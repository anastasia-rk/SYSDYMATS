% !TeX TXS-program:compile = txs:///pdflatex/[--shell-escape]

\documentclass[a4paper,11pt,twoside]{article}
%%%%%%%%%%%%%%%%%%%%%%%%%%%%%%%%%%%%%%%%%%%%%%%%%%%%%%%%%%%%%%%%%%%%%%%%%%%
% A generic report compiler for publishing rables and plots for a partcular choise of lags in the NARX system
% A path to media corresponding to the setting is defined below, as well as chosen parameter settings
\makeatletter
\def\input@path{{../Results_set_C_ny_4_nu_4/}}
\def\dataset{C}
\def\ny{4}
\def\nu{4}
\def\order{2}
\makeatother
%%%%%%%%%%%%%%%%%%%%%%%%%%%%%%%%%%%%%%%%%%%%%%%%%%%%%%%%%%%%%%%%%%%%%%%%%%%
% Packages
\usepackage[final]{pdfpages}
\usepackage{verbatim}
\usepackage{inputenc}
\usepackage{graphicx} 
\usepackage{amsmath,amssymb,mathrsfs,amsfonts}
\usepackage{mathtools}
\usepackage{amsthm}
\usepackage{mathtools}
\usepackage{calrsfs}
\usepackage{graphicx}
\usepackage{subfig}
\usepackage{eucal}    
\usepackage{amssymb}  
\usepackage{pifont}
\usepackage{color} 
\usepackage{cancel}
\usepackage[toc,page]{appendix}
\usepackage{pgfplots}
\pgfplotsset{every axis/.append style={line width=0.5pt},label style={font=\small},tick label style={font=\small},x tick label style={/pgf/number format/.cd,fixed,precision=3, set thousands separator={}},y tick label style={/pgf/number format/.cd,fixed,precision=3, set thousands separator={}},z tick label style={/pgf/number format/.cd,fixed,precision=3, set thousands separator={}},x label style={font={\small},at={(axis description cs:0.5,0.0)},anchor=north},y label style={font={\small},at={(axis description cs:0.05,0.5)},anchor=south},z label style={font={\small},at={(axis description cs:-0.25,0.5)},anchor=north}}
\usetikzlibrary{shapes,shadows,arrows,backgrounds,patterns,positioning,automata,calc,decorations.markings,decorations.pathreplacing,bayesnet,arrows.meta} 
\usepackage{varwidth}
\usepackage{lscape}
\usepackage{array} 
\usepackage[colorlinks=false,pdfborder={0 0 0}]{hyperref}
\usepackage{tabularx}
\usepackage{textcomp}
\usepackage{multicol} 
\usepackage{booktabs}
\usepackage{multirow}
\usepackage[font=small,labelfont=bf]{caption}                                                           
\usepackage{textcase}
\usepackage{bbm} 
\usepackage{fancyhdr}
\usepackage{enumitem}
\usepackage{soul}
\usepackage{wrapfig}
%%%%%%%%%%%%%%%%%%%%%%%%%%%%%%%%%%%%%%%%%%%%%%%%%%%%%%%%%%%%%%%%%%%%%%%%%%%
% Geometry
\setlength{\parindent}{2em}
\setlength{\parskip}{0.5em}
\renewcommand{\baselinestretch}{1.2}
\usepackage[left=2cm, right=2cm, top=2.5cm, bottom=3cm, headheight=13.6pt]{geometry}
\allowdisplaybreaks 
%%%%%%%%%%%%%%%%%%%%%%%%%%%%%%%%%%%%%%%%%%%%%%%%%%%%%%%%%%%%%%%%%%%%%%%%%%%
% Bibliography
\usepackage[backend=bibtex,style=ieee,sorting=none]{biblatex} 
\bibliography{bibliography}
\renewcommand*{\bibfont}{\scriptsize}
%%%%%%%%%%%%%%%%%%%%%%%%%%%%%%%%%%%%%%%%%%%%%%%%%%%%%%%%%%%%%%%%%%%%%%%%%%%
% Custom commands and operators
\makeatletter
\newcommand*{\rom}[1]{\expandafter\@slowromancap\romannumeral #1@}
\newcommand{\ie}{\textit{i.e.} }
\newcommand{\eg}{\textit{e.g.} }
\newcommand\id{\ensuremath{\mathbbm{1}}} 
\DeclareMathOperator{\E}{\mathbb{E}}
\DeclareMathOperator{\eye}{\mathbb{I}}
\DeclareMathOperator{\zeros}{\mathbb{O}}
\DeclareMathOperator{\tr}{\textrm{tr}}
\DeclareMathOperator{\vvec}{\textrm{vec}}
\DeclareMathOperator{\ik}{\mathrm{k}}
\DeclareMathOperator{\ip}{\mathrm{p}}
\DeclareMathOperator{\inn}{\mathrm{n}}
\DeclareMathOperator{\im}{\mathrm{m}}
\DeclareMathOperator{\td}{\mathrm{t}}
\DeclareMathOperator{\kd}{\mathrm{k}}
\DeclareMathOperator{\T}{\mathrm{T}}
\DeclareMathOperator{\K}{\mathrm{K}}
\DeclareMathOperator{\rk}{\mathrm{rk}}
\DeclareMathOperator{\vc}{\mathrm{vec}}
\DeclareSymbolFontAlphabet{\mathcal} {symbols}
\DeclareSymbolFont{symbols}{OMS}{cm}{m}{n}
\DeclareMathAlphabet{\mathbfit}{OML}{cmm}{b}{it}
\makeatother
% Number equations
%\numberwithin{equation}{section}

%%%%%%%%%%%%%%%%%%%%%%%%%%%%%%%%%%%%%%%%%%%%%%%%%%%%%%%%%%%%%%%%%%%%%%%%%%%
%Theorems
\newtheoremstyle{mytheoremstyle} % name
{.5em}                    % Space above
{.8em}                    % Space below
{\itshape}                % Body font
{1em}                           % Indent amount
{\bfseries}                   % Theorem head font
{:}                          % Punctuation after theorem head
{.5em}                       % Space after theorem head
{}  % Theorem head spec (can be left empty, meaning ‘normal’)

\theoremstyle{mytheoremstyle}
\newtheorem{theorem}{Theorem}[section]
\newtheorem{remark}{Remark}[section]
\newtheorem{assumption}{Assumption}[section]
\newtheorem{lemma}{Lemma}[section]
\newtheorem{condition}{Condition}[section]
\newtheorem{definition}{Definition}[section]
\newtheorem{property}{Property}[section]
\newtheorem{corollary}{Corollary}[section]
\renewcommand\qedsymbol{$\blacksquare$}
%%%%%%%%%%%%%%%%%%%%%%%%%%%%%%%%%%%%%%%%%%%%%%%%%%%%%%%%%%%%%%%%%%%%%%%%%%%
% Nomenclature
\usepackage[intoc]{nomencl}
\makenomenclature

\usepackage[ruled]{algorithm}
\usepackage{float}

\usepackage{algorithmic}
\algsetup{linenosize=\scriptsize}
\usepackage{etoolbox}
\AtBeginEnvironment{algorithmic}{\scriptsize}
\renewcommand{\thealgorithm}{\thechapter.\arabic{algorithm}} 
%\usepackage{chngcntr}
%\counterwithin{algorithm}{section}
% correct bad hyphenation here
\hyphenation{op-tical net-works semi-conduc-tor}
%%%%%%%%%%%%%%%%%%%%%%%%%%%%%%%%%%%%%%%%%%%%%%%%%%%%%%%%%%%%%%%%%%%%%%%%%%%%%
\usepackage[explicit]{titlesec}
\usepackage{titletoc}
\interfootnotelinepenalty=10000

\title{Structure and parameter identification of the dynamical model of auxetic foam}

\begin{document}
	\maketitle
\par Experimental data is described in the earlier reports. This report summarises the settings used for model identification and presents estimation and modelling results.
\section{Structure identification}
\par The following model structure is assumed. The output of the NARX model $\mathbfit{y}(t)$ is the measured load. The input vector is composed as
\begin{equation}
	\mathbfit{x}(t) = \{x_i(t)\}^{d}_{i=1} = \left[\{y(t - k)\}^{n_y}_{k=1} \quad \{u(t - k + n_y + 1)\}^{n_y + n_u}_{k= n_y + 1} \right]^{\top},
\end{equation}
where $n_u$ is the length of the input lag and $n_y$ is the length of the output lag in discrete time, and where $d = n_u + n_y$. In this case, the identification is performed under the following assumptions:
\begin{itemize}[noitemsep,topsep=0.3pt,parsep=0.3pt,partopsep=0.2pt,labelindent=1cm] 
	\item the output lag $n_y =\ny$.
	\item the input signal has a lag of length $n_u = \nu$.
\end{itemize}
The resultant input vector of the NARX model then takes the following form:
\begin{equation}
\mathbfit{x}(t) = \left[\begin{array}{cccccc}
u(t-n_u) & \dots & y(t-1) & u(t-n_u+1) & \dots & u(t)
\end{array}\right]^{\top}.
\end{equation}
The unknown model is approximated with a sum of polynomial basis functions up to second degree ($\lambda = \order$), rendering the following structure
\begin{equation}\label{eq:narx}
	\mathbfit{y}(t) = \theta^0 + \sum_{i=1}^{d} \theta_i x_i(t) + \sum_{i=1}^{d} \sum_{j=1}^{d} \theta_{i,j} x_i(t) x_j(t) + e(t).
\end{equation}
The number and order of significant terms are identified within the EFOR-CMSS algorithm based on the data from 8 out of 10 datasets. Figure \ref{fig:aamdl} illustrates the relationship between the number of model terms and the selected criterion of significance, AAMDL.
\begin{figure}[!h]
	\definecolor{mycolor1}{rgb}{0.00000,0.44700,0.74100}%
	\definecolor{mycolor2}{rgb}{0.85000,0.32500,0.09800}%
	\centering
	\subfloat[Sample size 2000.]{% This file was created by matlab2tikz.
%
\definecolor{mycolor1}{rgb}{0.00000,0.44700,0.74100}%
\definecolor{mycolor2}{rgb}{0.85000,0.32500,0.09800}%
%
\begin{tikzpicture}

\begin{axis}[%
width=5.706cm,
height=1.059cm,
at={(0cm,2.941cm)},
scale only axis,
xmin=1,
xmax=31,
xlabel style={font=\color{white!15!black}},
xlabel={Number of terms},
ymin=-0.429433771140034,
ymax=0.00127750520925408,
ylabel style={font=\color{white!15!black}},
ylabel={AAMDL},
axis background/.style={fill=white}
]
\addplot [color=mycolor1, draw=none, mark=o, mark options={solid, mycolor1}, forget plot]
  table[row sep=crcr]{%
1	0.00127750520925408\\
2	-0.0393982480025287\\
3	-0.126851839342105\\
4	-0.286792490956701\\
5	-0.429433771140034\\
6	-0.428179225773623\\
7	-0.426906740688297\\
8	-0.425572211840805\\
9	-0.424272108463344\\
10	-0.422941949752557\\
11	-0.421559364723936\\
12	-0.420170432302743\\
13	-0.418758586204635\\
14	-0.417326917026395\\
15	-0.415880887292387\\
16	-0.414508144190187\\
17	-0.413256641938182\\
18	-0.411810612204174\\
19	-0.41047927769617\\
20	-0.40909520865767\\
21	-0.407650647405457\\
22	-0.40627731975263\\
23	-0.404836457579981\\
24	-0.403482296117255\\
25	-0.402167183337615\\
26	-0.400749595190825\\
27	-0.399411640479846\\
28	-0.397980674851794\\
29	-0.396534645117786\\
30	-0.395177228746445\\
31	-0.393772305748151\\
};
\addplot [color=mycolor2, line width=5.0pt, draw=none, mark=asterisk, mark options={solid, mycolor2}, forget plot]
  table[row sep=crcr]{%
6	-0.428179225773623\\
};
\end{axis}

\begin{axis}[%
width=5.706cm,
height=1.059cm,
at={(0cm,1.471cm)},
scale only axis,
xmin=1,
xmax=31,
xlabel style={font=\color{white!15!black}},
xlabel={Number of terms},
ymin=-10000,
ymax=0,
ylabel style={font=\color{white!15!black}},
ylabel={BIC},
axis background/.style={fill=white}
]
\addplot [color=mycolor1, draw=none, mark=o, mark options={solid, mycolor1}, forget plot]
  table[row sep=crcr]{%
1	0\\
2	-785.282117884878\\
3	-2465.39356518993\\
4	-5522.90413184089\\
5	-8251.69204255499\\
6	-8246.17147617287\\
7	-8240.31001925192\\
8	-8233.26960676229\\
9	-8226.88334705418\\
10	-8219.92597590223\\
11	-8211.97239985879\\
12	-8203.89821066169\\
13	-8195.38861577585\\
14	-8186.50234272135\\
15	-8177.34319038615\\
16	-8169.57663062856\\
17	-8164.11388950637\\
18	-8154.95473717116\\
19	-8147.97502352048\\
20	-8139.99324832131\\
21	-8130.86200007717\\
22	-8123.08433268857\\
23	-8114.0233743543\\
24	-8106.60990291\\
25	-8099.9384345417\\
26	-8091.31972924681\\
27	-8084.2142184992\\
28	-8075.34131430536\\
29	-8066.18216197015\\
30	-8058.70684075235\\
31	-8050.32879860918\\
};
\addplot [color=mycolor2, line width=5.0pt, draw=none, mark=asterisk, mark options={solid, mycolor2}, forget plot]
  table[row sep=crcr]{%
6	-8246.17147617287\\
};
\end{axis}

\begin{axis}[%
width=5.706cm,
height=1.059cm,
at={(0cm,0cm)},
scale only axis,
xmin=1,
xmax=31,
xlabel style={font=\color{white!15!black}},
xlabel={Number of terms},
ymin=-10000,
ymax=0,
ylabel style={font=\color{white!15!black}},
ylabel={AIC},
axis background/.style={fill=white}
]
\addplot [color=mycolor1, draw=none, mark=o, mark options={solid, mycolor1}, forget plot]
  table[row sep=crcr]{%
1	0\\
2	-799.600422555292\\
3	-2486.87102219555\\
4	-5551.54074118172\\
5	-8287.48780423103\\
6	-8289.12639018411\\
7	-8290.42408559837\\
8	-8290.54282544394\\
9	-8291.31571807104\\
10	-8291.5174992543\\
11	-8290.72307554606\\
12	-8289.80803868416\\
13	-8288.45759613354\\
14	-8286.73047541425\\
15	-8284.73047541425\\
16	-8284.12306799186\\
17	-8285.81947920489\\
18	-8283.81947920489\\
19	-8283.99891788941\\
20	-8283.17629502545\\
21	-8281.20419911652\\
22	-8280.58568406312\\
23	-8278.68387806406\\
24	-8278.42955895496\\
25	-8278.91724292187\\
26	-8277.45768996219\\
27	-8277.51133154978\\
28	-8275.79757969115\\
29	-8273.79757969115\\
30	-8273.48141080855\\
31	-8272.26252100059\\
};
\addplot [color=mycolor2, line width=5.0pt, draw=none, mark=asterisk, mark options={solid, mycolor2}, forget plot]
  table[row sep=crcr]
	\subfloat[Sample size 4000.]{% This file was created by matlab2tikz.
%
\definecolor{mycolor1}{rgb}{0.00000,0.44700,0.74100}%
\definecolor{mycolor2}{rgb}{0.85000,0.32500,0.09800}%
%
\begin{tikzpicture}

\begin{axis}[%
width=5.706cm,
height=4cm,
at={(0cm,0cm)},
scale only axis,
xmin=1,
xmax=30,
xlabel style={font=\color{white!15!black}},
xlabel={Number of terms},
ymin=-2.02,
ymax=-1.93,
ylabel style={font=\color{white!15!black}},
ylabel={AAMDL},
axis background/.style={fill=white}
]
\addplot [color=mycolor1, draw=none, mark=o, mark options={solid, mycolor1}, forget plot]
  table[row sep=crcr]{%
1	-1.93736409910238\\
2	-1.97562482423252\\
3	-1.98557227378914\\
4	-2.00268972603383\\
5	-2.00704559304285\\
6	-2.0083350710199\\
7	-2.01210085697481\\
8	-2.01129351846349\\
9	-2.01034260062948\\
10	-2.00871092125821\\
11	-2.00735555052441\\
12	-2.00590619686071\\
13	-2.00424228983046\\
14	-2.00248225755915\\
15	-2.00092455375429\\
16	-1.99907498044788\\
17	-1.99721410861481\\
18	-1.99538053022978\\
19	-1.99355575101086\\
20	-1.99170778739878\\
21	-1.98986094147307\\
22	-1.98814243531639\\
23	-1.98632531638958\\
24	-1.98447087981026\\
25	-1.98267176150182\\
26	-1.98084700696409\\
27	-1.97897450145893\\
28	-1.97711029215376\\
29	-1.97523586160061\\
30	-1.97336882414125\\
};
\addplot [color=mycolor2, line width=5.0pt, draw=none, mark=asterisk, mark options={solid, mycolor2}, forget plot]
  table[row sep=crcr]
	\resizebox{!}{0cm}{
		\begin{minipage}{\textwidth}
			\begin{tikzpicture}	
			\begin{axis}[width=2cm,height=2cm]
			\addplot [color=mycolor2,line width=5.0pt,only marks,mark=asterisk,mark options={solid},forget plot]
			table[row sep=crcr]{%
				8	-3.00104696019893\\
			};\label{tikz:nterms}
			\end{axis}
			\end{tikzpicture}%
	\end{minipage}}
	\caption{Evolution of AAMDL with the respect to the number of terms for samples of different size. The optimal number of terms (\ref{tikz:nterms}) increases with the growing sample size.}\label{fig:aamdl}
\end{figure}	
\section{Parameter estimation}
\par The results of internal parameter estimation via the EFOR-CMSS method for sample sizes of 2000 and 4000 points are presented in Tables \ref{tab:thetas2000} and \ref{tab:thetas4000}, respectively.
\begin{table}[!h]
	\centering
	\caption{Estimated parameters for the sample length 2000.}\label{tab:thetas2000}
	\small
	\begin{tabular}{rrrrrrrrrrr}
Step & Terms & C1 & C2 & C4 & C5 & C6 & C7 & C8 & C10 & AERR($\%$) \\ 
\hline 
1 & $y(t-1)u(t)$ & -0.07 & -0.1 & -0.29 & -0.05 & -0.41 & -0.44 & -0.51 & -0.37 & 91.954 \\ 
2 & $y(t-1)u(t-1)$ & 0.04 & 0.08 & 0.28 & 0.04 & 0.3 & 0.35 & 0.41 & 0.29 & 4.929 \\ 
3 & $u(t-2)u(t-2)$ & 0.63 & 0.59 & 0.41 & 0.23 & -1.24 & 0.36 & -0.47 & 0.56 & 1.542 \\ 
4 & $y(t-2)u(t)$ & -0.17 & -0.19 & -0.13 & -0.21 & -0.21 & -0.23 & -0.22 & -0.21 & 0.325 \\ 
5 & $u(t-3)u(t)$ & 4.51 & 3.34 & 1.76 & 3.26 & 8.78 & 7.87 & 7.51 & 6.34 & 0.165 \\ 
6 & $u(t-3)$ & 24.65 & 18.68 & 11.08 & 19.07 & 46.29 & 45.79 & 40.38 & 35.38 & 0.089 \\ 
7 & $u(t)u(t)$ & -26.48 & -21.13 & -8.35 & -10.45 & -179.88 & -140.06 & -113.99 & -63.28 & 0.07 \\ 
8 & $u(t)$ & -128.76 & -105.22 & -43.88 & -51.12 & -837.17 & -659.26 & -535.31 & -300.36 & 0.031 \\ 
9 & $u(t-1)u(t)$ & 20.61 & 16.56 & 5.37 & 6.18 & 172.93 & 132.02 & 107.28 & 56.56 & 0.027 \\ 
10 & $u(t-1)$ & 100.95 & 84.09 & 29.25 & 28.52 & 796.2 & 616.45 & 498.36 & 267.13 & 0.04 \\ 
11 & $y(t-3)y(t-1)$ & 0 & 0 & 0 & -0.01 & 0 & 0 & 0 & 0 & 0.021 \\ 
12 & $y(t-2)y(t-2)$ & 0 & 0 & 0 & 0.01 & 0 & 0 & 0 & 0 & 0.006 \\ 
13 & $y(t-2)$ & -0.65 & -0.8 & -0.67 & -0.99 & -0.81 & -0.83 & -0.8 & -0.72 & 0.004 \\ 
14 & $y(t-1)y(t-1)$ & 0 & 0 & 0 & 0 & 0 & 0 & 0 & 0 & 0.004 \\ 
15 & $y(t-4)u(t)$ & -0.01 & 0 & 0.01 & 0 & -0.01 & -0.01 & -0.01 & -0.01 & 0.004 \\ 
16 & $y(t-3)u(t)$ & -0.02 & -0.02 & -0.01 & -0.01 & 0.01 & 0.01 & 0.01 & 0 & 0.003 \\ 
\hline 
\end{tabular}
\end{table}
\begin{table}[!h]
	\centering
	\caption{Estimated parameters for the sample length 4000.}\label{tab:thetas4000}
	\small
	\begin{tabular}{rrrrrrrrrrr}
Step & Terms & C1 & C2 & C4 & C5 & C6 & C7 & C9 & C10 & AEER($\%$) \\ 
\hline 
1 & $x_4,x_4$ & -18.97 & -15.33 & -8.34 & -8.94 & -138.21 & -114.23 & -63.58 & -50.3 & 88.667 \\ 
2 & $x_3$ & 63.2 & 53.09 & 30.09 & 24.55 & 426.09 & 362.14 & 208.4 & 168.78 & 9.494 \\ 
3 & $x_1,x_4$ & 4.29 & 3.45 & 2.03 & 3.11 & 34.39 & 27.1 & 14.45 & 10.2 & 0.12 \\ 
4 & $x_1,x_1$ & 0.6 & 0.87 & 0.31 & 0.56 & 9.24 & 5.54 & 4.15 & 4.1 & 0.042 \\ 
5 & $x_2$ & 2.34 & 0.07 & -1.9 & -2.13 & 43.4 & 34.96 & 18.62 & 12.71 & 0.036 \\ 
6 & $x_4$ & -157.87 & -128.81 & -69.1 & -69.29 & -1153.95 & -964.72 & -539.51 & -431.83 & 0.006 \\ 
7 & $c$ & -226.03 & -187.69 & -100.5 & -116.41 & -1722.47 & -1432.4 & -789.68 & -632.3 & 0.335 \\ 
8 & $x_3,x_3$ & 9.21 & 7.92 & 4.31 & 4.67 & 70 & 55.98 & 32.65 & 26.32 & 0.103 \\ 
9 & $x_1,x_3$ & -4.8 & -4.8 & -2.64 & -4.26 & -45.05 & -31.95 & -19.47 & -15.95 & 0.007 \\ 
\hline 
\end{tabular}
\end{table}
\par In order to link the external and internal parameters, an arbitrary polynomial function of two arguments is formed
\begin{equation}
\theta_i(L_{cut},D_{rlx}) = \beta_0 + \beta_1 L_{cut} + \beta_2 D_{rlx} + \beta_3 L_{cut}^{2} + \beta_4 D_{rlx}^{2} + \beta_5 L_{cut} D_{rlx}, \qquad i=1,\dots,N_s,
\end{equation}
where $L_{cut}$ and $D_{rlx}$ are the external parameters of manufacturing process, and where $N_s$ is the estimated number of the significant terms. The coefficients $B = \left[ \beta_0 \dots \beta_5 \right]$ are unknown and must be estimated from the internal parameter values available. The linear relationship between the batch of internal parameter values corresponding to each significant term $\Theta_i$ and the surface coefficients can be established in the following form
\begin{equation}\label{eq:linrel}
\left[\begin{array}{c}
\theta_{i}^{1} \\
\vdots \\
\theta_{i}^{k} \\
\vdots \\
\theta_{i}^{K}
\end{array}\right] =
\left[\begin{array}{cccccc}
1 & L& D& L\times L& D\times D& L\times D
\end{array}\right] B,
\end{equation}
where $k = 1, \dots, K$ is the index of the dataset, $i =1, \dots, N_s$ is the element index of the internal parameter vector, and where $\times$ denotes the by-element product such that
\begin{equation*}
L\times L = \Big[ L_{cut}^{(j)} L_{cut}^{(j)}\Big]^{N}_{j=1}.
\end{equation*}
Given the equation \eqref{eq:linrel}, the vector of unknown coefficients $B$ can be identified via the classical least squares (LS) method. The estimated coefficients are presented in Table \ref{tab:betas_all}, and the surface fitting results for each internal parameter are illustrated in Figure \ref{fig:surfaces_all} 
\begin{table}[!h]
	\centering
	\caption{Estimated polynomial coefficients for the sample length 2000.}\label{tab:betas_all}
	\small
	\begin{tabular}{rrrrrrr}
Terms & $\beta_0$ & $\beta_1$ & $\beta_2$ & $\beta_3$ & $\beta_4$ & $\beta_5$ \\ 
\hline 
$u(t)u(t)$ & -170.31 & 4.27 & 831.86 & -2.02 & -0.03 & -4562.38 \\ 
$u(t-1)$ & 143.25 & -3.37 & -1382.17 & -3.9 & 0.04 & 11446.44 \\ 
$u(t-3)u(t)$ & 5.32 & 0.17 & -203.84 & 2.88 & 0 & 161.25 \\ 
$u(t-3)u(t-3)$ & 1.85 & -0.2 & 78.58 & -1.4 & 0 & 39.42 \\ 
$u(t-2)$ & 76.05 & -1.76 & -434.89 & 5.27 & 0.01 & 953.45 \\ 
$u(t)$ & -1216.06 & 31.23 & 5342.46 & -8.72 & -0.25 & -30672.05 \\ 
$c$ & -2427.76 & 63.16 & 8882.74 & -21.5 & -0.49 & -45523.06 \\ 
$u(t-1)u(t)$ & 61.26 & -1.53 & -370.55 & 0.3 & 0.01 & 2478.05 \\ 
\hline 
\end{tabular}
\end{table}
The figure also shows values of the internal parameters computed for the external settings of experiments \dataset3 and \dataset8. The obtained internal parameters are substituted in the modified version of model \eqref{eq:narx} that only includes the identified significant  polynomial terms to validate the identified model structure. The simulation results are compared with true system outputs for \dataset3 and \dataset8 in Figure \ref{fig:c3all} and Figure \ref{fig:c8all}, respectively. Computed RMSEs for both lengths of the sample are presented in Table \ref{tab:RMSEs}.
\begin{table}[!h]
	\centering
	\caption{RMSE of the system output generated by the identified model.}\label{tab:RMSEs}
		\begin{tabular}{cc}
			Sample size 2000 & Sample size 4000 \\
			\hline	
			\begin{minipage}{2in} \verbatiminput{RMSEs_T_2000.txt}\end{minipage}& 
			\begin{minipage}{2in} \vspace{0.5cm}\verbatiminput{RMSEs_T_4000.txt}\end{minipage}\\
			\hline
		\end{tabular}
\end{table}

\begin{figure}[!t]
	\centering
	\resizebox{!}{0cm}{
		\begin{minipage}{\textwidth}
		% This file was created by matlab2tikz.
%
\definecolor{mycolor1}{rgb}{0.00000,0.44700,0.74100}%
\definecolor{mycolor2}{rgb}{0.85000,0.32500,0.09800}%
%
\begin{tikzpicture}

\begin{axis}[%
width=3.159cm,
height=3.097cm,
at={(0cm,12.903cm)},
scale only axis,
xmin=56,
xmax=74,
tick align=outside,
axis background/.style={fill=white},
xmajorgrids,
ymajorgrids,
zmajorgrids
]
\addplot3[only marks, mark=*, mark options={}, mark size=1.5000pt, color=mycolor1, fill=mycolor1] table[row sep=crcr]{%
x	y	z\\
74	0.123	-26.0353957891804\\
72	0.113	-20.9879322169279\\
61	0.095	-10.6920630070547\\
56	0.093	-10.9569379219045\\
};\label{tikz:thetas1}
\addplot3[only marks, mark=*, mark options={}, mark size=1.5000pt, color=mycolor2, fill=mycolor2] table[row sep=crcr]{%
x	y	z\\
67	0.276	-191.779551108501\\
66	0.255	-157.643535964989\\
62	0.209	-87.4196213968237\\
57	0.193	-69.8013569503948\\
};\label{tikz:thetas2}
\addplot3[only marks, mark=*, mark options={}, mark size=1.5000pt, color=black, fill=black] table[row sep=crcr]{%
x	y	z\\
69	0.104	-15.57705861622\\
};\label{tikz:thetaidentified}
\addplot3[only marks, mark=*, mark options={}, mark size=1.5000pt, color=black, fill=black] table[row sep=crcr]{%
x	y	z\\
64	0.23	-116.694087150299\\
};
\addplot3[%
surf,
fill opacity=0.7, shader=interp, colormap={mymap}{[1pt] rgb(0pt)=(1,0.905882,0); rgb(1pt)=(1,0.901964,0); rgb(2pt)=(1,0.898051,0); rgb(3pt)=(1,0.894144,0); rgb(4pt)=(1,0.890243,0); rgb(5pt)=(1,0.886349,0); rgb(6pt)=(1,0.88246,0); rgb(7pt)=(1,0.878577,0); rgb(8pt)=(1,0.8747,0); rgb(9pt)=(1,0.870829,0); rgb(10pt)=(1,0.866964,0); rgb(11pt)=(1,0.863106,0); rgb(12pt)=(1,0.859253,0); rgb(13pt)=(1,0.855406,0); rgb(14pt)=(1,0.851566,0); rgb(15pt)=(1,0.847732,0); rgb(16pt)=(1,0.843903,0); rgb(17pt)=(1,0.840081,0); rgb(18pt)=(1,0.836265,0); rgb(19pt)=(1,0.832455,0); rgb(20pt)=(1,0.828652,0); rgb(21pt)=(1,0.824854,0); rgb(22pt)=(1,0.821063,0); rgb(23pt)=(1,0.817278,0); rgb(24pt)=(1,0.8135,0); rgb(25pt)=(1,0.809727,0); rgb(26pt)=(1,0.805961,0); rgb(27pt)=(1,0.8022,0); rgb(28pt)=(1,0.798445,0); rgb(29pt)=(1,0.794696,0); rgb(30pt)=(1,0.790953,0); rgb(31pt)=(1,0.787215,0); rgb(32pt)=(1,0.783484,0); rgb(33pt)=(1,0.779758,0); rgb(34pt)=(1,0.776038,0); rgb(35pt)=(1,0.772324,0); rgb(36pt)=(1,0.768615,0); rgb(37pt)=(1,0.764913,0); rgb(38pt)=(1,0.761217,0); rgb(39pt)=(1,0.757527,0); rgb(40pt)=(1,0.753843,0); rgb(41pt)=(1,0.750165,0); rgb(42pt)=(1,0.746493,0); rgb(43pt)=(1,0.742827,0); rgb(44pt)=(1,0.739167,0); rgb(45pt)=(1,0.735514,0); rgb(46pt)=(1,0.731867,0); rgb(47pt)=(1,0.728226,0); rgb(48pt)=(1,0.724591,0); rgb(49pt)=(1,0.720963,0); rgb(50pt)=(1,0.717341,0); rgb(51pt)=(1,0.713725,0); rgb(52pt)=(0.999994,0.710077,0); rgb(53pt)=(0.999974,0.706363,0); rgb(54pt)=(0.999942,0.702592,0); rgb(55pt)=(0.999898,0.698775,0); rgb(56pt)=(0.999841,0.694921,0); rgb(57pt)=(0.999771,0.691039,0); rgb(58pt)=(0.99969,0.687139,0); rgb(59pt)=(0.999596,0.68323,0); rgb(60pt)=(0.99949,0.679323,0); rgb(61pt)=(0.999372,0.675427,0); rgb(62pt)=(0.999242,0.67155,0); rgb(63pt)=(0.9991,0.667704,0); rgb(64pt)=(0.998946,0.663897,0); rgb(65pt)=(0.998781,0.660138,0); rgb(66pt)=(0.998605,0.656439,0); rgb(67pt)=(0.998416,0.652807,0); rgb(68pt)=(0.998217,0.649253,0); rgb(69pt)=(0.998006,0.645786,0); rgb(70pt)=(0.997785,0.642416,0); rgb(71pt)=(0.997552,0.639152,0); rgb(72pt)=(0.997308,0.636004,0); rgb(73pt)=(0.997053,0.632982,0); rgb(74pt)=(0.996788,0.630095,0); rgb(75pt)=(0.996512,0.627352,0); rgb(76pt)=(0.996226,0.624763,0); rgb(77pt)=(0.995851,0.622329,0); rgb(78pt)=(0.99494,0.619997,0); rgb(79pt)=(0.99345,0.617753,0); rgb(80pt)=(0.991419,0.61559,0); rgb(81pt)=(0.988885,0.613503,0); rgb(82pt)=(0.985886,0.611486,0); rgb(83pt)=(0.98246,0.609532,0); rgb(84pt)=(0.978643,0.607636,0); rgb(85pt)=(0.974475,0.605791,0); rgb(86pt)=(0.969992,0.603992,0); rgb(87pt)=(0.965232,0.602233,0); rgb(88pt)=(0.960233,0.600507,0); rgb(89pt)=(0.955033,0.598808,0); rgb(90pt)=(0.949669,0.59713,0); rgb(91pt)=(0.94418,0.595468,0); rgb(92pt)=(0.938602,0.593815,0); rgb(93pt)=(0.932974,0.592166,0); rgb(94pt)=(0.927333,0.590513,0); rgb(95pt)=(0.921717,0.588852,0); rgb(96pt)=(0.916164,0.587176,0); rgb(97pt)=(0.910711,0.585479,0); rgb(98pt)=(0.905397,0.583755,0); rgb(99pt)=(0.900258,0.581999,0); rgb(100pt)=(0.895333,0.580203,0); rgb(101pt)=(0.890659,0.578362,0); rgb(102pt)=(0.886275,0.576471,0); rgb(103pt)=(0.882047,0.574545,0); rgb(104pt)=(0.877819,0.572608,0); rgb(105pt)=(0.873592,0.57066,0); rgb(106pt)=(0.869366,0.568701,0); rgb(107pt)=(0.865143,0.566733,0); rgb(108pt)=(0.860924,0.564756,0); rgb(109pt)=(0.856708,0.562771,0); rgb(110pt)=(0.852497,0.560778,0); rgb(111pt)=(0.848292,0.558779,0); rgb(112pt)=(0.844092,0.556774,0); rgb(113pt)=(0.8399,0.554763,0); rgb(114pt)=(0.835716,0.552749,0); rgb(115pt)=(0.831541,0.55073,0); rgb(116pt)=(0.827374,0.548709,0); rgb(117pt)=(0.823219,0.546686,0); rgb(118pt)=(0.819074,0.54466,0); rgb(119pt)=(0.81494,0.542635,0); rgb(120pt)=(0.81082,0.540609,0); rgb(121pt)=(0.806712,0.538584,0); rgb(122pt)=(0.802619,0.53656,0); rgb(123pt)=(0.798541,0.534539,0); rgb(124pt)=(0.794478,0.532521,0); rgb(125pt)=(0.790431,0.530506,0); rgb(126pt)=(0.786402,0.528496,0); rgb(127pt)=(0.782391,0.526491,0); rgb(128pt)=(0.77841,0.524489,0); rgb(129pt)=(0.774523,0.522478,0); rgb(130pt)=(0.770731,0.520455,0); rgb(131pt)=(0.767022,0.518424,0); rgb(132pt)=(0.763384,0.516385,0); rgb(133pt)=(0.759804,0.514339,0); rgb(134pt)=(0.756272,0.51229,0); rgb(135pt)=(0.752775,0.510237,0); rgb(136pt)=(0.749302,0.508182,0); rgb(137pt)=(0.74584,0.506128,0); rgb(138pt)=(0.742378,0.504075,0); rgb(139pt)=(0.738904,0.502025,0); rgb(140pt)=(0.735406,0.499979,0); rgb(141pt)=(0.731872,0.49794,0); rgb(142pt)=(0.72829,0.495909,0); rgb(143pt)=(0.724649,0.493887,0); rgb(144pt)=(0.720936,0.491875,0); rgb(145pt)=(0.71714,0.489876,0); rgb(146pt)=(0.713249,0.487891,0); rgb(147pt)=(0.709251,0.485921,0); rgb(148pt)=(0.705134,0.483968,0); rgb(149pt)=(0.700887,0.482033,0); rgb(150pt)=(0.696497,0.480118,0); rgb(151pt)=(0.691952,0.478225,0); rgb(152pt)=(0.687242,0.476355,0); rgb(153pt)=(0.682353,0.47451,0); rgb(154pt)=(0.677195,0.472696,0); rgb(155pt)=(0.6717,0.470916,0); rgb(156pt)=(0.665891,0.469169,0); rgb(157pt)=(0.659791,0.46745,0); rgb(158pt)=(0.653423,0.465756,0); rgb(159pt)=(0.64681,0.464084,0); rgb(160pt)=(0.639976,0.462432,0); rgb(161pt)=(0.632943,0.460795,0); rgb(162pt)=(0.625734,0.459171,0); rgb(163pt)=(0.618373,0.457556,0); rgb(164pt)=(0.610882,0.455948,0); rgb(165pt)=(0.603284,0.454343,0); rgb(166pt)=(0.595604,0.452737,0); rgb(167pt)=(0.587863,0.451129,0); rgb(168pt)=(0.580084,0.449514,0); rgb(169pt)=(0.572292,0.447889,0); rgb(170pt)=(0.564508,0.446252,0); rgb(171pt)=(0.556756,0.444599,0); rgb(172pt)=(0.549059,0.442927,0); rgb(173pt)=(0.54144,0.441232,0); rgb(174pt)=(0.533922,0.439512,0); rgb(175pt)=(0.526529,0.437764,0); rgb(176pt)=(0.519282,0.435983,0); rgb(177pt)=(0.512206,0.434168,0); rgb(178pt)=(0.505323,0.432315,0); rgb(179pt)=(0.498628,0.430422,3.92506e-06); rgb(180pt)=(0.491973,0.428504,3.49981e-05); rgb(181pt)=(0.485331,0.426562,9.63073e-05); rgb(182pt)=(0.478704,0.424596,0.000186979); rgb(183pt)=(0.472096,0.422609,0.000306141); rgb(184pt)=(0.465508,0.420599,0.00045292); rgb(185pt)=(0.458942,0.418567,0.000626441); rgb(186pt)=(0.452401,0.416515,0.000825833); rgb(187pt)=(0.445885,0.414441,0.00105022); rgb(188pt)=(0.439399,0.412348,0.00129873); rgb(189pt)=(0.432942,0.410234,0.00157049); rgb(190pt)=(0.426518,0.408102,0.00186463); rgb(191pt)=(0.420129,0.40595,0.00218028); rgb(192pt)=(0.413777,0.40378,0.00251655); rgb(193pt)=(0.407464,0.401592,0.00287258); rgb(194pt)=(0.401191,0.399386,0.00324749); rgb(195pt)=(0.394962,0.397164,0.00364042); rgb(196pt)=(0.388777,0.394925,0.00405048); rgb(197pt)=(0.38264,0.39267,0.00447681); rgb(198pt)=(0.376552,0.390399,0.00491852); rgb(199pt)=(0.370516,0.388113,0.00537476); rgb(200pt)=(0.364532,0.385812,0.00584464); rgb(201pt)=(0.358605,0.383497,0.00632729); rgb(202pt)=(0.352735,0.381168,0.00682184); rgb(203pt)=(0.346925,0.378826,0.00732741); rgb(204pt)=(0.341176,0.376471,0.00784314); rgb(205pt)=(0.335485,0.374093,0.00847245); rgb(206pt)=(0.329843,0.371682,0.00930909); rgb(207pt)=(0.324249,0.369242,0.0103377); rgb(208pt)=(0.318701,0.366772,0.0115428); rgb(209pt)=(0.313198,0.364275,0.0129091); rgb(210pt)=(0.307739,0.361753,0.0144211); rgb(211pt)=(0.302322,0.359206,0.0160634); rgb(212pt)=(0.296945,0.356637,0.0178207); rgb(213pt)=(0.291607,0.354048,0.0196776); rgb(214pt)=(0.286307,0.35144,0.0216186); rgb(215pt)=(0.281043,0.348814,0.0236284); rgb(216pt)=(0.275813,0.346172,0.0256916); rgb(217pt)=(0.270616,0.343517,0.0277927); rgb(218pt)=(0.265451,0.340849,0.0299163); rgb(219pt)=(0.260317,0.33817,0.0320472); rgb(220pt)=(0.25521,0.335482,0.0341698); rgb(221pt)=(0.250131,0.332786,0.0362688); rgb(222pt)=(0.245078,0.330085,0.0383287); rgb(223pt)=(0.240048,0.327379,0.0403343); rgb(224pt)=(0.235042,0.324671,0.04227); rgb(225pt)=(0.230056,0.321962,0.0441205); rgb(226pt)=(0.22509,0.319254,0.0458704); rgb(227pt)=(0.220142,0.316548,0.0475043); rgb(228pt)=(0.215212,0.313846,0.0490067); rgb(229pt)=(0.210296,0.311149,0.0503624); rgb(230pt)=(0.205395,0.308459,0.0515759); rgb(231pt)=(0.200514,0.305763,0.052757); rgb(232pt)=(0.195655,0.303061,0.0539242); rgb(233pt)=(0.190817,0.300353,0.0550763); rgb(234pt)=(0.186001,0.297639,0.0562123); rgb(235pt)=(0.181207,0.294918,0.0573313); rgb(236pt)=(0.176434,0.292191,0.0584321); rgb(237pt)=(0.171685,0.289458,0.0595136); rgb(238pt)=(0.166957,0.286719,0.060575); rgb(239pt)=(0.162252,0.283973,0.0616151); rgb(240pt)=(0.15757,0.281221,0.0626328); rgb(241pt)=(0.152911,0.278463,0.0636271); rgb(242pt)=(0.148275,0.275699,0.0645971); rgb(243pt)=(0.143663,0.272929,0.0655416); rgb(244pt)=(0.139074,0.270152,0.0664596); rgb(245pt)=(0.134508,0.26737,0.06735); rgb(246pt)=(0.129967,0.264581,0.0682118); rgb(247pt)=(0.125449,0.261787,0.0690441); rgb(248pt)=(0.120956,0.258986,0.0698456); rgb(249pt)=(0.116487,0.25618,0.0706154); rgb(250pt)=(0.112043,0.253367,0.0713525); rgb(251pt)=(0.107623,0.250549,0.0720557); rgb(252pt)=(0.103229,0.247724,0.0727241); rgb(253pt)=(0.0988592,0.244894,0.0733566); rgb(254pt)=(0.0945149,0.242058,0.0739522); rgb(255pt)=(0.0901961,0.239216,0.0745098)}, mesh/rows=49]
table[row sep=crcr, point meta=\thisrow{c}] {%
%
x	y	z	c\\
56	0.093	-10.9049623108072	-10.9049623108072\\
56	0.09666	-11.4423373867092	-11.4423373867092\\
56	0.10032	-12.101944154261	-12.101944154261\\
56	0.10398	-12.8837826134625	-12.8837826134625\\
56	0.10764	-13.7878527643139	-13.7878527643139\\
56	0.1113	-14.814154606815	-14.814154606815\\
56	0.11496	-15.9626881409659	-15.9626881409659\\
56	0.11862	-17.2334533667667	-17.2334533667667\\
56	0.12228	-18.6264502842172	-18.6264502842172\\
56	0.12594	-20.1416788933175	-20.1416788933175\\
56	0.1296	-21.7791391940677	-21.7791391940677\\
56	0.13326	-23.5388311864676	-23.5388311864676\\
56	0.13692	-25.4207548705173	-25.4207548705173\\
56	0.14058	-27.4249102462168	-27.4249102462168\\
56	0.14424	-29.5512973135661	-29.5512973135661\\
56	0.1479	-31.7999160725652	-31.7999160725652\\
56	0.15156	-34.1707665232141	-34.1707665232141\\
56	0.15522	-36.6638486655127	-36.6638486655127\\
56	0.15888	-39.2791624994612	-39.2791624994612\\
56	0.16254	-42.0167080250595	-42.0167080250595\\
56	0.1662	-44.8764852423076	-44.8764852423076\\
56	0.16986	-47.8584941512054	-47.8584941512054\\
56	0.17352	-50.9627347517531	-50.9627347517531\\
56	0.17718	-54.1892070439505	-54.1892070439505\\
56	0.18084	-57.5379110277977	-57.5379110277977\\
56	0.1845	-61.0088467032948	-61.0088467032948\\
56	0.18816	-64.6020140704416	-64.6020140704416\\
56	0.19182	-68.3174131292382	-68.3174131292382\\
56	0.19548	-72.1550438796846	-72.1550438796846\\
56	0.19914	-76.1149063217809	-76.1149063217809\\
56	0.2028	-80.1970004555269	-80.1970004555269\\
56	0.20646	-84.4013262809227	-84.4013262809227\\
56	0.21012	-88.7278837979683	-88.7278837979683\\
56	0.21378	-93.1766730066637	-93.1766730066637\\
56	0.21744	-97.7476939070088	-97.7476939070088\\
56	0.2211	-102.440946499004	-102.440946499004\\
56	0.22476	-107.256430782649	-107.256430782649\\
56	0.22842	-112.194146757943	-112.194146757943\\
56	0.23208	-117.254094424887	-117.254094424887\\
56	0.23574	-122.436273783482	-122.436273783482\\
56	0.2394	-127.740684833726	-127.740684833726\\
56	0.24306	-133.167327575619	-133.167327575619\\
56	0.24672	-138.716202009163	-138.716202009163\\
56	0.25038	-144.387308134356	-144.387308134356\\
56	0.25404	-150.180645951199	-150.180645951199\\
56	0.2577	-156.096215459692	-156.096215459692\\
56	0.26136	-162.134016659835	-162.134016659835\\
56	0.26502	-168.294049551628	-168.294049551628\\
56	0.26868	-174.57631413507	-174.57631413507\\
56	0.27234	-180.980810410162	-180.980810410162\\
56	0.276	-187.507538376904	-187.507538376904\\
56.375	0.093	-10.8126294158988	-10.8126294158988\\
56.375	0.09666	-11.3527833115828	-11.3527833115828\\
56.375	0.10032	-12.0151688989166	-12.0151688989166\\
56.375	0.10398	-12.7997861779002	-12.7997861779002\\
56.375	0.10764	-13.7066351485335	-13.7066351485335\\
56.375	0.1113	-14.7357158108167	-14.7357158108167\\
56.375	0.11496	-15.8870281647496	-15.8870281647496\\
56.375	0.11862	-17.1605722103324	-17.1605722103324\\
56.375	0.12228	-18.5563479475649	-18.5563479475649\\
56.375	0.12594	-20.0743553764473	-20.0743553764473\\
56.375	0.1296	-21.7145944969793	-21.7145944969793\\
56.375	0.13326	-23.4770653091613	-23.4770653091613\\
56.375	0.13692	-25.361767812993	-25.361767812993\\
56.375	0.14058	-27.3687020084745	-27.3687020084745\\
56.375	0.14424	-29.4978678956059	-29.4978678956059\\
56.375	0.1479	-31.7492654743869	-31.7492654743869\\
56.375	0.15156	-34.1228947448178	-34.1228947448178\\
56.375	0.15522	-36.6187557068985	-36.6187557068985\\
56.375	0.15888	-39.236848360629	-39.236848360629\\
56.375	0.16254	-41.9771727060093	-41.9771727060093\\
56.375	0.1662	-44.8397287430394	-44.8397287430394\\
56.375	0.16986	-47.8245164717192	-47.8245164717192\\
56.375	0.17352	-50.9315358920489	-50.9315358920489\\
56.375	0.17718	-54.1607870040283	-54.1607870040283\\
56.375	0.18084	-57.5122698076576	-57.5122698076576\\
56.375	0.1845	-60.9859843029366	-60.9859843029366\\
56.375	0.18816	-64.5819304898655	-64.5819304898655\\
56.375	0.19182	-68.3001083684441	-68.3001083684441\\
56.375	0.19548	-72.1405179386725	-72.1405179386725\\
56.375	0.19914	-76.1031592005507	-76.1031592005507\\
56.375	0.2028	-80.1880321540788	-80.1880321540788\\
56.375	0.20646	-84.3951367992566	-84.3951367992566\\
56.375	0.21012	-88.7244731360842	-88.7244731360842\\
56.375	0.21378	-93.1760411645616	-93.1760411645616\\
56.375	0.21744	-97.7498408846888	-97.7498408846888\\
56.375	0.2211	-102.445872296466	-102.445872296466\\
56.375	0.22476	-107.264135399893	-107.264135399893\\
56.375	0.22842	-112.204630194969	-112.204630194969\\
56.375	0.23208	-117.267356681695	-117.267356681695\\
56.375	0.23574	-122.452314860072	-122.452314860072\\
56.375	0.2394	-127.759504730098	-127.759504730098\\
56.375	0.24306	-133.188926291773	-133.188926291773\\
56.375	0.24672	-138.740579545099	-138.740579545099\\
56.375	0.25038	-144.414464490074	-144.414464490074\\
56.375	0.25404	-150.210581126699	-150.210581126699\\
56.375	0.2577	-156.128929454974	-156.128929454974\\
56.375	0.26136	-162.169509474899	-162.169509474899\\
56.375	0.26502	-168.332321186474	-168.332321186474\\
56.375	0.26868	-174.617364589698	-174.617364589698\\
56.375	0.27234	-181.024639684572	-181.024639684572\\
56.375	0.276	-187.554146471096	-187.554146471096\\
56.75	0.093	-10.7298928959079	-10.7298928959079\\
56.75	0.09666	-11.2728256113739	-11.2728256113739\\
56.75	0.10032	-11.9379900184897	-11.9379900184897\\
56.75	0.10398	-12.7253861172552	-12.7253861172552\\
56.75	0.10764	-13.6350139076706	-13.6350139076706\\
56.75	0.1113	-14.6668733897358	-14.6668733897358\\
56.75	0.11496	-15.8209645634508	-15.8209645634508\\
56.75	0.11862	-17.0972874288155	-17.0972874288155\\
56.75	0.12228	-18.49584198583	-18.49584198583\\
56.75	0.12594	-20.0166282344944	-20.0166282344944\\
56.75	0.1296	-21.6596461748085	-21.6596461748085\\
56.75	0.13326	-23.4248958067724	-23.4248958067724\\
56.75	0.13692	-25.3123771303862	-25.3123771303862\\
56.75	0.14058	-27.3220901456497	-27.3220901456497\\
56.75	0.14424	-29.454034852563	-29.454034852563\\
56.75	0.1479	-31.7082112511262	-31.7082112511262\\
56.75	0.15156	-34.084619341339	-34.084619341339\\
56.75	0.15522	-36.5832591232017	-36.5832591232017\\
56.75	0.15888	-39.2041305967142	-39.2041305967142\\
56.75	0.16254	-41.9472337618765	-41.9472337618765\\
56.75	0.1662	-44.8125686186886	-44.8125686186886\\
56.75	0.16986	-47.8001351671505	-47.8001351671505\\
56.75	0.17352	-50.9099334072622	-50.9099334072622\\
56.75	0.17718	-54.1419633390236	-54.1419633390236\\
56.75	0.18084	-57.4962249624349	-57.4962249624349\\
56.75	0.1845	-60.9727182774959	-60.9727182774959\\
56.75	0.18816	-64.5714432842068	-64.5714432842068\\
56.75	0.19182	-68.2923999825674	-68.2923999825674\\
56.75	0.19548	-72.1355883725778	-72.1355883725778\\
56.75	0.19914	-76.1010084542381	-76.1010084542381\\
56.75	0.2028	-80.1886602275481	-80.1886602275481\\
56.75	0.20646	-84.3985436925079	-84.3985436925079\\
56.75	0.21012	-88.7306588491175	-88.7306588491175\\
56.75	0.21378	-93.185005697377	-93.185005697377\\
56.75	0.21744	-97.7615842372861	-97.7615842372861\\
56.75	0.2211	-102.460394468845	-102.460394468845\\
56.75	0.22476	-107.281436392054	-107.281436392054\\
56.75	0.22842	-112.224710006913	-112.224710006913\\
56.75	0.23208	-117.290215313421	-117.290215313421\\
56.75	0.23574	-122.477952311579	-122.477952311579\\
56.75	0.2394	-127.787921001387	-127.787921001387\\
56.75	0.24306	-133.220121382845	-133.220121382845\\
56.75	0.24672	-138.774553455952	-138.774553455952\\
56.75	0.25038	-144.45121722071	-144.45121722071\\
56.75	0.25404	-150.250112677117	-150.250112677117\\
56.75	0.2577	-156.171239825174	-156.171239825174\\
56.75	0.26136	-162.214598664881	-162.214598664881\\
56.75	0.26502	-168.380189196237	-168.380189196237\\
56.75	0.26868	-174.668011419243	-174.668011419243\\
56.75	0.27234	-181.078065333899	-181.078065333899\\
56.75	0.276	-187.610350940205	-187.610350940205\\
57.125	0.093	-10.6567527508345	-10.6567527508345\\
57.125	0.09666	-11.2024642860824	-11.2024642860824\\
57.125	0.10032	-11.8704075129803	-11.8704075129803\\
57.125	0.10398	-12.6605824315278	-12.6605824315278\\
57.125	0.10764	-13.5729890417252	-13.5729890417252\\
57.125	0.1113	-14.6076273435724	-14.6076273435724\\
57.125	0.11496	-15.7644973370694	-15.7644973370694\\
57.125	0.11862	-17.0435990222161	-17.0435990222161\\
57.125	0.12228	-18.4449323990127	-18.4449323990127\\
57.125	0.12594	-19.968497467459	-19.968497467459\\
57.125	0.1296	-21.6142942275552	-21.6142942275552\\
57.125	0.13326	-23.3823226793011	-23.3823226793011\\
57.125	0.13692	-25.2725828226968	-25.2725828226968\\
57.125	0.14058	-27.2850746577424	-27.2850746577424\\
57.125	0.14424	-29.4197981844377	-29.4197981844377\\
57.125	0.1479	-31.6767534027828	-31.6767534027828\\
57.125	0.15156	-34.0559403127778	-34.0559403127778\\
57.125	0.15522	-36.5573589144225	-36.5573589144225\\
57.125	0.15888	-39.181009207717	-39.181009207717\\
57.125	0.16254	-41.9268911926613	-41.9268911926613\\
57.125	0.1662	-44.7950048692553	-44.7950048692553\\
57.125	0.16986	-47.7853502374992	-47.7853502374992\\
57.125	0.17352	-50.8979272973929	-50.8979272973929\\
57.125	0.17718	-54.1327360489364	-54.1327360489364\\
57.125	0.18084	-57.4897764921296	-57.4897764921296\\
57.125	0.1845	-60.9690486269727	-60.9690486269727\\
57.125	0.18816	-64.5705524534656	-64.5705524534656\\
57.125	0.19182	-68.2942879716082	-68.2942879716082\\
57.125	0.19548	-72.1402551814006	-72.1402551814006\\
57.125	0.19914	-76.1084540828429	-76.1084540828429\\
57.125	0.2028	-80.1988846759349	-80.1988846759349\\
57.125	0.20646	-84.4115469606768	-84.4115469606768\\
57.125	0.21012	-88.7464409370684	-88.7464409370684\\
57.125	0.21378	-93.2035666051098	-93.2035666051098\\
57.125	0.21744	-97.782923964801	-97.782923964801\\
57.125	0.2211	-102.484513016142	-102.484513016142\\
57.125	0.22476	-107.308333759133	-107.308333759133\\
57.125	0.22842	-112.254386193773	-112.254386193773\\
57.125	0.23208	-117.322670320064	-117.322670320064\\
57.125	0.23574	-122.513186138004	-122.513186138004\\
57.125	0.2394	-127.825933647594	-127.825933647594\\
57.125	0.24306	-133.260912848834	-133.260912848834\\
57.125	0.24672	-138.818123741723	-138.818123741723\\
57.125	0.25038	-144.497566326263	-144.497566326263\\
57.125	0.25404	-150.299240602452	-150.299240602452\\
57.125	0.2577	-156.223146570291	-156.223146570291\\
57.125	0.26136	-162.269284229779	-162.269284229779\\
57.125	0.26502	-168.437653580918	-168.437653580918\\
57.125	0.26868	-174.728254623706	-174.728254623706\\
57.125	0.27234	-181.141087358144	-181.141087358144\\
57.125	0.276	-187.676151784232	-187.676151784232\\
57.5	0.093	-10.5932089806785	-10.5932089806785\\
57.5	0.09666	-11.1416993357085	-11.1416993357085\\
57.5	0.10032	-11.8124213823883	-11.8124213823883\\
57.5	0.10398	-12.6053751207179	-12.6053751207179\\
57.5	0.10764	-13.5205605506973	-13.5205605506973\\
57.5	0.1113	-14.5579776723265	-14.5579776723265\\
57.5	0.11496	-15.7176264856055	-15.7176264856055\\
57.5	0.11862	-16.9995069905342	-16.9995069905342\\
57.5	0.12228	-18.4036191871128	-18.4036191871128\\
57.5	0.12594	-19.9299630753412	-19.9299630753412\\
57.5	0.1296	-21.5785386552193	-21.5785386552193\\
57.5	0.13326	-23.3493459267472	-23.3493459267472\\
57.5	0.13692	-25.242384889925	-25.242384889925\\
57.5	0.14058	-27.2576555447525	-27.2576555447525\\
57.5	0.14424	-29.3951578912299	-29.3951578912299\\
57.5	0.1479	-31.654891929357	-31.654891929357\\
57.5	0.15156	-34.0368576591339	-34.0368576591339\\
57.5	0.15522	-36.5410550805606	-36.5410550805606\\
57.5	0.15888	-39.1674841936372	-39.1674841936372\\
57.5	0.16254	-41.9161449983635	-41.9161449983635\\
57.5	0.1662	-44.7870374947396	-44.7870374947396\\
57.5	0.16986	-47.7801616827655	-47.7801616827655\\
57.5	0.17352	-50.8955175624411	-50.8955175624411\\
57.5	0.17718	-54.1331051337666	-54.1331051337666\\
57.5	0.18084	-57.4929243967419	-57.4929243967419\\
57.5	0.1845	-60.974975351367	-60.974975351367\\
57.5	0.18816	-64.5792579976418	-64.5792579976418\\
57.5	0.19182	-68.3057723355665	-68.3057723355665\\
57.5	0.19548	-72.1545183651409	-72.1545183651409\\
57.5	0.19914	-76.1254960863652	-76.1254960863652\\
57.5	0.2028	-80.2187054992393	-80.2187054992393\\
57.5	0.20646	-84.4341466037631	-84.4341466037631\\
57.5	0.21012	-88.7718193999367	-88.7718193999367\\
57.5	0.21378	-93.2317238877601	-93.2317238877601\\
57.5	0.21744	-97.8138600672334	-97.8138600672334\\
57.5	0.2211	-102.518227938356	-102.518227938356\\
57.5	0.22476	-107.344827501129	-107.344827501129\\
57.5	0.22842	-112.293658755552	-112.293658755552\\
57.5	0.23208	-117.364721701624	-117.364721701624\\
57.5	0.23574	-122.558016339346	-122.558016339346\\
57.5	0.2394	-127.873542668718	-127.873542668718\\
57.5	0.24306	-133.31130068974	-133.31130068974\\
57.5	0.24672	-138.871290402412	-138.871290402412\\
57.5	0.25038	-144.553511806733	-144.553511806733\\
57.5	0.25404	-150.357964902704	-150.357964902704\\
57.5	0.2577	-156.284649690325	-156.284649690325\\
57.5	0.26136	-162.333566169596	-162.333566169596\\
57.5	0.26502	-168.504714340516	-168.504714340516\\
57.5	0.26868	-174.798094203087	-174.798094203087\\
57.5	0.27234	-181.213705757307	-181.213705757307\\
57.5	0.276	-187.751549003177	-187.751549003177\\
57.875	0.093	-10.5392615854401	-10.5392615854401\\
57.875	0.09666	-11.0905307602521	-11.0905307602521\\
57.875	0.10032	-11.7640316267139	-11.7640316267139\\
57.875	0.10398	-12.5597641848255	-12.5597641848255\\
57.875	0.10764	-13.4777284345869	-13.4777284345869\\
57.875	0.1113	-14.5179243759981	-14.5179243759981\\
57.875	0.11496	-15.680352009059	-15.680352009059\\
57.875	0.11862	-16.9650113337699	-16.9650113337699\\
57.875	0.12228	-18.3719023501304	-18.3719023501304\\
57.875	0.12594	-19.9010250581408	-19.9010250581408\\
57.875	0.1296	-21.5523794578009	-21.5523794578009\\
57.875	0.13326	-23.3259655491109	-23.3259655491109\\
57.875	0.13692	-25.2217833320706	-25.2217833320706\\
57.875	0.14058	-27.2398328066802	-27.2398328066802\\
57.875	0.14424	-29.3801139729396	-29.3801139729396\\
57.875	0.1479	-31.6426268308487	-31.6426268308487\\
57.875	0.15156	-34.0273713804076	-34.0273713804076\\
57.875	0.15522	-36.5343476216163	-36.5343476216163\\
57.875	0.15888	-39.1635555544749	-39.1635555544749\\
57.875	0.16254	-41.9149951789832	-41.9149951789832\\
57.875	0.1662	-44.7886664951413	-44.7886664951413\\
57.875	0.16986	-47.7845695029492	-47.7845695029492\\
57.875	0.17352	-50.9027042024069	-50.9027042024069\\
57.875	0.17718	-54.1430705935144	-54.1430705935144\\
57.875	0.18084	-57.5056686762716	-57.5056686762716\\
57.875	0.1845	-60.9904984506787	-60.9904984506787\\
57.875	0.18816	-64.5975599167356	-64.5975599167356\\
57.875	0.19182	-68.3268530744423	-68.3268530744423\\
57.875	0.19548	-72.1783779237987	-72.1783779237987\\
57.875	0.19914	-76.152134464805	-76.152134464805\\
57.875	0.2028	-80.2481226974611	-80.2481226974611\\
57.875	0.20646	-84.4663426217669	-84.4663426217669\\
57.875	0.21012	-88.8067942377226	-88.8067942377226\\
57.875	0.21378	-93.269477545328	-93.269477545328\\
57.875	0.21744	-97.8543925445832	-97.8543925445832\\
57.875	0.2211	-102.561539235488	-102.561539235488\\
57.875	0.22476	-107.390917618043	-107.390917618043\\
57.875	0.22842	-112.342527692248	-112.342527692248\\
57.875	0.23208	-117.416369458102	-117.416369458102\\
57.875	0.23574	-122.612442915606	-122.612442915606\\
57.875	0.2394	-127.93074806476	-127.93074806476\\
57.875	0.24306	-133.371284905564	-133.371284905564\\
57.875	0.24672	-138.934053438018	-138.934053438018\\
57.875	0.25038	-144.619053662121	-144.619053662121\\
57.875	0.25404	-150.426285577874	-150.426285577874\\
57.875	0.2577	-156.355749185277	-156.355749185277\\
57.875	0.26136	-162.40744448433	-162.40744448433\\
57.875	0.26502	-168.581371475032	-168.581371475032\\
57.875	0.26868	-174.877530157385	-174.877530157385\\
57.875	0.27234	-181.295920531387	-181.295920531387\\
57.875	0.276	-187.836542597039	-187.836542597039\\
58.25	0.093	-10.4949105651191	-10.4949105651191\\
58.25	0.09666	-11.0489585597131	-11.0489585597131\\
58.25	0.10032	-11.7252382459569	-11.7252382459569\\
58.25	0.10398	-12.5237496238505	-12.5237496238505\\
58.25	0.10764	-13.444492693394	-13.444492693394\\
58.25	0.1113	-14.4874674545871	-14.4874674545871\\
58.25	0.11496	-15.6526739074302	-15.6526739074302\\
58.25	0.11862	-16.9401120519229	-16.9401120519229\\
58.25	0.12228	-18.3497818880655	-18.3497818880655\\
58.25	0.12594	-19.8816834158579	-19.8816834158579\\
58.25	0.1296	-21.5358166353001	-21.5358166353001\\
58.25	0.13326	-23.312181546392	-23.312181546392\\
58.25	0.13692	-25.2107781491338	-25.2107781491338\\
58.25	0.14058	-27.2316064435254	-27.2316064435254\\
58.25	0.14424	-29.3746664295667	-29.3746664295667\\
58.25	0.1479	-31.6399581072579	-31.6399581072579\\
58.25	0.15156	-34.0274814765988	-34.0274814765988\\
58.25	0.15522	-36.5372365375895	-36.5372365375895\\
58.25	0.15888	-39.1692232902301	-39.1692232902301\\
58.25	0.16254	-41.9234417345204	-41.9234417345204\\
58.25	0.1662	-44.7998918704605	-44.7998918704605\\
58.25	0.16986	-47.7985736980504	-47.7985736980504\\
58.25	0.17352	-50.9194872172901	-50.9194872172901\\
58.25	0.17718	-54.1626324281796	-54.1626324281796\\
58.25	0.18084	-57.5280093307189	-57.5280093307189\\
58.25	0.1845	-61.015617924908	-61.015617924908\\
58.25	0.18816	-64.6254582107469	-64.6254582107469\\
58.25	0.19182	-68.3575301882356	-68.3575301882356\\
58.25	0.19548	-72.211833857374	-72.211833857374\\
58.25	0.19914	-76.1883692181623	-76.1883692181623\\
58.25	0.2028	-80.2871362706004	-80.2871362706004\\
58.25	0.20646	-84.5081350146882	-84.5081350146882\\
58.25	0.21012	-88.8513654504259	-88.8513654504259\\
58.25	0.21378	-93.3168275778133	-93.3168275778133\\
58.25	0.21744	-97.9045213968505	-97.9045213968505\\
58.25	0.2211	-102.614446907538	-102.614446907538\\
58.25	0.22476	-107.446604109874	-107.446604109874\\
58.25	0.22842	-112.400993003861	-112.400993003861\\
58.25	0.23208	-117.477613589497	-117.477613589497\\
58.25	0.23574	-122.676465866784	-122.676465866784\\
58.25	0.2394	-127.99754983572	-127.99754983572\\
58.25	0.24306	-133.440865496305	-133.440865496305\\
58.25	0.24672	-139.006412848541	-139.006412848541\\
58.25	0.25038	-144.694191892426	-144.694191892426\\
58.25	0.25404	-150.504202627962	-150.504202627962\\
58.25	0.2577	-156.436445055147	-156.436445055147\\
58.25	0.26136	-162.490919173981	-162.490919173981\\
58.25	0.26502	-168.667624984466	-168.667624984466\\
58.25	0.26868	-174.9665624866	-174.9665624866\\
58.25	0.27234	-181.387731680384	-181.387731680384\\
58.25	0.276	-187.931132565818	-187.931132565818\\
58.625	0.093	-10.4601559197156	-10.4601559197156\\
58.625	0.09666	-11.0169827340916	-11.0169827340916\\
58.625	0.10032	-11.6960412401175	-11.6960412401175\\
58.625	0.10398	-12.4973314377931	-12.4973314377931\\
58.625	0.10764	-13.4208533271185	-13.4208533271185\\
58.625	0.1113	-14.4666069080937	-14.4666069080937\\
58.625	0.11496	-15.6345921807187	-15.6345921807187\\
58.625	0.11862	-16.9248091449935	-16.9248091449935\\
58.625	0.12228	-18.3372578009181	-18.3372578009181\\
58.625	0.12594	-19.8719381484925	-19.8719381484925\\
58.625	0.1296	-21.5288501877167	-21.5288501877167\\
58.625	0.13326	-23.3079939185906	-23.3079939185906\\
58.625	0.13692	-25.2093693411144	-25.2093693411144\\
58.625	0.14058	-27.232976455288	-27.232976455288\\
58.625	0.14424	-29.3788152611114	-29.3788152611114\\
58.625	0.1479	-31.6468857585845	-31.6468857585845\\
58.625	0.15156	-34.0371879477075	-34.0371879477075\\
58.625	0.15522	-36.5497218284802	-36.5497218284802\\
58.625	0.15888	-39.1844874009027	-39.1844874009027\\
58.625	0.16254	-41.9414846649751	-41.9414846649751\\
58.625	0.1662	-44.8207136206972	-44.8207136206972\\
58.625	0.16986	-47.8221742680691	-47.8221742680691\\
58.625	0.17352	-50.9458666070908	-50.9458666070908\\
58.625	0.17718	-54.1917906377623	-54.1917906377623\\
58.625	0.18084	-57.5599463600836	-57.5599463600836\\
58.625	0.1845	-61.0503337740547	-61.0503337740547\\
58.625	0.18816	-64.6629528796756	-64.6629528796756\\
58.625	0.19182	-68.3978036769463	-68.3978036769463\\
58.625	0.19548	-72.2548861658668	-72.2548861658668\\
58.625	0.19914	-76.234200346437	-76.234200346437\\
58.625	0.2028	-80.3357462186571	-80.3357462186571\\
58.625	0.20646	-84.559523782527	-84.559523782527\\
58.625	0.21012	-88.9055330380467	-88.9055330380467\\
58.625	0.21378	-93.3737739852161	-93.3737739852161\\
58.625	0.21744	-97.9642466240353	-97.9642466240353\\
58.625	0.2211	-102.676950954504	-102.676950954504\\
58.625	0.22476	-107.511886976623	-107.511886976623\\
58.625	0.22842	-112.469054690392	-112.469054690392\\
58.625	0.23208	-117.54845409581	-117.54845409581\\
58.625	0.23574	-122.750085192879	-122.750085192879\\
58.625	0.2394	-128.073947981596	-128.073947981596\\
58.625	0.24306	-133.520042461964	-133.520042461964\\
58.625	0.24672	-139.088368633982	-139.088368633982\\
58.625	0.25038	-144.778926497649	-144.778926497649\\
58.625	0.25404	-150.591716052967	-150.591716052967\\
58.625	0.2577	-156.526737299933	-156.526737299933\\
58.625	0.26136	-162.58399023855	-162.58399023855\\
58.625	0.26502	-168.763474868817	-168.763474868817\\
58.625	0.26868	-175.065191190733	-175.065191190733\\
58.625	0.27234	-181.489139204299	-181.489139204299\\
58.625	0.276	-188.035318909515	-188.035318909515\\
59	0.093	-10.4349976492296	-10.4349976492296\\
59	0.09666	-10.9946032833877	-10.9946032833877\\
59	0.10032	-11.6764406091955	-11.6764406091955\\
59	0.10398	-12.4805096266531	-12.4805096266531\\
59	0.10764	-13.4068103357606	-13.4068103357606\\
59	0.1113	-14.4553427365177	-14.4553427365177\\
59	0.11496	-15.6261068289248	-15.6261068289248\\
59	0.11862	-16.9191026129816	-16.9191026129816\\
59	0.12228	-18.3343300886882	-18.3343300886882\\
59	0.12594	-19.8717892560446	-19.8717892560446\\
59	0.1296	-21.5314801150508	-21.5314801150508\\
59	0.13326	-23.3134026657067	-23.3134026657067\\
59	0.13692	-25.2175569080125	-25.2175569080125\\
59	0.14058	-27.2439428419681	-27.2439428419681\\
59	0.14424	-29.3925604675735	-29.3925604675735\\
59	0.1479	-31.6634097848287	-31.6634097848287\\
59	0.15156	-34.0564907937336	-34.0564907937336\\
59	0.15522	-36.5718034942883	-36.5718034942883\\
59	0.15888	-39.2093478864929	-39.2093478864929\\
59	0.16254	-41.9691239703473	-41.9691239703473\\
59	0.1662	-44.8511317458514	-44.8511317458514\\
59	0.16986	-47.8553712130053	-47.8553712130053\\
59	0.17352	-50.981842371809	-50.981842371809\\
59	0.17718	-54.2305452222625	-54.2305452222625\\
59	0.18084	-57.6014797643658	-57.6014797643658\\
59	0.1845	-61.0946459981189	-61.0946459981189\\
59	0.18816	-64.7100439235219	-64.7100439235219\\
59	0.19182	-68.4476735405746	-68.4476735405746\\
59	0.19548	-72.307534849277	-72.307534849277\\
59	0.19914	-76.2896278496293	-76.2896278496293\\
59	0.2028	-80.3939525416314	-80.3939525416314\\
59	0.20646	-84.6205089252833	-84.6205089252833\\
59	0.21012	-88.969297000585	-88.969297000585\\
59	0.21378	-93.4403167675364	-93.4403167675364\\
59	0.21744	-98.0335682261377	-98.0335682261377\\
59	0.2211	-102.749051376389	-102.749051376389\\
59	0.22476	-107.58676621829	-107.58676621829\\
59	0.22842	-112.54671275184	-112.54671275184\\
59	0.23208	-117.628890977041	-117.628890977041\\
59	0.23574	-122.833300893891	-122.833300893891\\
59	0.2394	-128.159942502391	-128.159942502391\\
59	0.24306	-133.608815802541	-133.608815802541\\
59	0.24672	-139.17992079434	-139.17992079434\\
59	0.25038	-144.87325747779	-144.87325747779\\
59	0.25404	-150.688825852889	-150.688825852889\\
59	0.2577	-156.626625919638	-156.626625919638\\
59	0.26136	-162.686657678037	-162.686657678037\\
59	0.26502	-168.868921128085	-168.868921128085\\
59	0.26868	-175.173416269784	-175.173416269784\\
59	0.27234	-181.600143103132	-181.600143103132\\
59	0.276	-188.14910162813	-188.14910162813\\
59.375	0.093	-10.4194357536611	-10.4194357536611\\
59.375	0.09666	-10.9818202076011	-10.9818202076011\\
59.375	0.10032	-11.666436353191	-11.666436353191\\
59.375	0.10398	-12.4732841904306	-12.4732841904306\\
59.375	0.10764	-13.40236371932	-13.40236371932\\
59.375	0.1113	-14.4536749398593	-14.4536749398593\\
59.375	0.11496	-15.6272178520483	-15.6272178520483\\
59.375	0.11862	-16.9229924558871	-16.9229924558871\\
59.375	0.12228	-18.3409987513758	-18.3409987513758\\
59.375	0.12594	-19.8812367385142	-19.8812367385142\\
59.375	0.1296	-21.5437064173023	-21.5437064173023\\
59.375	0.13326	-23.3284077877404	-23.3284077877404\\
59.375	0.13692	-25.2353408498281	-25.2353408498281\\
59.375	0.14058	-27.2645056035657	-27.2645056035657\\
59.375	0.14424	-29.4159020489531	-29.4159020489531\\
59.375	0.1479	-31.6895301859903	-31.6895301859903\\
59.375	0.15156	-34.0853900146772	-34.0853900146772\\
59.375	0.15522	-36.603481535014	-36.603481535014\\
59.375	0.15888	-39.2438047470006	-39.2438047470006\\
59.375	0.16254	-42.0063596506369	-42.0063596506369\\
59.375	0.1662	-44.8911462459231	-44.8911462459231\\
59.375	0.16986	-47.898164532859	-47.898164532859\\
59.375	0.17352	-51.0274145114447	-51.0274145114447\\
59.375	0.17718	-54.2788961816802	-54.2788961816802\\
59.375	0.18084	-57.6526095435655	-57.6526095435655\\
59.375	0.1845	-61.1485545971007	-61.1485545971007\\
59.375	0.18816	-64.7667313422856	-64.7667313422856\\
59.375	0.19182	-68.5071397791203	-68.5071397791203\\
59.375	0.19548	-72.3697799076048	-72.3697799076048\\
59.375	0.19914	-76.3546517277391	-76.3546517277391\\
59.375	0.2028	-80.4617552395232	-80.4617552395232\\
59.375	0.20646	-84.6910904429571	-84.6910904429571\\
59.375	0.21012	-89.0426573380407	-89.0426573380407\\
59.375	0.21378	-93.5164559247742	-93.5164559247742\\
59.375	0.21744	-98.1124862031574	-98.1124862031574\\
59.375	0.2211	-102.830748173191	-102.830748173191\\
59.375	0.22476	-107.671241834873	-107.671241834873\\
59.375	0.22842	-112.633967188206	-112.633967188206\\
59.375	0.23208	-117.718924233188	-117.718924233188\\
59.375	0.23574	-122.926112969821	-122.926112969821\\
59.375	0.2394	-128.255533398103	-128.255533398103\\
59.375	0.24306	-133.707185518035	-133.707185518035\\
59.375	0.24672	-139.281069329616	-139.281069329616\\
59.375	0.25038	-144.977184832848	-144.977184832848\\
59.375	0.25404	-150.795532027729	-150.795532027729\\
59.375	0.2577	-156.73611091426	-156.73611091426\\
59.375	0.26136	-162.798921492441	-162.798921492441\\
59.375	0.26502	-168.983963762271	-168.983963762271\\
59.375	0.26868	-175.291237723752	-175.291237723752\\
59.375	0.27234	-181.720743376882	-181.720743376882\\
59.375	0.276	-188.272480721662	-188.272480721662\\
59.75	0.093	-10.4134702330102	-10.4134702330102\\
59.75	0.09666	-10.9786335067322	-10.9786335067322\\
59.75	0.10032	-11.6660284721041	-11.6660284721041\\
59.75	0.10398	-12.4756551291257	-12.4756551291257\\
59.75	0.10764	-13.4075134777971	-13.4075134777971\\
59.75	0.1113	-14.4616035181184	-14.4616035181184\\
59.75	0.11496	-15.6379252500894	-15.6379252500894\\
59.75	0.11862	-16.9364786737102	-16.9364786737102\\
59.75	0.12228	-18.3572637889809	-18.3572637889809\\
59.75	0.12594	-19.9002805959013	-19.9002805959013\\
59.75	0.1296	-21.5655290944715	-21.5655290944715\\
59.75	0.13326	-23.3530092846915	-23.3530092846915\\
59.75	0.13692	-25.2627211665612	-25.2627211665612\\
59.75	0.14058	-27.2946647400808	-27.2946647400808\\
59.75	0.14424	-29.4488400052502	-29.4488400052502\\
59.75	0.1479	-31.7252469620694	-31.7252469620694\\
59.75	0.15156	-34.1238856105384	-34.1238856105384\\
59.75	0.15522	-36.6447559506572	-36.6447559506572\\
59.75	0.15888	-39.2878579824258	-39.2878579824258\\
59.75	0.16254	-42.0531917058441	-42.0531917058441\\
59.75	0.1662	-44.9407571209123	-44.9407571209123\\
59.75	0.16986	-47.9505542276302	-47.9505542276302\\
59.75	0.17352	-51.0825830259979	-51.0825830259979\\
59.75	0.17718	-54.3368435160154	-54.3368435160154\\
59.75	0.18084	-57.7133356976828	-57.7133356976828\\
59.75	0.1845	-61.2120595709999	-61.2120595709999\\
59.75	0.18816	-64.8330151359668	-64.8330151359668\\
59.75	0.19182	-68.5762023925836	-68.5762023925836\\
59.75	0.19548	-72.44162134085	-72.44162134085\\
59.75	0.19914	-76.4292719807663	-76.4292719807663\\
59.75	0.2028	-80.5391543123325	-80.5391543123325\\
59.75	0.20646	-84.7712683355484	-84.7712683355484\\
59.75	0.21012	-89.1256140504141	-89.1256140504141\\
59.75	0.21378	-93.6021914569295	-93.6021914569295\\
59.75	0.21744	-98.2010005550948	-98.2010005550948\\
59.75	0.2211	-102.92204134491	-102.92204134491\\
59.75	0.22476	-107.765313826375	-107.765313826375\\
59.75	0.22842	-112.730817999489	-112.730817999489\\
59.75	0.23208	-117.818553864254	-117.818553864254\\
59.75	0.23574	-123.028521420668	-123.028521420668\\
59.75	0.2394	-128.360720668732	-128.360720668732\\
59.75	0.24306	-133.815151608446	-133.815151608446\\
59.75	0.24672	-139.39181423981	-139.39181423981\\
59.75	0.25038	-145.090708562823	-145.090708562823\\
59.75	0.25404	-150.911834577486	-150.911834577486\\
59.75	0.2577	-156.855192283799	-156.855192283799\\
59.75	0.26136	-162.920781681762	-162.920781681762\\
59.75	0.26502	-169.108602771375	-169.108602771375\\
59.75	0.26868	-175.418655552637	-175.418655552637\\
59.75	0.27234	-181.850940025549	-181.850940025549\\
59.75	0.276	-188.405456190111	-188.405456190111\\
60.125	0.093	-10.4171010872766	-10.4171010872766\\
60.125	0.09666	-10.9850431807806	-10.9850431807806\\
60.125	0.10032	-11.6752169659345	-11.6752169659345\\
60.125	0.10398	-12.4876224427381	-12.4876224427381\\
60.125	0.10764	-13.4222596111916	-13.4222596111916\\
60.125	0.1113	-14.4791284712948	-14.4791284712948\\
60.125	0.11496	-15.6582290230479	-15.6582290230479\\
60.125	0.11862	-16.9595612664507	-16.9595612664507\\
60.125	0.12228	-18.3831252015034	-18.3831252015034\\
60.125	0.12594	-19.9289208282058	-19.9289208282058\\
60.125	0.1296	-21.596948146558	-21.596948146558\\
60.125	0.13326	-23.38720715656	-23.38720715656\\
60.125	0.13692	-25.2996978582118	-25.2996978582118\\
60.125	0.14058	-27.3344202515134	-27.3344202515134\\
60.125	0.14424	-29.4913743364648	-29.4913743364648\\
60.125	0.1479	-31.770560113066	-31.770560113066\\
60.125	0.15156	-34.171977581317	-34.171977581317\\
60.125	0.15522	-36.6956267412178	-36.6956267412178\\
60.125	0.15888	-39.3415075927683	-39.3415075927683\\
60.125	0.16254	-42.1096201359687	-42.1096201359687\\
60.125	0.1662	-44.9999643708188	-44.9999643708188\\
60.125	0.16986	-48.0125402973188	-48.0125402973188\\
60.125	0.17352	-51.1473479154686	-51.1473479154686\\
60.125	0.17718	-54.4043872252681	-54.4043872252681\\
60.125	0.18084	-57.7836582267174	-57.7836582267174\\
60.125	0.1845	-61.2851609198165	-61.2851609198165\\
60.125	0.18816	-64.9088953045655	-64.9088953045655\\
60.125	0.19182	-68.6548613809642	-68.6548613809642\\
60.125	0.19548	-72.5230591490127	-72.5230591490127\\
60.125	0.19914	-76.513488608711	-76.513488608711\\
60.125	0.2028	-80.6261497600591	-80.6261497600591\\
60.125	0.20646	-84.8610426030571	-84.8610426030571\\
60.125	0.21012	-89.2181671377048	-89.2181671377048\\
60.125	0.21378	-93.6975233640023	-93.6975233640023\\
60.125	0.21744	-98.2991112819495	-98.2991112819495\\
60.125	0.2211	-103.022930891547	-103.022930891547\\
60.125	0.22476	-107.868982192793	-107.868982192793\\
60.125	0.22842	-112.83726518569	-112.83726518569\\
60.125	0.23208	-117.927779870237	-117.927779870237\\
60.125	0.23574	-123.140526246433	-123.140526246433\\
60.125	0.2394	-128.475504314279	-128.475504314279\\
60.125	0.24306	-133.932714073775	-133.932714073775\\
60.125	0.24672	-139.51215552492	-139.51215552492\\
60.125	0.25038	-145.213828667716	-145.213828667716\\
60.125	0.25404	-151.037733502161	-151.037733502161\\
60.125	0.2577	-156.983870028256	-156.983870028256\\
60.125	0.26136	-163.052238246001	-163.052238246001\\
60.125	0.26502	-169.242838155395	-169.242838155395\\
60.125	0.26868	-175.55566975644	-175.55566975644\\
60.125	0.27234	-181.990733049134	-181.990733049134\\
60.125	0.276	-188.548028033478	-188.548028033478\\
60.5	0.093	-10.4303283164606	-10.4303283164606\\
60.5	0.09666	-11.0010492297466	-11.0010492297466\\
60.5	0.10032	-11.6940018346825	-11.6940018346825\\
60.5	0.10398	-12.5091861312682	-12.5091861312682\\
60.5	0.10764	-13.4466021195036	-13.4466021195036\\
60.5	0.1113	-14.5062497993889	-14.5062497993889\\
60.5	0.11496	-15.6881291709239	-15.6881291709239\\
60.5	0.11862	-16.9922402341087	-16.9922402341087\\
60.5	0.12228	-18.4185829889434	-18.4185829889434\\
60.5	0.12594	-19.9671574354278	-19.9671574354278\\
60.5	0.1296	-21.637963573562	-21.637963573562\\
60.5	0.13326	-23.4310014033461	-23.4310014033461\\
60.5	0.13692	-25.3462709247799	-25.3462709247799\\
60.5	0.14058	-27.3837721378635	-27.3837721378635\\
60.5	0.14424	-29.5435050425969	-29.5435050425969\\
60.5	0.1479	-31.8254696389801	-31.8254696389801\\
60.5	0.15156	-34.2296659270131	-34.2296659270131\\
60.5	0.15522	-36.7560939066959	-36.7560939066959\\
60.5	0.15888	-39.4047535780285	-39.4047535780285\\
60.5	0.16254	-42.1756449410109	-42.1756449410109\\
60.5	0.1662	-45.068767995643	-45.068767995643\\
60.5	0.16986	-48.084122741925	-48.084122741925\\
60.5	0.17352	-51.2217091798567	-51.2217091798567\\
60.5	0.17718	-54.4815273094383	-54.4815273094383\\
60.5	0.18084	-57.8635771306696	-57.8635771306696\\
60.5	0.1845	-61.3678586435508	-61.3678586435508\\
60.5	0.18816	-64.9943718480817	-64.9943718480817\\
60.5	0.19182	-68.7431167442625	-68.7431167442625\\
60.5	0.19548	-72.6140933320929	-72.6140933320929\\
60.5	0.19914	-76.6073016115733	-76.6073016115733\\
60.5	0.2028	-80.7227415827034	-80.7227415827034\\
60.5	0.20646	-84.9604132454834	-84.9604132454834\\
60.5	0.21012	-89.320316599913	-89.320316599913\\
60.5	0.21378	-93.8024516459925	-93.8024516459925\\
60.5	0.21744	-98.4068183837218	-98.4068183837218\\
60.5	0.2211	-103.133416813101	-103.133416813101\\
60.5	0.22476	-107.98224693413	-107.98224693413\\
60.5	0.22842	-112.953308746808	-112.953308746808\\
60.5	0.23208	-118.046602251137	-118.046602251137\\
60.5	0.23574	-123.262127447115	-123.262127447115\\
60.5	0.2394	-128.599884334743	-128.599884334743\\
60.5	0.24306	-134.059872914021	-134.059872914021\\
60.5	0.24672	-139.642093184949	-139.642093184949\\
60.5	0.25038	-145.346545147526	-145.346545147526\\
60.5	0.25404	-151.173228801753	-151.173228801753\\
60.5	0.2577	-157.12214414763	-157.12214414763\\
60.5	0.26136	-163.193291185157	-163.193291185157\\
60.5	0.26502	-169.386669914334	-169.386669914334\\
60.5	0.26868	-175.70228033516	-175.70228033516\\
60.5	0.27234	-182.140122447636	-182.140122447636\\
60.5	0.276	-188.700196251762	-188.700196251762\\
60.875	0.093	-10.453151920562	-10.453151920562\\
60.875	0.09666	-11.02665165363	-11.02665165363\\
60.875	0.10032	-11.722383078348	-11.722383078348\\
60.875	0.10398	-12.5403461947156	-12.5403461947156\\
60.875	0.10764	-13.4805410027331	-13.4805410027331\\
60.875	0.1113	-14.5429675024004	-14.5429675024004\\
60.875	0.11496	-15.7276256937174	-15.7276256937174\\
60.875	0.11862	-17.0345155766842	-17.0345155766842\\
60.875	0.12228	-18.4636371513009	-18.4636371513009\\
60.875	0.12594	-20.0149904175673	-20.0149904175673\\
60.875	0.1296	-21.6885753754836	-21.6885753754836\\
60.875	0.13326	-23.4843920250496	-23.4843920250496\\
60.875	0.13692	-25.4024403662654	-25.4024403662654\\
60.875	0.14058	-27.4427203991311	-27.4427203991311\\
60.875	0.14424	-29.6052321236465	-29.6052321236465\\
60.875	0.1479	-31.8899755398117	-31.8899755398117\\
60.875	0.15156	-34.2969506476267	-34.2969506476267\\
60.875	0.15522	-36.8261574470915	-36.8261574470915\\
60.875	0.15888	-39.4775959382061	-39.4775959382061\\
60.875	0.16254	-42.2512661209704	-42.2512661209704\\
60.875	0.1662	-45.1471679953846	-45.1471679953846\\
60.875	0.16986	-48.1653015614486	-48.1653015614486\\
60.875	0.17352	-51.3056668191624	-51.3056668191624\\
60.875	0.17718	-54.5682637685259	-54.5682637685259\\
60.875	0.18084	-57.9530924095392	-57.9530924095392\\
60.875	0.1845	-61.4601527422024	-61.4601527422024\\
60.875	0.18816	-65.0894447665154	-65.0894447665154\\
60.875	0.19182	-68.8409684824781	-68.8409684824781\\
60.875	0.19548	-72.7147238900906	-72.7147238900906\\
60.875	0.19914	-76.710710989353	-76.710710989353\\
60.875	0.2028	-80.8289297802651	-80.8289297802651\\
60.875	0.20646	-85.0693802628271	-85.0693802628271\\
60.875	0.21012	-89.4320624370387	-89.4320624370387\\
60.875	0.21378	-93.9169763029002	-93.9169763029002\\
60.875	0.21744	-98.5241218604115	-98.5241218604115\\
60.875	0.2211	-103.253499109573	-103.253499109573\\
60.875	0.22476	-108.105108050384	-108.105108050384\\
60.875	0.22842	-113.078948682844	-113.078948682844\\
60.875	0.23208	-118.175021006955	-118.175021006955\\
60.875	0.23574	-123.393325022715	-123.393325022715\\
60.875	0.2394	-128.733860730125	-128.733860730125\\
60.875	0.24306	-134.196628129185	-134.196628129185\\
60.875	0.24672	-139.781627219895	-139.781627219895\\
60.875	0.25038	-145.488858002254	-145.488858002254\\
60.875	0.25404	-151.318320476263	-151.318320476263\\
60.875	0.2577	-157.270014641922	-157.270014641922\\
60.875	0.26136	-163.343940499231	-163.343940499231\\
60.875	0.26502	-169.54009804819	-169.54009804819\\
60.875	0.26868	-175.858487288798	-175.858487288798\\
60.875	0.27234	-182.299108221056	-182.299108221056\\
60.875	0.276	-188.861960844964	-188.861960844964\\
61.25	0.093	-10.485571899581	-10.485571899581\\
61.25	0.09666	-11.061850452431	-11.061850452431\\
61.25	0.10032	-11.7603606969309	-11.7603606969309\\
61.25	0.10398	-12.5811026330806	-12.5811026330806\\
61.25	0.10764	-13.5240762608801	-13.5240762608801\\
61.25	0.1113	-14.5892815803294	-14.5892815803294\\
61.25	0.11496	-15.7767185914284	-15.7767185914284\\
61.25	0.11862	-17.0863872941773	-17.0863872941773\\
61.25	0.12228	-18.518287688576	-18.518287688576\\
61.25	0.12594	-20.0724197746244	-20.0724197746244\\
61.25	0.1296	-21.7487835523226	-21.7487835523226\\
61.25	0.13326	-23.5473790216707	-23.5473790216707\\
61.25	0.13692	-25.4682061826685	-25.4682061826685\\
61.25	0.14058	-27.5112650353161	-27.5112650353161\\
61.25	0.14424	-29.6765555796136	-29.6765555796136\\
61.25	0.1479	-31.9640778155608	-31.9640778155608\\
61.25	0.15156	-34.3738317431578	-34.3738317431578\\
61.25	0.15522	-36.9058173624046	-36.9058173624046\\
61.25	0.15888	-39.5600346733012	-39.5600346733012\\
61.25	0.16254	-42.3364836758476	-42.3364836758476\\
61.25	0.1662	-45.2351643700438	-45.2351643700438\\
61.25	0.16986	-48.2560767558898	-48.2560767558898\\
61.25	0.17352	-51.3992208333855	-51.3992208333855\\
61.25	0.17718	-54.6645966025311	-54.6645966025311\\
61.25	0.18084	-58.0522040633264	-58.0522040633264\\
61.25	0.1845	-61.5620432157716	-61.5620432157716\\
61.25	0.18816	-65.1941140598666	-65.1941140598666\\
61.25	0.19182	-68.9484165956113	-68.9484165956113\\
61.25	0.19548	-72.8249508230058	-72.8249508230058\\
61.25	0.19914	-76.8237167420502	-76.8237167420502\\
61.25	0.2028	-80.9447143527443	-80.9447143527443\\
61.25	0.20646	-85.1879436550883	-85.1879436550883\\
61.25	0.21012	-89.553404649082	-89.553404649082\\
61.25	0.21378	-94.0410973347255	-94.0410973347255\\
61.25	0.21744	-98.6510217120188	-98.6510217120188\\
61.25	0.2211	-103.383177780962	-103.383177780962\\
61.25	0.22476	-108.237565541555	-108.237565541555\\
61.25	0.22842	-113.214184993797	-113.214184993797\\
61.25	0.23208	-118.31303613769	-118.31303613769\\
61.25	0.23574	-123.534118973232	-123.534118973232\\
61.25	0.2394	-128.877433500424	-128.877433500424\\
61.25	0.24306	-134.342979719266	-134.342979719266\\
61.25	0.24672	-139.930757629758	-139.930757629758\\
61.25	0.25038	-145.640767231899	-145.640767231899\\
61.25	0.25404	-151.473008525691	-151.473008525691\\
61.25	0.2577	-157.427481511132	-157.427481511132\\
61.25	0.26136	-163.504186188222	-163.504186188222\\
61.25	0.26502	-169.703122556963	-169.703122556963\\
61.25	0.26868	-176.024290617354	-176.024290617354\\
61.25	0.27234	-182.467690369394	-182.467690369394\\
61.25	0.276	-189.033321813084	-189.033321813084\\
61.625	0.093	-10.5275882535174	-10.5275882535174\\
61.625	0.09666	-11.1066456261495	-11.1066456261495\\
61.625	0.10032	-11.8079346904314	-11.8079346904314\\
61.625	0.10398	-12.6314554463631	-12.6314554463631\\
61.625	0.10764	-13.5772078939446	-13.5772078939446\\
61.625	0.1113	-14.6451920331759	-14.6451920331759\\
61.625	0.11496	-15.8354078640569	-15.8354078640569\\
61.625	0.11862	-17.1478553865878	-17.1478553865878\\
61.625	0.12228	-18.5825346007685	-18.5825346007685\\
61.625	0.12594	-20.1394455065989	-20.1394455065989\\
61.625	0.1296	-21.8185881040792	-21.8185881040792\\
61.625	0.13326	-23.6199623932092	-23.6199623932092\\
61.625	0.13692	-25.543568373989	-25.543568373989\\
61.625	0.14058	-27.5894060464187	-27.5894060464187\\
61.625	0.14424	-29.7574754104982	-29.7574754104982\\
61.625	0.1479	-32.0477764662273	-32.0477764662273\\
61.625	0.15156	-34.4603092136064	-34.4603092136064\\
61.625	0.15522	-36.9950736526352	-36.9950736526352\\
61.625	0.15888	-39.6520697833138	-39.6520697833138\\
61.625	0.16254	-42.4312976056422	-42.4312976056422\\
61.625	0.1662	-45.3327571196204	-45.3327571196204\\
61.625	0.16986	-48.3564483252484	-48.3564483252484\\
61.625	0.17352	-51.5023712225262	-51.5023712225262\\
61.625	0.17718	-54.7705258114537	-54.7705258114537\\
61.625	0.18084	-58.1609120920311	-58.1609120920311\\
61.625	0.1845	-61.6735300642583	-61.6735300642583\\
61.625	0.18816	-65.3083797281353	-65.3083797281353\\
61.625	0.19182	-69.065461083662	-69.065461083662\\
61.625	0.19548	-72.9447741308386	-72.9447741308386\\
61.625	0.19914	-76.9463188696649	-76.9463188696649\\
61.625	0.2028	-81.0700953001411	-81.0700953001411\\
61.625	0.20646	-85.316103422267	-85.316103422267\\
61.625	0.21012	-89.6843432360427	-89.6843432360427\\
61.625	0.21378	-94.1748147414682	-94.1748147414682\\
61.625	0.21744	-98.7875179385435	-98.7875179385435\\
61.625	0.2211	-103.522452827269	-103.522452827269\\
61.625	0.22476	-108.379619407644	-108.379619407644\\
61.625	0.22842	-113.359017679668	-113.359017679668\\
61.625	0.23208	-118.460647643343	-118.460647643343\\
61.625	0.23574	-123.684509298667	-123.684509298667\\
61.625	0.2394	-129.030602645641	-129.030602645641\\
61.625	0.24306	-134.498927684265	-134.498927684265\\
61.625	0.24672	-140.089484414539	-140.089484414539\\
61.625	0.25038	-145.802272836462	-145.802272836462\\
61.625	0.25404	-151.637292950035	-151.637292950035\\
61.625	0.2577	-157.594544755259	-157.594544755259\\
61.625	0.26136	-163.674028252131	-163.674028252131\\
61.625	0.26502	-169.875743440654	-169.875743440654\\
61.625	0.26868	-176.199690320826	-176.199690320826\\
61.625	0.27234	-182.645868892649	-182.645868892649\\
61.625	0.276	-189.214279156121	-189.214279156121\\
62	0.093	-10.5792009823713	-10.5792009823713\\
62	0.09666	-11.1610371747854	-11.1610371747854\\
62	0.10032	-11.8651050588493	-11.8651050588493\\
62	0.10398	-12.691404634563	-12.691404634563\\
62	0.10764	-13.6399359019265	-13.6399359019265\\
62	0.1113	-14.7106988609398	-14.7106988609398\\
62	0.11496	-15.9036935116028	-15.9036935116028\\
62	0.11862	-17.2189198539157	-17.2189198539157\\
62	0.12228	-18.6563778878784	-18.6563778878784\\
62	0.12594	-20.2160676134909	-20.2160676134909\\
62	0.1296	-21.8979890307531	-21.8979890307531\\
62	0.13326	-23.7021421396652	-23.7021421396652\\
62	0.13692	-25.628526940227	-25.628526940227\\
62	0.14058	-27.6771434324387	-27.6771434324387\\
62	0.14424	-29.8479916163001	-29.8479916163001\\
62	0.1479	-32.1410714918114	-32.1410714918114\\
62	0.15156	-34.5563830589724	-34.5563830589724\\
62	0.15522	-37.0939263177832	-37.0939263177832\\
62	0.15888	-39.7537012682439	-39.7537012682439\\
62	0.16254	-42.5357079103543	-42.5357079103543\\
62	0.1662	-45.4399462441145	-45.4399462441145\\
62	0.16986	-48.4664162695245	-48.4664162695245\\
62	0.17352	-51.6151179865843	-51.6151179865843\\
62	0.17718	-54.8860513952938	-54.8860513952938\\
62	0.18084	-58.2792164956532	-58.2792164956532\\
62	0.1845	-61.7946132876624	-61.7946132876624\\
62	0.18816	-65.4322417713214	-65.4322417713214\\
62	0.19182	-69.1921019466301	-69.1921019466301\\
62	0.19548	-73.0741938135887	-73.0741938135887\\
62	0.19914	-77.078517372197	-77.078517372197\\
62	0.2028	-81.2050726224552	-81.2050726224552\\
62	0.20646	-85.4538595643632	-85.4538595643632\\
62	0.21012	-89.8248781979209	-89.8248781979209\\
62	0.21378	-94.3181285231284	-94.3181285231284\\
62	0.21744	-98.9336105399857	-98.9336105399857\\
62	0.2211	-103.671324248493	-103.671324248493\\
62	0.22476	-108.53126964865	-108.53126964865\\
62	0.22842	-113.513446740456	-113.513446740456\\
62	0.23208	-118.617855523913	-118.617855523913\\
62	0.23574	-123.844495999019	-123.844495999019\\
62	0.2394	-129.193368165775	-129.193368165775\\
62	0.24306	-134.664472024181	-134.664472024181\\
62	0.24672	-140.257807574237	-140.257807574237\\
62	0.25038	-145.973374815942	-145.973374815942\\
62	0.25404	-151.811173749298	-151.811173749298\\
62	0.2577	-157.771204374303	-157.771204374303\\
62	0.26136	-163.853466690958	-163.853466690958\\
62	0.26502	-170.057960699262	-170.057960699262\\
62	0.26868	-176.384686399217	-176.384686399217\\
62	0.27234	-182.833643790821	-182.833643790821\\
62	0.276	-189.404832874075	-189.404832874075\\
62.375	0.093	-10.6404100861427	-10.6404100861427\\
62.375	0.09666	-11.2250250983388	-11.2250250983388\\
62.375	0.10032	-11.9318718021848	-11.9318718021848\\
62.375	0.10398	-12.7609501976805	-12.7609501976805\\
62.375	0.10764	-13.712260284826	-13.712260284826\\
62.375	0.1113	-14.7858020636213	-14.7858020636213\\
62.375	0.11496	-15.9815755340664	-15.9815755340664\\
62.375	0.11862	-17.2995806961613	-17.2995806961613\\
62.375	0.12228	-18.739817549906	-18.739817549906\\
62.375	0.12594	-20.3022860953004	-20.3022860953004\\
62.375	0.1296	-21.9869863323447	-21.9869863323447\\
62.375	0.13326	-23.7939182610387	-23.7939182610387\\
62.375	0.13692	-25.7230818813826	-25.7230818813826\\
62.375	0.14058	-27.7744771933763	-27.7744771933763\\
62.375	0.14424	-29.9481041970197	-29.9481041970197\\
62.375	0.1479	-32.243962892313	-32.243962892313\\
62.375	0.15156	-34.662053279256	-34.662053279256\\
62.375	0.15522	-37.2023753578488	-37.2023753578488\\
62.375	0.15888	-39.8649291280915	-39.8649291280915\\
62.375	0.16254	-42.6497145899839	-42.6497145899839\\
62.375	0.1662	-45.5567317435261	-45.5567317435261\\
62.375	0.16986	-48.5859805887181	-48.5859805887181\\
62.375	0.17352	-51.7374611255599	-51.7374611255599\\
62.375	0.17718	-55.0111733540515	-55.0111733540515\\
62.375	0.18084	-58.4071172741929	-58.4071172741929\\
62.375	0.1845	-61.925292885984	-61.925292885984\\
62.375	0.18816	-65.5657001894251	-65.5657001894251\\
62.375	0.19182	-69.3283391845158	-69.3283391845158\\
62.375	0.19548	-73.2132098712564	-73.2132098712564\\
62.375	0.19914	-77.2203122496468	-77.2203122496468\\
62.375	0.2028	-81.3496463196869	-81.3496463196869\\
62.375	0.20646	-85.6012120813769	-85.6012120813769\\
62.375	0.21012	-89.9750095347167	-89.9750095347167\\
62.375	0.21378	-94.4710386797062	-94.4710386797062\\
62.375	0.21744	-99.0892995163455	-99.0892995163455\\
62.375	0.2211	-103.829792044635	-103.829792044635\\
62.375	0.22476	-108.692516264574	-108.692516264574\\
62.375	0.22842	-113.677472176162	-113.677472176162\\
62.375	0.23208	-118.784659779401	-118.784659779401\\
62.375	0.23574	-124.014079074289	-124.014079074289\\
62.375	0.2394	-129.365730060827	-129.365730060827\\
62.375	0.24306	-134.839612739015	-134.839612739015\\
62.375	0.24672	-140.435727108853	-140.435727108853\\
62.375	0.25038	-146.15407317034	-146.15407317034\\
62.375	0.25404	-151.994650923478	-151.994650923478\\
62.375	0.2577	-157.957460368265	-157.957460368265\\
62.375	0.26136	-164.042501504701	-164.042501504701\\
62.375	0.26502	-170.249774332788	-170.249774332788\\
62.375	0.26868	-176.579278852525	-176.579278852525\\
62.375	0.27234	-183.031015063911	-183.031015063911\\
62.375	0.276	-189.604982966947	-189.604982966947\\
62.75	0.093	-10.7112155648316	-10.7112155648316\\
62.75	0.09666	-11.2986093968097	-11.2986093968097\\
62.75	0.10032	-12.0082349204377	-12.0082349204377\\
62.75	0.10398	-12.8400921357154	-12.8400921357154\\
62.75	0.10764	-13.7941810426428	-13.7941810426428\\
62.75	0.1113	-14.8705016412202	-14.8705016412202\\
62.75	0.11496	-16.0690539314473	-16.0690539314473\\
62.75	0.11862	-17.3898379133242	-17.3898379133242\\
62.75	0.12228	-18.8328535868509	-18.8328535868509\\
62.75	0.12594	-20.3981009520273	-20.3981009520273\\
62.75	0.1296	-22.0855800088536	-22.0855800088536\\
62.75	0.13326	-23.8952907573297	-23.8952907573297\\
62.75	0.13692	-25.8272331974555	-25.8272331974555\\
62.75	0.14058	-27.8814073292312	-27.8814073292312\\
62.75	0.14424	-30.0578131526567	-30.0578131526567\\
62.75	0.1479	-32.3564506677319	-32.3564506677319\\
62.75	0.15156	-34.777319874457	-34.777319874457\\
62.75	0.15522	-37.3204207728318	-37.3204207728318\\
62.75	0.15888	-39.9857533628565	-39.9857533628565\\
62.75	0.16254	-42.7733176445309	-42.7733176445309\\
62.75	0.1662	-45.6831136178551	-45.6831136178551\\
62.75	0.16986	-48.7151412828292	-48.7151412828292\\
62.75	0.17352	-51.8694006394529	-51.8694006394529\\
62.75	0.17718	-55.1458916877265	-55.1458916877265\\
62.75	0.18084	-58.5446144276499	-58.5446144276499\\
62.75	0.1845	-62.0655688592231	-62.0655688592231\\
62.75	0.18816	-65.7087549824462	-65.7087549824462\\
62.75	0.19182	-69.474172797319	-69.474172797319\\
62.75	0.19548	-73.3618223038415	-73.3618223038415\\
62.75	0.19914	-77.3717035020139	-77.3717035020139\\
62.75	0.2028	-81.5038163918361	-81.5038163918361\\
62.75	0.20646	-85.758160973308	-85.758160973308\\
62.75	0.21012	-90.1347372464298	-90.1347372464298\\
62.75	0.21378	-94.6335452112014	-94.6335452112014\\
62.75	0.21744	-99.2545848676226	-99.2545848676226\\
62.75	0.2211	-103.997856215694	-103.997856215694\\
62.75	0.22476	-108.863359255415	-108.863359255415\\
62.75	0.22842	-113.851093986785	-113.851093986785\\
62.75	0.23208	-118.961060409806	-118.961060409806\\
62.75	0.23574	-124.193258524476	-124.193258524476\\
62.75	0.2394	-129.547688330796	-129.547688330796\\
62.75	0.24306	-135.024349828766	-135.024349828766\\
62.75	0.24672	-140.623243018386	-140.623243018386\\
62.75	0.25038	-146.344367899656	-146.344367899656\\
62.75	0.25404	-152.187724472575	-152.187724472575\\
62.75	0.2577	-158.153312737144	-158.153312737144\\
62.75	0.26136	-164.241132693363	-164.241132693363\\
62.75	0.26502	-170.451184341231	-170.451184341231\\
62.75	0.26868	-176.78346768075	-176.78346768075\\
62.75	0.27234	-183.237982711918	-183.237982711918\\
62.75	0.276	-189.814729434736	-189.814729434736\\
63.125	0.093	-10.791617418438	-10.791617418438\\
63.125	0.09666	-11.3817900701981	-11.3817900701981\\
63.125	0.10032	-12.0941944136081	-12.0941944136081\\
63.125	0.10398	-12.9288304486678	-12.9288304486678\\
63.125	0.10764	-13.8856981753773	-13.8856981753773\\
63.125	0.1113	-14.9647975937367	-14.9647975937367\\
63.125	0.11496	-16.1661287037457	-16.1661287037457\\
63.125	0.11862	-17.4896915054046	-17.4896915054046\\
63.125	0.12228	-18.9354859987134	-18.9354859987134\\
63.125	0.12594	-20.5035121836718	-20.5035121836718\\
63.125	0.1296	-22.1937700602801	-22.1937700602801\\
63.125	0.13326	-24.0062596285382	-24.0062596285382\\
63.125	0.13692	-25.9409808884461	-25.9409808884461\\
63.125	0.14058	-27.9979338400038	-27.9979338400038\\
63.125	0.14424	-30.1771184832112	-30.1771184832112\\
63.125	0.1479	-32.4785348180685	-32.4785348180685\\
63.125	0.15156	-34.9021828445755	-34.9021828445755\\
63.125	0.15522	-37.4480625627324	-37.4480625627324\\
63.125	0.15888	-40.116173972539	-40.116173972539\\
63.125	0.16254	-42.9065170739955	-42.9065170739955\\
63.125	0.1662	-45.8190918671017	-45.8190918671017\\
63.125	0.16986	-48.8538983518577	-48.8538983518577\\
63.125	0.17352	-52.0109365282636	-52.0109365282636\\
63.125	0.17718	-55.2902063963191	-55.2902063963191\\
63.125	0.18084	-58.6917079560245	-58.6917079560245\\
63.125	0.1845	-62.2154412073798	-62.2154412073798\\
63.125	0.18816	-65.8614061503848	-65.8614061503848\\
63.125	0.19182	-69.6296027850395	-69.6296027850395\\
63.125	0.19548	-73.5200311113441	-73.5200311113441\\
63.125	0.19914	-77.5326911292985	-77.5326911292985\\
63.125	0.2028	-81.6675828389027	-81.6675828389027\\
63.125	0.20646	-85.9247062401567	-85.9247062401567\\
63.125	0.21012	-90.3040613330604	-90.3040613330604\\
63.125	0.21378	-94.805648117614	-94.805648117614\\
63.125	0.21744	-99.4294665938174	-99.4294665938174\\
63.125	0.2211	-104.175516761671	-104.175516761671\\
63.125	0.22476	-109.043798621173	-109.043798621173\\
63.125	0.22842	-114.034312172326	-114.034312172326\\
63.125	0.23208	-119.147057415129	-119.147057415129\\
63.125	0.23574	-124.382034349581	-124.382034349581\\
63.125	0.2394	-129.739242975683	-129.739242975683\\
63.125	0.24306	-135.218683293435	-135.218683293435\\
63.125	0.24672	-140.820355302837	-140.820355302837\\
63.125	0.25038	-146.544259003888	-146.544259003888\\
63.125	0.25404	-152.39039439659	-152.39039439659\\
63.125	0.2577	-158.358761480941	-158.358761480941\\
63.125	0.26136	-164.449360256942	-164.449360256942\\
63.125	0.26502	-170.662190724592	-170.662190724592\\
63.125	0.26868	-176.997252883893	-176.997252883893\\
63.125	0.27234	-183.454546734843	-183.454546734843\\
63.125	0.276	-190.034072277443	-190.034072277443\\
63.5	0.093	-10.8816156469619	-10.8816156469619\\
63.5	0.09666	-11.4745671185039	-11.4745671185039\\
63.5	0.10032	-12.1897502816959	-12.1897502816959\\
63.5	0.10398	-13.0271651365377	-13.0271651365377\\
63.5	0.10764	-13.9868116830292	-13.9868116830292\\
63.5	0.1113	-15.0686899211705	-15.0686899211705\\
63.5	0.11496	-16.2727998509616	-16.2727998509616\\
63.5	0.11862	-17.5991414724026	-17.5991414724026\\
63.5	0.12228	-19.0477147854933	-19.0477147854933\\
63.5	0.12594	-20.6185197902338	-20.6185197902338\\
63.5	0.1296	-22.3115564866241	-22.3115564866241\\
63.5	0.13326	-24.1268248746641	-24.1268248746641\\
63.5	0.13692	-26.064324954354	-26.064324954354\\
63.5	0.14058	-28.1240567256937	-28.1240567256937\\
63.5	0.14424	-30.3060201886832	-30.3060201886832\\
63.5	0.1479	-32.6102153433225	-32.6102153433225\\
63.5	0.15156	-35.0366421896115	-35.0366421896115\\
63.5	0.15522	-37.5853007275504	-37.5853007275504\\
63.5	0.15888	-40.2561909571391	-40.2561909571391\\
63.5	0.16254	-43.0493128783775	-43.0493128783775\\
63.5	0.1662	-45.9646664912657	-45.9646664912657\\
63.5	0.16986	-49.0022517958038	-49.0022517958038\\
63.5	0.17352	-52.1620687919916	-52.1620687919916\\
63.5	0.17718	-55.4441174798292	-55.4441174798292\\
63.5	0.18084	-58.8483978593166	-58.8483978593166\\
63.5	0.1845	-62.3749099304538	-62.3749099304538\\
63.5	0.18816	-66.0236536932408	-66.0236536932408\\
63.5	0.19182	-69.7946291476776	-69.7946291476776\\
63.5	0.19548	-73.6878362937642	-73.6878362937642\\
63.5	0.19914	-77.7032751315006	-77.7032751315006\\
63.5	0.2028	-81.8409456608868	-81.8409456608868\\
63.5	0.20646	-86.1008478819228	-86.1008478819228\\
63.5	0.21012	-90.4829817946085	-90.4829817946085\\
63.5	0.21378	-94.9873473989441	-94.9873473989441\\
63.5	0.21744	-99.6139446949295	-99.6139446949295\\
63.5	0.2211	-104.362773682565	-104.362773682565\\
63.5	0.22476	-109.23383436185	-109.23383436185\\
63.5	0.22842	-114.227126732784	-114.227126732784\\
63.5	0.23208	-119.342650795369	-119.342650795369\\
63.5	0.23574	-124.580406549603	-124.580406549603\\
63.5	0.2394	-129.940393995487	-129.940393995487\\
63.5	0.24306	-135.422613133021	-135.422613133021\\
63.5	0.24672	-141.027063962205	-141.027063962205\\
63.5	0.25038	-146.753746483039	-146.753746483039\\
63.5	0.25404	-152.602660695522	-152.602660695522\\
63.5	0.2577	-158.573806599655	-158.573806599655\\
63.5	0.26136	-164.667184195438	-164.667184195438\\
63.5	0.26502	-170.88279348287	-170.88279348287\\
63.5	0.26868	-177.220634461953	-177.220634461953\\
63.5	0.27234	-183.680707132685	-183.680707132685\\
63.5	0.276	-190.263011495067	-190.263011495067\\
63.875	0.093	-10.9812102504032	-10.9812102504032\\
63.875	0.09666	-11.5769405417274	-11.5769405417274\\
63.875	0.10032	-12.2949025247013	-12.2949025247013\\
63.875	0.10398	-13.1350961993251	-13.1350961993251\\
63.875	0.10764	-14.0975215655986	-14.0975215655986\\
63.875	0.1113	-15.182178623522	-15.182178623522\\
63.875	0.11496	-16.3890673730951	-16.3890673730951\\
63.875	0.11862	-17.718187814318	-17.718187814318\\
63.875	0.12228	-19.1695399471907	-19.1695399471907\\
63.875	0.12594	-20.7431237717133	-20.7431237717133\\
63.875	0.1296	-22.4389392878855	-22.4389392878855\\
63.875	0.13326	-24.2569864957076	-24.2569864957076\\
63.875	0.13692	-26.1972653951795	-26.1972653951795\\
63.875	0.14058	-28.2597759863012	-28.2597759863012\\
63.875	0.14424	-30.4445182690727	-30.4445182690727\\
63.875	0.1479	-32.751492243494	-32.751492243494\\
63.875	0.15156	-35.1806979095651	-35.1806979095651\\
63.875	0.15522	-37.7321352672859	-37.7321352672859\\
63.875	0.15888	-40.4058043166566	-40.4058043166566\\
63.875	0.16254	-43.201705057677	-43.201705057677\\
63.875	0.1662	-46.1198374903473	-46.1198374903473\\
63.875	0.16986	-49.1602016146674	-49.1602016146674\\
63.875	0.17352	-52.3227974306372	-52.3227974306372\\
63.875	0.17718	-55.6076249382568	-55.6076249382568\\
63.875	0.18084	-59.0146841375262	-59.0146841375262\\
63.875	0.1845	-62.5439750284455	-62.5439750284455\\
63.875	0.18816	-66.1954976110145	-66.1954976110145\\
63.875	0.19182	-69.9692518852333	-69.9692518852333\\
63.875	0.19548	-73.8652378511019	-73.8652378511019\\
63.875	0.19914	-77.8834555086203	-77.8834555086203\\
63.875	0.2028	-82.0239048577885	-82.0239048577885\\
63.875	0.20646	-86.2865858986065	-86.2865858986065\\
63.875	0.21012	-90.6714986310743	-90.6714986310743\\
63.875	0.21378	-95.1786430551919	-95.1786430551919\\
63.875	0.21744	-99.8080191709592	-99.8080191709592\\
63.875	0.2211	-104.559626978376	-104.559626978376\\
63.875	0.22476	-109.433466477443	-109.433466477443\\
63.875	0.22842	-114.42953766816	-114.42953766816\\
63.875	0.23208	-119.547840550527	-119.547840550527\\
63.875	0.23574	-124.788375124543	-124.788375124543\\
63.875	0.2394	-130.151141390209	-130.151141390209\\
63.875	0.24306	-135.636139347525	-135.636139347525\\
63.875	0.24672	-141.243368996491	-141.243368996491\\
63.875	0.25038	-146.972830337106	-146.972830337106\\
63.875	0.25404	-152.824523369372	-152.824523369372\\
63.875	0.2577	-158.798448093287	-158.798448093287\\
63.875	0.26136	-164.894604508852	-164.894604508852\\
63.875	0.26502	-171.112992616066	-171.112992616066\\
63.875	0.26868	-177.453612414931	-177.453612414931\\
63.875	0.27234	-183.916463905445	-183.916463905445\\
63.875	0.276	-190.501547087609	-190.501547087609\\
64.25	0.093	-11.090401228762	-11.090401228762\\
64.25	0.09666	-11.6889103398682	-11.6889103398682\\
64.25	0.10032	-12.4096511426241	-12.4096511426241\\
64.25	0.10398	-13.2526236370299	-13.2526236370299\\
64.25	0.10764	-14.2178278230854	-14.2178278230854\\
64.25	0.1113	-15.3052637007908	-15.3052637007908\\
64.25	0.11496	-16.5149312701459	-16.5149312701459\\
64.25	0.11862	-17.8468305311509	-17.8468305311509\\
64.25	0.12228	-19.3009614838056	-19.3009614838056\\
64.25	0.12594	-20.8773241281101	-20.8773241281101\\
64.25	0.1296	-22.5759184640644	-22.5759184640644\\
64.25	0.13326	-24.3967444916685	-24.3967444916685\\
64.25	0.13692	-26.3398022109224	-26.3398022109224\\
64.25	0.14058	-28.4050916218261	-28.4050916218261\\
64.25	0.14424	-30.5926127243796	-30.5926127243796\\
64.25	0.1479	-32.9023655185829	-32.9023655185829\\
64.25	0.15156	-35.334350004436	-35.334350004436\\
64.25	0.15522	-37.8885661819389	-37.8885661819389\\
64.25	0.15888	-40.5650140510916	-40.5650140510916\\
64.25	0.16254	-43.363693611894	-43.363693611894\\
64.25	0.1662	-46.2846048643463	-46.2846048643463\\
64.25	0.16986	-49.3277478084484	-49.3277478084484\\
64.25	0.17352	-52.4931224442002	-52.4931224442002\\
64.25	0.17718	-55.7807287716018	-55.7807287716018\\
64.25	0.18084	-59.1905667906532	-59.1905667906532\\
64.25	0.1845	-62.7226365013544	-62.7226365013544\\
64.25	0.18816	-66.3769379037055	-66.3769379037055\\
64.25	0.19182	-70.1534709977064	-70.1534709977064\\
64.25	0.19548	-74.0522357833569	-74.0522357833569\\
64.25	0.19914	-78.0732322606573	-78.0732322606573\\
64.25	0.2028	-82.2164604296076	-82.2164604296076\\
64.25	0.20646	-86.4819202902076	-86.4819202902076\\
64.25	0.21012	-90.8696118424574	-90.8696118424574\\
64.25	0.21378	-95.379535086357	-95.379535086357\\
64.25	0.21744	-100.011690021906	-100.011690021906\\
64.25	0.2211	-104.766076649105	-104.766076649105\\
64.25	0.22476	-109.642694967954	-109.642694967954\\
64.25	0.22842	-114.641544978453	-114.641544978453\\
64.25	0.23208	-119.762626680602	-119.762626680602\\
64.25	0.23574	-125.0059400744	-125.0059400744\\
64.25	0.2394	-130.371485159848	-130.371485159848\\
64.25	0.24306	-135.859261936946	-135.859261936946\\
64.25	0.24672	-141.469270405694	-141.469270405694\\
64.25	0.25038	-147.201510566092	-147.201510566092\\
64.25	0.25404	-153.055982418139	-153.055982418139\\
64.25	0.2577	-159.032685961836	-159.032685961836\\
64.25	0.26136	-165.131621197183	-165.131621197183\\
64.25	0.26502	-171.352788124179	-171.352788124179\\
64.25	0.26868	-177.696186742826	-177.696186742826\\
64.25	0.27234	-184.161817053122	-184.161817053122\\
64.25	0.276	-190.749679055068	-190.749679055068\\
64.625	0.093	-11.2091885820384	-11.2091885820384\\
64.625	0.09666	-11.8104765129265	-11.8104765129265\\
64.625	0.10032	-12.5339961354646	-12.5339961354646\\
64.625	0.10398	-13.3797474496523	-13.3797474496523\\
64.625	0.10764	-14.3477304554898	-14.3477304554898\\
64.625	0.1113	-15.4379451529772	-15.4379451529772\\
64.625	0.11496	-16.6503915421143	-16.6503915421143\\
64.625	0.11862	-17.9850696229013	-17.9850696229013\\
64.625	0.12228	-19.4419793953381	-19.4419793953381\\
64.625	0.12594	-21.0211208594246	-21.0211208594246\\
64.625	0.1296	-22.7224940151609	-22.7224940151609\\
64.625	0.13326	-24.546098862547	-24.546098862547\\
64.625	0.13692	-26.4919354015829	-26.4919354015829\\
64.625	0.14058	-28.5600036322686	-28.5600036322686\\
64.625	0.14424	-30.7503035546041	-30.7503035546041\\
64.625	0.1479	-33.0628351685894	-33.0628351685894\\
64.625	0.15156	-35.4975984742245	-35.4975984742245\\
64.625	0.15522	-38.0545934715094	-38.0545934715094\\
64.625	0.15888	-40.7338201604441	-40.7338201604441\\
64.625	0.16254	-43.5352785410286	-43.5352785410286\\
64.625	0.1662	-46.4589686132628	-46.4589686132628\\
64.625	0.16986	-49.5048903771469	-49.5048903771469\\
64.625	0.17352	-52.6730438326807	-52.6730438326807\\
64.625	0.17718	-55.9634289798644	-55.9634289798644\\
64.625	0.18084	-59.3760458186978	-59.3760458186978\\
64.625	0.1845	-62.9108943491811	-62.9108943491811\\
64.625	0.18816	-66.5679745713141	-66.5679745713141\\
64.625	0.19182	-70.3472864850969	-70.3472864850969\\
64.625	0.19548	-74.2488300905295	-74.2488300905295\\
64.625	0.19914	-78.272605387612	-78.272605387612\\
64.625	0.2028	-82.4186123763442	-82.4186123763442\\
64.625	0.20646	-86.6868510567262	-86.6868510567262\\
64.625	0.21012	-91.077321428758	-91.077321428758\\
64.625	0.21378	-95.5900234924396	-95.5900234924396\\
64.625	0.21744	-100.224957247771	-100.224957247771\\
64.625	0.2211	-104.982122694752	-104.982122694752\\
64.625	0.22476	-109.861519833383	-109.861519833383\\
64.625	0.22842	-114.863148663664	-114.863148663664\\
64.625	0.23208	-119.987009185594	-119.987009185594\\
64.625	0.23574	-125.233101399175	-125.233101399175\\
64.625	0.2394	-130.601425304405	-130.601425304405\\
64.625	0.24306	-136.091980901285	-136.091980901285\\
64.625	0.24672	-141.704768189815	-141.704768189815\\
64.625	0.25038	-147.439787169994	-147.439787169994\\
64.625	0.25404	-153.297037841824	-153.297037841824\\
64.625	0.2577	-159.276520205303	-159.276520205303\\
64.625	0.26136	-165.378234260432	-165.378234260432\\
64.625	0.26502	-171.60218000721	-171.60218000721\\
64.625	0.26868	-177.948357445639	-177.948357445639\\
64.625	0.27234	-184.416766575717	-184.416766575717\\
64.625	0.276	-191.007407397445	-191.007407397445\\
65	0.093	-11.3375723102322	-11.3375723102322\\
65	0.09666	-11.9416390609023	-11.9416390609023\\
65	0.10032	-12.6679375032224	-12.6679375032224\\
65	0.10398	-13.5164676371921	-13.5164676371921\\
65	0.10764	-14.4872294628117	-14.4872294628117\\
65	0.1113	-15.5802229800811	-15.5802229800811\\
65	0.11496	-16.7954481890002	-16.7954481890002\\
65	0.11862	-18.1329050895691	-18.1329050895691\\
65	0.12228	-19.5925936817879	-19.5925936817879\\
65	0.12594	-21.1745139656565	-21.1745139656565\\
65	0.1296	-22.8786659411747	-22.8786659411747\\
65	0.13326	-24.7050496083429	-24.7050496083429\\
65	0.13692	-26.6536649671608	-26.6536649671608\\
65	0.14058	-28.7245120176285	-28.7245120176285\\
65	0.14424	-30.917590759746	-30.917590759746\\
65	0.1479	-33.2329011935133	-33.2329011935133\\
65	0.15156	-35.6704433189305	-35.6704433189305\\
65	0.15522	-38.2302171359973	-38.2302171359973\\
65	0.15888	-40.912222644714	-40.912222644714\\
65	0.16254	-43.7164598450805	-43.7164598450805\\
65	0.1662	-46.6429287370968	-46.6429287370968\\
65	0.16986	-49.6916293207629	-49.6916293207629\\
65	0.17352	-52.8625615960787	-52.8625615960787\\
65	0.17718	-56.1557255630444	-56.1557255630444\\
65	0.18084	-59.5711212216598	-59.5711212216598\\
65	0.1845	-63.1087485719251	-63.1087485719251\\
65	0.18816	-66.7686076138401	-66.7686076138401\\
65	0.19182	-70.5506983474049	-70.5506983474049\\
65	0.19548	-74.4550207726196	-74.4550207726196\\
65	0.19914	-78.481574889484	-78.481574889484\\
65	0.2028	-82.6303606979983	-82.6303606979983\\
65	0.20646	-86.9013781981623	-86.9013781981623\\
65	0.21012	-91.2946273899761	-91.2946273899761\\
65	0.21378	-95.8101082734396	-95.8101082734396\\
65	0.21744	-100.447820848553	-100.447820848553\\
65	0.2211	-105.207765115316	-105.207765115316\\
65	0.22476	-110.089941073729	-110.089941073729\\
65	0.22842	-115.094348723792	-115.094348723792\\
65	0.23208	-120.220988065505	-120.220988065505\\
65	0.23574	-125.469859098867	-125.469859098867\\
65	0.2394	-130.840961823879	-130.840961823879\\
65	0.24306	-136.334296240541	-136.334296240541\\
65	0.24672	-141.949862348853	-141.949862348853\\
65	0.25038	-147.687660148814	-147.687660148814\\
65	0.25404	-153.547689640426	-153.547689640426\\
65	0.2577	-159.529950823687	-159.529950823687\\
65	0.26136	-165.634443698598	-165.634443698598\\
65	0.26502	-171.861168265158	-171.861168265158\\
65	0.26868	-178.210124523369	-178.210124523369\\
65	0.27234	-184.681312473229	-184.681312473229\\
65	0.276	-191.274732114739	-191.274732114739\\
65.375	0.093	-11.4755524133436	-11.4755524133436\\
65.375	0.09666	-12.0823979837957	-12.0823979837957\\
65.375	0.10032	-12.8114752458977	-12.8114752458977\\
65.375	0.10398	-13.6627841996495	-13.6627841996495\\
65.375	0.10764	-14.6363248450511	-14.6363248450511\\
65.375	0.1113	-15.7320971821025	-15.7320971821025\\
65.375	0.11496	-16.9501012108036	-16.9501012108036\\
65.375	0.11862	-18.2903369311546	-18.2903369311546\\
65.375	0.12228	-19.7528043431554	-19.7528043431554\\
65.375	0.12594	-21.3375034468059	-21.3375034468059\\
65.375	0.1296	-23.0444342421062	-23.0444342421062\\
65.375	0.13326	-24.8735967290564	-24.8735967290564\\
65.375	0.13692	-26.8249909076563	-26.8249909076563\\
65.375	0.14058	-28.898616777906	-28.898616777906\\
65.375	0.14424	-31.0944743398055	-31.0944743398055\\
65.375	0.1479	-33.4125635933548	-33.4125635933548\\
65.375	0.15156	-35.852884538554	-35.852884538554\\
65.375	0.15522	-38.4154371754029	-38.4154371754029\\
65.375	0.15888	-41.1002215039016	-41.1002215039016\\
65.375	0.16254	-43.9072375240501	-43.9072375240501\\
65.375	0.1662	-46.8364852358483	-46.8364852358483\\
65.375	0.16986	-49.8879646392964	-49.8879646392964\\
65.375	0.17352	-53.0616757343943	-53.0616757343943\\
65.375	0.17718	-56.3576185211419	-56.3576185211419\\
65.375	0.18084	-59.7757929995394	-59.7757929995394\\
65.375	0.1845	-63.3161991695866	-63.3161991695866\\
65.375	0.18816	-66.9788370312837	-66.9788370312837\\
65.375	0.19182	-70.7637065846306	-70.7637065846306\\
65.375	0.19548	-74.6708078296272	-74.6708078296272\\
65.375	0.19914	-78.7001407662736	-78.7001407662736\\
65.375	0.2028	-82.8517053945699	-82.8517053945699\\
65.375	0.20646	-87.1255017145159	-87.1255017145159\\
65.375	0.21012	-91.5215297261118	-91.5215297261118\\
65.375	0.21378	-96.0397894293573	-96.0397894293573\\
65.375	0.21744	-100.680280824253	-100.680280824253\\
65.375	0.2211	-105.443003910798	-105.443003910798\\
65.375	0.22476	-110.327958688993	-110.327958688993\\
65.375	0.22842	-115.335145158838	-115.335145158838\\
65.375	0.23208	-120.464563320332	-120.464563320332\\
65.375	0.23574	-125.716213173477	-125.716213173477\\
65.375	0.2394	-131.090094718271	-131.090094718271\\
65.375	0.24306	-136.586207954715	-136.586207954715\\
65.375	0.24672	-142.204552882809	-142.204552882809\\
65.375	0.25038	-147.945129502552	-147.945129502552\\
65.375	0.25404	-153.807937813946	-153.807937813946\\
65.375	0.2577	-159.792977816989	-159.792977816989\\
65.375	0.26136	-165.900249511682	-165.900249511682\\
65.375	0.26502	-172.129752898024	-172.129752898024\\
65.375	0.26868	-178.481487976017	-178.481487976017\\
65.375	0.27234	-184.955454745659	-184.955454745659\\
65.375	0.276	-191.551653206951	-191.551653206951\\
65.75	0.093	-11.6231288913724	-11.6231288913724\\
65.75	0.09666	-12.2327532816065	-12.2327532816065\\
65.75	0.10032	-12.9646093634906	-12.9646093634906\\
65.75	0.10398	-13.8186971370244	-13.8186971370244\\
65.75	0.10764	-14.7950166022079	-14.7950166022079\\
65.75	0.1113	-15.8935677590413	-15.8935677590413\\
65.75	0.11496	-17.1143506075245	-17.1143506075245\\
65.75	0.11862	-18.4573651476575	-18.4573651476575\\
65.75	0.12228	-19.9226113794402	-19.9226113794402\\
65.75	0.12594	-21.5100893028728	-21.5100893028728\\
65.75	0.1296	-23.2197989179551	-23.2197989179551\\
65.75	0.13326	-25.0517402246873	-25.0517402246873\\
65.75	0.13692	-27.0059132230692	-27.0059132230692\\
65.75	0.14058	-29.082317913101	-29.082317913101\\
65.75	0.14424	-31.2809542947825	-31.2809542947825\\
65.75	0.1479	-33.6018223681138	-33.6018223681138\\
65.75	0.15156	-36.0449221330949	-36.0449221330949\\
65.75	0.15522	-38.6102535897258	-38.6102535897258\\
65.75	0.15888	-41.2978167380066	-41.2978167380066\\
65.75	0.16254	-44.107611577937	-44.107611577937\\
65.75	0.1662	-47.0396381095173	-47.0396381095173\\
65.75	0.16986	-50.0938963327474	-50.0938963327474\\
65.75	0.17352	-53.2703862476273	-53.2703862476273\\
65.75	0.17718	-56.569107854157	-56.569107854157\\
65.75	0.18084	-59.9900611523364	-59.9900611523364\\
65.75	0.1845	-63.5332461421657	-63.5332461421657\\
65.75	0.18816	-67.1986628236448	-67.1986628236448\\
65.75	0.19182	-70.9863111967736	-70.9863111967736\\
65.75	0.19548	-74.8961912615523	-74.8961912615523\\
65.75	0.19914	-78.9283030179807	-78.9283030179807\\
65.75	0.2028	-83.082646466059	-83.082646466059\\
65.75	0.20646	-87.359221605787	-87.359221605787\\
65.75	0.21012	-91.7580284371648	-91.7580284371648\\
65.75	0.21378	-96.2790669601924	-96.2790669601924\\
65.75	0.21744	-100.92233717487	-100.92233717487\\
65.75	0.2211	-105.687839081197	-105.687839081197\\
65.75	0.22476	-110.575572679174	-110.575572679174\\
65.75	0.22842	-115.585537968801	-115.585537968801\\
65.75	0.23208	-120.717734950077	-120.717734950077\\
65.75	0.23574	-125.972163623004	-125.972163623004\\
65.75	0.2394	-131.34882398758	-131.34882398758\\
65.75	0.24306	-136.847716043806	-136.847716043806\\
65.75	0.24672	-142.468839791682	-142.468839791682\\
65.75	0.25038	-148.212195231207	-148.212195231207\\
65.75	0.25404	-154.077782362383	-154.077782362383\\
65.75	0.2577	-160.065601185208	-160.065601185208\\
65.75	0.26136	-166.175651699683	-166.175651699683\\
65.75	0.26502	-172.407933905808	-172.407933905808\\
65.75	0.26868	-178.762447803582	-178.762447803582\\
65.75	0.27234	-185.239193393006	-185.239193393006\\
65.75	0.276	-191.83817067408	-191.83817067408\\
66.125	0.093	-11.7803017443187	-11.7803017443187\\
66.125	0.09666	-12.3927049543348	-12.3927049543348\\
66.125	0.10032	-13.1273398560009	-13.1273398560009\\
66.125	0.10398	-13.9842064493166	-13.9842064493166\\
66.125	0.10764	-14.9633047342822	-14.9633047342822\\
66.125	0.1113	-16.0646347108977	-16.0646347108977\\
66.125	0.11496	-17.2881963791628	-17.2881963791628\\
66.125	0.11862	-18.6339897390778	-18.6339897390778\\
66.125	0.12228	-20.1020147906426	-20.1020147906426\\
66.125	0.12594	-21.6922715338572	-21.6922715338572\\
66.125	0.1296	-23.4047599687215	-23.4047599687215\\
66.125	0.13326	-25.2394800952356	-25.2394800952356\\
66.125	0.13692	-27.1964319133996	-27.1964319133996\\
66.125	0.14058	-29.2756154232134	-29.2756154232134\\
66.125	0.14424	-31.4770306246769	-31.4770306246769\\
66.125	0.1479	-33.8006775177902	-33.8006775177902\\
66.125	0.15156	-36.2465561025533	-36.2465561025533\\
66.125	0.15522	-38.8146663789663	-38.8146663789663\\
66.125	0.15888	-41.505008347029	-41.505008347029\\
66.125	0.16254	-44.3175820067415	-44.3175820067415\\
66.125	0.1662	-47.2523873581038	-47.2523873581038\\
66.125	0.16986	-50.3094244011159	-50.3094244011159\\
66.125	0.17352	-53.4886931357777	-53.4886931357777\\
66.125	0.17718	-56.7901935620894	-56.7901935620894\\
66.125	0.18084	-60.2139256800509	-60.2139256800509\\
66.125	0.1845	-63.7598894896622	-63.7598894896622\\
66.125	0.18816	-67.4280849909233	-67.4280849909233\\
66.125	0.19182	-71.2185121838341	-71.2185121838341\\
66.125	0.19548	-75.1311710683948	-75.1311710683948\\
66.125	0.19914	-79.1660616446052	-79.1660616446052\\
66.125	0.2028	-83.3231839124655	-83.3231839124655\\
66.125	0.20646	-87.6025378719756	-87.6025378719756\\
66.125	0.21012	-92.0041235231354	-92.0041235231354\\
66.125	0.21378	-96.527940865945	-96.527940865945\\
66.125	0.21744	-101.173989900404	-101.173989900404\\
66.125	0.2211	-105.942270626514	-105.942270626514\\
66.125	0.22476	-110.832783044273	-110.832783044273\\
66.125	0.22842	-115.845527153681	-115.845527153681\\
66.125	0.23208	-120.98050295474	-120.98050295474\\
66.125	0.23574	-126.237710447448	-126.237710447448\\
66.125	0.2394	-131.617149631807	-131.617149631807\\
66.125	0.24306	-137.118820507815	-137.118820507815\\
66.125	0.24672	-142.742723075472	-142.742723075472\\
66.125	0.25038	-148.48885733478	-148.48885733478\\
66.125	0.25404	-154.357223285737	-154.357223285737\\
66.125	0.2577	-160.347820928345	-160.347820928345\\
66.125	0.26136	-166.460650262602	-166.460650262602\\
66.125	0.26502	-172.695711288508	-172.695711288508\\
66.125	0.26868	-179.053004006065	-179.053004006065\\
66.125	0.27234	-185.532528415271	-185.532528415271\\
66.125	0.276	-192.134284516127	-192.134284516127\\
66.5	0.093	-11.9470709721825	-11.9470709721825\\
66.5	0.09666	-12.5622530019806	-12.5622530019806\\
66.5	0.10032	-13.2996667234287	-13.2996667234287\\
66.5	0.10398	-14.1593121365265	-14.1593121365265\\
66.5	0.10764	-15.1411892412741	-15.1411892412741\\
66.5	0.1113	-16.2452980376715	-16.2452980376715\\
66.5	0.11496	-17.4716385257186	-17.4716385257186\\
66.5	0.11862	-18.8202107054156	-18.8202107054156\\
66.5	0.12228	-20.2910145767624	-20.2910145767624\\
66.5	0.12594	-21.884050139759	-21.884050139759\\
66.5	0.1296	-23.5993173944053	-23.5993173944053\\
66.5	0.13326	-25.4368163407015	-25.4368163407015\\
66.5	0.13692	-27.3965469786475	-27.3965469786475\\
66.5	0.14058	-29.4785093082432	-29.4785093082432\\
66.5	0.14424	-31.6827033294888	-31.6827033294888\\
66.5	0.1479	-34.0091290423841	-34.0091290423841\\
66.5	0.15156	-36.4577864469293	-36.4577864469293\\
66.5	0.15522	-39.0286755431242	-39.0286755431242\\
66.5	0.15888	-41.7217963309689	-41.7217963309689\\
66.5	0.16254	-44.5371488104634	-44.5371488104634\\
66.5	0.1662	-47.4747329816077	-47.4747329816077\\
66.5	0.16986	-50.5345488444019	-50.5345488444019\\
66.5	0.17352	-53.7165963988458	-53.7165963988458\\
66.5	0.17718	-57.0208756449394	-57.0208756449394\\
66.5	0.18084	-60.4473865826829	-60.4473865826829\\
66.5	0.1845	-63.9961292120762	-63.9961292120762\\
66.5	0.18816	-67.6671035331193	-67.6671035331193\\
66.5	0.19182	-71.4603095458122	-71.4603095458122\\
66.5	0.19548	-75.3757472501548	-75.3757472501548\\
66.5	0.19914	-79.4134166461473	-79.4134166461473\\
66.5	0.2028	-83.5733177337896	-83.5733177337896\\
66.5	0.20646	-87.8554505130816	-87.8554505130816\\
66.5	0.21012	-92.2598149840235	-92.2598149840235\\
66.5	0.21378	-96.7864111466151	-96.7864111466151\\
66.5	0.21744	-101.435239000856	-101.435239000856\\
66.5	0.2211	-106.206298546748	-106.206298546748\\
66.5	0.22476	-111.099589784289	-111.099589784289\\
66.5	0.22842	-116.11511271348	-116.11511271348\\
66.5	0.23208	-121.25286733432	-121.25286733432\\
66.5	0.23574	-126.512853646811	-126.512853646811\\
66.5	0.2394	-131.895071650951	-131.895071650951\\
66.5	0.24306	-137.399521346741	-137.399521346741\\
66.5	0.24672	-143.026202734181	-143.026202734181\\
66.5	0.25038	-148.77511581327	-148.77511581327\\
66.5	0.25404	-154.64626058401	-154.64626058401\\
66.5	0.2577	-160.639637046399	-160.639637046399\\
66.5	0.26136	-166.755245200438	-166.755245200438\\
66.5	0.26502	-172.993085046126	-172.993085046126\\
66.5	0.26868	-179.353156583465	-179.353156583465\\
66.5	0.27234	-185.835459812453	-185.835459812453\\
66.5	0.276	-192.439994733091	-192.439994733091\\
66.875	0.093	-12.1234365749637	-12.1234365749637\\
66.875	0.09666	-12.7413974245439	-12.7413974245439\\
66.875	0.10032	-13.4815899657739	-13.4815899657739\\
66.875	0.10398	-14.3440141986537	-14.3440141986537\\
66.875	0.10764	-15.3286701231833	-15.3286701231833\\
66.875	0.1113	-16.4355577393628	-16.4355577393628\\
66.875	0.11496	-17.6646770471919	-17.6646770471919\\
66.875	0.11862	-19.016028046671	-19.016028046671\\
66.875	0.12228	-20.4896107377998	-20.4896107377998\\
66.875	0.12594	-22.0854251205783	-22.0854251205783\\
66.875	0.1296	-23.8034711950067	-23.8034711950067\\
66.875	0.13326	-25.6437489610849	-25.6437489610849\\
66.875	0.13692	-27.6062584188128	-27.6062584188128\\
66.875	0.14058	-29.6909995681906	-29.6909995681906\\
66.875	0.14424	-31.8979724092182	-31.8979724092182\\
66.875	0.1479	-34.2271769418955	-34.2271769418955\\
66.875	0.15156	-36.6786131662226	-36.6786131662226\\
66.875	0.15522	-39.2522810821996	-39.2522810821996\\
66.875	0.15888	-41.9481806898263	-41.9481806898263\\
66.875	0.16254	-44.7663119891029	-44.7663119891029\\
66.875	0.1662	-47.7066749800292	-47.7066749800292\\
66.875	0.16986	-50.7692696626053	-50.7692696626053\\
66.875	0.17352	-53.9540960368312	-53.9540960368312\\
66.875	0.17718	-57.2611541027069	-57.2611541027069\\
66.875	0.18084	-60.6904438602324	-60.6904438602324\\
66.875	0.1845	-64.2419653094076	-64.2419653094076\\
66.875	0.18816	-67.9157184502328	-67.9157184502328\\
66.875	0.19182	-71.7117032827077	-71.7117032827077\\
66.875	0.19548	-75.6299198068323	-75.6299198068323\\
66.875	0.19914	-79.6703680226068	-79.6703680226068\\
66.875	0.2028	-83.8330479300311	-83.8330479300311\\
66.875	0.20646	-88.1179595291051	-88.1179595291051\\
66.875	0.21012	-92.525102819829	-92.525102819829\\
66.875	0.21378	-97.0544778022027	-97.0544778022027\\
66.875	0.21744	-101.706084476226	-101.706084476226\\
66.875	0.2211	-106.479922841899	-106.479922841899\\
66.875	0.22476	-111.375992899222	-111.375992899222\\
66.875	0.22842	-116.394294648195	-116.394294648195\\
66.875	0.23208	-121.534828088818	-121.534828088818\\
66.875	0.23574	-126.79759322109	-126.79759322109\\
66.875	0.2394	-132.182590045012	-132.182590045012\\
66.875	0.24306	-137.689818560584	-137.689818560584\\
66.875	0.24672	-143.319278767806	-143.319278767806\\
66.875	0.25038	-149.070970666678	-149.070970666678\\
66.875	0.25404	-154.944894257199	-154.944894257199\\
66.875	0.2577	-160.94104953937	-160.94104953937\\
66.875	0.26136	-167.059436513191	-167.059436513191\\
66.875	0.26502	-173.300055178662	-173.300055178662\\
66.875	0.26868	-179.662905535783	-179.662905535783\\
66.875	0.27234	-186.147987584553	-186.147987584553\\
66.875	0.276	-192.755301324973	-192.755301324973\\
67.25	0.093	-12.3093985526625	-12.3093985526625\\
67.25	0.09666	-12.9301382220247	-12.9301382220247\\
67.25	0.10032	-13.6731095830367	-13.6731095830367\\
67.25	0.10398	-14.5383126356986	-14.5383126356986\\
67.25	0.10764	-15.5257473800102	-15.5257473800102\\
67.25	0.1113	-16.6354138159716	-16.6354138159716\\
67.25	0.11496	-17.8673119435828	-17.8673119435828\\
67.25	0.11862	-19.2214417628438	-19.2214417628438\\
67.25	0.12228	-20.6978032737546	-20.6978032737546\\
67.25	0.12594	-22.2963964763152	-22.2963964763152\\
67.25	0.1296	-24.0172213705256	-24.0172213705256\\
67.25	0.13326	-25.8602779563858	-25.8602779563858\\
67.25	0.13692	-27.8255662338957	-27.8255662338957\\
67.25	0.14058	-29.9130862030555	-29.9130862030555\\
67.25	0.14424	-32.1228378638651	-32.1228378638651\\
67.25	0.1479	-34.4548212163244	-34.4548212163244\\
67.25	0.15156	-36.9090362604336	-36.9090362604336\\
67.25	0.15522	-39.4854829961925	-39.4854829961925\\
67.25	0.15888	-42.1841614236013	-42.1841614236013\\
67.25	0.16254	-45.0050715426598	-45.0050715426598\\
67.25	0.1662	-47.9482133533681	-47.9482133533681\\
67.25	0.16986	-51.0135868557263	-51.0135868557263\\
67.25	0.17352	-54.2011920497342	-54.2011920497342\\
67.25	0.17718	-57.5110289353919	-57.5110289353919\\
67.25	0.18084	-60.9430975126994	-60.9430975126994\\
67.25	0.1845	-64.4973977816567	-64.4973977816567\\
67.25	0.18816	-68.1739297422638	-68.1739297422638\\
67.25	0.19182	-71.9726933945207	-71.9726933945207\\
67.25	0.19548	-75.8936887384274	-75.8936887384274\\
67.25	0.19914	-79.9369157739839	-79.9369157739839\\
67.25	0.2028	-84.1023745011902	-84.1023745011902\\
67.25	0.20646	-88.3900649200462	-88.3900649200462\\
67.25	0.21012	-92.7999870305521	-92.7999870305521\\
67.25	0.21378	-97.3321408327077	-97.3321408327077\\
67.25	0.21744	-101.986526326513	-101.986526326513\\
67.25	0.2211	-106.763143511968	-106.763143511968\\
67.25	0.22476	-111.661992389073	-111.661992389073\\
67.25	0.22842	-116.683072957828	-116.683072957828\\
67.25	0.23208	-121.826385218233	-121.826385218233\\
67.25	0.23574	-127.091929170287	-127.091929170287\\
67.25	0.2394	-132.479704813992	-132.479704813992\\
67.25	0.24306	-137.989712149346	-137.989712149346\\
67.25	0.24672	-143.621951176349	-143.621951176349\\
67.25	0.25038	-149.376421895003	-149.376421895003\\
67.25	0.25404	-155.253124305306	-155.253124305306\\
67.25	0.2577	-161.25205840726	-161.25205840726\\
67.25	0.26136	-167.373224200863	-167.373224200863\\
67.25	0.26502	-173.616621686115	-173.616621686115\\
67.25	0.26868	-179.982250863018	-179.982250863018\\
67.25	0.27234	-186.47011173157	-186.47011173157\\
67.25	0.276	-193.080204291772	-193.080204291772\\
67.625	0.093	-12.5049569052787	-12.5049569052787\\
67.625	0.09666	-13.1284753944229	-13.1284753944229\\
67.625	0.10032	-13.874225575217	-13.874225575217\\
67.625	0.10398	-14.7422074476608	-14.7422074476608\\
67.625	0.10764	-15.7324210117544	-15.7324210117544\\
67.625	0.1113	-16.8448662674979	-16.8448662674979\\
67.625	0.11496	-18.079543214891	-18.079543214891\\
67.625	0.11862	-19.4364518539341	-19.4364518539341\\
67.625	0.12228	-20.9155921846269	-20.9155921846269\\
67.625	0.12594	-22.5169642069695	-22.5169642069695\\
67.625	0.1296	-24.2405679209619	-24.2405679209619\\
67.625	0.13326	-26.0864033266041	-26.0864033266041\\
67.625	0.13692	-28.054470423896	-28.054470423896\\
67.625	0.14058	-30.1447692128378	-30.1447692128378\\
67.625	0.14424	-32.3572996934294	-32.3572996934294\\
67.625	0.1479	-34.6920618656708	-34.6920618656708\\
67.625	0.15156	-37.149055729562	-37.149055729562\\
67.625	0.15522	-39.7282812851029	-39.7282812851029\\
67.625	0.15888	-42.4297385322937	-42.4297385322937\\
67.625	0.16254	-45.2534274711342	-45.2534274711342\\
67.625	0.1662	-48.1993481016245	-48.1993481016245\\
67.625	0.16986	-51.2675004237647	-51.2675004237647\\
67.625	0.17352	-54.4578844375546	-54.4578844375546\\
67.625	0.17718	-57.7705001429943	-57.7705001429943\\
67.625	0.18084	-61.2053475400838	-61.2053475400838\\
67.625	0.1845	-64.7624266288232	-64.7624266288232\\
67.625	0.18816	-68.4417374092122	-68.4417374092122\\
67.625	0.19182	-72.2432798812511	-72.2432798812511\\
67.625	0.19548	-76.1670540449398	-76.1670540449398\\
67.625	0.19914	-80.2130599002783	-80.2130599002783\\
67.625	0.2028	-84.3812974472666	-84.3812974472666\\
67.625	0.20646	-88.6717666859047	-88.6717666859047\\
67.625	0.21012	-93.0844676161925	-93.0844676161925\\
67.625	0.21378	-97.6194002381302	-97.6194002381302\\
67.625	0.21744	-102.276564551718	-102.276564551718\\
67.625	0.2211	-107.055960556955	-107.055960556955\\
67.625	0.22476	-111.957588253842	-111.957588253842\\
67.625	0.22842	-116.981447642379	-116.981447642379\\
67.625	0.23208	-122.127538722565	-122.127538722565\\
67.625	0.23574	-127.395861494402	-127.395861494402\\
67.625	0.2394	-132.786415957888	-132.786415957888\\
67.625	0.24306	-138.299202113024	-138.299202113024\\
67.625	0.24672	-143.93421995981	-143.93421995981\\
67.625	0.25038	-149.691469498246	-149.691469498246\\
67.625	0.25404	-155.570950728331	-155.570950728331\\
67.625	0.2577	-161.572663650066	-161.572663650066\\
67.625	0.26136	-167.696608263451	-167.696608263451\\
67.625	0.26502	-173.942784568486	-173.942784568486\\
67.625	0.26868	-180.311192565171	-180.311192565171\\
67.625	0.27234	-186.801832253505	-186.801832253505\\
67.625	0.276	-193.414703633489	-193.414703633489\\
68	0.093	-12.7101116328124	-12.7101116328124\\
68	0.09666	-13.3364089417387	-13.3364089417387\\
68	0.10032	-14.0849379423148	-14.0849379423148\\
68	0.10398	-14.9556986345406	-14.9556986345406\\
68	0.10764	-15.9486910184162	-15.9486910184162\\
68	0.1113	-17.0639150939417	-17.0639150939417\\
68	0.11496	-18.3013708611169	-18.3013708611169\\
68	0.11862	-19.6610583199419	-19.6610583199419\\
68	0.12228	-21.1429774704167	-21.1429774704167\\
68	0.12594	-22.7471283125413	-22.7471283125413\\
68	0.1296	-24.4735108463157	-24.4735108463157\\
68	0.13326	-26.3221250717399	-26.3221250717399\\
68	0.13692	-28.2929709888139	-28.2929709888139\\
68	0.14058	-30.3860485975377	-30.3860485975377\\
68	0.14424	-32.6013578979113	-32.6013578979113\\
68	0.1479	-34.9388988899347	-34.9388988899347\\
68	0.15156	-37.3986715736078	-37.3986715736078\\
68	0.15522	-39.9806759489308	-39.9806759489308\\
68	0.15888	-42.6849120159036	-42.6849120159036\\
68	0.16254	-45.5113797745261	-45.5113797745261\\
68	0.1662	-48.4600792247985	-48.4600792247985\\
68	0.16986	-51.5310103667206	-51.5310103667206\\
68	0.17352	-54.7241732002926	-54.7241732002926\\
68	0.17718	-58.0395677255142	-58.0395677255142\\
68	0.18084	-61.4771939423858	-61.4771939423858\\
68	0.1845	-65.0370518509071	-65.0370518509071\\
68	0.18816	-68.7191414510782	-68.7191414510782\\
68	0.19182	-72.5234627428992	-72.5234627428992\\
68	0.19548	-76.4500157263698	-76.4500157263698\\
68	0.19914	-80.4988004014903	-80.4988004014903\\
68	0.2028	-84.6698167682607	-84.6698167682607\\
68	0.20646	-88.9630648266807	-88.9630648266807\\
68	0.21012	-93.3785445767506	-93.3785445767506\\
68	0.21378	-97.9162560184703	-97.9162560184703\\
68	0.21744	-102.57619915184	-102.57619915184\\
68	0.2211	-107.358373976859	-107.358373976859\\
68	0.22476	-112.262780493528	-112.262780493528\\
68	0.22842	-117.289418701847	-117.289418701847\\
68	0.23208	-122.438288601816	-122.438288601816\\
68	0.23574	-127.709390193434	-127.709390193434\\
68	0.2394	-133.102723476702	-133.102723476702\\
68	0.24306	-138.61828845162	-138.61828845162\\
68	0.24672	-144.256085118188	-144.256085118188\\
68	0.25038	-150.016113476406	-150.016113476406\\
68	0.25404	-155.898373526273	-155.898373526273\\
68	0.2577	-161.90286526779	-161.90286526779\\
68	0.26136	-168.029588700957	-168.029588700957\\
68	0.26502	-174.278543825774	-174.278543825774\\
68	0.26868	-180.649730642241	-180.649730642241\\
68	0.27234	-187.143149150357	-187.143149150357\\
68	0.276	-193.758799350123	-193.758799350123\\
68.375	0.093	-12.9248627352637	-12.9248627352637\\
68.375	0.09666	-13.5539388639719	-13.5539388639719\\
68.375	0.10032	-14.30524668433	-14.30524668433\\
68.375	0.10398	-15.1787861963378	-15.1787861963378\\
68.375	0.10764	-16.1745573999954	-16.1745573999954\\
68.375	0.1113	-17.2925602953029	-17.2925602953029\\
68.375	0.11496	-18.5327948822601	-18.5327948822601\\
68.375	0.11862	-19.8952611608671	-19.8952611608671\\
68.375	0.12228	-21.379959131124	-21.379959131124\\
68.375	0.12594	-22.9868887930306	-22.9868887930306\\
68.375	0.1296	-24.716050146587	-24.716050146587\\
68.375	0.13326	-26.5674431917932	-26.5674431917932\\
68.375	0.13692	-28.5410679286492	-28.5410679286492\\
68.375	0.14058	-30.636924357155	-30.636924357155\\
68.375	0.14424	-32.8550124773106	-32.8550124773106\\
68.375	0.1479	-35.195332289116	-35.195332289116\\
68.375	0.15156	-37.6578837925712	-37.6578837925712\\
68.375	0.15522	-40.2426669876761	-40.2426669876761\\
68.375	0.15888	-42.9496818744309	-42.9496818744309\\
68.375	0.16254	-45.7789284528355	-45.7789284528355\\
68.375	0.1662	-48.7304067228898	-48.7304067228898\\
68.375	0.16986	-51.804116684594	-51.804116684594\\
68.375	0.17352	-55.000058337948	-55.000058337948\\
68.375	0.17718	-58.3182316829516	-58.3182316829516\\
68.375	0.18084	-61.7586367196052	-61.7586367196052\\
68.375	0.1845	-65.3212734479085	-65.3212734479085\\
68.375	0.18816	-69.0061418678617	-69.0061418678617\\
68.375	0.19182	-72.8132419794646	-72.8132419794646\\
68.375	0.19548	-76.7425737827173	-76.7425737827173\\
68.375	0.19914	-80.7941372776198	-80.7941372776198\\
68.375	0.2028	-84.9679324641721	-84.9679324641721\\
68.375	0.20646	-89.2639593423742	-89.2639593423742\\
68.375	0.21012	-93.6822179122261	-93.6822179122261\\
68.375	0.21378	-98.2227081737278	-98.2227081737278\\
68.375	0.21744	-102.885430126879	-102.885430126879\\
68.375	0.2211	-107.67038377168	-107.67038377168\\
68.375	0.22476	-112.577569108132	-112.577569108132\\
68.375	0.22842	-117.606986136232	-117.606986136232\\
68.375	0.23208	-122.758634855983	-122.758634855983\\
68.375	0.23574	-128.032515267384	-128.032515267384\\
68.375	0.2394	-133.428627370434	-133.428627370434\\
68.375	0.24306	-138.946971165134	-138.946971165134\\
68.375	0.24672	-144.587546651484	-144.587546651484\\
68.375	0.25038	-150.350353829483	-150.350353829483\\
68.375	0.25404	-156.235392699133	-156.235392699133\\
68.375	0.2577	-162.242663260432	-162.242663260432\\
68.375	0.26136	-168.372165513381	-168.372165513381\\
68.375	0.26502	-174.62389945798	-174.62389945798\\
68.375	0.26868	-180.997865094228	-180.997865094228\\
68.375	0.27234	-187.494062422127	-187.494062422127\\
68.375	0.276	-194.112491441675	-194.112491441675\\
68.75	0.093	-13.1492102126324	-13.1492102126324\\
68.75	0.09666	-13.7810651611226	-13.7810651611226\\
68.75	0.10032	-14.5351518012627	-14.5351518012627\\
68.75	0.10398	-15.4114701330526	-15.4114701330526\\
68.75	0.10764	-16.4100201564922	-16.4100201564922\\
68.75	0.1113	-17.5308018715817	-17.5308018715817\\
68.75	0.11496	-18.7738152783209	-18.7738152783209\\
68.75	0.11862	-20.1390603767099	-20.1390603767099\\
68.75	0.12228	-21.6265371667488	-21.6265371667488\\
68.75	0.12594	-23.2362456484374	-23.2362456484374\\
68.75	0.1296	-24.9681858217758	-24.9681858217758\\
68.75	0.13326	-26.8223576867641	-26.8223576867641\\
68.75	0.13692	-28.7987612434021	-28.7987612434021\\
68.75	0.14058	-30.8973964916899	-30.8973964916899\\
68.75	0.14424	-33.1182634316275	-33.1182634316275\\
68.75	0.1479	-35.4613620632148	-35.4613620632148\\
68.75	0.15156	-37.9266923864521	-37.9266923864521\\
68.75	0.15522	-40.514254401339	-40.514254401339\\
68.75	0.15888	-43.2240481078758	-43.2240481078758\\
68.75	0.16254	-46.0560735060624	-46.0560735060624\\
68.75	0.1662	-49.0103305958988	-49.0103305958988\\
68.75	0.16986	-52.0868193773849	-52.0868193773849\\
68.75	0.17352	-55.2855398505208	-55.2855398505208\\
68.75	0.17718	-58.6064920153066	-58.6064920153066\\
68.75	0.18084	-62.0496758717421	-62.0496758717421\\
68.75	0.1845	-65.6150914198275	-65.6150914198275\\
68.75	0.18816	-69.3027386595626	-69.3027386595626\\
68.75	0.19182	-73.1126175909475	-73.1126175909475\\
68.75	0.19548	-77.0447282139823	-77.0447282139823\\
68.75	0.19914	-81.0990705286668	-81.0990705286668\\
68.75	0.2028	-85.2756445350011	-85.2756445350011\\
68.75	0.20646	-89.5744502329852	-89.5744502329852\\
68.75	0.21012	-93.9954876226191	-93.9954876226191\\
68.75	0.21378	-98.5387567039028	-98.5387567039028\\
68.75	0.21744	-103.204257476836	-103.204257476836\\
68.75	0.2211	-107.99198994142	-107.99198994142\\
68.75	0.22476	-112.901954097653	-112.901954097653\\
68.75	0.22842	-117.934149945535	-117.934149945535\\
68.75	0.23208	-123.088577485068	-123.088577485068\\
68.75	0.23574	-128.365236716251	-128.365236716251\\
68.75	0.2394	-133.764127639083	-133.764127639083\\
68.75	0.24306	-139.285250253565	-139.285250253565\\
68.75	0.24672	-144.928604559697	-144.928604559697\\
68.75	0.25038	-150.694190557478	-150.694190557478\\
68.75	0.25404	-156.58200824691	-156.58200824691\\
68.75	0.2577	-162.592057627991	-162.592057627991\\
68.75	0.26136	-168.724338700722	-168.724338700722\\
68.75	0.26502	-174.978851465103	-174.978851465103\\
68.75	0.26868	-181.355595921133	-181.355595921133\\
68.75	0.27234	-187.854572068814	-187.854572068814\\
68.75	0.276	-194.475779908144	-194.475779908144\\
69.125	0.093	-13.3831540649186	-13.3831540649186\\
69.125	0.09666	-14.0177878331909	-14.0177878331909\\
69.125	0.10032	-14.774653293113	-14.774653293113\\
69.125	0.10398	-15.6537504446848	-15.6537504446848\\
69.125	0.10764	-16.6550792879065	-16.6550792879065\\
69.125	0.1113	-17.778639822778	-17.778639822778\\
69.125	0.11496	-19.0244320492992	-19.0244320492992\\
69.125	0.11862	-20.3924559674702	-20.3924559674702\\
69.125	0.12228	-21.8827115772911	-21.8827115772911\\
69.125	0.12594	-23.4951988787617	-23.4951988787617\\
69.125	0.1296	-25.2299178718822	-25.2299178718822\\
69.125	0.13326	-27.0868685566524	-27.0868685566524\\
69.125	0.13692	-29.0660509330724	-29.0660509330724\\
69.125	0.14058	-31.1674650011422	-31.1674650011422\\
69.125	0.14424	-33.3911107608618	-33.3911107608618\\
69.125	0.1479	-35.7369882122312	-35.7369882122312\\
69.125	0.15156	-38.2050973552504	-38.2050973552504\\
69.125	0.15522	-40.7954381899194	-40.7954381899194\\
69.125	0.15888	-43.5080107162382	-43.5080107162382\\
69.125	0.16254	-46.3428149342068	-46.3428149342068\\
69.125	0.1662	-49.2998508438252	-49.2998508438252\\
69.125	0.16986	-52.3791184450934	-52.3791184450934\\
69.125	0.17352	-55.5806177380113	-55.5806177380113\\
69.125	0.17718	-58.904348722579	-58.904348722579\\
69.125	0.18084	-62.3503113987966	-62.3503113987966\\
69.125	0.1845	-65.9185057666639	-65.9185057666639\\
69.125	0.18816	-69.6089318261811	-69.6089318261811\\
69.125	0.19182	-73.421589577348	-73.421589577348\\
69.125	0.19548	-77.3564790201648	-77.3564790201648\\
69.125	0.19914	-81.4136001546313	-81.4136001546313\\
69.125	0.2028	-85.5929529807476	-85.5929529807476\\
69.125	0.20646	-89.8945374985138	-89.8945374985138\\
69.125	0.21012	-94.3183537079296	-94.3183537079296\\
69.125	0.21378	-98.8644016089954	-98.8644016089954\\
69.125	0.21744	-103.532681201711	-103.532681201711\\
69.125	0.2211	-108.323192486076	-108.323192486076\\
69.125	0.22476	-113.235935462091	-113.235935462091\\
69.125	0.22842	-118.270910129756	-118.270910129756\\
69.125	0.23208	-123.428116489071	-123.428116489071\\
69.125	0.23574	-128.707554540035	-128.707554540035\\
69.125	0.2394	-134.109224282649	-134.109224282649\\
69.125	0.24306	-139.633125716913	-139.633125716913\\
69.125	0.24672	-145.279258842827	-145.279258842827\\
69.125	0.25038	-151.047623660391	-151.047623660391\\
69.125	0.25404	-156.938220169604	-156.938220169604\\
69.125	0.2577	-162.951048370468	-162.951048370468\\
69.125	0.26136	-169.086108262981	-169.086108262981\\
69.125	0.26502	-175.343399847144	-175.343399847144\\
69.125	0.26868	-181.722923122956	-181.722923122956\\
69.125	0.27234	-188.224678090419	-188.224678090419\\
69.125	0.276	-194.848664749531	-194.848664749531\\
69.5	0.093	-13.6266942921223	-13.6266942921223\\
69.5	0.09666	-14.2641068801765	-14.2641068801765\\
69.5	0.10032	-15.0237511598807	-15.0237511598807\\
69.5	0.10398	-15.9056271312345	-15.9056271312345\\
69.5	0.10764	-16.9097347942382	-16.9097347942382\\
69.5	0.1113	-18.0360741488917	-18.0360741488917\\
69.5	0.11496	-19.2846451951949	-19.2846451951949\\
69.5	0.11862	-20.655447933148	-20.655447933148\\
69.5	0.12228	-22.1484823627509	-22.1484823627509\\
69.5	0.12594	-23.7637484840035	-23.7637484840035\\
69.5	0.1296	-25.5012462969059	-25.5012462969059\\
69.5	0.13326	-27.3609758014581	-27.3609758014581\\
69.5	0.13692	-29.3429369976602	-29.3429369976602\\
69.5	0.14058	-31.447129885512	-31.447129885512\\
69.5	0.14424	-33.6735544650136	-33.6735544650136\\
69.5	0.1479	-36.022210736165	-36.022210736165\\
69.5	0.15156	-38.4930986989662	-38.4930986989662\\
69.5	0.15522	-41.0862183534172	-41.0862183534172\\
69.5	0.15888	-43.8015696995181	-43.8015696995181\\
69.5	0.16254	-46.6391527372686	-46.6391527372686\\
69.5	0.1662	-49.598967466669	-49.598967466669\\
69.5	0.16986	-52.6810138877192	-52.6810138877192\\
69.5	0.17352	-55.8852920004192	-55.8852920004192\\
69.5	0.17718	-59.2118018047689	-59.2118018047689\\
69.5	0.18084	-62.6605433007685	-62.6605433007685\\
69.5	0.1845	-66.2315164884178	-66.2315164884178\\
69.5	0.18816	-69.924721367717	-69.924721367717\\
69.5	0.19182	-73.740157938666	-73.740157938666\\
69.5	0.19548	-77.6778262012647	-77.6778262012647\\
69.5	0.19914	-81.7377261555132	-81.7377261555132\\
69.5	0.2028	-85.9198578014115	-85.9198578014115\\
69.5	0.20646	-90.2242211389597	-90.2242211389597\\
69.5	0.21012	-94.6508161681576	-94.6508161681576\\
69.5	0.21378	-99.1996428890053	-99.1996428890053\\
69.5	0.21744	-103.870701301503	-103.870701301503\\
69.5	0.2211	-108.66399140565	-108.66399140565\\
69.5	0.22476	-113.579513201447	-113.579513201447\\
69.5	0.22842	-118.617266688894	-118.617266688894\\
69.5	0.23208	-123.777251867991	-123.777251867991\\
69.5	0.23574	-129.059468738737	-129.059468738737\\
69.5	0.2394	-134.463917301133	-134.463917301133\\
69.5	0.24306	-139.99059755518	-139.99059755518\\
69.5	0.24672	-145.639509500875	-145.639509500875\\
69.5	0.25038	-151.410653138221	-151.410653138221\\
69.5	0.25404	-157.304028467217	-157.304028467217\\
69.5	0.2577	-163.319635487862	-163.319635487862\\
69.5	0.26136	-169.457474200157	-169.457474200157\\
69.5	0.26502	-175.717544604102	-175.717544604102\\
69.5	0.26868	-182.099846699696	-182.099846699696\\
69.5	0.27234	-188.604380486941	-188.604380486941\\
69.5	0.276	-195.231145965835	-195.231145965835\\
69.875	0.093	-13.8798308942435	-13.8798308942435\\
69.875	0.09666	-14.5200223020797	-14.5200223020797\\
69.875	0.10032	-15.2824454015659	-15.2824454015659\\
69.875	0.10398	-16.1671001927017	-16.1671001927017\\
69.875	0.10764	-17.1739866754874	-17.1739866754874\\
69.875	0.1113	-18.3031048499229	-18.3031048499229\\
69.875	0.11496	-19.5544547160082	-19.5544547160082\\
69.875	0.11862	-20.9280362737432	-20.9280362737432\\
69.875	0.12228	-22.4238495231281	-22.4238495231281\\
69.875	0.12594	-24.0418944641628	-24.0418944641628\\
69.875	0.1296	-25.7821710968472	-25.7821710968472\\
69.875	0.13326	-27.6446794211814	-27.6446794211814\\
69.875	0.13692	-29.6294194371655	-29.6294194371655\\
69.875	0.14058	-31.7363911447993	-31.7363911447993\\
69.875	0.14424	-33.965594544083	-33.965594544083\\
69.875	0.1479	-36.3170296350163	-36.3170296350163\\
69.875	0.15156	-38.7906964175996	-38.7906964175996\\
69.875	0.15522	-41.3865948918326	-41.3865948918326\\
69.875	0.15888	-44.1047250577154	-44.1047250577154\\
69.875	0.16254	-46.945086915248	-46.945086915248\\
69.875	0.1662	-49.9076804644304	-49.9076804644304\\
69.875	0.16986	-52.9925057052626	-52.9925057052626\\
69.875	0.17352	-56.1995626377445	-56.1995626377445\\
69.875	0.17718	-59.5288512618763	-59.5288512618763\\
69.875	0.18084	-62.9803715776579	-62.9803715776579\\
69.875	0.1845	-66.5541235850893	-66.5541235850893\\
69.875	0.18816	-70.2501072841704	-70.2501072841704\\
69.875	0.19182	-74.0683226749013	-74.0683226749013\\
69.875	0.19548	-78.0087697572821	-78.0087697572821\\
69.875	0.19914	-82.0714485313127	-82.0714485313127\\
69.875	0.2028	-86.256358996993	-86.256358996993\\
69.875	0.20646	-90.5635011543232	-90.5635011543232\\
69.875	0.21012	-94.992875003303	-94.992875003303\\
69.875	0.21378	-99.5444805439328	-99.5444805439328\\
69.875	0.21744	-104.218317776212	-104.218317776212\\
69.875	0.2211	-109.014386700142	-109.014386700142\\
69.875	0.22476	-113.932687315721	-113.932687315721\\
69.875	0.22842	-118.97321962295	-118.97321962295\\
69.875	0.23208	-124.135983621828	-124.135983621828\\
69.875	0.23574	-129.420979312357	-129.420979312357\\
69.875	0.2394	-134.828206694535	-134.828206694535\\
69.875	0.24306	-140.357665768363	-140.357665768363\\
69.875	0.24672	-146.009356533841	-146.009356533841\\
69.875	0.25038	-151.783278990969	-151.783278990969\\
69.875	0.25404	-157.679433139746	-157.679433139746\\
69.875	0.2577	-163.697818980173	-163.697818980173\\
69.875	0.26136	-169.83843651225	-169.83843651225\\
69.875	0.26502	-176.101285735977	-176.101285735977\\
69.875	0.26868	-182.486366651354	-182.486366651354\\
69.875	0.27234	-188.99367925838	-188.99367925838\\
69.875	0.276	-195.623223557056	-195.623223557056\\
70.25	0.093	-14.1425638712821	-14.1425638712821\\
70.25	0.09666	-14.7855340989004	-14.7855340989004\\
70.25	0.10032	-15.5507360181685	-15.5507360181685\\
70.25	0.10398	-16.4381696290864	-16.4381696290864\\
70.25	0.10764	-17.4478349316541	-17.4478349316541\\
70.25	0.1113	-18.5797319258716	-18.5797319258716\\
70.25	0.11496	-19.8338606117388	-19.8338606117388\\
70.25	0.11862	-21.2102209892559	-21.2102209892559\\
70.25	0.12228	-22.7088130584228	-22.7088130584228\\
70.25	0.12594	-24.3296368192395	-24.3296368192395\\
70.25	0.1296	-26.0726922717059	-26.0726922717059\\
70.25	0.13326	-27.9379794158222	-27.9379794158222\\
70.25	0.13692	-29.9254982515882	-29.9254982515882\\
70.25	0.14058	-32.035248779004	-32.035248779004\\
70.25	0.14424	-34.2672309980697	-34.2672309980697\\
70.25	0.1479	-36.6214449087851	-36.6214449087851\\
70.25	0.15156	-39.0978905111504	-39.0978905111504\\
70.25	0.15522	-41.6965678051653	-41.6965678051653\\
70.25	0.15888	-44.4174767908302	-44.4174767908302\\
70.25	0.16254	-47.2606174681448	-47.2606174681448\\
70.25	0.1662	-50.2259898371092	-50.2259898371092\\
70.25	0.16986	-53.3135938977234	-53.3135938977234\\
70.25	0.17352	-56.5234296499873	-56.5234296499873\\
70.25	0.17718	-59.8554970939011	-59.8554970939011\\
70.25	0.18084	-63.3097962294647	-63.3097962294647\\
70.25	0.1845	-66.8863270566781	-66.8863270566781\\
70.25	0.18816	-70.5850895755413	-70.5850895755413\\
70.25	0.19182	-74.4060837860542	-74.4060837860542\\
70.25	0.19548	-78.3493096882169	-78.3493096882169\\
70.25	0.19914	-82.4147672820295	-82.4147672820295\\
70.25	0.2028	-86.6024565674919	-86.6024565674919\\
70.25	0.20646	-90.9123775446041	-90.9123775446041\\
70.25	0.21012	-95.3445302133659	-95.3445302133659\\
70.25	0.21378	-99.8989145737776	-99.8989145737776\\
70.25	0.21744	-104.575530625839	-104.575530625839\\
70.25	0.2211	-109.374378369551	-109.374378369551\\
70.25	0.22476	-114.295457804912	-114.295457804912\\
70.25	0.22842	-119.338768931923	-119.338768931923\\
70.25	0.23208	-124.504311750583	-124.504311750583\\
70.25	0.23574	-129.792086260894	-129.792086260894\\
70.25	0.2394	-135.202092462854	-135.202092462854\\
70.25	0.24306	-140.734330356464	-140.734330356464\\
70.25	0.24672	-146.388799941724	-146.388799941724\\
70.25	0.25038	-152.165501218634	-152.165501218634\\
70.25	0.25404	-158.064434187193	-158.064434187193\\
70.25	0.2577	-164.085598847402	-164.085598847402\\
70.25	0.26136	-170.228995199262	-170.228995199262\\
70.25	0.26502	-176.49462324277	-176.49462324277\\
70.25	0.26868	-182.882482977929	-182.882482977929\\
70.25	0.27234	-189.392574404737	-189.392574404737\\
70.25	0.276	-196.024897523196	-196.024897523196\\
70.625	0.093	-14.4148932232383	-14.4148932232383\\
70.625	0.09666	-15.0606422706386	-15.0606422706386\\
70.625	0.10032	-15.8286230096887	-15.8286230096887\\
70.625	0.10398	-16.7188354403886	-16.7188354403886\\
70.625	0.10764	-17.7312795627383	-17.7312795627383\\
70.625	0.1113	-18.8659553767379	-18.8659553767379\\
70.625	0.11496	-20.1228628823871	-20.1228628823871\\
70.625	0.11862	-21.5020020796862	-21.5020020796862\\
70.625	0.12228	-23.0033729686351	-23.0033729686351\\
70.625	0.12594	-24.6269755492338	-24.6269755492338\\
70.625	0.1296	-26.3728098214822	-26.3728098214822\\
70.625	0.13326	-28.2408757853805	-28.2408757853805\\
70.625	0.13692	-30.2311734409285	-30.2311734409285\\
70.625	0.14058	-32.3437027881264	-32.3437027881264\\
70.625	0.14424	-34.578463826974	-34.578463826974\\
70.625	0.1479	-36.9354565574715	-36.9354565574715\\
70.625	0.15156	-39.4146809796187	-39.4146809796187\\
70.625	0.15522	-42.0161370934158	-42.0161370934158\\
70.625	0.15888	-44.7398248988626	-44.7398248988626\\
70.625	0.16254	-47.5857443959591	-47.5857443959591\\
70.625	0.1662	-50.5538955847056	-50.5538955847056\\
70.625	0.16986	-53.6442784651018	-53.6442784651018\\
70.625	0.17352	-56.8568930371478	-56.8568930371478\\
70.625	0.17718	-60.1917393008435	-60.1917393008435\\
70.625	0.18084	-63.6488172561891	-63.6488172561891\\
70.625	0.1845	-67.2281269031845	-67.2281269031845\\
70.625	0.18816	-70.9296682418297	-70.9296682418297\\
70.625	0.19182	-74.7534412721247	-74.7534412721247\\
70.625	0.19548	-78.6994459940694	-78.6994459940694\\
70.625	0.19914	-82.767682407664	-82.767682407664\\
70.625	0.2028	-86.9581505129084	-86.9581505129084\\
70.625	0.20646	-91.2708503098026	-91.2708503098026\\
70.625	0.21012	-95.7057817983465	-95.7057817983465\\
70.625	0.21378	-100.26294497854	-100.26294497854\\
70.625	0.21744	-104.942339850384	-104.942339850384\\
70.625	0.2211	-109.743966413877	-109.743966413877\\
70.625	0.22476	-114.66782466902	-114.66782466902\\
70.625	0.22842	-119.713914615813	-119.713914615813\\
70.625	0.23208	-124.882236254256	-124.882236254256\\
70.625	0.23574	-130.172789584348	-130.172789584348\\
70.625	0.2394	-135.585574606091	-135.585574606091\\
70.625	0.24306	-141.120591319483	-141.120591319483\\
70.625	0.24672	-146.777839724525	-146.777839724525\\
70.625	0.25038	-152.557319821216	-152.557319821216\\
70.625	0.25404	-158.459031609558	-158.459031609558\\
70.625	0.2577	-164.482975089549	-164.482975089549\\
70.625	0.26136	-170.62915026119	-170.62915026119\\
70.625	0.26502	-176.897557124481	-176.897557124481\\
70.625	0.26868	-183.288195679422	-183.288195679422\\
70.625	0.27234	-189.801065926012	-189.801065926012\\
70.625	0.276	-196.436167864252	-196.436167864252\\
71	0.093	-14.6968189501119	-14.6968189501119\\
71	0.09666	-15.3453468172942	-15.3453468172942\\
71	0.10032	-16.1161063761264	-16.1161063761264\\
71	0.10398	-17.0090976266083	-17.0090976266083\\
71	0.10764	-18.02432056874	-18.02432056874\\
71	0.1113	-19.1617752025215	-19.1617752025215\\
71	0.11496	-20.4214615279528	-20.4214615279528\\
71	0.11862	-21.8033795450339	-21.8033795450339\\
71	0.12228	-23.3075292537648	-23.3075292537648\\
71	0.12594	-24.9339106541455	-24.9339106541455\\
71	0.1296	-26.6825237461759	-26.6825237461759\\
71	0.13326	-28.5533685298562	-28.5533685298562\\
71	0.13692	-30.5464450051862	-30.5464450051862\\
71	0.14058	-32.6617531721661	-32.6617531721661\\
71	0.14424	-34.8992930307958	-34.8992930307958\\
71	0.1479	-37.2590645810752	-37.2590645810752\\
71	0.15156	-39.7410678230044	-39.7410678230044\\
71	0.15522	-42.3453027565835	-42.3453027565835\\
71	0.15888	-45.0717693818123	-45.0717693818123\\
71	0.16254	-47.9204676986909	-47.9204676986909\\
71	0.1662	-50.8913977072194	-50.8913977072194\\
71	0.16986	-53.9845594073976	-53.9845594073976\\
71	0.17352	-57.1999527992256	-57.1999527992256\\
71	0.17718	-60.5375778827033	-60.5375778827033\\
71	0.18084	-63.997434657831	-63.997434657831\\
71	0.1845	-67.5795231246083	-67.5795231246083\\
71	0.18816	-71.2838432830355	-71.2838432830355\\
71	0.19182	-75.1103951331126	-75.1103951331126\\
71	0.19548	-79.0591786748393	-79.0591786748393\\
71	0.19914	-83.1301939082159	-83.1301939082159\\
71	0.2028	-87.3234408332422	-87.3234408332422\\
71	0.20646	-91.6389194499184	-91.6389194499184\\
71	0.21012	-96.0766297582444	-96.0766297582444\\
71	0.21378	-100.63657175822	-100.63657175822\\
71	0.21744	-105.318745449846	-105.318745449846\\
71	0.2211	-110.123150833121	-110.123150833121\\
71	0.22476	-115.049787908046	-115.049787908046\\
71	0.22842	-120.098656674621	-120.098656674621\\
71	0.23208	-125.269757132846	-125.269757132846\\
71	0.23574	-130.56308928272	-130.56308928272\\
71	0.2394	-135.978653124245	-135.978653124245\\
71	0.24306	-141.516448657419	-141.516448657419\\
71	0.24672	-147.176475882243	-147.176475882243\\
71	0.25038	-152.958734798716	-152.958734798716\\
71	0.25404	-158.86322540684	-158.86322540684\\
71	0.2577	-164.889947706613	-164.889947706613\\
71	0.26136	-171.038901698036	-171.038901698036\\
71	0.26502	-177.310087381109	-177.310087381109\\
71	0.26868	-183.703504755832	-183.703504755832\\
71	0.27234	-190.219153822204	-190.219153822204\\
71	0.276	-196.857034580226	-196.857034580226\\
71.375	0.093	-14.9883410519031	-14.9883410519031\\
71.375	0.09666	-15.6396477388673	-15.6396477388673\\
71.375	0.10032	-16.4131861174815	-16.4131861174815\\
71.375	0.10398	-17.3089561877454	-17.3089561877454\\
71.375	0.10764	-18.3269579496591	-18.3269579496591\\
71.375	0.1113	-19.4671914032227	-19.4671914032227\\
71.375	0.11496	-20.729656548436	-20.729656548436\\
71.375	0.11862	-22.114353385299	-22.114353385299\\
71.375	0.12228	-23.621281913812	-23.621281913812\\
71.375	0.12594	-25.2504421339747	-25.2504421339747\\
71.375	0.1296	-27.0018340457871	-27.0018340457871\\
71.375	0.13326	-28.8754576492494	-28.8754576492494\\
71.375	0.13692	-30.8713129443615	-30.8713129443615\\
71.375	0.14058	-32.9893999311234	-32.9893999311234\\
71.375	0.14424	-35.229718609535	-35.229718609535\\
71.375	0.1479	-37.5922689795965	-37.5922689795965\\
71.375	0.15156	-40.0770510413077	-40.0770510413077\\
71.375	0.15522	-42.6840647946687	-42.6840647946687\\
71.375	0.15888	-45.4133102396796	-45.4133102396796\\
71.375	0.16254	-48.2647873763403	-48.2647873763403\\
71.375	0.1662	-51.2384962046507	-51.2384962046507\\
71.375	0.16986	-54.3344367246109	-54.3344367246109\\
71.375	0.17352	-57.5526089362209	-57.5526089362209\\
71.375	0.17718	-60.8930128394807	-60.8930128394807\\
71.375	0.18084	-64.3556484343903	-64.3556484343903\\
71.375	0.1845	-67.9405157209497	-67.9405157209497\\
71.375	0.18816	-71.6476146991589	-71.6476146991589\\
71.375	0.19182	-75.4769453690179	-75.4769453690179\\
71.375	0.19548	-79.4285077305267	-79.4285077305267\\
71.375	0.19914	-83.5023017836852	-83.5023017836852\\
71.375	0.2028	-87.6983275284936	-87.6983275284936\\
71.375	0.20646	-92.0165849649518	-92.0165849649518\\
71.375	0.21012	-96.4570740930598	-96.4570740930598\\
71.375	0.21378	-101.019794912817	-101.019794912817\\
71.375	0.21744	-105.704747424225	-105.704747424225\\
71.375	0.2211	-110.511931627282	-110.511931627282\\
71.375	0.22476	-115.44134752199	-115.44134752199\\
71.375	0.22842	-120.492995108346	-120.492995108346\\
71.375	0.23208	-125.666874386353	-125.666874386353\\
71.375	0.23574	-130.96298535601	-130.96298535601\\
71.375	0.2394	-136.381328017316	-136.381328017316\\
71.375	0.24306	-141.921902370272	-141.921902370272\\
71.375	0.24672	-147.584708414878	-147.584708414878\\
71.375	0.25038	-153.369746151134	-153.369746151134\\
71.375	0.25404	-159.277015579039	-159.277015579039\\
71.375	0.2577	-165.306516698595	-165.306516698595\\
71.375	0.26136	-171.4582495098	-171.4582495098\\
71.375	0.26502	-177.732214012654	-177.732214012654\\
71.375	0.26868	-184.128410207159	-184.128410207159\\
71.375	0.27234	-190.646838093314	-190.646838093314\\
71.375	0.276	-197.287497671118	-197.287497671118\\
71.75	0.093	-15.2894595286117	-15.2894595286117\\
71.75	0.09666	-15.943545035358	-15.943545035358\\
71.75	0.10032	-16.7198622337541	-16.7198622337541\\
71.75	0.10398	-17.6184111238001	-17.6184111238001\\
71.75	0.10764	-18.6391917054958	-18.6391917054958\\
71.75	0.1113	-19.7822039788413	-19.7822039788413\\
71.75	0.11496	-21.0474479438366	-21.0474479438366\\
71.75	0.11862	-22.4349236004817	-22.4349236004817\\
71.75	0.12228	-23.9446309487766	-23.9446309487766\\
71.75	0.12594	-25.5765699887213	-25.5765699887213\\
71.75	0.1296	-27.3307407203158	-27.3307407203158\\
71.75	0.13326	-29.2071431435601	-29.2071431435601\\
71.75	0.13692	-31.2057772584542	-31.2057772584542\\
71.75	0.14058	-33.3266430649981	-33.3266430649981\\
71.75	0.14424	-35.5697405631917	-35.5697405631917\\
71.75	0.1479	-37.9350697530352	-37.9350697530352\\
71.75	0.15156	-40.4226306345285	-40.4226306345285\\
71.75	0.15522	-43.0324232076715	-43.0324232076715\\
71.75	0.15888	-45.7644474724644	-45.7644474724644\\
71.75	0.16254	-48.618703428907	-48.618703428907\\
71.75	0.1662	-51.5951910769994	-51.5951910769994\\
71.75	0.16986	-54.6939104167417	-54.6939104167417\\
71.75	0.17352	-57.9148614481337	-57.9148614481337\\
71.75	0.17718	-61.2580441711755	-61.2580441711755\\
71.75	0.18084	-64.7234585858671	-64.7234585858671\\
71.75	0.1845	-68.3111046922086	-68.3111046922086\\
71.75	0.18816	-72.0209824901997	-72.0209824901997\\
71.75	0.19182	-75.8530919798407	-75.8530919798407\\
71.75	0.19548	-79.8074331611315	-79.8074331611315\\
71.75	0.19914	-83.8840060340721	-83.8840060340721\\
71.75	0.2028	-88.0828105986625	-88.0828105986625\\
71.75	0.20646	-92.4038468549027	-92.4038468549027\\
71.75	0.21012	-96.8471148027926	-96.8471148027926\\
71.75	0.21378	-101.412614442332	-101.412614442332\\
71.75	0.21744	-106.100345773522	-106.100345773522\\
71.75	0.2211	-110.910308796361	-110.910308796361\\
71.75	0.22476	-115.842503510851	-115.842503510851\\
71.75	0.22842	-120.896929916989	-120.896929916989\\
71.75	0.23208	-126.073588014778	-126.073588014778\\
71.75	0.23574	-131.372477804217	-131.372477804217\\
71.75	0.2394	-136.793599285305	-136.793599285305\\
71.75	0.24306	-142.336952458043	-142.336952458043\\
71.75	0.24672	-148.002537322431	-148.002537322431\\
71.75	0.25038	-153.790353878469	-153.790353878469\\
71.75	0.25404	-159.700402126156	-159.700402126156\\
71.75	0.2577	-165.732682065494	-165.732682065494\\
71.75	0.26136	-171.887193696481	-171.887193696481\\
71.75	0.26502	-178.163937019117	-178.163937019117\\
71.75	0.26868	-184.562912033404	-184.562912033404\\
71.75	0.27234	-191.084118739341	-191.084118739341\\
71.75	0.276	-197.727557136927	-197.727557136927\\
72.125	0.093	-15.6001743802378	-15.6001743802378\\
72.125	0.09666	-16.2570387067661	-16.2570387067661\\
72.125	0.10032	-17.0361347249443	-17.0361347249443\\
72.125	0.10398	-17.9374624347722	-17.9374624347722\\
72.125	0.10764	-18.9610218362499	-18.9610218362499\\
72.125	0.1113	-20.1068129293775	-20.1068129293775\\
72.125	0.11496	-21.3748357141548	-21.3748357141548\\
72.125	0.11862	-22.7650901905819	-22.7650901905819\\
72.125	0.12228	-24.2775763586589	-24.2775763586589\\
72.125	0.12594	-25.9122942183856	-25.9122942183856\\
72.125	0.1296	-27.669243769762	-27.669243769762\\
72.125	0.13326	-29.5484250127883	-29.5484250127883\\
72.125	0.13692	-31.5498379474644	-31.5498379474644\\
72.125	0.14058	-33.6734825737903	-33.6734825737903\\
72.125	0.14424	-35.919358891766	-35.919358891766\\
72.125	0.1479	-38.2874669013915	-38.2874669013915\\
72.125	0.15156	-40.7778066026668	-40.7778066026668\\
72.125	0.15522	-43.3903779955918	-43.3903779955918\\
72.125	0.15888	-46.1251810801667	-46.1251810801667\\
72.125	0.16254	-48.9822158563913	-48.9822158563913\\
72.125	0.1662	-51.9614823242658	-51.9614823242658\\
72.125	0.16986	-55.06298048379	-55.06298048379\\
72.125	0.17352	-58.2867103349641	-58.2867103349641\\
72.125	0.17718	-61.6326718777879	-61.6326718777879\\
72.125	0.18084	-65.1008651122615	-65.1008651122615\\
72.125	0.1845	-68.6912900383849	-68.6912900383849\\
72.125	0.18816	-72.4039466561581	-72.4039466561581\\
72.125	0.19182	-76.2388349655812	-76.2388349655812\\
72.125	0.19548	-80.1959549666539	-80.1959549666539\\
72.125	0.19914	-84.2753066593765	-84.2753066593765\\
72.125	0.2028	-88.4768900437489	-88.4768900437489\\
72.125	0.20646	-92.8007051197711	-92.8007051197711\\
72.125	0.21012	-97.2467518874431	-97.2467518874431\\
72.125	0.21378	-101.815030346765	-101.815030346765\\
72.125	0.21744	-106.505540497736	-106.505540497736\\
72.125	0.2211	-111.318282340358	-111.318282340358\\
72.125	0.22476	-116.253255874629	-116.253255874629\\
72.125	0.22842	-121.31046110055	-121.31046110055\\
72.125	0.23208	-126.489898018121	-126.489898018121\\
72.125	0.23574	-131.791566627341	-131.791566627341\\
72.125	0.2394	-137.215466928211	-137.215466928211\\
72.125	0.24306	-142.761598920732	-142.761598920732\\
72.125	0.24672	-148.429962604902	-148.429962604902\\
72.125	0.25038	-154.220557980721	-154.220557980721\\
72.125	0.25404	-160.133385048191	-160.133385048191\\
72.125	0.2577	-166.16844380731	-166.16844380731\\
72.125	0.26136	-172.325734258079	-172.325734258079\\
72.125	0.26502	-178.605256400498	-178.605256400498\\
72.125	0.26868	-185.007010234567	-185.007010234567\\
72.125	0.27234	-191.530995760285	-191.530995760285\\
72.125	0.276	-198.177212977653	-198.177212977653\\
72.5	0.093	-15.9204856067813	-15.9204856067813\\
72.5	0.09666	-16.5801287530917	-16.5801287530917\\
72.5	0.10032	-17.3620035910519	-17.3620035910519\\
72.5	0.10398	-18.2661101206618	-18.2661101206618\\
72.5	0.10764	-19.2924483419215	-19.2924483419215\\
72.5	0.1113	-20.4410182548311	-20.4410182548311\\
72.5	0.11496	-21.7118198593904	-21.7118198593904\\
72.5	0.11862	-23.1048531555996	-23.1048531555996\\
72.5	0.12228	-24.6201181434585	-24.6201181434585\\
72.5	0.12594	-26.2576148229672	-26.2576148229672\\
72.5	0.1296	-28.0173431941257	-28.0173431941257\\
72.5	0.13326	-29.899303256934	-29.899303256934\\
72.5	0.13692	-31.9034950113921	-31.9034950113921\\
72.5	0.14058	-34.0299184575	-34.0299184575\\
72.5	0.14424	-36.2785735952577	-36.2785735952577\\
72.5	0.1479	-38.6494604246652	-38.6494604246652\\
72.5	0.15156	-41.1425789457224	-41.1425789457224\\
72.5	0.15522	-43.7579291584295	-43.7579291584295\\
72.5	0.15888	-46.4955110627864	-46.4955110627864\\
72.5	0.16254	-49.355324658793	-49.355324658793\\
72.5	0.1662	-52.3373699464495	-52.3373699464495\\
72.5	0.16986	-55.4416469257558	-55.4416469257558\\
72.5	0.17352	-58.6681555967118	-58.6681555967118\\
72.5	0.17718	-62.0168959593176	-62.0168959593176\\
72.5	0.18084	-65.4878680135732	-65.4878680135732\\
72.5	0.1845	-69.0810717594787	-69.0810717594787\\
72.5	0.18816	-72.7965071970339	-72.7965071970339\\
72.5	0.19182	-76.6341743262389	-76.6341743262389\\
72.5	0.19548	-80.5940731470937	-80.5940731470937\\
72.5	0.19914	-84.6762036595983	-84.6762036595983\\
72.5	0.2028	-88.8805658637528	-88.8805658637528\\
72.5	0.20646	-93.207159759557	-93.207159759557\\
72.5	0.21012	-97.655985347011	-97.655985347011\\
72.5	0.21378	-102.227042626115	-102.227042626115\\
72.5	0.21744	-106.920331596868	-106.920331596868\\
72.5	0.2211	-111.735852259272	-111.735852259272\\
72.5	0.22476	-116.673604613325	-116.673604613325\\
72.5	0.22842	-121.733588659028	-121.733588659028\\
72.5	0.23208	-126.91580439638	-126.91580439638\\
72.5	0.23574	-132.220251825383	-132.220251825383\\
72.5	0.2394	-137.646930946035	-137.646930946035\\
72.5	0.24306	-143.195841758337	-143.195841758337\\
72.5	0.24672	-148.866984262289	-148.866984262289\\
72.5	0.25038	-154.660358457891	-154.660358457891\\
72.5	0.25404	-160.575964345143	-160.575964345143\\
72.5	0.2577	-166.613801924044	-166.613801924044\\
72.5	0.26136	-172.773871194595	-172.773871194595\\
72.5	0.26502	-179.056172156796	-179.056172156796\\
72.5	0.26868	-185.460704810647	-185.460704810647\\
72.5	0.27234	-191.987469156147	-191.987469156147\\
72.5	0.276	-198.636465193297	-198.636465193297\\
72.875	0.093	-16.2503932082425	-16.2503932082425\\
72.875	0.09666	-16.9128151743348	-16.9128151743348\\
72.875	0.10032	-17.697468832077	-17.697468832077\\
72.875	0.10398	-18.604354181469	-18.604354181469\\
72.875	0.10764	-19.6334712225107	-19.6334712225107\\
72.875	0.1113	-20.7848199552023	-20.7848199552023\\
72.875	0.11496	-22.0584003795436	-22.0584003795436\\
72.875	0.11862	-23.4542124955347	-23.4542124955347\\
72.875	0.12228	-24.9722563031757	-24.9722563031757\\
72.875	0.12594	-26.6125318024664	-26.6125318024664\\
72.875	0.1296	-28.3750389934069	-28.3750389934069\\
72.875	0.13326	-30.2597778759972	-30.2597778759972\\
72.875	0.13692	-32.2667484502373	-32.2667484502373\\
72.875	0.14058	-34.3959507161272	-34.3959507161272\\
72.875	0.14424	-36.647384673667	-36.647384673667\\
72.875	0.1479	-39.0210503228564	-39.0210503228564\\
72.875	0.15156	-41.5169476636957	-41.5169476636957\\
72.875	0.15522	-44.1350766961848	-44.1350766961848\\
72.875	0.15888	-46.8754374203237	-46.8754374203237\\
72.875	0.16254	-49.7380298361124	-49.7380298361124\\
72.875	0.1662	-52.7228539435508	-52.7228539435508\\
72.875	0.16986	-55.8299097426391	-55.8299097426391\\
72.875	0.17352	-59.0591972333771	-59.0591972333771\\
72.875	0.17718	-62.410716415765	-62.410716415765\\
72.875	0.18084	-65.8844672898026	-65.8844672898026\\
72.875	0.1845	-69.48044985549	-69.48044985549\\
72.875	0.18816	-73.1986641128273	-73.1986641128273\\
72.875	0.19182	-77.0391100618143	-77.0391100618143\\
72.875	0.19548	-81.0017877024511	-81.0017877024511\\
72.875	0.19914	-85.0866970347377	-85.0866970347377\\
72.875	0.2028	-89.2938380586742	-89.2938380586742\\
72.875	0.20646	-93.6232107742604	-93.6232107742604\\
72.875	0.21012	-98.0748151814963	-98.0748151814963\\
72.875	0.21378	-102.648651280382	-102.648651280382\\
72.875	0.21744	-107.344719070918	-107.344719070918\\
72.875	0.2211	-112.163018553103	-112.163018553103\\
72.875	0.22476	-117.103549726938	-117.103549726938\\
72.875	0.22842	-122.166312592423	-122.166312592423\\
72.875	0.23208	-127.351307149558	-127.351307149558\\
72.875	0.23574	-132.658533398343	-132.658533398343\\
72.875	0.2394	-138.087991338777	-138.087991338777\\
72.875	0.24306	-143.639680970861	-143.639680970861\\
72.875	0.24672	-149.313602294595	-149.313602294595\\
72.875	0.25038	-155.109755309979	-155.109755309979\\
72.875	0.25404	-161.028140017012	-161.028140017012\\
72.875	0.2577	-167.068756415696	-167.068756415696\\
72.875	0.26136	-173.231604506029	-173.231604506029\\
72.875	0.26502	-179.516684288012	-179.516684288012\\
72.875	0.26868	-185.923995761644	-185.923995761644\\
72.875	0.27234	-192.453538926927	-192.453538926927\\
72.875	0.276	-199.105313783859	-199.105313783859\\
73.25	0.093	-16.5898971846211	-16.5898971846211\\
73.25	0.09666	-17.2550979704954	-17.2550979704954\\
73.25	0.10032	-18.0425304480196	-18.0425304480196\\
73.25	0.10398	-18.9521946171936	-18.9521946171936\\
73.25	0.10764	-19.9840904780173	-19.9840904780173\\
73.25	0.1113	-21.1382180304909	-21.1382180304909\\
73.25	0.11496	-22.4145772746142	-22.4145772746142\\
73.25	0.11862	-23.8131682103874	-23.8131682103874\\
73.25	0.12228	-25.3339908378104	-25.3339908378104\\
73.25	0.12594	-26.9770451568831	-26.9770451568831\\
73.25	0.1296	-28.7423311676056	-28.7423311676056\\
73.25	0.13326	-30.6298488699779	-30.6298488699779\\
73.25	0.13692	-32.6395982640001	-32.6395982640001\\
73.25	0.14058	-34.771579349672	-34.771579349672\\
73.25	0.14424	-37.0257921269937	-37.0257921269937\\
73.25	0.1479	-39.4022365959652	-39.4022365959652\\
73.25	0.15156	-41.9009127565865	-41.9009127565865\\
73.25	0.15522	-44.5218206088576	-44.5218206088576\\
73.25	0.15888	-47.2649601527785	-47.2649601527785\\
73.25	0.16254	-50.1303313883491	-50.1303313883491\\
73.25	0.1662	-53.1179343155696	-53.1179343155696\\
73.25	0.16986	-56.2277689344399	-56.2277689344399\\
73.25	0.17352	-59.4598352449599	-59.4598352449599\\
73.25	0.17718	-62.8141332471298	-62.8141332471298\\
73.25	0.18084	-66.2906629409494	-66.2906629409494\\
73.25	0.1845	-69.8894243264188	-69.8894243264188\\
73.25	0.18816	-73.6104174035381	-73.6104174035381\\
73.25	0.19182	-77.4536421723072	-77.4536421723072\\
73.25	0.19548	-81.419098632726	-81.419098632726\\
73.25	0.19914	-85.5067867847946	-85.5067867847946\\
73.25	0.2028	-89.716706628513	-89.716706628513\\
73.25	0.20646	-94.0488581638812	-94.0488581638812\\
73.25	0.21012	-98.5032413908993	-98.5032413908993\\
73.25	0.21378	-103.079856309567	-103.079856309567\\
73.25	0.21744	-107.778702919885	-107.778702919885\\
73.25	0.2211	-112.599781221852	-112.599781221852\\
73.25	0.22476	-117.543091215469	-117.543091215469\\
73.25	0.22842	-122.608632900736	-122.608632900736\\
73.25	0.23208	-127.796406277653	-127.796406277653\\
73.25	0.23574	-133.10641134622	-133.10641134622\\
73.25	0.2394	-138.538648106436	-138.538648106436\\
73.25	0.24306	-144.093116558302	-144.093116558302\\
73.25	0.24672	-149.769816701818	-149.769816701818\\
73.25	0.25038	-155.568748536984	-155.568748536984\\
73.25	0.25404	-161.489912063799	-161.489912063799\\
73.25	0.2577	-167.533307282265	-167.533307282265\\
73.25	0.26136	-173.69893419238	-173.69893419238\\
73.25	0.26502	-179.986792794145	-179.986792794145\\
73.25	0.26868	-186.396883087559	-186.396883087559\\
73.25	0.27234	-192.929205072624	-192.929205072624\\
73.25	0.276	-199.583758749338	-199.583758749338\\
73.625	0.093	-16.9389975359171	-16.9389975359171\\
73.625	0.09666	-17.6069771415735	-17.6069771415735\\
73.625	0.10032	-18.3971884388797	-18.3971884388797\\
73.625	0.10398	-19.3096314278356	-19.3096314278356\\
73.625	0.10764	-20.3443061084414	-20.3443061084414\\
73.625	0.1113	-21.501212480697	-21.501212480697\\
73.625	0.11496	-22.7803505446023	-22.7803505446023\\
73.625	0.11862	-24.1817203001575	-24.1817203001575\\
73.625	0.12228	-25.7053217473625	-25.7053217473625\\
73.625	0.12594	-27.3511548862172	-27.3511548862172\\
73.625	0.1296	-29.1192197167217	-29.1192197167217\\
73.625	0.13326	-31.0095162388761	-31.0095162388761\\
73.625	0.13692	-33.0220444526802	-33.0220444526802\\
73.625	0.14058	-35.1568043581341	-35.1568043581341\\
73.625	0.14424	-37.4137959552378	-37.4137959552378\\
73.625	0.1479	-39.7930192439913	-39.7930192439913\\
73.625	0.15156	-42.2944742243947	-42.2944742243947\\
73.625	0.15522	-44.9181608964478	-44.9181608964478\\
73.625	0.15888	-47.6640792601506	-47.6640792601506\\
73.625	0.16254	-50.5322293155033	-50.5322293155033\\
73.625	0.1662	-53.5226110625058	-53.5226110625058\\
73.625	0.16986	-56.6352245011581	-56.6352245011581\\
73.625	0.17352	-59.8700696314602	-59.8700696314602\\
73.625	0.17718	-63.227146453412	-63.227146453412\\
73.625	0.18084	-66.7064549670137	-66.7064549670137\\
73.625	0.1845	-70.3079951722651	-70.3079951722651\\
73.625	0.18816	-74.0317670691664	-74.0317670691664\\
73.625	0.19182	-77.8777706577174	-77.8777706577174\\
73.625	0.19548	-81.8460059379182	-81.8460059379182\\
73.625	0.19914	-85.9364729097689	-85.9364729097689\\
73.625	0.2028	-90.1491715732693	-90.1491715732693\\
73.625	0.20646	-94.4841019284196	-94.4841019284196\\
73.625	0.21012	-98.9412639752196	-98.9412639752196\\
73.625	0.21378	-103.520657713669	-103.520657713669\\
73.625	0.21744	-108.222283143769	-108.222283143769\\
73.625	0.2211	-113.046140265518	-113.046140265518\\
73.625	0.22476	-117.992229078917	-117.992229078917\\
73.625	0.22842	-123.060549583967	-123.060549583967\\
73.625	0.23208	-128.251101780665	-128.251101780665\\
73.625	0.23574	-133.563885669014	-133.563885669014\\
73.625	0.2394	-138.998901249012	-138.998901249012\\
73.625	0.24306	-144.55614852066	-144.55614852066\\
73.625	0.24672	-150.235627483958	-150.235627483958\\
73.625	0.25038	-156.037338138906	-156.037338138906\\
73.625	0.25404	-161.961280485504	-161.961280485504\\
73.625	0.2577	-168.007454523751	-168.007454523751\\
73.625	0.26136	-174.175860253648	-174.175860253648\\
73.625	0.26502	-180.466497675195	-180.466497675195\\
73.625	0.26868	-186.879366788392	-186.879366788392\\
73.625	0.27234	-193.414467593238	-193.414467593238\\
73.625	0.276	-200.071800089735	-200.071800089735\\
74	0.093	-17.2976942621307	-17.2976942621307\\
74	0.09666	-17.9684526875691	-17.9684526875691\\
74	0.10032	-18.7614428046573	-18.7614428046573\\
74	0.10398	-19.6766646133953	-19.6766646133953\\
74	0.10764	-20.714118113783	-20.714118113783\\
74	0.1113	-21.8738033058206	-21.8738033058206\\
74	0.11496	-23.155720189508	-23.155720189508\\
74	0.11862	-24.5598687648451	-24.5598687648451\\
74	0.12228	-26.0862490318321	-26.0862490318321\\
74	0.12594	-27.7348609904689	-27.7348609904689\\
74	0.1296	-29.5057046407554	-29.5057046407554\\
74	0.13326	-31.3987799826917	-31.3987799826917\\
74	0.13692	-33.4140870162778	-33.4140870162778\\
74	0.14058	-35.5516257415138	-35.5516257415138\\
74	0.14424	-37.8113961583996	-37.8113961583996\\
74	0.1479	-40.193398266935	-40.193398266935\\
74	0.15156	-42.6976320671204	-42.6976320671204\\
74	0.15522	-45.3240975589554	-45.3240975589554\\
74	0.15888	-48.0727947424404	-48.0727947424404\\
74	0.16254	-50.9437236175751	-50.9437236175751\\
74	0.1662	-53.9368841843596	-53.9368841843596\\
74	0.16986	-57.0522764427938	-57.0522764427938\\
74	0.17352	-60.2899003928779	-60.2899003928779\\
74	0.17718	-63.6497560346118	-63.6497560346118\\
74	0.18084	-67.1318433679955	-67.1318433679955\\
74	0.1845	-70.7361623930289	-70.7361623930289\\
74	0.18816	-74.4627131097122	-74.4627131097122\\
74	0.19182	-78.3114955180452	-78.3114955180452\\
74	0.19548	-82.2825096180281	-82.2825096180281\\
74	0.19914	-86.3757554096607	-86.3757554096607\\
74	0.2028	-90.5912328929432	-90.5912328929432\\
74	0.20646	-94.9289420678754	-94.9289420678754\\
74	0.21012	-99.3888829344574	-99.3888829344574\\
74	0.21378	-103.971055492689	-103.971055492689\\
74	0.21744	-108.675459742571	-108.675459742571\\
74	0.2211	-113.502095684102	-113.502095684102\\
74	0.22476	-118.450963317283	-118.450963317283\\
74	0.22842	-123.522062642114	-123.522062642114\\
74	0.23208	-128.715393658595	-128.715393658595\\
74	0.23574	-134.030956366726	-134.030956366726\\
74	0.2394	-139.468750766506	-139.468750766506\\
74	0.24306	-145.028776857936	-145.028776857936\\
74	0.24672	-150.711034641016	-150.711034641016\\
74	0.25038	-156.515524115746	-156.515524115746\\
74	0.25404	-162.442245282126	-162.442245282126\\
74	0.2577	-168.491198140155	-168.491198140155\\
74	0.26136	-174.662382689834	-174.662382689834\\
74	0.26502	-180.955798931163	-180.955798931163\\
74	0.26868	-187.371446864142	-187.371446864142\\
74	0.27234	-193.90932648877	-193.90932648877\\
74	0.276	-200.569437805049	-200.569437805049\\
};\label{tikz:theta_surf}
\end{axis}
\end{tikzpicture}%
	\end{minipage}}
	% This file was created by matlab2tikz.
%
\definecolor{mycolor1}{rgb}{0.00000,0.44700,0.74100}%
\definecolor{mycolor2}{rgb}{0.85000,0.32500,0.09800}%
%
\begin{tikzpicture}

\begin{axis}[%
width=6.159cm,
height=3.097cm,
at={(0cm,12.903cm)},
scale only axis,
xmin=56,
xmax=74,
tick align=outside,
xlabel style={font=\color{white!15!black}},
xlabel={$L_{cut}$},
ymin=0.093,
ymax=0.276,
ylabel style={font=\color{white!15!black}},
ylabel={$D_{rlx}$},
zmin=-200.569447858151,
zmax=0,
zlabel style={font=\color{white!15!black}},
zlabel={$u(t)u(t)$},
view={-140}{50},
axis background/.style={fill=white},
xmajorgrids,
ymajorgrids,
zmajorgrids
]
\addplot3[only marks, mark=*, mark options={}, mark size=1.5000pt, color=mycolor1, fill=mycolor1] table[row sep=crcr]{%
x	y	z\\
74	0.123	-26.0353957891804\\
72	0.113	-20.9879322169279\\
61	0.095	-10.6920630070547\\
56	0.093	-10.9569379219045\\
};
\addplot3[only marks, mark=*, mark options={}, mark size=1.5000pt, color=mycolor2, fill=mycolor2] table[row sep=crcr]{%
x	y	z\\
67	0.276	-191.779551108501\\
66	0.255	-157.643535964989\\
62	0.209	-87.4196213968237\\
57	0.193	-69.8013569503948\\
};
\addplot3[only marks, mark=*, mark options={}, mark size=1.5000pt, color=black, fill=black] table[row sep=crcr]{%
x	y	z\\
69	0.104	-15.5770567955152\\
};
\addplot3[only marks, mark=*, mark options={}, mark size=1.5000pt, color=black, fill=black] table[row sep=crcr]{%
x	y	z\\
64	0.23	-116.694088634917\\
};

\addplot3[%
surf,
fill opacity=0.7, shader=interp, colormap={mymap}{[1pt] rgb(0pt)=(1,0.905882,0); rgb(1pt)=(1,0.901964,0); rgb(2pt)=(1,0.898051,0); rgb(3pt)=(1,0.894144,0); rgb(4pt)=(1,0.890243,0); rgb(5pt)=(1,0.886349,0); rgb(6pt)=(1,0.88246,0); rgb(7pt)=(1,0.878577,0); rgb(8pt)=(1,0.8747,0); rgb(9pt)=(1,0.870829,0); rgb(10pt)=(1,0.866964,0); rgb(11pt)=(1,0.863106,0); rgb(12pt)=(1,0.859253,0); rgb(13pt)=(1,0.855406,0); rgb(14pt)=(1,0.851566,0); rgb(15pt)=(1,0.847732,0); rgb(16pt)=(1,0.843903,0); rgb(17pt)=(1,0.840081,0); rgb(18pt)=(1,0.836265,0); rgb(19pt)=(1,0.832455,0); rgb(20pt)=(1,0.828652,0); rgb(21pt)=(1,0.824854,0); rgb(22pt)=(1,0.821063,0); rgb(23pt)=(1,0.817278,0); rgb(24pt)=(1,0.8135,0); rgb(25pt)=(1,0.809727,0); rgb(26pt)=(1,0.805961,0); rgb(27pt)=(1,0.8022,0); rgb(28pt)=(1,0.798445,0); rgb(29pt)=(1,0.794696,0); rgb(30pt)=(1,0.790953,0); rgb(31pt)=(1,0.787215,0); rgb(32pt)=(1,0.783484,0); rgb(33pt)=(1,0.779758,0); rgb(34pt)=(1,0.776038,0); rgb(35pt)=(1,0.772324,0); rgb(36pt)=(1,0.768615,0); rgb(37pt)=(1,0.764913,0); rgb(38pt)=(1,0.761217,0); rgb(39pt)=(1,0.757527,0); rgb(40pt)=(1,0.753843,0); rgb(41pt)=(1,0.750165,0); rgb(42pt)=(1,0.746493,0); rgb(43pt)=(1,0.742827,0); rgb(44pt)=(1,0.739167,0); rgb(45pt)=(1,0.735514,0); rgb(46pt)=(1,0.731867,0); rgb(47pt)=(1,0.728226,0); rgb(48pt)=(1,0.724591,0); rgb(49pt)=(1,0.720963,0); rgb(50pt)=(1,0.717341,0); rgb(51pt)=(1,0.713725,0); rgb(52pt)=(0.999994,0.710077,0); rgb(53pt)=(0.999974,0.706363,0); rgb(54pt)=(0.999942,0.702592,0); rgb(55pt)=(0.999898,0.698775,0); rgb(56pt)=(0.999841,0.694921,0); rgb(57pt)=(0.999771,0.691039,0); rgb(58pt)=(0.99969,0.687139,0); rgb(59pt)=(0.999596,0.68323,0); rgb(60pt)=(0.99949,0.679323,0); rgb(61pt)=(0.999372,0.675427,0); rgb(62pt)=(0.999242,0.67155,0); rgb(63pt)=(0.9991,0.667704,0); rgb(64pt)=(0.998946,0.663897,0); rgb(65pt)=(0.998781,0.660138,0); rgb(66pt)=(0.998605,0.656439,0); rgb(67pt)=(0.998416,0.652807,0); rgb(68pt)=(0.998217,0.649253,0); rgb(69pt)=(0.998006,0.645786,0); rgb(70pt)=(0.997785,0.642416,0); rgb(71pt)=(0.997552,0.639152,0); rgb(72pt)=(0.997308,0.636004,0); rgb(73pt)=(0.997053,0.632982,0); rgb(74pt)=(0.996788,0.630095,0); rgb(75pt)=(0.996512,0.627352,0); rgb(76pt)=(0.996226,0.624763,0); rgb(77pt)=(0.995851,0.622329,0); rgb(78pt)=(0.99494,0.619997,0); rgb(79pt)=(0.99345,0.617753,0); rgb(80pt)=(0.991419,0.61559,0); rgb(81pt)=(0.988885,0.613503,0); rgb(82pt)=(0.985886,0.611486,0); rgb(83pt)=(0.98246,0.609532,0); rgb(84pt)=(0.978643,0.607636,0); rgb(85pt)=(0.974475,0.605791,0); rgb(86pt)=(0.969992,0.603992,0); rgb(87pt)=(0.965232,0.602233,0); rgb(88pt)=(0.960233,0.600507,0); rgb(89pt)=(0.955033,0.598808,0); rgb(90pt)=(0.949669,0.59713,0); rgb(91pt)=(0.94418,0.595468,0); rgb(92pt)=(0.938602,0.593815,0); rgb(93pt)=(0.932974,0.592166,0); rgb(94pt)=(0.927333,0.590513,0); rgb(95pt)=(0.921717,0.588852,0); rgb(96pt)=(0.916164,0.587176,0); rgb(97pt)=(0.910711,0.585479,0); rgb(98pt)=(0.905397,0.583755,0); rgb(99pt)=(0.900258,0.581999,0); rgb(100pt)=(0.895333,0.580203,0); rgb(101pt)=(0.890659,0.578362,0); rgb(102pt)=(0.886275,0.576471,0); rgb(103pt)=(0.882047,0.574545,0); rgb(104pt)=(0.877819,0.572608,0); rgb(105pt)=(0.873592,0.57066,0); rgb(106pt)=(0.869366,0.568701,0); rgb(107pt)=(0.865143,0.566733,0); rgb(108pt)=(0.860924,0.564756,0); rgb(109pt)=(0.856708,0.562771,0); rgb(110pt)=(0.852497,0.560778,0); rgb(111pt)=(0.848292,0.558779,0); rgb(112pt)=(0.844092,0.556774,0); rgb(113pt)=(0.8399,0.554763,0); rgb(114pt)=(0.835716,0.552749,0); rgb(115pt)=(0.831541,0.55073,0); rgb(116pt)=(0.827374,0.548709,0); rgb(117pt)=(0.823219,0.546686,0); rgb(118pt)=(0.819074,0.54466,0); rgb(119pt)=(0.81494,0.542635,0); rgb(120pt)=(0.81082,0.540609,0); rgb(121pt)=(0.806712,0.538584,0); rgb(122pt)=(0.802619,0.53656,0); rgb(123pt)=(0.798541,0.534539,0); rgb(124pt)=(0.794478,0.532521,0); rgb(125pt)=(0.790431,0.530506,0); rgb(126pt)=(0.786402,0.528496,0); rgb(127pt)=(0.782391,0.526491,0); rgb(128pt)=(0.77841,0.524489,0); rgb(129pt)=(0.774523,0.522478,0); rgb(130pt)=(0.770731,0.520455,0); rgb(131pt)=(0.767022,0.518424,0); rgb(132pt)=(0.763384,0.516385,0); rgb(133pt)=(0.759804,0.514339,0); rgb(134pt)=(0.756272,0.51229,0); rgb(135pt)=(0.752775,0.510237,0); rgb(136pt)=(0.749302,0.508182,0); rgb(137pt)=(0.74584,0.506128,0); rgb(138pt)=(0.742378,0.504075,0); rgb(139pt)=(0.738904,0.502025,0); rgb(140pt)=(0.735406,0.499979,0); rgb(141pt)=(0.731872,0.49794,0); rgb(142pt)=(0.72829,0.495909,0); rgb(143pt)=(0.724649,0.493887,0); rgb(144pt)=(0.720936,0.491875,0); rgb(145pt)=(0.71714,0.489876,0); rgb(146pt)=(0.713249,0.487891,0); rgb(147pt)=(0.709251,0.485921,0); rgb(148pt)=(0.705134,0.483968,0); rgb(149pt)=(0.700887,0.482033,0); rgb(150pt)=(0.696497,0.480118,0); rgb(151pt)=(0.691952,0.478225,0); rgb(152pt)=(0.687242,0.476355,0); rgb(153pt)=(0.682353,0.47451,0); rgb(154pt)=(0.677195,0.472696,0); rgb(155pt)=(0.6717,0.470916,0); rgb(156pt)=(0.665891,0.469169,0); rgb(157pt)=(0.659791,0.46745,0); rgb(158pt)=(0.653423,0.465756,0); rgb(159pt)=(0.64681,0.464084,0); rgb(160pt)=(0.639976,0.462432,0); rgb(161pt)=(0.632943,0.460795,0); rgb(162pt)=(0.625734,0.459171,0); rgb(163pt)=(0.618373,0.457556,0); rgb(164pt)=(0.610882,0.455948,0); rgb(165pt)=(0.603284,0.454343,0); rgb(166pt)=(0.595604,0.452737,0); rgb(167pt)=(0.587863,0.451129,0); rgb(168pt)=(0.580084,0.449514,0); rgb(169pt)=(0.572292,0.447889,0); rgb(170pt)=(0.564508,0.446252,0); rgb(171pt)=(0.556756,0.444599,0); rgb(172pt)=(0.549059,0.442927,0); rgb(173pt)=(0.54144,0.441232,0); rgb(174pt)=(0.533922,0.439512,0); rgb(175pt)=(0.526529,0.437764,0); rgb(176pt)=(0.519282,0.435983,0); rgb(177pt)=(0.512206,0.434168,0); rgb(178pt)=(0.505323,0.432315,0); rgb(179pt)=(0.498628,0.430422,3.92506e-06); rgb(180pt)=(0.491973,0.428504,3.49981e-05); rgb(181pt)=(0.485331,0.426562,9.63073e-05); rgb(182pt)=(0.478704,0.424596,0.000186979); rgb(183pt)=(0.472096,0.422609,0.000306141); rgb(184pt)=(0.465508,0.420599,0.00045292); rgb(185pt)=(0.458942,0.418567,0.000626441); rgb(186pt)=(0.452401,0.416515,0.000825833); rgb(187pt)=(0.445885,0.414441,0.00105022); rgb(188pt)=(0.439399,0.412348,0.00129873); rgb(189pt)=(0.432942,0.410234,0.00157049); rgb(190pt)=(0.426518,0.408102,0.00186463); rgb(191pt)=(0.420129,0.40595,0.00218028); rgb(192pt)=(0.413777,0.40378,0.00251655); rgb(193pt)=(0.407464,0.401592,0.00287258); rgb(194pt)=(0.401191,0.399386,0.00324749); rgb(195pt)=(0.394962,0.397164,0.00364042); rgb(196pt)=(0.388777,0.394925,0.00405048); rgb(197pt)=(0.38264,0.39267,0.00447681); rgb(198pt)=(0.376552,0.390399,0.00491852); rgb(199pt)=(0.370516,0.388113,0.00537476); rgb(200pt)=(0.364532,0.385812,0.00584464); rgb(201pt)=(0.358605,0.383497,0.00632729); rgb(202pt)=(0.352735,0.381168,0.00682184); rgb(203pt)=(0.346925,0.378826,0.00732741); rgb(204pt)=(0.341176,0.376471,0.00784314); rgb(205pt)=(0.335485,0.374093,0.00847245); rgb(206pt)=(0.329843,0.371682,0.00930909); rgb(207pt)=(0.324249,0.369242,0.0103377); rgb(208pt)=(0.318701,0.366772,0.0115428); rgb(209pt)=(0.313198,0.364275,0.0129091); rgb(210pt)=(0.307739,0.361753,0.0144211); rgb(211pt)=(0.302322,0.359206,0.0160634); rgb(212pt)=(0.296945,0.356637,0.0178207); rgb(213pt)=(0.291607,0.354048,0.0196776); rgb(214pt)=(0.286307,0.35144,0.0216186); rgb(215pt)=(0.281043,0.348814,0.0236284); rgb(216pt)=(0.275813,0.346172,0.0256916); rgb(217pt)=(0.270616,0.343517,0.0277927); rgb(218pt)=(0.265451,0.340849,0.0299163); rgb(219pt)=(0.260317,0.33817,0.0320472); rgb(220pt)=(0.25521,0.335482,0.0341698); rgb(221pt)=(0.250131,0.332786,0.0362688); rgb(222pt)=(0.245078,0.330085,0.0383287); rgb(223pt)=(0.240048,0.327379,0.0403343); rgb(224pt)=(0.235042,0.324671,0.04227); rgb(225pt)=(0.230056,0.321962,0.0441205); rgb(226pt)=(0.22509,0.319254,0.0458704); rgb(227pt)=(0.220142,0.316548,0.0475043); rgb(228pt)=(0.215212,0.313846,0.0490067); rgb(229pt)=(0.210296,0.311149,0.0503624); rgb(230pt)=(0.205395,0.308459,0.0515759); rgb(231pt)=(0.200514,0.305763,0.052757); rgb(232pt)=(0.195655,0.303061,0.0539242); rgb(233pt)=(0.190817,0.300353,0.0550763); rgb(234pt)=(0.186001,0.297639,0.0562123); rgb(235pt)=(0.181207,0.294918,0.0573313); rgb(236pt)=(0.176434,0.292191,0.0584321); rgb(237pt)=(0.171685,0.289458,0.0595136); rgb(238pt)=(0.166957,0.286719,0.060575); rgb(239pt)=(0.162252,0.283973,0.0616151); rgb(240pt)=(0.15757,0.281221,0.0626328); rgb(241pt)=(0.152911,0.278463,0.0636271); rgb(242pt)=(0.148275,0.275699,0.0645971); rgb(243pt)=(0.143663,0.272929,0.0655416); rgb(244pt)=(0.139074,0.270152,0.0664596); rgb(245pt)=(0.134508,0.26737,0.06735); rgb(246pt)=(0.129967,0.264581,0.0682118); rgb(247pt)=(0.125449,0.261787,0.0690441); rgb(248pt)=(0.120956,0.258986,0.0698456); rgb(249pt)=(0.116487,0.25618,0.0706154); rgb(250pt)=(0.112043,0.253367,0.0713525); rgb(251pt)=(0.107623,0.250549,0.0720557); rgb(252pt)=(0.103229,0.247724,0.0727241); rgb(253pt)=(0.0988592,0.244894,0.0733566); rgb(254pt)=(0.0945149,0.242058,0.0739522); rgb(255pt)=(0.0901961,0.239216,0.0745098)}, mesh/rows=49]
table[row sep=crcr, point meta=\thisrow{c}] {%
%
x	y	z	c\\
56	0.093	-10.9049629825366	-10.9049629825366\\
56	0.09666	-11.4423383223269	-11.4423383223269\\
56	0.10032	-12.1019453259281	-12.1019453259281\\
56	0.10398	-12.8837839933399	-12.8837839933399\\
56	0.10764	-13.7878543245625	-13.7878543245625\\
56	0.1113	-14.8141563195959	-14.8141563195959\\
56	0.11496	-15.9626899784401	-15.9626899784401\\
56	0.11862	-17.233455301095	-17.233455301095\\
56	0.12228	-18.6264522875606	-18.6264522875606\\
56	0.12594	-20.141680937837	-20.141680937837\\
56	0.1296	-21.7791412519242	-21.7791412519242\\
56	0.13326	-23.5388332298221	-23.5388332298221\\
56	0.13692	-25.4207568715308	-25.4207568715308\\
56	0.14058	-27.4249121770503	-27.4249121770503\\
56	0.14424	-29.5512991463805	-29.5512991463805\\
56	0.1479	-31.7999177795214	-31.7999177795214\\
56	0.15156	-34.1707680764732	-34.1707680764732\\
56	0.15522	-36.6638500372356	-36.6638500372356\\
56	0.15888	-39.2791636618089	-39.2791636618089\\
56	0.16254	-42.0167089501929	-42.0167089501929\\
56	0.1662	-44.8764859023877	-44.8764859023877\\
56	0.16986	-47.8584945183932	-47.8584945183932\\
56	0.17352	-50.9627347982094	-50.9627347982094\\
56	0.17718	-54.1892067418365	-54.1892067418365\\
56	0.18084	-57.5379103492742	-57.5379103492742\\
56	0.1845	-61.0088456205228	-61.0088456205228\\
56	0.18816	-64.6020125555821	-64.6020125555821\\
56	0.19182	-68.3174111544522	-68.3174111544522\\
56	0.19548	-72.155041417133	-72.155041417133\\
56	0.19914	-76.1149033436246	-76.1149033436246\\
56	0.2028	-80.196996933927	-80.196996933927\\
56	0.20646	-84.4013221880401	-84.4013221880401\\
56	0.21012	-88.7278791059639	-88.7278791059639\\
56	0.21378	-93.1766676876985	-93.1766676876985\\
56	0.21744	-97.7476879332439	-97.7476879332439\\
56	0.2211	-102.4409398426	-102.4409398426\\
56	0.22476	-107.256423415767	-107.256423415767\\
56	0.22842	-112.194138652745	-112.194138652745\\
56	0.23208	-117.254085553533	-117.254085553533\\
56	0.23574	-122.436264118132	-122.436264118132\\
56	0.2394	-127.740674346542	-127.740674346542\\
56	0.24306	-133.167316238763	-133.167316238763\\
56	0.24672	-138.716189794794	-138.716189794794\\
56	0.25038	-144.387295014636	-144.387295014636\\
56	0.25404	-150.180631898289	-150.180631898289\\
56	0.2577	-156.096200445753	-156.096200445753\\
56	0.26136	-162.134000657028	-162.134000657028\\
56	0.26502	-168.294032532113	-168.294032532113\\
56	0.26868	-174.576296071009	-174.576296071009\\
56	0.27234	-180.980791273716	-180.980791273716\\
56	0.276	-187.507518140233	-187.507518140233\\
56.375	0.093	-10.8126299877808	-10.8126299877808\\
56.375	0.09666	-11.3527841630815	-11.3527841630815\\
56.375	0.10032	-12.015170002193	-12.015170002193\\
56.375	0.10398	-12.7997875051152	-12.7997875051152\\
56.375	0.10764	-13.7066366718482	-13.7066366718482\\
56.375	0.1113	-14.7357175023919	-14.7357175023919\\
56.375	0.11496	-15.8870299967463	-15.8870299967463\\
56.375	0.11862	-17.1605741549116	-17.1605741549116\\
56.375	0.12228	-18.5563499768876	-18.5563499768876\\
56.375	0.12594	-20.0743574626744	-20.0743574626744\\
56.375	0.1296	-21.7145966122719	-21.7145966122719\\
56.375	0.13326	-23.4770674256802	-23.4770674256802\\
56.375	0.13692	-25.3617699028992	-25.3617699028992\\
56.375	0.14058	-27.368704043929	-27.368704043929\\
56.375	0.14424	-29.4978698487696	-29.4978698487696\\
56.375	0.1479	-31.7492673174209	-31.7492673174209\\
56.375	0.15156	-34.1228964498829	-34.1228964498829\\
56.375	0.15522	-36.6187572461558	-36.6187572461558\\
56.375	0.15888	-39.2368497062394	-39.2368497062394\\
56.375	0.16254	-41.9771738301337	-41.9771738301337\\
56.375	0.1662	-44.8397296178388	-44.8397296178388\\
56.375	0.16986	-47.8245170693547	-47.8245170693547\\
56.375	0.17352	-50.9315361846813	-50.9315361846813\\
56.375	0.17718	-54.1607869638187	-54.1607869638187\\
56.375	0.18084	-57.5122694067668	-57.5122694067668\\
56.375	0.1845	-60.9859835135257	-60.9859835135257\\
56.375	0.18816	-64.5819292840954	-64.5819292840954\\
56.375	0.19182	-68.3001067184758	-68.3001067184758\\
56.375	0.19548	-72.140515816667	-72.140515816667\\
56.375	0.19914	-76.1031565786689	-76.1031565786689\\
56.375	0.2028	-80.1880290044816	-80.1880290044816\\
56.375	0.20646	-84.3951330941051	-84.3951330941051\\
56.375	0.21012	-88.7244688475393	-88.7244688475393\\
56.375	0.21378	-93.1760362647842	-93.1760362647842\\
56.375	0.21744	-97.74983534584	-97.74983534584\\
56.375	0.2211	-102.445866090706	-102.445866090706\\
56.375	0.22476	-107.264128499384	-107.264128499384\\
56.375	0.22842	-112.204622571872	-112.204622571872\\
56.375	0.23208	-117.26734830817	-117.26734830817\\
56.375	0.23574	-122.45230570828	-122.45230570828\\
56.375	0.2394	-127.7594947722	-127.7594947722\\
56.375	0.24306	-133.188915499931	-133.188915499931\\
56.375	0.24672	-138.740567891473	-138.740567891473\\
56.375	0.25038	-144.414451946826	-144.414451946826\\
56.375	0.25404	-150.210567665989	-150.210567665989\\
56.375	0.2577	-156.128915048963	-156.128915048963\\
56.375	0.26136	-162.169494095748	-162.169494095748\\
56.375	0.26502	-168.332304806343	-168.332304806343\\
56.375	0.26868	-174.61734718075	-174.61734718075\\
56.375	0.27234	-181.024621218967	-181.024621218967\\
56.375	0.276	-187.554126920995	-187.554126920995\\
56.75	0.093	-10.7298933655794	-10.7298933655794\\
56.75	0.09666	-11.2728263763905	-11.2728263763905\\
56.75	0.10032	-11.9379910510123	-11.9379910510123\\
56.75	0.10398	-12.7253873894448	-12.7253873894448\\
56.75	0.10764	-13.6350153916882	-13.6350153916882\\
56.75	0.1113	-14.6668750577423	-14.6668750577423\\
56.75	0.11496	-15.8209663876071	-15.8209663876071\\
56.75	0.11862	-17.0972893812827	-17.0972893812827\\
56.75	0.12228	-18.495844038769	-18.495844038769\\
56.75	0.12594	-20.0166303600661	-20.0166303600661\\
56.75	0.1296	-21.659648345174	-21.659648345174\\
56.75	0.13326	-23.4248979940926	-23.4248979940926\\
56.75	0.13692	-25.312379306822	-25.312379306822\\
56.75	0.14058	-27.3220922833622	-27.3220922833622\\
56.75	0.14424	-29.454036923713	-29.454036923713\\
56.75	0.1479	-31.7082132278747	-31.7082132278747\\
56.75	0.15156	-34.0846211958471	-34.0846211958471\\
56.75	0.15522	-36.5832608276303	-36.5832608276303\\
56.75	0.15888	-39.2041321232243	-39.2041321232243\\
56.75	0.16254	-41.947235082629	-41.947235082629\\
56.75	0.1662	-44.8125697058444	-44.8125697058444\\
56.75	0.16986	-47.8001359928707	-47.8001359928707\\
56.75	0.17352	-50.9099339437076	-50.9099339437076\\
56.75	0.17718	-54.1419635583553	-54.1419635583553\\
56.75	0.18084	-57.4962248368138	-57.4962248368138\\
56.75	0.1845	-60.9727177790831	-60.9727177790831\\
56.75	0.18816	-64.571442385163	-64.571442385163\\
56.75	0.19182	-68.2923986550538	-68.2923986550538\\
56.75	0.19548	-72.1355865887553	-72.1355865887553\\
56.75	0.19914	-76.1010061862676	-76.1010061862676\\
56.75	0.2028	-80.1886574475907	-80.1886574475907\\
56.75	0.20646	-84.3985403727245	-84.3985403727245\\
56.75	0.21012	-88.730654961669	-88.730654961669\\
56.75	0.21378	-93.1850012144243	-93.1850012144243\\
56.75	0.21744	-97.7615791309904	-97.7615791309904\\
56.75	0.2211	-102.460388711367	-102.460388711367\\
56.75	0.22476	-107.281429955555	-107.281429955555\\
56.75	0.22842	-112.224702863553	-112.224702863553\\
56.75	0.23208	-117.290207435362	-117.290207435362\\
56.75	0.23574	-122.477943670982	-122.477943670982\\
56.75	0.2394	-127.787911570413	-127.787911570413\\
56.75	0.24306	-133.220111133654	-133.220111133654\\
56.75	0.24672	-138.774542360706	-138.774542360706\\
56.75	0.25038	-144.451205251569	-144.451205251569\\
56.75	0.25404	-150.250099806243	-150.250099806243\\
56.75	0.2577	-156.171226024727	-156.171226024727\\
56.75	0.26136	-162.214583907023	-162.214583907023\\
56.75	0.26502	-168.380173453128	-168.380173453128\\
56.75	0.26868	-174.667994663045	-174.667994663045\\
56.75	0.27234	-181.078047536773	-181.078047536773\\
56.75	0.276	-187.610332074311	-187.610332074311\\
57.125	0.093	-10.6567531159325	-10.6567531159325\\
57.125	0.09666	-11.2024649622539	-11.2024649622539\\
57.125	0.10032	-11.870408472386	-11.870408472386\\
57.125	0.10398	-12.6605836463289	-12.6605836463289\\
57.125	0.10764	-13.5729904840826	-13.5729904840826\\
57.125	0.1113	-14.607628985647	-14.607628985647\\
57.125	0.11496	-15.7644991510222	-15.7644991510222\\
57.125	0.11862	-17.0436009802082	-17.0436009802082\\
57.125	0.12228	-18.4449344732048	-18.4449344732048\\
57.125	0.12594	-19.9684996300123	-19.9684996300123\\
57.125	0.1296	-21.6142964506305	-21.6142964506305\\
57.125	0.13326	-23.3823249350595	-23.3823249350595\\
57.125	0.13692	-25.2725850832992	-25.2725850832992\\
57.125	0.14058	-27.2850768953497	-27.2850768953497\\
57.125	0.14424	-29.419800371211	-29.419800371211\\
57.125	0.1479	-31.676755510883	-31.676755510883\\
57.125	0.15156	-34.0559423143658	-34.0559423143658\\
57.125	0.15522	-36.5573607816593	-36.5573607816593\\
57.125	0.15888	-39.1810109127636	-39.1810109127636\\
57.125	0.16254	-41.9268927076786	-41.9268927076786\\
57.125	0.1662	-44.7950061664044	-44.7950061664044\\
57.125	0.16986	-47.785351288941	-47.785351288941\\
57.125	0.17352	-50.8979280752883	-50.8979280752883\\
57.125	0.17718	-54.1327365254464	-54.1327365254464\\
57.125	0.18084	-57.4897766394152	-57.4897766394152\\
57.125	0.1845	-60.9690484171948	-60.9690484171948\\
57.125	0.18816	-64.5705518587851	-64.5705518587851\\
57.125	0.19182	-68.2942869641863	-68.2942869641863\\
57.125	0.19548	-72.1402537333981	-72.1402537333981\\
57.125	0.19914	-76.1084521664207	-76.1084521664207\\
57.125	0.2028	-80.1988822632542	-80.1988822632542\\
57.125	0.20646	-84.4115440238983	-84.4115440238983\\
57.125	0.21012	-88.7464374483532	-88.7464374483532\\
57.125	0.21378	-93.2035625366189	-93.2035625366189\\
57.125	0.21744	-97.7829192886953	-97.7829192886953\\
57.125	0.2211	-102.484507704582	-102.484507704582\\
57.125	0.22476	-107.30832778428	-107.30832778428\\
57.125	0.22842	-112.254379527789	-112.254379527789\\
57.125	0.23208	-117.322662935109	-117.322662935109\\
57.125	0.23574	-122.513178006239	-122.513178006239\\
57.125	0.2394	-127.82592474118	-127.82592474118\\
57.125	0.24306	-133.260903139931	-133.260903139931\\
57.125	0.24672	-138.818113202494	-138.818113202494\\
57.125	0.25038	-144.497554928867	-144.497554928867\\
57.125	0.25404	-150.299228319051	-150.299228319051\\
57.125	0.2577	-156.223133373046	-156.223133373046\\
57.125	0.26136	-162.269270090851	-162.269270090851\\
57.125	0.26502	-168.437638472468	-168.437638472468\\
57.125	0.26868	-174.728238517895	-174.728238517895\\
57.125	0.27234	-181.141070227133	-181.141070227133\\
57.125	0.276	-187.676133600181	-187.676133600181\\
57.5	0.093	-10.5932092388399	-10.5932092388399\\
57.5	0.09666	-11.1416999206717	-11.1416999206717\\
57.5	0.10032	-11.8124222663142	-11.8124222663142\\
57.5	0.10398	-12.6053762757674	-12.6053762757674\\
57.5	0.10764	-13.5205619490314	-13.5205619490314\\
57.5	0.1113	-14.5579792861062	-14.5579792861062\\
57.5	0.11496	-15.7176282869917	-15.7176282869917\\
57.5	0.11862	-16.999508951688	-16.999508951688\\
57.5	0.12228	-18.403621280195	-18.403621280195\\
57.5	0.12594	-19.9299652725128	-19.9299652725128\\
57.5	0.1296	-21.5785409286414	-21.5785409286414\\
57.5	0.13326	-23.3493482485807	-23.3493482485807\\
57.5	0.13692	-25.2423872323308	-25.2423872323308\\
57.5	0.14058	-27.2576578798917	-27.2576578798917\\
57.5	0.14424	-29.3951601912633	-29.3951601912633\\
57.5	0.1479	-31.6548941664456	-31.6548941664456\\
57.5	0.15156	-34.0368598054387	-34.0368598054387\\
57.5	0.15522	-36.5410571082426	-36.5410571082426\\
57.5	0.15888	-39.1674860748573	-39.1674860748573\\
57.5	0.16254	-41.9161467052826	-41.9161467052826\\
57.5	0.1662	-44.7870389995188	-44.7870389995188\\
57.5	0.16986	-47.7801629575657	-47.7801629575657\\
57.5	0.17352	-50.8955185794233	-50.8955185794233\\
57.5	0.17718	-54.1331058650918	-54.1331058650918\\
57.5	0.18084	-57.492924814571	-57.492924814571\\
57.5	0.1845	-60.9749754278609	-60.9749754278609\\
57.5	0.18816	-64.5792577049616	-64.5792577049616\\
57.5	0.19182	-68.3057716458731	-68.3057716458731\\
57.5	0.19548	-72.1545172505953	-72.1545172505953\\
57.5	0.19914	-76.1254945191282	-76.1254945191282\\
57.5	0.2028	-80.218703451472	-80.218703451472\\
57.5	0.20646	-84.4341440476265	-84.4341440476265\\
57.5	0.21012	-88.7718163075918	-88.7718163075918\\
57.5	0.21378	-93.2317202313677	-93.2317202313677\\
57.5	0.21744	-97.8138558189545	-97.8138558189545\\
57.5	0.2211	-102.518223070352	-102.518223070352\\
57.5	0.22476	-107.34482198556	-107.34482198556\\
57.5	0.22842	-112.293652564579	-112.293652564579\\
57.5	0.23208	-117.364714807409	-117.364714807409\\
57.5	0.23574	-122.55800871405	-122.55800871405\\
57.5	0.2394	-127.873534284501	-127.873534284501\\
57.5	0.24306	-133.311291518763	-133.311291518763\\
57.5	0.24672	-138.871280416836	-138.871280416836\\
57.5	0.25038	-144.55350097872	-144.55350097872\\
57.5	0.25404	-150.357953204414	-150.357953204414\\
57.5	0.2577	-156.284637093919	-156.284637093919\\
57.5	0.26136	-162.333552647235	-162.333552647235\\
57.5	0.26502	-168.504699864362	-168.504699864362\\
57.5	0.26868	-174.798078745299	-174.798078745299\\
57.5	0.27234	-181.213689290047	-181.213689290047\\
57.5	0.276	-187.751531498606	-187.751531498606\\
57.875	0.093	-10.5392617343017	-10.5392617343017\\
57.875	0.09666	-11.0905312516438	-11.0905312516438\\
57.875	0.10032	-11.7640324327967	-11.7640324327967\\
57.875	0.10398	-12.5597652777603	-12.5597652777603\\
57.875	0.10764	-13.4777297865346	-13.4777297865346\\
57.875	0.1113	-14.5179259591197	-14.5179259591197\\
57.875	0.11496	-15.6803537955156	-15.6803537955156\\
57.875	0.11862	-16.9650132957222	-16.9650132957222\\
57.875	0.12228	-18.3719044597396	-18.3719044597396\\
57.875	0.12594	-19.9010272875678	-19.9010272875678\\
57.875	0.1296	-21.5523817792067	-21.5523817792067\\
57.875	0.13326	-23.3259679346564	-23.3259679346564\\
57.875	0.13692	-25.2217857539168	-25.2217857539168\\
57.875	0.14058	-27.239835236988	-27.239835236988\\
57.875	0.14424	-29.3801163838699	-29.3801163838699\\
57.875	0.1479	-31.6426291945626	-31.6426291945626\\
57.875	0.15156	-34.0273736690661	-34.0273736690661\\
57.875	0.15522	-36.5343498073803	-36.5343498073803\\
57.875	0.15888	-39.1635576095053	-39.1635576095053\\
57.875	0.16254	-41.9149970754411	-41.9149970754411\\
57.875	0.1662	-44.7886682051876	-44.7886682051876\\
57.875	0.16986	-47.7845709987448	-47.7845709987448\\
57.875	0.17352	-50.9027054561128	-50.9027054561128\\
57.875	0.17718	-54.1430715772916	-54.1430715772916\\
57.875	0.18084	-57.5056693622811	-57.5056693622811\\
57.875	0.1845	-60.9904988110814	-60.9904988110814\\
57.875	0.18816	-64.5975599236925	-64.5975599236925\\
57.875	0.19182	-68.3268527001143	-68.3268527001143\\
57.875	0.19548	-72.1783771403468	-72.1783771403468\\
57.875	0.19914	-76.1521332443902	-76.1521332443902\\
57.875	0.2028	-80.2481210122443	-80.2481210122443\\
57.875	0.20646	-84.4663404439091	-84.4663404439091\\
57.875	0.21012	-88.8067915393847	-88.8067915393847\\
57.875	0.21378	-93.2694742986711	-93.2694742986711\\
57.875	0.21744	-97.8543887217682	-97.8543887217682\\
57.875	0.2211	-102.561534808676	-102.561534808676\\
57.875	0.22476	-107.390912559395	-107.390912559395\\
57.875	0.22842	-112.342521973924	-112.342521973924\\
57.875	0.23208	-117.416363052264	-117.416363052264\\
57.875	0.23574	-122.612435794415	-122.612435794415\\
57.875	0.2394	-127.930740200377	-127.930740200377\\
57.875	0.24306	-133.371276270149	-133.371276270149\\
57.875	0.24672	-138.934044003732	-138.934044003732\\
57.875	0.25038	-144.619043401126	-144.619043401126\\
57.875	0.25404	-150.426274462331	-150.426274462331\\
57.875	0.2577	-156.355737187347	-156.355737187347\\
57.875	0.26136	-162.407431576173	-162.407431576173\\
57.875	0.26502	-168.58135762881	-168.58135762881\\
57.875	0.26868	-174.877515345258	-174.877515345258\\
57.875	0.27234	-181.295904725516	-181.295904725516\\
57.875	0.276	-187.836525769585	-187.836525769585\\
58.25	0.093	-10.4949106023179	-10.4949106023179\\
58.25	0.09666	-11.0489589551704	-11.0489589551704\\
58.25	0.10032	-11.7252389718336	-11.7252389718336\\
58.25	0.10398	-12.5237506523075	-12.5237506523075\\
58.25	0.10764	-13.4444939965922	-13.4444939965922\\
58.25	0.1113	-14.4874690046877	-14.4874690046877\\
58.25	0.11496	-15.652675676594	-15.652675676594\\
58.25	0.11862	-16.9401140123109	-16.9401140123109\\
58.25	0.12228	-18.3497840118387	-18.3497840118387\\
58.25	0.12594	-19.8816856751772	-19.8816856751772\\
58.25	0.1296	-21.5358190023264	-21.5358190023264\\
58.25	0.13326	-23.3121839932864	-23.3121839932864\\
58.25	0.13692	-25.2107806480572	-25.2107806480572\\
58.25	0.14058	-27.2316089666387	-27.2316089666387\\
58.25	0.14424	-29.3746689490311	-29.3746689490311\\
58.25	0.1479	-31.6399605952341	-31.6399605952341\\
58.25	0.15156	-34.0274839052479	-34.0274839052479\\
58.25	0.15522	-36.5372388790725	-36.5372388790725\\
58.25	0.15888	-39.1692255167078	-39.1692255167078\\
58.25	0.16254	-41.9234438181539	-41.9234438181539\\
58.25	0.1662	-44.7998937834108	-44.7998937834108\\
58.25	0.16986	-47.7985754124784	-47.7985754124784\\
58.25	0.17352	-50.9194887053567	-50.9194887053567\\
58.25	0.17718	-54.1626336620458	-54.1626336620458\\
58.25	0.18084	-57.5280102825457	-57.5280102825457\\
58.25	0.1845	-61.0156185668564	-61.0156185668564\\
58.25	0.18816	-64.6254585149777	-64.6254585149777\\
58.25	0.19182	-68.3575301269099	-68.3575301269099\\
58.25	0.19548	-72.2118334026528	-72.2118334026528\\
58.25	0.19914	-76.1883683422065	-76.1883683422065\\
58.25	0.2028	-80.287134945571	-80.287134945571\\
58.25	0.20646	-84.5081332127461	-84.5081332127461\\
58.25	0.21012	-88.8513631437321	-88.8513631437321\\
58.25	0.21378	-93.3168247385288	-93.3168247385288\\
58.25	0.21744	-97.9045179971362	-97.9045179971362\\
58.25	0.2211	-102.614442919554	-102.614442919554\\
58.25	0.22476	-107.446599505783	-107.446599505783\\
58.25	0.22842	-112.400987755823	-112.400987755823\\
58.25	0.23208	-117.477607669674	-117.477607669674\\
58.25	0.23574	-122.676459247335	-122.676459247335\\
58.25	0.2394	-127.997542488807	-127.997542488807\\
58.25	0.24306	-133.44085739409	-133.44085739409\\
58.25	0.24672	-139.006403963183	-139.006403963183\\
58.25	0.25038	-144.694182196088	-144.694182196088\\
58.25	0.25404	-150.504192092803	-150.504192092803\\
58.25	0.2577	-156.436433653328	-156.436433653328\\
58.25	0.26136	-162.490906877665	-162.490906877665\\
58.25	0.26502	-168.667611765812	-168.667611765812\\
58.25	0.26868	-174.96654831777	-174.96654831777\\
58.25	0.27234	-181.387716533539	-181.387716533539\\
58.25	0.276	-187.931116413119	-187.931116413119\\
58.625	0.093	-10.4601558428886	-10.4601558428886\\
58.625	0.09666	-11.0169830312514	-11.0169830312514\\
58.625	0.10032	-11.6960418834249	-11.6960418834249\\
58.625	0.10398	-12.4973323994092	-12.4973323994092\\
58.625	0.10764	-13.4208545792042	-13.4208545792042\\
58.625	0.1113	-14.46660842281	-14.46660842281\\
58.625	0.11496	-15.6345939302266	-15.6345939302266\\
58.625	0.11862	-16.924811101454	-16.924811101454\\
58.625	0.12228	-18.3372599364921	-18.3372599364921\\
58.625	0.12594	-19.8719404353409	-19.8719404353409\\
58.625	0.1296	-21.5288525980005	-21.5288525980005\\
58.625	0.13326	-23.3079964244709	-23.3079964244709\\
58.625	0.13692	-25.209371914752	-25.209371914752\\
58.625	0.14058	-27.2329790688439	-27.2329790688439\\
58.625	0.14424	-29.3788178867465	-29.3788178867465\\
58.625	0.1479	-31.6468883684599	-31.6468883684599\\
58.625	0.15156	-34.0371905139841	-34.0371905139841\\
58.625	0.15522	-36.549724323319	-36.549724323319\\
58.625	0.15888	-39.1844897964647	-39.1844897964647\\
58.625	0.16254	-41.9414869334211	-41.9414869334211\\
58.625	0.1662	-44.8207157341883	-44.8207157341883\\
58.625	0.16986	-47.8221761987663	-47.8221761987663\\
58.625	0.17352	-50.945868327155	-50.945868327155\\
58.625	0.17718	-54.1917921193544	-54.1917921193544\\
58.625	0.18084	-57.5599475753647	-57.5599475753647\\
58.625	0.1845	-61.0503346951857	-61.0503346951857\\
58.625	0.18816	-64.6629534788174	-64.6629534788174\\
58.625	0.19182	-68.3978039262599	-68.3978039262599\\
58.625	0.19548	-72.2548860375132	-72.2548860375132\\
58.625	0.19914	-76.2341998125772	-76.2341998125772\\
58.625	0.2028	-80.335745251452	-80.335745251452\\
58.625	0.20646	-84.5595223541375	-84.5595223541375\\
58.625	0.21012	-88.9055311206339	-88.9055311206339\\
58.625	0.21378	-93.3737715509408	-93.3737715509408\\
58.625	0.21744	-97.9642436450587	-97.9642436450587\\
58.625	0.2211	-102.676947402987	-102.676947402987\\
58.625	0.22476	-107.511882824727	-107.511882824727\\
58.625	0.22842	-112.469049910277	-112.469049910277\\
58.625	0.23208	-117.548448659638	-117.548448659638\\
58.625	0.23574	-122.750079072809	-122.750079072809\\
58.625	0.2394	-128.073941149791	-128.073941149791\\
58.625	0.24306	-133.520034890585	-133.520034890585\\
58.625	0.24672	-139.088360295189	-139.088360295189\\
58.625	0.25038	-144.778917363603	-144.778917363603\\
58.625	0.25404	-150.591706095828	-150.591706095828\\
58.625	0.2577	-156.526726491865	-156.526726491865\\
58.625	0.26136	-162.583978551712	-162.583978551712\\
58.625	0.26502	-168.763462275369	-168.763462275369\\
58.625	0.26868	-175.065177662838	-175.065177662838\\
58.625	0.27234	-181.489124714117	-181.489124714117\\
58.625	0.276	-188.035303429207	-188.035303429207\\
59	0.093	-10.4349974560135	-10.4349974560135\\
59	0.09666	-10.9946034798867	-10.9946034798867\\
59	0.10032	-11.6764411675706	-11.6764411675706\\
59	0.10398	-12.4805105190652	-12.4805105190652\\
59	0.10764	-13.4068115343706	-13.4068115343706\\
59	0.1113	-14.4553442134868	-14.4553442134868\\
59	0.11496	-15.6261085564137	-15.6261085564137\\
59	0.11862	-16.9191045631514	-16.9191045631514\\
59	0.12228	-18.3343322336999	-18.3343322336999\\
59	0.12594	-19.8717915680591	-19.8717915680591\\
59	0.1296	-21.531482566229	-21.531482566229\\
59	0.13326	-23.3134052282097	-23.3134052282097\\
59	0.13692	-25.2175595540012	-25.2175595540012\\
59	0.14058	-27.2439455436034	-27.2439455436034\\
59	0.14424	-29.3925631970164	-29.3925631970164\\
59	0.1479	-31.6634125142402	-31.6634125142402\\
59	0.15156	-34.0564934952747	-34.0564934952747\\
59	0.15522	-36.5718061401199	-36.5718061401199\\
59	0.15888	-39.209350448776	-39.209350448776\\
59	0.16254	-41.9691264212427	-41.9691264212427\\
59	0.1662	-44.8511340575203	-44.8511340575203\\
59	0.16986	-47.8553733576086	-47.8553733576086\\
59	0.17352	-50.9818443215076	-50.9818443215076\\
59	0.17718	-54.2305469492175	-54.2305469492175\\
59	0.18084	-57.601481240738	-57.601481240738\\
59	0.1845	-61.0946471960694	-61.0946471960694\\
59	0.18816	-64.7100448152114	-64.7100448152114\\
59	0.19182	-68.4476740981643	-68.4476740981643\\
59	0.19548	-72.3075350449279	-72.3075350449279\\
59	0.19914	-76.2896276555023	-76.2896276555023\\
59	0.2028	-80.3939519298874	-80.3939519298874\\
59	0.20646	-84.6205078680833	-84.6205078680833\\
59	0.21012	-88.96929547009	-88.96929547009\\
59	0.21378	-93.4403147359074	-93.4403147359074\\
59	0.21744	-98.0335656655355	-98.0335656655355\\
59	0.2211	-102.749048258974	-102.749048258974\\
59	0.22476	-107.586762516224	-107.586762516224\\
59	0.22842	-112.546708437285	-112.546708437285\\
59	0.23208	-117.628886022156	-117.628886022156\\
59	0.23574	-122.833295270838	-122.833295270838\\
59	0.2394	-128.15993618333	-128.15993618333\\
59	0.24306	-133.608808759634	-133.608808759634\\
59	0.24672	-139.179912999748	-139.179912999748\\
59	0.25038	-144.873248903673	-144.873248903673\\
59	0.25404	-150.688816471409	-150.688816471409\\
59	0.2577	-156.626615702955	-156.626615702955\\
59	0.26136	-162.686646598313	-162.686646598313\\
59	0.26502	-168.868909157481	-168.868909157481\\
59	0.26868	-175.173403380459	-175.173403380459\\
59	0.27234	-181.600129267249	-181.600129267249\\
59	0.276	-188.149086817849	-188.149086817849\\
59.375	0.093	-10.419435441693	-10.419435441693\\
59.375	0.09666	-10.9818203010765	-10.9818203010765\\
59.375	0.10032	-11.6664368242707	-11.6664368242707\\
59.375	0.10398	-12.4732850112757	-12.4732850112757\\
59.375	0.10764	-13.4023648620915	-13.4023648620915\\
59.375	0.1113	-14.453676376718	-14.453676376718\\
59.375	0.11496	-15.6272195551553	-15.6272195551553\\
59.375	0.11862	-16.9229943974033	-16.9229943974033\\
59.375	0.12228	-18.341000903462	-18.341000903462\\
59.375	0.12594	-19.8812390733316	-19.8812390733316\\
59.375	0.1296	-21.5437089070119	-21.5437089070119\\
59.375	0.13326	-23.3284104045029	-23.3284104045029\\
59.375	0.13692	-25.2353435658048	-25.2353435658048\\
59.375	0.14058	-27.2645083909174	-27.2645083909174\\
59.375	0.14424	-29.4159048798407	-29.4159048798407\\
59.375	0.1479	-31.6895330325748	-31.6895330325748\\
59.375	0.15156	-34.0853928491197	-34.0853928491197\\
59.375	0.15522	-36.6034843294753	-36.6034843294753\\
59.375	0.15888	-39.2438074736417	-39.2438074736417\\
59.375	0.16254	-42.0063622816188	-42.0063622816188\\
59.375	0.1662	-44.8911487534067	-44.8911487534067\\
59.375	0.16986	-47.8981668890053	-47.8981668890053\\
59.375	0.17352	-51.0274166884147	-51.0274166884147\\
59.375	0.17718	-54.2788981516349	-54.2788981516349\\
59.375	0.18084	-57.6526112786658	-57.6526112786658\\
59.375	0.1845	-61.1485560695075	-61.1485560695075\\
59.375	0.18816	-64.7667325241599	-64.7667325241599\\
59.375	0.19182	-68.5071406426231	-68.5071406426231\\
59.375	0.19548	-72.3697804248971	-72.3697804248971\\
59.375	0.19914	-76.3546518709818	-76.3546518709818\\
59.375	0.2028	-80.4617549808773	-80.4617549808773\\
59.375	0.20646	-84.6910897545836	-84.6910897545836\\
59.375	0.21012	-89.0426561921006	-89.0426561921006\\
59.375	0.21378	-93.5164542934283	-93.5164542934283\\
59.375	0.21744	-98.1124840585668	-98.1124840585668\\
59.375	0.2211	-102.830745487516	-102.830745487516\\
59.375	0.22476	-107.671238580276	-107.671238580276\\
59.375	0.22842	-112.633963336847	-112.633963336847\\
59.375	0.23208	-117.718919757228	-117.718919757228\\
59.375	0.23574	-122.926107841421	-122.926107841421\\
59.375	0.2394	-128.255527589424	-128.255527589424\\
59.375	0.24306	-133.707179001238	-133.707179001238\\
59.375	0.24672	-139.281062076862	-139.281062076862\\
59.375	0.25038	-144.977176816298	-144.977176816298\\
59.375	0.25404	-150.795523219544	-150.795523219544\\
59.375	0.2577	-156.7361012866	-156.7361012866\\
59.375	0.26136	-162.798911017468	-162.798911017468\\
59.375	0.26502	-168.983952412146	-168.983952412146\\
59.375	0.26868	-175.291225470636	-175.291225470636\\
59.375	0.27234	-181.720730192935	-181.720730192935\\
59.375	0.276	-188.272466579046	-188.272466579046\\
59.75	0.093	-10.4134697999268	-10.4134697999268\\
59.75	0.09666	-10.9786334948206	-10.9786334948206\\
59.75	0.10032	-11.6660288535252	-11.6660288535252\\
59.75	0.10398	-12.4756558760405	-12.4756558760405\\
59.75	0.10764	-13.4075145623667	-13.4075145623667\\
59.75	0.1113	-14.4616049125035	-14.4616049125035\\
59.75	0.11496	-15.6379269264511	-15.6379269264511\\
59.75	0.11862	-16.9364806042095	-16.9364806042095\\
59.75	0.12228	-18.3572659457786	-18.3572659457786\\
59.75	0.12594	-19.9002829511585	-19.9002829511585\\
59.75	0.1296	-21.5655316203492	-21.5655316203492\\
59.75	0.13326	-23.3530119533506	-23.3530119533506\\
59.75	0.13692	-25.2627239501627	-25.2627239501627\\
59.75	0.14058	-27.2946676107857	-27.2946676107857\\
59.75	0.14424	-29.4488429352194	-29.4488429352194\\
59.75	0.1479	-31.7252499234638	-31.7252499234638\\
59.75	0.15156	-34.123888575519	-34.123888575519\\
59.75	0.15522	-36.644758891385	-36.644758891385\\
59.75	0.15888	-39.2878608710617	-39.2878608710617\\
59.75	0.16254	-42.0531945145492	-42.0531945145492\\
59.75	0.1662	-44.9407598218474	-44.9407598218474\\
59.75	0.16986	-47.9505567929564	-47.9505567929564\\
59.75	0.17352	-51.0825854278761	-51.0825854278761\\
59.75	0.17718	-54.3368457266067	-54.3368457266067\\
59.75	0.18084	-57.7133376891479	-57.7133376891479\\
59.75	0.1845	-61.2120613155	-61.2120613155\\
59.75	0.18816	-64.8330166056628	-64.8330166056628\\
59.75	0.19182	-68.5762035596363	-68.5762035596363\\
59.75	0.19548	-72.4416221774206	-72.4416221774206\\
59.75	0.19914	-76.4292724590157	-76.4292724590157\\
59.75	0.2028	-80.5391544044216	-80.5391544044216\\
59.75	0.20646	-84.7712680136381	-84.7712680136381\\
59.75	0.21012	-89.1256132866654	-89.1256132866654\\
59.75	0.21378	-93.6021902235035	-93.6021902235035\\
59.75	0.21744	-98.2009988241524	-98.2009988241524\\
59.75	0.2211	-102.922039088612	-102.922039088612\\
59.75	0.22476	-107.765311016882	-107.765311016882\\
59.75	0.22842	-112.730814608963	-112.730814608963\\
59.75	0.23208	-117.818549864855	-117.818549864855\\
59.75	0.23574	-123.028516784558	-123.028516784558\\
59.75	0.2394	-128.360715368072	-128.360715368072\\
59.75	0.24306	-133.815145615396	-133.815145615396\\
59.75	0.24672	-139.391807526531	-139.391807526531\\
59.75	0.25038	-145.090701101476	-145.090701101476\\
59.75	0.25404	-150.911826340233	-150.911826340233\\
59.75	0.2577	-156.8551832428	-156.8551832428\\
59.75	0.26136	-162.920771809178	-162.920771809178\\
59.75	0.26502	-169.108592039367	-169.108592039367\\
59.75	0.26868	-175.418643933366	-175.418643933366\\
59.75	0.27234	-181.850927491176	-181.850927491176\\
59.75	0.276	-188.405442712797	-188.405442712797\\
60.125	0.093	-10.417100530715	-10.417100530715\\
60.125	0.09666	-10.9850430611191	-10.9850430611191\\
60.125	0.10032	-11.6752172553341	-11.6752172553341\\
60.125	0.10398	-12.4876231133598	-12.4876231133598\\
60.125	0.10764	-13.4222606351962	-13.4222606351962\\
60.125	0.1113	-14.4791298208435	-14.4791298208435\\
60.125	0.11496	-15.6582306703014	-15.6582306703014\\
60.125	0.11862	-16.9595631835701	-16.9595631835701\\
60.125	0.12228	-18.3831273606496	-18.3831273606496\\
60.125	0.12594	-19.9289232015398	-19.9289232015398\\
60.125	0.1296	-21.5969507062408	-21.5969507062408\\
60.125	0.13326	-23.3872098747526	-23.3872098747526\\
60.125	0.13692	-25.2997007070751	-25.2997007070751\\
60.125	0.14058	-27.3344232032084	-27.3344232032084\\
60.125	0.14424	-29.4913773631524	-29.4913773631524\\
60.125	0.1479	-31.7705631869072	-31.7705631869072\\
60.125	0.15156	-34.1719806744728	-34.1719806744728\\
60.125	0.15522	-36.6956298258491	-36.6956298258491\\
60.125	0.15888	-39.3415106410362	-39.3415106410362\\
60.125	0.16254	-42.109623120034	-42.109623120034\\
60.125	0.1662	-44.9999672628426	-44.9999672628426\\
60.125	0.16986	-48.0125430694619	-48.0125430694619\\
60.125	0.17352	-51.147350539892	-51.147350539892\\
60.125	0.17718	-54.4043896741329	-54.4043896741329\\
60.125	0.18084	-57.7836604721845	-57.7836604721845\\
60.125	0.1845	-61.2851629340469	-61.2851629340469\\
60.125	0.18816	-64.90889705972	-64.90889705972\\
60.125	0.19182	-68.6548628492039	-68.6548628492039\\
60.125	0.19548	-72.5230603024985	-72.5230603024985\\
60.125	0.19914	-76.513489419604	-76.513489419604\\
60.125	0.2028	-80.6261502005202	-80.6261502005202\\
60.125	0.20646	-84.8610426452471	-84.8610426452471\\
60.125	0.21012	-89.2181667537848	-89.2181667537848\\
60.125	0.21378	-93.6975225261332	-93.6975225261332\\
60.125	0.21744	-98.2991099622924	-98.2991099622924\\
60.125	0.2211	-103.022929062262	-103.022929062262\\
60.125	0.22476	-107.868979826043	-107.868979826043\\
60.125	0.22842	-112.837262253635	-112.837262253635\\
60.125	0.23208	-117.927776345037	-117.927776345037\\
60.125	0.23574	-123.14052210025	-123.14052210025\\
60.125	0.2394	-128.475499519274	-128.475499519274\\
60.125	0.24306	-133.932708602108	-133.932708602108\\
60.125	0.24672	-139.512149348753	-139.512149348753\\
60.125	0.25038	-145.213821759209	-145.213821759209\\
60.125	0.25404	-151.037725833476	-151.037725833476\\
60.125	0.2577	-156.983861571554	-156.983861571554\\
60.125	0.26136	-163.052228973442	-163.052228973442\\
60.125	0.26502	-169.242828039141	-169.242828039141\\
60.125	0.26868	-175.555658768651	-175.555658768651\\
60.125	0.27234	-181.990721161972	-181.990721161972\\
60.125	0.276	-188.548015219103	-188.548015219103\\
60.5	0.093	-10.4303276340576	-10.4303276340576\\
60.5	0.09666	-11.0010489999721	-11.0010489999721\\
60.5	0.10032	-11.6940020296974	-11.6940020296974\\
60.5	0.10398	-12.5091867232334	-12.5091867232334\\
60.5	0.10764	-13.4466030805802	-13.4466030805802\\
60.5	0.1113	-14.5062511017378	-14.5062511017378\\
60.5	0.11496	-15.6881307867061	-15.6881307867061\\
60.5	0.11862	-16.9922421354852	-16.9922421354852\\
60.5	0.12228	-18.418585148075	-18.418585148075\\
60.5	0.12594	-19.9671598244756	-19.9671598244756\\
60.5	0.1296	-21.6379661646869	-21.6379661646869\\
60.5	0.13326	-23.4310041687091	-23.4310041687091\\
60.5	0.13692	-25.3462738365419	-25.3462738365419\\
60.5	0.14058	-27.3837751681856	-27.3837751681856\\
60.5	0.14424	-29.5435081636399	-29.5435081636399\\
60.5	0.1479	-31.8254728229051	-31.8254728229051\\
60.5	0.15156	-34.229669145981	-34.229669145981\\
60.5	0.15522	-36.7560971328676	-36.7560971328676\\
60.5	0.15888	-39.404756783565	-39.404756783565\\
60.5	0.16254	-42.1756480980732	-42.1756480980732\\
60.5	0.1662	-45.0687710763922	-45.0687710763922\\
60.5	0.16986	-48.0841257185219	-48.0841257185219\\
60.5	0.17352	-51.2217120244623	-51.2217120244623\\
60.5	0.17718	-54.4815299942135	-54.4815299942135\\
60.5	0.18084	-57.8635796277754	-57.8635796277754\\
60.5	0.1845	-61.3678609251482	-61.3678609251482\\
60.5	0.18816	-64.9943738863317	-64.9943738863317\\
60.5	0.19182	-68.7431185113259	-68.7431185113259\\
60.5	0.19548	-72.6140948001309	-72.6140948001309\\
60.5	0.19914	-76.6073027527467	-76.6073027527467\\
60.5	0.2028	-80.7227423691732	-80.7227423691732\\
60.5	0.20646	-84.9604136494105	-84.9604136494105\\
60.5	0.21012	-89.3203165934586	-89.3203165934586\\
60.5	0.21378	-93.8024512013173	-93.8024512013173\\
60.5	0.21744	-98.4068174729869	-98.4068174729869\\
60.5	0.2211	-103.133415408467	-103.133415408467\\
60.5	0.22476	-107.982245007758	-107.982245007758\\
60.5	0.22842	-112.95330627086	-112.95330627086\\
60.5	0.23208	-118.046599197773	-118.046599197773\\
60.5	0.23574	-123.262123788496	-123.262123788496\\
60.5	0.2394	-128.59988004303	-128.59988004303\\
60.5	0.24306	-134.059867961375	-134.059867961375\\
60.5	0.24672	-139.642087543531	-139.642087543531\\
60.5	0.25038	-145.346538789497	-145.346538789497\\
60.5	0.25404	-151.173221699274	-151.173221699274\\
60.5	0.2577	-157.122136272862	-157.122136272862\\
60.5	0.26136	-163.193282510261	-163.193282510261\\
60.5	0.26502	-169.38666041147	-169.38666041147\\
60.5	0.26868	-175.70226997649	-175.70226997649\\
60.5	0.27234	-182.140111205321	-182.140111205321\\
60.5	0.276	-188.700184097963	-188.700184097963\\
60.875	0.093	-10.4531511099545	-10.4531511099545\\
60.875	0.09666	-11.0266513113794	-11.0266513113794\\
60.875	0.10032	-11.7223831766151	-11.7223831766151\\
60.875	0.10398	-12.5403467056614	-12.5403467056614\\
60.875	0.10764	-13.4805418985186	-13.4805418985186\\
60.875	0.1113	-14.5429687551865	-14.5429687551865\\
60.875	0.11496	-15.7276272756652	-15.7276272756652\\
60.875	0.11862	-17.0345174599546	-17.0345174599546\\
60.875	0.12228	-18.4636393080548	-18.4636393080548\\
60.875	0.12594	-20.0149928199657	-20.0149928199657\\
60.875	0.1296	-21.6885779956874	-21.6885779956874\\
60.875	0.13326	-23.4843948352198	-23.4843948352198\\
60.875	0.13692	-25.402443338563	-25.402443338563\\
60.875	0.14058	-27.442723505717	-27.442723505717\\
60.875	0.14424	-29.6052353366818	-29.6052353366818\\
60.875	0.1479	-31.8899788314572	-31.8899788314572\\
60.875	0.15156	-34.2969539900435	-34.2969539900435\\
60.875	0.15522	-36.8261608124405	-36.8261608124405\\
60.875	0.15888	-39.4775992986483	-39.4775992986483\\
60.875	0.16254	-42.2512694486668	-42.2512694486668\\
60.875	0.1662	-45.1471712624961	-45.1471712624961\\
60.875	0.16986	-48.1653047401361	-48.1653047401361\\
60.875	0.17352	-51.3056698815869	-51.3056698815869\\
60.875	0.17718	-54.5682666868485	-54.5682666868485\\
60.875	0.18084	-57.9530951559207	-57.9530951559207\\
60.875	0.1845	-61.4601552888039	-61.4601552888039\\
60.875	0.18816	-65.0894470854977	-65.0894470854977\\
60.875	0.19182	-68.8409705460023	-68.8409705460023\\
60.875	0.19548	-72.7147256703176	-72.7147256703176\\
60.875	0.19914	-76.7107124584437	-76.7107124584437\\
60.875	0.2028	-80.8289309103806	-80.8289309103806\\
60.875	0.20646	-85.0693810261283	-85.0693810261283\\
60.875	0.21012	-89.4320628056866	-89.4320628056866\\
60.875	0.21378	-93.9169762490558	-93.9169762490558\\
60.875	0.21744	-98.5241213562357	-98.5241213562357\\
60.875	0.2211	-103.253498127226	-103.253498127226\\
60.875	0.22476	-108.105106562028	-108.105106562028\\
60.875	0.22842	-113.07894666064	-113.07894666064\\
60.875	0.23208	-118.175018423063	-118.175018423063\\
60.875	0.23574	-123.393321849297	-123.393321849297\\
60.875	0.2394	-128.733856939341	-128.733856939341\\
60.875	0.24306	-134.196623693196	-134.196623693196\\
60.875	0.24672	-139.781622110862	-139.781622110862\\
60.875	0.25038	-145.488852192339	-145.488852192339\\
60.875	0.25404	-151.318313937626	-151.318313937626\\
60.875	0.2577	-157.270007346725	-157.270007346725\\
60.875	0.26136	-163.343932419634	-163.343932419634\\
60.875	0.26502	-169.540089156353	-169.540089156353\\
60.875	0.26868	-175.858477556884	-175.858477556884\\
60.875	0.27234	-182.299097621225	-182.299097621225\\
60.875	0.276	-188.861949349377	-188.861949349377\\
61.25	0.093	-10.4855709584059	-10.4855709584059\\
61.25	0.09666	-11.0618499953412	-11.0618499953412\\
61.25	0.10032	-11.7603606960871	-11.7603606960871\\
61.25	0.10398	-12.5811030606439	-12.5811030606439\\
61.25	0.10764	-13.5240770890114	-13.5240770890114\\
61.25	0.1113	-14.5892827811896	-14.5892827811896\\
61.25	0.11496	-15.7767201371787	-15.7767201371787\\
61.25	0.11862	-17.0863891569784	-17.0863891569784\\
61.25	0.12228	-18.5182898405889	-18.5182898405889\\
61.25	0.12594	-20.0724221880102	-20.0724221880102\\
61.25	0.1296	-21.7487861992422	-21.7487861992422\\
61.25	0.13326	-23.5473818742851	-23.5473818742851\\
61.25	0.13692	-25.4682092131386	-25.4682092131386\\
61.25	0.14058	-27.511268215803	-27.511268215803\\
61.25	0.14424	-29.676558882278	-29.676558882278\\
61.25	0.1479	-31.9640812125639	-31.9640812125639\\
61.25	0.15156	-34.3738352066605	-34.3738352066605\\
61.25	0.15522	-36.9058208645678	-36.9058208645678\\
61.25	0.15888	-39.5600381862859	-39.5600381862859\\
61.25	0.16254	-42.3364871718148	-42.3364871718148\\
61.25	0.1662	-45.2351678211544	-45.2351678211544\\
61.25	0.16986	-48.2560801343049	-48.2560801343049\\
61.25	0.17352	-51.3992241112659	-51.3992241112659\\
61.25	0.17718	-54.6645997520378	-54.6645997520378\\
61.25	0.18084	-58.0522070566205	-58.0522070566205\\
61.25	0.1845	-61.562046025014	-61.562046025014\\
61.25	0.18816	-65.1941166572181	-65.1941166572181\\
61.25	0.19182	-68.9484189532331	-68.9484189532331\\
61.25	0.19548	-72.8249529130588	-72.8249529130588\\
61.25	0.19914	-76.8237185366953	-76.8237185366953\\
61.25	0.2028	-80.9447158241425	-80.9447158241425\\
61.25	0.20646	-85.1879447754005	-85.1879447754005\\
61.25	0.21012	-89.5534053904692	-89.5534053904692\\
61.25	0.21378	-94.0410976693486	-94.0410976693486\\
61.25	0.21744	-98.6510216120389	-98.6510216120389\\
61.25	0.2211	-103.38317721854	-103.38317721854\\
61.25	0.22476	-108.237564488852	-108.237564488852\\
61.25	0.22842	-113.214183422974	-113.214183422974\\
61.25	0.23208	-118.313034020907	-118.313034020907\\
61.25	0.23574	-123.534116282652	-123.534116282652\\
61.25	0.2394	-128.877430208206	-128.877430208206\\
61.25	0.24306	-134.342975797572	-134.342975797572\\
61.25	0.24672	-139.930753050748	-139.930753050748\\
61.25	0.25038	-145.640761967735	-145.640761967735\\
61.25	0.25404	-151.473002548533	-151.473002548533\\
61.25	0.2577	-157.427474793142	-157.427474793142\\
61.25	0.26136	-163.504178701561	-163.504178701561\\
61.25	0.26502	-169.703114273791	-169.703114273791\\
61.25	0.26868	-176.024281509832	-176.024281509832\\
61.25	0.27234	-182.467680409684	-182.467680409684\\
61.25	0.276	-189.033310973346	-189.033310973346\\
61.625	0.093	-10.5275871794117	-10.5275871794117\\
61.625	0.09666	-11.1066450518573	-11.1066450518573\\
61.625	0.10032	-11.8079345881137	-11.8079345881137\\
61.625	0.10398	-12.6314557881807	-12.6314557881807\\
61.625	0.10764	-13.5772086520586	-13.5772086520586\\
61.625	0.1113	-14.6451931797472	-14.6451931797472\\
61.625	0.11496	-15.8354093712465	-15.8354093712465\\
61.625	0.11862	-17.1478572265567	-17.1478572265567\\
61.625	0.12228	-18.5825367456775	-18.5825367456775\\
61.625	0.12594	-20.1394479286092	-20.1394479286092\\
61.625	0.1296	-21.8185907753516	-21.8185907753516\\
61.625	0.13326	-23.6199652859047	-23.6199652859047\\
61.625	0.13692	-25.5435714602686	-25.5435714602686\\
61.625	0.14058	-27.5894092984433	-27.5894092984433\\
61.625	0.14424	-29.7574788004287	-29.7574788004287\\
61.625	0.1479	-32.0477799662249	-32.0477799662249\\
61.625	0.15156	-34.4603127958319	-34.4603127958319\\
61.625	0.15522	-36.9950772892496	-36.9950772892496\\
61.625	0.15888	-39.652073446478	-39.652073446478\\
61.625	0.16254	-42.4313012675172	-42.4313012675172\\
61.625	0.1662	-45.3327607523672	-45.3327607523672\\
61.625	0.16986	-48.356451901028	-48.356451901028\\
61.625	0.17352	-51.5023747134994	-51.5023747134994\\
61.625	0.17718	-54.7705291897817	-54.7705291897817\\
61.625	0.18084	-58.1609153298747	-58.1609153298747\\
61.625	0.1845	-61.6735331337785	-61.6735331337785\\
61.625	0.18816	-65.308382601493	-65.308382601493\\
61.625	0.19182	-69.0654637330183	-69.0654637330183\\
61.625	0.19548	-72.9447765283543	-72.9447765283543\\
61.625	0.19914	-76.9463209875011	-76.9463209875011\\
61.625	0.2028	-81.0700971104587	-81.0700971104587\\
61.625	0.20646	-85.3161048972271	-85.3161048972271\\
61.625	0.21012	-89.6843443478061	-89.6843443478061\\
61.625	0.21378	-94.1748154621959	-94.1748154621959\\
61.625	0.21744	-98.7875182403965	-98.7875182403965\\
61.625	0.2211	-103.522452682408	-103.522452682408\\
61.625	0.22476	-108.37961878823	-108.37961878823\\
61.625	0.22842	-113.359016557863	-113.359016557863\\
61.625	0.23208	-118.460645991307	-118.460645991307\\
61.625	0.23574	-123.684507088561	-123.684507088561\\
61.625	0.2394	-129.030599849626	-129.030599849626\\
61.625	0.24306	-134.498924274502	-134.498924274502\\
61.625	0.24672	-140.089480363189	-140.089480363189\\
61.625	0.25038	-145.802268115686	-145.802268115686\\
61.625	0.25404	-151.637287531994	-151.637287531994\\
61.625	0.2577	-157.594538612113	-157.594538612113\\
61.625	0.26136	-163.674021356043	-163.674021356043\\
61.625	0.26502	-169.875735763783	-169.875735763783\\
61.625	0.26868	-176.199681835335	-176.199681835335\\
61.625	0.27234	-182.645859570696	-182.645859570696\\
61.625	0.276	-189.214268969869	-189.214268969869\\
62	0.093	-10.5791997729719	-10.5791997729719\\
62	0.09666	-11.1610364809278	-11.1610364809278\\
62	0.10032	-11.8651048526945	-11.8651048526945\\
62	0.10398	-12.6914048882719	-12.6914048882719\\
62	0.10764	-13.6399365876601	-13.6399365876601\\
62	0.1113	-14.7106999508591	-14.7106999508591\\
62	0.11496	-15.9036949778688	-15.9036949778688\\
62	0.11862	-17.2189216686892	-17.2189216686892\\
62	0.12228	-18.6563800233205	-18.6563800233205\\
62	0.12594	-20.2160700417624	-20.2160700417624\\
62	0.1296	-21.8979917240152	-21.8979917240152\\
62	0.13326	-23.7021450700787	-23.7021450700787\\
62	0.13692	-25.6285300799529	-25.6285300799529\\
62	0.14058	-27.677146753638	-27.677146753638\\
62	0.14424	-29.8479950911337	-29.8479950911337\\
62	0.1479	-32.1410750924403	-32.1410750924403\\
62	0.15156	-34.5563867575575	-34.5563867575575\\
62	0.15522	-37.0939300864856	-37.0939300864856\\
62	0.15888	-39.7537050792244	-39.7537050792244\\
62	0.16254	-42.535711735774	-42.535711735774\\
62	0.1662	-45.4399500561343	-45.4399500561343\\
62	0.16986	-48.4664200403054	-48.4664200403054\\
62	0.17352	-51.6151216882872	-51.6151216882872\\
62	0.17718	-54.8860550000798	-54.8860550000798\\
62	0.18084	-58.2792199756832	-58.2792199756832\\
62	0.1845	-61.7946166150973	-61.7946166150973\\
62	0.18816	-65.4322449183222	-65.4322449183222\\
62	0.19182	-69.1921048853578	-69.1921048853578\\
62	0.19548	-73.0741965162042	-73.0741965162042\\
62	0.19914	-77.0785198108613	-77.0785198108613\\
62	0.2028	-81.2050747693293	-81.2050747693293\\
62	0.20646	-85.453861391608	-85.453861391608\\
62	0.21012	-89.8248796776974	-89.8248796776974\\
62	0.21378	-94.3181296275976	-94.3181296275976\\
62	0.21744	-98.9336112413085	-98.9336112413085\\
62	0.2211	-103.67132451883	-103.67132451883\\
62	0.22476	-108.531269460163	-108.531269460163\\
62	0.22842	-113.513446065306	-113.513446065306\\
62	0.23208	-118.61785433426	-118.61785433426\\
62	0.23574	-123.844494267025	-123.844494267025\\
62	0.2394	-129.1933658636	-129.1933658636\\
62	0.24306	-134.664469123986	-134.664469123986\\
62	0.24672	-140.257804048183	-140.257804048183\\
62	0.25038	-145.973370636191	-145.973370636191\\
62	0.25404	-151.81116888801	-151.81116888801\\
62	0.2577	-157.771198803639	-157.771198803639\\
62	0.26136	-163.853460383079	-163.853460383079\\
62	0.26502	-170.05795362633	-170.05795362633\\
62	0.26868	-176.384678533391	-176.384678533391\\
62	0.27234	-182.833635104264	-182.833635104264\\
62	0.276	-189.404823338947	-189.404823338947\\
62.375	0.093	-10.6404087390865	-10.6404087390865\\
62.375	0.09666	-11.2250242825528	-11.2250242825528\\
62.375	0.10032	-11.9318714898298	-11.9318714898298\\
62.375	0.10398	-12.7609503609176	-12.7609503609176\\
62.375	0.10764	-13.7122608958161	-13.7122608958161\\
62.375	0.1113	-14.7858030945254	-14.7858030945254\\
62.375	0.11496	-15.9815769570455	-15.9815769570455\\
62.375	0.11862	-17.2995824833763	-17.2995824833763\\
62.375	0.12228	-18.7398196735178	-18.7398196735178\\
62.375	0.12594	-20.3022885274702	-20.3022885274702\\
62.375	0.1296	-21.9869890452332	-21.9869890452332\\
62.375	0.13326	-23.7939212268071	-23.7939212268071\\
62.375	0.13692	-25.7230850721917	-25.7230850721917\\
62.375	0.14058	-27.7744805813871	-27.7744805813871\\
62.375	0.14424	-29.9481077543932	-29.9481077543932\\
62.375	0.1479	-32.2439665912101	-32.2439665912101\\
62.375	0.15156	-34.6620570918377	-34.6620570918377\\
62.375	0.15522	-37.2023792562761	-37.2023792562761\\
62.375	0.15888	-39.8649330845253	-39.8649330845253\\
62.375	0.16254	-42.6497185765852	-42.6497185765852\\
62.375	0.1662	-45.5567357324559	-45.5567357324559\\
62.375	0.16986	-48.5859845521373	-48.5859845521373\\
62.375	0.17352	-51.7374650356295	-51.7374650356295\\
62.375	0.17718	-55.0111771829324	-55.0111771829324\\
62.375	0.18084	-58.4071209940461	-58.4071209940461\\
62.375	0.1845	-61.9252964689706	-61.9252964689706\\
62.375	0.18816	-65.5657036077058	-65.5657036077058\\
62.375	0.19182	-69.3283424102518	-69.3283424102518\\
62.375	0.19548	-73.2132128766086	-73.2132128766086\\
62.375	0.19914	-77.2203150067761	-77.2203150067761\\
62.375	0.2028	-81.3496488007544	-81.3496488007544\\
62.375	0.20646	-85.6012142585434	-85.6012142585434\\
62.375	0.21012	-89.9750113801431	-89.9750113801431\\
62.375	0.21378	-94.4710401655537	-94.4710401655537\\
62.375	0.21744	-99.0893006147749	-99.0893006147749\\
62.375	0.2211	-103.829792727807	-103.829792727807\\
62.375	0.22476	-108.69251650465	-108.69251650465\\
62.375	0.22842	-113.677471945303	-113.677471945303\\
62.375	0.23208	-118.784659049768	-118.784659049768\\
62.375	0.23574	-124.014077818043	-124.014077818043\\
62.375	0.2394	-129.365728250129	-129.365728250129\\
62.375	0.24306	-134.839610346025	-134.839610346025\\
62.375	0.24672	-140.435724105733	-140.435724105733\\
62.375	0.25038	-146.154069529251	-146.154069529251\\
62.375	0.25404	-151.99464661658	-151.99464661658\\
62.375	0.2577	-157.957455367719	-157.957455367719\\
62.375	0.26136	-164.04249578267	-164.04249578267\\
62.375	0.26502	-170.249767861431	-170.249767861431\\
62.375	0.26868	-176.579271604003	-176.579271604003\\
62.375	0.27234	-183.031007010385	-183.031007010385\\
62.375	0.276	-189.604974080579	-189.604974080579\\
62.75	0.093	-10.7112140777555	-10.7112140777555\\
62.75	0.09666	-11.2986084567321	-11.2986084567321\\
62.75	0.10032	-12.0082344995195	-12.0082344995195\\
62.75	0.10398	-12.8400922061176	-12.8400922061176\\
62.75	0.10764	-13.7941815765265	-13.7941815765265\\
62.75	0.1113	-14.8705026107462	-14.8705026107462\\
62.75	0.11496	-16.0690553087766	-16.0690553087766\\
62.75	0.11862	-17.3898396706177	-17.3898396706177\\
62.75	0.12228	-18.8328556962696	-18.8328556962696\\
62.75	0.12594	-20.3981033857323	-20.3981033857323\\
62.75	0.1296	-22.0855827390057	-22.0855827390057\\
62.75	0.13326	-23.8952937560899	-23.8952937560899\\
62.75	0.13692	-25.8272364369849	-25.8272364369849\\
62.75	0.14058	-27.8814107816906	-27.8814107816906\\
62.75	0.14424	-30.0578167902071	-30.0578167902071\\
62.75	0.1479	-32.3564544625343	-32.3564544625343\\
62.75	0.15156	-34.7773237986723	-34.7773237986723\\
62.75	0.15522	-37.320424798621	-37.320424798621\\
62.75	0.15888	-39.9857574623806	-39.9857574623806\\
62.75	0.16254	-42.7733217899508	-42.7733217899508\\
62.75	0.1662	-45.6831177813318	-45.6831177813318\\
62.75	0.16986	-48.7151454365236	-48.7151454365236\\
62.75	0.17352	-51.8694047555261	-51.8694047555261\\
62.75	0.17718	-55.1458957383394	-55.1458957383394\\
62.75	0.18084	-58.5446183849635	-58.5446183849635\\
62.75	0.1845	-62.0655726953983	-62.0655726953983\\
62.75	0.18816	-65.7087586696439	-65.7087586696439\\
62.75	0.19182	-69.4741763077002	-69.4741763077002\\
62.75	0.19548	-73.3618256095673	-73.3618256095673\\
62.75	0.19914	-77.3717065752451	-77.3717065752451\\
62.75	0.2028	-81.5038192047338	-81.5038192047338\\
62.75	0.20646	-85.7581634980332	-85.7581634980332\\
62.75	0.21012	-90.1347394551433	-90.1347394551433\\
62.75	0.21378	-94.6335470760641	-94.6335470760641\\
62.75	0.21744	-99.2545863607958	-99.2545863607958\\
62.75	0.2211	-103.997857309338	-103.997857309338\\
62.75	0.22476	-108.863359921691	-108.863359921691\\
62.75	0.22842	-113.851094197855	-113.851094197855\\
62.75	0.23208	-118.96106013783	-118.96106013783\\
62.75	0.23574	-124.193257741615	-124.193257741615\\
62.75	0.2394	-129.547687009212	-129.547687009212\\
62.75	0.24306	-135.024347940619	-135.024347940619\\
62.75	0.24672	-140.623240535836	-140.623240535836\\
62.75	0.25038	-146.344364794865	-146.344364794865\\
62.75	0.25404	-152.187720717704	-152.187720717704\\
62.75	0.2577	-158.153308304354	-158.153308304354\\
62.75	0.26136	-164.241127554815	-164.241127554815\\
62.75	0.26502	-170.451178469086	-170.451178469086\\
62.75	0.26868	-176.783461047168	-176.783461047168\\
62.75	0.27234	-183.237975289061	-183.237975289061\\
62.75	0.276	-189.814721194765	-189.814721194765\\
63.125	0.093	-10.7916157889788	-10.7916157889788\\
63.125	0.09666	-11.3817890034658	-11.3817890034658\\
63.125	0.10032	-12.0941938817635	-12.0941938817635\\
63.125	0.10398	-12.928830423872	-12.928830423872\\
63.125	0.10764	-13.8856986297912	-13.8856986297912\\
63.125	0.1113	-14.9647984995213	-14.9647984995213\\
63.125	0.11496	-16.166130033062	-16.166130033062\\
63.125	0.11862	-17.4896932304135	-17.4896932304135\\
63.125	0.12228	-18.9354880915758	-18.9354880915758\\
63.125	0.12594	-20.5035146165488	-20.5035146165488\\
63.125	0.1296	-22.1937728053326	-22.1937728053326\\
63.125	0.13326	-24.0062626579271	-24.0062626579271\\
63.125	0.13692	-25.9409841743324	-25.9409841743324\\
63.125	0.14058	-27.9979373545485	-27.9979373545485\\
63.125	0.14424	-30.1771221985753	-30.1771221985753\\
63.125	0.1479	-32.4785387064128	-32.4785387064128\\
63.125	0.15156	-34.9021868780612	-34.9021868780612\\
63.125	0.15522	-37.4480667135203	-37.4480667135203\\
63.125	0.15888	-40.1161782127902	-40.1161782127902\\
63.125	0.16254	-42.9065213758708	-42.9065213758708\\
63.125	0.1662	-45.8190962027621	-45.8190962027621\\
63.125	0.16986	-48.8539026934643	-48.8539026934643\\
63.125	0.17352	-52.0109408479771	-52.0109408479771\\
63.125	0.17718	-55.2902106663008	-55.2902106663008\\
63.125	0.18084	-58.6917121484352	-58.6917121484352\\
63.125	0.1845	-62.2154452943804	-62.2154452943804\\
63.125	0.18816	-65.8614101041363	-65.8614101041363\\
63.125	0.19182	-69.6296065777029	-69.6296065777029\\
63.125	0.19548	-73.5200347150804	-73.5200347150804\\
63.125	0.19914	-77.5326945162685	-77.5326945162685\\
63.125	0.2028	-81.6675859812676	-81.6675859812676\\
63.125	0.20646	-85.9247091100773	-85.9247091100773\\
63.125	0.21012	-90.3040639026977	-90.3040639026977\\
63.125	0.21378	-94.8056503591289	-94.8056503591289\\
63.125	0.21744	-99.429468479371	-99.429468479371\\
63.125	0.2211	-104.175518263424	-104.175518263424\\
63.125	0.22476	-109.043799711287	-109.043799711287\\
63.125	0.22842	-114.034312822961	-114.034312822961\\
63.125	0.23208	-119.147057598446	-119.147057598446\\
63.125	0.23574	-124.382034037742	-124.382034037742\\
63.125	0.2394	-129.739242140849	-129.739242140849\\
63.125	0.24306	-135.218681907766	-135.218681907766\\
63.125	0.24672	-140.820353338494	-140.820353338494\\
63.125	0.25038	-146.544256433033	-146.544256433033\\
63.125	0.25404	-152.390391191383	-152.390391191383\\
63.125	0.2577	-158.358757613543	-158.358757613543\\
63.125	0.26136	-164.449355699514	-164.449355699514\\
63.125	0.26502	-170.662185449296	-170.662185449296\\
63.125	0.26868	-176.997246862888	-176.997246862888\\
63.125	0.27234	-183.454539940292	-183.454539940292\\
63.125	0.276	-190.034064681506	-190.034064681506\\
63.5	0.093	-10.8816138727566	-10.8816138727566\\
63.5	0.09666	-11.474565922754	-11.474565922754\\
63.5	0.10032	-12.189749636562	-12.189749636562\\
63.5	0.10398	-13.0271650141808	-13.0271650141808\\
63.5	0.10764	-13.9868120556104	-13.9868120556104\\
63.5	0.1113	-15.0686907608508	-15.0686907608508\\
63.5	0.11496	-16.2728011299019	-16.2728011299019\\
63.5	0.11862	-17.5991431627637	-17.5991431627637\\
63.5	0.12228	-19.0477168594363	-19.0477168594363\\
63.5	0.12594	-20.6185222199197	-20.6185222199197\\
63.5	0.1296	-22.3115592442138	-22.3115592442138\\
63.5	0.13326	-24.1268279323187	-24.1268279323187\\
63.5	0.13692	-26.0643282842344	-26.0643282842344\\
63.5	0.14058	-28.1240602999608	-28.1240602999608\\
63.5	0.14424	-30.306023979498	-30.306023979498\\
63.5	0.1479	-32.6102193228459	-32.6102193228459\\
63.5	0.15156	-35.0366463300046	-35.0366463300046\\
63.5	0.15522	-37.585305000974	-37.585305000974\\
63.5	0.15888	-40.2561953357542	-40.2561953357542\\
63.5	0.16254	-43.0493173343452	-43.0493173343452\\
63.5	0.1662	-45.9646709967469	-45.9646709967469\\
63.5	0.16986	-49.0022563229594	-49.0022563229594\\
63.5	0.17352	-52.1620733129826	-52.1620733129826\\
63.5	0.17718	-55.4441219668166	-55.4441219668166\\
63.5	0.18084	-58.8484022844613	-58.8484022844613\\
63.5	0.1845	-62.3749142659169	-62.3749142659169\\
63.5	0.18816	-66.0236579111831	-66.0236579111831\\
63.5	0.19182	-69.7946332202602	-69.7946332202602\\
63.5	0.19548	-73.687840193148	-73.687840193148\\
63.5	0.19914	-77.7032788298465	-77.7032788298465\\
63.5	0.2028	-81.8409491303558	-81.8409491303558\\
63.5	0.20646	-86.1008510946759	-86.1008510946759\\
63.5	0.21012	-90.4829847228067	-90.4829847228067\\
63.5	0.21378	-94.9873500147482	-94.9873500147482\\
63.5	0.21744	-99.6139469705006	-99.6139469705006\\
63.5	0.2211	-104.362775590064	-104.362775590064\\
63.5	0.22476	-109.233835873438	-109.233835873438\\
63.5	0.22842	-114.227127820622	-114.227127820622\\
63.5	0.23208	-119.342651431618	-119.342651431618\\
63.5	0.23574	-124.580406706424	-124.580406706424\\
63.5	0.2394	-129.940393645041	-129.940393645041\\
63.5	0.24306	-135.422612247468	-135.422612247468\\
63.5	0.24672	-141.027062513707	-141.027062513707\\
63.5	0.25038	-146.753744443756	-146.753744443756\\
63.5	0.25404	-152.602658037616	-152.602658037616\\
63.5	0.2577	-158.573803295286	-158.573803295286\\
63.5	0.26136	-164.667180216768	-164.667180216768\\
63.5	0.26502	-170.88278880206	-170.88278880206\\
63.5	0.26868	-177.220629051163	-177.220629051163\\
63.5	0.27234	-183.680700964077	-183.680700964077\\
63.5	0.276	-190.263004540801	-190.263004540801\\
63.875	0.093	-10.9812083290888	-10.9812083290888\\
63.875	0.09666	-11.5769392145965	-11.5769392145965\\
63.875	0.10032	-12.2949017639149	-12.2949017639149\\
63.875	0.10398	-13.1350959770441	-13.1350959770441\\
63.875	0.10764	-14.097521853984	-14.097521853984\\
63.875	0.1113	-15.1821793947347	-15.1821793947347\\
63.875	0.11496	-16.3890685992962	-16.3890685992962\\
63.875	0.11862	-17.7181894676684	-17.7181894676684\\
63.875	0.12228	-19.1695419998513	-19.1695419998513\\
63.875	0.12594	-20.743126195845	-20.743126195845\\
63.875	0.1296	-22.4389420556495	-22.4389420556495\\
63.875	0.13326	-24.2569895792647	-24.2569895792647\\
63.875	0.13692	-26.1972687666907	-26.1972687666907\\
63.875	0.14058	-28.2597796179275	-28.2597796179275\\
63.875	0.14424	-30.444522132975	-30.444522132975\\
63.875	0.1479	-32.7514963118333	-32.7514963118333\\
63.875	0.15156	-35.1807021545023	-35.1807021545023\\
63.875	0.15522	-37.7321396609821	-37.7321396609821\\
63.875	0.15888	-40.4058088312727	-40.4058088312727\\
63.875	0.16254	-43.201709665374	-43.201709665374\\
63.875	0.1662	-46.1198421632861	-46.1198421632861\\
63.875	0.16986	-49.1602063250089	-49.1602063250089\\
63.875	0.17352	-52.3228021505424	-52.3228021505424\\
63.875	0.17718	-55.6076296398868	-55.6076296398868\\
63.875	0.18084	-59.0146887930419	-59.0146887930419\\
63.875	0.1845	-62.5439796100077	-62.5439796100077\\
63.875	0.18816	-66.1955020907843	-66.1955020907843\\
63.875	0.19182	-69.9692562353717	-69.9692562353717\\
63.875	0.19548	-73.8652420437698	-73.8652420437698\\
63.875	0.19914	-77.8834595159787	-77.8834595159787\\
63.875	0.2028	-82.0239086519985	-82.0239086519985\\
63.875	0.20646	-86.2865894518289	-86.2865894518289\\
63.875	0.21012	-90.67150191547	-90.67150191547\\
63.875	0.21378	-95.1786460429219	-95.1786460429219\\
63.875	0.21744	-99.8080218341846	-99.8080218341846\\
63.875	0.2211	-104.559629289258	-104.559629289258\\
63.875	0.22476	-109.433468408142	-109.433468408142\\
63.875	0.22842	-114.429539190837	-114.429539190837\\
63.875	0.23208	-119.547841637343	-119.547841637343\\
63.875	0.23574	-124.78837574766	-124.78837574766\\
63.875	0.2394	-130.151141521787	-130.151141521787\\
63.875	0.24306	-135.636138959725	-135.636138959725\\
63.875	0.24672	-141.243368061473	-141.243368061473\\
63.875	0.25038	-146.972828827033	-146.972828827033\\
63.875	0.25404	-152.824521256403	-152.824521256403\\
63.875	0.2577	-158.798445349584	-158.798445349584\\
63.875	0.26136	-164.894601106576	-164.894601106576\\
63.875	0.26502	-171.112988527379	-171.112988527379\\
63.875	0.26868	-177.453607611992	-177.453607611992\\
63.875	0.27234	-183.916458360416	-183.916458360416\\
63.875	0.276	-190.501540772651	-190.501540772651\\
64.25	0.093	-11.0903991579754	-11.0903991579754\\
64.25	0.09666	-11.6889088789934	-11.6889088789934\\
64.25	0.10032	-12.4096502638221	-12.4096502638221\\
64.25	0.10398	-13.2526233124617	-13.2526233124617\\
64.25	0.10764	-14.2178280249119	-14.2178280249119\\
64.25	0.1113	-15.305264401173	-15.305264401173\\
64.25	0.11496	-16.5149324412448	-16.5149324412448\\
64.25	0.11862	-17.8468321451273	-17.8468321451273\\
64.25	0.12228	-19.3009635128206	-19.3009635128206\\
64.25	0.12594	-20.8773265443247	-20.8773265443247\\
64.25	0.1296	-22.5759212396395	-22.5759212396395\\
64.25	0.13326	-24.3967475987651	-24.3967475987651\\
64.25	0.13692	-26.3398056217015	-26.3398056217015\\
64.25	0.14058	-28.4050953084486	-28.4050953084486\\
64.25	0.14424	-30.5926166590064	-30.5926166590064\\
64.25	0.1479	-32.9023696733751	-32.9023696733751\\
64.25	0.15156	-35.3343543515544	-35.3343543515544\\
64.25	0.15522	-37.8885706935446	-37.8885706935446\\
64.25	0.15888	-40.5650186993455	-40.5650186993455\\
64.25	0.16254	-43.3636983689571	-43.3636983689571\\
64.25	0.1662	-46.2846097023795	-46.2846097023795\\
64.25	0.16986	-49.3277526996127	-49.3277526996127\\
64.25	0.17352	-52.4931273606566	-52.4931273606566\\
64.25	0.17718	-55.7807336855113	-55.7807336855113\\
64.25	0.18084	-59.1905716741768	-59.1905716741768\\
64.25	0.1845	-62.722641326653	-62.722641326653\\
64.25	0.18816	-66.3769426429399	-66.3769426429399\\
64.25	0.19182	-70.1534756230377	-70.1534756230377\\
64.25	0.19548	-74.0522402669461	-74.0522402669461\\
64.25	0.19914	-78.0732365746653	-78.0732365746653\\
64.25	0.2028	-82.2164645461954	-82.2164645461954\\
64.25	0.20646	-86.4819241815362	-86.4819241815362\\
64.25	0.21012	-90.8696154806877	-90.8696154806877\\
64.25	0.21378	-95.3795384436499	-95.3795384436499\\
64.25	0.21744	-100.011693070423	-100.011693070423\\
64.25	0.2211	-104.766079361007	-104.766079361007\\
64.25	0.22476	-109.642697315401	-109.642697315401\\
64.25	0.22842	-114.641546933607	-114.641546933607\\
64.25	0.23208	-119.762628215623	-119.762628215623\\
64.25	0.23574	-125.00594116145	-125.00594116145\\
64.25	0.2394	-130.371485771087	-130.371485771087\\
64.25	0.24306	-135.859262044535	-135.859262044535\\
64.25	0.24672	-141.469269981795	-141.469269981795\\
64.25	0.25038	-147.201509582864	-147.201509582864\\
64.25	0.25404	-153.055980847745	-153.055980847745\\
64.25	0.2577	-159.032683776436	-159.032683776436\\
64.25	0.26136	-165.131618368939	-165.131618368939\\
64.25	0.26502	-171.352784625251	-171.352784625251\\
64.25	0.26868	-177.696182545375	-177.696182545375\\
64.25	0.27234	-184.16181212931	-184.16181212931\\
64.25	0.276	-190.749673377055	-190.749673377055\\
64.625	0.093	-11.2091863594163	-11.2091863594163\\
64.625	0.09666	-11.8104749159447	-11.8104749159447\\
64.625	0.10032	-12.5339951362838	-12.5339951362838\\
64.625	0.10398	-13.3797470204337	-13.3797470204337\\
64.625	0.10764	-14.3477305683943	-14.3477305683943\\
64.625	0.1113	-15.4379457801657	-15.4379457801657\\
64.625	0.11496	-16.6503926557478	-16.6503926557478\\
64.625	0.11862	-17.9850711951407	-17.9850711951407\\
64.625	0.12228	-19.4419813983444	-19.4419813983444\\
64.625	0.12594	-21.0211232653588	-21.0211232653588\\
64.625	0.1296	-22.722496796184	-22.722496796184\\
64.625	0.13326	-24.5461019908199	-24.5461019908199\\
64.625	0.13692	-26.4919388492666	-26.4919388492666\\
64.625	0.14058	-28.560007371524	-28.560007371524\\
64.625	0.14424	-30.7503075575923	-30.7503075575923\\
64.625	0.1479	-33.0628394074712	-33.0628394074712\\
64.625	0.15156	-35.497602921161	-35.497602921161\\
64.625	0.15522	-38.0545980986615	-38.0545980986615\\
64.625	0.15888	-40.7338249399727	-40.7338249399727\\
64.625	0.16254	-43.5352834450947	-43.5352834450947\\
64.625	0.1662	-46.4589736140275	-46.4589736140275\\
64.625	0.16986	-49.504895446771	-49.504895446771\\
64.625	0.17352	-52.6730489433252	-52.6730489433252\\
64.625	0.17718	-55.9634341036903	-55.9634341036903\\
64.625	0.18084	-59.3760509278661	-59.3760509278661\\
64.625	0.1845	-62.9108994158527	-62.9108994158527\\
64.625	0.18816	-66.56797956765	-66.56797956765\\
64.625	0.19182	-70.3472913832581	-70.3472913832581\\
64.625	0.19548	-74.2488348626769	-74.2488348626769\\
64.625	0.19914	-78.2726100059065	-78.2726100059065\\
64.625	0.2028	-82.4186168129468	-82.4186168129468\\
64.625	0.20646	-86.6868552837979	-86.6868552837979\\
64.625	0.21012	-91.0773254184598	-91.0773254184598\\
64.625	0.21378	-95.5900272169324	-95.5900272169324\\
64.625	0.21744	-100.224960679216	-100.224960679216\\
64.625	0.2211	-104.98212580531	-104.98212580531\\
64.625	0.22476	-109.861522595215	-109.861522595215\\
64.625	0.22842	-114.86315104893	-114.86315104893\\
64.625	0.23208	-119.987011166457	-119.987011166457\\
64.625	0.23574	-125.233102947794	-125.233102947794\\
64.625	0.2394	-130.601426392942	-130.601426392942\\
64.625	0.24306	-136.091981501901	-136.091981501901\\
64.625	0.24672	-141.70476827467	-141.70476827467\\
64.625	0.25038	-147.43978671125	-147.43978671125\\
64.625	0.25404	-153.297036811641	-153.297036811641\\
64.625	0.2577	-159.276518575843	-159.276518575843\\
64.625	0.26136	-165.378232003856	-165.378232003856\\
64.625	0.26502	-171.602177095679	-171.602177095679\\
64.625	0.26868	-177.948353851313	-177.948353851313\\
64.625	0.27234	-184.416762270758	-184.416762270758\\
64.625	0.276	-191.007402354013	-191.007402354013\\
65	0.093	-11.3375699334117	-11.3375699334117\\
65	0.09666	-11.9416373254504	-11.9416373254504\\
65	0.10032	-12.6679363812999	-12.6679363812999\\
65	0.10398	-13.5164671009601	-13.5164671009601\\
65	0.10764	-14.4872294844311	-14.4872294844311\\
65	0.1113	-15.5802235317128	-15.5802235317128\\
65	0.11496	-16.7954492428053	-16.7954492428053\\
65	0.11862	-18.1329066177086	-18.1329066177086\\
65	0.12228	-19.5925956564226	-19.5925956564226\\
65	0.12594	-21.1745163589473	-21.1745163589473\\
65	0.1296	-22.8786687252828	-22.8786687252828\\
65	0.13326	-24.7050527554291	-24.7050527554291\\
65	0.13692	-26.6536684493861	-26.6536684493861\\
65	0.14058	-28.724515807154	-28.724515807154\\
65	0.14424	-30.9175948287325	-30.9175948287325\\
65	0.1479	-33.2329055141219	-33.2329055141219\\
65	0.15156	-35.6704478633219	-35.6704478633219\\
65	0.15522	-38.2302218763327	-38.2302218763327\\
65	0.15888	-40.9122275531544	-40.9122275531544\\
65	0.16254	-43.7164648937867	-43.7164648937867\\
65	0.1662	-46.6429338982298	-46.6429338982298\\
65	0.16986	-49.6916345664837	-49.6916345664837\\
65	0.17352	-52.8625668985483	-52.8625668985483\\
65	0.17718	-56.1557308944237	-56.1557308944237\\
65	0.18084	-59.5711265541098	-59.5711265541098\\
65	0.1845	-63.1087538776067	-63.1087538776067\\
65	0.18816	-66.7686128649144	-66.7686128649144\\
65	0.19182	-70.5507035160328	-70.5507035160328\\
65	0.19548	-74.4550258309619	-74.4550258309619\\
65	0.19914	-78.4815798097019	-78.4815798097019\\
65	0.2028	-82.6303654522526	-82.6303654522526\\
65	0.20646	-86.9013827586141	-86.9013827586141\\
65	0.21012	-91.2946317287863	-91.2946317287863\\
65	0.21378	-95.8101123627692	-95.8101123627692\\
65	0.21744	-100.447824660563	-100.447824660563\\
65	0.2211	-105.207768622167	-105.207768622167\\
65	0.22476	-110.089944247583	-110.089944247583\\
65	0.22842	-115.094351536809	-115.094351536809\\
65	0.23208	-120.220990489845	-120.220990489845\\
65	0.23574	-125.469861106693	-125.469861106693\\
65	0.2394	-130.840963387351	-130.840963387351\\
65	0.24306	-136.33429733182	-136.33429733182\\
65	0.24672	-141.9498629401	-141.9498629401\\
65	0.25038	-147.687660212191	-147.687660212191\\
65	0.25404	-153.547689148092	-153.547689148092\\
65	0.2577	-159.529949747804	-159.529949747804\\
65	0.26136	-165.634442011327	-165.634442011327\\
65	0.26502	-171.861165938661	-171.861165938661\\
65	0.26868	-178.210121529805	-178.210121529805\\
65	0.27234	-184.68130878476	-184.68130878476\\
65	0.276	-191.274727703526	-191.274727703526\\
65.375	0.093	-11.4755498799614	-11.4755498799614\\
65.375	0.09666	-12.0823961075105	-12.0823961075105\\
65.375	0.10032	-12.8114739988703	-12.8114739988703\\
65.375	0.10398	-13.6627835540409	-13.6627835540409\\
65.375	0.10764	-14.6363247730222	-14.6363247730222\\
65.375	0.1113	-15.7320976558143	-15.7320976558143\\
65.375	0.11496	-16.9501022024171	-16.9501022024171\\
65.375	0.11862	-18.2903384128307	-18.2903384128307\\
65.375	0.12228	-19.7528062870551	-19.7528062870551\\
65.375	0.12594	-21.3375058250902	-21.3375058250902\\
65.375	0.1296	-23.0444370269361	-23.0444370269361\\
65.375	0.13326	-24.8735998925927	-24.8735998925927\\
65.375	0.13692	-26.8249944220601	-26.8249944220601\\
65.375	0.14058	-28.8986206153382	-28.8986206153382\\
65.375	0.14424	-31.0944784724271	-31.0944784724271\\
65.375	0.1479	-33.4125679933268	-33.4125679933268\\
65.375	0.15156	-35.8528891780372	-35.8528891780372\\
65.375	0.15522	-38.4154420265584	-38.4154420265584\\
65.375	0.15888	-41.1002265388904	-41.1002265388904\\
65.375	0.16254	-43.907242715033	-43.907242715033\\
65.375	0.1662	-46.8364905549865	-46.8364905549865\\
65.375	0.16986	-49.8879700587507	-49.8879700587507\\
65.375	0.17352	-53.0616812263257	-53.0616812263257\\
65.375	0.17718	-56.3576240577114	-56.3576240577114\\
65.375	0.18084	-59.7757985529079	-59.7757985529079\\
65.375	0.1845	-63.3162047119152	-63.3162047119152\\
65.375	0.18816	-66.9788425347331	-66.9788425347331\\
65.375	0.19182	-70.7637120213619	-70.7637120213619\\
65.375	0.19548	-74.6708131718014	-74.6708131718014\\
65.375	0.19914	-78.7001459860517	-78.7001459860517\\
65.375	0.2028	-82.8517104641128	-82.8517104641128\\
65.375	0.20646	-87.1255066059846	-87.1255066059846\\
65.375	0.21012	-91.5215344116672	-91.5215344116672\\
65.375	0.21378	-96.0397938811605	-96.0397938811605\\
65.375	0.21744	-100.680285014465	-100.680285014465\\
65.375	0.2211	-105.443007811579	-105.443007811579\\
65.375	0.22476	-110.327962272505	-110.327962272505\\
65.375	0.22842	-115.335148397241	-115.335148397241\\
65.375	0.23208	-120.464566185788	-120.464566185788\\
65.375	0.23574	-125.716215638146	-125.716215638146\\
65.375	0.2394	-131.090096754315	-131.090096754315\\
65.375	0.24306	-136.586209534294	-136.586209534294\\
65.375	0.24672	-142.204553978085	-142.204553978085\\
65.375	0.25038	-147.945130085685	-147.945130085685\\
65.375	0.25404	-153.807937857097	-153.807937857097\\
65.375	0.2577	-159.79297729232	-159.79297729232\\
65.375	0.26136	-165.900248391353	-165.900248391353\\
65.375	0.26502	-172.129751154197	-172.129751154197\\
65.375	0.26868	-178.481485580851	-178.481485580851\\
65.375	0.27234	-184.955451671317	-184.955451671317\\
65.375	0.276	-191.551649425593	-191.551649425593\\
65.75	0.093	-11.6231261990656	-11.6231261990656\\
65.75	0.09666	-12.232751262125	-12.232751262125\\
65.75	0.10032	-12.9646079889952	-12.9646079889952\\
65.75	0.10398	-13.8186963796761	-13.8186963796761\\
65.75	0.10764	-14.7950164341677	-14.7950164341677\\
65.75	0.1113	-15.8935681524702	-15.8935681524702\\
65.75	0.11496	-17.1143515345834	-17.1143515345834\\
65.75	0.11862	-18.4573665805073	-18.4573665805073\\
65.75	0.12228	-19.922613290242	-19.922613290242\\
65.75	0.12594	-21.5100916637875	-21.5100916637875\\
65.75	0.1296	-23.2198017011437	-23.2198017011437\\
65.75	0.13326	-25.0517434023107	-25.0517434023107\\
65.75	0.13692	-27.0059167672884	-27.0059167672884\\
65.75	0.14058	-29.0823217960769	-29.0823217960769\\
65.75	0.14424	-31.2809584886762	-31.2809584886762\\
65.75	0.1479	-33.6018268450862	-33.6018268450862\\
65.75	0.15156	-36.0449268653069	-36.0449268653069\\
65.75	0.15522	-38.6102585493385	-38.6102585493385\\
65.75	0.15888	-41.2978218971808	-41.2978218971808\\
65.75	0.16254	-44.1076169088338	-44.1076169088338\\
65.75	0.1662	-47.0396435842976	-47.0396435842976\\
65.75	0.16986	-50.0939019235722	-50.0939019235722\\
65.75	0.17352	-53.2703919266575	-53.2703919266575\\
65.75	0.17718	-56.5691135935536	-56.5691135935536\\
65.75	0.18084	-59.9900669242604	-59.9900669242604\\
65.75	0.1845	-63.533251918778	-63.533251918778\\
65.75	0.18816	-67.1986685771064	-67.1986685771064\\
65.75	0.19182	-70.9863168992455	-70.9863168992455\\
65.75	0.19548	-74.8961968851954	-74.8961968851954\\
65.75	0.19914	-78.928308534956	-78.928308534956\\
65.75	0.2028	-83.0826518485274	-83.0826518485274\\
65.75	0.20646	-87.3592268259096	-87.3592268259096\\
65.75	0.21012	-91.7580334671025	-91.7580334671025\\
65.75	0.21378	-96.2790717721061	-96.2790717721061\\
65.75	0.21744	-100.922341740921	-100.922341740921\\
65.75	0.2211	-105.687843373546	-105.687843373546\\
65.75	0.22476	-110.575576669982	-110.575576669982\\
65.75	0.22842	-115.585541630228	-115.585541630228\\
65.75	0.23208	-120.717738254286	-120.717738254286\\
65.75	0.23574	-125.972166542154	-125.972166542154\\
65.75	0.2394	-131.348826493833	-131.348826493833\\
65.75	0.24306	-136.847718109323	-136.847718109323\\
65.75	0.24672	-142.468841388623	-142.468841388623\\
65.75	0.25038	-148.212196331735	-148.212196331735\\
65.75	0.25404	-154.077782938657	-154.077782938657\\
65.75	0.2577	-160.065601209389	-160.065601209389\\
65.75	0.26136	-166.175651143933	-166.175651143933\\
65.75	0.26502	-172.407932742287	-172.407932742287\\
65.75	0.26868	-178.762446004452	-178.762446004452\\
65.75	0.27234	-185.239190930428	-185.239190930428\\
65.75	0.276	-191.838167520215	-191.838167520215\\
66.125	0.093	-11.7802988907242	-11.7802988907242\\
66.125	0.09666	-12.3927027892939	-12.3927027892939\\
66.125	0.10032	-13.1273383516745	-13.1273383516745\\
66.125	0.10398	-13.9842055778657	-13.9842055778657\\
66.125	0.10764	-14.9633044678677	-14.9633044678677\\
66.125	0.1113	-16.0646350216805	-16.0646350216805\\
66.125	0.11496	-17.288197239304	-17.288197239304\\
66.125	0.11862	-18.6339911207383	-18.6339911207383\\
66.125	0.12228	-20.1020166659834	-20.1020166659834\\
66.125	0.12594	-21.6922738750392	-21.6922738750392\\
66.125	0.1296	-23.4047627479058	-23.4047627479058\\
66.125	0.13326	-25.2394832845831	-25.2394832845831\\
66.125	0.13692	-27.1964354850712	-27.1964354850712\\
66.125	0.14058	-29.27561934937	-29.27561934937\\
66.125	0.14424	-31.4770348774796	-31.4770348774796\\
66.125	0.1479	-33.8006820694	-33.8006820694\\
66.125	0.15156	-36.2465609251311	-36.2465609251311\\
66.125	0.15522	-38.814671444673	-38.814671444673\\
66.125	0.15888	-41.5050136280256	-41.5050136280256\\
66.125	0.16254	-44.317587475189	-44.317587475189\\
66.125	0.1662	-47.2523929861632	-47.2523929861632\\
66.125	0.16986	-50.3094301609481	-50.3094301609481\\
66.125	0.17352	-53.4886989995437	-53.4886989995437\\
66.125	0.17718	-56.7901995019502	-56.7901995019502\\
66.125	0.18084	-60.2139316681673	-60.2139316681673\\
66.125	0.1845	-63.7598954981953	-63.7598954981953\\
66.125	0.18816	-67.428090992034	-67.428090992034\\
66.125	0.19182	-71.2185181496835	-71.2185181496835\\
66.125	0.19548	-75.1311769711437	-75.1311769711437\\
66.125	0.19914	-79.1660674564147	-79.1660674564147\\
66.125	0.2028	-83.3231896054965	-83.3231896054965\\
66.125	0.20646	-87.602543418389	-87.602543418389\\
66.125	0.21012	-92.0041288950922	-92.0041288950922\\
66.125	0.21378	-96.5279460356062	-96.5279460356062\\
66.125	0.21744	-101.173994839931	-101.173994839931\\
66.125	0.2211	-105.942275308067	-105.942275308067\\
66.125	0.22476	-110.832787440013	-110.832787440013\\
66.125	0.22842	-115.84553123577	-115.84553123577\\
66.125	0.23208	-120.980506695338	-120.980506695338\\
66.125	0.23574	-126.237713818716	-126.237713818716\\
66.125	0.2394	-131.617152605906	-131.617152605906\\
66.125	0.24306	-137.118823056906	-137.118823056906\\
66.125	0.24672	-142.742725171716	-142.742725171716\\
66.125	0.25038	-148.488858950338	-148.488858950338\\
66.125	0.25404	-154.35722439277	-154.35722439277\\
66.125	0.2577	-160.347821499014	-160.347821499014\\
66.125	0.26136	-166.460650269067	-166.460650269067\\
66.125	0.26502	-172.695710702932	-172.695710702932\\
66.125	0.26868	-179.053002800607	-179.053002800607\\
66.125	0.27234	-185.532526562094	-185.532526562094\\
66.125	0.276	-192.13428198739	-192.13428198739\\
66.5	0.093	-11.9470679549371	-11.9470679549371\\
66.5	0.09666	-12.5622506890172	-12.5622506890172\\
66.5	0.10032	-13.299665086908	-13.299665086908\\
66.5	0.10398	-14.1593111486096	-14.1593111486096\\
66.5	0.10764	-15.141188874122	-15.141188874122\\
66.5	0.1113	-16.2452982634452	-16.2452982634452\\
66.5	0.11496	-17.4716393165791	-17.4716393165791\\
66.5	0.11862	-18.8202120335237	-18.8202120335237\\
66.5	0.12228	-20.2910164142791	-20.2910164142791\\
66.5	0.12594	-21.8840524588452	-21.8840524588452\\
66.5	0.1296	-23.5993201672221	-23.5993201672221\\
66.5	0.13326	-25.4368195394098	-25.4368195394098\\
66.5	0.13692	-27.3965505754082	-27.3965505754082\\
66.5	0.14058	-29.4785132752174	-29.4785132752174\\
66.5	0.14424	-31.6827076388374	-31.6827076388374\\
66.5	0.1479	-34.0091336662681	-34.0091336662681\\
66.5	0.15156	-36.4577913575096	-36.4577913575096\\
66.5	0.15522	-39.0286807125618	-39.0286807125618\\
66.5	0.15888	-41.7218017314248	-41.7218017314248\\
66.5	0.16254	-44.5371544140985	-44.5371544140985\\
66.5	0.1662	-47.474738760583	-47.474738760583\\
66.5	0.16986	-50.5345547708783	-50.5345547708783\\
66.5	0.17352	-53.7166024449843	-53.7166024449843\\
66.5	0.17718	-57.0208817829011	-57.0208817829011\\
66.5	0.18084	-60.4473927846286	-60.4473927846286\\
66.5	0.1845	-63.9961354501669	-63.9961354501669\\
66.5	0.18816	-67.6671097795159	-67.6671097795159\\
66.5	0.19182	-71.4603157726757	-71.4603157726757\\
66.5	0.19548	-75.3757534296463	-75.3757534296463\\
66.5	0.19914	-79.4134227504276	-79.4134227504276\\
66.5	0.2028	-83.5733237350198	-83.5733237350198\\
66.5	0.20646	-87.8554563834226	-87.8554563834226\\
66.5	0.21012	-92.2598206956362	-92.2598206956362\\
66.5	0.21378	-96.7864166716606	-96.7864166716606\\
66.5	0.21744	-101.435244311496	-101.435244311496\\
66.5	0.2211	-106.206303615142	-106.206303615142\\
66.5	0.22476	-111.099594582598	-111.099594582598\\
66.5	0.22842	-116.115117213866	-116.115117213866\\
66.5	0.23208	-121.252871508944	-121.252871508944\\
66.5	0.23574	-126.512857467833	-126.512857467833\\
66.5	0.2394	-131.895075090532	-131.895075090532\\
66.5	0.24306	-137.399524377043	-137.399524377043\\
66.5	0.24672	-143.026205327364	-143.026205327364\\
66.5	0.25038	-148.775117941496	-148.775117941496\\
66.5	0.25404	-154.646262219439	-154.646262219439\\
66.5	0.2577	-160.639638161192	-160.639638161192\\
66.5	0.26136	-166.755245766756	-166.755245766756\\
66.5	0.26502	-172.993085036131	-172.993085036131\\
66.5	0.26868	-179.353155969317	-179.353155969317\\
66.5	0.27234	-185.835458566314	-185.835458566314\\
66.5	0.276	-192.439992827121	-192.439992827121\\
66.875	0.093	-12.1234333917045	-12.1234333917045\\
66.875	0.09666	-12.7413949612949	-12.7413949612949\\
66.875	0.10032	-13.4815881946961	-13.4815881946961\\
66.875	0.10398	-14.344013091908	-14.344013091908\\
66.875	0.10764	-15.3286696529308	-15.3286696529308\\
66.875	0.1113	-16.4355578777643	-16.4355578777643\\
66.875	0.11496	-17.6646777664085	-17.6646777664085\\
66.875	0.11862	-19.0160293188634	-19.0160293188634\\
66.875	0.12228	-20.4896125351292	-20.4896125351292\\
66.875	0.12594	-22.0854274152057	-22.0854274152057\\
66.875	0.1296	-23.803473959093	-23.803473959093\\
66.875	0.13326	-25.643752166791	-25.643752166791\\
66.875	0.13692	-27.6062620382997	-27.6062620382997\\
66.875	0.14058	-29.6910035736193	-29.6910035736193\\
66.875	0.14424	-31.8979767727496	-31.8979767727496\\
66.875	0.1479	-34.2271816356907	-34.2271816356907\\
66.875	0.15156	-36.6786181624425	-36.6786181624425\\
66.875	0.15522	-39.2522863530051	-39.2522863530051\\
66.875	0.15888	-41.9481862073784	-41.9481862073784\\
66.875	0.16254	-44.7663177255625	-44.7663177255625\\
66.875	0.1662	-47.7066809075574	-47.7066809075574\\
66.875	0.16986	-50.7692757533629	-50.7692757533629\\
66.875	0.17352	-53.9541022629793	-53.9541022629793\\
66.875	0.17718	-57.2611604364064	-57.2611604364064\\
66.875	0.18084	-60.6904502736443	-60.6904502736443\\
66.875	0.1845	-64.241971774693	-64.241971774693\\
66.875	0.18816	-67.9157249395524	-67.9157249395524\\
66.875	0.19182	-71.7117097682225	-71.7117097682225\\
66.875	0.19548	-75.6299262607035	-75.6299262607035\\
66.875	0.19914	-79.6703744169951	-79.6703744169951\\
66.875	0.2028	-83.8330542370976	-83.8330542370976\\
66.875	0.20646	-88.1179657210108	-88.1179657210108\\
66.875	0.21012	-92.5251088687347	-92.5251088687347\\
66.875	0.21378	-97.0544836802694	-97.0544836802694\\
66.875	0.21744	-101.706090155615	-101.706090155615\\
66.875	0.2211	-106.479928294771	-106.479928294771\\
66.875	0.22476	-111.375998097738	-111.375998097738\\
66.875	0.22842	-116.394299564516	-116.394299564516\\
66.875	0.23208	-121.534832695104	-121.534832695104\\
66.875	0.23574	-126.797597489504	-126.797597489504\\
66.875	0.2394	-132.182593947714	-132.182593947714\\
66.875	0.24306	-137.689822069734	-137.689822069734\\
66.875	0.24672	-143.319281855566	-143.319281855566\\
66.875	0.25038	-149.070973305208	-149.070973305208\\
66.875	0.25404	-154.944896418661	-154.944896418661\\
66.875	0.2577	-160.941051195925	-160.941051195925\\
66.875	0.26136	-167.059437637	-167.059437637\\
66.875	0.26502	-173.300055741885	-173.300055741885\\
66.875	0.26868	-179.662905510581	-179.662905510581\\
66.875	0.27234	-186.147986943088	-186.147986943088\\
66.875	0.276	-192.755300039406	-192.755300039406\\
67.25	0.093	-12.3093952010262	-12.3093952010262\\
67.25	0.09666	-12.930135606127	-12.930135606127\\
67.25	0.10032	-13.6731076750386	-13.6731076750386\\
67.25	0.10398	-14.5383114077609	-14.5383114077609\\
67.25	0.10764	-15.5257468042939	-15.5257468042939\\
67.25	0.1113	-16.6354138646378	-16.6354138646378\\
67.25	0.11496	-17.8673125887923	-17.8673125887923\\
67.25	0.11862	-19.2214429767577	-19.2214429767577\\
67.25	0.12228	-20.6978050285338	-20.6978050285338\\
67.25	0.12594	-22.2963987441206	-22.2963987441206\\
67.25	0.1296	-24.0172241235182	-24.0172241235182\\
67.25	0.13326	-25.8602811667266	-25.8602811667266\\
67.25	0.13692	-27.8255698737457	-27.8255698737457\\
67.25	0.14058	-29.9130902445756	-29.9130902445756\\
67.25	0.14424	-32.1228422792163	-32.1228422792163\\
67.25	0.1479	-34.4548259776677	-34.4548259776677\\
67.25	0.15156	-36.9090413399298	-36.9090413399298\\
67.25	0.15522	-39.4854883660027	-39.4854883660027\\
67.25	0.15888	-42.1841670558864	-42.1841670558864\\
67.25	0.16254	-45.0050774095809	-45.0050774095809\\
67.25	0.1662	-47.9482194270861	-47.9482194270861\\
67.25	0.16986	-51.0135931084021	-51.0135931084021\\
67.25	0.17352	-54.2011984535287	-54.2011984535287\\
67.25	0.17718	-57.5110354624662	-57.5110354624662\\
67.25	0.18084	-60.9431041352144	-60.9431041352144\\
67.25	0.1845	-64.4974044717735	-64.4974044717735\\
67.25	0.18816	-68.1739364721431	-68.1739364721431\\
67.25	0.19182	-71.9727001363237	-71.9727001363237\\
67.25	0.19548	-75.8936954643149	-75.8936954643149\\
67.25	0.19914	-79.9369224561169	-79.9369224561169\\
67.25	0.2028	-84.1023811117298	-84.1023811117298\\
67.25	0.20646	-88.3900714311533	-88.3900714311533\\
67.25	0.21012	-92.7999934143876	-92.7999934143876\\
67.25	0.21378	-97.3321470614327	-97.3321470614327\\
67.25	0.21744	-101.986532372288	-101.986532372288\\
67.25	0.2211	-106.763149346955	-106.763149346955\\
67.25	0.22476	-111.661997985432	-111.661997985432\\
67.25	0.22842	-116.68307828772	-116.68307828772\\
67.25	0.23208	-121.826390253819	-121.826390253819\\
67.25	0.23574	-127.091933883729	-127.091933883729\\
67.25	0.2394	-132.479709177449	-132.479709177449\\
67.25	0.24306	-137.98971613498	-137.98971613498\\
67.25	0.24672	-143.621954756322	-143.621954756322\\
67.25	0.25038	-149.376425041475	-149.376425041475\\
67.25	0.25404	-155.253126990438	-155.253126990438\\
67.25	0.2577	-161.252060603213	-161.252060603213\\
67.25	0.26136	-167.373225879797	-167.373225879797\\
67.25	0.26502	-173.616622820193	-173.616622820193\\
67.25	0.26868	-179.9822514244	-179.9822514244\\
67.25	0.27234	-186.470111692417	-186.470111692417\\
67.25	0.276	-193.080203624245	-193.080203624245\\
67.625	0.093	-12.5049533829023	-12.5049533829023\\
67.625	0.09666	-13.1284726235134	-13.1284726235134\\
67.625	0.10032	-13.8742235279354	-13.8742235279354\\
67.625	0.10398	-14.742206096168	-14.742206096168\\
67.625	0.10764	-15.7324203282114	-15.7324203282114\\
67.625	0.1113	-16.8448662240656	-16.8448662240656\\
67.625	0.11496	-18.0795437837305	-18.0795437837305\\
67.625	0.11862	-19.4364530072062	-19.4364530072062\\
67.625	0.12228	-20.9155938944926	-20.9155938944926\\
67.625	0.12594	-22.5169664455898	-22.5169664455898\\
67.625	0.1296	-24.2405706604978	-24.2405706604978\\
67.625	0.13326	-26.0864065392165	-26.0864065392165\\
67.625	0.13692	-28.054474081746	-28.054474081746\\
67.625	0.14058	-30.1447732880862	-30.1447732880862\\
67.625	0.14424	-32.3573041582372	-32.3573041582372\\
67.625	0.1479	-34.692066692199	-34.692066692199\\
67.625	0.15156	-37.1490608899715	-37.1490608899715\\
67.625	0.15522	-39.7282867515548	-39.7282867515548\\
67.625	0.15888	-42.4297442769488	-42.4297442769488\\
67.625	0.16254	-45.2534334661536	-45.2534334661536\\
67.625	0.1662	-48.1993543191692	-48.1993543191692\\
67.625	0.16986	-51.2675068359955	-51.2675068359955\\
67.625	0.17352	-54.4578910166325	-54.4578910166325\\
67.625	0.17718	-57.7705068610803	-57.7705068610803\\
67.625	0.18084	-61.2053543693389	-61.2053543693389\\
67.625	0.1845	-64.7624335414082	-64.7624335414082\\
67.625	0.18816	-68.4417443772883	-68.4417443772883\\
67.625	0.19182	-72.2432868769791	-72.2432868769791\\
67.625	0.19548	-76.1670610404807	-76.1670610404807\\
67.625	0.19914	-80.2130668677931	-80.2130668677931\\
67.625	0.2028	-84.3813043589163	-84.3813043589163\\
67.625	0.20646	-88.6717735138502	-88.6717735138502\\
67.625	0.21012	-93.0844743325949	-93.0844743325949\\
67.625	0.21378	-97.6194068151503	-97.6194068151503\\
67.625	0.21744	-102.276570961516	-102.276570961516\\
67.625	0.2211	-107.055966771693	-107.055966771693\\
67.625	0.22476	-111.957594245681	-111.957594245681\\
67.625	0.22842	-116.981453383479	-116.981453383479\\
67.625	0.23208	-122.127544185089	-122.127544185089\\
67.625	0.23574	-127.395866650509	-127.395866650509\\
67.625	0.2394	-132.786420779739	-132.786420779739\\
67.625	0.24306	-138.299206572781	-138.299206572781\\
67.625	0.24672	-143.934224029633	-143.934224029633\\
67.625	0.25038	-149.691473150296	-149.691473150296\\
67.625	0.25404	-155.57095393477	-155.57095393477\\
67.625	0.2577	-161.572666383054	-161.572666383054\\
67.625	0.26136	-167.69661049515	-167.69661049515\\
67.625	0.26502	-173.942786271056	-173.942786271056\\
67.625	0.26868	-180.311193710772	-180.311193710772\\
67.625	0.27234	-186.8018328143	-186.8018328143\\
67.625	0.276	-193.414703581638	-193.414703581638\\
68	0.093	-12.7101079373329	-12.7101079373329\\
68	0.09666	-13.3364060134543	-13.3364060134543\\
68	0.10032	-14.0849357533866	-14.0849357533866\\
68	0.10398	-14.9556971571296	-14.9556971571296\\
68	0.10764	-15.9486902246833	-15.9486902246833\\
68	0.1113	-17.0639149560479	-17.0639149560479\\
68	0.11496	-18.3013713512232	-18.3013713512232\\
68	0.11862	-19.6610594102092	-19.6610594102092\\
68	0.12228	-21.142979133006	-21.142979133006\\
68	0.12594	-22.7471305196135	-22.7471305196135\\
68	0.1296	-24.4735135700318	-24.4735135700318\\
68	0.13326	-26.3221282842609	-26.3221282842609\\
68	0.13692	-28.2929746623007	-28.2929746623007\\
68	0.14058	-30.3860527041513	-30.3860527041513\\
68	0.14424	-32.6013624098126	-32.6013624098126\\
68	0.1479	-34.9389037792847	-34.9389037792847\\
68	0.15156	-37.3986768125676	-37.3986768125676\\
68	0.15522	-39.9806815096612	-39.9806815096612\\
68	0.15888	-42.6849178705656	-42.6849178705656\\
68	0.16254	-45.5113858952807	-45.5113858952807\\
68	0.1662	-48.4600855838066	-48.4600855838066\\
68	0.16986	-51.5310169361433	-51.5310169361433\\
68	0.17352	-54.7241799522907	-54.7241799522907\\
68	0.17718	-58.0395746322488	-58.0395746322488\\
68	0.18084	-61.4772009760178	-61.4772009760178\\
68	0.1845	-65.0370589835975	-65.0370589835975\\
68	0.18816	-68.7191486549879	-68.7191486549879\\
68	0.19182	-72.5234699901891	-72.5234699901891\\
68	0.19548	-76.4500229892011	-76.4500229892011\\
68	0.19914	-80.4988076520238	-80.4988076520238\\
68	0.2028	-84.6698239786573	-84.6698239786573\\
68	0.20646	-88.9630719691015	-88.9630719691015\\
68	0.21012	-93.3785516233566	-93.3785516233566\\
68	0.21378	-97.9162629414223	-97.9162629414223\\
68	0.21744	-102.576205923299	-102.576205923299\\
68	0.2211	-107.358380568986	-107.358380568986\\
68	0.22476	-112.262786878484	-112.262786878484\\
68	0.22842	-117.289424851793	-117.289424851793\\
68	0.23208	-122.438294488912	-122.438294488912\\
68	0.23574	-127.709395789843	-127.709395789843\\
68	0.2394	-133.102728754584	-133.102728754584\\
68	0.24306	-138.618293383136	-138.618293383136\\
68	0.24672	-144.256089675498	-144.256089675498\\
68	0.25038	-150.016117631672	-150.016117631672\\
68	0.25404	-155.898377251656	-155.898377251656\\
68	0.2577	-161.902868535451	-161.902868535451\\
68	0.26136	-168.029591483056	-168.029591483056\\
68	0.26502	-174.278546094473	-174.278546094473\\
68	0.26868	-180.6497323697	-180.6497323697\\
68	0.27234	-187.143150308738	-187.143150308738\\
68	0.276	-193.758799911586	-193.758799911586\\
68.375	0.093	-12.9248588643178	-12.9248588643178\\
68.375	0.09666	-13.5539357759497	-13.5539357759497\\
68.375	0.10032	-14.3052443513923	-14.3052443513923\\
68.375	0.10398	-15.1787845906456	-15.1787845906456\\
68.375	0.10764	-16.1745564937097	-16.1745564937097\\
68.375	0.1113	-17.2925600605846	-17.2925600605846\\
68.375	0.11496	-18.5327952912702	-18.5327952912702\\
68.375	0.11862	-19.8952621857666	-19.8952621857666\\
68.375	0.12228	-21.3799607440737	-21.3799607440737\\
68.375	0.12594	-22.9868909661916	-22.9868909661916\\
68.375	0.1296	-24.7160528521203	-24.7160528521203\\
68.375	0.13326	-26.5674464018597	-26.5674464018597\\
68.375	0.13692	-28.5410716154098	-28.5410716154098\\
68.375	0.14058	-30.6369284927708	-30.6369284927708\\
68.375	0.14424	-32.8550170339425	-32.8550170339425\\
68.375	0.1479	-35.1953372389249	-35.1953372389249\\
68.375	0.15156	-37.6578891077181	-37.6578891077181\\
68.375	0.15522	-40.2426726403221	-40.2426726403221\\
68.375	0.15888	-42.9496878367369	-42.9496878367369\\
68.375	0.16254	-45.7789346969624	-45.7789346969624\\
68.375	0.1662	-48.7304132209986	-48.7304132209986\\
68.375	0.16986	-51.8041234088456	-51.8041234088456\\
68.375	0.17352	-55.0000652605033	-55.0000652605033\\
68.375	0.17718	-58.3182387759718	-58.3182387759718\\
68.375	0.18084	-61.7586439552511	-61.7586439552511\\
68.375	0.1845	-65.3212807983411	-65.3212807983411\\
68.375	0.18816	-69.0061493052419	-69.0061493052419\\
68.375	0.19182	-72.8132494759535	-72.8132494759535\\
68.375	0.19548	-76.7425813104758	-76.7425813104758\\
68.375	0.19914	-80.7941448088088	-80.7941448088088\\
68.375	0.2028	-84.9679399709527	-84.9679399709527\\
68.375	0.20646	-89.2639667969073	-89.2639667969073\\
68.375	0.21012	-93.6822252866727	-93.6822252866727\\
68.375	0.21378	-98.2227154402487	-98.2227154402487\\
68.375	0.21744	-102.885437257636	-102.885437257636\\
68.375	0.2211	-107.670390738833	-107.670390738833\\
68.375	0.22476	-112.577575883842	-112.577575883842\\
68.375	0.22842	-117.606992692661	-117.606992692661\\
68.375	0.23208	-122.758641165291	-122.758641165291\\
68.375	0.23574	-128.032521301731	-128.032521301731\\
68.375	0.2394	-133.428633101983	-133.428633101983\\
68.375	0.24306	-138.946976566045	-138.946976566045\\
68.375	0.24672	-144.587551693918	-144.587551693918\\
68.375	0.25038	-150.350358485602	-150.350358485602\\
68.375	0.25404	-156.235396941096	-156.235396941096\\
68.375	0.2577	-162.242667060401	-162.242667060401\\
68.375	0.26136	-168.372168843517	-168.372168843517\\
68.375	0.26502	-174.623902290444	-174.623902290444\\
68.375	0.26868	-180.997867401181	-180.997867401181\\
68.375	0.27234	-187.49406417573	-187.49406417573\\
68.375	0.276	-194.112492614088	-194.112492614088\\
68.75	0.093	-13.1492061638571	-13.1492061638571\\
68.75	0.09666	-13.7810619109993	-13.7810619109993\\
68.75	0.10032	-14.5351493219523	-14.5351493219523\\
68.75	0.10398	-15.4114683967159	-15.4114683967159\\
68.75	0.10764	-16.4100191352904	-16.4100191352904\\
68.75	0.1113	-17.5308015376756	-17.5308015376756\\
68.75	0.11496	-18.7738156038716	-18.7738156038716\\
68.75	0.11862	-20.1390613338783	-20.1390613338783\\
68.75	0.12228	-21.6265387276958	-21.6265387276958\\
68.75	0.12594	-23.236247785324	-23.236247785324\\
68.75	0.1296	-24.968188506763	-24.968188506763\\
68.75	0.13326	-26.8223608920128	-26.8223608920128\\
68.75	0.13692	-28.7987649410733	-28.7987649410733\\
68.75	0.14058	-30.8974006539446	-30.8974006539446\\
68.75	0.14424	-33.1182680306267	-33.1182680306267\\
68.75	0.1479	-35.4613670711194	-35.4613670711194\\
68.75	0.15156	-37.926697775423	-37.926697775423\\
68.75	0.15522	-40.5142601435373	-40.5142601435373\\
68.75	0.15888	-43.2240541754624	-43.2240541754624\\
68.75	0.16254	-46.0560798711982	-46.0560798711982\\
68.75	0.1662	-49.0103372307448	-49.0103372307448\\
68.75	0.16986	-52.0868262541022	-52.0868262541022\\
68.75	0.17352	-55.2855469412702	-55.2855469412702\\
68.75	0.17718	-58.6064992922491	-58.6064992922491\\
68.75	0.18084	-62.0496833070387	-62.0496833070387\\
68.75	0.1845	-65.6150989856391	-65.6150989856391\\
68.75	0.18816	-69.3027463280502	-69.3027463280502\\
68.75	0.19182	-73.1126253342722	-73.1126253342722\\
68.75	0.19548	-77.0447360043048	-77.0447360043048\\
68.75	0.19914	-81.0990783381482	-81.0990783381482\\
68.75	0.2028	-85.2756523358024	-85.2756523358024\\
68.75	0.20646	-89.5744579972674	-89.5744579972674\\
68.75	0.21012	-93.9954953225431	-93.9954953225431\\
68.75	0.21378	-98.5387643116295	-98.5387643116295\\
68.75	0.21744	-103.204264964527	-103.204264964527\\
68.75	0.2211	-107.991997281235	-107.991997281235\\
68.75	0.22476	-112.901961261753	-112.901961261753\\
68.75	0.22842	-117.934156906083	-117.934156906083\\
68.75	0.23208	-123.088584214223	-123.088584214223\\
68.75	0.23574	-128.365243186174	-128.365243186174\\
68.75	0.2394	-133.764133821936	-133.764133821936\\
68.75	0.24306	-139.285256121508	-139.285256121508\\
68.75	0.24672	-144.928610084892	-144.928610084892\\
68.75	0.25038	-150.694195712086	-150.694195712086\\
68.75	0.25404	-156.582013003091	-156.582013003091\\
68.75	0.2577	-162.592061957906	-162.592061957906\\
68.75	0.26136	-168.724342576532	-168.724342576532\\
68.75	0.26502	-174.978854858969	-174.978854858969\\
68.75	0.26868	-181.355598805217	-181.355598805217\\
68.75	0.27234	-187.854574415276	-187.854574415276\\
68.75	0.276	-194.475781689145	-194.475781689145\\
69.125	0.093	-13.3831498359509	-13.3831498359509\\
69.125	0.09666	-14.0177844186034	-14.0177844186034\\
69.125	0.10032	-14.7746506650667	-14.7746506650667\\
69.125	0.10398	-15.6537485753407	-15.6537485753407\\
69.125	0.10764	-16.6550781494255	-16.6550781494255\\
69.125	0.1113	-17.7786393873211	-17.7786393873211\\
69.125	0.11496	-19.0244322890274	-19.0244322890274\\
69.125	0.11862	-20.3924568545445	-20.3924568545445\\
69.125	0.12228	-21.8827130838723	-21.8827130838723\\
69.125	0.12594	-23.4952009770109	-23.4952009770109\\
69.125	0.1296	-25.2299205339603	-25.2299205339603\\
69.125	0.13326	-27.0868717547204	-27.0868717547204\\
69.125	0.13692	-29.0660546392912	-29.0660546392912\\
69.125	0.14058	-31.1674691876729	-31.1674691876729\\
69.125	0.14424	-33.3911153998653	-33.3911153998653\\
69.125	0.1479	-35.7369932758684	-35.7369932758684\\
69.125	0.15156	-38.2051028156823	-38.2051028156823\\
69.125	0.15522	-40.795444019307	-40.795444019307\\
69.125	0.15888	-43.5080168867424	-43.5080168867424\\
69.125	0.16254	-46.3428214179885	-46.3428214179885\\
69.125	0.1662	-49.2998576130455	-49.2998576130455\\
69.125	0.16986	-52.3791254719132	-52.3791254719132\\
69.125	0.17352	-55.5806249945916	-55.5806249945916\\
69.125	0.17718	-58.9043561810809	-58.9043561810809\\
69.125	0.18084	-62.3503190313808	-62.3503190313808\\
69.125	0.1845	-65.9185135454916	-65.9185135454916\\
69.125	0.18816	-69.6089397234131	-69.6089397234131\\
69.125	0.19182	-73.4215975651454	-73.4215975651454\\
69.125	0.19548	-77.3564870706884	-77.3564870706884\\
69.125	0.19914	-81.4136082400421	-81.4136082400421\\
69.125	0.2028	-85.5929610732066	-85.5929610732066\\
69.125	0.20646	-89.8945455701819	-89.8945455701819\\
69.125	0.21012	-94.3183617309679	-94.3183617309679\\
69.125	0.21378	-98.8644095555647	-98.8644095555647\\
69.125	0.21744	-103.532689043972	-103.532689043972\\
69.125	0.2211	-108.323200196191	-108.323200196191\\
69.125	0.22476	-113.23594301222	-113.23594301222\\
69.125	0.22842	-118.270917492059	-118.270917492059\\
69.125	0.23208	-123.42812363571	-123.42812363571\\
69.125	0.23574	-128.707561443171	-128.707561443171\\
69.125	0.2394	-134.109230914444	-134.109230914444\\
69.125	0.24306	-139.633132049526	-139.633132049526\\
69.125	0.24672	-145.27926484842	-145.27926484842\\
69.125	0.25038	-151.047629311125	-151.047629311125\\
69.125	0.25404	-156.93822543764	-156.93822543764\\
69.125	0.2577	-162.951053227966	-162.951053227966\\
69.125	0.26136	-169.086112682102	-169.086112682102\\
69.125	0.26502	-175.34340380005	-175.34340380005\\
69.125	0.26868	-181.722926581808	-181.722926581808\\
69.125	0.27234	-188.224681027377	-188.224681027377\\
69.125	0.276	-194.848667136756	-194.848667136756\\
69.5	0.093	-13.626689880599	-13.626689880599\\
69.5	0.09666	-14.2641032987619	-14.2641032987619\\
69.5	0.10032	-15.0237483807355	-15.0237483807355\\
69.5	0.10398	-15.9056251265199	-15.9056251265199\\
69.5	0.10764	-16.9097335361151	-16.9097335361151\\
69.5	0.1113	-18.036073609521	-18.036073609521\\
69.5	0.11496	-19.2846453467376	-19.2846453467376\\
69.5	0.11862	-20.6554487477651	-20.6554487477651\\
69.5	0.12228	-22.1484838126032	-22.1484838126032\\
69.5	0.12594	-23.7637505412522	-23.7637505412522\\
69.5	0.1296	-25.5012489337119	-25.5012489337119\\
69.5	0.13326	-27.3609789899823	-27.3609789899823\\
69.5	0.13692	-29.3429407100635	-29.3429407100635\\
69.5	0.14058	-31.4471340939555	-31.4471340939555\\
69.5	0.14424	-33.6735591416583	-33.6735591416583\\
69.5	0.1479	-36.0222158531718	-36.0222158531718\\
69.5	0.15156	-38.493104228496	-38.493104228496\\
69.5	0.15522	-41.086224267631	-41.086224267631\\
69.5	0.15888	-43.8015759705768	-43.8015759705768\\
69.5	0.16254	-46.6391593373334	-46.6391593373334\\
69.5	0.1662	-49.5989743679007	-49.5989743679007\\
69.5	0.16986	-52.6810210622787	-52.6810210622787\\
69.5	0.17352	-55.8852994204674	-55.8852994204674\\
69.5	0.17718	-59.211809442467	-59.211809442467\\
69.5	0.18084	-62.6605511282773	-62.6605511282773\\
69.5	0.1845	-66.2315244778984	-66.2315244778984\\
69.5	0.18816	-69.9247294913302	-69.9247294913302\\
69.5	0.19182	-73.7401661685728	-73.7401661685728\\
69.5	0.19548	-77.6778345096261	-77.6778345096261\\
69.5	0.19914	-81.7377345144903	-81.7377345144903\\
69.5	0.2028	-85.9198661831652	-85.9198661831652\\
69.5	0.20646	-90.2242295156508	-90.2242295156508\\
69.5	0.21012	-94.6508245119472	-94.6508245119472\\
69.5	0.21378	-99.1996511720544	-99.1996511720544\\
69.5	0.21744	-103.870709495972	-103.870709495972\\
69.5	0.2211	-108.663999483701	-108.663999483701\\
69.5	0.22476	-113.57952113524	-113.57952113524\\
69.5	0.22842	-118.617274450591	-118.617274450591\\
69.5	0.23208	-123.777259429751	-123.777259429751\\
69.5	0.23574	-129.059476072723	-129.059476072723\\
69.5	0.2394	-134.463924379506	-134.463924379506\\
69.5	0.24306	-139.990604350099	-139.990604350099\\
69.5	0.24672	-145.639515984503	-145.639515984503\\
69.5	0.25038	-151.410659282718	-151.410659282718\\
69.5	0.25404	-157.304034244743	-157.304034244743\\
69.5	0.2577	-163.319640870579	-163.319640870579\\
69.5	0.26136	-169.457479160226	-169.457479160226\\
69.5	0.26502	-175.717549113684	-175.717549113684\\
69.5	0.26868	-182.099850730953	-182.099850730953\\
69.5	0.27234	-188.604384012032	-188.604384012032\\
69.5	0.276	-195.231148956922	-195.231148956922\\
69.875	0.093	-13.8798262978015	-13.8798262978015\\
69.875	0.09666	-14.5200185514747	-14.5200185514747\\
69.875	0.10032	-15.2824424689587	-15.2824424689587\\
69.875	0.10398	-16.1670980502534	-16.1670980502534\\
69.875	0.10764	-17.1739852953589	-17.1739852953589\\
69.875	0.1113	-18.3031042042752	-18.3031042042752\\
69.875	0.11496	-19.5544547770022	-19.5544547770022\\
69.875	0.11862	-20.92803701354	-20.92803701354\\
69.875	0.12228	-22.4238509138885	-22.4238509138885\\
69.875	0.12594	-24.0418964780478	-24.0418964780478\\
69.875	0.1296	-25.7821737060178	-25.7821737060178\\
69.875	0.13326	-27.6446825977986	-27.6446825977986\\
69.875	0.13692	-29.6294231533902	-29.6294231533902\\
69.875	0.14058	-31.7363953727925	-31.7363953727925\\
69.875	0.14424	-33.9655992560056	-33.9655992560056\\
69.875	0.1479	-36.3170348030295	-36.3170348030295\\
69.875	0.15156	-38.7907020138641	-38.7907020138641\\
69.875	0.15522	-41.3866008885094	-41.3866008885094\\
69.875	0.15888	-44.1047314269656	-44.1047314269656\\
69.875	0.16254	-46.9450936292324	-46.9450936292324\\
69.875	0.1662	-49.9076874953101	-49.9076874953101\\
69.875	0.16986	-52.9925130251985	-52.9925130251985\\
69.875	0.17352	-56.1995702188976	-56.1995702188976\\
69.875	0.17718	-59.5288590764075	-59.5288590764075\\
69.875	0.18084	-62.9803795977281	-62.9803795977281\\
69.875	0.1845	-66.5541317828596	-66.5541317828596\\
69.875	0.18816	-70.2501156318017	-70.2501156318017\\
69.875	0.19182	-74.0683311445547	-74.0683311445547\\
69.875	0.19548	-78.0087783211184	-78.0087783211184\\
69.875	0.19914	-82.0714571614928	-82.0714571614928\\
69.875	0.2028	-86.2563676656781	-86.2563676656781\\
69.875	0.20646	-90.5635098336741	-90.5635098336741\\
69.875	0.21012	-94.9928836654808	-94.9928836654808\\
69.875	0.21378	-99.5444891610983	-99.5444891610983\\
69.875	0.21744	-104.218326320527	-104.218326320527\\
69.875	0.2211	-109.014395143766	-109.014395143766\\
69.875	0.22476	-113.932695630815	-113.932695630815\\
69.875	0.22842	-118.973227781676	-118.973227781676\\
69.875	0.23208	-124.135991596347	-124.135991596347\\
69.875	0.23574	-129.420987074829	-129.420987074829\\
69.875	0.2394	-134.828214217122	-134.828214217122\\
69.875	0.24306	-140.357673023226	-140.357673023226\\
69.875	0.24672	-146.00936349314	-146.00936349314\\
69.875	0.25038	-151.783285626865	-151.783285626865\\
69.875	0.25404	-157.679439424401	-157.679439424401\\
69.875	0.2577	-163.697824885747	-163.697824885747\\
69.875	0.26136	-169.838442010905	-169.838442010905\\
69.875	0.26502	-176.101290799873	-176.101290799873\\
69.875	0.26868	-182.486371252652	-182.486371252652\\
69.875	0.27234	-188.993683369241	-188.993683369241\\
69.875	0.276	-195.623227149642	-195.623227149642\\
70.25	0.093	-14.1425590875584	-14.1425590875584\\
70.25	0.09666	-14.785530176742	-14.785530176742\\
70.25	0.10032	-15.5507329297363	-15.5507329297363\\
70.25	0.10398	-16.4381673465414	-16.4381673465414\\
70.25	0.10764	-17.4478334271573	-17.4478334271573\\
70.25	0.1113	-18.5797311715839	-18.5797311715839\\
70.25	0.11496	-19.8338605798212	-19.8338605798212\\
70.25	0.11862	-21.2102216518693	-21.2102216518693\\
70.25	0.12228	-22.7088143877282	-22.7088143877282\\
70.25	0.12594	-24.3296387873979	-24.3296387873979\\
70.25	0.1296	-26.0726948508782	-26.0726948508782\\
70.25	0.13326	-27.9379825781694	-27.9379825781694\\
70.25	0.13692	-29.9255019692713	-29.9255019692713\\
70.25	0.14058	-32.035253024184	-32.035253024184\\
70.25	0.14424	-34.2672357429074	-34.2672357429074\\
70.25	0.1479	-36.6214501254416	-36.6214501254416\\
70.25	0.15156	-39.0978961717866	-39.0978961717866\\
70.25	0.15522	-41.6965738819423	-41.6965738819423\\
70.25	0.15888	-44.4174832559087	-44.4174832559087\\
70.25	0.16254	-47.2606242936859	-47.2606242936859\\
70.25	0.1662	-50.2259969952739	-50.2259969952739\\
70.25	0.16986	-53.3136013606727	-53.3136013606727\\
70.25	0.17352	-56.5234373898822	-56.5234373898822\\
70.25	0.17718	-59.8555050829024	-59.8555050829024\\
70.25	0.18084	-63.3098044397334	-63.3098044397334\\
70.25	0.1845	-66.8863354603752	-66.8863354603752\\
70.25	0.18816	-70.5850981448278	-70.5850981448278\\
70.25	0.19182	-74.4060924930911	-74.4060924930911\\
70.25	0.19548	-78.3493185051651	-78.3493185051651\\
70.25	0.19914	-82.4147761810499	-82.4147761810499\\
70.25	0.2028	-86.6024655207455	-86.6024655207455\\
70.25	0.20646	-90.9123865242518	-90.9123865242518\\
70.25	0.21012	-95.3445391915689	-95.3445391915689\\
70.25	0.21378	-99.8989235226967	-99.8989235226967\\
70.25	0.21744	-104.575539517635	-104.575539517635\\
70.25	0.2211	-109.374387176385	-109.374387176385\\
70.25	0.22476	-114.295466498945	-114.295466498945\\
70.25	0.22842	-119.338777485316	-119.338777485316\\
70.25	0.23208	-124.504320135497	-124.504320135497\\
70.25	0.23574	-129.79209444949	-129.79209444949\\
70.25	0.2394	-135.202100427293	-135.202100427293\\
70.25	0.24306	-140.734338068907	-140.734338068907\\
70.25	0.24672	-146.388807374331	-146.388807374331\\
70.25	0.25038	-152.165508343567	-152.165508343567\\
70.25	0.25404	-158.064440976613	-158.064440976613\\
70.25	0.2577	-164.08560527347	-164.08560527347\\
70.25	0.26136	-170.229001234138	-170.229001234138\\
70.25	0.26502	-176.494628858616	-176.494628858616\\
70.25	0.26868	-182.882488146905	-182.882488146905\\
70.25	0.27234	-189.392579099005	-189.392579099005\\
70.25	0.276	-196.024901714916	-196.024901714916\\
70.625	0.093	-14.4148882498698	-14.4148882498698\\
70.625	0.09666	-15.0606381745637	-15.0606381745637\\
70.625	0.10032	-15.8286197630684	-15.8286197630684\\
70.625	0.10398	-16.7188330153838	-16.7188330153838\\
70.625	0.10764	-17.73127793151	-17.73127793151\\
70.625	0.1113	-18.865954511447	-18.865954511447\\
70.625	0.11496	-20.1228627551947	-20.1228627551947\\
70.625	0.11862	-21.5020026627531	-21.5020026627531\\
70.625	0.12228	-23.0033742341223	-23.0033742341223\\
70.625	0.12594	-24.6269774693023	-24.6269774693023\\
70.625	0.1296	-26.3728123682931	-26.3728123682931\\
70.625	0.13326	-28.2408789310946	-28.2408789310946\\
70.625	0.13692	-30.2311771577068	-30.2311771577068\\
70.625	0.14058	-32.3437070481298	-32.3437070481298\\
70.625	0.14424	-34.5784686023636	-34.5784686023636\\
70.625	0.1479	-36.9354618204082	-36.9354618204082\\
70.625	0.15156	-39.4146867022635	-39.4146867022635\\
70.625	0.15522	-42.0161432479295	-42.0161432479295\\
70.625	0.15888	-44.7398314574064	-44.7398314574064\\
70.625	0.16254	-47.5857513306939	-47.5857513306939\\
70.625	0.1662	-50.5539028677923	-50.5539028677923\\
70.625	0.16986	-53.6442860687014	-53.6442860687014\\
70.625	0.17352	-56.8569009334211	-56.8569009334211\\
70.625	0.17718	-60.1917474619518	-60.1917474619518\\
70.625	0.18084	-63.6488256542931	-63.6488256542931\\
70.625	0.1845	-67.2281355104452	-67.2281355104452\\
70.625	0.18816	-70.9296770304081	-70.9296770304081\\
70.625	0.19182	-74.7534502141817	-74.7534502141817\\
70.625	0.19548	-78.6994550617661	-78.6994550617661\\
70.625	0.19914	-82.7676915731613	-82.7676915731613\\
70.625	0.2028	-86.9581597483673	-86.9581597483673\\
70.625	0.20646	-91.2708595873839	-91.2708595873839\\
70.625	0.21012	-95.7057910902114	-95.7057910902114\\
70.625	0.21378	-100.26295425685	-100.26295425685\\
70.625	0.21744	-104.942349087299	-104.942349087299\\
70.625	0.2211	-109.743975581558	-109.743975581558\\
70.625	0.22476	-114.667833739629	-114.667833739629\\
70.625	0.22842	-119.71392356151	-119.71392356151\\
70.625	0.23208	-124.882245047202	-124.882245047202\\
70.625	0.23574	-130.172798196705	-130.172798196705\\
70.625	0.2394	-135.585583010018	-135.585583010018\\
70.625	0.24306	-141.120599487142	-141.120599487142\\
70.625	0.24672	-146.777847628077	-146.777847628077\\
70.625	0.25038	-152.557327432823	-152.557327432823\\
70.625	0.25404	-158.45903890138	-158.45903890138\\
70.625	0.2577	-164.482982033747	-164.482982033747\\
70.625	0.26136	-170.629156829925	-170.629156829925\\
70.625	0.26502	-176.897563289914	-176.897563289914\\
70.625	0.26868	-183.288201413713	-183.288201413713\\
70.625	0.27234	-189.801071201324	-189.801071201324\\
70.625	0.276	-196.436172652745	-196.436172652745\\
71	0.093	-14.6968137847354	-14.6968137847354\\
71	0.09666	-15.3453425449397	-15.3453425449397\\
71	0.10032	-16.1161029689547	-16.1161029689547\\
71	0.10398	-17.0090950567805	-17.0090950567805\\
71	0.10764	-18.024318808417	-18.024318808417\\
71	0.1113	-19.1617742238644	-19.1617742238644\\
71	0.11496	-20.4214613031224	-20.4214613031224\\
71	0.11862	-21.8033800461912	-21.8033800461912\\
71	0.12228	-23.3075304530708	-23.3075304530708\\
71	0.12594	-24.9339125237611	-24.9339125237611\\
71	0.1296	-26.6825262582622	-26.6825262582622\\
71	0.13326	-28.5533716565741	-28.5533716565741\\
71	0.13692	-30.5464487186967	-30.5464487186967\\
71	0.14058	-32.6617574446301	-32.6617574446301\\
71	0.14424	-34.8992978343742	-34.8992978343742\\
71	0.1479	-37.2590698879291	-37.2590698879291\\
71	0.15156	-39.7410736052947	-39.7410736052947\\
71	0.15522	-42.3453089864711	-42.3453089864711\\
71	0.15888	-45.0717760314583	-45.0717760314583\\
71	0.16254	-47.9204747402562	-47.9204747402562\\
71	0.1662	-50.8914051128649	-50.8914051128649\\
71	0.16986	-53.9845671492843	-53.9845671492843\\
71	0.17352	-57.1999608495145	-57.1999608495145\\
71	0.17718	-60.5375862135554	-60.5375862135554\\
71	0.18084	-63.9974432414071	-63.9974432414071\\
71	0.1845	-67.5795319330697	-67.5795319330697\\
71	0.18816	-71.2838522885428	-71.2838522885428\\
71	0.19182	-75.1104043078268	-75.1104043078268\\
71	0.19548	-79.0591879909216	-79.0591879909216\\
71	0.19914	-83.1302033378271	-83.1302033378271\\
71	0.2028	-87.3234503485434	-87.3234503485434\\
71	0.20646	-91.6389290230704	-91.6389290230704\\
71	0.21012	-96.0766393614082	-96.0766393614082\\
71	0.21378	-100.636581363557	-100.636581363557\\
71	0.21744	-105.318755029516	-105.318755029516\\
71	0.2211	-110.123160359286	-110.123160359286\\
71	0.22476	-115.049797352867	-115.049797352867\\
71	0.22842	-120.098666010258	-120.098666010258\\
71	0.23208	-125.269766331461	-125.269766331461\\
71	0.23574	-130.563098316474	-130.563098316474\\
71	0.2394	-135.978661965298	-135.978661965298\\
71	0.24306	-141.516457277932	-141.516457277932\\
71	0.24672	-147.176484254378	-147.176484254378\\
71	0.25038	-152.958742894634	-152.958742894634\\
71	0.25404	-158.863233198701	-158.863233198701\\
71	0.2577	-164.889955166578	-164.889955166578\\
71	0.26136	-171.038908798267	-171.038908798267\\
71	0.26502	-177.310094093766	-177.310094093766\\
71	0.26868	-183.703511053076	-183.703511053076\\
71	0.27234	-190.219159676196	-190.219159676196\\
71	0.276	-196.857039963128	-196.857039963128\\
71.375	0.093	-14.9883356921555	-14.9883356921555\\
71.375	0.09666	-15.6396432878702	-15.6396432878702\\
71.375	0.10032	-16.4131825473956	-16.4131825473956\\
71.375	0.10398	-17.3089534707317	-17.3089534707317\\
71.375	0.10764	-18.3269560578786	-18.3269560578786\\
71.375	0.1113	-19.4671903088362	-19.4671903088362\\
71.375	0.11496	-20.7296562236046	-20.7296562236046\\
71.375	0.11862	-22.1143538021838	-22.1143538021838\\
71.375	0.12228	-23.6212830445737	-23.6212830445737\\
71.375	0.12594	-25.2504439507744	-25.2504439507744\\
71.375	0.1296	-27.0018365207858	-27.0018365207858\\
71.375	0.13326	-28.875460754608	-28.875460754608\\
71.375	0.13692	-30.871316652241	-30.871316652241\\
71.375	0.14058	-32.9894042136847	-32.9894042136847\\
71.375	0.14424	-35.2297234389392	-35.2297234389392\\
71.375	0.1479	-37.5922743280044	-37.5922743280044\\
71.375	0.15156	-40.0770568808804	-40.0770568808804\\
71.375	0.15522	-42.6840710975672	-42.6840710975672\\
71.375	0.15888	-45.4133169780647	-45.4133169780647\\
71.375	0.16254	-48.2647945223729	-48.2647945223729\\
71.375	0.1662	-51.238503730492	-51.238503730492\\
71.375	0.16986	-54.3344446024217	-54.3344446024217\\
71.375	0.17352	-57.5526171381622	-57.5526171381622\\
71.375	0.17718	-60.8930213377135	-60.8930213377135\\
71.375	0.18084	-64.3556572010756	-64.3556572010756\\
71.375	0.1845	-67.9405247282485	-67.9405247282485\\
71.375	0.18816	-71.647623919232	-71.647623919232\\
71.375	0.19182	-75.4769547740264	-75.4769547740264\\
71.375	0.19548	-79.4285172926315	-79.4285172926315\\
71.375	0.19914	-83.5023114750473	-83.5023114750473\\
71.375	0.2028	-87.698337321274	-87.698337321274\\
71.375	0.20646	-92.0165948313113	-92.0165948313113\\
71.375	0.21012	-96.4570840051595	-96.4570840051595\\
71.375	0.21378	-101.019804842818	-101.019804842818\\
71.375	0.21744	-105.704757344288	-105.704757344288\\
71.375	0.2211	-110.511941509568	-110.511941509568\\
71.375	0.22476	-115.44135733866	-115.44135733866\\
71.375	0.22842	-120.493004831561	-120.493004831561\\
71.375	0.23208	-125.666883988274	-125.666883988274\\
71.375	0.23574	-130.962994808798	-130.962994808798\\
71.375	0.2394	-136.381337293132	-136.381337293132\\
71.375	0.24306	-141.921911441277	-141.921911441277\\
71.375	0.24672	-147.584717253232	-147.584717253232\\
71.375	0.25038	-153.369754728999	-153.369754728999\\
71.375	0.25404	-159.277023868576	-159.277023868576\\
71.375	0.2577	-165.306524671964	-165.306524671964\\
71.375	0.26136	-171.458257139163	-171.458257139163\\
71.375	0.26502	-177.732221270172	-177.732221270172\\
71.375	0.26868	-184.128417064993	-184.128417064993\\
71.375	0.27234	-190.646844523624	-190.646844523624\\
71.375	0.276	-197.287503646066	-197.287503646066\\
71.75	0.093	-15.28945397213	-15.28945397213\\
71.75	0.09666	-15.943540403355	-15.943540403355\\
71.75	0.10032	-16.7198584983908	-16.7198584983908\\
71.75	0.10398	-17.6184082572372	-17.6184082572372\\
71.75	0.10764	-18.6391896798945	-18.6391896798945\\
71.75	0.1113	-19.7822027663625	-19.7822027663625\\
71.75	0.11496	-21.0474475166412	-21.0474475166412\\
71.75	0.11862	-22.4349239307307	-22.4349239307307\\
71.75	0.12228	-23.944632008631	-23.944632008631\\
71.75	0.12594	-25.576571750342	-25.576571750342\\
71.75	0.1296	-27.3307431558638	-27.3307431558638\\
71.75	0.13326	-29.2071462251963	-29.2071462251963\\
71.75	0.13692	-31.2057809583397	-31.2057809583397\\
71.75	0.14058	-33.3266473552937	-33.3266473552937\\
71.75	0.14424	-35.5697454160586	-35.5697454160586\\
71.75	0.1479	-37.9350751406341	-37.9350751406341\\
71.75	0.15156	-40.4226365290205	-40.4226365290205\\
71.75	0.15522	-43.0324295812176	-43.0324295812176\\
71.75	0.15888	-45.7644542972255	-45.7644542972255\\
71.75	0.16254	-48.6187106770441	-48.6187106770441\\
71.75	0.1662	-51.5951987206735	-51.5951987206735\\
71.75	0.16986	-54.6939184281136	-54.6939184281136\\
71.75	0.17352	-57.9148697993644	-57.9148697993644\\
71.75	0.17718	-61.2580528344261	-61.2580528344261\\
71.75	0.18084	-64.7234675332985	-64.7234675332985\\
71.75	0.1845	-68.3111138959817	-68.3111138959817\\
71.75	0.18816	-72.0209919224755	-72.0209919224755\\
71.75	0.19182	-75.8531016127802	-75.8531016127802\\
71.75	0.19548	-79.8074429668956	-79.8074429668956\\
71.75	0.19914	-83.8840159848219	-83.8840159848219\\
71.75	0.2028	-88.0828206665589	-88.0828206665589\\
71.75	0.20646	-92.4038570121066	-92.4038570121066\\
71.75	0.21012	-96.8471250214651	-96.8471250214651\\
71.75	0.21378	-101.412624694634	-101.412624694634\\
71.75	0.21744	-106.100356031614	-106.100356031614\\
71.75	0.2211	-110.910319032405	-110.910319032405\\
71.75	0.22476	-115.842513697007	-115.842513697007\\
71.75	0.22842	-120.896940025419	-120.896940025419\\
71.75	0.23208	-126.073598017642	-126.073598017642\\
71.75	0.23574	-131.372487673676	-131.372487673676\\
71.75	0.2394	-136.79360899352	-136.79360899352\\
71.75	0.24306	-142.336961977176	-142.336961977176\\
71.75	0.24672	-148.002546624642	-148.002546624642\\
71.75	0.25038	-153.790362935918	-153.790362935918\\
71.75	0.25404	-159.700410911006	-159.700410911006\\
71.75	0.2577	-165.732690549904	-165.732690549904\\
71.75	0.26136	-171.887201852613	-171.887201852613\\
71.75	0.26502	-178.163944819133	-178.163944819133\\
71.75	0.26868	-184.562919449464	-184.562919449464\\
71.75	0.27234	-191.084125743605	-191.084125743605\\
71.75	0.276	-197.727563701557	-197.727563701557\\
72.125	0.093	-15.6001686246589	-15.6001686246589\\
72.125	0.09666	-16.2570338913942	-16.2570338913942\\
72.125	0.10032	-17.0361308219403	-17.0361308219403\\
72.125	0.10398	-17.9374594162971	-17.9374594162971\\
72.125	0.10764	-18.9610196744647	-18.9610196744647\\
72.125	0.1113	-20.1068115964431	-20.1068115964431\\
72.125	0.11496	-21.3748351822322	-21.3748351822322\\
72.125	0.11862	-22.765090431832	-22.765090431832\\
72.125	0.12228	-24.2775773452427	-24.2775773452427\\
72.125	0.12594	-25.912295922464	-25.912295922464\\
72.125	0.1296	-27.6692461634961	-27.6692461634961\\
72.125	0.13326	-29.548428068339	-29.548428068339\\
72.125	0.13692	-31.5498416369927	-31.5498416369927\\
72.125	0.14058	-33.6734868694571	-33.6734868694571\\
72.125	0.14424	-35.9193637657323	-35.9193637657323\\
72.125	0.1479	-38.2874723258182	-38.2874723258182\\
72.125	0.15156	-40.7778125497149	-40.7778125497149\\
72.125	0.15522	-43.3903844374224	-43.3903844374224\\
72.125	0.15888	-46.1251879889406	-46.1251879889406\\
72.125	0.16254	-48.9822232042696	-48.9822232042696\\
72.125	0.1662	-51.9614900834093	-51.9614900834093\\
72.125	0.16986	-55.0629886263598	-55.0629886263598\\
72.125	0.17352	-58.2867188331209	-58.2867188331209\\
72.125	0.17718	-61.6326807036929	-61.6326807036929\\
72.125	0.18084	-65.1008742380757	-65.1008742380757\\
72.125	0.1845	-68.6912994362692	-68.6912994362692\\
72.125	0.18816	-72.4039562982734	-72.4039562982734\\
72.125	0.19182	-76.2388448240885	-76.2388448240885\\
72.125	0.19548	-80.1959650137143	-80.1959650137143\\
72.125	0.19914	-84.2753168671508	-84.2753168671508\\
72.125	0.2028	-88.4769003843982	-88.4769003843982\\
72.125	0.20646	-92.8007155654563	-92.8007155654563\\
72.125	0.21012	-97.2467624103251	-97.2467624103251\\
72.125	0.21378	-101.815040919005	-101.815040919005\\
72.125	0.21744	-106.505551091495	-106.505551091495\\
72.125	0.2211	-111.318292927796	-111.318292927796\\
72.125	0.22476	-116.253266427908	-116.253266427908\\
72.125	0.22842	-121.310471591831	-121.310471591831\\
72.125	0.23208	-126.489908419564	-126.489908419564\\
72.125	0.23574	-131.791576911108	-131.791576911108\\
72.125	0.2394	-137.215477066463	-137.215477066463\\
72.125	0.24306	-142.761608885629	-142.761608885629\\
72.125	0.24672	-148.429972368605	-148.429972368605\\
72.125	0.25038	-154.220567515392	-154.220567515392\\
72.125	0.25404	-160.13339432599	-160.13339432599\\
72.125	0.2577	-166.168452800399	-166.168452800399\\
72.125	0.26136	-172.325742938618	-172.325742938618\\
72.125	0.26502	-178.605264740648	-178.605264740648\\
72.125	0.26868	-185.007018206489	-185.007018206489\\
72.125	0.27234	-191.531003336141	-191.531003336141\\
72.125	0.276	-198.177220129604	-198.177220129604\\
72.5	0.093	-15.9204796497422	-15.9204796497422\\
72.5	0.09666	-16.5801237519879	-16.5801237519879\\
72.5	0.10032	-17.3619995180443	-17.3619995180443\\
72.5	0.10398	-18.2661069479115	-18.2661069479115\\
72.5	0.10764	-19.2924460415895	-19.2924460415895\\
72.5	0.1113	-20.4410167990782	-20.4410167990782\\
72.5	0.11496	-21.7118192203776	-21.7118192203776\\
72.5	0.11862	-23.1048533054878	-23.1048533054878\\
72.5	0.12228	-24.6201190544088	-24.6201190544088\\
72.5	0.12594	-26.2576164671405	-26.2576164671405\\
72.5	0.1296	-28.017345543683	-28.017345543683\\
72.5	0.13326	-29.8993062840362	-29.8993062840362\\
72.5	0.13692	-31.9034986882002	-31.9034986882002\\
72.5	0.14058	-34.029922756175	-34.029922756175\\
72.5	0.14424	-36.2785784879605	-36.2785784879605\\
72.5	0.1479	-38.6494658835568	-38.6494658835568\\
72.5	0.15156	-41.1425849429638	-41.1425849429638\\
72.5	0.15522	-43.7579356661816	-43.7579356661816\\
72.5	0.15888	-46.4955180532101	-46.4955180532101\\
72.5	0.16254	-49.3553321040494	-49.3553321040494\\
72.5	0.1662	-52.3373778186995	-52.3373778186995\\
72.5	0.16986	-55.4416551971603	-55.4416551971603\\
72.5	0.17352	-58.6681642394319	-58.6681642394319\\
72.5	0.17718	-62.0169049455143	-62.0169049455143\\
72.5	0.18084	-65.4878773154074	-65.4878773154074\\
72.5	0.1845	-69.0810813491113	-69.0810813491113\\
72.5	0.18816	-72.7965170466259	-72.7965170466259\\
72.5	0.19182	-76.6341844079513	-76.6341844079513\\
72.5	0.19548	-80.5940834330874	-80.5940834330874\\
72.5	0.19914	-84.6762141220343	-84.6762141220343\\
72.5	0.2028	-88.880576474792	-88.880576474792\\
72.5	0.20646	-93.2071704913604	-93.2071704913604\\
72.5	0.21012	-97.6559961717396	-97.6559961717396\\
72.5	0.21378	-102.227053515929	-102.227053515929\\
72.5	0.21744	-106.92034252393	-106.92034252393\\
72.5	0.2211	-111.735863195742	-111.735863195742\\
72.5	0.22476	-116.673615531364	-116.673615531364\\
72.5	0.22842	-121.733599530797	-121.733599530797\\
72.5	0.23208	-126.91581519404	-126.91581519404\\
72.5	0.23574	-132.220262521095	-132.220262521095\\
72.5	0.2394	-137.64694151196	-137.64694151196\\
72.5	0.24306	-143.195852166636	-143.195852166636\\
72.5	0.24672	-148.866994485123	-148.866994485123\\
72.5	0.25038	-154.660368467421	-154.660368467421\\
72.5	0.25404	-160.575974113529	-160.575974113529\\
72.5	0.2577	-166.613811423448	-166.613811423448\\
72.5	0.26136	-172.773880397178	-172.773880397178\\
72.5	0.26502	-179.056181034718	-179.056181034718\\
72.5	0.26868	-185.46071333607	-185.46071333607\\
72.5	0.27234	-191.987477301232	-191.987477301232\\
72.5	0.276	-198.636472930204	-198.636472930204\\
72.875	0.093	-16.25038704738	-16.25038704738\\
72.875	0.09666	-16.912809985136	-16.912809985136\\
72.875	0.10032	-17.6974645867028	-17.6974645867028\\
72.875	0.10398	-18.6043508520803	-18.6043508520803\\
72.875	0.10764	-19.6334687812686	-19.6334687812686\\
72.875	0.1113	-20.7848183742676	-20.7848183742676\\
72.875	0.11496	-22.0583996310774	-22.0583996310774\\
72.875	0.11862	-23.454212551698	-23.454212551698\\
72.875	0.12228	-24.9722571361293	-24.9722571361293\\
72.875	0.12594	-26.6125333843713	-26.6125333843713\\
72.875	0.1296	-28.3750412964242	-28.3750412964242\\
72.875	0.13326	-30.2597808722878	-30.2597808722878\\
72.875	0.13692	-32.2667521119621	-32.2667521119621\\
72.875	0.14058	-34.3959550154472	-34.3959550154472\\
72.875	0.14424	-36.6473895827431	-36.6473895827431\\
72.875	0.1479	-39.0210558138497	-39.0210558138497\\
72.875	0.15156	-41.5169537087671	-41.5169537087671\\
72.875	0.15522	-44.1350832674952	-44.1350832674952\\
72.875	0.15888	-46.8754444900342	-46.8754444900342\\
72.875	0.16254	-49.7380373763838	-49.7380373763838\\
72.875	0.1662	-52.7228619265442	-52.7228619265442\\
72.875	0.16986	-55.8299181405154	-55.8299181405154\\
72.875	0.17352	-59.0592060182973	-59.0592060182973\\
72.875	0.17718	-62.41072555989	-62.41072555989\\
72.875	0.18084	-65.8844767652934	-65.8844767652934\\
72.875	0.1845	-69.4804596345077	-69.4804596345077\\
72.875	0.18816	-73.1986741675326	-73.1986741675326\\
72.875	0.19182	-77.0391203643683	-77.0391203643683\\
72.875	0.19548	-81.0017982250148	-81.0017982250148\\
72.875	0.19914	-85.0867077494721	-85.0867077494721\\
72.875	0.2028	-89.2938489377402	-89.2938489377402\\
72.875	0.20646	-93.6232217898189	-93.6232217898189\\
72.875	0.21012	-98.0748263057084	-98.0748263057084\\
72.875	0.21378	-102.648662485409	-102.648662485409\\
72.875	0.21744	-107.34473032892	-107.34473032892\\
72.875	0.2211	-112.163029836242	-112.163029836242\\
72.875	0.22476	-117.103561007374	-117.103561007374\\
72.875	0.22842	-122.166323842317	-122.166323842317\\
72.875	0.23208	-127.351318341071	-127.351318341071\\
72.875	0.23574	-132.658544503636	-132.658544503636\\
72.875	0.2394	-138.088002330012	-138.088002330012\\
72.875	0.24306	-143.639691820198	-143.639691820198\\
72.875	0.24672	-149.313612974195	-149.313612974195\\
72.875	0.25038	-155.109765792003	-155.109765792003\\
72.875	0.25404	-161.028150273622	-161.028150273622\\
72.875	0.2577	-167.068766419051	-167.068766419051\\
72.875	0.26136	-173.231614228291	-173.231614228291\\
72.875	0.26502	-179.516693701342	-179.516693701342\\
72.875	0.26868	-185.924004838204	-185.924004838204\\
72.875	0.27234	-192.453547638876	-192.453547638876\\
72.875	0.276	-199.105322103359	-199.105322103359\\
73.25	0.093	-16.589890817572	-16.589890817572\\
73.25	0.09666	-17.2550925908384	-17.2550925908384\\
73.25	0.10032	-18.0425260279155	-18.0425260279155\\
73.25	0.10398	-18.9521911288033	-18.9521911288033\\
73.25	0.10764	-19.984087893502	-19.984087893502\\
73.25	0.1113	-21.1382163220114	-21.1382163220114\\
73.25	0.11496	-22.4145764143315	-22.4145764143315\\
73.25	0.11862	-23.8131681704624	-23.8131681704624\\
73.25	0.12228	-25.3339915904041	-25.3339915904041\\
73.25	0.12594	-26.9770466741565	-26.9770466741565\\
73.25	0.1296	-28.7423334217197	-28.7423334217197\\
73.25	0.13326	-30.6298518330936	-30.6298518330936\\
73.25	0.13692	-32.6396019082783	-32.6396019082783\\
73.25	0.14058	-34.7715836472738	-34.7715836472738\\
73.25	0.14424	-37.02579705008	-37.02579705008\\
73.25	0.1479	-39.402242116697	-39.402242116697\\
73.25	0.15156	-41.9009188471247	-41.9009188471247\\
73.25	0.15522	-44.5218272413632	-44.5218272413632\\
73.25	0.15888	-47.2649672994125	-47.2649672994125\\
73.25	0.16254	-50.1303390212725	-50.1303390212725\\
73.25	0.1662	-53.1179424069432	-53.1179424069432\\
73.25	0.16986	-56.2277774564248	-56.2277774564248\\
73.25	0.17352	-59.459844169717	-59.459844169717\\
73.25	0.17718	-62.8141425468201	-62.8141425468201\\
73.25	0.18084	-66.2906725877338	-66.2906725877338\\
73.25	0.1845	-69.8894342924584	-69.8894342924584\\
73.25	0.18816	-73.6104276609937	-73.6104276609937\\
73.25	0.19182	-77.4536526933398	-77.4536526933398\\
73.25	0.19548	-81.4191093894966	-81.4191093894966\\
73.25	0.19914	-85.5067977494642	-85.5067977494642\\
73.25	0.2028	-89.7167177732426	-89.7167177732426\\
73.25	0.20646	-94.0488694608318	-94.0488694608318\\
73.25	0.21012	-98.5032528122316	-98.5032528122316\\
73.25	0.21378	-103.079867827442	-103.079867827442\\
73.25	0.21744	-107.778714506464	-107.778714506464\\
73.25	0.2211	-112.599792849296	-112.599792849296\\
73.25	0.22476	-117.543102855939	-117.543102855939\\
73.25	0.22842	-122.608644526392	-122.608644526392\\
73.25	0.23208	-127.796417860657	-127.796417860657\\
73.25	0.23574	-133.106422858732	-133.106422858732\\
73.25	0.2394	-138.538659520618	-138.538659520618\\
73.25	0.24306	-144.093127846315	-144.093127846315\\
73.25	0.24672	-149.769827835822	-149.769827835822\\
73.25	0.25038	-155.56875948914	-155.56875948914\\
73.25	0.25404	-161.489922806269	-161.489922806269\\
73.25	0.2577	-167.533317787209	-167.533317787209\\
73.25	0.26136	-173.698944431959	-173.698944431959\\
73.25	0.26502	-179.986802740521	-179.986802740521\\
73.25	0.26868	-186.396892712893	-186.396892712893\\
73.25	0.27234	-192.929214349075	-192.929214349075\\
73.25	0.276	-199.583767649069	-199.583767649069\\
73.625	0.093	-16.9389909603185	-16.9389909603185\\
73.625	0.09666	-17.6069715690952	-17.6069715690952\\
73.625	0.10032	-18.3971838416827	-18.3971838416827\\
73.625	0.10398	-19.3096277780809	-19.3096277780809\\
73.625	0.10764	-20.3443033782899	-20.3443033782899\\
73.625	0.1113	-21.5012106423096	-21.5012106423096\\
73.625	0.11496	-22.7803495701401	-22.7803495701401\\
73.625	0.11862	-24.1817201617814	-24.1817201617814\\
73.625	0.12228	-25.7053224172334	-25.7053224172334\\
73.625	0.12594	-27.3511563364962	-27.3511563364962\\
73.625	0.1296	-29.1192219195697	-29.1192219195697\\
73.625	0.13326	-31.009519166454	-31.009519166454\\
73.625	0.13692	-33.022048077149	-33.022048077149\\
73.625	0.14058	-35.1568086516548	-35.1568086516548\\
73.625	0.14424	-37.4138008899714	-37.4138008899714\\
73.625	0.1479	-39.7930247920987	-39.7930247920987\\
73.625	0.15156	-42.2944803580368	-42.2944803580368\\
73.625	0.15522	-44.9181675877856	-44.9181675877856\\
73.625	0.15888	-47.6640864813452	-47.6640864813452\\
73.625	0.16254	-50.5322370387156	-50.5322370387156\\
73.625	0.1662	-53.5226192598967	-53.5226192598967\\
73.625	0.16986	-56.6352331448885	-56.6352331448885\\
73.625	0.17352	-59.8700786936912	-59.8700786936912\\
73.625	0.17718	-63.2271559063045	-63.2271559063045\\
73.625	0.18084	-66.7064647827287	-66.7064647827287\\
73.625	0.1845	-70.3080053229636	-70.3080053229636\\
73.625	0.18816	-74.0317775270093	-74.0317775270093\\
73.625	0.19182	-77.8777813948657	-77.8777813948657\\
73.625	0.19548	-81.8460169265329	-81.8460169265329\\
73.625	0.19914	-85.9364841220109	-85.9364841220109\\
73.625	0.2028	-90.1491829812996	-90.1491829812996\\
73.625	0.20646	-94.484113504399	-94.484113504399\\
73.625	0.21012	-98.9412756913093	-98.9412756913093\\
73.625	0.21378	-103.52066954203	-103.52066954203\\
73.625	0.21744	-108.222295056562	-108.222295056562\\
73.625	0.2211	-113.046152234904	-113.046152234904\\
73.625	0.22476	-117.992241077058	-117.992241077058\\
73.625	0.22842	-123.060561583022	-123.060561583022\\
73.625	0.23208	-128.251113752796	-128.251113752796\\
73.625	0.23574	-133.563897586382	-133.563897586382\\
73.625	0.2394	-138.998913083778	-138.998913083778\\
73.625	0.24306	-144.556160244985	-144.556160244985\\
73.625	0.24672	-150.235639070003	-150.235639070003\\
73.625	0.25038	-156.037349558832	-156.037349558832\\
73.625	0.25404	-161.961291711471	-161.961291711471\\
73.625	0.2577	-168.007465527921	-168.007465527921\\
73.625	0.26136	-174.175871008182	-174.175871008182\\
73.625	0.26502	-180.466508152254	-180.466508152254\\
73.625	0.26868	-186.879376960136	-186.879376960136\\
73.625	0.27234	-193.414477431829	-193.414477431829\\
73.625	0.276	-200.071809567333	-200.071809567333\\
74	0.093	-17.2976874756194	-17.2976874756194\\
74	0.09666	-17.9684469199065	-17.9684469199065\\
74	0.10032	-18.7614380280043	-18.7614380280043\\
74	0.10398	-19.6766607999128	-19.6766607999128\\
74	0.10764	-20.7141152356322	-20.7141152356322\\
74	0.1113	-21.8738013351623	-21.8738013351623\\
74	0.11496	-23.1557190985031	-23.1557190985031\\
74	0.11862	-24.5598685256547	-24.5598685256547\\
74	0.12228	-26.0862496166171	-26.0862496166171\\
74	0.12594	-27.7348623713902	-27.7348623713902\\
74	0.1296	-29.505706789974	-29.505706789974\\
74	0.13326	-31.3987828723687	-31.3987828723687\\
74	0.13692	-33.414090618574	-33.414090618574\\
74	0.14058	-35.5516300285902	-35.5516300285902\\
74	0.14424	-37.8114011024172	-37.8114011024172\\
74	0.1479	-40.1934038400548	-40.1934038400548\\
74	0.15156	-42.6976382415032	-42.6976382415032\\
74	0.15522	-45.3241043067624	-45.3241043067624\\
74	0.15888	-48.0728020358324	-48.0728020358324\\
74	0.16254	-50.9437314287131	-50.9437314287131\\
74	0.1662	-53.9368924854046	-53.9368924854046\\
74	0.16986	-57.0522852059068	-57.0522852059068\\
74	0.17352	-60.2899095902197	-60.2899095902197\\
74	0.17718	-63.6497656383435	-63.6497656383435\\
74	0.18084	-67.1318533502779	-67.1318533502779\\
74	0.1845	-70.7361727260233	-70.7361727260233\\
74	0.18816	-74.4627237655792	-74.4627237655792\\
74	0.19182	-78.311506468946	-78.311506468946\\
74	0.19548	-82.2825208361235	-82.2825208361235\\
74	0.19914	-86.3757668671118	-86.3757668671118\\
74	0.2028	-90.5912445619109	-90.5912445619109\\
74	0.20646	-94.9289539205207	-94.9289539205207\\
74	0.21012	-99.3888949429413	-99.3888949429413\\
74	0.21378	-103.971067629173	-103.971067629173\\
74	0.21744	-108.675471979215	-108.675471979215\\
74	0.2211	-113.502107993068	-113.502107993068\\
74	0.22476	-118.450975670731	-118.450975670731\\
74	0.22842	-123.522075012205	-123.522075012205\\
74	0.23208	-128.715406017491	-128.715406017491\\
74	0.23574	-134.030968686587	-134.030968686587\\
74	0.2394	-139.468763019493	-139.468763019493\\
74	0.24306	-145.02878901621	-145.02878901621\\
74	0.24672	-150.711046676739	-150.711046676739\\
74	0.25038	-156.515536001077	-156.515536001077\\
74	0.25404	-162.442256989227	-162.442256989227\\
74	0.2577	-168.491209641188	-168.491209641188\\
74	0.26136	-174.662393956959	-174.662393956959\\
74	0.26502	-180.955809936541	-180.955809936541\\
74	0.26868	-187.371457579933	-187.371457579933\\
74	0.27234	-193.909336887137	-193.909336887137\\
74	0.276	-200.569447858151	-200.569447858151\\
};
\end{axis}

\begin{axis}[%
width=6.159cm,
height=3.097cm,
at={(8.104cm,12.903cm)},
scale only axis,
xmin=56,
xmax=74,
tick align=outside,
xlabel style={font=\color{white!15!black}},
xlabel={$L_{cut}$},
ymin=0.093,
ymax=0.276,
ylabel style={font=\color{white!15!black}},
ylabel={$D_{rlx}$},
zmin=0,
zmax=517.903549897876,
zlabel style={font=\color{white!15!black}},
zlabel={$u(t-1)$},
view={-140}{50},
axis background/.style={fill=white},
xmajorgrids,
ymajorgrids,
zmajorgrids
]
\addplot3[only marks, mark=*, mark options={}, mark size=1.5000pt, color=mycolor1, fill=mycolor1] table[row sep=crcr]{%
x	y	z\\
74	0.123	75.4245651900459\\
72	0.113	59.5782863482935\\
61	0.095	33.0003406266977\\
56	0.093	26.0569470415911\\
};
\addplot3[only marks, mark=*, mark options={}, mark size=1.5000pt, color=mycolor2, fill=mycolor2] table[row sep=crcr]{%
x	y	z\\
67	0.276	508.93731875208\\
66	0.255	419.53506953587\\
62	0.209	242.35439857544\\
57	0.193	195.15231965962\\
};
\addplot3[only marks, mark=*, mark options={}, mark size=1.5000pt, color=black, fill=black] table[row sep=crcr]{%
x	y	z\\
69	0.104	48.1088130226734\\
};
\addplot3[only marks, mark=*, mark options={}, mark size=1.5000pt, color=black, fill=black] table[row sep=crcr]{%
x	y	z\\
64	0.23	317.23878457902\\
};

\addplot3[%
surf,
fill opacity=0.7, shader=interp, colormap={mymap}{[1pt] rgb(0pt)=(1,0.905882,0); rgb(1pt)=(1,0.901964,0); rgb(2pt)=(1,0.898051,0); rgb(3pt)=(1,0.894144,0); rgb(4pt)=(1,0.890243,0); rgb(5pt)=(1,0.886349,0); rgb(6pt)=(1,0.88246,0); rgb(7pt)=(1,0.878577,0); rgb(8pt)=(1,0.8747,0); rgb(9pt)=(1,0.870829,0); rgb(10pt)=(1,0.866964,0); rgb(11pt)=(1,0.863106,0); rgb(12pt)=(1,0.859253,0); rgb(13pt)=(1,0.855406,0); rgb(14pt)=(1,0.851566,0); rgb(15pt)=(1,0.847732,0); rgb(16pt)=(1,0.843903,0); rgb(17pt)=(1,0.840081,0); rgb(18pt)=(1,0.836265,0); rgb(19pt)=(1,0.832455,0); rgb(20pt)=(1,0.828652,0); rgb(21pt)=(1,0.824854,0); rgb(22pt)=(1,0.821063,0); rgb(23pt)=(1,0.817278,0); rgb(24pt)=(1,0.8135,0); rgb(25pt)=(1,0.809727,0); rgb(26pt)=(1,0.805961,0); rgb(27pt)=(1,0.8022,0); rgb(28pt)=(1,0.798445,0); rgb(29pt)=(1,0.794696,0); rgb(30pt)=(1,0.790953,0); rgb(31pt)=(1,0.787215,0); rgb(32pt)=(1,0.783484,0); rgb(33pt)=(1,0.779758,0); rgb(34pt)=(1,0.776038,0); rgb(35pt)=(1,0.772324,0); rgb(36pt)=(1,0.768615,0); rgb(37pt)=(1,0.764913,0); rgb(38pt)=(1,0.761217,0); rgb(39pt)=(1,0.757527,0); rgb(40pt)=(1,0.753843,0); rgb(41pt)=(1,0.750165,0); rgb(42pt)=(1,0.746493,0); rgb(43pt)=(1,0.742827,0); rgb(44pt)=(1,0.739167,0); rgb(45pt)=(1,0.735514,0); rgb(46pt)=(1,0.731867,0); rgb(47pt)=(1,0.728226,0); rgb(48pt)=(1,0.724591,0); rgb(49pt)=(1,0.720963,0); rgb(50pt)=(1,0.717341,0); rgb(51pt)=(1,0.713725,0); rgb(52pt)=(0.999994,0.710077,0); rgb(53pt)=(0.999974,0.706363,0); rgb(54pt)=(0.999942,0.702592,0); rgb(55pt)=(0.999898,0.698775,0); rgb(56pt)=(0.999841,0.694921,0); rgb(57pt)=(0.999771,0.691039,0); rgb(58pt)=(0.99969,0.687139,0); rgb(59pt)=(0.999596,0.68323,0); rgb(60pt)=(0.99949,0.679323,0); rgb(61pt)=(0.999372,0.675427,0); rgb(62pt)=(0.999242,0.67155,0); rgb(63pt)=(0.9991,0.667704,0); rgb(64pt)=(0.998946,0.663897,0); rgb(65pt)=(0.998781,0.660138,0); rgb(66pt)=(0.998605,0.656439,0); rgb(67pt)=(0.998416,0.652807,0); rgb(68pt)=(0.998217,0.649253,0); rgb(69pt)=(0.998006,0.645786,0); rgb(70pt)=(0.997785,0.642416,0); rgb(71pt)=(0.997552,0.639152,0); rgb(72pt)=(0.997308,0.636004,0); rgb(73pt)=(0.997053,0.632982,0); rgb(74pt)=(0.996788,0.630095,0); rgb(75pt)=(0.996512,0.627352,0); rgb(76pt)=(0.996226,0.624763,0); rgb(77pt)=(0.995851,0.622329,0); rgb(78pt)=(0.99494,0.619997,0); rgb(79pt)=(0.99345,0.617753,0); rgb(80pt)=(0.991419,0.61559,0); rgb(81pt)=(0.988885,0.613503,0); rgb(82pt)=(0.985886,0.611486,0); rgb(83pt)=(0.98246,0.609532,0); rgb(84pt)=(0.978643,0.607636,0); rgb(85pt)=(0.974475,0.605791,0); rgb(86pt)=(0.969992,0.603992,0); rgb(87pt)=(0.965232,0.602233,0); rgb(88pt)=(0.960233,0.600507,0); rgb(89pt)=(0.955033,0.598808,0); rgb(90pt)=(0.949669,0.59713,0); rgb(91pt)=(0.94418,0.595468,0); rgb(92pt)=(0.938602,0.593815,0); rgb(93pt)=(0.932974,0.592166,0); rgb(94pt)=(0.927333,0.590513,0); rgb(95pt)=(0.921717,0.588852,0); rgb(96pt)=(0.916164,0.587176,0); rgb(97pt)=(0.910711,0.585479,0); rgb(98pt)=(0.905397,0.583755,0); rgb(99pt)=(0.900258,0.581999,0); rgb(100pt)=(0.895333,0.580203,0); rgb(101pt)=(0.890659,0.578362,0); rgb(102pt)=(0.886275,0.576471,0); rgb(103pt)=(0.882047,0.574545,0); rgb(104pt)=(0.877819,0.572608,0); rgb(105pt)=(0.873592,0.57066,0); rgb(106pt)=(0.869366,0.568701,0); rgb(107pt)=(0.865143,0.566733,0); rgb(108pt)=(0.860924,0.564756,0); rgb(109pt)=(0.856708,0.562771,0); rgb(110pt)=(0.852497,0.560778,0); rgb(111pt)=(0.848292,0.558779,0); rgb(112pt)=(0.844092,0.556774,0); rgb(113pt)=(0.8399,0.554763,0); rgb(114pt)=(0.835716,0.552749,0); rgb(115pt)=(0.831541,0.55073,0); rgb(116pt)=(0.827374,0.548709,0); rgb(117pt)=(0.823219,0.546686,0); rgb(118pt)=(0.819074,0.54466,0); rgb(119pt)=(0.81494,0.542635,0); rgb(120pt)=(0.81082,0.540609,0); rgb(121pt)=(0.806712,0.538584,0); rgb(122pt)=(0.802619,0.53656,0); rgb(123pt)=(0.798541,0.534539,0); rgb(124pt)=(0.794478,0.532521,0); rgb(125pt)=(0.790431,0.530506,0); rgb(126pt)=(0.786402,0.528496,0); rgb(127pt)=(0.782391,0.526491,0); rgb(128pt)=(0.77841,0.524489,0); rgb(129pt)=(0.774523,0.522478,0); rgb(130pt)=(0.770731,0.520455,0); rgb(131pt)=(0.767022,0.518424,0); rgb(132pt)=(0.763384,0.516385,0); rgb(133pt)=(0.759804,0.514339,0); rgb(134pt)=(0.756272,0.51229,0); rgb(135pt)=(0.752775,0.510237,0); rgb(136pt)=(0.749302,0.508182,0); rgb(137pt)=(0.74584,0.506128,0); rgb(138pt)=(0.742378,0.504075,0); rgb(139pt)=(0.738904,0.502025,0); rgb(140pt)=(0.735406,0.499979,0); rgb(141pt)=(0.731872,0.49794,0); rgb(142pt)=(0.72829,0.495909,0); rgb(143pt)=(0.724649,0.493887,0); rgb(144pt)=(0.720936,0.491875,0); rgb(145pt)=(0.71714,0.489876,0); rgb(146pt)=(0.713249,0.487891,0); rgb(147pt)=(0.709251,0.485921,0); rgb(148pt)=(0.705134,0.483968,0); rgb(149pt)=(0.700887,0.482033,0); rgb(150pt)=(0.696497,0.480118,0); rgb(151pt)=(0.691952,0.478225,0); rgb(152pt)=(0.687242,0.476355,0); rgb(153pt)=(0.682353,0.47451,0); rgb(154pt)=(0.677195,0.472696,0); rgb(155pt)=(0.6717,0.470916,0); rgb(156pt)=(0.665891,0.469169,0); rgb(157pt)=(0.659791,0.46745,0); rgb(158pt)=(0.653423,0.465756,0); rgb(159pt)=(0.64681,0.464084,0); rgb(160pt)=(0.639976,0.462432,0); rgb(161pt)=(0.632943,0.460795,0); rgb(162pt)=(0.625734,0.459171,0); rgb(163pt)=(0.618373,0.457556,0); rgb(164pt)=(0.610882,0.455948,0); rgb(165pt)=(0.603284,0.454343,0); rgb(166pt)=(0.595604,0.452737,0); rgb(167pt)=(0.587863,0.451129,0); rgb(168pt)=(0.580084,0.449514,0); rgb(169pt)=(0.572292,0.447889,0); rgb(170pt)=(0.564508,0.446252,0); rgb(171pt)=(0.556756,0.444599,0); rgb(172pt)=(0.549059,0.442927,0); rgb(173pt)=(0.54144,0.441232,0); rgb(174pt)=(0.533922,0.439512,0); rgb(175pt)=(0.526529,0.437764,0); rgb(176pt)=(0.519282,0.435983,0); rgb(177pt)=(0.512206,0.434168,0); rgb(178pt)=(0.505323,0.432315,0); rgb(179pt)=(0.498628,0.430422,3.92506e-06); rgb(180pt)=(0.491973,0.428504,3.49981e-05); rgb(181pt)=(0.485331,0.426562,9.63073e-05); rgb(182pt)=(0.478704,0.424596,0.000186979); rgb(183pt)=(0.472096,0.422609,0.000306141); rgb(184pt)=(0.465508,0.420599,0.00045292); rgb(185pt)=(0.458942,0.418567,0.000626441); rgb(186pt)=(0.452401,0.416515,0.000825833); rgb(187pt)=(0.445885,0.414441,0.00105022); rgb(188pt)=(0.439399,0.412348,0.00129873); rgb(189pt)=(0.432942,0.410234,0.00157049); rgb(190pt)=(0.426518,0.408102,0.00186463); rgb(191pt)=(0.420129,0.40595,0.00218028); rgb(192pt)=(0.413777,0.40378,0.00251655); rgb(193pt)=(0.407464,0.401592,0.00287258); rgb(194pt)=(0.401191,0.399386,0.00324749); rgb(195pt)=(0.394962,0.397164,0.00364042); rgb(196pt)=(0.388777,0.394925,0.00405048); rgb(197pt)=(0.38264,0.39267,0.00447681); rgb(198pt)=(0.376552,0.390399,0.00491852); rgb(199pt)=(0.370516,0.388113,0.00537476); rgb(200pt)=(0.364532,0.385812,0.00584464); rgb(201pt)=(0.358605,0.383497,0.00632729); rgb(202pt)=(0.352735,0.381168,0.00682184); rgb(203pt)=(0.346925,0.378826,0.00732741); rgb(204pt)=(0.341176,0.376471,0.00784314); rgb(205pt)=(0.335485,0.374093,0.00847245); rgb(206pt)=(0.329843,0.371682,0.00930909); rgb(207pt)=(0.324249,0.369242,0.0103377); rgb(208pt)=(0.318701,0.366772,0.0115428); rgb(209pt)=(0.313198,0.364275,0.0129091); rgb(210pt)=(0.307739,0.361753,0.0144211); rgb(211pt)=(0.302322,0.359206,0.0160634); rgb(212pt)=(0.296945,0.356637,0.0178207); rgb(213pt)=(0.291607,0.354048,0.0196776); rgb(214pt)=(0.286307,0.35144,0.0216186); rgb(215pt)=(0.281043,0.348814,0.0236284); rgb(216pt)=(0.275813,0.346172,0.0256916); rgb(217pt)=(0.270616,0.343517,0.0277927); rgb(218pt)=(0.265451,0.340849,0.0299163); rgb(219pt)=(0.260317,0.33817,0.0320472); rgb(220pt)=(0.25521,0.335482,0.0341698); rgb(221pt)=(0.250131,0.332786,0.0362688); rgb(222pt)=(0.245078,0.330085,0.0383287); rgb(223pt)=(0.240048,0.327379,0.0403343); rgb(224pt)=(0.235042,0.324671,0.04227); rgb(225pt)=(0.230056,0.321962,0.0441205); rgb(226pt)=(0.22509,0.319254,0.0458704); rgb(227pt)=(0.220142,0.316548,0.0475043); rgb(228pt)=(0.215212,0.313846,0.0490067); rgb(229pt)=(0.210296,0.311149,0.0503624); rgb(230pt)=(0.205395,0.308459,0.0515759); rgb(231pt)=(0.200514,0.305763,0.052757); rgb(232pt)=(0.195655,0.303061,0.0539242); rgb(233pt)=(0.190817,0.300353,0.0550763); rgb(234pt)=(0.186001,0.297639,0.0562123); rgb(235pt)=(0.181207,0.294918,0.0573313); rgb(236pt)=(0.176434,0.292191,0.0584321); rgb(237pt)=(0.171685,0.289458,0.0595136); rgb(238pt)=(0.166957,0.286719,0.060575); rgb(239pt)=(0.162252,0.283973,0.0616151); rgb(240pt)=(0.15757,0.281221,0.0626328); rgb(241pt)=(0.152911,0.278463,0.0636271); rgb(242pt)=(0.148275,0.275699,0.0645971); rgb(243pt)=(0.143663,0.272929,0.0655416); rgb(244pt)=(0.139074,0.270152,0.0664596); rgb(245pt)=(0.134508,0.26737,0.06735); rgb(246pt)=(0.129967,0.264581,0.0682118); rgb(247pt)=(0.125449,0.261787,0.0690441); rgb(248pt)=(0.120956,0.258986,0.0698456); rgb(249pt)=(0.116487,0.25618,0.0706154); rgb(250pt)=(0.112043,0.253367,0.0713525); rgb(251pt)=(0.107623,0.250549,0.0720557); rgb(252pt)=(0.103229,0.247724,0.0727241); rgb(253pt)=(0.0988592,0.244894,0.0733566); rgb(254pt)=(0.0945149,0.242058,0.0739522); rgb(255pt)=(0.0901961,0.239216,0.0745098)}, mesh/rows=49]
table[row sep=crcr, point meta=\thisrow{c}] {%
%
x	y	z	c\\
56	0.093	26.7122935396587	26.7122935396587\\
56	0.09666	28.8008150374213	28.8008150374213\\
56	0.10032	31.1960003360339	31.1960003360339\\
56	0.10398	33.8978494354966	33.8978494354966\\
56	0.10764	36.9063623358092	36.9063623358092\\
56	0.1113	40.2215390369718	40.2215390369718\\
56	0.11496	43.8433795389845	43.8433795389845\\
56	0.11862	47.7718838418471	47.7718838418471\\
56	0.12228	52.0070519455598	52.0070519455598\\
56	0.12594	56.5488838501225	56.5488838501225\\
56	0.1296	61.3973795555351	61.3973795555351\\
56	0.13326	66.5525390617977	66.5525390617977\\
56	0.13692	72.0143623689104	72.0143623689104\\
56	0.14058	77.7828494768732	77.7828494768732\\
56	0.14424	83.8580003856859	83.8580003856859\\
56	0.1479	90.2398150953486	90.2398150953486\\
56	0.15156	96.9282936058613	96.9282936058613\\
56	0.15522	103.923435917224	103.923435917224\\
56	0.15888	111.225242029437	111.225242029437\\
56	0.16254	118.8337119425	118.8337119425\\
56	0.1662	126.748845656412	126.748845656412\\
56	0.16986	134.970643171175	134.970643171175\\
56	0.17352	143.499104486788	143.499104486788\\
56	0.17718	152.334229603251	152.334229603251\\
56	0.18084	161.476018520563	161.476018520563\\
56	0.1845	170.924471238726	170.924471238726\\
56	0.18816	180.679587757739	180.679587757739\\
56	0.19182	190.741368077602	190.741368077602\\
56	0.19548	201.109812198315	201.109812198315\\
56	0.19914	211.784920119878	211.784920119878\\
56	0.2028	222.766691842291	222.766691842291\\
56	0.20646	234.055127365553	234.055127365553\\
56	0.21012	245.650226689666	245.650226689666\\
56	0.21378	257.551989814629	257.551989814629\\
56	0.21744	269.760416740442	269.760416740442\\
56	0.2211	282.275507467105	282.275507467105\\
56	0.22476	295.097261994618	295.097261994618\\
56	0.22842	308.225680322981	308.225680322981\\
56	0.23208	321.660762452193	321.660762452193\\
56	0.23574	335.402508382256	335.402508382256\\
56	0.2394	349.450918113169	349.450918113169\\
56	0.24306	363.805991644932	363.805991644932\\
56	0.24672	378.467728977545	378.467728977545\\
56	0.25038	393.436130111008	393.436130111008\\
56	0.25404	408.711195045321	408.711195045321\\
56	0.2577	424.292923780484	424.292923780484\\
56	0.26136	440.181316316497	440.181316316497\\
56	0.26502	456.37637265336	456.37637265336\\
56	0.26868	472.878092791073	472.878092791073\\
56	0.27234	489.686476729636	489.686476729636\\
56	0.276	506.801524469049	506.801524469049\\
56.375	0.093	26.9533925013348	26.9533925013348\\
56.375	0.09666	29.0365678713119	29.0365678713119\\
56.375	0.10032	31.426407042139	31.426407042139\\
56.375	0.10398	34.1229100138161	34.1229100138161\\
56.375	0.10764	37.1260767863432	37.1260767863432\\
56.375	0.1113	40.4359073597203	40.4359073597203\\
56.375	0.11496	44.0524017339474	44.0524017339474\\
56.375	0.11862	47.9755599090246	47.9755599090246\\
56.375	0.12228	52.2053818849518	52.2053818849518\\
56.375	0.12594	56.7418676617289	56.7418676617289\\
56.375	0.1296	61.5850172393561	61.5850172393561\\
56.375	0.13326	66.7348306178332	66.7348306178332\\
56.375	0.13692	72.1913077971604	72.1913077971604\\
56.375	0.14058	77.9544487773377	77.9544487773377\\
56.375	0.14424	84.0242535583649	84.0242535583649\\
56.375	0.1479	90.4007221402421	90.4007221402421\\
56.375	0.15156	97.0838545229693	97.0838545229693\\
56.375	0.15522	104.073650706547	104.073650706547\\
56.375	0.15888	111.370110690974	111.370110690974\\
56.375	0.16254	118.973234476251	118.973234476251\\
56.375	0.1662	126.883022062378	126.883022062378\\
56.375	0.16986	135.099473449356	135.099473449356\\
56.375	0.17352	143.622588637183	143.622588637183\\
56.375	0.17718	152.45236762586	152.45236762586\\
56.375	0.18084	161.588810415387	161.588810415387\\
56.375	0.1845	171.031917005765	171.031917005765\\
56.375	0.18816	180.781687396992	180.781687396992\\
56.375	0.19182	190.838121589069	190.838121589069\\
56.375	0.19548	201.201219581997	201.201219581997\\
56.375	0.19914	211.870981375774	211.870981375774\\
56.375	0.2028	222.847406970401	222.847406970401\\
56.375	0.20646	234.130496365879	234.130496365879\\
56.375	0.21012	245.720249562206	245.720249562206\\
56.375	0.21378	257.616666559383	257.616666559383\\
56.375	0.21744	269.819747357411	269.819747357411\\
56.375	0.2211	282.329491956288	282.329491956288\\
56.375	0.22476	295.145900356016	295.145900356016\\
56.375	0.22842	308.268972556593	308.268972556593\\
56.375	0.23208	321.69870855802	321.69870855802\\
56.375	0.23574	335.435108360298	335.435108360298\\
56.375	0.2394	349.478171963425	349.478171963425\\
56.375	0.24306	363.827899367403	363.827899367403\\
56.375	0.24672	378.48429057223	378.48429057223\\
56.375	0.25038	393.447345577908	393.447345577908\\
56.375	0.25404	408.717064384435	408.717064384435\\
56.375	0.2577	424.293446991813	424.293446991813\\
56.375	0.26136	440.17649340004	440.17649340004\\
56.375	0.26502	456.366203609117	456.366203609117\\
56.375	0.26868	472.862577619045	472.862577619045\\
56.375	0.27234	489.665615429822	489.665615429822\\
56.375	0.276	506.77531704145	506.77531704145\\
56.75	0.093	27.2054488938207	27.2054488938207\\
56.75	0.09666	29.2832781360123	29.2832781360123\\
56.75	0.10032	31.6677711790539	31.6677711790539\\
56.75	0.10398	34.3589280229455	34.3589280229455\\
56.75	0.10764	37.3567486676872	37.3567486676872\\
56.75	0.1113	40.6612331132787	40.6612331132787\\
56.75	0.11496	44.2723813597204	44.2723813597204\\
56.75	0.11862	48.190193407012	48.190193407012\\
56.75	0.12228	52.4146692551537	52.4146692551537\\
56.75	0.12594	56.9458089041454	56.9458089041454\\
56.75	0.1296	61.783612353987	61.783612353987\\
56.75	0.13326	66.9280796046787	66.9280796046787\\
56.75	0.13692	72.3792106562203	72.3792106562203\\
56.75	0.14058	78.1370055086121	78.1370055086121\\
56.75	0.14424	84.2014641618537	84.2014641618537\\
56.75	0.1479	90.5725866159455	90.5725866159455\\
56.75	0.15156	97.2503728708871	97.2503728708871\\
56.75	0.15522	104.234822926679	104.234822926679\\
56.75	0.15888	111.525936783321	111.525936783321\\
56.75	0.16254	119.123714440812	119.123714440812\\
56.75	0.1662	127.028155899154	127.028155899154\\
56.75	0.16986	135.239261158346	135.239261158346\\
56.75	0.17352	143.757030218388	143.757030218388\\
56.75	0.17718	152.581463079279	152.581463079279\\
56.75	0.18084	161.712559741021	161.712559741021\\
56.75	0.1845	171.150320203613	171.150320203613\\
56.75	0.18816	180.894744467055	180.894744467055\\
56.75	0.19182	190.945832531347	190.945832531347\\
56.75	0.19548	201.303584396488	201.303584396488\\
56.75	0.19914	211.96800006248	211.96800006248\\
56.75	0.2028	222.939079529322	222.939079529322\\
56.75	0.20646	234.216822797014	234.216822797014\\
56.75	0.21012	245.801229865556	245.801229865556\\
56.75	0.21378	257.692300734948	257.692300734948\\
56.75	0.21744	269.890035405189	269.890035405189\\
56.75	0.2211	282.394433876281	282.394433876281\\
56.75	0.22476	295.205496148223	295.205496148223\\
56.75	0.22842	308.323222221015	308.323222221015\\
56.75	0.23208	321.747612094657	321.747612094657\\
56.75	0.23574	335.478665769149	335.478665769149\\
56.75	0.2394	349.516383244491	349.516383244491\\
56.75	0.24306	363.860764520683	363.860764520683\\
56.75	0.24672	378.511809597725	378.511809597725\\
56.75	0.25038	393.469518475617	393.469518475617\\
56.75	0.25404	408.733891154359	408.733891154359\\
56.75	0.2577	424.304927633951	424.304927633951\\
56.75	0.26136	440.182627914393	440.182627914393\\
56.75	0.26502	456.366991995685	456.366991995685\\
56.75	0.26868	472.858019877827	472.858019877827\\
56.75	0.27234	489.655711560819	489.655711560819\\
56.75	0.276	506.76006704466	506.76006704466\\
57.125	0.093	27.4684627171166	27.4684627171166\\
57.125	0.09666	29.5409458315226	29.5409458315226\\
57.125	0.10032	31.9200927467787	31.9200927467787\\
57.125	0.10398	34.6059034628848	34.6059034628848\\
57.125	0.10764	37.598377979841	37.598377979841\\
57.125	0.1113	40.8975162976471	40.8975162976471\\
57.125	0.11496	44.5033184163032	44.5033184163032\\
57.125	0.11862	48.4157843358093	48.4157843358093\\
57.125	0.12228	52.6349140561655	52.6349140561655\\
57.125	0.12594	57.1607075773716	57.1607075773716\\
57.125	0.1296	61.9931648994278	61.9931648994278\\
57.125	0.13326	67.1322860223339	67.1322860223339\\
57.125	0.13692	72.5780709460901	72.5780709460901\\
57.125	0.14058	78.3305196706963	78.3305196706963\\
57.125	0.14424	84.3896321961525	84.3896321961525\\
57.125	0.1479	90.7554085224587	90.7554085224587\\
57.125	0.15156	97.4278486496149	97.4278486496149\\
57.125	0.15522	104.406952577621	104.406952577621\\
57.125	0.15888	111.692720306477	111.692720306477\\
57.125	0.16254	119.285151836184	119.285151836184\\
57.125	0.1662	127.18424716674	127.18424716674\\
57.125	0.16986	135.390006298146	135.390006298146\\
57.125	0.17352	143.902429230402	143.902429230402\\
57.125	0.17718	152.721515963509	152.721515963509\\
57.125	0.18084	161.847266497465	161.847266497465\\
57.125	0.1845	171.279680832271	171.279680832271\\
57.125	0.18816	181.018758967928	181.018758967928\\
57.125	0.19182	191.064500904434	191.064500904434\\
57.125	0.19548	201.41690664179	201.41690664179\\
57.125	0.19914	212.075976179996	212.075976179996\\
57.125	0.2028	223.041709519053	223.041709519053\\
57.125	0.20646	234.314106658959	234.314106658959\\
57.125	0.21012	245.893167599716	245.893167599716\\
57.125	0.21378	257.778892341322	257.778892341322\\
57.125	0.21744	269.971280883778	269.971280883778\\
57.125	0.2211	282.470333227085	282.470333227085\\
57.125	0.22476	295.276049371241	295.276049371241\\
57.125	0.22842	308.388429316247	308.388429316247\\
57.125	0.23208	321.807473062104	321.807473062104\\
57.125	0.23574	335.53318060881	335.53318060881\\
57.125	0.2394	349.565551956367	349.565551956367\\
57.125	0.24306	363.904587104773	363.904587104773\\
57.125	0.24672	378.550286054029	378.550286054029\\
57.125	0.25038	393.502648804136	393.502648804136\\
57.125	0.25404	408.761675355092	408.761675355092\\
57.125	0.2577	424.327365706899	424.327365706899\\
57.125	0.26136	440.199719859555	440.199719859555\\
57.125	0.26502	456.378737813062	456.378737813062\\
57.125	0.26868	472.864419567418	472.864419567418\\
57.125	0.27234	489.656765122625	489.656765122625\\
57.125	0.276	506.755774478681	506.755774478681\\
57.5	0.093	27.7424339712224	27.7424339712224\\
57.5	0.09666	29.8095709578429	29.8095709578429\\
57.5	0.10032	32.1833717453135	32.1833717453135\\
57.5	0.10398	34.8638363336341	34.8638363336341\\
57.5	0.10764	37.8509647228047	37.8509647228047\\
57.5	0.1113	41.1447569128253	41.1447569128253\\
57.5	0.11496	44.7452129036959	44.7452129036959\\
57.5	0.11862	48.6523326954166	48.6523326954166\\
57.5	0.12228	52.8661162879872	52.8661162879872\\
57.5	0.12594	57.3865636814079	57.3865636814079\\
57.5	0.1296	62.2136748756785	62.2136748756785\\
57.5	0.13326	67.3474498707992	67.3474498707992\\
57.5	0.13692	72.7878886667698	72.7878886667698\\
57.5	0.14058	78.5349912635905	78.5349912635905\\
57.5	0.14424	84.5887576612612	84.5887576612612\\
57.5	0.1479	90.9491878597819	90.9491878597819\\
57.5	0.15156	97.6162818591526	97.6162818591526\\
57.5	0.15522	104.590039659373	104.590039659373\\
57.5	0.15888	111.870461260444	111.870461260444\\
57.5	0.16254	119.457546662365	119.457546662365\\
57.5	0.1662	127.351295865136	127.351295865136\\
57.5	0.16986	135.551708868756	135.551708868756\\
57.5	0.17352	144.058785673227	144.058785673227\\
57.5	0.17718	152.872526278548	152.872526278548\\
57.5	0.18084	161.992930684719	161.992930684719\\
57.5	0.1845	171.419998891739	171.419998891739\\
57.5	0.18816	181.15373089961	181.15373089961\\
57.5	0.19182	191.194126708331	191.194126708331\\
57.5	0.19548	201.541186317902	201.541186317902\\
57.5	0.19914	212.194909728323	212.194909728323\\
57.5	0.2028	223.155296939594	223.155296939594\\
57.5	0.20646	234.422347951714	234.422347951714\\
57.5	0.21012	245.996062764685	245.996062764685\\
57.5	0.21378	257.876441378506	257.876441378506\\
57.5	0.21744	270.063483793177	270.063483793177\\
57.5	0.2211	282.557190008698	282.557190008698\\
57.5	0.22476	295.357560025069	295.357560025069\\
57.5	0.22842	308.464593842289	308.464593842289\\
57.5	0.23208	321.87829146036	321.87829146036\\
57.5	0.23574	335.598652879281	335.598652879281\\
57.5	0.2394	349.625678099052	349.625678099052\\
57.5	0.24306	363.959367119673	363.959367119673\\
57.5	0.24672	378.599719941144	378.599719941144\\
57.5	0.25038	393.546736563465	393.546736563465\\
57.5	0.25404	408.800416986636	408.800416986636\\
57.5	0.2577	424.360761210657	424.360761210657\\
57.5	0.26136	440.227769235528	440.227769235528\\
57.5	0.26502	456.401441061249	456.401441061249\\
57.5	0.26868	472.88177668782	472.88177668782\\
57.5	0.27234	489.668776115241	489.668776115241\\
57.5	0.276	506.762439343512	506.762439343512\\
57.875	0.093	28.0273626561381	28.0273626561381\\
57.875	0.09666	30.0891535149731	30.0891535149731\\
57.875	0.10032	32.4576081746582	32.4576081746582\\
57.875	0.10398	35.1327266351933	35.1327266351933\\
57.875	0.10764	38.1145088965784	38.1145088965784\\
57.875	0.1113	41.4029549588135	41.4029549588135\\
57.875	0.11496	44.9980648218986	44.9980648218986\\
57.875	0.11862	48.8998384858338	48.8998384858338\\
57.875	0.12228	53.1082759506189	53.1082759506189\\
57.875	0.12594	57.6233772162541	57.6233772162541\\
57.875	0.1296	62.4451422827392	62.4451422827392\\
57.875	0.13326	67.5735711500743	67.5735711500743\\
57.875	0.13692	73.0086638182595	73.0086638182595\\
57.875	0.14058	78.7504202872947	78.7504202872947\\
57.875	0.14424	84.7988405571799	84.7988405571799\\
57.875	0.1479	91.153924627915	91.153924627915\\
57.875	0.15156	97.8156724995002	97.8156724995002\\
57.875	0.15522	104.784084171935	104.784084171935\\
57.875	0.15888	112.059159645221	112.059159645221\\
57.875	0.16254	119.640898919356	119.640898919356\\
57.875	0.1662	127.529301994341	127.529301994341\\
57.875	0.16986	135.724368870176	135.724368870176\\
57.875	0.17352	144.226099546862	144.226099546862\\
57.875	0.17718	153.034494024397	153.034494024397\\
57.875	0.18084	162.149552302782	162.149552302782\\
57.875	0.1845	171.571274382017	171.571274382017\\
57.875	0.18816	181.299660262103	181.299660262103\\
57.875	0.19182	191.334709943038	191.334709943038\\
57.875	0.19548	201.676423424823	201.676423424823\\
57.875	0.19914	212.324800707459	212.324800707459\\
57.875	0.2028	223.279841790944	223.279841790944\\
57.875	0.20646	234.541546675279	234.541546675279\\
57.875	0.21012	246.109915360465	246.109915360465\\
57.875	0.21378	257.9849478465	257.9849478465\\
57.875	0.21744	270.166644133385	270.166644133385\\
57.875	0.2211	282.655004221121	282.655004221121\\
57.875	0.22476	295.450028109706	295.450028109706\\
57.875	0.22842	308.551715799141	308.551715799141\\
57.875	0.23208	321.960067289427	321.960067289427\\
57.875	0.23574	335.675082580562	335.675082580562\\
57.875	0.2394	349.696761672548	349.696761672548\\
57.875	0.24306	364.025104565383	364.025104565383\\
57.875	0.24672	378.660111259068	378.660111259068\\
57.875	0.25038	393.601781753604	393.601781753604\\
57.875	0.25404	408.850116048989	408.850116048989\\
57.875	0.2577	424.405114145225	424.405114145225\\
57.875	0.26136	440.26677604231	440.26677604231\\
57.875	0.26502	456.435101740246	456.435101740246\\
57.875	0.26868	472.910091239031	472.910091239031\\
57.875	0.27234	489.691744538667	489.691744538667\\
57.875	0.276	506.780061639152	506.780061639152\\
58.25	0.093	28.3232487718637	28.3232487718637\\
58.25	0.09666	30.3796935029133	30.3796935029133\\
58.25	0.10032	32.7428020348128	32.7428020348128\\
58.25	0.10398	35.4125743675624	35.4125743675624\\
58.25	0.10764	38.389010501162	38.389010501162\\
58.25	0.1113	41.6721104356116	41.6721104356116\\
58.25	0.11496	45.2618741709112	45.2618741709112\\
58.25	0.11862	49.1583017070608	49.1583017070608\\
58.25	0.12228	53.3613930440605	53.3613930440605\\
58.25	0.12594	57.8711481819101	57.8711481819101\\
58.25	0.1296	62.6875671206097	62.6875671206097\\
58.25	0.13326	67.8106498601593	67.8106498601593\\
58.25	0.13692	73.2403964005591	73.2403964005591\\
58.25	0.14058	78.9768067418088	78.9768067418088\\
58.25	0.14424	85.0198808839084	85.0198808839084\\
58.25	0.1479	91.3696188268581	91.3696188268581\\
58.25	0.15156	98.0260205706577	98.0260205706577\\
58.25	0.15522	104.989086115308	104.989086115308\\
58.25	0.15888	112.258815460807	112.258815460807\\
58.25	0.16254	119.835208607157	119.835208607157\\
58.25	0.1662	127.718265554357	127.718265554357\\
58.25	0.16986	135.907986302406	135.907986302406\\
58.25	0.17352	144.404370851306	144.404370851306\\
58.25	0.17718	153.207419201056	153.207419201056\\
58.25	0.18084	162.317131351656	162.317131351656\\
58.25	0.1845	171.733507303105	171.733507303105\\
58.25	0.18816	181.456547055405	181.456547055405\\
58.25	0.19182	191.486250608555	191.486250608555\\
58.25	0.19548	201.822617962555	201.822617962555\\
58.25	0.19914	212.465649117405	212.465649117405\\
58.25	0.2028	223.415344073105	223.415344073105\\
58.25	0.20646	234.671702829654	234.671702829654\\
58.25	0.21012	246.234725387054	246.234725387054\\
58.25	0.21378	258.104411745304	258.104411745304\\
58.25	0.21744	270.280761904404	270.280761904404\\
58.25	0.2211	282.763775864354	282.763775864354\\
58.25	0.22476	295.553453625154	295.553453625154\\
58.25	0.22842	308.649795186803	308.649795186803\\
58.25	0.23208	322.052800549303	322.052800549303\\
58.25	0.23574	335.762469712653	335.762469712653\\
58.25	0.2394	349.778802676853	349.778802676853\\
58.25	0.24306	364.101799441903	364.101799441903\\
58.25	0.24672	378.731460007803	378.731460007803\\
58.25	0.25038	393.667784374553	393.667784374553\\
58.25	0.25404	408.910772542153	408.910772542153\\
58.25	0.2577	424.460424510603	424.460424510603\\
58.25	0.26136	440.316740279903	440.316740279903\\
58.25	0.26502	456.479719850053	456.479719850053\\
58.25	0.26868	472.949363221052	472.949363221052\\
58.25	0.27234	489.725670392903	489.725670392903\\
58.25	0.276	506.808641365602	506.808641365602\\
58.625	0.093	28.6300923183992	28.6300923183992\\
58.625	0.09666	30.6811909216633	30.6811909216633\\
58.625	0.10032	33.0389533257774	33.0389533257774\\
58.625	0.10398	35.7033795307415	35.7033795307415\\
58.625	0.10764	38.6744695365556	38.6744695365556\\
58.625	0.1113	41.9522233432196	41.9522233432196\\
58.625	0.11496	45.5366409507338	45.5366409507338\\
58.625	0.11862	49.4277223590979	49.4277223590979\\
58.625	0.12228	53.625467568312	53.625467568312\\
58.625	0.12594	58.1298765783762	58.1298765783762\\
58.625	0.1296	62.9409493892903	62.9409493892903\\
58.625	0.13326	68.0586860010544	68.0586860010544\\
58.625	0.13692	73.4830864136686	73.4830864136686\\
58.625	0.14058	79.2141506271327	79.2141506271327\\
58.625	0.14424	85.251878641447	85.251878641447\\
58.625	0.1479	91.5962704566111	91.5962704566111\\
58.625	0.15156	98.2473260726252	98.2473260726252\\
58.625	0.15522	105.20504548949	105.20504548949\\
58.625	0.15888	112.469428707204	112.469428707204\\
58.625	0.16254	120.040475725768	120.040475725768\\
58.625	0.1662	127.918186545182	127.918186545182\\
58.625	0.16986	136.102561165446	136.102561165446\\
58.625	0.17352	144.593599586561	144.593599586561\\
58.625	0.17718	153.391301808525	153.391301808525\\
58.625	0.18084	162.495667831339	162.495667831339\\
58.625	0.1845	171.906697655003	171.906697655003\\
58.625	0.18816	181.624391279518	181.624391279518\\
58.625	0.19182	191.648748704882	191.648748704882\\
58.625	0.19548	201.979769931096	201.979769931096\\
58.625	0.19914	212.617454958161	212.617454958161\\
58.625	0.2028	223.561803786075	223.561803786075\\
58.625	0.20646	234.812816414839	234.812816414839\\
58.625	0.21012	246.370492844454	246.370492844454\\
58.625	0.21378	258.234833074918	258.234833074918\\
58.625	0.21744	270.405837106232	270.405837106232\\
58.625	0.2211	282.883504938397	282.883504938397\\
58.625	0.22476	295.667836571411	295.667836571411\\
58.625	0.22842	308.758832005275	308.758832005275\\
58.625	0.23208	322.15649123999	322.15649123999\\
58.625	0.23574	335.860814275554	335.860814275554\\
58.625	0.2394	349.871801111968	349.871801111968\\
58.625	0.24306	364.189451749233	364.189451749233\\
58.625	0.24672	378.813766187347	378.813766187347\\
58.625	0.25038	393.744744426312	393.744744426312\\
58.625	0.25404	408.982386466126	408.982386466126\\
58.625	0.2577	424.526692306791	424.526692306791\\
58.625	0.26136	440.377661948305	440.377661948305\\
58.625	0.26502	456.535295390669	456.535295390669\\
58.625	0.26868	472.999592633884	472.999592633884\\
58.625	0.27234	489.770553677948	489.770553677948\\
58.625	0.276	506.848178522863	506.848178522863\\
59	0.093	28.9478932957447	28.9478932957447\\
59	0.09666	30.9936457712232	30.9936457712232\\
59	0.10032	33.3460620475518	33.3460620475518\\
59	0.10398	36.0051421247304	36.0051421247304\\
59	0.10764	38.970886002759	38.970886002759\\
59	0.1113	42.2432936816376	42.2432936816376\\
59	0.11496	45.8223651613662	45.8223651613662\\
59	0.11862	49.7081004419448	49.7081004419448\\
59	0.12228	53.9004995233734	53.9004995233734\\
59	0.12594	58.399562405652	58.399562405652\\
59	0.1296	63.2052890887807	63.2052890887807\\
59	0.13326	68.3176795727593	68.3176795727593\\
59	0.13692	73.7367338575879	73.7367338575879\\
59	0.14058	79.4624519432666	79.4624519432666\\
59	0.14424	85.4948338297953	85.4948338297953\\
59	0.1479	91.833879517174	91.833879517174\\
59	0.15156	98.4795890054026	98.4795890054026\\
59	0.15522	105.431962294481	105.431962294481\\
59	0.15888	112.69099938441	112.69099938441\\
59	0.16254	120.256700275189	120.256700275189\\
59	0.1662	128.129064966817	128.129064966817\\
59	0.16986	136.308093459296	136.308093459296\\
59	0.17352	144.793785752625	144.793785752625\\
59	0.17718	153.586141846804	153.586141846804\\
59	0.18084	162.685161741832	162.685161741832\\
59	0.1845	172.090845437711	172.090845437711\\
59	0.18816	181.80319293444	181.80319293444\\
59	0.19182	191.822204232019	191.822204232019\\
59	0.19548	202.147879330448	202.147879330448\\
59	0.19914	212.780218229726	212.780218229726\\
59	0.2028	223.719220929855	223.719220929855\\
59	0.20646	234.964887430834	234.964887430834\\
59	0.21012	246.517217732663	246.517217732663\\
59	0.21378	258.376211835342	258.376211835342\\
59	0.21744	270.54186973887	270.54186973887\\
59	0.2211	283.014191443249	283.014191443249\\
59	0.22476	295.793176948478	295.793176948478\\
59	0.22842	308.878826254557	308.878826254557\\
59	0.23208	322.271139361486	322.271139361486\\
59	0.23574	335.970116269265	335.970116269265\\
59	0.2394	349.975756977894	349.975756977894\\
59	0.24306	364.288061487373	364.288061487373\\
59	0.24672	378.907029797701	378.907029797701\\
59	0.25038	393.832661908881	393.832661908881\\
59	0.25404	409.064957820909	409.064957820909\\
59	0.2577	424.603917533788	424.603917533788\\
59	0.26136	440.449541047517	440.449541047517\\
59	0.26502	456.601828362096	456.601828362096\\
59	0.26868	473.060779477525	473.060779477525\\
59	0.27234	489.826394393804	489.826394393804\\
59	0.276	506.898673110933	506.898673110933\\
59.375	0.093	29.2766517039001	29.2766517039001\\
59.375	0.09666	31.3170580515931	31.3170580515931\\
59.375	0.10032	33.6641282001362	33.6641282001362\\
59.375	0.10398	36.3178621495292	36.3178621495292\\
59.375	0.10764	39.2782598997723	39.2782598997723\\
59.375	0.1113	42.5453214508654	42.5453214508654\\
59.375	0.11496	46.1190468028085	46.1190468028085\\
59.375	0.11862	49.9994359556016	49.9994359556016\\
59.375	0.12228	54.1864889092448	54.1864889092448\\
59.375	0.12594	58.6802056637379	58.6802056637379\\
59.375	0.1296	63.480586219081	63.480586219081\\
59.375	0.13326	68.5876305752741	68.5876305752741\\
59.375	0.13692	74.0013387323172	74.0013387323172\\
59.375	0.14058	79.7217106902104	79.7217106902104\\
59.375	0.14424	85.7487464489536	85.7487464489536\\
59.375	0.1479	92.0824460085468	92.0824460085468\\
59.375	0.15156	98.7228093689899	98.7228093689899\\
59.375	0.15522	105.669836530283	105.669836530283\\
59.375	0.15888	112.923527492426	112.923527492426\\
59.375	0.16254	120.48388225542	120.48388225542\\
59.375	0.1662	128.350900819263	128.350900819263\\
59.375	0.16986	136.524583183956	136.524583183956\\
59.375	0.17352	145.004929349499	145.004929349499\\
59.375	0.17718	153.791939315892	153.791939315892\\
59.375	0.18084	162.885613083136	162.885613083136\\
59.375	0.1845	172.285950651229	172.285950651229\\
59.375	0.18816	181.992952020172	181.992952020172\\
59.375	0.19182	192.006617189966	192.006617189966\\
59.375	0.19548	202.326946160609	202.326946160609\\
59.375	0.19914	212.953938932102	212.953938932102\\
59.375	0.2028	223.887595504445	223.887595504445\\
59.375	0.20646	235.127915877639	235.127915877639\\
59.375	0.21012	246.674900051682	246.674900051682\\
59.375	0.21378	258.528548026575	258.528548026575\\
59.375	0.21744	270.688859802319	270.688859802319\\
59.375	0.2211	283.155835378912	283.155835378912\\
59.375	0.22476	295.929474756355	295.929474756355\\
59.375	0.22842	309.009777934649	309.009777934649\\
59.375	0.23208	322.396744913792	322.396744913792\\
59.375	0.23574	336.090375693786	336.090375693786\\
59.375	0.2394	350.090670274629	350.090670274629\\
59.375	0.24306	364.397628656322	364.397628656322\\
59.375	0.24672	379.011250838866	379.011250838866\\
59.375	0.25038	393.931536822259	393.931536822259\\
59.375	0.25404	409.158486606502	409.158486606502\\
59.375	0.2577	424.692100191596	424.692100191596\\
59.375	0.26136	440.532377577539	440.532377577539\\
59.375	0.26502	456.679318764333	456.679318764333\\
59.375	0.26868	473.132923751976	473.132923751976\\
59.375	0.27234	489.89319254047	489.89319254047\\
59.375	0.276	506.960125129813	506.960125129813\\
59.75	0.093	29.6163675428654	29.6163675428654\\
59.75	0.09666	31.6514277627729	31.6514277627729\\
59.75	0.10032	33.9931517835305	33.9931517835305\\
59.75	0.10398	36.641539605138	36.641539605138\\
59.75	0.10764	39.5965912275956	39.5965912275956\\
59.75	0.1113	42.8583066509032	42.8583066509032\\
59.75	0.11496	46.4266858750608	46.4266858750608\\
59.75	0.11862	50.3017289000684	50.3017289000684\\
59.75	0.12228	54.483435725926	54.483435725926\\
59.75	0.12594	58.9718063526336	58.9718063526336\\
59.75	0.1296	63.7668407801912	63.7668407801912\\
59.75	0.13326	68.8685390085988	68.8685390085988\\
59.75	0.13692	74.2769010378565	74.2769010378565\\
59.75	0.14058	79.9919268679642	79.9919268679642\\
59.75	0.14424	86.0136164989219	86.0136164989219\\
59.75	0.1479	92.3419699307295	92.3419699307295\\
59.75	0.15156	98.9769871633872	98.9769871633872\\
59.75	0.15522	105.918668196895	105.918668196895\\
59.75	0.15888	113.167013031253	113.167013031253\\
59.75	0.16254	120.72202166646	120.72202166646\\
59.75	0.1662	128.583694102518	128.583694102518\\
59.75	0.16986	136.752030339426	136.752030339426\\
59.75	0.17352	145.227030377183	145.227030377183\\
59.75	0.17718	154.008694215791	154.008694215791\\
59.75	0.18084	163.097021855249	163.097021855249\\
59.75	0.1845	172.492013295557	172.492013295557\\
59.75	0.18816	182.193668536714	182.193668536714\\
59.75	0.19182	192.201987578722	192.201987578722\\
59.75	0.19548	202.51697042158	202.51697042158\\
59.75	0.19914	213.138617065288	213.138617065288\\
59.75	0.2028	224.066927509846	224.066927509846\\
59.75	0.20646	235.301901755254	235.301901755254\\
59.75	0.21012	246.843539801511	246.843539801511\\
59.75	0.21378	258.691841648619	258.691841648619\\
59.75	0.21744	270.846807296577	270.846807296577\\
59.75	0.2211	283.308436745385	283.308436745385\\
59.75	0.22476	296.076729995043	296.076729995043\\
59.75	0.22842	309.15168704555	309.15168704555\\
59.75	0.23208	322.533307896908	322.533307896908\\
59.75	0.23574	336.221592549116	336.221592549116\\
59.75	0.2394	350.216541002174	350.216541002174\\
59.75	0.24306	364.518153256082	364.518153256082\\
59.75	0.24672	379.12642931084	379.12642931084\\
59.75	0.25038	394.041369166448	394.041369166448\\
59.75	0.25404	409.262972822906	409.262972822906\\
59.75	0.2577	424.791240280214	424.791240280214\\
59.75	0.26136	440.626171538372	440.626171538372\\
59.75	0.26502	456.767766597379	456.767766597379\\
59.75	0.26868	473.216025457237	473.216025457237\\
59.75	0.27234	489.970948117945	489.970948117945\\
59.75	0.276	507.032534579503	507.032534579503\\
60.125	0.093	29.9670408126406	29.9670408126406\\
60.125	0.09666	31.9967549047626	31.9967549047626\\
60.125	0.10032	34.3331327977347	34.3331327977347\\
60.125	0.10398	36.9761744915568	36.9761744915568\\
60.125	0.10764	39.9258799862288	39.9258799862288\\
60.125	0.1113	43.1822492817509	43.1822492817509\\
60.125	0.11496	46.745282378123	46.745282378123\\
60.125	0.11862	50.6149792753451	50.6149792753451\\
60.125	0.12228	54.7913399734172	54.7913399734172\\
60.125	0.12594	59.2743644723393	59.2743644723393\\
60.125	0.1296	64.0640527721114	64.0640527721114\\
60.125	0.13326	69.1604048727335	69.1604048727335\\
60.125	0.13692	74.5634207742057	74.5634207742057\\
60.125	0.14058	80.2731004765278	80.2731004765278\\
60.125	0.14424	86.2894439797	86.2894439797\\
60.125	0.1479	92.6124512837222	92.6124512837222\\
60.125	0.15156	99.2421223885943	99.2421223885943\\
60.125	0.15522	106.178457294317	106.178457294317\\
60.125	0.15888	113.421456000889	113.421456000889\\
60.125	0.16254	120.971118508311	120.971118508311\\
60.125	0.1662	128.827444816583	128.827444816583\\
60.125	0.16986	136.990434925705	136.990434925705\\
60.125	0.17352	145.460088835678	145.460088835678\\
60.125	0.17718	154.2364065465	154.2364065465\\
60.125	0.18084	163.319388058172	163.319388058172\\
60.125	0.1845	172.709033370694	172.709033370694\\
60.125	0.18816	182.405342484067	182.405342484067\\
60.125	0.19182	192.408315398289	192.408315398289\\
60.125	0.19548	202.717952113361	202.717952113361\\
60.125	0.19914	213.334252629283	213.334252629283\\
60.125	0.2028	224.257216946056	224.257216946056\\
60.125	0.20646	235.486845063678	235.486845063678\\
60.125	0.21012	247.02313698215	247.02313698215\\
60.125	0.21378	258.866092701473	258.866092701473\\
60.125	0.21744	271.015712221645	271.015712221645\\
60.125	0.2211	283.471995542667	283.471995542667\\
60.125	0.22476	296.23494266454	296.23494266454\\
60.125	0.22842	309.304553587262	309.304553587262\\
60.125	0.23208	322.680828310834	322.680828310834\\
60.125	0.23574	336.363766835257	336.363766835257\\
60.125	0.2394	350.353369160529	350.353369160529\\
60.125	0.24306	364.649635286651	364.649635286651\\
60.125	0.24672	379.252565213624	379.252565213624\\
60.125	0.25038	394.162158941446	394.162158941446\\
60.125	0.25404	409.378416470119	409.378416470119\\
60.125	0.2577	424.901337799641	424.901337799641\\
60.125	0.26136	440.730922930014	440.730922930014\\
60.125	0.26502	456.867171861236	456.867171861236\\
60.125	0.26868	473.310084593308	473.310084593308\\
60.125	0.27234	490.059661126231	490.059661126231\\
60.125	0.276	507.115901460003	507.115901460003\\
60.5	0.093	30.3286715132257	30.3286715132257\\
60.5	0.09666	32.3530394775622	32.3530394775622\\
60.5	0.10032	34.6840712427488	34.6840712427488\\
60.5	0.10398	37.3217668087853	37.3217668087853\\
60.5	0.10764	40.2661261756719	40.2661261756719\\
60.5	0.1113	43.5171493434085	43.5171493434085\\
60.5	0.11496	47.0748363119951	47.0748363119951\\
60.5	0.11862	50.9391870814317	50.9391870814317\\
60.5	0.12228	55.1102016517183	55.1102016517183\\
60.5	0.12594	59.5878800228549	59.5878800228549\\
60.5	0.1296	64.3722221948415	64.3722221948415\\
60.5	0.13326	69.4632281676781	69.4632281676781\\
60.5	0.13692	74.8608979413647	74.8608979413647\\
60.5	0.14058	80.5652315159014	80.5652315159014\\
60.5	0.14424	86.5762288912881	86.5762288912881\\
60.5	0.1479	92.8938900675247	92.8938900675247\\
60.5	0.15156	99.5182150446113	99.5182150446113\\
60.5	0.15522	106.449203822548	106.449203822548\\
60.5	0.15888	113.686856401335	113.686856401335\\
60.5	0.16254	121.231172780971	121.231172780971\\
60.5	0.1662	129.082152961458	129.082152961458\\
60.5	0.16986	137.239796942795	137.239796942795\\
60.5	0.17352	145.704104724982	145.704104724982\\
60.5	0.17718	154.475076308018	154.475076308018\\
60.5	0.18084	163.552711691905	163.552711691905\\
60.5	0.1845	172.937010876642	172.937010876642\\
60.5	0.18816	182.627973862229	182.627973862229\\
60.5	0.19182	192.625600648665	192.625600648665\\
60.5	0.19548	202.929891235952	202.929891235952\\
60.5	0.19914	213.540845624089	213.540845624089\\
60.5	0.2028	224.458463813076	224.458463813076\\
60.5	0.20646	235.682745802913	235.682745802913\\
60.5	0.21012	247.213691593599	247.213691593599\\
60.5	0.21378	259.051301185136	259.051301185136\\
60.5	0.21744	271.195574577523	271.195574577523\\
60.5	0.2211	283.64651177076	283.64651177076\\
60.5	0.22476	296.404112764847	296.404112764847\\
60.5	0.22842	309.468377559783	309.468377559783\\
60.5	0.23208	322.83930615557	322.83930615557\\
60.5	0.23574	336.516898552207	336.516898552207\\
60.5	0.2394	350.501154749694	350.501154749694\\
60.5	0.24306	364.792074748031	364.792074748031\\
60.5	0.24672	379.389658547218	379.389658547218\\
60.5	0.25038	394.293906147255	394.293906147255\\
60.5	0.25404	409.504817548141	409.504817548141\\
60.5	0.2577	425.022392749878	425.022392749878\\
60.5	0.26136	440.846631752465	440.846631752465\\
60.5	0.26502	456.977534555902	456.977534555902\\
60.5	0.26868	473.415101160189	473.415101160189\\
60.5	0.27234	490.159331565326	490.159331565326\\
60.5	0.276	507.210225771313	507.210225771313\\
60.875	0.093	30.7012596446208	30.7012596446208\\
60.875	0.09666	32.7202814811718	32.7202814811718\\
60.875	0.10032	35.0459671185728	35.0459671185728\\
60.875	0.10398	37.6783165568239	37.6783165568239\\
60.875	0.10764	40.617329795925	40.617329795925\\
60.875	0.1113	43.863006835876	43.863006835876\\
60.875	0.11496	47.4153476766771	47.4153476766771\\
60.875	0.11862	51.2743523183282	51.2743523183282\\
60.875	0.12228	55.4400207608293	55.4400207608293\\
60.875	0.12594	59.9123530041804	59.9123530041804\\
60.875	0.1296	64.6913490483815	64.6913490483815\\
60.875	0.13326	69.7770088934326	69.7770088934326\\
60.875	0.13692	75.1693325393337	75.1693325393337\\
60.875	0.14058	80.8683199860849	80.8683199860849\\
60.875	0.14424	86.8739712336861	86.8739712336861\\
60.875	0.1479	93.1862862821372	93.1862862821372\\
60.875	0.15156	99.8052651314383	99.8052651314383\\
60.875	0.15522	106.73090778159	106.73090778159\\
60.875	0.15888	113.963214232591	113.963214232591\\
60.875	0.16254	121.502184484442	121.502184484442\\
60.875	0.1662	129.347818537143	129.347818537143\\
60.875	0.16986	137.500116390694	137.500116390694\\
60.875	0.17352	145.959078045096	145.959078045096\\
60.875	0.17718	154.724703500347	154.724703500347\\
60.875	0.18084	163.796992756448	163.796992756448\\
60.875	0.1845	173.175945813399	173.175945813399\\
60.875	0.18816	182.8615626712	182.8615626712\\
60.875	0.19182	192.853843329852	192.853843329852\\
60.875	0.19548	203.152787789353	203.152787789353\\
60.875	0.19914	213.758396049704	213.758396049704\\
60.875	0.2028	224.670668110906	224.670668110906\\
60.875	0.20646	235.889603972957	235.889603972957\\
60.875	0.21012	247.415203635858	247.415203635858\\
60.875	0.21378	259.247467099609	259.247467099609\\
60.875	0.21744	271.386394364211	271.386394364211\\
60.875	0.2211	283.831985429662	283.831985429662\\
60.875	0.22476	296.584240295963	296.584240295963\\
60.875	0.22842	309.643158963115	309.643158963115\\
60.875	0.23208	323.008741431116	323.008741431116\\
60.875	0.23574	336.680987699968	336.680987699968\\
60.875	0.2394	350.659897769669	350.659897769669\\
60.875	0.24306	364.94547164022	364.94547164022\\
60.875	0.24672	379.537709311622	379.537709311622\\
60.875	0.25038	394.436610783873	394.436610783873\\
60.875	0.25404	409.642176056974	409.642176056974\\
60.875	0.2577	425.154405130926	425.154405130926\\
60.875	0.26136	440.973298005727	440.973298005727\\
60.875	0.26502	457.098854681379	457.098854681379\\
60.875	0.26868	473.53107515788	473.53107515788\\
60.875	0.27234	490.269959435231	490.269959435231\\
60.875	0.276	507.315507513433	507.315507513433\\
61.25	0.093	31.0848052068258	31.0848052068258\\
61.25	0.09666	33.0984809155913	33.0984809155913\\
61.25	0.10032	35.4188204252068	35.4188204252068\\
61.25	0.10398	38.0458237356724	38.0458237356724\\
61.25	0.10764	40.9794908469879	40.9794908469879\\
61.25	0.1113	44.2198217591535	44.2198217591535\\
61.25	0.11496	47.7668164721691	47.7668164721691\\
61.25	0.11862	51.6204749860347	51.6204749860347\\
61.25	0.12228	55.7807973007503	55.7807973007503\\
61.25	0.12594	60.2477834163158	60.2477834163158\\
61.25	0.1296	65.0214333327314	65.0214333327314\\
61.25	0.13326	70.101747049997	70.101747049997\\
61.25	0.13692	75.4887245681126	75.4887245681126\\
61.25	0.14058	81.1823658870783	81.1823658870783\\
61.25	0.14424	87.182671006894	87.182671006894\\
61.25	0.1479	93.4896399275596	93.4896399275596\\
61.25	0.15156	100.103272649075	100.103272649075\\
61.25	0.15522	107.023569171441	107.023569171441\\
61.25	0.15888	114.250529494657	114.250529494657\\
61.25	0.16254	121.784153618722	121.784153618722\\
61.25	0.1662	129.624441543638	129.624441543638\\
61.25	0.16986	137.771393269404	137.771393269404\\
61.25	0.17352	146.225008796019	146.225008796019\\
61.25	0.17718	154.985288123485	154.985288123485\\
61.25	0.18084	164.052231251801	164.052231251801\\
61.25	0.1845	173.425838180967	173.425838180967\\
61.25	0.18816	183.106108910982	183.106108910982\\
61.25	0.19182	193.093043441848	193.093043441848\\
61.25	0.19548	203.386641773564	203.386641773564\\
61.25	0.19914	213.98690390613	213.98690390613\\
61.25	0.2028	224.893829839545	224.893829839545\\
61.25	0.20646	236.107419573811	236.107419573811\\
61.25	0.21012	247.627673108927	247.627673108927\\
61.25	0.21378	259.454590444893	259.454590444893\\
61.25	0.21744	271.588171581708	271.588171581708\\
61.25	0.2211	284.028416519374	284.028416519374\\
61.25	0.22476	296.77532525789	296.77532525789\\
61.25	0.22842	309.828897797256	309.828897797256\\
61.25	0.23208	323.189134137472	323.189134137472\\
61.25	0.23574	336.856034278538	336.856034278538\\
61.25	0.2394	350.829598220454	350.829598220454\\
61.25	0.24306	365.109825963219	365.109825963219\\
61.25	0.24672	379.696717506835	379.696717506835\\
61.25	0.25038	394.590272851301	394.590272851301\\
61.25	0.25404	409.790491996617	409.790491996617\\
61.25	0.2577	425.297374942783	425.297374942783\\
61.25	0.26136	441.110921689799	441.110921689799\\
61.25	0.26502	457.231132237665	457.231132237665\\
61.25	0.26868	473.658006586381	473.658006586381\\
61.25	0.27234	490.391544735947	490.391544735947\\
61.25	0.276	507.431746686363	507.431746686363\\
61.625	0.093	31.4793081998407	31.4793081998407\\
61.625	0.09666	33.4876377808207	33.4876377808207\\
61.625	0.10032	35.8026311626507	35.8026311626507\\
61.625	0.10398	38.4242883453308	38.4242883453308\\
61.625	0.10764	41.3526093288608	41.3526093288608\\
61.625	0.1113	44.5875941132409	44.5875941132409\\
61.625	0.11496	48.129242698471	48.129242698471\\
61.625	0.11862	51.977555084551	51.977555084551\\
61.625	0.12228	56.1325312714811	56.1325312714811\\
61.625	0.12594	60.5941712592612	60.5941712592612\\
61.625	0.1296	65.3624750478913	65.3624750478913\\
61.625	0.13326	70.4374426373714	70.4374426373714\\
61.625	0.13692	75.8190740277015	75.8190740277015\\
61.625	0.14058	81.5073692188817	81.5073692188817\\
61.625	0.14424	87.5023282109118	87.5023282109118\\
61.625	0.1479	93.8039510037919	93.8039510037919\\
61.625	0.15156	100.412237597522	100.412237597522\\
61.625	0.15522	107.327187992102	107.327187992102\\
61.625	0.15888	114.548802187532	114.548802187532\\
61.625	0.16254	122.077080183813	122.077080183813\\
61.625	0.1662	129.912021980943	129.912021980943\\
61.625	0.16986	138.053627578923	138.053627578923\\
61.625	0.17352	146.501896977753	146.501896977753\\
61.625	0.17718	155.256830177433	155.256830177433\\
61.625	0.18084	164.318427177964	164.318427177964\\
61.625	0.1845	173.686687979344	173.686687979344\\
61.625	0.18816	183.361612581574	183.361612581574\\
61.625	0.19182	193.343200984654	193.343200984654\\
61.625	0.19548	203.631453188585	203.631453188585\\
61.625	0.19914	214.226369193365	214.226369193365\\
61.625	0.2028	225.127948998995	225.127948998995\\
61.625	0.20646	236.336192605476	236.336192605476\\
61.625	0.21012	247.851100012806	247.851100012806\\
61.625	0.21378	259.672671220986	259.672671220986\\
61.625	0.21744	271.800906230016	271.800906230016\\
61.625	0.2211	284.235805039897	284.235805039897\\
61.625	0.22476	296.977367650627	296.977367650627\\
61.625	0.22842	310.025594062207	310.025594062207\\
61.625	0.23208	323.380484274638	323.380484274638\\
61.625	0.23574	337.042038287918	337.042038287918\\
61.625	0.2394	351.010256102048	351.010256102048\\
61.625	0.24306	365.285137717029	365.285137717029\\
61.625	0.24672	379.866683132859	379.866683132859\\
61.625	0.25038	394.75489234954	394.75489234954\\
61.625	0.25404	409.94976536707	409.94976536707\\
61.625	0.2577	425.45130218545	425.45130218545\\
61.625	0.26136	441.259502804681	441.259502804681\\
61.625	0.26502	457.374367224761	457.374367224761\\
61.625	0.26868	473.795895445691	473.795895445691\\
61.625	0.27234	490.524087467472	490.524087467472\\
61.625	0.276	507.558943290102	507.558943290102\\
62	0.093	31.8847686236655	31.8847686236655\\
62	0.09666	33.88775207686	33.88775207686\\
62	0.10032	36.1973993309045	36.1973993309045\\
62	0.10398	38.813710385799	38.813710385799\\
62	0.10764	41.7366852415436	41.7366852415436\\
62	0.1113	44.9663238981381	44.9663238981381\\
62	0.11496	48.5026263555827	48.5026263555827\\
62	0.11862	52.3455926138772	52.3455926138772\\
62	0.12228	56.4952226730219	56.4952226730219\\
62	0.12594	60.9515165330165	60.9515165330165\\
62	0.1296	65.7144741938611	65.7144741938611\\
62	0.13326	70.7840956555556	70.7840956555556\\
62	0.13692	76.1603809181002	76.1603809181002\\
62	0.14058	81.8433299814949	81.8433299814949\\
62	0.14424	87.8329428457396	87.8329428457396\\
62	0.1479	94.1292195108342	94.1292195108342\\
62	0.15156	100.732159976779	100.732159976779\\
62	0.15522	107.641764243574	107.641764243574\\
62	0.15888	114.858032311218	114.858032311218\\
62	0.16254	122.380964179713	122.380964179713\\
62	0.1662	130.210559849057	130.210559849057\\
62	0.16986	138.346819319252	138.346819319252\\
62	0.17352	146.789742590297	146.789742590297\\
62	0.17718	155.539329662192	155.539329662192\\
62	0.18084	164.595580534936	164.595580534936\\
62	0.1845	173.958495208531	173.958495208531\\
62	0.18816	183.628073682976	183.628073682976\\
62	0.19182	193.60431595827	193.60431595827\\
62	0.19548	203.887222034415	203.887222034415\\
62	0.19914	214.47679191141	214.47679191141\\
62	0.2028	225.373025589255	225.373025589255\\
62	0.20646	236.57592306795	236.57592306795\\
62	0.21012	248.085484347494	248.085484347494\\
62	0.21378	259.901709427889	259.901709427889\\
62	0.21744	272.024598309134	272.024598309134\\
62	0.2211	284.454150991229	284.454150991229\\
62	0.22476	297.190367474174	297.190367474174\\
62	0.22842	310.233247757968	310.233247757968\\
62	0.23208	323.582791842613	323.582791842613\\
62	0.23574	337.238999728108	337.238999728108\\
62	0.2394	351.201871414453	351.201871414453\\
62	0.24306	365.471406901648	365.471406901648\\
62	0.24672	380.047606189693	380.047606189693\\
62	0.25038	394.930469278588	394.930469278588\\
62	0.25404	410.119996168332	410.119996168332\\
62	0.2577	425.616186858927	425.616186858927\\
62	0.26136	441.419041350372	441.419041350372\\
62	0.26502	457.528559642667	457.528559642667\\
62	0.26868	473.944741735812	473.944741735812\\
62	0.27234	490.667587629807	490.667587629807\\
62	0.276	507.697097324652	507.697097324652\\
62.375	0.093	32.3011864783002	32.3011864783002\\
62.375	0.09666	34.2988238037092	34.2988238037092\\
62.375	0.10032	36.6031249299683	36.6031249299683\\
62.375	0.10398	39.2140898570773	39.2140898570773\\
62.375	0.10764	42.1317185850363	42.1317185850363\\
62.375	0.1113	45.3560111138454	45.3560111138454\\
62.375	0.11496	48.8869674435044	48.8869674435044\\
62.375	0.11862	52.7245875740135	52.7245875740135\\
62.375	0.12228	56.8688715053726	56.8688715053726\\
62.375	0.12594	61.3198192375817	61.3198192375817\\
62.375	0.1296	66.0774307706407	66.0774307706407\\
62.375	0.13326	71.1417061045498	71.1417061045498\\
62.375	0.13692	76.512645239309	76.512645239309\\
62.375	0.14058	82.1902481749181	82.1902481749181\\
62.375	0.14424	88.1745149113772	88.1745149113772\\
62.375	0.1479	94.4654454486864	94.4654454486864\\
62.375	0.15156	101.063039786845	101.063039786845\\
62.375	0.15522	107.967297925855	107.967297925855\\
62.375	0.15888	115.178219865714	115.178219865714\\
62.375	0.16254	122.695805606423	122.695805606423\\
62.375	0.1662	130.520055147982	130.520055147982\\
62.375	0.16986	138.650968490391	138.650968490391\\
62.375	0.17352	147.088545633651	147.088545633651\\
62.375	0.17718	155.83278657776	155.83278657776\\
62.375	0.18084	164.883691322719	164.883691322719\\
62.375	0.1845	174.241259868528	174.241259868528\\
62.375	0.18816	183.905492215187	183.905492215187\\
62.375	0.19182	193.876388362697	193.876388362697\\
62.375	0.19548	204.153948311056	204.153948311056\\
62.375	0.19914	214.738172060265	214.738172060265\\
62.375	0.2028	225.629059610324	225.629059610324\\
62.375	0.20646	236.826610961234	236.826610961234\\
62.375	0.21012	248.330826112993	248.330826112993\\
62.375	0.21378	260.141705065602	260.141705065602\\
62.375	0.21744	272.259247819061	272.259247819061\\
62.375	0.2211	284.683454373371	284.683454373371\\
62.375	0.22476	297.41432472853	297.41432472853\\
62.375	0.22842	310.451858884539	310.451858884539\\
62.375	0.23208	323.796056841399	323.796056841399\\
62.375	0.23574	337.446918599108	337.446918599108\\
62.375	0.2394	351.404444157668	351.404444157668\\
62.375	0.24306	365.668633517077	365.668633517077\\
62.375	0.24672	380.239486677336	380.239486677336\\
62.375	0.25038	395.117003638446	395.117003638446\\
62.375	0.25404	410.301184400405	410.301184400405\\
62.375	0.2577	425.792028963214	425.792028963214\\
62.375	0.26136	441.589537326874	441.589537326874\\
62.375	0.26502	457.693709491383	457.693709491383\\
62.375	0.26868	474.104545456742	474.104545456742\\
62.375	0.27234	490.822045222952	490.822045222952\\
62.375	0.276	507.846208790011	507.846208790011\\
62.75	0.093	32.7285617637449	32.7285617637449\\
62.75	0.09666	34.7208529613684	34.7208529613684\\
62.75	0.10032	37.0198079598419	37.0198079598419\\
62.75	0.10398	39.6254267591654	39.6254267591654\\
62.75	0.10764	42.537709359339	42.537709359339\\
62.75	0.1113	45.7566557603625	45.7566557603625\\
62.75	0.11496	49.2822659622361	49.2822659622361\\
62.75	0.11862	53.1145399649596	53.1145399649596\\
62.75	0.12228	57.2534777685332	57.2534777685332\\
62.75	0.12594	61.6990793729568	61.6990793729568\\
62.75	0.1296	66.4513447782304	66.4513447782304\\
62.75	0.13326	71.5102739843539	71.5102739843539\\
62.75	0.13692	76.8758669913275	76.8758669913275\\
62.75	0.14058	82.5481237991511	82.5481237991511\\
62.75	0.14424	88.5270444078248	88.5270444078248\\
62.75	0.1479	94.8126288173484	94.8126288173484\\
62.75	0.15156	101.404877027722	101.404877027722\\
62.75	0.15522	108.303789038946	108.303789038946\\
62.75	0.15888	115.509364851019	115.509364851019\\
62.75	0.16254	123.021604463943	123.021604463943\\
62.75	0.1662	130.840507877717	130.840507877717\\
62.75	0.16986	138.96607509234	138.96607509234\\
62.75	0.17352	147.398306107814	147.398306107814\\
62.75	0.17718	156.137200924138	156.137200924138\\
62.75	0.18084	165.182759541311	165.182759541311\\
62.75	0.1845	174.534981959335	174.534981959335\\
62.75	0.18816	184.193868178209	184.193868178209\\
62.75	0.19182	194.159418197933	194.159418197933\\
62.75	0.19548	204.431632018506	204.431632018506\\
62.75	0.19914	215.01050963993	215.01050963993\\
62.75	0.2028	225.896051062204	225.896051062204\\
62.75	0.20646	237.088256285328	237.088256285328\\
62.75	0.21012	248.587125309301	248.587125309301\\
62.75	0.21378	260.392658134125	260.392658134125\\
62.75	0.21744	272.504854759799	272.504854759799\\
62.75	0.2211	284.923715186323	284.923715186323\\
62.75	0.22476	297.649239413697	297.649239413697\\
62.75	0.22842	310.68142744192	310.68142744192\\
62.75	0.23208	324.020279270994	324.020279270994\\
62.75	0.23574	337.665794900918	337.665794900918\\
62.75	0.2394	351.617974331692	351.617974331692\\
62.75	0.24306	365.876817563316	365.876817563316\\
62.75	0.24672	380.44232459579	380.44232459579\\
62.75	0.25038	395.314495429114	395.314495429114\\
62.75	0.25404	410.493330063287	410.493330063287\\
62.75	0.2577	425.978828498311	425.978828498311\\
62.75	0.26136	441.770990734185	441.770990734185\\
62.75	0.26502	457.869816770909	457.869816770909\\
62.75	0.26868	474.275306608483	474.275306608483\\
62.75	0.27234	490.987460246907	490.987460246907\\
62.75	0.276	508.006277686181	508.006277686181\\
63.125	0.093	33.1668944799994	33.1668944799994\\
63.125	0.09666	35.1538395498374	35.1538395498374\\
63.125	0.10032	37.4474484205254	37.4474484205254\\
63.125	0.10398	40.0477210920635	40.0477210920635\\
63.125	0.10764	42.9546575644515	42.9546575644515\\
63.125	0.1113	46.1682578376895	46.1682578376895\\
63.125	0.11496	49.6885219117776	49.6885219117776\\
63.125	0.11862	53.5154497867156	53.5154497867156\\
63.125	0.12228	57.6490414625037	57.6490414625037\\
63.125	0.12594	62.0892969391418	62.0892969391418\\
63.125	0.1296	66.8362162166299	66.8362162166299\\
63.125	0.13326	71.8897992949679	71.8897992949679\\
63.125	0.13692	77.250046174156	77.250046174156\\
63.125	0.14058	82.9169568541942	82.9169568541942\\
63.125	0.14424	88.8905313350823	88.8905313350823\\
63.125	0.1479	95.1707696168205	95.1707696168205\\
63.125	0.15156	101.757671699409	101.757671699409\\
63.125	0.15522	108.651237582847	108.651237582847\\
63.125	0.15888	115.851467267135	115.851467267135\\
63.125	0.16254	123.358360752273	123.358360752273\\
63.125	0.1662	131.171918038261	131.171918038261\\
63.125	0.16986	139.292139125099	139.292139125099\\
63.125	0.17352	147.719024012788	147.719024012788\\
63.125	0.17718	156.452572701326	156.452572701326\\
63.125	0.18084	165.492785190714	165.492785190714\\
63.125	0.1845	174.839661480952	174.839661480952\\
63.125	0.18816	184.49320157204	184.49320157204\\
63.125	0.19182	194.453405463979	194.453405463979\\
63.125	0.19548	204.720273156767	204.720273156767\\
63.125	0.19914	215.293804650405	215.293804650405\\
63.125	0.2028	226.173999944893	226.173999944893\\
63.125	0.20646	237.360859040232	237.360859040232\\
63.125	0.21012	248.85438193642	248.85438193642\\
63.125	0.21378	260.654568633458	260.654568633458\\
63.125	0.21744	272.761419131346	272.761419131346\\
63.125	0.2211	285.174933430085	285.174933430085\\
63.125	0.22476	297.895111529673	297.895111529673\\
63.125	0.22842	310.921953430111	310.921953430111\\
63.125	0.23208	324.2554591314	324.2554591314\\
63.125	0.23574	337.895628633538	337.895628633538\\
63.125	0.2394	351.842461936526	351.842461936526\\
63.125	0.24306	366.095959040365	366.095959040365\\
63.125	0.24672	380.656119945053	380.656119945053\\
63.125	0.25038	395.522944650591	395.522944650591\\
63.125	0.25404	410.69643315698	410.69643315698\\
63.125	0.2577	426.176585464218	426.176585464218\\
63.125	0.26136	441.963401572307	441.963401572307\\
63.125	0.26502	458.056881481245	458.056881481245\\
63.125	0.26868	474.457025191033	474.457025191033\\
63.125	0.27234	491.163832701672	491.163832701672\\
63.125	0.276	508.17730401316	508.17730401316\\
63.5	0.093	33.6161846270639	33.6161846270639\\
63.5	0.09666	35.5977835691164	35.5977835691164\\
63.5	0.10032	37.8860463120189	37.8860463120189\\
63.5	0.10398	40.4809728557715	40.4809728557715\\
63.5	0.10764	43.382563200374	43.382563200374\\
63.5	0.1113	46.5908173458265	46.5908173458265\\
63.5	0.11496	50.1057352921291	50.1057352921291\\
63.5	0.11862	53.9273170392816	53.9273170392816\\
63.5	0.12228	58.0555625872842	58.0555625872842\\
63.5	0.12594	62.4904719361368	62.4904719361368\\
63.5	0.1296	67.2320450858393	67.2320450858393\\
63.5	0.13326	72.2802820363919	72.2802820363919\\
63.5	0.13692	77.6351827877945	77.6351827877945\\
63.5	0.14058	83.2967473400471	83.2967473400471\\
63.5	0.14424	89.2649756931497	89.2649756931497\\
63.5	0.1479	95.5398678471024	95.5398678471024\\
63.5	0.15156	102.121423801905	102.121423801905\\
63.5	0.15522	109.009643557558	109.009643557558\\
63.5	0.15888	116.20452711406	116.20452711406\\
63.5	0.16254	123.706074471413	123.706074471413\\
63.5	0.1662	131.514285629616	131.514285629616\\
63.5	0.16986	139.629160588668	139.629160588668\\
63.5	0.17352	148.050699348571	148.050699348571\\
63.5	0.17718	156.778901909324	156.778901909324\\
63.5	0.18084	165.813768270926	165.813768270926\\
63.5	0.1845	175.155298433379	175.155298433379\\
63.5	0.18816	184.803492396682	184.803492396682\\
63.5	0.19182	194.758350160834	194.758350160834\\
63.5	0.19548	205.019871725837	205.019871725837\\
63.5	0.19914	215.58805709169	215.58805709169\\
63.5	0.2028	226.462906258393	226.462906258393\\
63.5	0.20646	237.644419225946	237.644419225946\\
63.5	0.21012	249.132595994348	249.132595994348\\
63.5	0.21378	260.927436563601	260.927436563601\\
63.5	0.21744	273.028940933704	273.028940933704\\
63.5	0.2211	285.437109104656	285.437109104656\\
63.5	0.22476	298.151941076459	298.151941076459\\
63.5	0.22842	311.173436849112	311.173436849112\\
63.5	0.23208	324.501596422615	324.501596422615\\
63.5	0.23574	338.136419796968	338.136419796968\\
63.5	0.2394	352.077906972171	352.077906972171\\
63.5	0.24306	366.326057948223	366.326057948223\\
63.5	0.24672	380.880872725126	380.880872725126\\
63.5	0.25038	395.742351302879	395.742351302879\\
63.5	0.25404	410.910493681482	410.910493681482\\
63.5	0.2577	426.385299860935	426.385299860935\\
63.5	0.26136	442.166769841238	442.166769841238\\
63.5	0.26502	458.254903622391	458.254903622391\\
63.5	0.26868	474.649701204394	474.649701204394\\
63.5	0.27234	491.351162587246	491.351162587246\\
63.5	0.276	508.359287770949	508.359287770949\\
63.875	0.093	34.0764322049384	34.0764322049384\\
63.875	0.09666	36.0526850192054	36.0526850192054\\
63.875	0.10032	38.3356016343224	38.3356016343224\\
63.875	0.10398	40.9251820502894	40.9251820502894\\
63.875	0.10764	43.8214262671064	43.8214262671064\\
63.875	0.1113	47.0243342847734	47.0243342847734\\
63.875	0.11496	50.5339061032905	50.5339061032905\\
63.875	0.11862	54.3501417226575	54.3501417226575\\
63.875	0.12228	58.4730411428746	58.4730411428746\\
63.875	0.12594	62.9026043639416	62.9026043639416\\
63.875	0.1296	67.6388313858587	67.6388313858587\\
63.875	0.13326	72.6817222086258	72.6817222086258\\
63.875	0.13692	78.0312768322428	78.0312768322428\\
63.875	0.14058	83.68749525671	83.68749525671\\
63.875	0.14424	89.6503774820271	89.6503774820271\\
63.875	0.1479	95.9199235081942	95.9199235081942\\
63.875	0.15156	102.496133335211	102.496133335211\\
63.875	0.15522	109.379006963078	109.379006963078\\
63.875	0.15888	116.568544391796	116.568544391796\\
63.875	0.16254	124.064745621363	124.064745621363\\
63.875	0.1662	131.86761065178	131.86761065178\\
63.875	0.16986	139.977139483047	139.977139483047\\
63.875	0.17352	148.393332115164	148.393332115164\\
63.875	0.17718	157.116188548131	157.116188548131\\
63.875	0.18084	166.145708781949	166.145708781949\\
63.875	0.1845	175.481892816616	175.481892816616\\
63.875	0.18816	185.124740652133	185.124740652133\\
63.875	0.19182	195.0742522885	195.0742522885\\
63.875	0.19548	205.330427725717	205.330427725717\\
63.875	0.19914	215.893266963785	215.893266963785\\
63.875	0.2028	226.762770002702	226.762770002702\\
63.875	0.20646	237.938936842469	237.938936842469\\
63.875	0.21012	249.421767483086	249.421767483086\\
63.875	0.21378	261.211261924554	261.211261924554\\
63.875	0.21744	273.307420166871	273.307420166871\\
63.875	0.2211	285.710242210038	285.710242210038\\
63.875	0.22476	298.419728054056	298.419728054056\\
63.875	0.22842	311.435877698923	311.435877698923\\
63.875	0.23208	324.75869114464	324.75869114464\\
63.875	0.23574	338.388168391208	338.388168391208\\
63.875	0.2394	352.324309438625	352.324309438625\\
63.875	0.24306	366.567114286892	366.567114286892\\
63.875	0.24672	381.11658293601	381.11658293601\\
63.875	0.25038	395.972715385977	395.972715385977\\
63.875	0.25404	411.135511636794	411.135511636794\\
63.875	0.2577	426.604971688462	426.604971688462\\
63.875	0.26136	442.381095540979	442.381095540979\\
63.875	0.26502	458.463883194346	458.463883194346\\
63.875	0.26868	474.853334648564	474.853334648564\\
63.875	0.27234	491.549449903631	491.549449903631\\
63.875	0.276	508.552228959548	508.552228959548\\
64.25	0.093	34.5476372136227	34.5476372136227\\
64.25	0.09666	36.5185439001042	36.5185439001042\\
64.25	0.10032	38.7961143874356	38.7961143874356\\
64.25	0.10398	41.3803486756172	41.3803486756172\\
64.25	0.10764	44.2712467646487	44.2712467646487\\
64.25	0.1113	47.4688086545302	47.4688086545302\\
64.25	0.11496	50.9730343452617	50.9730343452617\\
64.25	0.11862	54.7839238368433	54.7839238368433\\
64.25	0.12228	58.9014771292748	58.9014771292748\\
64.25	0.12594	63.3256942225564	63.3256942225564\\
64.25	0.1296	68.056575116688	68.056575116688\\
64.25	0.13326	73.0941198116695	73.0941198116695\\
64.25	0.13692	78.4383283075011	78.4383283075011\\
64.25	0.14058	84.0892006041828	84.0892006041828\\
64.25	0.14424	90.0467367017144	90.0467367017144\\
64.25	0.1479	96.310936600096	96.310936600096\\
64.25	0.15156	102.881800299327	102.881800299327\\
64.25	0.15522	109.759327799409	109.759327799409\\
64.25	0.15888	116.943519100341	116.943519100341\\
64.25	0.16254	124.434374202122	124.434374202122\\
64.25	0.1662	132.231893104754	132.231893104754\\
64.25	0.16986	140.336075808236	140.336075808236\\
64.25	0.17352	148.746922312567	148.746922312567\\
64.25	0.17718	157.464432617749	157.464432617749\\
64.25	0.18084	166.488606723781	166.488606723781\\
64.25	0.1845	175.819444630663	175.819444630663\\
64.25	0.18816	185.456946338394	185.456946338394\\
64.25	0.19182	195.401111846976	195.401111846976\\
64.25	0.19548	205.651941156408	205.651941156408\\
64.25	0.19914	216.209434266689	216.209434266689\\
64.25	0.2028	227.073591177821	227.073591177821\\
64.25	0.20646	238.244411889803	238.244411889803\\
64.25	0.21012	249.721896402635	249.721896402635\\
64.25	0.21378	261.506044716316	261.506044716316\\
64.25	0.21744	273.596856830848	273.596856830848\\
64.25	0.2211	285.99433274623	285.99433274623\\
64.25	0.22476	298.698472462462	298.698472462462\\
64.25	0.22842	311.709275979544	311.709275979544\\
64.25	0.23208	325.026743297475	325.026743297475\\
64.25	0.23574	338.650874416257	338.650874416257\\
64.25	0.2394	352.581669335889	352.581669335889\\
64.25	0.24306	366.819128056371	366.819128056371\\
64.25	0.24672	381.363250577703	381.363250577703\\
64.25	0.25038	396.214036899885	396.214036899885\\
64.25	0.25404	411.371487022916	411.371487022916\\
64.25	0.2577	426.835600946798	426.835600946798\\
64.25	0.26136	442.60637867153	442.60637867153\\
64.25	0.26502	458.683820197112	458.683820197112\\
64.25	0.26868	475.067925523544	475.067925523544\\
64.25	0.27234	491.758694650826	491.758694650826\\
64.25	0.276	508.756127578957	508.756127578957\\
64.625	0.093	35.0297996531169	35.0297996531169\\
64.625	0.09666	36.9953602118129	36.9953602118129\\
64.625	0.10032	39.2675845713589	39.2675845713589\\
64.625	0.10398	41.8464727317549	41.8464727317549\\
64.625	0.10764	44.7320246930009	44.7320246930009\\
64.625	0.1113	47.924240455097	47.924240455097\\
64.625	0.11496	51.423120018043	51.423120018043\\
64.625	0.11862	55.228663381839	55.228663381839\\
64.625	0.12228	59.3408705464851	59.3408705464851\\
64.625	0.12594	63.7597415119812	63.7597415119812\\
64.625	0.1296	68.4852762783272	68.4852762783272\\
64.625	0.13326	73.5174748455232	73.5174748455232\\
64.625	0.13692	78.8563372135693	78.8563372135693\\
64.625	0.14058	84.5018633824654	84.5018633824654\\
64.625	0.14424	90.4540533522116	90.4540533522116\\
64.625	0.1479	96.7129071228076	96.7129071228076\\
64.625	0.15156	103.278424694254	103.278424694254\\
64.625	0.15522	110.15060606655	110.15060606655\\
64.625	0.15888	117.329451239696	117.329451239696\\
64.625	0.16254	124.814960213692	124.814960213692\\
64.625	0.1662	132.607132988538	132.607132988538\\
64.625	0.16986	140.705969564234	140.705969564234\\
64.625	0.17352	149.111469940781	149.111469940781\\
64.625	0.17718	157.823634118177	157.823634118177\\
64.625	0.18084	166.842462096423	166.842462096423\\
64.625	0.1845	176.167953875519	176.167953875519\\
64.625	0.18816	185.800109455465	185.800109455465\\
64.625	0.19182	195.738928836262	195.738928836262\\
64.625	0.19548	205.984412017908	205.984412017908\\
64.625	0.19914	216.536559000404	216.536559000404\\
64.625	0.2028	227.39536978375	227.39536978375\\
64.625	0.20646	238.560844367947	238.560844367947\\
64.625	0.21012	250.032982752993	250.032982752993\\
64.625	0.21378	261.811784938889	261.811784938889\\
64.625	0.21744	273.897250925635	273.897250925635\\
64.625	0.2211	286.289380713231	286.289380713231\\
64.625	0.22476	298.988174301678	298.988174301678\\
64.625	0.22842	311.993631690974	311.993631690974\\
64.625	0.23208	325.30575288112	325.30575288112\\
64.625	0.23574	338.924537872117	338.924537872117\\
64.625	0.2394	352.849986663963	352.849986663963\\
64.625	0.24306	367.082099256659	367.082099256659\\
64.625	0.24672	381.620875650206	381.620875650206\\
64.625	0.25038	396.466315844602	396.466315844602\\
64.625	0.25404	411.618419839848	411.618419839848\\
64.625	0.2577	427.077187635945	427.077187635945\\
64.625	0.26136	442.842619232891	442.842619232891\\
64.625	0.26502	458.914714630688	458.914714630688\\
64.625	0.26868	475.293473829334	475.293473829334\\
64.625	0.27234	491.97889682883	491.97889682883\\
64.625	0.276	508.970983629177	508.970983629177\\
65	0.093	35.5229195234211	35.5229195234211\\
65	0.09666	37.4831339543316	37.4831339543316\\
65	0.10032	39.7500121860921	39.7500121860921\\
65	0.10398	42.3235542187026	42.3235542187026\\
65	0.10764	45.2037600521631	45.2037600521631\\
65	0.1113	48.3906296864736	48.3906296864736\\
65	0.11496	51.8841631216341	51.8841631216341\\
65	0.11862	55.6843603576447	55.6843603576447\\
65	0.12228	59.7912213945052	59.7912213945052\\
65	0.12594	64.2047462322158	64.2047462322158\\
65	0.1296	68.9249348707763	68.9249348707763\\
65	0.13326	73.9517873101869	73.9517873101869\\
65	0.13692	79.2853035504474	79.2853035504474\\
65	0.14058	84.925483591558	84.925483591558\\
65	0.14424	90.8723274335186	90.8723274335186\\
65	0.1479	97.1258350763292	97.1258350763292\\
65	0.15156	103.68600651999	103.68600651999\\
65	0.15522	110.5528417645	110.5528417645\\
65	0.15888	117.726340809861	117.726340809861\\
65	0.16254	125.206503656072	125.206503656072\\
65	0.1662	132.993330303132	132.993330303132\\
65	0.16986	141.086820751043	141.086820751043\\
65	0.17352	149.486974999804	149.486974999804\\
65	0.17718	158.193793049414	158.193793049414\\
65	0.18084	167.207274899875	167.207274899875\\
65	0.1845	176.527420551186	176.527420551186\\
65	0.18816	186.154230003346	186.154230003346\\
65	0.19182	196.087703256357	196.087703256357\\
65	0.19548	206.327840310218	206.327840310218\\
65	0.19914	216.874641164928	216.874641164928\\
65	0.2028	227.728105820489	227.728105820489\\
65	0.20646	238.8882342769	238.8882342769\\
65	0.21012	250.355026534161	250.355026534161\\
65	0.21378	262.128482592271	262.128482592271\\
65	0.21744	274.208602451232	274.208602451232\\
65	0.2211	286.595386111043	286.595386111043\\
65	0.22476	299.288833571704	299.288833571704\\
65	0.22842	312.288944833215	312.288944833215\\
65	0.23208	325.595719895575	325.595719895575\\
65	0.23574	339.209158758786	339.209158758786\\
65	0.2394	353.129261422847	353.129261422847\\
65	0.24306	367.356027887758	367.356027887758\\
65	0.24672	381.889458153519	381.889458153519\\
65	0.25038	396.72955222013	396.72955222013\\
65	0.25404	411.87631008759	411.87631008759\\
65	0.2577	427.329731755901	427.329731755901\\
65	0.26136	443.089817225062	443.089817225062\\
65	0.26502	459.156566495073	459.156566495073\\
65	0.26868	475.529979565934	475.529979565934\\
65	0.27234	492.210056437645	492.210056437645\\
65	0.276	509.196797110205	509.196797110205\\
65.375	0.093	36.0269968245352	36.0269968245352\\
65.375	0.09666	37.9818651276602	37.9818651276602\\
65.375	0.10032	40.2433972316352	40.2433972316352\\
65.375	0.10398	42.8115931364602	42.8115931364602\\
65.375	0.10764	45.6864528421352	45.6864528421352\\
65.375	0.1113	48.8679763486602	48.8679763486602\\
65.375	0.11496	52.3561636560352	52.3561636560352\\
65.375	0.11862	56.1510147642602	56.1510147642602\\
65.375	0.12228	60.2525296733353	60.2525296733353\\
65.375	0.12594	64.6607083832603	64.6607083832603\\
65.375	0.1296	69.3755508940353	69.3755508940353\\
65.375	0.13326	74.3970572056604	74.3970572056604\\
65.375	0.13692	79.7252273181354	79.7252273181354\\
65.375	0.14058	85.3600612314605	85.3600612314605\\
65.375	0.14424	91.3015589456357	91.3015589456357\\
65.375	0.1479	97.5497204606608	97.5497204606608\\
65.375	0.15156	104.104545776536	104.104545776536\\
65.375	0.15522	110.966034893261	110.966034893261\\
65.375	0.15888	118.134187810836	118.134187810836\\
65.375	0.16254	125.609004529261	125.609004529261\\
65.375	0.1662	133.390485048536	133.390485048536\\
65.375	0.16986	141.478629368662	141.478629368662\\
65.375	0.17352	149.873437489637	149.873437489637\\
65.375	0.17718	158.574909411462	158.574909411462\\
65.375	0.18084	167.583045134137	167.583045134137\\
65.375	0.1845	176.897844657662	176.897844657662\\
65.375	0.18816	186.519307982037	186.519307982037\\
65.375	0.19182	196.447435107263	196.447435107263\\
65.375	0.19548	206.682226033338	206.682226033338\\
65.375	0.19914	217.223680760263	217.223680760263\\
65.375	0.2028	228.071799288038	228.071799288038\\
65.375	0.20646	239.226581616664	239.226581616664\\
65.375	0.21012	250.688027746139	250.688027746139\\
65.375	0.21378	262.456137676464	262.456137676464\\
65.375	0.21744	274.530911407639	274.530911407639\\
65.375	0.2211	286.912348939664	286.912348939664\\
65.375	0.22476	299.60045027254	299.60045027254\\
65.375	0.22842	312.595215406265	312.595215406265\\
65.375	0.23208	325.89664434084	325.89664434084\\
65.375	0.23574	339.504737076266	339.504737076266\\
65.375	0.2394	353.419493612541	353.419493612541\\
65.375	0.24306	367.640913949666	367.640913949666\\
65.375	0.24672	382.168998087642	382.168998087642\\
65.375	0.25038	397.003746026467	397.003746026467\\
65.375	0.25404	412.145157766142	412.145157766142\\
65.375	0.2577	427.593233306668	427.593233306668\\
65.375	0.26136	443.347972648043	443.347972648043\\
65.375	0.26502	459.409375790268	459.409375790268\\
65.375	0.26868	475.777442733344	475.777442733344\\
65.375	0.27234	492.452173477269	492.452173477269\\
65.375	0.276	509.433568022044	509.433568022044\\
65.75	0.093	36.5420315564592	36.5420315564592\\
65.75	0.09666	38.4915537317987	38.4915537317987\\
65.75	0.10032	40.7477397079882	40.7477397079882\\
65.75	0.10398	43.3105894850276	43.3105894850276\\
65.75	0.10764	46.1801030629171	46.1801030629171\\
65.75	0.1113	49.3562804416566	49.3562804416566\\
65.75	0.11496	52.8391216212462	52.8391216212462\\
65.75	0.11862	56.6286266016857	56.6286266016857\\
65.75	0.12228	60.7247953829753	60.7247953829753\\
65.75	0.12594	65.1276279651148	65.1276279651148\\
65.75	0.1296	69.8371243481043	69.8371243481043\\
65.75	0.13326	74.8532845319438	74.8532845319438\\
65.75	0.13692	80.1761085166334	80.1761085166334\\
65.75	0.14058	85.805596302173	85.805596302173\\
65.75	0.14424	91.7417478885626	91.7417478885626\\
65.75	0.1479	97.9845632758021	97.9845632758021\\
65.75	0.15156	104.534042463892	104.534042463892\\
65.75	0.15522	111.390185452831	111.390185452831\\
65.75	0.15888	118.552992242621	118.552992242621\\
65.75	0.16254	126.022462833261	126.022462833261\\
65.75	0.1662	133.79859722475	133.79859722475\\
65.75	0.16986	141.88139541709	141.88139541709\\
65.75	0.17352	150.27085741028	150.27085741028\\
65.75	0.17718	158.966983204319	158.966983204319\\
65.75	0.18084	167.969772799209	167.969772799209\\
65.75	0.1845	177.279226194949	177.279226194949\\
65.75	0.18816	186.895343391538	186.895343391538\\
65.75	0.19182	196.818124388978	196.818124388978\\
65.75	0.19548	207.047569187268	207.047569187268\\
65.75	0.19914	217.583677786407	217.583677786407\\
65.75	0.2028	228.426450186397	228.426450186397\\
65.75	0.20646	239.575886387237	239.575886387237\\
65.75	0.21012	251.031986388927	251.031986388927\\
65.75	0.21378	262.794750191466	262.794750191466\\
65.75	0.21744	274.864177794856	274.864177794856\\
65.75	0.2211	287.240269199096	287.240269199096\\
65.75	0.22476	299.923024404186	299.923024404186\\
65.75	0.22842	312.912443410125	312.912443410125\\
65.75	0.23208	326.208526216915	326.208526216915\\
65.75	0.23574	339.811272824555	339.811272824555\\
65.75	0.2394	353.720683233045	353.720683233045\\
65.75	0.24306	367.936757442385	367.936757442385\\
65.75	0.24672	382.459495452574	382.459495452574\\
65.75	0.25038	397.288897263614	397.288897263614\\
65.75	0.25404	412.424962875504	412.424962875504\\
65.75	0.2577	427.867692288244	427.867692288244\\
65.75	0.26136	443.617085501834	443.617085501834\\
65.75	0.26502	459.673142516274	459.673142516274\\
65.75	0.26868	476.035863331563	476.035863331563\\
65.75	0.27234	492.705247947703	492.705247947703\\
65.75	0.276	509.681296364693	509.681296364693\\
66.125	0.093	37.0680237191932	37.0680237191932\\
66.125	0.09666	39.0121997667471	39.0121997667471\\
66.125	0.10032	41.2630396151511	41.2630396151511\\
66.125	0.10398	43.8205432644051	43.8205432644051\\
66.125	0.10764	46.6847107145091	46.6847107145091\\
66.125	0.1113	49.8555419654631	49.8555419654631\\
66.125	0.11496	53.3330370172671	53.3330370172671\\
66.125	0.11862	57.1171958699211	57.1171958699211\\
66.125	0.12228	61.2080185234252	61.2080185234252\\
66.125	0.12594	65.6055049777792	65.6055049777792\\
66.125	0.1296	70.3096552329832	70.3096552329832\\
66.125	0.13326	75.3204692890372	75.3204692890372\\
66.125	0.13692	80.6379471459413	80.6379471459413\\
66.125	0.14058	86.2620888036954	86.2620888036954\\
66.125	0.14424	92.1928942622995	92.1928942622995\\
66.125	0.1479	98.4303635217535	98.4303635217535\\
66.125	0.15156	104.974496582058	104.974496582058\\
66.125	0.15522	111.825293443212	111.825293443212\\
66.125	0.15888	118.982754105216	118.982754105216\\
66.125	0.16254	126.44687856807	126.44687856807\\
66.125	0.1662	134.217666831774	134.217666831774\\
66.125	0.16986	142.295118896328	142.295118896328\\
66.125	0.17352	150.679234761732	150.679234761732\\
66.125	0.17718	159.370014427987	159.370014427987\\
66.125	0.18084	168.367457895091	168.367457895091\\
66.125	0.1845	177.671565163045	177.671565163045\\
66.125	0.18816	187.282336231849	187.282336231849\\
66.125	0.19182	197.199771101503	197.199771101503\\
66.125	0.19548	207.423869772007	207.423869772007\\
66.125	0.19914	217.954632243362	217.954632243362\\
66.125	0.2028	228.792058515566	228.792058515566\\
66.125	0.20646	239.93614858862	239.93614858862\\
66.125	0.21012	251.386902462524	251.386902462524\\
66.125	0.21378	263.144320137279	263.144320137279\\
66.125	0.21744	275.208401612883	275.208401612883\\
66.125	0.2211	287.579146889337	287.579146889337\\
66.125	0.22476	300.256555966641	300.256555966641\\
66.125	0.22842	313.240628844796	313.240628844796\\
66.125	0.23208	326.5313655238	326.5313655238\\
66.125	0.23574	340.128766003654	340.128766003654\\
66.125	0.2394	354.032830284359	354.032830284359\\
66.125	0.24306	368.243558365913	368.243558365913\\
66.125	0.24672	382.760950248317	382.760950248317\\
66.125	0.25038	397.585005931572	397.585005931572\\
66.125	0.25404	412.715725415676	412.715725415676\\
66.125	0.2577	428.15310870063	428.15310870063\\
66.125	0.26136	443.897155786434	443.897155786434\\
66.125	0.26502	459.947866673089	459.947866673089\\
66.125	0.26868	476.305241360593	476.305241360593\\
66.125	0.27234	492.969279848948	492.969279848948\\
66.125	0.276	509.939982138152	509.939982138152\\
66.5	0.093	37.604973312737	37.604973312737\\
66.5	0.09666	39.5438032325054	39.5438032325054\\
66.5	0.10032	41.7892969531239	41.7892969531239\\
66.5	0.10398	44.3414544745924	44.3414544745924\\
66.5	0.10764	47.2002757969109	47.2002757969109\\
66.5	0.1113	50.3657609200794	50.3657609200794\\
66.5	0.11496	53.8379098440979	53.8379098440979\\
66.5	0.11862	57.6167225689664	57.6167225689664\\
66.5	0.12228	61.7021990946849	61.7021990946849\\
66.5	0.12594	66.0943394212535	66.0943394212535\\
66.5	0.1296	70.793143548672	70.793143548672\\
66.5	0.13326	75.7986114769405	75.7986114769405\\
66.5	0.13692	81.1107432060591	81.1107432060591\\
66.5	0.14058	86.7295387360276	86.7295387360276\\
66.5	0.14424	92.6549980668462	92.6549980668462\\
66.5	0.1479	98.8871211985148	98.8871211985148\\
66.5	0.15156	105.425908131033	105.425908131033\\
66.5	0.15522	112.271358864402	112.271358864402\\
66.5	0.15888	119.423473398621	119.423473398621\\
66.5	0.16254	126.882251733689	126.882251733689\\
66.5	0.1662	134.647693869608	134.647693869608\\
66.5	0.16986	142.719799806376	142.719799806376\\
66.5	0.17352	151.098569543995	151.098569543995\\
66.5	0.17718	159.784003082464	159.784003082464\\
66.5	0.18084	168.776100421782	168.776100421782\\
66.5	0.1845	178.074861561951	178.074861561951\\
66.5	0.18816	187.68028650297	187.68028650297\\
66.5	0.19182	197.592375244838	197.592375244838\\
66.5	0.19548	207.811127787557	207.811127787557\\
66.5	0.19914	218.336544131126	218.336544131126\\
66.5	0.2028	229.168624275545	229.168624275545\\
66.5	0.20646	240.307368220813	240.307368220813\\
66.5	0.21012	251.752775966932	251.752775966932\\
66.5	0.21378	263.504847513901	263.504847513901\\
66.5	0.21744	275.563582861719	275.563582861719\\
66.5	0.2211	287.928982010388	287.928982010388\\
66.5	0.22476	300.601044959907	300.601044959907\\
66.5	0.22842	313.579771710276	313.579771710276\\
66.5	0.23208	326.865162261495	326.865162261495\\
66.5	0.23574	340.457216613563	340.457216613563\\
66.5	0.2394	354.355934766482	354.355934766482\\
66.5	0.24306	368.561316720251	368.561316720251\\
66.5	0.24672	383.07336247487	383.07336247487\\
66.5	0.25038	397.892072030339	397.892072030339\\
66.5	0.25404	413.017445386657	413.017445386657\\
66.5	0.2577	428.449482543826	428.449482543826\\
66.5	0.26136	444.188183501845	444.188183501845\\
66.5	0.26502	460.233548260714	460.233548260714\\
66.5	0.26868	476.585576820433	476.585576820433\\
66.5	0.27234	493.244269181002	493.244269181002\\
66.5	0.276	510.20962534242	510.20962534242\\
66.875	0.093	38.1528803370908	38.1528803370908\\
66.875	0.09666	40.0863641290737	40.0863641290737\\
66.875	0.10032	42.3265117219067	42.3265117219067\\
66.875	0.10398	44.8733231155897	44.8733231155897\\
66.875	0.10764	47.7267983101226	47.7267983101226\\
66.875	0.1113	50.8869373055056	50.8869373055056\\
66.875	0.11496	54.3537401017386	54.3537401017386\\
66.875	0.11862	58.1272066988216	58.1272066988216\\
66.875	0.12228	62.2073370967547	62.2073370967547\\
66.875	0.12594	66.5941312955377	66.5941312955377\\
66.875	0.1296	71.2875892951708	71.2875892951708\\
66.875	0.13326	76.2877110956537	76.2877110956537\\
66.875	0.13692	81.5944966969868	81.5944966969868\\
66.875	0.14058	87.2079460991699	87.2079460991699\\
66.875	0.14424	93.1280593022029	93.1280593022029\\
66.875	0.1479	99.354836306086	99.354836306086\\
66.875	0.15156	105.888277110819	105.888277110819\\
66.875	0.15522	112.728381716402	112.728381716402\\
66.875	0.15888	119.875150122835	119.875150122835\\
66.875	0.16254	127.328582330118	127.328582330118\\
66.875	0.1662	135.088678338251	135.088678338251\\
66.875	0.16986	143.155438147235	143.155438147235\\
66.875	0.17352	151.528861757068	151.528861757068\\
66.875	0.17718	160.208949167751	160.208949167751\\
66.875	0.18084	169.195700379284	169.195700379284\\
66.875	0.1845	178.489115391667	178.489115391667\\
66.875	0.18816	188.0891942049	188.0891942049\\
66.875	0.19182	197.995936818984	197.995936818984\\
66.875	0.19548	208.209343233917	208.209343233917\\
66.875	0.19914	218.7294134497	218.7294134497\\
66.875	0.2028	229.556147466333	229.556147466333\\
66.875	0.20646	240.689545283816	240.689545283816\\
66.875	0.21012	252.12960690215	252.12960690215\\
66.875	0.21378	263.876332321333	263.876332321333\\
66.875	0.21744	275.929721541366	275.929721541366\\
66.875	0.2211	288.289774562249	288.289774562249\\
66.875	0.22476	300.956491383983	300.956491383983\\
66.875	0.22842	313.929872006566	313.929872006566\\
66.875	0.23208	327.209916429999	327.209916429999\\
66.875	0.23574	340.796624654282	340.796624654282\\
66.875	0.2394	354.689996679416	354.689996679416\\
66.875	0.24306	368.890032505399	368.890032505399\\
66.875	0.24672	383.396732132232	383.396732132232\\
66.875	0.25038	398.210095559916	398.210095559916\\
66.875	0.25404	413.330122788449	413.330122788449\\
66.875	0.2577	428.756813817832	428.756813817832\\
66.875	0.26136	444.490168648066	444.490168648066\\
66.875	0.26502	460.530187279149	460.530187279149\\
66.875	0.26868	476.876869711082	476.876869711082\\
66.875	0.27234	493.530215943866	493.530215943866\\
66.875	0.276	510.490225977499	510.490225977499\\
67.25	0.093	38.7117447922545	38.7117447922545\\
67.25	0.09666	40.6398824564519	40.6398824564519\\
67.25	0.10032	42.8746839214994	42.8746839214994\\
67.25	0.10398	45.4161491873968	45.4161491873968\\
67.25	0.10764	48.2642782541443	48.2642782541443\\
67.25	0.1113	51.4190711217418	51.4190711217418\\
67.25	0.11496	54.8805277901893	54.8805277901893\\
67.25	0.11862	58.6486482594868	58.6486482594868\\
67.25	0.12228	62.7234325296343	62.7234325296343\\
67.25	0.12594	67.1048806006318	67.1048806006318\\
67.25	0.1296	71.7929924724794	71.7929924724794\\
67.25	0.13326	76.7877681451768	76.7877681451768\\
67.25	0.13692	82.0892076187244	82.0892076187244\\
67.25	0.14058	87.697310893122	87.697310893122\\
67.25	0.14424	93.6120779683695	93.6120779683695\\
67.25	0.1479	99.8335088444671	99.8335088444671\\
67.25	0.15156	106.361603521415	106.361603521415\\
67.25	0.15522	113.196361999212	113.196361999212\\
67.25	0.15888	120.33778427786	120.33778427786\\
67.25	0.16254	127.785870357358	127.785870357358\\
67.25	0.1662	135.540620237705	135.540620237705\\
67.25	0.16986	143.602033918903	143.602033918903\\
67.25	0.17352	151.97011140095	151.97011140095\\
67.25	0.17718	160.644852683848	160.644852683848\\
67.25	0.18084	169.626257767596	169.626257767596\\
67.25	0.1845	178.914326652193	178.914326652193\\
67.25	0.18816	188.509059337641	188.509059337641\\
67.25	0.19182	198.410455823939	198.410455823939\\
67.25	0.19548	208.618516111086	208.618516111086\\
67.25	0.19914	219.133240199084	219.133240199084\\
67.25	0.2028	229.954628087932	229.954628087932\\
67.25	0.20646	241.08267977763	241.08267977763\\
67.25	0.21012	252.517395268177	252.517395268177\\
67.25	0.21378	264.258774559575	264.258774559575\\
67.25	0.21744	276.306817651823	276.306817651823\\
67.25	0.2211	288.66152454492	288.66152454492\\
67.25	0.22476	301.322895238868	301.322895238868\\
67.25	0.22842	314.290929733666	314.290929733666\\
67.25	0.23208	327.565628029314	327.565628029314\\
67.25	0.23574	341.146990125811	341.146990125811\\
67.25	0.2394	355.035016023159	355.035016023159\\
67.25	0.24306	369.229705721357	369.229705721357\\
67.25	0.24672	383.731059220405	383.731059220405\\
67.25	0.25038	398.539076520303	398.539076520303\\
67.25	0.25404	413.65375762105	413.65375762105\\
67.25	0.2577	429.075102522648	429.075102522648\\
67.25	0.26136	444.803111225096	444.803111225096\\
67.25	0.26502	460.837783728394	460.837783728394\\
67.25	0.26868	477.179120032542	477.179120032542\\
67.25	0.27234	493.82712013754	493.82712013754\\
67.25	0.276	510.781784043387	510.781784043387\\
67.625	0.093	39.2815666782281	39.2815666782281\\
67.625	0.09666	41.20435821464	41.20435821464\\
67.625	0.10032	43.4338135519019	43.4338135519019\\
67.625	0.10398	45.9699326900139	45.9699326900139\\
67.625	0.10764	48.8127156289759	48.8127156289759\\
67.625	0.1113	51.9621623687879	51.9621623687879\\
67.625	0.11496	55.4182729094499	55.4182729094499\\
67.625	0.11862	59.1810472509619	59.1810472509619\\
67.625	0.12228	63.2504853933239	63.2504853933239\\
67.625	0.12594	67.6265873365359	67.6265873365359\\
67.625	0.1296	72.3093530805979	72.3093530805979\\
67.625	0.13326	77.2987826255099	77.2987826255099\\
67.625	0.13692	82.5948759712719	82.5948759712719\\
67.625	0.14058	88.197633117884	88.197633117884\\
67.625	0.14424	94.1070540653461	94.1070540653461\\
67.625	0.1479	100.323138813658	100.323138813658\\
67.625	0.15156	106.84588736282	106.84588736282\\
67.625	0.15522	113.675299712832	113.675299712832\\
67.625	0.15888	120.811375863695	120.811375863695\\
67.625	0.16254	128.254115815407	128.254115815407\\
67.625	0.1662	136.003519567969	136.003519567969\\
67.625	0.16986	144.059587121381	144.059587121381\\
67.625	0.17352	152.422318475643	152.422318475643\\
67.625	0.17718	161.091713630755	161.091713630755\\
67.625	0.18084	170.067772586717	170.067772586717\\
67.625	0.1845	179.350495343529	179.350495343529\\
67.625	0.18816	188.939881901192	188.939881901192\\
67.625	0.19182	198.835932259704	198.835932259704\\
67.625	0.19548	209.038646419066	209.038646419066\\
67.625	0.19914	219.548024379278	219.548024379278\\
67.625	0.2028	230.36406614034	230.36406614034\\
67.625	0.20646	241.486771702253	241.486771702253\\
67.625	0.21012	252.916141065015	252.916141065015\\
67.625	0.21378	264.652174228627	264.652174228627\\
67.625	0.21744	276.694871193089	276.694871193089\\
67.625	0.2211	289.044231958401	289.044231958401\\
67.625	0.22476	301.700256524564	301.700256524564\\
67.625	0.22842	314.662944891576	314.662944891576\\
67.625	0.23208	327.932297059438	327.932297059438\\
67.625	0.23574	341.50831302815	341.50831302815\\
67.625	0.2394	355.390992797713	355.390992797713\\
67.625	0.24306	369.580336368125	369.580336368125\\
67.625	0.24672	384.076343739387	384.076343739387\\
67.625	0.25038	398.8790149115	398.8790149115\\
67.625	0.25404	413.988349884462	413.988349884462\\
67.625	0.2577	429.404348658274	429.404348658274\\
67.625	0.26136	445.127011232936	445.127011232936\\
67.625	0.26502	461.156337608449	461.156337608449\\
67.625	0.26868	477.492327784811	477.492327784811\\
67.625	0.27234	494.134981762023	494.134981762023\\
67.625	0.276	511.084299540086	511.084299540086\\
68	0.093	39.8623459950116	39.8623459950116\\
68	0.09666	41.779791403638	41.779791403638\\
68	0.10032	44.0039006131144	44.0039006131144\\
68	0.10398	46.5346736234409	46.5346736234409\\
68	0.10764	49.3721104346174	49.3721104346174\\
68	0.1113	52.5162110466438	52.5162110466438\\
68	0.11496	55.9669754595203	55.9669754595203\\
68	0.11862	59.7244036732469	59.7244036732469\\
68	0.12228	63.7884956878234	63.7884956878234\\
68	0.12594	68.1592515032499	68.1592515032499\\
68	0.1296	72.8366711195264	72.8366711195264\\
68	0.13326	77.8207545366529	77.8207545366529\\
68	0.13692	83.1115017546294	83.1115017546294\\
68	0.14058	88.708912773456	88.708912773456\\
68	0.14424	94.6129875931326	94.6129875931326\\
68	0.1479	100.823726213659	100.823726213659\\
68	0.15156	107.341128635036	107.341128635036\\
68	0.15522	114.165194857262	114.165194857262\\
68	0.15888	121.295924880339	121.295924880339\\
68	0.16254	128.733318704265	128.733318704265\\
68	0.1662	136.477376329042	136.477376329042\\
68	0.16986	144.528097754669	144.528097754669\\
68	0.17352	152.885482981145	152.885482981145\\
68	0.17718	161.549532008472	161.549532008472\\
68	0.18084	170.520244836648	170.520244836648\\
68	0.1845	179.797621465675	179.797621465675\\
68	0.18816	189.381661895552	189.381661895552\\
68	0.19182	199.272366126278	199.272366126278\\
68	0.19548	209.469734157855	209.469734157855\\
68	0.19914	219.973765990282	219.973765990282\\
68	0.2028	230.784461623559	230.784461623559\\
68	0.20646	241.901821057685	241.901821057685\\
68	0.21012	253.325844292662	253.325844292662\\
68	0.21378	265.056531328489	265.056531328489\\
68	0.21744	277.093882165165	277.093882165165\\
68	0.2211	289.437896802692	289.437896802692\\
68	0.22476	302.088575241069	302.088575241069\\
68	0.22842	315.045917480296	315.045917480296\\
68	0.23208	328.309923520372	328.309923520372\\
68	0.23574	341.880593361299	341.880593361299\\
68	0.2394	355.757927003076	355.757927003076\\
68	0.24306	369.941924445703	369.941924445703\\
68	0.24672	384.432585689179	384.432585689179\\
68	0.25038	399.229910733506	399.229910733506\\
68	0.25404	414.333899578683	414.333899578683\\
68	0.2577	429.74455222471	429.74455222471\\
68	0.26136	445.461868671587	445.461868671587\\
68	0.26502	461.485848919314	461.485848919314\\
68	0.26868	477.81649296789	477.81649296789\\
68	0.27234	494.453800817317	494.453800817317\\
68	0.276	511.397772467594	511.397772467594\\
68.375	0.093	40.454082742605	40.454082742605\\
68.375	0.09666	42.366182023446	42.366182023446\\
68.375	0.10032	44.5849451051369	44.5849451051369\\
68.375	0.10398	47.1103719876779	47.1103719876779\\
68.375	0.10764	49.9424626710689	49.9424626710689\\
68.375	0.1113	53.0812171553098	53.0812171553098\\
68.375	0.11496	56.5266354404008	56.5266354404008\\
68.375	0.11862	60.2787175263418	60.2787175263418\\
68.375	0.12228	64.3374634131328	64.3374634131328\\
68.375	0.12594	68.7028731007738	68.7028731007738\\
68.375	0.1296	73.3749465892648	73.3749465892648\\
68.375	0.13326	78.3536838786058	78.3536838786058\\
68.375	0.13692	83.6390849687968	83.6390849687968\\
68.375	0.14058	89.2311498598378	89.2311498598378\\
68.375	0.14424	95.1298785517289	95.1298785517289\\
68.375	0.1479	101.33527104447	101.33527104447\\
68.375	0.15156	107.847327338061	107.847327338061\\
68.375	0.15522	114.666047432502	114.666047432502\\
68.375	0.15888	121.791431327793	121.791431327793\\
68.375	0.16254	129.223479023934	129.223479023934\\
68.375	0.1662	136.962190520925	136.962190520925\\
68.375	0.16986	145.007565818766	145.007565818766\\
68.375	0.17352	153.359604917458	153.359604917458\\
68.375	0.17718	162.018307816999	162.018307816999\\
68.375	0.18084	170.98367451739	170.98367451739\\
68.375	0.1845	180.255705018631	180.255705018631\\
68.375	0.18816	189.834399320722	189.834399320722\\
68.375	0.19182	199.719757423663	199.719757423663\\
68.375	0.19548	209.911779327454	209.911779327454\\
68.375	0.19914	220.410465032096	220.410465032096\\
68.375	0.2028	231.215814537587	231.215814537587\\
68.375	0.20646	242.327827843928	242.327827843928\\
68.375	0.21012	253.746504951119	253.746504951119\\
68.375	0.21378	265.47184585916	265.47184585916\\
68.375	0.21744	277.503850568052	277.503850568052\\
68.375	0.2211	289.842519077793	289.842519077793\\
68.375	0.22476	302.487851388384	302.487851388384\\
68.375	0.22842	315.439847499825	315.439847499825\\
68.375	0.23208	328.698507412117	328.698507412117\\
68.375	0.23574	342.263831125258	342.263831125258\\
68.375	0.2394	356.135818639249	356.135818639249\\
68.375	0.24306	370.314469954091	370.314469954091\\
68.375	0.24672	384.799785069782	384.799785069782\\
68.375	0.25038	399.591763986323	399.591763986323\\
68.375	0.25404	414.690406703714	414.690406703714\\
68.375	0.2577	430.095713221956	430.095713221956\\
68.375	0.26136	445.807683541047	445.807683541047\\
68.375	0.26502	461.826317660988	461.826317660988\\
68.375	0.26868	478.15161558178	478.15161558178\\
68.375	0.27234	494.783577303421	494.783577303421\\
68.375	0.276	511.722202825912	511.722202825912\\
68.75	0.093	41.0567769210084	41.0567769210084\\
68.75	0.09666	42.9635300740638	42.9635300740638\\
68.75	0.10032	45.1769470279692	45.1769470279692\\
68.75	0.10398	47.6970277827247	47.6970277827247\\
68.75	0.10764	50.5237723383302	50.5237723383302\\
68.75	0.1113	53.6571806947856	53.6571806947856\\
68.75	0.11496	57.0972528520911	57.0972528520911\\
68.75	0.11862	60.8439888102466	60.8439888102466\\
68.75	0.12228	64.8973885692521	64.8973885692521\\
68.75	0.12594	69.2574521291076	69.2574521291076\\
68.75	0.1296	73.9241794898131	73.9241794898131\\
68.75	0.13326	78.8975706513685	78.8975706513685\\
68.75	0.13692	84.1776256137741	84.1776256137741\\
68.75	0.14058	89.7643443770297	89.7643443770297\\
68.75	0.14424	95.6577269411352	95.6577269411352\\
68.75	0.1479	101.857773306091	101.857773306091\\
68.75	0.15156	108.364483471896	108.364483471896\\
68.75	0.15522	115.177857438552	115.177857438552\\
68.75	0.15888	122.297895206058	122.297895206058\\
68.75	0.16254	129.724596774413	129.724596774413\\
68.75	0.1662	137.457962143619	137.457962143619\\
68.75	0.16986	145.497991313674	145.497991313674\\
68.75	0.17352	153.84468428458	153.84468428458\\
68.75	0.17718	162.498041056335	162.498041056335\\
68.75	0.18084	171.458061628941	171.458061628941\\
68.75	0.1845	180.724746002397	180.724746002397\\
68.75	0.18816	190.298094176702	190.298094176702\\
68.75	0.19182	200.178106151858	200.178106151858\\
68.75	0.19548	210.364781927864	210.364781927864\\
68.75	0.19914	220.858121504719	220.858121504719\\
68.75	0.2028	231.658124882425	231.658124882425\\
68.75	0.20646	242.764792060981	242.764792060981\\
68.75	0.21012	254.178123040386	254.178123040386\\
68.75	0.21378	265.898117820642	265.898117820642\\
68.75	0.21744	277.924776401748	277.924776401748\\
68.75	0.2211	290.258098783704	290.258098783704\\
68.75	0.22476	302.898084966509	302.898084966509\\
68.75	0.22842	315.844734950165	315.844734950165\\
68.75	0.23208	329.098048734671	329.098048734671\\
68.75	0.23574	342.658026320027	342.658026320027\\
68.75	0.2394	356.524667706232	356.524667706232\\
68.75	0.24306	370.697972893288	370.697972893288\\
68.75	0.24672	385.177941881194	385.177941881194\\
68.75	0.25038	399.96457466995	399.96457466995\\
68.75	0.25404	415.057871259555	415.057871259555\\
68.75	0.2577	430.457831650011	430.457831650011\\
68.75	0.26136	446.164455841317	446.164455841317\\
68.75	0.26502	462.177743833473	462.177743833473\\
68.75	0.26868	478.497695626479	478.497695626479\\
68.75	0.27234	495.124311220335	495.124311220335\\
68.75	0.276	512.05759061504	512.05759061504\\
69.125	0.093	41.6704285302217	41.6704285302217\\
69.125	0.09666	43.5718355554916	43.5718355554916\\
69.125	0.10032	45.7799063816115	45.7799063816115\\
69.125	0.10398	48.2946410085815	48.2946410085815\\
69.125	0.10764	51.1160394364015	51.1160394364015\\
69.125	0.1113	54.2441016650714	54.2441016650714\\
69.125	0.11496	57.6788276945914	57.6788276945914\\
69.125	0.11862	61.4202175249614	61.4202175249614\\
69.125	0.12228	65.4682711561813	65.4682711561813\\
69.125	0.12594	69.8229885882514	69.8229885882514\\
69.125	0.1296	74.4843698211714	74.4843698211714\\
69.125	0.13326	79.4524148549413	79.4524148549413\\
69.125	0.13692	84.7271236895614	84.7271236895614\\
69.125	0.14058	90.3084963250315	90.3084963250315\\
69.125	0.14424	96.1965327613515	96.1965327613515\\
69.125	0.1479	102.391232998522	102.391232998522\\
69.125	0.15156	108.892597036542	108.892597036542\\
69.125	0.15522	115.700624875412	115.700624875412\\
69.125	0.15888	122.815316515132	122.815316515132\\
69.125	0.16254	130.236671955702	130.236671955702\\
69.125	0.1662	137.964691197122	137.964691197122\\
69.125	0.16986	145.999374239392	145.999374239392\\
69.125	0.17352	154.340721082512	154.340721082512\\
69.125	0.17718	162.988731726482	162.988731726482\\
69.125	0.18084	171.943406171302	171.943406171302\\
69.125	0.1845	181.204744416972	181.204744416972\\
69.125	0.18816	190.772746463493	190.772746463493\\
69.125	0.19182	200.647412310863	200.647412310863\\
69.125	0.19548	210.828741959083	210.828741959083\\
69.125	0.19914	221.316735408153	221.316735408153\\
69.125	0.2028	232.111392658073	232.111392658073\\
69.125	0.20646	243.212713708844	243.212713708844\\
69.125	0.21012	254.620698560464	254.620698560464\\
69.125	0.21378	266.335347212934	266.335347212934\\
69.125	0.21744	278.356659666254	278.356659666254\\
69.125	0.2211	290.684635920424	290.684635920424\\
69.125	0.22476	303.319275975444	303.319275975444\\
69.125	0.22842	316.260579831315	316.260579831315\\
69.125	0.23208	329.508547488035	329.508547488035\\
69.125	0.23574	343.063178945605	343.063178945605\\
69.125	0.2394	356.924474204025	356.924474204025\\
69.125	0.24306	371.092433263296	371.092433263296\\
69.125	0.24672	385.567056123416	385.567056123416\\
69.125	0.25038	400.348342784386	400.348342784386\\
69.125	0.25404	415.436293246207	415.436293246207\\
69.125	0.2577	430.830907508877	430.830907508877\\
69.125	0.26136	446.532185572397	446.532185572397\\
69.125	0.26502	462.540127436768	462.540127436768\\
69.125	0.26868	478.854733101988	478.854733101988\\
69.125	0.27234	495.476002568058	495.476002568058\\
69.125	0.276	512.403935834978	512.403935834978\\
69.5	0.093	42.2950375702449	42.2950375702449\\
69.5	0.09666	44.1910984677293	44.1910984677293\\
69.5	0.10032	46.3938231660637	46.3938231660637\\
69.5	0.10398	48.9032116652482	48.9032116652482\\
69.5	0.10764	51.7192639652826	51.7192639652826\\
69.5	0.1113	54.8419800661671	54.8419800661671\\
69.5	0.11496	58.2713599679015	58.2713599679015\\
69.5	0.11862	62.007403670486	62.007403670486\\
69.5	0.12228	66.0501111739205	66.0501111739205\\
69.5	0.12594	70.399482478205	70.399482478205\\
69.5	0.1296	75.0555175833395	75.0555175833395\\
69.5	0.13326	80.0182164893239	80.0182164893239\\
69.5	0.13692	85.2875791961585	85.2875791961585\\
69.5	0.14058	90.863605703843	90.863605703843\\
69.5	0.14424	96.7462960123776	96.7462960123776\\
69.5	0.1479	102.935650121762	102.935650121762\\
69.5	0.15156	109.431668031997	109.431668031997\\
69.5	0.15522	116.234349743081	116.234349743081\\
69.5	0.15888	123.343695255016	123.343695255016\\
69.5	0.16254	130.7597045678	130.7597045678\\
69.5	0.1662	138.482377681435	138.482377681435\\
69.5	0.16986	146.51171459592	146.51171459592\\
69.5	0.17352	154.847715311254	154.847715311254\\
69.5	0.17718	163.490379827439	163.490379827439\\
69.5	0.18084	172.439708144473	172.439708144473\\
69.5	0.1845	181.695700262358	181.695700262358\\
69.5	0.18816	191.258356181093	191.258356181093\\
69.5	0.19182	201.127675900677	201.127675900677\\
69.5	0.19548	211.303659421112	211.303659421112\\
69.5	0.19914	221.786306742397	221.786306742397\\
69.5	0.2028	232.575617864531	232.575617864531\\
69.5	0.20646	243.671592787516	243.671592787516\\
69.5	0.21012	255.074231511351	255.074231511351\\
69.5	0.21378	266.783534036035	266.783534036035\\
69.5	0.21744	278.79950036157	278.79950036157\\
69.5	0.2211	291.122130487955	291.122130487955\\
69.5	0.22476	303.751424415189	303.751424415189\\
69.5	0.22842	316.687382143274	316.687382143274\\
69.5	0.23208	329.930003672209	329.930003672209\\
69.5	0.23574	343.479289001994	343.479289001994\\
69.5	0.2394	357.335238132628	357.335238132628\\
69.5	0.24306	371.497851064113	371.497851064113\\
69.5	0.24672	385.967127796448	385.967127796448\\
69.5	0.25038	400.743068329633	400.743068329633\\
69.5	0.25404	415.825672663668	415.825672663668\\
69.5	0.2577	431.214940798552	431.214940798552\\
69.5	0.26136	446.910872734287	446.910872734287\\
69.5	0.26502	462.913468470872	462.913468470872\\
69.5	0.26868	479.222728008307	479.222728008307\\
69.5	0.27234	495.838651346592	495.838651346592\\
69.5	0.276	512.761238485726	512.761238485726\\
69.875	0.093	42.930604041078	42.930604041078\\
69.875	0.09666	44.8213188107769	44.8213188107769\\
69.875	0.10032	47.0186973813259	47.0186973813259\\
69.875	0.10398	49.5227397527248	49.5227397527248\\
69.875	0.10764	52.3334459249737	52.3334459249737\\
69.875	0.1113	55.4508158980726	55.4508158980726\\
69.875	0.11496	58.8748496720216	58.8748496720216\\
69.875	0.11862	62.6055472468206	62.6055472468206\\
69.875	0.12228	66.6429086224696	66.6429086224696\\
69.875	0.12594	70.9869337989686	70.9869337989686\\
69.875	0.1296	75.6376227763175	75.6376227763175\\
69.875	0.13326	80.5949755545165	80.5949755545165\\
69.875	0.13692	85.8589921335655	85.8589921335655\\
69.875	0.14058	91.4296725134646	91.4296725134646\\
69.875	0.14424	97.3070166942137	97.3070166942137\\
69.875	0.1479	103.491024675813	103.491024675813\\
69.875	0.15156	109.981696458262	109.981696458262\\
69.875	0.15522	116.779032041561	116.779032041561\\
69.875	0.15888	123.88303142571	123.88303142571\\
69.875	0.16254	131.293694610709	131.293694610709\\
69.875	0.1662	139.011021596558	139.011021596558\\
69.875	0.16986	147.035012383257	147.035012383257\\
69.875	0.17352	155.365666970806	155.365666970806\\
69.875	0.17718	164.002985359205	164.002985359205\\
69.875	0.18084	172.946967548454	172.946967548454\\
69.875	0.1845	182.197613538554	182.197613538554\\
69.875	0.18816	191.754923329503	191.754923329503\\
69.875	0.19182	201.618896921302	201.618896921302\\
69.875	0.19548	211.789534313951	211.789534313951\\
69.875	0.19914	222.26683550745	222.26683550745\\
69.875	0.2028	233.050800501799	233.050800501799\\
69.875	0.20646	244.141429296999	244.141429296999\\
69.875	0.21012	255.538721893048	255.538721893048\\
69.875	0.21378	267.242678289947	267.242678289947\\
69.875	0.21744	279.253298487696	279.253298487696\\
69.875	0.2211	291.570582486295	291.570582486295\\
69.875	0.22476	304.194530285744	304.194530285744\\
69.875	0.22842	317.125141886044	317.125141886044\\
69.875	0.23208	330.362417287193	330.362417287193\\
69.875	0.23574	343.906356489192	343.906356489192\\
69.875	0.2394	357.756959492041	357.756959492041\\
69.875	0.24306	371.914226295741	371.914226295741\\
69.875	0.24672	386.37815690029	386.37815690029\\
69.875	0.25038	401.148751305689	401.148751305689\\
69.875	0.25404	416.226009511938	416.226009511938\\
69.875	0.2577	431.609931519038	431.609931519038\\
69.875	0.26136	447.300517326987	447.300517326987\\
69.875	0.26502	463.297766935786	463.297766935786\\
69.875	0.26868	479.601680345436	479.601680345436\\
69.875	0.27234	496.212257555935	496.212257555935\\
69.875	0.276	513.129498567284	513.129498567284\\
70.25	0.093	43.5771279427211	43.5771279427211\\
70.25	0.09666	45.4624965846345	45.4624965846345\\
70.25	0.10032	47.6545290273979	47.6545290273979\\
70.25	0.10398	50.1532252710113	50.1532252710113\\
70.25	0.10764	52.9585853154748	52.9585853154748\\
70.25	0.1113	56.0706091607882	56.0706091607882\\
70.25	0.11496	59.4892968069517	59.4892968069517\\
70.25	0.11862	63.2146482539652	63.2146482539652\\
70.25	0.12228	67.2466635018286	67.2466635018286\\
70.25	0.12594	71.5853425505421	71.5853425505421\\
70.25	0.1296	76.2306854001056	76.2306854001056\\
70.25	0.13326	81.182692050519	81.182692050519\\
70.25	0.13692	86.4413625017825	86.4413625017825\\
70.25	0.14058	92.0066967538961	92.0066967538961\\
70.25	0.14424	97.8786948068596	97.8786948068596\\
70.25	0.1479	104.057356660673	104.057356660673\\
70.25	0.15156	110.542682315337	110.542682315337\\
70.25	0.15522	117.33467177085	117.33467177085\\
70.25	0.15888	124.433325027214	124.433325027214\\
70.25	0.16254	131.838642084427	131.838642084427\\
70.25	0.1662	139.550622942491	139.550622942491\\
70.25	0.16986	147.569267601405	147.569267601405\\
70.25	0.17352	155.894576061168	155.894576061168\\
70.25	0.17718	164.526548321782	164.526548321782\\
70.25	0.18084	173.465184383245	173.465184383245\\
70.25	0.1845	182.710484245559	182.710484245559\\
70.25	0.18816	192.262447908723	192.262447908723\\
70.25	0.19182	202.121075372736	202.121075372736\\
70.25	0.19548	212.2863666376	212.2863666376\\
70.25	0.19914	222.758321703313	222.758321703313\\
70.25	0.2028	233.536940569877	233.536940569877\\
70.25	0.20646	244.622223237291	244.622223237291\\
70.25	0.21012	256.014169705555	256.014169705555\\
70.25	0.21378	267.712779974668	267.712779974668\\
70.25	0.21744	279.718054044632	279.718054044632\\
70.25	0.2211	292.029991915446	292.029991915446\\
70.25	0.22476	304.648593587109	304.648593587109\\
70.25	0.22842	317.573859059623	317.573859059623\\
70.25	0.23208	330.805788332987	330.805788332987\\
70.25	0.23574	344.344381407201	344.344381407201\\
70.25	0.2394	358.189638282264	358.189638282264\\
70.25	0.24306	372.341558958178	372.341558958178\\
70.25	0.24672	386.800143434942	386.800143434942\\
70.25	0.25038	401.565391712556	401.565391712556\\
70.25	0.25404	416.637303791019	416.637303791019\\
70.25	0.2577	432.015879670333	432.015879670333\\
70.25	0.26136	447.701119350497	447.701119350497\\
70.25	0.26502	463.693022831511	463.693022831511\\
70.25	0.26868	479.991590113374	479.991590113374\\
70.25	0.27234	496.596821196088	496.596821196088\\
70.25	0.276	513.508716079652	513.508716079652\\
70.625	0.093	44.234609275174	44.234609275174\\
70.625	0.09666	46.1146317893019	46.1146317893019\\
70.625	0.10032	48.3013181042798	48.3013181042798\\
70.625	0.10398	50.7946682201078	50.7946682201078\\
70.625	0.10764	53.5946821367857	53.5946821367857\\
70.625	0.1113	56.7013598543136	56.7013598543136\\
70.625	0.11496	60.1147013726916	60.1147013726916\\
70.625	0.11862	63.8347066919195	63.8347066919195\\
70.625	0.12228	67.8613758119975	67.8613758119975\\
70.625	0.12594	72.1947087329255	72.1947087329255\\
70.625	0.1296	76.8347054547035	76.8347054547035\\
70.625	0.13326	81.7813659773314	81.7813659773314\\
70.625	0.13692	87.0346903008094	87.0346903008094\\
70.625	0.14058	92.5946784251375	92.5946784251375\\
70.625	0.14424	98.4613303503155	98.4613303503155\\
70.625	0.1479	104.634646076344	104.634646076344\\
70.625	0.15156	111.114625603221	111.114625603221\\
70.625	0.15522	117.90126893095	117.90126893095\\
70.625	0.15888	124.994576059528	124.994576059528\\
70.625	0.16254	132.394546988956	132.394546988956\\
70.625	0.1662	140.101181719234	140.101181719234\\
70.625	0.16986	148.114480250362	148.114480250362\\
70.625	0.17352	156.43444258234	156.43444258234\\
70.625	0.17718	165.061068715168	165.061068715168\\
70.625	0.18084	173.994358648846	173.994358648846\\
70.625	0.1845	183.234312383374	183.234312383374\\
70.625	0.18816	192.780929918752	192.780929918752\\
70.625	0.19182	202.634211254981	202.634211254981\\
70.625	0.19548	212.794156392059	212.794156392059\\
70.625	0.19914	223.260765329987	223.260765329987\\
70.625	0.2028	234.034038068765	234.034038068765\\
70.625	0.20646	245.113974608393	245.113974608393\\
70.625	0.21012	256.500574948871	256.500574948871\\
70.625	0.21378	268.193839090199	268.193839090199\\
70.625	0.21744	280.193767032378	280.193767032378\\
70.625	0.2211	292.500358775406	292.500358775406\\
70.625	0.22476	305.113614319284	305.113614319284\\
70.625	0.22842	318.033533664012	318.033533664012\\
70.625	0.23208	331.26011680959	331.26011680959\\
70.625	0.23574	344.793363756019	344.793363756019\\
70.625	0.2394	358.633274503297	358.633274503297\\
70.625	0.24306	372.779849051425	372.779849051425\\
70.625	0.24672	387.233087400403	387.233087400403\\
70.625	0.25038	401.992989550232	401.992989550232\\
70.625	0.25404	417.05955550091	417.05955550091\\
70.625	0.2577	432.432785252438	432.432785252438\\
70.625	0.26136	448.112678804817	448.112678804817\\
70.625	0.26502	464.099236158045	464.099236158045\\
70.625	0.26868	480.392457312123	480.392457312123\\
70.625	0.27234	496.992342267052	496.992342267052\\
70.625	0.276	513.89889102283	513.89889102283\\
71	0.093	44.9030480384369	44.9030480384369\\
71	0.09666	46.7777244247793	46.7777244247793\\
71	0.10032	48.9590646119717	48.9590646119717\\
71	0.10398	51.4470686000141	51.4470686000141\\
71	0.10764	54.2417363889065	54.2417363889065\\
71	0.1113	57.343067978649	57.343067978649\\
71	0.11496	60.7510633692414	60.7510633692414\\
71	0.11862	64.4657225606839	64.4657225606839\\
71	0.12228	68.4870455529763	68.4870455529763\\
71	0.12594	72.8150323461188	72.8150323461188\\
71	0.1296	77.4496829401112	77.4496829401112\\
71	0.13326	82.3909973349537	82.3909973349537\\
71	0.13692	87.6389755306462	87.6389755306462\\
71	0.14058	93.1936175271888	93.1936175271888\\
71	0.14424	99.0549233245813	99.0549233245813\\
71	0.1479	105.222892922824	105.222892922824\\
71	0.15156	111.697526321916	111.697526321916\\
71	0.15522	118.478823521859	118.478823521859\\
71	0.15888	125.566784522652	125.566784522652\\
71	0.16254	132.961409324294	132.961409324294\\
71	0.1662	140.662697926787	140.662697926787\\
71	0.16986	148.670650330129	148.670650330129\\
71	0.17352	156.985266534322	156.985266534322\\
71	0.17718	165.606546539364	165.606546539364\\
71	0.18084	174.534490345257	174.534490345257\\
71	0.1845	183.769097952	183.769097952\\
71	0.18816	193.310369359592	193.310369359592\\
71	0.19182	203.158304568035	203.158304568035\\
71	0.19548	213.312903577327	213.312903577327\\
71	0.19914	223.77416638747	223.77416638747\\
71	0.2028	234.542092998463	234.542092998463\\
71	0.20646	245.616683410305	245.616683410305\\
71	0.21012	256.997937622998	256.997937622998\\
71	0.21378	268.685855636541	268.685855636541\\
71	0.21744	280.680437450933	280.680437450933\\
71	0.2211	292.981683066176	292.981683066176\\
71	0.22476	305.589592482269	305.589592482269\\
71	0.22842	318.504165699212	318.504165699212\\
71	0.23208	331.725402717004	331.725402717004\\
71	0.23574	345.253303535647	345.253303535647\\
71	0.2394	359.08786815514	359.08786815514\\
71	0.24306	373.229096575483	373.229096575483\\
71	0.24672	387.676988796675	387.676988796675\\
71	0.25038	402.431544818718	402.431544818718\\
71	0.25404	417.492764641611	417.492764641611\\
71	0.2577	432.860648265354	432.860648265354\\
71	0.26136	448.535195689946	448.535195689946\\
71	0.26502	464.516406915389	464.516406915389\\
71	0.26868	480.804281941682	480.804281941682\\
71	0.27234	497.398820768825	497.398820768825\\
71	0.276	514.300023396817	514.300023396817\\
71.375	0.093	45.5824442325097	45.5824442325097\\
71.375	0.09666	47.4517744910666	47.4517744910666\\
71.375	0.10032	49.6277685504735	49.6277685504735\\
71.375	0.10398	52.1104264107304	52.1104264107304\\
71.375	0.10764	54.8997480718374	54.8997480718374\\
71.375	0.1113	57.9957335337942	57.9957335337942\\
71.375	0.11496	61.3983827966012	61.3983827966012\\
71.375	0.11862	65.1076958602582	65.1076958602582\\
71.375	0.12228	69.1236727247652	69.1236727247652\\
71.375	0.12594	73.4463133901221	73.4463133901221\\
71.375	0.1296	78.075617856329	78.075617856329\\
71.375	0.13326	83.011586123386	83.011586123386\\
71.375	0.13692	88.254218191293	88.254218191293\\
71.375	0.14058	93.8035140600501	93.8035140600501\\
71.375	0.14424	99.6594737296571	99.6594737296571\\
71.375	0.1479	105.822097200114	105.822097200114\\
71.375	0.15156	112.291384471421	112.291384471421\\
71.375	0.15522	119.067335543578	119.067335543578\\
71.375	0.15888	126.149950416585	126.149950416585\\
71.375	0.16254	133.539229090442	133.539229090442\\
71.375	0.1662	141.235171565149	141.235171565149\\
71.375	0.16986	149.237777840706	149.237777840706\\
71.375	0.17352	157.547047917113	157.547047917113\\
71.375	0.17718	166.162981794371	166.162981794371\\
71.375	0.18084	175.085579472478	175.085579472478\\
71.375	0.1845	184.314840951435	184.314840951435\\
71.375	0.18816	193.850766231242	193.850766231242\\
71.375	0.19182	203.693355311899	203.693355311899\\
71.375	0.19548	213.842608193406	213.842608193406\\
71.375	0.19914	224.298524875763	224.298524875763\\
71.375	0.2028	235.06110535897	235.06110535897\\
71.375	0.20646	246.130349643028	246.130349643028\\
71.375	0.21012	257.506257727935	257.506257727935\\
71.375	0.21378	269.188829613692	269.188829613692\\
71.375	0.21744	281.178065300299	281.178065300299\\
71.375	0.2211	293.473964787756	293.473964787756\\
71.375	0.22476	306.076528076063	306.076528076063\\
71.375	0.22842	318.985755165221	318.985755165221\\
71.375	0.23208	332.201646055228	332.201646055228\\
71.375	0.23574	345.724200746085	345.724200746085\\
71.375	0.2394	359.553419237792	359.553419237792\\
71.375	0.24306	373.68930153035	373.68930153035\\
71.375	0.24672	388.131847623757	388.131847623757\\
71.375	0.25038	402.881057518014	402.881057518014\\
71.375	0.25404	417.936931213121	417.936931213121\\
71.375	0.2577	433.299468709079	433.299468709079\\
71.375	0.26136	448.968670005886	448.968670005886\\
71.375	0.26502	464.944535103543	464.944535103543\\
71.375	0.26868	481.22706400205	481.22706400205\\
71.375	0.27234	497.816256701408	497.816256701408\\
71.375	0.276	514.712113201615	514.712113201615\\
71.75	0.093	46.2727978573924	46.2727978573924\\
71.75	0.09666	48.1367819881638	48.1367819881638\\
71.75	0.10032	50.3074299197852	50.3074299197852\\
71.75	0.10398	52.7847416522566	52.7847416522566\\
71.75	0.10764	55.568717185578	55.568717185578\\
71.75	0.1113	58.6593565197495	58.6593565197495\\
71.75	0.11496	62.0566596547709	62.0566596547709\\
71.75	0.11862	65.7606265906423	65.7606265906423\\
71.75	0.12228	69.7712573273638	69.7712573273638\\
71.75	0.12594	74.0885518649353	74.0885518649353\\
71.75	0.1296	78.7125102033567	78.7125102033567\\
71.75	0.13326	83.6431323426281	83.6431323426281\\
71.75	0.13692	88.8804182827496	88.8804182827496\\
71.75	0.14058	94.4243680237212	94.4243680237212\\
71.75	0.14424	100.274981565543	100.274981565543\\
71.75	0.1479	106.432258908214	106.432258908214\\
71.75	0.15156	112.896200051736	112.896200051736\\
71.75	0.15522	119.666804996107	119.666804996107\\
71.75	0.15888	126.744073741329	126.744073741329\\
71.75	0.16254	134.1280062874	134.1280062874\\
71.75	0.1662	141.818602634322	141.818602634322\\
71.75	0.16986	149.815862782093	149.815862782093\\
71.75	0.17352	158.119786730715	158.119786730715\\
71.75	0.17718	166.730374480187	166.730374480187\\
71.75	0.18084	175.647626030508	175.647626030508\\
71.75	0.1845	184.87154138168	184.87154138168\\
71.75	0.18816	194.402120533701	194.402120533701\\
71.75	0.19182	204.239363486573	204.239363486573\\
71.75	0.19548	214.383270240295	214.383270240295\\
71.75	0.19914	224.833840794866	224.833840794866\\
71.75	0.2028	235.591075150288	235.591075150288\\
71.75	0.20646	246.65497330656	246.65497330656\\
71.75	0.21012	258.025535263681	258.025535263681\\
71.75	0.21378	269.702761021653	269.702761021653\\
71.75	0.21744	281.686650580475	281.686650580475\\
71.75	0.2211	293.977203940146	293.977203940146\\
71.75	0.22476	306.574421100668	306.574421100668\\
71.75	0.22842	319.47830206204	319.47830206204\\
71.75	0.23208	332.688846824261	332.688846824261\\
71.75	0.23574	346.206055387333	346.206055387333\\
71.75	0.2394	360.029927751255	360.029927751255\\
71.75	0.24306	374.160463916027	374.160463916027\\
71.75	0.24672	388.597663881648	388.597663881648\\
71.75	0.25038	403.34152764812	403.34152764812\\
71.75	0.25404	418.392055215442	418.392055215442\\
71.75	0.2577	433.749246583614	433.749246583614\\
71.75	0.26136	449.413101752635	449.413101752635\\
71.75	0.26502	465.383620722507	465.383620722507\\
71.75	0.26868	481.660803493229	481.660803493229\\
71.75	0.27234	498.244650064801	498.244650064801\\
71.75	0.276	515.135160437222	515.135160437222\\
72.125	0.093	46.9741089130851	46.9741089130851\\
72.125	0.09666	48.8327469160709	48.8327469160709\\
72.125	0.10032	50.9980487199068	50.9980487199068\\
72.125	0.10398	53.4700143245928	53.4700143245928\\
72.125	0.10764	56.2486437301286	56.2486437301286\\
72.125	0.1113	59.3339369365145	59.3339369365145\\
72.125	0.11496	62.7258939437505	62.7258939437505\\
72.125	0.11862	66.4245147518365	66.4245147518365\\
72.125	0.12228	70.4297993607724	70.4297993607724\\
72.125	0.12594	74.7417477705583	74.7417477705583\\
72.125	0.1296	79.3603599811943	79.3603599811943\\
72.125	0.13326	84.2856359926802	84.2856359926802\\
72.125	0.13692	89.5175758050162	89.5175758050162\\
72.125	0.14058	95.0561794182023	95.0561794182023\\
72.125	0.14424	100.901446832238	100.901446832238\\
72.125	0.1479	107.053378047124	107.053378047124\\
72.125	0.15156	113.51197306286	113.51197306286\\
72.125	0.15522	120.277231879446	120.277231879446\\
72.125	0.15888	127.349154496882	127.349154496882\\
72.125	0.16254	134.727740915168	134.727740915168\\
72.125	0.1662	142.412991134304	142.412991134304\\
72.125	0.16986	150.404905154291	150.404905154291\\
72.125	0.17352	158.703482975127	158.703482975127\\
72.125	0.17718	167.308724596813	167.308724596813\\
72.125	0.18084	176.220630019349	176.220630019349\\
72.125	0.1845	185.439199242735	185.439199242735\\
72.125	0.18816	194.964432266971	194.964432266971\\
72.125	0.19182	204.796329092057	204.796329092057\\
72.125	0.19548	214.934889717993	214.934889717993\\
72.125	0.19914	225.380114144779	225.380114144779\\
72.125	0.2028	236.132002372416	236.132002372416\\
72.125	0.20646	247.190554400902	247.190554400902\\
72.125	0.21012	258.555770230238	258.555770230238\\
72.125	0.21378	270.227649860424	270.227649860424\\
72.125	0.21744	282.20619329146	282.20619329146\\
72.125	0.2211	294.491400523346	294.491400523346\\
72.125	0.22476	307.083271556082	307.083271556082\\
72.125	0.22842	319.981806389669	319.981806389669\\
72.125	0.23208	333.187005024105	333.187005024105\\
72.125	0.23574	346.698867459391	346.698867459391\\
72.125	0.2394	360.517393695527	360.517393695527\\
72.125	0.24306	374.642583732514	374.642583732514\\
72.125	0.24672	389.07443757035	389.07443757035\\
72.125	0.25038	403.812955209036	403.812955209036\\
72.125	0.25404	418.858136648572	418.858136648572\\
72.125	0.2577	434.209981888959	434.209981888959\\
72.125	0.26136	449.868490930195	449.868490930195\\
72.125	0.26502	465.833663772281	465.833663772281\\
72.125	0.26868	482.105500415217	482.105500415217\\
72.125	0.27234	498.684000859004	498.684000859004\\
72.125	0.276	515.56916510364	515.56916510364\\
72.5	0.093	47.6863773995876	47.6863773995876\\
72.5	0.09666	49.539669274788	49.539669274788\\
72.5	0.10032	51.6996249508384	51.6996249508384\\
72.5	0.10398	54.1662444277388	54.1662444277388\\
72.5	0.10764	56.9395277054892	56.9395277054892\\
72.5	0.1113	60.0194747840896	60.0194747840896\\
72.5	0.11496	63.40608566354	63.40608566354\\
72.5	0.11862	67.0993603438405	67.0993603438405\\
72.5	0.12228	71.0992988249909	71.0992988249909\\
72.5	0.12594	75.4059011069914	75.4059011069914\\
72.5	0.1296	80.0191671898419	80.0191671898419\\
72.5	0.13326	84.9390970735423	84.9390970735423\\
72.5	0.13692	90.1656907580927	90.1656907580927\\
72.5	0.14058	95.6989482434933	95.6989482434933\\
72.5	0.14424	101.538869529744	101.538869529744\\
72.5	0.1479	107.685454616844	107.685454616844\\
72.5	0.15156	114.138703504795	114.138703504795\\
72.5	0.15522	120.898616193595	120.898616193595\\
72.5	0.15888	127.965192683246	127.965192683246\\
72.5	0.16254	135.338432973746	135.338432973746\\
72.5	0.1662	143.018337065097	143.018337065097\\
72.5	0.16986	151.004904957297	151.004904957297\\
72.5	0.17352	159.298136650348	159.298136650348\\
72.5	0.17718	167.898032144249	167.898032144249\\
72.5	0.18084	176.804591438999	176.804591438999\\
72.5	0.1845	186.0178145346	186.0178145346\\
72.5	0.18816	195.53770143105	195.53770143105\\
72.5	0.19182	205.364252128351	205.364252128351\\
72.5	0.19548	215.497466626502	215.497466626502\\
72.5	0.19914	225.937344925502	225.937344925502\\
72.5	0.2028	236.683887025353	236.683887025353\\
72.5	0.20646	247.737092926054	247.737092926054\\
72.5	0.21012	259.096962627604	259.096962627604\\
72.5	0.21378	270.763496130005	270.763496130005\\
72.5	0.21744	282.736693433255	282.736693433255\\
72.5	0.2211	295.016554537356	295.016554537356\\
72.5	0.22476	307.603079442307	307.603079442307\\
72.5	0.22842	320.496268148107	320.496268148107\\
72.5	0.23208	333.696120654758	333.696120654758\\
72.5	0.23574	347.202636962259	347.202636962259\\
72.5	0.2394	361.01581707061	361.01581707061\\
72.5	0.24306	375.13566097981	375.13566097981\\
72.5	0.24672	389.562168689861	389.562168689861\\
72.5	0.25038	404.295340200762	404.295340200762\\
72.5	0.25404	419.335175512513	419.335175512513\\
72.5	0.2577	434.681674625113	434.681674625113\\
72.5	0.26136	450.334837538564	450.334837538564\\
72.5	0.26502	466.294664252865	466.294664252865\\
72.5	0.26868	482.561154768016	482.561154768016\\
72.5	0.27234	499.134309084016	499.134309084016\\
72.5	0.276	516.014127200867	516.014127200867\\
72.875	0.093	48.4096033169001	48.4096033169001\\
72.875	0.09666	50.257549064315	50.257549064315\\
72.875	0.10032	52.4121586125799	52.4121586125799\\
72.875	0.10398	54.8734319616947	54.8734319616947\\
72.875	0.10764	57.6413691116597	57.6413691116597\\
72.875	0.1113	60.7159700624746	60.7159700624746\\
72.875	0.11496	64.0972348141395	64.0972348141395\\
72.875	0.11862	67.7851633666544	67.7851633666544\\
72.875	0.12228	71.7797557200194	71.7797557200194\\
72.875	0.12594	76.0810118742344	76.0810118742344\\
72.875	0.1296	80.6889318292993	80.6889318292993\\
72.875	0.13326	85.6035155852142	85.6035155852142\\
72.875	0.13692	90.8247631419792	90.8247631419792\\
72.875	0.14058	96.3526744995942	96.3526744995942\\
72.875	0.14424	102.187249658059	102.187249658059\\
72.875	0.1479	108.328488617374	108.328488617374\\
72.875	0.15156	114.776391377539	114.776391377539\\
72.875	0.15522	121.530957938554	121.530957938554\\
72.875	0.15888	128.592188300419	128.592188300419\\
72.875	0.16254	135.960082463134	135.960082463134\\
72.875	0.1662	143.634640426699	143.634640426699\\
72.875	0.16986	151.615862191114	151.615862191114\\
72.875	0.17352	159.903747756379	159.903747756379\\
72.875	0.17718	168.498297122494	168.498297122494\\
72.875	0.18084	177.399510289459	177.399510289459\\
72.875	0.1845	186.607387257275	186.607387257275\\
72.875	0.18816	196.12192802594	196.12192802594\\
72.875	0.19182	205.943132595455	205.943132595455\\
72.875	0.19548	216.07100096582	216.07100096582\\
72.875	0.19914	226.505533137035	226.505533137035\\
72.875	0.2028	237.2467291091	237.2467291091\\
72.875	0.20646	248.294588882015	248.294588882015\\
72.875	0.21012	259.64911245578	259.64911245578\\
72.875	0.21378	271.310299830396	271.310299830396\\
72.875	0.21744	283.278151005861	283.278151005861\\
72.875	0.2211	295.552665982176	295.552665982176\\
72.875	0.22476	308.133844759341	308.133844759341\\
72.875	0.22842	321.021687337356	321.021687337356\\
72.875	0.23208	334.216193716221	334.216193716221\\
72.875	0.23574	347.717363895937	347.717363895937\\
72.875	0.2394	361.525197876502	361.525197876502\\
72.875	0.24306	375.639695657917	375.639695657917\\
72.875	0.24672	390.060857240182	390.060857240182\\
72.875	0.25038	404.788682623298	404.788682623298\\
72.875	0.25404	419.823171807263	419.823171807263\\
72.875	0.2577	435.164324792078	435.164324792078\\
72.875	0.26136	450.812141577743	450.812141577743\\
72.875	0.26502	466.766622164259	466.766622164259\\
72.875	0.26868	483.027766551624	483.027766551624\\
72.875	0.27234	499.595574739839	499.595574739839\\
72.875	0.276	516.470046728904	516.470046728904\\
73.25	0.093	49.1437866650225	49.1437866650225\\
73.25	0.09666	50.9863862846518	50.9863862846518\\
73.25	0.10032	53.1356497051312	53.1356497051312\\
73.25	0.10398	55.5915769264606	55.5915769264606\\
73.25	0.10764	58.35416794864	58.35416794864\\
73.25	0.1113	61.4234227716694	61.4234227716694\\
73.25	0.11496	64.7993413955488	64.7993413955488\\
73.25	0.11862	68.4819238202782	68.4819238202782\\
73.25	0.12228	72.4711700458577	72.4711700458577\\
73.25	0.12594	76.7670800722871	76.7670800722871\\
73.25	0.1296	81.3696538995666	81.3696538995666\\
73.25	0.13326	86.278891527696	86.278891527696\\
73.25	0.13692	91.4947929566754	91.4947929566754\\
73.25	0.14058	97.017358186505	97.017358186505\\
73.25	0.14424	102.846587217184	102.846587217184\\
73.25	0.1479	108.982480048714	108.982480048714\\
73.25	0.15156	115.425036681093	115.425036681093\\
73.25	0.15522	122.174257114323	122.174257114323\\
73.25	0.15888	129.230141348403	129.230141348403\\
73.25	0.16254	136.592689383332	136.592689383332\\
73.25	0.1662	144.261901219112	144.261901219112\\
73.25	0.16986	152.237776855741	152.237776855741\\
73.25	0.17352	160.520316293221	160.520316293221\\
73.25	0.17718	169.10951953155	169.10951953155\\
73.25	0.18084	178.00538657073	178.00538657073\\
73.25	0.1845	187.207917410759	187.207917410759\\
73.25	0.18816	196.717112051639	196.717112051639\\
73.25	0.19182	206.532970493369	206.532970493369\\
73.25	0.19548	216.655492735948	216.655492735948\\
73.25	0.19914	227.084678779378	227.084678779378\\
73.25	0.2028	237.820528623658	237.820528623658\\
73.25	0.20646	248.863042268787	248.863042268787\\
73.25	0.21012	260.212219714767	260.212219714767\\
73.25	0.21378	271.868060961596	271.868060961596\\
73.25	0.21744	283.830566009276	283.830566009276\\
73.25	0.2211	296.099734857806	296.099734857806\\
73.25	0.22476	308.675567507185	308.675567507185\\
73.25	0.22842	321.558063957415	321.558063957415\\
73.25	0.23208	334.747224208495	334.747224208495\\
73.25	0.23574	348.243048260424	348.243048260424\\
73.25	0.2394	362.045536113204	362.045536113204\\
73.25	0.24306	376.154687766834	376.154687766834\\
73.25	0.24672	390.570503221314	390.570503221314\\
73.25	0.25038	405.292982476643	405.292982476643\\
73.25	0.25404	420.322125532823	420.322125532823\\
73.25	0.2577	435.657932389853	435.657932389853\\
73.25	0.26136	451.300403047733	451.300403047733\\
73.25	0.26502	467.249537506462	467.249537506462\\
73.25	0.26868	483.505335766042	483.505335766042\\
73.25	0.27234	500.067797826472	500.067797826472\\
73.25	0.276	516.936923687751	516.936923687751\\
73.625	0.093	49.8889274439548	49.8889274439548\\
73.625	0.09666	51.7261809357987	51.7261809357987\\
73.625	0.10032	53.8700982284926	53.8700982284926\\
73.625	0.10398	56.3206793220364	56.3206793220364\\
73.625	0.10764	59.0779242164303	59.0779242164303\\
73.625	0.1113	62.1418329116742	62.1418329116742\\
73.625	0.11496	65.5124054077681	65.5124054077681\\
73.625	0.11862	69.1896417047121	69.1896417047121\\
73.625	0.12228	73.173541802506	73.173541802506\\
73.625	0.12594	77.4641057011499	77.4641057011499\\
73.625	0.1296	82.0613334006439	82.0613334006439\\
73.625	0.13326	86.9652249009878	86.9652249009878\\
73.625	0.13692	92.1757802021817	92.1757802021817\\
73.625	0.14058	97.6929993042258	97.6929993042258\\
73.625	0.14424	103.51688220712	103.51688220712\\
73.625	0.1479	109.647428910864	109.647428910864\\
73.625	0.15156	116.084639415458	116.084639415458\\
73.625	0.15522	122.828513720902	122.828513720902\\
73.625	0.15888	129.879051827196	129.879051827196\\
73.625	0.16254	137.23625373434	137.23625373434\\
73.625	0.1662	144.900119442334	144.900119442334\\
73.625	0.16986	152.870648951178	152.870648951178\\
73.625	0.17352	161.147842260872	161.147842260872\\
73.625	0.17718	169.731699371416	169.731699371416\\
73.625	0.18084	178.62222028281	178.62222028281\\
73.625	0.1845	187.819404995054	187.819404995054\\
73.625	0.18816	197.323253508148	197.323253508148\\
73.625	0.19182	207.133765822092	207.133765822092\\
73.625	0.19548	217.250941936886	217.250941936886\\
73.625	0.19914	227.67478185253	227.67478185253\\
73.625	0.2028	238.405285569025	238.405285569025\\
73.625	0.20646	249.442453086369	249.442453086369\\
73.625	0.21012	260.786284404563	260.786284404563\\
73.625	0.21378	272.436779523607	272.436779523607\\
73.625	0.21744	284.393938443501	284.393938443501\\
73.625	0.2211	296.657761164245	296.657761164245\\
73.625	0.22476	309.22824768584	309.22824768584\\
73.625	0.22842	322.105398008284	322.105398008284\\
73.625	0.23208	335.289212131578	335.289212131578\\
73.625	0.23574	348.779690055722	348.779690055722\\
73.625	0.2394	362.576831780716	362.576831780716\\
73.625	0.24306	376.68063730656	376.68063730656\\
73.625	0.24672	391.091106633255	391.091106633255\\
73.625	0.25038	405.808239760799	405.808239760799\\
73.625	0.25404	420.832036689193	420.832036689193\\
73.625	0.2577	436.162497418437	436.162497418437\\
73.625	0.26136	451.799621948532	451.799621948532\\
73.625	0.26502	467.743410279476	467.743410279476\\
73.625	0.26868	483.99386241127	483.99386241127\\
73.625	0.27234	500.550978343914	500.550978343914\\
73.625	0.276	517.414758077409	517.414758077409\\
74	0.093	50.645025653697	50.645025653697\\
74	0.09666	52.4769330177554	52.4769330177554\\
74	0.10032	54.6155041826638	54.6155041826638\\
74	0.10398	57.0607391484222	57.0607391484222\\
74	0.10764	59.8126379150305	59.8126379150305\\
74	0.1113	62.8712004824889	62.8712004824889\\
74	0.11496	66.2364268507974	66.2364268507974\\
74	0.11862	69.9083170199557	69.9083170199557\\
74	0.12228	73.8868709899643	73.8868709899643\\
74	0.12594	78.1720887608226	78.1720887608226\\
74	0.1296	82.7639703325311	82.7639703325311\\
74	0.13326	87.6625157050895	87.6625157050895\\
74	0.13692	92.867724878498	92.867724878498\\
74	0.14058	98.3795978527564	98.3795978527564\\
74	0.14424	104.198134627865	104.198134627865\\
74	0.1479	110.323335203823	110.323335203823\\
74	0.15156	116.755199580632	116.755199580632\\
74	0.15522	123.49372775829	123.49372775829\\
74	0.15888	130.538919736799	130.538919736799\\
74	0.16254	137.890775516157	137.890775516157\\
74	0.1662	145.549295096366	145.549295096366\\
74	0.16986	153.514478477424	153.514478477424\\
74	0.17352	161.786325659333	161.786325659333\\
74	0.17718	170.364836642092	170.364836642092\\
74	0.18084	179.2500114257	179.2500114257\\
74	0.1845	188.441850010159	188.441850010159\\
74	0.18816	197.940352395467	197.940352395467\\
74	0.19182	207.745518581626	207.745518581626\\
74	0.19548	217.857348568634	217.857348568634\\
74	0.19914	228.275842356493	228.275842356493\\
74	0.2028	239.000999945202	239.000999945202\\
74	0.20646	250.03282133476	250.03282133476\\
74	0.21012	261.371306525169	261.371306525169\\
74	0.21378	273.016455516428	273.016455516428\\
74	0.21744	284.968268308536	284.968268308536\\
74	0.2211	297.226744901495	297.226744901495\\
74	0.22476	309.791885295303	309.791885295303\\
74	0.22842	322.663689489962	322.663689489962\\
74	0.23208	335.842157485471	335.842157485471\\
74	0.23574	349.32728928183	349.32728928183\\
74	0.2394	363.119084879038	363.119084879038\\
74	0.24306	377.217544277097	377.217544277097\\
74	0.24672	391.622667476006	391.622667476006\\
74	0.25038	406.334454475764	406.334454475764\\
74	0.25404	421.352905276373	421.352905276373\\
74	0.2577	436.678019877832	436.678019877832\\
74	0.26136	452.309798280141	452.309798280141\\
74	0.26502	468.248240483299	468.248240483299\\
74	0.26868	484.493346487308	484.493346487308\\
74	0.27234	501.045116292167	501.045116292167\\
74	0.276	517.903549897876	517.903549897876\\
};
\end{axis}

\begin{axis}[%
width=6.159cm,
height=3.097cm,
at={(0cm,8.602cm)},
scale only axis,
xmin=56,
xmax=74,
tick align=outside,
xlabel style={font=\color{white!15!black}},
xlabel={$L_{cut}$},
ymin=0.093,
ymax=0.276,
ylabel style={font=\color{white!15!black}},
ylabel={$D_{rlx}$},
zmin=-0.468524067538196,
zmax=12.0724352898114,
zlabel style={font=\color{white!15!black}},
zlabel={$u(t-3)u(t)$},
view={-140}{50},
axis background/.style={fill=white},
xmajorgrids,
ymajorgrids,
zmajorgrids
]
\addplot3[only marks, mark=*, mark options={}, mark size=1.5000pt, color=mycolor1, fill=mycolor1] table[row sep=crcr]{%
x	y	z\\
74	0.123	0.616147082677301\\
72	0.113	0.756244217181634\\
61	0.095	0.315136730386088\\
56	0.093	0.479225561787365\\
};
\addplot3[only marks, mark=*, mark options={}, mark size=1.5000pt, color=mycolor2, fill=mycolor2] table[row sep=crcr]{%
x	y	z\\
67	0.276	8.54895561081077\\
66	0.255	7.8265872893112\\
62	0.209	2.65659919911432\\
57	0.193	1.15263735041756\\
};
\addplot3[only marks, mark=*, mark options={}, mark size=1.5000pt, color=black, fill=black] table[row sep=crcr]{%
x	y	z\\
69	0.104	0.308897870052812\\
};
\addplot3[only marks, mark=*, mark options={}, mark size=1.5000pt, color=black, fill=black] table[row sep=crcr]{%
x	y	z\\
64	0.23	4.75084747148716\\
};

\addplot3[%
surf,
fill opacity=0.7, shader=interp, colormap={mymap}{[1pt] rgb(0pt)=(1,0.905882,0); rgb(1pt)=(1,0.901964,0); rgb(2pt)=(1,0.898051,0); rgb(3pt)=(1,0.894144,0); rgb(4pt)=(1,0.890243,0); rgb(5pt)=(1,0.886349,0); rgb(6pt)=(1,0.88246,0); rgb(7pt)=(1,0.878577,0); rgb(8pt)=(1,0.8747,0); rgb(9pt)=(1,0.870829,0); rgb(10pt)=(1,0.866964,0); rgb(11pt)=(1,0.863106,0); rgb(12pt)=(1,0.859253,0); rgb(13pt)=(1,0.855406,0); rgb(14pt)=(1,0.851566,0); rgb(15pt)=(1,0.847732,0); rgb(16pt)=(1,0.843903,0); rgb(17pt)=(1,0.840081,0); rgb(18pt)=(1,0.836265,0); rgb(19pt)=(1,0.832455,0); rgb(20pt)=(1,0.828652,0); rgb(21pt)=(1,0.824854,0); rgb(22pt)=(1,0.821063,0); rgb(23pt)=(1,0.817278,0); rgb(24pt)=(1,0.8135,0); rgb(25pt)=(1,0.809727,0); rgb(26pt)=(1,0.805961,0); rgb(27pt)=(1,0.8022,0); rgb(28pt)=(1,0.798445,0); rgb(29pt)=(1,0.794696,0); rgb(30pt)=(1,0.790953,0); rgb(31pt)=(1,0.787215,0); rgb(32pt)=(1,0.783484,0); rgb(33pt)=(1,0.779758,0); rgb(34pt)=(1,0.776038,0); rgb(35pt)=(1,0.772324,0); rgb(36pt)=(1,0.768615,0); rgb(37pt)=(1,0.764913,0); rgb(38pt)=(1,0.761217,0); rgb(39pt)=(1,0.757527,0); rgb(40pt)=(1,0.753843,0); rgb(41pt)=(1,0.750165,0); rgb(42pt)=(1,0.746493,0); rgb(43pt)=(1,0.742827,0); rgb(44pt)=(1,0.739167,0); rgb(45pt)=(1,0.735514,0); rgb(46pt)=(1,0.731867,0); rgb(47pt)=(1,0.728226,0); rgb(48pt)=(1,0.724591,0); rgb(49pt)=(1,0.720963,0); rgb(50pt)=(1,0.717341,0); rgb(51pt)=(1,0.713725,0); rgb(52pt)=(0.999994,0.710077,0); rgb(53pt)=(0.999974,0.706363,0); rgb(54pt)=(0.999942,0.702592,0); rgb(55pt)=(0.999898,0.698775,0); rgb(56pt)=(0.999841,0.694921,0); rgb(57pt)=(0.999771,0.691039,0); rgb(58pt)=(0.99969,0.687139,0); rgb(59pt)=(0.999596,0.68323,0); rgb(60pt)=(0.99949,0.679323,0); rgb(61pt)=(0.999372,0.675427,0); rgb(62pt)=(0.999242,0.67155,0); rgb(63pt)=(0.9991,0.667704,0); rgb(64pt)=(0.998946,0.663897,0); rgb(65pt)=(0.998781,0.660138,0); rgb(66pt)=(0.998605,0.656439,0); rgb(67pt)=(0.998416,0.652807,0); rgb(68pt)=(0.998217,0.649253,0); rgb(69pt)=(0.998006,0.645786,0); rgb(70pt)=(0.997785,0.642416,0); rgb(71pt)=(0.997552,0.639152,0); rgb(72pt)=(0.997308,0.636004,0); rgb(73pt)=(0.997053,0.632982,0); rgb(74pt)=(0.996788,0.630095,0); rgb(75pt)=(0.996512,0.627352,0); rgb(76pt)=(0.996226,0.624763,0); rgb(77pt)=(0.995851,0.622329,0); rgb(78pt)=(0.99494,0.619997,0); rgb(79pt)=(0.99345,0.617753,0); rgb(80pt)=(0.991419,0.61559,0); rgb(81pt)=(0.988885,0.613503,0); rgb(82pt)=(0.985886,0.611486,0); rgb(83pt)=(0.98246,0.609532,0); rgb(84pt)=(0.978643,0.607636,0); rgb(85pt)=(0.974475,0.605791,0); rgb(86pt)=(0.969992,0.603992,0); rgb(87pt)=(0.965232,0.602233,0); rgb(88pt)=(0.960233,0.600507,0); rgb(89pt)=(0.955033,0.598808,0); rgb(90pt)=(0.949669,0.59713,0); rgb(91pt)=(0.94418,0.595468,0); rgb(92pt)=(0.938602,0.593815,0); rgb(93pt)=(0.932974,0.592166,0); rgb(94pt)=(0.927333,0.590513,0); rgb(95pt)=(0.921717,0.588852,0); rgb(96pt)=(0.916164,0.587176,0); rgb(97pt)=(0.910711,0.585479,0); rgb(98pt)=(0.905397,0.583755,0); rgb(99pt)=(0.900258,0.581999,0); rgb(100pt)=(0.895333,0.580203,0); rgb(101pt)=(0.890659,0.578362,0); rgb(102pt)=(0.886275,0.576471,0); rgb(103pt)=(0.882047,0.574545,0); rgb(104pt)=(0.877819,0.572608,0); rgb(105pt)=(0.873592,0.57066,0); rgb(106pt)=(0.869366,0.568701,0); rgb(107pt)=(0.865143,0.566733,0); rgb(108pt)=(0.860924,0.564756,0); rgb(109pt)=(0.856708,0.562771,0); rgb(110pt)=(0.852497,0.560778,0); rgb(111pt)=(0.848292,0.558779,0); rgb(112pt)=(0.844092,0.556774,0); rgb(113pt)=(0.8399,0.554763,0); rgb(114pt)=(0.835716,0.552749,0); rgb(115pt)=(0.831541,0.55073,0); rgb(116pt)=(0.827374,0.548709,0); rgb(117pt)=(0.823219,0.546686,0); rgb(118pt)=(0.819074,0.54466,0); rgb(119pt)=(0.81494,0.542635,0); rgb(120pt)=(0.81082,0.540609,0); rgb(121pt)=(0.806712,0.538584,0); rgb(122pt)=(0.802619,0.53656,0); rgb(123pt)=(0.798541,0.534539,0); rgb(124pt)=(0.794478,0.532521,0); rgb(125pt)=(0.790431,0.530506,0); rgb(126pt)=(0.786402,0.528496,0); rgb(127pt)=(0.782391,0.526491,0); rgb(128pt)=(0.77841,0.524489,0); rgb(129pt)=(0.774523,0.522478,0); rgb(130pt)=(0.770731,0.520455,0); rgb(131pt)=(0.767022,0.518424,0); rgb(132pt)=(0.763384,0.516385,0); rgb(133pt)=(0.759804,0.514339,0); rgb(134pt)=(0.756272,0.51229,0); rgb(135pt)=(0.752775,0.510237,0); rgb(136pt)=(0.749302,0.508182,0); rgb(137pt)=(0.74584,0.506128,0); rgb(138pt)=(0.742378,0.504075,0); rgb(139pt)=(0.738904,0.502025,0); rgb(140pt)=(0.735406,0.499979,0); rgb(141pt)=(0.731872,0.49794,0); rgb(142pt)=(0.72829,0.495909,0); rgb(143pt)=(0.724649,0.493887,0); rgb(144pt)=(0.720936,0.491875,0); rgb(145pt)=(0.71714,0.489876,0); rgb(146pt)=(0.713249,0.487891,0); rgb(147pt)=(0.709251,0.485921,0); rgb(148pt)=(0.705134,0.483968,0); rgb(149pt)=(0.700887,0.482033,0); rgb(150pt)=(0.696497,0.480118,0); rgb(151pt)=(0.691952,0.478225,0); rgb(152pt)=(0.687242,0.476355,0); rgb(153pt)=(0.682353,0.47451,0); rgb(154pt)=(0.677195,0.472696,0); rgb(155pt)=(0.6717,0.470916,0); rgb(156pt)=(0.665891,0.469169,0); rgb(157pt)=(0.659791,0.46745,0); rgb(158pt)=(0.653423,0.465756,0); rgb(159pt)=(0.64681,0.464084,0); rgb(160pt)=(0.639976,0.462432,0); rgb(161pt)=(0.632943,0.460795,0); rgb(162pt)=(0.625734,0.459171,0); rgb(163pt)=(0.618373,0.457556,0); rgb(164pt)=(0.610882,0.455948,0); rgb(165pt)=(0.603284,0.454343,0); rgb(166pt)=(0.595604,0.452737,0); rgb(167pt)=(0.587863,0.451129,0); rgb(168pt)=(0.580084,0.449514,0); rgb(169pt)=(0.572292,0.447889,0); rgb(170pt)=(0.564508,0.446252,0); rgb(171pt)=(0.556756,0.444599,0); rgb(172pt)=(0.549059,0.442927,0); rgb(173pt)=(0.54144,0.441232,0); rgb(174pt)=(0.533922,0.439512,0); rgb(175pt)=(0.526529,0.437764,0); rgb(176pt)=(0.519282,0.435983,0); rgb(177pt)=(0.512206,0.434168,0); rgb(178pt)=(0.505323,0.432315,0); rgb(179pt)=(0.498628,0.430422,3.92506e-06); rgb(180pt)=(0.491973,0.428504,3.49981e-05); rgb(181pt)=(0.485331,0.426562,9.63073e-05); rgb(182pt)=(0.478704,0.424596,0.000186979); rgb(183pt)=(0.472096,0.422609,0.000306141); rgb(184pt)=(0.465508,0.420599,0.00045292); rgb(185pt)=(0.458942,0.418567,0.000626441); rgb(186pt)=(0.452401,0.416515,0.000825833); rgb(187pt)=(0.445885,0.414441,0.00105022); rgb(188pt)=(0.439399,0.412348,0.00129873); rgb(189pt)=(0.432942,0.410234,0.00157049); rgb(190pt)=(0.426518,0.408102,0.00186463); rgb(191pt)=(0.420129,0.40595,0.00218028); rgb(192pt)=(0.413777,0.40378,0.00251655); rgb(193pt)=(0.407464,0.401592,0.00287258); rgb(194pt)=(0.401191,0.399386,0.00324749); rgb(195pt)=(0.394962,0.397164,0.00364042); rgb(196pt)=(0.388777,0.394925,0.00405048); rgb(197pt)=(0.38264,0.39267,0.00447681); rgb(198pt)=(0.376552,0.390399,0.00491852); rgb(199pt)=(0.370516,0.388113,0.00537476); rgb(200pt)=(0.364532,0.385812,0.00584464); rgb(201pt)=(0.358605,0.383497,0.00632729); rgb(202pt)=(0.352735,0.381168,0.00682184); rgb(203pt)=(0.346925,0.378826,0.00732741); rgb(204pt)=(0.341176,0.376471,0.00784314); rgb(205pt)=(0.335485,0.374093,0.00847245); rgb(206pt)=(0.329843,0.371682,0.00930909); rgb(207pt)=(0.324249,0.369242,0.0103377); rgb(208pt)=(0.318701,0.366772,0.0115428); rgb(209pt)=(0.313198,0.364275,0.0129091); rgb(210pt)=(0.307739,0.361753,0.0144211); rgb(211pt)=(0.302322,0.359206,0.0160634); rgb(212pt)=(0.296945,0.356637,0.0178207); rgb(213pt)=(0.291607,0.354048,0.0196776); rgb(214pt)=(0.286307,0.35144,0.0216186); rgb(215pt)=(0.281043,0.348814,0.0236284); rgb(216pt)=(0.275813,0.346172,0.0256916); rgb(217pt)=(0.270616,0.343517,0.0277927); rgb(218pt)=(0.265451,0.340849,0.0299163); rgb(219pt)=(0.260317,0.33817,0.0320472); rgb(220pt)=(0.25521,0.335482,0.0341698); rgb(221pt)=(0.250131,0.332786,0.0362688); rgb(222pt)=(0.245078,0.330085,0.0383287); rgb(223pt)=(0.240048,0.327379,0.0403343); rgb(224pt)=(0.235042,0.324671,0.04227); rgb(225pt)=(0.230056,0.321962,0.0441205); rgb(226pt)=(0.22509,0.319254,0.0458704); rgb(227pt)=(0.220142,0.316548,0.0475043); rgb(228pt)=(0.215212,0.313846,0.0490067); rgb(229pt)=(0.210296,0.311149,0.0503624); rgb(230pt)=(0.205395,0.308459,0.0515759); rgb(231pt)=(0.200514,0.305763,0.052757); rgb(232pt)=(0.195655,0.303061,0.0539242); rgb(233pt)=(0.190817,0.300353,0.0550763); rgb(234pt)=(0.186001,0.297639,0.0562123); rgb(235pt)=(0.181207,0.294918,0.0573313); rgb(236pt)=(0.176434,0.292191,0.0584321); rgb(237pt)=(0.171685,0.289458,0.0595136); rgb(238pt)=(0.166957,0.286719,0.060575); rgb(239pt)=(0.162252,0.283973,0.0616151); rgb(240pt)=(0.15757,0.281221,0.0626328); rgb(241pt)=(0.152911,0.278463,0.0636271); rgb(242pt)=(0.148275,0.275699,0.0645971); rgb(243pt)=(0.143663,0.272929,0.0655416); rgb(244pt)=(0.139074,0.270152,0.0664596); rgb(245pt)=(0.134508,0.26737,0.06735); rgb(246pt)=(0.129967,0.264581,0.0682118); rgb(247pt)=(0.125449,0.261787,0.0690441); rgb(248pt)=(0.120956,0.258986,0.0698456); rgb(249pt)=(0.116487,0.25618,0.0706154); rgb(250pt)=(0.112043,0.253367,0.0713525); rgb(251pt)=(0.107623,0.250549,0.0720557); rgb(252pt)=(0.103229,0.247724,0.0727241); rgb(253pt)=(0.0988592,0.244894,0.0733566); rgb(254pt)=(0.0945149,0.242058,0.0739522); rgb(255pt)=(0.0901961,0.239216,0.0745098)}, mesh/rows=49]
table[row sep=crcr, point meta=\thisrow{c}] {%
%
x	y	z	c\\
56	0.093	0.414663957322075	0.414663957322075\\
56	0.09666	0.370128901927017	0.370128901927017\\
56	0.10032	0.329913908900332	0.329913908900332\\
56	0.10398	0.294018978242019	0.294018978242019\\
56	0.10764	0.262444109952074	0.262444109952074\\
56	0.1113	0.235189304030504	0.235189304030504\\
56	0.11496	0.212254560477309	0.212254560477309\\
56	0.11862	0.193639879292478	0.193639879292478\\
56	0.12228	0.179345260476023	0.179345260476023\\
56	0.12594	0.16937070402794	0.16937070402794\\
56	0.1296	0.163716209948227	0.163716209948227\\
56	0.13326	0.162381778236887	0.162381778236887\\
56	0.13692	0.165367408893918	0.165367408893918\\
56	0.14058	0.172673101919322	0.172673101919322\\
56	0.14424	0.184298857313093	0.184298857313093\\
56	0.1479	0.200244675075242	0.200244675075242\\
56	0.15156	0.22051055520576	0.22051055520576\\
56	0.15522	0.245096497704646	0.245096497704646\\
56	0.15888	0.274002502571911	0.274002502571911\\
56	0.16254	0.307228569807537	0.307228569807537\\
56	0.1662	0.344774699411541	0.344774699411541\\
56	0.16986	0.38664089138392	0.38664089138392\\
56	0.17352	0.432827145724668	0.432827145724668\\
56	0.17718	0.483333462433786	0.483333462433786\\
56	0.18084	0.538159841511277	0.538159841511277\\
56	0.1845	0.597306282957137	0.597306282957137\\
56	0.18816	0.660772786771368	0.660772786771368\\
56	0.19182	0.728559352953976	0.728559352953976\\
56	0.19548	0.800665981504953	0.800665981504953\\
56	0.19914	0.877092672424302	0.877092672424302\\
56	0.2028	0.957839425712025	0.957839425712025\\
56	0.20646	1.04290624136812	1.04290624136812\\
56	0.21012	1.13229311939257	1.13229311939257\\
56	0.21378	1.22600005978541	1.22600005978541\\
56	0.21744	1.32402706254661	1.32402706254661\\
56	0.2211	1.4263741276762	1.4263741276762\\
56	0.22476	1.53304125517415	1.53304125517415\\
56	0.22842	1.64402844504047	1.64402844504047\\
56	0.23208	1.75933569727515	1.75933569727515\\
56	0.23574	1.87896301187822	1.87896301187822\\
56	0.2394	2.00291038884967	2.00291038884967\\
56	0.24306	2.13117782818946	2.13117782818946\\
56	0.24672	2.26376532989765	2.26376532989765\\
56	0.25038	2.40067289397421	2.40067289397421\\
56	0.25404	2.54190052041912	2.54190052041912\\
56	0.2577	2.68744820923242	2.68744820923242\\
56	0.26136	2.83731596041408	2.83731596041408\\
56	0.26502	2.99150377396412	2.99150377396412\\
56	0.26868	3.15001164988254	3.15001164988254\\
56	0.27234	3.31283958816931	3.31283958816931\\
56	0.276	3.47998758882447	3.47998758882447\\
56.375	0.093	0.420826920209134	0.420826920209134\\
56.375	0.09666	0.38024004636651	0.38024004636651\\
56.375	0.10032	0.343973234892262	0.343973234892262\\
56.375	0.10398	0.312026485786387	0.312026485786387\\
56.375	0.10764	0.284399799048876	0.284399799048876\\
56.375	0.1113	0.261093174679745	0.261093174679745\\
56.375	0.11496	0.242106612678985	0.242106612678985\\
56.375	0.11862	0.227440113046589	0.227440113046589\\
56.375	0.12228	0.217093675782572	0.217093675782572\\
56.375	0.12594	0.211067300886927	0.211067300886927\\
56.375	0.1296	0.209360988359649	0.209360988359649\\
56.375	0.13326	0.211974738200743	0.211974738200743\\
56.375	0.13692	0.218908550410212	0.218908550410212\\
56.375	0.14058	0.230162424988051	0.230162424988051\\
56.375	0.14424	0.245736361934261	0.245736361934261\\
56.375	0.1479	0.265630361248845	0.265630361248845\\
56.375	0.15156	0.289844422931797	0.289844422931797\\
56.375	0.15522	0.318378546983121	0.318378546983121\\
56.375	0.15888	0.351232733402822	0.351232733402822\\
56.375	0.16254	0.388406982190886	0.388406982190886\\
56.375	0.1662	0.429901293347328	0.429901293347328\\
56.375	0.16986	0.475715666872138	0.475715666872138\\
56.375	0.17352	0.525850102765324	0.525850102765324\\
56.375	0.17718	0.580304601026881	0.580304601026881\\
56.375	0.18084	0.639079161656809	0.639079161656809\\
56.375	0.1845	0.7021737846551	0.7021737846551\\
56.375	0.18816	0.76958847002177	0.76958847002177\\
56.375	0.19182	0.841323217756813	0.841323217756813\\
56.375	0.19548	0.917378027860228	0.917378027860228\\
56.375	0.19914	0.997752900332012	0.997752900332012\\
56.375	0.2028	1.08244783517217	1.08244783517217\\
56.375	0.20646	1.1714628323807	1.1714628323807\\
56.375	0.21012	1.2647978919576	1.2647978919576\\
56.375	0.21378	1.36245301390286	1.36245301390286\\
56.375	0.21744	1.46442819821651	1.46442819821651\\
56.375	0.2211	1.57072344489852	1.57072344489852\\
56.375	0.22476	1.68133875394891	1.68133875394891\\
56.375	0.22842	1.79627412536767	1.79627412536767\\
56.375	0.23208	1.9155295591548	1.9155295591548\\
56.375	0.23574	2.0391050553103	2.0391050553103\\
56.375	0.2394	2.16700061383417	2.16700061383417\\
56.375	0.24306	2.29921623472642	2.29921623472642\\
56.375	0.24672	2.43575191798703	2.43575191798703\\
56.375	0.25038	2.57660766361601	2.57660766361601\\
56.375	0.25404	2.72178347161338	2.72178347161338\\
56.375	0.2577	2.87127934197911	2.87127934197911\\
56.375	0.26136	3.02509527471321	3.02509527471321\\
56.375	0.26502	3.18323126981569	3.18323126981569\\
56.375	0.26868	3.34568732728653	3.34568732728653\\
56.375	0.27234	3.51246344712576	3.51246344712576\\
56.375	0.276	3.68355962933334	3.68355962933334\\
56.75	0.093	0.425944661249165	0.425944661249165\\
56.75	0.09666	0.38930596895898	0.38930596895898\\
56.75	0.10032	0.356987339037167	0.356987339037167\\
56.75	0.10398	0.328988771483726	0.328988771483726\\
56.75	0.10764	0.305310266298657	0.305310266298657\\
56.75	0.1113	0.285951823481957	0.285951823481957\\
56.75	0.11496	0.270913443033635	0.270913443033635\\
56.75	0.11862	0.260195124953677	0.260195124953677\\
56.75	0.12228	0.253796869242095	0.253796869242095\\
56.75	0.12594	0.251718675898885	0.251718675898885\\
56.75	0.1296	0.253960544924042	0.253960544924042\\
56.75	0.13326	0.260522476317578	0.260522476317578\\
56.75	0.13692	0.271404470079482	0.271404470079482\\
56.75	0.14058	0.286606526209755	0.286606526209755\\
56.75	0.14424	0.306128644708399	0.306128644708399\\
56.75	0.1479	0.329970825575422	0.329970825575422\\
56.75	0.15156	0.358133068810812	0.358133068810812\\
56.75	0.15522	0.390615374414571	0.390615374414571\\
56.75	0.15888	0.42741774238671	0.42741774238671\\
56.75	0.16254	0.468540172727208	0.468540172727208\\
56.75	0.1662	0.513982665436085	0.513982665436085\\
56.75	0.16986	0.563745220513334	0.563745220513334\\
56.75	0.17352	0.617827837958954	0.617827837958954\\
56.75	0.17718	0.676230517772949	0.676230517772949\\
56.75	0.18084	0.738953259955313	0.738953259955313\\
56.75	0.1845	0.805996064506042	0.805996064506042\\
56.75	0.18816	0.877358931425146	0.877358931425146\\
56.75	0.19182	0.953041860712627	0.953041860712627\\
56.75	0.19548	1.03304485236848	1.03304485236848\\
56.75	0.19914	1.1173679063927	1.1173679063927\\
56.75	0.2028	1.20601102278529	1.20601102278529\\
56.75	0.20646	1.29897420154626	1.29897420154626\\
56.75	0.21012	1.39625744267559	1.39625744267559\\
56.75	0.21378	1.4978607461733	1.4978607461733\\
56.75	0.21744	1.60378411203938	1.60378411203938\\
56.75	0.2211	1.71402754027383	1.71402754027383\\
56.75	0.22476	1.82859103087665	1.82859103087665\\
56.75	0.22842	1.94747458384785	1.94747458384785\\
56.75	0.23208	2.07067819918741	2.07067819918741\\
56.75	0.23574	2.19820187689535	2.19820187689535\\
56.75	0.2394	2.33004561697165	2.33004561697165\\
56.75	0.24306	2.46620941941634	2.46620941941634\\
56.75	0.24672	2.60669328422939	2.60669328422939\\
56.75	0.25038	2.75149721141082	2.75149721141082\\
56.75	0.25404	2.90062120096061	2.90062120096061\\
56.75	0.2577	3.05406525287878	3.05406525287878\\
56.75	0.26136	3.21182936716532	3.21182936716532\\
56.75	0.26502	3.37391354382022	3.37391354382022\\
56.75	0.26868	3.54031778284351	3.54031778284351\\
56.75	0.27234	3.71104208423516	3.71104208423516\\
56.75	0.276	3.88608644799519	3.88608644799519\\
57.125	0.093	0.430017180442184	0.430017180442184\\
57.125	0.09666	0.397326669704433	0.397326669704433\\
57.125	0.10032	0.368956221335056	0.368956221335056\\
57.125	0.10398	0.344905835334054	0.344905835334054\\
57.125	0.10764	0.32517551170142	0.32517551170142\\
57.125	0.1113	0.309765250437158	0.309765250437158\\
57.125	0.11496	0.29867505154127	0.29867505154127\\
57.125	0.11862	0.291904915013747	0.291904915013747\\
57.125	0.12228	0.289454840854603	0.289454840854603\\
57.125	0.12594	0.291324829063831	0.291324829063831\\
57.125	0.1296	0.297514879641423	0.297514879641423\\
57.125	0.13326	0.308024992587394	0.308024992587394\\
57.125	0.13692	0.322855167901736	0.322855167901736\\
57.125	0.14058	0.342005405584444	0.342005405584444\\
57.125	0.14424	0.365475705635526	0.365475705635526\\
57.125	0.1479	0.393266068054984	0.393266068054984\\
57.125	0.15156	0.425376492842809	0.425376492842809\\
57.125	0.15522	0.461806979999006	0.461806979999006\\
57.125	0.15888	0.502557529523579	0.502557529523579\\
57.125	0.16254	0.547628141416516	0.547628141416516\\
57.125	0.1662	0.597018815677828	0.597018815677828\\
57.125	0.16986	0.650729552307515	0.650729552307515\\
57.125	0.17352	0.708760351305573	0.708760351305573\\
57.125	0.17718	0.771111212671999	0.771111212671999\\
57.125	0.18084	0.837782136406801	0.837782136406801\\
57.125	0.1845	0.908773122509968	0.908773122509968\\
57.125	0.18816	0.984084170981511	0.984084170981511\\
57.125	0.19182	1.06371528182142	1.06371528182142\\
57.125	0.19548	1.14766645502971	1.14766645502971\\
57.125	0.19914	1.23593769060637	1.23593769060637\\
57.125	0.2028	1.3285289885514	1.3285289885514\\
57.125	0.20646	1.4254403488648	1.4254403488648\\
57.125	0.21012	1.52667177154657	1.52667177154657\\
57.125	0.21378	1.63222325659672	1.63222325659672\\
57.125	0.21744	1.74209480401523	1.74209480401523\\
57.125	0.2211	1.85628641380212	1.85628641380212\\
57.125	0.22476	1.97479808595738	1.97479808595738\\
57.125	0.22842	2.09762982048101	2.09762982048101\\
57.125	0.23208	2.224781617373	2.224781617373\\
57.125	0.23574	2.35625347663338	2.35625347663338\\
57.125	0.2394	2.49204539826213	2.49204539826213\\
57.125	0.24306	2.63215738225924	2.63215738225924\\
57.125	0.24672	2.77658942862473	2.77658942862473\\
57.125	0.25038	2.92534153735859	2.92534153735859\\
57.125	0.25404	3.07841370846083	3.07841370846083\\
57.125	0.2577	3.23580594193143	3.23580594193143\\
57.125	0.26136	3.3975182377704	3.3975182377704\\
57.125	0.26502	3.56355059597775	3.56355059597775\\
57.125	0.26868	3.73390301655347	3.73390301655347\\
57.125	0.27234	3.90857549949756	3.90857549949756\\
57.125	0.276	4.08756804481003	4.08756804481003\\
57.5	0.093	0.433044477788178	0.433044477788178\\
57.5	0.09666	0.404302148602864	0.404302148602864\\
57.5	0.10032	0.379879881785923	0.379879881785923\\
57.5	0.10398	0.359777677337356	0.359777677337356\\
57.5	0.10764	0.34399553525716	0.34399553525716\\
57.5	0.1113	0.332533455545333	0.332533455545333\\
57.5	0.11496	0.325391438201883	0.325391438201883\\
57.5	0.11862	0.322569483226799	0.322569483226799\\
57.5	0.12228	0.32406759062009	0.32406759062009\\
57.5	0.12594	0.329885760381752	0.329885760381752\\
57.5	0.1296	0.340023992511782	0.340023992511782\\
57.5	0.13326	0.354482287010188	0.354482287010188\\
57.5	0.13692	0.373260643876964	0.373260643876964\\
57.5	0.14058	0.396359063112115	0.396359063112115\\
57.5	0.14424	0.423777544715632	0.423777544715632\\
57.5	0.1479	0.455516088687523	0.455516088687523\\
57.5	0.15156	0.491574695027787	0.491574695027787\\
57.5	0.15522	0.531953363736418	0.531953363736418\\
57.5	0.15888	0.57665209481343	0.57665209481343\\
57.5	0.16254	0.625670888258802	0.625670888258802\\
57.5	0.1662	0.679009744072552	0.679009744072552\\
57.5	0.16986	0.736668662254673	0.736668662254673\\
57.5	0.17352	0.798647642805166	0.798647642805166\\
57.5	0.17718	0.864946685724031	0.864946685724031\\
57.5	0.18084	0.935565791011271	0.935565791011271\\
57.5	0.1845	1.01050495866687	1.01050495866687\\
57.5	0.18816	1.08976418869085	1.08976418869085\\
57.5	0.19182	1.1733434810832	1.1733434810832\\
57.5	0.19548	1.26124283584393	1.26124283584393\\
57.5	0.19914	1.35346225297302	1.35346225297302\\
57.5	0.2028	1.45000173247048	1.45000173247048\\
57.5	0.20646	1.55086127433633	1.55086127433633\\
57.5	0.21012	1.65604087857053	1.65604087857053\\
57.5	0.21378	1.76554054517311	1.76554054517311\\
57.5	0.21744	1.87936027414406	1.87936027414406\\
57.5	0.2211	1.99750006548339	1.99750006548339\\
57.5	0.22476	2.11995991919108	2.11995991919108\\
57.5	0.22842	2.24673983526715	2.24673983526715\\
57.5	0.23208	2.37783981371159	2.37783981371159\\
57.5	0.23574	2.5132598545244	2.5132598545244\\
57.5	0.2394	2.65299995770558	2.65299995770558\\
57.5	0.24306	2.79706012325514	2.79706012325514\\
57.5	0.24672	2.94544035117305	2.94544035117305\\
57.5	0.25038	3.09814064145935	3.09814064145935\\
57.5	0.25404	3.25516099411403	3.25516099411403\\
57.5	0.2577	3.41650140913706	3.41650140913706\\
57.5	0.26136	3.58216188652847	3.58216188652847\\
57.5	0.26502	3.75214242628826	3.75214242628826\\
57.5	0.26868	3.92644302841641	3.92644302841641\\
57.5	0.27234	4.10506369291294	4.10506369291294\\
57.5	0.276	4.28800441977783	4.28800441977783\\
57.875	0.093	0.435026553287154	0.435026553287154\\
57.875	0.09666	0.410232405654276	0.410232405654276\\
57.875	0.10032	0.389758320389776	0.389758320389776\\
57.875	0.10398	0.373604297493643	0.373604297493643\\
57.875	0.10764	0.361770336965882	0.361770336965882\\
57.875	0.1113	0.354256438806493	0.354256438806493\\
57.875	0.11496	0.351062603015478	0.351062603015478\\
57.875	0.11862	0.352188829592828	0.352188829592828\\
57.875	0.12228	0.357635118538557	0.357635118538557\\
57.875	0.12594	0.367401469852655	0.367401469852655\\
57.875	0.1296	0.381487883535123	0.381487883535123\\
57.875	0.13326	0.399894359585963	0.399894359585963\\
57.875	0.13692	0.422620898005178	0.422620898005178\\
57.875	0.14058	0.449667498792763	0.449667498792763\\
57.875	0.14424	0.481034161948718	0.481034161948718\\
57.875	0.1479	0.516720887473045	0.516720887473045\\
57.875	0.15156	0.556727675365746	0.556727675365746\\
57.875	0.15522	0.601054525626813	0.601054525626813\\
57.875	0.15888	0.649701438256259	0.649701438256259\\
57.875	0.16254	0.702668413254069	0.702668413254069\\
57.875	0.1662	0.759955450620254	0.759955450620254\\
57.875	0.16986	0.821562550354813	0.821562550354813\\
57.875	0.17352	0.887489712457745	0.887489712457745\\
57.875	0.17718	0.957736936929044	0.957736936929044\\
57.875	0.18084	1.03230422376872	1.03230422376872\\
57.875	0.1845	1.11119157297676	1.11119157297676\\
57.875	0.18816	1.19439898455317	1.19439898455317\\
57.875	0.19182	1.28192645849796	1.28192645849796\\
57.875	0.19548	1.37377399481112	1.37377399481112\\
57.875	0.19914	1.46994159349265	1.46994159349265\\
57.875	0.2028	1.57042925454255	1.57042925454255\\
57.875	0.20646	1.67523697796083	1.67523697796083\\
57.875	0.21012	1.78436476374747	1.78436476374747\\
57.875	0.21378	1.89781261190249	1.89781261190249\\
57.875	0.21744	2.01558052242588	2.01558052242588\\
57.875	0.2211	2.13766849531763	2.13766849531763\\
57.875	0.22476	2.26407653057777	2.26407653057777\\
57.875	0.22842	2.39480462820627	2.39480462820627\\
57.875	0.23208	2.52985278820314	2.52985278820314\\
57.875	0.23574	2.66922101056838	2.66922101056838\\
57.875	0.2394	2.81290929530201	2.81290929530201\\
57.875	0.24306	2.960917642404	2.960917642404\\
57.875	0.24672	3.11324605187437	3.11324605187437\\
57.875	0.25038	3.2698945237131	3.2698945237131\\
57.875	0.25404	3.4308630579202	3.4308630579202\\
57.875	0.2577	3.59615165449568	3.59615165449568\\
57.875	0.26136	3.76576031343952	3.76576031343952\\
57.875	0.26502	3.93968903475173	3.93968903475173\\
57.875	0.26868	4.11793781843233	4.11793781843233\\
57.875	0.27234	4.30050666448129	4.30050666448129\\
57.875	0.276	4.48739557289864	4.48739557289864\\
58.25	0.093	0.435963406939106	0.435963406939106\\
58.25	0.09666	0.415117440858665	0.415117440858665\\
58.25	0.10032	0.398591537146599	0.398591537146599\\
58.25	0.10398	0.386385695802904	0.386385695802904\\
58.25	0.10764	0.378499916827578	0.378499916827578\\
58.25	0.1113	0.374934200220624	0.374934200220624\\
58.25	0.11496	0.375688545982047	0.375688545982047\\
58.25	0.11862	0.380762954111836	0.380762954111836\\
58.25	0.12228	0.390157424609999	0.390157424609999\\
58.25	0.12594	0.403871957476535	0.403871957476535\\
58.25	0.1296	0.421906552711438	0.421906552711438\\
58.25	0.13326	0.444261210314716	0.444261210314716\\
58.25	0.13692	0.470935930286366	0.470935930286366\\
58.25	0.14058	0.501930712626386	0.501930712626386\\
58.25	0.14424	0.537245557334776	0.537245557334776\\
58.25	0.1479	0.57688046441154	0.57688046441154\\
58.25	0.15156	0.620835433856677	0.620835433856677\\
58.25	0.15522	0.669110465670181	0.669110465670181\\
58.25	0.15888	0.721705559852066	0.721705559852066\\
58.25	0.16254	0.778620716402311	0.778620716402311\\
58.25	0.1662	0.839855935320934	0.839855935320934\\
58.25	0.16986	0.905411216607928	0.905411216607928\\
58.25	0.17352	0.975286560263294	0.975286560263294\\
58.25	0.17718	1.04948196628703	1.04948196628703\\
58.25	0.18084	1.12799743467914	1.12799743467914\\
58.25	0.1845	1.21083296543962	1.21083296543962\\
58.25	0.18816	1.29798855856847	1.29798855856847\\
58.25	0.19182	1.38946421406569	1.38946421406569\\
58.25	0.19548	1.48525993193129	1.48525993193129\\
58.25	0.19914	1.58537571216525	1.58537571216525\\
58.25	0.2028	1.68981155476759	1.68981155476759\\
58.25	0.20646	1.79856745973831	1.79856745973831\\
58.25	0.21012	1.91164342707738	1.91164342707738\\
58.25	0.21378	2.02903945678483	2.02903945678483\\
58.25	0.21744	2.15075554886066	2.15075554886066\\
58.25	0.2211	2.27679170330486	2.27679170330486\\
58.25	0.22476	2.40714792011742	2.40714792011742\\
58.25	0.22842	2.54182419929837	2.54182419929837\\
58.25	0.23208	2.68082054084767	2.68082054084767\\
58.25	0.23574	2.82413694476536	2.82413694476536\\
58.25	0.2394	2.97177341105141	2.97177341105141\\
58.25	0.24306	3.12372993970584	3.12372993970584\\
58.25	0.24672	3.28000653072863	3.28000653072863\\
58.25	0.25038	3.4406031841198	3.4406031841198\\
58.25	0.25404	3.60551989987935	3.60551989987935\\
58.25	0.2577	3.77475667800726	3.77475667800726\\
58.25	0.26136	3.94831351850354	3.94831351850354\\
58.25	0.26502	4.1261904213682	4.1261904213682\\
58.25	0.26868	4.30838738660123	4.30838738660123\\
58.25	0.27234	4.49490441420263	4.49490441420263\\
58.25	0.276	4.6857415041724	4.6857415041724\\
58.625	0.093	0.435855038744041	0.435855038744041\\
58.625	0.09666	0.418957254216036	0.418957254216036\\
58.625	0.10032	0.406379532056405	0.406379532056405\\
58.625	0.10398	0.398121872265145	0.398121872265145\\
58.625	0.10764	0.394184274842257	0.394184274842257\\
58.625	0.1113	0.394566739787741	0.394566739787741\\
58.625	0.11496	0.3992692671016	0.3992692671016\\
58.625	0.11862	0.408291856783823	0.408291856783823\\
58.625	0.12228	0.421634508834425	0.421634508834425\\
58.625	0.12594	0.439297223253395	0.439297223253395\\
58.625	0.1296	0.461280000040736	0.461280000040736\\
58.625	0.13326	0.487582839196449	0.487582839196449\\
58.625	0.13692	0.518205740720534	0.518205740720534\\
58.625	0.14058	0.553148704612992	0.553148704612992\\
58.625	0.14424	0.59241173087382	0.59241173087382\\
58.625	0.1479	0.635994819503019	0.635994819503019\\
58.625	0.15156	0.683897970500594	0.683897970500594\\
58.625	0.15522	0.736121183866533	0.736121183866533\\
58.625	0.15888	0.792664459600853	0.792664459600853\\
58.625	0.16254	0.853527797703536	0.853527797703536\\
58.625	0.1662	0.918711198174593	0.918711198174593\\
58.625	0.16986	0.988214661014026	0.988214661014026\\
58.625	0.17352	1.06203818622183	1.06203818622183\\
58.625	0.17718	1.140181773798	1.140181773798\\
58.625	0.18084	1.22264542374255	1.22264542374255\\
58.625	0.1845	1.30942913605546	1.30942913605546\\
58.625	0.18816	1.40053291073674	1.40053291073674\\
58.625	0.19182	1.49595674778641	1.49595674778641\\
58.625	0.19548	1.59570064720444	1.59570064720444\\
58.625	0.19914	1.69976460899084	1.69976460899084\\
58.625	0.2028	1.80814863314562	1.80814863314562\\
58.625	0.20646	1.92085271966877	1.92085271966877\\
58.625	0.21012	2.03787686856028	2.03787686856028\\
58.625	0.21378	2.15922107982017	2.15922107982017\\
58.625	0.21744	2.28488535344843	2.28488535344843\\
58.625	0.2211	2.41486968944507	2.41486968944507\\
58.625	0.22476	2.54917408781007	2.54917408781007\\
58.625	0.22842	2.68779854854345	2.68779854854345\\
58.625	0.23208	2.83074307164518	2.83074307164518\\
58.625	0.23574	2.97800765711531	2.97800765711531\\
58.625	0.2394	3.1295923049538	3.1295923049538\\
58.625	0.24306	3.28549701516066	3.28549701516066\\
58.625	0.24672	3.4457217877359	3.4457217877359\\
58.625	0.25038	3.6102666226795	3.6102666226795\\
58.625	0.25404	3.77913151999148	3.77913151999148\\
58.625	0.2577	3.95231647967183	3.95231647967183\\
58.625	0.26136	4.12982150172055	4.12982150172055\\
58.625	0.26502	4.31164658613764	4.31164658613764\\
58.625	0.26868	4.4977917329231	4.4977917329231\\
58.625	0.27234	4.68825694207694	4.68825694207694\\
58.625	0.276	4.88304221359915	4.88304221359915\\
59	0.093	0.434701448701953	0.434701448701953\\
59	0.09666	0.421751845726383	0.421751845726383\\
59	0.10032	0.41312230511919	0.41312230511919\\
59	0.10398	0.408812826880368	0.408812826880368\\
59	0.10764	0.408823411009915	0.408823411009915\\
59	0.1113	0.413154057507837	0.413154057507837\\
59	0.11496	0.42180476637413	0.42180476637413\\
59	0.11862	0.434775537608788	0.434775537608788\\
59	0.12228	0.452066371211825	0.452066371211825\\
59	0.12594	0.473677267183233	0.473677267183233\\
59	0.1296	0.499608225523009	0.499608225523009\\
59	0.13326	0.52985924623116	0.52985924623116\\
59	0.13692	0.564430329307683	0.564430329307683\\
59	0.14058	0.603321474752576	0.603321474752576\\
59	0.14424	0.646532682565839	0.646532682565839\\
59	0.1479	0.694063952747476	0.694063952747476\\
59	0.15156	0.745915285297486	0.745915285297486\\
59	0.15522	0.802086680215863	0.802086680215863\\
59	0.15888	0.862578137502621	0.862578137502621\\
59	0.16254	0.927389657157735	0.927389657157735\\
59	0.1662	0.996521239181231	0.996521239181231\\
59	0.16986	1.0699728835731	1.0699728835731\\
59	0.17352	1.14774459033334	1.14774459033334\\
59	0.17718	1.22983635946195	1.22983635946195\\
59	0.18084	1.31624819095893	1.31624819095893\\
59	0.1845	1.40698008482428	1.40698008482428\\
59	0.18816	1.502032041058	1.502032041058\\
59	0.19182	1.6014040596601	1.6014040596601\\
59	0.19548	1.70509614063057	1.70509614063057\\
59	0.19914	1.81310828396941	1.81310828396941\\
59	0.2028	1.92544048967662	1.92544048967662\\
59	0.20646	2.0420927577522	2.0420927577522\\
59	0.21012	2.16306508819615	2.16306508819615\\
59	0.21378	2.28835748100848	2.28835748100848\\
59	0.21744	2.41796993618918	2.41796993618918\\
59	0.2211	2.55190245373825	2.55190245373825\\
59	0.22476	2.69015503365569	2.69015503365569\\
59	0.22842	2.8327276759415	2.8327276759415\\
59	0.23208	2.97962038059567	2.97962038059567\\
59	0.23574	3.13083314761823	3.13083314761823\\
59	0.2394	3.28636597700917	3.28636597700917\\
59	0.24306	3.44621886876846	3.44621886876846\\
59	0.24672	3.61039182289614	3.61039182289614\\
59	0.25038	3.77888483939218	3.77888483939218\\
59	0.25404	3.95169791825659	3.95169791825659\\
59	0.2577	4.12883105948938	4.12883105948938\\
59	0.26136	4.31028426309053	4.31028426309053\\
59	0.26502	4.49605752906005	4.49605752906005\\
59	0.26868	4.68615085739796	4.68615085739796\\
59	0.27234	4.88056424810423	4.88056424810423\\
59	0.276	5.07929770117888	5.07929770117888\\
59.375	0.093	0.432502636812848	0.432502636812848\\
59.375	0.09666	0.423501215389714	0.423501215389714\\
59.375	0.10032	0.418819856334956	0.418819856334956\\
59.375	0.10398	0.418458559648572	0.418458559648572\\
59.375	0.10764	0.422417325330554	0.422417325330554\\
59.375	0.1113	0.430696153380911	0.430696153380911\\
59.375	0.11496	0.443295043799642	0.443295043799642\\
59.375	0.11862	0.460213996586738	0.460213996586738\\
59.375	0.12228	0.48145301174221	0.48145301174221\\
59.375	0.12594	0.507012089266056	0.507012089266056\\
59.375	0.1296	0.536891229158267	0.536891229158267\\
59.375	0.13326	0.571090431418857	0.571090431418857\\
59.375	0.13692	0.609609696047814	0.609609696047814\\
59.375	0.14058	0.652449023045145	0.652449023045145\\
59.375	0.14424	0.699608412410842	0.699608412410842\\
59.375	0.1479	0.751087864144915	0.751087864144915\\
59.375	0.15156	0.806887378247362	0.806887378247362\\
59.375	0.15522	0.867006954718175	0.867006954718175\\
59.375	0.15888	0.931446593557367	0.931446593557367\\
59.375	0.16254	1.00020629476492	1.00020629476492\\
59.375	0.1662	1.07328605834085	1.07328605834085\\
59.375	0.16986	1.15068588428516	1.15068588428516\\
59.375	0.17352	1.23240577259783	1.23240577259783\\
59.375	0.17718	1.31844572327888	1.31844572327888\\
59.375	0.18084	1.4088057363283	1.4088057363283\\
59.375	0.1845	1.50348581174608	1.50348581174608\\
59.375	0.18816	1.60248594953224	1.60248594953224\\
59.375	0.19182	1.70580614968677	1.70580614968677\\
59.375	0.19548	1.81344641220968	1.81344641220968\\
59.375	0.19914	1.92540673710096	1.92540673710096\\
59.375	0.2028	2.0416871243606	2.0416871243606\\
59.375	0.20646	2.16228757398862	2.16228757398862\\
59.375	0.21012	2.28720808598501	2.28720808598501\\
59.375	0.21378	2.41644866034977	2.41644866034977\\
59.375	0.21744	2.55000929708291	2.55000929708291\\
59.375	0.2211	2.68788999618441	2.68788999618441\\
59.375	0.22476	2.83009075765429	2.83009075765429\\
59.375	0.22842	2.97661158149254	2.97661158149254\\
59.375	0.23208	3.12745246769916	3.12745246769916\\
59.375	0.23574	3.28261341627415	3.28261341627415\\
59.375	0.2394	3.44209442721751	3.44209442721751\\
59.375	0.24306	3.60589550052925	3.60589550052925\\
59.375	0.24672	3.77401663620935	3.77401663620935\\
59.375	0.25038	3.94645783425783	3.94645783425783\\
59.375	0.25404	4.12321909467468	4.12321909467468\\
59.375	0.2577	4.3043004174599	4.3043004174599\\
59.375	0.26136	4.48970180261349	4.48970180261349\\
59.375	0.26502	4.67942325013546	4.67942325013546\\
59.375	0.26868	4.87346476002579	4.87346476002579\\
59.375	0.27234	5.07182633228451	5.07182633228451\\
59.375	0.276	5.27450796691159	5.27450796691159\\
59.75	0.093	0.429258603076717	0.429258603076717\\
59.75	0.09666	0.424205363206021	0.424205363206021\\
59.75	0.10032	0.423472185703698	0.423472185703698\\
59.75	0.10398	0.427059070569749	0.427059070569749\\
59.75	0.10764	0.434966017804169	0.434966017804169\\
59.75	0.1113	0.447193027406964	0.447193027406964\\
59.75	0.11496	0.46374009937813	0.46374009937813\\
59.75	0.11862	0.484607233717661	0.484607233717661\\
59.75	0.12228	0.509794430425571	0.509794430425571\\
59.75	0.12594	0.539301689501852	0.539301689501852\\
59.75	0.1296	0.573129010946501	0.573129010946501\\
59.75	0.13326	0.611276394759525	0.611276394759525\\
59.75	0.13692	0.653743840940921	0.653743840940921\\
59.75	0.14058	0.700531349490686	0.700531349490686\\
59.75	0.14424	0.751638920408822	0.751638920408822\\
59.75	0.1479	0.807066553695333	0.807066553695333\\
59.75	0.15156	0.866814249350212	0.866814249350212\\
59.75	0.15522	0.930882007373462	0.930882007373462\\
59.75	0.15888	0.999269827765089	0.999269827765089\\
59.75	0.16254	1.07197771052508	1.07197771052508\\
59.75	0.1662	1.14900565565345	1.14900565565345\\
59.75	0.16986	1.23035366315019	1.23035366315019\\
59.75	0.17352	1.3160217330153	1.3160217330153\\
59.75	0.17718	1.40600986524878	1.40600986524878\\
59.75	0.18084	1.50031805985064	1.50031805985064\\
59.75	0.1845	1.59894631682086	1.59894631682086\\
59.75	0.18816	1.70189463615946	1.70189463615946\\
59.75	0.19182	1.80916301786642	1.80916301786642\\
59.75	0.19548	1.92075146194177	1.92075146194177\\
59.75	0.19914	2.03665996838548	2.03665996838548\\
59.75	0.2028	2.15688853719756	2.15688853719756\\
59.75	0.20646	2.28143716837802	2.28143716837802\\
59.75	0.21012	2.41030586192684	2.41030586192684\\
59.75	0.21378	2.54349461784404	2.54349461784404\\
59.75	0.21744	2.68100343612961	2.68100343612961\\
59.75	0.2211	2.82283231678355	2.82283231678355\\
59.75	0.22476	2.96898125980587	2.96898125980587\\
59.75	0.22842	3.11945026519655	3.11945026519655\\
59.75	0.23208	3.27423933295561	3.27423933295561\\
59.75	0.23574	3.43334846308304	3.43334846308304\\
59.75	0.2394	3.59677765557883	3.59677765557883\\
59.75	0.24306	3.76452691044301	3.76452691044301\\
59.75	0.24672	3.93659622767555	3.93659622767555\\
59.75	0.25038	4.11298560727646	4.11298560727646\\
59.75	0.25404	4.29369504924575	4.29369504924575\\
59.75	0.2577	4.47872455358341	4.47872455358341\\
59.75	0.26136	4.66807412028943	4.66807412028943\\
59.75	0.26502	4.86174374936384	4.86174374936384\\
59.75	0.26868	5.05973344080661	5.05973344080661\\
59.75	0.27234	5.26204319461776	5.26204319461776\\
59.75	0.276	5.46867301079727	5.46867301079727\\
60.125	0.093	0.424969347493573	0.424969347493573\\
60.125	0.09666	0.423864289175312	0.423864289175312\\
60.125	0.10032	0.427079293225427	0.427079293225427\\
60.125	0.10398	0.434614359643916	0.434614359643916\\
60.125	0.10764	0.446469488430771	0.446469488430771\\
60.125	0.1113	0.462644679586001	0.462644679586001\\
60.125	0.11496	0.483139933109602	0.483139933109602\\
60.125	0.11862	0.507955249001574	0.507955249001574\\
60.125	0.12228	0.537090627261918	0.537090627261918\\
60.125	0.12594	0.570546067890635	0.570546067890635\\
60.125	0.1296	0.608321570887722	0.608321570887722\\
60.125	0.13326	0.650417136253181	0.650417136253181\\
60.125	0.13692	0.696832763987011	0.696832763987011\\
60.125	0.14058	0.747568454089215	0.747568454089215\\
60.125	0.14424	0.802624206559786	0.802624206559786\\
60.125	0.1479	0.862000021398734	0.862000021398734\\
60.125	0.15156	0.925695898606052	0.925695898606052\\
60.125	0.15522	0.993711838181737	0.993711838181737\\
60.125	0.15888	1.0660478401258	1.0660478401258\\
60.125	0.16254	1.14270390443823	1.14270390443823\\
60.125	0.1662	1.22368003111903	1.22368003111903\\
60.125	0.16986	1.30897622016821	1.30897622016821\\
60.125	0.17352	1.39859247158576	1.39859247158576\\
60.125	0.17718	1.49252878537168	1.49252878537168\\
60.125	0.18084	1.59078516152597	1.59078516152597\\
60.125	0.1845	1.69336160004862	1.69336160004862\\
60.125	0.18816	1.80025810093966	1.80025810093966\\
60.125	0.19182	1.91147466419907	1.91147466419907\\
60.125	0.19548	2.02701128982684	2.02701128982684\\
60.125	0.19914	2.14686797782299	2.14686797782299\\
60.125	0.2028	2.27104472818751	2.27104472818751\\
60.125	0.20646	2.3995415409204	2.3995415409204\\
60.125	0.21012	2.53235841602166	2.53235841602166\\
60.125	0.21378	2.66949535349129	2.66949535349129\\
60.125	0.21744	2.81095235332931	2.81095235332931\\
60.125	0.2211	2.95672941553569	2.95672941553569\\
60.125	0.22476	3.10682654011043	3.10682654011043\\
60.125	0.22842	3.26124372705356	3.26124372705356\\
60.125	0.23208	3.41998097636504	3.41998097636504\\
60.125	0.23574	3.58303828804491	3.58303828804491\\
60.125	0.2394	3.75041566209315	3.75041566209315\\
60.125	0.24306	3.92211309850975	3.92211309850975\\
60.125	0.24672	4.09813059729474	4.09813059729474\\
60.125	0.25038	4.27846815844808	4.27846815844808\\
60.125	0.25404	4.46312578196981	4.46312578196981\\
60.125	0.2577	4.65210346785991	4.65210346785991\\
60.125	0.26136	4.84540121611837	4.84540121611837\\
60.125	0.26502	5.0430190267452	5.0430190267452\\
60.125	0.26868	5.24495689974041	5.24495689974041\\
60.125	0.27234	5.45121483510399	5.45121483510399\\
60.125	0.276	5.66179283283595	5.66179283283595\\
60.5	0.093	0.419634870063401	0.419634870063401\\
60.5	0.09666	0.422477993297579	0.422477993297579\\
60.5	0.10032	0.429641178900132	0.429641178900132\\
60.5	0.10398	0.441124426871053	0.441124426871053\\
60.5	0.10764	0.456927737210346	0.456927737210346\\
60.5	0.1113	0.477051109918013	0.477051109918013\\
60.5	0.11496	0.501494544994049	0.501494544994049\\
60.5	0.11862	0.530258042438456	0.530258042438456\\
60.5	0.12228	0.563341602251239	0.563341602251239\\
60.5	0.12594	0.600745224432393	0.600745224432393\\
60.5	0.1296	0.642468908981915	0.642468908981915\\
60.5	0.13326	0.688512655899809	0.688512655899809\\
60.5	0.13692	0.738876465186078	0.738876465186078\\
60.5	0.14058	0.793560336840716	0.793560336840716\\
60.5	0.14424	0.852564270863725	0.852564270863725\\
60.5	0.1479	0.915888267255109	0.915888267255109\\
60.5	0.15156	0.983532326014864	0.983532326014864\\
60.5	0.15522	1.05549644714298	1.05549644714298\\
60.5	0.15888	1.13178063063949	1.13178063063949\\
60.5	0.16254	1.21238487650435	1.21238487650435\\
60.5	0.1662	1.29730918473759	1.29730918473759\\
60.5	0.16986	1.3865535553392	1.3865535553392\\
60.5	0.17352	1.48011798830919	1.48011798830919\\
60.5	0.17718	1.57800248364754	1.57800248364754\\
60.5	0.18084	1.68020704135427	1.68020704135427\\
60.5	0.1845	1.78673166142936	1.78673166142936\\
60.5	0.18816	1.89757634387284	1.89757634387284\\
60.5	0.19182	2.01274108868468	2.01274108868468\\
60.5	0.19548	2.13222589586489	2.13222589586489\\
60.5	0.19914	2.25603076541347	2.25603076541347\\
60.5	0.2028	2.38415569733043	2.38415569733043\\
60.5	0.20646	2.51660069161576	2.51660069161576\\
60.5	0.21012	2.65336574826946	2.65336574826946\\
60.5	0.21378	2.79445086729153	2.79445086729153\\
60.5	0.21744	2.93985604868197	2.93985604868197\\
60.5	0.2211	3.08958129244078	3.08958129244078\\
60.5	0.22476	3.24362659856797	3.24362659856797\\
60.5	0.22842	3.40199196706353	3.40199196706353\\
60.5	0.23208	3.56467739792745	3.56467739792745\\
60.5	0.23574	3.73168289115976	3.73168289115976\\
60.5	0.2394	3.90300844676042	3.90300844676042\\
60.5	0.24306	4.07865406472947	4.07865406472947\\
60.5	0.24672	4.25861974506688	4.25861974506688\\
60.5	0.25038	4.44290548777268	4.44290548777268\\
60.5	0.25404	4.63151129284684	4.63151129284684\\
60.5	0.2577	4.82443716028937	4.82443716028937\\
60.5	0.26136	5.02168309010027	5.02168309010027\\
60.5	0.26502	5.22324908227954	5.22324908227954\\
60.5	0.26868	5.42913513682718	5.42913513682718\\
60.5	0.27234	5.63934125374321	5.63934125374321\\
60.5	0.276	5.85386743302759	5.85386743302759\\
60.875	0.093	0.413255170786217	0.413255170786217\\
60.875	0.09666	0.420046475572829	0.420046475572829\\
60.875	0.10032	0.431157842727817	0.431157842727817\\
60.875	0.10398	0.446589272251176	0.446589272251176\\
60.875	0.10764	0.466340764142903	0.466340764142903\\
60.875	0.1113	0.490412318403006	0.490412318403006\\
60.875	0.11496	0.51880393503148	0.51880393503148\\
60.875	0.11862	0.551515614028325	0.551515614028325\\
60.875	0.12228	0.588547355393543	0.588547355393543\\
60.875	0.12594	0.629899159127132	0.629899159127132\\
60.875	0.1296	0.675571025229088	0.675571025229088\\
60.875	0.13326	0.725562953699424	0.725562953699424\\
60.875	0.13692	0.779874944538127	0.779874944538127\\
60.875	0.14058	0.838506997745204	0.838506997745204\\
60.875	0.14424	0.901459113320648	0.901459113320648\\
60.875	0.1479	0.968731291264466	0.968731291264466\\
60.875	0.15156	1.04032353157666	1.04032353157666\\
60.875	0.15522	1.11623583425721	1.11623583425721\\
60.875	0.15888	1.19646819930615	1.19646819930615\\
60.875	0.16254	1.28102062672345	1.28102062672345\\
60.875	0.1662	1.36989311650913	1.36989311650913\\
60.875	0.16986	1.46308566866318	1.46308566866318\\
60.875	0.17352	1.5605982831856	1.5605982831856\\
60.875	0.17718	1.66243096007639	1.66243096007639\\
60.875	0.18084	1.76858369933555	1.76858369933555\\
60.875	0.1845	1.87905650096308	1.87905650096308\\
60.875	0.18816	1.99384936495899	1.99384936495899\\
60.875	0.19182	2.11296229132327	2.11296229132327\\
60.875	0.19548	2.23639528005592	2.23639528005592\\
60.875	0.19914	2.36414833115694	2.36414833115694\\
60.875	0.2028	2.49622144462634	2.49622144462634\\
60.875	0.20646	2.6326146204641	2.6326146204641\\
60.875	0.21012	2.77332785867023	2.77332785867023\\
60.875	0.21378	2.91836115924474	2.91836115924474\\
60.875	0.21744	3.06771452218761	3.06771452218761\\
60.875	0.2211	3.22138794749887	3.22138794749887\\
60.875	0.22476	3.37938143517849	3.37938143517849\\
60.875	0.22842	3.54169498522649	3.54169498522649\\
60.875	0.23208	3.70832859764285	3.70832859764285\\
60.875	0.23574	3.87928227242758	3.87928227242758\\
60.875	0.2394	4.0545560095807	4.0545560095807\\
60.875	0.24306	4.23414980910217	4.23414980910217\\
60.875	0.24672	4.41806367099203	4.41806367099203\\
60.875	0.25038	4.60629759525025	4.60629759525025\\
60.875	0.25404	4.79885158187685	4.79885158187685\\
60.875	0.2577	4.99572563087182	4.99572563087182\\
60.875	0.26136	5.19691974223515	5.19691974223515\\
60.875	0.26502	5.40243391596686	5.40243391596686\\
60.875	0.26868	5.61226815206694	5.61226815206694\\
60.875	0.27234	5.82642245053539	5.82642245053539\\
60.875	0.276	6.04489681137223	6.04489681137223\\
61.25	0.093	0.405830249662005	0.405830249662005\\
61.25	0.09666	0.416569736001056	0.416569736001056\\
61.25	0.10032	0.431629284708478	0.431629284708478\\
61.25	0.10398	0.451008895784275	0.451008895784275\\
61.25	0.10764	0.474708569228437	0.474708569228437\\
61.25	0.1113	0.502728305040978	0.502728305040978\\
61.25	0.11496	0.535068103221887	0.535068103221887\\
61.25	0.11862	0.571727963771167	0.571727963771167\\
61.25	0.12228	0.612707886688823	0.612707886688823\\
61.25	0.12594	0.65800787197485	0.65800787197485\\
61.25	0.1296	0.707627919629241	0.707627919629241\\
61.25	0.13326	0.761568029652012	0.761568029652012\\
61.25	0.13692	0.819828202043153	0.819828202043153\\
61.25	0.14058	0.882408436802665	0.882408436802665\\
61.25	0.14424	0.949308733930543	0.949308733930543\\
61.25	0.1479	1.0205290934268	1.0205290934268\\
61.25	0.15156	1.09606951529143	1.09606951529143\\
61.25	0.15522	1.17592999952442	1.17592999952442\\
61.25	0.15888	1.2601105461258	1.2601105461258\\
61.25	0.16254	1.34861115509553	1.34861115509553\\
61.25	0.1662	1.44143182643364	1.44143182643364\\
61.25	0.16986	1.53857256014013	1.53857256014013\\
61.25	0.17352	1.64003335621499	1.64003335621499\\
61.25	0.17718	1.74581421465821	1.74581421465821\\
61.25	0.18084	1.85591513546982	1.85591513546982\\
61.25	0.1845	1.97033611864978	1.97033611864978\\
61.25	0.18816	2.08907716419812	2.08907716419812\\
61.25	0.19182	2.21213827211484	2.21213827211484\\
61.25	0.19548	2.33951944239992	2.33951944239992\\
61.25	0.19914	2.47122067505338	2.47122067505338\\
61.25	0.2028	2.60724197007521	2.60724197007521\\
61.25	0.20646	2.74758332746541	2.74758332746541\\
61.25	0.21012	2.89224474722398	2.89224474722398\\
61.25	0.21378	3.04122622935093	3.04122622935093\\
61.25	0.21744	3.19452777384624	3.19452777384624\\
61.25	0.2211	3.35214938070992	3.35214938070992\\
61.25	0.22476	3.51409104994199	3.51409104994199\\
61.25	0.22842	3.68035278154242	3.68035278154242\\
61.25	0.23208	3.85093457551122	3.85093457551122\\
61.25	0.23574	4.02583643184839	4.02583643184839\\
61.25	0.2394	4.20505835055394	4.20505835055394\\
61.25	0.24306	4.38860033162785	4.38860033162785\\
61.25	0.24672	4.57646237507014	4.57646237507014\\
61.25	0.25038	4.76864448088081	4.76864448088081\\
61.25	0.25404	4.96514664905984	4.96514664905984\\
61.25	0.2577	5.16596887960724	5.16596887960724\\
61.25	0.26136	5.37111117252301	5.37111117252301\\
61.25	0.26502	5.58057352780716	5.58057352780716\\
61.25	0.26868	5.79435594545967	5.79435594545967\\
61.25	0.27234	6.01245842548057	6.01245842548057\\
61.25	0.276	6.23488096786983	6.23488096786983\\
61.625	0.093	0.397360106690776	0.397360106690776\\
61.625	0.09666	0.412047774582262	0.412047774582262\\
61.625	0.10032	0.431055504842123	0.431055504842123\\
61.625	0.10398	0.454383297470354	0.454383297470354\\
61.625	0.10764	0.482031152466958	0.482031152466958\\
61.625	0.1113	0.51399906983193	0.51399906983193\\
61.625	0.11496	0.550287049565277	0.550287049565277\\
61.625	0.11862	0.590895091666995	0.590895091666995\\
61.625	0.12228	0.635823196137086	0.635823196137086\\
61.625	0.12594	0.685071362975548	0.685071362975548\\
61.625	0.1296	0.738639592182377	0.738639592182377\\
61.625	0.13326	0.796527883757582	0.796527883757582\\
61.625	0.13692	0.858736237701159	0.858736237701159\\
61.625	0.14058	0.925264654013108	0.925264654013108\\
61.625	0.14424	0.996113132693425	0.996113132693425\\
61.625	0.1479	1.07128167374212	1.07128167374212\\
61.625	0.15156	1.15077027715918	1.15077027715918\\
61.625	0.15522	1.23457894294461	1.23457894294461\\
61.625	0.15888	1.32270767109842	1.32270767109842\\
61.625	0.16254	1.41515646162059	1.41515646162059\\
61.625	0.1662	1.51192531451114	1.51192531451114\\
61.625	0.16986	1.61301422977006	1.61301422977006\\
61.625	0.17352	1.71842320739736	1.71842320739736\\
61.625	0.17718	1.82815224739302	1.82815224739302\\
61.625	0.18084	1.94220134975706	1.94220134975706\\
61.625	0.1845	2.06057051448946	2.06057051448946\\
61.625	0.18816	2.18325974159024	2.18325974159024\\
61.625	0.19182	2.31026903105939	2.31026903105939\\
61.625	0.19548	2.44159838289691	2.44159838289691\\
61.625	0.19914	2.57724779710281	2.57724779710281\\
61.625	0.2028	2.71721727367708	2.71721727367708\\
61.625	0.20646	2.86150681261971	2.86150681261971\\
61.625	0.21012	3.01011641393072	3.01011641393072\\
61.625	0.21378	3.1630460776101	3.1630460776101\\
61.625	0.21744	3.32029580365785	3.32029580365785\\
61.625	0.2211	3.48186559207397	3.48186559207397\\
61.625	0.22476	3.64775544285847	3.64775544285847\\
61.625	0.22842	3.81796535601134	3.81796535601134\\
61.625	0.23208	3.99249533153256	3.99249533153256\\
61.625	0.23574	4.17134536942218	4.17134536942218\\
61.625	0.2394	4.35451546968016	4.35451546968016\\
61.625	0.24306	4.54200563230651	4.54200563230651\\
61.625	0.24672	4.73381585730124	4.73381585730124\\
61.625	0.25038	4.92994614466434	4.92994614466434\\
61.625	0.25404	5.13039649439581	5.13039649439581\\
61.625	0.2577	5.33516690649565	5.33516690649565\\
61.625	0.26136	5.54425738096385	5.54425738096385\\
61.625	0.26502	5.75766791780043	5.75766791780043\\
61.625	0.26868	5.97539851700539	5.97539851700539\\
61.625	0.27234	6.19744917857871	6.19744917857871\\
61.625	0.276	6.42381990252042	6.42381990252042\\
62	0.093	0.387844741872531	0.387844741872531\\
62	0.09666	0.406480591316451	0.406480591316451\\
62	0.10032	0.429436503128747	0.429436503128747\\
62	0.10398	0.456712477309417	0.456712477309417\\
62	0.10764	0.488308513858455	0.488308513858455\\
62	0.1113	0.524224612775866	0.524224612775866\\
62	0.11496	0.564460774061647	0.564460774061647\\
62	0.11862	0.6090169977158	0.6090169977158\\
62	0.12228	0.657893283738329	0.657893283738329\\
62	0.12594	0.711089632129226	0.711089632129226\\
62	0.1296	0.768606042888493	0.768606042888493\\
62	0.13326	0.830442516016133	0.830442516016133\\
62	0.13692	0.896599051512148	0.896599051512148\\
62	0.14058	0.967075649376532	0.967075649376532\\
62	0.14424	1.04187230960929	1.04187230960929\\
62	0.1479	1.12098903221041	1.12098903221041\\
62	0.15156	1.20442581717991	1.20442581717991\\
62	0.15522	1.29218266451778	1.29218266451778\\
62	0.15888	1.38425957422403	1.38425957422403\\
62	0.16254	1.48065654629864	1.48065654629864\\
62	0.1662	1.58137358074162	1.58137358074162\\
62	0.16986	1.68641067755298	1.68641067755298\\
62	0.17352	1.79576783673271	1.79576783673271\\
62	0.17718	1.90944505828081	1.90944505828081\\
62	0.18084	2.02744234219728	2.02744234219728\\
62	0.1845	2.14975968848212	2.14975968848212\\
62	0.18816	2.27639709713534	2.27639709713534\\
62	0.19182	2.40735456815693	2.40735456815693\\
62	0.19548	2.54263210154688	2.54263210154688\\
62	0.19914	2.68222969730521	2.68222969730521\\
62	0.2028	2.82614735543192	2.82614735543192\\
62	0.20646	2.97438507592699	2.97438507592699\\
62	0.21012	3.12694285879043	3.12694285879043\\
62	0.21378	3.28382070402225	3.28382070402225\\
62	0.21744	3.44501861162244	3.44501861162244\\
62	0.2211	3.610536581591	3.610536581591\\
62	0.22476	3.78037461392792	3.78037461392792\\
62	0.22842	3.95453270863323	3.95453270863323\\
62	0.23208	4.1330108657069	4.1330108657069\\
62	0.23574	4.31580908514895	4.31580908514895\\
62	0.2394	4.50292736695936	4.50292736695936\\
62	0.24306	4.69436571113815	4.69436571113815\\
62	0.24672	4.89012411768532	4.89012411768532\\
62	0.25038	5.09020258660085	5.09020258660085\\
62	0.25404	5.29460111788475	5.29460111788475\\
62	0.2577	5.50331971153703	5.50331971153703\\
62	0.26136	5.71635836755768	5.71635836755768\\
62	0.26502	5.9337170859467	5.9337170859467\\
62	0.26868	6.15539586670408	6.15539586670408\\
62	0.27234	6.38139470982985	6.38139470982985\\
62	0.276	6.61171361532398	6.61171361532398\\
62.375	0.093	0.377284155207262	0.377284155207262\\
62.375	0.09666	0.39986818620362	0.39986818620362\\
62.375	0.10032	0.42677227956835	0.42677227956835\\
62.375	0.10398	0.457996435301455	0.457996435301455\\
62.375	0.10764	0.493540653402932	0.493540653402932\\
62.375	0.1113	0.533404933872777	0.533404933872777\\
62.375	0.11496	0.577589276710997	0.577589276710997\\
62.375	0.11862	0.626093681917585	0.626093681917585\\
62.375	0.12228	0.678918149492552	0.678918149492552\\
62.375	0.12594	0.736062679435887	0.736062679435887\\
62.375	0.1296	0.797527271747589	0.797527271747589\\
62.375	0.13326	0.863311926427667	0.863311926427667\\
62.375	0.13692	0.933416643476116	0.933416643476116\\
62.375	0.14058	1.00784142289294	1.00784142289294\\
62.375	0.14424	1.08658626467812	1.08658626467812\\
62.375	0.1479	1.16965116883169	1.16965116883169\\
62.375	0.15156	1.25703613535363	1.25703613535363\\
62.375	0.15522	1.34874116424393	1.34874116424393\\
62.375	0.15888	1.44476625550261	1.44476625550261\\
62.375	0.16254	1.54511140912966	1.54511140912966\\
62.375	0.1662	1.64977662512508	1.64977662512508\\
62.375	0.16986	1.75876190348887	1.75876190348887\\
62.375	0.17352	1.87206724422104	1.87206724422104\\
62.375	0.17718	1.98969264732158	1.98969264732158\\
62.375	0.18084	2.11163811279049	2.11163811279049\\
62.375	0.1845	2.23790364062776	2.23790364062776\\
62.375	0.18816	2.36848923083342	2.36848923083342\\
62.375	0.19182	2.50339488340744	2.50339488340744\\
62.375	0.19548	2.64262059834984	2.64262059834984\\
62.375	0.19914	2.7861663756606	2.7861663756606\\
62.375	0.2028	2.93403221533974	2.93403221533974\\
62.375	0.20646	3.08621811738725	3.08621811738725\\
62.375	0.21012	3.24272408180313	3.24272408180313\\
62.375	0.21378	3.40355010858738	3.40355010858738\\
62.375	0.21744	3.56869619774001	3.56869619774001\\
62.375	0.2211	3.738162349261	3.738162349261\\
62.375	0.22476	3.91194856315037	3.91194856315037\\
62.375	0.22842	4.09005483940811	4.09005483940811\\
62.375	0.23208	4.27248117803422	4.27248117803422\\
62.375	0.23574	4.4592275790287	4.4592275790287\\
62.375	0.2394	4.65029404239155	4.65029404239155\\
62.375	0.24306	4.84568056812278	4.84568056812278\\
62.375	0.24672	5.04538715622238	5.04538715622238\\
62.375	0.25038	5.24941380669035	5.24941380669035\\
62.375	0.25404	5.45776051952669	5.45776051952669\\
62.375	0.2577	5.6704272947314	5.6704272947314\\
62.375	0.26136	5.88741413230447	5.88741413230447\\
62.375	0.26502	6.10872103224594	6.10872103224594\\
62.375	0.26868	6.33434799455577	6.33434799455577\\
62.375	0.27234	6.56429501923397	6.56429501923397\\
62.375	0.276	6.79856210628054	6.79856210628054\\
62.75	0.093	0.365678346694972	0.365678346694972\\
62.75	0.09666	0.392210559243765	0.392210559243765\\
62.75	0.10032	0.423062834160937	0.423062834160937\\
62.75	0.10398	0.458235171446477	0.458235171446477\\
62.75	0.10764	0.497727571100389	0.497727571100389\\
62.75	0.1113	0.541540033122672	0.541540033122672\\
62.75	0.11496	0.589672557513326	0.589672557513326\\
62.75	0.11862	0.642125144272352	0.642125144272352\\
62.75	0.12228	0.69889779339975	0.69889779339975\\
62.75	0.12594	0.759990504895524	0.759990504895524\\
62.75	0.1296	0.825403278759664	0.825403278759664\\
62.75	0.13326	0.895136114992177	0.895136114992177\\
62.75	0.13692	0.969189013593065	0.969189013593065\\
62.75	0.14058	1.04756197456232	1.04756197456232\\
62.75	0.14424	1.13025499789995	1.13025499789995\\
62.75	0.1479	1.21726808360595	1.21726808360595\\
62.75	0.15156	1.30860123168032	1.30860123168032\\
62.75	0.15522	1.40425444212306	1.40425444212306\\
62.75	0.15888	1.50422771493418	1.50422771493418\\
62.75	0.16254	1.60852105011366	1.60852105011366\\
62.75	0.1662	1.71713444766152	1.71713444766152\\
62.75	0.16986	1.83006790757775	1.83006790757775\\
62.75	0.17352	1.94732142986235	1.94732142986235\\
62.75	0.17718	2.06889501451532	2.06889501451532\\
62.75	0.18084	2.19478866153667	2.19478866153667\\
62.75	0.1845	2.32500237092638	2.32500237092638\\
62.75	0.18816	2.45953614268447	2.45953614268447\\
62.75	0.19182	2.59838997681093	2.59838997681093\\
62.75	0.19548	2.74156387330576	2.74156387330576\\
62.75	0.19914	2.88905783216896	2.88905783216896\\
62.75	0.2028	3.04087185340054	3.04087185340054\\
62.75	0.20646	3.19700593700049	3.19700593700049\\
62.75	0.21012	3.3574600829688	3.3574600829688\\
62.75	0.21378	3.52223429130549	3.52223429130549\\
62.75	0.21744	3.69132856201055	3.69132856201055\\
62.75	0.2211	3.86474289508399	3.86474289508399\\
62.75	0.22476	4.04247729052579	4.04247729052579\\
62.75	0.22842	4.22453174833596	4.22453174833596\\
62.75	0.23208	4.4109062685145	4.4109062685145\\
62.75	0.23574	4.60160085106142	4.60160085106142\\
62.75	0.2394	4.79661549597672	4.79661549597672\\
62.75	0.24306	4.99595020326038	4.99595020326038\\
62.75	0.24672	5.19960497291242	5.19960497291242\\
62.75	0.25038	5.40757980493282	5.40757980493282\\
62.75	0.25404	5.61987469932159	5.61987469932159\\
62.75	0.2577	5.83648965607875	5.83648965607875\\
62.75	0.26136	6.05742467520427	6.05742467520427\\
62.75	0.26502	6.28267975669815	6.28267975669815\\
62.75	0.26868	6.51225490056043	6.51225490056043\\
62.75	0.27234	6.74615010679105	6.74615010679105\\
62.75	0.276	6.98436537539007	6.98436537539007\\
63.125	0.093	0.353027316335659	0.353027316335659\\
63.125	0.09666	0.38350771043689	0.38350771043689\\
63.125	0.10032	0.418308166906497	0.418308166906497\\
63.125	0.10398	0.457428685744475	0.457428685744475\\
63.125	0.10764	0.500869266950821	0.500869266950821\\
63.125	0.1113	0.548629910525539	0.548629910525539\\
63.125	0.11496	0.600710616468632	0.600710616468632\\
63.125	0.11862	0.657111384780093	0.657111384780093\\
63.125	0.12228	0.717832215459929	0.717832215459929\\
63.125	0.12594	0.782873108508137	0.782873108508137\\
63.125	0.1296	0.852234063924712	0.852234063924712\\
63.125	0.13326	0.925915081709663	0.925915081709663\\
63.125	0.13692	1.00391616186299	1.00391616186299\\
63.125	0.14058	1.08623730438468	1.08623730438468\\
63.125	0.14424	1.17287850927474	1.17287850927474\\
63.125	0.1479	1.26383977653318	1.26383977653318\\
63.125	0.15156	1.35912110615999	1.35912110615999\\
63.125	0.15522	1.45872249815517	1.45872249815517\\
63.125	0.15888	1.56264395251872	1.56264395251872\\
63.125	0.16254	1.67088546925064	1.67088546925064\\
63.125	0.1662	1.78344704835093	1.78344704835093\\
63.125	0.16986	1.9003286898196	1.9003286898196\\
63.125	0.17352	2.02153039365664	2.02153039365664\\
63.125	0.17718	2.14705215986205	2.14705215986205\\
63.125	0.18084	2.27689398843583	2.27689398843583\\
63.125	0.1845	2.41105587937798	2.41105587937798\\
63.125	0.18816	2.5495378326885	2.5495378326885\\
63.125	0.19182	2.6923398483674	2.6923398483674\\
63.125	0.19548	2.83946192641467	2.83946192641467\\
63.125	0.19914	2.99090406683031	2.99090406683031\\
63.125	0.2028	3.14666626961432	3.14666626961432\\
63.125	0.20646	3.3067485347667	3.3067485347667\\
63.125	0.21012	3.47115086228745	3.47115086228745\\
63.125	0.21378	3.63987325217657	3.63987325217657\\
63.125	0.21744	3.81291570443407	3.81291570443407\\
63.125	0.2211	3.99027821905994	3.99027821905994\\
63.125	0.22476	4.17196079605418	4.17196079605418\\
63.125	0.22842	4.3579634354168	4.3579634354168\\
63.125	0.23208	4.54828613714777	4.54828613714777\\
63.125	0.23574	4.74292890124713	4.74292890124713\\
63.125	0.2394	4.94189172771486	4.94189172771486\\
63.125	0.24306	5.14517461655096	5.14517461655096\\
63.125	0.24672	5.35277756775543	5.35277756775543\\
63.125	0.25038	5.56470058132827	5.56470058132827\\
63.125	0.25404	5.78094365726949	5.78094365726949\\
63.125	0.2577	6.00150679557907	6.00150679557907\\
63.125	0.26136	6.22638999625702	6.22638999625702\\
63.125	0.26502	6.45559325930335	6.45559325930335\\
63.125	0.26868	6.68911658471805	6.68911658471805\\
63.125	0.27234	6.92695997250113	6.92695997250113\\
63.125	0.276	7.16912342265257	7.16912342265257\\
63.5	0.093	0.339331064129327	0.339331064129327\\
63.5	0.09666	0.373759639782997	0.373759639782997\\
63.5	0.10032	0.412508277805038	0.412508277805038\\
63.5	0.10398	0.455576978195451	0.455576978195451\\
63.5	0.10764	0.502965740954235	0.502965740954235\\
63.5	0.1113	0.554674566081392	0.554674566081392\\
63.5	0.11496	0.610703453576919	0.610703453576919\\
63.5	0.11862	0.671052403440818	0.671052403440818\\
63.5	0.12228	0.735721415673089	0.735721415673089\\
63.5	0.12594	0.804710490273735	0.804710490273735\\
63.5	0.1296	0.878019627242745	0.878019627242745\\
63.5	0.13326	0.955648826580135	0.955648826580135\\
63.5	0.13692	1.03759808828589	1.03759808828589\\
63.5	0.14058	1.12386741236002	1.12386741236002\\
63.5	0.14424	1.21445679880252	1.21445679880252\\
63.5	0.1479	1.30936624761339	1.30936624761339\\
63.5	0.15156	1.40859575879264	1.40859575879264\\
63.5	0.15522	1.51214533234025	1.51214533234025\\
63.5	0.15888	1.62001496825624	1.62001496825624\\
63.5	0.16254	1.7322046665406	1.7322046665406\\
63.5	0.1662	1.84871442719333	1.84871442719333\\
63.5	0.16986	1.96954425021443	1.96954425021443\\
63.5	0.17352	2.09469413560391	2.09469413560391\\
63.5	0.17718	2.22416408336175	2.22416408336175\\
63.5	0.18084	2.35795409348797	2.35795409348797\\
63.5	0.1845	2.49606416598255	2.49606416598255\\
63.5	0.18816	2.63849430084552	2.63849430084552\\
63.5	0.19182	2.78524449807684	2.78524449807684\\
63.5	0.19548	2.93631475767655	2.93631475767655\\
63.5	0.19914	3.09170507964463	3.09170507964463\\
63.5	0.2028	3.25141546398107	3.25141546398107\\
63.5	0.20646	3.4154459106859	3.4154459106859\\
63.5	0.21012	3.58379641975908	3.58379641975908\\
63.5	0.21378	3.75646699120064	3.75646699120064\\
63.5	0.21744	3.93345762501058	3.93345762501058\\
63.5	0.2211	4.11476832118888	4.11476832118888\\
63.5	0.22476	4.30039907973556	4.30039907973556\\
63.5	0.22842	4.49034990065061	4.49034990065061\\
63.5	0.23208	4.68462078393402	4.68462078393402\\
63.5	0.23574	4.88321172958581	4.88321172958581\\
63.5	0.2394	5.08612273760598	5.08612273760598\\
63.5	0.24306	5.29335380799452	5.29335380799452\\
63.5	0.24672	5.50490494075143	5.50490494075143\\
63.5	0.25038	5.7207761358767	5.7207761358767\\
63.5	0.25404	5.94096739337035	5.94096739337035\\
63.5	0.2577	6.16547871323237	6.16547871323237\\
63.5	0.26136	6.39431009546276	6.39431009546276\\
63.5	0.26502	6.62746154006153	6.62746154006153\\
63.5	0.26868	6.86493304702867	6.86493304702867\\
63.5	0.27234	7.10672461636417	7.10672461636417\\
63.5	0.276	7.35283624806805	7.35283624806805\\
63.875	0.093	0.324589590075977	0.324589590075977\\
63.875	0.09666	0.362966347282081	0.362966347282081\\
63.875	0.10032	0.405663166856558	0.405663166856558\\
63.875	0.10398	0.452680048799408	0.452680048799408\\
63.875	0.10764	0.504016993110628	0.504016993110628\\
63.875	0.1113	0.559673999790222	0.559673999790222\\
63.875	0.11496	0.619651068838184	0.619651068838184\\
63.875	0.11862	0.683948200254518	0.683948200254518\\
63.875	0.12228	0.752565394039227	0.752565394039227\\
63.875	0.12594	0.825502650192308	0.825502650192308\\
63.875	0.1296	0.902759968713756	0.902759968713756\\
63.875	0.13326	0.98433734960358	0.98433734960358\\
63.875	0.13692	1.07023479286178	1.07023479286178\\
63.875	0.14058	1.16045229848834	1.16045229848834\\
63.875	0.14424	1.25498986648328	1.25498986648328\\
63.875	0.1479	1.35384749684659	1.35384749684659\\
63.875	0.15156	1.45702518957827	1.45702518957827\\
63.875	0.15522	1.56452294467832	1.56452294467832\\
63.875	0.15888	1.67634076214674	1.67634076214674\\
63.875	0.16254	1.79247864198353	1.79247864198353\\
63.875	0.1662	1.9129365841887	1.9129365841887\\
63.875	0.16986	2.03771458876224	2.03771458876224\\
63.875	0.17352	2.16681265570415	2.16681265570415\\
63.875	0.17718	2.30023078501444	2.30023078501444\\
63.875	0.18084	2.43796897669309	2.43796897669309\\
63.875	0.1845	2.58002723074011	2.58002723074011\\
63.875	0.18816	2.72640554715551	2.72640554715551\\
63.875	0.19182	2.87710392593927	2.87710392593927\\
63.875	0.19548	3.03212236709142	3.03212236709142\\
63.875	0.19914	3.19146087061193	3.19146087061193\\
63.875	0.2028	3.35511943650081	3.35511943650081\\
63.875	0.20646	3.52309806475807	3.52309806475807\\
63.875	0.21012	3.69539675538369	3.69539675538369\\
63.875	0.21378	3.87201550837769	3.87201550837769\\
63.875	0.21744	4.05295432374006	4.05295432374006\\
63.875	0.2211	4.2382132014708	4.2382132014708\\
63.875	0.22476	4.42779214156992	4.42779214156992\\
63.875	0.22842	4.6216911440374	4.6216911440374\\
63.875	0.23208	4.81991020887325	4.81991020887325\\
63.875	0.23574	5.02244933607748	5.02244933607748\\
63.875	0.2394	5.22930852565008	5.22930852565008\\
63.875	0.24306	5.44048777759105	5.44048777759105\\
63.875	0.24672	5.6559870919004	5.6559870919004\\
63.875	0.25038	5.87580646857811	5.87580646857811\\
63.875	0.25404	6.09994590762419	6.09994590762419\\
63.875	0.2577	6.32840540903866	6.32840540903866\\
63.875	0.26136	6.56118497282148	6.56118497282148\\
63.875	0.26502	6.79828459897267	6.79828459897267\\
63.875	0.26868	7.03970428749226	7.03970428749226\\
63.875	0.27234	7.28544403838019	7.28544403838019\\
63.875	0.276	7.53550385163652	7.53550385163652\\
64.25	0.093	0.308802894175607	0.308802894175607\\
64.25	0.09666	0.351127832934146	0.351127832934146\\
64.25	0.10032	0.39777283406106	0.39777283406106\\
64.25	0.10398	0.448737897556349	0.448737897556349\\
64.25	0.10764	0.504023023420003	0.504023023420003\\
64.25	0.1113	0.563628211652032	0.563628211652032\\
64.25	0.11496	0.627553462252433	0.627553462252433\\
64.25	0.11862	0.695798775221204	0.695798775221204\\
64.25	0.12228	0.768364150558349	0.768364150558349\\
64.25	0.12594	0.845249588263864	0.845249588263864\\
64.25	0.1296	0.926455088337751	0.926455088337751\\
64.25	0.13326	1.01198065078001	1.01198065078001\\
64.25	0.13692	1.10182627559064	1.10182627559064\\
64.25	0.14058	1.19599196276964	1.19599196276964\\
64.25	0.14424	1.29447771231702	1.29447771231702\\
64.25	0.1479	1.39728352423276	1.39728352423276\\
64.25	0.15156	1.50440939851688	1.50440939851688\\
64.25	0.15522	1.61585533516937	1.61585533516937\\
64.25	0.15888	1.73162133419023	1.73162133419023\\
64.25	0.16254	1.85170739557946	1.85170739557946\\
64.25	0.1662	1.97611351933706	1.97611351933706\\
64.25	0.16986	2.10483970546303	2.10483970546303\\
64.25	0.17352	2.23788595395738	2.23788595395738\\
64.25	0.17718	2.3752522648201	2.3752522648201\\
64.25	0.18084	2.51693863805119	2.51693863805119\\
64.25	0.1845	2.66294507365065	2.66294507365065\\
64.25	0.18816	2.81327157161848	2.81327157161848\\
64.25	0.19182	2.96791813195469	2.96791813195469\\
64.25	0.19548	3.12688475465926	3.12688475465926\\
64.25	0.19914	3.29017143973221	3.29017143973221\\
64.25	0.2028	3.45777818717353	3.45777818717353\\
64.25	0.20646	3.62970499698322	3.62970499698322\\
64.25	0.21012	3.80595186916129	3.80595186916129\\
64.25	0.21378	3.98651880370772	3.98651880370772\\
64.25	0.21744	4.17140580062252	4.17140580062252\\
64.25	0.2211	4.3606128599057	4.3606128599057\\
64.25	0.22476	4.55413998155725	4.55413998155725\\
64.25	0.22842	4.75198716557717	4.75198716557717\\
64.25	0.23208	4.95415441196546	4.95415441196546\\
64.25	0.23574	5.16064172072213	5.16064172072213\\
64.25	0.2394	5.37144909184716	5.37144909184716\\
64.25	0.24306	5.58657652534058	5.58657652534058\\
64.25	0.24672	5.80602402120235	5.80602402120235\\
64.25	0.25038	6.0297915794325	6.0297915794325\\
64.25	0.25404	6.25787920003103	6.25787920003103\\
64.25	0.2577	6.49028688299792	6.49028688299792\\
64.25	0.26136	6.72701462833317	6.72701462833317\\
64.25	0.26502	6.96806243603682	6.96806243603682\\
64.25	0.26868	7.21343030610883	7.21343030610883\\
64.25	0.27234	7.46311823854921	7.46311823854921\\
64.25	0.276	7.71712623335796	7.71712623335796\\
64.625	0.093	0.291970976428214	0.291970976428214\\
64.625	0.09666	0.338244096739191	0.338244096739191\\
64.625	0.10032	0.38883727941854	0.38883727941854\\
64.625	0.10398	0.443750524466264	0.443750524466264\\
64.625	0.10764	0.502983831882356	0.502983831882356\\
64.625	0.1113	0.566537201666824	0.566537201666824\\
64.625	0.11496	0.634410633819659	0.634410633819659\\
64.625	0.11862	0.706604128340866	0.706604128340866\\
64.625	0.12228	0.783117685230448	0.783117685230448\\
64.625	0.12594	0.863951304488402	0.863951304488402\\
64.625	0.1296	0.949104986114723	0.949104986114723\\
64.625	0.13326	1.03857873010942	1.03857873010942\\
64.625	0.13692	1.13237253647248	1.13237253647248\\
64.625	0.14058	1.23048640520393	1.23048640520393\\
64.625	0.14424	1.33292033630373	1.33292033630373\\
64.625	0.1479	1.43967432977191	1.43967432977191\\
64.625	0.15156	1.55074838560847	1.55074838560847\\
64.625	0.15522	1.66614250381339	1.66614250381339\\
64.625	0.15888	1.78585668438669	1.78585668438669\\
64.625	0.16254	1.90989092732835	1.90989092732835\\
64.625	0.1662	2.0382452326384	2.0382452326384\\
64.625	0.16986	2.17091960031681	2.17091960031681\\
64.625	0.17352	2.30791403036359	2.30791403036359\\
64.625	0.17718	2.44922852277875	2.44922852277875\\
64.625	0.18084	2.59486307756227	2.59486307756227\\
64.625	0.1845	2.74481769471417	2.74481769471417\\
64.625	0.18816	2.89909237423444	2.89909237423444\\
64.625	0.19182	3.05768711612307	3.05768711612307\\
64.625	0.19548	3.22060192038009	3.22060192038009\\
64.625	0.19914	3.38783678700547	3.38783678700547\\
64.625	0.2028	3.55939171599923	3.55939171599923\\
64.625	0.20646	3.73526670736136	3.73526670736136\\
64.625	0.21012	3.91546176109186	3.91546176109186\\
64.625	0.21378	4.09997687719073	4.09997687719073\\
64.625	0.21744	4.28881205565797	4.28881205565797\\
64.625	0.2211	4.48196729649358	4.48196729649358\\
64.625	0.22476	4.67944259969757	4.67944259969757\\
64.625	0.22842	4.88123796526993	4.88123796526993\\
64.625	0.23208	5.08735339321065	5.08735339321065\\
64.625	0.23574	5.29778888351976	5.29778888351976\\
64.625	0.2394	5.51254443619723	5.51254443619723\\
64.625	0.24306	5.73162005124307	5.73162005124307\\
64.625	0.24672	5.95501572865729	5.95501572865729\\
64.625	0.25038	6.18273146843987	6.18273146843987\\
64.625	0.25404	6.41476727059083	6.41476727059083\\
64.625	0.2577	6.65112313511017	6.65112313511017\\
64.625	0.26136	6.89179906199786	6.89179906199786\\
64.625	0.26502	7.13679505125394	7.13679505125394\\
64.625	0.26868	7.38611110287838	7.38611110287838\\
64.625	0.27234	7.6397472168712	7.6397472168712\\
64.625	0.276	7.89770339323239	7.89770339323239\\
65	0.093	0.274093836833803	0.274093836833803\\
65	0.09666	0.324315138697219	0.324315138697219\\
65	0.10032	0.378856502929006	0.378856502929006\\
65	0.10398	0.437717929529164	0.437717929529164\\
65	0.10764	0.500899418497691	0.500899418497691\\
65	0.1113	0.568400969834593	0.568400969834593\\
65	0.11496	0.640222583539867	0.640222583539867\\
65	0.11862	0.716364259613512	0.716364259613512\\
65	0.12228	0.796825998055529	0.796825998055529\\
65	0.12594	0.881607798865917	0.881607798865917\\
65	0.1296	0.970709662044673	0.970709662044673\\
65	0.13326	1.06413158759181	1.06413158759181\\
65	0.13692	1.16187357550731	1.16187357550731\\
65	0.14058	1.26393562579119	1.26393562579119\\
65	0.14424	1.37031773844343	1.37031773844343\\
65	0.1479	1.48101991346405	1.48101991346405\\
65	0.15156	1.59604215085304	1.59604215085304\\
65	0.15522	1.7153844506104	1.7153844506104\\
65	0.15888	1.83904681273614	1.83904681273614\\
65	0.16254	1.96702923723023	1.96702923723023\\
65	0.1662	2.09933172409271	2.09933172409271\\
65	0.16986	2.23595427332356	2.23595427332356\\
65	0.17352	2.37689688492278	2.37689688492278\\
65	0.17718	2.52215955889037	2.52215955889037\\
65	0.18084	2.67174229522634	2.67174229522634\\
65	0.1845	2.82564509393066	2.82564509393066\\
65	0.18816	2.98386795500337	2.98386795500337\\
65	0.19182	3.14641087844445	3.14641087844445\\
65	0.19548	3.3132738642539	3.3132738642539\\
65	0.19914	3.48445691243172	3.48445691243172\\
65	0.2028	3.65996002297791	3.65996002297791\\
65	0.20646	3.83978319589248	3.83978319589248\\
65	0.21012	4.02392643117541	4.02392643117541\\
65	0.21378	4.21238972882672	4.21238972882672\\
65	0.21744	4.40517308884639	4.40517308884639\\
65	0.2211	4.60227651123444	4.60227651123444\\
65	0.22476	4.80369999599087	4.80369999599087\\
65	0.22842	5.00944354311566	5.00944354311566\\
65	0.23208	5.21950715260882	5.21950715260882\\
65	0.23574	5.43389082447036	5.43389082447036\\
65	0.2394	5.65259455870027	5.65259455870027\\
65	0.24306	5.87561835529855	5.87561835529855\\
65	0.24672	6.1029622142652	6.1029622142652\\
65	0.25038	6.33462613560023	6.33462613560023\\
65	0.25404	6.57061011930362	6.57061011930362\\
65	0.2577	6.81091416537539	6.81091416537539\\
65	0.26136	7.05553827381553	7.05553827381553\\
65	0.26502	7.30448244462403	7.30448244462403\\
65	0.26868	7.55774667780092	7.55774667780092\\
65	0.27234	7.81533097334617	7.81533097334617\\
65	0.276	8.07723533125979	8.07723533125979\\
65.375	0.093	0.25517147539237	0.25517147539237\\
65.375	0.09666	0.309340958808217	0.309340958808217\\
65.375	0.10032	0.367830504592442	0.367830504592442\\
65.375	0.10398	0.430640112745035	0.430640112745035\\
65.375	0.10764	0.497769783266	0.497769783266\\
65.375	0.1113	0.569219516155341	0.569219516155341\\
65.375	0.11496	0.644989311413049	0.644989311413049\\
65.375	0.11862	0.725079169039129	0.725079169039129\\
65.375	0.12228	0.809489089033584	0.809489089033584\\
65.375	0.12594	0.898219071396411	0.898219071396411\\
65.375	0.1296	0.991269116127601	0.991269116127601\\
65.375	0.13326	1.08863922322717	1.08863922322717\\
65.375	0.13692	1.19032939269511	1.19032939269511\\
65.375	0.14058	1.29633962453142	1.29633962453142\\
65.375	0.14424	1.4066699187361	1.4066699187361\\
65.375	0.1479	1.52132027530916	1.52132027530916\\
65.375	0.15156	1.64029069425058	1.64029069425058\\
65.375	0.15522	1.76358117556038	1.76358117556038\\
65.375	0.15888	1.89119171923855	1.89119171923855\\
65.375	0.16254	2.02312232528509	2.02312232528509\\
65.375	0.1662	2.1593729937	2.1593729937\\
65.375	0.16986	2.29994372448329	2.29994372448329\\
65.375	0.17352	2.44483451763494	2.44483451763494\\
65.375	0.17718	2.59404537315497	2.59404537315497\\
65.375	0.18084	2.74757629104337	2.74757629104337\\
65.375	0.1845	2.90542727130014	2.90542727130014\\
65.375	0.18816	3.06759831392528	3.06759831392528\\
65.375	0.19182	3.23408941891879	3.23408941891879\\
65.375	0.19548	3.40490058628068	3.40490058628068\\
65.375	0.19914	3.58003181601094	3.58003181601094\\
65.375	0.2028	3.75948310810957	3.75948310810957\\
65.375	0.20646	3.94325446257657	3.94325446257657\\
65.375	0.21012	4.13134587941194	4.13134587941194\\
65.375	0.21378	4.32375735861568	4.32375735861568\\
65.375	0.21744	4.52048890018779	4.52048890018779\\
65.375	0.2211	4.72154050412828	4.72154050412828\\
65.375	0.22476	4.92691217043714	4.92691217043714\\
65.375	0.22842	5.13660389911437	5.13660389911437\\
65.375	0.23208	5.35061569015997	5.35061569015997\\
65.375	0.23574	5.56894754357394	5.56894754357394\\
65.375	0.2394	5.79159945935628	5.79159945935628\\
65.375	0.24306	6.01857143750701	6.01857143750701\\
65.375	0.24672	6.24986347802609	6.24986347802609\\
65.375	0.25038	6.48547558091355	6.48547558091355\\
65.375	0.25404	6.72540774616939	6.72540774616939\\
65.375	0.2577	6.96965997379359	6.96965997379359\\
65.375	0.26136	7.21823226378615	7.21823226378615\\
65.375	0.26502	7.4711246161471	7.4711246161471\\
65.375	0.26868	7.72833703087642	7.72833703087642\\
65.375	0.27234	7.98986950797411	7.98986950797411\\
65.375	0.276	8.25572204744017	8.25572204744017\\
65.75	0.093	0.235203892103917	0.235203892103917\\
65.75	0.09666	0.293321557072201	0.293321557072201\\
65.75	0.10032	0.355759284408862	0.355759284408862\\
65.75	0.10398	0.422517074113893	0.422517074113893\\
65.75	0.10764	0.493594926187293	0.493594926187293\\
65.75	0.1113	0.568992840629068	0.568992840629068\\
65.75	0.11496	0.648710817439214	0.648710817439214\\
65.75	0.11862	0.732748856617732	0.732748856617732\\
65.75	0.12228	0.821106958164622	0.821106958164622\\
65.75	0.12594	0.913785122079884	0.913785122079884\\
65.75	0.1296	1.01078334836351	1.01078334836351\\
65.75	0.13326	1.11210163701552	1.11210163701552\\
65.75	0.13692	1.21773998803589	1.21773998803589\\
65.75	0.14058	1.32769840142464	1.32769840142464\\
65.75	0.14424	1.44197687718176	1.44197687718176\\
65.75	0.1479	1.56057541530725	1.56057541530725\\
65.75	0.15156	1.68349401580112	1.68349401580112\\
65.75	0.15522	1.81073267866335	1.81073267866335\\
65.75	0.15888	1.94229140389395	1.94229140389395\\
65.75	0.16254	2.07817019149293	2.07817019149293\\
65.75	0.1662	2.21836904146028	2.21836904146028\\
65.75	0.16986	2.362887953796	2.362887953796\\
65.75	0.17352	2.51172692850009	2.51172692850009\\
65.75	0.17718	2.66488596557256	2.66488596557256\\
65.75	0.18084	2.82236506501339	2.82236506501339\\
65.75	0.1845	2.98416422682259	2.98416422682259\\
65.75	0.18816	3.15028345100017	3.15028345100017\\
65.75	0.19182	3.32072273754612	3.32072273754612\\
65.75	0.19548	3.49548208646044	3.49548208646044\\
65.75	0.19914	3.67456149774314	3.67456149774314\\
65.75	0.2028	3.85796097139421	3.85796097139421\\
65.75	0.20646	4.04568050741364	4.04568050741364\\
65.75	0.21012	4.23772010580144	4.23772010580144\\
65.75	0.21378	4.43407976655762	4.43407976655762\\
65.75	0.21744	4.63475948968217	4.63475948968217\\
65.75	0.2211	4.8397592751751	4.8397592751751\\
65.75	0.22476	5.04907912303639	5.04907912303639\\
65.75	0.22842	5.26271903326606	5.26271903326606\\
65.75	0.23208	5.4806790058641	5.4806790058641\\
65.75	0.23574	5.7029590408305	5.7029590408305\\
65.75	0.2394	5.92955913816529	5.92955913816529\\
65.75	0.24306	6.16047929786844	6.16047929786844\\
65.75	0.24672	6.39571951993997	6.39571951993997\\
65.75	0.25038	6.63527980437986	6.63527980437986\\
65.75	0.25404	6.87916015118813	6.87916015118813\\
65.75	0.2577	7.12736056036478	7.12736056036478\\
65.75	0.26136	7.37988103190978	7.37988103190978\\
65.75	0.26502	7.63672156582316	7.63672156582316\\
65.75	0.26868	7.89788216210492	7.89788216210492\\
65.75	0.27234	8.16336282075504	8.16336282075504\\
65.75	0.276	8.43316354177354	8.43316354177354\\
66.125	0.093	0.214191086968443	0.214191086968443\\
66.125	0.09666	0.276256933489166	0.276256933489166\\
66.125	0.10032	0.342642842378261	0.342642842378261\\
66.125	0.10398	0.413348813635731	0.413348813635731\\
66.125	0.10764	0.488374847261565	0.488374847261565\\
66.125	0.1113	0.567720943255779	0.567720943255779\\
66.125	0.11496	0.65138710161836	0.65138710161836\\
66.125	0.11862	0.739373322349312	0.739373322349312\\
66.125	0.12228	0.83167960544864	0.83167960544864\\
66.125	0.12594	0.928305950916337	0.928305950916337\\
66.125	0.1296	1.0292523587524	1.0292523587524\\
66.125	0.13326	1.13451882895685	1.13451882895685\\
66.125	0.13692	1.24410536152966	1.24410536152966\\
66.125	0.14058	1.35801195647084	1.35801195647084\\
66.125	0.14424	1.4762386137804	1.4762386137804\\
66.125	0.1479	1.59878533345832	1.59878533345832\\
66.125	0.15156	1.72565211550462	1.72565211550462\\
66.125	0.15522	1.85683895991929	1.85683895991929\\
66.125	0.15888	1.99234586670234	1.99234586670234\\
66.125	0.16254	2.13217283585375	2.13217283585375\\
66.125	0.1662	2.27631986737353	2.27631986737353\\
66.125	0.16986	2.42478696126169	2.42478696126169\\
66.125	0.17352	2.57757411751822	2.57757411751822\\
66.125	0.17718	2.73468133614312	2.73468133614312\\
66.125	0.18084	2.89610861713639	2.89610861713639\\
66.125	0.1845	3.06185596049803	3.06185596049803\\
66.125	0.18816	3.23192336622804	3.23192336622804\\
66.125	0.19182	3.40631083432643	3.40631083432643\\
66.125	0.19548	3.58501836479318	3.58501836479318\\
66.125	0.19914	3.76804595762832	3.76804595762832\\
66.125	0.2028	3.95539361283182	3.95539361283182\\
66.125	0.20646	4.1470613304037	4.1470613304037\\
66.125	0.21012	4.34304911034393	4.34304911034393\\
66.125	0.21378	4.54335695265255	4.54335695265255\\
66.125	0.21744	4.74798485732954	4.74798485732954\\
66.125	0.2211	4.9569328243749	4.9569328243749\\
66.125	0.22476	5.17020085378863	5.17020085378863\\
66.125	0.22842	5.38778894557073	5.38778894557073\\
66.125	0.23208	5.6096970997212	5.6096970997212\\
66.125	0.23574	5.83592531624005	5.83592531624005\\
66.125	0.2394	6.06647359512727	6.06647359512727\\
66.125	0.24306	6.30134193638285	6.30134193638285\\
66.125	0.24672	6.54053034000682	6.54053034000682\\
66.125	0.25038	6.78403880599915	6.78403880599915\\
66.125	0.25404	7.03186733435986	7.03186733435986\\
66.125	0.2577	7.28401592508893	7.28401592508893\\
66.125	0.26136	7.54048457818637	7.54048457818637\\
66.125	0.26502	7.80127329365219	7.80127329365219\\
66.125	0.26868	8.06638207148639	8.06638207148639\\
66.125	0.27234	8.33581091168895	8.33581091168895\\
66.125	0.276	8.60955981425989	8.60955981425989\\
66.5	0.093	0.192133059985947	0.192133059985947\\
66.5	0.09666	0.258147088059105	0.258147088059105\\
66.5	0.10032	0.328481178500638	0.328481178500638\\
66.5	0.10398	0.403135331310543	0.403135331310543\\
66.5	0.10764	0.482109546488815	0.482109546488815\\
66.5	0.1113	0.565403824035464	0.565403824035464\\
66.5	0.11496	0.653018163950483	0.653018163950483\\
66.5	0.11862	0.74495256623387	0.74495256623387\\
66.5	0.12228	0.841207030885637	0.841207030885637\\
66.5	0.12594	0.941781557905771	0.941781557905771\\
66.5	0.1296	1.04667614729427	1.04667614729427\\
66.5	0.13326	1.15589079905115	1.15589079905115\\
66.5	0.13692	1.2694255131764	1.2694255131764\\
66.5	0.14058	1.38728028967002	1.38728028967002\\
66.5	0.14424	1.50945512853201	1.50945512853201\\
66.5	0.1479	1.63595002976237	1.63595002976237\\
66.5	0.15156	1.76676499336111	1.76676499336111\\
66.5	0.15522	1.90190001932821	1.90190001932821\\
66.5	0.15888	2.04135510766369	2.04135510766369\\
66.5	0.16254	2.18513025836754	2.18513025836754\\
66.5	0.1662	2.33322547143976	2.33322547143976\\
66.5	0.16986	2.48564074688035	2.48564074688035\\
66.5	0.17352	2.64237608468932	2.64237608468932\\
66.5	0.17718	2.80343148486665	2.80343148486665\\
66.5	0.18084	2.96880694741237	2.96880694741237\\
66.5	0.1845	3.13850247232644	3.13850247232644\\
66.5	0.18816	3.31251805960889	3.31251805960889\\
66.5	0.19182	3.49085370925971	3.49085370925971\\
66.5	0.19548	3.67350942127891	3.67350942127891\\
66.5	0.19914	3.86048519566647	3.86048519566647\\
66.5	0.2028	4.05178103242242	4.05178103242242\\
66.5	0.20646	4.24739693154672	4.24739693154672\\
66.5	0.21012	4.4473328930394	4.4473328930394\\
66.5	0.21378	4.65158891690045	4.65158891690045\\
66.5	0.21744	4.86016500312988	4.86016500312988\\
66.5	0.2211	5.07306115172767	5.07306115172767\\
66.5	0.22476	5.29027736269384	5.29027736269384\\
66.5	0.22842	5.51181363602838	5.51181363602838\\
66.5	0.23208	5.73766997173129	5.73766997173129\\
66.5	0.23574	5.96784636980257	5.96784636980257\\
66.5	0.2394	6.20234283024223	6.20234283024223\\
66.5	0.24306	6.44115935305025	6.44115935305025\\
66.5	0.24672	6.68429593822665	6.68429593822665\\
66.5	0.25038	6.93175258577142	6.93175258577142\\
66.5	0.25404	7.18352929568456	7.18352929568456\\
66.5	0.2577	7.43962606796607	7.43962606796607\\
66.5	0.26136	7.70004290261595	7.70004290261595\\
66.5	0.26502	7.9647797996342	7.9647797996342\\
66.5	0.26868	8.23383675902083	8.23383675902083\\
66.5	0.27234	8.50721378077583	8.50721378077583\\
66.5	0.276	8.7849108648992	8.7849108648992\\
66.875	0.093	0.169029811156433	0.169029811156433\\
66.875	0.09666	0.238992020782029	0.238992020782029\\
66.875	0.10032	0.313274292775997	0.313274292775997\\
66.875	0.10398	0.39187662713834	0.39187662713834\\
66.875	0.10764	0.474799023869047	0.474799023869047\\
66.875	0.1113	0.562041482968133	0.562041482968133\\
66.875	0.11496	0.653604004435588	0.653604004435588\\
66.875	0.11862	0.749486588271413	0.749486588271413\\
66.875	0.12228	0.849689234475611	0.849689234475611\\
66.875	0.12594	0.954211943048183	0.954211943048183\\
66.875	0.1296	1.06305471398912	1.06305471398912\\
66.875	0.13326	1.17621754729844	1.17621754729844\\
66.875	0.13692	1.29370044297612	1.29370044297612\\
66.875	0.14058	1.41550340102218	1.41550340102218\\
66.875	0.14424	1.5416264214366	1.5416264214366\\
66.875	0.1479	1.67206950421941	1.67206950421941\\
66.875	0.15156	1.80683264937058	1.80683264937058\\
66.875	0.15522	1.94591585689012	1.94591585689012\\
66.875	0.15888	2.08931912677804	2.08931912677804\\
66.875	0.16254	2.23704245903432	2.23704245903432\\
66.875	0.1662	2.38908585365897	2.38908585365897\\
66.875	0.16986	2.545449310652	2.545449310652\\
66.875	0.17352	2.70613283001341	2.70613283001341\\
66.875	0.17718	2.87113641174318	2.87113641174318\\
66.875	0.18084	3.04046005584132	3.04046005584132\\
66.875	0.1845	3.21410376230784	3.21410376230784\\
66.875	0.18816	3.39206753114273	3.39206753114273\\
66.875	0.19182	3.57435136234598	3.57435136234598\\
66.875	0.19548	3.76095525591761	3.76095525591761\\
66.875	0.19914	3.95187921185762	3.95187921185762\\
66.875	0.2028	4.14712323016599	4.14712323016599\\
66.875	0.20646	4.34668731084274	4.34668731084274\\
66.875	0.21012	4.55057145388785	4.55057145388785\\
66.875	0.21378	4.75877565930134	4.75877565930134\\
66.875	0.21744	4.97129992708321	4.97129992708321\\
66.875	0.2211	5.18814425723344	5.18814425723344\\
66.875	0.22476	5.40930864975204	5.40930864975204\\
66.875	0.22842	5.63479310463901	5.63479310463901\\
66.875	0.23208	5.86459762189436	5.86459762189436\\
66.875	0.23574	6.09872220151808	6.09872220151808\\
66.875	0.2394	6.33716684351017	6.33716684351017\\
66.875	0.24306	6.57993154787062	6.57993154787062\\
66.875	0.24672	6.82701631459947	6.82701631459947\\
66.875	0.25038	7.07842114369668	7.07842114369668\\
66.875	0.25404	7.33414603516225	7.33414603516225\\
66.875	0.2577	7.5941909889962	7.5941909889962\\
66.875	0.26136	7.85855600519851	7.85855600519851\\
66.875	0.26502	8.1272410837692	8.1272410837692\\
66.875	0.26868	8.40024622470827	8.40024622470827\\
66.875	0.27234	8.6775714280157	8.6775714280157\\
66.875	0.276	8.95921669369152	8.95921669369152\\
67.25	0.093	0.144881340479904	0.144881340479904\\
67.25	0.09666	0.218791731657935	0.218791731657935\\
67.25	0.10032	0.297022185204337	0.297022185204337\\
67.25	0.10398	0.379572701119115	0.379572701119115\\
67.25	0.10764	0.46644327940226	0.46644327940226\\
67.25	0.1113	0.557633920053781	0.557633920053781\\
67.25	0.11496	0.653144623073674	0.653144623073674\\
67.25	0.11862	0.752975388461934	0.752975388461934\\
67.25	0.12228	0.85712621621857	0.85712621621857\\
67.25	0.12594	0.965597106343577	0.965597106343577\\
67.25	0.1296	1.07838805883695	1.07838805883695\\
67.25	0.13326	1.1954990736987	1.1954990736987\\
67.25	0.13692	1.31693015092882	1.31693015092882\\
67.25	0.14058	1.44268129052732	1.44268129052732\\
67.25	0.14424	1.57275249249418	1.57275249249418\\
67.25	0.1479	1.70714375682942	1.70714375682942\\
67.25	0.15156	1.84585508353303	1.84585508353303\\
67.25	0.15522	1.988886472605	1.988886472605\\
67.25	0.15888	2.13623792404536	2.13623792404536\\
67.25	0.16254	2.28790943785408	2.28790943785408\\
67.25	0.1662	2.44390101403117	2.44390101403117\\
67.25	0.16986	2.60421265257664	2.60421265257664\\
67.25	0.17352	2.76884435349048	2.76884435349048\\
67.25	0.17718	2.93779611677269	2.93779611677269\\
67.25	0.18084	3.11106794242327	3.11106794242327\\
67.25	0.1845	3.28865983044221	3.28865983044221\\
67.25	0.18816	3.47057178082954	3.47057178082954\\
67.25	0.19182	3.65680379358523	3.65680379358523\\
67.25	0.19548	3.8473558687093	3.8473558687093\\
67.25	0.19914	4.04222800620174	4.04222800620174\\
67.25	0.2028	4.24142020606255	4.24142020606255\\
67.25	0.20646	4.44493246829173	4.44493246829173\\
67.25	0.21012	4.65276479288928	4.65276479288928\\
67.25	0.21378	4.8649171798552	4.8649171798552\\
67.25	0.21744	5.0813896291895	5.0813896291895\\
67.25	0.2211	5.30218214089217	5.30218214089217\\
67.25	0.22476	5.52729471496321	5.52729471496321\\
67.25	0.22842	5.75672735140263	5.75672735140263\\
67.25	0.23208	5.9904800502104	5.9904800502104\\
67.25	0.23574	6.22855281138656	6.22855281138656\\
67.25	0.2394	6.47094563493108	6.47094563493108\\
67.25	0.24306	6.71765852084399	6.71765852084399\\
67.25	0.24672	6.96869146912525	6.96869146912525\\
67.25	0.25038	7.2240444797749	7.2240444797749\\
67.25	0.25404	7.48371755279292	7.48371755279292\\
67.25	0.2577	7.7477106881793	7.7477106881793\\
67.25	0.26136	8.01602388593405	8.01602388593405\\
67.25	0.26502	8.28865714605718	8.28865714605718\\
67.25	0.26868	8.56561046854867	8.56561046854867\\
67.25	0.27234	8.84688385340855	8.84688385340855\\
67.25	0.276	9.13247730063679	9.13247730063679\\
67.625	0.093	0.119687647956346	0.119687647956346\\
67.625	0.09666	0.197546220686811	0.197546220686811\\
67.625	0.10032	0.279724855785656	0.279724855785656\\
67.625	0.10398	0.366223553252867	0.366223553252867\\
67.625	0.10764	0.457042313088452	0.457042313088452\\
67.625	0.1113	0.552181135292407	0.552181135292407\\
67.625	0.11496	0.651640019864734	0.651640019864734\\
67.625	0.11862	0.755418966805433	0.755418966805433\\
67.625	0.12228	0.863517976114503	0.863517976114503\\
67.625	0.12594	0.975937047791949	0.975937047791949\\
67.625	0.1296	1.09267618183776	1.09267618183776\\
67.625	0.13326	1.21373537825195	1.21373537825195\\
67.625	0.13692	1.3391146370345	1.3391146370345\\
67.625	0.14058	1.46881395818543	1.46881395818543\\
67.625	0.14424	1.60283334170473	1.60283334170473\\
67.625	0.1479	1.74117278759241	1.74117278759241\\
67.625	0.15156	1.88383229584845	1.88383229584845\\
67.625	0.15522	2.03081186647286	2.03081186647286\\
67.625	0.15888	2.18211149946565	2.18211149946565\\
67.625	0.16254	2.33773119482681	2.33773119482681\\
67.625	0.1662	2.49767095255634	2.49767095255634\\
67.625	0.16986	2.66193077265424	2.66193077265424\\
67.625	0.17352	2.83051065512052	2.83051065512052\\
67.625	0.17718	3.00341059995516	3.00341059995516\\
67.625	0.18084	3.18063060715818	3.18063060715818\\
67.625	0.1845	3.36217067672956	3.36217067672956\\
67.625	0.18816	3.54803080866933	3.54803080866933\\
67.625	0.19182	3.73821100297745	3.73821100297745\\
67.625	0.19548	3.93271125965396	3.93271125965396\\
67.625	0.19914	4.13153157869884	4.13153157869884\\
67.625	0.2028	4.33467196011208	4.33467196011208\\
67.625	0.20646	4.5421324038937	4.5421324038937\\
67.625	0.21012	4.75391291004369	4.75391291004369\\
67.625	0.21378	4.97001347856205	4.97001347856205\\
67.625	0.21744	5.19043410944878	5.19043410944878\\
67.625	0.2211	5.41517480270389	5.41517480270389\\
67.625	0.22476	5.64423555832737	5.64423555832737\\
67.625	0.22842	5.87761637631921	5.87761637631921\\
67.625	0.23208	6.11531725667943	6.11531725667943\\
67.625	0.23574	6.35733819940802	6.35733819940802\\
67.625	0.2394	6.60367920450498	6.60367920450498\\
67.625	0.24306	6.85434027197032	6.85434027197032\\
67.625	0.24672	7.10932140180403	7.10932140180403\\
67.625	0.25038	7.3686225940061	7.3686225940061\\
67.625	0.25404	7.63224384857655	7.63224384857655\\
67.625	0.2577	7.90018516551538	7.90018516551538\\
67.625	0.26136	8.17244654482257	8.17244654482257\\
67.625	0.26502	8.44902798649813	8.44902798649813\\
67.625	0.26868	8.72992949054206	8.72992949054206\\
67.625	0.27234	9.01515105695437	9.01515105695437\\
67.625	0.276	9.30469268573506	9.30469268573506\\
68	0.093	0.0934487335857723	0.0934487335857723\\
68	0.09666	0.175255487868676	0.175255487868676\\
68	0.10032	0.261382304519955	0.261382304519955\\
68	0.10398	0.351829183539605	0.351829183539605\\
68	0.10764	0.446596124927624	0.446596124927624\\
68	0.1113	0.545683128684015	0.545683128684015\\
68	0.11496	0.64909019480878	0.64909019480878\\
68	0.11862	0.756817323301913	0.756817323301913\\
68	0.12228	0.868864514163422	0.868864514163422\\
68	0.12594	0.985231767393302	0.985231767393302\\
68	0.1296	1.10591908299155	1.10591908299155\\
68	0.13326	1.23092646095817	1.23092646095817\\
68	0.13692	1.36025390129317	1.36025390129317\\
68	0.14058	1.49390140399653	1.49390140399653\\
68	0.14424	1.63186896906827	1.63186896906827\\
68	0.1479	1.77415659650838	1.77415659650838\\
68	0.15156	1.92076428631686	1.92076428631686\\
68	0.15522	2.07169203849371	2.07169203849371\\
68	0.15888	2.22693985303893	2.22693985303893\\
68	0.16254	2.38650772995252	2.38650772995252\\
68	0.1662	2.55039566923449	2.55039566923449\\
68	0.16986	2.71860367088483	2.71860367088483\\
68	0.17352	2.89113173490354	2.89113173490354\\
68	0.17718	3.06797986129063	3.06797986129063\\
68	0.18084	3.24914805004608	3.24914805004608\\
68	0.1845	3.4346363011699	3.4346363011699\\
68	0.18816	3.6244446146621	3.6244446146621\\
68	0.19182	3.81857299052266	3.81857299052266\\
68	0.19548	4.01702142875161	4.01702142875161\\
68	0.19914	4.21978992934892	4.21978992934892\\
68	0.2028	4.4268784923146	4.4268784923146\\
68	0.20646	4.63828711764866	4.63828711764866\\
68	0.21012	4.85401580535108	4.85401580535108\\
68	0.21378	5.07406455542188	5.07406455542188\\
68	0.21744	5.29843336786104	5.29843336786104\\
68	0.2211	5.52712224266859	5.52712224266859\\
68	0.22476	5.7601311798445	5.7601311798445\\
68	0.22842	5.99746017938879	5.99746017938879\\
68	0.23208	6.23910924130143	6.23910924130143\\
68	0.23574	6.48507836558246	6.48507836558246\\
68	0.2394	6.73536755223187	6.73536755223187\\
68	0.24306	6.98997680124963	6.98997680124963\\
68	0.24672	7.24890611263578	7.24890611263578\\
68	0.25038	7.5121554863903	7.5121554863903\\
68	0.25404	7.77972492251318	7.77972492251318\\
68	0.2577	8.05161442100444	8.05161442100444\\
68	0.26136	8.32782398186406	8.32782398186406\\
68	0.26502	8.60835360509206	8.60835360509206\\
68	0.26868	8.89320329068843	8.89320329068843\\
68	0.27234	9.18237303865318	9.18237303865318\\
68	0.276	9.4758628489863	9.4758628489863\\
68.375	0.093	0.066164597368177	0.066164597368177\\
68.375	0.09666	0.151919533203515	0.151919533203515\\
68.375	0.10032	0.241994531407229	0.241994531407229\\
68.375	0.10398	0.336389591979314	0.336389591979314\\
68.375	0.10764	0.435104714919771	0.435104714919771\\
68.375	0.1113	0.5381399002286	0.5381399002286\\
68.375	0.11496	0.6454951479058	0.6454951479058\\
68.375	0.11862	0.757170457951371	0.757170457951371\\
68.375	0.12228	0.873165830365315	0.873165830365315\\
68.375	0.12594	0.99348126514763	0.99348126514763\\
68.375	0.1296	1.11811676229832	1.11811676229832\\
68.375	0.13326	1.24707232181737	1.24707232181737\\
68.375	0.13692	1.38034794370481	1.38034794370481\\
68.375	0.14058	1.51794362796061	1.51794362796061\\
68.375	0.14424	1.65985937458478	1.65985937458478\\
68.375	0.1479	1.80609518357732	1.80609518357732\\
68.375	0.15156	1.95665105493824	1.95665105493824\\
68.375	0.15522	2.11152698866753	2.11152698866753\\
68.375	0.15888	2.27072298476519	2.27072298476519\\
68.375	0.16254	2.43423904323122	2.43423904323122\\
68.375	0.1662	2.60207516406562	2.60207516406562\\
68.375	0.16986	2.7742313472684	2.7742313472684\\
68.375	0.17352	2.95070759283954	2.95070759283954\\
68.375	0.17718	3.13150390077906	3.13150390077906\\
68.375	0.18084	3.31662027108696	3.31662027108696\\
68.375	0.1845	3.50605670376321	3.50605670376321\\
68.375	0.18816	3.69981319880784	3.69981319880784\\
68.375	0.19182	3.89788975622085	3.89788975622085\\
68.375	0.19548	4.10028637600222	4.10028637600222\\
68.375	0.19914	4.30700305815197	4.30700305815197\\
68.375	0.2028	4.51803980267009	4.51803980267009\\
68.375	0.20646	4.73339660955658	4.73339660955658\\
68.375	0.21012	4.95307347881144	4.95307347881144\\
68.375	0.21378	5.17707041043468	5.17707041043468\\
68.375	0.21744	5.40538740442628	5.40538740442628\\
68.375	0.2211	5.63802446078626	5.63802446078626\\
68.375	0.22476	5.87498157951461	5.87498157951461\\
68.375	0.22842	6.11625876061134	6.11625876061134\\
68.375	0.23208	6.36185600407642	6.36185600407642\\
68.375	0.23574	6.61177330990989	6.61177330990989\\
68.375	0.2394	6.86601067811172	6.86601067811172\\
68.375	0.24306	7.12456810868193	7.12456810868193\\
68.375	0.24672	7.3874456016205	7.3874456016205\\
68.375	0.25038	7.65464315692745	7.65464315692745\\
68.375	0.25404	7.92616077460278	7.92616077460278\\
68.375	0.2577	8.20199845464647	8.20199845464647\\
68.375	0.26136	8.48215619705853	8.48215619705853\\
68.375	0.26502	8.76663400183898	8.76663400183898\\
68.375	0.26868	9.05543186898778	9.05543186898778\\
68.375	0.27234	9.34854979850496	9.34854979850496\\
68.375	0.276	9.64598779039051	9.64598779039051\\
68.75	0.093	0.0378352393035597	0.0378352393035597\\
68.75	0.09666	0.127538356691336	0.127538356691336\\
68.75	0.10032	0.221561536447485	0.221561536447485\\
68.75	0.10398	0.319904778572008	0.319904778572008\\
68.75	0.10764	0.4225680830649	0.4225680830649\\
68.75	0.1113	0.529551449926163	0.529551449926163\\
68.75	0.11496	0.640854879155802	0.640854879155802\\
68.75	0.11862	0.756478370753808	0.756478370753808\\
68.75	0.12228	0.876421924720189	0.876421924720189\\
68.75	0.12594	1.00068554105494	1.00068554105494\\
68.75	0.1296	1.12926921975806	1.12926921975806\\
68.75	0.13326	1.26217296082956	1.26217296082956\\
68.75	0.13692	1.39939676426942	1.39939676426942\\
68.75	0.14058	1.54094063007767	1.54094063007767\\
68.75	0.14424	1.68680455825427	1.68680455825427\\
68.75	0.1479	1.83698854879925	1.83698854879925\\
68.75	0.15156	1.99149260171261	1.99149260171261\\
68.75	0.15522	2.15031671699433	2.15031671699433\\
68.75	0.15888	2.31346089464443	2.31346089464443\\
68.75	0.16254	2.48092513466289	2.48092513466289\\
68.75	0.1662	2.65270943704973	2.65270943704973\\
68.75	0.16986	2.82881380180494	2.82881380180494\\
68.75	0.17352	3.00923822892852	3.00923822892852\\
68.75	0.17718	3.19398271842048	3.19398271842048\\
68.75	0.18084	3.38304727028081	3.38304727028081\\
68.75	0.1845	3.5764318845095	3.5764318845095\\
68.75	0.18816	3.77413656110657	3.77413656110657\\
68.75	0.19182	3.97616130007201	3.97616130007201\\
68.75	0.19548	4.18250610140583	4.18250610140583\\
68.75	0.19914	4.39317096510801	4.39317096510801\\
68.75	0.2028	4.60815589117856	4.60815589117856\\
68.75	0.20646	4.82746087961749	4.82746087961749\\
68.75	0.21012	5.05108593042479	5.05108593042479\\
68.75	0.21378	5.27903104360046	5.27903104360046\\
68.75	0.21744	5.5112962191445	5.5112962191445\\
68.75	0.2211	5.74788145705692	5.74788145705692\\
68.75	0.22476	5.9887867573377	5.9887867573377\\
68.75	0.22842	6.23401211998686	6.23401211998686\\
68.75	0.23208	6.48355754500438	6.48355754500438\\
68.75	0.23574	6.73742303239028	6.73742303239028\\
68.75	0.2394	6.99560858214456	6.99560858214456\\
68.75	0.24306	7.2581141942672	7.2581141942672\\
68.75	0.24672	7.52493986875822	7.52493986875822\\
68.75	0.25038	7.79608560561761	7.79608560561761\\
68.75	0.25404	8.07155140484536	8.07155140484536\\
68.75	0.2577	8.35133726644149	8.35133726644149\\
68.75	0.26136	8.63544319040599	8.63544319040599\\
68.75	0.26502	8.92386917673886	8.92386917673886\\
68.75	0.26868	9.21661522544011	9.21661522544011\\
68.75	0.27234	9.51368133650972	9.51368133650972\\
68.75	0.276	9.81506750994772	9.81506750994772\\
69.125	0.093	0.00846065939192386	0.00846065939192386\\
69.125	0.09666	0.102111958332135	0.102111958332135\\
69.125	0.10032	0.200083319640722	0.200083319640722\\
69.125	0.10398	0.302374743317684	0.302374743317684\\
69.125	0.10764	0.40898622936301	0.40898622936301\\
69.125	0.1113	0.519917777776712	0.519917777776712\\
69.125	0.11496	0.635169388558785	0.635169388558785\\
69.125	0.11862	0.754741061709229	0.754741061709229\\
69.125	0.12228	0.878632797228045	0.878632797228045\\
69.125	0.12594	1.00684459511523	1.00684459511523\\
69.125	0.1296	1.13937645537079	1.13937645537079\\
69.125	0.13326	1.27622837799472	1.27622837799472\\
69.125	0.13692	1.41740036298703	1.41740036298703\\
69.125	0.14058	1.5628924103477	1.5628924103477\\
69.125	0.14424	1.71270452007674	1.71270452007674\\
69.125	0.1479	1.86683669217417	1.86683669217417\\
69.125	0.15156	2.02528892663995	2.02528892663995\\
69.125	0.15522	2.18806122347411	2.18806122347411\\
69.125	0.15888	2.35515358267665	2.35515358267665\\
69.125	0.16254	2.52656600424755	2.52656600424755\\
69.125	0.1662	2.70229848818683	2.70229848818683\\
69.125	0.16986	2.88235103449447	2.88235103449447\\
69.125	0.17352	3.06672364317049	3.06672364317049\\
69.125	0.17718	3.25541631421488	3.25541631421488\\
69.125	0.18084	3.44842904762764	3.44842904762764\\
69.125	0.1845	3.64576184340877	3.64576184340877\\
69.125	0.18816	3.84741470155828	3.84741470155828\\
69.125	0.19182	4.05338762207616	4.05338762207616\\
69.125	0.19548	4.26368060496241	4.26368060496241\\
69.125	0.19914	4.47829365021703	4.47829365021703\\
69.125	0.2028	4.69722675784002	4.69722675784002\\
69.125	0.20646	4.92047992783139	4.92047992783139\\
69.125	0.21012	5.14805316019111	5.14805316019111\\
69.125	0.21378	5.37994645491922	5.37994645491922\\
69.125	0.21744	5.61615981201571	5.61615981201571\\
69.125	0.2211	5.85669323148055	5.85669323148055\\
69.125	0.22476	6.10154671331378	6.10154671331378\\
69.125	0.22842	6.35072025751538	6.35072025751538\\
69.125	0.23208	6.60421386408533	6.60421386408533\\
69.125	0.23574	6.86202753302367	6.86202753302367\\
69.125	0.2394	7.12416126433038	7.12416126433038\\
69.125	0.24306	7.39061505800546	7.39061505800546\\
69.125	0.24672	7.66138891404891	7.66138891404891\\
69.125	0.25038	7.93648283246073	7.93648283246073\\
69.125	0.25404	8.21589681324093	8.21589681324093\\
69.125	0.2577	8.49963085638949	8.49963085638949\\
69.125	0.26136	8.78768496190643	8.78768496190643\\
69.125	0.26502	9.08005912979174	9.08005912979174\\
69.125	0.26868	9.37675336004542	9.37675336004542\\
69.125	0.27234	9.67776765266747	9.67776765266747\\
69.125	0.276	9.9831020076579	9.9831020076579\\
69.5	0.093	-0.0219591423667305	-0.0219591423667305\\
69.5	0.09666	0.0756403381259192	0.0756403381259192\\
69.5	0.10032	0.177559880986941	0.177559880986941\\
69.5	0.10398	0.283799486216337	0.283799486216337\\
69.5	0.10764	0.394359153814102	0.394359153814102\\
69.5	0.1113	0.509238883780238	0.509238883780238\\
69.5	0.11496	0.628438676114749	0.628438676114749\\
69.5	0.11862	0.751958530817628	0.751958530817628\\
69.5	0.12228	0.879798447888883	0.879798447888883\\
69.5	0.12594	1.01195842732851	1.01195842732851\\
69.5	0.1296	1.1484384691365	1.1484384691365\\
69.5	0.13326	1.28923857331287	1.28923857331287\\
69.5	0.13692	1.43435873985761	1.43435873985761\\
69.5	0.14058	1.58379896877072	1.58379896877072\\
69.5	0.14424	1.7375592600522	1.7375592600522\\
69.5	0.1479	1.89563961370206	1.89563961370206\\
69.5	0.15156	2.05804002972028	2.05804002972028\\
69.5	0.15522	2.22476050810688	2.22476050810688\\
69.5	0.15888	2.39580104886185	2.39580104886185\\
69.5	0.16254	2.57116165198518	2.57116165198518\\
69.5	0.1662	2.7508423174769	2.7508423174769\\
69.5	0.16986	2.93484304533698	2.93484304533698\\
69.5	0.17352	3.12316383556544	3.12316383556544\\
69.5	0.17718	3.31580468816227	3.31580468816227\\
69.5	0.18084	3.51276560312746	3.51276560312746\\
69.5	0.1845	3.71404658046103	3.71404658046103\\
69.5	0.18816	3.91964762016297	3.91964762016297\\
69.5	0.19182	4.12956872223329	4.12956872223329\\
69.5	0.19548	4.34380988667197	4.34380988667197\\
69.5	0.19914	4.56237111347903	4.56237111347903\\
69.5	0.2028	4.78525240265446	4.78525240265446\\
69.5	0.20646	5.01245375419826	5.01245375419826\\
69.5	0.21012	5.24397516811043	5.24397516811043\\
69.5	0.21378	5.47981664439097	5.47981664439097\\
69.5	0.21744	5.71997818303988	5.71997818303988\\
69.5	0.2211	5.96445978405717	5.96445978405717\\
69.5	0.22476	6.21326144744283	6.21326144744283\\
69.5	0.22842	6.46638317319687	6.46638317319687\\
69.5	0.23208	6.72382496131926	6.72382496131926\\
69.5	0.23574	6.98558681181003	6.98558681181003\\
69.5	0.2394	7.25166872466917	7.25166872466917\\
69.5	0.24306	7.52207069989669	7.52207069989669\\
69.5	0.24672	7.79679273749258	7.79679273749258\\
69.5	0.25038	8.07583483745684	8.07583483745684\\
69.5	0.25404	8.35919699978948	8.35919699978948\\
69.5	0.2577	8.64687922449048	8.64687922449048\\
69.5	0.26136	8.93888151155985	8.93888151155985\\
69.5	0.26502	9.23520386099759	9.23520386099759\\
69.5	0.26868	9.53584627280371	9.53584627280371\\
69.5	0.27234	9.8408087469782	9.8408087469782\\
69.5	0.276	10.1500912835211	10.1500912835211\\
69.875	0.093	-0.0534241659724104	-0.0534241659724104\\
69.875	0.09666	0.0481234960726775	0.0481234960726775\\
69.875	0.10032	0.153991220486137	0.153991220486137\\
69.875	0.10398	0.264179007267968	0.264179007267968\\
69.875	0.10764	0.378686856418168	0.378686856418168\\
69.875	0.1113	0.497514767936742	0.497514767936742\\
69.875	0.11496	0.620662741823688	0.620662741823688\\
69.875	0.11862	0.748130778079002	0.748130778079002\\
69.875	0.12228	0.879918876702695	0.879918876702695\\
69.875	0.12594	1.01602703769476	1.01602703769476\\
69.875	0.1296	1.15645526105518	1.15645526105518\\
69.875	0.13326	1.30120354678399	1.30120354678399\\
69.875	0.13692	1.45027189488117	1.45027189488117\\
69.875	0.14058	1.60366030534671	1.60366030534671\\
69.875	0.14424	1.76136877818063	1.76136877818063\\
69.875	0.1479	1.92339731338292	1.92339731338292\\
69.875	0.15156	2.08974591095358	2.08974591095358\\
69.875	0.15522	2.26041457089261	2.26041457089261\\
69.875	0.15888	2.43540329320002	2.43540329320002\\
69.875	0.16254	2.61471207787579	2.61471207787579\\
69.875	0.1662	2.79834092491994	2.79834092491994\\
69.875	0.16986	2.98628983433246	2.98628983433246\\
69.875	0.17352	3.17855880611335	3.17855880611335\\
69.875	0.17718	3.37514784026262	3.37514784026262\\
69.875	0.18084	3.57605693678026	3.57605693678026\\
69.875	0.1845	3.78128609566626	3.78128609566626\\
69.875	0.18816	3.99083531692064	3.99083531692064\\
69.875	0.19182	4.20470460054339	4.20470460054339\\
69.875	0.19548	4.42289394653451	4.42289394653451\\
69.875	0.19914	4.645403354894	4.645403354894\\
69.875	0.2028	4.87223282562187	4.87223282562187\\
69.875	0.20646	5.10338235871811	5.10338235871811\\
69.875	0.21012	5.33885195418271	5.33885195418271\\
69.875	0.21378	5.57864161201569	5.57864161201569\\
69.875	0.21744	5.82275133221704	5.82275133221704\\
69.875	0.2211	6.07118111478676	6.07118111478676\\
69.875	0.22476	6.32393095972485	6.32393095972485\\
69.875	0.22842	6.58100086703133	6.58100086703133\\
69.875	0.23208	6.84239083670616	6.84239083670616\\
69.875	0.23574	7.10810086874937	7.10810086874937\\
69.875	0.2394	7.37813096316095	7.37813096316095\\
69.875	0.24306	7.6524811199409	7.6524811199409\\
69.875	0.24672	7.93115133908923	7.93115133908923\\
69.875	0.25038	8.21414162060593	8.21414162060593\\
69.875	0.25404	8.50145196449099	8.50145196449099\\
69.875	0.2577	8.79308237074444	8.79308237074444\\
69.875	0.26136	9.08903283936624	9.08903283936624\\
69.875	0.26502	9.38930337035641	9.38930337035641\\
69.875	0.26868	9.69389396371498	9.69389396371498\\
69.875	0.27234	10.0028046194419	10.0028046194419\\
69.875	0.276	10.3160353375372	10.3160353375372\\
70.25	0.093	-0.0859344114251053	-0.0859344114251053\\
70.25	0.09666	0.0195614321724138	0.0195614321724138\\
70.25	0.10032	0.129377338138312	0.129377338138312\\
70.25	0.10398	0.243513306472577	0.243513306472577\\
70.25	0.10764	0.361969337175215	0.361969337175215\\
70.25	0.1113	0.484745430246228	0.484745430246228\\
70.25	0.11496	0.611841585685609	0.611841585685609\\
70.25	0.11862	0.743257803493361	0.743257803493361\\
70.25	0.12228	0.878994083669488	0.878994083669488\\
70.25	0.12594	1.01905042621398	1.01905042621398\\
70.25	0.1296	1.16342683112685	1.16342683112685\\
70.25	0.13326	1.31212329840809	1.31212329840809\\
70.25	0.13692	1.4651398280577	1.4651398280577\\
70.25	0.14058	1.62247642007569	1.62247642007569\\
70.25	0.14424	1.78413307446204	1.78413307446204\\
70.25	0.1479	1.95010979121677	1.95010979121677\\
70.25	0.15156	2.12040657033987	2.12040657033987\\
70.25	0.15522	2.29502341183133	2.29502341183133\\
70.25	0.15888	2.47396031569118	2.47396031569118\\
70.25	0.16254	2.65721728191939	2.65721728191939\\
70.25	0.1662	2.84479431051597	2.84479431051597\\
70.25	0.16986	3.03669140148093	3.03669140148093\\
70.25	0.17352	3.23290855481426	3.23290855481426\\
70.25	0.17718	3.43344577051596	3.43344577051596\\
70.25	0.18084	3.63830304858603	3.63830304858603\\
70.25	0.1845	3.84748038902447	3.84748038902447\\
70.25	0.18816	4.06097779183129	4.06097779183129\\
70.25	0.19182	4.27879525700647	4.27879525700647\\
70.25	0.19548	4.50093278455003	4.50093278455003\\
70.25	0.19914	4.72739037446196	4.72739037446196\\
70.25	0.2028	4.95816802674226	4.95816802674226\\
70.25	0.20646	5.19326574139094	5.19326574139094\\
70.25	0.21012	5.43268351840797	5.43268351840797\\
70.25	0.21378	5.67642135779339	5.67642135779339\\
70.25	0.21744	5.92447925954718	5.92447925954718\\
70.25	0.2211	6.17685722366934	6.17685722366934\\
70.25	0.22476	6.43355525015987	6.43355525015987\\
70.25	0.22842	6.69457333901878	6.69457333901878\\
70.25	0.23208	6.95991149024604	6.95991149024604\\
70.25	0.23574	7.22956970384169	7.22956970384169\\
70.25	0.2394	7.50354797980571	7.50354797980571\\
70.25	0.24306	7.7818463181381	7.7818463181381\\
70.25	0.24672	8.06446471883885	8.06446471883885\\
70.25	0.25038	8.35140318190798	8.35140318190798\\
70.25	0.25404	8.6426617073455	8.6426617073455\\
70.25	0.2577	8.93824029515137	8.93824029515137\\
70.25	0.26136	9.23813894532561	9.23813894532561\\
70.25	0.26502	9.54235765786824	9.54235765786824\\
70.25	0.26868	9.85089643277921	9.85089643277921\\
70.25	0.27234	10.1637552700586	10.1637552700586\\
70.25	0.276	10.4809341697063	10.4809341697063\\
70.625	0.093	-0.119489878724826	-0.119489878724826\\
70.625	0.09666	-0.0100458535748684	-0.0100458535748684\\
70.625	0.10032	0.103718233943464	0.103718233943464\\
70.625	0.10398	0.221802383830168	0.221802383830168\\
70.625	0.10764	0.344206596085241	0.344206596085241\\
70.625	0.1113	0.470930870708688	0.470930870708688\\
70.625	0.11496	0.601975207700507	0.601975207700507\\
70.625	0.11862	0.737339607060694	0.737339607060694\\
70.625	0.12228	0.877024068789256	0.877024068789256\\
70.625	0.12594	1.02102859288619	1.02102859288619\\
70.625	0.1296	1.1693531793515	1.1693531793515\\
70.625	0.13326	1.32199782818517	1.32199782818517\\
70.625	0.13692	1.47896253938722	1.47896253938722\\
70.625	0.14058	1.64024731295764	1.64024731295764\\
70.625	0.14424	1.80585214889643	1.80585214889643\\
70.625	0.1479	1.97577704720359	1.97577704720359\\
70.625	0.15156	2.15002200787913	2.15002200787913\\
70.625	0.15522	2.32858703092303	2.32858703092303\\
70.625	0.15888	2.51147211633531	2.51147211633531\\
70.625	0.16254	2.69867726411596	2.69867726411596\\
70.625	0.1662	2.89020247426498	2.89020247426498\\
70.625	0.16986	3.08604774678237	3.08604774678237\\
70.625	0.17352	3.28621308166814	3.28621308166814\\
70.625	0.17718	3.49069847892227	3.49069847892227\\
70.625	0.18084	3.69950393854478	3.69950393854478\\
70.625	0.1845	3.91262946053565	3.91262946053565\\
70.625	0.18816	4.13007504489491	4.13007504489491\\
70.625	0.19182	4.35184069162253	4.35184069162253\\
70.625	0.19548	4.57792640071853	4.57792640071853\\
70.625	0.19914	4.80833217218289	4.80833217218289\\
70.625	0.2028	5.04305800601563	5.04305800601563\\
70.625	0.20646	5.28210390221674	5.28210390221674\\
70.625	0.21012	5.52546986078622	5.52546986078622\\
70.625	0.21378	5.77315588172407	5.77315588172407\\
70.625	0.21744	6.02516196503029	6.02516196503029\\
70.625	0.2211	6.28148811070489	6.28148811070489\\
70.625	0.22476	6.54213431874786	6.54213431874786\\
70.625	0.22842	6.8071005891592	6.8071005891592\\
70.625	0.23208	7.0763869219389	7.0763869219389\\
70.625	0.23574	7.34999331708699	7.34999331708699\\
70.625	0.2394	7.62791977460344	7.62791977460344\\
70.625	0.24306	7.91016629448827	7.91016629448827\\
70.625	0.24672	8.19673287674146	8.19673287674146\\
70.625	0.25038	8.48761952136303	8.48761952136303\\
70.625	0.25404	8.78282622835297	8.78282622835297\\
70.625	0.2577	9.08235299771129	9.08235299771129\\
70.625	0.26136	9.38619982943796	9.38619982943796\\
70.625	0.26502	9.69436672353302	9.69436672353302\\
70.625	0.26868	10.0068536799964	10.0068536799964\\
70.625	0.27234	10.3236606988282	10.3236606988282\\
70.625	0.276	10.6447877800284	10.6447877800284\\
71	0.093	-0.154090567871561	-0.154090567871561\\
71	0.09666	-0.0406983611691656	-0.0406983611691656\\
71	0.10032	0.0770139079016019	0.0770139079016019\\
71	0.10398	0.199046239340744	0.199046239340744\\
71	0.10764	0.325398633148251	0.325398633148251\\
71	0.1113	0.456071089324137	0.456071089324137\\
71	0.11496	0.59106360786839	0.59106360786839\\
71	0.11862	0.730376188781015	0.730376188781015\\
71	0.12228	0.874008832062013	0.874008832062013\\
71	0.12594	1.02196153771138	1.02196153771138\\
71	0.1296	1.17423430572912	1.17423430572912\\
71	0.13326	1.33082713611524	1.33082713611524\\
71	0.13692	1.49174002886972	1.49174002886972\\
71	0.14058	1.65697298399258	1.65697298399258\\
71	0.14424	1.8265260014838	1.8265260014838\\
71	0.1479	2.0003990813434	2.0003990813434\\
71	0.15156	2.17859222357138	2.17859222357138\\
71	0.15522	2.36110542816771	2.36110542816771\\
71	0.15888	2.54793869513244	2.54793869513244\\
71	0.16254	2.73909202446551	2.73909202446551\\
71	0.1662	2.93456541616697	2.93456541616697\\
71	0.16986	3.13435887023681	3.13435887023681\\
71	0.17352	3.338472386675	3.338472386675\\
71	0.17718	3.54690596548158	3.54690596548158\\
71	0.18084	3.75965960665652	3.75965960665652\\
71	0.1845	3.97673331019983	3.97673331019983\\
71	0.18816	4.19812707611152	4.19812707611152\\
71	0.19182	4.42384090439158	4.42384090439158\\
71	0.19548	4.65387479504001	4.65387479504001\\
71	0.19914	4.88822874805681	4.88822874805681\\
71	0.2028	5.12690276344199	5.12690276344199\\
71	0.20646	5.36989684119553	5.36989684119553\\
71	0.21012	5.61721098131744	5.61721098131744\\
71	0.21378	5.86884518380774	5.86884518380774\\
71	0.21744	6.12479944866639	6.12479944866639\\
71	0.2211	6.38507377589342	6.38507377589342\\
71	0.22476	6.64966816548883	6.64966816548883\\
71	0.22842	6.91858261745261	6.91858261745261\\
71	0.23208	7.19181713178475	7.19181713178475\\
71	0.23574	7.46937170848527	7.46937170848527\\
71	0.2394	7.75124634755415	7.75124634755415\\
71	0.24306	8.03744104899142	8.03744104899142\\
71	0.24672	8.32795581279705	8.32795581279705\\
71	0.25038	8.62279063897106	8.62279063897106\\
71	0.25404	8.92194552751343	8.92194552751343\\
71	0.2577	9.22542047842419	9.22542047842419\\
71	0.26136	9.5332154917033	9.5332154917033\\
71	0.26502	9.84533056735079	9.84533056735079\\
71	0.26868	10.1617657053666	10.1617657053666\\
71	0.27234	10.4825209057509	10.4825209057509\\
71	0.276	10.8075961685035	10.8075961685035\\
71.375	0.093	-0.189736478865319	-0.189736478865319\\
71.375	0.09666	-0.0723960906104884	-0.0723960906104884\\
71.375	0.10032	0.0492643600127174	0.0492643600127174\\
71.375	0.10398	0.175244873004294	0.175244873004294\\
71.375	0.10764	0.305545448364239	0.305545448364239\\
71.375	0.1113	0.44016608609256	0.44016608609256\\
71.375	0.11496	0.579106786189252	0.579106786189252\\
71.375	0.11862	0.722367548654312	0.722367548654312\\
71.375	0.12228	0.869948373487747	0.869948373487747\\
71.375	0.12594	1.02184926068955	1.02184926068955\\
71.375	0.1296	1.17807021025973	1.17807021025973\\
71.375	0.13326	1.33861122219828	1.33861122219828\\
71.375	0.13692	1.5034722965052	1.5034722965052\\
71.375	0.14058	1.67265343318049	1.67265343318049\\
71.375	0.14424	1.84615463222416	1.84615463222416\\
71.375	0.1479	2.02397589363619	2.02397589363619\\
71.375	0.15156	2.2061172174166	2.2061172174166\\
71.375	0.15522	2.39257860356538	2.39257860356538\\
71.375	0.15888	2.58336005208254	2.58336005208254\\
71.375	0.16254	2.77846156296805	2.77846156296805\\
71.375	0.1662	2.97788313622195	2.97788313622195\\
71.375	0.16986	3.18162477184421	3.18162477184421\\
71.375	0.17352	3.38968646983485	3.38968646983485\\
71.375	0.17718	3.60206823019386	3.60206823019386\\
71.375	0.18084	3.81877005292124	3.81877005292124\\
71.375	0.1845	4.03979193801699	4.03979193801699\\
71.375	0.18816	4.26513388548111	4.26513388548111\\
71.375	0.19182	4.49479589531361	4.49479589531361\\
71.375	0.19548	4.72877796751447	4.72877796751447\\
71.375	0.19914	4.96708010208371	4.96708010208371\\
71.375	0.2028	5.20970229902132	5.20970229902132\\
71.375	0.20646	5.4566445583273	5.4566445583273\\
71.375	0.21012	5.70790688000166	5.70790688000166\\
71.375	0.21378	5.96348926404438	5.96348926404438\\
71.375	0.21744	6.22339171045548	6.22339171045548\\
71.375	0.2211	6.48761421923495	6.48761421923495\\
71.375	0.22476	6.75615679038279	6.75615679038279\\
71.375	0.22842	7.029019423899	7.029019423899\\
71.375	0.23208	7.30620211978357	7.30620211978357\\
71.375	0.23574	7.58770487803653	7.58770487803653\\
71.375	0.2394	7.87352769865786	7.87352769865786\\
71.375	0.24306	8.16367058164756	8.16367058164756\\
71.375	0.24672	8.45813352700562	8.45813352700562\\
71.375	0.25038	8.75691653473206	8.75691653473206\\
71.375	0.25404	9.06001960482688	9.06001960482688\\
71.375	0.2577	9.36744273729006	9.36744273729006\\
71.375	0.26136	9.67918593212162	9.67918593212162\\
71.375	0.26502	9.99524918932155	9.99524918932155\\
71.375	0.26868	10.3156325088898	10.3156325088898\\
71.375	0.27234	10.6403358908265	10.6403358908265\\
71.375	0.276	10.9693593351316	10.9693593351316\\
71.75	0.093	-0.226427611706095	-0.226427611706095\\
71.75	0.09666	-0.10513904189883	-0.10513904189883\\
71.75	0.10032	0.0204695902768108	0.0204695902768108\\
71.75	0.10398	0.150398284820826	0.150398284820826\\
71.75	0.10764	0.284647041733206	0.284647041733206\\
71.75	0.1113	0.423215861013965	0.423215861013965\\
71.75	0.11496	0.566104742663091	0.566104742663091\\
71.75	0.11862	0.713313686680589	0.713313686680589\\
71.75	0.12228	0.864842693066459	0.864842693066459\\
71.75	0.12594	1.0206917618207	1.0206917618207\\
71.75	0.1296	1.18086089294331	1.18086089294331\\
71.75	0.13326	1.3453500864343	1.3453500864343\\
71.75	0.13692	1.51415934229366	1.51415934229366\\
71.75	0.14058	1.68728866052139	1.68728866052139\\
71.75	0.14424	1.86473804111749	1.86473804111749\\
71.75	0.1479	2.04650748408196	2.04650748408196\\
71.75	0.15156	2.2325969894148	2.2325969894148\\
71.75	0.15522	2.42300655711601	2.42300655711601\\
71.75	0.15888	2.61773618718561	2.61773618718561\\
71.75	0.16254	2.81678587962356	2.81678587962356\\
71.75	0.1662	3.02015563442989	3.02015563442989\\
71.75	0.16986	3.22784545160459	3.22784545160459\\
71.75	0.17352	3.43985533114767	3.43985533114767\\
71.75	0.17718	3.65618527305912	3.65618527305912\\
71.75	0.18084	3.87683527733893	3.87683527733893\\
71.75	0.1845	4.10180534398711	4.10180534398711\\
71.75	0.18816	4.33109547300368	4.33109547300368\\
71.75	0.19182	4.5647056643886	4.5647056643886\\
71.75	0.19548	4.80263591814191	4.80263591814191\\
71.75	0.19914	5.04488623426359	5.04488623426359\\
71.75	0.2028	5.29145661275363	5.29145661275363\\
71.75	0.20646	5.54234705361205	5.54234705361205\\
71.75	0.21012	5.79755755683884	5.79755755683884\\
71.75	0.21378	6.057088122434	6.057088122434\\
71.75	0.21744	6.32093875039753	6.32093875039753\\
71.75	0.2211	6.58910944072943	6.58910944072943\\
71.75	0.22476	6.86160019342971	6.86160019342971\\
71.75	0.22842	7.13841100849836	7.13841100849836\\
71.75	0.23208	7.41954188593537	7.41954188593537\\
71.75	0.23574	7.70499282574077	7.70499282574077\\
71.75	0.2394	7.99476382791453	7.99476382791453\\
71.75	0.24306	8.28885489245666	8.28885489245666\\
71.75	0.24672	8.58726601936717	8.58726601936717\\
71.75	0.25038	8.88999720864605	8.88999720864605\\
71.75	0.25404	9.1970484602933	9.1970484602933\\
71.75	0.2577	9.50841977430892	9.50841977430892\\
71.75	0.26136	9.82411115069291	9.82411115069291\\
71.75	0.26502	10.1441225894453	10.1441225894453\\
71.75	0.26868	10.468454090566	10.468454090566\\
71.75	0.27234	10.7971056540551	10.7971056540551\\
71.75	0.276	11.1300772799126	11.1300772799126\\
72.125	0.093	-0.264163966393896	-0.264163966393896\\
72.125	0.09666	-0.138927215034193	-0.138927215034193\\
72.125	0.10032	-0.00937040130611422	-0.00937040130611422\\
72.125	0.10398	0.124506474790332	0.124506474790332\\
72.125	0.10764	0.26270341325515	0.26270341325515\\
72.125	0.1113	0.405220414088344	0.405220414088344\\
72.125	0.11496	0.552057477289909	0.552057477289909\\
72.125	0.11862	0.703214602859841	0.703214602859841\\
72.125	0.12228	0.85869179079815	0.85869179079815\\
72.125	0.12594	1.01848904110483	1.01848904110483\\
72.125	0.1296	1.18260635377988	1.18260635377988\\
72.125	0.13326	1.3510437288233	1.3510437288233\\
72.125	0.13692	1.52380116623509	1.52380116623509\\
72.125	0.14058	1.70087866601526	1.70087866601526\\
72.125	0.14424	1.88227622816379	1.88227622816379\\
72.125	0.1479	2.0679938526807	2.0679938526807\\
72.125	0.15156	2.25803153956598	2.25803153956598\\
72.125	0.15522	2.45238928881964	2.45238928881964\\
72.125	0.15888	2.65106710044166	2.65106710044166\\
72.125	0.16254	2.85406497443205	2.85406497443205\\
72.125	0.1662	3.06138291079082	3.06138291079082\\
72.125	0.16986	3.27302090951795	3.27302090951795\\
72.125	0.17352	3.48897897061347	3.48897897061347\\
72.125	0.17718	3.70925709407735	3.70925709407735\\
72.125	0.18084	3.9338552799096	3.9338552799096\\
72.125	0.1845	4.16277352811022	4.16277352811022\\
72.125	0.18816	4.39601183867922	4.39601183867922\\
72.125	0.19182	4.63357021161658	4.63357021161658\\
72.125	0.19548	4.87544864692233	4.87544864692233\\
72.125	0.19914	5.12164714459644	5.12164714459644\\
72.125	0.2028	5.37216570463892	5.37216570463892\\
72.125	0.20646	5.62700432704978	5.62700432704978\\
72.125	0.21012	5.88616301182899	5.88616301182899\\
72.125	0.21378	6.1496417589766	6.1496417589766\\
72.125	0.21744	6.41744056849256	6.41744056849256\\
72.125	0.2211	6.68955944037691	6.68955944037691\\
72.125	0.22476	6.96599837462962	6.96599837462962\\
72.125	0.22842	7.24675737125071	7.24675737125071\\
72.125	0.23208	7.53183643024015	7.53183643024015\\
72.125	0.23574	7.82123555159798	7.82123555159798\\
72.125	0.2394	8.11495473532418	8.11495473532418\\
72.125	0.24306	8.41299398141875	8.41299398141875\\
72.125	0.24672	8.7153532898817	8.7153532898817\\
72.125	0.25038	9.02203266071301	9.02203266071301\\
72.125	0.25404	9.3330320939127	9.3330320939127\\
72.125	0.2577	9.64835158948075	9.64835158948075\\
72.125	0.26136	9.96799114741718	9.96799114741718\\
72.125	0.26502	10.291950767722	10.291950767722\\
72.125	0.26868	10.6202304503952	10.6202304503952\\
72.125	0.27234	10.9528301954367	10.9528301954367\\
72.125	0.276	11.2897500028466	11.2897500028466\\
72.5	0.093	-0.302945542928716	-0.302945542928716\\
72.5	0.09666	-0.173760610016578	-0.173760610016578\\
72.5	0.10032	-0.0402556147360613	-0.0402556147360613\\
72.5	0.10398	0.0975694429128231	0.0975694429128231\\
72.5	0.10764	0.23971456293008	0.23971456293008\\
72.5	0.1113	0.386179745315708	0.386179745315708\\
72.5	0.11496	0.536964990069707	0.536964990069707\\
72.5	0.11862	0.692070297192078	0.692070297192078\\
72.5	0.12228	0.851495666682821	0.851495666682821\\
72.5	0.12594	1.01524109854194	1.01524109854194\\
72.5	0.1296	1.18330659276942	1.18330659276942\\
72.5	0.13326	1.35569214936528	1.35569214936528\\
72.5	0.13692	1.53239776832951	1.53239776832951\\
72.5	0.14058	1.71342344966211	1.71342344966211\\
72.5	0.14424	1.89876919336308	1.89876919336308\\
72.5	0.1479	2.08843499943243	2.08843499943243\\
72.5	0.15156	2.28242086787015	2.28242086787015\\
72.5	0.15522	2.48072679867623	2.48072679867623\\
72.5	0.15888	2.68335279185069	2.68335279185069\\
72.5	0.16254	2.89029884739352	2.89029884739352\\
72.5	0.1662	3.10156496530473	3.10156496530473\\
72.5	0.16986	3.3171511455843	3.3171511455843\\
72.5	0.17352	3.53705738823225	3.53705738823225\\
72.5	0.17718	3.76128369324856	3.76128369324856\\
72.5	0.18084	3.98983006063325	3.98983006063325\\
72.5	0.1845	4.22269649038631	4.22269649038631\\
72.5	0.18816	4.45988298250774	4.45988298250774\\
72.5	0.19182	4.70138953699755	4.70138953699755\\
72.5	0.19548	4.94721615385572	4.94721615385572\\
72.5	0.19914	5.19736283308227	5.19736283308227\\
72.5	0.2028	5.45182957467719	5.45182957467719\\
72.5	0.20646	5.71061637864049	5.71061637864049\\
72.5	0.21012	5.97372324497214	5.97372324497214\\
72.5	0.21378	6.24115017367217	6.24115017367217\\
72.5	0.21744	6.51289716474058	6.51289716474058\\
72.5	0.2211	6.78896421817736	6.78896421817736\\
72.5	0.22476	7.06935133398251	7.06935133398251\\
72.5	0.22842	7.35405851215603	7.35405851215603\\
72.5	0.23208	7.64308575269791	7.64308575269791\\
72.5	0.23574	7.93643305560818	7.93643305560818\\
72.5	0.2394	8.23410042088681	8.23410042088681\\
72.5	0.24306	8.53608784853382	8.53608784853382\\
72.5	0.24672	8.8423953385492	8.8423953385492\\
72.5	0.25038	9.15302289093296	9.15302289093296\\
72.5	0.25404	9.46797050568508	9.46797050568508\\
72.5	0.2577	9.78723818280557	9.78723818280557\\
72.5	0.26136	10.1108259222944	10.1108259222944\\
72.5	0.26502	10.4387337241517	10.4387337241517\\
72.5	0.26868	10.7709615883773	10.7709615883773\\
72.5	0.27234	11.1075095149713	11.1075095149713\\
72.5	0.276	11.4483775039336	11.4483775039336\\
72.875	0.093	-0.342772341310558	-0.342772341310558\\
72.875	0.09666	-0.209639226845982	-0.209639226845982\\
72.875	0.10032	-0.0721860500130305	-0.0721860500130305\\
72.875	0.10398	0.0695871891882922	0.0695871891882922\\
72.875	0.10764	0.215680490757983	0.215680490757983\\
72.875	0.1113	0.366093854696046	0.366093854696046\\
72.875	0.11496	0.520827281002484	0.520827281002484\\
72.875	0.11862	0.67988076967729	0.67988076967729\\
72.875	0.12228	0.843254320720471	0.843254320720471\\
72.875	0.12594	1.01094793413202	1.01094793413202\\
72.875	0.1296	1.18296160991194	1.18296160991194\\
72.875	0.13326	1.35929534806024	1.35929534806024\\
72.875	0.13692	1.5399491485769	1.5399491485769\\
72.875	0.14058	1.72492301146194	1.72492301146194\\
72.875	0.14424	1.91421693671535	1.91421693671535\\
72.875	0.1479	2.10783092433713	2.10783092433713\\
72.875	0.15156	2.30576497432728	2.30576497432728\\
72.875	0.15522	2.50801908668581	2.50801908668581\\
72.875	0.15888	2.7145932614127	2.7145932614127\\
72.875	0.16254	2.92548749850797	2.92548749850797\\
72.875	0.1662	3.14070179797161	3.14070179797161\\
72.875	0.16986	3.36023615980362	3.36023615980362\\
72.875	0.17352	3.584090584004	3.584090584004\\
72.875	0.17718	3.81226507057276	3.81226507057276\\
72.875	0.18084	4.04475961950988	4.04475961950988\\
72.875	0.1845	4.28157423081538	4.28157423081538\\
72.875	0.18816	4.52270890448925	4.52270890448925\\
72.875	0.19182	4.76816364053149	4.76816364053149\\
72.875	0.19548	5.0179384389421	5.0179384389421\\
72.875	0.19914	5.27203329972109	5.27203329972109\\
72.875	0.2028	5.53044822286844	5.53044822286844\\
72.875	0.20646	5.79318320838417	5.79318320838417\\
72.875	0.21012	6.06023825626826	6.06023825626826\\
72.875	0.21378	6.33161336652073	6.33161336652073\\
72.875	0.21744	6.60730853914157	6.60730853914157\\
72.875	0.2211	6.88732377413079	6.88732377413079\\
72.875	0.22476	7.17165907148837	7.17165907148837\\
72.875	0.22842	7.46031443121433	7.46031443121433\\
72.875	0.23208	7.75328985330865	7.75328985330865\\
72.875	0.23574	8.05058533777135	8.05058533777135\\
72.875	0.2394	8.35220088460242	8.35220088460242\\
72.875	0.24306	8.65813649380186	8.65813649380186\\
72.875	0.24672	8.96839216536969	8.96839216536969\\
72.875	0.25038	9.28296789930588	9.28296789930588\\
72.875	0.25404	9.60186369561043	9.60186369561043\\
72.875	0.2577	9.92507955428337	9.92507955428337\\
72.875	0.26136	10.2526154753247	10.2526154753247\\
72.875	0.26502	10.5844714587343	10.5844714587343\\
72.875	0.26868	10.9206475045124	10.9206475045124\\
72.875	0.27234	11.2611436126588	11.2611436126588\\
72.875	0.276	11.6059597831736	11.6059597831736\\
73.25	0.093	-0.383644361539415	-0.383644361539415\\
73.25	0.09666	-0.246563065522404	-0.246563065522404\\
73.25	0.10032	-0.105161707137018	-0.105161707137018\\
73.25	0.10398	0.0405597136167393	0.0405597136167393\\
73.25	0.10764	0.190601196738869	0.190601196738869\\
73.25	0.1113	0.34496274222937	0.34496274222937\\
73.25	0.11496	0.503644350088242	0.503644350088242\\
73.25	0.11862	0.666646020315486	0.666646020315486\\
73.25	0.12228	0.833967752911102	0.833967752911102\\
73.25	0.12594	1.00560954787509	1.00560954787509\\
73.25	0.1296	1.18157140520744	1.18157140520744\\
73.25	0.13326	1.36185332490818	1.36185332490818\\
73.25	0.13692	1.54645530697728	1.54645530697728\\
73.25	0.14058	1.73537735141476	1.73537735141476\\
73.25	0.14424	1.9286194582206	1.9286194582206\\
73.25	0.1479	2.12618162739482	2.12618162739482\\
73.25	0.15156	2.32806385893741	2.32806385893741\\
73.25	0.15522	2.53426615284836	2.53426615284836\\
73.25	0.15888	2.7447885091277	2.7447885091277\\
73.25	0.16254	2.9596309277754	2.9596309277754\\
73.25	0.1662	3.17879340879148	3.17879340879148\\
73.25	0.16986	3.40227595217593	3.40227595217593\\
73.25	0.17352	3.63007855792874	3.63007855792874\\
73.25	0.17718	3.86220122604993	3.86220122604993\\
73.25	0.18084	4.0986439565395	4.0986439565395\\
73.25	0.1845	4.33940674939742	4.33940674939742\\
73.25	0.18816	4.58448960462373	4.58448960462373\\
73.25	0.19182	4.8338925222184	4.8338925222184\\
73.25	0.19548	5.08761550218146	5.08761550218146\\
73.25	0.19914	5.34565854451288	5.34565854451288\\
73.25	0.2028	5.60802164921267	5.60802164921267\\
73.25	0.20646	5.87470481628084	5.87470481628084\\
73.25	0.21012	6.14570804571737	6.14570804571737\\
73.25	0.21378	6.42103133752227	6.42103133752227\\
73.25	0.21744	6.70067469169555	6.70067469169555\\
73.25	0.2211	6.9846381082372	6.9846381082372\\
73.25	0.22476	7.27292158714722	7.27292158714722\\
73.25	0.22842	7.56552512842562	7.56552512842562\\
73.25	0.23208	7.86244873207237	7.86244873207237\\
73.25	0.23574	8.16369239808751	8.16369239808751\\
73.25	0.2394	8.46925612647102	8.46925612647102\\
73.25	0.24306	8.7791399172229	8.7791399172229\\
73.25	0.24672	9.09334377034315	9.09334377034315\\
73.25	0.25038	9.41186768583177	9.41186768583177\\
73.25	0.25404	9.73471166368877	9.73471166368877\\
73.25	0.2577	10.0618757039141	10.0618757039141\\
73.25	0.26136	10.3933598065079	10.3933598065079\\
73.25	0.26502	10.72916397147	10.72916397147\\
73.25	0.26868	11.0692881988005	11.0692881988005\\
73.25	0.27234	11.4137324884993	11.4137324884993\\
73.25	0.276	11.7624968405665	11.7624968405665\\
73.625	0.093	-0.425561603615298	-0.425561603615298\\
73.625	0.09666	-0.284532126045849	-0.284532126045849\\
73.625	0.10032	-0.139182586108028	-0.139182586108028\\
73.625	0.10398	0.0104870161981678	0.0104870161981678\\
73.625	0.10764	0.164476680872732	0.164476680872732\\
73.625	0.1113	0.322786407915668	0.322786407915668\\
73.625	0.11496	0.485416197326979	0.485416197326979\\
73.625	0.11862	0.652366049106657	0.652366049106657\\
73.625	0.12228	0.823635963254711	0.823635963254711\\
73.625	0.12594	0.999225939771137	0.999225939771137\\
73.625	0.1296	1.17913597865593	1.17913597865593\\
73.625	0.13326	1.3633660799091	1.3633660799091\\
73.625	0.13692	1.55191624353064	1.55191624353064\\
73.625	0.14058	1.74478646952055	1.74478646952055\\
73.625	0.14424	1.94197675787883	1.94197675787883\\
73.625	0.1479	2.14348710860548	2.14348710860548\\
73.625	0.15156	2.3493175217005	2.3493175217005\\
73.625	0.15522	2.5594679971639	2.5594679971639\\
73.625	0.15888	2.77393853499567	2.77393853499567\\
73.625	0.16254	2.99272913519581	2.99272913519581\\
73.625	0.1662	3.21583979776432	3.21583979776432\\
73.625	0.16986	3.4432705227012	3.4432705227012\\
73.625	0.17352	3.67502131000646	3.67502131000646\\
73.625	0.17718	3.91109215968009	3.91109215968009\\
73.625	0.18084	4.15148307172208	4.15148307172208\\
73.625	0.1845	4.39619404613245	4.39619404613245\\
73.625	0.18816	4.64522508291119	4.64522508291119\\
73.625	0.19182	4.8985761820583	4.8985761820583\\
73.625	0.19548	5.15624734357379	5.15624734357379\\
73.625	0.19914	5.41823856745765	5.41823856745765\\
73.625	0.2028	5.68454985370988	5.68454985370988\\
73.625	0.20646	5.95518120233048	5.95518120233048\\
73.625	0.21012	6.23013261331944	6.23013261331944\\
73.625	0.21378	6.50940408667678	6.50940408667678\\
73.625	0.21744	6.7929956224025	6.7929956224025\\
73.625	0.2211	7.08090722049659	7.08090722049659\\
73.625	0.22476	7.37313888095904	7.37313888095904\\
73.625	0.22842	7.66969060378988	7.66969060378988\\
73.625	0.23208	7.97056238898907	7.97056238898907\\
73.625	0.23574	8.27575423655664	8.27575423655664\\
73.625	0.2394	8.58526614649259	8.58526614649259\\
73.625	0.24306	8.89909811879691	8.89909811879691\\
73.625	0.24672	9.2172501534696	9.2172501534696\\
73.625	0.25038	9.53972225051066	9.53972225051066\\
73.625	0.25404	9.86651440992009	9.86651440992009\\
73.625	0.2577	10.1976266316979	10.1976266316979\\
73.625	0.26136	10.5330589158441	10.5330589158441\\
73.625	0.26502	10.8728112623586	10.8728112623586\\
73.625	0.26868	11.2168836712415	11.2168836712415\\
73.625	0.27234	11.5652761424928	11.5652761424928\\
73.625	0.276	11.9179886761125	11.9179886761125\\
74	0.093	-0.468524067538196	-0.468524067538196\\
74	0.09666	-0.323546408416312	-0.323546408416312\\
74	0.10032	-0.174248686926052	-0.174248686926052\\
74	0.10398	-0.0206309030674185	-0.0206309030674185\\
74	0.10764	0.13730694315958	0.13730694315958\\
74	0.1113	0.299564851754955	0.299564851754955\\
74	0.11496	0.4661428227187	0.4661428227187\\
74	0.11862	0.637040856050813	0.637040856050813\\
74	0.12228	0.812258951751306	0.812258951751306\\
74	0.12594	0.991797109820166	0.991797109820166\\
74	0.1296	1.17565533025739	1.17565533025739\\
74	0.13326	1.363833613063	1.363833613063\\
74	0.13692	1.55633195823698	1.55633195823698\\
74	0.14058	1.75315036577932	1.75315036577932\\
74	0.14424	1.95428883569004	1.95428883569004\\
74	0.1479	2.15974736796913	2.15974736796913\\
74	0.15156	2.36952596261659	2.36952596261659\\
74	0.15522	2.58362461963242	2.58362461963242\\
74	0.15888	2.80204333901663	2.80204333901663\\
74	0.16254	3.0247821207692	3.0247821207692\\
74	0.1662	3.25184096489015	3.25184096489015\\
74	0.16986	3.48321987137947	3.48321987137947\\
74	0.17352	3.71891884023716	3.71891884023716\\
74	0.17718	3.95893787146322	3.95893787146322\\
74	0.18084	4.20327696505766	4.20327696505766\\
74	0.1845	4.45193612102047	4.45193612102047\\
74	0.18816	4.70491533935165	4.70491533935165\\
74	0.19182	4.96221462005119	4.96221462005119\\
74	0.19548	5.22383396311912	5.22383396311912\\
74	0.19914	5.48977336855541	5.48977336855541\\
74	0.2028	5.76003283636007	5.76003283636007\\
74	0.20646	6.03461236653311	6.03461236653311\\
74	0.21012	6.31351195907451	6.31351195907451\\
74	0.21378	6.59673161398429	6.59673161398429\\
74	0.21744	6.88427133126244	6.88427133126244\\
74	0.2211	7.17613111090897	7.17613111090897\\
74	0.22476	7.47231095292386	7.47231095292386\\
74	0.22842	7.77281085730713	7.77281085730713\\
74	0.23208	8.07763082405876	8.07763082405876\\
74	0.23574	8.38677085317877	8.38677085317877\\
74	0.2394	8.70023094466715	8.70023094466715\\
74	0.24306	9.0180110985239	9.0180110985239\\
74	0.24672	9.34011131474903	9.34011131474903\\
74	0.25038	9.66653159334252	9.66653159334252\\
74	0.25404	9.99727193430439	9.99727193430439\\
74	0.2577	10.3323323376346	10.3323323376346\\
74	0.26136	10.6717128033332	10.6717128033332\\
74	0.26502	11.0154133314002	11.0154133314002\\
74	0.26868	11.3634339218356	11.3634339218356\\
74	0.27234	11.7157745746393	11.7157745746393\\
74	0.276	12.0724352898114	12.0724352898114\\
};
\end{axis}

\begin{axis}[%
width=6.159cm,
height=3.097cm,
at={(8.104cm,8.602cm)},
scale only axis,
xmin=56,
xmax=74,
tick align=outside,
xlabel style={font=\color{white!15!black}},
xlabel={$L_{cut}$},
ymin=0.093,
ymax=0.276,
ylabel style={font=\color{white!15!black}},
ylabel={$D_{rlx}$},
zmin=-1.52969257267885,
zmax=2.51190702643201,
zlabel style={font=\color{white!15!black}},
zlabel={$u(t-3)u(t-3)$},
view={-140}{50},
axis background/.style={fill=white},
xmajorgrids,
ymajorgrids,
zmajorgrids
]
\addplot3[only marks, mark=*, mark options={}, mark size=1.5000pt, color=mycolor1, fill=mycolor1] table[row sep=crcr]{%
x	y	z\\
74	0.123	0.00740370360905991\\
72	0.113	-0.186626587578763\\
61	0.095	-0.150607769785267\\
56	0.093	-0.216975707230133\\
};
\addplot3[only marks, mark=*, mark options={}, mark size=1.5000pt, color=mycolor2, fill=mycolor2] table[row sep=crcr]{%
x	y	z\\
67	0.276	0.0527524251509314\\
66	0.255	-0.475927501616939\\
62	0.209	0.435130634728223\\
57	0.193	0.76117565090573\\
};
\addplot3[only marks, mark=*, mark options={}, mark size=1.5000pt, color=black, fill=black] table[row sep=crcr]{%
x	y	z\\
69	0.104	-0.0638886768595199\\
};
\addplot3[only marks, mark=*, mark options={}, mark size=1.5000pt, color=black, fill=black] table[row sep=crcr]{%
x	y	z\\
64	0.23	0.0838527726713747\\
};

\addplot3[%
surf,
fill opacity=0.7, shader=interp, colormap={mymap}{[1pt] rgb(0pt)=(1,0.905882,0); rgb(1pt)=(1,0.901964,0); rgb(2pt)=(1,0.898051,0); rgb(3pt)=(1,0.894144,0); rgb(4pt)=(1,0.890243,0); rgb(5pt)=(1,0.886349,0); rgb(6pt)=(1,0.88246,0); rgb(7pt)=(1,0.878577,0); rgb(8pt)=(1,0.8747,0); rgb(9pt)=(1,0.870829,0); rgb(10pt)=(1,0.866964,0); rgb(11pt)=(1,0.863106,0); rgb(12pt)=(1,0.859253,0); rgb(13pt)=(1,0.855406,0); rgb(14pt)=(1,0.851566,0); rgb(15pt)=(1,0.847732,0); rgb(16pt)=(1,0.843903,0); rgb(17pt)=(1,0.840081,0); rgb(18pt)=(1,0.836265,0); rgb(19pt)=(1,0.832455,0); rgb(20pt)=(1,0.828652,0); rgb(21pt)=(1,0.824854,0); rgb(22pt)=(1,0.821063,0); rgb(23pt)=(1,0.817278,0); rgb(24pt)=(1,0.8135,0); rgb(25pt)=(1,0.809727,0); rgb(26pt)=(1,0.805961,0); rgb(27pt)=(1,0.8022,0); rgb(28pt)=(1,0.798445,0); rgb(29pt)=(1,0.794696,0); rgb(30pt)=(1,0.790953,0); rgb(31pt)=(1,0.787215,0); rgb(32pt)=(1,0.783484,0); rgb(33pt)=(1,0.779758,0); rgb(34pt)=(1,0.776038,0); rgb(35pt)=(1,0.772324,0); rgb(36pt)=(1,0.768615,0); rgb(37pt)=(1,0.764913,0); rgb(38pt)=(1,0.761217,0); rgb(39pt)=(1,0.757527,0); rgb(40pt)=(1,0.753843,0); rgb(41pt)=(1,0.750165,0); rgb(42pt)=(1,0.746493,0); rgb(43pt)=(1,0.742827,0); rgb(44pt)=(1,0.739167,0); rgb(45pt)=(1,0.735514,0); rgb(46pt)=(1,0.731867,0); rgb(47pt)=(1,0.728226,0); rgb(48pt)=(1,0.724591,0); rgb(49pt)=(1,0.720963,0); rgb(50pt)=(1,0.717341,0); rgb(51pt)=(1,0.713725,0); rgb(52pt)=(0.999994,0.710077,0); rgb(53pt)=(0.999974,0.706363,0); rgb(54pt)=(0.999942,0.702592,0); rgb(55pt)=(0.999898,0.698775,0); rgb(56pt)=(0.999841,0.694921,0); rgb(57pt)=(0.999771,0.691039,0); rgb(58pt)=(0.99969,0.687139,0); rgb(59pt)=(0.999596,0.68323,0); rgb(60pt)=(0.99949,0.679323,0); rgb(61pt)=(0.999372,0.675427,0); rgb(62pt)=(0.999242,0.67155,0); rgb(63pt)=(0.9991,0.667704,0); rgb(64pt)=(0.998946,0.663897,0); rgb(65pt)=(0.998781,0.660138,0); rgb(66pt)=(0.998605,0.656439,0); rgb(67pt)=(0.998416,0.652807,0); rgb(68pt)=(0.998217,0.649253,0); rgb(69pt)=(0.998006,0.645786,0); rgb(70pt)=(0.997785,0.642416,0); rgb(71pt)=(0.997552,0.639152,0); rgb(72pt)=(0.997308,0.636004,0); rgb(73pt)=(0.997053,0.632982,0); rgb(74pt)=(0.996788,0.630095,0); rgb(75pt)=(0.996512,0.627352,0); rgb(76pt)=(0.996226,0.624763,0); rgb(77pt)=(0.995851,0.622329,0); rgb(78pt)=(0.99494,0.619997,0); rgb(79pt)=(0.99345,0.617753,0); rgb(80pt)=(0.991419,0.61559,0); rgb(81pt)=(0.988885,0.613503,0); rgb(82pt)=(0.985886,0.611486,0); rgb(83pt)=(0.98246,0.609532,0); rgb(84pt)=(0.978643,0.607636,0); rgb(85pt)=(0.974475,0.605791,0); rgb(86pt)=(0.969992,0.603992,0); rgb(87pt)=(0.965232,0.602233,0); rgb(88pt)=(0.960233,0.600507,0); rgb(89pt)=(0.955033,0.598808,0); rgb(90pt)=(0.949669,0.59713,0); rgb(91pt)=(0.94418,0.595468,0); rgb(92pt)=(0.938602,0.593815,0); rgb(93pt)=(0.932974,0.592166,0); rgb(94pt)=(0.927333,0.590513,0); rgb(95pt)=(0.921717,0.588852,0); rgb(96pt)=(0.916164,0.587176,0); rgb(97pt)=(0.910711,0.585479,0); rgb(98pt)=(0.905397,0.583755,0); rgb(99pt)=(0.900258,0.581999,0); rgb(100pt)=(0.895333,0.580203,0); rgb(101pt)=(0.890659,0.578362,0); rgb(102pt)=(0.886275,0.576471,0); rgb(103pt)=(0.882047,0.574545,0); rgb(104pt)=(0.877819,0.572608,0); rgb(105pt)=(0.873592,0.57066,0); rgb(106pt)=(0.869366,0.568701,0); rgb(107pt)=(0.865143,0.566733,0); rgb(108pt)=(0.860924,0.564756,0); rgb(109pt)=(0.856708,0.562771,0); rgb(110pt)=(0.852497,0.560778,0); rgb(111pt)=(0.848292,0.558779,0); rgb(112pt)=(0.844092,0.556774,0); rgb(113pt)=(0.8399,0.554763,0); rgb(114pt)=(0.835716,0.552749,0); rgb(115pt)=(0.831541,0.55073,0); rgb(116pt)=(0.827374,0.548709,0); rgb(117pt)=(0.823219,0.546686,0); rgb(118pt)=(0.819074,0.54466,0); rgb(119pt)=(0.81494,0.542635,0); rgb(120pt)=(0.81082,0.540609,0); rgb(121pt)=(0.806712,0.538584,0); rgb(122pt)=(0.802619,0.53656,0); rgb(123pt)=(0.798541,0.534539,0); rgb(124pt)=(0.794478,0.532521,0); rgb(125pt)=(0.790431,0.530506,0); rgb(126pt)=(0.786402,0.528496,0); rgb(127pt)=(0.782391,0.526491,0); rgb(128pt)=(0.77841,0.524489,0); rgb(129pt)=(0.774523,0.522478,0); rgb(130pt)=(0.770731,0.520455,0); rgb(131pt)=(0.767022,0.518424,0); rgb(132pt)=(0.763384,0.516385,0); rgb(133pt)=(0.759804,0.514339,0); rgb(134pt)=(0.756272,0.51229,0); rgb(135pt)=(0.752775,0.510237,0); rgb(136pt)=(0.749302,0.508182,0); rgb(137pt)=(0.74584,0.506128,0); rgb(138pt)=(0.742378,0.504075,0); rgb(139pt)=(0.738904,0.502025,0); rgb(140pt)=(0.735406,0.499979,0); rgb(141pt)=(0.731872,0.49794,0); rgb(142pt)=(0.72829,0.495909,0); rgb(143pt)=(0.724649,0.493887,0); rgb(144pt)=(0.720936,0.491875,0); rgb(145pt)=(0.71714,0.489876,0); rgb(146pt)=(0.713249,0.487891,0); rgb(147pt)=(0.709251,0.485921,0); rgb(148pt)=(0.705134,0.483968,0); rgb(149pt)=(0.700887,0.482033,0); rgb(150pt)=(0.696497,0.480118,0); rgb(151pt)=(0.691952,0.478225,0); rgb(152pt)=(0.687242,0.476355,0); rgb(153pt)=(0.682353,0.47451,0); rgb(154pt)=(0.677195,0.472696,0); rgb(155pt)=(0.6717,0.470916,0); rgb(156pt)=(0.665891,0.469169,0); rgb(157pt)=(0.659791,0.46745,0); rgb(158pt)=(0.653423,0.465756,0); rgb(159pt)=(0.64681,0.464084,0); rgb(160pt)=(0.639976,0.462432,0); rgb(161pt)=(0.632943,0.460795,0); rgb(162pt)=(0.625734,0.459171,0); rgb(163pt)=(0.618373,0.457556,0); rgb(164pt)=(0.610882,0.455948,0); rgb(165pt)=(0.603284,0.454343,0); rgb(166pt)=(0.595604,0.452737,0); rgb(167pt)=(0.587863,0.451129,0); rgb(168pt)=(0.580084,0.449514,0); rgb(169pt)=(0.572292,0.447889,0); rgb(170pt)=(0.564508,0.446252,0); rgb(171pt)=(0.556756,0.444599,0); rgb(172pt)=(0.549059,0.442927,0); rgb(173pt)=(0.54144,0.441232,0); rgb(174pt)=(0.533922,0.439512,0); rgb(175pt)=(0.526529,0.437764,0); rgb(176pt)=(0.519282,0.435983,0); rgb(177pt)=(0.512206,0.434168,0); rgb(178pt)=(0.505323,0.432315,0); rgb(179pt)=(0.498628,0.430422,3.92506e-06); rgb(180pt)=(0.491973,0.428504,3.49981e-05); rgb(181pt)=(0.485331,0.426562,9.63073e-05); rgb(182pt)=(0.478704,0.424596,0.000186979); rgb(183pt)=(0.472096,0.422609,0.000306141); rgb(184pt)=(0.465508,0.420599,0.00045292); rgb(185pt)=(0.458942,0.418567,0.000626441); rgb(186pt)=(0.452401,0.416515,0.000825833); rgb(187pt)=(0.445885,0.414441,0.00105022); rgb(188pt)=(0.439399,0.412348,0.00129873); rgb(189pt)=(0.432942,0.410234,0.00157049); rgb(190pt)=(0.426518,0.408102,0.00186463); rgb(191pt)=(0.420129,0.40595,0.00218028); rgb(192pt)=(0.413777,0.40378,0.00251655); rgb(193pt)=(0.407464,0.401592,0.00287258); rgb(194pt)=(0.401191,0.399386,0.00324749); rgb(195pt)=(0.394962,0.397164,0.00364042); rgb(196pt)=(0.388777,0.394925,0.00405048); rgb(197pt)=(0.38264,0.39267,0.00447681); rgb(198pt)=(0.376552,0.390399,0.00491852); rgb(199pt)=(0.370516,0.388113,0.00537476); rgb(200pt)=(0.364532,0.385812,0.00584464); rgb(201pt)=(0.358605,0.383497,0.00632729); rgb(202pt)=(0.352735,0.381168,0.00682184); rgb(203pt)=(0.346925,0.378826,0.00732741); rgb(204pt)=(0.341176,0.376471,0.00784314); rgb(205pt)=(0.335485,0.374093,0.00847245); rgb(206pt)=(0.329843,0.371682,0.00930909); rgb(207pt)=(0.324249,0.369242,0.0103377); rgb(208pt)=(0.318701,0.366772,0.0115428); rgb(209pt)=(0.313198,0.364275,0.0129091); rgb(210pt)=(0.307739,0.361753,0.0144211); rgb(211pt)=(0.302322,0.359206,0.0160634); rgb(212pt)=(0.296945,0.356637,0.0178207); rgb(213pt)=(0.291607,0.354048,0.0196776); rgb(214pt)=(0.286307,0.35144,0.0216186); rgb(215pt)=(0.281043,0.348814,0.0236284); rgb(216pt)=(0.275813,0.346172,0.0256916); rgb(217pt)=(0.270616,0.343517,0.0277927); rgb(218pt)=(0.265451,0.340849,0.0299163); rgb(219pt)=(0.260317,0.33817,0.0320472); rgb(220pt)=(0.25521,0.335482,0.0341698); rgb(221pt)=(0.250131,0.332786,0.0362688); rgb(222pt)=(0.245078,0.330085,0.0383287); rgb(223pt)=(0.240048,0.327379,0.0403343); rgb(224pt)=(0.235042,0.324671,0.04227); rgb(225pt)=(0.230056,0.321962,0.0441205); rgb(226pt)=(0.22509,0.319254,0.0458704); rgb(227pt)=(0.220142,0.316548,0.0475043); rgb(228pt)=(0.215212,0.313846,0.0490067); rgb(229pt)=(0.210296,0.311149,0.0503624); rgb(230pt)=(0.205395,0.308459,0.0515759); rgb(231pt)=(0.200514,0.305763,0.052757); rgb(232pt)=(0.195655,0.303061,0.0539242); rgb(233pt)=(0.190817,0.300353,0.0550763); rgb(234pt)=(0.186001,0.297639,0.0562123); rgb(235pt)=(0.181207,0.294918,0.0573313); rgb(236pt)=(0.176434,0.292191,0.0584321); rgb(237pt)=(0.171685,0.289458,0.0595136); rgb(238pt)=(0.166957,0.286719,0.060575); rgb(239pt)=(0.162252,0.283973,0.0616151); rgb(240pt)=(0.15757,0.281221,0.0626328); rgb(241pt)=(0.152911,0.278463,0.0636271); rgb(242pt)=(0.148275,0.275699,0.0645971); rgb(243pt)=(0.143663,0.272929,0.0655416); rgb(244pt)=(0.139074,0.270152,0.0664596); rgb(245pt)=(0.134508,0.26737,0.06735); rgb(246pt)=(0.129967,0.264581,0.0682118); rgb(247pt)=(0.125449,0.261787,0.0690441); rgb(248pt)=(0.120956,0.258986,0.0698456); rgb(249pt)=(0.116487,0.25618,0.0706154); rgb(250pt)=(0.112043,0.253367,0.0713525); rgb(251pt)=(0.107623,0.250549,0.0720557); rgb(252pt)=(0.103229,0.247724,0.0727241); rgb(253pt)=(0.0988592,0.244894,0.0733566); rgb(254pt)=(0.0945149,0.242058,0.0739522); rgb(255pt)=(0.0901961,0.239216,0.0745098)}, mesh/rows=49]
table[row sep=crcr, point meta=\thisrow{c}] {%
%
x	y	z	c\\
56	0.093	-0.183093589236051	-0.183093589236051\\
56	0.09666	-0.155069262895937	-0.155069262895937\\
56	0.10032	-0.125988786107933	-0.125988786107933\\
56	0.10398	-0.0958521588720439	-0.0958521588720439\\
56	0.10764	-0.064659381188267	-0.064659381188267\\
56	0.1113	-0.0324104530566007	-0.0324104530566007\\
56	0.11496	0.000894625522951387	0.000894625522951387\\
56	0.11862	0.0352558545503928	0.0352558545503928\\
56	0.12228	0.0706732340257201	0.0706732340257201\\
56	0.12594	0.107146763948935	0.107146763948935\\
56	0.1296	0.144676444320039	0.144676444320039\\
56	0.13326	0.183262275139029	0.183262275139029\\
56	0.13692	0.222904256405907	0.222904256405907\\
56	0.14058	0.263602388120673	0.263602388120673\\
56	0.14424	0.305356670283327	0.305356670283327\\
56	0.1479	0.348167102893868	0.348167102893868\\
56	0.15156	0.392033685952298	0.392033685952298\\
56	0.15522	0.436956419458614	0.436956419458614\\
56	0.15888	0.482935303412818	0.482935303412818\\
56	0.16254	0.529970337814909	0.529970337814909\\
56	0.1662	0.578061522664889	0.578061522664889\\
56	0.16986	0.627208857962755	0.627208857962755\\
56	0.17352	0.677412343708509	0.677412343708509\\
56	0.17718	0.728671979902152	0.728671979902152\\
56	0.18084	0.780987766543681	0.780987766543681\\
56	0.1845	0.834359703633098	0.834359703633098\\
56	0.18816	0.888787791170404	0.888787791170404\\
56	0.19182	0.944272029155596	0.944272029155596\\
56	0.19548	1.00081241758868	1.00081241758868\\
56	0.19914	1.05840895646964	1.05840895646964\\
56	0.2028	1.1170616457985	1.1170616457985\\
56	0.20646	1.17677048557524	1.17677048557524\\
56	0.21012	1.23753547579987	1.23753547579987\\
56	0.21378	1.29935661647239	1.29935661647239\\
56	0.21744	1.3622339075928	1.3622339075928\\
56	0.2211	1.42616734916109	1.42616734916109\\
56	0.22476	1.49115694117727	1.49115694117727\\
56	0.22842	1.55720268364134	1.55720268364134\\
56	0.23208	1.6243045765533	1.6243045765533\\
56	0.23574	1.69246261991314	1.69246261991314\\
56	0.2394	1.76167681372087	1.76167681372087\\
56	0.24306	1.83194715797649	1.83194715797649\\
56	0.24672	1.90327365268	1.90327365268\\
56	0.25038	1.97565629783139	1.97565629783139\\
56	0.25404	2.04909509343067	2.04909509343067\\
56	0.2577	2.12359003947784	2.12359003947784\\
56	0.26136	2.1991411359729	2.1991411359729\\
56	0.26502	2.27574838291585	2.27574838291585\\
56	0.26868	2.35341178030668	2.35341178030668\\
56	0.27234	2.4321313281454	2.4321313281454\\
56	0.276	2.51190702643201	2.51190702643201\\
56.375	0.093	-0.189487201537818	-0.189487201537818\\
56.375	0.09666	-0.16338436814129	-0.16338436814129\\
56.375	0.10032	-0.136225384296876	-0.136225384296876\\
56.375	0.10398	-0.108010250004572	-0.108010250004572\\
56.375	0.10764	-0.0787389652643813	-0.0787389652643813\\
56.375	0.1113	-0.0484115300763026	-0.0484115300763026\\
56.375	0.11496	-0.0170279444403363	-0.0170279444403363\\
56.375	0.11862	0.0154117916435176	0.0154117916435176\\
56.375	0.12228	0.0489076781752591	0.0489076781752591\\
56.375	0.12594	0.0834597151548883	0.0834597151548883\\
56.375	0.1296	0.119067902582405	0.119067902582405\\
56.375	0.13326	0.155732240457807	0.155732240457807\\
56.375	0.13692	0.193452728781099	0.193452728781099\\
56.375	0.14058	0.232229367552279	0.232229367552279\\
56.375	0.14424	0.272062156771346	0.272062156771346\\
56.375	0.1479	0.312951096438299	0.312951096438299\\
56.375	0.15156	0.354896186553142	0.354896186553142\\
56.375	0.15522	0.397897427115872	0.397897427115872\\
56.375	0.15888	0.44195481812649	0.44195481812649\\
56.375	0.16254	0.487068359584993	0.487068359584993\\
56.375	0.1662	0.533238051491386	0.533238051491386\\
56.375	0.16986	0.580463893845666	0.580463893845666\\
56.375	0.17352	0.628745886647834	0.628745886647834\\
56.375	0.17718	0.67808402989789	0.67808402989789\\
56.375	0.18084	0.728478323595833	0.728478323595833\\
56.375	0.1845	0.779928767741662	0.779928767741662\\
56.375	0.18816	0.832435362335381	0.832435362335381\\
56.375	0.19182	0.885998107376987	0.885998107376987\\
56.375	0.19548	0.940617002866481	0.940617002866481\\
56.375	0.19914	0.996292048803861	0.996292048803861\\
56.375	0.2028	1.05302324518913	1.05302324518913\\
56.375	0.20646	1.11081059202229	1.11081059202229\\
56.375	0.21012	1.16965408930333	1.16965408930333\\
56.375	0.21378	1.22955373703226	1.22955373703226\\
56.375	0.21744	1.29050953520908	1.29050953520908\\
56.375	0.2211	1.35252148383379	1.35252148383379\\
56.375	0.22476	1.41558958290638	1.41558958290638\\
56.375	0.22842	1.47971383242687	1.47971383242687\\
56.375	0.23208	1.54489423239523	1.54489423239523\\
56.375	0.23574	1.61113078281149	1.61113078281149\\
56.375	0.2394	1.67842348367563	1.67842348367563\\
56.375	0.24306	1.74677233498767	1.74677233498767\\
56.375	0.24672	1.81617733674759	1.81617733674759\\
56.375	0.25038	1.8866384889554	1.8866384889554\\
56.375	0.25404	1.95815579161109	1.95815579161109\\
56.375	0.2577	2.03072924471467	2.03072924471467\\
56.375	0.26136	2.10435884826614	2.10435884826614\\
56.375	0.26502	2.1790446022655	2.1790446022655\\
56.375	0.26868	2.25478650671275	2.25478650671275\\
56.375	0.27234	2.33158456160788	2.33158456160788\\
56.375	0.276	2.4094387669509	2.4094387669509\\
56.75	0.093	-0.195103440740309	-0.195103440740309\\
56.75	0.09666	-0.17092210028737	-0.17092210028737\\
56.75	0.10032	-0.14568460938654	-0.14568460938654\\
56.75	0.10398	-0.119390968037824	-0.119390968037824\\
56.75	0.10764	-0.0920411762412204	-0.0920411762412204\\
56.75	0.1113	-0.0636352339967274	-0.0636352339967274\\
56.75	0.11496	-0.0341731413043487	-0.0341731413043487\\
56.75	0.11862	-0.0036548981640806	-0.0036548981640806\\
56.75	0.12228	0.0279194954240733	0.0279194954240733\\
56.75	0.12594	0.0605500394601149	0.0605500394601149\\
56.75	0.1296	0.094236733944044	0.094236733944044\\
56.75	0.13326	0.128979578875863	0.128979578875863\\
56.75	0.13692	0.164778574255567	0.164778574255567\\
56.75	0.14058	0.201633720083159	0.201633720083159\\
56.75	0.14424	0.239545016358639	0.239545016358639\\
56.75	0.1479	0.278512463082008	0.278512463082008\\
56.75	0.15156	0.318536060253263	0.318536060253263\\
56.75	0.15522	0.359615807872405	0.359615807872405\\
56.75	0.15888	0.401751705939435	0.401751705939435\\
56.75	0.16254	0.444943754454353	0.444943754454353\\
56.75	0.1662	0.48919195341716	0.48919195341716\\
56.75	0.16986	0.534496302827853	0.534496302827853\\
56.75	0.17352	0.580856802686433	0.580856802686433\\
56.75	0.17718	0.628273452992903	0.628273452992903\\
56.75	0.18084	0.676746253747259	0.676746253747259\\
56.75	0.1845	0.726275204949502	0.726275204949502\\
56.75	0.18816	0.776860306599634	0.776860306599634\\
56.75	0.19182	0.828501558697652	0.828501558697652\\
56.75	0.19548	0.88119896124356	0.88119896124356\\
56.75	0.19914	0.934952514237354	0.934952514237354\\
56.75	0.2028	0.989762217679035	0.989762217679035\\
56.75	0.20646	1.04562807156861	1.04562807156861\\
56.75	0.21012	1.10255007590606	1.10255007590606\\
56.75	0.21378	1.16052823069141	1.16052823069141\\
56.75	0.21744	1.21956253592464	1.21956253592464\\
56.75	0.2211	1.27965299160576	1.27965299160576\\
56.75	0.22476	1.34079959773477	1.34079959773477\\
56.75	0.22842	1.40300235431167	1.40300235431167\\
56.75	0.23208	1.46626126133645	1.46626126133645\\
56.75	0.23574	1.53057631880911	1.53057631880911\\
56.75	0.2394	1.59594752672967	1.59594752672967\\
56.75	0.24306	1.66237488509812	1.66237488509812\\
56.75	0.24672	1.72985839391445	1.72985839391445\\
56.75	0.25038	1.79839805317867	1.79839805317867\\
56.75	0.25404	1.86799386289078	1.86799386289078\\
56.75	0.2577	1.93864582305078	1.93864582305078\\
56.75	0.26136	2.01035393365866	2.01035393365866\\
56.75	0.26502	2.08311819471443	2.08311819471443\\
56.75	0.26868	2.15693860621809	2.15693860621809\\
56.75	0.27234	2.23181516816964	2.23181516816964\\
56.75	0.276	2.30774788056907	2.30774788056907\\
57.125	0.093	-0.199942306843528	-0.199942306843528\\
57.125	0.09666	-0.177682459334174	-0.177682459334174\\
57.125	0.10032	-0.154366461376932	-0.154366461376932\\
57.125	0.10398	-0.129994312971802	-0.129994312971802\\
57.125	0.10764	-0.104566014118786	-0.104566014118786\\
57.125	0.1113	-0.0780815648178805	-0.0780815648178805\\
57.125	0.11496	-0.0505409650690876	-0.0505409650690876\\
57.125	0.11862	-0.0219442148724071	-0.0219442148724071\\
57.125	0.12228	0.00770868577216111	0.00770868577216111\\
57.125	0.12594	0.0384177368646151	0.0384177368646151\\
57.125	0.1296	0.0701829384049584	0.0701829384049584\\
57.125	0.13326	0.103004290393189	0.103004290393189\\
57.125	0.13692	0.136881792829308	0.136881792829308\\
57.125	0.14058	0.171815445713313	0.171815445713313\\
57.125	0.14424	0.207805249045206	0.207805249045206\\
57.125	0.1479	0.244851202824988	0.244851202824988\\
57.125	0.15156	0.282953307052655	0.282953307052655\\
57.125	0.15522	0.322111561728212	0.322111561728212\\
57.125	0.15888	0.362325966851656	0.362325966851656\\
57.125	0.16254	0.403596522422986	0.403596522422986\\
57.125	0.1662	0.445923228442206	0.445923228442206\\
57.125	0.16986	0.489306084909313	0.489306084909313\\
57.125	0.17352	0.533745091824308	0.533745091824308\\
57.125	0.17718	0.579240249187188	0.579240249187188\\
57.125	0.18084	0.625791556997958	0.625791556997958\\
57.125	0.1845	0.673399015256616	0.673399015256616\\
57.125	0.18816	0.722062623963159	0.722062623963159\\
57.125	0.19182	0.771782383117592	0.771782383117592\\
57.125	0.19548	0.822558292719912	0.822558292719912\\
57.125	0.19914	0.87439035277012	0.87439035277012\\
57.125	0.2028	0.927278563268215	0.927278563268215\\
57.125	0.20646	0.9812229242142	0.9812229242142\\
57.125	0.21012	1.03622343560807	1.03622343560807\\
57.125	0.21378	1.09228009744983	1.09228009744983\\
57.125	0.21744	1.14939290973947	1.14939290973947\\
57.125	0.2211	1.207561872477	1.207561872477\\
57.125	0.22476	1.26678698566243	1.26678698566243\\
57.125	0.22842	1.32706824929573	1.32706824929573\\
57.125	0.23208	1.38840566337693	1.38840566337693\\
57.125	0.23574	1.45079922790601	1.45079922790601\\
57.125	0.2394	1.51424894288298	1.51424894288298\\
57.125	0.24306	1.57875480830784	1.57875480830784\\
57.125	0.24672	1.64431682418059	1.64431682418059\\
57.125	0.25038	1.71093499050123	1.71093499050123\\
57.125	0.25404	1.77860930726975	1.77860930726975\\
57.125	0.2577	1.84733977448616	1.84733977448616\\
57.125	0.26136	1.91712639215045	1.91712639215045\\
57.125	0.26502	1.98796916026264	1.98796916026264\\
57.125	0.26868	2.05986807882271	2.05986807882271\\
57.125	0.27234	2.13282314783067	2.13282314783067\\
57.125	0.276	2.20683436728652	2.20683436728652\\
57.5	0.093	-0.204003799847469	-0.204003799847469\\
57.5	0.09666	-0.183665445281702	-0.183665445281702\\
57.5	0.10032	-0.162270940268047	-0.162270940268047\\
57.5	0.10398	-0.139820284806505	-0.139820284806505\\
57.5	0.10764	-0.116313478897074	-0.116313478897074\\
57.5	0.1113	-0.0917505225397548	-0.0917505225397548\\
57.5	0.11496	-0.0661314157345494	-0.0661314157345494\\
57.5	0.11862	-0.0394561584814564	-0.0394561584814564\\
57.5	0.12228	-0.0117247507804741	-0.0117247507804741\\
57.5	0.12594	0.0170628073683942	0.0170628073683942\\
57.5	0.1296	0.0469065159651498	0.0469065159651498\\
57.5	0.13326	0.0778063750097933	0.0778063750097933\\
57.5	0.13692	0.109762384502326	0.109762384502326\\
57.5	0.14058	0.142774544442743	0.142774544442743\\
57.5	0.14424	0.176842854831051	0.176842854831051\\
57.5	0.1479	0.211967315667245	0.211967315667245\\
57.5	0.15156	0.248147926951326	0.248147926951326\\
57.5	0.15522	0.285384688683296	0.285384688683296\\
57.5	0.15888	0.323677600863153	0.323677600863153\\
57.5	0.16254	0.363026663490899	0.363026663490899\\
57.5	0.1662	0.403431876566531	0.403431876566531\\
57.5	0.16986	0.44489324009005	0.44489324009005\\
57.5	0.17352	0.487410754061457	0.487410754061457\\
57.5	0.17718	0.530984418480752	0.530984418480752\\
57.5	0.18084	0.575614233347936	0.575614233347936\\
57.5	0.1845	0.621300198663006	0.621300198663006\\
57.5	0.18816	0.668042314425964	0.668042314425964\\
57.5	0.19182	0.715840580636809	0.715840580636809\\
57.5	0.19548	0.764694997295542	0.764694997295542\\
57.5	0.19914	0.814605564402164	0.814605564402164\\
57.5	0.2028	0.865572281956671	0.865572281956671\\
57.5	0.20646	0.917595149959068	0.917595149959068\\
57.5	0.21012	0.970674168409351	0.970674168409351\\
57.5	0.21378	1.02480933730753	1.02480933730753\\
57.5	0.21744	1.08000065665358	1.08000065665358\\
57.5	0.2211	1.13624812644753	1.13624812644753\\
57.5	0.22476	1.19355174668936	1.19355174668936\\
57.5	0.22842	1.25191151737909	1.25191151737909\\
57.5	0.23208	1.31132743851669	1.31132743851669\\
57.5	0.23574	1.37179951010219	1.37179951010219\\
57.5	0.2394	1.43332773213558	1.43332773213558\\
57.5	0.24306	1.49591210461685	1.49591210461685\\
57.5	0.24672	1.55955262754601	1.55955262754601\\
57.5	0.25038	1.62424930092305	1.62424930092305\\
57.5	0.25404	1.69000212474799	1.69000212474799\\
57.5	0.2577	1.75681109902081	1.75681109902081\\
57.5	0.26136	1.82467622374152	1.82467622374152\\
57.5	0.26502	1.89359749891012	1.89359749891012\\
57.5	0.26868	1.96357492452661	1.96357492452661\\
57.5	0.27234	2.03460850059098	2.03460850059098\\
57.5	0.276	2.10669822710324	2.10669822710324\\
57.875	0.093	-0.207287919752138	-0.207287919752138\\
57.875	0.09666	-0.188871058129958	-0.188871058129958\\
57.875	0.10032	-0.169398046059889	-0.169398046059889\\
57.875	0.10398	-0.148868883541932	-0.148868883541932\\
57.875	0.10764	-0.127283570576088	-0.127283570576088\\
57.875	0.1113	-0.104642107162356	-0.104642107162356\\
57.875	0.11496	-0.0809444933007377	-0.0809444933007377\\
57.875	0.11862	-0.0561907289912306	-0.0561907289912306\\
57.875	0.12228	-0.0303808142338358	-0.0303808142338358\\
57.875	0.12594	-0.00351474902855509	-0.00351474902855509\\
57.875	0.1296	0.0244074666246148	0.0244074666246148\\
57.875	0.13326	0.0533858327256724	0.0533858327256724\\
57.875	0.13692	0.0834203492746158	0.0834203492746158\\
57.875	0.14058	0.114511016271449	0.114511016271449\\
57.875	0.14424	0.146657833716168	0.146657833716168\\
57.875	0.1479	0.179860801608776	0.179860801608776\\
57.875	0.15156	0.214119919949271	0.214119919949271\\
57.875	0.15522	0.249435188737653	0.249435188737653\\
57.875	0.15888	0.285806607973924	0.285806607973924\\
57.875	0.16254	0.323234177658083	0.323234177658083\\
57.875	0.1662	0.361717897790127	0.361717897790127\\
57.875	0.16986	0.401257768370061	0.401257768370061\\
57.875	0.17352	0.441853789397882	0.441853789397882\\
57.875	0.17718	0.483505960873589	0.483505960873589\\
57.875	0.18084	0.526214282797186	0.526214282797186\\
57.875	0.1845	0.56997875516867	0.56997875516867\\
57.875	0.18816	0.61479937798804	0.61479937798804\\
57.875	0.19182	0.6606761512553	0.6606761512553\\
57.875	0.19548	0.707609074970445	0.707609074970445\\
57.875	0.19914	0.755598149133481	0.755598149133481\\
57.875	0.2028	0.8046433737444	0.8046433737444\\
57.875	0.20646	0.854744748803214	0.854744748803214\\
57.875	0.21012	0.905902274309909	0.905902274309909\\
57.875	0.21378	0.958115950264493	0.958115950264493\\
57.875	0.21744	1.01138577666697	1.01138577666697\\
57.875	0.2211	1.06571175351733	1.06571175351733\\
57.875	0.22476	1.12109388081557	1.12109388081557\\
57.875	0.22842	1.17753215856171	1.17753215856171\\
57.875	0.23208	1.23502658675573	1.23502658675573\\
57.875	0.23574	1.29357716539764	1.29357716539764\\
57.875	0.2394	1.35318389448744	1.35318389448744\\
57.875	0.24306	1.41384677402512	1.41384677402512\\
57.875	0.24672	1.4755658040107	1.4755658040107\\
57.875	0.25038	1.53834098444416	1.53834098444416\\
57.875	0.25404	1.60217231532551	1.60217231532551\\
57.875	0.2577	1.66705979665474	1.66705979665474\\
57.875	0.26136	1.73300342843186	1.73300342843186\\
57.875	0.26502	1.80000321065688	1.80000321065688\\
57.875	0.26868	1.86805914332977	1.86805914332977\\
57.875	0.27234	1.93717122645056	1.93717122645056\\
57.875	0.276	2.00733946001924	2.00733946001924\\
58.25	0.093	-0.209794666557529	-0.209794666557529\\
58.25	0.09666	-0.193299297878933	-0.193299297878933\\
58.25	0.10032	-0.175747778752453	-0.175747778752453\\
58.25	0.10398	-0.157140109178084	-0.157140109178084\\
58.25	0.10764	-0.137476289155825	-0.137476289155825\\
58.25	0.1113	-0.116756318685681	-0.116756318685681\\
58.25	0.11496	-0.094980197767649	-0.094980197767649\\
58.25	0.11862	-0.0721479264017294	-0.0721479264017294\\
58.25	0.12228	-0.0482595045879204	-0.0482595045879204\\
58.25	0.12594	-0.0233149323262255	-0.0233149323262255\\
58.25	0.1296	0.00268579038335681	0.00268579038335681\\
58.25	0.13326	0.0297426635408269	0.0297426635408269\\
58.25	0.13692	0.0578556871461845	0.0578556871461845\\
58.25	0.14058	0.0870248611994301	0.0870248611994301\\
58.25	0.14424	0.117250185700563	0.117250185700563\\
58.25	0.1479	0.148531660649583	0.148531660649583\\
58.25	0.15156	0.180869286046492	0.180869286046492\\
58.25	0.15522	0.214263061891289	0.214263061891289\\
58.25	0.15888	0.248712988183971	0.248712988183971\\
58.25	0.16254	0.284219064924544	0.284219064924544\\
58.25	0.1662	0.320781292113002	0.320781292113002\\
58.25	0.16986	0.358399669749348	0.358399669749348\\
58.25	0.17352	0.397074197833582	0.397074197833582\\
58.25	0.17718	0.436804876365704	0.436804876365704\\
58.25	0.18084	0.477591705345712	0.477591705345712\\
58.25	0.1845	0.519434684773609	0.519434684773609\\
58.25	0.18816	0.562333814649394	0.562333814649394\\
58.25	0.19182	0.606289094973067	0.606289094973067\\
58.25	0.19548	0.651300525744627	0.651300525744627\\
58.25	0.19914	0.697368106964072	0.697368106964072\\
58.25	0.2028	0.744491838631407	0.744491838631407\\
58.25	0.20646	0.792671720746633	0.792671720746633\\
58.25	0.21012	0.841907753309741	0.841907753309741\\
58.25	0.21378	0.89219993632074	0.89219993632074\\
58.25	0.21744	0.943548269779627	0.943548269779627\\
58.25	0.2211	0.995952753686398	0.995952753686398\\
58.25	0.22476	1.04941338804106	1.04941338804106\\
58.25	0.22842	1.10393017284361	1.10393017284361\\
58.25	0.23208	1.15950310809404	1.15950310809404\\
58.25	0.23574	1.21613219379237	1.21613219379237\\
58.25	0.2394	1.27381742993857	1.27381742993857\\
58.25	0.24306	1.33255881653267	1.33255881653267\\
58.25	0.24672	1.39235635357466	1.39235635357466\\
58.25	0.25038	1.45321004106454	1.45321004106454\\
58.25	0.25404	1.5151198790023	1.5151198790023\\
58.25	0.2577	1.57808586738795	1.57808586738795\\
58.25	0.26136	1.64210800622148	1.64210800622148\\
58.25	0.26502	1.70718629550291	1.70718629550291\\
58.25	0.26868	1.77332073523222	1.77332073523222\\
58.25	0.27234	1.84051132540942	1.84051132540942\\
58.25	0.276	1.90875806603451	1.90875806603451\\
58.625	0.093	-0.211524040263647	-0.211524040263647\\
58.625	0.09666	-0.19695016452864	-0.19695016452864\\
58.625	0.10032	-0.181320138345744	-0.181320138345744\\
58.625	0.10398	-0.164633961714963	-0.164633961714963\\
58.625	0.10764	-0.146891634636292	-0.146891634636292\\
58.625	0.1113	-0.128093157109733	-0.128093157109733\\
58.625	0.11496	-0.108238529135289	-0.108238529135289\\
58.625	0.11862	-0.0873277507129547	-0.0873277507129547\\
58.625	0.12228	-0.0653608218427351	-0.0653608218427351\\
58.625	0.12594	-0.042337742524626	-0.042337742524626\\
58.625	0.1296	-0.0182585127586294	-0.0182585127586294\\
58.625	0.13326	0.00687686745525307	0.00687686745525307\\
58.625	0.13692	0.0330683981170249	0.0330683981170249\\
58.625	0.14058	0.0603160792266829	0.0603160792266829\\
58.625	0.14424	0.08861991078423	0.08861991078423\\
58.625	0.1479	0.117979892789663	0.117979892789663\\
58.625	0.15156	0.148396025242985	0.148396025242985\\
58.625	0.15522	0.179868308144193	0.179868308144193\\
58.625	0.15888	0.212396741493291	0.212396741493291\\
58.625	0.16254	0.245981325290275	0.245981325290275\\
58.625	0.1662	0.280622059535147	0.280622059535147\\
58.625	0.16986	0.316318944227908	0.316318944227908\\
58.625	0.17352	0.353071979368554	0.353071979368554\\
58.625	0.17718	0.39088116495709	0.39088116495709\\
58.625	0.18084	0.429746500993511	0.429746500993511\\
58.625	0.1845	0.469667987477822	0.469667987477822\\
58.625	0.18816	0.510645624410019	0.510645624410019\\
58.625	0.19182	0.552679411790105	0.552679411790105\\
58.625	0.19548	0.595769349618078	0.595769349618078\\
58.625	0.19914	0.639915437893936	0.639915437893936\\
58.625	0.2028	0.685117676617687	0.685117676617687\\
58.625	0.20646	0.731376065789322	0.731376065789322\\
58.625	0.21012	0.778690605408846	0.778690605408846\\
58.625	0.21378	0.827061295476258	0.827061295476258\\
58.625	0.21744	0.876488135991557	0.876488135991557\\
58.625	0.2211	0.926971126954744	0.926971126954744\\
58.625	0.22476	0.978510268365815	0.978510268365815\\
58.625	0.22842	1.03110556022478	1.03110556022478\\
58.625	0.23208	1.08475700253163	1.08475700253163\\
58.625	0.23574	1.13946459528636	1.13946459528636\\
58.625	0.2394	1.19522833848899	1.19522833848899\\
58.625	0.24306	1.2520482321395	1.2520482321395\\
58.625	0.24672	1.3099242762379	1.3099242762379\\
58.625	0.25038	1.36885647078419	1.36885647078419\\
58.625	0.25404	1.42884481577836	1.42884481577836\\
58.625	0.2577	1.48988931122042	1.48988931122042\\
58.625	0.26136	1.55198995711037	1.55198995711037\\
58.625	0.26502	1.61514675344821	1.61514675344821\\
58.625	0.26868	1.67935970023394	1.67935970023394\\
58.625	0.27234	1.74462879746755	1.74462879746755\\
58.625	0.276	1.81095404514905	1.81095404514905\\
59	0.093	-0.212476040870491	-0.212476040870491\\
59	0.09666	-0.19982365807907	-0.19982365807907\\
59	0.10032	-0.186115124839762	-0.186115124839762\\
59	0.10398	-0.171350441152566	-0.171350441152566\\
59	0.10764	-0.155529607017483	-0.155529607017483\\
59	0.1113	-0.138652622434511	-0.138652622434511\\
59	0.11496	-0.120719487403653	-0.120719487403653\\
59	0.11862	-0.101730201924907	-0.101730201924907\\
59	0.12228	-0.0816847659982727	-0.0816847659982727\\
59	0.12594	-0.0605831796237511	-0.0605831796237511\\
59	0.1296	-0.0384254428013404	-0.0384254428013404\\
59	0.13326	-0.0152115555310437	-0.0152115555310437\\
59	0.13692	0.00905848218714056	0.00905848218714056\\
59	0.14058	0.034384670353211	0.034384670353211\\
59	0.14424	0.0607670089671706	0.0607670089671706\\
59	0.1479	0.0882054980290177	0.0882054980290177\\
59	0.15156	0.116700137538754	0.116700137538754\\
59	0.15522	0.146250927496377	0.146250927496377\\
59	0.15888	0.176857867901885	0.176857867901885\\
59	0.16254	0.208520958755283	0.208520958755283\\
59	0.1662	0.241240200056568	0.241240200056568\\
59	0.16986	0.275015591805741	0.275015591805741\\
59	0.17352	0.309847134002803	0.309847134002803\\
59	0.17718	0.345734826647751	0.345734826647751\\
59	0.18084	0.382678669740586	0.382678669740586\\
59	0.1845	0.42067866328131	0.42067866328131\\
59	0.18816	0.459734807269919	0.459734807269919\\
59	0.19182	0.499847101706418	0.499847101706418\\
59	0.19548	0.541015546590804	0.541015546590804\\
59	0.19914	0.583240141923077	0.583240141923077\\
59	0.2028	0.626520887703241	0.626520887703241\\
59	0.20646	0.670857783931292	0.670857783931292\\
59	0.21012	0.716250830607228	0.716250830607228\\
59	0.21378	0.762700027731052	0.762700027731052\\
59	0.21744	0.810205375302764	0.810205375302764\\
59	0.2211	0.858766873322363	0.858766873322363\\
59	0.22476	0.90838452178985	0.90838452178985\\
59	0.22842	0.959058320705225	0.959058320705225\\
59	0.23208	1.01078827006849	1.01078827006849\\
59	0.23574	1.06357436987963	1.06357436987963\\
59	0.2394	1.11741662013867	1.11741662013867\\
59	0.24306	1.1723150208456	1.1723150208456\\
59	0.24672	1.22826957200041	1.22826957200041\\
59	0.25038	1.28528027360311	1.28528027360311\\
59	0.25404	1.3433471256537	1.3433471256537\\
59	0.2577	1.40247012815218	1.40247012815218\\
59	0.26136	1.46264928109854	1.46264928109854\\
59	0.26502	1.52388458449279	1.52388458449279\\
59	0.26868	1.58617603833493	1.58617603833493\\
59	0.27234	1.64952364262495	1.64952364262495\\
59	0.276	1.71392739736287	1.71392739736287\\
59.375	0.093	-0.212650668378058	-0.212650668378058\\
59.375	0.09666	-0.201919778530225	-0.201919778530225\\
59.375	0.10032	-0.190132738234502	-0.190132738234502\\
59.375	0.10398	-0.177289547490894	-0.177289547490894\\
59.375	0.10764	-0.163390206299396	-0.163390206299396\\
59.375	0.1113	-0.148434714660011	-0.148434714660011\\
59.375	0.11496	-0.13242307257274	-0.13242307257274\\
59.375	0.11862	-0.11535528003758	-0.11535528003758\\
59.375	0.12228	-0.0972313370545332	-0.0972313370545332\\
59.375	0.12594	-0.0780512436235975	-0.0780512436235975\\
59.375	0.1296	-0.0578149997447761	-0.0578149997447761\\
59.375	0.13326	-0.0365226054180652	-0.0365226054180652\\
59.375	0.13692	-0.0141740606434685	-0.0141740606434685\\
59.375	0.14058	0.00923063457901796	0.00923063457901796\\
59.375	0.14424	0.0336914802493899	0.0336914802493899\\
59.375	0.1479	0.0592084763676513	0.0592084763676513\\
59.375	0.15156	0.0857816229337984	0.0857816229337984\\
59.375	0.15522	0.113410919947835	0.113410919947835\\
59.375	0.15888	0.142096367409758	0.142096367409758\\
59.375	0.16254	0.171837965319568	0.171837965319568\\
59.375	0.1662	0.202635713677267	0.202635713677267\\
59.375	0.16986	0.234489612482852	0.234489612482852\\
59.375	0.17352	0.267399661736327	0.267399661736327\\
59.375	0.17718	0.301365861437688	0.301365861437688\\
59.375	0.18084	0.336388211586937	0.336388211586937\\
59.375	0.1845	0.372466712184075	0.372466712184075\\
59.375	0.18816	0.409601363229098	0.409601363229098\\
59.375	0.19182	0.447792164722009	0.447792164722009\\
59.375	0.19548	0.48703911666281	0.48703911666281\\
59.375	0.19914	0.527342219051496	0.527342219051496\\
59.375	0.2028	0.568701471888072	0.568701471888072\\
59.375	0.20646	0.611116875172535	0.611116875172535\\
59.375	0.21012	0.654588428904884	0.654588428904884\\
59.375	0.21378	0.699116133085124	0.699116133085124\\
59.375	0.21744	0.744699987713248	0.744699987713248\\
59.375	0.2211	0.79133999278926	0.79133999278926\\
59.375	0.22476	0.839036148313159	0.839036148313159\\
59.375	0.22842	0.88778845428495	0.88778845428495\\
59.375	0.23208	0.937596910704624	0.937596910704624\\
59.375	0.23574	0.988461517572184	0.988461517572184\\
59.375	0.2394	1.04038227488764	1.04038227488764\\
59.375	0.24306	1.09335918265097	1.09335918265097\\
59.375	0.24672	1.1473922408622	1.1473922408622\\
59.375	0.25038	1.20248144952132	1.20248144952132\\
59.375	0.25404	1.25862680862832	1.25862680862832\\
59.375	0.2577	1.3158283181832	1.3158283181832\\
59.375	0.26136	1.37408597818598	1.37408597818598\\
59.375	0.26502	1.43339978863665	1.43339978863665\\
59.375	0.26868	1.4937697495352	1.4937697495352\\
59.375	0.27234	1.55519586088164	1.55519586088164\\
59.375	0.276	1.61767812267597	1.61767812267597\\
59.75	0.093	-0.21204792278635	-0.21204792278635\\
59.75	0.09666	-0.203238525882101	-0.203238525882101\\
59.75	0.10032	-0.193372978529967	-0.193372978529967\\
59.75	0.10398	-0.182451280729945	-0.182451280729945\\
59.75	0.10764	-0.170473432482035	-0.170473432482035\\
59.75	0.1113	-0.157439433786237	-0.157439433786237\\
59.75	0.11496	-0.143349284642552	-0.143349284642552\\
59.75	0.11862	-0.128202985050979	-0.128202985050979\\
59.75	0.12228	-0.112000535011519	-0.112000535011519\\
59.75	0.12594	-0.0947419345241703	-0.0947419345241703\\
59.75	0.1296	-0.0764271835889347	-0.0764271835889347\\
59.75	0.13326	-0.0570562822058114	-0.0570562822058114\\
59.75	0.13692	-0.0366292303748005	-0.0366292303748005\\
59.75	0.14058	-0.0151460280959016	-0.0151460280959016\\
59.75	0.14424	0.0073933246308846	0.0073933246308846\\
59.75	0.1479	0.0309888278055583	0.0309888278055583\\
59.75	0.15156	0.0556404814281197	0.0556404814281197\\
59.75	0.15522	0.0813482854985688	0.0813482854985688\\
59.75	0.15888	0.108112240016904	0.108112240016904\\
59.75	0.16254	0.135932344983128	0.135932344983128\\
59.75	0.1662	0.16480860039724	0.16480860039724\\
59.75	0.16986	0.194741006259239	0.194741006259239\\
59.75	0.17352	0.225729562569126	0.225729562569126\\
59.75	0.17718	0.257774269326901	0.257774269326901\\
59.75	0.18084	0.290875126532563	0.290875126532563\\
59.75	0.1845	0.325032134186113	0.325032134186113\\
59.75	0.18816	0.360245292287551	0.360245292287551\\
59.75	0.19182	0.396514600836876	0.396514600836876\\
59.75	0.19548	0.433840059834087	0.433840059834087\\
59.75	0.19914	0.472221669279189	0.472221669279189\\
59.75	0.2028	0.511659429172178	0.511659429172178\\
59.75	0.20646	0.552153339513053	0.552153339513053\\
59.75	0.21012	0.593703400301818	0.593703400301818\\
59.75	0.21378	0.636309611538467	0.636309611538467\\
59.75	0.21744	0.679971973223007	0.679971973223007\\
59.75	0.2211	0.724690485355431	0.724690485355431\\
59.75	0.22476	0.770465147935747	0.770465147935747\\
59.75	0.22842	0.817295960963946	0.817295960963946\\
59.75	0.23208	0.865182924440037	0.865182924440037\\
59.75	0.23574	0.914126038364012	0.914126038364012\\
59.75	0.2394	0.964125302735874	0.964125302735874\\
59.75	0.24306	1.01518071755563	1.01518071755563\\
59.75	0.24672	1.06729228282327	1.06729228282327\\
59.75	0.25038	1.12045999853879	1.12045999853879\\
59.75	0.25404	1.17468386470221	1.17468386470221\\
59.75	0.2577	1.22996388131351	1.22996388131351\\
59.75	0.26136	1.2863000483727	1.2863000483727\\
59.75	0.26502	1.34369236587978	1.34369236587978\\
59.75	0.26868	1.40214083383474	1.40214083383474\\
59.75	0.27234	1.4616454522376	1.4616454522376\\
59.75	0.276	1.52220622108833	1.52220622108833\\
60.125	0.093	-0.210667804095368	-0.210667804095368\\
60.125	0.09666	-0.20377990013471	-0.20377990013471\\
60.125	0.10032	-0.195835845726161	-0.195835845726161\\
60.125	0.10398	-0.186835640869724	-0.186835640869724\\
60.125	0.10764	-0.176779285565402	-0.176779285565402\\
60.125	0.1113	-0.16566677981319	-0.16566677981319\\
60.125	0.11496	-0.153498123613092	-0.153498123613092\\
60.125	0.11862	-0.140273316965105	-0.140273316965105\\
60.125	0.12228	-0.125992359869232	-0.125992359869232\\
60.125	0.12594	-0.11065525232547	-0.11065525232547\\
60.125	0.1296	-0.0942619943338217	-0.0942619943338217\\
60.125	0.13326	-0.0768125858942841	-0.0768125858942841\\
60.125	0.13692	-0.0583070270068607	-0.0583070270068607\\
60.125	0.14058	-0.0387453176715494	-0.0387453176715494\\
60.125	0.14424	-0.018127457888349	-0.018127457888349\\
60.125	0.1479	0.00354655234273715	0.00354655234273715\\
60.125	0.15156	0.026276713021711	0.026276713021711\\
60.125	0.15522	0.0500630241485742	0.0500630241485742\\
60.125	0.15888	0.0749054857233237	0.0749054857233237\\
60.125	0.16254	0.10080409774596	0.10080409774596\\
60.125	0.1662	0.127758860216486	0.127758860216486\\
60.125	0.16986	0.155769773134898	0.155769773134898\\
60.125	0.17352	0.184836836501199	0.184836836501199\\
60.125	0.17718	0.214960050315387	0.214960050315387\\
60.125	0.18084	0.246139414577463	0.246139414577463\\
60.125	0.1845	0.278374929287425	0.278374929287425\\
60.125	0.18816	0.311666594445276	0.311666594445276\\
60.125	0.19182	0.346014410051013	0.346014410051013\\
60.125	0.19548	0.38141837610464	0.38141837610464\\
60.125	0.19914	0.417878492606155	0.417878492606155\\
60.125	0.2028	0.455394759555555	0.455394759555555\\
60.125	0.20646	0.493967176952847	0.493967176952847\\
60.125	0.21012	0.533595744798021	0.533595744798021\\
60.125	0.21378	0.574280463091086	0.574280463091086\\
60.125	0.21744	0.616021331832038	0.616021331832038\\
60.125	0.2211	0.658818351020878	0.658818351020878\\
60.125	0.22476	0.702671520657603	0.702671520657603\\
60.125	0.22842	0.747580840742218	0.747580840742218\\
60.125	0.23208	0.793546311274721	0.793546311274721\\
60.125	0.23574	0.840567932255109	0.840567932255109\\
60.125	0.2394	0.888645703683383	0.888645703683383\\
60.125	0.24306	0.937779625559549	0.937779625559549\\
60.125	0.24672	0.987969697883603	0.987969697883603\\
60.125	0.25038	1.03921592065554	1.03921592065554\\
60.125	0.25404	1.09151829387537	1.09151829387537\\
60.125	0.2577	1.14487681754309	1.14487681754309\\
60.125	0.26136	1.19929149165869	1.19929149165869\\
60.125	0.26502	1.25476231622218	1.25476231622218\\
60.125	0.26868	1.31128929123356	1.31128929123356\\
60.125	0.27234	1.36887241669282	1.36887241669282\\
60.125	0.276	1.42751169259998	1.42751169259998\\
60.5	0.093	-0.208510312305111	-0.208510312305111\\
60.5	0.09666	-0.203543901288037	-0.203543901288037\\
60.5	0.10032	-0.197521339823077	-0.197521339823077\\
60.5	0.10398	-0.190442627910228	-0.190442627910228\\
60.5	0.10764	-0.182307765549492	-0.182307765549492\\
60.5	0.1113	-0.173116752740867	-0.173116752740867\\
60.5	0.11496	-0.162869589484355	-0.162869589484355\\
60.5	0.11862	-0.151566275779956	-0.151566275779956\\
60.5	0.12228	-0.139206811627669	-0.139206811627669\\
60.5	0.12594	-0.125791197027494	-0.125791197027494\\
60.5	0.1296	-0.111319431979432	-0.111319431979432\\
60.5	0.13326	-0.0957915164834815	-0.0957915164834815\\
60.5	0.13692	-0.079207450539644	-0.079207450539644\\
60.5	0.14058	-0.0615672341479202	-0.0615672341479202\\
60.5	0.14424	-0.0428708673083074	-0.0428708673083074\\
60.5	0.1479	-0.023118350020807	-0.023118350020807\\
60.5	0.15156	-0.00230968228541895	-0.00230968228541895\\
60.5	0.15522	0.0195551358978567	0.0195551358978567\\
60.5	0.15888	0.0424761045290204	0.0424761045290204\\
60.5	0.16254	0.0664532236080713	0.0664532236080713\\
60.5	0.1662	0.0914864931350079	0.0914864931350079\\
60.5	0.16986	0.117575913109834	0.117575913109834\\
60.5	0.17352	0.144721483532548	0.144721483532548\\
60.5	0.17718	0.172923204403149	0.172923204403149\\
60.5	0.18084	0.202181075721638	0.202181075721638\\
60.5	0.1845	0.232495097488014	0.232495097488014\\
60.5	0.18816	0.263865269702279	0.263865269702279\\
60.5	0.19182	0.296291592364431	0.296291592364431\\
60.5	0.19548	0.32977406547447	0.32977406547447\\
60.5	0.19914	0.364312689032397	0.364312689032397\\
60.5	0.2028	0.39990746303821	0.39990746303821\\
60.5	0.20646	0.436558387491914	0.436558387491914\\
60.5	0.21012	0.474265462393504	0.474265462393504\\
60.5	0.21378	0.513028687742981	0.513028687742981\\
60.5	0.21744	0.552848063540346	0.552848063540346\\
60.5	0.2211	0.593723589785599	0.593723589785599\\
60.5	0.22476	0.635655266478739	0.635655266478739\\
60.5	0.22842	0.678643093619767	0.678643093619767\\
60.5	0.23208	0.722687071208683	0.722687071208683\\
60.5	0.23574	0.767787199245483	0.767787199245483\\
60.5	0.2394	0.813943477730173	0.813943477730173\\
60.5	0.24306	0.861155906662752	0.861155906662752\\
60.5	0.24672	0.909424486043218	0.909424486043218\\
60.5	0.25038	0.958749215871572	0.958749215871572\\
60.5	0.25404	1.00913009614781	1.00913009614781\\
60.5	0.2577	1.06056712687194	1.06056712687194\\
60.5	0.26136	1.11306030804396	1.11306030804396\\
60.5	0.26502	1.16660963966386	1.16660963966386\\
60.5	0.26868	1.22121512173165	1.22121512173165\\
60.5	0.27234	1.27687675424733	1.27687675424733\\
60.5	0.276	1.3335945372109	1.3335945372109\\
60.875	0.093	-0.205575447415579	-0.205575447415579\\
60.875	0.09666	-0.202530529342093	-0.202530529342093\\
60.875	0.10032	-0.198429460820719	-0.198429460820719\\
60.875	0.10398	-0.193272241851455	-0.193272241851455\\
60.875	0.10764	-0.187058872434306	-0.187058872434306\\
60.875	0.1113	-0.179789352569268	-0.179789352569268\\
60.875	0.11496	-0.171463682256343	-0.171463682256343\\
60.875	0.11862	-0.162081861495529	-0.162081861495529\\
60.875	0.12228	-0.15164389028683	-0.15164389028683\\
60.875	0.12594	-0.140149768630243	-0.140149768630243\\
60.875	0.1296	-0.127599496525766	-0.127599496525766\\
60.875	0.13326	-0.113993073973404	-0.113993073973404\\
60.875	0.13692	-0.0993305009731519	-0.0993305009731519\\
60.875	0.14058	-0.0836117775250139	-0.0836117775250139\\
60.875	0.14424	-0.0668369036289886	-0.0668369036289886\\
60.875	0.1479	-0.0490058792850758	-0.0490058792850758\\
60.875	0.15156	-0.0301187044932736	-0.0301187044932736\\
60.875	0.15522	-0.0101753792535855	-0.0101753792535855\\
60.875	0.15888	0.0108240964339906	0.0108240964339906\\
60.875	0.16254	0.0328797225694557	0.0328797225694557\\
60.875	0.1662	0.0559914991528065	0.0559914991528065\\
60.875	0.16986	0.080159426184045	0.080159426184045\\
60.875	0.17352	0.105383503663173	0.105383503663173\\
60.875	0.17718	0.131663731590187	0.131663731590187\\
60.875	0.18084	0.159000109965088	0.159000109965088\\
60.875	0.1845	0.187392638787879	0.187392638787879\\
60.875	0.18816	0.216841318058556	0.216841318058556\\
60.875	0.19182	0.24734614777712	0.24734614777712\\
60.875	0.19548	0.278907127943572	0.278907127943572\\
60.875	0.19914	0.311524258557915	0.311524258557915\\
60.875	0.2028	0.34519753962014	0.34519753962014\\
60.875	0.20646	0.379926971130257	0.379926971130257\\
60.875	0.21012	0.415712553088259	0.415712553088259\\
60.875	0.21378	0.452554285494152	0.452554285494152\\
60.875	0.21744	0.49045216834793	0.49045216834793\\
60.875	0.2211	0.529406201649595	0.529406201649595\\
60.875	0.22476	0.569416385399148	0.569416385399148\\
60.875	0.22842	0.610482719596588	0.610482719596588\\
60.875	0.23208	0.652605204241919	0.652605204241919\\
60.875	0.23574	0.695783839335132	0.695783839335132\\
60.875	0.2394	0.740018624876235	0.740018624876235\\
60.875	0.24306	0.785309560865226	0.785309560865226\\
60.875	0.24672	0.831656647302108	0.831656647302108\\
60.875	0.25038	0.879059884186874	0.879059884186874\\
60.875	0.25404	0.927519271519528	0.927519271519528\\
60.875	0.2577	0.977034809300072	0.977034809300072\\
60.875	0.26136	1.0276064975285	1.0276064975285\\
60.875	0.26502	1.07923433620482	1.07923433620482\\
60.875	0.26868	1.13191832532902	1.13191832532902\\
60.875	0.27234	1.18565846490111	1.18565846490111\\
60.875	0.276	1.2404547549211	1.2404547549211\\
61.25	0.093	-0.201863209426771	-0.201863209426771\\
61.25	0.09666	-0.200739784296873	-0.200739784296873\\
61.25	0.10032	-0.198560208719083	-0.198560208719083\\
61.25	0.10398	-0.195324482693409	-0.195324482693409\\
61.25	0.10764	-0.191032606219845	-0.191032606219845\\
61.25	0.1113	-0.185684579298394	-0.185684579298394\\
61.25	0.11496	-0.179280401929056	-0.179280401929056\\
61.25	0.11862	-0.17182007411183	-0.17182007411183\\
61.25	0.12228	-0.163303595846716	-0.163303595846716\\
61.25	0.12594	-0.153730967133714	-0.153730967133714\\
61.25	0.1296	-0.143102187972825	-0.143102187972825\\
61.25	0.13326	-0.131417258364049	-0.131417258364049\\
61.25	0.13692	-0.118676178307386	-0.118676178307386\\
61.25	0.14058	-0.104878947802834	-0.104878947802834\\
61.25	0.14424	-0.0900255668503946	-0.0900255668503946\\
61.25	0.1479	-0.0741160354500676	-0.0741160354500676\\
61.25	0.15156	-0.0571503536018547	-0.0571503536018547\\
61.25	0.15522	-0.0391285213057524	-0.0391285213057524\\
61.25	0.15888	-0.0200505385617621	-0.0200505385617621\\
61.25	0.16254	8.35946301154245e-05	8.35946301154245e-05\\
61.25	0.1662	0.0212738782698787	0.0212738782698787\\
61.25	0.16986	0.0435203123575314	0.0435203123575314\\
61.25	0.17352	0.0668228968930717	0.0668228968930717\\
61.25	0.17718	0.0911816318764997	0.0911816318764997\\
61.25	0.18084	0.116596517307815	0.116596517307815\\
61.25	0.1845	0.143067553187018	0.143067553187018\\
61.25	0.18816	0.170594739514108	0.170594739514108\\
61.25	0.19182	0.199178076289088	0.199178076289088\\
61.25	0.19548	0.228817563511952	0.228817563511952\\
61.25	0.19914	0.259513201182704	0.259513201182704\\
61.25	0.2028	0.291264989301346	0.291264989301346\\
61.25	0.20646	0.324072927867878	0.324072927867878\\
61.25	0.21012	0.357937016882293	0.357937016882293\\
61.25	0.21378	0.392857256344595	0.392857256344595\\
61.25	0.21744	0.428833646254788	0.428833646254788\\
61.25	0.2211	0.465866186612866	0.465866186612866\\
61.25	0.22476	0.503954877418835	0.503954877418835\\
61.25	0.22842	0.543099718672687	0.543099718672687\\
61.25	0.23208	0.583300710374428	0.583300710374428\\
61.25	0.23574	0.624557852524056	0.624557852524056\\
61.25	0.2394	0.666871145121575	0.666871145121575\\
61.25	0.24306	0.710240588166978	0.710240588166978\\
61.25	0.24672	0.754666181660273	0.754666181660273\\
61.25	0.25038	0.800147925601452	0.800147925601452\\
61.25	0.25404	0.846685819990521	0.846685819990521\\
61.25	0.2577	0.894279864827475	0.894279864827475\\
61.25	0.26136	0.94293006011232	0.94293006011232\\
61.25	0.26502	0.992636405845048	0.992636405845048\\
61.25	0.26868	1.04339890202567	1.04339890202567\\
61.25	0.27234	1.09521754865417	1.09521754865417\\
61.25	0.276	1.14809234573056	1.14809234573056\\
61.625	0.093	-0.197373598338689	-0.197373598338689\\
61.625	0.09666	-0.198171666152377	-0.198171666152377\\
61.625	0.10032	-0.197913583518175	-0.197913583518175\\
61.625	0.10398	-0.196599350436087	-0.196599350436087\\
61.625	0.10764	-0.194228966906111	-0.194228966906111\\
61.625	0.1113	-0.190802432928246	-0.190802432928246\\
61.625	0.11496	-0.186319748502495	-0.186319748502495\\
61.625	0.11862	-0.180780913628854	-0.180780913628854\\
61.625	0.12228	-0.174185928307328	-0.174185928307328\\
61.625	0.12594	-0.166534792537914	-0.166534792537914\\
61.625	0.1296	-0.157827506320613	-0.157827506320613\\
61.625	0.13326	-0.148064069655422	-0.148064069655422\\
61.625	0.13692	-0.137244482542345	-0.137244482542345\\
61.625	0.14058	-0.125368744981381	-0.125368744981381\\
61.625	0.14424	-0.112436856972529	-0.112436856972529\\
61.625	0.1479	-0.0984488185157877	-0.0984488185157877\\
61.625	0.15156	-0.0834046296111606	-0.0834046296111606\\
61.625	0.15522	-0.0673042902586458	-0.0673042902586458\\
61.625	0.15888	-0.0501478004582431	-0.0501478004582431\\
61.625	0.16254	-0.0319351602099531	-0.0319351602099531\\
61.625	0.1662	-0.0126663695137739	-0.0126663695137739\\
61.625	0.16986	0.00765857163029127	0.00765857163029127\\
61.625	0.17352	0.029039663222244	0.029039663222244\\
61.625	0.17718	0.0514769052620845	0.0514769052620845\\
61.625	0.18084	0.0749702977498141	0.0749702977498141\\
61.625	0.1845	0.0995198406854299	0.0995198406854299\\
61.625	0.18816	0.125125534068932	0.125125534068932\\
61.625	0.19182	0.151787377900324	0.151787377900324\\
61.625	0.19548	0.179505372179604	0.179505372179604\\
61.625	0.19914	0.208279516906769	0.208279516906769\\
61.625	0.2028	0.238109812081823	0.238109812081823\\
61.625	0.20646	0.268996257704768	0.268996257704768\\
61.625	0.21012	0.300938853775598	0.300938853775598\\
61.625	0.21378	0.333937600294313	0.333937600294313\\
61.625	0.21744	0.367992497260919	0.367992497260919\\
61.625	0.2211	0.403103544675412	0.403103544675412\\
61.625	0.22476	0.43927074253779	0.43927074253779\\
61.625	0.22842	0.476494090848059	0.476494090848059\\
61.625	0.23208	0.514773589606215	0.514773589606215\\
61.625	0.23574	0.554109238812253	0.554109238812253\\
61.625	0.2394	0.594501038466184	0.594501038466184\\
61.625	0.24306	0.635948988568003	0.635948988568003\\
61.625	0.24672	0.678453089117706	0.678453089117706\\
61.625	0.25038	0.722013340115301	0.722013340115301\\
61.625	0.25404	0.766629741560783	0.766629741560783\\
61.625	0.2577	0.812302293454153	0.812302293454153\\
61.625	0.26136	0.859030995795406	0.859030995795406\\
61.625	0.26502	0.906815848584551	0.906815848584551\\
61.625	0.26868	0.955656851821584	0.955656851821584\\
61.625	0.27234	1.0055540055065	1.0055540055065\\
61.625	0.276	1.05650730963931	1.05650730963931\\
62	0.093	-0.192106614151332	-0.192106614151332\\
62	0.09666	-0.194826174908605	-0.194826174908605\\
62	0.10032	-0.196489585217992	-0.196489585217992\\
62	0.10398	-0.19709684507949	-0.19709684507949\\
62	0.10764	-0.1966479544931	-0.1966479544931\\
62	0.1113	-0.195142913458822	-0.195142913458822\\
62	0.11496	-0.192581721976659	-0.192581721976659\\
62	0.11862	-0.188964380046606	-0.188964380046606\\
62	0.12228	-0.184290887668665	-0.184290887668665\\
62	0.12594	-0.178561244842837	-0.178561244842837\\
62	0.1296	-0.171775451569123	-0.171775451569123\\
62	0.13326	-0.16393350784752	-0.16393350784752\\
62	0.13692	-0.155035413678029	-0.155035413678029\\
62	0.14058	-0.145081169060652	-0.145081169060652\\
62	0.14424	-0.134070773995386	-0.134070773995386\\
62	0.1479	-0.122004228482232	-0.122004228482232\\
62	0.15156	-0.108881532521193	-0.108881532521193\\
62	0.15522	-0.094702686112264	-0.094702686112264\\
62	0.15888	-0.079467689255447	-0.079467689255447\\
62	0.16254	-0.0631765419507446	-0.0631765419507446\\
62	0.1662	-0.0458292441981529	-0.0458292441981529\\
62	0.16986	-0.0274257959976736	-0.0274257959976736\\
62	0.17352	-0.00796619734930659	-0.00796619734930659\\
62	0.17718	0.0125495517469463	0.0125495517469463\\
62	0.18084	0.0341214512910883	0.0341214512910883\\
62	0.1845	0.0567495012831165	0.0567495012831165\\
62	0.18816	0.0804337017230361	0.0804337017230361\\
62	0.19182	0.105174052610841	0.105174052610841\\
62	0.19548	0.13097055394653	0.13097055394653\\
62	0.19914	0.157823205730111	0.157823205730111\\
62	0.2028	0.185732007961577	0.185732007961577\\
62	0.20646	0.214696960640934	0.214696960640934\\
62	0.21012	0.244718063768177	0.244718063768177\\
62	0.21378	0.275795317343308	0.275795317343308\\
62	0.21744	0.307928721366326	0.307928721366326\\
62	0.2211	0.341118275837232	0.341118275837232\\
62	0.22476	0.375363980756026	0.375363980756026\\
62	0.22842	0.410665836122707	0.410665836122707\\
62	0.23208	0.447023841937272	0.447023841937272\\
62	0.23574	0.484437998199729	0.484437998199729\\
62	0.2394	0.522908304910073	0.522908304910073\\
62	0.24306	0.562434762068301	0.562434762068301\\
62	0.24672	0.60301736967442	0.60301736967442\\
62	0.25038	0.644656127728428	0.644656127728428\\
62	0.25404	0.687351036230322	0.687351036230322\\
62	0.2577	0.731102095180104	0.731102095180104\\
62	0.26136	0.775909304577774	0.775909304577774\\
62	0.26502	0.821772664423331	0.821772664423331\\
62	0.26868	0.868692174716776	0.868692174716776\\
62	0.27234	0.916667835458108	0.916667835458108\\
62	0.276	0.965699646647328	0.965699646647328\\
62.375	0.093	-0.186062256864699	-0.186062256864699\\
62.375	0.09666	-0.190703310565559	-0.190703310565559\\
62.375	0.10032	-0.194288213818532	-0.194288213818532\\
62.375	0.10398	-0.196816966623617	-0.196816966623617\\
62.375	0.10764	-0.198289568980813	-0.198289568980813\\
62.375	0.1113	-0.198706020890123	-0.198706020890123\\
62.375	0.11496	-0.198066322351545	-0.198066322351545\\
62.375	0.11862	-0.196370473365078	-0.196370473365078\\
62.375	0.12228	-0.193618473930725	-0.193618473930725\\
62.375	0.12594	-0.189810324048485	-0.189810324048485\\
62.375	0.1296	-0.184946023718357	-0.184946023718357\\
62.375	0.13326	-0.179025572940341	-0.179025572940341\\
62.375	0.13692	-0.172048971714436	-0.172048971714436\\
62.375	0.14058	-0.164016220040647	-0.164016220040647\\
62.375	0.14424	-0.154927317918966	-0.154927317918966\\
62.375	0.1479	-0.1447822653494	-0.1447822653494\\
62.375	0.15156	-0.133581062331946	-0.133581062331946\\
62.375	0.15522	-0.121323708866605	-0.121323708866605\\
62.375	0.15888	-0.108010204953376	-0.108010204953376\\
62.375	0.16254	-0.0936405505922591	-0.0936405505922591\\
62.375	0.1662	-0.0782147457832532	-0.0782147457832532\\
62.375	0.16986	-0.0617327905263614	-0.0617327905263614\\
62.375	0.17352	-0.044194684821582	-0.044194684821582\\
62.375	0.17718	-0.0256004286689149	-0.0256004286689149\\
62.375	0.18084	-0.00595002206836037	-0.00595002206836037\\
62.375	0.1845	0.014756534980082	0.014756534980082\\
62.375	0.18816	0.036519242476414	0.036519242476414\\
62.375	0.19182	0.0593381004206315	0.0593381004206315\\
62.375	0.19548	0.0832131088127366	0.0832131088127366\\
62.375	0.19914	0.108144267652729	0.108144267652729\\
62.375	0.2028	0.134131576940612	0.134131576940612\\
62.375	0.20646	0.161175036676382	0.161175036676382\\
62.375	0.21012	0.189274646860037	0.189274646860037\\
62.375	0.21378	0.21843040749158	0.21843040749158\\
62.375	0.21744	0.248642318571011	0.248642318571011\\
62.375	0.2211	0.279910380098329	0.279910380098329\\
62.375	0.22476	0.312234592073535	0.312234592073535\\
62.375	0.22842	0.345614954496629	0.345614954496629\\
62.375	0.23208	0.38005146736761	0.38005146736761\\
62.375	0.23574	0.41554413068648	0.41554413068648\\
62.375	0.2394	0.452092944453236	0.452092944453236\\
62.375	0.24306	0.48969790866788	0.48969790866788\\
62.375	0.24672	0.528359023330411	0.528359023330411\\
62.375	0.25038	0.568076288440835	0.568076288440835\\
62.375	0.25404	0.608849703999141	0.608849703999141\\
62.375	0.2577	0.650679270005336	0.650679270005336\\
62.375	0.26136	0.693564986459418	0.693564986459418\\
62.375	0.26502	0.737506853361388	0.737506853361388\\
62.375	0.26868	0.782504870711245	0.782504870711245\\
62.375	0.27234	0.82855903850899	0.82855903850899\\
62.375	0.276	0.875669356754623	0.875669356754623\\
62.75	0.093	-0.179240526478792	-0.179240526478792\\
62.75	0.09666	-0.18580307312324	-0.18580307312324\\
62.75	0.10032	-0.191309469319799	-0.191309469319799\\
62.75	0.10398	-0.19575971506847	-0.19575971506847\\
62.75	0.10764	-0.199153810369253	-0.199153810369253\\
62.75	0.1113	-0.201491755222149	-0.201491755222149\\
62.75	0.11496	-0.202773549627159	-0.202773549627159\\
62.75	0.11862	-0.202999193584279	-0.202999193584279\\
62.75	0.12228	-0.202168687093512	-0.202168687093512\\
62.75	0.12594	-0.200282030154859	-0.200282030154859\\
62.75	0.1296	-0.197339222768317	-0.197339222768317\\
62.75	0.13326	-0.193340264933887	-0.193340264933887\\
62.75	0.13692	-0.188285156651571	-0.188285156651571\\
62.75	0.14058	-0.182173897921366	-0.182173897921366\\
62.75	0.14424	-0.175006488743275	-0.175006488743275\\
62.75	0.1479	-0.166782929117294	-0.166782929117294\\
62.75	0.15156	-0.157503219043426	-0.157503219043426\\
62.75	0.15522	-0.147167358521673	-0.147167358521673\\
62.75	0.15888	-0.135775347552029	-0.135775347552029\\
62.75	0.16254	-0.1233271861345	-0.1233271861345\\
62.75	0.1662	-0.109822874269082	-0.109822874269082\\
62.75	0.16986	-0.0952624119557757	-0.0952624119557757\\
62.75	0.17352	-0.0796457991945838	-0.0796457991945838\\
62.75	0.17718	-0.0629730359855025	-0.0629730359855025\\
62.75	0.18084	-0.0452441223285338	-0.0452441223285338\\
62.75	0.1845	-0.026459058223679	-0.026459058223679\\
62.75	0.18816	-0.00661784367093454	-0.00661784367093454\\
62.75	0.19182	0.0142795213296953	0.0142795213296953\\
62.75	0.19548	0.0362330367782164	0.0362330367782164\\
62.75	0.19914	0.0592427026746216	0.0592427026746216\\
62.75	0.2028	0.0833085190189164	0.0833085190189164\\
62.75	0.20646	0.108430485811099	0.108430485811099\\
62.75	0.21012	0.134608603051167	0.134608603051167\\
62.75	0.21378	0.161842870739126	0.161842870739126\\
62.75	0.21744	0.190133288874969	0.190133288874969\\
62.75	0.2211	0.2194798574587	0.2194798574587\\
62.75	0.22476	0.249882576490322	0.249882576490322\\
62.75	0.22842	0.281341445969828	0.281341445969828\\
62.75	0.23208	0.313856465897221	0.313856465897221\\
62.75	0.23574	0.347427636272503	0.347427636272503\\
62.75	0.2394	0.382054957095672	0.382054957095672\\
62.75	0.24306	0.417738428366732	0.417738428366732\\
62.75	0.24672	0.454478050085676	0.454478050085676\\
62.75	0.25038	0.492273822252512	0.492273822252512\\
62.75	0.25404	0.531125744867231	0.531125744867231\\
62.75	0.2577	0.571033817929838	0.571033817929838\\
62.75	0.26136	0.611998041440336	0.611998041440336\\
62.75	0.26502	0.654018415398718	0.654018415398718\\
62.75	0.26868	0.697094939804988	0.697094939804988\\
62.75	0.27234	0.741227614659149	0.741227614659149\\
62.75	0.276	0.786416439961194	0.786416439961194\\
63.125	0.093	-0.17164142299361	-0.17164142299361\\
63.125	0.09666	-0.180125462581642	-0.180125462581642\\
63.125	0.10032	-0.18755335172179	-0.18755335172179\\
63.125	0.10398	-0.193925090414047	-0.193925090414047\\
63.125	0.10764	-0.199240678658418	-0.199240678658418\\
63.125	0.1113	-0.203500116454901	-0.203500116454901\\
63.125	0.11496	-0.206703403803496	-0.206703403803496\\
63.125	0.11862	-0.208850540704205	-0.208850540704205\\
63.125	0.12228	-0.209941527157023	-0.209941527157023\\
63.125	0.12594	-0.209976363161956	-0.209976363161956\\
63.125	0.1296	-0.208955048719002	-0.208955048719002\\
63.125	0.13326	-0.206877583828159	-0.206877583828159\\
63.125	0.13692	-0.203743968489429	-0.203743968489429\\
63.125	0.14058	-0.199554202702811	-0.199554202702811\\
63.125	0.14424	-0.194308286468306	-0.194308286468306\\
63.125	0.1479	-0.188006219785913	-0.188006219785913\\
63.125	0.15156	-0.180648002655633	-0.180648002655633\\
63.125	0.15522	-0.172233635077463	-0.172233635077463\\
63.125	0.15888	-0.162763117051409	-0.162763117051409\\
63.125	0.16254	-0.152236448577466	-0.152236448577466\\
63.125	0.1662	-0.140653629655633	-0.140653629655633\\
63.125	0.16986	-0.128014660285915	-0.128014660285915\\
63.125	0.17352	-0.114319540468309	-0.114319540468309\\
63.125	0.17718	-0.0995682702028149	-0.0995682702028149\\
63.125	0.18084	-0.0837608494894337	-0.0837608494894337\\
63.125	0.1845	-0.0668972783281647	-0.0668972783281647\\
63.125	0.18816	-0.0489775567190078	-0.0489775567190078\\
63.125	0.19182	-0.030001684661962	-0.030001684661962\\
63.125	0.19548	-0.00996966215703199	-0.00996966215703199\\
63.125	0.19914	0.0111185107957892	0.0111185107957892\\
63.125	0.2028	0.0332628341964964	0.0332628341964964\\
63.125	0.20646	0.0564633080450911	0.0564633080450911\\
63.125	0.21012	0.080719932341575	0.080719932341575\\
63.125	0.21378	0.106032707085943	0.106032707085943\\
63.125	0.21744	0.132401632278202	0.132401632278202\\
63.125	0.2211	0.159826707918349	0.159826707918349\\
63.125	0.22476	0.18830793400638	0.18830793400638\\
63.125	0.22842	0.217845310542302	0.217845310542302\\
63.125	0.23208	0.248438837526108	0.248438837526108\\
63.125	0.23574	0.280088514957802	0.280088514957802\\
63.125	0.2394	0.312794342837387	0.312794342837387\\
63.125	0.24306	0.346556321164856	0.346556321164856\\
63.125	0.24672	0.381374449940216	0.381374449940216\\
63.125	0.25038	0.417248729163464	0.417248729163464\\
63.125	0.25404	0.454179158834596	0.454179158834596\\
63.125	0.2577	0.492165738953619	0.492165738953619\\
63.125	0.26136	0.531208469520525	0.531208469520525\\
63.125	0.26502	0.571307350535323	0.571307350535323\\
63.125	0.26868	0.61246238199801	0.61246238199801\\
63.125	0.27234	0.654673563908579	0.654673563908579\\
63.125	0.276	0.69794089626704	0.69794089626704\\
63.5	0.093	-0.163264946409153	-0.163264946409153\\
63.5	0.09666	-0.173670478940772	-0.173670478940772\\
63.5	0.10032	-0.183019861024504	-0.183019861024504\\
63.5	0.10398	-0.19131309266035	-0.19131309266035\\
63.5	0.10764	-0.198550173848307	-0.198550173848307\\
63.5	0.1113	-0.204731104588376	-0.204731104588376\\
63.5	0.11496	-0.209855884880559	-0.209855884880559\\
63.5	0.11862	-0.213924514724853	-0.213924514724853\\
63.5	0.12228	-0.216936994121259	-0.216936994121259\\
63.5	0.12594	-0.21889332306978	-0.21889332306978\\
63.5	0.1296	-0.219793501570411	-0.219793501570411\\
63.5	0.13326	-0.219637529623156	-0.219637529623156\\
63.5	0.13692	-0.218425407228012	-0.218425407228012\\
63.5	0.14058	-0.216157134384982	-0.216157134384982\\
63.5	0.14424	-0.212832711094062	-0.212832711094062\\
63.5	0.1479	-0.208452137355257	-0.208452137355257\\
63.5	0.15156	-0.203015413168562	-0.203015413168562\\
63.5	0.15522	-0.19652253853398	-0.19652253853398\\
63.5	0.15888	-0.188973513451512	-0.188973513451512\\
63.5	0.16254	-0.180368337921154	-0.180368337921154\\
63.5	0.1662	-0.170707011942911	-0.170707011942911\\
63.5	0.16986	-0.159989535516778	-0.159989535516778\\
63.5	0.17352	-0.14821590864276	-0.14821590864276\\
63.5	0.17718	-0.135386131320852	-0.135386131320852\\
63.5	0.18084	-0.121500203551058	-0.121500203551058\\
63.5	0.1845	-0.106558125333375	-0.106558125333375\\
63.5	0.18816	-0.0905598966678058	-0.0905598966678058\\
63.5	0.19182	-0.0735055175543475	-0.0735055175543475\\
63.5	0.19548	-0.0553949879930016	-0.0553949879930016\\
63.5	0.19914	-0.036228307983768	-0.036228307983768\\
63.5	0.2028	-0.0160054775266483	-0.0160054775266483\\
63.5	0.20646	0.00527350337836241	0.00527350337836241\\
63.5	0.21012	0.0276086347312552	0.0276086347312552\\
63.5	0.21378	0.0509999165320392	0.0509999165320392\\
63.5	0.21744	0.0754473487807106	0.0754473487807106\\
63.5	0.2211	0.10095093147727	0.10095093147727\\
63.5	0.22476	0.127510664621717	0.127510664621717\\
63.5	0.22842	0.155126548214048	0.155126548214048\\
63.5	0.23208	0.18379858225427	0.18379858225427\\
63.5	0.23574	0.213526766742377	0.213526766742377\\
63.5	0.2394	0.244311101678373	0.244311101678373\\
63.5	0.24306	0.276151587062258	0.276151587062258\\
63.5	0.24672	0.309048222894031	0.309048222894031\\
63.5	0.25038	0.343001009173692	0.343001009173692\\
63.5	0.25404	0.378009945901239	0.378009945901239\\
63.5	0.2577	0.414075033076671	0.414075033076671\\
63.5	0.26136	0.451196270699994	0.451196270699994\\
63.5	0.26502	0.489373658771204	0.489373658771204\\
63.5	0.26868	0.528607197290303	0.528607197290303\\
63.5	0.27234	0.568896886257289	0.568896886257289\\
63.5	0.276	0.610242725672158	0.610242725672158\\
63.875	0.093	-0.154111096725422	-0.154111096725422\\
63.875	0.09666	-0.166438122200629	-0.166438122200629\\
63.875	0.10032	-0.177708997227947	-0.177708997227947\\
63.875	0.10398	-0.187923721807378	-0.187923721807378\\
63.875	0.10764	-0.197082295938923	-0.197082295938923\\
63.875	0.1113	-0.205184719622579	-0.205184719622579\\
63.875	0.11496	-0.212230992858348	-0.212230992858348\\
63.875	0.11862	-0.21822111564623	-0.21822111564623\\
63.875	0.12228	-0.223155087986224	-0.223155087986224\\
63.875	0.12594	-0.227032909878328	-0.227032909878328\\
63.875	0.1296	-0.229854581322547	-0.229854581322547\\
63.875	0.13326	-0.231620102318878	-0.231620102318878\\
63.875	0.13692	-0.232329472867321	-0.232329472867321\\
63.875	0.14058	-0.231982692967878	-0.231982692967878\\
63.875	0.14424	-0.230579762620545	-0.230579762620545\\
63.875	0.1479	-0.228120681825325	-0.228120681825325\\
63.875	0.15156	-0.224605450582218	-0.224605450582218\\
63.875	0.15522	-0.220034068891224	-0.220034068891224\\
63.875	0.15888	-0.214406536752343	-0.214406536752343\\
63.875	0.16254	-0.207722854165571	-0.207722854165571\\
63.875	0.1662	-0.199983021130914	-0.199983021130914\\
63.875	0.16986	-0.191187037648369	-0.191187037648369\\
63.875	0.17352	-0.181334903717936	-0.181334903717936\\
63.875	0.17718	-0.170426619339616	-0.170426619339616\\
63.875	0.18084	-0.158462184513408	-0.158462184513408\\
63.875	0.1845	-0.145441599239312	-0.145441599239312\\
63.875	0.18816	-0.13136486351733	-0.13136486351733\\
63.875	0.19182	-0.11623197734746	-0.11623197734746\\
63.875	0.19548	-0.100042940729701	-0.100042940729701\\
63.875	0.19914	-0.0827977536640552	-0.0827977536640552\\
63.875	0.2028	-0.0644964161505195	-0.0644964161505195\\
63.875	0.20646	-0.0451389281890964	-0.0451389281890964\\
63.875	0.21012	-0.0247252897797876	-0.0247252897797876\\
63.875	0.21378	-0.00325550092259119	-0.00325550092259119\\
63.875	0.21744	0.0192704383824926	0.0192704383824926\\
63.875	0.2211	0.0428525281354644	0.0428525281354644\\
63.875	0.22476	0.0674907683363237	0.0674907683363237\\
63.875	0.22842	0.0931851589850705	0.0931851589850705\\
63.875	0.23208	0.119935700081705	0.119935700081705\\
63.875	0.23574	0.147742391626224	0.147742391626224\\
63.875	0.2394	0.176605233618637	0.176605233618637\\
63.875	0.24306	0.206524226058935	0.206524226058935\\
63.875	0.24672	0.23749936894712	0.23749936894712\\
63.875	0.25038	0.269530662283193	0.269530662283193\\
63.875	0.25404	0.302618106067153	0.302618106067153\\
63.875	0.2577	0.336761700299001	0.336761700299001\\
63.875	0.26136	0.371961444978736	0.371961444978736\\
63.875	0.26502	0.408217340106359	0.408217340106359\\
63.875	0.26868	0.44552938568187	0.44552938568187\\
63.875	0.27234	0.483897581705268	0.483897581705268\\
63.875	0.276	0.523321928176554	0.523321928176554\\
64.25	0.093	-0.144179873942414	-0.144179873942414\\
64.25	0.09666	-0.158428392361205	-0.158428392361205\\
64.25	0.10032	-0.171620760332112	-0.171620760332112\\
64.25	0.10398	-0.183756977855131	-0.183756977855131\\
64.25	0.10764	-0.194837044930261	-0.194837044930261\\
64.25	0.1113	-0.204860961557504	-0.204860961557504\\
64.25	0.11496	-0.21382872773686	-0.21382872773686\\
64.25	0.11862	-0.221740343468328	-0.221740343468328\\
64.25	0.12228	-0.228595808751909	-0.228595808751909\\
64.25	0.12594	-0.234395123587601	-0.234395123587601\\
64.25	0.1296	-0.239138287975406	-0.239138287975406\\
64.25	0.13326	-0.242825301915324	-0.242825301915324\\
64.25	0.13692	-0.245456165407353	-0.245456165407353\\
64.25	0.14058	-0.247030878451496	-0.247030878451496\\
64.25	0.14424	-0.247549441047752	-0.247549441047752\\
64.25	0.1479	-0.247011853196119	-0.247011853196119\\
64.25	0.15156	-0.245418114896599	-0.245418114896599\\
64.25	0.15522	-0.24276822614919	-0.24276822614919\\
64.25	0.15888	-0.239062186953895	-0.239062186953895\\
64.25	0.16254	-0.234299997310711	-0.234299997310711\\
64.25	0.1662	-0.228481657219641	-0.228481657219641\\
64.25	0.16986	-0.221607166680682	-0.221607166680682\\
64.25	0.17352	-0.213676525693837	-0.213676525693837\\
64.25	0.17718	-0.204689734259102	-0.204689734259102\\
64.25	0.18084	-0.19464679237648	-0.19464679237648\\
64.25	0.1845	-0.183547700045972	-0.183547700045972\\
64.25	0.18816	-0.171392457267574	-0.171392457267574\\
64.25	0.19182	-0.158181064041291	-0.158181064041291\\
64.25	0.19548	-0.14391352036712	-0.14391352036712\\
64.25	0.19914	-0.128589826245062	-0.128589826245062\\
64.25	0.2028	-0.112209981675114	-0.112209981675114\\
64.25	0.20646	-0.0947739866572781	-0.0947739866572781\\
64.25	0.21012	-0.0762818411915569	-0.0762818411915569\\
64.25	0.21378	-0.0567335452779445	-0.0567335452779445\\
64.25	0.21744	-0.0361290989164482	-0.0361290989164482\\
64.25	0.2211	-0.0144685021070641	-0.0144685021070641\\
64.25	0.22476	0.00824824515021128	0.00824824515021128\\
64.25	0.22842	0.0320211428553705	0.0320211428553705\\
64.25	0.23208	0.0568501910084174	0.0568501910084174\\
64.25	0.23574	0.0827353896093528	0.0827353896093528\\
64.25	0.2394	0.109676738658175	0.109676738658175\\
64.25	0.24306	0.137674238154884	0.137674238154884\\
64.25	0.24672	0.166727888099482	0.166727888099482\\
64.25	0.25038	0.196837688491971	0.196837688491971\\
64.25	0.25404	0.228003639332343	0.228003639332343\\
64.25	0.2577	0.260225740620607	0.260225740620607\\
64.25	0.26136	0.293503992356755	0.293503992356755\\
64.25	0.26502	0.32783839454079	0.32783839454079\\
64.25	0.26868	0.363228947172713	0.363228947172713\\
64.25	0.27234	0.399675650252528	0.399675650252528\\
64.25	0.276	0.437178503780226	0.437178503780226\\
64.625	0.093	-0.133471278060132	-0.133471278060132\\
64.625	0.09666	-0.149641289422511	-0.149641289422511\\
64.625	0.10032	-0.164755150337006	-0.164755150337006\\
64.625	0.10398	-0.178812860803611	-0.178812860803611\\
64.625	0.10764	-0.191814420822329	-0.191814420822329\\
64.625	0.1113	-0.203759830393158	-0.203759830393158\\
64.625	0.11496	-0.214649089516101	-0.214649089516101\\
64.625	0.11862	-0.224482198191156	-0.224482198191156\\
64.625	0.12228	-0.233259156418323	-0.233259156418323\\
64.625	0.12594	-0.240979964197602	-0.240979964197602\\
64.625	0.1296	-0.247644621528994	-0.247644621528994\\
64.625	0.13326	-0.253253128412499	-0.253253128412499\\
64.625	0.13692	-0.257805484848116	-0.257805484848116\\
64.625	0.14058	-0.261301690835844	-0.261301690835844\\
64.625	0.14424	-0.263741746375686	-0.263741746375686\\
64.625	0.1479	-0.26512565146764	-0.26512565146764\\
64.625	0.15156	-0.265453406111706	-0.265453406111706\\
64.625	0.15522	-0.264725010307885	-0.264725010307885\\
64.625	0.15888	-0.262940464056177	-0.262940464056177\\
64.625	0.16254	-0.260099767356581	-0.260099767356581\\
64.625	0.1662	-0.256202920209097	-0.256202920209097\\
64.625	0.16986	-0.251249922613725	-0.251249922613725\\
64.625	0.17352	-0.245240774570466	-0.245240774570466\\
64.625	0.17718	-0.238175476079319	-0.238175476079319\\
64.625	0.18084	-0.230054027140282	-0.230054027140282\\
64.625	0.1845	-0.220876427753362	-0.220876427753362\\
64.625	0.18816	-0.210642677918552	-0.210642677918552\\
64.625	0.19182	-0.199352777635856	-0.199352777635856\\
64.625	0.19548	-0.187006726905269	-0.187006726905269\\
64.625	0.19914	-0.173604525726798	-0.173604525726798\\
64.625	0.2028	-0.159146174100438	-0.159146174100438\\
64.625	0.20646	-0.143631672026186	-0.143631672026186\\
64.625	0.21012	-0.127061019504053	-0.127061019504053\\
64.625	0.21378	-0.109434216534028	-0.109434216534028\\
64.625	0.21744	-0.0907512631161191	-0.0907512631161191\\
64.625	0.2211	-0.071012159250319	-0.071012159250319\\
64.625	0.22476	-0.0502169049366348	-0.0502169049366348\\
64.625	0.22842	-0.0283655001750596	-0.0283655001750596\\
64.625	0.23208	-0.00545794496560026	-0.00545794496560026\\
64.625	0.23574	0.0185057606917476	0.0185057606917476\\
64.625	0.2394	0.0435256167969817	0.0435256167969817\\
64.625	0.24306	0.0696016233501076	0.0696016233501076\\
64.625	0.24672	0.0967337803511175	0.0967337803511175\\
64.625	0.25038	0.124922087800019	0.124922087800019\\
64.625	0.25404	0.154166545696807	0.154166545696807\\
64.625	0.2577	0.18446715404148	0.18446715404148\\
64.625	0.26136	0.215823912834044	0.215823912834044\\
64.625	0.26502	0.248236822074491	0.248236822074491\\
64.625	0.26868	0.281705881762831	0.281705881762831\\
64.625	0.27234	0.316231091899054	0.316231091899054\\
64.625	0.276	0.351812452483168	0.351812452483168\\
65	0.093	-0.121985309078576	-0.121985309078576\\
65	0.09666	-0.140076813384542	-0.140076813384542\\
65	0.10032	-0.157112167242622	-0.157112167242622\\
65	0.10398	-0.173091370652813	-0.173091370652813\\
65	0.10764	-0.188014423615118	-0.188014423615118\\
65	0.1113	-0.201881326129534	-0.201881326129534\\
65	0.11496	-0.214692078196064	-0.214692078196064\\
65	0.11862	-0.226446679814705	-0.226446679814705\\
65	0.12228	-0.23714513098546	-0.23714513098546\\
65	0.12594	-0.246787431708325	-0.246787431708325\\
65	0.1296	-0.255373581983305	-0.255373581983305\\
65	0.13326	-0.262903581810395	-0.262903581810395\\
65	0.13692	-0.269377431189599	-0.269377431189599\\
65	0.14058	-0.274795130120915	-0.274795130120915\\
65	0.14424	-0.279156678604343	-0.279156678604343\\
65	0.1479	-0.282462076639884	-0.282462076639884\\
65	0.15156	-0.284711324227538	-0.284711324227538\\
65	0.15522	-0.285904421367303	-0.285904421367303\\
65	0.15888	-0.286041368059181	-0.286041368059181\\
65	0.16254	-0.285122164303172	-0.285122164303172\\
65	0.1662	-0.283146810099274	-0.283146810099274\\
65	0.16986	-0.280115305447489	-0.280115305447489\\
65	0.17352	-0.276027650347816	-0.276027650347816\\
65	0.17718	-0.270883844800255	-0.270883844800255\\
65	0.18084	-0.26468388880481	-0.26468388880481\\
65	0.1845	-0.257427782361473	-0.257427782361473\\
65	0.18816	-0.24911552547025	-0.24911552547025\\
65	0.19182	-0.239747118131139	-0.239747118131139\\
65	0.19548	-0.229322560344143	-0.229322560344143\\
65	0.19914	-0.217841852109256	-0.217841852109256\\
65	0.2028	-0.205304993426483	-0.205304993426483\\
65	0.20646	-0.191711984295819	-0.191711984295819\\
65	0.21012	-0.17706282471727	-0.17706282471727\\
65	0.21378	-0.161357514690836	-0.161357514690836\\
65	0.21744	-0.144596054216511	-0.144596054216511\\
65	0.2211	-0.126778443294299	-0.126778443294299\\
65	0.22476	-0.107904681924198	-0.107904681924198\\
65	0.22842	-0.0879747701062144	-0.0879747701062144\\
65	0.23208	-0.0669887078403391	-0.0669887078403391\\
65	0.23574	-0.0449464951265788	-0.0449464951265788\\
65	0.2394	-0.0218481319649286	-0.0218481319649286\\
65	0.24306	0.00230638164460606	0.00230638164460606\\
65	0.24672	0.027517045702032	0.027517045702032\\
65	0.25038	0.0537838602073459	0.0537838602073459\\
65	0.25404	0.0811068251605467	0.0811068251605467\\
65	0.2577	0.109485940561636	0.109485940561636\\
65	0.26136	0.138921206410608	0.138921206410608\\
65	0.26502	0.169412622707472	0.169412622707472\\
65	0.26868	0.200960189452223	0.200960189452223\\
65	0.27234	0.233563906644862	0.233563906644862\\
65	0.276	0.267223774285386	0.267223774285386\\
65.375	0.093	-0.109721966997744	-0.109721966997744\\
65.375	0.09666	-0.129734964247297	-0.129734964247297\\
65.375	0.10032	-0.148691811048964	-0.148691811048964\\
65.375	0.10398	-0.166592507402742	-0.166592507402742\\
65.375	0.10764	-0.183437053308633	-0.183437053308633\\
65.375	0.1113	-0.199225448766637	-0.199225448766637\\
65.375	0.11496	-0.213957693776752	-0.213957693776752\\
65.375	0.11862	-0.22763378833898	-0.22763378833898\\
65.375	0.12228	-0.240253732453321	-0.240253732453321\\
65.375	0.12594	-0.251817526119774	-0.251817526119774\\
65.375	0.1296	-0.262325169338339	-0.262325169338339\\
65.375	0.13326	-0.271776662109017	-0.271776662109017\\
65.375	0.13692	-0.280172004431807	-0.280172004431807\\
65.375	0.14058	-0.287511196306711	-0.287511196306711\\
65.375	0.14424	-0.293794237733726	-0.293794237733726\\
65.375	0.1479	-0.299021128712853	-0.299021128712853\\
65.375	0.15156	-0.303191869244093	-0.303191869244093\\
65.375	0.15522	-0.306306459327445	-0.306306459327445\\
65.375	0.15888	-0.308364898962911	-0.308364898962911\\
65.375	0.16254	-0.309367188150488	-0.309367188150488\\
65.375	0.1662	-0.309313326890177	-0.309313326890177\\
65.375	0.16986	-0.308203315181979	-0.308203315181979\\
65.375	0.17352	-0.306037153025893	-0.306037153025893\\
65.375	0.17718	-0.302814840421919	-0.302814840421919\\
65.375	0.18084	-0.298536377370058	-0.298536377370058\\
65.375	0.1845	-0.293201763870309	-0.293201763870309\\
65.375	0.18816	-0.286810999922674	-0.286810999922674\\
65.375	0.19182	-0.27936408552715	-0.27936408552715\\
65.375	0.19548	-0.270861020683738	-0.270861020683738\\
65.375	0.19914	-0.261301805392439	-0.261301805392439\\
65.375	0.2028	-0.250686439653253	-0.250686439653253\\
65.375	0.20646	-0.239014923466177	-0.239014923466177\\
65.375	0.21012	-0.226287256831215	-0.226287256831215\\
65.375	0.21378	-0.212503439748365	-0.212503439748365\\
65.375	0.21744	-0.197663472217628	-0.197663472217628\\
65.375	0.2211	-0.181767354239003	-0.181767354239003\\
65.375	0.22476	-0.16481508581249	-0.16481508581249\\
65.375	0.22842	-0.14680666693809	-0.14680666693809\\
65.375	0.23208	-0.127742097615803	-0.127742097615803\\
65.375	0.23574	-0.10762137784563	-0.10762137784563\\
65.375	0.2394	-0.0864445076275673	-0.0864445076275673\\
65.375	0.24306	-0.0642114869616166	-0.0642114869616166\\
65.375	0.24672	-0.0409223158477783	-0.0409223158477783\\
65.375	0.25038	-0.0165769942860519	-0.0165769942860519\\
65.375	0.25404	0.00882447772356132	0.00882447772356132\\
65.375	0.2577	0.0352821001810626	0.0352821001810626\\
65.375	0.26136	0.0627958730864511	0.0627958730864511\\
65.375	0.26502	0.0913657964397272	0.0913657964397272\\
65.375	0.26868	0.120991870240891	0.120991870240891\\
65.375	0.27234	0.151674094489943	0.151674094489943\\
65.375	0.276	0.183412469186882	0.183412469186882\\
65.75	0.093	-0.0966812518176328	-0.0966812518176328\\
65.75	0.09666	-0.118615742010774	-0.118615742010774\\
65.75	0.10032	-0.139494081756028	-0.139494081756028\\
65.75	0.10398	-0.159316271053394	-0.159316271053394\\
65.75	0.10764	-0.178082309902873	-0.178082309902873\\
65.75	0.1113	-0.195792198304462	-0.195792198304462\\
65.75	0.11496	-0.212445936258165	-0.212445936258165\\
65.75	0.11862	-0.228043523763979	-0.228043523763979\\
65.75	0.12228	-0.242584960821907	-0.242584960821907\\
65.75	0.12594	-0.256070247431946	-0.256070247431946\\
65.75	0.1296	-0.268499383594099	-0.268499383594099\\
65.75	0.13326	-0.279872369308364	-0.279872369308364\\
65.75	0.13692	-0.29018920457474	-0.29018920457474\\
65.75	0.14058	-0.29944988939323	-0.29944988939323\\
65.75	0.14424	-0.307654423763832	-0.307654423763832\\
65.75	0.1479	-0.314802807686547	-0.314802807686547\\
65.75	0.15156	-0.320895041161372	-0.320895041161372\\
65.75	0.15522	-0.325931124188312	-0.325931124188312\\
65.75	0.15888	-0.329911056767364	-0.329911056767364\\
65.75	0.16254	-0.332834838898528	-0.332834838898528\\
65.75	0.1662	-0.334702470581803	-0.334702470581803\\
65.75	0.16986	-0.335513951817192	-0.335513951817192\\
65.75	0.17352	-0.335269282604692	-0.335269282604692\\
65.75	0.17718	-0.333968462944308	-0.333968462944308\\
65.75	0.18084	-0.331611492836031	-0.331611492836031\\
65.75	0.1845	-0.328198372279869	-0.328198372279869\\
65.75	0.18816	-0.323729101275822	-0.323729101275822\\
65.75	0.19182	-0.318203679823882	-0.318203679823882\\
65.75	0.19548	-0.311622107924058	-0.311622107924058\\
65.75	0.19914	-0.303984385576346	-0.303984385576346\\
65.75	0.2028	-0.295290512780748	-0.295290512780748\\
65.75	0.20646	-0.285540489537256	-0.285540489537256\\
65.75	0.21012	-0.274734315845881	-0.274734315845881\\
65.75	0.21378	-0.262871991706619	-0.262871991706619\\
65.75	0.21744	-0.24995351711947	-0.24995351711947\\
65.75	0.2211	-0.235978892084432	-0.235978892084432\\
65.75	0.22476	-0.220948116601503	-0.220948116601503\\
65.75	0.22842	-0.204861190670691	-0.204861190670691\\
65.75	0.23208	-0.187718114291991	-0.187718114291991\\
65.75	0.23574	-0.169518887465406	-0.169518887465406\\
65.75	0.2394	-0.150263510190927	-0.150263510190927\\
65.75	0.24306	-0.129951982468564	-0.129951982468564\\
65.75	0.24672	-0.108584304298313	-0.108584304298313\\
65.75	0.25038	-0.0861604756801744	-0.0861604756801744\\
65.75	0.25404	-0.0626804966141452	-0.0626804966141452\\
65.75	0.2577	-0.0381443671002315	-0.0381443671002315\\
65.75	0.26136	-0.0125520871384306	-0.0125520871384306\\
65.75	0.26502	0.014096343271258	0.014096343271258\\
65.75	0.26868	0.0418009241288346	0.0418009241288346\\
65.75	0.27234	0.070561655434302	0.070561655434302\\
65.75	0.276	0.100378537187654	0.100378537187654\\
66.125	0.093	-0.0828631635382502	-0.0828631635382502\\
66.125	0.09666	-0.106719146674979	-0.106719146674979\\
66.125	0.10032	-0.129518979363819	-0.129518979363819\\
66.125	0.10398	-0.151262661604771	-0.151262661604771\\
66.125	0.10764	-0.171950193397837	-0.171950193397837\\
66.125	0.1113	-0.191581574743012	-0.191581574743012\\
66.125	0.11496	-0.210156805640302	-0.210156805640302\\
66.125	0.11862	-0.227675886089704	-0.227675886089704\\
66.125	0.12228	-0.244138816091218	-0.244138816091218\\
66.125	0.12594	-0.259545595644844	-0.259545595644844\\
66.125	0.1296	-0.273896224750583	-0.273896224750583\\
66.125	0.13326	-0.287190703408434	-0.287190703408434\\
66.125	0.13692	-0.299429031618399	-0.299429031618399\\
66.125	0.14058	-0.310611209380475	-0.310611209380475\\
66.125	0.14424	-0.320737236694663	-0.320737236694663\\
66.125	0.1479	-0.329807113560964	-0.329807113560964\\
66.125	0.15156	-0.337820839979379	-0.337820839979379\\
66.125	0.15522	-0.344778415949904	-0.344778415949904\\
66.125	0.15888	-0.350679841472541	-0.350679841472541\\
66.125	0.16254	-0.355525116547293	-0.355525116547293\\
66.125	0.1662	-0.359314241174156	-0.359314241174156\\
66.125	0.16986	-0.362047215353131	-0.362047215353131\\
66.125	0.17352	-0.363724039084216	-0.363724039084216\\
66.125	0.17718	-0.364344712367418	-0.364344712367418\\
66.125	0.18084	-0.363909235202732	-0.363909235202732\\
66.125	0.1845	-0.362417607590155	-0.362417607590155\\
66.125	0.18816	-0.359869829529695	-0.359869829529695\\
66.125	0.19182	-0.356265901021342	-0.356265901021342\\
66.125	0.19548	-0.351605822065106	-0.351605822065106\\
66.125	0.19914	-0.345889592660978	-0.345889592660978\\
66.125	0.2028	-0.339117212808968	-0.339117212808968\\
66.125	0.20646	-0.331288682509063	-0.331288682509063\\
66.125	0.21012	-0.322404001761276	-0.322404001761276\\
66.125	0.21378	-0.312463170565598	-0.312463170565598\\
66.125	0.21744	-0.301466188922036	-0.301466188922036\\
66.125	0.2211	-0.289413056830582	-0.289413056830582\\
66.125	0.22476	-0.276303774291245	-0.276303774291245\\
66.125	0.22842	-0.26213834130402	-0.26213834130402\\
66.125	0.23208	-0.246916757868904	-0.246916757868904\\
66.125	0.23574	-0.230639023985903	-0.230639023985903\\
66.125	0.2394	-0.213305139655015	-0.213305139655015\\
66.125	0.24306	-0.19491510487624	-0.19491510487624\\
66.125	0.24672	-0.175468919649573	-0.175468919649573\\
66.125	0.25038	-0.154966583975022	-0.154966583975022\\
66.125	0.25404	-0.13340809785258	-0.13340809785258\\
66.125	0.2577	-0.110793461282254	-0.110793461282254\\
66.125	0.26136	-0.087122674264037	-0.087122674264037\\
66.125	0.26502	-0.062395736797936	-0.062395736797936\\
66.125	0.26868	-0.0366126488839433	-0.0366126488839433\\
66.125	0.27234	-0.00977341052206704	-0.00977341052206704\\
66.125	0.276	0.0181219782876969	0.0181219782876969\\
66.5	0.093	-0.0682677021595942	-0.0682677021595942\\
66.5	0.09666	-0.0940451782399107	-0.0940451782399107\\
66.5	0.10032	-0.118766503872336	-0.118766503872336\\
66.5	0.10398	-0.142431679056875	-0.142431679056875\\
66.5	0.10764	-0.165040703793527	-0.165040703793527\\
66.5	0.1113	-0.18659357808229	-0.18659357808229\\
66.5	0.11496	-0.207090301923166	-0.207090301923166\\
66.5	0.11862	-0.226530875316154	-0.226530875316154\\
66.5	0.12228	-0.244915298261255	-0.244915298261255\\
66.5	0.12594	-0.262243570758469	-0.262243570758469\\
66.5	0.1296	-0.278515692807794	-0.278515692807794\\
66.5	0.13326	-0.293731664409232	-0.293731664409232\\
66.5	0.13692	-0.307891485562783	-0.307891485562783\\
66.5	0.14058	-0.320995156268446	-0.320995156268446\\
66.5	0.14424	-0.333042676526222	-0.333042676526222\\
66.5	0.1479	-0.344034046336109	-0.344034046336109\\
66.5	0.15156	-0.353969265698109	-0.353969265698109\\
66.5	0.15522	-0.362848334612222	-0.362848334612222\\
66.5	0.15888	-0.370671253078447	-0.370671253078447\\
66.5	0.16254	-0.377438021096783	-0.377438021096783\\
66.5	0.1662	-0.383148638667233	-0.383148638667233\\
66.5	0.16986	-0.387803105789796	-0.387803105789796\\
66.5	0.17352	-0.391401422464469	-0.391401422464469\\
66.5	0.17718	-0.393943588691258	-0.393943588691258\\
66.5	0.18084	-0.395429604470156	-0.395429604470156\\
66.5	0.1845	-0.395859469801166	-0.395859469801166\\
66.5	0.18816	-0.39523318468429	-0.39523318468429\\
66.5	0.19182	-0.393550749119529	-0.393550749119529\\
66.5	0.19548	-0.390812163106877	-0.390812163106877\\
66.5	0.19914	-0.387017426646336	-0.387017426646336\\
66.5	0.2028	-0.38216653973791	-0.38216653973791\\
66.5	0.20646	-0.376259502381596	-0.376259502381596\\
66.5	0.21012	-0.369296314577394	-0.369296314577394\\
66.5	0.21378	-0.361276976325307	-0.361276976325307\\
66.5	0.21744	-0.352201487625329	-0.352201487625329\\
66.5	0.2211	-0.342069848477463	-0.342069848477463\\
66.5	0.22476	-0.330882058881713	-0.330882058881713\\
66.5	0.22842	-0.318638118838072	-0.318638118838072\\
66.5	0.23208	-0.305338028346543	-0.305338028346543\\
66.5	0.23574	-0.29098178740713	-0.29098178740713\\
66.5	0.2394	-0.27556939601983	-0.27556939601983\\
66.5	0.24306	-0.259100854184638	-0.259100854184638\\
66.5	0.24672	-0.241576161901559	-0.241576161901559\\
66.5	0.25038	-0.222995319170592	-0.222995319170592\\
66.5	0.25404	-0.203358325991741	-0.203358325991741\\
66.5	0.2577	-0.182665182364999	-0.182665182364999\\
66.5	0.26136	-0.16091588829037	-0.16091588829037\\
66.5	0.26502	-0.138110443767856	-0.138110443767856\\
66.5	0.26868	-0.114248848797451	-0.114248848797451\\
66.5	0.27234	-0.0893311033791591	-0.0893311033791591\\
66.5	0.276	-0.0633572075129827	-0.0633572075129827\\
66.875	0.093	-0.052894867681661	-0.052894867681661\\
66.875	0.09666	-0.0805938367055615	-0.0805938367055615\\
66.875	0.10032	-0.107236655281576	-0.107236655281576\\
66.875	0.10398	-0.132823323409701	-0.132823323409701\\
66.875	0.10764	-0.157353841089939	-0.157353841089939\\
66.875	0.1113	-0.180828208322289	-0.180828208322289\\
66.875	0.11496	-0.203246425106753	-0.203246425106753\\
66.875	0.11862	-0.224608491443328	-0.224608491443328\\
66.875	0.12228	-0.244914407332016	-0.244914407332016\\
66.875	0.12594	-0.264164172772815	-0.264164172772815\\
66.875	0.1296	-0.282357787765729	-0.282357787765729\\
66.875	0.13326	-0.299495252310753	-0.299495252310753\\
66.875	0.13692	-0.31557656640789	-0.31557656640789\\
66.875	0.14058	-0.330601730057141	-0.330601730057141\\
66.875	0.14424	-0.344570743258503	-0.344570743258503\\
66.875	0.1479	-0.357483606011977	-0.357483606011977\\
66.875	0.15156	-0.369340318317563	-0.369340318317563\\
66.875	0.15522	-0.380140880175263	-0.380140880175263\\
66.875	0.15888	-0.389885291585074	-0.389885291585074\\
66.875	0.16254	-0.398573552546998	-0.398573552546998\\
66.875	0.1662	-0.406205663061036	-0.406205663061036\\
66.875	0.16986	-0.412781623127184	-0.412781623127184\\
66.875	0.17352	-0.418301432745445	-0.418301432745445\\
66.875	0.17718	-0.422765091915818	-0.422765091915818\\
66.875	0.18084	-0.426172600638303	-0.426172600638303\\
66.875	0.1845	-0.428523958912901	-0.428523958912901\\
66.875	0.18816	-0.429819166739613	-0.429819166739613\\
66.875	0.19182	-0.430058224118435	-0.430058224118435\\
66.875	0.19548	-0.42924113104937	-0.42924113104937\\
66.875	0.19914	-0.427367887532421	-0.427367887532421\\
66.875	0.2028	-0.424438493567579	-0.424438493567579\\
66.875	0.20646	-0.420452949154853	-0.420452949154853\\
66.875	0.21012	-0.415411254294238	-0.415411254294238\\
66.875	0.21378	-0.409313408985735	-0.409313408985735\\
66.875	0.21744	-0.402159413229344	-0.402159413229344\\
66.875	0.2211	-0.393949267025066	-0.393949267025066\\
66.875	0.22476	-0.3846829703729	-0.3846829703729\\
66.875	0.22842	-0.374360523272847	-0.374360523272847\\
66.875	0.23208	-0.362981925724906	-0.362981925724906\\
66.875	0.23574	-0.35054717772908	-0.35054717772908\\
66.875	0.2394	-0.337056279285364	-0.337056279285364\\
66.875	0.24306	-0.32250923039376	-0.32250923039376\\
66.875	0.24672	-0.306906031054268	-0.306906031054268\\
66.875	0.25038	-0.290246681266888	-0.290246681266888\\
66.875	0.25404	-0.272531181031622	-0.272531181031622\\
66.875	0.2577	-0.253759530348471	-0.253759530348471\\
66.875	0.26136	-0.233931729217429	-0.233931729217429\\
66.875	0.26502	-0.2130477776385	-0.2130477776385\\
66.875	0.26868	-0.191107675611682	-0.191107675611682\\
66.875	0.27234	-0.168111423136978	-0.168111423136978\\
66.875	0.276	-0.144059020214385	-0.144059020214385\\
67.25	0.093	-0.0367446601044544	-0.0367446601044544\\
67.25	0.09666	-0.0663651220719425	-0.0663651220719425\\
67.25	0.10032	-0.0949294335915428	-0.0949294335915428\\
67.25	0.10398	-0.122437594663256	-0.122437594663256\\
67.25	0.10764	-0.148889605287079	-0.148889605287079\\
67.25	0.1113	-0.174285465463017	-0.174285465463017\\
67.25	0.11496	-0.198625175191067	-0.198625175191067\\
67.25	0.11862	-0.221908734471227	-0.221908734471227\\
67.25	0.12228	-0.244136143303502	-0.244136143303502\\
67.25	0.12594	-0.265307401687889	-0.265307401687889\\
67.25	0.1296	-0.285422509624389	-0.285422509624389\\
67.25	0.13326	-0.304481467112999	-0.304481467112999\\
67.25	0.13692	-0.322484274153724	-0.322484274153724\\
67.25	0.14058	-0.33943093074656	-0.33943093074656\\
67.25	0.14424	-0.355321436891509	-0.355321436891509\\
67.25	0.1479	-0.370155792588571	-0.370155792588571\\
67.25	0.15156	-0.383933997837745	-0.383933997837745\\
67.25	0.15522	-0.396656052639029	-0.396656052639029\\
67.25	0.15888	-0.40832195699243	-0.40832195699243\\
67.25	0.16254	-0.418931710897939	-0.418931710897939\\
67.25	0.1662	-0.428485314355562	-0.428485314355562\\
67.25	0.16986	-0.436982767365298	-0.436982767365298\\
67.25	0.17352	-0.444424069927147	-0.444424069927147\\
67.25	0.17718	-0.450809222041107	-0.450809222041107\\
67.25	0.18084	-0.45613822370718	-0.45613822370718\\
67.25	0.1845	-0.460411074925362	-0.460411074925362\\
67.25	0.18816	-0.463627775695661	-0.463627775695661\\
67.25	0.19182	-0.465788326018072	-0.465788326018072\\
67.25	0.19548	-0.466892725892594	-0.466892725892594\\
67.25	0.19914	-0.466940975319229	-0.466940975319229\\
67.25	0.2028	-0.465933074297975	-0.465933074297975\\
67.25	0.20646	-0.463869022828832	-0.463869022828832\\
67.25	0.21012	-0.460748820911805	-0.460748820911805\\
67.25	0.21378	-0.456572468546889	-0.456572468546889\\
67.25	0.21744	-0.451339965734086	-0.451339965734086\\
67.25	0.2211	-0.445051312473396	-0.445051312473396\\
67.25	0.22476	-0.437706508764817	-0.437706508764817\\
67.25	0.22842	-0.429305554608348	-0.429305554608348\\
67.25	0.23208	-0.419848450003995	-0.419848450003995\\
67.25	0.23574	-0.409335194951756	-0.409335194951756\\
67.25	0.2394	-0.397765789451628	-0.397765789451628\\
67.25	0.24306	-0.385140233503611	-0.385140233503611\\
67.25	0.24672	-0.371458527107707	-0.371458527107707\\
67.25	0.25038	-0.356720670263912	-0.356720670263912\\
67.25	0.25404	-0.340926662972233	-0.340926662972233\\
67.25	0.2577	-0.324076505232666	-0.324076505232666\\
67.25	0.26136	-0.306170197045212	-0.306170197045212\\
67.25	0.26502	-0.28720773840987	-0.28720773840987\\
67.25	0.26868	-0.26718912932664	-0.26718912932664\\
67.25	0.27234	-0.246114369795523	-0.246114369795523\\
67.25	0.276	-0.223983459816514	-0.223983459816514\\
67.625	0.093	-0.0198170794279725	-0.0198170794279725\\
67.625	0.09666	-0.0513590343390481	-0.0513590343390481\\
67.625	0.10032	-0.081844838802236	-0.081844838802236\\
67.625	0.10398	-0.111274492817535	-0.111274492817535\\
67.625	0.10764	-0.139647996384946	-0.139647996384946\\
67.625	0.1113	-0.166965349504469	-0.166965349504469\\
67.625	0.11496	-0.193226552176107	-0.193226552176107\\
67.625	0.11862	-0.218431604399855	-0.218431604399855\\
67.625	0.12228	-0.242580506175716	-0.242580506175716\\
67.625	0.12594	-0.26567325750369	-0.26567325750369\\
67.625	0.1296	-0.287709858383776	-0.287709858383776\\
67.625	0.13326	-0.308690308815974	-0.308690308815974\\
67.625	0.13692	-0.328614608800284	-0.328614608800284\\
67.625	0.14058	-0.347482758336708	-0.347482758336708\\
67.625	0.14424	-0.365294757425245	-0.365294757425245\\
67.625	0.1479	-0.382050606065892	-0.382050606065892\\
67.625	0.15156	-0.397750304258652	-0.397750304258652\\
67.625	0.15522	-0.412393852003526	-0.412393852003526\\
67.625	0.15888	-0.42598124930051	-0.42598124930051\\
67.625	0.16254	-0.438512496149608	-0.438512496149608\\
67.625	0.1662	-0.449987592550818	-0.449987592550818\\
67.625	0.16986	-0.460406538504141	-0.460406538504141\\
67.625	0.17352	-0.469769334009573	-0.469769334009573\\
67.625	0.17718	-0.478075979067122	-0.478075979067122\\
67.625	0.18084	-0.485326473676782	-0.485326473676782\\
67.625	0.1845	-0.491520817838552	-0.491520817838552\\
67.625	0.18816	-0.496659011552439	-0.496659011552439\\
67.625	0.19182	-0.500741054818433	-0.500741054818433\\
67.625	0.19548	-0.503766947636543	-0.503766947636543\\
67.625	0.19914	-0.505736690006765	-0.505736690006765\\
67.625	0.2028	-0.506650281929098	-0.506650281929098\\
67.625	0.20646	-0.506507723403544	-0.506507723403544\\
67.625	0.21012	-0.5053090144301	-0.5053090144301\\
67.625	0.21378	-0.503054155008772	-0.503054155008772\\
67.625	0.21744	-0.499743145139557	-0.499743145139557\\
67.625	0.2211	-0.49537598482245	-0.49537598482245\\
67.625	0.22476	-0.489952674057459	-0.489952674057459\\
67.625	0.22842	-0.483473212844581	-0.483473212844581\\
67.625	0.23208	-0.475937601183812	-0.475937601183812\\
67.625	0.23574	-0.467345839075161	-0.467345839075161\\
67.625	0.2394	-0.457697926518617	-0.457697926518617\\
67.625	0.24306	-0.446993863514188	-0.446993863514188\\
67.625	0.24672	-0.435233650061871	-0.435233650061871\\
67.625	0.25038	-0.422417286161663	-0.422417286161663\\
67.625	0.25404	-0.408544771813572	-0.408544771813572\\
67.625	0.2577	-0.393616107017589	-0.393616107017589\\
67.625	0.26136	-0.377631291773722	-0.377631291773722\\
67.625	0.26502	-0.360590326081968	-0.360590326081968\\
67.625	0.26868	-0.342493209942322	-0.342493209942322\\
67.625	0.27234	-0.323339943354792	-0.323339943354792\\
67.625	0.276	-0.303130526319375	-0.303130526319375\\
68	0.093	-0.00211212565221702	-0.00211212565221702\\
68	0.09666	-0.0355755735068802	-0.0355755735068802\\
68	0.10032	-0.0679828709136522	-0.0679828709136522\\
68	0.10398	-0.0993340178725385	-0.0993340178725385\\
68	0.10764	-0.129629014383537	-0.129629014383537\\
68	0.1113	-0.158867860446648	-0.158867860446648\\
68	0.11496	-0.187050556061871	-0.187050556061871\\
68	0.11862	-0.214177101229205	-0.214177101229205\\
68	0.12228	-0.240247495948654	-0.240247495948654\\
68	0.12594	-0.265261740220214	-0.265261740220214\\
68	0.1296	-0.289219834043887	-0.289219834043887\\
68	0.13326	-0.312121777419673	-0.312121777419673\\
68	0.13692	-0.33396757034757	-0.33396757034757\\
68	0.14058	-0.35475721282758	-0.35475721282758\\
68	0.14424	-0.374490704859703	-0.374490704859703\\
68	0.1479	-0.393168046443938	-0.393168046443938\\
68	0.15156	-0.410789237580285	-0.410789237580285\\
68	0.15522	-0.427354278268743	-0.427354278268743\\
68	0.15888	-0.442863168509316	-0.442863168509316\\
68	0.16254	-0.457315908302001	-0.457315908302001\\
68	0.1662	-0.470712497646796	-0.470712497646796\\
68	0.16986	-0.483052936543707	-0.483052936543707\\
68	0.17352	-0.494337224992727	-0.494337224992727\\
68	0.17718	-0.504565362993862	-0.504565362993862\\
68	0.18084	-0.513737350547107	-0.513737350547107\\
68	0.1845	-0.521853187652464	-0.521853187652464\\
68	0.18816	-0.528912874309938	-0.528912874309938\\
68	0.19182	-0.53491641051952	-0.53491641051952\\
68	0.19548	-0.539863796281215	-0.539863796281215\\
68	0.19914	-0.543755031595025	-0.543755031595025\\
68	0.2028	-0.546590116460945	-0.546590116460945\\
68	0.20646	-0.548369050878974	-0.548369050878974\\
68	0.21012	-0.549091834849122	-0.549091834849122\\
68	0.21378	-0.548758468371378	-0.548758468371378\\
68	0.21744	-0.54736895144575	-0.54736895144575\\
68	0.2211	-0.544923284072231	-0.544923284072231\\
68	0.22476	-0.541421466250828	-0.541421466250828\\
68	0.22842	-0.536863497981534	-0.536863497981534\\
68	0.23208	-0.531249379264352	-0.531249379264352\\
68	0.23574	-0.524579110099288	-0.524579110099288\\
68	0.2394	-0.516852690486332	-0.516852690486332\\
68	0.24306	-0.508070120425487	-0.508070120425487\\
68	0.24672	-0.498231399916758	-0.498231399916758\\
68	0.25038	-0.487336528960137	-0.487336528960137\\
68	0.25404	-0.47538550755563	-0.47538550755563\\
68	0.2577	-0.462378335703238	-0.462378335703238\\
68	0.26136	-0.448315013402956	-0.448315013402956\\
68	0.26502	-0.433195540654789	-0.433195540654789\\
68	0.26868	-0.417019917458729	-0.417019917458729\\
68	0.27234	-0.399788143814787	-0.399788143814787\\
68	0.276	-0.381500219722957	-0.381500219722957\\
68.375	0.093	0.0163702012228119	0.0163702012228119\\
68.375	0.09666	-0.0190147395754353	-0.0190147395754353\\
68.375	0.10032	-0.0533435299257948	-0.0533435299257948\\
68.375	0.10398	-0.0866161698282687	-0.0866161698282687\\
68.375	0.10764	-0.118832659282853	-0.118832659282853\\
68.375	0.1113	-0.14999299828955	-0.14999299828955\\
68.375	0.11496	-0.180097186848361	-0.180097186848361\\
68.375	0.11862	-0.209145224959282	-0.209145224959282\\
68.375	0.12228	-0.237137112622316	-0.237137112622316\\
68.375	0.12594	-0.264072849837464	-0.264072849837464\\
68.375	0.1296	-0.289952436604723	-0.289952436604723\\
68.375	0.13326	-0.314775872924094	-0.314775872924094\\
68.375	0.13692	-0.33854315879558	-0.33854315879558\\
68.375	0.14058	-0.361254294219177	-0.361254294219177\\
68.375	0.14424	-0.382909279194885	-0.382909279194885\\
68.375	0.1479	-0.403508113722708	-0.403508113722708\\
68.375	0.15156	-0.423050797802641	-0.423050797802641\\
68.375	0.15522	-0.441537331434686	-0.441537331434686\\
68.375	0.15888	-0.458967714618846	-0.458967714618846\\
68.375	0.16254	-0.475341947355118	-0.475341947355118\\
68.375	0.1662	-0.4906600296435	-0.4906600296435\\
68.375	0.16986	-0.504921961483997	-0.504921961483997\\
68.375	0.17352	-0.518127742876605	-0.518127742876605\\
68.375	0.17718	-0.530277373821324	-0.530277373821324\\
68.375	0.18084	-0.541370854318157	-0.541370854318157\\
68.375	0.1845	-0.551408184367101	-0.551408184367101\\
68.375	0.18816	-0.56038936396816	-0.56038936396816\\
68.375	0.19182	-0.568314393121333	-0.568314393121333\\
68.375	0.19548	-0.575183271826614	-0.575183271826614\\
68.375	0.19914	-0.580996000084008	-0.580996000084008\\
68.375	0.2028	-0.585752577893516	-0.585752577893516\\
68.375	0.20646	-0.589453005255133	-0.589453005255133\\
68.375	0.21012	-0.592097282168865	-0.592097282168865\\
68.375	0.21378	-0.593685408634712	-0.593685408634712\\
68.375	0.21744	-0.594217384652668	-0.594217384652668\\
68.375	0.2211	-0.593693210222737	-0.593693210222737\\
68.375	0.22476	-0.592112885344918	-0.592112885344918\\
68.375	0.22842	-0.589476410019211	-0.589476410019211\\
68.375	0.23208	-0.585783784245617	-0.585783784245617\\
68.375	0.23574	-0.581035008024137	-0.581035008024137\\
68.375	0.2394	-0.575230081354772	-0.575230081354772\\
68.375	0.24306	-0.568369004237514	-0.568369004237514\\
68.375	0.24672	-0.56045177667237	-0.56045177667237\\
68.375	0.25038	-0.551478398659337	-0.551478398659337\\
68.375	0.25404	-0.541448870198417	-0.541448870198417\\
68.375	0.2577	-0.530363191289609	-0.530363191289609\\
68.375	0.26136	-0.518221361932917	-0.518221361932917\\
68.375	0.26502	-0.505023382128335	-0.505023382128335\\
68.375	0.26868	-0.490769251875864	-0.490769251875864\\
68.375	0.27234	-0.475458971175506	-0.475458971175506\\
68.375	0.276	-0.45909254002726	-0.45909254002726\\
68.75	0.093	0.0356299011971197	0.0356299011971197\\
68.75	0.09666	-0.00167653254471156	-0.00167653254471156\\
68.75	0.10032	-0.0379268158386604	-0.0379268158386604\\
68.75	0.10398	-0.0731209486847201	-0.0731209486847201\\
68.75	0.10764	-0.107258931082892	-0.107258931082892\\
68.75	0.1113	-0.140340763033176	-0.140340763033176\\
68.75	0.11496	-0.172366444535573	-0.172366444535573\\
68.75	0.11862	-0.20333597559008	-0.20333597559008\\
68.75	0.12228	-0.233249356196702	-0.233249356196702\\
68.75	0.12594	-0.262106586355436	-0.262106586355436\\
68.75	0.1296	-0.289907666066282	-0.289907666066282\\
68.75	0.13326	-0.316652595329241	-0.316652595329241\\
68.75	0.13692	-0.342341374144312	-0.342341374144312\\
68.75	0.14058	-0.366974002511497	-0.366974002511497\\
68.75	0.14424	-0.390550480430791	-0.390550480430791\\
68.75	0.1479	-0.413070807902199	-0.413070807902199\\
68.75	0.15156	-0.43453498492572	-0.43453498492572\\
68.75	0.15522	-0.454943011501353	-0.454943011501353\\
68.75	0.15888	-0.4742948876291	-0.4742948876291\\
68.75	0.16254	-0.492590613308958	-0.492590613308958\\
68.75	0.1662	-0.509830188540926	-0.509830188540926\\
68.75	0.16986	-0.526013613325011	-0.526013613325011\\
68.75	0.17352	-0.541140887661206	-0.541140887661206\\
68.75	0.17718	-0.555212011549513	-0.555212011549513\\
68.75	0.18084	-0.568226984989933	-0.568226984989933\\
68.75	0.1845	-0.580185807982465	-0.580185807982465\\
68.75	0.18816	-0.591088480527107	-0.591088480527107\\
68.75	0.19182	-0.600935002623864	-0.600935002623864\\
68.75	0.19548	-0.609725374272734	-0.609725374272734\\
68.75	0.19914	-0.617459595473715	-0.617459595473715\\
68.75	0.2028	-0.624137666226811	-0.624137666226811\\
68.75	0.20646	-0.629759586532015	-0.629759586532015\\
68.75	0.21012	-0.634325356389334	-0.634325356389334\\
68.75	0.21378	-0.637834975798766	-0.637834975798766\\
68.75	0.21744	-0.640288444760309	-0.640288444760309\\
68.75	0.2211	-0.641685763273965	-0.641685763273965\\
68.75	0.22476	-0.642026931339734	-0.642026931339734\\
68.75	0.22842	-0.641311948957615	-0.641311948957615\\
68.75	0.23208	-0.639540816127608	-0.639540816127608\\
68.75	0.23574	-0.636713532849713	-0.636713532849713\\
68.75	0.2394	-0.632830099123931	-0.632830099123931\\
68.75	0.24306	-0.627890514950261	-0.627890514950261\\
68.75	0.24672	-0.621894780328704	-0.621894780328704\\
68.75	0.25038	-0.614842895259259	-0.614842895259259\\
68.75	0.25404	-0.606734859741926	-0.606734859741926\\
68.75	0.2577	-0.597570673776706	-0.597570673776706\\
68.75	0.26136	-0.587350337363599	-0.587350337363599\\
68.75	0.26502	-0.576073850502604	-0.576073850502604\\
68.75	0.26868	-0.56374121319372	-0.56374121319372\\
68.75	0.27234	-0.55035242543695	-0.55035242543695\\
68.75	0.276	-0.535907487232292	-0.535907487232292\\
69.125	0.093	0.0556669742707045	0.0556669742707045\\
69.125	0.09666	0.0164390475852821	0.0164390475852821\\
69.125	0.10032	-0.0217327286522525	-0.0217327286522525\\
69.125	0.10398	-0.058848354441898	-0.058848354441898\\
69.125	0.10764	-0.0949078297836575	-0.0949078297836575\\
69.125	0.1113	-0.129911154677528	-0.129911154677528\\
69.125	0.11496	-0.163858329123512	-0.163858329123512\\
69.125	0.11862	-0.196749353121607	-0.196749353121607\\
69.125	0.12228	-0.228584226671814	-0.228584226671814\\
69.125	0.12594	-0.259362949774136	-0.259362949774136\\
69.125	0.1296	-0.289085522428568	-0.289085522428568\\
69.125	0.13326	-0.317751944635114	-0.317751944635114\\
69.125	0.13692	-0.345362216393771	-0.345362216393771\\
69.125	0.14058	-0.371916337704542	-0.371916337704542\\
69.125	0.14424	-0.397414308567425	-0.397414308567425\\
69.125	0.1479	-0.421856128982419	-0.421856128982419\\
69.125	0.15156	-0.445241798949527	-0.445241798949527\\
69.125	0.15522	-0.467571318468746	-0.467571318468746\\
69.125	0.15888	-0.488844687540079	-0.488844687540079\\
69.125	0.16254	-0.509061906163524	-0.509061906163524\\
69.125	0.1662	-0.52822297433908	-0.52822297433908\\
69.125	0.16986	-0.54632789206675	-0.54632789206675\\
69.125	0.17352	-0.563376659346529	-0.563376659346529\\
69.125	0.17718	-0.579369276178424	-0.579369276178424\\
69.125	0.18084	-0.594305742562432	-0.594305742562432\\
69.125	0.1845	-0.608186058498551	-0.608186058498551\\
69.125	0.18816	-0.621010223986781	-0.621010223986781\\
69.125	0.19182	-0.632778239027126	-0.632778239027126\\
69.125	0.19548	-0.643490103619583	-0.643490103619583\\
69.125	0.19914	-0.653145817764148	-0.653145817764148\\
69.125	0.2028	-0.661745381460832	-0.661745381460832\\
69.125	0.20646	-0.669288794709624	-0.669288794709624\\
69.125	0.21012	-0.675776057510527	-0.675776057510527\\
69.125	0.21378	-0.681207169863546	-0.681207169863546\\
69.125	0.21744	-0.685582131768677	-0.685582131768677\\
69.125	0.2211	-0.688900943225917	-0.688900943225917\\
69.125	0.22476	-0.691163604235273	-0.691163604235273\\
69.125	0.22842	-0.692370114796741	-0.692370114796741\\
69.125	0.23208	-0.692520474910322	-0.692520474910322\\
69.125	0.23574	-0.691614684576014	-0.691614684576014\\
69.125	0.2394	-0.68965274379382	-0.68965274379382\\
69.125	0.24306	-0.686634652563738	-0.686634652563738\\
69.125	0.24672	-0.682560410885765	-0.682560410885765\\
69.125	0.25038	-0.677430018759907	-0.677430018759907\\
69.125	0.25404	-0.671243476186159	-0.671243476186159\\
69.125	0.2577	-0.664000783164526	-0.664000783164526\\
69.125	0.26136	-0.655701939695006	-0.655701939695006\\
69.125	0.26502	-0.646346945777599	-0.646346945777599\\
69.125	0.26868	-0.6359358014123	-0.6359358014123\\
69.125	0.27234	-0.624468506599117	-0.624468506599117\\
69.125	0.276	-0.611945061338046	-0.611945061338046\\
69.5	0.093	0.076481420443561	0.076481420443561\\
69.5	0.09666	0.0353320008145511	0.0353320008145511\\
69.5	0.10032	-0.00476126836656932	-0.00476126836656932\\
69.5	0.10398	-0.0437983870998023	-0.0437983870998023\\
69.5	0.10764	-0.0817793553851476	-0.0817793553851476\\
69.5	0.1113	-0.118704173222605	-0.118704173222605\\
69.5	0.11496	-0.154572840612175	-0.154572840612175\\
69.5	0.11862	-0.189385357553858	-0.189385357553858\\
69.5	0.12228	-0.223141724047653	-0.223141724047653\\
69.5	0.12594	-0.25584194009356	-0.25584194009356\\
69.5	0.1296	-0.28748600569158	-0.28748600569158\\
69.5	0.13326	-0.318073920841712	-0.318073920841712\\
69.5	0.13692	-0.347605685543956	-0.347605685543956\\
69.5	0.14058	-0.376081299798313	-0.376081299798313\\
69.5	0.14424	-0.403500763604782	-0.403500763604782\\
69.5	0.1479	-0.429864076963364	-0.429864076963364\\
69.5	0.15156	-0.455171239874058	-0.455171239874058\\
69.5	0.15522	-0.479422252336864	-0.479422252336864\\
69.5	0.15888	-0.502617114351782	-0.502617114351782\\
69.5	0.16254	-0.524755825918814	-0.524755825918814\\
69.5	0.1662	-0.545838387037957	-0.545838387037957\\
69.5	0.16986	-0.565864797709215	-0.565864797709215\\
69.5	0.17352	-0.584835057932582	-0.584835057932582\\
69.5	0.17718	-0.602749167708064	-0.602749167708064\\
69.5	0.18084	-0.619607127035656	-0.619607127035656\\
69.5	0.1845	-0.635408935915363	-0.635408935915363\\
69.5	0.18816	-0.65015459434718	-0.65015459434718\\
69.5	0.19182	-0.663844102331109	-0.663844102331109\\
69.5	0.19548	-0.676477459867153	-0.676477459867153\\
69.5	0.19914	-0.688054666955306	-0.688054666955306\\
69.5	0.2028	-0.698575723595577	-0.698575723595577\\
69.5	0.20646	-0.708040629787953	-0.708040629787953\\
69.5	0.21012	-0.716449385532447	-0.716449385532447\\
69.5	0.21378	-0.72380199082905	-0.72380199082905\\
69.5	0.21744	-0.730098445677769	-0.730098445677769\\
69.5	0.2211	-0.735338750078597	-0.735338750078597\\
69.5	0.22476	-0.73952290403154	-0.73952290403154\\
69.5	0.22842	-0.742650907536593	-0.742650907536593\\
69.5	0.23208	-0.744722760593761	-0.744722760593761\\
69.5	0.23574	-0.745738463203041	-0.745738463203041\\
69.5	0.2394	-0.745698015364431	-0.745698015364431\\
69.5	0.24306	-0.744601417077936	-0.744601417077936\\
69.5	0.24672	-0.742448668343551	-0.742448668343551\\
69.5	0.25038	-0.73923976916128	-0.73923976916128\\
69.5	0.25404	-0.73497471953112	-0.73497471953112\\
69.5	0.2577	-0.729653519453075	-0.729653519453075\\
69.5	0.26136	-0.723276168927139	-0.723276168927139\\
69.5	0.26502	-0.715842667953319	-0.715842667953319\\
69.5	0.26868	-0.707353016531607	-0.707353016531607\\
69.5	0.27234	-0.697807214662012	-0.697807214662012\\
69.5	0.276	-0.687205262344525	-0.687205262344525\\
69.875	0.093	0.0980732397156929	0.0980732397156929\\
69.875	0.09666	0.0550023271430989	0.0550023271430989\\
69.875	0.10032	0.0129875650183892	0.0129875650183892\\
69.875	0.10398	-0.0279710466584296	-0.0279710466584296\\
69.875	0.10764	-0.0678735078873625	-0.0678735078873625\\
69.875	0.1113	-0.106719818668406	-0.106719818668406\\
69.875	0.11496	-0.144509979001564	-0.144509979001564\\
69.875	0.11862	-0.181243988886832	-0.181243988886832\\
69.875	0.12228	-0.216921848324214	-0.216921848324214\\
69.875	0.12594	-0.251543557313707	-0.251543557313707\\
69.875	0.1296	-0.285109115855315	-0.285109115855315\\
69.875	0.13326	-0.317618523949033	-0.317618523949033\\
69.875	0.13692	-0.349071781594865	-0.349071781594865\\
69.875	0.14058	-0.379468888792809	-0.379468888792809\\
69.875	0.14424	-0.408809845542864	-0.408809845542864\\
69.875	0.1479	-0.437094651845033	-0.437094651845033\\
69.875	0.15156	-0.464323307699313	-0.464323307699313\\
69.875	0.15522	-0.490495813105707	-0.490495813105707\\
69.875	0.15888	-0.515612168064213	-0.515612168064213\\
69.875	0.16254	-0.53967237257483	-0.53967237257483\\
69.875	0.1662	-0.562676426637559	-0.562676426637559\\
69.875	0.16986	-0.584624330252404	-0.584624330252404\\
69.875	0.17352	-0.605516083419358	-0.605516083419358\\
69.875	0.17718	-0.625351686138425	-0.625351686138425\\
69.875	0.18084	-0.644131138409608	-0.644131138409608\\
69.875	0.1845	-0.661854440232899	-0.661854440232899\\
69.875	0.18816	-0.678521591608304	-0.678521591608304\\
69.875	0.19182	-0.69413259253582	-0.69413259253582\\
69.875	0.19548	-0.708687443015448	-0.708687443015448\\
69.875	0.19914	-0.722186143047193	-0.722186143047193\\
69.875	0.2028	-0.734628692631047	-0.734628692631047\\
69.875	0.20646	-0.746015091767011	-0.746015091767011\\
69.875	0.21012	-0.756345340455089	-0.756345340455089\\
69.875	0.21378	-0.76561943869528	-0.76561943869528\\
69.875	0.21744	-0.773837386487586	-0.773837386487586\\
69.875	0.2211	-0.780999183832001	-0.780999183832001\\
69.875	0.22476	-0.787104830728529	-0.787104830728529\\
69.875	0.22842	-0.792154327177169	-0.792154327177169\\
69.875	0.23208	-0.796147673177925	-0.796147673177925\\
69.875	0.23574	-0.799084868730792	-0.799084868730792\\
69.875	0.2394	-0.80096591383577	-0.80096591383577\\
69.875	0.24306	-0.801790808492859	-0.801790808492859\\
69.875	0.24672	-0.801559552702065	-0.801559552702065\\
69.875	0.25038	-0.800272146463378	-0.800272146463378\\
69.875	0.25404	-0.797928589776805	-0.797928589776805\\
69.875	0.2577	-0.794528882642344	-0.794528882642344\\
69.875	0.26136	-0.790073025059996	-0.790073025059996\\
69.875	0.26502	-0.784561017029763	-0.784561017029763\\
69.875	0.26868	-0.777992858551639	-0.777992858551639\\
69.875	0.27234	-0.770368549625628	-0.770368549625628\\
69.875	0.276	-0.761688090251729	-0.761688090251729\\
70.25	0.093	0.1204424320871	0.1204424320871\\
70.25	0.09666	0.0754500265709185	0.0754500265709185\\
70.25	0.10032	0.0315137715026229	0.0315137715026229\\
70.25	0.10398	-0.0113663331177835	-0.0113663331177835\\
70.25	0.10764	-0.0531902872903021	-0.0531902872903021\\
70.25	0.1113	-0.0939580910149331	-0.0939580910149331\\
70.25	0.11496	-0.133669744291677	-0.133669744291677\\
70.25	0.11862	-0.172325247120532	-0.172325247120532\\
70.25	0.12228	-0.209924599501501	-0.209924599501501\\
70.25	0.12594	-0.246467801434581	-0.246467801434581\\
70.25	0.1296	-0.281954852919774	-0.281954852919774\\
70.25	0.13326	-0.31638575395708	-0.31638575395708\\
70.25	0.13692	-0.349760504546498	-0.349760504546498\\
70.25	0.14058	-0.382079104688029	-0.382079104688029\\
70.25	0.14424	-0.413341554381672	-0.413341554381672\\
70.25	0.1479	-0.443547853627427	-0.443547853627427\\
70.25	0.15156	-0.472698002425294	-0.472698002425294\\
70.25	0.15522	-0.500792000775274	-0.500792000775274\\
70.25	0.15888	-0.527829848677366	-0.527829848677366\\
70.25	0.16254	-0.55381154613157	-0.55381154613157\\
70.25	0.1662	-0.578737093137889	-0.578737093137889\\
70.25	0.16986	-0.602606489696318	-0.602606489696318\\
70.25	0.17352	-0.62541973580686	-0.62541973580686\\
70.25	0.17718	-0.647176831469514	-0.647176831469514\\
70.25	0.18084	-0.667877776684281	-0.667877776684281\\
70.25	0.1845	-0.687522571451159	-0.687522571451159\\
70.25	0.18816	-0.706111215770152	-0.706111215770152\\
70.25	0.19182	-0.723643709641256	-0.723643709641256\\
70.25	0.19548	-0.740120053064472	-0.740120053064472\\
70.25	0.19914	-0.7555402460398	-0.7555402460398\\
70.25	0.2028	-0.769904288567244	-0.769904288567244\\
70.25	0.20646	-0.783212180646794	-0.783212180646794\\
70.25	0.21012	-0.795463922278459	-0.795463922278459\\
70.25	0.21378	-0.806659513462237	-0.806659513462237\\
70.25	0.21744	-0.816798954198128	-0.816798954198128\\
70.25	0.2211	-0.825882244486131	-0.825882244486131\\
70.25	0.22476	-0.833909384326246	-0.833909384326246\\
70.25	0.22842	-0.840880373718473	-0.840880373718473\\
70.25	0.23208	-0.846795212662813	-0.846795212662813\\
70.25	0.23574	-0.851653901159268	-0.851653901159268\\
70.25	0.2394	-0.855456439207833	-0.855456439207833\\
70.25	0.24306	-0.85820282680851	-0.85820282680851\\
70.25	0.24672	-0.8598930639613	-0.8598930639613\\
70.25	0.25038	-0.860527150666198	-0.860527150666198\\
70.25	0.25404	-0.860105086923216	-0.860105086923216\\
70.25	0.2577	-0.858626872732342	-0.858626872732342\\
70.25	0.26136	-0.856092508093581	-0.856092508093581\\
70.25	0.26502	-0.852501993006933	-0.852501993006933\\
70.25	0.26868	-0.847855327472396	-0.847855327472396\\
70.25	0.27234	-0.842152511489973	-0.842152511489973\\
70.25	0.276	-0.835393545059661	-0.835393545059661\\
70.625	0.093	0.143588997557781	0.143588997557781\\
70.625	0.09666	0.0966750990980115	0.0966750990980115\\
70.625	0.10032	0.0508173510861302	0.0508173510861302\\
70.625	0.10398	0.00601575352213624	0.00601575352213624\\
70.625	0.10764	-0.0377296935939682	-0.0377296935939682\\
70.625	0.1113	-0.0804189902621868	-0.0804189902621868\\
70.625	0.11496	-0.122052136482518	-0.122052136482518\\
70.625	0.11862	-0.162629132254959	-0.162629132254959\\
70.625	0.12228	-0.202149977579515	-0.202149977579515\\
70.625	0.12594	-0.240614672456182	-0.240614672456182\\
70.625	0.1296	-0.278023216884962	-0.278023216884962\\
70.625	0.13326	-0.314375610865855	-0.314375610865855\\
70.625	0.13692	-0.349671854398859	-0.349671854398859\\
70.625	0.14058	-0.383911947483976	-0.383911947483976\\
70.625	0.14424	-0.417095890121207	-0.417095890121207\\
70.625	0.1479	-0.449223682310549	-0.449223682310549\\
70.625	0.15156	-0.480295324052002	-0.480295324052002\\
70.625	0.15522	-0.51031081534557	-0.51031081534557\\
70.625	0.15888	-0.539270156191249	-0.539270156191249\\
70.625	0.16254	-0.567173346589041	-0.567173346589041\\
70.625	0.1662	-0.594020386538946	-0.594020386538946\\
70.625	0.16986	-0.619811276040959	-0.619811276040959\\
70.625	0.17352	-0.644546015095088	-0.644546015095088\\
70.625	0.17718	-0.66822460370133	-0.66822460370133\\
70.625	0.18084	-0.690847041859684	-0.690847041859684\\
70.625	0.1845	-0.71241332957015	-0.71241332957015\\
70.625	0.18816	-0.732923466832727	-0.732923466832727\\
70.625	0.19182	-0.752377453647418	-0.752377453647418\\
70.625	0.19548	-0.770775290014222	-0.770775290014222\\
70.625	0.19914	-0.788116975933138	-0.788116975933138\\
70.625	0.2028	-0.804402511404162	-0.804402511404162\\
70.625	0.20646	-0.819631896427303	-0.819631896427303\\
70.625	0.21012	-0.833805131002556	-0.833805131002556\\
70.625	0.21378	-0.846922215129922	-0.846922215129922\\
70.625	0.21744	-0.8589831488094	-0.8589831488094\\
70.625	0.2211	-0.869987932040986	-0.869987932040986\\
70.625	0.22476	-0.879936564824689	-0.879936564824689\\
70.625	0.22842	-0.888829047160504	-0.888829047160504\\
70.625	0.23208	-0.896665379048432	-0.896665379048432\\
70.625	0.23574	-0.903445560488471	-0.903445560488471\\
70.625	0.2394	-0.909169591480623	-0.909169591480623\\
70.625	0.24306	-0.913837472024888	-0.913837472024888\\
70.625	0.24672	-0.917449202121265	-0.917449202121265\\
70.625	0.25038	-0.92000478176975	-0.92000478176975\\
70.625	0.25404	-0.921504210970352	-0.921504210970352\\
70.625	0.2577	-0.921947489723066	-0.921947489723066\\
70.625	0.26136	-0.921334618027893	-0.921334618027893\\
70.625	0.26502	-0.919665595884832	-0.919665595884832\\
70.625	0.26868	-0.91694042329388	-0.91694042329388\\
70.625	0.27234	-0.913159100255044	-0.913159100255044\\
70.625	0.276	-0.90832162676832	-0.90832162676832\\
71	0.093	0.167512936127738	0.167512936127738\\
71	0.09666	0.118677544724382	0.118677544724382\\
71	0.10032	0.0708983037689145	0.0708983037689145\\
71	0.10398	0.0241752132613348	0.0241752132613348\\
71	0.10764	-0.0214917267983573	-0.0214917267983573\\
71	0.1113	-0.0661025164101616	-0.0661025164101616\\
71	0.11496	-0.10965715557408	-0.10965715557408\\
71	0.11862	-0.152155644290109	-0.152155644290109\\
71	0.12228	-0.193597982558251	-0.193597982558251\\
71	0.12594	-0.233984170378505	-0.233984170378505\\
71	0.1296	-0.273314207750871	-0.273314207750871\\
71	0.13326	-0.31158809467535	-0.31158809467535\\
71	0.13692	-0.348805831151941	-0.348805831151941\\
71	0.14058	-0.384967417180646	-0.384967417180646\\
71	0.14424	-0.420072852761462	-0.420072852761462\\
71	0.1479	-0.454122137894391	-0.454122137894391\\
71	0.15156	-0.487115272579433	-0.487115272579433\\
71	0.15522	-0.519052256816586	-0.519052256816586\\
71	0.15888	-0.549933090605851	-0.549933090605851\\
71	0.16254	-0.579757773947229	-0.579757773947229\\
71	0.1662	-0.608526306840721	-0.608526306840721\\
71	0.16986	-0.636238689286322	-0.636238689286322\\
71	0.17352	-0.662894921284039	-0.662894921284039\\
71	0.17718	-0.688495002833868	-0.688495002833868\\
71	0.18084	-0.713038933935807	-0.713038933935807\\
71	0.1845	-0.736526714589862	-0.736526714589862\\
71	0.18816	-0.758958344796025	-0.758958344796025\\
71	0.19182	-0.780333824554303	-0.780333824554303\\
71	0.19548	-0.800653153864691	-0.800653153864691\\
71	0.19914	-0.819916332727195	-0.819916332727195\\
71	0.2028	-0.83812336114181	-0.83812336114181\\
71	0.20646	-0.855274239108535	-0.855274239108535\\
71	0.21012	-0.871368966627372	-0.871368966627372\\
71	0.21378	-0.886407543698325	-0.886407543698325\\
71	0.21744	-0.900389970321391	-0.900389970321391\\
71	0.2211	-0.913316246496565	-0.913316246496565\\
71	0.22476	-0.925186372223855	-0.925186372223855\\
71	0.22842	-0.936000347503255	-0.936000347503255\\
71	0.23208	-0.94575817233477	-0.94575817233477\\
71	0.23574	-0.954459846718396	-0.954459846718396\\
71	0.2394	-0.962105370654136	-0.962105370654136\\
71	0.24306	-0.968694744141985	-0.968694744141985\\
71	0.24672	-0.97422796718195	-0.97422796718195\\
71	0.25038	-0.978705039774022	-0.978705039774022\\
71	0.25404	-0.982125961918212	-0.982125961918212\\
71	0.2577	-0.98449073361451	-0.98449073361451\\
71	0.26136	-0.985799354862924	-0.985799354862924\\
71	0.26502	-0.986051825663451	-0.986051825663451\\
71	0.26868	-0.985248146016086	-0.985248146016086\\
71	0.27234	-0.983388315920838	-0.983388315920838\\
71	0.276	-0.980472335377698	-0.980472335377698\\
71.375	0.093	0.192214247796969	0.192214247796969\\
71.375	0.09666	0.141457363450025	0.141457363450025\\
71.375	0.10032	0.0917566295509723	0.0917566295509723\\
71.375	0.10398	0.043112046099805	0.043112046099805\\
71.375	0.10764	-0.0044763869034728	-0.0044763869034728\\
71.375	0.1113	-0.0510086694588647	-0.0510086694588647\\
71.375	0.11496	-0.0964848015663691	-0.0964848015663691\\
71.375	0.11862	-0.140904783225984	-0.140904783225984\\
71.375	0.12228	-0.184268614437713	-0.184268614437713\\
71.375	0.12594	-0.226576295201555	-0.226576295201555\\
71.375	0.1296	-0.267827825517507	-0.267827825517507\\
71.375	0.13326	-0.308023205385573	-0.308023205385573\\
71.375	0.13692	-0.347162434805752	-0.347162434805752\\
71.375	0.14058	-0.385245513778043	-0.385245513778043\\
71.375	0.14424	-0.422272442302446	-0.422272442302446\\
71.375	0.1479	-0.458243220378961	-0.458243220378961\\
71.375	0.15156	-0.493157848007589	-0.493157848007589\\
71.375	0.15522	-0.52701632518833	-0.52701632518833\\
71.375	0.15888	-0.559818651921182	-0.559818651921182\\
71.375	0.16254	-0.591564828206148	-0.591564828206148\\
71.375	0.1662	-0.622254854043224	-0.622254854043224\\
71.375	0.16986	-0.651888729432412	-0.651888729432412\\
71.375	0.17352	-0.680466454373717	-0.680466454373717\\
71.375	0.17718	-0.707988028867128	-0.707988028867128\\
71.375	0.18084	-0.734453452912657	-0.734453452912657\\
71.375	0.1845	-0.759862726510299	-0.759862726510299\\
71.375	0.18816	-0.784215849660049	-0.784215849660049\\
71.375	0.19182	-0.807512822361912	-0.807512822361912\\
71.375	0.19548	-0.82975364461589	-0.82975364461589\\
71.375	0.19914	-0.850938316421978	-0.850938316421978\\
71.375	0.2028	-0.871066837780178	-0.871066837780178\\
71.375	0.20646	-0.890139208690493	-0.890139208690493\\
71.375	0.21012	-0.908155429152918	-0.908155429152918\\
71.375	0.21378	-0.925115499167455	-0.925115499167455\\
71.375	0.21744	-0.941019418734109	-0.941019418734109\\
71.375	0.2211	-0.95586718785287	-0.95586718785287\\
71.375	0.22476	-0.969658806523745	-0.969658806523745\\
71.375	0.22842	-0.982394274746735	-0.982394274746735\\
71.375	0.23208	-0.994073592521834	-0.994073592521834\\
71.375	0.23574	-1.00469675984905	-1.00469675984905\\
71.375	0.2394	-1.01426377672837	-1.01426377672837\\
71.375	0.24306	-1.02277464315981	-1.02277464315981\\
71.375	0.24672	-1.03022935914336	-1.03022935914336\\
71.375	0.25038	-1.03662792467902	-1.03662792467902\\
71.375	0.25404	-1.0419703397668	-1.0419703397668\\
71.375	0.2577	-1.04625660440668	-1.04625660440668\\
71.375	0.26136	-1.04948671859868	-1.04948671859868\\
71.375	0.26502	-1.0516606823428	-1.0516606823428\\
71.375	0.26868	-1.05277849563902	-1.05277849563902\\
71.375	0.27234	-1.05284015848735	-1.05284015848735\\
71.375	0.276	-1.05184567088781	-1.05184567088781\\
71.75	0.093	0.217692932565476	0.217692932565476\\
71.75	0.09666	0.165014555274948	0.165014555274948\\
71.75	0.10032	0.113392328432305	0.113392328432305\\
71.75	0.10398	0.0628262520375523	0.0628262520375523\\
71.75	0.10764	0.0133163260906869	0.0133163260906869\\
71.75	0.1113	-0.0351374494082908	-0.0351374494082908\\
71.75	0.11496	-0.0825350744593827	-0.0825350744593827\\
71.75	0.11862	-0.128876549062585	-0.128876549062585\\
71.75	0.12228	-0.1741618732179	-0.1741618732179\\
71.75	0.12594	-0.218391046925328	-0.218391046925328\\
71.75	0.1296	-0.261564070184867	-0.261564070184867\\
71.75	0.13326	-0.303680942996521	-0.303680942996521\\
71.75	0.13692	-0.344741665360286	-0.344741665360286\\
71.75	0.14058	-0.384746237276164	-0.384746237276164\\
71.75	0.14424	-0.423694658744153	-0.423694658744153\\
71.75	0.1479	-0.461586929764255	-0.461586929764255\\
71.75	0.15156	-0.498423050336469	-0.498423050336469\\
71.75	0.15522	-0.534203020460797	-0.534203020460797\\
71.75	0.15888	-0.568926840137236	-0.568926840137236\\
71.75	0.16254	-0.602594509365787	-0.602594509365787\\
71.75	0.1662	-0.635206028146451	-0.635206028146451\\
71.75	0.16986	-0.666761396479232	-0.666761396479232\\
71.75	0.17352	-0.697260614364117	-0.697260614364117\\
71.75	0.17718	-0.726703681801121	-0.726703681801121\\
71.75	0.18084	-0.755090598790231	-0.755090598790231\\
71.75	0.1845	-0.78242136533146	-0.78242136533146\\
71.75	0.18816	-0.808695981424798	-0.808695981424798\\
71.75	0.19182	-0.833914447070248	-0.833914447070248\\
71.75	0.19548	-0.858076762267811	-0.858076762267811\\
71.75	0.19914	-0.881182927017486	-0.881182927017486\\
71.75	0.2028	-0.903232941319277	-0.903232941319277\\
71.75	0.20646	-0.924226805173173	-0.924226805173173\\
71.75	0.21012	-0.944164518579185	-0.944164518579185\\
71.75	0.21378	-0.96304608153731	-0.96304608153731\\
71.75	0.21744	-0.980871494047551	-0.980871494047551\\
71.75	0.2211	-0.9976407561099	-0.9976407561099\\
71.75	0.22476	-1.01335386772436	-1.01335386772436\\
71.75	0.22842	-1.02801082889094	-1.02801082889094\\
71.75	0.23208	-1.04161163960962	-1.04161163960962\\
71.75	0.23574	-1.05415629988042	-1.05415629988042\\
71.75	0.2394	-1.06564480970334	-1.06564480970334\\
71.75	0.24306	-1.07607716907836	-1.07607716907836\\
71.75	0.24672	-1.0854533780055	-1.0854533780055\\
71.75	0.25038	-1.09377343648474	-1.09377343648474\\
71.75	0.25404	-1.10103734451611	-1.10103734451611\\
71.75	0.2577	-1.10724510209958	-1.10724510209958\\
71.75	0.26136	-1.11239670923516	-1.11239670923516\\
71.75	0.26502	-1.11649216592287	-1.11649216592287\\
71.75	0.26868	-1.11953147216268	-1.11953147216268\\
71.75	0.27234	-1.1215146279546	-1.1215146279546\\
71.75	0.276	-1.12244163329864	-1.12244163329864\\
72.125	0.093	0.243948990433258	0.243948990433258\\
72.125	0.09666	0.189349120199142	0.189349120199142\\
72.125	0.10032	0.135805400412914	0.135805400412914\\
72.125	0.10398	0.0833178310745749	0.0833178310745749\\
72.125	0.10764	0.031886412184122	0.031886412184122\\
72.125	0.1113	-0.0184888562584433	-0.0184888562584433\\
72.125	0.11496	-0.067807974253121	-0.067807974253121\\
72.125	0.11862	-0.116070941799909	-0.116070941799909\\
72.125	0.12228	-0.163277758898812	-0.163277758898812\\
72.125	0.12594	-0.209428425549827	-0.209428425549827\\
72.125	0.1296	-0.254522941752954	-0.254522941752954\\
72.125	0.13326	-0.298561307508192	-0.298561307508192\\
72.125	0.13692	-0.341543522815544	-0.341543522815544\\
72.125	0.14058	-0.383469587675008	-0.383469587675008\\
72.125	0.14424	-0.424339502086585	-0.424339502086585\\
72.125	0.1479	-0.464153266050275	-0.464153266050275\\
72.125	0.15156	-0.502910879566076	-0.502910879566076\\
72.125	0.15522	-0.540612342633988	-0.540612342633988\\
72.125	0.15888	-0.577257655254016	-0.577257655254016\\
72.125	0.16254	-0.612846817426153	-0.612846817426153\\
72.125	0.1662	-0.647379829150408	-0.647379829150408\\
72.125	0.16986	-0.680856690426768	-0.680856690426768\\
72.125	0.17352	-0.713277401255241	-0.713277401255241\\
72.125	0.17718	-0.744641961635832	-0.744641961635832\\
72.125	0.18084	-0.77495037156853	-0.77495037156853\\
72.125	0.1845	-0.804202631053346	-0.804202631053346\\
72.125	0.18816	-0.832398740090271	-0.832398740090271\\
72.125	0.19182	-0.859538698679309	-0.859538698679309\\
72.125	0.19548	-0.885622506820456	-0.885622506820456\\
72.125	0.19914	-0.910650164513719	-0.910650164513719\\
72.125	0.2028	-0.934621671759094	-0.934621671759094\\
72.125	0.20646	-0.957537028556581	-0.957537028556581\\
72.125	0.21012	-0.979396234906181	-0.979396234906181\\
72.125	0.21378	-1.00019929080789	-1.00019929080789\\
72.125	0.21744	-1.01994619626172	-1.01994619626172\\
72.125	0.2211	-1.03863695126765	-1.03863695126765\\
72.125	0.22476	-1.0562715558257	-1.0562715558257\\
72.125	0.22842	-1.07285000993586	-1.07285000993586\\
72.125	0.23208	-1.08837231359814	-1.08837231359814\\
72.125	0.23574	-1.10283846681253	-1.10283846681253\\
72.125	0.2394	-1.11624846957903	-1.11624846957903\\
72.125	0.24306	-1.12860232189763	-1.12860232189763\\
72.125	0.24672	-1.13990002376836	-1.13990002376836\\
72.125	0.25038	-1.15014157519119	-1.15014157519119\\
72.125	0.25404	-1.15932697616614	-1.15932697616614\\
72.125	0.2577	-1.1674562266932	-1.1674562266932\\
72.125	0.26136	-1.17452932677238	-1.17452932677238\\
72.125	0.26502	-1.18054627640366	-1.18054627640366\\
72.125	0.26868	-1.18550707558706	-1.18550707558706\\
72.125	0.27234	-1.18941172432257	-1.18941172432257\\
72.125	0.276	-1.19226022261019	-1.19226022261019\\
72.5	0.093	0.270982421400315	0.270982421400315\\
72.5	0.09666	0.214461058222615	0.214461058222615\\
72.5	0.10032	0.158995845492797	0.158995845492797\\
72.5	0.10398	0.104586783210871	0.104586783210871\\
72.5	0.10764	0.0512338713768323	0.0512338713768323\\
72.5	0.1113	-0.00106289000931881	-0.00106289000931881\\
72.5	0.11496	-0.0523035009475841	-0.0523035009475841\\
72.5	0.11862	-0.10248796143796	-0.10248796143796\\
72.5	0.12228	-0.151616271480448	-0.151616271480448\\
72.5	0.12594	-0.199688431075049	-0.199688431075049\\
72.5	0.1296	-0.246704440221764	-0.246704440221764\\
72.5	0.13326	-0.292664298920589	-0.292664298920589\\
72.5	0.13692	-0.337568007171527	-0.337568007171527\\
72.5	0.14058	-0.381415564974579	-0.381415564974579\\
72.5	0.14424	-0.424206972329742	-0.424206972329742\\
72.5	0.1479	-0.465942229237019	-0.465942229237019\\
72.5	0.15156	-0.506621335696406	-0.506621335696406\\
72.5	0.15522	-0.546244291707906	-0.546244291707906\\
72.5	0.15888	-0.584811097271523	-0.584811097271523\\
72.5	0.16254	-0.622321752387246	-0.622321752387246\\
72.5	0.1662	-0.658776257055081	-0.658776257055081\\
72.5	0.16986	-0.694174611275036	-0.694174611275036\\
72.5	0.17352	-0.728516815047096	-0.728516815047096\\
72.5	0.17718	-0.761802868371268	-0.761802868371268\\
72.5	0.18084	-0.794032771247553	-0.794032771247553\\
72.5	0.1845	-0.825206523675957	-0.825206523675957\\
72.5	0.18816	-0.85532412565647	-0.85532412565647\\
72.5	0.19182	-0.884385577189092	-0.884385577189092\\
72.5	0.19548	-0.91239087827383	-0.91239087827383\\
72.5	0.19914	-0.939340028910676	-0.939340028910676\\
72.5	0.2028	-0.965233029099642	-0.965233029099642\\
72.5	0.20646	-0.990069878840714	-0.990069878840714\\
72.5	0.21012	-1.0138505781339	-1.0138505781339\\
72.5	0.21378	-1.0365751269792	-1.0365751269792\\
72.5	0.21744	-1.05824352537661	-1.05824352537661\\
72.5	0.2211	-1.07885577332613	-1.07885577332613\\
72.5	0.22476	-1.09841187082777	-1.09841187082777\\
72.5	0.22842	-1.11691181788151	-1.11691181788151\\
72.5	0.23208	-1.13435561448738	-1.13435561448738\\
72.5	0.23574	-1.15074326064535	-1.15074326064535\\
72.5	0.2394	-1.16607475635544	-1.16607475635544\\
72.5	0.24306	-1.18035010161763	-1.18035010161763\\
72.5	0.24672	-1.19356929643194	-1.19356929643194\\
72.5	0.25038	-1.20573234079837	-1.20573234079837\\
72.5	0.25404	-1.2168392347169	-1.2168392347169\\
72.5	0.2577	-1.22688997818755	-1.22688997818755\\
72.5	0.26136	-1.23588457121031	-1.23588457121031\\
72.5	0.26502	-1.24382301378518	-1.24382301378518\\
72.5	0.26868	-1.25070530591216	-1.25070530591216\\
72.5	0.27234	-1.25653144759126	-1.25653144759126\\
72.5	0.276	-1.26130143882247	-1.26130143882247\\
72.875	0.093	0.298793225466647	0.298793225466647\\
72.875	0.09666	0.24035036934536	0.24035036934536\\
72.875	0.10032	0.182963663671958	0.182963663671958\\
72.875	0.10398	0.126633108446444	0.126633108446444\\
72.875	0.10764	0.0713587036688179	0.0713587036688179\\
72.875	0.1113	0.0171404493390792	0.0171404493390792\\
72.875	0.11496	-0.0360216545427718	-0.0360216545427718\\
72.875	0.11862	-0.0881276079767336	-0.0881276079767336\\
72.875	0.12228	-0.139177410962809	-0.139177410962809\\
72.875	0.12594	-0.189171063500998	-0.189171063500998\\
72.875	0.1296	-0.238108565591298	-0.238108565591298\\
72.875	0.13326	-0.285989917233711	-0.285989917233711\\
72.875	0.13692	-0.332815118428235	-0.332815118428235\\
72.875	0.14058	-0.378584169174874	-0.378584169174874\\
72.875	0.14424	-0.423297069473625	-0.423297069473625\\
72.875	0.1479	-0.466953819324489	-0.466953819324489\\
72.875	0.15156	-0.509554418727462	-0.509554418727462\\
72.875	0.15522	-0.551098867682548	-0.551098867682548\\
72.875	0.15888	-0.591587166189745	-0.591587166189745\\
72.875	0.16254	-0.631019314249063	-0.631019314249063\\
72.875	0.1662	-0.669395311860486	-0.669395311860486\\
72.875	0.16986	-0.706715159024021	-0.706715159024021\\
72.875	0.17352	-0.742978855739669	-0.742978855739669\\
72.875	0.17718	-0.778186402007429	-0.778186402007429\\
72.875	0.18084	-0.812337797827308	-0.812337797827308\\
72.875	0.1845	-0.845433043199293	-0.845433043199293\\
72.875	0.18816	-0.877472138123389	-0.877472138123389\\
72.875	0.19182	-0.908455082599603	-0.908455082599603\\
72.875	0.19548	-0.938381876627924	-0.938381876627924\\
72.875	0.19914	-0.967252520208362	-0.967252520208362\\
72.875	0.2028	-0.995067013340909	-0.995067013340909\\
72.875	0.20646	-1.02182535602557	-1.02182535602557\\
72.875	0.21012	-1.04752754826234	-1.04752754826234\\
72.875	0.21378	-1.07217359005123	-1.07217359005123\\
72.875	0.21744	-1.09576348139223	-1.09576348139223\\
72.875	0.2211	-1.11829722228533	-1.11829722228533\\
72.875	0.22476	-1.13977481273056	-1.13977481273056\\
72.875	0.22842	-1.16019625272789	-1.16019625272789\\
72.875	0.23208	-1.17956154227734	-1.17956154227734\\
72.875	0.23574	-1.1978706813789	-1.1978706813789\\
72.875	0.2394	-1.21512367003257	-1.21512367003257\\
72.875	0.24306	-1.23132050823836	-1.23132050823836\\
72.875	0.24672	-1.24646119599626	-1.24646119599626\\
72.875	0.25038	-1.26054573330626	-1.26054573330626\\
72.875	0.25404	-1.27357412016838	-1.27357412016838\\
72.875	0.2577	-1.28554635658262	-1.28554635658262\\
72.875	0.26136	-1.29646244254896	-1.29646244254896\\
72.875	0.26502	-1.30632237806742	-1.30632237806742\\
72.875	0.26868	-1.31512616313799	-1.31512616313799\\
72.875	0.27234	-1.32287379776068	-1.32287379776068\\
72.875	0.276	-1.32956528193547	-1.32956528193547\\
73.25	0.093	0.327381402632253	0.327381402632253\\
73.25	0.09666	0.267017053567378	0.267017053567378\\
73.25	0.10032	0.207708854950391	0.207708854950391\\
73.25	0.10398	0.149456806781289	0.149456806781289\\
73.25	0.10764	0.0922609090600769	0.0922609090600769\\
73.25	0.1113	0.0361211617867525	0.0361211617867525\\
73.25	0.11496	-0.0189624350386861	-0.0189624350386861\\
73.25	0.11862	-0.0729898814162354	-0.0729898814162354\\
73.25	0.12228	-0.125961177345897	-0.125961177345897\\
73.25	0.12594	-0.177876322827673	-0.177876322827673\\
73.25	0.1296	-0.228735317861559	-0.228735317861559\\
73.25	0.13326	-0.278538162447558	-0.278538162447558\\
73.25	0.13692	-0.327284856585671	-0.327284856585671\\
73.25	0.14058	-0.374975400275896	-0.374975400275896\\
73.25	0.14424	-0.421609793518234	-0.421609793518234\\
73.25	0.1479	-0.467188036312683	-0.467188036312683\\
73.25	0.15156	-0.511710128659243	-0.511710128659243\\
73.25	0.15522	-0.555176070557917	-0.555176070557917\\
73.25	0.15888	-0.597585862008702	-0.597585862008702\\
73.25	0.16254	-0.6389395030116	-0.6389395030116\\
73.25	0.1662	-0.67923699356661	-0.67923699356661\\
73.25	0.16986	-0.71847833367374	-0.71847833367374\\
73.25	0.17352	-0.756663523332975	-0.756663523332975\\
73.25	0.17718	-0.793792562544323	-0.793792562544323\\
73.25	0.18084	-0.829865451307783	-0.829865451307783\\
73.25	0.1845	-0.864882189623355	-0.864882189623355\\
73.25	0.18816	-0.898842777491039	-0.898842777491039\\
73.25	0.19182	-0.931747214910836	-0.931747214910836\\
73.25	0.19548	-0.963595501882749	-0.963595501882749\\
73.25	0.19914	-0.994387638406771	-0.994387638406771\\
73.25	0.2028	-1.02412362448291	-1.02412362448291\\
73.25	0.20646	-1.05280346011115	-1.05280346011115\\
73.25	0.21012	-1.08042714529151	-1.08042714529151\\
73.25	0.21378	-1.10699468002398	-1.10699468002398\\
73.25	0.21744	-1.13250606430857	-1.13250606430857\\
73.25	0.2211	-1.15696129814527	-1.15696129814527\\
73.25	0.22476	-1.18036038153407	-1.18036038153407\\
73.25	0.22842	-1.202703314475	-1.202703314475\\
73.25	0.23208	-1.22399009696803	-1.22399009696803\\
73.25	0.23574	-1.24422072901318	-1.24422072901318\\
73.25	0.2394	-1.26339521061044	-1.26339521061044\\
73.25	0.24306	-1.28151354175981	-1.28151354175981\\
73.25	0.24672	-1.29857572246129	-1.29857572246129\\
73.25	0.25038	-1.31458175271489	-1.31458175271489\\
73.25	0.25404	-1.32953163252059	-1.32953163252059\\
73.25	0.2577	-1.34342536187841	-1.34342536187841\\
73.25	0.26136	-1.35626294078835	-1.35626294078835\\
73.25	0.26502	-1.36804436925039	-1.36804436925039\\
73.25	0.26868	-1.37876964726455	-1.37876964726455\\
73.25	0.27234	-1.38843877483082	-1.38843877483082\\
73.25	0.276	-1.3970517519492	-1.3970517519492\\
73.625	0.093	0.356746952897134	0.356746952897134\\
73.625	0.09666	0.294461110888671	0.294461110888671\\
73.625	0.10032	0.233231419328097	0.233231419328097\\
73.625	0.10398	0.173057878215409	0.173057878215409\\
73.625	0.10764	0.11394048755061	0.11394048755061\\
73.625	0.1113	0.0558792473336975	0.0558792473336975\\
73.625	0.11496	-0.0011258424353251	-0.0011258424353251\\
73.625	0.11862	-0.057074781756462	-0.057074781756462\\
73.625	0.12228	-0.111967570629711	-0.111967570629711\\
73.625	0.12594	-0.165804209055073	-0.165804209055073\\
73.625	0.1296	-0.218584697032547	-0.218584697032547\\
73.625	0.13326	-0.270309034562133	-0.270309034562133\\
73.625	0.13692	-0.32097722164383	-0.32097722164383\\
73.625	0.14058	-0.370589258277645	-0.370589258277645\\
73.625	0.14424	-0.419145144463565	-0.419145144463565\\
73.625	0.1479	-0.466644880201601	-0.466644880201601\\
73.625	0.15156	-0.513088465491749	-0.513088465491749\\
73.625	0.15522	-0.55847590033401	-0.55847590033401\\
73.625	0.15888	-0.602807184728383	-0.602807184728383\\
73.625	0.16254	-0.646082318674868	-0.646082318674868\\
73.625	0.1662	-0.688301302173466	-0.688301302173466\\
73.625	0.16986	-0.729464135224176	-0.729464135224176\\
73.625	0.17352	-0.769570817826999	-0.769570817826999\\
73.625	0.17718	-0.808621349981934	-0.808621349981934\\
73.625	0.18084	-0.846615731688982	-0.846615731688982\\
73.625	0.1845	-0.883553962948141	-0.883553962948141\\
73.625	0.18816	-0.919436043759417	-0.919436043759417\\
73.625	0.19182	-0.954261974122802	-0.954261974122802\\
73.625	0.19548	-0.988031754038299	-0.988031754038299\\
73.625	0.19914	-1.02074538350591	-1.02074538350591\\
73.625	0.2028	-1.05240286252563	-1.05240286252563\\
73.625	0.20646	-1.08300419109746	-1.08300419109746\\
73.625	0.21012	-1.11254936922141	-1.11254936922141\\
73.625	0.21378	-1.14103839689747	-1.14103839689747\\
73.625	0.21744	-1.16847127412564	-1.16847127412564\\
73.625	0.2211	-1.19484800090592	-1.19484800090592\\
73.625	0.22476	-1.22016857723832	-1.22016857723832\\
73.625	0.22842	-1.24443300312283	-1.24443300312283\\
73.625	0.23208	-1.26764127855945	-1.26764127855945\\
73.625	0.23574	-1.28979340354818	-1.28979340354818\\
73.625	0.2394	-1.31088937808903	-1.31088937808903\\
73.625	0.24306	-1.33092920218199	-1.33092920218199\\
73.625	0.24672	-1.34991287582706	-1.34991287582706\\
73.625	0.25038	-1.36784039902424	-1.36784039902424\\
73.625	0.25404	-1.38471177177353	-1.38471177177353\\
73.625	0.2577	-1.40052699407494	-1.40052699407494\\
73.625	0.26136	-1.41528606592846	-1.41528606592846\\
73.625	0.26502	-1.42898898733409	-1.42898898733409\\
73.625	0.26868	-1.44163575829184	-1.44163575829184\\
73.625	0.27234	-1.45322637880169	-1.45322637880169\\
73.625	0.276	-1.46376084886366	-1.46376084886366\\
74	0.093	0.38688987626129	0.38688987626129\\
74	0.09666	0.32268254130924	0.32268254130924\\
74	0.10032	0.25953135680508	0.25953135680508\\
74	0.10398	0.197436322748806	0.197436322748806\\
74	0.10764	0.136397439140419	0.136397439140419\\
74	0.1113	0.0764147059799214	0.0764147059799214\\
74	0.11496	0.0174881232673112	0.0174881232673112\\
74	0.11862	-0.0403823089974132	-0.0403823089974132\\
74	0.12228	-0.0971965908142483	-0.0971965908142483\\
74	0.12594	-0.152954722183197	-0.152954722183197\\
74	0.1296	-0.207656703104257	-0.207656703104257\\
74	0.13326	-0.261302533577431	-0.261302533577431\\
74	0.13692	-0.313892213602718	-0.313892213602718\\
74	0.14058	-0.365425743180116	-0.365425743180116\\
74	0.14424	-0.415903122309627	-0.415903122309627\\
74	0.1479	-0.465324350991247	-0.465324350991247\\
74	0.15156	-0.513689429224983	-0.513689429224983\\
74	0.15522	-0.560998357010831	-0.560998357010831\\
74	0.15888	-0.607251134348788	-0.607251134348788\\
74	0.16254	-0.652447761238861	-0.652447761238861\\
74	0.1662	-0.696588237681047	-0.696588237681047\\
74	0.16986	-0.739672563675345	-0.739672563675345\\
74	0.17352	-0.781700739221755	-0.781700739221755\\
74	0.17718	-0.822672764320278	-0.822672764320278\\
74	0.18084	-0.862588638970906	-0.862588638970906\\
74	0.1845	-0.901448363173653	-0.901448363173653\\
74	0.18816	-0.939251936928516	-0.939251936928516\\
74	0.19182	-0.975999360235485	-0.975999360235485\\
74	0.19548	-1.01169063309457	-1.01169063309457\\
74	0.19914	-1.04632575550577	-1.04632575550577\\
74	0.2028	-1.07990472746908	-1.07990472746908\\
74	0.20646	-1.11242754898449	-1.11242754898449\\
74	0.21012	-1.14389422005203	-1.14389422005203\\
74	0.21378	-1.17430474067167	-1.17430474067167\\
74	0.21744	-1.20365911084343	-1.20365911084343\\
74	0.2211	-1.2319573305673	-1.2319573305673\\
74	0.22476	-1.25919939984328	-1.25919939984328\\
74	0.22842	-1.28538531867138	-1.28538531867138\\
74	0.23208	-1.31051508705159	-1.31051508705159\\
74	0.23574	-1.33458870498391	-1.33458870498391\\
74	0.2394	-1.35760617246834	-1.35760617246834\\
74	0.24306	-1.37956748950488	-1.37956748950488\\
74	0.24672	-1.40047265609354	-1.40047265609354\\
74	0.25038	-1.42032167223431	-1.42032167223431\\
74	0.25404	-1.43911453792719	-1.43911453792719\\
74	0.2577	-1.45685125317219	-1.45685125317219\\
74	0.26136	-1.47353181796929	-1.47353181796929\\
74	0.26502	-1.48915623231851	-1.48915623231851\\
74	0.26868	-1.50372449621984	-1.50372449621984\\
74	0.27234	-1.51723660967329	-1.51723660967329\\
74	0.276	-1.52969257267885	-1.52969257267885\\
};
\end{axis}

\begin{axis}[%
width=6.159cm,
height=3.097cm,
at={(0cm,4.301cm)},
scale only axis,
xmin=56,
xmax=74,
tick align=outside,
xlabel style={font=\color{white!15!black}},
xlabel={$L_{cut}$},
ymin=0.093,
ymax=0.276,
ylabel style={font=\color{white!15!black}},
ylabel={$D_{rlx}$},
zmin=-3.90013564007816,
zmax=52.2197035510817,
zlabel style={font=\color{white!15!black}},
zlabel={$u(t-2)$},
view={-140}{50},
axis background/.style={fill=white},
xmajorgrids,
ymajorgrids,
zmajorgrids
]
\addplot3[only marks, mark=*, mark options={}, mark size=1.5000pt, color=mycolor1, fill=mycolor1] table[row sep=crcr]{%
x	y	z\\
74	0.123	0.706917161850943\\
72	0.113	-0.725188933625476\\
61	0.095	-2.23969084196957\\
56	0.093	-0.663995623424443\\
};
\addplot3[only marks, mark=*, mark options={}, mark size=1.5000pt, color=mycolor2, fill=mycolor2] table[row sep=crcr]{%
x	y	z\\
67	0.276	45.9418866067325\\
66	0.255	36.5734993512285\\
62	0.209	18.3955084875498\\
57	0.193	12.6775369279536\\
};
\addplot3[only marks, mark=*, mark options={}, mark size=1.5000pt, color=black, fill=black] table[row sep=crcr]{%
x	y	z\\
69	0.104	-2.29954754700143\\
};
\addplot3[only marks, mark=*, mark options={}, mark size=1.5000pt, color=black, fill=black] table[row sep=crcr]{%
x	y	z\\
64	0.23	25.9519556411493\\
};

\addplot3[%
surf,
fill opacity=0.7, shader=interp, colormap={mymap}{[1pt] rgb(0pt)=(1,0.905882,0); rgb(1pt)=(1,0.901964,0); rgb(2pt)=(1,0.898051,0); rgb(3pt)=(1,0.894144,0); rgb(4pt)=(1,0.890243,0); rgb(5pt)=(1,0.886349,0); rgb(6pt)=(1,0.88246,0); rgb(7pt)=(1,0.878577,0); rgb(8pt)=(1,0.8747,0); rgb(9pt)=(1,0.870829,0); rgb(10pt)=(1,0.866964,0); rgb(11pt)=(1,0.863106,0); rgb(12pt)=(1,0.859253,0); rgb(13pt)=(1,0.855406,0); rgb(14pt)=(1,0.851566,0); rgb(15pt)=(1,0.847732,0); rgb(16pt)=(1,0.843903,0); rgb(17pt)=(1,0.840081,0); rgb(18pt)=(1,0.836265,0); rgb(19pt)=(1,0.832455,0); rgb(20pt)=(1,0.828652,0); rgb(21pt)=(1,0.824854,0); rgb(22pt)=(1,0.821063,0); rgb(23pt)=(1,0.817278,0); rgb(24pt)=(1,0.8135,0); rgb(25pt)=(1,0.809727,0); rgb(26pt)=(1,0.805961,0); rgb(27pt)=(1,0.8022,0); rgb(28pt)=(1,0.798445,0); rgb(29pt)=(1,0.794696,0); rgb(30pt)=(1,0.790953,0); rgb(31pt)=(1,0.787215,0); rgb(32pt)=(1,0.783484,0); rgb(33pt)=(1,0.779758,0); rgb(34pt)=(1,0.776038,0); rgb(35pt)=(1,0.772324,0); rgb(36pt)=(1,0.768615,0); rgb(37pt)=(1,0.764913,0); rgb(38pt)=(1,0.761217,0); rgb(39pt)=(1,0.757527,0); rgb(40pt)=(1,0.753843,0); rgb(41pt)=(1,0.750165,0); rgb(42pt)=(1,0.746493,0); rgb(43pt)=(1,0.742827,0); rgb(44pt)=(1,0.739167,0); rgb(45pt)=(1,0.735514,0); rgb(46pt)=(1,0.731867,0); rgb(47pt)=(1,0.728226,0); rgb(48pt)=(1,0.724591,0); rgb(49pt)=(1,0.720963,0); rgb(50pt)=(1,0.717341,0); rgb(51pt)=(1,0.713725,0); rgb(52pt)=(0.999994,0.710077,0); rgb(53pt)=(0.999974,0.706363,0); rgb(54pt)=(0.999942,0.702592,0); rgb(55pt)=(0.999898,0.698775,0); rgb(56pt)=(0.999841,0.694921,0); rgb(57pt)=(0.999771,0.691039,0); rgb(58pt)=(0.99969,0.687139,0); rgb(59pt)=(0.999596,0.68323,0); rgb(60pt)=(0.99949,0.679323,0); rgb(61pt)=(0.999372,0.675427,0); rgb(62pt)=(0.999242,0.67155,0); rgb(63pt)=(0.9991,0.667704,0); rgb(64pt)=(0.998946,0.663897,0); rgb(65pt)=(0.998781,0.660138,0); rgb(66pt)=(0.998605,0.656439,0); rgb(67pt)=(0.998416,0.652807,0); rgb(68pt)=(0.998217,0.649253,0); rgb(69pt)=(0.998006,0.645786,0); rgb(70pt)=(0.997785,0.642416,0); rgb(71pt)=(0.997552,0.639152,0); rgb(72pt)=(0.997308,0.636004,0); rgb(73pt)=(0.997053,0.632982,0); rgb(74pt)=(0.996788,0.630095,0); rgb(75pt)=(0.996512,0.627352,0); rgb(76pt)=(0.996226,0.624763,0); rgb(77pt)=(0.995851,0.622329,0); rgb(78pt)=(0.99494,0.619997,0); rgb(79pt)=(0.99345,0.617753,0); rgb(80pt)=(0.991419,0.61559,0); rgb(81pt)=(0.988885,0.613503,0); rgb(82pt)=(0.985886,0.611486,0); rgb(83pt)=(0.98246,0.609532,0); rgb(84pt)=(0.978643,0.607636,0); rgb(85pt)=(0.974475,0.605791,0); rgb(86pt)=(0.969992,0.603992,0); rgb(87pt)=(0.965232,0.602233,0); rgb(88pt)=(0.960233,0.600507,0); rgb(89pt)=(0.955033,0.598808,0); rgb(90pt)=(0.949669,0.59713,0); rgb(91pt)=(0.94418,0.595468,0); rgb(92pt)=(0.938602,0.593815,0); rgb(93pt)=(0.932974,0.592166,0); rgb(94pt)=(0.927333,0.590513,0); rgb(95pt)=(0.921717,0.588852,0); rgb(96pt)=(0.916164,0.587176,0); rgb(97pt)=(0.910711,0.585479,0); rgb(98pt)=(0.905397,0.583755,0); rgb(99pt)=(0.900258,0.581999,0); rgb(100pt)=(0.895333,0.580203,0); rgb(101pt)=(0.890659,0.578362,0); rgb(102pt)=(0.886275,0.576471,0); rgb(103pt)=(0.882047,0.574545,0); rgb(104pt)=(0.877819,0.572608,0); rgb(105pt)=(0.873592,0.57066,0); rgb(106pt)=(0.869366,0.568701,0); rgb(107pt)=(0.865143,0.566733,0); rgb(108pt)=(0.860924,0.564756,0); rgb(109pt)=(0.856708,0.562771,0); rgb(110pt)=(0.852497,0.560778,0); rgb(111pt)=(0.848292,0.558779,0); rgb(112pt)=(0.844092,0.556774,0); rgb(113pt)=(0.8399,0.554763,0); rgb(114pt)=(0.835716,0.552749,0); rgb(115pt)=(0.831541,0.55073,0); rgb(116pt)=(0.827374,0.548709,0); rgb(117pt)=(0.823219,0.546686,0); rgb(118pt)=(0.819074,0.54466,0); rgb(119pt)=(0.81494,0.542635,0); rgb(120pt)=(0.81082,0.540609,0); rgb(121pt)=(0.806712,0.538584,0); rgb(122pt)=(0.802619,0.53656,0); rgb(123pt)=(0.798541,0.534539,0); rgb(124pt)=(0.794478,0.532521,0); rgb(125pt)=(0.790431,0.530506,0); rgb(126pt)=(0.786402,0.528496,0); rgb(127pt)=(0.782391,0.526491,0); rgb(128pt)=(0.77841,0.524489,0); rgb(129pt)=(0.774523,0.522478,0); rgb(130pt)=(0.770731,0.520455,0); rgb(131pt)=(0.767022,0.518424,0); rgb(132pt)=(0.763384,0.516385,0); rgb(133pt)=(0.759804,0.514339,0); rgb(134pt)=(0.756272,0.51229,0); rgb(135pt)=(0.752775,0.510237,0); rgb(136pt)=(0.749302,0.508182,0); rgb(137pt)=(0.74584,0.506128,0); rgb(138pt)=(0.742378,0.504075,0); rgb(139pt)=(0.738904,0.502025,0); rgb(140pt)=(0.735406,0.499979,0); rgb(141pt)=(0.731872,0.49794,0); rgb(142pt)=(0.72829,0.495909,0); rgb(143pt)=(0.724649,0.493887,0); rgb(144pt)=(0.720936,0.491875,0); rgb(145pt)=(0.71714,0.489876,0); rgb(146pt)=(0.713249,0.487891,0); rgb(147pt)=(0.709251,0.485921,0); rgb(148pt)=(0.705134,0.483968,0); rgb(149pt)=(0.700887,0.482033,0); rgb(150pt)=(0.696497,0.480118,0); rgb(151pt)=(0.691952,0.478225,0); rgb(152pt)=(0.687242,0.476355,0); rgb(153pt)=(0.682353,0.47451,0); rgb(154pt)=(0.677195,0.472696,0); rgb(155pt)=(0.6717,0.470916,0); rgb(156pt)=(0.665891,0.469169,0); rgb(157pt)=(0.659791,0.46745,0); rgb(158pt)=(0.653423,0.465756,0); rgb(159pt)=(0.64681,0.464084,0); rgb(160pt)=(0.639976,0.462432,0); rgb(161pt)=(0.632943,0.460795,0); rgb(162pt)=(0.625734,0.459171,0); rgb(163pt)=(0.618373,0.457556,0); rgb(164pt)=(0.610882,0.455948,0); rgb(165pt)=(0.603284,0.454343,0); rgb(166pt)=(0.595604,0.452737,0); rgb(167pt)=(0.587863,0.451129,0); rgb(168pt)=(0.580084,0.449514,0); rgb(169pt)=(0.572292,0.447889,0); rgb(170pt)=(0.564508,0.446252,0); rgb(171pt)=(0.556756,0.444599,0); rgb(172pt)=(0.549059,0.442927,0); rgb(173pt)=(0.54144,0.441232,0); rgb(174pt)=(0.533922,0.439512,0); rgb(175pt)=(0.526529,0.437764,0); rgb(176pt)=(0.519282,0.435983,0); rgb(177pt)=(0.512206,0.434168,0); rgb(178pt)=(0.505323,0.432315,0); rgb(179pt)=(0.498628,0.430422,3.92506e-06); rgb(180pt)=(0.491973,0.428504,3.49981e-05); rgb(181pt)=(0.485331,0.426562,9.63073e-05); rgb(182pt)=(0.478704,0.424596,0.000186979); rgb(183pt)=(0.472096,0.422609,0.000306141); rgb(184pt)=(0.465508,0.420599,0.00045292); rgb(185pt)=(0.458942,0.418567,0.000626441); rgb(186pt)=(0.452401,0.416515,0.000825833); rgb(187pt)=(0.445885,0.414441,0.00105022); rgb(188pt)=(0.439399,0.412348,0.00129873); rgb(189pt)=(0.432942,0.410234,0.00157049); rgb(190pt)=(0.426518,0.408102,0.00186463); rgb(191pt)=(0.420129,0.40595,0.00218028); rgb(192pt)=(0.413777,0.40378,0.00251655); rgb(193pt)=(0.407464,0.401592,0.00287258); rgb(194pt)=(0.401191,0.399386,0.00324749); rgb(195pt)=(0.394962,0.397164,0.00364042); rgb(196pt)=(0.388777,0.394925,0.00405048); rgb(197pt)=(0.38264,0.39267,0.00447681); rgb(198pt)=(0.376552,0.390399,0.00491852); rgb(199pt)=(0.370516,0.388113,0.00537476); rgb(200pt)=(0.364532,0.385812,0.00584464); rgb(201pt)=(0.358605,0.383497,0.00632729); rgb(202pt)=(0.352735,0.381168,0.00682184); rgb(203pt)=(0.346925,0.378826,0.00732741); rgb(204pt)=(0.341176,0.376471,0.00784314); rgb(205pt)=(0.335485,0.374093,0.00847245); rgb(206pt)=(0.329843,0.371682,0.00930909); rgb(207pt)=(0.324249,0.369242,0.0103377); rgb(208pt)=(0.318701,0.366772,0.0115428); rgb(209pt)=(0.313198,0.364275,0.0129091); rgb(210pt)=(0.307739,0.361753,0.0144211); rgb(211pt)=(0.302322,0.359206,0.0160634); rgb(212pt)=(0.296945,0.356637,0.0178207); rgb(213pt)=(0.291607,0.354048,0.0196776); rgb(214pt)=(0.286307,0.35144,0.0216186); rgb(215pt)=(0.281043,0.348814,0.0236284); rgb(216pt)=(0.275813,0.346172,0.0256916); rgb(217pt)=(0.270616,0.343517,0.0277927); rgb(218pt)=(0.265451,0.340849,0.0299163); rgb(219pt)=(0.260317,0.33817,0.0320472); rgb(220pt)=(0.25521,0.335482,0.0341698); rgb(221pt)=(0.250131,0.332786,0.0362688); rgb(222pt)=(0.245078,0.330085,0.0383287); rgb(223pt)=(0.240048,0.327379,0.0403343); rgb(224pt)=(0.235042,0.324671,0.04227); rgb(225pt)=(0.230056,0.321962,0.0441205); rgb(226pt)=(0.22509,0.319254,0.0458704); rgb(227pt)=(0.220142,0.316548,0.0475043); rgb(228pt)=(0.215212,0.313846,0.0490067); rgb(229pt)=(0.210296,0.311149,0.0503624); rgb(230pt)=(0.205395,0.308459,0.0515759); rgb(231pt)=(0.200514,0.305763,0.052757); rgb(232pt)=(0.195655,0.303061,0.0539242); rgb(233pt)=(0.190817,0.300353,0.0550763); rgb(234pt)=(0.186001,0.297639,0.0562123); rgb(235pt)=(0.181207,0.294918,0.0573313); rgb(236pt)=(0.176434,0.292191,0.0584321); rgb(237pt)=(0.171685,0.289458,0.0595136); rgb(238pt)=(0.166957,0.286719,0.060575); rgb(239pt)=(0.162252,0.283973,0.0616151); rgb(240pt)=(0.15757,0.281221,0.0626328); rgb(241pt)=(0.152911,0.278463,0.0636271); rgb(242pt)=(0.148275,0.275699,0.0645971); rgb(243pt)=(0.143663,0.272929,0.0655416); rgb(244pt)=(0.139074,0.270152,0.0664596); rgb(245pt)=(0.134508,0.26737,0.06735); rgb(246pt)=(0.129967,0.264581,0.0682118); rgb(247pt)=(0.125449,0.261787,0.0690441); rgb(248pt)=(0.120956,0.258986,0.0698456); rgb(249pt)=(0.116487,0.25618,0.0706154); rgb(250pt)=(0.112043,0.253367,0.0713525); rgb(251pt)=(0.107623,0.250549,0.0720557); rgb(252pt)=(0.103229,0.247724,0.0727241); rgb(253pt)=(0.0988592,0.244894,0.0733566); rgb(254pt)=(0.0945149,0.242058,0.0739522); rgb(255pt)=(0.0901961,0.239216,0.0745098)}, mesh/rows=49]
table[row sep=crcr, point meta=\thisrow{c}] {%
%
x	y	z	c\\
56	0.093	-0.774335008038994	-0.774335008038994\\
56	0.09666	-0.624732271062047	-0.624732271062047\\
56	0.10032	-0.449585546727148	-0.449585546727148\\
56	0.10398	-0.248894835034243	-0.248894835034243\\
56	0.10764	-0.022660135983358	-0.022660135983358\\
56	0.1113	0.229118550425509	0.229118550425509\\
56	0.11496	0.506441224192354	0.506441224192354\\
56	0.11862	0.809307885317162	0.809307885317162\\
56	0.12228	1.13771853379996	1.13771853379996\\
56	0.12594	1.49167316964073	1.49167316964073\\
56	0.1296	1.87117179283949	1.87117179283949\\
56	0.13326	2.27621440339622	2.27621440339622\\
56	0.13692	2.70680100131094	2.70680100131094\\
56	0.14058	3.16293158658365	3.16293158658365\\
56	0.14424	3.64460615921431	3.64460615921431\\
56	0.1479	4.15182471920297	4.15182471920297\\
56	0.15156	4.68458726654961	4.68458726654961\\
56	0.15522	5.24289380125422	5.24289380125422\\
56	0.15888	5.82674432331682	5.82674432331682\\
56	0.16254	6.4361388327374	6.4361388327374\\
56	0.1662	7.07107732951595	7.07107732951595\\
56	0.16986	7.73155981365247	7.73155981365247\\
56	0.17352	8.41758628514699	8.41758628514699\\
56	0.17718	9.12915674399949	9.12915674399949\\
56	0.18084	9.86627119020995	9.86627119020995\\
56	0.1845	10.6289296237784	10.6289296237784\\
56	0.18816	11.4171320447048	11.4171320447048\\
56	0.19182	12.2308784529892	12.2308784529892\\
56	0.19548	13.0701688486316	13.0701688486316\\
56	0.19914	13.935003231632	13.935003231632\\
56	0.2028	14.8253816019904	14.8253816019904\\
56	0.20646	15.7413039597067	15.7413039597067\\
56	0.21012	16.682770304781	16.682770304781\\
56	0.21378	17.6497806372133	17.6497806372133\\
56	0.21744	18.6423349570036	18.6423349570036\\
56	0.2211	19.6604332641518	19.6604332641518\\
56	0.22476	20.704075558658	20.704075558658\\
56	0.22842	21.7732618405222	21.7732618405222\\
56	0.23208	22.8679921097444	22.8679921097444\\
56	0.23574	23.9882663663246	23.9882663663246\\
56	0.2394	25.1340846102627	25.1340846102627\\
56	0.24306	26.3054468415589	26.3054468415589\\
56	0.24672	27.5023530602129	27.5023530602129\\
56	0.25038	28.724803266225	28.724803266225\\
56	0.25404	29.9727974595951	29.9727974595951\\
56	0.2577	31.2463356403232	31.2463356403232\\
56	0.26136	32.5454178084092	32.5454178084092\\
56	0.26502	33.8700439638532	33.8700439638532\\
56	0.26868	35.2202141066552	35.2202141066552\\
56	0.27234	36.5959282368151	36.5959282368151\\
56	0.276	37.997186354333	37.997186354333\\
56.375	0.093	-0.895119669503726	-0.895119669503726\\
56.375	0.09666	-0.738288466764782	-0.738288466764782\\
56.375	0.10032	-0.555913276667885	-0.555913276667885\\
56.375	0.10398	-0.347994099212983	-0.347994099212983\\
56.375	0.10764	-0.114530934400108	-0.114530934400108\\
56.375	0.1113	0.144476217770757	0.144476217770757\\
56.375	0.11496	0.429027357299592	0.429027357299592\\
56.375	0.11862	0.739122484186398	0.739122484186398\\
56.375	0.12228	1.07476159843119	1.07476159843119\\
56.375	0.12594	1.43594470003395	1.43594470003395\\
56.375	0.1296	1.82267178899471	1.82267178899471\\
56.375	0.13326	2.23494286531344	2.23494286531344\\
56.375	0.13692	2.67275792899015	2.67275792899015\\
56.375	0.14058	3.13611698002485	3.13611698002485\\
56.375	0.14424	3.62502001841751	3.62502001841751\\
56.375	0.1479	4.13946704416816	4.13946704416816\\
56.375	0.15156	4.6794580572768	4.6794580572768\\
56.375	0.15522	5.24499305774341	5.24499305774341\\
56.375	0.15888	5.83607204556801	5.83607204556801\\
56.375	0.16254	6.45269502075057	6.45269502075057\\
56.375	0.1662	7.09486198329112	7.09486198329112\\
56.375	0.16986	7.76257293318965	7.76257293318965\\
56.375	0.17352	8.45582787044615	8.45582787044615\\
56.375	0.17718	9.17462679506064	9.17462679506064\\
56.375	0.18084	9.91896970703311	9.91896970703311\\
56.375	0.1845	10.6888566063635	10.6888566063635\\
56.375	0.18816	11.484287493052	11.484287493052\\
56.375	0.19182	12.3052623670984	12.3052623670984\\
56.375	0.19548	13.1517812285028	13.1517812285028\\
56.375	0.19914	14.0238440772651	14.0238440772651\\
56.375	0.2028	14.9214509133855	14.9214509133855\\
56.375	0.20646	15.8446017368638	15.8446017368638\\
56.375	0.21012	16.7932965477001	16.7932965477001\\
56.375	0.21378	17.7675353458944	17.7675353458944\\
56.375	0.21744	18.7673181314467	18.7673181314467\\
56.375	0.2211	19.7926449043569	19.7926449043569\\
56.375	0.22476	20.8435156646251	20.8435156646251\\
56.375	0.22842	21.9199304122513	21.9199304122513\\
56.375	0.23208	23.0218891472355	23.0218891472355\\
56.375	0.23574	24.1493918695777	24.1493918695777\\
56.375	0.2394	25.3024385792778	25.3024385792778\\
56.375	0.24306	26.4810292763359	26.4810292763359\\
56.375	0.24672	27.685163960752	27.685163960752\\
56.375	0.25038	28.9148426325261	28.9148426325261\\
56.375	0.25404	30.1700652916582	30.1700652916582\\
56.375	0.2577	31.4508319381482	31.4508319381482\\
56.375	0.26136	32.7571425719962	32.7571425719962\\
56.375	0.26502	34.0889971932022	34.0889971932022\\
56.375	0.26868	35.4463958017662	35.4463958017662\\
56.375	0.27234	36.8293383976882	36.8293383976882\\
56.375	0.276	38.2378249809681	38.2378249809681\\
56.75	0.093	-1.0135356579913	-1.0135356579913\\
56.75	0.09666	-0.849475989490362	-0.849475989490362\\
56.75	0.10032	-0.659872333631446	-0.659872333631446\\
56.75	0.10398	-0.444724690414553	-0.444724690414553\\
56.75	0.10764	-0.204033059839695	-0.204033059839695\\
56.75	0.1113	0.0622025580931531	0.0622025580931531\\
56.75	0.11496	0.353982163383979	0.353982163383979\\
56.75	0.11862	0.671305756032789	0.671305756032789\\
56.75	0.12228	1.01417333603958	1.01417333603958\\
56.75	0.12594	1.38258490340435	1.38258490340435\\
56.75	0.1296	1.7765404581271	1.7765404581271\\
56.75	0.13326	2.19604000020782	2.19604000020782\\
56.75	0.13692	2.64108352964652	2.64108352964652\\
56.75	0.14058	3.11167104644321	3.11167104644321\\
56.75	0.14424	3.60780255059788	3.60780255059788\\
56.75	0.1479	4.12947804211052	4.12947804211052\\
56.75	0.15156	4.67669752098114	4.67669752098114\\
56.75	0.15522	5.24946098720975	5.24946098720975\\
56.75	0.15888	5.84776844079635	5.84776844079635\\
56.75	0.16254	6.47161988174091	6.47161988174091\\
56.75	0.1662	7.12101531004345	7.12101531004345\\
56.75	0.16986	7.79595472570396	7.79595472570396\\
56.75	0.17352	8.49643812872247	8.49643812872247\\
56.75	0.17718	9.22246551909895	9.22246551909895\\
56.75	0.18084	9.97403689683341	9.97403689683341\\
56.75	0.1845	10.7511522619258	10.7511522619258\\
56.75	0.18816	11.5538116143763	11.5538116143763\\
56.75	0.19182	12.3820149541847	12.3820149541847\\
56.75	0.19548	13.2357622813511	13.2357622813511\\
56.75	0.19914	14.1150535958754	14.1150535958754\\
56.75	0.2028	15.0198888977578	15.0198888977578\\
56.75	0.20646	15.9502681869981	15.9502681869981\\
56.75	0.21012	16.9061914635964	16.9061914635964\\
56.75	0.21378	17.8876587275527	17.8876587275527\\
56.75	0.21744	18.8946699788669	18.8946699788669\\
56.75	0.2211	19.9272252175391	19.9272252175391\\
56.75	0.22476	20.9853244435694	20.9853244435694\\
56.75	0.22842	22.0689676569576	22.0689676569576\\
56.75	0.23208	23.1781548577037	23.1781548577037\\
56.75	0.23574	24.3128860458079	24.3128860458079\\
56.75	0.2394	25.47316122127	25.47316122127\\
56.75	0.24306	26.6589803840901	26.6589803840901\\
56.75	0.24672	27.8703435342682	27.8703435342682\\
56.75	0.25038	29.1072506718043	29.1072506718043\\
56.75	0.25404	30.3697017966984	30.3697017966984\\
56.75	0.2577	31.6576969089504	31.6576969089504\\
56.75	0.26136	32.9712360085604	32.9712360085604\\
56.75	0.26502	34.3103190955284	34.3103190955284\\
56.75	0.26868	35.6749461698544	35.6749461698544\\
56.75	0.27234	37.0651172315383	37.0651172315383\\
56.75	0.276	38.4808322805802	38.4808322805802\\
57.125	0.093	-1.12958297350166	-1.12958297350166\\
57.125	0.09666	-0.958294839238722	-0.958294839238722\\
57.125	0.10032	-0.761462717617816	-0.761462717617816\\
57.125	0.10398	-0.539086608638925	-0.539086608638925\\
57.125	0.10764	-0.29116651230207	-0.29116651230207\\
57.125	0.1113	-0.0177024286072314	-0.0177024286072314\\
57.125	0.11496	0.281305642445599	0.281305642445599\\
57.125	0.11862	0.605857700856399	0.605857700856399\\
57.125	0.12228	0.955953746625189	0.955953746625189\\
57.125	0.12594	1.33159377975196	1.33159377975196\\
57.125	0.1296	1.7327778002367	1.7327778002367\\
57.125	0.13326	2.15950580807941	2.15950580807941\\
57.125	0.13692	2.61177780328011	2.61177780328011\\
57.125	0.14058	3.08959378583879	3.08959378583879\\
57.125	0.14424	3.59295375575546	3.59295375575546\\
57.125	0.1479	4.12185771303009	4.12185771303009\\
57.125	0.15156	4.67630565766271	4.67630565766271\\
57.125	0.15522	5.25629758965331	5.25629758965331\\
57.125	0.15888	5.8618335090019	5.8618335090019\\
57.125	0.16254	6.49291341570846	6.49291341570846\\
57.125	0.1662	7.149537309773	7.149537309773\\
57.125	0.16986	7.83170519119551	7.83170519119551\\
57.125	0.17352	8.539417059976	8.539417059976\\
57.125	0.17718	9.27267291611449	9.27267291611449\\
57.125	0.18084	10.0314727596109	10.0314727596109\\
57.125	0.1845	10.8158165904654	10.8158165904654\\
57.125	0.18816	11.6257044086778	11.6257044086778\\
57.125	0.19182	12.4611362142482	12.4611362142482\\
57.125	0.19548	13.3221120071766	13.3221120071766\\
57.125	0.19914	14.2086317874629	14.2086317874629\\
57.125	0.2028	15.1206955551073	15.1206955551073\\
57.125	0.20646	16.0583033101096	16.0583033101096\\
57.125	0.21012	17.0214550524699	17.0214550524699\\
57.125	0.21378	18.0101507821881	18.0101507821881\\
57.125	0.21744	19.0243904992644	19.0243904992644\\
57.125	0.2211	20.0641742036986	20.0641742036986\\
57.125	0.22476	21.1295018954908	21.1295018954908\\
57.125	0.22842	22.220373574641	22.220373574641\\
57.125	0.23208	23.3367892411492	23.3367892411492\\
57.125	0.23574	24.4787488950153	24.4787488950153\\
57.125	0.2394	25.6462525362395	25.6462525362395\\
57.125	0.24306	26.8393001648216	26.8393001648216\\
57.125	0.24672	28.0578917807617	28.0578917807617\\
57.125	0.25038	29.3020273840598	29.3020273840598\\
57.125	0.25404	30.5717069747158	30.5717069747158\\
57.125	0.2577	31.8669305527299	31.8669305527299\\
57.125	0.26136	33.1876981181018	33.1876981181018\\
57.125	0.26502	34.5340096708318	34.5340096708318\\
57.125	0.26868	35.9058652109198	35.9058652109198\\
57.125	0.27234	37.3032647383657	37.3032647383657\\
57.125	0.276	38.7262082531696	38.7262082531696\\
57.5	0.093	-1.24326161603483	-1.24326161603483\\
57.5	0.09666	-1.06474501600991	-1.06474501600991\\
57.5	0.10032	-0.860684428626991	-0.860684428626991\\
57.5	0.10398	-0.63107985388611	-0.63107985388611\\
57.5	0.10764	-0.375931291787264	-0.375931291787264\\
57.5	0.1113	-0.0952387423304284	-0.0952387423304284\\
57.5	0.11496	0.210997794484392	0.210997794484392\\
57.5	0.11862	0.54277831865719	0.54277831865719\\
57.5	0.12228	0.900102830187977	0.900102830187977\\
57.5	0.12594	1.28297132907674	1.28297132907674\\
57.5	0.1296	1.69138381532347	1.69138381532347\\
57.5	0.13326	2.12534028892818	2.12534028892818\\
57.5	0.13692	2.58484074989088	2.58484074989088\\
57.5	0.14058	3.06988519821155	3.06988519821155\\
57.5	0.14424	3.58047363389021	3.58047363389021\\
57.5	0.1479	4.11660605692684	4.11660605692684\\
57.5	0.15156	4.67828246732146	4.67828246732146\\
57.5	0.15522	5.26550286507405	5.26550286507405\\
57.5	0.15888	5.87826725018463	5.87826725018463\\
57.5	0.16254	6.51657562265319	6.51657562265319\\
57.5	0.1662	7.18042798247972	7.18042798247972\\
57.5	0.16986	7.86982432966423	7.86982432966423\\
57.5	0.17352	8.58476466420672	8.58476466420672\\
57.5	0.17718	9.32524898610719	9.32524898610719\\
57.5	0.18084	10.0912772953656	10.0912772953656\\
57.5	0.1845	10.8828495919821	10.8828495919821\\
57.5	0.18816	11.6999658759565	11.6999658759565\\
57.5	0.19182	12.5426261472889	12.5426261472889\\
57.5	0.19548	13.4108304059792	13.4108304059792\\
57.5	0.19914	14.3045786520276	14.3045786520276\\
57.5	0.2028	15.2238708854339	15.2238708854339\\
57.5	0.20646	16.1687071061982	16.1687071061982\\
57.5	0.21012	17.1390873143205	17.1390873143205\\
57.5	0.21378	18.1350115098008	18.1350115098008\\
57.5	0.21744	19.156479692639	19.156479692639\\
57.5	0.2211	20.2034918628353	20.2034918628353\\
57.5	0.22476	21.2760480203895	21.2760480203895\\
57.5	0.22842	22.3741481653017	22.3741481653017\\
57.5	0.23208	23.4977922975718	23.4977922975718\\
57.5	0.23574	24.6469804172	24.6469804172\\
57.5	0.2394	25.8217125241861	25.8217125241861\\
57.5	0.24306	27.0219886185302	27.0219886185302\\
57.5	0.24672	28.2478087002323	28.2478087002323\\
57.5	0.25038	29.4991727692923	29.4991727692923\\
57.5	0.25404	30.7760808257104	30.7760808257104\\
57.5	0.2577	32.0785328694864	32.0785328694864\\
57.5	0.26136	33.4065289006204	33.4065289006204\\
57.5	0.26502	34.7600689191124	34.7600689191124\\
57.5	0.26868	36.1391529249624	36.1391529249624\\
57.5	0.27234	37.5437809181703	37.5437809181703\\
57.5	0.276	38.9739528987362	38.9739528987362\\
57.875	0.093	-1.35457158559081	-1.35457158559081\\
57.875	0.09666	-1.16882651980389	-1.16882651980389\\
57.875	0.10032	-0.957537466658986	-0.957537466658986\\
57.875	0.10398	-0.720704426156107	-0.720704426156107\\
57.875	0.10764	-0.458327398295264	-0.458327398295264\\
57.875	0.1113	-0.170406383076438	-0.170406383076438\\
57.875	0.11496	0.14305861950038	0.14305861950038\\
57.875	0.11862	0.482067609435175	0.482067609435175\\
57.875	0.12228	0.846620586727953	0.846620586727953\\
57.875	0.12594	1.23671755137871	1.23671755137871\\
57.875	0.1296	1.65235850338744	1.65235850338744\\
57.875	0.13326	2.09354344275414	2.09354344275414\\
57.875	0.13692	2.56027236947883	2.56027236947883\\
57.875	0.14058	3.05254528356151	3.05254528356151\\
57.875	0.14424	3.57036218500216	3.57036218500216\\
57.875	0.1479	4.11372307380078	4.11372307380078\\
57.875	0.15156	4.6826279499574	4.6826279499574\\
57.875	0.15522	5.27707681347199	5.27707681347199\\
57.875	0.15888	5.89706966434456	5.89706966434456\\
57.875	0.16254	6.54260650257512	6.54260650257512\\
57.875	0.1662	7.21368732816364	7.21368732816364\\
57.875	0.16986	7.91031214111014	7.91031214111014\\
57.875	0.17352	8.63248094141463	8.63248094141463\\
57.875	0.17718	9.3801937290771	9.3801937290771\\
57.875	0.18084	10.1534505040975	10.1534505040975\\
57.875	0.1845	10.952251266476	10.952251266476\\
57.875	0.18816	11.7765960162124	11.7765960162124\\
57.875	0.19182	12.6264847533068	12.6264847533068\\
57.875	0.19548	13.5019174777591	13.5019174777591\\
57.875	0.19914	14.4028941895695	14.4028941895695\\
57.875	0.2028	15.3294148887378	15.3294148887378\\
57.875	0.20646	16.2814795752641	16.2814795752641\\
57.875	0.21012	17.2590882491484	17.2590882491484\\
57.875	0.21378	18.2622409103906	18.2622409103906\\
57.875	0.21744	19.2909375589909	19.2909375589909\\
57.875	0.2211	20.3451781949491	20.3451781949491\\
57.875	0.22476	21.4249628182653	21.4249628182653\\
57.875	0.22842	22.5302914289395	22.5302914289395\\
57.875	0.23208	23.6611640269717	23.6611640269717\\
57.875	0.23574	24.8175806123618	24.8175806123618\\
57.875	0.2394	25.9995411851099	25.9995411851099\\
57.875	0.24306	27.207045745216	27.207045745216\\
57.875	0.24672	28.4400942926801	28.4400942926801\\
57.875	0.25038	29.6986868275022	29.6986868275022\\
57.875	0.25404	30.9828233496822	30.9828233496822\\
57.875	0.2577	32.2925038592202	32.2925038592202\\
57.875	0.26136	33.6277283561162	33.6277283561162\\
57.875	0.26502	34.9884968403702	34.9884968403702\\
57.875	0.26868	36.3748093119821	36.3748093119821\\
57.875	0.27234	37.7866657709521	37.7866657709521\\
57.875	0.276	39.22406621728	39.22406621728\\
58.25	0.093	-1.46351288216961	-1.46351288216961\\
58.25	0.09666	-1.27053935062069	-1.27053935062069\\
58.25	0.10032	-1.05202183171379	-1.05202183171379\\
58.25	0.10398	-0.807960325448924	-0.807960325448924\\
58.25	0.10764	-0.538354831826084	-0.538354831826084\\
58.25	0.1113	-0.24320535084526	-0.24320535084526\\
58.25	0.11496	0.0774881174935551	0.0774881174935551\\
58.25	0.11862	0.423725573190341	0.423725573190341\\
58.25	0.12228	0.795507016245116	0.795507016245116\\
58.25	0.12594	1.19283244665786	1.19283244665786\\
58.25	0.1296	1.61570186442859	1.61570186442859\\
58.25	0.13326	2.06411526955728	2.06411526955728\\
58.25	0.13692	2.53807266204397	2.53807266204397\\
58.25	0.14058	3.03757404188865	3.03757404188865\\
58.25	0.14424	3.56261940909129	3.56261940909129\\
58.25	0.1479	4.11320876365191	4.11320876365191\\
58.25	0.15156	4.68934210557052	4.68934210557052\\
58.25	0.15522	5.2910194348471	5.2910194348471\\
58.25	0.15888	5.91824075148167	5.91824075148167\\
58.25	0.16254	6.57100605547421	6.57100605547421\\
58.25	0.1662	7.24931534682474	7.24931534682474\\
58.25	0.16986	7.95316862553324	7.95316862553324\\
58.25	0.17352	8.68256589159972	8.68256589159972\\
58.25	0.17718	9.43750714502418	9.43750714502418\\
58.25	0.18084	10.2179923858066	10.2179923858066\\
58.25	0.1845	11.024021613947	11.024021613947\\
58.25	0.18816	11.8555948294454	11.8555948294454\\
58.25	0.19182	12.7127120323018	12.7127120323018\\
58.25	0.19548	13.5953732225162	13.5953732225162\\
58.25	0.19914	14.5035784000885	14.5035784000885\\
58.25	0.2028	15.4373275650189	15.4373275650189\\
58.25	0.20646	16.3966207173071	16.3966207173071\\
58.25	0.21012	17.3814578569534	17.3814578569534\\
58.25	0.21378	18.3918389839577	18.3918389839577\\
58.25	0.21744	19.4277640983199	19.4277640983199\\
58.25	0.2211	20.4892332000401	20.4892332000401\\
58.25	0.22476	21.5762462891183	21.5762462891183\\
58.25	0.22842	22.6888033655545	22.6888033655545\\
58.25	0.23208	23.8269044293487	23.8269044293487\\
58.25	0.23574	24.9905494805008	24.9905494805008\\
58.25	0.2394	26.1797385190109	26.1797385190109\\
58.25	0.24306	27.394471544879	27.394471544879\\
58.25	0.24672	28.6347485581051	28.6347485581051\\
58.25	0.25038	29.9005695586891	29.9005695586891\\
58.25	0.25404	31.1919345466312	31.1919345466312\\
58.25	0.2577	32.5088435219312	32.5088435219312\\
58.25	0.26136	33.8512964845892	33.8512964845892\\
58.25	0.26502	35.2192934346051	35.2192934346051\\
58.25	0.26868	36.6128343719791	36.6128343719791\\
58.25	0.27234	38.031919296711	38.031919296711\\
58.25	0.276	39.4765482088009	39.4765482088009\\
58.625	0.093	-1.57008550577121	-1.57008550577121\\
58.625	0.09666	-1.3698835084603	-1.3698835084603\\
58.625	0.10032	-1.14413752379144	-1.14413752379144\\
58.625	0.10398	-0.892847551764572	-0.892847551764572\\
58.625	0.10764	-0.616013592379719	-0.616013592379719\\
58.625	0.1113	-0.313635645636884	-0.313635645636884\\
58.625	0.11496	0.0142862884639214	0.0142862884639214\\
58.625	0.11862	0.36775220992269	0.36775220992269\\
58.625	0.12228	0.746762118739456	0.746762118739456\\
58.625	0.12594	1.15131601491419	1.15131601491419\\
58.625	0.1296	1.58141389844691	1.58141389844691\\
58.625	0.13326	2.03705576933762	2.03705576933762\\
58.625	0.13692	2.5182416275863	2.5182416275863\\
58.625	0.14058	3.02497147319297	3.02497147319297\\
58.625	0.14424	3.55724530615759	3.55724530615759\\
58.625	0.1479	4.11506312648022	4.11506312648022\\
58.625	0.15156	4.69842493416082	4.69842493416082\\
58.625	0.15522	5.3073307291994	5.3073307291994\\
58.625	0.15888	5.94178051159597	5.94178051159597\\
58.625	0.16254	6.6017742813505	6.6017742813505\\
58.625	0.1662	7.28731203846303	7.28731203846303\\
58.625	0.16986	7.99839378293352	7.99839378293352\\
58.625	0.17352	8.73501951476199	8.73501951476199\\
58.625	0.17718	9.49718923394845	9.49718923394845\\
58.625	0.18084	10.2849029404929	10.2849029404929\\
58.625	0.1845	11.0981606343953	11.0981606343953\\
58.625	0.18816	11.9369623156557	11.9369623156557\\
58.625	0.19182	12.8013079842741	12.8013079842741\\
58.625	0.19548	13.6911976402504	13.6911976402504\\
58.625	0.19914	14.6066312835848	14.6066312835848\\
58.625	0.2028	15.5476089142771	15.5476089142771\\
58.625	0.20646	16.5141305323274	16.5141305323274\\
58.625	0.21012	17.5061961377356	17.5061961377356\\
58.625	0.21378	18.5238057305019	18.5238057305019\\
58.625	0.21744	19.5669593106261	19.5669593106261\\
58.625	0.2211	20.6356568781083	20.6356568781083\\
58.625	0.22476	21.7298984329485	21.7298984329485\\
58.625	0.22842	22.8496839751467	22.8496839751467\\
58.625	0.23208	23.9950135047029	23.9950135047029\\
58.625	0.23574	25.165887021617	25.165887021617\\
58.625	0.2394	26.3623045258891	26.3623045258891\\
58.625	0.24306	27.5842660175192	27.5842660175192\\
58.625	0.24672	28.8317714965072	28.8317714965072\\
58.625	0.25038	30.1048209628533	30.1048209628533\\
58.625	0.25404	31.4034144165573	31.4034144165573\\
58.625	0.2577	32.7275518576194	32.7275518576194\\
58.625	0.26136	34.0772332860393	34.0772332860393\\
58.625	0.26502	35.4524587018173	35.4524587018173\\
58.625	0.26868	36.8532281049533	36.8532281049533\\
58.625	0.27234	38.2795414954472	38.2795414954472\\
58.625	0.276	39.7313988732991	39.7313988732991\\
59	0.093	-1.67428945639563	-1.67428945639563\\
59	0.09666	-1.46685899332273	-1.46685899332273\\
59	0.10032	-1.23388454289187	-1.23388454289187\\
59	0.10398	-0.975366105103003	-0.975366105103003\\
59	0.10764	-0.69130367995616	-0.69130367995616\\
59	0.1113	-0.381697267451335	-0.381697267451335\\
59	0.11496	-0.0465468675885319	-0.0465468675885319\\
59	0.11862	0.314147519632234	0.314147519632234\\
59	0.12228	0.700385894210998	0.700385894210998\\
59	0.12594	1.11216825614772	1.11216825614772\\
59	0.1296	1.54949460544244	1.54949460544244\\
59	0.13326	2.01236494209514	2.01236494209514\\
59	0.13692	2.50077926610581	2.50077926610581\\
59	0.14058	3.01473757747447	3.01473757747447\\
59	0.14424	3.5542398762011	3.5542398762011\\
59	0.1479	4.11928616228572	4.11928616228572\\
59	0.15156	4.70987643572832	4.70987643572832\\
59	0.15522	5.32601069652889	5.32601069652889\\
59	0.15888	5.96768894468745	5.96768894468745\\
59	0.16254	6.63491118020399	6.63491118020399\\
59	0.1662	7.3276774030785	7.3276774030785\\
59	0.16986	8.04598761331099	8.04598761331099\\
59	0.17352	8.78984181090146	8.78984181090146\\
59	0.17718	9.55923999584992	9.55923999584992\\
59	0.18084	10.3541821681563	10.3541821681563\\
59	0.1845	11.1746683278207	11.1746683278207\\
59	0.18816	12.0206984748431	12.0206984748431\\
59	0.19182	12.8922726092235	12.8922726092235\\
59	0.19548	13.7893907309619	13.7893907309619\\
59	0.19914	14.7120528400582	14.7120528400582\\
59	0.2028	15.6602589365125	15.6602589365125\\
59	0.20646	16.6340090203248	16.6340090203248\\
59	0.21012	17.6333030914951	17.6333030914951\\
59	0.21378	18.6581411500233	18.6581411500233\\
59	0.21744	19.7085231959095	19.7085231959095\\
59	0.2211	20.7844492291537	20.7844492291537\\
59	0.22476	21.8859192497559	21.8859192497559\\
59	0.22842	23.0129332577161	23.0129332577161\\
59	0.23208	24.1654912530342	24.1654912530342\\
59	0.23574	25.3435932357104	25.3435932357104\\
59	0.2394	26.5472392057445	26.5472392057445\\
59	0.24306	27.7764291631365	27.7764291631365\\
59	0.24672	29.0311631078866	29.0311631078866\\
59	0.25038	30.3114410399947	30.3114410399947\\
59	0.25404	31.6172629594607	31.6172629594607\\
59	0.2577	32.9486288662847	32.9486288662847\\
59	0.26136	34.3055387604667	34.3055387604667\\
59	0.26502	35.6879926420067	35.6879926420067\\
59	0.26868	37.0959905109046	37.0959905109046\\
59	0.27234	38.5295323671605	38.5295323671605\\
59	0.276	39.9886182107744	39.9886182107744\\
59.375	0.093	-1.77612473404286	-1.77612473404286\\
59.375	0.09666	-1.56146580520796	-1.56146580520796\\
59.375	0.10032	-1.3212628890151	-1.3212628890151\\
59.375	0.10398	-1.05551598546425	-1.05551598546425\\
59.375	0.10764	-0.764225094555407	-0.764225094555407\\
59.375	0.1113	-0.447390216288584	-0.447390216288584\\
59.375	0.11496	-0.105011350663791	-0.105011350663791\\
59.375	0.11862	0.262911502318973	0.262911502318973\\
59.375	0.12228	0.656378342659726	0.656378342659726\\
59.375	0.12594	1.07538917035845	1.07538917035845\\
59.375	0.1296	1.51994398541517	1.51994398541517\\
59.375	0.13326	1.99004278782986	1.99004278782986\\
59.375	0.13692	2.48568557760253	2.48568557760253\\
59.375	0.14058	3.00687235473318	3.00687235473318\\
59.375	0.14424	3.5536031192218	3.5536031192218\\
59.375	0.1479	4.12587787106841	4.12587787106841\\
59.375	0.15156	4.72369661027301	4.72369661027301\\
59.375	0.15522	5.34705933683558	5.34705933683558\\
59.375	0.15888	5.99596605075613	5.99596605075613\\
59.375	0.16254	6.67041675203467	6.67041675203467\\
59.375	0.1662	7.37041144067117	7.37041144067117\\
59.375	0.16986	8.09595011666566	8.09595011666566\\
59.375	0.17352	8.84703278001813	8.84703278001813\\
59.375	0.17718	9.62365943072857	9.62365943072857\\
59.375	0.18084	10.425830068797	10.425830068797\\
59.375	0.1845	11.2535446942234	11.2535446942234\\
59.375	0.18816	12.1068033070078	12.1068033070078\\
59.375	0.19182	12.9856059071501	12.9856059071501\\
59.375	0.19548	13.8899524946505	13.8899524946505\\
59.375	0.19914	14.8198430695088	14.8198430695088\\
59.375	0.2028	15.7752776317251	15.7752776317251\\
59.375	0.20646	16.7562561812994	16.7562561812994\\
59.375	0.21012	17.7627787182317	17.7627787182317\\
59.375	0.21378	18.7948452425219	18.7948452425219\\
59.375	0.21744	19.8524557541701	19.8524557541701\\
59.375	0.2211	20.9356102531763	20.9356102531763\\
59.375	0.22476	22.0443087395405	22.0443087395405\\
59.375	0.22842	23.1785512132627	23.1785512132627\\
59.375	0.23208	24.3383376743428	24.3383376743428\\
59.375	0.23574	25.523668122781	25.523668122781\\
59.375	0.2394	26.7345425585771	26.7345425585771\\
59.375	0.24306	27.9709609817311	27.9709609817311\\
59.375	0.24672	29.2329233922432	29.2329233922432\\
59.375	0.25038	30.5204297901132	30.5204297901132\\
59.375	0.25404	31.8334801753412	31.8334801753412\\
59.375	0.2577	33.1720745479273	33.1720745479273\\
59.375	0.26136	34.5362129078712	34.5362129078712\\
59.375	0.26502	35.9258952551732	35.9258952551732\\
59.375	0.26868	37.3411215898331	37.3411215898331\\
59.375	0.27234	38.781891911851	38.781891911851\\
59.375	0.276	40.2482062212269	40.2482062212269\\
59.75	0.093	-1.8755913387129	-1.8755913387129\\
59.75	0.09666	-1.65370394411601	-1.65370394411601\\
59.75	0.10032	-1.40627256216116	-1.40627256216116\\
59.75	0.10398	-1.13329719284831	-1.13329719284831\\
59.75	0.10764	-0.834777836177473	-0.834777836177473\\
59.75	0.1113	-0.51071449214866	-0.51071449214866\\
59.75	0.11496	-0.161107160761862	-0.161107160761862\\
59.75	0.11862	0.214044157982892	0.214044157982892\\
59.75	0.12228	0.614739464085643	0.614739464085643\\
59.75	0.12594	1.04097875754636	1.04097875754636\\
59.75	0.1296	1.49276203836506	1.49276203836506\\
59.75	0.13326	1.97008930654176	1.97008930654176\\
59.75	0.13692	2.47296056207642	2.47296056207642\\
59.75	0.14058	3.00137580496907	3.00137580496907\\
59.75	0.14424	3.55533503521968	3.55533503521968\\
59.75	0.1479	4.1348382528283	4.1348382528283\\
59.75	0.15156	4.73988545779488	4.73988545779488\\
59.75	0.15522	5.37047665011945	5.37047665011945\\
59.75	0.15888	6.026611829802	6.026611829802\\
59.75	0.16254	6.70829099684252	6.70829099684252\\
59.75	0.1662	7.41551415124103	7.41551415124103\\
59.75	0.16986	8.1482812929975	8.1482812929975\\
59.75	0.17352	8.90659242211196	8.90659242211196\\
59.75	0.17718	9.69044753858441	9.69044753858441\\
59.75	0.18084	10.4998466424148	10.4998466424148\\
59.75	0.1845	11.3347897336032	11.3347897336032\\
59.75	0.18816	12.1952768121496	12.1952768121496\\
59.75	0.19182	13.081307878054	13.081307878054\\
59.75	0.19548	13.9928829313163	13.9928829313163\\
59.75	0.19914	14.9300019719366	14.9300019719366\\
59.75	0.2028	15.8926649999149	15.8926649999149\\
59.75	0.20646	16.8808720152512	16.8808720152512\\
59.75	0.21012	17.8946230179455	17.8946230179455\\
59.75	0.21378	18.9339180079977	18.9339180079977\\
59.75	0.21744	19.9987569854079	19.9987569854079\\
59.75	0.2211	21.0891399501761	21.0891399501761\\
59.75	0.22476	22.2050669023023	22.2050669023023\\
59.75	0.22842	23.3465378417865	23.3465378417865\\
59.75	0.23208	24.5135527686286	24.5135527686286\\
59.75	0.23574	25.7061116828287	25.7061116828287\\
59.75	0.2394	26.9242145843868	26.9242145843868\\
59.75	0.24306	28.1678614733029	28.1678614733029\\
59.75	0.24672	29.4370523495769	29.4370523495769\\
59.75	0.25038	30.731787213209	30.731787213209\\
59.75	0.25404	32.052066064199	32.052066064199\\
59.75	0.2577	33.397888902547	33.397888902547\\
59.75	0.26136	34.7692557282529	34.7692557282529\\
59.75	0.26502	36.1661665413169	36.1661665413169\\
59.75	0.26868	37.5886213417388	37.5886213417388\\
59.75	0.27234	39.0366201295187	39.0366201295187\\
59.75	0.276	40.5101629046566	40.5101629046566\\
60.125	0.093	-1.97268927040576	-1.97268927040576\\
60.125	0.09666	-1.74357341004688	-1.74357341004688\\
60.125	0.10032	-1.48891356233003	-1.48891356233003\\
60.125	0.10398	-1.20870972725518	-1.20870972725518\\
60.125	0.10764	-0.902961904822352	-0.902961904822352\\
60.125	0.1113	-0.571670095031541	-0.571670095031541\\
60.125	0.11496	-0.214834297882753	-0.214834297882753\\
60.125	0.11862	0.167545486623998	0.167545486623998\\
60.125	0.12228	0.575469258488747	0.575469258488747\\
60.125	0.12594	1.00893701771145	1.00893701771145\\
60.125	0.1296	1.46794876429216	1.46794876429216\\
60.125	0.13326	1.95250449823084	1.95250449823084\\
60.125	0.13692	2.4626042195275	2.4626042195275\\
60.125	0.14058	2.99824792818215	2.99824792818215\\
60.125	0.14424	3.55943562419476	3.55943562419476\\
60.125	0.1479	4.14616730756536	4.14616730756536\\
60.125	0.15156	4.75844297829394	4.75844297829394\\
60.125	0.15522	5.3962626363805	5.3962626363805\\
60.125	0.15888	6.05962628182505	6.05962628182505\\
60.125	0.16254	6.74853391462757	6.74853391462757\\
60.125	0.1662	7.46298553478807	7.46298553478807\\
60.125	0.16986	8.20298114230654	8.20298114230654\\
60.125	0.17352	8.968520737183	8.968520737183\\
60.125	0.17718	9.75960431941743	9.75960431941743\\
60.125	0.18084	10.5762318890098	10.5762318890098\\
60.125	0.1845	11.4184034459602	11.4184034459602\\
60.125	0.18816	12.2861189902686	12.2861189902686\\
60.125	0.19182	13.179378521935	13.179378521935\\
60.125	0.19548	14.0981820409593	14.0981820409593\\
60.125	0.19914	15.0425295473416	15.0425295473416\\
60.125	0.2028	16.0124210410819	16.0124210410819\\
60.125	0.20646	17.0078565221802	17.0078565221802\\
60.125	0.21012	18.0288359906365	18.0288359906365\\
60.125	0.21378	19.0753594464507	19.0753594464507\\
60.125	0.21744	20.1474268896229	20.1474268896229\\
60.125	0.2211	21.2450383201531	21.2450383201531\\
60.125	0.22476	22.3681937380412	22.3681937380412\\
60.125	0.22842	23.5168931432874	23.5168931432874\\
60.125	0.23208	24.6911365358915	24.6911365358915\\
60.125	0.23574	25.8909239158537	25.8909239158537\\
60.125	0.2394	27.1162552831738	27.1162552831738\\
60.125	0.24306	28.3671306378518	28.3671306378518\\
60.125	0.24672	29.6435499798878	29.6435499798878\\
60.125	0.25038	30.9455133092819	30.9455133092819\\
60.125	0.25404	32.2730206260339	32.2730206260339\\
60.125	0.2577	33.6260719301439	33.6260719301439\\
60.125	0.26136	35.0046672216119	35.0046672216119\\
60.125	0.26502	36.4088065004378	36.4088065004378\\
60.125	0.26868	37.8384897666217	37.8384897666217\\
60.125	0.27234	39.2937170201636	39.2937170201636\\
60.125	0.276	40.7744882610635	40.7744882610635\\
60.5	0.093	-2.06741852912145	-2.06741852912145\\
60.5	0.09666	-1.83107420300057	-1.83107420300057\\
60.5	0.10032	-1.5691858895217	-1.5691858895217\\
60.5	0.10398	-1.28175358868486	-1.28175358868486\\
60.5	0.10764	-0.968777300490046	-0.968777300490046\\
60.5	0.1113	-0.630257024937253	-0.630257024937253\\
60.5	0.11496	-0.266192762026474	-0.266192762026474\\
60.5	0.11862	0.123415488242289	0.123415488242289\\
60.5	0.12228	0.538567725869028	0.538567725869028\\
60.5	0.12594	0.979263950853751	0.979263950853751\\
60.5	0.1296	1.44550416319645	1.44550416319645\\
60.5	0.13326	1.93728836289711	1.93728836289711\\
60.5	0.13692	2.45461654995577	2.45461654995577\\
60.5	0.14058	2.99748872437241	2.99748872437241\\
60.5	0.14424	3.56590488614702	3.56590488614702\\
60.5	0.1479	4.15986503527961	4.15986503527961\\
60.5	0.15156	4.77936917177018	4.77936917177018\\
60.5	0.15522	5.42441729561875	5.42441729561875\\
60.5	0.15888	6.09500940682528	6.09500940682528\\
60.5	0.16254	6.7911455053898	6.7911455053898\\
60.5	0.1662	7.51282559131229	7.51282559131229\\
60.5	0.16986	8.26004966459276	8.26004966459276\\
60.5	0.17352	9.0328177252312	9.0328177252312\\
60.5	0.17718	9.83112977322764	9.83112977322764\\
60.5	0.18084	10.6549858085821	10.6549858085821\\
60.5	0.1845	11.5043858312944	11.5043858312944\\
60.5	0.18816	12.3793298413648	12.3793298413648\\
60.5	0.19182	13.2798178387932	13.2798178387932\\
60.5	0.19548	14.2058498235795	14.2058498235795\\
60.5	0.19914	15.1574257957238	15.1574257957238\\
60.5	0.2028	16.1345457552261	16.1345457552261\\
60.5	0.20646	17.1372097020864	17.1372097020864\\
60.5	0.21012	18.1654176363046	18.1654176363046\\
60.5	0.21378	19.2191695578808	19.2191695578808\\
60.5	0.21744	20.298465466815	20.298465466815\\
60.5	0.2211	21.4033053631072	21.4033053631072\\
60.5	0.22476	22.5336892467574	22.5336892467574\\
60.5	0.22842	23.6896171177655	23.6896171177655\\
60.5	0.23208	24.8710889761317	24.8710889761317\\
60.5	0.23574	26.0781048218558	26.0781048218558\\
60.5	0.2394	27.3106646549379	27.3106646549379\\
60.5	0.24306	28.5687684753779	28.5687684753779\\
60.5	0.24672	29.852416283176	29.852416283176\\
60.5	0.25038	31.161608078332	31.161608078332\\
60.5	0.25404	32.496343860846	32.496343860846\\
60.5	0.2577	33.856623630718	33.856623630718\\
60.5	0.26136	35.2424473879479	35.2424473879479\\
60.5	0.26502	36.6538151325359	36.6538151325359\\
60.5	0.26868	38.0907268644818	38.0907268644818\\
60.5	0.27234	39.5531825837857	39.5531825837857\\
60.5	0.276	41.0411822904476	41.0411822904476\\
60.875	0.093	-2.15977911485994	-2.15977911485994\\
60.875	0.09666	-1.91620632297705	-1.91620632297705\\
60.875	0.10032	-1.64708954373619	-1.64708954373619\\
60.875	0.10398	-1.35242877713736	-1.35242877713736\\
60.875	0.10764	-1.03222402318055	-1.03222402318055\\
60.875	0.1113	-0.686475281865762	-0.686475281865762\\
60.875	0.11496	-0.315182553192987	-0.315182553192987\\
60.875	0.11862	0.0816541628377738	0.0816541628377738\\
60.875	0.12228	0.50403486622651	0.50403486622651\\
60.875	0.12594	0.951959556973224	0.951959556973224\\
60.875	0.1296	1.42542823507792	1.42542823507792\\
60.875	0.13326	1.92444090054057	1.92444090054057\\
60.875	0.13692	2.44899755336122	2.44899755336122\\
60.875	0.14058	2.99909819353987	2.99909819353987\\
60.875	0.14424	3.57474282107648	3.57474282107648\\
60.875	0.1479	4.17593143597106	4.17593143597106\\
60.875	0.15156	4.80266403822363	4.80266403822363\\
60.875	0.15522	5.45494062783418	5.45494062783418\\
60.875	0.15888	6.13276120480271	6.13276120480271\\
60.875	0.16254	6.83612576912923	6.83612576912923\\
60.875	0.1662	7.56503432081371	7.56503432081371\\
60.875	0.16986	8.31948685985617	8.31948685985617\\
60.875	0.17352	9.09948338625662	9.09948338625662\\
60.875	0.17718	9.90502390001505	9.90502390001505\\
60.875	0.18084	10.7361084011314	10.7361084011314\\
60.875	0.1845	11.5927368896058	11.5927368896058\\
60.875	0.18816	12.4749093654382	12.4749093654382\\
60.875	0.19182	13.3826258286285	13.3826258286285\\
60.875	0.19548	14.3158862791769	14.3158862791769\\
60.875	0.19914	15.2746907170832	15.2746907170832\\
60.875	0.2028	16.2590391423475	16.2590391423475\\
60.875	0.20646	17.2689315549697	17.2689315549697\\
60.875	0.21012	18.30436795495	18.30436795495\\
60.875	0.21378	19.3653483422882	19.3653483422882\\
60.875	0.21744	20.4518727169844	20.4518727169844\\
60.875	0.2211	21.5639410790386	21.5639410790386\\
60.875	0.22476	22.7015534284507	22.7015534284507\\
60.875	0.22842	23.8647097652209	23.8647097652209\\
60.875	0.23208	25.053410089349	25.053410089349\\
60.875	0.23574	26.2676544008351	26.2676544008351\\
60.875	0.2394	27.5074426996792	27.5074426996792\\
60.875	0.24306	28.7727749858812	28.7727749858812\\
60.875	0.24672	30.0636512594413	30.0636512594413\\
60.875	0.25038	31.3800715203593	31.3800715203593\\
60.875	0.25404	32.7220357686353	32.7220357686353\\
60.875	0.2577	34.0895440042693	34.0895440042693\\
60.875	0.26136	35.4825962272612	35.4825962272612\\
60.875	0.26502	36.9011924376112	36.9011924376112\\
60.875	0.26868	38.3453326353191	38.3453326353191\\
60.875	0.27234	39.815016820385	39.815016820385\\
60.875	0.276	41.3102449928088	41.3102449928088\\
61.25	0.093	-2.24977102762122	-2.24977102762122\\
61.25	0.09666	-1.99896976997634	-1.99896976997634\\
61.25	0.10032	-1.72262452497348	-1.72262452497348\\
61.25	0.10398	-1.42073529261265	-1.42073529261265\\
61.25	0.10764	-1.09330207289386	-1.09330207289386\\
61.25	0.1113	-0.740324865817071	-0.740324865817071\\
61.25	0.11496	-0.361803671382297	-0.361803671382297\\
61.25	0.11862	0.0422615104104462	0.0422615104104462\\
61.25	0.12228	0.47187067956118	0.47187067956118\\
61.25	0.12594	0.927023836069891	0.927023836069891\\
61.25	0.1296	1.40772097993658	1.40772097993658\\
61.25	0.13326	1.91396211116124	1.91396211116124\\
61.25	0.13692	2.44574722974388	2.44574722974388\\
61.25	0.14058	3.00307633568451	3.00307633568451\\
61.25	0.14424	3.58594942898312	3.58594942898312\\
61.25	0.1479	4.19436650963969	4.19436650963969\\
61.25	0.15156	4.82832757765426	4.82832757765426\\
61.25	0.15522	5.48783263302681	5.48783263302681\\
61.25	0.15888	6.17288167575734	6.17288167575734\\
61.25	0.16254	6.88347470584584	6.88347470584584\\
61.25	0.1662	7.61961172329232	7.61961172329232\\
61.25	0.16986	8.38129272809678	8.38129272809678\\
61.25	0.17352	9.16851772025922	9.16851772025922\\
61.25	0.17718	9.98128669977965	9.98128669977965\\
61.25	0.18084	10.819599666658	10.819599666658\\
61.25	0.1845	11.6834566208944	11.6834566208944\\
61.25	0.18816	12.5728575624888	12.5728575624888\\
61.25	0.19182	13.4878024914411	13.4878024914411\\
61.25	0.19548	14.4282914077514	14.4282914077514\\
61.25	0.19914	15.3943243114197	15.3943243114197\\
61.25	0.2028	16.385901202446	16.385901202446\\
61.25	0.20646	17.4030220808303	17.4030220808303\\
61.25	0.21012	18.4456869465725	18.4456869465725\\
61.25	0.21378	19.5138957996727	19.5138957996727\\
61.25	0.21744	20.6076486401309	20.6076486401309\\
61.25	0.2211	21.7269454679471	21.7269454679471\\
61.25	0.22476	22.8717862831213	22.8717862831213\\
61.25	0.22842	24.0421710856534	24.0421710856534\\
61.25	0.23208	25.2380998755435	25.2380998755435\\
61.25	0.23574	26.4595726527916	26.4595726527916\\
61.25	0.2394	27.7065894173977	27.7065894173977\\
61.25	0.24306	28.9791501693617	28.9791501693617\\
61.25	0.24672	30.2772549086838	30.2772549086838\\
61.25	0.25038	31.6009036353638	31.6009036353638\\
61.25	0.25404	32.9500963494018	32.9500963494018\\
61.25	0.2577	34.3248330507978	34.3248330507978\\
61.25	0.26136	35.7251137395517	35.7251137395517\\
61.25	0.26502	37.1509384156636	37.1509384156636\\
61.25	0.26868	38.6023070791336	38.6023070791336\\
61.25	0.27234	40.0792197299614	40.0792197299614\\
61.25	0.276	41.5816763681473	41.5816763681473\\
61.625	0.093	-2.33739426740532	-2.33739426740532\\
61.625	0.09666	-2.07936454399846	-2.07936454399846\\
61.625	0.10032	-1.7957908332336	-1.7957908332336\\
61.625	0.10398	-1.48667313511077	-1.48667313511077\\
61.625	0.10764	-1.15201144962998	-1.15201144962998\\
61.625	0.1113	-0.791805776791199	-0.791805776791199\\
61.625	0.11496	-0.406056116594435	-0.406056116594435\\
61.625	0.11862	0.0052375309603061	0.0052375309603061\\
61.625	0.12228	0.442075165873037	0.442075165873037\\
61.625	0.12594	0.904456788143738	0.904456788143738\\
61.625	0.1296	1.39238239777243	1.39238239777243\\
61.625	0.13326	1.90585199475907	1.90585199475907\\
61.625	0.13692	2.44486557910372	2.44486557910372\\
61.625	0.14058	3.00942315080634	3.00942315080634\\
61.625	0.14424	3.59952470986694	3.59952470986694\\
61.625	0.1479	4.21517025628551	4.21517025628551\\
61.625	0.15156	4.85635979006208	4.85635979006208\\
61.625	0.15522	5.52309331119661	5.52309331119661\\
61.625	0.15888	6.21537081968914	6.21537081968914\\
61.625	0.16254	6.93319231553964	6.93319231553964\\
61.625	0.1662	7.67655779874812	7.67655779874812\\
61.625	0.16986	8.44546726931457	8.44546726931457\\
61.625	0.17352	9.239920727239	9.239920727239\\
61.625	0.17718	10.0599181725214	10.0599181725214\\
61.625	0.18084	10.9054596051618	10.9054596051618\\
61.625	0.1845	11.7765450251602	11.7765450251602\\
61.625	0.18816	12.6731744325165	12.6731744325165\\
61.625	0.19182	13.5953478272309	13.5953478272309\\
61.625	0.19548	14.5430652093032	14.5430652093032\\
61.625	0.19914	15.5163265787335	15.5163265787335\\
61.625	0.2028	16.5151319355218	16.5151319355218\\
61.625	0.20646	17.539481279668	17.539481279668\\
61.625	0.21012	18.5893746111723	18.5893746111723\\
61.625	0.21378	19.6648119300345	19.6648119300345\\
61.625	0.21744	20.7657932362547	20.7657932362547\\
61.625	0.2211	21.8923185298328	21.8923185298328\\
61.625	0.22476	23.044387810769	23.044387810769\\
61.625	0.22842	24.2220010790631	24.2220010790631\\
61.625	0.23208	25.4251583347152	25.4251583347152\\
61.625	0.23574	26.6538595777253	26.6538595777253\\
61.625	0.2394	27.9081048080934	27.9081048080934\\
61.625	0.24306	29.1878940258194	29.1878940258194\\
61.625	0.24672	30.4932272309035	30.4932272309035\\
61.625	0.25038	31.8241044233455	31.8241044233455\\
61.625	0.25404	33.1805256031455	33.1805256031455\\
61.625	0.2577	34.5624907703034	34.5624907703034\\
61.625	0.26136	35.9699999248194	35.9699999248194\\
61.625	0.26502	37.4030530666933	37.4030530666933\\
61.625	0.26868	38.8616501959252	38.8616501959252\\
61.625	0.27234	40.3457913125151	40.3457913125151\\
61.625	0.276	41.8554764164629	41.8554764164629\\
62	0.093	-2.42264883421223	-2.42264883421223\\
62	0.09666	-2.15739064504337	-2.15739064504337\\
62	0.10032	-1.86658846851652	-1.86658846851652\\
62	0.10398	-1.5502423046317	-1.5502423046317\\
62	0.10764	-1.20835215338891	-1.20835215338891\\
62	0.1113	-0.840918014788135	-0.840918014788135\\
62	0.11496	-0.447939888829374	-0.447939888829374\\
62	0.11862	-0.0294177755126359	-0.0294177755126359\\
62	0.12228	0.414648325162085	0.414648325162085\\
62	0.12594	0.884258413194784	0.884258413194784\\
62	0.1296	1.37941248858547	1.37941248858547\\
62	0.13326	1.90011055133411	1.90011055133411\\
62	0.13692	2.44635260144075	2.44635260144075\\
62	0.14058	3.01813863890537	3.01813863890537\\
62	0.14424	3.61546866372796	3.61546866372796\\
62	0.1479	4.23834267590853	4.23834267590853\\
62	0.15156	4.88676067544708	4.88676067544708\\
62	0.15522	5.56072266234363	5.56072266234363\\
62	0.15888	6.26022863659814	6.26022863659814\\
62	0.16254	6.98527859821064	6.98527859821064\\
62	0.1662	7.73587254718111	7.73587254718111\\
62	0.16986	8.51201048350955	8.51201048350955\\
62	0.17352	9.31369240719598	9.31369240719598\\
62	0.17718	10.1409183182404	10.1409183182404\\
62	0.18084	10.9936882166428	10.9936882166428\\
62	0.1845	11.8720021024032	11.8720021024032\\
62	0.18816	12.7758599755215	12.7758599755215\\
62	0.19182	13.7052618359978	13.7052618359978\\
62	0.19548	14.6602076838321	14.6602076838321\\
62	0.19914	15.6406975190244	15.6406975190244\\
62	0.2028	16.6467313415747	16.6467313415747\\
62	0.20646	17.678309151483	17.678309151483\\
62	0.21012	18.7354309487492	18.7354309487492\\
62	0.21378	19.8180967333734	19.8180967333734\\
62	0.21744	20.9263065053556	20.9263065053556\\
62	0.2211	22.0600602646957	22.0600602646957\\
62	0.22476	23.2193580113939	23.2193580113939\\
62	0.22842	24.40419974545	24.40419974545\\
62	0.23208	25.6145854668641	25.6145854668641\\
62	0.23574	26.8505151756362	26.8505151756362\\
62	0.2394	28.1119888717663	28.1119888717663\\
62	0.24306	29.3990065552543	29.3990065552543\\
62	0.24672	30.7115682261003	30.7115682261003\\
62	0.25038	32.0496738843044	32.0496738843044\\
62	0.25404	33.4133235298663	33.4133235298663\\
62	0.2577	34.8025171627863	34.8025171627863\\
62	0.26136	36.2172547830642	36.2172547830642\\
62	0.26502	37.6575363907002	37.6575363907002\\
62	0.26868	39.1233619856941	39.1233619856941\\
62	0.27234	40.6147315680459	40.6147315680459\\
62	0.276	42.1316451377558	42.1316451377558\\
62.375	0.093	-2.50553472804196	-2.50553472804196\\
62.375	0.09666	-2.2330480731111	-2.2330480731111\\
62.375	0.10032	-1.93501743082229	-1.93501743082229\\
62.375	0.10398	-1.61144280117547	-1.61144280117547\\
62.375	0.10764	-1.26232418417068	-1.26232418417068\\
62.375	0.1113	-0.887661579807888	-0.887661579807888\\
62.375	0.11496	-0.48745498808713	-0.48745498808713\\
62.375	0.11862	-0.0617044090084153	-0.0617044090084153\\
62.375	0.12228	0.389590157428303	0.389590157428303\\
62.375	0.12594	0.866428711222985	0.866428711222985\\
62.375	0.1296	1.36881125237565	1.36881125237565\\
62.375	0.13326	1.89673778088631	1.89673778088631\\
62.375	0.13692	2.45020829675494	2.45020829675494\\
62.375	0.14058	3.02922279998156	3.02922279998156\\
62.375	0.14424	3.63378129056614	3.63378129056614\\
62.375	0.1479	4.26388376850871	4.26388376850871\\
62.375	0.15156	4.91953023380926	4.91953023380926\\
62.375	0.15522	5.60072068646779	5.60072068646779\\
62.375	0.15888	6.30745512648431	6.30745512648431\\
62.375	0.16254	7.0397335538588	7.0397335538588\\
62.375	0.1662	7.79755596859126	7.79755596859126\\
62.375	0.16986	8.5809223706817	8.5809223706817\\
62.375	0.17352	9.38983276013013	9.38983276013013\\
62.375	0.17718	10.2242871369365	10.2242871369365\\
62.375	0.18084	11.0842855011009	11.0842855011009\\
62.375	0.1845	11.9698278526233	11.9698278526233\\
62.375	0.18816	12.8809141915036	12.8809141915036\\
62.375	0.19182	13.817544517742	13.817544517742\\
62.375	0.19548	14.7797188313383	14.7797188313383\\
62.375	0.19914	15.7674371322926	15.7674371322926\\
62.375	0.2028	16.7806994206048	16.7806994206048\\
62.375	0.20646	17.8195056962751	17.8195056962751\\
62.375	0.21012	18.8838559593033	18.8838559593033\\
62.375	0.21378	19.9737502096895	19.9737502096895\\
62.375	0.21744	21.0891884474337	21.0891884474337\\
62.375	0.2211	22.2301706725358	22.2301706725358\\
62.375	0.22476	23.3966968849959	23.3966968849959\\
62.375	0.22842	24.5887670848141	24.5887670848141\\
62.375	0.23208	25.8063812719902	25.8063812719902\\
62.375	0.23574	27.0495394465243	27.0495394465243\\
62.375	0.2394	28.3182416084163	28.3182416084163\\
62.375	0.24306	29.6124877576663	29.6124877576663\\
62.375	0.24672	30.9322778942744	30.9322778942744\\
62.375	0.25038	32.2776120182404	32.2776120182404\\
62.375	0.25404	33.6484901295644	33.6484901295644\\
62.375	0.2577	35.0449122282463	35.0449122282463\\
62.375	0.26136	36.4668783142862	36.4668783142862\\
62.375	0.26502	37.9143883876842	37.9143883876842\\
62.375	0.26868	39.3874424484401	39.3874424484401\\
62.375	0.27234	40.886040496554	40.886040496554\\
62.375	0.276	42.4101825320258	42.4101825320258\\
62.75	0.093	-2.58605194889449	-2.58605194889449\\
62.75	0.09666	-2.30633682820164	-2.30633682820164\\
62.75	0.10032	-2.00107772015083	-2.00107772015083\\
62.75	0.10398	-1.67027462474202	-1.67027462474202\\
62.75	0.10764	-1.31392754197523	-1.31392754197523\\
62.75	0.1113	-0.93203647185045	-0.93203647185045\\
62.75	0.11496	-0.524601414367694	-0.524601414367694\\
62.75	0.11862	-0.0916223695269824	-0.0916223695269824\\
62.75	0.12228	0.366900662671734	0.366900662671734\\
62.75	0.12594	0.850967682228406	0.850967682228406\\
62.75	0.1296	1.36057868914307	1.36057868914307\\
62.75	0.13326	1.89573368341572	1.89573368341572\\
62.75	0.13692	2.45643266504635	2.45643266504635\\
62.75	0.14058	3.04267563403496	3.04267563403496\\
62.75	0.14424	3.65446259038153	3.65446259038153\\
62.75	0.1479	4.2917935340861	4.2917935340861\\
62.75	0.15156	4.95466846514865	4.95466846514865\\
62.75	0.15522	5.64308738356917	5.64308738356917\\
62.75	0.15888	6.35705028934768	6.35705028934768\\
62.75	0.16254	7.09655718248417	7.09655718248417\\
62.75	0.1662	7.86160806297863	7.86160806297863\\
62.75	0.16986	8.65220293083106	8.65220293083106\\
62.75	0.17352	9.46834178604149	9.46834178604149\\
62.75	0.17718	10.3100246286099	10.3100246286099\\
62.75	0.18084	11.1772514585363	11.1772514585363\\
62.75	0.1845	12.0700222758206	12.0700222758206\\
62.75	0.18816	12.988337080463	12.988337080463\\
62.75	0.19182	13.9321958724633	13.9321958724633\\
62.75	0.19548	14.9015986518216	14.9015986518216\\
62.75	0.19914	15.8965454185379	15.8965454185379\\
62.75	0.2028	16.9170361726121	16.9170361726121\\
62.75	0.20646	17.9630709140444	17.9630709140444\\
62.75	0.21012	19.0346496428346	19.0346496428346\\
62.75	0.21378	20.1317723589828	20.1317723589828\\
62.75	0.21744	21.254439062489	21.254439062489\\
62.75	0.2211	22.4026497533531	22.4026497533531\\
62.75	0.22476	23.5764044315753	23.5764044315753\\
62.75	0.22842	24.7757030971554	24.7757030971554\\
62.75	0.23208	26.0005457500935	26.0005457500935\\
62.75	0.23574	27.2509323903895	27.2509323903895\\
62.75	0.2394	28.5268630180436	28.5268630180436\\
62.75	0.24306	29.8283376330556	29.8283376330556\\
62.75	0.24672	31.1553562354256	31.1553562354256\\
62.75	0.25038	32.5079188251537	32.5079188251537\\
62.75	0.25404	33.8860254022396	33.8860254022396\\
62.75	0.2577	35.2896759666836	35.2896759666836\\
62.75	0.26136	36.7188705184855	36.7188705184855\\
62.75	0.26502	38.1736090576454	38.1736090576454\\
62.75	0.26868	39.6538915841633	39.6538915841633\\
62.75	0.27234	41.1597180980392	41.1597180980392\\
62.75	0.276	42.691088599273	42.691088599273\\
63.125	0.093	-2.66420049676984	-2.66420049676984\\
63.125	0.09666	-2.37725691031499	-2.37725691031499\\
63.125	0.10032	-2.06476933650219	-2.06476933650219\\
63.125	0.10398	-1.72673777533138	-1.72673777533138\\
63.125	0.10764	-1.3631622268026	-1.3631622268026\\
63.125	0.1113	-0.974042690915821	-0.974042690915821\\
63.125	0.11496	-0.559379167671075	-0.559379167671075\\
63.125	0.11862	-0.119171657068366	-0.119171657068366\\
63.125	0.12228	0.346579840892341	0.346579840892341\\
63.125	0.12594	0.837875326211011	0.837875326211011\\
63.125	0.1296	1.35471479888767	1.35471479888767\\
63.125	0.13326	1.89709825892232	1.89709825892232\\
63.125	0.13692	2.46502570631494	2.46502570631494\\
63.125	0.14058	3.05849714106555	3.05849714106555\\
63.125	0.14424	3.67751256317412	3.67751256317412\\
63.125	0.1479	4.32207197264068	4.32207197264068\\
63.125	0.15156	4.99217536946522	4.99217536946522\\
63.125	0.15522	5.68782275364774	5.68782275364774\\
63.125	0.15888	6.40901412518824	6.40901412518824\\
63.125	0.16254	7.15574948408672	7.15574948408672\\
63.125	0.1662	7.92802883034318	7.92802883034318\\
63.125	0.16986	8.72585216395761	8.72585216395761\\
63.125	0.17352	9.54921948493002	9.54921948493002\\
63.125	0.17718	10.3981307932604	10.3981307932604\\
63.125	0.18084	11.2725860889488	11.2725860889488\\
63.125	0.1845	12.1725853719951	12.1725853719951\\
63.125	0.18816	13.0981286423995	13.0981286423995\\
63.125	0.19182	14.0492159001618	14.0492159001618\\
63.125	0.19548	15.0258471452821	15.0258471452821\\
63.125	0.19914	16.0280223777604	16.0280223777604\\
63.125	0.2028	17.0557415975966	17.0557415975966\\
63.125	0.20646	18.1090048047909	18.1090048047909\\
63.125	0.21012	19.1878119993431	19.1878119993431\\
63.125	0.21378	20.2921631812533	20.2921631812533\\
63.125	0.21744	21.4220583505214	21.4220583505214\\
63.125	0.2211	22.5774975071476	22.5774975071476\\
63.125	0.22476	23.7584806511317	23.7584806511317\\
63.125	0.22842	24.9650077824738	24.9650077824738\\
63.125	0.23208	26.1970789011739	26.1970789011739\\
63.125	0.23574	27.454694007232	27.454694007232\\
63.125	0.2394	28.737853100648	28.737853100648\\
63.125	0.24306	30.0465561814221	30.0465561814221\\
63.125	0.24672	31.3808032495541	31.3808032495541\\
63.125	0.25038	32.7405943050441	32.7405943050441\\
63.125	0.25404	34.125929347892	34.125929347892\\
63.125	0.2577	35.536808378098	35.536808378098\\
63.125	0.26136	36.9732313956619	36.9732313956619\\
63.125	0.26502	38.4351984005838	38.4351984005838\\
63.125	0.26868	39.9227093928637	39.9227093928637\\
63.125	0.27234	41.4357643725016	41.4357643725016\\
63.125	0.276	42.9743633394974	42.9743633394974\\
63.5	0.093	-2.73998037166801	-2.73998037166801\\
63.5	0.09666	-2.44580831945116	-2.44580831945116\\
63.5	0.10032	-2.12609227987636	-2.12609227987636\\
63.5	0.10398	-1.78083225294356	-1.78083225294356\\
63.5	0.10764	-1.41002823865278	-1.41002823865278\\
63.5	0.1113	-1.01368023700401	-1.01368023700401\\
63.5	0.11496	-0.591788247997272	-0.591788247997272\\
63.5	0.11862	-0.144352271632565	-0.144352271632565\\
63.5	0.12228	0.328627692090139	0.328627692090139\\
63.5	0.12594	0.827151643170799	0.827151643170799\\
63.5	0.1296	1.35121958160946	1.35121958160946\\
63.5	0.13326	1.9008315074061	1.9008315074061\\
63.5	0.13692	2.47598742056071	2.47598742056071\\
63.5	0.14058	3.07668732107331	3.07668732107331\\
63.5	0.14424	3.70293120894388	3.70293120894388\\
63.5	0.1479	4.35471908417244	4.35471908417244\\
63.5	0.15156	5.03205094675897	5.03205094675897\\
63.5	0.15522	5.73492679670349	5.73492679670349\\
63.5	0.15888	6.46334663400599	6.46334663400599\\
63.5	0.16254	7.21731045866646	7.21731045866646\\
63.5	0.1662	7.99681827068491	7.99681827068491\\
63.5	0.16986	8.80187007006134	8.80187007006134\\
63.5	0.17352	9.63246585679575	9.63246585679575\\
63.5	0.17718	10.4886056308881	10.4886056308881\\
63.5	0.18084	11.3702893923385	11.3702893923385\\
63.5	0.1845	12.2775171411469	12.2775171411469\\
63.5	0.18816	13.2102888773132	13.2102888773132\\
63.5	0.19182	14.1686046008375	14.1686046008375\\
63.5	0.19548	15.1524643117198	15.1524643117198\\
63.5	0.19914	16.1618680099601	16.1618680099601\\
63.5	0.2028	17.1968156955583	17.1968156955583\\
63.5	0.20646	18.2573073685145	18.2573073685145\\
63.5	0.21012	19.3433430288287	19.3433430288287\\
63.5	0.21378	20.4549226765009	20.4549226765009\\
63.5	0.21744	21.5920463115311	21.5920463115311\\
63.5	0.2211	22.7547139339192	22.7547139339192\\
63.5	0.22476	23.9429255436654	23.9429255436654\\
63.5	0.22842	25.1566811407695	25.1566811407695\\
63.5	0.23208	26.3959807252316	26.3959807252316\\
63.5	0.23574	27.6608242970516	27.6608242970516\\
63.5	0.2394	28.9512118562297	28.9512118562297\\
63.5	0.24306	30.2671434027657	30.2671434027657\\
63.5	0.24672	31.6086189366597	31.6086189366597\\
63.5	0.25038	32.9756384579117	32.9756384579117\\
63.5	0.25404	34.3682019665217	34.3682019665217\\
63.5	0.2577	35.7863094624896	35.7863094624896\\
63.5	0.26136	37.2299609458155	37.2299609458155\\
63.5	0.26502	38.6991564164994	38.6991564164994\\
63.5	0.26868	40.1938958745413	40.1938958745413\\
63.5	0.27234	41.7141793199411	41.7141793199411\\
63.5	0.276	43.260006752699	43.260006752699\\
63.875	0.093	-2.81339157358897	-2.81339157358897\\
63.875	0.09666	-2.51199105561013	-2.51199105561013\\
63.875	0.10032	-2.18504655027334	-2.18504655027334\\
63.875	0.10398	-1.83255805757854	-1.83255805757854\\
63.875	0.10764	-1.45452557752577	-1.45452557752577\\
63.875	0.1113	-1.050949110115	-1.050949110115\\
63.875	0.11496	-0.621828655346263	-0.621828655346263\\
63.875	0.11862	-0.167164213219566	-0.167164213219566\\
63.875	0.12228	0.313044216265128	0.313044216265128\\
63.875	0.12594	0.818796633107793	0.818796633107793\\
63.875	0.1296	1.35009303730844	1.35009303730844\\
63.875	0.13326	1.90693342886707	1.90693342886707\\
63.875	0.13692	2.48931780778369	2.48931780778369\\
63.875	0.14058	3.09724617405828	3.09724617405828\\
63.875	0.14424	3.73071852769084	3.73071852769084\\
63.875	0.1479	4.38973486868139	4.38973486868139\\
63.875	0.15156	5.07429519702992	5.07429519702992\\
63.875	0.15522	5.78439951273644	5.78439951273644\\
63.875	0.15888	6.52004781580093	6.52004781580093\\
63.875	0.16254	7.2812401062234	7.2812401062234\\
63.875	0.1662	8.06797638400385	8.06797638400385\\
63.875	0.16986	8.88025664914226	8.88025664914226\\
63.875	0.17352	9.71808090163868	9.71808090163868\\
63.875	0.17718	10.5814491414931	10.5814491414931\\
63.875	0.18084	11.4703613687054	11.4703613687054\\
63.875	0.1845	12.3848175832758	12.3848175832758\\
63.875	0.18816	13.3248177852041	13.3248177852041\\
63.875	0.19182	14.2903619744904	14.2903619744904\\
63.875	0.19548	15.2814501511347	15.2814501511347\\
63.875	0.19914	16.298082315137	16.298082315137\\
63.875	0.2028	17.3402584664972	17.3402584664972\\
63.875	0.20646	18.4079786052154	18.4079786052154\\
63.875	0.21012	19.5012427312916	19.5012427312916\\
63.875	0.21378	20.6200508447258	20.6200508447258\\
63.875	0.21744	21.764402945518	21.764402945518\\
63.875	0.2211	22.9342990336681	22.9342990336681\\
63.875	0.22476	24.1297391091762	24.1297391091762\\
63.875	0.22842	25.3507231720423	25.3507231720423\\
63.875	0.23208	26.5972512222664	26.5972512222664\\
63.875	0.23574	27.8693232598485	27.8693232598485\\
63.875	0.2394	29.1669392847885	29.1669392847885\\
63.875	0.24306	30.4900992970865	30.4900992970865\\
63.875	0.24672	31.8388032967425	31.8388032967425\\
63.875	0.25038	33.2130512837565	33.2130512837565\\
63.875	0.25404	34.6128432581285	34.6128432581285\\
63.875	0.2577	36.0381792198584	36.0381792198584\\
63.875	0.26136	37.4890591689463	37.4890591689463\\
63.875	0.26502	38.9654831053922	38.9654831053922\\
63.875	0.26868	40.4674510291961	40.4674510291961\\
63.875	0.27234	41.994962940358	41.994962940358\\
63.875	0.276	43.5480188388778	43.5480188388778\\
64.25	0.093	-2.88443410253276	-2.88443410253276\\
64.25	0.09666	-2.57580511879193	-2.57580511879193\\
64.25	0.10032	-2.24163214769314	-2.24163214769314\\
64.25	0.10398	-1.88191518923635	-1.88191518923635\\
64.25	0.10764	-1.49665424342158	-1.49665424342158\\
64.25	0.1113	-1.08584931024882	-1.08584931024882\\
64.25	0.11496	-0.649500389718085	-0.649500389718085\\
64.25	0.11862	-0.18760748182939	-0.18760748182939\\
64.25	0.12228	0.299829413417301	0.299829413417301\\
64.25	0.12594	0.812810296021956	0.812810296021956\\
64.25	0.1296	1.3513351659846	1.3513351659846\\
64.25	0.13326	1.91540402330523	1.91540402330523\\
64.25	0.13692	2.50501686798384	2.50501686798384\\
64.25	0.14058	3.12017370002043	3.12017370002043\\
64.25	0.14424	3.76087451941498	3.76087451941498\\
64.25	0.1479	4.42711932616753	4.42711932616753\\
64.25	0.15156	5.11890812027805	5.11890812027805\\
64.25	0.15522	5.83624090174656	5.83624090174656\\
64.25	0.15888	6.57911767057305	6.57911767057305\\
64.25	0.16254	7.34753842675751	7.34753842675751\\
64.25	0.1662	8.14150317029996	8.14150317029996\\
64.25	0.16986	8.96101190120037	8.96101190120037\\
64.25	0.17352	9.80606461945877	9.80606461945877\\
64.25	0.17718	10.6766613250752	10.6766613250752\\
64.25	0.18084	11.5728020180495	11.5728020180495\\
64.25	0.1845	12.4944866983818	12.4944866983818\\
64.25	0.18816	13.4417153660722	13.4417153660722\\
64.25	0.19182	14.4144880211205	14.4144880211205\\
64.25	0.19548	15.4128046635268	15.4128046635268\\
64.25	0.19914	16.436665293291	16.436665293291\\
64.25	0.2028	17.4860699104133	17.4860699104133\\
64.25	0.20646	18.5610185148935	18.5610185148935\\
64.25	0.21012	19.6615111067317	19.6615111067317\\
64.25	0.21378	20.7875476859278	20.7875476859278\\
64.25	0.21744	21.939128252482	21.939128252482\\
64.25	0.2211	23.1162528063941	23.1162528063941\\
64.25	0.22476	24.3189213476643	24.3189213476643\\
64.25	0.22842	25.5471338762923	25.5471338762923\\
64.25	0.23208	26.8008903922784	26.8008903922784\\
64.25	0.23574	28.0801908956225	28.0801908956225\\
64.25	0.2394	29.3850353863245	29.3850353863245\\
64.25	0.24306	30.7154238643845	30.7154238643845\\
64.25	0.24672	32.0713563298025	32.0713563298025\\
64.25	0.25038	33.4528327825785	33.4528327825785\\
64.25	0.25404	34.8598532227124	34.8598532227124\\
64.25	0.2577	36.2924176502044	36.2924176502044\\
64.25	0.26136	37.7505260650543	37.7505260650543\\
64.25	0.26502	39.2341784672622	39.2341784672622\\
64.25	0.26868	40.7433748568281	40.7433748568281\\
64.25	0.27234	42.2781152337519	42.2781152337519\\
64.25	0.276	43.8383995980337	43.8383995980337\\
64.625	0.093	-2.95310795849938	-2.95310795849938\\
64.625	0.09666	-2.63725050899655	-2.63725050899655\\
64.625	0.10032	-2.29584907213574	-2.29584907213574\\
64.625	0.10398	-1.92890364791695	-1.92890364791695\\
64.625	0.10764	-1.5364142363402	-1.5364142363402\\
64.625	0.1113	-1.11838083740547	-1.11838083740547\\
64.625	0.11496	-0.674803451112737	-0.674803451112737\\
64.625	0.11862	-0.205682077462031	-0.205682077462031\\
64.625	0.12228	0.288983283546651	0.288983283546651\\
64.625	0.12594	0.809192631913318	0.809192631913318\\
64.625	0.1296	1.35494596763796	1.35494596763796\\
64.625	0.13326	1.92624329072057	1.92624329072057\\
64.625	0.13692	2.52308460116117	2.52308460116117\\
64.625	0.14058	3.14546989895976	3.14546989895976\\
64.625	0.14424	3.79339918411632	3.79339918411632\\
64.625	0.1479	4.46687245663085	4.46687245663085\\
64.625	0.15156	5.16588971650337	5.16588971650337\\
64.625	0.15522	5.89045096373387	5.89045096373387\\
64.625	0.15888	6.64055619832235	6.64055619832235\\
64.625	0.16254	7.41620542026881	7.41620542026881\\
64.625	0.1662	8.21739862957325	8.21739862957325\\
64.625	0.16986	9.04413582623566	9.04413582623566\\
64.625	0.17352	9.89641701025606	9.89641701025606\\
64.625	0.17718	10.7742421816344	10.7742421816344\\
64.625	0.18084	11.6776113403708	11.6776113403708\\
64.625	0.1845	12.6065244864651	12.6065244864651\\
64.625	0.18816	13.5609816199174	13.5609816199174\\
64.625	0.19182	14.5409827407277	14.5409827407277\\
64.625	0.19548	15.546527848896	15.546527848896\\
64.625	0.19914	16.5776169444223	16.5776169444223\\
64.625	0.2028	17.6342500273065	17.6342500273065\\
64.625	0.20646	18.7164270975487	18.7164270975487\\
64.625	0.21012	19.8241481551489	19.8241481551489\\
64.625	0.21378	20.9574132001071	20.9574132001071\\
64.625	0.21744	22.1162222324232	22.1162222324232\\
64.625	0.2211	23.3005752520973	23.3005752520973\\
64.625	0.22476	24.5104722591295	24.5104722591295\\
64.625	0.22842	25.7459132535196	25.7459132535196\\
64.625	0.23208	27.0068982352676	27.0068982352676\\
64.625	0.23574	28.2934272043737	28.2934272043737\\
64.625	0.2394	29.6055001608377	29.6055001608377\\
64.625	0.24306	30.9431171046597	30.9431171046597\\
64.625	0.24672	32.3062780358397	32.3062780358397\\
64.625	0.25038	33.6949829543777	33.6949829543777\\
64.625	0.25404	35.1092318602736	35.1092318602736\\
64.625	0.2577	36.5490247535276	36.5490247535276\\
64.625	0.26136	38.0143616341395	38.0143616341395\\
64.625	0.26502	39.5052425021093	39.5052425021093\\
64.625	0.26868	41.0216673574372	41.0216673574372\\
64.625	0.27234	42.563636200123	42.563636200123\\
64.625	0.276	44.1311490301669	44.1311490301669\\
65	0.093	-3.01941314148879	-3.01941314148879\\
65	0.09666	-2.69632722622396	-2.69632722622396\\
65	0.10032	-2.34769732360115	-2.34769732360115\\
65	0.10398	-1.97352343362038	-1.97352343362038\\
65	0.10764	-1.57380555628163	-1.57380555628163\\
65	0.1113	-1.1485436915849	-1.1485436915849\\
65	0.11496	-0.697737839530177	-0.697737839530177\\
65	0.11862	-0.221388000117473	-0.221388000117473\\
65	0.12228	0.280505826653206	0.280505826653206\\
65	0.12594	0.807943640781863	0.807943640781863\\
65	0.1296	1.36092544226851	1.36092544226851\\
65	0.13326	1.93945123111311	1.93945123111311\\
65	0.13692	2.5435210073157	2.5435210073157\\
65	0.14058	3.17313477087628	3.17313477087628\\
65	0.14424	3.82829252179484	3.82829252179484\\
65	0.1479	4.50899426007136	4.50899426007136\\
65	0.15156	5.21523998570588	5.21523998570588\\
65	0.15522	5.94702969869838	5.94702969869838\\
65	0.15888	6.70436339904885	6.70436339904885\\
65	0.16254	7.48724108675731	7.48724108675731\\
65	0.1662	8.29566276182374	8.29566276182374\\
65	0.16986	9.12962842424814	9.12962842424814\\
65	0.17352	9.98913807403054	9.98913807403054\\
65	0.17718	10.8741917111709	10.8741917111709\\
65	0.18084	11.7847893356693	11.7847893356693\\
65	0.1845	12.7209309475256	12.7209309475256\\
65	0.18816	13.6826165467399	13.6826165467399\\
65	0.19182	14.6698461333122	14.6698461333122\\
65	0.19548	15.6826197072425	15.6826197072425\\
65	0.19914	16.7209372685307	16.7209372685307\\
65	0.2028	17.7847988171769	17.7847988171769\\
65	0.20646	18.8742043531811	18.8742043531811\\
65	0.21012	19.9891538765433	19.9891538765433\\
65	0.21378	21.1296473872635	21.1296473872635\\
65	0.21744	22.2956848853417	22.2956848853417\\
65	0.2211	23.4872663707778	23.4872663707778\\
65	0.22476	24.7043918435719	24.7043918435719\\
65	0.22842	25.947061303724	25.947061303724\\
65	0.23208	27.215274751234	27.215274751234\\
65	0.23574	28.5090321861021	28.5090321861021\\
65	0.2394	29.8283336083281	29.8283336083281\\
65	0.24306	31.1731790179121	31.1731790179121\\
65	0.24672	32.5435684148541	32.5435684148541\\
65	0.25038	33.939501799154	33.939501799154\\
65	0.25404	35.360979170812	35.360979170812\\
65	0.2577	36.8080005298279	36.8080005298279\\
65	0.26136	38.2805658762018	38.2805658762018\\
65	0.26502	39.7786752099337	39.7786752099337\\
65	0.26868	41.3023285310235	41.3023285310235\\
65	0.27234	42.8515258394714	42.8515258394714\\
65	0.276	44.4262671352772	44.4262671352772\\
65.375	0.093	-3.083349651501	-3.083349651501\\
65.375	0.09666	-2.75303527047419	-2.75303527047419\\
65.375	0.10032	-2.39717690208939	-2.39717690208939\\
65.375	0.10398	-2.01577454634661	-2.01577454634661\\
65.375	0.10764	-1.60882820324587	-1.60882820324587\\
65.375	0.1113	-1.17633787278714	-1.17633787278714\\
65.375	0.11496	-0.718303554970419	-0.718303554970419\\
65.375	0.11862	-0.234725249795725	-0.234725249795725\\
65.375	0.12228	0.274397042736952	0.274397042736952\\
65.375	0.12594	0.809063322627599	0.809063322627599\\
65.375	0.1296	1.36927358987623	1.36927358987623\\
65.375	0.13326	1.95502784448283	1.95502784448283\\
65.375	0.13692	2.56632608644742	2.56632608644742\\
65.375	0.14058	3.20316831577	3.20316831577\\
65.375	0.14424	3.86555453245056	3.86555453245056\\
65.375	0.1479	4.55348473648907	4.55348473648907\\
65.375	0.15156	5.26695892788558	5.26695892788558\\
65.375	0.15522	6.00597710664007	6.00597710664007\\
65.375	0.15888	6.77053927275255	6.77053927275255\\
65.375	0.16254	7.560645426223	7.560645426223\\
65.375	0.1662	8.37629556705143	8.37629556705143\\
65.375	0.16986	9.21748969523782	9.21748969523782\\
65.375	0.17352	10.0842278107822	10.0842278107822\\
65.375	0.17718	10.9765099136846	10.9765099136846\\
65.375	0.18084	11.8943360039449	11.8943360039449\\
65.375	0.1845	12.8377060815632	12.8377060815632\\
65.375	0.18816	13.8066201465395	13.8066201465395\\
65.375	0.19182	14.8010781988738	14.8010781988738\\
65.375	0.19548	15.8210802385661	15.8210802385661\\
65.375	0.19914	16.8666262656164	16.8666262656164\\
65.375	0.2028	17.9377162800246	17.9377162800246\\
65.375	0.20646	19.0343502817908	19.0343502817908\\
65.375	0.21012	20.1565282709149	20.1565282709149\\
65.375	0.21378	21.3042502473971	21.3042502473971\\
65.375	0.21744	22.4775162112373	22.4775162112373\\
65.375	0.2211	23.6763261624354	23.6763261624354\\
65.375	0.22476	24.9006801009915	24.9006801009915\\
65.375	0.22842	26.1505780269056	26.1505780269056\\
65.375	0.23208	27.4260199401776	27.4260199401776\\
65.375	0.23574	28.7270058408077	28.7270058408077\\
65.375	0.2394	30.0535357287957	30.0535357287957\\
65.375	0.24306	31.4056096041417	31.4056096041417\\
65.375	0.24672	32.7832274668457	32.7832274668457\\
65.375	0.25038	34.1863893169076	34.1863893169076\\
65.375	0.25404	35.6150951543275	35.6150951543275\\
65.375	0.2577	37.0693449791055	37.0693449791055\\
65.375	0.26136	38.5491387912414	38.5491387912414\\
65.375	0.26502	40.0544765907352	40.0544765907352\\
65.375	0.26868	41.5853583775871	41.5853583775871\\
65.375	0.27234	43.1417841517969	43.1417841517969\\
65.375	0.276	44.7237539133648	44.7237539133648\\
65.75	0.093	-3.14491748853605	-3.14491748853605\\
65.75	0.09666	-2.80737464174723	-2.80737464174723\\
65.75	0.10032	-2.44428780760043	-2.44428780760043\\
65.75	0.10398	-2.05565698609566	-2.05565698609566\\
65.75	0.10764	-1.64148217723293	-1.64148217723293\\
65.75	0.1113	-1.2017633810122	-1.2017633810122\\
65.75	0.11496	-0.736500597433491	-0.736500597433491\\
65.75	0.11862	-0.245693826496799	-0.245693826496799\\
65.75	0.12228	0.270656931797868	0.270656931797868\\
65.75	0.12594	0.81255167745052	0.81255167745052\\
65.75	0.1296	1.37999041046114	1.37999041046114\\
65.75	0.13326	1.97297313082974	1.97297313082974\\
65.75	0.13692	2.59149983855632	2.59149983855632\\
65.75	0.14058	3.23557053364089	3.23557053364089\\
65.75	0.14424	3.90518521608344	3.90518521608344\\
65.75	0.1479	4.60034388588396	4.60034388588396\\
65.75	0.15156	5.32104654304246	5.32104654304246\\
65.75	0.15522	6.06729318755895	6.06729318755895\\
65.75	0.15888	6.83908381943342	6.83908381943342\\
65.75	0.16254	7.63641843866586	7.63641843866586\\
65.75	0.1662	8.45929704525629	8.45929704525629\\
65.75	0.16986	9.30771963920467	9.30771963920467\\
65.75	0.17352	10.1816862205111	10.1816862205111\\
65.75	0.17718	11.0811967891754	11.0811967891754\\
65.75	0.18084	12.0062513451978	12.0062513451978\\
65.75	0.1845	12.9568498885781	12.9568498885781\\
65.75	0.18816	13.9329924193164	13.9329924193164\\
65.75	0.19182	14.9346789374127	14.9346789374127\\
65.75	0.19548	15.9619094428669	15.9619094428669\\
65.75	0.19914	17.0146839356792	17.0146839356792\\
65.75	0.2028	18.0930024158494	18.0930024158494\\
65.75	0.20646	19.1968648833776	19.1968648833776\\
65.75	0.21012	20.3262713382638	20.3262713382638\\
65.75	0.21378	21.4812217805079	21.4812217805079\\
65.75	0.21744	22.6617162101101	22.6617162101101\\
65.75	0.2211	23.8677546270702	23.8677546270702\\
65.75	0.22476	25.0993370313883	25.0993370313883\\
65.75	0.22842	26.3564634230643	26.3564634230643\\
65.75	0.23208	27.6391338020984	27.6391338020984\\
65.75	0.23574	28.9473481684904	28.9473481684904\\
65.75	0.2394	30.2811065222404	30.2811065222404\\
65.75	0.24306	31.6404088633484	31.6404088633484\\
65.75	0.24672	33.0252551918144	33.0252551918144\\
65.75	0.25038	34.4356455076383	34.4356455076383\\
65.75	0.25404	35.8715798108203	35.8715798108203\\
65.75	0.2577	37.3330581013602	37.3330581013602\\
65.75	0.26136	38.8200803792581	38.8200803792581\\
65.75	0.26502	40.332646644514	40.332646644514\\
65.75	0.26868	41.8707568971278	41.8707568971278\\
65.75	0.27234	43.4344111370996	43.4344111370996\\
65.75	0.276	45.0236093644294	45.0236093644294\\
66.125	0.093	-3.20411665259388	-3.20411665259388\\
66.125	0.09666	-2.85934534004307	-2.85934534004307\\
66.125	0.10032	-2.48903004013428	-2.48903004013428\\
66.125	0.10398	-2.09317075286751	-2.09317075286751\\
66.125	0.10764	-1.67176747824278	-1.67176747824278\\
66.125	0.1113	-1.22482021626006	-1.22482021626006\\
66.125	0.11496	-0.752328966919357	-0.752328966919357\\
66.125	0.11862	-0.254293730220676	-0.254293730220676\\
66.125	0.12228	0.269285493835996	0.269285493835996\\
66.125	0.12594	0.818408705250638	0.818408705250638\\
66.125	0.1296	1.39307590402326	1.39307590402326\\
66.125	0.13326	1.99328709015385	1.99328709015385\\
66.125	0.13692	2.61904226364243	2.61904226364243\\
66.125	0.14058	3.27034142448899	3.27034142448899\\
66.125	0.14424	3.94718457269354	3.94718457269354\\
66.125	0.1479	4.64957170825605	4.64957170825605\\
66.125	0.15156	5.37750283117655	5.37750283117655\\
66.125	0.15522	6.13097794145503	6.13097794145503\\
66.125	0.15888	6.90999703909149	6.90999703909149\\
66.125	0.16254	7.71456012408593	7.71456012408593\\
66.125	0.1662	8.54466719643835	8.54466719643835\\
66.125	0.16986	9.40031825614874	9.40031825614874\\
66.125	0.17352	10.2815133032171	10.2815133032171\\
66.125	0.17718	11.1882523376435	11.1882523376435\\
66.125	0.18084	12.1205353594278	12.1205353594278\\
66.125	0.1845	13.0783623685701	13.0783623685701\\
66.125	0.18816	14.0617333650704	14.0617333650704\\
66.125	0.19182	15.0706483489287	15.0706483489287\\
66.125	0.19548	16.1051073201449	16.1051073201449\\
66.125	0.19914	17.1651102787192	17.1651102787192\\
66.125	0.2028	18.2506572246514	18.2506572246514\\
66.125	0.20646	19.3617481579416	19.3617481579416\\
66.125	0.21012	20.4983830785897	20.4983830785897\\
66.125	0.21378	21.6605619865959	21.6605619865959\\
66.125	0.21744	22.84828488196	22.84828488196\\
66.125	0.2211	24.0615517646821	24.0615517646821\\
66.125	0.22476	25.3003626347622	25.3003626347622\\
66.125	0.22842	26.5647174922003	26.5647174922003\\
66.125	0.23208	27.8546163369964	27.8546163369964\\
66.125	0.23574	29.1700591691504	29.1700591691504\\
66.125	0.2394	30.5110459886624	30.5110459886624\\
66.125	0.24306	31.8775767955324	31.8775767955324\\
66.125	0.24672	33.2696515897604	33.2696515897604\\
66.125	0.25038	34.6872703713463	34.6872703713463\\
66.125	0.25404	36.1304331402902	36.1304331402902\\
66.125	0.2577	37.5991398965922	37.5991398965922\\
66.125	0.26136	39.093390640252	39.093390640252\\
66.125	0.26502	40.6131853712699	40.6131853712699\\
66.125	0.26868	42.1585240896457	42.1585240896457\\
66.125	0.27234	43.7294067953796	43.7294067953796\\
66.125	0.276	45.3258334884714	45.3258334884714\\
66.5	0.093	-3.26094714367454	-3.26094714367454\\
66.5	0.09666	-2.90894736536173	-2.90894736536173\\
66.5	0.10032	-2.53140359969098	-2.53140359969098\\
66.5	0.10398	-2.12831584666221	-2.12831584666221\\
66.5	0.10764	-1.69968410627547	-1.69968410627547\\
66.5	0.1113	-1.24550837853074	-1.24550837853074\\
66.5	0.11496	-0.76578866342804	-0.76578866342804\\
66.5	0.11862	-0.260524960967382	-0.260524960967382\\
66.5	0.12228	0.27028272885128	0.27028272885128\\
66.5	0.12594	0.826634406027905	0.826634406027905\\
66.5	0.1296	1.40853007056252	1.40853007056252\\
66.5	0.13326	2.01596972245512	2.01596972245512\\
66.5	0.13692	2.6489533617057	2.6489533617057\\
66.5	0.14058	3.30748098831426	3.30748098831426\\
66.5	0.14424	3.99155260228078	3.99155260228078\\
66.5	0.1479	4.7011682036053	4.7011682036053\\
66.5	0.15156	5.43632779228779	5.43632779228779\\
66.5	0.15522	6.19703136832827	6.19703136832827\\
66.5	0.15888	6.98327893172673	6.98327893172673\\
66.5	0.16254	7.79507048248317	7.79507048248317\\
66.5	0.1662	8.63240602059758	8.63240602059758\\
66.5	0.16986	9.49528554606995	9.49528554606995\\
66.5	0.17352	10.3837090589003	10.3837090589003\\
66.5	0.17718	11.2976765590887	11.2976765590887\\
66.5	0.18084	12.237188046635	12.237188046635\\
66.5	0.1845	13.2022435215393	13.2022435215393\\
66.5	0.18816	14.1928429838016	14.1928429838016\\
66.5	0.19182	15.2089864334219	15.2089864334219\\
66.5	0.19548	16.2506738704001	16.2506738704001\\
66.5	0.19914	17.3179052947364	17.3179052947364\\
66.5	0.2028	18.4106807064306	18.4106807064306\\
66.5	0.20646	19.5290001054828	19.5290001054828\\
66.5	0.21012	20.6728634918929	20.6728634918929\\
66.5	0.21378	21.8422708656611	21.8422708656611\\
66.5	0.21744	23.0372222267872	23.0372222267872\\
66.5	0.2211	24.2577175752713	24.2577175752713\\
66.5	0.22476	25.5037569111134	25.5037569111134\\
66.5	0.22842	26.7753402343135	26.7753402343135\\
66.5	0.23208	28.0724675448715	28.0724675448715\\
66.5	0.23574	29.3951388427875	29.3951388427875\\
66.5	0.2394	30.7433541280615	30.7433541280615\\
66.5	0.24306	32.1171134006935	32.1171134006935\\
66.5	0.24672	33.5164166606835	33.5164166606835\\
66.5	0.25038	34.9412639080314	34.9412639080314\\
66.5	0.25404	36.3916551427373	36.3916551427373\\
66.5	0.2577	37.8675903648013	37.8675903648013\\
66.5	0.26136	39.3690695742231	39.3690695742231\\
66.5	0.26502	40.896092771003	40.896092771003\\
66.5	0.26868	42.4486599551408	42.4486599551408\\
66.5	0.27234	44.0267711266367	44.0267711266367\\
66.5	0.276	45.6304262854904	45.6304262854904\\
66.875	0.093	-3.31540896177801	-3.31540896177801\\
66.875	0.09666	-2.95618071770321	-2.95618071770321\\
66.875	0.10032	-2.57140848627046	-2.57140848627046\\
66.875	0.10398	-2.1610922674797	-2.1610922674797\\
66.875	0.10764	-1.72523206133097	-1.72523206133097\\
66.875	0.1113	-1.26382786782425	-1.26382786782425\\
66.875	0.11496	-0.776879686959546	-0.776879686959546\\
66.875	0.11862	-0.264387518736891	-0.264387518736891\\
66.875	0.12228	0.273648636843769	0.273648636843769\\
66.875	0.12594	0.837228779782384	0.837228779782384\\
66.875	0.1296	1.426352910079	1.426352910079\\
66.875	0.13326	2.04102102773359	2.04102102773359\\
66.875	0.13692	2.68123313274616	2.68123313274616\\
66.875	0.14058	3.34698922511672	3.34698922511672\\
66.875	0.14424	4.03828930484523	4.03828930484523\\
66.875	0.1479	4.75513337193175	4.75513337193175\\
66.875	0.15156	5.49752142637623	5.49752142637623\\
66.875	0.15522	6.26545346817871	6.26545346817871\\
66.875	0.15888	7.05892949733916	7.05892949733916\\
66.875	0.16254	7.87794951385759	7.87794951385759\\
66.875	0.1662	8.722513517734	8.722513517734\\
66.875	0.16986	9.59262150896837	9.59262150896837\\
66.875	0.17352	10.4882734875607	10.4882734875607\\
66.875	0.17718	11.4094694535111	11.4094694535111\\
66.875	0.18084	12.3562094068194	12.3562094068194\\
66.875	0.1845	13.3284933474857	13.3284933474857\\
66.875	0.18816	14.32632127551	14.32632127551\\
66.875	0.19182	15.3496931908923	15.3496931908923\\
66.875	0.19548	16.3986090936325	16.3986090936325\\
66.875	0.19914	17.4730689837307	17.4730689837307\\
66.875	0.2028	18.5730728611869	18.5730728611869\\
66.875	0.20646	19.6986207260011	19.6986207260011\\
66.875	0.21012	20.8497125781733	20.8497125781733\\
66.875	0.21378	22.0263484177034	22.0263484177034\\
66.875	0.21744	23.2285282445916	23.2285282445916\\
66.875	0.2211	24.4562520588376	24.4562520588376\\
66.875	0.22476	25.7095198604417	25.7095198604417\\
66.875	0.22842	26.9883316494038	26.9883316494038\\
66.875	0.23208	28.2926874257238	28.2926874257238\\
66.875	0.23574	29.6225871894018	29.6225871894018\\
66.875	0.2394	30.9780309404379	30.9780309404379\\
66.875	0.24306	32.3590186788318	32.3590186788318\\
66.875	0.24672	33.7655504045838	33.7655504045838\\
66.875	0.25038	35.1976261176937	35.1976261176937\\
66.875	0.25404	36.6552458181616	36.6552458181616\\
66.875	0.2577	38.1384095059876	38.1384095059876\\
66.875	0.26136	39.6471171811714	39.6471171811714\\
66.875	0.26502	41.1813688437133	41.1813688437133\\
66.875	0.26868	42.7411644936131	42.7411644936131\\
66.875	0.27234	44.3265041308709	44.3265041308709\\
66.875	0.276	45.9373877554867	45.9373877554867\\
67.25	0.093	-3.36750210690428	-3.36750210690428\\
67.25	0.09666	-3.00104539706749	-3.00104539706749\\
67.25	0.10032	-2.60904469987273	-2.60904469987273\\
67.25	0.10398	-2.19150001531998	-2.19150001531998\\
67.25	0.10764	-1.74841134340926	-1.74841134340926\\
67.25	0.1113	-1.27977868414054	-1.27977868414054\\
67.25	0.11496	-0.785602037513847	-0.785602037513847\\
67.25	0.11862	-0.265881403529194	-0.265881403529194\\
67.25	0.12228	0.279383217813455	0.279383217813455\\
67.25	0.12594	0.850191826514076	0.850191826514076\\
67.25	0.1296	1.44654442257268	1.44654442257268\\
67.25	0.13326	2.06844100598927	2.06844100598927\\
67.25	0.13692	2.71588157676383	2.71588157676383\\
67.25	0.14058	3.38886613489639	3.38886613489639\\
67.25	0.14424	4.0873946803869	4.0873946803869\\
67.25	0.1479	4.8114672132354	4.8114672132354\\
67.25	0.15156	5.56108373344189	5.56108373344189\\
67.25	0.15522	6.33624424100635	6.33624424100635\\
67.25	0.15888	7.1369487359288	7.1369487359288\\
67.25	0.16254	7.96319721820923	7.96319721820923\\
67.25	0.1662	8.81498968784763	8.81498968784763\\
67.25	0.16986	9.69232614484401	9.69232614484401\\
67.25	0.17352	10.5952065891984	10.5952065891984\\
67.25	0.17718	11.5236310209107	11.5236310209107\\
67.25	0.18084	12.477599439981	12.477599439981\\
67.25	0.1845	13.4571118464093	13.4571118464093\\
67.25	0.18816	14.4621682401956	14.4621682401956\\
67.25	0.19182	15.4927686213399	15.4927686213399\\
67.25	0.19548	16.5489129898421	16.5489129898421\\
67.25	0.19914	17.6306013457023	17.6306013457023\\
67.25	0.2028	18.7378336889205	18.7378336889205\\
67.25	0.20646	19.8706100194967	19.8706100194967\\
67.25	0.21012	21.0289303374309	21.0289303374309\\
67.25	0.21378	22.212794642723	22.212794642723\\
67.25	0.21744	23.4222029353731	23.4222029353731\\
67.25	0.2211	24.6571552153812	24.6571552153812\\
67.25	0.22476	25.9176514827473	25.9176514827473\\
67.25	0.22842	27.2036917374713	27.2036917374713\\
67.25	0.23208	28.5152759795534	28.5152759795534\\
67.25	0.23574	29.8524042089934	29.8524042089934\\
67.25	0.2394	31.2150764257914	31.2150764257914\\
67.25	0.24306	32.6032926299473	32.6032926299473\\
67.25	0.24672	34.0170528214613	34.0170528214613\\
67.25	0.25038	35.4563570003332	35.4563570003332\\
67.25	0.25404	36.9212051665632	36.9212051665632\\
67.25	0.2577	38.4115973201511	38.4115973201511\\
67.25	0.26136	39.9275334610969	39.9275334610969\\
67.25	0.26502	41.4690135894008	41.4690135894008\\
67.25	0.26868	43.0360377050626	43.0360377050626\\
67.25	0.27234	44.6286058080824	44.6286058080824\\
67.25	0.276	46.2467178984602	46.2467178984602\\
67.625	0.093	-3.41722657905337	-3.41722657905337\\
67.625	0.09666	-3.04354140345458	-3.04354140345458\\
67.625	0.10032	-2.64431224049784	-2.64431224049784\\
67.625	0.10398	-2.21953909018309	-2.21953909018309\\
67.625	0.10764	-1.76922195251037	-1.76922195251037\\
67.625	0.1113	-1.29336082747965	-1.29336082747965\\
67.625	0.11496	-0.791955715090971	-0.791955715090971\\
67.625	0.11862	-0.265006615344321	-0.265006615344321\\
67.625	0.12228	0.287486471760326	0.287486471760326\\
67.625	0.12594	0.865523546222937	0.865523546222937\\
67.625	0.1296	1.46910460804354	1.46910460804354\\
67.625	0.13326	2.09822965722212	2.09822965722212\\
67.625	0.13692	2.75289869375868	2.75289869375868\\
67.625	0.14058	3.43311171765322	3.43311171765322\\
67.625	0.14424	4.13886872890573	4.13886872890573\\
67.625	0.1479	4.87016972751623	4.87016972751623\\
67.625	0.15156	5.62701471348472	5.62701471348472\\
67.625	0.15522	6.40940368681118	6.40940368681118\\
67.625	0.15888	7.21733664749562	7.21733664749562\\
67.625	0.16254	8.05081359553804	8.05081359553804\\
67.625	0.1662	8.90983453093844	8.90983453093844\\
67.625	0.16986	9.7943994536968	9.7943994536968\\
67.625	0.17352	10.7045083638132	10.7045083638132\\
67.625	0.17718	11.6401612612875	11.6401612612875\\
67.625	0.18084	12.6013581461198	12.6013581461198\\
67.625	0.1845	13.5880990183101	13.5880990183101\\
67.625	0.18816	14.6003838778584	14.6003838778584\\
67.625	0.19182	15.6382127247646	15.6382127247646\\
67.625	0.19548	16.7015855590289	16.7015855590289\\
67.625	0.19914	17.7905023806511	17.7905023806511\\
67.625	0.2028	18.9049631896313	18.9049631896313\\
67.625	0.20646	20.0449679859695	20.0449679859695\\
67.625	0.21012	21.2105167696656	21.2105167696656\\
67.625	0.21378	22.4016095407197	22.4016095407197\\
67.625	0.21744	23.6182462991318	23.6182462991318\\
67.625	0.2211	24.8604270449019	24.8604270449019\\
67.625	0.22476	26.12815177803	26.12815177803\\
67.625	0.22842	27.421420498516	27.421420498516\\
67.625	0.23208	28.7402332063601	28.7402332063601\\
67.625	0.23574	30.0845899015621	30.0845899015621\\
67.625	0.2394	31.4544905841221	31.4544905841221\\
67.625	0.24306	32.8499352540401	32.8499352540401\\
67.625	0.24672	34.270923911316	34.270923911316\\
67.625	0.25038	35.7174565559499	35.7174565559499\\
67.625	0.25404	37.1895331879418	37.1895331879418\\
67.625	0.2577	38.6871538072917	38.6871538072917\\
67.625	0.26136	40.2103184139996	40.2103184139996\\
67.625	0.26502	41.7590270080654	41.7590270080654\\
67.625	0.26868	43.3332795894893	43.3332795894893\\
67.625	0.27234	44.933076158271	44.933076158271\\
67.625	0.276	46.5584167144108	46.5584167144108\\
68	0.093	-3.46458237822528	-3.46458237822528\\
68	0.09666	-3.08366873686449	-3.08366873686449\\
68	0.10032	-2.67721110814575	-2.67721110814575\\
68	0.10398	-2.24520949206901	-2.24520949206901\\
68	0.10764	-1.78766388863429	-1.78766388863429\\
68	0.1113	-1.30457429784158	-1.30457429784158\\
68	0.11496	-0.795940719690897	-0.795940719690897\\
68	0.11862	-0.261763154182256	-0.261763154182256\\
68	0.12228	0.297958398684388	0.297958398684388\\
68	0.12594	0.883223938908989	0.883223938908989\\
68	0.1296	1.49403346649159	1.49403346649159\\
68	0.13326	2.13038698143216	2.13038698143216\\
68	0.13692	2.79228448373071	2.79228448373071\\
68	0.14058	3.47972597338726	3.47972597338726\\
68	0.14424	4.19271145040176	4.19271145040176\\
68	0.1479	4.93124091477426	4.93124091477426\\
68	0.15156	5.69531436650474	5.69531436650474\\
68	0.15522	6.48493180559319	6.48493180559319\\
68	0.15888	7.30009323203963	7.30009323203963\\
68	0.16254	8.14079864584405	8.14079864584405\\
68	0.1662	9.00704804700644	9.00704804700644\\
68	0.16986	9.8988414355268	9.8988414355268\\
68	0.17352	10.8161788114051	10.8161788114051\\
68	0.17718	11.7590601746415	11.7590601746415\\
68	0.18084	12.7274855252358	12.7274855252358\\
68	0.1845	13.7214548631881	13.7214548631881\\
68	0.18816	14.7409681884983	14.7409681884983\\
68	0.19182	15.7860255011666	15.7860255011666\\
68	0.19548	16.8566268011928	16.8566268011928\\
68	0.19914	17.952772088577	17.952772088577\\
68	0.2028	19.0744613633192	19.0744613633192\\
68	0.20646	20.2216946254194	20.2216946254194\\
68	0.21012	21.3944718748775	21.3944718748775\\
68	0.21378	22.5927931116937	22.5927931116937\\
68	0.21744	23.8166583358678	23.8166583358678\\
68	0.2211	25.0660675473999	25.0660675473999\\
68	0.22476	26.3410207462899	26.3410207462899\\
68	0.22842	27.641517932538	27.641517932538\\
68	0.23208	28.967559106144	28.967559106144\\
68	0.23574	30.319144267108	30.319144267108\\
68	0.2394	31.69627341543	31.69627341543\\
68	0.24306	33.0989465511099	33.0989465511099\\
68	0.24672	34.5271636741479	34.5271636741479\\
68	0.25038	35.9809247845438	35.9809247845438\\
68	0.25404	37.4602298822977	37.4602298822977\\
68	0.2577	38.9650789674096	38.9650789674096\\
68	0.26136	40.4954720398794	40.4954720398794\\
68	0.26502	42.0514090997073	42.0514090997073\\
68	0.26868	43.6328901468931	43.6328901468931\\
68	0.27234	45.2399151814369	45.2399151814369\\
68	0.276	46.8724842033387	46.8724842033387\\
68.375	0.093	-3.50956950442002	-3.50956950442002\\
68.375	0.09666	-3.12142739729724	-3.12142739729724\\
68.375	0.10032	-2.70774130281647	-2.70774130281647\\
68.375	0.10398	-2.26851122097774	-2.26851122097774\\
68.375	0.10764	-1.80373715178104	-1.80373715178104\\
68.375	0.1113	-1.31341909522635	-1.31341909522635\\
68.375	0.11496	-0.797557051313674	-0.797557051313674\\
68.375	0.11862	-0.256151020043022	-0.256151020043022\\
68.375	0.12228	0.310798998585613	0.310798998585613\\
68.375	0.12594	0.903293004572225	0.903293004572225\\
68.375	0.1296	1.52133099791682	1.52133099791682\\
68.375	0.13326	2.16491297861937	2.16491297861937\\
68.375	0.13692	2.83403894667993	2.83403894667993\\
68.375	0.14058	3.52870890209846	3.52870890209846\\
68.375	0.14424	4.24892284487497	4.24892284487497\\
68.375	0.1479	4.99468077500945	4.99468077500945\\
68.375	0.15156	5.76598269250193	5.76598269250193\\
68.375	0.15522	6.56282859735238	6.56282859735238\\
68.375	0.15888	7.38521848956081	7.38521848956081\\
68.375	0.16254	8.23315236912721	8.23315236912721\\
68.375	0.1662	9.1066302360516	9.1066302360516\\
68.375	0.16986	10.005652090334	10.005652090334\\
68.375	0.17352	10.9302179319743	10.9302179319743\\
68.375	0.17718	11.8803277609726	11.8803277609726\\
68.375	0.18084	12.8559815773289	12.8559815773289\\
68.375	0.1845	13.8571793810432	13.8571793810432\\
68.375	0.18816	14.8839211721155	14.8839211721155\\
68.375	0.19182	15.9362069505457	15.9362069505457\\
68.375	0.19548	17.0140367163339	17.0140367163339\\
68.375	0.19914	18.1174104694802	18.1174104694802\\
68.375	0.2028	19.2463282099844	19.2463282099844\\
68.375	0.20646	20.4007899378465	20.4007899378465\\
68.375	0.21012	21.5807956530667	21.5807956530667\\
68.375	0.21378	22.7863453556448	22.7863453556448\\
68.375	0.21744	24.0174390455809	24.0174390455809\\
68.375	0.2211	25.2740767228749	25.2740767228749\\
68.375	0.22476	26.556258387527	26.556258387527\\
68.375	0.22842	27.863984039537	27.863984039537\\
68.375	0.23208	29.1972536789051	29.1972536789051\\
68.375	0.23574	30.5560673056311	30.5560673056311\\
68.375	0.2394	31.940424919715	31.940424919715\\
68.375	0.24306	33.350326521157	33.350326521157\\
68.375	0.24672	34.7857721099569	34.7857721099569\\
68.375	0.25038	36.2467616861149	36.2467616861149\\
68.375	0.25404	37.7332952496308	37.7332952496308\\
68.375	0.2577	39.2453728005046	39.2453728005046\\
68.375	0.26136	40.7829943387365	40.7829943387365\\
68.375	0.26502	42.3461598643263	42.3461598643263\\
68.375	0.26868	43.9348693772741	43.9348693772741\\
68.375	0.27234	45.5491228775799	45.5491228775799\\
68.375	0.276	47.1889203652437	47.1889203652437\\
68.75	0.093	-3.55218795763754	-3.55218795763754\\
68.75	0.09666	-3.15681738475276	-3.15681738475276\\
68.75	0.10032	-2.73590282451001	-2.73590282451001\\
68.75	0.10398	-2.28944427690928	-2.28944427690928\\
68.75	0.10764	-1.81744174195058	-1.81744174195058\\
68.75	0.1113	-1.3198952196339	-1.3198952196339\\
68.75	0.11496	-0.796804709959225	-0.796804709959225\\
68.75	0.11862	-0.248170212926576	-0.248170212926576\\
68.75	0.12228	0.326008271464056	0.326008271464056\\
68.75	0.12594	0.925730743212659	0.925730743212659\\
68.75	0.1296	1.55099720231925	1.55099720231925\\
68.75	0.13326	2.2018076487838	2.2018076487838\\
68.75	0.13692	2.87816208260635	2.87816208260635\\
68.75	0.14058	3.58006050378688	3.58006050378688\\
68.75	0.14424	4.30750291232539	4.30750291232539\\
68.75	0.1479	5.06048930822186	5.06048930822186\\
68.75	0.15156	5.83901969147632	5.83901969147632\\
68.75	0.15522	6.64309406208877	6.64309406208877\\
68.75	0.15888	7.4727124200592	7.4727124200592\\
68.75	0.16254	8.3278747653876	8.3278747653876\\
68.75	0.1662	9.20858109807399	9.20858109807399\\
68.75	0.16986	10.1148314181183	10.1148314181183\\
68.75	0.17352	11.0466257255207	11.0466257255207\\
68.75	0.17718	12.003964020281	12.003964020281\\
68.75	0.18084	12.9868463023993	12.9868463023993\\
68.75	0.1845	13.9952725718756	13.9952725718756\\
68.75	0.18816	15.0292428287098	15.0292428287098\\
68.75	0.19182	16.0887570729021	16.0887570729021\\
68.75	0.19548	17.1738153044523	17.1738153044523\\
68.75	0.19914	18.2844175233605	18.2844175233605\\
68.75	0.2028	19.4205637296267	19.4205637296267\\
68.75	0.20646	20.5822539232509	20.5822539232509\\
68.75	0.21012	21.769488104233	21.769488104233\\
68.75	0.21378	22.9822662725731	22.9822662725731\\
68.75	0.21744	24.2205884282712	24.2205884282712\\
68.75	0.2211	25.4844545713272	25.4844545713272\\
68.75	0.22476	26.7738647017413	26.7738647017413\\
68.75	0.22842	28.0888188195133	28.0888188195133\\
68.75	0.23208	29.4293169246434	29.4293169246434\\
68.75	0.23574	30.7953590171314	30.7953590171314\\
68.75	0.2394	32.1869450969773	32.1869450969773\\
68.75	0.24306	33.6040751641813	33.6040751641813\\
68.75	0.24672	35.0467492187432	35.0467492187432\\
68.75	0.25038	36.5149672606632	36.5149672606632\\
68.75	0.25404	38.008729289941	38.008729289941\\
68.75	0.2577	39.5280353065769	39.5280353065769\\
68.75	0.26136	41.0728853105707	41.0728853105707\\
68.75	0.26502	42.6432793019226	42.6432793019226\\
68.75	0.26868	44.2392172806324	44.2392172806324\\
68.75	0.27234	45.8606992467002	45.8606992467002\\
68.75	0.276	47.5077252001259	47.5077252001259\\
69.125	0.093	-3.59243773787788	-3.59243773787788\\
69.125	0.09666	-3.18983869923111	-3.18983869923111\\
69.125	0.10032	-2.76169567322636	-2.76169567322636\\
69.125	0.10398	-2.30800865986363	-2.30800865986363\\
69.125	0.10764	-1.82877765914294	-1.82877765914294\\
69.125	0.1113	-1.32400267106426	-1.32400267106426\\
69.125	0.11496	-0.7936836956276	-0.7936836956276\\
69.125	0.11862	-0.237820732832953	-0.237820732832953\\
69.125	0.12228	0.34358621731967	0.34358621731967\\
69.125	0.12594	0.950537154830277	0.950537154830277\\
69.125	0.1296	1.58303207969886	1.58303207969886\\
69.125	0.13326	2.24107099192541	2.24107099192541\\
69.125	0.13692	2.92465389150994	2.92465389150994\\
69.125	0.14058	3.63378077845247	3.63378077845247\\
69.125	0.14424	4.36845165275297	4.36845165275297\\
69.125	0.1479	5.12866651441144	5.12866651441144\\
69.125	0.15156	5.9144253634279	5.9144253634279\\
69.125	0.15522	6.72572819980234	6.72572819980234\\
69.125	0.15888	7.56257502353476	7.56257502353476\\
69.125	0.16254	8.42496583462517	8.42496583462517\\
69.125	0.1662	9.31290063307354	9.31290063307354\\
69.125	0.16986	10.2263794188799	10.2263794188799\\
69.125	0.17352	11.1654021920442	11.1654021920442\\
69.125	0.17718	12.1299689525665	12.1299689525665\\
69.125	0.18084	13.1200797004468	13.1200797004468\\
69.125	0.1845	14.1357344356851	14.1357344356851\\
69.125	0.18816	15.1769331582814	15.1769331582814\\
69.125	0.19182	16.2436758682356	16.2436758682356\\
69.125	0.19548	17.3359625655478	17.3359625655478\\
69.125	0.19914	18.453793250218	18.453793250218\\
69.125	0.2028	19.5971679222462	19.5971679222462\\
69.125	0.20646	20.7660865816323	20.7660865816323\\
69.125	0.21012	21.9605492283765	21.9605492283765\\
69.125	0.21378	23.1805558624786	23.1805558624786\\
69.125	0.21744	24.4261064839387	24.4261064839387\\
69.125	0.2211	25.6972010927567	25.6972010927567\\
69.125	0.22476	26.9938396889328	26.9938396889328\\
69.125	0.22842	28.3160222724668	28.3160222724668\\
69.125	0.23208	29.6637488433588	29.6637488433588\\
69.125	0.23574	31.0370194016088	31.0370194016088\\
69.125	0.2394	32.4358339472168	32.4358339472168\\
69.125	0.24306	33.8601924801827	33.8601924801827\\
69.125	0.24672	35.3100950005066	35.3100950005066\\
69.125	0.25038	36.7855415081886	36.7855415081886\\
69.125	0.25404	38.2865320032285	38.2865320032285\\
69.125	0.2577	39.8130664856263	39.8130664856263\\
69.125	0.26136	41.3651449553822	41.3651449553822\\
69.125	0.26502	42.942767412496	42.942767412496\\
69.125	0.26868	44.5459338569678	44.5459338569678\\
69.125	0.27234	46.1746442887976	46.1746442887976\\
69.125	0.276	47.8288987079853	47.8288987079853\\
69.5	0.093	-3.63031884514102	-3.63031884514102\\
69.5	0.09666	-3.22049134073226	-3.22049134073226\\
69.5	0.10032	-2.78511984896551	-2.78511984896551\\
69.5	0.10398	-2.32420436984079	-2.32420436984079\\
69.5	0.10764	-1.8377449033581	-1.8377449033581\\
69.5	0.1113	-1.32574144951743	-1.32574144951743\\
69.5	0.11496	-0.788194008318769	-0.788194008318769\\
69.5	0.11862	-0.225102579762131	-0.225102579762131\\
69.5	0.12228	0.363532836152489	0.363532836152489\\
69.5	0.12594	0.977712239425086	0.977712239425086\\
69.5	0.1296	1.61743563005566	1.61743563005566\\
69.5	0.13326	2.28270300804421	2.28270300804421\\
69.5	0.13692	2.97351437339074	2.97351437339074\\
69.5	0.14058	3.68986972609526	3.68986972609526\\
69.5	0.14424	4.43176906615776	4.43176906615776\\
69.5	0.1479	5.19921239357822	5.19921239357822\\
69.5	0.15156	5.99219970835668	5.99219970835668\\
69.5	0.15522	6.81073101049311	6.81073101049311\\
69.5	0.15888	7.65480629998753	7.65480629998753\\
69.5	0.16254	8.52442557683993	8.52442557683993\\
69.5	0.1662	9.4195888410503	9.4195888410503\\
69.5	0.16986	10.3402960926186	10.3402960926186\\
69.5	0.17352	11.286547331545	11.286547331545\\
69.5	0.17718	12.2583425578293	12.2583425578293\\
69.5	0.18084	13.2556817714716	13.2556817714716\\
69.5	0.1845	14.2785649724718	14.2785649724718\\
69.5	0.18816	15.3269921608301	15.3269921608301\\
69.5	0.19182	16.4009633365463	16.4009633365463\\
69.5	0.19548	17.5004784996205	17.5004784996205\\
69.5	0.19914	18.6255376500527	18.6255376500527\\
69.5	0.2028	19.7761407878429	19.7761407878429\\
69.5	0.20646	20.9522879129911	20.9522879129911\\
69.5	0.21012	22.1539790254971	22.1539790254971\\
69.5	0.21378	23.3812141253613	23.3812141253613\\
69.5	0.21744	24.6339932125833	24.6339932125833\\
69.5	0.2211	25.9123162871634	25.9123162871634\\
69.5	0.22476	27.2161833491015	27.2161833491015\\
69.5	0.22842	28.5455943983975	28.5455943983975\\
69.5	0.23208	29.9005494350515	29.9005494350515\\
69.5	0.23574	31.2810484590635	31.2810484590635\\
69.5	0.2394	32.6870914704335	32.6870914704335\\
69.5	0.24306	34.1186784691614	34.1186784691614\\
69.5	0.24672	35.5758094552473	35.5758094552473\\
69.5	0.25038	37.0584844286912	37.0584844286912\\
69.5	0.25404	38.5667033894931	38.5667033894931\\
69.5	0.2577	40.100466337653	40.100466337653\\
69.5	0.26136	41.6597732731708	41.6597732731708\\
69.5	0.26502	43.2446241960466	43.2446241960466\\
69.5	0.26868	44.8550191062804	44.8550191062804\\
69.5	0.27234	46.4909580038722	46.4909580038722\\
69.5	0.276	48.152440888822	48.152440888822\\
69.875	0.093	-3.66583127942699	-3.66583127942699\\
69.875	0.09666	-3.24877530925623	-3.24877530925623\\
69.875	0.10032	-2.80617535172749	-2.80617535172749\\
69.875	0.10398	-2.33803140684077	-2.33803140684077\\
69.875	0.10764	-1.84434347459609	-1.84434347459609\\
69.875	0.1113	-1.32511155499343	-1.32511155499343\\
69.875	0.11496	-0.780335648032768	-0.780335648032768\\
69.875	0.11862	-0.210015753714133	-0.210015753714133\\
69.875	0.12228	0.385848127962477	0.385848127962477\\
69.875	0.12594	1.00725599699707	1.00725599699707\\
69.875	0.1296	1.65420785338965	1.65420785338965\\
69.875	0.13326	2.32670369714019	2.32670369714019\\
69.875	0.13692	3.02474352824871	3.02474352824871\\
69.875	0.14058	3.74832734671523	3.74832734671523\\
69.875	0.14424	4.49745515253972	4.49745515253972\\
69.875	0.1479	5.27212694572218	5.27212694572218\\
69.875	0.15156	6.07234272626263	6.07234272626263\\
69.875	0.15522	6.89810249416105	6.89810249416105\\
69.875	0.15888	7.74940624941748	7.74940624941748\\
69.875	0.16254	8.62625399203186	8.62625399203186\\
69.875	0.1662	9.52864572200423	9.52864572200423\\
69.875	0.16986	10.4565814393346	10.4565814393346\\
69.875	0.17352	11.4100611440229	11.4100611440229\\
69.875	0.17718	12.3890848360692	12.3890848360692\\
69.875	0.18084	13.3936525154735	13.3936525154735\\
69.875	0.1845	14.4237641822357	14.4237641822357\\
69.875	0.18816	15.479419836356	15.479419836356\\
69.875	0.19182	16.5606194778342	16.5606194778342\\
69.875	0.19548	17.6673631066704	17.6673631066704\\
69.875	0.19914	18.7996507228646	18.7996507228646\\
69.875	0.2028	19.9574823264168	19.9574823264168\\
69.875	0.20646	21.1408579173269	21.1408579173269\\
69.875	0.21012	22.349777495595	22.349777495595\\
69.875	0.21378	23.5842410612211	23.5842410612211\\
69.875	0.21744	24.8442486142052	24.8442486142052\\
69.875	0.2211	26.1298001545473	26.1298001545473\\
69.875	0.22476	27.4408956822473	27.4408956822473\\
69.875	0.22842	28.7775351973053	28.7775351973053\\
69.875	0.23208	30.1397186997213	30.1397186997213\\
69.875	0.23574	31.5274461894953	31.5274461894953\\
69.875	0.2394	32.9407176666273	32.9407176666273\\
69.875	0.24306	34.3795331311172	34.3795331311172\\
69.875	0.24672	35.8438925829651	35.8438925829651\\
69.875	0.25038	37.333796022171	37.333796022171\\
69.875	0.25404	38.8492434487349	38.8492434487349\\
69.875	0.2577	40.3902348626567	40.3902348626567\\
69.875	0.26136	41.9567702639366	41.9567702639366\\
69.875	0.26502	43.5488496525744	43.5488496525744\\
69.875	0.26868	45.1664730285702	45.1664730285702\\
69.875	0.27234	46.809640391924	46.809640391924\\
69.875	0.276	48.4783517426357	48.4783517426357\\
70.25	0.093	-3.69897504073576	-3.69897504073576\\
70.25	0.09666	-3.27469060480301	-3.27469060480301\\
70.25	0.10032	-2.8248621815123	-2.8248621815123\\
70.25	0.10398	-2.3494897708636	-2.3494897708636\\
70.25	0.10764	-1.8485733728569	-1.8485733728569\\
70.25	0.1113	-1.32211298749222	-1.32211298749222\\
70.25	0.11496	-0.770108614769576	-0.770108614769576\\
70.25	0.11862	-0.192560254688958	-0.192560254688958\\
70.25	0.12228	0.41053209274965	0.41053209274965\\
70.25	0.12594	1.03916842754622	1.03916842754622\\
70.25	0.1296	1.69334874970079	1.69334874970079\\
70.25	0.13326	2.37307305921334	2.37307305921334\\
70.25	0.13692	3.07834135608386	3.07834135608386\\
70.25	0.14058	3.80915364031237	3.80915364031237\\
70.25	0.14424	4.56550991189885	4.56550991189885\\
70.25	0.1479	5.34741017084331	5.34741017084331\\
70.25	0.15156	6.15485441714575	6.15485441714575\\
70.25	0.15522	6.98784265080618	6.98784265080618\\
70.25	0.15888	7.84637487182459	7.84637487182459\\
70.25	0.16254	8.73045108020098	8.73045108020098\\
70.25	0.1662	9.64007127593533	9.64007127593533\\
70.25	0.16986	10.5752354590277	10.5752354590277\\
70.25	0.17352	11.535943629478	11.535943629478\\
70.25	0.17718	12.5221957872863	12.5221957872863\\
70.25	0.18084	13.5339919324526	13.5339919324526\\
70.25	0.1845	14.5713320649768	14.5713320649768\\
70.25	0.18816	15.6342161848591	15.6342161848591\\
70.25	0.19182	16.7226442920993	16.7226442920993\\
70.25	0.19548	17.8366163866975	17.8366163866975\\
70.25	0.19914	18.9761324686537	18.9761324686537\\
70.25	0.2028	20.1411925379678	20.1411925379678\\
70.25	0.20646	21.33179659464	21.33179659464\\
70.25	0.21012	22.5479446386701	22.5479446386701\\
70.25	0.21378	23.7896366700582	23.7896366700582\\
70.25	0.21744	25.0568726888043	25.0568726888043\\
70.25	0.2211	26.3496526949083	26.3496526949083\\
70.25	0.22476	27.6679766883703	27.6679766883703\\
70.25	0.22842	29.0118446691903	29.0118446691903\\
70.25	0.23208	30.3812566373684	30.3812566373684\\
70.25	0.23574	31.7762125929043	31.7762125929043\\
70.25	0.2394	33.1967125357983	33.1967125357983\\
70.25	0.24306	34.6427564660502	34.6427564660502\\
70.25	0.24672	36.1143443836601	36.1143443836601\\
70.25	0.25038	37.611476288628	37.611476288628\\
70.25	0.25404	39.1341521809539	39.1341521809539\\
70.25	0.2577	40.6823720606377	40.6823720606377\\
70.25	0.26136	42.2561359276796	42.2561359276796\\
70.25	0.26502	43.8554437820794	43.8554437820794\\
70.25	0.26868	45.4802956238372	45.4802956238372\\
70.25	0.27234	47.1306914529529	47.1306914529529\\
70.25	0.276	48.8066312694267	48.8066312694267\\
70.625	0.093	-3.72975012906734	-3.72975012906734\\
70.625	0.09666	-3.2982372273726	-3.2982372273726\\
70.625	0.10032	-2.84118033831989	-2.84118033831989\\
70.625	0.10398	-2.35857946190918	-2.35857946190918\\
70.625	0.10764	-1.8504345981405	-1.8504345981405\\
70.625	0.1113	-1.31674574701383	-1.31674574701383\\
70.625	0.11496	-0.757512908529186	-0.757512908529186\\
70.625	0.11862	-0.172736082686571	-0.172736082686571\\
70.625	0.12228	0.437584730514034	0.437584730514034\\
70.625	0.12594	1.0734495310726	1.0734495310726\\
70.625	0.1296	1.73485831898916	1.73485831898916\\
70.625	0.13326	2.42181109426371	2.42181109426371\\
70.625	0.13692	3.13430785689622	3.13430785689622\\
70.625	0.14058	3.87234860688673	3.87234860688673\\
70.625	0.14424	4.63593334423519	4.63593334423519\\
70.625	0.1479	5.42506206894166	5.42506206894166\\
70.625	0.15156	6.2397347810061	6.2397347810061\\
70.625	0.15522	7.07995148042852	7.07995148042852\\
70.625	0.15888	7.94571216720892	7.94571216720892\\
70.625	0.16254	8.8370168413473	8.8370168413473\\
70.625	0.1662	9.75386550284366	9.75386550284366\\
70.625	0.16986	10.696258151698	10.696258151698\\
70.625	0.17352	11.6641947879103	11.6641947879103\\
70.625	0.17718	12.6576754114806	12.6576754114806\\
70.625	0.18084	13.6767000224089	13.6767000224089\\
70.625	0.1845	14.7212686206951	14.7212686206951\\
70.625	0.18816	15.7913812063394	15.7913812063394\\
70.625	0.19182	16.8870377793416	16.8870377793416\\
70.625	0.19548	18.0082383397018	18.0082383397018\\
70.625	0.19914	19.1549828874199	19.1549828874199\\
70.625	0.2028	20.3272714224961	20.3272714224961\\
70.625	0.20646	21.5251039449302	21.5251039449302\\
70.625	0.21012	22.7484804547223	22.7484804547223\\
70.625	0.21378	23.9974009518724	23.9974009518724\\
70.625	0.21744	25.2718654363805	25.2718654363805\\
70.625	0.2211	26.5718739082466	26.5718739082466\\
70.625	0.22476	27.8974263674706	27.8974263674706\\
70.625	0.22842	29.2485228140526	29.2485228140526\\
70.625	0.23208	30.6251632479926	30.6251632479926\\
70.625	0.23574	32.0273476692906	32.0273476692906\\
70.625	0.2394	33.4550760779465	33.4550760779465\\
70.625	0.24306	34.9083484739604	34.9083484739604\\
70.625	0.24672	36.3871648573323	36.3871648573323\\
70.625	0.25038	37.8915252280622	37.8915252280622\\
70.625	0.25404	39.4214295861501	39.4214295861501\\
70.625	0.2577	40.976877931596	40.976877931596\\
70.625	0.26136	42.5578702643997	42.5578702643997\\
70.625	0.26502	44.1644065845616	44.1644065845616\\
70.625	0.26868	45.7964868920813	45.7964868920813\\
70.625	0.27234	47.4541111869591	47.4541111869591\\
70.625	0.276	49.1372794691948	49.1372794691948\\
71	0.093	-3.75815654442175	-3.75815654442175\\
71	0.09666	-3.319415176965	-3.319415176965\\
71	0.10032	-2.8551298221503	-2.8551298221503\\
71	0.10398	-2.3653004799776	-2.3653004799776\\
71	0.10764	-1.84992715044692	-1.84992715044692\\
71	0.1113	-1.30900983355825	-1.30900983355825\\
71	0.11496	-0.742548529311613	-0.742548529311613\\
71	0.11862	-0.150543237707014	-0.150543237707014\\
71	0.12228	0.467006041255589	0.467006041255589\\
71	0.12594	1.11009930757615	1.11009930757615\\
71	0.1296	1.77873656125471	1.77873656125471\\
71	0.13326	2.47291780229125	2.47291780229125\\
71	0.13692	3.19264303068576	3.19264303068576\\
71	0.14058	3.93791224643826	3.93791224643826\\
71	0.14424	4.70872544954873	4.70872544954873\\
71	0.1479	5.50508264001718	5.50508264001718\\
71	0.15156	6.32698381784362	6.32698381784362\\
71	0.15522	7.17442898302803	7.17442898302803\\
71	0.15888	8.04741813557043	8.04741813557043\\
71	0.16254	8.9459512754708	8.9459512754708\\
71	0.1662	9.87002840272915	9.87002840272915\\
71	0.16986	10.8196495173455	10.8196495173455\\
71	0.17352	11.7948146193198	11.7948146193198\\
71	0.17718	12.7955237086521	12.7955237086521\\
71	0.18084	13.8217767853424	13.8217767853424\\
71	0.1845	14.8735738493906	14.8735738493906\\
71	0.18816	15.9509149007968	15.9509149007968\\
71	0.19182	17.053799939561	17.053799939561\\
71	0.19548	18.1822289656832	18.1822289656832\\
71	0.19914	19.3362019791634	19.3362019791634\\
71	0.2028	20.5157189800015	20.5157189800015\\
71	0.20646	21.7207799681977	21.7207799681977\\
71	0.21012	22.9513849437518	22.9513849437518\\
71	0.21378	24.2075339066639	24.2075339066639\\
71	0.21744	25.4892268569339	25.4892268569339\\
71	0.2211	26.7964637945619	26.7964637945619\\
71	0.22476	28.129244719548	28.129244719548\\
71	0.22842	29.487569631892	29.487569631892\\
71	0.23208	30.871438531594	30.871438531594\\
71	0.23574	32.280851418654	32.280851418654\\
71	0.2394	33.7158082930719	33.7158082930719\\
71	0.24306	35.1763091548478	35.1763091548478\\
71	0.24672	36.6623540039817	36.6623540039817\\
71	0.25038	38.1739428404736	38.1739428404736\\
71	0.25404	39.7110756643235	39.7110756643235\\
71	0.2577	41.2737524755313	41.2737524755313\\
71	0.26136	42.8619732740971	42.8619732740971\\
71	0.26502	44.4757380600209	44.4757380600209\\
71	0.26868	46.1150468333027	46.1150468333027\\
71	0.27234	47.7798995939425	47.7798995939425\\
71	0.276	49.4702963419402	49.4702963419402\\
71.375	0.093	-3.78419428679894	-3.78419428679894\\
71.375	0.09666	-3.33822445358021	-3.33822445358021\\
71.375	0.10032	-2.86671063300352	-2.86671063300352\\
71.375	0.10398	-2.36965282506882	-2.36965282506882\\
71.375	0.10764	-1.84705102977614	-1.84705102977614\\
71.375	0.1113	-1.29890524712548	-1.29890524712548\\
71.375	0.11496	-0.725215477116848	-0.725215477116848\\
71.375	0.11862	-0.125981719750252	-0.125981719750252\\
71.375	0.12228	0.498796024974348	0.498796024974348\\
71.375	0.12594	1.1491177570569	1.1491177570569\\
71.375	0.1296	1.82498347649745	1.82498347649745\\
71.375	0.13326	2.52639318329599	2.52639318329599\\
71.375	0.13692	3.2533468774525	3.2533468774525\\
71.375	0.14058	4.005844558967	4.005844558967\\
71.375	0.14424	4.78388622783945	4.78388622783945\\
71.375	0.1479	5.5874718840699	5.5874718840699\\
71.375	0.15156	6.41660152765833	6.41660152765833\\
71.375	0.15522	7.27127515860474	7.27127515860474\\
71.375	0.15888	8.15149277690914	8.15149277690914\\
71.375	0.16254	9.0572543825715	9.0572543825715\\
71.375	0.1662	9.98855997559185	9.98855997559185\\
71.375	0.16986	10.9454095559702	10.9454095559702\\
71.375	0.17352	11.9278031237065	11.9278031237065\\
71.375	0.17718	12.9357406788008	12.9357406788008\\
71.375	0.18084	13.969222221253	13.969222221253\\
71.375	0.1845	15.0282477510633	15.0282477510633\\
71.375	0.18816	16.1128172682315	16.1128172682315\\
71.375	0.19182	17.2229307727577	17.2229307727577\\
71.375	0.19548	18.3585882646419	18.3585882646419\\
71.375	0.19914	19.519789743884	19.519789743884\\
71.375	0.2028	20.7065352104842	20.7065352104842\\
71.375	0.20646	21.9188246644423	21.9188246644423\\
71.375	0.21012	23.1566581057584	23.1566581057584\\
71.375	0.21378	24.4200355344325	24.4200355344325\\
71.375	0.21744	25.7089569504646	25.7089569504646\\
71.375	0.2211	27.0234223538546	27.0234223538546\\
71.375	0.22476	28.3634317446026	28.3634317446026\\
71.375	0.22842	29.7289851227086	29.7289851227086\\
71.375	0.23208	31.1200824881726	31.1200824881726\\
71.375	0.23574	32.5367238409946	32.5367238409946\\
71.375	0.2394	33.9789091811745	33.9789091811745\\
71.375	0.24306	35.4466385087124	35.4466385087124\\
71.375	0.24672	36.9399118236083	36.9399118236083\\
71.375	0.25038	38.4587291258622	38.4587291258622\\
71.375	0.25404	40.003090415474	40.003090415474\\
71.375	0.2577	41.5729956924439	41.5729956924439\\
71.375	0.26136	43.1684449567717	43.1684449567717\\
71.375	0.26502	44.7894382084575	44.7894382084575\\
71.375	0.26868	46.4359754475012	46.4359754475012\\
71.375	0.27234	48.108056673903	48.108056673903\\
71.375	0.276	49.8056818876627	49.8056818876627\\
71.75	0.093	-3.80786335619897	-3.80786335619897\\
71.75	0.09666	-3.35466505721824	-3.35466505721824\\
71.75	0.10032	-2.87592277087955	-2.87592277087955\\
71.75	0.10398	-2.37163649718286	-2.37163649718286\\
71.75	0.10764	-1.84180623612819	-1.84180623612819\\
71.75	0.1113	-1.28643198771554	-1.28643198771554\\
71.75	0.11496	-0.705513751944906	-0.705513751944906\\
71.75	0.11862	-0.0990515288163127	-0.0990515288163127\\
71.75	0.12228	0.532954681670278	0.532954681670278\\
71.75	0.12594	1.19050487951483	1.19050487951483\\
71.75	0.1296	1.87359906471738	1.87359906471738\\
71.75	0.13326	2.5822372372779	2.5822372372779\\
71.75	0.13692	3.31641939719641	3.31641939719641\\
71.75	0.14058	4.0761455444729	4.0761455444729\\
71.75	0.14424	4.86141567910736	4.86141567910736\\
71.75	0.1479	5.6722298010998	5.6722298010998\\
71.75	0.15156	6.50858791045022	6.50858791045022\\
71.75	0.15522	7.37049000715863	7.37049000715863\\
71.75	0.15888	8.25793609122502	8.25793609122502\\
71.75	0.16254	9.17092616264938	9.17092616264938\\
71.75	0.1662	10.1094602214317	10.1094602214317\\
71.75	0.16986	11.073538267572	11.073538267572\\
71.75	0.17352	12.0631603010703	12.0631603010703\\
71.75	0.17718	13.0783263219266	13.0783263219266\\
71.75	0.18084	14.1190363301409	14.1190363301409\\
71.75	0.1845	15.1852903257131	15.1852903257131\\
71.75	0.18816	16.2770883086433	16.2770883086433\\
71.75	0.19182	17.3944302789315	17.3944302789315\\
71.75	0.19548	18.5373162365777	18.5373162365777\\
71.75	0.19914	19.7057461815819	19.7057461815819\\
71.75	0.2028	20.899720113944	20.899720113944\\
71.75	0.20646	22.1192380336641	22.1192380336641\\
71.75	0.21012	23.3642999407422	23.3642999407422\\
71.75	0.21378	24.6349058351783	24.6349058351783\\
71.75	0.21744	25.9310557169724	25.9310557169724\\
71.75	0.2211	27.2527495861244	27.2527495861244\\
71.75	0.22476	28.5999874426344	28.5999874426344\\
71.75	0.22842	29.9727692865024	29.9727692865024\\
71.75	0.23208	31.3710951177284	31.3710951177284\\
71.75	0.23574	32.7949649363123	32.7949649363123\\
71.75	0.2394	34.2443787422543	34.2443787422543\\
71.75	0.24306	35.7193365355542	35.7193365355542\\
71.75	0.24672	37.219838316212	37.219838316212\\
71.75	0.25038	38.745884084228	38.745884084228\\
71.75	0.25404	40.2974738396018	40.2974738396018\\
71.75	0.2577	41.8746075823336	41.8746075823336\\
71.75	0.26136	43.4772853124234	43.4772853124234\\
71.75	0.26502	45.1055070298712	45.1055070298712\\
71.75	0.26868	46.759272734677	46.759272734677\\
71.75	0.27234	48.4385824268407	48.4385824268407\\
71.75	0.276	50.1434361063625	50.1434361063625\\
72.125	0.093	-3.8291637526218	-3.8291637526218\\
72.125	0.09666	-3.36873698787907	-3.36873698787907\\
72.125	0.10032	-2.88276623577839	-2.88276623577839\\
72.125	0.10398	-2.3712514963197	-2.3712514963197\\
72.125	0.10764	-1.83419276950304	-1.83419276950304\\
72.125	0.1113	-1.27159005532839	-1.27159005532839\\
72.125	0.11496	-0.683443353795759	-0.683443353795759\\
72.125	0.11862	-0.0697526649051756	-0.0697526649051756\\
72.125	0.12228	0.569482011343412	0.569482011343412\\
72.125	0.12594	1.23426067494996	1.23426067494996\\
72.125	0.1296	1.9245833259145	1.9245833259145\\
72.125	0.13326	2.64044996423702	2.64044996423702\\
72.125	0.13692	3.38186058991752	3.38186058991752\\
72.125	0.14058	4.148815202956	4.148815202956\\
72.125	0.14424	4.94131380335245	4.94131380335245\\
72.125	0.1479	5.7593563911069	5.7593563911069\\
72.125	0.15156	6.60294296621931	6.60294296621931\\
72.125	0.15522	7.47207352868972	7.47207352868972\\
72.125	0.15888	8.3667480785181	8.3667480785181\\
72.125	0.16254	9.28696661570446	9.28696661570446\\
72.125	0.1662	10.2327291402488	10.2327291402488\\
72.125	0.16986	11.2040356521511	11.2040356521511\\
72.125	0.17352	12.2008861514114	12.2008861514114\\
72.125	0.17718	13.2232806380297	13.2232806380297\\
72.125	0.18084	14.2712191120059	14.2712191120059\\
72.125	0.1845	15.3447015733402	15.3447015733402\\
72.125	0.18816	16.4437280220324	16.4437280220324\\
72.125	0.19182	17.5682984580826	17.5682984580826\\
72.125	0.19548	18.7184128814907	18.7184128814907\\
72.125	0.19914	19.8940712922569	19.8940712922569\\
72.125	0.2028	21.0952736903811	21.0952736903811\\
72.125	0.20646	22.3220200758631	22.3220200758631\\
72.125	0.21012	23.5743104487032	23.5743104487032\\
72.125	0.21378	24.8521448089013	24.8521448089013\\
72.125	0.21744	26.1555231564574	26.1555231564574\\
72.125	0.2211	27.4844454913714	27.4844454913714\\
72.125	0.22476	28.8389118136434	28.8389118136434\\
72.125	0.22842	30.2189221232734	30.2189221232734\\
72.125	0.23208	31.6244764202613	31.6244764202613\\
72.125	0.23574	33.0555747046073	33.0555747046073\\
72.125	0.2394	34.5122169763112	34.5122169763112\\
72.125	0.24306	35.9944032353731	35.9944032353731\\
72.125	0.24672	37.502133481793	37.502133481793\\
72.125	0.25038	39.0354077155709	39.0354077155709\\
72.125	0.25404	40.5942259367067	40.5942259367067\\
72.125	0.2577	42.1785881452006	42.1785881452006\\
72.125	0.26136	43.7884943410523	43.7884943410523\\
72.125	0.26502	45.4239445242622	45.4239445242622\\
72.125	0.26868	47.0849386948299	47.0849386948299\\
72.125	0.27234	48.7714768527557	48.7714768527557\\
72.125	0.276	50.4835589980394	50.4835589980394\\
72.5	0.093	-3.84809547606747	-3.84809547606747\\
72.5	0.09666	-3.38044024556275	-3.38044024556275\\
72.5	0.10032	-2.88724102770004	-2.88724102770004\\
72.5	0.10398	-2.36849782247937	-2.36849782247937\\
72.5	0.10764	-1.82421062990072	-1.82421062990072\\
72.5	0.1113	-1.25437944996408	-1.25437944996408\\
72.5	0.11496	-0.659004282669464	-0.659004282669464\\
72.5	0.11862	-0.0380851280168688	-0.0380851280168688\\
72.5	0.12228	0.608378013993709	0.608378013993709\\
72.5	0.12594	1.28038514336227	1.28038514336227\\
72.5	0.1296	1.97793626008881	1.97793626008881\\
72.5	0.13326	2.70103136417331	2.70103136417331\\
72.5	0.13692	3.4496704556158	3.4496704556158\\
72.5	0.14058	4.22385353441628	4.22385353441628\\
72.5	0.14424	5.02358060057474	5.02358060057474\\
72.5	0.1479	5.84885165409116	5.84885165409116\\
72.5	0.15156	6.69966669496558	6.69966669496558\\
72.5	0.15522	7.57602572319797	7.57602572319797\\
72.5	0.15888	8.47792873878835	8.47792873878835\\
72.5	0.16254	9.4053757417367	9.4053757417367\\
72.5	0.1662	10.358366732043	10.358366732043\\
72.5	0.16986	11.3369017097073	11.3369017097073\\
72.5	0.17352	12.3409806747296	12.3409806747296\\
72.5	0.17718	13.3706036271099	13.3706036271099\\
72.5	0.18084	14.4257705668482	14.4257705668482\\
72.5	0.1845	15.5064814939444	15.5064814939444\\
72.5	0.18816	16.6127364083986	16.6127364083986\\
72.5	0.19182	17.7445353102108	17.7445353102108\\
72.5	0.19548	18.9018781993809	18.9018781993809\\
72.5	0.19914	20.0847650759091	20.0847650759091\\
72.5	0.2028	21.2931959397952	21.2931959397952\\
72.5	0.20646	22.5271707910393	22.5271707910393\\
72.5	0.21012	23.7866896296414	23.7866896296414\\
72.5	0.21378	25.0717524556015	25.0717524556015\\
72.5	0.21744	26.3823592689195	26.3823592689195\\
72.5	0.2211	27.7185100695955	27.7185100695955\\
72.5	0.22476	29.0802048576295	29.0802048576295\\
72.5	0.22842	30.4674436330215	30.4674436330215\\
72.5	0.23208	31.8802263957715	31.8802263957715\\
72.5	0.23574	33.3185531458795	33.3185531458795\\
72.5	0.2394	34.7824238833454	34.7824238833454\\
72.5	0.24306	36.2718386081693	36.2718386081693\\
72.5	0.24672	37.7867973203512	37.7867973203512\\
72.5	0.25038	39.327300019891	39.327300019891\\
72.5	0.25404	40.8933467067889	40.8933467067889\\
72.5	0.2577	42.4849373810447	42.4849373810447\\
72.5	0.26136	44.1020720426585	44.1020720426585\\
72.5	0.26502	45.7447506916303	45.7447506916303\\
72.5	0.26868	47.41297332796	47.41297332796\\
72.5	0.27234	49.1067399516478	49.1067399516478\\
72.5	0.276	50.8260505626934	50.8260505626934\\
72.875	0.093	-3.86465852653593	-3.86465852653593\\
72.875	0.09666	-3.38977483026921	-3.38977483026921\\
72.875	0.10032	-2.88934714664451	-2.88934714664451\\
72.875	0.10398	-2.36337547566183	-2.36337547566183\\
72.875	0.10764	-1.8118598173212	-1.8118598173212\\
72.875	0.1113	-1.23480017162256	-1.23480017162256\\
72.875	0.11496	-0.63219653856595	-0.63219653856595\\
72.875	0.11862	-0.00404891815135677	-0.00404891815135677\\
72.875	0.12228	0.649642689621219	0.649642689621219\\
72.875	0.12594	1.32887828475177	1.32887828475177\\
72.875	0.1296	2.0336578672403	2.0336578672403\\
72.875	0.13326	2.7639814370868	2.7639814370868\\
72.875	0.13692	3.51984899429129	3.51984899429129\\
72.875	0.14058	4.30126053885376	4.30126053885376\\
72.875	0.14424	5.10821607077422	5.10821607077422\\
72.875	0.1479	5.94071559005263	5.94071559005263\\
72.875	0.15156	6.79875909668904	6.79875909668904\\
72.875	0.15522	7.68234659068343	7.68234659068343\\
72.875	0.15888	8.59147807203581	8.59147807203581\\
72.875	0.16254	9.52615354074615	9.52615354074615\\
72.875	0.1662	10.4863729968145	10.4863729968145\\
72.875	0.16986	11.4721364402408	11.4721364402408\\
72.875	0.17352	12.4834438710251	12.4834438710251\\
72.875	0.17718	13.5202952891673	13.5202952891673\\
72.875	0.18084	14.5826906946676	14.5826906946676\\
72.875	0.1845	15.6706300875258	15.6706300875258\\
72.875	0.18816	16.784113467742	16.784113467742\\
72.875	0.19182	17.9231408353162	17.9231408353162\\
72.875	0.19548	19.0877121902484	19.0877121902484\\
72.875	0.19914	20.2778275325385	20.2778275325385\\
72.875	0.2028	21.4934868621866	21.4934868621866\\
72.875	0.20646	22.7346901791927	22.7346901791927\\
72.875	0.21012	24.0014374835568	24.0014374835568\\
72.875	0.21378	25.2937287752789	25.2937287752789\\
72.875	0.21744	26.6115640543589	26.6115640543589\\
72.875	0.2211	27.9549433207969	27.9549433207969\\
72.875	0.22476	29.3238665745929	29.3238665745929\\
72.875	0.22842	30.7183338157469	30.7183338157469\\
72.875	0.23208	32.1383450442589	32.1383450442589\\
72.875	0.23574	33.5839002601288	33.5839002601288\\
72.875	0.2394	35.0549994633567	35.0549994633567\\
72.875	0.24306	36.5516426539426	36.5516426539426\\
72.875	0.24672	38.0738298318865	38.0738298318865\\
72.875	0.25038	39.6215609971884	39.6215609971884\\
72.875	0.25404	41.1948361498482	41.1948361498482\\
72.875	0.2577	42.793655289866	42.793655289866\\
72.875	0.26136	44.4180184172418	44.4180184172418\\
72.875	0.26502	46.0679255319756	46.0679255319756\\
72.875	0.26868	47.7433766340673	47.7433766340673\\
72.875	0.27234	49.4443717235171	49.4443717235171\\
72.875	0.276	51.1709108003248	51.1709108003248\\
73.25	0.093	-3.8788529040272	-3.8788529040272\\
73.25	0.09666	-3.39674074199849	-3.39674074199849\\
73.25	0.10032	-2.88908459261179	-2.88908459261179\\
73.25	0.10398	-2.35588445586712	-2.35588445586712\\
73.25	0.10764	-1.79714033176448	-1.79714033176448\\
73.25	0.1113	-1.21285222030386	-1.21285222030386\\
73.25	0.11496	-0.603020121485251	-0.603020121485251\\
73.25	0.11862	0.0323559646913392	0.0323559646913392\\
73.25	0.12228	0.693276038225912	0.693276038225912\\
73.25	0.12594	1.37974009911846	1.37974009911846\\
73.25	0.1296	2.09174814736899	2.09174814736899\\
73.25	0.13326	2.82930018297747	2.82930018297747\\
73.25	0.13692	3.59239620594396	3.59239620594396\\
73.25	0.14058	4.38103621626843	4.38103621626843\\
73.25	0.14424	5.19522021395087	5.19522021395087\\
73.25	0.1479	6.03494819899129	6.03494819899129\\
73.25	0.15156	6.90022017138969	6.90022017138969\\
73.25	0.15522	7.79103613114608	7.79103613114608\\
73.25	0.15888	8.70739607826044	8.70739607826044\\
73.25	0.16254	9.64930001273279	9.64930001273279\\
73.25	0.1662	10.6167479345631	10.6167479345631\\
73.25	0.16986	11.6097398437514	11.6097398437514\\
73.25	0.17352	12.6282757402977	12.6282757402977\\
73.25	0.17718	13.6723556242019	13.6723556242019\\
73.25	0.18084	14.7419794954642	14.7419794954642\\
73.25	0.1845	15.8371473540844	15.8371473540844\\
73.25	0.18816	16.9578592000626	16.9578592000626\\
73.25	0.19182	18.1041150333988	18.1041150333988\\
73.25	0.19548	19.2759148540929	19.2759148540929\\
73.25	0.19914	20.4732586621451	20.4732586621451\\
73.25	0.2028	21.6961464575552	21.6961464575552\\
73.25	0.20646	22.9445782403233	22.9445782403233\\
73.25	0.21012	24.2185540104494	24.2185540104494\\
73.25	0.21378	25.5180737679334	25.5180737679334\\
73.25	0.21744	26.8431375127755	26.8431375127755\\
73.25	0.2211	28.1937452449755	28.1937452449755\\
73.25	0.22476	29.5698969645335	29.5698969645335\\
73.25	0.22842	30.9715926714494	30.9715926714494\\
73.25	0.23208	32.3988323657234	32.3988323657234\\
73.25	0.23574	33.8516160473553	33.8516160473553\\
73.25	0.2394	35.3299437163452	35.3299437163452\\
73.25	0.24306	36.8338153726931	36.8338153726931\\
73.25	0.24672	38.363231016399	38.363231016399\\
73.25	0.25038	39.9181906474629	39.9181906474629\\
73.25	0.25404	41.4986942658847	41.4986942658847\\
73.25	0.2577	43.1047418716645	43.1047418716645\\
73.25	0.26136	44.7363334648023	44.7363334648023\\
73.25	0.26502	46.3934690452981	46.3934690452981\\
73.25	0.26868	48.0761486131518	48.0761486131518\\
73.25	0.27234	49.7843721683635	49.7843721683635\\
73.25	0.276	51.5181397109332	51.5181397109332\\
73.625	0.093	-3.89067860854127	-3.89067860854127\\
73.625	0.09666	-3.40133798075057	-3.40133798075057\\
73.625	0.10032	-2.88645336560188	-2.88645336560188\\
73.625	0.10398	-2.34602476309521	-2.34602476309521\\
73.625	0.10764	-1.78005217323058	-1.78005217323058\\
73.625	0.1113	-1.18853559600796	-1.18853559600796\\
73.625	0.11496	-0.571475031427354	-0.571475031427354\\
73.625	0.11862	0.0711295205112261	0.0711295205112261\\
73.625	0.12228	0.739278059807789	0.739278059807789\\
73.625	0.12594	1.43297058646233	1.43297058646233\\
73.625	0.1296	2.15220710047486	2.15220710047486\\
73.625	0.13326	2.89698760184534	2.89698760184534\\
73.625	0.13692	3.66731209057382	3.66731209057382\\
73.625	0.14058	4.46318056666029	4.46318056666029\\
73.625	0.14424	5.28459303010473	5.28459303010473\\
73.625	0.1479	6.13154948090713	6.13154948090713\\
73.625	0.15156	7.00404991906753	7.00404991906753\\
73.625	0.15522	7.90209434458592	7.90209434458592\\
73.625	0.15888	8.82568275746227	8.82568275746227\\
73.625	0.16254	9.77481515769661	9.77481515769661\\
73.625	0.1662	10.7494915452889	10.7494915452889\\
73.625	0.16986	11.7497119202392	11.7497119202392\\
73.625	0.17352	12.7754762825475	12.7754762825475\\
73.625	0.17718	13.8267846322138	13.8267846322138\\
73.625	0.18084	14.903636969238	14.903636969238\\
73.625	0.1845	16.0060332936202	16.0060332936202\\
73.625	0.18816	17.1339736053604	17.1339736053604\\
73.625	0.19182	18.2874579044585	18.2874579044585\\
73.625	0.19548	19.4664861909147	19.4664861909147\\
73.625	0.19914	20.6710584647288	20.6710584647288\\
73.625	0.2028	21.901174725901	21.901174725901\\
73.625	0.20646	23.1568349744311	23.1568349744311\\
73.625	0.21012	24.4380392103191	24.4380392103191\\
73.625	0.21378	25.7447874335652	25.7447874335652\\
73.625	0.21744	27.0770796441692	27.0770796441692\\
73.625	0.2211	28.4349158421312	28.4349158421312\\
73.625	0.22476	29.8182960274512	29.8182960274512\\
73.625	0.22842	31.2272202001292	31.2272202001292\\
73.625	0.23208	32.6616883601651	32.6616883601651\\
73.625	0.23574	34.121700507559	34.121700507559\\
73.625	0.2394	35.607256642311	35.607256642311\\
73.625	0.24306	37.1183567644208	37.1183567644208\\
73.625	0.24672	38.6550008738887	38.6550008738887\\
73.625	0.25038	40.2171889707146	40.2171889707146\\
73.625	0.25404	41.8049210548984	41.8049210548984\\
73.625	0.2577	43.4181971264402	43.4181971264402\\
73.625	0.26136	45.05701718534	45.05701718534\\
73.625	0.26502	46.7213812315977	46.7213812315977\\
73.625	0.26868	48.4112892652135	48.4112892652135\\
73.625	0.27234	50.1267412861872	50.1267412861872\\
73.625	0.276	51.8677372945189	51.8677372945189\\
74	0.093	-3.90013564007816	-3.90013564007816\\
74	0.09666	-3.40356654652546	-3.40356654652546\\
74	0.10032	-2.88145346561477	-2.88145346561477\\
74	0.10398	-2.33379639734611	-2.33379639734611\\
74	0.10764	-1.76059534171949	-1.76059534171949\\
74	0.1113	-1.16185029873487	-1.16185029873487\\
74	0.11496	-0.537561268392274	-0.537561268392274\\
74	0.11862	0.112271749308304	0.112271749308304\\
74	0.12228	0.787648754366865	0.787648754366865\\
74	0.12594	1.4885697467834	1.4885697467834\\
74	0.1296	2.21503472655791	2.21503472655791\\
74	0.13326	2.9670436936904	2.9670436936904\\
74	0.13692	3.74459664818087	3.74459664818087\\
74	0.14058	4.54769359002934	4.54769359002934\\
74	0.14424	5.37633451923577	5.37633451923577\\
74	0.1479	6.23051943580017	6.23051943580017\\
74	0.15156	7.11024833972256	7.11024833972256\\
74	0.15522	8.01552123100294	8.01552123100294\\
74	0.15888	8.9463381096413	8.9463381096413\\
74	0.16254	9.90269897563763	9.90269897563763\\
74	0.1662	10.8846038289919	10.8846038289919\\
74	0.16986	11.8920526697042	11.8920526697042\\
74	0.17352	12.9250454977745	12.9250454977745\\
74	0.17718	13.9835823132027	13.9835823132027\\
74	0.18084	15.067663115989	15.067663115989\\
74	0.1845	16.1772879061332	16.1772879061332\\
74	0.18816	17.3124566836354	17.3124566836354\\
74	0.19182	18.4731694484956	18.4731694484956\\
74	0.19548	19.6594262007137	19.6594262007137\\
74	0.19914	20.8712269402898	20.8712269402898\\
74	0.2028	22.1085716672239	22.1085716672239\\
74	0.20646	23.371460381516	23.371460381516\\
74	0.21012	24.6598930831661	24.6598930831661\\
74	0.21378	25.9738697721741	25.9738697721741\\
74	0.21744	27.3133904485402	27.3133904485402\\
74	0.2211	28.6784551122641	28.6784551122641\\
74	0.22476	30.0690637633461	30.0690637633461\\
74	0.22842	31.4852164017861	31.4852164017861\\
74	0.23208	32.926913027584	32.926913027584\\
74	0.23574	34.39415364074	34.39415364074\\
74	0.2394	35.8869382412539	35.8869382412539\\
74	0.24306	37.4052668291258	37.4052668291258\\
74	0.24672	38.9491394043556	38.9491394043556\\
74	0.25038	40.5185559669435	40.5185559669435\\
74	0.25404	42.1135165168893	42.1135165168893\\
74	0.2577	43.7340210541931	43.7340210541931\\
74	0.26136	45.3800695788549	45.3800695788549\\
74	0.26502	47.0516620908746	47.0516620908746\\
74	0.26868	48.7487985902524	48.7487985902524\\
74	0.27234	50.4714790769881	50.4714790769881\\
74	0.276	52.2197035510817	52.2197035510817\\
};
\end{axis}

\begin{axis}[%
width=6.159cm,
height=3.097cm,
at={(8.104cm,4.301cm)},
scale only axis,
xmin=56,
xmax=74,
tick align=outside,
xlabel style={font=\color{white!15!black}},
xlabel={$L_{cut}$},
ymin=0.093,
ymax=0.276,
ylabel style={font=\color{white!15!black}},
ylabel={$D_{rlx}$},
zmin=-1328.01949078619,
zmax=0,
zlabel style={font=\color{white!15!black}},
zlabel={$u(t)$},
view={-140}{50},
axis background/.style={fill=white},
xmajorgrids,
ymajorgrids,
zmajorgrids
]
\addplot3[only marks, mark=*, mark options={}, mark size=1.5000pt, color=mycolor1, fill=mycolor1] table[row sep=crcr]{%
x	y	z\\
74	0.123	-171.244880190644\\
72	0.113	-139.217460396701\\
61	0.095	-69.6051966848396\\
56	0.093	-73.6870961160989\\
};
\addplot3[only marks, mark=*, mark options={}, mark size=1.5000pt, color=mycolor2, fill=mycolor2] table[row sep=crcr]{%
x	y	z\\
67	0.276	-1273.01999820883\\
66	0.255	-1046.38133574319\\
62	0.209	-579.722288788632\\
57	0.193	-465.590859275979\\
};
\addplot3[only marks, mark=*, mark options={}, mark size=1.5000pt, color=black, fill=black] table[row sep=crcr]{%
x	y	z\\
69	0.104	-102.29914898412\\
};
\addplot3[only marks, mark=*, mark options={}, mark size=1.5000pt, color=black, fill=black] table[row sep=crcr]{%
x	y	z\\
64	0.23	-773.899941054326\\
};

\addplot3[%
surf,
fill opacity=0.7, shader=interp, colormap={mymap}{[1pt] rgb(0pt)=(1,0.905882,0); rgb(1pt)=(1,0.901964,0); rgb(2pt)=(1,0.898051,0); rgb(3pt)=(1,0.894144,0); rgb(4pt)=(1,0.890243,0); rgb(5pt)=(1,0.886349,0); rgb(6pt)=(1,0.88246,0); rgb(7pt)=(1,0.878577,0); rgb(8pt)=(1,0.8747,0); rgb(9pt)=(1,0.870829,0); rgb(10pt)=(1,0.866964,0); rgb(11pt)=(1,0.863106,0); rgb(12pt)=(1,0.859253,0); rgb(13pt)=(1,0.855406,0); rgb(14pt)=(1,0.851566,0); rgb(15pt)=(1,0.847732,0); rgb(16pt)=(1,0.843903,0); rgb(17pt)=(1,0.840081,0); rgb(18pt)=(1,0.836265,0); rgb(19pt)=(1,0.832455,0); rgb(20pt)=(1,0.828652,0); rgb(21pt)=(1,0.824854,0); rgb(22pt)=(1,0.821063,0); rgb(23pt)=(1,0.817278,0); rgb(24pt)=(1,0.8135,0); rgb(25pt)=(1,0.809727,0); rgb(26pt)=(1,0.805961,0); rgb(27pt)=(1,0.8022,0); rgb(28pt)=(1,0.798445,0); rgb(29pt)=(1,0.794696,0); rgb(30pt)=(1,0.790953,0); rgb(31pt)=(1,0.787215,0); rgb(32pt)=(1,0.783484,0); rgb(33pt)=(1,0.779758,0); rgb(34pt)=(1,0.776038,0); rgb(35pt)=(1,0.772324,0); rgb(36pt)=(1,0.768615,0); rgb(37pt)=(1,0.764913,0); rgb(38pt)=(1,0.761217,0); rgb(39pt)=(1,0.757527,0); rgb(40pt)=(1,0.753843,0); rgb(41pt)=(1,0.750165,0); rgb(42pt)=(1,0.746493,0); rgb(43pt)=(1,0.742827,0); rgb(44pt)=(1,0.739167,0); rgb(45pt)=(1,0.735514,0); rgb(46pt)=(1,0.731867,0); rgb(47pt)=(1,0.728226,0); rgb(48pt)=(1,0.724591,0); rgb(49pt)=(1,0.720963,0); rgb(50pt)=(1,0.717341,0); rgb(51pt)=(1,0.713725,0); rgb(52pt)=(0.999994,0.710077,0); rgb(53pt)=(0.999974,0.706363,0); rgb(54pt)=(0.999942,0.702592,0); rgb(55pt)=(0.999898,0.698775,0); rgb(56pt)=(0.999841,0.694921,0); rgb(57pt)=(0.999771,0.691039,0); rgb(58pt)=(0.99969,0.687139,0); rgb(59pt)=(0.999596,0.68323,0); rgb(60pt)=(0.99949,0.679323,0); rgb(61pt)=(0.999372,0.675427,0); rgb(62pt)=(0.999242,0.67155,0); rgb(63pt)=(0.9991,0.667704,0); rgb(64pt)=(0.998946,0.663897,0); rgb(65pt)=(0.998781,0.660138,0); rgb(66pt)=(0.998605,0.656439,0); rgb(67pt)=(0.998416,0.652807,0); rgb(68pt)=(0.998217,0.649253,0); rgb(69pt)=(0.998006,0.645786,0); rgb(70pt)=(0.997785,0.642416,0); rgb(71pt)=(0.997552,0.639152,0); rgb(72pt)=(0.997308,0.636004,0); rgb(73pt)=(0.997053,0.632982,0); rgb(74pt)=(0.996788,0.630095,0); rgb(75pt)=(0.996512,0.627352,0); rgb(76pt)=(0.996226,0.624763,0); rgb(77pt)=(0.995851,0.622329,0); rgb(78pt)=(0.99494,0.619997,0); rgb(79pt)=(0.99345,0.617753,0); rgb(80pt)=(0.991419,0.61559,0); rgb(81pt)=(0.988885,0.613503,0); rgb(82pt)=(0.985886,0.611486,0); rgb(83pt)=(0.98246,0.609532,0); rgb(84pt)=(0.978643,0.607636,0); rgb(85pt)=(0.974475,0.605791,0); rgb(86pt)=(0.969992,0.603992,0); rgb(87pt)=(0.965232,0.602233,0); rgb(88pt)=(0.960233,0.600507,0); rgb(89pt)=(0.955033,0.598808,0); rgb(90pt)=(0.949669,0.59713,0); rgb(91pt)=(0.94418,0.595468,0); rgb(92pt)=(0.938602,0.593815,0); rgb(93pt)=(0.932974,0.592166,0); rgb(94pt)=(0.927333,0.590513,0); rgb(95pt)=(0.921717,0.588852,0); rgb(96pt)=(0.916164,0.587176,0); rgb(97pt)=(0.910711,0.585479,0); rgb(98pt)=(0.905397,0.583755,0); rgb(99pt)=(0.900258,0.581999,0); rgb(100pt)=(0.895333,0.580203,0); rgb(101pt)=(0.890659,0.578362,0); rgb(102pt)=(0.886275,0.576471,0); rgb(103pt)=(0.882047,0.574545,0); rgb(104pt)=(0.877819,0.572608,0); rgb(105pt)=(0.873592,0.57066,0); rgb(106pt)=(0.869366,0.568701,0); rgb(107pt)=(0.865143,0.566733,0); rgb(108pt)=(0.860924,0.564756,0); rgb(109pt)=(0.856708,0.562771,0); rgb(110pt)=(0.852497,0.560778,0); rgb(111pt)=(0.848292,0.558779,0); rgb(112pt)=(0.844092,0.556774,0); rgb(113pt)=(0.8399,0.554763,0); rgb(114pt)=(0.835716,0.552749,0); rgb(115pt)=(0.831541,0.55073,0); rgb(116pt)=(0.827374,0.548709,0); rgb(117pt)=(0.823219,0.546686,0); rgb(118pt)=(0.819074,0.54466,0); rgb(119pt)=(0.81494,0.542635,0); rgb(120pt)=(0.81082,0.540609,0); rgb(121pt)=(0.806712,0.538584,0); rgb(122pt)=(0.802619,0.53656,0); rgb(123pt)=(0.798541,0.534539,0); rgb(124pt)=(0.794478,0.532521,0); rgb(125pt)=(0.790431,0.530506,0); rgb(126pt)=(0.786402,0.528496,0); rgb(127pt)=(0.782391,0.526491,0); rgb(128pt)=(0.77841,0.524489,0); rgb(129pt)=(0.774523,0.522478,0); rgb(130pt)=(0.770731,0.520455,0); rgb(131pt)=(0.767022,0.518424,0); rgb(132pt)=(0.763384,0.516385,0); rgb(133pt)=(0.759804,0.514339,0); rgb(134pt)=(0.756272,0.51229,0); rgb(135pt)=(0.752775,0.510237,0); rgb(136pt)=(0.749302,0.508182,0); rgb(137pt)=(0.74584,0.506128,0); rgb(138pt)=(0.742378,0.504075,0); rgb(139pt)=(0.738904,0.502025,0); rgb(140pt)=(0.735406,0.499979,0); rgb(141pt)=(0.731872,0.49794,0); rgb(142pt)=(0.72829,0.495909,0); rgb(143pt)=(0.724649,0.493887,0); rgb(144pt)=(0.720936,0.491875,0); rgb(145pt)=(0.71714,0.489876,0); rgb(146pt)=(0.713249,0.487891,0); rgb(147pt)=(0.709251,0.485921,0); rgb(148pt)=(0.705134,0.483968,0); rgb(149pt)=(0.700887,0.482033,0); rgb(150pt)=(0.696497,0.480118,0); rgb(151pt)=(0.691952,0.478225,0); rgb(152pt)=(0.687242,0.476355,0); rgb(153pt)=(0.682353,0.47451,0); rgb(154pt)=(0.677195,0.472696,0); rgb(155pt)=(0.6717,0.470916,0); rgb(156pt)=(0.665891,0.469169,0); rgb(157pt)=(0.659791,0.46745,0); rgb(158pt)=(0.653423,0.465756,0); rgb(159pt)=(0.64681,0.464084,0); rgb(160pt)=(0.639976,0.462432,0); rgb(161pt)=(0.632943,0.460795,0); rgb(162pt)=(0.625734,0.459171,0); rgb(163pt)=(0.618373,0.457556,0); rgb(164pt)=(0.610882,0.455948,0); rgb(165pt)=(0.603284,0.454343,0); rgb(166pt)=(0.595604,0.452737,0); rgb(167pt)=(0.587863,0.451129,0); rgb(168pt)=(0.580084,0.449514,0); rgb(169pt)=(0.572292,0.447889,0); rgb(170pt)=(0.564508,0.446252,0); rgb(171pt)=(0.556756,0.444599,0); rgb(172pt)=(0.549059,0.442927,0); rgb(173pt)=(0.54144,0.441232,0); rgb(174pt)=(0.533922,0.439512,0); rgb(175pt)=(0.526529,0.437764,0); rgb(176pt)=(0.519282,0.435983,0); rgb(177pt)=(0.512206,0.434168,0); rgb(178pt)=(0.505323,0.432315,0); rgb(179pt)=(0.498628,0.430422,3.92506e-06); rgb(180pt)=(0.491973,0.428504,3.49981e-05); rgb(181pt)=(0.485331,0.426562,9.63073e-05); rgb(182pt)=(0.478704,0.424596,0.000186979); rgb(183pt)=(0.472096,0.422609,0.000306141); rgb(184pt)=(0.465508,0.420599,0.00045292); rgb(185pt)=(0.458942,0.418567,0.000626441); rgb(186pt)=(0.452401,0.416515,0.000825833); rgb(187pt)=(0.445885,0.414441,0.00105022); rgb(188pt)=(0.439399,0.412348,0.00129873); rgb(189pt)=(0.432942,0.410234,0.00157049); rgb(190pt)=(0.426518,0.408102,0.00186463); rgb(191pt)=(0.420129,0.40595,0.00218028); rgb(192pt)=(0.413777,0.40378,0.00251655); rgb(193pt)=(0.407464,0.401592,0.00287258); rgb(194pt)=(0.401191,0.399386,0.00324749); rgb(195pt)=(0.394962,0.397164,0.00364042); rgb(196pt)=(0.388777,0.394925,0.00405048); rgb(197pt)=(0.38264,0.39267,0.00447681); rgb(198pt)=(0.376552,0.390399,0.00491852); rgb(199pt)=(0.370516,0.388113,0.00537476); rgb(200pt)=(0.364532,0.385812,0.00584464); rgb(201pt)=(0.358605,0.383497,0.00632729); rgb(202pt)=(0.352735,0.381168,0.00682184); rgb(203pt)=(0.346925,0.378826,0.00732741); rgb(204pt)=(0.341176,0.376471,0.00784314); rgb(205pt)=(0.335485,0.374093,0.00847245); rgb(206pt)=(0.329843,0.371682,0.00930909); rgb(207pt)=(0.324249,0.369242,0.0103377); rgb(208pt)=(0.318701,0.366772,0.0115428); rgb(209pt)=(0.313198,0.364275,0.0129091); rgb(210pt)=(0.307739,0.361753,0.0144211); rgb(211pt)=(0.302322,0.359206,0.0160634); rgb(212pt)=(0.296945,0.356637,0.0178207); rgb(213pt)=(0.291607,0.354048,0.0196776); rgb(214pt)=(0.286307,0.35144,0.0216186); rgb(215pt)=(0.281043,0.348814,0.0236284); rgb(216pt)=(0.275813,0.346172,0.0256916); rgb(217pt)=(0.270616,0.343517,0.0277927); rgb(218pt)=(0.265451,0.340849,0.0299163); rgb(219pt)=(0.260317,0.33817,0.0320472); rgb(220pt)=(0.25521,0.335482,0.0341698); rgb(221pt)=(0.250131,0.332786,0.0362688); rgb(222pt)=(0.245078,0.330085,0.0383287); rgb(223pt)=(0.240048,0.327379,0.0403343); rgb(224pt)=(0.235042,0.324671,0.04227); rgb(225pt)=(0.230056,0.321962,0.0441205); rgb(226pt)=(0.22509,0.319254,0.0458704); rgb(227pt)=(0.220142,0.316548,0.0475043); rgb(228pt)=(0.215212,0.313846,0.0490067); rgb(229pt)=(0.210296,0.311149,0.0503624); rgb(230pt)=(0.205395,0.308459,0.0515759); rgb(231pt)=(0.200514,0.305763,0.052757); rgb(232pt)=(0.195655,0.303061,0.0539242); rgb(233pt)=(0.190817,0.300353,0.0550763); rgb(234pt)=(0.186001,0.297639,0.0562123); rgb(235pt)=(0.181207,0.294918,0.0573313); rgb(236pt)=(0.176434,0.292191,0.0584321); rgb(237pt)=(0.171685,0.289458,0.0595136); rgb(238pt)=(0.166957,0.286719,0.060575); rgb(239pt)=(0.162252,0.283973,0.0616151); rgb(240pt)=(0.15757,0.281221,0.0626328); rgb(241pt)=(0.152911,0.278463,0.0636271); rgb(242pt)=(0.148275,0.275699,0.0645971); rgb(243pt)=(0.143663,0.272929,0.0655416); rgb(244pt)=(0.139074,0.270152,0.0664596); rgb(245pt)=(0.134508,0.26737,0.06735); rgb(246pt)=(0.129967,0.264581,0.0682118); rgb(247pt)=(0.125449,0.261787,0.0690441); rgb(248pt)=(0.120956,0.258986,0.0698456); rgb(249pt)=(0.116487,0.25618,0.0706154); rgb(250pt)=(0.112043,0.253367,0.0713525); rgb(251pt)=(0.107623,0.250549,0.0720557); rgb(252pt)=(0.103229,0.247724,0.0727241); rgb(253pt)=(0.0988592,0.244894,0.0733566); rgb(254pt)=(0.0945149,0.242058,0.0739522); rgb(255pt)=(0.0901961,0.239216,0.0745098)}, mesh/rows=49]
table[row sep=crcr, point meta=\thisrow{c}] {%
%
x	y	z	c\\
56	0.093	-73.0247303221439	-73.0247303221439\\
56	0.09666	-76.5488919207614	-76.5488919207614\\
56	0.10032	-80.8947938841944	-80.8947938841944\\
56	0.10398	-86.0624362124423	-86.0624362124423\\
56	0.10764	-92.0518189055054	-92.0518189055054\\
56	0.1113	-98.8629419633837	-98.8629419633837\\
56	0.11496	-106.495805386077	-106.495805386077\\
56	0.11862	-114.950409173586	-114.950409173586\\
56	0.12228	-124.22675332591	-124.22675332591\\
56	0.12594	-134.324837843049	-134.324837843049\\
56	0.1296	-145.244662725003	-145.244662725003\\
56	0.13326	-156.986227971772	-156.986227971772\\
56	0.13692	-169.549533583356	-169.549533583356\\
56	0.14058	-182.934579559756	-182.934579559756\\
56	0.14424	-197.141365900971	-197.141365900971\\
56	0.1479	-212.169892607001	-212.169892607001\\
56	0.15156	-228.020159677847	-228.020159677847\\
56	0.15522	-244.692167113507	-244.692167113507\\
56	0.15888	-262.185914913982	-262.185914913982\\
56	0.16254	-280.501403079273	-280.501403079273\\
56	0.1662	-299.638631609379	-299.638631609379\\
56	0.16986	-319.597600504299	-319.597600504299\\
56	0.17352	-340.378309764036	-340.378309764036\\
56	0.17718	-361.980759388587	-361.980759388587\\
56	0.18084	-384.404949377953	-384.404949377953\\
56	0.1845	-407.650879732135	-407.650879732135\\
56	0.18816	-431.718550451132	-431.718550451132\\
56	0.19182	-456.607961534944	-456.607961534944\\
56	0.19548	-482.319112983571	-482.319112983571\\
56	0.19914	-508.852004797013	-508.852004797013\\
56	0.2028	-536.206636975271	-536.206636975271\\
56	0.20646	-564.383009518343	-564.383009518343\\
56	0.21012	-593.381122426231	-593.381122426231\\
56	0.21378	-623.200975698934	-623.200975698934\\
56	0.21744	-653.842569336452	-653.842569336452\\
56	0.2211	-685.305903338785	-685.305903338785\\
56	0.22476	-717.590977705934	-717.590977705934\\
56	0.22842	-750.697792437897	-750.697792437897\\
56	0.23208	-784.626347534676	-784.626347534676\\
56	0.23574	-819.37664299627	-819.37664299627\\
56	0.2394	-854.948678822679	-854.948678822679\\
56	0.24306	-891.342455013903	-891.342455013903\\
56	0.24672	-928.557971569943	-928.557971569943\\
56	0.25038	-966.595228490798	-966.595228490798\\
56	0.25404	-1005.45422577647	-1005.45422577647\\
56	0.2577	-1045.13496342695	-1045.13496342695\\
56	0.26136	-1085.63744144225	-1085.63744144225\\
56	0.26502	-1126.96165982237	-1126.96165982237\\
56	0.26868	-1169.1076185673	-1169.1076185673\\
56	0.27234	-1212.07531767704	-1212.07531767704\\
56	0.276	-1255.8647571516	-1255.8647571516\\
56.375	0.093	-72.2605324617118	-72.2605324617118\\
56.375	0.09666	-75.7966567181423	-75.7966567181423\\
56.375	0.10032	-80.1545213393885	-80.1545213393885\\
56.375	0.10398	-85.3341263254496	-85.3341263254496\\
56.375	0.10764	-91.3354716763255	-91.3354716763255\\
56.375	0.1113	-98.1585573920171	-98.1585573920171\\
56.375	0.11496	-105.803383472524	-105.803383472524\\
56.375	0.11862	-114.269949917845	-114.269949917845\\
56.375	0.12228	-123.558256727983	-123.558256727983\\
56.375	0.12594	-133.668303902934	-133.668303902934\\
56.375	0.1296	-144.600091442702	-144.600091442702\\
56.375	0.13326	-156.353619347284	-156.353619347284\\
56.375	0.13692	-168.928887616681	-168.928887616681\\
56.375	0.14058	-182.325896250894	-182.325896250894\\
56.375	0.14424	-196.544645249922	-196.544645249922\\
56.375	0.1479	-211.585134613765	-211.585134613765\\
56.375	0.15156	-227.447364342424	-227.447364342424\\
56.375	0.15522	-244.131334435897	-244.131334435897\\
56.375	0.15888	-261.637044894186	-261.637044894186\\
56.375	0.16254	-279.964495717289	-279.964495717289\\
56.375	0.1662	-299.113686905208	-299.113686905208\\
56.375	0.16986	-319.084618457942	-319.084618457942\\
56.375	0.17352	-339.877290375491	-339.877290375491\\
56.375	0.17718	-361.491702657856	-361.491702657856\\
56.375	0.18084	-383.927855305035	-383.927855305035\\
56.375	0.1845	-407.18574831703	-407.18574831703\\
56.375	0.18816	-431.26538169384	-431.26538169384\\
56.375	0.19182	-456.166755435465	-456.166755435465\\
56.375	0.19548	-481.889869541905	-481.889869541905\\
56.375	0.19914	-508.43472401316	-508.43472401316\\
56.375	0.2028	-535.801318849231	-535.801318849231\\
56.375	0.20646	-563.989654050117	-563.989654050117\\
56.375	0.21012	-592.999729615818	-592.999729615818\\
56.375	0.21378	-622.831545546334	-622.831545546334\\
56.375	0.21744	-653.485101841665	-653.485101841665\\
56.375	0.2211	-684.960398501811	-684.960398501811\\
56.375	0.22476	-717.257435526772	-717.257435526772\\
56.375	0.22842	-750.376212916549	-750.376212916549\\
56.375	0.23208	-784.316730671141	-784.316730671141\\
56.375	0.23574	-819.078988790548	-819.078988790548\\
56.375	0.2394	-854.66298727477	-854.66298727477\\
56.375	0.24306	-891.068726123808	-891.068726123808\\
56.375	0.24672	-928.29620533766	-928.29620533766\\
56.375	0.25038	-966.345424916328	-966.345424916328\\
56.375	0.25404	-1005.21638485981	-1005.21638485981\\
56.375	0.2577	-1044.90908516811	-1044.90908516811\\
56.375	0.26136	-1085.42352584122	-1085.42352584122\\
56.375	0.26502	-1126.75970687915	-1126.75970687915\\
56.375	0.26868	-1168.91762828189	-1168.91762828189\\
56.375	0.27234	-1211.89729004945	-1211.89729004945\\
56.375	0.276	-1255.69869218183	-1255.69869218183\\
56.75	0.093	-71.5673681581807	-71.5673681581807\\
56.75	0.09666	-75.1154550724245	-75.1154550724245\\
56.75	0.10032	-79.4852823514834	-79.4852823514834\\
56.75	0.10398	-84.6768499953577	-84.6768499953577\\
56.75	0.10764	-90.6901580040471	-90.6901580040471\\
56.75	0.1113	-97.5252063775514	-97.5252063775514\\
56.75	0.11496	-105.181995115871	-105.181995115871\\
56.75	0.11862	-113.660524219006	-113.660524219006\\
56.75	0.12228	-122.960793686956	-122.960793686956\\
56.75	0.12594	-133.082803519721	-133.082803519721\\
56.75	0.1296	-144.026553717301	-144.026553717301\\
56.75	0.13326	-155.792044279697	-155.792044279697\\
56.75	0.13692	-168.379275206908	-168.379275206908\\
56.75	0.14058	-181.788246498933	-181.788246498933\\
56.75	0.14424	-196.018958155774	-196.018958155774\\
56.75	0.1479	-211.07141017743	-211.07141017743\\
56.75	0.15156	-226.945602563902	-226.945602563902\\
56.75	0.15522	-243.641535315188	-243.641535315188\\
56.75	0.15888	-261.15920843129	-261.15920843129\\
56.75	0.16254	-279.498621912207	-279.498621912207\\
56.75	0.1662	-298.659775757939	-298.659775757939\\
56.75	0.16986	-318.642669968486	-318.642669968486\\
56.75	0.17352	-339.447304543848	-339.447304543848\\
56.75	0.17718	-361.073679484026	-361.073679484026\\
56.75	0.18084	-383.521794789018	-383.521794789018\\
56.75	0.1845	-406.791650458826	-406.791650458826\\
56.75	0.18816	-430.883246493449	-430.883246493449\\
56.75	0.19182	-455.796582892887	-455.796582892887\\
56.75	0.19548	-481.53165965714	-481.53165965714\\
56.75	0.19914	-508.088476786209	-508.088476786209\\
56.75	0.2028	-535.467034280092	-535.467034280092\\
56.75	0.20646	-563.667332138791	-563.667332138791\\
56.75	0.21012	-592.689370362305	-592.689370362305\\
56.75	0.21378	-622.533148950634	-622.533148950634\\
56.75	0.21744	-653.198667903778	-653.198667903778\\
56.75	0.2211	-684.685927221738	-684.685927221738\\
56.75	0.22476	-716.994926904512	-716.994926904512\\
56.75	0.22842	-750.125666952102	-750.125666952102\\
56.75	0.23208	-784.078147364507	-784.078147364507\\
56.75	0.23574	-818.852368141727	-818.852368141727\\
56.75	0.2394	-854.448329283762	-854.448329283762\\
56.75	0.24306	-890.866030790613	-890.866030790613\\
56.75	0.24672	-928.105472662278	-928.105472662278\\
56.75	0.25038	-966.166654898759	-966.166654898759\\
56.75	0.25404	-1005.04957750006	-1005.04957750006\\
56.75	0.2577	-1044.75424046617	-1044.75424046617\\
56.75	0.26136	-1085.28064379709	-1085.28064379709\\
56.75	0.26502	-1126.62878749283	-1126.62878749283\\
56.75	0.26868	-1168.79867155339	-1168.79867155339\\
56.75	0.27234	-1211.79029597876	-1211.79029597876\\
56.75	0.276	-1255.60366076895	-1255.60366076895\\
57.125	0.093	-70.9452374115499	-70.9452374115499\\
57.125	0.09666	-74.505286983607	-74.505286983607\\
57.125	0.10032	-78.8870769204788	-78.8870769204788\\
57.125	0.10398	-84.0906072221662	-84.0906072221662\\
57.125	0.10764	-90.1158778886685	-90.1158778886685\\
57.125	0.1113	-96.9628889199861	-96.9628889199861\\
57.125	0.11496	-104.631640316119	-104.631640316119\\
57.125	0.11862	-113.122132077067	-113.122132077067\\
57.125	0.12228	-122.43436420283	-122.43436420283\\
57.125	0.12594	-132.568336693408	-132.568336693408\\
57.125	0.1296	-143.524049548801	-143.524049548801\\
57.125	0.13326	-155.30150276901	-155.30150276901\\
57.125	0.13692	-167.900696354034	-167.900696354034\\
57.125	0.14058	-181.321630303873	-181.321630303873\\
57.125	0.14424	-195.564304618527	-195.564304618527\\
57.125	0.1479	-210.628719297996	-210.628719297996\\
57.125	0.15156	-226.51487434228	-226.51487434228\\
57.125	0.15522	-243.22276975138	-243.22276975138\\
57.125	0.15888	-260.752405525295	-260.752405525295\\
57.125	0.16254	-279.103781664024	-279.103781664024\\
57.125	0.1662	-298.27689816757	-298.27689816757\\
57.125	0.16986	-318.27175503593	-318.27175503593\\
57.125	0.17352	-339.088352269105	-339.088352269105\\
57.125	0.17718	-360.726689867096	-360.726689867096\\
57.125	0.18084	-383.186767829901	-383.186767829901\\
57.125	0.1845	-406.468586157522	-406.468586157522\\
57.125	0.18816	-430.572144849958	-430.572144849958\\
57.125	0.19182	-455.497443907209	-455.497443907209\\
57.125	0.19548	-481.244483329276	-481.244483329276\\
57.125	0.19914	-507.813263116157	-507.813263116157\\
57.125	0.2028	-535.203783267854	-535.203783267854\\
57.125	0.20646	-563.416043784366	-563.416043784366\\
57.125	0.21012	-592.450044665693	-592.450044665693\\
57.125	0.21378	-622.305785911835	-622.305785911835\\
57.125	0.21744	-652.983267522792	-652.983267522792\\
57.125	0.2211	-684.482489498565	-684.482489498565\\
57.125	0.22476	-716.803451839152	-716.803451839152\\
57.125	0.22842	-749.946154544555	-749.946154544555\\
57.125	0.23208	-783.910597614773	-783.910597614773\\
57.125	0.23574	-818.696781049807	-818.696781049807\\
57.125	0.2394	-854.304704849655	-854.304704849655\\
57.125	0.24306	-890.734369014318	-890.734369014318\\
57.125	0.24672	-927.985773543797	-927.985773543797\\
57.125	0.25038	-966.058918438091	-966.058918438091\\
57.125	0.25404	-1004.9538036972	-1004.9538036972\\
57.125	0.2577	-1044.67042932112	-1044.67042932112\\
57.125	0.26136	-1085.20879530986	-1085.20879530986\\
57.125	0.26502	-1126.56890166342	-1126.56890166342\\
57.125	0.26868	-1168.75074838179	-1168.75074838179\\
57.125	0.27234	-1211.75433546497	-1211.75433546497\\
57.125	0.276	-1255.57966291297	-1255.57966291297\\
57.5	0.093	-70.3941402218201	-70.3941402218201\\
57.5	0.09666	-73.9661524516899	-73.9661524516899\\
57.5	0.10032	-78.3599050463752	-78.3599050463752\\
57.5	0.10398	-83.5753980058756	-83.5753980058756\\
57.5	0.10764	-89.6126313301909	-89.6126313301909\\
57.5	0.1113	-96.4716050193217	-96.4716050193217\\
57.5	0.11496	-104.152319073268	-104.152319073268\\
57.5	0.11862	-112.654773492028	-112.654773492028\\
57.5	0.12228	-121.978968275605	-121.978968275605\\
57.5	0.12594	-132.124903423996	-132.124903423996\\
57.5	0.1296	-143.092578937202	-143.092578937202\\
57.5	0.13326	-154.881994815224	-154.881994815224\\
57.5	0.13692	-167.493151058061	-167.493151058061\\
57.5	0.14058	-180.926047665713	-180.926047665713\\
57.5	0.14424	-195.18068463818	-195.18068463818\\
57.5	0.1479	-210.257061975462	-210.257061975462\\
57.5	0.15156	-226.15517967756	-226.15517967756\\
57.5	0.15522	-242.875037744472	-242.875037744472\\
57.5	0.15888	-260.4166361762	-260.4166361762\\
57.5	0.16254	-278.779974972743	-278.779974972743\\
57.5	0.1662	-297.965054134101	-297.965054134101\\
57.5	0.16986	-317.971873660274	-317.971873660274\\
57.5	0.17352	-338.800433551263	-338.800433551263\\
57.5	0.17718	-360.450733807066	-360.450733807066\\
57.5	0.18084	-382.922774427685	-382.922774427685\\
57.5	0.1845	-406.216555413119	-406.216555413119\\
57.5	0.18816	-430.332076763368	-430.332076763368\\
57.5	0.19182	-455.269338478433	-455.269338478433\\
57.5	0.19548	-481.028340558312	-481.028340558312\\
57.5	0.19914	-507.609083003007	-507.609083003007\\
57.5	0.2028	-535.011565812517	-535.011565812517\\
57.5	0.20646	-563.235788986841	-563.235788986841\\
57.5	0.21012	-592.281752525982	-592.281752525982\\
57.5	0.21378	-622.149456429937	-622.149456429937\\
57.5	0.21744	-652.838900698707	-652.838900698707\\
57.5	0.2211	-684.350085332293	-684.350085332293\\
57.5	0.22476	-716.683010330693	-716.683010330693\\
57.5	0.22842	-749.837675693909	-749.837675693909\\
57.5	0.23208	-783.81408142194	-783.81408142194\\
57.5	0.23574	-818.612227514787	-818.612227514787\\
57.5	0.2394	-854.232113972448	-854.232113972448\\
57.5	0.24306	-890.673740794925	-890.673740794925\\
57.5	0.24672	-927.937107982217	-927.937107982217\\
57.5	0.25038	-966.022215534324	-966.022215534324\\
57.5	0.25404	-1004.92906345125	-1004.92906345125\\
57.5	0.2577	-1044.65765173298	-1044.65765173298\\
57.5	0.26136	-1085.20798037954	-1085.20798037954\\
57.5	0.26502	-1126.5800493909	-1126.5800493909\\
57.5	0.26868	-1168.77385876709	-1168.77385876709\\
57.5	0.27234	-1211.78940850808	-1211.78940850808\\
57.5	0.276	-1255.6266986139	-1255.6266986139\\
57.875	0.093	-69.9140765889906	-69.9140765889906\\
57.875	0.09666	-73.4980514766737	-73.4980514766737\\
57.875	0.10032	-77.903766729172	-77.903766729172\\
57.875	0.10398	-83.1312223464853	-83.1312223464853\\
57.875	0.10764	-89.1804183286137	-89.1804183286137\\
57.875	0.1113	-96.0513546755577	-96.0513546755577\\
57.875	0.11496	-103.744031387317	-103.744031387317\\
57.875	0.11862	-112.25844846389	-112.25844846389\\
57.875	0.12228	-121.59460590528	-121.59460590528\\
57.875	0.12594	-131.752503711484	-131.752503711484\\
57.875	0.1296	-142.732141882504	-142.732141882504\\
57.875	0.13326	-154.533520418338	-154.533520418338\\
57.875	0.13692	-167.156639318988	-167.156639318988\\
57.875	0.14058	-180.601498584453	-180.601498584453\\
57.875	0.14424	-194.868098214734	-194.868098214734\\
57.875	0.1479	-209.956438209829	-209.956438209829\\
57.875	0.15156	-225.86651856974	-225.86651856974\\
57.875	0.15522	-242.598339294465	-242.598339294465\\
57.875	0.15888	-260.151900384006	-260.151900384006\\
57.875	0.16254	-278.527201838362	-278.527201838362\\
57.875	0.1662	-297.724243657534	-297.724243657534\\
57.875	0.16986	-317.74302584152	-317.74302584152\\
57.875	0.17352	-338.583548390321	-338.583548390321\\
57.875	0.17718	-360.245811303938	-360.245811303938\\
57.875	0.18084	-382.72981458237	-382.72981458237\\
57.875	0.1845	-406.035558225617	-406.035558225617\\
57.875	0.18816	-430.163042233679	-430.163042233679\\
57.875	0.19182	-455.112266606556	-455.112266606556\\
57.875	0.19548	-480.883231344249	-480.883231344249\\
57.875	0.19914	-507.475936446756	-507.475936446756\\
57.875	0.2028	-534.89038191408	-534.89038191408\\
57.875	0.20646	-563.126567746217	-563.126567746217\\
57.875	0.21012	-592.184493943171	-592.184493943171\\
57.875	0.21378	-622.064160504939	-622.064160504939\\
57.875	0.21744	-652.765567431522	-652.765567431522\\
57.875	0.2211	-684.288714722921	-684.288714722921\\
57.875	0.22476	-716.633602379135	-716.633602379135\\
57.875	0.22842	-749.800230400164	-749.800230400164\\
57.875	0.23208	-783.788598786008	-783.788598786008\\
57.875	0.23574	-818.598707536668	-818.598707536668\\
57.875	0.2394	-854.230556652142	-854.230556652142\\
57.875	0.24306	-890.684146132431	-890.684146132431\\
57.875	0.24672	-927.959475977536	-927.959475977536\\
57.875	0.25038	-966.056546187457	-966.056546187457\\
57.875	0.25404	-1004.97535676219	-1004.97535676219\\
57.875	0.2577	-1044.71590770174	-1044.71590770174\\
57.875	0.26136	-1085.27819900611	-1085.27819900611\\
57.875	0.26502	-1126.66223067529	-1126.66223067529\\
57.875	0.26868	-1168.86800270928	-1168.86800270928\\
57.875	0.27234	-1211.89551510809	-1211.89551510809\\
57.875	0.276	-1255.74476787172	-1255.74476787172\\
58.25	0.093	-69.505046513062	-69.505046513062\\
58.25	0.09666	-73.1009840585583	-73.1009840585583\\
58.25	0.10032	-77.5186619688696	-77.5186619688696\\
58.25	0.10398	-82.7580802439959	-82.7580802439959\\
58.25	0.10764	-88.8192388839378	-88.8192388839378\\
58.25	0.1113	-95.7021378886943	-95.7021378886943\\
58.25	0.11496	-103.406777258266	-103.406777258266\\
58.25	0.11862	-111.933156992654	-111.933156992654\\
58.25	0.12228	-121.281277091856	-121.281277091856\\
58.25	0.12594	-131.451137555874	-131.451137555874\\
58.25	0.1296	-142.442738384706	-142.442738384706\\
58.25	0.13326	-154.256079578354	-154.256079578354\\
58.25	0.13692	-166.891161136817	-166.891161136817\\
58.25	0.14058	-180.347983060095	-180.347983060095\\
58.25	0.14424	-194.626545348188	-194.626545348188\\
58.25	0.1479	-209.726848001097	-209.726848001097\\
58.25	0.15156	-225.64889101882	-225.64889101882\\
58.25	0.15522	-242.392674401359	-242.392674401359\\
58.25	0.15888	-259.958198148713	-259.958198148713\\
58.25	0.16254	-278.345462260882	-278.345462260882\\
58.25	0.1662	-297.554466737866	-297.554466737866\\
58.25	0.16986	-317.585211579666	-317.585211579666\\
58.25	0.17352	-338.43769678628	-338.43769678628\\
58.25	0.17718	-360.11192235771	-360.11192235771\\
58.25	0.18084	-382.607888293955	-382.607888293955\\
58.25	0.1845	-405.925594595015	-405.925594595015\\
58.25	0.18816	-430.06504126089	-430.06504126089\\
58.25	0.19182	-455.026228291581	-455.026228291581\\
58.25	0.19548	-480.809155687086	-480.809155687086\\
58.25	0.19914	-507.413823447407	-507.413823447407\\
58.25	0.2028	-534.840231572543	-534.840231572543\\
58.25	0.20646	-563.088380062494	-563.088380062494\\
58.25	0.21012	-592.158268917261	-592.158268917261\\
58.25	0.21378	-622.049898136842	-622.049898136842\\
58.25	0.21744	-652.763267721238	-652.763267721238\\
58.25	0.2211	-684.29837767045	-684.29837767045\\
58.25	0.22476	-716.655227984477	-716.655227984477\\
58.25	0.22842	-749.833818663319	-749.833818663319\\
58.25	0.23208	-783.834149706976	-783.834149706976\\
58.25	0.23574	-818.656221115449	-818.656221115449\\
58.25	0.2394	-854.300032888736	-854.300032888736\\
58.25	0.24306	-890.765585026839	-890.765585026839\\
58.25	0.24672	-928.052877529757	-928.052877529757\\
58.25	0.25038	-966.16191039749	-966.16191039749\\
58.25	0.25404	-1005.09268363004	-1005.09268363004\\
58.25	0.2577	-1044.8451972274	-1044.8451972274\\
58.25	0.26136	-1085.41945118958	-1085.41945118958\\
58.25	0.26502	-1126.81544551657	-1126.81544551657\\
58.25	0.26868	-1169.03318020838	-1169.03318020838\\
58.25	0.27234	-1212.07265526501	-1212.07265526501\\
58.25	0.276	-1255.93387068645	-1255.93387068645\\
58.625	0.093	-69.1670499940341	-69.1670499940341\\
58.625	0.09666	-72.7749501973432	-72.7749501973432\\
58.625	0.10032	-77.2045907654675	-77.2045907654675\\
58.625	0.10398	-82.455971698407	-82.455971698407\\
58.625	0.10764	-88.5290929961619	-88.5290929961619\\
58.625	0.1113	-95.4239546587316	-95.4239546587316\\
58.625	0.11496	-103.140556686117	-103.140556686117\\
58.625	0.11862	-111.678899078317	-111.678899078317\\
58.625	0.12228	-121.038981835332	-121.038981835332\\
58.625	0.12594	-131.220804957163	-131.220804957163\\
58.625	0.1296	-142.224368443809	-142.224368443809\\
58.625	0.13326	-154.049672295269	-154.049672295269\\
58.625	0.13692	-166.696716511546	-166.696716511546\\
58.625	0.14058	-180.165501092637	-180.165501092637\\
58.625	0.14424	-194.456026038543	-194.456026038543\\
58.625	0.1479	-209.568291349265	-209.568291349265\\
58.625	0.15156	-225.502297024801	-225.502297024801\\
58.625	0.15522	-242.258043065153	-242.258043065153\\
58.625	0.15888	-259.83552947032	-259.83552947032\\
58.625	0.16254	-278.234756240302	-278.234756240302\\
58.625	0.1662	-297.4557233751	-297.4557233751\\
58.625	0.16986	-317.498430874712	-317.498430874712\\
58.625	0.17352	-338.36287873914	-338.36287873914\\
58.625	0.17718	-360.049066968383	-360.049066968383\\
58.625	0.18084	-382.556995562441	-382.556995562441\\
58.625	0.1845	-405.886664521314	-405.886664521314\\
58.625	0.18816	-430.038073845002	-430.038073845002\\
58.625	0.19182	-455.011223533506	-455.011223533506\\
58.625	0.19548	-480.806113586824	-480.806113586824\\
58.625	0.19914	-507.422744004958	-507.422744004958\\
58.625	0.2028	-534.861114787908	-534.861114787908\\
58.625	0.20646	-563.121225935672	-563.121225935672\\
58.625	0.21012	-592.203077448251	-592.203077448251\\
58.625	0.21378	-622.106669325645	-622.106669325645\\
58.625	0.21744	-652.832001567855	-652.832001567855\\
58.625	0.2211	-684.37907417488	-684.37907417488\\
58.625	0.22476	-716.74788714672	-716.74788714672\\
58.625	0.22842	-749.938440483375	-749.938440483375\\
58.625	0.23208	-783.950734184845	-783.950734184845\\
58.625	0.23574	-818.784768251131	-818.784768251131\\
58.625	0.2394	-854.440542682231	-854.440542682231\\
58.625	0.24306	-890.918057478147	-890.918057478147\\
58.625	0.24672	-928.217312638878	-928.217312638878\\
58.625	0.25038	-966.338308164424	-966.338308164424\\
58.625	0.25404	-1005.28104405479	-1005.28104405479\\
58.625	0.2577	-1045.04552030996	-1045.04552030996\\
58.625	0.26136	-1085.63173692995	-1085.63173692995\\
58.625	0.26502	-1127.03969391476	-1127.03969391476\\
58.625	0.26868	-1169.26939126438	-1169.26939126438\\
58.625	0.27234	-1212.32082897882	-1212.32082897882\\
58.625	0.276	-1256.19400705807	-1256.19400705807\\
59	0.093	-68.9000870319068	-68.9000870319068\\
59	0.09666	-72.5199498930289	-72.5199498930289\\
59	0.10032	-76.9615531189664	-76.9615531189664\\
59	0.10398	-82.2248967097192	-82.2248967097192\\
59	0.10764	-88.3099806652868	-88.3099806652868\\
59	0.1113	-95.2168049856698	-95.2168049856698\\
59	0.11496	-102.945369670868	-102.945369670868\\
59	0.11862	-111.495674720881	-111.495674720881\\
59	0.12228	-120.86772013571	-120.86772013571\\
59	0.12594	-131.061505915353	-131.061505915353\\
59	0.1296	-142.077032059812	-142.077032059812\\
59	0.13326	-153.914298569086	-153.914298569086\\
59	0.13692	-166.573305443175	-166.573305443175\\
59	0.14058	-180.054052682079	-180.054052682079\\
59	0.14424	-194.356540285799	-194.356540285799\\
59	0.1479	-209.480768254333	-209.480768254333\\
59	0.15156	-225.426736587683	-225.426736587683\\
59	0.15522	-242.194445285848	-242.194445285848\\
59	0.15888	-259.783894348828	-259.783894348828\\
59	0.16254	-278.195083776624	-278.195083776624\\
59	0.1662	-297.428013569234	-297.428013569234\\
59	0.16986	-317.48268372666	-317.48268372666\\
59	0.17352	-338.359094248901	-338.359094248901\\
59	0.17718	-360.057245135956	-360.057245135956\\
59	0.18084	-382.577136387827	-382.577136387827\\
59	0.1845	-405.918768004514	-405.918768004514\\
59	0.18816	-430.082139986015	-430.082139986015\\
59	0.19182	-455.067252332332	-455.067252332332\\
59	0.19548	-480.874105043463	-480.874105043463\\
59	0.19914	-507.50269811941	-507.50269811941\\
59	0.2028	-534.953031560173	-534.953031560173\\
59	0.20646	-563.22510536575	-563.22510536575\\
59	0.21012	-592.318919536142	-592.318919536142\\
59	0.21378	-622.23447407135	-622.23447407135\\
59	0.21744	-652.971768971373	-652.971768971373\\
59	0.2211	-684.53080423621	-684.53080423621\\
59	0.22476	-716.911579865863	-716.911579865863\\
59	0.22842	-750.114095860331	-750.114095860331\\
59	0.23208	-784.138352219615	-784.138352219615\\
59	0.23574	-818.984348943714	-818.984348943714\\
59	0.2394	-854.652086032627	-854.652086032627\\
59	0.24306	-891.141563486356	-891.141563486356\\
59	0.24672	-928.4527813049	-928.4527813049\\
59	0.25038	-966.58573948826	-966.58573948826\\
59	0.25404	-1005.54043803643	-1005.54043803643\\
59	0.2577	-1045.31687694942	-1045.31687694942\\
59	0.26136	-1085.91505622723	-1085.91505622723\\
59	0.26502	-1127.33497586985	-1127.33497586985\\
59	0.26868	-1169.57663587728	-1169.57663587728\\
59	0.27234	-1212.64003624953	-1212.64003624953\\
59	0.276	-1256.5251769866	-1256.5251769866\\
59.375	0.093	-68.7041576266799	-68.7041576266799\\
59.375	0.09666	-72.335983145615	-72.335983145615\\
59.375	0.10032	-76.7895490293658	-76.7895490293658\\
59.375	0.10398	-82.0648552779316	-82.0648552779316\\
59.375	0.10764	-88.1619018913122	-88.1619018913122\\
59.375	0.1113	-95.0806888695084	-95.0806888695084\\
59.375	0.11496	-102.82121621252	-102.82121621252\\
59.375	0.11862	-111.383483920346	-111.383483920346\\
59.375	0.12228	-120.767491992988	-120.767491992988\\
59.375	0.12594	-130.973240430444	-130.973240430444\\
59.375	0.1296	-142.000729232716	-142.000729232716\\
59.375	0.13326	-153.849958399803	-153.849958399803\\
59.375	0.13692	-166.520927931705	-166.520927931705\\
59.375	0.14058	-180.013637828423	-180.013637828423\\
59.375	0.14424	-194.328088089955	-194.328088089955\\
59.375	0.1479	-209.464278716303	-209.464278716303\\
59.375	0.15156	-225.422209707466	-225.422209707466\\
59.375	0.15522	-242.201881063444	-242.201881063444\\
59.375	0.15888	-259.803292784237	-259.803292784237\\
59.375	0.16254	-278.226444869845	-278.226444869845\\
59.375	0.1662	-297.471337320269	-297.471337320269\\
59.375	0.16986	-317.537970135507	-317.537970135507\\
59.375	0.17352	-338.426343315562	-338.426343315562\\
59.375	0.17718	-360.13645686043	-360.13645686043\\
59.375	0.18084	-382.668310770114	-382.668310770114\\
59.375	0.1845	-406.021905044614	-406.021905044614\\
59.375	0.18816	-430.197239683928	-430.197239683928\\
59.375	0.19182	-455.194314688058	-455.194314688058\\
59.375	0.19548	-481.013130057003	-481.013130057003\\
59.375	0.19914	-507.653685790763	-507.653685790763\\
59.375	0.2028	-535.115981889338	-535.115981889338\\
59.375	0.20646	-563.400018352728	-563.400018352728\\
59.375	0.21012	-592.505795180934	-592.505795180934\\
59.375	0.21378	-622.433312373955	-622.433312373955\\
59.375	0.21744	-653.18256993179	-653.18256993179\\
59.375	0.2211	-684.753567854441	-684.753567854441\\
59.375	0.22476	-717.146306141907	-717.146306141907\\
59.375	0.22842	-750.360784794189	-750.360784794189\\
59.375	0.23208	-784.397003811285	-784.397003811285\\
59.375	0.23574	-819.254963193197	-819.254963193197\\
59.375	0.2394	-854.934662939924	-854.934662939924\\
59.375	0.24306	-891.436103051466	-891.436103051466\\
59.375	0.24672	-928.759283527823	-928.759283527823\\
59.375	0.25038	-966.904204368995	-966.904204368995\\
59.375	0.25404	-1005.87086557498	-1005.87086557498\\
59.375	0.2577	-1045.65926714579	-1045.65926714579\\
59.375	0.26136	-1086.2694090814	-1086.2694090814\\
59.375	0.26502	-1127.70129138184	-1127.70129138184\\
59.375	0.26868	-1169.95491404708	-1169.95491404708\\
59.375	0.27234	-1213.03027707715	-1213.03027707715\\
59.375	0.276	-1256.92738047203	-1256.92738047203\\
59.75	0.093	-68.579261778354	-68.579261778354\\
59.75	0.09666	-72.2230499551025	-72.2230499551025\\
59.75	0.10032	-76.688578496666	-76.688578496666\\
59.75	0.10398	-81.9758474030448	-81.9758474030448\\
59.75	0.10764	-88.0848566742389	-88.0848566742389\\
59.75	0.1113	-95.0156063102479	-95.0156063102479\\
59.75	0.11496	-102.768096311072	-102.768096311072\\
59.75	0.11862	-111.342326676712	-111.342326676712\\
59.75	0.12228	-120.738297407166	-120.738297407166\\
59.75	0.12594	-130.956008502436	-130.956008502436\\
59.75	0.1296	-141.995459962521	-141.995459962521\\
59.75	0.13326	-153.856651787421	-153.856651787421\\
59.75	0.13692	-166.539583977136	-166.539583977136\\
59.75	0.14058	-180.044256531667	-180.044256531667\\
59.75	0.14424	-194.370669451012	-194.370669451012\\
59.75	0.1479	-209.518822735173	-209.518822735173\\
59.75	0.15156	-225.488716384149	-225.488716384149\\
59.75	0.15522	-242.28035039794	-242.28035039794\\
59.75	0.15888	-259.893724776546	-259.893724776546\\
59.75	0.16254	-278.328839519968	-278.328839519968\\
59.75	0.1662	-297.585694628204	-297.585694628204\\
59.75	0.16986	-317.664290101256	-317.664290101256\\
59.75	0.17352	-338.564625939123	-338.564625939123\\
59.75	0.17718	-360.286702141805	-360.286702141805\\
59.75	0.18084	-382.830518709302	-382.830518709302\\
59.75	0.1845	-406.196075641615	-406.196075641615\\
59.75	0.18816	-430.383372938742	-430.383372938742\\
59.75	0.19182	-455.392410600685	-455.392410600685\\
59.75	0.19548	-481.223188627443	-481.223188627443\\
59.75	0.19914	-507.875707019016	-507.875707019016\\
59.75	0.2028	-535.349965775405	-535.349965775405\\
59.75	0.20646	-563.645964896608	-563.645964896608\\
59.75	0.21012	-592.763704382626	-592.763704382626\\
59.75	0.21378	-622.70318423346	-622.70318423346\\
59.75	0.21744	-653.464404449109	-653.464404449109\\
59.75	0.2211	-685.047365029573	-685.047365029573\\
59.75	0.22476	-717.452065974852	-717.452065974852\\
59.75	0.22842	-750.678507284947	-750.678507284947\\
59.75	0.23208	-784.726688959856	-784.726688959856\\
59.75	0.23574	-819.596610999581	-819.596610999581\\
59.75	0.2394	-855.288273404121	-855.288273404121\\
59.75	0.24306	-891.801676173476	-891.801676173476\\
59.75	0.24672	-929.136819307646	-929.136819307646\\
59.75	0.25038	-967.293702806632	-967.293702806632\\
59.75	0.25404	-1006.27232667043	-1006.27232667043\\
59.75	0.2577	-1046.07269089905	-1046.07269089905\\
59.75	0.26136	-1086.69479549248	-1086.69479549248\\
59.75	0.26502	-1128.13864045072	-1128.13864045072\\
59.75	0.26868	-1170.40422577379	-1170.40422577379\\
59.75	0.27234	-1213.49155146166	-1213.49155146166\\
59.75	0.276	-1257.40061751435	-1257.40061751435\\
60.125	0.093	-68.5253994869283	-68.5253994869283\\
60.125	0.09666	-72.1811503214899	-72.1811503214899\\
60.125	0.10032	-76.6586415208666	-76.6586415208666\\
60.125	0.10398	-81.9578730850585	-81.9578730850585\\
60.125	0.10764	-88.0788450140655	-88.0788450140655\\
60.125	0.1113	-95.0215573078877	-95.0215573078877\\
60.125	0.11496	-102.786009966525	-102.786009966525\\
60.125	0.11862	-111.372202989978	-111.372202989978\\
60.125	0.12228	-120.780136378245	-120.780136378245\\
60.125	0.12594	-131.009810131328	-131.009810131328\\
60.125	0.1296	-142.061224249226	-142.061224249226\\
60.125	0.13326	-153.934378731939	-153.934378731939\\
60.125	0.13692	-166.629273579468	-166.629273579468\\
60.125	0.14058	-180.145908791811	-180.145908791811\\
60.125	0.14424	-194.48428436897	-194.48428436897\\
60.125	0.1479	-209.644400310944	-209.644400310944\\
60.125	0.15156	-225.626256617733	-225.626256617733\\
60.125	0.15522	-242.429853289337	-242.429853289337\\
60.125	0.15888	-260.055190325756	-260.055190325756\\
60.125	0.16254	-278.502267726991	-278.502267726991\\
60.125	0.1662	-297.77108549304	-297.77108549304\\
60.125	0.16986	-317.861643623905	-317.861643623905\\
60.125	0.17352	-338.773942119585	-338.773942119585\\
60.125	0.17718	-360.507980980081	-360.507980980081\\
60.125	0.18084	-383.063760205391	-383.063760205391\\
60.125	0.1845	-406.441279795516	-406.441279795516\\
60.125	0.18816	-430.640539750457	-430.640539750457\\
60.125	0.19182	-455.661540070213	-455.661540070213\\
60.125	0.19548	-481.504280754784	-481.504280754784\\
60.125	0.19914	-508.16876180417	-508.16876180417\\
60.125	0.2028	-535.654983218372	-535.654983218372\\
60.125	0.20646	-563.962944997388	-563.962944997388\\
60.125	0.21012	-593.092647141219	-593.092647141219\\
60.125	0.21378	-623.044089649866	-623.044089649866\\
60.125	0.21744	-653.817272523328	-653.817272523328\\
60.125	0.2211	-685.412195761605	-685.412195761605\\
60.125	0.22476	-717.828859364697	-717.828859364697\\
60.125	0.22842	-751.067263332605	-751.067263332605\\
60.125	0.23208	-785.127407665327	-785.127407665327\\
60.125	0.23574	-820.009292362865	-820.009292362865\\
60.125	0.2394	-855.712917425219	-855.712917425219\\
60.125	0.24306	-892.238282852387	-892.238282852387\\
60.125	0.24672	-929.58538864437	-929.58538864437\\
60.125	0.25038	-967.754234801169	-967.754234801169\\
60.125	0.25404	-1006.74482132278	-1006.74482132278\\
60.125	0.2577	-1046.55714820921	-1046.55714820921\\
60.125	0.26136	-1087.19121546046	-1087.19121546046\\
60.125	0.26502	-1128.64702307651	-1128.64702307651\\
60.125	0.26868	-1170.92457105739	-1170.92457105739\\
60.125	0.27234	-1214.02385940308	-1214.02385940308\\
60.125	0.276	-1257.94488811358	-1257.94488811358\\
60.5	0.093	-68.5425707524037	-68.5425707524037\\
60.5	0.09666	-72.2102842447782	-72.2102842447782\\
60.5	0.10032	-76.699738101968	-76.699738101968\\
60.5	0.10398	-82.010932323973	-82.010932323973\\
60.5	0.10764	-88.1438669107931	-88.1438669107931\\
60.5	0.1113	-95.0985418624283	-95.0985418624283\\
60.5	0.11496	-102.874957178879	-102.874957178879\\
60.5	0.11862	-111.473112860144	-111.473112860144\\
60.5	0.12228	-120.893008906225	-120.893008906225\\
60.5	0.12594	-131.134645317121	-131.134645317121\\
60.5	0.1296	-142.198022092832	-142.198022092832\\
60.5	0.13326	-154.083139233358	-154.083139233358\\
60.5	0.13692	-166.7899967387	-166.7899967387\\
60.5	0.14058	-180.318594608857	-180.318594608857\\
60.5	0.14424	-194.668932843828	-194.668932843828\\
60.5	0.1479	-209.841011443615	-209.841011443615\\
60.5	0.15156	-225.834830408218	-225.834830408218\\
60.5	0.15522	-242.650389737635	-242.650389737635\\
60.5	0.15888	-260.287689431867	-260.287689431867\\
60.5	0.16254	-278.746729490915	-278.746729490915\\
60.5	0.1662	-298.027509914777	-298.027509914777\\
60.5	0.16986	-318.130030703455	-318.130030703455\\
60.5	0.17352	-339.054291856949	-339.054291856949\\
60.5	0.17718	-360.800293375257	-360.800293375257\\
60.5	0.18084	-383.36803525838	-383.36803525838\\
60.5	0.1845	-406.757517506319	-406.757517506319\\
60.5	0.18816	-430.968740119072	-430.968740119072\\
60.5	0.19182	-456.001703096641	-456.001703096641\\
60.5	0.19548	-481.856406439025	-481.856406439025\\
60.5	0.19914	-508.532850146224	-508.532850146224\\
60.5	0.2028	-536.031034218239	-536.031034218239\\
60.5	0.20646	-564.350958655069	-564.350958655069\\
60.5	0.21012	-593.492623456713	-593.492623456713\\
60.5	0.21378	-623.456028623173	-623.456028623173\\
60.5	0.21744	-654.241174154448	-654.241174154448\\
60.5	0.2211	-685.848060050538	-685.848060050538\\
60.5	0.22476	-718.276686311444	-718.276686311444\\
60.5	0.22842	-751.527052937164	-751.527052937164\\
60.5	0.23208	-785.5991599277	-785.5991599277\\
60.5	0.23574	-820.493007283051	-820.493007283051\\
60.5	0.2394	-856.208595003217	-856.208595003217\\
60.5	0.24306	-892.745923088198	-892.745923088198\\
60.5	0.24672	-930.104991537995	-930.104991537995\\
60.5	0.25038	-968.285800352606	-968.285800352606\\
60.5	0.25404	-1007.28834953203	-1007.28834953203\\
60.5	0.2577	-1047.11263907628	-1047.11263907628\\
60.5	0.26136	-1087.75866898533	-1087.75866898533\\
60.5	0.26502	-1129.2264392592	-1129.2264392592\\
60.5	0.26868	-1171.51594989789	-1171.51594989789\\
60.5	0.27234	-1214.62720090139	-1214.62720090139\\
60.5	0.276	-1258.56019226971	-1258.56019226971\\
60.875	0.093	-68.6307755747795	-68.6307755747795\\
60.875	0.09666	-72.310451724967	-72.310451724967\\
60.875	0.10032	-76.81186823997	-76.81186823997\\
60.875	0.10398	-82.1350251197881	-82.1350251197881\\
60.875	0.10764	-88.2799223644211	-88.2799223644211\\
60.875	0.1113	-95.2465599738696	-95.2465599738696\\
60.875	0.11496	-103.034937948133	-103.034937948133\\
60.875	0.11862	-111.645056287212	-111.645056287212\\
60.875	0.12228	-121.076914991106	-121.076914991106\\
60.875	0.12594	-131.330514059815	-131.330514059815\\
60.875	0.1296	-142.405853493339	-142.405853493339\\
60.875	0.13326	-154.302933291678	-154.302933291678\\
60.875	0.13692	-167.021753454832	-167.021753454832\\
60.875	0.14058	-180.562313982802	-180.562313982802\\
60.875	0.14424	-194.924614875587	-194.924614875587\\
60.875	0.1479	-210.108656133187	-210.108656133187\\
60.875	0.15156	-226.114437755603	-226.114437755603\\
60.875	0.15522	-242.941959742833	-242.941959742833\\
60.875	0.15888	-260.591222094878	-260.591222094878\\
60.875	0.16254	-279.062224811739	-279.062224811739\\
60.875	0.1662	-298.354967893415	-298.354967893415\\
60.875	0.16986	-318.469451339906	-318.469451339906\\
60.875	0.17352	-339.405675151212	-339.405675151212\\
60.875	0.17718	-361.163639327333	-361.163639327333\\
60.875	0.18084	-383.74334386827	-383.74334386827\\
60.875	0.1845	-407.144788774021	-407.144788774021\\
60.875	0.18816	-431.367974044588	-431.367974044588\\
60.875	0.19182	-456.41289967997	-456.41289967997\\
60.875	0.19548	-482.279565680167	-482.279565680167\\
60.875	0.19914	-508.96797204518	-508.96797204518\\
60.875	0.2028	-536.478118775007	-536.478118775007\\
60.875	0.20646	-564.81000586965	-564.81000586965\\
60.875	0.21012	-593.963633329108	-593.963633329108\\
60.875	0.21378	-623.939001153381	-623.939001153381\\
60.875	0.21744	-654.736109342469	-654.736109342469\\
60.875	0.2211	-686.354957896372	-686.354957896372\\
60.875	0.22476	-718.79554681509	-718.79554681509\\
60.875	0.22842	-752.057876098624	-752.057876098624\\
60.875	0.23208	-786.141945746973	-786.141945746973\\
60.875	0.23574	-821.047755760137	-821.047755760137\\
60.875	0.2394	-856.775306138116	-856.775306138116\\
60.875	0.24306	-893.32459688091	-893.32459688091\\
60.875	0.24672	-930.69562798852	-930.69562798852\\
60.875	0.25038	-968.888399460944	-968.888399460944\\
60.875	0.25404	-1007.90291129818	-1007.90291129818\\
60.875	0.2577	-1047.73916350024	-1047.73916350024\\
60.875	0.26136	-1088.39715606711	-1088.39715606711\\
60.875	0.26502	-1129.87688899879	-1129.87688899879\\
60.875	0.26868	-1172.17836229529	-1172.17836229529\\
60.875	0.27234	-1215.30157595661	-1215.30157595661\\
60.875	0.276	-1259.24652998274	-1259.24652998274\\
61.25	0.093	-68.790013954056	-68.790013954056\\
61.25	0.09666	-72.481652762057	-72.481652762057\\
61.25	0.10032	-76.9950319348728	-76.9950319348728\\
61.25	0.10398	-82.3301514725038	-82.3301514725038\\
61.25	0.10764	-88.4870113749503	-88.4870113749503\\
61.25	0.1113	-95.4656116422116	-95.4656116422116\\
61.25	0.11496	-103.265952274288	-103.265952274288\\
61.25	0.11862	-111.88803327118	-111.88803327118\\
61.25	0.12228	-121.331854632887	-121.331854632887\\
61.25	0.12594	-131.597416359409	-131.597416359409\\
61.25	0.1296	-142.684718450746	-142.684718450746\\
61.25	0.13326	-154.593760906899	-154.593760906899\\
61.25	0.13692	-167.324543727866	-167.324543727866\\
61.25	0.14058	-180.877066913649	-180.877066913649\\
61.25	0.14424	-195.251330464247	-195.251330464247\\
61.25	0.1479	-210.44733437966	-210.44733437966\\
61.25	0.15156	-226.465078659888	-226.465078659888\\
61.25	0.15522	-243.304563304932	-243.304563304932\\
61.25	0.15888	-260.96578831479	-260.96578831479\\
61.25	0.16254	-279.448753689464	-279.448753689464\\
61.25	0.1662	-298.753459428953	-298.753459428953\\
61.25	0.16986	-318.879905533257	-318.879905533257\\
61.25	0.17352	-339.828092002376	-339.828092002376\\
61.25	0.17718	-361.598018836311	-361.598018836311\\
61.25	0.18084	-384.18968603506	-384.18968603506\\
61.25	0.1845	-407.603093598625	-407.603093598625\\
61.25	0.18816	-431.838241527005	-431.838241527005\\
61.25	0.19182	-456.8951298202	-456.8951298202\\
61.25	0.19548	-482.77375847821	-482.77375847821\\
61.25	0.19914	-509.474127501035	-509.474127501035\\
61.25	0.2028	-536.996236888676	-536.996236888676\\
61.25	0.20646	-565.340086641132	-565.340086641132\\
61.25	0.21012	-594.505676758403	-594.505676758403\\
61.25	0.21378	-624.493007240489	-624.493007240489\\
61.25	0.21744	-655.30207808739	-655.30207808739\\
61.25	0.2211	-686.932889299106	-686.932889299106\\
61.25	0.22476	-719.385440875638	-719.385440875638\\
61.25	0.22842	-752.659732816984	-752.659732816984\\
61.25	0.23208	-786.755765123146	-786.755765123146\\
61.25	0.23574	-821.673537794124	-821.673537794124\\
61.25	0.2394	-857.413050829916	-857.413050829916\\
61.25	0.24306	-893.974304230523	-893.974304230523\\
61.25	0.24672	-931.357297995946	-931.357297995946\\
61.25	0.25038	-969.562032126183	-969.562032126183\\
61.25	0.25404	-1008.58850662124	-1008.58850662124\\
61.25	0.2577	-1048.4367214811	-1048.4367214811\\
61.25	0.26136	-1089.10667670579	-1089.10667670579\\
61.25	0.26502	-1130.59837229529	-1130.59837229529\\
61.25	0.26868	-1172.9118082496	-1172.9118082496\\
61.25	0.27234	-1216.04698456873	-1216.04698456873\\
61.25	0.276	-1260.00390125267	-1260.00390125267\\
61.625	0.093	-69.020285890233	-69.020285890233\\
61.625	0.09666	-72.723887356047	-72.723887356047\\
61.625	0.10032	-77.249229186676	-77.249229186676\\
61.625	0.10398	-82.5963113821201	-82.5963113821201\\
61.625	0.10764	-88.7651339423796	-88.7651339423796\\
61.625	0.1113	-95.755696867454	-95.755696867454\\
61.625	0.11496	-103.568000157344	-103.568000157344\\
61.625	0.11862	-112.202043812049	-112.202043812049\\
61.625	0.12228	-121.657827831569	-121.657827831569\\
61.625	0.12594	-131.935352215904	-131.935352215904\\
61.625	0.1296	-143.034616965054	-143.034616965054\\
61.625	0.13326	-154.95562207902	-154.95562207902\\
61.625	0.13692	-167.6983675578	-167.6983675578\\
61.625	0.14058	-181.262853401396	-181.262853401396\\
61.625	0.14424	-195.649079609807	-195.649079609807\\
61.625	0.1479	-210.857046183033	-210.857046183033\\
61.625	0.15156	-226.886753121074	-226.886753121074\\
61.625	0.15522	-243.738200423931	-243.738200423931\\
61.625	0.15888	-261.411388091603	-261.411388091603\\
61.625	0.16254	-279.90631612409	-279.90631612409\\
61.625	0.1662	-299.222984521392	-299.222984521392\\
61.625	0.16986	-319.361393283509	-319.361393283509\\
61.625	0.17352	-340.321542410441	-340.321542410441\\
61.625	0.17718	-362.103431902189	-362.103431902189\\
61.625	0.18084	-384.707061758751	-384.707061758751\\
61.625	0.1845	-408.132431980129	-408.132431980129\\
61.625	0.18816	-432.379542566322	-432.379542566322\\
61.625	0.19182	-457.44839351733	-457.44839351733\\
61.625	0.19548	-483.338984833153	-483.338984833153\\
61.625	0.19914	-510.051316513792	-510.051316513792\\
61.625	0.2028	-537.585388559246	-537.585388559246\\
61.625	0.20646	-565.941200969514	-565.941200969514\\
61.625	0.21012	-595.118753744598	-595.118753744598\\
61.625	0.21378	-625.118046884497	-625.118046884497\\
61.625	0.21744	-655.939080389212	-655.939080389212\\
61.625	0.2211	-687.581854258741	-687.581854258741\\
61.625	0.22476	-720.046368493085	-720.046368493085\\
61.625	0.22842	-753.332623092245	-753.332623092245\\
61.625	0.23208	-787.44061805622	-787.44061805622\\
61.625	0.23574	-822.370353385011	-822.370353385011\\
61.625	0.2394	-858.121829078616	-858.121829078616\\
61.625	0.24306	-894.695045137036	-894.695045137036\\
61.625	0.24672	-932.090001560272	-932.090001560272\\
61.625	0.25038	-970.306698348323	-970.306698348323\\
61.625	0.25404	-1009.34513550119	-1009.34513550119\\
61.625	0.2577	-1049.20531301887	-1049.20531301887\\
61.625	0.26136	-1089.88723090137	-1089.88723090137\\
61.625	0.26502	-1131.39088914868	-1131.39088914868\\
61.625	0.26868	-1173.7162877608	-1173.7162877608\\
61.625	0.27234	-1216.86342673775	-1216.86342673775\\
61.625	0.276	-1260.8323060795	-1260.8323060795\\
62	0.093	-69.321591383311	-69.321591383311\\
62	0.09666	-73.0371555069378	-73.0371555069378\\
62	0.10032	-77.5744599953802	-77.5744599953802\\
62	0.10398	-82.9335048486375	-82.9335048486375\\
62	0.10764	-89.1142900667098	-89.1142900667098\\
62	0.1113	-96.1168156495975	-96.1168156495975\\
62	0.11496	-103.9410815973	-103.9410815973\\
62	0.11862	-112.587087909818	-112.587087909818\\
62	0.12228	-122.054834587151	-122.054834587151\\
62	0.12594	-132.344321629299	-132.344321629299\\
62	0.1296	-143.455549036263	-143.455549036263\\
62	0.13326	-155.388516808041	-155.388516808041\\
62	0.13692	-168.143224944635	-168.143224944635\\
62	0.14058	-181.719673446044	-181.719673446044\\
62	0.14424	-196.117862312268	-196.117862312268\\
62	0.1479	-211.337791543307	-211.337791543307\\
62	0.15156	-227.379461139162	-227.379461139162\\
62	0.15522	-244.242871099831	-244.242871099831\\
62	0.15888	-261.928021425316	-261.928021425316\\
62	0.16254	-280.434912115616	-280.434912115616\\
62	0.1662	-299.763543170731	-299.763543170731\\
62	0.16986	-319.913914590661	-319.913914590661\\
62	0.17352	-340.886026375407	-340.886026375407\\
62	0.17718	-362.679878524967	-362.679878524967\\
62	0.18084	-385.295471039343	-385.295471039343\\
62	0.1845	-408.732803918534	-408.732803918534\\
62	0.18816	-432.99187716254	-432.99187716254\\
62	0.19182	-458.072690771361	-458.072690771361\\
62	0.19548	-483.975244744998	-483.975244744998\\
62	0.19914	-510.699539083449	-510.699539083449\\
62	0.2028	-538.245573786716	-538.245573786716\\
62	0.20646	-566.613348854798	-566.613348854798\\
62	0.21012	-595.802864287695	-595.802864287695\\
62	0.21378	-625.814120085407	-625.814120085407\\
62	0.21744	-656.647116247934	-656.647116247934\\
62	0.2211	-688.301852775277	-688.301852775277\\
62	0.22476	-720.778329667434	-720.778329667434\\
62	0.22842	-754.076546924407	-754.076546924407\\
62	0.23208	-788.196504546195	-788.196504546195\\
62	0.23574	-823.138202532799	-823.138202532799\\
62	0.2394	-858.901640884217	-858.901640884217\\
62	0.24306	-895.486819600451	-895.486819600451\\
62	0.24672	-932.893738681499	-932.893738681499\\
62	0.25038	-971.122398127363	-971.122398127363\\
62	0.25404	-1010.17279793804	-1010.17279793804\\
62	0.2577	-1050.04493811354	-1050.04493811354\\
62	0.26136	-1090.73881865385	-1090.73881865385\\
62	0.26502	-1132.25443955897	-1132.25443955897\\
62	0.26868	-1174.59180082891	-1174.59180082891\\
62	0.27234	-1217.75090246366	-1217.75090246366\\
62	0.276	-1261.73174446323	-1261.73174446323\\
62.375	0.093	-69.6939304332896	-69.6939304332896\\
62.375	0.09666	-73.4214572147293	-73.4214572147293\\
62.375	0.10032	-77.9707243609845	-77.9707243609845\\
62.375	0.10398	-83.341731872055	-83.341731872055\\
62.375	0.10764	-89.5344797479403	-89.5344797479403\\
62.375	0.1113	-96.548967988641	-96.548967988641\\
62.375	0.11496	-104.385196594157	-104.385196594157\\
62.375	0.11862	-113.043165564488	-113.043165564488\\
62.375	0.12228	-122.522874899634	-122.522874899634\\
62.375	0.12594	-132.824324599595	-132.824324599595\\
62.375	0.1296	-143.947514664372	-143.947514664372\\
62.375	0.13326	-155.892445093964	-155.892445093964\\
62.375	0.13692	-168.65911588837	-168.65911588837\\
62.375	0.14058	-182.247527047593	-182.247527047593\\
62.375	0.14424	-196.65767857163	-196.65767857163\\
62.375	0.1479	-211.889570460482	-211.889570460482\\
62.375	0.15156	-227.94320271415	-227.94320271415\\
62.375	0.15522	-244.818575332632	-244.818575332632\\
62.375	0.15888	-262.51568831593	-262.51568831593\\
62.375	0.16254	-281.034541664043	-281.034541664043\\
62.375	0.1662	-300.375135376971	-300.375135376971\\
62.375	0.16986	-320.537469454714	-320.537469454714\\
62.375	0.17352	-341.521543897273	-341.521543897273\\
62.375	0.17718	-363.327358704646	-363.327358704646\\
62.375	0.18084	-385.954913876835	-385.954913876835\\
62.375	0.1845	-409.404209413839	-409.404209413839\\
62.375	0.18816	-433.675245315658	-433.675245315658\\
62.375	0.19182	-458.768021582293	-458.768021582293\\
62.375	0.19548	-484.682538213742	-484.682538213742\\
62.375	0.19914	-511.418795210007	-511.418795210007\\
62.375	0.2028	-538.976792571087	-538.976792571087\\
62.375	0.20646	-567.356530296982	-567.356530296982\\
62.375	0.21012	-596.558008387692	-596.558008387692\\
62.375	0.21378	-626.581226843217	-626.581226843217\\
62.375	0.21744	-657.426185663557	-657.426185663557\\
62.375	0.2211	-689.092884848713	-689.092884848713\\
62.375	0.22476	-721.581324398683	-721.581324398683\\
62.375	0.22842	-754.89150431347	-754.89150431347\\
62.375	0.23208	-789.023424593071	-789.023424593071\\
62.375	0.23574	-823.977085237487	-823.977085237487\\
62.375	0.2394	-859.752486246719	-859.752486246719\\
62.375	0.24306	-896.349627620765	-896.349627620765\\
62.375	0.24672	-933.768509359627	-933.768509359627\\
62.375	0.25038	-972.009131463304	-972.009131463304\\
62.375	0.25404	-1011.0714939318	-1011.0714939318\\
62.375	0.2577	-1050.9555967651	-1050.9555967651\\
62.375	0.26136	-1091.66143996323	-1091.66143996323\\
62.375	0.26502	-1133.18902352616	-1133.18902352616\\
62.375	0.26868	-1175.53834745392	-1175.53834745392\\
62.375	0.27234	-1218.70941174648	-1218.70941174648\\
62.375	0.276	-1262.70221640387	-1262.70221640387\\
62.75	0.093	-70.1373030401687	-70.1373030401687\\
62.75	0.09666	-73.8767924794217	-73.8767924794217\\
62.75	0.10032	-78.4380222834899	-78.4380222834899\\
62.75	0.10398	-83.8209924523734	-83.8209924523734\\
62.75	0.10764	-90.0257029860722	-90.0257029860722\\
62.75	0.1113	-97.0521538845857	-97.0521538845857\\
62.75	0.11496	-104.900345147915	-104.900345147915\\
62.75	0.11862	-113.570276776059	-113.570276776059\\
62.75	0.12228	-123.061948769018	-123.061948769018\\
62.75	0.12594	-133.375361126793	-133.375361126793\\
62.75	0.1296	-144.510513849382	-144.510513849382\\
62.75	0.13326	-156.467406936787	-156.467406936787\\
62.75	0.13692	-169.246040389007	-169.246040389007\\
62.75	0.14058	-182.846414206042	-182.846414206042\\
62.75	0.14424	-197.268528387892	-197.268528387892\\
62.75	0.1479	-212.512382934558	-212.512382934558\\
62.75	0.15156	-228.577977846038	-228.577977846038\\
62.75	0.15522	-245.465313122334	-245.465313122334\\
62.75	0.15888	-263.174388763445	-263.174388763445\\
62.75	0.16254	-281.705204769371	-281.705204769371\\
62.75	0.1662	-301.057761140112	-301.057761140112\\
62.75	0.16986	-321.232057875668	-321.232057875668\\
62.75	0.17352	-342.22809497604	-342.22809497604\\
62.75	0.17718	-364.045872441227	-364.045872441227\\
62.75	0.18084	-386.685390271228	-386.685390271228\\
62.75	0.1845	-410.146648466046	-410.146648466046\\
62.75	0.18816	-434.429647025678	-434.429647025678\\
62.75	0.19182	-459.534385950125	-459.534385950125\\
62.75	0.19548	-485.460865239388	-485.460865239388\\
62.75	0.19914	-512.209084893465	-512.209084893465\\
62.75	0.2028	-539.779044912358	-539.779044912358\\
62.75	0.20646	-568.170745296066	-568.170745296066\\
62.75	0.21012	-597.384186044589	-597.384186044589\\
62.75	0.21378	-627.419367157928	-627.419367157928\\
62.75	0.21744	-658.276288636081	-658.276288636081\\
62.75	0.2211	-689.95495047905	-689.95495047905\\
62.75	0.22476	-722.455352686834	-722.455352686834\\
62.75	0.22842	-755.777495259433	-755.777495259433\\
62.75	0.23208	-789.921378196847	-789.921378196847\\
62.75	0.23574	-824.887001499077	-824.887001499077\\
62.75	0.2394	-860.674365166121	-860.674365166121\\
62.75	0.24306	-897.28346919798	-897.28346919798\\
62.75	0.24672	-934.714313594655	-934.714313594655\\
62.75	0.25038	-972.966898356145	-972.966898356145\\
62.75	0.25404	-1012.04122348245	-1012.04122348245\\
62.75	0.2577	-1051.93728897357	-1051.93728897357\\
62.75	0.26136	-1092.65509482951	-1092.65509482951\\
62.75	0.26502	-1134.19464105026	-1134.19464105026\\
62.75	0.26868	-1176.55592763582	-1176.55592763582\\
62.75	0.27234	-1219.7389545862	-1219.7389545862\\
62.75	0.276	-1263.7437219014	-1263.7437219014\\
63.125	0.093	-70.6517092039484	-70.6517092039484\\
63.125	0.09666	-74.4031613010146	-74.4031613010146\\
63.125	0.10032	-78.9763537628958	-78.9763537628958\\
63.125	0.10398	-84.3712865895923	-84.3712865895923\\
63.125	0.10764	-90.5879597811041	-90.5879597811041\\
63.125	0.1113	-97.6263733374308	-97.6263733374308\\
63.125	0.11496	-105.486527258573	-105.486527258573\\
63.125	0.11862	-114.16842154453	-114.16842154453\\
63.125	0.12228	-123.672056195302	-123.672056195302\\
63.125	0.12594	-133.99743121089	-133.99743121089\\
63.125	0.1296	-145.144546591292	-145.144546591292\\
63.125	0.13326	-157.11340233651	-157.11340233651\\
63.125	0.13692	-169.903998446543	-169.903998446543\\
63.125	0.14058	-183.516334921391	-183.516334921391\\
63.125	0.14424	-197.950411761055	-197.950411761055\\
63.125	0.1479	-213.206228965533	-213.206228965533\\
63.125	0.15156	-229.283786534827	-229.283786534827\\
63.125	0.15522	-246.183084468936	-246.183084468936\\
63.125	0.15888	-263.90412276786	-263.90412276786\\
63.125	0.16254	-282.446901431599	-282.446901431599\\
63.125	0.1662	-301.811420460153	-301.811420460153\\
63.125	0.16986	-321.997679853523	-321.997679853523\\
63.125	0.17352	-343.005679611707	-343.005679611707\\
63.125	0.17718	-364.835419734707	-364.835419734707\\
63.125	0.18084	-387.486900222522	-387.486900222522\\
63.125	0.1845	-410.960121075152	-410.960121075152\\
63.125	0.18816	-435.255082292597	-435.255082292597\\
63.125	0.19182	-460.371783874858	-460.371783874858\\
63.125	0.19548	-486.310225821933	-486.310225821933\\
63.125	0.19914	-513.070408133824	-513.070408133824\\
63.125	0.2028	-540.65233081053	-540.65233081053\\
63.125	0.20646	-569.055993852051	-569.055993852051\\
63.125	0.21012	-598.281397258388	-598.281397258388\\
63.125	0.21378	-628.328541029539	-628.328541029539\\
63.125	0.21744	-659.197425165506	-659.197425165506\\
63.125	0.2211	-690.888049666287	-690.888049666287\\
63.125	0.22476	-723.400414531884	-723.400414531884\\
63.125	0.22842	-756.734519762296	-756.734519762296\\
63.125	0.23208	-790.890365357523	-790.890365357523\\
63.125	0.23574	-825.867951317566	-825.867951317566\\
63.125	0.2394	-861.667277642424	-861.667277642424\\
63.125	0.24306	-898.288344332097	-898.288344332097\\
63.125	0.24672	-935.731151386584	-935.731151386584\\
63.125	0.25038	-973.995698805888	-973.995698805888\\
63.125	0.25404	-1013.08198659001	-1013.08198659001\\
63.125	0.2577	-1052.99001473894	-1052.99001473894\\
63.125	0.26136	-1093.71978325269	-1093.71978325269\\
63.125	0.26502	-1135.27129213125	-1135.27129213125\\
63.125	0.26868	-1177.64454137463	-1177.64454137463\\
63.125	0.27234	-1220.83953098282	-1220.83953098282\\
63.125	0.276	-1264.85626095583	-1264.85626095583\\
63.5	0.093	-71.2371489246289	-71.2371489246289\\
63.5	0.09666	-75.0005636795079	-75.0005636795079\\
63.5	0.10032	-79.5857187992025	-79.5857187992025\\
63.5	0.10398	-84.992614283712	-84.992614283712\\
63.5	0.10764	-91.2212501330368	-91.2212501330368\\
63.5	0.1113	-98.2716263471767	-98.2716263471767\\
63.5	0.11496	-106.143742926132	-106.143742926132\\
63.5	0.11862	-114.837599869902	-114.837599869902\\
63.5	0.12228	-124.353197178488	-124.353197178488\\
63.5	0.12594	-134.690534851888	-134.690534851888\\
63.5	0.1296	-145.849612890104	-145.849612890104\\
63.5	0.13326	-157.830431293135	-157.830431293135\\
63.5	0.13692	-170.632990060981	-170.632990060981\\
63.5	0.14058	-184.257289193642	-184.257289193642\\
63.5	0.14424	-198.703328691119	-198.703328691119\\
63.5	0.1479	-213.97110855341	-213.97110855341\\
63.5	0.15156	-230.060628780517	-230.060628780517\\
63.5	0.15522	-246.971889372439	-246.971889372439\\
63.5	0.15888	-264.704890329176	-264.704890329176\\
63.5	0.16254	-283.259631650728	-283.259631650728\\
63.5	0.1662	-302.636113337095	-302.636113337095\\
63.5	0.16986	-322.834335388278	-322.834335388278\\
63.5	0.17352	-343.854297804276	-343.854297804276\\
63.5	0.17718	-365.696000585088	-365.696000585088\\
63.5	0.18084	-388.359443730716	-388.359443730716\\
63.5	0.1845	-411.84462724116	-411.84462724116\\
63.5	0.18816	-436.151551116418	-436.151551116418\\
63.5	0.19182	-461.280215356492	-461.280215356492\\
63.5	0.19548	-487.23061996138	-487.23061996138\\
63.5	0.19914	-514.002764931084	-514.002764931084\\
63.5	0.2028	-541.596650265603	-541.596650265603\\
63.5	0.20646	-570.012275964937	-570.012275964937\\
63.5	0.21012	-599.249642029087	-599.249642029087\\
63.5	0.21378	-629.308748458051	-629.308748458051\\
63.5	0.21744	-660.189595251831	-660.189595251831\\
63.5	0.2211	-691.892182410426	-691.892182410426\\
63.5	0.22476	-724.416509933836	-724.416509933836\\
63.5	0.22842	-757.762577822061	-757.762577822061\\
63.5	0.23208	-791.930386075101	-791.930386075101\\
63.5	0.23574	-826.919934692957	-826.919934692957\\
63.5	0.2394	-862.731223675627	-862.731223675627\\
63.5	0.24306	-899.364253023113	-899.364253023113\\
63.5	0.24672	-936.819022735414	-936.819022735414\\
63.5	0.25038	-975.095532812531	-975.095532812531\\
63.5	0.25404	-1014.19378325446	-1014.19378325446\\
63.5	0.2577	-1054.11377406121	-1054.11377406121\\
63.5	0.26136	-1094.85550523277	-1094.85550523277\\
63.5	0.26502	-1136.41897676915	-1136.41897676915\\
63.5	0.26868	-1178.80418867034	-1178.80418867034\\
63.5	0.27234	-1222.01114093635	-1222.01114093635\\
63.5	0.276	-1266.03983356717	-1266.03983356717\\
63.875	0.093	-71.8936222022097	-71.8936222022097\\
63.875	0.09666	-75.668999614902	-75.668999614902\\
63.875	0.10032	-80.2661173924094	-80.2661173924094\\
63.875	0.10398	-85.6849755347321	-85.6849755347321\\
63.875	0.10764	-91.9255740418699	-91.9255740418699\\
63.875	0.1113	-98.987912913823	-98.987912913823\\
63.875	0.11496	-106.871992150591	-106.871992150591\\
63.875	0.11862	-115.577811752174	-115.577811752174\\
63.875	0.12228	-125.105371718573	-125.105371718573\\
63.875	0.12594	-135.454672049787	-135.454672049787\\
63.875	0.1296	-146.625712745815	-146.625712745815\\
63.875	0.13326	-158.618493806659	-158.618493806659\\
63.875	0.13692	-171.433015232318	-171.433015232318\\
63.875	0.14058	-185.069277022793	-185.069277022793\\
63.875	0.14424	-199.527279178082	-199.527279178082\\
63.875	0.1479	-214.807021698187	-214.807021698187\\
63.875	0.15156	-230.908504583107	-230.908504583107\\
63.875	0.15522	-247.831727832842	-247.831727832842\\
63.875	0.15888	-265.576691447392	-265.576691447392\\
63.875	0.16254	-284.143395426757	-284.143395426757\\
63.875	0.1662	-303.531839770938	-303.531839770938\\
63.875	0.16986	-323.742024479933	-323.742024479933\\
63.875	0.17352	-344.773949553744	-344.773949553744\\
63.875	0.17718	-366.62761499237	-366.62761499237\\
63.875	0.18084	-389.303020795811	-389.303020795811\\
63.875	0.1845	-412.800166964068	-412.800166964068\\
63.875	0.18816	-437.119053497139	-437.119053497139\\
63.875	0.19182	-462.259680395025	-462.259680395025\\
63.875	0.19548	-488.222047657727	-488.222047657727\\
63.875	0.19914	-515.006155285244	-515.006155285244\\
63.875	0.2028	-542.612003277577	-542.612003277577\\
63.875	0.20646	-571.039591634724	-571.039591634724\\
63.875	0.21012	-600.288920356686	-600.288920356686\\
63.875	0.21378	-630.359989443464	-630.359989443464\\
63.875	0.21744	-661.252798895056	-661.252798895056\\
63.875	0.2211	-692.967348711464	-692.967348711464\\
63.875	0.22476	-725.503638892687	-725.503638892687\\
63.875	0.22842	-758.861669438726	-758.861669438726\\
63.875	0.23208	-793.041440349579	-793.041440349579\\
63.875	0.23574	-828.042951625248	-828.042951625248\\
63.875	0.2394	-863.866203265731	-863.866203265731\\
63.875	0.24306	-900.51119527103	-900.51119527103\\
63.875	0.24672	-937.977927641145	-937.977927641145\\
63.875	0.25038	-976.266400376074	-976.266400376074\\
63.875	0.25404	-1015.37661347582	-1015.37661347582\\
63.875	0.2577	-1055.30856694038	-1055.30856694038\\
63.875	0.26136	-1096.06226076975	-1096.06226076975\\
63.875	0.26502	-1137.63769496394	-1137.63769496394\\
63.875	0.26868	-1180.03486952295	-1180.03486952295\\
63.875	0.27234	-1223.25378444677	-1223.25378444677\\
63.875	0.276	-1267.2944397354	-1267.2944397354\\
64.25	0.093	-72.6211290366915	-72.6211290366915\\
64.25	0.09666	-76.408469107197	-76.408469107197\\
64.25	0.10032	-81.0175495425174	-81.0175495425174\\
64.25	0.10398	-86.4483703426532	-86.4483703426532\\
64.25	0.10764	-92.7009315076044	-92.7009315076044\\
64.25	0.1113	-99.7752330373703	-99.7752330373703\\
64.25	0.11496	-107.671274931952	-107.671274931952\\
64.25	0.11862	-116.389057191348	-116.389057191348\\
64.25	0.12228	-125.928579815559	-125.928579815559\\
64.25	0.12594	-136.289842804586	-136.289842804586\\
64.25	0.1296	-147.472846158428	-147.472846158428\\
64.25	0.13326	-159.477589877085	-159.477589877085\\
64.25	0.13692	-172.304073960557	-172.304073960557\\
64.25	0.14058	-185.952298408845	-185.952298408845\\
64.25	0.14424	-200.422263221947	-200.422263221947\\
64.25	0.1479	-215.713968399865	-215.713968399865\\
64.25	0.15156	-231.827413942598	-231.827413942598\\
64.25	0.15522	-248.762599850146	-248.762599850146\\
64.25	0.15888	-266.519526122509	-266.519526122509\\
64.25	0.16254	-285.098192759688	-285.098192759688\\
64.25	0.1662	-304.498599761681	-304.498599761681\\
64.25	0.16986	-324.72074712849	-324.72074712849\\
64.25	0.17352	-345.764634860113	-345.764634860113\\
64.25	0.17718	-367.630262956553	-367.630262956553\\
64.25	0.18084	-390.317631417807	-390.317631417807\\
64.25	0.1845	-413.826740243876	-413.826740243876\\
64.25	0.18816	-438.157589434761	-438.157589434761\\
64.25	0.19182	-463.310178990461	-463.310178990461\\
64.25	0.19548	-489.284508910975	-489.284508910975\\
64.25	0.19914	-516.080579196305	-516.080579196305\\
64.25	0.2028	-543.698389846451	-543.698389846451\\
64.25	0.20646	-572.137940861411	-572.137940861411\\
64.25	0.21012	-601.399232241186	-601.399232241186\\
64.25	0.21378	-631.482263985777	-631.482263985777\\
64.25	0.21744	-662.387036095183	-662.387036095183\\
64.25	0.2211	-694.113548569404	-694.113548569404\\
64.25	0.22476	-726.66180140844	-726.66180140844\\
64.25	0.22842	-760.031794612291	-760.031794612291\\
64.25	0.23208	-794.223528180958	-794.223528180958\\
64.25	0.23574	-829.23700211444	-829.23700211444\\
64.25	0.2394	-865.072216412736	-865.072216412736\\
64.25	0.24306	-901.729171075848	-901.729171075848\\
64.25	0.24672	-939.207866103776	-939.207866103776\\
64.25	0.25038	-977.508301496518	-977.508301496518\\
64.25	0.25404	-1016.63047725408	-1016.63047725408\\
64.25	0.2577	-1056.57439337645	-1056.57439337645\\
64.25	0.26136	-1097.34004986364	-1097.34004986364\\
64.25	0.26502	-1138.92744671564	-1138.92744671564\\
64.25	0.26868	-1181.33658393246	-1181.33658393246\\
64.25	0.27234	-1224.56746151409	-1224.56746151409\\
64.25	0.276	-1268.62007946054	-1268.62007946054\\
64.625	0.093	-73.4196694280737	-73.4196694280737\\
64.625	0.09666	-77.2189721563924	-77.2189721563924\\
64.625	0.10032	-81.8400152495258	-81.8400152495258\\
64.625	0.10398	-87.2827987074748	-87.2827987074748\\
64.625	0.10764	-93.5473225302388	-93.5473225302388\\
64.625	0.1113	-100.633586717818	-100.633586717818\\
64.625	0.11496	-108.541591270212	-108.541591270212\\
64.625	0.11862	-117.271336187422	-117.271336187422\\
64.625	0.12228	-126.822821469446	-126.822821469446\\
64.625	0.12594	-137.196047116286	-137.196047116286\\
64.625	0.1296	-148.391013127941	-148.391013127941\\
64.625	0.13326	-160.407719504411	-160.407719504411\\
64.625	0.13692	-173.246166245696	-173.246166245696\\
64.625	0.14058	-186.906353351797	-186.906353351797\\
64.625	0.14424	-201.388280822712	-201.388280822712\\
64.625	0.1479	-216.691948658443	-216.691948658443\\
64.625	0.15156	-232.817356858989	-232.817356858989\\
64.625	0.15522	-249.764505424351	-249.764505424351\\
64.625	0.15888	-267.533394354527	-267.533394354527\\
64.625	0.16254	-286.124023649518	-286.124023649518\\
64.625	0.1662	-305.536393309325	-305.536393309325\\
64.625	0.16986	-325.770503333947	-325.770503333947\\
64.625	0.17352	-346.826353723383	-346.826353723383\\
64.625	0.17718	-368.703944477636	-368.703944477636\\
64.625	0.18084	-391.403275596703	-391.403275596703\\
64.625	0.1845	-414.924347080585	-414.924347080585\\
64.625	0.18816	-439.267158929283	-439.267158929283\\
64.625	0.19182	-464.431711142796	-464.431711142796\\
64.625	0.19548	-490.418003721124	-490.418003721124\\
64.625	0.19914	-517.226036664267	-517.226036664267\\
64.625	0.2028	-544.855809972225	-544.855809972225\\
64.625	0.20646	-573.307323644999	-573.307323644999\\
64.625	0.21012	-602.580577682587	-602.580577682587\\
64.625	0.21378	-632.675572084991	-632.675572084991\\
64.625	0.21744	-663.59230685221	-663.59230685221\\
64.625	0.2211	-695.330781984244	-695.330781984244\\
64.625	0.22476	-727.890997481093	-727.890997481093\\
64.625	0.22842	-761.272953342757	-761.272953342757\\
64.625	0.23208	-795.476649569237	-795.476649569237\\
64.625	0.23574	-830.502086160532	-830.502086160532\\
64.625	0.2394	-866.349263116642	-866.349263116642\\
64.625	0.24306	-903.018180437567	-903.018180437567\\
64.625	0.24672	-940.508838123307	-940.508838123307\\
64.625	0.25038	-978.821236173863	-978.821236173863\\
64.625	0.25404	-1017.95537458923	-1017.95537458923\\
64.625	0.2577	-1057.91125336942	-1057.91125336942\\
64.625	0.26136	-1098.68887251442	-1098.68887251442\\
64.625	0.26502	-1140.28823202424	-1140.28823202424\\
64.625	0.26868	-1182.70933189887	-1182.70933189887\\
64.625	0.27234	-1225.95217213831	-1225.95217213831\\
64.625	0.276	-1270.01675274257	-1270.01675274257\\
65	0.093	-74.2892433763569	-74.2892433763569\\
65	0.09666	-78.1005087624884	-78.1005087624884\\
65	0.10032	-82.7335145134353	-82.7335145134353\\
65	0.10398	-88.1882606291973	-88.1882606291973\\
65	0.10764	-94.464747109774	-94.464747109774\\
65	0.1113	-101.562973955166	-101.562973955166\\
65	0.11496	-109.482941165374	-109.482941165374\\
65	0.11862	-118.224648740396	-118.224648740396\\
65	0.12228	-127.788096680234	-127.788096680234\\
65	0.12594	-138.173284984887	-138.173284984887\\
65	0.1296	-149.380213654355	-149.380213654355\\
65	0.13326	-161.408882688638	-161.408882688638\\
65	0.13692	-174.259292087737	-174.259292087737\\
65	0.14058	-187.93144185165	-187.93144185165\\
65	0.14424	-202.425331980379	-202.425331980379\\
65	0.1479	-217.740962473923	-217.740962473923\\
65	0.15156	-233.878333332282	-233.878333332282\\
65	0.15522	-250.837444555456	-250.837444555456\\
65	0.15888	-268.618296143445	-268.618296143445\\
65	0.16254	-287.22088809625	-287.22088809625\\
65	0.1662	-306.64522041387	-306.64522041387\\
65	0.16986	-326.891293096304	-326.891293096304\\
65	0.17352	-347.959106143555	-347.959106143555\\
65	0.17718	-369.84865955562	-369.84865955562\\
65	0.18084	-392.5599533325	-392.5599533325\\
65	0.1845	-416.092987474196	-416.092987474196\\
65	0.18816	-440.447761980706	-440.447761980706\\
65	0.19182	-465.624276852032	-465.624276852032\\
65	0.19548	-491.622532088173	-491.622532088173\\
65	0.19914	-518.442527689129	-518.442527689129\\
65	0.2028	-546.084263654901	-546.084263654901\\
65	0.20646	-574.547739985487	-574.547739985487\\
65	0.21012	-603.832956680889	-603.832956680889\\
65	0.21378	-633.939913741106	-633.939913741106\\
65	0.21744	-664.868611166138	-664.868611166138\\
65	0.2211	-696.619048955985	-696.619048955985\\
65	0.22476	-729.191227110647	-729.191227110647\\
65	0.22842	-762.585145630125	-762.585145630125\\
65	0.23208	-796.800804514417	-796.800804514417\\
65	0.23574	-831.838203763525	-831.838203763525\\
65	0.2394	-867.697343377448	-867.697343377448\\
65	0.24306	-904.378223356186	-904.378223356186\\
65	0.24672	-941.88084369974	-941.88084369974\\
65	0.25038	-980.205204408108	-980.205204408108\\
65	0.25404	-1019.35130548129	-1019.35130548129\\
65	0.2577	-1059.31914691929	-1059.31914691929\\
65	0.26136	-1100.1087287221	-1100.1087287221\\
65	0.26502	-1141.72005088973	-1141.72005088973\\
65	0.26868	-1184.15311342218	-1184.15311342218\\
65	0.27234	-1227.40791631944	-1227.40791631944\\
65	0.276	-1271.48445958151	-1271.48445958151\\
65.375	0.093	-75.2298508815403	-75.2298508815403\\
65.375	0.09666	-79.053078925485	-79.053078925485\\
65.375	0.10032	-83.6980473342449	-83.6980473342449\\
65.375	0.10398	-89.1647561078199	-89.1647561078199\\
65.375	0.10764	-95.4532052462098	-95.4532052462098\\
65.375	0.1113	-102.563394749415	-102.563394749415\\
65.375	0.11496	-110.495324617436	-110.495324617436\\
65.375	0.11862	-119.248994850271	-119.248994850271\\
65.375	0.12228	-128.824405447922	-128.824405447922\\
65.375	0.12594	-139.221556410388	-139.221556410388\\
65.375	0.1296	-150.440447737669	-150.440447737669\\
65.375	0.13326	-162.481079429766	-162.481079429766\\
65.375	0.13692	-175.343451486677	-175.343451486677\\
65.375	0.14058	-189.027563908404	-189.027563908404\\
65.375	0.14424	-203.533416694946	-203.533416694946\\
65.375	0.1479	-218.861009846302	-218.861009846302\\
65.375	0.15156	-235.010343362475	-235.010343362475\\
65.375	0.15522	-251.981417243462	-251.981417243462\\
65.375	0.15888	-269.774231489264	-269.774231489264\\
65.375	0.16254	-288.388786099882	-288.388786099882\\
65.375	0.1662	-307.825081075315	-307.825081075315\\
65.375	0.16986	-328.083116415562	-328.083116415562\\
65.375	0.17352	-349.162892120626	-349.162892120626\\
65.375	0.17718	-371.064408190504	-371.064408190504\\
65.375	0.18084	-393.787664625197	-393.787664625197\\
65.375	0.1845	-417.332661424706	-417.332661424706\\
65.375	0.18816	-441.69939858903	-441.69939858903\\
65.375	0.19182	-466.887876118169	-466.887876118169\\
65.375	0.19548	-492.898094012123	-492.898094012123\\
65.375	0.19914	-519.730052270892	-519.730052270892\\
65.375	0.2028	-547.383750894477	-547.383750894477\\
65.375	0.20646	-575.859189882876	-575.859189882876\\
65.375	0.21012	-605.156369236091	-605.156369236091\\
65.375	0.21378	-635.275288954121	-635.275288954121\\
65.375	0.21744	-666.215949036966	-666.215949036966\\
65.375	0.2211	-697.978349484626	-697.978349484626\\
65.375	0.22476	-730.562490297101	-730.562490297101\\
65.375	0.22842	-763.968371474392	-763.968371474392\\
65.375	0.23208	-798.195993016497	-798.195993016497\\
65.375	0.23574	-833.245354923419	-833.245354923419\\
65.375	0.2394	-869.116457195155	-869.116457195155\\
65.375	0.24306	-905.809299831706	-905.809299831706\\
65.375	0.24672	-943.323882833073	-943.323882833073\\
65.375	0.25038	-981.660206199254	-981.660206199254\\
65.375	0.25404	-1020.81826993025	-1020.81826993025\\
65.375	0.2577	-1060.79807402606	-1060.79807402606\\
65.375	0.26136	-1101.59961848669	-1101.59961848669\\
65.375	0.26502	-1143.22290331213	-1143.22290331213\\
65.375	0.26868	-1185.66792850239	-1185.66792850239\\
65.375	0.27234	-1228.93469405746	-1228.93469405746\\
65.375	0.276	-1273.02319997735	-1273.02319997735\\
65.75	0.093	-76.241491943625	-76.241491943625\\
65.75	0.09666	-80.0766826453827	-80.0766826453827\\
65.75	0.10032	-84.7336137119553	-84.7336137119553\\
65.75	0.10398	-90.2122851433435	-90.2122851433435\\
65.75	0.10764	-96.5126969395467	-96.5126969395467\\
65.75	0.1113	-103.634849100565	-103.634849100565\\
65.75	0.11496	-111.578741626399	-111.578741626399\\
65.75	0.11862	-120.344374517047	-120.344374517047\\
65.75	0.12228	-129.931747772511	-129.931747772511\\
65.75	0.12594	-140.340861392791	-140.340861392791\\
65.75	0.1296	-151.571715377884	-151.571715377884\\
65.75	0.13326	-163.624309727794	-163.624309727794\\
65.75	0.13692	-176.498644442518	-176.498644442518\\
65.75	0.14058	-190.194719522058	-190.194719522058\\
65.75	0.14424	-204.712534966413	-204.712534966413\\
65.75	0.1479	-220.052090775583	-220.052090775583\\
65.75	0.15156	-236.213386949568	-236.213386949568\\
65.75	0.15522	-253.196423488369	-253.196423488369\\
65.75	0.15888	-271.001200391984	-271.001200391984\\
65.75	0.16254	-289.627717660415	-289.627717660415\\
65.75	0.1662	-309.075975293661	-309.075975293661\\
65.75	0.16986	-329.345973291722	-329.345973291722\\
65.75	0.17352	-350.437711654598	-350.437711654598\\
65.75	0.17718	-372.351190382289	-372.351190382289\\
65.75	0.18084	-395.086409474796	-395.086409474796\\
65.75	0.1845	-418.643368932117	-418.643368932117\\
65.75	0.18816	-443.022068754254	-443.022068754254\\
65.75	0.19182	-468.222508941206	-468.222508941206\\
65.75	0.19548	-494.244689492974	-494.244689492974\\
65.75	0.19914	-521.088610409556	-521.088610409556\\
65.75	0.2028	-548.754271690954	-548.754271690954\\
65.75	0.20646	-577.241673337166	-577.241673337166\\
65.75	0.21012	-606.550815348194	-606.550815348194\\
65.75	0.21378	-636.681697724037	-636.681697724037\\
65.75	0.21744	-667.634320464695	-667.634320464695\\
65.75	0.2211	-699.408683570168	-699.408683570168\\
65.75	0.22476	-732.004787040456	-732.004787040456\\
65.75	0.22842	-765.42263087556	-765.42263087556\\
65.75	0.23208	-799.662215075479	-799.662215075479\\
65.75	0.23574	-834.723539640213	-834.723539640213\\
65.75	0.2394	-870.606604569763	-870.606604569763\\
65.75	0.24306	-907.311409864127	-907.311409864127\\
65.75	0.24672	-944.837955523306	-944.837955523306\\
65.75	0.25038	-983.186241547301	-983.186241547301\\
65.75	0.25404	-1022.35626793611	-1022.35626793611\\
65.75	0.2577	-1062.34803468974	-1062.34803468974\\
65.75	0.26136	-1103.16154180818	-1103.16154180818\\
65.75	0.26502	-1144.79678929143	-1144.79678929143\\
65.75	0.26868	-1187.2537771395	-1187.2537771395\\
65.75	0.27234	-1230.53250535239	-1230.53250535239\\
65.75	0.276	-1274.63297393009	-1274.63297393009\\
66.125	0.093	-77.3241665626098	-77.3241665626098\\
66.125	0.09666	-81.1713199221805	-81.1713199221805\\
66.125	0.10032	-85.8402136465664	-85.8402136465664\\
66.125	0.10398	-91.3308477357676	-91.3308477357676\\
66.125	0.10764	-97.643222189784	-97.643222189784\\
66.125	0.1113	-104.777337008615	-104.777337008615\\
66.125	0.11496	-112.733192192262	-112.733192192262\\
66.125	0.11862	-121.510787740724	-121.510787740724\\
66.125	0.12228	-131.110123654001	-131.110123654001\\
66.125	0.12594	-141.531199932093	-141.531199932093\\
66.125	0.1296	-152.774016575	-152.774016575\\
66.125	0.13326	-164.838573582723	-164.838573582723\\
66.125	0.13692	-177.72487095526	-177.72487095526\\
66.125	0.14058	-191.432908692613	-191.432908692613\\
66.125	0.14424	-205.962686794781	-205.962686794781\\
66.125	0.1479	-221.314205261764	-221.314205261764\\
66.125	0.15156	-237.487464093562	-237.487464093562\\
66.125	0.15522	-254.482463290176	-254.482463290176\\
66.125	0.15888	-272.299202851604	-272.299202851604\\
66.125	0.16254	-290.937682777848	-290.937682777848\\
66.125	0.1662	-310.397903068907	-310.397903068907\\
66.125	0.16986	-330.679863724781	-330.679863724781\\
66.125	0.17352	-351.78356474547	-351.78356474547\\
66.125	0.17718	-373.709006130975	-373.709006130975\\
66.125	0.18084	-396.456187881294	-396.456187881294\\
66.125	0.1845	-420.025109996429	-420.025109996429\\
66.125	0.18816	-444.415772476379	-444.415772476379\\
66.125	0.19182	-469.628175321144	-469.628175321144\\
66.125	0.19548	-495.662318530724	-495.662318530724\\
66.125	0.19914	-522.51820210512	-522.51820210512\\
66.125	0.2028	-550.195826044331	-550.195826044331\\
66.125	0.20646	-578.695190348356	-578.695190348356\\
66.125	0.21012	-608.016295017197	-608.016295017197\\
66.125	0.21378	-638.159140050853	-638.159140050853\\
66.125	0.21744	-669.123725449324	-669.123725449324\\
66.125	0.2211	-700.910051212611	-700.910051212611\\
66.125	0.22476	-733.518117340712	-733.518117340712\\
66.125	0.22842	-766.947923833629	-766.947923833629\\
66.125	0.23208	-801.199470691361	-801.199470691361\\
66.125	0.23574	-836.272757913908	-836.272757913908\\
66.125	0.2394	-872.167785501271	-872.167785501271\\
66.125	0.24306	-908.884553453448	-908.884553453448\\
66.125	0.24672	-946.42306177044	-946.42306177044\\
66.125	0.25038	-984.783310452248	-984.783310452248\\
66.125	0.25404	-1023.96529949887	-1023.96529949887\\
66.125	0.2577	-1063.96902891031	-1063.96902891031\\
66.125	0.26136	-1104.79449868656	-1104.79449868656\\
66.125	0.26502	-1146.44170882763	-1146.44170882763\\
66.125	0.26868	-1188.91065933351	-1188.91065933351\\
66.125	0.27234	-1232.20135020421	-1232.20135020421\\
66.125	0.276	-1276.31378143973	-1276.31378143973\\
66.5	0.093	-78.4778747384952	-78.4778747384952\\
66.5	0.09666	-82.3369907558791	-82.3369907558791\\
66.5	0.10032	-87.017847138078	-87.017847138078\\
66.5	0.10398	-92.5204438850922	-92.5204438850922\\
66.5	0.10764	-98.8447809969217	-98.8447809969217\\
66.5	0.1113	-105.990858473566	-105.990858473566\\
66.5	0.11496	-113.958676315026	-113.958676315026\\
66.5	0.11862	-122.748234521301	-122.748234521301\\
66.5	0.12228	-132.359533092391	-132.359533092391\\
66.5	0.12594	-142.792572028296	-142.792572028296\\
66.5	0.1296	-154.047351329016	-154.047351329016\\
66.5	0.13326	-166.123870994552	-166.123870994552\\
66.5	0.13692	-179.022131024903	-179.022131024903\\
66.5	0.14058	-192.742131420069	-192.742131420069\\
66.5	0.14424	-207.283872180049	-207.283872180049\\
66.5	0.1479	-222.647353304846	-222.647353304846\\
66.5	0.15156	-238.832574794457	-238.832574794457\\
66.5	0.15522	-255.839536648883	-255.839536648883\\
66.5	0.15888	-273.668238868125	-273.668238868125\\
66.5	0.16254	-292.318681452182	-292.318681452182\\
66.5	0.1662	-311.790864401054	-311.790864401054\\
66.5	0.16986	-332.084787714741	-332.084787714741\\
66.5	0.17352	-353.200451393243	-353.200451393243\\
66.5	0.17718	-375.137855436561	-375.137855436561\\
66.5	0.18084	-397.896999844694	-397.896999844694\\
66.5	0.1845	-421.477884617642	-421.477884617642\\
66.5	0.18816	-445.880509755405	-445.880509755405\\
66.5	0.19182	-471.104875257983	-471.104875257983\\
66.5	0.19548	-497.150981125376	-497.150981125376\\
66.5	0.19914	-524.018827357585	-524.018827357585\\
66.5	0.2028	-551.708413954608	-551.708413954608\\
66.5	0.20646	-580.219740916447	-580.219740916447\\
66.5	0.21012	-609.552808243101	-609.552808243101\\
66.5	0.21378	-639.70761593457	-639.70761593457\\
66.5	0.21744	-670.684163990855	-670.684163990855\\
66.5	0.2211	-702.482452411954	-702.482452411954\\
66.5	0.22476	-735.102481197868	-735.102481197868\\
66.5	0.22842	-768.544250348598	-768.544250348598\\
66.5	0.23208	-802.807759864143	-802.807759864143\\
66.5	0.23574	-837.893009744504	-837.893009744504\\
66.5	0.2394	-873.799999989679	-873.799999989679\\
66.5	0.24306	-910.528730599669	-910.528730599669\\
66.5	0.24672	-948.079201574475	-948.079201574475\\
66.5	0.25038	-986.451412914096	-986.451412914096\\
66.5	0.25404	-1025.64536461853	-1025.64536461853\\
66.5	0.2577	-1065.66105668778	-1065.66105668778\\
66.5	0.26136	-1106.49848912185	-1106.49848912185\\
66.5	0.26502	-1148.15766192073	-1148.15766192073\\
66.5	0.26868	-1190.63857508443	-1190.63857508443\\
66.5	0.27234	-1233.94122861294	-1233.94122861294\\
66.5	0.276	-1278.06562250627	-1278.06562250627\\
66.875	0.093	-79.7026164712817	-79.7026164712817\\
66.875	0.09666	-83.5736951464786	-83.5736951464786\\
66.875	0.10032	-88.2665141864907	-88.2665141864907\\
66.875	0.10398	-93.7810735913179	-93.7810735913179\\
66.875	0.10764	-100.117373360961	-100.117373360961\\
66.875	0.1113	-107.275413495418	-107.275413495418\\
66.875	0.11496	-115.255193994691	-115.255193994691\\
66.875	0.11862	-124.056714858779	-124.056714858779\\
66.875	0.12228	-133.679976087682	-133.679976087682\\
66.875	0.12594	-144.1249776814	-144.1249776814\\
66.875	0.1296	-155.391719639934	-155.391719639934\\
66.875	0.13326	-167.480201963282	-167.480201963282\\
66.875	0.13692	-180.390424651446	-180.390424651446\\
66.875	0.14058	-194.122387704425	-194.122387704425\\
66.875	0.14424	-208.676091122219	-208.676091122219\\
66.875	0.1479	-224.051534904828	-224.051534904828\\
66.875	0.15156	-240.248719052253	-240.248719052253\\
66.875	0.15522	-257.267643564492	-257.267643564492\\
66.875	0.15888	-275.108308441547	-275.108308441547\\
66.875	0.16254	-293.770713683417	-293.770713683417\\
66.875	0.1662	-313.254859290102	-313.254859290102\\
66.875	0.16986	-333.560745261603	-333.560745261603\\
66.875	0.17352	-354.688371597918	-354.688371597918\\
66.875	0.17718	-376.637738299049	-376.637738299049\\
66.875	0.18084	-399.408845364994	-399.408845364994\\
66.875	0.1845	-423.001692795755	-423.001692795755\\
66.875	0.18816	-447.416280591331	-447.416280591331\\
66.875	0.19182	-472.652608751722	-472.652608751722\\
66.875	0.19548	-498.710677276929	-498.710677276929\\
66.875	0.19914	-525.59048616695	-525.59048616695\\
66.875	0.2028	-553.292035421787	-553.292035421787\\
66.875	0.20646	-581.815325041439	-581.815325041439\\
66.875	0.21012	-611.160355025906	-611.160355025906\\
66.875	0.21378	-641.327125375188	-641.327125375188\\
66.875	0.21744	-672.315636089286	-672.315636089286\\
66.875	0.2211	-704.125887168198	-704.125887168198\\
66.875	0.22476	-736.757878611926	-736.757878611926\\
66.875	0.22842	-770.211610420469	-770.211610420469\\
66.875	0.23208	-804.487082593827	-804.487082593827\\
66.875	0.23574	-839.584295132	-839.584295132\\
66.875	0.2394	-875.503248034988	-875.503248034988\\
66.875	0.24306	-912.243941302792	-912.243941302792\\
66.875	0.24672	-949.806374935411	-949.806374935411\\
66.875	0.25038	-988.190548932845	-988.190548932845\\
66.875	0.25404	-1027.39646329509	-1027.39646329509\\
66.875	0.2577	-1067.42411802216	-1067.42411802216\\
66.875	0.26136	-1108.27351311404	-1108.27351311404\\
66.875	0.26502	-1149.94464857073	-1149.94464857073\\
66.875	0.26868	-1192.43752439224	-1192.43752439224\\
66.875	0.27234	-1235.75214057857	-1235.75214057857\\
66.875	0.276	-1279.88849712971	-1279.88849712971\\
67.25	0.093	-80.9983917609685	-80.9983917609685\\
67.25	0.09666	-84.8814330939787	-84.8814330939787\\
67.25	0.10032	-89.5862147918038	-89.5862147918038\\
67.25	0.10398	-95.112736854444	-95.112736854444\\
67.25	0.10764	-101.4609992819	-101.4609992819\\
67.25	0.1113	-108.63100207417	-108.63100207417\\
67.25	0.11496	-116.622745231256	-116.622745231256\\
67.25	0.11862	-125.436228753157	-125.436228753157\\
67.25	0.12228	-135.071452639873	-135.071452639873\\
67.25	0.12594	-145.528416891405	-145.528416891405\\
67.25	0.1296	-156.807121507751	-156.807121507751\\
67.25	0.13326	-168.907566488913	-168.907566488913\\
67.25	0.13692	-181.82975183489	-181.82975183489\\
67.25	0.14058	-195.573677545682	-195.573677545682\\
67.25	0.14424	-210.139343621289	-210.139343621289\\
67.25	0.1479	-225.526750061711	-225.526750061711\\
67.25	0.15156	-241.735896866949	-241.735896866949\\
67.25	0.15522	-258.766784037002	-258.766784037002\\
67.25	0.15888	-276.619411571869	-276.619411571869\\
67.25	0.16254	-295.293779471553	-295.293779471553\\
67.25	0.1662	-314.789887736051	-314.789887736051\\
67.25	0.16986	-335.107736365364	-335.107736365364\\
67.25	0.17352	-356.247325359492	-356.247325359492\\
67.25	0.17718	-378.208654718436	-378.208654718436\\
67.25	0.18084	-400.991724442195	-400.991724442195\\
67.25	0.1845	-424.596534530769	-424.596534530769\\
67.25	0.18816	-449.023084984158	-449.023084984158\\
67.25	0.19182	-474.271375802363	-474.271375802363\\
67.25	0.19548	-500.341406985382	-500.341406985382\\
67.25	0.19914	-527.233178533216	-527.233178533216\\
67.25	0.2028	-554.946690445867	-554.946690445867\\
67.25	0.20646	-583.481942723331	-583.481942723331\\
67.25	0.21012	-612.838935365612	-612.838935365612\\
67.25	0.21378	-643.017668372707	-643.017668372707\\
67.25	0.21744	-674.018141744617	-674.018141744617\\
67.25	0.2211	-705.840355481343	-705.840355481343\\
67.25	0.22476	-738.484309582883	-738.484309582883\\
67.25	0.22842	-771.95000404924	-771.95000404924\\
67.25	0.23208	-806.237438880411	-806.237438880411\\
67.25	0.23574	-841.346614076397	-841.346614076397\\
67.25	0.2394	-877.277529637198	-877.277529637198\\
67.25	0.24306	-914.030185562815	-914.030185562815\\
67.25	0.24672	-951.604581853248	-951.604581853248\\
67.25	0.25038	-990.000718508494	-990.000718508494\\
67.25	0.25404	-1029.21859552856	-1029.21859552856\\
67.25	0.2577	-1069.25821291343	-1069.25821291343\\
67.25	0.26136	-1110.11957066313	-1110.11957066313\\
67.25	0.26502	-1151.80266877763	-1151.80266877763\\
67.25	0.26868	-1194.30750725696	-1194.30750725696\\
67.25	0.27234	-1237.63408610109	-1237.63408610109\\
67.25	0.276	-1281.78240531005	-1281.78240531005\\
67.625	0.093	-82.3652006075559	-82.3652006075559\\
67.625	0.09666	-86.2602045983791	-86.2602045983791\\
67.625	0.10032	-90.9769489540172	-90.9769489540172\\
67.625	0.10398	-96.5154336744709	-96.5154336744709\\
67.625	0.10764	-102.87565875974	-102.87565875974\\
67.625	0.1113	-110.057624209823	-110.057624209823\\
67.625	0.11496	-118.061330024722	-118.061330024722\\
67.625	0.11862	-126.886776204436	-126.886776204436\\
67.625	0.12228	-136.533962748965	-136.533962748965\\
67.625	0.12594	-147.00288965831	-147.00288965831\\
67.625	0.1296	-158.29355693247	-158.29355693247\\
67.625	0.13326	-170.405964571444	-170.405964571444\\
67.625	0.13692	-183.340112575234	-183.340112575234\\
67.625	0.14058	-197.096000943839	-197.096000943839\\
67.625	0.14424	-211.67362967726	-211.67362967726\\
67.625	0.1479	-227.072998775495	-227.072998775495\\
67.625	0.15156	-243.294108238546	-243.294108238546\\
67.625	0.15522	-260.336958066411	-260.336958066411\\
67.625	0.15888	-278.201548259092	-278.201548259092\\
67.625	0.16254	-296.887878816589	-296.887878816589\\
67.625	0.1662	-316.3959497389	-316.3959497389\\
67.625	0.16986	-336.725761026026	-336.725761026026\\
67.625	0.17352	-357.877312677968	-357.877312677968\\
67.625	0.17718	-379.850604694725	-379.850604694725\\
67.625	0.18084	-402.645637076296	-402.645637076296\\
67.625	0.1845	-426.262409822684	-426.262409822684\\
67.625	0.18816	-450.700922933886	-450.700922933886\\
67.625	0.19182	-475.961176409903	-475.961176409903\\
67.625	0.19548	-502.043170250736	-502.043170250736\\
67.625	0.19914	-528.946904456383	-528.946904456383\\
67.625	0.2028	-556.672379026846	-556.672379026846\\
67.625	0.20646	-585.219593962124	-585.219593962124\\
67.625	0.21012	-614.588549262218	-614.588549262218\\
67.625	0.21378	-644.779244927126	-644.779244927126\\
67.625	0.21744	-675.791680956849	-675.791680956849\\
67.625	0.2211	-707.625857351388	-707.625857351388\\
67.625	0.22476	-740.281774110742	-740.281774110742\\
67.625	0.22842	-773.759431234911	-773.759431234911\\
67.625	0.23208	-808.058828723895	-808.058828723895\\
67.625	0.23574	-843.179966577695	-843.179966577695\\
67.625	0.2394	-879.122844796309	-879.122844796309\\
67.625	0.24306	-915.887463379739	-915.887463379739\\
67.625	0.24672	-953.473822327984	-953.473822327984\\
67.625	0.25038	-991.881921641044	-991.881921641044\\
67.625	0.25404	-1031.11176131892	-1031.11176131892\\
67.625	0.2577	-1071.16334136161	-1071.16334136161\\
67.625	0.26136	-1112.03666176911	-1112.03666176911\\
67.625	0.26502	-1153.73172254144	-1153.73172254144\\
67.625	0.26868	-1196.24852367857	-1196.24852367857\\
67.625	0.27234	-1239.58706518052	-1239.58706518052\\
67.625	0.276	-1283.74734704729	-1283.74734704729\\
68	0.093	-83.803043011044	-83.803043011044\\
68	0.09666	-87.7100096596804	-87.7100096596804\\
68	0.10032	-92.4387166731315	-92.4387166731315\\
68	0.10398	-97.9891640513982	-97.9891640513982\\
68	0.10764	-104.36135179448	-104.36135179448\\
68	0.1113	-111.555279902377	-111.555279902377\\
68	0.11496	-119.570948375089	-119.570948375089\\
68	0.11862	-128.408357212616	-128.408357212616\\
68	0.12228	-138.067506414958	-138.067506414958\\
68	0.12594	-148.548395982116	-148.548395982116\\
68	0.1296	-159.851025914088	-159.851025914088\\
68	0.13326	-171.975396210876	-171.975396210876\\
68	0.13692	-184.921506872479	-184.921506872479\\
68	0.14058	-198.689357898897	-198.689357898897\\
68	0.14424	-213.278949290131	-213.278949290131\\
68	0.1479	-228.690281046179	-228.690281046179\\
68	0.15156	-244.923353167043	-244.923353167043\\
68	0.15522	-261.978165652722	-261.978165652722\\
68	0.15888	-279.854718503216	-279.854718503216\\
68	0.16254	-298.553011718525	-298.553011718525\\
68	0.1662	-318.07304529865	-318.07304529865\\
68	0.16986	-338.414819243589	-338.414819243589\\
68	0.17352	-359.578333553343	-359.578333553343\\
68	0.17718	-381.563588227913	-381.563588227913\\
68	0.18084	-404.370583267298	-404.370583267298\\
68	0.1845	-427.999318671498	-427.999318671498\\
68	0.18816	-452.449794440514	-452.449794440514\\
68	0.19182	-477.722010574344	-477.722010574344\\
68	0.19548	-503.81596707299	-503.81596707299\\
68	0.19914	-530.731663936451	-530.731663936451\\
68	0.2028	-558.469101164727	-558.469101164727\\
68	0.20646	-587.028278757818	-587.028278757818\\
68	0.21012	-616.409196715724	-616.409196715724\\
68	0.21378	-646.611855038446	-646.611855038446\\
68	0.21744	-677.636253725982	-677.636253725982\\
68	0.2211	-709.482392778334	-709.482392778334\\
68	0.22476	-742.150272195501	-742.150272195501\\
68	0.22842	-775.639891977483	-775.639891977483\\
68	0.23208	-809.95125212428	-809.95125212428\\
68	0.23574	-845.084352635893	-845.084352635893\\
68	0.2394	-881.03919351232	-881.03919351232\\
68	0.24306	-917.815774753563	-917.815774753563\\
68	0.24672	-955.414096359621	-955.414096359621\\
68	0.25038	-993.834158330495	-993.834158330495\\
68	0.25404	-1033.07596066618	-1033.07596066618\\
68	0.2577	-1073.13950336669	-1073.13950336669\\
68	0.26136	-1114.024786432	-1114.024786432\\
68	0.26502	-1155.73180986214	-1155.73180986214\\
68	0.26868	-1198.26057365709	-1198.26057365709\\
68	0.27234	-1241.61107781685	-1241.61107781685\\
68	0.276	-1285.78332234143	-1285.78332234143\\
68.375	0.093	-85.3119189714331	-85.3119189714331\\
68.375	0.09666	-89.2308482778825	-89.2308482778825\\
68.375	0.10032	-93.9715179491469	-93.9715179491469\\
68.375	0.10398	-99.5339279852265	-99.5339279852265\\
68.375	0.10764	-105.918078386121	-105.918078386121\\
68.375	0.1113	-113.123969151831	-113.123969151831\\
68.375	0.11496	-121.151600282356	-121.151600282356\\
68.375	0.11862	-130.000971777697	-130.000971777697\\
68.375	0.12228	-139.672083637852	-139.672083637852\\
68.375	0.12594	-150.164935862823	-150.164935862823\\
68.375	0.1296	-161.479528452608	-161.479528452608\\
68.375	0.13326	-173.615861407209	-173.615861407209\\
68.375	0.13692	-186.573934726625	-186.573934726625\\
68.375	0.14058	-200.353748410857	-200.353748410857\\
68.375	0.14424	-214.955302459903	-214.955302459903\\
68.375	0.1479	-230.378596873765	-230.378596873765\\
68.375	0.15156	-246.623631652441	-246.623631652441\\
68.375	0.15522	-263.690406795933	-263.690406795933\\
68.375	0.15888	-281.57892230424	-281.57892230424\\
68.375	0.16254	-300.289178177363	-300.289178177363\\
68.375	0.1662	-319.8211744153	-319.8211744153\\
68.375	0.16986	-340.174911018053	-340.174911018053\\
68.375	0.17352	-361.35038798562	-361.35038798562\\
68.375	0.17718	-383.347605318003	-383.347605318003\\
68.375	0.18084	-406.166563015201	-406.166563015201\\
68.375	0.1845	-429.807261077215	-429.807261077215\\
68.375	0.18816	-454.269699504043	-454.269699504043\\
68.375	0.19182	-479.553878295687	-479.553878295687\\
68.375	0.19548	-505.659797452145	-505.659797452145\\
68.375	0.19914	-532.587456973419	-532.587456973419\\
68.375	0.2028	-560.336856859508	-560.336856859508\\
68.375	0.20646	-588.907997110412	-588.907997110412\\
68.375	0.21012	-618.300877726132	-618.300877726132\\
68.375	0.21378	-648.515498706666	-648.515498706666\\
68.375	0.21744	-679.551860052016	-679.551860052016\\
68.375	0.2211	-711.409961762181	-711.409961762181\\
68.375	0.22476	-744.089803837161	-744.089803837161\\
68.375	0.22842	-777.591386276956	-777.591386276956\\
68.375	0.23208	-811.914709081566	-811.914709081566\\
68.375	0.23574	-847.059772250992	-847.059772250992\\
68.375	0.2394	-883.026575785233	-883.026575785233\\
68.375	0.24306	-919.815119684289	-919.815119684289\\
68.375	0.24672	-957.42540394816	-957.42540394816\\
68.375	0.25038	-995.857428576846	-995.857428576846\\
68.375	0.25404	-1035.11119357035	-1035.11119357035\\
68.375	0.2577	-1075.18669892866	-1075.18669892866\\
68.375	0.26136	-1116.0839446518	-1116.0839446518\\
68.375	0.26502	-1157.80293073974	-1157.80293073974\\
68.375	0.26868	-1200.3436571925	-1200.3436571925\\
68.375	0.27234	-1243.70612401008	-1243.70612401008\\
68.375	0.276	-1287.89033119247	-1287.89033119247\\
68.75	0.093	-86.8918284887224	-86.8918284887224\\
68.75	0.09666	-90.822720452985	-90.822720452985\\
68.75	0.10032	-95.5753527820623	-95.5753527820623\\
68.75	0.10398	-101.149725475955	-101.149725475955\\
68.75	0.10764	-107.545838534663	-107.545838534663\\
68.75	0.1113	-114.763691958186	-114.763691958186\\
68.75	0.11496	-122.803285746524	-122.803285746524\\
68.75	0.11862	-131.664619899677	-131.664619899677\\
68.75	0.12228	-141.347694417646	-141.347694417646\\
68.75	0.12594	-151.85250930043	-151.85250930043\\
68.75	0.1296	-163.179064548028	-163.179064548028\\
68.75	0.13326	-175.327360160442	-175.327360160442\\
68.75	0.13692	-188.297396137672	-188.297396137672\\
68.75	0.14058	-202.089172479716	-202.089172479716\\
68.75	0.14424	-216.702689186575	-216.702689186575\\
68.75	0.1479	-232.13794625825	-232.13794625825\\
68.75	0.15156	-248.39494369474	-248.39494369474\\
68.75	0.15522	-265.473681496045	-265.473681496045\\
68.75	0.15888	-283.374159662165	-283.374159662165\\
68.75	0.16254	-302.0963781931	-302.0963781931\\
68.75	0.1662	-321.640337088851	-321.640337088851\\
68.75	0.16986	-342.006036349417	-342.006036349417\\
68.75	0.17352	-363.193475974797	-363.193475974797\\
68.75	0.17718	-385.202655964993	-385.202655964993\\
68.75	0.18084	-408.033576320004	-408.033576320004\\
68.75	0.1845	-431.686237039831	-431.686237039831\\
68.75	0.18816	-456.160638124472	-456.160638124472\\
68.75	0.19182	-481.456779573929	-481.456779573929\\
68.75	0.19548	-507.574661388201	-507.574661388201\\
68.75	0.19914	-534.514283567288	-534.514283567288\\
68.75	0.2028	-562.27564611119	-562.27564611119\\
68.75	0.20646	-590.858749019907	-590.858749019907\\
68.75	0.21012	-620.263592293439	-620.263592293439\\
68.75	0.21378	-650.490175931787	-650.490175931787\\
68.75	0.21744	-681.53849993495	-681.53849993495\\
68.75	0.2211	-713.408564302928	-713.408564302928\\
68.75	0.22476	-746.100369035721	-746.100369035721\\
68.75	0.22842	-779.613914133329	-779.613914133329\\
68.75	0.23208	-813.949199595753	-813.949199595753\\
68.75	0.23574	-849.106225422991	-849.106225422991\\
68.75	0.2394	-885.084991615045	-885.084991615045\\
68.75	0.24306	-921.885498171914	-921.885498171914\\
68.75	0.24672	-959.507745093598	-959.507745093598\\
68.75	0.25038	-997.951732380098	-997.951732380098\\
68.75	0.25404	-1037.21746003141	-1037.21746003141\\
68.75	0.2577	-1077.30492804754	-1077.30492804754\\
68.75	0.26136	-1118.21413642849	-1118.21413642849\\
68.75	0.26502	-1159.94508517425	-1159.94508517425\\
68.75	0.26868	-1202.49777428482	-1202.49777428482\\
68.75	0.27234	-1245.87220376021	-1245.87220376021\\
68.75	0.276	-1290.06837360042	-1290.06837360042\\
69.125	0.093	-88.5427715629127	-88.5427715629127\\
69.125	0.09666	-92.4856261849881	-92.4856261849881\\
69.125	0.10032	-97.2502211718786	-97.2502211718786\\
69.125	0.10398	-102.836556523584	-102.836556523584\\
69.125	0.10764	-109.244632240105	-109.244632240105\\
69.125	0.1113	-116.474448321441	-116.474448321441\\
69.125	0.11496	-124.526004767593	-124.526004767593\\
69.125	0.11862	-133.399301578559	-133.399301578559\\
69.125	0.12228	-143.094338754341	-143.094338754341\\
69.125	0.12594	-153.611116294937	-153.611116294937\\
69.125	0.1296	-164.949634200349	-164.949634200349\\
69.125	0.13326	-177.109892470576	-177.109892470576\\
69.125	0.13692	-190.091891105619	-190.091891105619\\
69.125	0.14058	-203.895630105476	-203.895630105476\\
69.125	0.14424	-218.521109470149	-218.521109470149\\
69.125	0.1479	-233.968329199637	-233.968329199637\\
69.125	0.15156	-250.237289293939	-250.237289293939\\
69.125	0.15522	-267.327989753057	-267.327989753057\\
69.125	0.15888	-285.240430576991	-285.240430576991\\
69.125	0.16254	-303.974611765739	-303.974611765739\\
69.125	0.1662	-323.530533319303	-323.530533319303\\
69.125	0.16986	-343.908195237681	-343.908195237681\\
69.125	0.17352	-365.107597520875	-365.107597520875\\
69.125	0.17718	-387.128740168884	-387.128740168884\\
69.125	0.18084	-409.971623181708	-409.971623181708\\
69.125	0.1845	-433.636246559348	-433.636246559348\\
69.125	0.18816	-458.122610301803	-458.122610301803\\
69.125	0.19182	-483.430714409072	-483.430714409072\\
69.125	0.19548	-509.560558881157	-509.560558881157\\
69.125	0.19914	-536.512143718057	-536.512143718057\\
69.125	0.2028	-564.285468919772	-564.285468919772\\
69.125	0.20646	-592.880534486303	-592.880534486303\\
69.125	0.21012	-622.297340417648	-622.297340417648\\
69.125	0.21378	-652.535886713809	-652.535886713809\\
69.125	0.21744	-683.596173374785	-683.596173374785\\
69.125	0.2211	-715.478200400576	-715.478200400576\\
69.125	0.22476	-748.181967791182	-748.181967791182\\
69.125	0.22842	-781.707475546603	-781.707475546603\\
69.125	0.23208	-816.05472366684	-816.05472366684\\
69.125	0.23574	-851.223712151892	-851.223712151892\\
69.125	0.2394	-887.214441001759	-887.214441001759\\
69.125	0.24306	-924.02691021644	-924.02691021644\\
69.125	0.24672	-961.661119795937	-961.661119795937\\
69.125	0.25038	-1000.11706974025	-1000.11706974025\\
69.125	0.25404	-1039.39476004938	-1039.39476004938\\
69.125	0.2577	-1079.49419072332	-1079.49419072332\\
69.125	0.26136	-1120.41536176208	-1120.41536176208\\
69.125	0.26502	-1162.15827316565	-1162.15827316565\\
69.125	0.26868	-1204.72292493404	-1204.72292493404\\
69.125	0.27234	-1248.10931706724	-1248.10931706724\\
69.125	0.276	-1292.31744956526	-1292.31744956526\\
69.5	0.093	-90.2647481940033	-90.2647481940033\\
69.5	0.09666	-94.219565473892	-94.219565473892\\
69.5	0.10032	-98.9961231185953	-98.9961231185953\\
69.5	0.10398	-104.594421128114	-104.594421128114\\
69.5	0.10764	-111.014459502448	-111.014459502448\\
69.5	0.1113	-118.256238241597	-118.256238241597\\
69.5	0.11496	-126.319757345562	-126.319757345562\\
69.5	0.11862	-135.205016814341	-135.205016814341\\
69.5	0.12228	-144.912016647936	-144.912016647936\\
69.5	0.12594	-155.440756846346	-155.440756846346\\
69.5	0.1296	-166.791237409571	-166.791237409571\\
69.5	0.13326	-178.963458337611	-178.963458337611\\
69.5	0.13692	-191.957419630466	-191.957419630466\\
69.5	0.14058	-205.773121288137	-205.773121288137\\
69.5	0.14424	-220.410563310622	-220.410563310622\\
69.5	0.1479	-235.869745697923	-235.869745697923\\
69.5	0.15156	-252.150668450039	-252.150668450039\\
69.5	0.15522	-269.25333156697	-269.25333156697\\
69.5	0.15888	-287.177735048717	-287.177735048717\\
69.5	0.16254	-305.923878895278	-305.923878895278\\
69.5	0.1662	-325.491763106655	-325.491763106655\\
69.5	0.16986	-345.881387682847	-345.881387682847\\
69.5	0.17352	-367.092752623853	-367.092752623853\\
69.5	0.17718	-389.125857929676	-389.125857929676\\
69.5	0.18084	-411.980703600313	-411.980703600313\\
69.5	0.1845	-435.657289635766	-435.657289635766\\
69.5	0.18816	-460.155616036033	-460.155616036033\\
69.5	0.19182	-485.475682801116	-485.475682801116\\
69.5	0.19548	-511.617489931014	-511.617489931014\\
69.5	0.19914	-538.581037425727	-538.581037425727\\
69.5	0.2028	-566.366325285256	-566.366325285256\\
69.5	0.20646	-594.973353509599	-594.973353509599\\
69.5	0.21012	-624.402122098757	-624.402122098757\\
69.5	0.21378	-654.652631052731	-654.652631052731\\
69.5	0.21744	-685.72488037152	-685.72488037152\\
69.5	0.2211	-717.618870055124	-717.618870055124\\
69.5	0.22476	-750.334600103543	-750.334600103543\\
69.5	0.22842	-783.872070516778	-783.872070516778\\
69.5	0.23208	-818.231281294827	-818.231281294827\\
69.5	0.23574	-853.412232437693	-853.412232437693\\
69.5	0.2394	-889.414923945372	-889.414923945372\\
69.5	0.24306	-926.239355817868	-926.239355817868\\
69.5	0.24672	-963.885528055177	-963.885528055177\\
69.5	0.25038	-1002.3534406573	-1002.3534406573\\
69.5	0.25404	-1041.64309362424	-1041.64309362424\\
69.5	0.2577	-1081.754486956	-1081.754486956\\
69.5	0.26136	-1122.68762065257	-1122.68762065257\\
69.5	0.26502	-1164.44249471396	-1164.44249471396\\
69.5	0.26868	-1207.01910914016	-1207.01910914016\\
69.5	0.27234	-1250.41746393117	-1250.41746393117\\
69.5	0.276	-1294.63755908701	-1294.63755908701\\
69.875	0.093	-92.0577583819951	-92.0577583819951\\
69.875	0.09666	-96.0245383196967	-96.0245383196967\\
69.875	0.10032	-100.813058622213	-100.813058622213\\
69.875	0.10398	-106.423319289545	-106.423319289545\\
69.875	0.10764	-112.855320321693	-112.855320321693\\
69.875	0.1113	-120.109061718655	-120.109061718655\\
69.875	0.11496	-128.184543480432	-128.184543480432\\
69.875	0.11862	-137.081765607025	-137.081765607025\\
69.875	0.12228	-146.800728098432	-146.800728098432\\
69.875	0.12594	-157.341430954655	-157.341430954655\\
69.875	0.1296	-168.703874175693	-168.703874175693\\
69.875	0.13326	-180.888057761547	-180.888057761547\\
69.875	0.13692	-193.893981712215	-193.893981712215\\
69.875	0.14058	-207.721646027699	-207.721646027699\\
69.875	0.14424	-222.371050707997	-222.371050707997\\
69.875	0.1479	-237.842195753111	-237.842195753111\\
69.875	0.15156	-254.13508116304	-254.13508116304\\
69.875	0.15522	-271.249706937785	-271.249706937785\\
69.875	0.15888	-289.186073077344	-289.186073077344\\
69.875	0.16254	-307.944179581718	-307.944179581718\\
69.875	0.1662	-327.524026450908	-327.524026450908\\
69.875	0.16986	-347.925613684913	-347.925613684913\\
69.875	0.17352	-369.148941283733	-369.148941283733\\
69.875	0.17718	-391.194009247369	-391.194009247369\\
69.875	0.18084	-414.060817575819	-414.060817575819\\
69.875	0.1845	-437.749366269084	-437.749366269084\\
69.875	0.18816	-462.259655327165	-462.259655327165\\
69.875	0.19182	-487.591684750061	-487.591684750061\\
69.875	0.19548	-513.745454537772	-513.745454537772\\
69.875	0.19914	-540.720964690298	-540.720964690298\\
69.875	0.2028	-568.51821520764	-568.51821520764\\
69.875	0.20646	-597.137206089796	-597.137206089796\\
69.875	0.21012	-626.577937336768	-626.577937336768\\
69.875	0.21378	-656.840408948555	-656.840408948555\\
69.875	0.21744	-687.924620925156	-687.924620925156\\
69.875	0.2211	-719.830573266574	-719.830573266574\\
69.875	0.22476	-752.558265972806	-752.558265972806\\
69.875	0.22842	-786.107699043854	-786.107699043854\\
69.875	0.23208	-820.478872479716	-820.478872479716\\
69.875	0.23574	-855.671786280394	-855.671786280394\\
69.875	0.2394	-891.686440445887	-891.686440445887\\
69.875	0.24306	-928.522834976195	-928.522834976195\\
69.875	0.24672	-966.180969871319	-966.180969871319\\
69.875	0.25038	-1004.66084513126	-1004.66084513126\\
69.875	0.25404	-1043.96246075601	-1043.96246075601\\
69.875	0.2577	-1084.08581674558	-1084.08581674558\\
69.875	0.26136	-1125.03091309996	-1125.03091309996\\
69.875	0.26502	-1166.79774981916	-1166.79774981916\\
69.875	0.26868	-1209.38632690318	-1209.38632690318\\
69.875	0.27234	-1252.79664435201	-1252.79664435201\\
69.875	0.276	-1297.02870216565	-1297.02870216565\\
70.25	0.093	-93.921802126887	-93.921802126887\\
70.25	0.09666	-97.9005447224016	-97.9005447224016\\
70.25	0.10032	-102.701027682731	-102.701027682731\\
70.25	0.10398	-108.323251007877	-108.323251007877\\
70.25	0.10764	-114.767214697837	-114.767214697837\\
70.25	0.1113	-122.032918752612	-122.032918752612\\
70.25	0.11496	-130.120363172202	-130.120363172202\\
70.25	0.11862	-139.029547956608	-139.029547956608\\
70.25	0.12228	-148.760473105829	-148.760473105829\\
70.25	0.12594	-159.313138619865	-159.313138619865\\
70.25	0.1296	-170.687544498716	-170.687544498716\\
70.25	0.13326	-182.883690742382	-182.883690742382\\
70.25	0.13692	-195.901577350864	-195.901577350864\\
70.25	0.14058	-209.74120432416	-209.74120432416\\
70.25	0.14424	-224.402571662272	-224.402571662272\\
70.25	0.1479	-239.8856793652	-239.8856793652\\
70.25	0.15156	-256.190527432941	-256.190527432941\\
70.25	0.15522	-273.317115865499	-273.317115865499\\
70.25	0.15888	-291.265444662871	-291.265444662871\\
70.25	0.16254	-310.035513825059	-310.035513825059\\
70.25	0.1662	-329.627323352062	-329.627323352062\\
70.25	0.16986	-350.04087324388	-350.04087324388\\
70.25	0.17352	-371.276163500513	-371.276163500513\\
70.25	0.17718	-393.333194121961	-393.333194121961\\
70.25	0.18084	-416.211965108224	-416.211965108224\\
70.25	0.1845	-439.912476459303	-439.912476459303\\
70.25	0.18816	-464.434728175197	-464.434728175197\\
70.25	0.19182	-489.778720255906	-489.778720255906\\
70.25	0.19548	-515.94445270143	-515.94445270143\\
70.25	0.19914	-542.931925511769	-542.931925511769\\
70.25	0.2028	-570.741138686924	-570.741138686924\\
70.25	0.20646	-599.372092226894	-599.372092226894\\
70.25	0.21012	-628.824786131679	-628.824786131679\\
70.25	0.21378	-659.099220401278	-659.099220401278\\
70.25	0.21744	-690.195395035693	-690.195395035693\\
70.25	0.2211	-722.113310034923	-722.113310034923\\
70.25	0.22476	-754.852965398969	-754.852965398969\\
70.25	0.22842	-788.41436112783	-788.41436112783\\
70.25	0.23208	-822.797497221505	-822.797497221505\\
70.25	0.23574	-858.002373679996	-858.002373679996\\
70.25	0.2394	-894.028990503302	-894.028990503302\\
70.25	0.24306	-930.877347691424	-930.877347691424\\
70.25	0.24672	-968.54744524436	-968.54744524436\\
70.25	0.25038	-1007.03928316211	-1007.03928316211\\
70.25	0.25404	-1046.35286144468	-1046.35286144468\\
70.25	0.2577	-1086.48818009206	-1086.48818009206\\
70.25	0.26136	-1127.44523910426	-1127.44523910426\\
70.25	0.26502	-1169.22403848127	-1169.22403848127\\
70.25	0.26868	-1211.8245782231	-1211.8245782231\\
70.25	0.27234	-1255.24685832974	-1255.24685832974\\
70.25	0.276	-1299.4908788012	-1299.4908788012\\
70.625	0.093	-95.8568794286795	-95.8568794286795\\
70.625	0.09666	-99.8475846820073	-99.8475846820073\\
70.625	0.10032	-104.66003030015	-104.66003030015\\
70.625	0.10398	-110.294216283108	-110.294216283108\\
70.625	0.10764	-116.750142630882	-116.750142630882\\
70.625	0.1113	-124.02780934347	-124.02780934347\\
70.625	0.11496	-132.127216420873	-132.127216420873\\
70.625	0.11862	-141.048363863092	-141.048363863092\\
70.625	0.12228	-150.791251670126	-150.791251670126\\
70.625	0.12594	-161.355879841975	-161.355879841975\\
70.625	0.1296	-172.742248378639	-172.742248378639\\
70.625	0.13326	-184.950357280119	-184.950357280119\\
70.625	0.13692	-197.980206546413	-197.980206546413\\
70.625	0.14058	-211.831796177523	-211.831796177523\\
70.625	0.14424	-226.505126173448	-226.505126173448\\
70.625	0.1479	-242.000196534188	-242.000196534188\\
70.625	0.15156	-258.317007259743	-258.317007259743\\
70.625	0.15522	-275.455558350114	-275.455558350114\\
70.625	0.15888	-293.415849805299	-293.415849805299\\
70.625	0.16254	-312.1978816253	-312.1978816253\\
70.625	0.1662	-331.801653810116	-331.801653810116\\
70.625	0.16986	-352.227166359747	-352.227166359747\\
70.625	0.17352	-373.474419274193	-373.474419274193\\
70.625	0.17718	-395.543412553455	-395.543412553455\\
70.625	0.18084	-418.434146197531	-418.434146197531\\
70.625	0.1845	-442.146620206423	-442.146620206423\\
70.625	0.18816	-466.68083458013	-466.68083458013\\
70.625	0.19182	-492.036789318652	-492.036789318652\\
70.625	0.19548	-518.214484421989	-518.214484421989\\
70.625	0.19914	-545.213919890141	-545.213919890141\\
70.625	0.2028	-573.035095723109	-573.035095723109\\
70.625	0.20646	-601.678011920892	-601.678011920892\\
70.625	0.21012	-631.14266848349	-631.14266848349\\
70.625	0.21378	-661.429065410902	-661.429065410902\\
70.625	0.21744	-692.53720270313	-692.53720270313\\
70.625	0.2211	-724.467080360174	-724.467080360174\\
70.625	0.22476	-757.218698382032	-757.218698382032\\
70.625	0.22842	-790.792056768706	-790.792056768706\\
70.625	0.23208	-825.187155520195	-825.187155520195\\
70.625	0.23574	-860.403994636499	-860.403994636499\\
70.625	0.2394	-896.442574117618	-896.442574117618\\
70.625	0.24306	-933.302893963552	-933.302893963552\\
70.625	0.24672	-970.984954174302	-970.984954174302\\
70.625	0.25038	-1009.48875474987	-1009.48875474987\\
70.625	0.25404	-1048.81429569025	-1048.81429569025\\
70.625	0.2577	-1088.96157699544	-1088.96157699544\\
70.625	0.26136	-1129.93059866545	-1129.93059866545\\
70.625	0.26502	-1171.72136070028	-1171.72136070028\\
70.625	0.26868	-1214.33386309992	-1214.33386309992\\
70.625	0.27234	-1257.76810586437	-1257.76810586437\\
70.625	0.276	-1302.02408899364	-1302.02408899364\\
71	0.093	-97.8629902873728	-97.8629902873728\\
71	0.09666	-101.865658198514	-101.865658198514\\
71	0.10032	-106.69006647447	-106.69006647447\\
71	0.10398	-112.336215115241	-112.336215115241\\
71	0.10764	-118.804104120827	-118.804104120827\\
71	0.1113	-126.093733491228	-126.093733491228\\
71	0.11496	-134.205103226445	-134.205103226445\\
71	0.11862	-143.138213326477	-143.138213326477\\
71	0.12228	-152.893063791324	-152.893063791324\\
71	0.12594	-163.469654620986	-163.469654620986\\
71	0.1296	-174.867985815463	-174.867985815463\\
71	0.13326	-187.088057374756	-187.088057374756\\
71	0.13692	-200.129869298864	-200.129869298864\\
71	0.14058	-213.993421587786	-213.993421587786\\
71	0.14424	-228.678714241524	-228.678714241524\\
71	0.1479	-244.185747260078	-244.185747260078\\
71	0.15156	-260.514520643446	-260.514520643446\\
71	0.15522	-277.665034391629	-277.665034391629\\
71	0.15888	-295.637288504628	-295.637288504628\\
71	0.16254	-314.431282982442	-314.431282982442\\
71	0.1662	-334.047017825071	-334.047017825071\\
71	0.16986	-354.484493032515	-354.484493032515\\
71	0.17352	-375.743708604774	-375.743708604774\\
71	0.17718	-397.824664541849	-397.824664541849\\
71	0.18084	-420.727360843738	-420.727360843738\\
71	0.1845	-444.451797510443	-444.451797510443\\
71	0.18816	-468.997974541963	-468.997974541963\\
71	0.19182	-494.365891938298	-494.365891938298\\
71	0.19548	-520.555549699448	-520.555549699448\\
71	0.19914	-547.566947825414	-547.566947825414\\
71	0.2028	-575.400086316195	-575.400086316195\\
71	0.20646	-604.05496517179	-604.05496517179\\
71	0.21012	-633.531584392201	-633.531584392201\\
71	0.21378	-663.829943977427	-663.829943977427\\
71	0.21744	-694.950043927468	-694.950043927468\\
71	0.2211	-726.891884242325	-726.891884242325\\
71	0.22476	-759.655464921996	-759.655464921996\\
71	0.22842	-793.240785966483	-793.240785966483\\
71	0.23208	-827.647847375785	-827.647847375785\\
71	0.23574	-862.876649149903	-862.876649149903\\
71	0.2394	-898.927191288834	-898.927191288834\\
71	0.24306	-935.799473792582	-935.799473792582\\
71	0.24672	-973.493496661145	-973.493496661145\\
71	0.25038	-1012.00925989452	-1012.00925989452\\
71	0.25404	-1051.34676349272	-1051.34676349272\\
71	0.2577	-1091.50600745572	-1091.50600745572\\
71	0.26136	-1132.48699178355	-1132.48699178355\\
71	0.26502	-1174.28971647619	-1174.28971647619\\
71	0.26868	-1216.91418153364	-1216.91418153364\\
71	0.27234	-1260.36038695591	-1260.36038695591\\
71	0.276	-1304.62833274299	-1304.62833274299\\
71.375	0.093	-99.9401347029672	-99.9401347029672\\
71.375	0.09666	-103.954765271921	-103.954765271921\\
71.375	0.10032	-108.79113620569	-108.79113620569\\
71.375	0.10398	-114.449247504274	-114.449247504274\\
71.375	0.10764	-120.929099167674	-120.929099167674\\
71.375	0.1113	-128.230691195888	-128.230691195888\\
71.375	0.11496	-136.354023588918	-136.354023588918\\
71.375	0.11862	-145.299096346763	-145.299096346763\\
71.375	0.12228	-155.065909469423	-155.065909469423\\
71.375	0.12594	-165.654462956899	-165.654462956899\\
71.375	0.1296	-177.064756809189	-177.064756809189\\
71.375	0.13326	-189.296791026294	-189.296791026294\\
71.375	0.13692	-202.350565608215	-202.350565608215\\
71.375	0.14058	-216.226080554951	-216.226080554951\\
71.375	0.14424	-230.923335866502	-230.923335866502\\
71.375	0.1479	-246.442331542868	-246.442331542868\\
71.375	0.15156	-262.783067584049	-262.783067584049\\
71.375	0.15522	-279.945543990046	-279.945543990046\\
71.375	0.15888	-297.929760760858	-297.929760760858\\
71.375	0.16254	-316.735717896485	-316.735717896485\\
71.375	0.1662	-336.363415396927	-336.363415396927\\
71.375	0.16986	-356.812853262184	-356.812853262184\\
71.375	0.17352	-378.084031492256	-378.084031492256\\
71.375	0.17718	-400.176950087144	-400.176950087144\\
71.375	0.18084	-423.091609046846	-423.091609046846\\
71.375	0.1845	-446.828008371364	-446.828008371364\\
71.375	0.18816	-471.386148060697	-471.386148060697\\
71.375	0.19182	-496.766028114846	-496.766028114846\\
71.375	0.19548	-522.967648533809	-522.967648533809\\
71.375	0.19914	-549.991009317588	-549.991009317588\\
71.375	0.2028	-577.836110466181	-577.836110466181\\
71.375	0.20646	-606.50295197959	-606.50295197959\\
71.375	0.21012	-635.991533857814	-635.991533857814\\
71.375	0.21378	-666.301856100853	-666.301856100853\\
71.375	0.21744	-697.433918708708	-697.433918708708\\
71.375	0.2211	-729.387721681377	-729.387721681377\\
71.375	0.22476	-762.163265018861	-762.163265018861\\
71.375	0.22842	-795.760548721161	-795.760548721161\\
71.375	0.23208	-830.179572788277	-830.179572788277\\
71.375	0.23574	-865.420337220207	-865.420337220207\\
71.375	0.2394	-901.482842016952	-901.482842016952\\
71.375	0.24306	-938.367087178513	-938.367087178513\\
71.375	0.24672	-976.073072704888	-976.073072704888\\
71.375	0.25038	-1014.60079859608	-1014.60079859608\\
71.375	0.25404	-1053.95026485208	-1053.95026485208\\
71.375	0.2577	-1094.12147147291	-1094.12147147291\\
71.375	0.26136	-1135.11441845854	-1135.11441845854\\
71.375	0.26502	-1176.92910580899	-1176.92910580899\\
71.375	0.26868	-1219.56553352426	-1219.56553352426\\
71.375	0.27234	-1263.02370160434	-1263.02370160434\\
71.375	0.276	-1307.30361004924	-1307.30361004924\\
71.75	0.093	-102.088312675462	-102.088312675462\\
71.75	0.09666	-106.114905902229	-106.114905902229\\
71.75	0.10032	-110.963239493811	-110.963239493811\\
71.75	0.10398	-116.633313450208	-116.633313450208\\
71.75	0.10764	-123.125127771421	-123.125127771421\\
71.75	0.1113	-130.438682457448	-130.438682457448\\
71.75	0.11496	-138.573977508291	-138.573977508291\\
71.75	0.11862	-147.531012923949	-147.531012923949\\
71.75	0.12228	-157.309788704422	-157.309788704422\\
71.75	0.12594	-167.910304849711	-167.910304849711\\
71.75	0.1296	-179.332561359814	-179.332561359814\\
71.75	0.13326	-191.576558234733	-191.576558234733\\
71.75	0.13692	-204.642295474467	-204.642295474467\\
71.75	0.14058	-218.529773079016	-218.529773079016\\
71.75	0.14424	-233.23899104838	-233.23899104838\\
71.75	0.1479	-248.769949382559	-248.769949382559\\
71.75	0.15156	-265.122648081553	-265.122648081553\\
71.75	0.15522	-282.297087145363	-282.297087145363\\
71.75	0.15888	-300.293266573988	-300.293266573988\\
71.75	0.16254	-319.111186367428	-319.111186367428\\
71.75	0.1662	-338.750846525683	-338.750846525683\\
71.75	0.16986	-359.212247048753	-359.212247048753\\
71.75	0.17352	-380.495387936638	-380.495387936638\\
71.75	0.17718	-402.600269189339	-402.600269189339\\
71.75	0.18084	-425.526890806855	-425.526890806855\\
71.75	0.1845	-449.275252789186	-449.275252789186\\
71.75	0.18816	-473.845355136332	-473.845355136332\\
71.75	0.19182	-499.237197848293	-499.237197848293\\
71.75	0.19548	-525.45078092507	-525.45078092507\\
71.75	0.19914	-552.486104366661	-552.486104366661\\
71.75	0.2028	-580.343168173068	-580.343168173068\\
71.75	0.20646	-609.02197234429	-609.02197234429\\
71.75	0.21012	-638.522516880327	-638.522516880327\\
71.75	0.21378	-668.84480178118	-668.84480178118\\
71.75	0.21744	-699.988827046847	-699.988827046847\\
71.75	0.2211	-731.954592677329	-731.954592677329\\
71.75	0.22476	-764.742098672627	-764.742098672627\\
71.75	0.22842	-798.35134503274	-798.35134503274\\
71.75	0.23208	-832.782331757668	-832.782331757668\\
71.75	0.23574	-868.035058847411	-868.035058847411\\
71.75	0.2394	-904.10952630197	-904.10952630197\\
71.75	0.24306	-941.005734121343	-941.005734121343\\
71.75	0.24672	-978.723682305532	-978.723682305532\\
71.75	0.25038	-1017.26337085454	-1017.26337085454\\
71.75	0.25404	-1056.62479976836	-1056.62479976836\\
71.75	0.2577	-1096.80796904699	-1096.80796904699\\
71.75	0.26136	-1137.81287869044	-1137.81287869044\\
71.75	0.26502	-1179.6395286987	-1179.6395286987\\
71.75	0.26868	-1222.28791907178	-1222.28791907178\\
71.75	0.27234	-1265.75804980968	-1265.75804980968\\
71.75	0.276	-1310.04992091239	-1310.04992091239\\
72.125	0.093	-104.307524204857	-104.307524204857\\
72.125	0.09666	-108.346080089437	-108.346080089437\\
72.125	0.10032	-113.206376338832	-113.206376338832\\
72.125	0.10398	-118.888412953043	-118.888412953043\\
72.125	0.10764	-125.392189932068	-125.392189932068\\
72.125	0.1113	-132.717707275909	-132.717707275909\\
72.125	0.11496	-140.864964984565	-140.864964984565\\
72.125	0.11862	-149.833963058036	-149.833963058036\\
72.125	0.12228	-159.624701496322	-159.624701496322\\
72.125	0.12594	-170.237180299424	-170.237180299424\\
72.125	0.1296	-181.67139946734	-181.67139946734\\
72.125	0.13326	-193.927359000072	-193.927359000072\\
72.125	0.13692	-207.005058897619	-207.005058897619\\
72.125	0.14058	-220.904499159981	-220.904499159981\\
72.125	0.14424	-235.625679787158	-235.625679787158\\
72.125	0.1479	-251.16860077915	-251.16860077915\\
72.125	0.15156	-267.533262135958	-267.533262135958\\
72.125	0.15522	-284.71966385758	-284.71966385758\\
72.125	0.15888	-302.727805944018	-302.727805944018\\
72.125	0.16254	-321.557688395271	-321.557688395271\\
72.125	0.1662	-341.20931121134	-341.20931121134\\
72.125	0.16986	-361.682674392223	-361.682674392223\\
72.125	0.17352	-382.977777937921	-382.977777937921\\
72.125	0.17718	-405.094621848435	-405.094621848435\\
72.125	0.18084	-428.033206123764	-428.033206123764\\
72.125	0.1845	-451.793530763908	-451.793530763908\\
72.125	0.18816	-476.375595768867	-476.375595768867\\
72.125	0.19182	-501.779401138642	-501.779401138642\\
72.125	0.19548	-528.004946873231	-528.004946873231\\
72.125	0.19914	-555.052232972636	-555.052232972636\\
72.125	0.2028	-582.921259436856	-582.921259436856\\
72.125	0.20646	-611.612026265891	-611.612026265891\\
72.125	0.21012	-641.124533459741	-641.124533459741\\
72.125	0.21378	-671.458781018406	-671.458781018406\\
72.125	0.21744	-702.614768941887	-702.614768941887\\
72.125	0.2211	-734.592497230182	-734.592497230182\\
72.125	0.22476	-767.391965883293	-767.391965883293\\
72.125	0.22842	-801.013174901219	-801.013174901219\\
72.125	0.23208	-835.45612428396	-835.45612428396\\
72.125	0.23574	-870.720814031517	-870.720814031517\\
72.125	0.2394	-906.807244143888	-906.807244143888\\
72.125	0.24306	-943.715414621075	-943.715414621075\\
72.125	0.24672	-981.445325463076	-981.445325463076\\
72.125	0.25038	-1019.99697666989	-1019.99697666989\\
72.125	0.25404	-1059.37036824153	-1059.37036824153\\
72.125	0.2577	-1099.56550017797	-1099.56550017797\\
72.125	0.26136	-1140.58237247924	-1140.58237247924\\
72.125	0.26502	-1182.42098514531	-1182.42098514531\\
72.125	0.26868	-1225.08133817621	-1225.08133817621\\
72.125	0.27234	-1268.56343157191	-1268.56343157191\\
72.125	0.276	-1312.86726533244	-1312.86726533244\\
72.5	0.093	-106.597769291153	-106.597769291153\\
72.5	0.09666	-110.648287833546	-110.648287833546\\
72.5	0.10032	-115.520546740754	-115.520546740754\\
72.5	0.10398	-121.214546012777	-121.214546012777\\
72.5	0.10764	-127.730285649616	-127.730285649616\\
72.5	0.1113	-135.06776565127	-135.06776565127\\
72.5	0.11496	-143.226986017739	-143.226986017739\\
72.5	0.11862	-152.207946749023	-152.207946749023\\
72.5	0.12228	-162.010647845122	-162.010647845122\\
72.5	0.12594	-172.635089306037	-172.635089306037\\
72.5	0.1296	-184.081271131766	-184.081271131766\\
72.5	0.13326	-196.349193322311	-196.349193322311\\
72.5	0.13692	-209.438855877671	-209.438855877671\\
72.5	0.14058	-223.350258797846	-223.350258797846\\
72.5	0.14424	-238.083402082837	-238.083402082837\\
72.5	0.1479	-253.638285732642	-253.638285732642\\
72.5	0.15156	-270.014909747263	-270.014909747263\\
72.5	0.15522	-287.213274126698	-287.213274126698\\
72.5	0.15888	-305.23337887095	-305.23337887095\\
72.5	0.16254	-324.075223980016	-324.075223980016\\
72.5	0.1662	-343.738809453897	-343.738809453897\\
72.5	0.16986	-364.224135292593	-364.224135292593\\
72.5	0.17352	-385.531201496105	-385.531201496105\\
72.5	0.17718	-407.660008064432	-407.660008064432\\
72.5	0.18084	-430.610554997574	-430.610554997574\\
72.5	0.1845	-454.382842295531	-454.382842295531\\
72.5	0.18816	-478.976869958303	-478.976869958303\\
72.5	0.19182	-504.392637985891	-504.392637985891\\
72.5	0.19548	-530.630146378293	-530.630146378293\\
72.5	0.19914	-557.689395135511	-557.689395135511\\
72.5	0.2028	-585.570384257544	-585.570384257544\\
72.5	0.20646	-614.273113744392	-614.273113744392\\
72.5	0.21012	-643.797583596055	-643.797583596055\\
72.5	0.21378	-674.143793812534	-674.143793812534\\
72.5	0.21744	-705.311744393827	-705.311744393827\\
72.5	0.2211	-737.301435339936	-737.301435339936\\
72.5	0.22476	-770.112866650859	-770.112866650859\\
72.5	0.22842	-803.746038326598	-803.746038326598\\
72.5	0.23208	-838.200950367153	-838.200950367153\\
72.5	0.23574	-873.477602772523	-873.477602772523\\
72.5	0.2394	-909.575995542707	-909.575995542707\\
72.5	0.24306	-946.496128677707	-946.496128677707\\
72.5	0.24672	-984.238002177521	-984.238002177521\\
72.5	0.25038	-1022.80161604215	-1022.80161604215\\
72.5	0.25404	-1062.1869702716	-1062.1869702716\\
72.5	0.2577	-1102.39406486586	-1102.39406486586\\
72.5	0.26136	-1143.42289982493	-1143.42289982493\\
72.5	0.26502	-1185.27347514882	-1185.27347514882\\
72.5	0.26868	-1227.94579083753	-1227.94579083753\\
72.5	0.27234	-1271.43984689105	-1271.43984689105\\
72.5	0.276	-1315.75564330939	-1315.75564330939\\
72.875	0.093	-108.95904793435	-108.95904793435\\
72.875	0.09666	-113.021529134556	-113.021529134556\\
72.875	0.10032	-117.905750699577	-117.905750699577\\
72.875	0.10398	-123.611712629414	-123.611712629414\\
72.875	0.10764	-130.139414924066	-130.139414924066\\
72.875	0.1113	-137.488857583532	-137.488857583532\\
72.875	0.11496	-145.660040607815	-145.660040607815\\
72.875	0.11862	-154.652963996912	-154.652963996912\\
72.875	0.12228	-164.467627750824	-164.467627750824\\
72.875	0.12594	-175.104031869552	-175.104031869552\\
72.875	0.1296	-186.562176353094	-186.562176353094\\
72.875	0.13326	-198.842061201452	-198.842061201452\\
72.875	0.13692	-211.943686414625	-211.943686414625\\
72.875	0.14058	-225.867051992613	-225.867051992613\\
72.875	0.14424	-240.612157935417	-240.612157935417\\
72.875	0.1479	-256.179004243035	-256.179004243035\\
72.875	0.15156	-272.567590915469	-272.567590915469\\
72.875	0.15522	-289.777917952718	-289.777917952718\\
72.875	0.15888	-307.809985354782	-307.809985354782\\
72.875	0.16254	-326.663793121661	-326.663793121661\\
72.875	0.1662	-346.339341253356	-346.339341253356\\
72.875	0.16986	-366.836629749865	-366.836629749865\\
72.875	0.17352	-388.15565861119	-388.15565861119\\
72.875	0.17718	-410.29642783733	-410.29642783733\\
72.875	0.18084	-433.258937428284	-433.258937428284\\
72.875	0.1845	-457.043187384055	-457.043187384055\\
72.875	0.18816	-481.64917770464	-481.64917770464\\
72.875	0.19182	-507.076908390041	-507.076908390041\\
72.875	0.19548	-533.326379440256	-533.326379440256\\
72.875	0.19914	-560.397590855287	-560.397590855287\\
72.875	0.2028	-588.290542635133	-588.290542635133\\
72.875	0.20646	-617.005234779794	-617.005234779794\\
72.875	0.21012	-646.541667289271	-646.541667289271\\
72.875	0.21378	-676.899840163562	-676.899840163562\\
72.875	0.21744	-708.079753402669	-708.079753402669\\
72.875	0.2211	-740.081407006591	-740.081407006591\\
72.875	0.22476	-772.904800975327	-772.904800975327\\
72.875	0.22842	-806.54993530888	-806.54993530888\\
72.875	0.23208	-841.016810007247	-841.016810007247\\
72.875	0.23574	-876.30542507043	-876.30542507043\\
72.875	0.2394	-912.415780498427	-912.415780498427\\
72.875	0.24306	-949.34787629124	-949.34787629124\\
72.875	0.24672	-987.101712448868	-987.101712448868\\
72.875	0.25038	-1025.67728897131	-1025.67728897131\\
72.875	0.25404	-1065.07460585857	-1065.07460585857\\
72.875	0.2577	-1105.29366311064	-1105.29366311064\\
72.875	0.26136	-1146.33446072753	-1146.33446072753\\
72.875	0.26502	-1188.19699870924	-1188.19699870924\\
72.875	0.26868	-1230.88127705575	-1230.88127705575\\
72.875	0.27234	-1274.38729576709	-1274.38729576709\\
72.875	0.276	-1318.71505484324	-1318.71505484324\\
73.25	0.093	-111.391360134447	-111.391360134447\\
73.25	0.09666	-115.465803992466	-115.465803992466\\
73.25	0.10032	-120.361988215301	-120.361988215301\\
73.25	0.10398	-126.07991280295	-126.07991280295\\
73.25	0.10764	-132.619577755415	-132.619577755415\\
73.25	0.1113	-139.980983072695	-139.980983072695\\
73.25	0.11496	-148.16412875479	-148.16412875479\\
73.25	0.11862	-157.1690148017	-157.1690148017\\
73.25	0.12228	-166.995641213426	-166.995641213426\\
73.25	0.12594	-177.644007989967	-177.644007989967\\
73.25	0.1296	-189.114115131322	-189.114115131322\\
73.25	0.13326	-201.405962637493	-201.405962637493\\
73.25	0.13692	-214.519550508479	-214.519550508479\\
73.25	0.14058	-228.454878744281	-228.454878744281\\
73.25	0.14424	-243.211947344897	-243.211947344897\\
73.25	0.1479	-258.790756310329	-258.790756310329\\
73.25	0.15156	-275.191305640575	-275.191305640575\\
73.25	0.15522	-292.413595335637	-292.413595335637\\
73.25	0.15888	-310.457625395514	-310.457625395514\\
73.25	0.16254	-329.323395820207	-329.323395820207\\
73.25	0.1662	-349.010906609714	-349.010906609714\\
73.25	0.16986	-369.520157764037	-369.520157764037\\
73.25	0.17352	-390.851149283175	-390.851149283175\\
73.25	0.17718	-413.003881167128	-413.003881167128\\
73.25	0.18084	-435.978353415895	-435.978353415895\\
73.25	0.1845	-459.774566029479	-459.774566029479\\
73.25	0.18816	-484.392519007878	-484.392519007878\\
73.25	0.19182	-509.832212351091	-509.832212351091\\
73.25	0.19548	-536.093646059119	-536.093646059119\\
73.25	0.19914	-563.176820131963	-563.176820131963\\
73.25	0.2028	-591.081734569623	-591.081734569623\\
73.25	0.20646	-619.808389372097	-619.808389372097\\
73.25	0.21012	-649.356784539387	-649.356784539387\\
73.25	0.21378	-679.726920071491	-679.726920071491\\
73.25	0.21744	-710.91879596841	-710.91879596841\\
73.25	0.2211	-742.932412230146	-742.932412230146\\
73.25	0.22476	-775.767768856696	-775.767768856696\\
73.25	0.22842	-809.424865848061	-809.424865848061\\
73.25	0.23208	-843.903703204241	-843.903703204241\\
73.25	0.23574	-879.204280925237	-879.204280925237\\
73.25	0.2394	-915.326599011047	-915.326599011047\\
73.25	0.24306	-952.270657461673	-952.270657461673\\
73.25	0.24672	-990.036456277115	-990.036456277115\\
73.25	0.25038	-1028.62399545737	-1028.62399545737\\
73.25	0.25404	-1068.03327500244	-1068.03327500244\\
73.25	0.2577	-1108.26429491233	-1108.26429491233\\
73.25	0.26136	-1149.31705518703	-1149.31705518703\\
73.25	0.26502	-1191.19155582655	-1191.19155582655\\
73.25	0.26868	-1233.88779683088	-1233.88779683088\\
73.25	0.27234	-1277.40577820003	-1277.40577820003\\
73.25	0.276	-1321.74549993399	-1321.74549993399\\
73.625	0.093	-113.894705891445	-113.894705891445\\
73.625	0.09666	-117.981112407277	-117.981112407277\\
73.625	0.10032	-122.889259287925	-122.889259287925\\
73.625	0.10398	-128.619146533387	-128.619146533387\\
73.625	0.10764	-135.170774143665	-135.170774143665\\
73.625	0.1113	-142.544142118758	-142.544142118758\\
73.625	0.11496	-150.739250458667	-150.739250458667\\
73.625	0.11862	-159.75609916339	-159.75609916339\\
73.625	0.12228	-169.594688232928	-169.594688232928\\
73.625	0.12594	-180.255017667282	-180.255017667282\\
73.625	0.1296	-191.737087466451	-191.737087466451\\
73.625	0.13326	-204.040897630435	-204.040897630435\\
73.625	0.13692	-217.166448159234	-217.166448159234\\
73.625	0.14058	-231.113739052849	-231.113739052849\\
73.625	0.14424	-245.882770311278	-245.882770311278\\
73.625	0.1479	-261.473541934523	-261.473541934523\\
73.625	0.15156	-277.886053922583	-277.886053922583\\
73.625	0.15522	-295.120306275458	-295.120306275458\\
73.625	0.15888	-313.176298993148	-313.176298993148\\
73.625	0.16254	-332.054032075653	-332.054032075653\\
73.625	0.1662	-351.753505522974	-351.753505522974\\
73.625	0.16986	-372.274719335109	-372.274719335109\\
73.625	0.17352	-393.61767351206	-393.61767351206\\
73.625	0.17718	-415.782368053826	-415.782368053826\\
73.625	0.18084	-438.768802960407	-438.768802960407\\
73.625	0.1845	-462.576978231804	-462.576978231804\\
73.625	0.18816	-487.206893868015	-487.206893868015\\
73.625	0.19182	-512.658549869042	-512.658549869042\\
73.625	0.19548	-538.931946234884	-538.931946234884\\
73.625	0.19914	-566.02708296554	-566.02708296554\\
73.625	0.2028	-593.943960061013	-593.943960061013\\
73.625	0.20646	-622.6825775213	-622.6825775213\\
73.625	0.21012	-652.242935346403	-652.242935346403\\
73.625	0.21378	-682.62503353632	-682.62503353632\\
73.625	0.21744	-713.828872091053	-713.828872091053\\
73.625	0.2211	-745.854451010601	-745.854451010601\\
73.625	0.22476	-778.701770294964	-778.701770294964\\
73.625	0.22842	-812.370829944143	-812.370829944143\\
73.625	0.23208	-846.861629958136	-846.861629958136\\
73.625	0.23574	-882.174170336945	-882.174170336945\\
73.625	0.2394	-918.308451080568	-918.308451080568\\
73.625	0.24306	-955.264472189007	-955.264472189007\\
73.625	0.24672	-993.042233662262	-993.042233662262\\
73.625	0.25038	-1031.64173550033	-1031.64173550033\\
73.625	0.25404	-1071.06297770322	-1071.06297770322\\
73.625	0.2577	-1111.30596027091	-1111.30596027091\\
73.625	0.26136	-1152.37068320343	-1152.37068320343\\
73.625	0.26502	-1194.25714650076	-1194.25714650076\\
73.625	0.26868	-1236.9653501629	-1236.9653501629\\
73.625	0.27234	-1280.49529418987	-1280.49529418987\\
73.625	0.276	-1324.84697858164	-1324.84697858164\\
74	0.093	-116.469085205343	-116.469085205343\\
74	0.09666	-120.567454378989	-120.567454378989\\
74	0.10032	-125.487563917449	-125.487563917449\\
74	0.10398	-131.229413820725	-131.229413820725\\
74	0.10764	-137.793004088816	-137.793004088816\\
74	0.1113	-145.178334721722	-145.178334721722\\
74	0.11496	-153.385405719443	-153.385405719443\\
74	0.11862	-162.41421708198	-162.41421708198\\
74	0.12228	-172.264768809331	-172.264768809331\\
74	0.12594	-182.937060901498	-182.937060901498\\
74	0.1296	-194.43109335848	-194.43109335848\\
74	0.13326	-206.746866180277	-206.746866180277\\
74	0.13692	-219.884379366889	-219.884379366889\\
74	0.14058	-233.843632918317	-233.843632918317\\
74	0.14424	-248.624626834559	-248.624626834559\\
74	0.1479	-264.227361115617	-264.227361115617\\
74	0.15156	-280.65183576149	-280.65183576149\\
74	0.15522	-297.898050772178	-297.898050772178\\
74	0.15888	-315.966006147682	-315.966006147682\\
74	0.16254	-334.855701888	-334.855701888\\
74	0.1662	-354.567137993134	-354.567137993134\\
74	0.16986	-375.100314463082	-375.100314463082\\
74	0.17352	-396.455231297846	-396.455231297846\\
74	0.17718	-418.631888497426	-418.631888497426\\
74	0.18084	-441.630286061819	-441.630286061819\\
74	0.1845	-465.450423991029	-465.450423991029\\
74	0.18816	-490.092302285054	-490.092302285054\\
74	0.19182	-515.555920943893	-515.555920943893\\
74	0.19548	-541.841279967548	-541.841279967548\\
74	0.19914	-568.948379356018	-568.948379356018\\
74	0.2028	-596.877219109304	-596.877219109304\\
74	0.20646	-625.627799227404	-625.627799227404\\
74	0.21012	-655.20011971032	-655.20011971032\\
74	0.21378	-685.594180558051	-685.594180558051\\
74	0.21744	-716.809981770596	-716.809981770596\\
74	0.2211	-748.847523347957	-748.847523347957\\
74	0.22476	-781.706805290133	-781.706805290133\\
74	0.22842	-815.387827597125	-815.387827597125\\
74	0.23208	-849.890590268931	-849.890590268931\\
74	0.23574	-885.215093305553	-885.215093305553\\
74	0.2394	-921.36133670699	-921.36133670699\\
74	0.24306	-958.329320473242	-958.329320473242\\
74	0.24672	-996.119044604309	-996.119044604309\\
74	0.25038	-1034.73050910019	-1034.73050910019\\
74	0.25404	-1074.16371396089	-1074.16371396089\\
74	0.2577	-1114.4186591864	-1114.4186591864\\
74	0.26136	-1155.49534477673	-1155.49534477673\\
74	0.26502	-1197.39377073187	-1197.39377073187\\
74	0.26868	-1240.11393705183	-1240.11393705183\\
74	0.27234	-1283.6558437366	-1283.6558437366\\
74	0.276	-1328.01949078619	-1328.01949078619\\
};
\end{axis}

\begin{axis}[%
width=6.159cm,
height=3.097cm,
at={(0cm,0cm)},
scale only axis,
xmin=56,
xmax=74,
tick align=outside,
xlabel style={font=\color{white!15!black}},
xlabel={$L_{cut}$},
ymin=0.093,
ymax=0.276,
ylabel style={font=\color{white!15!black}},
ylabel={$D_{rlx}$},
zmin=-2000,
zmax=0,
zlabel style={font=\color{white!15!black}},
zlabel={$c$},
view={-140}{50},
axis background/.style={fill=white},
xmajorgrids,
ymajorgrids,
zmajorgrids
]
\addplot3[only marks, mark=*, mark options={}, mark size=1.5000pt, color=mycolor1, fill=mycolor1] table[row sep=crcr]{%
x	y	z\\
74	0.123	-233.157966898601\\
72	0.113	-200.830593420783\\
61	0.095	-93.7391747783605\\
56	0.093	-119.696332564413\\
};
\addplot3[only marks, mark=*, mark options={}, mark size=1.5000pt, color=mycolor2, fill=mycolor2] table[row sep=crcr]{%
x	y	z\\
67	0.276	-1805.89673785913\\
66	0.255	-1488.55090215304\\
62	0.209	-803.703476355143\\
57	0.193	-648.796609601896\\
};
\addplot3[only marks, mark=*, mark options={}, mark size=1.5000pt, color=black, fill=black] table[row sep=crcr]{%
x	y	z\\
69	0.104	-138.62908786906\\
};
\addplot3[only marks, mark=*, mark options={}, mark size=1.5000pt, color=black, fill=black] table[row sep=crcr]{%
x	y	z\\
64	0.23	-1085.45178719795\\
};

\addplot3[%
surf,
fill opacity=0.7, shader=interp, colormap={mymap}{[1pt] rgb(0pt)=(1,0.905882,0); rgb(1pt)=(1,0.901964,0); rgb(2pt)=(1,0.898051,0); rgb(3pt)=(1,0.894144,0); rgb(4pt)=(1,0.890243,0); rgb(5pt)=(1,0.886349,0); rgb(6pt)=(1,0.88246,0); rgb(7pt)=(1,0.878577,0); rgb(8pt)=(1,0.8747,0); rgb(9pt)=(1,0.870829,0); rgb(10pt)=(1,0.866964,0); rgb(11pt)=(1,0.863106,0); rgb(12pt)=(1,0.859253,0); rgb(13pt)=(1,0.855406,0); rgb(14pt)=(1,0.851566,0); rgb(15pt)=(1,0.847732,0); rgb(16pt)=(1,0.843903,0); rgb(17pt)=(1,0.840081,0); rgb(18pt)=(1,0.836265,0); rgb(19pt)=(1,0.832455,0); rgb(20pt)=(1,0.828652,0); rgb(21pt)=(1,0.824854,0); rgb(22pt)=(1,0.821063,0); rgb(23pt)=(1,0.817278,0); rgb(24pt)=(1,0.8135,0); rgb(25pt)=(1,0.809727,0); rgb(26pt)=(1,0.805961,0); rgb(27pt)=(1,0.8022,0); rgb(28pt)=(1,0.798445,0); rgb(29pt)=(1,0.794696,0); rgb(30pt)=(1,0.790953,0); rgb(31pt)=(1,0.787215,0); rgb(32pt)=(1,0.783484,0); rgb(33pt)=(1,0.779758,0); rgb(34pt)=(1,0.776038,0); rgb(35pt)=(1,0.772324,0); rgb(36pt)=(1,0.768615,0); rgb(37pt)=(1,0.764913,0); rgb(38pt)=(1,0.761217,0); rgb(39pt)=(1,0.757527,0); rgb(40pt)=(1,0.753843,0); rgb(41pt)=(1,0.750165,0); rgb(42pt)=(1,0.746493,0); rgb(43pt)=(1,0.742827,0); rgb(44pt)=(1,0.739167,0); rgb(45pt)=(1,0.735514,0); rgb(46pt)=(1,0.731867,0); rgb(47pt)=(1,0.728226,0); rgb(48pt)=(1,0.724591,0); rgb(49pt)=(1,0.720963,0); rgb(50pt)=(1,0.717341,0); rgb(51pt)=(1,0.713725,0); rgb(52pt)=(0.999994,0.710077,0); rgb(53pt)=(0.999974,0.706363,0); rgb(54pt)=(0.999942,0.702592,0); rgb(55pt)=(0.999898,0.698775,0); rgb(56pt)=(0.999841,0.694921,0); rgb(57pt)=(0.999771,0.691039,0); rgb(58pt)=(0.99969,0.687139,0); rgb(59pt)=(0.999596,0.68323,0); rgb(60pt)=(0.99949,0.679323,0); rgb(61pt)=(0.999372,0.675427,0); rgb(62pt)=(0.999242,0.67155,0); rgb(63pt)=(0.9991,0.667704,0); rgb(64pt)=(0.998946,0.663897,0); rgb(65pt)=(0.998781,0.660138,0); rgb(66pt)=(0.998605,0.656439,0); rgb(67pt)=(0.998416,0.652807,0); rgb(68pt)=(0.998217,0.649253,0); rgb(69pt)=(0.998006,0.645786,0); rgb(70pt)=(0.997785,0.642416,0); rgb(71pt)=(0.997552,0.639152,0); rgb(72pt)=(0.997308,0.636004,0); rgb(73pt)=(0.997053,0.632982,0); rgb(74pt)=(0.996788,0.630095,0); rgb(75pt)=(0.996512,0.627352,0); rgb(76pt)=(0.996226,0.624763,0); rgb(77pt)=(0.995851,0.622329,0); rgb(78pt)=(0.99494,0.619997,0); rgb(79pt)=(0.99345,0.617753,0); rgb(80pt)=(0.991419,0.61559,0); rgb(81pt)=(0.988885,0.613503,0); rgb(82pt)=(0.985886,0.611486,0); rgb(83pt)=(0.98246,0.609532,0); rgb(84pt)=(0.978643,0.607636,0); rgb(85pt)=(0.974475,0.605791,0); rgb(86pt)=(0.969992,0.603992,0); rgb(87pt)=(0.965232,0.602233,0); rgb(88pt)=(0.960233,0.600507,0); rgb(89pt)=(0.955033,0.598808,0); rgb(90pt)=(0.949669,0.59713,0); rgb(91pt)=(0.94418,0.595468,0); rgb(92pt)=(0.938602,0.593815,0); rgb(93pt)=(0.932974,0.592166,0); rgb(94pt)=(0.927333,0.590513,0); rgb(95pt)=(0.921717,0.588852,0); rgb(96pt)=(0.916164,0.587176,0); rgb(97pt)=(0.910711,0.585479,0); rgb(98pt)=(0.905397,0.583755,0); rgb(99pt)=(0.900258,0.581999,0); rgb(100pt)=(0.895333,0.580203,0); rgb(101pt)=(0.890659,0.578362,0); rgb(102pt)=(0.886275,0.576471,0); rgb(103pt)=(0.882047,0.574545,0); rgb(104pt)=(0.877819,0.572608,0); rgb(105pt)=(0.873592,0.57066,0); rgb(106pt)=(0.869366,0.568701,0); rgb(107pt)=(0.865143,0.566733,0); rgb(108pt)=(0.860924,0.564756,0); rgb(109pt)=(0.856708,0.562771,0); rgb(110pt)=(0.852497,0.560778,0); rgb(111pt)=(0.848292,0.558779,0); rgb(112pt)=(0.844092,0.556774,0); rgb(113pt)=(0.8399,0.554763,0); rgb(114pt)=(0.835716,0.552749,0); rgb(115pt)=(0.831541,0.55073,0); rgb(116pt)=(0.827374,0.548709,0); rgb(117pt)=(0.823219,0.546686,0); rgb(118pt)=(0.819074,0.54466,0); rgb(119pt)=(0.81494,0.542635,0); rgb(120pt)=(0.81082,0.540609,0); rgb(121pt)=(0.806712,0.538584,0); rgb(122pt)=(0.802619,0.53656,0); rgb(123pt)=(0.798541,0.534539,0); rgb(124pt)=(0.794478,0.532521,0); rgb(125pt)=(0.790431,0.530506,0); rgb(126pt)=(0.786402,0.528496,0); rgb(127pt)=(0.782391,0.526491,0); rgb(128pt)=(0.77841,0.524489,0); rgb(129pt)=(0.774523,0.522478,0); rgb(130pt)=(0.770731,0.520455,0); rgb(131pt)=(0.767022,0.518424,0); rgb(132pt)=(0.763384,0.516385,0); rgb(133pt)=(0.759804,0.514339,0); rgb(134pt)=(0.756272,0.51229,0); rgb(135pt)=(0.752775,0.510237,0); rgb(136pt)=(0.749302,0.508182,0); rgb(137pt)=(0.74584,0.506128,0); rgb(138pt)=(0.742378,0.504075,0); rgb(139pt)=(0.738904,0.502025,0); rgb(140pt)=(0.735406,0.499979,0); rgb(141pt)=(0.731872,0.49794,0); rgb(142pt)=(0.72829,0.495909,0); rgb(143pt)=(0.724649,0.493887,0); rgb(144pt)=(0.720936,0.491875,0); rgb(145pt)=(0.71714,0.489876,0); rgb(146pt)=(0.713249,0.487891,0); rgb(147pt)=(0.709251,0.485921,0); rgb(148pt)=(0.705134,0.483968,0); rgb(149pt)=(0.700887,0.482033,0); rgb(150pt)=(0.696497,0.480118,0); rgb(151pt)=(0.691952,0.478225,0); rgb(152pt)=(0.687242,0.476355,0); rgb(153pt)=(0.682353,0.47451,0); rgb(154pt)=(0.677195,0.472696,0); rgb(155pt)=(0.6717,0.470916,0); rgb(156pt)=(0.665891,0.469169,0); rgb(157pt)=(0.659791,0.46745,0); rgb(158pt)=(0.653423,0.465756,0); rgb(159pt)=(0.64681,0.464084,0); rgb(160pt)=(0.639976,0.462432,0); rgb(161pt)=(0.632943,0.460795,0); rgb(162pt)=(0.625734,0.459171,0); rgb(163pt)=(0.618373,0.457556,0); rgb(164pt)=(0.610882,0.455948,0); rgb(165pt)=(0.603284,0.454343,0); rgb(166pt)=(0.595604,0.452737,0); rgb(167pt)=(0.587863,0.451129,0); rgb(168pt)=(0.580084,0.449514,0); rgb(169pt)=(0.572292,0.447889,0); rgb(170pt)=(0.564508,0.446252,0); rgb(171pt)=(0.556756,0.444599,0); rgb(172pt)=(0.549059,0.442927,0); rgb(173pt)=(0.54144,0.441232,0); rgb(174pt)=(0.533922,0.439512,0); rgb(175pt)=(0.526529,0.437764,0); rgb(176pt)=(0.519282,0.435983,0); rgb(177pt)=(0.512206,0.434168,0); rgb(178pt)=(0.505323,0.432315,0); rgb(179pt)=(0.498628,0.430422,3.92506e-06); rgb(180pt)=(0.491973,0.428504,3.49981e-05); rgb(181pt)=(0.485331,0.426562,9.63073e-05); rgb(182pt)=(0.478704,0.424596,0.000186979); rgb(183pt)=(0.472096,0.422609,0.000306141); rgb(184pt)=(0.465508,0.420599,0.00045292); rgb(185pt)=(0.458942,0.418567,0.000626441); rgb(186pt)=(0.452401,0.416515,0.000825833); rgb(187pt)=(0.445885,0.414441,0.00105022); rgb(188pt)=(0.439399,0.412348,0.00129873); rgb(189pt)=(0.432942,0.410234,0.00157049); rgb(190pt)=(0.426518,0.408102,0.00186463); rgb(191pt)=(0.420129,0.40595,0.00218028); rgb(192pt)=(0.413777,0.40378,0.00251655); rgb(193pt)=(0.407464,0.401592,0.00287258); rgb(194pt)=(0.401191,0.399386,0.00324749); rgb(195pt)=(0.394962,0.397164,0.00364042); rgb(196pt)=(0.388777,0.394925,0.00405048); rgb(197pt)=(0.38264,0.39267,0.00447681); rgb(198pt)=(0.376552,0.390399,0.00491852); rgb(199pt)=(0.370516,0.388113,0.00537476); rgb(200pt)=(0.364532,0.385812,0.00584464); rgb(201pt)=(0.358605,0.383497,0.00632729); rgb(202pt)=(0.352735,0.381168,0.00682184); rgb(203pt)=(0.346925,0.378826,0.00732741); rgb(204pt)=(0.341176,0.376471,0.00784314); rgb(205pt)=(0.335485,0.374093,0.00847245); rgb(206pt)=(0.329843,0.371682,0.00930909); rgb(207pt)=(0.324249,0.369242,0.0103377); rgb(208pt)=(0.318701,0.366772,0.0115428); rgb(209pt)=(0.313198,0.364275,0.0129091); rgb(210pt)=(0.307739,0.361753,0.0144211); rgb(211pt)=(0.302322,0.359206,0.0160634); rgb(212pt)=(0.296945,0.356637,0.0178207); rgb(213pt)=(0.291607,0.354048,0.0196776); rgb(214pt)=(0.286307,0.35144,0.0216186); rgb(215pt)=(0.281043,0.348814,0.0236284); rgb(216pt)=(0.275813,0.346172,0.0256916); rgb(217pt)=(0.270616,0.343517,0.0277927); rgb(218pt)=(0.265451,0.340849,0.0299163); rgb(219pt)=(0.260317,0.33817,0.0320472); rgb(220pt)=(0.25521,0.335482,0.0341698); rgb(221pt)=(0.250131,0.332786,0.0362688); rgb(222pt)=(0.245078,0.330085,0.0383287); rgb(223pt)=(0.240048,0.327379,0.0403343); rgb(224pt)=(0.235042,0.324671,0.04227); rgb(225pt)=(0.230056,0.321962,0.0441205); rgb(226pt)=(0.22509,0.319254,0.0458704); rgb(227pt)=(0.220142,0.316548,0.0475043); rgb(228pt)=(0.215212,0.313846,0.0490067); rgb(229pt)=(0.210296,0.311149,0.0503624); rgb(230pt)=(0.205395,0.308459,0.0515759); rgb(231pt)=(0.200514,0.305763,0.052757); rgb(232pt)=(0.195655,0.303061,0.0539242); rgb(233pt)=(0.190817,0.300353,0.0550763); rgb(234pt)=(0.186001,0.297639,0.0562123); rgb(235pt)=(0.181207,0.294918,0.0573313); rgb(236pt)=(0.176434,0.292191,0.0584321); rgb(237pt)=(0.171685,0.289458,0.0595136); rgb(238pt)=(0.166957,0.286719,0.060575); rgb(239pt)=(0.162252,0.283973,0.0616151); rgb(240pt)=(0.15757,0.281221,0.0626328); rgb(241pt)=(0.152911,0.278463,0.0636271); rgb(242pt)=(0.148275,0.275699,0.0645971); rgb(243pt)=(0.143663,0.272929,0.0655416); rgb(244pt)=(0.139074,0.270152,0.0664596); rgb(245pt)=(0.134508,0.26737,0.06735); rgb(246pt)=(0.129967,0.264581,0.0682118); rgb(247pt)=(0.125449,0.261787,0.0690441); rgb(248pt)=(0.120956,0.258986,0.0698456); rgb(249pt)=(0.116487,0.25618,0.0706154); rgb(250pt)=(0.112043,0.253367,0.0713525); rgb(251pt)=(0.107623,0.250549,0.0720557); rgb(252pt)=(0.103229,0.247724,0.0727241); rgb(253pt)=(0.0988592,0.244894,0.0733566); rgb(254pt)=(0.0945149,0.242058,0.0739522); rgb(255pt)=(0.0901961,0.239216,0.0745098)}, mesh/rows=49]
table[row sep=crcr, point meta=\thisrow{c}] {%
%
x	y	z	c\\
56	0.093	-115.683340162018	-115.683340162018\\
56	0.09666	-119.178304962119	-119.178304962119\\
56	0.10032	-123.892888394505	-123.892888394505\\
56	0.10398	-129.827090459174	-129.827090459174\\
56	0.10764	-136.980911156127	-136.980911156127\\
56	0.1113	-145.354350485365	-145.354350485365\\
56	0.11496	-154.947408446886	-154.947408446886\\
56	0.11862	-165.760085040692	-165.760085040692\\
56	0.12228	-177.79238026678	-177.79238026678\\
56	0.12594	-191.044294125154	-191.044294125154\\
56	0.1296	-205.515826615812	-205.515826615812\\
56	0.13326	-221.206977738753	-221.206977738753\\
56	0.13692	-238.117747493979	-238.117747493979\\
56	0.14058	-256.248135881489	-256.248135881489\\
56	0.14424	-275.598142901282	-275.598142901282\\
56	0.1479	-296.16776855336	-296.16776855336\\
56	0.15156	-317.957012837721	-317.957012837721\\
56	0.15522	-340.965875754368	-340.965875754368\\
56	0.15888	-365.194357303297	-365.194357303297\\
56	0.16254	-390.642457484511	-390.642457484511\\
56	0.1662	-417.310176298009	-417.310176298009\\
56	0.16986	-445.197513743791	-445.197513743791\\
56	0.17352	-474.304469821856	-474.304469821856\\
56	0.17718	-504.631044532207	-504.631044532207\\
56	0.18084	-536.17723787484	-536.17723787484\\
56	0.1845	-568.943049849759	-568.943049849759\\
56	0.18816	-602.928480456961	-602.928480456961\\
56	0.19182	-638.133529696446	-638.133529696446\\
56	0.19548	-674.558197568217	-674.558197568217\\
56	0.19914	-712.202484072271	-712.202484072271\\
56	0.2028	-751.066389208609	-751.066389208609\\
56	0.20646	-791.149912977232	-791.149912977232\\
56	0.21012	-832.453055378138	-832.453055378138\\
56	0.21378	-874.975816411328	-874.975816411328\\
56	0.21744	-918.718196076802	-918.718196076802\\
56	0.2211	-963.68019437456	-963.68019437456\\
56	0.22476	-1009.8618113046	-1009.8618113046\\
56	0.22842	-1057.26304686693	-1057.26304686693\\
56	0.23208	-1105.88390106154	-1105.88390106154\\
56	0.23574	-1155.72437388843	-1155.72437388843\\
56	0.2394	-1206.78446534761	-1206.78446534761\\
56	0.24306	-1259.06417543907	-1259.06417543907\\
56	0.24672	-1312.56350416282	-1312.56350416282\\
56	0.25038	-1367.28245151885	-1367.28245151885\\
56	0.25404	-1423.22101750717	-1423.22101750717\\
56	0.2577	-1480.37920212777	-1480.37920212777\\
56	0.26136	-1538.75700538065	-1538.75700538065\\
56	0.26502	-1598.35442726581	-1598.35442726581\\
56	0.26868	-1659.17146778326	-1659.17146778326\\
56	0.27234	-1721.208126933	-1721.208126933\\
56	0.276	-1784.46440471502	-1784.46440471502\\
56.375	0.093	-113.514388617176	-113.514388617176\\
56.375	0.09666	-117.038855221496	-117.038855221496\\
56.375	0.10032	-121.7829404581	-121.7829404581\\
56.375	0.10398	-127.746644326988	-127.746644326988\\
56.375	0.10764	-134.92996682816	-134.92996682816\\
56.375	0.1113	-143.332907961616	-143.332907961616\\
56.375	0.11496	-152.955467727356	-152.955467727356\\
56.375	0.11862	-163.797646125381	-163.797646125381\\
56.375	0.12228	-175.859443155688	-175.859443155688\\
56.375	0.12594	-189.140858818281	-189.140858818281\\
56.375	0.1296	-203.641893113157	-203.641893113157\\
56.375	0.13326	-219.362546040317	-219.362546040317\\
56.375	0.13692	-236.302817599762	-236.302817599762\\
56.375	0.14058	-254.46270779149	-254.46270779149\\
56.375	0.14424	-273.842216615502	-273.842216615502\\
56.375	0.1479	-294.441344071799	-294.441344071799\\
56.375	0.15156	-316.260090160379	-316.260090160379\\
56.375	0.15522	-339.298454881244	-339.298454881244\\
56.375	0.15888	-363.556438234392	-363.556438234392\\
56.375	0.16254	-389.034040219824	-389.034040219824\\
56.375	0.1662	-415.731260837541	-415.731260837541\\
56.375	0.16986	-443.648100087542	-443.648100087542\\
56.375	0.17352	-472.784557969826	-472.784557969826\\
56.375	0.17718	-503.140634484395	-503.140634484395\\
56.375	0.18084	-534.716329631247	-534.716329631247\\
56.375	0.1845	-567.511643410385	-567.511643410385\\
56.375	0.18816	-601.526575821805	-601.526575821805\\
56.375	0.19182	-636.76112686551	-636.76112686551\\
56.375	0.19548	-673.215296541498	-673.215296541498\\
56.375	0.19914	-710.889084849772	-710.889084849772\\
56.375	0.2028	-749.782491790329	-749.782491790329\\
56.375	0.20646	-789.89551736317	-789.89551736317\\
56.375	0.21012	-831.228161568294	-831.228161568294\\
56.375	0.21378	-873.780424405703	-873.780424405703\\
56.375	0.21744	-917.552305875396	-917.552305875396\\
56.375	0.2211	-962.543805977373	-962.543805977373\\
56.375	0.22476	-1008.75492471163	-1008.75492471163\\
56.375	0.22842	-1056.18566207818	-1056.18566207818\\
56.375	0.23208	-1104.83601807701	-1104.83601807701\\
56.375	0.23574	-1154.70599270812	-1154.70599270812\\
56.375	0.2394	-1205.79558597152	-1205.79558597152\\
56.375	0.24306	-1258.1047978672	-1258.1047978672\\
56.375	0.24672	-1311.63362839517	-1311.63362839517\\
56.375	0.25038	-1366.38207755542	-1366.38207755542\\
56.375	0.25404	-1422.35014534795	-1422.35014534795\\
56.375	0.2577	-1479.53783177277	-1479.53783177277\\
56.375	0.26136	-1537.94513682987	-1537.94513682987\\
56.375	0.26502	-1597.57206051925	-1597.57206051925\\
56.375	0.26868	-1658.41860284092	-1658.41860284092\\
56.375	0.27234	-1720.48476379487	-1720.48476379487\\
56.375	0.276	-1783.77054338111	-1783.77054338111\\
56.75	0.093	-111.484040437008	-111.484040437008\\
56.75	0.09666	-115.038008845547	-115.038008845547\\
56.75	0.10032	-119.811595886369	-119.811595886369\\
56.75	0.10398	-125.804801559476	-125.804801559476\\
56.75	0.10764	-133.017625864867	-133.017625864867\\
56.75	0.1113	-141.450068802541	-141.450068802541\\
56.75	0.11496	-151.102130372501	-151.102130372501\\
56.75	0.11862	-161.973810574743	-161.973810574743\\
56.75	0.12228	-174.065109409271	-174.065109409271\\
56.75	0.12594	-187.376026876081	-187.376026876081\\
56.75	0.1296	-201.906562975176	-201.906562975176\\
56.75	0.13326	-217.656717706555	-217.656717706555\\
56.75	0.13692	-234.626491070218	-234.626491070218\\
56.75	0.14058	-252.815883066165	-252.815883066165\\
56.75	0.14424	-272.224893694397	-272.224893694397\\
56.75	0.1479	-292.853522954911	-292.853522954911\\
56.75	0.15156	-314.701770847711	-314.701770847711\\
56.75	0.15522	-337.769637372793	-337.769637372793\\
56.75	0.15888	-362.057122530161	-362.057122530161\\
56.75	0.16254	-387.564226319812	-387.564226319812\\
56.75	0.1662	-414.290948741748	-414.290948741748\\
56.75	0.16986	-442.237289795966	-442.237289795966\\
56.75	0.17352	-471.40324948247	-471.40324948247\\
56.75	0.17718	-501.788827801257	-501.788827801257\\
56.75	0.18084	-533.394024752329	-533.394024752329\\
56.75	0.1845	-566.218840335684	-566.218840335684\\
56.75	0.18816	-600.263274551324	-600.263274551324\\
56.75	0.19182	-635.527327399247	-635.527327399247\\
56.75	0.19548	-672.010998879455	-672.010998879455\\
56.75	0.19914	-709.714288991946	-709.714288991946\\
56.75	0.2028	-748.637197736722	-748.637197736722\\
56.75	0.20646	-788.779725113782	-788.779725113782\\
56.75	0.21012	-830.141871123125	-830.141871123125\\
56.75	0.21378	-872.723635764753	-872.723635764753\\
56.75	0.21744	-916.525019038664	-916.525019038664\\
56.75	0.2211	-961.54602094486	-961.54602094486\\
56.75	0.22476	-1007.78664148334	-1007.78664148334\\
56.75	0.22842	-1055.2468806541	-1055.2468806541\\
56.75	0.23208	-1103.92673845715	-1103.92673845715\\
56.75	0.23574	-1153.82621489248	-1153.82621489248\\
56.75	0.2394	-1204.9453099601	-1204.9453099601\\
56.75	0.24306	-1257.28402366	-1257.28402366\\
56.75	0.24672	-1310.84235599218	-1310.84235599218\\
56.75	0.25038	-1365.62030695665	-1365.62030695665\\
56.75	0.25404	-1421.6178765534	-1421.6178765534\\
56.75	0.2577	-1478.83506478244	-1478.83506478244\\
56.75	0.26136	-1537.27187164376	-1537.27187164376\\
56.75	0.26502	-1596.92829713736	-1596.92829713736\\
56.75	0.26868	-1657.80434126325	-1657.80434126325\\
56.75	0.27234	-1719.90000402142	-1719.90000402142\\
56.75	0.276	-1783.21528541188	-1783.21528541188\\
57.125	0.093	-109.592295621513	-109.592295621513\\
57.125	0.09666	-113.17576583427	-113.17576583427\\
57.125	0.10032	-117.978854679312	-117.978854679312\\
57.125	0.10398	-124.001562156637	-124.001562156637\\
57.125	0.10764	-131.243888266247	-131.243888266247\\
57.125	0.1113	-139.70583300814	-139.70583300814\\
57.125	0.11496	-149.387396382318	-149.387396382318\\
57.125	0.11862	-160.288578388779	-160.288578388779\\
57.125	0.12228	-172.409379027525	-172.409379027525\\
57.125	0.12594	-185.749798298554	-185.749798298554\\
57.125	0.1296	-200.309836201868	-200.309836201868\\
57.125	0.13326	-216.089492737466	-216.089492737466\\
57.125	0.13692	-233.088767905348	-233.088767905348\\
57.125	0.14058	-251.307661705513	-251.307661705513\\
57.125	0.14424	-270.746174137963	-270.746174137963\\
57.125	0.1479	-291.404305202697	-291.404305202697\\
57.125	0.15156	-313.282054899715	-313.282054899715\\
57.125	0.15522	-336.379423229016	-336.379423229016\\
57.125	0.15888	-360.696410190603	-360.696410190603\\
57.125	0.16254	-386.233015784472	-386.233015784472\\
57.125	0.1662	-412.989240010627	-412.989240010627\\
57.125	0.16986	-440.965082869064	-440.965082869064\\
57.125	0.17352	-470.160544359787	-470.160544359787\\
57.125	0.17718	-500.575624482792	-500.575624482792\\
57.125	0.18084	-532.210323238083	-532.210323238083\\
57.125	0.1845	-565.064640625656	-565.064640625656\\
57.125	0.18816	-599.138576645515	-599.138576645515\\
57.125	0.19182	-634.432131297657	-634.432131297657\\
57.125	0.19548	-670.945304582084	-670.945304582084\\
57.125	0.19914	-708.678096498794	-708.678096498794\\
57.125	0.2028	-747.630507047788	-747.630507047788\\
57.125	0.20646	-787.802536229067	-787.802536229067\\
57.125	0.21012	-829.194184042629	-829.194184042629\\
57.125	0.21378	-871.805450488475	-871.805450488475\\
57.125	0.21744	-915.636335566605	-915.636335566605\\
57.125	0.2211	-960.68683927702	-960.68683927702\\
57.125	0.22476	-1006.95696161972	-1006.95696161972\\
57.125	0.22842	-1054.4467025947	-1054.4467025947\\
57.125	0.23208	-1103.15606220197	-1103.15606220197\\
57.125	0.23574	-1153.08504044152	-1153.08504044152\\
57.125	0.2394	-1204.23363731335	-1204.23363731335\\
57.125	0.24306	-1256.60185281747	-1256.60185281747\\
57.125	0.24672	-1310.18968695387	-1310.18968695387\\
57.125	0.25038	-1364.99713972256	-1364.99713972256\\
57.125	0.25404	-1421.02421112353	-1421.02421112353\\
57.125	0.2577	-1478.27090115679	-1478.27090115679\\
57.125	0.26136	-1536.73720982233	-1536.73720982233\\
57.125	0.26502	-1596.42313712015	-1596.42313712015\\
57.125	0.26868	-1657.32868305025	-1657.32868305025\\
57.125	0.27234	-1719.45384761265	-1719.45384761265\\
57.125	0.276	-1782.79863080732	-1782.79863080732\\
57.5	0.093	-107.839154170691	-107.839154170691\\
57.5	0.09666	-111.452126187667	-111.452126187667\\
57.5	0.10032	-116.284716836927	-116.284716836927\\
57.5	0.10398	-122.336926118471	-122.336926118471\\
57.5	0.10764	-129.608754032299	-129.608754032299\\
57.5	0.1113	-138.100200578411	-138.100200578411\\
57.5	0.11496	-147.811265756808	-147.811265756808\\
57.5	0.11862	-158.741949567488	-158.741949567488\\
57.5	0.12228	-170.892252010453	-170.892252010453\\
57.5	0.12594	-184.262173085701	-184.262173085701\\
57.5	0.1296	-198.851712793233	-198.851712793233\\
57.5	0.13326	-214.660871133049	-214.660871133049\\
57.5	0.13692	-231.68964810515	-231.68964810515\\
57.5	0.14058	-249.938043709534	-249.938043709534\\
57.5	0.14424	-269.406057946203	-269.406057946203\\
57.5	0.1479	-290.093690815155	-290.093690815155\\
57.5	0.15156	-312.000942316392	-312.000942316392\\
57.5	0.15522	-335.127812449912	-335.127812449912\\
57.5	0.15888	-359.474301215718	-359.474301215718\\
57.5	0.16254	-385.040408613806	-385.040408613806\\
57.5	0.1662	-411.826134644179	-411.826134644179\\
57.5	0.16986	-439.831479306835	-439.831479306835\\
57.5	0.17352	-469.056442601777	-469.056442601777\\
57.5	0.17718	-499.501024529	-499.501024529\\
57.5	0.18084	-531.16522508851	-531.16522508851\\
57.5	0.1845	-564.049044280302	-564.049044280302\\
57.5	0.18816	-598.15248210438	-598.15248210438\\
57.5	0.19182	-633.47553856074	-633.47553856074\\
57.5	0.19548	-670.018213649385	-670.018213649385\\
57.5	0.19914	-707.780507370314	-707.780507370314\\
57.5	0.2028	-746.762419723528	-746.762419723528\\
57.5	0.20646	-786.963950709025	-786.963950709025\\
57.5	0.21012	-828.385100326806	-828.385100326806\\
57.5	0.21378	-871.02586857687	-871.02586857687\\
57.5	0.21744	-914.88625545922	-914.88625545922\\
57.5	0.2211	-959.966260973853	-959.966260973853\\
57.5	0.22476	-1006.26588512077	-1006.26588512077\\
57.5	0.22842	-1053.78512789997	-1053.78512789997\\
57.5	0.23208	-1102.52398931146	-1102.52398931146\\
57.5	0.23574	-1152.48246935523	-1152.48246935523\\
57.5	0.2394	-1203.66056803128	-1203.66056803128\\
57.5	0.24306	-1256.05828533962	-1256.05828533962\\
57.5	0.24672	-1309.67562128024	-1309.67562128024\\
57.5	0.25038	-1364.51257585314	-1364.51257585314\\
57.5	0.25404	-1420.56914905833	-1420.56914905833\\
57.5	0.2577	-1477.84534089581	-1477.84534089581\\
57.5	0.26136	-1536.34115136556	-1536.34115136556\\
57.5	0.26502	-1596.05658046761	-1596.05658046761\\
57.5	0.26868	-1656.99162820193	-1656.99162820193\\
57.5	0.27234	-1719.14629456854	-1719.14629456854\\
57.5	0.276	-1782.52057956743	-1782.52057956743\\
57.875	0.093	-106.224616084543	-106.224616084543\\
57.875	0.09666	-109.867089905738	-109.867089905738\\
57.875	0.10032	-114.729182359216	-114.729182359216\\
57.875	0.10398	-120.810893444979	-120.810893444979\\
57.875	0.10764	-128.112223163026	-128.112223163026\\
57.875	0.1113	-136.633171513357	-136.633171513357\\
57.875	0.11496	-146.373738495972	-146.373738495972\\
57.875	0.11862	-157.333924110871	-157.333924110871\\
57.875	0.12228	-169.513728358054	-169.513728358054\\
57.875	0.12594	-182.913151237521	-182.913151237521\\
57.875	0.1296	-197.532192749272	-197.532192749272\\
57.875	0.13326	-213.370852893307	-213.370852893307\\
57.875	0.13692	-230.429131669626	-230.429131669626\\
57.875	0.14058	-248.70702907823	-248.70702907823\\
57.875	0.14424	-268.204545119117	-268.204545119117\\
57.875	0.1479	-288.921679792289	-288.921679792289\\
57.875	0.15156	-310.858433097743	-310.858433097743\\
57.875	0.15522	-334.014805035483	-334.014805035483\\
57.875	0.15888	-358.390795605506	-358.390795605506\\
57.875	0.16254	-383.986404807814	-383.986404807814\\
57.875	0.1662	-410.801632642405	-410.801632642405\\
57.875	0.16986	-438.836479109281	-438.836479109281\\
57.875	0.17352	-468.09094420844	-468.09094420844\\
57.875	0.17718	-498.565027939883	-498.565027939883\\
57.875	0.18084	-530.25873030361	-530.25873030361\\
57.875	0.1845	-563.172051299623	-563.172051299623\\
57.875	0.18816	-597.304990927918	-597.304990927918\\
57.875	0.19182	-632.657549188497	-632.657549188497\\
57.875	0.19548	-669.229726081361	-669.229726081361\\
57.875	0.19914	-707.021521606509	-707.021521606509\\
57.875	0.2028	-746.032935763941	-746.032935763941\\
57.875	0.20646	-786.263968553657	-786.263968553657\\
57.875	0.21012	-827.714619975657	-827.714619975657\\
57.875	0.21378	-870.38489002994	-870.38489002994\\
57.875	0.21744	-914.274778716508	-914.274778716508\\
57.875	0.2211	-959.38428603536	-959.38428603536\\
57.875	0.22476	-1005.7134119865	-1005.7134119865\\
57.875	0.22842	-1053.26215656992	-1053.26215656992\\
57.875	0.23208	-1102.03051978562	-1102.03051978562\\
57.875	0.23574	-1152.01850163361	-1152.01850163361\\
57.875	0.2394	-1203.22610211388	-1203.22610211388\\
57.875	0.24306	-1255.65332122644	-1255.65332122644\\
57.875	0.24672	-1309.30015897128	-1309.30015897128\\
57.875	0.25038	-1364.1666153484	-1364.1666153484\\
57.875	0.25404	-1420.25269035781	-1420.25269035781\\
57.875	0.2577	-1477.5583839995	-1477.5583839995\\
57.875	0.26136	-1536.08369627348	-1536.08369627348\\
57.875	0.26502	-1595.82862717974	-1595.82862717974\\
57.875	0.26868	-1656.79317671828	-1656.79317671828\\
57.875	0.27234	-1718.97734488911	-1718.97734488911\\
57.875	0.276	-1782.38113169222	-1782.38113169222\\
58.25	0.093	-104.748681363067	-104.748681363067\\
58.25	0.09666	-108.420656988481	-108.420656988481\\
58.25	0.10032	-113.312251246178	-113.312251246178\\
58.25	0.10398	-119.42346413616	-119.42346413616\\
58.25	0.10764	-126.754295658426	-126.754295658426\\
58.25	0.1113	-135.304745812976	-135.304745812976\\
58.25	0.11496	-145.074814599809	-145.074814599809\\
58.25	0.11862	-156.064502018927	-156.064502018927\\
58.25	0.12228	-168.273808070328	-168.273808070328\\
58.25	0.12594	-181.702732754014	-181.702732754014\\
58.25	0.1296	-196.351276069984	-196.351276069984\\
58.25	0.13326	-212.219438018238	-212.219438018238\\
58.25	0.13692	-229.307218598776	-229.307218598776\\
58.25	0.14058	-247.614617811598	-247.614617811598\\
58.25	0.14424	-267.141635656703	-267.141635656703\\
58.25	0.1479	-287.888272134094	-287.888272134094\\
58.25	0.15156	-309.854527243767	-309.854527243767\\
58.25	0.15522	-333.040400985726	-333.040400985726\\
58.25	0.15888	-357.445893359968	-357.445893359968\\
58.25	0.16254	-383.071004366494	-383.071004366494\\
58.25	0.1662	-409.915734005304	-409.915734005304\\
58.25	0.16986	-437.980082276398	-437.980082276398\\
58.25	0.17352	-467.264049179776	-467.264049179776\\
58.25	0.17718	-497.767634715439	-497.767634715439\\
58.25	0.18084	-529.490838883384	-529.490838883384\\
58.25	0.1845	-562.433661683615	-562.433661683615\\
58.25	0.18816	-596.596103116129	-596.596103116129\\
58.25	0.19182	-631.978163180927	-631.978163180927\\
58.25	0.19548	-668.57984187801	-668.57984187801\\
58.25	0.19914	-706.401139207377	-706.401139207377\\
58.25	0.2028	-745.442055169027	-745.442055169027\\
58.25	0.20646	-785.702589762962	-785.702589762962\\
58.25	0.21012	-827.18274298918	-827.18274298918\\
58.25	0.21378	-869.882514847682	-869.882514847682\\
58.25	0.21744	-913.801905338469	-913.801905338469\\
58.25	0.2211	-958.94091446154	-958.94091446154\\
58.25	0.22476	-1005.29954221689	-1005.29954221689\\
58.25	0.22842	-1052.87778860453	-1052.87778860453\\
58.25	0.23208	-1101.67565362446	-1101.67565362446\\
58.25	0.23574	-1151.69313727666	-1151.69313727666\\
58.25	0.2394	-1202.93023956115	-1202.93023956115\\
58.25	0.24306	-1255.38696047793	-1255.38696047793\\
58.25	0.24672	-1309.06330002699	-1309.06330002699\\
58.25	0.25038	-1363.95925820833	-1363.95925820833\\
58.25	0.25404	-1420.07483502196	-1420.07483502196\\
58.25	0.2577	-1477.41003046787	-1477.41003046787\\
58.25	0.26136	-1535.96484454606	-1535.96484454606\\
58.25	0.26502	-1595.73927725654	-1595.73927725654\\
58.25	0.26868	-1656.7333285993	-1656.7333285993\\
58.25	0.27234	-1718.94699857435	-1718.94699857435\\
58.25	0.276	-1782.38028718168	-1782.38028718168\\
58.625	0.093	-103.411350006266	-103.411350006266\\
58.625	0.09666	-107.112827435898	-107.112827435898\\
58.625	0.10032	-112.033923497815	-112.033923497815\\
58.625	0.10398	-118.174638192015	-118.174638192015\\
58.625	0.10764	-125.534971518499	-125.534971518499\\
58.625	0.1113	-134.114923477267	-134.114923477267\\
58.625	0.11496	-143.91449406832	-143.91449406832\\
58.625	0.11862	-154.933683291656	-154.933683291656\\
58.625	0.12228	-167.172491147277	-167.172491147277\\
58.625	0.12594	-180.630917635181	-180.630917635181\\
58.625	0.1296	-195.30896275537	-195.30896275537\\
58.625	0.13326	-211.206626507842	-211.206626507842\\
58.625	0.13692	-228.323908892599	-228.323908892599\\
58.625	0.14058	-246.660809909639	-246.660809909639\\
58.625	0.14424	-266.217329558965	-266.217329558965\\
58.625	0.1479	-286.993467840573	-286.993467840573\\
58.625	0.15156	-308.989224754466	-308.989224754466\\
58.625	0.15522	-332.204600300642	-332.204600300642\\
58.625	0.15888	-356.639594479103	-356.639594479103\\
58.625	0.16254	-382.294207289848	-382.294207289848\\
58.625	0.1662	-409.168438732877	-409.168438732877\\
58.625	0.16986	-437.26228880819	-437.26228880819\\
58.625	0.17352	-466.575757515787	-466.575757515787\\
58.625	0.17718	-497.108844855667	-497.108844855667\\
58.625	0.18084	-528.861550827833	-528.861550827833\\
58.625	0.1845	-561.833875432281	-561.833875432281\\
58.625	0.18816	-596.025818669015	-596.025818669015\\
58.625	0.19182	-631.437380538031	-631.437380538031\\
58.625	0.19548	-668.068561039333	-668.068561039333\\
58.625	0.19914	-705.919360172918	-705.919360172918\\
58.625	0.2028	-744.989777938787	-744.989777938787\\
58.625	0.20646	-785.279814336941	-785.279814336941\\
58.625	0.21012	-826.789469367377	-826.789469367377\\
58.625	0.21378	-869.518743030098	-869.518743030098\\
58.625	0.21744	-913.467635325104	-913.467635325104\\
58.625	0.2211	-958.636146252393	-958.636146252393\\
58.625	0.22476	-1005.02427581197	-1005.02427581197\\
58.625	0.22842	-1052.63202400382	-1052.63202400382\\
58.625	0.23208	-1101.45939082797	-1101.45939082797\\
58.625	0.23574	-1151.50637628439	-1151.50637628439\\
58.625	0.2394	-1202.7729803731	-1202.7729803731\\
58.625	0.24306	-1255.25920309409	-1255.25920309409\\
58.625	0.24672	-1308.96504444737	-1308.96504444737\\
58.625	0.25038	-1363.89050443293	-1363.89050443293\\
58.625	0.25404	-1420.03558305078	-1420.03558305078\\
58.625	0.2577	-1477.40028030091	-1477.40028030091\\
58.625	0.26136	-1535.98459618332	-1535.98459618332\\
58.625	0.26502	-1595.78853069802	-1595.78853069802\\
58.625	0.26868	-1656.812083845	-1656.812083845\\
58.625	0.27234	-1719.05525562427	-1719.05525562427\\
58.625	0.276	-1782.51804603582	-1782.51804603582\\
59	0.093	-102.212622014138	-102.212622014138\\
59	0.09666	-105.943601247988	-105.943601247988\\
59	0.10032	-110.894199114124	-110.894199114124\\
59	0.10398	-117.064415612542	-117.064415612542\\
59	0.10764	-124.454250743245	-124.454250743245\\
59	0.1113	-133.063704506232	-133.063704506232\\
59	0.11496	-142.892776901504	-142.892776901504\\
59	0.11862	-153.941467929059	-153.941467929059\\
59	0.12228	-166.209777588898	-166.209777588898\\
59	0.12594	-179.697705881021	-179.697705881021\\
59	0.1296	-194.405252805429	-194.405252805429\\
59	0.13326	-210.33241836212	-210.33241836212\\
59	0.13692	-227.479202551095	-227.479202551095\\
59	0.14058	-245.845605372354	-245.845605372354\\
59	0.14424	-265.431626825898	-265.431626825898\\
59	0.1479	-286.237266911725	-286.237266911725\\
59	0.15156	-308.262525629837	-308.262525629837\\
59	0.15522	-331.507402980232	-331.507402980232\\
59	0.15888	-355.971898962912	-355.971898962912\\
59	0.16254	-381.656013577875	-381.656013577875\\
59	0.1662	-408.559746825123	-408.559746825123\\
59	0.16986	-436.683098704654	-436.683098704654\\
59	0.17352	-466.02606921647	-466.02606921647\\
59	0.17718	-496.588658360569	-496.588658360569\\
59	0.18084	-528.370866136953	-528.370866136953\\
59	0.1845	-561.372692545621	-561.372692545621\\
59	0.18816	-595.594137586573	-595.594137586573\\
59	0.19182	-631.035201259808	-631.035201259808\\
59	0.19548	-667.695883565328	-667.695883565328\\
59	0.19914	-705.576184503132	-705.576184503132\\
59	0.2028	-744.67610407322	-744.67610407322\\
59	0.20646	-784.995642275592	-784.995642275592\\
59	0.21012	-826.534799110248	-826.534799110248\\
59	0.21378	-869.293574577187	-869.293574577187\\
59	0.21744	-913.271968676412	-913.271968676412\\
59	0.2211	-958.46998140792	-958.46998140792\\
59	0.22476	-1004.88761277171	-1004.88761277171\\
59	0.22842	-1052.52486276779	-1052.52486276779\\
59	0.23208	-1101.38173139615	-1101.38173139615\\
59	0.23574	-1151.45821865679	-1151.45821865679\\
59	0.2394	-1202.75432454972	-1202.75432454972\\
59	0.24306	-1255.27004907493	-1255.27004907493\\
59	0.24672	-1309.00539223243	-1309.00539223243\\
59	0.25038	-1363.96035402221	-1363.96035402221\\
59	0.25404	-1420.13493444427	-1420.13493444427\\
59	0.2577	-1477.52913349862	-1477.52913349862\\
59	0.26136	-1536.14295118525	-1536.14295118525\\
59	0.26502	-1595.97638750417	-1595.97638750417\\
59	0.26868	-1657.02944245537	-1657.02944245537\\
59	0.27234	-1719.30211603886	-1719.30211603886\\
59	0.276	-1782.79440825462	-1782.79440825462\\
59.375	0.093	-101.152497386683	-101.152497386683\\
59.375	0.09666	-104.912978424752	-104.912978424752\\
59.375	0.10032	-109.893078095106	-109.893078095106\\
59.375	0.10398	-116.092796397744	-116.092796397744\\
59.375	0.10764	-123.512133332666	-123.512133332666\\
59.375	0.1113	-132.151088899871	-132.151088899871\\
59.375	0.11496	-142.009663099361	-142.009663099361\\
59.375	0.11862	-153.087855931135	-153.087855931135\\
59.375	0.12228	-165.385667395193	-165.385667395193\\
59.375	0.12594	-178.903097491535	-178.903097491535\\
59.375	0.1296	-193.640146220161	-193.640146220161\\
59.375	0.13326	-209.596813581071	-209.596813581071\\
59.375	0.13692	-226.773099574265	-226.773099574265\\
59.375	0.14058	-245.169004199743	-245.169004199743\\
59.375	0.14424	-264.784527457505	-264.784527457505\\
59.375	0.1479	-285.619669347551	-285.619669347551\\
59.375	0.15156	-307.674429869881	-307.674429869881\\
59.375	0.15522	-330.948809024495	-330.948809024495\\
59.375	0.15888	-355.442806811394	-355.442806811394\\
59.375	0.16254	-381.156423230576	-381.156423230576\\
59.375	0.1662	-408.089658282042	-408.089658282042\\
59.375	0.16986	-436.242511965793	-436.242511965793\\
59.375	0.17352	-465.614984281827	-465.614984281827\\
59.375	0.17718	-496.207075230145	-496.207075230145\\
59.375	0.18084	-528.018784810747	-528.018784810747\\
59.375	0.1845	-561.050113023634	-561.050113023634\\
59.375	0.18816	-595.301059868805	-595.301059868805\\
59.375	0.19182	-630.771625346259	-630.771625346259\\
59.375	0.19548	-667.461809455997	-667.461809455997\\
59.375	0.19914	-705.37161219802	-705.37161219802\\
59.375	0.2028	-744.501033572327	-744.501033572327\\
59.375	0.20646	-784.850073578918	-784.850073578918\\
59.375	0.21012	-826.418732217792	-826.418732217792\\
59.375	0.21378	-869.207009488951	-869.207009488951\\
59.375	0.21744	-913.214905392393	-913.214905392393\\
59.375	0.2211	-958.44241992812	-958.44241992812\\
59.375	0.22476	-1004.88955309613	-1004.88955309613\\
59.375	0.22842	-1052.55630489643	-1052.55630489643\\
59.375	0.23208	-1101.442675329	-1101.442675329\\
59.375	0.23574	-1151.54866439387	-1151.54866439387\\
59.375	0.2394	-1202.87427209101	-1202.87427209101\\
59.375	0.24306	-1255.41949842045	-1255.41949842045\\
59.375	0.24672	-1309.18434338216	-1309.18434338216\\
59.375	0.25038	-1364.16880697616	-1364.16880697616\\
59.375	0.25404	-1420.37288920244	-1420.37288920244\\
59.375	0.2577	-1477.79659006101	-1477.79659006101\\
59.375	0.26136	-1536.43990955186	-1536.43990955186\\
59.375	0.26502	-1596.302847675	-1596.302847675\\
59.375	0.26868	-1657.38540443041	-1657.38540443041\\
59.375	0.27234	-1719.68757981812	-1719.68757981812\\
59.375	0.276	-1783.2093738381	-1783.2093738381\\
59.75	0.093	-100.230976123901	-100.230976123901\\
59.75	0.09666	-104.020958966189	-104.020958966189\\
59.75	0.10032	-109.030560440761	-109.030560440761\\
59.75	0.10398	-115.259780547618	-115.259780547618\\
59.75	0.10764	-122.708619286759	-122.708619286759\\
59.75	0.1113	-131.377076658183	-131.377076658183\\
59.75	0.11496	-141.265152661892	-141.265152661892\\
59.75	0.11862	-152.372847297885	-152.372847297885\\
59.75	0.12228	-164.700160566161	-164.700160566161\\
59.75	0.12594	-178.247092466722	-178.247092466722\\
59.75	0.1296	-193.013642999566	-193.013642999566\\
59.75	0.13326	-208.999812164695	-208.999812164695\\
59.75	0.13692	-226.205599962107	-226.205599962107\\
59.75	0.14058	-244.631006391805	-244.631006391805\\
59.75	0.14424	-264.276031453785	-264.276031453785\\
59.75	0.1479	-285.14067514805	-285.14067514805\\
59.75	0.15156	-307.224937474598	-307.224937474598\\
59.75	0.15522	-330.528818433432	-330.528818433432\\
59.75	0.15888	-355.052318024549	-355.052318024549\\
59.75	0.16254	-380.79543624795	-380.79543624795\\
59.75	0.1662	-407.758173103635	-407.758173103635\\
59.75	0.16986	-435.940528591604	-435.940528591604\\
59.75	0.17352	-465.342502711856	-465.342502711856\\
59.75	0.17718	-495.964095464394	-495.964095464394\\
59.75	0.18084	-527.805306849215	-527.805306849215\\
59.75	0.1845	-560.86613686632	-560.86613686632\\
59.75	0.18816	-595.146585515709	-595.146585515709\\
59.75	0.19182	-630.646652797382	-630.646652797382\\
59.75	0.19548	-667.36633871134	-667.36633871134\\
59.75	0.19914	-705.305643257582	-705.305643257582\\
59.75	0.2028	-744.464566436107	-744.464566436107\\
59.75	0.20646	-784.843108246916	-784.843108246916\\
59.75	0.21012	-826.441268690009	-826.441268690009\\
59.75	0.21378	-869.259047765386	-869.259047765386\\
59.75	0.21744	-913.296445473047	-913.296445473047\\
59.75	0.2211	-958.553461812993	-958.553461812993\\
59.75	0.22476	-1005.03009678522	-1005.03009678522\\
59.75	0.22842	-1052.72635038974	-1052.72635038974\\
59.75	0.23208	-1101.64222262653	-1101.64222262653\\
59.75	0.23574	-1151.77771349562	-1151.77771349562\\
59.75	0.2394	-1203.13282299698	-1203.13282299698\\
59.75	0.24306	-1255.70755113063	-1255.70755113063\\
59.75	0.24672	-1309.50189789656	-1309.50189789656\\
59.75	0.25038	-1364.51586329478	-1364.51586329478\\
59.75	0.25404	-1420.74944732528	-1420.74944732528\\
59.75	0.2577	-1478.20264998807	-1478.20264998807\\
59.75	0.26136	-1536.87547128314	-1536.87547128314\\
59.75	0.26502	-1596.76791121049	-1596.76791121049\\
59.75	0.26868	-1657.87996977013	-1657.87996977013\\
59.75	0.27234	-1720.21164696205	-1720.21164696205\\
59.75	0.276	-1783.76294278626	-1783.76294278626\\
60.125	0.093	-99.4480582257924	-99.4480582257924\\
60.125	0.09666	-103.2675428723	-103.2675428723\\
60.125	0.10032	-108.30664615109	-108.30664615109\\
60.125	0.10398	-114.565368062166	-114.565368062166\\
60.125	0.10764	-122.043708605525	-122.043708605525\\
60.125	0.1113	-130.741667781168	-130.741667781168\\
60.125	0.11496	-140.659245589095	-140.659245589095\\
60.125	0.11862	-151.796442029307	-151.796442029307\\
60.125	0.12228	-164.153257101802	-164.153257101802\\
60.125	0.12594	-177.729690806582	-177.729690806582\\
60.125	0.1296	-192.525743143645	-192.525743143645\\
60.125	0.13326	-208.541414112992	-208.541414112992\\
60.125	0.13692	-225.776703714624	-225.776703714624\\
60.125	0.14058	-244.23161194854	-244.23161194854\\
60.125	0.14424	-263.906138814738	-263.906138814738\\
60.125	0.1479	-284.800284313223	-284.800284313223\\
60.125	0.15156	-306.91404844399	-306.91404844399\\
60.125	0.15522	-330.247431207041	-330.247431207041\\
60.125	0.15888	-354.800432602377	-354.800432602377\\
60.125	0.16254	-380.573052629997	-380.573052629997\\
60.125	0.1662	-407.5652912899	-407.5652912899\\
60.125	0.16986	-435.777148582089	-435.777148582089\\
60.125	0.17352	-465.20862450656	-465.20862450656\\
60.125	0.17718	-495.859719063316	-495.859719063316\\
60.125	0.18084	-527.730432252355	-527.730432252355\\
60.125	0.1845	-560.82076407368	-560.82076407368\\
60.125	0.18816	-595.130714527287	-595.130714527287\\
60.125	0.19182	-630.660283613179	-630.660283613179\\
60.125	0.19548	-667.409471331355	-667.409471331355\\
60.125	0.19914	-705.378277681816	-705.378277681816\\
60.125	0.2028	-744.56670266456	-744.56670266456\\
60.125	0.20646	-784.974746279588	-784.974746279588\\
60.125	0.21012	-826.6024085269	-826.6024085269\\
60.125	0.21378	-869.449689406496	-869.449689406496\\
60.125	0.21744	-913.516588918375	-913.516588918375\\
60.125	0.2211	-958.80310706254	-958.80310706254\\
60.125	0.22476	-1005.30924383899	-1005.30924383899\\
60.125	0.22842	-1053.03499924772	-1053.03499924772\\
60.125	0.23208	-1101.98037328874	-1101.98037328874\\
60.125	0.23574	-1152.14536596204	-1152.14536596204\\
60.125	0.2394	-1203.52997726762	-1203.52997726762\\
60.125	0.24306	-1256.13420720549	-1256.13420720549\\
60.125	0.24672	-1309.95805577564	-1309.95805577564\\
60.125	0.25038	-1365.00152297808	-1365.00152297808\\
60.125	0.25404	-1421.2646088128	-1421.2646088128\\
60.125	0.2577	-1478.7473132798	-1478.7473132798\\
60.125	0.26136	-1537.44963637909	-1537.44963637909\\
60.125	0.26502	-1597.37157811066	-1597.37157811066\\
60.125	0.26868	-1658.51313847452	-1658.51313847452\\
60.125	0.27234	-1720.87431747066	-1720.87431747066\\
60.125	0.276	-1784.45511509909	-1784.45511509909\\
60.5	0.093	-98.8037436923577	-98.8037436923577\\
60.5	0.09666	-102.652730143084	-102.652730143084\\
60.5	0.10032	-107.721335226094	-107.721335226094\\
60.5	0.10398	-114.009558941387	-114.009558941387\\
60.5	0.10764	-121.517401288965	-121.517401288965\\
60.5	0.1113	-130.244862268827	-130.244862268827\\
60.5	0.11496	-140.191941880973	-140.191941880973\\
60.5	0.11862	-151.358640125403	-151.358640125403\\
60.5	0.12228	-163.744957002117	-163.744957002117\\
60.5	0.12594	-177.350892511115	-177.350892511115\\
60.5	0.1296	-192.176446652398	-192.176446652398\\
60.5	0.13326	-208.221619425963	-208.221619425963\\
60.5	0.13692	-225.486410831813	-225.486410831813\\
60.5	0.14058	-243.970820869948	-243.970820869948\\
60.5	0.14424	-263.674849540366	-263.674849540366\\
60.5	0.1479	-284.598496843068	-284.598496843068\\
60.5	0.15156	-306.741762778055	-306.741762778055\\
60.5	0.15522	-330.104647345325	-330.104647345325\\
60.5	0.15888	-354.68715054488	-354.68715054488\\
60.5	0.16254	-380.489272376718	-380.489272376718\\
60.5	0.1662	-407.511012840841	-407.511012840841\\
60.5	0.16986	-435.752371937247	-435.752371937247\\
60.5	0.17352	-465.213349665937	-465.213349665937\\
60.5	0.17718	-495.893946026911	-495.893946026911\\
60.5	0.18084	-527.79416102017	-527.79416102017\\
60.5	0.1845	-560.913994645713	-560.913994645713\\
60.5	0.18816	-595.253446903539	-595.253446903539\\
60.5	0.19182	-630.81251779365	-630.81251779365\\
60.5	0.19548	-667.591207316045	-667.591207316045\\
60.5	0.19914	-705.589515470723	-705.589515470723\\
60.5	0.2028	-744.807442257686	-744.807442257686\\
60.5	0.20646	-785.244987676933	-785.244987676933\\
60.5	0.21012	-826.902151728464	-826.902151728464\\
60.5	0.21378	-869.778934412278	-869.778934412278\\
60.5	0.21744	-913.875335728377	-913.875335728377\\
60.5	0.2211	-959.19135567676	-959.19135567676\\
60.5	0.22476	-1005.72699425743	-1005.72699425743\\
60.5	0.22842	-1053.48225147038	-1053.48225147038\\
60.5	0.23208	-1102.45712731561	-1102.45712731561\\
60.5	0.23574	-1152.65162179313	-1152.65162179313\\
60.5	0.2394	-1204.06573490294	-1204.06573490294\\
60.5	0.24306	-1256.69946664502	-1256.69946664502\\
60.5	0.24672	-1310.55281701939	-1310.55281701939\\
60.5	0.25038	-1365.62578602605	-1365.62578602605\\
60.5	0.25404	-1421.91837366499	-1421.91837366499\\
60.5	0.2577	-1479.43057993621	-1479.43057993621\\
60.5	0.26136	-1538.16240483972	-1538.16240483972\\
60.5	0.26502	-1598.11384837551	-1598.11384837551\\
60.5	0.26868	-1659.28491054358	-1659.28491054358\\
60.5	0.27234	-1721.67559134394	-1721.67559134394\\
60.5	0.276	-1785.28589077659	-1785.28589077659\\
60.875	0.093	-98.2980325235959	-98.2980325235959\\
60.875	0.09666	-102.17652077854	-102.17652077854\\
60.875	0.10032	-107.274627665769	-107.274627665769\\
60.875	0.10398	-113.592353185281	-113.592353185281\\
60.875	0.10764	-121.129697337078	-121.129697337078\\
60.875	0.1113	-129.886660121159	-129.886660121159\\
60.875	0.11496	-139.863241537524	-139.863241537524\\
60.875	0.11862	-151.059441586172	-151.059441586172\\
60.875	0.12228	-163.475260267105	-163.475260267105\\
60.875	0.12594	-177.110697580322	-177.110697580322\\
60.875	0.1296	-191.965753525823	-191.965753525823\\
60.875	0.13326	-208.040428103608	-208.040428103608\\
60.875	0.13692	-225.334721313676	-225.334721313676\\
60.875	0.14058	-243.848633156029	-243.848633156029\\
60.875	0.14424	-263.582163630667	-263.582163630667\\
60.875	0.1479	-284.535312737587	-284.535312737587\\
60.875	0.15156	-306.708080476792	-306.708080476792\\
60.875	0.15522	-330.100466848281	-330.100466848281\\
60.875	0.15888	-354.712471852054	-354.712471852054\\
60.875	0.16254	-380.544095488111	-380.544095488111\\
60.875	0.1662	-407.595337756453	-407.595337756453\\
60.875	0.16986	-435.866198657078	-435.866198657078\\
60.875	0.17352	-465.356678189987	-465.356678189987\\
60.875	0.17718	-496.06677635518	-496.06677635518\\
60.875	0.18084	-527.996493152657	-527.996493152657\\
60.875	0.1845	-561.145828582419	-561.145828582419\\
60.875	0.18816	-595.514782644465	-595.514782644465\\
60.875	0.19182	-631.103355338794	-631.103355338794\\
60.875	0.19548	-667.911546665407	-667.911546665407\\
60.875	0.19914	-705.939356624305	-705.939356624305\\
60.875	0.2028	-745.186785215486	-745.186785215486\\
60.875	0.20646	-785.653832438952	-785.653832438952\\
60.875	0.21012	-827.340498294701	-827.340498294701\\
60.875	0.21378	-870.246782782734	-870.246782782734\\
60.875	0.21744	-914.372685903052	-914.372685903052\\
60.875	0.2211	-959.718207655653	-959.718207655653\\
60.875	0.22476	-1006.28334804054	-1006.28334804054\\
60.875	0.22842	-1054.06810705771	-1054.06810705771\\
60.875	0.23208	-1103.07248470716	-1103.07248470716\\
60.875	0.23574	-1153.2964809889	-1153.2964809889\\
60.875	0.2394	-1204.74009590292	-1204.74009590292\\
60.875	0.24306	-1257.40332944923	-1257.40332944923\\
60.875	0.24672	-1311.28618162782	-1311.28618162782\\
60.875	0.25038	-1366.38865243869	-1366.38865243869\\
60.875	0.25404	-1422.71074188185	-1422.71074188185\\
60.875	0.2577	-1480.25244995729	-1480.25244995729\\
60.875	0.26136	-1539.01377666502	-1539.01377666502\\
60.875	0.26502	-1598.99472200503	-1598.99472200503\\
60.875	0.26868	-1660.19528597732	-1660.19528597732\\
60.875	0.27234	-1722.6154685819	-1722.6154685819\\
60.875	0.276	-1786.25526981876	-1786.25526981876\\
61.25	0.093	-97.9309247195077	-97.9309247195077\\
61.25	0.09666	-101.838914778671	-101.838914778671\\
61.25	0.10032	-106.966523470118	-106.966523470118\\
61.25	0.10398	-113.313750793849	-113.313750793849\\
61.25	0.10764	-120.880596749865	-120.880596749865\\
61.25	0.1113	-129.667061338165	-129.667061338165\\
61.25	0.11496	-139.673144558748	-139.673144558748\\
61.25	0.11862	-150.898846411616	-150.898846411616\\
61.25	0.12228	-163.344166896766	-163.344166896766\\
61.25	0.12594	-177.009106014202	-177.009106014202\\
61.25	0.1296	-191.893663763922	-191.893663763922\\
61.25	0.13326	-207.997840145925	-207.997840145925\\
61.25	0.13692	-225.321635160213	-225.321635160213\\
61.25	0.14058	-243.865048806785	-243.865048806785\\
61.25	0.14424	-263.62808108564	-263.62808108564\\
61.25	0.1479	-284.61073199678	-284.61073199678\\
61.25	0.15156	-306.813001540203	-306.813001540203\\
61.25	0.15522	-330.234889715912	-330.234889715912\\
61.25	0.15888	-354.876396523903	-354.876396523903\\
61.25	0.16254	-380.737521964179	-380.737521964179\\
61.25	0.1662	-407.818266036739	-407.818266036739\\
61.25	0.16986	-436.118628741583	-436.118628741583\\
61.25	0.17352	-465.638610078711	-465.638610078711\\
61.25	0.17718	-496.378210048123	-496.378210048123\\
61.25	0.18084	-528.337428649818	-528.337428649818\\
61.25	0.1845	-561.516265883799	-561.516265883799\\
61.25	0.18816	-595.914721750063	-595.914721750063\\
61.25	0.19182	-631.53279624861	-631.53279624861\\
61.25	0.19548	-668.370489379443	-668.370489379443\\
61.25	0.19914	-706.427801142559	-706.427801142559\\
61.25	0.2028	-745.70473153796	-745.70473153796\\
61.25	0.20646	-786.201280565644	-786.201280565644\\
61.25	0.21012	-827.917448225612	-827.917448225612\\
61.25	0.21378	-870.853234517864	-870.853234517864\\
61.25	0.21744	-915.0086394424	-915.0086394424\\
61.25	0.2211	-960.38366299922	-960.38366299922\\
61.25	0.22476	-1006.97830518832	-1006.97830518832\\
61.25	0.22842	-1054.79256600971	-1054.79256600971\\
61.25	0.23208	-1103.82644546339	-1103.82644546339\\
61.25	0.23574	-1154.07994354934	-1154.07994354934\\
61.25	0.2394	-1205.55306026758	-1205.55306026758\\
61.25	0.24306	-1258.24579561811	-1258.24579561811\\
61.25	0.24672	-1312.15814960092	-1312.15814960092\\
61.25	0.25038	-1367.29012221601	-1367.29012221601\\
61.25	0.25404	-1423.64171346338	-1423.64171346338\\
61.25	0.2577	-1481.21292334305	-1481.21292334305\\
61.25	0.26136	-1540.00375185499	-1540.00375185499\\
61.25	0.26502	-1600.01419899922	-1600.01419899922\\
61.25	0.26868	-1661.24426477573	-1661.24426477573\\
61.25	0.27234	-1723.69394918453	-1723.69394918453\\
61.25	0.276	-1787.36325222561	-1787.36325222561\\
61.625	0.093	-97.7024202800927	-97.7024202800927\\
61.625	0.09666	-101.639912143475	-101.639912143475\\
61.625	0.10032	-106.79702263914	-106.79702263914\\
61.625	0.10398	-113.173751767091	-113.173751767091\\
61.625	0.10764	-120.770099527325	-120.770099527325\\
61.625	0.1113	-129.586065919843	-129.586065919843\\
61.625	0.11496	-139.621650944645	-139.621650944645\\
61.625	0.11862	-150.876854601731	-150.876854601731\\
61.625	0.12228	-163.351676891101	-163.351676891101\\
61.625	0.12594	-177.046117812756	-177.046117812756\\
61.625	0.1296	-191.960177366694	-191.960177366694\\
61.625	0.13326	-208.093855552916	-208.093855552916\\
61.625	0.13692	-225.447152371423	-225.447152371423\\
61.625	0.14058	-244.020067822213	-244.020067822213\\
61.625	0.14424	-263.812601905287	-263.812601905287\\
61.625	0.1479	-284.824754620646	-284.824754620646\\
61.625	0.15156	-307.056525968288	-307.056525968288\\
61.625	0.15522	-330.507915948215	-330.507915948215\\
61.625	0.15888	-355.178924560425	-355.178924560425\\
61.625	0.16254	-381.06955180492	-381.06955180492\\
61.625	0.1662	-408.179797681698	-408.179797681698\\
61.625	0.16986	-436.509662190761	-436.509662190761\\
61.625	0.17352	-466.059145332107	-466.059145332107\\
61.625	0.17718	-496.828247105738	-496.828247105738\\
61.625	0.18084	-528.816967511652	-528.816967511652\\
61.625	0.1845	-562.025306549852	-562.025306549852\\
61.625	0.18816	-596.453264220334	-596.453264220334\\
61.625	0.19182	-632.100840523101	-632.100840523101\\
61.625	0.19548	-668.968035458152	-668.968035458152\\
61.625	0.19914	-707.054849025487	-707.054849025487\\
61.625	0.2028	-746.361281225106	-746.361281225106\\
61.625	0.20646	-786.887332057009	-786.887332057009\\
61.625	0.21012	-828.633001521196	-828.633001521196\\
61.625	0.21378	-871.598289617666	-871.598289617666\\
61.625	0.21744	-915.783196346422	-915.783196346422\\
61.625	0.2211	-961.18772170746	-961.18772170746\\
61.625	0.22476	-1007.81186570078	-1007.81186570078\\
61.625	0.22842	-1055.65562832639	-1055.65562832639\\
61.625	0.23208	-1104.71900958428	-1104.71900958428\\
61.625	0.23574	-1155.00200947446	-1155.00200947446\\
61.625	0.2394	-1206.50462799692	-1206.50462799692\\
61.625	0.24306	-1259.22686515166	-1259.22686515166\\
61.625	0.24672	-1313.16872093869	-1313.16872093869\\
61.625	0.25038	-1368.330195358	-1368.330195358\\
61.625	0.25404	-1424.71128840959	-1424.71128840959\\
61.625	0.2577	-1482.31200009347	-1482.31200009347\\
61.625	0.26136	-1541.13233040964	-1541.13233040964\\
61.625	0.26502	-1601.17227935808	-1601.17227935808\\
61.625	0.26868	-1662.43184693882	-1662.43184693882\\
61.625	0.27234	-1724.91103315183	-1724.91103315183\\
61.625	0.276	-1788.60983799713	-1788.60983799713\\
62	0.093	-97.6125192053513	-97.6125192053513\\
62	0.09666	-101.579512872952	-101.579512872952\\
62	0.10032	-106.766125172837	-106.766125172837\\
62	0.10398	-113.172356105005	-113.172356105005\\
62	0.10764	-120.798205669458	-120.798205669458\\
62	0.1113	-129.643673866195	-129.643673866195\\
62	0.11496	-139.708760695216	-139.708760695216\\
62	0.11862	-150.99346615652	-150.99346615652\\
62	0.12228	-163.49779025011	-163.49779025011\\
62	0.12594	-177.221732975983	-177.221732975983\\
62	0.1296	-192.16529433414	-192.16529433414\\
62	0.13326	-208.328474324581	-208.328474324581\\
62	0.13692	-225.711272947306	-225.711272947306\\
62	0.14058	-244.313690202314	-244.313690202314\\
62	0.14424	-264.135726089608	-264.135726089608\\
62	0.1479	-285.177380609185	-285.177380609185\\
62	0.15156	-307.438653761046	-307.438653761046\\
62	0.15522	-330.919545545191	-330.919545545191\\
62	0.15888	-355.620055961621	-355.620055961621\\
62	0.16254	-381.540185010334	-381.540185010334\\
62	0.1662	-408.679932691331	-408.679932691331\\
62	0.16986	-437.039299004612	-437.039299004612\\
62	0.17352	-466.618283950178	-466.618283950178\\
62	0.17718	-497.416887528027	-497.416887528027\\
62	0.18084	-529.435109738161	-529.435109738161\\
62	0.1845	-562.672950580578	-562.672950580578\\
62	0.18816	-597.13041005528	-597.13041005528\\
62	0.19182	-632.807488162265	-632.807488162265\\
62	0.19548	-669.704184901534	-669.704184901534\\
62	0.19914	-707.820500273088	-707.820500273088\\
62	0.2028	-747.156434276926	-747.156434276926\\
62	0.20646	-787.711986913048	-787.711986913048\\
62	0.21012	-829.487158181453	-829.487158181453\\
62	0.21378	-872.481948082142	-872.481948082142\\
62	0.21744	-916.696356615116	-916.696356615116\\
62	0.2211	-962.130383780374	-962.130383780374\\
62	0.22476	-1008.78402957792	-1008.78402957792\\
62	0.22842	-1056.65729400774	-1056.65729400774\\
62	0.23208	-1105.75017706985	-1105.75017706985\\
62	0.23574	-1156.06267876425	-1156.06267876425\\
62	0.2394	-1207.59479909092	-1207.59479909092\\
62	0.24306	-1260.34653804989	-1260.34653804989\\
62	0.24672	-1314.31789564113	-1314.31789564113\\
62	0.25038	-1369.50887186466	-1369.50887186466\\
62	0.25404	-1425.91946672048	-1425.91946672048\\
62	0.2577	-1483.54968020857	-1483.54968020857\\
62	0.26136	-1542.39951232896	-1542.39951232896\\
62	0.26502	-1602.46896308162	-1602.46896308162\\
62	0.26868	-1663.75803246657	-1663.75803246657\\
62	0.27234	-1726.26672048381	-1726.26672048381\\
62	0.276	-1789.99502713332	-1789.99502713332\\
62.375	0.093	-97.6612214952828	-97.6612214952828\\
62.375	0.09666	-101.657716967102	-101.657716967102\\
62.375	0.10032	-106.873831071206	-106.873831071206\\
62.375	0.10398	-113.309563807593	-113.309563807593\\
62.375	0.10764	-120.964915176265	-120.964915176265\\
62.375	0.1113	-129.83988517722	-129.83988517722\\
62.375	0.11496	-139.93447381046	-139.93447381046\\
62.375	0.11862	-151.248681075983	-151.248681075983\\
62.375	0.12228	-163.782506973791	-163.782506973791\\
62.375	0.12594	-177.535951503882	-177.535951503882\\
62.375	0.1296	-192.509014666259	-192.509014666259\\
62.375	0.13326	-208.701696460918	-208.701696460918\\
62.375	0.13692	-226.113996887862	-226.113996887862\\
62.375	0.14058	-244.745915947089	-244.745915947089\\
62.375	0.14424	-264.597453638602	-264.597453638602\\
62.375	0.1479	-285.668609962397	-285.668609962397\\
62.375	0.15156	-307.959384918477	-307.959384918477\\
62.375	0.15522	-331.469778506841	-331.469778506841\\
62.375	0.15888	-356.199790727489	-356.199790727489\\
62.375	0.16254	-382.149421580421	-382.149421580421\\
62.375	0.1662	-409.318671065637	-409.318671065637\\
62.375	0.16986	-437.707539183137	-437.707539183137\\
62.375	0.17352	-467.316025932921	-467.316025932921\\
62.375	0.17718	-498.144131314989	-498.144131314989\\
62.375	0.18084	-530.191855329341	-530.191855329341\\
62.375	0.1845	-563.459197975977	-563.459197975977\\
62.375	0.18816	-597.946159254898	-597.946159254898\\
62.375	0.19182	-633.652739166102	-633.652739166102\\
62.375	0.19548	-670.57893770959	-670.57893770959\\
62.375	0.19914	-708.724754885362	-708.724754885362\\
62.375	0.2028	-748.090190693419	-748.090190693419\\
62.375	0.20646	-788.67524513376	-788.67524513376\\
62.375	0.21012	-830.479918206383	-830.479918206383\\
62.375	0.21378	-873.504209911292	-873.504209911292\\
62.375	0.21744	-917.748120248484	-917.748120248484\\
62.375	0.2211	-963.211649217961	-963.211649217961\\
62.375	0.22476	-1009.89479681972	-1009.89479681972\\
62.375	0.22842	-1057.79756305377	-1057.79756305377\\
62.375	0.23208	-1106.91994792009	-1106.91994792009\\
62.375	0.23574	-1157.26195141871	-1157.26195141871\\
62.375	0.2394	-1208.8235735496	-1208.8235735496\\
62.375	0.24306	-1261.60481431278	-1261.60481431278\\
62.375	0.24672	-1315.60567370825	-1315.60567370825\\
62.375	0.25038	-1370.826151736	-1370.826151736\\
62.375	0.25404	-1427.26624839603	-1427.26624839603\\
62.375	0.2577	-1484.92596368835	-1484.92596368835\\
62.375	0.26136	-1543.80529761295	-1543.80529761295\\
62.375	0.26502	-1603.90425016983	-1603.90425016983\\
62.375	0.26868	-1665.222821359	-1665.222821359\\
62.375	0.27234	-1727.76101118045	-1727.76101118045\\
62.375	0.276	-1791.51881963419	-1791.51881963419\\
62.75	0.093	-97.8485271498881	-97.8485271498881\\
62.75	0.09666	-101.874524425926	-101.874524425926\\
62.75	0.10032	-107.120140334248	-107.120140334248\\
62.75	0.10398	-113.585374874855	-113.585374874855\\
62.75	0.10764	-121.270228047745	-121.270228047745\\
62.75	0.1113	-130.174699852919	-130.174699852919\\
62.75	0.11496	-140.298790290377	-140.298790290377\\
62.75	0.11862	-151.64249936012	-151.64249936012\\
62.75	0.12228	-164.205827062146	-164.205827062146\\
62.75	0.12594	-177.988773396457	-177.988773396457\\
62.75	0.1296	-192.991338363051	-192.991338363051\\
62.75	0.13326	-209.213521961929	-209.213521961929\\
62.75	0.13692	-226.655324193092	-226.655324193092\\
62.75	0.14058	-245.316745056538	-245.316745056538\\
62.75	0.14424	-265.197784552269	-265.197784552269\\
62.75	0.1479	-286.298442680284	-286.298442680284\\
62.75	0.15156	-308.618719440582	-308.618719440582\\
62.75	0.15522	-332.158614833165	-332.158614833165\\
62.75	0.15888	-356.918128858031	-356.918128858031\\
62.75	0.16254	-382.897261515182	-382.897261515182\\
62.75	0.1662	-410.096012804617	-410.096012804617\\
62.75	0.16986	-438.514382726336	-438.514382726336\\
62.75	0.17352	-468.152371280338	-468.152371280338\\
62.75	0.17718	-499.009978466625	-499.009978466625\\
62.75	0.18084	-531.087204285195	-531.087204285195\\
62.75	0.1845	-564.384048736051	-564.384048736051\\
62.75	0.18816	-598.900511819189	-598.900511819189\\
62.75	0.19182	-634.636593534612	-634.636593534612\\
62.75	0.19548	-671.59229388232	-671.59229388232\\
62.75	0.19914	-709.767612862311	-709.767612862311\\
62.75	0.2028	-749.162550474586	-749.162550474586\\
62.75	0.20646	-789.777106719145	-789.777106719145\\
62.75	0.21012	-831.611281595988	-831.611281595988\\
62.75	0.21378	-874.665075105115	-874.665075105115\\
62.75	0.21744	-918.938487246526	-918.938487246526\\
62.75	0.2211	-964.431518020221	-964.431518020221\\
62.75	0.22476	-1011.1441674262	-1011.1441674262\\
62.75	0.22842	-1059.07643546446	-1059.07643546446\\
62.75	0.23208	-1108.22832213501	-1108.22832213501\\
62.75	0.23574	-1158.59982743784	-1158.59982743784\\
62.75	0.2394	-1210.19095137296	-1210.19095137296\\
62.75	0.24306	-1263.00169394036	-1263.00169394036\\
62.75	0.24672	-1317.03205514004	-1317.03205514004\\
62.75	0.25038	-1372.28203497201	-1372.28203497201\\
62.75	0.25404	-1428.75163343626	-1428.75163343626\\
62.75	0.2577	-1486.4408505328	-1486.4408505328\\
62.75	0.26136	-1545.34968626161	-1545.34968626161\\
62.75	0.26502	-1605.47814062272	-1605.47814062272\\
62.75	0.26868	-1666.82621361611	-1666.82621361611\\
62.75	0.27234	-1729.39390524178	-1729.39390524178\\
62.75	0.276	-1793.18121549973	-1793.18121549973\\
63.125	0.093	-98.1744361691664	-98.1744361691664\\
63.125	0.09666	-102.229935249424	-102.229935249424\\
63.125	0.10032	-107.505052961964	-107.505052961964\\
63.125	0.10398	-113.999789306789	-113.999789306789\\
63.125	0.10764	-121.714144283898	-121.714144283898\\
63.125	0.1113	-130.648117893291	-130.648117893291\\
63.125	0.11496	-140.801710134968	-140.801710134968\\
63.125	0.11862	-152.174921008929	-152.174921008929\\
63.125	0.12228	-164.767750515174	-164.767750515174\\
63.125	0.12594	-178.580198653704	-178.580198653704\\
63.125	0.1296	-193.612265424516	-193.612265424516\\
63.125	0.13326	-209.863950827613	-209.863950827613\\
63.125	0.13692	-227.335254862994	-227.335254862994\\
63.125	0.14058	-246.02617753066	-246.02617753066\\
63.125	0.14424	-265.936718830609	-265.936718830609\\
63.125	0.1479	-287.066878762843	-287.066878762843\\
63.125	0.15156	-309.416657327359	-309.416657327359\\
63.125	0.15522	-332.986054524161	-332.986054524161\\
63.125	0.15888	-357.775070353247	-357.775070353247\\
63.125	0.16254	-383.783704814616	-383.783704814616\\
63.125	0.1662	-411.011957908269	-411.011957908269\\
63.125	0.16986	-439.459829634207	-439.459829634207\\
63.125	0.17352	-469.127319992428	-469.127319992428\\
63.125	0.17718	-500.014428982934	-500.014428982934\\
63.125	0.18084	-532.121156605723	-532.121156605723\\
63.125	0.1845	-565.447502860797	-565.447502860797\\
63.125	0.18816	-599.993467748155	-599.993467748155\\
63.125	0.19182	-635.759051267796	-635.759051267796\\
63.125	0.19548	-672.744253419722	-672.744253419722\\
63.125	0.19914	-710.949074203932	-710.949074203932\\
63.125	0.2028	-750.373513620426	-750.373513620426\\
63.125	0.20646	-791.017571669204	-791.017571669204\\
63.125	0.21012	-832.881248350265	-832.881248350265\\
63.125	0.21378	-875.96454366361	-875.96454366361\\
63.125	0.21744	-920.267457609241	-920.267457609241\\
63.125	0.2211	-965.789990187154	-965.789990187154\\
63.125	0.22476	-1012.53214139735	-1012.53214139735\\
63.125	0.22842	-1060.49391123983	-1060.49391123983\\
63.125	0.23208	-1109.6752997146	-1109.6752997146\\
63.125	0.23574	-1160.07630682165	-1160.07630682165\\
63.125	0.2394	-1211.69693256098	-1211.69693256098\\
63.125	0.24306	-1264.5371769326	-1264.5371769326\\
63.125	0.24672	-1318.5970399365	-1318.5970399365\\
63.125	0.25038	-1373.87652157269	-1373.87652157269\\
63.125	0.25404	-1430.37562184116	-1430.37562184116\\
63.125	0.2577	-1488.09434074192	-1488.09434074192\\
63.125	0.26136	-1547.03267827495	-1547.03267827495\\
63.125	0.26502	-1607.19063444028	-1607.19063444028\\
63.125	0.26868	-1668.56820923788	-1668.56820923788\\
63.125	0.27234	-1731.16540266777	-1731.16540266777\\
63.125	0.276	-1794.98221472995	-1794.98221472995\\
63.5	0.093	-98.6389485531184	-98.6389485531184\\
63.5	0.09666	-102.723949437594	-102.723949437594\\
63.5	0.10032	-108.028568954353	-108.028568954353\\
63.5	0.10398	-114.552807103397	-114.552807103397\\
63.5	0.10764	-122.296663884724	-122.296663884724\\
63.5	0.1113	-131.260139298336	-131.260139298336\\
63.5	0.11496	-141.443233344232	-141.443233344232\\
63.5	0.11862	-152.845946022412	-152.845946022412\\
63.5	0.12228	-165.468277332875	-165.468277332875\\
63.5	0.12594	-179.310227275623	-179.310227275623\\
63.5	0.1296	-194.371795850655	-194.371795850655\\
63.5	0.13326	-210.652983057971	-210.652983057971\\
63.5	0.13692	-228.153788897571	-228.153788897571\\
63.5	0.14058	-246.874213369455	-246.874213369455\\
63.5	0.14424	-266.814256473623	-266.814256473623\\
63.5	0.1479	-287.973918210075	-287.973918210075\\
63.5	0.15156	-310.353198578811	-310.353198578811\\
63.5	0.15522	-333.952097579831	-333.952097579831\\
63.5	0.15888	-358.770615213135	-358.770615213135\\
63.5	0.16254	-384.808751478723	-384.808751478723\\
63.5	0.1662	-412.066506376595	-412.066506376595\\
63.5	0.16986	-440.543879906752	-440.543879906752\\
63.5	0.17352	-470.240872069191	-470.240872069191\\
63.5	0.17718	-501.157482863916	-501.157482863916\\
63.5	0.18084	-533.293712290924	-533.293712290924\\
63.5	0.1845	-566.649560350217	-566.649560350217\\
63.5	0.18816	-601.225027041793	-601.225027041793\\
63.5	0.19182	-637.020112365653	-637.020112365653\\
63.5	0.19548	-674.034816321798	-674.034816321798\\
63.5	0.19914	-712.269138910227	-712.269138910227\\
63.5	0.2028	-751.723080130939	-751.723080130939\\
63.5	0.20646	-792.396639983936	-792.396639983936\\
63.5	0.21012	-834.289818469216	-834.289818469216\\
63.5	0.21378	-877.40261558678	-877.40261558678\\
63.5	0.21744	-921.735031336628	-921.735031336628\\
63.5	0.2211	-967.287065718761	-967.287065718761\\
63.5	0.22476	-1014.05871873318	-1014.05871873318\\
63.5	0.22842	-1062.04999037988	-1062.04999037988\\
63.5	0.23208	-1111.26088065886	-1111.26088065886\\
63.5	0.23574	-1161.69138957013	-1161.69138957013\\
63.5	0.2394	-1213.34151711368	-1213.34151711368\\
63.5	0.24306	-1266.21126328952	-1266.21126328952\\
63.5	0.24672	-1320.30062809764	-1320.30062809764\\
63.5	0.25038	-1375.60961153805	-1375.60961153805\\
63.5	0.25404	-1432.13821361074	-1432.13821361074\\
63.5	0.2577	-1489.88643431571	-1489.88643431571\\
63.5	0.26136	-1548.85427365297	-1548.85427365297\\
63.5	0.26502	-1609.04173162251	-1609.04173162251\\
63.5	0.26868	-1670.44880822433	-1670.44880822433\\
63.5	0.27234	-1733.07550345844	-1733.07550345844\\
63.5	0.276	-1796.92181732483	-1796.92181732483\\
63.875	0.093	-99.2420643017439	-99.2420643017439\\
63.875	0.09666	-103.356566990438	-103.356566990438\\
63.875	0.10032	-108.690688311416	-108.690688311416\\
63.875	0.10398	-115.244428264678	-115.244428264678\\
63.875	0.10764	-123.017786850225	-123.017786850225\\
63.875	0.1113	-132.010764068055	-132.010764068055\\
63.875	0.11496	-142.22335991817	-142.22335991817\\
63.875	0.11862	-153.655574400568	-153.655574400568\\
63.875	0.12228	-166.307407515251	-166.307407515251\\
63.875	0.12594	-180.178859262217	-180.178859262217\\
63.875	0.1296	-195.269929641468	-195.269929641468\\
63.875	0.13326	-211.580618653002	-211.580618653002\\
63.875	0.13692	-229.110926296821	-229.110926296821\\
63.875	0.14058	-247.860852572923	-247.860852572923\\
63.875	0.14424	-267.830397481311	-267.830397481311\\
63.875	0.1479	-289.019561021981	-289.019561021981\\
63.875	0.15156	-311.428343194936	-311.428343194936\\
63.875	0.15522	-335.056744000174	-335.056744000174\\
63.875	0.15888	-359.904763437697	-359.904763437697\\
63.875	0.16254	-385.972401507504	-385.972401507504\\
63.875	0.1662	-413.259658209595	-413.259658209595\\
63.875	0.16986	-441.76653354397	-441.76653354397\\
63.875	0.17352	-471.493027510629	-471.493027510629\\
63.875	0.17718	-502.439140109571	-502.439140109571\\
63.875	0.18084	-534.604871340799	-534.604871340799\\
63.875	0.1845	-567.990221204309	-567.990221204309\\
63.875	0.18816	-602.595189700105	-602.595189700105\\
63.875	0.19182	-638.419776828184	-638.419776828184\\
63.875	0.19548	-675.463982588547	-675.463982588547\\
63.875	0.19914	-713.727806981194	-713.727806981194\\
63.875	0.2028	-753.211250006126	-753.211250006126\\
63.875	0.20646	-793.914311663341	-793.914311663341\\
63.875	0.21012	-835.83699195284	-835.83699195284\\
63.875	0.21378	-878.979290874623	-878.979290874623\\
63.875	0.21744	-923.34120842869	-923.34120842869\\
63.875	0.2211	-968.922744615042	-968.922744615042\\
63.875	0.22476	-1015.72389943368	-1015.72389943368\\
63.875	0.22842	-1063.7446728846	-1063.7446728846\\
63.875	0.23208	-1112.9850649678	-1112.9850649678\\
63.875	0.23574	-1163.44507568329	-1163.44507568329\\
63.875	0.2394	-1215.12470503106	-1215.12470503106\\
63.875	0.24306	-1268.02395301111	-1268.02395301111\\
63.875	0.24672	-1322.14281962345	-1322.14281962345\\
63.875	0.25038	-1377.48130486808	-1377.48130486808\\
63.875	0.25404	-1434.03940874499	-1434.03940874499\\
63.875	0.2577	-1491.81713125418	-1491.81713125418\\
63.875	0.26136	-1550.81447239565	-1550.81447239565\\
63.875	0.26502	-1611.03143216941	-1611.03143216941\\
63.875	0.26868	-1672.46801057546	-1672.46801057546\\
63.875	0.27234	-1735.12420761378	-1735.12420761378\\
63.875	0.276	-1799.00002328439	-1799.00002328439\\
64.25	0.093	-99.9837834150422	-99.9837834150422\\
64.25	0.09666	-104.127787907955	-104.127787907955\\
64.25	0.10032	-109.491411033152	-109.491411033152\\
64.25	0.10398	-116.074652790633	-116.074652790633\\
64.25	0.10764	-123.877513180398	-123.877513180398\\
64.25	0.1113	-132.899992202447	-132.899992202447\\
64.25	0.11496	-143.142089856781	-143.142089856781\\
64.25	0.11862	-154.603806143397	-154.603806143397\\
64.25	0.12228	-167.285141062299	-167.285141062299\\
64.25	0.12594	-181.186094613484	-181.186094613484\\
64.25	0.1296	-196.306666796953	-196.306666796953\\
64.25	0.13326	-212.646857612706	-212.646857612706\\
64.25	0.13692	-230.206667060744	-230.206667060744\\
64.25	0.14058	-248.986095141065	-248.986095141065\\
64.25	0.14424	-268.985141853671	-268.985141853671\\
64.25	0.1479	-290.20380719856	-290.20380719856\\
64.25	0.15156	-312.642091175734	-312.642091175734\\
64.25	0.15522	-336.299993785191	-336.299993785191\\
64.25	0.15888	-361.177515026933	-361.177515026933\\
64.25	0.16254	-387.274654900958	-387.274654900958\\
64.25	0.1662	-414.591413407268	-414.591413407268\\
64.25	0.16986	-443.127790545861	-443.127790545861\\
64.25	0.17352	-472.883786316739	-472.883786316739\\
64.25	0.17718	-503.8594007199	-503.8594007199\\
64.25	0.18084	-536.054633755346	-536.054633755346\\
64.25	0.1845	-569.469485423076	-569.469485423076\\
64.25	0.18816	-604.10395572309	-604.10395572309\\
64.25	0.19182	-639.958044655388	-639.958044655388\\
64.25	0.19548	-677.031752219969	-677.031752219969\\
64.25	0.19914	-715.325078416835	-715.325078416835\\
64.25	0.2028	-754.838023245986	-754.838023245986\\
64.25	0.20646	-795.57058670742	-795.57058670742\\
64.25	0.21012	-837.522768801137	-837.522768801137\\
64.25	0.21378	-880.694569527139	-880.694569527139\\
64.25	0.21744	-925.085988885425	-925.085988885425\\
64.25	0.2211	-970.697026875995	-970.697026875995\\
64.25	0.22476	-1017.52768349885	-1017.52768349885\\
64.25	0.22842	-1065.57795875399	-1065.57795875399\\
64.25	0.23208	-1114.84785264141	-1114.84785264141\\
64.25	0.23574	-1165.33736516112	-1165.33736516112\\
64.25	0.2394	-1217.04649631311	-1217.04649631311\\
64.25	0.24306	-1269.97524609738	-1269.97524609738\\
64.25	0.24672	-1324.12361451394	-1324.12361451394\\
64.25	0.25038	-1379.49160156278	-1379.49160156278\\
64.25	0.25404	-1436.07920724391	-1436.07920724391\\
64.25	0.2577	-1493.88643155732	-1493.88643155732\\
64.25	0.26136	-1552.91327450301	-1552.91327450301\\
64.25	0.26502	-1613.15973608099	-1613.15973608099\\
64.25	0.26868	-1674.62581629125	-1674.62581629125\\
64.25	0.27234	-1737.3115151338	-1737.3115151338\\
64.25	0.276	-1801.21683260863	-1801.21683260863\\
64.625	0.093	-100.864105893014	-100.864105893014\\
64.625	0.09666	-105.037612190146	-105.037612190146\\
64.625	0.10032	-110.430737119561	-110.430737119561\\
64.625	0.10398	-117.043480681261	-117.043480681261\\
64.625	0.10764	-124.875842875245	-124.875842875245\\
64.625	0.1113	-133.927823701513	-133.927823701513\\
64.625	0.11496	-144.199423160064	-144.199423160064\\
64.625	0.11862	-155.690641250901	-155.690641250901\\
64.625	0.12228	-168.40147797402	-168.40147797402\\
64.625	0.12594	-182.331933329425	-182.331933329425\\
64.625	0.1296	-197.482007317112	-197.482007317112\\
64.625	0.13326	-213.851699937084	-213.851699937084\\
64.625	0.13692	-231.44101118934	-231.44101118934\\
64.625	0.14058	-250.249941073881	-250.249941073881\\
64.625	0.14424	-270.278489590704	-270.278489590704\\
64.625	0.1479	-291.526656739813	-291.526656739813\\
64.625	0.15156	-313.994442521205	-313.994442521205\\
64.625	0.15522	-337.681846934881	-337.681846934881\\
64.625	0.15888	-362.588869980841	-362.588869980841\\
64.625	0.16254	-388.715511659085	-388.715511659085\\
64.625	0.1662	-416.061771969614	-416.061771969614\\
64.625	0.16986	-444.627650912427	-444.627650912427\\
64.625	0.17352	-474.413148487522	-474.413148487522\\
64.625	0.17718	-505.418264694903	-505.418264694903\\
64.625	0.18084	-537.642999534567	-537.642999534567\\
64.625	0.1845	-571.087353006516	-571.087353006516\\
64.625	0.18816	-605.751325110749	-605.751325110749\\
64.625	0.19182	-641.634915847264	-641.634915847264\\
64.625	0.19548	-678.738125216066	-678.738125216066\\
64.625	0.19914	-717.060953217151	-717.060953217151\\
64.625	0.2028	-756.603399850519	-756.603399850519\\
64.625	0.20646	-797.365465116172	-797.365465116172\\
64.625	0.21012	-839.347149014108	-839.347149014108\\
64.625	0.21378	-882.548451544329	-882.548451544329\\
64.625	0.21744	-926.969372706833	-926.969372706833\\
64.625	0.2211	-972.609912501622	-972.609912501622\\
64.625	0.22476	-1019.47007092869	-1019.47007092869\\
64.625	0.22842	-1067.54984798805	-1067.54984798805\\
64.625	0.23208	-1116.84924367969	-1116.84924367969\\
64.625	0.23574	-1167.36825800362	-1167.36825800362\\
64.625	0.2394	-1219.10689095983	-1219.10689095983\\
64.625	0.24306	-1272.06514254832	-1272.06514254832\\
64.625	0.24672	-1326.2430127691	-1326.2430127691\\
64.625	0.25038	-1381.64050162216	-1381.64050162216\\
64.625	0.25404	-1438.2576091075	-1438.2576091075\\
64.625	0.2577	-1496.09433522513	-1496.09433522513\\
64.625	0.26136	-1555.15067997505	-1555.15067997505\\
64.625	0.26502	-1615.42664335724	-1615.42664335724\\
64.625	0.26868	-1676.92222537172	-1676.92222537172\\
64.625	0.27234	-1739.63742601849	-1739.63742601849\\
64.625	0.276	-1803.57224529754	-1803.57224529754\\
65	0.093	-101.883031735659	-101.883031735659\\
65	0.09666	-106.086039837009	-106.086039837009\\
65	0.10032	-111.508666570643	-111.508666570643\\
65	0.10398	-118.150911936562	-118.150911936562\\
65	0.10764	-126.012775934765	-126.012775934765\\
65	0.1113	-135.094258565251	-135.094258565251\\
65	0.11496	-145.395359828022	-145.395359828022\\
65	0.11862	-156.916079723077	-156.916079723077\\
65	0.12228	-169.656418250415	-169.656418250415\\
65	0.12594	-183.616375410038	-183.616375410038\\
65	0.1296	-198.795951201944	-198.795951201944\\
65	0.13326	-215.195145626135	-215.195145626135\\
65	0.13692	-232.81395868261	-232.81395868261\\
65	0.14058	-251.652390371369	-251.652390371369\\
65	0.14424	-271.710440692411	-271.710440692411\\
65	0.1479	-292.988109645739	-292.988109645739\\
65	0.15156	-315.485397231349	-315.485397231349\\
65	0.15522	-339.202303449244	-339.202303449244\\
65	0.15888	-364.138828299423	-364.138828299423\\
65	0.16254	-390.294971781886	-390.294971781886\\
65	0.1662	-417.670733896633	-417.670733896633\\
65	0.16986	-446.266114643665	-446.266114643665\\
65	0.17352	-476.081114022979	-476.081114022979\\
65	0.17718	-507.115732034578	-507.115732034578\\
65	0.18084	-539.369968678461	-539.369968678461\\
65	0.1845	-572.843823954629	-572.843823954629\\
65	0.18816	-607.53729786308	-607.53729786308\\
65	0.19182	-643.450390403815	-643.450390403815\\
65	0.19548	-680.583101576834	-680.583101576834\\
65	0.19914	-718.935431382138	-718.935431382138\\
65	0.2028	-758.507379819725	-758.507379819725\\
65	0.20646	-799.298946889597	-799.298946889597\\
65	0.21012	-841.310132591752	-841.310132591752\\
65	0.21378	-884.540936926191	-884.540936926191\\
65	0.21744	-928.991359892915	-928.991359892915\\
65	0.2211	-974.661401491922	-974.661401491922\\
65	0.22476	-1021.55106172321	-1021.55106172321\\
65	0.22842	-1069.66034058679	-1069.66034058679\\
65	0.23208	-1118.98923808265	-1118.98923808265\\
65	0.23574	-1169.53775421079	-1169.53775421079\\
65	0.2394	-1221.30588897122	-1221.30588897122\\
65	0.24306	-1274.29364236393	-1274.29364236393\\
65	0.24672	-1328.50101438893	-1328.50101438893\\
65	0.25038	-1383.92800504621	-1383.92800504621\\
65	0.25404	-1440.57461433577	-1440.57461433577\\
65	0.2577	-1498.44084225762	-1498.44084225762\\
65	0.26136	-1557.52668881175	-1557.52668881175\\
65	0.26502	-1617.83215399817	-1617.83215399817\\
65	0.26868	-1679.35723781687	-1679.35723781687\\
65	0.27234	-1742.10194026785	-1742.10194026785\\
65	0.276	-1806.06626135112	-1806.06626135112\\
65.375	0.093	-103.040560942978	-103.040560942978\\
65.375	0.09666	-107.273070848546	-107.273070848546\\
65.375	0.10032	-112.7251993864	-112.7251993864\\
65.375	0.10398	-119.396946556537	-119.396946556537\\
65.375	0.10764	-127.288312358958	-127.288312358958\\
65.375	0.1113	-136.399296793664	-136.399296793664\\
65.375	0.11496	-146.729899860652	-146.729899860652\\
65.375	0.11862	-158.280121559926	-158.280121559926\\
65.375	0.12228	-171.049961891483	-171.049961891483\\
65.375	0.12594	-185.039420855325	-185.039420855325\\
65.375	0.1296	-200.24849845145	-200.24849845145\\
65.375	0.13326	-216.677194679859	-216.677194679859\\
65.375	0.13692	-234.325509540553	-234.325509540553\\
65.375	0.14058	-253.193443033531	-253.193443033531\\
65.375	0.14424	-273.280995158792	-273.280995158792\\
65.375	0.1479	-294.588165916338	-294.588165916338\\
65.375	0.15156	-317.114955306167	-317.114955306167\\
65.375	0.15522	-340.861363328281	-340.861363328281\\
65.375	0.15888	-365.827389982679	-365.827389982679\\
65.375	0.16254	-392.01303526936	-392.01303526936\\
65.375	0.1662	-419.418299188326	-419.418299188326\\
65.375	0.16986	-448.043181739576	-448.043181739576\\
65.375	0.17352	-477.887682923109	-477.887682923109\\
65.375	0.17718	-508.951802738928	-508.951802738928\\
65.375	0.18084	-541.235541187029	-541.235541187029\\
65.375	0.1845	-574.738898267415	-574.738898267415\\
65.375	0.18816	-609.461873980085	-609.461873980085\\
65.375	0.19182	-645.404468325039	-645.404468325039\\
65.375	0.19548	-682.566681302277	-682.566681302277\\
65.375	0.19914	-720.948512911799	-720.948512911799\\
65.375	0.2028	-760.549963153605	-760.549963153605\\
65.375	0.20646	-801.371032027695	-801.371032027695\\
65.375	0.21012	-843.41171953407	-843.41171953407\\
65.375	0.21378	-886.672025672727	-886.672025672727\\
65.375	0.21744	-931.151950443669	-931.151950443669\\
65.375	0.2211	-976.851493846896	-976.851493846896\\
65.375	0.22476	-1023.77065588241	-1023.77065588241\\
65.375	0.22842	-1071.9094365502	-1071.9094365502\\
65.375	0.23208	-1121.26783585028	-1121.26783585028\\
65.375	0.23574	-1171.84585378264	-1171.84585378264\\
65.375	0.2394	-1223.64349034729	-1223.64349034729\\
65.375	0.24306	-1276.66074554422	-1276.66074554422\\
65.375	0.24672	-1330.89761937343	-1330.89761937343\\
65.375	0.25038	-1386.35411183493	-1386.35411183493\\
65.375	0.25404	-1443.03022292871	-1443.03022292871\\
65.375	0.2577	-1500.92595265478	-1500.92595265478\\
65.375	0.26136	-1560.04130101313	-1560.04130101313\\
65.375	0.26502	-1620.37626800376	-1620.37626800376\\
65.375	0.26868	-1681.93085362668	-1681.93085362668\\
65.375	0.27234	-1744.70505788189	-1744.70505788189\\
65.375	0.276	-1808.69888076937	-1808.69888076937\\
65.75	0.093	-104.33669351497	-104.33669351497\\
65.75	0.09666	-108.598705224758	-108.598705224758\\
65.75	0.10032	-114.080335566829	-114.080335566829\\
65.75	0.10398	-120.781584541186	-120.781584541186\\
65.75	0.10764	-128.702452147825	-128.702452147825\\
65.75	0.1113	-137.842938386749	-137.842938386749\\
65.75	0.11496	-148.203043257957	-148.203043257957\\
65.75	0.11862	-159.78276676145	-159.78276676145\\
65.75	0.12228	-172.582108897225	-172.582108897225\\
65.75	0.12594	-186.601069665286	-186.601069665286\\
65.75	0.1296	-201.83964906563	-201.83964906563\\
65.75	0.13326	-218.297847098258	-218.297847098258\\
65.75	0.13692	-235.97566376317	-235.97566376317\\
65.75	0.14058	-254.873099060367	-254.873099060367\\
65.75	0.14424	-274.990152989847	-274.990152989847\\
65.75	0.1479	-296.326825551611	-296.326825551611\\
65.75	0.15156	-318.883116745659	-318.883116745659\\
65.75	0.15522	-342.659026571992	-342.659026571992\\
65.75	0.15888	-367.654555030608	-367.654555030608\\
65.75	0.16254	-393.869702121508	-393.869702121508\\
65.75	0.1662	-421.304467844693	-421.304467844693\\
65.75	0.16986	-449.958852200162	-449.958852200162\\
65.75	0.17352	-479.832855187914	-479.832855187914\\
65.75	0.17718	-510.926476807951	-510.926476807951\\
65.75	0.18084	-543.239717060271	-543.239717060271\\
65.75	0.1845	-576.772575944876	-576.772575944876\\
65.75	0.18816	-611.525053461764	-611.525053461764\\
65.75	0.19182	-647.497149610937	-647.497149610937\\
65.75	0.19548	-684.688864392393	-684.688864392393\\
65.75	0.19914	-723.100197806135	-723.100197806135\\
65.75	0.2028	-762.731149852159	-762.731149852159\\
65.75	0.20646	-803.581720530468	-803.581720530468\\
65.75	0.21012	-845.651909841061	-845.651909841061\\
65.75	0.21378	-888.941717783937	-888.941717783937\\
65.75	0.21744	-933.451144359098	-933.451144359098\\
65.75	0.2211	-979.180189566543	-979.180189566543\\
65.75	0.22476	-1026.12885340627	-1026.12885340627\\
65.75	0.22842	-1074.29713587829	-1074.29713587829\\
65.75	0.23208	-1123.68503698258	-1123.68503698258\\
65.75	0.23574	-1174.29255671916	-1174.29255671916\\
65.75	0.2394	-1226.11969508803	-1226.11969508803\\
65.75	0.24306	-1279.16645208918	-1279.16645208918\\
65.75	0.24672	-1333.43282772261	-1333.43282772261\\
65.75	0.25038	-1388.91882198833	-1388.91882198833\\
65.75	0.25404	-1445.62443488633	-1445.62443488633\\
65.75	0.2577	-1503.54966641661	-1503.54966641661\\
65.75	0.26136	-1562.69451657918	-1562.69451657918\\
65.75	0.26502	-1623.05898537404	-1623.05898537404\\
65.75	0.26868	-1684.64307280117	-1684.64307280117\\
65.75	0.27234	-1747.4467788606	-1747.4467788606\\
65.75	0.276	-1811.4701035523	-1811.4701035523\\
66.125	0.093	-105.771429451635	-105.771429451635\\
66.125	0.09666	-110.062942965641	-110.062942965641\\
66.125	0.10032	-115.574075111931	-115.574075111931\\
66.125	0.10398	-122.304825890506	-122.304825890506\\
66.125	0.10764	-130.255195301365	-130.255195301365\\
66.125	0.1113	-139.425183344508	-139.425183344508\\
66.125	0.11496	-149.814790019934	-149.814790019934\\
66.125	0.11862	-161.424015327646	-161.424015327646\\
66.125	0.12228	-174.25285926764	-174.25285926764\\
66.125	0.12594	-188.301321839919	-188.301321839919\\
66.125	0.1296	-203.569403044482	-203.569403044482\\
66.125	0.13326	-220.057102881329	-220.057102881329\\
66.125	0.13692	-237.76442135046	-237.76442135046\\
66.125	0.14058	-256.691358451875	-256.691358451875\\
66.125	0.14424	-276.837914185573	-276.837914185573\\
66.125	0.1479	-298.204088551557	-298.204088551557\\
66.125	0.15156	-320.789881549823	-320.789881549823\\
66.125	0.15522	-344.595293180375	-344.595293180375\\
66.125	0.15888	-369.62032344321	-369.62032344321\\
66.125	0.16254	-395.864972338329	-395.864972338329\\
66.125	0.1662	-423.329239865732	-423.329239865732\\
66.125	0.16986	-452.01312602542	-452.01312602542\\
66.125	0.17352	-481.91663081739	-481.91663081739\\
66.125	0.17718	-513.039754241646	-513.039754241646\\
66.125	0.18084	-545.382496298185	-545.382496298185\\
66.125	0.1845	-578.944856987008	-578.944856987008\\
66.125	0.18816	-613.726836308115	-613.726836308115\\
66.125	0.19182	-649.728434261507	-649.728434261507\\
66.125	0.19548	-686.949650847182	-686.949650847182\\
66.125	0.19914	-725.390486065142	-725.390486065142\\
66.125	0.2028	-765.050939915386	-765.050939915386\\
66.125	0.20646	-805.931012397913	-805.931012397913\\
66.125	0.21012	-848.030703512724	-848.030703512724\\
66.125	0.21378	-891.35001325982	-891.35001325982\\
66.125	0.21744	-935.888941639199	-935.888941639199\\
66.125	0.2211	-981.647488650863	-981.647488650863\\
66.125	0.22476	-1028.62565429481	-1028.62565429481\\
66.125	0.22842	-1076.82343857104	-1076.82343857104\\
66.125	0.23208	-1126.24084147956	-1126.24084147956\\
66.125	0.23574	-1176.87786302036	-1176.87786302036\\
66.125	0.2394	-1228.73450319344	-1228.73450319344\\
66.125	0.24306	-1281.81076199881	-1281.81076199881\\
66.125	0.24672	-1336.10663943646	-1336.10663943646\\
66.125	0.25038	-1391.6221355064	-1391.6221355064\\
66.125	0.25404	-1448.35725020862	-1448.35725020862\\
66.125	0.2577	-1506.31198354312	-1506.31198354312\\
66.125	0.26136	-1565.48633550991	-1565.48633550991\\
66.125	0.26502	-1625.88030610898	-1625.88030610898\\
66.125	0.26868	-1687.49389534034	-1687.49389534034\\
66.125	0.27234	-1750.32710320398	-1750.32710320398\\
66.125	0.276	-1814.3799296999	-1814.3799296999\\
66.5	0.093	-107.344768752973	-107.344768752973\\
66.5	0.09666	-111.665784071198	-111.665784071198\\
66.5	0.10032	-117.206418021707	-117.206418021707\\
66.5	0.10398	-123.966670604501	-123.966670604501\\
66.5	0.10764	-131.946541819578	-131.946541819578\\
66.5	0.1113	-141.14603166694	-141.14603166694\\
66.5	0.11496	-151.565140146585	-151.565140146585\\
66.5	0.11862	-163.203867258515	-163.203867258515\\
66.5	0.12228	-176.062213002728	-176.062213002728\\
66.5	0.12594	-190.140177379226	-190.140177379226\\
66.5	0.1296	-205.437760388007	-205.437760388007\\
66.5	0.13326	-221.954962029073	-221.954962029073\\
66.5	0.13692	-239.691782302422	-239.691782302422\\
66.5	0.14058	-258.648221208056	-258.648221208056\\
66.5	0.14424	-278.824278745974	-278.824278745974\\
66.5	0.1479	-300.219954916176	-300.219954916176\\
66.5	0.15156	-322.835249718661	-322.835249718661\\
66.5	0.15522	-346.670163153431	-346.670163153431\\
66.5	0.15888	-371.724695220485	-371.724695220485\\
66.5	0.16254	-397.998845919823	-397.998845919823\\
66.5	0.1662	-425.492615251444	-425.492615251444\\
66.5	0.16986	-454.206003215351	-454.206003215351\\
66.5	0.17352	-484.13900981154	-484.13900981154\\
66.5	0.17718	-515.291635040015	-515.291635040015\\
66.5	0.18084	-547.663878900772	-547.663878900772\\
66.5	0.1845	-581.255741393815	-581.255741393815\\
66.5	0.18816	-616.06722251914	-616.06722251914\\
66.5	0.19182	-652.09832227675	-652.09832227675\\
66.5	0.19548	-689.349040666645	-689.349040666645\\
66.5	0.19914	-727.819377688823	-727.819377688823\\
66.5	0.2028	-767.509333343286	-767.509333343286\\
66.5	0.20646	-808.418907630032	-808.418907630032\\
66.5	0.21012	-850.548100549062	-850.548100549062\\
66.5	0.21378	-893.896912100376	-893.896912100376\\
66.5	0.21744	-938.465342283974	-938.465342283974\\
66.5	0.2211	-984.253391099856	-984.253391099856\\
66.5	0.22476	-1031.26105854802	-1031.26105854802\\
66.5	0.22842	-1079.48834462847	-1079.48834462847\\
66.5	0.23208	-1128.93524934121	-1128.93524934121\\
66.5	0.23574	-1179.60177268623	-1179.60177268623\\
66.5	0.2394	-1231.48791466353	-1231.48791466353\\
66.5	0.24306	-1284.59367527312	-1284.59367527312\\
66.5	0.24672	-1338.91905451499	-1338.91905451499\\
66.5	0.25038	-1394.46405238914	-1394.46405238914\\
66.5	0.25404	-1451.22866889558	-1451.22866889558\\
66.5	0.2577	-1509.2129040343	-1509.2129040343\\
66.5	0.26136	-1568.41675780531	-1568.41675780531\\
66.5	0.26502	-1628.8402302086	-1628.8402302086\\
66.5	0.26868	-1690.48332124417	-1690.48332124417\\
66.5	0.27234	-1753.34603091203	-1753.34603091203\\
66.5	0.276	-1817.42835921218	-1817.42835921218\\
66.875	0.093	-109.056711418985	-109.056711418985\\
66.875	0.09666	-113.407228541429	-113.407228541429\\
66.875	0.10032	-118.977364296157	-118.977364296157\\
66.875	0.10398	-125.767118683169	-125.767118683169\\
66.875	0.10764	-133.776491702465	-133.776491702465\\
66.875	0.1113	-143.005483354046	-143.005483354046\\
66.875	0.11496	-153.454093637909	-153.454093637909\\
66.875	0.11862	-165.122322554058	-165.122322554058\\
66.875	0.12228	-178.01017010249	-178.01017010249\\
66.875	0.12594	-192.117636283206	-192.117636283206\\
66.875	0.1296	-207.444721096206	-207.444721096206\\
66.875	0.13326	-223.99142454149	-223.99142454149\\
66.875	0.13692	-241.757746619059	-241.757746619059\\
66.875	0.14058	-260.743687328912	-260.743687328912\\
66.875	0.14424	-280.949246671048	-280.949246671048\\
66.875	0.1479	-302.374424645468	-302.374424645468\\
66.875	0.15156	-325.019221252172	-325.019221252172\\
66.875	0.15522	-348.883636491161	-348.883636491161\\
66.875	0.15888	-373.967670362434	-373.967670362434\\
66.875	0.16254	-400.27132286599	-400.27132286599\\
66.875	0.1662	-427.794594001831	-427.794594001831\\
66.875	0.16986	-456.537483769956	-456.537483769956\\
66.875	0.17352	-486.499992170364	-486.499992170364\\
66.875	0.17718	-517.682119203057	-517.682119203057\\
66.875	0.18084	-550.083864868033	-550.083864868033\\
66.875	0.1845	-583.705229165294	-583.705229165294\\
66.875	0.18816	-618.546212094839	-618.546212094839\\
66.875	0.19182	-654.606813656668	-654.606813656668\\
66.875	0.19548	-691.887033850781	-691.887033850781\\
66.875	0.19914	-730.386872677178	-730.386872677178\\
66.875	0.2028	-770.106330135859	-770.106330135859\\
66.875	0.20646	-811.045406226824	-811.045406226824\\
66.875	0.21012	-853.204100950073	-853.204100950073\\
66.875	0.21378	-896.582414305605	-896.582414305605\\
66.875	0.21744	-941.180346293422	-941.180346293422\\
66.875	0.2211	-986.997896913524	-986.997896913524\\
66.875	0.22476	-1034.03506616591	-1034.03506616591\\
66.875	0.22842	-1082.29185405058	-1082.29185405058\\
66.875	0.23208	-1131.76826056753	-1131.76826056753\\
66.875	0.23574	-1182.46428571677	-1182.46428571677\\
66.875	0.2394	-1234.37992949829	-1234.37992949829\\
66.875	0.24306	-1287.51519191209	-1287.51519191209\\
66.875	0.24672	-1341.87007295818	-1341.87007295818\\
66.875	0.25038	-1397.44457263656	-1397.44457263656\\
66.875	0.25404	-1454.23869094721	-1454.23869094721\\
66.875	0.2577	-1512.25242789016	-1512.25242789016\\
66.875	0.26136	-1571.48578346538	-1571.48578346538\\
66.875	0.26502	-1631.93875767289	-1631.93875767289\\
66.875	0.26868	-1693.61135051268	-1693.61135051268\\
66.875	0.27234	-1756.50356198476	-1756.50356198476\\
66.875	0.276	-1820.61539208912	-1820.61539208912\\
67.25	0.093	-110.90725744967	-110.90725744967\\
67.25	0.09666	-115.287276376333	-115.287276376333\\
67.25	0.10032	-120.886913935279	-120.886913935279\\
67.25	0.10398	-127.706170126511	-127.706170126511\\
67.25	0.10764	-135.745044950025	-135.745044950025\\
67.25	0.1113	-145.003538405824	-145.003538405824\\
67.25	0.11496	-155.481650493906	-155.481650493906\\
67.25	0.11862	-167.179381214274	-167.179381214274\\
67.25	0.12228	-180.096730566924	-180.096730566924\\
67.25	0.12594	-194.23369855186	-194.23369855186\\
67.25	0.1296	-209.590285169079	-209.590285169079\\
67.25	0.13326	-226.166490418582	-226.166490418582\\
67.25	0.13692	-243.962314300369	-243.962314300369\\
67.25	0.14058	-262.97775681444	-262.97775681444\\
67.25	0.14424	-283.212817960795	-283.212817960795\\
67.25	0.1479	-304.667497739434	-304.667497739434\\
67.25	0.15156	-327.341796150357	-327.341796150357\\
67.25	0.15522	-351.235713193565	-351.235713193565\\
67.25	0.15888	-376.349248869056	-376.349248869056\\
67.25	0.16254	-402.682403176831	-402.682403176831\\
67.25	0.1662	-430.23517611689	-430.23517611689\\
67.25	0.16986	-459.007567689234	-459.007567689234\\
67.25	0.17352	-488.999577893861	-488.999577893861\\
67.25	0.17718	-520.211206730773	-520.211206730773\\
67.25	0.18084	-552.642454199967	-552.642454199967\\
67.25	0.1845	-586.293320301448	-586.293320301448\\
67.25	0.18816	-621.163805035211	-621.163805035211\\
67.25	0.19182	-657.253908401258	-657.253908401258\\
67.25	0.19548	-694.56363039959	-694.56363039959\\
67.25	0.19914	-733.092971030206	-733.092971030206\\
67.25	0.2028	-772.841930293106	-772.841930293106\\
67.25	0.20646	-813.81050818829	-813.81050818829\\
67.25	0.21012	-855.998704715757	-855.998704715757\\
67.25	0.21378	-899.406519875508	-899.406519875508\\
67.25	0.21744	-944.033953667544	-944.033953667544\\
67.25	0.2211	-989.881006091864	-989.881006091864\\
67.25	0.22476	-1036.94767714847	-1036.94767714847\\
67.25	0.22842	-1085.23396683736	-1085.23396683736\\
67.25	0.23208	-1134.73987515853	-1134.73987515853\\
67.25	0.23574	-1185.46540211198	-1185.46540211198\\
67.25	0.2394	-1237.41054769772	-1237.41054769772\\
67.25	0.24306	-1290.57531191575	-1290.57531191575\\
67.25	0.24672	-1344.95969476606	-1344.95969476606\\
67.25	0.25038	-1400.56369624865	-1400.56369624865\\
67.25	0.25404	-1457.38731636352	-1457.38731636352\\
67.25	0.2577	-1515.43055511068	-1515.43055511068\\
67.25	0.26136	-1574.69341249013	-1574.69341249013\\
67.25	0.26502	-1635.17588850186	-1635.17588850186\\
67.25	0.26868	-1696.87798314587	-1696.87798314587\\
67.25	0.27234	-1759.79969642216	-1759.79969642216\\
67.25	0.276	-1823.94102833074	-1823.94102833074\\
67.625	0.093	-112.896406845029	-112.896406845029\\
67.625	0.09666	-117.305927575911	-117.305927575911\\
67.625	0.10032	-122.935066939076	-122.935066939076\\
67.625	0.10398	-129.783824934526	-129.783824934526\\
67.625	0.10764	-137.852201562259	-137.852201562259\\
67.625	0.1113	-147.140196822277	-147.140196822277\\
67.625	0.11496	-157.647810714578	-157.647810714578\\
67.625	0.11862	-169.375043239164	-169.375043239164\\
67.625	0.12228	-182.321894396033	-182.321894396033\\
67.625	0.12594	-196.488364185188	-196.488364185188\\
67.625	0.1296	-211.874452606625	-211.874452606625\\
67.625	0.13326	-228.480159660347	-228.480159660347\\
67.625	0.13692	-246.305485346352	-246.305485346352\\
67.625	0.14058	-265.350429664643	-265.350429664643\\
67.625	0.14424	-285.614992615216	-285.614992615216\\
67.625	0.1479	-307.099174198074	-307.099174198074\\
67.625	0.15156	-329.802974413216	-329.802974413216\\
67.625	0.15522	-353.726393260642	-353.726393260642\\
67.625	0.15888	-378.869430740352	-378.869430740352\\
67.625	0.16254	-405.232086852346	-405.232086852346\\
67.625	0.1662	-432.814361596624	-432.814361596624\\
67.625	0.16986	-461.616254973186	-461.616254973186\\
67.625	0.17352	-491.637766982032	-491.637766982032\\
67.625	0.17718	-522.878897623162	-522.878897623162\\
67.625	0.18084	-555.339646896576	-555.339646896576\\
67.625	0.1845	-589.020014802275	-589.020014802275\\
67.625	0.18816	-623.920001340257	-623.920001340257\\
67.625	0.19182	-660.039606510523	-660.039606510523\\
67.625	0.19548	-697.378830313073	-697.378830313073\\
67.625	0.19914	-735.937672747908	-735.937672747908\\
67.625	0.2028	-775.716133815026	-775.716133815026\\
67.625	0.20646	-816.714213514429	-816.714213514429\\
67.625	0.21012	-858.931911846115	-858.931911846115\\
67.625	0.21378	-902.369228810085	-902.369228810085\\
67.625	0.21744	-947.026164406339	-947.026164406339\\
67.625	0.2211	-992.902718634878	-992.902718634878\\
67.625	0.22476	-1039.9988914957	-1039.9988914957\\
67.625	0.22842	-1088.31468298881	-1088.31468298881\\
67.625	0.23208	-1137.8500931142	-1137.8500931142\\
67.625	0.23574	-1188.60512187187	-1188.60512187187\\
67.625	0.2394	-1240.57976926183	-1240.57976926183\\
67.625	0.24306	-1293.77403528407	-1293.77403528407\\
67.625	0.24672	-1348.1879199386	-1348.1879199386\\
67.625	0.25038	-1403.82142322541	-1403.82142322541\\
67.625	0.25404	-1460.67454514451	-1460.67454514451\\
67.625	0.2577	-1518.74728569588	-1518.74728569588\\
67.625	0.26136	-1578.03964487955	-1578.03964487955\\
67.625	0.26502	-1638.5516226955	-1638.5516226955\\
67.625	0.26868	-1700.28321914373	-1700.28321914373\\
67.625	0.27234	-1763.23443422424	-1763.23443422424\\
67.625	0.276	-1827.40526793704	-1827.40526793704\\
68	0.093	-115.024159605061	-115.024159605061\\
68	0.09666	-119.463182140161	-119.463182140161\\
68	0.10032	-125.121823307545	-125.121823307545\\
68	0.10398	-132.000083107214	-132.000083107214\\
68	0.10764	-140.097961539166	-140.097961539166\\
68	0.1113	-149.415458603403	-149.415458603403\\
68	0.11496	-159.952574299922	-159.952574299922\\
68	0.11862	-171.709308628727	-171.709308628727\\
68	0.12228	-184.685661589815	-184.685661589815\\
68	0.12594	-198.881633183188	-198.881633183188\\
68	0.1296	-214.297223408844	-214.297223408844\\
68	0.13326	-230.932432266785	-230.932432266785\\
68	0.13692	-248.787259757009	-248.787259757009\\
68	0.14058	-267.861705879518	-267.861705879518\\
68	0.14424	-288.15577063431	-288.15577063431\\
68	0.1479	-309.669454021387	-309.669454021387\\
68	0.15156	-332.402756040747	-332.402756040747\\
68	0.15522	-356.355676692392	-356.355676692392\\
68	0.15888	-381.528215976321	-381.528215976321\\
68	0.16254	-407.920373892533	-407.920373892533\\
68	0.1662	-435.53215044103	-435.53215044103\\
68	0.16986	-464.363545621811	-464.363545621811\\
68	0.17352	-494.414559434875	-494.414559434875\\
68	0.17718	-525.685191880224	-525.685191880224\\
68	0.18084	-558.175442957857	-558.175442957857\\
68	0.1845	-591.885312667775	-591.885312667775\\
68	0.18816	-626.814801009975	-626.814801009975\\
68	0.19182	-662.96390798446	-662.96390798446\\
68	0.19548	-700.332633591229	-700.332633591229\\
68	0.19914	-738.920977830282	-738.920977830282\\
68	0.2028	-778.72894070162	-778.72894070162\\
68	0.20646	-819.756522205241	-819.756522205241\\
68	0.21012	-862.003722341145	-862.003722341145\\
68	0.21378	-905.470541109335	-905.470541109335\\
68	0.21744	-950.156978509808	-950.156978509808\\
68	0.2211	-996.063034542565	-996.063034542565\\
68	0.22476	-1043.18870920761	-1043.18870920761\\
68	0.22842	-1091.53400250493	-1091.53400250493\\
68	0.23208	-1141.09891443454	-1141.09891443454\\
68	0.23574	-1191.88344499643	-1191.88344499643\\
68	0.2394	-1243.88759419061	-1243.88759419061\\
68	0.24306	-1297.11136201707	-1297.11136201707\\
68	0.24672	-1351.55474847582	-1351.55474847582\\
68	0.25038	-1407.21775356685	-1407.21775356685\\
68	0.25404	-1464.10037729016	-1464.10037729016\\
68	0.2577	-1522.20261964576	-1522.20261964576\\
68	0.26136	-1581.52448063364	-1581.52448063364\\
68	0.26502	-1642.06596025381	-1642.06596025381\\
68	0.26868	-1703.82705850626	-1703.82705850626\\
68	0.27234	-1766.80777539099	-1766.80777539099\\
68	0.276	-1831.00811090801	-1831.00811090801\\
68.375	0.093	-117.290515729766	-117.290515729766\\
68.375	0.09666	-121.759040069085	-121.759040069085\\
68.375	0.10032	-127.447183040688	-127.447183040688\\
68.375	0.10398	-134.354944644575	-134.354944644575\\
68.375	0.10764	-142.482324880746	-142.482324880746\\
68.375	0.1113	-151.829323749201	-151.829323749201\\
68.375	0.11496	-162.39594124994	-162.39594124994\\
68.375	0.11862	-174.182177382963	-174.182177382963\\
68.375	0.12228	-187.18803214827	-187.18803214827\\
68.375	0.12594	-201.413505545861	-201.413505545861\\
68.375	0.1296	-216.858597575736	-216.858597575736\\
68.375	0.13326	-233.523308237895	-233.523308237895\\
68.375	0.13692	-251.407637532338	-251.407637532338\\
68.375	0.14058	-270.511585459066	-270.511585459066\\
68.375	0.14424	-290.835152018077	-290.835152018077\\
68.375	0.1479	-312.378337209372	-312.378337209372\\
68.375	0.15156	-335.141141032951	-335.141141032951\\
68.375	0.15522	-359.123563488815	-359.123563488815\\
68.375	0.15888	-384.325604576962	-384.325604576962\\
68.375	0.16254	-410.747264297394	-410.747264297394\\
68.375	0.1662	-438.388542650109	-438.388542650109\\
68.375	0.16986	-467.249439635109	-467.249439635109\\
68.375	0.17352	-497.329955252392	-497.329955252392\\
68.375	0.17718	-528.63008950196	-528.63008950196\\
68.375	0.18084	-561.149842383811	-561.149842383811\\
68.375	0.1845	-594.889213897947	-594.889213897947\\
68.375	0.18816	-629.848204044367	-629.848204044367\\
68.375	0.19182	-666.02681282307	-666.02681282307\\
68.375	0.19548	-703.425040234058	-703.425040234058\\
68.375	0.19914	-742.042886277331	-742.042886277331\\
68.375	0.2028	-781.880350952886	-781.880350952886\\
68.375	0.20646	-822.937434260726	-822.937434260726\\
68.375	0.21012	-865.21413620085	-865.21413620085\\
68.375	0.21378	-908.710456773257	-908.710456773257\\
68.375	0.21744	-953.426395977949	-953.426395977949\\
68.375	0.2211	-999.361953814925	-999.361953814925\\
68.375	0.22476	-1046.51713028418	-1046.51713028418\\
68.375	0.22842	-1094.89192538573	-1094.89192538573\\
68.375	0.23208	-1144.48633911956	-1144.48633911956\\
68.375	0.23574	-1195.30037148567	-1195.30037148567\\
68.375	0.2394	-1247.33402248407	-1247.33402248407\\
68.375	0.24306	-1300.58729211475	-1300.58729211475\\
68.375	0.24672	-1355.06018037771	-1355.06018037771\\
68.375	0.25038	-1410.75268727296	-1410.75268727296\\
68.375	0.25404	-1467.66481280049	-1467.66481280049\\
68.375	0.2577	-1525.79655696031	-1525.79655696031\\
68.375	0.26136	-1585.14791975241	-1585.14791975241\\
68.375	0.26502	-1645.71890117679	-1645.71890117679\\
68.375	0.26868	-1707.50950123346	-1707.50950123346\\
68.375	0.27234	-1770.51971992241	-1770.51971992241\\
68.375	0.276	-1834.74955724365	-1834.74955724365\\
68.75	0.093	-119.695475219144	-119.695475219144\\
68.75	0.09666	-124.193501362682	-124.193501362682\\
68.75	0.10032	-129.911146138504	-129.911146138504\\
68.75	0.10398	-136.848409546609	-136.848409546609\\
68.75	0.10764	-145.005291586999	-145.005291586999\\
68.75	0.1113	-154.381792259673	-154.381792259673\\
68.75	0.11496	-164.97791156463	-164.97791156463\\
68.75	0.11862	-176.793649501872	-176.793649501872\\
68.75	0.12228	-189.829006071398	-189.829006071398\\
68.75	0.12594	-204.083981273208	-204.083981273208\\
68.75	0.1296	-219.558575107302	-219.558575107302\\
68.75	0.13326	-236.252787573679	-236.252787573679\\
68.75	0.13692	-254.166618672341	-254.166618672341\\
68.75	0.14058	-273.300068403287	-273.300068403287\\
68.75	0.14424	-293.653136766517	-293.653136766517\\
68.75	0.1479	-315.225823762032	-315.225823762032\\
68.75	0.15156	-338.018129389829	-338.018129389829\\
68.75	0.15522	-362.030053649911	-362.030053649911\\
68.75	0.15888	-387.261596542278	-387.261596542278\\
68.75	0.16254	-413.712758066928	-413.712758066928\\
68.75	0.1662	-441.383538223862	-441.383538223862\\
68.75	0.16986	-470.273937013081	-470.273937013081\\
68.75	0.17352	-500.383954434582	-500.383954434582\\
68.75	0.17718	-531.713590488369	-531.713590488369\\
68.75	0.18084	-564.262845174438	-564.262845174438\\
68.75	0.1845	-598.031718492794	-598.031718492794\\
68.75	0.18816	-633.020210443432	-633.020210443432\\
68.75	0.19182	-669.228321026354	-669.228321026354\\
68.75	0.19548	-706.65605024156	-706.65605024156\\
68.75	0.19914	-745.303398089051	-745.303398089051\\
68.75	0.2028	-785.170364568826	-785.170364568826\\
68.75	0.20646	-826.256949680884	-826.256949680884\\
68.75	0.21012	-868.563153425227	-868.563153425227\\
68.75	0.21378	-912.088975801853	-912.088975801853\\
68.75	0.21744	-956.834416810764	-956.834416810764\\
68.75	0.2211	-1002.79947645196	-1002.79947645196\\
68.75	0.22476	-1049.98415472544	-1049.98415472544\\
68.75	0.22842	-1098.3884516312	-1098.3884516312\\
68.75	0.23208	-1148.01236716925	-1148.01236716925\\
68.75	0.23574	-1198.85590133958	-1198.85590133958\\
68.75	0.2394	-1250.91905414219	-1250.91905414219\\
68.75	0.24306	-1304.20182557709	-1304.20182557709\\
68.75	0.24672	-1358.70421564427	-1358.70421564427\\
68.75	0.25038	-1414.42622434374	-1414.42622434374\\
68.75	0.25404	-1471.36785167549	-1471.36785167549\\
68.75	0.2577	-1529.52909763953	-1529.52909763953\\
68.75	0.26136	-1588.90996223585	-1588.90996223585\\
68.75	0.26502	-1649.51044546445	-1649.51044546445\\
68.75	0.26868	-1711.33054732534	-1711.33054732534\\
68.75	0.27234	-1774.37026781851	-1774.37026781851\\
68.75	0.276	-1838.62960694396	-1838.62960694396\\
69.125	0.093	-122.239038073197	-122.239038073197\\
69.125	0.09666	-126.766566020954	-126.766566020954\\
69.125	0.10032	-132.513712600993	-132.513712600993\\
69.125	0.10398	-139.480477813318	-139.480477813318\\
69.125	0.10764	-147.666861657927	-147.666861657927\\
69.125	0.1113	-157.072864134819	-157.072864134819\\
69.125	0.11496	-167.698485243995	-167.698485243995\\
69.125	0.11862	-179.543724985456	-179.543724985456\\
69.125	0.12228	-192.6085833592	-192.6085833592\\
69.125	0.12594	-206.893060365229	-206.893060365229\\
69.125	0.1296	-222.397156003541	-222.397156003541\\
69.125	0.13326	-239.120870274138	-239.120870274138\\
69.125	0.13692	-257.064203177018	-257.064203177018\\
69.125	0.14058	-276.227154712184	-276.227154712184\\
69.125	0.14424	-296.609724879632	-296.609724879632\\
69.125	0.1479	-318.211913679365	-318.211913679365\\
69.125	0.15156	-341.033721111381	-341.033721111381\\
69.125	0.15522	-365.075147175682	-365.075147175682\\
69.125	0.15888	-390.336191872267	-390.336191872267\\
69.125	0.16254	-416.816855201136	-416.816855201136\\
69.125	0.1662	-444.517137162289	-444.517137162289\\
69.125	0.16986	-473.437037755726	-473.437037755726\\
69.125	0.17352	-503.576556981446	-503.576556981446\\
69.125	0.17718	-534.935694839452	-534.935694839452\\
69.125	0.18084	-567.51445132974	-567.51445132974\\
69.125	0.1845	-601.312826452314	-601.312826452314\\
69.125	0.18816	-636.330820207171	-636.330820207171\\
69.125	0.19182	-672.568432594312	-672.568432594312\\
69.125	0.19548	-710.025663613737	-710.025663613737\\
69.125	0.19914	-748.702513265447	-748.702513265447\\
69.125	0.2028	-788.59898154944	-788.59898154944\\
69.125	0.20646	-829.715068465717	-829.715068465717\\
69.125	0.21012	-872.050774014278	-872.050774014278\\
69.125	0.21378	-915.606098195123	-915.606098195123\\
69.125	0.21744	-960.381041008252	-960.381041008252\\
69.125	0.2211	-1006.37560245367	-1006.37560245367\\
69.125	0.22476	-1053.58978253136	-1053.58978253136\\
69.125	0.22842	-1102.02358124134	-1102.02358124134\\
69.125	0.23208	-1151.67699858361	-1151.67699858361\\
69.125	0.23574	-1202.55003455816	-1202.55003455816\\
69.125	0.2394	-1254.64268916499	-1254.64268916499\\
69.125	0.24306	-1307.95496240411	-1307.95496240411\\
69.125	0.24672	-1362.48685427551	-1362.48685427551\\
69.125	0.25038	-1418.2383647792	-1418.2383647792\\
69.125	0.25404	-1475.20949391517	-1475.20949391517\\
69.125	0.2577	-1533.40024168342	-1533.40024168342\\
69.125	0.26136	-1592.81060808396	-1592.81060808396\\
69.125	0.26502	-1653.44059311678	-1653.44059311678\\
69.125	0.26868	-1715.29019678189	-1715.29019678189\\
69.125	0.27234	-1778.35941907928	-1778.35941907928\\
69.125	0.276	-1842.64826000895	-1842.64826000895\\
69.5	0.093	-124.921204291922	-124.921204291922\\
69.5	0.09666	-129.478234043897	-129.478234043897\\
69.5	0.10032	-135.254882428156	-135.254882428156\\
69.5	0.10398	-142.2511494447	-142.2511494447\\
69.5	0.10764	-150.467035093527	-150.467035093527\\
69.5	0.1113	-159.902539374638	-159.902539374638\\
69.5	0.11496	-170.557662288033	-170.557662288033\\
69.5	0.11862	-182.432403833712	-182.432403833712\\
69.5	0.12228	-195.526764011675	-195.526764011675\\
69.5	0.12594	-209.840742821923	-209.840742821923\\
69.5	0.1296	-225.374340264454	-225.374340264454\\
69.5	0.13326	-242.127556339269	-242.127556339269\\
69.5	0.13692	-260.100391046369	-260.100391046369\\
69.5	0.14058	-279.292844385752	-279.292844385752\\
69.5	0.14424	-299.704916357419	-299.704916357419\\
69.5	0.1479	-321.336606961371	-321.336606961371\\
69.5	0.15156	-344.187916197606	-344.187916197606\\
69.5	0.15522	-368.258844066126	-368.258844066126\\
69.5	0.15888	-393.549390566929	-393.549390566929\\
69.5	0.16254	-420.059555700017	-420.059555700017\\
69.5	0.1662	-447.789339465388	-447.789339465388\\
69.5	0.16986	-476.738741863045	-476.738741863045\\
69.5	0.17352	-506.907762892984	-506.907762892984\\
69.5	0.17718	-538.296402555208	-538.296402555208\\
69.5	0.18084	-570.904660849715	-570.904660849715\\
69.5	0.1845	-604.732537776507	-604.732537776507\\
69.5	0.18816	-639.780033335583	-639.780033335583\\
69.5	0.19182	-676.047147526942	-676.047147526942\\
69.5	0.19548	-713.533880350586	-713.533880350586\\
69.5	0.19914	-752.240231806515	-752.240231806515\\
69.5	0.2028	-792.166201894727	-792.166201894727\\
69.5	0.20646	-833.311790615223	-833.311790615223\\
69.5	0.21012	-875.676997968003	-875.676997968003\\
69.5	0.21378	-919.261823953066	-919.261823953066\\
69.5	0.21744	-964.066268570414	-964.066268570414\\
69.5	0.2211	-1010.09033182005	-1010.09033182005\\
69.5	0.22476	-1057.33401370196	-1057.33401370196\\
69.5	0.22842	-1105.79731421616	-1105.79731421616\\
69.5	0.23208	-1155.48023336265	-1155.48023336265\\
69.5	0.23574	-1206.38277114142	-1206.38277114142\\
69.5	0.2394	-1258.50492755247	-1258.50492755247\\
69.5	0.24306	-1311.8467025958	-1311.8467025958\\
69.5	0.24672	-1366.40809627142	-1366.40809627142\\
69.5	0.25038	-1422.18910857933	-1422.18910857933\\
69.5	0.25404	-1479.18973951952	-1479.18973951952\\
69.5	0.2577	-1537.40998909199	-1537.40998909199\\
69.5	0.26136	-1596.84985729675	-1596.84985729675\\
69.5	0.26502	-1657.50934413379	-1657.50934413379\\
69.5	0.26868	-1719.38844960311	-1719.38844960311\\
69.5	0.27234	-1782.48717370472	-1782.48717370472\\
69.5	0.276	-1846.80551643861	-1846.80551643861\\
69.875	0.093	-127.741973875321	-127.741973875321\\
69.875	0.09666	-132.328505431515	-132.328505431515\\
69.875	0.10032	-138.134655619992	-138.134655619992\\
69.875	0.10398	-145.160424440754	-145.160424440754\\
69.875	0.10764	-153.4058118938	-153.4058118938\\
69.875	0.1113	-162.87081797913	-162.87081797913\\
69.875	0.11496	-173.555442696744	-173.555442696744\\
69.875	0.11862	-185.459686046642	-185.459686046642\\
69.875	0.12228	-198.583548028823	-198.583548028823\\
69.875	0.12594	-212.92702864329	-212.92702864329\\
69.875	0.1296	-228.49012789004	-228.49012789004\\
69.875	0.13326	-245.272845769073	-245.272845769073\\
69.875	0.13692	-263.275182280392	-263.275182280392\\
69.875	0.14058	-282.497137423994	-282.497137423994\\
69.875	0.14424	-302.93871119988	-302.93871119988\\
69.875	0.1479	-324.59990360805	-324.59990360805\\
69.875	0.15156	-347.480714648504	-347.480714648504\\
69.875	0.15522	-371.581144321243	-371.581144321243\\
69.875	0.15888	-396.901192626265	-396.901192626265\\
69.875	0.16254	-423.440859563571	-423.440859563571\\
69.875	0.1662	-451.200145133162	-451.200145133162\\
69.875	0.16986	-480.179049335036	-480.179049335036\\
69.875	0.17352	-510.377572169194	-510.377572169194\\
69.875	0.17718	-541.795713635637	-541.795713635637\\
69.875	0.18084	-574.433473734363	-574.433473734363\\
69.875	0.1845	-608.290852465374	-608.290852465374\\
69.875	0.18816	-643.367849828668	-643.367849828668\\
69.875	0.19182	-679.664465824246	-679.664465824246\\
69.875	0.19548	-717.180700452109	-717.180700452109\\
69.875	0.19914	-755.916553712256	-755.916553712256\\
69.875	0.2028	-795.872025604687	-795.872025604687\\
69.875	0.20646	-837.047116129401	-837.047116129401\\
69.875	0.21012	-879.4418252864	-879.4418252864\\
69.875	0.21378	-923.056153075682	-923.056153075682\\
69.875	0.21744	-967.890099497249	-967.890099497249\\
69.875	0.2211	-1013.9436645511	-1013.9436645511\\
69.875	0.22476	-1061.21684823723	-1061.21684823723\\
69.875	0.22842	-1109.70965055565	-1109.70965055565\\
69.875	0.23208	-1159.42207150636	-1159.42207150636\\
69.875	0.23574	-1210.35411108934	-1210.35411108934\\
69.875	0.2394	-1262.50576930461	-1262.50576930461\\
69.875	0.24306	-1315.87704615217	-1315.87704615217\\
69.875	0.24672	-1370.46794163201	-1370.46794163201\\
69.875	0.25038	-1426.27845574413	-1426.27845574413\\
69.875	0.25404	-1483.30858848854	-1483.30858848854\\
69.875	0.2577	-1541.55833986523	-1541.55833986523\\
69.875	0.26136	-1601.02770987421	-1601.02770987421\\
69.875	0.26502	-1661.71669851546	-1661.71669851546\\
69.875	0.26868	-1723.62530578901	-1723.62530578901\\
69.875	0.27234	-1786.75353169483	-1786.75353169483\\
69.875	0.276	-1851.10137623295	-1851.10137623295\\
70.25	0.093	-130.701346823393	-130.701346823393\\
70.25	0.09666	-135.317380183805	-135.317380183805\\
70.25	0.10032	-141.153032176501	-141.153032176501\\
70.25	0.10398	-148.208302801482	-148.208302801482\\
70.25	0.10764	-156.483192058746	-156.483192058746\\
70.25	0.1113	-165.977699948295	-165.977699948295\\
70.25	0.11496	-176.691826470127	-176.691826470127\\
70.25	0.11862	-188.625571624245	-188.625571624245\\
70.25	0.12228	-201.778935410645	-201.778935410645\\
70.25	0.12594	-216.15191782933	-216.15191782933\\
70.25	0.1296	-231.744518880298	-231.744518880298\\
70.25	0.13326	-248.556738563551	-248.556738563551\\
70.25	0.13692	-266.588576879088	-266.588576879088\\
70.25	0.14058	-285.840033826909	-285.840033826909\\
70.25	0.14424	-306.311109407013	-306.311109407013\\
70.25	0.1479	-328.001803619403	-328.001803619403\\
70.25	0.15156	-350.912116464075	-350.912116464075\\
70.25	0.15522	-375.042047941032	-375.042047941032\\
70.25	0.15888	-400.391598050273	-400.391598050273\\
70.25	0.16254	-426.960766791798	-426.960766791798\\
70.25	0.1662	-454.749554165608	-454.749554165608\\
70.25	0.16986	-483.757960171701	-483.757960171701\\
70.25	0.17352	-513.985984810077	-513.985984810077\\
70.25	0.17718	-545.433628080739	-545.433628080739\\
70.25	0.18084	-578.100889983684	-578.100889983684\\
70.25	0.1845	-611.987770518913	-611.987770518913\\
70.25	0.18816	-647.094269686427	-647.094269686427\\
70.25	0.19182	-683.420387486223	-683.420387486223\\
70.25	0.19548	-720.966123918305	-720.966123918305\\
70.25	0.19914	-759.731478982671	-759.731478982671\\
70.25	0.2028	-799.71645267932	-799.71645267932\\
70.25	0.20646	-840.921045008253	-840.921045008253\\
70.25	0.21012	-883.345255969471	-883.345255969471\\
70.25	0.21378	-926.989085562971	-926.989085562971\\
70.25	0.21744	-971.852533788757	-971.852533788757\\
70.25	0.2211	-1017.93560064683	-1017.93560064683\\
70.25	0.22476	-1065.23828613718	-1065.23828613718\\
70.25	0.22842	-1113.76059025982	-1113.76059025982\\
70.25	0.23208	-1163.50251301474	-1163.50251301474\\
70.25	0.23574	-1214.46405440194	-1214.46405440194\\
70.25	0.2394	-1266.64521442143	-1266.64521442143\\
70.25	0.24306	-1320.04599307321	-1320.04599307321\\
70.25	0.24672	-1374.66639035727	-1374.66639035727\\
70.25	0.25038	-1430.50640627361	-1430.50640627361\\
70.25	0.25404	-1487.56604082223	-1487.56604082223\\
70.25	0.2577	-1545.84529400314	-1545.84529400314\\
70.25	0.26136	-1605.34416581634	-1605.34416581634\\
70.25	0.26502	-1666.06265626181	-1666.06265626181\\
70.25	0.26868	-1728.00076533958	-1728.00076533958\\
70.25	0.27234	-1791.15849304962	-1791.15849304962\\
70.25	0.276	-1855.53583939195	-1855.53583939195\\
70.625	0.093	-133.799323136138	-133.799323136138\\
70.625	0.09666	-138.444858300769	-138.444858300769\\
70.625	0.10032	-144.310012097684	-144.310012097684\\
70.625	0.10398	-151.394784526883	-151.394784526883\\
70.625	0.10764	-159.699175588367	-159.699175588367\\
70.625	0.1113	-169.223185282134	-169.223185282134\\
70.625	0.11496	-179.966813608185	-179.966813608185\\
70.625	0.11862	-191.930060566521	-191.930060566521\\
70.625	0.12228	-205.11292615714	-205.11292615714\\
70.625	0.12594	-219.515410380043	-219.515410380043\\
70.625	0.1296	-235.137513235231	-235.137513235231\\
70.625	0.13326	-251.979234722702	-251.979234722702\\
70.625	0.13692	-270.040574842458	-270.040574842458\\
70.625	0.14058	-289.321533594498	-289.321533594498\\
70.625	0.14424	-309.822110978821	-309.822110978821\\
70.625	0.1479	-331.542306995429	-331.542306995429\\
70.625	0.15156	-354.48212164432	-354.48212164432\\
70.625	0.15522	-378.641554925496	-378.641554925496\\
70.625	0.15888	-404.020606838955	-404.020606838955\\
70.625	0.16254	-430.619277384699	-430.619277384699\\
70.625	0.1662	-458.437566562727	-458.437566562727\\
70.625	0.16986	-487.475474373039	-487.475474373039\\
70.625	0.17352	-517.733000815634	-517.733000815634\\
70.625	0.17718	-549.210145890514	-549.210145890514\\
70.625	0.18084	-581.906909597678	-581.906909597678\\
70.625	0.1845	-615.823291937127	-615.823291937127\\
70.625	0.18816	-650.959292908858	-650.959292908858\\
70.625	0.19182	-687.314912512874	-687.314912512874\\
70.625	0.19548	-724.890150749174	-724.890150749174\\
70.625	0.19914	-763.685007617758	-763.685007617758\\
70.625	0.2028	-803.699483118627	-803.699483118627\\
70.625	0.20646	-844.933577251779	-844.933577251779\\
70.625	0.21012	-887.387290017215	-887.387290017215\\
70.625	0.21378	-931.060621414934	-931.060621414934\\
70.625	0.21744	-975.953571444938	-975.953571444938\\
70.625	0.2211	-1022.06614010723	-1022.06614010723\\
70.625	0.22476	-1069.3983274018	-1069.3983274018\\
70.625	0.22842	-1117.95013332866	-1117.95013332866\\
70.625	0.23208	-1167.7215578878	-1167.7215578878\\
70.625	0.23574	-1218.71260107922	-1218.71260107922\\
70.625	0.2394	-1270.92326290293	-1270.92326290293\\
70.625	0.24306	-1324.35354335892	-1324.35354335892\\
70.625	0.24672	-1379.0034424472	-1379.0034424472\\
70.625	0.25038	-1434.87296016776	-1434.87296016776\\
70.625	0.25404	-1491.9620965206	-1491.9620965206\\
70.625	0.2577	-1550.27085150573	-1550.27085150573\\
70.625	0.26136	-1609.79922512314	-1609.79922512314\\
70.625	0.26502	-1670.54721737284	-1670.54721737284\\
70.625	0.26868	-1732.51482825482	-1732.51482825482\\
70.625	0.27234	-1795.70205776909	-1795.70205776909\\
70.625	0.276	-1860.10890591563	-1860.10890591563\\
71	0.093	-137.035902813557	-137.035902813557\\
71	0.09666	-141.710939782407	-141.710939782407\\
71	0.10032	-147.605595383541	-147.605595383541\\
71	0.10398	-154.719869616959	-154.719869616959\\
71	0.10764	-163.053762482661	-163.053762482661\\
71	0.1113	-172.607273980647	-172.607273980647\\
71	0.11496	-183.380404110917	-183.380404110917\\
71	0.11862	-195.373152873471	-195.373152873471\\
71	0.12228	-208.585520268309	-208.585520268309\\
71	0.12594	-223.017506295432	-223.017506295432\\
71	0.1296	-238.669110954837	-238.669110954837\\
71	0.13326	-255.540334246527	-255.540334246527\\
71	0.13692	-273.631176170501	-273.631176170501\\
71	0.14058	-292.94163672676	-292.94163672676\\
71	0.14424	-313.471715915302	-313.471715915302\\
71	0.1479	-335.221413736129	-335.221413736129\\
71	0.15156	-358.190730189239	-358.190730189239\\
71	0.15522	-382.379665274633	-382.379665274633\\
71	0.15888	-407.788218992312	-407.788218992312\\
71	0.16254	-434.416391342274	-434.416391342274\\
71	0.1662	-462.264182324521	-462.264182324521\\
71	0.16986	-491.331591939051	-491.331591939051\\
71	0.17352	-521.618620185865	-521.618620185865\\
71	0.17718	-553.125267064964	-553.125267064964\\
71	0.18084	-585.851532576346	-585.851532576346\\
71	0.1845	-619.797416720014	-619.797416720014\\
71	0.18816	-654.962919495964	-654.962919495964\\
71	0.19182	-691.348040904199	-691.348040904199\\
71	0.19548	-728.952780944717	-728.952780944717\\
71	0.19914	-767.777139617521	-767.777139617521\\
71	0.2028	-807.821116922607	-807.821116922607\\
71	0.20646	-849.084712859978	-849.084712859978\\
71	0.21012	-891.567927429633	-891.567927429633\\
71	0.21378	-935.270760631571	-935.270760631571\\
71	0.21744	-980.193212465794	-980.193212465794\\
71	0.2211	-1026.3352829323	-1026.3352829323\\
71	0.22476	-1073.69697203109	-1073.69697203109\\
71	0.22842	-1122.27827976217	-1122.27827976217\\
71	0.23208	-1172.07920612553	-1172.07920612553\\
71	0.23574	-1223.09975112117	-1223.09975112117\\
71	0.2394	-1275.3399147491	-1275.3399147491\\
71	0.24306	-1328.79969700931	-1328.79969700931\\
71	0.24672	-1383.4790979018	-1383.4790979018\\
71	0.25038	-1439.37811742658	-1439.37811742658\\
71	0.25404	-1496.49675558365	-1496.49675558365\\
71	0.2577	-1554.83501237299	-1554.83501237299\\
71	0.26136	-1614.39288779462	-1614.39288779462\\
71	0.26502	-1675.17038184854	-1675.17038184854\\
71	0.26868	-1737.16749453474	-1737.16749453474\\
71	0.27234	-1800.38422585322	-1800.38422585322\\
71	0.276	-1864.82057580399	-1864.82057580399\\
71.375	0.093	-140.411085855649	-140.411085855649\\
71.375	0.09666	-145.115624628718	-145.115624628718\\
71.375	0.10032	-151.03978203407	-151.03978203407\\
71.375	0.10398	-158.183558071707	-158.183558071707\\
71.375	0.10764	-166.546952741628	-166.546952741628\\
71.375	0.1113	-176.129966043832	-176.129966043832\\
71.375	0.11496	-186.932597978321	-186.932597978321\\
71.375	0.11862	-198.954848545094	-198.954848545094\\
71.375	0.12228	-212.19671774415	-212.19671774415\\
71.375	0.12594	-226.658205575492	-226.658205575492\\
71.375	0.1296	-242.339312039116	-242.339312039116\\
71.375	0.13326	-259.240037135025	-259.240037135025\\
71.375	0.13692	-277.360380863218	-277.360380863218\\
71.375	0.14058	-296.700343223695	-296.700343223695\\
71.375	0.14424	-317.259924216456	-317.259924216456\\
71.375	0.1479	-339.039123841501	-339.039123841501\\
71.375	0.15156	-362.03794209883	-362.03794209883\\
71.375	0.15522	-386.256378988444	-386.256378988444\\
71.375	0.15888	-411.69443451034	-411.69443451034\\
71.375	0.16254	-438.352108664522	-438.352108664522\\
71.375	0.1662	-466.229401450987	-466.229401450987\\
71.375	0.16986	-495.326312869737	-495.326312869737\\
71.375	0.17352	-525.642842920769	-525.642842920769\\
71.375	0.17718	-557.178991604087	-557.178991604087\\
71.375	0.18084	-589.934758919688	-589.934758919688\\
71.375	0.1845	-623.910144867573	-623.910144867573\\
71.375	0.18816	-659.105149447743	-659.105149447743\\
71.375	0.19182	-695.519772660196	-695.519772660196\\
71.375	0.19548	-733.154014504933	-733.154014504933\\
71.375	0.19914	-772.007874981955	-772.007874981955\\
71.375	0.2028	-812.081354091261	-812.081354091261\\
71.375	0.20646	-853.37445183285	-853.37445183285\\
71.375	0.21012	-895.887168206724	-895.887168206724\\
71.375	0.21378	-939.619503212881	-939.619503212881\\
71.375	0.21744	-984.571456851323	-984.571456851323\\
71.375	0.2211	-1030.74302912205	-1030.74302912205\\
71.375	0.22476	-1078.13422002506	-1078.13422002506\\
71.375	0.22842	-1126.74502956035	-1126.74502956035\\
71.375	0.23208	-1176.57545772793	-1176.57545772793\\
71.375	0.23574	-1227.62550452779	-1227.62550452779\\
71.375	0.2394	-1279.89516995994	-1279.89516995994\\
71.375	0.24306	-1333.38445402437	-1333.38445402437\\
71.375	0.24672	-1388.09335672108	-1388.09335672108\\
71.375	0.25038	-1444.02187805008	-1444.02187805008\\
71.375	0.25404	-1501.17001801136	-1501.17001801136\\
71.375	0.2577	-1559.53777660493	-1559.53777660493\\
71.375	0.26136	-1619.12515383078	-1619.12515383078\\
71.375	0.26502	-1679.93214968891	-1679.93214968891\\
71.375	0.26868	-1741.95876417933	-1741.95876417933\\
71.375	0.27234	-1805.20499730203	-1805.20499730203\\
71.375	0.276	-1869.67084905702	-1869.67084905702\\
71.75	0.093	-143.924872262414	-143.924872262414\\
71.75	0.09666	-148.658912839702	-148.658912839702\\
71.75	0.10032	-154.612572049273	-154.612572049273\\
71.75	0.10398	-161.785849891128	-161.785849891128\\
71.75	0.10764	-170.178746365268	-170.178746365268\\
71.75	0.1113	-179.791261471691	-179.791261471691\\
71.75	0.11496	-190.623395210398	-190.623395210398\\
71.75	0.11862	-202.67514758139	-202.67514758139\\
71.75	0.12228	-215.946518584665	-215.946518584665\\
71.75	0.12594	-230.437508220225	-230.437508220225\\
71.75	0.1296	-246.148116488069	-246.148116488069\\
71.75	0.13326	-263.078343388196	-263.078343388196\\
71.75	0.13692	-281.228188920608	-281.228188920608\\
71.75	0.14058	-300.597653085304	-300.597653085304\\
71.75	0.14424	-321.186735882283	-321.186735882283\\
71.75	0.1479	-342.995437311547	-342.995437311547\\
71.75	0.15156	-366.023757373094	-366.023757373094\\
71.75	0.15522	-390.271696066927	-390.271696066927\\
71.75	0.15888	-415.739253393043	-415.739253393043\\
71.75	0.16254	-442.426429351442	-442.426429351442\\
71.75	0.1662	-470.333223942127	-470.333223942127\\
71.75	0.16986	-499.459637165095	-499.459637165095\\
71.75	0.17352	-529.805669020346	-529.805669020346\\
71.75	0.17718	-561.371319507882	-561.371319507882\\
71.75	0.18084	-594.156588627702	-594.156588627702\\
71.75	0.1845	-628.161476379807	-628.161476379807\\
71.75	0.18816	-663.385982764194	-663.385982764194\\
71.75	0.19182	-699.830107780866	-699.830107780866\\
71.75	0.19548	-737.493851429823	-737.493851429823\\
71.75	0.19914	-776.377213711063	-776.377213711063\\
71.75	0.2028	-816.480194624588	-816.480194624588\\
71.75	0.20646	-857.802794170396	-857.802794170396\\
71.75	0.21012	-900.345012348488	-900.345012348488\\
71.75	0.21378	-944.106849158863	-944.106849158863\\
71.75	0.21744	-989.088304601524	-989.088304601524\\
71.75	0.2211	-1035.28937867647	-1035.28937867647\\
71.75	0.22476	-1082.7100713837	-1082.7100713837\\
71.75	0.22842	-1131.35038272321	-1131.35038272321\\
71.75	0.23208	-1181.21031269501	-1181.21031269501\\
71.75	0.23574	-1232.28986129909	-1232.28986129909\\
71.75	0.2394	-1284.58902853545	-1284.58902853545\\
71.75	0.24306	-1338.1078144041	-1338.1078144041\\
71.75	0.24672	-1392.84621890503	-1392.84621890503\\
71.75	0.25038	-1448.80424203825	-1448.80424203825\\
71.75	0.25404	-1505.98188380375	-1505.98188380375\\
71.75	0.2577	-1564.37914420153	-1564.37914420153\\
71.75	0.26136	-1623.9960232316	-1623.9960232316\\
71.75	0.26502	-1684.83252089396	-1684.83252089396\\
71.75	0.26868	-1746.88863718859	-1746.88863718859\\
71.75	0.27234	-1810.16437211551	-1810.16437211551\\
71.75	0.276	-1874.65972567472	-1874.65972567472\\
72.125	0.093	-147.577262033853	-147.577262033853\\
72.125	0.09666	-152.340804415359	-152.340804415359\\
72.125	0.10032	-158.323965429149	-158.323965429149\\
72.125	0.10398	-165.526745075223	-165.526745075223\\
72.125	0.10764	-173.949143353581	-173.949143353581\\
72.125	0.1113	-183.591160264224	-183.591160264224\\
72.125	0.11496	-194.45279580715	-194.45279580715\\
72.125	0.11862	-206.53404998236	-206.53404998236\\
72.125	0.12228	-219.834922789854	-219.834922789854\\
72.125	0.12594	-234.355414229633	-234.355414229633\\
72.125	0.1296	-250.095524301694	-250.095524301694\\
72.125	0.13326	-267.055253006041	-267.055253006041\\
72.125	0.13692	-285.234600342671	-285.234600342671\\
72.125	0.14058	-304.633566311586	-304.633566311586\\
72.125	0.14424	-325.252150912784	-325.252150912784\\
72.125	0.1479	-347.090354146267	-347.090354146267\\
72.125	0.15156	-370.148176012033	-370.148176012033\\
72.125	0.15522	-394.425616510084	-394.425616510084\\
72.125	0.15888	-419.922675640418	-419.922675640418\\
72.125	0.16254	-446.639353403037	-446.639353403037\\
72.125	0.1662	-474.57564979794	-474.57564979794\\
72.125	0.16986	-503.731564825127	-503.731564825127\\
72.125	0.17352	-534.107098484597	-534.107098484597\\
72.125	0.17718	-565.702250776352	-565.702250776352\\
72.125	0.18084	-598.51702170039	-598.51702170039\\
72.125	0.1845	-632.551411256713	-632.551411256713\\
72.125	0.18816	-667.80541944532	-667.80541944532\\
72.125	0.19182	-704.279046266211	-704.279046266211\\
72.125	0.19548	-741.972291719386	-741.972291719386\\
72.125	0.19914	-780.885155804845	-780.885155804845\\
72.125	0.2028	-821.017638522588	-821.017638522588\\
72.125	0.20646	-862.369739872615	-862.369739872615\\
72.125	0.21012	-904.941459854926	-904.941459854926\\
72.125	0.21378	-948.73279846952	-948.73279846952\\
72.125	0.21744	-993.743755716399	-993.743755716399\\
72.125	0.2211	-1039.97433159556	-1039.97433159556\\
72.125	0.22476	-1087.42452610701	-1087.42452610701\\
72.125	0.22842	-1136.09433925074	-1136.09433925074\\
72.125	0.23208	-1185.98377102676	-1185.98377102676\\
72.125	0.23574	-1237.09282143506	-1237.09282143506\\
72.125	0.2394	-1289.42149047564	-1289.42149047564\\
72.125	0.24306	-1342.96977814851	-1342.96977814851\\
72.125	0.24672	-1397.73768445366	-1397.73768445366\\
72.125	0.25038	-1453.72520939109	-1453.72520939109\\
72.125	0.25404	-1510.93235296081	-1510.93235296081\\
72.125	0.2577	-1569.35911516282	-1569.35911516282\\
72.125	0.26136	-1629.0054959971	-1629.0054959971\\
72.125	0.26502	-1689.87149546367	-1689.87149546367\\
72.125	0.26868	-1751.95711356253	-1751.95711356253\\
72.125	0.27234	-1815.26235029367	-1815.26235029367\\
72.125	0.276	-1879.78720565709	-1879.78720565709\\
72.5	0.093	-151.368255169965	-151.368255169965\\
72.5	0.09666	-156.16129935569	-156.16129935569\\
72.5	0.10032	-162.173962173698	-162.173962173698\\
72.5	0.10398	-169.406243623991	-169.406243623991\\
72.5	0.10764	-177.858143706568	-177.858143706568\\
72.5	0.1113	-187.529662421429	-187.529662421429\\
72.5	0.11496	-198.420799768574	-198.420799768574\\
72.5	0.11862	-210.531555748003	-210.531555748003\\
72.5	0.12228	-223.861930359715	-223.861930359715\\
72.5	0.12594	-238.411923603713	-238.411923603713\\
72.5	0.1296	-254.181535479994	-254.181535479994\\
72.5	0.13326	-271.170765988559	-271.170765988559\\
72.5	0.13692	-289.379615129408	-289.379615129408\\
72.5	0.14058	-308.808082902541	-308.808082902541\\
72.5	0.14424	-329.456169307958	-329.456169307958\\
72.5	0.1479	-351.323874345659	-351.323874345659\\
72.5	0.15156	-374.411198015644	-374.411198015644\\
72.5	0.15522	-398.718140317914	-398.718140317914\\
72.5	0.15888	-424.244701252467	-424.244701252467\\
72.5	0.16254	-450.990880819304	-450.990880819304\\
72.5	0.1662	-478.956679018426	-478.956679018426\\
72.5	0.16986	-508.142095849831	-508.142095849831\\
72.5	0.17352	-538.54713131352	-538.54713131352\\
72.5	0.17718	-570.171785409494	-570.171785409494\\
72.5	0.18084	-603.016058137751	-603.016058137751\\
72.5	0.1845	-637.079949498293	-637.079949498293\\
72.5	0.18816	-672.363459491118	-672.363459491118\\
72.5	0.19182	-708.866588116227	-708.866588116227\\
72.5	0.19548	-746.589335373621	-746.589335373621\\
72.5	0.19914	-785.5317012633	-785.5317012633\\
72.5	0.2028	-825.693685785261	-825.693685785261\\
72.5	0.20646	-867.075288939507	-867.075288939507\\
72.5	0.21012	-909.676510726036	-909.676510726036\\
72.5	0.21378	-953.49735114485	-953.49735114485\\
72.5	0.21744	-998.537810195948	-998.537810195948\\
72.5	0.2211	-1044.79788787933	-1044.79788787933\\
72.5	0.22476	-1092.277584195	-1092.277584195\\
72.5	0.22842	-1140.97689914294	-1140.97689914294\\
72.5	0.23208	-1190.89583272318	-1190.89583272318\\
72.5	0.23574	-1242.0343849357	-1242.0343849357\\
72.5	0.2394	-1294.3925557805	-1294.3925557805\\
72.5	0.24306	-1347.97034525758	-1347.97034525758\\
72.5	0.24672	-1402.76775336695	-1402.76775336695\\
72.5	0.25038	-1458.78478010861	-1458.78478010861\\
72.5	0.25404	-1516.02142548255	-1516.02142548255\\
72.5	0.2577	-1574.47768948877	-1574.47768948877\\
72.5	0.26136	-1634.15357212728	-1634.15357212728\\
72.5	0.26502	-1695.04907339807	-1695.04907339807\\
72.5	0.26868	-1757.16419330114	-1757.16419330114\\
72.5	0.27234	-1820.4989318365	-1820.4989318365\\
72.5	0.276	-1885.05328900414	-1885.05328900414\\
72.875	0.093	-155.297851670751	-155.297851670751\\
72.875	0.09666	-160.120397660694	-160.120397660694\\
72.875	0.10032	-166.162562282921	-166.162562282921\\
72.875	0.10398	-173.424345537433	-173.424345537433\\
72.875	0.10764	-181.905747424229	-181.905747424229\\
72.875	0.1113	-191.606767943309	-191.606767943309\\
72.875	0.11496	-202.527407094672	-202.527407094672\\
72.875	0.11862	-214.66766487832	-214.66766487832\\
72.875	0.12228	-228.027541294251	-228.027541294251\\
72.875	0.12594	-242.607036342467	-242.607036342467\\
72.875	0.1296	-258.406150022967	-258.406150022967\\
72.875	0.13326	-275.42488233575	-275.42488233575\\
72.875	0.13692	-293.663233280818	-293.663233280818\\
72.875	0.14058	-313.12120285817	-313.12120285817\\
72.875	0.14424	-333.798791067806	-333.798791067806\\
72.875	0.1479	-355.695997909726	-355.695997909726\\
72.875	0.15156	-378.81282338393	-378.81282338393\\
72.875	0.15522	-403.149267490418	-403.149267490418\\
72.875	0.15888	-428.70533022919	-428.70533022919\\
72.875	0.16254	-455.481011600246	-455.481011600246\\
72.875	0.1662	-483.476311603586	-483.476311603586\\
72.875	0.16986	-512.69123023921	-512.69123023921\\
72.875	0.17352	-543.125767507118	-543.125767507118\\
72.875	0.17718	-574.77992340731	-574.77992340731\\
72.875	0.18084	-607.653697939786	-607.653697939786\\
72.875	0.1845	-641.747091104547	-641.747091104547\\
72.875	0.18816	-677.060102901591	-677.060102901591\\
72.875	0.19182	-713.592733330919	-713.592733330919\\
72.875	0.19548	-751.344982392531	-751.344982392531\\
72.875	0.19914	-790.316850086428	-790.316850086428\\
72.875	0.2028	-830.508336412609	-830.508336412609\\
72.875	0.20646	-871.919441371073	-871.919441371073\\
72.875	0.21012	-914.550164961821	-914.550164961821\\
72.875	0.21378	-958.400507184853	-958.400507184853\\
72.875	0.21744	-1003.47046804017	-1003.47046804017\\
72.875	0.2211	-1049.76004752777	-1049.76004752777\\
72.875	0.22476	-1097.26924564765	-1097.26924564765\\
72.875	0.22842	-1145.99806239982	-1145.99806239982\\
72.875	0.23208	-1195.94649778428	-1195.94649778428\\
72.875	0.23574	-1247.11455180101	-1247.11455180101\\
72.875	0.2394	-1299.50222445003	-1299.50222445003\\
72.875	0.24306	-1353.10951573134	-1353.10951573134\\
72.875	0.24672	-1407.93642564493	-1407.93642564493\\
72.875	0.25038	-1463.9829541908	-1463.9829541908\\
72.875	0.25404	-1521.24910136896	-1521.24910136896\\
72.875	0.2577	-1579.7348671794	-1579.7348671794\\
72.875	0.26136	-1639.44025162212	-1639.44025162212\\
72.875	0.26502	-1700.36525469713	-1700.36525469713\\
72.875	0.26868	-1762.50987640442	-1762.50987640442\\
72.875	0.27234	-1825.874116744	-1825.874116744\\
72.875	0.276	-1890.45797571586	-1890.45797571586\\
73.25	0.093	-159.36605153621	-159.36605153621\\
73.25	0.09666	-164.218099330372	-164.218099330372\\
73.25	0.10032	-170.289765756818	-170.289765756818\\
73.25	0.10398	-177.581050815548	-177.581050815548\\
73.25	0.10764	-186.091954506562	-186.091954506562\\
73.25	0.1113	-195.822476829861	-195.822476829861\\
73.25	0.11496	-206.772617785443	-206.772617785443\\
73.25	0.11862	-218.942377373309	-218.942377373309\\
73.25	0.12228	-232.331755593459	-232.331755593459\\
73.25	0.12594	-246.940752445895	-246.940752445895\\
73.25	0.1296	-262.769367930613	-262.769367930613\\
73.25	0.13326	-279.817602047615	-279.817602047615\\
73.25	0.13692	-298.085454796902	-298.085454796902\\
73.25	0.14058	-317.572926178472	-317.572926178472\\
73.25	0.14424	-338.280016192326	-338.280016192326\\
73.25	0.1479	-360.206724838465	-360.206724838465\\
73.25	0.15156	-383.353052116888	-383.353052116888\\
73.25	0.15522	-407.718998027595	-407.718998027595\\
73.25	0.15888	-433.304562570585	-433.304562570585\\
73.25	0.16254	-460.10974574586	-460.10974574586\\
73.25	0.1662	-488.134547553419	-488.134547553419\\
73.25	0.16986	-517.378967993262	-517.378967993262\\
73.25	0.17352	-547.843007065388	-547.843007065388\\
73.25	0.17718	-579.526664769799	-579.526664769799\\
73.25	0.18084	-612.429941106494	-612.429941106494\\
73.25	0.1845	-646.552836075473	-646.552836075473\\
73.25	0.18816	-681.895349676736	-681.895349676736\\
73.25	0.19182	-718.457481910283	-718.457481910283\\
73.25	0.19548	-756.239232776114	-756.239232776114\\
73.25	0.19914	-795.24060227423	-795.24060227423\\
73.25	0.2028	-835.461590404629	-835.461590404629\\
73.25	0.20646	-876.902197167312	-876.902197167312\\
73.25	0.21012	-919.562422562279	-919.562422562279\\
73.25	0.21378	-963.44226658953	-963.44226658953\\
73.25	0.21744	-1008.54172924906	-1008.54172924906\\
73.25	0.2211	-1054.86081054088	-1054.86081054088\\
73.25	0.22476	-1102.39951046499	-1102.39951046499\\
73.25	0.22842	-1151.15782902137	-1151.15782902137\\
73.25	0.23208	-1201.13576621005	-1201.13576621005\\
73.25	0.23574	-1252.333322031	-1252.333322031\\
73.25	0.2394	-1304.75049648424	-1304.75049648424\\
73.25	0.24306	-1358.38728956977	-1358.38728956977\\
73.25	0.24672	-1413.24370128757	-1413.24370128757\\
73.25	0.25038	-1469.31973163766	-1469.31973163766\\
73.25	0.25404	-1526.61538062004	-1526.61538062004\\
73.25	0.2577	-1585.1306482347	-1585.1306482347\\
73.25	0.26136	-1644.86553448164	-1644.86553448164\\
73.25	0.26502	-1705.82003936087	-1705.82003936087\\
73.25	0.26868	-1767.99416287238	-1767.99416287238\\
73.25	0.27234	-1831.38790501618	-1831.38790501618\\
73.25	0.276	-1896.00126579226	-1896.00126579226\\
73.625	0.093	-163.572854766341	-163.572854766341\\
73.625	0.09666	-168.454404364722	-168.454404364722\\
73.625	0.10032	-174.555572595387	-174.555572595387\\
73.625	0.10398	-181.876359458336	-181.876359458336\\
73.625	0.10764	-190.416764953569	-190.416764953569\\
73.625	0.1113	-200.176789081086	-200.176789081086\\
73.625	0.11496	-211.156431840887	-211.156431840887\\
73.625	0.11862	-223.355693232972	-223.355693232972\\
73.625	0.12228	-236.774573257341	-236.774573257341\\
73.625	0.12594	-251.413071913995	-251.413071913995\\
73.625	0.1296	-267.271189202932	-267.271189202932\\
73.625	0.13326	-284.348925124152	-284.348925124152\\
73.625	0.13692	-302.646279677658	-302.646279677658\\
73.625	0.14058	-322.163252863447	-322.163252863447\\
73.625	0.14424	-342.89984468152	-342.89984468152\\
73.625	0.1479	-364.856055131878	-364.856055131878\\
73.625	0.15156	-388.031884214519	-388.031884214519\\
73.625	0.15522	-412.427331929445	-412.427331929445\\
73.625	0.15888	-438.042398276654	-438.042398276654\\
73.625	0.16254	-464.877083256147	-464.877083256147\\
73.625	0.1662	-492.931386867925	-492.931386867925\\
73.625	0.16986	-522.205309111987	-522.205309111987\\
73.625	0.17352	-552.698849988332	-552.698849988332\\
73.625	0.17718	-584.412009496962	-584.412009496962\\
73.625	0.18084	-617.344787637875	-617.344787637875\\
73.625	0.1845	-651.497184411073	-651.497184411073\\
73.625	0.18816	-686.869199816554	-686.869199816554\\
73.625	0.19182	-723.46083385432	-723.46083385432\\
73.625	0.19548	-761.27208652437	-761.27208652437\\
73.625	0.19914	-800.302957826704	-800.302957826704\\
73.625	0.2028	-840.553447761322	-840.553447761322\\
73.625	0.20646	-882.023556328224	-882.023556328224\\
73.625	0.21012	-924.713283527409	-924.713283527409\\
73.625	0.21378	-968.622629358879	-968.622629358879\\
73.625	0.21744	-1013.75159382263	-1013.75159382263\\
73.625	0.2211	-1060.10017691867	-1060.10017691867\\
73.625	0.22476	-1107.66837864699	-1107.66837864699\\
73.625	0.22842	-1156.4561990076	-1156.4561990076\\
73.625	0.23208	-1206.46363800049	-1206.46363800049\\
73.625	0.23574	-1257.69069562566	-1257.69069562566\\
73.625	0.2394	-1310.13737188312	-1310.13737188312\\
73.625	0.24306	-1363.80366677286	-1363.80366677286\\
73.625	0.24672	-1418.68958029489	-1418.68958029489\\
73.625	0.25038	-1474.7951124492	-1474.7951124492\\
73.625	0.25404	-1532.12026323579	-1532.12026323579\\
73.625	0.2577	-1590.66503265467	-1590.66503265467\\
73.625	0.26136	-1650.42942070583	-1650.42942070583\\
73.625	0.26502	-1711.41342738928	-1711.41342738928\\
73.625	0.26868	-1773.61705270501	-1773.61705270501\\
73.625	0.27234	-1837.04029665303	-1837.04029665303\\
73.625	0.276	-1901.68315923332	-1901.68315923332\\
74	0.093	-167.918261361147	-167.918261361147\\
74	0.09666	-172.829312763746	-172.829312763746\\
74	0.10032	-178.95998279863	-178.95998279863\\
74	0.10398	-186.310271465797	-186.310271465797\\
74	0.10764	-194.880178765249	-194.880178765249\\
74	0.1113	-204.669704696985	-204.669704696985\\
74	0.11496	-215.678849261005	-215.678849261005\\
74	0.11862	-227.907612457309	-227.907612457309\\
74	0.12228	-241.355994285896	-241.355994285896\\
74	0.12594	-256.023994746768	-256.023994746768\\
74	0.1296	-271.911613839924	-271.911613839924\\
74	0.13326	-289.018851565364	-289.018851565364\\
74	0.13692	-307.345707923088	-307.345707923088\\
74	0.14058	-326.892182913096	-326.892182913096\\
74	0.14424	-347.658276535388	-347.658276535388\\
74	0.1479	-369.643988789964	-369.643988789964\\
74	0.15156	-392.849319676824	-392.849319676824\\
74	0.15522	-417.274269195968	-417.274269195968\\
74	0.15888	-442.918837347396	-442.918837347396\\
74	0.16254	-469.783024131108	-469.783024131108\\
74	0.1662	-497.866829547105	-497.866829547105\\
74	0.16986	-527.170253595385	-527.170253595385\\
74	0.17352	-557.693296275949	-557.693296275949\\
74	0.17718	-589.435957588797	-589.435957588797\\
74	0.18084	-622.398237533929	-622.398237533929\\
74	0.1845	-656.580136111346	-656.580136111346\\
74	0.18816	-691.981653321047	-691.981653321047\\
74	0.19182	-728.602789163031	-728.602789163031\\
74	0.19548	-766.443543637299	-766.443543637299\\
74	0.19914	-805.503916743852	-805.503916743852\\
74	0.2028	-845.783908482689	-845.783908482689\\
74	0.20646	-887.283518853809	-887.283518853809\\
74	0.21012	-930.002747857214	-930.002747857214\\
74	0.21378	-973.941595492902	-973.941595492902\\
74	0.21744	-1019.10006176087	-1019.10006176087\\
74	0.2211	-1065.47814666113	-1065.47814666113\\
74	0.22476	-1113.07585019367	-1113.07585019367\\
74	0.22842	-1161.8931723585	-1161.8931723585\\
74	0.23208	-1211.9301131556	-1211.9301131556\\
74	0.23574	-1263.186672585	-1263.186672585\\
74	0.2394	-1315.66285064668	-1315.66285064668\\
74	0.24306	-1369.35864734064	-1369.35864734064\\
74	0.24672	-1424.27406266688	-1424.27406266688\\
74	0.25038	-1480.40909662541	-1480.40909662541\\
74	0.25404	-1537.76374921622	-1537.76374921622\\
74	0.2577	-1596.33802043932	-1596.33802043932\\
74	0.26136	-1656.1319102947	-1656.1319102947\\
74	0.26502	-1717.14541878237	-1717.14541878237\\
74	0.26868	-1779.37854590232	-1779.37854590232\\
74	0.27234	-1842.83129165455	-1842.83129165455\\
74	0.276	-1907.50365603906	-1907.50365603906\\
};
\end{axis}

\begin{axis}[%
width=6.159cm,
height=3.097cm,
at={(8.104cm,0cm)},
scale only axis,
xmin=56,
xmax=74,
tick align=outside,
xlabel style={font=\color{white!15!black}},
xlabel={$L_{cut}$},
ymin=0.093,
ymax=0.276,
ylabel style={font=\color{white!15!black}},
ylabel={$D_{rlx}$},
zmin=0,
zmax=113.453202447659,
zlabel style={font=\color{white!15!black}},
zlabel={$u(t-1)u(t)$},
view={-140}{50},
axis background/.style={fill=white},
xmajorgrids,
ymajorgrids,
zmajorgrids
]
\addplot3[only marks, mark=*, mark options={}, mark size=1.5000pt, color=mycolor1, fill=mycolor1] table[row sep=crcr]{%
x	y	z\\
74	0.123	15.4680684467597\\
72	0.113	12.0981268654473\\
61	0.095	6.35969255839677\\
56	0.093	5.67589105085895\\
};
\addplot3[only marks, mark=*, mark options={}, mark size=1.5000pt, color=mycolor2, fill=mycolor2] table[row sep=crcr]{%
x	y	z\\
67	0.276	110.131370256376\\
66	0.255	90.4271696447197\\
62	0.209	51.7720891508705\\
57	0.193	41.4264697981385\\
};
\addplot3[only marks, mark=*, mark options={}, mark size=1.5000pt, color=black, fill=black] table[row sep=crcr]{%
x	y	z\\
69	0.104	9.34303245175057\\
};
\addplot3[only marks, mark=*, mark options={}, mark size=1.5000pt, color=black, fill=black] table[row sep=crcr]{%
x	y	z\\
64	0.23	68.0446101221447\\
};

\addplot3[%
surf,
fill opacity=0.7, shader=interp, colormap={mymap}{[1pt] rgb(0pt)=(1,0.905882,0); rgb(1pt)=(1,0.901964,0); rgb(2pt)=(1,0.898051,0); rgb(3pt)=(1,0.894144,0); rgb(4pt)=(1,0.890243,0); rgb(5pt)=(1,0.886349,0); rgb(6pt)=(1,0.88246,0); rgb(7pt)=(1,0.878577,0); rgb(8pt)=(1,0.8747,0); rgb(9pt)=(1,0.870829,0); rgb(10pt)=(1,0.866964,0); rgb(11pt)=(1,0.863106,0); rgb(12pt)=(1,0.859253,0); rgb(13pt)=(1,0.855406,0); rgb(14pt)=(1,0.851566,0); rgb(15pt)=(1,0.847732,0); rgb(16pt)=(1,0.843903,0); rgb(17pt)=(1,0.840081,0); rgb(18pt)=(1,0.836265,0); rgb(19pt)=(1,0.832455,0); rgb(20pt)=(1,0.828652,0); rgb(21pt)=(1,0.824854,0); rgb(22pt)=(1,0.821063,0); rgb(23pt)=(1,0.817278,0); rgb(24pt)=(1,0.8135,0); rgb(25pt)=(1,0.809727,0); rgb(26pt)=(1,0.805961,0); rgb(27pt)=(1,0.8022,0); rgb(28pt)=(1,0.798445,0); rgb(29pt)=(1,0.794696,0); rgb(30pt)=(1,0.790953,0); rgb(31pt)=(1,0.787215,0); rgb(32pt)=(1,0.783484,0); rgb(33pt)=(1,0.779758,0); rgb(34pt)=(1,0.776038,0); rgb(35pt)=(1,0.772324,0); rgb(36pt)=(1,0.768615,0); rgb(37pt)=(1,0.764913,0); rgb(38pt)=(1,0.761217,0); rgb(39pt)=(1,0.757527,0); rgb(40pt)=(1,0.753843,0); rgb(41pt)=(1,0.750165,0); rgb(42pt)=(1,0.746493,0); rgb(43pt)=(1,0.742827,0); rgb(44pt)=(1,0.739167,0); rgb(45pt)=(1,0.735514,0); rgb(46pt)=(1,0.731867,0); rgb(47pt)=(1,0.728226,0); rgb(48pt)=(1,0.724591,0); rgb(49pt)=(1,0.720963,0); rgb(50pt)=(1,0.717341,0); rgb(51pt)=(1,0.713725,0); rgb(52pt)=(0.999994,0.710077,0); rgb(53pt)=(0.999974,0.706363,0); rgb(54pt)=(0.999942,0.702592,0); rgb(55pt)=(0.999898,0.698775,0); rgb(56pt)=(0.999841,0.694921,0); rgb(57pt)=(0.999771,0.691039,0); rgb(58pt)=(0.99969,0.687139,0); rgb(59pt)=(0.999596,0.68323,0); rgb(60pt)=(0.99949,0.679323,0); rgb(61pt)=(0.999372,0.675427,0); rgb(62pt)=(0.999242,0.67155,0); rgb(63pt)=(0.9991,0.667704,0); rgb(64pt)=(0.998946,0.663897,0); rgb(65pt)=(0.998781,0.660138,0); rgb(66pt)=(0.998605,0.656439,0); rgb(67pt)=(0.998416,0.652807,0); rgb(68pt)=(0.998217,0.649253,0); rgb(69pt)=(0.998006,0.645786,0); rgb(70pt)=(0.997785,0.642416,0); rgb(71pt)=(0.997552,0.639152,0); rgb(72pt)=(0.997308,0.636004,0); rgb(73pt)=(0.997053,0.632982,0); rgb(74pt)=(0.996788,0.630095,0); rgb(75pt)=(0.996512,0.627352,0); rgb(76pt)=(0.996226,0.624763,0); rgb(77pt)=(0.995851,0.622329,0); rgb(78pt)=(0.99494,0.619997,0); rgb(79pt)=(0.99345,0.617753,0); rgb(80pt)=(0.991419,0.61559,0); rgb(81pt)=(0.988885,0.613503,0); rgb(82pt)=(0.985886,0.611486,0); rgb(83pt)=(0.98246,0.609532,0); rgb(84pt)=(0.978643,0.607636,0); rgb(85pt)=(0.974475,0.605791,0); rgb(86pt)=(0.969992,0.603992,0); rgb(87pt)=(0.965232,0.602233,0); rgb(88pt)=(0.960233,0.600507,0); rgb(89pt)=(0.955033,0.598808,0); rgb(90pt)=(0.949669,0.59713,0); rgb(91pt)=(0.94418,0.595468,0); rgb(92pt)=(0.938602,0.593815,0); rgb(93pt)=(0.932974,0.592166,0); rgb(94pt)=(0.927333,0.590513,0); rgb(95pt)=(0.921717,0.588852,0); rgb(96pt)=(0.916164,0.587176,0); rgb(97pt)=(0.910711,0.585479,0); rgb(98pt)=(0.905397,0.583755,0); rgb(99pt)=(0.900258,0.581999,0); rgb(100pt)=(0.895333,0.580203,0); rgb(101pt)=(0.890659,0.578362,0); rgb(102pt)=(0.886275,0.576471,0); rgb(103pt)=(0.882047,0.574545,0); rgb(104pt)=(0.877819,0.572608,0); rgb(105pt)=(0.873592,0.57066,0); rgb(106pt)=(0.869366,0.568701,0); rgb(107pt)=(0.865143,0.566733,0); rgb(108pt)=(0.860924,0.564756,0); rgb(109pt)=(0.856708,0.562771,0); rgb(110pt)=(0.852497,0.560778,0); rgb(111pt)=(0.848292,0.558779,0); rgb(112pt)=(0.844092,0.556774,0); rgb(113pt)=(0.8399,0.554763,0); rgb(114pt)=(0.835716,0.552749,0); rgb(115pt)=(0.831541,0.55073,0); rgb(116pt)=(0.827374,0.548709,0); rgb(117pt)=(0.823219,0.546686,0); rgb(118pt)=(0.819074,0.54466,0); rgb(119pt)=(0.81494,0.542635,0); rgb(120pt)=(0.81082,0.540609,0); rgb(121pt)=(0.806712,0.538584,0); rgb(122pt)=(0.802619,0.53656,0); rgb(123pt)=(0.798541,0.534539,0); rgb(124pt)=(0.794478,0.532521,0); rgb(125pt)=(0.790431,0.530506,0); rgb(126pt)=(0.786402,0.528496,0); rgb(127pt)=(0.782391,0.526491,0); rgb(128pt)=(0.77841,0.524489,0); rgb(129pt)=(0.774523,0.522478,0); rgb(130pt)=(0.770731,0.520455,0); rgb(131pt)=(0.767022,0.518424,0); rgb(132pt)=(0.763384,0.516385,0); rgb(133pt)=(0.759804,0.514339,0); rgb(134pt)=(0.756272,0.51229,0); rgb(135pt)=(0.752775,0.510237,0); rgb(136pt)=(0.749302,0.508182,0); rgb(137pt)=(0.74584,0.506128,0); rgb(138pt)=(0.742378,0.504075,0); rgb(139pt)=(0.738904,0.502025,0); rgb(140pt)=(0.735406,0.499979,0); rgb(141pt)=(0.731872,0.49794,0); rgb(142pt)=(0.72829,0.495909,0); rgb(143pt)=(0.724649,0.493887,0); rgb(144pt)=(0.720936,0.491875,0); rgb(145pt)=(0.71714,0.489876,0); rgb(146pt)=(0.713249,0.487891,0); rgb(147pt)=(0.709251,0.485921,0); rgb(148pt)=(0.705134,0.483968,0); rgb(149pt)=(0.700887,0.482033,0); rgb(150pt)=(0.696497,0.480118,0); rgb(151pt)=(0.691952,0.478225,0); rgb(152pt)=(0.687242,0.476355,0); rgb(153pt)=(0.682353,0.47451,0); rgb(154pt)=(0.677195,0.472696,0); rgb(155pt)=(0.6717,0.470916,0); rgb(156pt)=(0.665891,0.469169,0); rgb(157pt)=(0.659791,0.46745,0); rgb(158pt)=(0.653423,0.465756,0); rgb(159pt)=(0.64681,0.464084,0); rgb(160pt)=(0.639976,0.462432,0); rgb(161pt)=(0.632943,0.460795,0); rgb(162pt)=(0.625734,0.459171,0); rgb(163pt)=(0.618373,0.457556,0); rgb(164pt)=(0.610882,0.455948,0); rgb(165pt)=(0.603284,0.454343,0); rgb(166pt)=(0.595604,0.452737,0); rgb(167pt)=(0.587863,0.451129,0); rgb(168pt)=(0.580084,0.449514,0); rgb(169pt)=(0.572292,0.447889,0); rgb(170pt)=(0.564508,0.446252,0); rgb(171pt)=(0.556756,0.444599,0); rgb(172pt)=(0.549059,0.442927,0); rgb(173pt)=(0.54144,0.441232,0); rgb(174pt)=(0.533922,0.439512,0); rgb(175pt)=(0.526529,0.437764,0); rgb(176pt)=(0.519282,0.435983,0); rgb(177pt)=(0.512206,0.434168,0); rgb(178pt)=(0.505323,0.432315,0); rgb(179pt)=(0.498628,0.430422,3.92506e-06); rgb(180pt)=(0.491973,0.428504,3.49981e-05); rgb(181pt)=(0.485331,0.426562,9.63073e-05); rgb(182pt)=(0.478704,0.424596,0.000186979); rgb(183pt)=(0.472096,0.422609,0.000306141); rgb(184pt)=(0.465508,0.420599,0.00045292); rgb(185pt)=(0.458942,0.418567,0.000626441); rgb(186pt)=(0.452401,0.416515,0.000825833); rgb(187pt)=(0.445885,0.414441,0.00105022); rgb(188pt)=(0.439399,0.412348,0.00129873); rgb(189pt)=(0.432942,0.410234,0.00157049); rgb(190pt)=(0.426518,0.408102,0.00186463); rgb(191pt)=(0.420129,0.40595,0.00218028); rgb(192pt)=(0.413777,0.40378,0.00251655); rgb(193pt)=(0.407464,0.401592,0.00287258); rgb(194pt)=(0.401191,0.399386,0.00324749); rgb(195pt)=(0.394962,0.397164,0.00364042); rgb(196pt)=(0.388777,0.394925,0.00405048); rgb(197pt)=(0.38264,0.39267,0.00447681); rgb(198pt)=(0.376552,0.390399,0.00491852); rgb(199pt)=(0.370516,0.388113,0.00537476); rgb(200pt)=(0.364532,0.385812,0.00584464); rgb(201pt)=(0.358605,0.383497,0.00632729); rgb(202pt)=(0.352735,0.381168,0.00682184); rgb(203pt)=(0.346925,0.378826,0.00732741); rgb(204pt)=(0.341176,0.376471,0.00784314); rgb(205pt)=(0.335485,0.374093,0.00847245); rgb(206pt)=(0.329843,0.371682,0.00930909); rgb(207pt)=(0.324249,0.369242,0.0103377); rgb(208pt)=(0.318701,0.366772,0.0115428); rgb(209pt)=(0.313198,0.364275,0.0129091); rgb(210pt)=(0.307739,0.361753,0.0144211); rgb(211pt)=(0.302322,0.359206,0.0160634); rgb(212pt)=(0.296945,0.356637,0.0178207); rgb(213pt)=(0.291607,0.354048,0.0196776); rgb(214pt)=(0.286307,0.35144,0.0216186); rgb(215pt)=(0.281043,0.348814,0.0236284); rgb(216pt)=(0.275813,0.346172,0.0256916); rgb(217pt)=(0.270616,0.343517,0.0277927); rgb(218pt)=(0.265451,0.340849,0.0299163); rgb(219pt)=(0.260317,0.33817,0.0320472); rgb(220pt)=(0.25521,0.335482,0.0341698); rgb(221pt)=(0.250131,0.332786,0.0362688); rgb(222pt)=(0.245078,0.330085,0.0383287); rgb(223pt)=(0.240048,0.327379,0.0403343); rgb(224pt)=(0.235042,0.324671,0.04227); rgb(225pt)=(0.230056,0.321962,0.0441205); rgb(226pt)=(0.22509,0.319254,0.0458704); rgb(227pt)=(0.220142,0.316548,0.0475043); rgb(228pt)=(0.215212,0.313846,0.0490067); rgb(229pt)=(0.210296,0.311149,0.0503624); rgb(230pt)=(0.205395,0.308459,0.0515759); rgb(231pt)=(0.200514,0.305763,0.052757); rgb(232pt)=(0.195655,0.303061,0.0539242); rgb(233pt)=(0.190817,0.300353,0.0550763); rgb(234pt)=(0.186001,0.297639,0.0562123); rgb(235pt)=(0.181207,0.294918,0.0573313); rgb(236pt)=(0.176434,0.292191,0.0584321); rgb(237pt)=(0.171685,0.289458,0.0595136); rgb(238pt)=(0.166957,0.286719,0.060575); rgb(239pt)=(0.162252,0.283973,0.0616151); rgb(240pt)=(0.15757,0.281221,0.0626328); rgb(241pt)=(0.152911,0.278463,0.0636271); rgb(242pt)=(0.148275,0.275699,0.0645971); rgb(243pt)=(0.143663,0.272929,0.0655416); rgb(244pt)=(0.139074,0.270152,0.0664596); rgb(245pt)=(0.134508,0.26737,0.06735); rgb(246pt)=(0.129967,0.264581,0.0682118); rgb(247pt)=(0.125449,0.261787,0.0690441); rgb(248pt)=(0.120956,0.258986,0.0698456); rgb(249pt)=(0.116487,0.25618,0.0706154); rgb(250pt)=(0.112043,0.253367,0.0713525); rgb(251pt)=(0.107623,0.250549,0.0720557); rgb(252pt)=(0.103229,0.247724,0.0727241); rgb(253pt)=(0.0988592,0.244894,0.0733566); rgb(254pt)=(0.0945149,0.242058,0.0739522); rgb(255pt)=(0.0901961,0.239216,0.0745098)}, mesh/rows=49]
table[row sep=crcr, point meta=\thisrow{c}] {%
%
x	y	z	c\\
56	0.093	5.74613150884213	5.74613150884213\\
56	0.09666	6.17235692462004	6.17235692462004\\
56	0.10032	6.66497229086033	6.66497229086033\\
56	0.10398	7.223977607563	7.223977607563\\
56	0.10764	7.84937287472805	7.84937287472805\\
56	0.1113	8.54115809235549	8.54115809235549\\
56	0.11496	9.29933326044532	9.29933326044532\\
56	0.11862	10.1238983789975	10.1238983789975\\
56	0.12228	11.0148534480121	11.0148534480121\\
56	0.12594	11.9721984674891	11.9721984674891\\
56	0.1296	12.9959334374284	12.9959334374284\\
56	0.13326	14.0860583578301	14.0860583578301\\
56	0.13692	15.2425732286942	15.2425732286942\\
56	0.14058	16.4654780500207	16.4654780500207\\
56	0.14424	17.7547728218096	17.7547728218096\\
56	0.1479	19.1104575440608	19.1104575440608\\
56	0.15156	20.5325322167744	20.5325322167744\\
56	0.15522	22.0209968399504	22.0209968399504\\
56	0.15888	23.5758514135888	23.5758514135888\\
56	0.16254	25.1970959376896	25.1970959376896\\
56	0.1662	26.8847304122528	26.8847304122528\\
56	0.16986	28.6387548372783	28.6387548372783\\
56	0.17352	30.4591692127662	30.4591692127662\\
56	0.17718	32.3459735387165	32.3459735387165\\
56	0.18084	34.2991678151292	34.2991678151292\\
56	0.1845	36.3187520420042	36.3187520420042\\
56	0.18816	38.4047262193417	38.4047262193417\\
56	0.19182	40.5570903471415	40.5570903471415\\
56	0.19548	42.7758444254037	42.7758444254037\\
56	0.19914	45.0609884541283	45.0609884541283\\
56	0.2028	47.4125224333153	47.4125224333153\\
56	0.20646	49.8304463629646	49.8304463629646\\
56	0.21012	52.3147602430763	52.3147602430763\\
56	0.21378	54.8654640736505	54.8654640736505\\
56	0.21744	57.4825578546869	57.4825578546869\\
56	0.2211	60.1660415861858	60.1660415861858\\
56	0.22476	62.9159152681471	62.9159152681471\\
56	0.22842	65.7321789005707	65.7321789005707\\
56	0.23208	68.6148324834567	68.6148324834567\\
56	0.23574	71.5638760168051	71.5638760168051\\
56	0.2394	74.5793095006159	74.5793095006159\\
56	0.24306	77.6611329348891	77.6611329348891\\
56	0.24672	80.8093463196246	80.8093463196246\\
56	0.25038	84.0239496548225	84.0239496548225\\
56	0.25404	87.3049429404828	87.3049429404828\\
56	0.2577	90.6523261766055	90.6523261766055\\
56	0.26136	94.0660993631906	94.0660993631906\\
56	0.26502	97.546262500238	97.546262500238\\
56	0.26868	101.092815587748	101.092815587748\\
56	0.27234	104.70575862572	104.70575862572\\
56	0.276	108.385091614155	108.385091614155\\
56.375	0.093	5.74291436766217	5.74291436766217\\
56.375	0.09666	6.16955692276974	6.16955692276974\\
56.375	0.10032	6.6625894283397	6.6625894283397\\
56.375	0.10398	7.22201188437204	7.22201188437204\\
56.375	0.10764	7.84782429086675	7.84782429086675\\
56.375	0.1113	8.54002664782384	8.54002664782384\\
56.375	0.11496	9.29861895524333	9.29861895524333\\
56.375	0.11862	10.1236012131252	10.1236012131252\\
56.375	0.12228	11.0149734214694	11.0149734214694\\
56.375	0.12594	11.9727355802761	11.9727355802761\\
56.375	0.1296	12.9968876895451	12.9968876895451\\
56.375	0.13326	14.0874297492765	14.0874297492765\\
56.375	0.13692	15.2443617594702	15.2443617594702\\
56.375	0.14058	16.4676837201264	16.4676837201264\\
56.375	0.14424	17.7573956312449	17.7573956312449\\
56.375	0.1479	19.1134974928258	19.1134974928258\\
56.375	0.15156	20.5359893048691	20.5359893048691\\
56.375	0.15522	22.0248710673748	22.0248710673748\\
56.375	0.15888	23.5801427803429	23.5801427803429\\
56.375	0.16254	25.2018044437733	25.2018044437733\\
56.375	0.1662	26.8898560576661	26.8898560576661\\
56.375	0.16986	28.6442976220213	28.6442976220213\\
56.375	0.17352	30.4651291368389	30.4651291368389\\
56.375	0.17718	32.3523506021189	32.3523506021189\\
56.375	0.18084	34.3059620178612	34.3059620178612\\
56.375	0.1845	36.325963384066	36.325963384066\\
56.375	0.18816	38.4123547007331	38.4123547007331\\
56.375	0.19182	40.5651359678625	40.5651359678625\\
56.375	0.19548	42.7843071854544	42.7843071854544\\
56.375	0.19914	45.0698683535087	45.0698683535087\\
56.375	0.2028	47.4218194720253	47.4218194720253\\
56.375	0.20646	49.8401605410044	49.8401605410044\\
56.375	0.21012	52.3248915604457	52.3248915604457\\
56.375	0.21378	54.8760125303495	54.8760125303495\\
56.375	0.21744	57.4935234507156	57.4935234507156\\
56.375	0.2211	60.1774243215442	60.1774243215442\\
56.375	0.22476	62.9277151428351	62.9277151428351\\
56.375	0.22842	65.7443959145884	65.7443959145884\\
56.375	0.23208	68.6274666368041	68.6274666368041\\
56.375	0.23574	71.5769273094822	71.5769273094822\\
56.375	0.2394	74.5927779326226	74.5927779326226\\
56.375	0.24306	77.6750185062255	77.6750185062255\\
56.375	0.24672	80.8236490302907	80.8236490302907\\
56.375	0.25038	84.0386695048182	84.0386695048182\\
56.375	0.25404	87.3200799298082	87.3200799298082\\
56.375	0.2577	90.6678803052606	90.6678803052606\\
56.375	0.26136	94.0820706311753	94.0820706311753\\
56.375	0.26502	97.5626509075524	97.5626509075524\\
56.375	0.26868	101.109621134392	101.109621134392\\
56.375	0.27234	104.722981311694	104.722981311694\\
56.375	0.276	108.402731439458	108.402731439458\\
56.75	0.093	5.7434396016763	5.7434396016763\\
56.75	0.09666	6.17049929611354	6.17049929611354\\
56.75	0.10032	6.66394894101316	6.66394894101316\\
56.75	0.10398	7.22378853637516	7.22378853637516\\
56.75	0.10764	7.85001808219955	7.85001808219955\\
56.75	0.1113	8.54263757848633	8.54263757848633\\
56.75	0.11496	9.30164702523549	9.30164702523549\\
56.75	0.11862	10.127046422447	10.127046422447\\
56.75	0.12228	11.0188357701209	11.0188357701209\\
56.75	0.12594	11.9770150682572	11.9770150682572\\
56.75	0.1296	13.0015843168559	13.0015843168559\\
56.75	0.13326	14.0925435159169	14.0925435159169\\
56.75	0.13692	15.2498926654404	15.2498926654404\\
56.75	0.14058	16.4736317654262	16.4736317654262\\
56.75	0.14424	17.7637608158744	17.7637608158744\\
56.75	0.1479	19.120279816785	19.120279816785\\
56.75	0.15156	20.5431887681579	20.5431887681579\\
56.75	0.15522	22.0324876699933	22.0324876699933\\
56.75	0.15888	23.588176522291	23.588176522291\\
56.75	0.16254	25.2102553250511	25.2102553250511\\
56.75	0.1662	26.8987240782736	26.8987240782736\\
56.75	0.16986	28.6535827819585	28.6535827819585\\
56.75	0.17352	30.4748314361057	30.4748314361057\\
56.75	0.17718	32.3624700407154	32.3624700407154\\
56.75	0.18084	34.3164985957873	34.3164985957873\\
56.75	0.1845	36.3369171013218	36.3369171013218\\
56.75	0.18816	38.4237255573185	38.4237255573185\\
56.75	0.19182	40.5769239637777	40.5769239637777\\
56.75	0.19548	42.7965123206992	42.7965123206992\\
56.75	0.19914	45.0824906280832	45.0824906280832\\
56.75	0.2028	47.4348588859295	47.4348588859295\\
56.75	0.20646	49.8536170942382	49.8536170942382\\
56.75	0.21012	52.3387652530092	52.3387652530092\\
56.75	0.21378	54.8903033622426	54.8903033622426\\
56.75	0.21744	57.5082314219385	57.5082314219385\\
56.75	0.2211	60.1925494320967	60.1925494320967\\
56.75	0.22476	62.9432573927173	62.9432573927173\\
56.75	0.22842	65.7603553038003	65.7603553038003\\
56.75	0.23208	68.6438431653456	68.6438431653456\\
56.75	0.23574	71.5937209773533	71.5937209773533\\
56.75	0.2394	74.6099887398234	74.6099887398234\\
56.75	0.24306	77.6926464527559	77.6926464527559\\
56.75	0.24672	80.8416941161508	80.8416941161508\\
56.75	0.25038	84.0571317300081	84.0571317300081\\
56.75	0.25404	87.3389592943277	87.3389592943277\\
56.75	0.2577	90.6871768091097	90.6871768091097\\
56.75	0.26136	94.1017842743541	94.1017842743541\\
56.75	0.26502	97.5827816900609	97.5827816900609\\
56.75	0.26868	101.13016905623	101.13016905623\\
56.75	0.27234	104.743946372862	104.743946372862\\
56.75	0.276	108.424113639955	108.424113639955\\
57.125	0.093	5.74770721088454	5.74770721088454\\
57.125	0.09666	6.17518404465145	6.17518404465145\\
57.125	0.10032	6.66905082888075	6.66905082888075\\
57.125	0.10398	7.22930756357241	7.22930756357241\\
57.125	0.10764	7.85595424872646	7.85595424872646\\
57.125	0.1113	8.54899088434291	8.54899088434291\\
57.125	0.11496	9.30841747042173	9.30841747042173\\
57.125	0.11862	10.1342340069629	10.1342340069629\\
57.125	0.12228	11.0264404939665	11.0264404939665\\
57.125	0.12594	11.9850369314325	11.9850369314325\\
57.125	0.1296	13.0100233193608	13.0100233193608\\
57.125	0.13326	14.1013996577515	14.1013996577515\\
57.125	0.13692	15.2591659466046	15.2591659466046\\
57.125	0.14058	16.4833221859201	16.4833221859201\\
57.125	0.14424	17.773868375698	17.773868375698\\
57.125	0.1479	19.1308045159382	19.1308045159382\\
57.125	0.15156	20.5541306066409	20.5541306066409\\
57.125	0.15522	22.0438466478059	22.0438466478059\\
57.125	0.15888	23.5999526394333	23.5999526394333\\
57.125	0.16254	25.2224485815231	25.2224485815231\\
57.125	0.1662	26.9113344740752	26.9113344740752\\
57.125	0.16986	28.6666103170897	28.6666103170897\\
57.125	0.17352	30.4882761105666	30.4882761105666\\
57.125	0.17718	32.3763318545059	32.3763318545059\\
57.125	0.18084	34.3307775489076	34.3307775489076\\
57.125	0.1845	36.3516131937717	36.3516131937717\\
57.125	0.18816	38.4388387890981	38.4388387890981\\
57.125	0.19182	40.592454334887	40.592454334887\\
57.125	0.19548	42.8124598311382	42.8124598311382\\
57.125	0.19914	45.0988552778518	45.0988552778518\\
57.125	0.2028	47.4516406750277	47.4516406750277\\
57.125	0.20646	49.8708160226661	49.8708160226661\\
57.125	0.21012	52.3563813207668	52.3563813207668\\
57.125	0.21378	54.9083365693299	54.9083365693299\\
57.125	0.21744	57.5266817683554	57.5266817683554\\
57.125	0.2211	60.2114169178433	60.2114169178433\\
57.125	0.22476	62.9625420177935	62.9625420177935\\
57.125	0.22842	65.7800570682062	65.7800570682062\\
57.125	0.23208	68.6639620690812	68.6639620690812\\
57.125	0.23574	71.6142570204186	71.6142570204186\\
57.125	0.2394	74.6309419222184	74.6309419222184\\
57.125	0.24306	77.7140167744805	77.7140167744805\\
57.125	0.24672	80.8634815772051	80.8634815772051\\
57.125	0.25038	84.079336330392	84.079336330392\\
57.125	0.25404	87.3615810340413	87.3615810340413\\
57.125	0.2577	90.710215688153	90.710215688153\\
57.125	0.26136	94.1252402927271	94.1252402927271\\
57.125	0.26502	97.6066548477635	97.6066548477635\\
57.125	0.26868	101.154459353262	101.154459353262\\
57.125	0.27234	104.768653809223	104.768653809223\\
57.125	0.276	108.449238215647	108.449238215647\\
57.5	0.093	5.75571719528687	5.75571719528687\\
57.5	0.09666	6.18361116838344	6.18361116838344\\
57.5	0.10032	6.67789509194241	6.67789509194241\\
57.5	0.10398	7.23856896596375	7.23856896596375\\
57.5	0.10764	7.86563279044746	7.86563279044746\\
57.5	0.1113	8.55908656539355	8.55908656539355\\
57.5	0.11496	9.31893029080205	9.31893029080205\\
57.5	0.11862	10.1451639666729	10.1451639666729\\
57.5	0.12228	11.0377875930062	11.0377875930062\\
57.5	0.12594	11.9968011698018	11.9968011698018\\
57.5	0.1296	13.0222046970598	13.0222046970598\\
57.5	0.13326	14.1139981747802	14.1139981747802\\
57.5	0.13692	15.272181602963	15.272181602963\\
57.5	0.14058	16.4967549816081	16.4967549816081\\
57.5	0.14424	17.7877183107156	17.7877183107156\\
57.5	0.1479	19.1450715902856	19.1450715902856\\
57.5	0.15156	20.5688148203178	20.5688148203178\\
57.5	0.15522	22.0589480008125	22.0589480008125\\
57.5	0.15888	23.6154711317696	23.6154711317696\\
57.5	0.16254	25.238384213189	25.238384213189\\
57.5	0.1662	26.9276872450709	26.9276872450709\\
57.5	0.16986	28.6833802274151	28.6833802274151\\
57.5	0.17352	30.5054631602217	30.5054631602217\\
57.5	0.17718	32.3939360434906	32.3939360434906\\
57.5	0.18084	34.3487988772219	34.3487988772219\\
57.5	0.1845	36.3700516614157	36.3700516614157\\
57.5	0.18816	38.4576943960718	38.4576943960718\\
57.5	0.19182	40.6117270811903	40.6117270811903\\
57.5	0.19548	42.8321497167711	42.8321497167711\\
57.5	0.19914	45.1189623028144	45.1189623028144\\
57.5	0.2028	47.4721648393201	47.4721648393201\\
57.5	0.20646	49.8917573262881	49.8917573262881\\
57.5	0.21012	52.3777397637185	52.3777397637185\\
57.5	0.21378	54.9301121516112	54.9301121516112\\
57.5	0.21744	57.5488744899664	57.5488744899664\\
57.5	0.2211	60.2340267787839	60.2340267787839\\
57.5	0.22476	62.9855690180638	62.9855690180638\\
57.5	0.22842	65.8035012078062	65.8035012078062\\
57.5	0.23208	68.6878233480108	68.6878233480108\\
57.5	0.23574	71.6385354386779	71.6385354386779\\
57.5	0.2394	74.6556374798074	74.6556374798074\\
57.5	0.24306	77.7391294713992	77.7391294713992\\
57.5	0.24672	80.8890114134534	80.8890114134534\\
57.5	0.25038	84.10528330597	84.10528330597\\
57.5	0.25404	87.3879451489489	87.3879451489489\\
57.5	0.2577	90.7369969423903	90.7369969423903\\
57.5	0.26136	94.1524386862941	94.1524386862941\\
57.5	0.26502	97.6342703806602	97.6342703806602\\
57.5	0.26868	101.182492025489	101.182492025489\\
57.5	0.27234	104.79710362078	104.79710362078\\
57.5	0.276	108.478105166533	108.478105166533\\
57.875	0.093	5.76746955488329	5.76746955488329\\
57.875	0.09666	6.19578066730953	6.19578066730953\\
57.875	0.10032	6.69048173019815	6.69048173019815\\
57.875	0.10398	7.25157274354916	7.25157274354916\\
57.875	0.10764	7.87905370736255	7.87905370736255\\
57.875	0.1113	8.57292462163833	8.57292462163833\\
57.875	0.11496	9.33318548637649	9.33318548637649\\
57.875	0.11862	10.159836301577	10.159836301577\\
57.875	0.12228	11.0528770672399	11.0528770672399\\
57.875	0.12594	12.0123077833652	12.0123077833652\\
57.875	0.1296	13.0381284499529	13.0381284499529\\
57.875	0.13326	14.1303390670029	14.1303390670029\\
57.875	0.13692	15.2889396345154	15.2889396345154\\
57.875	0.14058	16.5139301524902	16.5139301524902\\
57.875	0.14424	17.8053106209274	17.8053106209274\\
57.875	0.1479	19.163081039827	19.163081039827\\
57.875	0.15156	20.587241409189	20.587241409189\\
57.875	0.15522	22.0777917290133	22.0777917290133\\
57.875	0.15888	23.6347319993	23.6347319993\\
57.875	0.16254	25.2580622200491	25.2580622200491\\
57.875	0.1662	26.9477823912606	26.9477823912606\\
57.875	0.16986	28.7038925129345	28.7038925129345\\
57.875	0.17352	30.5263925850707	30.5263925850707\\
57.875	0.17718	32.4152826076694	32.4152826076694\\
57.875	0.18084	34.3705625807304	34.3705625807304\\
57.875	0.1845	36.3922325042538	36.3922325042538\\
57.875	0.18816	38.4802923782396	38.4802923782396\\
57.875	0.19182	40.6347422026877	40.6347422026877\\
57.875	0.19548	42.8555819775983	42.8555819775983\\
57.875	0.19914	45.1428117029712	45.1428117029712\\
57.875	0.2028	47.4964313788065	47.4964313788065\\
57.875	0.20646	49.9164410051042	49.9164410051042\\
57.875	0.21012	52.4028405818642	52.4028405818642\\
57.875	0.21378	54.9556301090867	54.9556301090867\\
57.875	0.21744	57.5748095867715	57.5748095867715\\
57.875	0.2211	60.2603790149187	60.2603790149187\\
57.875	0.22476	63.0123383935283	63.0123383935283\\
57.875	0.22842	65.8306877226003	65.8306877226003\\
57.875	0.23208	68.7154270021346	68.7154270021346\\
57.875	0.23574	71.6665562321313	71.6665562321313\\
57.875	0.2394	74.6840754125905	74.6840754125905\\
57.875	0.24306	77.7679845435119	77.7679845435119\\
57.875	0.24672	80.9182836248958	80.9182836248958\\
57.875	0.25038	84.1349726567421	84.1349726567421\\
57.875	0.25404	87.4180516390507	87.4180516390507\\
57.875	0.2577	90.7675205718218	90.7675205718218\\
57.875	0.26136	94.1833794550552	94.1833794550552\\
57.875	0.26502	97.6656282887509	97.6656282887509\\
57.875	0.26868	101.214267072909	101.214267072909\\
57.875	0.27234	104.82929580753	104.82929580753\\
57.875	0.276	108.510714492613	108.510714492613\\
58.25	0.093	5.78296428967383	5.78296428967383\\
58.25	0.09666	6.21169254142973	6.21169254142973\\
58.25	0.10032	6.70681074364802	6.70681074364802\\
58.25	0.10398	7.26831889632869	7.26831889632869\\
58.25	0.10764	7.89621699947175	7.89621699947175\\
58.25	0.1113	8.5905050530772	8.5905050530772\\
58.25	0.11496	9.35118305714501	9.35118305714501\\
58.25	0.11862	10.1782510116752	10.1782510116752\\
58.25	0.12228	11.0717089166678	11.0717089166678\\
58.25	0.12594	12.0315567721228	12.0315567721228\\
58.25	0.1296	13.0577945780401	13.0577945780401\\
58.25	0.13326	14.1504223344198	14.1504223344198\\
58.25	0.13692	15.3094400412619	15.3094400412619\\
58.25	0.14058	16.5348476985664	16.5348476985664\\
58.25	0.14424	17.8266453063333	17.8266453063333\\
58.25	0.1479	19.1848328645625	19.1848328645625\\
58.25	0.15156	20.6094103732542	20.6094103732542\\
58.25	0.15522	22.1003778324082	22.1003778324082\\
58.25	0.15888	23.6577352420246	23.6577352420246\\
58.25	0.16254	25.2814826021033	25.2814826021033\\
58.25	0.1662	26.9716199126445	26.9716199126445\\
58.25	0.16986	28.728147173648	28.728147173648\\
58.25	0.17352	30.551064385114	30.551064385114\\
58.25	0.17718	32.4403715470422	32.4403715470422\\
58.25	0.18084	34.3960686594329	34.3960686594329\\
58.25	0.1845	36.418155722286	36.418155722286\\
58.25	0.18816	38.5066327356014	38.5066327356014\\
58.25	0.19182	40.6614996993793	40.6614996993793\\
58.25	0.19548	42.8827566136195	42.8827566136195\\
58.25	0.19914	45.1704034783221	45.1704034783221\\
58.25	0.2028	47.524440293487	47.524440293487\\
58.25	0.20646	49.9448670591144	49.9448670591144\\
58.25	0.21012	52.4316837752041	52.4316837752041\\
58.25	0.21378	54.9848904417562	54.9848904417562\\
58.25	0.21744	57.6044870587707	57.6044870587707\\
58.25	0.2211	60.2904736262476	60.2904736262476\\
58.25	0.22476	63.0428501441868	63.0428501441868\\
58.25	0.22842	65.8616166125885	65.8616166125885\\
58.25	0.23208	68.7467730314525	68.7467730314525\\
58.25	0.23574	71.6983194007789	71.6983194007789\\
58.25	0.2394	74.7162557205677	74.7162557205677\\
58.25	0.24306	77.8005819908188	77.8005819908188\\
58.25	0.24672	80.9512982115324	80.9512982115324\\
58.25	0.25038	84.1684043827083	84.1684043827083\\
58.25	0.25404	87.4519005043466	87.4519005043466\\
58.25	0.2577	90.8017865764473	90.8017865764473\\
58.25	0.26136	94.2180625990104	94.2180625990104\\
58.25	0.26502	97.7007285720358	97.7007285720358\\
58.25	0.26868	101.249784495524	101.249784495524\\
58.25	0.27234	104.865230369474	104.865230369474\\
58.25	0.276	108.547066193886	108.547066193886\\
58.625	0.093	5.80220139965844	5.80220139965844\\
58.625	0.09666	6.23134679074402	6.23134679074402\\
58.625	0.10032	6.72688213229199	6.72688213229199\\
58.625	0.10398	7.28880742430233	7.28880742430233\\
58.625	0.10764	7.91712266677504	7.91712266677504\\
58.625	0.1113	8.61182785971013	8.61182785971013\\
58.625	0.11496	9.37292300310763	9.37292300310763\\
58.625	0.11862	10.2004080969675	10.2004080969675\\
58.625	0.12228	11.0942831412897	11.0942831412897\\
58.625	0.12594	12.0545481360744	12.0545481360744\\
58.625	0.1296	13.0812030813214	13.0812030813214\\
58.625	0.13326	14.1742479770308	14.1742479770308\\
58.625	0.13692	15.3336828232025	15.3336828232025\\
58.625	0.14058	16.5595076198367	16.5595076198367\\
58.625	0.14424	17.8517223669332	17.8517223669332\\
58.625	0.1479	19.2103270644922	19.2103270644922\\
58.625	0.15156	20.6353217125134	20.6353217125134\\
58.625	0.15522	22.1267063109971	22.1267063109971\\
58.625	0.15888	23.6844808599432	23.6844808599432\\
58.625	0.16254	25.3086453593516	25.3086453593516\\
58.625	0.1662	26.9991998092225	26.9991998092225\\
58.625	0.16986	28.7561442095557	28.7561442095557\\
58.625	0.17352	30.5794785603513	30.5794785603513\\
58.625	0.17718	32.4692028616092	32.4692028616092\\
58.625	0.18084	34.4253171133296	34.4253171133296\\
58.625	0.1845	36.4478213155123	36.4478213155123\\
58.625	0.18816	38.5367154681574	38.5367154681574\\
58.625	0.19182	40.6919995712649	40.6919995712649\\
58.625	0.19548	42.9136736248348	42.9136736248348\\
58.625	0.19914	45.201737628867	45.201737628867\\
58.625	0.2028	47.5561915833617	47.5561915833617\\
58.625	0.20646	49.9770354883187	49.9770354883187\\
58.625	0.21012	52.4642693437381	52.4642693437381\\
58.625	0.21378	55.0178931496199	55.0178931496199\\
58.625	0.21744	57.637906905964	57.637906905964\\
58.625	0.2211	60.3243106127705	60.3243106127705\\
58.625	0.22476	63.0771042700395	63.0771042700395\\
58.625	0.22842	65.8962878777708	65.8962878777708\\
58.625	0.23208	68.7818614359644	68.7818614359644\\
58.625	0.23574	71.7338249446206	71.7338249446206\\
58.625	0.2394	74.752178403739	74.752178403739\\
58.625	0.24306	77.8369218133198	77.8369218133198\\
58.625	0.24672	80.988055173363	80.988055173363\\
58.625	0.25038	84.2055784838686	84.2055784838686\\
58.625	0.25404	87.4894917448366	87.4894917448366\\
58.625	0.2577	90.8397949562669	90.8397949562669\\
58.625	0.26136	94.2564881181597	94.2564881181597\\
58.625	0.26502	97.7395712305148	97.7395712305148\\
58.625	0.26868	101.289044293332	101.289044293332\\
58.625	0.27234	104.904907306612	104.904907306612\\
58.625	0.276	108.587160270354	108.587160270354\\
59	0.093	5.82518088483718	5.82518088483718\\
59	0.09666	6.25474341525242	6.25474341525242\\
59	0.10032	6.75069589613006	6.75069589613006\\
59	0.10398	7.31303832747006	7.31303832747006\\
59	0.10764	7.94177070927245	7.94177070927245\\
59	0.1113	8.63689304153721	8.63689304153721\\
59	0.11496	9.39840532426436	9.39840532426436\\
59	0.11862	10.2263075574539	10.2263075574539\\
59	0.12228	11.1205997411058	11.1205997411058\\
59	0.12594	12.0812818752201	12.0812818752201\\
59	0.1296	13.1083539597968	13.1083539597968\\
59	0.13326	14.2018159948358	14.2018159948358\\
59	0.13692	15.3616679803373	15.3616679803373\\
59	0.14058	16.5879099163011	16.5879099163011\\
59	0.14424	17.8805418027273	17.8805418027273\\
59	0.1479	19.2395636396159	19.2395636396159\\
59	0.15156	20.6649754269669	20.6649754269669\\
59	0.15522	22.1567771647802	22.1567771647802\\
59	0.15888	23.7149688530559	23.7149688530559\\
59	0.16254	25.339550491794	25.339550491794\\
59	0.1662	27.0305220809945	27.0305220809945\\
59	0.16986	28.7878836206574	28.7878836206574\\
59	0.17352	30.6116351107827	30.6116351107827\\
59	0.17718	32.5017765513703	32.5017765513703\\
59	0.18084	34.4583079424203	34.4583079424203\\
59	0.1845	36.4812292839327	36.4812292839327\\
59	0.18816	38.5705405759075	38.5705405759075\\
59	0.19182	40.7262418183446	40.7262418183446\\
59	0.19548	42.9483330112442	42.9483330112442\\
59	0.19914	45.2368141546061	45.2368141546061\\
59	0.2028	47.5916852484304	47.5916852484304\\
59	0.20646	50.0129462927171	50.0129462927171\\
59	0.21012	52.5005972874662	52.5005972874662\\
59	0.21378	55.0546382326776	55.0546382326776\\
59	0.21744	57.6750691283514	57.6750691283514\\
59	0.2211	60.3618899744876	60.3618899744876\\
59	0.22476	63.1151007710862	63.1151007710862\\
59	0.22842	65.9347015181472	65.9347015181472\\
59	0.23208	68.8206922156705	68.8206922156705\\
59	0.23574	71.7730728636563	71.7730728636563\\
59	0.2394	74.7918434621044	74.7918434621044\\
59	0.24306	77.8770040110149	77.8770040110149\\
59	0.24672	81.0285545103878	81.0285545103878\\
59	0.25038	84.246494960223	84.246494960223\\
59	0.25404	87.5308253605207	87.5308253605207\\
59	0.2577	90.8815457112807	90.8815457112807\\
59	0.26136	94.2986560125031	94.2986560125031\\
59	0.26502	97.7821562641878	97.7821562641878\\
59	0.26868	101.332046466335	101.332046466335\\
59	0.27234	104.948326618945	104.948326618945\\
59	0.276	108.630996722016	108.630996722016\\
59.375	0.093	5.85190274520999	5.85190274520999\\
59.375	0.09666	6.28188241495491	6.28188241495491\\
59.375	0.10032	6.77825203516219	6.77825203516219\\
59.375	0.10398	7.34101160583186	7.34101160583186\\
59.375	0.10764	7.97016112696393	7.97016112696393\\
59.375	0.1113	8.66570059855837	8.66570059855837\\
59.375	0.11496	9.4276300206152	9.4276300206152\\
59.375	0.11862	10.2559493931344	10.2559493931344\\
59.375	0.12228	11.150658716116	11.150658716116\\
59.375	0.12594	12.1117579895599	12.1117579895599\\
59.375	0.1296	13.1392472134663	13.1392472134663\\
59.375	0.13326	14.233126387835	14.233126387835\\
59.375	0.13692	15.3933955126661	15.3933955126661\\
59.375	0.14058	16.6200545879596	16.6200545879596\\
59.375	0.14424	17.9131036137155	17.9131036137155\\
59.375	0.1479	19.2725425899337	19.2725425899337\\
59.375	0.15156	20.6983715166143	20.6983715166143\\
59.375	0.15522	22.1905903937574	22.1905903937574\\
59.375	0.15888	23.7491992213628	23.7491992213628\\
59.375	0.16254	25.3741979994305	25.3741979994305\\
59.375	0.1662	27.0655867279607	27.0655867279607\\
59.375	0.16986	28.8233654069532	28.8233654069532\\
59.375	0.17352	30.6475340364081	30.6475340364081\\
59.375	0.17718	32.5380926163254	32.5380926163254\\
59.375	0.18084	34.4950411467051	34.4950411467051\\
59.375	0.1845	36.5183796275472	36.5183796275472\\
59.375	0.18816	38.6081080588516	38.6081080588516\\
59.375	0.19182	40.7642264406185	40.7642264406185\\
59.375	0.19548	42.9867347728477	42.9867347728477\\
59.375	0.19914	45.2756330555393	45.2756330555393\\
59.375	0.2028	47.6309212886932	47.6309212886932\\
59.375	0.20646	50.0525994723096	50.0525994723096\\
59.375	0.21012	52.5406676063883	52.5406676063883\\
59.375	0.21378	55.0951256909294	55.0951256909294\\
59.375	0.21744	57.7159737259329	57.7159737259329\\
59.375	0.2211	60.4032117113988	60.4032117113988\\
59.375	0.22476	63.156839647327	63.156839647327\\
59.375	0.22842	65.9768575337177	65.9768575337177\\
59.375	0.23208	68.8632653705707	68.8632653705707\\
59.375	0.23574	71.8160631578861	71.8160631578861\\
59.375	0.2394	74.8352508956639	74.8352508956639\\
59.375	0.24306	77.920828583904	77.920828583904\\
59.375	0.24672	81.0727962226066	81.0727962226066\\
59.375	0.25038	84.2911538117715	84.2911538117715\\
59.375	0.25404	87.5759013513988	87.5759013513988\\
59.375	0.2577	90.9270388414885	90.9270388414885\\
59.375	0.26136	94.3445662820406	94.3445662820406\\
59.375	0.26502	97.828483673055	97.828483673055\\
59.375	0.26868	101.378791014532	101.378791014532\\
59.375	0.27234	104.995488306471	104.995488306471\\
59.375	0.276	108.678575548873	108.678575548873\\
59.75	0.093	5.8823669807769	5.8823669807769\\
59.75	0.09666	6.31276378985148	6.31276378985148\\
59.75	0.10032	6.80955054938845	6.80955054938845\\
59.75	0.10398	7.37272725938779	7.37272725938779\\
59.75	0.10764	8.00229391984949	8.00229391984949\\
59.75	0.1113	8.69825053077359	8.69825053077359\\
59.75	0.11496	9.46059709216009	9.46059709216009\\
59.75	0.11862	10.289333604009	10.289333604009\\
59.75	0.12228	11.1844600663202	11.1844600663202\\
59.75	0.12594	12.1459764790938	12.1459764790938\\
59.75	0.1296	13.1738828423298	13.1738828423298\\
59.75	0.13326	14.2681791560282	14.2681791560282\\
59.75	0.13692	15.428865420189	15.428865420189\\
59.75	0.14058	16.6559416348122	16.6559416348122\\
59.75	0.14424	17.9494077998977	17.9494077998977\\
59.75	0.1479	19.3092639154456	19.3092639154456\\
59.75	0.15156	20.7355099814559	20.7355099814559\\
59.75	0.15522	22.2281459979286	22.2281459979286\\
59.75	0.15888	23.7871719648637	23.7871719648637\\
59.75	0.16254	25.4125878822611	25.4125878822611\\
59.75	0.1662	27.1043937501209	27.1043937501209\\
59.75	0.16986	28.8625895684431	28.8625895684431\\
59.75	0.17352	30.6871753372277	30.6871753372277\\
59.75	0.17718	32.5781510564747	32.5781510564747\\
59.75	0.18084	34.535516726184	34.535516726184\\
59.75	0.1845	36.5592723463558	36.5592723463558\\
59.75	0.18816	38.6494179169899	38.6494179169899\\
59.75	0.19182	40.8059534380864	40.8059534380864\\
59.75	0.19548	43.0288789096452	43.0288789096452\\
59.75	0.19914	45.3181943316665	45.3181943316665\\
59.75	0.2028	47.6738997041501	47.6738997041501\\
59.75	0.20646	50.0959950270962	50.0959950270962\\
59.75	0.21012	52.5844803005046	52.5844803005046\\
59.75	0.21378	55.1393555243753	55.1393555243753\\
59.75	0.21744	57.7606206987085	57.7606206987085\\
59.75	0.2211	60.448275823504	60.448275823504\\
59.75	0.22476	63.202320898762	63.202320898762\\
59.75	0.22842	66.0227559244823	66.0227559244823\\
59.75	0.23208	68.9095809006649	68.9095809006649\\
59.75	0.23574	71.86279582731	71.86279582731\\
59.75	0.2394	74.8824007044175	74.8824007044175\\
59.75	0.24306	77.9683955319873	77.9683955319873\\
59.75	0.24672	81.1207803100195	81.1207803100195\\
59.75	0.25038	84.3395550385141	84.3395550385141\\
59.75	0.25404	87.6247197174711	87.6247197174711\\
59.75	0.2577	90.9762743468904	90.9762743468904\\
59.75	0.26136	94.3942189267722	94.3942189267722\\
59.75	0.26502	97.8785534571163	97.8785534571163\\
59.75	0.26868	101.429277937923	101.429277937923\\
59.75	0.27234	105.046392369192	105.046392369192\\
59.75	0.276	108.729896750923	108.729896750923\\
60.125	0.093	5.91657359153793	5.91657359153793\\
60.125	0.09666	6.34738753994216	6.34738753994216\\
60.125	0.10032	6.8445914388088	6.8445914388088\\
60.125	0.10398	7.40818528813782	7.40818528813782\\
60.125	0.10764	8.03816908792918	8.03816908792918\\
60.125	0.1113	8.73454283818295	8.73454283818295\\
60.125	0.11496	9.49730653889912	9.49730653889912\\
60.125	0.11862	10.3264601900776	10.3264601900776\\
60.125	0.12228	11.2220037917186	11.2220037917186\\
60.125	0.12594	12.1839373438219	12.1839373438219\\
60.125	0.1296	13.2122608463875	13.2122608463875\\
60.125	0.13326	14.3069742994156	14.3069742994156\\
60.125	0.13692	15.468077702906	15.468077702906\\
60.125	0.14058	16.6955710568589	16.6955710568589\\
60.125	0.14424	17.9894543612741	17.9894543612741\\
60.125	0.1479	19.3497276161517	19.3497276161517\\
60.125	0.15156	20.7763908214916	20.7763908214916\\
60.125	0.15522	22.269443977294	22.269443977294\\
60.125	0.15888	23.8288870835587	23.8288870835587\\
60.125	0.16254	25.4547201402858	25.4547201402858\\
60.125	0.1662	27.1469431474753	27.1469431474753\\
60.125	0.16986	28.9055561051272	28.9055561051272\\
60.125	0.17352	30.7305590132414	30.7305590132414\\
60.125	0.17718	32.6219518718181	32.6219518718181\\
60.125	0.18084	34.5797346808571	34.5797346808571\\
60.125	0.1845	36.6039074403585	36.6039074403585\\
60.125	0.18816	38.6944701503222	38.6944701503222\\
60.125	0.19182	40.8514228107484	40.8514228107484\\
60.125	0.19548	43.0747654216369	43.0747654216369\\
60.125	0.19914	45.3644979829879	45.3644979829879\\
60.125	0.2028	47.7206204948012	47.7206204948012\\
60.125	0.20646	50.1431329570769	50.1431329570769\\
60.125	0.21012	52.6320353698149	52.6320353698149\\
60.125	0.21378	55.1873277330154	55.1873277330154\\
60.125	0.21744	57.8090100466782	57.8090100466782\\
60.125	0.2211	60.4970823108034	60.4970823108034\\
60.125	0.22476	63.251544525391	63.251544525391\\
60.125	0.22842	66.072396690441	66.072396690441\\
60.125	0.23208	68.9596388059533	68.9596388059533\\
60.125	0.23574	71.9132708719281	71.9132708719281\\
60.125	0.2394	74.9332928883652	74.9332928883652\\
60.125	0.24306	78.0197048552647	78.0197048552647\\
60.125	0.24672	81.1725067726265	81.1725067726265\\
60.125	0.25038	84.3916986404508	84.3916986404508\\
60.125	0.25404	87.6772804587374	87.6772804587374\\
60.125	0.2577	91.0292522274865	91.0292522274865\\
60.125	0.26136	94.4476139466979	94.4476139466979\\
60.125	0.26502	97.9323656163716	97.9323656163716\\
60.125	0.26868	101.483507236508	101.483507236508\\
60.125	0.27234	105.101038807106	105.101038807106\\
60.125	0.276	108.784960328167	108.784960328167\\
60.5	0.093	5.95452257749304	5.95452257749304\\
60.5	0.09666	6.38575366522694	6.38575366522694\\
60.5	0.10032	6.88337470342323	6.88337470342323\\
60.5	0.10398	7.44738569208191	7.44738569208191\\
60.5	0.10764	8.07778663120296	8.07778663120296\\
60.5	0.1113	8.77457752078642	8.77457752078642\\
60.5	0.11496	9.53775836083224	9.53775836083224\\
60.5	0.11862	10.3673291513404	10.3673291513404\\
60.5	0.12228	11.263289892311	11.263289892311\\
60.5	0.12594	12.225640583744	12.225640583744\\
60.5	0.1296	13.2543812256393	13.2543812256393\\
60.5	0.13326	14.349511817997	14.349511817997\\
60.5	0.13692	15.5110323608171	15.5110323608171\\
60.5	0.14058	16.7389428540996	16.7389428540996\\
60.5	0.14424	18.0332432978445	18.0332432978445\\
60.5	0.1479	19.3939336920518	19.3939336920518\\
60.5	0.15156	20.8210140367214	20.8210140367214\\
60.5	0.15522	22.3144843318534	22.3144843318534\\
60.5	0.15888	23.8743445774478	23.8743445774478\\
60.5	0.16254	25.5005947735046	25.5005947735046\\
60.5	0.1662	27.1932349200237	27.1932349200237\\
60.5	0.16986	28.9522650170053	28.9522650170053\\
60.5	0.17352	30.7776850644492	30.7776850644492\\
60.5	0.17718	32.6694950623555	32.6694950623555\\
60.5	0.18084	34.6276950107242	34.6276950107242\\
60.5	0.1845	36.6522849095552	36.6522849095552\\
60.5	0.18816	38.7432647588487	38.7432647588487\\
60.5	0.19182	40.9006345586045	40.9006345586045\\
60.5	0.19548	43.1243943088227	43.1243943088227\\
60.5	0.19914	45.4145440095033	45.4145440095033\\
60.5	0.2028	47.7710836606463	47.7710836606463\\
60.5	0.20646	50.1940132622517	50.1940132622517\\
60.5	0.21012	52.6833328143194	52.6833328143194\\
60.5	0.21378	55.2390423168495	55.2390423168495\\
60.5	0.21744	57.861141769842	57.861141769842\\
60.5	0.2211	60.5496311732969	60.5496311732969\\
60.5	0.22476	63.3045105272141	63.3045105272141\\
60.5	0.22842	66.1257798315938	66.1257798315938\\
60.5	0.23208	69.0134390864358	69.0134390864358\\
60.5	0.23574	71.9674882917402	71.9674882917402\\
60.5	0.2394	74.987927447507	74.987927447507\\
60.5	0.24306	78.0747565537361	78.0747565537361\\
60.5	0.24672	81.2279756104277	81.2279756104277\\
60.5	0.25038	84.4475846175816	84.4475846175816\\
60.5	0.25404	87.7335835751979	87.7335835751979\\
60.5	0.2577	91.0859724832766	91.0859724832766\\
60.5	0.26136	94.5047513418177	94.5047513418177\\
60.5	0.26502	97.9899201508211	97.9899201508211\\
60.5	0.26868	101.541478910287	101.541478910287\\
60.5	0.27234	105.159427620215	105.159427620215\\
60.5	0.276	108.843766280606	108.843766280606\\
60.875	0.093	5.99621393864223	5.99621393864223\\
60.875	0.09666	6.4278621657058	6.4278621657058\\
60.875	0.10032	6.92590034323178	6.92590034323178\\
60.875	0.10398	7.49032847122012	7.49032847122012\\
60.875	0.10764	8.12114654967083	8.12114654967083\\
60.875	0.1113	8.81835457858393	8.81835457858393\\
60.875	0.11496	9.58195255795943	9.58195255795943\\
60.875	0.11862	10.4119404877973	10.4119404877973\\
60.875	0.12228	11.3083183680975	11.3083183680975\\
60.875	0.12594	12.2710861988602	12.2710861988602\\
60.875	0.1296	13.3002439800852	13.3002439800852\\
60.875	0.13326	14.3957917117726	14.3957917117726\\
60.875	0.13692	15.5577293939224	15.5577293939224\\
60.875	0.14058	16.7860570265345	16.7860570265345\\
60.875	0.14424	18.0807746096091	18.0807746096091\\
60.875	0.1479	19.441882143146	19.441882143146\\
60.875	0.15156	20.8693796271453	20.8693796271453\\
60.875	0.15522	22.363267061607	22.363267061607\\
60.875	0.15888	23.923544446531	23.923544446531\\
60.875	0.16254	25.5502117819175	25.5502117819175\\
60.875	0.1662	27.2432690677663	27.2432690677663\\
60.875	0.16986	29.0027163040775	29.0027163040775\\
60.875	0.17352	30.8285534908511	30.8285534908511\\
60.875	0.17718	32.720780628087	32.720780628087\\
60.875	0.18084	34.6793977157854	34.6793977157854\\
60.875	0.1845	36.7044047539461	36.7044047539461\\
60.875	0.18816	38.7958017425692	38.7958017425692\\
60.875	0.19182	40.9535886816547	40.9535886816547\\
60.875	0.19548	43.1777655712026	43.1777655712026\\
60.875	0.19914	45.4683324112129	45.4683324112129\\
60.875	0.2028	47.8252892016855	47.8252892016855\\
60.875	0.20646	50.2486359426205	50.2486359426205\\
60.875	0.21012	52.7383726340179	52.7383726340179\\
60.875	0.21378	55.2944992758777	55.2944992758777\\
60.875	0.21744	57.9170158681998	57.9170158681998\\
60.875	0.2211	60.6059224109844	60.6059224109844\\
60.875	0.22476	63.3612189042313	63.3612189042313\\
60.875	0.22842	66.1829053479406	66.1829053479406\\
60.875	0.23208	69.0709817421123	69.0709817421123\\
60.875	0.23574	72.0254480867464	72.0254480867464\\
60.875	0.2394	75.0463043818429	75.0463043818429\\
60.875	0.24306	78.1335506274017	78.1335506274017\\
60.875	0.24672	81.2871868234229	81.2871868234229\\
60.875	0.25038	84.5072129699065	84.5072129699065\\
60.875	0.25404	87.7936290668524	87.7936290668524\\
60.875	0.2577	91.1464351142608	91.1464351142608\\
60.875	0.26136	94.5656311121315	94.5656311121315\\
60.875	0.26502	98.0512170604646	98.0512170604646\\
60.875	0.26868	101.60319295926	101.60319295926\\
60.875	0.27234	105.221558808518	105.221558808518\\
60.875	0.276	108.906314608238	108.906314608238\\
61.25	0.093	6.04164767498554	6.04164767498554\\
61.25	0.09666	6.47371304137879	6.47371304137879\\
61.25	0.10032	6.97216835823442	6.97216835823442\\
61.25	0.10398	7.53701362555243	7.53701362555243\\
61.25	0.10764	8.16824884333281	8.16824884333281\\
61.25	0.1113	8.86587401157558	8.86587401157558\\
61.25	0.11496	9.62988913028074	9.62988913028074\\
61.25	0.11862	10.4602941994483	10.4602941994483\\
61.25	0.12228	11.3570892190782	11.3570892190782\\
61.25	0.12594	12.3202741891705	12.3202741891705\\
61.25	0.1296	13.3498491097252	13.3498491097252\\
61.25	0.13326	14.4458139807422	14.4458139807422\\
61.25	0.13692	15.6081688022217	15.6081688022217\\
61.25	0.14058	16.8369135741635	16.8369135741635\\
61.25	0.14424	18.1320482965677	18.1320482965677\\
61.25	0.1479	19.4935729694343	19.4935729694343\\
61.25	0.15156	20.9214875927632	20.9214875927632\\
61.25	0.15522	22.4157921665546	22.4157921665546\\
61.25	0.15888	23.9764866908083	23.9764866908083\\
61.25	0.16254	25.6035711655244	25.6035711655244\\
61.25	0.1662	27.2970455907029	27.2970455907029\\
61.25	0.16986	29.0569099663438	29.0569099663438\\
61.25	0.17352	30.8831642924471	30.8831642924471\\
61.25	0.17718	32.7758085690127	32.7758085690127\\
61.25	0.18084	34.7348427960407	34.7348427960407\\
61.25	0.1845	36.7602669735311	36.7602669735311\\
61.25	0.18816	38.8520811014839	38.8520811014839\\
61.25	0.19182	41.010285179899	41.010285179899\\
61.25	0.19548	43.2348792087766	43.2348792087766\\
61.25	0.19914	45.5258631881165	45.5258631881165\\
61.25	0.2028	47.8832371179188	47.8832371179188\\
61.25	0.20646	50.3070009981835	50.3070009981835\\
61.25	0.21012	52.7971548289106	52.7971548289106\\
61.25	0.21378	55.3536986101	55.3536986101\\
61.25	0.21744	57.9766323417518	57.9766323417518\\
61.25	0.2211	60.665956023866	60.665956023866\\
61.25	0.22476	63.4216696564426	63.4216696564426\\
61.25	0.22842	66.2437732394816	66.2437732394816\\
61.25	0.23208	69.1322667729829	69.1322667729829\\
61.25	0.23574	72.0871502569467	72.0871502569467\\
61.25	0.2394	75.1084236913728	75.1084236913728\\
61.25	0.24306	78.1960870762613	78.1960870762613\\
61.25	0.24672	81.3501404116122	81.3501404116122\\
61.25	0.25038	84.5705836974255	84.5705836974255\\
61.25	0.25404	87.8574169337011	87.8574169337011\\
61.25	0.2577	91.2106401204391	91.2106401204391\\
61.25	0.26136	94.6302532576395	94.6302532576395\\
61.25	0.26502	98.1162563453023	98.1162563453023\\
61.25	0.26868	101.668649383427	101.668649383427\\
61.25	0.27234	105.287432372015	105.287432372015\\
61.25	0.276	108.972605311065	108.972605311065\\
61.625	0.093	6.09082378652294	6.09082378652294\\
61.625	0.09666	6.52330629224586	6.52330629224586\\
61.625	0.10032	7.02217874843113	7.02217874843113\\
61.625	0.10398	7.58744115507881	7.58744115507881\\
61.625	0.10764	8.21909351218888	8.21909351218888\\
61.625	0.1113	8.91713581976133	8.91713581976133\\
61.625	0.11496	9.68156807779616	9.68156807779616\\
61.625	0.11862	10.5123902862933	10.5123902862933\\
61.625	0.12228	11.4096024452529	11.4096024452529\\
61.625	0.12594	12.3732045546749	12.3732045546749\\
61.625	0.1296	13.4031966145592	13.4031966145592\\
61.625	0.13326	14.499578624906	14.499578624906\\
61.625	0.13692	15.6623505857151	15.6623505857151\\
61.625	0.14058	16.8915124969866	16.8915124969866\\
61.625	0.14424	18.1870643587204	18.1870643587204\\
61.625	0.1479	19.5490061709167	19.5490061709167\\
61.625	0.15156	20.9773379335753	20.9773379335753\\
61.625	0.15522	22.4720596466963	22.4720596466963\\
61.625	0.15888	24.0331713102797	24.0331713102797\\
61.625	0.16254	25.6606729243255	25.6606729243255\\
61.625	0.1662	27.3545644888337	27.3545644888337\\
61.625	0.16986	29.1148460038042	29.1148460038042\\
61.625	0.17352	30.9415174692371	30.9415174692371\\
61.625	0.17718	32.8345788851324	32.8345788851324\\
61.625	0.18084	34.7940302514901	34.7940302514901\\
61.625	0.1845	36.8198715683102	36.8198715683102\\
61.625	0.18816	38.9121028355926	38.9121028355926\\
61.625	0.19182	41.0707240533375	41.0707240533375\\
61.625	0.19548	43.2957352215447	43.2957352215447\\
61.625	0.19914	45.5871363402143	45.5871363402143\\
61.625	0.2028	47.9449274093462	47.9449274093462\\
61.625	0.20646	50.3691084289406	50.3691084289406\\
61.625	0.21012	52.8596793989973	52.8596793989973\\
61.625	0.21378	55.4166403195164	55.4166403195164\\
61.625	0.21744	58.0399911904979	58.0399911904979\\
61.625	0.2211	60.7297320119418	60.7297320119418\\
61.625	0.22476	63.4858627838481	63.4858627838481\\
61.625	0.22842	66.3083835062167	66.3083835062167\\
61.625	0.23208	69.1972941790477	69.1972941790477\\
61.625	0.23574	72.1525948023411	72.1525948023411\\
61.625	0.2394	75.1742853760969	75.1742853760969\\
61.625	0.24306	78.2623659003151	78.2623659003151\\
61.625	0.24672	81.4168363749956	81.4168363749956\\
61.625	0.25038	84.6376968001386	84.6376968001386\\
61.625	0.25404	87.9249471757439	87.9249471757439\\
61.625	0.2577	91.2785875018116	91.2785875018116\\
61.625	0.26136	94.6986177783416	94.6986177783416\\
61.625	0.26502	98.1850380053341	98.1850380053341\\
61.625	0.26868	101.737848182789	101.737848182789\\
61.625	0.27234	105.357048310706	105.357048310706\\
61.625	0.276	109.042638389086	109.042638389086\\
62	0.093	6.14374227325444	6.14374227325444\\
62	0.09666	6.57664191830701	6.57664191830701\\
62	0.10032	7.07593151382199	7.07593151382199\\
62	0.10398	7.64161105979932	7.64161105979932\\
62	0.10764	8.27368055623904	8.27368055623904\\
62	0.1113	8.97214000314113	8.97214000314113\\
62	0.11496	9.73698940050564	9.73698940050564\\
62	0.11862	10.5682287483325	10.5682287483325\\
62	0.12228	11.4658580466218	11.4658580466218\\
62	0.12594	12.4298772953734	12.4298772953734\\
62	0.1296	13.4602864945874	13.4602864945874\\
62	0.13326	14.5570856442638	14.5570856442638\\
62	0.13692	15.7202747444026	15.7202747444026\\
62	0.14058	16.9498537950037	16.9498537950037\\
62	0.14424	18.2458227960673	18.2458227960673\\
62	0.1479	19.6081817475932	19.6081817475932\\
62	0.15156	21.0369306495815	21.0369306495815\\
62	0.15522	22.5320695020322	22.5320695020322\\
62	0.15888	24.0935983049452	24.0935983049452\\
62	0.16254	25.7215170583207	25.7215170583207\\
62	0.1662	27.4158257621585	27.4158257621585\\
62	0.16986	29.1765244164587	29.1765244164587\\
62	0.17352	31.0036130212213	31.0036130212213\\
62	0.17718	32.8970915764463	32.8970915764463\\
62	0.18084	34.8569600821336	34.8569600821336\\
62	0.1845	36.8832185382834	36.8832185382834\\
62	0.18816	38.9758669448955	38.9758669448955\\
62	0.19182	41.13490530197	41.13490530197\\
62	0.19548	43.3603336095068	43.3603336095068\\
62	0.19914	45.6521518675061	45.6521518675061\\
62	0.2028	48.0103600759677	48.0103600759677\\
62	0.20646	50.4349582348918	50.4349582348918\\
62	0.21012	52.9259463442782	52.9259463442782\\
62	0.21378	55.4833244041269	55.4833244041269\\
62	0.21744	58.1070924144381	58.1070924144381\\
62	0.2211	60.7972503752116	60.7972503752116\\
62	0.22476	63.5537982864476	63.5537982864476\\
62	0.22842	66.3767361481459	66.3767361481459\\
62	0.23208	69.2660639603065	69.2660639603065\\
62	0.23574	72.2217817229296	72.2217817229296\\
62	0.2394	75.2438894360151	75.2438894360151\\
62	0.24306	78.3323870995629	78.3323870995629\\
62	0.24672	81.4872747135731	81.4872747135731\\
62	0.25038	84.7085522780457	84.7085522780457\\
62	0.25404	87.9962197929807	87.9962197929807\\
62	0.2577	91.3502772583781	91.3502772583781\\
62	0.26136	94.7707246742378	94.7707246742378\\
62	0.26502	98.2575620405599	98.2575620405599\\
62	0.26868	101.810789357344	101.810789357344\\
62	0.27234	105.430406624591	105.430406624591\\
62	0.276	109.116413842301	109.116413842301\\
62.375	0.093	6.20040313518003	6.20040313518003\\
62.375	0.09666	6.63371991956229	6.63371991956229\\
62.375	0.10032	7.13342665440692	7.13342665440692\\
62.375	0.10398	7.69952333971393	7.69952333971393\\
62.375	0.10764	8.33200997548332	8.33200997548332\\
62.375	0.1113	9.03088656171508	9.03088656171508\\
62.375	0.11496	9.79615309840923	9.79615309840923\\
62.375	0.11862	10.6278095855658	10.6278095855658\\
62.375	0.12228	11.5258560231847	11.5258560231847\\
62.375	0.12594	12.490292411266	12.490292411266\\
62.375	0.1296	13.5211187498097	13.5211187498097\\
62.375	0.13326	14.6183350388157	14.6183350388157\\
62.375	0.13692	15.7819412782842	15.7819412782842\\
62.375	0.14058	17.011937468215	17.011937468215\\
62.375	0.14424	18.3083236086082	18.3083236086082\\
62.375	0.1479	19.6710996994638	19.6710996994638\\
62.375	0.15156	21.1002657407818	21.1002657407818\\
62.375	0.15522	22.5958217325621	22.5958217325621\\
62.375	0.15888	24.1577676748048	24.1577676748048\\
62.375	0.16254	25.78610356751	25.78610356751\\
62.375	0.1662	27.4808294106775	27.4808294106775\\
62.375	0.16986	29.2419452043073	29.2419452043073\\
62.375	0.17352	31.0694509483996	31.0694509483996\\
62.375	0.17718	32.9633466429542	32.9633466429542\\
62.375	0.18084	34.9236322879712	34.9236322879712\\
62.375	0.1845	36.9503078834506	36.9503078834506\\
62.375	0.18816	39.0433734293924	39.0433734293924\\
62.375	0.19182	41.2028289257966	41.2028289257966\\
62.375	0.19548	43.4286743726631	43.4286743726631\\
62.375	0.19914	45.720909769992	45.720909769992\\
62.375	0.2028	48.0795351177834	48.0795351177834\\
62.375	0.20646	50.5045504160371	50.5045504160371\\
62.375	0.21012	52.9959556647531	52.9959556647531\\
62.375	0.21378	55.5537508639316	55.5537508639316\\
62.375	0.21744	58.1779360135724	58.1779360135724\\
62.375	0.2211	60.8685111136756	60.8685111136756\\
62.375	0.22476	63.6254761642412	63.6254761642412\\
62.375	0.22842	66.4488311652692	66.4488311652692\\
62.375	0.23208	69.3385761167595	69.3385761167595\\
62.375	0.23574	72.2947110187123	72.2947110187123\\
62.375	0.2394	75.3172358711274	75.3172358711274\\
62.375	0.24306	78.4061506740049	78.4061506740049\\
62.375	0.24672	81.5614554273448	81.5614554273448\\
62.375	0.25038	84.783150131147	84.783150131147\\
62.375	0.25404	88.0712347854116	88.0712347854116\\
62.375	0.2577	91.4257093901387	91.4257093901387\\
62.375	0.26136	94.8465739453281	94.8465739453281\\
62.375	0.26502	98.3338284509799	98.3338284509799\\
62.375	0.26868	101.887472907094	101.887472907094\\
62.375	0.27234	105.507507313671	105.507507313671\\
62.375	0.276	109.193931670709	109.193931670709\\
62.75	0.093	6.26080637229972	6.26080637229972\\
62.75	0.09666	6.69454029601163	6.69454029601163\\
62.75	0.10032	7.19466417018592	7.19466417018592\\
62.75	0.10398	7.76117799482261	7.76117799482261\\
62.75	0.10764	8.39408176992167	8.39408176992167\\
62.75	0.1113	9.09337549548312	9.09337549548312\\
62.75	0.11496	9.85905917150694	9.85905917150694\\
62.75	0.11862	10.6911327979931	10.6911327979931\\
62.75	0.12228	11.5895963749417	11.5895963749417\\
62.75	0.12594	12.5544499023527	12.5544499023527\\
62.75	0.1296	13.585693380226	13.585693380226\\
62.75	0.13326	14.6833268085617	14.6833268085617\\
62.75	0.13692	15.8473501873599	15.8473501873599\\
62.75	0.14058	17.0777635166204	17.0777635166204\\
62.75	0.14424	18.3745667963432	18.3745667963432\\
62.75	0.1479	19.7377600265285	19.7377600265285\\
62.75	0.15156	21.1673432071761	21.1673432071761\\
62.75	0.15522	22.6633163382861	22.6633163382861\\
62.75	0.15888	24.2256794198585	24.2256794198585\\
62.75	0.16254	25.8544324518933	25.8544324518933\\
62.75	0.1662	27.5495754343905	27.5495754343905\\
62.75	0.16986	29.31110836735	29.31110836735\\
62.75	0.17352	31.1390312507719	31.1390312507719\\
62.75	0.17718	33.0333440846562	33.0333440846562\\
62.75	0.18084	34.9940468690029	34.9940468690029\\
62.75	0.1845	37.021139603812	37.021139603812\\
62.75	0.18816	39.1146222890834	39.1146222890834\\
62.75	0.19182	41.2744949248173	41.2744949248173\\
62.75	0.19548	43.5007575110135	43.5007575110135\\
62.75	0.19914	45.7934100476721	45.7934100476721\\
62.75	0.2028	48.152452534793	48.152452534793\\
62.75	0.20646	50.5778849723764	50.5778849723764\\
62.75	0.21012	53.0697073604221	53.0697073604221\\
62.75	0.21378	55.6279196989302	55.6279196989302\\
62.75	0.21744	58.2525219879007	58.2525219879007\\
62.75	0.2211	60.9435142273336	60.9435142273336\\
62.75	0.22476	63.7008964172289	63.7008964172289\\
62.75	0.22842	66.5246685575865	66.5246685575865\\
62.75	0.23208	69.4148306484065	69.4148306484065\\
62.75	0.23574	72.371382689689	72.371382689689\\
62.75	0.2394	75.3943246814337	75.3943246814337\\
62.75	0.24306	78.4836566236409	78.4836566236409\\
62.75	0.24672	81.6393785163104	81.6393785163104\\
62.75	0.25038	84.8614903594424	84.8614903594424\\
62.75	0.25404	88.1499921530367	88.1499921530367\\
62.75	0.2577	91.5048838970934	91.5048838970934\\
62.75	0.26136	94.9261655916125	94.9261655916125\\
62.75	0.26502	98.4138372365939	98.4138372365939\\
62.75	0.26868	101.967898832038	101.967898832038\\
62.75	0.27234	105.588350377944	105.588350377944\\
62.75	0.276	109.275191874312	109.275191874312\\
63.125	0.093	6.32495198461351	6.32495198461351\\
63.125	0.09666	6.75910304765509	6.75910304765509\\
63.125	0.10032	7.25964406115906	7.25964406115906\\
63.125	0.10398	7.82657502512541	7.82657502512541\\
63.125	0.10764	8.45989593955412	8.45989593955412\\
63.125	0.1113	9.15960680444521	9.15960680444521\\
63.125	0.11496	9.92570761979871	9.92570761979871\\
63.125	0.11862	10.7581983856146	10.7581983856146\\
63.125	0.12228	11.6570791018928	11.6570791018928\\
63.125	0.12594	12.6223497686335	12.6223497686335\\
63.125	0.1296	13.6540103858365	13.6540103858365\\
63.125	0.13326	14.7520609535019	14.7520609535019\\
63.125	0.13692	15.9165014716297	15.9165014716297\\
63.125	0.14058	17.1473319402198	17.1473319402198\\
63.125	0.14424	18.4445523592724	18.4445523592724\\
63.125	0.1479	19.8081627287873	19.8081627287873\\
63.125	0.15156	21.2381630487646	21.2381630487646\\
63.125	0.15522	22.7345533192043	22.7345533192043\\
63.125	0.15888	24.2973335401063	24.2973335401063\\
63.125	0.16254	25.9265037114708	25.9265037114708\\
63.125	0.1662	27.6220638332976	27.6220638332976\\
63.125	0.16986	29.3840139055868	29.3840139055868\\
63.125	0.17352	31.2123539283384	31.2123539283384\\
63.125	0.17718	33.1070839015524	33.1070839015524\\
63.125	0.18084	35.0682038252287	35.0682038252287\\
63.125	0.1845	37.0957136993675	37.0957136993675\\
63.125	0.18816	39.1896135239686	39.1896135239686\\
63.125	0.19182	41.349903299032	41.349903299032\\
63.125	0.19548	43.5765830245579	43.5765830245579\\
63.125	0.19914	45.8696527005462	45.8696527005462\\
63.125	0.2028	48.2291123269968	48.2291123269968\\
63.125	0.20646	50.6549619039099	50.6549619039099\\
63.125	0.21012	53.1472014312853	53.1472014312853\\
63.125	0.21378	55.705830909123	55.705830909123\\
63.125	0.21744	58.3308503374232	58.3308503374232\\
63.125	0.2211	61.0222597161857	61.0222597161857\\
63.125	0.22476	63.7800590454107	63.7800590454107\\
63.125	0.22842	66.604248325098	66.604248325098\\
63.125	0.23208	69.4948275552476	69.4948275552476\\
63.125	0.23574	72.4517967358598	72.4517967358598\\
63.125	0.2394	75.4751558669342	75.4751558669342\\
63.125	0.24306	78.5649049484711	78.5649049484711\\
63.125	0.24672	81.7210439804703	81.7210439804703\\
63.125	0.25038	84.9435729629318	84.9435729629318\\
63.125	0.25404	88.2324918958558	88.2324918958558\\
63.125	0.2577	91.5878007792422	91.5878007792422\\
63.125	0.26136	95.0094996130909	95.0094996130909\\
63.125	0.26502	98.497588397402	98.497588397402\\
63.125	0.26868	102.052067132176	102.052067132176\\
63.125	0.27234	105.672935817411	105.672935817411\\
63.125	0.276	109.36019445311	109.36019445311\\
63.5	0.093	6.39283997212141	6.39283997212141\\
63.5	0.09666	6.82740817449265	6.82740817449265\\
63.5	0.10032	7.32836632732629	7.32836632732629\\
63.5	0.10398	7.89571443062231	7.89571443062231\\
63.5	0.10764	8.52945248438068	8.52945248438068\\
63.5	0.1113	9.22958048860145	9.22958048860145\\
63.5	0.11496	9.99609844328462	9.99609844328462\\
63.5	0.11862	10.8290063484301	10.8290063484301\\
63.5	0.12228	11.7283042040381	11.7283042040381\\
63.5	0.12594	12.6939920101084	12.6939920101084\\
63.5	0.1296	13.726069766641	13.726069766641\\
63.5	0.13326	14.8245374736361	14.8245374736361\\
63.5	0.13692	15.9893951310936	15.9893951310936\\
63.5	0.14058	17.2206427390134	17.2206427390134\\
63.5	0.14424	18.5182802973956	18.5182802973956\\
63.5	0.1479	19.8823078062402	19.8823078062402\\
63.5	0.15156	21.3127252655471	21.3127252655471\\
63.5	0.15522	22.8095326753165	22.8095326753165\\
63.5	0.15888	24.3727300355482	24.3727300355482\\
63.5	0.16254	26.0023173462424	26.0023173462424\\
63.5	0.1662	27.6982946073988	27.6982946073988\\
63.5	0.16986	29.4606618190177	29.4606618190177\\
63.5	0.17352	31.289418981099	31.289418981099\\
63.5	0.17718	33.1845660936426	33.1845660936426\\
63.5	0.18084	35.1461031566486	35.1461031566486\\
63.5	0.1845	37.174030170117	37.174030170117\\
63.5	0.18816	39.2683471340478	39.2683471340478\\
63.5	0.19182	41.429054048441	41.429054048441\\
63.5	0.19548	43.6561509132965	43.6561509132965\\
63.5	0.19914	45.9496377286144	45.9496377286144\\
63.5	0.2028	48.3095144943948	48.3095144943948\\
63.5	0.20646	50.7357812106375	50.7357812106375\\
63.5	0.21012	53.2284378773425	53.2284378773425\\
63.5	0.21378	55.78748449451	55.78748449451\\
63.5	0.21744	58.4129210621398	58.4129210621398\\
63.5	0.2211	61.104747580232	61.104747580232\\
63.5	0.22476	63.8629640487866	63.8629640487866\\
63.5	0.22842	66.6875704678036	66.6875704678036\\
63.5	0.23208	69.5785668372829	69.5785668372829\\
63.5	0.23574	72.5359531572247	72.5359531572247\\
63.5	0.2394	75.5597294276288	75.5597294276288\\
63.5	0.24306	78.6498956484953	78.6498956484953\\
63.5	0.24672	81.8064518198242	81.8064518198242\\
63.5	0.25038	85.0293979416154	85.0293979416154\\
63.5	0.25404	88.3187340138691	88.3187340138691\\
63.5	0.2577	91.6744600365851	91.6744600365851\\
63.5	0.26136	95.0965760097635	95.0965760097635\\
63.5	0.26502	98.5850819334043	98.5850819334043\\
63.5	0.26868	102.139977807507	102.139977807507\\
63.5	0.27234	105.761263632073	105.761263632073\\
63.5	0.276	109.448939407101	109.448939407101\\
63.875	0.093	6.46447033482339	6.46447033482339\\
63.875	0.09666	6.8994556765243	6.8994556765243\\
63.875	0.10032	7.40083096868758	7.40083096868758\\
63.875	0.10398	7.96859621131327	7.96859621131327\\
63.875	0.10764	8.60275140440132	8.60275140440132\\
63.875	0.1113	9.30329654795178	9.30329654795178\\
63.875	0.11496	10.0702316419646	10.0702316419646\\
63.875	0.11862	10.9035566864398	10.9035566864398\\
63.875	0.12228	11.8032716813774	11.8032716813774\\
63.875	0.12594	12.7693766267774	12.7693766267774\\
63.875	0.1296	13.8018715226397	13.8018715226397\\
63.875	0.13326	14.9007563689644	14.9007563689644\\
63.875	0.13692	16.0660311657515	16.0660311657515\\
63.875	0.14058	17.297695913001	17.297695913001\\
63.875	0.14424	18.5957506107129	18.5957506107129\\
63.875	0.1479	19.9601952588872	19.9601952588872\\
63.875	0.15156	21.3910298575238	21.3910298575238\\
63.875	0.15522	22.8882544066228	22.8882544066228\\
63.875	0.15888	24.4518689061842	24.4518689061842\\
63.875	0.16254	26.081873356208	26.081873356208\\
63.875	0.1662	27.7782677566942	27.7782677566942\\
63.875	0.16986	29.5410521076427	29.5410521076427\\
63.875	0.17352	31.3702264090536	31.3702264090536\\
63.875	0.17718	33.2657906609269	33.2657906609269\\
63.875	0.18084	35.2277448632626	35.2277448632626\\
63.875	0.1845	37.2560890160607	37.2560890160607\\
63.875	0.18816	39.3508231193211	39.3508231193211\\
63.875	0.19182	41.511947173044	41.511947173044\\
63.875	0.19548	43.7394611772292	43.7394611772292\\
63.875	0.19914	46.0333651318768	46.0333651318768\\
63.875	0.2028	48.3936590369867	48.3936590369867\\
63.875	0.20646	50.8203428925591	50.8203428925591\\
63.875	0.21012	53.3134166985938	53.3134166985938\\
63.875	0.21378	55.8728804550909	55.8728804550909\\
63.875	0.21744	58.4987341620504	58.4987341620504\\
63.875	0.2211	61.1909778194723	61.1909778194723\\
63.875	0.22476	63.9496114273566	63.9496114273566\\
63.875	0.22842	66.7746349857032	66.7746349857032\\
63.875	0.23208	69.6660484945122	69.6660484945122\\
63.875	0.23574	72.6238519537837	72.6238519537837\\
63.875	0.2394	75.6480453635174	75.6480453635174\\
63.875	0.24306	78.7386287237136	78.7386287237136\\
63.875	0.24672	81.8956020343722	81.8956020343722\\
63.875	0.25038	85.1189652954931	85.1189652954931\\
63.875	0.25404	88.4087185070764	88.4087185070764\\
63.875	0.2577	91.7648616691221	91.7648616691221\\
63.875	0.26136	95.1873947816302	95.1873947816302\\
63.875	0.26502	98.6763178446006	98.6763178446006\\
63.875	0.26868	102.231630858033	102.231630858033\\
63.875	0.27234	105.853333821929	105.853333821929\\
63.875	0.276	109.541426736286	109.541426736286\\
64.25	0.093	6.53984307271947	6.53984307271947\\
64.25	0.09666	6.97524555375006	6.97524555375006\\
64.25	0.10032	7.47703798524302	7.47703798524302\\
64.25	0.10398	8.04522036719836	8.04522036719836\\
64.25	0.10764	8.67979269961609	8.67979269961609\\
64.25	0.1113	9.3807549824962	9.3807549824962\\
64.25	0.11496	10.1481072158387	10.1481072158387\\
64.25	0.11862	10.9818493996436	10.9818493996436\\
64.25	0.12228	11.8819815339108	11.8819815339108\\
64.25	0.12594	12.8485036186405	12.8485036186405\\
64.25	0.1296	13.8814156538325	13.8814156538325\\
64.25	0.13326	14.9807176394869	14.9807176394869\\
64.25	0.13692	16.1464095756036	16.1464095756036\\
64.25	0.14058	17.3784914621828	17.3784914621828\\
64.25	0.14424	18.6769632992243	18.6769632992243\\
64.25	0.1479	20.0418250867283	20.0418250867283\\
64.25	0.15156	21.4730768246946	21.4730768246946\\
64.25	0.15522	22.9707185131232	22.9707185131232\\
64.25	0.15888	24.5347501520143	24.5347501520143\\
64.25	0.16254	26.1651717413678	26.1651717413678\\
64.25	0.1662	27.8619832811836	27.8619832811836\\
64.25	0.16986	29.6251847714618	29.6251847714618\\
64.25	0.17352	31.4547762122024	31.4547762122024\\
64.25	0.17718	33.3507576034054	33.3507576034054\\
64.25	0.18084	35.3131289450707	35.3131289450707\\
64.25	0.1845	37.3418902371984	37.3418902371984\\
64.25	0.18816	39.4370414797886	39.4370414797886\\
64.25	0.19182	41.5985826728411	41.5985826728411\\
64.25	0.19548	43.8265138163559	43.8265138163559\\
64.25	0.19914	46.1208349103332	46.1208349103332\\
64.25	0.2028	48.4815459547728	48.4815459547728\\
64.25	0.20646	50.9086469496749	50.9086469496749\\
64.25	0.21012	53.4021378950393	53.4021378950393\\
64.25	0.21378	55.962018790866	55.962018790866\\
64.25	0.21744	58.5882896371552	58.5882896371552\\
64.25	0.2211	61.2809504339067	61.2809504339067\\
64.25	0.22476	64.0400011811207	64.0400011811207\\
64.25	0.22842	66.865441878797	66.865441878797\\
64.25	0.23208	69.7572725269357	69.7572725269357\\
64.25	0.23574	72.7154931255368	72.7154931255368\\
64.25	0.2394	75.7401036746002	75.7401036746002\\
64.25	0.24306	78.831104174126	78.831104174126\\
64.25	0.24672	81.9884946241143	81.9884946241143\\
64.25	0.25038	85.2122750245649	85.2122750245649\\
64.25	0.25404	88.5024453754778	88.5024453754778\\
64.25	0.2577	91.8590056768532	91.8590056768532\\
64.25	0.26136	95.281955928691	95.281955928691\\
64.25	0.26502	98.7712961309911	98.7712961309911\\
64.25	0.26868	102.327026283754	102.327026283754\\
64.25	0.27234	105.949146386978	105.949146386978\\
64.25	0.276	109.637656440666	109.637656440666\\
64.625	0.093	6.61895818580965	6.61895818580965\\
64.625	0.09666	7.05477780616989	7.05477780616989\\
64.625	0.10032	7.55698737699253	7.55698737699253\\
64.625	0.10398	8.12558689827755	8.12558689827755\\
64.625	0.10764	8.76057637002493	8.76057637002493\\
64.625	0.1113	9.4619557922347	9.4619557922347\\
64.625	0.11496	10.2297251649069	10.2297251649069\\
64.625	0.11862	11.0638844880414	11.0638844880414\\
64.625	0.12228	11.9644337616383	11.9644337616383\\
64.625	0.12594	12.9313729856976	12.9313729856976\\
64.625	0.1296	13.9647021602193	13.9647021602193\\
64.625	0.13326	15.0644212852034	15.0644212852034\\
64.625	0.13692	16.2305303606498	16.2305303606498\\
64.625	0.14058	17.4630293865586	17.4630293865586\\
64.625	0.14424	18.7619183629298	18.7619183629298\\
64.625	0.1479	20.1271972897634	20.1271972897634\\
64.625	0.15156	21.5588661670594	21.5588661670594\\
64.625	0.15522	23.0569249948178	23.0569249948178\\
64.625	0.15888	24.6213737730385	24.6213737730385\\
64.625	0.16254	26.2522125017216	26.2522125017216\\
64.625	0.1662	27.9494411808671	27.9494411808671\\
64.625	0.16986	29.713059810475	29.713059810475\\
64.625	0.17352	31.5430683905452	31.5430683905452\\
64.625	0.17718	33.4394669210779	33.4394669210779\\
64.625	0.18084	35.4022554020729	35.4022554020729\\
64.625	0.1845	37.4314338335303	37.4314338335303\\
64.625	0.18816	39.5270022154501	39.5270022154501\\
64.625	0.19182	41.6889605478322	41.6889605478322\\
64.625	0.19548	43.9173088306768	43.9173088306768\\
64.625	0.19914	46.2120470639837	46.2120470639837\\
64.625	0.2028	48.573175247753	48.573175247753\\
64.625	0.20646	51.0006933819847	51.0006933819847\\
64.625	0.21012	53.4946014666788	53.4946014666788\\
64.625	0.21378	56.0548995018352	56.0548995018352\\
64.625	0.21744	58.681587487454	58.681587487454\\
64.625	0.2211	61.3746654235352	61.3746654235352\\
64.625	0.22476	64.1341333100789	64.1341333100789\\
64.625	0.22842	66.9599911470848	66.9599911470848\\
64.625	0.23208	69.8522389345532	69.8522389345532\\
64.625	0.23574	72.810876672484	72.810876672484\\
64.625	0.2394	75.8359043608771	75.8359043608771\\
64.625	0.24306	78.9273219997326	78.9273219997326\\
64.625	0.24672	82.0851295890505	82.0851295890505\\
64.625	0.25038	85.3093271288307	85.3093271288307\\
64.625	0.25404	88.5999146190734	88.5999146190734\\
64.625	0.2577	91.9568920597784	91.9568920597784\\
64.625	0.26136	95.3802594509458	95.3802594509458\\
64.625	0.26502	98.8700167925755	98.8700167925755\\
64.625	0.26868	102.426164084668	102.426164084668\\
64.625	0.27234	106.048701327222	106.048701327222\\
64.625	0.276	109.737628520239	109.737628520239\\
65	0.093	6.70181567409392	6.70181567409392\\
65	0.09666	7.13805243378384	7.13805243378384\\
65	0.10032	7.64067914393612	7.64067914393612\\
65	0.10398	8.2096958045508	8.2096958045508\\
65	0.10764	8.84510241562787	8.84510241562787\\
65	0.1113	9.54689897716732	9.54689897716732\\
65	0.11496	10.3150854891692	10.3150854891692\\
65	0.11862	11.1496619516333	11.1496619516333\\
65	0.12228	12.0506283645599	12.0506283645599\\
65	0.12594	13.0179847279489	13.0179847279489\\
65	0.1296	14.0517310418002	14.0517310418002\\
65	0.13326	15.151867306114	15.151867306114\\
65	0.13692	16.3183935208901	16.3183935208901\\
65	0.14058	17.5513096861286	17.5513096861286\\
65	0.14424	18.8506158018295	18.8506158018295\\
65	0.1479	20.2163118679927	20.2163118679927\\
65	0.15156	21.6483978846183	21.6483978846183\\
65	0.15522	23.1468738517064	23.1468738517064\\
65	0.15888	24.7117397692568	24.7117397692568\\
65	0.16254	26.3429956372696	26.3429956372696\\
65	0.1662	28.0406414557447	28.0406414557447\\
65	0.16986	29.8046772246823	29.8046772246823\\
65	0.17352	31.6351029440822	31.6351029440822\\
65	0.17718	33.5319186139445	33.5319186139445\\
65	0.18084	35.4951242342691	35.4951242342691\\
65	0.1845	37.5247198050562	37.5247198050562\\
65	0.18816	39.6207053263057	39.6207053263057\\
65	0.19182	41.7830807980175	41.7830807980175\\
65	0.19548	44.0118462201917	44.0118462201917\\
65	0.19914	46.3070015928283	46.3070015928283\\
65	0.2028	48.6685469159273	48.6685469159273\\
65	0.20646	51.0964821894887	51.0964821894887\\
65	0.21012	53.5908074135124	53.5908074135124\\
65	0.21378	56.1515225879985	56.1515225879985\\
65	0.21744	58.778627712947	58.778627712947\\
65	0.2211	61.4721227883579	61.4721227883579\\
65	0.22476	64.2320078142311	64.2320078142311\\
65	0.22842	67.0582827905668	67.0582827905668\\
65	0.23208	69.9509477173648	69.9509477173648\\
65	0.23574	72.9100025946252	72.9100025946252\\
65	0.2394	75.935447422348	75.935447422348\\
65	0.24306	79.0272822005332	79.0272822005332\\
65	0.24672	82.1855069291807	82.1855069291807\\
65	0.25038	85.4101216082907	85.4101216082907\\
65	0.25404	88.701126237863	88.701126237863\\
65	0.2577	92.0585208178977	92.0585208178977\\
65	0.26136	95.4823053483948	95.4823053483948\\
65	0.26502	98.9724798293542	98.9724798293542\\
65	0.26868	102.529044260776	102.529044260776\\
65	0.27234	106.15199864266	106.15199864266\\
65	0.276	109.841342975007	109.841342975007\\
65.375	0.093	6.78841553757231	6.78841553757231\\
65.375	0.09666	7.22506943659188	7.22506943659188\\
65.375	0.10032	7.72811328607385	7.72811328607385\\
65.375	0.10398	8.29754708601818	8.29754708601818\\
65.375	0.10764	8.93337083642492	8.93337083642492\\
65.375	0.1113	9.63558453729403	9.63558453729403\\
65.375	0.11496	10.4041881886255	10.4041881886255\\
65.375	0.11862	11.2391817904194	11.2391817904194\\
65.375	0.12228	12.1405653426757	12.1405653426757\\
65.375	0.12594	13.1083388453943	13.1083388453943\\
65.375	0.1296	14.1425022985753	14.1425022985753\\
65.375	0.13326	15.2430557022187	15.2430557022187\\
65.375	0.13692	16.4099990563245	16.4099990563245\\
65.375	0.14058	17.6433323608926	17.6433323608926\\
65.375	0.14424	18.9430556159232	18.9430556159232\\
65.375	0.1479	20.3091688214161	20.3091688214161\\
65.375	0.15156	21.7416719773714	21.7416719773714\\
65.375	0.15522	23.2405650837891	23.2405650837891\\
65.375	0.15888	24.8058481406692	24.8058481406692\\
65.375	0.16254	26.4375211480116	26.4375211480116\\
65.375	0.1662	28.1355841058164	28.1355841058164\\
65.375	0.16986	29.9000370140836	29.9000370140836\\
65.375	0.17352	31.7308798728132	31.7308798728132\\
65.375	0.17718	33.6281126820052	33.6281126820052\\
65.375	0.18084	35.5917354416595	35.5917354416595\\
65.375	0.1845	37.6217481517763	37.6217481517763\\
65.375	0.18816	39.7181508123554	39.7181508123554\\
65.375	0.19182	41.8809434233969	41.8809434233969\\
65.375	0.19548	44.1101259849008	44.1101259849008\\
65.375	0.19914	46.4056984968671	46.4056984968671\\
65.375	0.2028	48.7676609592957	48.7676609592957\\
65.375	0.20646	51.1960133721867	51.1960133721867\\
65.375	0.21012	53.6907557355401	53.6907557355401\\
65.375	0.21378	56.2518880493559	56.2518880493559\\
65.375	0.21744	58.8794103136341	58.8794103136341\\
65.375	0.2211	61.5733225283746	61.5733225283746\\
65.375	0.22476	64.3336246935775	64.3336246935775\\
65.375	0.22842	67.1603168092429	67.1603168092429\\
65.375	0.23208	70.0533988753705	70.0533988753705\\
65.375	0.23574	73.0128708919606	73.0128708919606\\
65.375	0.2394	76.0387328590131	76.0387328590131\\
65.375	0.24306	79.1309847765279	79.1309847765279\\
65.375	0.24672	82.2896266445051	82.2896266445051\\
65.375	0.25038	85.5146584629447	85.5146584629447\\
65.375	0.25404	88.8060802318467	88.8060802318467\\
65.375	0.2577	92.1638919512111	92.1638919512111\\
65.375	0.26136	95.5880936210378	95.5880936210378\\
65.375	0.26502	99.0786852413269	99.0786852413269\\
65.375	0.26868	102.635666812078	102.635666812078\\
65.375	0.27234	106.259038333292	106.259038333292\\
65.375	0.276	109.948799804969	109.948799804969\\
65.75	0.093	6.87875777624475	6.87875777624475\\
65.75	0.09666	7.31582881459401	7.31582881459401\\
65.75	0.10032	7.81928980340565	7.81928980340565\\
65.75	0.10398	8.38914074267966	8.38914074267966\\
65.75	0.10764	9.02538163241605	9.02538163241605\\
65.75	0.1113	9.72801247261481	9.72801247261481\\
65.75	0.11496	10.497033263276	10.497033263276\\
65.75	0.11862	11.3324440043995	11.3324440043995\\
65.75	0.12228	12.2342446959854	12.2342446959854\\
65.75	0.12594	13.2024353380338	13.2024353380338\\
65.75	0.1296	14.2370159305444	14.2370159305444\\
65.75	0.13326	15.3379864735175	15.3379864735175\\
65.75	0.13692	16.5053469669529	16.5053469669529\\
65.75	0.14058	17.7390974108508	17.7390974108508\\
65.75	0.14424	19.039237805211	19.039237805211\\
65.75	0.1479	20.4057681500336	20.4057681500336\\
65.75	0.15156	21.8386884453185	21.8386884453185\\
65.75	0.15522	23.3379986910659	23.3379986910659\\
65.75	0.15888	24.9036988872756	24.9036988872756\\
65.75	0.16254	26.5357890339477	26.5357890339477\\
65.75	0.1662	28.2342691310822	28.2342691310822\\
65.75	0.16986	29.9991391786791	29.9991391786791\\
65.75	0.17352	31.8303991767384	31.8303991767384\\
65.75	0.17718	33.72804912526	33.72804912526\\
65.75	0.18084	35.692089024244	35.692089024244\\
65.75	0.1845	37.7225188736904	37.7225188736904\\
65.75	0.18816	39.8193386735992	39.8193386735992\\
65.75	0.19182	41.9825484239704	41.9825484239704\\
65.75	0.19548	44.2121481248039	44.2121481248039\\
65.75	0.19914	46.5081377760999	46.5081377760999\\
65.75	0.2028	48.8705173778582	48.8705173778582\\
65.75	0.20646	51.2992869300789	51.2992869300789\\
65.75	0.21012	53.7944464327619	53.7944464327619\\
65.75	0.21378	56.3559958859074	56.3559958859074\\
65.75	0.21744	58.9839352895152	58.9839352895152\\
65.75	0.2211	61.6782646435854	61.6782646435854\\
65.75	0.22476	64.438983948118	64.438983948118\\
65.75	0.22842	67.266093203113	67.266093203113\\
65.75	0.23208	70.1595924085703	70.1595924085703\\
65.75	0.23574	73.1194815644901	73.1194815644901\\
65.75	0.2394	76.1457606708722	76.1457606708722\\
65.75	0.24306	79.2384297277167	79.2384297277167\\
65.75	0.24672	82.3974887350236	82.3974887350236\\
65.75	0.25038	85.6229376927929	85.6229376927929\\
65.75	0.25404	88.9147766010245	88.9147766010245\\
65.75	0.2577	92.2730054597185	92.2730054597185\\
65.75	0.26136	95.697624268875	95.697624268875\\
65.75	0.26502	99.1886330284937	99.1886330284937\\
65.75	0.26868	102.746031738575	102.746031738575\\
65.75	0.27234	106.369820399118	106.369820399118\\
65.75	0.276	110.059999010124	110.059999010124\\
66.125	0.093	6.97284239011132	6.97284239011132\\
66.125	0.09666	7.41033056779024	7.41033056779024\\
66.125	0.10032	7.91420869593153	7.91420869593153\\
66.125	0.10398	8.48447677453522	8.48447677453522\\
66.125	0.10764	9.12113480360129	9.12113480360129\\
66.125	0.1113	9.82418278312974	9.82418278312974\\
66.125	0.11496	10.5936207131206	10.5936207131206\\
66.125	0.11862	11.4294485935738	11.4294485935738\\
66.125	0.12228	12.3316664244894	12.3316664244894\\
66.125	0.12594	13.3002742058673	13.3002742058673\\
66.125	0.1296	14.3352719377077	14.3352719377077\\
66.125	0.13326	15.4366596200104	15.4366596200104\\
66.125	0.13692	16.6044372527755	16.6044372527755\\
66.125	0.14058	17.838604836003	17.838604836003\\
66.125	0.14424	19.1391623696929	19.1391623696929\\
66.125	0.1479	20.5061098538451	20.5061098538451\\
66.125	0.15156	21.9394472884598	21.9394472884598\\
66.125	0.15522	23.4391746735368	23.4391746735368\\
66.125	0.15888	25.0052920090762	25.0052920090762\\
66.125	0.16254	26.637799295078	26.637799295078\\
66.125	0.1662	28.3366965315422	28.3366965315422\\
66.125	0.16986	30.1019837184687	30.1019837184687\\
66.125	0.17352	31.9336608558576	31.9336608558576\\
66.125	0.17718	33.8317279437089	33.8317279437089\\
66.125	0.18084	35.7961849820226	35.7961849820226\\
66.125	0.1845	37.8270319707987	37.8270319707987\\
66.125	0.18816	39.9242689100371	39.9242689100371\\
66.125	0.19182	42.087895799738	42.087895799738\\
66.125	0.19548	44.3179126399012	44.3179126399012\\
66.125	0.19914	46.6143194305268	46.6143194305268\\
66.125	0.2028	48.9771161716147	48.9771161716147\\
66.125	0.20646	51.4063028631651	51.4063028631651\\
66.125	0.21012	53.9018795051778	53.9018795051778\\
66.125	0.21378	56.4638460976529	56.4638460976529\\
66.125	0.21744	59.0922026405905	59.0922026405905\\
66.125	0.2211	61.7869491339903	61.7869491339903\\
66.125	0.22476	64.5480855778526	64.5480855778526\\
66.125	0.22842	67.3756119721772	67.3756119721772\\
66.125	0.23208	70.2695283169643	70.2695283169643\\
66.125	0.23574	73.2298346122137	73.2298346122137\\
66.125	0.2394	76.2565308579255	76.2565308579255\\
66.125	0.24306	79.3496170540996	79.3496170540996\\
66.125	0.24672	82.5090932007362	82.5090932007362\\
66.125	0.25038	85.7349592978351	85.7349592978351\\
66.125	0.25404	89.0272153453964	89.0272153453964\\
66.125	0.2577	92.3858613434201	92.3858613434201\\
66.125	0.26136	95.8108972919062	95.8108972919062\\
66.125	0.26502	99.3023231908547	99.3023231908547\\
66.125	0.26868	102.860139040266	102.860139040266\\
66.125	0.27234	106.484344840139	106.484344840139\\
66.125	0.276	110.174940590474	110.174940590474\\
66.5	0.093	7.070669379172	7.070669379172\\
66.5	0.09666	7.50857469618058	7.50857469618058\\
66.5	0.10032	8.01286996365154	8.01286996365154\\
66.5	0.10398	8.58355518158489	8.58355518158489\\
66.5	0.10764	9.22063034998062	9.22063034998062\\
66.5	0.1113	9.92409546883873	9.92409546883873\\
66.5	0.11496	10.6939505381592	10.6939505381592\\
66.5	0.11862	11.5301955579421	11.5301955579421\\
66.5	0.12228	12.4328305281874	12.4328305281874\\
66.5	0.12594	13.401855448895	13.401855448895\\
66.5	0.1296	14.437270320065	14.437270320065\\
66.5	0.13326	15.5390751416974	15.5390751416974\\
66.5	0.13692	16.7072699137922	16.7072699137922\\
66.5	0.14058	17.9418546363493	17.9418546363493\\
66.5	0.14424	19.2428293093689	19.2428293093689\\
66.5	0.1479	20.6101939328508	20.6101939328508\\
66.5	0.15156	22.0439485067951	22.0439485067951\\
66.5	0.15522	23.5440930312018	23.5440930312018\\
66.5	0.15888	25.1106275060709	25.1106275060709\\
66.5	0.16254	26.7435519314023	26.7435519314023\\
66.5	0.1662	28.4428663071962	28.4428663071962\\
66.5	0.16986	30.2085706334524	30.2085706334524\\
66.5	0.17352	32.0406649101709	32.0406649101709\\
66.5	0.17718	33.9391491373519	33.9391491373519\\
66.5	0.18084	35.9040233149953	35.9040233149953\\
66.5	0.1845	37.935287443101	37.935287443101\\
66.5	0.18816	40.0329415216691	40.0329415216691\\
66.5	0.19182	42.1969855506996	42.1969855506996\\
66.5	0.19548	44.4274195301925	44.4274195301925\\
66.5	0.19914	46.7242434601478	46.7242434601478\\
66.5	0.2028	49.0874573405654	49.0874573405654\\
66.5	0.20646	51.5170611714454	51.5170611714454\\
66.5	0.21012	54.0130549527878	54.0130549527878\\
66.5	0.21378	56.5754386845926	56.5754386845926\\
66.5	0.21744	59.2042123668598	59.2042123668598\\
66.5	0.2211	61.8993759995893	61.8993759995893\\
66.5	0.22476	64.6609295827813	64.6609295827813\\
66.5	0.22842	67.4888731164356	67.4888731164356\\
66.5	0.23208	70.3832066005523	70.3832066005523\\
66.5	0.23574	73.3439300351314	73.3439300351314\\
66.5	0.2394	76.3710434201728	76.3710434201728\\
66.5	0.24306	79.4645467556766	79.4645467556766\\
66.5	0.24672	82.6244400416429	82.6244400416429\\
66.5	0.25038	85.8507232780715	85.8507232780715\\
66.5	0.25404	89.1433964649625	89.1433964649625\\
66.5	0.2577	92.5024596023158	92.5024596023158\\
66.5	0.26136	95.9279126901315	95.9279126901315\\
66.5	0.26502	99.4197557284097	99.4197557284097\\
66.5	0.26868	102.97798871715	102.97798871715\\
66.5	0.27234	106.602611656353	106.602611656353\\
66.5	0.276	110.293624546018	110.293624546018\\
66.875	0.093	7.17223874342676	7.17223874342676\\
66.875	0.09666	7.610561199765	7.610561199765\\
66.875	0.10032	8.11527360656565	8.11527360656565\\
66.875	0.10398	8.68637596382867	8.68637596382867\\
66.875	0.10764	9.32386827155404	9.32386827155404\\
66.875	0.1113	10.0277505297418	10.0277505297418\\
66.875	0.11496	10.798022738392	10.798022738392\\
66.875	0.11862	11.6346848975045	11.6346848975045\\
66.875	0.12228	12.5377370070794	12.5377370070794\\
66.875	0.12594	13.5071790671168	13.5071790671168\\
66.875	0.1296	14.5430110776164	14.5430110776164\\
66.875	0.13326	15.6452330385785	15.6452330385785\\
66.875	0.13692	16.8138449500029	16.8138449500029\\
66.875	0.14058	18.0488468118898	18.0488468118898\\
66.875	0.14424	19.350238624239	19.350238624239\\
66.875	0.1479	20.7180203870506	20.7180203870506\\
66.875	0.15156	22.1521921003245	22.1521921003245\\
66.875	0.15522	23.6527537640609	23.6527537640609\\
66.875	0.15888	25.2197053782596	25.2197053782596\\
66.875	0.16254	26.8530469429208	26.8530469429208\\
66.875	0.1662	28.5527784580443	28.5527784580443\\
66.875	0.16986	30.3188999236301	30.3188999236301\\
66.875	0.17352	32.1514113396784	32.1514113396784\\
66.875	0.17718	34.050312706189	34.050312706189\\
66.875	0.18084	36.015604023162	36.015604023162\\
66.875	0.1845	38.0472852905975	38.0472852905975\\
66.875	0.18816	40.1453565084952	40.1453565084952\\
66.875	0.19182	42.3098176768554	42.3098176768554\\
66.875	0.19548	44.5406687956779	44.5406687956779\\
66.875	0.19914	46.8379098649629	46.8379098649629\\
66.875	0.2028	49.2015408847102	49.2015408847102\\
66.875	0.20646	51.6315618549199	51.6315618549199\\
66.875	0.21012	54.127972775592	54.127972775592\\
66.875	0.21378	56.6907736467264	56.6907736467264\\
66.875	0.21744	59.3199644683232	59.3199644683232\\
66.875	0.2211	62.0155452403824	62.0155452403824\\
66.875	0.22476	64.777515962904	64.777515962904\\
66.875	0.22842	67.605876635888	67.605876635888\\
66.875	0.23208	70.5006272593344	70.5006272593344\\
66.875	0.23574	73.4617678332432	73.4617678332432\\
66.875	0.2394	76.4892983576143	76.4892983576143\\
66.875	0.24306	79.5832188324478	79.5832188324478\\
66.875	0.24672	82.7435292577437	82.7435292577437\\
66.875	0.25038	85.9702296335019	85.9702296335019\\
66.875	0.25404	89.2633199597226	89.2633199597226\\
66.875	0.2577	92.6228002364056	92.6228002364056\\
66.875	0.26136	96.048670463551	96.048670463551\\
66.875	0.26502	99.5409306411588	99.5409306411588\\
66.875	0.26868	103.099580769229	103.099580769229\\
66.875	0.27234	106.724620847761	106.724620847761\\
66.875	0.276	110.416050876756	110.416050876756\\
67.25	0.093	7.2775504828756	7.2775504828756\\
67.25	0.09666	7.71629007854352	7.71629007854352\\
67.25	0.10032	8.22141962467381	8.22141962467381\\
67.25	0.10398	8.7929391212665	8.7929391212665\\
67.25	0.10764	9.43084856832156	9.43084856832156\\
67.25	0.1113	10.135147965839	10.135147965839\\
67.25	0.11496	10.9058373138188	10.9058373138188\\
67.25	0.11862	11.7429166122611	11.7429166122611\\
67.25	0.12228	12.6463858611656	12.6463858611656\\
67.25	0.12594	13.6162450605326	13.6162450605326\\
67.25	0.1296	14.6524942103619	14.6524942103619\\
67.25	0.13326	15.7551333106537	15.7551333106537\\
67.25	0.13692	16.9241623614078	16.9241623614078\\
67.25	0.14058	18.1595813626243	18.1595813626243\\
67.25	0.14424	19.4613903143032	19.4613903143032\\
67.25	0.1479	20.8295892164444	20.8295892164444\\
67.25	0.15156	22.2641780690481	22.2641780690481\\
67.25	0.15522	23.7651568721141	23.7651568721141\\
67.25	0.15888	25.3325256256425	25.3325256256425\\
67.25	0.16254	26.9662843296333	26.9662843296333\\
67.25	0.1662	28.6664329840864	28.6664329840864\\
67.25	0.16986	30.432971589002	30.432971589002\\
67.25	0.17352	32.2659001443799	32.2659001443799\\
67.25	0.17718	34.1652186502202	34.1652186502202\\
67.25	0.18084	36.1309271065229	36.1309271065229\\
67.25	0.1845	38.163025513288	38.163025513288\\
67.25	0.18816	40.2615138705154	40.2615138705154\\
67.25	0.19182	42.4263921782053	42.4263921782053\\
67.25	0.19548	44.6576604363575	44.6576604363575\\
67.25	0.19914	46.9553186449721	46.9553186449721\\
67.25	0.2028	49.319366804049	49.319366804049\\
67.25	0.20646	51.7498049135884	51.7498049135884\\
67.25	0.21012	54.2466329735901	54.2466329735901\\
67.25	0.21378	56.8098509840543	56.8098509840543\\
67.25	0.21744	59.4394589449807	59.4394589449807\\
67.25	0.2211	62.1354568563696	62.1354568563696\\
67.25	0.22476	64.8978447182209	64.8978447182209\\
67.25	0.22842	67.7266225305346	67.7266225305346\\
67.25	0.23208	70.6217902933106	70.6217902933106\\
67.25	0.23574	73.583348006549	73.583348006549\\
67.25	0.2394	76.6112956702498	76.6112956702498\\
67.25	0.24306	79.705633284413	79.705633284413\\
67.25	0.24672	82.8663608490385	82.8663608490385\\
67.25	0.25038	86.0934783641264	86.0934783641264\\
67.25	0.25404	89.3869858296768	89.3869858296768\\
67.25	0.2577	92.7468832456895	92.7468832456895\\
67.25	0.26136	96.1731706121645	96.1731706121645\\
67.25	0.26502	99.665847929102	99.665847929102\\
67.25	0.26868	103.224915196502	103.224915196502\\
67.25	0.27234	106.850372414364	106.850372414364\\
67.25	0.276	110.542219582689	110.542219582689\\
67.625	0.093	7.38660459751857	7.38660459751857\\
67.625	0.09666	7.82576133251617	7.82576133251617\\
67.625	0.10032	8.33130801797612	8.33130801797612\\
67.625	0.10398	8.90324465389847	8.90324465389847\\
67.625	0.10764	9.5415712402832	9.5415712402832\\
67.625	0.1113	10.2462877771303	10.2462877771303\\
67.625	0.11496	11.0173942644398	11.0173942644398\\
67.625	0.11862	11.8548907022117	11.8548907022117\\
67.625	0.12228	12.7587770904459	12.7587770904459\\
67.625	0.12594	13.7290534291426	13.7290534291426\\
67.625	0.1296	14.7657197183016	14.7657197183016\\
67.625	0.13326	15.868775957923	15.868775957923\\
67.625	0.13692	17.0382221480068	17.0382221480068\\
67.625	0.14058	18.2740582885529	18.2740582885529\\
67.625	0.14424	19.5762843795615	19.5762843795615\\
67.625	0.1479	20.9449004210324	20.9449004210324\\
67.625	0.15156	22.3799064129657	22.3799064129657\\
67.625	0.15522	23.8813023553614	23.8813023553614\\
67.625	0.15888	25.4490882482195	25.4490882482195\\
67.625	0.16254	27.0832640915399	27.0832640915399\\
67.625	0.1662	28.7838298853228	28.7838298853228\\
67.625	0.16986	30.550785629568	30.550785629568\\
67.625	0.17352	32.3841313242756	32.3841313242756\\
67.625	0.17718	34.2838669694455	34.2838669694455\\
67.625	0.18084	36.2499925650779	36.2499925650779\\
67.625	0.1845	38.2825081111726	38.2825081111726\\
67.625	0.18816	40.3814136077297	40.3814136077297\\
67.625	0.19182	42.5467090547492	42.5467090547492\\
67.625	0.19548	44.7783944522311	44.7783944522311\\
67.625	0.19914	47.0764698001754	47.0764698001754\\
67.625	0.2028	49.440935098582	49.440935098582\\
67.625	0.20646	51.8717903474511	51.8717903474511\\
67.625	0.21012	54.3690355467824	54.3690355467824\\
67.625	0.21378	56.9326706965762	56.9326706965762\\
67.625	0.21744	59.5626957968324	59.5626957968324\\
67.625	0.2211	62.259110847551	62.259110847551\\
67.625	0.22476	65.0219158487319	65.0219158487319\\
67.625	0.22842	67.8511108003752	67.8511108003752\\
67.625	0.23208	70.7466957024809	70.7466957024809\\
67.625	0.23574	73.708670555049	73.708670555049\\
67.625	0.2394	76.7370353580794	76.7370353580794\\
67.625	0.24306	79.8317901115722	79.8317901115722\\
67.625	0.24672	82.9929348155275	82.9929348155275\\
67.625	0.25038	86.2204694699451	86.2204694699451\\
67.625	0.25404	89.5143940748251	89.5143940748251\\
67.625	0.2577	92.8747086301674	92.8747086301674\\
67.625	0.26136	96.3014131359722	96.3014131359722\\
67.625	0.26502	99.7945075922393	99.7945075922393\\
67.625	0.26868	103.353991998969	103.353991998969\\
67.625	0.27234	106.979866356161	106.979866356161\\
67.625	0.276	110.672130663815	110.672130663815\\
68	0.093	7.49940108735561	7.49940108735561\\
68	0.09666	7.93897496168287	7.93897496168287\\
68	0.10032	8.44493878647251	8.44493878647251\\
68	0.10398	9.01729256172452	9.01729256172452\\
68	0.10764	9.65603628743891	9.65603628743891\\
68	0.1113	10.3611699636157	10.3611699636157\\
68	0.11496	11.1326935902548	11.1326935902548\\
68	0.11862	11.9706071673564	11.9706071673564\\
68	0.12228	12.8749106949203	12.8749106949203\\
68	0.12594	13.8456041729466	13.8456041729466\\
68	0.1296	14.8826876014353	14.8826876014353\\
68	0.13326	15.9861609803864	15.9861609803864\\
68	0.13692	17.1560243097998	17.1560243097998\\
68	0.14058	18.3922775896756	18.3922775896756\\
68	0.14424	19.6949208200139	19.6949208200139\\
68	0.1479	21.0639540008144	21.0639540008144\\
68	0.15156	22.4993771320774	22.4993771320774\\
68	0.15522	24.0011902138028	24.0011902138028\\
68	0.15888	25.5693932459905	25.5693932459905\\
68	0.16254	27.2039862286406	27.2039862286406\\
68	0.1662	28.9049691617531	28.9049691617531\\
68	0.16986	30.672342045328	30.672342045328\\
68	0.17352	32.5061048793653	32.5061048793653\\
68	0.17718	34.4062576638649	34.4062576638649\\
68	0.18084	36.3728003988269	36.3728003988269\\
68	0.1845	38.4057330842514	38.4057330842514\\
68	0.18816	40.5050557201381	40.5050557201381\\
68	0.19182	42.6707683064873	42.6707683064873\\
68	0.19548	44.9028708432988	44.9028708432988\\
68	0.19914	47.2013633305728	47.2013633305728\\
68	0.2028	49.5662457683091	49.5662457683091\\
68	0.20646	51.9975181565078	51.9975181565078\\
68	0.21012	54.4951804951689	54.4951804951689\\
68	0.21378	57.0592327842923	57.0592327842923\\
68	0.21744	59.6896750238781	59.6896750238781\\
68	0.2211	62.3865072139263	62.3865072139263\\
68	0.22476	65.1497293544369	65.1497293544369\\
68	0.22842	67.9793414454099	67.9793414454099\\
68	0.23208	70.8753434868453	70.8753434868453\\
68	0.23574	73.837735478743	73.837735478743\\
68	0.2394	76.8665174211032	76.8665174211032\\
68	0.24306	79.9616893139257	79.9616893139257\\
68	0.24672	83.1232511572106	83.1232511572106\\
68	0.25038	86.3512029509578	86.3512029509578\\
68	0.25404	89.6455446951675	89.6455446951675\\
68	0.2577	93.0062763898395	93.0062763898395\\
68	0.26136	96.4333980349739	96.4333980349739\\
68	0.26502	99.9269096305707	99.9269096305707\\
68	0.26868	103.48681117663	103.48681117663\\
68	0.27234	107.113102673151	107.113102673151\\
68	0.276	110.805784120135	110.805784120135\\
68.375	0.093	7.61593995238678	7.61593995238678\\
68.375	0.09666	8.0559309660437	8.0559309660437\\
68.375	0.10032	8.562311930163	8.562311930163\\
68.375	0.10398	9.13508284474469	9.13508284474469\\
68.375	0.10764	9.77424370978874	9.77424370978874\\
68.375	0.1113	10.4797945252952	10.4797945252952\\
68.375	0.11496	11.251735291264	11.251735291264\\
68.375	0.11862	12.0900660076952	12.0900660076952\\
68.375	0.12228	12.9947866745888	12.9947866745888\\
68.375	0.12594	13.9658972919448	13.9658972919448\\
68.375	0.1296	15.0033978597631	15.0033978597631\\
68.375	0.13326	16.1072883780439	16.1072883780439\\
68.375	0.13692	17.277568846787	17.277568846787\\
68.375	0.14058	18.5142392659925	18.5142392659925\\
68.375	0.14424	19.8172996356604	19.8172996356604\\
68.375	0.1479	21.1867499557906	21.1867499557906\\
68.375	0.15156	22.6225902263833	22.6225902263833\\
68.375	0.15522	24.1248204474383	24.1248204474383\\
68.375	0.15888	25.6934406189557	25.6934406189557\\
68.375	0.16254	27.3284507409355	27.3284507409355\\
68.375	0.1662	29.0298508133776	29.0298508133776\\
68.375	0.16986	30.7976408362822	30.7976408362822\\
68.375	0.17352	32.6318208096491	32.6318208096491\\
68.375	0.17718	34.5323907334784	34.5323907334784\\
68.375	0.18084	36.4993506077701	36.4993506077701\\
68.375	0.1845	38.5327004325242	38.5327004325242\\
68.375	0.18816	40.6324402077406	40.6324402077406\\
68.375	0.19182	42.7985699334195	42.7985699334195\\
68.375	0.19548	45.0310896095607	45.0310896095607\\
68.375	0.19914	47.3299992361643	47.3299992361643\\
68.375	0.2028	49.6952988132303	49.6952988132303\\
68.375	0.20646	52.1269883407586	52.1269883407586\\
68.375	0.21012	54.6250678187494	54.6250678187494\\
68.375	0.21378	57.1895372472025	57.1895372472025\\
68.375	0.21744	59.8203966261179	59.8203966261179\\
68.375	0.2211	62.5176459554958	62.5176459554958\\
68.375	0.22476	65.2812852353361	65.2812852353361\\
68.375	0.22842	68.1113144656388	68.1113144656388\\
68.375	0.23208	71.0077336464038	71.0077336464038\\
68.375	0.23574	73.9705427776312	73.9705427776312\\
68.375	0.2394	76.999741859321	76.999741859321\\
68.375	0.24306	80.0953308914732	80.0953308914732\\
68.375	0.24672	83.2573098740877	83.2573098740877\\
68.375	0.25038	86.4856788071646	86.4856788071646\\
68.375	0.25404	89.780437690704	89.780437690704\\
68.375	0.2577	93.1415865247057	93.1415865247057\\
68.375	0.26136	96.5691253091697	96.5691253091697\\
68.375	0.26502	100.063054044096	100.063054044096\\
68.375	0.26868	103.623372729485	103.623372729485\\
68.375	0.27234	107.250081365336	107.250081365336\\
68.375	0.276	110.94317995165	110.94317995165\\
68.75	0.093	7.73622119261202	7.73622119261202\\
68.75	0.09666	8.1766293455986	8.1766293455986\\
68.75	0.10032	8.68342744904757	8.68342744904757\\
68.75	0.10398	9.25661550295891	9.25661550295891\\
68.75	0.10764	9.89619350733265	9.89619350733265\\
68.75	0.1113	10.6021614621688	10.6021614621688\\
68.75	0.11496	11.3745193674673	11.3745193674673\\
68.75	0.11862	12.2132672232281	12.2132672232281\\
68.75	0.12228	13.1184050294514	13.1184050294514\\
68.75	0.12594	14.089932786137	14.089932786137\\
68.75	0.1296	15.127850493285	15.127850493285\\
68.75	0.13326	16.2321581508954	16.2321581508954\\
68.75	0.13692	17.4028557589682	17.4028557589682\\
68.75	0.14058	18.6399433175034	18.6399433175034\\
68.75	0.14424	19.9434208265009	19.9434208265009\\
68.75	0.1479	21.3132882859609	21.3132882859609\\
68.75	0.15156	22.7495456958832	22.7495456958832\\
68.75	0.15522	24.2521930562679	24.2521930562679\\
68.75	0.15888	25.8212303671149	25.8212303671149\\
68.75	0.16254	27.4566576284244	27.4566576284244\\
68.75	0.1662	29.1584748401962	29.1584748401962\\
68.75	0.16986	30.9266820024304	30.9266820024304\\
68.75	0.17352	32.761279115127	32.761279115127\\
68.75	0.17718	34.662266178286	34.662266178286\\
68.75	0.18084	36.6296431919073	36.6296431919073\\
68.75	0.1845	38.6634101559911	38.6634101559911\\
68.75	0.18816	40.7635670705372	40.7635670705372\\
68.75	0.19182	42.9301139355457	42.9301139355457\\
68.75	0.19548	45.1630507510166	45.1630507510166\\
68.75	0.19914	47.4623775169499	47.4623775169499\\
68.75	0.2028	49.8280942333455	49.8280942333455\\
68.75	0.20646	52.2602009002035	52.2602009002035\\
68.75	0.21012	54.7586975175239	54.7586975175239\\
68.75	0.21378	57.3235840853067	57.3235840853067\\
68.75	0.21744	59.9548606035519	59.9548606035519\\
68.75	0.2211	62.6525270722594	62.6525270722594\\
68.75	0.22476	65.4165834914294	65.4165834914294\\
68.75	0.22842	68.2470298610617	68.2470298610617\\
68.75	0.23208	71.1438661811564	71.1438661811564\\
68.75	0.23574	74.1070924517135	74.1070924517135\\
68.75	0.2394	77.1367086727329	77.1367086727329\\
68.75	0.24306	80.2327148442147	80.2327148442147\\
68.75	0.24672	83.395110966159	83.395110966159\\
68.75	0.25038	86.6238970385656	86.6238970385656\\
68.75	0.25404	89.9190730614346	89.9190730614346\\
68.75	0.2577	93.280639034766	93.280639034766\\
68.75	0.26136	96.7085949585597	96.7085949585597\\
68.75	0.26502	100.202940832816	100.202940832816\\
68.75	0.26868	103.763676657534	103.763676657534\\
68.75	0.27234	107.390802432715	107.390802432715\\
68.75	0.276	111.084318158358	111.084318158358\\
69.125	0.093	7.86024480803135	7.86024480803135\\
69.125	0.09666	8.30107010034762	8.30107010034762\\
69.125	0.10032	8.80828534312625	8.80828534312625\\
69.125	0.10398	9.38189053636727	9.38189053636727\\
69.125	0.10764	10.0218856800707	10.0218856800707\\
69.125	0.1113	10.7282707742364	10.7282707742364\\
69.125	0.11496	11.5010458188646	11.5010458188646\\
69.125	0.11862	12.3402108139551	12.3402108139551\\
69.125	0.12228	13.2457657595081	13.2457657595081\\
69.125	0.12594	14.2177106555234	14.2177106555234\\
69.125	0.1296	15.2560455020011	15.2560455020011\\
69.125	0.13326	16.3607702989411	16.3607702989411\\
69.125	0.13692	17.5318850463436	17.5318850463436\\
69.125	0.14058	18.7693897442084	18.7693897442084\\
69.125	0.14424	20.0732843925356	20.0732843925356\\
69.125	0.1479	21.4435689913252	21.4435689913252\\
69.125	0.15156	22.8802435405772	22.8802435405772\\
69.125	0.15522	24.3833080402915	24.3833080402915\\
69.125	0.15888	25.9527624904683	25.9527624904683\\
69.125	0.16254	27.5886068911074	27.5886068911074\\
69.125	0.1662	29.2908412422089	29.2908412422089\\
69.125	0.16986	31.0594655437728	31.0594655437728\\
69.125	0.17352	32.894479795799	32.894479795799\\
69.125	0.17718	34.7958839982877	34.7958839982877\\
69.125	0.18084	36.7636781512387	36.7636781512387\\
69.125	0.1845	38.7978622546521	38.7978622546521\\
69.125	0.18816	40.8984363085279	40.8984363085279\\
69.125	0.19182	43.065400312866	43.065400312866\\
69.125	0.19548	45.2987542676666	45.2987542676666\\
69.125	0.19914	47.5984981729295	47.5984981729295\\
69.125	0.2028	49.9646320286549	49.9646320286549\\
69.125	0.20646	52.3971558348426	52.3971558348426\\
69.125	0.21012	54.8960695914926	54.8960695914926\\
69.125	0.21378	57.4613732986051	57.4613732986051\\
69.125	0.21744	60.0930669561799	60.0930669561799\\
69.125	0.2211	62.7911505642171	62.7911505642171\\
69.125	0.22476	65.5556241227167	65.5556241227167\\
69.125	0.22842	68.3864876316787	68.3864876316787\\
69.125	0.23208	71.283741091103	71.283741091103\\
69.125	0.23574	74.2473845009898	74.2473845009898\\
69.125	0.2394	77.277417861339	77.277417861339\\
69.125	0.24306	80.3738411721505	80.3738411721505\\
69.125	0.24672	83.5366544334243	83.5366544334243\\
69.125	0.25038	86.7658576451606	86.7658576451606\\
69.125	0.25404	90.0614508073593	90.0614508073593\\
69.125	0.2577	93.4234339200203	93.4234339200203\\
69.125	0.26136	96.8518069831437	96.8518069831437\\
69.125	0.26502	100.346569996729	100.346569996729\\
69.125	0.26868	103.907722960778	103.907722960778\\
69.125	0.27234	107.535265875288	107.535265875288\\
69.125	0.276	111.229198740261	111.229198740261\\
69.5	0.093	7.9880107986448	7.9880107986448\\
69.5	0.09666	8.42925323029072	8.42925323029072\\
69.5	0.10032	8.93688561239903	8.93688561239903\\
69.5	0.10398	9.51090794496972	9.51090794496972\\
69.5	0.10764	10.1513202280028	10.1513202280028\\
69.5	0.1113	10.8581224614982	10.8581224614982\\
69.5	0.11496	11.631314645456	11.631314645456\\
69.5	0.11862	12.4708967798762	12.4708967798762\\
69.5	0.12228	13.3768688647588	13.3768688647588\\
69.5	0.12594	14.3492309001038	14.3492309001038\\
69.5	0.1296	15.3879828859112	15.3879828859112\\
69.5	0.13326	16.4931248221809	16.4931248221809\\
69.5	0.13692	17.664656708913	17.664656708913\\
69.5	0.14058	18.9025785461075	18.9025785461075\\
69.5	0.14424	20.2068903337644	20.2068903337644\\
69.5	0.1479	21.5775920718837	21.5775920718837\\
69.5	0.15156	23.0146837604653	23.0146837604653\\
69.5	0.15522	24.5181653995093	24.5181653995093\\
69.5	0.15888	26.0880369890157	26.0880369890157\\
69.5	0.16254	27.7242985289845	27.7242985289845\\
69.5	0.1662	29.4269500194157	29.4269500194157\\
69.5	0.16986	31.1959914603092	31.1959914603092\\
69.5	0.17352	33.0314228516652	33.0314228516652\\
69.5	0.17718	34.9332441934835	34.9332441934835\\
69.5	0.18084	36.9014554857641	36.9014554857641\\
69.5	0.1845	38.9360567285072	38.9360567285072\\
69.5	0.18816	41.0370479217127	41.0370479217127\\
69.5	0.19182	43.2044290653805	43.2044290653805\\
69.5	0.19548	45.4382001595107	45.4382001595107\\
69.5	0.19914	47.7383612041033	47.7383612041033\\
69.5	0.2028	50.1049121991583	50.1049121991583\\
69.5	0.20646	52.5378531446757	52.5378531446757\\
69.5	0.21012	55.0371840406554	55.0371840406554\\
69.5	0.21378	57.6029048870975	57.6029048870975\\
69.5	0.21744	60.235015684002	60.235015684002\\
69.5	0.2211	62.9335164313689	62.9335164313689\\
69.5	0.22476	65.6984071291982	65.6984071291982\\
69.5	0.22842	68.5296877774898	68.5296877774898\\
69.5	0.23208	71.4273583762438	71.4273583762438\\
69.5	0.23574	74.3914189254603	74.3914189254603\\
69.5	0.2394	77.4218694251391	77.4218694251391\\
69.5	0.24306	80.5187098752802	80.5187098752802\\
69.5	0.24672	83.6819402758838	83.6819402758838\\
69.5	0.25038	86.9115606269497	86.9115606269497\\
69.5	0.25404	90.207570928478	90.207570928478\\
69.5	0.2577	93.5699711804687	93.5699711804687\\
69.5	0.26136	96.9987613829218	96.9987613829218\\
69.5	0.26502	100.493941535837	100.493941535837\\
69.5	0.26868	104.055511639215	104.055511639215\\
69.5	0.27234	107.683471693055	107.683471693055\\
69.5	0.276	111.377821697358	111.377821697358\\
69.875	0.093	8.11951916445235	8.11951916445235\\
69.875	0.09666	8.56117873542794	8.56117873542794\\
69.875	0.10032	9.0692282568659	9.0692282568659\\
69.875	0.10398	9.64366772876625	9.64366772876625\\
69.875	0.10764	10.284497151129	10.284497151129\\
69.875	0.1113	10.9917165239541	10.9917165239541\\
69.875	0.11496	11.7653258472416	11.7653258472416\\
69.875	0.11862	12.6053251209915	12.6053251209915\\
69.875	0.12228	13.5117143452037	13.5117143452037\\
69.875	0.12594	14.4844935198784	14.4844935198784\\
69.875	0.1296	15.5236626450154	15.5236626450154\\
69.875	0.13326	16.6292217206148	16.6292217206148\\
69.875	0.13692	17.8011707466766	17.8011707466766\\
69.875	0.14058	19.0395097232007	19.0395097232007\\
69.875	0.14424	20.3442386501873	20.3442386501873\\
69.875	0.1479	21.7153575276362	21.7153575276362\\
69.875	0.15156	23.1528663555475	23.1528663555475\\
69.875	0.15522	24.6567651339212	24.6567651339212\\
69.875	0.15888	26.2270538627573	26.2270538627573\\
69.875	0.16254	27.8637325420557	27.8637325420557\\
69.875	0.1662	29.5668011718166	29.5668011718166\\
69.875	0.16986	31.3362597520398	31.3362597520398\\
69.875	0.17352	33.1721082827254	33.1721082827254\\
69.875	0.17718	35.0743467638733	35.0743467638733\\
69.875	0.18084	37.0429751954837	37.0429751954837\\
69.875	0.1845	39.0779935775564	39.0779935775564\\
69.875	0.18816	41.1794019100916	41.1794019100916\\
69.875	0.19182	43.3472001930891	43.3472001930891\\
69.875	0.19548	45.581388426549	45.581388426549\\
69.875	0.19914	47.8819666104712	47.8819666104712\\
69.875	0.2028	50.2489347448559	50.2489347448559\\
69.875	0.20646	52.6822928297029	52.6822928297029\\
69.875	0.21012	55.1820408650123	55.1820408650123\\
69.875	0.21378	57.7481788507841	57.7481788507841\\
69.875	0.21744	60.3807067870182	60.3807067870182\\
69.875	0.2211	63.0796246737148	63.0796246737148\\
69.875	0.22476	65.8449325108737	65.8449325108737\\
69.875	0.22842	68.6766302984951	68.6766302984951\\
69.875	0.23208	71.5747180365787	71.5747180365787\\
69.875	0.23574	74.5391957251248	74.5391957251248\\
69.875	0.2394	77.5700633641333	77.5700633641333\\
69.875	0.24306	80.6673209536041	80.6673209536041\\
69.875	0.24672	83.8309684935373	83.8309684935373\\
69.875	0.25038	87.0610059839329	87.0610059839329\\
69.875	0.25404	90.3574334247909	90.3574334247909\\
69.875	0.2577	93.7202508161113	93.7202508161113\\
69.875	0.26136	97.1494581578941	97.1494581578941\\
69.875	0.26502	100.645055450139	100.645055450139\\
69.875	0.26868	104.207042692847	104.207042692847\\
69.875	0.27234	107.835419886017	107.835419886017\\
69.875	0.276	111.530187029649	111.530187029649\\
70.25	0.093	8.25476990545398	8.25476990545398\\
70.25	0.09666	8.69684661575922	8.69684661575922\\
70.25	0.10032	9.20531327652687	9.20531327652687\\
70.25	0.10398	9.78016988775689	9.78016988775689\\
70.25	0.10764	10.4214164494493	10.4214164494493\\
70.25	0.1113	11.129052961604	11.129052961604\\
70.25	0.11496	11.9030794242212	11.9030794242212\\
70.25	0.11862	12.7434958373008	12.7434958373008\\
70.25	0.12228	13.6503022008427	13.6503022008427\\
70.25	0.12594	14.623498514847	14.623498514847\\
70.25	0.1296	15.6630847793137	15.6630847793137\\
70.25	0.13326	16.7690609942427	16.7690609942427\\
70.25	0.13692	17.9414271596342	17.9414271596342\\
70.25	0.14058	19.180183275488	19.180183275488\\
70.25	0.14424	20.4853293418043	20.4853293418043\\
70.25	0.1479	21.8568653585828	21.8568653585828\\
70.25	0.15156	23.2947913258238	23.2947913258238\\
70.25	0.15522	24.7991072435272	24.7991072435272\\
70.25	0.15888	26.3698131116929	26.3698131116929\\
70.25	0.16254	28.006908930321	28.006908930321\\
70.25	0.1662	29.7103946994115	29.7103946994115\\
70.25	0.16986	31.4802704189644	31.4802704189644\\
70.25	0.17352	33.3165360889797	33.3165360889797\\
70.25	0.17718	35.2191917094573	35.2191917094573\\
70.25	0.18084	37.1882372803973	37.1882372803973\\
70.25	0.1845	39.2236728017998	39.2236728017998\\
70.25	0.18816	41.3254982736645	41.3254982736645\\
70.25	0.19182	43.4937136959917	43.4937136959917\\
70.25	0.19548	45.7283190687812	45.7283190687812\\
70.25	0.19914	48.0293143920332	48.0293143920332\\
70.25	0.2028	50.3966996657475	50.3966996657475\\
70.25	0.20646	52.8304748899242	52.8304748899242\\
70.25	0.21012	55.3306400645633	55.3306400645633\\
70.25	0.21378	57.8971951896647	57.8971951896647\\
70.25	0.21744	60.5301402652285	60.5301402652285\\
70.25	0.2211	63.2294752912547	63.2294752912547\\
70.25	0.22476	65.9952002677434	65.9952002677434\\
70.25	0.22842	68.8273151946943	68.8273151946943\\
70.25	0.23208	71.7258200721077	71.7258200721077\\
70.25	0.23574	74.6907148999835	74.6907148999835\\
70.25	0.2394	77.7219996783216	77.7219996783216\\
70.25	0.24306	80.8196744071221	80.8196744071221\\
70.25	0.24672	83.983739086385	83.983739086385\\
70.25	0.25038	87.2141937161103	87.2141937161103\\
70.25	0.25404	90.5110382962979	90.5110382962979\\
70.25	0.2577	93.8742728269479	93.8742728269479\\
70.25	0.26136	97.3038973080604	97.3038973080604\\
70.25	0.26502	100.799911739635	100.799911739635\\
70.25	0.26868	104.362316121672	104.362316121672\\
70.25	0.27234	107.991110454172	107.991110454172\\
70.25	0.276	111.686294737134	111.686294737134\\
70.625	0.093	8.39376302164971	8.39376302164971\\
70.625	0.09666	8.83625687128464	8.83625687128464\\
70.625	0.10032	9.34514067138194	9.34514067138194\\
70.625	0.10398	9.92041442194164	9.92041442194164\\
70.625	0.10764	10.5620781229637	10.5620781229637\\
70.625	0.1113	11.2701317744481	11.2701317744481\\
70.625	0.11496	12.044575376395	12.044575376395\\
70.625	0.11862	12.8854089288042	12.8854089288042\\
70.625	0.12228	13.7926324316758	13.7926324316758\\
70.625	0.12594	14.7662458850097	14.7662458850097\\
70.625	0.1296	15.8062492888061	15.8062492888061\\
70.625	0.13326	16.9126426430648	16.9126426430648\\
70.625	0.13692	18.0854259477859	18.0854259477859\\
70.625	0.14058	19.3245992029694	19.3245992029694\\
70.625	0.14424	20.6301624086153	20.6301624086153\\
70.625	0.1479	22.0021155647236	22.0021155647236\\
70.625	0.15156	23.4404586712942	23.4404586712942\\
70.625	0.15522	24.9451917283273	24.9451917283273\\
70.625	0.15888	26.5163147358227	26.5163147358227\\
70.625	0.16254	28.1538276937805	28.1538276937805\\
70.625	0.1662	29.8577306022006	29.8577306022006\\
70.625	0.16986	31.6280234610832	31.6280234610832\\
70.625	0.17352	33.4647062704281	33.4647062704281\\
70.625	0.17718	35.3677790302354	35.3677790302354\\
70.625	0.18084	37.3372417405051	37.3372417405051\\
70.625	0.1845	39.3730944012372	39.3730944012372\\
70.625	0.18816	41.4753370124316	41.4753370124316\\
70.625	0.19182	43.6439695740884	43.6439695740884\\
70.625	0.19548	45.8789920862077	45.8789920862077\\
70.625	0.19914	48.1804045487893	48.1804045487893\\
70.625	0.2028	50.5482069618333	50.5482069618333\\
70.625	0.20646	52.9823993253396	52.9823993253396\\
70.625	0.21012	55.4829816393084	55.4829816393084\\
70.625	0.21378	58.0499539037395	58.0499539037395\\
70.625	0.21744	60.683316118633	60.683316118633\\
70.625	0.2211	63.3830682839888	63.3830682839888\\
70.625	0.22476	66.1492103998071	66.1492103998071\\
70.625	0.22842	68.9817424660878	68.9817424660878\\
70.625	0.23208	71.8806644828308	71.8806644828308\\
70.625	0.23574	74.8459764500362	74.8459764500362\\
70.625	0.2394	77.877678367704	77.877678367704\\
70.625	0.24306	80.9757702358342	80.9757702358342\\
70.625	0.24672	84.1402520544267	84.1402520544267\\
70.625	0.25038	87.3711238234817	87.3711238234817\\
70.625	0.25404	90.668385542999	90.668385542999\\
70.625	0.2577	94.0320372129787	94.0320372129787\\
70.625	0.26136	97.4620788334208	97.4620788334208\\
70.625	0.26502	100.958510404325	100.958510404325\\
70.625	0.26868	104.521331925692	104.521331925692\\
70.625	0.27234	108.150543397521	108.150543397521\\
70.625	0.276	111.846144819813	111.846144819813\\
71	0.093	8.53649851303954	8.53649851303954\\
71	0.09666	8.97940950200414	8.97940950200414\\
71	0.10032	9.4887104414311	9.4887104414311\\
71	0.10398	10.0644013313204	10.0644013313204\\
71	0.10764	10.7064821716722	10.7064821716722\\
71	0.1113	11.4149529624863	11.4149529624863\\
71	0.11496	12.1898137037628	12.1898137037628\\
71	0.11862	13.0310643955017	13.0310643955017\\
71	0.12228	13.9387050377029	13.9387050377029\\
71	0.12594	14.9127356303666	14.9127356303666\\
71	0.1296	15.9531561734926	15.9531561734926\\
71	0.13326	17.059966667081	17.059966667081\\
71	0.13692	18.2331671111318	18.2331671111318\\
71	0.14058	19.4727575056449	19.4727575056449\\
71	0.14424	20.7787378506205	20.7787378506205\\
71	0.1479	22.1511081460584	22.1511081460584\\
71	0.15156	23.5898683919587	23.5898683919587\\
71	0.15522	25.0950185883214	25.0950185883214\\
71	0.15888	26.6665587351465	26.6665587351465\\
71	0.16254	28.3044888324339	28.3044888324339\\
71	0.1662	30.0088088801838	30.0088088801838\\
71	0.16986	31.779518878396	31.779518878396\\
71	0.17352	33.6166188270706	33.6166188270706\\
71	0.17718	35.5201087262076	35.5201087262076\\
71	0.18084	37.4899885758069	37.4899885758069\\
71	0.1845	39.5262583758687	39.5262583758687\\
71	0.18816	41.6289181263928	41.6289181263928\\
71	0.19182	43.7979678273793	43.7979678273793\\
71	0.19548	46.0334074788282	46.0334074788282\\
71	0.19914	48.3352370807394	48.3352370807394\\
71	0.2028	50.7034566331131	50.7034566331131\\
71	0.20646	53.1380661359491	53.1380661359491\\
71	0.21012	55.6390655892475	55.6390655892475\\
71	0.21378	58.2064549930083	58.2064549930083\\
71	0.21744	60.8402343472315	60.8402343472315\\
71	0.2211	63.540403651917	63.540403651917\\
71	0.22476	66.306962907065	66.306962907065\\
71	0.22842	69.1399121126753	69.1399121126753\\
71	0.23208	72.039251268748	72.039251268748\\
71	0.23574	75.0049803752831	75.0049803752831\\
71	0.2394	78.0370994322805	78.0370994322805\\
71	0.24306	81.1356084397404	81.1356084397404\\
71	0.24672	84.3005073976626	84.3005073976626\\
71	0.25038	87.5317963060472	87.5317963060472\\
71	0.25404	90.8294751648942	90.8294751648942\\
71	0.2577	94.1935439742035	94.1935439742035\\
71	0.26136	97.6240027339753	97.6240027339753\\
71	0.26502	101.120851444209	101.120851444209\\
71	0.26868	104.684090104906	104.684090104906\\
71	0.27234	108.313718716065	108.313718716065\\
71	0.276	112.009737277686	112.009737277686\\
71.375	0.093	8.68297637962346	8.68297637962346\\
71.375	0.09666	9.12630450791772	9.12630450791772\\
71.375	0.10032	9.63602258667436	9.63602258667436\\
71.375	0.10398	10.2121306158934	10.2121306158934\\
71.375	0.10764	10.8546285955748	10.8546285955748\\
71.375	0.1113	11.5635165257185	11.5635165257185\\
71.375	0.11496	12.3387944063247	12.3387944063247\\
71.375	0.11862	13.1804622373933	13.1804622373933\\
71.375	0.12228	14.0885200189242	14.0885200189242\\
71.375	0.12594	15.0629677509175	15.0629677509175\\
71.375	0.1296	16.1038054333732	16.1038054333732\\
71.375	0.13326	17.2110330662912	17.2110330662912\\
71.375	0.13692	18.3846506496717	18.3846506496717\\
71.375	0.14058	19.6246581835145	19.6246581835145\\
71.375	0.14424	20.9310556678198	20.9310556678198\\
71.375	0.1479	22.3038431025873	22.3038431025873\\
71.375	0.15156	23.7430204878173	23.7430204878173\\
71.375	0.15522	25.2485878235097	25.2485878235097\\
71.375	0.15888	26.8205451096644	26.8205451096644\\
71.375	0.16254	28.4588923462815	28.4588923462815\\
71.375	0.1662	30.163629533361	30.163629533361\\
71.375	0.16986	31.9347566709029	31.9347566709029\\
71.375	0.17352	33.7722737589072	33.7722737589072\\
71.375	0.17718	35.6761807973738	35.6761807973738\\
71.375	0.18084	37.6464777863028	37.6464777863028\\
71.375	0.1845	39.6831647256943	39.6831647256943\\
71.375	0.18816	41.786241615548	41.786241615548\\
71.375	0.19182	43.9557084558642	43.9557084558642\\
71.375	0.19548	46.1915652466428	46.1915652466428\\
71.375	0.19914	48.4938119878837	48.4938119878837\\
71.375	0.2028	50.862448679587	50.862448679587\\
71.375	0.20646	53.2974753217527	53.2974753217527\\
71.375	0.21012	55.7988919143808	55.7988919143808\\
71.375	0.21378	58.3666984574713	58.3666984574713\\
71.375	0.21744	61.0008949510241	61.0008949510241\\
71.375	0.2211	63.7014813950393	63.7014813950393\\
71.375	0.22476	66.4684577895169	66.4684577895169\\
71.375	0.22842	69.3018241344569	69.3018241344569\\
71.375	0.23208	72.2015804298592	72.2015804298592\\
71.375	0.23574	75.167726675724	75.167726675724\\
71.375	0.2394	78.2002628720511	78.2002628720511\\
71.375	0.24306	81.2991890188406	81.2991890188406\\
71.375	0.24672	84.4645051160925	84.4645051160925\\
71.375	0.25038	87.6962111638068	87.6962111638068\\
71.375	0.25404	90.9943071619834	90.9943071619834\\
71.375	0.2577	94.3587931106225	94.3587931106225\\
71.375	0.26136	97.7896690097239	97.7896690097239\\
71.375	0.26502	101.286934859288	101.286934859288\\
71.375	0.26868	104.850590659314	104.850590659314\\
71.375	0.27234	108.480636409802	108.480636409802\\
71.375	0.276	112.177072110753	112.177072110753\\
71.75	0.093	8.83319662140149	8.83319662140149\\
71.75	0.09666	9.27694188902542	9.27694188902542\\
71.75	0.10032	9.78707710711172	9.78707710711172\\
71.75	0.10398	10.3636022756604	10.3636022756604\\
71.75	0.10764	11.0065173946715	11.0065173946715\\
71.75	0.1113	11.7158224641449	11.7158224641449\\
71.75	0.11496	12.4915174840807	12.4915174840807\\
71.75	0.11862	13.333602454479	13.333602454479\\
71.75	0.12228	14.2420773753395	14.2420773753395\\
71.75	0.12594	15.2169422466625	15.2169422466625\\
71.75	0.1296	16.2581970684479	16.2581970684479\\
71.75	0.13326	17.3658418406956	17.3658418406956\\
71.75	0.13692	18.5398765634057	18.5398765634057\\
71.75	0.14058	19.7803012365782	19.7803012365782\\
71.75	0.14424	21.0871158602131	21.0871158602131\\
71.75	0.1479	22.4603204343104	22.4603204343104\\
71.75	0.15156	23.89991495887	23.89991495887\\
71.75	0.15522	25.405899433892	25.405899433892\\
71.75	0.15888	26.9782738593765	26.9782738593765\\
71.75	0.16254	28.6170382353232	28.6170382353232\\
71.75	0.1662	30.3221925617324	30.3221925617324\\
71.75	0.16986	32.093736838604	32.093736838604\\
71.75	0.17352	33.9316710659379	33.9316710659379\\
71.75	0.17718	35.8359952437342	35.8359952437342\\
71.75	0.18084	37.8067093719929	37.8067093719929\\
71.75	0.1845	39.843813450714	39.843813450714\\
71.75	0.18816	41.9473074798974	41.9473074798974\\
71.75	0.19182	44.1171914595432	44.1171914595432\\
71.75	0.19548	46.3534653896515	46.3534653896515\\
71.75	0.19914	48.6561292702221	48.6561292702221\\
71.75	0.2028	51.0251831012551	51.0251831012551\\
71.75	0.20646	53.4606268827504	53.4606268827504\\
71.75	0.21012	55.9624606147082	55.9624606147082\\
71.75	0.21378	58.5306842971283	58.5306842971283\\
71.75	0.21744	61.1652979300108	61.1652979300108\\
71.75	0.2211	63.8663015133557	63.8663015133557\\
71.75	0.22476	66.6336950471629	66.6336950471629\\
71.75	0.22842	69.4674785314326	69.4674785314326\\
71.75	0.23208	72.3676519661646	72.3676519661646\\
71.75	0.23574	75.334215351359	75.334215351359\\
71.75	0.2394	78.3671686870158	78.3671686870158\\
71.75	0.24306	81.466511973135	81.466511973135\\
71.75	0.24672	84.6322452097166	84.6322452097166\\
71.75	0.25038	87.8643683967605	87.8643683967605\\
71.75	0.25404	91.1628815342668	91.1628815342668\\
71.75	0.2577	94.5277846222355	94.5277846222355\\
71.75	0.26136	97.9590776606666	97.9590776606666\\
71.75	0.26502	101.45676064956	101.45676064956\\
71.75	0.26868	105.020833588916	105.020833588916\\
71.75	0.27234	108.651296478734	108.651296478734\\
71.75	0.276	112.348149319015	112.348149319015\\
72.125	0.093	8.98715923837361	8.98715923837361\\
72.125	0.09666	9.43132164532719	9.43132164532719\\
72.125	0.10032	9.94187400274317	9.94187400274317\\
72.125	0.10398	10.5188163106215	10.5188163106215\\
72.125	0.10764	11.1621485689623	11.1621485689623\\
72.125	0.1113	11.8718707777654	11.8718707777654\\
72.125	0.11496	12.6479829370309	12.6479829370309\\
72.125	0.11862	13.4904850467588	13.4904850467588\\
72.125	0.12228	14.399377106949	14.399377106949\\
72.125	0.12594	15.3746591176017	15.3746591176017\\
72.125	0.1296	16.4163310787167	16.4163310787167\\
72.125	0.13326	17.5243929902941	17.5243929902941\\
72.125	0.13692	18.6988448523338	18.6988448523338\\
72.125	0.14058	19.939686664836	19.939686664836\\
72.125	0.14424	21.2469184278006	21.2469184278006\\
72.125	0.1479	22.6205401412275	22.6205401412275\\
72.125	0.15156	24.0605518051168	24.0605518051168\\
72.125	0.15522	25.5669534194685	25.5669534194685\\
72.125	0.15888	27.1397449842826	27.1397449842826\\
72.125	0.16254	28.778926499559	28.778926499559\\
72.125	0.1662	30.4844979652979	30.4844979652979\\
72.125	0.16986	32.2564593814991	32.2564593814991\\
72.125	0.17352	34.0948107481627	34.0948107481627\\
72.125	0.17718	35.9995520652886	35.9995520652886\\
72.125	0.18084	37.970683332877	37.970683332877\\
72.125	0.1845	40.0082045509278	40.0082045509278\\
72.125	0.18816	42.1121157194409	42.1121157194409\\
72.125	0.19182	44.2824168384164	44.2824168384164\\
72.125	0.19548	46.5191079078543	46.5191079078543\\
72.125	0.19914	48.8221889277545	48.8221889277545\\
72.125	0.2028	51.1916598981172	51.1916598981172\\
72.125	0.20646	53.6275208189422	53.6275208189422\\
72.125	0.21012	56.1297716902296	56.1297716902296\\
72.125	0.21378	58.6984125119794	58.6984125119794\\
72.125	0.21744	61.3334432841916	61.3334432841916\\
72.125	0.2211	64.0348640068661	64.0348640068661\\
72.125	0.22476	66.8026746800031	66.8026746800031\\
72.125	0.22842	69.6368753036024	69.6368753036024\\
72.125	0.23208	72.5374658776641	72.5374658776641\\
72.125	0.23574	75.5044464021882	75.5044464021882\\
72.125	0.2394	78.5378168771746	78.5378168771746\\
72.125	0.24306	81.6375773026235	81.6375773026235\\
72.125	0.24672	84.8037276785347	84.8037276785347\\
72.125	0.25038	88.0362680049083	88.0362680049083\\
72.125	0.25404	91.3351982817443	91.3351982817443\\
72.125	0.2577	94.7005185090427	94.7005185090427\\
72.125	0.26136	98.1322286868034	98.1322286868034\\
72.125	0.26502	101.630328815027	101.630328815027\\
72.125	0.26868	105.194818893712	105.194818893712\\
72.125	0.27234	108.82569892286	108.82569892286\\
72.125	0.276	112.52296890247	112.52296890247\\
72.5	0.093	9.14486423053983	9.14486423053983\\
72.5	0.09666	9.5894437768231	9.5894437768231\\
72.5	0.10032	10.1004132735687	10.1004132735687\\
72.5	0.10398	10.6777727207768	10.6777727207768\\
72.5	0.10764	11.3215221184471	11.3215221184471\\
72.5	0.1113	12.0316614665799	12.0316614665799\\
72.5	0.11496	12.8081907651751	12.8081907651751\\
72.5	0.11862	13.6511100142326	13.6511100142326\\
72.5	0.12228	14.5604192137526	14.5604192137526\\
72.5	0.12594	15.5361183637349	15.5361183637349\\
72.5	0.1296	16.5782074641796	16.5782074641796\\
72.5	0.13326	17.6866865150866	17.6866865150866\\
72.5	0.13692	18.8615555164561	18.8615555164561\\
72.5	0.14058	20.1028144682879	20.1028144682879\\
72.5	0.14424	21.4104633705821	21.4104633705821\\
72.5	0.1479	22.7845022233387	22.7845022233387\\
72.5	0.15156	24.2249310265577	24.2249310265577\\
72.5	0.15522	25.7317497802391	25.7317497802391\\
72.5	0.15888	27.3049584843828	27.3049584843828\\
72.5	0.16254	28.9445571389889	28.9445571389889\\
72.5	0.1662	30.6505457440574	30.6505457440574\\
72.5	0.16986	32.4229242995883	32.4229242995883\\
72.5	0.17352	34.2616928055816	34.2616928055816\\
72.5	0.17718	36.1668512620372	36.1668512620372\\
72.5	0.18084	38.1383996689552	38.1383996689552\\
72.5	0.1845	40.1763380263357	40.1763380263357\\
72.5	0.18816	42.2806663341784	42.2806663341784\\
72.5	0.19182	44.4513845924836	44.4513845924836\\
72.5	0.19548	46.6884928012512	46.6884928012512\\
72.5	0.19914	48.9919909604811	48.9919909604811\\
72.5	0.2028	51.3618790701734	51.3618790701734\\
72.5	0.20646	53.7981571303281	53.7981571303281\\
72.5	0.21012	56.3008251409452	56.3008251409452\\
72.5	0.21378	58.8698831020246	58.8698831020246\\
72.5	0.21744	61.5053310135665	61.5053310135665\\
72.5	0.2211	64.2071688755707	64.2071688755707\\
72.5	0.22476	66.9753966880373	66.9753966880373\\
72.5	0.22842	69.8100144509663	69.8100144509663\\
72.5	0.23208	72.7110221643576	72.7110221643576\\
72.5	0.23574	75.6784198282114	75.6784198282114\\
72.5	0.2394	78.7122074425275	78.7122074425275\\
72.5	0.24306	81.812385007306	81.812385007306\\
72.5	0.24672	84.9789525225469	84.9789525225469\\
72.5	0.25038	88.2119099882502	88.2119099882502\\
72.5	0.25404	91.5112574044159	91.5112574044159\\
72.5	0.2577	94.8769947710439	94.8769947710439\\
72.5	0.26136	98.3091220881343	98.3091220881343\\
72.5	0.26502	101.807639355687	101.807639355687\\
72.5	0.26868	105.372546573702	105.372546573702\\
72.5	0.27234	109.00384374218	109.00384374218\\
72.5	0.276	112.70153086112	112.70153086112\\
72.875	0.093	9.30631159790014	9.30631159790014\\
72.875	0.09666	9.75130828351306	9.75130828351306\\
72.875	0.10032	10.2626949195884	10.2626949195884\\
72.875	0.10398	10.8404715061261	10.8404715061261\\
72.875	0.10764	11.4846380431261	11.4846380431261\\
72.875	0.1113	12.1951945305886	12.1951945305886\\
72.875	0.11496	12.9721409685134	12.9721409685134\\
72.875	0.11862	13.8154773569006	13.8154773569006\\
72.875	0.12228	14.7252036957502	14.7252036957502\\
72.875	0.12594	15.7013199850622	15.7013199850622\\
72.875	0.1296	16.7438262248365	16.7438262248365\\
72.875	0.13326	17.8527224150733	17.8527224150733\\
72.875	0.13692	19.0280085557724	19.0280085557724\\
72.875	0.14058	20.2696846469339	20.2696846469339\\
72.875	0.14424	21.5777506885578	21.5777506885578\\
72.875	0.1479	22.952206680644	22.952206680644\\
72.875	0.15156	24.3930526231927	24.3930526231927\\
72.875	0.15522	25.9002885162037	25.9002885162037\\
72.875	0.15888	27.4739143596771	27.4739143596771\\
72.875	0.16254	29.1139301536129	29.1139301536129\\
72.875	0.1662	30.8203358980111	30.8203358980111\\
72.875	0.16986	32.5931315928716	32.5931315928716\\
72.875	0.17352	34.4323172381946	34.4323172381946\\
72.875	0.17718	36.3378928339799	36.3378928339799\\
72.875	0.18084	38.3098583802276	38.3098583802276\\
72.875	0.1845	40.3482138769376	40.3482138769376\\
72.875	0.18816	42.4529593241101	42.4529593241101\\
72.875	0.19182	44.6240947217449	44.6240947217449\\
72.875	0.19548	46.8616200698421	46.8616200698421\\
72.875	0.19914	49.1655353684017	49.1655353684017\\
72.875	0.2028	51.5358406174237	51.5358406174237\\
72.875	0.20646	53.9725358169081	53.9725358169081\\
72.875	0.21012	56.4756209668548	56.4756209668548\\
72.875	0.21378	59.045096067264	59.045096067264\\
72.875	0.21744	61.6809611181355	61.6809611181355\\
72.875	0.2211	64.3832161194693	64.3832161194693\\
72.875	0.22476	67.1518610712656	67.1518610712656\\
72.875	0.22842	69.9868959735243	69.9868959735243\\
72.875	0.23208	72.8883208262453	72.8883208262453\\
72.875	0.23574	75.8561356294287	75.8561356294287\\
72.875	0.2394	78.8903403830745	78.8903403830745\\
72.875	0.24306	81.9909350871827	81.9909350871827\\
72.875	0.24672	85.1579197417533	85.1579197417533\\
72.875	0.25038	88.3912943467862	88.3912943467862\\
72.875	0.25404	91.6910589022815	91.6910589022815\\
72.875	0.2577	95.0572134082392	95.0572134082392\\
72.875	0.26136	98.4897578646593	98.4897578646593\\
72.875	0.26502	101.988692271542	101.988692271542\\
72.875	0.26868	105.554016628887	105.554016628887\\
72.875	0.27234	109.185730936694	109.185730936694\\
72.875	0.276	112.883835194963	112.883835194963\\
73.25	0.093	9.47150134045456	9.47150134045456\\
73.25	0.09666	9.91691516539715	9.91691516539715\\
73.25	0.10032	10.4287189408021	10.4287189408021\\
73.25	0.10398	11.0069126666695	11.0069126666695\\
73.25	0.10764	11.6514963429992	11.6514963429992\\
73.25	0.1113	12.3624699697913	12.3624699697913\\
73.25	0.11496	13.1398335470458	13.1398335470458\\
73.25	0.11862	13.9835870747627	13.9835870747627\\
73.25	0.12228	14.893730552942	14.893730552942\\
73.25	0.12594	15.8702639815836	15.8702639815836\\
73.25	0.1296	16.9131873606876	16.9131873606876\\
73.25	0.13326	18.022500690254	18.022500690254\\
73.25	0.13692	19.1982039702828	19.1982039702828\\
73.25	0.14058	20.440297200774	20.440297200774\\
73.25	0.14424	21.7487803817275	21.7487803817275\\
73.25	0.1479	23.1236535131435	23.1236535131435\\
73.25	0.15156	24.5649165950218	24.5649165950218\\
73.25	0.15522	26.0725696273625	26.0725696273625\\
73.25	0.15888	27.6466126101655	27.6466126101655\\
73.25	0.16254	29.287045543431	29.287045543431\\
73.25	0.1662	30.9938684271588	30.9938684271588\\
73.25	0.16986	32.767081261349	32.767081261349\\
73.25	0.17352	34.6066840460016	34.6066840460016\\
73.25	0.17718	36.5126767811166	36.5126767811166\\
73.25	0.18084	38.485059466694	38.485059466694\\
73.25	0.1845	40.5238321027337	40.5238321027337\\
73.25	0.18816	42.6289946892359	42.6289946892359\\
73.25	0.19182	44.8005472262004	44.8005472262004\\
73.25	0.19548	47.0384897136272	47.0384897136272\\
73.25	0.19914	49.3428221515165	49.3428221515165\\
73.25	0.2028	51.7135445398682	51.7135445398682\\
73.25	0.20646	54.1506568786822	54.1506568786822\\
73.25	0.21012	56.6541591679586	56.6541591679586\\
73.25	0.21378	59.2240514076974	59.2240514076974\\
73.25	0.21744	61.8603335978986	61.8603335978986\\
73.25	0.2211	64.5630057385621	64.5630057385621\\
73.25	0.22476	67.332067829688	67.332067829688\\
73.25	0.22842	70.1675198712764	70.1675198712764\\
73.25	0.23208	73.069361863327	73.069361863327\\
73.25	0.23574	76.0375938058402	76.0375938058402\\
73.25	0.2394	79.0722156988156	79.0722156988156\\
73.25	0.24306	82.1732275422535	82.1732275422535\\
73.25	0.24672	85.3406293361537	85.3406293361537\\
73.25	0.25038	88.5744210805163	88.5744210805163\\
73.25	0.25404	91.8746027753413	91.8746027753413\\
73.25	0.2577	95.2411744206287	95.2411744206287\\
73.25	0.26136	98.6741360163784	98.6741360163784\\
73.25	0.26502	102.173487562591	102.173487562591\\
73.25	0.26868	105.739229059265	105.739229059265\\
73.25	0.27234	109.371360506402	109.371360506402\\
73.25	0.276	113.069881904001	113.069881904001\\
73.625	0.093	9.64043345820307	9.64043345820307\\
73.625	0.09666	10.0862644224753	10.0862644224753\\
73.625	0.10032	10.59848533721	10.59848533721\\
73.625	0.10398	11.177096202407	11.177096202407\\
73.625	0.10764	11.8220970180664	11.8220970180664\\
73.625	0.1113	12.5334877841882	12.5334877841882\\
73.625	0.11496	13.3112685007723	13.3112685007723\\
73.625	0.11862	14.1554391678189	14.1554391678189\\
73.625	0.12228	15.0659997853278	15.0659997853278\\
73.625	0.12594	16.0429503532991	16.0429503532991\\
73.625	0.1296	17.0862908717328	17.0862908717328\\
73.625	0.13326	18.1960213406289	18.1960213406289\\
73.625	0.13692	19.3721417599873	19.3721417599873\\
73.625	0.14058	20.6146521298082	20.6146521298082\\
73.625	0.14424	21.9235524500914	21.9235524500914\\
73.625	0.1479	23.298842720837	23.298842720837\\
73.625	0.15156	24.7405229420449	24.7405229420449\\
73.625	0.15522	26.2485931137153	26.2485931137153\\
73.625	0.15888	27.8230532358481	27.8230532358481\\
73.625	0.16254	29.4639033084432	29.4639033084432\\
73.625	0.1662	31.1711433315007	31.1711433315007\\
73.625	0.16986	32.9447733050206	32.9447733050206\\
73.625	0.17352	34.7847932290028	34.7847932290028\\
73.625	0.17718	36.6912031034475	36.6912031034475\\
73.625	0.18084	38.6640029283545	38.6640029283545\\
73.625	0.1845	40.7031927037239	40.7031927037239\\
73.625	0.18816	42.8087724295557	42.8087724295557\\
73.625	0.19182	44.9807421058499	44.9807421058499\\
73.625	0.19548	47.2191017326064	47.2191017326064\\
73.625	0.19914	49.5238513098254	49.5238513098254\\
73.625	0.2028	51.8949908375067	51.8949908375067\\
73.625	0.20646	54.3325203156504	54.3325203156504\\
73.625	0.21012	56.8364397442565	56.8364397442565\\
73.625	0.21378	59.4067491233249	59.4067491233249\\
73.625	0.21744	62.0434484528557	62.0434484528557\\
73.625	0.2211	64.746537732849	64.746537732849\\
73.625	0.22476	67.5160169633046	67.5160169633046\\
73.625	0.22842	70.3518861442226	70.3518861442226\\
73.625	0.23208	73.2541452756029	73.2541452756029\\
73.625	0.23574	76.2227943574457	76.2227943574457\\
73.625	0.2394	79.2578333897508	79.2578333897508\\
73.625	0.24306	82.3592623725183	82.3592623725183\\
73.625	0.24672	85.5270813057482	85.5270813057482\\
73.625	0.25038	88.7612901894405	88.7612901894405\\
73.625	0.25404	92.0618890235951	92.0618890235951\\
73.625	0.2577	95.4288778082122	95.4288778082122\\
73.625	0.26136	98.8622565432916	98.8622565432916\\
73.625	0.26502	102.362025228833	102.362025228833\\
73.625	0.26868	105.928183864838	105.928183864838\\
73.625	0.27234	109.560732451304	109.560732451304\\
73.625	0.276	113.259670988233	113.259670988233\\
74	0.093	9.81310795114566	9.81310795114566\\
74	0.09666	10.2593560547476	10.2593560547476\\
74	0.10032	10.7719941088119	10.7719941088119\\
74	0.10398	11.3510221133386	11.3510221133386\\
74	0.10764	11.9964400683276	11.9964400683276\\
74	0.1113	12.7082479737791	12.7082479737791\\
74	0.11496	13.4864458296929	13.4864458296929\\
74	0.11862	14.3310336360691	14.3310336360691\\
74	0.12228	15.2420113929077	15.2420113929077\\
74	0.12594	16.2193791002087	16.2193791002087\\
74	0.1296	17.2631367579721	17.2631367579721\\
74	0.13326	18.3732843661978	18.3732843661978\\
74	0.13692	19.5498219248859	19.5498219248859\\
74	0.14058	20.7927494340364	20.7927494340364\\
74	0.14424	22.1020668936493	22.1020668936493\\
74	0.1479	23.4777743037246	23.4777743037246\\
74	0.15156	24.9198716642622	24.9198716642622\\
74	0.15522	26.4283589752623	26.4283589752623\\
74	0.15888	28.0032362367247	28.0032362367247\\
74	0.16254	29.6445034486495	29.6445034486495\\
74	0.1662	31.3521606110366	31.3521606110366\\
74	0.16986	33.1262077238862	33.1262077238862\\
74	0.17352	34.9666447871981	34.9666447871981\\
74	0.17718	36.8734718009724	36.8734718009724\\
74	0.18084	38.8466887652091	38.8466887652091\\
74	0.1845	40.8862956799082	40.8862956799082\\
74	0.18816	42.9922925450696	42.9922925450696\\
74	0.19182	45.1646793606935	45.1646793606935\\
74	0.19548	47.4034561267797	47.4034561267797\\
74	0.19914	49.7086228433283	49.7086228433283\\
74	0.2028	52.0801795103393	52.0801795103393\\
74	0.20646	54.5181261278127	54.5181261278127\\
74	0.21012	57.0224626957484	57.0224626957484\\
74	0.21378	59.5931892141465	59.5931892141465\\
74	0.21744	62.230305683007	62.230305683007\\
74	0.2211	64.9338121023299	64.9338121023299\\
74	0.22476	67.7037084721152	67.7037084721152\\
74	0.22842	70.5399947923628	70.5399947923628\\
74	0.23208	73.4426710630729	73.4426710630729\\
74	0.23574	76.4117372842453	76.4117372842453\\
74	0.2394	79.4471934558801	79.4471934558801\\
74	0.24306	82.5490395779773	82.5490395779773\\
74	0.24672	85.7172756505368	85.7172756505368\\
74	0.25038	88.9519016735588	88.9519016735588\\
74	0.25404	92.2529176470431	92.2529176470431\\
74	0.2577	95.6203235709898	95.6203235709898\\
74	0.26136	99.0541194453989	99.0541194453989\\
74	0.26502	102.55430527027	102.55430527027\\
74	0.26868	106.120881045604	106.120881045604\\
74	0.27234	109.7538467714	109.7538467714\\
74	0.276	113.453202447659	113.453202447659\\
};
\end{axis}
\end{tikzpicture}%
	\caption{The quantitative relationship between external parameters ($L_{cut}, D_{rlx}$) and internal parameters $\theta$ is obtained by fitting a polynomial surface to the internal parameter values estimated for the sets \dataset1-\dataset5 (\ref{tikz:thetas1}) and sets \dataset6-\dataset10 (\ref{tikz:thetas2}). The fitted surfaces (\ref{tikz:theta_surf}) are then used to compute the internal model parameters corresponding to the settings of choice (\ref{tikz:thetaidentified}).}\label{fig:surfaces_all}
\end{figure}
\begin{figure}[!t]
	\centering
	\subfloat[\dataset3]{% This file was created by matlab2tikz.
%
\definecolor{mycolor1}{rgb}{0.00000,0.44700,0.74100}%
\definecolor{mycolor2}{rgb}{0.85000,0.32500,0.09800}%
%
\begin{tikzpicture}

\begin{axis}[%
width=11.411cm,
height=4cm,
at={(0cm,0cm)},
scale only axis,
xmin=1000,
xmax=1500,
ymin=-45,
ymax=0,
axis background/.style={fill=white},
legend style={legend cell align=left, align=left, draw=white!15!black}
]
\addplot [color=mycolor1]
  table[row sep=crcr]{%
1001	-17.09\\
1002	-19.531\\
1003	-14.648\\
1004	-14.648\\
1005	-19.531\\
1006	-18.311\\
1007	-23.193\\
1008	-18.311\\
1009	-9.766\\
1010	-15.869\\
1011	-12.207\\
1012	-14.648\\
1013	-10.986\\
1014	-3.662\\
1015	-2.441\\
1016	-4.883\\
1017	-6.104\\
1018	-14.648\\
1019	-17.09\\
1020	-17.09\\
1021	-13.428\\
1022	-9.766\\
1023	-18.311\\
1024	-14.648\\
1025	-10.986\\
1026	-13.428\\
1027	-12.207\\
1028	-10.986\\
1029	-20.752\\
1030	-19.531\\
1031	-12.207\\
1032	-20.752\\
1033	-20.752\\
1034	-15.869\\
1035	-13.428\\
1036	-9.766\\
1037	-14.648\\
1038	-14.648\\
1039	-14.648\\
1040	-14.648\\
1041	-14.648\\
1042	-15.869\\
1043	-21.973\\
1044	-20.752\\
1045	-13.428\\
1046	-13.428\\
1047	-8.545\\
1048	-12.207\\
1049	-12.207\\
1050	-13.428\\
1051	-10.986\\
1052	-13.428\\
1053	-14.648\\
1054	-13.428\\
1055	-20.752\\
1056	-13.428\\
1057	-9.766\\
1058	-7.324\\
1059	-8.545\\
1060	-10.986\\
1061	-6.104\\
1062	-9.766\\
1063	-10.986\\
1064	-6.104\\
1065	-6.104\\
1066	-9.766\\
1067	-9.766\\
1068	-15.869\\
1069	-14.648\\
1070	-17.09\\
1071	-15.869\\
1072	-18.311\\
1073	-13.428\\
1074	-14.648\\
1075	-12.207\\
1076	-10.986\\
1077	-12.207\\
1078	-23.193\\
1079	-28.076\\
1080	-30.518\\
1081	-29.297\\
1082	-18.311\\
1083	-28.076\\
1084	-32.959\\
1085	-31.738\\
1086	-20.752\\
1087	-34.18\\
1088	-40.283\\
1089	-29.297\\
1090	-24.414\\
1091	-19.531\\
1092	-15.869\\
1093	-12.207\\
1094	-14.648\\
1095	-9.766\\
1096	-7.324\\
1097	-7.324\\
1098	-10.986\\
1099	-13.428\\
1100	-12.207\\
1101	-12.207\\
1102	-15.869\\
1103	-18.311\\
1104	-19.531\\
1105	-15.869\\
1106	-21.973\\
1107	-15.869\\
1108	-15.869\\
1109	-17.09\\
1110	-12.207\\
1111	-12.207\\
1112	-15.869\\
1113	-14.648\\
1114	-8.545\\
1115	-12.207\\
1116	-8.545\\
1117	-9.766\\
1118	-9.766\\
1119	-7.324\\
1120	-9.766\\
1121	-12.207\\
1122	-17.09\\
1123	-17.09\\
1124	-17.09\\
1125	-10.986\\
1126	-12.207\\
1127	-23.193\\
1128	-15.869\\
1129	-20.752\\
1130	-21.973\\
1131	-12.207\\
1132	-9.766\\
1133	-12.207\\
1134	-13.428\\
1135	-21.973\\
1136	-25.635\\
1137	-26.855\\
1138	-19.531\\
1139	-20.752\\
1140	-18.311\\
1141	-17.09\\
1142	-18.311\\
1143	-13.428\\
1144	-12.207\\
1145	-9.766\\
1146	-9.766\\
1147	-10.986\\
1148	-15.869\\
1149	-23.193\\
1150	-17.09\\
1151	-13.428\\
1152	-10.986\\
1153	-10.986\\
1154	-7.324\\
1155	-6.104\\
1156	-6.104\\
1157	-12.207\\
1158	-7.324\\
1159	-8.545\\
1160	-8.545\\
1161	-8.545\\
1162	-7.324\\
1163	-4.883\\
1164	-3.662\\
1165	-8.545\\
1166	-15.869\\
1167	-18.311\\
1168	-20.752\\
1169	-15.869\\
1170	-10.986\\
1171	-8.545\\
1172	-4.883\\
1173	-9.766\\
1174	-8.545\\
1175	-15.869\\
1176	-17.09\\
1177	-29.297\\
1178	-32.959\\
1179	-25.635\\
1180	-25.635\\
1181	-18.311\\
1182	-23.193\\
1183	-21.973\\
1184	-24.414\\
1185	-15.869\\
1186	-17.09\\
1187	-17.09\\
1188	-15.869\\
1189	-15.869\\
1190	-13.428\\
1191	-23.193\\
1192	-25.635\\
1193	-19.531\\
1194	-13.428\\
1195	-14.648\\
1196	-23.193\\
1197	-24.414\\
1198	-28.076\\
1199	-30.518\\
1200	-29.297\\
1201	-23.193\\
1202	-21.973\\
1203	-25.635\\
1204	-28.076\\
1205	-18.311\\
1206	-12.207\\
1207	-19.531\\
1208	-14.648\\
1209	-9.766\\
1210	-13.428\\
1211	-14.648\\
1212	-10.986\\
1213	-10.986\\
1214	-10.986\\
1215	-8.545\\
1216	-10.986\\
1217	-18.311\\
1218	-17.09\\
1219	-12.207\\
1220	-12.207\\
1221	-18.311\\
1222	-26.855\\
1223	-17.09\\
1224	-14.648\\
1225	-10.986\\
1226	-13.428\\
1227	-10.986\\
1228	-8.545\\
1229	-8.545\\
1230	-10.986\\
1231	-13.428\\
1232	-14.648\\
1233	-12.207\\
1234	-15.869\\
1235	-20.752\\
1236	-17.09\\
1237	-12.207\\
1238	-15.869\\
1239	-20.752\\
1240	-13.428\\
1241	-7.324\\
1242	-13.428\\
1243	-10.986\\
1244	-12.207\\
1245	-9.766\\
1246	-8.545\\
1247	-13.428\\
1248	-13.428\\
1249	-9.766\\
1250	-12.207\\
1251	-9.766\\
1252	-7.324\\
1253	-12.207\\
1254	-8.545\\
1255	-9.766\\
1256	-7.324\\
1257	-4.883\\
1258	-12.207\\
1259	-15.869\\
1260	-17.09\\
1261	-23.193\\
1262	-15.869\\
1263	-9.766\\
1264	-8.545\\
1265	-8.545\\
1266	-10.986\\
1267	-8.545\\
1268	-4.883\\
1269	-10.986\\
1270	-15.869\\
1271	-13.428\\
1272	-20.752\\
1273	-15.869\\
1274	-14.648\\
1275	-8.545\\
1276	-10.986\\
1277	-4.883\\
1278	-3.662\\
1279	-4.883\\
1280	-9.766\\
1281	-10.986\\
1282	-12.207\\
1283	-13.428\\
1284	-20.752\\
1285	-17.09\\
1286	-14.648\\
1287	-17.09\\
1288	-19.531\\
1289	-20.752\\
1290	-17.09\\
1291	-13.428\\
1292	-7.324\\
1293	-6.104\\
1294	-4.883\\
1295	-7.324\\
1296	-7.324\\
1297	-4.883\\
1298	-8.545\\
1299	-12.207\\
1300	-10.986\\
1301	-12.207\\
1302	-12.207\\
1303	-8.545\\
1304	-4.883\\
1305	-9.766\\
1306	-15.869\\
1307	-13.428\\
1308	-15.869\\
1309	-13.428\\
1310	-9.766\\
1311	-14.648\\
1312	-18.311\\
1313	-18.311\\
1314	-13.428\\
1315	-20.752\\
1316	-15.869\\
1317	-17.09\\
1318	-18.311\\
1319	-19.531\\
1320	-13.428\\
1321	-13.428\\
1322	-18.311\\
1323	-26.855\\
1324	-21.973\\
1325	-13.428\\
1326	-13.428\\
1327	-13.428\\
1328	-15.869\\
1329	-17.09\\
1330	-12.207\\
1331	-14.648\\
1332	-18.311\\
1333	-20.752\\
1334	-13.428\\
1335	-12.207\\
1336	-15.869\\
1337	-23.193\\
1338	-20.752\\
1339	-21.973\\
1340	-14.648\\
1341	-14.648\\
1342	-12.207\\
1343	-9.766\\
1344	-4.883\\
1345	-3.662\\
1346	-3.662\\
1347	-7.324\\
1348	-13.428\\
1349	-14.648\\
1350	-9.766\\
1351	-12.207\\
1352	-13.428\\
1353	-9.766\\
1354	-8.545\\
1355	-8.545\\
1356	-6.104\\
1357	-9.766\\
1358	-14.648\\
1359	-15.869\\
1360	-10.986\\
1361	-8.545\\
1362	-8.545\\
1363	-8.545\\
1364	-7.324\\
1365	-10.986\\
1366	-20.752\\
1367	-18.311\\
1368	-18.311\\
1369	-24.414\\
1370	-23.193\\
1371	-18.311\\
1372	-14.648\\
1373	-14.648\\
1374	-14.648\\
1375	-17.09\\
1376	-19.531\\
1377	-19.531\\
1378	-25.635\\
1379	-26.855\\
1380	-31.738\\
1381	-24.414\\
1382	-25.635\\
1383	-26.855\\
1384	-21.973\\
1385	-24.414\\
1386	-20.752\\
1387	-13.428\\
1388	-9.766\\
1389	-9.766\\
1390	-12.207\\
1391	-9.766\\
1392	-7.324\\
1393	-10.986\\
1394	-8.545\\
1395	-6.104\\
1396	-7.324\\
1397	-9.766\\
1398	-6.104\\
1399	-12.207\\
1400	-14.648\\
1401	-10.986\\
1402	-17.09\\
1403	-24.414\\
1404	-24.414\\
1405	-28.076\\
1406	-23.193\\
1407	-21.973\\
1408	-17.09\\
1409	-9.766\\
1410	-10.986\\
1411	-10.986\\
1412	-7.324\\
1413	-10.986\\
1414	-9.766\\
1415	-2.441\\
1416	-6.104\\
1417	-9.766\\
1418	-12.207\\
1419	-14.648\\
1420	-17.09\\
1421	-14.648\\
1422	-17.09\\
1423	-14.648\\
1424	-13.428\\
1425	-12.207\\
1426	-10.986\\
1427	-14.648\\
1428	-13.428\\
1429	-10.986\\
1430	-13.428\\
1431	-18.311\\
1432	-13.428\\
1433	-10.986\\
1434	-10.986\\
1435	-10.986\\
1436	-8.545\\
1437	-7.324\\
1438	-10.986\\
1439	-13.428\\
1440	-13.428\\
1441	-10.986\\
1442	-13.428\\
1443	-13.428\\
1444	-8.545\\
1445	-7.324\\
1446	-15.869\\
1447	-19.531\\
1448	-18.311\\
1449	-18.311\\
1450	-15.869\\
1451	-13.428\\
1452	-10.986\\
1453	-9.766\\
1454	-13.428\\
1455	-13.428\\
1456	-8.545\\
1457	-12.207\\
1458	-14.648\\
1459	-14.648\\
1460	-17.09\\
1461	-17.09\\
1462	-20.752\\
1463	-28.076\\
1464	-30.518\\
1465	-20.752\\
1466	-13.428\\
1467	-8.545\\
1468	-6.104\\
1469	-8.545\\
1470	-7.324\\
1471	-10.986\\
1472	-12.207\\
1473	-12.207\\
1474	-13.428\\
1475	-15.869\\
1476	-19.531\\
1477	-19.531\\
1478	-28.076\\
1479	-23.193\\
1480	-18.311\\
1481	-14.648\\
1482	-14.648\\
1483	-14.648\\
1484	-17.09\\
1485	-14.648\\
1486	-12.207\\
1487	-13.428\\
1488	-8.545\\
1489	-7.324\\
1490	-17.09\\
1491	-14.648\\
1492	-12.207\\
1493	-23.193\\
1494	-18.311\\
1495	-15.869\\
1496	-21.973\\
1497	-23.193\\
1498	-14.648\\
1499	-8.545\\
1500	-9.766\\
};
\addlegendentry{True output}

\addplot [color=mycolor2, dashed]
  table[row sep=crcr]{%
1001	-16.254527795835\\
1002	-19.6785535483227\\
1003	-18.4695238735799\\
1004	-16.8085595785455\\
1005	-20.6967288010317\\
1006	-20.9229581833036\\
1007	-22.1588523355377\\
1008	-18.9522242268178\\
1009	-11.852930282059\\
1010	-12.7097929583578\\
1011	-13.8176911214572\\
1012	-15.6680960627051\\
1013	-14.3387146598432\\
1014	-7.44412900453769\\
1015	-4.8418820233322\\
1016	-4.54558709548009\\
1017	-10.2805005439112\\
1018	-18.6656315953563\\
1019	-18.1550065451774\\
1020	-17.4193008050967\\
1021	-13.4253527706572\\
1022	-9.17876081794417\\
1023	-16.2819251515142\\
1024	-13.7572983830717\\
1025	-10.5808002198238\\
1026	-14.2600863261157\\
1027	-15.1542363761452\\
1028	-12.4175435697077\\
1029	-17.5324658075764\\
1030	-17.7337281700071\\
1031	-13.9936533067158\\
1032	-18.0981665740989\\
1033	-18.7879458582808\\
1034	-15.4981038856351\\
1035	-13.5604273513669\\
1036	-12.1798403398373\\
1037	-13.1052897796164\\
1038	-15.729572443885\\
1039	-15.6325009114338\\
1040	-14.7766484584953\\
1041	-15.2801081420831\\
1042	-16.1268641691837\\
1043	-20.7296060663079\\
1044	-20.1844053248746\\
1045	-14.5952861338106\\
1046	-13.2928458118398\\
1047	-9.39819069122964\\
1048	-8.85449698520292\\
1049	-11.7984229473124\\
1050	-11.0473327882388\\
1051	-10.7574669425996\\
1052	-13.1991465980444\\
1053	-14.7951825921551\\
1054	-13.9364492543648\\
1055	-19.082990888115\\
1056	-16.7251096070586\\
1057	-10.227987990023\\
1058	-8.6944446207486\\
1059	-11.1021082020429\\
1060	-12.987407754281\\
1061	-9.34960794134911\\
1062	-7.73616153788004\\
1063	-10.3744371753598\\
1064	-10.0456299877943\\
1065	-8.48840928991694\\
1066	-9.6316171234278\\
1067	-10.9790178347557\\
1068	-15.0190079150973\\
1069	-15.2807843502467\\
1070	-16.1021289661051\\
1071	-15.1503437204151\\
1072	-16.8886526779588\\
1073	-15.2239566275856\\
1074	-13.9144134600621\\
1075	-13.4039923990204\\
1076	-13.3909125715513\\
1077	-13.2245011419285\\
1078	-19.4481189159192\\
1079	-28.091645619237\\
1080	-28.0501288434693\\
1081	-26.3405135189048\\
1082	-17.8013324181827\\
1083	-23.906540522589\\
1084	-29.4196670314917\\
1085	-29.4309146016869\\
1086	-24.5335176563917\\
1087	-29.1049706221812\\
1088	-36.7577763759706\\
1089	-31.0429113667165\\
1090	-25.9574621226679\\
1091	-19.3997407738875\\
1092	-16.3208382956374\\
1093	-14.9976353638659\\
1094	-16.8288067247173\\
1095	-15.2467666705987\\
1096	-10.2245672999938\\
1097	-8.8740427225416\\
1098	-10.9635938146415\\
1099	-15.1540628805987\\
1100	-14.9940159324954\\
1101	-14.3166688145998\\
1102	-16.4750779984649\\
1103	-17.2239579442626\\
1104	-22.8154402334777\\
1105	-20.4912215099301\\
1106	-21.1855093503453\\
1107	-18.2179026934775\\
1108	-15.5838902605412\\
1109	-18.355664073965\\
1110	-15.501712454729\\
1111	-13.3586533476621\\
1112	-15.8308478142802\\
1113	-16.2283199398048\\
1114	-13.6861082792868\\
1115	-12.9045301721076\\
1116	-11.3626974195098\\
1117	-10.9783638949845\\
1118	-12.7443803010989\\
1119	-9.68145238691905\\
1120	-10.3058880847157\\
1121	-12.9002524636774\\
1122	-17.4204453936312\\
1123	-17.6193672838978\\
1124	-17.4694960644632\\
1125	-12.9323145396955\\
1126	-15.7641869846631\\
1127	-23.3624231155263\\
1128	-18.9656345302036\\
1129	-20.012980140676\\
1130	-20.0481065704999\\
1131	-14.4373939232766\\
1132	-10.8743990493386\\
1133	-13.082178709204\\
1134	-15.4735497451192\\
1135	-22.1783709924051\\
1136	-26.5876104811415\\
1137	-24.8768382978155\\
1138	-20.5084016374023\\
1139	-19.9682365619472\\
1140	-19.9427406522857\\
1141	-19.1195046734225\\
1142	-19.0103087108665\\
1143	-14.9406316584238\\
1144	-12.899175481349\\
1145	-12.8556498626275\\
1146	-12.3783206506346\\
1147	-13.0025409567306\\
1148	-17.002186423881\\
1149	-23.167838836912\\
1150	-20.3952811327998\\
1151	-14.3353286428408\\
1152	-12.3102166181048\\
1153	-12.7581966598952\\
1154	-9.2196826444159\\
1155	-7.03589110955713\\
1156	-7.63726237002204\\
1157	-12.0325255948571\\
1158	-11.0240397076226\\
1159	-10.3259686383801\\
1160	-10.9526523320839\\
1161	-10.4953721786074\\
1162	-8.89803779510969\\
1163	-6.66330239103735\\
1164	-5.69840034993383\\
1165	-7.31957075641177\\
1166	-14.1187053213359\\
1167	-19.9866815805115\\
1168	-20.8178976007142\\
1169	-14.3109217006899\\
1170	-10.4343188026074\\
1171	-8.17000945054156\\
1172	-6.2465499929547\\
1173	-11.1074007793417\\
1174	-12.2888153858427\\
1175	-14.9268597597601\\
1176	-18.3691731175147\\
1177	-27.8513269533019\\
1178	-32.4578740558389\\
1179	-26.7152328147948\\
1180	-25.5304894902503\\
1181	-20.259450484194\\
1182	-20.1583961130355\\
1183	-22.3871671558667\\
1184	-23.5833452090276\\
1185	-19.9505573548944\\
1186	-18.4519290179529\\
1187	-18.8049251837313\\
1188	-17.5501866153547\\
1189	-19.9453873305425\\
1190	-16.7210750892428\\
1191	-21.6428881321012\\
1192	-26.220111931263\\
1193	-20.2356679425326\\
1194	-13.7333816837161\\
1195	-14.3875038416724\\
1196	-21.705050301303\\
1197	-25.7403160454778\\
1198	-28.9326860356227\\
1199	-30.0813993969761\\
1200	-29.7381542813311\\
1201	-24.4603578932273\\
1202	-22.9668464998021\\
1203	-25.5049417203235\\
1204	-28.1972277516781\\
1205	-21.0503513126442\\
1206	-14.0731618692229\\
1207	-18.1000201335959\\
1208	-17.4539263490992\\
1209	-12.5129380674449\\
1210	-14.6042131013128\\
1211	-16.9681001059637\\
1212	-14.5590786761075\\
1213	-11.7891039889261\\
1214	-13.6318372330255\\
1215	-13.3120183067239\\
1216	-15.3281518331053\\
1217	-20.0030473902354\\
1218	-18.1718972602771\\
1219	-14.1277157405455\\
1220	-13.5271316909107\\
1221	-18.8822301962699\\
1222	-28.2617253947708\\
1223	-23.592769336167\\
1224	-15.299850257314\\
1225	-12.6924792244913\\
1226	-13.6120441448873\\
1227	-14.5224185894347\\
1228	-11.7059717562675\\
1229	-12.158241059464\\
1230	-13.4091608908626\\
1231	-16.1410094942187\\
1232	-17.2231901419689\\
1233	-12.6126641344204\\
1234	-16.2248855191873\\
1235	-19.8500997959637\\
1236	-18.9720796162841\\
1237	-16.1615203138505\\
1238	-16.6115883716647\\
1239	-21.3277290294522\\
1240	-16.067235998004\\
1241	-8.27973024702612\\
1242	-9.36537116083202\\
1243	-11.7601257947719\\
1244	-15.3979061430704\\
1245	-14.3126363561289\\
1246	-11.4079194223811\\
1247	-14.5272282595769\\
1248	-14.9396312644322\\
1249	-11.9386321191322\\
1250	-13.2743821968853\\
1251	-12.8287581134105\\
1252	-12.1373731367485\\
1253	-12.8002555314779\\
1254	-9.75457553358341\\
1255	-9.78250893382767\\
1256	-8.6581591269492\\
1257	-8.9275421278963\\
1258	-13.6388183209548\\
1259	-15.5373757232236\\
1260	-18.5672623307148\\
1261	-23.5023433971908\\
1262	-17.3025232320516\\
1263	-10.9772017414028\\
1264	-8.85163409732329\\
1265	-9.10450981702048\\
1266	-12.3758765311307\\
1267	-9.62063661681329\\
1268	-9.36224065893933\\
1269	-11.4390492147414\\
1270	-16.7926063803233\\
1271	-16.8554309936768\\
1272	-19.2792427915847\\
1273	-15.1183768158966\\
1274	-13.1446291398401\\
1275	-8.76776715046061\\
1276	-7.95348582306552\\
1277	-6.24843965216333\\
1278	-6.30508598808701\\
1279	-7.65721823512418\\
1280	-12.5553901779878\\
1281	-12.7646688523679\\
1282	-13.6707945274289\\
1283	-13.9275803281834\\
1284	-20.2787501270059\\
1285	-20.0041757101421\\
1286	-14.4991900359391\\
1287	-16.2774395222776\\
1288	-17.2476013330773\\
1289	-21.5795066573139\\
1290	-18.7869449244937\\
1291	-13.5857972315294\\
1292	-8.3878088311381\\
1293	-6.13792391714111\\
1294	-6.72686302117015\\
1295	-8.00948052364981\\
1296	-8.37711807869147\\
1297	-8.15579287248692\\
1298	-9.27479097513148\\
1299	-12.8271829493702\\
1300	-12.8269327510388\\
1301	-11.9872358127937\\
1302	-13.2602639198292\\
1303	-9.79893019224206\\
1304	-6.1853120244679\\
1305	-9.675668477782\\
1306	-14.3218418123058\\
1307	-14.6440547537416\\
1308	-14.3704148412383\\
1309	-13.2574504010815\\
1310	-10.8235106869072\\
1311	-13.0789023398639\\
1312	-17.6017018440589\\
1313	-18.1385340801341\\
1314	-13.6087009677457\\
1315	-19.9853081693773\\
1316	-17.7789728664422\\
1317	-16.0428111563599\\
1318	-18.4954923331756\\
1319	-17.9580684429222\\
1320	-15.2503556948337\\
1321	-13.5020601212662\\
1322	-18.5732809355128\\
1323	-26.7350471694415\\
1324	-21.4408447226086\\
1325	-13.5667059798754\\
1326	-12.6952514290382\\
1327	-14.6815799019769\\
1328	-17.068341169465\\
1329	-19.2059227517558\\
1330	-15.5598299909801\\
1331	-16.0453525269692\\
1332	-19.4234522494844\\
1333	-20.7797578699979\\
1334	-15.937117413586\\
1335	-13.3223849922964\\
1336	-16.0125489432341\\
1337	-24.1148100099088\\
1338	-21.9831162094355\\
1339	-21.1318656350575\\
1340	-14.6802707914842\\
1341	-13.0009519660553\\
1342	-13.971223732964\\
1343	-9.67469731275744\\
1344	-7.41623229586748\\
1345	-5.93007691785131\\
1346	-5.45582507700952\\
1347	-9.28718366490923\\
1348	-13.5515890616187\\
1349	-15.2525475006181\\
1350	-11.9154953951825\\
1351	-11.7335282946571\\
1352	-12.9447763382902\\
1353	-10.0644345857967\\
1354	-9.53258524621465\\
1355	-10.015489529515\\
1356	-8.71148726825974\\
1357	-9.39301877148121\\
1358	-13.1628195012754\\
1359	-16.5860743970687\\
1360	-11.8954665532997\\
1361	-9.1421813809755\\
1362	-8.31749497293008\\
1363	-9.76715218928246\\
1364	-7.95353919690366\\
1365	-13.6081705172173\\
1366	-21.5595723163141\\
1367	-17.5464657775623\\
1368	-20.4379762514503\\
1369	-24.341148174161\\
1370	-23.4699707777061\\
1371	-19.4499760845079\\
1372	-15.1142519270206\\
1373	-15.2265647768365\\
1374	-16.1809074874407\\
1375	-18.0950330067858\\
1376	-19.8949602244743\\
1377	-20.2228871319075\\
1378	-23.3212541231441\\
1379	-26.5669211777671\\
1380	-30.4315367225362\\
1381	-23.775419929132\\
1382	-23.3015105469787\\
1383	-27.3418601373673\\
1384	-23.5783881054257\\
1385	-24.3340120500237\\
1386	-23.0673247624779\\
1387	-14.184429351209\\
1388	-10.3090176960213\\
1389	-11.1355103039195\\
1390	-15.7144261899986\\
1391	-14.25914050165\\
1392	-10.5566995707071\\
1393	-12.2620722202176\\
1394	-10.543572180114\\
1395	-8.42730554003665\\
1396	-9.50826378345298\\
1397	-10.5181865419145\\
1398	-8.90316586150725\\
1399	-12.2868492331555\\
1400	-16.3023310144022\\
1401	-13.3159135428293\\
1402	-17.305066541887\\
1403	-25.0988159381525\\
1404	-23.5152499234897\\
1405	-26.6860650707936\\
1406	-21.3247851197159\\
1407	-20.7891086896782\\
1408	-17.1262336553421\\
1409	-11.5255754137825\\
1410	-13.8255411949341\\
1411	-12.3155582832819\\
1412	-10.0010259361444\\
1413	-12.2036150425358\\
1414	-8.78843567813996\\
1415	-6.14879677121817\\
1416	-6.77118555399161\\
1417	-11.0260177346679\\
1418	-15.4909779483148\\
1419	-17.2159964813981\\
1420	-16.7825400598738\\
1421	-15.7355500312245\\
1422	-16.9711750893266\\
1423	-15.425405139149\\
1424	-13.2519091976085\\
1425	-13.9450276484476\\
1426	-12.4106816766141\\
1427	-16.3078971182963\\
1428	-13.1633888023794\\
1429	-10.7083193505805\\
1430	-14.0985565673601\\
1431	-19.2114843717261\\
1432	-16.3238202334337\\
1433	-12.3393575505876\\
1434	-11.1157565422425\\
1435	-12.4837610687157\\
1436	-10.0528221524678\\
1437	-9.03558130101\\
1438	-11.499184511203\\
1439	-13.3357172133222\\
1440	-14.8521272968782\\
1441	-12.6347769127954\\
1442	-13.9802424646758\\
1443	-14.0723875880144\\
1444	-9.91747179431477\\
1445	-9.37127282430209\\
1446	-14.9939434726051\\
1447	-21.0316215843534\\
1448	-19.9531149521046\\
1449	-16.9508001862369\\
1450	-16.4829143846543\\
1451	-14.2275409423853\\
1452	-13.631215740707\\
1453	-12.0478524217997\\
1454	-14.7132221435115\\
1455	-14.4416298909729\\
1456	-10.2631108537023\\
1457	-12.851309902905\\
1458	-14.5653629895465\\
1459	-16.9708127493046\\
1460	-17.3674837997607\\
1461	-17.0743807602671\\
1462	-21.7845132103158\\
1463	-26.3724203450924\\
1464	-29.3748731887452\\
1465	-21.4248514285985\\
1466	-13.4369519171937\\
1467	-9.79249352914372\\
1468	-7.23445340804922\\
1469	-9.58503481369203\\
1470	-10.6836604050061\\
1471	-14.8814284126221\\
1472	-14.5220824611656\\
1473	-12.1271810994553\\
1474	-14.5019473200892\\
1475	-16.3747886762306\\
1476	-18.0655033650769\\
1477	-19.1206578141706\\
1478	-26.869766440414\\
1479	-24.8689926895787\\
1480	-19.9413257879833\\
1481	-15.0995932212746\\
1482	-14.4371602819976\\
1483	-16.2573487568386\\
1484	-18.6633593431921\\
1485	-15.4379615859187\\
1486	-14.5316315181258\\
1487	-14.6254876485864\\
1488	-9.92892991027907\\
1489	-12.9615144123974\\
1490	-18.3167599441571\\
1491	-14.7783357958607\\
1492	-14.3303168789501\\
1493	-22.6430306650768\\
1494	-19.4062633963547\\
1495	-17.3111761839129\\
1496	-22.3021896506457\\
1497	-22.0856899954143\\
1498	-16.260267648596\\
1499	-10.4113296930648\\
1500	-11.3791239579802\\
};
\addlegendentry{Generated output}

\end{axis}
\end{tikzpicture}%\label{fig:c3all}}\\
	\subfloat[\dataset8]{% This file was created by matlab2tikz.
%
\definecolor{mycolor1}{rgb}{0.00000,0.44700,0.74100}%
\definecolor{mycolor2}{rgb}{0.85000,0.32500,0.09800}%
%
\begin{tikzpicture}

\begin{axis}[%
width=11.411cm,
height=6cm,
at={(0cm,0cm)},
scale only axis,
xmin=1000,
xmax=1500,
ymin=-300,
ymax=0,
axis background/.style={fill=white},
legend style={legend cell align=left, align=left, draw=white!15!black}
]
\addplot [color=mycolor1]
  table[row sep=crcr]{%
1001	-96.436\\
1002	-122.07\\
1003	-100.098\\
1004	-93.994\\
1005	-130.615\\
1006	-125.732\\
1007	-153.809\\
1008	-115.967\\
1009	-61.035\\
1010	-74.463\\
1011	-73.242\\
1012	-86.67\\
1013	-73.242\\
1014	-34.18\\
1015	-20.752\\
1016	-23.193\\
1017	-58.594\\
1018	-100.098\\
1019	-109.863\\
1020	-114.746\\
1021	-83.008\\
1022	-52.49\\
1023	-104.98\\
1024	-81.787\\
1025	-59.814\\
1026	-97.656\\
1027	-83.008\\
1028	-72.021\\
1029	-118.408\\
1030	-107.422\\
1031	-83.008\\
1032	-119.629\\
1033	-123.291\\
1034	-95.215\\
1035	-80.566\\
1036	-62.256\\
1037	-75.684\\
1038	-95.215\\
1039	-87.891\\
1040	-91.553\\
1041	-96.436\\
1042	-102.539\\
1043	-141.602\\
1044	-128.174\\
1045	-85.449\\
1046	-79.346\\
1047	-47.607\\
1048	-48.828\\
1049	-68.359\\
1050	-52.49\\
1051	-57.373\\
1052	-75.684\\
1053	-89.111\\
1054	-81.787\\
1055	-128.174\\
1056	-98.877\\
1057	-57.373\\
1058	-45.166\\
1059	-62.256\\
1060	-72.021\\
1061	-40.283\\
1062	-42.725\\
1063	-56.152\\
1064	-46.387\\
1065	-40.283\\
1066	-52.49\\
1067	-62.256\\
1068	-92.773\\
1069	-90.332\\
1070	-107.422\\
1071	-98.877\\
1072	-115.967\\
1073	-91.553\\
1074	-91.553\\
1075	-79.346\\
1076	-75.684\\
1077	-76.904\\
1078	-133.057\\
1079	-189.209\\
1080	-194.092\\
1081	-189.209\\
1082	-119.629\\
1083	-170.898\\
1084	-213.623\\
1085	-219.727\\
1086	-169.678\\
1087	-219.727\\
1088	-279.541\\
1089	-197.754\\
1090	-161.133\\
1091	-109.863\\
1092	-85.449\\
1093	-73.242\\
1094	-91.553\\
1095	-62.256\\
1096	-46.387\\
1097	-42.725\\
1098	-54.932\\
1099	-79.346\\
1100	-74.463\\
1101	-75.684\\
1102	-100.098\\
1103	-103.76\\
1104	-141.602\\
1105	-114.746\\
1106	-142.822\\
1107	-104.98\\
1108	-90.332\\
1109	-103.76\\
1110	-76.904\\
1111	-69.58\\
1112	-84.229\\
1113	-86.67\\
1114	-59.814\\
1115	-64.697\\
1116	-48.828\\
1117	-53.711\\
1118	-57.373\\
1119	-41.504\\
1120	-52.49\\
1121	-65.918\\
1122	-101.318\\
1123	-104.98\\
1124	-114.746\\
1125	-68.359\\
1126	-93.994\\
1127	-150.146\\
1128	-114.746\\
1129	-125.732\\
1130	-128.174\\
1131	-80.566\\
1132	-53.711\\
1133	-75.684\\
1134	-85.449\\
1135	-137.939\\
1136	-172.119\\
1137	-170.898\\
1138	-123.291\\
1139	-130.615\\
1140	-115.967\\
1141	-113.525\\
1142	-111.084\\
1143	-76.904\\
1144	-68.359\\
1145	-61.035\\
1146	-57.373\\
1147	-62.256\\
1148	-91.553\\
1149	-140.381\\
1150	-111.084\\
1151	-79.346\\
1152	-64.697\\
1153	-62.256\\
1154	-40.283\\
1155	-29.297\\
1156	-36.621\\
1157	-63.477\\
1158	-51.27\\
1159	-52.49\\
1160	-58.594\\
1161	-54.932\\
1162	-37.842\\
1163	-28.076\\
1164	-23.193\\
1165	-39.063\\
1166	-83.008\\
1167	-117.188\\
1168	-130.615\\
1169	-86.67\\
1170	-58.594\\
1171	-45.166\\
1172	-32.959\\
1173	-67.139\\
1174	-65.918\\
1175	-91.553\\
1176	-117.188\\
1177	-186.768\\
1178	-216.064\\
1179	-177.002\\
1180	-168.457\\
1181	-109.863\\
1182	-122.07\\
1183	-131.836\\
1184	-146.484\\
1185	-108.643\\
1186	-100.098\\
1187	-98.877\\
1188	-89.111\\
1189	-106.201\\
1190	-78.125\\
1191	-130.615\\
1192	-159.912\\
1193	-113.525\\
1194	-70.801\\
1195	-73.242\\
1196	-123.291\\
1197	-153.809\\
1198	-189.209\\
1199	-202.637\\
1200	-202.637\\
1201	-153.809\\
1202	-137.939\\
1203	-158.691\\
1204	-187.988\\
1205	-109.863\\
1206	-72.021\\
1207	-101.318\\
1208	-86.67\\
1209	-53.711\\
1210	-74.463\\
1211	-84.229\\
1212	-67.139\\
1213	-51.27\\
1214	-70.801\\
1215	-58.594\\
1216	-79.346\\
1217	-107.422\\
1218	-100.098\\
1219	-69.58\\
1220	-70.801\\
1221	-107.422\\
1222	-177.002\\
1223	-128.174\\
1224	-79.346\\
1225	-61.035\\
1226	-70.801\\
1227	-64.697\\
1228	-48.828\\
1229	-53.711\\
1230	-65.918\\
1231	-83.008\\
1232	-91.553\\
1233	-63.477\\
1234	-104.98\\
1235	-124.512\\
1236	-118.408\\
1237	-91.553\\
1238	-100.098\\
1239	-135.498\\
1240	-87.891\\
1241	-42.725\\
1242	-57.373\\
1243	-62.256\\
1244	-81.787\\
1245	-72.021\\
1246	-52.49\\
1247	-79.346\\
1248	-80.566\\
1249	-57.373\\
1250	-70.801\\
1251	-63.477\\
1252	-56.152\\
1253	-67.139\\
1254	-41.504\\
1255	-48.828\\
1256	-36.621\\
1257	-40.283\\
1258	-75.684\\
1259	-84.229\\
1260	-113.525\\
1261	-146.484\\
1262	-98.877\\
1263	-58.594\\
1264	-46.387\\
1265	-43.945\\
1266	-63.477\\
1267	-42.725\\
1268	-47.607\\
1269	-61.035\\
1270	-102.539\\
1271	-93.994\\
1272	-134.277\\
1273	-89.111\\
1274	-81.787\\
1275	-46.387\\
1276	-45.166\\
1277	-28.076\\
1278	-35.4\\
1279	-37.842\\
1280	-67.139\\
1281	-70.801\\
1282	-79.346\\
1283	-85.449\\
1284	-125.732\\
1285	-112.305\\
1286	-84.229\\
1287	-102.539\\
1288	-109.863\\
1289	-144.043\\
1290	-115.967\\
1291	-75.684\\
1292	-40.283\\
1293	-29.297\\
1294	-29.297\\
1295	-36.621\\
1296	-39.063\\
1297	-37.842\\
1298	-50.049\\
1299	-74.463\\
1300	-68.359\\
1301	-70.801\\
1302	-83.008\\
1303	-51.27\\
1304	-30.518\\
1305	-58.594\\
1306	-90.332\\
1307	-87.891\\
1308	-97.656\\
1309	-83.008\\
1310	-61.035\\
1311	-85.449\\
1312	-118.408\\
1313	-118.408\\
1314	-84.229\\
1315	-136.719\\
1316	-100.098\\
1317	-101.318\\
1318	-117.188\\
1319	-115.967\\
1320	-86.67\\
1321	-76.904\\
1322	-128.174\\
1323	-178.223\\
1324	-139.16\\
1325	-79.346\\
1326	-76.904\\
1327	-79.346\\
1328	-98.877\\
1329	-114.746\\
1330	-85.449\\
1331	-90.332\\
1332	-120.85\\
1333	-131.836\\
1334	-87.891\\
1335	-68.359\\
1336	-91.553\\
1337	-156.25\\
1338	-137.939\\
1339	-140.381\\
1340	-84.229\\
1341	-76.904\\
1342	-72.021\\
1343	-47.607\\
1344	-30.518\\
1345	-26.855\\
1346	-23.193\\
1347	-54.932\\
1348	-70.801\\
1349	-87.891\\
1350	-65.918\\
1351	-73.242\\
1352	-83.008\\
1353	-53.711\\
1354	-56.152\\
1355	-53.711\\
1356	-40.283\\
1357	-51.27\\
1358	-84.229\\
1359	-106.201\\
1360	-63.477\\
1361	-48.828\\
1362	-40.283\\
1363	-50.049\\
1364	-35.4\\
1365	-76.904\\
1366	-130.615\\
1367	-104.98\\
1368	-139.16\\
1369	-162.354\\
1370	-164.795\\
1371	-124.512\\
1372	-90.332\\
1373	-89.111\\
1374	-89.111\\
1375	-106.201\\
1376	-125.732\\
1377	-125.732\\
1378	-168.457\\
1379	-183.105\\
1380	-219.727\\
1381	-147.705\\
1382	-166.016\\
1383	-184.326\\
1384	-140.381\\
1385	-162.354\\
1386	-139.16\\
1387	-80.566\\
1388	-52.49\\
1389	-56.152\\
1390	-81.787\\
1391	-65.918\\
1392	-43.945\\
1393	-62.256\\
1394	-47.607\\
1395	-36.621\\
1396	-46.387\\
1397	-52.49\\
1398	-36.621\\
1399	-70.801\\
1400	-92.773\\
1401	-68.359\\
1402	-114.746\\
1403	-164.795\\
1404	-150.146\\
1405	-196.533\\
1406	-131.836\\
1407	-144.043\\
1408	-96.436\\
1409	-63.477\\
1410	-76.904\\
1411	-59.814\\
1412	-45.166\\
1413	-64.697\\
1414	-36.621\\
1415	-28.076\\
1416	-34.18\\
1417	-70.801\\
1418	-86.67\\
1419	-107.422\\
1420	-103.76\\
1421	-100.098\\
1422	-114.746\\
1423	-93.994\\
1424	-76.904\\
1425	-76.904\\
1426	-62.256\\
1427	-97.656\\
1428	-68.359\\
1429	-56.152\\
1430	-81.787\\
1431	-113.525\\
1432	-86.67\\
1433	-65.918\\
1434	-56.152\\
1435	-65.918\\
1436	-45.166\\
1437	-45.166\\
1438	-59.814\\
1439	-75.684\\
1440	-84.229\\
1441	-65.918\\
1442	-86.67\\
1443	-79.346\\
1444	-47.607\\
1445	-50.049\\
1446	-95.215\\
1447	-131.836\\
1448	-131.836\\
1449	-107.422\\
1450	-103.76\\
1451	-79.346\\
1452	-76.904\\
1453	-61.035\\
1454	-89.111\\
1455	-75.684\\
1456	-50.049\\
1457	-74.463\\
1458	-85.449\\
1459	-101.318\\
1460	-109.863\\
1461	-109.863\\
1462	-148.926\\
1463	-181.885\\
1464	-205.078\\
1465	-129.395\\
1466	-73.242\\
1467	-47.607\\
1468	-34.18\\
1469	-54.932\\
1470	-46.387\\
1471	-80.566\\
1472	-72.021\\
1473	-64.697\\
1474	-85.449\\
1475	-98.877\\
1476	-117.188\\
1477	-123.291\\
1478	-175.781\\
1479	-150.146\\
1480	-117.188\\
1481	-80.566\\
1482	-80.566\\
1483	-83.008\\
1484	-106.201\\
1485	-75.684\\
1486	-79.346\\
1487	-74.463\\
1488	-42.725\\
1489	-74.463\\
1490	-107.422\\
1491	-73.242\\
1492	-80.566\\
1493	-147.705\\
1494	-109.863\\
1495	-106.201\\
1496	-147.705\\
1497	-140.381\\
1498	-87.891\\
1499	-56.152\\
1500	-58.594\\
};
\addlegendentry{True output}

\addplot [color=mycolor2, dashed]
  table[row sep=crcr]{%
1001	-85.8756471558183\\
1002	-108.781975897838\\
1003	-93.9974712580271\\
1004	-89.4772939880649\\
1005	-119.592201601983\\
1006	-105.370464520536\\
1007	-132.748351280165\\
1008	-104.931762756019\\
1009	-50.7605166987932\\
1010	-64.7981708146356\\
1011	-63.0915843052091\\
1012	-84.4129891825257\\
1013	-69.2122289037765\\
1014	-26.7702783148575\\
1015	-18.330887815544\\
1016	-15.8811324664409\\
1017	-53.0750856773175\\
1018	-97.3231810491477\\
1019	-102.86396249747\\
1020	-98.012687664862\\
1021	-69.5591603474269\\
1022	-38.1340493794365\\
1023	-92.1827780839704\\
1024	-61.2327889727107\\
1025	-51.6239957421089\\
1026	-87.1615955457352\\
1027	-71.4207821629431\\
1028	-63.5639501835796\\
1029	-100.656932405948\\
1030	-87.9960556079452\\
1031	-70.9656747799672\\
1032	-98.0233763731248\\
1033	-94.939874638527\\
1034	-80.752029344413\\
1035	-68.4948585985856\\
1036	-54.3485939604692\\
1037	-66.2225762139784\\
1038	-77.4629737823706\\
1039	-78.0912498278969\\
1040	-77.8276936973815\\
1041	-79.7049346251984\\
1042	-80.8794046746441\\
1043	-114.384922924191\\
1044	-105.335106603163\\
1045	-75.0889341430586\\
1046	-67.6859302134342\\
1047	-36.4191846509842\\
1048	-39.5086324797758\\
1049	-59.4444915089967\\
1050	-46.775582839451\\
1051	-51.5217532613501\\
1052	-68.641422517927\\
1053	-74.7721428773677\\
1054	-71.5515352011179\\
1055	-109.227122446292\\
1056	-79.7405793146656\\
1057	-48.111800036796\\
1058	-39.4822982607182\\
1059	-49.3079252339567\\
1060	-63.3152175355686\\
1061	-34.8496634270999\\
1062	-38.5340080664833\\
1063	-49.2208994835417\\
1064	-40.8319707841164\\
1065	-38.4988284743211\\
1066	-47.8755317930123\\
1067	-48.9109484243007\\
1068	-86.5104040109645\\
1069	-73.0384082533741\\
1070	-91.127798514347\\
1071	-80.5813627921775\\
1072	-87.0683229749219\\
1073	-74.7062858384942\\
1074	-71.5424063502072\\
1075	-66.2452288527443\\
1076	-66.3632111139621\\
1077	-64.9169035346241\\
1078	-105.871987366112\\
1079	-156.124622644753\\
1080	-161.950716519448\\
1081	-152.348807514668\\
1082	-97.4793216870099\\
1083	-138.238908462946\\
1084	-167.527896347524\\
1085	-175.70596256708\\
1086	-145.926551753494\\
1087	-173.466196576734\\
1088	-230.097844876827\\
1089	-177.48688171353\\
1090	-151.995568904892\\
1091	-100.855255766391\\
1092	-78.4157507840204\\
1093	-71.2112690532629\\
1094	-89.2219883239786\\
1095	-64.7731948291869\\
1096	-41.5762137311027\\
1097	-38.1976692003346\\
1098	-51.9331652671483\\
1099	-78.1824194842909\\
1100	-72.7748189646537\\
1101	-73.9527041323413\\
1102	-90.0840866965444\\
1103	-94.0698962690143\\
1104	-126.594176860882\\
1105	-111.533874231513\\
1106	-120.376813174784\\
1107	-94.8674922357301\\
1108	-79.6203973396839\\
1109	-95.1114844859548\\
1110	-72.3359268753987\\
1111	-67.8749189805355\\
1112	-83.0984459460096\\
1113	-81.8298272552157\\
1114	-64.2417485673695\\
1115	-66.0850897165648\\
1116	-48.3034192436949\\
1117	-50.7107738202193\\
1118	-58.4496895906325\\
1119	-44.3352942891684\\
1120	-53.1781351682474\\
1121	-63.3660758196293\\
1122	-98.3250286173859\\
1123	-91.4235983371455\\
1124	-104.603077537049\\
1125	-57.7555129035452\\
1126	-84.0872272647323\\
1127	-127.746436319712\\
1128	-103.125595576166\\
1129	-114.128078556118\\
1130	-108.023883095345\\
1131	-68.5556811028296\\
1132	-44.7650426369743\\
1133	-67.8571081248201\\
1134	-75.5626528091764\\
1135	-128.194063423056\\
1136	-153.481066618615\\
1137	-147.864349551485\\
1138	-106.517986880385\\
1139	-116.335406321578\\
1140	-102.681839004007\\
1141	-102.062316477502\\
1142	-105.155739136035\\
1143	-70.2749130666348\\
1144	-66.5958306265896\\
1145	-58.9902785598199\\
1146	-55.1201483838191\\
1147	-65.1696302739873\\
1148	-90.9770976354053\\
1149	-127.914728648067\\
1150	-111.906356807174\\
1151	-72.9946464124578\\
1152	-63.8775619633677\\
1153	-58.6783003597715\\
1154	-35.8639992019743\\
1155	-26.6158544787682\\
1156	-33.2295453682164\\
1157	-56.3501679575868\\
1158	-47.3306539114569\\
1159	-53.8166028582985\\
1160	-53.1880151769902\\
1161	-51.209649876813\\
1162	-36.8133711526278\\
1163	-29.1913287008618\\
1164	-21.5784139097438\\
1165	-32.4736398419213\\
1166	-75.1281829591923\\
1167	-110.077291061382\\
1168	-116.120052987912\\
1169	-78.3149851711529\\
1170	-48.7263578175166\\
1171	-35.3462100821797\\
1172	-24.1734263727574\\
1173	-57.6065085652929\\
1174	-55.4799217260177\\
1175	-80.5909021921064\\
1176	-101.087142277794\\
1177	-167.281172858719\\
1178	-194.111560602298\\
1179	-160.480985585159\\
1180	-142.3964177489\\
1181	-94.7777875540258\\
1182	-106.060930175999\\
1183	-120.715599585085\\
1184	-128.484864583248\\
1185	-108.391927096851\\
1186	-90.930285192826\\
1187	-99.4741307308238\\
1188	-90.7320252418332\\
1189	-101.89046119946\\
1190	-79.7289216784303\\
1191	-123.158016374611\\
1192	-150.711815539327\\
1193	-107.698527017693\\
1194	-72.6268787537307\\
1195	-67.8930808923659\\
1196	-113.050519763141\\
1197	-142.891314858754\\
1198	-163.723789497519\\
1199	-172.263877927729\\
1200	-173.84029981327\\
1201	-133.20839631612\\
1202	-126.158570328342\\
1203	-140.597118237057\\
1204	-163.350770713792\\
1205	-101.403548379271\\
1206	-65.2394580900068\\
1207	-100.684042198925\\
1208	-84.1819585630751\\
1209	-55.9673254717629\\
1210	-75.4745577663282\\
1211	-83.2561289181423\\
1212	-69.6611766822554\\
1213	-61.6790005874582\\
1214	-71.446116784998\\
1215	-61.9728889934971\\
1216	-74.7905104883132\\
1217	-110.025164831449\\
1218	-95.8660943225174\\
1219	-72.9686695127974\\
1220	-69.0698483611327\\
1221	-98.9452931398126\\
1222	-162.62935285924\\
1223	-131.064085672889\\
1224	-76.2726607590403\\
1225	-61.1110337253837\\
1226	-65.7494772530113\\
1227	-67.4707303343973\\
1228	-49.184487901371\\
1229	-57.8857255419981\\
1230	-64.8107963805425\\
1231	-80.2593388202381\\
1232	-91.5658197586818\\
1233	-63.9236791661498\\
1234	-97.3876536518374\\
1235	-115.409400335629\\
1236	-104.741272318663\\
1237	-85.7316778262958\\
1238	-89.7698398821794\\
1239	-115.925294977797\\
1240	-88.8139853045268\\
1241	-33.2363764279707\\
1242	-46.5751911987759\\
1243	-54.8038908593412\\
1244	-80.0853166822502\\
1245	-73.198581916334\\
1246	-56.4011035268316\\
1247	-80.0465405844137\\
1248	-71.2055409660861\\
1249	-54.4111288873932\\
1250	-70.809456924989\\
1251	-59.4729137075886\\
1252	-54.6788293961354\\
1253	-66.8546137233178\\
1254	-39.9647406143725\\
1255	-48.7845733275513\\
1256	-34.7757703239027\\
1257	-39.2536325530202\\
1258	-73.0581654781187\\
1259	-74.7946611004563\\
1260	-100.500590551516\\
1261	-130.69484428458\\
1262	-87.8099115799422\\
1263	-51.7337369055579\\
1264	-37.2439088948131\\
1265	-36.4284481152178\\
1266	-61.9997521243629\\
1267	-40.6450368856659\\
1268	-46.4611474593169\\
1269	-55.6030436347039\\
1270	-94.0905369557472\\
1271	-86.1146952295014\\
1272	-119.317270306922\\
1273	-81.2771577675406\\
1274	-66.8840394056158\\
1275	-34.9827539411686\\
1276	-34.4921217703222\\
1277	-23.0453047554055\\
1278	-29.7138966761695\\
1279	-34.8109649282721\\
1280	-59.4233191219673\\
1281	-64.4157765704903\\
1282	-69.0598776304082\\
1283	-70.8608433854086\\
1284	-105.817506673668\\
1285	-104.228341995503\\
1286	-73.7343425264116\\
1287	-84.8250239504844\\
1288	-91.8183609307161\\
1289	-117.469356706803\\
1290	-96.8508574023282\\
1291	-65.3883836588961\\
1292	-27.4953272699391\\
1293	-19.5863117798302\\
1294	-22.9012764922793\\
1295	-31.5126277299564\\
1296	-36.397220443073\\
1297	-36.5727948055756\\
1298	-42.3790852711318\\
1299	-68.7833360996199\\
1300	-62.3213105521933\\
1301	-60.3271378712103\\
1302	-67.475031441762\\
1303	-41.8627120561024\\
1304	-26.1540209508996\\
1305	-47.4369012882117\\
1306	-73.4422586009234\\
1307	-73.4571870983283\\
1308	-84.7306936354444\\
1309	-66.3034122085303\\
1310	-49.7077760912864\\
1311	-67.6607393203343\\
1312	-92.9700130417137\\
1313	-96.5345194540836\\
1314	-71.0875502000962\\
1315	-111.01206409922\\
1316	-85.6637373933219\\
1317	-80.2079491417663\\
1318	-100.801126898798\\
1319	-94.050725115293\\
1320	-78.2906044390477\\
1321	-70.7631626569724\\
1322	-108.254861544063\\
1323	-155.32202100177\\
1324	-121.966031174252\\
1325	-71.3431775721947\\
1326	-63.3984558161095\\
1327	-72.7782671031525\\
1328	-86.1915770469237\\
1329	-100.446714167238\\
1330	-79.182330001914\\
1331	-86.385562735571\\
1332	-105.628028690026\\
1333	-108.872978846652\\
1334	-78.568494676037\\
1335	-68.3452069759672\\
1336	-83.6253378013735\\
1337	-142.326385609468\\
1338	-125.26319396611\\
1339	-121.78644473948\\
1340	-75.6878088906359\\
1341	-63.9988657197975\\
1342	-66.3264825351296\\
1343	-41.4795963396261\\
1344	-26.8361437550982\\
1345	-21.6075153293021\\
1346	-19.6457621343238\\
1347	-45.6766490248392\\
1348	-66.0599086651134\\
1349	-81.9621864662466\\
1350	-61.6584650112253\\
1351	-60.1548117950183\\
1352	-67.4130260577116\\
1353	-41.5371716935127\\
1354	-46.8778444321015\\
1355	-44.6308342480773\\
1356	-34.5444255582418\\
1357	-46.6416918428564\\
1358	-69.8962584487476\\
1359	-90.4882583194646\\
1360	-54.7842905200869\\
1361	-42.5561624596549\\
1362	-35.7678840465475\\
1363	-44.3542092314954\\
1364	-30.8607919089673\\
1365	-71.5955780653889\\
1366	-117.823030255056\\
1367	-96.6477299291345\\
1368	-117.085886058351\\
1369	-131.279840981503\\
1370	-134.935040377696\\
1371	-101.027942970684\\
1372	-74.8852086083915\\
1373	-78.6810661094642\\
1374	-77.6426376099907\\
1375	-94.62684790852\\
1376	-109.768206918528\\
1377	-106.469685728954\\
1378	-140.67909660764\\
1379	-155.272568428374\\
1380	-179.080960225594\\
1381	-130.118275265813\\
1382	-143.109207960522\\
1383	-151.229815627344\\
1384	-127.7735433281\\
1385	-143.473200277865\\
1386	-127.514563467725\\
1387	-74.2189351874832\\
1388	-44.1386176769837\\
1389	-48.6829986222211\\
1390	-75.0993367114458\\
1391	-65.0783137454596\\
1392	-48.0033602277612\\
1393	-64.4305265797044\\
1394	-46.1489302920758\\
1395	-38.6028092105506\\
1396	-46.6238538488804\\
1397	-45.9894745233666\\
1398	-40.5272628437784\\
1399	-64.5952865619058\\
1400	-84.9274073270748\\
1401	-64.7306704351326\\
1402	-107.983297362985\\
1403	-154.572808763304\\
1404	-139.632048006407\\
1405	-169.689905543426\\
1406	-115.742525519512\\
1407	-125.923530701896\\
1408	-86.2481819449752\\
1409	-50.6562361987688\\
1410	-68.6041923654971\\
1411	-57.7798792489252\\
1412	-44.3090320719579\\
1413	-61.0547392391141\\
1414	-40.0094952324102\\
1415	-24.2627975862001\\
1416	-31.0052601561606\\
1417	-61.1039119975748\\
1418	-84.1614784784435\\
1419	-98.3167177316726\\
1420	-94.8527524964739\\
1421	-83.1201272937762\\
1422	-93.3749262581091\\
1423	-78.6580837617323\\
1424	-67.1801869245947\\
1425	-69.3659691597715\\
1426	-57.0802958073391\\
1427	-88.6940222564625\\
1428	-59.3587490352998\\
1429	-52.3273445808056\\
1430	-77.1928808209879\\
1431	-97.5083724354539\\
1432	-78.243073278266\\
1433	-61.2223618500198\\
1434	-53.7295847177309\\
1435	-60.7584881634399\\
1436	-42.3136247001052\\
1437	-41.969765006024\\
1438	-56.4874704385603\\
1439	-64.9891970392833\\
1440	-75.0558110957581\\
1441	-62.9354389468223\\
1442	-74.2987443150698\\
1443	-66.7046221421842\\
1444	-41.9109777609738\\
1445	-45.8929603089634\\
1446	-82.522618495864\\
1447	-122.290796633796\\
1448	-114.61174257422\\
1449	-92.929295583135\\
1450	-85.7555605676239\\
1451	-71.501159182665\\
1452	-66.0037305441524\\
1453	-58.9969904786343\\
1454	-74.9227343805362\\
1455	-73.6597455674379\\
1456	-46.2222933708766\\
1457	-70.5869418771113\\
1458	-72.0296214269847\\
1459	-86.7305910509578\\
1460	-97.9148631674249\\
1461	-88.8822818039363\\
1462	-130.04880794556\\
1463	-153.802720848001\\
1464	-171.910300059338\\
1465	-119.781430066221\\
1466	-59.5550303619634\\
1467	-34.5462721671724\\
1468	-23.6585844618198\\
1469	-43.9700992560787\\
1470	-41.4967819951212\\
1471	-77.2748572125817\\
1472	-67.5502438328214\\
1473	-61.1456686455546\\
1474	-76.3723721801872\\
1475	-82.7228933580437\\
1476	-100.454321559855\\
1477	-104.927116168167\\
1478	-150.565523516917\\
1479	-137.570189035985\\
1480	-103.062479689649\\
1481	-75.9491345705617\\
1482	-69.1772775993997\\
1483	-81.0941223821902\\
1484	-96.9778918666842\\
1485	-74.4545346071711\\
1486	-74.1614206884306\\
1487	-77.8984003096345\\
1488	-36.4628772571252\\
1489	-68.8615862616538\\
1490	-96.6519930458516\\
1491	-71.7693255688265\\
1492	-71.9605221742408\\
1493	-138.916057225027\\
1494	-95.8535280172065\\
1495	-98.0005547531333\\
1496	-126.559534249726\\
1497	-118.62926555078\\
1498	-83.1282598517199\\
1499	-51.5170785587445\\
1500	-51.4928349932362\\
};
\addlegendentry{Generated output}

\end{axis}
\end{tikzpicture}%\label{fig:c8all}}
	\caption{Samples of the output obtained experimentally and the output generated by the identified model.}\label{fig:Callout}
\end{figure}
\end{document}