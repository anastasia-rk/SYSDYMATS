% !TeX TXS-program:compile = txs:///pdflatex/[--shell-escape]

\documentclass[a4paper,11pt,twoside]{article}
%%%%%%%%%%%%%%%%%%%%%%%%%%%%%%%%%%%%%%%%%%%%%%%%%%%%%%%%%%%%%%%%%%%%%%%%%%%
% A generic report compiler for publishing rables and plots for a partcular choise of lags in the NARX system
% A path to media corresponding to the setting is defined below, as well as chosen parameter settings
\makeatletter
\def\input@path{{../delta_results_cv_f_set_V_lambda_2/}}
\def\dataset{C}
\def\ny{4}
\def\nu{4}
\def\order{3}
\makeatother
%%%%%%%%%%%%%%%%%%%%%%%%%%%%%%%%%%%%%%%%%%%%%%%%%%%%%%%%%%%%%%%%%%%%%%%%%%%
% Packages
\usepackage[final]{pdfpages}
\usepackage{verbatim}
\usepackage{inputenc}
\usepackage{graphicx} 
\usepackage{amsmath,amssymb,mathrsfs,amsfonts}
\usepackage{mathtools}
\usepackage{amsthm}
\usepackage{mathtools}
\usepackage{calrsfs}
\usepackage{graphicx}
\usepackage{subfig}
\usepackage{eucal}    
\usepackage{amssymb}  
\usepackage{pifont}
\usepackage{color} 
\usepackage{cancel}
\usepackage[toc,page]{appendix}
\usepackage{pgfplots}
\pgfplotsset{compat=newest,every axis/.append style={line width=0.5pt},x label style={font={\small},at={(axis description cs:0.5,-0.15)},anchor=north},y label style={font={\small},at={(axis description cs:-0.15,0.5)},anchor=south},z label style={font={\small},at={(axis description cs:-0.25,0.5)},anchor=north},label style={font=\small},tick label style={font=\small},title style={font=\small}}
\usetikzlibrary{shapes,shadows,arrows,backgrounds,patterns,positioning,automata,calc,decorations.markings,decorations.pathreplacing,bayesnet,arrows.meta} 
\usepackage{varwidth}
\usepackage{lscape}
\usepackage{array} 
\usepackage[colorlinks=false,pdfborder={0 0 0}]{hyperref}
\usepackage{tabularx}
\usepackage{textcomp}
\usepackage{multicol} 
\usepackage{booktabs}
\usepackage{multirow}
\usepackage[font=small,labelfont=bf]{caption}                                                           
\usepackage{textcase}
\usepackage{bbm} 
\usepackage{fancyhdr}
\usepackage{enumitem}
\usepackage{soul}
\usepackage{wrapfig}
%%%%%%%%%%%%%%%%%%%%%%%%%%%%%%%%%%%%%%%%%%%%%%%%%%%%%%%%%%%%%%%%%%%%%%%%%%%
% Geometry
\setlength{\parindent}{2em}
\setlength{\parskip}{0.5em}
\renewcommand{\baselinestretch}{1.2}
\usepackage[left=2cm, right=2cm, top=2.5cm, bottom=3cm, headheight=13.6pt]{geometry}
\allowdisplaybreaks 
%%%%%%%%%%%%%%%%%%%%%%%%%%%%%%%%%%%%%%%%%%%%%%%%%%%%%%%%%%%%%%%%%%%%%%%%%%%
% Bibliography
\usepackage[backend=bibtex,style=ieee,sorting=none]{biblatex} 
\bibliography{bibliography_ridge}
\renewcommand*{\bibfont}{\scriptsize}
%%%%%%%%%%%%%%%%%%%%%%%%%%%%%%%%%%%%%%%%%%%%%%%%%%%%%%%%%%%%%%%%%%%%%%%%%%%
% Custom commands and operators
\makeatletter
\newcommand*{\rom}[1]{\expandafter\@slowromancap\romannumeral #1@}
\newcommand{\ie}{\textit{i.e.} }
\newcommand{\eg}{\textit{e.g.} }
\newcommand\id{\ensuremath{\mathbbm{1}}} 
\newcommand{\norm}[1]{\left\lVert#1\right\rVert}
\DeclareMathOperator{\E}{\mathbb{E}}
\DeclareMathOperator{\eye}{\mathbb{I}}
\DeclareMathOperator{\zeros}{\mathbb{O}}
\DeclareMathOperator{\tr}{\textrm{tr}}
\DeclareMathOperator{\vvec}{\textrm{vec}}
\DeclareMathOperator{\ik}{\mathrm{k}}
\DeclareMathOperator{\ip}{\mathrm{p}}
\DeclareMathOperator{\inn}{\mathrm{n}}
\DeclareMathOperator{\im}{\mathrm{m}}
\DeclareMathOperator{\td}{\mathrm{t}}
\DeclareMathOperator{\kd}{\mathrm{k}}
\DeclareMathOperator{\T}{\mathrm{T}}
\DeclareMathOperator{\K}{\mathrm{K}}
\DeclareMathOperator{\rk}{\mathrm{rk}}
\DeclareMathOperator{\vc}{\mathrm{vec}}
\DeclareSymbolFontAlphabet{\mathcal} {symbols}
\DeclareSymbolFont{symbols}{OMS}{cm}{m}{n}
\DeclareMathAlphabet{\mathbfit}{OML}{cmm}{b}{it}
\makeatother
% Number equations
%\numberwithin{equation}{section}

%%%%%%%%%%%%%%%%%%%%%%%%%%%%%%%%%%%%%%%%%%%%%%%%%%%%%%%%%%%%%%%%%%%%%%%%%%%
%Theorems
\newtheoremstyle{mytheoremstyle} % name
{.5em}                    % Space above
{.8em}                    % Space below
{\itshape}                % Body font
{1em}                           % Indent amount
{\bfseries}                   % Theorem head font
{:}                          % Punctuation after theorem head
{.5em}                       % Space after theorem head
{}  % Theorem head spec (can be left empty, meaning ‘normal’)

\theoremstyle{mytheoremstyle}
\newtheorem{theorem}{Theorem}[section]
\newtheorem{remark}{Remark}[section]
\newtheorem{assumption}{Assumption}[section]
\newtheorem{lemma}{Lemma}[section]
\newtheorem{condition}{Condition}[section]
\newtheorem{definition}{Definition}[section]
\newtheorem{property}{Property}[section]
\newtheorem{corollary}{Corollary}[section]
\renewcommand\qedsymbol{$\blacksquare$}
%%%%%%%%%%%%%%%%%%%%%%%%%%%%%%%%%%%%%%%%%%%%%%%%%%%%%%%%%%%%%%%%%%%%%%%%%%%
% Nomenclature
\usepackage[intoc]{nomencl}
\makenomenclature

\usepackage[ruled]{algorithm}
\usepackage{float}

\usepackage{algorithmic}
\algsetup{linenosize=\scriptsize}
\usepackage{etoolbox}
\AtBeginEnvironment{algorithmic}{\scriptsize}
\renewcommand{\thealgorithm}{\thechapter.\arabic{algorithm}} 
%\usepackage{chngcntr}
%\counterwithin{algorithm}{section}
% correct bad hyphenation here
\hyphenation{op-tical net-works semi-conduc-tor}
%%%%%%%%%%%%%%%%%%%%%%%%%%%%%%%%%%%%%%%%%%%%%%%%%%%%%%%%%%%%%%%%%%%%%%%%%%%%%
\usepackage[explicit]{titlesec}
\usepackage{titletoc}
\interfootnotelinepenalty=10000

\title{Constrained parameter estimation of $\delta$-domain models}

\begin{document}
	\maketitle
\par This report demonstrates the performance of the $\delta$-domain identification framework on a simulation example.
\section{Model structure identification}
\par Equivalence between the lag models and $\delta$-domain models was established in \cite{ANDERSON20071859}. For a input-output model with a defined lags
\begin{equation}
	\mathbfit{y}(t) = f\big( \{y(t - k)\}^{n_y}_{k=1}, \{u(t - k + n_y + 1)\}^{n_y + n_u}_{k= n_y + 1} \big),
\end{equation}
there can be established an equivalent representation with $\delta$-operator, where the output of the NARX model is the highest order derivative of the registered output, $\delta^{n} y(t)\delta t^{n} y(t)$, and input vector takes the following form:
\begin{equation}
\mathbfit{x}(t) = \left[\begin{array}{cccccccc}
\delta^{n-1} y(t) & \dots & \delta y(t) & y(t) & \delta^{n-1}u(t-1) & \dots  & \delta u(t) & u(t)
\end{array}\right]^{\top}.
\end{equation}
The unknown model is approximated with a sum of polynomial basis functions up to second degree ($\lambda = \order$), rendering the following structure
\begin{equation}\label{eq:narx}
	\mathbfit{y}(t) = \theta^0 + \sum_{i=1}^{d} \theta_i x_i(t) + \sum_{i=1}^{d} \sum_{j=i}^{d} \theta_{i,j} x_i(t) x_j(t) +  \sum_{i=1}^{d} \sum_{j=i}^{d} \sum_{k=j}^{d} \theta_{i,j} x_i(t) x_j(t) x_k(t) + e(t).
\end{equation}
\par The performance of $\delta$-domain identification framework is tested on a simulated data for Van-der-Pol oscillator (VDPO) with varying damping strength. The dynamics of VDPO is described by a non-linear second-order ODE:
\begin{equation}
\frac{\delta^2}{\delta t^2} y(t) = \mu(1 - y^2(t))\frac{\delta}{\delta t} y(t) - y(t) + u(t),
\end{equation} 
where $u(t)$ is an excitation signal (in this case, sum of sinusoids). The damping coefficient $\mu$ was assigned the following values for the MC simulations
\begin{equation*}
\mu = \left[\begin{array}{ccccccc}0.0625 & 0.125 & 0.25 & 0.3 & 0.5 & 0.8 & 1.
\end{array}\right]
\end{equation*}
\par The number and order of significant terms are identified within the EFOR-CMSS truncated using Bayesian information criteria 
\begin{figure}[!h]
	\definecolor{mycolor1}{rgb}{0.00000,0.44700,0.74100}%
	\definecolor{mycolor2}{rgb}{0.85000,0.32500,0.09800}%
	\centering
	% This file was created by matlab2tikz.
%
\definecolor{mycolor1}{rgb}{0.00000,0.44700,0.74100}%
\definecolor{mycolor2}{rgb}{0.85000,0.32500,0.09800}%
%
\begin{tikzpicture}

\begin{axis}[%
width=5.832cm,
height=6cm,
at={(0cm,0cm)},
scale only axis,
xmin=0,
xmax=30,
xlabel style={font=\color{white!15!black}},
xlabel={Terms},
ymin=-93200,
ymax=-91800,
ylabel style={font=\color{white!15!black}},
ylabel={BIC},
axis background/.style={fill=white}
]
\addplot [color=mycolor1, forget plot]
  table[row sep=crcr]{%
1	-91984.3170620367\\
2	-92671.8941741199\\
3	-93024.9777751433\\
4	-93069.5581143147\\
5	-93070.2385875783\\
6	-93067.8240361525\\
7	-93074.3321718975\\
8	-93067.4204378494\\
9	-93069.967023701\\
10	-93063.8091780021\\
11	-93055.6327964656\\
12	-93048.2950270999\\
13	-93041.5378723637\\
14	-93033.5561546957\\
15	-93028.0214444791\\
16	-93019.5816145041\\
17	-93011.9984280773\\
18	-93004.2673469732\\
19	-92998.9543281201\\
20	-92994.4896362584\\
21	-92987.3398578412\\
22	-92979.7126260436\\
23	-92971.8868886109\\
24	-92963.3527874333\\
25	-92954.7555429889\\
26	-92946.0830430522\\
27	-92938.0387555589\\
28	-92929.4371242625\\
29	-92920.7259468241\\
30	-92911.6633754532\\
};
\addplot [color=mycolor2, draw=none, mark=asterisk, mark options={solid, mycolor2}, forget plot]
  table[row sep=crcr]{%
4	-93069.5581143147\\
};
\end{axis}
\end{tikzpicture}%
	\resizebox{!}{0cm}{
		\begin{minipage}{\textwidth}
			\begin{tikzpicture}	
			\begin{axis}[width=2cm,height=2cm]
			\addplot [color=mycolor2,line width=5.0pt,only marks,mark=asterisk,mark options={solid},forget plot]
			table[row sep=crcr]{%
				8	-3.00104696019893\\
			};\label{tikz:nterms}
			\end{axis}
			\end{tikzpicture}%
	\end{minipage}}
	\caption{Evolution of BIC with growing number of parameters in the model}\label{fig:bic}
\end{figure}	
\par The significant terms identified by the algorithm are presented in Table \ref{tab:thetas_all}
\begin{table}[!h]
	\centering
	\caption{Significant terms and corresponding coefficients identified in EFOR-CMSS algorithm.}\label{tab:theetas_all}
	\small
	\begin{tabular}{rrrrrrrrrrr}
Step & Terms & V1 & V2 & V3 & V4 & V5 & V6 & V7 & AERR($\%$) & BIC \\ 
\hline 
1 & $y(t-1)$ & 2 & 2 & 2.01 & 2.02 & 2.04 & 2.07 & 2.09 & 98.991 & 0 \\ 
2 & $y(t-2)$ & -1.01 & -1.01 & -1.02 & -1.03 & -1.05 & -1.08 & -1.1 & 1.002 & -90731.5566 \\ 
3 & $u(t-1)$ & 0.01 & 0.01 & 0.01 & 0.01 & 0.01 & 0.01 & 0.01 & 0.004 & -98870.2779 \\ 
4 & $y(t-2)y(t-2)y(t-2)$ & 0 & 0 & -0.01 & -0.01 & -0.02 & -0.04 & -0.05 & 0 & -99124.2257 \\ 
5 & $y(t-1)y(t-1)y(t-1)$ & 0 & -0.01 & -0.02 & -0.02 & -0.03 & -0.06 & -0.07 & 0.003 & -112106.805 \\ 
6 & $y(t-2)y(t-2)y(t-1)$ & 0.01 & 0.01 & 0.03 & 0.03 & 0.05 & 0.09 & 0.12 & 0 & -113689.0794 \\ 
7 & $y(t-1)y(t-1)u(t-1)$ & 0 & 0 & 0 & 0 & 0 & 0 & 0 & 0 & -113982.8189 \\ 
8 & $u(t-2)$ & 0 & 0 & 0 & 0 & 0 & 0 & 0 & 0 & -114016.8129 \\ 
\hline 
\end{tabular}
\end{table} 
\section{Direct estimation of external model parameters}
\par In order to link the external and internal parameters, an arbitrary polynomial function is formed from either a single parameter vector or a pair of vectors. The number of unknown parameters is defined by the number of avaliable datasets. This section demonstrates the direct estimation procedure and the justification for selecting polynomial terms for curve fitting.
\par Model structure for $K$ datasets:
\begin{equation}\label{eq:batchtimeser}
\underbrace{\bar{\mathbf{Y}}}_{\T K\times 1} = \underbrace{\bar{\Phi}}_{\T K \times NK} \underbrace{\bar{\Theta}}_{NK \times 1},
\end{equation}
where the block matrices have the following structure:
\begin{equation}
\underbrace{\bar{\mathbf{Y}}}_{\T K\times 1} = \left[\begin{array}{c} 
\underbrace{\mathbfit{y}^1}_{\T\times 1} \\
\underbrace{\mathbfit{y}^2}_{\T\times 1} \\
\vdots \\
\underbrace{\mathbfit{y}^K}_{\T\times 1}
\end{array}\right]; \quad 
\underbrace{\bar{\Phi}}_{\T K \times NK} = \left[\begin{array}{cccc} 
\underbrace{\Phi^1}_{\T \times N} & \dots & \dots & \zeros \\
\zeros & \underbrace{\Phi^2}_{\T \times N} & \dots & \zeros \\
\vdots & \vdots & \vdots & \vdots  \\
\zeros & \dots & \dots & \underbrace{\Phi^K}_{\T \times N}
\end{array}\right]; \quad 
\underbrace{\bar{\Theta}}_{NK \times 1} = \left[\begin{array}{c} 
\underbrace{\theta^1}_{N \times 1} \\
\underbrace{\theta^2}_{N \times 1} \\
\vdots \\
\underbrace{\theta^K}_{N \times 1}
\end{array}\right].
\end{equation}
\par The relationship of the design parameters known from the experiments and the internal parameters of NARMAX model  is defined by the following linear function:
\begin{equation}
\underbrace{\Theta}_{N \times K} = \underbrace{B}_{N \times L} \underbrace{A}_{L \times K},
\end{equation}
where $A$ is the matrix where each row is a function of the vector of design parameters. The example structure is
\begin{equation}
A = \left[\begin{array}{cccccc}
\eye_{K \times 1} & L_{K \times 1} & D_{K \times 1} & L D_{K \times 1} &  L^{2}_{K \times 1} & D^{2}_{K \times 1} 
\end{array}\right]^{\top},
\end{equation} 
and where $B$ denotes the matrix of unknown coefficients of a hypersurface of order $L$ that maps a point in external parameter space, $\xi^k = (L_k, D_k)$, onto the point in the space of internal parameters, $\theta^k$.
In can be seen that $\bar{\Theta} = \text{vec}(\Theta)$, then
\begin{equation}
\underbrace{\bar{\Theta}}_{NK \times 1} = \text{vec}\left(\underbrace{B}_{N \times L} \underbrace{A}_{L \times K}\right).
\end{equation}
This vectorisation can be obtained using Kronecker product:
\begin{equation}
\text{vec}\left(\underbrace{B}_{N \times L} \underbrace{A}_{L \times K}\right) = (\underbrace{A^{\top}}_{K \times L} \otimes \underbrace{\eye}_{N \times N}) \underbrace{\text{vec}(B)}_{NL \times 1}.
\end{equation}
Denoting the result of Kronecker product as $\underbrace{\mathbf{Kr}}_{NK \times NL} \triangleq (\underbrace{A^{\top}}_{K \times L} \otimes \underbrace{\eye}_{N \times N})$ and vectorised coefficient matrix as $\underbrace{\bar{\mathbf{B}}}_{NL \times 1} \triangleq \text{vec}(B)$  yields the following:
\begin{equation}\label{eq:BtoTheta}
\underbrace{\bar{\Theta}}_{NK \times 1} = \underbrace{\mathbf{Kr}}_{KN \times NL} \underbrace{\bar{\mathbf{B}}}_{NL \times 1}.
\end{equation}
Substituting the above expression into \eqref{eq:batchtimeser} renders an expression that directly links the design parameters and the timeseries data
\begin{equation}\label{eq:BtoY}
\underbrace{\bar{\mathbf{Y}}}_{\T K\times 1} = \underbrace{\bar{\Phi}}_{\T K \times NK} \underbrace{\mathbf{Kr}}_{KN \times NL} \underbrace{\bar{\mathbf{B}}}_{NL \times 1},
\end{equation}
where $\bar{\mathbf{B}}$ is the unknown vector and all other factors are known form the experiments or defined prior to structure identification. 
\par The representation \eqref{eq:BtoY} allows estimating the coefficients in  $\bar{\mathbf{B}}$ directly from the timeseries data bypassing the intermediate estimation of the internal coefficients in NARX model.
\par The following condition must be satisfied:
\begin{equation}\label{eq:rankcond}
\rk(\bar{\Phi}\mathbf{Kr}) \geq NL.
\end{equation}
The rank of the linear system \eqref{eq:BtoY} satisfies the following:
\begin{equation}
\rk (\bar{\Phi}\mathbf{Kr}) \leq \min\left(\rk(\bar{\Phi}), \rk(\mathbf{Kr})\right),
\end{equation}
where the rank of Kronecker product can be found as
\begin{equation}
\rk(\mathbf{Kr}) = \rk(\underbrace{A^{\top}}_{K \times L})\rk(\underbrace{\eye}_{N \times N}),
\end{equation}
thus the matrix $A$ composed of by-element combinations of external parameter vector(s) must be of rank $K$.
\section{Constrained estimation}
\par A common treatment for models that suffer from bad generalisation is to introduce a constrained LS problem, where the constraint is normally posed on the unknown parameter vector:
\begin{equation}\label{eq:rls_const}
\hat{\mathbf{\beta}}^i = \arg \min \norm{ \biggl(\bar{\theta}^i - X \mathbf{\beta}^i \biggr) }^2, \quad f_R(\mathbf{\beta}^i) < \gamma.
\end{equation}
where $\lambda$ is a pre-specified parameter that defines the size of the constraint in the parameter space $\Xi$.
Lagrangian formulation of the constrained problem is called regularised Least Squares (RLS),
\begin{equation}
\mathbf{\beta}^i = \arg \min \Biggl \{\norm{ \biggl(\bar{\theta}^i - X \mathbf{\beta}^i \biggr) }^2 + \lambda f_R(\mathbf{\beta}^i)\Biggr\}
\end{equation}
where the regularisation coefficient $\lambda$ is directly linked to  $\gamma$ in \eqref{eq:rls_const}. Different types of the constraint function are be considered depending on the problem.
\subsection{Tikhonov regularisation}
Innovation regularisation  constrains the 2-norm of the parameter vector
\begin{equation}
f_{R}(\mathbf{\beta}^i) = \norm{\mathbf{\beta}^i}^{2}_{2}.
\end{equation}
This is the only RLS formulation that has a closed form solution that is usually obtained for the normalised data. 
\begin{equation}\label{eq:ridge}
\hat{\mathbf{\beta}}^{i}_{RLS} = (\mathbf{R}^{*}_{aa} + \lambda \eye_M)^{-1} (\mathbf{A}^{\star})^{\top}\bar{\theta}^i,  
\end{equation}
This solution is referred to as ridge regression, because increasing $\lambda$ shrinks the coefficients $\beta_j$. The shrinkage is shown is best interpreted via singular values decomposition (SVD) of the normalised data matrix. Denote the SVD of $\mathbf{A}^{\star}$ as
\begin{equation}
\underbrace{\mathbf{A}^{\star}}_{K \times M} = \underbrace{\mathbf{U}}_{K \times M}\underbrace{\mathbf{D}}_{M \times M}\underbrace{\mathbf{V}^{\top}}_{M \times M},
\end{equation}
where columns of $\mathbf{U}$ are principal components of $\mathbf{A}^{\star}$, diagonal elements of  $\mathbf{D}$ are the singular values, and where $\mathbf{V}$ is the rotation. All orthonormality assumption are the same as in the general case. The ridge regression \eqref{eq:ridge} then takes form
\begin{equation}
\hat{\mathbf{\beta}}^{i}_{RLS} = \left( \mathbf{V}\mathbf{D}\mathbf{U}^{\top}\mathbf{U}\mathbf{D}\mathbf{V}^{\top}  + \lambda \eye_M\right)^{-1} \left( \mathbf{U}\mathbf{D}\mathbf{V}^{\top}\right)^{\top}\bar{\theta}^i.
\end{equation}
Simple linear algebra yields the following
\begin{equation}
\hat{\mathbf{\beta}}^{i}_{RLS} =  \mathbf{V}\left(\mathbf{D}^2  + \lambda \eye_M\right)^{-1}\mathbf{V}^{\top} \mathbf{V}\mathbf{D}\mathbf{U}^{\top}\bar{\theta}^i.
\end{equation}
The finale expression is
\begin{equation}\label{eq:rls_svd}
\hat{\mathbf{\beta}}^{i}_{RLS} = \mathbf{V}\left(\mathbf{D}^2  + \lambda \eye_M\right)^{-1}\mathbf{D}\mathbf{U}^{\top}\bar{\theta}^i,
\end{equation}
which can be compared to the SVD of OLS regression:
\begin{equation}\label{eq:ols_svd}
\hat{\mathbf{\beta}}^{i}_{OLS} = \mathbf{V}\mathbf{D}^{-1}\mathbf{U}^{\top}\bar{\theta}^i \qquad \left( \approx (\mathbf{A}^{\star})^{-1}\bar{\theta}^i\right).
\end{equation}
It can be seen from \eqref{eq:rls_svd}-\eqref{eq:ols_svd} that in Tikhonov regularisation the inverse of the diagonal matrix is obtained as $\mathbf{D} / (\mathbf{D}^2  + \lambda \eye_M)$ where $\lambda$ is non-negative. Increasing regularisation coefficient thus leads to shrinkage of the singular values, and the estimates asymptotically approach zero. This shows that in Tikhonov regularisation $\lambda$ quantifies the trade-off between the bias and the variance in the estimates. Small regularisation coefficient leads to near-OLS solution that overfits the model to the training data, while large value leads to biased estimates and drives all coefficients to near-zero values.
\subsection{LASSO regularisation}
\par While in the ridge regression LSEs asymptotically approach zero, none of the parameters can be zeroed-out explicitly if the model structure is overly detailed. Another formulation of RLS, called Least absolute shrinkage and selection operator (LASSO) regression, performs variable selection and the regularisation simultaneously by imposing the $l_1$ penalty: 
\begin{equation}
f_{R}(\mathbf{\beta}^i) = \norm{\mathbf{\beta}^i}_{1}.
\end{equation}
The Lagrangian optimisation problem then takes for of basis pursuit de-noising that can be solved numerically using quadratic programming or convex techniques. This report uses the shooting algorithm proposed in \cite{Fu1998} because of its relative simplicity. The results of ridge estimation are used as the initial point in convex optimisation searching for the LASSO solution.
\subsection{Selection of the regularisation coefficient}
\par The important stage of solving RLS problem is selecting the regularisation parameter that will result into a interpretable but parsimonious model. The constraint $\gamma$ in \eqref{eq:rls_constr} is often selected arbitrary since the shape of the parameter space is unknown. As a result, finding $\lambda$ relies on iterative schemes most of which do not guarantee convergence [CITE MANY]. For the lack of universal approach for selecting the optimal value of $\lambda$, the choice of the method remains application-specific.
\par In this report, ridge estimation is applied to a fixed model structure, hence Aikaike’s information criterion (AIC) may be used
\begin{equation}
\text{AIC}_{\lambda} = 2 p - 2\log p(\bar{\theta}^i \mid \mathbf{\beta}^i, \lambda),
\end{equation}
where the first term quantifies model complexity and the second term is the log-likelihood of the selected model fitting the data. The model complexity is determined as simply the trace of the hat matrix of the ridge estimator
\begin{equation}
p = \tr(\mathbf{H}_{RLS}) = \tr((\mathbf{A}^{\top})(\mathbf{R}_{aa} + \lambda \eye_M)^{-1} \mathbf{A}^{\top})
\end{equation}.
Bayesian information criterion (BIC), assigns a larger penalty to the model complexity 
\begin{equation}
\text{BIC}_{\lambda} = 2\log(n) p - 2\log p(\bar{\theta}^i \mid \mathbf{\beta}^i, \lambda),
\end{equation}
where $n$ is the number of data points used for parameter estimation. Both AIC and BIC are aim to estimate 

\par When it comes to finding a parsimonious model, it may be more reasonable to access model's prediction performance instead of explicitly penalising its complexity. Cross-validation procedure  
This work 
\par Selcting regularisation coefficient for LASSO regression is more nuanced as different model structure may arise for different values of $\lambda$. The most popular approach described in the literature uses cross-validation. Both the data matrix and the response vector are partitioned into pairs. Then each pair is exuded from the 
\section{Results}
The estimated coefficients are presented in Table \ref{tab:betas_all}, and the surface fitting results for each internal parameter are illustrated in Figure \ref{fig:surfaces_all}

\begin{table}[!h]
	\centering
	\caption{Polynomial coefficients estimated via ordinary LS.}\label{tab:betas_all}
	\small
	\begin{tabular}{rrrrrrrrrrr}
Step & Terms & $\beta_{0}$ & $\beta_{1}$ & $\beta_{2}$ & $\beta_{3}$ & $\beta_{4}$ & $\beta_{5}$ & $\beta_{6}$ & $\beta_{7}$ & $\beta_{8}$ \\ 
\hline 
1 & $\delta^1 y(t)$ & 1175.87 & -10095.27 & 20091 & -50.49 & 395.03 & -686.33 & 0.45 & -3.58 & 5.77 \\ 
2 & $y(t)y(t)y(t)$ & -128886.14 & -20167.02 & -4694.76 & -7041 & 71024.93 & -31152.47 & 119.95 & -1018.17 & 509.13 \\ 
3 & $c$ & -962.4 & 1876.76 & -13254.42 & -84.69 & 1864.26 & -7013.18 & 1.31 & -26.02 & 100.2 \\ 
4 & $y(t)$ & 5843.59 & -85876.07 & -20912.33 & -874.68 & 10156.69 & -19925.03 & 10.76 & -128.35 & 295.04 \\ 
5 & $y(t)\delta^1 u(t)$ & 34257.29 & 4105.26 & -230.18 & -3735.84 & 21112.09 & -21037.45 & 41.81 & -264.05 & 240.53 \\ 
6 & $y(t)\delta^1 u(t)$ & 34257.27 & 4104.98 & -230.12 & -3735.84 & 21112.09 & -21037.45 & 41.81 & -264.05 & 240.53 \\ 
\hline 
\end{tabular}
\end{table} 
\begin{table}[!h]
	\centering
	\caption{Polynomial coefficients estimated via Tikhonov regularisation.}\label{tab:betas_tikh}
	\small
	\begin{tabular}{rrrrrr}
Step & Terms & $\beta_{0}$ & $\beta_{1}$ & $\beta_{2}$ & $\beta_{3}$ \\ 
\hline 
1 & $y(t)$ & -99.66 & -36.65 & -86.47 & 62.92 \\ 
2 & $u(t)$ & 95.83 & 7.52 & -19.58 & 13.48 \\ 
3 & $y(t)y(t)\delta^1 y(t)$ & -0.01 & -9.42 & -0.47 & 0.16 \\ 
4 & $\delta^1 u(t)$ & 0.92 & 0.13 & -0.48 & 0.32 \\ 
5 & $\delta^1 y(t)$ & -1.02 & 9.5 & -1.36 & 0.03 \\ 
6 & $y(t)y(t)y(t)$ & -0.29 & 10.27 & 24.3 & -17.61 \\ 
7 & $y(t)\delta^1 y(t)u(t)$ & 0.04 & -1.04 & 1.69 & -1.66 \\ 
8 & $y(t)y(t)u(t)$ & -0.43 & 1.17 & -10.6 & 8.54 \\ 
\hline 
\end{tabular}
\end{table}

\begin{table}[!h]
	\centering
	\caption{Polynomial coefficients estimated via LASSO regularisation.}\label{tab:betas_lass}
	\small
	\begin{tabular}{rrrrrr}
Step & Terms & $\beta_{0}$ & $\beta_{1}$ & $\beta_{2}$ & $\beta_{3}$ \\ 
\hline 
1 & $y(t)$ & -99.66 & -36.65 & -86.47 & 62.92 \\ 
2 & $u(t)$ & 95.83 & 7.52 & -19.58 & 13.48 \\ 
3 & $y(t)y(t)\delta^1 y(t)$ & -0.01 & -9.42 & -0.47 & 0.16 \\ 
4 & $\delta^1 u(t)$ & 0.92 & 0.13 & -0.48 & 0.32 \\ 
5 & $\delta^1 y(t)$ & -1.02 & 9.5 & -1.36 & 0.03 \\ 
6 & $y(t)y(t)y(t)$ & -0.29 & 10.27 & 24.3 & -17.61 \\ 
7 & $y(t)\delta^1 y(t)u(t)$ & 0.04 & -1.04 & 1.69 & -1.66 \\ 
8 & $y(t)y(t)u(t)$ & -0.43 & 1.17 & -10.6 & 8.54 \\ 
\hline 
\end{tabular}
\end{table}
%The figure also shows values of the internal parameters computed for the external settings of experiments \dataset3 and \dataset8. The obtained internal parameters are substituted in the modified version of model \eqref{eq:narx} that only includes the identified significant  polynomial terms to validate the identified model structure. The simulation results are compared with true system outputs for \dataset3 and \dataset8 in Figure \ref{fig:c3all} and Figure \ref{fig:c8all}, respectively. Computed RMSEs for both lengths of the sample are presented in Table \ref{tab:RMSEs}.
%\begin{table}[!h]
%	\centering
%	\caption{RMSE of the system output generated by the identified model.}\label{tab:RMSEs}
%		\begin{tabular}{cc}
%			Sample size 2000 & Sample size 4000 \\
%			\hline	
%			\begin{minipage}{2in} \verbatiminput{RMSEs_T_2000.txt}\end{minipage}& 
%			\begin{minipage}{2in} \vspace{0.5cm}\verbatiminput{RMSEs_T_4000.txt}\end{minipage}\\
%			\hline
%		\end{tabular}
%\end{table}

%\begin{figure}[!t]
%	\centering
%	\resizebox{!}{0cm}{
%		\begin{minipage}{\textwidth}
%		% This file was created by matlab2tikz.
%
\definecolor{mycolor1}{rgb}{0.00000,0.44700,0.74100}%
\definecolor{mycolor2}{rgb}{0.85000,0.32500,0.09800}%
%
\begin{tikzpicture}

\begin{axis}[%
width=3.159cm,
height=3.097cm,
at={(0cm,12.903cm)},
scale only axis,
xmin=56,
xmax=74,
tick align=outside,
axis background/.style={fill=white},
xmajorgrids,
ymajorgrids,
zmajorgrids
]
\addplot3[only marks, mark=*, mark options={}, mark size=1.5000pt, color=mycolor1, fill=mycolor1] table[row sep=crcr]{%
x	y	z\\
74	0.123	-26.0353957891804\\
72	0.113	-20.9879322169279\\
61	0.095	-10.6920630070547\\
56	0.093	-10.9569379219045\\
};\label{tikz:thetas1}
\addplot3[only marks, mark=*, mark options={}, mark size=1.5000pt, color=mycolor2, fill=mycolor2] table[row sep=crcr]{%
x	y	z\\
67	0.276	-191.779551108501\\
66	0.255	-157.643535964989\\
62	0.209	-87.4196213968237\\
57	0.193	-69.8013569503948\\
};\label{tikz:thetas2}
\addplot3[only marks, mark=*, mark options={}, mark size=1.5000pt, color=black, fill=black] table[row sep=crcr]{%
x	y	z\\
69	0.104	-15.57705861622\\
};\label{tikz:thetaidentified}
\addplot3[only marks, mark=*, mark options={}, mark size=1.5000pt, color=black, fill=black] table[row sep=crcr]{%
x	y	z\\
64	0.23	-116.694087150299\\
};
\addplot3[%
surf,
fill opacity=0.7, shader=interp, colormap={mymap}{[1pt] rgb(0pt)=(1,0.905882,0); rgb(1pt)=(1,0.901964,0); rgb(2pt)=(1,0.898051,0); rgb(3pt)=(1,0.894144,0); rgb(4pt)=(1,0.890243,0); rgb(5pt)=(1,0.886349,0); rgb(6pt)=(1,0.88246,0); rgb(7pt)=(1,0.878577,0); rgb(8pt)=(1,0.8747,0); rgb(9pt)=(1,0.870829,0); rgb(10pt)=(1,0.866964,0); rgb(11pt)=(1,0.863106,0); rgb(12pt)=(1,0.859253,0); rgb(13pt)=(1,0.855406,0); rgb(14pt)=(1,0.851566,0); rgb(15pt)=(1,0.847732,0); rgb(16pt)=(1,0.843903,0); rgb(17pt)=(1,0.840081,0); rgb(18pt)=(1,0.836265,0); rgb(19pt)=(1,0.832455,0); rgb(20pt)=(1,0.828652,0); rgb(21pt)=(1,0.824854,0); rgb(22pt)=(1,0.821063,0); rgb(23pt)=(1,0.817278,0); rgb(24pt)=(1,0.8135,0); rgb(25pt)=(1,0.809727,0); rgb(26pt)=(1,0.805961,0); rgb(27pt)=(1,0.8022,0); rgb(28pt)=(1,0.798445,0); rgb(29pt)=(1,0.794696,0); rgb(30pt)=(1,0.790953,0); rgb(31pt)=(1,0.787215,0); rgb(32pt)=(1,0.783484,0); rgb(33pt)=(1,0.779758,0); rgb(34pt)=(1,0.776038,0); rgb(35pt)=(1,0.772324,0); rgb(36pt)=(1,0.768615,0); rgb(37pt)=(1,0.764913,0); rgb(38pt)=(1,0.761217,0); rgb(39pt)=(1,0.757527,0); rgb(40pt)=(1,0.753843,0); rgb(41pt)=(1,0.750165,0); rgb(42pt)=(1,0.746493,0); rgb(43pt)=(1,0.742827,0); rgb(44pt)=(1,0.739167,0); rgb(45pt)=(1,0.735514,0); rgb(46pt)=(1,0.731867,0); rgb(47pt)=(1,0.728226,0); rgb(48pt)=(1,0.724591,0); rgb(49pt)=(1,0.720963,0); rgb(50pt)=(1,0.717341,0); rgb(51pt)=(1,0.713725,0); rgb(52pt)=(0.999994,0.710077,0); rgb(53pt)=(0.999974,0.706363,0); rgb(54pt)=(0.999942,0.702592,0); rgb(55pt)=(0.999898,0.698775,0); rgb(56pt)=(0.999841,0.694921,0); rgb(57pt)=(0.999771,0.691039,0); rgb(58pt)=(0.99969,0.687139,0); rgb(59pt)=(0.999596,0.68323,0); rgb(60pt)=(0.99949,0.679323,0); rgb(61pt)=(0.999372,0.675427,0); rgb(62pt)=(0.999242,0.67155,0); rgb(63pt)=(0.9991,0.667704,0); rgb(64pt)=(0.998946,0.663897,0); rgb(65pt)=(0.998781,0.660138,0); rgb(66pt)=(0.998605,0.656439,0); rgb(67pt)=(0.998416,0.652807,0); rgb(68pt)=(0.998217,0.649253,0); rgb(69pt)=(0.998006,0.645786,0); rgb(70pt)=(0.997785,0.642416,0); rgb(71pt)=(0.997552,0.639152,0); rgb(72pt)=(0.997308,0.636004,0); rgb(73pt)=(0.997053,0.632982,0); rgb(74pt)=(0.996788,0.630095,0); rgb(75pt)=(0.996512,0.627352,0); rgb(76pt)=(0.996226,0.624763,0); rgb(77pt)=(0.995851,0.622329,0); rgb(78pt)=(0.99494,0.619997,0); rgb(79pt)=(0.99345,0.617753,0); rgb(80pt)=(0.991419,0.61559,0); rgb(81pt)=(0.988885,0.613503,0); rgb(82pt)=(0.985886,0.611486,0); rgb(83pt)=(0.98246,0.609532,0); rgb(84pt)=(0.978643,0.607636,0); rgb(85pt)=(0.974475,0.605791,0); rgb(86pt)=(0.969992,0.603992,0); rgb(87pt)=(0.965232,0.602233,0); rgb(88pt)=(0.960233,0.600507,0); rgb(89pt)=(0.955033,0.598808,0); rgb(90pt)=(0.949669,0.59713,0); rgb(91pt)=(0.94418,0.595468,0); rgb(92pt)=(0.938602,0.593815,0); rgb(93pt)=(0.932974,0.592166,0); rgb(94pt)=(0.927333,0.590513,0); rgb(95pt)=(0.921717,0.588852,0); rgb(96pt)=(0.916164,0.587176,0); rgb(97pt)=(0.910711,0.585479,0); rgb(98pt)=(0.905397,0.583755,0); rgb(99pt)=(0.900258,0.581999,0); rgb(100pt)=(0.895333,0.580203,0); rgb(101pt)=(0.890659,0.578362,0); rgb(102pt)=(0.886275,0.576471,0); rgb(103pt)=(0.882047,0.574545,0); rgb(104pt)=(0.877819,0.572608,0); rgb(105pt)=(0.873592,0.57066,0); rgb(106pt)=(0.869366,0.568701,0); rgb(107pt)=(0.865143,0.566733,0); rgb(108pt)=(0.860924,0.564756,0); rgb(109pt)=(0.856708,0.562771,0); rgb(110pt)=(0.852497,0.560778,0); rgb(111pt)=(0.848292,0.558779,0); rgb(112pt)=(0.844092,0.556774,0); rgb(113pt)=(0.8399,0.554763,0); rgb(114pt)=(0.835716,0.552749,0); rgb(115pt)=(0.831541,0.55073,0); rgb(116pt)=(0.827374,0.548709,0); rgb(117pt)=(0.823219,0.546686,0); rgb(118pt)=(0.819074,0.54466,0); rgb(119pt)=(0.81494,0.542635,0); rgb(120pt)=(0.81082,0.540609,0); rgb(121pt)=(0.806712,0.538584,0); rgb(122pt)=(0.802619,0.53656,0); rgb(123pt)=(0.798541,0.534539,0); rgb(124pt)=(0.794478,0.532521,0); rgb(125pt)=(0.790431,0.530506,0); rgb(126pt)=(0.786402,0.528496,0); rgb(127pt)=(0.782391,0.526491,0); rgb(128pt)=(0.77841,0.524489,0); rgb(129pt)=(0.774523,0.522478,0); rgb(130pt)=(0.770731,0.520455,0); rgb(131pt)=(0.767022,0.518424,0); rgb(132pt)=(0.763384,0.516385,0); rgb(133pt)=(0.759804,0.514339,0); rgb(134pt)=(0.756272,0.51229,0); rgb(135pt)=(0.752775,0.510237,0); rgb(136pt)=(0.749302,0.508182,0); rgb(137pt)=(0.74584,0.506128,0); rgb(138pt)=(0.742378,0.504075,0); rgb(139pt)=(0.738904,0.502025,0); rgb(140pt)=(0.735406,0.499979,0); rgb(141pt)=(0.731872,0.49794,0); rgb(142pt)=(0.72829,0.495909,0); rgb(143pt)=(0.724649,0.493887,0); rgb(144pt)=(0.720936,0.491875,0); rgb(145pt)=(0.71714,0.489876,0); rgb(146pt)=(0.713249,0.487891,0); rgb(147pt)=(0.709251,0.485921,0); rgb(148pt)=(0.705134,0.483968,0); rgb(149pt)=(0.700887,0.482033,0); rgb(150pt)=(0.696497,0.480118,0); rgb(151pt)=(0.691952,0.478225,0); rgb(152pt)=(0.687242,0.476355,0); rgb(153pt)=(0.682353,0.47451,0); rgb(154pt)=(0.677195,0.472696,0); rgb(155pt)=(0.6717,0.470916,0); rgb(156pt)=(0.665891,0.469169,0); rgb(157pt)=(0.659791,0.46745,0); rgb(158pt)=(0.653423,0.465756,0); rgb(159pt)=(0.64681,0.464084,0); rgb(160pt)=(0.639976,0.462432,0); rgb(161pt)=(0.632943,0.460795,0); rgb(162pt)=(0.625734,0.459171,0); rgb(163pt)=(0.618373,0.457556,0); rgb(164pt)=(0.610882,0.455948,0); rgb(165pt)=(0.603284,0.454343,0); rgb(166pt)=(0.595604,0.452737,0); rgb(167pt)=(0.587863,0.451129,0); rgb(168pt)=(0.580084,0.449514,0); rgb(169pt)=(0.572292,0.447889,0); rgb(170pt)=(0.564508,0.446252,0); rgb(171pt)=(0.556756,0.444599,0); rgb(172pt)=(0.549059,0.442927,0); rgb(173pt)=(0.54144,0.441232,0); rgb(174pt)=(0.533922,0.439512,0); rgb(175pt)=(0.526529,0.437764,0); rgb(176pt)=(0.519282,0.435983,0); rgb(177pt)=(0.512206,0.434168,0); rgb(178pt)=(0.505323,0.432315,0); rgb(179pt)=(0.498628,0.430422,3.92506e-06); rgb(180pt)=(0.491973,0.428504,3.49981e-05); rgb(181pt)=(0.485331,0.426562,9.63073e-05); rgb(182pt)=(0.478704,0.424596,0.000186979); rgb(183pt)=(0.472096,0.422609,0.000306141); rgb(184pt)=(0.465508,0.420599,0.00045292); rgb(185pt)=(0.458942,0.418567,0.000626441); rgb(186pt)=(0.452401,0.416515,0.000825833); rgb(187pt)=(0.445885,0.414441,0.00105022); rgb(188pt)=(0.439399,0.412348,0.00129873); rgb(189pt)=(0.432942,0.410234,0.00157049); rgb(190pt)=(0.426518,0.408102,0.00186463); rgb(191pt)=(0.420129,0.40595,0.00218028); rgb(192pt)=(0.413777,0.40378,0.00251655); rgb(193pt)=(0.407464,0.401592,0.00287258); rgb(194pt)=(0.401191,0.399386,0.00324749); rgb(195pt)=(0.394962,0.397164,0.00364042); rgb(196pt)=(0.388777,0.394925,0.00405048); rgb(197pt)=(0.38264,0.39267,0.00447681); rgb(198pt)=(0.376552,0.390399,0.00491852); rgb(199pt)=(0.370516,0.388113,0.00537476); rgb(200pt)=(0.364532,0.385812,0.00584464); rgb(201pt)=(0.358605,0.383497,0.00632729); rgb(202pt)=(0.352735,0.381168,0.00682184); rgb(203pt)=(0.346925,0.378826,0.00732741); rgb(204pt)=(0.341176,0.376471,0.00784314); rgb(205pt)=(0.335485,0.374093,0.00847245); rgb(206pt)=(0.329843,0.371682,0.00930909); rgb(207pt)=(0.324249,0.369242,0.0103377); rgb(208pt)=(0.318701,0.366772,0.0115428); rgb(209pt)=(0.313198,0.364275,0.0129091); rgb(210pt)=(0.307739,0.361753,0.0144211); rgb(211pt)=(0.302322,0.359206,0.0160634); rgb(212pt)=(0.296945,0.356637,0.0178207); rgb(213pt)=(0.291607,0.354048,0.0196776); rgb(214pt)=(0.286307,0.35144,0.0216186); rgb(215pt)=(0.281043,0.348814,0.0236284); rgb(216pt)=(0.275813,0.346172,0.0256916); rgb(217pt)=(0.270616,0.343517,0.0277927); rgb(218pt)=(0.265451,0.340849,0.0299163); rgb(219pt)=(0.260317,0.33817,0.0320472); rgb(220pt)=(0.25521,0.335482,0.0341698); rgb(221pt)=(0.250131,0.332786,0.0362688); rgb(222pt)=(0.245078,0.330085,0.0383287); rgb(223pt)=(0.240048,0.327379,0.0403343); rgb(224pt)=(0.235042,0.324671,0.04227); rgb(225pt)=(0.230056,0.321962,0.0441205); rgb(226pt)=(0.22509,0.319254,0.0458704); rgb(227pt)=(0.220142,0.316548,0.0475043); rgb(228pt)=(0.215212,0.313846,0.0490067); rgb(229pt)=(0.210296,0.311149,0.0503624); rgb(230pt)=(0.205395,0.308459,0.0515759); rgb(231pt)=(0.200514,0.305763,0.052757); rgb(232pt)=(0.195655,0.303061,0.0539242); rgb(233pt)=(0.190817,0.300353,0.0550763); rgb(234pt)=(0.186001,0.297639,0.0562123); rgb(235pt)=(0.181207,0.294918,0.0573313); rgb(236pt)=(0.176434,0.292191,0.0584321); rgb(237pt)=(0.171685,0.289458,0.0595136); rgb(238pt)=(0.166957,0.286719,0.060575); rgb(239pt)=(0.162252,0.283973,0.0616151); rgb(240pt)=(0.15757,0.281221,0.0626328); rgb(241pt)=(0.152911,0.278463,0.0636271); rgb(242pt)=(0.148275,0.275699,0.0645971); rgb(243pt)=(0.143663,0.272929,0.0655416); rgb(244pt)=(0.139074,0.270152,0.0664596); rgb(245pt)=(0.134508,0.26737,0.06735); rgb(246pt)=(0.129967,0.264581,0.0682118); rgb(247pt)=(0.125449,0.261787,0.0690441); rgb(248pt)=(0.120956,0.258986,0.0698456); rgb(249pt)=(0.116487,0.25618,0.0706154); rgb(250pt)=(0.112043,0.253367,0.0713525); rgb(251pt)=(0.107623,0.250549,0.0720557); rgb(252pt)=(0.103229,0.247724,0.0727241); rgb(253pt)=(0.0988592,0.244894,0.0733566); rgb(254pt)=(0.0945149,0.242058,0.0739522); rgb(255pt)=(0.0901961,0.239216,0.0745098)}, mesh/rows=49]
table[row sep=crcr, point meta=\thisrow{c}] {%
%
x	y	z	c\\
56	0.093	-10.9049623108072	-10.9049623108072\\
56	0.09666	-11.4423373867092	-11.4423373867092\\
56	0.10032	-12.101944154261	-12.101944154261\\
56	0.10398	-12.8837826134625	-12.8837826134625\\
56	0.10764	-13.7878527643139	-13.7878527643139\\
56	0.1113	-14.814154606815	-14.814154606815\\
56	0.11496	-15.9626881409659	-15.9626881409659\\
56	0.11862	-17.2334533667667	-17.2334533667667\\
56	0.12228	-18.6264502842172	-18.6264502842172\\
56	0.12594	-20.1416788933175	-20.1416788933175\\
56	0.1296	-21.7791391940677	-21.7791391940677\\
56	0.13326	-23.5388311864676	-23.5388311864676\\
56	0.13692	-25.4207548705173	-25.4207548705173\\
56	0.14058	-27.4249102462168	-27.4249102462168\\
56	0.14424	-29.5512973135661	-29.5512973135661\\
56	0.1479	-31.7999160725652	-31.7999160725652\\
56	0.15156	-34.1707665232141	-34.1707665232141\\
56	0.15522	-36.6638486655127	-36.6638486655127\\
56	0.15888	-39.2791624994612	-39.2791624994612\\
56	0.16254	-42.0167080250595	-42.0167080250595\\
56	0.1662	-44.8764852423076	-44.8764852423076\\
56	0.16986	-47.8584941512054	-47.8584941512054\\
56	0.17352	-50.9627347517531	-50.9627347517531\\
56	0.17718	-54.1892070439505	-54.1892070439505\\
56	0.18084	-57.5379110277977	-57.5379110277977\\
56	0.1845	-61.0088467032948	-61.0088467032948\\
56	0.18816	-64.6020140704416	-64.6020140704416\\
56	0.19182	-68.3174131292382	-68.3174131292382\\
56	0.19548	-72.1550438796846	-72.1550438796846\\
56	0.19914	-76.1149063217809	-76.1149063217809\\
56	0.2028	-80.1970004555269	-80.1970004555269\\
56	0.20646	-84.4013262809227	-84.4013262809227\\
56	0.21012	-88.7278837979683	-88.7278837979683\\
56	0.21378	-93.1766730066637	-93.1766730066637\\
56	0.21744	-97.7476939070088	-97.7476939070088\\
56	0.2211	-102.440946499004	-102.440946499004\\
56	0.22476	-107.256430782649	-107.256430782649\\
56	0.22842	-112.194146757943	-112.194146757943\\
56	0.23208	-117.254094424887	-117.254094424887\\
56	0.23574	-122.436273783482	-122.436273783482\\
56	0.2394	-127.740684833726	-127.740684833726\\
56	0.24306	-133.167327575619	-133.167327575619\\
56	0.24672	-138.716202009163	-138.716202009163\\
56	0.25038	-144.387308134356	-144.387308134356\\
56	0.25404	-150.180645951199	-150.180645951199\\
56	0.2577	-156.096215459692	-156.096215459692\\
56	0.26136	-162.134016659835	-162.134016659835\\
56	0.26502	-168.294049551628	-168.294049551628\\
56	0.26868	-174.57631413507	-174.57631413507\\
56	0.27234	-180.980810410162	-180.980810410162\\
56	0.276	-187.507538376904	-187.507538376904\\
56.375	0.093	-10.8126294158988	-10.8126294158988\\
56.375	0.09666	-11.3527833115828	-11.3527833115828\\
56.375	0.10032	-12.0151688989166	-12.0151688989166\\
56.375	0.10398	-12.7997861779002	-12.7997861779002\\
56.375	0.10764	-13.7066351485335	-13.7066351485335\\
56.375	0.1113	-14.7357158108167	-14.7357158108167\\
56.375	0.11496	-15.8870281647496	-15.8870281647496\\
56.375	0.11862	-17.1605722103324	-17.1605722103324\\
56.375	0.12228	-18.5563479475649	-18.5563479475649\\
56.375	0.12594	-20.0743553764473	-20.0743553764473\\
56.375	0.1296	-21.7145944969793	-21.7145944969793\\
56.375	0.13326	-23.4770653091613	-23.4770653091613\\
56.375	0.13692	-25.361767812993	-25.361767812993\\
56.375	0.14058	-27.3687020084745	-27.3687020084745\\
56.375	0.14424	-29.4978678956059	-29.4978678956059\\
56.375	0.1479	-31.7492654743869	-31.7492654743869\\
56.375	0.15156	-34.1228947448178	-34.1228947448178\\
56.375	0.15522	-36.6187557068985	-36.6187557068985\\
56.375	0.15888	-39.236848360629	-39.236848360629\\
56.375	0.16254	-41.9771727060093	-41.9771727060093\\
56.375	0.1662	-44.8397287430394	-44.8397287430394\\
56.375	0.16986	-47.8245164717192	-47.8245164717192\\
56.375	0.17352	-50.9315358920489	-50.9315358920489\\
56.375	0.17718	-54.1607870040283	-54.1607870040283\\
56.375	0.18084	-57.5122698076576	-57.5122698076576\\
56.375	0.1845	-60.9859843029366	-60.9859843029366\\
56.375	0.18816	-64.5819304898655	-64.5819304898655\\
56.375	0.19182	-68.3001083684441	-68.3001083684441\\
56.375	0.19548	-72.1405179386725	-72.1405179386725\\
56.375	0.19914	-76.1031592005507	-76.1031592005507\\
56.375	0.2028	-80.1880321540788	-80.1880321540788\\
56.375	0.20646	-84.3951367992566	-84.3951367992566\\
56.375	0.21012	-88.7244731360842	-88.7244731360842\\
56.375	0.21378	-93.1760411645616	-93.1760411645616\\
56.375	0.21744	-97.7498408846888	-97.7498408846888\\
56.375	0.2211	-102.445872296466	-102.445872296466\\
56.375	0.22476	-107.264135399893	-107.264135399893\\
56.375	0.22842	-112.204630194969	-112.204630194969\\
56.375	0.23208	-117.267356681695	-117.267356681695\\
56.375	0.23574	-122.452314860072	-122.452314860072\\
56.375	0.2394	-127.759504730098	-127.759504730098\\
56.375	0.24306	-133.188926291773	-133.188926291773\\
56.375	0.24672	-138.740579545099	-138.740579545099\\
56.375	0.25038	-144.414464490074	-144.414464490074\\
56.375	0.25404	-150.210581126699	-150.210581126699\\
56.375	0.2577	-156.128929454974	-156.128929454974\\
56.375	0.26136	-162.169509474899	-162.169509474899\\
56.375	0.26502	-168.332321186474	-168.332321186474\\
56.375	0.26868	-174.617364589698	-174.617364589698\\
56.375	0.27234	-181.024639684572	-181.024639684572\\
56.375	0.276	-187.554146471096	-187.554146471096\\
56.75	0.093	-10.7298928959079	-10.7298928959079\\
56.75	0.09666	-11.2728256113739	-11.2728256113739\\
56.75	0.10032	-11.9379900184897	-11.9379900184897\\
56.75	0.10398	-12.7253861172552	-12.7253861172552\\
56.75	0.10764	-13.6350139076706	-13.6350139076706\\
56.75	0.1113	-14.6668733897358	-14.6668733897358\\
56.75	0.11496	-15.8209645634508	-15.8209645634508\\
56.75	0.11862	-17.0972874288155	-17.0972874288155\\
56.75	0.12228	-18.49584198583	-18.49584198583\\
56.75	0.12594	-20.0166282344944	-20.0166282344944\\
56.75	0.1296	-21.6596461748085	-21.6596461748085\\
56.75	0.13326	-23.4248958067724	-23.4248958067724\\
56.75	0.13692	-25.3123771303862	-25.3123771303862\\
56.75	0.14058	-27.3220901456497	-27.3220901456497\\
56.75	0.14424	-29.454034852563	-29.454034852563\\
56.75	0.1479	-31.7082112511262	-31.7082112511262\\
56.75	0.15156	-34.084619341339	-34.084619341339\\
56.75	0.15522	-36.5832591232017	-36.5832591232017\\
56.75	0.15888	-39.2041305967142	-39.2041305967142\\
56.75	0.16254	-41.9472337618765	-41.9472337618765\\
56.75	0.1662	-44.8125686186886	-44.8125686186886\\
56.75	0.16986	-47.8001351671505	-47.8001351671505\\
56.75	0.17352	-50.9099334072622	-50.9099334072622\\
56.75	0.17718	-54.1419633390236	-54.1419633390236\\
56.75	0.18084	-57.4962249624349	-57.4962249624349\\
56.75	0.1845	-60.9727182774959	-60.9727182774959\\
56.75	0.18816	-64.5714432842068	-64.5714432842068\\
56.75	0.19182	-68.2923999825674	-68.2923999825674\\
56.75	0.19548	-72.1355883725778	-72.1355883725778\\
56.75	0.19914	-76.1010084542381	-76.1010084542381\\
56.75	0.2028	-80.1886602275481	-80.1886602275481\\
56.75	0.20646	-84.3985436925079	-84.3985436925079\\
56.75	0.21012	-88.7306588491175	-88.7306588491175\\
56.75	0.21378	-93.185005697377	-93.185005697377\\
56.75	0.21744	-97.7615842372861	-97.7615842372861\\
56.75	0.2211	-102.460394468845	-102.460394468845\\
56.75	0.22476	-107.281436392054	-107.281436392054\\
56.75	0.22842	-112.224710006913	-112.224710006913\\
56.75	0.23208	-117.290215313421	-117.290215313421\\
56.75	0.23574	-122.477952311579	-122.477952311579\\
56.75	0.2394	-127.787921001387	-127.787921001387\\
56.75	0.24306	-133.220121382845	-133.220121382845\\
56.75	0.24672	-138.774553455952	-138.774553455952\\
56.75	0.25038	-144.45121722071	-144.45121722071\\
56.75	0.25404	-150.250112677117	-150.250112677117\\
56.75	0.2577	-156.171239825174	-156.171239825174\\
56.75	0.26136	-162.214598664881	-162.214598664881\\
56.75	0.26502	-168.380189196237	-168.380189196237\\
56.75	0.26868	-174.668011419243	-174.668011419243\\
56.75	0.27234	-181.078065333899	-181.078065333899\\
56.75	0.276	-187.610350940205	-187.610350940205\\
57.125	0.093	-10.6567527508345	-10.6567527508345\\
57.125	0.09666	-11.2024642860824	-11.2024642860824\\
57.125	0.10032	-11.8704075129803	-11.8704075129803\\
57.125	0.10398	-12.6605824315278	-12.6605824315278\\
57.125	0.10764	-13.5729890417252	-13.5729890417252\\
57.125	0.1113	-14.6076273435724	-14.6076273435724\\
57.125	0.11496	-15.7644973370694	-15.7644973370694\\
57.125	0.11862	-17.0435990222161	-17.0435990222161\\
57.125	0.12228	-18.4449323990127	-18.4449323990127\\
57.125	0.12594	-19.968497467459	-19.968497467459\\
57.125	0.1296	-21.6142942275552	-21.6142942275552\\
57.125	0.13326	-23.3823226793011	-23.3823226793011\\
57.125	0.13692	-25.2725828226968	-25.2725828226968\\
57.125	0.14058	-27.2850746577424	-27.2850746577424\\
57.125	0.14424	-29.4197981844377	-29.4197981844377\\
57.125	0.1479	-31.6767534027828	-31.6767534027828\\
57.125	0.15156	-34.0559403127778	-34.0559403127778\\
57.125	0.15522	-36.5573589144225	-36.5573589144225\\
57.125	0.15888	-39.181009207717	-39.181009207717\\
57.125	0.16254	-41.9268911926613	-41.9268911926613\\
57.125	0.1662	-44.7950048692553	-44.7950048692553\\
57.125	0.16986	-47.7853502374992	-47.7853502374992\\
57.125	0.17352	-50.8979272973929	-50.8979272973929\\
57.125	0.17718	-54.1327360489364	-54.1327360489364\\
57.125	0.18084	-57.4897764921296	-57.4897764921296\\
57.125	0.1845	-60.9690486269727	-60.9690486269727\\
57.125	0.18816	-64.5705524534656	-64.5705524534656\\
57.125	0.19182	-68.2942879716082	-68.2942879716082\\
57.125	0.19548	-72.1402551814006	-72.1402551814006\\
57.125	0.19914	-76.1084540828429	-76.1084540828429\\
57.125	0.2028	-80.1988846759349	-80.1988846759349\\
57.125	0.20646	-84.4115469606768	-84.4115469606768\\
57.125	0.21012	-88.7464409370684	-88.7464409370684\\
57.125	0.21378	-93.2035666051098	-93.2035666051098\\
57.125	0.21744	-97.782923964801	-97.782923964801\\
57.125	0.2211	-102.484513016142	-102.484513016142\\
57.125	0.22476	-107.308333759133	-107.308333759133\\
57.125	0.22842	-112.254386193773	-112.254386193773\\
57.125	0.23208	-117.322670320064	-117.322670320064\\
57.125	0.23574	-122.513186138004	-122.513186138004\\
57.125	0.2394	-127.825933647594	-127.825933647594\\
57.125	0.24306	-133.260912848834	-133.260912848834\\
57.125	0.24672	-138.818123741723	-138.818123741723\\
57.125	0.25038	-144.497566326263	-144.497566326263\\
57.125	0.25404	-150.299240602452	-150.299240602452\\
57.125	0.2577	-156.223146570291	-156.223146570291\\
57.125	0.26136	-162.269284229779	-162.269284229779\\
57.125	0.26502	-168.437653580918	-168.437653580918\\
57.125	0.26868	-174.728254623706	-174.728254623706\\
57.125	0.27234	-181.141087358144	-181.141087358144\\
57.125	0.276	-187.676151784232	-187.676151784232\\
57.5	0.093	-10.5932089806785	-10.5932089806785\\
57.5	0.09666	-11.1416993357085	-11.1416993357085\\
57.5	0.10032	-11.8124213823883	-11.8124213823883\\
57.5	0.10398	-12.6053751207179	-12.6053751207179\\
57.5	0.10764	-13.5205605506973	-13.5205605506973\\
57.5	0.1113	-14.5579776723265	-14.5579776723265\\
57.5	0.11496	-15.7176264856055	-15.7176264856055\\
57.5	0.11862	-16.9995069905342	-16.9995069905342\\
57.5	0.12228	-18.4036191871128	-18.4036191871128\\
57.5	0.12594	-19.9299630753412	-19.9299630753412\\
57.5	0.1296	-21.5785386552193	-21.5785386552193\\
57.5	0.13326	-23.3493459267472	-23.3493459267472\\
57.5	0.13692	-25.242384889925	-25.242384889925\\
57.5	0.14058	-27.2576555447525	-27.2576555447525\\
57.5	0.14424	-29.3951578912299	-29.3951578912299\\
57.5	0.1479	-31.654891929357	-31.654891929357\\
57.5	0.15156	-34.0368576591339	-34.0368576591339\\
57.5	0.15522	-36.5410550805606	-36.5410550805606\\
57.5	0.15888	-39.1674841936372	-39.1674841936372\\
57.5	0.16254	-41.9161449983635	-41.9161449983635\\
57.5	0.1662	-44.7870374947396	-44.7870374947396\\
57.5	0.16986	-47.7801616827655	-47.7801616827655\\
57.5	0.17352	-50.8955175624411	-50.8955175624411\\
57.5	0.17718	-54.1331051337666	-54.1331051337666\\
57.5	0.18084	-57.4929243967419	-57.4929243967419\\
57.5	0.1845	-60.974975351367	-60.974975351367\\
57.5	0.18816	-64.5792579976418	-64.5792579976418\\
57.5	0.19182	-68.3057723355665	-68.3057723355665\\
57.5	0.19548	-72.1545183651409	-72.1545183651409\\
57.5	0.19914	-76.1254960863652	-76.1254960863652\\
57.5	0.2028	-80.2187054992393	-80.2187054992393\\
57.5	0.20646	-84.4341466037631	-84.4341466037631\\
57.5	0.21012	-88.7718193999367	-88.7718193999367\\
57.5	0.21378	-93.2317238877601	-93.2317238877601\\
57.5	0.21744	-97.8138600672334	-97.8138600672334\\
57.5	0.2211	-102.518227938356	-102.518227938356\\
57.5	0.22476	-107.344827501129	-107.344827501129\\
57.5	0.22842	-112.293658755552	-112.293658755552\\
57.5	0.23208	-117.364721701624	-117.364721701624\\
57.5	0.23574	-122.558016339346	-122.558016339346\\
57.5	0.2394	-127.873542668718	-127.873542668718\\
57.5	0.24306	-133.31130068974	-133.31130068974\\
57.5	0.24672	-138.871290402412	-138.871290402412\\
57.5	0.25038	-144.553511806733	-144.553511806733\\
57.5	0.25404	-150.357964902704	-150.357964902704\\
57.5	0.2577	-156.284649690325	-156.284649690325\\
57.5	0.26136	-162.333566169596	-162.333566169596\\
57.5	0.26502	-168.504714340516	-168.504714340516\\
57.5	0.26868	-174.798094203087	-174.798094203087\\
57.5	0.27234	-181.213705757307	-181.213705757307\\
57.5	0.276	-187.751549003177	-187.751549003177\\
57.875	0.093	-10.5392615854401	-10.5392615854401\\
57.875	0.09666	-11.0905307602521	-11.0905307602521\\
57.875	0.10032	-11.7640316267139	-11.7640316267139\\
57.875	0.10398	-12.5597641848255	-12.5597641848255\\
57.875	0.10764	-13.4777284345869	-13.4777284345869\\
57.875	0.1113	-14.5179243759981	-14.5179243759981\\
57.875	0.11496	-15.680352009059	-15.680352009059\\
57.875	0.11862	-16.9650113337699	-16.9650113337699\\
57.875	0.12228	-18.3719023501304	-18.3719023501304\\
57.875	0.12594	-19.9010250581408	-19.9010250581408\\
57.875	0.1296	-21.5523794578009	-21.5523794578009\\
57.875	0.13326	-23.3259655491109	-23.3259655491109\\
57.875	0.13692	-25.2217833320706	-25.2217833320706\\
57.875	0.14058	-27.2398328066802	-27.2398328066802\\
57.875	0.14424	-29.3801139729396	-29.3801139729396\\
57.875	0.1479	-31.6426268308487	-31.6426268308487\\
57.875	0.15156	-34.0273713804076	-34.0273713804076\\
57.875	0.15522	-36.5343476216163	-36.5343476216163\\
57.875	0.15888	-39.1635555544749	-39.1635555544749\\
57.875	0.16254	-41.9149951789832	-41.9149951789832\\
57.875	0.1662	-44.7886664951413	-44.7886664951413\\
57.875	0.16986	-47.7845695029492	-47.7845695029492\\
57.875	0.17352	-50.9027042024069	-50.9027042024069\\
57.875	0.17718	-54.1430705935144	-54.1430705935144\\
57.875	0.18084	-57.5056686762716	-57.5056686762716\\
57.875	0.1845	-60.9904984506787	-60.9904984506787\\
57.875	0.18816	-64.5975599167356	-64.5975599167356\\
57.875	0.19182	-68.3268530744423	-68.3268530744423\\
57.875	0.19548	-72.1783779237987	-72.1783779237987\\
57.875	0.19914	-76.152134464805	-76.152134464805\\
57.875	0.2028	-80.2481226974611	-80.2481226974611\\
57.875	0.20646	-84.4663426217669	-84.4663426217669\\
57.875	0.21012	-88.8067942377226	-88.8067942377226\\
57.875	0.21378	-93.269477545328	-93.269477545328\\
57.875	0.21744	-97.8543925445832	-97.8543925445832\\
57.875	0.2211	-102.561539235488	-102.561539235488\\
57.875	0.22476	-107.390917618043	-107.390917618043\\
57.875	0.22842	-112.342527692248	-112.342527692248\\
57.875	0.23208	-117.416369458102	-117.416369458102\\
57.875	0.23574	-122.612442915606	-122.612442915606\\
57.875	0.2394	-127.93074806476	-127.93074806476\\
57.875	0.24306	-133.371284905564	-133.371284905564\\
57.875	0.24672	-138.934053438018	-138.934053438018\\
57.875	0.25038	-144.619053662121	-144.619053662121\\
57.875	0.25404	-150.426285577874	-150.426285577874\\
57.875	0.2577	-156.355749185277	-156.355749185277\\
57.875	0.26136	-162.40744448433	-162.40744448433\\
57.875	0.26502	-168.581371475032	-168.581371475032\\
57.875	0.26868	-174.877530157385	-174.877530157385\\
57.875	0.27234	-181.295920531387	-181.295920531387\\
57.875	0.276	-187.836542597039	-187.836542597039\\
58.25	0.093	-10.4949105651191	-10.4949105651191\\
58.25	0.09666	-11.0489585597131	-11.0489585597131\\
58.25	0.10032	-11.7252382459569	-11.7252382459569\\
58.25	0.10398	-12.5237496238505	-12.5237496238505\\
58.25	0.10764	-13.444492693394	-13.444492693394\\
58.25	0.1113	-14.4874674545871	-14.4874674545871\\
58.25	0.11496	-15.6526739074302	-15.6526739074302\\
58.25	0.11862	-16.9401120519229	-16.9401120519229\\
58.25	0.12228	-18.3497818880655	-18.3497818880655\\
58.25	0.12594	-19.8816834158579	-19.8816834158579\\
58.25	0.1296	-21.5358166353001	-21.5358166353001\\
58.25	0.13326	-23.312181546392	-23.312181546392\\
58.25	0.13692	-25.2107781491338	-25.2107781491338\\
58.25	0.14058	-27.2316064435254	-27.2316064435254\\
58.25	0.14424	-29.3746664295667	-29.3746664295667\\
58.25	0.1479	-31.6399581072579	-31.6399581072579\\
58.25	0.15156	-34.0274814765988	-34.0274814765988\\
58.25	0.15522	-36.5372365375895	-36.5372365375895\\
58.25	0.15888	-39.1692232902301	-39.1692232902301\\
58.25	0.16254	-41.9234417345204	-41.9234417345204\\
58.25	0.1662	-44.7998918704605	-44.7998918704605\\
58.25	0.16986	-47.7985736980504	-47.7985736980504\\
58.25	0.17352	-50.9194872172901	-50.9194872172901\\
58.25	0.17718	-54.1626324281796	-54.1626324281796\\
58.25	0.18084	-57.5280093307189	-57.5280093307189\\
58.25	0.1845	-61.015617924908	-61.015617924908\\
58.25	0.18816	-64.6254582107469	-64.6254582107469\\
58.25	0.19182	-68.3575301882356	-68.3575301882356\\
58.25	0.19548	-72.211833857374	-72.211833857374\\
58.25	0.19914	-76.1883692181623	-76.1883692181623\\
58.25	0.2028	-80.2871362706004	-80.2871362706004\\
58.25	0.20646	-84.5081350146882	-84.5081350146882\\
58.25	0.21012	-88.8513654504259	-88.8513654504259\\
58.25	0.21378	-93.3168275778133	-93.3168275778133\\
58.25	0.21744	-97.9045213968505	-97.9045213968505\\
58.25	0.2211	-102.614446907538	-102.614446907538\\
58.25	0.22476	-107.446604109874	-107.446604109874\\
58.25	0.22842	-112.400993003861	-112.400993003861\\
58.25	0.23208	-117.477613589497	-117.477613589497\\
58.25	0.23574	-122.676465866784	-122.676465866784\\
58.25	0.2394	-127.99754983572	-127.99754983572\\
58.25	0.24306	-133.440865496305	-133.440865496305\\
58.25	0.24672	-139.006412848541	-139.006412848541\\
58.25	0.25038	-144.694191892426	-144.694191892426\\
58.25	0.25404	-150.504202627962	-150.504202627962\\
58.25	0.2577	-156.436445055147	-156.436445055147\\
58.25	0.26136	-162.490919173981	-162.490919173981\\
58.25	0.26502	-168.667624984466	-168.667624984466\\
58.25	0.26868	-174.9665624866	-174.9665624866\\
58.25	0.27234	-181.387731680384	-181.387731680384\\
58.25	0.276	-187.931132565818	-187.931132565818\\
58.625	0.093	-10.4601559197156	-10.4601559197156\\
58.625	0.09666	-11.0169827340916	-11.0169827340916\\
58.625	0.10032	-11.6960412401175	-11.6960412401175\\
58.625	0.10398	-12.4973314377931	-12.4973314377931\\
58.625	0.10764	-13.4208533271185	-13.4208533271185\\
58.625	0.1113	-14.4666069080937	-14.4666069080937\\
58.625	0.11496	-15.6345921807187	-15.6345921807187\\
58.625	0.11862	-16.9248091449935	-16.9248091449935\\
58.625	0.12228	-18.3372578009181	-18.3372578009181\\
58.625	0.12594	-19.8719381484925	-19.8719381484925\\
58.625	0.1296	-21.5288501877167	-21.5288501877167\\
58.625	0.13326	-23.3079939185906	-23.3079939185906\\
58.625	0.13692	-25.2093693411144	-25.2093693411144\\
58.625	0.14058	-27.232976455288	-27.232976455288\\
58.625	0.14424	-29.3788152611114	-29.3788152611114\\
58.625	0.1479	-31.6468857585845	-31.6468857585845\\
58.625	0.15156	-34.0371879477075	-34.0371879477075\\
58.625	0.15522	-36.5497218284802	-36.5497218284802\\
58.625	0.15888	-39.1844874009027	-39.1844874009027\\
58.625	0.16254	-41.9414846649751	-41.9414846649751\\
58.625	0.1662	-44.8207136206972	-44.8207136206972\\
58.625	0.16986	-47.8221742680691	-47.8221742680691\\
58.625	0.17352	-50.9458666070908	-50.9458666070908\\
58.625	0.17718	-54.1917906377623	-54.1917906377623\\
58.625	0.18084	-57.5599463600836	-57.5599463600836\\
58.625	0.1845	-61.0503337740547	-61.0503337740547\\
58.625	0.18816	-64.6629528796756	-64.6629528796756\\
58.625	0.19182	-68.3978036769463	-68.3978036769463\\
58.625	0.19548	-72.2548861658668	-72.2548861658668\\
58.625	0.19914	-76.234200346437	-76.234200346437\\
58.625	0.2028	-80.3357462186571	-80.3357462186571\\
58.625	0.20646	-84.559523782527	-84.559523782527\\
58.625	0.21012	-88.9055330380467	-88.9055330380467\\
58.625	0.21378	-93.3737739852161	-93.3737739852161\\
58.625	0.21744	-97.9642466240353	-97.9642466240353\\
58.625	0.2211	-102.676950954504	-102.676950954504\\
58.625	0.22476	-107.511886976623	-107.511886976623\\
58.625	0.22842	-112.469054690392	-112.469054690392\\
58.625	0.23208	-117.54845409581	-117.54845409581\\
58.625	0.23574	-122.750085192879	-122.750085192879\\
58.625	0.2394	-128.073947981596	-128.073947981596\\
58.625	0.24306	-133.520042461964	-133.520042461964\\
58.625	0.24672	-139.088368633982	-139.088368633982\\
58.625	0.25038	-144.778926497649	-144.778926497649\\
58.625	0.25404	-150.591716052967	-150.591716052967\\
58.625	0.2577	-156.526737299933	-156.526737299933\\
58.625	0.26136	-162.58399023855	-162.58399023855\\
58.625	0.26502	-168.763474868817	-168.763474868817\\
58.625	0.26868	-175.065191190733	-175.065191190733\\
58.625	0.27234	-181.489139204299	-181.489139204299\\
58.625	0.276	-188.035318909515	-188.035318909515\\
59	0.093	-10.4349976492296	-10.4349976492296\\
59	0.09666	-10.9946032833877	-10.9946032833877\\
59	0.10032	-11.6764406091955	-11.6764406091955\\
59	0.10398	-12.4805096266531	-12.4805096266531\\
59	0.10764	-13.4068103357606	-13.4068103357606\\
59	0.1113	-14.4553427365177	-14.4553427365177\\
59	0.11496	-15.6261068289248	-15.6261068289248\\
59	0.11862	-16.9191026129816	-16.9191026129816\\
59	0.12228	-18.3343300886882	-18.3343300886882\\
59	0.12594	-19.8717892560446	-19.8717892560446\\
59	0.1296	-21.5314801150508	-21.5314801150508\\
59	0.13326	-23.3134026657067	-23.3134026657067\\
59	0.13692	-25.2175569080125	-25.2175569080125\\
59	0.14058	-27.2439428419681	-27.2439428419681\\
59	0.14424	-29.3925604675735	-29.3925604675735\\
59	0.1479	-31.6634097848287	-31.6634097848287\\
59	0.15156	-34.0564907937336	-34.0564907937336\\
59	0.15522	-36.5718034942883	-36.5718034942883\\
59	0.15888	-39.2093478864929	-39.2093478864929\\
59	0.16254	-41.9691239703473	-41.9691239703473\\
59	0.1662	-44.8511317458514	-44.8511317458514\\
59	0.16986	-47.8553712130053	-47.8553712130053\\
59	0.17352	-50.981842371809	-50.981842371809\\
59	0.17718	-54.2305452222625	-54.2305452222625\\
59	0.18084	-57.6014797643658	-57.6014797643658\\
59	0.1845	-61.0946459981189	-61.0946459981189\\
59	0.18816	-64.7100439235219	-64.7100439235219\\
59	0.19182	-68.4476735405746	-68.4476735405746\\
59	0.19548	-72.307534849277	-72.307534849277\\
59	0.19914	-76.2896278496293	-76.2896278496293\\
59	0.2028	-80.3939525416314	-80.3939525416314\\
59	0.20646	-84.6205089252833	-84.6205089252833\\
59	0.21012	-88.969297000585	-88.969297000585\\
59	0.21378	-93.4403167675364	-93.4403167675364\\
59	0.21744	-98.0335682261377	-98.0335682261377\\
59	0.2211	-102.749051376389	-102.749051376389\\
59	0.22476	-107.58676621829	-107.58676621829\\
59	0.22842	-112.54671275184	-112.54671275184\\
59	0.23208	-117.628890977041	-117.628890977041\\
59	0.23574	-122.833300893891	-122.833300893891\\
59	0.2394	-128.159942502391	-128.159942502391\\
59	0.24306	-133.608815802541	-133.608815802541\\
59	0.24672	-139.17992079434	-139.17992079434\\
59	0.25038	-144.87325747779	-144.87325747779\\
59	0.25404	-150.688825852889	-150.688825852889\\
59	0.2577	-156.626625919638	-156.626625919638\\
59	0.26136	-162.686657678037	-162.686657678037\\
59	0.26502	-168.868921128085	-168.868921128085\\
59	0.26868	-175.173416269784	-175.173416269784\\
59	0.27234	-181.600143103132	-181.600143103132\\
59	0.276	-188.14910162813	-188.14910162813\\
59.375	0.093	-10.4194357536611	-10.4194357536611\\
59.375	0.09666	-10.9818202076011	-10.9818202076011\\
59.375	0.10032	-11.666436353191	-11.666436353191\\
59.375	0.10398	-12.4732841904306	-12.4732841904306\\
59.375	0.10764	-13.40236371932	-13.40236371932\\
59.375	0.1113	-14.4536749398593	-14.4536749398593\\
59.375	0.11496	-15.6272178520483	-15.6272178520483\\
59.375	0.11862	-16.9229924558871	-16.9229924558871\\
59.375	0.12228	-18.3409987513758	-18.3409987513758\\
59.375	0.12594	-19.8812367385142	-19.8812367385142\\
59.375	0.1296	-21.5437064173023	-21.5437064173023\\
59.375	0.13326	-23.3284077877404	-23.3284077877404\\
59.375	0.13692	-25.2353408498281	-25.2353408498281\\
59.375	0.14058	-27.2645056035657	-27.2645056035657\\
59.375	0.14424	-29.4159020489531	-29.4159020489531\\
59.375	0.1479	-31.6895301859903	-31.6895301859903\\
59.375	0.15156	-34.0853900146772	-34.0853900146772\\
59.375	0.15522	-36.603481535014	-36.603481535014\\
59.375	0.15888	-39.2438047470006	-39.2438047470006\\
59.375	0.16254	-42.0063596506369	-42.0063596506369\\
59.375	0.1662	-44.8911462459231	-44.8911462459231\\
59.375	0.16986	-47.898164532859	-47.898164532859\\
59.375	0.17352	-51.0274145114447	-51.0274145114447\\
59.375	0.17718	-54.2788961816802	-54.2788961816802\\
59.375	0.18084	-57.6526095435655	-57.6526095435655\\
59.375	0.1845	-61.1485545971007	-61.1485545971007\\
59.375	0.18816	-64.7667313422856	-64.7667313422856\\
59.375	0.19182	-68.5071397791203	-68.5071397791203\\
59.375	0.19548	-72.3697799076048	-72.3697799076048\\
59.375	0.19914	-76.3546517277391	-76.3546517277391\\
59.375	0.2028	-80.4617552395232	-80.4617552395232\\
59.375	0.20646	-84.6910904429571	-84.6910904429571\\
59.375	0.21012	-89.0426573380407	-89.0426573380407\\
59.375	0.21378	-93.5164559247742	-93.5164559247742\\
59.375	0.21744	-98.1124862031574	-98.1124862031574\\
59.375	0.2211	-102.830748173191	-102.830748173191\\
59.375	0.22476	-107.671241834873	-107.671241834873\\
59.375	0.22842	-112.633967188206	-112.633967188206\\
59.375	0.23208	-117.718924233188	-117.718924233188\\
59.375	0.23574	-122.926112969821	-122.926112969821\\
59.375	0.2394	-128.255533398103	-128.255533398103\\
59.375	0.24306	-133.707185518035	-133.707185518035\\
59.375	0.24672	-139.281069329616	-139.281069329616\\
59.375	0.25038	-144.977184832848	-144.977184832848\\
59.375	0.25404	-150.795532027729	-150.795532027729\\
59.375	0.2577	-156.73611091426	-156.73611091426\\
59.375	0.26136	-162.798921492441	-162.798921492441\\
59.375	0.26502	-168.983963762271	-168.983963762271\\
59.375	0.26868	-175.291237723752	-175.291237723752\\
59.375	0.27234	-181.720743376882	-181.720743376882\\
59.375	0.276	-188.272480721662	-188.272480721662\\
59.75	0.093	-10.4134702330102	-10.4134702330102\\
59.75	0.09666	-10.9786335067322	-10.9786335067322\\
59.75	0.10032	-11.6660284721041	-11.6660284721041\\
59.75	0.10398	-12.4756551291257	-12.4756551291257\\
59.75	0.10764	-13.4075134777971	-13.4075134777971\\
59.75	0.1113	-14.4616035181184	-14.4616035181184\\
59.75	0.11496	-15.6379252500894	-15.6379252500894\\
59.75	0.11862	-16.9364786737102	-16.9364786737102\\
59.75	0.12228	-18.3572637889809	-18.3572637889809\\
59.75	0.12594	-19.9002805959013	-19.9002805959013\\
59.75	0.1296	-21.5655290944715	-21.5655290944715\\
59.75	0.13326	-23.3530092846915	-23.3530092846915\\
59.75	0.13692	-25.2627211665612	-25.2627211665612\\
59.75	0.14058	-27.2946647400808	-27.2946647400808\\
59.75	0.14424	-29.4488400052502	-29.4488400052502\\
59.75	0.1479	-31.7252469620694	-31.7252469620694\\
59.75	0.15156	-34.1238856105384	-34.1238856105384\\
59.75	0.15522	-36.6447559506572	-36.6447559506572\\
59.75	0.15888	-39.2878579824258	-39.2878579824258\\
59.75	0.16254	-42.0531917058441	-42.0531917058441\\
59.75	0.1662	-44.9407571209123	-44.9407571209123\\
59.75	0.16986	-47.9505542276302	-47.9505542276302\\
59.75	0.17352	-51.0825830259979	-51.0825830259979\\
59.75	0.17718	-54.3368435160154	-54.3368435160154\\
59.75	0.18084	-57.7133356976828	-57.7133356976828\\
59.75	0.1845	-61.2120595709999	-61.2120595709999\\
59.75	0.18816	-64.8330151359668	-64.8330151359668\\
59.75	0.19182	-68.5762023925836	-68.5762023925836\\
59.75	0.19548	-72.44162134085	-72.44162134085\\
59.75	0.19914	-76.4292719807663	-76.4292719807663\\
59.75	0.2028	-80.5391543123325	-80.5391543123325\\
59.75	0.20646	-84.7712683355484	-84.7712683355484\\
59.75	0.21012	-89.1256140504141	-89.1256140504141\\
59.75	0.21378	-93.6021914569295	-93.6021914569295\\
59.75	0.21744	-98.2010005550948	-98.2010005550948\\
59.75	0.2211	-102.92204134491	-102.92204134491\\
59.75	0.22476	-107.765313826375	-107.765313826375\\
59.75	0.22842	-112.730817999489	-112.730817999489\\
59.75	0.23208	-117.818553864254	-117.818553864254\\
59.75	0.23574	-123.028521420668	-123.028521420668\\
59.75	0.2394	-128.360720668732	-128.360720668732\\
59.75	0.24306	-133.815151608446	-133.815151608446\\
59.75	0.24672	-139.39181423981	-139.39181423981\\
59.75	0.25038	-145.090708562823	-145.090708562823\\
59.75	0.25404	-150.911834577486	-150.911834577486\\
59.75	0.2577	-156.855192283799	-156.855192283799\\
59.75	0.26136	-162.920781681762	-162.920781681762\\
59.75	0.26502	-169.108602771375	-169.108602771375\\
59.75	0.26868	-175.418655552637	-175.418655552637\\
59.75	0.27234	-181.850940025549	-181.850940025549\\
59.75	0.276	-188.405456190111	-188.405456190111\\
60.125	0.093	-10.4171010872766	-10.4171010872766\\
60.125	0.09666	-10.9850431807806	-10.9850431807806\\
60.125	0.10032	-11.6752169659345	-11.6752169659345\\
60.125	0.10398	-12.4876224427381	-12.4876224427381\\
60.125	0.10764	-13.4222596111916	-13.4222596111916\\
60.125	0.1113	-14.4791284712948	-14.4791284712948\\
60.125	0.11496	-15.6582290230479	-15.6582290230479\\
60.125	0.11862	-16.9595612664507	-16.9595612664507\\
60.125	0.12228	-18.3831252015034	-18.3831252015034\\
60.125	0.12594	-19.9289208282058	-19.9289208282058\\
60.125	0.1296	-21.596948146558	-21.596948146558\\
60.125	0.13326	-23.38720715656	-23.38720715656\\
60.125	0.13692	-25.2996978582118	-25.2996978582118\\
60.125	0.14058	-27.3344202515134	-27.3344202515134\\
60.125	0.14424	-29.4913743364648	-29.4913743364648\\
60.125	0.1479	-31.770560113066	-31.770560113066\\
60.125	0.15156	-34.171977581317	-34.171977581317\\
60.125	0.15522	-36.6956267412178	-36.6956267412178\\
60.125	0.15888	-39.3415075927683	-39.3415075927683\\
60.125	0.16254	-42.1096201359687	-42.1096201359687\\
60.125	0.1662	-44.9999643708188	-44.9999643708188\\
60.125	0.16986	-48.0125402973188	-48.0125402973188\\
60.125	0.17352	-51.1473479154686	-51.1473479154686\\
60.125	0.17718	-54.4043872252681	-54.4043872252681\\
60.125	0.18084	-57.7836582267174	-57.7836582267174\\
60.125	0.1845	-61.2851609198165	-61.2851609198165\\
60.125	0.18816	-64.9088953045655	-64.9088953045655\\
60.125	0.19182	-68.6548613809642	-68.6548613809642\\
60.125	0.19548	-72.5230591490127	-72.5230591490127\\
60.125	0.19914	-76.513488608711	-76.513488608711\\
60.125	0.2028	-80.6261497600591	-80.6261497600591\\
60.125	0.20646	-84.8610426030571	-84.8610426030571\\
60.125	0.21012	-89.2181671377048	-89.2181671377048\\
60.125	0.21378	-93.6975233640023	-93.6975233640023\\
60.125	0.21744	-98.2991112819495	-98.2991112819495\\
60.125	0.2211	-103.022930891547	-103.022930891547\\
60.125	0.22476	-107.868982192793	-107.868982192793\\
60.125	0.22842	-112.83726518569	-112.83726518569\\
60.125	0.23208	-117.927779870237	-117.927779870237\\
60.125	0.23574	-123.140526246433	-123.140526246433\\
60.125	0.2394	-128.475504314279	-128.475504314279\\
60.125	0.24306	-133.932714073775	-133.932714073775\\
60.125	0.24672	-139.51215552492	-139.51215552492\\
60.125	0.25038	-145.213828667716	-145.213828667716\\
60.125	0.25404	-151.037733502161	-151.037733502161\\
60.125	0.2577	-156.983870028256	-156.983870028256\\
60.125	0.26136	-163.052238246001	-163.052238246001\\
60.125	0.26502	-169.242838155395	-169.242838155395\\
60.125	0.26868	-175.55566975644	-175.55566975644\\
60.125	0.27234	-181.990733049134	-181.990733049134\\
60.125	0.276	-188.548028033478	-188.548028033478\\
60.5	0.093	-10.4303283164606	-10.4303283164606\\
60.5	0.09666	-11.0010492297466	-11.0010492297466\\
60.5	0.10032	-11.6940018346825	-11.6940018346825\\
60.5	0.10398	-12.5091861312682	-12.5091861312682\\
60.5	0.10764	-13.4466021195036	-13.4466021195036\\
60.5	0.1113	-14.5062497993889	-14.5062497993889\\
60.5	0.11496	-15.6881291709239	-15.6881291709239\\
60.5	0.11862	-16.9922402341087	-16.9922402341087\\
60.5	0.12228	-18.4185829889434	-18.4185829889434\\
60.5	0.12594	-19.9671574354278	-19.9671574354278\\
60.5	0.1296	-21.637963573562	-21.637963573562\\
60.5	0.13326	-23.4310014033461	-23.4310014033461\\
60.5	0.13692	-25.3462709247799	-25.3462709247799\\
60.5	0.14058	-27.3837721378635	-27.3837721378635\\
60.5	0.14424	-29.5435050425969	-29.5435050425969\\
60.5	0.1479	-31.8254696389801	-31.8254696389801\\
60.5	0.15156	-34.2296659270131	-34.2296659270131\\
60.5	0.15522	-36.7560939066959	-36.7560939066959\\
60.5	0.15888	-39.4047535780285	-39.4047535780285\\
60.5	0.16254	-42.1756449410109	-42.1756449410109\\
60.5	0.1662	-45.068767995643	-45.068767995643\\
60.5	0.16986	-48.084122741925	-48.084122741925\\
60.5	0.17352	-51.2217091798567	-51.2217091798567\\
60.5	0.17718	-54.4815273094383	-54.4815273094383\\
60.5	0.18084	-57.8635771306696	-57.8635771306696\\
60.5	0.1845	-61.3678586435508	-61.3678586435508\\
60.5	0.18816	-64.9943718480817	-64.9943718480817\\
60.5	0.19182	-68.7431167442625	-68.7431167442625\\
60.5	0.19548	-72.6140933320929	-72.6140933320929\\
60.5	0.19914	-76.6073016115733	-76.6073016115733\\
60.5	0.2028	-80.7227415827034	-80.7227415827034\\
60.5	0.20646	-84.9604132454834	-84.9604132454834\\
60.5	0.21012	-89.320316599913	-89.320316599913\\
60.5	0.21378	-93.8024516459925	-93.8024516459925\\
60.5	0.21744	-98.4068183837218	-98.4068183837218\\
60.5	0.2211	-103.133416813101	-103.133416813101\\
60.5	0.22476	-107.98224693413	-107.98224693413\\
60.5	0.22842	-112.953308746808	-112.953308746808\\
60.5	0.23208	-118.046602251137	-118.046602251137\\
60.5	0.23574	-123.262127447115	-123.262127447115\\
60.5	0.2394	-128.599884334743	-128.599884334743\\
60.5	0.24306	-134.059872914021	-134.059872914021\\
60.5	0.24672	-139.642093184949	-139.642093184949\\
60.5	0.25038	-145.346545147526	-145.346545147526\\
60.5	0.25404	-151.173228801753	-151.173228801753\\
60.5	0.2577	-157.12214414763	-157.12214414763\\
60.5	0.26136	-163.193291185157	-163.193291185157\\
60.5	0.26502	-169.386669914334	-169.386669914334\\
60.5	0.26868	-175.70228033516	-175.70228033516\\
60.5	0.27234	-182.140122447636	-182.140122447636\\
60.5	0.276	-188.700196251762	-188.700196251762\\
60.875	0.093	-10.453151920562	-10.453151920562\\
60.875	0.09666	-11.02665165363	-11.02665165363\\
60.875	0.10032	-11.722383078348	-11.722383078348\\
60.875	0.10398	-12.5403461947156	-12.5403461947156\\
60.875	0.10764	-13.4805410027331	-13.4805410027331\\
60.875	0.1113	-14.5429675024004	-14.5429675024004\\
60.875	0.11496	-15.7276256937174	-15.7276256937174\\
60.875	0.11862	-17.0345155766842	-17.0345155766842\\
60.875	0.12228	-18.4636371513009	-18.4636371513009\\
60.875	0.12594	-20.0149904175673	-20.0149904175673\\
60.875	0.1296	-21.6885753754836	-21.6885753754836\\
60.875	0.13326	-23.4843920250496	-23.4843920250496\\
60.875	0.13692	-25.4024403662654	-25.4024403662654\\
60.875	0.14058	-27.4427203991311	-27.4427203991311\\
60.875	0.14424	-29.6052321236465	-29.6052321236465\\
60.875	0.1479	-31.8899755398117	-31.8899755398117\\
60.875	0.15156	-34.2969506476267	-34.2969506476267\\
60.875	0.15522	-36.8261574470915	-36.8261574470915\\
60.875	0.15888	-39.4775959382061	-39.4775959382061\\
60.875	0.16254	-42.2512661209704	-42.2512661209704\\
60.875	0.1662	-45.1471679953846	-45.1471679953846\\
60.875	0.16986	-48.1653015614486	-48.1653015614486\\
60.875	0.17352	-51.3056668191624	-51.3056668191624\\
60.875	0.17718	-54.5682637685259	-54.5682637685259\\
60.875	0.18084	-57.9530924095392	-57.9530924095392\\
60.875	0.1845	-61.4601527422024	-61.4601527422024\\
60.875	0.18816	-65.0894447665154	-65.0894447665154\\
60.875	0.19182	-68.8409684824781	-68.8409684824781\\
60.875	0.19548	-72.7147238900906	-72.7147238900906\\
60.875	0.19914	-76.710710989353	-76.710710989353\\
60.875	0.2028	-80.8289297802651	-80.8289297802651\\
60.875	0.20646	-85.0693802628271	-85.0693802628271\\
60.875	0.21012	-89.4320624370387	-89.4320624370387\\
60.875	0.21378	-93.9169763029002	-93.9169763029002\\
60.875	0.21744	-98.5241218604115	-98.5241218604115\\
60.875	0.2211	-103.253499109573	-103.253499109573\\
60.875	0.22476	-108.105108050384	-108.105108050384\\
60.875	0.22842	-113.078948682844	-113.078948682844\\
60.875	0.23208	-118.175021006955	-118.175021006955\\
60.875	0.23574	-123.393325022715	-123.393325022715\\
60.875	0.2394	-128.733860730125	-128.733860730125\\
60.875	0.24306	-134.196628129185	-134.196628129185\\
60.875	0.24672	-139.781627219895	-139.781627219895\\
60.875	0.25038	-145.488858002254	-145.488858002254\\
60.875	0.25404	-151.318320476263	-151.318320476263\\
60.875	0.2577	-157.270014641922	-157.270014641922\\
60.875	0.26136	-163.343940499231	-163.343940499231\\
60.875	0.26502	-169.54009804819	-169.54009804819\\
60.875	0.26868	-175.858487288798	-175.858487288798\\
60.875	0.27234	-182.299108221056	-182.299108221056\\
60.875	0.276	-188.861960844964	-188.861960844964\\
61.25	0.093	-10.485571899581	-10.485571899581\\
61.25	0.09666	-11.061850452431	-11.061850452431\\
61.25	0.10032	-11.7603606969309	-11.7603606969309\\
61.25	0.10398	-12.5811026330806	-12.5811026330806\\
61.25	0.10764	-13.5240762608801	-13.5240762608801\\
61.25	0.1113	-14.5892815803294	-14.5892815803294\\
61.25	0.11496	-15.7767185914284	-15.7767185914284\\
61.25	0.11862	-17.0863872941773	-17.0863872941773\\
61.25	0.12228	-18.518287688576	-18.518287688576\\
61.25	0.12594	-20.0724197746244	-20.0724197746244\\
61.25	0.1296	-21.7487835523226	-21.7487835523226\\
61.25	0.13326	-23.5473790216707	-23.5473790216707\\
61.25	0.13692	-25.4682061826685	-25.4682061826685\\
61.25	0.14058	-27.5112650353161	-27.5112650353161\\
61.25	0.14424	-29.6765555796136	-29.6765555796136\\
61.25	0.1479	-31.9640778155608	-31.9640778155608\\
61.25	0.15156	-34.3738317431578	-34.3738317431578\\
61.25	0.15522	-36.9058173624046	-36.9058173624046\\
61.25	0.15888	-39.5600346733012	-39.5600346733012\\
61.25	0.16254	-42.3364836758476	-42.3364836758476\\
61.25	0.1662	-45.2351643700438	-45.2351643700438\\
61.25	0.16986	-48.2560767558898	-48.2560767558898\\
61.25	0.17352	-51.3992208333855	-51.3992208333855\\
61.25	0.17718	-54.6645966025311	-54.6645966025311\\
61.25	0.18084	-58.0522040633264	-58.0522040633264\\
61.25	0.1845	-61.5620432157716	-61.5620432157716\\
61.25	0.18816	-65.1941140598666	-65.1941140598666\\
61.25	0.19182	-68.9484165956113	-68.9484165956113\\
61.25	0.19548	-72.8249508230058	-72.8249508230058\\
61.25	0.19914	-76.8237167420502	-76.8237167420502\\
61.25	0.2028	-80.9447143527443	-80.9447143527443\\
61.25	0.20646	-85.1879436550883	-85.1879436550883\\
61.25	0.21012	-89.553404649082	-89.553404649082\\
61.25	0.21378	-94.0410973347255	-94.0410973347255\\
61.25	0.21744	-98.6510217120188	-98.6510217120188\\
61.25	0.2211	-103.383177780962	-103.383177780962\\
61.25	0.22476	-108.237565541555	-108.237565541555\\
61.25	0.22842	-113.214184993797	-113.214184993797\\
61.25	0.23208	-118.31303613769	-118.31303613769\\
61.25	0.23574	-123.534118973232	-123.534118973232\\
61.25	0.2394	-128.877433500424	-128.877433500424\\
61.25	0.24306	-134.342979719266	-134.342979719266\\
61.25	0.24672	-139.930757629758	-139.930757629758\\
61.25	0.25038	-145.640767231899	-145.640767231899\\
61.25	0.25404	-151.473008525691	-151.473008525691\\
61.25	0.2577	-157.427481511132	-157.427481511132\\
61.25	0.26136	-163.504186188222	-163.504186188222\\
61.25	0.26502	-169.703122556963	-169.703122556963\\
61.25	0.26868	-176.024290617354	-176.024290617354\\
61.25	0.27234	-182.467690369394	-182.467690369394\\
61.25	0.276	-189.033321813084	-189.033321813084\\
61.625	0.093	-10.5275882535174	-10.5275882535174\\
61.625	0.09666	-11.1066456261495	-11.1066456261495\\
61.625	0.10032	-11.8079346904314	-11.8079346904314\\
61.625	0.10398	-12.6314554463631	-12.6314554463631\\
61.625	0.10764	-13.5772078939446	-13.5772078939446\\
61.625	0.1113	-14.6451920331759	-14.6451920331759\\
61.625	0.11496	-15.8354078640569	-15.8354078640569\\
61.625	0.11862	-17.1478553865878	-17.1478553865878\\
61.625	0.12228	-18.5825346007685	-18.5825346007685\\
61.625	0.12594	-20.1394455065989	-20.1394455065989\\
61.625	0.1296	-21.8185881040792	-21.8185881040792\\
61.625	0.13326	-23.6199623932092	-23.6199623932092\\
61.625	0.13692	-25.543568373989	-25.543568373989\\
61.625	0.14058	-27.5894060464187	-27.5894060464187\\
61.625	0.14424	-29.7574754104982	-29.7574754104982\\
61.625	0.1479	-32.0477764662273	-32.0477764662273\\
61.625	0.15156	-34.4603092136064	-34.4603092136064\\
61.625	0.15522	-36.9950736526352	-36.9950736526352\\
61.625	0.15888	-39.6520697833138	-39.6520697833138\\
61.625	0.16254	-42.4312976056422	-42.4312976056422\\
61.625	0.1662	-45.3327571196204	-45.3327571196204\\
61.625	0.16986	-48.3564483252484	-48.3564483252484\\
61.625	0.17352	-51.5023712225262	-51.5023712225262\\
61.625	0.17718	-54.7705258114537	-54.7705258114537\\
61.625	0.18084	-58.1609120920311	-58.1609120920311\\
61.625	0.1845	-61.6735300642583	-61.6735300642583\\
61.625	0.18816	-65.3083797281353	-65.3083797281353\\
61.625	0.19182	-69.065461083662	-69.065461083662\\
61.625	0.19548	-72.9447741308386	-72.9447741308386\\
61.625	0.19914	-76.9463188696649	-76.9463188696649\\
61.625	0.2028	-81.0700953001411	-81.0700953001411\\
61.625	0.20646	-85.316103422267	-85.316103422267\\
61.625	0.21012	-89.6843432360427	-89.6843432360427\\
61.625	0.21378	-94.1748147414682	-94.1748147414682\\
61.625	0.21744	-98.7875179385435	-98.7875179385435\\
61.625	0.2211	-103.522452827269	-103.522452827269\\
61.625	0.22476	-108.379619407644	-108.379619407644\\
61.625	0.22842	-113.359017679668	-113.359017679668\\
61.625	0.23208	-118.460647643343	-118.460647643343\\
61.625	0.23574	-123.684509298667	-123.684509298667\\
61.625	0.2394	-129.030602645641	-129.030602645641\\
61.625	0.24306	-134.498927684265	-134.498927684265\\
61.625	0.24672	-140.089484414539	-140.089484414539\\
61.625	0.25038	-145.802272836462	-145.802272836462\\
61.625	0.25404	-151.637292950035	-151.637292950035\\
61.625	0.2577	-157.594544755259	-157.594544755259\\
61.625	0.26136	-163.674028252131	-163.674028252131\\
61.625	0.26502	-169.875743440654	-169.875743440654\\
61.625	0.26868	-176.199690320826	-176.199690320826\\
61.625	0.27234	-182.645868892649	-182.645868892649\\
61.625	0.276	-189.214279156121	-189.214279156121\\
62	0.093	-10.5792009823713	-10.5792009823713\\
62	0.09666	-11.1610371747854	-11.1610371747854\\
62	0.10032	-11.8651050588493	-11.8651050588493\\
62	0.10398	-12.691404634563	-12.691404634563\\
62	0.10764	-13.6399359019265	-13.6399359019265\\
62	0.1113	-14.7106988609398	-14.7106988609398\\
62	0.11496	-15.9036935116028	-15.9036935116028\\
62	0.11862	-17.2189198539157	-17.2189198539157\\
62	0.12228	-18.6563778878784	-18.6563778878784\\
62	0.12594	-20.2160676134909	-20.2160676134909\\
62	0.1296	-21.8979890307531	-21.8979890307531\\
62	0.13326	-23.7021421396652	-23.7021421396652\\
62	0.13692	-25.628526940227	-25.628526940227\\
62	0.14058	-27.6771434324387	-27.6771434324387\\
62	0.14424	-29.8479916163001	-29.8479916163001\\
62	0.1479	-32.1410714918114	-32.1410714918114\\
62	0.15156	-34.5563830589724	-34.5563830589724\\
62	0.15522	-37.0939263177832	-37.0939263177832\\
62	0.15888	-39.7537012682439	-39.7537012682439\\
62	0.16254	-42.5357079103543	-42.5357079103543\\
62	0.1662	-45.4399462441145	-45.4399462441145\\
62	0.16986	-48.4664162695245	-48.4664162695245\\
62	0.17352	-51.6151179865843	-51.6151179865843\\
62	0.17718	-54.8860513952938	-54.8860513952938\\
62	0.18084	-58.2792164956532	-58.2792164956532\\
62	0.1845	-61.7946132876624	-61.7946132876624\\
62	0.18816	-65.4322417713214	-65.4322417713214\\
62	0.19182	-69.1921019466301	-69.1921019466301\\
62	0.19548	-73.0741938135887	-73.0741938135887\\
62	0.19914	-77.078517372197	-77.078517372197\\
62	0.2028	-81.2050726224552	-81.2050726224552\\
62	0.20646	-85.4538595643632	-85.4538595643632\\
62	0.21012	-89.8248781979209	-89.8248781979209\\
62	0.21378	-94.3181285231284	-94.3181285231284\\
62	0.21744	-98.9336105399857	-98.9336105399857\\
62	0.2211	-103.671324248493	-103.671324248493\\
62	0.22476	-108.53126964865	-108.53126964865\\
62	0.22842	-113.513446740456	-113.513446740456\\
62	0.23208	-118.617855523913	-118.617855523913\\
62	0.23574	-123.844495999019	-123.844495999019\\
62	0.2394	-129.193368165775	-129.193368165775\\
62	0.24306	-134.664472024181	-134.664472024181\\
62	0.24672	-140.257807574237	-140.257807574237\\
62	0.25038	-145.973374815942	-145.973374815942\\
62	0.25404	-151.811173749298	-151.811173749298\\
62	0.2577	-157.771204374303	-157.771204374303\\
62	0.26136	-163.853466690958	-163.853466690958\\
62	0.26502	-170.057960699262	-170.057960699262\\
62	0.26868	-176.384686399217	-176.384686399217\\
62	0.27234	-182.833643790821	-182.833643790821\\
62	0.276	-189.404832874075	-189.404832874075\\
62.375	0.093	-10.6404100861427	-10.6404100861427\\
62.375	0.09666	-11.2250250983388	-11.2250250983388\\
62.375	0.10032	-11.9318718021848	-11.9318718021848\\
62.375	0.10398	-12.7609501976805	-12.7609501976805\\
62.375	0.10764	-13.712260284826	-13.712260284826\\
62.375	0.1113	-14.7858020636213	-14.7858020636213\\
62.375	0.11496	-15.9815755340664	-15.9815755340664\\
62.375	0.11862	-17.2995806961613	-17.2995806961613\\
62.375	0.12228	-18.739817549906	-18.739817549906\\
62.375	0.12594	-20.3022860953004	-20.3022860953004\\
62.375	0.1296	-21.9869863323447	-21.9869863323447\\
62.375	0.13326	-23.7939182610387	-23.7939182610387\\
62.375	0.13692	-25.7230818813826	-25.7230818813826\\
62.375	0.14058	-27.7744771933763	-27.7744771933763\\
62.375	0.14424	-29.9481041970197	-29.9481041970197\\
62.375	0.1479	-32.243962892313	-32.243962892313\\
62.375	0.15156	-34.662053279256	-34.662053279256\\
62.375	0.15522	-37.2023753578488	-37.2023753578488\\
62.375	0.15888	-39.8649291280915	-39.8649291280915\\
62.375	0.16254	-42.6497145899839	-42.6497145899839\\
62.375	0.1662	-45.5567317435261	-45.5567317435261\\
62.375	0.16986	-48.5859805887181	-48.5859805887181\\
62.375	0.17352	-51.7374611255599	-51.7374611255599\\
62.375	0.17718	-55.0111733540515	-55.0111733540515\\
62.375	0.18084	-58.4071172741929	-58.4071172741929\\
62.375	0.1845	-61.925292885984	-61.925292885984\\
62.375	0.18816	-65.5657001894251	-65.5657001894251\\
62.375	0.19182	-69.3283391845158	-69.3283391845158\\
62.375	0.19548	-73.2132098712564	-73.2132098712564\\
62.375	0.19914	-77.2203122496468	-77.2203122496468\\
62.375	0.2028	-81.3496463196869	-81.3496463196869\\
62.375	0.20646	-85.6012120813769	-85.6012120813769\\
62.375	0.21012	-89.9750095347167	-89.9750095347167\\
62.375	0.21378	-94.4710386797062	-94.4710386797062\\
62.375	0.21744	-99.0892995163455	-99.0892995163455\\
62.375	0.2211	-103.829792044635	-103.829792044635\\
62.375	0.22476	-108.692516264574	-108.692516264574\\
62.375	0.22842	-113.677472176162	-113.677472176162\\
62.375	0.23208	-118.784659779401	-118.784659779401\\
62.375	0.23574	-124.014079074289	-124.014079074289\\
62.375	0.2394	-129.365730060827	-129.365730060827\\
62.375	0.24306	-134.839612739015	-134.839612739015\\
62.375	0.24672	-140.435727108853	-140.435727108853\\
62.375	0.25038	-146.15407317034	-146.15407317034\\
62.375	0.25404	-151.994650923478	-151.994650923478\\
62.375	0.2577	-157.957460368265	-157.957460368265\\
62.375	0.26136	-164.042501504701	-164.042501504701\\
62.375	0.26502	-170.249774332788	-170.249774332788\\
62.375	0.26868	-176.579278852525	-176.579278852525\\
62.375	0.27234	-183.031015063911	-183.031015063911\\
62.375	0.276	-189.604982966947	-189.604982966947\\
62.75	0.093	-10.7112155648316	-10.7112155648316\\
62.75	0.09666	-11.2986093968097	-11.2986093968097\\
62.75	0.10032	-12.0082349204377	-12.0082349204377\\
62.75	0.10398	-12.8400921357154	-12.8400921357154\\
62.75	0.10764	-13.7941810426428	-13.7941810426428\\
62.75	0.1113	-14.8705016412202	-14.8705016412202\\
62.75	0.11496	-16.0690539314473	-16.0690539314473\\
62.75	0.11862	-17.3898379133242	-17.3898379133242\\
62.75	0.12228	-18.8328535868509	-18.8328535868509\\
62.75	0.12594	-20.3981009520273	-20.3981009520273\\
62.75	0.1296	-22.0855800088536	-22.0855800088536\\
62.75	0.13326	-23.8952907573297	-23.8952907573297\\
62.75	0.13692	-25.8272331974555	-25.8272331974555\\
62.75	0.14058	-27.8814073292312	-27.8814073292312\\
62.75	0.14424	-30.0578131526567	-30.0578131526567\\
62.75	0.1479	-32.3564506677319	-32.3564506677319\\
62.75	0.15156	-34.777319874457	-34.777319874457\\
62.75	0.15522	-37.3204207728318	-37.3204207728318\\
62.75	0.15888	-39.9857533628565	-39.9857533628565\\
62.75	0.16254	-42.7733176445309	-42.7733176445309\\
62.75	0.1662	-45.6831136178551	-45.6831136178551\\
62.75	0.16986	-48.7151412828292	-48.7151412828292\\
62.75	0.17352	-51.8694006394529	-51.8694006394529\\
62.75	0.17718	-55.1458916877265	-55.1458916877265\\
62.75	0.18084	-58.5446144276499	-58.5446144276499\\
62.75	0.1845	-62.0655688592231	-62.0655688592231\\
62.75	0.18816	-65.7087549824462	-65.7087549824462\\
62.75	0.19182	-69.474172797319	-69.474172797319\\
62.75	0.19548	-73.3618223038415	-73.3618223038415\\
62.75	0.19914	-77.3717035020139	-77.3717035020139\\
62.75	0.2028	-81.5038163918361	-81.5038163918361\\
62.75	0.20646	-85.758160973308	-85.758160973308\\
62.75	0.21012	-90.1347372464298	-90.1347372464298\\
62.75	0.21378	-94.6335452112014	-94.6335452112014\\
62.75	0.21744	-99.2545848676226	-99.2545848676226\\
62.75	0.2211	-103.997856215694	-103.997856215694\\
62.75	0.22476	-108.863359255415	-108.863359255415\\
62.75	0.22842	-113.851093986785	-113.851093986785\\
62.75	0.23208	-118.961060409806	-118.961060409806\\
62.75	0.23574	-124.193258524476	-124.193258524476\\
62.75	0.2394	-129.547688330796	-129.547688330796\\
62.75	0.24306	-135.024349828766	-135.024349828766\\
62.75	0.24672	-140.623243018386	-140.623243018386\\
62.75	0.25038	-146.344367899656	-146.344367899656\\
62.75	0.25404	-152.187724472575	-152.187724472575\\
62.75	0.2577	-158.153312737144	-158.153312737144\\
62.75	0.26136	-164.241132693363	-164.241132693363\\
62.75	0.26502	-170.451184341231	-170.451184341231\\
62.75	0.26868	-176.78346768075	-176.78346768075\\
62.75	0.27234	-183.237982711918	-183.237982711918\\
62.75	0.276	-189.814729434736	-189.814729434736\\
63.125	0.093	-10.791617418438	-10.791617418438\\
63.125	0.09666	-11.3817900701981	-11.3817900701981\\
63.125	0.10032	-12.0941944136081	-12.0941944136081\\
63.125	0.10398	-12.9288304486678	-12.9288304486678\\
63.125	0.10764	-13.8856981753773	-13.8856981753773\\
63.125	0.1113	-14.9647975937367	-14.9647975937367\\
63.125	0.11496	-16.1661287037457	-16.1661287037457\\
63.125	0.11862	-17.4896915054046	-17.4896915054046\\
63.125	0.12228	-18.9354859987134	-18.9354859987134\\
63.125	0.12594	-20.5035121836718	-20.5035121836718\\
63.125	0.1296	-22.1937700602801	-22.1937700602801\\
63.125	0.13326	-24.0062596285382	-24.0062596285382\\
63.125	0.13692	-25.9409808884461	-25.9409808884461\\
63.125	0.14058	-27.9979338400038	-27.9979338400038\\
63.125	0.14424	-30.1771184832112	-30.1771184832112\\
63.125	0.1479	-32.4785348180685	-32.4785348180685\\
63.125	0.15156	-34.9021828445755	-34.9021828445755\\
63.125	0.15522	-37.4480625627324	-37.4480625627324\\
63.125	0.15888	-40.116173972539	-40.116173972539\\
63.125	0.16254	-42.9065170739955	-42.9065170739955\\
63.125	0.1662	-45.8190918671017	-45.8190918671017\\
63.125	0.16986	-48.8538983518577	-48.8538983518577\\
63.125	0.17352	-52.0109365282636	-52.0109365282636\\
63.125	0.17718	-55.2902063963191	-55.2902063963191\\
63.125	0.18084	-58.6917079560245	-58.6917079560245\\
63.125	0.1845	-62.2154412073798	-62.2154412073798\\
63.125	0.18816	-65.8614061503848	-65.8614061503848\\
63.125	0.19182	-69.6296027850395	-69.6296027850395\\
63.125	0.19548	-73.5200311113441	-73.5200311113441\\
63.125	0.19914	-77.5326911292985	-77.5326911292985\\
63.125	0.2028	-81.6675828389027	-81.6675828389027\\
63.125	0.20646	-85.9247062401567	-85.9247062401567\\
63.125	0.21012	-90.3040613330604	-90.3040613330604\\
63.125	0.21378	-94.805648117614	-94.805648117614\\
63.125	0.21744	-99.4294665938174	-99.4294665938174\\
63.125	0.2211	-104.175516761671	-104.175516761671\\
63.125	0.22476	-109.043798621173	-109.043798621173\\
63.125	0.22842	-114.034312172326	-114.034312172326\\
63.125	0.23208	-119.147057415129	-119.147057415129\\
63.125	0.23574	-124.382034349581	-124.382034349581\\
63.125	0.2394	-129.739242975683	-129.739242975683\\
63.125	0.24306	-135.218683293435	-135.218683293435\\
63.125	0.24672	-140.820355302837	-140.820355302837\\
63.125	0.25038	-146.544259003888	-146.544259003888\\
63.125	0.25404	-152.39039439659	-152.39039439659\\
63.125	0.2577	-158.358761480941	-158.358761480941\\
63.125	0.26136	-164.449360256942	-164.449360256942\\
63.125	0.26502	-170.662190724592	-170.662190724592\\
63.125	0.26868	-176.997252883893	-176.997252883893\\
63.125	0.27234	-183.454546734843	-183.454546734843\\
63.125	0.276	-190.034072277443	-190.034072277443\\
63.5	0.093	-10.8816156469619	-10.8816156469619\\
63.5	0.09666	-11.4745671185039	-11.4745671185039\\
63.5	0.10032	-12.1897502816959	-12.1897502816959\\
63.5	0.10398	-13.0271651365377	-13.0271651365377\\
63.5	0.10764	-13.9868116830292	-13.9868116830292\\
63.5	0.1113	-15.0686899211705	-15.0686899211705\\
63.5	0.11496	-16.2727998509616	-16.2727998509616\\
63.5	0.11862	-17.5991414724026	-17.5991414724026\\
63.5	0.12228	-19.0477147854933	-19.0477147854933\\
63.5	0.12594	-20.6185197902338	-20.6185197902338\\
63.5	0.1296	-22.3115564866241	-22.3115564866241\\
63.5	0.13326	-24.1268248746641	-24.1268248746641\\
63.5	0.13692	-26.064324954354	-26.064324954354\\
63.5	0.14058	-28.1240567256937	-28.1240567256937\\
63.5	0.14424	-30.3060201886832	-30.3060201886832\\
63.5	0.1479	-32.6102153433225	-32.6102153433225\\
63.5	0.15156	-35.0366421896115	-35.0366421896115\\
63.5	0.15522	-37.5853007275504	-37.5853007275504\\
63.5	0.15888	-40.2561909571391	-40.2561909571391\\
63.5	0.16254	-43.0493128783775	-43.0493128783775\\
63.5	0.1662	-45.9646664912657	-45.9646664912657\\
63.5	0.16986	-49.0022517958038	-49.0022517958038\\
63.5	0.17352	-52.1620687919916	-52.1620687919916\\
63.5	0.17718	-55.4441174798292	-55.4441174798292\\
63.5	0.18084	-58.8483978593166	-58.8483978593166\\
63.5	0.1845	-62.3749099304538	-62.3749099304538\\
63.5	0.18816	-66.0236536932408	-66.0236536932408\\
63.5	0.19182	-69.7946291476776	-69.7946291476776\\
63.5	0.19548	-73.6878362937642	-73.6878362937642\\
63.5	0.19914	-77.7032751315006	-77.7032751315006\\
63.5	0.2028	-81.8409456608868	-81.8409456608868\\
63.5	0.20646	-86.1008478819228	-86.1008478819228\\
63.5	0.21012	-90.4829817946085	-90.4829817946085\\
63.5	0.21378	-94.9873473989441	-94.9873473989441\\
63.5	0.21744	-99.6139446949295	-99.6139446949295\\
63.5	0.2211	-104.362773682565	-104.362773682565\\
63.5	0.22476	-109.23383436185	-109.23383436185\\
63.5	0.22842	-114.227126732784	-114.227126732784\\
63.5	0.23208	-119.342650795369	-119.342650795369\\
63.5	0.23574	-124.580406549603	-124.580406549603\\
63.5	0.2394	-129.940393995487	-129.940393995487\\
63.5	0.24306	-135.422613133021	-135.422613133021\\
63.5	0.24672	-141.027063962205	-141.027063962205\\
63.5	0.25038	-146.753746483039	-146.753746483039\\
63.5	0.25404	-152.602660695522	-152.602660695522\\
63.5	0.2577	-158.573806599655	-158.573806599655\\
63.5	0.26136	-164.667184195438	-164.667184195438\\
63.5	0.26502	-170.88279348287	-170.88279348287\\
63.5	0.26868	-177.220634461953	-177.220634461953\\
63.5	0.27234	-183.680707132685	-183.680707132685\\
63.5	0.276	-190.263011495067	-190.263011495067\\
63.875	0.093	-10.9812102504032	-10.9812102504032\\
63.875	0.09666	-11.5769405417274	-11.5769405417274\\
63.875	0.10032	-12.2949025247013	-12.2949025247013\\
63.875	0.10398	-13.1350961993251	-13.1350961993251\\
63.875	0.10764	-14.0975215655986	-14.0975215655986\\
63.875	0.1113	-15.182178623522	-15.182178623522\\
63.875	0.11496	-16.3890673730951	-16.3890673730951\\
63.875	0.11862	-17.718187814318	-17.718187814318\\
63.875	0.12228	-19.1695399471907	-19.1695399471907\\
63.875	0.12594	-20.7431237717133	-20.7431237717133\\
63.875	0.1296	-22.4389392878855	-22.4389392878855\\
63.875	0.13326	-24.2569864957076	-24.2569864957076\\
63.875	0.13692	-26.1972653951795	-26.1972653951795\\
63.875	0.14058	-28.2597759863012	-28.2597759863012\\
63.875	0.14424	-30.4445182690727	-30.4445182690727\\
63.875	0.1479	-32.751492243494	-32.751492243494\\
63.875	0.15156	-35.1806979095651	-35.1806979095651\\
63.875	0.15522	-37.7321352672859	-37.7321352672859\\
63.875	0.15888	-40.4058043166566	-40.4058043166566\\
63.875	0.16254	-43.201705057677	-43.201705057677\\
63.875	0.1662	-46.1198374903473	-46.1198374903473\\
63.875	0.16986	-49.1602016146674	-49.1602016146674\\
63.875	0.17352	-52.3227974306372	-52.3227974306372\\
63.875	0.17718	-55.6076249382568	-55.6076249382568\\
63.875	0.18084	-59.0146841375262	-59.0146841375262\\
63.875	0.1845	-62.5439750284455	-62.5439750284455\\
63.875	0.18816	-66.1954976110145	-66.1954976110145\\
63.875	0.19182	-69.9692518852333	-69.9692518852333\\
63.875	0.19548	-73.8652378511019	-73.8652378511019\\
63.875	0.19914	-77.8834555086203	-77.8834555086203\\
63.875	0.2028	-82.0239048577885	-82.0239048577885\\
63.875	0.20646	-86.2865858986065	-86.2865858986065\\
63.875	0.21012	-90.6714986310743	-90.6714986310743\\
63.875	0.21378	-95.1786430551919	-95.1786430551919\\
63.875	0.21744	-99.8080191709592	-99.8080191709592\\
63.875	0.2211	-104.559626978376	-104.559626978376\\
63.875	0.22476	-109.433466477443	-109.433466477443\\
63.875	0.22842	-114.42953766816	-114.42953766816\\
63.875	0.23208	-119.547840550527	-119.547840550527\\
63.875	0.23574	-124.788375124543	-124.788375124543\\
63.875	0.2394	-130.151141390209	-130.151141390209\\
63.875	0.24306	-135.636139347525	-135.636139347525\\
63.875	0.24672	-141.243368996491	-141.243368996491\\
63.875	0.25038	-146.972830337106	-146.972830337106\\
63.875	0.25404	-152.824523369372	-152.824523369372\\
63.875	0.2577	-158.798448093287	-158.798448093287\\
63.875	0.26136	-164.894604508852	-164.894604508852\\
63.875	0.26502	-171.112992616066	-171.112992616066\\
63.875	0.26868	-177.453612414931	-177.453612414931\\
63.875	0.27234	-183.916463905445	-183.916463905445\\
63.875	0.276	-190.501547087609	-190.501547087609\\
64.25	0.093	-11.090401228762	-11.090401228762\\
64.25	0.09666	-11.6889103398682	-11.6889103398682\\
64.25	0.10032	-12.4096511426241	-12.4096511426241\\
64.25	0.10398	-13.2526236370299	-13.2526236370299\\
64.25	0.10764	-14.2178278230854	-14.2178278230854\\
64.25	0.1113	-15.3052637007908	-15.3052637007908\\
64.25	0.11496	-16.5149312701459	-16.5149312701459\\
64.25	0.11862	-17.8468305311509	-17.8468305311509\\
64.25	0.12228	-19.3009614838056	-19.3009614838056\\
64.25	0.12594	-20.8773241281101	-20.8773241281101\\
64.25	0.1296	-22.5759184640644	-22.5759184640644\\
64.25	0.13326	-24.3967444916685	-24.3967444916685\\
64.25	0.13692	-26.3398022109224	-26.3398022109224\\
64.25	0.14058	-28.4050916218261	-28.4050916218261\\
64.25	0.14424	-30.5926127243796	-30.5926127243796\\
64.25	0.1479	-32.9023655185829	-32.9023655185829\\
64.25	0.15156	-35.334350004436	-35.334350004436\\
64.25	0.15522	-37.8885661819389	-37.8885661819389\\
64.25	0.15888	-40.5650140510916	-40.5650140510916\\
64.25	0.16254	-43.363693611894	-43.363693611894\\
64.25	0.1662	-46.2846048643463	-46.2846048643463\\
64.25	0.16986	-49.3277478084484	-49.3277478084484\\
64.25	0.17352	-52.4931224442002	-52.4931224442002\\
64.25	0.17718	-55.7807287716018	-55.7807287716018\\
64.25	0.18084	-59.1905667906532	-59.1905667906532\\
64.25	0.1845	-62.7226365013544	-62.7226365013544\\
64.25	0.18816	-66.3769379037055	-66.3769379037055\\
64.25	0.19182	-70.1534709977064	-70.1534709977064\\
64.25	0.19548	-74.0522357833569	-74.0522357833569\\
64.25	0.19914	-78.0732322606573	-78.0732322606573\\
64.25	0.2028	-82.2164604296076	-82.2164604296076\\
64.25	0.20646	-86.4819202902076	-86.4819202902076\\
64.25	0.21012	-90.8696118424574	-90.8696118424574\\
64.25	0.21378	-95.379535086357	-95.379535086357\\
64.25	0.21744	-100.011690021906	-100.011690021906\\
64.25	0.2211	-104.766076649105	-104.766076649105\\
64.25	0.22476	-109.642694967954	-109.642694967954\\
64.25	0.22842	-114.641544978453	-114.641544978453\\
64.25	0.23208	-119.762626680602	-119.762626680602\\
64.25	0.23574	-125.0059400744	-125.0059400744\\
64.25	0.2394	-130.371485159848	-130.371485159848\\
64.25	0.24306	-135.859261936946	-135.859261936946\\
64.25	0.24672	-141.469270405694	-141.469270405694\\
64.25	0.25038	-147.201510566092	-147.201510566092\\
64.25	0.25404	-153.055982418139	-153.055982418139\\
64.25	0.2577	-159.032685961836	-159.032685961836\\
64.25	0.26136	-165.131621197183	-165.131621197183\\
64.25	0.26502	-171.352788124179	-171.352788124179\\
64.25	0.26868	-177.696186742826	-177.696186742826\\
64.25	0.27234	-184.161817053122	-184.161817053122\\
64.25	0.276	-190.749679055068	-190.749679055068\\
64.625	0.093	-11.2091885820384	-11.2091885820384\\
64.625	0.09666	-11.8104765129265	-11.8104765129265\\
64.625	0.10032	-12.5339961354646	-12.5339961354646\\
64.625	0.10398	-13.3797474496523	-13.3797474496523\\
64.625	0.10764	-14.3477304554898	-14.3477304554898\\
64.625	0.1113	-15.4379451529772	-15.4379451529772\\
64.625	0.11496	-16.6503915421143	-16.6503915421143\\
64.625	0.11862	-17.9850696229013	-17.9850696229013\\
64.625	0.12228	-19.4419793953381	-19.4419793953381\\
64.625	0.12594	-21.0211208594246	-21.0211208594246\\
64.625	0.1296	-22.7224940151609	-22.7224940151609\\
64.625	0.13326	-24.546098862547	-24.546098862547\\
64.625	0.13692	-26.4919354015829	-26.4919354015829\\
64.625	0.14058	-28.5600036322686	-28.5600036322686\\
64.625	0.14424	-30.7503035546041	-30.7503035546041\\
64.625	0.1479	-33.0628351685894	-33.0628351685894\\
64.625	0.15156	-35.4975984742245	-35.4975984742245\\
64.625	0.15522	-38.0545934715094	-38.0545934715094\\
64.625	0.15888	-40.7338201604441	-40.7338201604441\\
64.625	0.16254	-43.5352785410286	-43.5352785410286\\
64.625	0.1662	-46.4589686132628	-46.4589686132628\\
64.625	0.16986	-49.5048903771469	-49.5048903771469\\
64.625	0.17352	-52.6730438326807	-52.6730438326807\\
64.625	0.17718	-55.9634289798644	-55.9634289798644\\
64.625	0.18084	-59.3760458186978	-59.3760458186978\\
64.625	0.1845	-62.9108943491811	-62.9108943491811\\
64.625	0.18816	-66.5679745713141	-66.5679745713141\\
64.625	0.19182	-70.3472864850969	-70.3472864850969\\
64.625	0.19548	-74.2488300905295	-74.2488300905295\\
64.625	0.19914	-78.272605387612	-78.272605387612\\
64.625	0.2028	-82.4186123763442	-82.4186123763442\\
64.625	0.20646	-86.6868510567262	-86.6868510567262\\
64.625	0.21012	-91.077321428758	-91.077321428758\\
64.625	0.21378	-95.5900234924396	-95.5900234924396\\
64.625	0.21744	-100.224957247771	-100.224957247771\\
64.625	0.2211	-104.982122694752	-104.982122694752\\
64.625	0.22476	-109.861519833383	-109.861519833383\\
64.625	0.22842	-114.863148663664	-114.863148663664\\
64.625	0.23208	-119.987009185594	-119.987009185594\\
64.625	0.23574	-125.233101399175	-125.233101399175\\
64.625	0.2394	-130.601425304405	-130.601425304405\\
64.625	0.24306	-136.091980901285	-136.091980901285\\
64.625	0.24672	-141.704768189815	-141.704768189815\\
64.625	0.25038	-147.439787169994	-147.439787169994\\
64.625	0.25404	-153.297037841824	-153.297037841824\\
64.625	0.2577	-159.276520205303	-159.276520205303\\
64.625	0.26136	-165.378234260432	-165.378234260432\\
64.625	0.26502	-171.60218000721	-171.60218000721\\
64.625	0.26868	-177.948357445639	-177.948357445639\\
64.625	0.27234	-184.416766575717	-184.416766575717\\
64.625	0.276	-191.007407397445	-191.007407397445\\
65	0.093	-11.3375723102322	-11.3375723102322\\
65	0.09666	-11.9416390609023	-11.9416390609023\\
65	0.10032	-12.6679375032224	-12.6679375032224\\
65	0.10398	-13.5164676371921	-13.5164676371921\\
65	0.10764	-14.4872294628117	-14.4872294628117\\
65	0.1113	-15.5802229800811	-15.5802229800811\\
65	0.11496	-16.7954481890002	-16.7954481890002\\
65	0.11862	-18.1329050895691	-18.1329050895691\\
65	0.12228	-19.5925936817879	-19.5925936817879\\
65	0.12594	-21.1745139656565	-21.1745139656565\\
65	0.1296	-22.8786659411747	-22.8786659411747\\
65	0.13326	-24.7050496083429	-24.7050496083429\\
65	0.13692	-26.6536649671608	-26.6536649671608\\
65	0.14058	-28.7245120176285	-28.7245120176285\\
65	0.14424	-30.917590759746	-30.917590759746\\
65	0.1479	-33.2329011935133	-33.2329011935133\\
65	0.15156	-35.6704433189305	-35.6704433189305\\
65	0.15522	-38.2302171359973	-38.2302171359973\\
65	0.15888	-40.912222644714	-40.912222644714\\
65	0.16254	-43.7164598450805	-43.7164598450805\\
65	0.1662	-46.6429287370968	-46.6429287370968\\
65	0.16986	-49.6916293207629	-49.6916293207629\\
65	0.17352	-52.8625615960787	-52.8625615960787\\
65	0.17718	-56.1557255630444	-56.1557255630444\\
65	0.18084	-59.5711212216598	-59.5711212216598\\
65	0.1845	-63.1087485719251	-63.1087485719251\\
65	0.18816	-66.7686076138401	-66.7686076138401\\
65	0.19182	-70.5506983474049	-70.5506983474049\\
65	0.19548	-74.4550207726196	-74.4550207726196\\
65	0.19914	-78.481574889484	-78.481574889484\\
65	0.2028	-82.6303606979983	-82.6303606979983\\
65	0.20646	-86.9013781981623	-86.9013781981623\\
65	0.21012	-91.2946273899761	-91.2946273899761\\
65	0.21378	-95.8101082734396	-95.8101082734396\\
65	0.21744	-100.447820848553	-100.447820848553\\
65	0.2211	-105.207765115316	-105.207765115316\\
65	0.22476	-110.089941073729	-110.089941073729\\
65	0.22842	-115.094348723792	-115.094348723792\\
65	0.23208	-120.220988065505	-120.220988065505\\
65	0.23574	-125.469859098867	-125.469859098867\\
65	0.2394	-130.840961823879	-130.840961823879\\
65	0.24306	-136.334296240541	-136.334296240541\\
65	0.24672	-141.949862348853	-141.949862348853\\
65	0.25038	-147.687660148814	-147.687660148814\\
65	0.25404	-153.547689640426	-153.547689640426\\
65	0.2577	-159.529950823687	-159.529950823687\\
65	0.26136	-165.634443698598	-165.634443698598\\
65	0.26502	-171.861168265158	-171.861168265158\\
65	0.26868	-178.210124523369	-178.210124523369\\
65	0.27234	-184.681312473229	-184.681312473229\\
65	0.276	-191.274732114739	-191.274732114739\\
65.375	0.093	-11.4755524133436	-11.4755524133436\\
65.375	0.09666	-12.0823979837957	-12.0823979837957\\
65.375	0.10032	-12.8114752458977	-12.8114752458977\\
65.375	0.10398	-13.6627841996495	-13.6627841996495\\
65.375	0.10764	-14.6363248450511	-14.6363248450511\\
65.375	0.1113	-15.7320971821025	-15.7320971821025\\
65.375	0.11496	-16.9501012108036	-16.9501012108036\\
65.375	0.11862	-18.2903369311546	-18.2903369311546\\
65.375	0.12228	-19.7528043431554	-19.7528043431554\\
65.375	0.12594	-21.3375034468059	-21.3375034468059\\
65.375	0.1296	-23.0444342421062	-23.0444342421062\\
65.375	0.13326	-24.8735967290564	-24.8735967290564\\
65.375	0.13692	-26.8249909076563	-26.8249909076563\\
65.375	0.14058	-28.898616777906	-28.898616777906\\
65.375	0.14424	-31.0944743398055	-31.0944743398055\\
65.375	0.1479	-33.4125635933548	-33.4125635933548\\
65.375	0.15156	-35.852884538554	-35.852884538554\\
65.375	0.15522	-38.4154371754029	-38.4154371754029\\
65.375	0.15888	-41.1002215039016	-41.1002215039016\\
65.375	0.16254	-43.9072375240501	-43.9072375240501\\
65.375	0.1662	-46.8364852358483	-46.8364852358483\\
65.375	0.16986	-49.8879646392964	-49.8879646392964\\
65.375	0.17352	-53.0616757343943	-53.0616757343943\\
65.375	0.17718	-56.3576185211419	-56.3576185211419\\
65.375	0.18084	-59.7757929995394	-59.7757929995394\\
65.375	0.1845	-63.3161991695866	-63.3161991695866\\
65.375	0.18816	-66.9788370312837	-66.9788370312837\\
65.375	0.19182	-70.7637065846306	-70.7637065846306\\
65.375	0.19548	-74.6708078296272	-74.6708078296272\\
65.375	0.19914	-78.7001407662736	-78.7001407662736\\
65.375	0.2028	-82.8517053945699	-82.8517053945699\\
65.375	0.20646	-87.1255017145159	-87.1255017145159\\
65.375	0.21012	-91.5215297261118	-91.5215297261118\\
65.375	0.21378	-96.0397894293573	-96.0397894293573\\
65.375	0.21744	-100.680280824253	-100.680280824253\\
65.375	0.2211	-105.443003910798	-105.443003910798\\
65.375	0.22476	-110.327958688993	-110.327958688993\\
65.375	0.22842	-115.335145158838	-115.335145158838\\
65.375	0.23208	-120.464563320332	-120.464563320332\\
65.375	0.23574	-125.716213173477	-125.716213173477\\
65.375	0.2394	-131.090094718271	-131.090094718271\\
65.375	0.24306	-136.586207954715	-136.586207954715\\
65.375	0.24672	-142.204552882809	-142.204552882809\\
65.375	0.25038	-147.945129502552	-147.945129502552\\
65.375	0.25404	-153.807937813946	-153.807937813946\\
65.375	0.2577	-159.792977816989	-159.792977816989\\
65.375	0.26136	-165.900249511682	-165.900249511682\\
65.375	0.26502	-172.129752898024	-172.129752898024\\
65.375	0.26868	-178.481487976017	-178.481487976017\\
65.375	0.27234	-184.955454745659	-184.955454745659\\
65.375	0.276	-191.551653206951	-191.551653206951\\
65.75	0.093	-11.6231288913724	-11.6231288913724\\
65.75	0.09666	-12.2327532816065	-12.2327532816065\\
65.75	0.10032	-12.9646093634906	-12.9646093634906\\
65.75	0.10398	-13.8186971370244	-13.8186971370244\\
65.75	0.10764	-14.7950166022079	-14.7950166022079\\
65.75	0.1113	-15.8935677590413	-15.8935677590413\\
65.75	0.11496	-17.1143506075245	-17.1143506075245\\
65.75	0.11862	-18.4573651476575	-18.4573651476575\\
65.75	0.12228	-19.9226113794402	-19.9226113794402\\
65.75	0.12594	-21.5100893028728	-21.5100893028728\\
65.75	0.1296	-23.2197989179551	-23.2197989179551\\
65.75	0.13326	-25.0517402246873	-25.0517402246873\\
65.75	0.13692	-27.0059132230692	-27.0059132230692\\
65.75	0.14058	-29.082317913101	-29.082317913101\\
65.75	0.14424	-31.2809542947825	-31.2809542947825\\
65.75	0.1479	-33.6018223681138	-33.6018223681138\\
65.75	0.15156	-36.0449221330949	-36.0449221330949\\
65.75	0.15522	-38.6102535897258	-38.6102535897258\\
65.75	0.15888	-41.2978167380066	-41.2978167380066\\
65.75	0.16254	-44.107611577937	-44.107611577937\\
65.75	0.1662	-47.0396381095173	-47.0396381095173\\
65.75	0.16986	-50.0938963327474	-50.0938963327474\\
65.75	0.17352	-53.2703862476273	-53.2703862476273\\
65.75	0.17718	-56.569107854157	-56.569107854157\\
65.75	0.18084	-59.9900611523364	-59.9900611523364\\
65.75	0.1845	-63.5332461421657	-63.5332461421657\\
65.75	0.18816	-67.1986628236448	-67.1986628236448\\
65.75	0.19182	-70.9863111967736	-70.9863111967736\\
65.75	0.19548	-74.8961912615523	-74.8961912615523\\
65.75	0.19914	-78.9283030179807	-78.9283030179807\\
65.75	0.2028	-83.082646466059	-83.082646466059\\
65.75	0.20646	-87.359221605787	-87.359221605787\\
65.75	0.21012	-91.7580284371648	-91.7580284371648\\
65.75	0.21378	-96.2790669601924	-96.2790669601924\\
65.75	0.21744	-100.92233717487	-100.92233717487\\
65.75	0.2211	-105.687839081197	-105.687839081197\\
65.75	0.22476	-110.575572679174	-110.575572679174\\
65.75	0.22842	-115.585537968801	-115.585537968801\\
65.75	0.23208	-120.717734950077	-120.717734950077\\
65.75	0.23574	-125.972163623004	-125.972163623004\\
65.75	0.2394	-131.34882398758	-131.34882398758\\
65.75	0.24306	-136.847716043806	-136.847716043806\\
65.75	0.24672	-142.468839791682	-142.468839791682\\
65.75	0.25038	-148.212195231207	-148.212195231207\\
65.75	0.25404	-154.077782362383	-154.077782362383\\
65.75	0.2577	-160.065601185208	-160.065601185208\\
65.75	0.26136	-166.175651699683	-166.175651699683\\
65.75	0.26502	-172.407933905808	-172.407933905808\\
65.75	0.26868	-178.762447803582	-178.762447803582\\
65.75	0.27234	-185.239193393006	-185.239193393006\\
65.75	0.276	-191.83817067408	-191.83817067408\\
66.125	0.093	-11.7803017443187	-11.7803017443187\\
66.125	0.09666	-12.3927049543348	-12.3927049543348\\
66.125	0.10032	-13.1273398560009	-13.1273398560009\\
66.125	0.10398	-13.9842064493166	-13.9842064493166\\
66.125	0.10764	-14.9633047342822	-14.9633047342822\\
66.125	0.1113	-16.0646347108977	-16.0646347108977\\
66.125	0.11496	-17.2881963791628	-17.2881963791628\\
66.125	0.11862	-18.6339897390778	-18.6339897390778\\
66.125	0.12228	-20.1020147906426	-20.1020147906426\\
66.125	0.12594	-21.6922715338572	-21.6922715338572\\
66.125	0.1296	-23.4047599687215	-23.4047599687215\\
66.125	0.13326	-25.2394800952356	-25.2394800952356\\
66.125	0.13692	-27.1964319133996	-27.1964319133996\\
66.125	0.14058	-29.2756154232134	-29.2756154232134\\
66.125	0.14424	-31.4770306246769	-31.4770306246769\\
66.125	0.1479	-33.8006775177902	-33.8006775177902\\
66.125	0.15156	-36.2465561025533	-36.2465561025533\\
66.125	0.15522	-38.8146663789663	-38.8146663789663\\
66.125	0.15888	-41.505008347029	-41.505008347029\\
66.125	0.16254	-44.3175820067415	-44.3175820067415\\
66.125	0.1662	-47.2523873581038	-47.2523873581038\\
66.125	0.16986	-50.3094244011159	-50.3094244011159\\
66.125	0.17352	-53.4886931357777	-53.4886931357777\\
66.125	0.17718	-56.7901935620894	-56.7901935620894\\
66.125	0.18084	-60.2139256800509	-60.2139256800509\\
66.125	0.1845	-63.7598894896622	-63.7598894896622\\
66.125	0.18816	-67.4280849909233	-67.4280849909233\\
66.125	0.19182	-71.2185121838341	-71.2185121838341\\
66.125	0.19548	-75.1311710683948	-75.1311710683948\\
66.125	0.19914	-79.1660616446052	-79.1660616446052\\
66.125	0.2028	-83.3231839124655	-83.3231839124655\\
66.125	0.20646	-87.6025378719756	-87.6025378719756\\
66.125	0.21012	-92.0041235231354	-92.0041235231354\\
66.125	0.21378	-96.527940865945	-96.527940865945\\
66.125	0.21744	-101.173989900404	-101.173989900404\\
66.125	0.2211	-105.942270626514	-105.942270626514\\
66.125	0.22476	-110.832783044273	-110.832783044273\\
66.125	0.22842	-115.845527153681	-115.845527153681\\
66.125	0.23208	-120.98050295474	-120.98050295474\\
66.125	0.23574	-126.237710447448	-126.237710447448\\
66.125	0.2394	-131.617149631807	-131.617149631807\\
66.125	0.24306	-137.118820507815	-137.118820507815\\
66.125	0.24672	-142.742723075472	-142.742723075472\\
66.125	0.25038	-148.48885733478	-148.48885733478\\
66.125	0.25404	-154.357223285737	-154.357223285737\\
66.125	0.2577	-160.347820928345	-160.347820928345\\
66.125	0.26136	-166.460650262602	-166.460650262602\\
66.125	0.26502	-172.695711288508	-172.695711288508\\
66.125	0.26868	-179.053004006065	-179.053004006065\\
66.125	0.27234	-185.532528415271	-185.532528415271\\
66.125	0.276	-192.134284516127	-192.134284516127\\
66.5	0.093	-11.9470709721825	-11.9470709721825\\
66.5	0.09666	-12.5622530019806	-12.5622530019806\\
66.5	0.10032	-13.2996667234287	-13.2996667234287\\
66.5	0.10398	-14.1593121365265	-14.1593121365265\\
66.5	0.10764	-15.1411892412741	-15.1411892412741\\
66.5	0.1113	-16.2452980376715	-16.2452980376715\\
66.5	0.11496	-17.4716385257186	-17.4716385257186\\
66.5	0.11862	-18.8202107054156	-18.8202107054156\\
66.5	0.12228	-20.2910145767624	-20.2910145767624\\
66.5	0.12594	-21.884050139759	-21.884050139759\\
66.5	0.1296	-23.5993173944053	-23.5993173944053\\
66.5	0.13326	-25.4368163407015	-25.4368163407015\\
66.5	0.13692	-27.3965469786475	-27.3965469786475\\
66.5	0.14058	-29.4785093082432	-29.4785093082432\\
66.5	0.14424	-31.6827033294888	-31.6827033294888\\
66.5	0.1479	-34.0091290423841	-34.0091290423841\\
66.5	0.15156	-36.4577864469293	-36.4577864469293\\
66.5	0.15522	-39.0286755431242	-39.0286755431242\\
66.5	0.15888	-41.7217963309689	-41.7217963309689\\
66.5	0.16254	-44.5371488104634	-44.5371488104634\\
66.5	0.1662	-47.4747329816077	-47.4747329816077\\
66.5	0.16986	-50.5345488444019	-50.5345488444019\\
66.5	0.17352	-53.7165963988458	-53.7165963988458\\
66.5	0.17718	-57.0208756449394	-57.0208756449394\\
66.5	0.18084	-60.4473865826829	-60.4473865826829\\
66.5	0.1845	-63.9961292120762	-63.9961292120762\\
66.5	0.18816	-67.6671035331193	-67.6671035331193\\
66.5	0.19182	-71.4603095458122	-71.4603095458122\\
66.5	0.19548	-75.3757472501548	-75.3757472501548\\
66.5	0.19914	-79.4134166461473	-79.4134166461473\\
66.5	0.2028	-83.5733177337896	-83.5733177337896\\
66.5	0.20646	-87.8554505130816	-87.8554505130816\\
66.5	0.21012	-92.2598149840235	-92.2598149840235\\
66.5	0.21378	-96.7864111466151	-96.7864111466151\\
66.5	0.21744	-101.435239000856	-101.435239000856\\
66.5	0.2211	-106.206298546748	-106.206298546748\\
66.5	0.22476	-111.099589784289	-111.099589784289\\
66.5	0.22842	-116.11511271348	-116.11511271348\\
66.5	0.23208	-121.25286733432	-121.25286733432\\
66.5	0.23574	-126.512853646811	-126.512853646811\\
66.5	0.2394	-131.895071650951	-131.895071650951\\
66.5	0.24306	-137.399521346741	-137.399521346741\\
66.5	0.24672	-143.026202734181	-143.026202734181\\
66.5	0.25038	-148.77511581327	-148.77511581327\\
66.5	0.25404	-154.64626058401	-154.64626058401\\
66.5	0.2577	-160.639637046399	-160.639637046399\\
66.5	0.26136	-166.755245200438	-166.755245200438\\
66.5	0.26502	-172.993085046126	-172.993085046126\\
66.5	0.26868	-179.353156583465	-179.353156583465\\
66.5	0.27234	-185.835459812453	-185.835459812453\\
66.5	0.276	-192.439994733091	-192.439994733091\\
66.875	0.093	-12.1234365749637	-12.1234365749637\\
66.875	0.09666	-12.7413974245439	-12.7413974245439\\
66.875	0.10032	-13.4815899657739	-13.4815899657739\\
66.875	0.10398	-14.3440141986537	-14.3440141986537\\
66.875	0.10764	-15.3286701231833	-15.3286701231833\\
66.875	0.1113	-16.4355577393628	-16.4355577393628\\
66.875	0.11496	-17.6646770471919	-17.6646770471919\\
66.875	0.11862	-19.016028046671	-19.016028046671\\
66.875	0.12228	-20.4896107377998	-20.4896107377998\\
66.875	0.12594	-22.0854251205783	-22.0854251205783\\
66.875	0.1296	-23.8034711950067	-23.8034711950067\\
66.875	0.13326	-25.6437489610849	-25.6437489610849\\
66.875	0.13692	-27.6062584188128	-27.6062584188128\\
66.875	0.14058	-29.6909995681906	-29.6909995681906\\
66.875	0.14424	-31.8979724092182	-31.8979724092182\\
66.875	0.1479	-34.2271769418955	-34.2271769418955\\
66.875	0.15156	-36.6786131662226	-36.6786131662226\\
66.875	0.15522	-39.2522810821996	-39.2522810821996\\
66.875	0.15888	-41.9481806898263	-41.9481806898263\\
66.875	0.16254	-44.7663119891029	-44.7663119891029\\
66.875	0.1662	-47.7066749800292	-47.7066749800292\\
66.875	0.16986	-50.7692696626053	-50.7692696626053\\
66.875	0.17352	-53.9540960368312	-53.9540960368312\\
66.875	0.17718	-57.2611541027069	-57.2611541027069\\
66.875	0.18084	-60.6904438602324	-60.6904438602324\\
66.875	0.1845	-64.2419653094076	-64.2419653094076\\
66.875	0.18816	-67.9157184502328	-67.9157184502328\\
66.875	0.19182	-71.7117032827077	-71.7117032827077\\
66.875	0.19548	-75.6299198068323	-75.6299198068323\\
66.875	0.19914	-79.6703680226068	-79.6703680226068\\
66.875	0.2028	-83.8330479300311	-83.8330479300311\\
66.875	0.20646	-88.1179595291051	-88.1179595291051\\
66.875	0.21012	-92.525102819829	-92.525102819829\\
66.875	0.21378	-97.0544778022027	-97.0544778022027\\
66.875	0.21744	-101.706084476226	-101.706084476226\\
66.875	0.2211	-106.479922841899	-106.479922841899\\
66.875	0.22476	-111.375992899222	-111.375992899222\\
66.875	0.22842	-116.394294648195	-116.394294648195\\
66.875	0.23208	-121.534828088818	-121.534828088818\\
66.875	0.23574	-126.79759322109	-126.79759322109\\
66.875	0.2394	-132.182590045012	-132.182590045012\\
66.875	0.24306	-137.689818560584	-137.689818560584\\
66.875	0.24672	-143.319278767806	-143.319278767806\\
66.875	0.25038	-149.070970666678	-149.070970666678\\
66.875	0.25404	-154.944894257199	-154.944894257199\\
66.875	0.2577	-160.94104953937	-160.94104953937\\
66.875	0.26136	-167.059436513191	-167.059436513191\\
66.875	0.26502	-173.300055178662	-173.300055178662\\
66.875	0.26868	-179.662905535783	-179.662905535783\\
66.875	0.27234	-186.147987584553	-186.147987584553\\
66.875	0.276	-192.755301324973	-192.755301324973\\
67.25	0.093	-12.3093985526625	-12.3093985526625\\
67.25	0.09666	-12.9301382220247	-12.9301382220247\\
67.25	0.10032	-13.6731095830367	-13.6731095830367\\
67.25	0.10398	-14.5383126356986	-14.5383126356986\\
67.25	0.10764	-15.5257473800102	-15.5257473800102\\
67.25	0.1113	-16.6354138159716	-16.6354138159716\\
67.25	0.11496	-17.8673119435828	-17.8673119435828\\
67.25	0.11862	-19.2214417628438	-19.2214417628438\\
67.25	0.12228	-20.6978032737546	-20.6978032737546\\
67.25	0.12594	-22.2963964763152	-22.2963964763152\\
67.25	0.1296	-24.0172213705256	-24.0172213705256\\
67.25	0.13326	-25.8602779563858	-25.8602779563858\\
67.25	0.13692	-27.8255662338957	-27.8255662338957\\
67.25	0.14058	-29.9130862030555	-29.9130862030555\\
67.25	0.14424	-32.1228378638651	-32.1228378638651\\
67.25	0.1479	-34.4548212163244	-34.4548212163244\\
67.25	0.15156	-36.9090362604336	-36.9090362604336\\
67.25	0.15522	-39.4854829961925	-39.4854829961925\\
67.25	0.15888	-42.1841614236013	-42.1841614236013\\
67.25	0.16254	-45.0050715426598	-45.0050715426598\\
67.25	0.1662	-47.9482133533681	-47.9482133533681\\
67.25	0.16986	-51.0135868557263	-51.0135868557263\\
67.25	0.17352	-54.2011920497342	-54.2011920497342\\
67.25	0.17718	-57.5110289353919	-57.5110289353919\\
67.25	0.18084	-60.9430975126994	-60.9430975126994\\
67.25	0.1845	-64.4973977816567	-64.4973977816567\\
67.25	0.18816	-68.1739297422638	-68.1739297422638\\
67.25	0.19182	-71.9726933945207	-71.9726933945207\\
67.25	0.19548	-75.8936887384274	-75.8936887384274\\
67.25	0.19914	-79.9369157739839	-79.9369157739839\\
67.25	0.2028	-84.1023745011902	-84.1023745011902\\
67.25	0.20646	-88.3900649200462	-88.3900649200462\\
67.25	0.21012	-92.7999870305521	-92.7999870305521\\
67.25	0.21378	-97.3321408327077	-97.3321408327077\\
67.25	0.21744	-101.986526326513	-101.986526326513\\
67.25	0.2211	-106.763143511968	-106.763143511968\\
67.25	0.22476	-111.661992389073	-111.661992389073\\
67.25	0.22842	-116.683072957828	-116.683072957828\\
67.25	0.23208	-121.826385218233	-121.826385218233\\
67.25	0.23574	-127.091929170287	-127.091929170287\\
67.25	0.2394	-132.479704813992	-132.479704813992\\
67.25	0.24306	-137.989712149346	-137.989712149346\\
67.25	0.24672	-143.621951176349	-143.621951176349\\
67.25	0.25038	-149.376421895003	-149.376421895003\\
67.25	0.25404	-155.253124305306	-155.253124305306\\
67.25	0.2577	-161.25205840726	-161.25205840726\\
67.25	0.26136	-167.373224200863	-167.373224200863\\
67.25	0.26502	-173.616621686115	-173.616621686115\\
67.25	0.26868	-179.982250863018	-179.982250863018\\
67.25	0.27234	-186.47011173157	-186.47011173157\\
67.25	0.276	-193.080204291772	-193.080204291772\\
67.625	0.093	-12.5049569052787	-12.5049569052787\\
67.625	0.09666	-13.1284753944229	-13.1284753944229\\
67.625	0.10032	-13.874225575217	-13.874225575217\\
67.625	0.10398	-14.7422074476608	-14.7422074476608\\
67.625	0.10764	-15.7324210117544	-15.7324210117544\\
67.625	0.1113	-16.8448662674979	-16.8448662674979\\
67.625	0.11496	-18.079543214891	-18.079543214891\\
67.625	0.11862	-19.4364518539341	-19.4364518539341\\
67.625	0.12228	-20.9155921846269	-20.9155921846269\\
67.625	0.12594	-22.5169642069695	-22.5169642069695\\
67.625	0.1296	-24.2405679209619	-24.2405679209619\\
67.625	0.13326	-26.0864033266041	-26.0864033266041\\
67.625	0.13692	-28.054470423896	-28.054470423896\\
67.625	0.14058	-30.1447692128378	-30.1447692128378\\
67.625	0.14424	-32.3572996934294	-32.3572996934294\\
67.625	0.1479	-34.6920618656708	-34.6920618656708\\
67.625	0.15156	-37.149055729562	-37.149055729562\\
67.625	0.15522	-39.7282812851029	-39.7282812851029\\
67.625	0.15888	-42.4297385322937	-42.4297385322937\\
67.625	0.16254	-45.2534274711342	-45.2534274711342\\
67.625	0.1662	-48.1993481016245	-48.1993481016245\\
67.625	0.16986	-51.2675004237647	-51.2675004237647\\
67.625	0.17352	-54.4578844375546	-54.4578844375546\\
67.625	0.17718	-57.7705001429943	-57.7705001429943\\
67.625	0.18084	-61.2053475400838	-61.2053475400838\\
67.625	0.1845	-64.7624266288232	-64.7624266288232\\
67.625	0.18816	-68.4417374092122	-68.4417374092122\\
67.625	0.19182	-72.2432798812511	-72.2432798812511\\
67.625	0.19548	-76.1670540449398	-76.1670540449398\\
67.625	0.19914	-80.2130599002783	-80.2130599002783\\
67.625	0.2028	-84.3812974472666	-84.3812974472666\\
67.625	0.20646	-88.6717666859047	-88.6717666859047\\
67.625	0.21012	-93.0844676161925	-93.0844676161925\\
67.625	0.21378	-97.6194002381302	-97.6194002381302\\
67.625	0.21744	-102.276564551718	-102.276564551718\\
67.625	0.2211	-107.055960556955	-107.055960556955\\
67.625	0.22476	-111.957588253842	-111.957588253842\\
67.625	0.22842	-116.981447642379	-116.981447642379\\
67.625	0.23208	-122.127538722565	-122.127538722565\\
67.625	0.23574	-127.395861494402	-127.395861494402\\
67.625	0.2394	-132.786415957888	-132.786415957888\\
67.625	0.24306	-138.299202113024	-138.299202113024\\
67.625	0.24672	-143.93421995981	-143.93421995981\\
67.625	0.25038	-149.691469498246	-149.691469498246\\
67.625	0.25404	-155.570950728331	-155.570950728331\\
67.625	0.2577	-161.572663650066	-161.572663650066\\
67.625	0.26136	-167.696608263451	-167.696608263451\\
67.625	0.26502	-173.942784568486	-173.942784568486\\
67.625	0.26868	-180.311192565171	-180.311192565171\\
67.625	0.27234	-186.801832253505	-186.801832253505\\
67.625	0.276	-193.414703633489	-193.414703633489\\
68	0.093	-12.7101116328124	-12.7101116328124\\
68	0.09666	-13.3364089417387	-13.3364089417387\\
68	0.10032	-14.0849379423148	-14.0849379423148\\
68	0.10398	-14.9556986345406	-14.9556986345406\\
68	0.10764	-15.9486910184162	-15.9486910184162\\
68	0.1113	-17.0639150939417	-17.0639150939417\\
68	0.11496	-18.3013708611169	-18.3013708611169\\
68	0.11862	-19.6610583199419	-19.6610583199419\\
68	0.12228	-21.1429774704167	-21.1429774704167\\
68	0.12594	-22.7471283125413	-22.7471283125413\\
68	0.1296	-24.4735108463157	-24.4735108463157\\
68	0.13326	-26.3221250717399	-26.3221250717399\\
68	0.13692	-28.2929709888139	-28.2929709888139\\
68	0.14058	-30.3860485975377	-30.3860485975377\\
68	0.14424	-32.6013578979113	-32.6013578979113\\
68	0.1479	-34.9388988899347	-34.9388988899347\\
68	0.15156	-37.3986715736078	-37.3986715736078\\
68	0.15522	-39.9806759489308	-39.9806759489308\\
68	0.15888	-42.6849120159036	-42.6849120159036\\
68	0.16254	-45.5113797745261	-45.5113797745261\\
68	0.1662	-48.4600792247985	-48.4600792247985\\
68	0.16986	-51.5310103667206	-51.5310103667206\\
68	0.17352	-54.7241732002926	-54.7241732002926\\
68	0.17718	-58.0395677255142	-58.0395677255142\\
68	0.18084	-61.4771939423858	-61.4771939423858\\
68	0.1845	-65.0370518509071	-65.0370518509071\\
68	0.18816	-68.7191414510782	-68.7191414510782\\
68	0.19182	-72.5234627428992	-72.5234627428992\\
68	0.19548	-76.4500157263698	-76.4500157263698\\
68	0.19914	-80.4988004014903	-80.4988004014903\\
68	0.2028	-84.6698167682607	-84.6698167682607\\
68	0.20646	-88.9630648266807	-88.9630648266807\\
68	0.21012	-93.3785445767506	-93.3785445767506\\
68	0.21378	-97.9162560184703	-97.9162560184703\\
68	0.21744	-102.57619915184	-102.57619915184\\
68	0.2211	-107.358373976859	-107.358373976859\\
68	0.22476	-112.262780493528	-112.262780493528\\
68	0.22842	-117.289418701847	-117.289418701847\\
68	0.23208	-122.438288601816	-122.438288601816\\
68	0.23574	-127.709390193434	-127.709390193434\\
68	0.2394	-133.102723476702	-133.102723476702\\
68	0.24306	-138.61828845162	-138.61828845162\\
68	0.24672	-144.256085118188	-144.256085118188\\
68	0.25038	-150.016113476406	-150.016113476406\\
68	0.25404	-155.898373526273	-155.898373526273\\
68	0.2577	-161.90286526779	-161.90286526779\\
68	0.26136	-168.029588700957	-168.029588700957\\
68	0.26502	-174.278543825774	-174.278543825774\\
68	0.26868	-180.649730642241	-180.649730642241\\
68	0.27234	-187.143149150357	-187.143149150357\\
68	0.276	-193.758799350123	-193.758799350123\\
68.375	0.093	-12.9248627352637	-12.9248627352637\\
68.375	0.09666	-13.5539388639719	-13.5539388639719\\
68.375	0.10032	-14.30524668433	-14.30524668433\\
68.375	0.10398	-15.1787861963378	-15.1787861963378\\
68.375	0.10764	-16.1745573999954	-16.1745573999954\\
68.375	0.1113	-17.2925602953029	-17.2925602953029\\
68.375	0.11496	-18.5327948822601	-18.5327948822601\\
68.375	0.11862	-19.8952611608671	-19.8952611608671\\
68.375	0.12228	-21.379959131124	-21.379959131124\\
68.375	0.12594	-22.9868887930306	-22.9868887930306\\
68.375	0.1296	-24.716050146587	-24.716050146587\\
68.375	0.13326	-26.5674431917932	-26.5674431917932\\
68.375	0.13692	-28.5410679286492	-28.5410679286492\\
68.375	0.14058	-30.636924357155	-30.636924357155\\
68.375	0.14424	-32.8550124773106	-32.8550124773106\\
68.375	0.1479	-35.195332289116	-35.195332289116\\
68.375	0.15156	-37.6578837925712	-37.6578837925712\\
68.375	0.15522	-40.2426669876761	-40.2426669876761\\
68.375	0.15888	-42.9496818744309	-42.9496818744309\\
68.375	0.16254	-45.7789284528355	-45.7789284528355\\
68.375	0.1662	-48.7304067228898	-48.7304067228898\\
68.375	0.16986	-51.804116684594	-51.804116684594\\
68.375	0.17352	-55.000058337948	-55.000058337948\\
68.375	0.17718	-58.3182316829516	-58.3182316829516\\
68.375	0.18084	-61.7586367196052	-61.7586367196052\\
68.375	0.1845	-65.3212734479085	-65.3212734479085\\
68.375	0.18816	-69.0061418678617	-69.0061418678617\\
68.375	0.19182	-72.8132419794646	-72.8132419794646\\
68.375	0.19548	-76.7425737827173	-76.7425737827173\\
68.375	0.19914	-80.7941372776198	-80.7941372776198\\
68.375	0.2028	-84.9679324641721	-84.9679324641721\\
68.375	0.20646	-89.2639593423742	-89.2639593423742\\
68.375	0.21012	-93.6822179122261	-93.6822179122261\\
68.375	0.21378	-98.2227081737278	-98.2227081737278\\
68.375	0.21744	-102.885430126879	-102.885430126879\\
68.375	0.2211	-107.67038377168	-107.67038377168\\
68.375	0.22476	-112.577569108132	-112.577569108132\\
68.375	0.22842	-117.606986136232	-117.606986136232\\
68.375	0.23208	-122.758634855983	-122.758634855983\\
68.375	0.23574	-128.032515267384	-128.032515267384\\
68.375	0.2394	-133.428627370434	-133.428627370434\\
68.375	0.24306	-138.946971165134	-138.946971165134\\
68.375	0.24672	-144.587546651484	-144.587546651484\\
68.375	0.25038	-150.350353829483	-150.350353829483\\
68.375	0.25404	-156.235392699133	-156.235392699133\\
68.375	0.2577	-162.242663260432	-162.242663260432\\
68.375	0.26136	-168.372165513381	-168.372165513381\\
68.375	0.26502	-174.62389945798	-174.62389945798\\
68.375	0.26868	-180.997865094228	-180.997865094228\\
68.375	0.27234	-187.494062422127	-187.494062422127\\
68.375	0.276	-194.112491441675	-194.112491441675\\
68.75	0.093	-13.1492102126324	-13.1492102126324\\
68.75	0.09666	-13.7810651611226	-13.7810651611226\\
68.75	0.10032	-14.5351518012627	-14.5351518012627\\
68.75	0.10398	-15.4114701330526	-15.4114701330526\\
68.75	0.10764	-16.4100201564922	-16.4100201564922\\
68.75	0.1113	-17.5308018715817	-17.5308018715817\\
68.75	0.11496	-18.7738152783209	-18.7738152783209\\
68.75	0.11862	-20.1390603767099	-20.1390603767099\\
68.75	0.12228	-21.6265371667488	-21.6265371667488\\
68.75	0.12594	-23.2362456484374	-23.2362456484374\\
68.75	0.1296	-24.9681858217758	-24.9681858217758\\
68.75	0.13326	-26.8223576867641	-26.8223576867641\\
68.75	0.13692	-28.7987612434021	-28.7987612434021\\
68.75	0.14058	-30.8973964916899	-30.8973964916899\\
68.75	0.14424	-33.1182634316275	-33.1182634316275\\
68.75	0.1479	-35.4613620632148	-35.4613620632148\\
68.75	0.15156	-37.9266923864521	-37.9266923864521\\
68.75	0.15522	-40.514254401339	-40.514254401339\\
68.75	0.15888	-43.2240481078758	-43.2240481078758\\
68.75	0.16254	-46.0560735060624	-46.0560735060624\\
68.75	0.1662	-49.0103305958988	-49.0103305958988\\
68.75	0.16986	-52.0868193773849	-52.0868193773849\\
68.75	0.17352	-55.2855398505208	-55.2855398505208\\
68.75	0.17718	-58.6064920153066	-58.6064920153066\\
68.75	0.18084	-62.0496758717421	-62.0496758717421\\
68.75	0.1845	-65.6150914198275	-65.6150914198275\\
68.75	0.18816	-69.3027386595626	-69.3027386595626\\
68.75	0.19182	-73.1126175909475	-73.1126175909475\\
68.75	0.19548	-77.0447282139823	-77.0447282139823\\
68.75	0.19914	-81.0990705286668	-81.0990705286668\\
68.75	0.2028	-85.2756445350011	-85.2756445350011\\
68.75	0.20646	-89.5744502329852	-89.5744502329852\\
68.75	0.21012	-93.9954876226191	-93.9954876226191\\
68.75	0.21378	-98.5387567039028	-98.5387567039028\\
68.75	0.21744	-103.204257476836	-103.204257476836\\
68.75	0.2211	-107.99198994142	-107.99198994142\\
68.75	0.22476	-112.901954097653	-112.901954097653\\
68.75	0.22842	-117.934149945535	-117.934149945535\\
68.75	0.23208	-123.088577485068	-123.088577485068\\
68.75	0.23574	-128.365236716251	-128.365236716251\\
68.75	0.2394	-133.764127639083	-133.764127639083\\
68.75	0.24306	-139.285250253565	-139.285250253565\\
68.75	0.24672	-144.928604559697	-144.928604559697\\
68.75	0.25038	-150.694190557478	-150.694190557478\\
68.75	0.25404	-156.58200824691	-156.58200824691\\
68.75	0.2577	-162.592057627991	-162.592057627991\\
68.75	0.26136	-168.724338700722	-168.724338700722\\
68.75	0.26502	-174.978851465103	-174.978851465103\\
68.75	0.26868	-181.355595921133	-181.355595921133\\
68.75	0.27234	-187.854572068814	-187.854572068814\\
68.75	0.276	-194.475779908144	-194.475779908144\\
69.125	0.093	-13.3831540649186	-13.3831540649186\\
69.125	0.09666	-14.0177878331909	-14.0177878331909\\
69.125	0.10032	-14.774653293113	-14.774653293113\\
69.125	0.10398	-15.6537504446848	-15.6537504446848\\
69.125	0.10764	-16.6550792879065	-16.6550792879065\\
69.125	0.1113	-17.778639822778	-17.778639822778\\
69.125	0.11496	-19.0244320492992	-19.0244320492992\\
69.125	0.11862	-20.3924559674702	-20.3924559674702\\
69.125	0.12228	-21.8827115772911	-21.8827115772911\\
69.125	0.12594	-23.4951988787617	-23.4951988787617\\
69.125	0.1296	-25.2299178718822	-25.2299178718822\\
69.125	0.13326	-27.0868685566524	-27.0868685566524\\
69.125	0.13692	-29.0660509330724	-29.0660509330724\\
69.125	0.14058	-31.1674650011422	-31.1674650011422\\
69.125	0.14424	-33.3911107608618	-33.3911107608618\\
69.125	0.1479	-35.7369882122312	-35.7369882122312\\
69.125	0.15156	-38.2050973552504	-38.2050973552504\\
69.125	0.15522	-40.7954381899194	-40.7954381899194\\
69.125	0.15888	-43.5080107162382	-43.5080107162382\\
69.125	0.16254	-46.3428149342068	-46.3428149342068\\
69.125	0.1662	-49.2998508438252	-49.2998508438252\\
69.125	0.16986	-52.3791184450934	-52.3791184450934\\
69.125	0.17352	-55.5806177380113	-55.5806177380113\\
69.125	0.17718	-58.904348722579	-58.904348722579\\
69.125	0.18084	-62.3503113987966	-62.3503113987966\\
69.125	0.1845	-65.9185057666639	-65.9185057666639\\
69.125	0.18816	-69.6089318261811	-69.6089318261811\\
69.125	0.19182	-73.421589577348	-73.421589577348\\
69.125	0.19548	-77.3564790201648	-77.3564790201648\\
69.125	0.19914	-81.4136001546313	-81.4136001546313\\
69.125	0.2028	-85.5929529807476	-85.5929529807476\\
69.125	0.20646	-89.8945374985138	-89.8945374985138\\
69.125	0.21012	-94.3183537079296	-94.3183537079296\\
69.125	0.21378	-98.8644016089954	-98.8644016089954\\
69.125	0.21744	-103.532681201711	-103.532681201711\\
69.125	0.2211	-108.323192486076	-108.323192486076\\
69.125	0.22476	-113.235935462091	-113.235935462091\\
69.125	0.22842	-118.270910129756	-118.270910129756\\
69.125	0.23208	-123.428116489071	-123.428116489071\\
69.125	0.23574	-128.707554540035	-128.707554540035\\
69.125	0.2394	-134.109224282649	-134.109224282649\\
69.125	0.24306	-139.633125716913	-139.633125716913\\
69.125	0.24672	-145.279258842827	-145.279258842827\\
69.125	0.25038	-151.047623660391	-151.047623660391\\
69.125	0.25404	-156.938220169604	-156.938220169604\\
69.125	0.2577	-162.951048370468	-162.951048370468\\
69.125	0.26136	-169.086108262981	-169.086108262981\\
69.125	0.26502	-175.343399847144	-175.343399847144\\
69.125	0.26868	-181.722923122956	-181.722923122956\\
69.125	0.27234	-188.224678090419	-188.224678090419\\
69.125	0.276	-194.848664749531	-194.848664749531\\
69.5	0.093	-13.6266942921223	-13.6266942921223\\
69.5	0.09666	-14.2641068801765	-14.2641068801765\\
69.5	0.10032	-15.0237511598807	-15.0237511598807\\
69.5	0.10398	-15.9056271312345	-15.9056271312345\\
69.5	0.10764	-16.9097347942382	-16.9097347942382\\
69.5	0.1113	-18.0360741488917	-18.0360741488917\\
69.5	0.11496	-19.2846451951949	-19.2846451951949\\
69.5	0.11862	-20.655447933148	-20.655447933148\\
69.5	0.12228	-22.1484823627509	-22.1484823627509\\
69.5	0.12594	-23.7637484840035	-23.7637484840035\\
69.5	0.1296	-25.5012462969059	-25.5012462969059\\
69.5	0.13326	-27.3609758014581	-27.3609758014581\\
69.5	0.13692	-29.3429369976602	-29.3429369976602\\
69.5	0.14058	-31.447129885512	-31.447129885512\\
69.5	0.14424	-33.6735544650136	-33.6735544650136\\
69.5	0.1479	-36.022210736165	-36.022210736165\\
69.5	0.15156	-38.4930986989662	-38.4930986989662\\
69.5	0.15522	-41.0862183534172	-41.0862183534172\\
69.5	0.15888	-43.8015696995181	-43.8015696995181\\
69.5	0.16254	-46.6391527372686	-46.6391527372686\\
69.5	0.1662	-49.598967466669	-49.598967466669\\
69.5	0.16986	-52.6810138877192	-52.6810138877192\\
69.5	0.17352	-55.8852920004192	-55.8852920004192\\
69.5	0.17718	-59.2118018047689	-59.2118018047689\\
69.5	0.18084	-62.6605433007685	-62.6605433007685\\
69.5	0.1845	-66.2315164884178	-66.2315164884178\\
69.5	0.18816	-69.924721367717	-69.924721367717\\
69.5	0.19182	-73.740157938666	-73.740157938666\\
69.5	0.19548	-77.6778262012647	-77.6778262012647\\
69.5	0.19914	-81.7377261555132	-81.7377261555132\\
69.5	0.2028	-85.9198578014115	-85.9198578014115\\
69.5	0.20646	-90.2242211389597	-90.2242211389597\\
69.5	0.21012	-94.6508161681576	-94.6508161681576\\
69.5	0.21378	-99.1996428890053	-99.1996428890053\\
69.5	0.21744	-103.870701301503	-103.870701301503\\
69.5	0.2211	-108.66399140565	-108.66399140565\\
69.5	0.22476	-113.579513201447	-113.579513201447\\
69.5	0.22842	-118.617266688894	-118.617266688894\\
69.5	0.23208	-123.777251867991	-123.777251867991\\
69.5	0.23574	-129.059468738737	-129.059468738737\\
69.5	0.2394	-134.463917301133	-134.463917301133\\
69.5	0.24306	-139.99059755518	-139.99059755518\\
69.5	0.24672	-145.639509500875	-145.639509500875\\
69.5	0.25038	-151.410653138221	-151.410653138221\\
69.5	0.25404	-157.304028467217	-157.304028467217\\
69.5	0.2577	-163.319635487862	-163.319635487862\\
69.5	0.26136	-169.457474200157	-169.457474200157\\
69.5	0.26502	-175.717544604102	-175.717544604102\\
69.5	0.26868	-182.099846699696	-182.099846699696\\
69.5	0.27234	-188.604380486941	-188.604380486941\\
69.5	0.276	-195.231145965835	-195.231145965835\\
69.875	0.093	-13.8798308942435	-13.8798308942435\\
69.875	0.09666	-14.5200223020797	-14.5200223020797\\
69.875	0.10032	-15.2824454015659	-15.2824454015659\\
69.875	0.10398	-16.1671001927017	-16.1671001927017\\
69.875	0.10764	-17.1739866754874	-17.1739866754874\\
69.875	0.1113	-18.3031048499229	-18.3031048499229\\
69.875	0.11496	-19.5544547160082	-19.5544547160082\\
69.875	0.11862	-20.9280362737432	-20.9280362737432\\
69.875	0.12228	-22.4238495231281	-22.4238495231281\\
69.875	0.12594	-24.0418944641628	-24.0418944641628\\
69.875	0.1296	-25.7821710968472	-25.7821710968472\\
69.875	0.13326	-27.6446794211814	-27.6446794211814\\
69.875	0.13692	-29.6294194371655	-29.6294194371655\\
69.875	0.14058	-31.7363911447993	-31.7363911447993\\
69.875	0.14424	-33.965594544083	-33.965594544083\\
69.875	0.1479	-36.3170296350163	-36.3170296350163\\
69.875	0.15156	-38.7906964175996	-38.7906964175996\\
69.875	0.15522	-41.3865948918326	-41.3865948918326\\
69.875	0.15888	-44.1047250577154	-44.1047250577154\\
69.875	0.16254	-46.945086915248	-46.945086915248\\
69.875	0.1662	-49.9076804644304	-49.9076804644304\\
69.875	0.16986	-52.9925057052626	-52.9925057052626\\
69.875	0.17352	-56.1995626377445	-56.1995626377445\\
69.875	0.17718	-59.5288512618763	-59.5288512618763\\
69.875	0.18084	-62.9803715776579	-62.9803715776579\\
69.875	0.1845	-66.5541235850893	-66.5541235850893\\
69.875	0.18816	-70.2501072841704	-70.2501072841704\\
69.875	0.19182	-74.0683226749013	-74.0683226749013\\
69.875	0.19548	-78.0087697572821	-78.0087697572821\\
69.875	0.19914	-82.0714485313127	-82.0714485313127\\
69.875	0.2028	-86.256358996993	-86.256358996993\\
69.875	0.20646	-90.5635011543232	-90.5635011543232\\
69.875	0.21012	-94.992875003303	-94.992875003303\\
69.875	0.21378	-99.5444805439328	-99.5444805439328\\
69.875	0.21744	-104.218317776212	-104.218317776212\\
69.875	0.2211	-109.014386700142	-109.014386700142\\
69.875	0.22476	-113.932687315721	-113.932687315721\\
69.875	0.22842	-118.97321962295	-118.97321962295\\
69.875	0.23208	-124.135983621828	-124.135983621828\\
69.875	0.23574	-129.420979312357	-129.420979312357\\
69.875	0.2394	-134.828206694535	-134.828206694535\\
69.875	0.24306	-140.357665768363	-140.357665768363\\
69.875	0.24672	-146.009356533841	-146.009356533841\\
69.875	0.25038	-151.783278990969	-151.783278990969\\
69.875	0.25404	-157.679433139746	-157.679433139746\\
69.875	0.2577	-163.697818980173	-163.697818980173\\
69.875	0.26136	-169.83843651225	-169.83843651225\\
69.875	0.26502	-176.101285735977	-176.101285735977\\
69.875	0.26868	-182.486366651354	-182.486366651354\\
69.875	0.27234	-188.99367925838	-188.99367925838\\
69.875	0.276	-195.623223557056	-195.623223557056\\
70.25	0.093	-14.1425638712821	-14.1425638712821\\
70.25	0.09666	-14.7855340989004	-14.7855340989004\\
70.25	0.10032	-15.5507360181685	-15.5507360181685\\
70.25	0.10398	-16.4381696290864	-16.4381696290864\\
70.25	0.10764	-17.4478349316541	-17.4478349316541\\
70.25	0.1113	-18.5797319258716	-18.5797319258716\\
70.25	0.11496	-19.8338606117388	-19.8338606117388\\
70.25	0.11862	-21.2102209892559	-21.2102209892559\\
70.25	0.12228	-22.7088130584228	-22.7088130584228\\
70.25	0.12594	-24.3296368192395	-24.3296368192395\\
70.25	0.1296	-26.0726922717059	-26.0726922717059\\
70.25	0.13326	-27.9379794158222	-27.9379794158222\\
70.25	0.13692	-29.9254982515882	-29.9254982515882\\
70.25	0.14058	-32.035248779004	-32.035248779004\\
70.25	0.14424	-34.2672309980697	-34.2672309980697\\
70.25	0.1479	-36.6214449087851	-36.6214449087851\\
70.25	0.15156	-39.0978905111504	-39.0978905111504\\
70.25	0.15522	-41.6965678051653	-41.6965678051653\\
70.25	0.15888	-44.4174767908302	-44.4174767908302\\
70.25	0.16254	-47.2606174681448	-47.2606174681448\\
70.25	0.1662	-50.2259898371092	-50.2259898371092\\
70.25	0.16986	-53.3135938977234	-53.3135938977234\\
70.25	0.17352	-56.5234296499873	-56.5234296499873\\
70.25	0.17718	-59.8554970939011	-59.8554970939011\\
70.25	0.18084	-63.3097962294647	-63.3097962294647\\
70.25	0.1845	-66.8863270566781	-66.8863270566781\\
70.25	0.18816	-70.5850895755413	-70.5850895755413\\
70.25	0.19182	-74.4060837860542	-74.4060837860542\\
70.25	0.19548	-78.3493096882169	-78.3493096882169\\
70.25	0.19914	-82.4147672820295	-82.4147672820295\\
70.25	0.2028	-86.6024565674919	-86.6024565674919\\
70.25	0.20646	-90.9123775446041	-90.9123775446041\\
70.25	0.21012	-95.3445302133659	-95.3445302133659\\
70.25	0.21378	-99.8989145737776	-99.8989145737776\\
70.25	0.21744	-104.575530625839	-104.575530625839\\
70.25	0.2211	-109.374378369551	-109.374378369551\\
70.25	0.22476	-114.295457804912	-114.295457804912\\
70.25	0.22842	-119.338768931923	-119.338768931923\\
70.25	0.23208	-124.504311750583	-124.504311750583\\
70.25	0.23574	-129.792086260894	-129.792086260894\\
70.25	0.2394	-135.202092462854	-135.202092462854\\
70.25	0.24306	-140.734330356464	-140.734330356464\\
70.25	0.24672	-146.388799941724	-146.388799941724\\
70.25	0.25038	-152.165501218634	-152.165501218634\\
70.25	0.25404	-158.064434187193	-158.064434187193\\
70.25	0.2577	-164.085598847402	-164.085598847402\\
70.25	0.26136	-170.228995199262	-170.228995199262\\
70.25	0.26502	-176.49462324277	-176.49462324277\\
70.25	0.26868	-182.882482977929	-182.882482977929\\
70.25	0.27234	-189.392574404737	-189.392574404737\\
70.25	0.276	-196.024897523196	-196.024897523196\\
70.625	0.093	-14.4148932232383	-14.4148932232383\\
70.625	0.09666	-15.0606422706386	-15.0606422706386\\
70.625	0.10032	-15.8286230096887	-15.8286230096887\\
70.625	0.10398	-16.7188354403886	-16.7188354403886\\
70.625	0.10764	-17.7312795627383	-17.7312795627383\\
70.625	0.1113	-18.8659553767379	-18.8659553767379\\
70.625	0.11496	-20.1228628823871	-20.1228628823871\\
70.625	0.11862	-21.5020020796862	-21.5020020796862\\
70.625	0.12228	-23.0033729686351	-23.0033729686351\\
70.625	0.12594	-24.6269755492338	-24.6269755492338\\
70.625	0.1296	-26.3728098214822	-26.3728098214822\\
70.625	0.13326	-28.2408757853805	-28.2408757853805\\
70.625	0.13692	-30.2311734409285	-30.2311734409285\\
70.625	0.14058	-32.3437027881264	-32.3437027881264\\
70.625	0.14424	-34.578463826974	-34.578463826974\\
70.625	0.1479	-36.9354565574715	-36.9354565574715\\
70.625	0.15156	-39.4146809796187	-39.4146809796187\\
70.625	0.15522	-42.0161370934158	-42.0161370934158\\
70.625	0.15888	-44.7398248988626	-44.7398248988626\\
70.625	0.16254	-47.5857443959591	-47.5857443959591\\
70.625	0.1662	-50.5538955847056	-50.5538955847056\\
70.625	0.16986	-53.6442784651018	-53.6442784651018\\
70.625	0.17352	-56.8568930371478	-56.8568930371478\\
70.625	0.17718	-60.1917393008435	-60.1917393008435\\
70.625	0.18084	-63.6488172561891	-63.6488172561891\\
70.625	0.1845	-67.2281269031845	-67.2281269031845\\
70.625	0.18816	-70.9296682418297	-70.9296682418297\\
70.625	0.19182	-74.7534412721247	-74.7534412721247\\
70.625	0.19548	-78.6994459940694	-78.6994459940694\\
70.625	0.19914	-82.767682407664	-82.767682407664\\
70.625	0.2028	-86.9581505129084	-86.9581505129084\\
70.625	0.20646	-91.2708503098026	-91.2708503098026\\
70.625	0.21012	-95.7057817983465	-95.7057817983465\\
70.625	0.21378	-100.26294497854	-100.26294497854\\
70.625	0.21744	-104.942339850384	-104.942339850384\\
70.625	0.2211	-109.743966413877	-109.743966413877\\
70.625	0.22476	-114.66782466902	-114.66782466902\\
70.625	0.22842	-119.713914615813	-119.713914615813\\
70.625	0.23208	-124.882236254256	-124.882236254256\\
70.625	0.23574	-130.172789584348	-130.172789584348\\
70.625	0.2394	-135.585574606091	-135.585574606091\\
70.625	0.24306	-141.120591319483	-141.120591319483\\
70.625	0.24672	-146.777839724525	-146.777839724525\\
70.625	0.25038	-152.557319821216	-152.557319821216\\
70.625	0.25404	-158.459031609558	-158.459031609558\\
70.625	0.2577	-164.482975089549	-164.482975089549\\
70.625	0.26136	-170.62915026119	-170.62915026119\\
70.625	0.26502	-176.897557124481	-176.897557124481\\
70.625	0.26868	-183.288195679422	-183.288195679422\\
70.625	0.27234	-189.801065926012	-189.801065926012\\
70.625	0.276	-196.436167864252	-196.436167864252\\
71	0.093	-14.6968189501119	-14.6968189501119\\
71	0.09666	-15.3453468172942	-15.3453468172942\\
71	0.10032	-16.1161063761264	-16.1161063761264\\
71	0.10398	-17.0090976266083	-17.0090976266083\\
71	0.10764	-18.02432056874	-18.02432056874\\
71	0.1113	-19.1617752025215	-19.1617752025215\\
71	0.11496	-20.4214615279528	-20.4214615279528\\
71	0.11862	-21.8033795450339	-21.8033795450339\\
71	0.12228	-23.3075292537648	-23.3075292537648\\
71	0.12594	-24.9339106541455	-24.9339106541455\\
71	0.1296	-26.6825237461759	-26.6825237461759\\
71	0.13326	-28.5533685298562	-28.5533685298562\\
71	0.13692	-30.5464450051862	-30.5464450051862\\
71	0.14058	-32.6617531721661	-32.6617531721661\\
71	0.14424	-34.8992930307958	-34.8992930307958\\
71	0.1479	-37.2590645810752	-37.2590645810752\\
71	0.15156	-39.7410678230044	-39.7410678230044\\
71	0.15522	-42.3453027565835	-42.3453027565835\\
71	0.15888	-45.0717693818123	-45.0717693818123\\
71	0.16254	-47.9204676986909	-47.9204676986909\\
71	0.1662	-50.8913977072194	-50.8913977072194\\
71	0.16986	-53.9845594073976	-53.9845594073976\\
71	0.17352	-57.1999527992256	-57.1999527992256\\
71	0.17718	-60.5375778827033	-60.5375778827033\\
71	0.18084	-63.997434657831	-63.997434657831\\
71	0.1845	-67.5795231246083	-67.5795231246083\\
71	0.18816	-71.2838432830355	-71.2838432830355\\
71	0.19182	-75.1103951331126	-75.1103951331126\\
71	0.19548	-79.0591786748393	-79.0591786748393\\
71	0.19914	-83.1301939082159	-83.1301939082159\\
71	0.2028	-87.3234408332422	-87.3234408332422\\
71	0.20646	-91.6389194499184	-91.6389194499184\\
71	0.21012	-96.0766297582444	-96.0766297582444\\
71	0.21378	-100.63657175822	-100.63657175822\\
71	0.21744	-105.318745449846	-105.318745449846\\
71	0.2211	-110.123150833121	-110.123150833121\\
71	0.22476	-115.049787908046	-115.049787908046\\
71	0.22842	-120.098656674621	-120.098656674621\\
71	0.23208	-125.269757132846	-125.269757132846\\
71	0.23574	-130.56308928272	-130.56308928272\\
71	0.2394	-135.978653124245	-135.978653124245\\
71	0.24306	-141.516448657419	-141.516448657419\\
71	0.24672	-147.176475882243	-147.176475882243\\
71	0.25038	-152.958734798716	-152.958734798716\\
71	0.25404	-158.86322540684	-158.86322540684\\
71	0.2577	-164.889947706613	-164.889947706613\\
71	0.26136	-171.038901698036	-171.038901698036\\
71	0.26502	-177.310087381109	-177.310087381109\\
71	0.26868	-183.703504755832	-183.703504755832\\
71	0.27234	-190.219153822204	-190.219153822204\\
71	0.276	-196.857034580226	-196.857034580226\\
71.375	0.093	-14.9883410519031	-14.9883410519031\\
71.375	0.09666	-15.6396477388673	-15.6396477388673\\
71.375	0.10032	-16.4131861174815	-16.4131861174815\\
71.375	0.10398	-17.3089561877454	-17.3089561877454\\
71.375	0.10764	-18.3269579496591	-18.3269579496591\\
71.375	0.1113	-19.4671914032227	-19.4671914032227\\
71.375	0.11496	-20.729656548436	-20.729656548436\\
71.375	0.11862	-22.114353385299	-22.114353385299\\
71.375	0.12228	-23.621281913812	-23.621281913812\\
71.375	0.12594	-25.2504421339747	-25.2504421339747\\
71.375	0.1296	-27.0018340457871	-27.0018340457871\\
71.375	0.13326	-28.8754576492494	-28.8754576492494\\
71.375	0.13692	-30.8713129443615	-30.8713129443615\\
71.375	0.14058	-32.9893999311234	-32.9893999311234\\
71.375	0.14424	-35.229718609535	-35.229718609535\\
71.375	0.1479	-37.5922689795965	-37.5922689795965\\
71.375	0.15156	-40.0770510413077	-40.0770510413077\\
71.375	0.15522	-42.6840647946687	-42.6840647946687\\
71.375	0.15888	-45.4133102396796	-45.4133102396796\\
71.375	0.16254	-48.2647873763403	-48.2647873763403\\
71.375	0.1662	-51.2384962046507	-51.2384962046507\\
71.375	0.16986	-54.3344367246109	-54.3344367246109\\
71.375	0.17352	-57.5526089362209	-57.5526089362209\\
71.375	0.17718	-60.8930128394807	-60.8930128394807\\
71.375	0.18084	-64.3556484343903	-64.3556484343903\\
71.375	0.1845	-67.9405157209497	-67.9405157209497\\
71.375	0.18816	-71.6476146991589	-71.6476146991589\\
71.375	0.19182	-75.4769453690179	-75.4769453690179\\
71.375	0.19548	-79.4285077305267	-79.4285077305267\\
71.375	0.19914	-83.5023017836852	-83.5023017836852\\
71.375	0.2028	-87.6983275284936	-87.6983275284936\\
71.375	0.20646	-92.0165849649518	-92.0165849649518\\
71.375	0.21012	-96.4570740930598	-96.4570740930598\\
71.375	0.21378	-101.019794912817	-101.019794912817\\
71.375	0.21744	-105.704747424225	-105.704747424225\\
71.375	0.2211	-110.511931627282	-110.511931627282\\
71.375	0.22476	-115.44134752199	-115.44134752199\\
71.375	0.22842	-120.492995108346	-120.492995108346\\
71.375	0.23208	-125.666874386353	-125.666874386353\\
71.375	0.23574	-130.96298535601	-130.96298535601\\
71.375	0.2394	-136.381328017316	-136.381328017316\\
71.375	0.24306	-141.921902370272	-141.921902370272\\
71.375	0.24672	-147.584708414878	-147.584708414878\\
71.375	0.25038	-153.369746151134	-153.369746151134\\
71.375	0.25404	-159.277015579039	-159.277015579039\\
71.375	0.2577	-165.306516698595	-165.306516698595\\
71.375	0.26136	-171.4582495098	-171.4582495098\\
71.375	0.26502	-177.732214012654	-177.732214012654\\
71.375	0.26868	-184.128410207159	-184.128410207159\\
71.375	0.27234	-190.646838093314	-190.646838093314\\
71.375	0.276	-197.287497671118	-197.287497671118\\
71.75	0.093	-15.2894595286117	-15.2894595286117\\
71.75	0.09666	-15.943545035358	-15.943545035358\\
71.75	0.10032	-16.7198622337541	-16.7198622337541\\
71.75	0.10398	-17.6184111238001	-17.6184111238001\\
71.75	0.10764	-18.6391917054958	-18.6391917054958\\
71.75	0.1113	-19.7822039788413	-19.7822039788413\\
71.75	0.11496	-21.0474479438366	-21.0474479438366\\
71.75	0.11862	-22.4349236004817	-22.4349236004817\\
71.75	0.12228	-23.9446309487766	-23.9446309487766\\
71.75	0.12594	-25.5765699887213	-25.5765699887213\\
71.75	0.1296	-27.3307407203158	-27.3307407203158\\
71.75	0.13326	-29.2071431435601	-29.2071431435601\\
71.75	0.13692	-31.2057772584542	-31.2057772584542\\
71.75	0.14058	-33.3266430649981	-33.3266430649981\\
71.75	0.14424	-35.5697405631917	-35.5697405631917\\
71.75	0.1479	-37.9350697530352	-37.9350697530352\\
71.75	0.15156	-40.4226306345285	-40.4226306345285\\
71.75	0.15522	-43.0324232076715	-43.0324232076715\\
71.75	0.15888	-45.7644474724644	-45.7644474724644\\
71.75	0.16254	-48.618703428907	-48.618703428907\\
71.75	0.1662	-51.5951910769994	-51.5951910769994\\
71.75	0.16986	-54.6939104167417	-54.6939104167417\\
71.75	0.17352	-57.9148614481337	-57.9148614481337\\
71.75	0.17718	-61.2580441711755	-61.2580441711755\\
71.75	0.18084	-64.7234585858671	-64.7234585858671\\
71.75	0.1845	-68.3111046922086	-68.3111046922086\\
71.75	0.18816	-72.0209824901997	-72.0209824901997\\
71.75	0.19182	-75.8530919798407	-75.8530919798407\\
71.75	0.19548	-79.8074331611315	-79.8074331611315\\
71.75	0.19914	-83.8840060340721	-83.8840060340721\\
71.75	0.2028	-88.0828105986625	-88.0828105986625\\
71.75	0.20646	-92.4038468549027	-92.4038468549027\\
71.75	0.21012	-96.8471148027926	-96.8471148027926\\
71.75	0.21378	-101.412614442332	-101.412614442332\\
71.75	0.21744	-106.100345773522	-106.100345773522\\
71.75	0.2211	-110.910308796361	-110.910308796361\\
71.75	0.22476	-115.842503510851	-115.842503510851\\
71.75	0.22842	-120.896929916989	-120.896929916989\\
71.75	0.23208	-126.073588014778	-126.073588014778\\
71.75	0.23574	-131.372477804217	-131.372477804217\\
71.75	0.2394	-136.793599285305	-136.793599285305\\
71.75	0.24306	-142.336952458043	-142.336952458043\\
71.75	0.24672	-148.002537322431	-148.002537322431\\
71.75	0.25038	-153.790353878469	-153.790353878469\\
71.75	0.25404	-159.700402126156	-159.700402126156\\
71.75	0.2577	-165.732682065494	-165.732682065494\\
71.75	0.26136	-171.887193696481	-171.887193696481\\
71.75	0.26502	-178.163937019117	-178.163937019117\\
71.75	0.26868	-184.562912033404	-184.562912033404\\
71.75	0.27234	-191.084118739341	-191.084118739341\\
71.75	0.276	-197.727557136927	-197.727557136927\\
72.125	0.093	-15.6001743802378	-15.6001743802378\\
72.125	0.09666	-16.2570387067661	-16.2570387067661\\
72.125	0.10032	-17.0361347249443	-17.0361347249443\\
72.125	0.10398	-17.9374624347722	-17.9374624347722\\
72.125	0.10764	-18.9610218362499	-18.9610218362499\\
72.125	0.1113	-20.1068129293775	-20.1068129293775\\
72.125	0.11496	-21.3748357141548	-21.3748357141548\\
72.125	0.11862	-22.7650901905819	-22.7650901905819\\
72.125	0.12228	-24.2775763586589	-24.2775763586589\\
72.125	0.12594	-25.9122942183856	-25.9122942183856\\
72.125	0.1296	-27.669243769762	-27.669243769762\\
72.125	0.13326	-29.5484250127883	-29.5484250127883\\
72.125	0.13692	-31.5498379474644	-31.5498379474644\\
72.125	0.14058	-33.6734825737903	-33.6734825737903\\
72.125	0.14424	-35.919358891766	-35.919358891766\\
72.125	0.1479	-38.2874669013915	-38.2874669013915\\
72.125	0.15156	-40.7778066026668	-40.7778066026668\\
72.125	0.15522	-43.3903779955918	-43.3903779955918\\
72.125	0.15888	-46.1251810801667	-46.1251810801667\\
72.125	0.16254	-48.9822158563913	-48.9822158563913\\
72.125	0.1662	-51.9614823242658	-51.9614823242658\\
72.125	0.16986	-55.06298048379	-55.06298048379\\
72.125	0.17352	-58.2867103349641	-58.2867103349641\\
72.125	0.17718	-61.6326718777879	-61.6326718777879\\
72.125	0.18084	-65.1008651122615	-65.1008651122615\\
72.125	0.1845	-68.6912900383849	-68.6912900383849\\
72.125	0.18816	-72.4039466561581	-72.4039466561581\\
72.125	0.19182	-76.2388349655812	-76.2388349655812\\
72.125	0.19548	-80.1959549666539	-80.1959549666539\\
72.125	0.19914	-84.2753066593765	-84.2753066593765\\
72.125	0.2028	-88.4768900437489	-88.4768900437489\\
72.125	0.20646	-92.8007051197711	-92.8007051197711\\
72.125	0.21012	-97.2467518874431	-97.2467518874431\\
72.125	0.21378	-101.815030346765	-101.815030346765\\
72.125	0.21744	-106.505540497736	-106.505540497736\\
72.125	0.2211	-111.318282340358	-111.318282340358\\
72.125	0.22476	-116.253255874629	-116.253255874629\\
72.125	0.22842	-121.31046110055	-121.31046110055\\
72.125	0.23208	-126.489898018121	-126.489898018121\\
72.125	0.23574	-131.791566627341	-131.791566627341\\
72.125	0.2394	-137.215466928211	-137.215466928211\\
72.125	0.24306	-142.761598920732	-142.761598920732\\
72.125	0.24672	-148.429962604902	-148.429962604902\\
72.125	0.25038	-154.220557980721	-154.220557980721\\
72.125	0.25404	-160.133385048191	-160.133385048191\\
72.125	0.2577	-166.16844380731	-166.16844380731\\
72.125	0.26136	-172.325734258079	-172.325734258079\\
72.125	0.26502	-178.605256400498	-178.605256400498\\
72.125	0.26868	-185.007010234567	-185.007010234567\\
72.125	0.27234	-191.530995760285	-191.530995760285\\
72.125	0.276	-198.177212977653	-198.177212977653\\
72.5	0.093	-15.9204856067813	-15.9204856067813\\
72.5	0.09666	-16.5801287530917	-16.5801287530917\\
72.5	0.10032	-17.3620035910519	-17.3620035910519\\
72.5	0.10398	-18.2661101206618	-18.2661101206618\\
72.5	0.10764	-19.2924483419215	-19.2924483419215\\
72.5	0.1113	-20.4410182548311	-20.4410182548311\\
72.5	0.11496	-21.7118198593904	-21.7118198593904\\
72.5	0.11862	-23.1048531555996	-23.1048531555996\\
72.5	0.12228	-24.6201181434585	-24.6201181434585\\
72.5	0.12594	-26.2576148229672	-26.2576148229672\\
72.5	0.1296	-28.0173431941257	-28.0173431941257\\
72.5	0.13326	-29.899303256934	-29.899303256934\\
72.5	0.13692	-31.9034950113921	-31.9034950113921\\
72.5	0.14058	-34.0299184575	-34.0299184575\\
72.5	0.14424	-36.2785735952577	-36.2785735952577\\
72.5	0.1479	-38.6494604246652	-38.6494604246652\\
72.5	0.15156	-41.1425789457224	-41.1425789457224\\
72.5	0.15522	-43.7579291584295	-43.7579291584295\\
72.5	0.15888	-46.4955110627864	-46.4955110627864\\
72.5	0.16254	-49.355324658793	-49.355324658793\\
72.5	0.1662	-52.3373699464495	-52.3373699464495\\
72.5	0.16986	-55.4416469257558	-55.4416469257558\\
72.5	0.17352	-58.6681555967118	-58.6681555967118\\
72.5	0.17718	-62.0168959593176	-62.0168959593176\\
72.5	0.18084	-65.4878680135732	-65.4878680135732\\
72.5	0.1845	-69.0810717594787	-69.0810717594787\\
72.5	0.18816	-72.7965071970339	-72.7965071970339\\
72.5	0.19182	-76.6341743262389	-76.6341743262389\\
72.5	0.19548	-80.5940731470937	-80.5940731470937\\
72.5	0.19914	-84.6762036595983	-84.6762036595983\\
72.5	0.2028	-88.8805658637528	-88.8805658637528\\
72.5	0.20646	-93.207159759557	-93.207159759557\\
72.5	0.21012	-97.655985347011	-97.655985347011\\
72.5	0.21378	-102.227042626115	-102.227042626115\\
72.5	0.21744	-106.920331596868	-106.920331596868\\
72.5	0.2211	-111.735852259272	-111.735852259272\\
72.5	0.22476	-116.673604613325	-116.673604613325\\
72.5	0.22842	-121.733588659028	-121.733588659028\\
72.5	0.23208	-126.91580439638	-126.91580439638\\
72.5	0.23574	-132.220251825383	-132.220251825383\\
72.5	0.2394	-137.646930946035	-137.646930946035\\
72.5	0.24306	-143.195841758337	-143.195841758337\\
72.5	0.24672	-148.866984262289	-148.866984262289\\
72.5	0.25038	-154.660358457891	-154.660358457891\\
72.5	0.25404	-160.575964345143	-160.575964345143\\
72.5	0.2577	-166.613801924044	-166.613801924044\\
72.5	0.26136	-172.773871194595	-172.773871194595\\
72.5	0.26502	-179.056172156796	-179.056172156796\\
72.5	0.26868	-185.460704810647	-185.460704810647\\
72.5	0.27234	-191.987469156147	-191.987469156147\\
72.5	0.276	-198.636465193297	-198.636465193297\\
72.875	0.093	-16.2503932082425	-16.2503932082425\\
72.875	0.09666	-16.9128151743348	-16.9128151743348\\
72.875	0.10032	-17.697468832077	-17.697468832077\\
72.875	0.10398	-18.604354181469	-18.604354181469\\
72.875	0.10764	-19.6334712225107	-19.6334712225107\\
72.875	0.1113	-20.7848199552023	-20.7848199552023\\
72.875	0.11496	-22.0584003795436	-22.0584003795436\\
72.875	0.11862	-23.4542124955347	-23.4542124955347\\
72.875	0.12228	-24.9722563031757	-24.9722563031757\\
72.875	0.12594	-26.6125318024664	-26.6125318024664\\
72.875	0.1296	-28.3750389934069	-28.3750389934069\\
72.875	0.13326	-30.2597778759972	-30.2597778759972\\
72.875	0.13692	-32.2667484502373	-32.2667484502373\\
72.875	0.14058	-34.3959507161272	-34.3959507161272\\
72.875	0.14424	-36.647384673667	-36.647384673667\\
72.875	0.1479	-39.0210503228564	-39.0210503228564\\
72.875	0.15156	-41.5169476636957	-41.5169476636957\\
72.875	0.15522	-44.1350766961848	-44.1350766961848\\
72.875	0.15888	-46.8754374203237	-46.8754374203237\\
72.875	0.16254	-49.7380298361124	-49.7380298361124\\
72.875	0.1662	-52.7228539435508	-52.7228539435508\\
72.875	0.16986	-55.8299097426391	-55.8299097426391\\
72.875	0.17352	-59.0591972333771	-59.0591972333771\\
72.875	0.17718	-62.410716415765	-62.410716415765\\
72.875	0.18084	-65.8844672898026	-65.8844672898026\\
72.875	0.1845	-69.48044985549	-69.48044985549\\
72.875	0.18816	-73.1986641128273	-73.1986641128273\\
72.875	0.19182	-77.0391100618143	-77.0391100618143\\
72.875	0.19548	-81.0017877024511	-81.0017877024511\\
72.875	0.19914	-85.0866970347377	-85.0866970347377\\
72.875	0.2028	-89.2938380586742	-89.2938380586742\\
72.875	0.20646	-93.6232107742604	-93.6232107742604\\
72.875	0.21012	-98.0748151814963	-98.0748151814963\\
72.875	0.21378	-102.648651280382	-102.648651280382\\
72.875	0.21744	-107.344719070918	-107.344719070918\\
72.875	0.2211	-112.163018553103	-112.163018553103\\
72.875	0.22476	-117.103549726938	-117.103549726938\\
72.875	0.22842	-122.166312592423	-122.166312592423\\
72.875	0.23208	-127.351307149558	-127.351307149558\\
72.875	0.23574	-132.658533398343	-132.658533398343\\
72.875	0.2394	-138.087991338777	-138.087991338777\\
72.875	0.24306	-143.639680970861	-143.639680970861\\
72.875	0.24672	-149.313602294595	-149.313602294595\\
72.875	0.25038	-155.109755309979	-155.109755309979\\
72.875	0.25404	-161.028140017012	-161.028140017012\\
72.875	0.2577	-167.068756415696	-167.068756415696\\
72.875	0.26136	-173.231604506029	-173.231604506029\\
72.875	0.26502	-179.516684288012	-179.516684288012\\
72.875	0.26868	-185.923995761644	-185.923995761644\\
72.875	0.27234	-192.453538926927	-192.453538926927\\
72.875	0.276	-199.105313783859	-199.105313783859\\
73.25	0.093	-16.5898971846211	-16.5898971846211\\
73.25	0.09666	-17.2550979704954	-17.2550979704954\\
73.25	0.10032	-18.0425304480196	-18.0425304480196\\
73.25	0.10398	-18.9521946171936	-18.9521946171936\\
73.25	0.10764	-19.9840904780173	-19.9840904780173\\
73.25	0.1113	-21.1382180304909	-21.1382180304909\\
73.25	0.11496	-22.4145772746142	-22.4145772746142\\
73.25	0.11862	-23.8131682103874	-23.8131682103874\\
73.25	0.12228	-25.3339908378104	-25.3339908378104\\
73.25	0.12594	-26.9770451568831	-26.9770451568831\\
73.25	0.1296	-28.7423311676056	-28.7423311676056\\
73.25	0.13326	-30.6298488699779	-30.6298488699779\\
73.25	0.13692	-32.6395982640001	-32.6395982640001\\
73.25	0.14058	-34.771579349672	-34.771579349672\\
73.25	0.14424	-37.0257921269937	-37.0257921269937\\
73.25	0.1479	-39.4022365959652	-39.4022365959652\\
73.25	0.15156	-41.9009127565865	-41.9009127565865\\
73.25	0.15522	-44.5218206088576	-44.5218206088576\\
73.25	0.15888	-47.2649601527785	-47.2649601527785\\
73.25	0.16254	-50.1303313883491	-50.1303313883491\\
73.25	0.1662	-53.1179343155696	-53.1179343155696\\
73.25	0.16986	-56.2277689344399	-56.2277689344399\\
73.25	0.17352	-59.4598352449599	-59.4598352449599\\
73.25	0.17718	-62.8141332471298	-62.8141332471298\\
73.25	0.18084	-66.2906629409494	-66.2906629409494\\
73.25	0.1845	-69.8894243264188	-69.8894243264188\\
73.25	0.18816	-73.6104174035381	-73.6104174035381\\
73.25	0.19182	-77.4536421723072	-77.4536421723072\\
73.25	0.19548	-81.419098632726	-81.419098632726\\
73.25	0.19914	-85.5067867847946	-85.5067867847946\\
73.25	0.2028	-89.716706628513	-89.716706628513\\
73.25	0.20646	-94.0488581638812	-94.0488581638812\\
73.25	0.21012	-98.5032413908993	-98.5032413908993\\
73.25	0.21378	-103.079856309567	-103.079856309567\\
73.25	0.21744	-107.778702919885	-107.778702919885\\
73.25	0.2211	-112.599781221852	-112.599781221852\\
73.25	0.22476	-117.543091215469	-117.543091215469\\
73.25	0.22842	-122.608632900736	-122.608632900736\\
73.25	0.23208	-127.796406277653	-127.796406277653\\
73.25	0.23574	-133.10641134622	-133.10641134622\\
73.25	0.2394	-138.538648106436	-138.538648106436\\
73.25	0.24306	-144.093116558302	-144.093116558302\\
73.25	0.24672	-149.769816701818	-149.769816701818\\
73.25	0.25038	-155.568748536984	-155.568748536984\\
73.25	0.25404	-161.489912063799	-161.489912063799\\
73.25	0.2577	-167.533307282265	-167.533307282265\\
73.25	0.26136	-173.69893419238	-173.69893419238\\
73.25	0.26502	-179.986792794145	-179.986792794145\\
73.25	0.26868	-186.396883087559	-186.396883087559\\
73.25	0.27234	-192.929205072624	-192.929205072624\\
73.25	0.276	-199.583758749338	-199.583758749338\\
73.625	0.093	-16.9389975359171	-16.9389975359171\\
73.625	0.09666	-17.6069771415735	-17.6069771415735\\
73.625	0.10032	-18.3971884388797	-18.3971884388797\\
73.625	0.10398	-19.3096314278356	-19.3096314278356\\
73.625	0.10764	-20.3443061084414	-20.3443061084414\\
73.625	0.1113	-21.501212480697	-21.501212480697\\
73.625	0.11496	-22.7803505446023	-22.7803505446023\\
73.625	0.11862	-24.1817203001575	-24.1817203001575\\
73.625	0.12228	-25.7053217473625	-25.7053217473625\\
73.625	0.12594	-27.3511548862172	-27.3511548862172\\
73.625	0.1296	-29.1192197167217	-29.1192197167217\\
73.625	0.13326	-31.0095162388761	-31.0095162388761\\
73.625	0.13692	-33.0220444526802	-33.0220444526802\\
73.625	0.14058	-35.1568043581341	-35.1568043581341\\
73.625	0.14424	-37.4137959552378	-37.4137959552378\\
73.625	0.1479	-39.7930192439913	-39.7930192439913\\
73.625	0.15156	-42.2944742243947	-42.2944742243947\\
73.625	0.15522	-44.9181608964478	-44.9181608964478\\
73.625	0.15888	-47.6640792601506	-47.6640792601506\\
73.625	0.16254	-50.5322293155033	-50.5322293155033\\
73.625	0.1662	-53.5226110625058	-53.5226110625058\\
73.625	0.16986	-56.6352245011581	-56.6352245011581\\
73.625	0.17352	-59.8700696314602	-59.8700696314602\\
73.625	0.17718	-63.227146453412	-63.227146453412\\
73.625	0.18084	-66.7064549670137	-66.7064549670137\\
73.625	0.1845	-70.3079951722651	-70.3079951722651\\
73.625	0.18816	-74.0317670691664	-74.0317670691664\\
73.625	0.19182	-77.8777706577174	-77.8777706577174\\
73.625	0.19548	-81.8460059379182	-81.8460059379182\\
73.625	0.19914	-85.9364729097689	-85.9364729097689\\
73.625	0.2028	-90.1491715732693	-90.1491715732693\\
73.625	0.20646	-94.4841019284196	-94.4841019284196\\
73.625	0.21012	-98.9412639752196	-98.9412639752196\\
73.625	0.21378	-103.520657713669	-103.520657713669\\
73.625	0.21744	-108.222283143769	-108.222283143769\\
73.625	0.2211	-113.046140265518	-113.046140265518\\
73.625	0.22476	-117.992229078917	-117.992229078917\\
73.625	0.22842	-123.060549583967	-123.060549583967\\
73.625	0.23208	-128.251101780665	-128.251101780665\\
73.625	0.23574	-133.563885669014	-133.563885669014\\
73.625	0.2394	-138.998901249012	-138.998901249012\\
73.625	0.24306	-144.55614852066	-144.55614852066\\
73.625	0.24672	-150.235627483958	-150.235627483958\\
73.625	0.25038	-156.037338138906	-156.037338138906\\
73.625	0.25404	-161.961280485504	-161.961280485504\\
73.625	0.2577	-168.007454523751	-168.007454523751\\
73.625	0.26136	-174.175860253648	-174.175860253648\\
73.625	0.26502	-180.466497675195	-180.466497675195\\
73.625	0.26868	-186.879366788392	-186.879366788392\\
73.625	0.27234	-193.414467593238	-193.414467593238\\
73.625	0.276	-200.071800089735	-200.071800089735\\
74	0.093	-17.2976942621307	-17.2976942621307\\
74	0.09666	-17.9684526875691	-17.9684526875691\\
74	0.10032	-18.7614428046573	-18.7614428046573\\
74	0.10398	-19.6766646133953	-19.6766646133953\\
74	0.10764	-20.714118113783	-20.714118113783\\
74	0.1113	-21.8738033058206	-21.8738033058206\\
74	0.11496	-23.155720189508	-23.155720189508\\
74	0.11862	-24.5598687648451	-24.5598687648451\\
74	0.12228	-26.0862490318321	-26.0862490318321\\
74	0.12594	-27.7348609904689	-27.7348609904689\\
74	0.1296	-29.5057046407554	-29.5057046407554\\
74	0.13326	-31.3987799826917	-31.3987799826917\\
74	0.13692	-33.4140870162778	-33.4140870162778\\
74	0.14058	-35.5516257415138	-35.5516257415138\\
74	0.14424	-37.8113961583996	-37.8113961583996\\
74	0.1479	-40.193398266935	-40.193398266935\\
74	0.15156	-42.6976320671204	-42.6976320671204\\
74	0.15522	-45.3240975589554	-45.3240975589554\\
74	0.15888	-48.0727947424404	-48.0727947424404\\
74	0.16254	-50.9437236175751	-50.9437236175751\\
74	0.1662	-53.9368841843596	-53.9368841843596\\
74	0.16986	-57.0522764427938	-57.0522764427938\\
74	0.17352	-60.2899003928779	-60.2899003928779\\
74	0.17718	-63.6497560346118	-63.6497560346118\\
74	0.18084	-67.1318433679955	-67.1318433679955\\
74	0.1845	-70.7361623930289	-70.7361623930289\\
74	0.18816	-74.4627131097122	-74.4627131097122\\
74	0.19182	-78.3114955180452	-78.3114955180452\\
74	0.19548	-82.2825096180281	-82.2825096180281\\
74	0.19914	-86.3757554096607	-86.3757554096607\\
74	0.2028	-90.5912328929432	-90.5912328929432\\
74	0.20646	-94.9289420678754	-94.9289420678754\\
74	0.21012	-99.3888829344574	-99.3888829344574\\
74	0.21378	-103.971055492689	-103.971055492689\\
74	0.21744	-108.675459742571	-108.675459742571\\
74	0.2211	-113.502095684102	-113.502095684102\\
74	0.22476	-118.450963317283	-118.450963317283\\
74	0.22842	-123.522062642114	-123.522062642114\\
74	0.23208	-128.715393658595	-128.715393658595\\
74	0.23574	-134.030956366726	-134.030956366726\\
74	0.2394	-139.468750766506	-139.468750766506\\
74	0.24306	-145.028776857936	-145.028776857936\\
74	0.24672	-150.711034641016	-150.711034641016\\
74	0.25038	-156.515524115746	-156.515524115746\\
74	0.25404	-162.442245282126	-162.442245282126\\
74	0.2577	-168.491198140155	-168.491198140155\\
74	0.26136	-174.662382689834	-174.662382689834\\
74	0.26502	-180.955798931163	-180.955798931163\\
74	0.26868	-187.371446864142	-187.371446864142\\
74	0.27234	-193.90932648877	-193.90932648877\\
74	0.276	-200.569437805049	-200.569437805049\\
};\label{tikz:theta_surf}
\end{axis}
\end{tikzpicture}%
%	\end{minipage}}
%	% This file was created by matlab2tikz.
%
\definecolor{mycolor1}{rgb}{0.00000,0.44700,0.74100}%
\definecolor{mycolor2}{rgb}{0.85000,0.32500,0.09800}%
%
\begin{tikzpicture}

\begin{axis}[%
width=6.159cm,
height=3.097cm,
at={(0cm,12.903cm)},
scale only axis,
xmin=56,
xmax=74,
tick align=outside,
xlabel style={font=\color{white!15!black}},
xlabel={$L_{cut}$},
ymin=0.093,
ymax=0.276,
ylabel style={font=\color{white!15!black}},
ylabel={$D_{rlx}$},
zmin=-200.569447858151,
zmax=0,
zlabel style={font=\color{white!15!black}},
zlabel={$u(t)u(t)$},
view={-140}{50},
axis background/.style={fill=white},
xmajorgrids,
ymajorgrids,
zmajorgrids
]
\addplot3[only marks, mark=*, mark options={}, mark size=1.5000pt, color=mycolor1, fill=mycolor1] table[row sep=crcr]{%
x	y	z\\
74	0.123	-26.0353957891804\\
72	0.113	-20.9879322169279\\
61	0.095	-10.6920630070547\\
56	0.093	-10.9569379219045\\
};
\addplot3[only marks, mark=*, mark options={}, mark size=1.5000pt, color=mycolor2, fill=mycolor2] table[row sep=crcr]{%
x	y	z\\
67	0.276	-191.779551108501\\
66	0.255	-157.643535964989\\
62	0.209	-87.4196213968237\\
57	0.193	-69.8013569503948\\
};
\addplot3[only marks, mark=*, mark options={}, mark size=1.5000pt, color=black, fill=black] table[row sep=crcr]{%
x	y	z\\
69	0.104	-15.5770567955152\\
};
\addplot3[only marks, mark=*, mark options={}, mark size=1.5000pt, color=black, fill=black] table[row sep=crcr]{%
x	y	z\\
64	0.23	-116.694088634917\\
};

\addplot3[%
surf,
fill opacity=0.7, shader=interp, colormap={mymap}{[1pt] rgb(0pt)=(1,0.905882,0); rgb(1pt)=(1,0.901964,0); rgb(2pt)=(1,0.898051,0); rgb(3pt)=(1,0.894144,0); rgb(4pt)=(1,0.890243,0); rgb(5pt)=(1,0.886349,0); rgb(6pt)=(1,0.88246,0); rgb(7pt)=(1,0.878577,0); rgb(8pt)=(1,0.8747,0); rgb(9pt)=(1,0.870829,0); rgb(10pt)=(1,0.866964,0); rgb(11pt)=(1,0.863106,0); rgb(12pt)=(1,0.859253,0); rgb(13pt)=(1,0.855406,0); rgb(14pt)=(1,0.851566,0); rgb(15pt)=(1,0.847732,0); rgb(16pt)=(1,0.843903,0); rgb(17pt)=(1,0.840081,0); rgb(18pt)=(1,0.836265,0); rgb(19pt)=(1,0.832455,0); rgb(20pt)=(1,0.828652,0); rgb(21pt)=(1,0.824854,0); rgb(22pt)=(1,0.821063,0); rgb(23pt)=(1,0.817278,0); rgb(24pt)=(1,0.8135,0); rgb(25pt)=(1,0.809727,0); rgb(26pt)=(1,0.805961,0); rgb(27pt)=(1,0.8022,0); rgb(28pt)=(1,0.798445,0); rgb(29pt)=(1,0.794696,0); rgb(30pt)=(1,0.790953,0); rgb(31pt)=(1,0.787215,0); rgb(32pt)=(1,0.783484,0); rgb(33pt)=(1,0.779758,0); rgb(34pt)=(1,0.776038,0); rgb(35pt)=(1,0.772324,0); rgb(36pt)=(1,0.768615,0); rgb(37pt)=(1,0.764913,0); rgb(38pt)=(1,0.761217,0); rgb(39pt)=(1,0.757527,0); rgb(40pt)=(1,0.753843,0); rgb(41pt)=(1,0.750165,0); rgb(42pt)=(1,0.746493,0); rgb(43pt)=(1,0.742827,0); rgb(44pt)=(1,0.739167,0); rgb(45pt)=(1,0.735514,0); rgb(46pt)=(1,0.731867,0); rgb(47pt)=(1,0.728226,0); rgb(48pt)=(1,0.724591,0); rgb(49pt)=(1,0.720963,0); rgb(50pt)=(1,0.717341,0); rgb(51pt)=(1,0.713725,0); rgb(52pt)=(0.999994,0.710077,0); rgb(53pt)=(0.999974,0.706363,0); rgb(54pt)=(0.999942,0.702592,0); rgb(55pt)=(0.999898,0.698775,0); rgb(56pt)=(0.999841,0.694921,0); rgb(57pt)=(0.999771,0.691039,0); rgb(58pt)=(0.99969,0.687139,0); rgb(59pt)=(0.999596,0.68323,0); rgb(60pt)=(0.99949,0.679323,0); rgb(61pt)=(0.999372,0.675427,0); rgb(62pt)=(0.999242,0.67155,0); rgb(63pt)=(0.9991,0.667704,0); rgb(64pt)=(0.998946,0.663897,0); rgb(65pt)=(0.998781,0.660138,0); rgb(66pt)=(0.998605,0.656439,0); rgb(67pt)=(0.998416,0.652807,0); rgb(68pt)=(0.998217,0.649253,0); rgb(69pt)=(0.998006,0.645786,0); rgb(70pt)=(0.997785,0.642416,0); rgb(71pt)=(0.997552,0.639152,0); rgb(72pt)=(0.997308,0.636004,0); rgb(73pt)=(0.997053,0.632982,0); rgb(74pt)=(0.996788,0.630095,0); rgb(75pt)=(0.996512,0.627352,0); rgb(76pt)=(0.996226,0.624763,0); rgb(77pt)=(0.995851,0.622329,0); rgb(78pt)=(0.99494,0.619997,0); rgb(79pt)=(0.99345,0.617753,0); rgb(80pt)=(0.991419,0.61559,0); rgb(81pt)=(0.988885,0.613503,0); rgb(82pt)=(0.985886,0.611486,0); rgb(83pt)=(0.98246,0.609532,0); rgb(84pt)=(0.978643,0.607636,0); rgb(85pt)=(0.974475,0.605791,0); rgb(86pt)=(0.969992,0.603992,0); rgb(87pt)=(0.965232,0.602233,0); rgb(88pt)=(0.960233,0.600507,0); rgb(89pt)=(0.955033,0.598808,0); rgb(90pt)=(0.949669,0.59713,0); rgb(91pt)=(0.94418,0.595468,0); rgb(92pt)=(0.938602,0.593815,0); rgb(93pt)=(0.932974,0.592166,0); rgb(94pt)=(0.927333,0.590513,0); rgb(95pt)=(0.921717,0.588852,0); rgb(96pt)=(0.916164,0.587176,0); rgb(97pt)=(0.910711,0.585479,0); rgb(98pt)=(0.905397,0.583755,0); rgb(99pt)=(0.900258,0.581999,0); rgb(100pt)=(0.895333,0.580203,0); rgb(101pt)=(0.890659,0.578362,0); rgb(102pt)=(0.886275,0.576471,0); rgb(103pt)=(0.882047,0.574545,0); rgb(104pt)=(0.877819,0.572608,0); rgb(105pt)=(0.873592,0.57066,0); rgb(106pt)=(0.869366,0.568701,0); rgb(107pt)=(0.865143,0.566733,0); rgb(108pt)=(0.860924,0.564756,0); rgb(109pt)=(0.856708,0.562771,0); rgb(110pt)=(0.852497,0.560778,0); rgb(111pt)=(0.848292,0.558779,0); rgb(112pt)=(0.844092,0.556774,0); rgb(113pt)=(0.8399,0.554763,0); rgb(114pt)=(0.835716,0.552749,0); rgb(115pt)=(0.831541,0.55073,0); rgb(116pt)=(0.827374,0.548709,0); rgb(117pt)=(0.823219,0.546686,0); rgb(118pt)=(0.819074,0.54466,0); rgb(119pt)=(0.81494,0.542635,0); rgb(120pt)=(0.81082,0.540609,0); rgb(121pt)=(0.806712,0.538584,0); rgb(122pt)=(0.802619,0.53656,0); rgb(123pt)=(0.798541,0.534539,0); rgb(124pt)=(0.794478,0.532521,0); rgb(125pt)=(0.790431,0.530506,0); rgb(126pt)=(0.786402,0.528496,0); rgb(127pt)=(0.782391,0.526491,0); rgb(128pt)=(0.77841,0.524489,0); rgb(129pt)=(0.774523,0.522478,0); rgb(130pt)=(0.770731,0.520455,0); rgb(131pt)=(0.767022,0.518424,0); rgb(132pt)=(0.763384,0.516385,0); rgb(133pt)=(0.759804,0.514339,0); rgb(134pt)=(0.756272,0.51229,0); rgb(135pt)=(0.752775,0.510237,0); rgb(136pt)=(0.749302,0.508182,0); rgb(137pt)=(0.74584,0.506128,0); rgb(138pt)=(0.742378,0.504075,0); rgb(139pt)=(0.738904,0.502025,0); rgb(140pt)=(0.735406,0.499979,0); rgb(141pt)=(0.731872,0.49794,0); rgb(142pt)=(0.72829,0.495909,0); rgb(143pt)=(0.724649,0.493887,0); rgb(144pt)=(0.720936,0.491875,0); rgb(145pt)=(0.71714,0.489876,0); rgb(146pt)=(0.713249,0.487891,0); rgb(147pt)=(0.709251,0.485921,0); rgb(148pt)=(0.705134,0.483968,0); rgb(149pt)=(0.700887,0.482033,0); rgb(150pt)=(0.696497,0.480118,0); rgb(151pt)=(0.691952,0.478225,0); rgb(152pt)=(0.687242,0.476355,0); rgb(153pt)=(0.682353,0.47451,0); rgb(154pt)=(0.677195,0.472696,0); rgb(155pt)=(0.6717,0.470916,0); rgb(156pt)=(0.665891,0.469169,0); rgb(157pt)=(0.659791,0.46745,0); rgb(158pt)=(0.653423,0.465756,0); rgb(159pt)=(0.64681,0.464084,0); rgb(160pt)=(0.639976,0.462432,0); rgb(161pt)=(0.632943,0.460795,0); rgb(162pt)=(0.625734,0.459171,0); rgb(163pt)=(0.618373,0.457556,0); rgb(164pt)=(0.610882,0.455948,0); rgb(165pt)=(0.603284,0.454343,0); rgb(166pt)=(0.595604,0.452737,0); rgb(167pt)=(0.587863,0.451129,0); rgb(168pt)=(0.580084,0.449514,0); rgb(169pt)=(0.572292,0.447889,0); rgb(170pt)=(0.564508,0.446252,0); rgb(171pt)=(0.556756,0.444599,0); rgb(172pt)=(0.549059,0.442927,0); rgb(173pt)=(0.54144,0.441232,0); rgb(174pt)=(0.533922,0.439512,0); rgb(175pt)=(0.526529,0.437764,0); rgb(176pt)=(0.519282,0.435983,0); rgb(177pt)=(0.512206,0.434168,0); rgb(178pt)=(0.505323,0.432315,0); rgb(179pt)=(0.498628,0.430422,3.92506e-06); rgb(180pt)=(0.491973,0.428504,3.49981e-05); rgb(181pt)=(0.485331,0.426562,9.63073e-05); rgb(182pt)=(0.478704,0.424596,0.000186979); rgb(183pt)=(0.472096,0.422609,0.000306141); rgb(184pt)=(0.465508,0.420599,0.00045292); rgb(185pt)=(0.458942,0.418567,0.000626441); rgb(186pt)=(0.452401,0.416515,0.000825833); rgb(187pt)=(0.445885,0.414441,0.00105022); rgb(188pt)=(0.439399,0.412348,0.00129873); rgb(189pt)=(0.432942,0.410234,0.00157049); rgb(190pt)=(0.426518,0.408102,0.00186463); rgb(191pt)=(0.420129,0.40595,0.00218028); rgb(192pt)=(0.413777,0.40378,0.00251655); rgb(193pt)=(0.407464,0.401592,0.00287258); rgb(194pt)=(0.401191,0.399386,0.00324749); rgb(195pt)=(0.394962,0.397164,0.00364042); rgb(196pt)=(0.388777,0.394925,0.00405048); rgb(197pt)=(0.38264,0.39267,0.00447681); rgb(198pt)=(0.376552,0.390399,0.00491852); rgb(199pt)=(0.370516,0.388113,0.00537476); rgb(200pt)=(0.364532,0.385812,0.00584464); rgb(201pt)=(0.358605,0.383497,0.00632729); rgb(202pt)=(0.352735,0.381168,0.00682184); rgb(203pt)=(0.346925,0.378826,0.00732741); rgb(204pt)=(0.341176,0.376471,0.00784314); rgb(205pt)=(0.335485,0.374093,0.00847245); rgb(206pt)=(0.329843,0.371682,0.00930909); rgb(207pt)=(0.324249,0.369242,0.0103377); rgb(208pt)=(0.318701,0.366772,0.0115428); rgb(209pt)=(0.313198,0.364275,0.0129091); rgb(210pt)=(0.307739,0.361753,0.0144211); rgb(211pt)=(0.302322,0.359206,0.0160634); rgb(212pt)=(0.296945,0.356637,0.0178207); rgb(213pt)=(0.291607,0.354048,0.0196776); rgb(214pt)=(0.286307,0.35144,0.0216186); rgb(215pt)=(0.281043,0.348814,0.0236284); rgb(216pt)=(0.275813,0.346172,0.0256916); rgb(217pt)=(0.270616,0.343517,0.0277927); rgb(218pt)=(0.265451,0.340849,0.0299163); rgb(219pt)=(0.260317,0.33817,0.0320472); rgb(220pt)=(0.25521,0.335482,0.0341698); rgb(221pt)=(0.250131,0.332786,0.0362688); rgb(222pt)=(0.245078,0.330085,0.0383287); rgb(223pt)=(0.240048,0.327379,0.0403343); rgb(224pt)=(0.235042,0.324671,0.04227); rgb(225pt)=(0.230056,0.321962,0.0441205); rgb(226pt)=(0.22509,0.319254,0.0458704); rgb(227pt)=(0.220142,0.316548,0.0475043); rgb(228pt)=(0.215212,0.313846,0.0490067); rgb(229pt)=(0.210296,0.311149,0.0503624); rgb(230pt)=(0.205395,0.308459,0.0515759); rgb(231pt)=(0.200514,0.305763,0.052757); rgb(232pt)=(0.195655,0.303061,0.0539242); rgb(233pt)=(0.190817,0.300353,0.0550763); rgb(234pt)=(0.186001,0.297639,0.0562123); rgb(235pt)=(0.181207,0.294918,0.0573313); rgb(236pt)=(0.176434,0.292191,0.0584321); rgb(237pt)=(0.171685,0.289458,0.0595136); rgb(238pt)=(0.166957,0.286719,0.060575); rgb(239pt)=(0.162252,0.283973,0.0616151); rgb(240pt)=(0.15757,0.281221,0.0626328); rgb(241pt)=(0.152911,0.278463,0.0636271); rgb(242pt)=(0.148275,0.275699,0.0645971); rgb(243pt)=(0.143663,0.272929,0.0655416); rgb(244pt)=(0.139074,0.270152,0.0664596); rgb(245pt)=(0.134508,0.26737,0.06735); rgb(246pt)=(0.129967,0.264581,0.0682118); rgb(247pt)=(0.125449,0.261787,0.0690441); rgb(248pt)=(0.120956,0.258986,0.0698456); rgb(249pt)=(0.116487,0.25618,0.0706154); rgb(250pt)=(0.112043,0.253367,0.0713525); rgb(251pt)=(0.107623,0.250549,0.0720557); rgb(252pt)=(0.103229,0.247724,0.0727241); rgb(253pt)=(0.0988592,0.244894,0.0733566); rgb(254pt)=(0.0945149,0.242058,0.0739522); rgb(255pt)=(0.0901961,0.239216,0.0745098)}, mesh/rows=49]
table[row sep=crcr, point meta=\thisrow{c}] {%
%
x	y	z	c\\
56	0.093	-10.9049629825366	-10.9049629825366\\
56	0.09666	-11.4423383223269	-11.4423383223269\\
56	0.10032	-12.1019453259281	-12.1019453259281\\
56	0.10398	-12.8837839933399	-12.8837839933399\\
56	0.10764	-13.7878543245625	-13.7878543245625\\
56	0.1113	-14.8141563195959	-14.8141563195959\\
56	0.11496	-15.9626899784401	-15.9626899784401\\
56	0.11862	-17.233455301095	-17.233455301095\\
56	0.12228	-18.6264522875606	-18.6264522875606\\
56	0.12594	-20.141680937837	-20.141680937837\\
56	0.1296	-21.7791412519242	-21.7791412519242\\
56	0.13326	-23.5388332298221	-23.5388332298221\\
56	0.13692	-25.4207568715308	-25.4207568715308\\
56	0.14058	-27.4249121770503	-27.4249121770503\\
56	0.14424	-29.5512991463805	-29.5512991463805\\
56	0.1479	-31.7999177795214	-31.7999177795214\\
56	0.15156	-34.1707680764732	-34.1707680764732\\
56	0.15522	-36.6638500372356	-36.6638500372356\\
56	0.15888	-39.2791636618089	-39.2791636618089\\
56	0.16254	-42.0167089501929	-42.0167089501929\\
56	0.1662	-44.8764859023877	-44.8764859023877\\
56	0.16986	-47.8584945183932	-47.8584945183932\\
56	0.17352	-50.9627347982094	-50.9627347982094\\
56	0.17718	-54.1892067418365	-54.1892067418365\\
56	0.18084	-57.5379103492742	-57.5379103492742\\
56	0.1845	-61.0088456205228	-61.0088456205228\\
56	0.18816	-64.6020125555821	-64.6020125555821\\
56	0.19182	-68.3174111544522	-68.3174111544522\\
56	0.19548	-72.155041417133	-72.155041417133\\
56	0.19914	-76.1149033436246	-76.1149033436246\\
56	0.2028	-80.196996933927	-80.196996933927\\
56	0.20646	-84.4013221880401	-84.4013221880401\\
56	0.21012	-88.7278791059639	-88.7278791059639\\
56	0.21378	-93.1766676876985	-93.1766676876985\\
56	0.21744	-97.7476879332439	-97.7476879332439\\
56	0.2211	-102.4409398426	-102.4409398426\\
56	0.22476	-107.256423415767	-107.256423415767\\
56	0.22842	-112.194138652745	-112.194138652745\\
56	0.23208	-117.254085553533	-117.254085553533\\
56	0.23574	-122.436264118132	-122.436264118132\\
56	0.2394	-127.740674346542	-127.740674346542\\
56	0.24306	-133.167316238763	-133.167316238763\\
56	0.24672	-138.716189794794	-138.716189794794\\
56	0.25038	-144.387295014636	-144.387295014636\\
56	0.25404	-150.180631898289	-150.180631898289\\
56	0.2577	-156.096200445753	-156.096200445753\\
56	0.26136	-162.134000657028	-162.134000657028\\
56	0.26502	-168.294032532113	-168.294032532113\\
56	0.26868	-174.576296071009	-174.576296071009\\
56	0.27234	-180.980791273716	-180.980791273716\\
56	0.276	-187.507518140233	-187.507518140233\\
56.375	0.093	-10.8126299877808	-10.8126299877808\\
56.375	0.09666	-11.3527841630815	-11.3527841630815\\
56.375	0.10032	-12.015170002193	-12.015170002193\\
56.375	0.10398	-12.7997875051152	-12.7997875051152\\
56.375	0.10764	-13.7066366718482	-13.7066366718482\\
56.375	0.1113	-14.7357175023919	-14.7357175023919\\
56.375	0.11496	-15.8870299967463	-15.8870299967463\\
56.375	0.11862	-17.1605741549116	-17.1605741549116\\
56.375	0.12228	-18.5563499768876	-18.5563499768876\\
56.375	0.12594	-20.0743574626744	-20.0743574626744\\
56.375	0.1296	-21.7145966122719	-21.7145966122719\\
56.375	0.13326	-23.4770674256802	-23.4770674256802\\
56.375	0.13692	-25.3617699028992	-25.3617699028992\\
56.375	0.14058	-27.368704043929	-27.368704043929\\
56.375	0.14424	-29.4978698487696	-29.4978698487696\\
56.375	0.1479	-31.7492673174209	-31.7492673174209\\
56.375	0.15156	-34.1228964498829	-34.1228964498829\\
56.375	0.15522	-36.6187572461558	-36.6187572461558\\
56.375	0.15888	-39.2368497062394	-39.2368497062394\\
56.375	0.16254	-41.9771738301337	-41.9771738301337\\
56.375	0.1662	-44.8397296178388	-44.8397296178388\\
56.375	0.16986	-47.8245170693547	-47.8245170693547\\
56.375	0.17352	-50.9315361846813	-50.9315361846813\\
56.375	0.17718	-54.1607869638187	-54.1607869638187\\
56.375	0.18084	-57.5122694067668	-57.5122694067668\\
56.375	0.1845	-60.9859835135257	-60.9859835135257\\
56.375	0.18816	-64.5819292840954	-64.5819292840954\\
56.375	0.19182	-68.3001067184758	-68.3001067184758\\
56.375	0.19548	-72.140515816667	-72.140515816667\\
56.375	0.19914	-76.1031565786689	-76.1031565786689\\
56.375	0.2028	-80.1880290044816	-80.1880290044816\\
56.375	0.20646	-84.3951330941051	-84.3951330941051\\
56.375	0.21012	-88.7244688475393	-88.7244688475393\\
56.375	0.21378	-93.1760362647842	-93.1760362647842\\
56.375	0.21744	-97.74983534584	-97.74983534584\\
56.375	0.2211	-102.445866090706	-102.445866090706\\
56.375	0.22476	-107.264128499384	-107.264128499384\\
56.375	0.22842	-112.204622571872	-112.204622571872\\
56.375	0.23208	-117.26734830817	-117.26734830817\\
56.375	0.23574	-122.45230570828	-122.45230570828\\
56.375	0.2394	-127.7594947722	-127.7594947722\\
56.375	0.24306	-133.188915499931	-133.188915499931\\
56.375	0.24672	-138.740567891473	-138.740567891473\\
56.375	0.25038	-144.414451946826	-144.414451946826\\
56.375	0.25404	-150.210567665989	-150.210567665989\\
56.375	0.2577	-156.128915048963	-156.128915048963\\
56.375	0.26136	-162.169494095748	-162.169494095748\\
56.375	0.26502	-168.332304806343	-168.332304806343\\
56.375	0.26868	-174.61734718075	-174.61734718075\\
56.375	0.27234	-181.024621218967	-181.024621218967\\
56.375	0.276	-187.554126920995	-187.554126920995\\
56.75	0.093	-10.7298933655794	-10.7298933655794\\
56.75	0.09666	-11.2728263763905	-11.2728263763905\\
56.75	0.10032	-11.9379910510123	-11.9379910510123\\
56.75	0.10398	-12.7253873894448	-12.7253873894448\\
56.75	0.10764	-13.6350153916882	-13.6350153916882\\
56.75	0.1113	-14.6668750577423	-14.6668750577423\\
56.75	0.11496	-15.8209663876071	-15.8209663876071\\
56.75	0.11862	-17.0972893812827	-17.0972893812827\\
56.75	0.12228	-18.495844038769	-18.495844038769\\
56.75	0.12594	-20.0166303600661	-20.0166303600661\\
56.75	0.1296	-21.659648345174	-21.659648345174\\
56.75	0.13326	-23.4248979940926	-23.4248979940926\\
56.75	0.13692	-25.312379306822	-25.312379306822\\
56.75	0.14058	-27.3220922833622	-27.3220922833622\\
56.75	0.14424	-29.454036923713	-29.454036923713\\
56.75	0.1479	-31.7082132278747	-31.7082132278747\\
56.75	0.15156	-34.0846211958471	-34.0846211958471\\
56.75	0.15522	-36.5832608276303	-36.5832608276303\\
56.75	0.15888	-39.2041321232243	-39.2041321232243\\
56.75	0.16254	-41.947235082629	-41.947235082629\\
56.75	0.1662	-44.8125697058444	-44.8125697058444\\
56.75	0.16986	-47.8001359928707	-47.8001359928707\\
56.75	0.17352	-50.9099339437076	-50.9099339437076\\
56.75	0.17718	-54.1419635583553	-54.1419635583553\\
56.75	0.18084	-57.4962248368138	-57.4962248368138\\
56.75	0.1845	-60.9727177790831	-60.9727177790831\\
56.75	0.18816	-64.571442385163	-64.571442385163\\
56.75	0.19182	-68.2923986550538	-68.2923986550538\\
56.75	0.19548	-72.1355865887553	-72.1355865887553\\
56.75	0.19914	-76.1010061862676	-76.1010061862676\\
56.75	0.2028	-80.1886574475907	-80.1886574475907\\
56.75	0.20646	-84.3985403727245	-84.3985403727245\\
56.75	0.21012	-88.730654961669	-88.730654961669\\
56.75	0.21378	-93.1850012144243	-93.1850012144243\\
56.75	0.21744	-97.7615791309904	-97.7615791309904\\
56.75	0.2211	-102.460388711367	-102.460388711367\\
56.75	0.22476	-107.281429955555	-107.281429955555\\
56.75	0.22842	-112.224702863553	-112.224702863553\\
56.75	0.23208	-117.290207435362	-117.290207435362\\
56.75	0.23574	-122.477943670982	-122.477943670982\\
56.75	0.2394	-127.787911570413	-127.787911570413\\
56.75	0.24306	-133.220111133654	-133.220111133654\\
56.75	0.24672	-138.774542360706	-138.774542360706\\
56.75	0.25038	-144.451205251569	-144.451205251569\\
56.75	0.25404	-150.250099806243	-150.250099806243\\
56.75	0.2577	-156.171226024727	-156.171226024727\\
56.75	0.26136	-162.214583907023	-162.214583907023\\
56.75	0.26502	-168.380173453128	-168.380173453128\\
56.75	0.26868	-174.667994663045	-174.667994663045\\
56.75	0.27234	-181.078047536773	-181.078047536773\\
56.75	0.276	-187.610332074311	-187.610332074311\\
57.125	0.093	-10.6567531159325	-10.6567531159325\\
57.125	0.09666	-11.2024649622539	-11.2024649622539\\
57.125	0.10032	-11.870408472386	-11.870408472386\\
57.125	0.10398	-12.6605836463289	-12.6605836463289\\
57.125	0.10764	-13.5729904840826	-13.5729904840826\\
57.125	0.1113	-14.607628985647	-14.607628985647\\
57.125	0.11496	-15.7644991510222	-15.7644991510222\\
57.125	0.11862	-17.0436009802082	-17.0436009802082\\
57.125	0.12228	-18.4449344732048	-18.4449344732048\\
57.125	0.12594	-19.9684996300123	-19.9684996300123\\
57.125	0.1296	-21.6142964506305	-21.6142964506305\\
57.125	0.13326	-23.3823249350595	-23.3823249350595\\
57.125	0.13692	-25.2725850832992	-25.2725850832992\\
57.125	0.14058	-27.2850768953497	-27.2850768953497\\
57.125	0.14424	-29.419800371211	-29.419800371211\\
57.125	0.1479	-31.676755510883	-31.676755510883\\
57.125	0.15156	-34.0559423143658	-34.0559423143658\\
57.125	0.15522	-36.5573607816593	-36.5573607816593\\
57.125	0.15888	-39.1810109127636	-39.1810109127636\\
57.125	0.16254	-41.9268927076786	-41.9268927076786\\
57.125	0.1662	-44.7950061664044	-44.7950061664044\\
57.125	0.16986	-47.785351288941	-47.785351288941\\
57.125	0.17352	-50.8979280752883	-50.8979280752883\\
57.125	0.17718	-54.1327365254464	-54.1327365254464\\
57.125	0.18084	-57.4897766394152	-57.4897766394152\\
57.125	0.1845	-60.9690484171948	-60.9690484171948\\
57.125	0.18816	-64.5705518587851	-64.5705518587851\\
57.125	0.19182	-68.2942869641863	-68.2942869641863\\
57.125	0.19548	-72.1402537333981	-72.1402537333981\\
57.125	0.19914	-76.1084521664207	-76.1084521664207\\
57.125	0.2028	-80.1988822632542	-80.1988822632542\\
57.125	0.20646	-84.4115440238983	-84.4115440238983\\
57.125	0.21012	-88.7464374483532	-88.7464374483532\\
57.125	0.21378	-93.2035625366189	-93.2035625366189\\
57.125	0.21744	-97.7829192886953	-97.7829192886953\\
57.125	0.2211	-102.484507704582	-102.484507704582\\
57.125	0.22476	-107.30832778428	-107.30832778428\\
57.125	0.22842	-112.254379527789	-112.254379527789\\
57.125	0.23208	-117.322662935109	-117.322662935109\\
57.125	0.23574	-122.513178006239	-122.513178006239\\
57.125	0.2394	-127.82592474118	-127.82592474118\\
57.125	0.24306	-133.260903139931	-133.260903139931\\
57.125	0.24672	-138.818113202494	-138.818113202494\\
57.125	0.25038	-144.497554928867	-144.497554928867\\
57.125	0.25404	-150.299228319051	-150.299228319051\\
57.125	0.2577	-156.223133373046	-156.223133373046\\
57.125	0.26136	-162.269270090851	-162.269270090851\\
57.125	0.26502	-168.437638472468	-168.437638472468\\
57.125	0.26868	-174.728238517895	-174.728238517895\\
57.125	0.27234	-181.141070227133	-181.141070227133\\
57.125	0.276	-187.676133600181	-187.676133600181\\
57.5	0.093	-10.5932092388399	-10.5932092388399\\
57.5	0.09666	-11.1416999206717	-11.1416999206717\\
57.5	0.10032	-11.8124222663142	-11.8124222663142\\
57.5	0.10398	-12.6053762757674	-12.6053762757674\\
57.5	0.10764	-13.5205619490314	-13.5205619490314\\
57.5	0.1113	-14.5579792861062	-14.5579792861062\\
57.5	0.11496	-15.7176282869917	-15.7176282869917\\
57.5	0.11862	-16.999508951688	-16.999508951688\\
57.5	0.12228	-18.403621280195	-18.403621280195\\
57.5	0.12594	-19.9299652725128	-19.9299652725128\\
57.5	0.1296	-21.5785409286414	-21.5785409286414\\
57.5	0.13326	-23.3493482485807	-23.3493482485807\\
57.5	0.13692	-25.2423872323308	-25.2423872323308\\
57.5	0.14058	-27.2576578798917	-27.2576578798917\\
57.5	0.14424	-29.3951601912633	-29.3951601912633\\
57.5	0.1479	-31.6548941664456	-31.6548941664456\\
57.5	0.15156	-34.0368598054387	-34.0368598054387\\
57.5	0.15522	-36.5410571082426	-36.5410571082426\\
57.5	0.15888	-39.1674860748573	-39.1674860748573\\
57.5	0.16254	-41.9161467052826	-41.9161467052826\\
57.5	0.1662	-44.7870389995188	-44.7870389995188\\
57.5	0.16986	-47.7801629575657	-47.7801629575657\\
57.5	0.17352	-50.8955185794233	-50.8955185794233\\
57.5	0.17718	-54.1331058650918	-54.1331058650918\\
57.5	0.18084	-57.492924814571	-57.492924814571\\
57.5	0.1845	-60.9749754278609	-60.9749754278609\\
57.5	0.18816	-64.5792577049616	-64.5792577049616\\
57.5	0.19182	-68.3057716458731	-68.3057716458731\\
57.5	0.19548	-72.1545172505953	-72.1545172505953\\
57.5	0.19914	-76.1254945191282	-76.1254945191282\\
57.5	0.2028	-80.218703451472	-80.218703451472\\
57.5	0.20646	-84.4341440476265	-84.4341440476265\\
57.5	0.21012	-88.7718163075918	-88.7718163075918\\
57.5	0.21378	-93.2317202313677	-93.2317202313677\\
57.5	0.21744	-97.8138558189545	-97.8138558189545\\
57.5	0.2211	-102.518223070352	-102.518223070352\\
57.5	0.22476	-107.34482198556	-107.34482198556\\
57.5	0.22842	-112.293652564579	-112.293652564579\\
57.5	0.23208	-117.364714807409	-117.364714807409\\
57.5	0.23574	-122.55800871405	-122.55800871405\\
57.5	0.2394	-127.873534284501	-127.873534284501\\
57.5	0.24306	-133.311291518763	-133.311291518763\\
57.5	0.24672	-138.871280416836	-138.871280416836\\
57.5	0.25038	-144.55350097872	-144.55350097872\\
57.5	0.25404	-150.357953204414	-150.357953204414\\
57.5	0.2577	-156.284637093919	-156.284637093919\\
57.5	0.26136	-162.333552647235	-162.333552647235\\
57.5	0.26502	-168.504699864362	-168.504699864362\\
57.5	0.26868	-174.798078745299	-174.798078745299\\
57.5	0.27234	-181.213689290047	-181.213689290047\\
57.5	0.276	-187.751531498606	-187.751531498606\\
57.875	0.093	-10.5392617343017	-10.5392617343017\\
57.875	0.09666	-11.0905312516438	-11.0905312516438\\
57.875	0.10032	-11.7640324327967	-11.7640324327967\\
57.875	0.10398	-12.5597652777603	-12.5597652777603\\
57.875	0.10764	-13.4777297865346	-13.4777297865346\\
57.875	0.1113	-14.5179259591197	-14.5179259591197\\
57.875	0.11496	-15.6803537955156	-15.6803537955156\\
57.875	0.11862	-16.9650132957222	-16.9650132957222\\
57.875	0.12228	-18.3719044597396	-18.3719044597396\\
57.875	0.12594	-19.9010272875678	-19.9010272875678\\
57.875	0.1296	-21.5523817792067	-21.5523817792067\\
57.875	0.13326	-23.3259679346564	-23.3259679346564\\
57.875	0.13692	-25.2217857539168	-25.2217857539168\\
57.875	0.14058	-27.239835236988	-27.239835236988\\
57.875	0.14424	-29.3801163838699	-29.3801163838699\\
57.875	0.1479	-31.6426291945626	-31.6426291945626\\
57.875	0.15156	-34.0273736690661	-34.0273736690661\\
57.875	0.15522	-36.5343498073803	-36.5343498073803\\
57.875	0.15888	-39.1635576095053	-39.1635576095053\\
57.875	0.16254	-41.9149970754411	-41.9149970754411\\
57.875	0.1662	-44.7886682051876	-44.7886682051876\\
57.875	0.16986	-47.7845709987448	-47.7845709987448\\
57.875	0.17352	-50.9027054561128	-50.9027054561128\\
57.875	0.17718	-54.1430715772916	-54.1430715772916\\
57.875	0.18084	-57.5056693622811	-57.5056693622811\\
57.875	0.1845	-60.9904988110814	-60.9904988110814\\
57.875	0.18816	-64.5975599236925	-64.5975599236925\\
57.875	0.19182	-68.3268527001143	-68.3268527001143\\
57.875	0.19548	-72.1783771403468	-72.1783771403468\\
57.875	0.19914	-76.1521332443902	-76.1521332443902\\
57.875	0.2028	-80.2481210122443	-80.2481210122443\\
57.875	0.20646	-84.4663404439091	-84.4663404439091\\
57.875	0.21012	-88.8067915393847	-88.8067915393847\\
57.875	0.21378	-93.2694742986711	-93.2694742986711\\
57.875	0.21744	-97.8543887217682	-97.8543887217682\\
57.875	0.2211	-102.561534808676	-102.561534808676\\
57.875	0.22476	-107.390912559395	-107.390912559395\\
57.875	0.22842	-112.342521973924	-112.342521973924\\
57.875	0.23208	-117.416363052264	-117.416363052264\\
57.875	0.23574	-122.612435794415	-122.612435794415\\
57.875	0.2394	-127.930740200377	-127.930740200377\\
57.875	0.24306	-133.371276270149	-133.371276270149\\
57.875	0.24672	-138.934044003732	-138.934044003732\\
57.875	0.25038	-144.619043401126	-144.619043401126\\
57.875	0.25404	-150.426274462331	-150.426274462331\\
57.875	0.2577	-156.355737187347	-156.355737187347\\
57.875	0.26136	-162.407431576173	-162.407431576173\\
57.875	0.26502	-168.58135762881	-168.58135762881\\
57.875	0.26868	-174.877515345258	-174.877515345258\\
57.875	0.27234	-181.295904725516	-181.295904725516\\
57.875	0.276	-187.836525769585	-187.836525769585\\
58.25	0.093	-10.4949106023179	-10.4949106023179\\
58.25	0.09666	-11.0489589551704	-11.0489589551704\\
58.25	0.10032	-11.7252389718336	-11.7252389718336\\
58.25	0.10398	-12.5237506523075	-12.5237506523075\\
58.25	0.10764	-13.4444939965922	-13.4444939965922\\
58.25	0.1113	-14.4874690046877	-14.4874690046877\\
58.25	0.11496	-15.652675676594	-15.652675676594\\
58.25	0.11862	-16.9401140123109	-16.9401140123109\\
58.25	0.12228	-18.3497840118387	-18.3497840118387\\
58.25	0.12594	-19.8816856751772	-19.8816856751772\\
58.25	0.1296	-21.5358190023264	-21.5358190023264\\
58.25	0.13326	-23.3121839932864	-23.3121839932864\\
58.25	0.13692	-25.2107806480572	-25.2107806480572\\
58.25	0.14058	-27.2316089666387	-27.2316089666387\\
58.25	0.14424	-29.3746689490311	-29.3746689490311\\
58.25	0.1479	-31.6399605952341	-31.6399605952341\\
58.25	0.15156	-34.0274839052479	-34.0274839052479\\
58.25	0.15522	-36.5372388790725	-36.5372388790725\\
58.25	0.15888	-39.1692255167078	-39.1692255167078\\
58.25	0.16254	-41.9234438181539	-41.9234438181539\\
58.25	0.1662	-44.7998937834108	-44.7998937834108\\
58.25	0.16986	-47.7985754124784	-47.7985754124784\\
58.25	0.17352	-50.9194887053567	-50.9194887053567\\
58.25	0.17718	-54.1626336620458	-54.1626336620458\\
58.25	0.18084	-57.5280102825457	-57.5280102825457\\
58.25	0.1845	-61.0156185668564	-61.0156185668564\\
58.25	0.18816	-64.6254585149777	-64.6254585149777\\
58.25	0.19182	-68.3575301269099	-68.3575301269099\\
58.25	0.19548	-72.2118334026528	-72.2118334026528\\
58.25	0.19914	-76.1883683422065	-76.1883683422065\\
58.25	0.2028	-80.287134945571	-80.287134945571\\
58.25	0.20646	-84.5081332127461	-84.5081332127461\\
58.25	0.21012	-88.8513631437321	-88.8513631437321\\
58.25	0.21378	-93.3168247385288	-93.3168247385288\\
58.25	0.21744	-97.9045179971362	-97.9045179971362\\
58.25	0.2211	-102.614442919554	-102.614442919554\\
58.25	0.22476	-107.446599505783	-107.446599505783\\
58.25	0.22842	-112.400987755823	-112.400987755823\\
58.25	0.23208	-117.477607669674	-117.477607669674\\
58.25	0.23574	-122.676459247335	-122.676459247335\\
58.25	0.2394	-127.997542488807	-127.997542488807\\
58.25	0.24306	-133.44085739409	-133.44085739409\\
58.25	0.24672	-139.006403963183	-139.006403963183\\
58.25	0.25038	-144.694182196088	-144.694182196088\\
58.25	0.25404	-150.504192092803	-150.504192092803\\
58.25	0.2577	-156.436433653328	-156.436433653328\\
58.25	0.26136	-162.490906877665	-162.490906877665\\
58.25	0.26502	-168.667611765812	-168.667611765812\\
58.25	0.26868	-174.96654831777	-174.96654831777\\
58.25	0.27234	-181.387716533539	-181.387716533539\\
58.25	0.276	-187.931116413119	-187.931116413119\\
58.625	0.093	-10.4601558428886	-10.4601558428886\\
58.625	0.09666	-11.0169830312514	-11.0169830312514\\
58.625	0.10032	-11.6960418834249	-11.6960418834249\\
58.625	0.10398	-12.4973323994092	-12.4973323994092\\
58.625	0.10764	-13.4208545792042	-13.4208545792042\\
58.625	0.1113	-14.46660842281	-14.46660842281\\
58.625	0.11496	-15.6345939302266	-15.6345939302266\\
58.625	0.11862	-16.924811101454	-16.924811101454\\
58.625	0.12228	-18.3372599364921	-18.3372599364921\\
58.625	0.12594	-19.8719404353409	-19.8719404353409\\
58.625	0.1296	-21.5288525980005	-21.5288525980005\\
58.625	0.13326	-23.3079964244709	-23.3079964244709\\
58.625	0.13692	-25.209371914752	-25.209371914752\\
58.625	0.14058	-27.2329790688439	-27.2329790688439\\
58.625	0.14424	-29.3788178867465	-29.3788178867465\\
58.625	0.1479	-31.6468883684599	-31.6468883684599\\
58.625	0.15156	-34.0371905139841	-34.0371905139841\\
58.625	0.15522	-36.549724323319	-36.549724323319\\
58.625	0.15888	-39.1844897964647	-39.1844897964647\\
58.625	0.16254	-41.9414869334211	-41.9414869334211\\
58.625	0.1662	-44.8207157341883	-44.8207157341883\\
58.625	0.16986	-47.8221761987663	-47.8221761987663\\
58.625	0.17352	-50.945868327155	-50.945868327155\\
58.625	0.17718	-54.1917921193544	-54.1917921193544\\
58.625	0.18084	-57.5599475753647	-57.5599475753647\\
58.625	0.1845	-61.0503346951857	-61.0503346951857\\
58.625	0.18816	-64.6629534788174	-64.6629534788174\\
58.625	0.19182	-68.3978039262599	-68.3978039262599\\
58.625	0.19548	-72.2548860375132	-72.2548860375132\\
58.625	0.19914	-76.2341998125772	-76.2341998125772\\
58.625	0.2028	-80.335745251452	-80.335745251452\\
58.625	0.20646	-84.5595223541375	-84.5595223541375\\
58.625	0.21012	-88.9055311206339	-88.9055311206339\\
58.625	0.21378	-93.3737715509408	-93.3737715509408\\
58.625	0.21744	-97.9642436450587	-97.9642436450587\\
58.625	0.2211	-102.676947402987	-102.676947402987\\
58.625	0.22476	-107.511882824727	-107.511882824727\\
58.625	0.22842	-112.469049910277	-112.469049910277\\
58.625	0.23208	-117.548448659638	-117.548448659638\\
58.625	0.23574	-122.750079072809	-122.750079072809\\
58.625	0.2394	-128.073941149791	-128.073941149791\\
58.625	0.24306	-133.520034890585	-133.520034890585\\
58.625	0.24672	-139.088360295189	-139.088360295189\\
58.625	0.25038	-144.778917363603	-144.778917363603\\
58.625	0.25404	-150.591706095828	-150.591706095828\\
58.625	0.2577	-156.526726491865	-156.526726491865\\
58.625	0.26136	-162.583978551712	-162.583978551712\\
58.625	0.26502	-168.763462275369	-168.763462275369\\
58.625	0.26868	-175.065177662838	-175.065177662838\\
58.625	0.27234	-181.489124714117	-181.489124714117\\
58.625	0.276	-188.035303429207	-188.035303429207\\
59	0.093	-10.4349974560135	-10.4349974560135\\
59	0.09666	-10.9946034798867	-10.9946034798867\\
59	0.10032	-11.6764411675706	-11.6764411675706\\
59	0.10398	-12.4805105190652	-12.4805105190652\\
59	0.10764	-13.4068115343706	-13.4068115343706\\
59	0.1113	-14.4553442134868	-14.4553442134868\\
59	0.11496	-15.6261085564137	-15.6261085564137\\
59	0.11862	-16.9191045631514	-16.9191045631514\\
59	0.12228	-18.3343322336999	-18.3343322336999\\
59	0.12594	-19.8717915680591	-19.8717915680591\\
59	0.1296	-21.531482566229	-21.531482566229\\
59	0.13326	-23.3134052282097	-23.3134052282097\\
59	0.13692	-25.2175595540012	-25.2175595540012\\
59	0.14058	-27.2439455436034	-27.2439455436034\\
59	0.14424	-29.3925631970164	-29.3925631970164\\
59	0.1479	-31.6634125142402	-31.6634125142402\\
59	0.15156	-34.0564934952747	-34.0564934952747\\
59	0.15522	-36.5718061401199	-36.5718061401199\\
59	0.15888	-39.209350448776	-39.209350448776\\
59	0.16254	-41.9691264212427	-41.9691264212427\\
59	0.1662	-44.8511340575203	-44.8511340575203\\
59	0.16986	-47.8553733576086	-47.8553733576086\\
59	0.17352	-50.9818443215076	-50.9818443215076\\
59	0.17718	-54.2305469492175	-54.2305469492175\\
59	0.18084	-57.601481240738	-57.601481240738\\
59	0.1845	-61.0946471960694	-61.0946471960694\\
59	0.18816	-64.7100448152114	-64.7100448152114\\
59	0.19182	-68.4476740981643	-68.4476740981643\\
59	0.19548	-72.3075350449279	-72.3075350449279\\
59	0.19914	-76.2896276555023	-76.2896276555023\\
59	0.2028	-80.3939519298874	-80.3939519298874\\
59	0.20646	-84.6205078680833	-84.6205078680833\\
59	0.21012	-88.96929547009	-88.96929547009\\
59	0.21378	-93.4403147359074	-93.4403147359074\\
59	0.21744	-98.0335656655355	-98.0335656655355\\
59	0.2211	-102.749048258974	-102.749048258974\\
59	0.22476	-107.586762516224	-107.586762516224\\
59	0.22842	-112.546708437285	-112.546708437285\\
59	0.23208	-117.628886022156	-117.628886022156\\
59	0.23574	-122.833295270838	-122.833295270838\\
59	0.2394	-128.15993618333	-128.15993618333\\
59	0.24306	-133.608808759634	-133.608808759634\\
59	0.24672	-139.179912999748	-139.179912999748\\
59	0.25038	-144.873248903673	-144.873248903673\\
59	0.25404	-150.688816471409	-150.688816471409\\
59	0.2577	-156.626615702955	-156.626615702955\\
59	0.26136	-162.686646598313	-162.686646598313\\
59	0.26502	-168.868909157481	-168.868909157481\\
59	0.26868	-175.173403380459	-175.173403380459\\
59	0.27234	-181.600129267249	-181.600129267249\\
59	0.276	-188.149086817849	-188.149086817849\\
59.375	0.093	-10.419435441693	-10.419435441693\\
59.375	0.09666	-10.9818203010765	-10.9818203010765\\
59.375	0.10032	-11.6664368242707	-11.6664368242707\\
59.375	0.10398	-12.4732850112757	-12.4732850112757\\
59.375	0.10764	-13.4023648620915	-13.4023648620915\\
59.375	0.1113	-14.453676376718	-14.453676376718\\
59.375	0.11496	-15.6272195551553	-15.6272195551553\\
59.375	0.11862	-16.9229943974033	-16.9229943974033\\
59.375	0.12228	-18.341000903462	-18.341000903462\\
59.375	0.12594	-19.8812390733316	-19.8812390733316\\
59.375	0.1296	-21.5437089070119	-21.5437089070119\\
59.375	0.13326	-23.3284104045029	-23.3284104045029\\
59.375	0.13692	-25.2353435658048	-25.2353435658048\\
59.375	0.14058	-27.2645083909174	-27.2645083909174\\
59.375	0.14424	-29.4159048798407	-29.4159048798407\\
59.375	0.1479	-31.6895330325748	-31.6895330325748\\
59.375	0.15156	-34.0853928491197	-34.0853928491197\\
59.375	0.15522	-36.6034843294753	-36.6034843294753\\
59.375	0.15888	-39.2438074736417	-39.2438074736417\\
59.375	0.16254	-42.0063622816188	-42.0063622816188\\
59.375	0.1662	-44.8911487534067	-44.8911487534067\\
59.375	0.16986	-47.8981668890053	-47.8981668890053\\
59.375	0.17352	-51.0274166884147	-51.0274166884147\\
59.375	0.17718	-54.2788981516349	-54.2788981516349\\
59.375	0.18084	-57.6526112786658	-57.6526112786658\\
59.375	0.1845	-61.1485560695075	-61.1485560695075\\
59.375	0.18816	-64.7667325241599	-64.7667325241599\\
59.375	0.19182	-68.5071406426231	-68.5071406426231\\
59.375	0.19548	-72.3697804248971	-72.3697804248971\\
59.375	0.19914	-76.3546518709818	-76.3546518709818\\
59.375	0.2028	-80.4617549808773	-80.4617549808773\\
59.375	0.20646	-84.6910897545836	-84.6910897545836\\
59.375	0.21012	-89.0426561921006	-89.0426561921006\\
59.375	0.21378	-93.5164542934283	-93.5164542934283\\
59.375	0.21744	-98.1124840585668	-98.1124840585668\\
59.375	0.2211	-102.830745487516	-102.830745487516\\
59.375	0.22476	-107.671238580276	-107.671238580276\\
59.375	0.22842	-112.633963336847	-112.633963336847\\
59.375	0.23208	-117.718919757228	-117.718919757228\\
59.375	0.23574	-122.926107841421	-122.926107841421\\
59.375	0.2394	-128.255527589424	-128.255527589424\\
59.375	0.24306	-133.707179001238	-133.707179001238\\
59.375	0.24672	-139.281062076862	-139.281062076862\\
59.375	0.25038	-144.977176816298	-144.977176816298\\
59.375	0.25404	-150.795523219544	-150.795523219544\\
59.375	0.2577	-156.7361012866	-156.7361012866\\
59.375	0.26136	-162.798911017468	-162.798911017468\\
59.375	0.26502	-168.983952412146	-168.983952412146\\
59.375	0.26868	-175.291225470636	-175.291225470636\\
59.375	0.27234	-181.720730192935	-181.720730192935\\
59.375	0.276	-188.272466579046	-188.272466579046\\
59.75	0.093	-10.4134697999268	-10.4134697999268\\
59.75	0.09666	-10.9786334948206	-10.9786334948206\\
59.75	0.10032	-11.6660288535252	-11.6660288535252\\
59.75	0.10398	-12.4756558760405	-12.4756558760405\\
59.75	0.10764	-13.4075145623667	-13.4075145623667\\
59.75	0.1113	-14.4616049125035	-14.4616049125035\\
59.75	0.11496	-15.6379269264511	-15.6379269264511\\
59.75	0.11862	-16.9364806042095	-16.9364806042095\\
59.75	0.12228	-18.3572659457786	-18.3572659457786\\
59.75	0.12594	-19.9002829511585	-19.9002829511585\\
59.75	0.1296	-21.5655316203492	-21.5655316203492\\
59.75	0.13326	-23.3530119533506	-23.3530119533506\\
59.75	0.13692	-25.2627239501627	-25.2627239501627\\
59.75	0.14058	-27.2946676107857	-27.2946676107857\\
59.75	0.14424	-29.4488429352194	-29.4488429352194\\
59.75	0.1479	-31.7252499234638	-31.7252499234638\\
59.75	0.15156	-34.123888575519	-34.123888575519\\
59.75	0.15522	-36.644758891385	-36.644758891385\\
59.75	0.15888	-39.2878608710617	-39.2878608710617\\
59.75	0.16254	-42.0531945145492	-42.0531945145492\\
59.75	0.1662	-44.9407598218474	-44.9407598218474\\
59.75	0.16986	-47.9505567929564	-47.9505567929564\\
59.75	0.17352	-51.0825854278761	-51.0825854278761\\
59.75	0.17718	-54.3368457266067	-54.3368457266067\\
59.75	0.18084	-57.7133376891479	-57.7133376891479\\
59.75	0.1845	-61.2120613155	-61.2120613155\\
59.75	0.18816	-64.8330166056628	-64.8330166056628\\
59.75	0.19182	-68.5762035596363	-68.5762035596363\\
59.75	0.19548	-72.4416221774206	-72.4416221774206\\
59.75	0.19914	-76.4292724590157	-76.4292724590157\\
59.75	0.2028	-80.5391544044216	-80.5391544044216\\
59.75	0.20646	-84.7712680136381	-84.7712680136381\\
59.75	0.21012	-89.1256132866654	-89.1256132866654\\
59.75	0.21378	-93.6021902235035	-93.6021902235035\\
59.75	0.21744	-98.2009988241524	-98.2009988241524\\
59.75	0.2211	-102.922039088612	-102.922039088612\\
59.75	0.22476	-107.765311016882	-107.765311016882\\
59.75	0.22842	-112.730814608963	-112.730814608963\\
59.75	0.23208	-117.818549864855	-117.818549864855\\
59.75	0.23574	-123.028516784558	-123.028516784558\\
59.75	0.2394	-128.360715368072	-128.360715368072\\
59.75	0.24306	-133.815145615396	-133.815145615396\\
59.75	0.24672	-139.391807526531	-139.391807526531\\
59.75	0.25038	-145.090701101476	-145.090701101476\\
59.75	0.25404	-150.911826340233	-150.911826340233\\
59.75	0.2577	-156.8551832428	-156.8551832428\\
59.75	0.26136	-162.920771809178	-162.920771809178\\
59.75	0.26502	-169.108592039367	-169.108592039367\\
59.75	0.26868	-175.418643933366	-175.418643933366\\
59.75	0.27234	-181.850927491176	-181.850927491176\\
59.75	0.276	-188.405442712797	-188.405442712797\\
60.125	0.093	-10.417100530715	-10.417100530715\\
60.125	0.09666	-10.9850430611191	-10.9850430611191\\
60.125	0.10032	-11.6752172553341	-11.6752172553341\\
60.125	0.10398	-12.4876231133598	-12.4876231133598\\
60.125	0.10764	-13.4222606351962	-13.4222606351962\\
60.125	0.1113	-14.4791298208435	-14.4791298208435\\
60.125	0.11496	-15.6582306703014	-15.6582306703014\\
60.125	0.11862	-16.9595631835701	-16.9595631835701\\
60.125	0.12228	-18.3831273606496	-18.3831273606496\\
60.125	0.12594	-19.9289232015398	-19.9289232015398\\
60.125	0.1296	-21.5969507062408	-21.5969507062408\\
60.125	0.13326	-23.3872098747526	-23.3872098747526\\
60.125	0.13692	-25.2997007070751	-25.2997007070751\\
60.125	0.14058	-27.3344232032084	-27.3344232032084\\
60.125	0.14424	-29.4913773631524	-29.4913773631524\\
60.125	0.1479	-31.7705631869072	-31.7705631869072\\
60.125	0.15156	-34.1719806744728	-34.1719806744728\\
60.125	0.15522	-36.6956298258491	-36.6956298258491\\
60.125	0.15888	-39.3415106410362	-39.3415106410362\\
60.125	0.16254	-42.109623120034	-42.109623120034\\
60.125	0.1662	-44.9999672628426	-44.9999672628426\\
60.125	0.16986	-48.0125430694619	-48.0125430694619\\
60.125	0.17352	-51.147350539892	-51.147350539892\\
60.125	0.17718	-54.4043896741329	-54.4043896741329\\
60.125	0.18084	-57.7836604721845	-57.7836604721845\\
60.125	0.1845	-61.2851629340469	-61.2851629340469\\
60.125	0.18816	-64.90889705972	-64.90889705972\\
60.125	0.19182	-68.6548628492039	-68.6548628492039\\
60.125	0.19548	-72.5230603024985	-72.5230603024985\\
60.125	0.19914	-76.513489419604	-76.513489419604\\
60.125	0.2028	-80.6261502005202	-80.6261502005202\\
60.125	0.20646	-84.8610426452471	-84.8610426452471\\
60.125	0.21012	-89.2181667537848	-89.2181667537848\\
60.125	0.21378	-93.6975225261332	-93.6975225261332\\
60.125	0.21744	-98.2991099622924	-98.2991099622924\\
60.125	0.2211	-103.022929062262	-103.022929062262\\
60.125	0.22476	-107.868979826043	-107.868979826043\\
60.125	0.22842	-112.837262253635	-112.837262253635\\
60.125	0.23208	-117.927776345037	-117.927776345037\\
60.125	0.23574	-123.14052210025	-123.14052210025\\
60.125	0.2394	-128.475499519274	-128.475499519274\\
60.125	0.24306	-133.932708602108	-133.932708602108\\
60.125	0.24672	-139.512149348753	-139.512149348753\\
60.125	0.25038	-145.213821759209	-145.213821759209\\
60.125	0.25404	-151.037725833476	-151.037725833476\\
60.125	0.2577	-156.983861571554	-156.983861571554\\
60.125	0.26136	-163.052228973442	-163.052228973442\\
60.125	0.26502	-169.242828039141	-169.242828039141\\
60.125	0.26868	-175.555658768651	-175.555658768651\\
60.125	0.27234	-181.990721161972	-181.990721161972\\
60.125	0.276	-188.548015219103	-188.548015219103\\
60.5	0.093	-10.4303276340576	-10.4303276340576\\
60.5	0.09666	-11.0010489999721	-11.0010489999721\\
60.5	0.10032	-11.6940020296974	-11.6940020296974\\
60.5	0.10398	-12.5091867232334	-12.5091867232334\\
60.5	0.10764	-13.4466030805802	-13.4466030805802\\
60.5	0.1113	-14.5062511017378	-14.5062511017378\\
60.5	0.11496	-15.6881307867061	-15.6881307867061\\
60.5	0.11862	-16.9922421354852	-16.9922421354852\\
60.5	0.12228	-18.418585148075	-18.418585148075\\
60.5	0.12594	-19.9671598244756	-19.9671598244756\\
60.5	0.1296	-21.6379661646869	-21.6379661646869\\
60.5	0.13326	-23.4310041687091	-23.4310041687091\\
60.5	0.13692	-25.3462738365419	-25.3462738365419\\
60.5	0.14058	-27.3837751681856	-27.3837751681856\\
60.5	0.14424	-29.5435081636399	-29.5435081636399\\
60.5	0.1479	-31.8254728229051	-31.8254728229051\\
60.5	0.15156	-34.229669145981	-34.229669145981\\
60.5	0.15522	-36.7560971328676	-36.7560971328676\\
60.5	0.15888	-39.404756783565	-39.404756783565\\
60.5	0.16254	-42.1756480980732	-42.1756480980732\\
60.5	0.1662	-45.0687710763922	-45.0687710763922\\
60.5	0.16986	-48.0841257185219	-48.0841257185219\\
60.5	0.17352	-51.2217120244623	-51.2217120244623\\
60.5	0.17718	-54.4815299942135	-54.4815299942135\\
60.5	0.18084	-57.8635796277754	-57.8635796277754\\
60.5	0.1845	-61.3678609251482	-61.3678609251482\\
60.5	0.18816	-64.9943738863317	-64.9943738863317\\
60.5	0.19182	-68.7431185113259	-68.7431185113259\\
60.5	0.19548	-72.6140948001309	-72.6140948001309\\
60.5	0.19914	-76.6073027527467	-76.6073027527467\\
60.5	0.2028	-80.7227423691732	-80.7227423691732\\
60.5	0.20646	-84.9604136494105	-84.9604136494105\\
60.5	0.21012	-89.3203165934586	-89.3203165934586\\
60.5	0.21378	-93.8024512013173	-93.8024512013173\\
60.5	0.21744	-98.4068174729869	-98.4068174729869\\
60.5	0.2211	-103.133415408467	-103.133415408467\\
60.5	0.22476	-107.982245007758	-107.982245007758\\
60.5	0.22842	-112.95330627086	-112.95330627086\\
60.5	0.23208	-118.046599197773	-118.046599197773\\
60.5	0.23574	-123.262123788496	-123.262123788496\\
60.5	0.2394	-128.59988004303	-128.59988004303\\
60.5	0.24306	-134.059867961375	-134.059867961375\\
60.5	0.24672	-139.642087543531	-139.642087543531\\
60.5	0.25038	-145.346538789497	-145.346538789497\\
60.5	0.25404	-151.173221699274	-151.173221699274\\
60.5	0.2577	-157.122136272862	-157.122136272862\\
60.5	0.26136	-163.193282510261	-163.193282510261\\
60.5	0.26502	-169.38666041147	-169.38666041147\\
60.5	0.26868	-175.70226997649	-175.70226997649\\
60.5	0.27234	-182.140111205321	-182.140111205321\\
60.5	0.276	-188.700184097963	-188.700184097963\\
60.875	0.093	-10.4531511099545	-10.4531511099545\\
60.875	0.09666	-11.0266513113794	-11.0266513113794\\
60.875	0.10032	-11.7223831766151	-11.7223831766151\\
60.875	0.10398	-12.5403467056614	-12.5403467056614\\
60.875	0.10764	-13.4805418985186	-13.4805418985186\\
60.875	0.1113	-14.5429687551865	-14.5429687551865\\
60.875	0.11496	-15.7276272756652	-15.7276272756652\\
60.875	0.11862	-17.0345174599546	-17.0345174599546\\
60.875	0.12228	-18.4636393080548	-18.4636393080548\\
60.875	0.12594	-20.0149928199657	-20.0149928199657\\
60.875	0.1296	-21.6885779956874	-21.6885779956874\\
60.875	0.13326	-23.4843948352198	-23.4843948352198\\
60.875	0.13692	-25.402443338563	-25.402443338563\\
60.875	0.14058	-27.442723505717	-27.442723505717\\
60.875	0.14424	-29.6052353366818	-29.6052353366818\\
60.875	0.1479	-31.8899788314572	-31.8899788314572\\
60.875	0.15156	-34.2969539900435	-34.2969539900435\\
60.875	0.15522	-36.8261608124405	-36.8261608124405\\
60.875	0.15888	-39.4775992986483	-39.4775992986483\\
60.875	0.16254	-42.2512694486668	-42.2512694486668\\
60.875	0.1662	-45.1471712624961	-45.1471712624961\\
60.875	0.16986	-48.1653047401361	-48.1653047401361\\
60.875	0.17352	-51.3056698815869	-51.3056698815869\\
60.875	0.17718	-54.5682666868485	-54.5682666868485\\
60.875	0.18084	-57.9530951559207	-57.9530951559207\\
60.875	0.1845	-61.4601552888039	-61.4601552888039\\
60.875	0.18816	-65.0894470854977	-65.0894470854977\\
60.875	0.19182	-68.8409705460023	-68.8409705460023\\
60.875	0.19548	-72.7147256703176	-72.7147256703176\\
60.875	0.19914	-76.7107124584437	-76.7107124584437\\
60.875	0.2028	-80.8289309103806	-80.8289309103806\\
60.875	0.20646	-85.0693810261283	-85.0693810261283\\
60.875	0.21012	-89.4320628056866	-89.4320628056866\\
60.875	0.21378	-93.9169762490558	-93.9169762490558\\
60.875	0.21744	-98.5241213562357	-98.5241213562357\\
60.875	0.2211	-103.253498127226	-103.253498127226\\
60.875	0.22476	-108.105106562028	-108.105106562028\\
60.875	0.22842	-113.07894666064	-113.07894666064\\
60.875	0.23208	-118.175018423063	-118.175018423063\\
60.875	0.23574	-123.393321849297	-123.393321849297\\
60.875	0.2394	-128.733856939341	-128.733856939341\\
60.875	0.24306	-134.196623693196	-134.196623693196\\
60.875	0.24672	-139.781622110862	-139.781622110862\\
60.875	0.25038	-145.488852192339	-145.488852192339\\
60.875	0.25404	-151.318313937626	-151.318313937626\\
60.875	0.2577	-157.270007346725	-157.270007346725\\
60.875	0.26136	-163.343932419634	-163.343932419634\\
60.875	0.26502	-169.540089156353	-169.540089156353\\
60.875	0.26868	-175.858477556884	-175.858477556884\\
60.875	0.27234	-182.299097621225	-182.299097621225\\
60.875	0.276	-188.861949349377	-188.861949349377\\
61.25	0.093	-10.4855709584059	-10.4855709584059\\
61.25	0.09666	-11.0618499953412	-11.0618499953412\\
61.25	0.10032	-11.7603606960871	-11.7603606960871\\
61.25	0.10398	-12.5811030606439	-12.5811030606439\\
61.25	0.10764	-13.5240770890114	-13.5240770890114\\
61.25	0.1113	-14.5892827811896	-14.5892827811896\\
61.25	0.11496	-15.7767201371787	-15.7767201371787\\
61.25	0.11862	-17.0863891569784	-17.0863891569784\\
61.25	0.12228	-18.5182898405889	-18.5182898405889\\
61.25	0.12594	-20.0724221880102	-20.0724221880102\\
61.25	0.1296	-21.7487861992422	-21.7487861992422\\
61.25	0.13326	-23.5473818742851	-23.5473818742851\\
61.25	0.13692	-25.4682092131386	-25.4682092131386\\
61.25	0.14058	-27.511268215803	-27.511268215803\\
61.25	0.14424	-29.676558882278	-29.676558882278\\
61.25	0.1479	-31.9640812125639	-31.9640812125639\\
61.25	0.15156	-34.3738352066605	-34.3738352066605\\
61.25	0.15522	-36.9058208645678	-36.9058208645678\\
61.25	0.15888	-39.5600381862859	-39.5600381862859\\
61.25	0.16254	-42.3364871718148	-42.3364871718148\\
61.25	0.1662	-45.2351678211544	-45.2351678211544\\
61.25	0.16986	-48.2560801343049	-48.2560801343049\\
61.25	0.17352	-51.3992241112659	-51.3992241112659\\
61.25	0.17718	-54.6645997520378	-54.6645997520378\\
61.25	0.18084	-58.0522070566205	-58.0522070566205\\
61.25	0.1845	-61.562046025014	-61.562046025014\\
61.25	0.18816	-65.1941166572181	-65.1941166572181\\
61.25	0.19182	-68.9484189532331	-68.9484189532331\\
61.25	0.19548	-72.8249529130588	-72.8249529130588\\
61.25	0.19914	-76.8237185366953	-76.8237185366953\\
61.25	0.2028	-80.9447158241425	-80.9447158241425\\
61.25	0.20646	-85.1879447754005	-85.1879447754005\\
61.25	0.21012	-89.5534053904692	-89.5534053904692\\
61.25	0.21378	-94.0410976693486	-94.0410976693486\\
61.25	0.21744	-98.6510216120389	-98.6510216120389\\
61.25	0.2211	-103.38317721854	-103.38317721854\\
61.25	0.22476	-108.237564488852	-108.237564488852\\
61.25	0.22842	-113.214183422974	-113.214183422974\\
61.25	0.23208	-118.313034020907	-118.313034020907\\
61.25	0.23574	-123.534116282652	-123.534116282652\\
61.25	0.2394	-128.877430208206	-128.877430208206\\
61.25	0.24306	-134.342975797572	-134.342975797572\\
61.25	0.24672	-139.930753050748	-139.930753050748\\
61.25	0.25038	-145.640761967735	-145.640761967735\\
61.25	0.25404	-151.473002548533	-151.473002548533\\
61.25	0.2577	-157.427474793142	-157.427474793142\\
61.25	0.26136	-163.504178701561	-163.504178701561\\
61.25	0.26502	-169.703114273791	-169.703114273791\\
61.25	0.26868	-176.024281509832	-176.024281509832\\
61.25	0.27234	-182.467680409684	-182.467680409684\\
61.25	0.276	-189.033310973346	-189.033310973346\\
61.625	0.093	-10.5275871794117	-10.5275871794117\\
61.625	0.09666	-11.1066450518573	-11.1066450518573\\
61.625	0.10032	-11.8079345881137	-11.8079345881137\\
61.625	0.10398	-12.6314557881807	-12.6314557881807\\
61.625	0.10764	-13.5772086520586	-13.5772086520586\\
61.625	0.1113	-14.6451931797472	-14.6451931797472\\
61.625	0.11496	-15.8354093712465	-15.8354093712465\\
61.625	0.11862	-17.1478572265567	-17.1478572265567\\
61.625	0.12228	-18.5825367456775	-18.5825367456775\\
61.625	0.12594	-20.1394479286092	-20.1394479286092\\
61.625	0.1296	-21.8185907753516	-21.8185907753516\\
61.625	0.13326	-23.6199652859047	-23.6199652859047\\
61.625	0.13692	-25.5435714602686	-25.5435714602686\\
61.625	0.14058	-27.5894092984433	-27.5894092984433\\
61.625	0.14424	-29.7574788004287	-29.7574788004287\\
61.625	0.1479	-32.0477799662249	-32.0477799662249\\
61.625	0.15156	-34.4603127958319	-34.4603127958319\\
61.625	0.15522	-36.9950772892496	-36.9950772892496\\
61.625	0.15888	-39.652073446478	-39.652073446478\\
61.625	0.16254	-42.4313012675172	-42.4313012675172\\
61.625	0.1662	-45.3327607523672	-45.3327607523672\\
61.625	0.16986	-48.356451901028	-48.356451901028\\
61.625	0.17352	-51.5023747134994	-51.5023747134994\\
61.625	0.17718	-54.7705291897817	-54.7705291897817\\
61.625	0.18084	-58.1609153298747	-58.1609153298747\\
61.625	0.1845	-61.6735331337785	-61.6735331337785\\
61.625	0.18816	-65.308382601493	-65.308382601493\\
61.625	0.19182	-69.0654637330183	-69.0654637330183\\
61.625	0.19548	-72.9447765283543	-72.9447765283543\\
61.625	0.19914	-76.9463209875011	-76.9463209875011\\
61.625	0.2028	-81.0700971104587	-81.0700971104587\\
61.625	0.20646	-85.3161048972271	-85.3161048972271\\
61.625	0.21012	-89.6843443478061	-89.6843443478061\\
61.625	0.21378	-94.1748154621959	-94.1748154621959\\
61.625	0.21744	-98.7875182403965	-98.7875182403965\\
61.625	0.2211	-103.522452682408	-103.522452682408\\
61.625	0.22476	-108.37961878823	-108.37961878823\\
61.625	0.22842	-113.359016557863	-113.359016557863\\
61.625	0.23208	-118.460645991307	-118.460645991307\\
61.625	0.23574	-123.684507088561	-123.684507088561\\
61.625	0.2394	-129.030599849626	-129.030599849626\\
61.625	0.24306	-134.498924274502	-134.498924274502\\
61.625	0.24672	-140.089480363189	-140.089480363189\\
61.625	0.25038	-145.802268115686	-145.802268115686\\
61.625	0.25404	-151.637287531994	-151.637287531994\\
61.625	0.2577	-157.594538612113	-157.594538612113\\
61.625	0.26136	-163.674021356043	-163.674021356043\\
61.625	0.26502	-169.875735763783	-169.875735763783\\
61.625	0.26868	-176.199681835335	-176.199681835335\\
61.625	0.27234	-182.645859570696	-182.645859570696\\
61.625	0.276	-189.214268969869	-189.214268969869\\
62	0.093	-10.5791997729719	-10.5791997729719\\
62	0.09666	-11.1610364809278	-11.1610364809278\\
62	0.10032	-11.8651048526945	-11.8651048526945\\
62	0.10398	-12.6914048882719	-12.6914048882719\\
62	0.10764	-13.6399365876601	-13.6399365876601\\
62	0.1113	-14.7106999508591	-14.7106999508591\\
62	0.11496	-15.9036949778688	-15.9036949778688\\
62	0.11862	-17.2189216686892	-17.2189216686892\\
62	0.12228	-18.6563800233205	-18.6563800233205\\
62	0.12594	-20.2160700417624	-20.2160700417624\\
62	0.1296	-21.8979917240152	-21.8979917240152\\
62	0.13326	-23.7021450700787	-23.7021450700787\\
62	0.13692	-25.6285300799529	-25.6285300799529\\
62	0.14058	-27.677146753638	-27.677146753638\\
62	0.14424	-29.8479950911337	-29.8479950911337\\
62	0.1479	-32.1410750924403	-32.1410750924403\\
62	0.15156	-34.5563867575575	-34.5563867575575\\
62	0.15522	-37.0939300864856	-37.0939300864856\\
62	0.15888	-39.7537050792244	-39.7537050792244\\
62	0.16254	-42.535711735774	-42.535711735774\\
62	0.1662	-45.4399500561343	-45.4399500561343\\
62	0.16986	-48.4664200403054	-48.4664200403054\\
62	0.17352	-51.6151216882872	-51.6151216882872\\
62	0.17718	-54.8860550000798	-54.8860550000798\\
62	0.18084	-58.2792199756832	-58.2792199756832\\
62	0.1845	-61.7946166150973	-61.7946166150973\\
62	0.18816	-65.4322449183222	-65.4322449183222\\
62	0.19182	-69.1921048853578	-69.1921048853578\\
62	0.19548	-73.0741965162042	-73.0741965162042\\
62	0.19914	-77.0785198108613	-77.0785198108613\\
62	0.2028	-81.2050747693293	-81.2050747693293\\
62	0.20646	-85.453861391608	-85.453861391608\\
62	0.21012	-89.8248796776974	-89.8248796776974\\
62	0.21378	-94.3181296275976	-94.3181296275976\\
62	0.21744	-98.9336112413085	-98.9336112413085\\
62	0.2211	-103.67132451883	-103.67132451883\\
62	0.22476	-108.531269460163	-108.531269460163\\
62	0.22842	-113.513446065306	-113.513446065306\\
62	0.23208	-118.61785433426	-118.61785433426\\
62	0.23574	-123.844494267025	-123.844494267025\\
62	0.2394	-129.1933658636	-129.1933658636\\
62	0.24306	-134.664469123986	-134.664469123986\\
62	0.24672	-140.257804048183	-140.257804048183\\
62	0.25038	-145.973370636191	-145.973370636191\\
62	0.25404	-151.81116888801	-151.81116888801\\
62	0.2577	-157.771198803639	-157.771198803639\\
62	0.26136	-163.853460383079	-163.853460383079\\
62	0.26502	-170.05795362633	-170.05795362633\\
62	0.26868	-176.384678533391	-176.384678533391\\
62	0.27234	-182.833635104264	-182.833635104264\\
62	0.276	-189.404823338947	-189.404823338947\\
62.375	0.093	-10.6404087390865	-10.6404087390865\\
62.375	0.09666	-11.2250242825528	-11.2250242825528\\
62.375	0.10032	-11.9318714898298	-11.9318714898298\\
62.375	0.10398	-12.7609503609176	-12.7609503609176\\
62.375	0.10764	-13.7122608958161	-13.7122608958161\\
62.375	0.1113	-14.7858030945254	-14.7858030945254\\
62.375	0.11496	-15.9815769570455	-15.9815769570455\\
62.375	0.11862	-17.2995824833763	-17.2995824833763\\
62.375	0.12228	-18.7398196735178	-18.7398196735178\\
62.375	0.12594	-20.3022885274702	-20.3022885274702\\
62.375	0.1296	-21.9869890452332	-21.9869890452332\\
62.375	0.13326	-23.7939212268071	-23.7939212268071\\
62.375	0.13692	-25.7230850721917	-25.7230850721917\\
62.375	0.14058	-27.7744805813871	-27.7744805813871\\
62.375	0.14424	-29.9481077543932	-29.9481077543932\\
62.375	0.1479	-32.2439665912101	-32.2439665912101\\
62.375	0.15156	-34.6620570918377	-34.6620570918377\\
62.375	0.15522	-37.2023792562761	-37.2023792562761\\
62.375	0.15888	-39.8649330845253	-39.8649330845253\\
62.375	0.16254	-42.6497185765852	-42.6497185765852\\
62.375	0.1662	-45.5567357324559	-45.5567357324559\\
62.375	0.16986	-48.5859845521373	-48.5859845521373\\
62.375	0.17352	-51.7374650356295	-51.7374650356295\\
62.375	0.17718	-55.0111771829324	-55.0111771829324\\
62.375	0.18084	-58.4071209940461	-58.4071209940461\\
62.375	0.1845	-61.9252964689706	-61.9252964689706\\
62.375	0.18816	-65.5657036077058	-65.5657036077058\\
62.375	0.19182	-69.3283424102518	-69.3283424102518\\
62.375	0.19548	-73.2132128766086	-73.2132128766086\\
62.375	0.19914	-77.2203150067761	-77.2203150067761\\
62.375	0.2028	-81.3496488007544	-81.3496488007544\\
62.375	0.20646	-85.6012142585434	-85.6012142585434\\
62.375	0.21012	-89.9750113801431	-89.9750113801431\\
62.375	0.21378	-94.4710401655537	-94.4710401655537\\
62.375	0.21744	-99.0893006147749	-99.0893006147749\\
62.375	0.2211	-103.829792727807	-103.829792727807\\
62.375	0.22476	-108.69251650465	-108.69251650465\\
62.375	0.22842	-113.677471945303	-113.677471945303\\
62.375	0.23208	-118.784659049768	-118.784659049768\\
62.375	0.23574	-124.014077818043	-124.014077818043\\
62.375	0.2394	-129.365728250129	-129.365728250129\\
62.375	0.24306	-134.839610346025	-134.839610346025\\
62.375	0.24672	-140.435724105733	-140.435724105733\\
62.375	0.25038	-146.154069529251	-146.154069529251\\
62.375	0.25404	-151.99464661658	-151.99464661658\\
62.375	0.2577	-157.957455367719	-157.957455367719\\
62.375	0.26136	-164.04249578267	-164.04249578267\\
62.375	0.26502	-170.249767861431	-170.249767861431\\
62.375	0.26868	-176.579271604003	-176.579271604003\\
62.375	0.27234	-183.031007010385	-183.031007010385\\
62.375	0.276	-189.604974080579	-189.604974080579\\
62.75	0.093	-10.7112140777555	-10.7112140777555\\
62.75	0.09666	-11.2986084567321	-11.2986084567321\\
62.75	0.10032	-12.0082344995195	-12.0082344995195\\
62.75	0.10398	-12.8400922061176	-12.8400922061176\\
62.75	0.10764	-13.7941815765265	-13.7941815765265\\
62.75	0.1113	-14.8705026107462	-14.8705026107462\\
62.75	0.11496	-16.0690553087766	-16.0690553087766\\
62.75	0.11862	-17.3898396706177	-17.3898396706177\\
62.75	0.12228	-18.8328556962696	-18.8328556962696\\
62.75	0.12594	-20.3981033857323	-20.3981033857323\\
62.75	0.1296	-22.0855827390057	-22.0855827390057\\
62.75	0.13326	-23.8952937560899	-23.8952937560899\\
62.75	0.13692	-25.8272364369849	-25.8272364369849\\
62.75	0.14058	-27.8814107816906	-27.8814107816906\\
62.75	0.14424	-30.0578167902071	-30.0578167902071\\
62.75	0.1479	-32.3564544625343	-32.3564544625343\\
62.75	0.15156	-34.7773237986723	-34.7773237986723\\
62.75	0.15522	-37.320424798621	-37.320424798621\\
62.75	0.15888	-39.9857574623806	-39.9857574623806\\
62.75	0.16254	-42.7733217899508	-42.7733217899508\\
62.75	0.1662	-45.6831177813318	-45.6831177813318\\
62.75	0.16986	-48.7151454365236	-48.7151454365236\\
62.75	0.17352	-51.8694047555261	-51.8694047555261\\
62.75	0.17718	-55.1458957383394	-55.1458957383394\\
62.75	0.18084	-58.5446183849635	-58.5446183849635\\
62.75	0.1845	-62.0655726953983	-62.0655726953983\\
62.75	0.18816	-65.7087586696439	-65.7087586696439\\
62.75	0.19182	-69.4741763077002	-69.4741763077002\\
62.75	0.19548	-73.3618256095673	-73.3618256095673\\
62.75	0.19914	-77.3717065752451	-77.3717065752451\\
62.75	0.2028	-81.5038192047338	-81.5038192047338\\
62.75	0.20646	-85.7581634980332	-85.7581634980332\\
62.75	0.21012	-90.1347394551433	-90.1347394551433\\
62.75	0.21378	-94.6335470760641	-94.6335470760641\\
62.75	0.21744	-99.2545863607958	-99.2545863607958\\
62.75	0.2211	-103.997857309338	-103.997857309338\\
62.75	0.22476	-108.863359921691	-108.863359921691\\
62.75	0.22842	-113.851094197855	-113.851094197855\\
62.75	0.23208	-118.96106013783	-118.96106013783\\
62.75	0.23574	-124.193257741615	-124.193257741615\\
62.75	0.2394	-129.547687009212	-129.547687009212\\
62.75	0.24306	-135.024347940619	-135.024347940619\\
62.75	0.24672	-140.623240535836	-140.623240535836\\
62.75	0.25038	-146.344364794865	-146.344364794865\\
62.75	0.25404	-152.187720717704	-152.187720717704\\
62.75	0.2577	-158.153308304354	-158.153308304354\\
62.75	0.26136	-164.241127554815	-164.241127554815\\
62.75	0.26502	-170.451178469086	-170.451178469086\\
62.75	0.26868	-176.783461047168	-176.783461047168\\
62.75	0.27234	-183.237975289061	-183.237975289061\\
62.75	0.276	-189.814721194765	-189.814721194765\\
63.125	0.093	-10.7916157889788	-10.7916157889788\\
63.125	0.09666	-11.3817890034658	-11.3817890034658\\
63.125	0.10032	-12.0941938817635	-12.0941938817635\\
63.125	0.10398	-12.928830423872	-12.928830423872\\
63.125	0.10764	-13.8856986297912	-13.8856986297912\\
63.125	0.1113	-14.9647984995213	-14.9647984995213\\
63.125	0.11496	-16.166130033062	-16.166130033062\\
63.125	0.11862	-17.4896932304135	-17.4896932304135\\
63.125	0.12228	-18.9354880915758	-18.9354880915758\\
63.125	0.12594	-20.5035146165488	-20.5035146165488\\
63.125	0.1296	-22.1937728053326	-22.1937728053326\\
63.125	0.13326	-24.0062626579271	-24.0062626579271\\
63.125	0.13692	-25.9409841743324	-25.9409841743324\\
63.125	0.14058	-27.9979373545485	-27.9979373545485\\
63.125	0.14424	-30.1771221985753	-30.1771221985753\\
63.125	0.1479	-32.4785387064128	-32.4785387064128\\
63.125	0.15156	-34.9021868780612	-34.9021868780612\\
63.125	0.15522	-37.4480667135203	-37.4480667135203\\
63.125	0.15888	-40.1161782127902	-40.1161782127902\\
63.125	0.16254	-42.9065213758708	-42.9065213758708\\
63.125	0.1662	-45.8190962027621	-45.8190962027621\\
63.125	0.16986	-48.8539026934643	-48.8539026934643\\
63.125	0.17352	-52.0109408479771	-52.0109408479771\\
63.125	0.17718	-55.2902106663008	-55.2902106663008\\
63.125	0.18084	-58.6917121484352	-58.6917121484352\\
63.125	0.1845	-62.2154452943804	-62.2154452943804\\
63.125	0.18816	-65.8614101041363	-65.8614101041363\\
63.125	0.19182	-69.6296065777029	-69.6296065777029\\
63.125	0.19548	-73.5200347150804	-73.5200347150804\\
63.125	0.19914	-77.5326945162685	-77.5326945162685\\
63.125	0.2028	-81.6675859812676	-81.6675859812676\\
63.125	0.20646	-85.9247091100773	-85.9247091100773\\
63.125	0.21012	-90.3040639026977	-90.3040639026977\\
63.125	0.21378	-94.8056503591289	-94.8056503591289\\
63.125	0.21744	-99.429468479371	-99.429468479371\\
63.125	0.2211	-104.175518263424	-104.175518263424\\
63.125	0.22476	-109.043799711287	-109.043799711287\\
63.125	0.22842	-114.034312822961	-114.034312822961\\
63.125	0.23208	-119.147057598446	-119.147057598446\\
63.125	0.23574	-124.382034037742	-124.382034037742\\
63.125	0.2394	-129.739242140849	-129.739242140849\\
63.125	0.24306	-135.218681907766	-135.218681907766\\
63.125	0.24672	-140.820353338494	-140.820353338494\\
63.125	0.25038	-146.544256433033	-146.544256433033\\
63.125	0.25404	-152.390391191383	-152.390391191383\\
63.125	0.2577	-158.358757613543	-158.358757613543\\
63.125	0.26136	-164.449355699514	-164.449355699514\\
63.125	0.26502	-170.662185449296	-170.662185449296\\
63.125	0.26868	-176.997246862888	-176.997246862888\\
63.125	0.27234	-183.454539940292	-183.454539940292\\
63.125	0.276	-190.034064681506	-190.034064681506\\
63.5	0.093	-10.8816138727566	-10.8816138727566\\
63.5	0.09666	-11.474565922754	-11.474565922754\\
63.5	0.10032	-12.189749636562	-12.189749636562\\
63.5	0.10398	-13.0271650141808	-13.0271650141808\\
63.5	0.10764	-13.9868120556104	-13.9868120556104\\
63.5	0.1113	-15.0686907608508	-15.0686907608508\\
63.5	0.11496	-16.2728011299019	-16.2728011299019\\
63.5	0.11862	-17.5991431627637	-17.5991431627637\\
63.5	0.12228	-19.0477168594363	-19.0477168594363\\
63.5	0.12594	-20.6185222199197	-20.6185222199197\\
63.5	0.1296	-22.3115592442138	-22.3115592442138\\
63.5	0.13326	-24.1268279323187	-24.1268279323187\\
63.5	0.13692	-26.0643282842344	-26.0643282842344\\
63.5	0.14058	-28.1240602999608	-28.1240602999608\\
63.5	0.14424	-30.306023979498	-30.306023979498\\
63.5	0.1479	-32.6102193228459	-32.6102193228459\\
63.5	0.15156	-35.0366463300046	-35.0366463300046\\
63.5	0.15522	-37.585305000974	-37.585305000974\\
63.5	0.15888	-40.2561953357542	-40.2561953357542\\
63.5	0.16254	-43.0493173343452	-43.0493173343452\\
63.5	0.1662	-45.9646709967469	-45.9646709967469\\
63.5	0.16986	-49.0022563229594	-49.0022563229594\\
63.5	0.17352	-52.1620733129826	-52.1620733129826\\
63.5	0.17718	-55.4441219668166	-55.4441219668166\\
63.5	0.18084	-58.8484022844613	-58.8484022844613\\
63.5	0.1845	-62.3749142659169	-62.3749142659169\\
63.5	0.18816	-66.0236579111831	-66.0236579111831\\
63.5	0.19182	-69.7946332202602	-69.7946332202602\\
63.5	0.19548	-73.687840193148	-73.687840193148\\
63.5	0.19914	-77.7032788298465	-77.7032788298465\\
63.5	0.2028	-81.8409491303558	-81.8409491303558\\
63.5	0.20646	-86.1008510946759	-86.1008510946759\\
63.5	0.21012	-90.4829847228067	-90.4829847228067\\
63.5	0.21378	-94.9873500147482	-94.9873500147482\\
63.5	0.21744	-99.6139469705006	-99.6139469705006\\
63.5	0.2211	-104.362775590064	-104.362775590064\\
63.5	0.22476	-109.233835873438	-109.233835873438\\
63.5	0.22842	-114.227127820622	-114.227127820622\\
63.5	0.23208	-119.342651431618	-119.342651431618\\
63.5	0.23574	-124.580406706424	-124.580406706424\\
63.5	0.2394	-129.940393645041	-129.940393645041\\
63.5	0.24306	-135.422612247468	-135.422612247468\\
63.5	0.24672	-141.027062513707	-141.027062513707\\
63.5	0.25038	-146.753744443756	-146.753744443756\\
63.5	0.25404	-152.602658037616	-152.602658037616\\
63.5	0.2577	-158.573803295286	-158.573803295286\\
63.5	0.26136	-164.667180216768	-164.667180216768\\
63.5	0.26502	-170.88278880206	-170.88278880206\\
63.5	0.26868	-177.220629051163	-177.220629051163\\
63.5	0.27234	-183.680700964077	-183.680700964077\\
63.5	0.276	-190.263004540801	-190.263004540801\\
63.875	0.093	-10.9812083290888	-10.9812083290888\\
63.875	0.09666	-11.5769392145965	-11.5769392145965\\
63.875	0.10032	-12.2949017639149	-12.2949017639149\\
63.875	0.10398	-13.1350959770441	-13.1350959770441\\
63.875	0.10764	-14.097521853984	-14.097521853984\\
63.875	0.1113	-15.1821793947347	-15.1821793947347\\
63.875	0.11496	-16.3890685992962	-16.3890685992962\\
63.875	0.11862	-17.7181894676684	-17.7181894676684\\
63.875	0.12228	-19.1695419998513	-19.1695419998513\\
63.875	0.12594	-20.743126195845	-20.743126195845\\
63.875	0.1296	-22.4389420556495	-22.4389420556495\\
63.875	0.13326	-24.2569895792647	-24.2569895792647\\
63.875	0.13692	-26.1972687666907	-26.1972687666907\\
63.875	0.14058	-28.2597796179275	-28.2597796179275\\
63.875	0.14424	-30.444522132975	-30.444522132975\\
63.875	0.1479	-32.7514963118333	-32.7514963118333\\
63.875	0.15156	-35.1807021545023	-35.1807021545023\\
63.875	0.15522	-37.7321396609821	-37.7321396609821\\
63.875	0.15888	-40.4058088312727	-40.4058088312727\\
63.875	0.16254	-43.201709665374	-43.201709665374\\
63.875	0.1662	-46.1198421632861	-46.1198421632861\\
63.875	0.16986	-49.1602063250089	-49.1602063250089\\
63.875	0.17352	-52.3228021505424	-52.3228021505424\\
63.875	0.17718	-55.6076296398868	-55.6076296398868\\
63.875	0.18084	-59.0146887930419	-59.0146887930419\\
63.875	0.1845	-62.5439796100077	-62.5439796100077\\
63.875	0.18816	-66.1955020907843	-66.1955020907843\\
63.875	0.19182	-69.9692562353717	-69.9692562353717\\
63.875	0.19548	-73.8652420437698	-73.8652420437698\\
63.875	0.19914	-77.8834595159787	-77.8834595159787\\
63.875	0.2028	-82.0239086519985	-82.0239086519985\\
63.875	0.20646	-86.2865894518289	-86.2865894518289\\
63.875	0.21012	-90.67150191547	-90.67150191547\\
63.875	0.21378	-95.1786460429219	-95.1786460429219\\
63.875	0.21744	-99.8080218341846	-99.8080218341846\\
63.875	0.2211	-104.559629289258	-104.559629289258\\
63.875	0.22476	-109.433468408142	-109.433468408142\\
63.875	0.22842	-114.429539190837	-114.429539190837\\
63.875	0.23208	-119.547841637343	-119.547841637343\\
63.875	0.23574	-124.78837574766	-124.78837574766\\
63.875	0.2394	-130.151141521787	-130.151141521787\\
63.875	0.24306	-135.636138959725	-135.636138959725\\
63.875	0.24672	-141.243368061473	-141.243368061473\\
63.875	0.25038	-146.972828827033	-146.972828827033\\
63.875	0.25404	-152.824521256403	-152.824521256403\\
63.875	0.2577	-158.798445349584	-158.798445349584\\
63.875	0.26136	-164.894601106576	-164.894601106576\\
63.875	0.26502	-171.112988527379	-171.112988527379\\
63.875	0.26868	-177.453607611992	-177.453607611992\\
63.875	0.27234	-183.916458360416	-183.916458360416\\
63.875	0.276	-190.501540772651	-190.501540772651\\
64.25	0.093	-11.0903991579754	-11.0903991579754\\
64.25	0.09666	-11.6889088789934	-11.6889088789934\\
64.25	0.10032	-12.4096502638221	-12.4096502638221\\
64.25	0.10398	-13.2526233124617	-13.2526233124617\\
64.25	0.10764	-14.2178280249119	-14.2178280249119\\
64.25	0.1113	-15.305264401173	-15.305264401173\\
64.25	0.11496	-16.5149324412448	-16.5149324412448\\
64.25	0.11862	-17.8468321451273	-17.8468321451273\\
64.25	0.12228	-19.3009635128206	-19.3009635128206\\
64.25	0.12594	-20.8773265443247	-20.8773265443247\\
64.25	0.1296	-22.5759212396395	-22.5759212396395\\
64.25	0.13326	-24.3967475987651	-24.3967475987651\\
64.25	0.13692	-26.3398056217015	-26.3398056217015\\
64.25	0.14058	-28.4050953084486	-28.4050953084486\\
64.25	0.14424	-30.5926166590064	-30.5926166590064\\
64.25	0.1479	-32.9023696733751	-32.9023696733751\\
64.25	0.15156	-35.3343543515544	-35.3343543515544\\
64.25	0.15522	-37.8885706935446	-37.8885706935446\\
64.25	0.15888	-40.5650186993455	-40.5650186993455\\
64.25	0.16254	-43.3636983689571	-43.3636983689571\\
64.25	0.1662	-46.2846097023795	-46.2846097023795\\
64.25	0.16986	-49.3277526996127	-49.3277526996127\\
64.25	0.17352	-52.4931273606566	-52.4931273606566\\
64.25	0.17718	-55.7807336855113	-55.7807336855113\\
64.25	0.18084	-59.1905716741768	-59.1905716741768\\
64.25	0.1845	-62.722641326653	-62.722641326653\\
64.25	0.18816	-66.3769426429399	-66.3769426429399\\
64.25	0.19182	-70.1534756230377	-70.1534756230377\\
64.25	0.19548	-74.0522402669461	-74.0522402669461\\
64.25	0.19914	-78.0732365746653	-78.0732365746653\\
64.25	0.2028	-82.2164645461954	-82.2164645461954\\
64.25	0.20646	-86.4819241815362	-86.4819241815362\\
64.25	0.21012	-90.8696154806877	-90.8696154806877\\
64.25	0.21378	-95.3795384436499	-95.3795384436499\\
64.25	0.21744	-100.011693070423	-100.011693070423\\
64.25	0.2211	-104.766079361007	-104.766079361007\\
64.25	0.22476	-109.642697315401	-109.642697315401\\
64.25	0.22842	-114.641546933607	-114.641546933607\\
64.25	0.23208	-119.762628215623	-119.762628215623\\
64.25	0.23574	-125.00594116145	-125.00594116145\\
64.25	0.2394	-130.371485771087	-130.371485771087\\
64.25	0.24306	-135.859262044535	-135.859262044535\\
64.25	0.24672	-141.469269981795	-141.469269981795\\
64.25	0.25038	-147.201509582864	-147.201509582864\\
64.25	0.25404	-153.055980847745	-153.055980847745\\
64.25	0.2577	-159.032683776436	-159.032683776436\\
64.25	0.26136	-165.131618368939	-165.131618368939\\
64.25	0.26502	-171.352784625251	-171.352784625251\\
64.25	0.26868	-177.696182545375	-177.696182545375\\
64.25	0.27234	-184.16181212931	-184.16181212931\\
64.25	0.276	-190.749673377055	-190.749673377055\\
64.625	0.093	-11.2091863594163	-11.2091863594163\\
64.625	0.09666	-11.8104749159447	-11.8104749159447\\
64.625	0.10032	-12.5339951362838	-12.5339951362838\\
64.625	0.10398	-13.3797470204337	-13.3797470204337\\
64.625	0.10764	-14.3477305683943	-14.3477305683943\\
64.625	0.1113	-15.4379457801657	-15.4379457801657\\
64.625	0.11496	-16.6503926557478	-16.6503926557478\\
64.625	0.11862	-17.9850711951407	-17.9850711951407\\
64.625	0.12228	-19.4419813983444	-19.4419813983444\\
64.625	0.12594	-21.0211232653588	-21.0211232653588\\
64.625	0.1296	-22.722496796184	-22.722496796184\\
64.625	0.13326	-24.5461019908199	-24.5461019908199\\
64.625	0.13692	-26.4919388492666	-26.4919388492666\\
64.625	0.14058	-28.560007371524	-28.560007371524\\
64.625	0.14424	-30.7503075575923	-30.7503075575923\\
64.625	0.1479	-33.0628394074712	-33.0628394074712\\
64.625	0.15156	-35.497602921161	-35.497602921161\\
64.625	0.15522	-38.0545980986615	-38.0545980986615\\
64.625	0.15888	-40.7338249399727	-40.7338249399727\\
64.625	0.16254	-43.5352834450947	-43.5352834450947\\
64.625	0.1662	-46.4589736140275	-46.4589736140275\\
64.625	0.16986	-49.504895446771	-49.504895446771\\
64.625	0.17352	-52.6730489433252	-52.6730489433252\\
64.625	0.17718	-55.9634341036903	-55.9634341036903\\
64.625	0.18084	-59.3760509278661	-59.3760509278661\\
64.625	0.1845	-62.9108994158527	-62.9108994158527\\
64.625	0.18816	-66.56797956765	-66.56797956765\\
64.625	0.19182	-70.3472913832581	-70.3472913832581\\
64.625	0.19548	-74.2488348626769	-74.2488348626769\\
64.625	0.19914	-78.2726100059065	-78.2726100059065\\
64.625	0.2028	-82.4186168129468	-82.4186168129468\\
64.625	0.20646	-86.6868552837979	-86.6868552837979\\
64.625	0.21012	-91.0773254184598	-91.0773254184598\\
64.625	0.21378	-95.5900272169324	-95.5900272169324\\
64.625	0.21744	-100.224960679216	-100.224960679216\\
64.625	0.2211	-104.98212580531	-104.98212580531\\
64.625	0.22476	-109.861522595215	-109.861522595215\\
64.625	0.22842	-114.86315104893	-114.86315104893\\
64.625	0.23208	-119.987011166457	-119.987011166457\\
64.625	0.23574	-125.233102947794	-125.233102947794\\
64.625	0.2394	-130.601426392942	-130.601426392942\\
64.625	0.24306	-136.091981501901	-136.091981501901\\
64.625	0.24672	-141.70476827467	-141.70476827467\\
64.625	0.25038	-147.43978671125	-147.43978671125\\
64.625	0.25404	-153.297036811641	-153.297036811641\\
64.625	0.2577	-159.276518575843	-159.276518575843\\
64.625	0.26136	-165.378232003856	-165.378232003856\\
64.625	0.26502	-171.602177095679	-171.602177095679\\
64.625	0.26868	-177.948353851313	-177.948353851313\\
64.625	0.27234	-184.416762270758	-184.416762270758\\
64.625	0.276	-191.007402354013	-191.007402354013\\
65	0.093	-11.3375699334117	-11.3375699334117\\
65	0.09666	-11.9416373254504	-11.9416373254504\\
65	0.10032	-12.6679363812999	-12.6679363812999\\
65	0.10398	-13.5164671009601	-13.5164671009601\\
65	0.10764	-14.4872294844311	-14.4872294844311\\
65	0.1113	-15.5802235317128	-15.5802235317128\\
65	0.11496	-16.7954492428053	-16.7954492428053\\
65	0.11862	-18.1329066177086	-18.1329066177086\\
65	0.12228	-19.5925956564226	-19.5925956564226\\
65	0.12594	-21.1745163589473	-21.1745163589473\\
65	0.1296	-22.8786687252828	-22.8786687252828\\
65	0.13326	-24.7050527554291	-24.7050527554291\\
65	0.13692	-26.6536684493861	-26.6536684493861\\
65	0.14058	-28.724515807154	-28.724515807154\\
65	0.14424	-30.9175948287325	-30.9175948287325\\
65	0.1479	-33.2329055141219	-33.2329055141219\\
65	0.15156	-35.6704478633219	-35.6704478633219\\
65	0.15522	-38.2302218763327	-38.2302218763327\\
65	0.15888	-40.9122275531544	-40.9122275531544\\
65	0.16254	-43.7164648937867	-43.7164648937867\\
65	0.1662	-46.6429338982298	-46.6429338982298\\
65	0.16986	-49.6916345664837	-49.6916345664837\\
65	0.17352	-52.8625668985483	-52.8625668985483\\
65	0.17718	-56.1557308944237	-56.1557308944237\\
65	0.18084	-59.5711265541098	-59.5711265541098\\
65	0.1845	-63.1087538776067	-63.1087538776067\\
65	0.18816	-66.7686128649144	-66.7686128649144\\
65	0.19182	-70.5507035160328	-70.5507035160328\\
65	0.19548	-74.4550258309619	-74.4550258309619\\
65	0.19914	-78.4815798097019	-78.4815798097019\\
65	0.2028	-82.6303654522526	-82.6303654522526\\
65	0.20646	-86.9013827586141	-86.9013827586141\\
65	0.21012	-91.2946317287863	-91.2946317287863\\
65	0.21378	-95.8101123627692	-95.8101123627692\\
65	0.21744	-100.447824660563	-100.447824660563\\
65	0.2211	-105.207768622167	-105.207768622167\\
65	0.22476	-110.089944247583	-110.089944247583\\
65	0.22842	-115.094351536809	-115.094351536809\\
65	0.23208	-120.220990489845	-120.220990489845\\
65	0.23574	-125.469861106693	-125.469861106693\\
65	0.2394	-130.840963387351	-130.840963387351\\
65	0.24306	-136.33429733182	-136.33429733182\\
65	0.24672	-141.9498629401	-141.9498629401\\
65	0.25038	-147.687660212191	-147.687660212191\\
65	0.25404	-153.547689148092	-153.547689148092\\
65	0.2577	-159.529949747804	-159.529949747804\\
65	0.26136	-165.634442011327	-165.634442011327\\
65	0.26502	-171.861165938661	-171.861165938661\\
65	0.26868	-178.210121529805	-178.210121529805\\
65	0.27234	-184.68130878476	-184.68130878476\\
65	0.276	-191.274727703526	-191.274727703526\\
65.375	0.093	-11.4755498799614	-11.4755498799614\\
65.375	0.09666	-12.0823961075105	-12.0823961075105\\
65.375	0.10032	-12.8114739988703	-12.8114739988703\\
65.375	0.10398	-13.6627835540409	-13.6627835540409\\
65.375	0.10764	-14.6363247730222	-14.6363247730222\\
65.375	0.1113	-15.7320976558143	-15.7320976558143\\
65.375	0.11496	-16.9501022024171	-16.9501022024171\\
65.375	0.11862	-18.2903384128307	-18.2903384128307\\
65.375	0.12228	-19.7528062870551	-19.7528062870551\\
65.375	0.12594	-21.3375058250902	-21.3375058250902\\
65.375	0.1296	-23.0444370269361	-23.0444370269361\\
65.375	0.13326	-24.8735998925927	-24.8735998925927\\
65.375	0.13692	-26.8249944220601	-26.8249944220601\\
65.375	0.14058	-28.8986206153382	-28.8986206153382\\
65.375	0.14424	-31.0944784724271	-31.0944784724271\\
65.375	0.1479	-33.4125679933268	-33.4125679933268\\
65.375	0.15156	-35.8528891780372	-35.8528891780372\\
65.375	0.15522	-38.4154420265584	-38.4154420265584\\
65.375	0.15888	-41.1002265388904	-41.1002265388904\\
65.375	0.16254	-43.907242715033	-43.907242715033\\
65.375	0.1662	-46.8364905549865	-46.8364905549865\\
65.375	0.16986	-49.8879700587507	-49.8879700587507\\
65.375	0.17352	-53.0616812263257	-53.0616812263257\\
65.375	0.17718	-56.3576240577114	-56.3576240577114\\
65.375	0.18084	-59.7757985529079	-59.7757985529079\\
65.375	0.1845	-63.3162047119152	-63.3162047119152\\
65.375	0.18816	-66.9788425347331	-66.9788425347331\\
65.375	0.19182	-70.7637120213619	-70.7637120213619\\
65.375	0.19548	-74.6708131718014	-74.6708131718014\\
65.375	0.19914	-78.7001459860517	-78.7001459860517\\
65.375	0.2028	-82.8517104641128	-82.8517104641128\\
65.375	0.20646	-87.1255066059846	-87.1255066059846\\
65.375	0.21012	-91.5215344116672	-91.5215344116672\\
65.375	0.21378	-96.0397938811605	-96.0397938811605\\
65.375	0.21744	-100.680285014465	-100.680285014465\\
65.375	0.2211	-105.443007811579	-105.443007811579\\
65.375	0.22476	-110.327962272505	-110.327962272505\\
65.375	0.22842	-115.335148397241	-115.335148397241\\
65.375	0.23208	-120.464566185788	-120.464566185788\\
65.375	0.23574	-125.716215638146	-125.716215638146\\
65.375	0.2394	-131.090096754315	-131.090096754315\\
65.375	0.24306	-136.586209534294	-136.586209534294\\
65.375	0.24672	-142.204553978085	-142.204553978085\\
65.375	0.25038	-147.945130085685	-147.945130085685\\
65.375	0.25404	-153.807937857097	-153.807937857097\\
65.375	0.2577	-159.79297729232	-159.79297729232\\
65.375	0.26136	-165.900248391353	-165.900248391353\\
65.375	0.26502	-172.129751154197	-172.129751154197\\
65.375	0.26868	-178.481485580851	-178.481485580851\\
65.375	0.27234	-184.955451671317	-184.955451671317\\
65.375	0.276	-191.551649425593	-191.551649425593\\
65.75	0.093	-11.6231261990656	-11.6231261990656\\
65.75	0.09666	-12.232751262125	-12.232751262125\\
65.75	0.10032	-12.9646079889952	-12.9646079889952\\
65.75	0.10398	-13.8186963796761	-13.8186963796761\\
65.75	0.10764	-14.7950164341677	-14.7950164341677\\
65.75	0.1113	-15.8935681524702	-15.8935681524702\\
65.75	0.11496	-17.1143515345834	-17.1143515345834\\
65.75	0.11862	-18.4573665805073	-18.4573665805073\\
65.75	0.12228	-19.922613290242	-19.922613290242\\
65.75	0.12594	-21.5100916637875	-21.5100916637875\\
65.75	0.1296	-23.2198017011437	-23.2198017011437\\
65.75	0.13326	-25.0517434023107	-25.0517434023107\\
65.75	0.13692	-27.0059167672884	-27.0059167672884\\
65.75	0.14058	-29.0823217960769	-29.0823217960769\\
65.75	0.14424	-31.2809584886762	-31.2809584886762\\
65.75	0.1479	-33.6018268450862	-33.6018268450862\\
65.75	0.15156	-36.0449268653069	-36.0449268653069\\
65.75	0.15522	-38.6102585493385	-38.6102585493385\\
65.75	0.15888	-41.2978218971808	-41.2978218971808\\
65.75	0.16254	-44.1076169088338	-44.1076169088338\\
65.75	0.1662	-47.0396435842976	-47.0396435842976\\
65.75	0.16986	-50.0939019235722	-50.0939019235722\\
65.75	0.17352	-53.2703919266575	-53.2703919266575\\
65.75	0.17718	-56.5691135935536	-56.5691135935536\\
65.75	0.18084	-59.9900669242604	-59.9900669242604\\
65.75	0.1845	-63.533251918778	-63.533251918778\\
65.75	0.18816	-67.1986685771064	-67.1986685771064\\
65.75	0.19182	-70.9863168992455	-70.9863168992455\\
65.75	0.19548	-74.8961968851954	-74.8961968851954\\
65.75	0.19914	-78.928308534956	-78.928308534956\\
65.75	0.2028	-83.0826518485274	-83.0826518485274\\
65.75	0.20646	-87.3592268259096	-87.3592268259096\\
65.75	0.21012	-91.7580334671025	-91.7580334671025\\
65.75	0.21378	-96.2790717721061	-96.2790717721061\\
65.75	0.21744	-100.922341740921	-100.922341740921\\
65.75	0.2211	-105.687843373546	-105.687843373546\\
65.75	0.22476	-110.575576669982	-110.575576669982\\
65.75	0.22842	-115.585541630228	-115.585541630228\\
65.75	0.23208	-120.717738254286	-120.717738254286\\
65.75	0.23574	-125.972166542154	-125.972166542154\\
65.75	0.2394	-131.348826493833	-131.348826493833\\
65.75	0.24306	-136.847718109323	-136.847718109323\\
65.75	0.24672	-142.468841388623	-142.468841388623\\
65.75	0.25038	-148.212196331735	-148.212196331735\\
65.75	0.25404	-154.077782938657	-154.077782938657\\
65.75	0.2577	-160.065601209389	-160.065601209389\\
65.75	0.26136	-166.175651143933	-166.175651143933\\
65.75	0.26502	-172.407932742287	-172.407932742287\\
65.75	0.26868	-178.762446004452	-178.762446004452\\
65.75	0.27234	-185.239190930428	-185.239190930428\\
65.75	0.276	-191.838167520215	-191.838167520215\\
66.125	0.093	-11.7802988907242	-11.7802988907242\\
66.125	0.09666	-12.3927027892939	-12.3927027892939\\
66.125	0.10032	-13.1273383516745	-13.1273383516745\\
66.125	0.10398	-13.9842055778657	-13.9842055778657\\
66.125	0.10764	-14.9633044678677	-14.9633044678677\\
66.125	0.1113	-16.0646350216805	-16.0646350216805\\
66.125	0.11496	-17.288197239304	-17.288197239304\\
66.125	0.11862	-18.6339911207383	-18.6339911207383\\
66.125	0.12228	-20.1020166659834	-20.1020166659834\\
66.125	0.12594	-21.6922738750392	-21.6922738750392\\
66.125	0.1296	-23.4047627479058	-23.4047627479058\\
66.125	0.13326	-25.2394832845831	-25.2394832845831\\
66.125	0.13692	-27.1964354850712	-27.1964354850712\\
66.125	0.14058	-29.27561934937	-29.27561934937\\
66.125	0.14424	-31.4770348774796	-31.4770348774796\\
66.125	0.1479	-33.8006820694	-33.8006820694\\
66.125	0.15156	-36.2465609251311	-36.2465609251311\\
66.125	0.15522	-38.814671444673	-38.814671444673\\
66.125	0.15888	-41.5050136280256	-41.5050136280256\\
66.125	0.16254	-44.317587475189	-44.317587475189\\
66.125	0.1662	-47.2523929861632	-47.2523929861632\\
66.125	0.16986	-50.3094301609481	-50.3094301609481\\
66.125	0.17352	-53.4886989995437	-53.4886989995437\\
66.125	0.17718	-56.7901995019502	-56.7901995019502\\
66.125	0.18084	-60.2139316681673	-60.2139316681673\\
66.125	0.1845	-63.7598954981953	-63.7598954981953\\
66.125	0.18816	-67.428090992034	-67.428090992034\\
66.125	0.19182	-71.2185181496835	-71.2185181496835\\
66.125	0.19548	-75.1311769711437	-75.1311769711437\\
66.125	0.19914	-79.1660674564147	-79.1660674564147\\
66.125	0.2028	-83.3231896054965	-83.3231896054965\\
66.125	0.20646	-87.602543418389	-87.602543418389\\
66.125	0.21012	-92.0041288950922	-92.0041288950922\\
66.125	0.21378	-96.5279460356062	-96.5279460356062\\
66.125	0.21744	-101.173994839931	-101.173994839931\\
66.125	0.2211	-105.942275308067	-105.942275308067\\
66.125	0.22476	-110.832787440013	-110.832787440013\\
66.125	0.22842	-115.84553123577	-115.84553123577\\
66.125	0.23208	-120.980506695338	-120.980506695338\\
66.125	0.23574	-126.237713818716	-126.237713818716\\
66.125	0.2394	-131.617152605906	-131.617152605906\\
66.125	0.24306	-137.118823056906	-137.118823056906\\
66.125	0.24672	-142.742725171716	-142.742725171716\\
66.125	0.25038	-148.488858950338	-148.488858950338\\
66.125	0.25404	-154.35722439277	-154.35722439277\\
66.125	0.2577	-160.347821499014	-160.347821499014\\
66.125	0.26136	-166.460650269067	-166.460650269067\\
66.125	0.26502	-172.695710702932	-172.695710702932\\
66.125	0.26868	-179.053002800607	-179.053002800607\\
66.125	0.27234	-185.532526562094	-185.532526562094\\
66.125	0.276	-192.13428198739	-192.13428198739\\
66.5	0.093	-11.9470679549371	-11.9470679549371\\
66.5	0.09666	-12.5622506890172	-12.5622506890172\\
66.5	0.10032	-13.299665086908	-13.299665086908\\
66.5	0.10398	-14.1593111486096	-14.1593111486096\\
66.5	0.10764	-15.141188874122	-15.141188874122\\
66.5	0.1113	-16.2452982634452	-16.2452982634452\\
66.5	0.11496	-17.4716393165791	-17.4716393165791\\
66.5	0.11862	-18.8202120335237	-18.8202120335237\\
66.5	0.12228	-20.2910164142791	-20.2910164142791\\
66.5	0.12594	-21.8840524588452	-21.8840524588452\\
66.5	0.1296	-23.5993201672221	-23.5993201672221\\
66.5	0.13326	-25.4368195394098	-25.4368195394098\\
66.5	0.13692	-27.3965505754082	-27.3965505754082\\
66.5	0.14058	-29.4785132752174	-29.4785132752174\\
66.5	0.14424	-31.6827076388374	-31.6827076388374\\
66.5	0.1479	-34.0091336662681	-34.0091336662681\\
66.5	0.15156	-36.4577913575096	-36.4577913575096\\
66.5	0.15522	-39.0286807125618	-39.0286807125618\\
66.5	0.15888	-41.7218017314248	-41.7218017314248\\
66.5	0.16254	-44.5371544140985	-44.5371544140985\\
66.5	0.1662	-47.474738760583	-47.474738760583\\
66.5	0.16986	-50.5345547708783	-50.5345547708783\\
66.5	0.17352	-53.7166024449843	-53.7166024449843\\
66.5	0.17718	-57.0208817829011	-57.0208817829011\\
66.5	0.18084	-60.4473927846286	-60.4473927846286\\
66.5	0.1845	-63.9961354501669	-63.9961354501669\\
66.5	0.18816	-67.6671097795159	-67.6671097795159\\
66.5	0.19182	-71.4603157726757	-71.4603157726757\\
66.5	0.19548	-75.3757534296463	-75.3757534296463\\
66.5	0.19914	-79.4134227504276	-79.4134227504276\\
66.5	0.2028	-83.5733237350198	-83.5733237350198\\
66.5	0.20646	-87.8554563834226	-87.8554563834226\\
66.5	0.21012	-92.2598206956362	-92.2598206956362\\
66.5	0.21378	-96.7864166716606	-96.7864166716606\\
66.5	0.21744	-101.435244311496	-101.435244311496\\
66.5	0.2211	-106.206303615142	-106.206303615142\\
66.5	0.22476	-111.099594582598	-111.099594582598\\
66.5	0.22842	-116.115117213866	-116.115117213866\\
66.5	0.23208	-121.252871508944	-121.252871508944\\
66.5	0.23574	-126.512857467833	-126.512857467833\\
66.5	0.2394	-131.895075090532	-131.895075090532\\
66.5	0.24306	-137.399524377043	-137.399524377043\\
66.5	0.24672	-143.026205327364	-143.026205327364\\
66.5	0.25038	-148.775117941496	-148.775117941496\\
66.5	0.25404	-154.646262219439	-154.646262219439\\
66.5	0.2577	-160.639638161192	-160.639638161192\\
66.5	0.26136	-166.755245766756	-166.755245766756\\
66.5	0.26502	-172.993085036131	-172.993085036131\\
66.5	0.26868	-179.353155969317	-179.353155969317\\
66.5	0.27234	-185.835458566314	-185.835458566314\\
66.5	0.276	-192.439992827121	-192.439992827121\\
66.875	0.093	-12.1234333917045	-12.1234333917045\\
66.875	0.09666	-12.7413949612949	-12.7413949612949\\
66.875	0.10032	-13.4815881946961	-13.4815881946961\\
66.875	0.10398	-14.344013091908	-14.344013091908\\
66.875	0.10764	-15.3286696529308	-15.3286696529308\\
66.875	0.1113	-16.4355578777643	-16.4355578777643\\
66.875	0.11496	-17.6646777664085	-17.6646777664085\\
66.875	0.11862	-19.0160293188634	-19.0160293188634\\
66.875	0.12228	-20.4896125351292	-20.4896125351292\\
66.875	0.12594	-22.0854274152057	-22.0854274152057\\
66.875	0.1296	-23.803473959093	-23.803473959093\\
66.875	0.13326	-25.643752166791	-25.643752166791\\
66.875	0.13692	-27.6062620382997	-27.6062620382997\\
66.875	0.14058	-29.6910035736193	-29.6910035736193\\
66.875	0.14424	-31.8979767727496	-31.8979767727496\\
66.875	0.1479	-34.2271816356907	-34.2271816356907\\
66.875	0.15156	-36.6786181624425	-36.6786181624425\\
66.875	0.15522	-39.2522863530051	-39.2522863530051\\
66.875	0.15888	-41.9481862073784	-41.9481862073784\\
66.875	0.16254	-44.7663177255625	-44.7663177255625\\
66.875	0.1662	-47.7066809075574	-47.7066809075574\\
66.875	0.16986	-50.7692757533629	-50.7692757533629\\
66.875	0.17352	-53.9541022629793	-53.9541022629793\\
66.875	0.17718	-57.2611604364064	-57.2611604364064\\
66.875	0.18084	-60.6904502736443	-60.6904502736443\\
66.875	0.1845	-64.241971774693	-64.241971774693\\
66.875	0.18816	-67.9157249395524	-67.9157249395524\\
66.875	0.19182	-71.7117097682225	-71.7117097682225\\
66.875	0.19548	-75.6299262607035	-75.6299262607035\\
66.875	0.19914	-79.6703744169951	-79.6703744169951\\
66.875	0.2028	-83.8330542370976	-83.8330542370976\\
66.875	0.20646	-88.1179657210108	-88.1179657210108\\
66.875	0.21012	-92.5251088687347	-92.5251088687347\\
66.875	0.21378	-97.0544836802694	-97.0544836802694\\
66.875	0.21744	-101.706090155615	-101.706090155615\\
66.875	0.2211	-106.479928294771	-106.479928294771\\
66.875	0.22476	-111.375998097738	-111.375998097738\\
66.875	0.22842	-116.394299564516	-116.394299564516\\
66.875	0.23208	-121.534832695104	-121.534832695104\\
66.875	0.23574	-126.797597489504	-126.797597489504\\
66.875	0.2394	-132.182593947714	-132.182593947714\\
66.875	0.24306	-137.689822069734	-137.689822069734\\
66.875	0.24672	-143.319281855566	-143.319281855566\\
66.875	0.25038	-149.070973305208	-149.070973305208\\
66.875	0.25404	-154.944896418661	-154.944896418661\\
66.875	0.2577	-160.941051195925	-160.941051195925\\
66.875	0.26136	-167.059437637	-167.059437637\\
66.875	0.26502	-173.300055741885	-173.300055741885\\
66.875	0.26868	-179.662905510581	-179.662905510581\\
66.875	0.27234	-186.147986943088	-186.147986943088\\
66.875	0.276	-192.755300039406	-192.755300039406\\
67.25	0.093	-12.3093952010262	-12.3093952010262\\
67.25	0.09666	-12.930135606127	-12.930135606127\\
67.25	0.10032	-13.6731076750386	-13.6731076750386\\
67.25	0.10398	-14.5383114077609	-14.5383114077609\\
67.25	0.10764	-15.5257468042939	-15.5257468042939\\
67.25	0.1113	-16.6354138646378	-16.6354138646378\\
67.25	0.11496	-17.8673125887923	-17.8673125887923\\
67.25	0.11862	-19.2214429767577	-19.2214429767577\\
67.25	0.12228	-20.6978050285338	-20.6978050285338\\
67.25	0.12594	-22.2963987441206	-22.2963987441206\\
67.25	0.1296	-24.0172241235182	-24.0172241235182\\
67.25	0.13326	-25.8602811667266	-25.8602811667266\\
67.25	0.13692	-27.8255698737457	-27.8255698737457\\
67.25	0.14058	-29.9130902445756	-29.9130902445756\\
67.25	0.14424	-32.1228422792163	-32.1228422792163\\
67.25	0.1479	-34.4548259776677	-34.4548259776677\\
67.25	0.15156	-36.9090413399298	-36.9090413399298\\
67.25	0.15522	-39.4854883660027	-39.4854883660027\\
67.25	0.15888	-42.1841670558864	-42.1841670558864\\
67.25	0.16254	-45.0050774095809	-45.0050774095809\\
67.25	0.1662	-47.9482194270861	-47.9482194270861\\
67.25	0.16986	-51.0135931084021	-51.0135931084021\\
67.25	0.17352	-54.2011984535287	-54.2011984535287\\
67.25	0.17718	-57.5110354624662	-57.5110354624662\\
67.25	0.18084	-60.9431041352144	-60.9431041352144\\
67.25	0.1845	-64.4974044717735	-64.4974044717735\\
67.25	0.18816	-68.1739364721431	-68.1739364721431\\
67.25	0.19182	-71.9727001363237	-71.9727001363237\\
67.25	0.19548	-75.8936954643149	-75.8936954643149\\
67.25	0.19914	-79.9369224561169	-79.9369224561169\\
67.25	0.2028	-84.1023811117298	-84.1023811117298\\
67.25	0.20646	-88.3900714311533	-88.3900714311533\\
67.25	0.21012	-92.7999934143876	-92.7999934143876\\
67.25	0.21378	-97.3321470614327	-97.3321470614327\\
67.25	0.21744	-101.986532372288	-101.986532372288\\
67.25	0.2211	-106.763149346955	-106.763149346955\\
67.25	0.22476	-111.661997985432	-111.661997985432\\
67.25	0.22842	-116.68307828772	-116.68307828772\\
67.25	0.23208	-121.826390253819	-121.826390253819\\
67.25	0.23574	-127.091933883729	-127.091933883729\\
67.25	0.2394	-132.479709177449	-132.479709177449\\
67.25	0.24306	-137.98971613498	-137.98971613498\\
67.25	0.24672	-143.621954756322	-143.621954756322\\
67.25	0.25038	-149.376425041475	-149.376425041475\\
67.25	0.25404	-155.253126990438	-155.253126990438\\
67.25	0.2577	-161.252060603213	-161.252060603213\\
67.25	0.26136	-167.373225879797	-167.373225879797\\
67.25	0.26502	-173.616622820193	-173.616622820193\\
67.25	0.26868	-179.9822514244	-179.9822514244\\
67.25	0.27234	-186.470111692417	-186.470111692417\\
67.25	0.276	-193.080203624245	-193.080203624245\\
67.625	0.093	-12.5049533829023	-12.5049533829023\\
67.625	0.09666	-13.1284726235134	-13.1284726235134\\
67.625	0.10032	-13.8742235279354	-13.8742235279354\\
67.625	0.10398	-14.742206096168	-14.742206096168\\
67.625	0.10764	-15.7324203282114	-15.7324203282114\\
67.625	0.1113	-16.8448662240656	-16.8448662240656\\
67.625	0.11496	-18.0795437837305	-18.0795437837305\\
67.625	0.11862	-19.4364530072062	-19.4364530072062\\
67.625	0.12228	-20.9155938944926	-20.9155938944926\\
67.625	0.12594	-22.5169664455898	-22.5169664455898\\
67.625	0.1296	-24.2405706604978	-24.2405706604978\\
67.625	0.13326	-26.0864065392165	-26.0864065392165\\
67.625	0.13692	-28.054474081746	-28.054474081746\\
67.625	0.14058	-30.1447732880862	-30.1447732880862\\
67.625	0.14424	-32.3573041582372	-32.3573041582372\\
67.625	0.1479	-34.692066692199	-34.692066692199\\
67.625	0.15156	-37.1490608899715	-37.1490608899715\\
67.625	0.15522	-39.7282867515548	-39.7282867515548\\
67.625	0.15888	-42.4297442769488	-42.4297442769488\\
67.625	0.16254	-45.2534334661536	-45.2534334661536\\
67.625	0.1662	-48.1993543191692	-48.1993543191692\\
67.625	0.16986	-51.2675068359955	-51.2675068359955\\
67.625	0.17352	-54.4578910166325	-54.4578910166325\\
67.625	0.17718	-57.7705068610803	-57.7705068610803\\
67.625	0.18084	-61.2053543693389	-61.2053543693389\\
67.625	0.1845	-64.7624335414082	-64.7624335414082\\
67.625	0.18816	-68.4417443772883	-68.4417443772883\\
67.625	0.19182	-72.2432868769791	-72.2432868769791\\
67.625	0.19548	-76.1670610404807	-76.1670610404807\\
67.625	0.19914	-80.2130668677931	-80.2130668677931\\
67.625	0.2028	-84.3813043589163	-84.3813043589163\\
67.625	0.20646	-88.6717735138502	-88.6717735138502\\
67.625	0.21012	-93.0844743325949	-93.0844743325949\\
67.625	0.21378	-97.6194068151503	-97.6194068151503\\
67.625	0.21744	-102.276570961516	-102.276570961516\\
67.625	0.2211	-107.055966771693	-107.055966771693\\
67.625	0.22476	-111.957594245681	-111.957594245681\\
67.625	0.22842	-116.981453383479	-116.981453383479\\
67.625	0.23208	-122.127544185089	-122.127544185089\\
67.625	0.23574	-127.395866650509	-127.395866650509\\
67.625	0.2394	-132.786420779739	-132.786420779739\\
67.625	0.24306	-138.299206572781	-138.299206572781\\
67.625	0.24672	-143.934224029633	-143.934224029633\\
67.625	0.25038	-149.691473150296	-149.691473150296\\
67.625	0.25404	-155.57095393477	-155.57095393477\\
67.625	0.2577	-161.572666383054	-161.572666383054\\
67.625	0.26136	-167.69661049515	-167.69661049515\\
67.625	0.26502	-173.942786271056	-173.942786271056\\
67.625	0.26868	-180.311193710772	-180.311193710772\\
67.625	0.27234	-186.8018328143	-186.8018328143\\
67.625	0.276	-193.414703581638	-193.414703581638\\
68	0.093	-12.7101079373329	-12.7101079373329\\
68	0.09666	-13.3364060134543	-13.3364060134543\\
68	0.10032	-14.0849357533866	-14.0849357533866\\
68	0.10398	-14.9556971571296	-14.9556971571296\\
68	0.10764	-15.9486902246833	-15.9486902246833\\
68	0.1113	-17.0639149560479	-17.0639149560479\\
68	0.11496	-18.3013713512232	-18.3013713512232\\
68	0.11862	-19.6610594102092	-19.6610594102092\\
68	0.12228	-21.142979133006	-21.142979133006\\
68	0.12594	-22.7471305196135	-22.7471305196135\\
68	0.1296	-24.4735135700318	-24.4735135700318\\
68	0.13326	-26.3221282842609	-26.3221282842609\\
68	0.13692	-28.2929746623007	-28.2929746623007\\
68	0.14058	-30.3860527041513	-30.3860527041513\\
68	0.14424	-32.6013624098126	-32.6013624098126\\
68	0.1479	-34.9389037792847	-34.9389037792847\\
68	0.15156	-37.3986768125676	-37.3986768125676\\
68	0.15522	-39.9806815096612	-39.9806815096612\\
68	0.15888	-42.6849178705656	-42.6849178705656\\
68	0.16254	-45.5113858952807	-45.5113858952807\\
68	0.1662	-48.4600855838066	-48.4600855838066\\
68	0.16986	-51.5310169361433	-51.5310169361433\\
68	0.17352	-54.7241799522907	-54.7241799522907\\
68	0.17718	-58.0395746322488	-58.0395746322488\\
68	0.18084	-61.4772009760178	-61.4772009760178\\
68	0.1845	-65.0370589835975	-65.0370589835975\\
68	0.18816	-68.7191486549879	-68.7191486549879\\
68	0.19182	-72.5234699901891	-72.5234699901891\\
68	0.19548	-76.4500229892011	-76.4500229892011\\
68	0.19914	-80.4988076520238	-80.4988076520238\\
68	0.2028	-84.6698239786573	-84.6698239786573\\
68	0.20646	-88.9630719691015	-88.9630719691015\\
68	0.21012	-93.3785516233566	-93.3785516233566\\
68	0.21378	-97.9162629414223	-97.9162629414223\\
68	0.21744	-102.576205923299	-102.576205923299\\
68	0.2211	-107.358380568986	-107.358380568986\\
68	0.22476	-112.262786878484	-112.262786878484\\
68	0.22842	-117.289424851793	-117.289424851793\\
68	0.23208	-122.438294488912	-122.438294488912\\
68	0.23574	-127.709395789843	-127.709395789843\\
68	0.2394	-133.102728754584	-133.102728754584\\
68	0.24306	-138.618293383136	-138.618293383136\\
68	0.24672	-144.256089675498	-144.256089675498\\
68	0.25038	-150.016117631672	-150.016117631672\\
68	0.25404	-155.898377251656	-155.898377251656\\
68	0.2577	-161.902868535451	-161.902868535451\\
68	0.26136	-168.029591483056	-168.029591483056\\
68	0.26502	-174.278546094473	-174.278546094473\\
68	0.26868	-180.6497323697	-180.6497323697\\
68	0.27234	-187.143150308738	-187.143150308738\\
68	0.276	-193.758799911586	-193.758799911586\\
68.375	0.093	-12.9248588643178	-12.9248588643178\\
68.375	0.09666	-13.5539357759497	-13.5539357759497\\
68.375	0.10032	-14.3052443513923	-14.3052443513923\\
68.375	0.10398	-15.1787845906456	-15.1787845906456\\
68.375	0.10764	-16.1745564937097	-16.1745564937097\\
68.375	0.1113	-17.2925600605846	-17.2925600605846\\
68.375	0.11496	-18.5327952912702	-18.5327952912702\\
68.375	0.11862	-19.8952621857666	-19.8952621857666\\
68.375	0.12228	-21.3799607440737	-21.3799607440737\\
68.375	0.12594	-22.9868909661916	-22.9868909661916\\
68.375	0.1296	-24.7160528521203	-24.7160528521203\\
68.375	0.13326	-26.5674464018597	-26.5674464018597\\
68.375	0.13692	-28.5410716154098	-28.5410716154098\\
68.375	0.14058	-30.6369284927708	-30.6369284927708\\
68.375	0.14424	-32.8550170339425	-32.8550170339425\\
68.375	0.1479	-35.1953372389249	-35.1953372389249\\
68.375	0.15156	-37.6578891077181	-37.6578891077181\\
68.375	0.15522	-40.2426726403221	-40.2426726403221\\
68.375	0.15888	-42.9496878367369	-42.9496878367369\\
68.375	0.16254	-45.7789346969624	-45.7789346969624\\
68.375	0.1662	-48.7304132209986	-48.7304132209986\\
68.375	0.16986	-51.8041234088456	-51.8041234088456\\
68.375	0.17352	-55.0000652605033	-55.0000652605033\\
68.375	0.17718	-58.3182387759718	-58.3182387759718\\
68.375	0.18084	-61.7586439552511	-61.7586439552511\\
68.375	0.1845	-65.3212807983411	-65.3212807983411\\
68.375	0.18816	-69.0061493052419	-69.0061493052419\\
68.375	0.19182	-72.8132494759535	-72.8132494759535\\
68.375	0.19548	-76.7425813104758	-76.7425813104758\\
68.375	0.19914	-80.7941448088088	-80.7941448088088\\
68.375	0.2028	-84.9679399709527	-84.9679399709527\\
68.375	0.20646	-89.2639667969073	-89.2639667969073\\
68.375	0.21012	-93.6822252866727	-93.6822252866727\\
68.375	0.21378	-98.2227154402487	-98.2227154402487\\
68.375	0.21744	-102.885437257636	-102.885437257636\\
68.375	0.2211	-107.670390738833	-107.670390738833\\
68.375	0.22476	-112.577575883842	-112.577575883842\\
68.375	0.22842	-117.606992692661	-117.606992692661\\
68.375	0.23208	-122.758641165291	-122.758641165291\\
68.375	0.23574	-128.032521301731	-128.032521301731\\
68.375	0.2394	-133.428633101983	-133.428633101983\\
68.375	0.24306	-138.946976566045	-138.946976566045\\
68.375	0.24672	-144.587551693918	-144.587551693918\\
68.375	0.25038	-150.350358485602	-150.350358485602\\
68.375	0.25404	-156.235396941096	-156.235396941096\\
68.375	0.2577	-162.242667060401	-162.242667060401\\
68.375	0.26136	-168.372168843517	-168.372168843517\\
68.375	0.26502	-174.623902290444	-174.623902290444\\
68.375	0.26868	-180.997867401181	-180.997867401181\\
68.375	0.27234	-187.49406417573	-187.49406417573\\
68.375	0.276	-194.112492614088	-194.112492614088\\
68.75	0.093	-13.1492061638571	-13.1492061638571\\
68.75	0.09666	-13.7810619109993	-13.7810619109993\\
68.75	0.10032	-14.5351493219523	-14.5351493219523\\
68.75	0.10398	-15.4114683967159	-15.4114683967159\\
68.75	0.10764	-16.4100191352904	-16.4100191352904\\
68.75	0.1113	-17.5308015376756	-17.5308015376756\\
68.75	0.11496	-18.7738156038716	-18.7738156038716\\
68.75	0.11862	-20.1390613338783	-20.1390613338783\\
68.75	0.12228	-21.6265387276958	-21.6265387276958\\
68.75	0.12594	-23.236247785324	-23.236247785324\\
68.75	0.1296	-24.968188506763	-24.968188506763\\
68.75	0.13326	-26.8223608920128	-26.8223608920128\\
68.75	0.13692	-28.7987649410733	-28.7987649410733\\
68.75	0.14058	-30.8974006539446	-30.8974006539446\\
68.75	0.14424	-33.1182680306267	-33.1182680306267\\
68.75	0.1479	-35.4613670711194	-35.4613670711194\\
68.75	0.15156	-37.926697775423	-37.926697775423\\
68.75	0.15522	-40.5142601435373	-40.5142601435373\\
68.75	0.15888	-43.2240541754624	-43.2240541754624\\
68.75	0.16254	-46.0560798711982	-46.0560798711982\\
68.75	0.1662	-49.0103372307448	-49.0103372307448\\
68.75	0.16986	-52.0868262541022	-52.0868262541022\\
68.75	0.17352	-55.2855469412702	-55.2855469412702\\
68.75	0.17718	-58.6064992922491	-58.6064992922491\\
68.75	0.18084	-62.0496833070387	-62.0496833070387\\
68.75	0.1845	-65.6150989856391	-65.6150989856391\\
68.75	0.18816	-69.3027463280502	-69.3027463280502\\
68.75	0.19182	-73.1126253342722	-73.1126253342722\\
68.75	0.19548	-77.0447360043048	-77.0447360043048\\
68.75	0.19914	-81.0990783381482	-81.0990783381482\\
68.75	0.2028	-85.2756523358024	-85.2756523358024\\
68.75	0.20646	-89.5744579972674	-89.5744579972674\\
68.75	0.21012	-93.9954953225431	-93.9954953225431\\
68.75	0.21378	-98.5387643116295	-98.5387643116295\\
68.75	0.21744	-103.204264964527	-103.204264964527\\
68.75	0.2211	-107.991997281235	-107.991997281235\\
68.75	0.22476	-112.901961261753	-112.901961261753\\
68.75	0.22842	-117.934156906083	-117.934156906083\\
68.75	0.23208	-123.088584214223	-123.088584214223\\
68.75	0.23574	-128.365243186174	-128.365243186174\\
68.75	0.2394	-133.764133821936	-133.764133821936\\
68.75	0.24306	-139.285256121508	-139.285256121508\\
68.75	0.24672	-144.928610084892	-144.928610084892\\
68.75	0.25038	-150.694195712086	-150.694195712086\\
68.75	0.25404	-156.582013003091	-156.582013003091\\
68.75	0.2577	-162.592061957906	-162.592061957906\\
68.75	0.26136	-168.724342576532	-168.724342576532\\
68.75	0.26502	-174.978854858969	-174.978854858969\\
68.75	0.26868	-181.355598805217	-181.355598805217\\
68.75	0.27234	-187.854574415276	-187.854574415276\\
68.75	0.276	-194.475781689145	-194.475781689145\\
69.125	0.093	-13.3831498359509	-13.3831498359509\\
69.125	0.09666	-14.0177844186034	-14.0177844186034\\
69.125	0.10032	-14.7746506650667	-14.7746506650667\\
69.125	0.10398	-15.6537485753407	-15.6537485753407\\
69.125	0.10764	-16.6550781494255	-16.6550781494255\\
69.125	0.1113	-17.7786393873211	-17.7786393873211\\
69.125	0.11496	-19.0244322890274	-19.0244322890274\\
69.125	0.11862	-20.3924568545445	-20.3924568545445\\
69.125	0.12228	-21.8827130838723	-21.8827130838723\\
69.125	0.12594	-23.4952009770109	-23.4952009770109\\
69.125	0.1296	-25.2299205339603	-25.2299205339603\\
69.125	0.13326	-27.0868717547204	-27.0868717547204\\
69.125	0.13692	-29.0660546392912	-29.0660546392912\\
69.125	0.14058	-31.1674691876729	-31.1674691876729\\
69.125	0.14424	-33.3911153998653	-33.3911153998653\\
69.125	0.1479	-35.7369932758684	-35.7369932758684\\
69.125	0.15156	-38.2051028156823	-38.2051028156823\\
69.125	0.15522	-40.795444019307	-40.795444019307\\
69.125	0.15888	-43.5080168867424	-43.5080168867424\\
69.125	0.16254	-46.3428214179885	-46.3428214179885\\
69.125	0.1662	-49.2998576130455	-49.2998576130455\\
69.125	0.16986	-52.3791254719132	-52.3791254719132\\
69.125	0.17352	-55.5806249945916	-55.5806249945916\\
69.125	0.17718	-58.9043561810809	-58.9043561810809\\
69.125	0.18084	-62.3503190313808	-62.3503190313808\\
69.125	0.1845	-65.9185135454916	-65.9185135454916\\
69.125	0.18816	-69.6089397234131	-69.6089397234131\\
69.125	0.19182	-73.4215975651454	-73.4215975651454\\
69.125	0.19548	-77.3564870706884	-77.3564870706884\\
69.125	0.19914	-81.4136082400421	-81.4136082400421\\
69.125	0.2028	-85.5929610732066	-85.5929610732066\\
69.125	0.20646	-89.8945455701819	-89.8945455701819\\
69.125	0.21012	-94.3183617309679	-94.3183617309679\\
69.125	0.21378	-98.8644095555647	-98.8644095555647\\
69.125	0.21744	-103.532689043972	-103.532689043972\\
69.125	0.2211	-108.323200196191	-108.323200196191\\
69.125	0.22476	-113.23594301222	-113.23594301222\\
69.125	0.22842	-118.270917492059	-118.270917492059\\
69.125	0.23208	-123.42812363571	-123.42812363571\\
69.125	0.23574	-128.707561443171	-128.707561443171\\
69.125	0.2394	-134.109230914444	-134.109230914444\\
69.125	0.24306	-139.633132049526	-139.633132049526\\
69.125	0.24672	-145.27926484842	-145.27926484842\\
69.125	0.25038	-151.047629311125	-151.047629311125\\
69.125	0.25404	-156.93822543764	-156.93822543764\\
69.125	0.2577	-162.951053227966	-162.951053227966\\
69.125	0.26136	-169.086112682102	-169.086112682102\\
69.125	0.26502	-175.34340380005	-175.34340380005\\
69.125	0.26868	-181.722926581808	-181.722926581808\\
69.125	0.27234	-188.224681027377	-188.224681027377\\
69.125	0.276	-194.848667136756	-194.848667136756\\
69.5	0.093	-13.626689880599	-13.626689880599\\
69.5	0.09666	-14.2641032987619	-14.2641032987619\\
69.5	0.10032	-15.0237483807355	-15.0237483807355\\
69.5	0.10398	-15.9056251265199	-15.9056251265199\\
69.5	0.10764	-16.9097335361151	-16.9097335361151\\
69.5	0.1113	-18.036073609521	-18.036073609521\\
69.5	0.11496	-19.2846453467376	-19.2846453467376\\
69.5	0.11862	-20.6554487477651	-20.6554487477651\\
69.5	0.12228	-22.1484838126032	-22.1484838126032\\
69.5	0.12594	-23.7637505412522	-23.7637505412522\\
69.5	0.1296	-25.5012489337119	-25.5012489337119\\
69.5	0.13326	-27.3609789899823	-27.3609789899823\\
69.5	0.13692	-29.3429407100635	-29.3429407100635\\
69.5	0.14058	-31.4471340939555	-31.4471340939555\\
69.5	0.14424	-33.6735591416583	-33.6735591416583\\
69.5	0.1479	-36.0222158531718	-36.0222158531718\\
69.5	0.15156	-38.493104228496	-38.493104228496\\
69.5	0.15522	-41.086224267631	-41.086224267631\\
69.5	0.15888	-43.8015759705768	-43.8015759705768\\
69.5	0.16254	-46.6391593373334	-46.6391593373334\\
69.5	0.1662	-49.5989743679007	-49.5989743679007\\
69.5	0.16986	-52.6810210622787	-52.6810210622787\\
69.5	0.17352	-55.8852994204674	-55.8852994204674\\
69.5	0.17718	-59.211809442467	-59.211809442467\\
69.5	0.18084	-62.6605511282773	-62.6605511282773\\
69.5	0.1845	-66.2315244778984	-66.2315244778984\\
69.5	0.18816	-69.9247294913302	-69.9247294913302\\
69.5	0.19182	-73.7401661685728	-73.7401661685728\\
69.5	0.19548	-77.6778345096261	-77.6778345096261\\
69.5	0.19914	-81.7377345144903	-81.7377345144903\\
69.5	0.2028	-85.9198661831652	-85.9198661831652\\
69.5	0.20646	-90.2242295156508	-90.2242295156508\\
69.5	0.21012	-94.6508245119472	-94.6508245119472\\
69.5	0.21378	-99.1996511720544	-99.1996511720544\\
69.5	0.21744	-103.870709495972	-103.870709495972\\
69.5	0.2211	-108.663999483701	-108.663999483701\\
69.5	0.22476	-113.57952113524	-113.57952113524\\
69.5	0.22842	-118.617274450591	-118.617274450591\\
69.5	0.23208	-123.777259429751	-123.777259429751\\
69.5	0.23574	-129.059476072723	-129.059476072723\\
69.5	0.2394	-134.463924379506	-134.463924379506\\
69.5	0.24306	-139.990604350099	-139.990604350099\\
69.5	0.24672	-145.639515984503	-145.639515984503\\
69.5	0.25038	-151.410659282718	-151.410659282718\\
69.5	0.25404	-157.304034244743	-157.304034244743\\
69.5	0.2577	-163.319640870579	-163.319640870579\\
69.5	0.26136	-169.457479160226	-169.457479160226\\
69.5	0.26502	-175.717549113684	-175.717549113684\\
69.5	0.26868	-182.099850730953	-182.099850730953\\
69.5	0.27234	-188.604384012032	-188.604384012032\\
69.5	0.276	-195.231148956922	-195.231148956922\\
69.875	0.093	-13.8798262978015	-13.8798262978015\\
69.875	0.09666	-14.5200185514747	-14.5200185514747\\
69.875	0.10032	-15.2824424689587	-15.2824424689587\\
69.875	0.10398	-16.1670980502534	-16.1670980502534\\
69.875	0.10764	-17.1739852953589	-17.1739852953589\\
69.875	0.1113	-18.3031042042752	-18.3031042042752\\
69.875	0.11496	-19.5544547770022	-19.5544547770022\\
69.875	0.11862	-20.92803701354	-20.92803701354\\
69.875	0.12228	-22.4238509138885	-22.4238509138885\\
69.875	0.12594	-24.0418964780478	-24.0418964780478\\
69.875	0.1296	-25.7821737060178	-25.7821737060178\\
69.875	0.13326	-27.6446825977986	-27.6446825977986\\
69.875	0.13692	-29.6294231533902	-29.6294231533902\\
69.875	0.14058	-31.7363953727925	-31.7363953727925\\
69.875	0.14424	-33.9655992560056	-33.9655992560056\\
69.875	0.1479	-36.3170348030295	-36.3170348030295\\
69.875	0.15156	-38.7907020138641	-38.7907020138641\\
69.875	0.15522	-41.3866008885094	-41.3866008885094\\
69.875	0.15888	-44.1047314269656	-44.1047314269656\\
69.875	0.16254	-46.9450936292324	-46.9450936292324\\
69.875	0.1662	-49.9076874953101	-49.9076874953101\\
69.875	0.16986	-52.9925130251985	-52.9925130251985\\
69.875	0.17352	-56.1995702188976	-56.1995702188976\\
69.875	0.17718	-59.5288590764075	-59.5288590764075\\
69.875	0.18084	-62.9803795977281	-62.9803795977281\\
69.875	0.1845	-66.5541317828596	-66.5541317828596\\
69.875	0.18816	-70.2501156318017	-70.2501156318017\\
69.875	0.19182	-74.0683311445547	-74.0683311445547\\
69.875	0.19548	-78.0087783211184	-78.0087783211184\\
69.875	0.19914	-82.0714571614928	-82.0714571614928\\
69.875	0.2028	-86.2563676656781	-86.2563676656781\\
69.875	0.20646	-90.5635098336741	-90.5635098336741\\
69.875	0.21012	-94.9928836654808	-94.9928836654808\\
69.875	0.21378	-99.5444891610983	-99.5444891610983\\
69.875	0.21744	-104.218326320527	-104.218326320527\\
69.875	0.2211	-109.014395143766	-109.014395143766\\
69.875	0.22476	-113.932695630815	-113.932695630815\\
69.875	0.22842	-118.973227781676	-118.973227781676\\
69.875	0.23208	-124.135991596347	-124.135991596347\\
69.875	0.23574	-129.420987074829	-129.420987074829\\
69.875	0.2394	-134.828214217122	-134.828214217122\\
69.875	0.24306	-140.357673023226	-140.357673023226\\
69.875	0.24672	-146.00936349314	-146.00936349314\\
69.875	0.25038	-151.783285626865	-151.783285626865\\
69.875	0.25404	-157.679439424401	-157.679439424401\\
69.875	0.2577	-163.697824885747	-163.697824885747\\
69.875	0.26136	-169.838442010905	-169.838442010905\\
69.875	0.26502	-176.101290799873	-176.101290799873\\
69.875	0.26868	-182.486371252652	-182.486371252652\\
69.875	0.27234	-188.993683369241	-188.993683369241\\
69.875	0.276	-195.623227149642	-195.623227149642\\
70.25	0.093	-14.1425590875584	-14.1425590875584\\
70.25	0.09666	-14.785530176742	-14.785530176742\\
70.25	0.10032	-15.5507329297363	-15.5507329297363\\
70.25	0.10398	-16.4381673465414	-16.4381673465414\\
70.25	0.10764	-17.4478334271573	-17.4478334271573\\
70.25	0.1113	-18.5797311715839	-18.5797311715839\\
70.25	0.11496	-19.8338605798212	-19.8338605798212\\
70.25	0.11862	-21.2102216518693	-21.2102216518693\\
70.25	0.12228	-22.7088143877282	-22.7088143877282\\
70.25	0.12594	-24.3296387873979	-24.3296387873979\\
70.25	0.1296	-26.0726948508782	-26.0726948508782\\
70.25	0.13326	-27.9379825781694	-27.9379825781694\\
70.25	0.13692	-29.9255019692713	-29.9255019692713\\
70.25	0.14058	-32.035253024184	-32.035253024184\\
70.25	0.14424	-34.2672357429074	-34.2672357429074\\
70.25	0.1479	-36.6214501254416	-36.6214501254416\\
70.25	0.15156	-39.0978961717866	-39.0978961717866\\
70.25	0.15522	-41.6965738819423	-41.6965738819423\\
70.25	0.15888	-44.4174832559087	-44.4174832559087\\
70.25	0.16254	-47.2606242936859	-47.2606242936859\\
70.25	0.1662	-50.2259969952739	-50.2259969952739\\
70.25	0.16986	-53.3136013606727	-53.3136013606727\\
70.25	0.17352	-56.5234373898822	-56.5234373898822\\
70.25	0.17718	-59.8555050829024	-59.8555050829024\\
70.25	0.18084	-63.3098044397334	-63.3098044397334\\
70.25	0.1845	-66.8863354603752	-66.8863354603752\\
70.25	0.18816	-70.5850981448278	-70.5850981448278\\
70.25	0.19182	-74.4060924930911	-74.4060924930911\\
70.25	0.19548	-78.3493185051651	-78.3493185051651\\
70.25	0.19914	-82.4147761810499	-82.4147761810499\\
70.25	0.2028	-86.6024655207455	-86.6024655207455\\
70.25	0.20646	-90.9123865242518	-90.9123865242518\\
70.25	0.21012	-95.3445391915689	-95.3445391915689\\
70.25	0.21378	-99.8989235226967	-99.8989235226967\\
70.25	0.21744	-104.575539517635	-104.575539517635\\
70.25	0.2211	-109.374387176385	-109.374387176385\\
70.25	0.22476	-114.295466498945	-114.295466498945\\
70.25	0.22842	-119.338777485316	-119.338777485316\\
70.25	0.23208	-124.504320135497	-124.504320135497\\
70.25	0.23574	-129.79209444949	-129.79209444949\\
70.25	0.2394	-135.202100427293	-135.202100427293\\
70.25	0.24306	-140.734338068907	-140.734338068907\\
70.25	0.24672	-146.388807374331	-146.388807374331\\
70.25	0.25038	-152.165508343567	-152.165508343567\\
70.25	0.25404	-158.064440976613	-158.064440976613\\
70.25	0.2577	-164.08560527347	-164.08560527347\\
70.25	0.26136	-170.229001234138	-170.229001234138\\
70.25	0.26502	-176.494628858616	-176.494628858616\\
70.25	0.26868	-182.882488146905	-182.882488146905\\
70.25	0.27234	-189.392579099005	-189.392579099005\\
70.25	0.276	-196.024901714916	-196.024901714916\\
70.625	0.093	-14.4148882498698	-14.4148882498698\\
70.625	0.09666	-15.0606381745637	-15.0606381745637\\
70.625	0.10032	-15.8286197630684	-15.8286197630684\\
70.625	0.10398	-16.7188330153838	-16.7188330153838\\
70.625	0.10764	-17.73127793151	-17.73127793151\\
70.625	0.1113	-18.865954511447	-18.865954511447\\
70.625	0.11496	-20.1228627551947	-20.1228627551947\\
70.625	0.11862	-21.5020026627531	-21.5020026627531\\
70.625	0.12228	-23.0033742341223	-23.0033742341223\\
70.625	0.12594	-24.6269774693023	-24.6269774693023\\
70.625	0.1296	-26.3728123682931	-26.3728123682931\\
70.625	0.13326	-28.2408789310946	-28.2408789310946\\
70.625	0.13692	-30.2311771577068	-30.2311771577068\\
70.625	0.14058	-32.3437070481298	-32.3437070481298\\
70.625	0.14424	-34.5784686023636	-34.5784686023636\\
70.625	0.1479	-36.9354618204082	-36.9354618204082\\
70.625	0.15156	-39.4146867022635	-39.4146867022635\\
70.625	0.15522	-42.0161432479295	-42.0161432479295\\
70.625	0.15888	-44.7398314574064	-44.7398314574064\\
70.625	0.16254	-47.5857513306939	-47.5857513306939\\
70.625	0.1662	-50.5539028677923	-50.5539028677923\\
70.625	0.16986	-53.6442860687014	-53.6442860687014\\
70.625	0.17352	-56.8569009334211	-56.8569009334211\\
70.625	0.17718	-60.1917474619518	-60.1917474619518\\
70.625	0.18084	-63.6488256542931	-63.6488256542931\\
70.625	0.1845	-67.2281355104452	-67.2281355104452\\
70.625	0.18816	-70.9296770304081	-70.9296770304081\\
70.625	0.19182	-74.7534502141817	-74.7534502141817\\
70.625	0.19548	-78.6994550617661	-78.6994550617661\\
70.625	0.19914	-82.7676915731613	-82.7676915731613\\
70.625	0.2028	-86.9581597483673	-86.9581597483673\\
70.625	0.20646	-91.2708595873839	-91.2708595873839\\
70.625	0.21012	-95.7057910902114	-95.7057910902114\\
70.625	0.21378	-100.26295425685	-100.26295425685\\
70.625	0.21744	-104.942349087299	-104.942349087299\\
70.625	0.2211	-109.743975581558	-109.743975581558\\
70.625	0.22476	-114.667833739629	-114.667833739629\\
70.625	0.22842	-119.71392356151	-119.71392356151\\
70.625	0.23208	-124.882245047202	-124.882245047202\\
70.625	0.23574	-130.172798196705	-130.172798196705\\
70.625	0.2394	-135.585583010018	-135.585583010018\\
70.625	0.24306	-141.120599487142	-141.120599487142\\
70.625	0.24672	-146.777847628077	-146.777847628077\\
70.625	0.25038	-152.557327432823	-152.557327432823\\
70.625	0.25404	-158.45903890138	-158.45903890138\\
70.625	0.2577	-164.482982033747	-164.482982033747\\
70.625	0.26136	-170.629156829925	-170.629156829925\\
70.625	0.26502	-176.897563289914	-176.897563289914\\
70.625	0.26868	-183.288201413713	-183.288201413713\\
70.625	0.27234	-189.801071201324	-189.801071201324\\
70.625	0.276	-196.436172652745	-196.436172652745\\
71	0.093	-14.6968137847354	-14.6968137847354\\
71	0.09666	-15.3453425449397	-15.3453425449397\\
71	0.10032	-16.1161029689547	-16.1161029689547\\
71	0.10398	-17.0090950567805	-17.0090950567805\\
71	0.10764	-18.024318808417	-18.024318808417\\
71	0.1113	-19.1617742238644	-19.1617742238644\\
71	0.11496	-20.4214613031224	-20.4214613031224\\
71	0.11862	-21.8033800461912	-21.8033800461912\\
71	0.12228	-23.3075304530708	-23.3075304530708\\
71	0.12594	-24.9339125237611	-24.9339125237611\\
71	0.1296	-26.6825262582622	-26.6825262582622\\
71	0.13326	-28.5533716565741	-28.5533716565741\\
71	0.13692	-30.5464487186967	-30.5464487186967\\
71	0.14058	-32.6617574446301	-32.6617574446301\\
71	0.14424	-34.8992978343742	-34.8992978343742\\
71	0.1479	-37.2590698879291	-37.2590698879291\\
71	0.15156	-39.7410736052947	-39.7410736052947\\
71	0.15522	-42.3453089864711	-42.3453089864711\\
71	0.15888	-45.0717760314583	-45.0717760314583\\
71	0.16254	-47.9204747402562	-47.9204747402562\\
71	0.1662	-50.8914051128649	-50.8914051128649\\
71	0.16986	-53.9845671492843	-53.9845671492843\\
71	0.17352	-57.1999608495145	-57.1999608495145\\
71	0.17718	-60.5375862135554	-60.5375862135554\\
71	0.18084	-63.9974432414071	-63.9974432414071\\
71	0.1845	-67.5795319330697	-67.5795319330697\\
71	0.18816	-71.2838522885428	-71.2838522885428\\
71	0.19182	-75.1104043078268	-75.1104043078268\\
71	0.19548	-79.0591879909216	-79.0591879909216\\
71	0.19914	-83.1302033378271	-83.1302033378271\\
71	0.2028	-87.3234503485434	-87.3234503485434\\
71	0.20646	-91.6389290230704	-91.6389290230704\\
71	0.21012	-96.0766393614082	-96.0766393614082\\
71	0.21378	-100.636581363557	-100.636581363557\\
71	0.21744	-105.318755029516	-105.318755029516\\
71	0.2211	-110.123160359286	-110.123160359286\\
71	0.22476	-115.049797352867	-115.049797352867\\
71	0.22842	-120.098666010258	-120.098666010258\\
71	0.23208	-125.269766331461	-125.269766331461\\
71	0.23574	-130.563098316474	-130.563098316474\\
71	0.2394	-135.978661965298	-135.978661965298\\
71	0.24306	-141.516457277932	-141.516457277932\\
71	0.24672	-147.176484254378	-147.176484254378\\
71	0.25038	-152.958742894634	-152.958742894634\\
71	0.25404	-158.863233198701	-158.863233198701\\
71	0.2577	-164.889955166578	-164.889955166578\\
71	0.26136	-171.038908798267	-171.038908798267\\
71	0.26502	-177.310094093766	-177.310094093766\\
71	0.26868	-183.703511053076	-183.703511053076\\
71	0.27234	-190.219159676196	-190.219159676196\\
71	0.276	-196.857039963128	-196.857039963128\\
71.375	0.093	-14.9883356921555	-14.9883356921555\\
71.375	0.09666	-15.6396432878702	-15.6396432878702\\
71.375	0.10032	-16.4131825473956	-16.4131825473956\\
71.375	0.10398	-17.3089534707317	-17.3089534707317\\
71.375	0.10764	-18.3269560578786	-18.3269560578786\\
71.375	0.1113	-19.4671903088362	-19.4671903088362\\
71.375	0.11496	-20.7296562236046	-20.7296562236046\\
71.375	0.11862	-22.1143538021838	-22.1143538021838\\
71.375	0.12228	-23.6212830445737	-23.6212830445737\\
71.375	0.12594	-25.2504439507744	-25.2504439507744\\
71.375	0.1296	-27.0018365207858	-27.0018365207858\\
71.375	0.13326	-28.875460754608	-28.875460754608\\
71.375	0.13692	-30.871316652241	-30.871316652241\\
71.375	0.14058	-32.9894042136847	-32.9894042136847\\
71.375	0.14424	-35.2297234389392	-35.2297234389392\\
71.375	0.1479	-37.5922743280044	-37.5922743280044\\
71.375	0.15156	-40.0770568808804	-40.0770568808804\\
71.375	0.15522	-42.6840710975672	-42.6840710975672\\
71.375	0.15888	-45.4133169780647	-45.4133169780647\\
71.375	0.16254	-48.2647945223729	-48.2647945223729\\
71.375	0.1662	-51.238503730492	-51.238503730492\\
71.375	0.16986	-54.3344446024217	-54.3344446024217\\
71.375	0.17352	-57.5526171381622	-57.5526171381622\\
71.375	0.17718	-60.8930213377135	-60.8930213377135\\
71.375	0.18084	-64.3556572010756	-64.3556572010756\\
71.375	0.1845	-67.9405247282485	-67.9405247282485\\
71.375	0.18816	-71.647623919232	-71.647623919232\\
71.375	0.19182	-75.4769547740264	-75.4769547740264\\
71.375	0.19548	-79.4285172926315	-79.4285172926315\\
71.375	0.19914	-83.5023114750473	-83.5023114750473\\
71.375	0.2028	-87.698337321274	-87.698337321274\\
71.375	0.20646	-92.0165948313113	-92.0165948313113\\
71.375	0.21012	-96.4570840051595	-96.4570840051595\\
71.375	0.21378	-101.019804842818	-101.019804842818\\
71.375	0.21744	-105.704757344288	-105.704757344288\\
71.375	0.2211	-110.511941509568	-110.511941509568\\
71.375	0.22476	-115.44135733866	-115.44135733866\\
71.375	0.22842	-120.493004831561	-120.493004831561\\
71.375	0.23208	-125.666883988274	-125.666883988274\\
71.375	0.23574	-130.962994808798	-130.962994808798\\
71.375	0.2394	-136.381337293132	-136.381337293132\\
71.375	0.24306	-141.921911441277	-141.921911441277\\
71.375	0.24672	-147.584717253232	-147.584717253232\\
71.375	0.25038	-153.369754728999	-153.369754728999\\
71.375	0.25404	-159.277023868576	-159.277023868576\\
71.375	0.2577	-165.306524671964	-165.306524671964\\
71.375	0.26136	-171.458257139163	-171.458257139163\\
71.375	0.26502	-177.732221270172	-177.732221270172\\
71.375	0.26868	-184.128417064993	-184.128417064993\\
71.375	0.27234	-190.646844523624	-190.646844523624\\
71.375	0.276	-197.287503646066	-197.287503646066\\
71.75	0.093	-15.28945397213	-15.28945397213\\
71.75	0.09666	-15.943540403355	-15.943540403355\\
71.75	0.10032	-16.7198584983908	-16.7198584983908\\
71.75	0.10398	-17.6184082572372	-17.6184082572372\\
71.75	0.10764	-18.6391896798945	-18.6391896798945\\
71.75	0.1113	-19.7822027663625	-19.7822027663625\\
71.75	0.11496	-21.0474475166412	-21.0474475166412\\
71.75	0.11862	-22.4349239307307	-22.4349239307307\\
71.75	0.12228	-23.944632008631	-23.944632008631\\
71.75	0.12594	-25.576571750342	-25.576571750342\\
71.75	0.1296	-27.3307431558638	-27.3307431558638\\
71.75	0.13326	-29.2071462251963	-29.2071462251963\\
71.75	0.13692	-31.2057809583397	-31.2057809583397\\
71.75	0.14058	-33.3266473552937	-33.3266473552937\\
71.75	0.14424	-35.5697454160586	-35.5697454160586\\
71.75	0.1479	-37.9350751406341	-37.9350751406341\\
71.75	0.15156	-40.4226365290205	-40.4226365290205\\
71.75	0.15522	-43.0324295812176	-43.0324295812176\\
71.75	0.15888	-45.7644542972255	-45.7644542972255\\
71.75	0.16254	-48.6187106770441	-48.6187106770441\\
71.75	0.1662	-51.5951987206735	-51.5951987206735\\
71.75	0.16986	-54.6939184281136	-54.6939184281136\\
71.75	0.17352	-57.9148697993644	-57.9148697993644\\
71.75	0.17718	-61.2580528344261	-61.2580528344261\\
71.75	0.18084	-64.7234675332985	-64.7234675332985\\
71.75	0.1845	-68.3111138959817	-68.3111138959817\\
71.75	0.18816	-72.0209919224755	-72.0209919224755\\
71.75	0.19182	-75.8531016127802	-75.8531016127802\\
71.75	0.19548	-79.8074429668956	-79.8074429668956\\
71.75	0.19914	-83.8840159848219	-83.8840159848219\\
71.75	0.2028	-88.0828206665589	-88.0828206665589\\
71.75	0.20646	-92.4038570121066	-92.4038570121066\\
71.75	0.21012	-96.8471250214651	-96.8471250214651\\
71.75	0.21378	-101.412624694634	-101.412624694634\\
71.75	0.21744	-106.100356031614	-106.100356031614\\
71.75	0.2211	-110.910319032405	-110.910319032405\\
71.75	0.22476	-115.842513697007	-115.842513697007\\
71.75	0.22842	-120.896940025419	-120.896940025419\\
71.75	0.23208	-126.073598017642	-126.073598017642\\
71.75	0.23574	-131.372487673676	-131.372487673676\\
71.75	0.2394	-136.79360899352	-136.79360899352\\
71.75	0.24306	-142.336961977176	-142.336961977176\\
71.75	0.24672	-148.002546624642	-148.002546624642\\
71.75	0.25038	-153.790362935918	-153.790362935918\\
71.75	0.25404	-159.700410911006	-159.700410911006\\
71.75	0.2577	-165.732690549904	-165.732690549904\\
71.75	0.26136	-171.887201852613	-171.887201852613\\
71.75	0.26502	-178.163944819133	-178.163944819133\\
71.75	0.26868	-184.562919449464	-184.562919449464\\
71.75	0.27234	-191.084125743605	-191.084125743605\\
71.75	0.276	-197.727563701557	-197.727563701557\\
72.125	0.093	-15.6001686246589	-15.6001686246589\\
72.125	0.09666	-16.2570338913942	-16.2570338913942\\
72.125	0.10032	-17.0361308219403	-17.0361308219403\\
72.125	0.10398	-17.9374594162971	-17.9374594162971\\
72.125	0.10764	-18.9610196744647	-18.9610196744647\\
72.125	0.1113	-20.1068115964431	-20.1068115964431\\
72.125	0.11496	-21.3748351822322	-21.3748351822322\\
72.125	0.11862	-22.765090431832	-22.765090431832\\
72.125	0.12228	-24.2775773452427	-24.2775773452427\\
72.125	0.12594	-25.912295922464	-25.912295922464\\
72.125	0.1296	-27.6692461634961	-27.6692461634961\\
72.125	0.13326	-29.548428068339	-29.548428068339\\
72.125	0.13692	-31.5498416369927	-31.5498416369927\\
72.125	0.14058	-33.6734868694571	-33.6734868694571\\
72.125	0.14424	-35.9193637657323	-35.9193637657323\\
72.125	0.1479	-38.2874723258182	-38.2874723258182\\
72.125	0.15156	-40.7778125497149	-40.7778125497149\\
72.125	0.15522	-43.3903844374224	-43.3903844374224\\
72.125	0.15888	-46.1251879889406	-46.1251879889406\\
72.125	0.16254	-48.9822232042696	-48.9822232042696\\
72.125	0.1662	-51.9614900834093	-51.9614900834093\\
72.125	0.16986	-55.0629886263598	-55.0629886263598\\
72.125	0.17352	-58.2867188331209	-58.2867188331209\\
72.125	0.17718	-61.6326807036929	-61.6326807036929\\
72.125	0.18084	-65.1008742380757	-65.1008742380757\\
72.125	0.1845	-68.6912994362692	-68.6912994362692\\
72.125	0.18816	-72.4039562982734	-72.4039562982734\\
72.125	0.19182	-76.2388448240885	-76.2388448240885\\
72.125	0.19548	-80.1959650137143	-80.1959650137143\\
72.125	0.19914	-84.2753168671508	-84.2753168671508\\
72.125	0.2028	-88.4769003843982	-88.4769003843982\\
72.125	0.20646	-92.8007155654563	-92.8007155654563\\
72.125	0.21012	-97.2467624103251	-97.2467624103251\\
72.125	0.21378	-101.815040919005	-101.815040919005\\
72.125	0.21744	-106.505551091495	-106.505551091495\\
72.125	0.2211	-111.318292927796	-111.318292927796\\
72.125	0.22476	-116.253266427908	-116.253266427908\\
72.125	0.22842	-121.310471591831	-121.310471591831\\
72.125	0.23208	-126.489908419564	-126.489908419564\\
72.125	0.23574	-131.791576911108	-131.791576911108\\
72.125	0.2394	-137.215477066463	-137.215477066463\\
72.125	0.24306	-142.761608885629	-142.761608885629\\
72.125	0.24672	-148.429972368605	-148.429972368605\\
72.125	0.25038	-154.220567515392	-154.220567515392\\
72.125	0.25404	-160.13339432599	-160.13339432599\\
72.125	0.2577	-166.168452800399	-166.168452800399\\
72.125	0.26136	-172.325742938618	-172.325742938618\\
72.125	0.26502	-178.605264740648	-178.605264740648\\
72.125	0.26868	-185.007018206489	-185.007018206489\\
72.125	0.27234	-191.531003336141	-191.531003336141\\
72.125	0.276	-198.177220129604	-198.177220129604\\
72.5	0.093	-15.9204796497422	-15.9204796497422\\
72.5	0.09666	-16.5801237519879	-16.5801237519879\\
72.5	0.10032	-17.3619995180443	-17.3619995180443\\
72.5	0.10398	-18.2661069479115	-18.2661069479115\\
72.5	0.10764	-19.2924460415895	-19.2924460415895\\
72.5	0.1113	-20.4410167990782	-20.4410167990782\\
72.5	0.11496	-21.7118192203776	-21.7118192203776\\
72.5	0.11862	-23.1048533054878	-23.1048533054878\\
72.5	0.12228	-24.6201190544088	-24.6201190544088\\
72.5	0.12594	-26.2576164671405	-26.2576164671405\\
72.5	0.1296	-28.017345543683	-28.017345543683\\
72.5	0.13326	-29.8993062840362	-29.8993062840362\\
72.5	0.13692	-31.9034986882002	-31.9034986882002\\
72.5	0.14058	-34.029922756175	-34.029922756175\\
72.5	0.14424	-36.2785784879605	-36.2785784879605\\
72.5	0.1479	-38.6494658835568	-38.6494658835568\\
72.5	0.15156	-41.1425849429638	-41.1425849429638\\
72.5	0.15522	-43.7579356661816	-43.7579356661816\\
72.5	0.15888	-46.4955180532101	-46.4955180532101\\
72.5	0.16254	-49.3553321040494	-49.3553321040494\\
72.5	0.1662	-52.3373778186995	-52.3373778186995\\
72.5	0.16986	-55.4416551971603	-55.4416551971603\\
72.5	0.17352	-58.6681642394319	-58.6681642394319\\
72.5	0.17718	-62.0169049455143	-62.0169049455143\\
72.5	0.18084	-65.4878773154074	-65.4878773154074\\
72.5	0.1845	-69.0810813491113	-69.0810813491113\\
72.5	0.18816	-72.7965170466259	-72.7965170466259\\
72.5	0.19182	-76.6341844079513	-76.6341844079513\\
72.5	0.19548	-80.5940834330874	-80.5940834330874\\
72.5	0.19914	-84.6762141220343	-84.6762141220343\\
72.5	0.2028	-88.880576474792	-88.880576474792\\
72.5	0.20646	-93.2071704913604	-93.2071704913604\\
72.5	0.21012	-97.6559961717396	-97.6559961717396\\
72.5	0.21378	-102.227053515929	-102.227053515929\\
72.5	0.21744	-106.92034252393	-106.92034252393\\
72.5	0.2211	-111.735863195742	-111.735863195742\\
72.5	0.22476	-116.673615531364	-116.673615531364\\
72.5	0.22842	-121.733599530797	-121.733599530797\\
72.5	0.23208	-126.91581519404	-126.91581519404\\
72.5	0.23574	-132.220262521095	-132.220262521095\\
72.5	0.2394	-137.64694151196	-137.64694151196\\
72.5	0.24306	-143.195852166636	-143.195852166636\\
72.5	0.24672	-148.866994485123	-148.866994485123\\
72.5	0.25038	-154.660368467421	-154.660368467421\\
72.5	0.25404	-160.575974113529	-160.575974113529\\
72.5	0.2577	-166.613811423448	-166.613811423448\\
72.5	0.26136	-172.773880397178	-172.773880397178\\
72.5	0.26502	-179.056181034718	-179.056181034718\\
72.5	0.26868	-185.46071333607	-185.46071333607\\
72.5	0.27234	-191.987477301232	-191.987477301232\\
72.5	0.276	-198.636472930204	-198.636472930204\\
72.875	0.093	-16.25038704738	-16.25038704738\\
72.875	0.09666	-16.912809985136	-16.912809985136\\
72.875	0.10032	-17.6974645867028	-17.6974645867028\\
72.875	0.10398	-18.6043508520803	-18.6043508520803\\
72.875	0.10764	-19.6334687812686	-19.6334687812686\\
72.875	0.1113	-20.7848183742676	-20.7848183742676\\
72.875	0.11496	-22.0583996310774	-22.0583996310774\\
72.875	0.11862	-23.454212551698	-23.454212551698\\
72.875	0.12228	-24.9722571361293	-24.9722571361293\\
72.875	0.12594	-26.6125333843713	-26.6125333843713\\
72.875	0.1296	-28.3750412964242	-28.3750412964242\\
72.875	0.13326	-30.2597808722878	-30.2597808722878\\
72.875	0.13692	-32.2667521119621	-32.2667521119621\\
72.875	0.14058	-34.3959550154472	-34.3959550154472\\
72.875	0.14424	-36.6473895827431	-36.6473895827431\\
72.875	0.1479	-39.0210558138497	-39.0210558138497\\
72.875	0.15156	-41.5169537087671	-41.5169537087671\\
72.875	0.15522	-44.1350832674952	-44.1350832674952\\
72.875	0.15888	-46.8754444900342	-46.8754444900342\\
72.875	0.16254	-49.7380373763838	-49.7380373763838\\
72.875	0.1662	-52.7228619265442	-52.7228619265442\\
72.875	0.16986	-55.8299181405154	-55.8299181405154\\
72.875	0.17352	-59.0592060182973	-59.0592060182973\\
72.875	0.17718	-62.41072555989	-62.41072555989\\
72.875	0.18084	-65.8844767652934	-65.8844767652934\\
72.875	0.1845	-69.4804596345077	-69.4804596345077\\
72.875	0.18816	-73.1986741675326	-73.1986741675326\\
72.875	0.19182	-77.0391203643683	-77.0391203643683\\
72.875	0.19548	-81.0017982250148	-81.0017982250148\\
72.875	0.19914	-85.0867077494721	-85.0867077494721\\
72.875	0.2028	-89.2938489377402	-89.2938489377402\\
72.875	0.20646	-93.6232217898189	-93.6232217898189\\
72.875	0.21012	-98.0748263057084	-98.0748263057084\\
72.875	0.21378	-102.648662485409	-102.648662485409\\
72.875	0.21744	-107.34473032892	-107.34473032892\\
72.875	0.2211	-112.163029836242	-112.163029836242\\
72.875	0.22476	-117.103561007374	-117.103561007374\\
72.875	0.22842	-122.166323842317	-122.166323842317\\
72.875	0.23208	-127.351318341071	-127.351318341071\\
72.875	0.23574	-132.658544503636	-132.658544503636\\
72.875	0.2394	-138.088002330012	-138.088002330012\\
72.875	0.24306	-143.639691820198	-143.639691820198\\
72.875	0.24672	-149.313612974195	-149.313612974195\\
72.875	0.25038	-155.109765792003	-155.109765792003\\
72.875	0.25404	-161.028150273622	-161.028150273622\\
72.875	0.2577	-167.068766419051	-167.068766419051\\
72.875	0.26136	-173.231614228291	-173.231614228291\\
72.875	0.26502	-179.516693701342	-179.516693701342\\
72.875	0.26868	-185.924004838204	-185.924004838204\\
72.875	0.27234	-192.453547638876	-192.453547638876\\
72.875	0.276	-199.105322103359	-199.105322103359\\
73.25	0.093	-16.589890817572	-16.589890817572\\
73.25	0.09666	-17.2550925908384	-17.2550925908384\\
73.25	0.10032	-18.0425260279155	-18.0425260279155\\
73.25	0.10398	-18.9521911288033	-18.9521911288033\\
73.25	0.10764	-19.984087893502	-19.984087893502\\
73.25	0.1113	-21.1382163220114	-21.1382163220114\\
73.25	0.11496	-22.4145764143315	-22.4145764143315\\
73.25	0.11862	-23.8131681704624	-23.8131681704624\\
73.25	0.12228	-25.3339915904041	-25.3339915904041\\
73.25	0.12594	-26.9770466741565	-26.9770466741565\\
73.25	0.1296	-28.7423334217197	-28.7423334217197\\
73.25	0.13326	-30.6298518330936	-30.6298518330936\\
73.25	0.13692	-32.6396019082783	-32.6396019082783\\
73.25	0.14058	-34.7715836472738	-34.7715836472738\\
73.25	0.14424	-37.02579705008	-37.02579705008\\
73.25	0.1479	-39.402242116697	-39.402242116697\\
73.25	0.15156	-41.9009188471247	-41.9009188471247\\
73.25	0.15522	-44.5218272413632	-44.5218272413632\\
73.25	0.15888	-47.2649672994125	-47.2649672994125\\
73.25	0.16254	-50.1303390212725	-50.1303390212725\\
73.25	0.1662	-53.1179424069432	-53.1179424069432\\
73.25	0.16986	-56.2277774564248	-56.2277774564248\\
73.25	0.17352	-59.459844169717	-59.459844169717\\
73.25	0.17718	-62.8141425468201	-62.8141425468201\\
73.25	0.18084	-66.2906725877338	-66.2906725877338\\
73.25	0.1845	-69.8894342924584	-69.8894342924584\\
73.25	0.18816	-73.6104276609937	-73.6104276609937\\
73.25	0.19182	-77.4536526933398	-77.4536526933398\\
73.25	0.19548	-81.4191093894966	-81.4191093894966\\
73.25	0.19914	-85.5067977494642	-85.5067977494642\\
73.25	0.2028	-89.7167177732426	-89.7167177732426\\
73.25	0.20646	-94.0488694608318	-94.0488694608318\\
73.25	0.21012	-98.5032528122316	-98.5032528122316\\
73.25	0.21378	-103.079867827442	-103.079867827442\\
73.25	0.21744	-107.778714506464	-107.778714506464\\
73.25	0.2211	-112.599792849296	-112.599792849296\\
73.25	0.22476	-117.543102855939	-117.543102855939\\
73.25	0.22842	-122.608644526392	-122.608644526392\\
73.25	0.23208	-127.796417860657	-127.796417860657\\
73.25	0.23574	-133.106422858732	-133.106422858732\\
73.25	0.2394	-138.538659520618	-138.538659520618\\
73.25	0.24306	-144.093127846315	-144.093127846315\\
73.25	0.24672	-149.769827835822	-149.769827835822\\
73.25	0.25038	-155.56875948914	-155.56875948914\\
73.25	0.25404	-161.489922806269	-161.489922806269\\
73.25	0.2577	-167.533317787209	-167.533317787209\\
73.25	0.26136	-173.698944431959	-173.698944431959\\
73.25	0.26502	-179.986802740521	-179.986802740521\\
73.25	0.26868	-186.396892712893	-186.396892712893\\
73.25	0.27234	-192.929214349075	-192.929214349075\\
73.25	0.276	-199.583767649069	-199.583767649069\\
73.625	0.093	-16.9389909603185	-16.9389909603185\\
73.625	0.09666	-17.6069715690952	-17.6069715690952\\
73.625	0.10032	-18.3971838416827	-18.3971838416827\\
73.625	0.10398	-19.3096277780809	-19.3096277780809\\
73.625	0.10764	-20.3443033782899	-20.3443033782899\\
73.625	0.1113	-21.5012106423096	-21.5012106423096\\
73.625	0.11496	-22.7803495701401	-22.7803495701401\\
73.625	0.11862	-24.1817201617814	-24.1817201617814\\
73.625	0.12228	-25.7053224172334	-25.7053224172334\\
73.625	0.12594	-27.3511563364962	-27.3511563364962\\
73.625	0.1296	-29.1192219195697	-29.1192219195697\\
73.625	0.13326	-31.009519166454	-31.009519166454\\
73.625	0.13692	-33.022048077149	-33.022048077149\\
73.625	0.14058	-35.1568086516548	-35.1568086516548\\
73.625	0.14424	-37.4138008899714	-37.4138008899714\\
73.625	0.1479	-39.7930247920987	-39.7930247920987\\
73.625	0.15156	-42.2944803580368	-42.2944803580368\\
73.625	0.15522	-44.9181675877856	-44.9181675877856\\
73.625	0.15888	-47.6640864813452	-47.6640864813452\\
73.625	0.16254	-50.5322370387156	-50.5322370387156\\
73.625	0.1662	-53.5226192598967	-53.5226192598967\\
73.625	0.16986	-56.6352331448885	-56.6352331448885\\
73.625	0.17352	-59.8700786936912	-59.8700786936912\\
73.625	0.17718	-63.2271559063045	-63.2271559063045\\
73.625	0.18084	-66.7064647827287	-66.7064647827287\\
73.625	0.1845	-70.3080053229636	-70.3080053229636\\
73.625	0.18816	-74.0317775270093	-74.0317775270093\\
73.625	0.19182	-77.8777813948657	-77.8777813948657\\
73.625	0.19548	-81.8460169265329	-81.8460169265329\\
73.625	0.19914	-85.9364841220109	-85.9364841220109\\
73.625	0.2028	-90.1491829812996	-90.1491829812996\\
73.625	0.20646	-94.484113504399	-94.484113504399\\
73.625	0.21012	-98.9412756913093	-98.9412756913093\\
73.625	0.21378	-103.52066954203	-103.52066954203\\
73.625	0.21744	-108.222295056562	-108.222295056562\\
73.625	0.2211	-113.046152234904	-113.046152234904\\
73.625	0.22476	-117.992241077058	-117.992241077058\\
73.625	0.22842	-123.060561583022	-123.060561583022\\
73.625	0.23208	-128.251113752796	-128.251113752796\\
73.625	0.23574	-133.563897586382	-133.563897586382\\
73.625	0.2394	-138.998913083778	-138.998913083778\\
73.625	0.24306	-144.556160244985	-144.556160244985\\
73.625	0.24672	-150.235639070003	-150.235639070003\\
73.625	0.25038	-156.037349558832	-156.037349558832\\
73.625	0.25404	-161.961291711471	-161.961291711471\\
73.625	0.2577	-168.007465527921	-168.007465527921\\
73.625	0.26136	-174.175871008182	-174.175871008182\\
73.625	0.26502	-180.466508152254	-180.466508152254\\
73.625	0.26868	-186.879376960136	-186.879376960136\\
73.625	0.27234	-193.414477431829	-193.414477431829\\
73.625	0.276	-200.071809567333	-200.071809567333\\
74	0.093	-17.2976874756194	-17.2976874756194\\
74	0.09666	-17.9684469199065	-17.9684469199065\\
74	0.10032	-18.7614380280043	-18.7614380280043\\
74	0.10398	-19.6766607999128	-19.6766607999128\\
74	0.10764	-20.7141152356322	-20.7141152356322\\
74	0.1113	-21.8738013351623	-21.8738013351623\\
74	0.11496	-23.1557190985031	-23.1557190985031\\
74	0.11862	-24.5598685256547	-24.5598685256547\\
74	0.12228	-26.0862496166171	-26.0862496166171\\
74	0.12594	-27.7348623713902	-27.7348623713902\\
74	0.1296	-29.505706789974	-29.505706789974\\
74	0.13326	-31.3987828723687	-31.3987828723687\\
74	0.13692	-33.414090618574	-33.414090618574\\
74	0.14058	-35.5516300285902	-35.5516300285902\\
74	0.14424	-37.8114011024172	-37.8114011024172\\
74	0.1479	-40.1934038400548	-40.1934038400548\\
74	0.15156	-42.6976382415032	-42.6976382415032\\
74	0.15522	-45.3241043067624	-45.3241043067624\\
74	0.15888	-48.0728020358324	-48.0728020358324\\
74	0.16254	-50.9437314287131	-50.9437314287131\\
74	0.1662	-53.9368924854046	-53.9368924854046\\
74	0.16986	-57.0522852059068	-57.0522852059068\\
74	0.17352	-60.2899095902197	-60.2899095902197\\
74	0.17718	-63.6497656383435	-63.6497656383435\\
74	0.18084	-67.1318533502779	-67.1318533502779\\
74	0.1845	-70.7361727260233	-70.7361727260233\\
74	0.18816	-74.4627237655792	-74.4627237655792\\
74	0.19182	-78.311506468946	-78.311506468946\\
74	0.19548	-82.2825208361235	-82.2825208361235\\
74	0.19914	-86.3757668671118	-86.3757668671118\\
74	0.2028	-90.5912445619109	-90.5912445619109\\
74	0.20646	-94.9289539205207	-94.9289539205207\\
74	0.21012	-99.3888949429413	-99.3888949429413\\
74	0.21378	-103.971067629173	-103.971067629173\\
74	0.21744	-108.675471979215	-108.675471979215\\
74	0.2211	-113.502107993068	-113.502107993068\\
74	0.22476	-118.450975670731	-118.450975670731\\
74	0.22842	-123.522075012205	-123.522075012205\\
74	0.23208	-128.715406017491	-128.715406017491\\
74	0.23574	-134.030968686587	-134.030968686587\\
74	0.2394	-139.468763019493	-139.468763019493\\
74	0.24306	-145.02878901621	-145.02878901621\\
74	0.24672	-150.711046676739	-150.711046676739\\
74	0.25038	-156.515536001077	-156.515536001077\\
74	0.25404	-162.442256989227	-162.442256989227\\
74	0.2577	-168.491209641188	-168.491209641188\\
74	0.26136	-174.662393956959	-174.662393956959\\
74	0.26502	-180.955809936541	-180.955809936541\\
74	0.26868	-187.371457579933	-187.371457579933\\
74	0.27234	-193.909336887137	-193.909336887137\\
74	0.276	-200.569447858151	-200.569447858151\\
};
\end{axis}

\begin{axis}[%
width=6.159cm,
height=3.097cm,
at={(8.104cm,12.903cm)},
scale only axis,
xmin=56,
xmax=74,
tick align=outside,
xlabel style={font=\color{white!15!black}},
xlabel={$L_{cut}$},
ymin=0.093,
ymax=0.276,
ylabel style={font=\color{white!15!black}},
ylabel={$D_{rlx}$},
zmin=0,
zmax=517.903549897876,
zlabel style={font=\color{white!15!black}},
zlabel={$u(t-1)$},
view={-140}{50},
axis background/.style={fill=white},
xmajorgrids,
ymajorgrids,
zmajorgrids
]
\addplot3[only marks, mark=*, mark options={}, mark size=1.5000pt, color=mycolor1, fill=mycolor1] table[row sep=crcr]{%
x	y	z\\
74	0.123	75.4245651900459\\
72	0.113	59.5782863482935\\
61	0.095	33.0003406266977\\
56	0.093	26.0569470415911\\
};
\addplot3[only marks, mark=*, mark options={}, mark size=1.5000pt, color=mycolor2, fill=mycolor2] table[row sep=crcr]{%
x	y	z\\
67	0.276	508.93731875208\\
66	0.255	419.53506953587\\
62	0.209	242.35439857544\\
57	0.193	195.15231965962\\
};
\addplot3[only marks, mark=*, mark options={}, mark size=1.5000pt, color=black, fill=black] table[row sep=crcr]{%
x	y	z\\
69	0.104	48.1088130226734\\
};
\addplot3[only marks, mark=*, mark options={}, mark size=1.5000pt, color=black, fill=black] table[row sep=crcr]{%
x	y	z\\
64	0.23	317.23878457902\\
};

\addplot3[%
surf,
fill opacity=0.7, shader=interp, colormap={mymap}{[1pt] rgb(0pt)=(1,0.905882,0); rgb(1pt)=(1,0.901964,0); rgb(2pt)=(1,0.898051,0); rgb(3pt)=(1,0.894144,0); rgb(4pt)=(1,0.890243,0); rgb(5pt)=(1,0.886349,0); rgb(6pt)=(1,0.88246,0); rgb(7pt)=(1,0.878577,0); rgb(8pt)=(1,0.8747,0); rgb(9pt)=(1,0.870829,0); rgb(10pt)=(1,0.866964,0); rgb(11pt)=(1,0.863106,0); rgb(12pt)=(1,0.859253,0); rgb(13pt)=(1,0.855406,0); rgb(14pt)=(1,0.851566,0); rgb(15pt)=(1,0.847732,0); rgb(16pt)=(1,0.843903,0); rgb(17pt)=(1,0.840081,0); rgb(18pt)=(1,0.836265,0); rgb(19pt)=(1,0.832455,0); rgb(20pt)=(1,0.828652,0); rgb(21pt)=(1,0.824854,0); rgb(22pt)=(1,0.821063,0); rgb(23pt)=(1,0.817278,0); rgb(24pt)=(1,0.8135,0); rgb(25pt)=(1,0.809727,0); rgb(26pt)=(1,0.805961,0); rgb(27pt)=(1,0.8022,0); rgb(28pt)=(1,0.798445,0); rgb(29pt)=(1,0.794696,0); rgb(30pt)=(1,0.790953,0); rgb(31pt)=(1,0.787215,0); rgb(32pt)=(1,0.783484,0); rgb(33pt)=(1,0.779758,0); rgb(34pt)=(1,0.776038,0); rgb(35pt)=(1,0.772324,0); rgb(36pt)=(1,0.768615,0); rgb(37pt)=(1,0.764913,0); rgb(38pt)=(1,0.761217,0); rgb(39pt)=(1,0.757527,0); rgb(40pt)=(1,0.753843,0); rgb(41pt)=(1,0.750165,0); rgb(42pt)=(1,0.746493,0); rgb(43pt)=(1,0.742827,0); rgb(44pt)=(1,0.739167,0); rgb(45pt)=(1,0.735514,0); rgb(46pt)=(1,0.731867,0); rgb(47pt)=(1,0.728226,0); rgb(48pt)=(1,0.724591,0); rgb(49pt)=(1,0.720963,0); rgb(50pt)=(1,0.717341,0); rgb(51pt)=(1,0.713725,0); rgb(52pt)=(0.999994,0.710077,0); rgb(53pt)=(0.999974,0.706363,0); rgb(54pt)=(0.999942,0.702592,0); rgb(55pt)=(0.999898,0.698775,0); rgb(56pt)=(0.999841,0.694921,0); rgb(57pt)=(0.999771,0.691039,0); rgb(58pt)=(0.99969,0.687139,0); rgb(59pt)=(0.999596,0.68323,0); rgb(60pt)=(0.99949,0.679323,0); rgb(61pt)=(0.999372,0.675427,0); rgb(62pt)=(0.999242,0.67155,0); rgb(63pt)=(0.9991,0.667704,0); rgb(64pt)=(0.998946,0.663897,0); rgb(65pt)=(0.998781,0.660138,0); rgb(66pt)=(0.998605,0.656439,0); rgb(67pt)=(0.998416,0.652807,0); rgb(68pt)=(0.998217,0.649253,0); rgb(69pt)=(0.998006,0.645786,0); rgb(70pt)=(0.997785,0.642416,0); rgb(71pt)=(0.997552,0.639152,0); rgb(72pt)=(0.997308,0.636004,0); rgb(73pt)=(0.997053,0.632982,0); rgb(74pt)=(0.996788,0.630095,0); rgb(75pt)=(0.996512,0.627352,0); rgb(76pt)=(0.996226,0.624763,0); rgb(77pt)=(0.995851,0.622329,0); rgb(78pt)=(0.99494,0.619997,0); rgb(79pt)=(0.99345,0.617753,0); rgb(80pt)=(0.991419,0.61559,0); rgb(81pt)=(0.988885,0.613503,0); rgb(82pt)=(0.985886,0.611486,0); rgb(83pt)=(0.98246,0.609532,0); rgb(84pt)=(0.978643,0.607636,0); rgb(85pt)=(0.974475,0.605791,0); rgb(86pt)=(0.969992,0.603992,0); rgb(87pt)=(0.965232,0.602233,0); rgb(88pt)=(0.960233,0.600507,0); rgb(89pt)=(0.955033,0.598808,0); rgb(90pt)=(0.949669,0.59713,0); rgb(91pt)=(0.94418,0.595468,0); rgb(92pt)=(0.938602,0.593815,0); rgb(93pt)=(0.932974,0.592166,0); rgb(94pt)=(0.927333,0.590513,0); rgb(95pt)=(0.921717,0.588852,0); rgb(96pt)=(0.916164,0.587176,0); rgb(97pt)=(0.910711,0.585479,0); rgb(98pt)=(0.905397,0.583755,0); rgb(99pt)=(0.900258,0.581999,0); rgb(100pt)=(0.895333,0.580203,0); rgb(101pt)=(0.890659,0.578362,0); rgb(102pt)=(0.886275,0.576471,0); rgb(103pt)=(0.882047,0.574545,0); rgb(104pt)=(0.877819,0.572608,0); rgb(105pt)=(0.873592,0.57066,0); rgb(106pt)=(0.869366,0.568701,0); rgb(107pt)=(0.865143,0.566733,0); rgb(108pt)=(0.860924,0.564756,0); rgb(109pt)=(0.856708,0.562771,0); rgb(110pt)=(0.852497,0.560778,0); rgb(111pt)=(0.848292,0.558779,0); rgb(112pt)=(0.844092,0.556774,0); rgb(113pt)=(0.8399,0.554763,0); rgb(114pt)=(0.835716,0.552749,0); rgb(115pt)=(0.831541,0.55073,0); rgb(116pt)=(0.827374,0.548709,0); rgb(117pt)=(0.823219,0.546686,0); rgb(118pt)=(0.819074,0.54466,0); rgb(119pt)=(0.81494,0.542635,0); rgb(120pt)=(0.81082,0.540609,0); rgb(121pt)=(0.806712,0.538584,0); rgb(122pt)=(0.802619,0.53656,0); rgb(123pt)=(0.798541,0.534539,0); rgb(124pt)=(0.794478,0.532521,0); rgb(125pt)=(0.790431,0.530506,0); rgb(126pt)=(0.786402,0.528496,0); rgb(127pt)=(0.782391,0.526491,0); rgb(128pt)=(0.77841,0.524489,0); rgb(129pt)=(0.774523,0.522478,0); rgb(130pt)=(0.770731,0.520455,0); rgb(131pt)=(0.767022,0.518424,0); rgb(132pt)=(0.763384,0.516385,0); rgb(133pt)=(0.759804,0.514339,0); rgb(134pt)=(0.756272,0.51229,0); rgb(135pt)=(0.752775,0.510237,0); rgb(136pt)=(0.749302,0.508182,0); rgb(137pt)=(0.74584,0.506128,0); rgb(138pt)=(0.742378,0.504075,0); rgb(139pt)=(0.738904,0.502025,0); rgb(140pt)=(0.735406,0.499979,0); rgb(141pt)=(0.731872,0.49794,0); rgb(142pt)=(0.72829,0.495909,0); rgb(143pt)=(0.724649,0.493887,0); rgb(144pt)=(0.720936,0.491875,0); rgb(145pt)=(0.71714,0.489876,0); rgb(146pt)=(0.713249,0.487891,0); rgb(147pt)=(0.709251,0.485921,0); rgb(148pt)=(0.705134,0.483968,0); rgb(149pt)=(0.700887,0.482033,0); rgb(150pt)=(0.696497,0.480118,0); rgb(151pt)=(0.691952,0.478225,0); rgb(152pt)=(0.687242,0.476355,0); rgb(153pt)=(0.682353,0.47451,0); rgb(154pt)=(0.677195,0.472696,0); rgb(155pt)=(0.6717,0.470916,0); rgb(156pt)=(0.665891,0.469169,0); rgb(157pt)=(0.659791,0.46745,0); rgb(158pt)=(0.653423,0.465756,0); rgb(159pt)=(0.64681,0.464084,0); rgb(160pt)=(0.639976,0.462432,0); rgb(161pt)=(0.632943,0.460795,0); rgb(162pt)=(0.625734,0.459171,0); rgb(163pt)=(0.618373,0.457556,0); rgb(164pt)=(0.610882,0.455948,0); rgb(165pt)=(0.603284,0.454343,0); rgb(166pt)=(0.595604,0.452737,0); rgb(167pt)=(0.587863,0.451129,0); rgb(168pt)=(0.580084,0.449514,0); rgb(169pt)=(0.572292,0.447889,0); rgb(170pt)=(0.564508,0.446252,0); rgb(171pt)=(0.556756,0.444599,0); rgb(172pt)=(0.549059,0.442927,0); rgb(173pt)=(0.54144,0.441232,0); rgb(174pt)=(0.533922,0.439512,0); rgb(175pt)=(0.526529,0.437764,0); rgb(176pt)=(0.519282,0.435983,0); rgb(177pt)=(0.512206,0.434168,0); rgb(178pt)=(0.505323,0.432315,0); rgb(179pt)=(0.498628,0.430422,3.92506e-06); rgb(180pt)=(0.491973,0.428504,3.49981e-05); rgb(181pt)=(0.485331,0.426562,9.63073e-05); rgb(182pt)=(0.478704,0.424596,0.000186979); rgb(183pt)=(0.472096,0.422609,0.000306141); rgb(184pt)=(0.465508,0.420599,0.00045292); rgb(185pt)=(0.458942,0.418567,0.000626441); rgb(186pt)=(0.452401,0.416515,0.000825833); rgb(187pt)=(0.445885,0.414441,0.00105022); rgb(188pt)=(0.439399,0.412348,0.00129873); rgb(189pt)=(0.432942,0.410234,0.00157049); rgb(190pt)=(0.426518,0.408102,0.00186463); rgb(191pt)=(0.420129,0.40595,0.00218028); rgb(192pt)=(0.413777,0.40378,0.00251655); rgb(193pt)=(0.407464,0.401592,0.00287258); rgb(194pt)=(0.401191,0.399386,0.00324749); rgb(195pt)=(0.394962,0.397164,0.00364042); rgb(196pt)=(0.388777,0.394925,0.00405048); rgb(197pt)=(0.38264,0.39267,0.00447681); rgb(198pt)=(0.376552,0.390399,0.00491852); rgb(199pt)=(0.370516,0.388113,0.00537476); rgb(200pt)=(0.364532,0.385812,0.00584464); rgb(201pt)=(0.358605,0.383497,0.00632729); rgb(202pt)=(0.352735,0.381168,0.00682184); rgb(203pt)=(0.346925,0.378826,0.00732741); rgb(204pt)=(0.341176,0.376471,0.00784314); rgb(205pt)=(0.335485,0.374093,0.00847245); rgb(206pt)=(0.329843,0.371682,0.00930909); rgb(207pt)=(0.324249,0.369242,0.0103377); rgb(208pt)=(0.318701,0.366772,0.0115428); rgb(209pt)=(0.313198,0.364275,0.0129091); rgb(210pt)=(0.307739,0.361753,0.0144211); rgb(211pt)=(0.302322,0.359206,0.0160634); rgb(212pt)=(0.296945,0.356637,0.0178207); rgb(213pt)=(0.291607,0.354048,0.0196776); rgb(214pt)=(0.286307,0.35144,0.0216186); rgb(215pt)=(0.281043,0.348814,0.0236284); rgb(216pt)=(0.275813,0.346172,0.0256916); rgb(217pt)=(0.270616,0.343517,0.0277927); rgb(218pt)=(0.265451,0.340849,0.0299163); rgb(219pt)=(0.260317,0.33817,0.0320472); rgb(220pt)=(0.25521,0.335482,0.0341698); rgb(221pt)=(0.250131,0.332786,0.0362688); rgb(222pt)=(0.245078,0.330085,0.0383287); rgb(223pt)=(0.240048,0.327379,0.0403343); rgb(224pt)=(0.235042,0.324671,0.04227); rgb(225pt)=(0.230056,0.321962,0.0441205); rgb(226pt)=(0.22509,0.319254,0.0458704); rgb(227pt)=(0.220142,0.316548,0.0475043); rgb(228pt)=(0.215212,0.313846,0.0490067); rgb(229pt)=(0.210296,0.311149,0.0503624); rgb(230pt)=(0.205395,0.308459,0.0515759); rgb(231pt)=(0.200514,0.305763,0.052757); rgb(232pt)=(0.195655,0.303061,0.0539242); rgb(233pt)=(0.190817,0.300353,0.0550763); rgb(234pt)=(0.186001,0.297639,0.0562123); rgb(235pt)=(0.181207,0.294918,0.0573313); rgb(236pt)=(0.176434,0.292191,0.0584321); rgb(237pt)=(0.171685,0.289458,0.0595136); rgb(238pt)=(0.166957,0.286719,0.060575); rgb(239pt)=(0.162252,0.283973,0.0616151); rgb(240pt)=(0.15757,0.281221,0.0626328); rgb(241pt)=(0.152911,0.278463,0.0636271); rgb(242pt)=(0.148275,0.275699,0.0645971); rgb(243pt)=(0.143663,0.272929,0.0655416); rgb(244pt)=(0.139074,0.270152,0.0664596); rgb(245pt)=(0.134508,0.26737,0.06735); rgb(246pt)=(0.129967,0.264581,0.0682118); rgb(247pt)=(0.125449,0.261787,0.0690441); rgb(248pt)=(0.120956,0.258986,0.0698456); rgb(249pt)=(0.116487,0.25618,0.0706154); rgb(250pt)=(0.112043,0.253367,0.0713525); rgb(251pt)=(0.107623,0.250549,0.0720557); rgb(252pt)=(0.103229,0.247724,0.0727241); rgb(253pt)=(0.0988592,0.244894,0.0733566); rgb(254pt)=(0.0945149,0.242058,0.0739522); rgb(255pt)=(0.0901961,0.239216,0.0745098)}, mesh/rows=49]
table[row sep=crcr, point meta=\thisrow{c}] {%
%
x	y	z	c\\
56	0.093	26.7122935396587	26.7122935396587\\
56	0.09666	28.8008150374213	28.8008150374213\\
56	0.10032	31.1960003360339	31.1960003360339\\
56	0.10398	33.8978494354966	33.8978494354966\\
56	0.10764	36.9063623358092	36.9063623358092\\
56	0.1113	40.2215390369718	40.2215390369718\\
56	0.11496	43.8433795389845	43.8433795389845\\
56	0.11862	47.7718838418471	47.7718838418471\\
56	0.12228	52.0070519455598	52.0070519455598\\
56	0.12594	56.5488838501225	56.5488838501225\\
56	0.1296	61.3973795555351	61.3973795555351\\
56	0.13326	66.5525390617977	66.5525390617977\\
56	0.13692	72.0143623689104	72.0143623689104\\
56	0.14058	77.7828494768732	77.7828494768732\\
56	0.14424	83.8580003856859	83.8580003856859\\
56	0.1479	90.2398150953486	90.2398150953486\\
56	0.15156	96.9282936058613	96.9282936058613\\
56	0.15522	103.923435917224	103.923435917224\\
56	0.15888	111.225242029437	111.225242029437\\
56	0.16254	118.8337119425	118.8337119425\\
56	0.1662	126.748845656412	126.748845656412\\
56	0.16986	134.970643171175	134.970643171175\\
56	0.17352	143.499104486788	143.499104486788\\
56	0.17718	152.334229603251	152.334229603251\\
56	0.18084	161.476018520563	161.476018520563\\
56	0.1845	170.924471238726	170.924471238726\\
56	0.18816	180.679587757739	180.679587757739\\
56	0.19182	190.741368077602	190.741368077602\\
56	0.19548	201.109812198315	201.109812198315\\
56	0.19914	211.784920119878	211.784920119878\\
56	0.2028	222.766691842291	222.766691842291\\
56	0.20646	234.055127365553	234.055127365553\\
56	0.21012	245.650226689666	245.650226689666\\
56	0.21378	257.551989814629	257.551989814629\\
56	0.21744	269.760416740442	269.760416740442\\
56	0.2211	282.275507467105	282.275507467105\\
56	0.22476	295.097261994618	295.097261994618\\
56	0.22842	308.225680322981	308.225680322981\\
56	0.23208	321.660762452193	321.660762452193\\
56	0.23574	335.402508382256	335.402508382256\\
56	0.2394	349.450918113169	349.450918113169\\
56	0.24306	363.805991644932	363.805991644932\\
56	0.24672	378.467728977545	378.467728977545\\
56	0.25038	393.436130111008	393.436130111008\\
56	0.25404	408.711195045321	408.711195045321\\
56	0.2577	424.292923780484	424.292923780484\\
56	0.26136	440.181316316497	440.181316316497\\
56	0.26502	456.37637265336	456.37637265336\\
56	0.26868	472.878092791073	472.878092791073\\
56	0.27234	489.686476729636	489.686476729636\\
56	0.276	506.801524469049	506.801524469049\\
56.375	0.093	26.9533925013348	26.9533925013348\\
56.375	0.09666	29.0365678713119	29.0365678713119\\
56.375	0.10032	31.426407042139	31.426407042139\\
56.375	0.10398	34.1229100138161	34.1229100138161\\
56.375	0.10764	37.1260767863432	37.1260767863432\\
56.375	0.1113	40.4359073597203	40.4359073597203\\
56.375	0.11496	44.0524017339474	44.0524017339474\\
56.375	0.11862	47.9755599090246	47.9755599090246\\
56.375	0.12228	52.2053818849518	52.2053818849518\\
56.375	0.12594	56.7418676617289	56.7418676617289\\
56.375	0.1296	61.5850172393561	61.5850172393561\\
56.375	0.13326	66.7348306178332	66.7348306178332\\
56.375	0.13692	72.1913077971604	72.1913077971604\\
56.375	0.14058	77.9544487773377	77.9544487773377\\
56.375	0.14424	84.0242535583649	84.0242535583649\\
56.375	0.1479	90.4007221402421	90.4007221402421\\
56.375	0.15156	97.0838545229693	97.0838545229693\\
56.375	0.15522	104.073650706547	104.073650706547\\
56.375	0.15888	111.370110690974	111.370110690974\\
56.375	0.16254	118.973234476251	118.973234476251\\
56.375	0.1662	126.883022062378	126.883022062378\\
56.375	0.16986	135.099473449356	135.099473449356\\
56.375	0.17352	143.622588637183	143.622588637183\\
56.375	0.17718	152.45236762586	152.45236762586\\
56.375	0.18084	161.588810415387	161.588810415387\\
56.375	0.1845	171.031917005765	171.031917005765\\
56.375	0.18816	180.781687396992	180.781687396992\\
56.375	0.19182	190.838121589069	190.838121589069\\
56.375	0.19548	201.201219581997	201.201219581997\\
56.375	0.19914	211.870981375774	211.870981375774\\
56.375	0.2028	222.847406970401	222.847406970401\\
56.375	0.20646	234.130496365879	234.130496365879\\
56.375	0.21012	245.720249562206	245.720249562206\\
56.375	0.21378	257.616666559383	257.616666559383\\
56.375	0.21744	269.819747357411	269.819747357411\\
56.375	0.2211	282.329491956288	282.329491956288\\
56.375	0.22476	295.145900356016	295.145900356016\\
56.375	0.22842	308.268972556593	308.268972556593\\
56.375	0.23208	321.69870855802	321.69870855802\\
56.375	0.23574	335.435108360298	335.435108360298\\
56.375	0.2394	349.478171963425	349.478171963425\\
56.375	0.24306	363.827899367403	363.827899367403\\
56.375	0.24672	378.48429057223	378.48429057223\\
56.375	0.25038	393.447345577908	393.447345577908\\
56.375	0.25404	408.717064384435	408.717064384435\\
56.375	0.2577	424.293446991813	424.293446991813\\
56.375	0.26136	440.17649340004	440.17649340004\\
56.375	0.26502	456.366203609117	456.366203609117\\
56.375	0.26868	472.862577619045	472.862577619045\\
56.375	0.27234	489.665615429822	489.665615429822\\
56.375	0.276	506.77531704145	506.77531704145\\
56.75	0.093	27.2054488938207	27.2054488938207\\
56.75	0.09666	29.2832781360123	29.2832781360123\\
56.75	0.10032	31.6677711790539	31.6677711790539\\
56.75	0.10398	34.3589280229455	34.3589280229455\\
56.75	0.10764	37.3567486676872	37.3567486676872\\
56.75	0.1113	40.6612331132787	40.6612331132787\\
56.75	0.11496	44.2723813597204	44.2723813597204\\
56.75	0.11862	48.190193407012	48.190193407012\\
56.75	0.12228	52.4146692551537	52.4146692551537\\
56.75	0.12594	56.9458089041454	56.9458089041454\\
56.75	0.1296	61.783612353987	61.783612353987\\
56.75	0.13326	66.9280796046787	66.9280796046787\\
56.75	0.13692	72.3792106562203	72.3792106562203\\
56.75	0.14058	78.1370055086121	78.1370055086121\\
56.75	0.14424	84.2014641618537	84.2014641618537\\
56.75	0.1479	90.5725866159455	90.5725866159455\\
56.75	0.15156	97.2503728708871	97.2503728708871\\
56.75	0.15522	104.234822926679	104.234822926679\\
56.75	0.15888	111.525936783321	111.525936783321\\
56.75	0.16254	119.123714440812	119.123714440812\\
56.75	0.1662	127.028155899154	127.028155899154\\
56.75	0.16986	135.239261158346	135.239261158346\\
56.75	0.17352	143.757030218388	143.757030218388\\
56.75	0.17718	152.581463079279	152.581463079279\\
56.75	0.18084	161.712559741021	161.712559741021\\
56.75	0.1845	171.150320203613	171.150320203613\\
56.75	0.18816	180.894744467055	180.894744467055\\
56.75	0.19182	190.945832531347	190.945832531347\\
56.75	0.19548	201.303584396488	201.303584396488\\
56.75	0.19914	211.96800006248	211.96800006248\\
56.75	0.2028	222.939079529322	222.939079529322\\
56.75	0.20646	234.216822797014	234.216822797014\\
56.75	0.21012	245.801229865556	245.801229865556\\
56.75	0.21378	257.692300734948	257.692300734948\\
56.75	0.21744	269.890035405189	269.890035405189\\
56.75	0.2211	282.394433876281	282.394433876281\\
56.75	0.22476	295.205496148223	295.205496148223\\
56.75	0.22842	308.323222221015	308.323222221015\\
56.75	0.23208	321.747612094657	321.747612094657\\
56.75	0.23574	335.478665769149	335.478665769149\\
56.75	0.2394	349.516383244491	349.516383244491\\
56.75	0.24306	363.860764520683	363.860764520683\\
56.75	0.24672	378.511809597725	378.511809597725\\
56.75	0.25038	393.469518475617	393.469518475617\\
56.75	0.25404	408.733891154359	408.733891154359\\
56.75	0.2577	424.304927633951	424.304927633951\\
56.75	0.26136	440.182627914393	440.182627914393\\
56.75	0.26502	456.366991995685	456.366991995685\\
56.75	0.26868	472.858019877827	472.858019877827\\
56.75	0.27234	489.655711560819	489.655711560819\\
56.75	0.276	506.76006704466	506.76006704466\\
57.125	0.093	27.4684627171166	27.4684627171166\\
57.125	0.09666	29.5409458315226	29.5409458315226\\
57.125	0.10032	31.9200927467787	31.9200927467787\\
57.125	0.10398	34.6059034628848	34.6059034628848\\
57.125	0.10764	37.598377979841	37.598377979841\\
57.125	0.1113	40.8975162976471	40.8975162976471\\
57.125	0.11496	44.5033184163032	44.5033184163032\\
57.125	0.11862	48.4157843358093	48.4157843358093\\
57.125	0.12228	52.6349140561655	52.6349140561655\\
57.125	0.12594	57.1607075773716	57.1607075773716\\
57.125	0.1296	61.9931648994278	61.9931648994278\\
57.125	0.13326	67.1322860223339	67.1322860223339\\
57.125	0.13692	72.5780709460901	72.5780709460901\\
57.125	0.14058	78.3305196706963	78.3305196706963\\
57.125	0.14424	84.3896321961525	84.3896321961525\\
57.125	0.1479	90.7554085224587	90.7554085224587\\
57.125	0.15156	97.4278486496149	97.4278486496149\\
57.125	0.15522	104.406952577621	104.406952577621\\
57.125	0.15888	111.692720306477	111.692720306477\\
57.125	0.16254	119.285151836184	119.285151836184\\
57.125	0.1662	127.18424716674	127.18424716674\\
57.125	0.16986	135.390006298146	135.390006298146\\
57.125	0.17352	143.902429230402	143.902429230402\\
57.125	0.17718	152.721515963509	152.721515963509\\
57.125	0.18084	161.847266497465	161.847266497465\\
57.125	0.1845	171.279680832271	171.279680832271\\
57.125	0.18816	181.018758967928	181.018758967928\\
57.125	0.19182	191.064500904434	191.064500904434\\
57.125	0.19548	201.41690664179	201.41690664179\\
57.125	0.19914	212.075976179996	212.075976179996\\
57.125	0.2028	223.041709519053	223.041709519053\\
57.125	0.20646	234.314106658959	234.314106658959\\
57.125	0.21012	245.893167599716	245.893167599716\\
57.125	0.21378	257.778892341322	257.778892341322\\
57.125	0.21744	269.971280883778	269.971280883778\\
57.125	0.2211	282.470333227085	282.470333227085\\
57.125	0.22476	295.276049371241	295.276049371241\\
57.125	0.22842	308.388429316247	308.388429316247\\
57.125	0.23208	321.807473062104	321.807473062104\\
57.125	0.23574	335.53318060881	335.53318060881\\
57.125	0.2394	349.565551956367	349.565551956367\\
57.125	0.24306	363.904587104773	363.904587104773\\
57.125	0.24672	378.550286054029	378.550286054029\\
57.125	0.25038	393.502648804136	393.502648804136\\
57.125	0.25404	408.761675355092	408.761675355092\\
57.125	0.2577	424.327365706899	424.327365706899\\
57.125	0.26136	440.199719859555	440.199719859555\\
57.125	0.26502	456.378737813062	456.378737813062\\
57.125	0.26868	472.864419567418	472.864419567418\\
57.125	0.27234	489.656765122625	489.656765122625\\
57.125	0.276	506.755774478681	506.755774478681\\
57.5	0.093	27.7424339712224	27.7424339712224\\
57.5	0.09666	29.8095709578429	29.8095709578429\\
57.5	0.10032	32.1833717453135	32.1833717453135\\
57.5	0.10398	34.8638363336341	34.8638363336341\\
57.5	0.10764	37.8509647228047	37.8509647228047\\
57.5	0.1113	41.1447569128253	41.1447569128253\\
57.5	0.11496	44.7452129036959	44.7452129036959\\
57.5	0.11862	48.6523326954166	48.6523326954166\\
57.5	0.12228	52.8661162879872	52.8661162879872\\
57.5	0.12594	57.3865636814079	57.3865636814079\\
57.5	0.1296	62.2136748756785	62.2136748756785\\
57.5	0.13326	67.3474498707992	67.3474498707992\\
57.5	0.13692	72.7878886667698	72.7878886667698\\
57.5	0.14058	78.5349912635905	78.5349912635905\\
57.5	0.14424	84.5887576612612	84.5887576612612\\
57.5	0.1479	90.9491878597819	90.9491878597819\\
57.5	0.15156	97.6162818591526	97.6162818591526\\
57.5	0.15522	104.590039659373	104.590039659373\\
57.5	0.15888	111.870461260444	111.870461260444\\
57.5	0.16254	119.457546662365	119.457546662365\\
57.5	0.1662	127.351295865136	127.351295865136\\
57.5	0.16986	135.551708868756	135.551708868756\\
57.5	0.17352	144.058785673227	144.058785673227\\
57.5	0.17718	152.872526278548	152.872526278548\\
57.5	0.18084	161.992930684719	161.992930684719\\
57.5	0.1845	171.419998891739	171.419998891739\\
57.5	0.18816	181.15373089961	181.15373089961\\
57.5	0.19182	191.194126708331	191.194126708331\\
57.5	0.19548	201.541186317902	201.541186317902\\
57.5	0.19914	212.194909728323	212.194909728323\\
57.5	0.2028	223.155296939594	223.155296939594\\
57.5	0.20646	234.422347951714	234.422347951714\\
57.5	0.21012	245.996062764685	245.996062764685\\
57.5	0.21378	257.876441378506	257.876441378506\\
57.5	0.21744	270.063483793177	270.063483793177\\
57.5	0.2211	282.557190008698	282.557190008698\\
57.5	0.22476	295.357560025069	295.357560025069\\
57.5	0.22842	308.464593842289	308.464593842289\\
57.5	0.23208	321.87829146036	321.87829146036\\
57.5	0.23574	335.598652879281	335.598652879281\\
57.5	0.2394	349.625678099052	349.625678099052\\
57.5	0.24306	363.959367119673	363.959367119673\\
57.5	0.24672	378.599719941144	378.599719941144\\
57.5	0.25038	393.546736563465	393.546736563465\\
57.5	0.25404	408.800416986636	408.800416986636\\
57.5	0.2577	424.360761210657	424.360761210657\\
57.5	0.26136	440.227769235528	440.227769235528\\
57.5	0.26502	456.401441061249	456.401441061249\\
57.5	0.26868	472.88177668782	472.88177668782\\
57.5	0.27234	489.668776115241	489.668776115241\\
57.5	0.276	506.762439343512	506.762439343512\\
57.875	0.093	28.0273626561381	28.0273626561381\\
57.875	0.09666	30.0891535149731	30.0891535149731\\
57.875	0.10032	32.4576081746582	32.4576081746582\\
57.875	0.10398	35.1327266351933	35.1327266351933\\
57.875	0.10764	38.1145088965784	38.1145088965784\\
57.875	0.1113	41.4029549588135	41.4029549588135\\
57.875	0.11496	44.9980648218986	44.9980648218986\\
57.875	0.11862	48.8998384858338	48.8998384858338\\
57.875	0.12228	53.1082759506189	53.1082759506189\\
57.875	0.12594	57.6233772162541	57.6233772162541\\
57.875	0.1296	62.4451422827392	62.4451422827392\\
57.875	0.13326	67.5735711500743	67.5735711500743\\
57.875	0.13692	73.0086638182595	73.0086638182595\\
57.875	0.14058	78.7504202872947	78.7504202872947\\
57.875	0.14424	84.7988405571799	84.7988405571799\\
57.875	0.1479	91.153924627915	91.153924627915\\
57.875	0.15156	97.8156724995002	97.8156724995002\\
57.875	0.15522	104.784084171935	104.784084171935\\
57.875	0.15888	112.059159645221	112.059159645221\\
57.875	0.16254	119.640898919356	119.640898919356\\
57.875	0.1662	127.529301994341	127.529301994341\\
57.875	0.16986	135.724368870176	135.724368870176\\
57.875	0.17352	144.226099546862	144.226099546862\\
57.875	0.17718	153.034494024397	153.034494024397\\
57.875	0.18084	162.149552302782	162.149552302782\\
57.875	0.1845	171.571274382017	171.571274382017\\
57.875	0.18816	181.299660262103	181.299660262103\\
57.875	0.19182	191.334709943038	191.334709943038\\
57.875	0.19548	201.676423424823	201.676423424823\\
57.875	0.19914	212.324800707459	212.324800707459\\
57.875	0.2028	223.279841790944	223.279841790944\\
57.875	0.20646	234.541546675279	234.541546675279\\
57.875	0.21012	246.109915360465	246.109915360465\\
57.875	0.21378	257.9849478465	257.9849478465\\
57.875	0.21744	270.166644133385	270.166644133385\\
57.875	0.2211	282.655004221121	282.655004221121\\
57.875	0.22476	295.450028109706	295.450028109706\\
57.875	0.22842	308.551715799141	308.551715799141\\
57.875	0.23208	321.960067289427	321.960067289427\\
57.875	0.23574	335.675082580562	335.675082580562\\
57.875	0.2394	349.696761672548	349.696761672548\\
57.875	0.24306	364.025104565383	364.025104565383\\
57.875	0.24672	378.660111259068	378.660111259068\\
57.875	0.25038	393.601781753604	393.601781753604\\
57.875	0.25404	408.850116048989	408.850116048989\\
57.875	0.2577	424.405114145225	424.405114145225\\
57.875	0.26136	440.26677604231	440.26677604231\\
57.875	0.26502	456.435101740246	456.435101740246\\
57.875	0.26868	472.910091239031	472.910091239031\\
57.875	0.27234	489.691744538667	489.691744538667\\
57.875	0.276	506.780061639152	506.780061639152\\
58.25	0.093	28.3232487718637	28.3232487718637\\
58.25	0.09666	30.3796935029133	30.3796935029133\\
58.25	0.10032	32.7428020348128	32.7428020348128\\
58.25	0.10398	35.4125743675624	35.4125743675624\\
58.25	0.10764	38.389010501162	38.389010501162\\
58.25	0.1113	41.6721104356116	41.6721104356116\\
58.25	0.11496	45.2618741709112	45.2618741709112\\
58.25	0.11862	49.1583017070608	49.1583017070608\\
58.25	0.12228	53.3613930440605	53.3613930440605\\
58.25	0.12594	57.8711481819101	57.8711481819101\\
58.25	0.1296	62.6875671206097	62.6875671206097\\
58.25	0.13326	67.8106498601593	67.8106498601593\\
58.25	0.13692	73.2403964005591	73.2403964005591\\
58.25	0.14058	78.9768067418088	78.9768067418088\\
58.25	0.14424	85.0198808839084	85.0198808839084\\
58.25	0.1479	91.3696188268581	91.3696188268581\\
58.25	0.15156	98.0260205706577	98.0260205706577\\
58.25	0.15522	104.989086115308	104.989086115308\\
58.25	0.15888	112.258815460807	112.258815460807\\
58.25	0.16254	119.835208607157	119.835208607157\\
58.25	0.1662	127.718265554357	127.718265554357\\
58.25	0.16986	135.907986302406	135.907986302406\\
58.25	0.17352	144.404370851306	144.404370851306\\
58.25	0.17718	153.207419201056	153.207419201056\\
58.25	0.18084	162.317131351656	162.317131351656\\
58.25	0.1845	171.733507303105	171.733507303105\\
58.25	0.18816	181.456547055405	181.456547055405\\
58.25	0.19182	191.486250608555	191.486250608555\\
58.25	0.19548	201.822617962555	201.822617962555\\
58.25	0.19914	212.465649117405	212.465649117405\\
58.25	0.2028	223.415344073105	223.415344073105\\
58.25	0.20646	234.671702829654	234.671702829654\\
58.25	0.21012	246.234725387054	246.234725387054\\
58.25	0.21378	258.104411745304	258.104411745304\\
58.25	0.21744	270.280761904404	270.280761904404\\
58.25	0.2211	282.763775864354	282.763775864354\\
58.25	0.22476	295.553453625154	295.553453625154\\
58.25	0.22842	308.649795186803	308.649795186803\\
58.25	0.23208	322.052800549303	322.052800549303\\
58.25	0.23574	335.762469712653	335.762469712653\\
58.25	0.2394	349.778802676853	349.778802676853\\
58.25	0.24306	364.101799441903	364.101799441903\\
58.25	0.24672	378.731460007803	378.731460007803\\
58.25	0.25038	393.667784374553	393.667784374553\\
58.25	0.25404	408.910772542153	408.910772542153\\
58.25	0.2577	424.460424510603	424.460424510603\\
58.25	0.26136	440.316740279903	440.316740279903\\
58.25	0.26502	456.479719850053	456.479719850053\\
58.25	0.26868	472.949363221052	472.949363221052\\
58.25	0.27234	489.725670392903	489.725670392903\\
58.25	0.276	506.808641365602	506.808641365602\\
58.625	0.093	28.6300923183992	28.6300923183992\\
58.625	0.09666	30.6811909216633	30.6811909216633\\
58.625	0.10032	33.0389533257774	33.0389533257774\\
58.625	0.10398	35.7033795307415	35.7033795307415\\
58.625	0.10764	38.6744695365556	38.6744695365556\\
58.625	0.1113	41.9522233432196	41.9522233432196\\
58.625	0.11496	45.5366409507338	45.5366409507338\\
58.625	0.11862	49.4277223590979	49.4277223590979\\
58.625	0.12228	53.625467568312	53.625467568312\\
58.625	0.12594	58.1298765783762	58.1298765783762\\
58.625	0.1296	62.9409493892903	62.9409493892903\\
58.625	0.13326	68.0586860010544	68.0586860010544\\
58.625	0.13692	73.4830864136686	73.4830864136686\\
58.625	0.14058	79.2141506271327	79.2141506271327\\
58.625	0.14424	85.251878641447	85.251878641447\\
58.625	0.1479	91.5962704566111	91.5962704566111\\
58.625	0.15156	98.2473260726252	98.2473260726252\\
58.625	0.15522	105.20504548949	105.20504548949\\
58.625	0.15888	112.469428707204	112.469428707204\\
58.625	0.16254	120.040475725768	120.040475725768\\
58.625	0.1662	127.918186545182	127.918186545182\\
58.625	0.16986	136.102561165446	136.102561165446\\
58.625	0.17352	144.593599586561	144.593599586561\\
58.625	0.17718	153.391301808525	153.391301808525\\
58.625	0.18084	162.495667831339	162.495667831339\\
58.625	0.1845	171.906697655003	171.906697655003\\
58.625	0.18816	181.624391279518	181.624391279518\\
58.625	0.19182	191.648748704882	191.648748704882\\
58.625	0.19548	201.979769931096	201.979769931096\\
58.625	0.19914	212.617454958161	212.617454958161\\
58.625	0.2028	223.561803786075	223.561803786075\\
58.625	0.20646	234.812816414839	234.812816414839\\
58.625	0.21012	246.370492844454	246.370492844454\\
58.625	0.21378	258.234833074918	258.234833074918\\
58.625	0.21744	270.405837106232	270.405837106232\\
58.625	0.2211	282.883504938397	282.883504938397\\
58.625	0.22476	295.667836571411	295.667836571411\\
58.625	0.22842	308.758832005275	308.758832005275\\
58.625	0.23208	322.15649123999	322.15649123999\\
58.625	0.23574	335.860814275554	335.860814275554\\
58.625	0.2394	349.871801111968	349.871801111968\\
58.625	0.24306	364.189451749233	364.189451749233\\
58.625	0.24672	378.813766187347	378.813766187347\\
58.625	0.25038	393.744744426312	393.744744426312\\
58.625	0.25404	408.982386466126	408.982386466126\\
58.625	0.2577	424.526692306791	424.526692306791\\
58.625	0.26136	440.377661948305	440.377661948305\\
58.625	0.26502	456.535295390669	456.535295390669\\
58.625	0.26868	472.999592633884	472.999592633884\\
58.625	0.27234	489.770553677948	489.770553677948\\
58.625	0.276	506.848178522863	506.848178522863\\
59	0.093	28.9478932957447	28.9478932957447\\
59	0.09666	30.9936457712232	30.9936457712232\\
59	0.10032	33.3460620475518	33.3460620475518\\
59	0.10398	36.0051421247304	36.0051421247304\\
59	0.10764	38.970886002759	38.970886002759\\
59	0.1113	42.2432936816376	42.2432936816376\\
59	0.11496	45.8223651613662	45.8223651613662\\
59	0.11862	49.7081004419448	49.7081004419448\\
59	0.12228	53.9004995233734	53.9004995233734\\
59	0.12594	58.399562405652	58.399562405652\\
59	0.1296	63.2052890887807	63.2052890887807\\
59	0.13326	68.3176795727593	68.3176795727593\\
59	0.13692	73.7367338575879	73.7367338575879\\
59	0.14058	79.4624519432666	79.4624519432666\\
59	0.14424	85.4948338297953	85.4948338297953\\
59	0.1479	91.833879517174	91.833879517174\\
59	0.15156	98.4795890054026	98.4795890054026\\
59	0.15522	105.431962294481	105.431962294481\\
59	0.15888	112.69099938441	112.69099938441\\
59	0.16254	120.256700275189	120.256700275189\\
59	0.1662	128.129064966817	128.129064966817\\
59	0.16986	136.308093459296	136.308093459296\\
59	0.17352	144.793785752625	144.793785752625\\
59	0.17718	153.586141846804	153.586141846804\\
59	0.18084	162.685161741832	162.685161741832\\
59	0.1845	172.090845437711	172.090845437711\\
59	0.18816	181.80319293444	181.80319293444\\
59	0.19182	191.822204232019	191.822204232019\\
59	0.19548	202.147879330448	202.147879330448\\
59	0.19914	212.780218229726	212.780218229726\\
59	0.2028	223.719220929855	223.719220929855\\
59	0.20646	234.964887430834	234.964887430834\\
59	0.21012	246.517217732663	246.517217732663\\
59	0.21378	258.376211835342	258.376211835342\\
59	0.21744	270.54186973887	270.54186973887\\
59	0.2211	283.014191443249	283.014191443249\\
59	0.22476	295.793176948478	295.793176948478\\
59	0.22842	308.878826254557	308.878826254557\\
59	0.23208	322.271139361486	322.271139361486\\
59	0.23574	335.970116269265	335.970116269265\\
59	0.2394	349.975756977894	349.975756977894\\
59	0.24306	364.288061487373	364.288061487373\\
59	0.24672	378.907029797701	378.907029797701\\
59	0.25038	393.832661908881	393.832661908881\\
59	0.25404	409.064957820909	409.064957820909\\
59	0.2577	424.603917533788	424.603917533788\\
59	0.26136	440.449541047517	440.449541047517\\
59	0.26502	456.601828362096	456.601828362096\\
59	0.26868	473.060779477525	473.060779477525\\
59	0.27234	489.826394393804	489.826394393804\\
59	0.276	506.898673110933	506.898673110933\\
59.375	0.093	29.2766517039001	29.2766517039001\\
59.375	0.09666	31.3170580515931	31.3170580515931\\
59.375	0.10032	33.6641282001362	33.6641282001362\\
59.375	0.10398	36.3178621495292	36.3178621495292\\
59.375	0.10764	39.2782598997723	39.2782598997723\\
59.375	0.1113	42.5453214508654	42.5453214508654\\
59.375	0.11496	46.1190468028085	46.1190468028085\\
59.375	0.11862	49.9994359556016	49.9994359556016\\
59.375	0.12228	54.1864889092448	54.1864889092448\\
59.375	0.12594	58.6802056637379	58.6802056637379\\
59.375	0.1296	63.480586219081	63.480586219081\\
59.375	0.13326	68.5876305752741	68.5876305752741\\
59.375	0.13692	74.0013387323172	74.0013387323172\\
59.375	0.14058	79.7217106902104	79.7217106902104\\
59.375	0.14424	85.7487464489536	85.7487464489536\\
59.375	0.1479	92.0824460085468	92.0824460085468\\
59.375	0.15156	98.7228093689899	98.7228093689899\\
59.375	0.15522	105.669836530283	105.669836530283\\
59.375	0.15888	112.923527492426	112.923527492426\\
59.375	0.16254	120.48388225542	120.48388225542\\
59.375	0.1662	128.350900819263	128.350900819263\\
59.375	0.16986	136.524583183956	136.524583183956\\
59.375	0.17352	145.004929349499	145.004929349499\\
59.375	0.17718	153.791939315892	153.791939315892\\
59.375	0.18084	162.885613083136	162.885613083136\\
59.375	0.1845	172.285950651229	172.285950651229\\
59.375	0.18816	181.992952020172	181.992952020172\\
59.375	0.19182	192.006617189966	192.006617189966\\
59.375	0.19548	202.326946160609	202.326946160609\\
59.375	0.19914	212.953938932102	212.953938932102\\
59.375	0.2028	223.887595504445	223.887595504445\\
59.375	0.20646	235.127915877639	235.127915877639\\
59.375	0.21012	246.674900051682	246.674900051682\\
59.375	0.21378	258.528548026575	258.528548026575\\
59.375	0.21744	270.688859802319	270.688859802319\\
59.375	0.2211	283.155835378912	283.155835378912\\
59.375	0.22476	295.929474756355	295.929474756355\\
59.375	0.22842	309.009777934649	309.009777934649\\
59.375	0.23208	322.396744913792	322.396744913792\\
59.375	0.23574	336.090375693786	336.090375693786\\
59.375	0.2394	350.090670274629	350.090670274629\\
59.375	0.24306	364.397628656322	364.397628656322\\
59.375	0.24672	379.011250838866	379.011250838866\\
59.375	0.25038	393.931536822259	393.931536822259\\
59.375	0.25404	409.158486606502	409.158486606502\\
59.375	0.2577	424.692100191596	424.692100191596\\
59.375	0.26136	440.532377577539	440.532377577539\\
59.375	0.26502	456.679318764333	456.679318764333\\
59.375	0.26868	473.132923751976	473.132923751976\\
59.375	0.27234	489.89319254047	489.89319254047\\
59.375	0.276	506.960125129813	506.960125129813\\
59.75	0.093	29.6163675428654	29.6163675428654\\
59.75	0.09666	31.6514277627729	31.6514277627729\\
59.75	0.10032	33.9931517835305	33.9931517835305\\
59.75	0.10398	36.641539605138	36.641539605138\\
59.75	0.10764	39.5965912275956	39.5965912275956\\
59.75	0.1113	42.8583066509032	42.8583066509032\\
59.75	0.11496	46.4266858750608	46.4266858750608\\
59.75	0.11862	50.3017289000684	50.3017289000684\\
59.75	0.12228	54.483435725926	54.483435725926\\
59.75	0.12594	58.9718063526336	58.9718063526336\\
59.75	0.1296	63.7668407801912	63.7668407801912\\
59.75	0.13326	68.8685390085988	68.8685390085988\\
59.75	0.13692	74.2769010378565	74.2769010378565\\
59.75	0.14058	79.9919268679642	79.9919268679642\\
59.75	0.14424	86.0136164989219	86.0136164989219\\
59.75	0.1479	92.3419699307295	92.3419699307295\\
59.75	0.15156	98.9769871633872	98.9769871633872\\
59.75	0.15522	105.918668196895	105.918668196895\\
59.75	0.15888	113.167013031253	113.167013031253\\
59.75	0.16254	120.72202166646	120.72202166646\\
59.75	0.1662	128.583694102518	128.583694102518\\
59.75	0.16986	136.752030339426	136.752030339426\\
59.75	0.17352	145.227030377183	145.227030377183\\
59.75	0.17718	154.008694215791	154.008694215791\\
59.75	0.18084	163.097021855249	163.097021855249\\
59.75	0.1845	172.492013295557	172.492013295557\\
59.75	0.18816	182.193668536714	182.193668536714\\
59.75	0.19182	192.201987578722	192.201987578722\\
59.75	0.19548	202.51697042158	202.51697042158\\
59.75	0.19914	213.138617065288	213.138617065288\\
59.75	0.2028	224.066927509846	224.066927509846\\
59.75	0.20646	235.301901755254	235.301901755254\\
59.75	0.21012	246.843539801511	246.843539801511\\
59.75	0.21378	258.691841648619	258.691841648619\\
59.75	0.21744	270.846807296577	270.846807296577\\
59.75	0.2211	283.308436745385	283.308436745385\\
59.75	0.22476	296.076729995043	296.076729995043\\
59.75	0.22842	309.15168704555	309.15168704555\\
59.75	0.23208	322.533307896908	322.533307896908\\
59.75	0.23574	336.221592549116	336.221592549116\\
59.75	0.2394	350.216541002174	350.216541002174\\
59.75	0.24306	364.518153256082	364.518153256082\\
59.75	0.24672	379.12642931084	379.12642931084\\
59.75	0.25038	394.041369166448	394.041369166448\\
59.75	0.25404	409.262972822906	409.262972822906\\
59.75	0.2577	424.791240280214	424.791240280214\\
59.75	0.26136	440.626171538372	440.626171538372\\
59.75	0.26502	456.767766597379	456.767766597379\\
59.75	0.26868	473.216025457237	473.216025457237\\
59.75	0.27234	489.970948117945	489.970948117945\\
59.75	0.276	507.032534579503	507.032534579503\\
60.125	0.093	29.9670408126406	29.9670408126406\\
60.125	0.09666	31.9967549047626	31.9967549047626\\
60.125	0.10032	34.3331327977347	34.3331327977347\\
60.125	0.10398	36.9761744915568	36.9761744915568\\
60.125	0.10764	39.9258799862288	39.9258799862288\\
60.125	0.1113	43.1822492817509	43.1822492817509\\
60.125	0.11496	46.745282378123	46.745282378123\\
60.125	0.11862	50.6149792753451	50.6149792753451\\
60.125	0.12228	54.7913399734172	54.7913399734172\\
60.125	0.12594	59.2743644723393	59.2743644723393\\
60.125	0.1296	64.0640527721114	64.0640527721114\\
60.125	0.13326	69.1604048727335	69.1604048727335\\
60.125	0.13692	74.5634207742057	74.5634207742057\\
60.125	0.14058	80.2731004765278	80.2731004765278\\
60.125	0.14424	86.2894439797	86.2894439797\\
60.125	0.1479	92.6124512837222	92.6124512837222\\
60.125	0.15156	99.2421223885943	99.2421223885943\\
60.125	0.15522	106.178457294317	106.178457294317\\
60.125	0.15888	113.421456000889	113.421456000889\\
60.125	0.16254	120.971118508311	120.971118508311\\
60.125	0.1662	128.827444816583	128.827444816583\\
60.125	0.16986	136.990434925705	136.990434925705\\
60.125	0.17352	145.460088835678	145.460088835678\\
60.125	0.17718	154.2364065465	154.2364065465\\
60.125	0.18084	163.319388058172	163.319388058172\\
60.125	0.1845	172.709033370694	172.709033370694\\
60.125	0.18816	182.405342484067	182.405342484067\\
60.125	0.19182	192.408315398289	192.408315398289\\
60.125	0.19548	202.717952113361	202.717952113361\\
60.125	0.19914	213.334252629283	213.334252629283\\
60.125	0.2028	224.257216946056	224.257216946056\\
60.125	0.20646	235.486845063678	235.486845063678\\
60.125	0.21012	247.02313698215	247.02313698215\\
60.125	0.21378	258.866092701473	258.866092701473\\
60.125	0.21744	271.015712221645	271.015712221645\\
60.125	0.2211	283.471995542667	283.471995542667\\
60.125	0.22476	296.23494266454	296.23494266454\\
60.125	0.22842	309.304553587262	309.304553587262\\
60.125	0.23208	322.680828310834	322.680828310834\\
60.125	0.23574	336.363766835257	336.363766835257\\
60.125	0.2394	350.353369160529	350.353369160529\\
60.125	0.24306	364.649635286651	364.649635286651\\
60.125	0.24672	379.252565213624	379.252565213624\\
60.125	0.25038	394.162158941446	394.162158941446\\
60.125	0.25404	409.378416470119	409.378416470119\\
60.125	0.2577	424.901337799641	424.901337799641\\
60.125	0.26136	440.730922930014	440.730922930014\\
60.125	0.26502	456.867171861236	456.867171861236\\
60.125	0.26868	473.310084593308	473.310084593308\\
60.125	0.27234	490.059661126231	490.059661126231\\
60.125	0.276	507.115901460003	507.115901460003\\
60.5	0.093	30.3286715132257	30.3286715132257\\
60.5	0.09666	32.3530394775622	32.3530394775622\\
60.5	0.10032	34.6840712427488	34.6840712427488\\
60.5	0.10398	37.3217668087853	37.3217668087853\\
60.5	0.10764	40.2661261756719	40.2661261756719\\
60.5	0.1113	43.5171493434085	43.5171493434085\\
60.5	0.11496	47.0748363119951	47.0748363119951\\
60.5	0.11862	50.9391870814317	50.9391870814317\\
60.5	0.12228	55.1102016517183	55.1102016517183\\
60.5	0.12594	59.5878800228549	59.5878800228549\\
60.5	0.1296	64.3722221948415	64.3722221948415\\
60.5	0.13326	69.4632281676781	69.4632281676781\\
60.5	0.13692	74.8608979413647	74.8608979413647\\
60.5	0.14058	80.5652315159014	80.5652315159014\\
60.5	0.14424	86.5762288912881	86.5762288912881\\
60.5	0.1479	92.8938900675247	92.8938900675247\\
60.5	0.15156	99.5182150446113	99.5182150446113\\
60.5	0.15522	106.449203822548	106.449203822548\\
60.5	0.15888	113.686856401335	113.686856401335\\
60.5	0.16254	121.231172780971	121.231172780971\\
60.5	0.1662	129.082152961458	129.082152961458\\
60.5	0.16986	137.239796942795	137.239796942795\\
60.5	0.17352	145.704104724982	145.704104724982\\
60.5	0.17718	154.475076308018	154.475076308018\\
60.5	0.18084	163.552711691905	163.552711691905\\
60.5	0.1845	172.937010876642	172.937010876642\\
60.5	0.18816	182.627973862229	182.627973862229\\
60.5	0.19182	192.625600648665	192.625600648665\\
60.5	0.19548	202.929891235952	202.929891235952\\
60.5	0.19914	213.540845624089	213.540845624089\\
60.5	0.2028	224.458463813076	224.458463813076\\
60.5	0.20646	235.682745802913	235.682745802913\\
60.5	0.21012	247.213691593599	247.213691593599\\
60.5	0.21378	259.051301185136	259.051301185136\\
60.5	0.21744	271.195574577523	271.195574577523\\
60.5	0.2211	283.64651177076	283.64651177076\\
60.5	0.22476	296.404112764847	296.404112764847\\
60.5	0.22842	309.468377559783	309.468377559783\\
60.5	0.23208	322.83930615557	322.83930615557\\
60.5	0.23574	336.516898552207	336.516898552207\\
60.5	0.2394	350.501154749694	350.501154749694\\
60.5	0.24306	364.792074748031	364.792074748031\\
60.5	0.24672	379.389658547218	379.389658547218\\
60.5	0.25038	394.293906147255	394.293906147255\\
60.5	0.25404	409.504817548141	409.504817548141\\
60.5	0.2577	425.022392749878	425.022392749878\\
60.5	0.26136	440.846631752465	440.846631752465\\
60.5	0.26502	456.977534555902	456.977534555902\\
60.5	0.26868	473.415101160189	473.415101160189\\
60.5	0.27234	490.159331565326	490.159331565326\\
60.5	0.276	507.210225771313	507.210225771313\\
60.875	0.093	30.7012596446208	30.7012596446208\\
60.875	0.09666	32.7202814811718	32.7202814811718\\
60.875	0.10032	35.0459671185728	35.0459671185728\\
60.875	0.10398	37.6783165568239	37.6783165568239\\
60.875	0.10764	40.617329795925	40.617329795925\\
60.875	0.1113	43.863006835876	43.863006835876\\
60.875	0.11496	47.4153476766771	47.4153476766771\\
60.875	0.11862	51.2743523183282	51.2743523183282\\
60.875	0.12228	55.4400207608293	55.4400207608293\\
60.875	0.12594	59.9123530041804	59.9123530041804\\
60.875	0.1296	64.6913490483815	64.6913490483815\\
60.875	0.13326	69.7770088934326	69.7770088934326\\
60.875	0.13692	75.1693325393337	75.1693325393337\\
60.875	0.14058	80.8683199860849	80.8683199860849\\
60.875	0.14424	86.8739712336861	86.8739712336861\\
60.875	0.1479	93.1862862821372	93.1862862821372\\
60.875	0.15156	99.8052651314383	99.8052651314383\\
60.875	0.15522	106.73090778159	106.73090778159\\
60.875	0.15888	113.963214232591	113.963214232591\\
60.875	0.16254	121.502184484442	121.502184484442\\
60.875	0.1662	129.347818537143	129.347818537143\\
60.875	0.16986	137.500116390694	137.500116390694\\
60.875	0.17352	145.959078045096	145.959078045096\\
60.875	0.17718	154.724703500347	154.724703500347\\
60.875	0.18084	163.796992756448	163.796992756448\\
60.875	0.1845	173.175945813399	173.175945813399\\
60.875	0.18816	182.8615626712	182.8615626712\\
60.875	0.19182	192.853843329852	192.853843329852\\
60.875	0.19548	203.152787789353	203.152787789353\\
60.875	0.19914	213.758396049704	213.758396049704\\
60.875	0.2028	224.670668110906	224.670668110906\\
60.875	0.20646	235.889603972957	235.889603972957\\
60.875	0.21012	247.415203635858	247.415203635858\\
60.875	0.21378	259.247467099609	259.247467099609\\
60.875	0.21744	271.386394364211	271.386394364211\\
60.875	0.2211	283.831985429662	283.831985429662\\
60.875	0.22476	296.584240295963	296.584240295963\\
60.875	0.22842	309.643158963115	309.643158963115\\
60.875	0.23208	323.008741431116	323.008741431116\\
60.875	0.23574	336.680987699968	336.680987699968\\
60.875	0.2394	350.659897769669	350.659897769669\\
60.875	0.24306	364.94547164022	364.94547164022\\
60.875	0.24672	379.537709311622	379.537709311622\\
60.875	0.25038	394.436610783873	394.436610783873\\
60.875	0.25404	409.642176056974	409.642176056974\\
60.875	0.2577	425.154405130926	425.154405130926\\
60.875	0.26136	440.973298005727	440.973298005727\\
60.875	0.26502	457.098854681379	457.098854681379\\
60.875	0.26868	473.53107515788	473.53107515788\\
60.875	0.27234	490.269959435231	490.269959435231\\
60.875	0.276	507.315507513433	507.315507513433\\
61.25	0.093	31.0848052068258	31.0848052068258\\
61.25	0.09666	33.0984809155913	33.0984809155913\\
61.25	0.10032	35.4188204252068	35.4188204252068\\
61.25	0.10398	38.0458237356724	38.0458237356724\\
61.25	0.10764	40.9794908469879	40.9794908469879\\
61.25	0.1113	44.2198217591535	44.2198217591535\\
61.25	0.11496	47.7668164721691	47.7668164721691\\
61.25	0.11862	51.6204749860347	51.6204749860347\\
61.25	0.12228	55.7807973007503	55.7807973007503\\
61.25	0.12594	60.2477834163158	60.2477834163158\\
61.25	0.1296	65.0214333327314	65.0214333327314\\
61.25	0.13326	70.101747049997	70.101747049997\\
61.25	0.13692	75.4887245681126	75.4887245681126\\
61.25	0.14058	81.1823658870783	81.1823658870783\\
61.25	0.14424	87.182671006894	87.182671006894\\
61.25	0.1479	93.4896399275596	93.4896399275596\\
61.25	0.15156	100.103272649075	100.103272649075\\
61.25	0.15522	107.023569171441	107.023569171441\\
61.25	0.15888	114.250529494657	114.250529494657\\
61.25	0.16254	121.784153618722	121.784153618722\\
61.25	0.1662	129.624441543638	129.624441543638\\
61.25	0.16986	137.771393269404	137.771393269404\\
61.25	0.17352	146.225008796019	146.225008796019\\
61.25	0.17718	154.985288123485	154.985288123485\\
61.25	0.18084	164.052231251801	164.052231251801\\
61.25	0.1845	173.425838180967	173.425838180967\\
61.25	0.18816	183.106108910982	183.106108910982\\
61.25	0.19182	193.093043441848	193.093043441848\\
61.25	0.19548	203.386641773564	203.386641773564\\
61.25	0.19914	213.98690390613	213.98690390613\\
61.25	0.2028	224.893829839545	224.893829839545\\
61.25	0.20646	236.107419573811	236.107419573811\\
61.25	0.21012	247.627673108927	247.627673108927\\
61.25	0.21378	259.454590444893	259.454590444893\\
61.25	0.21744	271.588171581708	271.588171581708\\
61.25	0.2211	284.028416519374	284.028416519374\\
61.25	0.22476	296.77532525789	296.77532525789\\
61.25	0.22842	309.828897797256	309.828897797256\\
61.25	0.23208	323.189134137472	323.189134137472\\
61.25	0.23574	336.856034278538	336.856034278538\\
61.25	0.2394	350.829598220454	350.829598220454\\
61.25	0.24306	365.109825963219	365.109825963219\\
61.25	0.24672	379.696717506835	379.696717506835\\
61.25	0.25038	394.590272851301	394.590272851301\\
61.25	0.25404	409.790491996617	409.790491996617\\
61.25	0.2577	425.297374942783	425.297374942783\\
61.25	0.26136	441.110921689799	441.110921689799\\
61.25	0.26502	457.231132237665	457.231132237665\\
61.25	0.26868	473.658006586381	473.658006586381\\
61.25	0.27234	490.391544735947	490.391544735947\\
61.25	0.276	507.431746686363	507.431746686363\\
61.625	0.093	31.4793081998407	31.4793081998407\\
61.625	0.09666	33.4876377808207	33.4876377808207\\
61.625	0.10032	35.8026311626507	35.8026311626507\\
61.625	0.10398	38.4242883453308	38.4242883453308\\
61.625	0.10764	41.3526093288608	41.3526093288608\\
61.625	0.1113	44.5875941132409	44.5875941132409\\
61.625	0.11496	48.129242698471	48.129242698471\\
61.625	0.11862	51.977555084551	51.977555084551\\
61.625	0.12228	56.1325312714811	56.1325312714811\\
61.625	0.12594	60.5941712592612	60.5941712592612\\
61.625	0.1296	65.3624750478913	65.3624750478913\\
61.625	0.13326	70.4374426373714	70.4374426373714\\
61.625	0.13692	75.8190740277015	75.8190740277015\\
61.625	0.14058	81.5073692188817	81.5073692188817\\
61.625	0.14424	87.5023282109118	87.5023282109118\\
61.625	0.1479	93.8039510037919	93.8039510037919\\
61.625	0.15156	100.412237597522	100.412237597522\\
61.625	0.15522	107.327187992102	107.327187992102\\
61.625	0.15888	114.548802187532	114.548802187532\\
61.625	0.16254	122.077080183813	122.077080183813\\
61.625	0.1662	129.912021980943	129.912021980943\\
61.625	0.16986	138.053627578923	138.053627578923\\
61.625	0.17352	146.501896977753	146.501896977753\\
61.625	0.17718	155.256830177433	155.256830177433\\
61.625	0.18084	164.318427177964	164.318427177964\\
61.625	0.1845	173.686687979344	173.686687979344\\
61.625	0.18816	183.361612581574	183.361612581574\\
61.625	0.19182	193.343200984654	193.343200984654\\
61.625	0.19548	203.631453188585	203.631453188585\\
61.625	0.19914	214.226369193365	214.226369193365\\
61.625	0.2028	225.127948998995	225.127948998995\\
61.625	0.20646	236.336192605476	236.336192605476\\
61.625	0.21012	247.851100012806	247.851100012806\\
61.625	0.21378	259.672671220986	259.672671220986\\
61.625	0.21744	271.800906230016	271.800906230016\\
61.625	0.2211	284.235805039897	284.235805039897\\
61.625	0.22476	296.977367650627	296.977367650627\\
61.625	0.22842	310.025594062207	310.025594062207\\
61.625	0.23208	323.380484274638	323.380484274638\\
61.625	0.23574	337.042038287918	337.042038287918\\
61.625	0.2394	351.010256102048	351.010256102048\\
61.625	0.24306	365.285137717029	365.285137717029\\
61.625	0.24672	379.866683132859	379.866683132859\\
61.625	0.25038	394.75489234954	394.75489234954\\
61.625	0.25404	409.94976536707	409.94976536707\\
61.625	0.2577	425.45130218545	425.45130218545\\
61.625	0.26136	441.259502804681	441.259502804681\\
61.625	0.26502	457.374367224761	457.374367224761\\
61.625	0.26868	473.795895445691	473.795895445691\\
61.625	0.27234	490.524087467472	490.524087467472\\
61.625	0.276	507.558943290102	507.558943290102\\
62	0.093	31.8847686236655	31.8847686236655\\
62	0.09666	33.88775207686	33.88775207686\\
62	0.10032	36.1973993309045	36.1973993309045\\
62	0.10398	38.813710385799	38.813710385799\\
62	0.10764	41.7366852415436	41.7366852415436\\
62	0.1113	44.9663238981381	44.9663238981381\\
62	0.11496	48.5026263555827	48.5026263555827\\
62	0.11862	52.3455926138772	52.3455926138772\\
62	0.12228	56.4952226730219	56.4952226730219\\
62	0.12594	60.9515165330165	60.9515165330165\\
62	0.1296	65.7144741938611	65.7144741938611\\
62	0.13326	70.7840956555556	70.7840956555556\\
62	0.13692	76.1603809181002	76.1603809181002\\
62	0.14058	81.8433299814949	81.8433299814949\\
62	0.14424	87.8329428457396	87.8329428457396\\
62	0.1479	94.1292195108342	94.1292195108342\\
62	0.15156	100.732159976779	100.732159976779\\
62	0.15522	107.641764243574	107.641764243574\\
62	0.15888	114.858032311218	114.858032311218\\
62	0.16254	122.380964179713	122.380964179713\\
62	0.1662	130.210559849057	130.210559849057\\
62	0.16986	138.346819319252	138.346819319252\\
62	0.17352	146.789742590297	146.789742590297\\
62	0.17718	155.539329662192	155.539329662192\\
62	0.18084	164.595580534936	164.595580534936\\
62	0.1845	173.958495208531	173.958495208531\\
62	0.18816	183.628073682976	183.628073682976\\
62	0.19182	193.60431595827	193.60431595827\\
62	0.19548	203.887222034415	203.887222034415\\
62	0.19914	214.47679191141	214.47679191141\\
62	0.2028	225.373025589255	225.373025589255\\
62	0.20646	236.57592306795	236.57592306795\\
62	0.21012	248.085484347494	248.085484347494\\
62	0.21378	259.901709427889	259.901709427889\\
62	0.21744	272.024598309134	272.024598309134\\
62	0.2211	284.454150991229	284.454150991229\\
62	0.22476	297.190367474174	297.190367474174\\
62	0.22842	310.233247757968	310.233247757968\\
62	0.23208	323.582791842613	323.582791842613\\
62	0.23574	337.238999728108	337.238999728108\\
62	0.2394	351.201871414453	351.201871414453\\
62	0.24306	365.471406901648	365.471406901648\\
62	0.24672	380.047606189693	380.047606189693\\
62	0.25038	394.930469278588	394.930469278588\\
62	0.25404	410.119996168332	410.119996168332\\
62	0.2577	425.616186858927	425.616186858927\\
62	0.26136	441.419041350372	441.419041350372\\
62	0.26502	457.528559642667	457.528559642667\\
62	0.26868	473.944741735812	473.944741735812\\
62	0.27234	490.667587629807	490.667587629807\\
62	0.276	507.697097324652	507.697097324652\\
62.375	0.093	32.3011864783002	32.3011864783002\\
62.375	0.09666	34.2988238037092	34.2988238037092\\
62.375	0.10032	36.6031249299683	36.6031249299683\\
62.375	0.10398	39.2140898570773	39.2140898570773\\
62.375	0.10764	42.1317185850363	42.1317185850363\\
62.375	0.1113	45.3560111138454	45.3560111138454\\
62.375	0.11496	48.8869674435044	48.8869674435044\\
62.375	0.11862	52.7245875740135	52.7245875740135\\
62.375	0.12228	56.8688715053726	56.8688715053726\\
62.375	0.12594	61.3198192375817	61.3198192375817\\
62.375	0.1296	66.0774307706407	66.0774307706407\\
62.375	0.13326	71.1417061045498	71.1417061045498\\
62.375	0.13692	76.512645239309	76.512645239309\\
62.375	0.14058	82.1902481749181	82.1902481749181\\
62.375	0.14424	88.1745149113772	88.1745149113772\\
62.375	0.1479	94.4654454486864	94.4654454486864\\
62.375	0.15156	101.063039786845	101.063039786845\\
62.375	0.15522	107.967297925855	107.967297925855\\
62.375	0.15888	115.178219865714	115.178219865714\\
62.375	0.16254	122.695805606423	122.695805606423\\
62.375	0.1662	130.520055147982	130.520055147982\\
62.375	0.16986	138.650968490391	138.650968490391\\
62.375	0.17352	147.088545633651	147.088545633651\\
62.375	0.17718	155.83278657776	155.83278657776\\
62.375	0.18084	164.883691322719	164.883691322719\\
62.375	0.1845	174.241259868528	174.241259868528\\
62.375	0.18816	183.905492215187	183.905492215187\\
62.375	0.19182	193.876388362697	193.876388362697\\
62.375	0.19548	204.153948311056	204.153948311056\\
62.375	0.19914	214.738172060265	214.738172060265\\
62.375	0.2028	225.629059610324	225.629059610324\\
62.375	0.20646	236.826610961234	236.826610961234\\
62.375	0.21012	248.330826112993	248.330826112993\\
62.375	0.21378	260.141705065602	260.141705065602\\
62.375	0.21744	272.259247819061	272.259247819061\\
62.375	0.2211	284.683454373371	284.683454373371\\
62.375	0.22476	297.41432472853	297.41432472853\\
62.375	0.22842	310.451858884539	310.451858884539\\
62.375	0.23208	323.796056841399	323.796056841399\\
62.375	0.23574	337.446918599108	337.446918599108\\
62.375	0.2394	351.404444157668	351.404444157668\\
62.375	0.24306	365.668633517077	365.668633517077\\
62.375	0.24672	380.239486677336	380.239486677336\\
62.375	0.25038	395.117003638446	395.117003638446\\
62.375	0.25404	410.301184400405	410.301184400405\\
62.375	0.2577	425.792028963214	425.792028963214\\
62.375	0.26136	441.589537326874	441.589537326874\\
62.375	0.26502	457.693709491383	457.693709491383\\
62.375	0.26868	474.104545456742	474.104545456742\\
62.375	0.27234	490.822045222952	490.822045222952\\
62.375	0.276	507.846208790011	507.846208790011\\
62.75	0.093	32.7285617637449	32.7285617637449\\
62.75	0.09666	34.7208529613684	34.7208529613684\\
62.75	0.10032	37.0198079598419	37.0198079598419\\
62.75	0.10398	39.6254267591654	39.6254267591654\\
62.75	0.10764	42.537709359339	42.537709359339\\
62.75	0.1113	45.7566557603625	45.7566557603625\\
62.75	0.11496	49.2822659622361	49.2822659622361\\
62.75	0.11862	53.1145399649596	53.1145399649596\\
62.75	0.12228	57.2534777685332	57.2534777685332\\
62.75	0.12594	61.6990793729568	61.6990793729568\\
62.75	0.1296	66.4513447782304	66.4513447782304\\
62.75	0.13326	71.5102739843539	71.5102739843539\\
62.75	0.13692	76.8758669913275	76.8758669913275\\
62.75	0.14058	82.5481237991511	82.5481237991511\\
62.75	0.14424	88.5270444078248	88.5270444078248\\
62.75	0.1479	94.8126288173484	94.8126288173484\\
62.75	0.15156	101.404877027722	101.404877027722\\
62.75	0.15522	108.303789038946	108.303789038946\\
62.75	0.15888	115.509364851019	115.509364851019\\
62.75	0.16254	123.021604463943	123.021604463943\\
62.75	0.1662	130.840507877717	130.840507877717\\
62.75	0.16986	138.96607509234	138.96607509234\\
62.75	0.17352	147.398306107814	147.398306107814\\
62.75	0.17718	156.137200924138	156.137200924138\\
62.75	0.18084	165.182759541311	165.182759541311\\
62.75	0.1845	174.534981959335	174.534981959335\\
62.75	0.18816	184.193868178209	184.193868178209\\
62.75	0.19182	194.159418197933	194.159418197933\\
62.75	0.19548	204.431632018506	204.431632018506\\
62.75	0.19914	215.01050963993	215.01050963993\\
62.75	0.2028	225.896051062204	225.896051062204\\
62.75	0.20646	237.088256285328	237.088256285328\\
62.75	0.21012	248.587125309301	248.587125309301\\
62.75	0.21378	260.392658134125	260.392658134125\\
62.75	0.21744	272.504854759799	272.504854759799\\
62.75	0.2211	284.923715186323	284.923715186323\\
62.75	0.22476	297.649239413697	297.649239413697\\
62.75	0.22842	310.68142744192	310.68142744192\\
62.75	0.23208	324.020279270994	324.020279270994\\
62.75	0.23574	337.665794900918	337.665794900918\\
62.75	0.2394	351.617974331692	351.617974331692\\
62.75	0.24306	365.876817563316	365.876817563316\\
62.75	0.24672	380.44232459579	380.44232459579\\
62.75	0.25038	395.314495429114	395.314495429114\\
62.75	0.25404	410.493330063287	410.493330063287\\
62.75	0.2577	425.978828498311	425.978828498311\\
62.75	0.26136	441.770990734185	441.770990734185\\
62.75	0.26502	457.869816770909	457.869816770909\\
62.75	0.26868	474.275306608483	474.275306608483\\
62.75	0.27234	490.987460246907	490.987460246907\\
62.75	0.276	508.006277686181	508.006277686181\\
63.125	0.093	33.1668944799994	33.1668944799994\\
63.125	0.09666	35.1538395498374	35.1538395498374\\
63.125	0.10032	37.4474484205254	37.4474484205254\\
63.125	0.10398	40.0477210920635	40.0477210920635\\
63.125	0.10764	42.9546575644515	42.9546575644515\\
63.125	0.1113	46.1682578376895	46.1682578376895\\
63.125	0.11496	49.6885219117776	49.6885219117776\\
63.125	0.11862	53.5154497867156	53.5154497867156\\
63.125	0.12228	57.6490414625037	57.6490414625037\\
63.125	0.12594	62.0892969391418	62.0892969391418\\
63.125	0.1296	66.8362162166299	66.8362162166299\\
63.125	0.13326	71.8897992949679	71.8897992949679\\
63.125	0.13692	77.250046174156	77.250046174156\\
63.125	0.14058	82.9169568541942	82.9169568541942\\
63.125	0.14424	88.8905313350823	88.8905313350823\\
63.125	0.1479	95.1707696168205	95.1707696168205\\
63.125	0.15156	101.757671699409	101.757671699409\\
63.125	0.15522	108.651237582847	108.651237582847\\
63.125	0.15888	115.851467267135	115.851467267135\\
63.125	0.16254	123.358360752273	123.358360752273\\
63.125	0.1662	131.171918038261	131.171918038261\\
63.125	0.16986	139.292139125099	139.292139125099\\
63.125	0.17352	147.719024012788	147.719024012788\\
63.125	0.17718	156.452572701326	156.452572701326\\
63.125	0.18084	165.492785190714	165.492785190714\\
63.125	0.1845	174.839661480952	174.839661480952\\
63.125	0.18816	184.49320157204	184.49320157204\\
63.125	0.19182	194.453405463979	194.453405463979\\
63.125	0.19548	204.720273156767	204.720273156767\\
63.125	0.19914	215.293804650405	215.293804650405\\
63.125	0.2028	226.173999944893	226.173999944893\\
63.125	0.20646	237.360859040232	237.360859040232\\
63.125	0.21012	248.85438193642	248.85438193642\\
63.125	0.21378	260.654568633458	260.654568633458\\
63.125	0.21744	272.761419131346	272.761419131346\\
63.125	0.2211	285.174933430085	285.174933430085\\
63.125	0.22476	297.895111529673	297.895111529673\\
63.125	0.22842	310.921953430111	310.921953430111\\
63.125	0.23208	324.2554591314	324.2554591314\\
63.125	0.23574	337.895628633538	337.895628633538\\
63.125	0.2394	351.842461936526	351.842461936526\\
63.125	0.24306	366.095959040365	366.095959040365\\
63.125	0.24672	380.656119945053	380.656119945053\\
63.125	0.25038	395.522944650591	395.522944650591\\
63.125	0.25404	410.69643315698	410.69643315698\\
63.125	0.2577	426.176585464218	426.176585464218\\
63.125	0.26136	441.963401572307	441.963401572307\\
63.125	0.26502	458.056881481245	458.056881481245\\
63.125	0.26868	474.457025191033	474.457025191033\\
63.125	0.27234	491.163832701672	491.163832701672\\
63.125	0.276	508.17730401316	508.17730401316\\
63.5	0.093	33.6161846270639	33.6161846270639\\
63.5	0.09666	35.5977835691164	35.5977835691164\\
63.5	0.10032	37.8860463120189	37.8860463120189\\
63.5	0.10398	40.4809728557715	40.4809728557715\\
63.5	0.10764	43.382563200374	43.382563200374\\
63.5	0.1113	46.5908173458265	46.5908173458265\\
63.5	0.11496	50.1057352921291	50.1057352921291\\
63.5	0.11862	53.9273170392816	53.9273170392816\\
63.5	0.12228	58.0555625872842	58.0555625872842\\
63.5	0.12594	62.4904719361368	62.4904719361368\\
63.5	0.1296	67.2320450858393	67.2320450858393\\
63.5	0.13326	72.2802820363919	72.2802820363919\\
63.5	0.13692	77.6351827877945	77.6351827877945\\
63.5	0.14058	83.2967473400471	83.2967473400471\\
63.5	0.14424	89.2649756931497	89.2649756931497\\
63.5	0.1479	95.5398678471024	95.5398678471024\\
63.5	0.15156	102.121423801905	102.121423801905\\
63.5	0.15522	109.009643557558	109.009643557558\\
63.5	0.15888	116.20452711406	116.20452711406\\
63.5	0.16254	123.706074471413	123.706074471413\\
63.5	0.1662	131.514285629616	131.514285629616\\
63.5	0.16986	139.629160588668	139.629160588668\\
63.5	0.17352	148.050699348571	148.050699348571\\
63.5	0.17718	156.778901909324	156.778901909324\\
63.5	0.18084	165.813768270926	165.813768270926\\
63.5	0.1845	175.155298433379	175.155298433379\\
63.5	0.18816	184.803492396682	184.803492396682\\
63.5	0.19182	194.758350160834	194.758350160834\\
63.5	0.19548	205.019871725837	205.019871725837\\
63.5	0.19914	215.58805709169	215.58805709169\\
63.5	0.2028	226.462906258393	226.462906258393\\
63.5	0.20646	237.644419225946	237.644419225946\\
63.5	0.21012	249.132595994348	249.132595994348\\
63.5	0.21378	260.927436563601	260.927436563601\\
63.5	0.21744	273.028940933704	273.028940933704\\
63.5	0.2211	285.437109104656	285.437109104656\\
63.5	0.22476	298.151941076459	298.151941076459\\
63.5	0.22842	311.173436849112	311.173436849112\\
63.5	0.23208	324.501596422615	324.501596422615\\
63.5	0.23574	338.136419796968	338.136419796968\\
63.5	0.2394	352.077906972171	352.077906972171\\
63.5	0.24306	366.326057948223	366.326057948223\\
63.5	0.24672	380.880872725126	380.880872725126\\
63.5	0.25038	395.742351302879	395.742351302879\\
63.5	0.25404	410.910493681482	410.910493681482\\
63.5	0.2577	426.385299860935	426.385299860935\\
63.5	0.26136	442.166769841238	442.166769841238\\
63.5	0.26502	458.254903622391	458.254903622391\\
63.5	0.26868	474.649701204394	474.649701204394\\
63.5	0.27234	491.351162587246	491.351162587246\\
63.5	0.276	508.359287770949	508.359287770949\\
63.875	0.093	34.0764322049384	34.0764322049384\\
63.875	0.09666	36.0526850192054	36.0526850192054\\
63.875	0.10032	38.3356016343224	38.3356016343224\\
63.875	0.10398	40.9251820502894	40.9251820502894\\
63.875	0.10764	43.8214262671064	43.8214262671064\\
63.875	0.1113	47.0243342847734	47.0243342847734\\
63.875	0.11496	50.5339061032905	50.5339061032905\\
63.875	0.11862	54.3501417226575	54.3501417226575\\
63.875	0.12228	58.4730411428746	58.4730411428746\\
63.875	0.12594	62.9026043639416	62.9026043639416\\
63.875	0.1296	67.6388313858587	67.6388313858587\\
63.875	0.13326	72.6817222086258	72.6817222086258\\
63.875	0.13692	78.0312768322428	78.0312768322428\\
63.875	0.14058	83.68749525671	83.68749525671\\
63.875	0.14424	89.6503774820271	89.6503774820271\\
63.875	0.1479	95.9199235081942	95.9199235081942\\
63.875	0.15156	102.496133335211	102.496133335211\\
63.875	0.15522	109.379006963078	109.379006963078\\
63.875	0.15888	116.568544391796	116.568544391796\\
63.875	0.16254	124.064745621363	124.064745621363\\
63.875	0.1662	131.86761065178	131.86761065178\\
63.875	0.16986	139.977139483047	139.977139483047\\
63.875	0.17352	148.393332115164	148.393332115164\\
63.875	0.17718	157.116188548131	157.116188548131\\
63.875	0.18084	166.145708781949	166.145708781949\\
63.875	0.1845	175.481892816616	175.481892816616\\
63.875	0.18816	185.124740652133	185.124740652133\\
63.875	0.19182	195.0742522885	195.0742522885\\
63.875	0.19548	205.330427725717	205.330427725717\\
63.875	0.19914	215.893266963785	215.893266963785\\
63.875	0.2028	226.762770002702	226.762770002702\\
63.875	0.20646	237.938936842469	237.938936842469\\
63.875	0.21012	249.421767483086	249.421767483086\\
63.875	0.21378	261.211261924554	261.211261924554\\
63.875	0.21744	273.307420166871	273.307420166871\\
63.875	0.2211	285.710242210038	285.710242210038\\
63.875	0.22476	298.419728054056	298.419728054056\\
63.875	0.22842	311.435877698923	311.435877698923\\
63.875	0.23208	324.75869114464	324.75869114464\\
63.875	0.23574	338.388168391208	338.388168391208\\
63.875	0.2394	352.324309438625	352.324309438625\\
63.875	0.24306	366.567114286892	366.567114286892\\
63.875	0.24672	381.11658293601	381.11658293601\\
63.875	0.25038	395.972715385977	395.972715385977\\
63.875	0.25404	411.135511636794	411.135511636794\\
63.875	0.2577	426.604971688462	426.604971688462\\
63.875	0.26136	442.381095540979	442.381095540979\\
63.875	0.26502	458.463883194346	458.463883194346\\
63.875	0.26868	474.853334648564	474.853334648564\\
63.875	0.27234	491.549449903631	491.549449903631\\
63.875	0.276	508.552228959548	508.552228959548\\
64.25	0.093	34.5476372136227	34.5476372136227\\
64.25	0.09666	36.5185439001042	36.5185439001042\\
64.25	0.10032	38.7961143874356	38.7961143874356\\
64.25	0.10398	41.3803486756172	41.3803486756172\\
64.25	0.10764	44.2712467646487	44.2712467646487\\
64.25	0.1113	47.4688086545302	47.4688086545302\\
64.25	0.11496	50.9730343452617	50.9730343452617\\
64.25	0.11862	54.7839238368433	54.7839238368433\\
64.25	0.12228	58.9014771292748	58.9014771292748\\
64.25	0.12594	63.3256942225564	63.3256942225564\\
64.25	0.1296	68.056575116688	68.056575116688\\
64.25	0.13326	73.0941198116695	73.0941198116695\\
64.25	0.13692	78.4383283075011	78.4383283075011\\
64.25	0.14058	84.0892006041828	84.0892006041828\\
64.25	0.14424	90.0467367017144	90.0467367017144\\
64.25	0.1479	96.310936600096	96.310936600096\\
64.25	0.15156	102.881800299327	102.881800299327\\
64.25	0.15522	109.759327799409	109.759327799409\\
64.25	0.15888	116.943519100341	116.943519100341\\
64.25	0.16254	124.434374202122	124.434374202122\\
64.25	0.1662	132.231893104754	132.231893104754\\
64.25	0.16986	140.336075808236	140.336075808236\\
64.25	0.17352	148.746922312567	148.746922312567\\
64.25	0.17718	157.464432617749	157.464432617749\\
64.25	0.18084	166.488606723781	166.488606723781\\
64.25	0.1845	175.819444630663	175.819444630663\\
64.25	0.18816	185.456946338394	185.456946338394\\
64.25	0.19182	195.401111846976	195.401111846976\\
64.25	0.19548	205.651941156408	205.651941156408\\
64.25	0.19914	216.209434266689	216.209434266689\\
64.25	0.2028	227.073591177821	227.073591177821\\
64.25	0.20646	238.244411889803	238.244411889803\\
64.25	0.21012	249.721896402635	249.721896402635\\
64.25	0.21378	261.506044716316	261.506044716316\\
64.25	0.21744	273.596856830848	273.596856830848\\
64.25	0.2211	285.99433274623	285.99433274623\\
64.25	0.22476	298.698472462462	298.698472462462\\
64.25	0.22842	311.709275979544	311.709275979544\\
64.25	0.23208	325.026743297475	325.026743297475\\
64.25	0.23574	338.650874416257	338.650874416257\\
64.25	0.2394	352.581669335889	352.581669335889\\
64.25	0.24306	366.819128056371	366.819128056371\\
64.25	0.24672	381.363250577703	381.363250577703\\
64.25	0.25038	396.214036899885	396.214036899885\\
64.25	0.25404	411.371487022916	411.371487022916\\
64.25	0.2577	426.835600946798	426.835600946798\\
64.25	0.26136	442.60637867153	442.60637867153\\
64.25	0.26502	458.683820197112	458.683820197112\\
64.25	0.26868	475.067925523544	475.067925523544\\
64.25	0.27234	491.758694650826	491.758694650826\\
64.25	0.276	508.756127578957	508.756127578957\\
64.625	0.093	35.0297996531169	35.0297996531169\\
64.625	0.09666	36.9953602118129	36.9953602118129\\
64.625	0.10032	39.2675845713589	39.2675845713589\\
64.625	0.10398	41.8464727317549	41.8464727317549\\
64.625	0.10764	44.7320246930009	44.7320246930009\\
64.625	0.1113	47.924240455097	47.924240455097\\
64.625	0.11496	51.423120018043	51.423120018043\\
64.625	0.11862	55.228663381839	55.228663381839\\
64.625	0.12228	59.3408705464851	59.3408705464851\\
64.625	0.12594	63.7597415119812	63.7597415119812\\
64.625	0.1296	68.4852762783272	68.4852762783272\\
64.625	0.13326	73.5174748455232	73.5174748455232\\
64.625	0.13692	78.8563372135693	78.8563372135693\\
64.625	0.14058	84.5018633824654	84.5018633824654\\
64.625	0.14424	90.4540533522116	90.4540533522116\\
64.625	0.1479	96.7129071228076	96.7129071228076\\
64.625	0.15156	103.278424694254	103.278424694254\\
64.625	0.15522	110.15060606655	110.15060606655\\
64.625	0.15888	117.329451239696	117.329451239696\\
64.625	0.16254	124.814960213692	124.814960213692\\
64.625	0.1662	132.607132988538	132.607132988538\\
64.625	0.16986	140.705969564234	140.705969564234\\
64.625	0.17352	149.111469940781	149.111469940781\\
64.625	0.17718	157.823634118177	157.823634118177\\
64.625	0.18084	166.842462096423	166.842462096423\\
64.625	0.1845	176.167953875519	176.167953875519\\
64.625	0.18816	185.800109455465	185.800109455465\\
64.625	0.19182	195.738928836262	195.738928836262\\
64.625	0.19548	205.984412017908	205.984412017908\\
64.625	0.19914	216.536559000404	216.536559000404\\
64.625	0.2028	227.39536978375	227.39536978375\\
64.625	0.20646	238.560844367947	238.560844367947\\
64.625	0.21012	250.032982752993	250.032982752993\\
64.625	0.21378	261.811784938889	261.811784938889\\
64.625	0.21744	273.897250925635	273.897250925635\\
64.625	0.2211	286.289380713231	286.289380713231\\
64.625	0.22476	298.988174301678	298.988174301678\\
64.625	0.22842	311.993631690974	311.993631690974\\
64.625	0.23208	325.30575288112	325.30575288112\\
64.625	0.23574	338.924537872117	338.924537872117\\
64.625	0.2394	352.849986663963	352.849986663963\\
64.625	0.24306	367.082099256659	367.082099256659\\
64.625	0.24672	381.620875650206	381.620875650206\\
64.625	0.25038	396.466315844602	396.466315844602\\
64.625	0.25404	411.618419839848	411.618419839848\\
64.625	0.2577	427.077187635945	427.077187635945\\
64.625	0.26136	442.842619232891	442.842619232891\\
64.625	0.26502	458.914714630688	458.914714630688\\
64.625	0.26868	475.293473829334	475.293473829334\\
64.625	0.27234	491.97889682883	491.97889682883\\
64.625	0.276	508.970983629177	508.970983629177\\
65	0.093	35.5229195234211	35.5229195234211\\
65	0.09666	37.4831339543316	37.4831339543316\\
65	0.10032	39.7500121860921	39.7500121860921\\
65	0.10398	42.3235542187026	42.3235542187026\\
65	0.10764	45.2037600521631	45.2037600521631\\
65	0.1113	48.3906296864736	48.3906296864736\\
65	0.11496	51.8841631216341	51.8841631216341\\
65	0.11862	55.6843603576447	55.6843603576447\\
65	0.12228	59.7912213945052	59.7912213945052\\
65	0.12594	64.2047462322158	64.2047462322158\\
65	0.1296	68.9249348707763	68.9249348707763\\
65	0.13326	73.9517873101869	73.9517873101869\\
65	0.13692	79.2853035504474	79.2853035504474\\
65	0.14058	84.925483591558	84.925483591558\\
65	0.14424	90.8723274335186	90.8723274335186\\
65	0.1479	97.1258350763292	97.1258350763292\\
65	0.15156	103.68600651999	103.68600651999\\
65	0.15522	110.5528417645	110.5528417645\\
65	0.15888	117.726340809861	117.726340809861\\
65	0.16254	125.206503656072	125.206503656072\\
65	0.1662	132.993330303132	132.993330303132\\
65	0.16986	141.086820751043	141.086820751043\\
65	0.17352	149.486974999804	149.486974999804\\
65	0.17718	158.193793049414	158.193793049414\\
65	0.18084	167.207274899875	167.207274899875\\
65	0.1845	176.527420551186	176.527420551186\\
65	0.18816	186.154230003346	186.154230003346\\
65	0.19182	196.087703256357	196.087703256357\\
65	0.19548	206.327840310218	206.327840310218\\
65	0.19914	216.874641164928	216.874641164928\\
65	0.2028	227.728105820489	227.728105820489\\
65	0.20646	238.8882342769	238.8882342769\\
65	0.21012	250.355026534161	250.355026534161\\
65	0.21378	262.128482592271	262.128482592271\\
65	0.21744	274.208602451232	274.208602451232\\
65	0.2211	286.595386111043	286.595386111043\\
65	0.22476	299.288833571704	299.288833571704\\
65	0.22842	312.288944833215	312.288944833215\\
65	0.23208	325.595719895575	325.595719895575\\
65	0.23574	339.209158758786	339.209158758786\\
65	0.2394	353.129261422847	353.129261422847\\
65	0.24306	367.356027887758	367.356027887758\\
65	0.24672	381.889458153519	381.889458153519\\
65	0.25038	396.72955222013	396.72955222013\\
65	0.25404	411.87631008759	411.87631008759\\
65	0.2577	427.329731755901	427.329731755901\\
65	0.26136	443.089817225062	443.089817225062\\
65	0.26502	459.156566495073	459.156566495073\\
65	0.26868	475.529979565934	475.529979565934\\
65	0.27234	492.210056437645	492.210056437645\\
65	0.276	509.196797110205	509.196797110205\\
65.375	0.093	36.0269968245352	36.0269968245352\\
65.375	0.09666	37.9818651276602	37.9818651276602\\
65.375	0.10032	40.2433972316352	40.2433972316352\\
65.375	0.10398	42.8115931364602	42.8115931364602\\
65.375	0.10764	45.6864528421352	45.6864528421352\\
65.375	0.1113	48.8679763486602	48.8679763486602\\
65.375	0.11496	52.3561636560352	52.3561636560352\\
65.375	0.11862	56.1510147642602	56.1510147642602\\
65.375	0.12228	60.2525296733353	60.2525296733353\\
65.375	0.12594	64.6607083832603	64.6607083832603\\
65.375	0.1296	69.3755508940353	69.3755508940353\\
65.375	0.13326	74.3970572056604	74.3970572056604\\
65.375	0.13692	79.7252273181354	79.7252273181354\\
65.375	0.14058	85.3600612314605	85.3600612314605\\
65.375	0.14424	91.3015589456357	91.3015589456357\\
65.375	0.1479	97.5497204606608	97.5497204606608\\
65.375	0.15156	104.104545776536	104.104545776536\\
65.375	0.15522	110.966034893261	110.966034893261\\
65.375	0.15888	118.134187810836	118.134187810836\\
65.375	0.16254	125.609004529261	125.609004529261\\
65.375	0.1662	133.390485048536	133.390485048536\\
65.375	0.16986	141.478629368662	141.478629368662\\
65.375	0.17352	149.873437489637	149.873437489637\\
65.375	0.17718	158.574909411462	158.574909411462\\
65.375	0.18084	167.583045134137	167.583045134137\\
65.375	0.1845	176.897844657662	176.897844657662\\
65.375	0.18816	186.519307982037	186.519307982037\\
65.375	0.19182	196.447435107263	196.447435107263\\
65.375	0.19548	206.682226033338	206.682226033338\\
65.375	0.19914	217.223680760263	217.223680760263\\
65.375	0.2028	228.071799288038	228.071799288038\\
65.375	0.20646	239.226581616664	239.226581616664\\
65.375	0.21012	250.688027746139	250.688027746139\\
65.375	0.21378	262.456137676464	262.456137676464\\
65.375	0.21744	274.530911407639	274.530911407639\\
65.375	0.2211	286.912348939664	286.912348939664\\
65.375	0.22476	299.60045027254	299.60045027254\\
65.375	0.22842	312.595215406265	312.595215406265\\
65.375	0.23208	325.89664434084	325.89664434084\\
65.375	0.23574	339.504737076266	339.504737076266\\
65.375	0.2394	353.419493612541	353.419493612541\\
65.375	0.24306	367.640913949666	367.640913949666\\
65.375	0.24672	382.168998087642	382.168998087642\\
65.375	0.25038	397.003746026467	397.003746026467\\
65.375	0.25404	412.145157766142	412.145157766142\\
65.375	0.2577	427.593233306668	427.593233306668\\
65.375	0.26136	443.347972648043	443.347972648043\\
65.375	0.26502	459.409375790268	459.409375790268\\
65.375	0.26868	475.777442733344	475.777442733344\\
65.375	0.27234	492.452173477269	492.452173477269\\
65.375	0.276	509.433568022044	509.433568022044\\
65.75	0.093	36.5420315564592	36.5420315564592\\
65.75	0.09666	38.4915537317987	38.4915537317987\\
65.75	0.10032	40.7477397079882	40.7477397079882\\
65.75	0.10398	43.3105894850276	43.3105894850276\\
65.75	0.10764	46.1801030629171	46.1801030629171\\
65.75	0.1113	49.3562804416566	49.3562804416566\\
65.75	0.11496	52.8391216212462	52.8391216212462\\
65.75	0.11862	56.6286266016857	56.6286266016857\\
65.75	0.12228	60.7247953829753	60.7247953829753\\
65.75	0.12594	65.1276279651148	65.1276279651148\\
65.75	0.1296	69.8371243481043	69.8371243481043\\
65.75	0.13326	74.8532845319438	74.8532845319438\\
65.75	0.13692	80.1761085166334	80.1761085166334\\
65.75	0.14058	85.805596302173	85.805596302173\\
65.75	0.14424	91.7417478885626	91.7417478885626\\
65.75	0.1479	97.9845632758021	97.9845632758021\\
65.75	0.15156	104.534042463892	104.534042463892\\
65.75	0.15522	111.390185452831	111.390185452831\\
65.75	0.15888	118.552992242621	118.552992242621\\
65.75	0.16254	126.022462833261	126.022462833261\\
65.75	0.1662	133.79859722475	133.79859722475\\
65.75	0.16986	141.88139541709	141.88139541709\\
65.75	0.17352	150.27085741028	150.27085741028\\
65.75	0.17718	158.966983204319	158.966983204319\\
65.75	0.18084	167.969772799209	167.969772799209\\
65.75	0.1845	177.279226194949	177.279226194949\\
65.75	0.18816	186.895343391538	186.895343391538\\
65.75	0.19182	196.818124388978	196.818124388978\\
65.75	0.19548	207.047569187268	207.047569187268\\
65.75	0.19914	217.583677786407	217.583677786407\\
65.75	0.2028	228.426450186397	228.426450186397\\
65.75	0.20646	239.575886387237	239.575886387237\\
65.75	0.21012	251.031986388927	251.031986388927\\
65.75	0.21378	262.794750191466	262.794750191466\\
65.75	0.21744	274.864177794856	274.864177794856\\
65.75	0.2211	287.240269199096	287.240269199096\\
65.75	0.22476	299.923024404186	299.923024404186\\
65.75	0.22842	312.912443410125	312.912443410125\\
65.75	0.23208	326.208526216915	326.208526216915\\
65.75	0.23574	339.811272824555	339.811272824555\\
65.75	0.2394	353.720683233045	353.720683233045\\
65.75	0.24306	367.936757442385	367.936757442385\\
65.75	0.24672	382.459495452574	382.459495452574\\
65.75	0.25038	397.288897263614	397.288897263614\\
65.75	0.25404	412.424962875504	412.424962875504\\
65.75	0.2577	427.867692288244	427.867692288244\\
65.75	0.26136	443.617085501834	443.617085501834\\
65.75	0.26502	459.673142516274	459.673142516274\\
65.75	0.26868	476.035863331563	476.035863331563\\
65.75	0.27234	492.705247947703	492.705247947703\\
65.75	0.276	509.681296364693	509.681296364693\\
66.125	0.093	37.0680237191932	37.0680237191932\\
66.125	0.09666	39.0121997667471	39.0121997667471\\
66.125	0.10032	41.2630396151511	41.2630396151511\\
66.125	0.10398	43.8205432644051	43.8205432644051\\
66.125	0.10764	46.6847107145091	46.6847107145091\\
66.125	0.1113	49.8555419654631	49.8555419654631\\
66.125	0.11496	53.3330370172671	53.3330370172671\\
66.125	0.11862	57.1171958699211	57.1171958699211\\
66.125	0.12228	61.2080185234252	61.2080185234252\\
66.125	0.12594	65.6055049777792	65.6055049777792\\
66.125	0.1296	70.3096552329832	70.3096552329832\\
66.125	0.13326	75.3204692890372	75.3204692890372\\
66.125	0.13692	80.6379471459413	80.6379471459413\\
66.125	0.14058	86.2620888036954	86.2620888036954\\
66.125	0.14424	92.1928942622995	92.1928942622995\\
66.125	0.1479	98.4303635217535	98.4303635217535\\
66.125	0.15156	104.974496582058	104.974496582058\\
66.125	0.15522	111.825293443212	111.825293443212\\
66.125	0.15888	118.982754105216	118.982754105216\\
66.125	0.16254	126.44687856807	126.44687856807\\
66.125	0.1662	134.217666831774	134.217666831774\\
66.125	0.16986	142.295118896328	142.295118896328\\
66.125	0.17352	150.679234761732	150.679234761732\\
66.125	0.17718	159.370014427987	159.370014427987\\
66.125	0.18084	168.367457895091	168.367457895091\\
66.125	0.1845	177.671565163045	177.671565163045\\
66.125	0.18816	187.282336231849	187.282336231849\\
66.125	0.19182	197.199771101503	197.199771101503\\
66.125	0.19548	207.423869772007	207.423869772007\\
66.125	0.19914	217.954632243362	217.954632243362\\
66.125	0.2028	228.792058515566	228.792058515566\\
66.125	0.20646	239.93614858862	239.93614858862\\
66.125	0.21012	251.386902462524	251.386902462524\\
66.125	0.21378	263.144320137279	263.144320137279\\
66.125	0.21744	275.208401612883	275.208401612883\\
66.125	0.2211	287.579146889337	287.579146889337\\
66.125	0.22476	300.256555966641	300.256555966641\\
66.125	0.22842	313.240628844796	313.240628844796\\
66.125	0.23208	326.5313655238	326.5313655238\\
66.125	0.23574	340.128766003654	340.128766003654\\
66.125	0.2394	354.032830284359	354.032830284359\\
66.125	0.24306	368.243558365913	368.243558365913\\
66.125	0.24672	382.760950248317	382.760950248317\\
66.125	0.25038	397.585005931572	397.585005931572\\
66.125	0.25404	412.715725415676	412.715725415676\\
66.125	0.2577	428.15310870063	428.15310870063\\
66.125	0.26136	443.897155786434	443.897155786434\\
66.125	0.26502	459.947866673089	459.947866673089\\
66.125	0.26868	476.305241360593	476.305241360593\\
66.125	0.27234	492.969279848948	492.969279848948\\
66.125	0.276	509.939982138152	509.939982138152\\
66.5	0.093	37.604973312737	37.604973312737\\
66.5	0.09666	39.5438032325054	39.5438032325054\\
66.5	0.10032	41.7892969531239	41.7892969531239\\
66.5	0.10398	44.3414544745924	44.3414544745924\\
66.5	0.10764	47.2002757969109	47.2002757969109\\
66.5	0.1113	50.3657609200794	50.3657609200794\\
66.5	0.11496	53.8379098440979	53.8379098440979\\
66.5	0.11862	57.6167225689664	57.6167225689664\\
66.5	0.12228	61.7021990946849	61.7021990946849\\
66.5	0.12594	66.0943394212535	66.0943394212535\\
66.5	0.1296	70.793143548672	70.793143548672\\
66.5	0.13326	75.7986114769405	75.7986114769405\\
66.5	0.13692	81.1107432060591	81.1107432060591\\
66.5	0.14058	86.7295387360276	86.7295387360276\\
66.5	0.14424	92.6549980668462	92.6549980668462\\
66.5	0.1479	98.8871211985148	98.8871211985148\\
66.5	0.15156	105.425908131033	105.425908131033\\
66.5	0.15522	112.271358864402	112.271358864402\\
66.5	0.15888	119.423473398621	119.423473398621\\
66.5	0.16254	126.882251733689	126.882251733689\\
66.5	0.1662	134.647693869608	134.647693869608\\
66.5	0.16986	142.719799806376	142.719799806376\\
66.5	0.17352	151.098569543995	151.098569543995\\
66.5	0.17718	159.784003082464	159.784003082464\\
66.5	0.18084	168.776100421782	168.776100421782\\
66.5	0.1845	178.074861561951	178.074861561951\\
66.5	0.18816	187.68028650297	187.68028650297\\
66.5	0.19182	197.592375244838	197.592375244838\\
66.5	0.19548	207.811127787557	207.811127787557\\
66.5	0.19914	218.336544131126	218.336544131126\\
66.5	0.2028	229.168624275545	229.168624275545\\
66.5	0.20646	240.307368220813	240.307368220813\\
66.5	0.21012	251.752775966932	251.752775966932\\
66.5	0.21378	263.504847513901	263.504847513901\\
66.5	0.21744	275.563582861719	275.563582861719\\
66.5	0.2211	287.928982010388	287.928982010388\\
66.5	0.22476	300.601044959907	300.601044959907\\
66.5	0.22842	313.579771710276	313.579771710276\\
66.5	0.23208	326.865162261495	326.865162261495\\
66.5	0.23574	340.457216613563	340.457216613563\\
66.5	0.2394	354.355934766482	354.355934766482\\
66.5	0.24306	368.561316720251	368.561316720251\\
66.5	0.24672	383.07336247487	383.07336247487\\
66.5	0.25038	397.892072030339	397.892072030339\\
66.5	0.25404	413.017445386657	413.017445386657\\
66.5	0.2577	428.449482543826	428.449482543826\\
66.5	0.26136	444.188183501845	444.188183501845\\
66.5	0.26502	460.233548260714	460.233548260714\\
66.5	0.26868	476.585576820433	476.585576820433\\
66.5	0.27234	493.244269181002	493.244269181002\\
66.5	0.276	510.20962534242	510.20962534242\\
66.875	0.093	38.1528803370908	38.1528803370908\\
66.875	0.09666	40.0863641290737	40.0863641290737\\
66.875	0.10032	42.3265117219067	42.3265117219067\\
66.875	0.10398	44.8733231155897	44.8733231155897\\
66.875	0.10764	47.7267983101226	47.7267983101226\\
66.875	0.1113	50.8869373055056	50.8869373055056\\
66.875	0.11496	54.3537401017386	54.3537401017386\\
66.875	0.11862	58.1272066988216	58.1272066988216\\
66.875	0.12228	62.2073370967547	62.2073370967547\\
66.875	0.12594	66.5941312955377	66.5941312955377\\
66.875	0.1296	71.2875892951708	71.2875892951708\\
66.875	0.13326	76.2877110956537	76.2877110956537\\
66.875	0.13692	81.5944966969868	81.5944966969868\\
66.875	0.14058	87.2079460991699	87.2079460991699\\
66.875	0.14424	93.1280593022029	93.1280593022029\\
66.875	0.1479	99.354836306086	99.354836306086\\
66.875	0.15156	105.888277110819	105.888277110819\\
66.875	0.15522	112.728381716402	112.728381716402\\
66.875	0.15888	119.875150122835	119.875150122835\\
66.875	0.16254	127.328582330118	127.328582330118\\
66.875	0.1662	135.088678338251	135.088678338251\\
66.875	0.16986	143.155438147235	143.155438147235\\
66.875	0.17352	151.528861757068	151.528861757068\\
66.875	0.17718	160.208949167751	160.208949167751\\
66.875	0.18084	169.195700379284	169.195700379284\\
66.875	0.1845	178.489115391667	178.489115391667\\
66.875	0.18816	188.0891942049	188.0891942049\\
66.875	0.19182	197.995936818984	197.995936818984\\
66.875	0.19548	208.209343233917	208.209343233917\\
66.875	0.19914	218.7294134497	218.7294134497\\
66.875	0.2028	229.556147466333	229.556147466333\\
66.875	0.20646	240.689545283816	240.689545283816\\
66.875	0.21012	252.12960690215	252.12960690215\\
66.875	0.21378	263.876332321333	263.876332321333\\
66.875	0.21744	275.929721541366	275.929721541366\\
66.875	0.2211	288.289774562249	288.289774562249\\
66.875	0.22476	300.956491383983	300.956491383983\\
66.875	0.22842	313.929872006566	313.929872006566\\
66.875	0.23208	327.209916429999	327.209916429999\\
66.875	0.23574	340.796624654282	340.796624654282\\
66.875	0.2394	354.689996679416	354.689996679416\\
66.875	0.24306	368.890032505399	368.890032505399\\
66.875	0.24672	383.396732132232	383.396732132232\\
66.875	0.25038	398.210095559916	398.210095559916\\
66.875	0.25404	413.330122788449	413.330122788449\\
66.875	0.2577	428.756813817832	428.756813817832\\
66.875	0.26136	444.490168648066	444.490168648066\\
66.875	0.26502	460.530187279149	460.530187279149\\
66.875	0.26868	476.876869711082	476.876869711082\\
66.875	0.27234	493.530215943866	493.530215943866\\
66.875	0.276	510.490225977499	510.490225977499\\
67.25	0.093	38.7117447922545	38.7117447922545\\
67.25	0.09666	40.6398824564519	40.6398824564519\\
67.25	0.10032	42.8746839214994	42.8746839214994\\
67.25	0.10398	45.4161491873968	45.4161491873968\\
67.25	0.10764	48.2642782541443	48.2642782541443\\
67.25	0.1113	51.4190711217418	51.4190711217418\\
67.25	0.11496	54.8805277901893	54.8805277901893\\
67.25	0.11862	58.6486482594868	58.6486482594868\\
67.25	0.12228	62.7234325296343	62.7234325296343\\
67.25	0.12594	67.1048806006318	67.1048806006318\\
67.25	0.1296	71.7929924724794	71.7929924724794\\
67.25	0.13326	76.7877681451768	76.7877681451768\\
67.25	0.13692	82.0892076187244	82.0892076187244\\
67.25	0.14058	87.697310893122	87.697310893122\\
67.25	0.14424	93.6120779683695	93.6120779683695\\
67.25	0.1479	99.8335088444671	99.8335088444671\\
67.25	0.15156	106.361603521415	106.361603521415\\
67.25	0.15522	113.196361999212	113.196361999212\\
67.25	0.15888	120.33778427786	120.33778427786\\
67.25	0.16254	127.785870357358	127.785870357358\\
67.25	0.1662	135.540620237705	135.540620237705\\
67.25	0.16986	143.602033918903	143.602033918903\\
67.25	0.17352	151.97011140095	151.97011140095\\
67.25	0.17718	160.644852683848	160.644852683848\\
67.25	0.18084	169.626257767596	169.626257767596\\
67.25	0.1845	178.914326652193	178.914326652193\\
67.25	0.18816	188.509059337641	188.509059337641\\
67.25	0.19182	198.410455823939	198.410455823939\\
67.25	0.19548	208.618516111086	208.618516111086\\
67.25	0.19914	219.133240199084	219.133240199084\\
67.25	0.2028	229.954628087932	229.954628087932\\
67.25	0.20646	241.08267977763	241.08267977763\\
67.25	0.21012	252.517395268177	252.517395268177\\
67.25	0.21378	264.258774559575	264.258774559575\\
67.25	0.21744	276.306817651823	276.306817651823\\
67.25	0.2211	288.66152454492	288.66152454492\\
67.25	0.22476	301.322895238868	301.322895238868\\
67.25	0.22842	314.290929733666	314.290929733666\\
67.25	0.23208	327.565628029314	327.565628029314\\
67.25	0.23574	341.146990125811	341.146990125811\\
67.25	0.2394	355.035016023159	355.035016023159\\
67.25	0.24306	369.229705721357	369.229705721357\\
67.25	0.24672	383.731059220405	383.731059220405\\
67.25	0.25038	398.539076520303	398.539076520303\\
67.25	0.25404	413.65375762105	413.65375762105\\
67.25	0.2577	429.075102522648	429.075102522648\\
67.25	0.26136	444.803111225096	444.803111225096\\
67.25	0.26502	460.837783728394	460.837783728394\\
67.25	0.26868	477.179120032542	477.179120032542\\
67.25	0.27234	493.82712013754	493.82712013754\\
67.25	0.276	510.781784043387	510.781784043387\\
67.625	0.093	39.2815666782281	39.2815666782281\\
67.625	0.09666	41.20435821464	41.20435821464\\
67.625	0.10032	43.4338135519019	43.4338135519019\\
67.625	0.10398	45.9699326900139	45.9699326900139\\
67.625	0.10764	48.8127156289759	48.8127156289759\\
67.625	0.1113	51.9621623687879	51.9621623687879\\
67.625	0.11496	55.4182729094499	55.4182729094499\\
67.625	0.11862	59.1810472509619	59.1810472509619\\
67.625	0.12228	63.2504853933239	63.2504853933239\\
67.625	0.12594	67.6265873365359	67.6265873365359\\
67.625	0.1296	72.3093530805979	72.3093530805979\\
67.625	0.13326	77.2987826255099	77.2987826255099\\
67.625	0.13692	82.5948759712719	82.5948759712719\\
67.625	0.14058	88.197633117884	88.197633117884\\
67.625	0.14424	94.1070540653461	94.1070540653461\\
67.625	0.1479	100.323138813658	100.323138813658\\
67.625	0.15156	106.84588736282	106.84588736282\\
67.625	0.15522	113.675299712832	113.675299712832\\
67.625	0.15888	120.811375863695	120.811375863695\\
67.625	0.16254	128.254115815407	128.254115815407\\
67.625	0.1662	136.003519567969	136.003519567969\\
67.625	0.16986	144.059587121381	144.059587121381\\
67.625	0.17352	152.422318475643	152.422318475643\\
67.625	0.17718	161.091713630755	161.091713630755\\
67.625	0.18084	170.067772586717	170.067772586717\\
67.625	0.1845	179.350495343529	179.350495343529\\
67.625	0.18816	188.939881901192	188.939881901192\\
67.625	0.19182	198.835932259704	198.835932259704\\
67.625	0.19548	209.038646419066	209.038646419066\\
67.625	0.19914	219.548024379278	219.548024379278\\
67.625	0.2028	230.36406614034	230.36406614034\\
67.625	0.20646	241.486771702253	241.486771702253\\
67.625	0.21012	252.916141065015	252.916141065015\\
67.625	0.21378	264.652174228627	264.652174228627\\
67.625	0.21744	276.694871193089	276.694871193089\\
67.625	0.2211	289.044231958401	289.044231958401\\
67.625	0.22476	301.700256524564	301.700256524564\\
67.625	0.22842	314.662944891576	314.662944891576\\
67.625	0.23208	327.932297059438	327.932297059438\\
67.625	0.23574	341.50831302815	341.50831302815\\
67.625	0.2394	355.390992797713	355.390992797713\\
67.625	0.24306	369.580336368125	369.580336368125\\
67.625	0.24672	384.076343739387	384.076343739387\\
67.625	0.25038	398.8790149115	398.8790149115\\
67.625	0.25404	413.988349884462	413.988349884462\\
67.625	0.2577	429.404348658274	429.404348658274\\
67.625	0.26136	445.127011232936	445.127011232936\\
67.625	0.26502	461.156337608449	461.156337608449\\
67.625	0.26868	477.492327784811	477.492327784811\\
67.625	0.27234	494.134981762023	494.134981762023\\
67.625	0.276	511.084299540086	511.084299540086\\
68	0.093	39.8623459950116	39.8623459950116\\
68	0.09666	41.779791403638	41.779791403638\\
68	0.10032	44.0039006131144	44.0039006131144\\
68	0.10398	46.5346736234409	46.5346736234409\\
68	0.10764	49.3721104346174	49.3721104346174\\
68	0.1113	52.5162110466438	52.5162110466438\\
68	0.11496	55.9669754595203	55.9669754595203\\
68	0.11862	59.7244036732469	59.7244036732469\\
68	0.12228	63.7884956878234	63.7884956878234\\
68	0.12594	68.1592515032499	68.1592515032499\\
68	0.1296	72.8366711195264	72.8366711195264\\
68	0.13326	77.8207545366529	77.8207545366529\\
68	0.13692	83.1115017546294	83.1115017546294\\
68	0.14058	88.708912773456	88.708912773456\\
68	0.14424	94.6129875931326	94.6129875931326\\
68	0.1479	100.823726213659	100.823726213659\\
68	0.15156	107.341128635036	107.341128635036\\
68	0.15522	114.165194857262	114.165194857262\\
68	0.15888	121.295924880339	121.295924880339\\
68	0.16254	128.733318704265	128.733318704265\\
68	0.1662	136.477376329042	136.477376329042\\
68	0.16986	144.528097754669	144.528097754669\\
68	0.17352	152.885482981145	152.885482981145\\
68	0.17718	161.549532008472	161.549532008472\\
68	0.18084	170.520244836648	170.520244836648\\
68	0.1845	179.797621465675	179.797621465675\\
68	0.18816	189.381661895552	189.381661895552\\
68	0.19182	199.272366126278	199.272366126278\\
68	0.19548	209.469734157855	209.469734157855\\
68	0.19914	219.973765990282	219.973765990282\\
68	0.2028	230.784461623559	230.784461623559\\
68	0.20646	241.901821057685	241.901821057685\\
68	0.21012	253.325844292662	253.325844292662\\
68	0.21378	265.056531328489	265.056531328489\\
68	0.21744	277.093882165165	277.093882165165\\
68	0.2211	289.437896802692	289.437896802692\\
68	0.22476	302.088575241069	302.088575241069\\
68	0.22842	315.045917480296	315.045917480296\\
68	0.23208	328.309923520372	328.309923520372\\
68	0.23574	341.880593361299	341.880593361299\\
68	0.2394	355.757927003076	355.757927003076\\
68	0.24306	369.941924445703	369.941924445703\\
68	0.24672	384.432585689179	384.432585689179\\
68	0.25038	399.229910733506	399.229910733506\\
68	0.25404	414.333899578683	414.333899578683\\
68	0.2577	429.74455222471	429.74455222471\\
68	0.26136	445.461868671587	445.461868671587\\
68	0.26502	461.485848919314	461.485848919314\\
68	0.26868	477.81649296789	477.81649296789\\
68	0.27234	494.453800817317	494.453800817317\\
68	0.276	511.397772467594	511.397772467594\\
68.375	0.093	40.454082742605	40.454082742605\\
68.375	0.09666	42.366182023446	42.366182023446\\
68.375	0.10032	44.5849451051369	44.5849451051369\\
68.375	0.10398	47.1103719876779	47.1103719876779\\
68.375	0.10764	49.9424626710689	49.9424626710689\\
68.375	0.1113	53.0812171553098	53.0812171553098\\
68.375	0.11496	56.5266354404008	56.5266354404008\\
68.375	0.11862	60.2787175263418	60.2787175263418\\
68.375	0.12228	64.3374634131328	64.3374634131328\\
68.375	0.12594	68.7028731007738	68.7028731007738\\
68.375	0.1296	73.3749465892648	73.3749465892648\\
68.375	0.13326	78.3536838786058	78.3536838786058\\
68.375	0.13692	83.6390849687968	83.6390849687968\\
68.375	0.14058	89.2311498598378	89.2311498598378\\
68.375	0.14424	95.1298785517289	95.1298785517289\\
68.375	0.1479	101.33527104447	101.33527104447\\
68.375	0.15156	107.847327338061	107.847327338061\\
68.375	0.15522	114.666047432502	114.666047432502\\
68.375	0.15888	121.791431327793	121.791431327793\\
68.375	0.16254	129.223479023934	129.223479023934\\
68.375	0.1662	136.962190520925	136.962190520925\\
68.375	0.16986	145.007565818766	145.007565818766\\
68.375	0.17352	153.359604917458	153.359604917458\\
68.375	0.17718	162.018307816999	162.018307816999\\
68.375	0.18084	170.98367451739	170.98367451739\\
68.375	0.1845	180.255705018631	180.255705018631\\
68.375	0.18816	189.834399320722	189.834399320722\\
68.375	0.19182	199.719757423663	199.719757423663\\
68.375	0.19548	209.911779327454	209.911779327454\\
68.375	0.19914	220.410465032096	220.410465032096\\
68.375	0.2028	231.215814537587	231.215814537587\\
68.375	0.20646	242.327827843928	242.327827843928\\
68.375	0.21012	253.746504951119	253.746504951119\\
68.375	0.21378	265.47184585916	265.47184585916\\
68.375	0.21744	277.503850568052	277.503850568052\\
68.375	0.2211	289.842519077793	289.842519077793\\
68.375	0.22476	302.487851388384	302.487851388384\\
68.375	0.22842	315.439847499825	315.439847499825\\
68.375	0.23208	328.698507412117	328.698507412117\\
68.375	0.23574	342.263831125258	342.263831125258\\
68.375	0.2394	356.135818639249	356.135818639249\\
68.375	0.24306	370.314469954091	370.314469954091\\
68.375	0.24672	384.799785069782	384.799785069782\\
68.375	0.25038	399.591763986323	399.591763986323\\
68.375	0.25404	414.690406703714	414.690406703714\\
68.375	0.2577	430.095713221956	430.095713221956\\
68.375	0.26136	445.807683541047	445.807683541047\\
68.375	0.26502	461.826317660988	461.826317660988\\
68.375	0.26868	478.15161558178	478.15161558178\\
68.375	0.27234	494.783577303421	494.783577303421\\
68.375	0.276	511.722202825912	511.722202825912\\
68.75	0.093	41.0567769210084	41.0567769210084\\
68.75	0.09666	42.9635300740638	42.9635300740638\\
68.75	0.10032	45.1769470279692	45.1769470279692\\
68.75	0.10398	47.6970277827247	47.6970277827247\\
68.75	0.10764	50.5237723383302	50.5237723383302\\
68.75	0.1113	53.6571806947856	53.6571806947856\\
68.75	0.11496	57.0972528520911	57.0972528520911\\
68.75	0.11862	60.8439888102466	60.8439888102466\\
68.75	0.12228	64.8973885692521	64.8973885692521\\
68.75	0.12594	69.2574521291076	69.2574521291076\\
68.75	0.1296	73.9241794898131	73.9241794898131\\
68.75	0.13326	78.8975706513685	78.8975706513685\\
68.75	0.13692	84.1776256137741	84.1776256137741\\
68.75	0.14058	89.7643443770297	89.7643443770297\\
68.75	0.14424	95.6577269411352	95.6577269411352\\
68.75	0.1479	101.857773306091	101.857773306091\\
68.75	0.15156	108.364483471896	108.364483471896\\
68.75	0.15522	115.177857438552	115.177857438552\\
68.75	0.15888	122.297895206058	122.297895206058\\
68.75	0.16254	129.724596774413	129.724596774413\\
68.75	0.1662	137.457962143619	137.457962143619\\
68.75	0.16986	145.497991313674	145.497991313674\\
68.75	0.17352	153.84468428458	153.84468428458\\
68.75	0.17718	162.498041056335	162.498041056335\\
68.75	0.18084	171.458061628941	171.458061628941\\
68.75	0.1845	180.724746002397	180.724746002397\\
68.75	0.18816	190.298094176702	190.298094176702\\
68.75	0.19182	200.178106151858	200.178106151858\\
68.75	0.19548	210.364781927864	210.364781927864\\
68.75	0.19914	220.858121504719	220.858121504719\\
68.75	0.2028	231.658124882425	231.658124882425\\
68.75	0.20646	242.764792060981	242.764792060981\\
68.75	0.21012	254.178123040386	254.178123040386\\
68.75	0.21378	265.898117820642	265.898117820642\\
68.75	0.21744	277.924776401748	277.924776401748\\
68.75	0.2211	290.258098783704	290.258098783704\\
68.75	0.22476	302.898084966509	302.898084966509\\
68.75	0.22842	315.844734950165	315.844734950165\\
68.75	0.23208	329.098048734671	329.098048734671\\
68.75	0.23574	342.658026320027	342.658026320027\\
68.75	0.2394	356.524667706232	356.524667706232\\
68.75	0.24306	370.697972893288	370.697972893288\\
68.75	0.24672	385.177941881194	385.177941881194\\
68.75	0.25038	399.96457466995	399.96457466995\\
68.75	0.25404	415.057871259555	415.057871259555\\
68.75	0.2577	430.457831650011	430.457831650011\\
68.75	0.26136	446.164455841317	446.164455841317\\
68.75	0.26502	462.177743833473	462.177743833473\\
68.75	0.26868	478.497695626479	478.497695626479\\
68.75	0.27234	495.124311220335	495.124311220335\\
68.75	0.276	512.05759061504	512.05759061504\\
69.125	0.093	41.6704285302217	41.6704285302217\\
69.125	0.09666	43.5718355554916	43.5718355554916\\
69.125	0.10032	45.7799063816115	45.7799063816115\\
69.125	0.10398	48.2946410085815	48.2946410085815\\
69.125	0.10764	51.1160394364015	51.1160394364015\\
69.125	0.1113	54.2441016650714	54.2441016650714\\
69.125	0.11496	57.6788276945914	57.6788276945914\\
69.125	0.11862	61.4202175249614	61.4202175249614\\
69.125	0.12228	65.4682711561813	65.4682711561813\\
69.125	0.12594	69.8229885882514	69.8229885882514\\
69.125	0.1296	74.4843698211714	74.4843698211714\\
69.125	0.13326	79.4524148549413	79.4524148549413\\
69.125	0.13692	84.7271236895614	84.7271236895614\\
69.125	0.14058	90.3084963250315	90.3084963250315\\
69.125	0.14424	96.1965327613515	96.1965327613515\\
69.125	0.1479	102.391232998522	102.391232998522\\
69.125	0.15156	108.892597036542	108.892597036542\\
69.125	0.15522	115.700624875412	115.700624875412\\
69.125	0.15888	122.815316515132	122.815316515132\\
69.125	0.16254	130.236671955702	130.236671955702\\
69.125	0.1662	137.964691197122	137.964691197122\\
69.125	0.16986	145.999374239392	145.999374239392\\
69.125	0.17352	154.340721082512	154.340721082512\\
69.125	0.17718	162.988731726482	162.988731726482\\
69.125	0.18084	171.943406171302	171.943406171302\\
69.125	0.1845	181.204744416972	181.204744416972\\
69.125	0.18816	190.772746463493	190.772746463493\\
69.125	0.19182	200.647412310863	200.647412310863\\
69.125	0.19548	210.828741959083	210.828741959083\\
69.125	0.19914	221.316735408153	221.316735408153\\
69.125	0.2028	232.111392658073	232.111392658073\\
69.125	0.20646	243.212713708844	243.212713708844\\
69.125	0.21012	254.620698560464	254.620698560464\\
69.125	0.21378	266.335347212934	266.335347212934\\
69.125	0.21744	278.356659666254	278.356659666254\\
69.125	0.2211	290.684635920424	290.684635920424\\
69.125	0.22476	303.319275975444	303.319275975444\\
69.125	0.22842	316.260579831315	316.260579831315\\
69.125	0.23208	329.508547488035	329.508547488035\\
69.125	0.23574	343.063178945605	343.063178945605\\
69.125	0.2394	356.924474204025	356.924474204025\\
69.125	0.24306	371.092433263296	371.092433263296\\
69.125	0.24672	385.567056123416	385.567056123416\\
69.125	0.25038	400.348342784386	400.348342784386\\
69.125	0.25404	415.436293246207	415.436293246207\\
69.125	0.2577	430.830907508877	430.830907508877\\
69.125	0.26136	446.532185572397	446.532185572397\\
69.125	0.26502	462.540127436768	462.540127436768\\
69.125	0.26868	478.854733101988	478.854733101988\\
69.125	0.27234	495.476002568058	495.476002568058\\
69.125	0.276	512.403935834978	512.403935834978\\
69.5	0.093	42.2950375702449	42.2950375702449\\
69.5	0.09666	44.1910984677293	44.1910984677293\\
69.5	0.10032	46.3938231660637	46.3938231660637\\
69.5	0.10398	48.9032116652482	48.9032116652482\\
69.5	0.10764	51.7192639652826	51.7192639652826\\
69.5	0.1113	54.8419800661671	54.8419800661671\\
69.5	0.11496	58.2713599679015	58.2713599679015\\
69.5	0.11862	62.007403670486	62.007403670486\\
69.5	0.12228	66.0501111739205	66.0501111739205\\
69.5	0.12594	70.399482478205	70.399482478205\\
69.5	0.1296	75.0555175833395	75.0555175833395\\
69.5	0.13326	80.0182164893239	80.0182164893239\\
69.5	0.13692	85.2875791961585	85.2875791961585\\
69.5	0.14058	90.863605703843	90.863605703843\\
69.5	0.14424	96.7462960123776	96.7462960123776\\
69.5	0.1479	102.935650121762	102.935650121762\\
69.5	0.15156	109.431668031997	109.431668031997\\
69.5	0.15522	116.234349743081	116.234349743081\\
69.5	0.15888	123.343695255016	123.343695255016\\
69.5	0.16254	130.7597045678	130.7597045678\\
69.5	0.1662	138.482377681435	138.482377681435\\
69.5	0.16986	146.51171459592	146.51171459592\\
69.5	0.17352	154.847715311254	154.847715311254\\
69.5	0.17718	163.490379827439	163.490379827439\\
69.5	0.18084	172.439708144473	172.439708144473\\
69.5	0.1845	181.695700262358	181.695700262358\\
69.5	0.18816	191.258356181093	191.258356181093\\
69.5	0.19182	201.127675900677	201.127675900677\\
69.5	0.19548	211.303659421112	211.303659421112\\
69.5	0.19914	221.786306742397	221.786306742397\\
69.5	0.2028	232.575617864531	232.575617864531\\
69.5	0.20646	243.671592787516	243.671592787516\\
69.5	0.21012	255.074231511351	255.074231511351\\
69.5	0.21378	266.783534036035	266.783534036035\\
69.5	0.21744	278.79950036157	278.79950036157\\
69.5	0.2211	291.122130487955	291.122130487955\\
69.5	0.22476	303.751424415189	303.751424415189\\
69.5	0.22842	316.687382143274	316.687382143274\\
69.5	0.23208	329.930003672209	329.930003672209\\
69.5	0.23574	343.479289001994	343.479289001994\\
69.5	0.2394	357.335238132628	357.335238132628\\
69.5	0.24306	371.497851064113	371.497851064113\\
69.5	0.24672	385.967127796448	385.967127796448\\
69.5	0.25038	400.743068329633	400.743068329633\\
69.5	0.25404	415.825672663668	415.825672663668\\
69.5	0.2577	431.214940798552	431.214940798552\\
69.5	0.26136	446.910872734287	446.910872734287\\
69.5	0.26502	462.913468470872	462.913468470872\\
69.5	0.26868	479.222728008307	479.222728008307\\
69.5	0.27234	495.838651346592	495.838651346592\\
69.5	0.276	512.761238485726	512.761238485726\\
69.875	0.093	42.930604041078	42.930604041078\\
69.875	0.09666	44.8213188107769	44.8213188107769\\
69.875	0.10032	47.0186973813259	47.0186973813259\\
69.875	0.10398	49.5227397527248	49.5227397527248\\
69.875	0.10764	52.3334459249737	52.3334459249737\\
69.875	0.1113	55.4508158980726	55.4508158980726\\
69.875	0.11496	58.8748496720216	58.8748496720216\\
69.875	0.11862	62.6055472468206	62.6055472468206\\
69.875	0.12228	66.6429086224696	66.6429086224696\\
69.875	0.12594	70.9869337989686	70.9869337989686\\
69.875	0.1296	75.6376227763175	75.6376227763175\\
69.875	0.13326	80.5949755545165	80.5949755545165\\
69.875	0.13692	85.8589921335655	85.8589921335655\\
69.875	0.14058	91.4296725134646	91.4296725134646\\
69.875	0.14424	97.3070166942137	97.3070166942137\\
69.875	0.1479	103.491024675813	103.491024675813\\
69.875	0.15156	109.981696458262	109.981696458262\\
69.875	0.15522	116.779032041561	116.779032041561\\
69.875	0.15888	123.88303142571	123.88303142571\\
69.875	0.16254	131.293694610709	131.293694610709\\
69.875	0.1662	139.011021596558	139.011021596558\\
69.875	0.16986	147.035012383257	147.035012383257\\
69.875	0.17352	155.365666970806	155.365666970806\\
69.875	0.17718	164.002985359205	164.002985359205\\
69.875	0.18084	172.946967548454	172.946967548454\\
69.875	0.1845	182.197613538554	182.197613538554\\
69.875	0.18816	191.754923329503	191.754923329503\\
69.875	0.19182	201.618896921302	201.618896921302\\
69.875	0.19548	211.789534313951	211.789534313951\\
69.875	0.19914	222.26683550745	222.26683550745\\
69.875	0.2028	233.050800501799	233.050800501799\\
69.875	0.20646	244.141429296999	244.141429296999\\
69.875	0.21012	255.538721893048	255.538721893048\\
69.875	0.21378	267.242678289947	267.242678289947\\
69.875	0.21744	279.253298487696	279.253298487696\\
69.875	0.2211	291.570582486295	291.570582486295\\
69.875	0.22476	304.194530285744	304.194530285744\\
69.875	0.22842	317.125141886044	317.125141886044\\
69.875	0.23208	330.362417287193	330.362417287193\\
69.875	0.23574	343.906356489192	343.906356489192\\
69.875	0.2394	357.756959492041	357.756959492041\\
69.875	0.24306	371.914226295741	371.914226295741\\
69.875	0.24672	386.37815690029	386.37815690029\\
69.875	0.25038	401.148751305689	401.148751305689\\
69.875	0.25404	416.226009511938	416.226009511938\\
69.875	0.2577	431.609931519038	431.609931519038\\
69.875	0.26136	447.300517326987	447.300517326987\\
69.875	0.26502	463.297766935786	463.297766935786\\
69.875	0.26868	479.601680345436	479.601680345436\\
69.875	0.27234	496.212257555935	496.212257555935\\
69.875	0.276	513.129498567284	513.129498567284\\
70.25	0.093	43.5771279427211	43.5771279427211\\
70.25	0.09666	45.4624965846345	45.4624965846345\\
70.25	0.10032	47.6545290273979	47.6545290273979\\
70.25	0.10398	50.1532252710113	50.1532252710113\\
70.25	0.10764	52.9585853154748	52.9585853154748\\
70.25	0.1113	56.0706091607882	56.0706091607882\\
70.25	0.11496	59.4892968069517	59.4892968069517\\
70.25	0.11862	63.2146482539652	63.2146482539652\\
70.25	0.12228	67.2466635018286	67.2466635018286\\
70.25	0.12594	71.5853425505421	71.5853425505421\\
70.25	0.1296	76.2306854001056	76.2306854001056\\
70.25	0.13326	81.182692050519	81.182692050519\\
70.25	0.13692	86.4413625017825	86.4413625017825\\
70.25	0.14058	92.0066967538961	92.0066967538961\\
70.25	0.14424	97.8786948068596	97.8786948068596\\
70.25	0.1479	104.057356660673	104.057356660673\\
70.25	0.15156	110.542682315337	110.542682315337\\
70.25	0.15522	117.33467177085	117.33467177085\\
70.25	0.15888	124.433325027214	124.433325027214\\
70.25	0.16254	131.838642084427	131.838642084427\\
70.25	0.1662	139.550622942491	139.550622942491\\
70.25	0.16986	147.569267601405	147.569267601405\\
70.25	0.17352	155.894576061168	155.894576061168\\
70.25	0.17718	164.526548321782	164.526548321782\\
70.25	0.18084	173.465184383245	173.465184383245\\
70.25	0.1845	182.710484245559	182.710484245559\\
70.25	0.18816	192.262447908723	192.262447908723\\
70.25	0.19182	202.121075372736	202.121075372736\\
70.25	0.19548	212.2863666376	212.2863666376\\
70.25	0.19914	222.758321703313	222.758321703313\\
70.25	0.2028	233.536940569877	233.536940569877\\
70.25	0.20646	244.622223237291	244.622223237291\\
70.25	0.21012	256.014169705555	256.014169705555\\
70.25	0.21378	267.712779974668	267.712779974668\\
70.25	0.21744	279.718054044632	279.718054044632\\
70.25	0.2211	292.029991915446	292.029991915446\\
70.25	0.22476	304.648593587109	304.648593587109\\
70.25	0.22842	317.573859059623	317.573859059623\\
70.25	0.23208	330.805788332987	330.805788332987\\
70.25	0.23574	344.344381407201	344.344381407201\\
70.25	0.2394	358.189638282264	358.189638282264\\
70.25	0.24306	372.341558958178	372.341558958178\\
70.25	0.24672	386.800143434942	386.800143434942\\
70.25	0.25038	401.565391712556	401.565391712556\\
70.25	0.25404	416.637303791019	416.637303791019\\
70.25	0.2577	432.015879670333	432.015879670333\\
70.25	0.26136	447.701119350497	447.701119350497\\
70.25	0.26502	463.693022831511	463.693022831511\\
70.25	0.26868	479.991590113374	479.991590113374\\
70.25	0.27234	496.596821196088	496.596821196088\\
70.25	0.276	513.508716079652	513.508716079652\\
70.625	0.093	44.234609275174	44.234609275174\\
70.625	0.09666	46.1146317893019	46.1146317893019\\
70.625	0.10032	48.3013181042798	48.3013181042798\\
70.625	0.10398	50.7946682201078	50.7946682201078\\
70.625	0.10764	53.5946821367857	53.5946821367857\\
70.625	0.1113	56.7013598543136	56.7013598543136\\
70.625	0.11496	60.1147013726916	60.1147013726916\\
70.625	0.11862	63.8347066919195	63.8347066919195\\
70.625	0.12228	67.8613758119975	67.8613758119975\\
70.625	0.12594	72.1947087329255	72.1947087329255\\
70.625	0.1296	76.8347054547035	76.8347054547035\\
70.625	0.13326	81.7813659773314	81.7813659773314\\
70.625	0.13692	87.0346903008094	87.0346903008094\\
70.625	0.14058	92.5946784251375	92.5946784251375\\
70.625	0.14424	98.4613303503155	98.4613303503155\\
70.625	0.1479	104.634646076344	104.634646076344\\
70.625	0.15156	111.114625603221	111.114625603221\\
70.625	0.15522	117.90126893095	117.90126893095\\
70.625	0.15888	124.994576059528	124.994576059528\\
70.625	0.16254	132.394546988956	132.394546988956\\
70.625	0.1662	140.101181719234	140.101181719234\\
70.625	0.16986	148.114480250362	148.114480250362\\
70.625	0.17352	156.43444258234	156.43444258234\\
70.625	0.17718	165.061068715168	165.061068715168\\
70.625	0.18084	173.994358648846	173.994358648846\\
70.625	0.1845	183.234312383374	183.234312383374\\
70.625	0.18816	192.780929918752	192.780929918752\\
70.625	0.19182	202.634211254981	202.634211254981\\
70.625	0.19548	212.794156392059	212.794156392059\\
70.625	0.19914	223.260765329987	223.260765329987\\
70.625	0.2028	234.034038068765	234.034038068765\\
70.625	0.20646	245.113974608393	245.113974608393\\
70.625	0.21012	256.500574948871	256.500574948871\\
70.625	0.21378	268.193839090199	268.193839090199\\
70.625	0.21744	280.193767032378	280.193767032378\\
70.625	0.2211	292.500358775406	292.500358775406\\
70.625	0.22476	305.113614319284	305.113614319284\\
70.625	0.22842	318.033533664012	318.033533664012\\
70.625	0.23208	331.26011680959	331.26011680959\\
70.625	0.23574	344.793363756019	344.793363756019\\
70.625	0.2394	358.633274503297	358.633274503297\\
70.625	0.24306	372.779849051425	372.779849051425\\
70.625	0.24672	387.233087400403	387.233087400403\\
70.625	0.25038	401.992989550232	401.992989550232\\
70.625	0.25404	417.05955550091	417.05955550091\\
70.625	0.2577	432.432785252438	432.432785252438\\
70.625	0.26136	448.112678804817	448.112678804817\\
70.625	0.26502	464.099236158045	464.099236158045\\
70.625	0.26868	480.392457312123	480.392457312123\\
70.625	0.27234	496.992342267052	496.992342267052\\
70.625	0.276	513.89889102283	513.89889102283\\
71	0.093	44.9030480384369	44.9030480384369\\
71	0.09666	46.7777244247793	46.7777244247793\\
71	0.10032	48.9590646119717	48.9590646119717\\
71	0.10398	51.4470686000141	51.4470686000141\\
71	0.10764	54.2417363889065	54.2417363889065\\
71	0.1113	57.343067978649	57.343067978649\\
71	0.11496	60.7510633692414	60.7510633692414\\
71	0.11862	64.4657225606839	64.4657225606839\\
71	0.12228	68.4870455529763	68.4870455529763\\
71	0.12594	72.8150323461188	72.8150323461188\\
71	0.1296	77.4496829401112	77.4496829401112\\
71	0.13326	82.3909973349537	82.3909973349537\\
71	0.13692	87.6389755306462	87.6389755306462\\
71	0.14058	93.1936175271888	93.1936175271888\\
71	0.14424	99.0549233245813	99.0549233245813\\
71	0.1479	105.222892922824	105.222892922824\\
71	0.15156	111.697526321916	111.697526321916\\
71	0.15522	118.478823521859	118.478823521859\\
71	0.15888	125.566784522652	125.566784522652\\
71	0.16254	132.961409324294	132.961409324294\\
71	0.1662	140.662697926787	140.662697926787\\
71	0.16986	148.670650330129	148.670650330129\\
71	0.17352	156.985266534322	156.985266534322\\
71	0.17718	165.606546539364	165.606546539364\\
71	0.18084	174.534490345257	174.534490345257\\
71	0.1845	183.769097952	183.769097952\\
71	0.18816	193.310369359592	193.310369359592\\
71	0.19182	203.158304568035	203.158304568035\\
71	0.19548	213.312903577327	213.312903577327\\
71	0.19914	223.77416638747	223.77416638747\\
71	0.2028	234.542092998463	234.542092998463\\
71	0.20646	245.616683410305	245.616683410305\\
71	0.21012	256.997937622998	256.997937622998\\
71	0.21378	268.685855636541	268.685855636541\\
71	0.21744	280.680437450933	280.680437450933\\
71	0.2211	292.981683066176	292.981683066176\\
71	0.22476	305.589592482269	305.589592482269\\
71	0.22842	318.504165699212	318.504165699212\\
71	0.23208	331.725402717004	331.725402717004\\
71	0.23574	345.253303535647	345.253303535647\\
71	0.2394	359.08786815514	359.08786815514\\
71	0.24306	373.229096575483	373.229096575483\\
71	0.24672	387.676988796675	387.676988796675\\
71	0.25038	402.431544818718	402.431544818718\\
71	0.25404	417.492764641611	417.492764641611\\
71	0.2577	432.860648265354	432.860648265354\\
71	0.26136	448.535195689946	448.535195689946\\
71	0.26502	464.516406915389	464.516406915389\\
71	0.26868	480.804281941682	480.804281941682\\
71	0.27234	497.398820768825	497.398820768825\\
71	0.276	514.300023396817	514.300023396817\\
71.375	0.093	45.5824442325097	45.5824442325097\\
71.375	0.09666	47.4517744910666	47.4517744910666\\
71.375	0.10032	49.6277685504735	49.6277685504735\\
71.375	0.10398	52.1104264107304	52.1104264107304\\
71.375	0.10764	54.8997480718374	54.8997480718374\\
71.375	0.1113	57.9957335337942	57.9957335337942\\
71.375	0.11496	61.3983827966012	61.3983827966012\\
71.375	0.11862	65.1076958602582	65.1076958602582\\
71.375	0.12228	69.1236727247652	69.1236727247652\\
71.375	0.12594	73.4463133901221	73.4463133901221\\
71.375	0.1296	78.075617856329	78.075617856329\\
71.375	0.13326	83.011586123386	83.011586123386\\
71.375	0.13692	88.254218191293	88.254218191293\\
71.375	0.14058	93.8035140600501	93.8035140600501\\
71.375	0.14424	99.6594737296571	99.6594737296571\\
71.375	0.1479	105.822097200114	105.822097200114\\
71.375	0.15156	112.291384471421	112.291384471421\\
71.375	0.15522	119.067335543578	119.067335543578\\
71.375	0.15888	126.149950416585	126.149950416585\\
71.375	0.16254	133.539229090442	133.539229090442\\
71.375	0.1662	141.235171565149	141.235171565149\\
71.375	0.16986	149.237777840706	149.237777840706\\
71.375	0.17352	157.547047917113	157.547047917113\\
71.375	0.17718	166.162981794371	166.162981794371\\
71.375	0.18084	175.085579472478	175.085579472478\\
71.375	0.1845	184.314840951435	184.314840951435\\
71.375	0.18816	193.850766231242	193.850766231242\\
71.375	0.19182	203.693355311899	203.693355311899\\
71.375	0.19548	213.842608193406	213.842608193406\\
71.375	0.19914	224.298524875763	224.298524875763\\
71.375	0.2028	235.06110535897	235.06110535897\\
71.375	0.20646	246.130349643028	246.130349643028\\
71.375	0.21012	257.506257727935	257.506257727935\\
71.375	0.21378	269.188829613692	269.188829613692\\
71.375	0.21744	281.178065300299	281.178065300299\\
71.375	0.2211	293.473964787756	293.473964787756\\
71.375	0.22476	306.076528076063	306.076528076063\\
71.375	0.22842	318.985755165221	318.985755165221\\
71.375	0.23208	332.201646055228	332.201646055228\\
71.375	0.23574	345.724200746085	345.724200746085\\
71.375	0.2394	359.553419237792	359.553419237792\\
71.375	0.24306	373.68930153035	373.68930153035\\
71.375	0.24672	388.131847623757	388.131847623757\\
71.375	0.25038	402.881057518014	402.881057518014\\
71.375	0.25404	417.936931213121	417.936931213121\\
71.375	0.2577	433.299468709079	433.299468709079\\
71.375	0.26136	448.968670005886	448.968670005886\\
71.375	0.26502	464.944535103543	464.944535103543\\
71.375	0.26868	481.22706400205	481.22706400205\\
71.375	0.27234	497.816256701408	497.816256701408\\
71.375	0.276	514.712113201615	514.712113201615\\
71.75	0.093	46.2727978573924	46.2727978573924\\
71.75	0.09666	48.1367819881638	48.1367819881638\\
71.75	0.10032	50.3074299197852	50.3074299197852\\
71.75	0.10398	52.7847416522566	52.7847416522566\\
71.75	0.10764	55.568717185578	55.568717185578\\
71.75	0.1113	58.6593565197495	58.6593565197495\\
71.75	0.11496	62.0566596547709	62.0566596547709\\
71.75	0.11862	65.7606265906423	65.7606265906423\\
71.75	0.12228	69.7712573273638	69.7712573273638\\
71.75	0.12594	74.0885518649353	74.0885518649353\\
71.75	0.1296	78.7125102033567	78.7125102033567\\
71.75	0.13326	83.6431323426281	83.6431323426281\\
71.75	0.13692	88.8804182827496	88.8804182827496\\
71.75	0.14058	94.4243680237212	94.4243680237212\\
71.75	0.14424	100.274981565543	100.274981565543\\
71.75	0.1479	106.432258908214	106.432258908214\\
71.75	0.15156	112.896200051736	112.896200051736\\
71.75	0.15522	119.666804996107	119.666804996107\\
71.75	0.15888	126.744073741329	126.744073741329\\
71.75	0.16254	134.1280062874	134.1280062874\\
71.75	0.1662	141.818602634322	141.818602634322\\
71.75	0.16986	149.815862782093	149.815862782093\\
71.75	0.17352	158.119786730715	158.119786730715\\
71.75	0.17718	166.730374480187	166.730374480187\\
71.75	0.18084	175.647626030508	175.647626030508\\
71.75	0.1845	184.87154138168	184.87154138168\\
71.75	0.18816	194.402120533701	194.402120533701\\
71.75	0.19182	204.239363486573	204.239363486573\\
71.75	0.19548	214.383270240295	214.383270240295\\
71.75	0.19914	224.833840794866	224.833840794866\\
71.75	0.2028	235.591075150288	235.591075150288\\
71.75	0.20646	246.65497330656	246.65497330656\\
71.75	0.21012	258.025535263681	258.025535263681\\
71.75	0.21378	269.702761021653	269.702761021653\\
71.75	0.21744	281.686650580475	281.686650580475\\
71.75	0.2211	293.977203940146	293.977203940146\\
71.75	0.22476	306.574421100668	306.574421100668\\
71.75	0.22842	319.47830206204	319.47830206204\\
71.75	0.23208	332.688846824261	332.688846824261\\
71.75	0.23574	346.206055387333	346.206055387333\\
71.75	0.2394	360.029927751255	360.029927751255\\
71.75	0.24306	374.160463916027	374.160463916027\\
71.75	0.24672	388.597663881648	388.597663881648\\
71.75	0.25038	403.34152764812	403.34152764812\\
71.75	0.25404	418.392055215442	418.392055215442\\
71.75	0.2577	433.749246583614	433.749246583614\\
71.75	0.26136	449.413101752635	449.413101752635\\
71.75	0.26502	465.383620722507	465.383620722507\\
71.75	0.26868	481.660803493229	481.660803493229\\
71.75	0.27234	498.244650064801	498.244650064801\\
71.75	0.276	515.135160437222	515.135160437222\\
72.125	0.093	46.9741089130851	46.9741089130851\\
72.125	0.09666	48.8327469160709	48.8327469160709\\
72.125	0.10032	50.9980487199068	50.9980487199068\\
72.125	0.10398	53.4700143245928	53.4700143245928\\
72.125	0.10764	56.2486437301286	56.2486437301286\\
72.125	0.1113	59.3339369365145	59.3339369365145\\
72.125	0.11496	62.7258939437505	62.7258939437505\\
72.125	0.11862	66.4245147518365	66.4245147518365\\
72.125	0.12228	70.4297993607724	70.4297993607724\\
72.125	0.12594	74.7417477705583	74.7417477705583\\
72.125	0.1296	79.3603599811943	79.3603599811943\\
72.125	0.13326	84.2856359926802	84.2856359926802\\
72.125	0.13692	89.5175758050162	89.5175758050162\\
72.125	0.14058	95.0561794182023	95.0561794182023\\
72.125	0.14424	100.901446832238	100.901446832238\\
72.125	0.1479	107.053378047124	107.053378047124\\
72.125	0.15156	113.51197306286	113.51197306286\\
72.125	0.15522	120.277231879446	120.277231879446\\
72.125	0.15888	127.349154496882	127.349154496882\\
72.125	0.16254	134.727740915168	134.727740915168\\
72.125	0.1662	142.412991134304	142.412991134304\\
72.125	0.16986	150.404905154291	150.404905154291\\
72.125	0.17352	158.703482975127	158.703482975127\\
72.125	0.17718	167.308724596813	167.308724596813\\
72.125	0.18084	176.220630019349	176.220630019349\\
72.125	0.1845	185.439199242735	185.439199242735\\
72.125	0.18816	194.964432266971	194.964432266971\\
72.125	0.19182	204.796329092057	204.796329092057\\
72.125	0.19548	214.934889717993	214.934889717993\\
72.125	0.19914	225.380114144779	225.380114144779\\
72.125	0.2028	236.132002372416	236.132002372416\\
72.125	0.20646	247.190554400902	247.190554400902\\
72.125	0.21012	258.555770230238	258.555770230238\\
72.125	0.21378	270.227649860424	270.227649860424\\
72.125	0.21744	282.20619329146	282.20619329146\\
72.125	0.2211	294.491400523346	294.491400523346\\
72.125	0.22476	307.083271556082	307.083271556082\\
72.125	0.22842	319.981806389669	319.981806389669\\
72.125	0.23208	333.187005024105	333.187005024105\\
72.125	0.23574	346.698867459391	346.698867459391\\
72.125	0.2394	360.517393695527	360.517393695527\\
72.125	0.24306	374.642583732514	374.642583732514\\
72.125	0.24672	389.07443757035	389.07443757035\\
72.125	0.25038	403.812955209036	403.812955209036\\
72.125	0.25404	418.858136648572	418.858136648572\\
72.125	0.2577	434.209981888959	434.209981888959\\
72.125	0.26136	449.868490930195	449.868490930195\\
72.125	0.26502	465.833663772281	465.833663772281\\
72.125	0.26868	482.105500415217	482.105500415217\\
72.125	0.27234	498.684000859004	498.684000859004\\
72.125	0.276	515.56916510364	515.56916510364\\
72.5	0.093	47.6863773995876	47.6863773995876\\
72.5	0.09666	49.539669274788	49.539669274788\\
72.5	0.10032	51.6996249508384	51.6996249508384\\
72.5	0.10398	54.1662444277388	54.1662444277388\\
72.5	0.10764	56.9395277054892	56.9395277054892\\
72.5	0.1113	60.0194747840896	60.0194747840896\\
72.5	0.11496	63.40608566354	63.40608566354\\
72.5	0.11862	67.0993603438405	67.0993603438405\\
72.5	0.12228	71.0992988249909	71.0992988249909\\
72.5	0.12594	75.4059011069914	75.4059011069914\\
72.5	0.1296	80.0191671898419	80.0191671898419\\
72.5	0.13326	84.9390970735423	84.9390970735423\\
72.5	0.13692	90.1656907580927	90.1656907580927\\
72.5	0.14058	95.6989482434933	95.6989482434933\\
72.5	0.14424	101.538869529744	101.538869529744\\
72.5	0.1479	107.685454616844	107.685454616844\\
72.5	0.15156	114.138703504795	114.138703504795\\
72.5	0.15522	120.898616193595	120.898616193595\\
72.5	0.15888	127.965192683246	127.965192683246\\
72.5	0.16254	135.338432973746	135.338432973746\\
72.5	0.1662	143.018337065097	143.018337065097\\
72.5	0.16986	151.004904957297	151.004904957297\\
72.5	0.17352	159.298136650348	159.298136650348\\
72.5	0.17718	167.898032144249	167.898032144249\\
72.5	0.18084	176.804591438999	176.804591438999\\
72.5	0.1845	186.0178145346	186.0178145346\\
72.5	0.18816	195.53770143105	195.53770143105\\
72.5	0.19182	205.364252128351	205.364252128351\\
72.5	0.19548	215.497466626502	215.497466626502\\
72.5	0.19914	225.937344925502	225.937344925502\\
72.5	0.2028	236.683887025353	236.683887025353\\
72.5	0.20646	247.737092926054	247.737092926054\\
72.5	0.21012	259.096962627604	259.096962627604\\
72.5	0.21378	270.763496130005	270.763496130005\\
72.5	0.21744	282.736693433255	282.736693433255\\
72.5	0.2211	295.016554537356	295.016554537356\\
72.5	0.22476	307.603079442307	307.603079442307\\
72.5	0.22842	320.496268148107	320.496268148107\\
72.5	0.23208	333.696120654758	333.696120654758\\
72.5	0.23574	347.202636962259	347.202636962259\\
72.5	0.2394	361.01581707061	361.01581707061\\
72.5	0.24306	375.13566097981	375.13566097981\\
72.5	0.24672	389.562168689861	389.562168689861\\
72.5	0.25038	404.295340200762	404.295340200762\\
72.5	0.25404	419.335175512513	419.335175512513\\
72.5	0.2577	434.681674625113	434.681674625113\\
72.5	0.26136	450.334837538564	450.334837538564\\
72.5	0.26502	466.294664252865	466.294664252865\\
72.5	0.26868	482.561154768016	482.561154768016\\
72.5	0.27234	499.134309084016	499.134309084016\\
72.5	0.276	516.014127200867	516.014127200867\\
72.875	0.093	48.4096033169001	48.4096033169001\\
72.875	0.09666	50.257549064315	50.257549064315\\
72.875	0.10032	52.4121586125799	52.4121586125799\\
72.875	0.10398	54.8734319616947	54.8734319616947\\
72.875	0.10764	57.6413691116597	57.6413691116597\\
72.875	0.1113	60.7159700624746	60.7159700624746\\
72.875	0.11496	64.0972348141395	64.0972348141395\\
72.875	0.11862	67.7851633666544	67.7851633666544\\
72.875	0.12228	71.7797557200194	71.7797557200194\\
72.875	0.12594	76.0810118742344	76.0810118742344\\
72.875	0.1296	80.6889318292993	80.6889318292993\\
72.875	0.13326	85.6035155852142	85.6035155852142\\
72.875	0.13692	90.8247631419792	90.8247631419792\\
72.875	0.14058	96.3526744995942	96.3526744995942\\
72.875	0.14424	102.187249658059	102.187249658059\\
72.875	0.1479	108.328488617374	108.328488617374\\
72.875	0.15156	114.776391377539	114.776391377539\\
72.875	0.15522	121.530957938554	121.530957938554\\
72.875	0.15888	128.592188300419	128.592188300419\\
72.875	0.16254	135.960082463134	135.960082463134\\
72.875	0.1662	143.634640426699	143.634640426699\\
72.875	0.16986	151.615862191114	151.615862191114\\
72.875	0.17352	159.903747756379	159.903747756379\\
72.875	0.17718	168.498297122494	168.498297122494\\
72.875	0.18084	177.399510289459	177.399510289459\\
72.875	0.1845	186.607387257275	186.607387257275\\
72.875	0.18816	196.12192802594	196.12192802594\\
72.875	0.19182	205.943132595455	205.943132595455\\
72.875	0.19548	216.07100096582	216.07100096582\\
72.875	0.19914	226.505533137035	226.505533137035\\
72.875	0.2028	237.2467291091	237.2467291091\\
72.875	0.20646	248.294588882015	248.294588882015\\
72.875	0.21012	259.64911245578	259.64911245578\\
72.875	0.21378	271.310299830396	271.310299830396\\
72.875	0.21744	283.278151005861	283.278151005861\\
72.875	0.2211	295.552665982176	295.552665982176\\
72.875	0.22476	308.133844759341	308.133844759341\\
72.875	0.22842	321.021687337356	321.021687337356\\
72.875	0.23208	334.216193716221	334.216193716221\\
72.875	0.23574	347.717363895937	347.717363895937\\
72.875	0.2394	361.525197876502	361.525197876502\\
72.875	0.24306	375.639695657917	375.639695657917\\
72.875	0.24672	390.060857240182	390.060857240182\\
72.875	0.25038	404.788682623298	404.788682623298\\
72.875	0.25404	419.823171807263	419.823171807263\\
72.875	0.2577	435.164324792078	435.164324792078\\
72.875	0.26136	450.812141577743	450.812141577743\\
72.875	0.26502	466.766622164259	466.766622164259\\
72.875	0.26868	483.027766551624	483.027766551624\\
72.875	0.27234	499.595574739839	499.595574739839\\
72.875	0.276	516.470046728904	516.470046728904\\
73.25	0.093	49.1437866650225	49.1437866650225\\
73.25	0.09666	50.9863862846518	50.9863862846518\\
73.25	0.10032	53.1356497051312	53.1356497051312\\
73.25	0.10398	55.5915769264606	55.5915769264606\\
73.25	0.10764	58.35416794864	58.35416794864\\
73.25	0.1113	61.4234227716694	61.4234227716694\\
73.25	0.11496	64.7993413955488	64.7993413955488\\
73.25	0.11862	68.4819238202782	68.4819238202782\\
73.25	0.12228	72.4711700458577	72.4711700458577\\
73.25	0.12594	76.7670800722871	76.7670800722871\\
73.25	0.1296	81.3696538995666	81.3696538995666\\
73.25	0.13326	86.278891527696	86.278891527696\\
73.25	0.13692	91.4947929566754	91.4947929566754\\
73.25	0.14058	97.017358186505	97.017358186505\\
73.25	0.14424	102.846587217184	102.846587217184\\
73.25	0.1479	108.982480048714	108.982480048714\\
73.25	0.15156	115.425036681093	115.425036681093\\
73.25	0.15522	122.174257114323	122.174257114323\\
73.25	0.15888	129.230141348403	129.230141348403\\
73.25	0.16254	136.592689383332	136.592689383332\\
73.25	0.1662	144.261901219112	144.261901219112\\
73.25	0.16986	152.237776855741	152.237776855741\\
73.25	0.17352	160.520316293221	160.520316293221\\
73.25	0.17718	169.10951953155	169.10951953155\\
73.25	0.18084	178.00538657073	178.00538657073\\
73.25	0.1845	187.207917410759	187.207917410759\\
73.25	0.18816	196.717112051639	196.717112051639\\
73.25	0.19182	206.532970493369	206.532970493369\\
73.25	0.19548	216.655492735948	216.655492735948\\
73.25	0.19914	227.084678779378	227.084678779378\\
73.25	0.2028	237.820528623658	237.820528623658\\
73.25	0.20646	248.863042268787	248.863042268787\\
73.25	0.21012	260.212219714767	260.212219714767\\
73.25	0.21378	271.868060961596	271.868060961596\\
73.25	0.21744	283.830566009276	283.830566009276\\
73.25	0.2211	296.099734857806	296.099734857806\\
73.25	0.22476	308.675567507185	308.675567507185\\
73.25	0.22842	321.558063957415	321.558063957415\\
73.25	0.23208	334.747224208495	334.747224208495\\
73.25	0.23574	348.243048260424	348.243048260424\\
73.25	0.2394	362.045536113204	362.045536113204\\
73.25	0.24306	376.154687766834	376.154687766834\\
73.25	0.24672	390.570503221314	390.570503221314\\
73.25	0.25038	405.292982476643	405.292982476643\\
73.25	0.25404	420.322125532823	420.322125532823\\
73.25	0.2577	435.657932389853	435.657932389853\\
73.25	0.26136	451.300403047733	451.300403047733\\
73.25	0.26502	467.249537506462	467.249537506462\\
73.25	0.26868	483.505335766042	483.505335766042\\
73.25	0.27234	500.067797826472	500.067797826472\\
73.25	0.276	516.936923687751	516.936923687751\\
73.625	0.093	49.8889274439548	49.8889274439548\\
73.625	0.09666	51.7261809357987	51.7261809357987\\
73.625	0.10032	53.8700982284926	53.8700982284926\\
73.625	0.10398	56.3206793220364	56.3206793220364\\
73.625	0.10764	59.0779242164303	59.0779242164303\\
73.625	0.1113	62.1418329116742	62.1418329116742\\
73.625	0.11496	65.5124054077681	65.5124054077681\\
73.625	0.11862	69.1896417047121	69.1896417047121\\
73.625	0.12228	73.173541802506	73.173541802506\\
73.625	0.12594	77.4641057011499	77.4641057011499\\
73.625	0.1296	82.0613334006439	82.0613334006439\\
73.625	0.13326	86.9652249009878	86.9652249009878\\
73.625	0.13692	92.1757802021817	92.1757802021817\\
73.625	0.14058	97.6929993042258	97.6929993042258\\
73.625	0.14424	103.51688220712	103.51688220712\\
73.625	0.1479	109.647428910864	109.647428910864\\
73.625	0.15156	116.084639415458	116.084639415458\\
73.625	0.15522	122.828513720902	122.828513720902\\
73.625	0.15888	129.879051827196	129.879051827196\\
73.625	0.16254	137.23625373434	137.23625373434\\
73.625	0.1662	144.900119442334	144.900119442334\\
73.625	0.16986	152.870648951178	152.870648951178\\
73.625	0.17352	161.147842260872	161.147842260872\\
73.625	0.17718	169.731699371416	169.731699371416\\
73.625	0.18084	178.62222028281	178.62222028281\\
73.625	0.1845	187.819404995054	187.819404995054\\
73.625	0.18816	197.323253508148	197.323253508148\\
73.625	0.19182	207.133765822092	207.133765822092\\
73.625	0.19548	217.250941936886	217.250941936886\\
73.625	0.19914	227.67478185253	227.67478185253\\
73.625	0.2028	238.405285569025	238.405285569025\\
73.625	0.20646	249.442453086369	249.442453086369\\
73.625	0.21012	260.786284404563	260.786284404563\\
73.625	0.21378	272.436779523607	272.436779523607\\
73.625	0.21744	284.393938443501	284.393938443501\\
73.625	0.2211	296.657761164245	296.657761164245\\
73.625	0.22476	309.22824768584	309.22824768584\\
73.625	0.22842	322.105398008284	322.105398008284\\
73.625	0.23208	335.289212131578	335.289212131578\\
73.625	0.23574	348.779690055722	348.779690055722\\
73.625	0.2394	362.576831780716	362.576831780716\\
73.625	0.24306	376.68063730656	376.68063730656\\
73.625	0.24672	391.091106633255	391.091106633255\\
73.625	0.25038	405.808239760799	405.808239760799\\
73.625	0.25404	420.832036689193	420.832036689193\\
73.625	0.2577	436.162497418437	436.162497418437\\
73.625	0.26136	451.799621948532	451.799621948532\\
73.625	0.26502	467.743410279476	467.743410279476\\
73.625	0.26868	483.99386241127	483.99386241127\\
73.625	0.27234	500.550978343914	500.550978343914\\
73.625	0.276	517.414758077409	517.414758077409\\
74	0.093	50.645025653697	50.645025653697\\
74	0.09666	52.4769330177554	52.4769330177554\\
74	0.10032	54.6155041826638	54.6155041826638\\
74	0.10398	57.0607391484222	57.0607391484222\\
74	0.10764	59.8126379150305	59.8126379150305\\
74	0.1113	62.8712004824889	62.8712004824889\\
74	0.11496	66.2364268507974	66.2364268507974\\
74	0.11862	69.9083170199557	69.9083170199557\\
74	0.12228	73.8868709899643	73.8868709899643\\
74	0.12594	78.1720887608226	78.1720887608226\\
74	0.1296	82.7639703325311	82.7639703325311\\
74	0.13326	87.6625157050895	87.6625157050895\\
74	0.13692	92.867724878498	92.867724878498\\
74	0.14058	98.3795978527564	98.3795978527564\\
74	0.14424	104.198134627865	104.198134627865\\
74	0.1479	110.323335203823	110.323335203823\\
74	0.15156	116.755199580632	116.755199580632\\
74	0.15522	123.49372775829	123.49372775829\\
74	0.15888	130.538919736799	130.538919736799\\
74	0.16254	137.890775516157	137.890775516157\\
74	0.1662	145.549295096366	145.549295096366\\
74	0.16986	153.514478477424	153.514478477424\\
74	0.17352	161.786325659333	161.786325659333\\
74	0.17718	170.364836642092	170.364836642092\\
74	0.18084	179.2500114257	179.2500114257\\
74	0.1845	188.441850010159	188.441850010159\\
74	0.18816	197.940352395467	197.940352395467\\
74	0.19182	207.745518581626	207.745518581626\\
74	0.19548	217.857348568634	217.857348568634\\
74	0.19914	228.275842356493	228.275842356493\\
74	0.2028	239.000999945202	239.000999945202\\
74	0.20646	250.03282133476	250.03282133476\\
74	0.21012	261.371306525169	261.371306525169\\
74	0.21378	273.016455516428	273.016455516428\\
74	0.21744	284.968268308536	284.968268308536\\
74	0.2211	297.226744901495	297.226744901495\\
74	0.22476	309.791885295303	309.791885295303\\
74	0.22842	322.663689489962	322.663689489962\\
74	0.23208	335.842157485471	335.842157485471\\
74	0.23574	349.32728928183	349.32728928183\\
74	0.2394	363.119084879038	363.119084879038\\
74	0.24306	377.217544277097	377.217544277097\\
74	0.24672	391.622667476006	391.622667476006\\
74	0.25038	406.334454475764	406.334454475764\\
74	0.25404	421.352905276373	421.352905276373\\
74	0.2577	436.678019877832	436.678019877832\\
74	0.26136	452.309798280141	452.309798280141\\
74	0.26502	468.248240483299	468.248240483299\\
74	0.26868	484.493346487308	484.493346487308\\
74	0.27234	501.045116292167	501.045116292167\\
74	0.276	517.903549897876	517.903549897876\\
};
\end{axis}

\begin{axis}[%
width=6.159cm,
height=3.097cm,
at={(0cm,8.602cm)},
scale only axis,
xmin=56,
xmax=74,
tick align=outside,
xlabel style={font=\color{white!15!black}},
xlabel={$L_{cut}$},
ymin=0.093,
ymax=0.276,
ylabel style={font=\color{white!15!black}},
ylabel={$D_{rlx}$},
zmin=-0.468524067538196,
zmax=12.0724352898114,
zlabel style={font=\color{white!15!black}},
zlabel={$u(t-3)u(t)$},
view={-140}{50},
axis background/.style={fill=white},
xmajorgrids,
ymajorgrids,
zmajorgrids
]
\addplot3[only marks, mark=*, mark options={}, mark size=1.5000pt, color=mycolor1, fill=mycolor1] table[row sep=crcr]{%
x	y	z\\
74	0.123	0.616147082677301\\
72	0.113	0.756244217181634\\
61	0.095	0.315136730386088\\
56	0.093	0.479225561787365\\
};
\addplot3[only marks, mark=*, mark options={}, mark size=1.5000pt, color=mycolor2, fill=mycolor2] table[row sep=crcr]{%
x	y	z\\
67	0.276	8.54895561081077\\
66	0.255	7.8265872893112\\
62	0.209	2.65659919911432\\
57	0.193	1.15263735041756\\
};
\addplot3[only marks, mark=*, mark options={}, mark size=1.5000pt, color=black, fill=black] table[row sep=crcr]{%
x	y	z\\
69	0.104	0.308897870052812\\
};
\addplot3[only marks, mark=*, mark options={}, mark size=1.5000pt, color=black, fill=black] table[row sep=crcr]{%
x	y	z\\
64	0.23	4.75084747148716\\
};

\addplot3[%
surf,
fill opacity=0.7, shader=interp, colormap={mymap}{[1pt] rgb(0pt)=(1,0.905882,0); rgb(1pt)=(1,0.901964,0); rgb(2pt)=(1,0.898051,0); rgb(3pt)=(1,0.894144,0); rgb(4pt)=(1,0.890243,0); rgb(5pt)=(1,0.886349,0); rgb(6pt)=(1,0.88246,0); rgb(7pt)=(1,0.878577,0); rgb(8pt)=(1,0.8747,0); rgb(9pt)=(1,0.870829,0); rgb(10pt)=(1,0.866964,0); rgb(11pt)=(1,0.863106,0); rgb(12pt)=(1,0.859253,0); rgb(13pt)=(1,0.855406,0); rgb(14pt)=(1,0.851566,0); rgb(15pt)=(1,0.847732,0); rgb(16pt)=(1,0.843903,0); rgb(17pt)=(1,0.840081,0); rgb(18pt)=(1,0.836265,0); rgb(19pt)=(1,0.832455,0); rgb(20pt)=(1,0.828652,0); rgb(21pt)=(1,0.824854,0); rgb(22pt)=(1,0.821063,0); rgb(23pt)=(1,0.817278,0); rgb(24pt)=(1,0.8135,0); rgb(25pt)=(1,0.809727,0); rgb(26pt)=(1,0.805961,0); rgb(27pt)=(1,0.8022,0); rgb(28pt)=(1,0.798445,0); rgb(29pt)=(1,0.794696,0); rgb(30pt)=(1,0.790953,0); rgb(31pt)=(1,0.787215,0); rgb(32pt)=(1,0.783484,0); rgb(33pt)=(1,0.779758,0); rgb(34pt)=(1,0.776038,0); rgb(35pt)=(1,0.772324,0); rgb(36pt)=(1,0.768615,0); rgb(37pt)=(1,0.764913,0); rgb(38pt)=(1,0.761217,0); rgb(39pt)=(1,0.757527,0); rgb(40pt)=(1,0.753843,0); rgb(41pt)=(1,0.750165,0); rgb(42pt)=(1,0.746493,0); rgb(43pt)=(1,0.742827,0); rgb(44pt)=(1,0.739167,0); rgb(45pt)=(1,0.735514,0); rgb(46pt)=(1,0.731867,0); rgb(47pt)=(1,0.728226,0); rgb(48pt)=(1,0.724591,0); rgb(49pt)=(1,0.720963,0); rgb(50pt)=(1,0.717341,0); rgb(51pt)=(1,0.713725,0); rgb(52pt)=(0.999994,0.710077,0); rgb(53pt)=(0.999974,0.706363,0); rgb(54pt)=(0.999942,0.702592,0); rgb(55pt)=(0.999898,0.698775,0); rgb(56pt)=(0.999841,0.694921,0); rgb(57pt)=(0.999771,0.691039,0); rgb(58pt)=(0.99969,0.687139,0); rgb(59pt)=(0.999596,0.68323,0); rgb(60pt)=(0.99949,0.679323,0); rgb(61pt)=(0.999372,0.675427,0); rgb(62pt)=(0.999242,0.67155,0); rgb(63pt)=(0.9991,0.667704,0); rgb(64pt)=(0.998946,0.663897,0); rgb(65pt)=(0.998781,0.660138,0); rgb(66pt)=(0.998605,0.656439,0); rgb(67pt)=(0.998416,0.652807,0); rgb(68pt)=(0.998217,0.649253,0); rgb(69pt)=(0.998006,0.645786,0); rgb(70pt)=(0.997785,0.642416,0); rgb(71pt)=(0.997552,0.639152,0); rgb(72pt)=(0.997308,0.636004,0); rgb(73pt)=(0.997053,0.632982,0); rgb(74pt)=(0.996788,0.630095,0); rgb(75pt)=(0.996512,0.627352,0); rgb(76pt)=(0.996226,0.624763,0); rgb(77pt)=(0.995851,0.622329,0); rgb(78pt)=(0.99494,0.619997,0); rgb(79pt)=(0.99345,0.617753,0); rgb(80pt)=(0.991419,0.61559,0); rgb(81pt)=(0.988885,0.613503,0); rgb(82pt)=(0.985886,0.611486,0); rgb(83pt)=(0.98246,0.609532,0); rgb(84pt)=(0.978643,0.607636,0); rgb(85pt)=(0.974475,0.605791,0); rgb(86pt)=(0.969992,0.603992,0); rgb(87pt)=(0.965232,0.602233,0); rgb(88pt)=(0.960233,0.600507,0); rgb(89pt)=(0.955033,0.598808,0); rgb(90pt)=(0.949669,0.59713,0); rgb(91pt)=(0.94418,0.595468,0); rgb(92pt)=(0.938602,0.593815,0); rgb(93pt)=(0.932974,0.592166,0); rgb(94pt)=(0.927333,0.590513,0); rgb(95pt)=(0.921717,0.588852,0); rgb(96pt)=(0.916164,0.587176,0); rgb(97pt)=(0.910711,0.585479,0); rgb(98pt)=(0.905397,0.583755,0); rgb(99pt)=(0.900258,0.581999,0); rgb(100pt)=(0.895333,0.580203,0); rgb(101pt)=(0.890659,0.578362,0); rgb(102pt)=(0.886275,0.576471,0); rgb(103pt)=(0.882047,0.574545,0); rgb(104pt)=(0.877819,0.572608,0); rgb(105pt)=(0.873592,0.57066,0); rgb(106pt)=(0.869366,0.568701,0); rgb(107pt)=(0.865143,0.566733,0); rgb(108pt)=(0.860924,0.564756,0); rgb(109pt)=(0.856708,0.562771,0); rgb(110pt)=(0.852497,0.560778,0); rgb(111pt)=(0.848292,0.558779,0); rgb(112pt)=(0.844092,0.556774,0); rgb(113pt)=(0.8399,0.554763,0); rgb(114pt)=(0.835716,0.552749,0); rgb(115pt)=(0.831541,0.55073,0); rgb(116pt)=(0.827374,0.548709,0); rgb(117pt)=(0.823219,0.546686,0); rgb(118pt)=(0.819074,0.54466,0); rgb(119pt)=(0.81494,0.542635,0); rgb(120pt)=(0.81082,0.540609,0); rgb(121pt)=(0.806712,0.538584,0); rgb(122pt)=(0.802619,0.53656,0); rgb(123pt)=(0.798541,0.534539,0); rgb(124pt)=(0.794478,0.532521,0); rgb(125pt)=(0.790431,0.530506,0); rgb(126pt)=(0.786402,0.528496,0); rgb(127pt)=(0.782391,0.526491,0); rgb(128pt)=(0.77841,0.524489,0); rgb(129pt)=(0.774523,0.522478,0); rgb(130pt)=(0.770731,0.520455,0); rgb(131pt)=(0.767022,0.518424,0); rgb(132pt)=(0.763384,0.516385,0); rgb(133pt)=(0.759804,0.514339,0); rgb(134pt)=(0.756272,0.51229,0); rgb(135pt)=(0.752775,0.510237,0); rgb(136pt)=(0.749302,0.508182,0); rgb(137pt)=(0.74584,0.506128,0); rgb(138pt)=(0.742378,0.504075,0); rgb(139pt)=(0.738904,0.502025,0); rgb(140pt)=(0.735406,0.499979,0); rgb(141pt)=(0.731872,0.49794,0); rgb(142pt)=(0.72829,0.495909,0); rgb(143pt)=(0.724649,0.493887,0); rgb(144pt)=(0.720936,0.491875,0); rgb(145pt)=(0.71714,0.489876,0); rgb(146pt)=(0.713249,0.487891,0); rgb(147pt)=(0.709251,0.485921,0); rgb(148pt)=(0.705134,0.483968,0); rgb(149pt)=(0.700887,0.482033,0); rgb(150pt)=(0.696497,0.480118,0); rgb(151pt)=(0.691952,0.478225,0); rgb(152pt)=(0.687242,0.476355,0); rgb(153pt)=(0.682353,0.47451,0); rgb(154pt)=(0.677195,0.472696,0); rgb(155pt)=(0.6717,0.470916,0); rgb(156pt)=(0.665891,0.469169,0); rgb(157pt)=(0.659791,0.46745,0); rgb(158pt)=(0.653423,0.465756,0); rgb(159pt)=(0.64681,0.464084,0); rgb(160pt)=(0.639976,0.462432,0); rgb(161pt)=(0.632943,0.460795,0); rgb(162pt)=(0.625734,0.459171,0); rgb(163pt)=(0.618373,0.457556,0); rgb(164pt)=(0.610882,0.455948,0); rgb(165pt)=(0.603284,0.454343,0); rgb(166pt)=(0.595604,0.452737,0); rgb(167pt)=(0.587863,0.451129,0); rgb(168pt)=(0.580084,0.449514,0); rgb(169pt)=(0.572292,0.447889,0); rgb(170pt)=(0.564508,0.446252,0); rgb(171pt)=(0.556756,0.444599,0); rgb(172pt)=(0.549059,0.442927,0); rgb(173pt)=(0.54144,0.441232,0); rgb(174pt)=(0.533922,0.439512,0); rgb(175pt)=(0.526529,0.437764,0); rgb(176pt)=(0.519282,0.435983,0); rgb(177pt)=(0.512206,0.434168,0); rgb(178pt)=(0.505323,0.432315,0); rgb(179pt)=(0.498628,0.430422,3.92506e-06); rgb(180pt)=(0.491973,0.428504,3.49981e-05); rgb(181pt)=(0.485331,0.426562,9.63073e-05); rgb(182pt)=(0.478704,0.424596,0.000186979); rgb(183pt)=(0.472096,0.422609,0.000306141); rgb(184pt)=(0.465508,0.420599,0.00045292); rgb(185pt)=(0.458942,0.418567,0.000626441); rgb(186pt)=(0.452401,0.416515,0.000825833); rgb(187pt)=(0.445885,0.414441,0.00105022); rgb(188pt)=(0.439399,0.412348,0.00129873); rgb(189pt)=(0.432942,0.410234,0.00157049); rgb(190pt)=(0.426518,0.408102,0.00186463); rgb(191pt)=(0.420129,0.40595,0.00218028); rgb(192pt)=(0.413777,0.40378,0.00251655); rgb(193pt)=(0.407464,0.401592,0.00287258); rgb(194pt)=(0.401191,0.399386,0.00324749); rgb(195pt)=(0.394962,0.397164,0.00364042); rgb(196pt)=(0.388777,0.394925,0.00405048); rgb(197pt)=(0.38264,0.39267,0.00447681); rgb(198pt)=(0.376552,0.390399,0.00491852); rgb(199pt)=(0.370516,0.388113,0.00537476); rgb(200pt)=(0.364532,0.385812,0.00584464); rgb(201pt)=(0.358605,0.383497,0.00632729); rgb(202pt)=(0.352735,0.381168,0.00682184); rgb(203pt)=(0.346925,0.378826,0.00732741); rgb(204pt)=(0.341176,0.376471,0.00784314); rgb(205pt)=(0.335485,0.374093,0.00847245); rgb(206pt)=(0.329843,0.371682,0.00930909); rgb(207pt)=(0.324249,0.369242,0.0103377); rgb(208pt)=(0.318701,0.366772,0.0115428); rgb(209pt)=(0.313198,0.364275,0.0129091); rgb(210pt)=(0.307739,0.361753,0.0144211); rgb(211pt)=(0.302322,0.359206,0.0160634); rgb(212pt)=(0.296945,0.356637,0.0178207); rgb(213pt)=(0.291607,0.354048,0.0196776); rgb(214pt)=(0.286307,0.35144,0.0216186); rgb(215pt)=(0.281043,0.348814,0.0236284); rgb(216pt)=(0.275813,0.346172,0.0256916); rgb(217pt)=(0.270616,0.343517,0.0277927); rgb(218pt)=(0.265451,0.340849,0.0299163); rgb(219pt)=(0.260317,0.33817,0.0320472); rgb(220pt)=(0.25521,0.335482,0.0341698); rgb(221pt)=(0.250131,0.332786,0.0362688); rgb(222pt)=(0.245078,0.330085,0.0383287); rgb(223pt)=(0.240048,0.327379,0.0403343); rgb(224pt)=(0.235042,0.324671,0.04227); rgb(225pt)=(0.230056,0.321962,0.0441205); rgb(226pt)=(0.22509,0.319254,0.0458704); rgb(227pt)=(0.220142,0.316548,0.0475043); rgb(228pt)=(0.215212,0.313846,0.0490067); rgb(229pt)=(0.210296,0.311149,0.0503624); rgb(230pt)=(0.205395,0.308459,0.0515759); rgb(231pt)=(0.200514,0.305763,0.052757); rgb(232pt)=(0.195655,0.303061,0.0539242); rgb(233pt)=(0.190817,0.300353,0.0550763); rgb(234pt)=(0.186001,0.297639,0.0562123); rgb(235pt)=(0.181207,0.294918,0.0573313); rgb(236pt)=(0.176434,0.292191,0.0584321); rgb(237pt)=(0.171685,0.289458,0.0595136); rgb(238pt)=(0.166957,0.286719,0.060575); rgb(239pt)=(0.162252,0.283973,0.0616151); rgb(240pt)=(0.15757,0.281221,0.0626328); rgb(241pt)=(0.152911,0.278463,0.0636271); rgb(242pt)=(0.148275,0.275699,0.0645971); rgb(243pt)=(0.143663,0.272929,0.0655416); rgb(244pt)=(0.139074,0.270152,0.0664596); rgb(245pt)=(0.134508,0.26737,0.06735); rgb(246pt)=(0.129967,0.264581,0.0682118); rgb(247pt)=(0.125449,0.261787,0.0690441); rgb(248pt)=(0.120956,0.258986,0.0698456); rgb(249pt)=(0.116487,0.25618,0.0706154); rgb(250pt)=(0.112043,0.253367,0.0713525); rgb(251pt)=(0.107623,0.250549,0.0720557); rgb(252pt)=(0.103229,0.247724,0.0727241); rgb(253pt)=(0.0988592,0.244894,0.0733566); rgb(254pt)=(0.0945149,0.242058,0.0739522); rgb(255pt)=(0.0901961,0.239216,0.0745098)}, mesh/rows=49]
table[row sep=crcr, point meta=\thisrow{c}] {%
%
x	y	z	c\\
56	0.093	0.414663957322075	0.414663957322075\\
56	0.09666	0.370128901927017	0.370128901927017\\
56	0.10032	0.329913908900332	0.329913908900332\\
56	0.10398	0.294018978242019	0.294018978242019\\
56	0.10764	0.262444109952074	0.262444109952074\\
56	0.1113	0.235189304030504	0.235189304030504\\
56	0.11496	0.212254560477309	0.212254560477309\\
56	0.11862	0.193639879292478	0.193639879292478\\
56	0.12228	0.179345260476023	0.179345260476023\\
56	0.12594	0.16937070402794	0.16937070402794\\
56	0.1296	0.163716209948227	0.163716209948227\\
56	0.13326	0.162381778236887	0.162381778236887\\
56	0.13692	0.165367408893918	0.165367408893918\\
56	0.14058	0.172673101919322	0.172673101919322\\
56	0.14424	0.184298857313093	0.184298857313093\\
56	0.1479	0.200244675075242	0.200244675075242\\
56	0.15156	0.22051055520576	0.22051055520576\\
56	0.15522	0.245096497704646	0.245096497704646\\
56	0.15888	0.274002502571911	0.274002502571911\\
56	0.16254	0.307228569807537	0.307228569807537\\
56	0.1662	0.344774699411541	0.344774699411541\\
56	0.16986	0.38664089138392	0.38664089138392\\
56	0.17352	0.432827145724668	0.432827145724668\\
56	0.17718	0.483333462433786	0.483333462433786\\
56	0.18084	0.538159841511277	0.538159841511277\\
56	0.1845	0.597306282957137	0.597306282957137\\
56	0.18816	0.660772786771368	0.660772786771368\\
56	0.19182	0.728559352953976	0.728559352953976\\
56	0.19548	0.800665981504953	0.800665981504953\\
56	0.19914	0.877092672424302	0.877092672424302\\
56	0.2028	0.957839425712025	0.957839425712025\\
56	0.20646	1.04290624136812	1.04290624136812\\
56	0.21012	1.13229311939257	1.13229311939257\\
56	0.21378	1.22600005978541	1.22600005978541\\
56	0.21744	1.32402706254661	1.32402706254661\\
56	0.2211	1.4263741276762	1.4263741276762\\
56	0.22476	1.53304125517415	1.53304125517415\\
56	0.22842	1.64402844504047	1.64402844504047\\
56	0.23208	1.75933569727515	1.75933569727515\\
56	0.23574	1.87896301187822	1.87896301187822\\
56	0.2394	2.00291038884967	2.00291038884967\\
56	0.24306	2.13117782818946	2.13117782818946\\
56	0.24672	2.26376532989765	2.26376532989765\\
56	0.25038	2.40067289397421	2.40067289397421\\
56	0.25404	2.54190052041912	2.54190052041912\\
56	0.2577	2.68744820923242	2.68744820923242\\
56	0.26136	2.83731596041408	2.83731596041408\\
56	0.26502	2.99150377396412	2.99150377396412\\
56	0.26868	3.15001164988254	3.15001164988254\\
56	0.27234	3.31283958816931	3.31283958816931\\
56	0.276	3.47998758882447	3.47998758882447\\
56.375	0.093	0.420826920209134	0.420826920209134\\
56.375	0.09666	0.38024004636651	0.38024004636651\\
56.375	0.10032	0.343973234892262	0.343973234892262\\
56.375	0.10398	0.312026485786387	0.312026485786387\\
56.375	0.10764	0.284399799048876	0.284399799048876\\
56.375	0.1113	0.261093174679745	0.261093174679745\\
56.375	0.11496	0.242106612678985	0.242106612678985\\
56.375	0.11862	0.227440113046589	0.227440113046589\\
56.375	0.12228	0.217093675782572	0.217093675782572\\
56.375	0.12594	0.211067300886927	0.211067300886927\\
56.375	0.1296	0.209360988359649	0.209360988359649\\
56.375	0.13326	0.211974738200743	0.211974738200743\\
56.375	0.13692	0.218908550410212	0.218908550410212\\
56.375	0.14058	0.230162424988051	0.230162424988051\\
56.375	0.14424	0.245736361934261	0.245736361934261\\
56.375	0.1479	0.265630361248845	0.265630361248845\\
56.375	0.15156	0.289844422931797	0.289844422931797\\
56.375	0.15522	0.318378546983121	0.318378546983121\\
56.375	0.15888	0.351232733402822	0.351232733402822\\
56.375	0.16254	0.388406982190886	0.388406982190886\\
56.375	0.1662	0.429901293347328	0.429901293347328\\
56.375	0.16986	0.475715666872138	0.475715666872138\\
56.375	0.17352	0.525850102765324	0.525850102765324\\
56.375	0.17718	0.580304601026881	0.580304601026881\\
56.375	0.18084	0.639079161656809	0.639079161656809\\
56.375	0.1845	0.7021737846551	0.7021737846551\\
56.375	0.18816	0.76958847002177	0.76958847002177\\
56.375	0.19182	0.841323217756813	0.841323217756813\\
56.375	0.19548	0.917378027860228	0.917378027860228\\
56.375	0.19914	0.997752900332012	0.997752900332012\\
56.375	0.2028	1.08244783517217	1.08244783517217\\
56.375	0.20646	1.1714628323807	1.1714628323807\\
56.375	0.21012	1.2647978919576	1.2647978919576\\
56.375	0.21378	1.36245301390286	1.36245301390286\\
56.375	0.21744	1.46442819821651	1.46442819821651\\
56.375	0.2211	1.57072344489852	1.57072344489852\\
56.375	0.22476	1.68133875394891	1.68133875394891\\
56.375	0.22842	1.79627412536767	1.79627412536767\\
56.375	0.23208	1.9155295591548	1.9155295591548\\
56.375	0.23574	2.0391050553103	2.0391050553103\\
56.375	0.2394	2.16700061383417	2.16700061383417\\
56.375	0.24306	2.29921623472642	2.29921623472642\\
56.375	0.24672	2.43575191798703	2.43575191798703\\
56.375	0.25038	2.57660766361601	2.57660766361601\\
56.375	0.25404	2.72178347161338	2.72178347161338\\
56.375	0.2577	2.87127934197911	2.87127934197911\\
56.375	0.26136	3.02509527471321	3.02509527471321\\
56.375	0.26502	3.18323126981569	3.18323126981569\\
56.375	0.26868	3.34568732728653	3.34568732728653\\
56.375	0.27234	3.51246344712576	3.51246344712576\\
56.375	0.276	3.68355962933334	3.68355962933334\\
56.75	0.093	0.425944661249165	0.425944661249165\\
56.75	0.09666	0.38930596895898	0.38930596895898\\
56.75	0.10032	0.356987339037167	0.356987339037167\\
56.75	0.10398	0.328988771483726	0.328988771483726\\
56.75	0.10764	0.305310266298657	0.305310266298657\\
56.75	0.1113	0.285951823481957	0.285951823481957\\
56.75	0.11496	0.270913443033635	0.270913443033635\\
56.75	0.11862	0.260195124953677	0.260195124953677\\
56.75	0.12228	0.253796869242095	0.253796869242095\\
56.75	0.12594	0.251718675898885	0.251718675898885\\
56.75	0.1296	0.253960544924042	0.253960544924042\\
56.75	0.13326	0.260522476317578	0.260522476317578\\
56.75	0.13692	0.271404470079482	0.271404470079482\\
56.75	0.14058	0.286606526209755	0.286606526209755\\
56.75	0.14424	0.306128644708399	0.306128644708399\\
56.75	0.1479	0.329970825575422	0.329970825575422\\
56.75	0.15156	0.358133068810812	0.358133068810812\\
56.75	0.15522	0.390615374414571	0.390615374414571\\
56.75	0.15888	0.42741774238671	0.42741774238671\\
56.75	0.16254	0.468540172727208	0.468540172727208\\
56.75	0.1662	0.513982665436085	0.513982665436085\\
56.75	0.16986	0.563745220513334	0.563745220513334\\
56.75	0.17352	0.617827837958954	0.617827837958954\\
56.75	0.17718	0.676230517772949	0.676230517772949\\
56.75	0.18084	0.738953259955313	0.738953259955313\\
56.75	0.1845	0.805996064506042	0.805996064506042\\
56.75	0.18816	0.877358931425146	0.877358931425146\\
56.75	0.19182	0.953041860712627	0.953041860712627\\
56.75	0.19548	1.03304485236848	1.03304485236848\\
56.75	0.19914	1.1173679063927	1.1173679063927\\
56.75	0.2028	1.20601102278529	1.20601102278529\\
56.75	0.20646	1.29897420154626	1.29897420154626\\
56.75	0.21012	1.39625744267559	1.39625744267559\\
56.75	0.21378	1.4978607461733	1.4978607461733\\
56.75	0.21744	1.60378411203938	1.60378411203938\\
56.75	0.2211	1.71402754027383	1.71402754027383\\
56.75	0.22476	1.82859103087665	1.82859103087665\\
56.75	0.22842	1.94747458384785	1.94747458384785\\
56.75	0.23208	2.07067819918741	2.07067819918741\\
56.75	0.23574	2.19820187689535	2.19820187689535\\
56.75	0.2394	2.33004561697165	2.33004561697165\\
56.75	0.24306	2.46620941941634	2.46620941941634\\
56.75	0.24672	2.60669328422939	2.60669328422939\\
56.75	0.25038	2.75149721141082	2.75149721141082\\
56.75	0.25404	2.90062120096061	2.90062120096061\\
56.75	0.2577	3.05406525287878	3.05406525287878\\
56.75	0.26136	3.21182936716532	3.21182936716532\\
56.75	0.26502	3.37391354382022	3.37391354382022\\
56.75	0.26868	3.54031778284351	3.54031778284351\\
56.75	0.27234	3.71104208423516	3.71104208423516\\
56.75	0.276	3.88608644799519	3.88608644799519\\
57.125	0.093	0.430017180442184	0.430017180442184\\
57.125	0.09666	0.397326669704433	0.397326669704433\\
57.125	0.10032	0.368956221335056	0.368956221335056\\
57.125	0.10398	0.344905835334054	0.344905835334054\\
57.125	0.10764	0.32517551170142	0.32517551170142\\
57.125	0.1113	0.309765250437158	0.309765250437158\\
57.125	0.11496	0.29867505154127	0.29867505154127\\
57.125	0.11862	0.291904915013747	0.291904915013747\\
57.125	0.12228	0.289454840854603	0.289454840854603\\
57.125	0.12594	0.291324829063831	0.291324829063831\\
57.125	0.1296	0.297514879641423	0.297514879641423\\
57.125	0.13326	0.308024992587394	0.308024992587394\\
57.125	0.13692	0.322855167901736	0.322855167901736\\
57.125	0.14058	0.342005405584444	0.342005405584444\\
57.125	0.14424	0.365475705635526	0.365475705635526\\
57.125	0.1479	0.393266068054984	0.393266068054984\\
57.125	0.15156	0.425376492842809	0.425376492842809\\
57.125	0.15522	0.461806979999006	0.461806979999006\\
57.125	0.15888	0.502557529523579	0.502557529523579\\
57.125	0.16254	0.547628141416516	0.547628141416516\\
57.125	0.1662	0.597018815677828	0.597018815677828\\
57.125	0.16986	0.650729552307515	0.650729552307515\\
57.125	0.17352	0.708760351305573	0.708760351305573\\
57.125	0.17718	0.771111212671999	0.771111212671999\\
57.125	0.18084	0.837782136406801	0.837782136406801\\
57.125	0.1845	0.908773122509968	0.908773122509968\\
57.125	0.18816	0.984084170981511	0.984084170981511\\
57.125	0.19182	1.06371528182142	1.06371528182142\\
57.125	0.19548	1.14766645502971	1.14766645502971\\
57.125	0.19914	1.23593769060637	1.23593769060637\\
57.125	0.2028	1.3285289885514	1.3285289885514\\
57.125	0.20646	1.4254403488648	1.4254403488648\\
57.125	0.21012	1.52667177154657	1.52667177154657\\
57.125	0.21378	1.63222325659672	1.63222325659672\\
57.125	0.21744	1.74209480401523	1.74209480401523\\
57.125	0.2211	1.85628641380212	1.85628641380212\\
57.125	0.22476	1.97479808595738	1.97479808595738\\
57.125	0.22842	2.09762982048101	2.09762982048101\\
57.125	0.23208	2.224781617373	2.224781617373\\
57.125	0.23574	2.35625347663338	2.35625347663338\\
57.125	0.2394	2.49204539826213	2.49204539826213\\
57.125	0.24306	2.63215738225924	2.63215738225924\\
57.125	0.24672	2.77658942862473	2.77658942862473\\
57.125	0.25038	2.92534153735859	2.92534153735859\\
57.125	0.25404	3.07841370846083	3.07841370846083\\
57.125	0.2577	3.23580594193143	3.23580594193143\\
57.125	0.26136	3.3975182377704	3.3975182377704\\
57.125	0.26502	3.56355059597775	3.56355059597775\\
57.125	0.26868	3.73390301655347	3.73390301655347\\
57.125	0.27234	3.90857549949756	3.90857549949756\\
57.125	0.276	4.08756804481003	4.08756804481003\\
57.5	0.093	0.433044477788178	0.433044477788178\\
57.5	0.09666	0.404302148602864	0.404302148602864\\
57.5	0.10032	0.379879881785923	0.379879881785923\\
57.5	0.10398	0.359777677337356	0.359777677337356\\
57.5	0.10764	0.34399553525716	0.34399553525716\\
57.5	0.1113	0.332533455545333	0.332533455545333\\
57.5	0.11496	0.325391438201883	0.325391438201883\\
57.5	0.11862	0.322569483226799	0.322569483226799\\
57.5	0.12228	0.32406759062009	0.32406759062009\\
57.5	0.12594	0.329885760381752	0.329885760381752\\
57.5	0.1296	0.340023992511782	0.340023992511782\\
57.5	0.13326	0.354482287010188	0.354482287010188\\
57.5	0.13692	0.373260643876964	0.373260643876964\\
57.5	0.14058	0.396359063112115	0.396359063112115\\
57.5	0.14424	0.423777544715632	0.423777544715632\\
57.5	0.1479	0.455516088687523	0.455516088687523\\
57.5	0.15156	0.491574695027787	0.491574695027787\\
57.5	0.15522	0.531953363736418	0.531953363736418\\
57.5	0.15888	0.57665209481343	0.57665209481343\\
57.5	0.16254	0.625670888258802	0.625670888258802\\
57.5	0.1662	0.679009744072552	0.679009744072552\\
57.5	0.16986	0.736668662254673	0.736668662254673\\
57.5	0.17352	0.798647642805166	0.798647642805166\\
57.5	0.17718	0.864946685724031	0.864946685724031\\
57.5	0.18084	0.935565791011271	0.935565791011271\\
57.5	0.1845	1.01050495866687	1.01050495866687\\
57.5	0.18816	1.08976418869085	1.08976418869085\\
57.5	0.19182	1.1733434810832	1.1733434810832\\
57.5	0.19548	1.26124283584393	1.26124283584393\\
57.5	0.19914	1.35346225297302	1.35346225297302\\
57.5	0.2028	1.45000173247048	1.45000173247048\\
57.5	0.20646	1.55086127433633	1.55086127433633\\
57.5	0.21012	1.65604087857053	1.65604087857053\\
57.5	0.21378	1.76554054517311	1.76554054517311\\
57.5	0.21744	1.87936027414406	1.87936027414406\\
57.5	0.2211	1.99750006548339	1.99750006548339\\
57.5	0.22476	2.11995991919108	2.11995991919108\\
57.5	0.22842	2.24673983526715	2.24673983526715\\
57.5	0.23208	2.37783981371159	2.37783981371159\\
57.5	0.23574	2.5132598545244	2.5132598545244\\
57.5	0.2394	2.65299995770558	2.65299995770558\\
57.5	0.24306	2.79706012325514	2.79706012325514\\
57.5	0.24672	2.94544035117305	2.94544035117305\\
57.5	0.25038	3.09814064145935	3.09814064145935\\
57.5	0.25404	3.25516099411403	3.25516099411403\\
57.5	0.2577	3.41650140913706	3.41650140913706\\
57.5	0.26136	3.58216188652847	3.58216188652847\\
57.5	0.26502	3.75214242628826	3.75214242628826\\
57.5	0.26868	3.92644302841641	3.92644302841641\\
57.5	0.27234	4.10506369291294	4.10506369291294\\
57.5	0.276	4.28800441977783	4.28800441977783\\
57.875	0.093	0.435026553287154	0.435026553287154\\
57.875	0.09666	0.410232405654276	0.410232405654276\\
57.875	0.10032	0.389758320389776	0.389758320389776\\
57.875	0.10398	0.373604297493643	0.373604297493643\\
57.875	0.10764	0.361770336965882	0.361770336965882\\
57.875	0.1113	0.354256438806493	0.354256438806493\\
57.875	0.11496	0.351062603015478	0.351062603015478\\
57.875	0.11862	0.352188829592828	0.352188829592828\\
57.875	0.12228	0.357635118538557	0.357635118538557\\
57.875	0.12594	0.367401469852655	0.367401469852655\\
57.875	0.1296	0.381487883535123	0.381487883535123\\
57.875	0.13326	0.399894359585963	0.399894359585963\\
57.875	0.13692	0.422620898005178	0.422620898005178\\
57.875	0.14058	0.449667498792763	0.449667498792763\\
57.875	0.14424	0.481034161948718	0.481034161948718\\
57.875	0.1479	0.516720887473045	0.516720887473045\\
57.875	0.15156	0.556727675365746	0.556727675365746\\
57.875	0.15522	0.601054525626813	0.601054525626813\\
57.875	0.15888	0.649701438256259	0.649701438256259\\
57.875	0.16254	0.702668413254069	0.702668413254069\\
57.875	0.1662	0.759955450620254	0.759955450620254\\
57.875	0.16986	0.821562550354813	0.821562550354813\\
57.875	0.17352	0.887489712457745	0.887489712457745\\
57.875	0.17718	0.957736936929044	0.957736936929044\\
57.875	0.18084	1.03230422376872	1.03230422376872\\
57.875	0.1845	1.11119157297676	1.11119157297676\\
57.875	0.18816	1.19439898455317	1.19439898455317\\
57.875	0.19182	1.28192645849796	1.28192645849796\\
57.875	0.19548	1.37377399481112	1.37377399481112\\
57.875	0.19914	1.46994159349265	1.46994159349265\\
57.875	0.2028	1.57042925454255	1.57042925454255\\
57.875	0.20646	1.67523697796083	1.67523697796083\\
57.875	0.21012	1.78436476374747	1.78436476374747\\
57.875	0.21378	1.89781261190249	1.89781261190249\\
57.875	0.21744	2.01558052242588	2.01558052242588\\
57.875	0.2211	2.13766849531763	2.13766849531763\\
57.875	0.22476	2.26407653057777	2.26407653057777\\
57.875	0.22842	2.39480462820627	2.39480462820627\\
57.875	0.23208	2.52985278820314	2.52985278820314\\
57.875	0.23574	2.66922101056838	2.66922101056838\\
57.875	0.2394	2.81290929530201	2.81290929530201\\
57.875	0.24306	2.960917642404	2.960917642404\\
57.875	0.24672	3.11324605187437	3.11324605187437\\
57.875	0.25038	3.2698945237131	3.2698945237131\\
57.875	0.25404	3.4308630579202	3.4308630579202\\
57.875	0.2577	3.59615165449568	3.59615165449568\\
57.875	0.26136	3.76576031343952	3.76576031343952\\
57.875	0.26502	3.93968903475173	3.93968903475173\\
57.875	0.26868	4.11793781843233	4.11793781843233\\
57.875	0.27234	4.30050666448129	4.30050666448129\\
57.875	0.276	4.48739557289864	4.48739557289864\\
58.25	0.093	0.435963406939106	0.435963406939106\\
58.25	0.09666	0.415117440858665	0.415117440858665\\
58.25	0.10032	0.398591537146599	0.398591537146599\\
58.25	0.10398	0.386385695802904	0.386385695802904\\
58.25	0.10764	0.378499916827578	0.378499916827578\\
58.25	0.1113	0.374934200220624	0.374934200220624\\
58.25	0.11496	0.375688545982047	0.375688545982047\\
58.25	0.11862	0.380762954111836	0.380762954111836\\
58.25	0.12228	0.390157424609999	0.390157424609999\\
58.25	0.12594	0.403871957476535	0.403871957476535\\
58.25	0.1296	0.421906552711438	0.421906552711438\\
58.25	0.13326	0.444261210314716	0.444261210314716\\
58.25	0.13692	0.470935930286366	0.470935930286366\\
58.25	0.14058	0.501930712626386	0.501930712626386\\
58.25	0.14424	0.537245557334776	0.537245557334776\\
58.25	0.1479	0.57688046441154	0.57688046441154\\
58.25	0.15156	0.620835433856677	0.620835433856677\\
58.25	0.15522	0.669110465670181	0.669110465670181\\
58.25	0.15888	0.721705559852066	0.721705559852066\\
58.25	0.16254	0.778620716402311	0.778620716402311\\
58.25	0.1662	0.839855935320934	0.839855935320934\\
58.25	0.16986	0.905411216607928	0.905411216607928\\
58.25	0.17352	0.975286560263294	0.975286560263294\\
58.25	0.17718	1.04948196628703	1.04948196628703\\
58.25	0.18084	1.12799743467914	1.12799743467914\\
58.25	0.1845	1.21083296543962	1.21083296543962\\
58.25	0.18816	1.29798855856847	1.29798855856847\\
58.25	0.19182	1.38946421406569	1.38946421406569\\
58.25	0.19548	1.48525993193129	1.48525993193129\\
58.25	0.19914	1.58537571216525	1.58537571216525\\
58.25	0.2028	1.68981155476759	1.68981155476759\\
58.25	0.20646	1.79856745973831	1.79856745973831\\
58.25	0.21012	1.91164342707738	1.91164342707738\\
58.25	0.21378	2.02903945678483	2.02903945678483\\
58.25	0.21744	2.15075554886066	2.15075554886066\\
58.25	0.2211	2.27679170330486	2.27679170330486\\
58.25	0.22476	2.40714792011742	2.40714792011742\\
58.25	0.22842	2.54182419929837	2.54182419929837\\
58.25	0.23208	2.68082054084767	2.68082054084767\\
58.25	0.23574	2.82413694476536	2.82413694476536\\
58.25	0.2394	2.97177341105141	2.97177341105141\\
58.25	0.24306	3.12372993970584	3.12372993970584\\
58.25	0.24672	3.28000653072863	3.28000653072863\\
58.25	0.25038	3.4406031841198	3.4406031841198\\
58.25	0.25404	3.60551989987935	3.60551989987935\\
58.25	0.2577	3.77475667800726	3.77475667800726\\
58.25	0.26136	3.94831351850354	3.94831351850354\\
58.25	0.26502	4.1261904213682	4.1261904213682\\
58.25	0.26868	4.30838738660123	4.30838738660123\\
58.25	0.27234	4.49490441420263	4.49490441420263\\
58.25	0.276	4.6857415041724	4.6857415041724\\
58.625	0.093	0.435855038744041	0.435855038744041\\
58.625	0.09666	0.418957254216036	0.418957254216036\\
58.625	0.10032	0.406379532056405	0.406379532056405\\
58.625	0.10398	0.398121872265145	0.398121872265145\\
58.625	0.10764	0.394184274842257	0.394184274842257\\
58.625	0.1113	0.394566739787741	0.394566739787741\\
58.625	0.11496	0.3992692671016	0.3992692671016\\
58.625	0.11862	0.408291856783823	0.408291856783823\\
58.625	0.12228	0.421634508834425	0.421634508834425\\
58.625	0.12594	0.439297223253395	0.439297223253395\\
58.625	0.1296	0.461280000040736	0.461280000040736\\
58.625	0.13326	0.487582839196449	0.487582839196449\\
58.625	0.13692	0.518205740720534	0.518205740720534\\
58.625	0.14058	0.553148704612992	0.553148704612992\\
58.625	0.14424	0.59241173087382	0.59241173087382\\
58.625	0.1479	0.635994819503019	0.635994819503019\\
58.625	0.15156	0.683897970500594	0.683897970500594\\
58.625	0.15522	0.736121183866533	0.736121183866533\\
58.625	0.15888	0.792664459600853	0.792664459600853\\
58.625	0.16254	0.853527797703536	0.853527797703536\\
58.625	0.1662	0.918711198174593	0.918711198174593\\
58.625	0.16986	0.988214661014026	0.988214661014026\\
58.625	0.17352	1.06203818622183	1.06203818622183\\
58.625	0.17718	1.140181773798	1.140181773798\\
58.625	0.18084	1.22264542374255	1.22264542374255\\
58.625	0.1845	1.30942913605546	1.30942913605546\\
58.625	0.18816	1.40053291073674	1.40053291073674\\
58.625	0.19182	1.49595674778641	1.49595674778641\\
58.625	0.19548	1.59570064720444	1.59570064720444\\
58.625	0.19914	1.69976460899084	1.69976460899084\\
58.625	0.2028	1.80814863314562	1.80814863314562\\
58.625	0.20646	1.92085271966877	1.92085271966877\\
58.625	0.21012	2.03787686856028	2.03787686856028\\
58.625	0.21378	2.15922107982017	2.15922107982017\\
58.625	0.21744	2.28488535344843	2.28488535344843\\
58.625	0.2211	2.41486968944507	2.41486968944507\\
58.625	0.22476	2.54917408781007	2.54917408781007\\
58.625	0.22842	2.68779854854345	2.68779854854345\\
58.625	0.23208	2.83074307164518	2.83074307164518\\
58.625	0.23574	2.97800765711531	2.97800765711531\\
58.625	0.2394	3.1295923049538	3.1295923049538\\
58.625	0.24306	3.28549701516066	3.28549701516066\\
58.625	0.24672	3.4457217877359	3.4457217877359\\
58.625	0.25038	3.6102666226795	3.6102666226795\\
58.625	0.25404	3.77913151999148	3.77913151999148\\
58.625	0.2577	3.95231647967183	3.95231647967183\\
58.625	0.26136	4.12982150172055	4.12982150172055\\
58.625	0.26502	4.31164658613764	4.31164658613764\\
58.625	0.26868	4.4977917329231	4.4977917329231\\
58.625	0.27234	4.68825694207694	4.68825694207694\\
58.625	0.276	4.88304221359915	4.88304221359915\\
59	0.093	0.434701448701953	0.434701448701953\\
59	0.09666	0.421751845726383	0.421751845726383\\
59	0.10032	0.41312230511919	0.41312230511919\\
59	0.10398	0.408812826880368	0.408812826880368\\
59	0.10764	0.408823411009915	0.408823411009915\\
59	0.1113	0.413154057507837	0.413154057507837\\
59	0.11496	0.42180476637413	0.42180476637413\\
59	0.11862	0.434775537608788	0.434775537608788\\
59	0.12228	0.452066371211825	0.452066371211825\\
59	0.12594	0.473677267183233	0.473677267183233\\
59	0.1296	0.499608225523009	0.499608225523009\\
59	0.13326	0.52985924623116	0.52985924623116\\
59	0.13692	0.564430329307683	0.564430329307683\\
59	0.14058	0.603321474752576	0.603321474752576\\
59	0.14424	0.646532682565839	0.646532682565839\\
59	0.1479	0.694063952747476	0.694063952747476\\
59	0.15156	0.745915285297486	0.745915285297486\\
59	0.15522	0.802086680215863	0.802086680215863\\
59	0.15888	0.862578137502621	0.862578137502621\\
59	0.16254	0.927389657157735	0.927389657157735\\
59	0.1662	0.996521239181231	0.996521239181231\\
59	0.16986	1.0699728835731	1.0699728835731\\
59	0.17352	1.14774459033334	1.14774459033334\\
59	0.17718	1.22983635946195	1.22983635946195\\
59	0.18084	1.31624819095893	1.31624819095893\\
59	0.1845	1.40698008482428	1.40698008482428\\
59	0.18816	1.502032041058	1.502032041058\\
59	0.19182	1.6014040596601	1.6014040596601\\
59	0.19548	1.70509614063057	1.70509614063057\\
59	0.19914	1.81310828396941	1.81310828396941\\
59	0.2028	1.92544048967662	1.92544048967662\\
59	0.20646	2.0420927577522	2.0420927577522\\
59	0.21012	2.16306508819615	2.16306508819615\\
59	0.21378	2.28835748100848	2.28835748100848\\
59	0.21744	2.41796993618918	2.41796993618918\\
59	0.2211	2.55190245373825	2.55190245373825\\
59	0.22476	2.69015503365569	2.69015503365569\\
59	0.22842	2.8327276759415	2.8327276759415\\
59	0.23208	2.97962038059567	2.97962038059567\\
59	0.23574	3.13083314761823	3.13083314761823\\
59	0.2394	3.28636597700917	3.28636597700917\\
59	0.24306	3.44621886876846	3.44621886876846\\
59	0.24672	3.61039182289614	3.61039182289614\\
59	0.25038	3.77888483939218	3.77888483939218\\
59	0.25404	3.95169791825659	3.95169791825659\\
59	0.2577	4.12883105948938	4.12883105948938\\
59	0.26136	4.31028426309053	4.31028426309053\\
59	0.26502	4.49605752906005	4.49605752906005\\
59	0.26868	4.68615085739796	4.68615085739796\\
59	0.27234	4.88056424810423	4.88056424810423\\
59	0.276	5.07929770117888	5.07929770117888\\
59.375	0.093	0.432502636812848	0.432502636812848\\
59.375	0.09666	0.423501215389714	0.423501215389714\\
59.375	0.10032	0.418819856334956	0.418819856334956\\
59.375	0.10398	0.418458559648572	0.418458559648572\\
59.375	0.10764	0.422417325330554	0.422417325330554\\
59.375	0.1113	0.430696153380911	0.430696153380911\\
59.375	0.11496	0.443295043799642	0.443295043799642\\
59.375	0.11862	0.460213996586738	0.460213996586738\\
59.375	0.12228	0.48145301174221	0.48145301174221\\
59.375	0.12594	0.507012089266056	0.507012089266056\\
59.375	0.1296	0.536891229158267	0.536891229158267\\
59.375	0.13326	0.571090431418857	0.571090431418857\\
59.375	0.13692	0.609609696047814	0.609609696047814\\
59.375	0.14058	0.652449023045145	0.652449023045145\\
59.375	0.14424	0.699608412410842	0.699608412410842\\
59.375	0.1479	0.751087864144915	0.751087864144915\\
59.375	0.15156	0.806887378247362	0.806887378247362\\
59.375	0.15522	0.867006954718175	0.867006954718175\\
59.375	0.15888	0.931446593557367	0.931446593557367\\
59.375	0.16254	1.00020629476492	1.00020629476492\\
59.375	0.1662	1.07328605834085	1.07328605834085\\
59.375	0.16986	1.15068588428516	1.15068588428516\\
59.375	0.17352	1.23240577259783	1.23240577259783\\
59.375	0.17718	1.31844572327888	1.31844572327888\\
59.375	0.18084	1.4088057363283	1.4088057363283\\
59.375	0.1845	1.50348581174608	1.50348581174608\\
59.375	0.18816	1.60248594953224	1.60248594953224\\
59.375	0.19182	1.70580614968677	1.70580614968677\\
59.375	0.19548	1.81344641220968	1.81344641220968\\
59.375	0.19914	1.92540673710096	1.92540673710096\\
59.375	0.2028	2.0416871243606	2.0416871243606\\
59.375	0.20646	2.16228757398862	2.16228757398862\\
59.375	0.21012	2.28720808598501	2.28720808598501\\
59.375	0.21378	2.41644866034977	2.41644866034977\\
59.375	0.21744	2.55000929708291	2.55000929708291\\
59.375	0.2211	2.68788999618441	2.68788999618441\\
59.375	0.22476	2.83009075765429	2.83009075765429\\
59.375	0.22842	2.97661158149254	2.97661158149254\\
59.375	0.23208	3.12745246769916	3.12745246769916\\
59.375	0.23574	3.28261341627415	3.28261341627415\\
59.375	0.2394	3.44209442721751	3.44209442721751\\
59.375	0.24306	3.60589550052925	3.60589550052925\\
59.375	0.24672	3.77401663620935	3.77401663620935\\
59.375	0.25038	3.94645783425783	3.94645783425783\\
59.375	0.25404	4.12321909467468	4.12321909467468\\
59.375	0.2577	4.3043004174599	4.3043004174599\\
59.375	0.26136	4.48970180261349	4.48970180261349\\
59.375	0.26502	4.67942325013546	4.67942325013546\\
59.375	0.26868	4.87346476002579	4.87346476002579\\
59.375	0.27234	5.07182633228451	5.07182633228451\\
59.375	0.276	5.27450796691159	5.27450796691159\\
59.75	0.093	0.429258603076717	0.429258603076717\\
59.75	0.09666	0.424205363206021	0.424205363206021\\
59.75	0.10032	0.423472185703698	0.423472185703698\\
59.75	0.10398	0.427059070569749	0.427059070569749\\
59.75	0.10764	0.434966017804169	0.434966017804169\\
59.75	0.1113	0.447193027406964	0.447193027406964\\
59.75	0.11496	0.46374009937813	0.46374009937813\\
59.75	0.11862	0.484607233717661	0.484607233717661\\
59.75	0.12228	0.509794430425571	0.509794430425571\\
59.75	0.12594	0.539301689501852	0.539301689501852\\
59.75	0.1296	0.573129010946501	0.573129010946501\\
59.75	0.13326	0.611276394759525	0.611276394759525\\
59.75	0.13692	0.653743840940921	0.653743840940921\\
59.75	0.14058	0.700531349490686	0.700531349490686\\
59.75	0.14424	0.751638920408822	0.751638920408822\\
59.75	0.1479	0.807066553695333	0.807066553695333\\
59.75	0.15156	0.866814249350212	0.866814249350212\\
59.75	0.15522	0.930882007373462	0.930882007373462\\
59.75	0.15888	0.999269827765089	0.999269827765089\\
59.75	0.16254	1.07197771052508	1.07197771052508\\
59.75	0.1662	1.14900565565345	1.14900565565345\\
59.75	0.16986	1.23035366315019	1.23035366315019\\
59.75	0.17352	1.3160217330153	1.3160217330153\\
59.75	0.17718	1.40600986524878	1.40600986524878\\
59.75	0.18084	1.50031805985064	1.50031805985064\\
59.75	0.1845	1.59894631682086	1.59894631682086\\
59.75	0.18816	1.70189463615946	1.70189463615946\\
59.75	0.19182	1.80916301786642	1.80916301786642\\
59.75	0.19548	1.92075146194177	1.92075146194177\\
59.75	0.19914	2.03665996838548	2.03665996838548\\
59.75	0.2028	2.15688853719756	2.15688853719756\\
59.75	0.20646	2.28143716837802	2.28143716837802\\
59.75	0.21012	2.41030586192684	2.41030586192684\\
59.75	0.21378	2.54349461784404	2.54349461784404\\
59.75	0.21744	2.68100343612961	2.68100343612961\\
59.75	0.2211	2.82283231678355	2.82283231678355\\
59.75	0.22476	2.96898125980587	2.96898125980587\\
59.75	0.22842	3.11945026519655	3.11945026519655\\
59.75	0.23208	3.27423933295561	3.27423933295561\\
59.75	0.23574	3.43334846308304	3.43334846308304\\
59.75	0.2394	3.59677765557883	3.59677765557883\\
59.75	0.24306	3.76452691044301	3.76452691044301\\
59.75	0.24672	3.93659622767555	3.93659622767555\\
59.75	0.25038	4.11298560727646	4.11298560727646\\
59.75	0.25404	4.29369504924575	4.29369504924575\\
59.75	0.2577	4.47872455358341	4.47872455358341\\
59.75	0.26136	4.66807412028943	4.66807412028943\\
59.75	0.26502	4.86174374936384	4.86174374936384\\
59.75	0.26868	5.05973344080661	5.05973344080661\\
59.75	0.27234	5.26204319461776	5.26204319461776\\
59.75	0.276	5.46867301079727	5.46867301079727\\
60.125	0.093	0.424969347493573	0.424969347493573\\
60.125	0.09666	0.423864289175312	0.423864289175312\\
60.125	0.10032	0.427079293225427	0.427079293225427\\
60.125	0.10398	0.434614359643916	0.434614359643916\\
60.125	0.10764	0.446469488430771	0.446469488430771\\
60.125	0.1113	0.462644679586001	0.462644679586001\\
60.125	0.11496	0.483139933109602	0.483139933109602\\
60.125	0.11862	0.507955249001574	0.507955249001574\\
60.125	0.12228	0.537090627261918	0.537090627261918\\
60.125	0.12594	0.570546067890635	0.570546067890635\\
60.125	0.1296	0.608321570887722	0.608321570887722\\
60.125	0.13326	0.650417136253181	0.650417136253181\\
60.125	0.13692	0.696832763987011	0.696832763987011\\
60.125	0.14058	0.747568454089215	0.747568454089215\\
60.125	0.14424	0.802624206559786	0.802624206559786\\
60.125	0.1479	0.862000021398734	0.862000021398734\\
60.125	0.15156	0.925695898606052	0.925695898606052\\
60.125	0.15522	0.993711838181737	0.993711838181737\\
60.125	0.15888	1.0660478401258	1.0660478401258\\
60.125	0.16254	1.14270390443823	1.14270390443823\\
60.125	0.1662	1.22368003111903	1.22368003111903\\
60.125	0.16986	1.30897622016821	1.30897622016821\\
60.125	0.17352	1.39859247158576	1.39859247158576\\
60.125	0.17718	1.49252878537168	1.49252878537168\\
60.125	0.18084	1.59078516152597	1.59078516152597\\
60.125	0.1845	1.69336160004862	1.69336160004862\\
60.125	0.18816	1.80025810093966	1.80025810093966\\
60.125	0.19182	1.91147466419907	1.91147466419907\\
60.125	0.19548	2.02701128982684	2.02701128982684\\
60.125	0.19914	2.14686797782299	2.14686797782299\\
60.125	0.2028	2.27104472818751	2.27104472818751\\
60.125	0.20646	2.3995415409204	2.3995415409204\\
60.125	0.21012	2.53235841602166	2.53235841602166\\
60.125	0.21378	2.66949535349129	2.66949535349129\\
60.125	0.21744	2.81095235332931	2.81095235332931\\
60.125	0.2211	2.95672941553569	2.95672941553569\\
60.125	0.22476	3.10682654011043	3.10682654011043\\
60.125	0.22842	3.26124372705356	3.26124372705356\\
60.125	0.23208	3.41998097636504	3.41998097636504\\
60.125	0.23574	3.58303828804491	3.58303828804491\\
60.125	0.2394	3.75041566209315	3.75041566209315\\
60.125	0.24306	3.92211309850975	3.92211309850975\\
60.125	0.24672	4.09813059729474	4.09813059729474\\
60.125	0.25038	4.27846815844808	4.27846815844808\\
60.125	0.25404	4.46312578196981	4.46312578196981\\
60.125	0.2577	4.65210346785991	4.65210346785991\\
60.125	0.26136	4.84540121611837	4.84540121611837\\
60.125	0.26502	5.0430190267452	5.0430190267452\\
60.125	0.26868	5.24495689974041	5.24495689974041\\
60.125	0.27234	5.45121483510399	5.45121483510399\\
60.125	0.276	5.66179283283595	5.66179283283595\\
60.5	0.093	0.419634870063401	0.419634870063401\\
60.5	0.09666	0.422477993297579	0.422477993297579\\
60.5	0.10032	0.429641178900132	0.429641178900132\\
60.5	0.10398	0.441124426871053	0.441124426871053\\
60.5	0.10764	0.456927737210346	0.456927737210346\\
60.5	0.1113	0.477051109918013	0.477051109918013\\
60.5	0.11496	0.501494544994049	0.501494544994049\\
60.5	0.11862	0.530258042438456	0.530258042438456\\
60.5	0.12228	0.563341602251239	0.563341602251239\\
60.5	0.12594	0.600745224432393	0.600745224432393\\
60.5	0.1296	0.642468908981915	0.642468908981915\\
60.5	0.13326	0.688512655899809	0.688512655899809\\
60.5	0.13692	0.738876465186078	0.738876465186078\\
60.5	0.14058	0.793560336840716	0.793560336840716\\
60.5	0.14424	0.852564270863725	0.852564270863725\\
60.5	0.1479	0.915888267255109	0.915888267255109\\
60.5	0.15156	0.983532326014864	0.983532326014864\\
60.5	0.15522	1.05549644714298	1.05549644714298\\
60.5	0.15888	1.13178063063949	1.13178063063949\\
60.5	0.16254	1.21238487650435	1.21238487650435\\
60.5	0.1662	1.29730918473759	1.29730918473759\\
60.5	0.16986	1.3865535553392	1.3865535553392\\
60.5	0.17352	1.48011798830919	1.48011798830919\\
60.5	0.17718	1.57800248364754	1.57800248364754\\
60.5	0.18084	1.68020704135427	1.68020704135427\\
60.5	0.1845	1.78673166142936	1.78673166142936\\
60.5	0.18816	1.89757634387284	1.89757634387284\\
60.5	0.19182	2.01274108868468	2.01274108868468\\
60.5	0.19548	2.13222589586489	2.13222589586489\\
60.5	0.19914	2.25603076541347	2.25603076541347\\
60.5	0.2028	2.38415569733043	2.38415569733043\\
60.5	0.20646	2.51660069161576	2.51660069161576\\
60.5	0.21012	2.65336574826946	2.65336574826946\\
60.5	0.21378	2.79445086729153	2.79445086729153\\
60.5	0.21744	2.93985604868197	2.93985604868197\\
60.5	0.2211	3.08958129244078	3.08958129244078\\
60.5	0.22476	3.24362659856797	3.24362659856797\\
60.5	0.22842	3.40199196706353	3.40199196706353\\
60.5	0.23208	3.56467739792745	3.56467739792745\\
60.5	0.23574	3.73168289115976	3.73168289115976\\
60.5	0.2394	3.90300844676042	3.90300844676042\\
60.5	0.24306	4.07865406472947	4.07865406472947\\
60.5	0.24672	4.25861974506688	4.25861974506688\\
60.5	0.25038	4.44290548777268	4.44290548777268\\
60.5	0.25404	4.63151129284684	4.63151129284684\\
60.5	0.2577	4.82443716028937	4.82443716028937\\
60.5	0.26136	5.02168309010027	5.02168309010027\\
60.5	0.26502	5.22324908227954	5.22324908227954\\
60.5	0.26868	5.42913513682718	5.42913513682718\\
60.5	0.27234	5.63934125374321	5.63934125374321\\
60.5	0.276	5.85386743302759	5.85386743302759\\
60.875	0.093	0.413255170786217	0.413255170786217\\
60.875	0.09666	0.420046475572829	0.420046475572829\\
60.875	0.10032	0.431157842727817	0.431157842727817\\
60.875	0.10398	0.446589272251176	0.446589272251176\\
60.875	0.10764	0.466340764142903	0.466340764142903\\
60.875	0.1113	0.490412318403006	0.490412318403006\\
60.875	0.11496	0.51880393503148	0.51880393503148\\
60.875	0.11862	0.551515614028325	0.551515614028325\\
60.875	0.12228	0.588547355393543	0.588547355393543\\
60.875	0.12594	0.629899159127132	0.629899159127132\\
60.875	0.1296	0.675571025229088	0.675571025229088\\
60.875	0.13326	0.725562953699424	0.725562953699424\\
60.875	0.13692	0.779874944538127	0.779874944538127\\
60.875	0.14058	0.838506997745204	0.838506997745204\\
60.875	0.14424	0.901459113320648	0.901459113320648\\
60.875	0.1479	0.968731291264466	0.968731291264466\\
60.875	0.15156	1.04032353157666	1.04032353157666\\
60.875	0.15522	1.11623583425721	1.11623583425721\\
60.875	0.15888	1.19646819930615	1.19646819930615\\
60.875	0.16254	1.28102062672345	1.28102062672345\\
60.875	0.1662	1.36989311650913	1.36989311650913\\
60.875	0.16986	1.46308566866318	1.46308566866318\\
60.875	0.17352	1.5605982831856	1.5605982831856\\
60.875	0.17718	1.66243096007639	1.66243096007639\\
60.875	0.18084	1.76858369933555	1.76858369933555\\
60.875	0.1845	1.87905650096308	1.87905650096308\\
60.875	0.18816	1.99384936495899	1.99384936495899\\
60.875	0.19182	2.11296229132327	2.11296229132327\\
60.875	0.19548	2.23639528005592	2.23639528005592\\
60.875	0.19914	2.36414833115694	2.36414833115694\\
60.875	0.2028	2.49622144462634	2.49622144462634\\
60.875	0.20646	2.6326146204641	2.6326146204641\\
60.875	0.21012	2.77332785867023	2.77332785867023\\
60.875	0.21378	2.91836115924474	2.91836115924474\\
60.875	0.21744	3.06771452218761	3.06771452218761\\
60.875	0.2211	3.22138794749887	3.22138794749887\\
60.875	0.22476	3.37938143517849	3.37938143517849\\
60.875	0.22842	3.54169498522649	3.54169498522649\\
60.875	0.23208	3.70832859764285	3.70832859764285\\
60.875	0.23574	3.87928227242758	3.87928227242758\\
60.875	0.2394	4.0545560095807	4.0545560095807\\
60.875	0.24306	4.23414980910217	4.23414980910217\\
60.875	0.24672	4.41806367099203	4.41806367099203\\
60.875	0.25038	4.60629759525025	4.60629759525025\\
60.875	0.25404	4.79885158187685	4.79885158187685\\
60.875	0.2577	4.99572563087182	4.99572563087182\\
60.875	0.26136	5.19691974223515	5.19691974223515\\
60.875	0.26502	5.40243391596686	5.40243391596686\\
60.875	0.26868	5.61226815206694	5.61226815206694\\
60.875	0.27234	5.82642245053539	5.82642245053539\\
60.875	0.276	6.04489681137223	6.04489681137223\\
61.25	0.093	0.405830249662005	0.405830249662005\\
61.25	0.09666	0.416569736001056	0.416569736001056\\
61.25	0.10032	0.431629284708478	0.431629284708478\\
61.25	0.10398	0.451008895784275	0.451008895784275\\
61.25	0.10764	0.474708569228437	0.474708569228437\\
61.25	0.1113	0.502728305040978	0.502728305040978\\
61.25	0.11496	0.535068103221887	0.535068103221887\\
61.25	0.11862	0.571727963771167	0.571727963771167\\
61.25	0.12228	0.612707886688823	0.612707886688823\\
61.25	0.12594	0.65800787197485	0.65800787197485\\
61.25	0.1296	0.707627919629241	0.707627919629241\\
61.25	0.13326	0.761568029652012	0.761568029652012\\
61.25	0.13692	0.819828202043153	0.819828202043153\\
61.25	0.14058	0.882408436802665	0.882408436802665\\
61.25	0.14424	0.949308733930543	0.949308733930543\\
61.25	0.1479	1.0205290934268	1.0205290934268\\
61.25	0.15156	1.09606951529143	1.09606951529143\\
61.25	0.15522	1.17592999952442	1.17592999952442\\
61.25	0.15888	1.2601105461258	1.2601105461258\\
61.25	0.16254	1.34861115509553	1.34861115509553\\
61.25	0.1662	1.44143182643364	1.44143182643364\\
61.25	0.16986	1.53857256014013	1.53857256014013\\
61.25	0.17352	1.64003335621499	1.64003335621499\\
61.25	0.17718	1.74581421465821	1.74581421465821\\
61.25	0.18084	1.85591513546982	1.85591513546982\\
61.25	0.1845	1.97033611864978	1.97033611864978\\
61.25	0.18816	2.08907716419812	2.08907716419812\\
61.25	0.19182	2.21213827211484	2.21213827211484\\
61.25	0.19548	2.33951944239992	2.33951944239992\\
61.25	0.19914	2.47122067505338	2.47122067505338\\
61.25	0.2028	2.60724197007521	2.60724197007521\\
61.25	0.20646	2.74758332746541	2.74758332746541\\
61.25	0.21012	2.89224474722398	2.89224474722398\\
61.25	0.21378	3.04122622935093	3.04122622935093\\
61.25	0.21744	3.19452777384624	3.19452777384624\\
61.25	0.2211	3.35214938070992	3.35214938070992\\
61.25	0.22476	3.51409104994199	3.51409104994199\\
61.25	0.22842	3.68035278154242	3.68035278154242\\
61.25	0.23208	3.85093457551122	3.85093457551122\\
61.25	0.23574	4.02583643184839	4.02583643184839\\
61.25	0.2394	4.20505835055394	4.20505835055394\\
61.25	0.24306	4.38860033162785	4.38860033162785\\
61.25	0.24672	4.57646237507014	4.57646237507014\\
61.25	0.25038	4.76864448088081	4.76864448088081\\
61.25	0.25404	4.96514664905984	4.96514664905984\\
61.25	0.2577	5.16596887960724	5.16596887960724\\
61.25	0.26136	5.37111117252301	5.37111117252301\\
61.25	0.26502	5.58057352780716	5.58057352780716\\
61.25	0.26868	5.79435594545967	5.79435594545967\\
61.25	0.27234	6.01245842548057	6.01245842548057\\
61.25	0.276	6.23488096786983	6.23488096786983\\
61.625	0.093	0.397360106690776	0.397360106690776\\
61.625	0.09666	0.412047774582262	0.412047774582262\\
61.625	0.10032	0.431055504842123	0.431055504842123\\
61.625	0.10398	0.454383297470354	0.454383297470354\\
61.625	0.10764	0.482031152466958	0.482031152466958\\
61.625	0.1113	0.51399906983193	0.51399906983193\\
61.625	0.11496	0.550287049565277	0.550287049565277\\
61.625	0.11862	0.590895091666995	0.590895091666995\\
61.625	0.12228	0.635823196137086	0.635823196137086\\
61.625	0.12594	0.685071362975548	0.685071362975548\\
61.625	0.1296	0.738639592182377	0.738639592182377\\
61.625	0.13326	0.796527883757582	0.796527883757582\\
61.625	0.13692	0.858736237701159	0.858736237701159\\
61.625	0.14058	0.925264654013108	0.925264654013108\\
61.625	0.14424	0.996113132693425	0.996113132693425\\
61.625	0.1479	1.07128167374212	1.07128167374212\\
61.625	0.15156	1.15077027715918	1.15077027715918\\
61.625	0.15522	1.23457894294461	1.23457894294461\\
61.625	0.15888	1.32270767109842	1.32270767109842\\
61.625	0.16254	1.41515646162059	1.41515646162059\\
61.625	0.1662	1.51192531451114	1.51192531451114\\
61.625	0.16986	1.61301422977006	1.61301422977006\\
61.625	0.17352	1.71842320739736	1.71842320739736\\
61.625	0.17718	1.82815224739302	1.82815224739302\\
61.625	0.18084	1.94220134975706	1.94220134975706\\
61.625	0.1845	2.06057051448946	2.06057051448946\\
61.625	0.18816	2.18325974159024	2.18325974159024\\
61.625	0.19182	2.31026903105939	2.31026903105939\\
61.625	0.19548	2.44159838289691	2.44159838289691\\
61.625	0.19914	2.57724779710281	2.57724779710281\\
61.625	0.2028	2.71721727367708	2.71721727367708\\
61.625	0.20646	2.86150681261971	2.86150681261971\\
61.625	0.21012	3.01011641393072	3.01011641393072\\
61.625	0.21378	3.1630460776101	3.1630460776101\\
61.625	0.21744	3.32029580365785	3.32029580365785\\
61.625	0.2211	3.48186559207397	3.48186559207397\\
61.625	0.22476	3.64775544285847	3.64775544285847\\
61.625	0.22842	3.81796535601134	3.81796535601134\\
61.625	0.23208	3.99249533153256	3.99249533153256\\
61.625	0.23574	4.17134536942218	4.17134536942218\\
61.625	0.2394	4.35451546968016	4.35451546968016\\
61.625	0.24306	4.54200563230651	4.54200563230651\\
61.625	0.24672	4.73381585730124	4.73381585730124\\
61.625	0.25038	4.92994614466434	4.92994614466434\\
61.625	0.25404	5.13039649439581	5.13039649439581\\
61.625	0.2577	5.33516690649565	5.33516690649565\\
61.625	0.26136	5.54425738096385	5.54425738096385\\
61.625	0.26502	5.75766791780043	5.75766791780043\\
61.625	0.26868	5.97539851700539	5.97539851700539\\
61.625	0.27234	6.19744917857871	6.19744917857871\\
61.625	0.276	6.42381990252042	6.42381990252042\\
62	0.093	0.387844741872531	0.387844741872531\\
62	0.09666	0.406480591316451	0.406480591316451\\
62	0.10032	0.429436503128747	0.429436503128747\\
62	0.10398	0.456712477309417	0.456712477309417\\
62	0.10764	0.488308513858455	0.488308513858455\\
62	0.1113	0.524224612775866	0.524224612775866\\
62	0.11496	0.564460774061647	0.564460774061647\\
62	0.11862	0.6090169977158	0.6090169977158\\
62	0.12228	0.657893283738329	0.657893283738329\\
62	0.12594	0.711089632129226	0.711089632129226\\
62	0.1296	0.768606042888493	0.768606042888493\\
62	0.13326	0.830442516016133	0.830442516016133\\
62	0.13692	0.896599051512148	0.896599051512148\\
62	0.14058	0.967075649376532	0.967075649376532\\
62	0.14424	1.04187230960929	1.04187230960929\\
62	0.1479	1.12098903221041	1.12098903221041\\
62	0.15156	1.20442581717991	1.20442581717991\\
62	0.15522	1.29218266451778	1.29218266451778\\
62	0.15888	1.38425957422403	1.38425957422403\\
62	0.16254	1.48065654629864	1.48065654629864\\
62	0.1662	1.58137358074162	1.58137358074162\\
62	0.16986	1.68641067755298	1.68641067755298\\
62	0.17352	1.79576783673271	1.79576783673271\\
62	0.17718	1.90944505828081	1.90944505828081\\
62	0.18084	2.02744234219728	2.02744234219728\\
62	0.1845	2.14975968848212	2.14975968848212\\
62	0.18816	2.27639709713534	2.27639709713534\\
62	0.19182	2.40735456815693	2.40735456815693\\
62	0.19548	2.54263210154688	2.54263210154688\\
62	0.19914	2.68222969730521	2.68222969730521\\
62	0.2028	2.82614735543192	2.82614735543192\\
62	0.20646	2.97438507592699	2.97438507592699\\
62	0.21012	3.12694285879043	3.12694285879043\\
62	0.21378	3.28382070402225	3.28382070402225\\
62	0.21744	3.44501861162244	3.44501861162244\\
62	0.2211	3.610536581591	3.610536581591\\
62	0.22476	3.78037461392792	3.78037461392792\\
62	0.22842	3.95453270863323	3.95453270863323\\
62	0.23208	4.1330108657069	4.1330108657069\\
62	0.23574	4.31580908514895	4.31580908514895\\
62	0.2394	4.50292736695936	4.50292736695936\\
62	0.24306	4.69436571113815	4.69436571113815\\
62	0.24672	4.89012411768532	4.89012411768532\\
62	0.25038	5.09020258660085	5.09020258660085\\
62	0.25404	5.29460111788475	5.29460111788475\\
62	0.2577	5.50331971153703	5.50331971153703\\
62	0.26136	5.71635836755768	5.71635836755768\\
62	0.26502	5.9337170859467	5.9337170859467\\
62	0.26868	6.15539586670408	6.15539586670408\\
62	0.27234	6.38139470982985	6.38139470982985\\
62	0.276	6.61171361532398	6.61171361532398\\
62.375	0.093	0.377284155207262	0.377284155207262\\
62.375	0.09666	0.39986818620362	0.39986818620362\\
62.375	0.10032	0.42677227956835	0.42677227956835\\
62.375	0.10398	0.457996435301455	0.457996435301455\\
62.375	0.10764	0.493540653402932	0.493540653402932\\
62.375	0.1113	0.533404933872777	0.533404933872777\\
62.375	0.11496	0.577589276710997	0.577589276710997\\
62.375	0.11862	0.626093681917585	0.626093681917585\\
62.375	0.12228	0.678918149492552	0.678918149492552\\
62.375	0.12594	0.736062679435887	0.736062679435887\\
62.375	0.1296	0.797527271747589	0.797527271747589\\
62.375	0.13326	0.863311926427667	0.863311926427667\\
62.375	0.13692	0.933416643476116	0.933416643476116\\
62.375	0.14058	1.00784142289294	1.00784142289294\\
62.375	0.14424	1.08658626467812	1.08658626467812\\
62.375	0.1479	1.16965116883169	1.16965116883169\\
62.375	0.15156	1.25703613535363	1.25703613535363\\
62.375	0.15522	1.34874116424393	1.34874116424393\\
62.375	0.15888	1.44476625550261	1.44476625550261\\
62.375	0.16254	1.54511140912966	1.54511140912966\\
62.375	0.1662	1.64977662512508	1.64977662512508\\
62.375	0.16986	1.75876190348887	1.75876190348887\\
62.375	0.17352	1.87206724422104	1.87206724422104\\
62.375	0.17718	1.98969264732158	1.98969264732158\\
62.375	0.18084	2.11163811279049	2.11163811279049\\
62.375	0.1845	2.23790364062776	2.23790364062776\\
62.375	0.18816	2.36848923083342	2.36848923083342\\
62.375	0.19182	2.50339488340744	2.50339488340744\\
62.375	0.19548	2.64262059834984	2.64262059834984\\
62.375	0.19914	2.7861663756606	2.7861663756606\\
62.375	0.2028	2.93403221533974	2.93403221533974\\
62.375	0.20646	3.08621811738725	3.08621811738725\\
62.375	0.21012	3.24272408180313	3.24272408180313\\
62.375	0.21378	3.40355010858738	3.40355010858738\\
62.375	0.21744	3.56869619774001	3.56869619774001\\
62.375	0.2211	3.738162349261	3.738162349261\\
62.375	0.22476	3.91194856315037	3.91194856315037\\
62.375	0.22842	4.09005483940811	4.09005483940811\\
62.375	0.23208	4.27248117803422	4.27248117803422\\
62.375	0.23574	4.4592275790287	4.4592275790287\\
62.375	0.2394	4.65029404239155	4.65029404239155\\
62.375	0.24306	4.84568056812278	4.84568056812278\\
62.375	0.24672	5.04538715622238	5.04538715622238\\
62.375	0.25038	5.24941380669035	5.24941380669035\\
62.375	0.25404	5.45776051952669	5.45776051952669\\
62.375	0.2577	5.6704272947314	5.6704272947314\\
62.375	0.26136	5.88741413230447	5.88741413230447\\
62.375	0.26502	6.10872103224594	6.10872103224594\\
62.375	0.26868	6.33434799455577	6.33434799455577\\
62.375	0.27234	6.56429501923397	6.56429501923397\\
62.375	0.276	6.79856210628054	6.79856210628054\\
62.75	0.093	0.365678346694972	0.365678346694972\\
62.75	0.09666	0.392210559243765	0.392210559243765\\
62.75	0.10032	0.423062834160937	0.423062834160937\\
62.75	0.10398	0.458235171446477	0.458235171446477\\
62.75	0.10764	0.497727571100389	0.497727571100389\\
62.75	0.1113	0.541540033122672	0.541540033122672\\
62.75	0.11496	0.589672557513326	0.589672557513326\\
62.75	0.11862	0.642125144272352	0.642125144272352\\
62.75	0.12228	0.69889779339975	0.69889779339975\\
62.75	0.12594	0.759990504895524	0.759990504895524\\
62.75	0.1296	0.825403278759664	0.825403278759664\\
62.75	0.13326	0.895136114992177	0.895136114992177\\
62.75	0.13692	0.969189013593065	0.969189013593065\\
62.75	0.14058	1.04756197456232	1.04756197456232\\
62.75	0.14424	1.13025499789995	1.13025499789995\\
62.75	0.1479	1.21726808360595	1.21726808360595\\
62.75	0.15156	1.30860123168032	1.30860123168032\\
62.75	0.15522	1.40425444212306	1.40425444212306\\
62.75	0.15888	1.50422771493418	1.50422771493418\\
62.75	0.16254	1.60852105011366	1.60852105011366\\
62.75	0.1662	1.71713444766152	1.71713444766152\\
62.75	0.16986	1.83006790757775	1.83006790757775\\
62.75	0.17352	1.94732142986235	1.94732142986235\\
62.75	0.17718	2.06889501451532	2.06889501451532\\
62.75	0.18084	2.19478866153667	2.19478866153667\\
62.75	0.1845	2.32500237092638	2.32500237092638\\
62.75	0.18816	2.45953614268447	2.45953614268447\\
62.75	0.19182	2.59838997681093	2.59838997681093\\
62.75	0.19548	2.74156387330576	2.74156387330576\\
62.75	0.19914	2.88905783216896	2.88905783216896\\
62.75	0.2028	3.04087185340054	3.04087185340054\\
62.75	0.20646	3.19700593700049	3.19700593700049\\
62.75	0.21012	3.3574600829688	3.3574600829688\\
62.75	0.21378	3.52223429130549	3.52223429130549\\
62.75	0.21744	3.69132856201055	3.69132856201055\\
62.75	0.2211	3.86474289508399	3.86474289508399\\
62.75	0.22476	4.04247729052579	4.04247729052579\\
62.75	0.22842	4.22453174833596	4.22453174833596\\
62.75	0.23208	4.4109062685145	4.4109062685145\\
62.75	0.23574	4.60160085106142	4.60160085106142\\
62.75	0.2394	4.79661549597672	4.79661549597672\\
62.75	0.24306	4.99595020326038	4.99595020326038\\
62.75	0.24672	5.19960497291242	5.19960497291242\\
62.75	0.25038	5.40757980493282	5.40757980493282\\
62.75	0.25404	5.61987469932159	5.61987469932159\\
62.75	0.2577	5.83648965607875	5.83648965607875\\
62.75	0.26136	6.05742467520427	6.05742467520427\\
62.75	0.26502	6.28267975669815	6.28267975669815\\
62.75	0.26868	6.51225490056043	6.51225490056043\\
62.75	0.27234	6.74615010679105	6.74615010679105\\
62.75	0.276	6.98436537539007	6.98436537539007\\
63.125	0.093	0.353027316335659	0.353027316335659\\
63.125	0.09666	0.38350771043689	0.38350771043689\\
63.125	0.10032	0.418308166906497	0.418308166906497\\
63.125	0.10398	0.457428685744475	0.457428685744475\\
63.125	0.10764	0.500869266950821	0.500869266950821\\
63.125	0.1113	0.548629910525539	0.548629910525539\\
63.125	0.11496	0.600710616468632	0.600710616468632\\
63.125	0.11862	0.657111384780093	0.657111384780093\\
63.125	0.12228	0.717832215459929	0.717832215459929\\
63.125	0.12594	0.782873108508137	0.782873108508137\\
63.125	0.1296	0.852234063924712	0.852234063924712\\
63.125	0.13326	0.925915081709663	0.925915081709663\\
63.125	0.13692	1.00391616186299	1.00391616186299\\
63.125	0.14058	1.08623730438468	1.08623730438468\\
63.125	0.14424	1.17287850927474	1.17287850927474\\
63.125	0.1479	1.26383977653318	1.26383977653318\\
63.125	0.15156	1.35912110615999	1.35912110615999\\
63.125	0.15522	1.45872249815517	1.45872249815517\\
63.125	0.15888	1.56264395251872	1.56264395251872\\
63.125	0.16254	1.67088546925064	1.67088546925064\\
63.125	0.1662	1.78344704835093	1.78344704835093\\
63.125	0.16986	1.9003286898196	1.9003286898196\\
63.125	0.17352	2.02153039365664	2.02153039365664\\
63.125	0.17718	2.14705215986205	2.14705215986205\\
63.125	0.18084	2.27689398843583	2.27689398843583\\
63.125	0.1845	2.41105587937798	2.41105587937798\\
63.125	0.18816	2.5495378326885	2.5495378326885\\
63.125	0.19182	2.6923398483674	2.6923398483674\\
63.125	0.19548	2.83946192641467	2.83946192641467\\
63.125	0.19914	2.99090406683031	2.99090406683031\\
63.125	0.2028	3.14666626961432	3.14666626961432\\
63.125	0.20646	3.3067485347667	3.3067485347667\\
63.125	0.21012	3.47115086228745	3.47115086228745\\
63.125	0.21378	3.63987325217657	3.63987325217657\\
63.125	0.21744	3.81291570443407	3.81291570443407\\
63.125	0.2211	3.99027821905994	3.99027821905994\\
63.125	0.22476	4.17196079605418	4.17196079605418\\
63.125	0.22842	4.3579634354168	4.3579634354168\\
63.125	0.23208	4.54828613714777	4.54828613714777\\
63.125	0.23574	4.74292890124713	4.74292890124713\\
63.125	0.2394	4.94189172771486	4.94189172771486\\
63.125	0.24306	5.14517461655096	5.14517461655096\\
63.125	0.24672	5.35277756775543	5.35277756775543\\
63.125	0.25038	5.56470058132827	5.56470058132827\\
63.125	0.25404	5.78094365726949	5.78094365726949\\
63.125	0.2577	6.00150679557907	6.00150679557907\\
63.125	0.26136	6.22638999625702	6.22638999625702\\
63.125	0.26502	6.45559325930335	6.45559325930335\\
63.125	0.26868	6.68911658471805	6.68911658471805\\
63.125	0.27234	6.92695997250113	6.92695997250113\\
63.125	0.276	7.16912342265257	7.16912342265257\\
63.5	0.093	0.339331064129327	0.339331064129327\\
63.5	0.09666	0.373759639782997	0.373759639782997\\
63.5	0.10032	0.412508277805038	0.412508277805038\\
63.5	0.10398	0.455576978195451	0.455576978195451\\
63.5	0.10764	0.502965740954235	0.502965740954235\\
63.5	0.1113	0.554674566081392	0.554674566081392\\
63.5	0.11496	0.610703453576919	0.610703453576919\\
63.5	0.11862	0.671052403440818	0.671052403440818\\
63.5	0.12228	0.735721415673089	0.735721415673089\\
63.5	0.12594	0.804710490273735	0.804710490273735\\
63.5	0.1296	0.878019627242745	0.878019627242745\\
63.5	0.13326	0.955648826580135	0.955648826580135\\
63.5	0.13692	1.03759808828589	1.03759808828589\\
63.5	0.14058	1.12386741236002	1.12386741236002\\
63.5	0.14424	1.21445679880252	1.21445679880252\\
63.5	0.1479	1.30936624761339	1.30936624761339\\
63.5	0.15156	1.40859575879264	1.40859575879264\\
63.5	0.15522	1.51214533234025	1.51214533234025\\
63.5	0.15888	1.62001496825624	1.62001496825624\\
63.5	0.16254	1.7322046665406	1.7322046665406\\
63.5	0.1662	1.84871442719333	1.84871442719333\\
63.5	0.16986	1.96954425021443	1.96954425021443\\
63.5	0.17352	2.09469413560391	2.09469413560391\\
63.5	0.17718	2.22416408336175	2.22416408336175\\
63.5	0.18084	2.35795409348797	2.35795409348797\\
63.5	0.1845	2.49606416598255	2.49606416598255\\
63.5	0.18816	2.63849430084552	2.63849430084552\\
63.5	0.19182	2.78524449807684	2.78524449807684\\
63.5	0.19548	2.93631475767655	2.93631475767655\\
63.5	0.19914	3.09170507964463	3.09170507964463\\
63.5	0.2028	3.25141546398107	3.25141546398107\\
63.5	0.20646	3.4154459106859	3.4154459106859\\
63.5	0.21012	3.58379641975908	3.58379641975908\\
63.5	0.21378	3.75646699120064	3.75646699120064\\
63.5	0.21744	3.93345762501058	3.93345762501058\\
63.5	0.2211	4.11476832118888	4.11476832118888\\
63.5	0.22476	4.30039907973556	4.30039907973556\\
63.5	0.22842	4.49034990065061	4.49034990065061\\
63.5	0.23208	4.68462078393402	4.68462078393402\\
63.5	0.23574	4.88321172958581	4.88321172958581\\
63.5	0.2394	5.08612273760598	5.08612273760598\\
63.5	0.24306	5.29335380799452	5.29335380799452\\
63.5	0.24672	5.50490494075143	5.50490494075143\\
63.5	0.25038	5.7207761358767	5.7207761358767\\
63.5	0.25404	5.94096739337035	5.94096739337035\\
63.5	0.2577	6.16547871323237	6.16547871323237\\
63.5	0.26136	6.39431009546276	6.39431009546276\\
63.5	0.26502	6.62746154006153	6.62746154006153\\
63.5	0.26868	6.86493304702867	6.86493304702867\\
63.5	0.27234	7.10672461636417	7.10672461636417\\
63.5	0.276	7.35283624806805	7.35283624806805\\
63.875	0.093	0.324589590075977	0.324589590075977\\
63.875	0.09666	0.362966347282081	0.362966347282081\\
63.875	0.10032	0.405663166856558	0.405663166856558\\
63.875	0.10398	0.452680048799408	0.452680048799408\\
63.875	0.10764	0.504016993110628	0.504016993110628\\
63.875	0.1113	0.559673999790222	0.559673999790222\\
63.875	0.11496	0.619651068838184	0.619651068838184\\
63.875	0.11862	0.683948200254518	0.683948200254518\\
63.875	0.12228	0.752565394039227	0.752565394039227\\
63.875	0.12594	0.825502650192308	0.825502650192308\\
63.875	0.1296	0.902759968713756	0.902759968713756\\
63.875	0.13326	0.98433734960358	0.98433734960358\\
63.875	0.13692	1.07023479286178	1.07023479286178\\
63.875	0.14058	1.16045229848834	1.16045229848834\\
63.875	0.14424	1.25498986648328	1.25498986648328\\
63.875	0.1479	1.35384749684659	1.35384749684659\\
63.875	0.15156	1.45702518957827	1.45702518957827\\
63.875	0.15522	1.56452294467832	1.56452294467832\\
63.875	0.15888	1.67634076214674	1.67634076214674\\
63.875	0.16254	1.79247864198353	1.79247864198353\\
63.875	0.1662	1.9129365841887	1.9129365841887\\
63.875	0.16986	2.03771458876224	2.03771458876224\\
63.875	0.17352	2.16681265570415	2.16681265570415\\
63.875	0.17718	2.30023078501444	2.30023078501444\\
63.875	0.18084	2.43796897669309	2.43796897669309\\
63.875	0.1845	2.58002723074011	2.58002723074011\\
63.875	0.18816	2.72640554715551	2.72640554715551\\
63.875	0.19182	2.87710392593927	2.87710392593927\\
63.875	0.19548	3.03212236709142	3.03212236709142\\
63.875	0.19914	3.19146087061193	3.19146087061193\\
63.875	0.2028	3.35511943650081	3.35511943650081\\
63.875	0.20646	3.52309806475807	3.52309806475807\\
63.875	0.21012	3.69539675538369	3.69539675538369\\
63.875	0.21378	3.87201550837769	3.87201550837769\\
63.875	0.21744	4.05295432374006	4.05295432374006\\
63.875	0.2211	4.2382132014708	4.2382132014708\\
63.875	0.22476	4.42779214156992	4.42779214156992\\
63.875	0.22842	4.6216911440374	4.6216911440374\\
63.875	0.23208	4.81991020887325	4.81991020887325\\
63.875	0.23574	5.02244933607748	5.02244933607748\\
63.875	0.2394	5.22930852565008	5.22930852565008\\
63.875	0.24306	5.44048777759105	5.44048777759105\\
63.875	0.24672	5.6559870919004	5.6559870919004\\
63.875	0.25038	5.87580646857811	5.87580646857811\\
63.875	0.25404	6.09994590762419	6.09994590762419\\
63.875	0.2577	6.32840540903866	6.32840540903866\\
63.875	0.26136	6.56118497282148	6.56118497282148\\
63.875	0.26502	6.79828459897267	6.79828459897267\\
63.875	0.26868	7.03970428749226	7.03970428749226\\
63.875	0.27234	7.28544403838019	7.28544403838019\\
63.875	0.276	7.53550385163652	7.53550385163652\\
64.25	0.093	0.308802894175607	0.308802894175607\\
64.25	0.09666	0.351127832934146	0.351127832934146\\
64.25	0.10032	0.39777283406106	0.39777283406106\\
64.25	0.10398	0.448737897556349	0.448737897556349\\
64.25	0.10764	0.504023023420003	0.504023023420003\\
64.25	0.1113	0.563628211652032	0.563628211652032\\
64.25	0.11496	0.627553462252433	0.627553462252433\\
64.25	0.11862	0.695798775221204	0.695798775221204\\
64.25	0.12228	0.768364150558349	0.768364150558349\\
64.25	0.12594	0.845249588263864	0.845249588263864\\
64.25	0.1296	0.926455088337751	0.926455088337751\\
64.25	0.13326	1.01198065078001	1.01198065078001\\
64.25	0.13692	1.10182627559064	1.10182627559064\\
64.25	0.14058	1.19599196276964	1.19599196276964\\
64.25	0.14424	1.29447771231702	1.29447771231702\\
64.25	0.1479	1.39728352423276	1.39728352423276\\
64.25	0.15156	1.50440939851688	1.50440939851688\\
64.25	0.15522	1.61585533516937	1.61585533516937\\
64.25	0.15888	1.73162133419023	1.73162133419023\\
64.25	0.16254	1.85170739557946	1.85170739557946\\
64.25	0.1662	1.97611351933706	1.97611351933706\\
64.25	0.16986	2.10483970546303	2.10483970546303\\
64.25	0.17352	2.23788595395738	2.23788595395738\\
64.25	0.17718	2.3752522648201	2.3752522648201\\
64.25	0.18084	2.51693863805119	2.51693863805119\\
64.25	0.1845	2.66294507365065	2.66294507365065\\
64.25	0.18816	2.81327157161848	2.81327157161848\\
64.25	0.19182	2.96791813195469	2.96791813195469\\
64.25	0.19548	3.12688475465926	3.12688475465926\\
64.25	0.19914	3.29017143973221	3.29017143973221\\
64.25	0.2028	3.45777818717353	3.45777818717353\\
64.25	0.20646	3.62970499698322	3.62970499698322\\
64.25	0.21012	3.80595186916129	3.80595186916129\\
64.25	0.21378	3.98651880370772	3.98651880370772\\
64.25	0.21744	4.17140580062252	4.17140580062252\\
64.25	0.2211	4.3606128599057	4.3606128599057\\
64.25	0.22476	4.55413998155725	4.55413998155725\\
64.25	0.22842	4.75198716557717	4.75198716557717\\
64.25	0.23208	4.95415441196546	4.95415441196546\\
64.25	0.23574	5.16064172072213	5.16064172072213\\
64.25	0.2394	5.37144909184716	5.37144909184716\\
64.25	0.24306	5.58657652534058	5.58657652534058\\
64.25	0.24672	5.80602402120235	5.80602402120235\\
64.25	0.25038	6.0297915794325	6.0297915794325\\
64.25	0.25404	6.25787920003103	6.25787920003103\\
64.25	0.2577	6.49028688299792	6.49028688299792\\
64.25	0.26136	6.72701462833317	6.72701462833317\\
64.25	0.26502	6.96806243603682	6.96806243603682\\
64.25	0.26868	7.21343030610883	7.21343030610883\\
64.25	0.27234	7.46311823854921	7.46311823854921\\
64.25	0.276	7.71712623335796	7.71712623335796\\
64.625	0.093	0.291970976428214	0.291970976428214\\
64.625	0.09666	0.338244096739191	0.338244096739191\\
64.625	0.10032	0.38883727941854	0.38883727941854\\
64.625	0.10398	0.443750524466264	0.443750524466264\\
64.625	0.10764	0.502983831882356	0.502983831882356\\
64.625	0.1113	0.566537201666824	0.566537201666824\\
64.625	0.11496	0.634410633819659	0.634410633819659\\
64.625	0.11862	0.706604128340866	0.706604128340866\\
64.625	0.12228	0.783117685230448	0.783117685230448\\
64.625	0.12594	0.863951304488402	0.863951304488402\\
64.625	0.1296	0.949104986114723	0.949104986114723\\
64.625	0.13326	1.03857873010942	1.03857873010942\\
64.625	0.13692	1.13237253647248	1.13237253647248\\
64.625	0.14058	1.23048640520393	1.23048640520393\\
64.625	0.14424	1.33292033630373	1.33292033630373\\
64.625	0.1479	1.43967432977191	1.43967432977191\\
64.625	0.15156	1.55074838560847	1.55074838560847\\
64.625	0.15522	1.66614250381339	1.66614250381339\\
64.625	0.15888	1.78585668438669	1.78585668438669\\
64.625	0.16254	1.90989092732835	1.90989092732835\\
64.625	0.1662	2.0382452326384	2.0382452326384\\
64.625	0.16986	2.17091960031681	2.17091960031681\\
64.625	0.17352	2.30791403036359	2.30791403036359\\
64.625	0.17718	2.44922852277875	2.44922852277875\\
64.625	0.18084	2.59486307756227	2.59486307756227\\
64.625	0.1845	2.74481769471417	2.74481769471417\\
64.625	0.18816	2.89909237423444	2.89909237423444\\
64.625	0.19182	3.05768711612307	3.05768711612307\\
64.625	0.19548	3.22060192038009	3.22060192038009\\
64.625	0.19914	3.38783678700547	3.38783678700547\\
64.625	0.2028	3.55939171599923	3.55939171599923\\
64.625	0.20646	3.73526670736136	3.73526670736136\\
64.625	0.21012	3.91546176109186	3.91546176109186\\
64.625	0.21378	4.09997687719073	4.09997687719073\\
64.625	0.21744	4.28881205565797	4.28881205565797\\
64.625	0.2211	4.48196729649358	4.48196729649358\\
64.625	0.22476	4.67944259969757	4.67944259969757\\
64.625	0.22842	4.88123796526993	4.88123796526993\\
64.625	0.23208	5.08735339321065	5.08735339321065\\
64.625	0.23574	5.29778888351976	5.29778888351976\\
64.625	0.2394	5.51254443619723	5.51254443619723\\
64.625	0.24306	5.73162005124307	5.73162005124307\\
64.625	0.24672	5.95501572865729	5.95501572865729\\
64.625	0.25038	6.18273146843987	6.18273146843987\\
64.625	0.25404	6.41476727059083	6.41476727059083\\
64.625	0.2577	6.65112313511017	6.65112313511017\\
64.625	0.26136	6.89179906199786	6.89179906199786\\
64.625	0.26502	7.13679505125394	7.13679505125394\\
64.625	0.26868	7.38611110287838	7.38611110287838\\
64.625	0.27234	7.6397472168712	7.6397472168712\\
64.625	0.276	7.89770339323239	7.89770339323239\\
65	0.093	0.274093836833803	0.274093836833803\\
65	0.09666	0.324315138697219	0.324315138697219\\
65	0.10032	0.378856502929006	0.378856502929006\\
65	0.10398	0.437717929529164	0.437717929529164\\
65	0.10764	0.500899418497691	0.500899418497691\\
65	0.1113	0.568400969834593	0.568400969834593\\
65	0.11496	0.640222583539867	0.640222583539867\\
65	0.11862	0.716364259613512	0.716364259613512\\
65	0.12228	0.796825998055529	0.796825998055529\\
65	0.12594	0.881607798865917	0.881607798865917\\
65	0.1296	0.970709662044673	0.970709662044673\\
65	0.13326	1.06413158759181	1.06413158759181\\
65	0.13692	1.16187357550731	1.16187357550731\\
65	0.14058	1.26393562579119	1.26393562579119\\
65	0.14424	1.37031773844343	1.37031773844343\\
65	0.1479	1.48101991346405	1.48101991346405\\
65	0.15156	1.59604215085304	1.59604215085304\\
65	0.15522	1.7153844506104	1.7153844506104\\
65	0.15888	1.83904681273614	1.83904681273614\\
65	0.16254	1.96702923723023	1.96702923723023\\
65	0.1662	2.09933172409271	2.09933172409271\\
65	0.16986	2.23595427332356	2.23595427332356\\
65	0.17352	2.37689688492278	2.37689688492278\\
65	0.17718	2.52215955889037	2.52215955889037\\
65	0.18084	2.67174229522634	2.67174229522634\\
65	0.1845	2.82564509393066	2.82564509393066\\
65	0.18816	2.98386795500337	2.98386795500337\\
65	0.19182	3.14641087844445	3.14641087844445\\
65	0.19548	3.3132738642539	3.3132738642539\\
65	0.19914	3.48445691243172	3.48445691243172\\
65	0.2028	3.65996002297791	3.65996002297791\\
65	0.20646	3.83978319589248	3.83978319589248\\
65	0.21012	4.02392643117541	4.02392643117541\\
65	0.21378	4.21238972882672	4.21238972882672\\
65	0.21744	4.40517308884639	4.40517308884639\\
65	0.2211	4.60227651123444	4.60227651123444\\
65	0.22476	4.80369999599087	4.80369999599087\\
65	0.22842	5.00944354311566	5.00944354311566\\
65	0.23208	5.21950715260882	5.21950715260882\\
65	0.23574	5.43389082447036	5.43389082447036\\
65	0.2394	5.65259455870027	5.65259455870027\\
65	0.24306	5.87561835529855	5.87561835529855\\
65	0.24672	6.1029622142652	6.1029622142652\\
65	0.25038	6.33462613560023	6.33462613560023\\
65	0.25404	6.57061011930362	6.57061011930362\\
65	0.2577	6.81091416537539	6.81091416537539\\
65	0.26136	7.05553827381553	7.05553827381553\\
65	0.26502	7.30448244462403	7.30448244462403\\
65	0.26868	7.55774667780092	7.55774667780092\\
65	0.27234	7.81533097334617	7.81533097334617\\
65	0.276	8.07723533125979	8.07723533125979\\
65.375	0.093	0.25517147539237	0.25517147539237\\
65.375	0.09666	0.309340958808217	0.309340958808217\\
65.375	0.10032	0.367830504592442	0.367830504592442\\
65.375	0.10398	0.430640112745035	0.430640112745035\\
65.375	0.10764	0.497769783266	0.497769783266\\
65.375	0.1113	0.569219516155341	0.569219516155341\\
65.375	0.11496	0.644989311413049	0.644989311413049\\
65.375	0.11862	0.725079169039129	0.725079169039129\\
65.375	0.12228	0.809489089033584	0.809489089033584\\
65.375	0.12594	0.898219071396411	0.898219071396411\\
65.375	0.1296	0.991269116127601	0.991269116127601\\
65.375	0.13326	1.08863922322717	1.08863922322717\\
65.375	0.13692	1.19032939269511	1.19032939269511\\
65.375	0.14058	1.29633962453142	1.29633962453142\\
65.375	0.14424	1.4066699187361	1.4066699187361\\
65.375	0.1479	1.52132027530916	1.52132027530916\\
65.375	0.15156	1.64029069425058	1.64029069425058\\
65.375	0.15522	1.76358117556038	1.76358117556038\\
65.375	0.15888	1.89119171923855	1.89119171923855\\
65.375	0.16254	2.02312232528509	2.02312232528509\\
65.375	0.1662	2.1593729937	2.1593729937\\
65.375	0.16986	2.29994372448329	2.29994372448329\\
65.375	0.17352	2.44483451763494	2.44483451763494\\
65.375	0.17718	2.59404537315497	2.59404537315497\\
65.375	0.18084	2.74757629104337	2.74757629104337\\
65.375	0.1845	2.90542727130014	2.90542727130014\\
65.375	0.18816	3.06759831392528	3.06759831392528\\
65.375	0.19182	3.23408941891879	3.23408941891879\\
65.375	0.19548	3.40490058628068	3.40490058628068\\
65.375	0.19914	3.58003181601094	3.58003181601094\\
65.375	0.2028	3.75948310810957	3.75948310810957\\
65.375	0.20646	3.94325446257657	3.94325446257657\\
65.375	0.21012	4.13134587941194	4.13134587941194\\
65.375	0.21378	4.32375735861568	4.32375735861568\\
65.375	0.21744	4.52048890018779	4.52048890018779\\
65.375	0.2211	4.72154050412828	4.72154050412828\\
65.375	0.22476	4.92691217043714	4.92691217043714\\
65.375	0.22842	5.13660389911437	5.13660389911437\\
65.375	0.23208	5.35061569015997	5.35061569015997\\
65.375	0.23574	5.56894754357394	5.56894754357394\\
65.375	0.2394	5.79159945935628	5.79159945935628\\
65.375	0.24306	6.01857143750701	6.01857143750701\\
65.375	0.24672	6.24986347802609	6.24986347802609\\
65.375	0.25038	6.48547558091355	6.48547558091355\\
65.375	0.25404	6.72540774616939	6.72540774616939\\
65.375	0.2577	6.96965997379359	6.96965997379359\\
65.375	0.26136	7.21823226378615	7.21823226378615\\
65.375	0.26502	7.4711246161471	7.4711246161471\\
65.375	0.26868	7.72833703087642	7.72833703087642\\
65.375	0.27234	7.98986950797411	7.98986950797411\\
65.375	0.276	8.25572204744017	8.25572204744017\\
65.75	0.093	0.235203892103917	0.235203892103917\\
65.75	0.09666	0.293321557072201	0.293321557072201\\
65.75	0.10032	0.355759284408862	0.355759284408862\\
65.75	0.10398	0.422517074113893	0.422517074113893\\
65.75	0.10764	0.493594926187293	0.493594926187293\\
65.75	0.1113	0.568992840629068	0.568992840629068\\
65.75	0.11496	0.648710817439214	0.648710817439214\\
65.75	0.11862	0.732748856617732	0.732748856617732\\
65.75	0.12228	0.821106958164622	0.821106958164622\\
65.75	0.12594	0.913785122079884	0.913785122079884\\
65.75	0.1296	1.01078334836351	1.01078334836351\\
65.75	0.13326	1.11210163701552	1.11210163701552\\
65.75	0.13692	1.21773998803589	1.21773998803589\\
65.75	0.14058	1.32769840142464	1.32769840142464\\
65.75	0.14424	1.44197687718176	1.44197687718176\\
65.75	0.1479	1.56057541530725	1.56057541530725\\
65.75	0.15156	1.68349401580112	1.68349401580112\\
65.75	0.15522	1.81073267866335	1.81073267866335\\
65.75	0.15888	1.94229140389395	1.94229140389395\\
65.75	0.16254	2.07817019149293	2.07817019149293\\
65.75	0.1662	2.21836904146028	2.21836904146028\\
65.75	0.16986	2.362887953796	2.362887953796\\
65.75	0.17352	2.51172692850009	2.51172692850009\\
65.75	0.17718	2.66488596557256	2.66488596557256\\
65.75	0.18084	2.82236506501339	2.82236506501339\\
65.75	0.1845	2.98416422682259	2.98416422682259\\
65.75	0.18816	3.15028345100017	3.15028345100017\\
65.75	0.19182	3.32072273754612	3.32072273754612\\
65.75	0.19548	3.49548208646044	3.49548208646044\\
65.75	0.19914	3.67456149774314	3.67456149774314\\
65.75	0.2028	3.85796097139421	3.85796097139421\\
65.75	0.20646	4.04568050741364	4.04568050741364\\
65.75	0.21012	4.23772010580144	4.23772010580144\\
65.75	0.21378	4.43407976655762	4.43407976655762\\
65.75	0.21744	4.63475948968217	4.63475948968217\\
65.75	0.2211	4.8397592751751	4.8397592751751\\
65.75	0.22476	5.04907912303639	5.04907912303639\\
65.75	0.22842	5.26271903326606	5.26271903326606\\
65.75	0.23208	5.4806790058641	5.4806790058641\\
65.75	0.23574	5.7029590408305	5.7029590408305\\
65.75	0.2394	5.92955913816529	5.92955913816529\\
65.75	0.24306	6.16047929786844	6.16047929786844\\
65.75	0.24672	6.39571951993997	6.39571951993997\\
65.75	0.25038	6.63527980437986	6.63527980437986\\
65.75	0.25404	6.87916015118813	6.87916015118813\\
65.75	0.2577	7.12736056036478	7.12736056036478\\
65.75	0.26136	7.37988103190978	7.37988103190978\\
65.75	0.26502	7.63672156582316	7.63672156582316\\
65.75	0.26868	7.89788216210492	7.89788216210492\\
65.75	0.27234	8.16336282075504	8.16336282075504\\
65.75	0.276	8.43316354177354	8.43316354177354\\
66.125	0.093	0.214191086968443	0.214191086968443\\
66.125	0.09666	0.276256933489166	0.276256933489166\\
66.125	0.10032	0.342642842378261	0.342642842378261\\
66.125	0.10398	0.413348813635731	0.413348813635731\\
66.125	0.10764	0.488374847261565	0.488374847261565\\
66.125	0.1113	0.567720943255779	0.567720943255779\\
66.125	0.11496	0.65138710161836	0.65138710161836\\
66.125	0.11862	0.739373322349312	0.739373322349312\\
66.125	0.12228	0.83167960544864	0.83167960544864\\
66.125	0.12594	0.928305950916337	0.928305950916337\\
66.125	0.1296	1.0292523587524	1.0292523587524\\
66.125	0.13326	1.13451882895685	1.13451882895685\\
66.125	0.13692	1.24410536152966	1.24410536152966\\
66.125	0.14058	1.35801195647084	1.35801195647084\\
66.125	0.14424	1.4762386137804	1.4762386137804\\
66.125	0.1479	1.59878533345832	1.59878533345832\\
66.125	0.15156	1.72565211550462	1.72565211550462\\
66.125	0.15522	1.85683895991929	1.85683895991929\\
66.125	0.15888	1.99234586670234	1.99234586670234\\
66.125	0.16254	2.13217283585375	2.13217283585375\\
66.125	0.1662	2.27631986737353	2.27631986737353\\
66.125	0.16986	2.42478696126169	2.42478696126169\\
66.125	0.17352	2.57757411751822	2.57757411751822\\
66.125	0.17718	2.73468133614312	2.73468133614312\\
66.125	0.18084	2.89610861713639	2.89610861713639\\
66.125	0.1845	3.06185596049803	3.06185596049803\\
66.125	0.18816	3.23192336622804	3.23192336622804\\
66.125	0.19182	3.40631083432643	3.40631083432643\\
66.125	0.19548	3.58501836479318	3.58501836479318\\
66.125	0.19914	3.76804595762832	3.76804595762832\\
66.125	0.2028	3.95539361283182	3.95539361283182\\
66.125	0.20646	4.1470613304037	4.1470613304037\\
66.125	0.21012	4.34304911034393	4.34304911034393\\
66.125	0.21378	4.54335695265255	4.54335695265255\\
66.125	0.21744	4.74798485732954	4.74798485732954\\
66.125	0.2211	4.9569328243749	4.9569328243749\\
66.125	0.22476	5.17020085378863	5.17020085378863\\
66.125	0.22842	5.38778894557073	5.38778894557073\\
66.125	0.23208	5.6096970997212	5.6096970997212\\
66.125	0.23574	5.83592531624005	5.83592531624005\\
66.125	0.2394	6.06647359512727	6.06647359512727\\
66.125	0.24306	6.30134193638285	6.30134193638285\\
66.125	0.24672	6.54053034000682	6.54053034000682\\
66.125	0.25038	6.78403880599915	6.78403880599915\\
66.125	0.25404	7.03186733435986	7.03186733435986\\
66.125	0.2577	7.28401592508893	7.28401592508893\\
66.125	0.26136	7.54048457818637	7.54048457818637\\
66.125	0.26502	7.80127329365219	7.80127329365219\\
66.125	0.26868	8.06638207148639	8.06638207148639\\
66.125	0.27234	8.33581091168895	8.33581091168895\\
66.125	0.276	8.60955981425989	8.60955981425989\\
66.5	0.093	0.192133059985947	0.192133059985947\\
66.5	0.09666	0.258147088059105	0.258147088059105\\
66.5	0.10032	0.328481178500638	0.328481178500638\\
66.5	0.10398	0.403135331310543	0.403135331310543\\
66.5	0.10764	0.482109546488815	0.482109546488815\\
66.5	0.1113	0.565403824035464	0.565403824035464\\
66.5	0.11496	0.653018163950483	0.653018163950483\\
66.5	0.11862	0.74495256623387	0.74495256623387\\
66.5	0.12228	0.841207030885637	0.841207030885637\\
66.5	0.12594	0.941781557905771	0.941781557905771\\
66.5	0.1296	1.04667614729427	1.04667614729427\\
66.5	0.13326	1.15589079905115	1.15589079905115\\
66.5	0.13692	1.2694255131764	1.2694255131764\\
66.5	0.14058	1.38728028967002	1.38728028967002\\
66.5	0.14424	1.50945512853201	1.50945512853201\\
66.5	0.1479	1.63595002976237	1.63595002976237\\
66.5	0.15156	1.76676499336111	1.76676499336111\\
66.5	0.15522	1.90190001932821	1.90190001932821\\
66.5	0.15888	2.04135510766369	2.04135510766369\\
66.5	0.16254	2.18513025836754	2.18513025836754\\
66.5	0.1662	2.33322547143976	2.33322547143976\\
66.5	0.16986	2.48564074688035	2.48564074688035\\
66.5	0.17352	2.64237608468932	2.64237608468932\\
66.5	0.17718	2.80343148486665	2.80343148486665\\
66.5	0.18084	2.96880694741237	2.96880694741237\\
66.5	0.1845	3.13850247232644	3.13850247232644\\
66.5	0.18816	3.31251805960889	3.31251805960889\\
66.5	0.19182	3.49085370925971	3.49085370925971\\
66.5	0.19548	3.67350942127891	3.67350942127891\\
66.5	0.19914	3.86048519566647	3.86048519566647\\
66.5	0.2028	4.05178103242242	4.05178103242242\\
66.5	0.20646	4.24739693154672	4.24739693154672\\
66.5	0.21012	4.4473328930394	4.4473328930394\\
66.5	0.21378	4.65158891690045	4.65158891690045\\
66.5	0.21744	4.86016500312988	4.86016500312988\\
66.5	0.2211	5.07306115172767	5.07306115172767\\
66.5	0.22476	5.29027736269384	5.29027736269384\\
66.5	0.22842	5.51181363602838	5.51181363602838\\
66.5	0.23208	5.73766997173129	5.73766997173129\\
66.5	0.23574	5.96784636980257	5.96784636980257\\
66.5	0.2394	6.20234283024223	6.20234283024223\\
66.5	0.24306	6.44115935305025	6.44115935305025\\
66.5	0.24672	6.68429593822665	6.68429593822665\\
66.5	0.25038	6.93175258577142	6.93175258577142\\
66.5	0.25404	7.18352929568456	7.18352929568456\\
66.5	0.2577	7.43962606796607	7.43962606796607\\
66.5	0.26136	7.70004290261595	7.70004290261595\\
66.5	0.26502	7.9647797996342	7.9647797996342\\
66.5	0.26868	8.23383675902083	8.23383675902083\\
66.5	0.27234	8.50721378077583	8.50721378077583\\
66.5	0.276	8.7849108648992	8.7849108648992\\
66.875	0.093	0.169029811156433	0.169029811156433\\
66.875	0.09666	0.238992020782029	0.238992020782029\\
66.875	0.10032	0.313274292775997	0.313274292775997\\
66.875	0.10398	0.39187662713834	0.39187662713834\\
66.875	0.10764	0.474799023869047	0.474799023869047\\
66.875	0.1113	0.562041482968133	0.562041482968133\\
66.875	0.11496	0.653604004435588	0.653604004435588\\
66.875	0.11862	0.749486588271413	0.749486588271413\\
66.875	0.12228	0.849689234475611	0.849689234475611\\
66.875	0.12594	0.954211943048183	0.954211943048183\\
66.875	0.1296	1.06305471398912	1.06305471398912\\
66.875	0.13326	1.17621754729844	1.17621754729844\\
66.875	0.13692	1.29370044297612	1.29370044297612\\
66.875	0.14058	1.41550340102218	1.41550340102218\\
66.875	0.14424	1.5416264214366	1.5416264214366\\
66.875	0.1479	1.67206950421941	1.67206950421941\\
66.875	0.15156	1.80683264937058	1.80683264937058\\
66.875	0.15522	1.94591585689012	1.94591585689012\\
66.875	0.15888	2.08931912677804	2.08931912677804\\
66.875	0.16254	2.23704245903432	2.23704245903432\\
66.875	0.1662	2.38908585365897	2.38908585365897\\
66.875	0.16986	2.545449310652	2.545449310652\\
66.875	0.17352	2.70613283001341	2.70613283001341\\
66.875	0.17718	2.87113641174318	2.87113641174318\\
66.875	0.18084	3.04046005584132	3.04046005584132\\
66.875	0.1845	3.21410376230784	3.21410376230784\\
66.875	0.18816	3.39206753114273	3.39206753114273\\
66.875	0.19182	3.57435136234598	3.57435136234598\\
66.875	0.19548	3.76095525591761	3.76095525591761\\
66.875	0.19914	3.95187921185762	3.95187921185762\\
66.875	0.2028	4.14712323016599	4.14712323016599\\
66.875	0.20646	4.34668731084274	4.34668731084274\\
66.875	0.21012	4.55057145388785	4.55057145388785\\
66.875	0.21378	4.75877565930134	4.75877565930134\\
66.875	0.21744	4.97129992708321	4.97129992708321\\
66.875	0.2211	5.18814425723344	5.18814425723344\\
66.875	0.22476	5.40930864975204	5.40930864975204\\
66.875	0.22842	5.63479310463901	5.63479310463901\\
66.875	0.23208	5.86459762189436	5.86459762189436\\
66.875	0.23574	6.09872220151808	6.09872220151808\\
66.875	0.2394	6.33716684351017	6.33716684351017\\
66.875	0.24306	6.57993154787062	6.57993154787062\\
66.875	0.24672	6.82701631459947	6.82701631459947\\
66.875	0.25038	7.07842114369668	7.07842114369668\\
66.875	0.25404	7.33414603516225	7.33414603516225\\
66.875	0.2577	7.5941909889962	7.5941909889962\\
66.875	0.26136	7.85855600519851	7.85855600519851\\
66.875	0.26502	8.1272410837692	8.1272410837692\\
66.875	0.26868	8.40024622470827	8.40024622470827\\
66.875	0.27234	8.6775714280157	8.6775714280157\\
66.875	0.276	8.95921669369152	8.95921669369152\\
67.25	0.093	0.144881340479904	0.144881340479904\\
67.25	0.09666	0.218791731657935	0.218791731657935\\
67.25	0.10032	0.297022185204337	0.297022185204337\\
67.25	0.10398	0.379572701119115	0.379572701119115\\
67.25	0.10764	0.46644327940226	0.46644327940226\\
67.25	0.1113	0.557633920053781	0.557633920053781\\
67.25	0.11496	0.653144623073674	0.653144623073674\\
67.25	0.11862	0.752975388461934	0.752975388461934\\
67.25	0.12228	0.85712621621857	0.85712621621857\\
67.25	0.12594	0.965597106343577	0.965597106343577\\
67.25	0.1296	1.07838805883695	1.07838805883695\\
67.25	0.13326	1.1954990736987	1.1954990736987\\
67.25	0.13692	1.31693015092882	1.31693015092882\\
67.25	0.14058	1.44268129052732	1.44268129052732\\
67.25	0.14424	1.57275249249418	1.57275249249418\\
67.25	0.1479	1.70714375682942	1.70714375682942\\
67.25	0.15156	1.84585508353303	1.84585508353303\\
67.25	0.15522	1.988886472605	1.988886472605\\
67.25	0.15888	2.13623792404536	2.13623792404536\\
67.25	0.16254	2.28790943785408	2.28790943785408\\
67.25	0.1662	2.44390101403117	2.44390101403117\\
67.25	0.16986	2.60421265257664	2.60421265257664\\
67.25	0.17352	2.76884435349048	2.76884435349048\\
67.25	0.17718	2.93779611677269	2.93779611677269\\
67.25	0.18084	3.11106794242327	3.11106794242327\\
67.25	0.1845	3.28865983044221	3.28865983044221\\
67.25	0.18816	3.47057178082954	3.47057178082954\\
67.25	0.19182	3.65680379358523	3.65680379358523\\
67.25	0.19548	3.8473558687093	3.8473558687093\\
67.25	0.19914	4.04222800620174	4.04222800620174\\
67.25	0.2028	4.24142020606255	4.24142020606255\\
67.25	0.20646	4.44493246829173	4.44493246829173\\
67.25	0.21012	4.65276479288928	4.65276479288928\\
67.25	0.21378	4.8649171798552	4.8649171798552\\
67.25	0.21744	5.0813896291895	5.0813896291895\\
67.25	0.2211	5.30218214089217	5.30218214089217\\
67.25	0.22476	5.52729471496321	5.52729471496321\\
67.25	0.22842	5.75672735140263	5.75672735140263\\
67.25	0.23208	5.9904800502104	5.9904800502104\\
67.25	0.23574	6.22855281138656	6.22855281138656\\
67.25	0.2394	6.47094563493108	6.47094563493108\\
67.25	0.24306	6.71765852084399	6.71765852084399\\
67.25	0.24672	6.96869146912525	6.96869146912525\\
67.25	0.25038	7.2240444797749	7.2240444797749\\
67.25	0.25404	7.48371755279292	7.48371755279292\\
67.25	0.2577	7.7477106881793	7.7477106881793\\
67.25	0.26136	8.01602388593405	8.01602388593405\\
67.25	0.26502	8.28865714605718	8.28865714605718\\
67.25	0.26868	8.56561046854867	8.56561046854867\\
67.25	0.27234	8.84688385340855	8.84688385340855\\
67.25	0.276	9.13247730063679	9.13247730063679\\
67.625	0.093	0.119687647956346	0.119687647956346\\
67.625	0.09666	0.197546220686811	0.197546220686811\\
67.625	0.10032	0.279724855785656	0.279724855785656\\
67.625	0.10398	0.366223553252867	0.366223553252867\\
67.625	0.10764	0.457042313088452	0.457042313088452\\
67.625	0.1113	0.552181135292407	0.552181135292407\\
67.625	0.11496	0.651640019864734	0.651640019864734\\
67.625	0.11862	0.755418966805433	0.755418966805433\\
67.625	0.12228	0.863517976114503	0.863517976114503\\
67.625	0.12594	0.975937047791949	0.975937047791949\\
67.625	0.1296	1.09267618183776	1.09267618183776\\
67.625	0.13326	1.21373537825195	1.21373537825195\\
67.625	0.13692	1.3391146370345	1.3391146370345\\
67.625	0.14058	1.46881395818543	1.46881395818543\\
67.625	0.14424	1.60283334170473	1.60283334170473\\
67.625	0.1479	1.74117278759241	1.74117278759241\\
67.625	0.15156	1.88383229584845	1.88383229584845\\
67.625	0.15522	2.03081186647286	2.03081186647286\\
67.625	0.15888	2.18211149946565	2.18211149946565\\
67.625	0.16254	2.33773119482681	2.33773119482681\\
67.625	0.1662	2.49767095255634	2.49767095255634\\
67.625	0.16986	2.66193077265424	2.66193077265424\\
67.625	0.17352	2.83051065512052	2.83051065512052\\
67.625	0.17718	3.00341059995516	3.00341059995516\\
67.625	0.18084	3.18063060715818	3.18063060715818\\
67.625	0.1845	3.36217067672956	3.36217067672956\\
67.625	0.18816	3.54803080866933	3.54803080866933\\
67.625	0.19182	3.73821100297745	3.73821100297745\\
67.625	0.19548	3.93271125965396	3.93271125965396\\
67.625	0.19914	4.13153157869884	4.13153157869884\\
67.625	0.2028	4.33467196011208	4.33467196011208\\
67.625	0.20646	4.5421324038937	4.5421324038937\\
67.625	0.21012	4.75391291004369	4.75391291004369\\
67.625	0.21378	4.97001347856205	4.97001347856205\\
67.625	0.21744	5.19043410944878	5.19043410944878\\
67.625	0.2211	5.41517480270389	5.41517480270389\\
67.625	0.22476	5.64423555832737	5.64423555832737\\
67.625	0.22842	5.87761637631921	5.87761637631921\\
67.625	0.23208	6.11531725667943	6.11531725667943\\
67.625	0.23574	6.35733819940802	6.35733819940802\\
67.625	0.2394	6.60367920450498	6.60367920450498\\
67.625	0.24306	6.85434027197032	6.85434027197032\\
67.625	0.24672	7.10932140180403	7.10932140180403\\
67.625	0.25038	7.3686225940061	7.3686225940061\\
67.625	0.25404	7.63224384857655	7.63224384857655\\
67.625	0.2577	7.90018516551538	7.90018516551538\\
67.625	0.26136	8.17244654482257	8.17244654482257\\
67.625	0.26502	8.44902798649813	8.44902798649813\\
67.625	0.26868	8.72992949054206	8.72992949054206\\
67.625	0.27234	9.01515105695437	9.01515105695437\\
67.625	0.276	9.30469268573506	9.30469268573506\\
68	0.093	0.0934487335857723	0.0934487335857723\\
68	0.09666	0.175255487868676	0.175255487868676\\
68	0.10032	0.261382304519955	0.261382304519955\\
68	0.10398	0.351829183539605	0.351829183539605\\
68	0.10764	0.446596124927624	0.446596124927624\\
68	0.1113	0.545683128684015	0.545683128684015\\
68	0.11496	0.64909019480878	0.64909019480878\\
68	0.11862	0.756817323301913	0.756817323301913\\
68	0.12228	0.868864514163422	0.868864514163422\\
68	0.12594	0.985231767393302	0.985231767393302\\
68	0.1296	1.10591908299155	1.10591908299155\\
68	0.13326	1.23092646095817	1.23092646095817\\
68	0.13692	1.36025390129317	1.36025390129317\\
68	0.14058	1.49390140399653	1.49390140399653\\
68	0.14424	1.63186896906827	1.63186896906827\\
68	0.1479	1.77415659650838	1.77415659650838\\
68	0.15156	1.92076428631686	1.92076428631686\\
68	0.15522	2.07169203849371	2.07169203849371\\
68	0.15888	2.22693985303893	2.22693985303893\\
68	0.16254	2.38650772995252	2.38650772995252\\
68	0.1662	2.55039566923449	2.55039566923449\\
68	0.16986	2.71860367088483	2.71860367088483\\
68	0.17352	2.89113173490354	2.89113173490354\\
68	0.17718	3.06797986129063	3.06797986129063\\
68	0.18084	3.24914805004608	3.24914805004608\\
68	0.1845	3.4346363011699	3.4346363011699\\
68	0.18816	3.6244446146621	3.6244446146621\\
68	0.19182	3.81857299052266	3.81857299052266\\
68	0.19548	4.01702142875161	4.01702142875161\\
68	0.19914	4.21978992934892	4.21978992934892\\
68	0.2028	4.4268784923146	4.4268784923146\\
68	0.20646	4.63828711764866	4.63828711764866\\
68	0.21012	4.85401580535108	4.85401580535108\\
68	0.21378	5.07406455542188	5.07406455542188\\
68	0.21744	5.29843336786104	5.29843336786104\\
68	0.2211	5.52712224266859	5.52712224266859\\
68	0.22476	5.7601311798445	5.7601311798445\\
68	0.22842	5.99746017938879	5.99746017938879\\
68	0.23208	6.23910924130143	6.23910924130143\\
68	0.23574	6.48507836558246	6.48507836558246\\
68	0.2394	6.73536755223187	6.73536755223187\\
68	0.24306	6.98997680124963	6.98997680124963\\
68	0.24672	7.24890611263578	7.24890611263578\\
68	0.25038	7.5121554863903	7.5121554863903\\
68	0.25404	7.77972492251318	7.77972492251318\\
68	0.2577	8.05161442100444	8.05161442100444\\
68	0.26136	8.32782398186406	8.32782398186406\\
68	0.26502	8.60835360509206	8.60835360509206\\
68	0.26868	8.89320329068843	8.89320329068843\\
68	0.27234	9.18237303865318	9.18237303865318\\
68	0.276	9.4758628489863	9.4758628489863\\
68.375	0.093	0.066164597368177	0.066164597368177\\
68.375	0.09666	0.151919533203515	0.151919533203515\\
68.375	0.10032	0.241994531407229	0.241994531407229\\
68.375	0.10398	0.336389591979314	0.336389591979314\\
68.375	0.10764	0.435104714919771	0.435104714919771\\
68.375	0.1113	0.5381399002286	0.5381399002286\\
68.375	0.11496	0.6454951479058	0.6454951479058\\
68.375	0.11862	0.757170457951371	0.757170457951371\\
68.375	0.12228	0.873165830365315	0.873165830365315\\
68.375	0.12594	0.99348126514763	0.99348126514763\\
68.375	0.1296	1.11811676229832	1.11811676229832\\
68.375	0.13326	1.24707232181737	1.24707232181737\\
68.375	0.13692	1.38034794370481	1.38034794370481\\
68.375	0.14058	1.51794362796061	1.51794362796061\\
68.375	0.14424	1.65985937458478	1.65985937458478\\
68.375	0.1479	1.80609518357732	1.80609518357732\\
68.375	0.15156	1.95665105493824	1.95665105493824\\
68.375	0.15522	2.11152698866753	2.11152698866753\\
68.375	0.15888	2.27072298476519	2.27072298476519\\
68.375	0.16254	2.43423904323122	2.43423904323122\\
68.375	0.1662	2.60207516406562	2.60207516406562\\
68.375	0.16986	2.7742313472684	2.7742313472684\\
68.375	0.17352	2.95070759283954	2.95070759283954\\
68.375	0.17718	3.13150390077906	3.13150390077906\\
68.375	0.18084	3.31662027108696	3.31662027108696\\
68.375	0.1845	3.50605670376321	3.50605670376321\\
68.375	0.18816	3.69981319880784	3.69981319880784\\
68.375	0.19182	3.89788975622085	3.89788975622085\\
68.375	0.19548	4.10028637600222	4.10028637600222\\
68.375	0.19914	4.30700305815197	4.30700305815197\\
68.375	0.2028	4.51803980267009	4.51803980267009\\
68.375	0.20646	4.73339660955658	4.73339660955658\\
68.375	0.21012	4.95307347881144	4.95307347881144\\
68.375	0.21378	5.17707041043468	5.17707041043468\\
68.375	0.21744	5.40538740442628	5.40538740442628\\
68.375	0.2211	5.63802446078626	5.63802446078626\\
68.375	0.22476	5.87498157951461	5.87498157951461\\
68.375	0.22842	6.11625876061134	6.11625876061134\\
68.375	0.23208	6.36185600407642	6.36185600407642\\
68.375	0.23574	6.61177330990989	6.61177330990989\\
68.375	0.2394	6.86601067811172	6.86601067811172\\
68.375	0.24306	7.12456810868193	7.12456810868193\\
68.375	0.24672	7.3874456016205	7.3874456016205\\
68.375	0.25038	7.65464315692745	7.65464315692745\\
68.375	0.25404	7.92616077460278	7.92616077460278\\
68.375	0.2577	8.20199845464647	8.20199845464647\\
68.375	0.26136	8.48215619705853	8.48215619705853\\
68.375	0.26502	8.76663400183898	8.76663400183898\\
68.375	0.26868	9.05543186898778	9.05543186898778\\
68.375	0.27234	9.34854979850496	9.34854979850496\\
68.375	0.276	9.64598779039051	9.64598779039051\\
68.75	0.093	0.0378352393035597	0.0378352393035597\\
68.75	0.09666	0.127538356691336	0.127538356691336\\
68.75	0.10032	0.221561536447485	0.221561536447485\\
68.75	0.10398	0.319904778572008	0.319904778572008\\
68.75	0.10764	0.4225680830649	0.4225680830649\\
68.75	0.1113	0.529551449926163	0.529551449926163\\
68.75	0.11496	0.640854879155802	0.640854879155802\\
68.75	0.11862	0.756478370753808	0.756478370753808\\
68.75	0.12228	0.876421924720189	0.876421924720189\\
68.75	0.12594	1.00068554105494	1.00068554105494\\
68.75	0.1296	1.12926921975806	1.12926921975806\\
68.75	0.13326	1.26217296082956	1.26217296082956\\
68.75	0.13692	1.39939676426942	1.39939676426942\\
68.75	0.14058	1.54094063007767	1.54094063007767\\
68.75	0.14424	1.68680455825427	1.68680455825427\\
68.75	0.1479	1.83698854879925	1.83698854879925\\
68.75	0.15156	1.99149260171261	1.99149260171261\\
68.75	0.15522	2.15031671699433	2.15031671699433\\
68.75	0.15888	2.31346089464443	2.31346089464443\\
68.75	0.16254	2.48092513466289	2.48092513466289\\
68.75	0.1662	2.65270943704973	2.65270943704973\\
68.75	0.16986	2.82881380180494	2.82881380180494\\
68.75	0.17352	3.00923822892852	3.00923822892852\\
68.75	0.17718	3.19398271842048	3.19398271842048\\
68.75	0.18084	3.38304727028081	3.38304727028081\\
68.75	0.1845	3.5764318845095	3.5764318845095\\
68.75	0.18816	3.77413656110657	3.77413656110657\\
68.75	0.19182	3.97616130007201	3.97616130007201\\
68.75	0.19548	4.18250610140583	4.18250610140583\\
68.75	0.19914	4.39317096510801	4.39317096510801\\
68.75	0.2028	4.60815589117856	4.60815589117856\\
68.75	0.20646	4.82746087961749	4.82746087961749\\
68.75	0.21012	5.05108593042479	5.05108593042479\\
68.75	0.21378	5.27903104360046	5.27903104360046\\
68.75	0.21744	5.5112962191445	5.5112962191445\\
68.75	0.2211	5.74788145705692	5.74788145705692\\
68.75	0.22476	5.9887867573377	5.9887867573377\\
68.75	0.22842	6.23401211998686	6.23401211998686\\
68.75	0.23208	6.48355754500438	6.48355754500438\\
68.75	0.23574	6.73742303239028	6.73742303239028\\
68.75	0.2394	6.99560858214456	6.99560858214456\\
68.75	0.24306	7.2581141942672	7.2581141942672\\
68.75	0.24672	7.52493986875822	7.52493986875822\\
68.75	0.25038	7.79608560561761	7.79608560561761\\
68.75	0.25404	8.07155140484536	8.07155140484536\\
68.75	0.2577	8.35133726644149	8.35133726644149\\
68.75	0.26136	8.63544319040599	8.63544319040599\\
68.75	0.26502	8.92386917673886	8.92386917673886\\
68.75	0.26868	9.21661522544011	9.21661522544011\\
68.75	0.27234	9.51368133650972	9.51368133650972\\
68.75	0.276	9.81506750994772	9.81506750994772\\
69.125	0.093	0.00846065939192386	0.00846065939192386\\
69.125	0.09666	0.102111958332135	0.102111958332135\\
69.125	0.10032	0.200083319640722	0.200083319640722\\
69.125	0.10398	0.302374743317684	0.302374743317684\\
69.125	0.10764	0.40898622936301	0.40898622936301\\
69.125	0.1113	0.519917777776712	0.519917777776712\\
69.125	0.11496	0.635169388558785	0.635169388558785\\
69.125	0.11862	0.754741061709229	0.754741061709229\\
69.125	0.12228	0.878632797228045	0.878632797228045\\
69.125	0.12594	1.00684459511523	1.00684459511523\\
69.125	0.1296	1.13937645537079	1.13937645537079\\
69.125	0.13326	1.27622837799472	1.27622837799472\\
69.125	0.13692	1.41740036298703	1.41740036298703\\
69.125	0.14058	1.5628924103477	1.5628924103477\\
69.125	0.14424	1.71270452007674	1.71270452007674\\
69.125	0.1479	1.86683669217417	1.86683669217417\\
69.125	0.15156	2.02528892663995	2.02528892663995\\
69.125	0.15522	2.18806122347411	2.18806122347411\\
69.125	0.15888	2.35515358267665	2.35515358267665\\
69.125	0.16254	2.52656600424755	2.52656600424755\\
69.125	0.1662	2.70229848818683	2.70229848818683\\
69.125	0.16986	2.88235103449447	2.88235103449447\\
69.125	0.17352	3.06672364317049	3.06672364317049\\
69.125	0.17718	3.25541631421488	3.25541631421488\\
69.125	0.18084	3.44842904762764	3.44842904762764\\
69.125	0.1845	3.64576184340877	3.64576184340877\\
69.125	0.18816	3.84741470155828	3.84741470155828\\
69.125	0.19182	4.05338762207616	4.05338762207616\\
69.125	0.19548	4.26368060496241	4.26368060496241\\
69.125	0.19914	4.47829365021703	4.47829365021703\\
69.125	0.2028	4.69722675784002	4.69722675784002\\
69.125	0.20646	4.92047992783139	4.92047992783139\\
69.125	0.21012	5.14805316019111	5.14805316019111\\
69.125	0.21378	5.37994645491922	5.37994645491922\\
69.125	0.21744	5.61615981201571	5.61615981201571\\
69.125	0.2211	5.85669323148055	5.85669323148055\\
69.125	0.22476	6.10154671331378	6.10154671331378\\
69.125	0.22842	6.35072025751538	6.35072025751538\\
69.125	0.23208	6.60421386408533	6.60421386408533\\
69.125	0.23574	6.86202753302367	6.86202753302367\\
69.125	0.2394	7.12416126433038	7.12416126433038\\
69.125	0.24306	7.39061505800546	7.39061505800546\\
69.125	0.24672	7.66138891404891	7.66138891404891\\
69.125	0.25038	7.93648283246073	7.93648283246073\\
69.125	0.25404	8.21589681324093	8.21589681324093\\
69.125	0.2577	8.49963085638949	8.49963085638949\\
69.125	0.26136	8.78768496190643	8.78768496190643\\
69.125	0.26502	9.08005912979174	9.08005912979174\\
69.125	0.26868	9.37675336004542	9.37675336004542\\
69.125	0.27234	9.67776765266747	9.67776765266747\\
69.125	0.276	9.9831020076579	9.9831020076579\\
69.5	0.093	-0.0219591423667305	-0.0219591423667305\\
69.5	0.09666	0.0756403381259192	0.0756403381259192\\
69.5	0.10032	0.177559880986941	0.177559880986941\\
69.5	0.10398	0.283799486216337	0.283799486216337\\
69.5	0.10764	0.394359153814102	0.394359153814102\\
69.5	0.1113	0.509238883780238	0.509238883780238\\
69.5	0.11496	0.628438676114749	0.628438676114749\\
69.5	0.11862	0.751958530817628	0.751958530817628\\
69.5	0.12228	0.879798447888883	0.879798447888883\\
69.5	0.12594	1.01195842732851	1.01195842732851\\
69.5	0.1296	1.1484384691365	1.1484384691365\\
69.5	0.13326	1.28923857331287	1.28923857331287\\
69.5	0.13692	1.43435873985761	1.43435873985761\\
69.5	0.14058	1.58379896877072	1.58379896877072\\
69.5	0.14424	1.7375592600522	1.7375592600522\\
69.5	0.1479	1.89563961370206	1.89563961370206\\
69.5	0.15156	2.05804002972028	2.05804002972028\\
69.5	0.15522	2.22476050810688	2.22476050810688\\
69.5	0.15888	2.39580104886185	2.39580104886185\\
69.5	0.16254	2.57116165198518	2.57116165198518\\
69.5	0.1662	2.7508423174769	2.7508423174769\\
69.5	0.16986	2.93484304533698	2.93484304533698\\
69.5	0.17352	3.12316383556544	3.12316383556544\\
69.5	0.17718	3.31580468816227	3.31580468816227\\
69.5	0.18084	3.51276560312746	3.51276560312746\\
69.5	0.1845	3.71404658046103	3.71404658046103\\
69.5	0.18816	3.91964762016297	3.91964762016297\\
69.5	0.19182	4.12956872223329	4.12956872223329\\
69.5	0.19548	4.34380988667197	4.34380988667197\\
69.5	0.19914	4.56237111347903	4.56237111347903\\
69.5	0.2028	4.78525240265446	4.78525240265446\\
69.5	0.20646	5.01245375419826	5.01245375419826\\
69.5	0.21012	5.24397516811043	5.24397516811043\\
69.5	0.21378	5.47981664439097	5.47981664439097\\
69.5	0.21744	5.71997818303988	5.71997818303988\\
69.5	0.2211	5.96445978405717	5.96445978405717\\
69.5	0.22476	6.21326144744283	6.21326144744283\\
69.5	0.22842	6.46638317319687	6.46638317319687\\
69.5	0.23208	6.72382496131926	6.72382496131926\\
69.5	0.23574	6.98558681181003	6.98558681181003\\
69.5	0.2394	7.25166872466917	7.25166872466917\\
69.5	0.24306	7.52207069989669	7.52207069989669\\
69.5	0.24672	7.79679273749258	7.79679273749258\\
69.5	0.25038	8.07583483745684	8.07583483745684\\
69.5	0.25404	8.35919699978948	8.35919699978948\\
69.5	0.2577	8.64687922449048	8.64687922449048\\
69.5	0.26136	8.93888151155985	8.93888151155985\\
69.5	0.26502	9.23520386099759	9.23520386099759\\
69.5	0.26868	9.53584627280371	9.53584627280371\\
69.5	0.27234	9.8408087469782	9.8408087469782\\
69.5	0.276	10.1500912835211	10.1500912835211\\
69.875	0.093	-0.0534241659724104	-0.0534241659724104\\
69.875	0.09666	0.0481234960726775	0.0481234960726775\\
69.875	0.10032	0.153991220486137	0.153991220486137\\
69.875	0.10398	0.264179007267968	0.264179007267968\\
69.875	0.10764	0.378686856418168	0.378686856418168\\
69.875	0.1113	0.497514767936742	0.497514767936742\\
69.875	0.11496	0.620662741823688	0.620662741823688\\
69.875	0.11862	0.748130778079002	0.748130778079002\\
69.875	0.12228	0.879918876702695	0.879918876702695\\
69.875	0.12594	1.01602703769476	1.01602703769476\\
69.875	0.1296	1.15645526105518	1.15645526105518\\
69.875	0.13326	1.30120354678399	1.30120354678399\\
69.875	0.13692	1.45027189488117	1.45027189488117\\
69.875	0.14058	1.60366030534671	1.60366030534671\\
69.875	0.14424	1.76136877818063	1.76136877818063\\
69.875	0.1479	1.92339731338292	1.92339731338292\\
69.875	0.15156	2.08974591095358	2.08974591095358\\
69.875	0.15522	2.26041457089261	2.26041457089261\\
69.875	0.15888	2.43540329320002	2.43540329320002\\
69.875	0.16254	2.61471207787579	2.61471207787579\\
69.875	0.1662	2.79834092491994	2.79834092491994\\
69.875	0.16986	2.98628983433246	2.98628983433246\\
69.875	0.17352	3.17855880611335	3.17855880611335\\
69.875	0.17718	3.37514784026262	3.37514784026262\\
69.875	0.18084	3.57605693678026	3.57605693678026\\
69.875	0.1845	3.78128609566626	3.78128609566626\\
69.875	0.18816	3.99083531692064	3.99083531692064\\
69.875	0.19182	4.20470460054339	4.20470460054339\\
69.875	0.19548	4.42289394653451	4.42289394653451\\
69.875	0.19914	4.645403354894	4.645403354894\\
69.875	0.2028	4.87223282562187	4.87223282562187\\
69.875	0.20646	5.10338235871811	5.10338235871811\\
69.875	0.21012	5.33885195418271	5.33885195418271\\
69.875	0.21378	5.57864161201569	5.57864161201569\\
69.875	0.21744	5.82275133221704	5.82275133221704\\
69.875	0.2211	6.07118111478676	6.07118111478676\\
69.875	0.22476	6.32393095972485	6.32393095972485\\
69.875	0.22842	6.58100086703133	6.58100086703133\\
69.875	0.23208	6.84239083670616	6.84239083670616\\
69.875	0.23574	7.10810086874937	7.10810086874937\\
69.875	0.2394	7.37813096316095	7.37813096316095\\
69.875	0.24306	7.6524811199409	7.6524811199409\\
69.875	0.24672	7.93115133908923	7.93115133908923\\
69.875	0.25038	8.21414162060593	8.21414162060593\\
69.875	0.25404	8.50145196449099	8.50145196449099\\
69.875	0.2577	8.79308237074444	8.79308237074444\\
69.875	0.26136	9.08903283936624	9.08903283936624\\
69.875	0.26502	9.38930337035641	9.38930337035641\\
69.875	0.26868	9.69389396371498	9.69389396371498\\
69.875	0.27234	10.0028046194419	10.0028046194419\\
69.875	0.276	10.3160353375372	10.3160353375372\\
70.25	0.093	-0.0859344114251053	-0.0859344114251053\\
70.25	0.09666	0.0195614321724138	0.0195614321724138\\
70.25	0.10032	0.129377338138312	0.129377338138312\\
70.25	0.10398	0.243513306472577	0.243513306472577\\
70.25	0.10764	0.361969337175215	0.361969337175215\\
70.25	0.1113	0.484745430246228	0.484745430246228\\
70.25	0.11496	0.611841585685609	0.611841585685609\\
70.25	0.11862	0.743257803493361	0.743257803493361\\
70.25	0.12228	0.878994083669488	0.878994083669488\\
70.25	0.12594	1.01905042621398	1.01905042621398\\
70.25	0.1296	1.16342683112685	1.16342683112685\\
70.25	0.13326	1.31212329840809	1.31212329840809\\
70.25	0.13692	1.4651398280577	1.4651398280577\\
70.25	0.14058	1.62247642007569	1.62247642007569\\
70.25	0.14424	1.78413307446204	1.78413307446204\\
70.25	0.1479	1.95010979121677	1.95010979121677\\
70.25	0.15156	2.12040657033987	2.12040657033987\\
70.25	0.15522	2.29502341183133	2.29502341183133\\
70.25	0.15888	2.47396031569118	2.47396031569118\\
70.25	0.16254	2.65721728191939	2.65721728191939\\
70.25	0.1662	2.84479431051597	2.84479431051597\\
70.25	0.16986	3.03669140148093	3.03669140148093\\
70.25	0.17352	3.23290855481426	3.23290855481426\\
70.25	0.17718	3.43344577051596	3.43344577051596\\
70.25	0.18084	3.63830304858603	3.63830304858603\\
70.25	0.1845	3.84748038902447	3.84748038902447\\
70.25	0.18816	4.06097779183129	4.06097779183129\\
70.25	0.19182	4.27879525700647	4.27879525700647\\
70.25	0.19548	4.50093278455003	4.50093278455003\\
70.25	0.19914	4.72739037446196	4.72739037446196\\
70.25	0.2028	4.95816802674226	4.95816802674226\\
70.25	0.20646	5.19326574139094	5.19326574139094\\
70.25	0.21012	5.43268351840797	5.43268351840797\\
70.25	0.21378	5.67642135779339	5.67642135779339\\
70.25	0.21744	5.92447925954718	5.92447925954718\\
70.25	0.2211	6.17685722366934	6.17685722366934\\
70.25	0.22476	6.43355525015987	6.43355525015987\\
70.25	0.22842	6.69457333901878	6.69457333901878\\
70.25	0.23208	6.95991149024604	6.95991149024604\\
70.25	0.23574	7.22956970384169	7.22956970384169\\
70.25	0.2394	7.50354797980571	7.50354797980571\\
70.25	0.24306	7.7818463181381	7.7818463181381\\
70.25	0.24672	8.06446471883885	8.06446471883885\\
70.25	0.25038	8.35140318190798	8.35140318190798\\
70.25	0.25404	8.6426617073455	8.6426617073455\\
70.25	0.2577	8.93824029515137	8.93824029515137\\
70.25	0.26136	9.23813894532561	9.23813894532561\\
70.25	0.26502	9.54235765786824	9.54235765786824\\
70.25	0.26868	9.85089643277921	9.85089643277921\\
70.25	0.27234	10.1637552700586	10.1637552700586\\
70.25	0.276	10.4809341697063	10.4809341697063\\
70.625	0.093	-0.119489878724826	-0.119489878724826\\
70.625	0.09666	-0.0100458535748684	-0.0100458535748684\\
70.625	0.10032	0.103718233943464	0.103718233943464\\
70.625	0.10398	0.221802383830168	0.221802383830168\\
70.625	0.10764	0.344206596085241	0.344206596085241\\
70.625	0.1113	0.470930870708688	0.470930870708688\\
70.625	0.11496	0.601975207700507	0.601975207700507\\
70.625	0.11862	0.737339607060694	0.737339607060694\\
70.625	0.12228	0.877024068789256	0.877024068789256\\
70.625	0.12594	1.02102859288619	1.02102859288619\\
70.625	0.1296	1.1693531793515	1.1693531793515\\
70.625	0.13326	1.32199782818517	1.32199782818517\\
70.625	0.13692	1.47896253938722	1.47896253938722\\
70.625	0.14058	1.64024731295764	1.64024731295764\\
70.625	0.14424	1.80585214889643	1.80585214889643\\
70.625	0.1479	1.97577704720359	1.97577704720359\\
70.625	0.15156	2.15002200787913	2.15002200787913\\
70.625	0.15522	2.32858703092303	2.32858703092303\\
70.625	0.15888	2.51147211633531	2.51147211633531\\
70.625	0.16254	2.69867726411596	2.69867726411596\\
70.625	0.1662	2.89020247426498	2.89020247426498\\
70.625	0.16986	3.08604774678237	3.08604774678237\\
70.625	0.17352	3.28621308166814	3.28621308166814\\
70.625	0.17718	3.49069847892227	3.49069847892227\\
70.625	0.18084	3.69950393854478	3.69950393854478\\
70.625	0.1845	3.91262946053565	3.91262946053565\\
70.625	0.18816	4.13007504489491	4.13007504489491\\
70.625	0.19182	4.35184069162253	4.35184069162253\\
70.625	0.19548	4.57792640071853	4.57792640071853\\
70.625	0.19914	4.80833217218289	4.80833217218289\\
70.625	0.2028	5.04305800601563	5.04305800601563\\
70.625	0.20646	5.28210390221674	5.28210390221674\\
70.625	0.21012	5.52546986078622	5.52546986078622\\
70.625	0.21378	5.77315588172407	5.77315588172407\\
70.625	0.21744	6.02516196503029	6.02516196503029\\
70.625	0.2211	6.28148811070489	6.28148811070489\\
70.625	0.22476	6.54213431874786	6.54213431874786\\
70.625	0.22842	6.8071005891592	6.8071005891592\\
70.625	0.23208	7.0763869219389	7.0763869219389\\
70.625	0.23574	7.34999331708699	7.34999331708699\\
70.625	0.2394	7.62791977460344	7.62791977460344\\
70.625	0.24306	7.91016629448827	7.91016629448827\\
70.625	0.24672	8.19673287674146	8.19673287674146\\
70.625	0.25038	8.48761952136303	8.48761952136303\\
70.625	0.25404	8.78282622835297	8.78282622835297\\
70.625	0.2577	9.08235299771129	9.08235299771129\\
70.625	0.26136	9.38619982943796	9.38619982943796\\
70.625	0.26502	9.69436672353302	9.69436672353302\\
70.625	0.26868	10.0068536799964	10.0068536799964\\
70.625	0.27234	10.3236606988282	10.3236606988282\\
70.625	0.276	10.6447877800284	10.6447877800284\\
71	0.093	-0.154090567871561	-0.154090567871561\\
71	0.09666	-0.0406983611691656	-0.0406983611691656\\
71	0.10032	0.0770139079016019	0.0770139079016019\\
71	0.10398	0.199046239340744	0.199046239340744\\
71	0.10764	0.325398633148251	0.325398633148251\\
71	0.1113	0.456071089324137	0.456071089324137\\
71	0.11496	0.59106360786839	0.59106360786839\\
71	0.11862	0.730376188781015	0.730376188781015\\
71	0.12228	0.874008832062013	0.874008832062013\\
71	0.12594	1.02196153771138	1.02196153771138\\
71	0.1296	1.17423430572912	1.17423430572912\\
71	0.13326	1.33082713611524	1.33082713611524\\
71	0.13692	1.49174002886972	1.49174002886972\\
71	0.14058	1.65697298399258	1.65697298399258\\
71	0.14424	1.8265260014838	1.8265260014838\\
71	0.1479	2.0003990813434	2.0003990813434\\
71	0.15156	2.17859222357138	2.17859222357138\\
71	0.15522	2.36110542816771	2.36110542816771\\
71	0.15888	2.54793869513244	2.54793869513244\\
71	0.16254	2.73909202446551	2.73909202446551\\
71	0.1662	2.93456541616697	2.93456541616697\\
71	0.16986	3.13435887023681	3.13435887023681\\
71	0.17352	3.338472386675	3.338472386675\\
71	0.17718	3.54690596548158	3.54690596548158\\
71	0.18084	3.75965960665652	3.75965960665652\\
71	0.1845	3.97673331019983	3.97673331019983\\
71	0.18816	4.19812707611152	4.19812707611152\\
71	0.19182	4.42384090439158	4.42384090439158\\
71	0.19548	4.65387479504001	4.65387479504001\\
71	0.19914	4.88822874805681	4.88822874805681\\
71	0.2028	5.12690276344199	5.12690276344199\\
71	0.20646	5.36989684119553	5.36989684119553\\
71	0.21012	5.61721098131744	5.61721098131744\\
71	0.21378	5.86884518380774	5.86884518380774\\
71	0.21744	6.12479944866639	6.12479944866639\\
71	0.2211	6.38507377589342	6.38507377589342\\
71	0.22476	6.64966816548883	6.64966816548883\\
71	0.22842	6.91858261745261	6.91858261745261\\
71	0.23208	7.19181713178475	7.19181713178475\\
71	0.23574	7.46937170848527	7.46937170848527\\
71	0.2394	7.75124634755415	7.75124634755415\\
71	0.24306	8.03744104899142	8.03744104899142\\
71	0.24672	8.32795581279705	8.32795581279705\\
71	0.25038	8.62279063897106	8.62279063897106\\
71	0.25404	8.92194552751343	8.92194552751343\\
71	0.2577	9.22542047842419	9.22542047842419\\
71	0.26136	9.5332154917033	9.5332154917033\\
71	0.26502	9.84533056735079	9.84533056735079\\
71	0.26868	10.1617657053666	10.1617657053666\\
71	0.27234	10.4825209057509	10.4825209057509\\
71	0.276	10.8075961685035	10.8075961685035\\
71.375	0.093	-0.189736478865319	-0.189736478865319\\
71.375	0.09666	-0.0723960906104884	-0.0723960906104884\\
71.375	0.10032	0.0492643600127174	0.0492643600127174\\
71.375	0.10398	0.175244873004294	0.175244873004294\\
71.375	0.10764	0.305545448364239	0.305545448364239\\
71.375	0.1113	0.44016608609256	0.44016608609256\\
71.375	0.11496	0.579106786189252	0.579106786189252\\
71.375	0.11862	0.722367548654312	0.722367548654312\\
71.375	0.12228	0.869948373487747	0.869948373487747\\
71.375	0.12594	1.02184926068955	1.02184926068955\\
71.375	0.1296	1.17807021025973	1.17807021025973\\
71.375	0.13326	1.33861122219828	1.33861122219828\\
71.375	0.13692	1.5034722965052	1.5034722965052\\
71.375	0.14058	1.67265343318049	1.67265343318049\\
71.375	0.14424	1.84615463222416	1.84615463222416\\
71.375	0.1479	2.02397589363619	2.02397589363619\\
71.375	0.15156	2.2061172174166	2.2061172174166\\
71.375	0.15522	2.39257860356538	2.39257860356538\\
71.375	0.15888	2.58336005208254	2.58336005208254\\
71.375	0.16254	2.77846156296805	2.77846156296805\\
71.375	0.1662	2.97788313622195	2.97788313622195\\
71.375	0.16986	3.18162477184421	3.18162477184421\\
71.375	0.17352	3.38968646983485	3.38968646983485\\
71.375	0.17718	3.60206823019386	3.60206823019386\\
71.375	0.18084	3.81877005292124	3.81877005292124\\
71.375	0.1845	4.03979193801699	4.03979193801699\\
71.375	0.18816	4.26513388548111	4.26513388548111\\
71.375	0.19182	4.49479589531361	4.49479589531361\\
71.375	0.19548	4.72877796751447	4.72877796751447\\
71.375	0.19914	4.96708010208371	4.96708010208371\\
71.375	0.2028	5.20970229902132	5.20970229902132\\
71.375	0.20646	5.4566445583273	5.4566445583273\\
71.375	0.21012	5.70790688000166	5.70790688000166\\
71.375	0.21378	5.96348926404438	5.96348926404438\\
71.375	0.21744	6.22339171045548	6.22339171045548\\
71.375	0.2211	6.48761421923495	6.48761421923495\\
71.375	0.22476	6.75615679038279	6.75615679038279\\
71.375	0.22842	7.029019423899	7.029019423899\\
71.375	0.23208	7.30620211978357	7.30620211978357\\
71.375	0.23574	7.58770487803653	7.58770487803653\\
71.375	0.2394	7.87352769865786	7.87352769865786\\
71.375	0.24306	8.16367058164756	8.16367058164756\\
71.375	0.24672	8.45813352700562	8.45813352700562\\
71.375	0.25038	8.75691653473206	8.75691653473206\\
71.375	0.25404	9.06001960482688	9.06001960482688\\
71.375	0.2577	9.36744273729006	9.36744273729006\\
71.375	0.26136	9.67918593212162	9.67918593212162\\
71.375	0.26502	9.99524918932155	9.99524918932155\\
71.375	0.26868	10.3156325088898	10.3156325088898\\
71.375	0.27234	10.6403358908265	10.6403358908265\\
71.375	0.276	10.9693593351316	10.9693593351316\\
71.75	0.093	-0.226427611706095	-0.226427611706095\\
71.75	0.09666	-0.10513904189883	-0.10513904189883\\
71.75	0.10032	0.0204695902768108	0.0204695902768108\\
71.75	0.10398	0.150398284820826	0.150398284820826\\
71.75	0.10764	0.284647041733206	0.284647041733206\\
71.75	0.1113	0.423215861013965	0.423215861013965\\
71.75	0.11496	0.566104742663091	0.566104742663091\\
71.75	0.11862	0.713313686680589	0.713313686680589\\
71.75	0.12228	0.864842693066459	0.864842693066459\\
71.75	0.12594	1.0206917618207	1.0206917618207\\
71.75	0.1296	1.18086089294331	1.18086089294331\\
71.75	0.13326	1.3453500864343	1.3453500864343\\
71.75	0.13692	1.51415934229366	1.51415934229366\\
71.75	0.14058	1.68728866052139	1.68728866052139\\
71.75	0.14424	1.86473804111749	1.86473804111749\\
71.75	0.1479	2.04650748408196	2.04650748408196\\
71.75	0.15156	2.2325969894148	2.2325969894148\\
71.75	0.15522	2.42300655711601	2.42300655711601\\
71.75	0.15888	2.61773618718561	2.61773618718561\\
71.75	0.16254	2.81678587962356	2.81678587962356\\
71.75	0.1662	3.02015563442989	3.02015563442989\\
71.75	0.16986	3.22784545160459	3.22784545160459\\
71.75	0.17352	3.43985533114767	3.43985533114767\\
71.75	0.17718	3.65618527305912	3.65618527305912\\
71.75	0.18084	3.87683527733893	3.87683527733893\\
71.75	0.1845	4.10180534398711	4.10180534398711\\
71.75	0.18816	4.33109547300368	4.33109547300368\\
71.75	0.19182	4.5647056643886	4.5647056643886\\
71.75	0.19548	4.80263591814191	4.80263591814191\\
71.75	0.19914	5.04488623426359	5.04488623426359\\
71.75	0.2028	5.29145661275363	5.29145661275363\\
71.75	0.20646	5.54234705361205	5.54234705361205\\
71.75	0.21012	5.79755755683884	5.79755755683884\\
71.75	0.21378	6.057088122434	6.057088122434\\
71.75	0.21744	6.32093875039753	6.32093875039753\\
71.75	0.2211	6.58910944072943	6.58910944072943\\
71.75	0.22476	6.86160019342971	6.86160019342971\\
71.75	0.22842	7.13841100849836	7.13841100849836\\
71.75	0.23208	7.41954188593537	7.41954188593537\\
71.75	0.23574	7.70499282574077	7.70499282574077\\
71.75	0.2394	7.99476382791453	7.99476382791453\\
71.75	0.24306	8.28885489245666	8.28885489245666\\
71.75	0.24672	8.58726601936717	8.58726601936717\\
71.75	0.25038	8.88999720864605	8.88999720864605\\
71.75	0.25404	9.1970484602933	9.1970484602933\\
71.75	0.2577	9.50841977430892	9.50841977430892\\
71.75	0.26136	9.82411115069291	9.82411115069291\\
71.75	0.26502	10.1441225894453	10.1441225894453\\
71.75	0.26868	10.468454090566	10.468454090566\\
71.75	0.27234	10.7971056540551	10.7971056540551\\
71.75	0.276	11.1300772799126	11.1300772799126\\
72.125	0.093	-0.264163966393896	-0.264163966393896\\
72.125	0.09666	-0.138927215034193	-0.138927215034193\\
72.125	0.10032	-0.00937040130611422	-0.00937040130611422\\
72.125	0.10398	0.124506474790332	0.124506474790332\\
72.125	0.10764	0.26270341325515	0.26270341325515\\
72.125	0.1113	0.405220414088344	0.405220414088344\\
72.125	0.11496	0.552057477289909	0.552057477289909\\
72.125	0.11862	0.703214602859841	0.703214602859841\\
72.125	0.12228	0.85869179079815	0.85869179079815\\
72.125	0.12594	1.01848904110483	1.01848904110483\\
72.125	0.1296	1.18260635377988	1.18260635377988\\
72.125	0.13326	1.3510437288233	1.3510437288233\\
72.125	0.13692	1.52380116623509	1.52380116623509\\
72.125	0.14058	1.70087866601526	1.70087866601526\\
72.125	0.14424	1.88227622816379	1.88227622816379\\
72.125	0.1479	2.0679938526807	2.0679938526807\\
72.125	0.15156	2.25803153956598	2.25803153956598\\
72.125	0.15522	2.45238928881964	2.45238928881964\\
72.125	0.15888	2.65106710044166	2.65106710044166\\
72.125	0.16254	2.85406497443205	2.85406497443205\\
72.125	0.1662	3.06138291079082	3.06138291079082\\
72.125	0.16986	3.27302090951795	3.27302090951795\\
72.125	0.17352	3.48897897061347	3.48897897061347\\
72.125	0.17718	3.70925709407735	3.70925709407735\\
72.125	0.18084	3.9338552799096	3.9338552799096\\
72.125	0.1845	4.16277352811022	4.16277352811022\\
72.125	0.18816	4.39601183867922	4.39601183867922\\
72.125	0.19182	4.63357021161658	4.63357021161658\\
72.125	0.19548	4.87544864692233	4.87544864692233\\
72.125	0.19914	5.12164714459644	5.12164714459644\\
72.125	0.2028	5.37216570463892	5.37216570463892\\
72.125	0.20646	5.62700432704978	5.62700432704978\\
72.125	0.21012	5.88616301182899	5.88616301182899\\
72.125	0.21378	6.1496417589766	6.1496417589766\\
72.125	0.21744	6.41744056849256	6.41744056849256\\
72.125	0.2211	6.68955944037691	6.68955944037691\\
72.125	0.22476	6.96599837462962	6.96599837462962\\
72.125	0.22842	7.24675737125071	7.24675737125071\\
72.125	0.23208	7.53183643024015	7.53183643024015\\
72.125	0.23574	7.82123555159798	7.82123555159798\\
72.125	0.2394	8.11495473532418	8.11495473532418\\
72.125	0.24306	8.41299398141875	8.41299398141875\\
72.125	0.24672	8.7153532898817	8.7153532898817\\
72.125	0.25038	9.02203266071301	9.02203266071301\\
72.125	0.25404	9.3330320939127	9.3330320939127\\
72.125	0.2577	9.64835158948075	9.64835158948075\\
72.125	0.26136	9.96799114741718	9.96799114741718\\
72.125	0.26502	10.291950767722	10.291950767722\\
72.125	0.26868	10.6202304503952	10.6202304503952\\
72.125	0.27234	10.9528301954367	10.9528301954367\\
72.125	0.276	11.2897500028466	11.2897500028466\\
72.5	0.093	-0.302945542928716	-0.302945542928716\\
72.5	0.09666	-0.173760610016578	-0.173760610016578\\
72.5	0.10032	-0.0402556147360613	-0.0402556147360613\\
72.5	0.10398	0.0975694429128231	0.0975694429128231\\
72.5	0.10764	0.23971456293008	0.23971456293008\\
72.5	0.1113	0.386179745315708	0.386179745315708\\
72.5	0.11496	0.536964990069707	0.536964990069707\\
72.5	0.11862	0.692070297192078	0.692070297192078\\
72.5	0.12228	0.851495666682821	0.851495666682821\\
72.5	0.12594	1.01524109854194	1.01524109854194\\
72.5	0.1296	1.18330659276942	1.18330659276942\\
72.5	0.13326	1.35569214936528	1.35569214936528\\
72.5	0.13692	1.53239776832951	1.53239776832951\\
72.5	0.14058	1.71342344966211	1.71342344966211\\
72.5	0.14424	1.89876919336308	1.89876919336308\\
72.5	0.1479	2.08843499943243	2.08843499943243\\
72.5	0.15156	2.28242086787015	2.28242086787015\\
72.5	0.15522	2.48072679867623	2.48072679867623\\
72.5	0.15888	2.68335279185069	2.68335279185069\\
72.5	0.16254	2.89029884739352	2.89029884739352\\
72.5	0.1662	3.10156496530473	3.10156496530473\\
72.5	0.16986	3.3171511455843	3.3171511455843\\
72.5	0.17352	3.53705738823225	3.53705738823225\\
72.5	0.17718	3.76128369324856	3.76128369324856\\
72.5	0.18084	3.98983006063325	3.98983006063325\\
72.5	0.1845	4.22269649038631	4.22269649038631\\
72.5	0.18816	4.45988298250774	4.45988298250774\\
72.5	0.19182	4.70138953699755	4.70138953699755\\
72.5	0.19548	4.94721615385572	4.94721615385572\\
72.5	0.19914	5.19736283308227	5.19736283308227\\
72.5	0.2028	5.45182957467719	5.45182957467719\\
72.5	0.20646	5.71061637864049	5.71061637864049\\
72.5	0.21012	5.97372324497214	5.97372324497214\\
72.5	0.21378	6.24115017367217	6.24115017367217\\
72.5	0.21744	6.51289716474058	6.51289716474058\\
72.5	0.2211	6.78896421817736	6.78896421817736\\
72.5	0.22476	7.06935133398251	7.06935133398251\\
72.5	0.22842	7.35405851215603	7.35405851215603\\
72.5	0.23208	7.64308575269791	7.64308575269791\\
72.5	0.23574	7.93643305560818	7.93643305560818\\
72.5	0.2394	8.23410042088681	8.23410042088681\\
72.5	0.24306	8.53608784853382	8.53608784853382\\
72.5	0.24672	8.8423953385492	8.8423953385492\\
72.5	0.25038	9.15302289093296	9.15302289093296\\
72.5	0.25404	9.46797050568508	9.46797050568508\\
72.5	0.2577	9.78723818280557	9.78723818280557\\
72.5	0.26136	10.1108259222944	10.1108259222944\\
72.5	0.26502	10.4387337241517	10.4387337241517\\
72.5	0.26868	10.7709615883773	10.7709615883773\\
72.5	0.27234	11.1075095149713	11.1075095149713\\
72.5	0.276	11.4483775039336	11.4483775039336\\
72.875	0.093	-0.342772341310558	-0.342772341310558\\
72.875	0.09666	-0.209639226845982	-0.209639226845982\\
72.875	0.10032	-0.0721860500130305	-0.0721860500130305\\
72.875	0.10398	0.0695871891882922	0.0695871891882922\\
72.875	0.10764	0.215680490757983	0.215680490757983\\
72.875	0.1113	0.366093854696046	0.366093854696046\\
72.875	0.11496	0.520827281002484	0.520827281002484\\
72.875	0.11862	0.67988076967729	0.67988076967729\\
72.875	0.12228	0.843254320720471	0.843254320720471\\
72.875	0.12594	1.01094793413202	1.01094793413202\\
72.875	0.1296	1.18296160991194	1.18296160991194\\
72.875	0.13326	1.35929534806024	1.35929534806024\\
72.875	0.13692	1.5399491485769	1.5399491485769\\
72.875	0.14058	1.72492301146194	1.72492301146194\\
72.875	0.14424	1.91421693671535	1.91421693671535\\
72.875	0.1479	2.10783092433713	2.10783092433713\\
72.875	0.15156	2.30576497432728	2.30576497432728\\
72.875	0.15522	2.50801908668581	2.50801908668581\\
72.875	0.15888	2.7145932614127	2.7145932614127\\
72.875	0.16254	2.92548749850797	2.92548749850797\\
72.875	0.1662	3.14070179797161	3.14070179797161\\
72.875	0.16986	3.36023615980362	3.36023615980362\\
72.875	0.17352	3.584090584004	3.584090584004\\
72.875	0.17718	3.81226507057276	3.81226507057276\\
72.875	0.18084	4.04475961950988	4.04475961950988\\
72.875	0.1845	4.28157423081538	4.28157423081538\\
72.875	0.18816	4.52270890448925	4.52270890448925\\
72.875	0.19182	4.76816364053149	4.76816364053149\\
72.875	0.19548	5.0179384389421	5.0179384389421\\
72.875	0.19914	5.27203329972109	5.27203329972109\\
72.875	0.2028	5.53044822286844	5.53044822286844\\
72.875	0.20646	5.79318320838417	5.79318320838417\\
72.875	0.21012	6.06023825626826	6.06023825626826\\
72.875	0.21378	6.33161336652073	6.33161336652073\\
72.875	0.21744	6.60730853914157	6.60730853914157\\
72.875	0.2211	6.88732377413079	6.88732377413079\\
72.875	0.22476	7.17165907148837	7.17165907148837\\
72.875	0.22842	7.46031443121433	7.46031443121433\\
72.875	0.23208	7.75328985330865	7.75328985330865\\
72.875	0.23574	8.05058533777135	8.05058533777135\\
72.875	0.2394	8.35220088460242	8.35220088460242\\
72.875	0.24306	8.65813649380186	8.65813649380186\\
72.875	0.24672	8.96839216536969	8.96839216536969\\
72.875	0.25038	9.28296789930588	9.28296789930588\\
72.875	0.25404	9.60186369561043	9.60186369561043\\
72.875	0.2577	9.92507955428337	9.92507955428337\\
72.875	0.26136	10.2526154753247	10.2526154753247\\
72.875	0.26502	10.5844714587343	10.5844714587343\\
72.875	0.26868	10.9206475045124	10.9206475045124\\
72.875	0.27234	11.2611436126588	11.2611436126588\\
72.875	0.276	11.6059597831736	11.6059597831736\\
73.25	0.093	-0.383644361539415	-0.383644361539415\\
73.25	0.09666	-0.246563065522404	-0.246563065522404\\
73.25	0.10032	-0.105161707137018	-0.105161707137018\\
73.25	0.10398	0.0405597136167393	0.0405597136167393\\
73.25	0.10764	0.190601196738869	0.190601196738869\\
73.25	0.1113	0.34496274222937	0.34496274222937\\
73.25	0.11496	0.503644350088242	0.503644350088242\\
73.25	0.11862	0.666646020315486	0.666646020315486\\
73.25	0.12228	0.833967752911102	0.833967752911102\\
73.25	0.12594	1.00560954787509	1.00560954787509\\
73.25	0.1296	1.18157140520744	1.18157140520744\\
73.25	0.13326	1.36185332490818	1.36185332490818\\
73.25	0.13692	1.54645530697728	1.54645530697728\\
73.25	0.14058	1.73537735141476	1.73537735141476\\
73.25	0.14424	1.9286194582206	1.9286194582206\\
73.25	0.1479	2.12618162739482	2.12618162739482\\
73.25	0.15156	2.32806385893741	2.32806385893741\\
73.25	0.15522	2.53426615284836	2.53426615284836\\
73.25	0.15888	2.7447885091277	2.7447885091277\\
73.25	0.16254	2.9596309277754	2.9596309277754\\
73.25	0.1662	3.17879340879148	3.17879340879148\\
73.25	0.16986	3.40227595217593	3.40227595217593\\
73.25	0.17352	3.63007855792874	3.63007855792874\\
73.25	0.17718	3.86220122604993	3.86220122604993\\
73.25	0.18084	4.0986439565395	4.0986439565395\\
73.25	0.1845	4.33940674939742	4.33940674939742\\
73.25	0.18816	4.58448960462373	4.58448960462373\\
73.25	0.19182	4.8338925222184	4.8338925222184\\
73.25	0.19548	5.08761550218146	5.08761550218146\\
73.25	0.19914	5.34565854451288	5.34565854451288\\
73.25	0.2028	5.60802164921267	5.60802164921267\\
73.25	0.20646	5.87470481628084	5.87470481628084\\
73.25	0.21012	6.14570804571737	6.14570804571737\\
73.25	0.21378	6.42103133752227	6.42103133752227\\
73.25	0.21744	6.70067469169555	6.70067469169555\\
73.25	0.2211	6.9846381082372	6.9846381082372\\
73.25	0.22476	7.27292158714722	7.27292158714722\\
73.25	0.22842	7.56552512842562	7.56552512842562\\
73.25	0.23208	7.86244873207237	7.86244873207237\\
73.25	0.23574	8.16369239808751	8.16369239808751\\
73.25	0.2394	8.46925612647102	8.46925612647102\\
73.25	0.24306	8.7791399172229	8.7791399172229\\
73.25	0.24672	9.09334377034315	9.09334377034315\\
73.25	0.25038	9.41186768583177	9.41186768583177\\
73.25	0.25404	9.73471166368877	9.73471166368877\\
73.25	0.2577	10.0618757039141	10.0618757039141\\
73.25	0.26136	10.3933598065079	10.3933598065079\\
73.25	0.26502	10.72916397147	10.72916397147\\
73.25	0.26868	11.0692881988005	11.0692881988005\\
73.25	0.27234	11.4137324884993	11.4137324884993\\
73.25	0.276	11.7624968405665	11.7624968405665\\
73.625	0.093	-0.425561603615298	-0.425561603615298\\
73.625	0.09666	-0.284532126045849	-0.284532126045849\\
73.625	0.10032	-0.139182586108028	-0.139182586108028\\
73.625	0.10398	0.0104870161981678	0.0104870161981678\\
73.625	0.10764	0.164476680872732	0.164476680872732\\
73.625	0.1113	0.322786407915668	0.322786407915668\\
73.625	0.11496	0.485416197326979	0.485416197326979\\
73.625	0.11862	0.652366049106657	0.652366049106657\\
73.625	0.12228	0.823635963254711	0.823635963254711\\
73.625	0.12594	0.999225939771137	0.999225939771137\\
73.625	0.1296	1.17913597865593	1.17913597865593\\
73.625	0.13326	1.3633660799091	1.3633660799091\\
73.625	0.13692	1.55191624353064	1.55191624353064\\
73.625	0.14058	1.74478646952055	1.74478646952055\\
73.625	0.14424	1.94197675787883	1.94197675787883\\
73.625	0.1479	2.14348710860548	2.14348710860548\\
73.625	0.15156	2.3493175217005	2.3493175217005\\
73.625	0.15522	2.5594679971639	2.5594679971639\\
73.625	0.15888	2.77393853499567	2.77393853499567\\
73.625	0.16254	2.99272913519581	2.99272913519581\\
73.625	0.1662	3.21583979776432	3.21583979776432\\
73.625	0.16986	3.4432705227012	3.4432705227012\\
73.625	0.17352	3.67502131000646	3.67502131000646\\
73.625	0.17718	3.91109215968009	3.91109215968009\\
73.625	0.18084	4.15148307172208	4.15148307172208\\
73.625	0.1845	4.39619404613245	4.39619404613245\\
73.625	0.18816	4.64522508291119	4.64522508291119\\
73.625	0.19182	4.8985761820583	4.8985761820583\\
73.625	0.19548	5.15624734357379	5.15624734357379\\
73.625	0.19914	5.41823856745765	5.41823856745765\\
73.625	0.2028	5.68454985370988	5.68454985370988\\
73.625	0.20646	5.95518120233048	5.95518120233048\\
73.625	0.21012	6.23013261331944	6.23013261331944\\
73.625	0.21378	6.50940408667678	6.50940408667678\\
73.625	0.21744	6.7929956224025	6.7929956224025\\
73.625	0.2211	7.08090722049659	7.08090722049659\\
73.625	0.22476	7.37313888095904	7.37313888095904\\
73.625	0.22842	7.66969060378988	7.66969060378988\\
73.625	0.23208	7.97056238898907	7.97056238898907\\
73.625	0.23574	8.27575423655664	8.27575423655664\\
73.625	0.2394	8.58526614649259	8.58526614649259\\
73.625	0.24306	8.89909811879691	8.89909811879691\\
73.625	0.24672	9.2172501534696	9.2172501534696\\
73.625	0.25038	9.53972225051066	9.53972225051066\\
73.625	0.25404	9.86651440992009	9.86651440992009\\
73.625	0.2577	10.1976266316979	10.1976266316979\\
73.625	0.26136	10.5330589158441	10.5330589158441\\
73.625	0.26502	10.8728112623586	10.8728112623586\\
73.625	0.26868	11.2168836712415	11.2168836712415\\
73.625	0.27234	11.5652761424928	11.5652761424928\\
73.625	0.276	11.9179886761125	11.9179886761125\\
74	0.093	-0.468524067538196	-0.468524067538196\\
74	0.09666	-0.323546408416312	-0.323546408416312\\
74	0.10032	-0.174248686926052	-0.174248686926052\\
74	0.10398	-0.0206309030674185	-0.0206309030674185\\
74	0.10764	0.13730694315958	0.13730694315958\\
74	0.1113	0.299564851754955	0.299564851754955\\
74	0.11496	0.4661428227187	0.4661428227187\\
74	0.11862	0.637040856050813	0.637040856050813\\
74	0.12228	0.812258951751306	0.812258951751306\\
74	0.12594	0.991797109820166	0.991797109820166\\
74	0.1296	1.17565533025739	1.17565533025739\\
74	0.13326	1.363833613063	1.363833613063\\
74	0.13692	1.55633195823698	1.55633195823698\\
74	0.14058	1.75315036577932	1.75315036577932\\
74	0.14424	1.95428883569004	1.95428883569004\\
74	0.1479	2.15974736796913	2.15974736796913\\
74	0.15156	2.36952596261659	2.36952596261659\\
74	0.15522	2.58362461963242	2.58362461963242\\
74	0.15888	2.80204333901663	2.80204333901663\\
74	0.16254	3.0247821207692	3.0247821207692\\
74	0.1662	3.25184096489015	3.25184096489015\\
74	0.16986	3.48321987137947	3.48321987137947\\
74	0.17352	3.71891884023716	3.71891884023716\\
74	0.17718	3.95893787146322	3.95893787146322\\
74	0.18084	4.20327696505766	4.20327696505766\\
74	0.1845	4.45193612102047	4.45193612102047\\
74	0.18816	4.70491533935165	4.70491533935165\\
74	0.19182	4.96221462005119	4.96221462005119\\
74	0.19548	5.22383396311912	5.22383396311912\\
74	0.19914	5.48977336855541	5.48977336855541\\
74	0.2028	5.76003283636007	5.76003283636007\\
74	0.20646	6.03461236653311	6.03461236653311\\
74	0.21012	6.31351195907451	6.31351195907451\\
74	0.21378	6.59673161398429	6.59673161398429\\
74	0.21744	6.88427133126244	6.88427133126244\\
74	0.2211	7.17613111090897	7.17613111090897\\
74	0.22476	7.47231095292386	7.47231095292386\\
74	0.22842	7.77281085730713	7.77281085730713\\
74	0.23208	8.07763082405876	8.07763082405876\\
74	0.23574	8.38677085317877	8.38677085317877\\
74	0.2394	8.70023094466715	8.70023094466715\\
74	0.24306	9.0180110985239	9.0180110985239\\
74	0.24672	9.34011131474903	9.34011131474903\\
74	0.25038	9.66653159334252	9.66653159334252\\
74	0.25404	9.99727193430439	9.99727193430439\\
74	0.2577	10.3323323376346	10.3323323376346\\
74	0.26136	10.6717128033332	10.6717128033332\\
74	0.26502	11.0154133314002	11.0154133314002\\
74	0.26868	11.3634339218356	11.3634339218356\\
74	0.27234	11.7157745746393	11.7157745746393\\
74	0.276	12.0724352898114	12.0724352898114\\
};
\end{axis}

\begin{axis}[%
width=6.159cm,
height=3.097cm,
at={(8.104cm,8.602cm)},
scale only axis,
xmin=56,
xmax=74,
tick align=outside,
xlabel style={font=\color{white!15!black}},
xlabel={$L_{cut}$},
ymin=0.093,
ymax=0.276,
ylabel style={font=\color{white!15!black}},
ylabel={$D_{rlx}$},
zmin=-1.52969257267885,
zmax=2.51190702643201,
zlabel style={font=\color{white!15!black}},
zlabel={$u(t-3)u(t-3)$},
view={-140}{50},
axis background/.style={fill=white},
xmajorgrids,
ymajorgrids,
zmajorgrids
]
\addplot3[only marks, mark=*, mark options={}, mark size=1.5000pt, color=mycolor1, fill=mycolor1] table[row sep=crcr]{%
x	y	z\\
74	0.123	0.00740370360905991\\
72	0.113	-0.186626587578763\\
61	0.095	-0.150607769785267\\
56	0.093	-0.216975707230133\\
};
\addplot3[only marks, mark=*, mark options={}, mark size=1.5000pt, color=mycolor2, fill=mycolor2] table[row sep=crcr]{%
x	y	z\\
67	0.276	0.0527524251509314\\
66	0.255	-0.475927501616939\\
62	0.209	0.435130634728223\\
57	0.193	0.76117565090573\\
};
\addplot3[only marks, mark=*, mark options={}, mark size=1.5000pt, color=black, fill=black] table[row sep=crcr]{%
x	y	z\\
69	0.104	-0.0638886768595199\\
};
\addplot3[only marks, mark=*, mark options={}, mark size=1.5000pt, color=black, fill=black] table[row sep=crcr]{%
x	y	z\\
64	0.23	0.0838527726713747\\
};

\addplot3[%
surf,
fill opacity=0.7, shader=interp, colormap={mymap}{[1pt] rgb(0pt)=(1,0.905882,0); rgb(1pt)=(1,0.901964,0); rgb(2pt)=(1,0.898051,0); rgb(3pt)=(1,0.894144,0); rgb(4pt)=(1,0.890243,0); rgb(5pt)=(1,0.886349,0); rgb(6pt)=(1,0.88246,0); rgb(7pt)=(1,0.878577,0); rgb(8pt)=(1,0.8747,0); rgb(9pt)=(1,0.870829,0); rgb(10pt)=(1,0.866964,0); rgb(11pt)=(1,0.863106,0); rgb(12pt)=(1,0.859253,0); rgb(13pt)=(1,0.855406,0); rgb(14pt)=(1,0.851566,0); rgb(15pt)=(1,0.847732,0); rgb(16pt)=(1,0.843903,0); rgb(17pt)=(1,0.840081,0); rgb(18pt)=(1,0.836265,0); rgb(19pt)=(1,0.832455,0); rgb(20pt)=(1,0.828652,0); rgb(21pt)=(1,0.824854,0); rgb(22pt)=(1,0.821063,0); rgb(23pt)=(1,0.817278,0); rgb(24pt)=(1,0.8135,0); rgb(25pt)=(1,0.809727,0); rgb(26pt)=(1,0.805961,0); rgb(27pt)=(1,0.8022,0); rgb(28pt)=(1,0.798445,0); rgb(29pt)=(1,0.794696,0); rgb(30pt)=(1,0.790953,0); rgb(31pt)=(1,0.787215,0); rgb(32pt)=(1,0.783484,0); rgb(33pt)=(1,0.779758,0); rgb(34pt)=(1,0.776038,0); rgb(35pt)=(1,0.772324,0); rgb(36pt)=(1,0.768615,0); rgb(37pt)=(1,0.764913,0); rgb(38pt)=(1,0.761217,0); rgb(39pt)=(1,0.757527,0); rgb(40pt)=(1,0.753843,0); rgb(41pt)=(1,0.750165,0); rgb(42pt)=(1,0.746493,0); rgb(43pt)=(1,0.742827,0); rgb(44pt)=(1,0.739167,0); rgb(45pt)=(1,0.735514,0); rgb(46pt)=(1,0.731867,0); rgb(47pt)=(1,0.728226,0); rgb(48pt)=(1,0.724591,0); rgb(49pt)=(1,0.720963,0); rgb(50pt)=(1,0.717341,0); rgb(51pt)=(1,0.713725,0); rgb(52pt)=(0.999994,0.710077,0); rgb(53pt)=(0.999974,0.706363,0); rgb(54pt)=(0.999942,0.702592,0); rgb(55pt)=(0.999898,0.698775,0); rgb(56pt)=(0.999841,0.694921,0); rgb(57pt)=(0.999771,0.691039,0); rgb(58pt)=(0.99969,0.687139,0); rgb(59pt)=(0.999596,0.68323,0); rgb(60pt)=(0.99949,0.679323,0); rgb(61pt)=(0.999372,0.675427,0); rgb(62pt)=(0.999242,0.67155,0); rgb(63pt)=(0.9991,0.667704,0); rgb(64pt)=(0.998946,0.663897,0); rgb(65pt)=(0.998781,0.660138,0); rgb(66pt)=(0.998605,0.656439,0); rgb(67pt)=(0.998416,0.652807,0); rgb(68pt)=(0.998217,0.649253,0); rgb(69pt)=(0.998006,0.645786,0); rgb(70pt)=(0.997785,0.642416,0); rgb(71pt)=(0.997552,0.639152,0); rgb(72pt)=(0.997308,0.636004,0); rgb(73pt)=(0.997053,0.632982,0); rgb(74pt)=(0.996788,0.630095,0); rgb(75pt)=(0.996512,0.627352,0); rgb(76pt)=(0.996226,0.624763,0); rgb(77pt)=(0.995851,0.622329,0); rgb(78pt)=(0.99494,0.619997,0); rgb(79pt)=(0.99345,0.617753,0); rgb(80pt)=(0.991419,0.61559,0); rgb(81pt)=(0.988885,0.613503,0); rgb(82pt)=(0.985886,0.611486,0); rgb(83pt)=(0.98246,0.609532,0); rgb(84pt)=(0.978643,0.607636,0); rgb(85pt)=(0.974475,0.605791,0); rgb(86pt)=(0.969992,0.603992,0); rgb(87pt)=(0.965232,0.602233,0); rgb(88pt)=(0.960233,0.600507,0); rgb(89pt)=(0.955033,0.598808,0); rgb(90pt)=(0.949669,0.59713,0); rgb(91pt)=(0.94418,0.595468,0); rgb(92pt)=(0.938602,0.593815,0); rgb(93pt)=(0.932974,0.592166,0); rgb(94pt)=(0.927333,0.590513,0); rgb(95pt)=(0.921717,0.588852,0); rgb(96pt)=(0.916164,0.587176,0); rgb(97pt)=(0.910711,0.585479,0); rgb(98pt)=(0.905397,0.583755,0); rgb(99pt)=(0.900258,0.581999,0); rgb(100pt)=(0.895333,0.580203,0); rgb(101pt)=(0.890659,0.578362,0); rgb(102pt)=(0.886275,0.576471,0); rgb(103pt)=(0.882047,0.574545,0); rgb(104pt)=(0.877819,0.572608,0); rgb(105pt)=(0.873592,0.57066,0); rgb(106pt)=(0.869366,0.568701,0); rgb(107pt)=(0.865143,0.566733,0); rgb(108pt)=(0.860924,0.564756,0); rgb(109pt)=(0.856708,0.562771,0); rgb(110pt)=(0.852497,0.560778,0); rgb(111pt)=(0.848292,0.558779,0); rgb(112pt)=(0.844092,0.556774,0); rgb(113pt)=(0.8399,0.554763,0); rgb(114pt)=(0.835716,0.552749,0); rgb(115pt)=(0.831541,0.55073,0); rgb(116pt)=(0.827374,0.548709,0); rgb(117pt)=(0.823219,0.546686,0); rgb(118pt)=(0.819074,0.54466,0); rgb(119pt)=(0.81494,0.542635,0); rgb(120pt)=(0.81082,0.540609,0); rgb(121pt)=(0.806712,0.538584,0); rgb(122pt)=(0.802619,0.53656,0); rgb(123pt)=(0.798541,0.534539,0); rgb(124pt)=(0.794478,0.532521,0); rgb(125pt)=(0.790431,0.530506,0); rgb(126pt)=(0.786402,0.528496,0); rgb(127pt)=(0.782391,0.526491,0); rgb(128pt)=(0.77841,0.524489,0); rgb(129pt)=(0.774523,0.522478,0); rgb(130pt)=(0.770731,0.520455,0); rgb(131pt)=(0.767022,0.518424,0); rgb(132pt)=(0.763384,0.516385,0); rgb(133pt)=(0.759804,0.514339,0); rgb(134pt)=(0.756272,0.51229,0); rgb(135pt)=(0.752775,0.510237,0); rgb(136pt)=(0.749302,0.508182,0); rgb(137pt)=(0.74584,0.506128,0); rgb(138pt)=(0.742378,0.504075,0); rgb(139pt)=(0.738904,0.502025,0); rgb(140pt)=(0.735406,0.499979,0); rgb(141pt)=(0.731872,0.49794,0); rgb(142pt)=(0.72829,0.495909,0); rgb(143pt)=(0.724649,0.493887,0); rgb(144pt)=(0.720936,0.491875,0); rgb(145pt)=(0.71714,0.489876,0); rgb(146pt)=(0.713249,0.487891,0); rgb(147pt)=(0.709251,0.485921,0); rgb(148pt)=(0.705134,0.483968,0); rgb(149pt)=(0.700887,0.482033,0); rgb(150pt)=(0.696497,0.480118,0); rgb(151pt)=(0.691952,0.478225,0); rgb(152pt)=(0.687242,0.476355,0); rgb(153pt)=(0.682353,0.47451,0); rgb(154pt)=(0.677195,0.472696,0); rgb(155pt)=(0.6717,0.470916,0); rgb(156pt)=(0.665891,0.469169,0); rgb(157pt)=(0.659791,0.46745,0); rgb(158pt)=(0.653423,0.465756,0); rgb(159pt)=(0.64681,0.464084,0); rgb(160pt)=(0.639976,0.462432,0); rgb(161pt)=(0.632943,0.460795,0); rgb(162pt)=(0.625734,0.459171,0); rgb(163pt)=(0.618373,0.457556,0); rgb(164pt)=(0.610882,0.455948,0); rgb(165pt)=(0.603284,0.454343,0); rgb(166pt)=(0.595604,0.452737,0); rgb(167pt)=(0.587863,0.451129,0); rgb(168pt)=(0.580084,0.449514,0); rgb(169pt)=(0.572292,0.447889,0); rgb(170pt)=(0.564508,0.446252,0); rgb(171pt)=(0.556756,0.444599,0); rgb(172pt)=(0.549059,0.442927,0); rgb(173pt)=(0.54144,0.441232,0); rgb(174pt)=(0.533922,0.439512,0); rgb(175pt)=(0.526529,0.437764,0); rgb(176pt)=(0.519282,0.435983,0); rgb(177pt)=(0.512206,0.434168,0); rgb(178pt)=(0.505323,0.432315,0); rgb(179pt)=(0.498628,0.430422,3.92506e-06); rgb(180pt)=(0.491973,0.428504,3.49981e-05); rgb(181pt)=(0.485331,0.426562,9.63073e-05); rgb(182pt)=(0.478704,0.424596,0.000186979); rgb(183pt)=(0.472096,0.422609,0.000306141); rgb(184pt)=(0.465508,0.420599,0.00045292); rgb(185pt)=(0.458942,0.418567,0.000626441); rgb(186pt)=(0.452401,0.416515,0.000825833); rgb(187pt)=(0.445885,0.414441,0.00105022); rgb(188pt)=(0.439399,0.412348,0.00129873); rgb(189pt)=(0.432942,0.410234,0.00157049); rgb(190pt)=(0.426518,0.408102,0.00186463); rgb(191pt)=(0.420129,0.40595,0.00218028); rgb(192pt)=(0.413777,0.40378,0.00251655); rgb(193pt)=(0.407464,0.401592,0.00287258); rgb(194pt)=(0.401191,0.399386,0.00324749); rgb(195pt)=(0.394962,0.397164,0.00364042); rgb(196pt)=(0.388777,0.394925,0.00405048); rgb(197pt)=(0.38264,0.39267,0.00447681); rgb(198pt)=(0.376552,0.390399,0.00491852); rgb(199pt)=(0.370516,0.388113,0.00537476); rgb(200pt)=(0.364532,0.385812,0.00584464); rgb(201pt)=(0.358605,0.383497,0.00632729); rgb(202pt)=(0.352735,0.381168,0.00682184); rgb(203pt)=(0.346925,0.378826,0.00732741); rgb(204pt)=(0.341176,0.376471,0.00784314); rgb(205pt)=(0.335485,0.374093,0.00847245); rgb(206pt)=(0.329843,0.371682,0.00930909); rgb(207pt)=(0.324249,0.369242,0.0103377); rgb(208pt)=(0.318701,0.366772,0.0115428); rgb(209pt)=(0.313198,0.364275,0.0129091); rgb(210pt)=(0.307739,0.361753,0.0144211); rgb(211pt)=(0.302322,0.359206,0.0160634); rgb(212pt)=(0.296945,0.356637,0.0178207); rgb(213pt)=(0.291607,0.354048,0.0196776); rgb(214pt)=(0.286307,0.35144,0.0216186); rgb(215pt)=(0.281043,0.348814,0.0236284); rgb(216pt)=(0.275813,0.346172,0.0256916); rgb(217pt)=(0.270616,0.343517,0.0277927); rgb(218pt)=(0.265451,0.340849,0.0299163); rgb(219pt)=(0.260317,0.33817,0.0320472); rgb(220pt)=(0.25521,0.335482,0.0341698); rgb(221pt)=(0.250131,0.332786,0.0362688); rgb(222pt)=(0.245078,0.330085,0.0383287); rgb(223pt)=(0.240048,0.327379,0.0403343); rgb(224pt)=(0.235042,0.324671,0.04227); rgb(225pt)=(0.230056,0.321962,0.0441205); rgb(226pt)=(0.22509,0.319254,0.0458704); rgb(227pt)=(0.220142,0.316548,0.0475043); rgb(228pt)=(0.215212,0.313846,0.0490067); rgb(229pt)=(0.210296,0.311149,0.0503624); rgb(230pt)=(0.205395,0.308459,0.0515759); rgb(231pt)=(0.200514,0.305763,0.052757); rgb(232pt)=(0.195655,0.303061,0.0539242); rgb(233pt)=(0.190817,0.300353,0.0550763); rgb(234pt)=(0.186001,0.297639,0.0562123); rgb(235pt)=(0.181207,0.294918,0.0573313); rgb(236pt)=(0.176434,0.292191,0.0584321); rgb(237pt)=(0.171685,0.289458,0.0595136); rgb(238pt)=(0.166957,0.286719,0.060575); rgb(239pt)=(0.162252,0.283973,0.0616151); rgb(240pt)=(0.15757,0.281221,0.0626328); rgb(241pt)=(0.152911,0.278463,0.0636271); rgb(242pt)=(0.148275,0.275699,0.0645971); rgb(243pt)=(0.143663,0.272929,0.0655416); rgb(244pt)=(0.139074,0.270152,0.0664596); rgb(245pt)=(0.134508,0.26737,0.06735); rgb(246pt)=(0.129967,0.264581,0.0682118); rgb(247pt)=(0.125449,0.261787,0.0690441); rgb(248pt)=(0.120956,0.258986,0.0698456); rgb(249pt)=(0.116487,0.25618,0.0706154); rgb(250pt)=(0.112043,0.253367,0.0713525); rgb(251pt)=(0.107623,0.250549,0.0720557); rgb(252pt)=(0.103229,0.247724,0.0727241); rgb(253pt)=(0.0988592,0.244894,0.0733566); rgb(254pt)=(0.0945149,0.242058,0.0739522); rgb(255pt)=(0.0901961,0.239216,0.0745098)}, mesh/rows=49]
table[row sep=crcr, point meta=\thisrow{c}] {%
%
x	y	z	c\\
56	0.093	-0.183093589236051	-0.183093589236051\\
56	0.09666	-0.155069262895937	-0.155069262895937\\
56	0.10032	-0.125988786107933	-0.125988786107933\\
56	0.10398	-0.0958521588720439	-0.0958521588720439\\
56	0.10764	-0.064659381188267	-0.064659381188267\\
56	0.1113	-0.0324104530566007	-0.0324104530566007\\
56	0.11496	0.000894625522951387	0.000894625522951387\\
56	0.11862	0.0352558545503928	0.0352558545503928\\
56	0.12228	0.0706732340257201	0.0706732340257201\\
56	0.12594	0.107146763948935	0.107146763948935\\
56	0.1296	0.144676444320039	0.144676444320039\\
56	0.13326	0.183262275139029	0.183262275139029\\
56	0.13692	0.222904256405907	0.222904256405907\\
56	0.14058	0.263602388120673	0.263602388120673\\
56	0.14424	0.305356670283327	0.305356670283327\\
56	0.1479	0.348167102893868	0.348167102893868\\
56	0.15156	0.392033685952298	0.392033685952298\\
56	0.15522	0.436956419458614	0.436956419458614\\
56	0.15888	0.482935303412818	0.482935303412818\\
56	0.16254	0.529970337814909	0.529970337814909\\
56	0.1662	0.578061522664889	0.578061522664889\\
56	0.16986	0.627208857962755	0.627208857962755\\
56	0.17352	0.677412343708509	0.677412343708509\\
56	0.17718	0.728671979902152	0.728671979902152\\
56	0.18084	0.780987766543681	0.780987766543681\\
56	0.1845	0.834359703633098	0.834359703633098\\
56	0.18816	0.888787791170404	0.888787791170404\\
56	0.19182	0.944272029155596	0.944272029155596\\
56	0.19548	1.00081241758868	1.00081241758868\\
56	0.19914	1.05840895646964	1.05840895646964\\
56	0.2028	1.1170616457985	1.1170616457985\\
56	0.20646	1.17677048557524	1.17677048557524\\
56	0.21012	1.23753547579987	1.23753547579987\\
56	0.21378	1.29935661647239	1.29935661647239\\
56	0.21744	1.3622339075928	1.3622339075928\\
56	0.2211	1.42616734916109	1.42616734916109\\
56	0.22476	1.49115694117727	1.49115694117727\\
56	0.22842	1.55720268364134	1.55720268364134\\
56	0.23208	1.6243045765533	1.6243045765533\\
56	0.23574	1.69246261991314	1.69246261991314\\
56	0.2394	1.76167681372087	1.76167681372087\\
56	0.24306	1.83194715797649	1.83194715797649\\
56	0.24672	1.90327365268	1.90327365268\\
56	0.25038	1.97565629783139	1.97565629783139\\
56	0.25404	2.04909509343067	2.04909509343067\\
56	0.2577	2.12359003947784	2.12359003947784\\
56	0.26136	2.1991411359729	2.1991411359729\\
56	0.26502	2.27574838291585	2.27574838291585\\
56	0.26868	2.35341178030668	2.35341178030668\\
56	0.27234	2.4321313281454	2.4321313281454\\
56	0.276	2.51190702643201	2.51190702643201\\
56.375	0.093	-0.189487201537818	-0.189487201537818\\
56.375	0.09666	-0.16338436814129	-0.16338436814129\\
56.375	0.10032	-0.136225384296876	-0.136225384296876\\
56.375	0.10398	-0.108010250004572	-0.108010250004572\\
56.375	0.10764	-0.0787389652643813	-0.0787389652643813\\
56.375	0.1113	-0.0484115300763026	-0.0484115300763026\\
56.375	0.11496	-0.0170279444403363	-0.0170279444403363\\
56.375	0.11862	0.0154117916435176	0.0154117916435176\\
56.375	0.12228	0.0489076781752591	0.0489076781752591\\
56.375	0.12594	0.0834597151548883	0.0834597151548883\\
56.375	0.1296	0.119067902582405	0.119067902582405\\
56.375	0.13326	0.155732240457807	0.155732240457807\\
56.375	0.13692	0.193452728781099	0.193452728781099\\
56.375	0.14058	0.232229367552279	0.232229367552279\\
56.375	0.14424	0.272062156771346	0.272062156771346\\
56.375	0.1479	0.312951096438299	0.312951096438299\\
56.375	0.15156	0.354896186553142	0.354896186553142\\
56.375	0.15522	0.397897427115872	0.397897427115872\\
56.375	0.15888	0.44195481812649	0.44195481812649\\
56.375	0.16254	0.487068359584993	0.487068359584993\\
56.375	0.1662	0.533238051491386	0.533238051491386\\
56.375	0.16986	0.580463893845666	0.580463893845666\\
56.375	0.17352	0.628745886647834	0.628745886647834\\
56.375	0.17718	0.67808402989789	0.67808402989789\\
56.375	0.18084	0.728478323595833	0.728478323595833\\
56.375	0.1845	0.779928767741662	0.779928767741662\\
56.375	0.18816	0.832435362335381	0.832435362335381\\
56.375	0.19182	0.885998107376987	0.885998107376987\\
56.375	0.19548	0.940617002866481	0.940617002866481\\
56.375	0.19914	0.996292048803861	0.996292048803861\\
56.375	0.2028	1.05302324518913	1.05302324518913\\
56.375	0.20646	1.11081059202229	1.11081059202229\\
56.375	0.21012	1.16965408930333	1.16965408930333\\
56.375	0.21378	1.22955373703226	1.22955373703226\\
56.375	0.21744	1.29050953520908	1.29050953520908\\
56.375	0.2211	1.35252148383379	1.35252148383379\\
56.375	0.22476	1.41558958290638	1.41558958290638\\
56.375	0.22842	1.47971383242687	1.47971383242687\\
56.375	0.23208	1.54489423239523	1.54489423239523\\
56.375	0.23574	1.61113078281149	1.61113078281149\\
56.375	0.2394	1.67842348367563	1.67842348367563\\
56.375	0.24306	1.74677233498767	1.74677233498767\\
56.375	0.24672	1.81617733674759	1.81617733674759\\
56.375	0.25038	1.8866384889554	1.8866384889554\\
56.375	0.25404	1.95815579161109	1.95815579161109\\
56.375	0.2577	2.03072924471467	2.03072924471467\\
56.375	0.26136	2.10435884826614	2.10435884826614\\
56.375	0.26502	2.1790446022655	2.1790446022655\\
56.375	0.26868	2.25478650671275	2.25478650671275\\
56.375	0.27234	2.33158456160788	2.33158456160788\\
56.375	0.276	2.4094387669509	2.4094387669509\\
56.75	0.093	-0.195103440740309	-0.195103440740309\\
56.75	0.09666	-0.17092210028737	-0.17092210028737\\
56.75	0.10032	-0.14568460938654	-0.14568460938654\\
56.75	0.10398	-0.119390968037824	-0.119390968037824\\
56.75	0.10764	-0.0920411762412204	-0.0920411762412204\\
56.75	0.1113	-0.0636352339967274	-0.0636352339967274\\
56.75	0.11496	-0.0341731413043487	-0.0341731413043487\\
56.75	0.11862	-0.0036548981640806	-0.0036548981640806\\
56.75	0.12228	0.0279194954240733	0.0279194954240733\\
56.75	0.12594	0.0605500394601149	0.0605500394601149\\
56.75	0.1296	0.094236733944044	0.094236733944044\\
56.75	0.13326	0.128979578875863	0.128979578875863\\
56.75	0.13692	0.164778574255567	0.164778574255567\\
56.75	0.14058	0.201633720083159	0.201633720083159\\
56.75	0.14424	0.239545016358639	0.239545016358639\\
56.75	0.1479	0.278512463082008	0.278512463082008\\
56.75	0.15156	0.318536060253263	0.318536060253263\\
56.75	0.15522	0.359615807872405	0.359615807872405\\
56.75	0.15888	0.401751705939435	0.401751705939435\\
56.75	0.16254	0.444943754454353	0.444943754454353\\
56.75	0.1662	0.48919195341716	0.48919195341716\\
56.75	0.16986	0.534496302827853	0.534496302827853\\
56.75	0.17352	0.580856802686433	0.580856802686433\\
56.75	0.17718	0.628273452992903	0.628273452992903\\
56.75	0.18084	0.676746253747259	0.676746253747259\\
56.75	0.1845	0.726275204949502	0.726275204949502\\
56.75	0.18816	0.776860306599634	0.776860306599634\\
56.75	0.19182	0.828501558697652	0.828501558697652\\
56.75	0.19548	0.88119896124356	0.88119896124356\\
56.75	0.19914	0.934952514237354	0.934952514237354\\
56.75	0.2028	0.989762217679035	0.989762217679035\\
56.75	0.20646	1.04562807156861	1.04562807156861\\
56.75	0.21012	1.10255007590606	1.10255007590606\\
56.75	0.21378	1.16052823069141	1.16052823069141\\
56.75	0.21744	1.21956253592464	1.21956253592464\\
56.75	0.2211	1.27965299160576	1.27965299160576\\
56.75	0.22476	1.34079959773477	1.34079959773477\\
56.75	0.22842	1.40300235431167	1.40300235431167\\
56.75	0.23208	1.46626126133645	1.46626126133645\\
56.75	0.23574	1.53057631880911	1.53057631880911\\
56.75	0.2394	1.59594752672967	1.59594752672967\\
56.75	0.24306	1.66237488509812	1.66237488509812\\
56.75	0.24672	1.72985839391445	1.72985839391445\\
56.75	0.25038	1.79839805317867	1.79839805317867\\
56.75	0.25404	1.86799386289078	1.86799386289078\\
56.75	0.2577	1.93864582305078	1.93864582305078\\
56.75	0.26136	2.01035393365866	2.01035393365866\\
56.75	0.26502	2.08311819471443	2.08311819471443\\
56.75	0.26868	2.15693860621809	2.15693860621809\\
56.75	0.27234	2.23181516816964	2.23181516816964\\
56.75	0.276	2.30774788056907	2.30774788056907\\
57.125	0.093	-0.199942306843528	-0.199942306843528\\
57.125	0.09666	-0.177682459334174	-0.177682459334174\\
57.125	0.10032	-0.154366461376932	-0.154366461376932\\
57.125	0.10398	-0.129994312971802	-0.129994312971802\\
57.125	0.10764	-0.104566014118786	-0.104566014118786\\
57.125	0.1113	-0.0780815648178805	-0.0780815648178805\\
57.125	0.11496	-0.0505409650690876	-0.0505409650690876\\
57.125	0.11862	-0.0219442148724071	-0.0219442148724071\\
57.125	0.12228	0.00770868577216111	0.00770868577216111\\
57.125	0.12594	0.0384177368646151	0.0384177368646151\\
57.125	0.1296	0.0701829384049584	0.0701829384049584\\
57.125	0.13326	0.103004290393189	0.103004290393189\\
57.125	0.13692	0.136881792829308	0.136881792829308\\
57.125	0.14058	0.171815445713313	0.171815445713313\\
57.125	0.14424	0.207805249045206	0.207805249045206\\
57.125	0.1479	0.244851202824988	0.244851202824988\\
57.125	0.15156	0.282953307052655	0.282953307052655\\
57.125	0.15522	0.322111561728212	0.322111561728212\\
57.125	0.15888	0.362325966851656	0.362325966851656\\
57.125	0.16254	0.403596522422986	0.403596522422986\\
57.125	0.1662	0.445923228442206	0.445923228442206\\
57.125	0.16986	0.489306084909313	0.489306084909313\\
57.125	0.17352	0.533745091824308	0.533745091824308\\
57.125	0.17718	0.579240249187188	0.579240249187188\\
57.125	0.18084	0.625791556997958	0.625791556997958\\
57.125	0.1845	0.673399015256616	0.673399015256616\\
57.125	0.18816	0.722062623963159	0.722062623963159\\
57.125	0.19182	0.771782383117592	0.771782383117592\\
57.125	0.19548	0.822558292719912	0.822558292719912\\
57.125	0.19914	0.87439035277012	0.87439035277012\\
57.125	0.2028	0.927278563268215	0.927278563268215\\
57.125	0.20646	0.9812229242142	0.9812229242142\\
57.125	0.21012	1.03622343560807	1.03622343560807\\
57.125	0.21378	1.09228009744983	1.09228009744983\\
57.125	0.21744	1.14939290973947	1.14939290973947\\
57.125	0.2211	1.207561872477	1.207561872477\\
57.125	0.22476	1.26678698566243	1.26678698566243\\
57.125	0.22842	1.32706824929573	1.32706824929573\\
57.125	0.23208	1.38840566337693	1.38840566337693\\
57.125	0.23574	1.45079922790601	1.45079922790601\\
57.125	0.2394	1.51424894288298	1.51424894288298\\
57.125	0.24306	1.57875480830784	1.57875480830784\\
57.125	0.24672	1.64431682418059	1.64431682418059\\
57.125	0.25038	1.71093499050123	1.71093499050123\\
57.125	0.25404	1.77860930726975	1.77860930726975\\
57.125	0.2577	1.84733977448616	1.84733977448616\\
57.125	0.26136	1.91712639215045	1.91712639215045\\
57.125	0.26502	1.98796916026264	1.98796916026264\\
57.125	0.26868	2.05986807882271	2.05986807882271\\
57.125	0.27234	2.13282314783067	2.13282314783067\\
57.125	0.276	2.20683436728652	2.20683436728652\\
57.5	0.093	-0.204003799847469	-0.204003799847469\\
57.5	0.09666	-0.183665445281702	-0.183665445281702\\
57.5	0.10032	-0.162270940268047	-0.162270940268047\\
57.5	0.10398	-0.139820284806505	-0.139820284806505\\
57.5	0.10764	-0.116313478897074	-0.116313478897074\\
57.5	0.1113	-0.0917505225397548	-0.0917505225397548\\
57.5	0.11496	-0.0661314157345494	-0.0661314157345494\\
57.5	0.11862	-0.0394561584814564	-0.0394561584814564\\
57.5	0.12228	-0.0117247507804741	-0.0117247507804741\\
57.5	0.12594	0.0170628073683942	0.0170628073683942\\
57.5	0.1296	0.0469065159651498	0.0469065159651498\\
57.5	0.13326	0.0778063750097933	0.0778063750097933\\
57.5	0.13692	0.109762384502326	0.109762384502326\\
57.5	0.14058	0.142774544442743	0.142774544442743\\
57.5	0.14424	0.176842854831051	0.176842854831051\\
57.5	0.1479	0.211967315667245	0.211967315667245\\
57.5	0.15156	0.248147926951326	0.248147926951326\\
57.5	0.15522	0.285384688683296	0.285384688683296\\
57.5	0.15888	0.323677600863153	0.323677600863153\\
57.5	0.16254	0.363026663490899	0.363026663490899\\
57.5	0.1662	0.403431876566531	0.403431876566531\\
57.5	0.16986	0.44489324009005	0.44489324009005\\
57.5	0.17352	0.487410754061457	0.487410754061457\\
57.5	0.17718	0.530984418480752	0.530984418480752\\
57.5	0.18084	0.575614233347936	0.575614233347936\\
57.5	0.1845	0.621300198663006	0.621300198663006\\
57.5	0.18816	0.668042314425964	0.668042314425964\\
57.5	0.19182	0.715840580636809	0.715840580636809\\
57.5	0.19548	0.764694997295542	0.764694997295542\\
57.5	0.19914	0.814605564402164	0.814605564402164\\
57.5	0.2028	0.865572281956671	0.865572281956671\\
57.5	0.20646	0.917595149959068	0.917595149959068\\
57.5	0.21012	0.970674168409351	0.970674168409351\\
57.5	0.21378	1.02480933730753	1.02480933730753\\
57.5	0.21744	1.08000065665358	1.08000065665358\\
57.5	0.2211	1.13624812644753	1.13624812644753\\
57.5	0.22476	1.19355174668936	1.19355174668936\\
57.5	0.22842	1.25191151737909	1.25191151737909\\
57.5	0.23208	1.31132743851669	1.31132743851669\\
57.5	0.23574	1.37179951010219	1.37179951010219\\
57.5	0.2394	1.43332773213558	1.43332773213558\\
57.5	0.24306	1.49591210461685	1.49591210461685\\
57.5	0.24672	1.55955262754601	1.55955262754601\\
57.5	0.25038	1.62424930092305	1.62424930092305\\
57.5	0.25404	1.69000212474799	1.69000212474799\\
57.5	0.2577	1.75681109902081	1.75681109902081\\
57.5	0.26136	1.82467622374152	1.82467622374152\\
57.5	0.26502	1.89359749891012	1.89359749891012\\
57.5	0.26868	1.96357492452661	1.96357492452661\\
57.5	0.27234	2.03460850059098	2.03460850059098\\
57.5	0.276	2.10669822710324	2.10669822710324\\
57.875	0.093	-0.207287919752138	-0.207287919752138\\
57.875	0.09666	-0.188871058129958	-0.188871058129958\\
57.875	0.10032	-0.169398046059889	-0.169398046059889\\
57.875	0.10398	-0.148868883541932	-0.148868883541932\\
57.875	0.10764	-0.127283570576088	-0.127283570576088\\
57.875	0.1113	-0.104642107162356	-0.104642107162356\\
57.875	0.11496	-0.0809444933007377	-0.0809444933007377\\
57.875	0.11862	-0.0561907289912306	-0.0561907289912306\\
57.875	0.12228	-0.0303808142338358	-0.0303808142338358\\
57.875	0.12594	-0.00351474902855509	-0.00351474902855509\\
57.875	0.1296	0.0244074666246148	0.0244074666246148\\
57.875	0.13326	0.0533858327256724	0.0533858327256724\\
57.875	0.13692	0.0834203492746158	0.0834203492746158\\
57.875	0.14058	0.114511016271449	0.114511016271449\\
57.875	0.14424	0.146657833716168	0.146657833716168\\
57.875	0.1479	0.179860801608776	0.179860801608776\\
57.875	0.15156	0.214119919949271	0.214119919949271\\
57.875	0.15522	0.249435188737653	0.249435188737653\\
57.875	0.15888	0.285806607973924	0.285806607973924\\
57.875	0.16254	0.323234177658083	0.323234177658083\\
57.875	0.1662	0.361717897790127	0.361717897790127\\
57.875	0.16986	0.401257768370061	0.401257768370061\\
57.875	0.17352	0.441853789397882	0.441853789397882\\
57.875	0.17718	0.483505960873589	0.483505960873589\\
57.875	0.18084	0.526214282797186	0.526214282797186\\
57.875	0.1845	0.56997875516867	0.56997875516867\\
57.875	0.18816	0.61479937798804	0.61479937798804\\
57.875	0.19182	0.6606761512553	0.6606761512553\\
57.875	0.19548	0.707609074970445	0.707609074970445\\
57.875	0.19914	0.755598149133481	0.755598149133481\\
57.875	0.2028	0.8046433737444	0.8046433737444\\
57.875	0.20646	0.854744748803214	0.854744748803214\\
57.875	0.21012	0.905902274309909	0.905902274309909\\
57.875	0.21378	0.958115950264493	0.958115950264493\\
57.875	0.21744	1.01138577666697	1.01138577666697\\
57.875	0.2211	1.06571175351733	1.06571175351733\\
57.875	0.22476	1.12109388081557	1.12109388081557\\
57.875	0.22842	1.17753215856171	1.17753215856171\\
57.875	0.23208	1.23502658675573	1.23502658675573\\
57.875	0.23574	1.29357716539764	1.29357716539764\\
57.875	0.2394	1.35318389448744	1.35318389448744\\
57.875	0.24306	1.41384677402512	1.41384677402512\\
57.875	0.24672	1.4755658040107	1.4755658040107\\
57.875	0.25038	1.53834098444416	1.53834098444416\\
57.875	0.25404	1.60217231532551	1.60217231532551\\
57.875	0.2577	1.66705979665474	1.66705979665474\\
57.875	0.26136	1.73300342843186	1.73300342843186\\
57.875	0.26502	1.80000321065688	1.80000321065688\\
57.875	0.26868	1.86805914332977	1.86805914332977\\
57.875	0.27234	1.93717122645056	1.93717122645056\\
57.875	0.276	2.00733946001924	2.00733946001924\\
58.25	0.093	-0.209794666557529	-0.209794666557529\\
58.25	0.09666	-0.193299297878933	-0.193299297878933\\
58.25	0.10032	-0.175747778752453	-0.175747778752453\\
58.25	0.10398	-0.157140109178084	-0.157140109178084\\
58.25	0.10764	-0.137476289155825	-0.137476289155825\\
58.25	0.1113	-0.116756318685681	-0.116756318685681\\
58.25	0.11496	-0.094980197767649	-0.094980197767649\\
58.25	0.11862	-0.0721479264017294	-0.0721479264017294\\
58.25	0.12228	-0.0482595045879204	-0.0482595045879204\\
58.25	0.12594	-0.0233149323262255	-0.0233149323262255\\
58.25	0.1296	0.00268579038335681	0.00268579038335681\\
58.25	0.13326	0.0297426635408269	0.0297426635408269\\
58.25	0.13692	0.0578556871461845	0.0578556871461845\\
58.25	0.14058	0.0870248611994301	0.0870248611994301\\
58.25	0.14424	0.117250185700563	0.117250185700563\\
58.25	0.1479	0.148531660649583	0.148531660649583\\
58.25	0.15156	0.180869286046492	0.180869286046492\\
58.25	0.15522	0.214263061891289	0.214263061891289\\
58.25	0.15888	0.248712988183971	0.248712988183971\\
58.25	0.16254	0.284219064924544	0.284219064924544\\
58.25	0.1662	0.320781292113002	0.320781292113002\\
58.25	0.16986	0.358399669749348	0.358399669749348\\
58.25	0.17352	0.397074197833582	0.397074197833582\\
58.25	0.17718	0.436804876365704	0.436804876365704\\
58.25	0.18084	0.477591705345712	0.477591705345712\\
58.25	0.1845	0.519434684773609	0.519434684773609\\
58.25	0.18816	0.562333814649394	0.562333814649394\\
58.25	0.19182	0.606289094973067	0.606289094973067\\
58.25	0.19548	0.651300525744627	0.651300525744627\\
58.25	0.19914	0.697368106964072	0.697368106964072\\
58.25	0.2028	0.744491838631407	0.744491838631407\\
58.25	0.20646	0.792671720746633	0.792671720746633\\
58.25	0.21012	0.841907753309741	0.841907753309741\\
58.25	0.21378	0.89219993632074	0.89219993632074\\
58.25	0.21744	0.943548269779627	0.943548269779627\\
58.25	0.2211	0.995952753686398	0.995952753686398\\
58.25	0.22476	1.04941338804106	1.04941338804106\\
58.25	0.22842	1.10393017284361	1.10393017284361\\
58.25	0.23208	1.15950310809404	1.15950310809404\\
58.25	0.23574	1.21613219379237	1.21613219379237\\
58.25	0.2394	1.27381742993857	1.27381742993857\\
58.25	0.24306	1.33255881653267	1.33255881653267\\
58.25	0.24672	1.39235635357466	1.39235635357466\\
58.25	0.25038	1.45321004106454	1.45321004106454\\
58.25	0.25404	1.5151198790023	1.5151198790023\\
58.25	0.2577	1.57808586738795	1.57808586738795\\
58.25	0.26136	1.64210800622148	1.64210800622148\\
58.25	0.26502	1.70718629550291	1.70718629550291\\
58.25	0.26868	1.77332073523222	1.77332073523222\\
58.25	0.27234	1.84051132540942	1.84051132540942\\
58.25	0.276	1.90875806603451	1.90875806603451\\
58.625	0.093	-0.211524040263647	-0.211524040263647\\
58.625	0.09666	-0.19695016452864	-0.19695016452864\\
58.625	0.10032	-0.181320138345744	-0.181320138345744\\
58.625	0.10398	-0.164633961714963	-0.164633961714963\\
58.625	0.10764	-0.146891634636292	-0.146891634636292\\
58.625	0.1113	-0.128093157109733	-0.128093157109733\\
58.625	0.11496	-0.108238529135289	-0.108238529135289\\
58.625	0.11862	-0.0873277507129547	-0.0873277507129547\\
58.625	0.12228	-0.0653608218427351	-0.0653608218427351\\
58.625	0.12594	-0.042337742524626	-0.042337742524626\\
58.625	0.1296	-0.0182585127586294	-0.0182585127586294\\
58.625	0.13326	0.00687686745525307	0.00687686745525307\\
58.625	0.13692	0.0330683981170249	0.0330683981170249\\
58.625	0.14058	0.0603160792266829	0.0603160792266829\\
58.625	0.14424	0.08861991078423	0.08861991078423\\
58.625	0.1479	0.117979892789663	0.117979892789663\\
58.625	0.15156	0.148396025242985	0.148396025242985\\
58.625	0.15522	0.179868308144193	0.179868308144193\\
58.625	0.15888	0.212396741493291	0.212396741493291\\
58.625	0.16254	0.245981325290275	0.245981325290275\\
58.625	0.1662	0.280622059535147	0.280622059535147\\
58.625	0.16986	0.316318944227908	0.316318944227908\\
58.625	0.17352	0.353071979368554	0.353071979368554\\
58.625	0.17718	0.39088116495709	0.39088116495709\\
58.625	0.18084	0.429746500993511	0.429746500993511\\
58.625	0.1845	0.469667987477822	0.469667987477822\\
58.625	0.18816	0.510645624410019	0.510645624410019\\
58.625	0.19182	0.552679411790105	0.552679411790105\\
58.625	0.19548	0.595769349618078	0.595769349618078\\
58.625	0.19914	0.639915437893936	0.639915437893936\\
58.625	0.2028	0.685117676617687	0.685117676617687\\
58.625	0.20646	0.731376065789322	0.731376065789322\\
58.625	0.21012	0.778690605408846	0.778690605408846\\
58.625	0.21378	0.827061295476258	0.827061295476258\\
58.625	0.21744	0.876488135991557	0.876488135991557\\
58.625	0.2211	0.926971126954744	0.926971126954744\\
58.625	0.22476	0.978510268365815	0.978510268365815\\
58.625	0.22842	1.03110556022478	1.03110556022478\\
58.625	0.23208	1.08475700253163	1.08475700253163\\
58.625	0.23574	1.13946459528636	1.13946459528636\\
58.625	0.2394	1.19522833848899	1.19522833848899\\
58.625	0.24306	1.2520482321395	1.2520482321395\\
58.625	0.24672	1.3099242762379	1.3099242762379\\
58.625	0.25038	1.36885647078419	1.36885647078419\\
58.625	0.25404	1.42884481577836	1.42884481577836\\
58.625	0.2577	1.48988931122042	1.48988931122042\\
58.625	0.26136	1.55198995711037	1.55198995711037\\
58.625	0.26502	1.61514675344821	1.61514675344821\\
58.625	0.26868	1.67935970023394	1.67935970023394\\
58.625	0.27234	1.74462879746755	1.74462879746755\\
58.625	0.276	1.81095404514905	1.81095404514905\\
59	0.093	-0.212476040870491	-0.212476040870491\\
59	0.09666	-0.19982365807907	-0.19982365807907\\
59	0.10032	-0.186115124839762	-0.186115124839762\\
59	0.10398	-0.171350441152566	-0.171350441152566\\
59	0.10764	-0.155529607017483	-0.155529607017483\\
59	0.1113	-0.138652622434511	-0.138652622434511\\
59	0.11496	-0.120719487403653	-0.120719487403653\\
59	0.11862	-0.101730201924907	-0.101730201924907\\
59	0.12228	-0.0816847659982727	-0.0816847659982727\\
59	0.12594	-0.0605831796237511	-0.0605831796237511\\
59	0.1296	-0.0384254428013404	-0.0384254428013404\\
59	0.13326	-0.0152115555310437	-0.0152115555310437\\
59	0.13692	0.00905848218714056	0.00905848218714056\\
59	0.14058	0.034384670353211	0.034384670353211\\
59	0.14424	0.0607670089671706	0.0607670089671706\\
59	0.1479	0.0882054980290177	0.0882054980290177\\
59	0.15156	0.116700137538754	0.116700137538754\\
59	0.15522	0.146250927496377	0.146250927496377\\
59	0.15888	0.176857867901885	0.176857867901885\\
59	0.16254	0.208520958755283	0.208520958755283\\
59	0.1662	0.241240200056568	0.241240200056568\\
59	0.16986	0.275015591805741	0.275015591805741\\
59	0.17352	0.309847134002803	0.309847134002803\\
59	0.17718	0.345734826647751	0.345734826647751\\
59	0.18084	0.382678669740586	0.382678669740586\\
59	0.1845	0.42067866328131	0.42067866328131\\
59	0.18816	0.459734807269919	0.459734807269919\\
59	0.19182	0.499847101706418	0.499847101706418\\
59	0.19548	0.541015546590804	0.541015546590804\\
59	0.19914	0.583240141923077	0.583240141923077\\
59	0.2028	0.626520887703241	0.626520887703241\\
59	0.20646	0.670857783931292	0.670857783931292\\
59	0.21012	0.716250830607228	0.716250830607228\\
59	0.21378	0.762700027731052	0.762700027731052\\
59	0.21744	0.810205375302764	0.810205375302764\\
59	0.2211	0.858766873322363	0.858766873322363\\
59	0.22476	0.90838452178985	0.90838452178985\\
59	0.22842	0.959058320705225	0.959058320705225\\
59	0.23208	1.01078827006849	1.01078827006849\\
59	0.23574	1.06357436987963	1.06357436987963\\
59	0.2394	1.11741662013867	1.11741662013867\\
59	0.24306	1.1723150208456	1.1723150208456\\
59	0.24672	1.22826957200041	1.22826957200041\\
59	0.25038	1.28528027360311	1.28528027360311\\
59	0.25404	1.3433471256537	1.3433471256537\\
59	0.2577	1.40247012815218	1.40247012815218\\
59	0.26136	1.46264928109854	1.46264928109854\\
59	0.26502	1.52388458449279	1.52388458449279\\
59	0.26868	1.58617603833493	1.58617603833493\\
59	0.27234	1.64952364262495	1.64952364262495\\
59	0.276	1.71392739736287	1.71392739736287\\
59.375	0.093	-0.212650668378058	-0.212650668378058\\
59.375	0.09666	-0.201919778530225	-0.201919778530225\\
59.375	0.10032	-0.190132738234502	-0.190132738234502\\
59.375	0.10398	-0.177289547490894	-0.177289547490894\\
59.375	0.10764	-0.163390206299396	-0.163390206299396\\
59.375	0.1113	-0.148434714660011	-0.148434714660011\\
59.375	0.11496	-0.13242307257274	-0.13242307257274\\
59.375	0.11862	-0.11535528003758	-0.11535528003758\\
59.375	0.12228	-0.0972313370545332	-0.0972313370545332\\
59.375	0.12594	-0.0780512436235975	-0.0780512436235975\\
59.375	0.1296	-0.0578149997447761	-0.0578149997447761\\
59.375	0.13326	-0.0365226054180652	-0.0365226054180652\\
59.375	0.13692	-0.0141740606434685	-0.0141740606434685\\
59.375	0.14058	0.00923063457901796	0.00923063457901796\\
59.375	0.14424	0.0336914802493899	0.0336914802493899\\
59.375	0.1479	0.0592084763676513	0.0592084763676513\\
59.375	0.15156	0.0857816229337984	0.0857816229337984\\
59.375	0.15522	0.113410919947835	0.113410919947835\\
59.375	0.15888	0.142096367409758	0.142096367409758\\
59.375	0.16254	0.171837965319568	0.171837965319568\\
59.375	0.1662	0.202635713677267	0.202635713677267\\
59.375	0.16986	0.234489612482852	0.234489612482852\\
59.375	0.17352	0.267399661736327	0.267399661736327\\
59.375	0.17718	0.301365861437688	0.301365861437688\\
59.375	0.18084	0.336388211586937	0.336388211586937\\
59.375	0.1845	0.372466712184075	0.372466712184075\\
59.375	0.18816	0.409601363229098	0.409601363229098\\
59.375	0.19182	0.447792164722009	0.447792164722009\\
59.375	0.19548	0.48703911666281	0.48703911666281\\
59.375	0.19914	0.527342219051496	0.527342219051496\\
59.375	0.2028	0.568701471888072	0.568701471888072\\
59.375	0.20646	0.611116875172535	0.611116875172535\\
59.375	0.21012	0.654588428904884	0.654588428904884\\
59.375	0.21378	0.699116133085124	0.699116133085124\\
59.375	0.21744	0.744699987713248	0.744699987713248\\
59.375	0.2211	0.79133999278926	0.79133999278926\\
59.375	0.22476	0.839036148313159	0.839036148313159\\
59.375	0.22842	0.88778845428495	0.88778845428495\\
59.375	0.23208	0.937596910704624	0.937596910704624\\
59.375	0.23574	0.988461517572184	0.988461517572184\\
59.375	0.2394	1.04038227488764	1.04038227488764\\
59.375	0.24306	1.09335918265097	1.09335918265097\\
59.375	0.24672	1.1473922408622	1.1473922408622\\
59.375	0.25038	1.20248144952132	1.20248144952132\\
59.375	0.25404	1.25862680862832	1.25862680862832\\
59.375	0.2577	1.3158283181832	1.3158283181832\\
59.375	0.26136	1.37408597818598	1.37408597818598\\
59.375	0.26502	1.43339978863665	1.43339978863665\\
59.375	0.26868	1.4937697495352	1.4937697495352\\
59.375	0.27234	1.55519586088164	1.55519586088164\\
59.375	0.276	1.61767812267597	1.61767812267597\\
59.75	0.093	-0.21204792278635	-0.21204792278635\\
59.75	0.09666	-0.203238525882101	-0.203238525882101\\
59.75	0.10032	-0.193372978529967	-0.193372978529967\\
59.75	0.10398	-0.182451280729945	-0.182451280729945\\
59.75	0.10764	-0.170473432482035	-0.170473432482035\\
59.75	0.1113	-0.157439433786237	-0.157439433786237\\
59.75	0.11496	-0.143349284642552	-0.143349284642552\\
59.75	0.11862	-0.128202985050979	-0.128202985050979\\
59.75	0.12228	-0.112000535011519	-0.112000535011519\\
59.75	0.12594	-0.0947419345241703	-0.0947419345241703\\
59.75	0.1296	-0.0764271835889347	-0.0764271835889347\\
59.75	0.13326	-0.0570562822058114	-0.0570562822058114\\
59.75	0.13692	-0.0366292303748005	-0.0366292303748005\\
59.75	0.14058	-0.0151460280959016	-0.0151460280959016\\
59.75	0.14424	0.0073933246308846	0.0073933246308846\\
59.75	0.1479	0.0309888278055583	0.0309888278055583\\
59.75	0.15156	0.0556404814281197	0.0556404814281197\\
59.75	0.15522	0.0813482854985688	0.0813482854985688\\
59.75	0.15888	0.108112240016904	0.108112240016904\\
59.75	0.16254	0.135932344983128	0.135932344983128\\
59.75	0.1662	0.16480860039724	0.16480860039724\\
59.75	0.16986	0.194741006259239	0.194741006259239\\
59.75	0.17352	0.225729562569126	0.225729562569126\\
59.75	0.17718	0.257774269326901	0.257774269326901\\
59.75	0.18084	0.290875126532563	0.290875126532563\\
59.75	0.1845	0.325032134186113	0.325032134186113\\
59.75	0.18816	0.360245292287551	0.360245292287551\\
59.75	0.19182	0.396514600836876	0.396514600836876\\
59.75	0.19548	0.433840059834087	0.433840059834087\\
59.75	0.19914	0.472221669279189	0.472221669279189\\
59.75	0.2028	0.511659429172178	0.511659429172178\\
59.75	0.20646	0.552153339513053	0.552153339513053\\
59.75	0.21012	0.593703400301818	0.593703400301818\\
59.75	0.21378	0.636309611538467	0.636309611538467\\
59.75	0.21744	0.679971973223007	0.679971973223007\\
59.75	0.2211	0.724690485355431	0.724690485355431\\
59.75	0.22476	0.770465147935747	0.770465147935747\\
59.75	0.22842	0.817295960963946	0.817295960963946\\
59.75	0.23208	0.865182924440037	0.865182924440037\\
59.75	0.23574	0.914126038364012	0.914126038364012\\
59.75	0.2394	0.964125302735874	0.964125302735874\\
59.75	0.24306	1.01518071755563	1.01518071755563\\
59.75	0.24672	1.06729228282327	1.06729228282327\\
59.75	0.25038	1.12045999853879	1.12045999853879\\
59.75	0.25404	1.17468386470221	1.17468386470221\\
59.75	0.2577	1.22996388131351	1.22996388131351\\
59.75	0.26136	1.2863000483727	1.2863000483727\\
59.75	0.26502	1.34369236587978	1.34369236587978\\
59.75	0.26868	1.40214083383474	1.40214083383474\\
59.75	0.27234	1.4616454522376	1.4616454522376\\
59.75	0.276	1.52220622108833	1.52220622108833\\
60.125	0.093	-0.210667804095368	-0.210667804095368\\
60.125	0.09666	-0.20377990013471	-0.20377990013471\\
60.125	0.10032	-0.195835845726161	-0.195835845726161\\
60.125	0.10398	-0.186835640869724	-0.186835640869724\\
60.125	0.10764	-0.176779285565402	-0.176779285565402\\
60.125	0.1113	-0.16566677981319	-0.16566677981319\\
60.125	0.11496	-0.153498123613092	-0.153498123613092\\
60.125	0.11862	-0.140273316965105	-0.140273316965105\\
60.125	0.12228	-0.125992359869232	-0.125992359869232\\
60.125	0.12594	-0.11065525232547	-0.11065525232547\\
60.125	0.1296	-0.0942619943338217	-0.0942619943338217\\
60.125	0.13326	-0.0768125858942841	-0.0768125858942841\\
60.125	0.13692	-0.0583070270068607	-0.0583070270068607\\
60.125	0.14058	-0.0387453176715494	-0.0387453176715494\\
60.125	0.14424	-0.018127457888349	-0.018127457888349\\
60.125	0.1479	0.00354655234273715	0.00354655234273715\\
60.125	0.15156	0.026276713021711	0.026276713021711\\
60.125	0.15522	0.0500630241485742	0.0500630241485742\\
60.125	0.15888	0.0749054857233237	0.0749054857233237\\
60.125	0.16254	0.10080409774596	0.10080409774596\\
60.125	0.1662	0.127758860216486	0.127758860216486\\
60.125	0.16986	0.155769773134898	0.155769773134898\\
60.125	0.17352	0.184836836501199	0.184836836501199\\
60.125	0.17718	0.214960050315387	0.214960050315387\\
60.125	0.18084	0.246139414577463	0.246139414577463\\
60.125	0.1845	0.278374929287425	0.278374929287425\\
60.125	0.18816	0.311666594445276	0.311666594445276\\
60.125	0.19182	0.346014410051013	0.346014410051013\\
60.125	0.19548	0.38141837610464	0.38141837610464\\
60.125	0.19914	0.417878492606155	0.417878492606155\\
60.125	0.2028	0.455394759555555	0.455394759555555\\
60.125	0.20646	0.493967176952847	0.493967176952847\\
60.125	0.21012	0.533595744798021	0.533595744798021\\
60.125	0.21378	0.574280463091086	0.574280463091086\\
60.125	0.21744	0.616021331832038	0.616021331832038\\
60.125	0.2211	0.658818351020878	0.658818351020878\\
60.125	0.22476	0.702671520657603	0.702671520657603\\
60.125	0.22842	0.747580840742218	0.747580840742218\\
60.125	0.23208	0.793546311274721	0.793546311274721\\
60.125	0.23574	0.840567932255109	0.840567932255109\\
60.125	0.2394	0.888645703683383	0.888645703683383\\
60.125	0.24306	0.937779625559549	0.937779625559549\\
60.125	0.24672	0.987969697883603	0.987969697883603\\
60.125	0.25038	1.03921592065554	1.03921592065554\\
60.125	0.25404	1.09151829387537	1.09151829387537\\
60.125	0.2577	1.14487681754309	1.14487681754309\\
60.125	0.26136	1.19929149165869	1.19929149165869\\
60.125	0.26502	1.25476231622218	1.25476231622218\\
60.125	0.26868	1.31128929123356	1.31128929123356\\
60.125	0.27234	1.36887241669282	1.36887241669282\\
60.125	0.276	1.42751169259998	1.42751169259998\\
60.5	0.093	-0.208510312305111	-0.208510312305111\\
60.5	0.09666	-0.203543901288037	-0.203543901288037\\
60.5	0.10032	-0.197521339823077	-0.197521339823077\\
60.5	0.10398	-0.190442627910228	-0.190442627910228\\
60.5	0.10764	-0.182307765549492	-0.182307765549492\\
60.5	0.1113	-0.173116752740867	-0.173116752740867\\
60.5	0.11496	-0.162869589484355	-0.162869589484355\\
60.5	0.11862	-0.151566275779956	-0.151566275779956\\
60.5	0.12228	-0.139206811627669	-0.139206811627669\\
60.5	0.12594	-0.125791197027494	-0.125791197027494\\
60.5	0.1296	-0.111319431979432	-0.111319431979432\\
60.5	0.13326	-0.0957915164834815	-0.0957915164834815\\
60.5	0.13692	-0.079207450539644	-0.079207450539644\\
60.5	0.14058	-0.0615672341479202	-0.0615672341479202\\
60.5	0.14424	-0.0428708673083074	-0.0428708673083074\\
60.5	0.1479	-0.023118350020807	-0.023118350020807\\
60.5	0.15156	-0.00230968228541895	-0.00230968228541895\\
60.5	0.15522	0.0195551358978567	0.0195551358978567\\
60.5	0.15888	0.0424761045290204	0.0424761045290204\\
60.5	0.16254	0.0664532236080713	0.0664532236080713\\
60.5	0.1662	0.0914864931350079	0.0914864931350079\\
60.5	0.16986	0.117575913109834	0.117575913109834\\
60.5	0.17352	0.144721483532548	0.144721483532548\\
60.5	0.17718	0.172923204403149	0.172923204403149\\
60.5	0.18084	0.202181075721638	0.202181075721638\\
60.5	0.1845	0.232495097488014	0.232495097488014\\
60.5	0.18816	0.263865269702279	0.263865269702279\\
60.5	0.19182	0.296291592364431	0.296291592364431\\
60.5	0.19548	0.32977406547447	0.32977406547447\\
60.5	0.19914	0.364312689032397	0.364312689032397\\
60.5	0.2028	0.39990746303821	0.39990746303821\\
60.5	0.20646	0.436558387491914	0.436558387491914\\
60.5	0.21012	0.474265462393504	0.474265462393504\\
60.5	0.21378	0.513028687742981	0.513028687742981\\
60.5	0.21744	0.552848063540346	0.552848063540346\\
60.5	0.2211	0.593723589785599	0.593723589785599\\
60.5	0.22476	0.635655266478739	0.635655266478739\\
60.5	0.22842	0.678643093619767	0.678643093619767\\
60.5	0.23208	0.722687071208683	0.722687071208683\\
60.5	0.23574	0.767787199245483	0.767787199245483\\
60.5	0.2394	0.813943477730173	0.813943477730173\\
60.5	0.24306	0.861155906662752	0.861155906662752\\
60.5	0.24672	0.909424486043218	0.909424486043218\\
60.5	0.25038	0.958749215871572	0.958749215871572\\
60.5	0.25404	1.00913009614781	1.00913009614781\\
60.5	0.2577	1.06056712687194	1.06056712687194\\
60.5	0.26136	1.11306030804396	1.11306030804396\\
60.5	0.26502	1.16660963966386	1.16660963966386\\
60.5	0.26868	1.22121512173165	1.22121512173165\\
60.5	0.27234	1.27687675424733	1.27687675424733\\
60.5	0.276	1.3335945372109	1.3335945372109\\
60.875	0.093	-0.205575447415579	-0.205575447415579\\
60.875	0.09666	-0.202530529342093	-0.202530529342093\\
60.875	0.10032	-0.198429460820719	-0.198429460820719\\
60.875	0.10398	-0.193272241851455	-0.193272241851455\\
60.875	0.10764	-0.187058872434306	-0.187058872434306\\
60.875	0.1113	-0.179789352569268	-0.179789352569268\\
60.875	0.11496	-0.171463682256343	-0.171463682256343\\
60.875	0.11862	-0.162081861495529	-0.162081861495529\\
60.875	0.12228	-0.15164389028683	-0.15164389028683\\
60.875	0.12594	-0.140149768630243	-0.140149768630243\\
60.875	0.1296	-0.127599496525766	-0.127599496525766\\
60.875	0.13326	-0.113993073973404	-0.113993073973404\\
60.875	0.13692	-0.0993305009731519	-0.0993305009731519\\
60.875	0.14058	-0.0836117775250139	-0.0836117775250139\\
60.875	0.14424	-0.0668369036289886	-0.0668369036289886\\
60.875	0.1479	-0.0490058792850758	-0.0490058792850758\\
60.875	0.15156	-0.0301187044932736	-0.0301187044932736\\
60.875	0.15522	-0.0101753792535855	-0.0101753792535855\\
60.875	0.15888	0.0108240964339906	0.0108240964339906\\
60.875	0.16254	0.0328797225694557	0.0328797225694557\\
60.875	0.1662	0.0559914991528065	0.0559914991528065\\
60.875	0.16986	0.080159426184045	0.080159426184045\\
60.875	0.17352	0.105383503663173	0.105383503663173\\
60.875	0.17718	0.131663731590187	0.131663731590187\\
60.875	0.18084	0.159000109965088	0.159000109965088\\
60.875	0.1845	0.187392638787879	0.187392638787879\\
60.875	0.18816	0.216841318058556	0.216841318058556\\
60.875	0.19182	0.24734614777712	0.24734614777712\\
60.875	0.19548	0.278907127943572	0.278907127943572\\
60.875	0.19914	0.311524258557915	0.311524258557915\\
60.875	0.2028	0.34519753962014	0.34519753962014\\
60.875	0.20646	0.379926971130257	0.379926971130257\\
60.875	0.21012	0.415712553088259	0.415712553088259\\
60.875	0.21378	0.452554285494152	0.452554285494152\\
60.875	0.21744	0.49045216834793	0.49045216834793\\
60.875	0.2211	0.529406201649595	0.529406201649595\\
60.875	0.22476	0.569416385399148	0.569416385399148\\
60.875	0.22842	0.610482719596588	0.610482719596588\\
60.875	0.23208	0.652605204241919	0.652605204241919\\
60.875	0.23574	0.695783839335132	0.695783839335132\\
60.875	0.2394	0.740018624876235	0.740018624876235\\
60.875	0.24306	0.785309560865226	0.785309560865226\\
60.875	0.24672	0.831656647302108	0.831656647302108\\
60.875	0.25038	0.879059884186874	0.879059884186874\\
60.875	0.25404	0.927519271519528	0.927519271519528\\
60.875	0.2577	0.977034809300072	0.977034809300072\\
60.875	0.26136	1.0276064975285	1.0276064975285\\
60.875	0.26502	1.07923433620482	1.07923433620482\\
60.875	0.26868	1.13191832532902	1.13191832532902\\
60.875	0.27234	1.18565846490111	1.18565846490111\\
60.875	0.276	1.2404547549211	1.2404547549211\\
61.25	0.093	-0.201863209426771	-0.201863209426771\\
61.25	0.09666	-0.200739784296873	-0.200739784296873\\
61.25	0.10032	-0.198560208719083	-0.198560208719083\\
61.25	0.10398	-0.195324482693409	-0.195324482693409\\
61.25	0.10764	-0.191032606219845	-0.191032606219845\\
61.25	0.1113	-0.185684579298394	-0.185684579298394\\
61.25	0.11496	-0.179280401929056	-0.179280401929056\\
61.25	0.11862	-0.17182007411183	-0.17182007411183\\
61.25	0.12228	-0.163303595846716	-0.163303595846716\\
61.25	0.12594	-0.153730967133714	-0.153730967133714\\
61.25	0.1296	-0.143102187972825	-0.143102187972825\\
61.25	0.13326	-0.131417258364049	-0.131417258364049\\
61.25	0.13692	-0.118676178307386	-0.118676178307386\\
61.25	0.14058	-0.104878947802834	-0.104878947802834\\
61.25	0.14424	-0.0900255668503946	-0.0900255668503946\\
61.25	0.1479	-0.0741160354500676	-0.0741160354500676\\
61.25	0.15156	-0.0571503536018547	-0.0571503536018547\\
61.25	0.15522	-0.0391285213057524	-0.0391285213057524\\
61.25	0.15888	-0.0200505385617621	-0.0200505385617621\\
61.25	0.16254	8.35946301154245e-05	8.35946301154245e-05\\
61.25	0.1662	0.0212738782698787	0.0212738782698787\\
61.25	0.16986	0.0435203123575314	0.0435203123575314\\
61.25	0.17352	0.0668228968930717	0.0668228968930717\\
61.25	0.17718	0.0911816318764997	0.0911816318764997\\
61.25	0.18084	0.116596517307815	0.116596517307815\\
61.25	0.1845	0.143067553187018	0.143067553187018\\
61.25	0.18816	0.170594739514108	0.170594739514108\\
61.25	0.19182	0.199178076289088	0.199178076289088\\
61.25	0.19548	0.228817563511952	0.228817563511952\\
61.25	0.19914	0.259513201182704	0.259513201182704\\
61.25	0.2028	0.291264989301346	0.291264989301346\\
61.25	0.20646	0.324072927867878	0.324072927867878\\
61.25	0.21012	0.357937016882293	0.357937016882293\\
61.25	0.21378	0.392857256344595	0.392857256344595\\
61.25	0.21744	0.428833646254788	0.428833646254788\\
61.25	0.2211	0.465866186612866	0.465866186612866\\
61.25	0.22476	0.503954877418835	0.503954877418835\\
61.25	0.22842	0.543099718672687	0.543099718672687\\
61.25	0.23208	0.583300710374428	0.583300710374428\\
61.25	0.23574	0.624557852524056	0.624557852524056\\
61.25	0.2394	0.666871145121575	0.666871145121575\\
61.25	0.24306	0.710240588166978	0.710240588166978\\
61.25	0.24672	0.754666181660273	0.754666181660273\\
61.25	0.25038	0.800147925601452	0.800147925601452\\
61.25	0.25404	0.846685819990521	0.846685819990521\\
61.25	0.2577	0.894279864827475	0.894279864827475\\
61.25	0.26136	0.94293006011232	0.94293006011232\\
61.25	0.26502	0.992636405845048	0.992636405845048\\
61.25	0.26868	1.04339890202567	1.04339890202567\\
61.25	0.27234	1.09521754865417	1.09521754865417\\
61.25	0.276	1.14809234573056	1.14809234573056\\
61.625	0.093	-0.197373598338689	-0.197373598338689\\
61.625	0.09666	-0.198171666152377	-0.198171666152377\\
61.625	0.10032	-0.197913583518175	-0.197913583518175\\
61.625	0.10398	-0.196599350436087	-0.196599350436087\\
61.625	0.10764	-0.194228966906111	-0.194228966906111\\
61.625	0.1113	-0.190802432928246	-0.190802432928246\\
61.625	0.11496	-0.186319748502495	-0.186319748502495\\
61.625	0.11862	-0.180780913628854	-0.180780913628854\\
61.625	0.12228	-0.174185928307328	-0.174185928307328\\
61.625	0.12594	-0.166534792537914	-0.166534792537914\\
61.625	0.1296	-0.157827506320613	-0.157827506320613\\
61.625	0.13326	-0.148064069655422	-0.148064069655422\\
61.625	0.13692	-0.137244482542345	-0.137244482542345\\
61.625	0.14058	-0.125368744981381	-0.125368744981381\\
61.625	0.14424	-0.112436856972529	-0.112436856972529\\
61.625	0.1479	-0.0984488185157877	-0.0984488185157877\\
61.625	0.15156	-0.0834046296111606	-0.0834046296111606\\
61.625	0.15522	-0.0673042902586458	-0.0673042902586458\\
61.625	0.15888	-0.0501478004582431	-0.0501478004582431\\
61.625	0.16254	-0.0319351602099531	-0.0319351602099531\\
61.625	0.1662	-0.0126663695137739	-0.0126663695137739\\
61.625	0.16986	0.00765857163029127	0.00765857163029127\\
61.625	0.17352	0.029039663222244	0.029039663222244\\
61.625	0.17718	0.0514769052620845	0.0514769052620845\\
61.625	0.18084	0.0749702977498141	0.0749702977498141\\
61.625	0.1845	0.0995198406854299	0.0995198406854299\\
61.625	0.18816	0.125125534068932	0.125125534068932\\
61.625	0.19182	0.151787377900324	0.151787377900324\\
61.625	0.19548	0.179505372179604	0.179505372179604\\
61.625	0.19914	0.208279516906769	0.208279516906769\\
61.625	0.2028	0.238109812081823	0.238109812081823\\
61.625	0.20646	0.268996257704768	0.268996257704768\\
61.625	0.21012	0.300938853775598	0.300938853775598\\
61.625	0.21378	0.333937600294313	0.333937600294313\\
61.625	0.21744	0.367992497260919	0.367992497260919\\
61.625	0.2211	0.403103544675412	0.403103544675412\\
61.625	0.22476	0.43927074253779	0.43927074253779\\
61.625	0.22842	0.476494090848059	0.476494090848059\\
61.625	0.23208	0.514773589606215	0.514773589606215\\
61.625	0.23574	0.554109238812253	0.554109238812253\\
61.625	0.2394	0.594501038466184	0.594501038466184\\
61.625	0.24306	0.635948988568003	0.635948988568003\\
61.625	0.24672	0.678453089117706	0.678453089117706\\
61.625	0.25038	0.722013340115301	0.722013340115301\\
61.625	0.25404	0.766629741560783	0.766629741560783\\
61.625	0.2577	0.812302293454153	0.812302293454153\\
61.625	0.26136	0.859030995795406	0.859030995795406\\
61.625	0.26502	0.906815848584551	0.906815848584551\\
61.625	0.26868	0.955656851821584	0.955656851821584\\
61.625	0.27234	1.0055540055065	1.0055540055065\\
61.625	0.276	1.05650730963931	1.05650730963931\\
62	0.093	-0.192106614151332	-0.192106614151332\\
62	0.09666	-0.194826174908605	-0.194826174908605\\
62	0.10032	-0.196489585217992	-0.196489585217992\\
62	0.10398	-0.19709684507949	-0.19709684507949\\
62	0.10764	-0.1966479544931	-0.1966479544931\\
62	0.1113	-0.195142913458822	-0.195142913458822\\
62	0.11496	-0.192581721976659	-0.192581721976659\\
62	0.11862	-0.188964380046606	-0.188964380046606\\
62	0.12228	-0.184290887668665	-0.184290887668665\\
62	0.12594	-0.178561244842837	-0.178561244842837\\
62	0.1296	-0.171775451569123	-0.171775451569123\\
62	0.13326	-0.16393350784752	-0.16393350784752\\
62	0.13692	-0.155035413678029	-0.155035413678029\\
62	0.14058	-0.145081169060652	-0.145081169060652\\
62	0.14424	-0.134070773995386	-0.134070773995386\\
62	0.1479	-0.122004228482232	-0.122004228482232\\
62	0.15156	-0.108881532521193	-0.108881532521193\\
62	0.15522	-0.094702686112264	-0.094702686112264\\
62	0.15888	-0.079467689255447	-0.079467689255447\\
62	0.16254	-0.0631765419507446	-0.0631765419507446\\
62	0.1662	-0.0458292441981529	-0.0458292441981529\\
62	0.16986	-0.0274257959976736	-0.0274257959976736\\
62	0.17352	-0.00796619734930659	-0.00796619734930659\\
62	0.17718	0.0125495517469463	0.0125495517469463\\
62	0.18084	0.0341214512910883	0.0341214512910883\\
62	0.1845	0.0567495012831165	0.0567495012831165\\
62	0.18816	0.0804337017230361	0.0804337017230361\\
62	0.19182	0.105174052610841	0.105174052610841\\
62	0.19548	0.13097055394653	0.13097055394653\\
62	0.19914	0.157823205730111	0.157823205730111\\
62	0.2028	0.185732007961577	0.185732007961577\\
62	0.20646	0.214696960640934	0.214696960640934\\
62	0.21012	0.244718063768177	0.244718063768177\\
62	0.21378	0.275795317343308	0.275795317343308\\
62	0.21744	0.307928721366326	0.307928721366326\\
62	0.2211	0.341118275837232	0.341118275837232\\
62	0.22476	0.375363980756026	0.375363980756026\\
62	0.22842	0.410665836122707	0.410665836122707\\
62	0.23208	0.447023841937272	0.447023841937272\\
62	0.23574	0.484437998199729	0.484437998199729\\
62	0.2394	0.522908304910073	0.522908304910073\\
62	0.24306	0.562434762068301	0.562434762068301\\
62	0.24672	0.60301736967442	0.60301736967442\\
62	0.25038	0.644656127728428	0.644656127728428\\
62	0.25404	0.687351036230322	0.687351036230322\\
62	0.2577	0.731102095180104	0.731102095180104\\
62	0.26136	0.775909304577774	0.775909304577774\\
62	0.26502	0.821772664423331	0.821772664423331\\
62	0.26868	0.868692174716776	0.868692174716776\\
62	0.27234	0.916667835458108	0.916667835458108\\
62	0.276	0.965699646647328	0.965699646647328\\
62.375	0.093	-0.186062256864699	-0.186062256864699\\
62.375	0.09666	-0.190703310565559	-0.190703310565559\\
62.375	0.10032	-0.194288213818532	-0.194288213818532\\
62.375	0.10398	-0.196816966623617	-0.196816966623617\\
62.375	0.10764	-0.198289568980813	-0.198289568980813\\
62.375	0.1113	-0.198706020890123	-0.198706020890123\\
62.375	0.11496	-0.198066322351545	-0.198066322351545\\
62.375	0.11862	-0.196370473365078	-0.196370473365078\\
62.375	0.12228	-0.193618473930725	-0.193618473930725\\
62.375	0.12594	-0.189810324048485	-0.189810324048485\\
62.375	0.1296	-0.184946023718357	-0.184946023718357\\
62.375	0.13326	-0.179025572940341	-0.179025572940341\\
62.375	0.13692	-0.172048971714436	-0.172048971714436\\
62.375	0.14058	-0.164016220040647	-0.164016220040647\\
62.375	0.14424	-0.154927317918966	-0.154927317918966\\
62.375	0.1479	-0.1447822653494	-0.1447822653494\\
62.375	0.15156	-0.133581062331946	-0.133581062331946\\
62.375	0.15522	-0.121323708866605	-0.121323708866605\\
62.375	0.15888	-0.108010204953376	-0.108010204953376\\
62.375	0.16254	-0.0936405505922591	-0.0936405505922591\\
62.375	0.1662	-0.0782147457832532	-0.0782147457832532\\
62.375	0.16986	-0.0617327905263614	-0.0617327905263614\\
62.375	0.17352	-0.044194684821582	-0.044194684821582\\
62.375	0.17718	-0.0256004286689149	-0.0256004286689149\\
62.375	0.18084	-0.00595002206836037	-0.00595002206836037\\
62.375	0.1845	0.014756534980082	0.014756534980082\\
62.375	0.18816	0.036519242476414	0.036519242476414\\
62.375	0.19182	0.0593381004206315	0.0593381004206315\\
62.375	0.19548	0.0832131088127366	0.0832131088127366\\
62.375	0.19914	0.108144267652729	0.108144267652729\\
62.375	0.2028	0.134131576940612	0.134131576940612\\
62.375	0.20646	0.161175036676382	0.161175036676382\\
62.375	0.21012	0.189274646860037	0.189274646860037\\
62.375	0.21378	0.21843040749158	0.21843040749158\\
62.375	0.21744	0.248642318571011	0.248642318571011\\
62.375	0.2211	0.279910380098329	0.279910380098329\\
62.375	0.22476	0.312234592073535	0.312234592073535\\
62.375	0.22842	0.345614954496629	0.345614954496629\\
62.375	0.23208	0.38005146736761	0.38005146736761\\
62.375	0.23574	0.41554413068648	0.41554413068648\\
62.375	0.2394	0.452092944453236	0.452092944453236\\
62.375	0.24306	0.48969790866788	0.48969790866788\\
62.375	0.24672	0.528359023330411	0.528359023330411\\
62.375	0.25038	0.568076288440835	0.568076288440835\\
62.375	0.25404	0.608849703999141	0.608849703999141\\
62.375	0.2577	0.650679270005336	0.650679270005336\\
62.375	0.26136	0.693564986459418	0.693564986459418\\
62.375	0.26502	0.737506853361388	0.737506853361388\\
62.375	0.26868	0.782504870711245	0.782504870711245\\
62.375	0.27234	0.82855903850899	0.82855903850899\\
62.375	0.276	0.875669356754623	0.875669356754623\\
62.75	0.093	-0.179240526478792	-0.179240526478792\\
62.75	0.09666	-0.18580307312324	-0.18580307312324\\
62.75	0.10032	-0.191309469319799	-0.191309469319799\\
62.75	0.10398	-0.19575971506847	-0.19575971506847\\
62.75	0.10764	-0.199153810369253	-0.199153810369253\\
62.75	0.1113	-0.201491755222149	-0.201491755222149\\
62.75	0.11496	-0.202773549627159	-0.202773549627159\\
62.75	0.11862	-0.202999193584279	-0.202999193584279\\
62.75	0.12228	-0.202168687093512	-0.202168687093512\\
62.75	0.12594	-0.200282030154859	-0.200282030154859\\
62.75	0.1296	-0.197339222768317	-0.197339222768317\\
62.75	0.13326	-0.193340264933887	-0.193340264933887\\
62.75	0.13692	-0.188285156651571	-0.188285156651571\\
62.75	0.14058	-0.182173897921366	-0.182173897921366\\
62.75	0.14424	-0.175006488743275	-0.175006488743275\\
62.75	0.1479	-0.166782929117294	-0.166782929117294\\
62.75	0.15156	-0.157503219043426	-0.157503219043426\\
62.75	0.15522	-0.147167358521673	-0.147167358521673\\
62.75	0.15888	-0.135775347552029	-0.135775347552029\\
62.75	0.16254	-0.1233271861345	-0.1233271861345\\
62.75	0.1662	-0.109822874269082	-0.109822874269082\\
62.75	0.16986	-0.0952624119557757	-0.0952624119557757\\
62.75	0.17352	-0.0796457991945838	-0.0796457991945838\\
62.75	0.17718	-0.0629730359855025	-0.0629730359855025\\
62.75	0.18084	-0.0452441223285338	-0.0452441223285338\\
62.75	0.1845	-0.026459058223679	-0.026459058223679\\
62.75	0.18816	-0.00661784367093454	-0.00661784367093454\\
62.75	0.19182	0.0142795213296953	0.0142795213296953\\
62.75	0.19548	0.0362330367782164	0.0362330367782164\\
62.75	0.19914	0.0592427026746216	0.0592427026746216\\
62.75	0.2028	0.0833085190189164	0.0833085190189164\\
62.75	0.20646	0.108430485811099	0.108430485811099\\
62.75	0.21012	0.134608603051167	0.134608603051167\\
62.75	0.21378	0.161842870739126	0.161842870739126\\
62.75	0.21744	0.190133288874969	0.190133288874969\\
62.75	0.2211	0.2194798574587	0.2194798574587\\
62.75	0.22476	0.249882576490322	0.249882576490322\\
62.75	0.22842	0.281341445969828	0.281341445969828\\
62.75	0.23208	0.313856465897221	0.313856465897221\\
62.75	0.23574	0.347427636272503	0.347427636272503\\
62.75	0.2394	0.382054957095672	0.382054957095672\\
62.75	0.24306	0.417738428366732	0.417738428366732\\
62.75	0.24672	0.454478050085676	0.454478050085676\\
62.75	0.25038	0.492273822252512	0.492273822252512\\
62.75	0.25404	0.531125744867231	0.531125744867231\\
62.75	0.2577	0.571033817929838	0.571033817929838\\
62.75	0.26136	0.611998041440336	0.611998041440336\\
62.75	0.26502	0.654018415398718	0.654018415398718\\
62.75	0.26868	0.697094939804988	0.697094939804988\\
62.75	0.27234	0.741227614659149	0.741227614659149\\
62.75	0.276	0.786416439961194	0.786416439961194\\
63.125	0.093	-0.17164142299361	-0.17164142299361\\
63.125	0.09666	-0.180125462581642	-0.180125462581642\\
63.125	0.10032	-0.18755335172179	-0.18755335172179\\
63.125	0.10398	-0.193925090414047	-0.193925090414047\\
63.125	0.10764	-0.199240678658418	-0.199240678658418\\
63.125	0.1113	-0.203500116454901	-0.203500116454901\\
63.125	0.11496	-0.206703403803496	-0.206703403803496\\
63.125	0.11862	-0.208850540704205	-0.208850540704205\\
63.125	0.12228	-0.209941527157023	-0.209941527157023\\
63.125	0.12594	-0.209976363161956	-0.209976363161956\\
63.125	0.1296	-0.208955048719002	-0.208955048719002\\
63.125	0.13326	-0.206877583828159	-0.206877583828159\\
63.125	0.13692	-0.203743968489429	-0.203743968489429\\
63.125	0.14058	-0.199554202702811	-0.199554202702811\\
63.125	0.14424	-0.194308286468306	-0.194308286468306\\
63.125	0.1479	-0.188006219785913	-0.188006219785913\\
63.125	0.15156	-0.180648002655633	-0.180648002655633\\
63.125	0.15522	-0.172233635077463	-0.172233635077463\\
63.125	0.15888	-0.162763117051409	-0.162763117051409\\
63.125	0.16254	-0.152236448577466	-0.152236448577466\\
63.125	0.1662	-0.140653629655633	-0.140653629655633\\
63.125	0.16986	-0.128014660285915	-0.128014660285915\\
63.125	0.17352	-0.114319540468309	-0.114319540468309\\
63.125	0.17718	-0.0995682702028149	-0.0995682702028149\\
63.125	0.18084	-0.0837608494894337	-0.0837608494894337\\
63.125	0.1845	-0.0668972783281647	-0.0668972783281647\\
63.125	0.18816	-0.0489775567190078	-0.0489775567190078\\
63.125	0.19182	-0.030001684661962	-0.030001684661962\\
63.125	0.19548	-0.00996966215703199	-0.00996966215703199\\
63.125	0.19914	0.0111185107957892	0.0111185107957892\\
63.125	0.2028	0.0332628341964964	0.0332628341964964\\
63.125	0.20646	0.0564633080450911	0.0564633080450911\\
63.125	0.21012	0.080719932341575	0.080719932341575\\
63.125	0.21378	0.106032707085943	0.106032707085943\\
63.125	0.21744	0.132401632278202	0.132401632278202\\
63.125	0.2211	0.159826707918349	0.159826707918349\\
63.125	0.22476	0.18830793400638	0.18830793400638\\
63.125	0.22842	0.217845310542302	0.217845310542302\\
63.125	0.23208	0.248438837526108	0.248438837526108\\
63.125	0.23574	0.280088514957802	0.280088514957802\\
63.125	0.2394	0.312794342837387	0.312794342837387\\
63.125	0.24306	0.346556321164856	0.346556321164856\\
63.125	0.24672	0.381374449940216	0.381374449940216\\
63.125	0.25038	0.417248729163464	0.417248729163464\\
63.125	0.25404	0.454179158834596	0.454179158834596\\
63.125	0.2577	0.492165738953619	0.492165738953619\\
63.125	0.26136	0.531208469520525	0.531208469520525\\
63.125	0.26502	0.571307350535323	0.571307350535323\\
63.125	0.26868	0.61246238199801	0.61246238199801\\
63.125	0.27234	0.654673563908579	0.654673563908579\\
63.125	0.276	0.69794089626704	0.69794089626704\\
63.5	0.093	-0.163264946409153	-0.163264946409153\\
63.5	0.09666	-0.173670478940772	-0.173670478940772\\
63.5	0.10032	-0.183019861024504	-0.183019861024504\\
63.5	0.10398	-0.19131309266035	-0.19131309266035\\
63.5	0.10764	-0.198550173848307	-0.198550173848307\\
63.5	0.1113	-0.204731104588376	-0.204731104588376\\
63.5	0.11496	-0.209855884880559	-0.209855884880559\\
63.5	0.11862	-0.213924514724853	-0.213924514724853\\
63.5	0.12228	-0.216936994121259	-0.216936994121259\\
63.5	0.12594	-0.21889332306978	-0.21889332306978\\
63.5	0.1296	-0.219793501570411	-0.219793501570411\\
63.5	0.13326	-0.219637529623156	-0.219637529623156\\
63.5	0.13692	-0.218425407228012	-0.218425407228012\\
63.5	0.14058	-0.216157134384982	-0.216157134384982\\
63.5	0.14424	-0.212832711094062	-0.212832711094062\\
63.5	0.1479	-0.208452137355257	-0.208452137355257\\
63.5	0.15156	-0.203015413168562	-0.203015413168562\\
63.5	0.15522	-0.19652253853398	-0.19652253853398\\
63.5	0.15888	-0.188973513451512	-0.188973513451512\\
63.5	0.16254	-0.180368337921154	-0.180368337921154\\
63.5	0.1662	-0.170707011942911	-0.170707011942911\\
63.5	0.16986	-0.159989535516778	-0.159989535516778\\
63.5	0.17352	-0.14821590864276	-0.14821590864276\\
63.5	0.17718	-0.135386131320852	-0.135386131320852\\
63.5	0.18084	-0.121500203551058	-0.121500203551058\\
63.5	0.1845	-0.106558125333375	-0.106558125333375\\
63.5	0.18816	-0.0905598966678058	-0.0905598966678058\\
63.5	0.19182	-0.0735055175543475	-0.0735055175543475\\
63.5	0.19548	-0.0553949879930016	-0.0553949879930016\\
63.5	0.19914	-0.036228307983768	-0.036228307983768\\
63.5	0.2028	-0.0160054775266483	-0.0160054775266483\\
63.5	0.20646	0.00527350337836241	0.00527350337836241\\
63.5	0.21012	0.0276086347312552	0.0276086347312552\\
63.5	0.21378	0.0509999165320392	0.0509999165320392\\
63.5	0.21744	0.0754473487807106	0.0754473487807106\\
63.5	0.2211	0.10095093147727	0.10095093147727\\
63.5	0.22476	0.127510664621717	0.127510664621717\\
63.5	0.22842	0.155126548214048	0.155126548214048\\
63.5	0.23208	0.18379858225427	0.18379858225427\\
63.5	0.23574	0.213526766742377	0.213526766742377\\
63.5	0.2394	0.244311101678373	0.244311101678373\\
63.5	0.24306	0.276151587062258	0.276151587062258\\
63.5	0.24672	0.309048222894031	0.309048222894031\\
63.5	0.25038	0.343001009173692	0.343001009173692\\
63.5	0.25404	0.378009945901239	0.378009945901239\\
63.5	0.2577	0.414075033076671	0.414075033076671\\
63.5	0.26136	0.451196270699994	0.451196270699994\\
63.5	0.26502	0.489373658771204	0.489373658771204\\
63.5	0.26868	0.528607197290303	0.528607197290303\\
63.5	0.27234	0.568896886257289	0.568896886257289\\
63.5	0.276	0.610242725672158	0.610242725672158\\
63.875	0.093	-0.154111096725422	-0.154111096725422\\
63.875	0.09666	-0.166438122200629	-0.166438122200629\\
63.875	0.10032	-0.177708997227947	-0.177708997227947\\
63.875	0.10398	-0.187923721807378	-0.187923721807378\\
63.875	0.10764	-0.197082295938923	-0.197082295938923\\
63.875	0.1113	-0.205184719622579	-0.205184719622579\\
63.875	0.11496	-0.212230992858348	-0.212230992858348\\
63.875	0.11862	-0.21822111564623	-0.21822111564623\\
63.875	0.12228	-0.223155087986224	-0.223155087986224\\
63.875	0.12594	-0.227032909878328	-0.227032909878328\\
63.875	0.1296	-0.229854581322547	-0.229854581322547\\
63.875	0.13326	-0.231620102318878	-0.231620102318878\\
63.875	0.13692	-0.232329472867321	-0.232329472867321\\
63.875	0.14058	-0.231982692967878	-0.231982692967878\\
63.875	0.14424	-0.230579762620545	-0.230579762620545\\
63.875	0.1479	-0.228120681825325	-0.228120681825325\\
63.875	0.15156	-0.224605450582218	-0.224605450582218\\
63.875	0.15522	-0.220034068891224	-0.220034068891224\\
63.875	0.15888	-0.214406536752343	-0.214406536752343\\
63.875	0.16254	-0.207722854165571	-0.207722854165571\\
63.875	0.1662	-0.199983021130914	-0.199983021130914\\
63.875	0.16986	-0.191187037648369	-0.191187037648369\\
63.875	0.17352	-0.181334903717936	-0.181334903717936\\
63.875	0.17718	-0.170426619339616	-0.170426619339616\\
63.875	0.18084	-0.158462184513408	-0.158462184513408\\
63.875	0.1845	-0.145441599239312	-0.145441599239312\\
63.875	0.18816	-0.13136486351733	-0.13136486351733\\
63.875	0.19182	-0.11623197734746	-0.11623197734746\\
63.875	0.19548	-0.100042940729701	-0.100042940729701\\
63.875	0.19914	-0.0827977536640552	-0.0827977536640552\\
63.875	0.2028	-0.0644964161505195	-0.0644964161505195\\
63.875	0.20646	-0.0451389281890964	-0.0451389281890964\\
63.875	0.21012	-0.0247252897797876	-0.0247252897797876\\
63.875	0.21378	-0.00325550092259119	-0.00325550092259119\\
63.875	0.21744	0.0192704383824926	0.0192704383824926\\
63.875	0.2211	0.0428525281354644	0.0428525281354644\\
63.875	0.22476	0.0674907683363237	0.0674907683363237\\
63.875	0.22842	0.0931851589850705	0.0931851589850705\\
63.875	0.23208	0.119935700081705	0.119935700081705\\
63.875	0.23574	0.147742391626224	0.147742391626224\\
63.875	0.2394	0.176605233618637	0.176605233618637\\
63.875	0.24306	0.206524226058935	0.206524226058935\\
63.875	0.24672	0.23749936894712	0.23749936894712\\
63.875	0.25038	0.269530662283193	0.269530662283193\\
63.875	0.25404	0.302618106067153	0.302618106067153\\
63.875	0.2577	0.336761700299001	0.336761700299001\\
63.875	0.26136	0.371961444978736	0.371961444978736\\
63.875	0.26502	0.408217340106359	0.408217340106359\\
63.875	0.26868	0.44552938568187	0.44552938568187\\
63.875	0.27234	0.483897581705268	0.483897581705268\\
63.875	0.276	0.523321928176554	0.523321928176554\\
64.25	0.093	-0.144179873942414	-0.144179873942414\\
64.25	0.09666	-0.158428392361205	-0.158428392361205\\
64.25	0.10032	-0.171620760332112	-0.171620760332112\\
64.25	0.10398	-0.183756977855131	-0.183756977855131\\
64.25	0.10764	-0.194837044930261	-0.194837044930261\\
64.25	0.1113	-0.204860961557504	-0.204860961557504\\
64.25	0.11496	-0.21382872773686	-0.21382872773686\\
64.25	0.11862	-0.221740343468328	-0.221740343468328\\
64.25	0.12228	-0.228595808751909	-0.228595808751909\\
64.25	0.12594	-0.234395123587601	-0.234395123587601\\
64.25	0.1296	-0.239138287975406	-0.239138287975406\\
64.25	0.13326	-0.242825301915324	-0.242825301915324\\
64.25	0.13692	-0.245456165407353	-0.245456165407353\\
64.25	0.14058	-0.247030878451496	-0.247030878451496\\
64.25	0.14424	-0.247549441047752	-0.247549441047752\\
64.25	0.1479	-0.247011853196119	-0.247011853196119\\
64.25	0.15156	-0.245418114896599	-0.245418114896599\\
64.25	0.15522	-0.24276822614919	-0.24276822614919\\
64.25	0.15888	-0.239062186953895	-0.239062186953895\\
64.25	0.16254	-0.234299997310711	-0.234299997310711\\
64.25	0.1662	-0.228481657219641	-0.228481657219641\\
64.25	0.16986	-0.221607166680682	-0.221607166680682\\
64.25	0.17352	-0.213676525693837	-0.213676525693837\\
64.25	0.17718	-0.204689734259102	-0.204689734259102\\
64.25	0.18084	-0.19464679237648	-0.19464679237648\\
64.25	0.1845	-0.183547700045972	-0.183547700045972\\
64.25	0.18816	-0.171392457267574	-0.171392457267574\\
64.25	0.19182	-0.158181064041291	-0.158181064041291\\
64.25	0.19548	-0.14391352036712	-0.14391352036712\\
64.25	0.19914	-0.128589826245062	-0.128589826245062\\
64.25	0.2028	-0.112209981675114	-0.112209981675114\\
64.25	0.20646	-0.0947739866572781	-0.0947739866572781\\
64.25	0.21012	-0.0762818411915569	-0.0762818411915569\\
64.25	0.21378	-0.0567335452779445	-0.0567335452779445\\
64.25	0.21744	-0.0361290989164482	-0.0361290989164482\\
64.25	0.2211	-0.0144685021070641	-0.0144685021070641\\
64.25	0.22476	0.00824824515021128	0.00824824515021128\\
64.25	0.22842	0.0320211428553705	0.0320211428553705\\
64.25	0.23208	0.0568501910084174	0.0568501910084174\\
64.25	0.23574	0.0827353896093528	0.0827353896093528\\
64.25	0.2394	0.109676738658175	0.109676738658175\\
64.25	0.24306	0.137674238154884	0.137674238154884\\
64.25	0.24672	0.166727888099482	0.166727888099482\\
64.25	0.25038	0.196837688491971	0.196837688491971\\
64.25	0.25404	0.228003639332343	0.228003639332343\\
64.25	0.2577	0.260225740620607	0.260225740620607\\
64.25	0.26136	0.293503992356755	0.293503992356755\\
64.25	0.26502	0.32783839454079	0.32783839454079\\
64.25	0.26868	0.363228947172713	0.363228947172713\\
64.25	0.27234	0.399675650252528	0.399675650252528\\
64.25	0.276	0.437178503780226	0.437178503780226\\
64.625	0.093	-0.133471278060132	-0.133471278060132\\
64.625	0.09666	-0.149641289422511	-0.149641289422511\\
64.625	0.10032	-0.164755150337006	-0.164755150337006\\
64.625	0.10398	-0.178812860803611	-0.178812860803611\\
64.625	0.10764	-0.191814420822329	-0.191814420822329\\
64.625	0.1113	-0.203759830393158	-0.203759830393158\\
64.625	0.11496	-0.214649089516101	-0.214649089516101\\
64.625	0.11862	-0.224482198191156	-0.224482198191156\\
64.625	0.12228	-0.233259156418323	-0.233259156418323\\
64.625	0.12594	-0.240979964197602	-0.240979964197602\\
64.625	0.1296	-0.247644621528994	-0.247644621528994\\
64.625	0.13326	-0.253253128412499	-0.253253128412499\\
64.625	0.13692	-0.257805484848116	-0.257805484848116\\
64.625	0.14058	-0.261301690835844	-0.261301690835844\\
64.625	0.14424	-0.263741746375686	-0.263741746375686\\
64.625	0.1479	-0.26512565146764	-0.26512565146764\\
64.625	0.15156	-0.265453406111706	-0.265453406111706\\
64.625	0.15522	-0.264725010307885	-0.264725010307885\\
64.625	0.15888	-0.262940464056177	-0.262940464056177\\
64.625	0.16254	-0.260099767356581	-0.260099767356581\\
64.625	0.1662	-0.256202920209097	-0.256202920209097\\
64.625	0.16986	-0.251249922613725	-0.251249922613725\\
64.625	0.17352	-0.245240774570466	-0.245240774570466\\
64.625	0.17718	-0.238175476079319	-0.238175476079319\\
64.625	0.18084	-0.230054027140282	-0.230054027140282\\
64.625	0.1845	-0.220876427753362	-0.220876427753362\\
64.625	0.18816	-0.210642677918552	-0.210642677918552\\
64.625	0.19182	-0.199352777635856	-0.199352777635856\\
64.625	0.19548	-0.187006726905269	-0.187006726905269\\
64.625	0.19914	-0.173604525726798	-0.173604525726798\\
64.625	0.2028	-0.159146174100438	-0.159146174100438\\
64.625	0.20646	-0.143631672026186	-0.143631672026186\\
64.625	0.21012	-0.127061019504053	-0.127061019504053\\
64.625	0.21378	-0.109434216534028	-0.109434216534028\\
64.625	0.21744	-0.0907512631161191	-0.0907512631161191\\
64.625	0.2211	-0.071012159250319	-0.071012159250319\\
64.625	0.22476	-0.0502169049366348	-0.0502169049366348\\
64.625	0.22842	-0.0283655001750596	-0.0283655001750596\\
64.625	0.23208	-0.00545794496560026	-0.00545794496560026\\
64.625	0.23574	0.0185057606917476	0.0185057606917476\\
64.625	0.2394	0.0435256167969817	0.0435256167969817\\
64.625	0.24306	0.0696016233501076	0.0696016233501076\\
64.625	0.24672	0.0967337803511175	0.0967337803511175\\
64.625	0.25038	0.124922087800019	0.124922087800019\\
64.625	0.25404	0.154166545696807	0.154166545696807\\
64.625	0.2577	0.18446715404148	0.18446715404148\\
64.625	0.26136	0.215823912834044	0.215823912834044\\
64.625	0.26502	0.248236822074491	0.248236822074491\\
64.625	0.26868	0.281705881762831	0.281705881762831\\
64.625	0.27234	0.316231091899054	0.316231091899054\\
64.625	0.276	0.351812452483168	0.351812452483168\\
65	0.093	-0.121985309078576	-0.121985309078576\\
65	0.09666	-0.140076813384542	-0.140076813384542\\
65	0.10032	-0.157112167242622	-0.157112167242622\\
65	0.10398	-0.173091370652813	-0.173091370652813\\
65	0.10764	-0.188014423615118	-0.188014423615118\\
65	0.1113	-0.201881326129534	-0.201881326129534\\
65	0.11496	-0.214692078196064	-0.214692078196064\\
65	0.11862	-0.226446679814705	-0.226446679814705\\
65	0.12228	-0.23714513098546	-0.23714513098546\\
65	0.12594	-0.246787431708325	-0.246787431708325\\
65	0.1296	-0.255373581983305	-0.255373581983305\\
65	0.13326	-0.262903581810395	-0.262903581810395\\
65	0.13692	-0.269377431189599	-0.269377431189599\\
65	0.14058	-0.274795130120915	-0.274795130120915\\
65	0.14424	-0.279156678604343	-0.279156678604343\\
65	0.1479	-0.282462076639884	-0.282462076639884\\
65	0.15156	-0.284711324227538	-0.284711324227538\\
65	0.15522	-0.285904421367303	-0.285904421367303\\
65	0.15888	-0.286041368059181	-0.286041368059181\\
65	0.16254	-0.285122164303172	-0.285122164303172\\
65	0.1662	-0.283146810099274	-0.283146810099274\\
65	0.16986	-0.280115305447489	-0.280115305447489\\
65	0.17352	-0.276027650347816	-0.276027650347816\\
65	0.17718	-0.270883844800255	-0.270883844800255\\
65	0.18084	-0.26468388880481	-0.26468388880481\\
65	0.1845	-0.257427782361473	-0.257427782361473\\
65	0.18816	-0.24911552547025	-0.24911552547025\\
65	0.19182	-0.239747118131139	-0.239747118131139\\
65	0.19548	-0.229322560344143	-0.229322560344143\\
65	0.19914	-0.217841852109256	-0.217841852109256\\
65	0.2028	-0.205304993426483	-0.205304993426483\\
65	0.20646	-0.191711984295819	-0.191711984295819\\
65	0.21012	-0.17706282471727	-0.17706282471727\\
65	0.21378	-0.161357514690836	-0.161357514690836\\
65	0.21744	-0.144596054216511	-0.144596054216511\\
65	0.2211	-0.126778443294299	-0.126778443294299\\
65	0.22476	-0.107904681924198	-0.107904681924198\\
65	0.22842	-0.0879747701062144	-0.0879747701062144\\
65	0.23208	-0.0669887078403391	-0.0669887078403391\\
65	0.23574	-0.0449464951265788	-0.0449464951265788\\
65	0.2394	-0.0218481319649286	-0.0218481319649286\\
65	0.24306	0.00230638164460606	0.00230638164460606\\
65	0.24672	0.027517045702032	0.027517045702032\\
65	0.25038	0.0537838602073459	0.0537838602073459\\
65	0.25404	0.0811068251605467	0.0811068251605467\\
65	0.2577	0.109485940561636	0.109485940561636\\
65	0.26136	0.138921206410608	0.138921206410608\\
65	0.26502	0.169412622707472	0.169412622707472\\
65	0.26868	0.200960189452223	0.200960189452223\\
65	0.27234	0.233563906644862	0.233563906644862\\
65	0.276	0.267223774285386	0.267223774285386\\
65.375	0.093	-0.109721966997744	-0.109721966997744\\
65.375	0.09666	-0.129734964247297	-0.129734964247297\\
65.375	0.10032	-0.148691811048964	-0.148691811048964\\
65.375	0.10398	-0.166592507402742	-0.166592507402742\\
65.375	0.10764	-0.183437053308633	-0.183437053308633\\
65.375	0.1113	-0.199225448766637	-0.199225448766637\\
65.375	0.11496	-0.213957693776752	-0.213957693776752\\
65.375	0.11862	-0.22763378833898	-0.22763378833898\\
65.375	0.12228	-0.240253732453321	-0.240253732453321\\
65.375	0.12594	-0.251817526119774	-0.251817526119774\\
65.375	0.1296	-0.262325169338339	-0.262325169338339\\
65.375	0.13326	-0.271776662109017	-0.271776662109017\\
65.375	0.13692	-0.280172004431807	-0.280172004431807\\
65.375	0.14058	-0.287511196306711	-0.287511196306711\\
65.375	0.14424	-0.293794237733726	-0.293794237733726\\
65.375	0.1479	-0.299021128712853	-0.299021128712853\\
65.375	0.15156	-0.303191869244093	-0.303191869244093\\
65.375	0.15522	-0.306306459327445	-0.306306459327445\\
65.375	0.15888	-0.308364898962911	-0.308364898962911\\
65.375	0.16254	-0.309367188150488	-0.309367188150488\\
65.375	0.1662	-0.309313326890177	-0.309313326890177\\
65.375	0.16986	-0.308203315181979	-0.308203315181979\\
65.375	0.17352	-0.306037153025893	-0.306037153025893\\
65.375	0.17718	-0.302814840421919	-0.302814840421919\\
65.375	0.18084	-0.298536377370058	-0.298536377370058\\
65.375	0.1845	-0.293201763870309	-0.293201763870309\\
65.375	0.18816	-0.286810999922674	-0.286810999922674\\
65.375	0.19182	-0.27936408552715	-0.27936408552715\\
65.375	0.19548	-0.270861020683738	-0.270861020683738\\
65.375	0.19914	-0.261301805392439	-0.261301805392439\\
65.375	0.2028	-0.250686439653253	-0.250686439653253\\
65.375	0.20646	-0.239014923466177	-0.239014923466177\\
65.375	0.21012	-0.226287256831215	-0.226287256831215\\
65.375	0.21378	-0.212503439748365	-0.212503439748365\\
65.375	0.21744	-0.197663472217628	-0.197663472217628\\
65.375	0.2211	-0.181767354239003	-0.181767354239003\\
65.375	0.22476	-0.16481508581249	-0.16481508581249\\
65.375	0.22842	-0.14680666693809	-0.14680666693809\\
65.375	0.23208	-0.127742097615803	-0.127742097615803\\
65.375	0.23574	-0.10762137784563	-0.10762137784563\\
65.375	0.2394	-0.0864445076275673	-0.0864445076275673\\
65.375	0.24306	-0.0642114869616166	-0.0642114869616166\\
65.375	0.24672	-0.0409223158477783	-0.0409223158477783\\
65.375	0.25038	-0.0165769942860519	-0.0165769942860519\\
65.375	0.25404	0.00882447772356132	0.00882447772356132\\
65.375	0.2577	0.0352821001810626	0.0352821001810626\\
65.375	0.26136	0.0627958730864511	0.0627958730864511\\
65.375	0.26502	0.0913657964397272	0.0913657964397272\\
65.375	0.26868	0.120991870240891	0.120991870240891\\
65.375	0.27234	0.151674094489943	0.151674094489943\\
65.375	0.276	0.183412469186882	0.183412469186882\\
65.75	0.093	-0.0966812518176328	-0.0966812518176328\\
65.75	0.09666	-0.118615742010774	-0.118615742010774\\
65.75	0.10032	-0.139494081756028	-0.139494081756028\\
65.75	0.10398	-0.159316271053394	-0.159316271053394\\
65.75	0.10764	-0.178082309902873	-0.178082309902873\\
65.75	0.1113	-0.195792198304462	-0.195792198304462\\
65.75	0.11496	-0.212445936258165	-0.212445936258165\\
65.75	0.11862	-0.228043523763979	-0.228043523763979\\
65.75	0.12228	-0.242584960821907	-0.242584960821907\\
65.75	0.12594	-0.256070247431946	-0.256070247431946\\
65.75	0.1296	-0.268499383594099	-0.268499383594099\\
65.75	0.13326	-0.279872369308364	-0.279872369308364\\
65.75	0.13692	-0.29018920457474	-0.29018920457474\\
65.75	0.14058	-0.29944988939323	-0.29944988939323\\
65.75	0.14424	-0.307654423763832	-0.307654423763832\\
65.75	0.1479	-0.314802807686547	-0.314802807686547\\
65.75	0.15156	-0.320895041161372	-0.320895041161372\\
65.75	0.15522	-0.325931124188312	-0.325931124188312\\
65.75	0.15888	-0.329911056767364	-0.329911056767364\\
65.75	0.16254	-0.332834838898528	-0.332834838898528\\
65.75	0.1662	-0.334702470581803	-0.334702470581803\\
65.75	0.16986	-0.335513951817192	-0.335513951817192\\
65.75	0.17352	-0.335269282604692	-0.335269282604692\\
65.75	0.17718	-0.333968462944308	-0.333968462944308\\
65.75	0.18084	-0.331611492836031	-0.331611492836031\\
65.75	0.1845	-0.328198372279869	-0.328198372279869\\
65.75	0.18816	-0.323729101275822	-0.323729101275822\\
65.75	0.19182	-0.318203679823882	-0.318203679823882\\
65.75	0.19548	-0.311622107924058	-0.311622107924058\\
65.75	0.19914	-0.303984385576346	-0.303984385576346\\
65.75	0.2028	-0.295290512780748	-0.295290512780748\\
65.75	0.20646	-0.285540489537256	-0.285540489537256\\
65.75	0.21012	-0.274734315845881	-0.274734315845881\\
65.75	0.21378	-0.262871991706619	-0.262871991706619\\
65.75	0.21744	-0.24995351711947	-0.24995351711947\\
65.75	0.2211	-0.235978892084432	-0.235978892084432\\
65.75	0.22476	-0.220948116601503	-0.220948116601503\\
65.75	0.22842	-0.204861190670691	-0.204861190670691\\
65.75	0.23208	-0.187718114291991	-0.187718114291991\\
65.75	0.23574	-0.169518887465406	-0.169518887465406\\
65.75	0.2394	-0.150263510190927	-0.150263510190927\\
65.75	0.24306	-0.129951982468564	-0.129951982468564\\
65.75	0.24672	-0.108584304298313	-0.108584304298313\\
65.75	0.25038	-0.0861604756801744	-0.0861604756801744\\
65.75	0.25404	-0.0626804966141452	-0.0626804966141452\\
65.75	0.2577	-0.0381443671002315	-0.0381443671002315\\
65.75	0.26136	-0.0125520871384306	-0.0125520871384306\\
65.75	0.26502	0.014096343271258	0.014096343271258\\
65.75	0.26868	0.0418009241288346	0.0418009241288346\\
65.75	0.27234	0.070561655434302	0.070561655434302\\
65.75	0.276	0.100378537187654	0.100378537187654\\
66.125	0.093	-0.0828631635382502	-0.0828631635382502\\
66.125	0.09666	-0.106719146674979	-0.106719146674979\\
66.125	0.10032	-0.129518979363819	-0.129518979363819\\
66.125	0.10398	-0.151262661604771	-0.151262661604771\\
66.125	0.10764	-0.171950193397837	-0.171950193397837\\
66.125	0.1113	-0.191581574743012	-0.191581574743012\\
66.125	0.11496	-0.210156805640302	-0.210156805640302\\
66.125	0.11862	-0.227675886089704	-0.227675886089704\\
66.125	0.12228	-0.244138816091218	-0.244138816091218\\
66.125	0.12594	-0.259545595644844	-0.259545595644844\\
66.125	0.1296	-0.273896224750583	-0.273896224750583\\
66.125	0.13326	-0.287190703408434	-0.287190703408434\\
66.125	0.13692	-0.299429031618399	-0.299429031618399\\
66.125	0.14058	-0.310611209380475	-0.310611209380475\\
66.125	0.14424	-0.320737236694663	-0.320737236694663\\
66.125	0.1479	-0.329807113560964	-0.329807113560964\\
66.125	0.15156	-0.337820839979379	-0.337820839979379\\
66.125	0.15522	-0.344778415949904	-0.344778415949904\\
66.125	0.15888	-0.350679841472541	-0.350679841472541\\
66.125	0.16254	-0.355525116547293	-0.355525116547293\\
66.125	0.1662	-0.359314241174156	-0.359314241174156\\
66.125	0.16986	-0.362047215353131	-0.362047215353131\\
66.125	0.17352	-0.363724039084216	-0.363724039084216\\
66.125	0.17718	-0.364344712367418	-0.364344712367418\\
66.125	0.18084	-0.363909235202732	-0.363909235202732\\
66.125	0.1845	-0.362417607590155	-0.362417607590155\\
66.125	0.18816	-0.359869829529695	-0.359869829529695\\
66.125	0.19182	-0.356265901021342	-0.356265901021342\\
66.125	0.19548	-0.351605822065106	-0.351605822065106\\
66.125	0.19914	-0.345889592660978	-0.345889592660978\\
66.125	0.2028	-0.339117212808968	-0.339117212808968\\
66.125	0.20646	-0.331288682509063	-0.331288682509063\\
66.125	0.21012	-0.322404001761276	-0.322404001761276\\
66.125	0.21378	-0.312463170565598	-0.312463170565598\\
66.125	0.21744	-0.301466188922036	-0.301466188922036\\
66.125	0.2211	-0.289413056830582	-0.289413056830582\\
66.125	0.22476	-0.276303774291245	-0.276303774291245\\
66.125	0.22842	-0.26213834130402	-0.26213834130402\\
66.125	0.23208	-0.246916757868904	-0.246916757868904\\
66.125	0.23574	-0.230639023985903	-0.230639023985903\\
66.125	0.2394	-0.213305139655015	-0.213305139655015\\
66.125	0.24306	-0.19491510487624	-0.19491510487624\\
66.125	0.24672	-0.175468919649573	-0.175468919649573\\
66.125	0.25038	-0.154966583975022	-0.154966583975022\\
66.125	0.25404	-0.13340809785258	-0.13340809785258\\
66.125	0.2577	-0.110793461282254	-0.110793461282254\\
66.125	0.26136	-0.087122674264037	-0.087122674264037\\
66.125	0.26502	-0.062395736797936	-0.062395736797936\\
66.125	0.26868	-0.0366126488839433	-0.0366126488839433\\
66.125	0.27234	-0.00977341052206704	-0.00977341052206704\\
66.125	0.276	0.0181219782876969	0.0181219782876969\\
66.5	0.093	-0.0682677021595942	-0.0682677021595942\\
66.5	0.09666	-0.0940451782399107	-0.0940451782399107\\
66.5	0.10032	-0.118766503872336	-0.118766503872336\\
66.5	0.10398	-0.142431679056875	-0.142431679056875\\
66.5	0.10764	-0.165040703793527	-0.165040703793527\\
66.5	0.1113	-0.18659357808229	-0.18659357808229\\
66.5	0.11496	-0.207090301923166	-0.207090301923166\\
66.5	0.11862	-0.226530875316154	-0.226530875316154\\
66.5	0.12228	-0.244915298261255	-0.244915298261255\\
66.5	0.12594	-0.262243570758469	-0.262243570758469\\
66.5	0.1296	-0.278515692807794	-0.278515692807794\\
66.5	0.13326	-0.293731664409232	-0.293731664409232\\
66.5	0.13692	-0.307891485562783	-0.307891485562783\\
66.5	0.14058	-0.320995156268446	-0.320995156268446\\
66.5	0.14424	-0.333042676526222	-0.333042676526222\\
66.5	0.1479	-0.344034046336109	-0.344034046336109\\
66.5	0.15156	-0.353969265698109	-0.353969265698109\\
66.5	0.15522	-0.362848334612222	-0.362848334612222\\
66.5	0.15888	-0.370671253078447	-0.370671253078447\\
66.5	0.16254	-0.377438021096783	-0.377438021096783\\
66.5	0.1662	-0.383148638667233	-0.383148638667233\\
66.5	0.16986	-0.387803105789796	-0.387803105789796\\
66.5	0.17352	-0.391401422464469	-0.391401422464469\\
66.5	0.17718	-0.393943588691258	-0.393943588691258\\
66.5	0.18084	-0.395429604470156	-0.395429604470156\\
66.5	0.1845	-0.395859469801166	-0.395859469801166\\
66.5	0.18816	-0.39523318468429	-0.39523318468429\\
66.5	0.19182	-0.393550749119529	-0.393550749119529\\
66.5	0.19548	-0.390812163106877	-0.390812163106877\\
66.5	0.19914	-0.387017426646336	-0.387017426646336\\
66.5	0.2028	-0.38216653973791	-0.38216653973791\\
66.5	0.20646	-0.376259502381596	-0.376259502381596\\
66.5	0.21012	-0.369296314577394	-0.369296314577394\\
66.5	0.21378	-0.361276976325307	-0.361276976325307\\
66.5	0.21744	-0.352201487625329	-0.352201487625329\\
66.5	0.2211	-0.342069848477463	-0.342069848477463\\
66.5	0.22476	-0.330882058881713	-0.330882058881713\\
66.5	0.22842	-0.318638118838072	-0.318638118838072\\
66.5	0.23208	-0.305338028346543	-0.305338028346543\\
66.5	0.23574	-0.29098178740713	-0.29098178740713\\
66.5	0.2394	-0.27556939601983	-0.27556939601983\\
66.5	0.24306	-0.259100854184638	-0.259100854184638\\
66.5	0.24672	-0.241576161901559	-0.241576161901559\\
66.5	0.25038	-0.222995319170592	-0.222995319170592\\
66.5	0.25404	-0.203358325991741	-0.203358325991741\\
66.5	0.2577	-0.182665182364999	-0.182665182364999\\
66.5	0.26136	-0.16091588829037	-0.16091588829037\\
66.5	0.26502	-0.138110443767856	-0.138110443767856\\
66.5	0.26868	-0.114248848797451	-0.114248848797451\\
66.5	0.27234	-0.0893311033791591	-0.0893311033791591\\
66.5	0.276	-0.0633572075129827	-0.0633572075129827\\
66.875	0.093	-0.052894867681661	-0.052894867681661\\
66.875	0.09666	-0.0805938367055615	-0.0805938367055615\\
66.875	0.10032	-0.107236655281576	-0.107236655281576\\
66.875	0.10398	-0.132823323409701	-0.132823323409701\\
66.875	0.10764	-0.157353841089939	-0.157353841089939\\
66.875	0.1113	-0.180828208322289	-0.180828208322289\\
66.875	0.11496	-0.203246425106753	-0.203246425106753\\
66.875	0.11862	-0.224608491443328	-0.224608491443328\\
66.875	0.12228	-0.244914407332016	-0.244914407332016\\
66.875	0.12594	-0.264164172772815	-0.264164172772815\\
66.875	0.1296	-0.282357787765729	-0.282357787765729\\
66.875	0.13326	-0.299495252310753	-0.299495252310753\\
66.875	0.13692	-0.31557656640789	-0.31557656640789\\
66.875	0.14058	-0.330601730057141	-0.330601730057141\\
66.875	0.14424	-0.344570743258503	-0.344570743258503\\
66.875	0.1479	-0.357483606011977	-0.357483606011977\\
66.875	0.15156	-0.369340318317563	-0.369340318317563\\
66.875	0.15522	-0.380140880175263	-0.380140880175263\\
66.875	0.15888	-0.389885291585074	-0.389885291585074\\
66.875	0.16254	-0.398573552546998	-0.398573552546998\\
66.875	0.1662	-0.406205663061036	-0.406205663061036\\
66.875	0.16986	-0.412781623127184	-0.412781623127184\\
66.875	0.17352	-0.418301432745445	-0.418301432745445\\
66.875	0.17718	-0.422765091915818	-0.422765091915818\\
66.875	0.18084	-0.426172600638303	-0.426172600638303\\
66.875	0.1845	-0.428523958912901	-0.428523958912901\\
66.875	0.18816	-0.429819166739613	-0.429819166739613\\
66.875	0.19182	-0.430058224118435	-0.430058224118435\\
66.875	0.19548	-0.42924113104937	-0.42924113104937\\
66.875	0.19914	-0.427367887532421	-0.427367887532421\\
66.875	0.2028	-0.424438493567579	-0.424438493567579\\
66.875	0.20646	-0.420452949154853	-0.420452949154853\\
66.875	0.21012	-0.415411254294238	-0.415411254294238\\
66.875	0.21378	-0.409313408985735	-0.409313408985735\\
66.875	0.21744	-0.402159413229344	-0.402159413229344\\
66.875	0.2211	-0.393949267025066	-0.393949267025066\\
66.875	0.22476	-0.3846829703729	-0.3846829703729\\
66.875	0.22842	-0.374360523272847	-0.374360523272847\\
66.875	0.23208	-0.362981925724906	-0.362981925724906\\
66.875	0.23574	-0.35054717772908	-0.35054717772908\\
66.875	0.2394	-0.337056279285364	-0.337056279285364\\
66.875	0.24306	-0.32250923039376	-0.32250923039376\\
66.875	0.24672	-0.306906031054268	-0.306906031054268\\
66.875	0.25038	-0.290246681266888	-0.290246681266888\\
66.875	0.25404	-0.272531181031622	-0.272531181031622\\
66.875	0.2577	-0.253759530348471	-0.253759530348471\\
66.875	0.26136	-0.233931729217429	-0.233931729217429\\
66.875	0.26502	-0.2130477776385	-0.2130477776385\\
66.875	0.26868	-0.191107675611682	-0.191107675611682\\
66.875	0.27234	-0.168111423136978	-0.168111423136978\\
66.875	0.276	-0.144059020214385	-0.144059020214385\\
67.25	0.093	-0.0367446601044544	-0.0367446601044544\\
67.25	0.09666	-0.0663651220719425	-0.0663651220719425\\
67.25	0.10032	-0.0949294335915428	-0.0949294335915428\\
67.25	0.10398	-0.122437594663256	-0.122437594663256\\
67.25	0.10764	-0.148889605287079	-0.148889605287079\\
67.25	0.1113	-0.174285465463017	-0.174285465463017\\
67.25	0.11496	-0.198625175191067	-0.198625175191067\\
67.25	0.11862	-0.221908734471227	-0.221908734471227\\
67.25	0.12228	-0.244136143303502	-0.244136143303502\\
67.25	0.12594	-0.265307401687889	-0.265307401687889\\
67.25	0.1296	-0.285422509624389	-0.285422509624389\\
67.25	0.13326	-0.304481467112999	-0.304481467112999\\
67.25	0.13692	-0.322484274153724	-0.322484274153724\\
67.25	0.14058	-0.33943093074656	-0.33943093074656\\
67.25	0.14424	-0.355321436891509	-0.355321436891509\\
67.25	0.1479	-0.370155792588571	-0.370155792588571\\
67.25	0.15156	-0.383933997837745	-0.383933997837745\\
67.25	0.15522	-0.396656052639029	-0.396656052639029\\
67.25	0.15888	-0.40832195699243	-0.40832195699243\\
67.25	0.16254	-0.418931710897939	-0.418931710897939\\
67.25	0.1662	-0.428485314355562	-0.428485314355562\\
67.25	0.16986	-0.436982767365298	-0.436982767365298\\
67.25	0.17352	-0.444424069927147	-0.444424069927147\\
67.25	0.17718	-0.450809222041107	-0.450809222041107\\
67.25	0.18084	-0.45613822370718	-0.45613822370718\\
67.25	0.1845	-0.460411074925362	-0.460411074925362\\
67.25	0.18816	-0.463627775695661	-0.463627775695661\\
67.25	0.19182	-0.465788326018072	-0.465788326018072\\
67.25	0.19548	-0.466892725892594	-0.466892725892594\\
67.25	0.19914	-0.466940975319229	-0.466940975319229\\
67.25	0.2028	-0.465933074297975	-0.465933074297975\\
67.25	0.20646	-0.463869022828832	-0.463869022828832\\
67.25	0.21012	-0.460748820911805	-0.460748820911805\\
67.25	0.21378	-0.456572468546889	-0.456572468546889\\
67.25	0.21744	-0.451339965734086	-0.451339965734086\\
67.25	0.2211	-0.445051312473396	-0.445051312473396\\
67.25	0.22476	-0.437706508764817	-0.437706508764817\\
67.25	0.22842	-0.429305554608348	-0.429305554608348\\
67.25	0.23208	-0.419848450003995	-0.419848450003995\\
67.25	0.23574	-0.409335194951756	-0.409335194951756\\
67.25	0.2394	-0.397765789451628	-0.397765789451628\\
67.25	0.24306	-0.385140233503611	-0.385140233503611\\
67.25	0.24672	-0.371458527107707	-0.371458527107707\\
67.25	0.25038	-0.356720670263912	-0.356720670263912\\
67.25	0.25404	-0.340926662972233	-0.340926662972233\\
67.25	0.2577	-0.324076505232666	-0.324076505232666\\
67.25	0.26136	-0.306170197045212	-0.306170197045212\\
67.25	0.26502	-0.28720773840987	-0.28720773840987\\
67.25	0.26868	-0.26718912932664	-0.26718912932664\\
67.25	0.27234	-0.246114369795523	-0.246114369795523\\
67.25	0.276	-0.223983459816514	-0.223983459816514\\
67.625	0.093	-0.0198170794279725	-0.0198170794279725\\
67.625	0.09666	-0.0513590343390481	-0.0513590343390481\\
67.625	0.10032	-0.081844838802236	-0.081844838802236\\
67.625	0.10398	-0.111274492817535	-0.111274492817535\\
67.625	0.10764	-0.139647996384946	-0.139647996384946\\
67.625	0.1113	-0.166965349504469	-0.166965349504469\\
67.625	0.11496	-0.193226552176107	-0.193226552176107\\
67.625	0.11862	-0.218431604399855	-0.218431604399855\\
67.625	0.12228	-0.242580506175716	-0.242580506175716\\
67.625	0.12594	-0.26567325750369	-0.26567325750369\\
67.625	0.1296	-0.287709858383776	-0.287709858383776\\
67.625	0.13326	-0.308690308815974	-0.308690308815974\\
67.625	0.13692	-0.328614608800284	-0.328614608800284\\
67.625	0.14058	-0.347482758336708	-0.347482758336708\\
67.625	0.14424	-0.365294757425245	-0.365294757425245\\
67.625	0.1479	-0.382050606065892	-0.382050606065892\\
67.625	0.15156	-0.397750304258652	-0.397750304258652\\
67.625	0.15522	-0.412393852003526	-0.412393852003526\\
67.625	0.15888	-0.42598124930051	-0.42598124930051\\
67.625	0.16254	-0.438512496149608	-0.438512496149608\\
67.625	0.1662	-0.449987592550818	-0.449987592550818\\
67.625	0.16986	-0.460406538504141	-0.460406538504141\\
67.625	0.17352	-0.469769334009573	-0.469769334009573\\
67.625	0.17718	-0.478075979067122	-0.478075979067122\\
67.625	0.18084	-0.485326473676782	-0.485326473676782\\
67.625	0.1845	-0.491520817838552	-0.491520817838552\\
67.625	0.18816	-0.496659011552439	-0.496659011552439\\
67.625	0.19182	-0.500741054818433	-0.500741054818433\\
67.625	0.19548	-0.503766947636543	-0.503766947636543\\
67.625	0.19914	-0.505736690006765	-0.505736690006765\\
67.625	0.2028	-0.506650281929098	-0.506650281929098\\
67.625	0.20646	-0.506507723403544	-0.506507723403544\\
67.625	0.21012	-0.5053090144301	-0.5053090144301\\
67.625	0.21378	-0.503054155008772	-0.503054155008772\\
67.625	0.21744	-0.499743145139557	-0.499743145139557\\
67.625	0.2211	-0.49537598482245	-0.49537598482245\\
67.625	0.22476	-0.489952674057459	-0.489952674057459\\
67.625	0.22842	-0.483473212844581	-0.483473212844581\\
67.625	0.23208	-0.475937601183812	-0.475937601183812\\
67.625	0.23574	-0.467345839075161	-0.467345839075161\\
67.625	0.2394	-0.457697926518617	-0.457697926518617\\
67.625	0.24306	-0.446993863514188	-0.446993863514188\\
67.625	0.24672	-0.435233650061871	-0.435233650061871\\
67.625	0.25038	-0.422417286161663	-0.422417286161663\\
67.625	0.25404	-0.408544771813572	-0.408544771813572\\
67.625	0.2577	-0.393616107017589	-0.393616107017589\\
67.625	0.26136	-0.377631291773722	-0.377631291773722\\
67.625	0.26502	-0.360590326081968	-0.360590326081968\\
67.625	0.26868	-0.342493209942322	-0.342493209942322\\
67.625	0.27234	-0.323339943354792	-0.323339943354792\\
67.625	0.276	-0.303130526319375	-0.303130526319375\\
68	0.093	-0.00211212565221702	-0.00211212565221702\\
68	0.09666	-0.0355755735068802	-0.0355755735068802\\
68	0.10032	-0.0679828709136522	-0.0679828709136522\\
68	0.10398	-0.0993340178725385	-0.0993340178725385\\
68	0.10764	-0.129629014383537	-0.129629014383537\\
68	0.1113	-0.158867860446648	-0.158867860446648\\
68	0.11496	-0.187050556061871	-0.187050556061871\\
68	0.11862	-0.214177101229205	-0.214177101229205\\
68	0.12228	-0.240247495948654	-0.240247495948654\\
68	0.12594	-0.265261740220214	-0.265261740220214\\
68	0.1296	-0.289219834043887	-0.289219834043887\\
68	0.13326	-0.312121777419673	-0.312121777419673\\
68	0.13692	-0.33396757034757	-0.33396757034757\\
68	0.14058	-0.35475721282758	-0.35475721282758\\
68	0.14424	-0.374490704859703	-0.374490704859703\\
68	0.1479	-0.393168046443938	-0.393168046443938\\
68	0.15156	-0.410789237580285	-0.410789237580285\\
68	0.15522	-0.427354278268743	-0.427354278268743\\
68	0.15888	-0.442863168509316	-0.442863168509316\\
68	0.16254	-0.457315908302001	-0.457315908302001\\
68	0.1662	-0.470712497646796	-0.470712497646796\\
68	0.16986	-0.483052936543707	-0.483052936543707\\
68	0.17352	-0.494337224992727	-0.494337224992727\\
68	0.17718	-0.504565362993862	-0.504565362993862\\
68	0.18084	-0.513737350547107	-0.513737350547107\\
68	0.1845	-0.521853187652464	-0.521853187652464\\
68	0.18816	-0.528912874309938	-0.528912874309938\\
68	0.19182	-0.53491641051952	-0.53491641051952\\
68	0.19548	-0.539863796281215	-0.539863796281215\\
68	0.19914	-0.543755031595025	-0.543755031595025\\
68	0.2028	-0.546590116460945	-0.546590116460945\\
68	0.20646	-0.548369050878974	-0.548369050878974\\
68	0.21012	-0.549091834849122	-0.549091834849122\\
68	0.21378	-0.548758468371378	-0.548758468371378\\
68	0.21744	-0.54736895144575	-0.54736895144575\\
68	0.2211	-0.544923284072231	-0.544923284072231\\
68	0.22476	-0.541421466250828	-0.541421466250828\\
68	0.22842	-0.536863497981534	-0.536863497981534\\
68	0.23208	-0.531249379264352	-0.531249379264352\\
68	0.23574	-0.524579110099288	-0.524579110099288\\
68	0.2394	-0.516852690486332	-0.516852690486332\\
68	0.24306	-0.508070120425487	-0.508070120425487\\
68	0.24672	-0.498231399916758	-0.498231399916758\\
68	0.25038	-0.487336528960137	-0.487336528960137\\
68	0.25404	-0.47538550755563	-0.47538550755563\\
68	0.2577	-0.462378335703238	-0.462378335703238\\
68	0.26136	-0.448315013402956	-0.448315013402956\\
68	0.26502	-0.433195540654789	-0.433195540654789\\
68	0.26868	-0.417019917458729	-0.417019917458729\\
68	0.27234	-0.399788143814787	-0.399788143814787\\
68	0.276	-0.381500219722957	-0.381500219722957\\
68.375	0.093	0.0163702012228119	0.0163702012228119\\
68.375	0.09666	-0.0190147395754353	-0.0190147395754353\\
68.375	0.10032	-0.0533435299257948	-0.0533435299257948\\
68.375	0.10398	-0.0866161698282687	-0.0866161698282687\\
68.375	0.10764	-0.118832659282853	-0.118832659282853\\
68.375	0.1113	-0.14999299828955	-0.14999299828955\\
68.375	0.11496	-0.180097186848361	-0.180097186848361\\
68.375	0.11862	-0.209145224959282	-0.209145224959282\\
68.375	0.12228	-0.237137112622316	-0.237137112622316\\
68.375	0.12594	-0.264072849837464	-0.264072849837464\\
68.375	0.1296	-0.289952436604723	-0.289952436604723\\
68.375	0.13326	-0.314775872924094	-0.314775872924094\\
68.375	0.13692	-0.33854315879558	-0.33854315879558\\
68.375	0.14058	-0.361254294219177	-0.361254294219177\\
68.375	0.14424	-0.382909279194885	-0.382909279194885\\
68.375	0.1479	-0.403508113722708	-0.403508113722708\\
68.375	0.15156	-0.423050797802641	-0.423050797802641\\
68.375	0.15522	-0.441537331434686	-0.441537331434686\\
68.375	0.15888	-0.458967714618846	-0.458967714618846\\
68.375	0.16254	-0.475341947355118	-0.475341947355118\\
68.375	0.1662	-0.4906600296435	-0.4906600296435\\
68.375	0.16986	-0.504921961483997	-0.504921961483997\\
68.375	0.17352	-0.518127742876605	-0.518127742876605\\
68.375	0.17718	-0.530277373821324	-0.530277373821324\\
68.375	0.18084	-0.541370854318157	-0.541370854318157\\
68.375	0.1845	-0.551408184367101	-0.551408184367101\\
68.375	0.18816	-0.56038936396816	-0.56038936396816\\
68.375	0.19182	-0.568314393121333	-0.568314393121333\\
68.375	0.19548	-0.575183271826614	-0.575183271826614\\
68.375	0.19914	-0.580996000084008	-0.580996000084008\\
68.375	0.2028	-0.585752577893516	-0.585752577893516\\
68.375	0.20646	-0.589453005255133	-0.589453005255133\\
68.375	0.21012	-0.592097282168865	-0.592097282168865\\
68.375	0.21378	-0.593685408634712	-0.593685408634712\\
68.375	0.21744	-0.594217384652668	-0.594217384652668\\
68.375	0.2211	-0.593693210222737	-0.593693210222737\\
68.375	0.22476	-0.592112885344918	-0.592112885344918\\
68.375	0.22842	-0.589476410019211	-0.589476410019211\\
68.375	0.23208	-0.585783784245617	-0.585783784245617\\
68.375	0.23574	-0.581035008024137	-0.581035008024137\\
68.375	0.2394	-0.575230081354772	-0.575230081354772\\
68.375	0.24306	-0.568369004237514	-0.568369004237514\\
68.375	0.24672	-0.56045177667237	-0.56045177667237\\
68.375	0.25038	-0.551478398659337	-0.551478398659337\\
68.375	0.25404	-0.541448870198417	-0.541448870198417\\
68.375	0.2577	-0.530363191289609	-0.530363191289609\\
68.375	0.26136	-0.518221361932917	-0.518221361932917\\
68.375	0.26502	-0.505023382128335	-0.505023382128335\\
68.375	0.26868	-0.490769251875864	-0.490769251875864\\
68.375	0.27234	-0.475458971175506	-0.475458971175506\\
68.375	0.276	-0.45909254002726	-0.45909254002726\\
68.75	0.093	0.0356299011971197	0.0356299011971197\\
68.75	0.09666	-0.00167653254471156	-0.00167653254471156\\
68.75	0.10032	-0.0379268158386604	-0.0379268158386604\\
68.75	0.10398	-0.0731209486847201	-0.0731209486847201\\
68.75	0.10764	-0.107258931082892	-0.107258931082892\\
68.75	0.1113	-0.140340763033176	-0.140340763033176\\
68.75	0.11496	-0.172366444535573	-0.172366444535573\\
68.75	0.11862	-0.20333597559008	-0.20333597559008\\
68.75	0.12228	-0.233249356196702	-0.233249356196702\\
68.75	0.12594	-0.262106586355436	-0.262106586355436\\
68.75	0.1296	-0.289907666066282	-0.289907666066282\\
68.75	0.13326	-0.316652595329241	-0.316652595329241\\
68.75	0.13692	-0.342341374144312	-0.342341374144312\\
68.75	0.14058	-0.366974002511497	-0.366974002511497\\
68.75	0.14424	-0.390550480430791	-0.390550480430791\\
68.75	0.1479	-0.413070807902199	-0.413070807902199\\
68.75	0.15156	-0.43453498492572	-0.43453498492572\\
68.75	0.15522	-0.454943011501353	-0.454943011501353\\
68.75	0.15888	-0.4742948876291	-0.4742948876291\\
68.75	0.16254	-0.492590613308958	-0.492590613308958\\
68.75	0.1662	-0.509830188540926	-0.509830188540926\\
68.75	0.16986	-0.526013613325011	-0.526013613325011\\
68.75	0.17352	-0.541140887661206	-0.541140887661206\\
68.75	0.17718	-0.555212011549513	-0.555212011549513\\
68.75	0.18084	-0.568226984989933	-0.568226984989933\\
68.75	0.1845	-0.580185807982465	-0.580185807982465\\
68.75	0.18816	-0.591088480527107	-0.591088480527107\\
68.75	0.19182	-0.600935002623864	-0.600935002623864\\
68.75	0.19548	-0.609725374272734	-0.609725374272734\\
68.75	0.19914	-0.617459595473715	-0.617459595473715\\
68.75	0.2028	-0.624137666226811	-0.624137666226811\\
68.75	0.20646	-0.629759586532015	-0.629759586532015\\
68.75	0.21012	-0.634325356389334	-0.634325356389334\\
68.75	0.21378	-0.637834975798766	-0.637834975798766\\
68.75	0.21744	-0.640288444760309	-0.640288444760309\\
68.75	0.2211	-0.641685763273965	-0.641685763273965\\
68.75	0.22476	-0.642026931339734	-0.642026931339734\\
68.75	0.22842	-0.641311948957615	-0.641311948957615\\
68.75	0.23208	-0.639540816127608	-0.639540816127608\\
68.75	0.23574	-0.636713532849713	-0.636713532849713\\
68.75	0.2394	-0.632830099123931	-0.632830099123931\\
68.75	0.24306	-0.627890514950261	-0.627890514950261\\
68.75	0.24672	-0.621894780328704	-0.621894780328704\\
68.75	0.25038	-0.614842895259259	-0.614842895259259\\
68.75	0.25404	-0.606734859741926	-0.606734859741926\\
68.75	0.2577	-0.597570673776706	-0.597570673776706\\
68.75	0.26136	-0.587350337363599	-0.587350337363599\\
68.75	0.26502	-0.576073850502604	-0.576073850502604\\
68.75	0.26868	-0.56374121319372	-0.56374121319372\\
68.75	0.27234	-0.55035242543695	-0.55035242543695\\
68.75	0.276	-0.535907487232292	-0.535907487232292\\
69.125	0.093	0.0556669742707045	0.0556669742707045\\
69.125	0.09666	0.0164390475852821	0.0164390475852821\\
69.125	0.10032	-0.0217327286522525	-0.0217327286522525\\
69.125	0.10398	-0.058848354441898	-0.058848354441898\\
69.125	0.10764	-0.0949078297836575	-0.0949078297836575\\
69.125	0.1113	-0.129911154677528	-0.129911154677528\\
69.125	0.11496	-0.163858329123512	-0.163858329123512\\
69.125	0.11862	-0.196749353121607	-0.196749353121607\\
69.125	0.12228	-0.228584226671814	-0.228584226671814\\
69.125	0.12594	-0.259362949774136	-0.259362949774136\\
69.125	0.1296	-0.289085522428568	-0.289085522428568\\
69.125	0.13326	-0.317751944635114	-0.317751944635114\\
69.125	0.13692	-0.345362216393771	-0.345362216393771\\
69.125	0.14058	-0.371916337704542	-0.371916337704542\\
69.125	0.14424	-0.397414308567425	-0.397414308567425\\
69.125	0.1479	-0.421856128982419	-0.421856128982419\\
69.125	0.15156	-0.445241798949527	-0.445241798949527\\
69.125	0.15522	-0.467571318468746	-0.467571318468746\\
69.125	0.15888	-0.488844687540079	-0.488844687540079\\
69.125	0.16254	-0.509061906163524	-0.509061906163524\\
69.125	0.1662	-0.52822297433908	-0.52822297433908\\
69.125	0.16986	-0.54632789206675	-0.54632789206675\\
69.125	0.17352	-0.563376659346529	-0.563376659346529\\
69.125	0.17718	-0.579369276178424	-0.579369276178424\\
69.125	0.18084	-0.594305742562432	-0.594305742562432\\
69.125	0.1845	-0.608186058498551	-0.608186058498551\\
69.125	0.18816	-0.621010223986781	-0.621010223986781\\
69.125	0.19182	-0.632778239027126	-0.632778239027126\\
69.125	0.19548	-0.643490103619583	-0.643490103619583\\
69.125	0.19914	-0.653145817764148	-0.653145817764148\\
69.125	0.2028	-0.661745381460832	-0.661745381460832\\
69.125	0.20646	-0.669288794709624	-0.669288794709624\\
69.125	0.21012	-0.675776057510527	-0.675776057510527\\
69.125	0.21378	-0.681207169863546	-0.681207169863546\\
69.125	0.21744	-0.685582131768677	-0.685582131768677\\
69.125	0.2211	-0.688900943225917	-0.688900943225917\\
69.125	0.22476	-0.691163604235273	-0.691163604235273\\
69.125	0.22842	-0.692370114796741	-0.692370114796741\\
69.125	0.23208	-0.692520474910322	-0.692520474910322\\
69.125	0.23574	-0.691614684576014	-0.691614684576014\\
69.125	0.2394	-0.68965274379382	-0.68965274379382\\
69.125	0.24306	-0.686634652563738	-0.686634652563738\\
69.125	0.24672	-0.682560410885765	-0.682560410885765\\
69.125	0.25038	-0.677430018759907	-0.677430018759907\\
69.125	0.25404	-0.671243476186159	-0.671243476186159\\
69.125	0.2577	-0.664000783164526	-0.664000783164526\\
69.125	0.26136	-0.655701939695006	-0.655701939695006\\
69.125	0.26502	-0.646346945777599	-0.646346945777599\\
69.125	0.26868	-0.6359358014123	-0.6359358014123\\
69.125	0.27234	-0.624468506599117	-0.624468506599117\\
69.125	0.276	-0.611945061338046	-0.611945061338046\\
69.5	0.093	0.076481420443561	0.076481420443561\\
69.5	0.09666	0.0353320008145511	0.0353320008145511\\
69.5	0.10032	-0.00476126836656932	-0.00476126836656932\\
69.5	0.10398	-0.0437983870998023	-0.0437983870998023\\
69.5	0.10764	-0.0817793553851476	-0.0817793553851476\\
69.5	0.1113	-0.118704173222605	-0.118704173222605\\
69.5	0.11496	-0.154572840612175	-0.154572840612175\\
69.5	0.11862	-0.189385357553858	-0.189385357553858\\
69.5	0.12228	-0.223141724047653	-0.223141724047653\\
69.5	0.12594	-0.25584194009356	-0.25584194009356\\
69.5	0.1296	-0.28748600569158	-0.28748600569158\\
69.5	0.13326	-0.318073920841712	-0.318073920841712\\
69.5	0.13692	-0.347605685543956	-0.347605685543956\\
69.5	0.14058	-0.376081299798313	-0.376081299798313\\
69.5	0.14424	-0.403500763604782	-0.403500763604782\\
69.5	0.1479	-0.429864076963364	-0.429864076963364\\
69.5	0.15156	-0.455171239874058	-0.455171239874058\\
69.5	0.15522	-0.479422252336864	-0.479422252336864\\
69.5	0.15888	-0.502617114351782	-0.502617114351782\\
69.5	0.16254	-0.524755825918814	-0.524755825918814\\
69.5	0.1662	-0.545838387037957	-0.545838387037957\\
69.5	0.16986	-0.565864797709215	-0.565864797709215\\
69.5	0.17352	-0.584835057932582	-0.584835057932582\\
69.5	0.17718	-0.602749167708064	-0.602749167708064\\
69.5	0.18084	-0.619607127035656	-0.619607127035656\\
69.5	0.1845	-0.635408935915363	-0.635408935915363\\
69.5	0.18816	-0.65015459434718	-0.65015459434718\\
69.5	0.19182	-0.663844102331109	-0.663844102331109\\
69.5	0.19548	-0.676477459867153	-0.676477459867153\\
69.5	0.19914	-0.688054666955306	-0.688054666955306\\
69.5	0.2028	-0.698575723595577	-0.698575723595577\\
69.5	0.20646	-0.708040629787953	-0.708040629787953\\
69.5	0.21012	-0.716449385532447	-0.716449385532447\\
69.5	0.21378	-0.72380199082905	-0.72380199082905\\
69.5	0.21744	-0.730098445677769	-0.730098445677769\\
69.5	0.2211	-0.735338750078597	-0.735338750078597\\
69.5	0.22476	-0.73952290403154	-0.73952290403154\\
69.5	0.22842	-0.742650907536593	-0.742650907536593\\
69.5	0.23208	-0.744722760593761	-0.744722760593761\\
69.5	0.23574	-0.745738463203041	-0.745738463203041\\
69.5	0.2394	-0.745698015364431	-0.745698015364431\\
69.5	0.24306	-0.744601417077936	-0.744601417077936\\
69.5	0.24672	-0.742448668343551	-0.742448668343551\\
69.5	0.25038	-0.73923976916128	-0.73923976916128\\
69.5	0.25404	-0.73497471953112	-0.73497471953112\\
69.5	0.2577	-0.729653519453075	-0.729653519453075\\
69.5	0.26136	-0.723276168927139	-0.723276168927139\\
69.5	0.26502	-0.715842667953319	-0.715842667953319\\
69.5	0.26868	-0.707353016531607	-0.707353016531607\\
69.5	0.27234	-0.697807214662012	-0.697807214662012\\
69.5	0.276	-0.687205262344525	-0.687205262344525\\
69.875	0.093	0.0980732397156929	0.0980732397156929\\
69.875	0.09666	0.0550023271430989	0.0550023271430989\\
69.875	0.10032	0.0129875650183892	0.0129875650183892\\
69.875	0.10398	-0.0279710466584296	-0.0279710466584296\\
69.875	0.10764	-0.0678735078873625	-0.0678735078873625\\
69.875	0.1113	-0.106719818668406	-0.106719818668406\\
69.875	0.11496	-0.144509979001564	-0.144509979001564\\
69.875	0.11862	-0.181243988886832	-0.181243988886832\\
69.875	0.12228	-0.216921848324214	-0.216921848324214\\
69.875	0.12594	-0.251543557313707	-0.251543557313707\\
69.875	0.1296	-0.285109115855315	-0.285109115855315\\
69.875	0.13326	-0.317618523949033	-0.317618523949033\\
69.875	0.13692	-0.349071781594865	-0.349071781594865\\
69.875	0.14058	-0.379468888792809	-0.379468888792809\\
69.875	0.14424	-0.408809845542864	-0.408809845542864\\
69.875	0.1479	-0.437094651845033	-0.437094651845033\\
69.875	0.15156	-0.464323307699313	-0.464323307699313\\
69.875	0.15522	-0.490495813105707	-0.490495813105707\\
69.875	0.15888	-0.515612168064213	-0.515612168064213\\
69.875	0.16254	-0.53967237257483	-0.53967237257483\\
69.875	0.1662	-0.562676426637559	-0.562676426637559\\
69.875	0.16986	-0.584624330252404	-0.584624330252404\\
69.875	0.17352	-0.605516083419358	-0.605516083419358\\
69.875	0.17718	-0.625351686138425	-0.625351686138425\\
69.875	0.18084	-0.644131138409608	-0.644131138409608\\
69.875	0.1845	-0.661854440232899	-0.661854440232899\\
69.875	0.18816	-0.678521591608304	-0.678521591608304\\
69.875	0.19182	-0.69413259253582	-0.69413259253582\\
69.875	0.19548	-0.708687443015448	-0.708687443015448\\
69.875	0.19914	-0.722186143047193	-0.722186143047193\\
69.875	0.2028	-0.734628692631047	-0.734628692631047\\
69.875	0.20646	-0.746015091767011	-0.746015091767011\\
69.875	0.21012	-0.756345340455089	-0.756345340455089\\
69.875	0.21378	-0.76561943869528	-0.76561943869528\\
69.875	0.21744	-0.773837386487586	-0.773837386487586\\
69.875	0.2211	-0.780999183832001	-0.780999183832001\\
69.875	0.22476	-0.787104830728529	-0.787104830728529\\
69.875	0.22842	-0.792154327177169	-0.792154327177169\\
69.875	0.23208	-0.796147673177925	-0.796147673177925\\
69.875	0.23574	-0.799084868730792	-0.799084868730792\\
69.875	0.2394	-0.80096591383577	-0.80096591383577\\
69.875	0.24306	-0.801790808492859	-0.801790808492859\\
69.875	0.24672	-0.801559552702065	-0.801559552702065\\
69.875	0.25038	-0.800272146463378	-0.800272146463378\\
69.875	0.25404	-0.797928589776805	-0.797928589776805\\
69.875	0.2577	-0.794528882642344	-0.794528882642344\\
69.875	0.26136	-0.790073025059996	-0.790073025059996\\
69.875	0.26502	-0.784561017029763	-0.784561017029763\\
69.875	0.26868	-0.777992858551639	-0.777992858551639\\
69.875	0.27234	-0.770368549625628	-0.770368549625628\\
69.875	0.276	-0.761688090251729	-0.761688090251729\\
70.25	0.093	0.1204424320871	0.1204424320871\\
70.25	0.09666	0.0754500265709185	0.0754500265709185\\
70.25	0.10032	0.0315137715026229	0.0315137715026229\\
70.25	0.10398	-0.0113663331177835	-0.0113663331177835\\
70.25	0.10764	-0.0531902872903021	-0.0531902872903021\\
70.25	0.1113	-0.0939580910149331	-0.0939580910149331\\
70.25	0.11496	-0.133669744291677	-0.133669744291677\\
70.25	0.11862	-0.172325247120532	-0.172325247120532\\
70.25	0.12228	-0.209924599501501	-0.209924599501501\\
70.25	0.12594	-0.246467801434581	-0.246467801434581\\
70.25	0.1296	-0.281954852919774	-0.281954852919774\\
70.25	0.13326	-0.31638575395708	-0.31638575395708\\
70.25	0.13692	-0.349760504546498	-0.349760504546498\\
70.25	0.14058	-0.382079104688029	-0.382079104688029\\
70.25	0.14424	-0.413341554381672	-0.413341554381672\\
70.25	0.1479	-0.443547853627427	-0.443547853627427\\
70.25	0.15156	-0.472698002425294	-0.472698002425294\\
70.25	0.15522	-0.500792000775274	-0.500792000775274\\
70.25	0.15888	-0.527829848677366	-0.527829848677366\\
70.25	0.16254	-0.55381154613157	-0.55381154613157\\
70.25	0.1662	-0.578737093137889	-0.578737093137889\\
70.25	0.16986	-0.602606489696318	-0.602606489696318\\
70.25	0.17352	-0.62541973580686	-0.62541973580686\\
70.25	0.17718	-0.647176831469514	-0.647176831469514\\
70.25	0.18084	-0.667877776684281	-0.667877776684281\\
70.25	0.1845	-0.687522571451159	-0.687522571451159\\
70.25	0.18816	-0.706111215770152	-0.706111215770152\\
70.25	0.19182	-0.723643709641256	-0.723643709641256\\
70.25	0.19548	-0.740120053064472	-0.740120053064472\\
70.25	0.19914	-0.7555402460398	-0.7555402460398\\
70.25	0.2028	-0.769904288567244	-0.769904288567244\\
70.25	0.20646	-0.783212180646794	-0.783212180646794\\
70.25	0.21012	-0.795463922278459	-0.795463922278459\\
70.25	0.21378	-0.806659513462237	-0.806659513462237\\
70.25	0.21744	-0.816798954198128	-0.816798954198128\\
70.25	0.2211	-0.825882244486131	-0.825882244486131\\
70.25	0.22476	-0.833909384326246	-0.833909384326246\\
70.25	0.22842	-0.840880373718473	-0.840880373718473\\
70.25	0.23208	-0.846795212662813	-0.846795212662813\\
70.25	0.23574	-0.851653901159268	-0.851653901159268\\
70.25	0.2394	-0.855456439207833	-0.855456439207833\\
70.25	0.24306	-0.85820282680851	-0.85820282680851\\
70.25	0.24672	-0.8598930639613	-0.8598930639613\\
70.25	0.25038	-0.860527150666198	-0.860527150666198\\
70.25	0.25404	-0.860105086923216	-0.860105086923216\\
70.25	0.2577	-0.858626872732342	-0.858626872732342\\
70.25	0.26136	-0.856092508093581	-0.856092508093581\\
70.25	0.26502	-0.852501993006933	-0.852501993006933\\
70.25	0.26868	-0.847855327472396	-0.847855327472396\\
70.25	0.27234	-0.842152511489973	-0.842152511489973\\
70.25	0.276	-0.835393545059661	-0.835393545059661\\
70.625	0.093	0.143588997557781	0.143588997557781\\
70.625	0.09666	0.0966750990980115	0.0966750990980115\\
70.625	0.10032	0.0508173510861302	0.0508173510861302\\
70.625	0.10398	0.00601575352213624	0.00601575352213624\\
70.625	0.10764	-0.0377296935939682	-0.0377296935939682\\
70.625	0.1113	-0.0804189902621868	-0.0804189902621868\\
70.625	0.11496	-0.122052136482518	-0.122052136482518\\
70.625	0.11862	-0.162629132254959	-0.162629132254959\\
70.625	0.12228	-0.202149977579515	-0.202149977579515\\
70.625	0.12594	-0.240614672456182	-0.240614672456182\\
70.625	0.1296	-0.278023216884962	-0.278023216884962\\
70.625	0.13326	-0.314375610865855	-0.314375610865855\\
70.625	0.13692	-0.349671854398859	-0.349671854398859\\
70.625	0.14058	-0.383911947483976	-0.383911947483976\\
70.625	0.14424	-0.417095890121207	-0.417095890121207\\
70.625	0.1479	-0.449223682310549	-0.449223682310549\\
70.625	0.15156	-0.480295324052002	-0.480295324052002\\
70.625	0.15522	-0.51031081534557	-0.51031081534557\\
70.625	0.15888	-0.539270156191249	-0.539270156191249\\
70.625	0.16254	-0.567173346589041	-0.567173346589041\\
70.625	0.1662	-0.594020386538946	-0.594020386538946\\
70.625	0.16986	-0.619811276040959	-0.619811276040959\\
70.625	0.17352	-0.644546015095088	-0.644546015095088\\
70.625	0.17718	-0.66822460370133	-0.66822460370133\\
70.625	0.18084	-0.690847041859684	-0.690847041859684\\
70.625	0.1845	-0.71241332957015	-0.71241332957015\\
70.625	0.18816	-0.732923466832727	-0.732923466832727\\
70.625	0.19182	-0.752377453647418	-0.752377453647418\\
70.625	0.19548	-0.770775290014222	-0.770775290014222\\
70.625	0.19914	-0.788116975933138	-0.788116975933138\\
70.625	0.2028	-0.804402511404162	-0.804402511404162\\
70.625	0.20646	-0.819631896427303	-0.819631896427303\\
70.625	0.21012	-0.833805131002556	-0.833805131002556\\
70.625	0.21378	-0.846922215129922	-0.846922215129922\\
70.625	0.21744	-0.8589831488094	-0.8589831488094\\
70.625	0.2211	-0.869987932040986	-0.869987932040986\\
70.625	0.22476	-0.879936564824689	-0.879936564824689\\
70.625	0.22842	-0.888829047160504	-0.888829047160504\\
70.625	0.23208	-0.896665379048432	-0.896665379048432\\
70.625	0.23574	-0.903445560488471	-0.903445560488471\\
70.625	0.2394	-0.909169591480623	-0.909169591480623\\
70.625	0.24306	-0.913837472024888	-0.913837472024888\\
70.625	0.24672	-0.917449202121265	-0.917449202121265\\
70.625	0.25038	-0.92000478176975	-0.92000478176975\\
70.625	0.25404	-0.921504210970352	-0.921504210970352\\
70.625	0.2577	-0.921947489723066	-0.921947489723066\\
70.625	0.26136	-0.921334618027893	-0.921334618027893\\
70.625	0.26502	-0.919665595884832	-0.919665595884832\\
70.625	0.26868	-0.91694042329388	-0.91694042329388\\
70.625	0.27234	-0.913159100255044	-0.913159100255044\\
70.625	0.276	-0.90832162676832	-0.90832162676832\\
71	0.093	0.167512936127738	0.167512936127738\\
71	0.09666	0.118677544724382	0.118677544724382\\
71	0.10032	0.0708983037689145	0.0708983037689145\\
71	0.10398	0.0241752132613348	0.0241752132613348\\
71	0.10764	-0.0214917267983573	-0.0214917267983573\\
71	0.1113	-0.0661025164101616	-0.0661025164101616\\
71	0.11496	-0.10965715557408	-0.10965715557408\\
71	0.11862	-0.152155644290109	-0.152155644290109\\
71	0.12228	-0.193597982558251	-0.193597982558251\\
71	0.12594	-0.233984170378505	-0.233984170378505\\
71	0.1296	-0.273314207750871	-0.273314207750871\\
71	0.13326	-0.31158809467535	-0.31158809467535\\
71	0.13692	-0.348805831151941	-0.348805831151941\\
71	0.14058	-0.384967417180646	-0.384967417180646\\
71	0.14424	-0.420072852761462	-0.420072852761462\\
71	0.1479	-0.454122137894391	-0.454122137894391\\
71	0.15156	-0.487115272579433	-0.487115272579433\\
71	0.15522	-0.519052256816586	-0.519052256816586\\
71	0.15888	-0.549933090605851	-0.549933090605851\\
71	0.16254	-0.579757773947229	-0.579757773947229\\
71	0.1662	-0.608526306840721	-0.608526306840721\\
71	0.16986	-0.636238689286322	-0.636238689286322\\
71	0.17352	-0.662894921284039	-0.662894921284039\\
71	0.17718	-0.688495002833868	-0.688495002833868\\
71	0.18084	-0.713038933935807	-0.713038933935807\\
71	0.1845	-0.736526714589862	-0.736526714589862\\
71	0.18816	-0.758958344796025	-0.758958344796025\\
71	0.19182	-0.780333824554303	-0.780333824554303\\
71	0.19548	-0.800653153864691	-0.800653153864691\\
71	0.19914	-0.819916332727195	-0.819916332727195\\
71	0.2028	-0.83812336114181	-0.83812336114181\\
71	0.20646	-0.855274239108535	-0.855274239108535\\
71	0.21012	-0.871368966627372	-0.871368966627372\\
71	0.21378	-0.886407543698325	-0.886407543698325\\
71	0.21744	-0.900389970321391	-0.900389970321391\\
71	0.2211	-0.913316246496565	-0.913316246496565\\
71	0.22476	-0.925186372223855	-0.925186372223855\\
71	0.22842	-0.936000347503255	-0.936000347503255\\
71	0.23208	-0.94575817233477	-0.94575817233477\\
71	0.23574	-0.954459846718396	-0.954459846718396\\
71	0.2394	-0.962105370654136	-0.962105370654136\\
71	0.24306	-0.968694744141985	-0.968694744141985\\
71	0.24672	-0.97422796718195	-0.97422796718195\\
71	0.25038	-0.978705039774022	-0.978705039774022\\
71	0.25404	-0.982125961918212	-0.982125961918212\\
71	0.2577	-0.98449073361451	-0.98449073361451\\
71	0.26136	-0.985799354862924	-0.985799354862924\\
71	0.26502	-0.986051825663451	-0.986051825663451\\
71	0.26868	-0.985248146016086	-0.985248146016086\\
71	0.27234	-0.983388315920838	-0.983388315920838\\
71	0.276	-0.980472335377698	-0.980472335377698\\
71.375	0.093	0.192214247796969	0.192214247796969\\
71.375	0.09666	0.141457363450025	0.141457363450025\\
71.375	0.10032	0.0917566295509723	0.0917566295509723\\
71.375	0.10398	0.043112046099805	0.043112046099805\\
71.375	0.10764	-0.0044763869034728	-0.0044763869034728\\
71.375	0.1113	-0.0510086694588647	-0.0510086694588647\\
71.375	0.11496	-0.0964848015663691	-0.0964848015663691\\
71.375	0.11862	-0.140904783225984	-0.140904783225984\\
71.375	0.12228	-0.184268614437713	-0.184268614437713\\
71.375	0.12594	-0.226576295201555	-0.226576295201555\\
71.375	0.1296	-0.267827825517507	-0.267827825517507\\
71.375	0.13326	-0.308023205385573	-0.308023205385573\\
71.375	0.13692	-0.347162434805752	-0.347162434805752\\
71.375	0.14058	-0.385245513778043	-0.385245513778043\\
71.375	0.14424	-0.422272442302446	-0.422272442302446\\
71.375	0.1479	-0.458243220378961	-0.458243220378961\\
71.375	0.15156	-0.493157848007589	-0.493157848007589\\
71.375	0.15522	-0.52701632518833	-0.52701632518833\\
71.375	0.15888	-0.559818651921182	-0.559818651921182\\
71.375	0.16254	-0.591564828206148	-0.591564828206148\\
71.375	0.1662	-0.622254854043224	-0.622254854043224\\
71.375	0.16986	-0.651888729432412	-0.651888729432412\\
71.375	0.17352	-0.680466454373717	-0.680466454373717\\
71.375	0.17718	-0.707988028867128	-0.707988028867128\\
71.375	0.18084	-0.734453452912657	-0.734453452912657\\
71.375	0.1845	-0.759862726510299	-0.759862726510299\\
71.375	0.18816	-0.784215849660049	-0.784215849660049\\
71.375	0.19182	-0.807512822361912	-0.807512822361912\\
71.375	0.19548	-0.82975364461589	-0.82975364461589\\
71.375	0.19914	-0.850938316421978	-0.850938316421978\\
71.375	0.2028	-0.871066837780178	-0.871066837780178\\
71.375	0.20646	-0.890139208690493	-0.890139208690493\\
71.375	0.21012	-0.908155429152918	-0.908155429152918\\
71.375	0.21378	-0.925115499167455	-0.925115499167455\\
71.375	0.21744	-0.941019418734109	-0.941019418734109\\
71.375	0.2211	-0.95586718785287	-0.95586718785287\\
71.375	0.22476	-0.969658806523745	-0.969658806523745\\
71.375	0.22842	-0.982394274746735	-0.982394274746735\\
71.375	0.23208	-0.994073592521834	-0.994073592521834\\
71.375	0.23574	-1.00469675984905	-1.00469675984905\\
71.375	0.2394	-1.01426377672837	-1.01426377672837\\
71.375	0.24306	-1.02277464315981	-1.02277464315981\\
71.375	0.24672	-1.03022935914336	-1.03022935914336\\
71.375	0.25038	-1.03662792467902	-1.03662792467902\\
71.375	0.25404	-1.0419703397668	-1.0419703397668\\
71.375	0.2577	-1.04625660440668	-1.04625660440668\\
71.375	0.26136	-1.04948671859868	-1.04948671859868\\
71.375	0.26502	-1.0516606823428	-1.0516606823428\\
71.375	0.26868	-1.05277849563902	-1.05277849563902\\
71.375	0.27234	-1.05284015848735	-1.05284015848735\\
71.375	0.276	-1.05184567088781	-1.05184567088781\\
71.75	0.093	0.217692932565476	0.217692932565476\\
71.75	0.09666	0.165014555274948	0.165014555274948\\
71.75	0.10032	0.113392328432305	0.113392328432305\\
71.75	0.10398	0.0628262520375523	0.0628262520375523\\
71.75	0.10764	0.0133163260906869	0.0133163260906869\\
71.75	0.1113	-0.0351374494082908	-0.0351374494082908\\
71.75	0.11496	-0.0825350744593827	-0.0825350744593827\\
71.75	0.11862	-0.128876549062585	-0.128876549062585\\
71.75	0.12228	-0.1741618732179	-0.1741618732179\\
71.75	0.12594	-0.218391046925328	-0.218391046925328\\
71.75	0.1296	-0.261564070184867	-0.261564070184867\\
71.75	0.13326	-0.303680942996521	-0.303680942996521\\
71.75	0.13692	-0.344741665360286	-0.344741665360286\\
71.75	0.14058	-0.384746237276164	-0.384746237276164\\
71.75	0.14424	-0.423694658744153	-0.423694658744153\\
71.75	0.1479	-0.461586929764255	-0.461586929764255\\
71.75	0.15156	-0.498423050336469	-0.498423050336469\\
71.75	0.15522	-0.534203020460797	-0.534203020460797\\
71.75	0.15888	-0.568926840137236	-0.568926840137236\\
71.75	0.16254	-0.602594509365787	-0.602594509365787\\
71.75	0.1662	-0.635206028146451	-0.635206028146451\\
71.75	0.16986	-0.666761396479232	-0.666761396479232\\
71.75	0.17352	-0.697260614364117	-0.697260614364117\\
71.75	0.17718	-0.726703681801121	-0.726703681801121\\
71.75	0.18084	-0.755090598790231	-0.755090598790231\\
71.75	0.1845	-0.78242136533146	-0.78242136533146\\
71.75	0.18816	-0.808695981424798	-0.808695981424798\\
71.75	0.19182	-0.833914447070248	-0.833914447070248\\
71.75	0.19548	-0.858076762267811	-0.858076762267811\\
71.75	0.19914	-0.881182927017486	-0.881182927017486\\
71.75	0.2028	-0.903232941319277	-0.903232941319277\\
71.75	0.20646	-0.924226805173173	-0.924226805173173\\
71.75	0.21012	-0.944164518579185	-0.944164518579185\\
71.75	0.21378	-0.96304608153731	-0.96304608153731\\
71.75	0.21744	-0.980871494047551	-0.980871494047551\\
71.75	0.2211	-0.9976407561099	-0.9976407561099\\
71.75	0.22476	-1.01335386772436	-1.01335386772436\\
71.75	0.22842	-1.02801082889094	-1.02801082889094\\
71.75	0.23208	-1.04161163960962	-1.04161163960962\\
71.75	0.23574	-1.05415629988042	-1.05415629988042\\
71.75	0.2394	-1.06564480970334	-1.06564480970334\\
71.75	0.24306	-1.07607716907836	-1.07607716907836\\
71.75	0.24672	-1.0854533780055	-1.0854533780055\\
71.75	0.25038	-1.09377343648474	-1.09377343648474\\
71.75	0.25404	-1.10103734451611	-1.10103734451611\\
71.75	0.2577	-1.10724510209958	-1.10724510209958\\
71.75	0.26136	-1.11239670923516	-1.11239670923516\\
71.75	0.26502	-1.11649216592287	-1.11649216592287\\
71.75	0.26868	-1.11953147216268	-1.11953147216268\\
71.75	0.27234	-1.1215146279546	-1.1215146279546\\
71.75	0.276	-1.12244163329864	-1.12244163329864\\
72.125	0.093	0.243948990433258	0.243948990433258\\
72.125	0.09666	0.189349120199142	0.189349120199142\\
72.125	0.10032	0.135805400412914	0.135805400412914\\
72.125	0.10398	0.0833178310745749	0.0833178310745749\\
72.125	0.10764	0.031886412184122	0.031886412184122\\
72.125	0.1113	-0.0184888562584433	-0.0184888562584433\\
72.125	0.11496	-0.067807974253121	-0.067807974253121\\
72.125	0.11862	-0.116070941799909	-0.116070941799909\\
72.125	0.12228	-0.163277758898812	-0.163277758898812\\
72.125	0.12594	-0.209428425549827	-0.209428425549827\\
72.125	0.1296	-0.254522941752954	-0.254522941752954\\
72.125	0.13326	-0.298561307508192	-0.298561307508192\\
72.125	0.13692	-0.341543522815544	-0.341543522815544\\
72.125	0.14058	-0.383469587675008	-0.383469587675008\\
72.125	0.14424	-0.424339502086585	-0.424339502086585\\
72.125	0.1479	-0.464153266050275	-0.464153266050275\\
72.125	0.15156	-0.502910879566076	-0.502910879566076\\
72.125	0.15522	-0.540612342633988	-0.540612342633988\\
72.125	0.15888	-0.577257655254016	-0.577257655254016\\
72.125	0.16254	-0.612846817426153	-0.612846817426153\\
72.125	0.1662	-0.647379829150408	-0.647379829150408\\
72.125	0.16986	-0.680856690426768	-0.680856690426768\\
72.125	0.17352	-0.713277401255241	-0.713277401255241\\
72.125	0.17718	-0.744641961635832	-0.744641961635832\\
72.125	0.18084	-0.77495037156853	-0.77495037156853\\
72.125	0.1845	-0.804202631053346	-0.804202631053346\\
72.125	0.18816	-0.832398740090271	-0.832398740090271\\
72.125	0.19182	-0.859538698679309	-0.859538698679309\\
72.125	0.19548	-0.885622506820456	-0.885622506820456\\
72.125	0.19914	-0.910650164513719	-0.910650164513719\\
72.125	0.2028	-0.934621671759094	-0.934621671759094\\
72.125	0.20646	-0.957537028556581	-0.957537028556581\\
72.125	0.21012	-0.979396234906181	-0.979396234906181\\
72.125	0.21378	-1.00019929080789	-1.00019929080789\\
72.125	0.21744	-1.01994619626172	-1.01994619626172\\
72.125	0.2211	-1.03863695126765	-1.03863695126765\\
72.125	0.22476	-1.0562715558257	-1.0562715558257\\
72.125	0.22842	-1.07285000993586	-1.07285000993586\\
72.125	0.23208	-1.08837231359814	-1.08837231359814\\
72.125	0.23574	-1.10283846681253	-1.10283846681253\\
72.125	0.2394	-1.11624846957903	-1.11624846957903\\
72.125	0.24306	-1.12860232189763	-1.12860232189763\\
72.125	0.24672	-1.13990002376836	-1.13990002376836\\
72.125	0.25038	-1.15014157519119	-1.15014157519119\\
72.125	0.25404	-1.15932697616614	-1.15932697616614\\
72.125	0.2577	-1.1674562266932	-1.1674562266932\\
72.125	0.26136	-1.17452932677238	-1.17452932677238\\
72.125	0.26502	-1.18054627640366	-1.18054627640366\\
72.125	0.26868	-1.18550707558706	-1.18550707558706\\
72.125	0.27234	-1.18941172432257	-1.18941172432257\\
72.125	0.276	-1.19226022261019	-1.19226022261019\\
72.5	0.093	0.270982421400315	0.270982421400315\\
72.5	0.09666	0.214461058222615	0.214461058222615\\
72.5	0.10032	0.158995845492797	0.158995845492797\\
72.5	0.10398	0.104586783210871	0.104586783210871\\
72.5	0.10764	0.0512338713768323	0.0512338713768323\\
72.5	0.1113	-0.00106289000931881	-0.00106289000931881\\
72.5	0.11496	-0.0523035009475841	-0.0523035009475841\\
72.5	0.11862	-0.10248796143796	-0.10248796143796\\
72.5	0.12228	-0.151616271480448	-0.151616271480448\\
72.5	0.12594	-0.199688431075049	-0.199688431075049\\
72.5	0.1296	-0.246704440221764	-0.246704440221764\\
72.5	0.13326	-0.292664298920589	-0.292664298920589\\
72.5	0.13692	-0.337568007171527	-0.337568007171527\\
72.5	0.14058	-0.381415564974579	-0.381415564974579\\
72.5	0.14424	-0.424206972329742	-0.424206972329742\\
72.5	0.1479	-0.465942229237019	-0.465942229237019\\
72.5	0.15156	-0.506621335696406	-0.506621335696406\\
72.5	0.15522	-0.546244291707906	-0.546244291707906\\
72.5	0.15888	-0.584811097271523	-0.584811097271523\\
72.5	0.16254	-0.622321752387246	-0.622321752387246\\
72.5	0.1662	-0.658776257055081	-0.658776257055081\\
72.5	0.16986	-0.694174611275036	-0.694174611275036\\
72.5	0.17352	-0.728516815047096	-0.728516815047096\\
72.5	0.17718	-0.761802868371268	-0.761802868371268\\
72.5	0.18084	-0.794032771247553	-0.794032771247553\\
72.5	0.1845	-0.825206523675957	-0.825206523675957\\
72.5	0.18816	-0.85532412565647	-0.85532412565647\\
72.5	0.19182	-0.884385577189092	-0.884385577189092\\
72.5	0.19548	-0.91239087827383	-0.91239087827383\\
72.5	0.19914	-0.939340028910676	-0.939340028910676\\
72.5	0.2028	-0.965233029099642	-0.965233029099642\\
72.5	0.20646	-0.990069878840714	-0.990069878840714\\
72.5	0.21012	-1.0138505781339	-1.0138505781339\\
72.5	0.21378	-1.0365751269792	-1.0365751269792\\
72.5	0.21744	-1.05824352537661	-1.05824352537661\\
72.5	0.2211	-1.07885577332613	-1.07885577332613\\
72.5	0.22476	-1.09841187082777	-1.09841187082777\\
72.5	0.22842	-1.11691181788151	-1.11691181788151\\
72.5	0.23208	-1.13435561448738	-1.13435561448738\\
72.5	0.23574	-1.15074326064535	-1.15074326064535\\
72.5	0.2394	-1.16607475635544	-1.16607475635544\\
72.5	0.24306	-1.18035010161763	-1.18035010161763\\
72.5	0.24672	-1.19356929643194	-1.19356929643194\\
72.5	0.25038	-1.20573234079837	-1.20573234079837\\
72.5	0.25404	-1.2168392347169	-1.2168392347169\\
72.5	0.2577	-1.22688997818755	-1.22688997818755\\
72.5	0.26136	-1.23588457121031	-1.23588457121031\\
72.5	0.26502	-1.24382301378518	-1.24382301378518\\
72.5	0.26868	-1.25070530591216	-1.25070530591216\\
72.5	0.27234	-1.25653144759126	-1.25653144759126\\
72.5	0.276	-1.26130143882247	-1.26130143882247\\
72.875	0.093	0.298793225466647	0.298793225466647\\
72.875	0.09666	0.24035036934536	0.24035036934536\\
72.875	0.10032	0.182963663671958	0.182963663671958\\
72.875	0.10398	0.126633108446444	0.126633108446444\\
72.875	0.10764	0.0713587036688179	0.0713587036688179\\
72.875	0.1113	0.0171404493390792	0.0171404493390792\\
72.875	0.11496	-0.0360216545427718	-0.0360216545427718\\
72.875	0.11862	-0.0881276079767336	-0.0881276079767336\\
72.875	0.12228	-0.139177410962809	-0.139177410962809\\
72.875	0.12594	-0.189171063500998	-0.189171063500998\\
72.875	0.1296	-0.238108565591298	-0.238108565591298\\
72.875	0.13326	-0.285989917233711	-0.285989917233711\\
72.875	0.13692	-0.332815118428235	-0.332815118428235\\
72.875	0.14058	-0.378584169174874	-0.378584169174874\\
72.875	0.14424	-0.423297069473625	-0.423297069473625\\
72.875	0.1479	-0.466953819324489	-0.466953819324489\\
72.875	0.15156	-0.509554418727462	-0.509554418727462\\
72.875	0.15522	-0.551098867682548	-0.551098867682548\\
72.875	0.15888	-0.591587166189745	-0.591587166189745\\
72.875	0.16254	-0.631019314249063	-0.631019314249063\\
72.875	0.1662	-0.669395311860486	-0.669395311860486\\
72.875	0.16986	-0.706715159024021	-0.706715159024021\\
72.875	0.17352	-0.742978855739669	-0.742978855739669\\
72.875	0.17718	-0.778186402007429	-0.778186402007429\\
72.875	0.18084	-0.812337797827308	-0.812337797827308\\
72.875	0.1845	-0.845433043199293	-0.845433043199293\\
72.875	0.18816	-0.877472138123389	-0.877472138123389\\
72.875	0.19182	-0.908455082599603	-0.908455082599603\\
72.875	0.19548	-0.938381876627924	-0.938381876627924\\
72.875	0.19914	-0.967252520208362	-0.967252520208362\\
72.875	0.2028	-0.995067013340909	-0.995067013340909\\
72.875	0.20646	-1.02182535602557	-1.02182535602557\\
72.875	0.21012	-1.04752754826234	-1.04752754826234\\
72.875	0.21378	-1.07217359005123	-1.07217359005123\\
72.875	0.21744	-1.09576348139223	-1.09576348139223\\
72.875	0.2211	-1.11829722228533	-1.11829722228533\\
72.875	0.22476	-1.13977481273056	-1.13977481273056\\
72.875	0.22842	-1.16019625272789	-1.16019625272789\\
72.875	0.23208	-1.17956154227734	-1.17956154227734\\
72.875	0.23574	-1.1978706813789	-1.1978706813789\\
72.875	0.2394	-1.21512367003257	-1.21512367003257\\
72.875	0.24306	-1.23132050823836	-1.23132050823836\\
72.875	0.24672	-1.24646119599626	-1.24646119599626\\
72.875	0.25038	-1.26054573330626	-1.26054573330626\\
72.875	0.25404	-1.27357412016838	-1.27357412016838\\
72.875	0.2577	-1.28554635658262	-1.28554635658262\\
72.875	0.26136	-1.29646244254896	-1.29646244254896\\
72.875	0.26502	-1.30632237806742	-1.30632237806742\\
72.875	0.26868	-1.31512616313799	-1.31512616313799\\
72.875	0.27234	-1.32287379776068	-1.32287379776068\\
72.875	0.276	-1.32956528193547	-1.32956528193547\\
73.25	0.093	0.327381402632253	0.327381402632253\\
73.25	0.09666	0.267017053567378	0.267017053567378\\
73.25	0.10032	0.207708854950391	0.207708854950391\\
73.25	0.10398	0.149456806781289	0.149456806781289\\
73.25	0.10764	0.0922609090600769	0.0922609090600769\\
73.25	0.1113	0.0361211617867525	0.0361211617867525\\
73.25	0.11496	-0.0189624350386861	-0.0189624350386861\\
73.25	0.11862	-0.0729898814162354	-0.0729898814162354\\
73.25	0.12228	-0.125961177345897	-0.125961177345897\\
73.25	0.12594	-0.177876322827673	-0.177876322827673\\
73.25	0.1296	-0.228735317861559	-0.228735317861559\\
73.25	0.13326	-0.278538162447558	-0.278538162447558\\
73.25	0.13692	-0.327284856585671	-0.327284856585671\\
73.25	0.14058	-0.374975400275896	-0.374975400275896\\
73.25	0.14424	-0.421609793518234	-0.421609793518234\\
73.25	0.1479	-0.467188036312683	-0.467188036312683\\
73.25	0.15156	-0.511710128659243	-0.511710128659243\\
73.25	0.15522	-0.555176070557917	-0.555176070557917\\
73.25	0.15888	-0.597585862008702	-0.597585862008702\\
73.25	0.16254	-0.6389395030116	-0.6389395030116\\
73.25	0.1662	-0.67923699356661	-0.67923699356661\\
73.25	0.16986	-0.71847833367374	-0.71847833367374\\
73.25	0.17352	-0.756663523332975	-0.756663523332975\\
73.25	0.17718	-0.793792562544323	-0.793792562544323\\
73.25	0.18084	-0.829865451307783	-0.829865451307783\\
73.25	0.1845	-0.864882189623355	-0.864882189623355\\
73.25	0.18816	-0.898842777491039	-0.898842777491039\\
73.25	0.19182	-0.931747214910836	-0.931747214910836\\
73.25	0.19548	-0.963595501882749	-0.963595501882749\\
73.25	0.19914	-0.994387638406771	-0.994387638406771\\
73.25	0.2028	-1.02412362448291	-1.02412362448291\\
73.25	0.20646	-1.05280346011115	-1.05280346011115\\
73.25	0.21012	-1.08042714529151	-1.08042714529151\\
73.25	0.21378	-1.10699468002398	-1.10699468002398\\
73.25	0.21744	-1.13250606430857	-1.13250606430857\\
73.25	0.2211	-1.15696129814527	-1.15696129814527\\
73.25	0.22476	-1.18036038153407	-1.18036038153407\\
73.25	0.22842	-1.202703314475	-1.202703314475\\
73.25	0.23208	-1.22399009696803	-1.22399009696803\\
73.25	0.23574	-1.24422072901318	-1.24422072901318\\
73.25	0.2394	-1.26339521061044	-1.26339521061044\\
73.25	0.24306	-1.28151354175981	-1.28151354175981\\
73.25	0.24672	-1.29857572246129	-1.29857572246129\\
73.25	0.25038	-1.31458175271489	-1.31458175271489\\
73.25	0.25404	-1.32953163252059	-1.32953163252059\\
73.25	0.2577	-1.34342536187841	-1.34342536187841\\
73.25	0.26136	-1.35626294078835	-1.35626294078835\\
73.25	0.26502	-1.36804436925039	-1.36804436925039\\
73.25	0.26868	-1.37876964726455	-1.37876964726455\\
73.25	0.27234	-1.38843877483082	-1.38843877483082\\
73.25	0.276	-1.3970517519492	-1.3970517519492\\
73.625	0.093	0.356746952897134	0.356746952897134\\
73.625	0.09666	0.294461110888671	0.294461110888671\\
73.625	0.10032	0.233231419328097	0.233231419328097\\
73.625	0.10398	0.173057878215409	0.173057878215409\\
73.625	0.10764	0.11394048755061	0.11394048755061\\
73.625	0.1113	0.0558792473336975	0.0558792473336975\\
73.625	0.11496	-0.0011258424353251	-0.0011258424353251\\
73.625	0.11862	-0.057074781756462	-0.057074781756462\\
73.625	0.12228	-0.111967570629711	-0.111967570629711\\
73.625	0.12594	-0.165804209055073	-0.165804209055073\\
73.625	0.1296	-0.218584697032547	-0.218584697032547\\
73.625	0.13326	-0.270309034562133	-0.270309034562133\\
73.625	0.13692	-0.32097722164383	-0.32097722164383\\
73.625	0.14058	-0.370589258277645	-0.370589258277645\\
73.625	0.14424	-0.419145144463565	-0.419145144463565\\
73.625	0.1479	-0.466644880201601	-0.466644880201601\\
73.625	0.15156	-0.513088465491749	-0.513088465491749\\
73.625	0.15522	-0.55847590033401	-0.55847590033401\\
73.625	0.15888	-0.602807184728383	-0.602807184728383\\
73.625	0.16254	-0.646082318674868	-0.646082318674868\\
73.625	0.1662	-0.688301302173466	-0.688301302173466\\
73.625	0.16986	-0.729464135224176	-0.729464135224176\\
73.625	0.17352	-0.769570817826999	-0.769570817826999\\
73.625	0.17718	-0.808621349981934	-0.808621349981934\\
73.625	0.18084	-0.846615731688982	-0.846615731688982\\
73.625	0.1845	-0.883553962948141	-0.883553962948141\\
73.625	0.18816	-0.919436043759417	-0.919436043759417\\
73.625	0.19182	-0.954261974122802	-0.954261974122802\\
73.625	0.19548	-0.988031754038299	-0.988031754038299\\
73.625	0.19914	-1.02074538350591	-1.02074538350591\\
73.625	0.2028	-1.05240286252563	-1.05240286252563\\
73.625	0.20646	-1.08300419109746	-1.08300419109746\\
73.625	0.21012	-1.11254936922141	-1.11254936922141\\
73.625	0.21378	-1.14103839689747	-1.14103839689747\\
73.625	0.21744	-1.16847127412564	-1.16847127412564\\
73.625	0.2211	-1.19484800090592	-1.19484800090592\\
73.625	0.22476	-1.22016857723832	-1.22016857723832\\
73.625	0.22842	-1.24443300312283	-1.24443300312283\\
73.625	0.23208	-1.26764127855945	-1.26764127855945\\
73.625	0.23574	-1.28979340354818	-1.28979340354818\\
73.625	0.2394	-1.31088937808903	-1.31088937808903\\
73.625	0.24306	-1.33092920218199	-1.33092920218199\\
73.625	0.24672	-1.34991287582706	-1.34991287582706\\
73.625	0.25038	-1.36784039902424	-1.36784039902424\\
73.625	0.25404	-1.38471177177353	-1.38471177177353\\
73.625	0.2577	-1.40052699407494	-1.40052699407494\\
73.625	0.26136	-1.41528606592846	-1.41528606592846\\
73.625	0.26502	-1.42898898733409	-1.42898898733409\\
73.625	0.26868	-1.44163575829184	-1.44163575829184\\
73.625	0.27234	-1.45322637880169	-1.45322637880169\\
73.625	0.276	-1.46376084886366	-1.46376084886366\\
74	0.093	0.38688987626129	0.38688987626129\\
74	0.09666	0.32268254130924	0.32268254130924\\
74	0.10032	0.25953135680508	0.25953135680508\\
74	0.10398	0.197436322748806	0.197436322748806\\
74	0.10764	0.136397439140419	0.136397439140419\\
74	0.1113	0.0764147059799214	0.0764147059799214\\
74	0.11496	0.0174881232673112	0.0174881232673112\\
74	0.11862	-0.0403823089974132	-0.0403823089974132\\
74	0.12228	-0.0971965908142483	-0.0971965908142483\\
74	0.12594	-0.152954722183197	-0.152954722183197\\
74	0.1296	-0.207656703104257	-0.207656703104257\\
74	0.13326	-0.261302533577431	-0.261302533577431\\
74	0.13692	-0.313892213602718	-0.313892213602718\\
74	0.14058	-0.365425743180116	-0.365425743180116\\
74	0.14424	-0.415903122309627	-0.415903122309627\\
74	0.1479	-0.465324350991247	-0.465324350991247\\
74	0.15156	-0.513689429224983	-0.513689429224983\\
74	0.15522	-0.560998357010831	-0.560998357010831\\
74	0.15888	-0.607251134348788	-0.607251134348788\\
74	0.16254	-0.652447761238861	-0.652447761238861\\
74	0.1662	-0.696588237681047	-0.696588237681047\\
74	0.16986	-0.739672563675345	-0.739672563675345\\
74	0.17352	-0.781700739221755	-0.781700739221755\\
74	0.17718	-0.822672764320278	-0.822672764320278\\
74	0.18084	-0.862588638970906	-0.862588638970906\\
74	0.1845	-0.901448363173653	-0.901448363173653\\
74	0.18816	-0.939251936928516	-0.939251936928516\\
74	0.19182	-0.975999360235485	-0.975999360235485\\
74	0.19548	-1.01169063309457	-1.01169063309457\\
74	0.19914	-1.04632575550577	-1.04632575550577\\
74	0.2028	-1.07990472746908	-1.07990472746908\\
74	0.20646	-1.11242754898449	-1.11242754898449\\
74	0.21012	-1.14389422005203	-1.14389422005203\\
74	0.21378	-1.17430474067167	-1.17430474067167\\
74	0.21744	-1.20365911084343	-1.20365911084343\\
74	0.2211	-1.2319573305673	-1.2319573305673\\
74	0.22476	-1.25919939984328	-1.25919939984328\\
74	0.22842	-1.28538531867138	-1.28538531867138\\
74	0.23208	-1.31051508705159	-1.31051508705159\\
74	0.23574	-1.33458870498391	-1.33458870498391\\
74	0.2394	-1.35760617246834	-1.35760617246834\\
74	0.24306	-1.37956748950488	-1.37956748950488\\
74	0.24672	-1.40047265609354	-1.40047265609354\\
74	0.25038	-1.42032167223431	-1.42032167223431\\
74	0.25404	-1.43911453792719	-1.43911453792719\\
74	0.2577	-1.45685125317219	-1.45685125317219\\
74	0.26136	-1.47353181796929	-1.47353181796929\\
74	0.26502	-1.48915623231851	-1.48915623231851\\
74	0.26868	-1.50372449621984	-1.50372449621984\\
74	0.27234	-1.51723660967329	-1.51723660967329\\
74	0.276	-1.52969257267885	-1.52969257267885\\
};
\end{axis}

\begin{axis}[%
width=6.159cm,
height=3.097cm,
at={(0cm,4.301cm)},
scale only axis,
xmin=56,
xmax=74,
tick align=outside,
xlabel style={font=\color{white!15!black}},
xlabel={$L_{cut}$},
ymin=0.093,
ymax=0.276,
ylabel style={font=\color{white!15!black}},
ylabel={$D_{rlx}$},
zmin=-3.90013564007816,
zmax=52.2197035510817,
zlabel style={font=\color{white!15!black}},
zlabel={$u(t-2)$},
view={-140}{50},
axis background/.style={fill=white},
xmajorgrids,
ymajorgrids,
zmajorgrids
]
\addplot3[only marks, mark=*, mark options={}, mark size=1.5000pt, color=mycolor1, fill=mycolor1] table[row sep=crcr]{%
x	y	z\\
74	0.123	0.706917161850943\\
72	0.113	-0.725188933625476\\
61	0.095	-2.23969084196957\\
56	0.093	-0.663995623424443\\
};
\addplot3[only marks, mark=*, mark options={}, mark size=1.5000pt, color=mycolor2, fill=mycolor2] table[row sep=crcr]{%
x	y	z\\
67	0.276	45.9418866067325\\
66	0.255	36.5734993512285\\
62	0.209	18.3955084875498\\
57	0.193	12.6775369279536\\
};
\addplot3[only marks, mark=*, mark options={}, mark size=1.5000pt, color=black, fill=black] table[row sep=crcr]{%
x	y	z\\
69	0.104	-2.29954754700143\\
};
\addplot3[only marks, mark=*, mark options={}, mark size=1.5000pt, color=black, fill=black] table[row sep=crcr]{%
x	y	z\\
64	0.23	25.9519556411493\\
};

\addplot3[%
surf,
fill opacity=0.7, shader=interp, colormap={mymap}{[1pt] rgb(0pt)=(1,0.905882,0); rgb(1pt)=(1,0.901964,0); rgb(2pt)=(1,0.898051,0); rgb(3pt)=(1,0.894144,0); rgb(4pt)=(1,0.890243,0); rgb(5pt)=(1,0.886349,0); rgb(6pt)=(1,0.88246,0); rgb(7pt)=(1,0.878577,0); rgb(8pt)=(1,0.8747,0); rgb(9pt)=(1,0.870829,0); rgb(10pt)=(1,0.866964,0); rgb(11pt)=(1,0.863106,0); rgb(12pt)=(1,0.859253,0); rgb(13pt)=(1,0.855406,0); rgb(14pt)=(1,0.851566,0); rgb(15pt)=(1,0.847732,0); rgb(16pt)=(1,0.843903,0); rgb(17pt)=(1,0.840081,0); rgb(18pt)=(1,0.836265,0); rgb(19pt)=(1,0.832455,0); rgb(20pt)=(1,0.828652,0); rgb(21pt)=(1,0.824854,0); rgb(22pt)=(1,0.821063,0); rgb(23pt)=(1,0.817278,0); rgb(24pt)=(1,0.8135,0); rgb(25pt)=(1,0.809727,0); rgb(26pt)=(1,0.805961,0); rgb(27pt)=(1,0.8022,0); rgb(28pt)=(1,0.798445,0); rgb(29pt)=(1,0.794696,0); rgb(30pt)=(1,0.790953,0); rgb(31pt)=(1,0.787215,0); rgb(32pt)=(1,0.783484,0); rgb(33pt)=(1,0.779758,0); rgb(34pt)=(1,0.776038,0); rgb(35pt)=(1,0.772324,0); rgb(36pt)=(1,0.768615,0); rgb(37pt)=(1,0.764913,0); rgb(38pt)=(1,0.761217,0); rgb(39pt)=(1,0.757527,0); rgb(40pt)=(1,0.753843,0); rgb(41pt)=(1,0.750165,0); rgb(42pt)=(1,0.746493,0); rgb(43pt)=(1,0.742827,0); rgb(44pt)=(1,0.739167,0); rgb(45pt)=(1,0.735514,0); rgb(46pt)=(1,0.731867,0); rgb(47pt)=(1,0.728226,0); rgb(48pt)=(1,0.724591,0); rgb(49pt)=(1,0.720963,0); rgb(50pt)=(1,0.717341,0); rgb(51pt)=(1,0.713725,0); rgb(52pt)=(0.999994,0.710077,0); rgb(53pt)=(0.999974,0.706363,0); rgb(54pt)=(0.999942,0.702592,0); rgb(55pt)=(0.999898,0.698775,0); rgb(56pt)=(0.999841,0.694921,0); rgb(57pt)=(0.999771,0.691039,0); rgb(58pt)=(0.99969,0.687139,0); rgb(59pt)=(0.999596,0.68323,0); rgb(60pt)=(0.99949,0.679323,0); rgb(61pt)=(0.999372,0.675427,0); rgb(62pt)=(0.999242,0.67155,0); rgb(63pt)=(0.9991,0.667704,0); rgb(64pt)=(0.998946,0.663897,0); rgb(65pt)=(0.998781,0.660138,0); rgb(66pt)=(0.998605,0.656439,0); rgb(67pt)=(0.998416,0.652807,0); rgb(68pt)=(0.998217,0.649253,0); rgb(69pt)=(0.998006,0.645786,0); rgb(70pt)=(0.997785,0.642416,0); rgb(71pt)=(0.997552,0.639152,0); rgb(72pt)=(0.997308,0.636004,0); rgb(73pt)=(0.997053,0.632982,0); rgb(74pt)=(0.996788,0.630095,0); rgb(75pt)=(0.996512,0.627352,0); rgb(76pt)=(0.996226,0.624763,0); rgb(77pt)=(0.995851,0.622329,0); rgb(78pt)=(0.99494,0.619997,0); rgb(79pt)=(0.99345,0.617753,0); rgb(80pt)=(0.991419,0.61559,0); rgb(81pt)=(0.988885,0.613503,0); rgb(82pt)=(0.985886,0.611486,0); rgb(83pt)=(0.98246,0.609532,0); rgb(84pt)=(0.978643,0.607636,0); rgb(85pt)=(0.974475,0.605791,0); rgb(86pt)=(0.969992,0.603992,0); rgb(87pt)=(0.965232,0.602233,0); rgb(88pt)=(0.960233,0.600507,0); rgb(89pt)=(0.955033,0.598808,0); rgb(90pt)=(0.949669,0.59713,0); rgb(91pt)=(0.94418,0.595468,0); rgb(92pt)=(0.938602,0.593815,0); rgb(93pt)=(0.932974,0.592166,0); rgb(94pt)=(0.927333,0.590513,0); rgb(95pt)=(0.921717,0.588852,0); rgb(96pt)=(0.916164,0.587176,0); rgb(97pt)=(0.910711,0.585479,0); rgb(98pt)=(0.905397,0.583755,0); rgb(99pt)=(0.900258,0.581999,0); rgb(100pt)=(0.895333,0.580203,0); rgb(101pt)=(0.890659,0.578362,0); rgb(102pt)=(0.886275,0.576471,0); rgb(103pt)=(0.882047,0.574545,0); rgb(104pt)=(0.877819,0.572608,0); rgb(105pt)=(0.873592,0.57066,0); rgb(106pt)=(0.869366,0.568701,0); rgb(107pt)=(0.865143,0.566733,0); rgb(108pt)=(0.860924,0.564756,0); rgb(109pt)=(0.856708,0.562771,0); rgb(110pt)=(0.852497,0.560778,0); rgb(111pt)=(0.848292,0.558779,0); rgb(112pt)=(0.844092,0.556774,0); rgb(113pt)=(0.8399,0.554763,0); rgb(114pt)=(0.835716,0.552749,0); rgb(115pt)=(0.831541,0.55073,0); rgb(116pt)=(0.827374,0.548709,0); rgb(117pt)=(0.823219,0.546686,0); rgb(118pt)=(0.819074,0.54466,0); rgb(119pt)=(0.81494,0.542635,0); rgb(120pt)=(0.81082,0.540609,0); rgb(121pt)=(0.806712,0.538584,0); rgb(122pt)=(0.802619,0.53656,0); rgb(123pt)=(0.798541,0.534539,0); rgb(124pt)=(0.794478,0.532521,0); rgb(125pt)=(0.790431,0.530506,0); rgb(126pt)=(0.786402,0.528496,0); rgb(127pt)=(0.782391,0.526491,0); rgb(128pt)=(0.77841,0.524489,0); rgb(129pt)=(0.774523,0.522478,0); rgb(130pt)=(0.770731,0.520455,0); rgb(131pt)=(0.767022,0.518424,0); rgb(132pt)=(0.763384,0.516385,0); rgb(133pt)=(0.759804,0.514339,0); rgb(134pt)=(0.756272,0.51229,0); rgb(135pt)=(0.752775,0.510237,0); rgb(136pt)=(0.749302,0.508182,0); rgb(137pt)=(0.74584,0.506128,0); rgb(138pt)=(0.742378,0.504075,0); rgb(139pt)=(0.738904,0.502025,0); rgb(140pt)=(0.735406,0.499979,0); rgb(141pt)=(0.731872,0.49794,0); rgb(142pt)=(0.72829,0.495909,0); rgb(143pt)=(0.724649,0.493887,0); rgb(144pt)=(0.720936,0.491875,0); rgb(145pt)=(0.71714,0.489876,0); rgb(146pt)=(0.713249,0.487891,0); rgb(147pt)=(0.709251,0.485921,0); rgb(148pt)=(0.705134,0.483968,0); rgb(149pt)=(0.700887,0.482033,0); rgb(150pt)=(0.696497,0.480118,0); rgb(151pt)=(0.691952,0.478225,0); rgb(152pt)=(0.687242,0.476355,0); rgb(153pt)=(0.682353,0.47451,0); rgb(154pt)=(0.677195,0.472696,0); rgb(155pt)=(0.6717,0.470916,0); rgb(156pt)=(0.665891,0.469169,0); rgb(157pt)=(0.659791,0.46745,0); rgb(158pt)=(0.653423,0.465756,0); rgb(159pt)=(0.64681,0.464084,0); rgb(160pt)=(0.639976,0.462432,0); rgb(161pt)=(0.632943,0.460795,0); rgb(162pt)=(0.625734,0.459171,0); rgb(163pt)=(0.618373,0.457556,0); rgb(164pt)=(0.610882,0.455948,0); rgb(165pt)=(0.603284,0.454343,0); rgb(166pt)=(0.595604,0.452737,0); rgb(167pt)=(0.587863,0.451129,0); rgb(168pt)=(0.580084,0.449514,0); rgb(169pt)=(0.572292,0.447889,0); rgb(170pt)=(0.564508,0.446252,0); rgb(171pt)=(0.556756,0.444599,0); rgb(172pt)=(0.549059,0.442927,0); rgb(173pt)=(0.54144,0.441232,0); rgb(174pt)=(0.533922,0.439512,0); rgb(175pt)=(0.526529,0.437764,0); rgb(176pt)=(0.519282,0.435983,0); rgb(177pt)=(0.512206,0.434168,0); rgb(178pt)=(0.505323,0.432315,0); rgb(179pt)=(0.498628,0.430422,3.92506e-06); rgb(180pt)=(0.491973,0.428504,3.49981e-05); rgb(181pt)=(0.485331,0.426562,9.63073e-05); rgb(182pt)=(0.478704,0.424596,0.000186979); rgb(183pt)=(0.472096,0.422609,0.000306141); rgb(184pt)=(0.465508,0.420599,0.00045292); rgb(185pt)=(0.458942,0.418567,0.000626441); rgb(186pt)=(0.452401,0.416515,0.000825833); rgb(187pt)=(0.445885,0.414441,0.00105022); rgb(188pt)=(0.439399,0.412348,0.00129873); rgb(189pt)=(0.432942,0.410234,0.00157049); rgb(190pt)=(0.426518,0.408102,0.00186463); rgb(191pt)=(0.420129,0.40595,0.00218028); rgb(192pt)=(0.413777,0.40378,0.00251655); rgb(193pt)=(0.407464,0.401592,0.00287258); rgb(194pt)=(0.401191,0.399386,0.00324749); rgb(195pt)=(0.394962,0.397164,0.00364042); rgb(196pt)=(0.388777,0.394925,0.00405048); rgb(197pt)=(0.38264,0.39267,0.00447681); rgb(198pt)=(0.376552,0.390399,0.00491852); rgb(199pt)=(0.370516,0.388113,0.00537476); rgb(200pt)=(0.364532,0.385812,0.00584464); rgb(201pt)=(0.358605,0.383497,0.00632729); rgb(202pt)=(0.352735,0.381168,0.00682184); rgb(203pt)=(0.346925,0.378826,0.00732741); rgb(204pt)=(0.341176,0.376471,0.00784314); rgb(205pt)=(0.335485,0.374093,0.00847245); rgb(206pt)=(0.329843,0.371682,0.00930909); rgb(207pt)=(0.324249,0.369242,0.0103377); rgb(208pt)=(0.318701,0.366772,0.0115428); rgb(209pt)=(0.313198,0.364275,0.0129091); rgb(210pt)=(0.307739,0.361753,0.0144211); rgb(211pt)=(0.302322,0.359206,0.0160634); rgb(212pt)=(0.296945,0.356637,0.0178207); rgb(213pt)=(0.291607,0.354048,0.0196776); rgb(214pt)=(0.286307,0.35144,0.0216186); rgb(215pt)=(0.281043,0.348814,0.0236284); rgb(216pt)=(0.275813,0.346172,0.0256916); rgb(217pt)=(0.270616,0.343517,0.0277927); rgb(218pt)=(0.265451,0.340849,0.0299163); rgb(219pt)=(0.260317,0.33817,0.0320472); rgb(220pt)=(0.25521,0.335482,0.0341698); rgb(221pt)=(0.250131,0.332786,0.0362688); rgb(222pt)=(0.245078,0.330085,0.0383287); rgb(223pt)=(0.240048,0.327379,0.0403343); rgb(224pt)=(0.235042,0.324671,0.04227); rgb(225pt)=(0.230056,0.321962,0.0441205); rgb(226pt)=(0.22509,0.319254,0.0458704); rgb(227pt)=(0.220142,0.316548,0.0475043); rgb(228pt)=(0.215212,0.313846,0.0490067); rgb(229pt)=(0.210296,0.311149,0.0503624); rgb(230pt)=(0.205395,0.308459,0.0515759); rgb(231pt)=(0.200514,0.305763,0.052757); rgb(232pt)=(0.195655,0.303061,0.0539242); rgb(233pt)=(0.190817,0.300353,0.0550763); rgb(234pt)=(0.186001,0.297639,0.0562123); rgb(235pt)=(0.181207,0.294918,0.0573313); rgb(236pt)=(0.176434,0.292191,0.0584321); rgb(237pt)=(0.171685,0.289458,0.0595136); rgb(238pt)=(0.166957,0.286719,0.060575); rgb(239pt)=(0.162252,0.283973,0.0616151); rgb(240pt)=(0.15757,0.281221,0.0626328); rgb(241pt)=(0.152911,0.278463,0.0636271); rgb(242pt)=(0.148275,0.275699,0.0645971); rgb(243pt)=(0.143663,0.272929,0.0655416); rgb(244pt)=(0.139074,0.270152,0.0664596); rgb(245pt)=(0.134508,0.26737,0.06735); rgb(246pt)=(0.129967,0.264581,0.0682118); rgb(247pt)=(0.125449,0.261787,0.0690441); rgb(248pt)=(0.120956,0.258986,0.0698456); rgb(249pt)=(0.116487,0.25618,0.0706154); rgb(250pt)=(0.112043,0.253367,0.0713525); rgb(251pt)=(0.107623,0.250549,0.0720557); rgb(252pt)=(0.103229,0.247724,0.0727241); rgb(253pt)=(0.0988592,0.244894,0.0733566); rgb(254pt)=(0.0945149,0.242058,0.0739522); rgb(255pt)=(0.0901961,0.239216,0.0745098)}, mesh/rows=49]
table[row sep=crcr, point meta=\thisrow{c}] {%
%
x	y	z	c\\
56	0.093	-0.774335008038994	-0.774335008038994\\
56	0.09666	-0.624732271062047	-0.624732271062047\\
56	0.10032	-0.449585546727148	-0.449585546727148\\
56	0.10398	-0.248894835034243	-0.248894835034243\\
56	0.10764	-0.022660135983358	-0.022660135983358\\
56	0.1113	0.229118550425509	0.229118550425509\\
56	0.11496	0.506441224192354	0.506441224192354\\
56	0.11862	0.809307885317162	0.809307885317162\\
56	0.12228	1.13771853379996	1.13771853379996\\
56	0.12594	1.49167316964073	1.49167316964073\\
56	0.1296	1.87117179283949	1.87117179283949\\
56	0.13326	2.27621440339622	2.27621440339622\\
56	0.13692	2.70680100131094	2.70680100131094\\
56	0.14058	3.16293158658365	3.16293158658365\\
56	0.14424	3.64460615921431	3.64460615921431\\
56	0.1479	4.15182471920297	4.15182471920297\\
56	0.15156	4.68458726654961	4.68458726654961\\
56	0.15522	5.24289380125422	5.24289380125422\\
56	0.15888	5.82674432331682	5.82674432331682\\
56	0.16254	6.4361388327374	6.4361388327374\\
56	0.1662	7.07107732951595	7.07107732951595\\
56	0.16986	7.73155981365247	7.73155981365247\\
56	0.17352	8.41758628514699	8.41758628514699\\
56	0.17718	9.12915674399949	9.12915674399949\\
56	0.18084	9.86627119020995	9.86627119020995\\
56	0.1845	10.6289296237784	10.6289296237784\\
56	0.18816	11.4171320447048	11.4171320447048\\
56	0.19182	12.2308784529892	12.2308784529892\\
56	0.19548	13.0701688486316	13.0701688486316\\
56	0.19914	13.935003231632	13.935003231632\\
56	0.2028	14.8253816019904	14.8253816019904\\
56	0.20646	15.7413039597067	15.7413039597067\\
56	0.21012	16.682770304781	16.682770304781\\
56	0.21378	17.6497806372133	17.6497806372133\\
56	0.21744	18.6423349570036	18.6423349570036\\
56	0.2211	19.6604332641518	19.6604332641518\\
56	0.22476	20.704075558658	20.704075558658\\
56	0.22842	21.7732618405222	21.7732618405222\\
56	0.23208	22.8679921097444	22.8679921097444\\
56	0.23574	23.9882663663246	23.9882663663246\\
56	0.2394	25.1340846102627	25.1340846102627\\
56	0.24306	26.3054468415589	26.3054468415589\\
56	0.24672	27.5023530602129	27.5023530602129\\
56	0.25038	28.724803266225	28.724803266225\\
56	0.25404	29.9727974595951	29.9727974595951\\
56	0.2577	31.2463356403232	31.2463356403232\\
56	0.26136	32.5454178084092	32.5454178084092\\
56	0.26502	33.8700439638532	33.8700439638532\\
56	0.26868	35.2202141066552	35.2202141066552\\
56	0.27234	36.5959282368151	36.5959282368151\\
56	0.276	37.997186354333	37.997186354333\\
56.375	0.093	-0.895119669503726	-0.895119669503726\\
56.375	0.09666	-0.738288466764782	-0.738288466764782\\
56.375	0.10032	-0.555913276667885	-0.555913276667885\\
56.375	0.10398	-0.347994099212983	-0.347994099212983\\
56.375	0.10764	-0.114530934400108	-0.114530934400108\\
56.375	0.1113	0.144476217770757	0.144476217770757\\
56.375	0.11496	0.429027357299592	0.429027357299592\\
56.375	0.11862	0.739122484186398	0.739122484186398\\
56.375	0.12228	1.07476159843119	1.07476159843119\\
56.375	0.12594	1.43594470003395	1.43594470003395\\
56.375	0.1296	1.82267178899471	1.82267178899471\\
56.375	0.13326	2.23494286531344	2.23494286531344\\
56.375	0.13692	2.67275792899015	2.67275792899015\\
56.375	0.14058	3.13611698002485	3.13611698002485\\
56.375	0.14424	3.62502001841751	3.62502001841751\\
56.375	0.1479	4.13946704416816	4.13946704416816\\
56.375	0.15156	4.6794580572768	4.6794580572768\\
56.375	0.15522	5.24499305774341	5.24499305774341\\
56.375	0.15888	5.83607204556801	5.83607204556801\\
56.375	0.16254	6.45269502075057	6.45269502075057\\
56.375	0.1662	7.09486198329112	7.09486198329112\\
56.375	0.16986	7.76257293318965	7.76257293318965\\
56.375	0.17352	8.45582787044615	8.45582787044615\\
56.375	0.17718	9.17462679506064	9.17462679506064\\
56.375	0.18084	9.91896970703311	9.91896970703311\\
56.375	0.1845	10.6888566063635	10.6888566063635\\
56.375	0.18816	11.484287493052	11.484287493052\\
56.375	0.19182	12.3052623670984	12.3052623670984\\
56.375	0.19548	13.1517812285028	13.1517812285028\\
56.375	0.19914	14.0238440772651	14.0238440772651\\
56.375	0.2028	14.9214509133855	14.9214509133855\\
56.375	0.20646	15.8446017368638	15.8446017368638\\
56.375	0.21012	16.7932965477001	16.7932965477001\\
56.375	0.21378	17.7675353458944	17.7675353458944\\
56.375	0.21744	18.7673181314467	18.7673181314467\\
56.375	0.2211	19.7926449043569	19.7926449043569\\
56.375	0.22476	20.8435156646251	20.8435156646251\\
56.375	0.22842	21.9199304122513	21.9199304122513\\
56.375	0.23208	23.0218891472355	23.0218891472355\\
56.375	0.23574	24.1493918695777	24.1493918695777\\
56.375	0.2394	25.3024385792778	25.3024385792778\\
56.375	0.24306	26.4810292763359	26.4810292763359\\
56.375	0.24672	27.685163960752	27.685163960752\\
56.375	0.25038	28.9148426325261	28.9148426325261\\
56.375	0.25404	30.1700652916582	30.1700652916582\\
56.375	0.2577	31.4508319381482	31.4508319381482\\
56.375	0.26136	32.7571425719962	32.7571425719962\\
56.375	0.26502	34.0889971932022	34.0889971932022\\
56.375	0.26868	35.4463958017662	35.4463958017662\\
56.375	0.27234	36.8293383976882	36.8293383976882\\
56.375	0.276	38.2378249809681	38.2378249809681\\
56.75	0.093	-1.0135356579913	-1.0135356579913\\
56.75	0.09666	-0.849475989490362	-0.849475989490362\\
56.75	0.10032	-0.659872333631446	-0.659872333631446\\
56.75	0.10398	-0.444724690414553	-0.444724690414553\\
56.75	0.10764	-0.204033059839695	-0.204033059839695\\
56.75	0.1113	0.0622025580931531	0.0622025580931531\\
56.75	0.11496	0.353982163383979	0.353982163383979\\
56.75	0.11862	0.671305756032789	0.671305756032789\\
56.75	0.12228	1.01417333603958	1.01417333603958\\
56.75	0.12594	1.38258490340435	1.38258490340435\\
56.75	0.1296	1.7765404581271	1.7765404581271\\
56.75	0.13326	2.19604000020782	2.19604000020782\\
56.75	0.13692	2.64108352964652	2.64108352964652\\
56.75	0.14058	3.11167104644321	3.11167104644321\\
56.75	0.14424	3.60780255059788	3.60780255059788\\
56.75	0.1479	4.12947804211052	4.12947804211052\\
56.75	0.15156	4.67669752098114	4.67669752098114\\
56.75	0.15522	5.24946098720975	5.24946098720975\\
56.75	0.15888	5.84776844079635	5.84776844079635\\
56.75	0.16254	6.47161988174091	6.47161988174091\\
56.75	0.1662	7.12101531004345	7.12101531004345\\
56.75	0.16986	7.79595472570396	7.79595472570396\\
56.75	0.17352	8.49643812872247	8.49643812872247\\
56.75	0.17718	9.22246551909895	9.22246551909895\\
56.75	0.18084	9.97403689683341	9.97403689683341\\
56.75	0.1845	10.7511522619258	10.7511522619258\\
56.75	0.18816	11.5538116143763	11.5538116143763\\
56.75	0.19182	12.3820149541847	12.3820149541847\\
56.75	0.19548	13.2357622813511	13.2357622813511\\
56.75	0.19914	14.1150535958754	14.1150535958754\\
56.75	0.2028	15.0198888977578	15.0198888977578\\
56.75	0.20646	15.9502681869981	15.9502681869981\\
56.75	0.21012	16.9061914635964	16.9061914635964\\
56.75	0.21378	17.8876587275527	17.8876587275527\\
56.75	0.21744	18.8946699788669	18.8946699788669\\
56.75	0.2211	19.9272252175391	19.9272252175391\\
56.75	0.22476	20.9853244435694	20.9853244435694\\
56.75	0.22842	22.0689676569576	22.0689676569576\\
56.75	0.23208	23.1781548577037	23.1781548577037\\
56.75	0.23574	24.3128860458079	24.3128860458079\\
56.75	0.2394	25.47316122127	25.47316122127\\
56.75	0.24306	26.6589803840901	26.6589803840901\\
56.75	0.24672	27.8703435342682	27.8703435342682\\
56.75	0.25038	29.1072506718043	29.1072506718043\\
56.75	0.25404	30.3697017966984	30.3697017966984\\
56.75	0.2577	31.6576969089504	31.6576969089504\\
56.75	0.26136	32.9712360085604	32.9712360085604\\
56.75	0.26502	34.3103190955284	34.3103190955284\\
56.75	0.26868	35.6749461698544	35.6749461698544\\
56.75	0.27234	37.0651172315383	37.0651172315383\\
56.75	0.276	38.4808322805802	38.4808322805802\\
57.125	0.093	-1.12958297350166	-1.12958297350166\\
57.125	0.09666	-0.958294839238722	-0.958294839238722\\
57.125	0.10032	-0.761462717617816	-0.761462717617816\\
57.125	0.10398	-0.539086608638925	-0.539086608638925\\
57.125	0.10764	-0.29116651230207	-0.29116651230207\\
57.125	0.1113	-0.0177024286072314	-0.0177024286072314\\
57.125	0.11496	0.281305642445599	0.281305642445599\\
57.125	0.11862	0.605857700856399	0.605857700856399\\
57.125	0.12228	0.955953746625189	0.955953746625189\\
57.125	0.12594	1.33159377975196	1.33159377975196\\
57.125	0.1296	1.7327778002367	1.7327778002367\\
57.125	0.13326	2.15950580807941	2.15950580807941\\
57.125	0.13692	2.61177780328011	2.61177780328011\\
57.125	0.14058	3.08959378583879	3.08959378583879\\
57.125	0.14424	3.59295375575546	3.59295375575546\\
57.125	0.1479	4.12185771303009	4.12185771303009\\
57.125	0.15156	4.67630565766271	4.67630565766271\\
57.125	0.15522	5.25629758965331	5.25629758965331\\
57.125	0.15888	5.8618335090019	5.8618335090019\\
57.125	0.16254	6.49291341570846	6.49291341570846\\
57.125	0.1662	7.149537309773	7.149537309773\\
57.125	0.16986	7.83170519119551	7.83170519119551\\
57.125	0.17352	8.539417059976	8.539417059976\\
57.125	0.17718	9.27267291611449	9.27267291611449\\
57.125	0.18084	10.0314727596109	10.0314727596109\\
57.125	0.1845	10.8158165904654	10.8158165904654\\
57.125	0.18816	11.6257044086778	11.6257044086778\\
57.125	0.19182	12.4611362142482	12.4611362142482\\
57.125	0.19548	13.3221120071766	13.3221120071766\\
57.125	0.19914	14.2086317874629	14.2086317874629\\
57.125	0.2028	15.1206955551073	15.1206955551073\\
57.125	0.20646	16.0583033101096	16.0583033101096\\
57.125	0.21012	17.0214550524699	17.0214550524699\\
57.125	0.21378	18.0101507821881	18.0101507821881\\
57.125	0.21744	19.0243904992644	19.0243904992644\\
57.125	0.2211	20.0641742036986	20.0641742036986\\
57.125	0.22476	21.1295018954908	21.1295018954908\\
57.125	0.22842	22.220373574641	22.220373574641\\
57.125	0.23208	23.3367892411492	23.3367892411492\\
57.125	0.23574	24.4787488950153	24.4787488950153\\
57.125	0.2394	25.6462525362395	25.6462525362395\\
57.125	0.24306	26.8393001648216	26.8393001648216\\
57.125	0.24672	28.0578917807617	28.0578917807617\\
57.125	0.25038	29.3020273840598	29.3020273840598\\
57.125	0.25404	30.5717069747158	30.5717069747158\\
57.125	0.2577	31.8669305527299	31.8669305527299\\
57.125	0.26136	33.1876981181018	33.1876981181018\\
57.125	0.26502	34.5340096708318	34.5340096708318\\
57.125	0.26868	35.9058652109198	35.9058652109198\\
57.125	0.27234	37.3032647383657	37.3032647383657\\
57.125	0.276	38.7262082531696	38.7262082531696\\
57.5	0.093	-1.24326161603483	-1.24326161603483\\
57.5	0.09666	-1.06474501600991	-1.06474501600991\\
57.5	0.10032	-0.860684428626991	-0.860684428626991\\
57.5	0.10398	-0.63107985388611	-0.63107985388611\\
57.5	0.10764	-0.375931291787264	-0.375931291787264\\
57.5	0.1113	-0.0952387423304284	-0.0952387423304284\\
57.5	0.11496	0.210997794484392	0.210997794484392\\
57.5	0.11862	0.54277831865719	0.54277831865719\\
57.5	0.12228	0.900102830187977	0.900102830187977\\
57.5	0.12594	1.28297132907674	1.28297132907674\\
57.5	0.1296	1.69138381532347	1.69138381532347\\
57.5	0.13326	2.12534028892818	2.12534028892818\\
57.5	0.13692	2.58484074989088	2.58484074989088\\
57.5	0.14058	3.06988519821155	3.06988519821155\\
57.5	0.14424	3.58047363389021	3.58047363389021\\
57.5	0.1479	4.11660605692684	4.11660605692684\\
57.5	0.15156	4.67828246732146	4.67828246732146\\
57.5	0.15522	5.26550286507405	5.26550286507405\\
57.5	0.15888	5.87826725018463	5.87826725018463\\
57.5	0.16254	6.51657562265319	6.51657562265319\\
57.5	0.1662	7.18042798247972	7.18042798247972\\
57.5	0.16986	7.86982432966423	7.86982432966423\\
57.5	0.17352	8.58476466420672	8.58476466420672\\
57.5	0.17718	9.32524898610719	9.32524898610719\\
57.5	0.18084	10.0912772953656	10.0912772953656\\
57.5	0.1845	10.8828495919821	10.8828495919821\\
57.5	0.18816	11.6999658759565	11.6999658759565\\
57.5	0.19182	12.5426261472889	12.5426261472889\\
57.5	0.19548	13.4108304059792	13.4108304059792\\
57.5	0.19914	14.3045786520276	14.3045786520276\\
57.5	0.2028	15.2238708854339	15.2238708854339\\
57.5	0.20646	16.1687071061982	16.1687071061982\\
57.5	0.21012	17.1390873143205	17.1390873143205\\
57.5	0.21378	18.1350115098008	18.1350115098008\\
57.5	0.21744	19.156479692639	19.156479692639\\
57.5	0.2211	20.2034918628353	20.2034918628353\\
57.5	0.22476	21.2760480203895	21.2760480203895\\
57.5	0.22842	22.3741481653017	22.3741481653017\\
57.5	0.23208	23.4977922975718	23.4977922975718\\
57.5	0.23574	24.6469804172	24.6469804172\\
57.5	0.2394	25.8217125241861	25.8217125241861\\
57.5	0.24306	27.0219886185302	27.0219886185302\\
57.5	0.24672	28.2478087002323	28.2478087002323\\
57.5	0.25038	29.4991727692923	29.4991727692923\\
57.5	0.25404	30.7760808257104	30.7760808257104\\
57.5	0.2577	32.0785328694864	32.0785328694864\\
57.5	0.26136	33.4065289006204	33.4065289006204\\
57.5	0.26502	34.7600689191124	34.7600689191124\\
57.5	0.26868	36.1391529249624	36.1391529249624\\
57.5	0.27234	37.5437809181703	37.5437809181703\\
57.5	0.276	38.9739528987362	38.9739528987362\\
57.875	0.093	-1.35457158559081	-1.35457158559081\\
57.875	0.09666	-1.16882651980389	-1.16882651980389\\
57.875	0.10032	-0.957537466658986	-0.957537466658986\\
57.875	0.10398	-0.720704426156107	-0.720704426156107\\
57.875	0.10764	-0.458327398295264	-0.458327398295264\\
57.875	0.1113	-0.170406383076438	-0.170406383076438\\
57.875	0.11496	0.14305861950038	0.14305861950038\\
57.875	0.11862	0.482067609435175	0.482067609435175\\
57.875	0.12228	0.846620586727953	0.846620586727953\\
57.875	0.12594	1.23671755137871	1.23671755137871\\
57.875	0.1296	1.65235850338744	1.65235850338744\\
57.875	0.13326	2.09354344275414	2.09354344275414\\
57.875	0.13692	2.56027236947883	2.56027236947883\\
57.875	0.14058	3.05254528356151	3.05254528356151\\
57.875	0.14424	3.57036218500216	3.57036218500216\\
57.875	0.1479	4.11372307380078	4.11372307380078\\
57.875	0.15156	4.6826279499574	4.6826279499574\\
57.875	0.15522	5.27707681347199	5.27707681347199\\
57.875	0.15888	5.89706966434456	5.89706966434456\\
57.875	0.16254	6.54260650257512	6.54260650257512\\
57.875	0.1662	7.21368732816364	7.21368732816364\\
57.875	0.16986	7.91031214111014	7.91031214111014\\
57.875	0.17352	8.63248094141463	8.63248094141463\\
57.875	0.17718	9.3801937290771	9.3801937290771\\
57.875	0.18084	10.1534505040975	10.1534505040975\\
57.875	0.1845	10.952251266476	10.952251266476\\
57.875	0.18816	11.7765960162124	11.7765960162124\\
57.875	0.19182	12.6264847533068	12.6264847533068\\
57.875	0.19548	13.5019174777591	13.5019174777591\\
57.875	0.19914	14.4028941895695	14.4028941895695\\
57.875	0.2028	15.3294148887378	15.3294148887378\\
57.875	0.20646	16.2814795752641	16.2814795752641\\
57.875	0.21012	17.2590882491484	17.2590882491484\\
57.875	0.21378	18.2622409103906	18.2622409103906\\
57.875	0.21744	19.2909375589909	19.2909375589909\\
57.875	0.2211	20.3451781949491	20.3451781949491\\
57.875	0.22476	21.4249628182653	21.4249628182653\\
57.875	0.22842	22.5302914289395	22.5302914289395\\
57.875	0.23208	23.6611640269717	23.6611640269717\\
57.875	0.23574	24.8175806123618	24.8175806123618\\
57.875	0.2394	25.9995411851099	25.9995411851099\\
57.875	0.24306	27.207045745216	27.207045745216\\
57.875	0.24672	28.4400942926801	28.4400942926801\\
57.875	0.25038	29.6986868275022	29.6986868275022\\
57.875	0.25404	30.9828233496822	30.9828233496822\\
57.875	0.2577	32.2925038592202	32.2925038592202\\
57.875	0.26136	33.6277283561162	33.6277283561162\\
57.875	0.26502	34.9884968403702	34.9884968403702\\
57.875	0.26868	36.3748093119821	36.3748093119821\\
57.875	0.27234	37.7866657709521	37.7866657709521\\
57.875	0.276	39.22406621728	39.22406621728\\
58.25	0.093	-1.46351288216961	-1.46351288216961\\
58.25	0.09666	-1.27053935062069	-1.27053935062069\\
58.25	0.10032	-1.05202183171379	-1.05202183171379\\
58.25	0.10398	-0.807960325448924	-0.807960325448924\\
58.25	0.10764	-0.538354831826084	-0.538354831826084\\
58.25	0.1113	-0.24320535084526	-0.24320535084526\\
58.25	0.11496	0.0774881174935551	0.0774881174935551\\
58.25	0.11862	0.423725573190341	0.423725573190341\\
58.25	0.12228	0.795507016245116	0.795507016245116\\
58.25	0.12594	1.19283244665786	1.19283244665786\\
58.25	0.1296	1.61570186442859	1.61570186442859\\
58.25	0.13326	2.06411526955728	2.06411526955728\\
58.25	0.13692	2.53807266204397	2.53807266204397\\
58.25	0.14058	3.03757404188865	3.03757404188865\\
58.25	0.14424	3.56261940909129	3.56261940909129\\
58.25	0.1479	4.11320876365191	4.11320876365191\\
58.25	0.15156	4.68934210557052	4.68934210557052\\
58.25	0.15522	5.2910194348471	5.2910194348471\\
58.25	0.15888	5.91824075148167	5.91824075148167\\
58.25	0.16254	6.57100605547421	6.57100605547421\\
58.25	0.1662	7.24931534682474	7.24931534682474\\
58.25	0.16986	7.95316862553324	7.95316862553324\\
58.25	0.17352	8.68256589159972	8.68256589159972\\
58.25	0.17718	9.43750714502418	9.43750714502418\\
58.25	0.18084	10.2179923858066	10.2179923858066\\
58.25	0.1845	11.024021613947	11.024021613947\\
58.25	0.18816	11.8555948294454	11.8555948294454\\
58.25	0.19182	12.7127120323018	12.7127120323018\\
58.25	0.19548	13.5953732225162	13.5953732225162\\
58.25	0.19914	14.5035784000885	14.5035784000885\\
58.25	0.2028	15.4373275650189	15.4373275650189\\
58.25	0.20646	16.3966207173071	16.3966207173071\\
58.25	0.21012	17.3814578569534	17.3814578569534\\
58.25	0.21378	18.3918389839577	18.3918389839577\\
58.25	0.21744	19.4277640983199	19.4277640983199\\
58.25	0.2211	20.4892332000401	20.4892332000401\\
58.25	0.22476	21.5762462891183	21.5762462891183\\
58.25	0.22842	22.6888033655545	22.6888033655545\\
58.25	0.23208	23.8269044293487	23.8269044293487\\
58.25	0.23574	24.9905494805008	24.9905494805008\\
58.25	0.2394	26.1797385190109	26.1797385190109\\
58.25	0.24306	27.394471544879	27.394471544879\\
58.25	0.24672	28.6347485581051	28.6347485581051\\
58.25	0.25038	29.9005695586891	29.9005695586891\\
58.25	0.25404	31.1919345466312	31.1919345466312\\
58.25	0.2577	32.5088435219312	32.5088435219312\\
58.25	0.26136	33.8512964845892	33.8512964845892\\
58.25	0.26502	35.2192934346051	35.2192934346051\\
58.25	0.26868	36.6128343719791	36.6128343719791\\
58.25	0.27234	38.031919296711	38.031919296711\\
58.25	0.276	39.4765482088009	39.4765482088009\\
58.625	0.093	-1.57008550577121	-1.57008550577121\\
58.625	0.09666	-1.3698835084603	-1.3698835084603\\
58.625	0.10032	-1.14413752379144	-1.14413752379144\\
58.625	0.10398	-0.892847551764572	-0.892847551764572\\
58.625	0.10764	-0.616013592379719	-0.616013592379719\\
58.625	0.1113	-0.313635645636884	-0.313635645636884\\
58.625	0.11496	0.0142862884639214	0.0142862884639214\\
58.625	0.11862	0.36775220992269	0.36775220992269\\
58.625	0.12228	0.746762118739456	0.746762118739456\\
58.625	0.12594	1.15131601491419	1.15131601491419\\
58.625	0.1296	1.58141389844691	1.58141389844691\\
58.625	0.13326	2.03705576933762	2.03705576933762\\
58.625	0.13692	2.5182416275863	2.5182416275863\\
58.625	0.14058	3.02497147319297	3.02497147319297\\
58.625	0.14424	3.55724530615759	3.55724530615759\\
58.625	0.1479	4.11506312648022	4.11506312648022\\
58.625	0.15156	4.69842493416082	4.69842493416082\\
58.625	0.15522	5.3073307291994	5.3073307291994\\
58.625	0.15888	5.94178051159597	5.94178051159597\\
58.625	0.16254	6.6017742813505	6.6017742813505\\
58.625	0.1662	7.28731203846303	7.28731203846303\\
58.625	0.16986	7.99839378293352	7.99839378293352\\
58.625	0.17352	8.73501951476199	8.73501951476199\\
58.625	0.17718	9.49718923394845	9.49718923394845\\
58.625	0.18084	10.2849029404929	10.2849029404929\\
58.625	0.1845	11.0981606343953	11.0981606343953\\
58.625	0.18816	11.9369623156557	11.9369623156557\\
58.625	0.19182	12.8013079842741	12.8013079842741\\
58.625	0.19548	13.6911976402504	13.6911976402504\\
58.625	0.19914	14.6066312835848	14.6066312835848\\
58.625	0.2028	15.5476089142771	15.5476089142771\\
58.625	0.20646	16.5141305323274	16.5141305323274\\
58.625	0.21012	17.5061961377356	17.5061961377356\\
58.625	0.21378	18.5238057305019	18.5238057305019\\
58.625	0.21744	19.5669593106261	19.5669593106261\\
58.625	0.2211	20.6356568781083	20.6356568781083\\
58.625	0.22476	21.7298984329485	21.7298984329485\\
58.625	0.22842	22.8496839751467	22.8496839751467\\
58.625	0.23208	23.9950135047029	23.9950135047029\\
58.625	0.23574	25.165887021617	25.165887021617\\
58.625	0.2394	26.3623045258891	26.3623045258891\\
58.625	0.24306	27.5842660175192	27.5842660175192\\
58.625	0.24672	28.8317714965072	28.8317714965072\\
58.625	0.25038	30.1048209628533	30.1048209628533\\
58.625	0.25404	31.4034144165573	31.4034144165573\\
58.625	0.2577	32.7275518576194	32.7275518576194\\
58.625	0.26136	34.0772332860393	34.0772332860393\\
58.625	0.26502	35.4524587018173	35.4524587018173\\
58.625	0.26868	36.8532281049533	36.8532281049533\\
58.625	0.27234	38.2795414954472	38.2795414954472\\
58.625	0.276	39.7313988732991	39.7313988732991\\
59	0.093	-1.67428945639563	-1.67428945639563\\
59	0.09666	-1.46685899332273	-1.46685899332273\\
59	0.10032	-1.23388454289187	-1.23388454289187\\
59	0.10398	-0.975366105103003	-0.975366105103003\\
59	0.10764	-0.69130367995616	-0.69130367995616\\
59	0.1113	-0.381697267451335	-0.381697267451335\\
59	0.11496	-0.0465468675885319	-0.0465468675885319\\
59	0.11862	0.314147519632234	0.314147519632234\\
59	0.12228	0.700385894210998	0.700385894210998\\
59	0.12594	1.11216825614772	1.11216825614772\\
59	0.1296	1.54949460544244	1.54949460544244\\
59	0.13326	2.01236494209514	2.01236494209514\\
59	0.13692	2.50077926610581	2.50077926610581\\
59	0.14058	3.01473757747447	3.01473757747447\\
59	0.14424	3.5542398762011	3.5542398762011\\
59	0.1479	4.11928616228572	4.11928616228572\\
59	0.15156	4.70987643572832	4.70987643572832\\
59	0.15522	5.32601069652889	5.32601069652889\\
59	0.15888	5.96768894468745	5.96768894468745\\
59	0.16254	6.63491118020399	6.63491118020399\\
59	0.1662	7.3276774030785	7.3276774030785\\
59	0.16986	8.04598761331099	8.04598761331099\\
59	0.17352	8.78984181090146	8.78984181090146\\
59	0.17718	9.55923999584992	9.55923999584992\\
59	0.18084	10.3541821681563	10.3541821681563\\
59	0.1845	11.1746683278207	11.1746683278207\\
59	0.18816	12.0206984748431	12.0206984748431\\
59	0.19182	12.8922726092235	12.8922726092235\\
59	0.19548	13.7893907309619	13.7893907309619\\
59	0.19914	14.7120528400582	14.7120528400582\\
59	0.2028	15.6602589365125	15.6602589365125\\
59	0.20646	16.6340090203248	16.6340090203248\\
59	0.21012	17.6333030914951	17.6333030914951\\
59	0.21378	18.6581411500233	18.6581411500233\\
59	0.21744	19.7085231959095	19.7085231959095\\
59	0.2211	20.7844492291537	20.7844492291537\\
59	0.22476	21.8859192497559	21.8859192497559\\
59	0.22842	23.0129332577161	23.0129332577161\\
59	0.23208	24.1654912530342	24.1654912530342\\
59	0.23574	25.3435932357104	25.3435932357104\\
59	0.2394	26.5472392057445	26.5472392057445\\
59	0.24306	27.7764291631365	27.7764291631365\\
59	0.24672	29.0311631078866	29.0311631078866\\
59	0.25038	30.3114410399947	30.3114410399947\\
59	0.25404	31.6172629594607	31.6172629594607\\
59	0.2577	32.9486288662847	32.9486288662847\\
59	0.26136	34.3055387604667	34.3055387604667\\
59	0.26502	35.6879926420067	35.6879926420067\\
59	0.26868	37.0959905109046	37.0959905109046\\
59	0.27234	38.5295323671605	38.5295323671605\\
59	0.276	39.9886182107744	39.9886182107744\\
59.375	0.093	-1.77612473404286	-1.77612473404286\\
59.375	0.09666	-1.56146580520796	-1.56146580520796\\
59.375	0.10032	-1.3212628890151	-1.3212628890151\\
59.375	0.10398	-1.05551598546425	-1.05551598546425\\
59.375	0.10764	-0.764225094555407	-0.764225094555407\\
59.375	0.1113	-0.447390216288584	-0.447390216288584\\
59.375	0.11496	-0.105011350663791	-0.105011350663791\\
59.375	0.11862	0.262911502318973	0.262911502318973\\
59.375	0.12228	0.656378342659726	0.656378342659726\\
59.375	0.12594	1.07538917035845	1.07538917035845\\
59.375	0.1296	1.51994398541517	1.51994398541517\\
59.375	0.13326	1.99004278782986	1.99004278782986\\
59.375	0.13692	2.48568557760253	2.48568557760253\\
59.375	0.14058	3.00687235473318	3.00687235473318\\
59.375	0.14424	3.5536031192218	3.5536031192218\\
59.375	0.1479	4.12587787106841	4.12587787106841\\
59.375	0.15156	4.72369661027301	4.72369661027301\\
59.375	0.15522	5.34705933683558	5.34705933683558\\
59.375	0.15888	5.99596605075613	5.99596605075613\\
59.375	0.16254	6.67041675203467	6.67041675203467\\
59.375	0.1662	7.37041144067117	7.37041144067117\\
59.375	0.16986	8.09595011666566	8.09595011666566\\
59.375	0.17352	8.84703278001813	8.84703278001813\\
59.375	0.17718	9.62365943072857	9.62365943072857\\
59.375	0.18084	10.425830068797	10.425830068797\\
59.375	0.1845	11.2535446942234	11.2535446942234\\
59.375	0.18816	12.1068033070078	12.1068033070078\\
59.375	0.19182	12.9856059071501	12.9856059071501\\
59.375	0.19548	13.8899524946505	13.8899524946505\\
59.375	0.19914	14.8198430695088	14.8198430695088\\
59.375	0.2028	15.7752776317251	15.7752776317251\\
59.375	0.20646	16.7562561812994	16.7562561812994\\
59.375	0.21012	17.7627787182317	17.7627787182317\\
59.375	0.21378	18.7948452425219	18.7948452425219\\
59.375	0.21744	19.8524557541701	19.8524557541701\\
59.375	0.2211	20.9356102531763	20.9356102531763\\
59.375	0.22476	22.0443087395405	22.0443087395405\\
59.375	0.22842	23.1785512132627	23.1785512132627\\
59.375	0.23208	24.3383376743428	24.3383376743428\\
59.375	0.23574	25.523668122781	25.523668122781\\
59.375	0.2394	26.7345425585771	26.7345425585771\\
59.375	0.24306	27.9709609817311	27.9709609817311\\
59.375	0.24672	29.2329233922432	29.2329233922432\\
59.375	0.25038	30.5204297901132	30.5204297901132\\
59.375	0.25404	31.8334801753412	31.8334801753412\\
59.375	0.2577	33.1720745479273	33.1720745479273\\
59.375	0.26136	34.5362129078712	34.5362129078712\\
59.375	0.26502	35.9258952551732	35.9258952551732\\
59.375	0.26868	37.3411215898331	37.3411215898331\\
59.375	0.27234	38.781891911851	38.781891911851\\
59.375	0.276	40.2482062212269	40.2482062212269\\
59.75	0.093	-1.8755913387129	-1.8755913387129\\
59.75	0.09666	-1.65370394411601	-1.65370394411601\\
59.75	0.10032	-1.40627256216116	-1.40627256216116\\
59.75	0.10398	-1.13329719284831	-1.13329719284831\\
59.75	0.10764	-0.834777836177473	-0.834777836177473\\
59.75	0.1113	-0.51071449214866	-0.51071449214866\\
59.75	0.11496	-0.161107160761862	-0.161107160761862\\
59.75	0.11862	0.214044157982892	0.214044157982892\\
59.75	0.12228	0.614739464085643	0.614739464085643\\
59.75	0.12594	1.04097875754636	1.04097875754636\\
59.75	0.1296	1.49276203836506	1.49276203836506\\
59.75	0.13326	1.97008930654176	1.97008930654176\\
59.75	0.13692	2.47296056207642	2.47296056207642\\
59.75	0.14058	3.00137580496907	3.00137580496907\\
59.75	0.14424	3.55533503521968	3.55533503521968\\
59.75	0.1479	4.1348382528283	4.1348382528283\\
59.75	0.15156	4.73988545779488	4.73988545779488\\
59.75	0.15522	5.37047665011945	5.37047665011945\\
59.75	0.15888	6.026611829802	6.026611829802\\
59.75	0.16254	6.70829099684252	6.70829099684252\\
59.75	0.1662	7.41551415124103	7.41551415124103\\
59.75	0.16986	8.1482812929975	8.1482812929975\\
59.75	0.17352	8.90659242211196	8.90659242211196\\
59.75	0.17718	9.69044753858441	9.69044753858441\\
59.75	0.18084	10.4998466424148	10.4998466424148\\
59.75	0.1845	11.3347897336032	11.3347897336032\\
59.75	0.18816	12.1952768121496	12.1952768121496\\
59.75	0.19182	13.081307878054	13.081307878054\\
59.75	0.19548	13.9928829313163	13.9928829313163\\
59.75	0.19914	14.9300019719366	14.9300019719366\\
59.75	0.2028	15.8926649999149	15.8926649999149\\
59.75	0.20646	16.8808720152512	16.8808720152512\\
59.75	0.21012	17.8946230179455	17.8946230179455\\
59.75	0.21378	18.9339180079977	18.9339180079977\\
59.75	0.21744	19.9987569854079	19.9987569854079\\
59.75	0.2211	21.0891399501761	21.0891399501761\\
59.75	0.22476	22.2050669023023	22.2050669023023\\
59.75	0.22842	23.3465378417865	23.3465378417865\\
59.75	0.23208	24.5135527686286	24.5135527686286\\
59.75	0.23574	25.7061116828287	25.7061116828287\\
59.75	0.2394	26.9242145843868	26.9242145843868\\
59.75	0.24306	28.1678614733029	28.1678614733029\\
59.75	0.24672	29.4370523495769	29.4370523495769\\
59.75	0.25038	30.731787213209	30.731787213209\\
59.75	0.25404	32.052066064199	32.052066064199\\
59.75	0.2577	33.397888902547	33.397888902547\\
59.75	0.26136	34.7692557282529	34.7692557282529\\
59.75	0.26502	36.1661665413169	36.1661665413169\\
59.75	0.26868	37.5886213417388	37.5886213417388\\
59.75	0.27234	39.0366201295187	39.0366201295187\\
59.75	0.276	40.5101629046566	40.5101629046566\\
60.125	0.093	-1.97268927040576	-1.97268927040576\\
60.125	0.09666	-1.74357341004688	-1.74357341004688\\
60.125	0.10032	-1.48891356233003	-1.48891356233003\\
60.125	0.10398	-1.20870972725518	-1.20870972725518\\
60.125	0.10764	-0.902961904822352	-0.902961904822352\\
60.125	0.1113	-0.571670095031541	-0.571670095031541\\
60.125	0.11496	-0.214834297882753	-0.214834297882753\\
60.125	0.11862	0.167545486623998	0.167545486623998\\
60.125	0.12228	0.575469258488747	0.575469258488747\\
60.125	0.12594	1.00893701771145	1.00893701771145\\
60.125	0.1296	1.46794876429216	1.46794876429216\\
60.125	0.13326	1.95250449823084	1.95250449823084\\
60.125	0.13692	2.4626042195275	2.4626042195275\\
60.125	0.14058	2.99824792818215	2.99824792818215\\
60.125	0.14424	3.55943562419476	3.55943562419476\\
60.125	0.1479	4.14616730756536	4.14616730756536\\
60.125	0.15156	4.75844297829394	4.75844297829394\\
60.125	0.15522	5.3962626363805	5.3962626363805\\
60.125	0.15888	6.05962628182505	6.05962628182505\\
60.125	0.16254	6.74853391462757	6.74853391462757\\
60.125	0.1662	7.46298553478807	7.46298553478807\\
60.125	0.16986	8.20298114230654	8.20298114230654\\
60.125	0.17352	8.968520737183	8.968520737183\\
60.125	0.17718	9.75960431941743	9.75960431941743\\
60.125	0.18084	10.5762318890098	10.5762318890098\\
60.125	0.1845	11.4184034459602	11.4184034459602\\
60.125	0.18816	12.2861189902686	12.2861189902686\\
60.125	0.19182	13.179378521935	13.179378521935\\
60.125	0.19548	14.0981820409593	14.0981820409593\\
60.125	0.19914	15.0425295473416	15.0425295473416\\
60.125	0.2028	16.0124210410819	16.0124210410819\\
60.125	0.20646	17.0078565221802	17.0078565221802\\
60.125	0.21012	18.0288359906365	18.0288359906365\\
60.125	0.21378	19.0753594464507	19.0753594464507\\
60.125	0.21744	20.1474268896229	20.1474268896229\\
60.125	0.2211	21.2450383201531	21.2450383201531\\
60.125	0.22476	22.3681937380412	22.3681937380412\\
60.125	0.22842	23.5168931432874	23.5168931432874\\
60.125	0.23208	24.6911365358915	24.6911365358915\\
60.125	0.23574	25.8909239158537	25.8909239158537\\
60.125	0.2394	27.1162552831738	27.1162552831738\\
60.125	0.24306	28.3671306378518	28.3671306378518\\
60.125	0.24672	29.6435499798878	29.6435499798878\\
60.125	0.25038	30.9455133092819	30.9455133092819\\
60.125	0.25404	32.2730206260339	32.2730206260339\\
60.125	0.2577	33.6260719301439	33.6260719301439\\
60.125	0.26136	35.0046672216119	35.0046672216119\\
60.125	0.26502	36.4088065004378	36.4088065004378\\
60.125	0.26868	37.8384897666217	37.8384897666217\\
60.125	0.27234	39.2937170201636	39.2937170201636\\
60.125	0.276	40.7744882610635	40.7744882610635\\
60.5	0.093	-2.06741852912145	-2.06741852912145\\
60.5	0.09666	-1.83107420300057	-1.83107420300057\\
60.5	0.10032	-1.5691858895217	-1.5691858895217\\
60.5	0.10398	-1.28175358868486	-1.28175358868486\\
60.5	0.10764	-0.968777300490046	-0.968777300490046\\
60.5	0.1113	-0.630257024937253	-0.630257024937253\\
60.5	0.11496	-0.266192762026474	-0.266192762026474\\
60.5	0.11862	0.123415488242289	0.123415488242289\\
60.5	0.12228	0.538567725869028	0.538567725869028\\
60.5	0.12594	0.979263950853751	0.979263950853751\\
60.5	0.1296	1.44550416319645	1.44550416319645\\
60.5	0.13326	1.93728836289711	1.93728836289711\\
60.5	0.13692	2.45461654995577	2.45461654995577\\
60.5	0.14058	2.99748872437241	2.99748872437241\\
60.5	0.14424	3.56590488614702	3.56590488614702\\
60.5	0.1479	4.15986503527961	4.15986503527961\\
60.5	0.15156	4.77936917177018	4.77936917177018\\
60.5	0.15522	5.42441729561875	5.42441729561875\\
60.5	0.15888	6.09500940682528	6.09500940682528\\
60.5	0.16254	6.7911455053898	6.7911455053898\\
60.5	0.1662	7.51282559131229	7.51282559131229\\
60.5	0.16986	8.26004966459276	8.26004966459276\\
60.5	0.17352	9.0328177252312	9.0328177252312\\
60.5	0.17718	9.83112977322764	9.83112977322764\\
60.5	0.18084	10.6549858085821	10.6549858085821\\
60.5	0.1845	11.5043858312944	11.5043858312944\\
60.5	0.18816	12.3793298413648	12.3793298413648\\
60.5	0.19182	13.2798178387932	13.2798178387932\\
60.5	0.19548	14.2058498235795	14.2058498235795\\
60.5	0.19914	15.1574257957238	15.1574257957238\\
60.5	0.2028	16.1345457552261	16.1345457552261\\
60.5	0.20646	17.1372097020864	17.1372097020864\\
60.5	0.21012	18.1654176363046	18.1654176363046\\
60.5	0.21378	19.2191695578808	19.2191695578808\\
60.5	0.21744	20.298465466815	20.298465466815\\
60.5	0.2211	21.4033053631072	21.4033053631072\\
60.5	0.22476	22.5336892467574	22.5336892467574\\
60.5	0.22842	23.6896171177655	23.6896171177655\\
60.5	0.23208	24.8710889761317	24.8710889761317\\
60.5	0.23574	26.0781048218558	26.0781048218558\\
60.5	0.2394	27.3106646549379	27.3106646549379\\
60.5	0.24306	28.5687684753779	28.5687684753779\\
60.5	0.24672	29.852416283176	29.852416283176\\
60.5	0.25038	31.161608078332	31.161608078332\\
60.5	0.25404	32.496343860846	32.496343860846\\
60.5	0.2577	33.856623630718	33.856623630718\\
60.5	0.26136	35.2424473879479	35.2424473879479\\
60.5	0.26502	36.6538151325359	36.6538151325359\\
60.5	0.26868	38.0907268644818	38.0907268644818\\
60.5	0.27234	39.5531825837857	39.5531825837857\\
60.5	0.276	41.0411822904476	41.0411822904476\\
60.875	0.093	-2.15977911485994	-2.15977911485994\\
60.875	0.09666	-1.91620632297705	-1.91620632297705\\
60.875	0.10032	-1.64708954373619	-1.64708954373619\\
60.875	0.10398	-1.35242877713736	-1.35242877713736\\
60.875	0.10764	-1.03222402318055	-1.03222402318055\\
60.875	0.1113	-0.686475281865762	-0.686475281865762\\
60.875	0.11496	-0.315182553192987	-0.315182553192987\\
60.875	0.11862	0.0816541628377738	0.0816541628377738\\
60.875	0.12228	0.50403486622651	0.50403486622651\\
60.875	0.12594	0.951959556973224	0.951959556973224\\
60.875	0.1296	1.42542823507792	1.42542823507792\\
60.875	0.13326	1.92444090054057	1.92444090054057\\
60.875	0.13692	2.44899755336122	2.44899755336122\\
60.875	0.14058	2.99909819353987	2.99909819353987\\
60.875	0.14424	3.57474282107648	3.57474282107648\\
60.875	0.1479	4.17593143597106	4.17593143597106\\
60.875	0.15156	4.80266403822363	4.80266403822363\\
60.875	0.15522	5.45494062783418	5.45494062783418\\
60.875	0.15888	6.13276120480271	6.13276120480271\\
60.875	0.16254	6.83612576912923	6.83612576912923\\
60.875	0.1662	7.56503432081371	7.56503432081371\\
60.875	0.16986	8.31948685985617	8.31948685985617\\
60.875	0.17352	9.09948338625662	9.09948338625662\\
60.875	0.17718	9.90502390001505	9.90502390001505\\
60.875	0.18084	10.7361084011314	10.7361084011314\\
60.875	0.1845	11.5927368896058	11.5927368896058\\
60.875	0.18816	12.4749093654382	12.4749093654382\\
60.875	0.19182	13.3826258286285	13.3826258286285\\
60.875	0.19548	14.3158862791769	14.3158862791769\\
60.875	0.19914	15.2746907170832	15.2746907170832\\
60.875	0.2028	16.2590391423475	16.2590391423475\\
60.875	0.20646	17.2689315549697	17.2689315549697\\
60.875	0.21012	18.30436795495	18.30436795495\\
60.875	0.21378	19.3653483422882	19.3653483422882\\
60.875	0.21744	20.4518727169844	20.4518727169844\\
60.875	0.2211	21.5639410790386	21.5639410790386\\
60.875	0.22476	22.7015534284507	22.7015534284507\\
60.875	0.22842	23.8647097652209	23.8647097652209\\
60.875	0.23208	25.053410089349	25.053410089349\\
60.875	0.23574	26.2676544008351	26.2676544008351\\
60.875	0.2394	27.5074426996792	27.5074426996792\\
60.875	0.24306	28.7727749858812	28.7727749858812\\
60.875	0.24672	30.0636512594413	30.0636512594413\\
60.875	0.25038	31.3800715203593	31.3800715203593\\
60.875	0.25404	32.7220357686353	32.7220357686353\\
60.875	0.2577	34.0895440042693	34.0895440042693\\
60.875	0.26136	35.4825962272612	35.4825962272612\\
60.875	0.26502	36.9011924376112	36.9011924376112\\
60.875	0.26868	38.3453326353191	38.3453326353191\\
60.875	0.27234	39.815016820385	39.815016820385\\
60.875	0.276	41.3102449928088	41.3102449928088\\
61.25	0.093	-2.24977102762122	-2.24977102762122\\
61.25	0.09666	-1.99896976997634	-1.99896976997634\\
61.25	0.10032	-1.72262452497348	-1.72262452497348\\
61.25	0.10398	-1.42073529261265	-1.42073529261265\\
61.25	0.10764	-1.09330207289386	-1.09330207289386\\
61.25	0.1113	-0.740324865817071	-0.740324865817071\\
61.25	0.11496	-0.361803671382297	-0.361803671382297\\
61.25	0.11862	0.0422615104104462	0.0422615104104462\\
61.25	0.12228	0.47187067956118	0.47187067956118\\
61.25	0.12594	0.927023836069891	0.927023836069891\\
61.25	0.1296	1.40772097993658	1.40772097993658\\
61.25	0.13326	1.91396211116124	1.91396211116124\\
61.25	0.13692	2.44574722974388	2.44574722974388\\
61.25	0.14058	3.00307633568451	3.00307633568451\\
61.25	0.14424	3.58594942898312	3.58594942898312\\
61.25	0.1479	4.19436650963969	4.19436650963969\\
61.25	0.15156	4.82832757765426	4.82832757765426\\
61.25	0.15522	5.48783263302681	5.48783263302681\\
61.25	0.15888	6.17288167575734	6.17288167575734\\
61.25	0.16254	6.88347470584584	6.88347470584584\\
61.25	0.1662	7.61961172329232	7.61961172329232\\
61.25	0.16986	8.38129272809678	8.38129272809678\\
61.25	0.17352	9.16851772025922	9.16851772025922\\
61.25	0.17718	9.98128669977965	9.98128669977965\\
61.25	0.18084	10.819599666658	10.819599666658\\
61.25	0.1845	11.6834566208944	11.6834566208944\\
61.25	0.18816	12.5728575624888	12.5728575624888\\
61.25	0.19182	13.4878024914411	13.4878024914411\\
61.25	0.19548	14.4282914077514	14.4282914077514\\
61.25	0.19914	15.3943243114197	15.3943243114197\\
61.25	0.2028	16.385901202446	16.385901202446\\
61.25	0.20646	17.4030220808303	17.4030220808303\\
61.25	0.21012	18.4456869465725	18.4456869465725\\
61.25	0.21378	19.5138957996727	19.5138957996727\\
61.25	0.21744	20.6076486401309	20.6076486401309\\
61.25	0.2211	21.7269454679471	21.7269454679471\\
61.25	0.22476	22.8717862831213	22.8717862831213\\
61.25	0.22842	24.0421710856534	24.0421710856534\\
61.25	0.23208	25.2380998755435	25.2380998755435\\
61.25	0.23574	26.4595726527916	26.4595726527916\\
61.25	0.2394	27.7065894173977	27.7065894173977\\
61.25	0.24306	28.9791501693617	28.9791501693617\\
61.25	0.24672	30.2772549086838	30.2772549086838\\
61.25	0.25038	31.6009036353638	31.6009036353638\\
61.25	0.25404	32.9500963494018	32.9500963494018\\
61.25	0.2577	34.3248330507978	34.3248330507978\\
61.25	0.26136	35.7251137395517	35.7251137395517\\
61.25	0.26502	37.1509384156636	37.1509384156636\\
61.25	0.26868	38.6023070791336	38.6023070791336\\
61.25	0.27234	40.0792197299614	40.0792197299614\\
61.25	0.276	41.5816763681473	41.5816763681473\\
61.625	0.093	-2.33739426740532	-2.33739426740532\\
61.625	0.09666	-2.07936454399846	-2.07936454399846\\
61.625	0.10032	-1.7957908332336	-1.7957908332336\\
61.625	0.10398	-1.48667313511077	-1.48667313511077\\
61.625	0.10764	-1.15201144962998	-1.15201144962998\\
61.625	0.1113	-0.791805776791199	-0.791805776791199\\
61.625	0.11496	-0.406056116594435	-0.406056116594435\\
61.625	0.11862	0.0052375309603061	0.0052375309603061\\
61.625	0.12228	0.442075165873037	0.442075165873037\\
61.625	0.12594	0.904456788143738	0.904456788143738\\
61.625	0.1296	1.39238239777243	1.39238239777243\\
61.625	0.13326	1.90585199475907	1.90585199475907\\
61.625	0.13692	2.44486557910372	2.44486557910372\\
61.625	0.14058	3.00942315080634	3.00942315080634\\
61.625	0.14424	3.59952470986694	3.59952470986694\\
61.625	0.1479	4.21517025628551	4.21517025628551\\
61.625	0.15156	4.85635979006208	4.85635979006208\\
61.625	0.15522	5.52309331119661	5.52309331119661\\
61.625	0.15888	6.21537081968914	6.21537081968914\\
61.625	0.16254	6.93319231553964	6.93319231553964\\
61.625	0.1662	7.67655779874812	7.67655779874812\\
61.625	0.16986	8.44546726931457	8.44546726931457\\
61.625	0.17352	9.239920727239	9.239920727239\\
61.625	0.17718	10.0599181725214	10.0599181725214\\
61.625	0.18084	10.9054596051618	10.9054596051618\\
61.625	0.1845	11.7765450251602	11.7765450251602\\
61.625	0.18816	12.6731744325165	12.6731744325165\\
61.625	0.19182	13.5953478272309	13.5953478272309\\
61.625	0.19548	14.5430652093032	14.5430652093032\\
61.625	0.19914	15.5163265787335	15.5163265787335\\
61.625	0.2028	16.5151319355218	16.5151319355218\\
61.625	0.20646	17.539481279668	17.539481279668\\
61.625	0.21012	18.5893746111723	18.5893746111723\\
61.625	0.21378	19.6648119300345	19.6648119300345\\
61.625	0.21744	20.7657932362547	20.7657932362547\\
61.625	0.2211	21.8923185298328	21.8923185298328\\
61.625	0.22476	23.044387810769	23.044387810769\\
61.625	0.22842	24.2220010790631	24.2220010790631\\
61.625	0.23208	25.4251583347152	25.4251583347152\\
61.625	0.23574	26.6538595777253	26.6538595777253\\
61.625	0.2394	27.9081048080934	27.9081048080934\\
61.625	0.24306	29.1878940258194	29.1878940258194\\
61.625	0.24672	30.4932272309035	30.4932272309035\\
61.625	0.25038	31.8241044233455	31.8241044233455\\
61.625	0.25404	33.1805256031455	33.1805256031455\\
61.625	0.2577	34.5624907703034	34.5624907703034\\
61.625	0.26136	35.9699999248194	35.9699999248194\\
61.625	0.26502	37.4030530666933	37.4030530666933\\
61.625	0.26868	38.8616501959252	38.8616501959252\\
61.625	0.27234	40.3457913125151	40.3457913125151\\
61.625	0.276	41.8554764164629	41.8554764164629\\
62	0.093	-2.42264883421223	-2.42264883421223\\
62	0.09666	-2.15739064504337	-2.15739064504337\\
62	0.10032	-1.86658846851652	-1.86658846851652\\
62	0.10398	-1.5502423046317	-1.5502423046317\\
62	0.10764	-1.20835215338891	-1.20835215338891\\
62	0.1113	-0.840918014788135	-0.840918014788135\\
62	0.11496	-0.447939888829374	-0.447939888829374\\
62	0.11862	-0.0294177755126359	-0.0294177755126359\\
62	0.12228	0.414648325162085	0.414648325162085\\
62	0.12594	0.884258413194784	0.884258413194784\\
62	0.1296	1.37941248858547	1.37941248858547\\
62	0.13326	1.90011055133411	1.90011055133411\\
62	0.13692	2.44635260144075	2.44635260144075\\
62	0.14058	3.01813863890537	3.01813863890537\\
62	0.14424	3.61546866372796	3.61546866372796\\
62	0.1479	4.23834267590853	4.23834267590853\\
62	0.15156	4.88676067544708	4.88676067544708\\
62	0.15522	5.56072266234363	5.56072266234363\\
62	0.15888	6.26022863659814	6.26022863659814\\
62	0.16254	6.98527859821064	6.98527859821064\\
62	0.1662	7.73587254718111	7.73587254718111\\
62	0.16986	8.51201048350955	8.51201048350955\\
62	0.17352	9.31369240719598	9.31369240719598\\
62	0.17718	10.1409183182404	10.1409183182404\\
62	0.18084	10.9936882166428	10.9936882166428\\
62	0.1845	11.8720021024032	11.8720021024032\\
62	0.18816	12.7758599755215	12.7758599755215\\
62	0.19182	13.7052618359978	13.7052618359978\\
62	0.19548	14.6602076838321	14.6602076838321\\
62	0.19914	15.6406975190244	15.6406975190244\\
62	0.2028	16.6467313415747	16.6467313415747\\
62	0.20646	17.678309151483	17.678309151483\\
62	0.21012	18.7354309487492	18.7354309487492\\
62	0.21378	19.8180967333734	19.8180967333734\\
62	0.21744	20.9263065053556	20.9263065053556\\
62	0.2211	22.0600602646957	22.0600602646957\\
62	0.22476	23.2193580113939	23.2193580113939\\
62	0.22842	24.40419974545	24.40419974545\\
62	0.23208	25.6145854668641	25.6145854668641\\
62	0.23574	26.8505151756362	26.8505151756362\\
62	0.2394	28.1119888717663	28.1119888717663\\
62	0.24306	29.3990065552543	29.3990065552543\\
62	0.24672	30.7115682261003	30.7115682261003\\
62	0.25038	32.0496738843044	32.0496738843044\\
62	0.25404	33.4133235298663	33.4133235298663\\
62	0.2577	34.8025171627863	34.8025171627863\\
62	0.26136	36.2172547830642	36.2172547830642\\
62	0.26502	37.6575363907002	37.6575363907002\\
62	0.26868	39.1233619856941	39.1233619856941\\
62	0.27234	40.6147315680459	40.6147315680459\\
62	0.276	42.1316451377558	42.1316451377558\\
62.375	0.093	-2.50553472804196	-2.50553472804196\\
62.375	0.09666	-2.2330480731111	-2.2330480731111\\
62.375	0.10032	-1.93501743082229	-1.93501743082229\\
62.375	0.10398	-1.61144280117547	-1.61144280117547\\
62.375	0.10764	-1.26232418417068	-1.26232418417068\\
62.375	0.1113	-0.887661579807888	-0.887661579807888\\
62.375	0.11496	-0.48745498808713	-0.48745498808713\\
62.375	0.11862	-0.0617044090084153	-0.0617044090084153\\
62.375	0.12228	0.389590157428303	0.389590157428303\\
62.375	0.12594	0.866428711222985	0.866428711222985\\
62.375	0.1296	1.36881125237565	1.36881125237565\\
62.375	0.13326	1.89673778088631	1.89673778088631\\
62.375	0.13692	2.45020829675494	2.45020829675494\\
62.375	0.14058	3.02922279998156	3.02922279998156\\
62.375	0.14424	3.63378129056614	3.63378129056614\\
62.375	0.1479	4.26388376850871	4.26388376850871\\
62.375	0.15156	4.91953023380926	4.91953023380926\\
62.375	0.15522	5.60072068646779	5.60072068646779\\
62.375	0.15888	6.30745512648431	6.30745512648431\\
62.375	0.16254	7.0397335538588	7.0397335538588\\
62.375	0.1662	7.79755596859126	7.79755596859126\\
62.375	0.16986	8.5809223706817	8.5809223706817\\
62.375	0.17352	9.38983276013013	9.38983276013013\\
62.375	0.17718	10.2242871369365	10.2242871369365\\
62.375	0.18084	11.0842855011009	11.0842855011009\\
62.375	0.1845	11.9698278526233	11.9698278526233\\
62.375	0.18816	12.8809141915036	12.8809141915036\\
62.375	0.19182	13.817544517742	13.817544517742\\
62.375	0.19548	14.7797188313383	14.7797188313383\\
62.375	0.19914	15.7674371322926	15.7674371322926\\
62.375	0.2028	16.7806994206048	16.7806994206048\\
62.375	0.20646	17.8195056962751	17.8195056962751\\
62.375	0.21012	18.8838559593033	18.8838559593033\\
62.375	0.21378	19.9737502096895	19.9737502096895\\
62.375	0.21744	21.0891884474337	21.0891884474337\\
62.375	0.2211	22.2301706725358	22.2301706725358\\
62.375	0.22476	23.3966968849959	23.3966968849959\\
62.375	0.22842	24.5887670848141	24.5887670848141\\
62.375	0.23208	25.8063812719902	25.8063812719902\\
62.375	0.23574	27.0495394465243	27.0495394465243\\
62.375	0.2394	28.3182416084163	28.3182416084163\\
62.375	0.24306	29.6124877576663	29.6124877576663\\
62.375	0.24672	30.9322778942744	30.9322778942744\\
62.375	0.25038	32.2776120182404	32.2776120182404\\
62.375	0.25404	33.6484901295644	33.6484901295644\\
62.375	0.2577	35.0449122282463	35.0449122282463\\
62.375	0.26136	36.4668783142862	36.4668783142862\\
62.375	0.26502	37.9143883876842	37.9143883876842\\
62.375	0.26868	39.3874424484401	39.3874424484401\\
62.375	0.27234	40.886040496554	40.886040496554\\
62.375	0.276	42.4101825320258	42.4101825320258\\
62.75	0.093	-2.58605194889449	-2.58605194889449\\
62.75	0.09666	-2.30633682820164	-2.30633682820164\\
62.75	0.10032	-2.00107772015083	-2.00107772015083\\
62.75	0.10398	-1.67027462474202	-1.67027462474202\\
62.75	0.10764	-1.31392754197523	-1.31392754197523\\
62.75	0.1113	-0.93203647185045	-0.93203647185045\\
62.75	0.11496	-0.524601414367694	-0.524601414367694\\
62.75	0.11862	-0.0916223695269824	-0.0916223695269824\\
62.75	0.12228	0.366900662671734	0.366900662671734\\
62.75	0.12594	0.850967682228406	0.850967682228406\\
62.75	0.1296	1.36057868914307	1.36057868914307\\
62.75	0.13326	1.89573368341572	1.89573368341572\\
62.75	0.13692	2.45643266504635	2.45643266504635\\
62.75	0.14058	3.04267563403496	3.04267563403496\\
62.75	0.14424	3.65446259038153	3.65446259038153\\
62.75	0.1479	4.2917935340861	4.2917935340861\\
62.75	0.15156	4.95466846514865	4.95466846514865\\
62.75	0.15522	5.64308738356917	5.64308738356917\\
62.75	0.15888	6.35705028934768	6.35705028934768\\
62.75	0.16254	7.09655718248417	7.09655718248417\\
62.75	0.1662	7.86160806297863	7.86160806297863\\
62.75	0.16986	8.65220293083106	8.65220293083106\\
62.75	0.17352	9.46834178604149	9.46834178604149\\
62.75	0.17718	10.3100246286099	10.3100246286099\\
62.75	0.18084	11.1772514585363	11.1772514585363\\
62.75	0.1845	12.0700222758206	12.0700222758206\\
62.75	0.18816	12.988337080463	12.988337080463\\
62.75	0.19182	13.9321958724633	13.9321958724633\\
62.75	0.19548	14.9015986518216	14.9015986518216\\
62.75	0.19914	15.8965454185379	15.8965454185379\\
62.75	0.2028	16.9170361726121	16.9170361726121\\
62.75	0.20646	17.9630709140444	17.9630709140444\\
62.75	0.21012	19.0346496428346	19.0346496428346\\
62.75	0.21378	20.1317723589828	20.1317723589828\\
62.75	0.21744	21.254439062489	21.254439062489\\
62.75	0.2211	22.4026497533531	22.4026497533531\\
62.75	0.22476	23.5764044315753	23.5764044315753\\
62.75	0.22842	24.7757030971554	24.7757030971554\\
62.75	0.23208	26.0005457500935	26.0005457500935\\
62.75	0.23574	27.2509323903895	27.2509323903895\\
62.75	0.2394	28.5268630180436	28.5268630180436\\
62.75	0.24306	29.8283376330556	29.8283376330556\\
62.75	0.24672	31.1553562354256	31.1553562354256\\
62.75	0.25038	32.5079188251537	32.5079188251537\\
62.75	0.25404	33.8860254022396	33.8860254022396\\
62.75	0.2577	35.2896759666836	35.2896759666836\\
62.75	0.26136	36.7188705184855	36.7188705184855\\
62.75	0.26502	38.1736090576454	38.1736090576454\\
62.75	0.26868	39.6538915841633	39.6538915841633\\
62.75	0.27234	41.1597180980392	41.1597180980392\\
62.75	0.276	42.691088599273	42.691088599273\\
63.125	0.093	-2.66420049676984	-2.66420049676984\\
63.125	0.09666	-2.37725691031499	-2.37725691031499\\
63.125	0.10032	-2.06476933650219	-2.06476933650219\\
63.125	0.10398	-1.72673777533138	-1.72673777533138\\
63.125	0.10764	-1.3631622268026	-1.3631622268026\\
63.125	0.1113	-0.974042690915821	-0.974042690915821\\
63.125	0.11496	-0.559379167671075	-0.559379167671075\\
63.125	0.11862	-0.119171657068366	-0.119171657068366\\
63.125	0.12228	0.346579840892341	0.346579840892341\\
63.125	0.12594	0.837875326211011	0.837875326211011\\
63.125	0.1296	1.35471479888767	1.35471479888767\\
63.125	0.13326	1.89709825892232	1.89709825892232\\
63.125	0.13692	2.46502570631494	2.46502570631494\\
63.125	0.14058	3.05849714106555	3.05849714106555\\
63.125	0.14424	3.67751256317412	3.67751256317412\\
63.125	0.1479	4.32207197264068	4.32207197264068\\
63.125	0.15156	4.99217536946522	4.99217536946522\\
63.125	0.15522	5.68782275364774	5.68782275364774\\
63.125	0.15888	6.40901412518824	6.40901412518824\\
63.125	0.16254	7.15574948408672	7.15574948408672\\
63.125	0.1662	7.92802883034318	7.92802883034318\\
63.125	0.16986	8.72585216395761	8.72585216395761\\
63.125	0.17352	9.54921948493002	9.54921948493002\\
63.125	0.17718	10.3981307932604	10.3981307932604\\
63.125	0.18084	11.2725860889488	11.2725860889488\\
63.125	0.1845	12.1725853719951	12.1725853719951\\
63.125	0.18816	13.0981286423995	13.0981286423995\\
63.125	0.19182	14.0492159001618	14.0492159001618\\
63.125	0.19548	15.0258471452821	15.0258471452821\\
63.125	0.19914	16.0280223777604	16.0280223777604\\
63.125	0.2028	17.0557415975966	17.0557415975966\\
63.125	0.20646	18.1090048047909	18.1090048047909\\
63.125	0.21012	19.1878119993431	19.1878119993431\\
63.125	0.21378	20.2921631812533	20.2921631812533\\
63.125	0.21744	21.4220583505214	21.4220583505214\\
63.125	0.2211	22.5774975071476	22.5774975071476\\
63.125	0.22476	23.7584806511317	23.7584806511317\\
63.125	0.22842	24.9650077824738	24.9650077824738\\
63.125	0.23208	26.1970789011739	26.1970789011739\\
63.125	0.23574	27.454694007232	27.454694007232\\
63.125	0.2394	28.737853100648	28.737853100648\\
63.125	0.24306	30.0465561814221	30.0465561814221\\
63.125	0.24672	31.3808032495541	31.3808032495541\\
63.125	0.25038	32.7405943050441	32.7405943050441\\
63.125	0.25404	34.125929347892	34.125929347892\\
63.125	0.2577	35.536808378098	35.536808378098\\
63.125	0.26136	36.9732313956619	36.9732313956619\\
63.125	0.26502	38.4351984005838	38.4351984005838\\
63.125	0.26868	39.9227093928637	39.9227093928637\\
63.125	0.27234	41.4357643725016	41.4357643725016\\
63.125	0.276	42.9743633394974	42.9743633394974\\
63.5	0.093	-2.73998037166801	-2.73998037166801\\
63.5	0.09666	-2.44580831945116	-2.44580831945116\\
63.5	0.10032	-2.12609227987636	-2.12609227987636\\
63.5	0.10398	-1.78083225294356	-1.78083225294356\\
63.5	0.10764	-1.41002823865278	-1.41002823865278\\
63.5	0.1113	-1.01368023700401	-1.01368023700401\\
63.5	0.11496	-0.591788247997272	-0.591788247997272\\
63.5	0.11862	-0.144352271632565	-0.144352271632565\\
63.5	0.12228	0.328627692090139	0.328627692090139\\
63.5	0.12594	0.827151643170799	0.827151643170799\\
63.5	0.1296	1.35121958160946	1.35121958160946\\
63.5	0.13326	1.9008315074061	1.9008315074061\\
63.5	0.13692	2.47598742056071	2.47598742056071\\
63.5	0.14058	3.07668732107331	3.07668732107331\\
63.5	0.14424	3.70293120894388	3.70293120894388\\
63.5	0.1479	4.35471908417244	4.35471908417244\\
63.5	0.15156	5.03205094675897	5.03205094675897\\
63.5	0.15522	5.73492679670349	5.73492679670349\\
63.5	0.15888	6.46334663400599	6.46334663400599\\
63.5	0.16254	7.21731045866646	7.21731045866646\\
63.5	0.1662	7.99681827068491	7.99681827068491\\
63.5	0.16986	8.80187007006134	8.80187007006134\\
63.5	0.17352	9.63246585679575	9.63246585679575\\
63.5	0.17718	10.4886056308881	10.4886056308881\\
63.5	0.18084	11.3702893923385	11.3702893923385\\
63.5	0.1845	12.2775171411469	12.2775171411469\\
63.5	0.18816	13.2102888773132	13.2102888773132\\
63.5	0.19182	14.1686046008375	14.1686046008375\\
63.5	0.19548	15.1524643117198	15.1524643117198\\
63.5	0.19914	16.1618680099601	16.1618680099601\\
63.5	0.2028	17.1968156955583	17.1968156955583\\
63.5	0.20646	18.2573073685145	18.2573073685145\\
63.5	0.21012	19.3433430288287	19.3433430288287\\
63.5	0.21378	20.4549226765009	20.4549226765009\\
63.5	0.21744	21.5920463115311	21.5920463115311\\
63.5	0.2211	22.7547139339192	22.7547139339192\\
63.5	0.22476	23.9429255436654	23.9429255436654\\
63.5	0.22842	25.1566811407695	25.1566811407695\\
63.5	0.23208	26.3959807252316	26.3959807252316\\
63.5	0.23574	27.6608242970516	27.6608242970516\\
63.5	0.2394	28.9512118562297	28.9512118562297\\
63.5	0.24306	30.2671434027657	30.2671434027657\\
63.5	0.24672	31.6086189366597	31.6086189366597\\
63.5	0.25038	32.9756384579117	32.9756384579117\\
63.5	0.25404	34.3682019665217	34.3682019665217\\
63.5	0.2577	35.7863094624896	35.7863094624896\\
63.5	0.26136	37.2299609458155	37.2299609458155\\
63.5	0.26502	38.6991564164994	38.6991564164994\\
63.5	0.26868	40.1938958745413	40.1938958745413\\
63.5	0.27234	41.7141793199411	41.7141793199411\\
63.5	0.276	43.260006752699	43.260006752699\\
63.875	0.093	-2.81339157358897	-2.81339157358897\\
63.875	0.09666	-2.51199105561013	-2.51199105561013\\
63.875	0.10032	-2.18504655027334	-2.18504655027334\\
63.875	0.10398	-1.83255805757854	-1.83255805757854\\
63.875	0.10764	-1.45452557752577	-1.45452557752577\\
63.875	0.1113	-1.050949110115	-1.050949110115\\
63.875	0.11496	-0.621828655346263	-0.621828655346263\\
63.875	0.11862	-0.167164213219566	-0.167164213219566\\
63.875	0.12228	0.313044216265128	0.313044216265128\\
63.875	0.12594	0.818796633107793	0.818796633107793\\
63.875	0.1296	1.35009303730844	1.35009303730844\\
63.875	0.13326	1.90693342886707	1.90693342886707\\
63.875	0.13692	2.48931780778369	2.48931780778369\\
63.875	0.14058	3.09724617405828	3.09724617405828\\
63.875	0.14424	3.73071852769084	3.73071852769084\\
63.875	0.1479	4.38973486868139	4.38973486868139\\
63.875	0.15156	5.07429519702992	5.07429519702992\\
63.875	0.15522	5.78439951273644	5.78439951273644\\
63.875	0.15888	6.52004781580093	6.52004781580093\\
63.875	0.16254	7.2812401062234	7.2812401062234\\
63.875	0.1662	8.06797638400385	8.06797638400385\\
63.875	0.16986	8.88025664914226	8.88025664914226\\
63.875	0.17352	9.71808090163868	9.71808090163868\\
63.875	0.17718	10.5814491414931	10.5814491414931\\
63.875	0.18084	11.4703613687054	11.4703613687054\\
63.875	0.1845	12.3848175832758	12.3848175832758\\
63.875	0.18816	13.3248177852041	13.3248177852041\\
63.875	0.19182	14.2903619744904	14.2903619744904\\
63.875	0.19548	15.2814501511347	15.2814501511347\\
63.875	0.19914	16.298082315137	16.298082315137\\
63.875	0.2028	17.3402584664972	17.3402584664972\\
63.875	0.20646	18.4079786052154	18.4079786052154\\
63.875	0.21012	19.5012427312916	19.5012427312916\\
63.875	0.21378	20.6200508447258	20.6200508447258\\
63.875	0.21744	21.764402945518	21.764402945518\\
63.875	0.2211	22.9342990336681	22.9342990336681\\
63.875	0.22476	24.1297391091762	24.1297391091762\\
63.875	0.22842	25.3507231720423	25.3507231720423\\
63.875	0.23208	26.5972512222664	26.5972512222664\\
63.875	0.23574	27.8693232598485	27.8693232598485\\
63.875	0.2394	29.1669392847885	29.1669392847885\\
63.875	0.24306	30.4900992970865	30.4900992970865\\
63.875	0.24672	31.8388032967425	31.8388032967425\\
63.875	0.25038	33.2130512837565	33.2130512837565\\
63.875	0.25404	34.6128432581285	34.6128432581285\\
63.875	0.2577	36.0381792198584	36.0381792198584\\
63.875	0.26136	37.4890591689463	37.4890591689463\\
63.875	0.26502	38.9654831053922	38.9654831053922\\
63.875	0.26868	40.4674510291961	40.4674510291961\\
63.875	0.27234	41.994962940358	41.994962940358\\
63.875	0.276	43.5480188388778	43.5480188388778\\
64.25	0.093	-2.88443410253276	-2.88443410253276\\
64.25	0.09666	-2.57580511879193	-2.57580511879193\\
64.25	0.10032	-2.24163214769314	-2.24163214769314\\
64.25	0.10398	-1.88191518923635	-1.88191518923635\\
64.25	0.10764	-1.49665424342158	-1.49665424342158\\
64.25	0.1113	-1.08584931024882	-1.08584931024882\\
64.25	0.11496	-0.649500389718085	-0.649500389718085\\
64.25	0.11862	-0.18760748182939	-0.18760748182939\\
64.25	0.12228	0.299829413417301	0.299829413417301\\
64.25	0.12594	0.812810296021956	0.812810296021956\\
64.25	0.1296	1.3513351659846	1.3513351659846\\
64.25	0.13326	1.91540402330523	1.91540402330523\\
64.25	0.13692	2.50501686798384	2.50501686798384\\
64.25	0.14058	3.12017370002043	3.12017370002043\\
64.25	0.14424	3.76087451941498	3.76087451941498\\
64.25	0.1479	4.42711932616753	4.42711932616753\\
64.25	0.15156	5.11890812027805	5.11890812027805\\
64.25	0.15522	5.83624090174656	5.83624090174656\\
64.25	0.15888	6.57911767057305	6.57911767057305\\
64.25	0.16254	7.34753842675751	7.34753842675751\\
64.25	0.1662	8.14150317029996	8.14150317029996\\
64.25	0.16986	8.96101190120037	8.96101190120037\\
64.25	0.17352	9.80606461945877	9.80606461945877\\
64.25	0.17718	10.6766613250752	10.6766613250752\\
64.25	0.18084	11.5728020180495	11.5728020180495\\
64.25	0.1845	12.4944866983818	12.4944866983818\\
64.25	0.18816	13.4417153660722	13.4417153660722\\
64.25	0.19182	14.4144880211205	14.4144880211205\\
64.25	0.19548	15.4128046635268	15.4128046635268\\
64.25	0.19914	16.436665293291	16.436665293291\\
64.25	0.2028	17.4860699104133	17.4860699104133\\
64.25	0.20646	18.5610185148935	18.5610185148935\\
64.25	0.21012	19.6615111067317	19.6615111067317\\
64.25	0.21378	20.7875476859278	20.7875476859278\\
64.25	0.21744	21.939128252482	21.939128252482\\
64.25	0.2211	23.1162528063941	23.1162528063941\\
64.25	0.22476	24.3189213476643	24.3189213476643\\
64.25	0.22842	25.5471338762923	25.5471338762923\\
64.25	0.23208	26.8008903922784	26.8008903922784\\
64.25	0.23574	28.0801908956225	28.0801908956225\\
64.25	0.2394	29.3850353863245	29.3850353863245\\
64.25	0.24306	30.7154238643845	30.7154238643845\\
64.25	0.24672	32.0713563298025	32.0713563298025\\
64.25	0.25038	33.4528327825785	33.4528327825785\\
64.25	0.25404	34.8598532227124	34.8598532227124\\
64.25	0.2577	36.2924176502044	36.2924176502044\\
64.25	0.26136	37.7505260650543	37.7505260650543\\
64.25	0.26502	39.2341784672622	39.2341784672622\\
64.25	0.26868	40.7433748568281	40.7433748568281\\
64.25	0.27234	42.2781152337519	42.2781152337519\\
64.25	0.276	43.8383995980337	43.8383995980337\\
64.625	0.093	-2.95310795849938	-2.95310795849938\\
64.625	0.09666	-2.63725050899655	-2.63725050899655\\
64.625	0.10032	-2.29584907213574	-2.29584907213574\\
64.625	0.10398	-1.92890364791695	-1.92890364791695\\
64.625	0.10764	-1.5364142363402	-1.5364142363402\\
64.625	0.1113	-1.11838083740547	-1.11838083740547\\
64.625	0.11496	-0.674803451112737	-0.674803451112737\\
64.625	0.11862	-0.205682077462031	-0.205682077462031\\
64.625	0.12228	0.288983283546651	0.288983283546651\\
64.625	0.12594	0.809192631913318	0.809192631913318\\
64.625	0.1296	1.35494596763796	1.35494596763796\\
64.625	0.13326	1.92624329072057	1.92624329072057\\
64.625	0.13692	2.52308460116117	2.52308460116117\\
64.625	0.14058	3.14546989895976	3.14546989895976\\
64.625	0.14424	3.79339918411632	3.79339918411632\\
64.625	0.1479	4.46687245663085	4.46687245663085\\
64.625	0.15156	5.16588971650337	5.16588971650337\\
64.625	0.15522	5.89045096373387	5.89045096373387\\
64.625	0.15888	6.64055619832235	6.64055619832235\\
64.625	0.16254	7.41620542026881	7.41620542026881\\
64.625	0.1662	8.21739862957325	8.21739862957325\\
64.625	0.16986	9.04413582623566	9.04413582623566\\
64.625	0.17352	9.89641701025606	9.89641701025606\\
64.625	0.17718	10.7742421816344	10.7742421816344\\
64.625	0.18084	11.6776113403708	11.6776113403708\\
64.625	0.1845	12.6065244864651	12.6065244864651\\
64.625	0.18816	13.5609816199174	13.5609816199174\\
64.625	0.19182	14.5409827407277	14.5409827407277\\
64.625	0.19548	15.546527848896	15.546527848896\\
64.625	0.19914	16.5776169444223	16.5776169444223\\
64.625	0.2028	17.6342500273065	17.6342500273065\\
64.625	0.20646	18.7164270975487	18.7164270975487\\
64.625	0.21012	19.8241481551489	19.8241481551489\\
64.625	0.21378	20.9574132001071	20.9574132001071\\
64.625	0.21744	22.1162222324232	22.1162222324232\\
64.625	0.2211	23.3005752520973	23.3005752520973\\
64.625	0.22476	24.5104722591295	24.5104722591295\\
64.625	0.22842	25.7459132535196	25.7459132535196\\
64.625	0.23208	27.0068982352676	27.0068982352676\\
64.625	0.23574	28.2934272043737	28.2934272043737\\
64.625	0.2394	29.6055001608377	29.6055001608377\\
64.625	0.24306	30.9431171046597	30.9431171046597\\
64.625	0.24672	32.3062780358397	32.3062780358397\\
64.625	0.25038	33.6949829543777	33.6949829543777\\
64.625	0.25404	35.1092318602736	35.1092318602736\\
64.625	0.2577	36.5490247535276	36.5490247535276\\
64.625	0.26136	38.0143616341395	38.0143616341395\\
64.625	0.26502	39.5052425021093	39.5052425021093\\
64.625	0.26868	41.0216673574372	41.0216673574372\\
64.625	0.27234	42.563636200123	42.563636200123\\
64.625	0.276	44.1311490301669	44.1311490301669\\
65	0.093	-3.01941314148879	-3.01941314148879\\
65	0.09666	-2.69632722622396	-2.69632722622396\\
65	0.10032	-2.34769732360115	-2.34769732360115\\
65	0.10398	-1.97352343362038	-1.97352343362038\\
65	0.10764	-1.57380555628163	-1.57380555628163\\
65	0.1113	-1.1485436915849	-1.1485436915849\\
65	0.11496	-0.697737839530177	-0.697737839530177\\
65	0.11862	-0.221388000117473	-0.221388000117473\\
65	0.12228	0.280505826653206	0.280505826653206\\
65	0.12594	0.807943640781863	0.807943640781863\\
65	0.1296	1.36092544226851	1.36092544226851\\
65	0.13326	1.93945123111311	1.93945123111311\\
65	0.13692	2.5435210073157	2.5435210073157\\
65	0.14058	3.17313477087628	3.17313477087628\\
65	0.14424	3.82829252179484	3.82829252179484\\
65	0.1479	4.50899426007136	4.50899426007136\\
65	0.15156	5.21523998570588	5.21523998570588\\
65	0.15522	5.94702969869838	5.94702969869838\\
65	0.15888	6.70436339904885	6.70436339904885\\
65	0.16254	7.48724108675731	7.48724108675731\\
65	0.1662	8.29566276182374	8.29566276182374\\
65	0.16986	9.12962842424814	9.12962842424814\\
65	0.17352	9.98913807403054	9.98913807403054\\
65	0.17718	10.8741917111709	10.8741917111709\\
65	0.18084	11.7847893356693	11.7847893356693\\
65	0.1845	12.7209309475256	12.7209309475256\\
65	0.18816	13.6826165467399	13.6826165467399\\
65	0.19182	14.6698461333122	14.6698461333122\\
65	0.19548	15.6826197072425	15.6826197072425\\
65	0.19914	16.7209372685307	16.7209372685307\\
65	0.2028	17.7847988171769	17.7847988171769\\
65	0.20646	18.8742043531811	18.8742043531811\\
65	0.21012	19.9891538765433	19.9891538765433\\
65	0.21378	21.1296473872635	21.1296473872635\\
65	0.21744	22.2956848853417	22.2956848853417\\
65	0.2211	23.4872663707778	23.4872663707778\\
65	0.22476	24.7043918435719	24.7043918435719\\
65	0.22842	25.947061303724	25.947061303724\\
65	0.23208	27.215274751234	27.215274751234\\
65	0.23574	28.5090321861021	28.5090321861021\\
65	0.2394	29.8283336083281	29.8283336083281\\
65	0.24306	31.1731790179121	31.1731790179121\\
65	0.24672	32.5435684148541	32.5435684148541\\
65	0.25038	33.939501799154	33.939501799154\\
65	0.25404	35.360979170812	35.360979170812\\
65	0.2577	36.8080005298279	36.8080005298279\\
65	0.26136	38.2805658762018	38.2805658762018\\
65	0.26502	39.7786752099337	39.7786752099337\\
65	0.26868	41.3023285310235	41.3023285310235\\
65	0.27234	42.8515258394714	42.8515258394714\\
65	0.276	44.4262671352772	44.4262671352772\\
65.375	0.093	-3.083349651501	-3.083349651501\\
65.375	0.09666	-2.75303527047419	-2.75303527047419\\
65.375	0.10032	-2.39717690208939	-2.39717690208939\\
65.375	0.10398	-2.01577454634661	-2.01577454634661\\
65.375	0.10764	-1.60882820324587	-1.60882820324587\\
65.375	0.1113	-1.17633787278714	-1.17633787278714\\
65.375	0.11496	-0.718303554970419	-0.718303554970419\\
65.375	0.11862	-0.234725249795725	-0.234725249795725\\
65.375	0.12228	0.274397042736952	0.274397042736952\\
65.375	0.12594	0.809063322627599	0.809063322627599\\
65.375	0.1296	1.36927358987623	1.36927358987623\\
65.375	0.13326	1.95502784448283	1.95502784448283\\
65.375	0.13692	2.56632608644742	2.56632608644742\\
65.375	0.14058	3.20316831577	3.20316831577\\
65.375	0.14424	3.86555453245056	3.86555453245056\\
65.375	0.1479	4.55348473648907	4.55348473648907\\
65.375	0.15156	5.26695892788558	5.26695892788558\\
65.375	0.15522	6.00597710664007	6.00597710664007\\
65.375	0.15888	6.77053927275255	6.77053927275255\\
65.375	0.16254	7.560645426223	7.560645426223\\
65.375	0.1662	8.37629556705143	8.37629556705143\\
65.375	0.16986	9.21748969523782	9.21748969523782\\
65.375	0.17352	10.0842278107822	10.0842278107822\\
65.375	0.17718	10.9765099136846	10.9765099136846\\
65.375	0.18084	11.8943360039449	11.8943360039449\\
65.375	0.1845	12.8377060815632	12.8377060815632\\
65.375	0.18816	13.8066201465395	13.8066201465395\\
65.375	0.19182	14.8010781988738	14.8010781988738\\
65.375	0.19548	15.8210802385661	15.8210802385661\\
65.375	0.19914	16.8666262656164	16.8666262656164\\
65.375	0.2028	17.9377162800246	17.9377162800246\\
65.375	0.20646	19.0343502817908	19.0343502817908\\
65.375	0.21012	20.1565282709149	20.1565282709149\\
65.375	0.21378	21.3042502473971	21.3042502473971\\
65.375	0.21744	22.4775162112373	22.4775162112373\\
65.375	0.2211	23.6763261624354	23.6763261624354\\
65.375	0.22476	24.9006801009915	24.9006801009915\\
65.375	0.22842	26.1505780269056	26.1505780269056\\
65.375	0.23208	27.4260199401776	27.4260199401776\\
65.375	0.23574	28.7270058408077	28.7270058408077\\
65.375	0.2394	30.0535357287957	30.0535357287957\\
65.375	0.24306	31.4056096041417	31.4056096041417\\
65.375	0.24672	32.7832274668457	32.7832274668457\\
65.375	0.25038	34.1863893169076	34.1863893169076\\
65.375	0.25404	35.6150951543275	35.6150951543275\\
65.375	0.2577	37.0693449791055	37.0693449791055\\
65.375	0.26136	38.5491387912414	38.5491387912414\\
65.375	0.26502	40.0544765907352	40.0544765907352\\
65.375	0.26868	41.5853583775871	41.5853583775871\\
65.375	0.27234	43.1417841517969	43.1417841517969\\
65.375	0.276	44.7237539133648	44.7237539133648\\
65.75	0.093	-3.14491748853605	-3.14491748853605\\
65.75	0.09666	-2.80737464174723	-2.80737464174723\\
65.75	0.10032	-2.44428780760043	-2.44428780760043\\
65.75	0.10398	-2.05565698609566	-2.05565698609566\\
65.75	0.10764	-1.64148217723293	-1.64148217723293\\
65.75	0.1113	-1.2017633810122	-1.2017633810122\\
65.75	0.11496	-0.736500597433491	-0.736500597433491\\
65.75	0.11862	-0.245693826496799	-0.245693826496799\\
65.75	0.12228	0.270656931797868	0.270656931797868\\
65.75	0.12594	0.81255167745052	0.81255167745052\\
65.75	0.1296	1.37999041046114	1.37999041046114\\
65.75	0.13326	1.97297313082974	1.97297313082974\\
65.75	0.13692	2.59149983855632	2.59149983855632\\
65.75	0.14058	3.23557053364089	3.23557053364089\\
65.75	0.14424	3.90518521608344	3.90518521608344\\
65.75	0.1479	4.60034388588396	4.60034388588396\\
65.75	0.15156	5.32104654304246	5.32104654304246\\
65.75	0.15522	6.06729318755895	6.06729318755895\\
65.75	0.15888	6.83908381943342	6.83908381943342\\
65.75	0.16254	7.63641843866586	7.63641843866586\\
65.75	0.1662	8.45929704525629	8.45929704525629\\
65.75	0.16986	9.30771963920467	9.30771963920467\\
65.75	0.17352	10.1816862205111	10.1816862205111\\
65.75	0.17718	11.0811967891754	11.0811967891754\\
65.75	0.18084	12.0062513451978	12.0062513451978\\
65.75	0.1845	12.9568498885781	12.9568498885781\\
65.75	0.18816	13.9329924193164	13.9329924193164\\
65.75	0.19182	14.9346789374127	14.9346789374127\\
65.75	0.19548	15.9619094428669	15.9619094428669\\
65.75	0.19914	17.0146839356792	17.0146839356792\\
65.75	0.2028	18.0930024158494	18.0930024158494\\
65.75	0.20646	19.1968648833776	19.1968648833776\\
65.75	0.21012	20.3262713382638	20.3262713382638\\
65.75	0.21378	21.4812217805079	21.4812217805079\\
65.75	0.21744	22.6617162101101	22.6617162101101\\
65.75	0.2211	23.8677546270702	23.8677546270702\\
65.75	0.22476	25.0993370313883	25.0993370313883\\
65.75	0.22842	26.3564634230643	26.3564634230643\\
65.75	0.23208	27.6391338020984	27.6391338020984\\
65.75	0.23574	28.9473481684904	28.9473481684904\\
65.75	0.2394	30.2811065222404	30.2811065222404\\
65.75	0.24306	31.6404088633484	31.6404088633484\\
65.75	0.24672	33.0252551918144	33.0252551918144\\
65.75	0.25038	34.4356455076383	34.4356455076383\\
65.75	0.25404	35.8715798108203	35.8715798108203\\
65.75	0.2577	37.3330581013602	37.3330581013602\\
65.75	0.26136	38.8200803792581	38.8200803792581\\
65.75	0.26502	40.332646644514	40.332646644514\\
65.75	0.26868	41.8707568971278	41.8707568971278\\
65.75	0.27234	43.4344111370996	43.4344111370996\\
65.75	0.276	45.0236093644294	45.0236093644294\\
66.125	0.093	-3.20411665259388	-3.20411665259388\\
66.125	0.09666	-2.85934534004307	-2.85934534004307\\
66.125	0.10032	-2.48903004013428	-2.48903004013428\\
66.125	0.10398	-2.09317075286751	-2.09317075286751\\
66.125	0.10764	-1.67176747824278	-1.67176747824278\\
66.125	0.1113	-1.22482021626006	-1.22482021626006\\
66.125	0.11496	-0.752328966919357	-0.752328966919357\\
66.125	0.11862	-0.254293730220676	-0.254293730220676\\
66.125	0.12228	0.269285493835996	0.269285493835996\\
66.125	0.12594	0.818408705250638	0.818408705250638\\
66.125	0.1296	1.39307590402326	1.39307590402326\\
66.125	0.13326	1.99328709015385	1.99328709015385\\
66.125	0.13692	2.61904226364243	2.61904226364243\\
66.125	0.14058	3.27034142448899	3.27034142448899\\
66.125	0.14424	3.94718457269354	3.94718457269354\\
66.125	0.1479	4.64957170825605	4.64957170825605\\
66.125	0.15156	5.37750283117655	5.37750283117655\\
66.125	0.15522	6.13097794145503	6.13097794145503\\
66.125	0.15888	6.90999703909149	6.90999703909149\\
66.125	0.16254	7.71456012408593	7.71456012408593\\
66.125	0.1662	8.54466719643835	8.54466719643835\\
66.125	0.16986	9.40031825614874	9.40031825614874\\
66.125	0.17352	10.2815133032171	10.2815133032171\\
66.125	0.17718	11.1882523376435	11.1882523376435\\
66.125	0.18084	12.1205353594278	12.1205353594278\\
66.125	0.1845	13.0783623685701	13.0783623685701\\
66.125	0.18816	14.0617333650704	14.0617333650704\\
66.125	0.19182	15.0706483489287	15.0706483489287\\
66.125	0.19548	16.1051073201449	16.1051073201449\\
66.125	0.19914	17.1651102787192	17.1651102787192\\
66.125	0.2028	18.2506572246514	18.2506572246514\\
66.125	0.20646	19.3617481579416	19.3617481579416\\
66.125	0.21012	20.4983830785897	20.4983830785897\\
66.125	0.21378	21.6605619865959	21.6605619865959\\
66.125	0.21744	22.84828488196	22.84828488196\\
66.125	0.2211	24.0615517646821	24.0615517646821\\
66.125	0.22476	25.3003626347622	25.3003626347622\\
66.125	0.22842	26.5647174922003	26.5647174922003\\
66.125	0.23208	27.8546163369964	27.8546163369964\\
66.125	0.23574	29.1700591691504	29.1700591691504\\
66.125	0.2394	30.5110459886624	30.5110459886624\\
66.125	0.24306	31.8775767955324	31.8775767955324\\
66.125	0.24672	33.2696515897604	33.2696515897604\\
66.125	0.25038	34.6872703713463	34.6872703713463\\
66.125	0.25404	36.1304331402902	36.1304331402902\\
66.125	0.2577	37.5991398965922	37.5991398965922\\
66.125	0.26136	39.093390640252	39.093390640252\\
66.125	0.26502	40.6131853712699	40.6131853712699\\
66.125	0.26868	42.1585240896457	42.1585240896457\\
66.125	0.27234	43.7294067953796	43.7294067953796\\
66.125	0.276	45.3258334884714	45.3258334884714\\
66.5	0.093	-3.26094714367454	-3.26094714367454\\
66.5	0.09666	-2.90894736536173	-2.90894736536173\\
66.5	0.10032	-2.53140359969098	-2.53140359969098\\
66.5	0.10398	-2.12831584666221	-2.12831584666221\\
66.5	0.10764	-1.69968410627547	-1.69968410627547\\
66.5	0.1113	-1.24550837853074	-1.24550837853074\\
66.5	0.11496	-0.76578866342804	-0.76578866342804\\
66.5	0.11862	-0.260524960967382	-0.260524960967382\\
66.5	0.12228	0.27028272885128	0.27028272885128\\
66.5	0.12594	0.826634406027905	0.826634406027905\\
66.5	0.1296	1.40853007056252	1.40853007056252\\
66.5	0.13326	2.01596972245512	2.01596972245512\\
66.5	0.13692	2.6489533617057	2.6489533617057\\
66.5	0.14058	3.30748098831426	3.30748098831426\\
66.5	0.14424	3.99155260228078	3.99155260228078\\
66.5	0.1479	4.7011682036053	4.7011682036053\\
66.5	0.15156	5.43632779228779	5.43632779228779\\
66.5	0.15522	6.19703136832827	6.19703136832827\\
66.5	0.15888	6.98327893172673	6.98327893172673\\
66.5	0.16254	7.79507048248317	7.79507048248317\\
66.5	0.1662	8.63240602059758	8.63240602059758\\
66.5	0.16986	9.49528554606995	9.49528554606995\\
66.5	0.17352	10.3837090589003	10.3837090589003\\
66.5	0.17718	11.2976765590887	11.2976765590887\\
66.5	0.18084	12.237188046635	12.237188046635\\
66.5	0.1845	13.2022435215393	13.2022435215393\\
66.5	0.18816	14.1928429838016	14.1928429838016\\
66.5	0.19182	15.2089864334219	15.2089864334219\\
66.5	0.19548	16.2506738704001	16.2506738704001\\
66.5	0.19914	17.3179052947364	17.3179052947364\\
66.5	0.2028	18.4106807064306	18.4106807064306\\
66.5	0.20646	19.5290001054828	19.5290001054828\\
66.5	0.21012	20.6728634918929	20.6728634918929\\
66.5	0.21378	21.8422708656611	21.8422708656611\\
66.5	0.21744	23.0372222267872	23.0372222267872\\
66.5	0.2211	24.2577175752713	24.2577175752713\\
66.5	0.22476	25.5037569111134	25.5037569111134\\
66.5	0.22842	26.7753402343135	26.7753402343135\\
66.5	0.23208	28.0724675448715	28.0724675448715\\
66.5	0.23574	29.3951388427875	29.3951388427875\\
66.5	0.2394	30.7433541280615	30.7433541280615\\
66.5	0.24306	32.1171134006935	32.1171134006935\\
66.5	0.24672	33.5164166606835	33.5164166606835\\
66.5	0.25038	34.9412639080314	34.9412639080314\\
66.5	0.25404	36.3916551427373	36.3916551427373\\
66.5	0.2577	37.8675903648013	37.8675903648013\\
66.5	0.26136	39.3690695742231	39.3690695742231\\
66.5	0.26502	40.896092771003	40.896092771003\\
66.5	0.26868	42.4486599551408	42.4486599551408\\
66.5	0.27234	44.0267711266367	44.0267711266367\\
66.5	0.276	45.6304262854904	45.6304262854904\\
66.875	0.093	-3.31540896177801	-3.31540896177801\\
66.875	0.09666	-2.95618071770321	-2.95618071770321\\
66.875	0.10032	-2.57140848627046	-2.57140848627046\\
66.875	0.10398	-2.1610922674797	-2.1610922674797\\
66.875	0.10764	-1.72523206133097	-1.72523206133097\\
66.875	0.1113	-1.26382786782425	-1.26382786782425\\
66.875	0.11496	-0.776879686959546	-0.776879686959546\\
66.875	0.11862	-0.264387518736891	-0.264387518736891\\
66.875	0.12228	0.273648636843769	0.273648636843769\\
66.875	0.12594	0.837228779782384	0.837228779782384\\
66.875	0.1296	1.426352910079	1.426352910079\\
66.875	0.13326	2.04102102773359	2.04102102773359\\
66.875	0.13692	2.68123313274616	2.68123313274616\\
66.875	0.14058	3.34698922511672	3.34698922511672\\
66.875	0.14424	4.03828930484523	4.03828930484523\\
66.875	0.1479	4.75513337193175	4.75513337193175\\
66.875	0.15156	5.49752142637623	5.49752142637623\\
66.875	0.15522	6.26545346817871	6.26545346817871\\
66.875	0.15888	7.05892949733916	7.05892949733916\\
66.875	0.16254	7.87794951385759	7.87794951385759\\
66.875	0.1662	8.722513517734	8.722513517734\\
66.875	0.16986	9.59262150896837	9.59262150896837\\
66.875	0.17352	10.4882734875607	10.4882734875607\\
66.875	0.17718	11.4094694535111	11.4094694535111\\
66.875	0.18084	12.3562094068194	12.3562094068194\\
66.875	0.1845	13.3284933474857	13.3284933474857\\
66.875	0.18816	14.32632127551	14.32632127551\\
66.875	0.19182	15.3496931908923	15.3496931908923\\
66.875	0.19548	16.3986090936325	16.3986090936325\\
66.875	0.19914	17.4730689837307	17.4730689837307\\
66.875	0.2028	18.5730728611869	18.5730728611869\\
66.875	0.20646	19.6986207260011	19.6986207260011\\
66.875	0.21012	20.8497125781733	20.8497125781733\\
66.875	0.21378	22.0263484177034	22.0263484177034\\
66.875	0.21744	23.2285282445916	23.2285282445916\\
66.875	0.2211	24.4562520588376	24.4562520588376\\
66.875	0.22476	25.7095198604417	25.7095198604417\\
66.875	0.22842	26.9883316494038	26.9883316494038\\
66.875	0.23208	28.2926874257238	28.2926874257238\\
66.875	0.23574	29.6225871894018	29.6225871894018\\
66.875	0.2394	30.9780309404379	30.9780309404379\\
66.875	0.24306	32.3590186788318	32.3590186788318\\
66.875	0.24672	33.7655504045838	33.7655504045838\\
66.875	0.25038	35.1976261176937	35.1976261176937\\
66.875	0.25404	36.6552458181616	36.6552458181616\\
66.875	0.2577	38.1384095059876	38.1384095059876\\
66.875	0.26136	39.6471171811714	39.6471171811714\\
66.875	0.26502	41.1813688437133	41.1813688437133\\
66.875	0.26868	42.7411644936131	42.7411644936131\\
66.875	0.27234	44.3265041308709	44.3265041308709\\
66.875	0.276	45.9373877554867	45.9373877554867\\
67.25	0.093	-3.36750210690428	-3.36750210690428\\
67.25	0.09666	-3.00104539706749	-3.00104539706749\\
67.25	0.10032	-2.60904469987273	-2.60904469987273\\
67.25	0.10398	-2.19150001531998	-2.19150001531998\\
67.25	0.10764	-1.74841134340926	-1.74841134340926\\
67.25	0.1113	-1.27977868414054	-1.27977868414054\\
67.25	0.11496	-0.785602037513847	-0.785602037513847\\
67.25	0.11862	-0.265881403529194	-0.265881403529194\\
67.25	0.12228	0.279383217813455	0.279383217813455\\
67.25	0.12594	0.850191826514076	0.850191826514076\\
67.25	0.1296	1.44654442257268	1.44654442257268\\
67.25	0.13326	2.06844100598927	2.06844100598927\\
67.25	0.13692	2.71588157676383	2.71588157676383\\
67.25	0.14058	3.38886613489639	3.38886613489639\\
67.25	0.14424	4.0873946803869	4.0873946803869\\
67.25	0.1479	4.8114672132354	4.8114672132354\\
67.25	0.15156	5.56108373344189	5.56108373344189\\
67.25	0.15522	6.33624424100635	6.33624424100635\\
67.25	0.15888	7.1369487359288	7.1369487359288\\
67.25	0.16254	7.96319721820923	7.96319721820923\\
67.25	0.1662	8.81498968784763	8.81498968784763\\
67.25	0.16986	9.69232614484401	9.69232614484401\\
67.25	0.17352	10.5952065891984	10.5952065891984\\
67.25	0.17718	11.5236310209107	11.5236310209107\\
67.25	0.18084	12.477599439981	12.477599439981\\
67.25	0.1845	13.4571118464093	13.4571118464093\\
67.25	0.18816	14.4621682401956	14.4621682401956\\
67.25	0.19182	15.4927686213399	15.4927686213399\\
67.25	0.19548	16.5489129898421	16.5489129898421\\
67.25	0.19914	17.6306013457023	17.6306013457023\\
67.25	0.2028	18.7378336889205	18.7378336889205\\
67.25	0.20646	19.8706100194967	19.8706100194967\\
67.25	0.21012	21.0289303374309	21.0289303374309\\
67.25	0.21378	22.212794642723	22.212794642723\\
67.25	0.21744	23.4222029353731	23.4222029353731\\
67.25	0.2211	24.6571552153812	24.6571552153812\\
67.25	0.22476	25.9176514827473	25.9176514827473\\
67.25	0.22842	27.2036917374713	27.2036917374713\\
67.25	0.23208	28.5152759795534	28.5152759795534\\
67.25	0.23574	29.8524042089934	29.8524042089934\\
67.25	0.2394	31.2150764257914	31.2150764257914\\
67.25	0.24306	32.6032926299473	32.6032926299473\\
67.25	0.24672	34.0170528214613	34.0170528214613\\
67.25	0.25038	35.4563570003332	35.4563570003332\\
67.25	0.25404	36.9212051665632	36.9212051665632\\
67.25	0.2577	38.4115973201511	38.4115973201511\\
67.25	0.26136	39.9275334610969	39.9275334610969\\
67.25	0.26502	41.4690135894008	41.4690135894008\\
67.25	0.26868	43.0360377050626	43.0360377050626\\
67.25	0.27234	44.6286058080824	44.6286058080824\\
67.25	0.276	46.2467178984602	46.2467178984602\\
67.625	0.093	-3.41722657905337	-3.41722657905337\\
67.625	0.09666	-3.04354140345458	-3.04354140345458\\
67.625	0.10032	-2.64431224049784	-2.64431224049784\\
67.625	0.10398	-2.21953909018309	-2.21953909018309\\
67.625	0.10764	-1.76922195251037	-1.76922195251037\\
67.625	0.1113	-1.29336082747965	-1.29336082747965\\
67.625	0.11496	-0.791955715090971	-0.791955715090971\\
67.625	0.11862	-0.265006615344321	-0.265006615344321\\
67.625	0.12228	0.287486471760326	0.287486471760326\\
67.625	0.12594	0.865523546222937	0.865523546222937\\
67.625	0.1296	1.46910460804354	1.46910460804354\\
67.625	0.13326	2.09822965722212	2.09822965722212\\
67.625	0.13692	2.75289869375868	2.75289869375868\\
67.625	0.14058	3.43311171765322	3.43311171765322\\
67.625	0.14424	4.13886872890573	4.13886872890573\\
67.625	0.1479	4.87016972751623	4.87016972751623\\
67.625	0.15156	5.62701471348472	5.62701471348472\\
67.625	0.15522	6.40940368681118	6.40940368681118\\
67.625	0.15888	7.21733664749562	7.21733664749562\\
67.625	0.16254	8.05081359553804	8.05081359553804\\
67.625	0.1662	8.90983453093844	8.90983453093844\\
67.625	0.16986	9.7943994536968	9.7943994536968\\
67.625	0.17352	10.7045083638132	10.7045083638132\\
67.625	0.17718	11.6401612612875	11.6401612612875\\
67.625	0.18084	12.6013581461198	12.6013581461198\\
67.625	0.1845	13.5880990183101	13.5880990183101\\
67.625	0.18816	14.6003838778584	14.6003838778584\\
67.625	0.19182	15.6382127247646	15.6382127247646\\
67.625	0.19548	16.7015855590289	16.7015855590289\\
67.625	0.19914	17.7905023806511	17.7905023806511\\
67.625	0.2028	18.9049631896313	18.9049631896313\\
67.625	0.20646	20.0449679859695	20.0449679859695\\
67.625	0.21012	21.2105167696656	21.2105167696656\\
67.625	0.21378	22.4016095407197	22.4016095407197\\
67.625	0.21744	23.6182462991318	23.6182462991318\\
67.625	0.2211	24.8604270449019	24.8604270449019\\
67.625	0.22476	26.12815177803	26.12815177803\\
67.625	0.22842	27.421420498516	27.421420498516\\
67.625	0.23208	28.7402332063601	28.7402332063601\\
67.625	0.23574	30.0845899015621	30.0845899015621\\
67.625	0.2394	31.4544905841221	31.4544905841221\\
67.625	0.24306	32.8499352540401	32.8499352540401\\
67.625	0.24672	34.270923911316	34.270923911316\\
67.625	0.25038	35.7174565559499	35.7174565559499\\
67.625	0.25404	37.1895331879418	37.1895331879418\\
67.625	0.2577	38.6871538072917	38.6871538072917\\
67.625	0.26136	40.2103184139996	40.2103184139996\\
67.625	0.26502	41.7590270080654	41.7590270080654\\
67.625	0.26868	43.3332795894893	43.3332795894893\\
67.625	0.27234	44.933076158271	44.933076158271\\
67.625	0.276	46.5584167144108	46.5584167144108\\
68	0.093	-3.46458237822528	-3.46458237822528\\
68	0.09666	-3.08366873686449	-3.08366873686449\\
68	0.10032	-2.67721110814575	-2.67721110814575\\
68	0.10398	-2.24520949206901	-2.24520949206901\\
68	0.10764	-1.78766388863429	-1.78766388863429\\
68	0.1113	-1.30457429784158	-1.30457429784158\\
68	0.11496	-0.795940719690897	-0.795940719690897\\
68	0.11862	-0.261763154182256	-0.261763154182256\\
68	0.12228	0.297958398684388	0.297958398684388\\
68	0.12594	0.883223938908989	0.883223938908989\\
68	0.1296	1.49403346649159	1.49403346649159\\
68	0.13326	2.13038698143216	2.13038698143216\\
68	0.13692	2.79228448373071	2.79228448373071\\
68	0.14058	3.47972597338726	3.47972597338726\\
68	0.14424	4.19271145040176	4.19271145040176\\
68	0.1479	4.93124091477426	4.93124091477426\\
68	0.15156	5.69531436650474	5.69531436650474\\
68	0.15522	6.48493180559319	6.48493180559319\\
68	0.15888	7.30009323203963	7.30009323203963\\
68	0.16254	8.14079864584405	8.14079864584405\\
68	0.1662	9.00704804700644	9.00704804700644\\
68	0.16986	9.8988414355268	9.8988414355268\\
68	0.17352	10.8161788114051	10.8161788114051\\
68	0.17718	11.7590601746415	11.7590601746415\\
68	0.18084	12.7274855252358	12.7274855252358\\
68	0.1845	13.7214548631881	13.7214548631881\\
68	0.18816	14.7409681884983	14.7409681884983\\
68	0.19182	15.7860255011666	15.7860255011666\\
68	0.19548	16.8566268011928	16.8566268011928\\
68	0.19914	17.952772088577	17.952772088577\\
68	0.2028	19.0744613633192	19.0744613633192\\
68	0.20646	20.2216946254194	20.2216946254194\\
68	0.21012	21.3944718748775	21.3944718748775\\
68	0.21378	22.5927931116937	22.5927931116937\\
68	0.21744	23.8166583358678	23.8166583358678\\
68	0.2211	25.0660675473999	25.0660675473999\\
68	0.22476	26.3410207462899	26.3410207462899\\
68	0.22842	27.641517932538	27.641517932538\\
68	0.23208	28.967559106144	28.967559106144\\
68	0.23574	30.319144267108	30.319144267108\\
68	0.2394	31.69627341543	31.69627341543\\
68	0.24306	33.0989465511099	33.0989465511099\\
68	0.24672	34.5271636741479	34.5271636741479\\
68	0.25038	35.9809247845438	35.9809247845438\\
68	0.25404	37.4602298822977	37.4602298822977\\
68	0.2577	38.9650789674096	38.9650789674096\\
68	0.26136	40.4954720398794	40.4954720398794\\
68	0.26502	42.0514090997073	42.0514090997073\\
68	0.26868	43.6328901468931	43.6328901468931\\
68	0.27234	45.2399151814369	45.2399151814369\\
68	0.276	46.8724842033387	46.8724842033387\\
68.375	0.093	-3.50956950442002	-3.50956950442002\\
68.375	0.09666	-3.12142739729724	-3.12142739729724\\
68.375	0.10032	-2.70774130281647	-2.70774130281647\\
68.375	0.10398	-2.26851122097774	-2.26851122097774\\
68.375	0.10764	-1.80373715178104	-1.80373715178104\\
68.375	0.1113	-1.31341909522635	-1.31341909522635\\
68.375	0.11496	-0.797557051313674	-0.797557051313674\\
68.375	0.11862	-0.256151020043022	-0.256151020043022\\
68.375	0.12228	0.310798998585613	0.310798998585613\\
68.375	0.12594	0.903293004572225	0.903293004572225\\
68.375	0.1296	1.52133099791682	1.52133099791682\\
68.375	0.13326	2.16491297861937	2.16491297861937\\
68.375	0.13692	2.83403894667993	2.83403894667993\\
68.375	0.14058	3.52870890209846	3.52870890209846\\
68.375	0.14424	4.24892284487497	4.24892284487497\\
68.375	0.1479	4.99468077500945	4.99468077500945\\
68.375	0.15156	5.76598269250193	5.76598269250193\\
68.375	0.15522	6.56282859735238	6.56282859735238\\
68.375	0.15888	7.38521848956081	7.38521848956081\\
68.375	0.16254	8.23315236912721	8.23315236912721\\
68.375	0.1662	9.1066302360516	9.1066302360516\\
68.375	0.16986	10.005652090334	10.005652090334\\
68.375	0.17352	10.9302179319743	10.9302179319743\\
68.375	0.17718	11.8803277609726	11.8803277609726\\
68.375	0.18084	12.8559815773289	12.8559815773289\\
68.375	0.1845	13.8571793810432	13.8571793810432\\
68.375	0.18816	14.8839211721155	14.8839211721155\\
68.375	0.19182	15.9362069505457	15.9362069505457\\
68.375	0.19548	17.0140367163339	17.0140367163339\\
68.375	0.19914	18.1174104694802	18.1174104694802\\
68.375	0.2028	19.2463282099844	19.2463282099844\\
68.375	0.20646	20.4007899378465	20.4007899378465\\
68.375	0.21012	21.5807956530667	21.5807956530667\\
68.375	0.21378	22.7863453556448	22.7863453556448\\
68.375	0.21744	24.0174390455809	24.0174390455809\\
68.375	0.2211	25.2740767228749	25.2740767228749\\
68.375	0.22476	26.556258387527	26.556258387527\\
68.375	0.22842	27.863984039537	27.863984039537\\
68.375	0.23208	29.1972536789051	29.1972536789051\\
68.375	0.23574	30.5560673056311	30.5560673056311\\
68.375	0.2394	31.940424919715	31.940424919715\\
68.375	0.24306	33.350326521157	33.350326521157\\
68.375	0.24672	34.7857721099569	34.7857721099569\\
68.375	0.25038	36.2467616861149	36.2467616861149\\
68.375	0.25404	37.7332952496308	37.7332952496308\\
68.375	0.2577	39.2453728005046	39.2453728005046\\
68.375	0.26136	40.7829943387365	40.7829943387365\\
68.375	0.26502	42.3461598643263	42.3461598643263\\
68.375	0.26868	43.9348693772741	43.9348693772741\\
68.375	0.27234	45.5491228775799	45.5491228775799\\
68.375	0.276	47.1889203652437	47.1889203652437\\
68.75	0.093	-3.55218795763754	-3.55218795763754\\
68.75	0.09666	-3.15681738475276	-3.15681738475276\\
68.75	0.10032	-2.73590282451001	-2.73590282451001\\
68.75	0.10398	-2.28944427690928	-2.28944427690928\\
68.75	0.10764	-1.81744174195058	-1.81744174195058\\
68.75	0.1113	-1.3198952196339	-1.3198952196339\\
68.75	0.11496	-0.796804709959225	-0.796804709959225\\
68.75	0.11862	-0.248170212926576	-0.248170212926576\\
68.75	0.12228	0.326008271464056	0.326008271464056\\
68.75	0.12594	0.925730743212659	0.925730743212659\\
68.75	0.1296	1.55099720231925	1.55099720231925\\
68.75	0.13326	2.2018076487838	2.2018076487838\\
68.75	0.13692	2.87816208260635	2.87816208260635\\
68.75	0.14058	3.58006050378688	3.58006050378688\\
68.75	0.14424	4.30750291232539	4.30750291232539\\
68.75	0.1479	5.06048930822186	5.06048930822186\\
68.75	0.15156	5.83901969147632	5.83901969147632\\
68.75	0.15522	6.64309406208877	6.64309406208877\\
68.75	0.15888	7.4727124200592	7.4727124200592\\
68.75	0.16254	8.3278747653876	8.3278747653876\\
68.75	0.1662	9.20858109807399	9.20858109807399\\
68.75	0.16986	10.1148314181183	10.1148314181183\\
68.75	0.17352	11.0466257255207	11.0466257255207\\
68.75	0.17718	12.003964020281	12.003964020281\\
68.75	0.18084	12.9868463023993	12.9868463023993\\
68.75	0.1845	13.9952725718756	13.9952725718756\\
68.75	0.18816	15.0292428287098	15.0292428287098\\
68.75	0.19182	16.0887570729021	16.0887570729021\\
68.75	0.19548	17.1738153044523	17.1738153044523\\
68.75	0.19914	18.2844175233605	18.2844175233605\\
68.75	0.2028	19.4205637296267	19.4205637296267\\
68.75	0.20646	20.5822539232509	20.5822539232509\\
68.75	0.21012	21.769488104233	21.769488104233\\
68.75	0.21378	22.9822662725731	22.9822662725731\\
68.75	0.21744	24.2205884282712	24.2205884282712\\
68.75	0.2211	25.4844545713272	25.4844545713272\\
68.75	0.22476	26.7738647017413	26.7738647017413\\
68.75	0.22842	28.0888188195133	28.0888188195133\\
68.75	0.23208	29.4293169246434	29.4293169246434\\
68.75	0.23574	30.7953590171314	30.7953590171314\\
68.75	0.2394	32.1869450969773	32.1869450969773\\
68.75	0.24306	33.6040751641813	33.6040751641813\\
68.75	0.24672	35.0467492187432	35.0467492187432\\
68.75	0.25038	36.5149672606632	36.5149672606632\\
68.75	0.25404	38.008729289941	38.008729289941\\
68.75	0.2577	39.5280353065769	39.5280353065769\\
68.75	0.26136	41.0728853105707	41.0728853105707\\
68.75	0.26502	42.6432793019226	42.6432793019226\\
68.75	0.26868	44.2392172806324	44.2392172806324\\
68.75	0.27234	45.8606992467002	45.8606992467002\\
68.75	0.276	47.5077252001259	47.5077252001259\\
69.125	0.093	-3.59243773787788	-3.59243773787788\\
69.125	0.09666	-3.18983869923111	-3.18983869923111\\
69.125	0.10032	-2.76169567322636	-2.76169567322636\\
69.125	0.10398	-2.30800865986363	-2.30800865986363\\
69.125	0.10764	-1.82877765914294	-1.82877765914294\\
69.125	0.1113	-1.32400267106426	-1.32400267106426\\
69.125	0.11496	-0.7936836956276	-0.7936836956276\\
69.125	0.11862	-0.237820732832953	-0.237820732832953\\
69.125	0.12228	0.34358621731967	0.34358621731967\\
69.125	0.12594	0.950537154830277	0.950537154830277\\
69.125	0.1296	1.58303207969886	1.58303207969886\\
69.125	0.13326	2.24107099192541	2.24107099192541\\
69.125	0.13692	2.92465389150994	2.92465389150994\\
69.125	0.14058	3.63378077845247	3.63378077845247\\
69.125	0.14424	4.36845165275297	4.36845165275297\\
69.125	0.1479	5.12866651441144	5.12866651441144\\
69.125	0.15156	5.9144253634279	5.9144253634279\\
69.125	0.15522	6.72572819980234	6.72572819980234\\
69.125	0.15888	7.56257502353476	7.56257502353476\\
69.125	0.16254	8.42496583462517	8.42496583462517\\
69.125	0.1662	9.31290063307354	9.31290063307354\\
69.125	0.16986	10.2263794188799	10.2263794188799\\
69.125	0.17352	11.1654021920442	11.1654021920442\\
69.125	0.17718	12.1299689525665	12.1299689525665\\
69.125	0.18084	13.1200797004468	13.1200797004468\\
69.125	0.1845	14.1357344356851	14.1357344356851\\
69.125	0.18816	15.1769331582814	15.1769331582814\\
69.125	0.19182	16.2436758682356	16.2436758682356\\
69.125	0.19548	17.3359625655478	17.3359625655478\\
69.125	0.19914	18.453793250218	18.453793250218\\
69.125	0.2028	19.5971679222462	19.5971679222462\\
69.125	0.20646	20.7660865816323	20.7660865816323\\
69.125	0.21012	21.9605492283765	21.9605492283765\\
69.125	0.21378	23.1805558624786	23.1805558624786\\
69.125	0.21744	24.4261064839387	24.4261064839387\\
69.125	0.2211	25.6972010927567	25.6972010927567\\
69.125	0.22476	26.9938396889328	26.9938396889328\\
69.125	0.22842	28.3160222724668	28.3160222724668\\
69.125	0.23208	29.6637488433588	29.6637488433588\\
69.125	0.23574	31.0370194016088	31.0370194016088\\
69.125	0.2394	32.4358339472168	32.4358339472168\\
69.125	0.24306	33.8601924801827	33.8601924801827\\
69.125	0.24672	35.3100950005066	35.3100950005066\\
69.125	0.25038	36.7855415081886	36.7855415081886\\
69.125	0.25404	38.2865320032285	38.2865320032285\\
69.125	0.2577	39.8130664856263	39.8130664856263\\
69.125	0.26136	41.3651449553822	41.3651449553822\\
69.125	0.26502	42.942767412496	42.942767412496\\
69.125	0.26868	44.5459338569678	44.5459338569678\\
69.125	0.27234	46.1746442887976	46.1746442887976\\
69.125	0.276	47.8288987079853	47.8288987079853\\
69.5	0.093	-3.63031884514102	-3.63031884514102\\
69.5	0.09666	-3.22049134073226	-3.22049134073226\\
69.5	0.10032	-2.78511984896551	-2.78511984896551\\
69.5	0.10398	-2.32420436984079	-2.32420436984079\\
69.5	0.10764	-1.8377449033581	-1.8377449033581\\
69.5	0.1113	-1.32574144951743	-1.32574144951743\\
69.5	0.11496	-0.788194008318769	-0.788194008318769\\
69.5	0.11862	-0.225102579762131	-0.225102579762131\\
69.5	0.12228	0.363532836152489	0.363532836152489\\
69.5	0.12594	0.977712239425086	0.977712239425086\\
69.5	0.1296	1.61743563005566	1.61743563005566\\
69.5	0.13326	2.28270300804421	2.28270300804421\\
69.5	0.13692	2.97351437339074	2.97351437339074\\
69.5	0.14058	3.68986972609526	3.68986972609526\\
69.5	0.14424	4.43176906615776	4.43176906615776\\
69.5	0.1479	5.19921239357822	5.19921239357822\\
69.5	0.15156	5.99219970835668	5.99219970835668\\
69.5	0.15522	6.81073101049311	6.81073101049311\\
69.5	0.15888	7.65480629998753	7.65480629998753\\
69.5	0.16254	8.52442557683993	8.52442557683993\\
69.5	0.1662	9.4195888410503	9.4195888410503\\
69.5	0.16986	10.3402960926186	10.3402960926186\\
69.5	0.17352	11.286547331545	11.286547331545\\
69.5	0.17718	12.2583425578293	12.2583425578293\\
69.5	0.18084	13.2556817714716	13.2556817714716\\
69.5	0.1845	14.2785649724718	14.2785649724718\\
69.5	0.18816	15.3269921608301	15.3269921608301\\
69.5	0.19182	16.4009633365463	16.4009633365463\\
69.5	0.19548	17.5004784996205	17.5004784996205\\
69.5	0.19914	18.6255376500527	18.6255376500527\\
69.5	0.2028	19.7761407878429	19.7761407878429\\
69.5	0.20646	20.9522879129911	20.9522879129911\\
69.5	0.21012	22.1539790254971	22.1539790254971\\
69.5	0.21378	23.3812141253613	23.3812141253613\\
69.5	0.21744	24.6339932125833	24.6339932125833\\
69.5	0.2211	25.9123162871634	25.9123162871634\\
69.5	0.22476	27.2161833491015	27.2161833491015\\
69.5	0.22842	28.5455943983975	28.5455943983975\\
69.5	0.23208	29.9005494350515	29.9005494350515\\
69.5	0.23574	31.2810484590635	31.2810484590635\\
69.5	0.2394	32.6870914704335	32.6870914704335\\
69.5	0.24306	34.1186784691614	34.1186784691614\\
69.5	0.24672	35.5758094552473	35.5758094552473\\
69.5	0.25038	37.0584844286912	37.0584844286912\\
69.5	0.25404	38.5667033894931	38.5667033894931\\
69.5	0.2577	40.100466337653	40.100466337653\\
69.5	0.26136	41.6597732731708	41.6597732731708\\
69.5	0.26502	43.2446241960466	43.2446241960466\\
69.5	0.26868	44.8550191062804	44.8550191062804\\
69.5	0.27234	46.4909580038722	46.4909580038722\\
69.5	0.276	48.152440888822	48.152440888822\\
69.875	0.093	-3.66583127942699	-3.66583127942699\\
69.875	0.09666	-3.24877530925623	-3.24877530925623\\
69.875	0.10032	-2.80617535172749	-2.80617535172749\\
69.875	0.10398	-2.33803140684077	-2.33803140684077\\
69.875	0.10764	-1.84434347459609	-1.84434347459609\\
69.875	0.1113	-1.32511155499343	-1.32511155499343\\
69.875	0.11496	-0.780335648032768	-0.780335648032768\\
69.875	0.11862	-0.210015753714133	-0.210015753714133\\
69.875	0.12228	0.385848127962477	0.385848127962477\\
69.875	0.12594	1.00725599699707	1.00725599699707\\
69.875	0.1296	1.65420785338965	1.65420785338965\\
69.875	0.13326	2.32670369714019	2.32670369714019\\
69.875	0.13692	3.02474352824871	3.02474352824871\\
69.875	0.14058	3.74832734671523	3.74832734671523\\
69.875	0.14424	4.49745515253972	4.49745515253972\\
69.875	0.1479	5.27212694572218	5.27212694572218\\
69.875	0.15156	6.07234272626263	6.07234272626263\\
69.875	0.15522	6.89810249416105	6.89810249416105\\
69.875	0.15888	7.74940624941748	7.74940624941748\\
69.875	0.16254	8.62625399203186	8.62625399203186\\
69.875	0.1662	9.52864572200423	9.52864572200423\\
69.875	0.16986	10.4565814393346	10.4565814393346\\
69.875	0.17352	11.4100611440229	11.4100611440229\\
69.875	0.17718	12.3890848360692	12.3890848360692\\
69.875	0.18084	13.3936525154735	13.3936525154735\\
69.875	0.1845	14.4237641822357	14.4237641822357\\
69.875	0.18816	15.479419836356	15.479419836356\\
69.875	0.19182	16.5606194778342	16.5606194778342\\
69.875	0.19548	17.6673631066704	17.6673631066704\\
69.875	0.19914	18.7996507228646	18.7996507228646\\
69.875	0.2028	19.9574823264168	19.9574823264168\\
69.875	0.20646	21.1408579173269	21.1408579173269\\
69.875	0.21012	22.349777495595	22.349777495595\\
69.875	0.21378	23.5842410612211	23.5842410612211\\
69.875	0.21744	24.8442486142052	24.8442486142052\\
69.875	0.2211	26.1298001545473	26.1298001545473\\
69.875	0.22476	27.4408956822473	27.4408956822473\\
69.875	0.22842	28.7775351973053	28.7775351973053\\
69.875	0.23208	30.1397186997213	30.1397186997213\\
69.875	0.23574	31.5274461894953	31.5274461894953\\
69.875	0.2394	32.9407176666273	32.9407176666273\\
69.875	0.24306	34.3795331311172	34.3795331311172\\
69.875	0.24672	35.8438925829651	35.8438925829651\\
69.875	0.25038	37.333796022171	37.333796022171\\
69.875	0.25404	38.8492434487349	38.8492434487349\\
69.875	0.2577	40.3902348626567	40.3902348626567\\
69.875	0.26136	41.9567702639366	41.9567702639366\\
69.875	0.26502	43.5488496525744	43.5488496525744\\
69.875	0.26868	45.1664730285702	45.1664730285702\\
69.875	0.27234	46.809640391924	46.809640391924\\
69.875	0.276	48.4783517426357	48.4783517426357\\
70.25	0.093	-3.69897504073576	-3.69897504073576\\
70.25	0.09666	-3.27469060480301	-3.27469060480301\\
70.25	0.10032	-2.8248621815123	-2.8248621815123\\
70.25	0.10398	-2.3494897708636	-2.3494897708636\\
70.25	0.10764	-1.8485733728569	-1.8485733728569\\
70.25	0.1113	-1.32211298749222	-1.32211298749222\\
70.25	0.11496	-0.770108614769576	-0.770108614769576\\
70.25	0.11862	-0.192560254688958	-0.192560254688958\\
70.25	0.12228	0.41053209274965	0.41053209274965\\
70.25	0.12594	1.03916842754622	1.03916842754622\\
70.25	0.1296	1.69334874970079	1.69334874970079\\
70.25	0.13326	2.37307305921334	2.37307305921334\\
70.25	0.13692	3.07834135608386	3.07834135608386\\
70.25	0.14058	3.80915364031237	3.80915364031237\\
70.25	0.14424	4.56550991189885	4.56550991189885\\
70.25	0.1479	5.34741017084331	5.34741017084331\\
70.25	0.15156	6.15485441714575	6.15485441714575\\
70.25	0.15522	6.98784265080618	6.98784265080618\\
70.25	0.15888	7.84637487182459	7.84637487182459\\
70.25	0.16254	8.73045108020098	8.73045108020098\\
70.25	0.1662	9.64007127593533	9.64007127593533\\
70.25	0.16986	10.5752354590277	10.5752354590277\\
70.25	0.17352	11.535943629478	11.535943629478\\
70.25	0.17718	12.5221957872863	12.5221957872863\\
70.25	0.18084	13.5339919324526	13.5339919324526\\
70.25	0.1845	14.5713320649768	14.5713320649768\\
70.25	0.18816	15.6342161848591	15.6342161848591\\
70.25	0.19182	16.7226442920993	16.7226442920993\\
70.25	0.19548	17.8366163866975	17.8366163866975\\
70.25	0.19914	18.9761324686537	18.9761324686537\\
70.25	0.2028	20.1411925379678	20.1411925379678\\
70.25	0.20646	21.33179659464	21.33179659464\\
70.25	0.21012	22.5479446386701	22.5479446386701\\
70.25	0.21378	23.7896366700582	23.7896366700582\\
70.25	0.21744	25.0568726888043	25.0568726888043\\
70.25	0.2211	26.3496526949083	26.3496526949083\\
70.25	0.22476	27.6679766883703	27.6679766883703\\
70.25	0.22842	29.0118446691903	29.0118446691903\\
70.25	0.23208	30.3812566373684	30.3812566373684\\
70.25	0.23574	31.7762125929043	31.7762125929043\\
70.25	0.2394	33.1967125357983	33.1967125357983\\
70.25	0.24306	34.6427564660502	34.6427564660502\\
70.25	0.24672	36.1143443836601	36.1143443836601\\
70.25	0.25038	37.611476288628	37.611476288628\\
70.25	0.25404	39.1341521809539	39.1341521809539\\
70.25	0.2577	40.6823720606377	40.6823720606377\\
70.25	0.26136	42.2561359276796	42.2561359276796\\
70.25	0.26502	43.8554437820794	43.8554437820794\\
70.25	0.26868	45.4802956238372	45.4802956238372\\
70.25	0.27234	47.1306914529529	47.1306914529529\\
70.25	0.276	48.8066312694267	48.8066312694267\\
70.625	0.093	-3.72975012906734	-3.72975012906734\\
70.625	0.09666	-3.2982372273726	-3.2982372273726\\
70.625	0.10032	-2.84118033831989	-2.84118033831989\\
70.625	0.10398	-2.35857946190918	-2.35857946190918\\
70.625	0.10764	-1.8504345981405	-1.8504345981405\\
70.625	0.1113	-1.31674574701383	-1.31674574701383\\
70.625	0.11496	-0.757512908529186	-0.757512908529186\\
70.625	0.11862	-0.172736082686571	-0.172736082686571\\
70.625	0.12228	0.437584730514034	0.437584730514034\\
70.625	0.12594	1.0734495310726	1.0734495310726\\
70.625	0.1296	1.73485831898916	1.73485831898916\\
70.625	0.13326	2.42181109426371	2.42181109426371\\
70.625	0.13692	3.13430785689622	3.13430785689622\\
70.625	0.14058	3.87234860688673	3.87234860688673\\
70.625	0.14424	4.63593334423519	4.63593334423519\\
70.625	0.1479	5.42506206894166	5.42506206894166\\
70.625	0.15156	6.2397347810061	6.2397347810061\\
70.625	0.15522	7.07995148042852	7.07995148042852\\
70.625	0.15888	7.94571216720892	7.94571216720892\\
70.625	0.16254	8.8370168413473	8.8370168413473\\
70.625	0.1662	9.75386550284366	9.75386550284366\\
70.625	0.16986	10.696258151698	10.696258151698\\
70.625	0.17352	11.6641947879103	11.6641947879103\\
70.625	0.17718	12.6576754114806	12.6576754114806\\
70.625	0.18084	13.6767000224089	13.6767000224089\\
70.625	0.1845	14.7212686206951	14.7212686206951\\
70.625	0.18816	15.7913812063394	15.7913812063394\\
70.625	0.19182	16.8870377793416	16.8870377793416\\
70.625	0.19548	18.0082383397018	18.0082383397018\\
70.625	0.19914	19.1549828874199	19.1549828874199\\
70.625	0.2028	20.3272714224961	20.3272714224961\\
70.625	0.20646	21.5251039449302	21.5251039449302\\
70.625	0.21012	22.7484804547223	22.7484804547223\\
70.625	0.21378	23.9974009518724	23.9974009518724\\
70.625	0.21744	25.2718654363805	25.2718654363805\\
70.625	0.2211	26.5718739082466	26.5718739082466\\
70.625	0.22476	27.8974263674706	27.8974263674706\\
70.625	0.22842	29.2485228140526	29.2485228140526\\
70.625	0.23208	30.6251632479926	30.6251632479926\\
70.625	0.23574	32.0273476692906	32.0273476692906\\
70.625	0.2394	33.4550760779465	33.4550760779465\\
70.625	0.24306	34.9083484739604	34.9083484739604\\
70.625	0.24672	36.3871648573323	36.3871648573323\\
70.625	0.25038	37.8915252280622	37.8915252280622\\
70.625	0.25404	39.4214295861501	39.4214295861501\\
70.625	0.2577	40.976877931596	40.976877931596\\
70.625	0.26136	42.5578702643997	42.5578702643997\\
70.625	0.26502	44.1644065845616	44.1644065845616\\
70.625	0.26868	45.7964868920813	45.7964868920813\\
70.625	0.27234	47.4541111869591	47.4541111869591\\
70.625	0.276	49.1372794691948	49.1372794691948\\
71	0.093	-3.75815654442175	-3.75815654442175\\
71	0.09666	-3.319415176965	-3.319415176965\\
71	0.10032	-2.8551298221503	-2.8551298221503\\
71	0.10398	-2.3653004799776	-2.3653004799776\\
71	0.10764	-1.84992715044692	-1.84992715044692\\
71	0.1113	-1.30900983355825	-1.30900983355825\\
71	0.11496	-0.742548529311613	-0.742548529311613\\
71	0.11862	-0.150543237707014	-0.150543237707014\\
71	0.12228	0.467006041255589	0.467006041255589\\
71	0.12594	1.11009930757615	1.11009930757615\\
71	0.1296	1.77873656125471	1.77873656125471\\
71	0.13326	2.47291780229125	2.47291780229125\\
71	0.13692	3.19264303068576	3.19264303068576\\
71	0.14058	3.93791224643826	3.93791224643826\\
71	0.14424	4.70872544954873	4.70872544954873\\
71	0.1479	5.50508264001718	5.50508264001718\\
71	0.15156	6.32698381784362	6.32698381784362\\
71	0.15522	7.17442898302803	7.17442898302803\\
71	0.15888	8.04741813557043	8.04741813557043\\
71	0.16254	8.9459512754708	8.9459512754708\\
71	0.1662	9.87002840272915	9.87002840272915\\
71	0.16986	10.8196495173455	10.8196495173455\\
71	0.17352	11.7948146193198	11.7948146193198\\
71	0.17718	12.7955237086521	12.7955237086521\\
71	0.18084	13.8217767853424	13.8217767853424\\
71	0.1845	14.8735738493906	14.8735738493906\\
71	0.18816	15.9509149007968	15.9509149007968\\
71	0.19182	17.053799939561	17.053799939561\\
71	0.19548	18.1822289656832	18.1822289656832\\
71	0.19914	19.3362019791634	19.3362019791634\\
71	0.2028	20.5157189800015	20.5157189800015\\
71	0.20646	21.7207799681977	21.7207799681977\\
71	0.21012	22.9513849437518	22.9513849437518\\
71	0.21378	24.2075339066639	24.2075339066639\\
71	0.21744	25.4892268569339	25.4892268569339\\
71	0.2211	26.7964637945619	26.7964637945619\\
71	0.22476	28.129244719548	28.129244719548\\
71	0.22842	29.487569631892	29.487569631892\\
71	0.23208	30.871438531594	30.871438531594\\
71	0.23574	32.280851418654	32.280851418654\\
71	0.2394	33.7158082930719	33.7158082930719\\
71	0.24306	35.1763091548478	35.1763091548478\\
71	0.24672	36.6623540039817	36.6623540039817\\
71	0.25038	38.1739428404736	38.1739428404736\\
71	0.25404	39.7110756643235	39.7110756643235\\
71	0.2577	41.2737524755313	41.2737524755313\\
71	0.26136	42.8619732740971	42.8619732740971\\
71	0.26502	44.4757380600209	44.4757380600209\\
71	0.26868	46.1150468333027	46.1150468333027\\
71	0.27234	47.7798995939425	47.7798995939425\\
71	0.276	49.4702963419402	49.4702963419402\\
71.375	0.093	-3.78419428679894	-3.78419428679894\\
71.375	0.09666	-3.33822445358021	-3.33822445358021\\
71.375	0.10032	-2.86671063300352	-2.86671063300352\\
71.375	0.10398	-2.36965282506882	-2.36965282506882\\
71.375	0.10764	-1.84705102977614	-1.84705102977614\\
71.375	0.1113	-1.29890524712548	-1.29890524712548\\
71.375	0.11496	-0.725215477116848	-0.725215477116848\\
71.375	0.11862	-0.125981719750252	-0.125981719750252\\
71.375	0.12228	0.498796024974348	0.498796024974348\\
71.375	0.12594	1.1491177570569	1.1491177570569\\
71.375	0.1296	1.82498347649745	1.82498347649745\\
71.375	0.13326	2.52639318329599	2.52639318329599\\
71.375	0.13692	3.2533468774525	3.2533468774525\\
71.375	0.14058	4.005844558967	4.005844558967\\
71.375	0.14424	4.78388622783945	4.78388622783945\\
71.375	0.1479	5.5874718840699	5.5874718840699\\
71.375	0.15156	6.41660152765833	6.41660152765833\\
71.375	0.15522	7.27127515860474	7.27127515860474\\
71.375	0.15888	8.15149277690914	8.15149277690914\\
71.375	0.16254	9.0572543825715	9.0572543825715\\
71.375	0.1662	9.98855997559185	9.98855997559185\\
71.375	0.16986	10.9454095559702	10.9454095559702\\
71.375	0.17352	11.9278031237065	11.9278031237065\\
71.375	0.17718	12.9357406788008	12.9357406788008\\
71.375	0.18084	13.969222221253	13.969222221253\\
71.375	0.1845	15.0282477510633	15.0282477510633\\
71.375	0.18816	16.1128172682315	16.1128172682315\\
71.375	0.19182	17.2229307727577	17.2229307727577\\
71.375	0.19548	18.3585882646419	18.3585882646419\\
71.375	0.19914	19.519789743884	19.519789743884\\
71.375	0.2028	20.7065352104842	20.7065352104842\\
71.375	0.20646	21.9188246644423	21.9188246644423\\
71.375	0.21012	23.1566581057584	23.1566581057584\\
71.375	0.21378	24.4200355344325	24.4200355344325\\
71.375	0.21744	25.7089569504646	25.7089569504646\\
71.375	0.2211	27.0234223538546	27.0234223538546\\
71.375	0.22476	28.3634317446026	28.3634317446026\\
71.375	0.22842	29.7289851227086	29.7289851227086\\
71.375	0.23208	31.1200824881726	31.1200824881726\\
71.375	0.23574	32.5367238409946	32.5367238409946\\
71.375	0.2394	33.9789091811745	33.9789091811745\\
71.375	0.24306	35.4466385087124	35.4466385087124\\
71.375	0.24672	36.9399118236083	36.9399118236083\\
71.375	0.25038	38.4587291258622	38.4587291258622\\
71.375	0.25404	40.003090415474	40.003090415474\\
71.375	0.2577	41.5729956924439	41.5729956924439\\
71.375	0.26136	43.1684449567717	43.1684449567717\\
71.375	0.26502	44.7894382084575	44.7894382084575\\
71.375	0.26868	46.4359754475012	46.4359754475012\\
71.375	0.27234	48.108056673903	48.108056673903\\
71.375	0.276	49.8056818876627	49.8056818876627\\
71.75	0.093	-3.80786335619897	-3.80786335619897\\
71.75	0.09666	-3.35466505721824	-3.35466505721824\\
71.75	0.10032	-2.87592277087955	-2.87592277087955\\
71.75	0.10398	-2.37163649718286	-2.37163649718286\\
71.75	0.10764	-1.84180623612819	-1.84180623612819\\
71.75	0.1113	-1.28643198771554	-1.28643198771554\\
71.75	0.11496	-0.705513751944906	-0.705513751944906\\
71.75	0.11862	-0.0990515288163127	-0.0990515288163127\\
71.75	0.12228	0.532954681670278	0.532954681670278\\
71.75	0.12594	1.19050487951483	1.19050487951483\\
71.75	0.1296	1.87359906471738	1.87359906471738\\
71.75	0.13326	2.5822372372779	2.5822372372779\\
71.75	0.13692	3.31641939719641	3.31641939719641\\
71.75	0.14058	4.0761455444729	4.0761455444729\\
71.75	0.14424	4.86141567910736	4.86141567910736\\
71.75	0.1479	5.6722298010998	5.6722298010998\\
71.75	0.15156	6.50858791045022	6.50858791045022\\
71.75	0.15522	7.37049000715863	7.37049000715863\\
71.75	0.15888	8.25793609122502	8.25793609122502\\
71.75	0.16254	9.17092616264938	9.17092616264938\\
71.75	0.1662	10.1094602214317	10.1094602214317\\
71.75	0.16986	11.073538267572	11.073538267572\\
71.75	0.17352	12.0631603010703	12.0631603010703\\
71.75	0.17718	13.0783263219266	13.0783263219266\\
71.75	0.18084	14.1190363301409	14.1190363301409\\
71.75	0.1845	15.1852903257131	15.1852903257131\\
71.75	0.18816	16.2770883086433	16.2770883086433\\
71.75	0.19182	17.3944302789315	17.3944302789315\\
71.75	0.19548	18.5373162365777	18.5373162365777\\
71.75	0.19914	19.7057461815819	19.7057461815819\\
71.75	0.2028	20.899720113944	20.899720113944\\
71.75	0.20646	22.1192380336641	22.1192380336641\\
71.75	0.21012	23.3642999407422	23.3642999407422\\
71.75	0.21378	24.6349058351783	24.6349058351783\\
71.75	0.21744	25.9310557169724	25.9310557169724\\
71.75	0.2211	27.2527495861244	27.2527495861244\\
71.75	0.22476	28.5999874426344	28.5999874426344\\
71.75	0.22842	29.9727692865024	29.9727692865024\\
71.75	0.23208	31.3710951177284	31.3710951177284\\
71.75	0.23574	32.7949649363123	32.7949649363123\\
71.75	0.2394	34.2443787422543	34.2443787422543\\
71.75	0.24306	35.7193365355542	35.7193365355542\\
71.75	0.24672	37.219838316212	37.219838316212\\
71.75	0.25038	38.745884084228	38.745884084228\\
71.75	0.25404	40.2974738396018	40.2974738396018\\
71.75	0.2577	41.8746075823336	41.8746075823336\\
71.75	0.26136	43.4772853124234	43.4772853124234\\
71.75	0.26502	45.1055070298712	45.1055070298712\\
71.75	0.26868	46.759272734677	46.759272734677\\
71.75	0.27234	48.4385824268407	48.4385824268407\\
71.75	0.276	50.1434361063625	50.1434361063625\\
72.125	0.093	-3.8291637526218	-3.8291637526218\\
72.125	0.09666	-3.36873698787907	-3.36873698787907\\
72.125	0.10032	-2.88276623577839	-2.88276623577839\\
72.125	0.10398	-2.3712514963197	-2.3712514963197\\
72.125	0.10764	-1.83419276950304	-1.83419276950304\\
72.125	0.1113	-1.27159005532839	-1.27159005532839\\
72.125	0.11496	-0.683443353795759	-0.683443353795759\\
72.125	0.11862	-0.0697526649051756	-0.0697526649051756\\
72.125	0.12228	0.569482011343412	0.569482011343412\\
72.125	0.12594	1.23426067494996	1.23426067494996\\
72.125	0.1296	1.9245833259145	1.9245833259145\\
72.125	0.13326	2.64044996423702	2.64044996423702\\
72.125	0.13692	3.38186058991752	3.38186058991752\\
72.125	0.14058	4.148815202956	4.148815202956\\
72.125	0.14424	4.94131380335245	4.94131380335245\\
72.125	0.1479	5.7593563911069	5.7593563911069\\
72.125	0.15156	6.60294296621931	6.60294296621931\\
72.125	0.15522	7.47207352868972	7.47207352868972\\
72.125	0.15888	8.3667480785181	8.3667480785181\\
72.125	0.16254	9.28696661570446	9.28696661570446\\
72.125	0.1662	10.2327291402488	10.2327291402488\\
72.125	0.16986	11.2040356521511	11.2040356521511\\
72.125	0.17352	12.2008861514114	12.2008861514114\\
72.125	0.17718	13.2232806380297	13.2232806380297\\
72.125	0.18084	14.2712191120059	14.2712191120059\\
72.125	0.1845	15.3447015733402	15.3447015733402\\
72.125	0.18816	16.4437280220324	16.4437280220324\\
72.125	0.19182	17.5682984580826	17.5682984580826\\
72.125	0.19548	18.7184128814907	18.7184128814907\\
72.125	0.19914	19.8940712922569	19.8940712922569\\
72.125	0.2028	21.0952736903811	21.0952736903811\\
72.125	0.20646	22.3220200758631	22.3220200758631\\
72.125	0.21012	23.5743104487032	23.5743104487032\\
72.125	0.21378	24.8521448089013	24.8521448089013\\
72.125	0.21744	26.1555231564574	26.1555231564574\\
72.125	0.2211	27.4844454913714	27.4844454913714\\
72.125	0.22476	28.8389118136434	28.8389118136434\\
72.125	0.22842	30.2189221232734	30.2189221232734\\
72.125	0.23208	31.6244764202613	31.6244764202613\\
72.125	0.23574	33.0555747046073	33.0555747046073\\
72.125	0.2394	34.5122169763112	34.5122169763112\\
72.125	0.24306	35.9944032353731	35.9944032353731\\
72.125	0.24672	37.502133481793	37.502133481793\\
72.125	0.25038	39.0354077155709	39.0354077155709\\
72.125	0.25404	40.5942259367067	40.5942259367067\\
72.125	0.2577	42.1785881452006	42.1785881452006\\
72.125	0.26136	43.7884943410523	43.7884943410523\\
72.125	0.26502	45.4239445242622	45.4239445242622\\
72.125	0.26868	47.0849386948299	47.0849386948299\\
72.125	0.27234	48.7714768527557	48.7714768527557\\
72.125	0.276	50.4835589980394	50.4835589980394\\
72.5	0.093	-3.84809547606747	-3.84809547606747\\
72.5	0.09666	-3.38044024556275	-3.38044024556275\\
72.5	0.10032	-2.88724102770004	-2.88724102770004\\
72.5	0.10398	-2.36849782247937	-2.36849782247937\\
72.5	0.10764	-1.82421062990072	-1.82421062990072\\
72.5	0.1113	-1.25437944996408	-1.25437944996408\\
72.5	0.11496	-0.659004282669464	-0.659004282669464\\
72.5	0.11862	-0.0380851280168688	-0.0380851280168688\\
72.5	0.12228	0.608378013993709	0.608378013993709\\
72.5	0.12594	1.28038514336227	1.28038514336227\\
72.5	0.1296	1.97793626008881	1.97793626008881\\
72.5	0.13326	2.70103136417331	2.70103136417331\\
72.5	0.13692	3.4496704556158	3.4496704556158\\
72.5	0.14058	4.22385353441628	4.22385353441628\\
72.5	0.14424	5.02358060057474	5.02358060057474\\
72.5	0.1479	5.84885165409116	5.84885165409116\\
72.5	0.15156	6.69966669496558	6.69966669496558\\
72.5	0.15522	7.57602572319797	7.57602572319797\\
72.5	0.15888	8.47792873878835	8.47792873878835\\
72.5	0.16254	9.4053757417367	9.4053757417367\\
72.5	0.1662	10.358366732043	10.358366732043\\
72.5	0.16986	11.3369017097073	11.3369017097073\\
72.5	0.17352	12.3409806747296	12.3409806747296\\
72.5	0.17718	13.3706036271099	13.3706036271099\\
72.5	0.18084	14.4257705668482	14.4257705668482\\
72.5	0.1845	15.5064814939444	15.5064814939444\\
72.5	0.18816	16.6127364083986	16.6127364083986\\
72.5	0.19182	17.7445353102108	17.7445353102108\\
72.5	0.19548	18.9018781993809	18.9018781993809\\
72.5	0.19914	20.0847650759091	20.0847650759091\\
72.5	0.2028	21.2931959397952	21.2931959397952\\
72.5	0.20646	22.5271707910393	22.5271707910393\\
72.5	0.21012	23.7866896296414	23.7866896296414\\
72.5	0.21378	25.0717524556015	25.0717524556015\\
72.5	0.21744	26.3823592689195	26.3823592689195\\
72.5	0.2211	27.7185100695955	27.7185100695955\\
72.5	0.22476	29.0802048576295	29.0802048576295\\
72.5	0.22842	30.4674436330215	30.4674436330215\\
72.5	0.23208	31.8802263957715	31.8802263957715\\
72.5	0.23574	33.3185531458795	33.3185531458795\\
72.5	0.2394	34.7824238833454	34.7824238833454\\
72.5	0.24306	36.2718386081693	36.2718386081693\\
72.5	0.24672	37.7867973203512	37.7867973203512\\
72.5	0.25038	39.327300019891	39.327300019891\\
72.5	0.25404	40.8933467067889	40.8933467067889\\
72.5	0.2577	42.4849373810447	42.4849373810447\\
72.5	0.26136	44.1020720426585	44.1020720426585\\
72.5	0.26502	45.7447506916303	45.7447506916303\\
72.5	0.26868	47.41297332796	47.41297332796\\
72.5	0.27234	49.1067399516478	49.1067399516478\\
72.5	0.276	50.8260505626934	50.8260505626934\\
72.875	0.093	-3.86465852653593	-3.86465852653593\\
72.875	0.09666	-3.38977483026921	-3.38977483026921\\
72.875	0.10032	-2.88934714664451	-2.88934714664451\\
72.875	0.10398	-2.36337547566183	-2.36337547566183\\
72.875	0.10764	-1.8118598173212	-1.8118598173212\\
72.875	0.1113	-1.23480017162256	-1.23480017162256\\
72.875	0.11496	-0.63219653856595	-0.63219653856595\\
72.875	0.11862	-0.00404891815135677	-0.00404891815135677\\
72.875	0.12228	0.649642689621219	0.649642689621219\\
72.875	0.12594	1.32887828475177	1.32887828475177\\
72.875	0.1296	2.0336578672403	2.0336578672403\\
72.875	0.13326	2.7639814370868	2.7639814370868\\
72.875	0.13692	3.51984899429129	3.51984899429129\\
72.875	0.14058	4.30126053885376	4.30126053885376\\
72.875	0.14424	5.10821607077422	5.10821607077422\\
72.875	0.1479	5.94071559005263	5.94071559005263\\
72.875	0.15156	6.79875909668904	6.79875909668904\\
72.875	0.15522	7.68234659068343	7.68234659068343\\
72.875	0.15888	8.59147807203581	8.59147807203581\\
72.875	0.16254	9.52615354074615	9.52615354074615\\
72.875	0.1662	10.4863729968145	10.4863729968145\\
72.875	0.16986	11.4721364402408	11.4721364402408\\
72.875	0.17352	12.4834438710251	12.4834438710251\\
72.875	0.17718	13.5202952891673	13.5202952891673\\
72.875	0.18084	14.5826906946676	14.5826906946676\\
72.875	0.1845	15.6706300875258	15.6706300875258\\
72.875	0.18816	16.784113467742	16.784113467742\\
72.875	0.19182	17.9231408353162	17.9231408353162\\
72.875	0.19548	19.0877121902484	19.0877121902484\\
72.875	0.19914	20.2778275325385	20.2778275325385\\
72.875	0.2028	21.4934868621866	21.4934868621866\\
72.875	0.20646	22.7346901791927	22.7346901791927\\
72.875	0.21012	24.0014374835568	24.0014374835568\\
72.875	0.21378	25.2937287752789	25.2937287752789\\
72.875	0.21744	26.6115640543589	26.6115640543589\\
72.875	0.2211	27.9549433207969	27.9549433207969\\
72.875	0.22476	29.3238665745929	29.3238665745929\\
72.875	0.22842	30.7183338157469	30.7183338157469\\
72.875	0.23208	32.1383450442589	32.1383450442589\\
72.875	0.23574	33.5839002601288	33.5839002601288\\
72.875	0.2394	35.0549994633567	35.0549994633567\\
72.875	0.24306	36.5516426539426	36.5516426539426\\
72.875	0.24672	38.0738298318865	38.0738298318865\\
72.875	0.25038	39.6215609971884	39.6215609971884\\
72.875	0.25404	41.1948361498482	41.1948361498482\\
72.875	0.2577	42.793655289866	42.793655289866\\
72.875	0.26136	44.4180184172418	44.4180184172418\\
72.875	0.26502	46.0679255319756	46.0679255319756\\
72.875	0.26868	47.7433766340673	47.7433766340673\\
72.875	0.27234	49.4443717235171	49.4443717235171\\
72.875	0.276	51.1709108003248	51.1709108003248\\
73.25	0.093	-3.8788529040272	-3.8788529040272\\
73.25	0.09666	-3.39674074199849	-3.39674074199849\\
73.25	0.10032	-2.88908459261179	-2.88908459261179\\
73.25	0.10398	-2.35588445586712	-2.35588445586712\\
73.25	0.10764	-1.79714033176448	-1.79714033176448\\
73.25	0.1113	-1.21285222030386	-1.21285222030386\\
73.25	0.11496	-0.603020121485251	-0.603020121485251\\
73.25	0.11862	0.0323559646913392	0.0323559646913392\\
73.25	0.12228	0.693276038225912	0.693276038225912\\
73.25	0.12594	1.37974009911846	1.37974009911846\\
73.25	0.1296	2.09174814736899	2.09174814736899\\
73.25	0.13326	2.82930018297747	2.82930018297747\\
73.25	0.13692	3.59239620594396	3.59239620594396\\
73.25	0.14058	4.38103621626843	4.38103621626843\\
73.25	0.14424	5.19522021395087	5.19522021395087\\
73.25	0.1479	6.03494819899129	6.03494819899129\\
73.25	0.15156	6.90022017138969	6.90022017138969\\
73.25	0.15522	7.79103613114608	7.79103613114608\\
73.25	0.15888	8.70739607826044	8.70739607826044\\
73.25	0.16254	9.64930001273279	9.64930001273279\\
73.25	0.1662	10.6167479345631	10.6167479345631\\
73.25	0.16986	11.6097398437514	11.6097398437514\\
73.25	0.17352	12.6282757402977	12.6282757402977\\
73.25	0.17718	13.6723556242019	13.6723556242019\\
73.25	0.18084	14.7419794954642	14.7419794954642\\
73.25	0.1845	15.8371473540844	15.8371473540844\\
73.25	0.18816	16.9578592000626	16.9578592000626\\
73.25	0.19182	18.1041150333988	18.1041150333988\\
73.25	0.19548	19.2759148540929	19.2759148540929\\
73.25	0.19914	20.4732586621451	20.4732586621451\\
73.25	0.2028	21.6961464575552	21.6961464575552\\
73.25	0.20646	22.9445782403233	22.9445782403233\\
73.25	0.21012	24.2185540104494	24.2185540104494\\
73.25	0.21378	25.5180737679334	25.5180737679334\\
73.25	0.21744	26.8431375127755	26.8431375127755\\
73.25	0.2211	28.1937452449755	28.1937452449755\\
73.25	0.22476	29.5698969645335	29.5698969645335\\
73.25	0.22842	30.9715926714494	30.9715926714494\\
73.25	0.23208	32.3988323657234	32.3988323657234\\
73.25	0.23574	33.8516160473553	33.8516160473553\\
73.25	0.2394	35.3299437163452	35.3299437163452\\
73.25	0.24306	36.8338153726931	36.8338153726931\\
73.25	0.24672	38.363231016399	38.363231016399\\
73.25	0.25038	39.9181906474629	39.9181906474629\\
73.25	0.25404	41.4986942658847	41.4986942658847\\
73.25	0.2577	43.1047418716645	43.1047418716645\\
73.25	0.26136	44.7363334648023	44.7363334648023\\
73.25	0.26502	46.3934690452981	46.3934690452981\\
73.25	0.26868	48.0761486131518	48.0761486131518\\
73.25	0.27234	49.7843721683635	49.7843721683635\\
73.25	0.276	51.5181397109332	51.5181397109332\\
73.625	0.093	-3.89067860854127	-3.89067860854127\\
73.625	0.09666	-3.40133798075057	-3.40133798075057\\
73.625	0.10032	-2.88645336560188	-2.88645336560188\\
73.625	0.10398	-2.34602476309521	-2.34602476309521\\
73.625	0.10764	-1.78005217323058	-1.78005217323058\\
73.625	0.1113	-1.18853559600796	-1.18853559600796\\
73.625	0.11496	-0.571475031427354	-0.571475031427354\\
73.625	0.11862	0.0711295205112261	0.0711295205112261\\
73.625	0.12228	0.739278059807789	0.739278059807789\\
73.625	0.12594	1.43297058646233	1.43297058646233\\
73.625	0.1296	2.15220710047486	2.15220710047486\\
73.625	0.13326	2.89698760184534	2.89698760184534\\
73.625	0.13692	3.66731209057382	3.66731209057382\\
73.625	0.14058	4.46318056666029	4.46318056666029\\
73.625	0.14424	5.28459303010473	5.28459303010473\\
73.625	0.1479	6.13154948090713	6.13154948090713\\
73.625	0.15156	7.00404991906753	7.00404991906753\\
73.625	0.15522	7.90209434458592	7.90209434458592\\
73.625	0.15888	8.82568275746227	8.82568275746227\\
73.625	0.16254	9.77481515769661	9.77481515769661\\
73.625	0.1662	10.7494915452889	10.7494915452889\\
73.625	0.16986	11.7497119202392	11.7497119202392\\
73.625	0.17352	12.7754762825475	12.7754762825475\\
73.625	0.17718	13.8267846322138	13.8267846322138\\
73.625	0.18084	14.903636969238	14.903636969238\\
73.625	0.1845	16.0060332936202	16.0060332936202\\
73.625	0.18816	17.1339736053604	17.1339736053604\\
73.625	0.19182	18.2874579044585	18.2874579044585\\
73.625	0.19548	19.4664861909147	19.4664861909147\\
73.625	0.19914	20.6710584647288	20.6710584647288\\
73.625	0.2028	21.901174725901	21.901174725901\\
73.625	0.20646	23.1568349744311	23.1568349744311\\
73.625	0.21012	24.4380392103191	24.4380392103191\\
73.625	0.21378	25.7447874335652	25.7447874335652\\
73.625	0.21744	27.0770796441692	27.0770796441692\\
73.625	0.2211	28.4349158421312	28.4349158421312\\
73.625	0.22476	29.8182960274512	29.8182960274512\\
73.625	0.22842	31.2272202001292	31.2272202001292\\
73.625	0.23208	32.6616883601651	32.6616883601651\\
73.625	0.23574	34.121700507559	34.121700507559\\
73.625	0.2394	35.607256642311	35.607256642311\\
73.625	0.24306	37.1183567644208	37.1183567644208\\
73.625	0.24672	38.6550008738887	38.6550008738887\\
73.625	0.25038	40.2171889707146	40.2171889707146\\
73.625	0.25404	41.8049210548984	41.8049210548984\\
73.625	0.2577	43.4181971264402	43.4181971264402\\
73.625	0.26136	45.05701718534	45.05701718534\\
73.625	0.26502	46.7213812315977	46.7213812315977\\
73.625	0.26868	48.4112892652135	48.4112892652135\\
73.625	0.27234	50.1267412861872	50.1267412861872\\
73.625	0.276	51.8677372945189	51.8677372945189\\
74	0.093	-3.90013564007816	-3.90013564007816\\
74	0.09666	-3.40356654652546	-3.40356654652546\\
74	0.10032	-2.88145346561477	-2.88145346561477\\
74	0.10398	-2.33379639734611	-2.33379639734611\\
74	0.10764	-1.76059534171949	-1.76059534171949\\
74	0.1113	-1.16185029873487	-1.16185029873487\\
74	0.11496	-0.537561268392274	-0.537561268392274\\
74	0.11862	0.112271749308304	0.112271749308304\\
74	0.12228	0.787648754366865	0.787648754366865\\
74	0.12594	1.4885697467834	1.4885697467834\\
74	0.1296	2.21503472655791	2.21503472655791\\
74	0.13326	2.9670436936904	2.9670436936904\\
74	0.13692	3.74459664818087	3.74459664818087\\
74	0.14058	4.54769359002934	4.54769359002934\\
74	0.14424	5.37633451923577	5.37633451923577\\
74	0.1479	6.23051943580017	6.23051943580017\\
74	0.15156	7.11024833972256	7.11024833972256\\
74	0.15522	8.01552123100294	8.01552123100294\\
74	0.15888	8.9463381096413	8.9463381096413\\
74	0.16254	9.90269897563763	9.90269897563763\\
74	0.1662	10.8846038289919	10.8846038289919\\
74	0.16986	11.8920526697042	11.8920526697042\\
74	0.17352	12.9250454977745	12.9250454977745\\
74	0.17718	13.9835823132027	13.9835823132027\\
74	0.18084	15.067663115989	15.067663115989\\
74	0.1845	16.1772879061332	16.1772879061332\\
74	0.18816	17.3124566836354	17.3124566836354\\
74	0.19182	18.4731694484956	18.4731694484956\\
74	0.19548	19.6594262007137	19.6594262007137\\
74	0.19914	20.8712269402898	20.8712269402898\\
74	0.2028	22.1085716672239	22.1085716672239\\
74	0.20646	23.371460381516	23.371460381516\\
74	0.21012	24.6598930831661	24.6598930831661\\
74	0.21378	25.9738697721741	25.9738697721741\\
74	0.21744	27.3133904485402	27.3133904485402\\
74	0.2211	28.6784551122641	28.6784551122641\\
74	0.22476	30.0690637633461	30.0690637633461\\
74	0.22842	31.4852164017861	31.4852164017861\\
74	0.23208	32.926913027584	32.926913027584\\
74	0.23574	34.39415364074	34.39415364074\\
74	0.2394	35.8869382412539	35.8869382412539\\
74	0.24306	37.4052668291258	37.4052668291258\\
74	0.24672	38.9491394043556	38.9491394043556\\
74	0.25038	40.5185559669435	40.5185559669435\\
74	0.25404	42.1135165168893	42.1135165168893\\
74	0.2577	43.7340210541931	43.7340210541931\\
74	0.26136	45.3800695788549	45.3800695788549\\
74	0.26502	47.0516620908746	47.0516620908746\\
74	0.26868	48.7487985902524	48.7487985902524\\
74	0.27234	50.4714790769881	50.4714790769881\\
74	0.276	52.2197035510817	52.2197035510817\\
};
\end{axis}

\begin{axis}[%
width=6.159cm,
height=3.097cm,
at={(8.104cm,4.301cm)},
scale only axis,
xmin=56,
xmax=74,
tick align=outside,
xlabel style={font=\color{white!15!black}},
xlabel={$L_{cut}$},
ymin=0.093,
ymax=0.276,
ylabel style={font=\color{white!15!black}},
ylabel={$D_{rlx}$},
zmin=-1328.01949078619,
zmax=0,
zlabel style={font=\color{white!15!black}},
zlabel={$u(t)$},
view={-140}{50},
axis background/.style={fill=white},
xmajorgrids,
ymajorgrids,
zmajorgrids
]
\addplot3[only marks, mark=*, mark options={}, mark size=1.5000pt, color=mycolor1, fill=mycolor1] table[row sep=crcr]{%
x	y	z\\
74	0.123	-171.244880190644\\
72	0.113	-139.217460396701\\
61	0.095	-69.6051966848396\\
56	0.093	-73.6870961160989\\
};
\addplot3[only marks, mark=*, mark options={}, mark size=1.5000pt, color=mycolor2, fill=mycolor2] table[row sep=crcr]{%
x	y	z\\
67	0.276	-1273.01999820883\\
66	0.255	-1046.38133574319\\
62	0.209	-579.722288788632\\
57	0.193	-465.590859275979\\
};
\addplot3[only marks, mark=*, mark options={}, mark size=1.5000pt, color=black, fill=black] table[row sep=crcr]{%
x	y	z\\
69	0.104	-102.29914898412\\
};
\addplot3[only marks, mark=*, mark options={}, mark size=1.5000pt, color=black, fill=black] table[row sep=crcr]{%
x	y	z\\
64	0.23	-773.899941054326\\
};

\addplot3[%
surf,
fill opacity=0.7, shader=interp, colormap={mymap}{[1pt] rgb(0pt)=(1,0.905882,0); rgb(1pt)=(1,0.901964,0); rgb(2pt)=(1,0.898051,0); rgb(3pt)=(1,0.894144,0); rgb(4pt)=(1,0.890243,0); rgb(5pt)=(1,0.886349,0); rgb(6pt)=(1,0.88246,0); rgb(7pt)=(1,0.878577,0); rgb(8pt)=(1,0.8747,0); rgb(9pt)=(1,0.870829,0); rgb(10pt)=(1,0.866964,0); rgb(11pt)=(1,0.863106,0); rgb(12pt)=(1,0.859253,0); rgb(13pt)=(1,0.855406,0); rgb(14pt)=(1,0.851566,0); rgb(15pt)=(1,0.847732,0); rgb(16pt)=(1,0.843903,0); rgb(17pt)=(1,0.840081,0); rgb(18pt)=(1,0.836265,0); rgb(19pt)=(1,0.832455,0); rgb(20pt)=(1,0.828652,0); rgb(21pt)=(1,0.824854,0); rgb(22pt)=(1,0.821063,0); rgb(23pt)=(1,0.817278,0); rgb(24pt)=(1,0.8135,0); rgb(25pt)=(1,0.809727,0); rgb(26pt)=(1,0.805961,0); rgb(27pt)=(1,0.8022,0); rgb(28pt)=(1,0.798445,0); rgb(29pt)=(1,0.794696,0); rgb(30pt)=(1,0.790953,0); rgb(31pt)=(1,0.787215,0); rgb(32pt)=(1,0.783484,0); rgb(33pt)=(1,0.779758,0); rgb(34pt)=(1,0.776038,0); rgb(35pt)=(1,0.772324,0); rgb(36pt)=(1,0.768615,0); rgb(37pt)=(1,0.764913,0); rgb(38pt)=(1,0.761217,0); rgb(39pt)=(1,0.757527,0); rgb(40pt)=(1,0.753843,0); rgb(41pt)=(1,0.750165,0); rgb(42pt)=(1,0.746493,0); rgb(43pt)=(1,0.742827,0); rgb(44pt)=(1,0.739167,0); rgb(45pt)=(1,0.735514,0); rgb(46pt)=(1,0.731867,0); rgb(47pt)=(1,0.728226,0); rgb(48pt)=(1,0.724591,0); rgb(49pt)=(1,0.720963,0); rgb(50pt)=(1,0.717341,0); rgb(51pt)=(1,0.713725,0); rgb(52pt)=(0.999994,0.710077,0); rgb(53pt)=(0.999974,0.706363,0); rgb(54pt)=(0.999942,0.702592,0); rgb(55pt)=(0.999898,0.698775,0); rgb(56pt)=(0.999841,0.694921,0); rgb(57pt)=(0.999771,0.691039,0); rgb(58pt)=(0.99969,0.687139,0); rgb(59pt)=(0.999596,0.68323,0); rgb(60pt)=(0.99949,0.679323,0); rgb(61pt)=(0.999372,0.675427,0); rgb(62pt)=(0.999242,0.67155,0); rgb(63pt)=(0.9991,0.667704,0); rgb(64pt)=(0.998946,0.663897,0); rgb(65pt)=(0.998781,0.660138,0); rgb(66pt)=(0.998605,0.656439,0); rgb(67pt)=(0.998416,0.652807,0); rgb(68pt)=(0.998217,0.649253,0); rgb(69pt)=(0.998006,0.645786,0); rgb(70pt)=(0.997785,0.642416,0); rgb(71pt)=(0.997552,0.639152,0); rgb(72pt)=(0.997308,0.636004,0); rgb(73pt)=(0.997053,0.632982,0); rgb(74pt)=(0.996788,0.630095,0); rgb(75pt)=(0.996512,0.627352,0); rgb(76pt)=(0.996226,0.624763,0); rgb(77pt)=(0.995851,0.622329,0); rgb(78pt)=(0.99494,0.619997,0); rgb(79pt)=(0.99345,0.617753,0); rgb(80pt)=(0.991419,0.61559,0); rgb(81pt)=(0.988885,0.613503,0); rgb(82pt)=(0.985886,0.611486,0); rgb(83pt)=(0.98246,0.609532,0); rgb(84pt)=(0.978643,0.607636,0); rgb(85pt)=(0.974475,0.605791,0); rgb(86pt)=(0.969992,0.603992,0); rgb(87pt)=(0.965232,0.602233,0); rgb(88pt)=(0.960233,0.600507,0); rgb(89pt)=(0.955033,0.598808,0); rgb(90pt)=(0.949669,0.59713,0); rgb(91pt)=(0.94418,0.595468,0); rgb(92pt)=(0.938602,0.593815,0); rgb(93pt)=(0.932974,0.592166,0); rgb(94pt)=(0.927333,0.590513,0); rgb(95pt)=(0.921717,0.588852,0); rgb(96pt)=(0.916164,0.587176,0); rgb(97pt)=(0.910711,0.585479,0); rgb(98pt)=(0.905397,0.583755,0); rgb(99pt)=(0.900258,0.581999,0); rgb(100pt)=(0.895333,0.580203,0); rgb(101pt)=(0.890659,0.578362,0); rgb(102pt)=(0.886275,0.576471,0); rgb(103pt)=(0.882047,0.574545,0); rgb(104pt)=(0.877819,0.572608,0); rgb(105pt)=(0.873592,0.57066,0); rgb(106pt)=(0.869366,0.568701,0); rgb(107pt)=(0.865143,0.566733,0); rgb(108pt)=(0.860924,0.564756,0); rgb(109pt)=(0.856708,0.562771,0); rgb(110pt)=(0.852497,0.560778,0); rgb(111pt)=(0.848292,0.558779,0); rgb(112pt)=(0.844092,0.556774,0); rgb(113pt)=(0.8399,0.554763,0); rgb(114pt)=(0.835716,0.552749,0); rgb(115pt)=(0.831541,0.55073,0); rgb(116pt)=(0.827374,0.548709,0); rgb(117pt)=(0.823219,0.546686,0); rgb(118pt)=(0.819074,0.54466,0); rgb(119pt)=(0.81494,0.542635,0); rgb(120pt)=(0.81082,0.540609,0); rgb(121pt)=(0.806712,0.538584,0); rgb(122pt)=(0.802619,0.53656,0); rgb(123pt)=(0.798541,0.534539,0); rgb(124pt)=(0.794478,0.532521,0); rgb(125pt)=(0.790431,0.530506,0); rgb(126pt)=(0.786402,0.528496,0); rgb(127pt)=(0.782391,0.526491,0); rgb(128pt)=(0.77841,0.524489,0); rgb(129pt)=(0.774523,0.522478,0); rgb(130pt)=(0.770731,0.520455,0); rgb(131pt)=(0.767022,0.518424,0); rgb(132pt)=(0.763384,0.516385,0); rgb(133pt)=(0.759804,0.514339,0); rgb(134pt)=(0.756272,0.51229,0); rgb(135pt)=(0.752775,0.510237,0); rgb(136pt)=(0.749302,0.508182,0); rgb(137pt)=(0.74584,0.506128,0); rgb(138pt)=(0.742378,0.504075,0); rgb(139pt)=(0.738904,0.502025,0); rgb(140pt)=(0.735406,0.499979,0); rgb(141pt)=(0.731872,0.49794,0); rgb(142pt)=(0.72829,0.495909,0); rgb(143pt)=(0.724649,0.493887,0); rgb(144pt)=(0.720936,0.491875,0); rgb(145pt)=(0.71714,0.489876,0); rgb(146pt)=(0.713249,0.487891,0); rgb(147pt)=(0.709251,0.485921,0); rgb(148pt)=(0.705134,0.483968,0); rgb(149pt)=(0.700887,0.482033,0); rgb(150pt)=(0.696497,0.480118,0); rgb(151pt)=(0.691952,0.478225,0); rgb(152pt)=(0.687242,0.476355,0); rgb(153pt)=(0.682353,0.47451,0); rgb(154pt)=(0.677195,0.472696,0); rgb(155pt)=(0.6717,0.470916,0); rgb(156pt)=(0.665891,0.469169,0); rgb(157pt)=(0.659791,0.46745,0); rgb(158pt)=(0.653423,0.465756,0); rgb(159pt)=(0.64681,0.464084,0); rgb(160pt)=(0.639976,0.462432,0); rgb(161pt)=(0.632943,0.460795,0); rgb(162pt)=(0.625734,0.459171,0); rgb(163pt)=(0.618373,0.457556,0); rgb(164pt)=(0.610882,0.455948,0); rgb(165pt)=(0.603284,0.454343,0); rgb(166pt)=(0.595604,0.452737,0); rgb(167pt)=(0.587863,0.451129,0); rgb(168pt)=(0.580084,0.449514,0); rgb(169pt)=(0.572292,0.447889,0); rgb(170pt)=(0.564508,0.446252,0); rgb(171pt)=(0.556756,0.444599,0); rgb(172pt)=(0.549059,0.442927,0); rgb(173pt)=(0.54144,0.441232,0); rgb(174pt)=(0.533922,0.439512,0); rgb(175pt)=(0.526529,0.437764,0); rgb(176pt)=(0.519282,0.435983,0); rgb(177pt)=(0.512206,0.434168,0); rgb(178pt)=(0.505323,0.432315,0); rgb(179pt)=(0.498628,0.430422,3.92506e-06); rgb(180pt)=(0.491973,0.428504,3.49981e-05); rgb(181pt)=(0.485331,0.426562,9.63073e-05); rgb(182pt)=(0.478704,0.424596,0.000186979); rgb(183pt)=(0.472096,0.422609,0.000306141); rgb(184pt)=(0.465508,0.420599,0.00045292); rgb(185pt)=(0.458942,0.418567,0.000626441); rgb(186pt)=(0.452401,0.416515,0.000825833); rgb(187pt)=(0.445885,0.414441,0.00105022); rgb(188pt)=(0.439399,0.412348,0.00129873); rgb(189pt)=(0.432942,0.410234,0.00157049); rgb(190pt)=(0.426518,0.408102,0.00186463); rgb(191pt)=(0.420129,0.40595,0.00218028); rgb(192pt)=(0.413777,0.40378,0.00251655); rgb(193pt)=(0.407464,0.401592,0.00287258); rgb(194pt)=(0.401191,0.399386,0.00324749); rgb(195pt)=(0.394962,0.397164,0.00364042); rgb(196pt)=(0.388777,0.394925,0.00405048); rgb(197pt)=(0.38264,0.39267,0.00447681); rgb(198pt)=(0.376552,0.390399,0.00491852); rgb(199pt)=(0.370516,0.388113,0.00537476); rgb(200pt)=(0.364532,0.385812,0.00584464); rgb(201pt)=(0.358605,0.383497,0.00632729); rgb(202pt)=(0.352735,0.381168,0.00682184); rgb(203pt)=(0.346925,0.378826,0.00732741); rgb(204pt)=(0.341176,0.376471,0.00784314); rgb(205pt)=(0.335485,0.374093,0.00847245); rgb(206pt)=(0.329843,0.371682,0.00930909); rgb(207pt)=(0.324249,0.369242,0.0103377); rgb(208pt)=(0.318701,0.366772,0.0115428); rgb(209pt)=(0.313198,0.364275,0.0129091); rgb(210pt)=(0.307739,0.361753,0.0144211); rgb(211pt)=(0.302322,0.359206,0.0160634); rgb(212pt)=(0.296945,0.356637,0.0178207); rgb(213pt)=(0.291607,0.354048,0.0196776); rgb(214pt)=(0.286307,0.35144,0.0216186); rgb(215pt)=(0.281043,0.348814,0.0236284); rgb(216pt)=(0.275813,0.346172,0.0256916); rgb(217pt)=(0.270616,0.343517,0.0277927); rgb(218pt)=(0.265451,0.340849,0.0299163); rgb(219pt)=(0.260317,0.33817,0.0320472); rgb(220pt)=(0.25521,0.335482,0.0341698); rgb(221pt)=(0.250131,0.332786,0.0362688); rgb(222pt)=(0.245078,0.330085,0.0383287); rgb(223pt)=(0.240048,0.327379,0.0403343); rgb(224pt)=(0.235042,0.324671,0.04227); rgb(225pt)=(0.230056,0.321962,0.0441205); rgb(226pt)=(0.22509,0.319254,0.0458704); rgb(227pt)=(0.220142,0.316548,0.0475043); rgb(228pt)=(0.215212,0.313846,0.0490067); rgb(229pt)=(0.210296,0.311149,0.0503624); rgb(230pt)=(0.205395,0.308459,0.0515759); rgb(231pt)=(0.200514,0.305763,0.052757); rgb(232pt)=(0.195655,0.303061,0.0539242); rgb(233pt)=(0.190817,0.300353,0.0550763); rgb(234pt)=(0.186001,0.297639,0.0562123); rgb(235pt)=(0.181207,0.294918,0.0573313); rgb(236pt)=(0.176434,0.292191,0.0584321); rgb(237pt)=(0.171685,0.289458,0.0595136); rgb(238pt)=(0.166957,0.286719,0.060575); rgb(239pt)=(0.162252,0.283973,0.0616151); rgb(240pt)=(0.15757,0.281221,0.0626328); rgb(241pt)=(0.152911,0.278463,0.0636271); rgb(242pt)=(0.148275,0.275699,0.0645971); rgb(243pt)=(0.143663,0.272929,0.0655416); rgb(244pt)=(0.139074,0.270152,0.0664596); rgb(245pt)=(0.134508,0.26737,0.06735); rgb(246pt)=(0.129967,0.264581,0.0682118); rgb(247pt)=(0.125449,0.261787,0.0690441); rgb(248pt)=(0.120956,0.258986,0.0698456); rgb(249pt)=(0.116487,0.25618,0.0706154); rgb(250pt)=(0.112043,0.253367,0.0713525); rgb(251pt)=(0.107623,0.250549,0.0720557); rgb(252pt)=(0.103229,0.247724,0.0727241); rgb(253pt)=(0.0988592,0.244894,0.0733566); rgb(254pt)=(0.0945149,0.242058,0.0739522); rgb(255pt)=(0.0901961,0.239216,0.0745098)}, mesh/rows=49]
table[row sep=crcr, point meta=\thisrow{c}] {%
%
x	y	z	c\\
56	0.093	-73.0247303221439	-73.0247303221439\\
56	0.09666	-76.5488919207614	-76.5488919207614\\
56	0.10032	-80.8947938841944	-80.8947938841944\\
56	0.10398	-86.0624362124423	-86.0624362124423\\
56	0.10764	-92.0518189055054	-92.0518189055054\\
56	0.1113	-98.8629419633837	-98.8629419633837\\
56	0.11496	-106.495805386077	-106.495805386077\\
56	0.11862	-114.950409173586	-114.950409173586\\
56	0.12228	-124.22675332591	-124.22675332591\\
56	0.12594	-134.324837843049	-134.324837843049\\
56	0.1296	-145.244662725003	-145.244662725003\\
56	0.13326	-156.986227971772	-156.986227971772\\
56	0.13692	-169.549533583356	-169.549533583356\\
56	0.14058	-182.934579559756	-182.934579559756\\
56	0.14424	-197.141365900971	-197.141365900971\\
56	0.1479	-212.169892607001	-212.169892607001\\
56	0.15156	-228.020159677847	-228.020159677847\\
56	0.15522	-244.692167113507	-244.692167113507\\
56	0.15888	-262.185914913982	-262.185914913982\\
56	0.16254	-280.501403079273	-280.501403079273\\
56	0.1662	-299.638631609379	-299.638631609379\\
56	0.16986	-319.597600504299	-319.597600504299\\
56	0.17352	-340.378309764036	-340.378309764036\\
56	0.17718	-361.980759388587	-361.980759388587\\
56	0.18084	-384.404949377953	-384.404949377953\\
56	0.1845	-407.650879732135	-407.650879732135\\
56	0.18816	-431.718550451132	-431.718550451132\\
56	0.19182	-456.607961534944	-456.607961534944\\
56	0.19548	-482.319112983571	-482.319112983571\\
56	0.19914	-508.852004797013	-508.852004797013\\
56	0.2028	-536.206636975271	-536.206636975271\\
56	0.20646	-564.383009518343	-564.383009518343\\
56	0.21012	-593.381122426231	-593.381122426231\\
56	0.21378	-623.200975698934	-623.200975698934\\
56	0.21744	-653.842569336452	-653.842569336452\\
56	0.2211	-685.305903338785	-685.305903338785\\
56	0.22476	-717.590977705934	-717.590977705934\\
56	0.22842	-750.697792437897	-750.697792437897\\
56	0.23208	-784.626347534676	-784.626347534676\\
56	0.23574	-819.37664299627	-819.37664299627\\
56	0.2394	-854.948678822679	-854.948678822679\\
56	0.24306	-891.342455013903	-891.342455013903\\
56	0.24672	-928.557971569943	-928.557971569943\\
56	0.25038	-966.595228490798	-966.595228490798\\
56	0.25404	-1005.45422577647	-1005.45422577647\\
56	0.2577	-1045.13496342695	-1045.13496342695\\
56	0.26136	-1085.63744144225	-1085.63744144225\\
56	0.26502	-1126.96165982237	-1126.96165982237\\
56	0.26868	-1169.1076185673	-1169.1076185673\\
56	0.27234	-1212.07531767704	-1212.07531767704\\
56	0.276	-1255.8647571516	-1255.8647571516\\
56.375	0.093	-72.2605324617118	-72.2605324617118\\
56.375	0.09666	-75.7966567181423	-75.7966567181423\\
56.375	0.10032	-80.1545213393885	-80.1545213393885\\
56.375	0.10398	-85.3341263254496	-85.3341263254496\\
56.375	0.10764	-91.3354716763255	-91.3354716763255\\
56.375	0.1113	-98.1585573920171	-98.1585573920171\\
56.375	0.11496	-105.803383472524	-105.803383472524\\
56.375	0.11862	-114.269949917845	-114.269949917845\\
56.375	0.12228	-123.558256727983	-123.558256727983\\
56.375	0.12594	-133.668303902934	-133.668303902934\\
56.375	0.1296	-144.600091442702	-144.600091442702\\
56.375	0.13326	-156.353619347284	-156.353619347284\\
56.375	0.13692	-168.928887616681	-168.928887616681\\
56.375	0.14058	-182.325896250894	-182.325896250894\\
56.375	0.14424	-196.544645249922	-196.544645249922\\
56.375	0.1479	-211.585134613765	-211.585134613765\\
56.375	0.15156	-227.447364342424	-227.447364342424\\
56.375	0.15522	-244.131334435897	-244.131334435897\\
56.375	0.15888	-261.637044894186	-261.637044894186\\
56.375	0.16254	-279.964495717289	-279.964495717289\\
56.375	0.1662	-299.113686905208	-299.113686905208\\
56.375	0.16986	-319.084618457942	-319.084618457942\\
56.375	0.17352	-339.877290375491	-339.877290375491\\
56.375	0.17718	-361.491702657856	-361.491702657856\\
56.375	0.18084	-383.927855305035	-383.927855305035\\
56.375	0.1845	-407.18574831703	-407.18574831703\\
56.375	0.18816	-431.26538169384	-431.26538169384\\
56.375	0.19182	-456.166755435465	-456.166755435465\\
56.375	0.19548	-481.889869541905	-481.889869541905\\
56.375	0.19914	-508.43472401316	-508.43472401316\\
56.375	0.2028	-535.801318849231	-535.801318849231\\
56.375	0.20646	-563.989654050117	-563.989654050117\\
56.375	0.21012	-592.999729615818	-592.999729615818\\
56.375	0.21378	-622.831545546334	-622.831545546334\\
56.375	0.21744	-653.485101841665	-653.485101841665\\
56.375	0.2211	-684.960398501811	-684.960398501811\\
56.375	0.22476	-717.257435526772	-717.257435526772\\
56.375	0.22842	-750.376212916549	-750.376212916549\\
56.375	0.23208	-784.316730671141	-784.316730671141\\
56.375	0.23574	-819.078988790548	-819.078988790548\\
56.375	0.2394	-854.66298727477	-854.66298727477\\
56.375	0.24306	-891.068726123808	-891.068726123808\\
56.375	0.24672	-928.29620533766	-928.29620533766\\
56.375	0.25038	-966.345424916328	-966.345424916328\\
56.375	0.25404	-1005.21638485981	-1005.21638485981\\
56.375	0.2577	-1044.90908516811	-1044.90908516811\\
56.375	0.26136	-1085.42352584122	-1085.42352584122\\
56.375	0.26502	-1126.75970687915	-1126.75970687915\\
56.375	0.26868	-1168.91762828189	-1168.91762828189\\
56.375	0.27234	-1211.89729004945	-1211.89729004945\\
56.375	0.276	-1255.69869218183	-1255.69869218183\\
56.75	0.093	-71.5673681581807	-71.5673681581807\\
56.75	0.09666	-75.1154550724245	-75.1154550724245\\
56.75	0.10032	-79.4852823514834	-79.4852823514834\\
56.75	0.10398	-84.6768499953577	-84.6768499953577\\
56.75	0.10764	-90.6901580040471	-90.6901580040471\\
56.75	0.1113	-97.5252063775514	-97.5252063775514\\
56.75	0.11496	-105.181995115871	-105.181995115871\\
56.75	0.11862	-113.660524219006	-113.660524219006\\
56.75	0.12228	-122.960793686956	-122.960793686956\\
56.75	0.12594	-133.082803519721	-133.082803519721\\
56.75	0.1296	-144.026553717301	-144.026553717301\\
56.75	0.13326	-155.792044279697	-155.792044279697\\
56.75	0.13692	-168.379275206908	-168.379275206908\\
56.75	0.14058	-181.788246498933	-181.788246498933\\
56.75	0.14424	-196.018958155774	-196.018958155774\\
56.75	0.1479	-211.07141017743	-211.07141017743\\
56.75	0.15156	-226.945602563902	-226.945602563902\\
56.75	0.15522	-243.641535315188	-243.641535315188\\
56.75	0.15888	-261.15920843129	-261.15920843129\\
56.75	0.16254	-279.498621912207	-279.498621912207\\
56.75	0.1662	-298.659775757939	-298.659775757939\\
56.75	0.16986	-318.642669968486	-318.642669968486\\
56.75	0.17352	-339.447304543848	-339.447304543848\\
56.75	0.17718	-361.073679484026	-361.073679484026\\
56.75	0.18084	-383.521794789018	-383.521794789018\\
56.75	0.1845	-406.791650458826	-406.791650458826\\
56.75	0.18816	-430.883246493449	-430.883246493449\\
56.75	0.19182	-455.796582892887	-455.796582892887\\
56.75	0.19548	-481.53165965714	-481.53165965714\\
56.75	0.19914	-508.088476786209	-508.088476786209\\
56.75	0.2028	-535.467034280092	-535.467034280092\\
56.75	0.20646	-563.667332138791	-563.667332138791\\
56.75	0.21012	-592.689370362305	-592.689370362305\\
56.75	0.21378	-622.533148950634	-622.533148950634\\
56.75	0.21744	-653.198667903778	-653.198667903778\\
56.75	0.2211	-684.685927221738	-684.685927221738\\
56.75	0.22476	-716.994926904512	-716.994926904512\\
56.75	0.22842	-750.125666952102	-750.125666952102\\
56.75	0.23208	-784.078147364507	-784.078147364507\\
56.75	0.23574	-818.852368141727	-818.852368141727\\
56.75	0.2394	-854.448329283762	-854.448329283762\\
56.75	0.24306	-890.866030790613	-890.866030790613\\
56.75	0.24672	-928.105472662278	-928.105472662278\\
56.75	0.25038	-966.166654898759	-966.166654898759\\
56.75	0.25404	-1005.04957750006	-1005.04957750006\\
56.75	0.2577	-1044.75424046617	-1044.75424046617\\
56.75	0.26136	-1085.28064379709	-1085.28064379709\\
56.75	0.26502	-1126.62878749283	-1126.62878749283\\
56.75	0.26868	-1168.79867155339	-1168.79867155339\\
56.75	0.27234	-1211.79029597876	-1211.79029597876\\
56.75	0.276	-1255.60366076895	-1255.60366076895\\
57.125	0.093	-70.9452374115499	-70.9452374115499\\
57.125	0.09666	-74.505286983607	-74.505286983607\\
57.125	0.10032	-78.8870769204788	-78.8870769204788\\
57.125	0.10398	-84.0906072221662	-84.0906072221662\\
57.125	0.10764	-90.1158778886685	-90.1158778886685\\
57.125	0.1113	-96.9628889199861	-96.9628889199861\\
57.125	0.11496	-104.631640316119	-104.631640316119\\
57.125	0.11862	-113.122132077067	-113.122132077067\\
57.125	0.12228	-122.43436420283	-122.43436420283\\
57.125	0.12594	-132.568336693408	-132.568336693408\\
57.125	0.1296	-143.524049548801	-143.524049548801\\
57.125	0.13326	-155.30150276901	-155.30150276901\\
57.125	0.13692	-167.900696354034	-167.900696354034\\
57.125	0.14058	-181.321630303873	-181.321630303873\\
57.125	0.14424	-195.564304618527	-195.564304618527\\
57.125	0.1479	-210.628719297996	-210.628719297996\\
57.125	0.15156	-226.51487434228	-226.51487434228\\
57.125	0.15522	-243.22276975138	-243.22276975138\\
57.125	0.15888	-260.752405525295	-260.752405525295\\
57.125	0.16254	-279.103781664024	-279.103781664024\\
57.125	0.1662	-298.27689816757	-298.27689816757\\
57.125	0.16986	-318.27175503593	-318.27175503593\\
57.125	0.17352	-339.088352269105	-339.088352269105\\
57.125	0.17718	-360.726689867096	-360.726689867096\\
57.125	0.18084	-383.186767829901	-383.186767829901\\
57.125	0.1845	-406.468586157522	-406.468586157522\\
57.125	0.18816	-430.572144849958	-430.572144849958\\
57.125	0.19182	-455.497443907209	-455.497443907209\\
57.125	0.19548	-481.244483329276	-481.244483329276\\
57.125	0.19914	-507.813263116157	-507.813263116157\\
57.125	0.2028	-535.203783267854	-535.203783267854\\
57.125	0.20646	-563.416043784366	-563.416043784366\\
57.125	0.21012	-592.450044665693	-592.450044665693\\
57.125	0.21378	-622.305785911835	-622.305785911835\\
57.125	0.21744	-652.983267522792	-652.983267522792\\
57.125	0.2211	-684.482489498565	-684.482489498565\\
57.125	0.22476	-716.803451839152	-716.803451839152\\
57.125	0.22842	-749.946154544555	-749.946154544555\\
57.125	0.23208	-783.910597614773	-783.910597614773\\
57.125	0.23574	-818.696781049807	-818.696781049807\\
57.125	0.2394	-854.304704849655	-854.304704849655\\
57.125	0.24306	-890.734369014318	-890.734369014318\\
57.125	0.24672	-927.985773543797	-927.985773543797\\
57.125	0.25038	-966.058918438091	-966.058918438091\\
57.125	0.25404	-1004.9538036972	-1004.9538036972\\
57.125	0.2577	-1044.67042932112	-1044.67042932112\\
57.125	0.26136	-1085.20879530986	-1085.20879530986\\
57.125	0.26502	-1126.56890166342	-1126.56890166342\\
57.125	0.26868	-1168.75074838179	-1168.75074838179\\
57.125	0.27234	-1211.75433546497	-1211.75433546497\\
57.125	0.276	-1255.57966291297	-1255.57966291297\\
57.5	0.093	-70.3941402218201	-70.3941402218201\\
57.5	0.09666	-73.9661524516899	-73.9661524516899\\
57.5	0.10032	-78.3599050463752	-78.3599050463752\\
57.5	0.10398	-83.5753980058756	-83.5753980058756\\
57.5	0.10764	-89.6126313301909	-89.6126313301909\\
57.5	0.1113	-96.4716050193217	-96.4716050193217\\
57.5	0.11496	-104.152319073268	-104.152319073268\\
57.5	0.11862	-112.654773492028	-112.654773492028\\
57.5	0.12228	-121.978968275605	-121.978968275605\\
57.5	0.12594	-132.124903423996	-132.124903423996\\
57.5	0.1296	-143.092578937202	-143.092578937202\\
57.5	0.13326	-154.881994815224	-154.881994815224\\
57.5	0.13692	-167.493151058061	-167.493151058061\\
57.5	0.14058	-180.926047665713	-180.926047665713\\
57.5	0.14424	-195.18068463818	-195.18068463818\\
57.5	0.1479	-210.257061975462	-210.257061975462\\
57.5	0.15156	-226.15517967756	-226.15517967756\\
57.5	0.15522	-242.875037744472	-242.875037744472\\
57.5	0.15888	-260.4166361762	-260.4166361762\\
57.5	0.16254	-278.779974972743	-278.779974972743\\
57.5	0.1662	-297.965054134101	-297.965054134101\\
57.5	0.16986	-317.971873660274	-317.971873660274\\
57.5	0.17352	-338.800433551263	-338.800433551263\\
57.5	0.17718	-360.450733807066	-360.450733807066\\
57.5	0.18084	-382.922774427685	-382.922774427685\\
57.5	0.1845	-406.216555413119	-406.216555413119\\
57.5	0.18816	-430.332076763368	-430.332076763368\\
57.5	0.19182	-455.269338478433	-455.269338478433\\
57.5	0.19548	-481.028340558312	-481.028340558312\\
57.5	0.19914	-507.609083003007	-507.609083003007\\
57.5	0.2028	-535.011565812517	-535.011565812517\\
57.5	0.20646	-563.235788986841	-563.235788986841\\
57.5	0.21012	-592.281752525982	-592.281752525982\\
57.5	0.21378	-622.149456429937	-622.149456429937\\
57.5	0.21744	-652.838900698707	-652.838900698707\\
57.5	0.2211	-684.350085332293	-684.350085332293\\
57.5	0.22476	-716.683010330693	-716.683010330693\\
57.5	0.22842	-749.837675693909	-749.837675693909\\
57.5	0.23208	-783.81408142194	-783.81408142194\\
57.5	0.23574	-818.612227514787	-818.612227514787\\
57.5	0.2394	-854.232113972448	-854.232113972448\\
57.5	0.24306	-890.673740794925	-890.673740794925\\
57.5	0.24672	-927.937107982217	-927.937107982217\\
57.5	0.25038	-966.022215534324	-966.022215534324\\
57.5	0.25404	-1004.92906345125	-1004.92906345125\\
57.5	0.2577	-1044.65765173298	-1044.65765173298\\
57.5	0.26136	-1085.20798037954	-1085.20798037954\\
57.5	0.26502	-1126.5800493909	-1126.5800493909\\
57.5	0.26868	-1168.77385876709	-1168.77385876709\\
57.5	0.27234	-1211.78940850808	-1211.78940850808\\
57.5	0.276	-1255.6266986139	-1255.6266986139\\
57.875	0.093	-69.9140765889906	-69.9140765889906\\
57.875	0.09666	-73.4980514766737	-73.4980514766737\\
57.875	0.10032	-77.903766729172	-77.903766729172\\
57.875	0.10398	-83.1312223464853	-83.1312223464853\\
57.875	0.10764	-89.1804183286137	-89.1804183286137\\
57.875	0.1113	-96.0513546755577	-96.0513546755577\\
57.875	0.11496	-103.744031387317	-103.744031387317\\
57.875	0.11862	-112.25844846389	-112.25844846389\\
57.875	0.12228	-121.59460590528	-121.59460590528\\
57.875	0.12594	-131.752503711484	-131.752503711484\\
57.875	0.1296	-142.732141882504	-142.732141882504\\
57.875	0.13326	-154.533520418338	-154.533520418338\\
57.875	0.13692	-167.156639318988	-167.156639318988\\
57.875	0.14058	-180.601498584453	-180.601498584453\\
57.875	0.14424	-194.868098214734	-194.868098214734\\
57.875	0.1479	-209.956438209829	-209.956438209829\\
57.875	0.15156	-225.86651856974	-225.86651856974\\
57.875	0.15522	-242.598339294465	-242.598339294465\\
57.875	0.15888	-260.151900384006	-260.151900384006\\
57.875	0.16254	-278.527201838362	-278.527201838362\\
57.875	0.1662	-297.724243657534	-297.724243657534\\
57.875	0.16986	-317.74302584152	-317.74302584152\\
57.875	0.17352	-338.583548390321	-338.583548390321\\
57.875	0.17718	-360.245811303938	-360.245811303938\\
57.875	0.18084	-382.72981458237	-382.72981458237\\
57.875	0.1845	-406.035558225617	-406.035558225617\\
57.875	0.18816	-430.163042233679	-430.163042233679\\
57.875	0.19182	-455.112266606556	-455.112266606556\\
57.875	0.19548	-480.883231344249	-480.883231344249\\
57.875	0.19914	-507.475936446756	-507.475936446756\\
57.875	0.2028	-534.89038191408	-534.89038191408\\
57.875	0.20646	-563.126567746217	-563.126567746217\\
57.875	0.21012	-592.184493943171	-592.184493943171\\
57.875	0.21378	-622.064160504939	-622.064160504939\\
57.875	0.21744	-652.765567431522	-652.765567431522\\
57.875	0.2211	-684.288714722921	-684.288714722921\\
57.875	0.22476	-716.633602379135	-716.633602379135\\
57.875	0.22842	-749.800230400164	-749.800230400164\\
57.875	0.23208	-783.788598786008	-783.788598786008\\
57.875	0.23574	-818.598707536668	-818.598707536668\\
57.875	0.2394	-854.230556652142	-854.230556652142\\
57.875	0.24306	-890.684146132431	-890.684146132431\\
57.875	0.24672	-927.959475977536	-927.959475977536\\
57.875	0.25038	-966.056546187457	-966.056546187457\\
57.875	0.25404	-1004.97535676219	-1004.97535676219\\
57.875	0.2577	-1044.71590770174	-1044.71590770174\\
57.875	0.26136	-1085.27819900611	-1085.27819900611\\
57.875	0.26502	-1126.66223067529	-1126.66223067529\\
57.875	0.26868	-1168.86800270928	-1168.86800270928\\
57.875	0.27234	-1211.89551510809	-1211.89551510809\\
57.875	0.276	-1255.74476787172	-1255.74476787172\\
58.25	0.093	-69.505046513062	-69.505046513062\\
58.25	0.09666	-73.1009840585583	-73.1009840585583\\
58.25	0.10032	-77.5186619688696	-77.5186619688696\\
58.25	0.10398	-82.7580802439959	-82.7580802439959\\
58.25	0.10764	-88.8192388839378	-88.8192388839378\\
58.25	0.1113	-95.7021378886943	-95.7021378886943\\
58.25	0.11496	-103.406777258266	-103.406777258266\\
58.25	0.11862	-111.933156992654	-111.933156992654\\
58.25	0.12228	-121.281277091856	-121.281277091856\\
58.25	0.12594	-131.451137555874	-131.451137555874\\
58.25	0.1296	-142.442738384706	-142.442738384706\\
58.25	0.13326	-154.256079578354	-154.256079578354\\
58.25	0.13692	-166.891161136817	-166.891161136817\\
58.25	0.14058	-180.347983060095	-180.347983060095\\
58.25	0.14424	-194.626545348188	-194.626545348188\\
58.25	0.1479	-209.726848001097	-209.726848001097\\
58.25	0.15156	-225.64889101882	-225.64889101882\\
58.25	0.15522	-242.392674401359	-242.392674401359\\
58.25	0.15888	-259.958198148713	-259.958198148713\\
58.25	0.16254	-278.345462260882	-278.345462260882\\
58.25	0.1662	-297.554466737866	-297.554466737866\\
58.25	0.16986	-317.585211579666	-317.585211579666\\
58.25	0.17352	-338.43769678628	-338.43769678628\\
58.25	0.17718	-360.11192235771	-360.11192235771\\
58.25	0.18084	-382.607888293955	-382.607888293955\\
58.25	0.1845	-405.925594595015	-405.925594595015\\
58.25	0.18816	-430.06504126089	-430.06504126089\\
58.25	0.19182	-455.026228291581	-455.026228291581\\
58.25	0.19548	-480.809155687086	-480.809155687086\\
58.25	0.19914	-507.413823447407	-507.413823447407\\
58.25	0.2028	-534.840231572543	-534.840231572543\\
58.25	0.20646	-563.088380062494	-563.088380062494\\
58.25	0.21012	-592.158268917261	-592.158268917261\\
58.25	0.21378	-622.049898136842	-622.049898136842\\
58.25	0.21744	-652.763267721238	-652.763267721238\\
58.25	0.2211	-684.29837767045	-684.29837767045\\
58.25	0.22476	-716.655227984477	-716.655227984477\\
58.25	0.22842	-749.833818663319	-749.833818663319\\
58.25	0.23208	-783.834149706976	-783.834149706976\\
58.25	0.23574	-818.656221115449	-818.656221115449\\
58.25	0.2394	-854.300032888736	-854.300032888736\\
58.25	0.24306	-890.765585026839	-890.765585026839\\
58.25	0.24672	-928.052877529757	-928.052877529757\\
58.25	0.25038	-966.16191039749	-966.16191039749\\
58.25	0.25404	-1005.09268363004	-1005.09268363004\\
58.25	0.2577	-1044.8451972274	-1044.8451972274\\
58.25	0.26136	-1085.41945118958	-1085.41945118958\\
58.25	0.26502	-1126.81544551657	-1126.81544551657\\
58.25	0.26868	-1169.03318020838	-1169.03318020838\\
58.25	0.27234	-1212.07265526501	-1212.07265526501\\
58.25	0.276	-1255.93387068645	-1255.93387068645\\
58.625	0.093	-69.1670499940341	-69.1670499940341\\
58.625	0.09666	-72.7749501973432	-72.7749501973432\\
58.625	0.10032	-77.2045907654675	-77.2045907654675\\
58.625	0.10398	-82.455971698407	-82.455971698407\\
58.625	0.10764	-88.5290929961619	-88.5290929961619\\
58.625	0.1113	-95.4239546587316	-95.4239546587316\\
58.625	0.11496	-103.140556686117	-103.140556686117\\
58.625	0.11862	-111.678899078317	-111.678899078317\\
58.625	0.12228	-121.038981835332	-121.038981835332\\
58.625	0.12594	-131.220804957163	-131.220804957163\\
58.625	0.1296	-142.224368443809	-142.224368443809\\
58.625	0.13326	-154.049672295269	-154.049672295269\\
58.625	0.13692	-166.696716511546	-166.696716511546\\
58.625	0.14058	-180.165501092637	-180.165501092637\\
58.625	0.14424	-194.456026038543	-194.456026038543\\
58.625	0.1479	-209.568291349265	-209.568291349265\\
58.625	0.15156	-225.502297024801	-225.502297024801\\
58.625	0.15522	-242.258043065153	-242.258043065153\\
58.625	0.15888	-259.83552947032	-259.83552947032\\
58.625	0.16254	-278.234756240302	-278.234756240302\\
58.625	0.1662	-297.4557233751	-297.4557233751\\
58.625	0.16986	-317.498430874712	-317.498430874712\\
58.625	0.17352	-338.36287873914	-338.36287873914\\
58.625	0.17718	-360.049066968383	-360.049066968383\\
58.625	0.18084	-382.556995562441	-382.556995562441\\
58.625	0.1845	-405.886664521314	-405.886664521314\\
58.625	0.18816	-430.038073845002	-430.038073845002\\
58.625	0.19182	-455.011223533506	-455.011223533506\\
58.625	0.19548	-480.806113586824	-480.806113586824\\
58.625	0.19914	-507.422744004958	-507.422744004958\\
58.625	0.2028	-534.861114787908	-534.861114787908\\
58.625	0.20646	-563.121225935672	-563.121225935672\\
58.625	0.21012	-592.203077448251	-592.203077448251\\
58.625	0.21378	-622.106669325645	-622.106669325645\\
58.625	0.21744	-652.832001567855	-652.832001567855\\
58.625	0.2211	-684.37907417488	-684.37907417488\\
58.625	0.22476	-716.74788714672	-716.74788714672\\
58.625	0.22842	-749.938440483375	-749.938440483375\\
58.625	0.23208	-783.950734184845	-783.950734184845\\
58.625	0.23574	-818.784768251131	-818.784768251131\\
58.625	0.2394	-854.440542682231	-854.440542682231\\
58.625	0.24306	-890.918057478147	-890.918057478147\\
58.625	0.24672	-928.217312638878	-928.217312638878\\
58.625	0.25038	-966.338308164424	-966.338308164424\\
58.625	0.25404	-1005.28104405479	-1005.28104405479\\
58.625	0.2577	-1045.04552030996	-1045.04552030996\\
58.625	0.26136	-1085.63173692995	-1085.63173692995\\
58.625	0.26502	-1127.03969391476	-1127.03969391476\\
58.625	0.26868	-1169.26939126438	-1169.26939126438\\
58.625	0.27234	-1212.32082897882	-1212.32082897882\\
58.625	0.276	-1256.19400705807	-1256.19400705807\\
59	0.093	-68.9000870319068	-68.9000870319068\\
59	0.09666	-72.5199498930289	-72.5199498930289\\
59	0.10032	-76.9615531189664	-76.9615531189664\\
59	0.10398	-82.2248967097192	-82.2248967097192\\
59	0.10764	-88.3099806652868	-88.3099806652868\\
59	0.1113	-95.2168049856698	-95.2168049856698\\
59	0.11496	-102.945369670868	-102.945369670868\\
59	0.11862	-111.495674720881	-111.495674720881\\
59	0.12228	-120.86772013571	-120.86772013571\\
59	0.12594	-131.061505915353	-131.061505915353\\
59	0.1296	-142.077032059812	-142.077032059812\\
59	0.13326	-153.914298569086	-153.914298569086\\
59	0.13692	-166.573305443175	-166.573305443175\\
59	0.14058	-180.054052682079	-180.054052682079\\
59	0.14424	-194.356540285799	-194.356540285799\\
59	0.1479	-209.480768254333	-209.480768254333\\
59	0.15156	-225.426736587683	-225.426736587683\\
59	0.15522	-242.194445285848	-242.194445285848\\
59	0.15888	-259.783894348828	-259.783894348828\\
59	0.16254	-278.195083776624	-278.195083776624\\
59	0.1662	-297.428013569234	-297.428013569234\\
59	0.16986	-317.48268372666	-317.48268372666\\
59	0.17352	-338.359094248901	-338.359094248901\\
59	0.17718	-360.057245135956	-360.057245135956\\
59	0.18084	-382.577136387827	-382.577136387827\\
59	0.1845	-405.918768004514	-405.918768004514\\
59	0.18816	-430.082139986015	-430.082139986015\\
59	0.19182	-455.067252332332	-455.067252332332\\
59	0.19548	-480.874105043463	-480.874105043463\\
59	0.19914	-507.50269811941	-507.50269811941\\
59	0.2028	-534.953031560173	-534.953031560173\\
59	0.20646	-563.22510536575	-563.22510536575\\
59	0.21012	-592.318919536142	-592.318919536142\\
59	0.21378	-622.23447407135	-622.23447407135\\
59	0.21744	-652.971768971373	-652.971768971373\\
59	0.2211	-684.53080423621	-684.53080423621\\
59	0.22476	-716.911579865863	-716.911579865863\\
59	0.22842	-750.114095860331	-750.114095860331\\
59	0.23208	-784.138352219615	-784.138352219615\\
59	0.23574	-818.984348943714	-818.984348943714\\
59	0.2394	-854.652086032627	-854.652086032627\\
59	0.24306	-891.141563486356	-891.141563486356\\
59	0.24672	-928.4527813049	-928.4527813049\\
59	0.25038	-966.58573948826	-966.58573948826\\
59	0.25404	-1005.54043803643	-1005.54043803643\\
59	0.2577	-1045.31687694942	-1045.31687694942\\
59	0.26136	-1085.91505622723	-1085.91505622723\\
59	0.26502	-1127.33497586985	-1127.33497586985\\
59	0.26868	-1169.57663587728	-1169.57663587728\\
59	0.27234	-1212.64003624953	-1212.64003624953\\
59	0.276	-1256.5251769866	-1256.5251769866\\
59.375	0.093	-68.7041576266799	-68.7041576266799\\
59.375	0.09666	-72.335983145615	-72.335983145615\\
59.375	0.10032	-76.7895490293658	-76.7895490293658\\
59.375	0.10398	-82.0648552779316	-82.0648552779316\\
59.375	0.10764	-88.1619018913122	-88.1619018913122\\
59.375	0.1113	-95.0806888695084	-95.0806888695084\\
59.375	0.11496	-102.82121621252	-102.82121621252\\
59.375	0.11862	-111.383483920346	-111.383483920346\\
59.375	0.12228	-120.767491992988	-120.767491992988\\
59.375	0.12594	-130.973240430444	-130.973240430444\\
59.375	0.1296	-142.000729232716	-142.000729232716\\
59.375	0.13326	-153.849958399803	-153.849958399803\\
59.375	0.13692	-166.520927931705	-166.520927931705\\
59.375	0.14058	-180.013637828423	-180.013637828423\\
59.375	0.14424	-194.328088089955	-194.328088089955\\
59.375	0.1479	-209.464278716303	-209.464278716303\\
59.375	0.15156	-225.422209707466	-225.422209707466\\
59.375	0.15522	-242.201881063444	-242.201881063444\\
59.375	0.15888	-259.803292784237	-259.803292784237\\
59.375	0.16254	-278.226444869845	-278.226444869845\\
59.375	0.1662	-297.471337320269	-297.471337320269\\
59.375	0.16986	-317.537970135507	-317.537970135507\\
59.375	0.17352	-338.426343315562	-338.426343315562\\
59.375	0.17718	-360.13645686043	-360.13645686043\\
59.375	0.18084	-382.668310770114	-382.668310770114\\
59.375	0.1845	-406.021905044614	-406.021905044614\\
59.375	0.18816	-430.197239683928	-430.197239683928\\
59.375	0.19182	-455.194314688058	-455.194314688058\\
59.375	0.19548	-481.013130057003	-481.013130057003\\
59.375	0.19914	-507.653685790763	-507.653685790763\\
59.375	0.2028	-535.115981889338	-535.115981889338\\
59.375	0.20646	-563.400018352728	-563.400018352728\\
59.375	0.21012	-592.505795180934	-592.505795180934\\
59.375	0.21378	-622.433312373955	-622.433312373955\\
59.375	0.21744	-653.18256993179	-653.18256993179\\
59.375	0.2211	-684.753567854441	-684.753567854441\\
59.375	0.22476	-717.146306141907	-717.146306141907\\
59.375	0.22842	-750.360784794189	-750.360784794189\\
59.375	0.23208	-784.397003811285	-784.397003811285\\
59.375	0.23574	-819.254963193197	-819.254963193197\\
59.375	0.2394	-854.934662939924	-854.934662939924\\
59.375	0.24306	-891.436103051466	-891.436103051466\\
59.375	0.24672	-928.759283527823	-928.759283527823\\
59.375	0.25038	-966.904204368995	-966.904204368995\\
59.375	0.25404	-1005.87086557498	-1005.87086557498\\
59.375	0.2577	-1045.65926714579	-1045.65926714579\\
59.375	0.26136	-1086.2694090814	-1086.2694090814\\
59.375	0.26502	-1127.70129138184	-1127.70129138184\\
59.375	0.26868	-1169.95491404708	-1169.95491404708\\
59.375	0.27234	-1213.03027707715	-1213.03027707715\\
59.375	0.276	-1256.92738047203	-1256.92738047203\\
59.75	0.093	-68.579261778354	-68.579261778354\\
59.75	0.09666	-72.2230499551025	-72.2230499551025\\
59.75	0.10032	-76.688578496666	-76.688578496666\\
59.75	0.10398	-81.9758474030448	-81.9758474030448\\
59.75	0.10764	-88.0848566742389	-88.0848566742389\\
59.75	0.1113	-95.0156063102479	-95.0156063102479\\
59.75	0.11496	-102.768096311072	-102.768096311072\\
59.75	0.11862	-111.342326676712	-111.342326676712\\
59.75	0.12228	-120.738297407166	-120.738297407166\\
59.75	0.12594	-130.956008502436	-130.956008502436\\
59.75	0.1296	-141.995459962521	-141.995459962521\\
59.75	0.13326	-153.856651787421	-153.856651787421\\
59.75	0.13692	-166.539583977136	-166.539583977136\\
59.75	0.14058	-180.044256531667	-180.044256531667\\
59.75	0.14424	-194.370669451012	-194.370669451012\\
59.75	0.1479	-209.518822735173	-209.518822735173\\
59.75	0.15156	-225.488716384149	-225.488716384149\\
59.75	0.15522	-242.28035039794	-242.28035039794\\
59.75	0.15888	-259.893724776546	-259.893724776546\\
59.75	0.16254	-278.328839519968	-278.328839519968\\
59.75	0.1662	-297.585694628204	-297.585694628204\\
59.75	0.16986	-317.664290101256	-317.664290101256\\
59.75	0.17352	-338.564625939123	-338.564625939123\\
59.75	0.17718	-360.286702141805	-360.286702141805\\
59.75	0.18084	-382.830518709302	-382.830518709302\\
59.75	0.1845	-406.196075641615	-406.196075641615\\
59.75	0.18816	-430.383372938742	-430.383372938742\\
59.75	0.19182	-455.392410600685	-455.392410600685\\
59.75	0.19548	-481.223188627443	-481.223188627443\\
59.75	0.19914	-507.875707019016	-507.875707019016\\
59.75	0.2028	-535.349965775405	-535.349965775405\\
59.75	0.20646	-563.645964896608	-563.645964896608\\
59.75	0.21012	-592.763704382626	-592.763704382626\\
59.75	0.21378	-622.70318423346	-622.70318423346\\
59.75	0.21744	-653.464404449109	-653.464404449109\\
59.75	0.2211	-685.047365029573	-685.047365029573\\
59.75	0.22476	-717.452065974852	-717.452065974852\\
59.75	0.22842	-750.678507284947	-750.678507284947\\
59.75	0.23208	-784.726688959856	-784.726688959856\\
59.75	0.23574	-819.596610999581	-819.596610999581\\
59.75	0.2394	-855.288273404121	-855.288273404121\\
59.75	0.24306	-891.801676173476	-891.801676173476\\
59.75	0.24672	-929.136819307646	-929.136819307646\\
59.75	0.25038	-967.293702806632	-967.293702806632\\
59.75	0.25404	-1006.27232667043	-1006.27232667043\\
59.75	0.2577	-1046.07269089905	-1046.07269089905\\
59.75	0.26136	-1086.69479549248	-1086.69479549248\\
59.75	0.26502	-1128.13864045072	-1128.13864045072\\
59.75	0.26868	-1170.40422577379	-1170.40422577379\\
59.75	0.27234	-1213.49155146166	-1213.49155146166\\
59.75	0.276	-1257.40061751435	-1257.40061751435\\
60.125	0.093	-68.5253994869283	-68.5253994869283\\
60.125	0.09666	-72.1811503214899	-72.1811503214899\\
60.125	0.10032	-76.6586415208666	-76.6586415208666\\
60.125	0.10398	-81.9578730850585	-81.9578730850585\\
60.125	0.10764	-88.0788450140655	-88.0788450140655\\
60.125	0.1113	-95.0215573078877	-95.0215573078877\\
60.125	0.11496	-102.786009966525	-102.786009966525\\
60.125	0.11862	-111.372202989978	-111.372202989978\\
60.125	0.12228	-120.780136378245	-120.780136378245\\
60.125	0.12594	-131.009810131328	-131.009810131328\\
60.125	0.1296	-142.061224249226	-142.061224249226\\
60.125	0.13326	-153.934378731939	-153.934378731939\\
60.125	0.13692	-166.629273579468	-166.629273579468\\
60.125	0.14058	-180.145908791811	-180.145908791811\\
60.125	0.14424	-194.48428436897	-194.48428436897\\
60.125	0.1479	-209.644400310944	-209.644400310944\\
60.125	0.15156	-225.626256617733	-225.626256617733\\
60.125	0.15522	-242.429853289337	-242.429853289337\\
60.125	0.15888	-260.055190325756	-260.055190325756\\
60.125	0.16254	-278.502267726991	-278.502267726991\\
60.125	0.1662	-297.77108549304	-297.77108549304\\
60.125	0.16986	-317.861643623905	-317.861643623905\\
60.125	0.17352	-338.773942119585	-338.773942119585\\
60.125	0.17718	-360.507980980081	-360.507980980081\\
60.125	0.18084	-383.063760205391	-383.063760205391\\
60.125	0.1845	-406.441279795516	-406.441279795516\\
60.125	0.18816	-430.640539750457	-430.640539750457\\
60.125	0.19182	-455.661540070213	-455.661540070213\\
60.125	0.19548	-481.504280754784	-481.504280754784\\
60.125	0.19914	-508.16876180417	-508.16876180417\\
60.125	0.2028	-535.654983218372	-535.654983218372\\
60.125	0.20646	-563.962944997388	-563.962944997388\\
60.125	0.21012	-593.092647141219	-593.092647141219\\
60.125	0.21378	-623.044089649866	-623.044089649866\\
60.125	0.21744	-653.817272523328	-653.817272523328\\
60.125	0.2211	-685.412195761605	-685.412195761605\\
60.125	0.22476	-717.828859364697	-717.828859364697\\
60.125	0.22842	-751.067263332605	-751.067263332605\\
60.125	0.23208	-785.127407665327	-785.127407665327\\
60.125	0.23574	-820.009292362865	-820.009292362865\\
60.125	0.2394	-855.712917425219	-855.712917425219\\
60.125	0.24306	-892.238282852387	-892.238282852387\\
60.125	0.24672	-929.58538864437	-929.58538864437\\
60.125	0.25038	-967.754234801169	-967.754234801169\\
60.125	0.25404	-1006.74482132278	-1006.74482132278\\
60.125	0.2577	-1046.55714820921	-1046.55714820921\\
60.125	0.26136	-1087.19121546046	-1087.19121546046\\
60.125	0.26502	-1128.64702307651	-1128.64702307651\\
60.125	0.26868	-1170.92457105739	-1170.92457105739\\
60.125	0.27234	-1214.02385940308	-1214.02385940308\\
60.125	0.276	-1257.94488811358	-1257.94488811358\\
60.5	0.093	-68.5425707524037	-68.5425707524037\\
60.5	0.09666	-72.2102842447782	-72.2102842447782\\
60.5	0.10032	-76.699738101968	-76.699738101968\\
60.5	0.10398	-82.010932323973	-82.010932323973\\
60.5	0.10764	-88.1438669107931	-88.1438669107931\\
60.5	0.1113	-95.0985418624283	-95.0985418624283\\
60.5	0.11496	-102.874957178879	-102.874957178879\\
60.5	0.11862	-111.473112860144	-111.473112860144\\
60.5	0.12228	-120.893008906225	-120.893008906225\\
60.5	0.12594	-131.134645317121	-131.134645317121\\
60.5	0.1296	-142.198022092832	-142.198022092832\\
60.5	0.13326	-154.083139233358	-154.083139233358\\
60.5	0.13692	-166.7899967387	-166.7899967387\\
60.5	0.14058	-180.318594608857	-180.318594608857\\
60.5	0.14424	-194.668932843828	-194.668932843828\\
60.5	0.1479	-209.841011443615	-209.841011443615\\
60.5	0.15156	-225.834830408218	-225.834830408218\\
60.5	0.15522	-242.650389737635	-242.650389737635\\
60.5	0.15888	-260.287689431867	-260.287689431867\\
60.5	0.16254	-278.746729490915	-278.746729490915\\
60.5	0.1662	-298.027509914777	-298.027509914777\\
60.5	0.16986	-318.130030703455	-318.130030703455\\
60.5	0.17352	-339.054291856949	-339.054291856949\\
60.5	0.17718	-360.800293375257	-360.800293375257\\
60.5	0.18084	-383.36803525838	-383.36803525838\\
60.5	0.1845	-406.757517506319	-406.757517506319\\
60.5	0.18816	-430.968740119072	-430.968740119072\\
60.5	0.19182	-456.001703096641	-456.001703096641\\
60.5	0.19548	-481.856406439025	-481.856406439025\\
60.5	0.19914	-508.532850146224	-508.532850146224\\
60.5	0.2028	-536.031034218239	-536.031034218239\\
60.5	0.20646	-564.350958655069	-564.350958655069\\
60.5	0.21012	-593.492623456713	-593.492623456713\\
60.5	0.21378	-623.456028623173	-623.456028623173\\
60.5	0.21744	-654.241174154448	-654.241174154448\\
60.5	0.2211	-685.848060050538	-685.848060050538\\
60.5	0.22476	-718.276686311444	-718.276686311444\\
60.5	0.22842	-751.527052937164	-751.527052937164\\
60.5	0.23208	-785.5991599277	-785.5991599277\\
60.5	0.23574	-820.493007283051	-820.493007283051\\
60.5	0.2394	-856.208595003217	-856.208595003217\\
60.5	0.24306	-892.745923088198	-892.745923088198\\
60.5	0.24672	-930.104991537995	-930.104991537995\\
60.5	0.25038	-968.285800352606	-968.285800352606\\
60.5	0.25404	-1007.28834953203	-1007.28834953203\\
60.5	0.2577	-1047.11263907628	-1047.11263907628\\
60.5	0.26136	-1087.75866898533	-1087.75866898533\\
60.5	0.26502	-1129.2264392592	-1129.2264392592\\
60.5	0.26868	-1171.51594989789	-1171.51594989789\\
60.5	0.27234	-1214.62720090139	-1214.62720090139\\
60.5	0.276	-1258.56019226971	-1258.56019226971\\
60.875	0.093	-68.6307755747795	-68.6307755747795\\
60.875	0.09666	-72.310451724967	-72.310451724967\\
60.875	0.10032	-76.81186823997	-76.81186823997\\
60.875	0.10398	-82.1350251197881	-82.1350251197881\\
60.875	0.10764	-88.2799223644211	-88.2799223644211\\
60.875	0.1113	-95.2465599738696	-95.2465599738696\\
60.875	0.11496	-103.034937948133	-103.034937948133\\
60.875	0.11862	-111.645056287212	-111.645056287212\\
60.875	0.12228	-121.076914991106	-121.076914991106\\
60.875	0.12594	-131.330514059815	-131.330514059815\\
60.875	0.1296	-142.405853493339	-142.405853493339\\
60.875	0.13326	-154.302933291678	-154.302933291678\\
60.875	0.13692	-167.021753454832	-167.021753454832\\
60.875	0.14058	-180.562313982802	-180.562313982802\\
60.875	0.14424	-194.924614875587	-194.924614875587\\
60.875	0.1479	-210.108656133187	-210.108656133187\\
60.875	0.15156	-226.114437755603	-226.114437755603\\
60.875	0.15522	-242.941959742833	-242.941959742833\\
60.875	0.15888	-260.591222094878	-260.591222094878\\
60.875	0.16254	-279.062224811739	-279.062224811739\\
60.875	0.1662	-298.354967893415	-298.354967893415\\
60.875	0.16986	-318.469451339906	-318.469451339906\\
60.875	0.17352	-339.405675151212	-339.405675151212\\
60.875	0.17718	-361.163639327333	-361.163639327333\\
60.875	0.18084	-383.74334386827	-383.74334386827\\
60.875	0.1845	-407.144788774021	-407.144788774021\\
60.875	0.18816	-431.367974044588	-431.367974044588\\
60.875	0.19182	-456.41289967997	-456.41289967997\\
60.875	0.19548	-482.279565680167	-482.279565680167\\
60.875	0.19914	-508.96797204518	-508.96797204518\\
60.875	0.2028	-536.478118775007	-536.478118775007\\
60.875	0.20646	-564.81000586965	-564.81000586965\\
60.875	0.21012	-593.963633329108	-593.963633329108\\
60.875	0.21378	-623.939001153381	-623.939001153381\\
60.875	0.21744	-654.736109342469	-654.736109342469\\
60.875	0.2211	-686.354957896372	-686.354957896372\\
60.875	0.22476	-718.79554681509	-718.79554681509\\
60.875	0.22842	-752.057876098624	-752.057876098624\\
60.875	0.23208	-786.141945746973	-786.141945746973\\
60.875	0.23574	-821.047755760137	-821.047755760137\\
60.875	0.2394	-856.775306138116	-856.775306138116\\
60.875	0.24306	-893.32459688091	-893.32459688091\\
60.875	0.24672	-930.69562798852	-930.69562798852\\
60.875	0.25038	-968.888399460944	-968.888399460944\\
60.875	0.25404	-1007.90291129818	-1007.90291129818\\
60.875	0.2577	-1047.73916350024	-1047.73916350024\\
60.875	0.26136	-1088.39715606711	-1088.39715606711\\
60.875	0.26502	-1129.87688899879	-1129.87688899879\\
60.875	0.26868	-1172.17836229529	-1172.17836229529\\
60.875	0.27234	-1215.30157595661	-1215.30157595661\\
60.875	0.276	-1259.24652998274	-1259.24652998274\\
61.25	0.093	-68.790013954056	-68.790013954056\\
61.25	0.09666	-72.481652762057	-72.481652762057\\
61.25	0.10032	-76.9950319348728	-76.9950319348728\\
61.25	0.10398	-82.3301514725038	-82.3301514725038\\
61.25	0.10764	-88.4870113749503	-88.4870113749503\\
61.25	0.1113	-95.4656116422116	-95.4656116422116\\
61.25	0.11496	-103.265952274288	-103.265952274288\\
61.25	0.11862	-111.88803327118	-111.88803327118\\
61.25	0.12228	-121.331854632887	-121.331854632887\\
61.25	0.12594	-131.597416359409	-131.597416359409\\
61.25	0.1296	-142.684718450746	-142.684718450746\\
61.25	0.13326	-154.593760906899	-154.593760906899\\
61.25	0.13692	-167.324543727866	-167.324543727866\\
61.25	0.14058	-180.877066913649	-180.877066913649\\
61.25	0.14424	-195.251330464247	-195.251330464247\\
61.25	0.1479	-210.44733437966	-210.44733437966\\
61.25	0.15156	-226.465078659888	-226.465078659888\\
61.25	0.15522	-243.304563304932	-243.304563304932\\
61.25	0.15888	-260.96578831479	-260.96578831479\\
61.25	0.16254	-279.448753689464	-279.448753689464\\
61.25	0.1662	-298.753459428953	-298.753459428953\\
61.25	0.16986	-318.879905533257	-318.879905533257\\
61.25	0.17352	-339.828092002376	-339.828092002376\\
61.25	0.17718	-361.598018836311	-361.598018836311\\
61.25	0.18084	-384.18968603506	-384.18968603506\\
61.25	0.1845	-407.603093598625	-407.603093598625\\
61.25	0.18816	-431.838241527005	-431.838241527005\\
61.25	0.19182	-456.8951298202	-456.8951298202\\
61.25	0.19548	-482.77375847821	-482.77375847821\\
61.25	0.19914	-509.474127501035	-509.474127501035\\
61.25	0.2028	-536.996236888676	-536.996236888676\\
61.25	0.20646	-565.340086641132	-565.340086641132\\
61.25	0.21012	-594.505676758403	-594.505676758403\\
61.25	0.21378	-624.493007240489	-624.493007240489\\
61.25	0.21744	-655.30207808739	-655.30207808739\\
61.25	0.2211	-686.932889299106	-686.932889299106\\
61.25	0.22476	-719.385440875638	-719.385440875638\\
61.25	0.22842	-752.659732816984	-752.659732816984\\
61.25	0.23208	-786.755765123146	-786.755765123146\\
61.25	0.23574	-821.673537794124	-821.673537794124\\
61.25	0.2394	-857.413050829916	-857.413050829916\\
61.25	0.24306	-893.974304230523	-893.974304230523\\
61.25	0.24672	-931.357297995946	-931.357297995946\\
61.25	0.25038	-969.562032126183	-969.562032126183\\
61.25	0.25404	-1008.58850662124	-1008.58850662124\\
61.25	0.2577	-1048.4367214811	-1048.4367214811\\
61.25	0.26136	-1089.10667670579	-1089.10667670579\\
61.25	0.26502	-1130.59837229529	-1130.59837229529\\
61.25	0.26868	-1172.9118082496	-1172.9118082496\\
61.25	0.27234	-1216.04698456873	-1216.04698456873\\
61.25	0.276	-1260.00390125267	-1260.00390125267\\
61.625	0.093	-69.020285890233	-69.020285890233\\
61.625	0.09666	-72.723887356047	-72.723887356047\\
61.625	0.10032	-77.249229186676	-77.249229186676\\
61.625	0.10398	-82.5963113821201	-82.5963113821201\\
61.625	0.10764	-88.7651339423796	-88.7651339423796\\
61.625	0.1113	-95.755696867454	-95.755696867454\\
61.625	0.11496	-103.568000157344	-103.568000157344\\
61.625	0.11862	-112.202043812049	-112.202043812049\\
61.625	0.12228	-121.657827831569	-121.657827831569\\
61.625	0.12594	-131.935352215904	-131.935352215904\\
61.625	0.1296	-143.034616965054	-143.034616965054\\
61.625	0.13326	-154.95562207902	-154.95562207902\\
61.625	0.13692	-167.6983675578	-167.6983675578\\
61.625	0.14058	-181.262853401396	-181.262853401396\\
61.625	0.14424	-195.649079609807	-195.649079609807\\
61.625	0.1479	-210.857046183033	-210.857046183033\\
61.625	0.15156	-226.886753121074	-226.886753121074\\
61.625	0.15522	-243.738200423931	-243.738200423931\\
61.625	0.15888	-261.411388091603	-261.411388091603\\
61.625	0.16254	-279.90631612409	-279.90631612409\\
61.625	0.1662	-299.222984521392	-299.222984521392\\
61.625	0.16986	-319.361393283509	-319.361393283509\\
61.625	0.17352	-340.321542410441	-340.321542410441\\
61.625	0.17718	-362.103431902189	-362.103431902189\\
61.625	0.18084	-384.707061758751	-384.707061758751\\
61.625	0.1845	-408.132431980129	-408.132431980129\\
61.625	0.18816	-432.379542566322	-432.379542566322\\
61.625	0.19182	-457.44839351733	-457.44839351733\\
61.625	0.19548	-483.338984833153	-483.338984833153\\
61.625	0.19914	-510.051316513792	-510.051316513792\\
61.625	0.2028	-537.585388559246	-537.585388559246\\
61.625	0.20646	-565.941200969514	-565.941200969514\\
61.625	0.21012	-595.118753744598	-595.118753744598\\
61.625	0.21378	-625.118046884497	-625.118046884497\\
61.625	0.21744	-655.939080389212	-655.939080389212\\
61.625	0.2211	-687.581854258741	-687.581854258741\\
61.625	0.22476	-720.046368493085	-720.046368493085\\
61.625	0.22842	-753.332623092245	-753.332623092245\\
61.625	0.23208	-787.44061805622	-787.44061805622\\
61.625	0.23574	-822.370353385011	-822.370353385011\\
61.625	0.2394	-858.121829078616	-858.121829078616\\
61.625	0.24306	-894.695045137036	-894.695045137036\\
61.625	0.24672	-932.090001560272	-932.090001560272\\
61.625	0.25038	-970.306698348323	-970.306698348323\\
61.625	0.25404	-1009.34513550119	-1009.34513550119\\
61.625	0.2577	-1049.20531301887	-1049.20531301887\\
61.625	0.26136	-1089.88723090137	-1089.88723090137\\
61.625	0.26502	-1131.39088914868	-1131.39088914868\\
61.625	0.26868	-1173.7162877608	-1173.7162877608\\
61.625	0.27234	-1216.86342673775	-1216.86342673775\\
61.625	0.276	-1260.8323060795	-1260.8323060795\\
62	0.093	-69.321591383311	-69.321591383311\\
62	0.09666	-73.0371555069378	-73.0371555069378\\
62	0.10032	-77.5744599953802	-77.5744599953802\\
62	0.10398	-82.9335048486375	-82.9335048486375\\
62	0.10764	-89.1142900667098	-89.1142900667098\\
62	0.1113	-96.1168156495975	-96.1168156495975\\
62	0.11496	-103.9410815973	-103.9410815973\\
62	0.11862	-112.587087909818	-112.587087909818\\
62	0.12228	-122.054834587151	-122.054834587151\\
62	0.12594	-132.344321629299	-132.344321629299\\
62	0.1296	-143.455549036263	-143.455549036263\\
62	0.13326	-155.388516808041	-155.388516808041\\
62	0.13692	-168.143224944635	-168.143224944635\\
62	0.14058	-181.719673446044	-181.719673446044\\
62	0.14424	-196.117862312268	-196.117862312268\\
62	0.1479	-211.337791543307	-211.337791543307\\
62	0.15156	-227.379461139162	-227.379461139162\\
62	0.15522	-244.242871099831	-244.242871099831\\
62	0.15888	-261.928021425316	-261.928021425316\\
62	0.16254	-280.434912115616	-280.434912115616\\
62	0.1662	-299.763543170731	-299.763543170731\\
62	0.16986	-319.913914590661	-319.913914590661\\
62	0.17352	-340.886026375407	-340.886026375407\\
62	0.17718	-362.679878524967	-362.679878524967\\
62	0.18084	-385.295471039343	-385.295471039343\\
62	0.1845	-408.732803918534	-408.732803918534\\
62	0.18816	-432.99187716254	-432.99187716254\\
62	0.19182	-458.072690771361	-458.072690771361\\
62	0.19548	-483.975244744998	-483.975244744998\\
62	0.19914	-510.699539083449	-510.699539083449\\
62	0.2028	-538.245573786716	-538.245573786716\\
62	0.20646	-566.613348854798	-566.613348854798\\
62	0.21012	-595.802864287695	-595.802864287695\\
62	0.21378	-625.814120085407	-625.814120085407\\
62	0.21744	-656.647116247934	-656.647116247934\\
62	0.2211	-688.301852775277	-688.301852775277\\
62	0.22476	-720.778329667434	-720.778329667434\\
62	0.22842	-754.076546924407	-754.076546924407\\
62	0.23208	-788.196504546195	-788.196504546195\\
62	0.23574	-823.138202532799	-823.138202532799\\
62	0.2394	-858.901640884217	-858.901640884217\\
62	0.24306	-895.486819600451	-895.486819600451\\
62	0.24672	-932.893738681499	-932.893738681499\\
62	0.25038	-971.122398127363	-971.122398127363\\
62	0.25404	-1010.17279793804	-1010.17279793804\\
62	0.2577	-1050.04493811354	-1050.04493811354\\
62	0.26136	-1090.73881865385	-1090.73881865385\\
62	0.26502	-1132.25443955897	-1132.25443955897\\
62	0.26868	-1174.59180082891	-1174.59180082891\\
62	0.27234	-1217.75090246366	-1217.75090246366\\
62	0.276	-1261.73174446323	-1261.73174446323\\
62.375	0.093	-69.6939304332896	-69.6939304332896\\
62.375	0.09666	-73.4214572147293	-73.4214572147293\\
62.375	0.10032	-77.9707243609845	-77.9707243609845\\
62.375	0.10398	-83.341731872055	-83.341731872055\\
62.375	0.10764	-89.5344797479403	-89.5344797479403\\
62.375	0.1113	-96.548967988641	-96.548967988641\\
62.375	0.11496	-104.385196594157	-104.385196594157\\
62.375	0.11862	-113.043165564488	-113.043165564488\\
62.375	0.12228	-122.522874899634	-122.522874899634\\
62.375	0.12594	-132.824324599595	-132.824324599595\\
62.375	0.1296	-143.947514664372	-143.947514664372\\
62.375	0.13326	-155.892445093964	-155.892445093964\\
62.375	0.13692	-168.65911588837	-168.65911588837\\
62.375	0.14058	-182.247527047593	-182.247527047593\\
62.375	0.14424	-196.65767857163	-196.65767857163\\
62.375	0.1479	-211.889570460482	-211.889570460482\\
62.375	0.15156	-227.94320271415	-227.94320271415\\
62.375	0.15522	-244.818575332632	-244.818575332632\\
62.375	0.15888	-262.51568831593	-262.51568831593\\
62.375	0.16254	-281.034541664043	-281.034541664043\\
62.375	0.1662	-300.375135376971	-300.375135376971\\
62.375	0.16986	-320.537469454714	-320.537469454714\\
62.375	0.17352	-341.521543897273	-341.521543897273\\
62.375	0.17718	-363.327358704646	-363.327358704646\\
62.375	0.18084	-385.954913876835	-385.954913876835\\
62.375	0.1845	-409.404209413839	-409.404209413839\\
62.375	0.18816	-433.675245315658	-433.675245315658\\
62.375	0.19182	-458.768021582293	-458.768021582293\\
62.375	0.19548	-484.682538213742	-484.682538213742\\
62.375	0.19914	-511.418795210007	-511.418795210007\\
62.375	0.2028	-538.976792571087	-538.976792571087\\
62.375	0.20646	-567.356530296982	-567.356530296982\\
62.375	0.21012	-596.558008387692	-596.558008387692\\
62.375	0.21378	-626.581226843217	-626.581226843217\\
62.375	0.21744	-657.426185663557	-657.426185663557\\
62.375	0.2211	-689.092884848713	-689.092884848713\\
62.375	0.22476	-721.581324398683	-721.581324398683\\
62.375	0.22842	-754.89150431347	-754.89150431347\\
62.375	0.23208	-789.023424593071	-789.023424593071\\
62.375	0.23574	-823.977085237487	-823.977085237487\\
62.375	0.2394	-859.752486246719	-859.752486246719\\
62.375	0.24306	-896.349627620765	-896.349627620765\\
62.375	0.24672	-933.768509359627	-933.768509359627\\
62.375	0.25038	-972.009131463304	-972.009131463304\\
62.375	0.25404	-1011.0714939318	-1011.0714939318\\
62.375	0.2577	-1050.9555967651	-1050.9555967651\\
62.375	0.26136	-1091.66143996323	-1091.66143996323\\
62.375	0.26502	-1133.18902352616	-1133.18902352616\\
62.375	0.26868	-1175.53834745392	-1175.53834745392\\
62.375	0.27234	-1218.70941174648	-1218.70941174648\\
62.375	0.276	-1262.70221640387	-1262.70221640387\\
62.75	0.093	-70.1373030401687	-70.1373030401687\\
62.75	0.09666	-73.8767924794217	-73.8767924794217\\
62.75	0.10032	-78.4380222834899	-78.4380222834899\\
62.75	0.10398	-83.8209924523734	-83.8209924523734\\
62.75	0.10764	-90.0257029860722	-90.0257029860722\\
62.75	0.1113	-97.0521538845857	-97.0521538845857\\
62.75	0.11496	-104.900345147915	-104.900345147915\\
62.75	0.11862	-113.570276776059	-113.570276776059\\
62.75	0.12228	-123.061948769018	-123.061948769018\\
62.75	0.12594	-133.375361126793	-133.375361126793\\
62.75	0.1296	-144.510513849382	-144.510513849382\\
62.75	0.13326	-156.467406936787	-156.467406936787\\
62.75	0.13692	-169.246040389007	-169.246040389007\\
62.75	0.14058	-182.846414206042	-182.846414206042\\
62.75	0.14424	-197.268528387892	-197.268528387892\\
62.75	0.1479	-212.512382934558	-212.512382934558\\
62.75	0.15156	-228.577977846038	-228.577977846038\\
62.75	0.15522	-245.465313122334	-245.465313122334\\
62.75	0.15888	-263.174388763445	-263.174388763445\\
62.75	0.16254	-281.705204769371	-281.705204769371\\
62.75	0.1662	-301.057761140112	-301.057761140112\\
62.75	0.16986	-321.232057875668	-321.232057875668\\
62.75	0.17352	-342.22809497604	-342.22809497604\\
62.75	0.17718	-364.045872441227	-364.045872441227\\
62.75	0.18084	-386.685390271228	-386.685390271228\\
62.75	0.1845	-410.146648466046	-410.146648466046\\
62.75	0.18816	-434.429647025678	-434.429647025678\\
62.75	0.19182	-459.534385950125	-459.534385950125\\
62.75	0.19548	-485.460865239388	-485.460865239388\\
62.75	0.19914	-512.209084893465	-512.209084893465\\
62.75	0.2028	-539.779044912358	-539.779044912358\\
62.75	0.20646	-568.170745296066	-568.170745296066\\
62.75	0.21012	-597.384186044589	-597.384186044589\\
62.75	0.21378	-627.419367157928	-627.419367157928\\
62.75	0.21744	-658.276288636081	-658.276288636081\\
62.75	0.2211	-689.95495047905	-689.95495047905\\
62.75	0.22476	-722.455352686834	-722.455352686834\\
62.75	0.22842	-755.777495259433	-755.777495259433\\
62.75	0.23208	-789.921378196847	-789.921378196847\\
62.75	0.23574	-824.887001499077	-824.887001499077\\
62.75	0.2394	-860.674365166121	-860.674365166121\\
62.75	0.24306	-897.28346919798	-897.28346919798\\
62.75	0.24672	-934.714313594655	-934.714313594655\\
62.75	0.25038	-972.966898356145	-972.966898356145\\
62.75	0.25404	-1012.04122348245	-1012.04122348245\\
62.75	0.2577	-1051.93728897357	-1051.93728897357\\
62.75	0.26136	-1092.65509482951	-1092.65509482951\\
62.75	0.26502	-1134.19464105026	-1134.19464105026\\
62.75	0.26868	-1176.55592763582	-1176.55592763582\\
62.75	0.27234	-1219.7389545862	-1219.7389545862\\
62.75	0.276	-1263.7437219014	-1263.7437219014\\
63.125	0.093	-70.6517092039484	-70.6517092039484\\
63.125	0.09666	-74.4031613010146	-74.4031613010146\\
63.125	0.10032	-78.9763537628958	-78.9763537628958\\
63.125	0.10398	-84.3712865895923	-84.3712865895923\\
63.125	0.10764	-90.5879597811041	-90.5879597811041\\
63.125	0.1113	-97.6263733374308	-97.6263733374308\\
63.125	0.11496	-105.486527258573	-105.486527258573\\
63.125	0.11862	-114.16842154453	-114.16842154453\\
63.125	0.12228	-123.672056195302	-123.672056195302\\
63.125	0.12594	-133.99743121089	-133.99743121089\\
63.125	0.1296	-145.144546591292	-145.144546591292\\
63.125	0.13326	-157.11340233651	-157.11340233651\\
63.125	0.13692	-169.903998446543	-169.903998446543\\
63.125	0.14058	-183.516334921391	-183.516334921391\\
63.125	0.14424	-197.950411761055	-197.950411761055\\
63.125	0.1479	-213.206228965533	-213.206228965533\\
63.125	0.15156	-229.283786534827	-229.283786534827\\
63.125	0.15522	-246.183084468936	-246.183084468936\\
63.125	0.15888	-263.90412276786	-263.90412276786\\
63.125	0.16254	-282.446901431599	-282.446901431599\\
63.125	0.1662	-301.811420460153	-301.811420460153\\
63.125	0.16986	-321.997679853523	-321.997679853523\\
63.125	0.17352	-343.005679611707	-343.005679611707\\
63.125	0.17718	-364.835419734707	-364.835419734707\\
63.125	0.18084	-387.486900222522	-387.486900222522\\
63.125	0.1845	-410.960121075152	-410.960121075152\\
63.125	0.18816	-435.255082292597	-435.255082292597\\
63.125	0.19182	-460.371783874858	-460.371783874858\\
63.125	0.19548	-486.310225821933	-486.310225821933\\
63.125	0.19914	-513.070408133824	-513.070408133824\\
63.125	0.2028	-540.65233081053	-540.65233081053\\
63.125	0.20646	-569.055993852051	-569.055993852051\\
63.125	0.21012	-598.281397258388	-598.281397258388\\
63.125	0.21378	-628.328541029539	-628.328541029539\\
63.125	0.21744	-659.197425165506	-659.197425165506\\
63.125	0.2211	-690.888049666287	-690.888049666287\\
63.125	0.22476	-723.400414531884	-723.400414531884\\
63.125	0.22842	-756.734519762296	-756.734519762296\\
63.125	0.23208	-790.890365357523	-790.890365357523\\
63.125	0.23574	-825.867951317566	-825.867951317566\\
63.125	0.2394	-861.667277642424	-861.667277642424\\
63.125	0.24306	-898.288344332097	-898.288344332097\\
63.125	0.24672	-935.731151386584	-935.731151386584\\
63.125	0.25038	-973.995698805888	-973.995698805888\\
63.125	0.25404	-1013.08198659001	-1013.08198659001\\
63.125	0.2577	-1052.99001473894	-1052.99001473894\\
63.125	0.26136	-1093.71978325269	-1093.71978325269\\
63.125	0.26502	-1135.27129213125	-1135.27129213125\\
63.125	0.26868	-1177.64454137463	-1177.64454137463\\
63.125	0.27234	-1220.83953098282	-1220.83953098282\\
63.125	0.276	-1264.85626095583	-1264.85626095583\\
63.5	0.093	-71.2371489246289	-71.2371489246289\\
63.5	0.09666	-75.0005636795079	-75.0005636795079\\
63.5	0.10032	-79.5857187992025	-79.5857187992025\\
63.5	0.10398	-84.992614283712	-84.992614283712\\
63.5	0.10764	-91.2212501330368	-91.2212501330368\\
63.5	0.1113	-98.2716263471767	-98.2716263471767\\
63.5	0.11496	-106.143742926132	-106.143742926132\\
63.5	0.11862	-114.837599869902	-114.837599869902\\
63.5	0.12228	-124.353197178488	-124.353197178488\\
63.5	0.12594	-134.690534851888	-134.690534851888\\
63.5	0.1296	-145.849612890104	-145.849612890104\\
63.5	0.13326	-157.830431293135	-157.830431293135\\
63.5	0.13692	-170.632990060981	-170.632990060981\\
63.5	0.14058	-184.257289193642	-184.257289193642\\
63.5	0.14424	-198.703328691119	-198.703328691119\\
63.5	0.1479	-213.97110855341	-213.97110855341\\
63.5	0.15156	-230.060628780517	-230.060628780517\\
63.5	0.15522	-246.971889372439	-246.971889372439\\
63.5	0.15888	-264.704890329176	-264.704890329176\\
63.5	0.16254	-283.259631650728	-283.259631650728\\
63.5	0.1662	-302.636113337095	-302.636113337095\\
63.5	0.16986	-322.834335388278	-322.834335388278\\
63.5	0.17352	-343.854297804276	-343.854297804276\\
63.5	0.17718	-365.696000585088	-365.696000585088\\
63.5	0.18084	-388.359443730716	-388.359443730716\\
63.5	0.1845	-411.84462724116	-411.84462724116\\
63.5	0.18816	-436.151551116418	-436.151551116418\\
63.5	0.19182	-461.280215356492	-461.280215356492\\
63.5	0.19548	-487.23061996138	-487.23061996138\\
63.5	0.19914	-514.002764931084	-514.002764931084\\
63.5	0.2028	-541.596650265603	-541.596650265603\\
63.5	0.20646	-570.012275964937	-570.012275964937\\
63.5	0.21012	-599.249642029087	-599.249642029087\\
63.5	0.21378	-629.308748458051	-629.308748458051\\
63.5	0.21744	-660.189595251831	-660.189595251831\\
63.5	0.2211	-691.892182410426	-691.892182410426\\
63.5	0.22476	-724.416509933836	-724.416509933836\\
63.5	0.22842	-757.762577822061	-757.762577822061\\
63.5	0.23208	-791.930386075101	-791.930386075101\\
63.5	0.23574	-826.919934692957	-826.919934692957\\
63.5	0.2394	-862.731223675627	-862.731223675627\\
63.5	0.24306	-899.364253023113	-899.364253023113\\
63.5	0.24672	-936.819022735414	-936.819022735414\\
63.5	0.25038	-975.095532812531	-975.095532812531\\
63.5	0.25404	-1014.19378325446	-1014.19378325446\\
63.5	0.2577	-1054.11377406121	-1054.11377406121\\
63.5	0.26136	-1094.85550523277	-1094.85550523277\\
63.5	0.26502	-1136.41897676915	-1136.41897676915\\
63.5	0.26868	-1178.80418867034	-1178.80418867034\\
63.5	0.27234	-1222.01114093635	-1222.01114093635\\
63.5	0.276	-1266.03983356717	-1266.03983356717\\
63.875	0.093	-71.8936222022097	-71.8936222022097\\
63.875	0.09666	-75.668999614902	-75.668999614902\\
63.875	0.10032	-80.2661173924094	-80.2661173924094\\
63.875	0.10398	-85.6849755347321	-85.6849755347321\\
63.875	0.10764	-91.9255740418699	-91.9255740418699\\
63.875	0.1113	-98.987912913823	-98.987912913823\\
63.875	0.11496	-106.871992150591	-106.871992150591\\
63.875	0.11862	-115.577811752174	-115.577811752174\\
63.875	0.12228	-125.105371718573	-125.105371718573\\
63.875	0.12594	-135.454672049787	-135.454672049787\\
63.875	0.1296	-146.625712745815	-146.625712745815\\
63.875	0.13326	-158.618493806659	-158.618493806659\\
63.875	0.13692	-171.433015232318	-171.433015232318\\
63.875	0.14058	-185.069277022793	-185.069277022793\\
63.875	0.14424	-199.527279178082	-199.527279178082\\
63.875	0.1479	-214.807021698187	-214.807021698187\\
63.875	0.15156	-230.908504583107	-230.908504583107\\
63.875	0.15522	-247.831727832842	-247.831727832842\\
63.875	0.15888	-265.576691447392	-265.576691447392\\
63.875	0.16254	-284.143395426757	-284.143395426757\\
63.875	0.1662	-303.531839770938	-303.531839770938\\
63.875	0.16986	-323.742024479933	-323.742024479933\\
63.875	0.17352	-344.773949553744	-344.773949553744\\
63.875	0.17718	-366.62761499237	-366.62761499237\\
63.875	0.18084	-389.303020795811	-389.303020795811\\
63.875	0.1845	-412.800166964068	-412.800166964068\\
63.875	0.18816	-437.119053497139	-437.119053497139\\
63.875	0.19182	-462.259680395025	-462.259680395025\\
63.875	0.19548	-488.222047657727	-488.222047657727\\
63.875	0.19914	-515.006155285244	-515.006155285244\\
63.875	0.2028	-542.612003277577	-542.612003277577\\
63.875	0.20646	-571.039591634724	-571.039591634724\\
63.875	0.21012	-600.288920356686	-600.288920356686\\
63.875	0.21378	-630.359989443464	-630.359989443464\\
63.875	0.21744	-661.252798895056	-661.252798895056\\
63.875	0.2211	-692.967348711464	-692.967348711464\\
63.875	0.22476	-725.503638892687	-725.503638892687\\
63.875	0.22842	-758.861669438726	-758.861669438726\\
63.875	0.23208	-793.041440349579	-793.041440349579\\
63.875	0.23574	-828.042951625248	-828.042951625248\\
63.875	0.2394	-863.866203265731	-863.866203265731\\
63.875	0.24306	-900.51119527103	-900.51119527103\\
63.875	0.24672	-937.977927641145	-937.977927641145\\
63.875	0.25038	-976.266400376074	-976.266400376074\\
63.875	0.25404	-1015.37661347582	-1015.37661347582\\
63.875	0.2577	-1055.30856694038	-1055.30856694038\\
63.875	0.26136	-1096.06226076975	-1096.06226076975\\
63.875	0.26502	-1137.63769496394	-1137.63769496394\\
63.875	0.26868	-1180.03486952295	-1180.03486952295\\
63.875	0.27234	-1223.25378444677	-1223.25378444677\\
63.875	0.276	-1267.2944397354	-1267.2944397354\\
64.25	0.093	-72.6211290366915	-72.6211290366915\\
64.25	0.09666	-76.408469107197	-76.408469107197\\
64.25	0.10032	-81.0175495425174	-81.0175495425174\\
64.25	0.10398	-86.4483703426532	-86.4483703426532\\
64.25	0.10764	-92.7009315076044	-92.7009315076044\\
64.25	0.1113	-99.7752330373703	-99.7752330373703\\
64.25	0.11496	-107.671274931952	-107.671274931952\\
64.25	0.11862	-116.389057191348	-116.389057191348\\
64.25	0.12228	-125.928579815559	-125.928579815559\\
64.25	0.12594	-136.289842804586	-136.289842804586\\
64.25	0.1296	-147.472846158428	-147.472846158428\\
64.25	0.13326	-159.477589877085	-159.477589877085\\
64.25	0.13692	-172.304073960557	-172.304073960557\\
64.25	0.14058	-185.952298408845	-185.952298408845\\
64.25	0.14424	-200.422263221947	-200.422263221947\\
64.25	0.1479	-215.713968399865	-215.713968399865\\
64.25	0.15156	-231.827413942598	-231.827413942598\\
64.25	0.15522	-248.762599850146	-248.762599850146\\
64.25	0.15888	-266.519526122509	-266.519526122509\\
64.25	0.16254	-285.098192759688	-285.098192759688\\
64.25	0.1662	-304.498599761681	-304.498599761681\\
64.25	0.16986	-324.72074712849	-324.72074712849\\
64.25	0.17352	-345.764634860113	-345.764634860113\\
64.25	0.17718	-367.630262956553	-367.630262956553\\
64.25	0.18084	-390.317631417807	-390.317631417807\\
64.25	0.1845	-413.826740243876	-413.826740243876\\
64.25	0.18816	-438.157589434761	-438.157589434761\\
64.25	0.19182	-463.310178990461	-463.310178990461\\
64.25	0.19548	-489.284508910975	-489.284508910975\\
64.25	0.19914	-516.080579196305	-516.080579196305\\
64.25	0.2028	-543.698389846451	-543.698389846451\\
64.25	0.20646	-572.137940861411	-572.137940861411\\
64.25	0.21012	-601.399232241186	-601.399232241186\\
64.25	0.21378	-631.482263985777	-631.482263985777\\
64.25	0.21744	-662.387036095183	-662.387036095183\\
64.25	0.2211	-694.113548569404	-694.113548569404\\
64.25	0.22476	-726.66180140844	-726.66180140844\\
64.25	0.22842	-760.031794612291	-760.031794612291\\
64.25	0.23208	-794.223528180958	-794.223528180958\\
64.25	0.23574	-829.23700211444	-829.23700211444\\
64.25	0.2394	-865.072216412736	-865.072216412736\\
64.25	0.24306	-901.729171075848	-901.729171075848\\
64.25	0.24672	-939.207866103776	-939.207866103776\\
64.25	0.25038	-977.508301496518	-977.508301496518\\
64.25	0.25404	-1016.63047725408	-1016.63047725408\\
64.25	0.2577	-1056.57439337645	-1056.57439337645\\
64.25	0.26136	-1097.34004986364	-1097.34004986364\\
64.25	0.26502	-1138.92744671564	-1138.92744671564\\
64.25	0.26868	-1181.33658393246	-1181.33658393246\\
64.25	0.27234	-1224.56746151409	-1224.56746151409\\
64.25	0.276	-1268.62007946054	-1268.62007946054\\
64.625	0.093	-73.4196694280737	-73.4196694280737\\
64.625	0.09666	-77.2189721563924	-77.2189721563924\\
64.625	0.10032	-81.8400152495258	-81.8400152495258\\
64.625	0.10398	-87.2827987074748	-87.2827987074748\\
64.625	0.10764	-93.5473225302388	-93.5473225302388\\
64.625	0.1113	-100.633586717818	-100.633586717818\\
64.625	0.11496	-108.541591270212	-108.541591270212\\
64.625	0.11862	-117.271336187422	-117.271336187422\\
64.625	0.12228	-126.822821469446	-126.822821469446\\
64.625	0.12594	-137.196047116286	-137.196047116286\\
64.625	0.1296	-148.391013127941	-148.391013127941\\
64.625	0.13326	-160.407719504411	-160.407719504411\\
64.625	0.13692	-173.246166245696	-173.246166245696\\
64.625	0.14058	-186.906353351797	-186.906353351797\\
64.625	0.14424	-201.388280822712	-201.388280822712\\
64.625	0.1479	-216.691948658443	-216.691948658443\\
64.625	0.15156	-232.817356858989	-232.817356858989\\
64.625	0.15522	-249.764505424351	-249.764505424351\\
64.625	0.15888	-267.533394354527	-267.533394354527\\
64.625	0.16254	-286.124023649518	-286.124023649518\\
64.625	0.1662	-305.536393309325	-305.536393309325\\
64.625	0.16986	-325.770503333947	-325.770503333947\\
64.625	0.17352	-346.826353723383	-346.826353723383\\
64.625	0.17718	-368.703944477636	-368.703944477636\\
64.625	0.18084	-391.403275596703	-391.403275596703\\
64.625	0.1845	-414.924347080585	-414.924347080585\\
64.625	0.18816	-439.267158929283	-439.267158929283\\
64.625	0.19182	-464.431711142796	-464.431711142796\\
64.625	0.19548	-490.418003721124	-490.418003721124\\
64.625	0.19914	-517.226036664267	-517.226036664267\\
64.625	0.2028	-544.855809972225	-544.855809972225\\
64.625	0.20646	-573.307323644999	-573.307323644999\\
64.625	0.21012	-602.580577682587	-602.580577682587\\
64.625	0.21378	-632.675572084991	-632.675572084991\\
64.625	0.21744	-663.59230685221	-663.59230685221\\
64.625	0.2211	-695.330781984244	-695.330781984244\\
64.625	0.22476	-727.890997481093	-727.890997481093\\
64.625	0.22842	-761.272953342757	-761.272953342757\\
64.625	0.23208	-795.476649569237	-795.476649569237\\
64.625	0.23574	-830.502086160532	-830.502086160532\\
64.625	0.2394	-866.349263116642	-866.349263116642\\
64.625	0.24306	-903.018180437567	-903.018180437567\\
64.625	0.24672	-940.508838123307	-940.508838123307\\
64.625	0.25038	-978.821236173863	-978.821236173863\\
64.625	0.25404	-1017.95537458923	-1017.95537458923\\
64.625	0.2577	-1057.91125336942	-1057.91125336942\\
64.625	0.26136	-1098.68887251442	-1098.68887251442\\
64.625	0.26502	-1140.28823202424	-1140.28823202424\\
64.625	0.26868	-1182.70933189887	-1182.70933189887\\
64.625	0.27234	-1225.95217213831	-1225.95217213831\\
64.625	0.276	-1270.01675274257	-1270.01675274257\\
65	0.093	-74.2892433763569	-74.2892433763569\\
65	0.09666	-78.1005087624884	-78.1005087624884\\
65	0.10032	-82.7335145134353	-82.7335145134353\\
65	0.10398	-88.1882606291973	-88.1882606291973\\
65	0.10764	-94.464747109774	-94.464747109774\\
65	0.1113	-101.562973955166	-101.562973955166\\
65	0.11496	-109.482941165374	-109.482941165374\\
65	0.11862	-118.224648740396	-118.224648740396\\
65	0.12228	-127.788096680234	-127.788096680234\\
65	0.12594	-138.173284984887	-138.173284984887\\
65	0.1296	-149.380213654355	-149.380213654355\\
65	0.13326	-161.408882688638	-161.408882688638\\
65	0.13692	-174.259292087737	-174.259292087737\\
65	0.14058	-187.93144185165	-187.93144185165\\
65	0.14424	-202.425331980379	-202.425331980379\\
65	0.1479	-217.740962473923	-217.740962473923\\
65	0.15156	-233.878333332282	-233.878333332282\\
65	0.15522	-250.837444555456	-250.837444555456\\
65	0.15888	-268.618296143445	-268.618296143445\\
65	0.16254	-287.22088809625	-287.22088809625\\
65	0.1662	-306.64522041387	-306.64522041387\\
65	0.16986	-326.891293096304	-326.891293096304\\
65	0.17352	-347.959106143555	-347.959106143555\\
65	0.17718	-369.84865955562	-369.84865955562\\
65	0.18084	-392.5599533325	-392.5599533325\\
65	0.1845	-416.092987474196	-416.092987474196\\
65	0.18816	-440.447761980706	-440.447761980706\\
65	0.19182	-465.624276852032	-465.624276852032\\
65	0.19548	-491.622532088173	-491.622532088173\\
65	0.19914	-518.442527689129	-518.442527689129\\
65	0.2028	-546.084263654901	-546.084263654901\\
65	0.20646	-574.547739985487	-574.547739985487\\
65	0.21012	-603.832956680889	-603.832956680889\\
65	0.21378	-633.939913741106	-633.939913741106\\
65	0.21744	-664.868611166138	-664.868611166138\\
65	0.2211	-696.619048955985	-696.619048955985\\
65	0.22476	-729.191227110647	-729.191227110647\\
65	0.22842	-762.585145630125	-762.585145630125\\
65	0.23208	-796.800804514417	-796.800804514417\\
65	0.23574	-831.838203763525	-831.838203763525\\
65	0.2394	-867.697343377448	-867.697343377448\\
65	0.24306	-904.378223356186	-904.378223356186\\
65	0.24672	-941.88084369974	-941.88084369974\\
65	0.25038	-980.205204408108	-980.205204408108\\
65	0.25404	-1019.35130548129	-1019.35130548129\\
65	0.2577	-1059.31914691929	-1059.31914691929\\
65	0.26136	-1100.1087287221	-1100.1087287221\\
65	0.26502	-1141.72005088973	-1141.72005088973\\
65	0.26868	-1184.15311342218	-1184.15311342218\\
65	0.27234	-1227.40791631944	-1227.40791631944\\
65	0.276	-1271.48445958151	-1271.48445958151\\
65.375	0.093	-75.2298508815403	-75.2298508815403\\
65.375	0.09666	-79.053078925485	-79.053078925485\\
65.375	0.10032	-83.6980473342449	-83.6980473342449\\
65.375	0.10398	-89.1647561078199	-89.1647561078199\\
65.375	0.10764	-95.4532052462098	-95.4532052462098\\
65.375	0.1113	-102.563394749415	-102.563394749415\\
65.375	0.11496	-110.495324617436	-110.495324617436\\
65.375	0.11862	-119.248994850271	-119.248994850271\\
65.375	0.12228	-128.824405447922	-128.824405447922\\
65.375	0.12594	-139.221556410388	-139.221556410388\\
65.375	0.1296	-150.440447737669	-150.440447737669\\
65.375	0.13326	-162.481079429766	-162.481079429766\\
65.375	0.13692	-175.343451486677	-175.343451486677\\
65.375	0.14058	-189.027563908404	-189.027563908404\\
65.375	0.14424	-203.533416694946	-203.533416694946\\
65.375	0.1479	-218.861009846302	-218.861009846302\\
65.375	0.15156	-235.010343362475	-235.010343362475\\
65.375	0.15522	-251.981417243462	-251.981417243462\\
65.375	0.15888	-269.774231489264	-269.774231489264\\
65.375	0.16254	-288.388786099882	-288.388786099882\\
65.375	0.1662	-307.825081075315	-307.825081075315\\
65.375	0.16986	-328.083116415562	-328.083116415562\\
65.375	0.17352	-349.162892120626	-349.162892120626\\
65.375	0.17718	-371.064408190504	-371.064408190504\\
65.375	0.18084	-393.787664625197	-393.787664625197\\
65.375	0.1845	-417.332661424706	-417.332661424706\\
65.375	0.18816	-441.69939858903	-441.69939858903\\
65.375	0.19182	-466.887876118169	-466.887876118169\\
65.375	0.19548	-492.898094012123	-492.898094012123\\
65.375	0.19914	-519.730052270892	-519.730052270892\\
65.375	0.2028	-547.383750894477	-547.383750894477\\
65.375	0.20646	-575.859189882876	-575.859189882876\\
65.375	0.21012	-605.156369236091	-605.156369236091\\
65.375	0.21378	-635.275288954121	-635.275288954121\\
65.375	0.21744	-666.215949036966	-666.215949036966\\
65.375	0.2211	-697.978349484626	-697.978349484626\\
65.375	0.22476	-730.562490297101	-730.562490297101\\
65.375	0.22842	-763.968371474392	-763.968371474392\\
65.375	0.23208	-798.195993016497	-798.195993016497\\
65.375	0.23574	-833.245354923419	-833.245354923419\\
65.375	0.2394	-869.116457195155	-869.116457195155\\
65.375	0.24306	-905.809299831706	-905.809299831706\\
65.375	0.24672	-943.323882833073	-943.323882833073\\
65.375	0.25038	-981.660206199254	-981.660206199254\\
65.375	0.25404	-1020.81826993025	-1020.81826993025\\
65.375	0.2577	-1060.79807402606	-1060.79807402606\\
65.375	0.26136	-1101.59961848669	-1101.59961848669\\
65.375	0.26502	-1143.22290331213	-1143.22290331213\\
65.375	0.26868	-1185.66792850239	-1185.66792850239\\
65.375	0.27234	-1228.93469405746	-1228.93469405746\\
65.375	0.276	-1273.02319997735	-1273.02319997735\\
65.75	0.093	-76.241491943625	-76.241491943625\\
65.75	0.09666	-80.0766826453827	-80.0766826453827\\
65.75	0.10032	-84.7336137119553	-84.7336137119553\\
65.75	0.10398	-90.2122851433435	-90.2122851433435\\
65.75	0.10764	-96.5126969395467	-96.5126969395467\\
65.75	0.1113	-103.634849100565	-103.634849100565\\
65.75	0.11496	-111.578741626399	-111.578741626399\\
65.75	0.11862	-120.344374517047	-120.344374517047\\
65.75	0.12228	-129.931747772511	-129.931747772511\\
65.75	0.12594	-140.340861392791	-140.340861392791\\
65.75	0.1296	-151.571715377884	-151.571715377884\\
65.75	0.13326	-163.624309727794	-163.624309727794\\
65.75	0.13692	-176.498644442518	-176.498644442518\\
65.75	0.14058	-190.194719522058	-190.194719522058\\
65.75	0.14424	-204.712534966413	-204.712534966413\\
65.75	0.1479	-220.052090775583	-220.052090775583\\
65.75	0.15156	-236.213386949568	-236.213386949568\\
65.75	0.15522	-253.196423488369	-253.196423488369\\
65.75	0.15888	-271.001200391984	-271.001200391984\\
65.75	0.16254	-289.627717660415	-289.627717660415\\
65.75	0.1662	-309.075975293661	-309.075975293661\\
65.75	0.16986	-329.345973291722	-329.345973291722\\
65.75	0.17352	-350.437711654598	-350.437711654598\\
65.75	0.17718	-372.351190382289	-372.351190382289\\
65.75	0.18084	-395.086409474796	-395.086409474796\\
65.75	0.1845	-418.643368932117	-418.643368932117\\
65.75	0.18816	-443.022068754254	-443.022068754254\\
65.75	0.19182	-468.222508941206	-468.222508941206\\
65.75	0.19548	-494.244689492974	-494.244689492974\\
65.75	0.19914	-521.088610409556	-521.088610409556\\
65.75	0.2028	-548.754271690954	-548.754271690954\\
65.75	0.20646	-577.241673337166	-577.241673337166\\
65.75	0.21012	-606.550815348194	-606.550815348194\\
65.75	0.21378	-636.681697724037	-636.681697724037\\
65.75	0.21744	-667.634320464695	-667.634320464695\\
65.75	0.2211	-699.408683570168	-699.408683570168\\
65.75	0.22476	-732.004787040456	-732.004787040456\\
65.75	0.22842	-765.42263087556	-765.42263087556\\
65.75	0.23208	-799.662215075479	-799.662215075479\\
65.75	0.23574	-834.723539640213	-834.723539640213\\
65.75	0.2394	-870.606604569763	-870.606604569763\\
65.75	0.24306	-907.311409864127	-907.311409864127\\
65.75	0.24672	-944.837955523306	-944.837955523306\\
65.75	0.25038	-983.186241547301	-983.186241547301\\
65.75	0.25404	-1022.35626793611	-1022.35626793611\\
65.75	0.2577	-1062.34803468974	-1062.34803468974\\
65.75	0.26136	-1103.16154180818	-1103.16154180818\\
65.75	0.26502	-1144.79678929143	-1144.79678929143\\
65.75	0.26868	-1187.2537771395	-1187.2537771395\\
65.75	0.27234	-1230.53250535239	-1230.53250535239\\
65.75	0.276	-1274.63297393009	-1274.63297393009\\
66.125	0.093	-77.3241665626098	-77.3241665626098\\
66.125	0.09666	-81.1713199221805	-81.1713199221805\\
66.125	0.10032	-85.8402136465664	-85.8402136465664\\
66.125	0.10398	-91.3308477357676	-91.3308477357676\\
66.125	0.10764	-97.643222189784	-97.643222189784\\
66.125	0.1113	-104.777337008615	-104.777337008615\\
66.125	0.11496	-112.733192192262	-112.733192192262\\
66.125	0.11862	-121.510787740724	-121.510787740724\\
66.125	0.12228	-131.110123654001	-131.110123654001\\
66.125	0.12594	-141.531199932093	-141.531199932093\\
66.125	0.1296	-152.774016575	-152.774016575\\
66.125	0.13326	-164.838573582723	-164.838573582723\\
66.125	0.13692	-177.72487095526	-177.72487095526\\
66.125	0.14058	-191.432908692613	-191.432908692613\\
66.125	0.14424	-205.962686794781	-205.962686794781\\
66.125	0.1479	-221.314205261764	-221.314205261764\\
66.125	0.15156	-237.487464093562	-237.487464093562\\
66.125	0.15522	-254.482463290176	-254.482463290176\\
66.125	0.15888	-272.299202851604	-272.299202851604\\
66.125	0.16254	-290.937682777848	-290.937682777848\\
66.125	0.1662	-310.397903068907	-310.397903068907\\
66.125	0.16986	-330.679863724781	-330.679863724781\\
66.125	0.17352	-351.78356474547	-351.78356474547\\
66.125	0.17718	-373.709006130975	-373.709006130975\\
66.125	0.18084	-396.456187881294	-396.456187881294\\
66.125	0.1845	-420.025109996429	-420.025109996429\\
66.125	0.18816	-444.415772476379	-444.415772476379\\
66.125	0.19182	-469.628175321144	-469.628175321144\\
66.125	0.19548	-495.662318530724	-495.662318530724\\
66.125	0.19914	-522.51820210512	-522.51820210512\\
66.125	0.2028	-550.195826044331	-550.195826044331\\
66.125	0.20646	-578.695190348356	-578.695190348356\\
66.125	0.21012	-608.016295017197	-608.016295017197\\
66.125	0.21378	-638.159140050853	-638.159140050853\\
66.125	0.21744	-669.123725449324	-669.123725449324\\
66.125	0.2211	-700.910051212611	-700.910051212611\\
66.125	0.22476	-733.518117340712	-733.518117340712\\
66.125	0.22842	-766.947923833629	-766.947923833629\\
66.125	0.23208	-801.199470691361	-801.199470691361\\
66.125	0.23574	-836.272757913908	-836.272757913908\\
66.125	0.2394	-872.167785501271	-872.167785501271\\
66.125	0.24306	-908.884553453448	-908.884553453448\\
66.125	0.24672	-946.42306177044	-946.42306177044\\
66.125	0.25038	-984.783310452248	-984.783310452248\\
66.125	0.25404	-1023.96529949887	-1023.96529949887\\
66.125	0.2577	-1063.96902891031	-1063.96902891031\\
66.125	0.26136	-1104.79449868656	-1104.79449868656\\
66.125	0.26502	-1146.44170882763	-1146.44170882763\\
66.125	0.26868	-1188.91065933351	-1188.91065933351\\
66.125	0.27234	-1232.20135020421	-1232.20135020421\\
66.125	0.276	-1276.31378143973	-1276.31378143973\\
66.5	0.093	-78.4778747384952	-78.4778747384952\\
66.5	0.09666	-82.3369907558791	-82.3369907558791\\
66.5	0.10032	-87.017847138078	-87.017847138078\\
66.5	0.10398	-92.5204438850922	-92.5204438850922\\
66.5	0.10764	-98.8447809969217	-98.8447809969217\\
66.5	0.1113	-105.990858473566	-105.990858473566\\
66.5	0.11496	-113.958676315026	-113.958676315026\\
66.5	0.11862	-122.748234521301	-122.748234521301\\
66.5	0.12228	-132.359533092391	-132.359533092391\\
66.5	0.12594	-142.792572028296	-142.792572028296\\
66.5	0.1296	-154.047351329016	-154.047351329016\\
66.5	0.13326	-166.123870994552	-166.123870994552\\
66.5	0.13692	-179.022131024903	-179.022131024903\\
66.5	0.14058	-192.742131420069	-192.742131420069\\
66.5	0.14424	-207.283872180049	-207.283872180049\\
66.5	0.1479	-222.647353304846	-222.647353304846\\
66.5	0.15156	-238.832574794457	-238.832574794457\\
66.5	0.15522	-255.839536648883	-255.839536648883\\
66.5	0.15888	-273.668238868125	-273.668238868125\\
66.5	0.16254	-292.318681452182	-292.318681452182\\
66.5	0.1662	-311.790864401054	-311.790864401054\\
66.5	0.16986	-332.084787714741	-332.084787714741\\
66.5	0.17352	-353.200451393243	-353.200451393243\\
66.5	0.17718	-375.137855436561	-375.137855436561\\
66.5	0.18084	-397.896999844694	-397.896999844694\\
66.5	0.1845	-421.477884617642	-421.477884617642\\
66.5	0.18816	-445.880509755405	-445.880509755405\\
66.5	0.19182	-471.104875257983	-471.104875257983\\
66.5	0.19548	-497.150981125376	-497.150981125376\\
66.5	0.19914	-524.018827357585	-524.018827357585\\
66.5	0.2028	-551.708413954608	-551.708413954608\\
66.5	0.20646	-580.219740916447	-580.219740916447\\
66.5	0.21012	-609.552808243101	-609.552808243101\\
66.5	0.21378	-639.70761593457	-639.70761593457\\
66.5	0.21744	-670.684163990855	-670.684163990855\\
66.5	0.2211	-702.482452411954	-702.482452411954\\
66.5	0.22476	-735.102481197868	-735.102481197868\\
66.5	0.22842	-768.544250348598	-768.544250348598\\
66.5	0.23208	-802.807759864143	-802.807759864143\\
66.5	0.23574	-837.893009744504	-837.893009744504\\
66.5	0.2394	-873.799999989679	-873.799999989679\\
66.5	0.24306	-910.528730599669	-910.528730599669\\
66.5	0.24672	-948.079201574475	-948.079201574475\\
66.5	0.25038	-986.451412914096	-986.451412914096\\
66.5	0.25404	-1025.64536461853	-1025.64536461853\\
66.5	0.2577	-1065.66105668778	-1065.66105668778\\
66.5	0.26136	-1106.49848912185	-1106.49848912185\\
66.5	0.26502	-1148.15766192073	-1148.15766192073\\
66.5	0.26868	-1190.63857508443	-1190.63857508443\\
66.5	0.27234	-1233.94122861294	-1233.94122861294\\
66.5	0.276	-1278.06562250627	-1278.06562250627\\
66.875	0.093	-79.7026164712817	-79.7026164712817\\
66.875	0.09666	-83.5736951464786	-83.5736951464786\\
66.875	0.10032	-88.2665141864907	-88.2665141864907\\
66.875	0.10398	-93.7810735913179	-93.7810735913179\\
66.875	0.10764	-100.117373360961	-100.117373360961\\
66.875	0.1113	-107.275413495418	-107.275413495418\\
66.875	0.11496	-115.255193994691	-115.255193994691\\
66.875	0.11862	-124.056714858779	-124.056714858779\\
66.875	0.12228	-133.679976087682	-133.679976087682\\
66.875	0.12594	-144.1249776814	-144.1249776814\\
66.875	0.1296	-155.391719639934	-155.391719639934\\
66.875	0.13326	-167.480201963282	-167.480201963282\\
66.875	0.13692	-180.390424651446	-180.390424651446\\
66.875	0.14058	-194.122387704425	-194.122387704425\\
66.875	0.14424	-208.676091122219	-208.676091122219\\
66.875	0.1479	-224.051534904828	-224.051534904828\\
66.875	0.15156	-240.248719052253	-240.248719052253\\
66.875	0.15522	-257.267643564492	-257.267643564492\\
66.875	0.15888	-275.108308441547	-275.108308441547\\
66.875	0.16254	-293.770713683417	-293.770713683417\\
66.875	0.1662	-313.254859290102	-313.254859290102\\
66.875	0.16986	-333.560745261603	-333.560745261603\\
66.875	0.17352	-354.688371597918	-354.688371597918\\
66.875	0.17718	-376.637738299049	-376.637738299049\\
66.875	0.18084	-399.408845364994	-399.408845364994\\
66.875	0.1845	-423.001692795755	-423.001692795755\\
66.875	0.18816	-447.416280591331	-447.416280591331\\
66.875	0.19182	-472.652608751722	-472.652608751722\\
66.875	0.19548	-498.710677276929	-498.710677276929\\
66.875	0.19914	-525.59048616695	-525.59048616695\\
66.875	0.2028	-553.292035421787	-553.292035421787\\
66.875	0.20646	-581.815325041439	-581.815325041439\\
66.875	0.21012	-611.160355025906	-611.160355025906\\
66.875	0.21378	-641.327125375188	-641.327125375188\\
66.875	0.21744	-672.315636089286	-672.315636089286\\
66.875	0.2211	-704.125887168198	-704.125887168198\\
66.875	0.22476	-736.757878611926	-736.757878611926\\
66.875	0.22842	-770.211610420469	-770.211610420469\\
66.875	0.23208	-804.487082593827	-804.487082593827\\
66.875	0.23574	-839.584295132	-839.584295132\\
66.875	0.2394	-875.503248034988	-875.503248034988\\
66.875	0.24306	-912.243941302792	-912.243941302792\\
66.875	0.24672	-949.806374935411	-949.806374935411\\
66.875	0.25038	-988.190548932845	-988.190548932845\\
66.875	0.25404	-1027.39646329509	-1027.39646329509\\
66.875	0.2577	-1067.42411802216	-1067.42411802216\\
66.875	0.26136	-1108.27351311404	-1108.27351311404\\
66.875	0.26502	-1149.94464857073	-1149.94464857073\\
66.875	0.26868	-1192.43752439224	-1192.43752439224\\
66.875	0.27234	-1235.75214057857	-1235.75214057857\\
66.875	0.276	-1279.88849712971	-1279.88849712971\\
67.25	0.093	-80.9983917609685	-80.9983917609685\\
67.25	0.09666	-84.8814330939787	-84.8814330939787\\
67.25	0.10032	-89.5862147918038	-89.5862147918038\\
67.25	0.10398	-95.112736854444	-95.112736854444\\
67.25	0.10764	-101.4609992819	-101.4609992819\\
67.25	0.1113	-108.63100207417	-108.63100207417\\
67.25	0.11496	-116.622745231256	-116.622745231256\\
67.25	0.11862	-125.436228753157	-125.436228753157\\
67.25	0.12228	-135.071452639873	-135.071452639873\\
67.25	0.12594	-145.528416891405	-145.528416891405\\
67.25	0.1296	-156.807121507751	-156.807121507751\\
67.25	0.13326	-168.907566488913	-168.907566488913\\
67.25	0.13692	-181.82975183489	-181.82975183489\\
67.25	0.14058	-195.573677545682	-195.573677545682\\
67.25	0.14424	-210.139343621289	-210.139343621289\\
67.25	0.1479	-225.526750061711	-225.526750061711\\
67.25	0.15156	-241.735896866949	-241.735896866949\\
67.25	0.15522	-258.766784037002	-258.766784037002\\
67.25	0.15888	-276.619411571869	-276.619411571869\\
67.25	0.16254	-295.293779471553	-295.293779471553\\
67.25	0.1662	-314.789887736051	-314.789887736051\\
67.25	0.16986	-335.107736365364	-335.107736365364\\
67.25	0.17352	-356.247325359492	-356.247325359492\\
67.25	0.17718	-378.208654718436	-378.208654718436\\
67.25	0.18084	-400.991724442195	-400.991724442195\\
67.25	0.1845	-424.596534530769	-424.596534530769\\
67.25	0.18816	-449.023084984158	-449.023084984158\\
67.25	0.19182	-474.271375802363	-474.271375802363\\
67.25	0.19548	-500.341406985382	-500.341406985382\\
67.25	0.19914	-527.233178533216	-527.233178533216\\
67.25	0.2028	-554.946690445867	-554.946690445867\\
67.25	0.20646	-583.481942723331	-583.481942723331\\
67.25	0.21012	-612.838935365612	-612.838935365612\\
67.25	0.21378	-643.017668372707	-643.017668372707\\
67.25	0.21744	-674.018141744617	-674.018141744617\\
67.25	0.2211	-705.840355481343	-705.840355481343\\
67.25	0.22476	-738.484309582883	-738.484309582883\\
67.25	0.22842	-771.95000404924	-771.95000404924\\
67.25	0.23208	-806.237438880411	-806.237438880411\\
67.25	0.23574	-841.346614076397	-841.346614076397\\
67.25	0.2394	-877.277529637198	-877.277529637198\\
67.25	0.24306	-914.030185562815	-914.030185562815\\
67.25	0.24672	-951.604581853248	-951.604581853248\\
67.25	0.25038	-990.000718508494	-990.000718508494\\
67.25	0.25404	-1029.21859552856	-1029.21859552856\\
67.25	0.2577	-1069.25821291343	-1069.25821291343\\
67.25	0.26136	-1110.11957066313	-1110.11957066313\\
67.25	0.26502	-1151.80266877763	-1151.80266877763\\
67.25	0.26868	-1194.30750725696	-1194.30750725696\\
67.25	0.27234	-1237.63408610109	-1237.63408610109\\
67.25	0.276	-1281.78240531005	-1281.78240531005\\
67.625	0.093	-82.3652006075559	-82.3652006075559\\
67.625	0.09666	-86.2602045983791	-86.2602045983791\\
67.625	0.10032	-90.9769489540172	-90.9769489540172\\
67.625	0.10398	-96.5154336744709	-96.5154336744709\\
67.625	0.10764	-102.87565875974	-102.87565875974\\
67.625	0.1113	-110.057624209823	-110.057624209823\\
67.625	0.11496	-118.061330024722	-118.061330024722\\
67.625	0.11862	-126.886776204436	-126.886776204436\\
67.625	0.12228	-136.533962748965	-136.533962748965\\
67.625	0.12594	-147.00288965831	-147.00288965831\\
67.625	0.1296	-158.29355693247	-158.29355693247\\
67.625	0.13326	-170.405964571444	-170.405964571444\\
67.625	0.13692	-183.340112575234	-183.340112575234\\
67.625	0.14058	-197.096000943839	-197.096000943839\\
67.625	0.14424	-211.67362967726	-211.67362967726\\
67.625	0.1479	-227.072998775495	-227.072998775495\\
67.625	0.15156	-243.294108238546	-243.294108238546\\
67.625	0.15522	-260.336958066411	-260.336958066411\\
67.625	0.15888	-278.201548259092	-278.201548259092\\
67.625	0.16254	-296.887878816589	-296.887878816589\\
67.625	0.1662	-316.3959497389	-316.3959497389\\
67.625	0.16986	-336.725761026026	-336.725761026026\\
67.625	0.17352	-357.877312677968	-357.877312677968\\
67.625	0.17718	-379.850604694725	-379.850604694725\\
67.625	0.18084	-402.645637076296	-402.645637076296\\
67.625	0.1845	-426.262409822684	-426.262409822684\\
67.625	0.18816	-450.700922933886	-450.700922933886\\
67.625	0.19182	-475.961176409903	-475.961176409903\\
67.625	0.19548	-502.043170250736	-502.043170250736\\
67.625	0.19914	-528.946904456383	-528.946904456383\\
67.625	0.2028	-556.672379026846	-556.672379026846\\
67.625	0.20646	-585.219593962124	-585.219593962124\\
67.625	0.21012	-614.588549262218	-614.588549262218\\
67.625	0.21378	-644.779244927126	-644.779244927126\\
67.625	0.21744	-675.791680956849	-675.791680956849\\
67.625	0.2211	-707.625857351388	-707.625857351388\\
67.625	0.22476	-740.281774110742	-740.281774110742\\
67.625	0.22842	-773.759431234911	-773.759431234911\\
67.625	0.23208	-808.058828723895	-808.058828723895\\
67.625	0.23574	-843.179966577695	-843.179966577695\\
67.625	0.2394	-879.122844796309	-879.122844796309\\
67.625	0.24306	-915.887463379739	-915.887463379739\\
67.625	0.24672	-953.473822327984	-953.473822327984\\
67.625	0.25038	-991.881921641044	-991.881921641044\\
67.625	0.25404	-1031.11176131892	-1031.11176131892\\
67.625	0.2577	-1071.16334136161	-1071.16334136161\\
67.625	0.26136	-1112.03666176911	-1112.03666176911\\
67.625	0.26502	-1153.73172254144	-1153.73172254144\\
67.625	0.26868	-1196.24852367857	-1196.24852367857\\
67.625	0.27234	-1239.58706518052	-1239.58706518052\\
67.625	0.276	-1283.74734704729	-1283.74734704729\\
68	0.093	-83.803043011044	-83.803043011044\\
68	0.09666	-87.7100096596804	-87.7100096596804\\
68	0.10032	-92.4387166731315	-92.4387166731315\\
68	0.10398	-97.9891640513982	-97.9891640513982\\
68	0.10764	-104.36135179448	-104.36135179448\\
68	0.1113	-111.555279902377	-111.555279902377\\
68	0.11496	-119.570948375089	-119.570948375089\\
68	0.11862	-128.408357212616	-128.408357212616\\
68	0.12228	-138.067506414958	-138.067506414958\\
68	0.12594	-148.548395982116	-148.548395982116\\
68	0.1296	-159.851025914088	-159.851025914088\\
68	0.13326	-171.975396210876	-171.975396210876\\
68	0.13692	-184.921506872479	-184.921506872479\\
68	0.14058	-198.689357898897	-198.689357898897\\
68	0.14424	-213.278949290131	-213.278949290131\\
68	0.1479	-228.690281046179	-228.690281046179\\
68	0.15156	-244.923353167043	-244.923353167043\\
68	0.15522	-261.978165652722	-261.978165652722\\
68	0.15888	-279.854718503216	-279.854718503216\\
68	0.16254	-298.553011718525	-298.553011718525\\
68	0.1662	-318.07304529865	-318.07304529865\\
68	0.16986	-338.414819243589	-338.414819243589\\
68	0.17352	-359.578333553343	-359.578333553343\\
68	0.17718	-381.563588227913	-381.563588227913\\
68	0.18084	-404.370583267298	-404.370583267298\\
68	0.1845	-427.999318671498	-427.999318671498\\
68	0.18816	-452.449794440514	-452.449794440514\\
68	0.19182	-477.722010574344	-477.722010574344\\
68	0.19548	-503.81596707299	-503.81596707299\\
68	0.19914	-530.731663936451	-530.731663936451\\
68	0.2028	-558.469101164727	-558.469101164727\\
68	0.20646	-587.028278757818	-587.028278757818\\
68	0.21012	-616.409196715724	-616.409196715724\\
68	0.21378	-646.611855038446	-646.611855038446\\
68	0.21744	-677.636253725982	-677.636253725982\\
68	0.2211	-709.482392778334	-709.482392778334\\
68	0.22476	-742.150272195501	-742.150272195501\\
68	0.22842	-775.639891977483	-775.639891977483\\
68	0.23208	-809.95125212428	-809.95125212428\\
68	0.23574	-845.084352635893	-845.084352635893\\
68	0.2394	-881.03919351232	-881.03919351232\\
68	0.24306	-917.815774753563	-917.815774753563\\
68	0.24672	-955.414096359621	-955.414096359621\\
68	0.25038	-993.834158330495	-993.834158330495\\
68	0.25404	-1033.07596066618	-1033.07596066618\\
68	0.2577	-1073.13950336669	-1073.13950336669\\
68	0.26136	-1114.024786432	-1114.024786432\\
68	0.26502	-1155.73180986214	-1155.73180986214\\
68	0.26868	-1198.26057365709	-1198.26057365709\\
68	0.27234	-1241.61107781685	-1241.61107781685\\
68	0.276	-1285.78332234143	-1285.78332234143\\
68.375	0.093	-85.3119189714331	-85.3119189714331\\
68.375	0.09666	-89.2308482778825	-89.2308482778825\\
68.375	0.10032	-93.9715179491469	-93.9715179491469\\
68.375	0.10398	-99.5339279852265	-99.5339279852265\\
68.375	0.10764	-105.918078386121	-105.918078386121\\
68.375	0.1113	-113.123969151831	-113.123969151831\\
68.375	0.11496	-121.151600282356	-121.151600282356\\
68.375	0.11862	-130.000971777697	-130.000971777697\\
68.375	0.12228	-139.672083637852	-139.672083637852\\
68.375	0.12594	-150.164935862823	-150.164935862823\\
68.375	0.1296	-161.479528452608	-161.479528452608\\
68.375	0.13326	-173.615861407209	-173.615861407209\\
68.375	0.13692	-186.573934726625	-186.573934726625\\
68.375	0.14058	-200.353748410857	-200.353748410857\\
68.375	0.14424	-214.955302459903	-214.955302459903\\
68.375	0.1479	-230.378596873765	-230.378596873765\\
68.375	0.15156	-246.623631652441	-246.623631652441\\
68.375	0.15522	-263.690406795933	-263.690406795933\\
68.375	0.15888	-281.57892230424	-281.57892230424\\
68.375	0.16254	-300.289178177363	-300.289178177363\\
68.375	0.1662	-319.8211744153	-319.8211744153\\
68.375	0.16986	-340.174911018053	-340.174911018053\\
68.375	0.17352	-361.35038798562	-361.35038798562\\
68.375	0.17718	-383.347605318003	-383.347605318003\\
68.375	0.18084	-406.166563015201	-406.166563015201\\
68.375	0.1845	-429.807261077215	-429.807261077215\\
68.375	0.18816	-454.269699504043	-454.269699504043\\
68.375	0.19182	-479.553878295687	-479.553878295687\\
68.375	0.19548	-505.659797452145	-505.659797452145\\
68.375	0.19914	-532.587456973419	-532.587456973419\\
68.375	0.2028	-560.336856859508	-560.336856859508\\
68.375	0.20646	-588.907997110412	-588.907997110412\\
68.375	0.21012	-618.300877726132	-618.300877726132\\
68.375	0.21378	-648.515498706666	-648.515498706666\\
68.375	0.21744	-679.551860052016	-679.551860052016\\
68.375	0.2211	-711.409961762181	-711.409961762181\\
68.375	0.22476	-744.089803837161	-744.089803837161\\
68.375	0.22842	-777.591386276956	-777.591386276956\\
68.375	0.23208	-811.914709081566	-811.914709081566\\
68.375	0.23574	-847.059772250992	-847.059772250992\\
68.375	0.2394	-883.026575785233	-883.026575785233\\
68.375	0.24306	-919.815119684289	-919.815119684289\\
68.375	0.24672	-957.42540394816	-957.42540394816\\
68.375	0.25038	-995.857428576846	-995.857428576846\\
68.375	0.25404	-1035.11119357035	-1035.11119357035\\
68.375	0.2577	-1075.18669892866	-1075.18669892866\\
68.375	0.26136	-1116.0839446518	-1116.0839446518\\
68.375	0.26502	-1157.80293073974	-1157.80293073974\\
68.375	0.26868	-1200.3436571925	-1200.3436571925\\
68.375	0.27234	-1243.70612401008	-1243.70612401008\\
68.375	0.276	-1287.89033119247	-1287.89033119247\\
68.75	0.093	-86.8918284887224	-86.8918284887224\\
68.75	0.09666	-90.822720452985	-90.822720452985\\
68.75	0.10032	-95.5753527820623	-95.5753527820623\\
68.75	0.10398	-101.149725475955	-101.149725475955\\
68.75	0.10764	-107.545838534663	-107.545838534663\\
68.75	0.1113	-114.763691958186	-114.763691958186\\
68.75	0.11496	-122.803285746524	-122.803285746524\\
68.75	0.11862	-131.664619899677	-131.664619899677\\
68.75	0.12228	-141.347694417646	-141.347694417646\\
68.75	0.12594	-151.85250930043	-151.85250930043\\
68.75	0.1296	-163.179064548028	-163.179064548028\\
68.75	0.13326	-175.327360160442	-175.327360160442\\
68.75	0.13692	-188.297396137672	-188.297396137672\\
68.75	0.14058	-202.089172479716	-202.089172479716\\
68.75	0.14424	-216.702689186575	-216.702689186575\\
68.75	0.1479	-232.13794625825	-232.13794625825\\
68.75	0.15156	-248.39494369474	-248.39494369474\\
68.75	0.15522	-265.473681496045	-265.473681496045\\
68.75	0.15888	-283.374159662165	-283.374159662165\\
68.75	0.16254	-302.0963781931	-302.0963781931\\
68.75	0.1662	-321.640337088851	-321.640337088851\\
68.75	0.16986	-342.006036349417	-342.006036349417\\
68.75	0.17352	-363.193475974797	-363.193475974797\\
68.75	0.17718	-385.202655964993	-385.202655964993\\
68.75	0.18084	-408.033576320004	-408.033576320004\\
68.75	0.1845	-431.686237039831	-431.686237039831\\
68.75	0.18816	-456.160638124472	-456.160638124472\\
68.75	0.19182	-481.456779573929	-481.456779573929\\
68.75	0.19548	-507.574661388201	-507.574661388201\\
68.75	0.19914	-534.514283567288	-534.514283567288\\
68.75	0.2028	-562.27564611119	-562.27564611119\\
68.75	0.20646	-590.858749019907	-590.858749019907\\
68.75	0.21012	-620.263592293439	-620.263592293439\\
68.75	0.21378	-650.490175931787	-650.490175931787\\
68.75	0.21744	-681.53849993495	-681.53849993495\\
68.75	0.2211	-713.408564302928	-713.408564302928\\
68.75	0.22476	-746.100369035721	-746.100369035721\\
68.75	0.22842	-779.613914133329	-779.613914133329\\
68.75	0.23208	-813.949199595753	-813.949199595753\\
68.75	0.23574	-849.106225422991	-849.106225422991\\
68.75	0.2394	-885.084991615045	-885.084991615045\\
68.75	0.24306	-921.885498171914	-921.885498171914\\
68.75	0.24672	-959.507745093598	-959.507745093598\\
68.75	0.25038	-997.951732380098	-997.951732380098\\
68.75	0.25404	-1037.21746003141	-1037.21746003141\\
68.75	0.2577	-1077.30492804754	-1077.30492804754\\
68.75	0.26136	-1118.21413642849	-1118.21413642849\\
68.75	0.26502	-1159.94508517425	-1159.94508517425\\
68.75	0.26868	-1202.49777428482	-1202.49777428482\\
68.75	0.27234	-1245.87220376021	-1245.87220376021\\
68.75	0.276	-1290.06837360042	-1290.06837360042\\
69.125	0.093	-88.5427715629127	-88.5427715629127\\
69.125	0.09666	-92.4856261849881	-92.4856261849881\\
69.125	0.10032	-97.2502211718786	-97.2502211718786\\
69.125	0.10398	-102.836556523584	-102.836556523584\\
69.125	0.10764	-109.244632240105	-109.244632240105\\
69.125	0.1113	-116.474448321441	-116.474448321441\\
69.125	0.11496	-124.526004767593	-124.526004767593\\
69.125	0.11862	-133.399301578559	-133.399301578559\\
69.125	0.12228	-143.094338754341	-143.094338754341\\
69.125	0.12594	-153.611116294937	-153.611116294937\\
69.125	0.1296	-164.949634200349	-164.949634200349\\
69.125	0.13326	-177.109892470576	-177.109892470576\\
69.125	0.13692	-190.091891105619	-190.091891105619\\
69.125	0.14058	-203.895630105476	-203.895630105476\\
69.125	0.14424	-218.521109470149	-218.521109470149\\
69.125	0.1479	-233.968329199637	-233.968329199637\\
69.125	0.15156	-250.237289293939	-250.237289293939\\
69.125	0.15522	-267.327989753057	-267.327989753057\\
69.125	0.15888	-285.240430576991	-285.240430576991\\
69.125	0.16254	-303.974611765739	-303.974611765739\\
69.125	0.1662	-323.530533319303	-323.530533319303\\
69.125	0.16986	-343.908195237681	-343.908195237681\\
69.125	0.17352	-365.107597520875	-365.107597520875\\
69.125	0.17718	-387.128740168884	-387.128740168884\\
69.125	0.18084	-409.971623181708	-409.971623181708\\
69.125	0.1845	-433.636246559348	-433.636246559348\\
69.125	0.18816	-458.122610301803	-458.122610301803\\
69.125	0.19182	-483.430714409072	-483.430714409072\\
69.125	0.19548	-509.560558881157	-509.560558881157\\
69.125	0.19914	-536.512143718057	-536.512143718057\\
69.125	0.2028	-564.285468919772	-564.285468919772\\
69.125	0.20646	-592.880534486303	-592.880534486303\\
69.125	0.21012	-622.297340417648	-622.297340417648\\
69.125	0.21378	-652.535886713809	-652.535886713809\\
69.125	0.21744	-683.596173374785	-683.596173374785\\
69.125	0.2211	-715.478200400576	-715.478200400576\\
69.125	0.22476	-748.181967791182	-748.181967791182\\
69.125	0.22842	-781.707475546603	-781.707475546603\\
69.125	0.23208	-816.05472366684	-816.05472366684\\
69.125	0.23574	-851.223712151892	-851.223712151892\\
69.125	0.2394	-887.214441001759	-887.214441001759\\
69.125	0.24306	-924.02691021644	-924.02691021644\\
69.125	0.24672	-961.661119795937	-961.661119795937\\
69.125	0.25038	-1000.11706974025	-1000.11706974025\\
69.125	0.25404	-1039.39476004938	-1039.39476004938\\
69.125	0.2577	-1079.49419072332	-1079.49419072332\\
69.125	0.26136	-1120.41536176208	-1120.41536176208\\
69.125	0.26502	-1162.15827316565	-1162.15827316565\\
69.125	0.26868	-1204.72292493404	-1204.72292493404\\
69.125	0.27234	-1248.10931706724	-1248.10931706724\\
69.125	0.276	-1292.31744956526	-1292.31744956526\\
69.5	0.093	-90.2647481940033	-90.2647481940033\\
69.5	0.09666	-94.219565473892	-94.219565473892\\
69.5	0.10032	-98.9961231185953	-98.9961231185953\\
69.5	0.10398	-104.594421128114	-104.594421128114\\
69.5	0.10764	-111.014459502448	-111.014459502448\\
69.5	0.1113	-118.256238241597	-118.256238241597\\
69.5	0.11496	-126.319757345562	-126.319757345562\\
69.5	0.11862	-135.205016814341	-135.205016814341\\
69.5	0.12228	-144.912016647936	-144.912016647936\\
69.5	0.12594	-155.440756846346	-155.440756846346\\
69.5	0.1296	-166.791237409571	-166.791237409571\\
69.5	0.13326	-178.963458337611	-178.963458337611\\
69.5	0.13692	-191.957419630466	-191.957419630466\\
69.5	0.14058	-205.773121288137	-205.773121288137\\
69.5	0.14424	-220.410563310622	-220.410563310622\\
69.5	0.1479	-235.869745697923	-235.869745697923\\
69.5	0.15156	-252.150668450039	-252.150668450039\\
69.5	0.15522	-269.25333156697	-269.25333156697\\
69.5	0.15888	-287.177735048717	-287.177735048717\\
69.5	0.16254	-305.923878895278	-305.923878895278\\
69.5	0.1662	-325.491763106655	-325.491763106655\\
69.5	0.16986	-345.881387682847	-345.881387682847\\
69.5	0.17352	-367.092752623853	-367.092752623853\\
69.5	0.17718	-389.125857929676	-389.125857929676\\
69.5	0.18084	-411.980703600313	-411.980703600313\\
69.5	0.1845	-435.657289635766	-435.657289635766\\
69.5	0.18816	-460.155616036033	-460.155616036033\\
69.5	0.19182	-485.475682801116	-485.475682801116\\
69.5	0.19548	-511.617489931014	-511.617489931014\\
69.5	0.19914	-538.581037425727	-538.581037425727\\
69.5	0.2028	-566.366325285256	-566.366325285256\\
69.5	0.20646	-594.973353509599	-594.973353509599\\
69.5	0.21012	-624.402122098757	-624.402122098757\\
69.5	0.21378	-654.652631052731	-654.652631052731\\
69.5	0.21744	-685.72488037152	-685.72488037152\\
69.5	0.2211	-717.618870055124	-717.618870055124\\
69.5	0.22476	-750.334600103543	-750.334600103543\\
69.5	0.22842	-783.872070516778	-783.872070516778\\
69.5	0.23208	-818.231281294827	-818.231281294827\\
69.5	0.23574	-853.412232437693	-853.412232437693\\
69.5	0.2394	-889.414923945372	-889.414923945372\\
69.5	0.24306	-926.239355817868	-926.239355817868\\
69.5	0.24672	-963.885528055177	-963.885528055177\\
69.5	0.25038	-1002.3534406573	-1002.3534406573\\
69.5	0.25404	-1041.64309362424	-1041.64309362424\\
69.5	0.2577	-1081.754486956	-1081.754486956\\
69.5	0.26136	-1122.68762065257	-1122.68762065257\\
69.5	0.26502	-1164.44249471396	-1164.44249471396\\
69.5	0.26868	-1207.01910914016	-1207.01910914016\\
69.5	0.27234	-1250.41746393117	-1250.41746393117\\
69.5	0.276	-1294.63755908701	-1294.63755908701\\
69.875	0.093	-92.0577583819951	-92.0577583819951\\
69.875	0.09666	-96.0245383196967	-96.0245383196967\\
69.875	0.10032	-100.813058622213	-100.813058622213\\
69.875	0.10398	-106.423319289545	-106.423319289545\\
69.875	0.10764	-112.855320321693	-112.855320321693\\
69.875	0.1113	-120.109061718655	-120.109061718655\\
69.875	0.11496	-128.184543480432	-128.184543480432\\
69.875	0.11862	-137.081765607025	-137.081765607025\\
69.875	0.12228	-146.800728098432	-146.800728098432\\
69.875	0.12594	-157.341430954655	-157.341430954655\\
69.875	0.1296	-168.703874175693	-168.703874175693\\
69.875	0.13326	-180.888057761547	-180.888057761547\\
69.875	0.13692	-193.893981712215	-193.893981712215\\
69.875	0.14058	-207.721646027699	-207.721646027699\\
69.875	0.14424	-222.371050707997	-222.371050707997\\
69.875	0.1479	-237.842195753111	-237.842195753111\\
69.875	0.15156	-254.13508116304	-254.13508116304\\
69.875	0.15522	-271.249706937785	-271.249706937785\\
69.875	0.15888	-289.186073077344	-289.186073077344\\
69.875	0.16254	-307.944179581718	-307.944179581718\\
69.875	0.1662	-327.524026450908	-327.524026450908\\
69.875	0.16986	-347.925613684913	-347.925613684913\\
69.875	0.17352	-369.148941283733	-369.148941283733\\
69.875	0.17718	-391.194009247369	-391.194009247369\\
69.875	0.18084	-414.060817575819	-414.060817575819\\
69.875	0.1845	-437.749366269084	-437.749366269084\\
69.875	0.18816	-462.259655327165	-462.259655327165\\
69.875	0.19182	-487.591684750061	-487.591684750061\\
69.875	0.19548	-513.745454537772	-513.745454537772\\
69.875	0.19914	-540.720964690298	-540.720964690298\\
69.875	0.2028	-568.51821520764	-568.51821520764\\
69.875	0.20646	-597.137206089796	-597.137206089796\\
69.875	0.21012	-626.577937336768	-626.577937336768\\
69.875	0.21378	-656.840408948555	-656.840408948555\\
69.875	0.21744	-687.924620925156	-687.924620925156\\
69.875	0.2211	-719.830573266574	-719.830573266574\\
69.875	0.22476	-752.558265972806	-752.558265972806\\
69.875	0.22842	-786.107699043854	-786.107699043854\\
69.875	0.23208	-820.478872479716	-820.478872479716\\
69.875	0.23574	-855.671786280394	-855.671786280394\\
69.875	0.2394	-891.686440445887	-891.686440445887\\
69.875	0.24306	-928.522834976195	-928.522834976195\\
69.875	0.24672	-966.180969871319	-966.180969871319\\
69.875	0.25038	-1004.66084513126	-1004.66084513126\\
69.875	0.25404	-1043.96246075601	-1043.96246075601\\
69.875	0.2577	-1084.08581674558	-1084.08581674558\\
69.875	0.26136	-1125.03091309996	-1125.03091309996\\
69.875	0.26502	-1166.79774981916	-1166.79774981916\\
69.875	0.26868	-1209.38632690318	-1209.38632690318\\
69.875	0.27234	-1252.79664435201	-1252.79664435201\\
69.875	0.276	-1297.02870216565	-1297.02870216565\\
70.25	0.093	-93.921802126887	-93.921802126887\\
70.25	0.09666	-97.9005447224016	-97.9005447224016\\
70.25	0.10032	-102.701027682731	-102.701027682731\\
70.25	0.10398	-108.323251007877	-108.323251007877\\
70.25	0.10764	-114.767214697837	-114.767214697837\\
70.25	0.1113	-122.032918752612	-122.032918752612\\
70.25	0.11496	-130.120363172202	-130.120363172202\\
70.25	0.11862	-139.029547956608	-139.029547956608\\
70.25	0.12228	-148.760473105829	-148.760473105829\\
70.25	0.12594	-159.313138619865	-159.313138619865\\
70.25	0.1296	-170.687544498716	-170.687544498716\\
70.25	0.13326	-182.883690742382	-182.883690742382\\
70.25	0.13692	-195.901577350864	-195.901577350864\\
70.25	0.14058	-209.74120432416	-209.74120432416\\
70.25	0.14424	-224.402571662272	-224.402571662272\\
70.25	0.1479	-239.8856793652	-239.8856793652\\
70.25	0.15156	-256.190527432941	-256.190527432941\\
70.25	0.15522	-273.317115865499	-273.317115865499\\
70.25	0.15888	-291.265444662871	-291.265444662871\\
70.25	0.16254	-310.035513825059	-310.035513825059\\
70.25	0.1662	-329.627323352062	-329.627323352062\\
70.25	0.16986	-350.04087324388	-350.04087324388\\
70.25	0.17352	-371.276163500513	-371.276163500513\\
70.25	0.17718	-393.333194121961	-393.333194121961\\
70.25	0.18084	-416.211965108224	-416.211965108224\\
70.25	0.1845	-439.912476459303	-439.912476459303\\
70.25	0.18816	-464.434728175197	-464.434728175197\\
70.25	0.19182	-489.778720255906	-489.778720255906\\
70.25	0.19548	-515.94445270143	-515.94445270143\\
70.25	0.19914	-542.931925511769	-542.931925511769\\
70.25	0.2028	-570.741138686924	-570.741138686924\\
70.25	0.20646	-599.372092226894	-599.372092226894\\
70.25	0.21012	-628.824786131679	-628.824786131679\\
70.25	0.21378	-659.099220401278	-659.099220401278\\
70.25	0.21744	-690.195395035693	-690.195395035693\\
70.25	0.2211	-722.113310034923	-722.113310034923\\
70.25	0.22476	-754.852965398969	-754.852965398969\\
70.25	0.22842	-788.41436112783	-788.41436112783\\
70.25	0.23208	-822.797497221505	-822.797497221505\\
70.25	0.23574	-858.002373679996	-858.002373679996\\
70.25	0.2394	-894.028990503302	-894.028990503302\\
70.25	0.24306	-930.877347691424	-930.877347691424\\
70.25	0.24672	-968.54744524436	-968.54744524436\\
70.25	0.25038	-1007.03928316211	-1007.03928316211\\
70.25	0.25404	-1046.35286144468	-1046.35286144468\\
70.25	0.2577	-1086.48818009206	-1086.48818009206\\
70.25	0.26136	-1127.44523910426	-1127.44523910426\\
70.25	0.26502	-1169.22403848127	-1169.22403848127\\
70.25	0.26868	-1211.8245782231	-1211.8245782231\\
70.25	0.27234	-1255.24685832974	-1255.24685832974\\
70.25	0.276	-1299.4908788012	-1299.4908788012\\
70.625	0.093	-95.8568794286795	-95.8568794286795\\
70.625	0.09666	-99.8475846820073	-99.8475846820073\\
70.625	0.10032	-104.66003030015	-104.66003030015\\
70.625	0.10398	-110.294216283108	-110.294216283108\\
70.625	0.10764	-116.750142630882	-116.750142630882\\
70.625	0.1113	-124.02780934347	-124.02780934347\\
70.625	0.11496	-132.127216420873	-132.127216420873\\
70.625	0.11862	-141.048363863092	-141.048363863092\\
70.625	0.12228	-150.791251670126	-150.791251670126\\
70.625	0.12594	-161.355879841975	-161.355879841975\\
70.625	0.1296	-172.742248378639	-172.742248378639\\
70.625	0.13326	-184.950357280119	-184.950357280119\\
70.625	0.13692	-197.980206546413	-197.980206546413\\
70.625	0.14058	-211.831796177523	-211.831796177523\\
70.625	0.14424	-226.505126173448	-226.505126173448\\
70.625	0.1479	-242.000196534188	-242.000196534188\\
70.625	0.15156	-258.317007259743	-258.317007259743\\
70.625	0.15522	-275.455558350114	-275.455558350114\\
70.625	0.15888	-293.415849805299	-293.415849805299\\
70.625	0.16254	-312.1978816253	-312.1978816253\\
70.625	0.1662	-331.801653810116	-331.801653810116\\
70.625	0.16986	-352.227166359747	-352.227166359747\\
70.625	0.17352	-373.474419274193	-373.474419274193\\
70.625	0.17718	-395.543412553455	-395.543412553455\\
70.625	0.18084	-418.434146197531	-418.434146197531\\
70.625	0.1845	-442.146620206423	-442.146620206423\\
70.625	0.18816	-466.68083458013	-466.68083458013\\
70.625	0.19182	-492.036789318652	-492.036789318652\\
70.625	0.19548	-518.214484421989	-518.214484421989\\
70.625	0.19914	-545.213919890141	-545.213919890141\\
70.625	0.2028	-573.035095723109	-573.035095723109\\
70.625	0.20646	-601.678011920892	-601.678011920892\\
70.625	0.21012	-631.14266848349	-631.14266848349\\
70.625	0.21378	-661.429065410902	-661.429065410902\\
70.625	0.21744	-692.53720270313	-692.53720270313\\
70.625	0.2211	-724.467080360174	-724.467080360174\\
70.625	0.22476	-757.218698382032	-757.218698382032\\
70.625	0.22842	-790.792056768706	-790.792056768706\\
70.625	0.23208	-825.187155520195	-825.187155520195\\
70.625	0.23574	-860.403994636499	-860.403994636499\\
70.625	0.2394	-896.442574117618	-896.442574117618\\
70.625	0.24306	-933.302893963552	-933.302893963552\\
70.625	0.24672	-970.984954174302	-970.984954174302\\
70.625	0.25038	-1009.48875474987	-1009.48875474987\\
70.625	0.25404	-1048.81429569025	-1048.81429569025\\
70.625	0.2577	-1088.96157699544	-1088.96157699544\\
70.625	0.26136	-1129.93059866545	-1129.93059866545\\
70.625	0.26502	-1171.72136070028	-1171.72136070028\\
70.625	0.26868	-1214.33386309992	-1214.33386309992\\
70.625	0.27234	-1257.76810586437	-1257.76810586437\\
70.625	0.276	-1302.02408899364	-1302.02408899364\\
71	0.093	-97.8629902873728	-97.8629902873728\\
71	0.09666	-101.865658198514	-101.865658198514\\
71	0.10032	-106.69006647447	-106.69006647447\\
71	0.10398	-112.336215115241	-112.336215115241\\
71	0.10764	-118.804104120827	-118.804104120827\\
71	0.1113	-126.093733491228	-126.093733491228\\
71	0.11496	-134.205103226445	-134.205103226445\\
71	0.11862	-143.138213326477	-143.138213326477\\
71	0.12228	-152.893063791324	-152.893063791324\\
71	0.12594	-163.469654620986	-163.469654620986\\
71	0.1296	-174.867985815463	-174.867985815463\\
71	0.13326	-187.088057374756	-187.088057374756\\
71	0.13692	-200.129869298864	-200.129869298864\\
71	0.14058	-213.993421587786	-213.993421587786\\
71	0.14424	-228.678714241524	-228.678714241524\\
71	0.1479	-244.185747260078	-244.185747260078\\
71	0.15156	-260.514520643446	-260.514520643446\\
71	0.15522	-277.665034391629	-277.665034391629\\
71	0.15888	-295.637288504628	-295.637288504628\\
71	0.16254	-314.431282982442	-314.431282982442\\
71	0.1662	-334.047017825071	-334.047017825071\\
71	0.16986	-354.484493032515	-354.484493032515\\
71	0.17352	-375.743708604774	-375.743708604774\\
71	0.17718	-397.824664541849	-397.824664541849\\
71	0.18084	-420.727360843738	-420.727360843738\\
71	0.1845	-444.451797510443	-444.451797510443\\
71	0.18816	-468.997974541963	-468.997974541963\\
71	0.19182	-494.365891938298	-494.365891938298\\
71	0.19548	-520.555549699448	-520.555549699448\\
71	0.19914	-547.566947825414	-547.566947825414\\
71	0.2028	-575.400086316195	-575.400086316195\\
71	0.20646	-604.05496517179	-604.05496517179\\
71	0.21012	-633.531584392201	-633.531584392201\\
71	0.21378	-663.829943977427	-663.829943977427\\
71	0.21744	-694.950043927468	-694.950043927468\\
71	0.2211	-726.891884242325	-726.891884242325\\
71	0.22476	-759.655464921996	-759.655464921996\\
71	0.22842	-793.240785966483	-793.240785966483\\
71	0.23208	-827.647847375785	-827.647847375785\\
71	0.23574	-862.876649149903	-862.876649149903\\
71	0.2394	-898.927191288834	-898.927191288834\\
71	0.24306	-935.799473792582	-935.799473792582\\
71	0.24672	-973.493496661145	-973.493496661145\\
71	0.25038	-1012.00925989452	-1012.00925989452\\
71	0.25404	-1051.34676349272	-1051.34676349272\\
71	0.2577	-1091.50600745572	-1091.50600745572\\
71	0.26136	-1132.48699178355	-1132.48699178355\\
71	0.26502	-1174.28971647619	-1174.28971647619\\
71	0.26868	-1216.91418153364	-1216.91418153364\\
71	0.27234	-1260.36038695591	-1260.36038695591\\
71	0.276	-1304.62833274299	-1304.62833274299\\
71.375	0.093	-99.9401347029672	-99.9401347029672\\
71.375	0.09666	-103.954765271921	-103.954765271921\\
71.375	0.10032	-108.79113620569	-108.79113620569\\
71.375	0.10398	-114.449247504274	-114.449247504274\\
71.375	0.10764	-120.929099167674	-120.929099167674\\
71.375	0.1113	-128.230691195888	-128.230691195888\\
71.375	0.11496	-136.354023588918	-136.354023588918\\
71.375	0.11862	-145.299096346763	-145.299096346763\\
71.375	0.12228	-155.065909469423	-155.065909469423\\
71.375	0.12594	-165.654462956899	-165.654462956899\\
71.375	0.1296	-177.064756809189	-177.064756809189\\
71.375	0.13326	-189.296791026294	-189.296791026294\\
71.375	0.13692	-202.350565608215	-202.350565608215\\
71.375	0.14058	-216.226080554951	-216.226080554951\\
71.375	0.14424	-230.923335866502	-230.923335866502\\
71.375	0.1479	-246.442331542868	-246.442331542868\\
71.375	0.15156	-262.783067584049	-262.783067584049\\
71.375	0.15522	-279.945543990046	-279.945543990046\\
71.375	0.15888	-297.929760760858	-297.929760760858\\
71.375	0.16254	-316.735717896485	-316.735717896485\\
71.375	0.1662	-336.363415396927	-336.363415396927\\
71.375	0.16986	-356.812853262184	-356.812853262184\\
71.375	0.17352	-378.084031492256	-378.084031492256\\
71.375	0.17718	-400.176950087144	-400.176950087144\\
71.375	0.18084	-423.091609046846	-423.091609046846\\
71.375	0.1845	-446.828008371364	-446.828008371364\\
71.375	0.18816	-471.386148060697	-471.386148060697\\
71.375	0.19182	-496.766028114846	-496.766028114846\\
71.375	0.19548	-522.967648533809	-522.967648533809\\
71.375	0.19914	-549.991009317588	-549.991009317588\\
71.375	0.2028	-577.836110466181	-577.836110466181\\
71.375	0.20646	-606.50295197959	-606.50295197959\\
71.375	0.21012	-635.991533857814	-635.991533857814\\
71.375	0.21378	-666.301856100853	-666.301856100853\\
71.375	0.21744	-697.433918708708	-697.433918708708\\
71.375	0.2211	-729.387721681377	-729.387721681377\\
71.375	0.22476	-762.163265018861	-762.163265018861\\
71.375	0.22842	-795.760548721161	-795.760548721161\\
71.375	0.23208	-830.179572788277	-830.179572788277\\
71.375	0.23574	-865.420337220207	-865.420337220207\\
71.375	0.2394	-901.482842016952	-901.482842016952\\
71.375	0.24306	-938.367087178513	-938.367087178513\\
71.375	0.24672	-976.073072704888	-976.073072704888\\
71.375	0.25038	-1014.60079859608	-1014.60079859608\\
71.375	0.25404	-1053.95026485208	-1053.95026485208\\
71.375	0.2577	-1094.12147147291	-1094.12147147291\\
71.375	0.26136	-1135.11441845854	-1135.11441845854\\
71.375	0.26502	-1176.92910580899	-1176.92910580899\\
71.375	0.26868	-1219.56553352426	-1219.56553352426\\
71.375	0.27234	-1263.02370160434	-1263.02370160434\\
71.375	0.276	-1307.30361004924	-1307.30361004924\\
71.75	0.093	-102.088312675462	-102.088312675462\\
71.75	0.09666	-106.114905902229	-106.114905902229\\
71.75	0.10032	-110.963239493811	-110.963239493811\\
71.75	0.10398	-116.633313450208	-116.633313450208\\
71.75	0.10764	-123.125127771421	-123.125127771421\\
71.75	0.1113	-130.438682457448	-130.438682457448\\
71.75	0.11496	-138.573977508291	-138.573977508291\\
71.75	0.11862	-147.531012923949	-147.531012923949\\
71.75	0.12228	-157.309788704422	-157.309788704422\\
71.75	0.12594	-167.910304849711	-167.910304849711\\
71.75	0.1296	-179.332561359814	-179.332561359814\\
71.75	0.13326	-191.576558234733	-191.576558234733\\
71.75	0.13692	-204.642295474467	-204.642295474467\\
71.75	0.14058	-218.529773079016	-218.529773079016\\
71.75	0.14424	-233.23899104838	-233.23899104838\\
71.75	0.1479	-248.769949382559	-248.769949382559\\
71.75	0.15156	-265.122648081553	-265.122648081553\\
71.75	0.15522	-282.297087145363	-282.297087145363\\
71.75	0.15888	-300.293266573988	-300.293266573988\\
71.75	0.16254	-319.111186367428	-319.111186367428\\
71.75	0.1662	-338.750846525683	-338.750846525683\\
71.75	0.16986	-359.212247048753	-359.212247048753\\
71.75	0.17352	-380.495387936638	-380.495387936638\\
71.75	0.17718	-402.600269189339	-402.600269189339\\
71.75	0.18084	-425.526890806855	-425.526890806855\\
71.75	0.1845	-449.275252789186	-449.275252789186\\
71.75	0.18816	-473.845355136332	-473.845355136332\\
71.75	0.19182	-499.237197848293	-499.237197848293\\
71.75	0.19548	-525.45078092507	-525.45078092507\\
71.75	0.19914	-552.486104366661	-552.486104366661\\
71.75	0.2028	-580.343168173068	-580.343168173068\\
71.75	0.20646	-609.02197234429	-609.02197234429\\
71.75	0.21012	-638.522516880327	-638.522516880327\\
71.75	0.21378	-668.84480178118	-668.84480178118\\
71.75	0.21744	-699.988827046847	-699.988827046847\\
71.75	0.2211	-731.954592677329	-731.954592677329\\
71.75	0.22476	-764.742098672627	-764.742098672627\\
71.75	0.22842	-798.35134503274	-798.35134503274\\
71.75	0.23208	-832.782331757668	-832.782331757668\\
71.75	0.23574	-868.035058847411	-868.035058847411\\
71.75	0.2394	-904.10952630197	-904.10952630197\\
71.75	0.24306	-941.005734121343	-941.005734121343\\
71.75	0.24672	-978.723682305532	-978.723682305532\\
71.75	0.25038	-1017.26337085454	-1017.26337085454\\
71.75	0.25404	-1056.62479976836	-1056.62479976836\\
71.75	0.2577	-1096.80796904699	-1096.80796904699\\
71.75	0.26136	-1137.81287869044	-1137.81287869044\\
71.75	0.26502	-1179.6395286987	-1179.6395286987\\
71.75	0.26868	-1222.28791907178	-1222.28791907178\\
71.75	0.27234	-1265.75804980968	-1265.75804980968\\
71.75	0.276	-1310.04992091239	-1310.04992091239\\
72.125	0.093	-104.307524204857	-104.307524204857\\
72.125	0.09666	-108.346080089437	-108.346080089437\\
72.125	0.10032	-113.206376338832	-113.206376338832\\
72.125	0.10398	-118.888412953043	-118.888412953043\\
72.125	0.10764	-125.392189932068	-125.392189932068\\
72.125	0.1113	-132.717707275909	-132.717707275909\\
72.125	0.11496	-140.864964984565	-140.864964984565\\
72.125	0.11862	-149.833963058036	-149.833963058036\\
72.125	0.12228	-159.624701496322	-159.624701496322\\
72.125	0.12594	-170.237180299424	-170.237180299424\\
72.125	0.1296	-181.67139946734	-181.67139946734\\
72.125	0.13326	-193.927359000072	-193.927359000072\\
72.125	0.13692	-207.005058897619	-207.005058897619\\
72.125	0.14058	-220.904499159981	-220.904499159981\\
72.125	0.14424	-235.625679787158	-235.625679787158\\
72.125	0.1479	-251.16860077915	-251.16860077915\\
72.125	0.15156	-267.533262135958	-267.533262135958\\
72.125	0.15522	-284.71966385758	-284.71966385758\\
72.125	0.15888	-302.727805944018	-302.727805944018\\
72.125	0.16254	-321.557688395271	-321.557688395271\\
72.125	0.1662	-341.20931121134	-341.20931121134\\
72.125	0.16986	-361.682674392223	-361.682674392223\\
72.125	0.17352	-382.977777937921	-382.977777937921\\
72.125	0.17718	-405.094621848435	-405.094621848435\\
72.125	0.18084	-428.033206123764	-428.033206123764\\
72.125	0.1845	-451.793530763908	-451.793530763908\\
72.125	0.18816	-476.375595768867	-476.375595768867\\
72.125	0.19182	-501.779401138642	-501.779401138642\\
72.125	0.19548	-528.004946873231	-528.004946873231\\
72.125	0.19914	-555.052232972636	-555.052232972636\\
72.125	0.2028	-582.921259436856	-582.921259436856\\
72.125	0.20646	-611.612026265891	-611.612026265891\\
72.125	0.21012	-641.124533459741	-641.124533459741\\
72.125	0.21378	-671.458781018406	-671.458781018406\\
72.125	0.21744	-702.614768941887	-702.614768941887\\
72.125	0.2211	-734.592497230182	-734.592497230182\\
72.125	0.22476	-767.391965883293	-767.391965883293\\
72.125	0.22842	-801.013174901219	-801.013174901219\\
72.125	0.23208	-835.45612428396	-835.45612428396\\
72.125	0.23574	-870.720814031517	-870.720814031517\\
72.125	0.2394	-906.807244143888	-906.807244143888\\
72.125	0.24306	-943.715414621075	-943.715414621075\\
72.125	0.24672	-981.445325463076	-981.445325463076\\
72.125	0.25038	-1019.99697666989	-1019.99697666989\\
72.125	0.25404	-1059.37036824153	-1059.37036824153\\
72.125	0.2577	-1099.56550017797	-1099.56550017797\\
72.125	0.26136	-1140.58237247924	-1140.58237247924\\
72.125	0.26502	-1182.42098514531	-1182.42098514531\\
72.125	0.26868	-1225.08133817621	-1225.08133817621\\
72.125	0.27234	-1268.56343157191	-1268.56343157191\\
72.125	0.276	-1312.86726533244	-1312.86726533244\\
72.5	0.093	-106.597769291153	-106.597769291153\\
72.5	0.09666	-110.648287833546	-110.648287833546\\
72.5	0.10032	-115.520546740754	-115.520546740754\\
72.5	0.10398	-121.214546012777	-121.214546012777\\
72.5	0.10764	-127.730285649616	-127.730285649616\\
72.5	0.1113	-135.06776565127	-135.06776565127\\
72.5	0.11496	-143.226986017739	-143.226986017739\\
72.5	0.11862	-152.207946749023	-152.207946749023\\
72.5	0.12228	-162.010647845122	-162.010647845122\\
72.5	0.12594	-172.635089306037	-172.635089306037\\
72.5	0.1296	-184.081271131766	-184.081271131766\\
72.5	0.13326	-196.349193322311	-196.349193322311\\
72.5	0.13692	-209.438855877671	-209.438855877671\\
72.5	0.14058	-223.350258797846	-223.350258797846\\
72.5	0.14424	-238.083402082837	-238.083402082837\\
72.5	0.1479	-253.638285732642	-253.638285732642\\
72.5	0.15156	-270.014909747263	-270.014909747263\\
72.5	0.15522	-287.213274126698	-287.213274126698\\
72.5	0.15888	-305.23337887095	-305.23337887095\\
72.5	0.16254	-324.075223980016	-324.075223980016\\
72.5	0.1662	-343.738809453897	-343.738809453897\\
72.5	0.16986	-364.224135292593	-364.224135292593\\
72.5	0.17352	-385.531201496105	-385.531201496105\\
72.5	0.17718	-407.660008064432	-407.660008064432\\
72.5	0.18084	-430.610554997574	-430.610554997574\\
72.5	0.1845	-454.382842295531	-454.382842295531\\
72.5	0.18816	-478.976869958303	-478.976869958303\\
72.5	0.19182	-504.392637985891	-504.392637985891\\
72.5	0.19548	-530.630146378293	-530.630146378293\\
72.5	0.19914	-557.689395135511	-557.689395135511\\
72.5	0.2028	-585.570384257544	-585.570384257544\\
72.5	0.20646	-614.273113744392	-614.273113744392\\
72.5	0.21012	-643.797583596055	-643.797583596055\\
72.5	0.21378	-674.143793812534	-674.143793812534\\
72.5	0.21744	-705.311744393827	-705.311744393827\\
72.5	0.2211	-737.301435339936	-737.301435339936\\
72.5	0.22476	-770.112866650859	-770.112866650859\\
72.5	0.22842	-803.746038326598	-803.746038326598\\
72.5	0.23208	-838.200950367153	-838.200950367153\\
72.5	0.23574	-873.477602772523	-873.477602772523\\
72.5	0.2394	-909.575995542707	-909.575995542707\\
72.5	0.24306	-946.496128677707	-946.496128677707\\
72.5	0.24672	-984.238002177521	-984.238002177521\\
72.5	0.25038	-1022.80161604215	-1022.80161604215\\
72.5	0.25404	-1062.1869702716	-1062.1869702716\\
72.5	0.2577	-1102.39406486586	-1102.39406486586\\
72.5	0.26136	-1143.42289982493	-1143.42289982493\\
72.5	0.26502	-1185.27347514882	-1185.27347514882\\
72.5	0.26868	-1227.94579083753	-1227.94579083753\\
72.5	0.27234	-1271.43984689105	-1271.43984689105\\
72.5	0.276	-1315.75564330939	-1315.75564330939\\
72.875	0.093	-108.95904793435	-108.95904793435\\
72.875	0.09666	-113.021529134556	-113.021529134556\\
72.875	0.10032	-117.905750699577	-117.905750699577\\
72.875	0.10398	-123.611712629414	-123.611712629414\\
72.875	0.10764	-130.139414924066	-130.139414924066\\
72.875	0.1113	-137.488857583532	-137.488857583532\\
72.875	0.11496	-145.660040607815	-145.660040607815\\
72.875	0.11862	-154.652963996912	-154.652963996912\\
72.875	0.12228	-164.467627750824	-164.467627750824\\
72.875	0.12594	-175.104031869552	-175.104031869552\\
72.875	0.1296	-186.562176353094	-186.562176353094\\
72.875	0.13326	-198.842061201452	-198.842061201452\\
72.875	0.13692	-211.943686414625	-211.943686414625\\
72.875	0.14058	-225.867051992613	-225.867051992613\\
72.875	0.14424	-240.612157935417	-240.612157935417\\
72.875	0.1479	-256.179004243035	-256.179004243035\\
72.875	0.15156	-272.567590915469	-272.567590915469\\
72.875	0.15522	-289.777917952718	-289.777917952718\\
72.875	0.15888	-307.809985354782	-307.809985354782\\
72.875	0.16254	-326.663793121661	-326.663793121661\\
72.875	0.1662	-346.339341253356	-346.339341253356\\
72.875	0.16986	-366.836629749865	-366.836629749865\\
72.875	0.17352	-388.15565861119	-388.15565861119\\
72.875	0.17718	-410.29642783733	-410.29642783733\\
72.875	0.18084	-433.258937428284	-433.258937428284\\
72.875	0.1845	-457.043187384055	-457.043187384055\\
72.875	0.18816	-481.64917770464	-481.64917770464\\
72.875	0.19182	-507.076908390041	-507.076908390041\\
72.875	0.19548	-533.326379440256	-533.326379440256\\
72.875	0.19914	-560.397590855287	-560.397590855287\\
72.875	0.2028	-588.290542635133	-588.290542635133\\
72.875	0.20646	-617.005234779794	-617.005234779794\\
72.875	0.21012	-646.541667289271	-646.541667289271\\
72.875	0.21378	-676.899840163562	-676.899840163562\\
72.875	0.21744	-708.079753402669	-708.079753402669\\
72.875	0.2211	-740.081407006591	-740.081407006591\\
72.875	0.22476	-772.904800975327	-772.904800975327\\
72.875	0.22842	-806.54993530888	-806.54993530888\\
72.875	0.23208	-841.016810007247	-841.016810007247\\
72.875	0.23574	-876.30542507043	-876.30542507043\\
72.875	0.2394	-912.415780498427	-912.415780498427\\
72.875	0.24306	-949.34787629124	-949.34787629124\\
72.875	0.24672	-987.101712448868	-987.101712448868\\
72.875	0.25038	-1025.67728897131	-1025.67728897131\\
72.875	0.25404	-1065.07460585857	-1065.07460585857\\
72.875	0.2577	-1105.29366311064	-1105.29366311064\\
72.875	0.26136	-1146.33446072753	-1146.33446072753\\
72.875	0.26502	-1188.19699870924	-1188.19699870924\\
72.875	0.26868	-1230.88127705575	-1230.88127705575\\
72.875	0.27234	-1274.38729576709	-1274.38729576709\\
72.875	0.276	-1318.71505484324	-1318.71505484324\\
73.25	0.093	-111.391360134447	-111.391360134447\\
73.25	0.09666	-115.465803992466	-115.465803992466\\
73.25	0.10032	-120.361988215301	-120.361988215301\\
73.25	0.10398	-126.07991280295	-126.07991280295\\
73.25	0.10764	-132.619577755415	-132.619577755415\\
73.25	0.1113	-139.980983072695	-139.980983072695\\
73.25	0.11496	-148.16412875479	-148.16412875479\\
73.25	0.11862	-157.1690148017	-157.1690148017\\
73.25	0.12228	-166.995641213426	-166.995641213426\\
73.25	0.12594	-177.644007989967	-177.644007989967\\
73.25	0.1296	-189.114115131322	-189.114115131322\\
73.25	0.13326	-201.405962637493	-201.405962637493\\
73.25	0.13692	-214.519550508479	-214.519550508479\\
73.25	0.14058	-228.454878744281	-228.454878744281\\
73.25	0.14424	-243.211947344897	-243.211947344897\\
73.25	0.1479	-258.790756310329	-258.790756310329\\
73.25	0.15156	-275.191305640575	-275.191305640575\\
73.25	0.15522	-292.413595335637	-292.413595335637\\
73.25	0.15888	-310.457625395514	-310.457625395514\\
73.25	0.16254	-329.323395820207	-329.323395820207\\
73.25	0.1662	-349.010906609714	-349.010906609714\\
73.25	0.16986	-369.520157764037	-369.520157764037\\
73.25	0.17352	-390.851149283175	-390.851149283175\\
73.25	0.17718	-413.003881167128	-413.003881167128\\
73.25	0.18084	-435.978353415895	-435.978353415895\\
73.25	0.1845	-459.774566029479	-459.774566029479\\
73.25	0.18816	-484.392519007878	-484.392519007878\\
73.25	0.19182	-509.832212351091	-509.832212351091\\
73.25	0.19548	-536.093646059119	-536.093646059119\\
73.25	0.19914	-563.176820131963	-563.176820131963\\
73.25	0.2028	-591.081734569623	-591.081734569623\\
73.25	0.20646	-619.808389372097	-619.808389372097\\
73.25	0.21012	-649.356784539387	-649.356784539387\\
73.25	0.21378	-679.726920071491	-679.726920071491\\
73.25	0.21744	-710.91879596841	-710.91879596841\\
73.25	0.2211	-742.932412230146	-742.932412230146\\
73.25	0.22476	-775.767768856696	-775.767768856696\\
73.25	0.22842	-809.424865848061	-809.424865848061\\
73.25	0.23208	-843.903703204241	-843.903703204241\\
73.25	0.23574	-879.204280925237	-879.204280925237\\
73.25	0.2394	-915.326599011047	-915.326599011047\\
73.25	0.24306	-952.270657461673	-952.270657461673\\
73.25	0.24672	-990.036456277115	-990.036456277115\\
73.25	0.25038	-1028.62399545737	-1028.62399545737\\
73.25	0.25404	-1068.03327500244	-1068.03327500244\\
73.25	0.2577	-1108.26429491233	-1108.26429491233\\
73.25	0.26136	-1149.31705518703	-1149.31705518703\\
73.25	0.26502	-1191.19155582655	-1191.19155582655\\
73.25	0.26868	-1233.88779683088	-1233.88779683088\\
73.25	0.27234	-1277.40577820003	-1277.40577820003\\
73.25	0.276	-1321.74549993399	-1321.74549993399\\
73.625	0.093	-113.894705891445	-113.894705891445\\
73.625	0.09666	-117.981112407277	-117.981112407277\\
73.625	0.10032	-122.889259287925	-122.889259287925\\
73.625	0.10398	-128.619146533387	-128.619146533387\\
73.625	0.10764	-135.170774143665	-135.170774143665\\
73.625	0.1113	-142.544142118758	-142.544142118758\\
73.625	0.11496	-150.739250458667	-150.739250458667\\
73.625	0.11862	-159.75609916339	-159.75609916339\\
73.625	0.12228	-169.594688232928	-169.594688232928\\
73.625	0.12594	-180.255017667282	-180.255017667282\\
73.625	0.1296	-191.737087466451	-191.737087466451\\
73.625	0.13326	-204.040897630435	-204.040897630435\\
73.625	0.13692	-217.166448159234	-217.166448159234\\
73.625	0.14058	-231.113739052849	-231.113739052849\\
73.625	0.14424	-245.882770311278	-245.882770311278\\
73.625	0.1479	-261.473541934523	-261.473541934523\\
73.625	0.15156	-277.886053922583	-277.886053922583\\
73.625	0.15522	-295.120306275458	-295.120306275458\\
73.625	0.15888	-313.176298993148	-313.176298993148\\
73.625	0.16254	-332.054032075653	-332.054032075653\\
73.625	0.1662	-351.753505522974	-351.753505522974\\
73.625	0.16986	-372.274719335109	-372.274719335109\\
73.625	0.17352	-393.61767351206	-393.61767351206\\
73.625	0.17718	-415.782368053826	-415.782368053826\\
73.625	0.18084	-438.768802960407	-438.768802960407\\
73.625	0.1845	-462.576978231804	-462.576978231804\\
73.625	0.18816	-487.206893868015	-487.206893868015\\
73.625	0.19182	-512.658549869042	-512.658549869042\\
73.625	0.19548	-538.931946234884	-538.931946234884\\
73.625	0.19914	-566.02708296554	-566.02708296554\\
73.625	0.2028	-593.943960061013	-593.943960061013\\
73.625	0.20646	-622.6825775213	-622.6825775213\\
73.625	0.21012	-652.242935346403	-652.242935346403\\
73.625	0.21378	-682.62503353632	-682.62503353632\\
73.625	0.21744	-713.828872091053	-713.828872091053\\
73.625	0.2211	-745.854451010601	-745.854451010601\\
73.625	0.22476	-778.701770294964	-778.701770294964\\
73.625	0.22842	-812.370829944143	-812.370829944143\\
73.625	0.23208	-846.861629958136	-846.861629958136\\
73.625	0.23574	-882.174170336945	-882.174170336945\\
73.625	0.2394	-918.308451080568	-918.308451080568\\
73.625	0.24306	-955.264472189007	-955.264472189007\\
73.625	0.24672	-993.042233662262	-993.042233662262\\
73.625	0.25038	-1031.64173550033	-1031.64173550033\\
73.625	0.25404	-1071.06297770322	-1071.06297770322\\
73.625	0.2577	-1111.30596027091	-1111.30596027091\\
73.625	0.26136	-1152.37068320343	-1152.37068320343\\
73.625	0.26502	-1194.25714650076	-1194.25714650076\\
73.625	0.26868	-1236.9653501629	-1236.9653501629\\
73.625	0.27234	-1280.49529418987	-1280.49529418987\\
73.625	0.276	-1324.84697858164	-1324.84697858164\\
74	0.093	-116.469085205343	-116.469085205343\\
74	0.09666	-120.567454378989	-120.567454378989\\
74	0.10032	-125.487563917449	-125.487563917449\\
74	0.10398	-131.229413820725	-131.229413820725\\
74	0.10764	-137.793004088816	-137.793004088816\\
74	0.1113	-145.178334721722	-145.178334721722\\
74	0.11496	-153.385405719443	-153.385405719443\\
74	0.11862	-162.41421708198	-162.41421708198\\
74	0.12228	-172.264768809331	-172.264768809331\\
74	0.12594	-182.937060901498	-182.937060901498\\
74	0.1296	-194.43109335848	-194.43109335848\\
74	0.13326	-206.746866180277	-206.746866180277\\
74	0.13692	-219.884379366889	-219.884379366889\\
74	0.14058	-233.843632918317	-233.843632918317\\
74	0.14424	-248.624626834559	-248.624626834559\\
74	0.1479	-264.227361115617	-264.227361115617\\
74	0.15156	-280.65183576149	-280.65183576149\\
74	0.15522	-297.898050772178	-297.898050772178\\
74	0.15888	-315.966006147682	-315.966006147682\\
74	0.16254	-334.855701888	-334.855701888\\
74	0.1662	-354.567137993134	-354.567137993134\\
74	0.16986	-375.100314463082	-375.100314463082\\
74	0.17352	-396.455231297846	-396.455231297846\\
74	0.17718	-418.631888497426	-418.631888497426\\
74	0.18084	-441.630286061819	-441.630286061819\\
74	0.1845	-465.450423991029	-465.450423991029\\
74	0.18816	-490.092302285054	-490.092302285054\\
74	0.19182	-515.555920943893	-515.555920943893\\
74	0.19548	-541.841279967548	-541.841279967548\\
74	0.19914	-568.948379356018	-568.948379356018\\
74	0.2028	-596.877219109304	-596.877219109304\\
74	0.20646	-625.627799227404	-625.627799227404\\
74	0.21012	-655.20011971032	-655.20011971032\\
74	0.21378	-685.594180558051	-685.594180558051\\
74	0.21744	-716.809981770596	-716.809981770596\\
74	0.2211	-748.847523347957	-748.847523347957\\
74	0.22476	-781.706805290133	-781.706805290133\\
74	0.22842	-815.387827597125	-815.387827597125\\
74	0.23208	-849.890590268931	-849.890590268931\\
74	0.23574	-885.215093305553	-885.215093305553\\
74	0.2394	-921.36133670699	-921.36133670699\\
74	0.24306	-958.329320473242	-958.329320473242\\
74	0.24672	-996.119044604309	-996.119044604309\\
74	0.25038	-1034.73050910019	-1034.73050910019\\
74	0.25404	-1074.16371396089	-1074.16371396089\\
74	0.2577	-1114.4186591864	-1114.4186591864\\
74	0.26136	-1155.49534477673	-1155.49534477673\\
74	0.26502	-1197.39377073187	-1197.39377073187\\
74	0.26868	-1240.11393705183	-1240.11393705183\\
74	0.27234	-1283.6558437366	-1283.6558437366\\
74	0.276	-1328.01949078619	-1328.01949078619\\
};
\end{axis}

\begin{axis}[%
width=6.159cm,
height=3.097cm,
at={(0cm,0cm)},
scale only axis,
xmin=56,
xmax=74,
tick align=outside,
xlabel style={font=\color{white!15!black}},
xlabel={$L_{cut}$},
ymin=0.093,
ymax=0.276,
ylabel style={font=\color{white!15!black}},
ylabel={$D_{rlx}$},
zmin=-2000,
zmax=0,
zlabel style={font=\color{white!15!black}},
zlabel={$c$},
view={-140}{50},
axis background/.style={fill=white},
xmajorgrids,
ymajorgrids,
zmajorgrids
]
\addplot3[only marks, mark=*, mark options={}, mark size=1.5000pt, color=mycolor1, fill=mycolor1] table[row sep=crcr]{%
x	y	z\\
74	0.123	-233.157966898601\\
72	0.113	-200.830593420783\\
61	0.095	-93.7391747783605\\
56	0.093	-119.696332564413\\
};
\addplot3[only marks, mark=*, mark options={}, mark size=1.5000pt, color=mycolor2, fill=mycolor2] table[row sep=crcr]{%
x	y	z\\
67	0.276	-1805.89673785913\\
66	0.255	-1488.55090215304\\
62	0.209	-803.703476355143\\
57	0.193	-648.796609601896\\
};
\addplot3[only marks, mark=*, mark options={}, mark size=1.5000pt, color=black, fill=black] table[row sep=crcr]{%
x	y	z\\
69	0.104	-138.62908786906\\
};
\addplot3[only marks, mark=*, mark options={}, mark size=1.5000pt, color=black, fill=black] table[row sep=crcr]{%
x	y	z\\
64	0.23	-1085.45178719795\\
};

\addplot3[%
surf,
fill opacity=0.7, shader=interp, colormap={mymap}{[1pt] rgb(0pt)=(1,0.905882,0); rgb(1pt)=(1,0.901964,0); rgb(2pt)=(1,0.898051,0); rgb(3pt)=(1,0.894144,0); rgb(4pt)=(1,0.890243,0); rgb(5pt)=(1,0.886349,0); rgb(6pt)=(1,0.88246,0); rgb(7pt)=(1,0.878577,0); rgb(8pt)=(1,0.8747,0); rgb(9pt)=(1,0.870829,0); rgb(10pt)=(1,0.866964,0); rgb(11pt)=(1,0.863106,0); rgb(12pt)=(1,0.859253,0); rgb(13pt)=(1,0.855406,0); rgb(14pt)=(1,0.851566,0); rgb(15pt)=(1,0.847732,0); rgb(16pt)=(1,0.843903,0); rgb(17pt)=(1,0.840081,0); rgb(18pt)=(1,0.836265,0); rgb(19pt)=(1,0.832455,0); rgb(20pt)=(1,0.828652,0); rgb(21pt)=(1,0.824854,0); rgb(22pt)=(1,0.821063,0); rgb(23pt)=(1,0.817278,0); rgb(24pt)=(1,0.8135,0); rgb(25pt)=(1,0.809727,0); rgb(26pt)=(1,0.805961,0); rgb(27pt)=(1,0.8022,0); rgb(28pt)=(1,0.798445,0); rgb(29pt)=(1,0.794696,0); rgb(30pt)=(1,0.790953,0); rgb(31pt)=(1,0.787215,0); rgb(32pt)=(1,0.783484,0); rgb(33pt)=(1,0.779758,0); rgb(34pt)=(1,0.776038,0); rgb(35pt)=(1,0.772324,0); rgb(36pt)=(1,0.768615,0); rgb(37pt)=(1,0.764913,0); rgb(38pt)=(1,0.761217,0); rgb(39pt)=(1,0.757527,0); rgb(40pt)=(1,0.753843,0); rgb(41pt)=(1,0.750165,0); rgb(42pt)=(1,0.746493,0); rgb(43pt)=(1,0.742827,0); rgb(44pt)=(1,0.739167,0); rgb(45pt)=(1,0.735514,0); rgb(46pt)=(1,0.731867,0); rgb(47pt)=(1,0.728226,0); rgb(48pt)=(1,0.724591,0); rgb(49pt)=(1,0.720963,0); rgb(50pt)=(1,0.717341,0); rgb(51pt)=(1,0.713725,0); rgb(52pt)=(0.999994,0.710077,0); rgb(53pt)=(0.999974,0.706363,0); rgb(54pt)=(0.999942,0.702592,0); rgb(55pt)=(0.999898,0.698775,0); rgb(56pt)=(0.999841,0.694921,0); rgb(57pt)=(0.999771,0.691039,0); rgb(58pt)=(0.99969,0.687139,0); rgb(59pt)=(0.999596,0.68323,0); rgb(60pt)=(0.99949,0.679323,0); rgb(61pt)=(0.999372,0.675427,0); rgb(62pt)=(0.999242,0.67155,0); rgb(63pt)=(0.9991,0.667704,0); rgb(64pt)=(0.998946,0.663897,0); rgb(65pt)=(0.998781,0.660138,0); rgb(66pt)=(0.998605,0.656439,0); rgb(67pt)=(0.998416,0.652807,0); rgb(68pt)=(0.998217,0.649253,0); rgb(69pt)=(0.998006,0.645786,0); rgb(70pt)=(0.997785,0.642416,0); rgb(71pt)=(0.997552,0.639152,0); rgb(72pt)=(0.997308,0.636004,0); rgb(73pt)=(0.997053,0.632982,0); rgb(74pt)=(0.996788,0.630095,0); rgb(75pt)=(0.996512,0.627352,0); rgb(76pt)=(0.996226,0.624763,0); rgb(77pt)=(0.995851,0.622329,0); rgb(78pt)=(0.99494,0.619997,0); rgb(79pt)=(0.99345,0.617753,0); rgb(80pt)=(0.991419,0.61559,0); rgb(81pt)=(0.988885,0.613503,0); rgb(82pt)=(0.985886,0.611486,0); rgb(83pt)=(0.98246,0.609532,0); rgb(84pt)=(0.978643,0.607636,0); rgb(85pt)=(0.974475,0.605791,0); rgb(86pt)=(0.969992,0.603992,0); rgb(87pt)=(0.965232,0.602233,0); rgb(88pt)=(0.960233,0.600507,0); rgb(89pt)=(0.955033,0.598808,0); rgb(90pt)=(0.949669,0.59713,0); rgb(91pt)=(0.94418,0.595468,0); rgb(92pt)=(0.938602,0.593815,0); rgb(93pt)=(0.932974,0.592166,0); rgb(94pt)=(0.927333,0.590513,0); rgb(95pt)=(0.921717,0.588852,0); rgb(96pt)=(0.916164,0.587176,0); rgb(97pt)=(0.910711,0.585479,0); rgb(98pt)=(0.905397,0.583755,0); rgb(99pt)=(0.900258,0.581999,0); rgb(100pt)=(0.895333,0.580203,0); rgb(101pt)=(0.890659,0.578362,0); rgb(102pt)=(0.886275,0.576471,0); rgb(103pt)=(0.882047,0.574545,0); rgb(104pt)=(0.877819,0.572608,0); rgb(105pt)=(0.873592,0.57066,0); rgb(106pt)=(0.869366,0.568701,0); rgb(107pt)=(0.865143,0.566733,0); rgb(108pt)=(0.860924,0.564756,0); rgb(109pt)=(0.856708,0.562771,0); rgb(110pt)=(0.852497,0.560778,0); rgb(111pt)=(0.848292,0.558779,0); rgb(112pt)=(0.844092,0.556774,0); rgb(113pt)=(0.8399,0.554763,0); rgb(114pt)=(0.835716,0.552749,0); rgb(115pt)=(0.831541,0.55073,0); rgb(116pt)=(0.827374,0.548709,0); rgb(117pt)=(0.823219,0.546686,0); rgb(118pt)=(0.819074,0.54466,0); rgb(119pt)=(0.81494,0.542635,0); rgb(120pt)=(0.81082,0.540609,0); rgb(121pt)=(0.806712,0.538584,0); rgb(122pt)=(0.802619,0.53656,0); rgb(123pt)=(0.798541,0.534539,0); rgb(124pt)=(0.794478,0.532521,0); rgb(125pt)=(0.790431,0.530506,0); rgb(126pt)=(0.786402,0.528496,0); rgb(127pt)=(0.782391,0.526491,0); rgb(128pt)=(0.77841,0.524489,0); rgb(129pt)=(0.774523,0.522478,0); rgb(130pt)=(0.770731,0.520455,0); rgb(131pt)=(0.767022,0.518424,0); rgb(132pt)=(0.763384,0.516385,0); rgb(133pt)=(0.759804,0.514339,0); rgb(134pt)=(0.756272,0.51229,0); rgb(135pt)=(0.752775,0.510237,0); rgb(136pt)=(0.749302,0.508182,0); rgb(137pt)=(0.74584,0.506128,0); rgb(138pt)=(0.742378,0.504075,0); rgb(139pt)=(0.738904,0.502025,0); rgb(140pt)=(0.735406,0.499979,0); rgb(141pt)=(0.731872,0.49794,0); rgb(142pt)=(0.72829,0.495909,0); rgb(143pt)=(0.724649,0.493887,0); rgb(144pt)=(0.720936,0.491875,0); rgb(145pt)=(0.71714,0.489876,0); rgb(146pt)=(0.713249,0.487891,0); rgb(147pt)=(0.709251,0.485921,0); rgb(148pt)=(0.705134,0.483968,0); rgb(149pt)=(0.700887,0.482033,0); rgb(150pt)=(0.696497,0.480118,0); rgb(151pt)=(0.691952,0.478225,0); rgb(152pt)=(0.687242,0.476355,0); rgb(153pt)=(0.682353,0.47451,0); rgb(154pt)=(0.677195,0.472696,0); rgb(155pt)=(0.6717,0.470916,0); rgb(156pt)=(0.665891,0.469169,0); rgb(157pt)=(0.659791,0.46745,0); rgb(158pt)=(0.653423,0.465756,0); rgb(159pt)=(0.64681,0.464084,0); rgb(160pt)=(0.639976,0.462432,0); rgb(161pt)=(0.632943,0.460795,0); rgb(162pt)=(0.625734,0.459171,0); rgb(163pt)=(0.618373,0.457556,0); rgb(164pt)=(0.610882,0.455948,0); rgb(165pt)=(0.603284,0.454343,0); rgb(166pt)=(0.595604,0.452737,0); rgb(167pt)=(0.587863,0.451129,0); rgb(168pt)=(0.580084,0.449514,0); rgb(169pt)=(0.572292,0.447889,0); rgb(170pt)=(0.564508,0.446252,0); rgb(171pt)=(0.556756,0.444599,0); rgb(172pt)=(0.549059,0.442927,0); rgb(173pt)=(0.54144,0.441232,0); rgb(174pt)=(0.533922,0.439512,0); rgb(175pt)=(0.526529,0.437764,0); rgb(176pt)=(0.519282,0.435983,0); rgb(177pt)=(0.512206,0.434168,0); rgb(178pt)=(0.505323,0.432315,0); rgb(179pt)=(0.498628,0.430422,3.92506e-06); rgb(180pt)=(0.491973,0.428504,3.49981e-05); rgb(181pt)=(0.485331,0.426562,9.63073e-05); rgb(182pt)=(0.478704,0.424596,0.000186979); rgb(183pt)=(0.472096,0.422609,0.000306141); rgb(184pt)=(0.465508,0.420599,0.00045292); rgb(185pt)=(0.458942,0.418567,0.000626441); rgb(186pt)=(0.452401,0.416515,0.000825833); rgb(187pt)=(0.445885,0.414441,0.00105022); rgb(188pt)=(0.439399,0.412348,0.00129873); rgb(189pt)=(0.432942,0.410234,0.00157049); rgb(190pt)=(0.426518,0.408102,0.00186463); rgb(191pt)=(0.420129,0.40595,0.00218028); rgb(192pt)=(0.413777,0.40378,0.00251655); rgb(193pt)=(0.407464,0.401592,0.00287258); rgb(194pt)=(0.401191,0.399386,0.00324749); rgb(195pt)=(0.394962,0.397164,0.00364042); rgb(196pt)=(0.388777,0.394925,0.00405048); rgb(197pt)=(0.38264,0.39267,0.00447681); rgb(198pt)=(0.376552,0.390399,0.00491852); rgb(199pt)=(0.370516,0.388113,0.00537476); rgb(200pt)=(0.364532,0.385812,0.00584464); rgb(201pt)=(0.358605,0.383497,0.00632729); rgb(202pt)=(0.352735,0.381168,0.00682184); rgb(203pt)=(0.346925,0.378826,0.00732741); rgb(204pt)=(0.341176,0.376471,0.00784314); rgb(205pt)=(0.335485,0.374093,0.00847245); rgb(206pt)=(0.329843,0.371682,0.00930909); rgb(207pt)=(0.324249,0.369242,0.0103377); rgb(208pt)=(0.318701,0.366772,0.0115428); rgb(209pt)=(0.313198,0.364275,0.0129091); rgb(210pt)=(0.307739,0.361753,0.0144211); rgb(211pt)=(0.302322,0.359206,0.0160634); rgb(212pt)=(0.296945,0.356637,0.0178207); rgb(213pt)=(0.291607,0.354048,0.0196776); rgb(214pt)=(0.286307,0.35144,0.0216186); rgb(215pt)=(0.281043,0.348814,0.0236284); rgb(216pt)=(0.275813,0.346172,0.0256916); rgb(217pt)=(0.270616,0.343517,0.0277927); rgb(218pt)=(0.265451,0.340849,0.0299163); rgb(219pt)=(0.260317,0.33817,0.0320472); rgb(220pt)=(0.25521,0.335482,0.0341698); rgb(221pt)=(0.250131,0.332786,0.0362688); rgb(222pt)=(0.245078,0.330085,0.0383287); rgb(223pt)=(0.240048,0.327379,0.0403343); rgb(224pt)=(0.235042,0.324671,0.04227); rgb(225pt)=(0.230056,0.321962,0.0441205); rgb(226pt)=(0.22509,0.319254,0.0458704); rgb(227pt)=(0.220142,0.316548,0.0475043); rgb(228pt)=(0.215212,0.313846,0.0490067); rgb(229pt)=(0.210296,0.311149,0.0503624); rgb(230pt)=(0.205395,0.308459,0.0515759); rgb(231pt)=(0.200514,0.305763,0.052757); rgb(232pt)=(0.195655,0.303061,0.0539242); rgb(233pt)=(0.190817,0.300353,0.0550763); rgb(234pt)=(0.186001,0.297639,0.0562123); rgb(235pt)=(0.181207,0.294918,0.0573313); rgb(236pt)=(0.176434,0.292191,0.0584321); rgb(237pt)=(0.171685,0.289458,0.0595136); rgb(238pt)=(0.166957,0.286719,0.060575); rgb(239pt)=(0.162252,0.283973,0.0616151); rgb(240pt)=(0.15757,0.281221,0.0626328); rgb(241pt)=(0.152911,0.278463,0.0636271); rgb(242pt)=(0.148275,0.275699,0.0645971); rgb(243pt)=(0.143663,0.272929,0.0655416); rgb(244pt)=(0.139074,0.270152,0.0664596); rgb(245pt)=(0.134508,0.26737,0.06735); rgb(246pt)=(0.129967,0.264581,0.0682118); rgb(247pt)=(0.125449,0.261787,0.0690441); rgb(248pt)=(0.120956,0.258986,0.0698456); rgb(249pt)=(0.116487,0.25618,0.0706154); rgb(250pt)=(0.112043,0.253367,0.0713525); rgb(251pt)=(0.107623,0.250549,0.0720557); rgb(252pt)=(0.103229,0.247724,0.0727241); rgb(253pt)=(0.0988592,0.244894,0.0733566); rgb(254pt)=(0.0945149,0.242058,0.0739522); rgb(255pt)=(0.0901961,0.239216,0.0745098)}, mesh/rows=49]
table[row sep=crcr, point meta=\thisrow{c}] {%
%
x	y	z	c\\
56	0.093	-115.683340162018	-115.683340162018\\
56	0.09666	-119.178304962119	-119.178304962119\\
56	0.10032	-123.892888394505	-123.892888394505\\
56	0.10398	-129.827090459174	-129.827090459174\\
56	0.10764	-136.980911156127	-136.980911156127\\
56	0.1113	-145.354350485365	-145.354350485365\\
56	0.11496	-154.947408446886	-154.947408446886\\
56	0.11862	-165.760085040692	-165.760085040692\\
56	0.12228	-177.79238026678	-177.79238026678\\
56	0.12594	-191.044294125154	-191.044294125154\\
56	0.1296	-205.515826615812	-205.515826615812\\
56	0.13326	-221.206977738753	-221.206977738753\\
56	0.13692	-238.117747493979	-238.117747493979\\
56	0.14058	-256.248135881489	-256.248135881489\\
56	0.14424	-275.598142901282	-275.598142901282\\
56	0.1479	-296.16776855336	-296.16776855336\\
56	0.15156	-317.957012837721	-317.957012837721\\
56	0.15522	-340.965875754368	-340.965875754368\\
56	0.15888	-365.194357303297	-365.194357303297\\
56	0.16254	-390.642457484511	-390.642457484511\\
56	0.1662	-417.310176298009	-417.310176298009\\
56	0.16986	-445.197513743791	-445.197513743791\\
56	0.17352	-474.304469821856	-474.304469821856\\
56	0.17718	-504.631044532207	-504.631044532207\\
56	0.18084	-536.17723787484	-536.17723787484\\
56	0.1845	-568.943049849759	-568.943049849759\\
56	0.18816	-602.928480456961	-602.928480456961\\
56	0.19182	-638.133529696446	-638.133529696446\\
56	0.19548	-674.558197568217	-674.558197568217\\
56	0.19914	-712.202484072271	-712.202484072271\\
56	0.2028	-751.066389208609	-751.066389208609\\
56	0.20646	-791.149912977232	-791.149912977232\\
56	0.21012	-832.453055378138	-832.453055378138\\
56	0.21378	-874.975816411328	-874.975816411328\\
56	0.21744	-918.718196076802	-918.718196076802\\
56	0.2211	-963.68019437456	-963.68019437456\\
56	0.22476	-1009.8618113046	-1009.8618113046\\
56	0.22842	-1057.26304686693	-1057.26304686693\\
56	0.23208	-1105.88390106154	-1105.88390106154\\
56	0.23574	-1155.72437388843	-1155.72437388843\\
56	0.2394	-1206.78446534761	-1206.78446534761\\
56	0.24306	-1259.06417543907	-1259.06417543907\\
56	0.24672	-1312.56350416282	-1312.56350416282\\
56	0.25038	-1367.28245151885	-1367.28245151885\\
56	0.25404	-1423.22101750717	-1423.22101750717\\
56	0.2577	-1480.37920212777	-1480.37920212777\\
56	0.26136	-1538.75700538065	-1538.75700538065\\
56	0.26502	-1598.35442726581	-1598.35442726581\\
56	0.26868	-1659.17146778326	-1659.17146778326\\
56	0.27234	-1721.208126933	-1721.208126933\\
56	0.276	-1784.46440471502	-1784.46440471502\\
56.375	0.093	-113.514388617176	-113.514388617176\\
56.375	0.09666	-117.038855221496	-117.038855221496\\
56.375	0.10032	-121.7829404581	-121.7829404581\\
56.375	0.10398	-127.746644326988	-127.746644326988\\
56.375	0.10764	-134.92996682816	-134.92996682816\\
56.375	0.1113	-143.332907961616	-143.332907961616\\
56.375	0.11496	-152.955467727356	-152.955467727356\\
56.375	0.11862	-163.797646125381	-163.797646125381\\
56.375	0.12228	-175.859443155688	-175.859443155688\\
56.375	0.12594	-189.140858818281	-189.140858818281\\
56.375	0.1296	-203.641893113157	-203.641893113157\\
56.375	0.13326	-219.362546040317	-219.362546040317\\
56.375	0.13692	-236.302817599762	-236.302817599762\\
56.375	0.14058	-254.46270779149	-254.46270779149\\
56.375	0.14424	-273.842216615502	-273.842216615502\\
56.375	0.1479	-294.441344071799	-294.441344071799\\
56.375	0.15156	-316.260090160379	-316.260090160379\\
56.375	0.15522	-339.298454881244	-339.298454881244\\
56.375	0.15888	-363.556438234392	-363.556438234392\\
56.375	0.16254	-389.034040219824	-389.034040219824\\
56.375	0.1662	-415.731260837541	-415.731260837541\\
56.375	0.16986	-443.648100087542	-443.648100087542\\
56.375	0.17352	-472.784557969826	-472.784557969826\\
56.375	0.17718	-503.140634484395	-503.140634484395\\
56.375	0.18084	-534.716329631247	-534.716329631247\\
56.375	0.1845	-567.511643410385	-567.511643410385\\
56.375	0.18816	-601.526575821805	-601.526575821805\\
56.375	0.19182	-636.76112686551	-636.76112686551\\
56.375	0.19548	-673.215296541498	-673.215296541498\\
56.375	0.19914	-710.889084849772	-710.889084849772\\
56.375	0.2028	-749.782491790329	-749.782491790329\\
56.375	0.20646	-789.89551736317	-789.89551736317\\
56.375	0.21012	-831.228161568294	-831.228161568294\\
56.375	0.21378	-873.780424405703	-873.780424405703\\
56.375	0.21744	-917.552305875396	-917.552305875396\\
56.375	0.2211	-962.543805977373	-962.543805977373\\
56.375	0.22476	-1008.75492471163	-1008.75492471163\\
56.375	0.22842	-1056.18566207818	-1056.18566207818\\
56.375	0.23208	-1104.83601807701	-1104.83601807701\\
56.375	0.23574	-1154.70599270812	-1154.70599270812\\
56.375	0.2394	-1205.79558597152	-1205.79558597152\\
56.375	0.24306	-1258.1047978672	-1258.1047978672\\
56.375	0.24672	-1311.63362839517	-1311.63362839517\\
56.375	0.25038	-1366.38207755542	-1366.38207755542\\
56.375	0.25404	-1422.35014534795	-1422.35014534795\\
56.375	0.2577	-1479.53783177277	-1479.53783177277\\
56.375	0.26136	-1537.94513682987	-1537.94513682987\\
56.375	0.26502	-1597.57206051925	-1597.57206051925\\
56.375	0.26868	-1658.41860284092	-1658.41860284092\\
56.375	0.27234	-1720.48476379487	-1720.48476379487\\
56.375	0.276	-1783.77054338111	-1783.77054338111\\
56.75	0.093	-111.484040437008	-111.484040437008\\
56.75	0.09666	-115.038008845547	-115.038008845547\\
56.75	0.10032	-119.811595886369	-119.811595886369\\
56.75	0.10398	-125.804801559476	-125.804801559476\\
56.75	0.10764	-133.017625864867	-133.017625864867\\
56.75	0.1113	-141.450068802541	-141.450068802541\\
56.75	0.11496	-151.102130372501	-151.102130372501\\
56.75	0.11862	-161.973810574743	-161.973810574743\\
56.75	0.12228	-174.065109409271	-174.065109409271\\
56.75	0.12594	-187.376026876081	-187.376026876081\\
56.75	0.1296	-201.906562975176	-201.906562975176\\
56.75	0.13326	-217.656717706555	-217.656717706555\\
56.75	0.13692	-234.626491070218	-234.626491070218\\
56.75	0.14058	-252.815883066165	-252.815883066165\\
56.75	0.14424	-272.224893694397	-272.224893694397\\
56.75	0.1479	-292.853522954911	-292.853522954911\\
56.75	0.15156	-314.701770847711	-314.701770847711\\
56.75	0.15522	-337.769637372793	-337.769637372793\\
56.75	0.15888	-362.057122530161	-362.057122530161\\
56.75	0.16254	-387.564226319812	-387.564226319812\\
56.75	0.1662	-414.290948741748	-414.290948741748\\
56.75	0.16986	-442.237289795966	-442.237289795966\\
56.75	0.17352	-471.40324948247	-471.40324948247\\
56.75	0.17718	-501.788827801257	-501.788827801257\\
56.75	0.18084	-533.394024752329	-533.394024752329\\
56.75	0.1845	-566.218840335684	-566.218840335684\\
56.75	0.18816	-600.263274551324	-600.263274551324\\
56.75	0.19182	-635.527327399247	-635.527327399247\\
56.75	0.19548	-672.010998879455	-672.010998879455\\
56.75	0.19914	-709.714288991946	-709.714288991946\\
56.75	0.2028	-748.637197736722	-748.637197736722\\
56.75	0.20646	-788.779725113782	-788.779725113782\\
56.75	0.21012	-830.141871123125	-830.141871123125\\
56.75	0.21378	-872.723635764753	-872.723635764753\\
56.75	0.21744	-916.525019038664	-916.525019038664\\
56.75	0.2211	-961.54602094486	-961.54602094486\\
56.75	0.22476	-1007.78664148334	-1007.78664148334\\
56.75	0.22842	-1055.2468806541	-1055.2468806541\\
56.75	0.23208	-1103.92673845715	-1103.92673845715\\
56.75	0.23574	-1153.82621489248	-1153.82621489248\\
56.75	0.2394	-1204.9453099601	-1204.9453099601\\
56.75	0.24306	-1257.28402366	-1257.28402366\\
56.75	0.24672	-1310.84235599218	-1310.84235599218\\
56.75	0.25038	-1365.62030695665	-1365.62030695665\\
56.75	0.25404	-1421.6178765534	-1421.6178765534\\
56.75	0.2577	-1478.83506478244	-1478.83506478244\\
56.75	0.26136	-1537.27187164376	-1537.27187164376\\
56.75	0.26502	-1596.92829713736	-1596.92829713736\\
56.75	0.26868	-1657.80434126325	-1657.80434126325\\
56.75	0.27234	-1719.90000402142	-1719.90000402142\\
56.75	0.276	-1783.21528541188	-1783.21528541188\\
57.125	0.093	-109.592295621513	-109.592295621513\\
57.125	0.09666	-113.17576583427	-113.17576583427\\
57.125	0.10032	-117.978854679312	-117.978854679312\\
57.125	0.10398	-124.001562156637	-124.001562156637\\
57.125	0.10764	-131.243888266247	-131.243888266247\\
57.125	0.1113	-139.70583300814	-139.70583300814\\
57.125	0.11496	-149.387396382318	-149.387396382318\\
57.125	0.11862	-160.288578388779	-160.288578388779\\
57.125	0.12228	-172.409379027525	-172.409379027525\\
57.125	0.12594	-185.749798298554	-185.749798298554\\
57.125	0.1296	-200.309836201868	-200.309836201868\\
57.125	0.13326	-216.089492737466	-216.089492737466\\
57.125	0.13692	-233.088767905348	-233.088767905348\\
57.125	0.14058	-251.307661705513	-251.307661705513\\
57.125	0.14424	-270.746174137963	-270.746174137963\\
57.125	0.1479	-291.404305202697	-291.404305202697\\
57.125	0.15156	-313.282054899715	-313.282054899715\\
57.125	0.15522	-336.379423229016	-336.379423229016\\
57.125	0.15888	-360.696410190603	-360.696410190603\\
57.125	0.16254	-386.233015784472	-386.233015784472\\
57.125	0.1662	-412.989240010627	-412.989240010627\\
57.125	0.16986	-440.965082869064	-440.965082869064\\
57.125	0.17352	-470.160544359787	-470.160544359787\\
57.125	0.17718	-500.575624482792	-500.575624482792\\
57.125	0.18084	-532.210323238083	-532.210323238083\\
57.125	0.1845	-565.064640625656	-565.064640625656\\
57.125	0.18816	-599.138576645515	-599.138576645515\\
57.125	0.19182	-634.432131297657	-634.432131297657\\
57.125	0.19548	-670.945304582084	-670.945304582084\\
57.125	0.19914	-708.678096498794	-708.678096498794\\
57.125	0.2028	-747.630507047788	-747.630507047788\\
57.125	0.20646	-787.802536229067	-787.802536229067\\
57.125	0.21012	-829.194184042629	-829.194184042629\\
57.125	0.21378	-871.805450488475	-871.805450488475\\
57.125	0.21744	-915.636335566605	-915.636335566605\\
57.125	0.2211	-960.68683927702	-960.68683927702\\
57.125	0.22476	-1006.95696161972	-1006.95696161972\\
57.125	0.22842	-1054.4467025947	-1054.4467025947\\
57.125	0.23208	-1103.15606220197	-1103.15606220197\\
57.125	0.23574	-1153.08504044152	-1153.08504044152\\
57.125	0.2394	-1204.23363731335	-1204.23363731335\\
57.125	0.24306	-1256.60185281747	-1256.60185281747\\
57.125	0.24672	-1310.18968695387	-1310.18968695387\\
57.125	0.25038	-1364.99713972256	-1364.99713972256\\
57.125	0.25404	-1421.02421112353	-1421.02421112353\\
57.125	0.2577	-1478.27090115679	-1478.27090115679\\
57.125	0.26136	-1536.73720982233	-1536.73720982233\\
57.125	0.26502	-1596.42313712015	-1596.42313712015\\
57.125	0.26868	-1657.32868305025	-1657.32868305025\\
57.125	0.27234	-1719.45384761265	-1719.45384761265\\
57.125	0.276	-1782.79863080732	-1782.79863080732\\
57.5	0.093	-107.839154170691	-107.839154170691\\
57.5	0.09666	-111.452126187667	-111.452126187667\\
57.5	0.10032	-116.284716836927	-116.284716836927\\
57.5	0.10398	-122.336926118471	-122.336926118471\\
57.5	0.10764	-129.608754032299	-129.608754032299\\
57.5	0.1113	-138.100200578411	-138.100200578411\\
57.5	0.11496	-147.811265756808	-147.811265756808\\
57.5	0.11862	-158.741949567488	-158.741949567488\\
57.5	0.12228	-170.892252010453	-170.892252010453\\
57.5	0.12594	-184.262173085701	-184.262173085701\\
57.5	0.1296	-198.851712793233	-198.851712793233\\
57.5	0.13326	-214.660871133049	-214.660871133049\\
57.5	0.13692	-231.68964810515	-231.68964810515\\
57.5	0.14058	-249.938043709534	-249.938043709534\\
57.5	0.14424	-269.406057946203	-269.406057946203\\
57.5	0.1479	-290.093690815155	-290.093690815155\\
57.5	0.15156	-312.000942316392	-312.000942316392\\
57.5	0.15522	-335.127812449912	-335.127812449912\\
57.5	0.15888	-359.474301215718	-359.474301215718\\
57.5	0.16254	-385.040408613806	-385.040408613806\\
57.5	0.1662	-411.826134644179	-411.826134644179\\
57.5	0.16986	-439.831479306835	-439.831479306835\\
57.5	0.17352	-469.056442601777	-469.056442601777\\
57.5	0.17718	-499.501024529	-499.501024529\\
57.5	0.18084	-531.16522508851	-531.16522508851\\
57.5	0.1845	-564.049044280302	-564.049044280302\\
57.5	0.18816	-598.15248210438	-598.15248210438\\
57.5	0.19182	-633.47553856074	-633.47553856074\\
57.5	0.19548	-670.018213649385	-670.018213649385\\
57.5	0.19914	-707.780507370314	-707.780507370314\\
57.5	0.2028	-746.762419723528	-746.762419723528\\
57.5	0.20646	-786.963950709025	-786.963950709025\\
57.5	0.21012	-828.385100326806	-828.385100326806\\
57.5	0.21378	-871.02586857687	-871.02586857687\\
57.5	0.21744	-914.88625545922	-914.88625545922\\
57.5	0.2211	-959.966260973853	-959.966260973853\\
57.5	0.22476	-1006.26588512077	-1006.26588512077\\
57.5	0.22842	-1053.78512789997	-1053.78512789997\\
57.5	0.23208	-1102.52398931146	-1102.52398931146\\
57.5	0.23574	-1152.48246935523	-1152.48246935523\\
57.5	0.2394	-1203.66056803128	-1203.66056803128\\
57.5	0.24306	-1256.05828533962	-1256.05828533962\\
57.5	0.24672	-1309.67562128024	-1309.67562128024\\
57.5	0.25038	-1364.51257585314	-1364.51257585314\\
57.5	0.25404	-1420.56914905833	-1420.56914905833\\
57.5	0.2577	-1477.84534089581	-1477.84534089581\\
57.5	0.26136	-1536.34115136556	-1536.34115136556\\
57.5	0.26502	-1596.05658046761	-1596.05658046761\\
57.5	0.26868	-1656.99162820193	-1656.99162820193\\
57.5	0.27234	-1719.14629456854	-1719.14629456854\\
57.5	0.276	-1782.52057956743	-1782.52057956743\\
57.875	0.093	-106.224616084543	-106.224616084543\\
57.875	0.09666	-109.867089905738	-109.867089905738\\
57.875	0.10032	-114.729182359216	-114.729182359216\\
57.875	0.10398	-120.810893444979	-120.810893444979\\
57.875	0.10764	-128.112223163026	-128.112223163026\\
57.875	0.1113	-136.633171513357	-136.633171513357\\
57.875	0.11496	-146.373738495972	-146.373738495972\\
57.875	0.11862	-157.333924110871	-157.333924110871\\
57.875	0.12228	-169.513728358054	-169.513728358054\\
57.875	0.12594	-182.913151237521	-182.913151237521\\
57.875	0.1296	-197.532192749272	-197.532192749272\\
57.875	0.13326	-213.370852893307	-213.370852893307\\
57.875	0.13692	-230.429131669626	-230.429131669626\\
57.875	0.14058	-248.70702907823	-248.70702907823\\
57.875	0.14424	-268.204545119117	-268.204545119117\\
57.875	0.1479	-288.921679792289	-288.921679792289\\
57.875	0.15156	-310.858433097743	-310.858433097743\\
57.875	0.15522	-334.014805035483	-334.014805035483\\
57.875	0.15888	-358.390795605506	-358.390795605506\\
57.875	0.16254	-383.986404807814	-383.986404807814\\
57.875	0.1662	-410.801632642405	-410.801632642405\\
57.875	0.16986	-438.836479109281	-438.836479109281\\
57.875	0.17352	-468.09094420844	-468.09094420844\\
57.875	0.17718	-498.565027939883	-498.565027939883\\
57.875	0.18084	-530.25873030361	-530.25873030361\\
57.875	0.1845	-563.172051299623	-563.172051299623\\
57.875	0.18816	-597.304990927918	-597.304990927918\\
57.875	0.19182	-632.657549188497	-632.657549188497\\
57.875	0.19548	-669.229726081361	-669.229726081361\\
57.875	0.19914	-707.021521606509	-707.021521606509\\
57.875	0.2028	-746.032935763941	-746.032935763941\\
57.875	0.20646	-786.263968553657	-786.263968553657\\
57.875	0.21012	-827.714619975657	-827.714619975657\\
57.875	0.21378	-870.38489002994	-870.38489002994\\
57.875	0.21744	-914.274778716508	-914.274778716508\\
57.875	0.2211	-959.38428603536	-959.38428603536\\
57.875	0.22476	-1005.7134119865	-1005.7134119865\\
57.875	0.22842	-1053.26215656992	-1053.26215656992\\
57.875	0.23208	-1102.03051978562	-1102.03051978562\\
57.875	0.23574	-1152.01850163361	-1152.01850163361\\
57.875	0.2394	-1203.22610211388	-1203.22610211388\\
57.875	0.24306	-1255.65332122644	-1255.65332122644\\
57.875	0.24672	-1309.30015897128	-1309.30015897128\\
57.875	0.25038	-1364.1666153484	-1364.1666153484\\
57.875	0.25404	-1420.25269035781	-1420.25269035781\\
57.875	0.2577	-1477.5583839995	-1477.5583839995\\
57.875	0.26136	-1536.08369627348	-1536.08369627348\\
57.875	0.26502	-1595.82862717974	-1595.82862717974\\
57.875	0.26868	-1656.79317671828	-1656.79317671828\\
57.875	0.27234	-1718.97734488911	-1718.97734488911\\
57.875	0.276	-1782.38113169222	-1782.38113169222\\
58.25	0.093	-104.748681363067	-104.748681363067\\
58.25	0.09666	-108.420656988481	-108.420656988481\\
58.25	0.10032	-113.312251246178	-113.312251246178\\
58.25	0.10398	-119.42346413616	-119.42346413616\\
58.25	0.10764	-126.754295658426	-126.754295658426\\
58.25	0.1113	-135.304745812976	-135.304745812976\\
58.25	0.11496	-145.074814599809	-145.074814599809\\
58.25	0.11862	-156.064502018927	-156.064502018927\\
58.25	0.12228	-168.273808070328	-168.273808070328\\
58.25	0.12594	-181.702732754014	-181.702732754014\\
58.25	0.1296	-196.351276069984	-196.351276069984\\
58.25	0.13326	-212.219438018238	-212.219438018238\\
58.25	0.13692	-229.307218598776	-229.307218598776\\
58.25	0.14058	-247.614617811598	-247.614617811598\\
58.25	0.14424	-267.141635656703	-267.141635656703\\
58.25	0.1479	-287.888272134094	-287.888272134094\\
58.25	0.15156	-309.854527243767	-309.854527243767\\
58.25	0.15522	-333.040400985726	-333.040400985726\\
58.25	0.15888	-357.445893359968	-357.445893359968\\
58.25	0.16254	-383.071004366494	-383.071004366494\\
58.25	0.1662	-409.915734005304	-409.915734005304\\
58.25	0.16986	-437.980082276398	-437.980082276398\\
58.25	0.17352	-467.264049179776	-467.264049179776\\
58.25	0.17718	-497.767634715439	-497.767634715439\\
58.25	0.18084	-529.490838883384	-529.490838883384\\
58.25	0.1845	-562.433661683615	-562.433661683615\\
58.25	0.18816	-596.596103116129	-596.596103116129\\
58.25	0.19182	-631.978163180927	-631.978163180927\\
58.25	0.19548	-668.57984187801	-668.57984187801\\
58.25	0.19914	-706.401139207377	-706.401139207377\\
58.25	0.2028	-745.442055169027	-745.442055169027\\
58.25	0.20646	-785.702589762962	-785.702589762962\\
58.25	0.21012	-827.18274298918	-827.18274298918\\
58.25	0.21378	-869.882514847682	-869.882514847682\\
58.25	0.21744	-913.801905338469	-913.801905338469\\
58.25	0.2211	-958.94091446154	-958.94091446154\\
58.25	0.22476	-1005.29954221689	-1005.29954221689\\
58.25	0.22842	-1052.87778860453	-1052.87778860453\\
58.25	0.23208	-1101.67565362446	-1101.67565362446\\
58.25	0.23574	-1151.69313727666	-1151.69313727666\\
58.25	0.2394	-1202.93023956115	-1202.93023956115\\
58.25	0.24306	-1255.38696047793	-1255.38696047793\\
58.25	0.24672	-1309.06330002699	-1309.06330002699\\
58.25	0.25038	-1363.95925820833	-1363.95925820833\\
58.25	0.25404	-1420.07483502196	-1420.07483502196\\
58.25	0.2577	-1477.41003046787	-1477.41003046787\\
58.25	0.26136	-1535.96484454606	-1535.96484454606\\
58.25	0.26502	-1595.73927725654	-1595.73927725654\\
58.25	0.26868	-1656.7333285993	-1656.7333285993\\
58.25	0.27234	-1718.94699857435	-1718.94699857435\\
58.25	0.276	-1782.38028718168	-1782.38028718168\\
58.625	0.093	-103.411350006266	-103.411350006266\\
58.625	0.09666	-107.112827435898	-107.112827435898\\
58.625	0.10032	-112.033923497815	-112.033923497815\\
58.625	0.10398	-118.174638192015	-118.174638192015\\
58.625	0.10764	-125.534971518499	-125.534971518499\\
58.625	0.1113	-134.114923477267	-134.114923477267\\
58.625	0.11496	-143.91449406832	-143.91449406832\\
58.625	0.11862	-154.933683291656	-154.933683291656\\
58.625	0.12228	-167.172491147277	-167.172491147277\\
58.625	0.12594	-180.630917635181	-180.630917635181\\
58.625	0.1296	-195.30896275537	-195.30896275537\\
58.625	0.13326	-211.206626507842	-211.206626507842\\
58.625	0.13692	-228.323908892599	-228.323908892599\\
58.625	0.14058	-246.660809909639	-246.660809909639\\
58.625	0.14424	-266.217329558965	-266.217329558965\\
58.625	0.1479	-286.993467840573	-286.993467840573\\
58.625	0.15156	-308.989224754466	-308.989224754466\\
58.625	0.15522	-332.204600300642	-332.204600300642\\
58.625	0.15888	-356.639594479103	-356.639594479103\\
58.625	0.16254	-382.294207289848	-382.294207289848\\
58.625	0.1662	-409.168438732877	-409.168438732877\\
58.625	0.16986	-437.26228880819	-437.26228880819\\
58.625	0.17352	-466.575757515787	-466.575757515787\\
58.625	0.17718	-497.108844855667	-497.108844855667\\
58.625	0.18084	-528.861550827833	-528.861550827833\\
58.625	0.1845	-561.833875432281	-561.833875432281\\
58.625	0.18816	-596.025818669015	-596.025818669015\\
58.625	0.19182	-631.437380538031	-631.437380538031\\
58.625	0.19548	-668.068561039333	-668.068561039333\\
58.625	0.19914	-705.919360172918	-705.919360172918\\
58.625	0.2028	-744.989777938787	-744.989777938787\\
58.625	0.20646	-785.279814336941	-785.279814336941\\
58.625	0.21012	-826.789469367377	-826.789469367377\\
58.625	0.21378	-869.518743030098	-869.518743030098\\
58.625	0.21744	-913.467635325104	-913.467635325104\\
58.625	0.2211	-958.636146252393	-958.636146252393\\
58.625	0.22476	-1005.02427581197	-1005.02427581197\\
58.625	0.22842	-1052.63202400382	-1052.63202400382\\
58.625	0.23208	-1101.45939082797	-1101.45939082797\\
58.625	0.23574	-1151.50637628439	-1151.50637628439\\
58.625	0.2394	-1202.7729803731	-1202.7729803731\\
58.625	0.24306	-1255.25920309409	-1255.25920309409\\
58.625	0.24672	-1308.96504444737	-1308.96504444737\\
58.625	0.25038	-1363.89050443293	-1363.89050443293\\
58.625	0.25404	-1420.03558305078	-1420.03558305078\\
58.625	0.2577	-1477.40028030091	-1477.40028030091\\
58.625	0.26136	-1535.98459618332	-1535.98459618332\\
58.625	0.26502	-1595.78853069802	-1595.78853069802\\
58.625	0.26868	-1656.812083845	-1656.812083845\\
58.625	0.27234	-1719.05525562427	-1719.05525562427\\
58.625	0.276	-1782.51804603582	-1782.51804603582\\
59	0.093	-102.212622014138	-102.212622014138\\
59	0.09666	-105.943601247988	-105.943601247988\\
59	0.10032	-110.894199114124	-110.894199114124\\
59	0.10398	-117.064415612542	-117.064415612542\\
59	0.10764	-124.454250743245	-124.454250743245\\
59	0.1113	-133.063704506232	-133.063704506232\\
59	0.11496	-142.892776901504	-142.892776901504\\
59	0.11862	-153.941467929059	-153.941467929059\\
59	0.12228	-166.209777588898	-166.209777588898\\
59	0.12594	-179.697705881021	-179.697705881021\\
59	0.1296	-194.405252805429	-194.405252805429\\
59	0.13326	-210.33241836212	-210.33241836212\\
59	0.13692	-227.479202551095	-227.479202551095\\
59	0.14058	-245.845605372354	-245.845605372354\\
59	0.14424	-265.431626825898	-265.431626825898\\
59	0.1479	-286.237266911725	-286.237266911725\\
59	0.15156	-308.262525629837	-308.262525629837\\
59	0.15522	-331.507402980232	-331.507402980232\\
59	0.15888	-355.971898962912	-355.971898962912\\
59	0.16254	-381.656013577875	-381.656013577875\\
59	0.1662	-408.559746825123	-408.559746825123\\
59	0.16986	-436.683098704654	-436.683098704654\\
59	0.17352	-466.02606921647	-466.02606921647\\
59	0.17718	-496.588658360569	-496.588658360569\\
59	0.18084	-528.370866136953	-528.370866136953\\
59	0.1845	-561.372692545621	-561.372692545621\\
59	0.18816	-595.594137586573	-595.594137586573\\
59	0.19182	-631.035201259808	-631.035201259808\\
59	0.19548	-667.695883565328	-667.695883565328\\
59	0.19914	-705.576184503132	-705.576184503132\\
59	0.2028	-744.67610407322	-744.67610407322\\
59	0.20646	-784.995642275592	-784.995642275592\\
59	0.21012	-826.534799110248	-826.534799110248\\
59	0.21378	-869.293574577187	-869.293574577187\\
59	0.21744	-913.271968676412	-913.271968676412\\
59	0.2211	-958.46998140792	-958.46998140792\\
59	0.22476	-1004.88761277171	-1004.88761277171\\
59	0.22842	-1052.52486276779	-1052.52486276779\\
59	0.23208	-1101.38173139615	-1101.38173139615\\
59	0.23574	-1151.45821865679	-1151.45821865679\\
59	0.2394	-1202.75432454972	-1202.75432454972\\
59	0.24306	-1255.27004907493	-1255.27004907493\\
59	0.24672	-1309.00539223243	-1309.00539223243\\
59	0.25038	-1363.96035402221	-1363.96035402221\\
59	0.25404	-1420.13493444427	-1420.13493444427\\
59	0.2577	-1477.52913349862	-1477.52913349862\\
59	0.26136	-1536.14295118525	-1536.14295118525\\
59	0.26502	-1595.97638750417	-1595.97638750417\\
59	0.26868	-1657.02944245537	-1657.02944245537\\
59	0.27234	-1719.30211603886	-1719.30211603886\\
59	0.276	-1782.79440825462	-1782.79440825462\\
59.375	0.093	-101.152497386683	-101.152497386683\\
59.375	0.09666	-104.912978424752	-104.912978424752\\
59.375	0.10032	-109.893078095106	-109.893078095106\\
59.375	0.10398	-116.092796397744	-116.092796397744\\
59.375	0.10764	-123.512133332666	-123.512133332666\\
59.375	0.1113	-132.151088899871	-132.151088899871\\
59.375	0.11496	-142.009663099361	-142.009663099361\\
59.375	0.11862	-153.087855931135	-153.087855931135\\
59.375	0.12228	-165.385667395193	-165.385667395193\\
59.375	0.12594	-178.903097491535	-178.903097491535\\
59.375	0.1296	-193.640146220161	-193.640146220161\\
59.375	0.13326	-209.596813581071	-209.596813581071\\
59.375	0.13692	-226.773099574265	-226.773099574265\\
59.375	0.14058	-245.169004199743	-245.169004199743\\
59.375	0.14424	-264.784527457505	-264.784527457505\\
59.375	0.1479	-285.619669347551	-285.619669347551\\
59.375	0.15156	-307.674429869881	-307.674429869881\\
59.375	0.15522	-330.948809024495	-330.948809024495\\
59.375	0.15888	-355.442806811394	-355.442806811394\\
59.375	0.16254	-381.156423230576	-381.156423230576\\
59.375	0.1662	-408.089658282042	-408.089658282042\\
59.375	0.16986	-436.242511965793	-436.242511965793\\
59.375	0.17352	-465.614984281827	-465.614984281827\\
59.375	0.17718	-496.207075230145	-496.207075230145\\
59.375	0.18084	-528.018784810747	-528.018784810747\\
59.375	0.1845	-561.050113023634	-561.050113023634\\
59.375	0.18816	-595.301059868805	-595.301059868805\\
59.375	0.19182	-630.771625346259	-630.771625346259\\
59.375	0.19548	-667.461809455997	-667.461809455997\\
59.375	0.19914	-705.37161219802	-705.37161219802\\
59.375	0.2028	-744.501033572327	-744.501033572327\\
59.375	0.20646	-784.850073578918	-784.850073578918\\
59.375	0.21012	-826.418732217792	-826.418732217792\\
59.375	0.21378	-869.207009488951	-869.207009488951\\
59.375	0.21744	-913.214905392393	-913.214905392393\\
59.375	0.2211	-958.44241992812	-958.44241992812\\
59.375	0.22476	-1004.88955309613	-1004.88955309613\\
59.375	0.22842	-1052.55630489643	-1052.55630489643\\
59.375	0.23208	-1101.442675329	-1101.442675329\\
59.375	0.23574	-1151.54866439387	-1151.54866439387\\
59.375	0.2394	-1202.87427209101	-1202.87427209101\\
59.375	0.24306	-1255.41949842045	-1255.41949842045\\
59.375	0.24672	-1309.18434338216	-1309.18434338216\\
59.375	0.25038	-1364.16880697616	-1364.16880697616\\
59.375	0.25404	-1420.37288920244	-1420.37288920244\\
59.375	0.2577	-1477.79659006101	-1477.79659006101\\
59.375	0.26136	-1536.43990955186	-1536.43990955186\\
59.375	0.26502	-1596.302847675	-1596.302847675\\
59.375	0.26868	-1657.38540443041	-1657.38540443041\\
59.375	0.27234	-1719.68757981812	-1719.68757981812\\
59.375	0.276	-1783.2093738381	-1783.2093738381\\
59.75	0.093	-100.230976123901	-100.230976123901\\
59.75	0.09666	-104.020958966189	-104.020958966189\\
59.75	0.10032	-109.030560440761	-109.030560440761\\
59.75	0.10398	-115.259780547618	-115.259780547618\\
59.75	0.10764	-122.708619286759	-122.708619286759\\
59.75	0.1113	-131.377076658183	-131.377076658183\\
59.75	0.11496	-141.265152661892	-141.265152661892\\
59.75	0.11862	-152.372847297885	-152.372847297885\\
59.75	0.12228	-164.700160566161	-164.700160566161\\
59.75	0.12594	-178.247092466722	-178.247092466722\\
59.75	0.1296	-193.013642999566	-193.013642999566\\
59.75	0.13326	-208.999812164695	-208.999812164695\\
59.75	0.13692	-226.205599962107	-226.205599962107\\
59.75	0.14058	-244.631006391805	-244.631006391805\\
59.75	0.14424	-264.276031453785	-264.276031453785\\
59.75	0.1479	-285.14067514805	-285.14067514805\\
59.75	0.15156	-307.224937474598	-307.224937474598\\
59.75	0.15522	-330.528818433432	-330.528818433432\\
59.75	0.15888	-355.052318024549	-355.052318024549\\
59.75	0.16254	-380.79543624795	-380.79543624795\\
59.75	0.1662	-407.758173103635	-407.758173103635\\
59.75	0.16986	-435.940528591604	-435.940528591604\\
59.75	0.17352	-465.342502711856	-465.342502711856\\
59.75	0.17718	-495.964095464394	-495.964095464394\\
59.75	0.18084	-527.805306849215	-527.805306849215\\
59.75	0.1845	-560.86613686632	-560.86613686632\\
59.75	0.18816	-595.146585515709	-595.146585515709\\
59.75	0.19182	-630.646652797382	-630.646652797382\\
59.75	0.19548	-667.36633871134	-667.36633871134\\
59.75	0.19914	-705.305643257582	-705.305643257582\\
59.75	0.2028	-744.464566436107	-744.464566436107\\
59.75	0.20646	-784.843108246916	-784.843108246916\\
59.75	0.21012	-826.441268690009	-826.441268690009\\
59.75	0.21378	-869.259047765386	-869.259047765386\\
59.75	0.21744	-913.296445473047	-913.296445473047\\
59.75	0.2211	-958.553461812993	-958.553461812993\\
59.75	0.22476	-1005.03009678522	-1005.03009678522\\
59.75	0.22842	-1052.72635038974	-1052.72635038974\\
59.75	0.23208	-1101.64222262653	-1101.64222262653\\
59.75	0.23574	-1151.77771349562	-1151.77771349562\\
59.75	0.2394	-1203.13282299698	-1203.13282299698\\
59.75	0.24306	-1255.70755113063	-1255.70755113063\\
59.75	0.24672	-1309.50189789656	-1309.50189789656\\
59.75	0.25038	-1364.51586329478	-1364.51586329478\\
59.75	0.25404	-1420.74944732528	-1420.74944732528\\
59.75	0.2577	-1478.20264998807	-1478.20264998807\\
59.75	0.26136	-1536.87547128314	-1536.87547128314\\
59.75	0.26502	-1596.76791121049	-1596.76791121049\\
59.75	0.26868	-1657.87996977013	-1657.87996977013\\
59.75	0.27234	-1720.21164696205	-1720.21164696205\\
59.75	0.276	-1783.76294278626	-1783.76294278626\\
60.125	0.093	-99.4480582257924	-99.4480582257924\\
60.125	0.09666	-103.2675428723	-103.2675428723\\
60.125	0.10032	-108.30664615109	-108.30664615109\\
60.125	0.10398	-114.565368062166	-114.565368062166\\
60.125	0.10764	-122.043708605525	-122.043708605525\\
60.125	0.1113	-130.741667781168	-130.741667781168\\
60.125	0.11496	-140.659245589095	-140.659245589095\\
60.125	0.11862	-151.796442029307	-151.796442029307\\
60.125	0.12228	-164.153257101802	-164.153257101802\\
60.125	0.12594	-177.729690806582	-177.729690806582\\
60.125	0.1296	-192.525743143645	-192.525743143645\\
60.125	0.13326	-208.541414112992	-208.541414112992\\
60.125	0.13692	-225.776703714624	-225.776703714624\\
60.125	0.14058	-244.23161194854	-244.23161194854\\
60.125	0.14424	-263.906138814738	-263.906138814738\\
60.125	0.1479	-284.800284313223	-284.800284313223\\
60.125	0.15156	-306.91404844399	-306.91404844399\\
60.125	0.15522	-330.247431207041	-330.247431207041\\
60.125	0.15888	-354.800432602377	-354.800432602377\\
60.125	0.16254	-380.573052629997	-380.573052629997\\
60.125	0.1662	-407.5652912899	-407.5652912899\\
60.125	0.16986	-435.777148582089	-435.777148582089\\
60.125	0.17352	-465.20862450656	-465.20862450656\\
60.125	0.17718	-495.859719063316	-495.859719063316\\
60.125	0.18084	-527.730432252355	-527.730432252355\\
60.125	0.1845	-560.82076407368	-560.82076407368\\
60.125	0.18816	-595.130714527287	-595.130714527287\\
60.125	0.19182	-630.660283613179	-630.660283613179\\
60.125	0.19548	-667.409471331355	-667.409471331355\\
60.125	0.19914	-705.378277681816	-705.378277681816\\
60.125	0.2028	-744.56670266456	-744.56670266456\\
60.125	0.20646	-784.974746279588	-784.974746279588\\
60.125	0.21012	-826.6024085269	-826.6024085269\\
60.125	0.21378	-869.449689406496	-869.449689406496\\
60.125	0.21744	-913.516588918375	-913.516588918375\\
60.125	0.2211	-958.80310706254	-958.80310706254\\
60.125	0.22476	-1005.30924383899	-1005.30924383899\\
60.125	0.22842	-1053.03499924772	-1053.03499924772\\
60.125	0.23208	-1101.98037328874	-1101.98037328874\\
60.125	0.23574	-1152.14536596204	-1152.14536596204\\
60.125	0.2394	-1203.52997726762	-1203.52997726762\\
60.125	0.24306	-1256.13420720549	-1256.13420720549\\
60.125	0.24672	-1309.95805577564	-1309.95805577564\\
60.125	0.25038	-1365.00152297808	-1365.00152297808\\
60.125	0.25404	-1421.2646088128	-1421.2646088128\\
60.125	0.2577	-1478.7473132798	-1478.7473132798\\
60.125	0.26136	-1537.44963637909	-1537.44963637909\\
60.125	0.26502	-1597.37157811066	-1597.37157811066\\
60.125	0.26868	-1658.51313847452	-1658.51313847452\\
60.125	0.27234	-1720.87431747066	-1720.87431747066\\
60.125	0.276	-1784.45511509909	-1784.45511509909\\
60.5	0.093	-98.8037436923577	-98.8037436923577\\
60.5	0.09666	-102.652730143084	-102.652730143084\\
60.5	0.10032	-107.721335226094	-107.721335226094\\
60.5	0.10398	-114.009558941387	-114.009558941387\\
60.5	0.10764	-121.517401288965	-121.517401288965\\
60.5	0.1113	-130.244862268827	-130.244862268827\\
60.5	0.11496	-140.191941880973	-140.191941880973\\
60.5	0.11862	-151.358640125403	-151.358640125403\\
60.5	0.12228	-163.744957002117	-163.744957002117\\
60.5	0.12594	-177.350892511115	-177.350892511115\\
60.5	0.1296	-192.176446652398	-192.176446652398\\
60.5	0.13326	-208.221619425963	-208.221619425963\\
60.5	0.13692	-225.486410831813	-225.486410831813\\
60.5	0.14058	-243.970820869948	-243.970820869948\\
60.5	0.14424	-263.674849540366	-263.674849540366\\
60.5	0.1479	-284.598496843068	-284.598496843068\\
60.5	0.15156	-306.741762778055	-306.741762778055\\
60.5	0.15522	-330.104647345325	-330.104647345325\\
60.5	0.15888	-354.68715054488	-354.68715054488\\
60.5	0.16254	-380.489272376718	-380.489272376718\\
60.5	0.1662	-407.511012840841	-407.511012840841\\
60.5	0.16986	-435.752371937247	-435.752371937247\\
60.5	0.17352	-465.213349665937	-465.213349665937\\
60.5	0.17718	-495.893946026911	-495.893946026911\\
60.5	0.18084	-527.79416102017	-527.79416102017\\
60.5	0.1845	-560.913994645713	-560.913994645713\\
60.5	0.18816	-595.253446903539	-595.253446903539\\
60.5	0.19182	-630.81251779365	-630.81251779365\\
60.5	0.19548	-667.591207316045	-667.591207316045\\
60.5	0.19914	-705.589515470723	-705.589515470723\\
60.5	0.2028	-744.807442257686	-744.807442257686\\
60.5	0.20646	-785.244987676933	-785.244987676933\\
60.5	0.21012	-826.902151728464	-826.902151728464\\
60.5	0.21378	-869.778934412278	-869.778934412278\\
60.5	0.21744	-913.875335728377	-913.875335728377\\
60.5	0.2211	-959.19135567676	-959.19135567676\\
60.5	0.22476	-1005.72699425743	-1005.72699425743\\
60.5	0.22842	-1053.48225147038	-1053.48225147038\\
60.5	0.23208	-1102.45712731561	-1102.45712731561\\
60.5	0.23574	-1152.65162179313	-1152.65162179313\\
60.5	0.2394	-1204.06573490294	-1204.06573490294\\
60.5	0.24306	-1256.69946664502	-1256.69946664502\\
60.5	0.24672	-1310.55281701939	-1310.55281701939\\
60.5	0.25038	-1365.62578602605	-1365.62578602605\\
60.5	0.25404	-1421.91837366499	-1421.91837366499\\
60.5	0.2577	-1479.43057993621	-1479.43057993621\\
60.5	0.26136	-1538.16240483972	-1538.16240483972\\
60.5	0.26502	-1598.11384837551	-1598.11384837551\\
60.5	0.26868	-1659.28491054358	-1659.28491054358\\
60.5	0.27234	-1721.67559134394	-1721.67559134394\\
60.5	0.276	-1785.28589077659	-1785.28589077659\\
60.875	0.093	-98.2980325235959	-98.2980325235959\\
60.875	0.09666	-102.17652077854	-102.17652077854\\
60.875	0.10032	-107.274627665769	-107.274627665769\\
60.875	0.10398	-113.592353185281	-113.592353185281\\
60.875	0.10764	-121.129697337078	-121.129697337078\\
60.875	0.1113	-129.886660121159	-129.886660121159\\
60.875	0.11496	-139.863241537524	-139.863241537524\\
60.875	0.11862	-151.059441586172	-151.059441586172\\
60.875	0.12228	-163.475260267105	-163.475260267105\\
60.875	0.12594	-177.110697580322	-177.110697580322\\
60.875	0.1296	-191.965753525823	-191.965753525823\\
60.875	0.13326	-208.040428103608	-208.040428103608\\
60.875	0.13692	-225.334721313676	-225.334721313676\\
60.875	0.14058	-243.848633156029	-243.848633156029\\
60.875	0.14424	-263.582163630667	-263.582163630667\\
60.875	0.1479	-284.535312737587	-284.535312737587\\
60.875	0.15156	-306.708080476792	-306.708080476792\\
60.875	0.15522	-330.100466848281	-330.100466848281\\
60.875	0.15888	-354.712471852054	-354.712471852054\\
60.875	0.16254	-380.544095488111	-380.544095488111\\
60.875	0.1662	-407.595337756453	-407.595337756453\\
60.875	0.16986	-435.866198657078	-435.866198657078\\
60.875	0.17352	-465.356678189987	-465.356678189987\\
60.875	0.17718	-496.06677635518	-496.06677635518\\
60.875	0.18084	-527.996493152657	-527.996493152657\\
60.875	0.1845	-561.145828582419	-561.145828582419\\
60.875	0.18816	-595.514782644465	-595.514782644465\\
60.875	0.19182	-631.103355338794	-631.103355338794\\
60.875	0.19548	-667.911546665407	-667.911546665407\\
60.875	0.19914	-705.939356624305	-705.939356624305\\
60.875	0.2028	-745.186785215486	-745.186785215486\\
60.875	0.20646	-785.653832438952	-785.653832438952\\
60.875	0.21012	-827.340498294701	-827.340498294701\\
60.875	0.21378	-870.246782782734	-870.246782782734\\
60.875	0.21744	-914.372685903052	-914.372685903052\\
60.875	0.2211	-959.718207655653	-959.718207655653\\
60.875	0.22476	-1006.28334804054	-1006.28334804054\\
60.875	0.22842	-1054.06810705771	-1054.06810705771\\
60.875	0.23208	-1103.07248470716	-1103.07248470716\\
60.875	0.23574	-1153.2964809889	-1153.2964809889\\
60.875	0.2394	-1204.74009590292	-1204.74009590292\\
60.875	0.24306	-1257.40332944923	-1257.40332944923\\
60.875	0.24672	-1311.28618162782	-1311.28618162782\\
60.875	0.25038	-1366.38865243869	-1366.38865243869\\
60.875	0.25404	-1422.71074188185	-1422.71074188185\\
60.875	0.2577	-1480.25244995729	-1480.25244995729\\
60.875	0.26136	-1539.01377666502	-1539.01377666502\\
60.875	0.26502	-1598.99472200503	-1598.99472200503\\
60.875	0.26868	-1660.19528597732	-1660.19528597732\\
60.875	0.27234	-1722.6154685819	-1722.6154685819\\
60.875	0.276	-1786.25526981876	-1786.25526981876\\
61.25	0.093	-97.9309247195077	-97.9309247195077\\
61.25	0.09666	-101.838914778671	-101.838914778671\\
61.25	0.10032	-106.966523470118	-106.966523470118\\
61.25	0.10398	-113.313750793849	-113.313750793849\\
61.25	0.10764	-120.880596749865	-120.880596749865\\
61.25	0.1113	-129.667061338165	-129.667061338165\\
61.25	0.11496	-139.673144558748	-139.673144558748\\
61.25	0.11862	-150.898846411616	-150.898846411616\\
61.25	0.12228	-163.344166896766	-163.344166896766\\
61.25	0.12594	-177.009106014202	-177.009106014202\\
61.25	0.1296	-191.893663763922	-191.893663763922\\
61.25	0.13326	-207.997840145925	-207.997840145925\\
61.25	0.13692	-225.321635160213	-225.321635160213\\
61.25	0.14058	-243.865048806785	-243.865048806785\\
61.25	0.14424	-263.62808108564	-263.62808108564\\
61.25	0.1479	-284.61073199678	-284.61073199678\\
61.25	0.15156	-306.813001540203	-306.813001540203\\
61.25	0.15522	-330.234889715912	-330.234889715912\\
61.25	0.15888	-354.876396523903	-354.876396523903\\
61.25	0.16254	-380.737521964179	-380.737521964179\\
61.25	0.1662	-407.818266036739	-407.818266036739\\
61.25	0.16986	-436.118628741583	-436.118628741583\\
61.25	0.17352	-465.638610078711	-465.638610078711\\
61.25	0.17718	-496.378210048123	-496.378210048123\\
61.25	0.18084	-528.337428649818	-528.337428649818\\
61.25	0.1845	-561.516265883799	-561.516265883799\\
61.25	0.18816	-595.914721750063	-595.914721750063\\
61.25	0.19182	-631.53279624861	-631.53279624861\\
61.25	0.19548	-668.370489379443	-668.370489379443\\
61.25	0.19914	-706.427801142559	-706.427801142559\\
61.25	0.2028	-745.70473153796	-745.70473153796\\
61.25	0.20646	-786.201280565644	-786.201280565644\\
61.25	0.21012	-827.917448225612	-827.917448225612\\
61.25	0.21378	-870.853234517864	-870.853234517864\\
61.25	0.21744	-915.0086394424	-915.0086394424\\
61.25	0.2211	-960.38366299922	-960.38366299922\\
61.25	0.22476	-1006.97830518832	-1006.97830518832\\
61.25	0.22842	-1054.79256600971	-1054.79256600971\\
61.25	0.23208	-1103.82644546339	-1103.82644546339\\
61.25	0.23574	-1154.07994354934	-1154.07994354934\\
61.25	0.2394	-1205.55306026758	-1205.55306026758\\
61.25	0.24306	-1258.24579561811	-1258.24579561811\\
61.25	0.24672	-1312.15814960092	-1312.15814960092\\
61.25	0.25038	-1367.29012221601	-1367.29012221601\\
61.25	0.25404	-1423.64171346338	-1423.64171346338\\
61.25	0.2577	-1481.21292334305	-1481.21292334305\\
61.25	0.26136	-1540.00375185499	-1540.00375185499\\
61.25	0.26502	-1600.01419899922	-1600.01419899922\\
61.25	0.26868	-1661.24426477573	-1661.24426477573\\
61.25	0.27234	-1723.69394918453	-1723.69394918453\\
61.25	0.276	-1787.36325222561	-1787.36325222561\\
61.625	0.093	-97.7024202800927	-97.7024202800927\\
61.625	0.09666	-101.639912143475	-101.639912143475\\
61.625	0.10032	-106.79702263914	-106.79702263914\\
61.625	0.10398	-113.173751767091	-113.173751767091\\
61.625	0.10764	-120.770099527325	-120.770099527325\\
61.625	0.1113	-129.586065919843	-129.586065919843\\
61.625	0.11496	-139.621650944645	-139.621650944645\\
61.625	0.11862	-150.876854601731	-150.876854601731\\
61.625	0.12228	-163.351676891101	-163.351676891101\\
61.625	0.12594	-177.046117812756	-177.046117812756\\
61.625	0.1296	-191.960177366694	-191.960177366694\\
61.625	0.13326	-208.093855552916	-208.093855552916\\
61.625	0.13692	-225.447152371423	-225.447152371423\\
61.625	0.14058	-244.020067822213	-244.020067822213\\
61.625	0.14424	-263.812601905287	-263.812601905287\\
61.625	0.1479	-284.824754620646	-284.824754620646\\
61.625	0.15156	-307.056525968288	-307.056525968288\\
61.625	0.15522	-330.507915948215	-330.507915948215\\
61.625	0.15888	-355.178924560425	-355.178924560425\\
61.625	0.16254	-381.06955180492	-381.06955180492\\
61.625	0.1662	-408.179797681698	-408.179797681698\\
61.625	0.16986	-436.509662190761	-436.509662190761\\
61.625	0.17352	-466.059145332107	-466.059145332107\\
61.625	0.17718	-496.828247105738	-496.828247105738\\
61.625	0.18084	-528.816967511652	-528.816967511652\\
61.625	0.1845	-562.025306549852	-562.025306549852\\
61.625	0.18816	-596.453264220334	-596.453264220334\\
61.625	0.19182	-632.100840523101	-632.100840523101\\
61.625	0.19548	-668.968035458152	-668.968035458152\\
61.625	0.19914	-707.054849025487	-707.054849025487\\
61.625	0.2028	-746.361281225106	-746.361281225106\\
61.625	0.20646	-786.887332057009	-786.887332057009\\
61.625	0.21012	-828.633001521196	-828.633001521196\\
61.625	0.21378	-871.598289617666	-871.598289617666\\
61.625	0.21744	-915.783196346422	-915.783196346422\\
61.625	0.2211	-961.18772170746	-961.18772170746\\
61.625	0.22476	-1007.81186570078	-1007.81186570078\\
61.625	0.22842	-1055.65562832639	-1055.65562832639\\
61.625	0.23208	-1104.71900958428	-1104.71900958428\\
61.625	0.23574	-1155.00200947446	-1155.00200947446\\
61.625	0.2394	-1206.50462799692	-1206.50462799692\\
61.625	0.24306	-1259.22686515166	-1259.22686515166\\
61.625	0.24672	-1313.16872093869	-1313.16872093869\\
61.625	0.25038	-1368.330195358	-1368.330195358\\
61.625	0.25404	-1424.71128840959	-1424.71128840959\\
61.625	0.2577	-1482.31200009347	-1482.31200009347\\
61.625	0.26136	-1541.13233040964	-1541.13233040964\\
61.625	0.26502	-1601.17227935808	-1601.17227935808\\
61.625	0.26868	-1662.43184693882	-1662.43184693882\\
61.625	0.27234	-1724.91103315183	-1724.91103315183\\
61.625	0.276	-1788.60983799713	-1788.60983799713\\
62	0.093	-97.6125192053513	-97.6125192053513\\
62	0.09666	-101.579512872952	-101.579512872952\\
62	0.10032	-106.766125172837	-106.766125172837\\
62	0.10398	-113.172356105005	-113.172356105005\\
62	0.10764	-120.798205669458	-120.798205669458\\
62	0.1113	-129.643673866195	-129.643673866195\\
62	0.11496	-139.708760695216	-139.708760695216\\
62	0.11862	-150.99346615652	-150.99346615652\\
62	0.12228	-163.49779025011	-163.49779025011\\
62	0.12594	-177.221732975983	-177.221732975983\\
62	0.1296	-192.16529433414	-192.16529433414\\
62	0.13326	-208.328474324581	-208.328474324581\\
62	0.13692	-225.711272947306	-225.711272947306\\
62	0.14058	-244.313690202314	-244.313690202314\\
62	0.14424	-264.135726089608	-264.135726089608\\
62	0.1479	-285.177380609185	-285.177380609185\\
62	0.15156	-307.438653761046	-307.438653761046\\
62	0.15522	-330.919545545191	-330.919545545191\\
62	0.15888	-355.620055961621	-355.620055961621\\
62	0.16254	-381.540185010334	-381.540185010334\\
62	0.1662	-408.679932691331	-408.679932691331\\
62	0.16986	-437.039299004612	-437.039299004612\\
62	0.17352	-466.618283950178	-466.618283950178\\
62	0.17718	-497.416887528027	-497.416887528027\\
62	0.18084	-529.435109738161	-529.435109738161\\
62	0.1845	-562.672950580578	-562.672950580578\\
62	0.18816	-597.13041005528	-597.13041005528\\
62	0.19182	-632.807488162265	-632.807488162265\\
62	0.19548	-669.704184901534	-669.704184901534\\
62	0.19914	-707.820500273088	-707.820500273088\\
62	0.2028	-747.156434276926	-747.156434276926\\
62	0.20646	-787.711986913048	-787.711986913048\\
62	0.21012	-829.487158181453	-829.487158181453\\
62	0.21378	-872.481948082142	-872.481948082142\\
62	0.21744	-916.696356615116	-916.696356615116\\
62	0.2211	-962.130383780374	-962.130383780374\\
62	0.22476	-1008.78402957792	-1008.78402957792\\
62	0.22842	-1056.65729400774	-1056.65729400774\\
62	0.23208	-1105.75017706985	-1105.75017706985\\
62	0.23574	-1156.06267876425	-1156.06267876425\\
62	0.2394	-1207.59479909092	-1207.59479909092\\
62	0.24306	-1260.34653804989	-1260.34653804989\\
62	0.24672	-1314.31789564113	-1314.31789564113\\
62	0.25038	-1369.50887186466	-1369.50887186466\\
62	0.25404	-1425.91946672048	-1425.91946672048\\
62	0.2577	-1483.54968020857	-1483.54968020857\\
62	0.26136	-1542.39951232896	-1542.39951232896\\
62	0.26502	-1602.46896308162	-1602.46896308162\\
62	0.26868	-1663.75803246657	-1663.75803246657\\
62	0.27234	-1726.26672048381	-1726.26672048381\\
62	0.276	-1789.99502713332	-1789.99502713332\\
62.375	0.093	-97.6612214952828	-97.6612214952828\\
62.375	0.09666	-101.657716967102	-101.657716967102\\
62.375	0.10032	-106.873831071206	-106.873831071206\\
62.375	0.10398	-113.309563807593	-113.309563807593\\
62.375	0.10764	-120.964915176265	-120.964915176265\\
62.375	0.1113	-129.83988517722	-129.83988517722\\
62.375	0.11496	-139.93447381046	-139.93447381046\\
62.375	0.11862	-151.248681075983	-151.248681075983\\
62.375	0.12228	-163.782506973791	-163.782506973791\\
62.375	0.12594	-177.535951503882	-177.535951503882\\
62.375	0.1296	-192.509014666259	-192.509014666259\\
62.375	0.13326	-208.701696460918	-208.701696460918\\
62.375	0.13692	-226.113996887862	-226.113996887862\\
62.375	0.14058	-244.745915947089	-244.745915947089\\
62.375	0.14424	-264.597453638602	-264.597453638602\\
62.375	0.1479	-285.668609962397	-285.668609962397\\
62.375	0.15156	-307.959384918477	-307.959384918477\\
62.375	0.15522	-331.469778506841	-331.469778506841\\
62.375	0.15888	-356.199790727489	-356.199790727489\\
62.375	0.16254	-382.149421580421	-382.149421580421\\
62.375	0.1662	-409.318671065637	-409.318671065637\\
62.375	0.16986	-437.707539183137	-437.707539183137\\
62.375	0.17352	-467.316025932921	-467.316025932921\\
62.375	0.17718	-498.144131314989	-498.144131314989\\
62.375	0.18084	-530.191855329341	-530.191855329341\\
62.375	0.1845	-563.459197975977	-563.459197975977\\
62.375	0.18816	-597.946159254898	-597.946159254898\\
62.375	0.19182	-633.652739166102	-633.652739166102\\
62.375	0.19548	-670.57893770959	-670.57893770959\\
62.375	0.19914	-708.724754885362	-708.724754885362\\
62.375	0.2028	-748.090190693419	-748.090190693419\\
62.375	0.20646	-788.67524513376	-788.67524513376\\
62.375	0.21012	-830.479918206383	-830.479918206383\\
62.375	0.21378	-873.504209911292	-873.504209911292\\
62.375	0.21744	-917.748120248484	-917.748120248484\\
62.375	0.2211	-963.211649217961	-963.211649217961\\
62.375	0.22476	-1009.89479681972	-1009.89479681972\\
62.375	0.22842	-1057.79756305377	-1057.79756305377\\
62.375	0.23208	-1106.91994792009	-1106.91994792009\\
62.375	0.23574	-1157.26195141871	-1157.26195141871\\
62.375	0.2394	-1208.8235735496	-1208.8235735496\\
62.375	0.24306	-1261.60481431278	-1261.60481431278\\
62.375	0.24672	-1315.60567370825	-1315.60567370825\\
62.375	0.25038	-1370.826151736	-1370.826151736\\
62.375	0.25404	-1427.26624839603	-1427.26624839603\\
62.375	0.2577	-1484.92596368835	-1484.92596368835\\
62.375	0.26136	-1543.80529761295	-1543.80529761295\\
62.375	0.26502	-1603.90425016983	-1603.90425016983\\
62.375	0.26868	-1665.222821359	-1665.222821359\\
62.375	0.27234	-1727.76101118045	-1727.76101118045\\
62.375	0.276	-1791.51881963419	-1791.51881963419\\
62.75	0.093	-97.8485271498881	-97.8485271498881\\
62.75	0.09666	-101.874524425926	-101.874524425926\\
62.75	0.10032	-107.120140334248	-107.120140334248\\
62.75	0.10398	-113.585374874855	-113.585374874855\\
62.75	0.10764	-121.270228047745	-121.270228047745\\
62.75	0.1113	-130.174699852919	-130.174699852919\\
62.75	0.11496	-140.298790290377	-140.298790290377\\
62.75	0.11862	-151.64249936012	-151.64249936012\\
62.75	0.12228	-164.205827062146	-164.205827062146\\
62.75	0.12594	-177.988773396457	-177.988773396457\\
62.75	0.1296	-192.991338363051	-192.991338363051\\
62.75	0.13326	-209.213521961929	-209.213521961929\\
62.75	0.13692	-226.655324193092	-226.655324193092\\
62.75	0.14058	-245.316745056538	-245.316745056538\\
62.75	0.14424	-265.197784552269	-265.197784552269\\
62.75	0.1479	-286.298442680284	-286.298442680284\\
62.75	0.15156	-308.618719440582	-308.618719440582\\
62.75	0.15522	-332.158614833165	-332.158614833165\\
62.75	0.15888	-356.918128858031	-356.918128858031\\
62.75	0.16254	-382.897261515182	-382.897261515182\\
62.75	0.1662	-410.096012804617	-410.096012804617\\
62.75	0.16986	-438.514382726336	-438.514382726336\\
62.75	0.17352	-468.152371280338	-468.152371280338\\
62.75	0.17718	-499.009978466625	-499.009978466625\\
62.75	0.18084	-531.087204285195	-531.087204285195\\
62.75	0.1845	-564.384048736051	-564.384048736051\\
62.75	0.18816	-598.900511819189	-598.900511819189\\
62.75	0.19182	-634.636593534612	-634.636593534612\\
62.75	0.19548	-671.59229388232	-671.59229388232\\
62.75	0.19914	-709.767612862311	-709.767612862311\\
62.75	0.2028	-749.162550474586	-749.162550474586\\
62.75	0.20646	-789.777106719145	-789.777106719145\\
62.75	0.21012	-831.611281595988	-831.611281595988\\
62.75	0.21378	-874.665075105115	-874.665075105115\\
62.75	0.21744	-918.938487246526	-918.938487246526\\
62.75	0.2211	-964.431518020221	-964.431518020221\\
62.75	0.22476	-1011.1441674262	-1011.1441674262\\
62.75	0.22842	-1059.07643546446	-1059.07643546446\\
62.75	0.23208	-1108.22832213501	-1108.22832213501\\
62.75	0.23574	-1158.59982743784	-1158.59982743784\\
62.75	0.2394	-1210.19095137296	-1210.19095137296\\
62.75	0.24306	-1263.00169394036	-1263.00169394036\\
62.75	0.24672	-1317.03205514004	-1317.03205514004\\
62.75	0.25038	-1372.28203497201	-1372.28203497201\\
62.75	0.25404	-1428.75163343626	-1428.75163343626\\
62.75	0.2577	-1486.4408505328	-1486.4408505328\\
62.75	0.26136	-1545.34968626161	-1545.34968626161\\
62.75	0.26502	-1605.47814062272	-1605.47814062272\\
62.75	0.26868	-1666.82621361611	-1666.82621361611\\
62.75	0.27234	-1729.39390524178	-1729.39390524178\\
62.75	0.276	-1793.18121549973	-1793.18121549973\\
63.125	0.093	-98.1744361691664	-98.1744361691664\\
63.125	0.09666	-102.229935249424	-102.229935249424\\
63.125	0.10032	-107.505052961964	-107.505052961964\\
63.125	0.10398	-113.999789306789	-113.999789306789\\
63.125	0.10764	-121.714144283898	-121.714144283898\\
63.125	0.1113	-130.648117893291	-130.648117893291\\
63.125	0.11496	-140.801710134968	-140.801710134968\\
63.125	0.11862	-152.174921008929	-152.174921008929\\
63.125	0.12228	-164.767750515174	-164.767750515174\\
63.125	0.12594	-178.580198653704	-178.580198653704\\
63.125	0.1296	-193.612265424516	-193.612265424516\\
63.125	0.13326	-209.863950827613	-209.863950827613\\
63.125	0.13692	-227.335254862994	-227.335254862994\\
63.125	0.14058	-246.02617753066	-246.02617753066\\
63.125	0.14424	-265.936718830609	-265.936718830609\\
63.125	0.1479	-287.066878762843	-287.066878762843\\
63.125	0.15156	-309.416657327359	-309.416657327359\\
63.125	0.15522	-332.986054524161	-332.986054524161\\
63.125	0.15888	-357.775070353247	-357.775070353247\\
63.125	0.16254	-383.783704814616	-383.783704814616\\
63.125	0.1662	-411.011957908269	-411.011957908269\\
63.125	0.16986	-439.459829634207	-439.459829634207\\
63.125	0.17352	-469.127319992428	-469.127319992428\\
63.125	0.17718	-500.014428982934	-500.014428982934\\
63.125	0.18084	-532.121156605723	-532.121156605723\\
63.125	0.1845	-565.447502860797	-565.447502860797\\
63.125	0.18816	-599.993467748155	-599.993467748155\\
63.125	0.19182	-635.759051267796	-635.759051267796\\
63.125	0.19548	-672.744253419722	-672.744253419722\\
63.125	0.19914	-710.949074203932	-710.949074203932\\
63.125	0.2028	-750.373513620426	-750.373513620426\\
63.125	0.20646	-791.017571669204	-791.017571669204\\
63.125	0.21012	-832.881248350265	-832.881248350265\\
63.125	0.21378	-875.96454366361	-875.96454366361\\
63.125	0.21744	-920.267457609241	-920.267457609241\\
63.125	0.2211	-965.789990187154	-965.789990187154\\
63.125	0.22476	-1012.53214139735	-1012.53214139735\\
63.125	0.22842	-1060.49391123983	-1060.49391123983\\
63.125	0.23208	-1109.6752997146	-1109.6752997146\\
63.125	0.23574	-1160.07630682165	-1160.07630682165\\
63.125	0.2394	-1211.69693256098	-1211.69693256098\\
63.125	0.24306	-1264.5371769326	-1264.5371769326\\
63.125	0.24672	-1318.5970399365	-1318.5970399365\\
63.125	0.25038	-1373.87652157269	-1373.87652157269\\
63.125	0.25404	-1430.37562184116	-1430.37562184116\\
63.125	0.2577	-1488.09434074192	-1488.09434074192\\
63.125	0.26136	-1547.03267827495	-1547.03267827495\\
63.125	0.26502	-1607.19063444028	-1607.19063444028\\
63.125	0.26868	-1668.56820923788	-1668.56820923788\\
63.125	0.27234	-1731.16540266777	-1731.16540266777\\
63.125	0.276	-1794.98221472995	-1794.98221472995\\
63.5	0.093	-98.6389485531184	-98.6389485531184\\
63.5	0.09666	-102.723949437594	-102.723949437594\\
63.5	0.10032	-108.028568954353	-108.028568954353\\
63.5	0.10398	-114.552807103397	-114.552807103397\\
63.5	0.10764	-122.296663884724	-122.296663884724\\
63.5	0.1113	-131.260139298336	-131.260139298336\\
63.5	0.11496	-141.443233344232	-141.443233344232\\
63.5	0.11862	-152.845946022412	-152.845946022412\\
63.5	0.12228	-165.468277332875	-165.468277332875\\
63.5	0.12594	-179.310227275623	-179.310227275623\\
63.5	0.1296	-194.371795850655	-194.371795850655\\
63.5	0.13326	-210.652983057971	-210.652983057971\\
63.5	0.13692	-228.153788897571	-228.153788897571\\
63.5	0.14058	-246.874213369455	-246.874213369455\\
63.5	0.14424	-266.814256473623	-266.814256473623\\
63.5	0.1479	-287.973918210075	-287.973918210075\\
63.5	0.15156	-310.353198578811	-310.353198578811\\
63.5	0.15522	-333.952097579831	-333.952097579831\\
63.5	0.15888	-358.770615213135	-358.770615213135\\
63.5	0.16254	-384.808751478723	-384.808751478723\\
63.5	0.1662	-412.066506376595	-412.066506376595\\
63.5	0.16986	-440.543879906752	-440.543879906752\\
63.5	0.17352	-470.240872069191	-470.240872069191\\
63.5	0.17718	-501.157482863916	-501.157482863916\\
63.5	0.18084	-533.293712290924	-533.293712290924\\
63.5	0.1845	-566.649560350217	-566.649560350217\\
63.5	0.18816	-601.225027041793	-601.225027041793\\
63.5	0.19182	-637.020112365653	-637.020112365653\\
63.5	0.19548	-674.034816321798	-674.034816321798\\
63.5	0.19914	-712.269138910227	-712.269138910227\\
63.5	0.2028	-751.723080130939	-751.723080130939\\
63.5	0.20646	-792.396639983936	-792.396639983936\\
63.5	0.21012	-834.289818469216	-834.289818469216\\
63.5	0.21378	-877.40261558678	-877.40261558678\\
63.5	0.21744	-921.735031336628	-921.735031336628\\
63.5	0.2211	-967.287065718761	-967.287065718761\\
63.5	0.22476	-1014.05871873318	-1014.05871873318\\
63.5	0.22842	-1062.04999037988	-1062.04999037988\\
63.5	0.23208	-1111.26088065886	-1111.26088065886\\
63.5	0.23574	-1161.69138957013	-1161.69138957013\\
63.5	0.2394	-1213.34151711368	-1213.34151711368\\
63.5	0.24306	-1266.21126328952	-1266.21126328952\\
63.5	0.24672	-1320.30062809764	-1320.30062809764\\
63.5	0.25038	-1375.60961153805	-1375.60961153805\\
63.5	0.25404	-1432.13821361074	-1432.13821361074\\
63.5	0.2577	-1489.88643431571	-1489.88643431571\\
63.5	0.26136	-1548.85427365297	-1548.85427365297\\
63.5	0.26502	-1609.04173162251	-1609.04173162251\\
63.5	0.26868	-1670.44880822433	-1670.44880822433\\
63.5	0.27234	-1733.07550345844	-1733.07550345844\\
63.5	0.276	-1796.92181732483	-1796.92181732483\\
63.875	0.093	-99.2420643017439	-99.2420643017439\\
63.875	0.09666	-103.356566990438	-103.356566990438\\
63.875	0.10032	-108.690688311416	-108.690688311416\\
63.875	0.10398	-115.244428264678	-115.244428264678\\
63.875	0.10764	-123.017786850225	-123.017786850225\\
63.875	0.1113	-132.010764068055	-132.010764068055\\
63.875	0.11496	-142.22335991817	-142.22335991817\\
63.875	0.11862	-153.655574400568	-153.655574400568\\
63.875	0.12228	-166.307407515251	-166.307407515251\\
63.875	0.12594	-180.178859262217	-180.178859262217\\
63.875	0.1296	-195.269929641468	-195.269929641468\\
63.875	0.13326	-211.580618653002	-211.580618653002\\
63.875	0.13692	-229.110926296821	-229.110926296821\\
63.875	0.14058	-247.860852572923	-247.860852572923\\
63.875	0.14424	-267.830397481311	-267.830397481311\\
63.875	0.1479	-289.019561021981	-289.019561021981\\
63.875	0.15156	-311.428343194936	-311.428343194936\\
63.875	0.15522	-335.056744000174	-335.056744000174\\
63.875	0.15888	-359.904763437697	-359.904763437697\\
63.875	0.16254	-385.972401507504	-385.972401507504\\
63.875	0.1662	-413.259658209595	-413.259658209595\\
63.875	0.16986	-441.76653354397	-441.76653354397\\
63.875	0.17352	-471.493027510629	-471.493027510629\\
63.875	0.17718	-502.439140109571	-502.439140109571\\
63.875	0.18084	-534.604871340799	-534.604871340799\\
63.875	0.1845	-567.990221204309	-567.990221204309\\
63.875	0.18816	-602.595189700105	-602.595189700105\\
63.875	0.19182	-638.419776828184	-638.419776828184\\
63.875	0.19548	-675.463982588547	-675.463982588547\\
63.875	0.19914	-713.727806981194	-713.727806981194\\
63.875	0.2028	-753.211250006126	-753.211250006126\\
63.875	0.20646	-793.914311663341	-793.914311663341\\
63.875	0.21012	-835.83699195284	-835.83699195284\\
63.875	0.21378	-878.979290874623	-878.979290874623\\
63.875	0.21744	-923.34120842869	-923.34120842869\\
63.875	0.2211	-968.922744615042	-968.922744615042\\
63.875	0.22476	-1015.72389943368	-1015.72389943368\\
63.875	0.22842	-1063.7446728846	-1063.7446728846\\
63.875	0.23208	-1112.9850649678	-1112.9850649678\\
63.875	0.23574	-1163.44507568329	-1163.44507568329\\
63.875	0.2394	-1215.12470503106	-1215.12470503106\\
63.875	0.24306	-1268.02395301111	-1268.02395301111\\
63.875	0.24672	-1322.14281962345	-1322.14281962345\\
63.875	0.25038	-1377.48130486808	-1377.48130486808\\
63.875	0.25404	-1434.03940874499	-1434.03940874499\\
63.875	0.2577	-1491.81713125418	-1491.81713125418\\
63.875	0.26136	-1550.81447239565	-1550.81447239565\\
63.875	0.26502	-1611.03143216941	-1611.03143216941\\
63.875	0.26868	-1672.46801057546	-1672.46801057546\\
63.875	0.27234	-1735.12420761378	-1735.12420761378\\
63.875	0.276	-1799.00002328439	-1799.00002328439\\
64.25	0.093	-99.9837834150422	-99.9837834150422\\
64.25	0.09666	-104.127787907955	-104.127787907955\\
64.25	0.10032	-109.491411033152	-109.491411033152\\
64.25	0.10398	-116.074652790633	-116.074652790633\\
64.25	0.10764	-123.877513180398	-123.877513180398\\
64.25	0.1113	-132.899992202447	-132.899992202447\\
64.25	0.11496	-143.142089856781	-143.142089856781\\
64.25	0.11862	-154.603806143397	-154.603806143397\\
64.25	0.12228	-167.285141062299	-167.285141062299\\
64.25	0.12594	-181.186094613484	-181.186094613484\\
64.25	0.1296	-196.306666796953	-196.306666796953\\
64.25	0.13326	-212.646857612706	-212.646857612706\\
64.25	0.13692	-230.206667060744	-230.206667060744\\
64.25	0.14058	-248.986095141065	-248.986095141065\\
64.25	0.14424	-268.985141853671	-268.985141853671\\
64.25	0.1479	-290.20380719856	-290.20380719856\\
64.25	0.15156	-312.642091175734	-312.642091175734\\
64.25	0.15522	-336.299993785191	-336.299993785191\\
64.25	0.15888	-361.177515026933	-361.177515026933\\
64.25	0.16254	-387.274654900958	-387.274654900958\\
64.25	0.1662	-414.591413407268	-414.591413407268\\
64.25	0.16986	-443.127790545861	-443.127790545861\\
64.25	0.17352	-472.883786316739	-472.883786316739\\
64.25	0.17718	-503.8594007199	-503.8594007199\\
64.25	0.18084	-536.054633755346	-536.054633755346\\
64.25	0.1845	-569.469485423076	-569.469485423076\\
64.25	0.18816	-604.10395572309	-604.10395572309\\
64.25	0.19182	-639.958044655388	-639.958044655388\\
64.25	0.19548	-677.031752219969	-677.031752219969\\
64.25	0.19914	-715.325078416835	-715.325078416835\\
64.25	0.2028	-754.838023245986	-754.838023245986\\
64.25	0.20646	-795.57058670742	-795.57058670742\\
64.25	0.21012	-837.522768801137	-837.522768801137\\
64.25	0.21378	-880.694569527139	-880.694569527139\\
64.25	0.21744	-925.085988885425	-925.085988885425\\
64.25	0.2211	-970.697026875995	-970.697026875995\\
64.25	0.22476	-1017.52768349885	-1017.52768349885\\
64.25	0.22842	-1065.57795875399	-1065.57795875399\\
64.25	0.23208	-1114.84785264141	-1114.84785264141\\
64.25	0.23574	-1165.33736516112	-1165.33736516112\\
64.25	0.2394	-1217.04649631311	-1217.04649631311\\
64.25	0.24306	-1269.97524609738	-1269.97524609738\\
64.25	0.24672	-1324.12361451394	-1324.12361451394\\
64.25	0.25038	-1379.49160156278	-1379.49160156278\\
64.25	0.25404	-1436.07920724391	-1436.07920724391\\
64.25	0.2577	-1493.88643155732	-1493.88643155732\\
64.25	0.26136	-1552.91327450301	-1552.91327450301\\
64.25	0.26502	-1613.15973608099	-1613.15973608099\\
64.25	0.26868	-1674.62581629125	-1674.62581629125\\
64.25	0.27234	-1737.3115151338	-1737.3115151338\\
64.25	0.276	-1801.21683260863	-1801.21683260863\\
64.625	0.093	-100.864105893014	-100.864105893014\\
64.625	0.09666	-105.037612190146	-105.037612190146\\
64.625	0.10032	-110.430737119561	-110.430737119561\\
64.625	0.10398	-117.043480681261	-117.043480681261\\
64.625	0.10764	-124.875842875245	-124.875842875245\\
64.625	0.1113	-133.927823701513	-133.927823701513\\
64.625	0.11496	-144.199423160064	-144.199423160064\\
64.625	0.11862	-155.690641250901	-155.690641250901\\
64.625	0.12228	-168.40147797402	-168.40147797402\\
64.625	0.12594	-182.331933329425	-182.331933329425\\
64.625	0.1296	-197.482007317112	-197.482007317112\\
64.625	0.13326	-213.851699937084	-213.851699937084\\
64.625	0.13692	-231.44101118934	-231.44101118934\\
64.625	0.14058	-250.249941073881	-250.249941073881\\
64.625	0.14424	-270.278489590704	-270.278489590704\\
64.625	0.1479	-291.526656739813	-291.526656739813\\
64.625	0.15156	-313.994442521205	-313.994442521205\\
64.625	0.15522	-337.681846934881	-337.681846934881\\
64.625	0.15888	-362.588869980841	-362.588869980841\\
64.625	0.16254	-388.715511659085	-388.715511659085\\
64.625	0.1662	-416.061771969614	-416.061771969614\\
64.625	0.16986	-444.627650912427	-444.627650912427\\
64.625	0.17352	-474.413148487522	-474.413148487522\\
64.625	0.17718	-505.418264694903	-505.418264694903\\
64.625	0.18084	-537.642999534567	-537.642999534567\\
64.625	0.1845	-571.087353006516	-571.087353006516\\
64.625	0.18816	-605.751325110749	-605.751325110749\\
64.625	0.19182	-641.634915847264	-641.634915847264\\
64.625	0.19548	-678.738125216066	-678.738125216066\\
64.625	0.19914	-717.060953217151	-717.060953217151\\
64.625	0.2028	-756.603399850519	-756.603399850519\\
64.625	0.20646	-797.365465116172	-797.365465116172\\
64.625	0.21012	-839.347149014108	-839.347149014108\\
64.625	0.21378	-882.548451544329	-882.548451544329\\
64.625	0.21744	-926.969372706833	-926.969372706833\\
64.625	0.2211	-972.609912501622	-972.609912501622\\
64.625	0.22476	-1019.47007092869	-1019.47007092869\\
64.625	0.22842	-1067.54984798805	-1067.54984798805\\
64.625	0.23208	-1116.84924367969	-1116.84924367969\\
64.625	0.23574	-1167.36825800362	-1167.36825800362\\
64.625	0.2394	-1219.10689095983	-1219.10689095983\\
64.625	0.24306	-1272.06514254832	-1272.06514254832\\
64.625	0.24672	-1326.2430127691	-1326.2430127691\\
64.625	0.25038	-1381.64050162216	-1381.64050162216\\
64.625	0.25404	-1438.2576091075	-1438.2576091075\\
64.625	0.2577	-1496.09433522513	-1496.09433522513\\
64.625	0.26136	-1555.15067997505	-1555.15067997505\\
64.625	0.26502	-1615.42664335724	-1615.42664335724\\
64.625	0.26868	-1676.92222537172	-1676.92222537172\\
64.625	0.27234	-1739.63742601849	-1739.63742601849\\
64.625	0.276	-1803.57224529754	-1803.57224529754\\
65	0.093	-101.883031735659	-101.883031735659\\
65	0.09666	-106.086039837009	-106.086039837009\\
65	0.10032	-111.508666570643	-111.508666570643\\
65	0.10398	-118.150911936562	-118.150911936562\\
65	0.10764	-126.012775934765	-126.012775934765\\
65	0.1113	-135.094258565251	-135.094258565251\\
65	0.11496	-145.395359828022	-145.395359828022\\
65	0.11862	-156.916079723077	-156.916079723077\\
65	0.12228	-169.656418250415	-169.656418250415\\
65	0.12594	-183.616375410038	-183.616375410038\\
65	0.1296	-198.795951201944	-198.795951201944\\
65	0.13326	-215.195145626135	-215.195145626135\\
65	0.13692	-232.81395868261	-232.81395868261\\
65	0.14058	-251.652390371369	-251.652390371369\\
65	0.14424	-271.710440692411	-271.710440692411\\
65	0.1479	-292.988109645739	-292.988109645739\\
65	0.15156	-315.485397231349	-315.485397231349\\
65	0.15522	-339.202303449244	-339.202303449244\\
65	0.15888	-364.138828299423	-364.138828299423\\
65	0.16254	-390.294971781886	-390.294971781886\\
65	0.1662	-417.670733896633	-417.670733896633\\
65	0.16986	-446.266114643665	-446.266114643665\\
65	0.17352	-476.081114022979	-476.081114022979\\
65	0.17718	-507.115732034578	-507.115732034578\\
65	0.18084	-539.369968678461	-539.369968678461\\
65	0.1845	-572.843823954629	-572.843823954629\\
65	0.18816	-607.53729786308	-607.53729786308\\
65	0.19182	-643.450390403815	-643.450390403815\\
65	0.19548	-680.583101576834	-680.583101576834\\
65	0.19914	-718.935431382138	-718.935431382138\\
65	0.2028	-758.507379819725	-758.507379819725\\
65	0.20646	-799.298946889597	-799.298946889597\\
65	0.21012	-841.310132591752	-841.310132591752\\
65	0.21378	-884.540936926191	-884.540936926191\\
65	0.21744	-928.991359892915	-928.991359892915\\
65	0.2211	-974.661401491922	-974.661401491922\\
65	0.22476	-1021.55106172321	-1021.55106172321\\
65	0.22842	-1069.66034058679	-1069.66034058679\\
65	0.23208	-1118.98923808265	-1118.98923808265\\
65	0.23574	-1169.53775421079	-1169.53775421079\\
65	0.2394	-1221.30588897122	-1221.30588897122\\
65	0.24306	-1274.29364236393	-1274.29364236393\\
65	0.24672	-1328.50101438893	-1328.50101438893\\
65	0.25038	-1383.92800504621	-1383.92800504621\\
65	0.25404	-1440.57461433577	-1440.57461433577\\
65	0.2577	-1498.44084225762	-1498.44084225762\\
65	0.26136	-1557.52668881175	-1557.52668881175\\
65	0.26502	-1617.83215399817	-1617.83215399817\\
65	0.26868	-1679.35723781687	-1679.35723781687\\
65	0.27234	-1742.10194026785	-1742.10194026785\\
65	0.276	-1806.06626135112	-1806.06626135112\\
65.375	0.093	-103.040560942978	-103.040560942978\\
65.375	0.09666	-107.273070848546	-107.273070848546\\
65.375	0.10032	-112.7251993864	-112.7251993864\\
65.375	0.10398	-119.396946556537	-119.396946556537\\
65.375	0.10764	-127.288312358958	-127.288312358958\\
65.375	0.1113	-136.399296793664	-136.399296793664\\
65.375	0.11496	-146.729899860652	-146.729899860652\\
65.375	0.11862	-158.280121559926	-158.280121559926\\
65.375	0.12228	-171.049961891483	-171.049961891483\\
65.375	0.12594	-185.039420855325	-185.039420855325\\
65.375	0.1296	-200.24849845145	-200.24849845145\\
65.375	0.13326	-216.677194679859	-216.677194679859\\
65.375	0.13692	-234.325509540553	-234.325509540553\\
65.375	0.14058	-253.193443033531	-253.193443033531\\
65.375	0.14424	-273.280995158792	-273.280995158792\\
65.375	0.1479	-294.588165916338	-294.588165916338\\
65.375	0.15156	-317.114955306167	-317.114955306167\\
65.375	0.15522	-340.861363328281	-340.861363328281\\
65.375	0.15888	-365.827389982679	-365.827389982679\\
65.375	0.16254	-392.01303526936	-392.01303526936\\
65.375	0.1662	-419.418299188326	-419.418299188326\\
65.375	0.16986	-448.043181739576	-448.043181739576\\
65.375	0.17352	-477.887682923109	-477.887682923109\\
65.375	0.17718	-508.951802738928	-508.951802738928\\
65.375	0.18084	-541.235541187029	-541.235541187029\\
65.375	0.1845	-574.738898267415	-574.738898267415\\
65.375	0.18816	-609.461873980085	-609.461873980085\\
65.375	0.19182	-645.404468325039	-645.404468325039\\
65.375	0.19548	-682.566681302277	-682.566681302277\\
65.375	0.19914	-720.948512911799	-720.948512911799\\
65.375	0.2028	-760.549963153605	-760.549963153605\\
65.375	0.20646	-801.371032027695	-801.371032027695\\
65.375	0.21012	-843.41171953407	-843.41171953407\\
65.375	0.21378	-886.672025672727	-886.672025672727\\
65.375	0.21744	-931.151950443669	-931.151950443669\\
65.375	0.2211	-976.851493846896	-976.851493846896\\
65.375	0.22476	-1023.77065588241	-1023.77065588241\\
65.375	0.22842	-1071.9094365502	-1071.9094365502\\
65.375	0.23208	-1121.26783585028	-1121.26783585028\\
65.375	0.23574	-1171.84585378264	-1171.84585378264\\
65.375	0.2394	-1223.64349034729	-1223.64349034729\\
65.375	0.24306	-1276.66074554422	-1276.66074554422\\
65.375	0.24672	-1330.89761937343	-1330.89761937343\\
65.375	0.25038	-1386.35411183493	-1386.35411183493\\
65.375	0.25404	-1443.03022292871	-1443.03022292871\\
65.375	0.2577	-1500.92595265478	-1500.92595265478\\
65.375	0.26136	-1560.04130101313	-1560.04130101313\\
65.375	0.26502	-1620.37626800376	-1620.37626800376\\
65.375	0.26868	-1681.93085362668	-1681.93085362668\\
65.375	0.27234	-1744.70505788189	-1744.70505788189\\
65.375	0.276	-1808.69888076937	-1808.69888076937\\
65.75	0.093	-104.33669351497	-104.33669351497\\
65.75	0.09666	-108.598705224758	-108.598705224758\\
65.75	0.10032	-114.080335566829	-114.080335566829\\
65.75	0.10398	-120.781584541186	-120.781584541186\\
65.75	0.10764	-128.702452147825	-128.702452147825\\
65.75	0.1113	-137.842938386749	-137.842938386749\\
65.75	0.11496	-148.203043257957	-148.203043257957\\
65.75	0.11862	-159.78276676145	-159.78276676145\\
65.75	0.12228	-172.582108897225	-172.582108897225\\
65.75	0.12594	-186.601069665286	-186.601069665286\\
65.75	0.1296	-201.83964906563	-201.83964906563\\
65.75	0.13326	-218.297847098258	-218.297847098258\\
65.75	0.13692	-235.97566376317	-235.97566376317\\
65.75	0.14058	-254.873099060367	-254.873099060367\\
65.75	0.14424	-274.990152989847	-274.990152989847\\
65.75	0.1479	-296.326825551611	-296.326825551611\\
65.75	0.15156	-318.883116745659	-318.883116745659\\
65.75	0.15522	-342.659026571992	-342.659026571992\\
65.75	0.15888	-367.654555030608	-367.654555030608\\
65.75	0.16254	-393.869702121508	-393.869702121508\\
65.75	0.1662	-421.304467844693	-421.304467844693\\
65.75	0.16986	-449.958852200162	-449.958852200162\\
65.75	0.17352	-479.832855187914	-479.832855187914\\
65.75	0.17718	-510.926476807951	-510.926476807951\\
65.75	0.18084	-543.239717060271	-543.239717060271\\
65.75	0.1845	-576.772575944876	-576.772575944876\\
65.75	0.18816	-611.525053461764	-611.525053461764\\
65.75	0.19182	-647.497149610937	-647.497149610937\\
65.75	0.19548	-684.688864392393	-684.688864392393\\
65.75	0.19914	-723.100197806135	-723.100197806135\\
65.75	0.2028	-762.731149852159	-762.731149852159\\
65.75	0.20646	-803.581720530468	-803.581720530468\\
65.75	0.21012	-845.651909841061	-845.651909841061\\
65.75	0.21378	-888.941717783937	-888.941717783937\\
65.75	0.21744	-933.451144359098	-933.451144359098\\
65.75	0.2211	-979.180189566543	-979.180189566543\\
65.75	0.22476	-1026.12885340627	-1026.12885340627\\
65.75	0.22842	-1074.29713587829	-1074.29713587829\\
65.75	0.23208	-1123.68503698258	-1123.68503698258\\
65.75	0.23574	-1174.29255671916	-1174.29255671916\\
65.75	0.2394	-1226.11969508803	-1226.11969508803\\
65.75	0.24306	-1279.16645208918	-1279.16645208918\\
65.75	0.24672	-1333.43282772261	-1333.43282772261\\
65.75	0.25038	-1388.91882198833	-1388.91882198833\\
65.75	0.25404	-1445.62443488633	-1445.62443488633\\
65.75	0.2577	-1503.54966641661	-1503.54966641661\\
65.75	0.26136	-1562.69451657918	-1562.69451657918\\
65.75	0.26502	-1623.05898537404	-1623.05898537404\\
65.75	0.26868	-1684.64307280117	-1684.64307280117\\
65.75	0.27234	-1747.4467788606	-1747.4467788606\\
65.75	0.276	-1811.4701035523	-1811.4701035523\\
66.125	0.093	-105.771429451635	-105.771429451635\\
66.125	0.09666	-110.062942965641	-110.062942965641\\
66.125	0.10032	-115.574075111931	-115.574075111931\\
66.125	0.10398	-122.304825890506	-122.304825890506\\
66.125	0.10764	-130.255195301365	-130.255195301365\\
66.125	0.1113	-139.425183344508	-139.425183344508\\
66.125	0.11496	-149.814790019934	-149.814790019934\\
66.125	0.11862	-161.424015327646	-161.424015327646\\
66.125	0.12228	-174.25285926764	-174.25285926764\\
66.125	0.12594	-188.301321839919	-188.301321839919\\
66.125	0.1296	-203.569403044482	-203.569403044482\\
66.125	0.13326	-220.057102881329	-220.057102881329\\
66.125	0.13692	-237.76442135046	-237.76442135046\\
66.125	0.14058	-256.691358451875	-256.691358451875\\
66.125	0.14424	-276.837914185573	-276.837914185573\\
66.125	0.1479	-298.204088551557	-298.204088551557\\
66.125	0.15156	-320.789881549823	-320.789881549823\\
66.125	0.15522	-344.595293180375	-344.595293180375\\
66.125	0.15888	-369.62032344321	-369.62032344321\\
66.125	0.16254	-395.864972338329	-395.864972338329\\
66.125	0.1662	-423.329239865732	-423.329239865732\\
66.125	0.16986	-452.01312602542	-452.01312602542\\
66.125	0.17352	-481.91663081739	-481.91663081739\\
66.125	0.17718	-513.039754241646	-513.039754241646\\
66.125	0.18084	-545.382496298185	-545.382496298185\\
66.125	0.1845	-578.944856987008	-578.944856987008\\
66.125	0.18816	-613.726836308115	-613.726836308115\\
66.125	0.19182	-649.728434261507	-649.728434261507\\
66.125	0.19548	-686.949650847182	-686.949650847182\\
66.125	0.19914	-725.390486065142	-725.390486065142\\
66.125	0.2028	-765.050939915386	-765.050939915386\\
66.125	0.20646	-805.931012397913	-805.931012397913\\
66.125	0.21012	-848.030703512724	-848.030703512724\\
66.125	0.21378	-891.35001325982	-891.35001325982\\
66.125	0.21744	-935.888941639199	-935.888941639199\\
66.125	0.2211	-981.647488650863	-981.647488650863\\
66.125	0.22476	-1028.62565429481	-1028.62565429481\\
66.125	0.22842	-1076.82343857104	-1076.82343857104\\
66.125	0.23208	-1126.24084147956	-1126.24084147956\\
66.125	0.23574	-1176.87786302036	-1176.87786302036\\
66.125	0.2394	-1228.73450319344	-1228.73450319344\\
66.125	0.24306	-1281.81076199881	-1281.81076199881\\
66.125	0.24672	-1336.10663943646	-1336.10663943646\\
66.125	0.25038	-1391.6221355064	-1391.6221355064\\
66.125	0.25404	-1448.35725020862	-1448.35725020862\\
66.125	0.2577	-1506.31198354312	-1506.31198354312\\
66.125	0.26136	-1565.48633550991	-1565.48633550991\\
66.125	0.26502	-1625.88030610898	-1625.88030610898\\
66.125	0.26868	-1687.49389534034	-1687.49389534034\\
66.125	0.27234	-1750.32710320398	-1750.32710320398\\
66.125	0.276	-1814.3799296999	-1814.3799296999\\
66.5	0.093	-107.344768752973	-107.344768752973\\
66.5	0.09666	-111.665784071198	-111.665784071198\\
66.5	0.10032	-117.206418021707	-117.206418021707\\
66.5	0.10398	-123.966670604501	-123.966670604501\\
66.5	0.10764	-131.946541819578	-131.946541819578\\
66.5	0.1113	-141.14603166694	-141.14603166694\\
66.5	0.11496	-151.565140146585	-151.565140146585\\
66.5	0.11862	-163.203867258515	-163.203867258515\\
66.5	0.12228	-176.062213002728	-176.062213002728\\
66.5	0.12594	-190.140177379226	-190.140177379226\\
66.5	0.1296	-205.437760388007	-205.437760388007\\
66.5	0.13326	-221.954962029073	-221.954962029073\\
66.5	0.13692	-239.691782302422	-239.691782302422\\
66.5	0.14058	-258.648221208056	-258.648221208056\\
66.5	0.14424	-278.824278745974	-278.824278745974\\
66.5	0.1479	-300.219954916176	-300.219954916176\\
66.5	0.15156	-322.835249718661	-322.835249718661\\
66.5	0.15522	-346.670163153431	-346.670163153431\\
66.5	0.15888	-371.724695220485	-371.724695220485\\
66.5	0.16254	-397.998845919823	-397.998845919823\\
66.5	0.1662	-425.492615251444	-425.492615251444\\
66.5	0.16986	-454.206003215351	-454.206003215351\\
66.5	0.17352	-484.13900981154	-484.13900981154\\
66.5	0.17718	-515.291635040015	-515.291635040015\\
66.5	0.18084	-547.663878900772	-547.663878900772\\
66.5	0.1845	-581.255741393815	-581.255741393815\\
66.5	0.18816	-616.06722251914	-616.06722251914\\
66.5	0.19182	-652.09832227675	-652.09832227675\\
66.5	0.19548	-689.349040666645	-689.349040666645\\
66.5	0.19914	-727.819377688823	-727.819377688823\\
66.5	0.2028	-767.509333343286	-767.509333343286\\
66.5	0.20646	-808.418907630032	-808.418907630032\\
66.5	0.21012	-850.548100549062	-850.548100549062\\
66.5	0.21378	-893.896912100376	-893.896912100376\\
66.5	0.21744	-938.465342283974	-938.465342283974\\
66.5	0.2211	-984.253391099856	-984.253391099856\\
66.5	0.22476	-1031.26105854802	-1031.26105854802\\
66.5	0.22842	-1079.48834462847	-1079.48834462847\\
66.5	0.23208	-1128.93524934121	-1128.93524934121\\
66.5	0.23574	-1179.60177268623	-1179.60177268623\\
66.5	0.2394	-1231.48791466353	-1231.48791466353\\
66.5	0.24306	-1284.59367527312	-1284.59367527312\\
66.5	0.24672	-1338.91905451499	-1338.91905451499\\
66.5	0.25038	-1394.46405238914	-1394.46405238914\\
66.5	0.25404	-1451.22866889558	-1451.22866889558\\
66.5	0.2577	-1509.2129040343	-1509.2129040343\\
66.5	0.26136	-1568.41675780531	-1568.41675780531\\
66.5	0.26502	-1628.8402302086	-1628.8402302086\\
66.5	0.26868	-1690.48332124417	-1690.48332124417\\
66.5	0.27234	-1753.34603091203	-1753.34603091203\\
66.5	0.276	-1817.42835921218	-1817.42835921218\\
66.875	0.093	-109.056711418985	-109.056711418985\\
66.875	0.09666	-113.407228541429	-113.407228541429\\
66.875	0.10032	-118.977364296157	-118.977364296157\\
66.875	0.10398	-125.767118683169	-125.767118683169\\
66.875	0.10764	-133.776491702465	-133.776491702465\\
66.875	0.1113	-143.005483354046	-143.005483354046\\
66.875	0.11496	-153.454093637909	-153.454093637909\\
66.875	0.11862	-165.122322554058	-165.122322554058\\
66.875	0.12228	-178.01017010249	-178.01017010249\\
66.875	0.12594	-192.117636283206	-192.117636283206\\
66.875	0.1296	-207.444721096206	-207.444721096206\\
66.875	0.13326	-223.99142454149	-223.99142454149\\
66.875	0.13692	-241.757746619059	-241.757746619059\\
66.875	0.14058	-260.743687328912	-260.743687328912\\
66.875	0.14424	-280.949246671048	-280.949246671048\\
66.875	0.1479	-302.374424645468	-302.374424645468\\
66.875	0.15156	-325.019221252172	-325.019221252172\\
66.875	0.15522	-348.883636491161	-348.883636491161\\
66.875	0.15888	-373.967670362434	-373.967670362434\\
66.875	0.16254	-400.27132286599	-400.27132286599\\
66.875	0.1662	-427.794594001831	-427.794594001831\\
66.875	0.16986	-456.537483769956	-456.537483769956\\
66.875	0.17352	-486.499992170364	-486.499992170364\\
66.875	0.17718	-517.682119203057	-517.682119203057\\
66.875	0.18084	-550.083864868033	-550.083864868033\\
66.875	0.1845	-583.705229165294	-583.705229165294\\
66.875	0.18816	-618.546212094839	-618.546212094839\\
66.875	0.19182	-654.606813656668	-654.606813656668\\
66.875	0.19548	-691.887033850781	-691.887033850781\\
66.875	0.19914	-730.386872677178	-730.386872677178\\
66.875	0.2028	-770.106330135859	-770.106330135859\\
66.875	0.20646	-811.045406226824	-811.045406226824\\
66.875	0.21012	-853.204100950073	-853.204100950073\\
66.875	0.21378	-896.582414305605	-896.582414305605\\
66.875	0.21744	-941.180346293422	-941.180346293422\\
66.875	0.2211	-986.997896913524	-986.997896913524\\
66.875	0.22476	-1034.03506616591	-1034.03506616591\\
66.875	0.22842	-1082.29185405058	-1082.29185405058\\
66.875	0.23208	-1131.76826056753	-1131.76826056753\\
66.875	0.23574	-1182.46428571677	-1182.46428571677\\
66.875	0.2394	-1234.37992949829	-1234.37992949829\\
66.875	0.24306	-1287.51519191209	-1287.51519191209\\
66.875	0.24672	-1341.87007295818	-1341.87007295818\\
66.875	0.25038	-1397.44457263656	-1397.44457263656\\
66.875	0.25404	-1454.23869094721	-1454.23869094721\\
66.875	0.2577	-1512.25242789016	-1512.25242789016\\
66.875	0.26136	-1571.48578346538	-1571.48578346538\\
66.875	0.26502	-1631.93875767289	-1631.93875767289\\
66.875	0.26868	-1693.61135051268	-1693.61135051268\\
66.875	0.27234	-1756.50356198476	-1756.50356198476\\
66.875	0.276	-1820.61539208912	-1820.61539208912\\
67.25	0.093	-110.90725744967	-110.90725744967\\
67.25	0.09666	-115.287276376333	-115.287276376333\\
67.25	0.10032	-120.886913935279	-120.886913935279\\
67.25	0.10398	-127.706170126511	-127.706170126511\\
67.25	0.10764	-135.745044950025	-135.745044950025\\
67.25	0.1113	-145.003538405824	-145.003538405824\\
67.25	0.11496	-155.481650493906	-155.481650493906\\
67.25	0.11862	-167.179381214274	-167.179381214274\\
67.25	0.12228	-180.096730566924	-180.096730566924\\
67.25	0.12594	-194.23369855186	-194.23369855186\\
67.25	0.1296	-209.590285169079	-209.590285169079\\
67.25	0.13326	-226.166490418582	-226.166490418582\\
67.25	0.13692	-243.962314300369	-243.962314300369\\
67.25	0.14058	-262.97775681444	-262.97775681444\\
67.25	0.14424	-283.212817960795	-283.212817960795\\
67.25	0.1479	-304.667497739434	-304.667497739434\\
67.25	0.15156	-327.341796150357	-327.341796150357\\
67.25	0.15522	-351.235713193565	-351.235713193565\\
67.25	0.15888	-376.349248869056	-376.349248869056\\
67.25	0.16254	-402.682403176831	-402.682403176831\\
67.25	0.1662	-430.23517611689	-430.23517611689\\
67.25	0.16986	-459.007567689234	-459.007567689234\\
67.25	0.17352	-488.999577893861	-488.999577893861\\
67.25	0.17718	-520.211206730773	-520.211206730773\\
67.25	0.18084	-552.642454199967	-552.642454199967\\
67.25	0.1845	-586.293320301448	-586.293320301448\\
67.25	0.18816	-621.163805035211	-621.163805035211\\
67.25	0.19182	-657.253908401258	-657.253908401258\\
67.25	0.19548	-694.56363039959	-694.56363039959\\
67.25	0.19914	-733.092971030206	-733.092971030206\\
67.25	0.2028	-772.841930293106	-772.841930293106\\
67.25	0.20646	-813.81050818829	-813.81050818829\\
67.25	0.21012	-855.998704715757	-855.998704715757\\
67.25	0.21378	-899.406519875508	-899.406519875508\\
67.25	0.21744	-944.033953667544	-944.033953667544\\
67.25	0.2211	-989.881006091864	-989.881006091864\\
67.25	0.22476	-1036.94767714847	-1036.94767714847\\
67.25	0.22842	-1085.23396683736	-1085.23396683736\\
67.25	0.23208	-1134.73987515853	-1134.73987515853\\
67.25	0.23574	-1185.46540211198	-1185.46540211198\\
67.25	0.2394	-1237.41054769772	-1237.41054769772\\
67.25	0.24306	-1290.57531191575	-1290.57531191575\\
67.25	0.24672	-1344.95969476606	-1344.95969476606\\
67.25	0.25038	-1400.56369624865	-1400.56369624865\\
67.25	0.25404	-1457.38731636352	-1457.38731636352\\
67.25	0.2577	-1515.43055511068	-1515.43055511068\\
67.25	0.26136	-1574.69341249013	-1574.69341249013\\
67.25	0.26502	-1635.17588850186	-1635.17588850186\\
67.25	0.26868	-1696.87798314587	-1696.87798314587\\
67.25	0.27234	-1759.79969642216	-1759.79969642216\\
67.25	0.276	-1823.94102833074	-1823.94102833074\\
67.625	0.093	-112.896406845029	-112.896406845029\\
67.625	0.09666	-117.305927575911	-117.305927575911\\
67.625	0.10032	-122.935066939076	-122.935066939076\\
67.625	0.10398	-129.783824934526	-129.783824934526\\
67.625	0.10764	-137.852201562259	-137.852201562259\\
67.625	0.1113	-147.140196822277	-147.140196822277\\
67.625	0.11496	-157.647810714578	-157.647810714578\\
67.625	0.11862	-169.375043239164	-169.375043239164\\
67.625	0.12228	-182.321894396033	-182.321894396033\\
67.625	0.12594	-196.488364185188	-196.488364185188\\
67.625	0.1296	-211.874452606625	-211.874452606625\\
67.625	0.13326	-228.480159660347	-228.480159660347\\
67.625	0.13692	-246.305485346352	-246.305485346352\\
67.625	0.14058	-265.350429664643	-265.350429664643\\
67.625	0.14424	-285.614992615216	-285.614992615216\\
67.625	0.1479	-307.099174198074	-307.099174198074\\
67.625	0.15156	-329.802974413216	-329.802974413216\\
67.625	0.15522	-353.726393260642	-353.726393260642\\
67.625	0.15888	-378.869430740352	-378.869430740352\\
67.625	0.16254	-405.232086852346	-405.232086852346\\
67.625	0.1662	-432.814361596624	-432.814361596624\\
67.625	0.16986	-461.616254973186	-461.616254973186\\
67.625	0.17352	-491.637766982032	-491.637766982032\\
67.625	0.17718	-522.878897623162	-522.878897623162\\
67.625	0.18084	-555.339646896576	-555.339646896576\\
67.625	0.1845	-589.020014802275	-589.020014802275\\
67.625	0.18816	-623.920001340257	-623.920001340257\\
67.625	0.19182	-660.039606510523	-660.039606510523\\
67.625	0.19548	-697.378830313073	-697.378830313073\\
67.625	0.19914	-735.937672747908	-735.937672747908\\
67.625	0.2028	-775.716133815026	-775.716133815026\\
67.625	0.20646	-816.714213514429	-816.714213514429\\
67.625	0.21012	-858.931911846115	-858.931911846115\\
67.625	0.21378	-902.369228810085	-902.369228810085\\
67.625	0.21744	-947.026164406339	-947.026164406339\\
67.625	0.2211	-992.902718634878	-992.902718634878\\
67.625	0.22476	-1039.9988914957	-1039.9988914957\\
67.625	0.22842	-1088.31468298881	-1088.31468298881\\
67.625	0.23208	-1137.8500931142	-1137.8500931142\\
67.625	0.23574	-1188.60512187187	-1188.60512187187\\
67.625	0.2394	-1240.57976926183	-1240.57976926183\\
67.625	0.24306	-1293.77403528407	-1293.77403528407\\
67.625	0.24672	-1348.1879199386	-1348.1879199386\\
67.625	0.25038	-1403.82142322541	-1403.82142322541\\
67.625	0.25404	-1460.67454514451	-1460.67454514451\\
67.625	0.2577	-1518.74728569588	-1518.74728569588\\
67.625	0.26136	-1578.03964487955	-1578.03964487955\\
67.625	0.26502	-1638.5516226955	-1638.5516226955\\
67.625	0.26868	-1700.28321914373	-1700.28321914373\\
67.625	0.27234	-1763.23443422424	-1763.23443422424\\
67.625	0.276	-1827.40526793704	-1827.40526793704\\
68	0.093	-115.024159605061	-115.024159605061\\
68	0.09666	-119.463182140161	-119.463182140161\\
68	0.10032	-125.121823307545	-125.121823307545\\
68	0.10398	-132.000083107214	-132.000083107214\\
68	0.10764	-140.097961539166	-140.097961539166\\
68	0.1113	-149.415458603403	-149.415458603403\\
68	0.11496	-159.952574299922	-159.952574299922\\
68	0.11862	-171.709308628727	-171.709308628727\\
68	0.12228	-184.685661589815	-184.685661589815\\
68	0.12594	-198.881633183188	-198.881633183188\\
68	0.1296	-214.297223408844	-214.297223408844\\
68	0.13326	-230.932432266785	-230.932432266785\\
68	0.13692	-248.787259757009	-248.787259757009\\
68	0.14058	-267.861705879518	-267.861705879518\\
68	0.14424	-288.15577063431	-288.15577063431\\
68	0.1479	-309.669454021387	-309.669454021387\\
68	0.15156	-332.402756040747	-332.402756040747\\
68	0.15522	-356.355676692392	-356.355676692392\\
68	0.15888	-381.528215976321	-381.528215976321\\
68	0.16254	-407.920373892533	-407.920373892533\\
68	0.1662	-435.53215044103	-435.53215044103\\
68	0.16986	-464.363545621811	-464.363545621811\\
68	0.17352	-494.414559434875	-494.414559434875\\
68	0.17718	-525.685191880224	-525.685191880224\\
68	0.18084	-558.175442957857	-558.175442957857\\
68	0.1845	-591.885312667775	-591.885312667775\\
68	0.18816	-626.814801009975	-626.814801009975\\
68	0.19182	-662.96390798446	-662.96390798446\\
68	0.19548	-700.332633591229	-700.332633591229\\
68	0.19914	-738.920977830282	-738.920977830282\\
68	0.2028	-778.72894070162	-778.72894070162\\
68	0.20646	-819.756522205241	-819.756522205241\\
68	0.21012	-862.003722341145	-862.003722341145\\
68	0.21378	-905.470541109335	-905.470541109335\\
68	0.21744	-950.156978509808	-950.156978509808\\
68	0.2211	-996.063034542565	-996.063034542565\\
68	0.22476	-1043.18870920761	-1043.18870920761\\
68	0.22842	-1091.53400250493	-1091.53400250493\\
68	0.23208	-1141.09891443454	-1141.09891443454\\
68	0.23574	-1191.88344499643	-1191.88344499643\\
68	0.2394	-1243.88759419061	-1243.88759419061\\
68	0.24306	-1297.11136201707	-1297.11136201707\\
68	0.24672	-1351.55474847582	-1351.55474847582\\
68	0.25038	-1407.21775356685	-1407.21775356685\\
68	0.25404	-1464.10037729016	-1464.10037729016\\
68	0.2577	-1522.20261964576	-1522.20261964576\\
68	0.26136	-1581.52448063364	-1581.52448063364\\
68	0.26502	-1642.06596025381	-1642.06596025381\\
68	0.26868	-1703.82705850626	-1703.82705850626\\
68	0.27234	-1766.80777539099	-1766.80777539099\\
68	0.276	-1831.00811090801	-1831.00811090801\\
68.375	0.093	-117.290515729766	-117.290515729766\\
68.375	0.09666	-121.759040069085	-121.759040069085\\
68.375	0.10032	-127.447183040688	-127.447183040688\\
68.375	0.10398	-134.354944644575	-134.354944644575\\
68.375	0.10764	-142.482324880746	-142.482324880746\\
68.375	0.1113	-151.829323749201	-151.829323749201\\
68.375	0.11496	-162.39594124994	-162.39594124994\\
68.375	0.11862	-174.182177382963	-174.182177382963\\
68.375	0.12228	-187.18803214827	-187.18803214827\\
68.375	0.12594	-201.413505545861	-201.413505545861\\
68.375	0.1296	-216.858597575736	-216.858597575736\\
68.375	0.13326	-233.523308237895	-233.523308237895\\
68.375	0.13692	-251.407637532338	-251.407637532338\\
68.375	0.14058	-270.511585459066	-270.511585459066\\
68.375	0.14424	-290.835152018077	-290.835152018077\\
68.375	0.1479	-312.378337209372	-312.378337209372\\
68.375	0.15156	-335.141141032951	-335.141141032951\\
68.375	0.15522	-359.123563488815	-359.123563488815\\
68.375	0.15888	-384.325604576962	-384.325604576962\\
68.375	0.16254	-410.747264297394	-410.747264297394\\
68.375	0.1662	-438.388542650109	-438.388542650109\\
68.375	0.16986	-467.249439635109	-467.249439635109\\
68.375	0.17352	-497.329955252392	-497.329955252392\\
68.375	0.17718	-528.63008950196	-528.63008950196\\
68.375	0.18084	-561.149842383811	-561.149842383811\\
68.375	0.1845	-594.889213897947	-594.889213897947\\
68.375	0.18816	-629.848204044367	-629.848204044367\\
68.375	0.19182	-666.02681282307	-666.02681282307\\
68.375	0.19548	-703.425040234058	-703.425040234058\\
68.375	0.19914	-742.042886277331	-742.042886277331\\
68.375	0.2028	-781.880350952886	-781.880350952886\\
68.375	0.20646	-822.937434260726	-822.937434260726\\
68.375	0.21012	-865.21413620085	-865.21413620085\\
68.375	0.21378	-908.710456773257	-908.710456773257\\
68.375	0.21744	-953.426395977949	-953.426395977949\\
68.375	0.2211	-999.361953814925	-999.361953814925\\
68.375	0.22476	-1046.51713028418	-1046.51713028418\\
68.375	0.22842	-1094.89192538573	-1094.89192538573\\
68.375	0.23208	-1144.48633911956	-1144.48633911956\\
68.375	0.23574	-1195.30037148567	-1195.30037148567\\
68.375	0.2394	-1247.33402248407	-1247.33402248407\\
68.375	0.24306	-1300.58729211475	-1300.58729211475\\
68.375	0.24672	-1355.06018037771	-1355.06018037771\\
68.375	0.25038	-1410.75268727296	-1410.75268727296\\
68.375	0.25404	-1467.66481280049	-1467.66481280049\\
68.375	0.2577	-1525.79655696031	-1525.79655696031\\
68.375	0.26136	-1585.14791975241	-1585.14791975241\\
68.375	0.26502	-1645.71890117679	-1645.71890117679\\
68.375	0.26868	-1707.50950123346	-1707.50950123346\\
68.375	0.27234	-1770.51971992241	-1770.51971992241\\
68.375	0.276	-1834.74955724365	-1834.74955724365\\
68.75	0.093	-119.695475219144	-119.695475219144\\
68.75	0.09666	-124.193501362682	-124.193501362682\\
68.75	0.10032	-129.911146138504	-129.911146138504\\
68.75	0.10398	-136.848409546609	-136.848409546609\\
68.75	0.10764	-145.005291586999	-145.005291586999\\
68.75	0.1113	-154.381792259673	-154.381792259673\\
68.75	0.11496	-164.97791156463	-164.97791156463\\
68.75	0.11862	-176.793649501872	-176.793649501872\\
68.75	0.12228	-189.829006071398	-189.829006071398\\
68.75	0.12594	-204.083981273208	-204.083981273208\\
68.75	0.1296	-219.558575107302	-219.558575107302\\
68.75	0.13326	-236.252787573679	-236.252787573679\\
68.75	0.13692	-254.166618672341	-254.166618672341\\
68.75	0.14058	-273.300068403287	-273.300068403287\\
68.75	0.14424	-293.653136766517	-293.653136766517\\
68.75	0.1479	-315.225823762032	-315.225823762032\\
68.75	0.15156	-338.018129389829	-338.018129389829\\
68.75	0.15522	-362.030053649911	-362.030053649911\\
68.75	0.15888	-387.261596542278	-387.261596542278\\
68.75	0.16254	-413.712758066928	-413.712758066928\\
68.75	0.1662	-441.383538223862	-441.383538223862\\
68.75	0.16986	-470.273937013081	-470.273937013081\\
68.75	0.17352	-500.383954434582	-500.383954434582\\
68.75	0.17718	-531.713590488369	-531.713590488369\\
68.75	0.18084	-564.262845174438	-564.262845174438\\
68.75	0.1845	-598.031718492794	-598.031718492794\\
68.75	0.18816	-633.020210443432	-633.020210443432\\
68.75	0.19182	-669.228321026354	-669.228321026354\\
68.75	0.19548	-706.65605024156	-706.65605024156\\
68.75	0.19914	-745.303398089051	-745.303398089051\\
68.75	0.2028	-785.170364568826	-785.170364568826\\
68.75	0.20646	-826.256949680884	-826.256949680884\\
68.75	0.21012	-868.563153425227	-868.563153425227\\
68.75	0.21378	-912.088975801853	-912.088975801853\\
68.75	0.21744	-956.834416810764	-956.834416810764\\
68.75	0.2211	-1002.79947645196	-1002.79947645196\\
68.75	0.22476	-1049.98415472544	-1049.98415472544\\
68.75	0.22842	-1098.3884516312	-1098.3884516312\\
68.75	0.23208	-1148.01236716925	-1148.01236716925\\
68.75	0.23574	-1198.85590133958	-1198.85590133958\\
68.75	0.2394	-1250.91905414219	-1250.91905414219\\
68.75	0.24306	-1304.20182557709	-1304.20182557709\\
68.75	0.24672	-1358.70421564427	-1358.70421564427\\
68.75	0.25038	-1414.42622434374	-1414.42622434374\\
68.75	0.25404	-1471.36785167549	-1471.36785167549\\
68.75	0.2577	-1529.52909763953	-1529.52909763953\\
68.75	0.26136	-1588.90996223585	-1588.90996223585\\
68.75	0.26502	-1649.51044546445	-1649.51044546445\\
68.75	0.26868	-1711.33054732534	-1711.33054732534\\
68.75	0.27234	-1774.37026781851	-1774.37026781851\\
68.75	0.276	-1838.62960694396	-1838.62960694396\\
69.125	0.093	-122.239038073197	-122.239038073197\\
69.125	0.09666	-126.766566020954	-126.766566020954\\
69.125	0.10032	-132.513712600993	-132.513712600993\\
69.125	0.10398	-139.480477813318	-139.480477813318\\
69.125	0.10764	-147.666861657927	-147.666861657927\\
69.125	0.1113	-157.072864134819	-157.072864134819\\
69.125	0.11496	-167.698485243995	-167.698485243995\\
69.125	0.11862	-179.543724985456	-179.543724985456\\
69.125	0.12228	-192.6085833592	-192.6085833592\\
69.125	0.12594	-206.893060365229	-206.893060365229\\
69.125	0.1296	-222.397156003541	-222.397156003541\\
69.125	0.13326	-239.120870274138	-239.120870274138\\
69.125	0.13692	-257.064203177018	-257.064203177018\\
69.125	0.14058	-276.227154712184	-276.227154712184\\
69.125	0.14424	-296.609724879632	-296.609724879632\\
69.125	0.1479	-318.211913679365	-318.211913679365\\
69.125	0.15156	-341.033721111381	-341.033721111381\\
69.125	0.15522	-365.075147175682	-365.075147175682\\
69.125	0.15888	-390.336191872267	-390.336191872267\\
69.125	0.16254	-416.816855201136	-416.816855201136\\
69.125	0.1662	-444.517137162289	-444.517137162289\\
69.125	0.16986	-473.437037755726	-473.437037755726\\
69.125	0.17352	-503.576556981446	-503.576556981446\\
69.125	0.17718	-534.935694839452	-534.935694839452\\
69.125	0.18084	-567.51445132974	-567.51445132974\\
69.125	0.1845	-601.312826452314	-601.312826452314\\
69.125	0.18816	-636.330820207171	-636.330820207171\\
69.125	0.19182	-672.568432594312	-672.568432594312\\
69.125	0.19548	-710.025663613737	-710.025663613737\\
69.125	0.19914	-748.702513265447	-748.702513265447\\
69.125	0.2028	-788.59898154944	-788.59898154944\\
69.125	0.20646	-829.715068465717	-829.715068465717\\
69.125	0.21012	-872.050774014278	-872.050774014278\\
69.125	0.21378	-915.606098195123	-915.606098195123\\
69.125	0.21744	-960.381041008252	-960.381041008252\\
69.125	0.2211	-1006.37560245367	-1006.37560245367\\
69.125	0.22476	-1053.58978253136	-1053.58978253136\\
69.125	0.22842	-1102.02358124134	-1102.02358124134\\
69.125	0.23208	-1151.67699858361	-1151.67699858361\\
69.125	0.23574	-1202.55003455816	-1202.55003455816\\
69.125	0.2394	-1254.64268916499	-1254.64268916499\\
69.125	0.24306	-1307.95496240411	-1307.95496240411\\
69.125	0.24672	-1362.48685427551	-1362.48685427551\\
69.125	0.25038	-1418.2383647792	-1418.2383647792\\
69.125	0.25404	-1475.20949391517	-1475.20949391517\\
69.125	0.2577	-1533.40024168342	-1533.40024168342\\
69.125	0.26136	-1592.81060808396	-1592.81060808396\\
69.125	0.26502	-1653.44059311678	-1653.44059311678\\
69.125	0.26868	-1715.29019678189	-1715.29019678189\\
69.125	0.27234	-1778.35941907928	-1778.35941907928\\
69.125	0.276	-1842.64826000895	-1842.64826000895\\
69.5	0.093	-124.921204291922	-124.921204291922\\
69.5	0.09666	-129.478234043897	-129.478234043897\\
69.5	0.10032	-135.254882428156	-135.254882428156\\
69.5	0.10398	-142.2511494447	-142.2511494447\\
69.5	0.10764	-150.467035093527	-150.467035093527\\
69.5	0.1113	-159.902539374638	-159.902539374638\\
69.5	0.11496	-170.557662288033	-170.557662288033\\
69.5	0.11862	-182.432403833712	-182.432403833712\\
69.5	0.12228	-195.526764011675	-195.526764011675\\
69.5	0.12594	-209.840742821923	-209.840742821923\\
69.5	0.1296	-225.374340264454	-225.374340264454\\
69.5	0.13326	-242.127556339269	-242.127556339269\\
69.5	0.13692	-260.100391046369	-260.100391046369\\
69.5	0.14058	-279.292844385752	-279.292844385752\\
69.5	0.14424	-299.704916357419	-299.704916357419\\
69.5	0.1479	-321.336606961371	-321.336606961371\\
69.5	0.15156	-344.187916197606	-344.187916197606\\
69.5	0.15522	-368.258844066126	-368.258844066126\\
69.5	0.15888	-393.549390566929	-393.549390566929\\
69.5	0.16254	-420.059555700017	-420.059555700017\\
69.5	0.1662	-447.789339465388	-447.789339465388\\
69.5	0.16986	-476.738741863045	-476.738741863045\\
69.5	0.17352	-506.907762892984	-506.907762892984\\
69.5	0.17718	-538.296402555208	-538.296402555208\\
69.5	0.18084	-570.904660849715	-570.904660849715\\
69.5	0.1845	-604.732537776507	-604.732537776507\\
69.5	0.18816	-639.780033335583	-639.780033335583\\
69.5	0.19182	-676.047147526942	-676.047147526942\\
69.5	0.19548	-713.533880350586	-713.533880350586\\
69.5	0.19914	-752.240231806515	-752.240231806515\\
69.5	0.2028	-792.166201894727	-792.166201894727\\
69.5	0.20646	-833.311790615223	-833.311790615223\\
69.5	0.21012	-875.676997968003	-875.676997968003\\
69.5	0.21378	-919.261823953066	-919.261823953066\\
69.5	0.21744	-964.066268570414	-964.066268570414\\
69.5	0.2211	-1010.09033182005	-1010.09033182005\\
69.5	0.22476	-1057.33401370196	-1057.33401370196\\
69.5	0.22842	-1105.79731421616	-1105.79731421616\\
69.5	0.23208	-1155.48023336265	-1155.48023336265\\
69.5	0.23574	-1206.38277114142	-1206.38277114142\\
69.5	0.2394	-1258.50492755247	-1258.50492755247\\
69.5	0.24306	-1311.8467025958	-1311.8467025958\\
69.5	0.24672	-1366.40809627142	-1366.40809627142\\
69.5	0.25038	-1422.18910857933	-1422.18910857933\\
69.5	0.25404	-1479.18973951952	-1479.18973951952\\
69.5	0.2577	-1537.40998909199	-1537.40998909199\\
69.5	0.26136	-1596.84985729675	-1596.84985729675\\
69.5	0.26502	-1657.50934413379	-1657.50934413379\\
69.5	0.26868	-1719.38844960311	-1719.38844960311\\
69.5	0.27234	-1782.48717370472	-1782.48717370472\\
69.5	0.276	-1846.80551643861	-1846.80551643861\\
69.875	0.093	-127.741973875321	-127.741973875321\\
69.875	0.09666	-132.328505431515	-132.328505431515\\
69.875	0.10032	-138.134655619992	-138.134655619992\\
69.875	0.10398	-145.160424440754	-145.160424440754\\
69.875	0.10764	-153.4058118938	-153.4058118938\\
69.875	0.1113	-162.87081797913	-162.87081797913\\
69.875	0.11496	-173.555442696744	-173.555442696744\\
69.875	0.11862	-185.459686046642	-185.459686046642\\
69.875	0.12228	-198.583548028823	-198.583548028823\\
69.875	0.12594	-212.92702864329	-212.92702864329\\
69.875	0.1296	-228.49012789004	-228.49012789004\\
69.875	0.13326	-245.272845769073	-245.272845769073\\
69.875	0.13692	-263.275182280392	-263.275182280392\\
69.875	0.14058	-282.497137423994	-282.497137423994\\
69.875	0.14424	-302.93871119988	-302.93871119988\\
69.875	0.1479	-324.59990360805	-324.59990360805\\
69.875	0.15156	-347.480714648504	-347.480714648504\\
69.875	0.15522	-371.581144321243	-371.581144321243\\
69.875	0.15888	-396.901192626265	-396.901192626265\\
69.875	0.16254	-423.440859563571	-423.440859563571\\
69.875	0.1662	-451.200145133162	-451.200145133162\\
69.875	0.16986	-480.179049335036	-480.179049335036\\
69.875	0.17352	-510.377572169194	-510.377572169194\\
69.875	0.17718	-541.795713635637	-541.795713635637\\
69.875	0.18084	-574.433473734363	-574.433473734363\\
69.875	0.1845	-608.290852465374	-608.290852465374\\
69.875	0.18816	-643.367849828668	-643.367849828668\\
69.875	0.19182	-679.664465824246	-679.664465824246\\
69.875	0.19548	-717.180700452109	-717.180700452109\\
69.875	0.19914	-755.916553712256	-755.916553712256\\
69.875	0.2028	-795.872025604687	-795.872025604687\\
69.875	0.20646	-837.047116129401	-837.047116129401\\
69.875	0.21012	-879.4418252864	-879.4418252864\\
69.875	0.21378	-923.056153075682	-923.056153075682\\
69.875	0.21744	-967.890099497249	-967.890099497249\\
69.875	0.2211	-1013.9436645511	-1013.9436645511\\
69.875	0.22476	-1061.21684823723	-1061.21684823723\\
69.875	0.22842	-1109.70965055565	-1109.70965055565\\
69.875	0.23208	-1159.42207150636	-1159.42207150636\\
69.875	0.23574	-1210.35411108934	-1210.35411108934\\
69.875	0.2394	-1262.50576930461	-1262.50576930461\\
69.875	0.24306	-1315.87704615217	-1315.87704615217\\
69.875	0.24672	-1370.46794163201	-1370.46794163201\\
69.875	0.25038	-1426.27845574413	-1426.27845574413\\
69.875	0.25404	-1483.30858848854	-1483.30858848854\\
69.875	0.2577	-1541.55833986523	-1541.55833986523\\
69.875	0.26136	-1601.02770987421	-1601.02770987421\\
69.875	0.26502	-1661.71669851546	-1661.71669851546\\
69.875	0.26868	-1723.62530578901	-1723.62530578901\\
69.875	0.27234	-1786.75353169483	-1786.75353169483\\
69.875	0.276	-1851.10137623295	-1851.10137623295\\
70.25	0.093	-130.701346823393	-130.701346823393\\
70.25	0.09666	-135.317380183805	-135.317380183805\\
70.25	0.10032	-141.153032176501	-141.153032176501\\
70.25	0.10398	-148.208302801482	-148.208302801482\\
70.25	0.10764	-156.483192058746	-156.483192058746\\
70.25	0.1113	-165.977699948295	-165.977699948295\\
70.25	0.11496	-176.691826470127	-176.691826470127\\
70.25	0.11862	-188.625571624245	-188.625571624245\\
70.25	0.12228	-201.778935410645	-201.778935410645\\
70.25	0.12594	-216.15191782933	-216.15191782933\\
70.25	0.1296	-231.744518880298	-231.744518880298\\
70.25	0.13326	-248.556738563551	-248.556738563551\\
70.25	0.13692	-266.588576879088	-266.588576879088\\
70.25	0.14058	-285.840033826909	-285.840033826909\\
70.25	0.14424	-306.311109407013	-306.311109407013\\
70.25	0.1479	-328.001803619403	-328.001803619403\\
70.25	0.15156	-350.912116464075	-350.912116464075\\
70.25	0.15522	-375.042047941032	-375.042047941032\\
70.25	0.15888	-400.391598050273	-400.391598050273\\
70.25	0.16254	-426.960766791798	-426.960766791798\\
70.25	0.1662	-454.749554165608	-454.749554165608\\
70.25	0.16986	-483.757960171701	-483.757960171701\\
70.25	0.17352	-513.985984810077	-513.985984810077\\
70.25	0.17718	-545.433628080739	-545.433628080739\\
70.25	0.18084	-578.100889983684	-578.100889983684\\
70.25	0.1845	-611.987770518913	-611.987770518913\\
70.25	0.18816	-647.094269686427	-647.094269686427\\
70.25	0.19182	-683.420387486223	-683.420387486223\\
70.25	0.19548	-720.966123918305	-720.966123918305\\
70.25	0.19914	-759.731478982671	-759.731478982671\\
70.25	0.2028	-799.71645267932	-799.71645267932\\
70.25	0.20646	-840.921045008253	-840.921045008253\\
70.25	0.21012	-883.345255969471	-883.345255969471\\
70.25	0.21378	-926.989085562971	-926.989085562971\\
70.25	0.21744	-971.852533788757	-971.852533788757\\
70.25	0.2211	-1017.93560064683	-1017.93560064683\\
70.25	0.22476	-1065.23828613718	-1065.23828613718\\
70.25	0.22842	-1113.76059025982	-1113.76059025982\\
70.25	0.23208	-1163.50251301474	-1163.50251301474\\
70.25	0.23574	-1214.46405440194	-1214.46405440194\\
70.25	0.2394	-1266.64521442143	-1266.64521442143\\
70.25	0.24306	-1320.04599307321	-1320.04599307321\\
70.25	0.24672	-1374.66639035727	-1374.66639035727\\
70.25	0.25038	-1430.50640627361	-1430.50640627361\\
70.25	0.25404	-1487.56604082223	-1487.56604082223\\
70.25	0.2577	-1545.84529400314	-1545.84529400314\\
70.25	0.26136	-1605.34416581634	-1605.34416581634\\
70.25	0.26502	-1666.06265626181	-1666.06265626181\\
70.25	0.26868	-1728.00076533958	-1728.00076533958\\
70.25	0.27234	-1791.15849304962	-1791.15849304962\\
70.25	0.276	-1855.53583939195	-1855.53583939195\\
70.625	0.093	-133.799323136138	-133.799323136138\\
70.625	0.09666	-138.444858300769	-138.444858300769\\
70.625	0.10032	-144.310012097684	-144.310012097684\\
70.625	0.10398	-151.394784526883	-151.394784526883\\
70.625	0.10764	-159.699175588367	-159.699175588367\\
70.625	0.1113	-169.223185282134	-169.223185282134\\
70.625	0.11496	-179.966813608185	-179.966813608185\\
70.625	0.11862	-191.930060566521	-191.930060566521\\
70.625	0.12228	-205.11292615714	-205.11292615714\\
70.625	0.12594	-219.515410380043	-219.515410380043\\
70.625	0.1296	-235.137513235231	-235.137513235231\\
70.625	0.13326	-251.979234722702	-251.979234722702\\
70.625	0.13692	-270.040574842458	-270.040574842458\\
70.625	0.14058	-289.321533594498	-289.321533594498\\
70.625	0.14424	-309.822110978821	-309.822110978821\\
70.625	0.1479	-331.542306995429	-331.542306995429\\
70.625	0.15156	-354.48212164432	-354.48212164432\\
70.625	0.15522	-378.641554925496	-378.641554925496\\
70.625	0.15888	-404.020606838955	-404.020606838955\\
70.625	0.16254	-430.619277384699	-430.619277384699\\
70.625	0.1662	-458.437566562727	-458.437566562727\\
70.625	0.16986	-487.475474373039	-487.475474373039\\
70.625	0.17352	-517.733000815634	-517.733000815634\\
70.625	0.17718	-549.210145890514	-549.210145890514\\
70.625	0.18084	-581.906909597678	-581.906909597678\\
70.625	0.1845	-615.823291937127	-615.823291937127\\
70.625	0.18816	-650.959292908858	-650.959292908858\\
70.625	0.19182	-687.314912512874	-687.314912512874\\
70.625	0.19548	-724.890150749174	-724.890150749174\\
70.625	0.19914	-763.685007617758	-763.685007617758\\
70.625	0.2028	-803.699483118627	-803.699483118627\\
70.625	0.20646	-844.933577251779	-844.933577251779\\
70.625	0.21012	-887.387290017215	-887.387290017215\\
70.625	0.21378	-931.060621414934	-931.060621414934\\
70.625	0.21744	-975.953571444938	-975.953571444938\\
70.625	0.2211	-1022.06614010723	-1022.06614010723\\
70.625	0.22476	-1069.3983274018	-1069.3983274018\\
70.625	0.22842	-1117.95013332866	-1117.95013332866\\
70.625	0.23208	-1167.7215578878	-1167.7215578878\\
70.625	0.23574	-1218.71260107922	-1218.71260107922\\
70.625	0.2394	-1270.92326290293	-1270.92326290293\\
70.625	0.24306	-1324.35354335892	-1324.35354335892\\
70.625	0.24672	-1379.0034424472	-1379.0034424472\\
70.625	0.25038	-1434.87296016776	-1434.87296016776\\
70.625	0.25404	-1491.9620965206	-1491.9620965206\\
70.625	0.2577	-1550.27085150573	-1550.27085150573\\
70.625	0.26136	-1609.79922512314	-1609.79922512314\\
70.625	0.26502	-1670.54721737284	-1670.54721737284\\
70.625	0.26868	-1732.51482825482	-1732.51482825482\\
70.625	0.27234	-1795.70205776909	-1795.70205776909\\
70.625	0.276	-1860.10890591563	-1860.10890591563\\
71	0.093	-137.035902813557	-137.035902813557\\
71	0.09666	-141.710939782407	-141.710939782407\\
71	0.10032	-147.605595383541	-147.605595383541\\
71	0.10398	-154.719869616959	-154.719869616959\\
71	0.10764	-163.053762482661	-163.053762482661\\
71	0.1113	-172.607273980647	-172.607273980647\\
71	0.11496	-183.380404110917	-183.380404110917\\
71	0.11862	-195.373152873471	-195.373152873471\\
71	0.12228	-208.585520268309	-208.585520268309\\
71	0.12594	-223.017506295432	-223.017506295432\\
71	0.1296	-238.669110954837	-238.669110954837\\
71	0.13326	-255.540334246527	-255.540334246527\\
71	0.13692	-273.631176170501	-273.631176170501\\
71	0.14058	-292.94163672676	-292.94163672676\\
71	0.14424	-313.471715915302	-313.471715915302\\
71	0.1479	-335.221413736129	-335.221413736129\\
71	0.15156	-358.190730189239	-358.190730189239\\
71	0.15522	-382.379665274633	-382.379665274633\\
71	0.15888	-407.788218992312	-407.788218992312\\
71	0.16254	-434.416391342274	-434.416391342274\\
71	0.1662	-462.264182324521	-462.264182324521\\
71	0.16986	-491.331591939051	-491.331591939051\\
71	0.17352	-521.618620185865	-521.618620185865\\
71	0.17718	-553.125267064964	-553.125267064964\\
71	0.18084	-585.851532576346	-585.851532576346\\
71	0.1845	-619.797416720014	-619.797416720014\\
71	0.18816	-654.962919495964	-654.962919495964\\
71	0.19182	-691.348040904199	-691.348040904199\\
71	0.19548	-728.952780944717	-728.952780944717\\
71	0.19914	-767.777139617521	-767.777139617521\\
71	0.2028	-807.821116922607	-807.821116922607\\
71	0.20646	-849.084712859978	-849.084712859978\\
71	0.21012	-891.567927429633	-891.567927429633\\
71	0.21378	-935.270760631571	-935.270760631571\\
71	0.21744	-980.193212465794	-980.193212465794\\
71	0.2211	-1026.3352829323	-1026.3352829323\\
71	0.22476	-1073.69697203109	-1073.69697203109\\
71	0.22842	-1122.27827976217	-1122.27827976217\\
71	0.23208	-1172.07920612553	-1172.07920612553\\
71	0.23574	-1223.09975112117	-1223.09975112117\\
71	0.2394	-1275.3399147491	-1275.3399147491\\
71	0.24306	-1328.79969700931	-1328.79969700931\\
71	0.24672	-1383.4790979018	-1383.4790979018\\
71	0.25038	-1439.37811742658	-1439.37811742658\\
71	0.25404	-1496.49675558365	-1496.49675558365\\
71	0.2577	-1554.83501237299	-1554.83501237299\\
71	0.26136	-1614.39288779462	-1614.39288779462\\
71	0.26502	-1675.17038184854	-1675.17038184854\\
71	0.26868	-1737.16749453474	-1737.16749453474\\
71	0.27234	-1800.38422585322	-1800.38422585322\\
71	0.276	-1864.82057580399	-1864.82057580399\\
71.375	0.093	-140.411085855649	-140.411085855649\\
71.375	0.09666	-145.115624628718	-145.115624628718\\
71.375	0.10032	-151.03978203407	-151.03978203407\\
71.375	0.10398	-158.183558071707	-158.183558071707\\
71.375	0.10764	-166.546952741628	-166.546952741628\\
71.375	0.1113	-176.129966043832	-176.129966043832\\
71.375	0.11496	-186.932597978321	-186.932597978321\\
71.375	0.11862	-198.954848545094	-198.954848545094\\
71.375	0.12228	-212.19671774415	-212.19671774415\\
71.375	0.12594	-226.658205575492	-226.658205575492\\
71.375	0.1296	-242.339312039116	-242.339312039116\\
71.375	0.13326	-259.240037135025	-259.240037135025\\
71.375	0.13692	-277.360380863218	-277.360380863218\\
71.375	0.14058	-296.700343223695	-296.700343223695\\
71.375	0.14424	-317.259924216456	-317.259924216456\\
71.375	0.1479	-339.039123841501	-339.039123841501\\
71.375	0.15156	-362.03794209883	-362.03794209883\\
71.375	0.15522	-386.256378988444	-386.256378988444\\
71.375	0.15888	-411.69443451034	-411.69443451034\\
71.375	0.16254	-438.352108664522	-438.352108664522\\
71.375	0.1662	-466.229401450987	-466.229401450987\\
71.375	0.16986	-495.326312869737	-495.326312869737\\
71.375	0.17352	-525.642842920769	-525.642842920769\\
71.375	0.17718	-557.178991604087	-557.178991604087\\
71.375	0.18084	-589.934758919688	-589.934758919688\\
71.375	0.1845	-623.910144867573	-623.910144867573\\
71.375	0.18816	-659.105149447743	-659.105149447743\\
71.375	0.19182	-695.519772660196	-695.519772660196\\
71.375	0.19548	-733.154014504933	-733.154014504933\\
71.375	0.19914	-772.007874981955	-772.007874981955\\
71.375	0.2028	-812.081354091261	-812.081354091261\\
71.375	0.20646	-853.37445183285	-853.37445183285\\
71.375	0.21012	-895.887168206724	-895.887168206724\\
71.375	0.21378	-939.619503212881	-939.619503212881\\
71.375	0.21744	-984.571456851323	-984.571456851323\\
71.375	0.2211	-1030.74302912205	-1030.74302912205\\
71.375	0.22476	-1078.13422002506	-1078.13422002506\\
71.375	0.22842	-1126.74502956035	-1126.74502956035\\
71.375	0.23208	-1176.57545772793	-1176.57545772793\\
71.375	0.23574	-1227.62550452779	-1227.62550452779\\
71.375	0.2394	-1279.89516995994	-1279.89516995994\\
71.375	0.24306	-1333.38445402437	-1333.38445402437\\
71.375	0.24672	-1388.09335672108	-1388.09335672108\\
71.375	0.25038	-1444.02187805008	-1444.02187805008\\
71.375	0.25404	-1501.17001801136	-1501.17001801136\\
71.375	0.2577	-1559.53777660493	-1559.53777660493\\
71.375	0.26136	-1619.12515383078	-1619.12515383078\\
71.375	0.26502	-1679.93214968891	-1679.93214968891\\
71.375	0.26868	-1741.95876417933	-1741.95876417933\\
71.375	0.27234	-1805.20499730203	-1805.20499730203\\
71.375	0.276	-1869.67084905702	-1869.67084905702\\
71.75	0.093	-143.924872262414	-143.924872262414\\
71.75	0.09666	-148.658912839702	-148.658912839702\\
71.75	0.10032	-154.612572049273	-154.612572049273\\
71.75	0.10398	-161.785849891128	-161.785849891128\\
71.75	0.10764	-170.178746365268	-170.178746365268\\
71.75	0.1113	-179.791261471691	-179.791261471691\\
71.75	0.11496	-190.623395210398	-190.623395210398\\
71.75	0.11862	-202.67514758139	-202.67514758139\\
71.75	0.12228	-215.946518584665	-215.946518584665\\
71.75	0.12594	-230.437508220225	-230.437508220225\\
71.75	0.1296	-246.148116488069	-246.148116488069\\
71.75	0.13326	-263.078343388196	-263.078343388196\\
71.75	0.13692	-281.228188920608	-281.228188920608\\
71.75	0.14058	-300.597653085304	-300.597653085304\\
71.75	0.14424	-321.186735882283	-321.186735882283\\
71.75	0.1479	-342.995437311547	-342.995437311547\\
71.75	0.15156	-366.023757373094	-366.023757373094\\
71.75	0.15522	-390.271696066927	-390.271696066927\\
71.75	0.15888	-415.739253393043	-415.739253393043\\
71.75	0.16254	-442.426429351442	-442.426429351442\\
71.75	0.1662	-470.333223942127	-470.333223942127\\
71.75	0.16986	-499.459637165095	-499.459637165095\\
71.75	0.17352	-529.805669020346	-529.805669020346\\
71.75	0.17718	-561.371319507882	-561.371319507882\\
71.75	0.18084	-594.156588627702	-594.156588627702\\
71.75	0.1845	-628.161476379807	-628.161476379807\\
71.75	0.18816	-663.385982764194	-663.385982764194\\
71.75	0.19182	-699.830107780866	-699.830107780866\\
71.75	0.19548	-737.493851429823	-737.493851429823\\
71.75	0.19914	-776.377213711063	-776.377213711063\\
71.75	0.2028	-816.480194624588	-816.480194624588\\
71.75	0.20646	-857.802794170396	-857.802794170396\\
71.75	0.21012	-900.345012348488	-900.345012348488\\
71.75	0.21378	-944.106849158863	-944.106849158863\\
71.75	0.21744	-989.088304601524	-989.088304601524\\
71.75	0.2211	-1035.28937867647	-1035.28937867647\\
71.75	0.22476	-1082.7100713837	-1082.7100713837\\
71.75	0.22842	-1131.35038272321	-1131.35038272321\\
71.75	0.23208	-1181.21031269501	-1181.21031269501\\
71.75	0.23574	-1232.28986129909	-1232.28986129909\\
71.75	0.2394	-1284.58902853545	-1284.58902853545\\
71.75	0.24306	-1338.1078144041	-1338.1078144041\\
71.75	0.24672	-1392.84621890503	-1392.84621890503\\
71.75	0.25038	-1448.80424203825	-1448.80424203825\\
71.75	0.25404	-1505.98188380375	-1505.98188380375\\
71.75	0.2577	-1564.37914420153	-1564.37914420153\\
71.75	0.26136	-1623.9960232316	-1623.9960232316\\
71.75	0.26502	-1684.83252089396	-1684.83252089396\\
71.75	0.26868	-1746.88863718859	-1746.88863718859\\
71.75	0.27234	-1810.16437211551	-1810.16437211551\\
71.75	0.276	-1874.65972567472	-1874.65972567472\\
72.125	0.093	-147.577262033853	-147.577262033853\\
72.125	0.09666	-152.340804415359	-152.340804415359\\
72.125	0.10032	-158.323965429149	-158.323965429149\\
72.125	0.10398	-165.526745075223	-165.526745075223\\
72.125	0.10764	-173.949143353581	-173.949143353581\\
72.125	0.1113	-183.591160264224	-183.591160264224\\
72.125	0.11496	-194.45279580715	-194.45279580715\\
72.125	0.11862	-206.53404998236	-206.53404998236\\
72.125	0.12228	-219.834922789854	-219.834922789854\\
72.125	0.12594	-234.355414229633	-234.355414229633\\
72.125	0.1296	-250.095524301694	-250.095524301694\\
72.125	0.13326	-267.055253006041	-267.055253006041\\
72.125	0.13692	-285.234600342671	-285.234600342671\\
72.125	0.14058	-304.633566311586	-304.633566311586\\
72.125	0.14424	-325.252150912784	-325.252150912784\\
72.125	0.1479	-347.090354146267	-347.090354146267\\
72.125	0.15156	-370.148176012033	-370.148176012033\\
72.125	0.15522	-394.425616510084	-394.425616510084\\
72.125	0.15888	-419.922675640418	-419.922675640418\\
72.125	0.16254	-446.639353403037	-446.639353403037\\
72.125	0.1662	-474.57564979794	-474.57564979794\\
72.125	0.16986	-503.731564825127	-503.731564825127\\
72.125	0.17352	-534.107098484597	-534.107098484597\\
72.125	0.17718	-565.702250776352	-565.702250776352\\
72.125	0.18084	-598.51702170039	-598.51702170039\\
72.125	0.1845	-632.551411256713	-632.551411256713\\
72.125	0.18816	-667.80541944532	-667.80541944532\\
72.125	0.19182	-704.279046266211	-704.279046266211\\
72.125	0.19548	-741.972291719386	-741.972291719386\\
72.125	0.19914	-780.885155804845	-780.885155804845\\
72.125	0.2028	-821.017638522588	-821.017638522588\\
72.125	0.20646	-862.369739872615	-862.369739872615\\
72.125	0.21012	-904.941459854926	-904.941459854926\\
72.125	0.21378	-948.73279846952	-948.73279846952\\
72.125	0.21744	-993.743755716399	-993.743755716399\\
72.125	0.2211	-1039.97433159556	-1039.97433159556\\
72.125	0.22476	-1087.42452610701	-1087.42452610701\\
72.125	0.22842	-1136.09433925074	-1136.09433925074\\
72.125	0.23208	-1185.98377102676	-1185.98377102676\\
72.125	0.23574	-1237.09282143506	-1237.09282143506\\
72.125	0.2394	-1289.42149047564	-1289.42149047564\\
72.125	0.24306	-1342.96977814851	-1342.96977814851\\
72.125	0.24672	-1397.73768445366	-1397.73768445366\\
72.125	0.25038	-1453.72520939109	-1453.72520939109\\
72.125	0.25404	-1510.93235296081	-1510.93235296081\\
72.125	0.2577	-1569.35911516282	-1569.35911516282\\
72.125	0.26136	-1629.0054959971	-1629.0054959971\\
72.125	0.26502	-1689.87149546367	-1689.87149546367\\
72.125	0.26868	-1751.95711356253	-1751.95711356253\\
72.125	0.27234	-1815.26235029367	-1815.26235029367\\
72.125	0.276	-1879.78720565709	-1879.78720565709\\
72.5	0.093	-151.368255169965	-151.368255169965\\
72.5	0.09666	-156.16129935569	-156.16129935569\\
72.5	0.10032	-162.173962173698	-162.173962173698\\
72.5	0.10398	-169.406243623991	-169.406243623991\\
72.5	0.10764	-177.858143706568	-177.858143706568\\
72.5	0.1113	-187.529662421429	-187.529662421429\\
72.5	0.11496	-198.420799768574	-198.420799768574\\
72.5	0.11862	-210.531555748003	-210.531555748003\\
72.5	0.12228	-223.861930359715	-223.861930359715\\
72.5	0.12594	-238.411923603713	-238.411923603713\\
72.5	0.1296	-254.181535479994	-254.181535479994\\
72.5	0.13326	-271.170765988559	-271.170765988559\\
72.5	0.13692	-289.379615129408	-289.379615129408\\
72.5	0.14058	-308.808082902541	-308.808082902541\\
72.5	0.14424	-329.456169307958	-329.456169307958\\
72.5	0.1479	-351.323874345659	-351.323874345659\\
72.5	0.15156	-374.411198015644	-374.411198015644\\
72.5	0.15522	-398.718140317914	-398.718140317914\\
72.5	0.15888	-424.244701252467	-424.244701252467\\
72.5	0.16254	-450.990880819304	-450.990880819304\\
72.5	0.1662	-478.956679018426	-478.956679018426\\
72.5	0.16986	-508.142095849831	-508.142095849831\\
72.5	0.17352	-538.54713131352	-538.54713131352\\
72.5	0.17718	-570.171785409494	-570.171785409494\\
72.5	0.18084	-603.016058137751	-603.016058137751\\
72.5	0.1845	-637.079949498293	-637.079949498293\\
72.5	0.18816	-672.363459491118	-672.363459491118\\
72.5	0.19182	-708.866588116227	-708.866588116227\\
72.5	0.19548	-746.589335373621	-746.589335373621\\
72.5	0.19914	-785.5317012633	-785.5317012633\\
72.5	0.2028	-825.693685785261	-825.693685785261\\
72.5	0.20646	-867.075288939507	-867.075288939507\\
72.5	0.21012	-909.676510726036	-909.676510726036\\
72.5	0.21378	-953.49735114485	-953.49735114485\\
72.5	0.21744	-998.537810195948	-998.537810195948\\
72.5	0.2211	-1044.79788787933	-1044.79788787933\\
72.5	0.22476	-1092.277584195	-1092.277584195\\
72.5	0.22842	-1140.97689914294	-1140.97689914294\\
72.5	0.23208	-1190.89583272318	-1190.89583272318\\
72.5	0.23574	-1242.0343849357	-1242.0343849357\\
72.5	0.2394	-1294.3925557805	-1294.3925557805\\
72.5	0.24306	-1347.97034525758	-1347.97034525758\\
72.5	0.24672	-1402.76775336695	-1402.76775336695\\
72.5	0.25038	-1458.78478010861	-1458.78478010861\\
72.5	0.25404	-1516.02142548255	-1516.02142548255\\
72.5	0.2577	-1574.47768948877	-1574.47768948877\\
72.5	0.26136	-1634.15357212728	-1634.15357212728\\
72.5	0.26502	-1695.04907339807	-1695.04907339807\\
72.5	0.26868	-1757.16419330114	-1757.16419330114\\
72.5	0.27234	-1820.4989318365	-1820.4989318365\\
72.5	0.276	-1885.05328900414	-1885.05328900414\\
72.875	0.093	-155.297851670751	-155.297851670751\\
72.875	0.09666	-160.120397660694	-160.120397660694\\
72.875	0.10032	-166.162562282921	-166.162562282921\\
72.875	0.10398	-173.424345537433	-173.424345537433\\
72.875	0.10764	-181.905747424229	-181.905747424229\\
72.875	0.1113	-191.606767943309	-191.606767943309\\
72.875	0.11496	-202.527407094672	-202.527407094672\\
72.875	0.11862	-214.66766487832	-214.66766487832\\
72.875	0.12228	-228.027541294251	-228.027541294251\\
72.875	0.12594	-242.607036342467	-242.607036342467\\
72.875	0.1296	-258.406150022967	-258.406150022967\\
72.875	0.13326	-275.42488233575	-275.42488233575\\
72.875	0.13692	-293.663233280818	-293.663233280818\\
72.875	0.14058	-313.12120285817	-313.12120285817\\
72.875	0.14424	-333.798791067806	-333.798791067806\\
72.875	0.1479	-355.695997909726	-355.695997909726\\
72.875	0.15156	-378.81282338393	-378.81282338393\\
72.875	0.15522	-403.149267490418	-403.149267490418\\
72.875	0.15888	-428.70533022919	-428.70533022919\\
72.875	0.16254	-455.481011600246	-455.481011600246\\
72.875	0.1662	-483.476311603586	-483.476311603586\\
72.875	0.16986	-512.69123023921	-512.69123023921\\
72.875	0.17352	-543.125767507118	-543.125767507118\\
72.875	0.17718	-574.77992340731	-574.77992340731\\
72.875	0.18084	-607.653697939786	-607.653697939786\\
72.875	0.1845	-641.747091104547	-641.747091104547\\
72.875	0.18816	-677.060102901591	-677.060102901591\\
72.875	0.19182	-713.592733330919	-713.592733330919\\
72.875	0.19548	-751.344982392531	-751.344982392531\\
72.875	0.19914	-790.316850086428	-790.316850086428\\
72.875	0.2028	-830.508336412609	-830.508336412609\\
72.875	0.20646	-871.919441371073	-871.919441371073\\
72.875	0.21012	-914.550164961821	-914.550164961821\\
72.875	0.21378	-958.400507184853	-958.400507184853\\
72.875	0.21744	-1003.47046804017	-1003.47046804017\\
72.875	0.2211	-1049.76004752777	-1049.76004752777\\
72.875	0.22476	-1097.26924564765	-1097.26924564765\\
72.875	0.22842	-1145.99806239982	-1145.99806239982\\
72.875	0.23208	-1195.94649778428	-1195.94649778428\\
72.875	0.23574	-1247.11455180101	-1247.11455180101\\
72.875	0.2394	-1299.50222445003	-1299.50222445003\\
72.875	0.24306	-1353.10951573134	-1353.10951573134\\
72.875	0.24672	-1407.93642564493	-1407.93642564493\\
72.875	0.25038	-1463.9829541908	-1463.9829541908\\
72.875	0.25404	-1521.24910136896	-1521.24910136896\\
72.875	0.2577	-1579.7348671794	-1579.7348671794\\
72.875	0.26136	-1639.44025162212	-1639.44025162212\\
72.875	0.26502	-1700.36525469713	-1700.36525469713\\
72.875	0.26868	-1762.50987640442	-1762.50987640442\\
72.875	0.27234	-1825.874116744	-1825.874116744\\
72.875	0.276	-1890.45797571586	-1890.45797571586\\
73.25	0.093	-159.36605153621	-159.36605153621\\
73.25	0.09666	-164.218099330372	-164.218099330372\\
73.25	0.10032	-170.289765756818	-170.289765756818\\
73.25	0.10398	-177.581050815548	-177.581050815548\\
73.25	0.10764	-186.091954506562	-186.091954506562\\
73.25	0.1113	-195.822476829861	-195.822476829861\\
73.25	0.11496	-206.772617785443	-206.772617785443\\
73.25	0.11862	-218.942377373309	-218.942377373309\\
73.25	0.12228	-232.331755593459	-232.331755593459\\
73.25	0.12594	-246.940752445895	-246.940752445895\\
73.25	0.1296	-262.769367930613	-262.769367930613\\
73.25	0.13326	-279.817602047615	-279.817602047615\\
73.25	0.13692	-298.085454796902	-298.085454796902\\
73.25	0.14058	-317.572926178472	-317.572926178472\\
73.25	0.14424	-338.280016192326	-338.280016192326\\
73.25	0.1479	-360.206724838465	-360.206724838465\\
73.25	0.15156	-383.353052116888	-383.353052116888\\
73.25	0.15522	-407.718998027595	-407.718998027595\\
73.25	0.15888	-433.304562570585	-433.304562570585\\
73.25	0.16254	-460.10974574586	-460.10974574586\\
73.25	0.1662	-488.134547553419	-488.134547553419\\
73.25	0.16986	-517.378967993262	-517.378967993262\\
73.25	0.17352	-547.843007065388	-547.843007065388\\
73.25	0.17718	-579.526664769799	-579.526664769799\\
73.25	0.18084	-612.429941106494	-612.429941106494\\
73.25	0.1845	-646.552836075473	-646.552836075473\\
73.25	0.18816	-681.895349676736	-681.895349676736\\
73.25	0.19182	-718.457481910283	-718.457481910283\\
73.25	0.19548	-756.239232776114	-756.239232776114\\
73.25	0.19914	-795.24060227423	-795.24060227423\\
73.25	0.2028	-835.461590404629	-835.461590404629\\
73.25	0.20646	-876.902197167312	-876.902197167312\\
73.25	0.21012	-919.562422562279	-919.562422562279\\
73.25	0.21378	-963.44226658953	-963.44226658953\\
73.25	0.21744	-1008.54172924906	-1008.54172924906\\
73.25	0.2211	-1054.86081054088	-1054.86081054088\\
73.25	0.22476	-1102.39951046499	-1102.39951046499\\
73.25	0.22842	-1151.15782902137	-1151.15782902137\\
73.25	0.23208	-1201.13576621005	-1201.13576621005\\
73.25	0.23574	-1252.333322031	-1252.333322031\\
73.25	0.2394	-1304.75049648424	-1304.75049648424\\
73.25	0.24306	-1358.38728956977	-1358.38728956977\\
73.25	0.24672	-1413.24370128757	-1413.24370128757\\
73.25	0.25038	-1469.31973163766	-1469.31973163766\\
73.25	0.25404	-1526.61538062004	-1526.61538062004\\
73.25	0.2577	-1585.1306482347	-1585.1306482347\\
73.25	0.26136	-1644.86553448164	-1644.86553448164\\
73.25	0.26502	-1705.82003936087	-1705.82003936087\\
73.25	0.26868	-1767.99416287238	-1767.99416287238\\
73.25	0.27234	-1831.38790501618	-1831.38790501618\\
73.25	0.276	-1896.00126579226	-1896.00126579226\\
73.625	0.093	-163.572854766341	-163.572854766341\\
73.625	0.09666	-168.454404364722	-168.454404364722\\
73.625	0.10032	-174.555572595387	-174.555572595387\\
73.625	0.10398	-181.876359458336	-181.876359458336\\
73.625	0.10764	-190.416764953569	-190.416764953569\\
73.625	0.1113	-200.176789081086	-200.176789081086\\
73.625	0.11496	-211.156431840887	-211.156431840887\\
73.625	0.11862	-223.355693232972	-223.355693232972\\
73.625	0.12228	-236.774573257341	-236.774573257341\\
73.625	0.12594	-251.413071913995	-251.413071913995\\
73.625	0.1296	-267.271189202932	-267.271189202932\\
73.625	0.13326	-284.348925124152	-284.348925124152\\
73.625	0.13692	-302.646279677658	-302.646279677658\\
73.625	0.14058	-322.163252863447	-322.163252863447\\
73.625	0.14424	-342.89984468152	-342.89984468152\\
73.625	0.1479	-364.856055131878	-364.856055131878\\
73.625	0.15156	-388.031884214519	-388.031884214519\\
73.625	0.15522	-412.427331929445	-412.427331929445\\
73.625	0.15888	-438.042398276654	-438.042398276654\\
73.625	0.16254	-464.877083256147	-464.877083256147\\
73.625	0.1662	-492.931386867925	-492.931386867925\\
73.625	0.16986	-522.205309111987	-522.205309111987\\
73.625	0.17352	-552.698849988332	-552.698849988332\\
73.625	0.17718	-584.412009496962	-584.412009496962\\
73.625	0.18084	-617.344787637875	-617.344787637875\\
73.625	0.1845	-651.497184411073	-651.497184411073\\
73.625	0.18816	-686.869199816554	-686.869199816554\\
73.625	0.19182	-723.46083385432	-723.46083385432\\
73.625	0.19548	-761.27208652437	-761.27208652437\\
73.625	0.19914	-800.302957826704	-800.302957826704\\
73.625	0.2028	-840.553447761322	-840.553447761322\\
73.625	0.20646	-882.023556328224	-882.023556328224\\
73.625	0.21012	-924.713283527409	-924.713283527409\\
73.625	0.21378	-968.622629358879	-968.622629358879\\
73.625	0.21744	-1013.75159382263	-1013.75159382263\\
73.625	0.2211	-1060.10017691867	-1060.10017691867\\
73.625	0.22476	-1107.66837864699	-1107.66837864699\\
73.625	0.22842	-1156.4561990076	-1156.4561990076\\
73.625	0.23208	-1206.46363800049	-1206.46363800049\\
73.625	0.23574	-1257.69069562566	-1257.69069562566\\
73.625	0.2394	-1310.13737188312	-1310.13737188312\\
73.625	0.24306	-1363.80366677286	-1363.80366677286\\
73.625	0.24672	-1418.68958029489	-1418.68958029489\\
73.625	0.25038	-1474.7951124492	-1474.7951124492\\
73.625	0.25404	-1532.12026323579	-1532.12026323579\\
73.625	0.2577	-1590.66503265467	-1590.66503265467\\
73.625	0.26136	-1650.42942070583	-1650.42942070583\\
73.625	0.26502	-1711.41342738928	-1711.41342738928\\
73.625	0.26868	-1773.61705270501	-1773.61705270501\\
73.625	0.27234	-1837.04029665303	-1837.04029665303\\
73.625	0.276	-1901.68315923332	-1901.68315923332\\
74	0.093	-167.918261361147	-167.918261361147\\
74	0.09666	-172.829312763746	-172.829312763746\\
74	0.10032	-178.95998279863	-178.95998279863\\
74	0.10398	-186.310271465797	-186.310271465797\\
74	0.10764	-194.880178765249	-194.880178765249\\
74	0.1113	-204.669704696985	-204.669704696985\\
74	0.11496	-215.678849261005	-215.678849261005\\
74	0.11862	-227.907612457309	-227.907612457309\\
74	0.12228	-241.355994285896	-241.355994285896\\
74	0.12594	-256.023994746768	-256.023994746768\\
74	0.1296	-271.911613839924	-271.911613839924\\
74	0.13326	-289.018851565364	-289.018851565364\\
74	0.13692	-307.345707923088	-307.345707923088\\
74	0.14058	-326.892182913096	-326.892182913096\\
74	0.14424	-347.658276535388	-347.658276535388\\
74	0.1479	-369.643988789964	-369.643988789964\\
74	0.15156	-392.849319676824	-392.849319676824\\
74	0.15522	-417.274269195968	-417.274269195968\\
74	0.15888	-442.918837347396	-442.918837347396\\
74	0.16254	-469.783024131108	-469.783024131108\\
74	0.1662	-497.866829547105	-497.866829547105\\
74	0.16986	-527.170253595385	-527.170253595385\\
74	0.17352	-557.693296275949	-557.693296275949\\
74	0.17718	-589.435957588797	-589.435957588797\\
74	0.18084	-622.398237533929	-622.398237533929\\
74	0.1845	-656.580136111346	-656.580136111346\\
74	0.18816	-691.981653321047	-691.981653321047\\
74	0.19182	-728.602789163031	-728.602789163031\\
74	0.19548	-766.443543637299	-766.443543637299\\
74	0.19914	-805.503916743852	-805.503916743852\\
74	0.2028	-845.783908482689	-845.783908482689\\
74	0.20646	-887.283518853809	-887.283518853809\\
74	0.21012	-930.002747857214	-930.002747857214\\
74	0.21378	-973.941595492902	-973.941595492902\\
74	0.21744	-1019.10006176087	-1019.10006176087\\
74	0.2211	-1065.47814666113	-1065.47814666113\\
74	0.22476	-1113.07585019367	-1113.07585019367\\
74	0.22842	-1161.8931723585	-1161.8931723585\\
74	0.23208	-1211.9301131556	-1211.9301131556\\
74	0.23574	-1263.186672585	-1263.186672585\\
74	0.2394	-1315.66285064668	-1315.66285064668\\
74	0.24306	-1369.35864734064	-1369.35864734064\\
74	0.24672	-1424.27406266688	-1424.27406266688\\
74	0.25038	-1480.40909662541	-1480.40909662541\\
74	0.25404	-1537.76374921622	-1537.76374921622\\
74	0.2577	-1596.33802043932	-1596.33802043932\\
74	0.26136	-1656.1319102947	-1656.1319102947\\
74	0.26502	-1717.14541878237	-1717.14541878237\\
74	0.26868	-1779.37854590232	-1779.37854590232\\
74	0.27234	-1842.83129165455	-1842.83129165455\\
74	0.276	-1907.50365603906	-1907.50365603906\\
};
\end{axis}

\begin{axis}[%
width=6.159cm,
height=3.097cm,
at={(8.104cm,0cm)},
scale only axis,
xmin=56,
xmax=74,
tick align=outside,
xlabel style={font=\color{white!15!black}},
xlabel={$L_{cut}$},
ymin=0.093,
ymax=0.276,
ylabel style={font=\color{white!15!black}},
ylabel={$D_{rlx}$},
zmin=0,
zmax=113.453202447659,
zlabel style={font=\color{white!15!black}},
zlabel={$u(t-1)u(t)$},
view={-140}{50},
axis background/.style={fill=white},
xmajorgrids,
ymajorgrids,
zmajorgrids
]
\addplot3[only marks, mark=*, mark options={}, mark size=1.5000pt, color=mycolor1, fill=mycolor1] table[row sep=crcr]{%
x	y	z\\
74	0.123	15.4680684467597\\
72	0.113	12.0981268654473\\
61	0.095	6.35969255839677\\
56	0.093	5.67589105085895\\
};
\addplot3[only marks, mark=*, mark options={}, mark size=1.5000pt, color=mycolor2, fill=mycolor2] table[row sep=crcr]{%
x	y	z\\
67	0.276	110.131370256376\\
66	0.255	90.4271696447197\\
62	0.209	51.7720891508705\\
57	0.193	41.4264697981385\\
};
\addplot3[only marks, mark=*, mark options={}, mark size=1.5000pt, color=black, fill=black] table[row sep=crcr]{%
x	y	z\\
69	0.104	9.34303245175057\\
};
\addplot3[only marks, mark=*, mark options={}, mark size=1.5000pt, color=black, fill=black] table[row sep=crcr]{%
x	y	z\\
64	0.23	68.0446101221447\\
};

\addplot3[%
surf,
fill opacity=0.7, shader=interp, colormap={mymap}{[1pt] rgb(0pt)=(1,0.905882,0); rgb(1pt)=(1,0.901964,0); rgb(2pt)=(1,0.898051,0); rgb(3pt)=(1,0.894144,0); rgb(4pt)=(1,0.890243,0); rgb(5pt)=(1,0.886349,0); rgb(6pt)=(1,0.88246,0); rgb(7pt)=(1,0.878577,0); rgb(8pt)=(1,0.8747,0); rgb(9pt)=(1,0.870829,0); rgb(10pt)=(1,0.866964,0); rgb(11pt)=(1,0.863106,0); rgb(12pt)=(1,0.859253,0); rgb(13pt)=(1,0.855406,0); rgb(14pt)=(1,0.851566,0); rgb(15pt)=(1,0.847732,0); rgb(16pt)=(1,0.843903,0); rgb(17pt)=(1,0.840081,0); rgb(18pt)=(1,0.836265,0); rgb(19pt)=(1,0.832455,0); rgb(20pt)=(1,0.828652,0); rgb(21pt)=(1,0.824854,0); rgb(22pt)=(1,0.821063,0); rgb(23pt)=(1,0.817278,0); rgb(24pt)=(1,0.8135,0); rgb(25pt)=(1,0.809727,0); rgb(26pt)=(1,0.805961,0); rgb(27pt)=(1,0.8022,0); rgb(28pt)=(1,0.798445,0); rgb(29pt)=(1,0.794696,0); rgb(30pt)=(1,0.790953,0); rgb(31pt)=(1,0.787215,0); rgb(32pt)=(1,0.783484,0); rgb(33pt)=(1,0.779758,0); rgb(34pt)=(1,0.776038,0); rgb(35pt)=(1,0.772324,0); rgb(36pt)=(1,0.768615,0); rgb(37pt)=(1,0.764913,0); rgb(38pt)=(1,0.761217,0); rgb(39pt)=(1,0.757527,0); rgb(40pt)=(1,0.753843,0); rgb(41pt)=(1,0.750165,0); rgb(42pt)=(1,0.746493,0); rgb(43pt)=(1,0.742827,0); rgb(44pt)=(1,0.739167,0); rgb(45pt)=(1,0.735514,0); rgb(46pt)=(1,0.731867,0); rgb(47pt)=(1,0.728226,0); rgb(48pt)=(1,0.724591,0); rgb(49pt)=(1,0.720963,0); rgb(50pt)=(1,0.717341,0); rgb(51pt)=(1,0.713725,0); rgb(52pt)=(0.999994,0.710077,0); rgb(53pt)=(0.999974,0.706363,0); rgb(54pt)=(0.999942,0.702592,0); rgb(55pt)=(0.999898,0.698775,0); rgb(56pt)=(0.999841,0.694921,0); rgb(57pt)=(0.999771,0.691039,0); rgb(58pt)=(0.99969,0.687139,0); rgb(59pt)=(0.999596,0.68323,0); rgb(60pt)=(0.99949,0.679323,0); rgb(61pt)=(0.999372,0.675427,0); rgb(62pt)=(0.999242,0.67155,0); rgb(63pt)=(0.9991,0.667704,0); rgb(64pt)=(0.998946,0.663897,0); rgb(65pt)=(0.998781,0.660138,0); rgb(66pt)=(0.998605,0.656439,0); rgb(67pt)=(0.998416,0.652807,0); rgb(68pt)=(0.998217,0.649253,0); rgb(69pt)=(0.998006,0.645786,0); rgb(70pt)=(0.997785,0.642416,0); rgb(71pt)=(0.997552,0.639152,0); rgb(72pt)=(0.997308,0.636004,0); rgb(73pt)=(0.997053,0.632982,0); rgb(74pt)=(0.996788,0.630095,0); rgb(75pt)=(0.996512,0.627352,0); rgb(76pt)=(0.996226,0.624763,0); rgb(77pt)=(0.995851,0.622329,0); rgb(78pt)=(0.99494,0.619997,0); rgb(79pt)=(0.99345,0.617753,0); rgb(80pt)=(0.991419,0.61559,0); rgb(81pt)=(0.988885,0.613503,0); rgb(82pt)=(0.985886,0.611486,0); rgb(83pt)=(0.98246,0.609532,0); rgb(84pt)=(0.978643,0.607636,0); rgb(85pt)=(0.974475,0.605791,0); rgb(86pt)=(0.969992,0.603992,0); rgb(87pt)=(0.965232,0.602233,0); rgb(88pt)=(0.960233,0.600507,0); rgb(89pt)=(0.955033,0.598808,0); rgb(90pt)=(0.949669,0.59713,0); rgb(91pt)=(0.94418,0.595468,0); rgb(92pt)=(0.938602,0.593815,0); rgb(93pt)=(0.932974,0.592166,0); rgb(94pt)=(0.927333,0.590513,0); rgb(95pt)=(0.921717,0.588852,0); rgb(96pt)=(0.916164,0.587176,0); rgb(97pt)=(0.910711,0.585479,0); rgb(98pt)=(0.905397,0.583755,0); rgb(99pt)=(0.900258,0.581999,0); rgb(100pt)=(0.895333,0.580203,0); rgb(101pt)=(0.890659,0.578362,0); rgb(102pt)=(0.886275,0.576471,0); rgb(103pt)=(0.882047,0.574545,0); rgb(104pt)=(0.877819,0.572608,0); rgb(105pt)=(0.873592,0.57066,0); rgb(106pt)=(0.869366,0.568701,0); rgb(107pt)=(0.865143,0.566733,0); rgb(108pt)=(0.860924,0.564756,0); rgb(109pt)=(0.856708,0.562771,0); rgb(110pt)=(0.852497,0.560778,0); rgb(111pt)=(0.848292,0.558779,0); rgb(112pt)=(0.844092,0.556774,0); rgb(113pt)=(0.8399,0.554763,0); rgb(114pt)=(0.835716,0.552749,0); rgb(115pt)=(0.831541,0.55073,0); rgb(116pt)=(0.827374,0.548709,0); rgb(117pt)=(0.823219,0.546686,0); rgb(118pt)=(0.819074,0.54466,0); rgb(119pt)=(0.81494,0.542635,0); rgb(120pt)=(0.81082,0.540609,0); rgb(121pt)=(0.806712,0.538584,0); rgb(122pt)=(0.802619,0.53656,0); rgb(123pt)=(0.798541,0.534539,0); rgb(124pt)=(0.794478,0.532521,0); rgb(125pt)=(0.790431,0.530506,0); rgb(126pt)=(0.786402,0.528496,0); rgb(127pt)=(0.782391,0.526491,0); rgb(128pt)=(0.77841,0.524489,0); rgb(129pt)=(0.774523,0.522478,0); rgb(130pt)=(0.770731,0.520455,0); rgb(131pt)=(0.767022,0.518424,0); rgb(132pt)=(0.763384,0.516385,0); rgb(133pt)=(0.759804,0.514339,0); rgb(134pt)=(0.756272,0.51229,0); rgb(135pt)=(0.752775,0.510237,0); rgb(136pt)=(0.749302,0.508182,0); rgb(137pt)=(0.74584,0.506128,0); rgb(138pt)=(0.742378,0.504075,0); rgb(139pt)=(0.738904,0.502025,0); rgb(140pt)=(0.735406,0.499979,0); rgb(141pt)=(0.731872,0.49794,0); rgb(142pt)=(0.72829,0.495909,0); rgb(143pt)=(0.724649,0.493887,0); rgb(144pt)=(0.720936,0.491875,0); rgb(145pt)=(0.71714,0.489876,0); rgb(146pt)=(0.713249,0.487891,0); rgb(147pt)=(0.709251,0.485921,0); rgb(148pt)=(0.705134,0.483968,0); rgb(149pt)=(0.700887,0.482033,0); rgb(150pt)=(0.696497,0.480118,0); rgb(151pt)=(0.691952,0.478225,0); rgb(152pt)=(0.687242,0.476355,0); rgb(153pt)=(0.682353,0.47451,0); rgb(154pt)=(0.677195,0.472696,0); rgb(155pt)=(0.6717,0.470916,0); rgb(156pt)=(0.665891,0.469169,0); rgb(157pt)=(0.659791,0.46745,0); rgb(158pt)=(0.653423,0.465756,0); rgb(159pt)=(0.64681,0.464084,0); rgb(160pt)=(0.639976,0.462432,0); rgb(161pt)=(0.632943,0.460795,0); rgb(162pt)=(0.625734,0.459171,0); rgb(163pt)=(0.618373,0.457556,0); rgb(164pt)=(0.610882,0.455948,0); rgb(165pt)=(0.603284,0.454343,0); rgb(166pt)=(0.595604,0.452737,0); rgb(167pt)=(0.587863,0.451129,0); rgb(168pt)=(0.580084,0.449514,0); rgb(169pt)=(0.572292,0.447889,0); rgb(170pt)=(0.564508,0.446252,0); rgb(171pt)=(0.556756,0.444599,0); rgb(172pt)=(0.549059,0.442927,0); rgb(173pt)=(0.54144,0.441232,0); rgb(174pt)=(0.533922,0.439512,0); rgb(175pt)=(0.526529,0.437764,0); rgb(176pt)=(0.519282,0.435983,0); rgb(177pt)=(0.512206,0.434168,0); rgb(178pt)=(0.505323,0.432315,0); rgb(179pt)=(0.498628,0.430422,3.92506e-06); rgb(180pt)=(0.491973,0.428504,3.49981e-05); rgb(181pt)=(0.485331,0.426562,9.63073e-05); rgb(182pt)=(0.478704,0.424596,0.000186979); rgb(183pt)=(0.472096,0.422609,0.000306141); rgb(184pt)=(0.465508,0.420599,0.00045292); rgb(185pt)=(0.458942,0.418567,0.000626441); rgb(186pt)=(0.452401,0.416515,0.000825833); rgb(187pt)=(0.445885,0.414441,0.00105022); rgb(188pt)=(0.439399,0.412348,0.00129873); rgb(189pt)=(0.432942,0.410234,0.00157049); rgb(190pt)=(0.426518,0.408102,0.00186463); rgb(191pt)=(0.420129,0.40595,0.00218028); rgb(192pt)=(0.413777,0.40378,0.00251655); rgb(193pt)=(0.407464,0.401592,0.00287258); rgb(194pt)=(0.401191,0.399386,0.00324749); rgb(195pt)=(0.394962,0.397164,0.00364042); rgb(196pt)=(0.388777,0.394925,0.00405048); rgb(197pt)=(0.38264,0.39267,0.00447681); rgb(198pt)=(0.376552,0.390399,0.00491852); rgb(199pt)=(0.370516,0.388113,0.00537476); rgb(200pt)=(0.364532,0.385812,0.00584464); rgb(201pt)=(0.358605,0.383497,0.00632729); rgb(202pt)=(0.352735,0.381168,0.00682184); rgb(203pt)=(0.346925,0.378826,0.00732741); rgb(204pt)=(0.341176,0.376471,0.00784314); rgb(205pt)=(0.335485,0.374093,0.00847245); rgb(206pt)=(0.329843,0.371682,0.00930909); rgb(207pt)=(0.324249,0.369242,0.0103377); rgb(208pt)=(0.318701,0.366772,0.0115428); rgb(209pt)=(0.313198,0.364275,0.0129091); rgb(210pt)=(0.307739,0.361753,0.0144211); rgb(211pt)=(0.302322,0.359206,0.0160634); rgb(212pt)=(0.296945,0.356637,0.0178207); rgb(213pt)=(0.291607,0.354048,0.0196776); rgb(214pt)=(0.286307,0.35144,0.0216186); rgb(215pt)=(0.281043,0.348814,0.0236284); rgb(216pt)=(0.275813,0.346172,0.0256916); rgb(217pt)=(0.270616,0.343517,0.0277927); rgb(218pt)=(0.265451,0.340849,0.0299163); rgb(219pt)=(0.260317,0.33817,0.0320472); rgb(220pt)=(0.25521,0.335482,0.0341698); rgb(221pt)=(0.250131,0.332786,0.0362688); rgb(222pt)=(0.245078,0.330085,0.0383287); rgb(223pt)=(0.240048,0.327379,0.0403343); rgb(224pt)=(0.235042,0.324671,0.04227); rgb(225pt)=(0.230056,0.321962,0.0441205); rgb(226pt)=(0.22509,0.319254,0.0458704); rgb(227pt)=(0.220142,0.316548,0.0475043); rgb(228pt)=(0.215212,0.313846,0.0490067); rgb(229pt)=(0.210296,0.311149,0.0503624); rgb(230pt)=(0.205395,0.308459,0.0515759); rgb(231pt)=(0.200514,0.305763,0.052757); rgb(232pt)=(0.195655,0.303061,0.0539242); rgb(233pt)=(0.190817,0.300353,0.0550763); rgb(234pt)=(0.186001,0.297639,0.0562123); rgb(235pt)=(0.181207,0.294918,0.0573313); rgb(236pt)=(0.176434,0.292191,0.0584321); rgb(237pt)=(0.171685,0.289458,0.0595136); rgb(238pt)=(0.166957,0.286719,0.060575); rgb(239pt)=(0.162252,0.283973,0.0616151); rgb(240pt)=(0.15757,0.281221,0.0626328); rgb(241pt)=(0.152911,0.278463,0.0636271); rgb(242pt)=(0.148275,0.275699,0.0645971); rgb(243pt)=(0.143663,0.272929,0.0655416); rgb(244pt)=(0.139074,0.270152,0.0664596); rgb(245pt)=(0.134508,0.26737,0.06735); rgb(246pt)=(0.129967,0.264581,0.0682118); rgb(247pt)=(0.125449,0.261787,0.0690441); rgb(248pt)=(0.120956,0.258986,0.0698456); rgb(249pt)=(0.116487,0.25618,0.0706154); rgb(250pt)=(0.112043,0.253367,0.0713525); rgb(251pt)=(0.107623,0.250549,0.0720557); rgb(252pt)=(0.103229,0.247724,0.0727241); rgb(253pt)=(0.0988592,0.244894,0.0733566); rgb(254pt)=(0.0945149,0.242058,0.0739522); rgb(255pt)=(0.0901961,0.239216,0.0745098)}, mesh/rows=49]
table[row sep=crcr, point meta=\thisrow{c}] {%
%
x	y	z	c\\
56	0.093	5.74613150884213	5.74613150884213\\
56	0.09666	6.17235692462004	6.17235692462004\\
56	0.10032	6.66497229086033	6.66497229086033\\
56	0.10398	7.223977607563	7.223977607563\\
56	0.10764	7.84937287472805	7.84937287472805\\
56	0.1113	8.54115809235549	8.54115809235549\\
56	0.11496	9.29933326044532	9.29933326044532\\
56	0.11862	10.1238983789975	10.1238983789975\\
56	0.12228	11.0148534480121	11.0148534480121\\
56	0.12594	11.9721984674891	11.9721984674891\\
56	0.1296	12.9959334374284	12.9959334374284\\
56	0.13326	14.0860583578301	14.0860583578301\\
56	0.13692	15.2425732286942	15.2425732286942\\
56	0.14058	16.4654780500207	16.4654780500207\\
56	0.14424	17.7547728218096	17.7547728218096\\
56	0.1479	19.1104575440608	19.1104575440608\\
56	0.15156	20.5325322167744	20.5325322167744\\
56	0.15522	22.0209968399504	22.0209968399504\\
56	0.15888	23.5758514135888	23.5758514135888\\
56	0.16254	25.1970959376896	25.1970959376896\\
56	0.1662	26.8847304122528	26.8847304122528\\
56	0.16986	28.6387548372783	28.6387548372783\\
56	0.17352	30.4591692127662	30.4591692127662\\
56	0.17718	32.3459735387165	32.3459735387165\\
56	0.18084	34.2991678151292	34.2991678151292\\
56	0.1845	36.3187520420042	36.3187520420042\\
56	0.18816	38.4047262193417	38.4047262193417\\
56	0.19182	40.5570903471415	40.5570903471415\\
56	0.19548	42.7758444254037	42.7758444254037\\
56	0.19914	45.0609884541283	45.0609884541283\\
56	0.2028	47.4125224333153	47.4125224333153\\
56	0.20646	49.8304463629646	49.8304463629646\\
56	0.21012	52.3147602430763	52.3147602430763\\
56	0.21378	54.8654640736505	54.8654640736505\\
56	0.21744	57.4825578546869	57.4825578546869\\
56	0.2211	60.1660415861858	60.1660415861858\\
56	0.22476	62.9159152681471	62.9159152681471\\
56	0.22842	65.7321789005707	65.7321789005707\\
56	0.23208	68.6148324834567	68.6148324834567\\
56	0.23574	71.5638760168051	71.5638760168051\\
56	0.2394	74.5793095006159	74.5793095006159\\
56	0.24306	77.6611329348891	77.6611329348891\\
56	0.24672	80.8093463196246	80.8093463196246\\
56	0.25038	84.0239496548225	84.0239496548225\\
56	0.25404	87.3049429404828	87.3049429404828\\
56	0.2577	90.6523261766055	90.6523261766055\\
56	0.26136	94.0660993631906	94.0660993631906\\
56	0.26502	97.546262500238	97.546262500238\\
56	0.26868	101.092815587748	101.092815587748\\
56	0.27234	104.70575862572	104.70575862572\\
56	0.276	108.385091614155	108.385091614155\\
56.375	0.093	5.74291436766217	5.74291436766217\\
56.375	0.09666	6.16955692276974	6.16955692276974\\
56.375	0.10032	6.6625894283397	6.6625894283397\\
56.375	0.10398	7.22201188437204	7.22201188437204\\
56.375	0.10764	7.84782429086675	7.84782429086675\\
56.375	0.1113	8.54002664782384	8.54002664782384\\
56.375	0.11496	9.29861895524333	9.29861895524333\\
56.375	0.11862	10.1236012131252	10.1236012131252\\
56.375	0.12228	11.0149734214694	11.0149734214694\\
56.375	0.12594	11.9727355802761	11.9727355802761\\
56.375	0.1296	12.9968876895451	12.9968876895451\\
56.375	0.13326	14.0874297492765	14.0874297492765\\
56.375	0.13692	15.2443617594702	15.2443617594702\\
56.375	0.14058	16.4676837201264	16.4676837201264\\
56.375	0.14424	17.7573956312449	17.7573956312449\\
56.375	0.1479	19.1134974928258	19.1134974928258\\
56.375	0.15156	20.5359893048691	20.5359893048691\\
56.375	0.15522	22.0248710673748	22.0248710673748\\
56.375	0.15888	23.5801427803429	23.5801427803429\\
56.375	0.16254	25.2018044437733	25.2018044437733\\
56.375	0.1662	26.8898560576661	26.8898560576661\\
56.375	0.16986	28.6442976220213	28.6442976220213\\
56.375	0.17352	30.4651291368389	30.4651291368389\\
56.375	0.17718	32.3523506021189	32.3523506021189\\
56.375	0.18084	34.3059620178612	34.3059620178612\\
56.375	0.1845	36.325963384066	36.325963384066\\
56.375	0.18816	38.4123547007331	38.4123547007331\\
56.375	0.19182	40.5651359678625	40.5651359678625\\
56.375	0.19548	42.7843071854544	42.7843071854544\\
56.375	0.19914	45.0698683535087	45.0698683535087\\
56.375	0.2028	47.4218194720253	47.4218194720253\\
56.375	0.20646	49.8401605410044	49.8401605410044\\
56.375	0.21012	52.3248915604457	52.3248915604457\\
56.375	0.21378	54.8760125303495	54.8760125303495\\
56.375	0.21744	57.4935234507156	57.4935234507156\\
56.375	0.2211	60.1774243215442	60.1774243215442\\
56.375	0.22476	62.9277151428351	62.9277151428351\\
56.375	0.22842	65.7443959145884	65.7443959145884\\
56.375	0.23208	68.6274666368041	68.6274666368041\\
56.375	0.23574	71.5769273094822	71.5769273094822\\
56.375	0.2394	74.5927779326226	74.5927779326226\\
56.375	0.24306	77.6750185062255	77.6750185062255\\
56.375	0.24672	80.8236490302907	80.8236490302907\\
56.375	0.25038	84.0386695048182	84.0386695048182\\
56.375	0.25404	87.3200799298082	87.3200799298082\\
56.375	0.2577	90.6678803052606	90.6678803052606\\
56.375	0.26136	94.0820706311753	94.0820706311753\\
56.375	0.26502	97.5626509075524	97.5626509075524\\
56.375	0.26868	101.109621134392	101.109621134392\\
56.375	0.27234	104.722981311694	104.722981311694\\
56.375	0.276	108.402731439458	108.402731439458\\
56.75	0.093	5.7434396016763	5.7434396016763\\
56.75	0.09666	6.17049929611354	6.17049929611354\\
56.75	0.10032	6.66394894101316	6.66394894101316\\
56.75	0.10398	7.22378853637516	7.22378853637516\\
56.75	0.10764	7.85001808219955	7.85001808219955\\
56.75	0.1113	8.54263757848633	8.54263757848633\\
56.75	0.11496	9.30164702523549	9.30164702523549\\
56.75	0.11862	10.127046422447	10.127046422447\\
56.75	0.12228	11.0188357701209	11.0188357701209\\
56.75	0.12594	11.9770150682572	11.9770150682572\\
56.75	0.1296	13.0015843168559	13.0015843168559\\
56.75	0.13326	14.0925435159169	14.0925435159169\\
56.75	0.13692	15.2498926654404	15.2498926654404\\
56.75	0.14058	16.4736317654262	16.4736317654262\\
56.75	0.14424	17.7637608158744	17.7637608158744\\
56.75	0.1479	19.120279816785	19.120279816785\\
56.75	0.15156	20.5431887681579	20.5431887681579\\
56.75	0.15522	22.0324876699933	22.0324876699933\\
56.75	0.15888	23.588176522291	23.588176522291\\
56.75	0.16254	25.2102553250511	25.2102553250511\\
56.75	0.1662	26.8987240782736	26.8987240782736\\
56.75	0.16986	28.6535827819585	28.6535827819585\\
56.75	0.17352	30.4748314361057	30.4748314361057\\
56.75	0.17718	32.3624700407154	32.3624700407154\\
56.75	0.18084	34.3164985957873	34.3164985957873\\
56.75	0.1845	36.3369171013218	36.3369171013218\\
56.75	0.18816	38.4237255573185	38.4237255573185\\
56.75	0.19182	40.5769239637777	40.5769239637777\\
56.75	0.19548	42.7965123206992	42.7965123206992\\
56.75	0.19914	45.0824906280832	45.0824906280832\\
56.75	0.2028	47.4348588859295	47.4348588859295\\
56.75	0.20646	49.8536170942382	49.8536170942382\\
56.75	0.21012	52.3387652530092	52.3387652530092\\
56.75	0.21378	54.8903033622426	54.8903033622426\\
56.75	0.21744	57.5082314219385	57.5082314219385\\
56.75	0.2211	60.1925494320967	60.1925494320967\\
56.75	0.22476	62.9432573927173	62.9432573927173\\
56.75	0.22842	65.7603553038003	65.7603553038003\\
56.75	0.23208	68.6438431653456	68.6438431653456\\
56.75	0.23574	71.5937209773533	71.5937209773533\\
56.75	0.2394	74.6099887398234	74.6099887398234\\
56.75	0.24306	77.6926464527559	77.6926464527559\\
56.75	0.24672	80.8416941161508	80.8416941161508\\
56.75	0.25038	84.0571317300081	84.0571317300081\\
56.75	0.25404	87.3389592943277	87.3389592943277\\
56.75	0.2577	90.6871768091097	90.6871768091097\\
56.75	0.26136	94.1017842743541	94.1017842743541\\
56.75	0.26502	97.5827816900609	97.5827816900609\\
56.75	0.26868	101.13016905623	101.13016905623\\
56.75	0.27234	104.743946372862	104.743946372862\\
56.75	0.276	108.424113639955	108.424113639955\\
57.125	0.093	5.74770721088454	5.74770721088454\\
57.125	0.09666	6.17518404465145	6.17518404465145\\
57.125	0.10032	6.66905082888075	6.66905082888075\\
57.125	0.10398	7.22930756357241	7.22930756357241\\
57.125	0.10764	7.85595424872646	7.85595424872646\\
57.125	0.1113	8.54899088434291	8.54899088434291\\
57.125	0.11496	9.30841747042173	9.30841747042173\\
57.125	0.11862	10.1342340069629	10.1342340069629\\
57.125	0.12228	11.0264404939665	11.0264404939665\\
57.125	0.12594	11.9850369314325	11.9850369314325\\
57.125	0.1296	13.0100233193608	13.0100233193608\\
57.125	0.13326	14.1013996577515	14.1013996577515\\
57.125	0.13692	15.2591659466046	15.2591659466046\\
57.125	0.14058	16.4833221859201	16.4833221859201\\
57.125	0.14424	17.773868375698	17.773868375698\\
57.125	0.1479	19.1308045159382	19.1308045159382\\
57.125	0.15156	20.5541306066409	20.5541306066409\\
57.125	0.15522	22.0438466478059	22.0438466478059\\
57.125	0.15888	23.5999526394333	23.5999526394333\\
57.125	0.16254	25.2224485815231	25.2224485815231\\
57.125	0.1662	26.9113344740752	26.9113344740752\\
57.125	0.16986	28.6666103170897	28.6666103170897\\
57.125	0.17352	30.4882761105666	30.4882761105666\\
57.125	0.17718	32.3763318545059	32.3763318545059\\
57.125	0.18084	34.3307775489076	34.3307775489076\\
57.125	0.1845	36.3516131937717	36.3516131937717\\
57.125	0.18816	38.4388387890981	38.4388387890981\\
57.125	0.19182	40.592454334887	40.592454334887\\
57.125	0.19548	42.8124598311382	42.8124598311382\\
57.125	0.19914	45.0988552778518	45.0988552778518\\
57.125	0.2028	47.4516406750277	47.4516406750277\\
57.125	0.20646	49.8708160226661	49.8708160226661\\
57.125	0.21012	52.3563813207668	52.3563813207668\\
57.125	0.21378	54.9083365693299	54.9083365693299\\
57.125	0.21744	57.5266817683554	57.5266817683554\\
57.125	0.2211	60.2114169178433	60.2114169178433\\
57.125	0.22476	62.9625420177935	62.9625420177935\\
57.125	0.22842	65.7800570682062	65.7800570682062\\
57.125	0.23208	68.6639620690812	68.6639620690812\\
57.125	0.23574	71.6142570204186	71.6142570204186\\
57.125	0.2394	74.6309419222184	74.6309419222184\\
57.125	0.24306	77.7140167744805	77.7140167744805\\
57.125	0.24672	80.8634815772051	80.8634815772051\\
57.125	0.25038	84.079336330392	84.079336330392\\
57.125	0.25404	87.3615810340413	87.3615810340413\\
57.125	0.2577	90.710215688153	90.710215688153\\
57.125	0.26136	94.1252402927271	94.1252402927271\\
57.125	0.26502	97.6066548477635	97.6066548477635\\
57.125	0.26868	101.154459353262	101.154459353262\\
57.125	0.27234	104.768653809223	104.768653809223\\
57.125	0.276	108.449238215647	108.449238215647\\
57.5	0.093	5.75571719528687	5.75571719528687\\
57.5	0.09666	6.18361116838344	6.18361116838344\\
57.5	0.10032	6.67789509194241	6.67789509194241\\
57.5	0.10398	7.23856896596375	7.23856896596375\\
57.5	0.10764	7.86563279044746	7.86563279044746\\
57.5	0.1113	8.55908656539355	8.55908656539355\\
57.5	0.11496	9.31893029080205	9.31893029080205\\
57.5	0.11862	10.1451639666729	10.1451639666729\\
57.5	0.12228	11.0377875930062	11.0377875930062\\
57.5	0.12594	11.9968011698018	11.9968011698018\\
57.5	0.1296	13.0222046970598	13.0222046970598\\
57.5	0.13326	14.1139981747802	14.1139981747802\\
57.5	0.13692	15.272181602963	15.272181602963\\
57.5	0.14058	16.4967549816081	16.4967549816081\\
57.5	0.14424	17.7877183107156	17.7877183107156\\
57.5	0.1479	19.1450715902856	19.1450715902856\\
57.5	0.15156	20.5688148203178	20.5688148203178\\
57.5	0.15522	22.0589480008125	22.0589480008125\\
57.5	0.15888	23.6154711317696	23.6154711317696\\
57.5	0.16254	25.238384213189	25.238384213189\\
57.5	0.1662	26.9276872450709	26.9276872450709\\
57.5	0.16986	28.6833802274151	28.6833802274151\\
57.5	0.17352	30.5054631602217	30.5054631602217\\
57.5	0.17718	32.3939360434906	32.3939360434906\\
57.5	0.18084	34.3487988772219	34.3487988772219\\
57.5	0.1845	36.3700516614157	36.3700516614157\\
57.5	0.18816	38.4576943960718	38.4576943960718\\
57.5	0.19182	40.6117270811903	40.6117270811903\\
57.5	0.19548	42.8321497167711	42.8321497167711\\
57.5	0.19914	45.1189623028144	45.1189623028144\\
57.5	0.2028	47.4721648393201	47.4721648393201\\
57.5	0.20646	49.8917573262881	49.8917573262881\\
57.5	0.21012	52.3777397637185	52.3777397637185\\
57.5	0.21378	54.9301121516112	54.9301121516112\\
57.5	0.21744	57.5488744899664	57.5488744899664\\
57.5	0.2211	60.2340267787839	60.2340267787839\\
57.5	0.22476	62.9855690180638	62.9855690180638\\
57.5	0.22842	65.8035012078062	65.8035012078062\\
57.5	0.23208	68.6878233480108	68.6878233480108\\
57.5	0.23574	71.6385354386779	71.6385354386779\\
57.5	0.2394	74.6556374798074	74.6556374798074\\
57.5	0.24306	77.7391294713992	77.7391294713992\\
57.5	0.24672	80.8890114134534	80.8890114134534\\
57.5	0.25038	84.10528330597	84.10528330597\\
57.5	0.25404	87.3879451489489	87.3879451489489\\
57.5	0.2577	90.7369969423903	90.7369969423903\\
57.5	0.26136	94.1524386862941	94.1524386862941\\
57.5	0.26502	97.6342703806602	97.6342703806602\\
57.5	0.26868	101.182492025489	101.182492025489\\
57.5	0.27234	104.79710362078	104.79710362078\\
57.5	0.276	108.478105166533	108.478105166533\\
57.875	0.093	5.76746955488329	5.76746955488329\\
57.875	0.09666	6.19578066730953	6.19578066730953\\
57.875	0.10032	6.69048173019815	6.69048173019815\\
57.875	0.10398	7.25157274354916	7.25157274354916\\
57.875	0.10764	7.87905370736255	7.87905370736255\\
57.875	0.1113	8.57292462163833	8.57292462163833\\
57.875	0.11496	9.33318548637649	9.33318548637649\\
57.875	0.11862	10.159836301577	10.159836301577\\
57.875	0.12228	11.0528770672399	11.0528770672399\\
57.875	0.12594	12.0123077833652	12.0123077833652\\
57.875	0.1296	13.0381284499529	13.0381284499529\\
57.875	0.13326	14.1303390670029	14.1303390670029\\
57.875	0.13692	15.2889396345154	15.2889396345154\\
57.875	0.14058	16.5139301524902	16.5139301524902\\
57.875	0.14424	17.8053106209274	17.8053106209274\\
57.875	0.1479	19.163081039827	19.163081039827\\
57.875	0.15156	20.587241409189	20.587241409189\\
57.875	0.15522	22.0777917290133	22.0777917290133\\
57.875	0.15888	23.6347319993	23.6347319993\\
57.875	0.16254	25.2580622200491	25.2580622200491\\
57.875	0.1662	26.9477823912606	26.9477823912606\\
57.875	0.16986	28.7038925129345	28.7038925129345\\
57.875	0.17352	30.5263925850707	30.5263925850707\\
57.875	0.17718	32.4152826076694	32.4152826076694\\
57.875	0.18084	34.3705625807304	34.3705625807304\\
57.875	0.1845	36.3922325042538	36.3922325042538\\
57.875	0.18816	38.4802923782396	38.4802923782396\\
57.875	0.19182	40.6347422026877	40.6347422026877\\
57.875	0.19548	42.8555819775983	42.8555819775983\\
57.875	0.19914	45.1428117029712	45.1428117029712\\
57.875	0.2028	47.4964313788065	47.4964313788065\\
57.875	0.20646	49.9164410051042	49.9164410051042\\
57.875	0.21012	52.4028405818642	52.4028405818642\\
57.875	0.21378	54.9556301090867	54.9556301090867\\
57.875	0.21744	57.5748095867715	57.5748095867715\\
57.875	0.2211	60.2603790149187	60.2603790149187\\
57.875	0.22476	63.0123383935283	63.0123383935283\\
57.875	0.22842	65.8306877226003	65.8306877226003\\
57.875	0.23208	68.7154270021346	68.7154270021346\\
57.875	0.23574	71.6665562321313	71.6665562321313\\
57.875	0.2394	74.6840754125905	74.6840754125905\\
57.875	0.24306	77.7679845435119	77.7679845435119\\
57.875	0.24672	80.9182836248958	80.9182836248958\\
57.875	0.25038	84.1349726567421	84.1349726567421\\
57.875	0.25404	87.4180516390507	87.4180516390507\\
57.875	0.2577	90.7675205718218	90.7675205718218\\
57.875	0.26136	94.1833794550552	94.1833794550552\\
57.875	0.26502	97.6656282887509	97.6656282887509\\
57.875	0.26868	101.214267072909	101.214267072909\\
57.875	0.27234	104.82929580753	104.82929580753\\
57.875	0.276	108.510714492613	108.510714492613\\
58.25	0.093	5.78296428967383	5.78296428967383\\
58.25	0.09666	6.21169254142973	6.21169254142973\\
58.25	0.10032	6.70681074364802	6.70681074364802\\
58.25	0.10398	7.26831889632869	7.26831889632869\\
58.25	0.10764	7.89621699947175	7.89621699947175\\
58.25	0.1113	8.5905050530772	8.5905050530772\\
58.25	0.11496	9.35118305714501	9.35118305714501\\
58.25	0.11862	10.1782510116752	10.1782510116752\\
58.25	0.12228	11.0717089166678	11.0717089166678\\
58.25	0.12594	12.0315567721228	12.0315567721228\\
58.25	0.1296	13.0577945780401	13.0577945780401\\
58.25	0.13326	14.1504223344198	14.1504223344198\\
58.25	0.13692	15.3094400412619	15.3094400412619\\
58.25	0.14058	16.5348476985664	16.5348476985664\\
58.25	0.14424	17.8266453063333	17.8266453063333\\
58.25	0.1479	19.1848328645625	19.1848328645625\\
58.25	0.15156	20.6094103732542	20.6094103732542\\
58.25	0.15522	22.1003778324082	22.1003778324082\\
58.25	0.15888	23.6577352420246	23.6577352420246\\
58.25	0.16254	25.2814826021033	25.2814826021033\\
58.25	0.1662	26.9716199126445	26.9716199126445\\
58.25	0.16986	28.728147173648	28.728147173648\\
58.25	0.17352	30.551064385114	30.551064385114\\
58.25	0.17718	32.4403715470422	32.4403715470422\\
58.25	0.18084	34.3960686594329	34.3960686594329\\
58.25	0.1845	36.418155722286	36.418155722286\\
58.25	0.18816	38.5066327356014	38.5066327356014\\
58.25	0.19182	40.6614996993793	40.6614996993793\\
58.25	0.19548	42.8827566136195	42.8827566136195\\
58.25	0.19914	45.1704034783221	45.1704034783221\\
58.25	0.2028	47.524440293487	47.524440293487\\
58.25	0.20646	49.9448670591144	49.9448670591144\\
58.25	0.21012	52.4316837752041	52.4316837752041\\
58.25	0.21378	54.9848904417562	54.9848904417562\\
58.25	0.21744	57.6044870587707	57.6044870587707\\
58.25	0.2211	60.2904736262476	60.2904736262476\\
58.25	0.22476	63.0428501441868	63.0428501441868\\
58.25	0.22842	65.8616166125885	65.8616166125885\\
58.25	0.23208	68.7467730314525	68.7467730314525\\
58.25	0.23574	71.6983194007789	71.6983194007789\\
58.25	0.2394	74.7162557205677	74.7162557205677\\
58.25	0.24306	77.8005819908188	77.8005819908188\\
58.25	0.24672	80.9512982115324	80.9512982115324\\
58.25	0.25038	84.1684043827083	84.1684043827083\\
58.25	0.25404	87.4519005043466	87.4519005043466\\
58.25	0.2577	90.8017865764473	90.8017865764473\\
58.25	0.26136	94.2180625990104	94.2180625990104\\
58.25	0.26502	97.7007285720358	97.7007285720358\\
58.25	0.26868	101.249784495524	101.249784495524\\
58.25	0.27234	104.865230369474	104.865230369474\\
58.25	0.276	108.547066193886	108.547066193886\\
58.625	0.093	5.80220139965844	5.80220139965844\\
58.625	0.09666	6.23134679074402	6.23134679074402\\
58.625	0.10032	6.72688213229199	6.72688213229199\\
58.625	0.10398	7.28880742430233	7.28880742430233\\
58.625	0.10764	7.91712266677504	7.91712266677504\\
58.625	0.1113	8.61182785971013	8.61182785971013\\
58.625	0.11496	9.37292300310763	9.37292300310763\\
58.625	0.11862	10.2004080969675	10.2004080969675\\
58.625	0.12228	11.0942831412897	11.0942831412897\\
58.625	0.12594	12.0545481360744	12.0545481360744\\
58.625	0.1296	13.0812030813214	13.0812030813214\\
58.625	0.13326	14.1742479770308	14.1742479770308\\
58.625	0.13692	15.3336828232025	15.3336828232025\\
58.625	0.14058	16.5595076198367	16.5595076198367\\
58.625	0.14424	17.8517223669332	17.8517223669332\\
58.625	0.1479	19.2103270644922	19.2103270644922\\
58.625	0.15156	20.6353217125134	20.6353217125134\\
58.625	0.15522	22.1267063109971	22.1267063109971\\
58.625	0.15888	23.6844808599432	23.6844808599432\\
58.625	0.16254	25.3086453593516	25.3086453593516\\
58.625	0.1662	26.9991998092225	26.9991998092225\\
58.625	0.16986	28.7561442095557	28.7561442095557\\
58.625	0.17352	30.5794785603513	30.5794785603513\\
58.625	0.17718	32.4692028616092	32.4692028616092\\
58.625	0.18084	34.4253171133296	34.4253171133296\\
58.625	0.1845	36.4478213155123	36.4478213155123\\
58.625	0.18816	38.5367154681574	38.5367154681574\\
58.625	0.19182	40.6919995712649	40.6919995712649\\
58.625	0.19548	42.9136736248348	42.9136736248348\\
58.625	0.19914	45.201737628867	45.201737628867\\
58.625	0.2028	47.5561915833617	47.5561915833617\\
58.625	0.20646	49.9770354883187	49.9770354883187\\
58.625	0.21012	52.4642693437381	52.4642693437381\\
58.625	0.21378	55.0178931496199	55.0178931496199\\
58.625	0.21744	57.637906905964	57.637906905964\\
58.625	0.2211	60.3243106127705	60.3243106127705\\
58.625	0.22476	63.0771042700395	63.0771042700395\\
58.625	0.22842	65.8962878777708	65.8962878777708\\
58.625	0.23208	68.7818614359644	68.7818614359644\\
58.625	0.23574	71.7338249446206	71.7338249446206\\
58.625	0.2394	74.752178403739	74.752178403739\\
58.625	0.24306	77.8369218133198	77.8369218133198\\
58.625	0.24672	80.988055173363	80.988055173363\\
58.625	0.25038	84.2055784838686	84.2055784838686\\
58.625	0.25404	87.4894917448366	87.4894917448366\\
58.625	0.2577	90.8397949562669	90.8397949562669\\
58.625	0.26136	94.2564881181597	94.2564881181597\\
58.625	0.26502	97.7395712305148	97.7395712305148\\
58.625	0.26868	101.289044293332	101.289044293332\\
58.625	0.27234	104.904907306612	104.904907306612\\
58.625	0.276	108.587160270354	108.587160270354\\
59	0.093	5.82518088483718	5.82518088483718\\
59	0.09666	6.25474341525242	6.25474341525242\\
59	0.10032	6.75069589613006	6.75069589613006\\
59	0.10398	7.31303832747006	7.31303832747006\\
59	0.10764	7.94177070927245	7.94177070927245\\
59	0.1113	8.63689304153721	8.63689304153721\\
59	0.11496	9.39840532426436	9.39840532426436\\
59	0.11862	10.2263075574539	10.2263075574539\\
59	0.12228	11.1205997411058	11.1205997411058\\
59	0.12594	12.0812818752201	12.0812818752201\\
59	0.1296	13.1083539597968	13.1083539597968\\
59	0.13326	14.2018159948358	14.2018159948358\\
59	0.13692	15.3616679803373	15.3616679803373\\
59	0.14058	16.5879099163011	16.5879099163011\\
59	0.14424	17.8805418027273	17.8805418027273\\
59	0.1479	19.2395636396159	19.2395636396159\\
59	0.15156	20.6649754269669	20.6649754269669\\
59	0.15522	22.1567771647802	22.1567771647802\\
59	0.15888	23.7149688530559	23.7149688530559\\
59	0.16254	25.339550491794	25.339550491794\\
59	0.1662	27.0305220809945	27.0305220809945\\
59	0.16986	28.7878836206574	28.7878836206574\\
59	0.17352	30.6116351107827	30.6116351107827\\
59	0.17718	32.5017765513703	32.5017765513703\\
59	0.18084	34.4583079424203	34.4583079424203\\
59	0.1845	36.4812292839327	36.4812292839327\\
59	0.18816	38.5705405759075	38.5705405759075\\
59	0.19182	40.7262418183446	40.7262418183446\\
59	0.19548	42.9483330112442	42.9483330112442\\
59	0.19914	45.2368141546061	45.2368141546061\\
59	0.2028	47.5916852484304	47.5916852484304\\
59	0.20646	50.0129462927171	50.0129462927171\\
59	0.21012	52.5005972874662	52.5005972874662\\
59	0.21378	55.0546382326776	55.0546382326776\\
59	0.21744	57.6750691283514	57.6750691283514\\
59	0.2211	60.3618899744876	60.3618899744876\\
59	0.22476	63.1151007710862	63.1151007710862\\
59	0.22842	65.9347015181472	65.9347015181472\\
59	0.23208	68.8206922156705	68.8206922156705\\
59	0.23574	71.7730728636563	71.7730728636563\\
59	0.2394	74.7918434621044	74.7918434621044\\
59	0.24306	77.8770040110149	77.8770040110149\\
59	0.24672	81.0285545103878	81.0285545103878\\
59	0.25038	84.246494960223	84.246494960223\\
59	0.25404	87.5308253605207	87.5308253605207\\
59	0.2577	90.8815457112807	90.8815457112807\\
59	0.26136	94.2986560125031	94.2986560125031\\
59	0.26502	97.7821562641878	97.7821562641878\\
59	0.26868	101.332046466335	101.332046466335\\
59	0.27234	104.948326618945	104.948326618945\\
59	0.276	108.630996722016	108.630996722016\\
59.375	0.093	5.85190274520999	5.85190274520999\\
59.375	0.09666	6.28188241495491	6.28188241495491\\
59.375	0.10032	6.77825203516219	6.77825203516219\\
59.375	0.10398	7.34101160583186	7.34101160583186\\
59.375	0.10764	7.97016112696393	7.97016112696393\\
59.375	0.1113	8.66570059855837	8.66570059855837\\
59.375	0.11496	9.4276300206152	9.4276300206152\\
59.375	0.11862	10.2559493931344	10.2559493931344\\
59.375	0.12228	11.150658716116	11.150658716116\\
59.375	0.12594	12.1117579895599	12.1117579895599\\
59.375	0.1296	13.1392472134663	13.1392472134663\\
59.375	0.13326	14.233126387835	14.233126387835\\
59.375	0.13692	15.3933955126661	15.3933955126661\\
59.375	0.14058	16.6200545879596	16.6200545879596\\
59.375	0.14424	17.9131036137155	17.9131036137155\\
59.375	0.1479	19.2725425899337	19.2725425899337\\
59.375	0.15156	20.6983715166143	20.6983715166143\\
59.375	0.15522	22.1905903937574	22.1905903937574\\
59.375	0.15888	23.7491992213628	23.7491992213628\\
59.375	0.16254	25.3741979994305	25.3741979994305\\
59.375	0.1662	27.0655867279607	27.0655867279607\\
59.375	0.16986	28.8233654069532	28.8233654069532\\
59.375	0.17352	30.6475340364081	30.6475340364081\\
59.375	0.17718	32.5380926163254	32.5380926163254\\
59.375	0.18084	34.4950411467051	34.4950411467051\\
59.375	0.1845	36.5183796275472	36.5183796275472\\
59.375	0.18816	38.6081080588516	38.6081080588516\\
59.375	0.19182	40.7642264406185	40.7642264406185\\
59.375	0.19548	42.9867347728477	42.9867347728477\\
59.375	0.19914	45.2756330555393	45.2756330555393\\
59.375	0.2028	47.6309212886932	47.6309212886932\\
59.375	0.20646	50.0525994723096	50.0525994723096\\
59.375	0.21012	52.5406676063883	52.5406676063883\\
59.375	0.21378	55.0951256909294	55.0951256909294\\
59.375	0.21744	57.7159737259329	57.7159737259329\\
59.375	0.2211	60.4032117113988	60.4032117113988\\
59.375	0.22476	63.156839647327	63.156839647327\\
59.375	0.22842	65.9768575337177	65.9768575337177\\
59.375	0.23208	68.8632653705707	68.8632653705707\\
59.375	0.23574	71.8160631578861	71.8160631578861\\
59.375	0.2394	74.8352508956639	74.8352508956639\\
59.375	0.24306	77.920828583904	77.920828583904\\
59.375	0.24672	81.0727962226066	81.0727962226066\\
59.375	0.25038	84.2911538117715	84.2911538117715\\
59.375	0.25404	87.5759013513988	87.5759013513988\\
59.375	0.2577	90.9270388414885	90.9270388414885\\
59.375	0.26136	94.3445662820406	94.3445662820406\\
59.375	0.26502	97.828483673055	97.828483673055\\
59.375	0.26868	101.378791014532	101.378791014532\\
59.375	0.27234	104.995488306471	104.995488306471\\
59.375	0.276	108.678575548873	108.678575548873\\
59.75	0.093	5.8823669807769	5.8823669807769\\
59.75	0.09666	6.31276378985148	6.31276378985148\\
59.75	0.10032	6.80955054938845	6.80955054938845\\
59.75	0.10398	7.37272725938779	7.37272725938779\\
59.75	0.10764	8.00229391984949	8.00229391984949\\
59.75	0.1113	8.69825053077359	8.69825053077359\\
59.75	0.11496	9.46059709216009	9.46059709216009\\
59.75	0.11862	10.289333604009	10.289333604009\\
59.75	0.12228	11.1844600663202	11.1844600663202\\
59.75	0.12594	12.1459764790938	12.1459764790938\\
59.75	0.1296	13.1738828423298	13.1738828423298\\
59.75	0.13326	14.2681791560282	14.2681791560282\\
59.75	0.13692	15.428865420189	15.428865420189\\
59.75	0.14058	16.6559416348122	16.6559416348122\\
59.75	0.14424	17.9494077998977	17.9494077998977\\
59.75	0.1479	19.3092639154456	19.3092639154456\\
59.75	0.15156	20.7355099814559	20.7355099814559\\
59.75	0.15522	22.2281459979286	22.2281459979286\\
59.75	0.15888	23.7871719648637	23.7871719648637\\
59.75	0.16254	25.4125878822611	25.4125878822611\\
59.75	0.1662	27.1043937501209	27.1043937501209\\
59.75	0.16986	28.8625895684431	28.8625895684431\\
59.75	0.17352	30.6871753372277	30.6871753372277\\
59.75	0.17718	32.5781510564747	32.5781510564747\\
59.75	0.18084	34.535516726184	34.535516726184\\
59.75	0.1845	36.5592723463558	36.5592723463558\\
59.75	0.18816	38.6494179169899	38.6494179169899\\
59.75	0.19182	40.8059534380864	40.8059534380864\\
59.75	0.19548	43.0288789096452	43.0288789096452\\
59.75	0.19914	45.3181943316665	45.3181943316665\\
59.75	0.2028	47.6738997041501	47.6738997041501\\
59.75	0.20646	50.0959950270962	50.0959950270962\\
59.75	0.21012	52.5844803005046	52.5844803005046\\
59.75	0.21378	55.1393555243753	55.1393555243753\\
59.75	0.21744	57.7606206987085	57.7606206987085\\
59.75	0.2211	60.448275823504	60.448275823504\\
59.75	0.22476	63.202320898762	63.202320898762\\
59.75	0.22842	66.0227559244823	66.0227559244823\\
59.75	0.23208	68.9095809006649	68.9095809006649\\
59.75	0.23574	71.86279582731	71.86279582731\\
59.75	0.2394	74.8824007044175	74.8824007044175\\
59.75	0.24306	77.9683955319873	77.9683955319873\\
59.75	0.24672	81.1207803100195	81.1207803100195\\
59.75	0.25038	84.3395550385141	84.3395550385141\\
59.75	0.25404	87.6247197174711	87.6247197174711\\
59.75	0.2577	90.9762743468904	90.9762743468904\\
59.75	0.26136	94.3942189267722	94.3942189267722\\
59.75	0.26502	97.8785534571163	97.8785534571163\\
59.75	0.26868	101.429277937923	101.429277937923\\
59.75	0.27234	105.046392369192	105.046392369192\\
59.75	0.276	108.729896750923	108.729896750923\\
60.125	0.093	5.91657359153793	5.91657359153793\\
60.125	0.09666	6.34738753994216	6.34738753994216\\
60.125	0.10032	6.8445914388088	6.8445914388088\\
60.125	0.10398	7.40818528813782	7.40818528813782\\
60.125	0.10764	8.03816908792918	8.03816908792918\\
60.125	0.1113	8.73454283818295	8.73454283818295\\
60.125	0.11496	9.49730653889912	9.49730653889912\\
60.125	0.11862	10.3264601900776	10.3264601900776\\
60.125	0.12228	11.2220037917186	11.2220037917186\\
60.125	0.12594	12.1839373438219	12.1839373438219\\
60.125	0.1296	13.2122608463875	13.2122608463875\\
60.125	0.13326	14.3069742994156	14.3069742994156\\
60.125	0.13692	15.468077702906	15.468077702906\\
60.125	0.14058	16.6955710568589	16.6955710568589\\
60.125	0.14424	17.9894543612741	17.9894543612741\\
60.125	0.1479	19.3497276161517	19.3497276161517\\
60.125	0.15156	20.7763908214916	20.7763908214916\\
60.125	0.15522	22.269443977294	22.269443977294\\
60.125	0.15888	23.8288870835587	23.8288870835587\\
60.125	0.16254	25.4547201402858	25.4547201402858\\
60.125	0.1662	27.1469431474753	27.1469431474753\\
60.125	0.16986	28.9055561051272	28.9055561051272\\
60.125	0.17352	30.7305590132414	30.7305590132414\\
60.125	0.17718	32.6219518718181	32.6219518718181\\
60.125	0.18084	34.5797346808571	34.5797346808571\\
60.125	0.1845	36.6039074403585	36.6039074403585\\
60.125	0.18816	38.6944701503222	38.6944701503222\\
60.125	0.19182	40.8514228107484	40.8514228107484\\
60.125	0.19548	43.0747654216369	43.0747654216369\\
60.125	0.19914	45.3644979829879	45.3644979829879\\
60.125	0.2028	47.7206204948012	47.7206204948012\\
60.125	0.20646	50.1431329570769	50.1431329570769\\
60.125	0.21012	52.6320353698149	52.6320353698149\\
60.125	0.21378	55.1873277330154	55.1873277330154\\
60.125	0.21744	57.8090100466782	57.8090100466782\\
60.125	0.2211	60.4970823108034	60.4970823108034\\
60.125	0.22476	63.251544525391	63.251544525391\\
60.125	0.22842	66.072396690441	66.072396690441\\
60.125	0.23208	68.9596388059533	68.9596388059533\\
60.125	0.23574	71.9132708719281	71.9132708719281\\
60.125	0.2394	74.9332928883652	74.9332928883652\\
60.125	0.24306	78.0197048552647	78.0197048552647\\
60.125	0.24672	81.1725067726265	81.1725067726265\\
60.125	0.25038	84.3916986404508	84.3916986404508\\
60.125	0.25404	87.6772804587374	87.6772804587374\\
60.125	0.2577	91.0292522274865	91.0292522274865\\
60.125	0.26136	94.4476139466979	94.4476139466979\\
60.125	0.26502	97.9323656163716	97.9323656163716\\
60.125	0.26868	101.483507236508	101.483507236508\\
60.125	0.27234	105.101038807106	105.101038807106\\
60.125	0.276	108.784960328167	108.784960328167\\
60.5	0.093	5.95452257749304	5.95452257749304\\
60.5	0.09666	6.38575366522694	6.38575366522694\\
60.5	0.10032	6.88337470342323	6.88337470342323\\
60.5	0.10398	7.44738569208191	7.44738569208191\\
60.5	0.10764	8.07778663120296	8.07778663120296\\
60.5	0.1113	8.77457752078642	8.77457752078642\\
60.5	0.11496	9.53775836083224	9.53775836083224\\
60.5	0.11862	10.3673291513404	10.3673291513404\\
60.5	0.12228	11.263289892311	11.263289892311\\
60.5	0.12594	12.225640583744	12.225640583744\\
60.5	0.1296	13.2543812256393	13.2543812256393\\
60.5	0.13326	14.349511817997	14.349511817997\\
60.5	0.13692	15.5110323608171	15.5110323608171\\
60.5	0.14058	16.7389428540996	16.7389428540996\\
60.5	0.14424	18.0332432978445	18.0332432978445\\
60.5	0.1479	19.3939336920518	19.3939336920518\\
60.5	0.15156	20.8210140367214	20.8210140367214\\
60.5	0.15522	22.3144843318534	22.3144843318534\\
60.5	0.15888	23.8743445774478	23.8743445774478\\
60.5	0.16254	25.5005947735046	25.5005947735046\\
60.5	0.1662	27.1932349200237	27.1932349200237\\
60.5	0.16986	28.9522650170053	28.9522650170053\\
60.5	0.17352	30.7776850644492	30.7776850644492\\
60.5	0.17718	32.6694950623555	32.6694950623555\\
60.5	0.18084	34.6276950107242	34.6276950107242\\
60.5	0.1845	36.6522849095552	36.6522849095552\\
60.5	0.18816	38.7432647588487	38.7432647588487\\
60.5	0.19182	40.9006345586045	40.9006345586045\\
60.5	0.19548	43.1243943088227	43.1243943088227\\
60.5	0.19914	45.4145440095033	45.4145440095033\\
60.5	0.2028	47.7710836606463	47.7710836606463\\
60.5	0.20646	50.1940132622517	50.1940132622517\\
60.5	0.21012	52.6833328143194	52.6833328143194\\
60.5	0.21378	55.2390423168495	55.2390423168495\\
60.5	0.21744	57.861141769842	57.861141769842\\
60.5	0.2211	60.5496311732969	60.5496311732969\\
60.5	0.22476	63.3045105272141	63.3045105272141\\
60.5	0.22842	66.1257798315938	66.1257798315938\\
60.5	0.23208	69.0134390864358	69.0134390864358\\
60.5	0.23574	71.9674882917402	71.9674882917402\\
60.5	0.2394	74.987927447507	74.987927447507\\
60.5	0.24306	78.0747565537361	78.0747565537361\\
60.5	0.24672	81.2279756104277	81.2279756104277\\
60.5	0.25038	84.4475846175816	84.4475846175816\\
60.5	0.25404	87.7335835751979	87.7335835751979\\
60.5	0.2577	91.0859724832766	91.0859724832766\\
60.5	0.26136	94.5047513418177	94.5047513418177\\
60.5	0.26502	97.9899201508211	97.9899201508211\\
60.5	0.26868	101.541478910287	101.541478910287\\
60.5	0.27234	105.159427620215	105.159427620215\\
60.5	0.276	108.843766280606	108.843766280606\\
60.875	0.093	5.99621393864223	5.99621393864223\\
60.875	0.09666	6.4278621657058	6.4278621657058\\
60.875	0.10032	6.92590034323178	6.92590034323178\\
60.875	0.10398	7.49032847122012	7.49032847122012\\
60.875	0.10764	8.12114654967083	8.12114654967083\\
60.875	0.1113	8.81835457858393	8.81835457858393\\
60.875	0.11496	9.58195255795943	9.58195255795943\\
60.875	0.11862	10.4119404877973	10.4119404877973\\
60.875	0.12228	11.3083183680975	11.3083183680975\\
60.875	0.12594	12.2710861988602	12.2710861988602\\
60.875	0.1296	13.3002439800852	13.3002439800852\\
60.875	0.13326	14.3957917117726	14.3957917117726\\
60.875	0.13692	15.5577293939224	15.5577293939224\\
60.875	0.14058	16.7860570265345	16.7860570265345\\
60.875	0.14424	18.0807746096091	18.0807746096091\\
60.875	0.1479	19.441882143146	19.441882143146\\
60.875	0.15156	20.8693796271453	20.8693796271453\\
60.875	0.15522	22.363267061607	22.363267061607\\
60.875	0.15888	23.923544446531	23.923544446531\\
60.875	0.16254	25.5502117819175	25.5502117819175\\
60.875	0.1662	27.2432690677663	27.2432690677663\\
60.875	0.16986	29.0027163040775	29.0027163040775\\
60.875	0.17352	30.8285534908511	30.8285534908511\\
60.875	0.17718	32.720780628087	32.720780628087\\
60.875	0.18084	34.6793977157854	34.6793977157854\\
60.875	0.1845	36.7044047539461	36.7044047539461\\
60.875	0.18816	38.7958017425692	38.7958017425692\\
60.875	0.19182	40.9535886816547	40.9535886816547\\
60.875	0.19548	43.1777655712026	43.1777655712026\\
60.875	0.19914	45.4683324112129	45.4683324112129\\
60.875	0.2028	47.8252892016855	47.8252892016855\\
60.875	0.20646	50.2486359426205	50.2486359426205\\
60.875	0.21012	52.7383726340179	52.7383726340179\\
60.875	0.21378	55.2944992758777	55.2944992758777\\
60.875	0.21744	57.9170158681998	57.9170158681998\\
60.875	0.2211	60.6059224109844	60.6059224109844\\
60.875	0.22476	63.3612189042313	63.3612189042313\\
60.875	0.22842	66.1829053479406	66.1829053479406\\
60.875	0.23208	69.0709817421123	69.0709817421123\\
60.875	0.23574	72.0254480867464	72.0254480867464\\
60.875	0.2394	75.0463043818429	75.0463043818429\\
60.875	0.24306	78.1335506274017	78.1335506274017\\
60.875	0.24672	81.2871868234229	81.2871868234229\\
60.875	0.25038	84.5072129699065	84.5072129699065\\
60.875	0.25404	87.7936290668524	87.7936290668524\\
60.875	0.2577	91.1464351142608	91.1464351142608\\
60.875	0.26136	94.5656311121315	94.5656311121315\\
60.875	0.26502	98.0512170604646	98.0512170604646\\
60.875	0.26868	101.60319295926	101.60319295926\\
60.875	0.27234	105.221558808518	105.221558808518\\
60.875	0.276	108.906314608238	108.906314608238\\
61.25	0.093	6.04164767498554	6.04164767498554\\
61.25	0.09666	6.47371304137879	6.47371304137879\\
61.25	0.10032	6.97216835823442	6.97216835823442\\
61.25	0.10398	7.53701362555243	7.53701362555243\\
61.25	0.10764	8.16824884333281	8.16824884333281\\
61.25	0.1113	8.86587401157558	8.86587401157558\\
61.25	0.11496	9.62988913028074	9.62988913028074\\
61.25	0.11862	10.4602941994483	10.4602941994483\\
61.25	0.12228	11.3570892190782	11.3570892190782\\
61.25	0.12594	12.3202741891705	12.3202741891705\\
61.25	0.1296	13.3498491097252	13.3498491097252\\
61.25	0.13326	14.4458139807422	14.4458139807422\\
61.25	0.13692	15.6081688022217	15.6081688022217\\
61.25	0.14058	16.8369135741635	16.8369135741635\\
61.25	0.14424	18.1320482965677	18.1320482965677\\
61.25	0.1479	19.4935729694343	19.4935729694343\\
61.25	0.15156	20.9214875927632	20.9214875927632\\
61.25	0.15522	22.4157921665546	22.4157921665546\\
61.25	0.15888	23.9764866908083	23.9764866908083\\
61.25	0.16254	25.6035711655244	25.6035711655244\\
61.25	0.1662	27.2970455907029	27.2970455907029\\
61.25	0.16986	29.0569099663438	29.0569099663438\\
61.25	0.17352	30.8831642924471	30.8831642924471\\
61.25	0.17718	32.7758085690127	32.7758085690127\\
61.25	0.18084	34.7348427960407	34.7348427960407\\
61.25	0.1845	36.7602669735311	36.7602669735311\\
61.25	0.18816	38.8520811014839	38.8520811014839\\
61.25	0.19182	41.010285179899	41.010285179899\\
61.25	0.19548	43.2348792087766	43.2348792087766\\
61.25	0.19914	45.5258631881165	45.5258631881165\\
61.25	0.2028	47.8832371179188	47.8832371179188\\
61.25	0.20646	50.3070009981835	50.3070009981835\\
61.25	0.21012	52.7971548289106	52.7971548289106\\
61.25	0.21378	55.3536986101	55.3536986101\\
61.25	0.21744	57.9766323417518	57.9766323417518\\
61.25	0.2211	60.665956023866	60.665956023866\\
61.25	0.22476	63.4216696564426	63.4216696564426\\
61.25	0.22842	66.2437732394816	66.2437732394816\\
61.25	0.23208	69.1322667729829	69.1322667729829\\
61.25	0.23574	72.0871502569467	72.0871502569467\\
61.25	0.2394	75.1084236913728	75.1084236913728\\
61.25	0.24306	78.1960870762613	78.1960870762613\\
61.25	0.24672	81.3501404116122	81.3501404116122\\
61.25	0.25038	84.5705836974255	84.5705836974255\\
61.25	0.25404	87.8574169337011	87.8574169337011\\
61.25	0.2577	91.2106401204391	91.2106401204391\\
61.25	0.26136	94.6302532576395	94.6302532576395\\
61.25	0.26502	98.1162563453023	98.1162563453023\\
61.25	0.26868	101.668649383427	101.668649383427\\
61.25	0.27234	105.287432372015	105.287432372015\\
61.25	0.276	108.972605311065	108.972605311065\\
61.625	0.093	6.09082378652294	6.09082378652294\\
61.625	0.09666	6.52330629224586	6.52330629224586\\
61.625	0.10032	7.02217874843113	7.02217874843113\\
61.625	0.10398	7.58744115507881	7.58744115507881\\
61.625	0.10764	8.21909351218888	8.21909351218888\\
61.625	0.1113	8.91713581976133	8.91713581976133\\
61.625	0.11496	9.68156807779616	9.68156807779616\\
61.625	0.11862	10.5123902862933	10.5123902862933\\
61.625	0.12228	11.4096024452529	11.4096024452529\\
61.625	0.12594	12.3732045546749	12.3732045546749\\
61.625	0.1296	13.4031966145592	13.4031966145592\\
61.625	0.13326	14.499578624906	14.499578624906\\
61.625	0.13692	15.6623505857151	15.6623505857151\\
61.625	0.14058	16.8915124969866	16.8915124969866\\
61.625	0.14424	18.1870643587204	18.1870643587204\\
61.625	0.1479	19.5490061709167	19.5490061709167\\
61.625	0.15156	20.9773379335753	20.9773379335753\\
61.625	0.15522	22.4720596466963	22.4720596466963\\
61.625	0.15888	24.0331713102797	24.0331713102797\\
61.625	0.16254	25.6606729243255	25.6606729243255\\
61.625	0.1662	27.3545644888337	27.3545644888337\\
61.625	0.16986	29.1148460038042	29.1148460038042\\
61.625	0.17352	30.9415174692371	30.9415174692371\\
61.625	0.17718	32.8345788851324	32.8345788851324\\
61.625	0.18084	34.7940302514901	34.7940302514901\\
61.625	0.1845	36.8198715683102	36.8198715683102\\
61.625	0.18816	38.9121028355926	38.9121028355926\\
61.625	0.19182	41.0707240533375	41.0707240533375\\
61.625	0.19548	43.2957352215447	43.2957352215447\\
61.625	0.19914	45.5871363402143	45.5871363402143\\
61.625	0.2028	47.9449274093462	47.9449274093462\\
61.625	0.20646	50.3691084289406	50.3691084289406\\
61.625	0.21012	52.8596793989973	52.8596793989973\\
61.625	0.21378	55.4166403195164	55.4166403195164\\
61.625	0.21744	58.0399911904979	58.0399911904979\\
61.625	0.2211	60.7297320119418	60.7297320119418\\
61.625	0.22476	63.4858627838481	63.4858627838481\\
61.625	0.22842	66.3083835062167	66.3083835062167\\
61.625	0.23208	69.1972941790477	69.1972941790477\\
61.625	0.23574	72.1525948023411	72.1525948023411\\
61.625	0.2394	75.1742853760969	75.1742853760969\\
61.625	0.24306	78.2623659003151	78.2623659003151\\
61.625	0.24672	81.4168363749956	81.4168363749956\\
61.625	0.25038	84.6376968001386	84.6376968001386\\
61.625	0.25404	87.9249471757439	87.9249471757439\\
61.625	0.2577	91.2785875018116	91.2785875018116\\
61.625	0.26136	94.6986177783416	94.6986177783416\\
61.625	0.26502	98.1850380053341	98.1850380053341\\
61.625	0.26868	101.737848182789	101.737848182789\\
61.625	0.27234	105.357048310706	105.357048310706\\
61.625	0.276	109.042638389086	109.042638389086\\
62	0.093	6.14374227325444	6.14374227325444\\
62	0.09666	6.57664191830701	6.57664191830701\\
62	0.10032	7.07593151382199	7.07593151382199\\
62	0.10398	7.64161105979932	7.64161105979932\\
62	0.10764	8.27368055623904	8.27368055623904\\
62	0.1113	8.97214000314113	8.97214000314113\\
62	0.11496	9.73698940050564	9.73698940050564\\
62	0.11862	10.5682287483325	10.5682287483325\\
62	0.12228	11.4658580466218	11.4658580466218\\
62	0.12594	12.4298772953734	12.4298772953734\\
62	0.1296	13.4602864945874	13.4602864945874\\
62	0.13326	14.5570856442638	14.5570856442638\\
62	0.13692	15.7202747444026	15.7202747444026\\
62	0.14058	16.9498537950037	16.9498537950037\\
62	0.14424	18.2458227960673	18.2458227960673\\
62	0.1479	19.6081817475932	19.6081817475932\\
62	0.15156	21.0369306495815	21.0369306495815\\
62	0.15522	22.5320695020322	22.5320695020322\\
62	0.15888	24.0935983049452	24.0935983049452\\
62	0.16254	25.7215170583207	25.7215170583207\\
62	0.1662	27.4158257621585	27.4158257621585\\
62	0.16986	29.1765244164587	29.1765244164587\\
62	0.17352	31.0036130212213	31.0036130212213\\
62	0.17718	32.8970915764463	32.8970915764463\\
62	0.18084	34.8569600821336	34.8569600821336\\
62	0.1845	36.8832185382834	36.8832185382834\\
62	0.18816	38.9758669448955	38.9758669448955\\
62	0.19182	41.13490530197	41.13490530197\\
62	0.19548	43.3603336095068	43.3603336095068\\
62	0.19914	45.6521518675061	45.6521518675061\\
62	0.2028	48.0103600759677	48.0103600759677\\
62	0.20646	50.4349582348918	50.4349582348918\\
62	0.21012	52.9259463442782	52.9259463442782\\
62	0.21378	55.4833244041269	55.4833244041269\\
62	0.21744	58.1070924144381	58.1070924144381\\
62	0.2211	60.7972503752116	60.7972503752116\\
62	0.22476	63.5537982864476	63.5537982864476\\
62	0.22842	66.3767361481459	66.3767361481459\\
62	0.23208	69.2660639603065	69.2660639603065\\
62	0.23574	72.2217817229296	72.2217817229296\\
62	0.2394	75.2438894360151	75.2438894360151\\
62	0.24306	78.3323870995629	78.3323870995629\\
62	0.24672	81.4872747135731	81.4872747135731\\
62	0.25038	84.7085522780457	84.7085522780457\\
62	0.25404	87.9962197929807	87.9962197929807\\
62	0.2577	91.3502772583781	91.3502772583781\\
62	0.26136	94.7707246742378	94.7707246742378\\
62	0.26502	98.2575620405599	98.2575620405599\\
62	0.26868	101.810789357344	101.810789357344\\
62	0.27234	105.430406624591	105.430406624591\\
62	0.276	109.116413842301	109.116413842301\\
62.375	0.093	6.20040313518003	6.20040313518003\\
62.375	0.09666	6.63371991956229	6.63371991956229\\
62.375	0.10032	7.13342665440692	7.13342665440692\\
62.375	0.10398	7.69952333971393	7.69952333971393\\
62.375	0.10764	8.33200997548332	8.33200997548332\\
62.375	0.1113	9.03088656171508	9.03088656171508\\
62.375	0.11496	9.79615309840923	9.79615309840923\\
62.375	0.11862	10.6278095855658	10.6278095855658\\
62.375	0.12228	11.5258560231847	11.5258560231847\\
62.375	0.12594	12.490292411266	12.490292411266\\
62.375	0.1296	13.5211187498097	13.5211187498097\\
62.375	0.13326	14.6183350388157	14.6183350388157\\
62.375	0.13692	15.7819412782842	15.7819412782842\\
62.375	0.14058	17.011937468215	17.011937468215\\
62.375	0.14424	18.3083236086082	18.3083236086082\\
62.375	0.1479	19.6710996994638	19.6710996994638\\
62.375	0.15156	21.1002657407818	21.1002657407818\\
62.375	0.15522	22.5958217325621	22.5958217325621\\
62.375	0.15888	24.1577676748048	24.1577676748048\\
62.375	0.16254	25.78610356751	25.78610356751\\
62.375	0.1662	27.4808294106775	27.4808294106775\\
62.375	0.16986	29.2419452043073	29.2419452043073\\
62.375	0.17352	31.0694509483996	31.0694509483996\\
62.375	0.17718	32.9633466429542	32.9633466429542\\
62.375	0.18084	34.9236322879712	34.9236322879712\\
62.375	0.1845	36.9503078834506	36.9503078834506\\
62.375	0.18816	39.0433734293924	39.0433734293924\\
62.375	0.19182	41.2028289257966	41.2028289257966\\
62.375	0.19548	43.4286743726631	43.4286743726631\\
62.375	0.19914	45.720909769992	45.720909769992\\
62.375	0.2028	48.0795351177834	48.0795351177834\\
62.375	0.20646	50.5045504160371	50.5045504160371\\
62.375	0.21012	52.9959556647531	52.9959556647531\\
62.375	0.21378	55.5537508639316	55.5537508639316\\
62.375	0.21744	58.1779360135724	58.1779360135724\\
62.375	0.2211	60.8685111136756	60.8685111136756\\
62.375	0.22476	63.6254761642412	63.6254761642412\\
62.375	0.22842	66.4488311652692	66.4488311652692\\
62.375	0.23208	69.3385761167595	69.3385761167595\\
62.375	0.23574	72.2947110187123	72.2947110187123\\
62.375	0.2394	75.3172358711274	75.3172358711274\\
62.375	0.24306	78.4061506740049	78.4061506740049\\
62.375	0.24672	81.5614554273448	81.5614554273448\\
62.375	0.25038	84.783150131147	84.783150131147\\
62.375	0.25404	88.0712347854116	88.0712347854116\\
62.375	0.2577	91.4257093901387	91.4257093901387\\
62.375	0.26136	94.8465739453281	94.8465739453281\\
62.375	0.26502	98.3338284509799	98.3338284509799\\
62.375	0.26868	101.887472907094	101.887472907094\\
62.375	0.27234	105.507507313671	105.507507313671\\
62.375	0.276	109.193931670709	109.193931670709\\
62.75	0.093	6.26080637229972	6.26080637229972\\
62.75	0.09666	6.69454029601163	6.69454029601163\\
62.75	0.10032	7.19466417018592	7.19466417018592\\
62.75	0.10398	7.76117799482261	7.76117799482261\\
62.75	0.10764	8.39408176992167	8.39408176992167\\
62.75	0.1113	9.09337549548312	9.09337549548312\\
62.75	0.11496	9.85905917150694	9.85905917150694\\
62.75	0.11862	10.6911327979931	10.6911327979931\\
62.75	0.12228	11.5895963749417	11.5895963749417\\
62.75	0.12594	12.5544499023527	12.5544499023527\\
62.75	0.1296	13.585693380226	13.585693380226\\
62.75	0.13326	14.6833268085617	14.6833268085617\\
62.75	0.13692	15.8473501873599	15.8473501873599\\
62.75	0.14058	17.0777635166204	17.0777635166204\\
62.75	0.14424	18.3745667963432	18.3745667963432\\
62.75	0.1479	19.7377600265285	19.7377600265285\\
62.75	0.15156	21.1673432071761	21.1673432071761\\
62.75	0.15522	22.6633163382861	22.6633163382861\\
62.75	0.15888	24.2256794198585	24.2256794198585\\
62.75	0.16254	25.8544324518933	25.8544324518933\\
62.75	0.1662	27.5495754343905	27.5495754343905\\
62.75	0.16986	29.31110836735	29.31110836735\\
62.75	0.17352	31.1390312507719	31.1390312507719\\
62.75	0.17718	33.0333440846562	33.0333440846562\\
62.75	0.18084	34.9940468690029	34.9940468690029\\
62.75	0.1845	37.021139603812	37.021139603812\\
62.75	0.18816	39.1146222890834	39.1146222890834\\
62.75	0.19182	41.2744949248173	41.2744949248173\\
62.75	0.19548	43.5007575110135	43.5007575110135\\
62.75	0.19914	45.7934100476721	45.7934100476721\\
62.75	0.2028	48.152452534793	48.152452534793\\
62.75	0.20646	50.5778849723764	50.5778849723764\\
62.75	0.21012	53.0697073604221	53.0697073604221\\
62.75	0.21378	55.6279196989302	55.6279196989302\\
62.75	0.21744	58.2525219879007	58.2525219879007\\
62.75	0.2211	60.9435142273336	60.9435142273336\\
62.75	0.22476	63.7008964172289	63.7008964172289\\
62.75	0.22842	66.5246685575865	66.5246685575865\\
62.75	0.23208	69.4148306484065	69.4148306484065\\
62.75	0.23574	72.371382689689	72.371382689689\\
62.75	0.2394	75.3943246814337	75.3943246814337\\
62.75	0.24306	78.4836566236409	78.4836566236409\\
62.75	0.24672	81.6393785163104	81.6393785163104\\
62.75	0.25038	84.8614903594424	84.8614903594424\\
62.75	0.25404	88.1499921530367	88.1499921530367\\
62.75	0.2577	91.5048838970934	91.5048838970934\\
62.75	0.26136	94.9261655916125	94.9261655916125\\
62.75	0.26502	98.4138372365939	98.4138372365939\\
62.75	0.26868	101.967898832038	101.967898832038\\
62.75	0.27234	105.588350377944	105.588350377944\\
62.75	0.276	109.275191874312	109.275191874312\\
63.125	0.093	6.32495198461351	6.32495198461351\\
63.125	0.09666	6.75910304765509	6.75910304765509\\
63.125	0.10032	7.25964406115906	7.25964406115906\\
63.125	0.10398	7.82657502512541	7.82657502512541\\
63.125	0.10764	8.45989593955412	8.45989593955412\\
63.125	0.1113	9.15960680444521	9.15960680444521\\
63.125	0.11496	9.92570761979871	9.92570761979871\\
63.125	0.11862	10.7581983856146	10.7581983856146\\
63.125	0.12228	11.6570791018928	11.6570791018928\\
63.125	0.12594	12.6223497686335	12.6223497686335\\
63.125	0.1296	13.6540103858365	13.6540103858365\\
63.125	0.13326	14.7520609535019	14.7520609535019\\
63.125	0.13692	15.9165014716297	15.9165014716297\\
63.125	0.14058	17.1473319402198	17.1473319402198\\
63.125	0.14424	18.4445523592724	18.4445523592724\\
63.125	0.1479	19.8081627287873	19.8081627287873\\
63.125	0.15156	21.2381630487646	21.2381630487646\\
63.125	0.15522	22.7345533192043	22.7345533192043\\
63.125	0.15888	24.2973335401063	24.2973335401063\\
63.125	0.16254	25.9265037114708	25.9265037114708\\
63.125	0.1662	27.6220638332976	27.6220638332976\\
63.125	0.16986	29.3840139055868	29.3840139055868\\
63.125	0.17352	31.2123539283384	31.2123539283384\\
63.125	0.17718	33.1070839015524	33.1070839015524\\
63.125	0.18084	35.0682038252287	35.0682038252287\\
63.125	0.1845	37.0957136993675	37.0957136993675\\
63.125	0.18816	39.1896135239686	39.1896135239686\\
63.125	0.19182	41.349903299032	41.349903299032\\
63.125	0.19548	43.5765830245579	43.5765830245579\\
63.125	0.19914	45.8696527005462	45.8696527005462\\
63.125	0.2028	48.2291123269968	48.2291123269968\\
63.125	0.20646	50.6549619039099	50.6549619039099\\
63.125	0.21012	53.1472014312853	53.1472014312853\\
63.125	0.21378	55.705830909123	55.705830909123\\
63.125	0.21744	58.3308503374232	58.3308503374232\\
63.125	0.2211	61.0222597161857	61.0222597161857\\
63.125	0.22476	63.7800590454107	63.7800590454107\\
63.125	0.22842	66.604248325098	66.604248325098\\
63.125	0.23208	69.4948275552476	69.4948275552476\\
63.125	0.23574	72.4517967358598	72.4517967358598\\
63.125	0.2394	75.4751558669342	75.4751558669342\\
63.125	0.24306	78.5649049484711	78.5649049484711\\
63.125	0.24672	81.7210439804703	81.7210439804703\\
63.125	0.25038	84.9435729629318	84.9435729629318\\
63.125	0.25404	88.2324918958558	88.2324918958558\\
63.125	0.2577	91.5878007792422	91.5878007792422\\
63.125	0.26136	95.0094996130909	95.0094996130909\\
63.125	0.26502	98.497588397402	98.497588397402\\
63.125	0.26868	102.052067132176	102.052067132176\\
63.125	0.27234	105.672935817411	105.672935817411\\
63.125	0.276	109.36019445311	109.36019445311\\
63.5	0.093	6.39283997212141	6.39283997212141\\
63.5	0.09666	6.82740817449265	6.82740817449265\\
63.5	0.10032	7.32836632732629	7.32836632732629\\
63.5	0.10398	7.89571443062231	7.89571443062231\\
63.5	0.10764	8.52945248438068	8.52945248438068\\
63.5	0.1113	9.22958048860145	9.22958048860145\\
63.5	0.11496	9.99609844328462	9.99609844328462\\
63.5	0.11862	10.8290063484301	10.8290063484301\\
63.5	0.12228	11.7283042040381	11.7283042040381\\
63.5	0.12594	12.6939920101084	12.6939920101084\\
63.5	0.1296	13.726069766641	13.726069766641\\
63.5	0.13326	14.8245374736361	14.8245374736361\\
63.5	0.13692	15.9893951310936	15.9893951310936\\
63.5	0.14058	17.2206427390134	17.2206427390134\\
63.5	0.14424	18.5182802973956	18.5182802973956\\
63.5	0.1479	19.8823078062402	19.8823078062402\\
63.5	0.15156	21.3127252655471	21.3127252655471\\
63.5	0.15522	22.8095326753165	22.8095326753165\\
63.5	0.15888	24.3727300355482	24.3727300355482\\
63.5	0.16254	26.0023173462424	26.0023173462424\\
63.5	0.1662	27.6982946073988	27.6982946073988\\
63.5	0.16986	29.4606618190177	29.4606618190177\\
63.5	0.17352	31.289418981099	31.289418981099\\
63.5	0.17718	33.1845660936426	33.1845660936426\\
63.5	0.18084	35.1461031566486	35.1461031566486\\
63.5	0.1845	37.174030170117	37.174030170117\\
63.5	0.18816	39.2683471340478	39.2683471340478\\
63.5	0.19182	41.429054048441	41.429054048441\\
63.5	0.19548	43.6561509132965	43.6561509132965\\
63.5	0.19914	45.9496377286144	45.9496377286144\\
63.5	0.2028	48.3095144943948	48.3095144943948\\
63.5	0.20646	50.7357812106375	50.7357812106375\\
63.5	0.21012	53.2284378773425	53.2284378773425\\
63.5	0.21378	55.78748449451	55.78748449451\\
63.5	0.21744	58.4129210621398	58.4129210621398\\
63.5	0.2211	61.104747580232	61.104747580232\\
63.5	0.22476	63.8629640487866	63.8629640487866\\
63.5	0.22842	66.6875704678036	66.6875704678036\\
63.5	0.23208	69.5785668372829	69.5785668372829\\
63.5	0.23574	72.5359531572247	72.5359531572247\\
63.5	0.2394	75.5597294276288	75.5597294276288\\
63.5	0.24306	78.6498956484953	78.6498956484953\\
63.5	0.24672	81.8064518198242	81.8064518198242\\
63.5	0.25038	85.0293979416154	85.0293979416154\\
63.5	0.25404	88.3187340138691	88.3187340138691\\
63.5	0.2577	91.6744600365851	91.6744600365851\\
63.5	0.26136	95.0965760097635	95.0965760097635\\
63.5	0.26502	98.5850819334043	98.5850819334043\\
63.5	0.26868	102.139977807507	102.139977807507\\
63.5	0.27234	105.761263632073	105.761263632073\\
63.5	0.276	109.448939407101	109.448939407101\\
63.875	0.093	6.46447033482339	6.46447033482339\\
63.875	0.09666	6.8994556765243	6.8994556765243\\
63.875	0.10032	7.40083096868758	7.40083096868758\\
63.875	0.10398	7.96859621131327	7.96859621131327\\
63.875	0.10764	8.60275140440132	8.60275140440132\\
63.875	0.1113	9.30329654795178	9.30329654795178\\
63.875	0.11496	10.0702316419646	10.0702316419646\\
63.875	0.11862	10.9035566864398	10.9035566864398\\
63.875	0.12228	11.8032716813774	11.8032716813774\\
63.875	0.12594	12.7693766267774	12.7693766267774\\
63.875	0.1296	13.8018715226397	13.8018715226397\\
63.875	0.13326	14.9007563689644	14.9007563689644\\
63.875	0.13692	16.0660311657515	16.0660311657515\\
63.875	0.14058	17.297695913001	17.297695913001\\
63.875	0.14424	18.5957506107129	18.5957506107129\\
63.875	0.1479	19.9601952588872	19.9601952588872\\
63.875	0.15156	21.3910298575238	21.3910298575238\\
63.875	0.15522	22.8882544066228	22.8882544066228\\
63.875	0.15888	24.4518689061842	24.4518689061842\\
63.875	0.16254	26.081873356208	26.081873356208\\
63.875	0.1662	27.7782677566942	27.7782677566942\\
63.875	0.16986	29.5410521076427	29.5410521076427\\
63.875	0.17352	31.3702264090536	31.3702264090536\\
63.875	0.17718	33.2657906609269	33.2657906609269\\
63.875	0.18084	35.2277448632626	35.2277448632626\\
63.875	0.1845	37.2560890160607	37.2560890160607\\
63.875	0.18816	39.3508231193211	39.3508231193211\\
63.875	0.19182	41.511947173044	41.511947173044\\
63.875	0.19548	43.7394611772292	43.7394611772292\\
63.875	0.19914	46.0333651318768	46.0333651318768\\
63.875	0.2028	48.3936590369867	48.3936590369867\\
63.875	0.20646	50.8203428925591	50.8203428925591\\
63.875	0.21012	53.3134166985938	53.3134166985938\\
63.875	0.21378	55.8728804550909	55.8728804550909\\
63.875	0.21744	58.4987341620504	58.4987341620504\\
63.875	0.2211	61.1909778194723	61.1909778194723\\
63.875	0.22476	63.9496114273566	63.9496114273566\\
63.875	0.22842	66.7746349857032	66.7746349857032\\
63.875	0.23208	69.6660484945122	69.6660484945122\\
63.875	0.23574	72.6238519537837	72.6238519537837\\
63.875	0.2394	75.6480453635174	75.6480453635174\\
63.875	0.24306	78.7386287237136	78.7386287237136\\
63.875	0.24672	81.8956020343722	81.8956020343722\\
63.875	0.25038	85.1189652954931	85.1189652954931\\
63.875	0.25404	88.4087185070764	88.4087185070764\\
63.875	0.2577	91.7648616691221	91.7648616691221\\
63.875	0.26136	95.1873947816302	95.1873947816302\\
63.875	0.26502	98.6763178446006	98.6763178446006\\
63.875	0.26868	102.231630858033	102.231630858033\\
63.875	0.27234	105.853333821929	105.853333821929\\
63.875	0.276	109.541426736286	109.541426736286\\
64.25	0.093	6.53984307271947	6.53984307271947\\
64.25	0.09666	6.97524555375006	6.97524555375006\\
64.25	0.10032	7.47703798524302	7.47703798524302\\
64.25	0.10398	8.04522036719836	8.04522036719836\\
64.25	0.10764	8.67979269961609	8.67979269961609\\
64.25	0.1113	9.3807549824962	9.3807549824962\\
64.25	0.11496	10.1481072158387	10.1481072158387\\
64.25	0.11862	10.9818493996436	10.9818493996436\\
64.25	0.12228	11.8819815339108	11.8819815339108\\
64.25	0.12594	12.8485036186405	12.8485036186405\\
64.25	0.1296	13.8814156538325	13.8814156538325\\
64.25	0.13326	14.9807176394869	14.9807176394869\\
64.25	0.13692	16.1464095756036	16.1464095756036\\
64.25	0.14058	17.3784914621828	17.3784914621828\\
64.25	0.14424	18.6769632992243	18.6769632992243\\
64.25	0.1479	20.0418250867283	20.0418250867283\\
64.25	0.15156	21.4730768246946	21.4730768246946\\
64.25	0.15522	22.9707185131232	22.9707185131232\\
64.25	0.15888	24.5347501520143	24.5347501520143\\
64.25	0.16254	26.1651717413678	26.1651717413678\\
64.25	0.1662	27.8619832811836	27.8619832811836\\
64.25	0.16986	29.6251847714618	29.6251847714618\\
64.25	0.17352	31.4547762122024	31.4547762122024\\
64.25	0.17718	33.3507576034054	33.3507576034054\\
64.25	0.18084	35.3131289450707	35.3131289450707\\
64.25	0.1845	37.3418902371984	37.3418902371984\\
64.25	0.18816	39.4370414797886	39.4370414797886\\
64.25	0.19182	41.5985826728411	41.5985826728411\\
64.25	0.19548	43.8265138163559	43.8265138163559\\
64.25	0.19914	46.1208349103332	46.1208349103332\\
64.25	0.2028	48.4815459547728	48.4815459547728\\
64.25	0.20646	50.9086469496749	50.9086469496749\\
64.25	0.21012	53.4021378950393	53.4021378950393\\
64.25	0.21378	55.962018790866	55.962018790866\\
64.25	0.21744	58.5882896371552	58.5882896371552\\
64.25	0.2211	61.2809504339067	61.2809504339067\\
64.25	0.22476	64.0400011811207	64.0400011811207\\
64.25	0.22842	66.865441878797	66.865441878797\\
64.25	0.23208	69.7572725269357	69.7572725269357\\
64.25	0.23574	72.7154931255368	72.7154931255368\\
64.25	0.2394	75.7401036746002	75.7401036746002\\
64.25	0.24306	78.831104174126	78.831104174126\\
64.25	0.24672	81.9884946241143	81.9884946241143\\
64.25	0.25038	85.2122750245649	85.2122750245649\\
64.25	0.25404	88.5024453754778	88.5024453754778\\
64.25	0.2577	91.8590056768532	91.8590056768532\\
64.25	0.26136	95.281955928691	95.281955928691\\
64.25	0.26502	98.7712961309911	98.7712961309911\\
64.25	0.26868	102.327026283754	102.327026283754\\
64.25	0.27234	105.949146386978	105.949146386978\\
64.25	0.276	109.637656440666	109.637656440666\\
64.625	0.093	6.61895818580965	6.61895818580965\\
64.625	0.09666	7.05477780616989	7.05477780616989\\
64.625	0.10032	7.55698737699253	7.55698737699253\\
64.625	0.10398	8.12558689827755	8.12558689827755\\
64.625	0.10764	8.76057637002493	8.76057637002493\\
64.625	0.1113	9.4619557922347	9.4619557922347\\
64.625	0.11496	10.2297251649069	10.2297251649069\\
64.625	0.11862	11.0638844880414	11.0638844880414\\
64.625	0.12228	11.9644337616383	11.9644337616383\\
64.625	0.12594	12.9313729856976	12.9313729856976\\
64.625	0.1296	13.9647021602193	13.9647021602193\\
64.625	0.13326	15.0644212852034	15.0644212852034\\
64.625	0.13692	16.2305303606498	16.2305303606498\\
64.625	0.14058	17.4630293865586	17.4630293865586\\
64.625	0.14424	18.7619183629298	18.7619183629298\\
64.625	0.1479	20.1271972897634	20.1271972897634\\
64.625	0.15156	21.5588661670594	21.5588661670594\\
64.625	0.15522	23.0569249948178	23.0569249948178\\
64.625	0.15888	24.6213737730385	24.6213737730385\\
64.625	0.16254	26.2522125017216	26.2522125017216\\
64.625	0.1662	27.9494411808671	27.9494411808671\\
64.625	0.16986	29.713059810475	29.713059810475\\
64.625	0.17352	31.5430683905452	31.5430683905452\\
64.625	0.17718	33.4394669210779	33.4394669210779\\
64.625	0.18084	35.4022554020729	35.4022554020729\\
64.625	0.1845	37.4314338335303	37.4314338335303\\
64.625	0.18816	39.5270022154501	39.5270022154501\\
64.625	0.19182	41.6889605478322	41.6889605478322\\
64.625	0.19548	43.9173088306768	43.9173088306768\\
64.625	0.19914	46.2120470639837	46.2120470639837\\
64.625	0.2028	48.573175247753	48.573175247753\\
64.625	0.20646	51.0006933819847	51.0006933819847\\
64.625	0.21012	53.4946014666788	53.4946014666788\\
64.625	0.21378	56.0548995018352	56.0548995018352\\
64.625	0.21744	58.681587487454	58.681587487454\\
64.625	0.2211	61.3746654235352	61.3746654235352\\
64.625	0.22476	64.1341333100789	64.1341333100789\\
64.625	0.22842	66.9599911470848	66.9599911470848\\
64.625	0.23208	69.8522389345532	69.8522389345532\\
64.625	0.23574	72.810876672484	72.810876672484\\
64.625	0.2394	75.8359043608771	75.8359043608771\\
64.625	0.24306	78.9273219997326	78.9273219997326\\
64.625	0.24672	82.0851295890505	82.0851295890505\\
64.625	0.25038	85.3093271288307	85.3093271288307\\
64.625	0.25404	88.5999146190734	88.5999146190734\\
64.625	0.2577	91.9568920597784	91.9568920597784\\
64.625	0.26136	95.3802594509458	95.3802594509458\\
64.625	0.26502	98.8700167925755	98.8700167925755\\
64.625	0.26868	102.426164084668	102.426164084668\\
64.625	0.27234	106.048701327222	106.048701327222\\
64.625	0.276	109.737628520239	109.737628520239\\
65	0.093	6.70181567409392	6.70181567409392\\
65	0.09666	7.13805243378384	7.13805243378384\\
65	0.10032	7.64067914393612	7.64067914393612\\
65	0.10398	8.2096958045508	8.2096958045508\\
65	0.10764	8.84510241562787	8.84510241562787\\
65	0.1113	9.54689897716732	9.54689897716732\\
65	0.11496	10.3150854891692	10.3150854891692\\
65	0.11862	11.1496619516333	11.1496619516333\\
65	0.12228	12.0506283645599	12.0506283645599\\
65	0.12594	13.0179847279489	13.0179847279489\\
65	0.1296	14.0517310418002	14.0517310418002\\
65	0.13326	15.151867306114	15.151867306114\\
65	0.13692	16.3183935208901	16.3183935208901\\
65	0.14058	17.5513096861286	17.5513096861286\\
65	0.14424	18.8506158018295	18.8506158018295\\
65	0.1479	20.2163118679927	20.2163118679927\\
65	0.15156	21.6483978846183	21.6483978846183\\
65	0.15522	23.1468738517064	23.1468738517064\\
65	0.15888	24.7117397692568	24.7117397692568\\
65	0.16254	26.3429956372696	26.3429956372696\\
65	0.1662	28.0406414557447	28.0406414557447\\
65	0.16986	29.8046772246823	29.8046772246823\\
65	0.17352	31.6351029440822	31.6351029440822\\
65	0.17718	33.5319186139445	33.5319186139445\\
65	0.18084	35.4951242342691	35.4951242342691\\
65	0.1845	37.5247198050562	37.5247198050562\\
65	0.18816	39.6207053263057	39.6207053263057\\
65	0.19182	41.7830807980175	41.7830807980175\\
65	0.19548	44.0118462201917	44.0118462201917\\
65	0.19914	46.3070015928283	46.3070015928283\\
65	0.2028	48.6685469159273	48.6685469159273\\
65	0.20646	51.0964821894887	51.0964821894887\\
65	0.21012	53.5908074135124	53.5908074135124\\
65	0.21378	56.1515225879985	56.1515225879985\\
65	0.21744	58.778627712947	58.778627712947\\
65	0.2211	61.4721227883579	61.4721227883579\\
65	0.22476	64.2320078142311	64.2320078142311\\
65	0.22842	67.0582827905668	67.0582827905668\\
65	0.23208	69.9509477173648	69.9509477173648\\
65	0.23574	72.9100025946252	72.9100025946252\\
65	0.2394	75.935447422348	75.935447422348\\
65	0.24306	79.0272822005332	79.0272822005332\\
65	0.24672	82.1855069291807	82.1855069291807\\
65	0.25038	85.4101216082907	85.4101216082907\\
65	0.25404	88.701126237863	88.701126237863\\
65	0.2577	92.0585208178977	92.0585208178977\\
65	0.26136	95.4823053483948	95.4823053483948\\
65	0.26502	98.9724798293542	98.9724798293542\\
65	0.26868	102.529044260776	102.529044260776\\
65	0.27234	106.15199864266	106.15199864266\\
65	0.276	109.841342975007	109.841342975007\\
65.375	0.093	6.78841553757231	6.78841553757231\\
65.375	0.09666	7.22506943659188	7.22506943659188\\
65.375	0.10032	7.72811328607385	7.72811328607385\\
65.375	0.10398	8.29754708601818	8.29754708601818\\
65.375	0.10764	8.93337083642492	8.93337083642492\\
65.375	0.1113	9.63558453729403	9.63558453729403\\
65.375	0.11496	10.4041881886255	10.4041881886255\\
65.375	0.11862	11.2391817904194	11.2391817904194\\
65.375	0.12228	12.1405653426757	12.1405653426757\\
65.375	0.12594	13.1083388453943	13.1083388453943\\
65.375	0.1296	14.1425022985753	14.1425022985753\\
65.375	0.13326	15.2430557022187	15.2430557022187\\
65.375	0.13692	16.4099990563245	16.4099990563245\\
65.375	0.14058	17.6433323608926	17.6433323608926\\
65.375	0.14424	18.9430556159232	18.9430556159232\\
65.375	0.1479	20.3091688214161	20.3091688214161\\
65.375	0.15156	21.7416719773714	21.7416719773714\\
65.375	0.15522	23.2405650837891	23.2405650837891\\
65.375	0.15888	24.8058481406692	24.8058481406692\\
65.375	0.16254	26.4375211480116	26.4375211480116\\
65.375	0.1662	28.1355841058164	28.1355841058164\\
65.375	0.16986	29.9000370140836	29.9000370140836\\
65.375	0.17352	31.7308798728132	31.7308798728132\\
65.375	0.17718	33.6281126820052	33.6281126820052\\
65.375	0.18084	35.5917354416595	35.5917354416595\\
65.375	0.1845	37.6217481517763	37.6217481517763\\
65.375	0.18816	39.7181508123554	39.7181508123554\\
65.375	0.19182	41.8809434233969	41.8809434233969\\
65.375	0.19548	44.1101259849008	44.1101259849008\\
65.375	0.19914	46.4056984968671	46.4056984968671\\
65.375	0.2028	48.7676609592957	48.7676609592957\\
65.375	0.20646	51.1960133721867	51.1960133721867\\
65.375	0.21012	53.6907557355401	53.6907557355401\\
65.375	0.21378	56.2518880493559	56.2518880493559\\
65.375	0.21744	58.8794103136341	58.8794103136341\\
65.375	0.2211	61.5733225283746	61.5733225283746\\
65.375	0.22476	64.3336246935775	64.3336246935775\\
65.375	0.22842	67.1603168092429	67.1603168092429\\
65.375	0.23208	70.0533988753705	70.0533988753705\\
65.375	0.23574	73.0128708919606	73.0128708919606\\
65.375	0.2394	76.0387328590131	76.0387328590131\\
65.375	0.24306	79.1309847765279	79.1309847765279\\
65.375	0.24672	82.2896266445051	82.2896266445051\\
65.375	0.25038	85.5146584629447	85.5146584629447\\
65.375	0.25404	88.8060802318467	88.8060802318467\\
65.375	0.2577	92.1638919512111	92.1638919512111\\
65.375	0.26136	95.5880936210378	95.5880936210378\\
65.375	0.26502	99.0786852413269	99.0786852413269\\
65.375	0.26868	102.635666812078	102.635666812078\\
65.375	0.27234	106.259038333292	106.259038333292\\
65.375	0.276	109.948799804969	109.948799804969\\
65.75	0.093	6.87875777624475	6.87875777624475\\
65.75	0.09666	7.31582881459401	7.31582881459401\\
65.75	0.10032	7.81928980340565	7.81928980340565\\
65.75	0.10398	8.38914074267966	8.38914074267966\\
65.75	0.10764	9.02538163241605	9.02538163241605\\
65.75	0.1113	9.72801247261481	9.72801247261481\\
65.75	0.11496	10.497033263276	10.497033263276\\
65.75	0.11862	11.3324440043995	11.3324440043995\\
65.75	0.12228	12.2342446959854	12.2342446959854\\
65.75	0.12594	13.2024353380338	13.2024353380338\\
65.75	0.1296	14.2370159305444	14.2370159305444\\
65.75	0.13326	15.3379864735175	15.3379864735175\\
65.75	0.13692	16.5053469669529	16.5053469669529\\
65.75	0.14058	17.7390974108508	17.7390974108508\\
65.75	0.14424	19.039237805211	19.039237805211\\
65.75	0.1479	20.4057681500336	20.4057681500336\\
65.75	0.15156	21.8386884453185	21.8386884453185\\
65.75	0.15522	23.3379986910659	23.3379986910659\\
65.75	0.15888	24.9036988872756	24.9036988872756\\
65.75	0.16254	26.5357890339477	26.5357890339477\\
65.75	0.1662	28.2342691310822	28.2342691310822\\
65.75	0.16986	29.9991391786791	29.9991391786791\\
65.75	0.17352	31.8303991767384	31.8303991767384\\
65.75	0.17718	33.72804912526	33.72804912526\\
65.75	0.18084	35.692089024244	35.692089024244\\
65.75	0.1845	37.7225188736904	37.7225188736904\\
65.75	0.18816	39.8193386735992	39.8193386735992\\
65.75	0.19182	41.9825484239704	41.9825484239704\\
65.75	0.19548	44.2121481248039	44.2121481248039\\
65.75	0.19914	46.5081377760999	46.5081377760999\\
65.75	0.2028	48.8705173778582	48.8705173778582\\
65.75	0.20646	51.2992869300789	51.2992869300789\\
65.75	0.21012	53.7944464327619	53.7944464327619\\
65.75	0.21378	56.3559958859074	56.3559958859074\\
65.75	0.21744	58.9839352895152	58.9839352895152\\
65.75	0.2211	61.6782646435854	61.6782646435854\\
65.75	0.22476	64.438983948118	64.438983948118\\
65.75	0.22842	67.266093203113	67.266093203113\\
65.75	0.23208	70.1595924085703	70.1595924085703\\
65.75	0.23574	73.1194815644901	73.1194815644901\\
65.75	0.2394	76.1457606708722	76.1457606708722\\
65.75	0.24306	79.2384297277167	79.2384297277167\\
65.75	0.24672	82.3974887350236	82.3974887350236\\
65.75	0.25038	85.6229376927929	85.6229376927929\\
65.75	0.25404	88.9147766010245	88.9147766010245\\
65.75	0.2577	92.2730054597185	92.2730054597185\\
65.75	0.26136	95.697624268875	95.697624268875\\
65.75	0.26502	99.1886330284937	99.1886330284937\\
65.75	0.26868	102.746031738575	102.746031738575\\
65.75	0.27234	106.369820399118	106.369820399118\\
65.75	0.276	110.059999010124	110.059999010124\\
66.125	0.093	6.97284239011132	6.97284239011132\\
66.125	0.09666	7.41033056779024	7.41033056779024\\
66.125	0.10032	7.91420869593153	7.91420869593153\\
66.125	0.10398	8.48447677453522	8.48447677453522\\
66.125	0.10764	9.12113480360129	9.12113480360129\\
66.125	0.1113	9.82418278312974	9.82418278312974\\
66.125	0.11496	10.5936207131206	10.5936207131206\\
66.125	0.11862	11.4294485935738	11.4294485935738\\
66.125	0.12228	12.3316664244894	12.3316664244894\\
66.125	0.12594	13.3002742058673	13.3002742058673\\
66.125	0.1296	14.3352719377077	14.3352719377077\\
66.125	0.13326	15.4366596200104	15.4366596200104\\
66.125	0.13692	16.6044372527755	16.6044372527755\\
66.125	0.14058	17.838604836003	17.838604836003\\
66.125	0.14424	19.1391623696929	19.1391623696929\\
66.125	0.1479	20.5061098538451	20.5061098538451\\
66.125	0.15156	21.9394472884598	21.9394472884598\\
66.125	0.15522	23.4391746735368	23.4391746735368\\
66.125	0.15888	25.0052920090762	25.0052920090762\\
66.125	0.16254	26.637799295078	26.637799295078\\
66.125	0.1662	28.3366965315422	28.3366965315422\\
66.125	0.16986	30.1019837184687	30.1019837184687\\
66.125	0.17352	31.9336608558576	31.9336608558576\\
66.125	0.17718	33.8317279437089	33.8317279437089\\
66.125	0.18084	35.7961849820226	35.7961849820226\\
66.125	0.1845	37.8270319707987	37.8270319707987\\
66.125	0.18816	39.9242689100371	39.9242689100371\\
66.125	0.19182	42.087895799738	42.087895799738\\
66.125	0.19548	44.3179126399012	44.3179126399012\\
66.125	0.19914	46.6143194305268	46.6143194305268\\
66.125	0.2028	48.9771161716147	48.9771161716147\\
66.125	0.20646	51.4063028631651	51.4063028631651\\
66.125	0.21012	53.9018795051778	53.9018795051778\\
66.125	0.21378	56.4638460976529	56.4638460976529\\
66.125	0.21744	59.0922026405905	59.0922026405905\\
66.125	0.2211	61.7869491339903	61.7869491339903\\
66.125	0.22476	64.5480855778526	64.5480855778526\\
66.125	0.22842	67.3756119721772	67.3756119721772\\
66.125	0.23208	70.2695283169643	70.2695283169643\\
66.125	0.23574	73.2298346122137	73.2298346122137\\
66.125	0.2394	76.2565308579255	76.2565308579255\\
66.125	0.24306	79.3496170540996	79.3496170540996\\
66.125	0.24672	82.5090932007362	82.5090932007362\\
66.125	0.25038	85.7349592978351	85.7349592978351\\
66.125	0.25404	89.0272153453964	89.0272153453964\\
66.125	0.2577	92.3858613434201	92.3858613434201\\
66.125	0.26136	95.8108972919062	95.8108972919062\\
66.125	0.26502	99.3023231908547	99.3023231908547\\
66.125	0.26868	102.860139040266	102.860139040266\\
66.125	0.27234	106.484344840139	106.484344840139\\
66.125	0.276	110.174940590474	110.174940590474\\
66.5	0.093	7.070669379172	7.070669379172\\
66.5	0.09666	7.50857469618058	7.50857469618058\\
66.5	0.10032	8.01286996365154	8.01286996365154\\
66.5	0.10398	8.58355518158489	8.58355518158489\\
66.5	0.10764	9.22063034998062	9.22063034998062\\
66.5	0.1113	9.92409546883873	9.92409546883873\\
66.5	0.11496	10.6939505381592	10.6939505381592\\
66.5	0.11862	11.5301955579421	11.5301955579421\\
66.5	0.12228	12.4328305281874	12.4328305281874\\
66.5	0.12594	13.401855448895	13.401855448895\\
66.5	0.1296	14.437270320065	14.437270320065\\
66.5	0.13326	15.5390751416974	15.5390751416974\\
66.5	0.13692	16.7072699137922	16.7072699137922\\
66.5	0.14058	17.9418546363493	17.9418546363493\\
66.5	0.14424	19.2428293093689	19.2428293093689\\
66.5	0.1479	20.6101939328508	20.6101939328508\\
66.5	0.15156	22.0439485067951	22.0439485067951\\
66.5	0.15522	23.5440930312018	23.5440930312018\\
66.5	0.15888	25.1106275060709	25.1106275060709\\
66.5	0.16254	26.7435519314023	26.7435519314023\\
66.5	0.1662	28.4428663071962	28.4428663071962\\
66.5	0.16986	30.2085706334524	30.2085706334524\\
66.5	0.17352	32.0406649101709	32.0406649101709\\
66.5	0.17718	33.9391491373519	33.9391491373519\\
66.5	0.18084	35.9040233149953	35.9040233149953\\
66.5	0.1845	37.935287443101	37.935287443101\\
66.5	0.18816	40.0329415216691	40.0329415216691\\
66.5	0.19182	42.1969855506996	42.1969855506996\\
66.5	0.19548	44.4274195301925	44.4274195301925\\
66.5	0.19914	46.7242434601478	46.7242434601478\\
66.5	0.2028	49.0874573405654	49.0874573405654\\
66.5	0.20646	51.5170611714454	51.5170611714454\\
66.5	0.21012	54.0130549527878	54.0130549527878\\
66.5	0.21378	56.5754386845926	56.5754386845926\\
66.5	0.21744	59.2042123668598	59.2042123668598\\
66.5	0.2211	61.8993759995893	61.8993759995893\\
66.5	0.22476	64.6609295827813	64.6609295827813\\
66.5	0.22842	67.4888731164356	67.4888731164356\\
66.5	0.23208	70.3832066005523	70.3832066005523\\
66.5	0.23574	73.3439300351314	73.3439300351314\\
66.5	0.2394	76.3710434201728	76.3710434201728\\
66.5	0.24306	79.4645467556766	79.4645467556766\\
66.5	0.24672	82.6244400416429	82.6244400416429\\
66.5	0.25038	85.8507232780715	85.8507232780715\\
66.5	0.25404	89.1433964649625	89.1433964649625\\
66.5	0.2577	92.5024596023158	92.5024596023158\\
66.5	0.26136	95.9279126901315	95.9279126901315\\
66.5	0.26502	99.4197557284097	99.4197557284097\\
66.5	0.26868	102.97798871715	102.97798871715\\
66.5	0.27234	106.602611656353	106.602611656353\\
66.5	0.276	110.293624546018	110.293624546018\\
66.875	0.093	7.17223874342676	7.17223874342676\\
66.875	0.09666	7.610561199765	7.610561199765\\
66.875	0.10032	8.11527360656565	8.11527360656565\\
66.875	0.10398	8.68637596382867	8.68637596382867\\
66.875	0.10764	9.32386827155404	9.32386827155404\\
66.875	0.1113	10.0277505297418	10.0277505297418\\
66.875	0.11496	10.798022738392	10.798022738392\\
66.875	0.11862	11.6346848975045	11.6346848975045\\
66.875	0.12228	12.5377370070794	12.5377370070794\\
66.875	0.12594	13.5071790671168	13.5071790671168\\
66.875	0.1296	14.5430110776164	14.5430110776164\\
66.875	0.13326	15.6452330385785	15.6452330385785\\
66.875	0.13692	16.8138449500029	16.8138449500029\\
66.875	0.14058	18.0488468118898	18.0488468118898\\
66.875	0.14424	19.350238624239	19.350238624239\\
66.875	0.1479	20.7180203870506	20.7180203870506\\
66.875	0.15156	22.1521921003245	22.1521921003245\\
66.875	0.15522	23.6527537640609	23.6527537640609\\
66.875	0.15888	25.2197053782596	25.2197053782596\\
66.875	0.16254	26.8530469429208	26.8530469429208\\
66.875	0.1662	28.5527784580443	28.5527784580443\\
66.875	0.16986	30.3188999236301	30.3188999236301\\
66.875	0.17352	32.1514113396784	32.1514113396784\\
66.875	0.17718	34.050312706189	34.050312706189\\
66.875	0.18084	36.015604023162	36.015604023162\\
66.875	0.1845	38.0472852905975	38.0472852905975\\
66.875	0.18816	40.1453565084952	40.1453565084952\\
66.875	0.19182	42.3098176768554	42.3098176768554\\
66.875	0.19548	44.5406687956779	44.5406687956779\\
66.875	0.19914	46.8379098649629	46.8379098649629\\
66.875	0.2028	49.2015408847102	49.2015408847102\\
66.875	0.20646	51.6315618549199	51.6315618549199\\
66.875	0.21012	54.127972775592	54.127972775592\\
66.875	0.21378	56.6907736467264	56.6907736467264\\
66.875	0.21744	59.3199644683232	59.3199644683232\\
66.875	0.2211	62.0155452403824	62.0155452403824\\
66.875	0.22476	64.777515962904	64.777515962904\\
66.875	0.22842	67.605876635888	67.605876635888\\
66.875	0.23208	70.5006272593344	70.5006272593344\\
66.875	0.23574	73.4617678332432	73.4617678332432\\
66.875	0.2394	76.4892983576143	76.4892983576143\\
66.875	0.24306	79.5832188324478	79.5832188324478\\
66.875	0.24672	82.7435292577437	82.7435292577437\\
66.875	0.25038	85.9702296335019	85.9702296335019\\
66.875	0.25404	89.2633199597226	89.2633199597226\\
66.875	0.2577	92.6228002364056	92.6228002364056\\
66.875	0.26136	96.048670463551	96.048670463551\\
66.875	0.26502	99.5409306411588	99.5409306411588\\
66.875	0.26868	103.099580769229	103.099580769229\\
66.875	0.27234	106.724620847761	106.724620847761\\
66.875	0.276	110.416050876756	110.416050876756\\
67.25	0.093	7.2775504828756	7.2775504828756\\
67.25	0.09666	7.71629007854352	7.71629007854352\\
67.25	0.10032	8.22141962467381	8.22141962467381\\
67.25	0.10398	8.7929391212665	8.7929391212665\\
67.25	0.10764	9.43084856832156	9.43084856832156\\
67.25	0.1113	10.135147965839	10.135147965839\\
67.25	0.11496	10.9058373138188	10.9058373138188\\
67.25	0.11862	11.7429166122611	11.7429166122611\\
67.25	0.12228	12.6463858611656	12.6463858611656\\
67.25	0.12594	13.6162450605326	13.6162450605326\\
67.25	0.1296	14.6524942103619	14.6524942103619\\
67.25	0.13326	15.7551333106537	15.7551333106537\\
67.25	0.13692	16.9241623614078	16.9241623614078\\
67.25	0.14058	18.1595813626243	18.1595813626243\\
67.25	0.14424	19.4613903143032	19.4613903143032\\
67.25	0.1479	20.8295892164444	20.8295892164444\\
67.25	0.15156	22.2641780690481	22.2641780690481\\
67.25	0.15522	23.7651568721141	23.7651568721141\\
67.25	0.15888	25.3325256256425	25.3325256256425\\
67.25	0.16254	26.9662843296333	26.9662843296333\\
67.25	0.1662	28.6664329840864	28.6664329840864\\
67.25	0.16986	30.432971589002	30.432971589002\\
67.25	0.17352	32.2659001443799	32.2659001443799\\
67.25	0.17718	34.1652186502202	34.1652186502202\\
67.25	0.18084	36.1309271065229	36.1309271065229\\
67.25	0.1845	38.163025513288	38.163025513288\\
67.25	0.18816	40.2615138705154	40.2615138705154\\
67.25	0.19182	42.4263921782053	42.4263921782053\\
67.25	0.19548	44.6576604363575	44.6576604363575\\
67.25	0.19914	46.9553186449721	46.9553186449721\\
67.25	0.2028	49.319366804049	49.319366804049\\
67.25	0.20646	51.7498049135884	51.7498049135884\\
67.25	0.21012	54.2466329735901	54.2466329735901\\
67.25	0.21378	56.8098509840543	56.8098509840543\\
67.25	0.21744	59.4394589449807	59.4394589449807\\
67.25	0.2211	62.1354568563696	62.1354568563696\\
67.25	0.22476	64.8978447182209	64.8978447182209\\
67.25	0.22842	67.7266225305346	67.7266225305346\\
67.25	0.23208	70.6217902933106	70.6217902933106\\
67.25	0.23574	73.583348006549	73.583348006549\\
67.25	0.2394	76.6112956702498	76.6112956702498\\
67.25	0.24306	79.705633284413	79.705633284413\\
67.25	0.24672	82.8663608490385	82.8663608490385\\
67.25	0.25038	86.0934783641264	86.0934783641264\\
67.25	0.25404	89.3869858296768	89.3869858296768\\
67.25	0.2577	92.7468832456895	92.7468832456895\\
67.25	0.26136	96.1731706121645	96.1731706121645\\
67.25	0.26502	99.665847929102	99.665847929102\\
67.25	0.26868	103.224915196502	103.224915196502\\
67.25	0.27234	106.850372414364	106.850372414364\\
67.25	0.276	110.542219582689	110.542219582689\\
67.625	0.093	7.38660459751857	7.38660459751857\\
67.625	0.09666	7.82576133251617	7.82576133251617\\
67.625	0.10032	8.33130801797612	8.33130801797612\\
67.625	0.10398	8.90324465389847	8.90324465389847\\
67.625	0.10764	9.5415712402832	9.5415712402832\\
67.625	0.1113	10.2462877771303	10.2462877771303\\
67.625	0.11496	11.0173942644398	11.0173942644398\\
67.625	0.11862	11.8548907022117	11.8548907022117\\
67.625	0.12228	12.7587770904459	12.7587770904459\\
67.625	0.12594	13.7290534291426	13.7290534291426\\
67.625	0.1296	14.7657197183016	14.7657197183016\\
67.625	0.13326	15.868775957923	15.868775957923\\
67.625	0.13692	17.0382221480068	17.0382221480068\\
67.625	0.14058	18.2740582885529	18.2740582885529\\
67.625	0.14424	19.5762843795615	19.5762843795615\\
67.625	0.1479	20.9449004210324	20.9449004210324\\
67.625	0.15156	22.3799064129657	22.3799064129657\\
67.625	0.15522	23.8813023553614	23.8813023553614\\
67.625	0.15888	25.4490882482195	25.4490882482195\\
67.625	0.16254	27.0832640915399	27.0832640915399\\
67.625	0.1662	28.7838298853228	28.7838298853228\\
67.625	0.16986	30.550785629568	30.550785629568\\
67.625	0.17352	32.3841313242756	32.3841313242756\\
67.625	0.17718	34.2838669694455	34.2838669694455\\
67.625	0.18084	36.2499925650779	36.2499925650779\\
67.625	0.1845	38.2825081111726	38.2825081111726\\
67.625	0.18816	40.3814136077297	40.3814136077297\\
67.625	0.19182	42.5467090547492	42.5467090547492\\
67.625	0.19548	44.7783944522311	44.7783944522311\\
67.625	0.19914	47.0764698001754	47.0764698001754\\
67.625	0.2028	49.440935098582	49.440935098582\\
67.625	0.20646	51.8717903474511	51.8717903474511\\
67.625	0.21012	54.3690355467824	54.3690355467824\\
67.625	0.21378	56.9326706965762	56.9326706965762\\
67.625	0.21744	59.5626957968324	59.5626957968324\\
67.625	0.2211	62.259110847551	62.259110847551\\
67.625	0.22476	65.0219158487319	65.0219158487319\\
67.625	0.22842	67.8511108003752	67.8511108003752\\
67.625	0.23208	70.7466957024809	70.7466957024809\\
67.625	0.23574	73.708670555049	73.708670555049\\
67.625	0.2394	76.7370353580794	76.7370353580794\\
67.625	0.24306	79.8317901115722	79.8317901115722\\
67.625	0.24672	82.9929348155275	82.9929348155275\\
67.625	0.25038	86.2204694699451	86.2204694699451\\
67.625	0.25404	89.5143940748251	89.5143940748251\\
67.625	0.2577	92.8747086301674	92.8747086301674\\
67.625	0.26136	96.3014131359722	96.3014131359722\\
67.625	0.26502	99.7945075922393	99.7945075922393\\
67.625	0.26868	103.353991998969	103.353991998969\\
67.625	0.27234	106.979866356161	106.979866356161\\
67.625	0.276	110.672130663815	110.672130663815\\
68	0.093	7.49940108735561	7.49940108735561\\
68	0.09666	7.93897496168287	7.93897496168287\\
68	0.10032	8.44493878647251	8.44493878647251\\
68	0.10398	9.01729256172452	9.01729256172452\\
68	0.10764	9.65603628743891	9.65603628743891\\
68	0.1113	10.3611699636157	10.3611699636157\\
68	0.11496	11.1326935902548	11.1326935902548\\
68	0.11862	11.9706071673564	11.9706071673564\\
68	0.12228	12.8749106949203	12.8749106949203\\
68	0.12594	13.8456041729466	13.8456041729466\\
68	0.1296	14.8826876014353	14.8826876014353\\
68	0.13326	15.9861609803864	15.9861609803864\\
68	0.13692	17.1560243097998	17.1560243097998\\
68	0.14058	18.3922775896756	18.3922775896756\\
68	0.14424	19.6949208200139	19.6949208200139\\
68	0.1479	21.0639540008144	21.0639540008144\\
68	0.15156	22.4993771320774	22.4993771320774\\
68	0.15522	24.0011902138028	24.0011902138028\\
68	0.15888	25.5693932459905	25.5693932459905\\
68	0.16254	27.2039862286406	27.2039862286406\\
68	0.1662	28.9049691617531	28.9049691617531\\
68	0.16986	30.672342045328	30.672342045328\\
68	0.17352	32.5061048793653	32.5061048793653\\
68	0.17718	34.4062576638649	34.4062576638649\\
68	0.18084	36.3728003988269	36.3728003988269\\
68	0.1845	38.4057330842514	38.4057330842514\\
68	0.18816	40.5050557201381	40.5050557201381\\
68	0.19182	42.6707683064873	42.6707683064873\\
68	0.19548	44.9028708432988	44.9028708432988\\
68	0.19914	47.2013633305728	47.2013633305728\\
68	0.2028	49.5662457683091	49.5662457683091\\
68	0.20646	51.9975181565078	51.9975181565078\\
68	0.21012	54.4951804951689	54.4951804951689\\
68	0.21378	57.0592327842923	57.0592327842923\\
68	0.21744	59.6896750238781	59.6896750238781\\
68	0.2211	62.3865072139263	62.3865072139263\\
68	0.22476	65.1497293544369	65.1497293544369\\
68	0.22842	67.9793414454099	67.9793414454099\\
68	0.23208	70.8753434868453	70.8753434868453\\
68	0.23574	73.837735478743	73.837735478743\\
68	0.2394	76.8665174211032	76.8665174211032\\
68	0.24306	79.9616893139257	79.9616893139257\\
68	0.24672	83.1232511572106	83.1232511572106\\
68	0.25038	86.3512029509578	86.3512029509578\\
68	0.25404	89.6455446951675	89.6455446951675\\
68	0.2577	93.0062763898395	93.0062763898395\\
68	0.26136	96.4333980349739	96.4333980349739\\
68	0.26502	99.9269096305707	99.9269096305707\\
68	0.26868	103.48681117663	103.48681117663\\
68	0.27234	107.113102673151	107.113102673151\\
68	0.276	110.805784120135	110.805784120135\\
68.375	0.093	7.61593995238678	7.61593995238678\\
68.375	0.09666	8.0559309660437	8.0559309660437\\
68.375	0.10032	8.562311930163	8.562311930163\\
68.375	0.10398	9.13508284474469	9.13508284474469\\
68.375	0.10764	9.77424370978874	9.77424370978874\\
68.375	0.1113	10.4797945252952	10.4797945252952\\
68.375	0.11496	11.251735291264	11.251735291264\\
68.375	0.11862	12.0900660076952	12.0900660076952\\
68.375	0.12228	12.9947866745888	12.9947866745888\\
68.375	0.12594	13.9658972919448	13.9658972919448\\
68.375	0.1296	15.0033978597631	15.0033978597631\\
68.375	0.13326	16.1072883780439	16.1072883780439\\
68.375	0.13692	17.277568846787	17.277568846787\\
68.375	0.14058	18.5142392659925	18.5142392659925\\
68.375	0.14424	19.8172996356604	19.8172996356604\\
68.375	0.1479	21.1867499557906	21.1867499557906\\
68.375	0.15156	22.6225902263833	22.6225902263833\\
68.375	0.15522	24.1248204474383	24.1248204474383\\
68.375	0.15888	25.6934406189557	25.6934406189557\\
68.375	0.16254	27.3284507409355	27.3284507409355\\
68.375	0.1662	29.0298508133776	29.0298508133776\\
68.375	0.16986	30.7976408362822	30.7976408362822\\
68.375	0.17352	32.6318208096491	32.6318208096491\\
68.375	0.17718	34.5323907334784	34.5323907334784\\
68.375	0.18084	36.4993506077701	36.4993506077701\\
68.375	0.1845	38.5327004325242	38.5327004325242\\
68.375	0.18816	40.6324402077406	40.6324402077406\\
68.375	0.19182	42.7985699334195	42.7985699334195\\
68.375	0.19548	45.0310896095607	45.0310896095607\\
68.375	0.19914	47.3299992361643	47.3299992361643\\
68.375	0.2028	49.6952988132303	49.6952988132303\\
68.375	0.20646	52.1269883407586	52.1269883407586\\
68.375	0.21012	54.6250678187494	54.6250678187494\\
68.375	0.21378	57.1895372472025	57.1895372472025\\
68.375	0.21744	59.8203966261179	59.8203966261179\\
68.375	0.2211	62.5176459554958	62.5176459554958\\
68.375	0.22476	65.2812852353361	65.2812852353361\\
68.375	0.22842	68.1113144656388	68.1113144656388\\
68.375	0.23208	71.0077336464038	71.0077336464038\\
68.375	0.23574	73.9705427776312	73.9705427776312\\
68.375	0.2394	76.999741859321	76.999741859321\\
68.375	0.24306	80.0953308914732	80.0953308914732\\
68.375	0.24672	83.2573098740877	83.2573098740877\\
68.375	0.25038	86.4856788071646	86.4856788071646\\
68.375	0.25404	89.780437690704	89.780437690704\\
68.375	0.2577	93.1415865247057	93.1415865247057\\
68.375	0.26136	96.5691253091697	96.5691253091697\\
68.375	0.26502	100.063054044096	100.063054044096\\
68.375	0.26868	103.623372729485	103.623372729485\\
68.375	0.27234	107.250081365336	107.250081365336\\
68.375	0.276	110.94317995165	110.94317995165\\
68.75	0.093	7.73622119261202	7.73622119261202\\
68.75	0.09666	8.1766293455986	8.1766293455986\\
68.75	0.10032	8.68342744904757	8.68342744904757\\
68.75	0.10398	9.25661550295891	9.25661550295891\\
68.75	0.10764	9.89619350733265	9.89619350733265\\
68.75	0.1113	10.6021614621688	10.6021614621688\\
68.75	0.11496	11.3745193674673	11.3745193674673\\
68.75	0.11862	12.2132672232281	12.2132672232281\\
68.75	0.12228	13.1184050294514	13.1184050294514\\
68.75	0.12594	14.089932786137	14.089932786137\\
68.75	0.1296	15.127850493285	15.127850493285\\
68.75	0.13326	16.2321581508954	16.2321581508954\\
68.75	0.13692	17.4028557589682	17.4028557589682\\
68.75	0.14058	18.6399433175034	18.6399433175034\\
68.75	0.14424	19.9434208265009	19.9434208265009\\
68.75	0.1479	21.3132882859609	21.3132882859609\\
68.75	0.15156	22.7495456958832	22.7495456958832\\
68.75	0.15522	24.2521930562679	24.2521930562679\\
68.75	0.15888	25.8212303671149	25.8212303671149\\
68.75	0.16254	27.4566576284244	27.4566576284244\\
68.75	0.1662	29.1584748401962	29.1584748401962\\
68.75	0.16986	30.9266820024304	30.9266820024304\\
68.75	0.17352	32.761279115127	32.761279115127\\
68.75	0.17718	34.662266178286	34.662266178286\\
68.75	0.18084	36.6296431919073	36.6296431919073\\
68.75	0.1845	38.6634101559911	38.6634101559911\\
68.75	0.18816	40.7635670705372	40.7635670705372\\
68.75	0.19182	42.9301139355457	42.9301139355457\\
68.75	0.19548	45.1630507510166	45.1630507510166\\
68.75	0.19914	47.4623775169499	47.4623775169499\\
68.75	0.2028	49.8280942333455	49.8280942333455\\
68.75	0.20646	52.2602009002035	52.2602009002035\\
68.75	0.21012	54.7586975175239	54.7586975175239\\
68.75	0.21378	57.3235840853067	57.3235840853067\\
68.75	0.21744	59.9548606035519	59.9548606035519\\
68.75	0.2211	62.6525270722594	62.6525270722594\\
68.75	0.22476	65.4165834914294	65.4165834914294\\
68.75	0.22842	68.2470298610617	68.2470298610617\\
68.75	0.23208	71.1438661811564	71.1438661811564\\
68.75	0.23574	74.1070924517135	74.1070924517135\\
68.75	0.2394	77.1367086727329	77.1367086727329\\
68.75	0.24306	80.2327148442147	80.2327148442147\\
68.75	0.24672	83.395110966159	83.395110966159\\
68.75	0.25038	86.6238970385656	86.6238970385656\\
68.75	0.25404	89.9190730614346	89.9190730614346\\
68.75	0.2577	93.280639034766	93.280639034766\\
68.75	0.26136	96.7085949585597	96.7085949585597\\
68.75	0.26502	100.202940832816	100.202940832816\\
68.75	0.26868	103.763676657534	103.763676657534\\
68.75	0.27234	107.390802432715	107.390802432715\\
68.75	0.276	111.084318158358	111.084318158358\\
69.125	0.093	7.86024480803135	7.86024480803135\\
69.125	0.09666	8.30107010034762	8.30107010034762\\
69.125	0.10032	8.80828534312625	8.80828534312625\\
69.125	0.10398	9.38189053636727	9.38189053636727\\
69.125	0.10764	10.0218856800707	10.0218856800707\\
69.125	0.1113	10.7282707742364	10.7282707742364\\
69.125	0.11496	11.5010458188646	11.5010458188646\\
69.125	0.11862	12.3402108139551	12.3402108139551\\
69.125	0.12228	13.2457657595081	13.2457657595081\\
69.125	0.12594	14.2177106555234	14.2177106555234\\
69.125	0.1296	15.2560455020011	15.2560455020011\\
69.125	0.13326	16.3607702989411	16.3607702989411\\
69.125	0.13692	17.5318850463436	17.5318850463436\\
69.125	0.14058	18.7693897442084	18.7693897442084\\
69.125	0.14424	20.0732843925356	20.0732843925356\\
69.125	0.1479	21.4435689913252	21.4435689913252\\
69.125	0.15156	22.8802435405772	22.8802435405772\\
69.125	0.15522	24.3833080402915	24.3833080402915\\
69.125	0.15888	25.9527624904683	25.9527624904683\\
69.125	0.16254	27.5886068911074	27.5886068911074\\
69.125	0.1662	29.2908412422089	29.2908412422089\\
69.125	0.16986	31.0594655437728	31.0594655437728\\
69.125	0.17352	32.894479795799	32.894479795799\\
69.125	0.17718	34.7958839982877	34.7958839982877\\
69.125	0.18084	36.7636781512387	36.7636781512387\\
69.125	0.1845	38.7978622546521	38.7978622546521\\
69.125	0.18816	40.8984363085279	40.8984363085279\\
69.125	0.19182	43.065400312866	43.065400312866\\
69.125	0.19548	45.2987542676666	45.2987542676666\\
69.125	0.19914	47.5984981729295	47.5984981729295\\
69.125	0.2028	49.9646320286549	49.9646320286549\\
69.125	0.20646	52.3971558348426	52.3971558348426\\
69.125	0.21012	54.8960695914926	54.8960695914926\\
69.125	0.21378	57.4613732986051	57.4613732986051\\
69.125	0.21744	60.0930669561799	60.0930669561799\\
69.125	0.2211	62.7911505642171	62.7911505642171\\
69.125	0.22476	65.5556241227167	65.5556241227167\\
69.125	0.22842	68.3864876316787	68.3864876316787\\
69.125	0.23208	71.283741091103	71.283741091103\\
69.125	0.23574	74.2473845009898	74.2473845009898\\
69.125	0.2394	77.277417861339	77.277417861339\\
69.125	0.24306	80.3738411721505	80.3738411721505\\
69.125	0.24672	83.5366544334243	83.5366544334243\\
69.125	0.25038	86.7658576451606	86.7658576451606\\
69.125	0.25404	90.0614508073593	90.0614508073593\\
69.125	0.2577	93.4234339200203	93.4234339200203\\
69.125	0.26136	96.8518069831437	96.8518069831437\\
69.125	0.26502	100.346569996729	100.346569996729\\
69.125	0.26868	103.907722960778	103.907722960778\\
69.125	0.27234	107.535265875288	107.535265875288\\
69.125	0.276	111.229198740261	111.229198740261\\
69.5	0.093	7.9880107986448	7.9880107986448\\
69.5	0.09666	8.42925323029072	8.42925323029072\\
69.5	0.10032	8.93688561239903	8.93688561239903\\
69.5	0.10398	9.51090794496972	9.51090794496972\\
69.5	0.10764	10.1513202280028	10.1513202280028\\
69.5	0.1113	10.8581224614982	10.8581224614982\\
69.5	0.11496	11.631314645456	11.631314645456\\
69.5	0.11862	12.4708967798762	12.4708967798762\\
69.5	0.12228	13.3768688647588	13.3768688647588\\
69.5	0.12594	14.3492309001038	14.3492309001038\\
69.5	0.1296	15.3879828859112	15.3879828859112\\
69.5	0.13326	16.4931248221809	16.4931248221809\\
69.5	0.13692	17.664656708913	17.664656708913\\
69.5	0.14058	18.9025785461075	18.9025785461075\\
69.5	0.14424	20.2068903337644	20.2068903337644\\
69.5	0.1479	21.5775920718837	21.5775920718837\\
69.5	0.15156	23.0146837604653	23.0146837604653\\
69.5	0.15522	24.5181653995093	24.5181653995093\\
69.5	0.15888	26.0880369890157	26.0880369890157\\
69.5	0.16254	27.7242985289845	27.7242985289845\\
69.5	0.1662	29.4269500194157	29.4269500194157\\
69.5	0.16986	31.1959914603092	31.1959914603092\\
69.5	0.17352	33.0314228516652	33.0314228516652\\
69.5	0.17718	34.9332441934835	34.9332441934835\\
69.5	0.18084	36.9014554857641	36.9014554857641\\
69.5	0.1845	38.9360567285072	38.9360567285072\\
69.5	0.18816	41.0370479217127	41.0370479217127\\
69.5	0.19182	43.2044290653805	43.2044290653805\\
69.5	0.19548	45.4382001595107	45.4382001595107\\
69.5	0.19914	47.7383612041033	47.7383612041033\\
69.5	0.2028	50.1049121991583	50.1049121991583\\
69.5	0.20646	52.5378531446757	52.5378531446757\\
69.5	0.21012	55.0371840406554	55.0371840406554\\
69.5	0.21378	57.6029048870975	57.6029048870975\\
69.5	0.21744	60.235015684002	60.235015684002\\
69.5	0.2211	62.9335164313689	62.9335164313689\\
69.5	0.22476	65.6984071291982	65.6984071291982\\
69.5	0.22842	68.5296877774898	68.5296877774898\\
69.5	0.23208	71.4273583762438	71.4273583762438\\
69.5	0.23574	74.3914189254603	74.3914189254603\\
69.5	0.2394	77.4218694251391	77.4218694251391\\
69.5	0.24306	80.5187098752802	80.5187098752802\\
69.5	0.24672	83.6819402758838	83.6819402758838\\
69.5	0.25038	86.9115606269497	86.9115606269497\\
69.5	0.25404	90.207570928478	90.207570928478\\
69.5	0.2577	93.5699711804687	93.5699711804687\\
69.5	0.26136	96.9987613829218	96.9987613829218\\
69.5	0.26502	100.493941535837	100.493941535837\\
69.5	0.26868	104.055511639215	104.055511639215\\
69.5	0.27234	107.683471693055	107.683471693055\\
69.5	0.276	111.377821697358	111.377821697358\\
69.875	0.093	8.11951916445235	8.11951916445235\\
69.875	0.09666	8.56117873542794	8.56117873542794\\
69.875	0.10032	9.0692282568659	9.0692282568659\\
69.875	0.10398	9.64366772876625	9.64366772876625\\
69.875	0.10764	10.284497151129	10.284497151129\\
69.875	0.1113	10.9917165239541	10.9917165239541\\
69.875	0.11496	11.7653258472416	11.7653258472416\\
69.875	0.11862	12.6053251209915	12.6053251209915\\
69.875	0.12228	13.5117143452037	13.5117143452037\\
69.875	0.12594	14.4844935198784	14.4844935198784\\
69.875	0.1296	15.5236626450154	15.5236626450154\\
69.875	0.13326	16.6292217206148	16.6292217206148\\
69.875	0.13692	17.8011707466766	17.8011707466766\\
69.875	0.14058	19.0395097232007	19.0395097232007\\
69.875	0.14424	20.3442386501873	20.3442386501873\\
69.875	0.1479	21.7153575276362	21.7153575276362\\
69.875	0.15156	23.1528663555475	23.1528663555475\\
69.875	0.15522	24.6567651339212	24.6567651339212\\
69.875	0.15888	26.2270538627573	26.2270538627573\\
69.875	0.16254	27.8637325420557	27.8637325420557\\
69.875	0.1662	29.5668011718166	29.5668011718166\\
69.875	0.16986	31.3362597520398	31.3362597520398\\
69.875	0.17352	33.1721082827254	33.1721082827254\\
69.875	0.17718	35.0743467638733	35.0743467638733\\
69.875	0.18084	37.0429751954837	37.0429751954837\\
69.875	0.1845	39.0779935775564	39.0779935775564\\
69.875	0.18816	41.1794019100916	41.1794019100916\\
69.875	0.19182	43.3472001930891	43.3472001930891\\
69.875	0.19548	45.581388426549	45.581388426549\\
69.875	0.19914	47.8819666104712	47.8819666104712\\
69.875	0.2028	50.2489347448559	50.2489347448559\\
69.875	0.20646	52.6822928297029	52.6822928297029\\
69.875	0.21012	55.1820408650123	55.1820408650123\\
69.875	0.21378	57.7481788507841	57.7481788507841\\
69.875	0.21744	60.3807067870182	60.3807067870182\\
69.875	0.2211	63.0796246737148	63.0796246737148\\
69.875	0.22476	65.8449325108737	65.8449325108737\\
69.875	0.22842	68.6766302984951	68.6766302984951\\
69.875	0.23208	71.5747180365787	71.5747180365787\\
69.875	0.23574	74.5391957251248	74.5391957251248\\
69.875	0.2394	77.5700633641333	77.5700633641333\\
69.875	0.24306	80.6673209536041	80.6673209536041\\
69.875	0.24672	83.8309684935373	83.8309684935373\\
69.875	0.25038	87.0610059839329	87.0610059839329\\
69.875	0.25404	90.3574334247909	90.3574334247909\\
69.875	0.2577	93.7202508161113	93.7202508161113\\
69.875	0.26136	97.1494581578941	97.1494581578941\\
69.875	0.26502	100.645055450139	100.645055450139\\
69.875	0.26868	104.207042692847	104.207042692847\\
69.875	0.27234	107.835419886017	107.835419886017\\
69.875	0.276	111.530187029649	111.530187029649\\
70.25	0.093	8.25476990545398	8.25476990545398\\
70.25	0.09666	8.69684661575922	8.69684661575922\\
70.25	0.10032	9.20531327652687	9.20531327652687\\
70.25	0.10398	9.78016988775689	9.78016988775689\\
70.25	0.10764	10.4214164494493	10.4214164494493\\
70.25	0.1113	11.129052961604	11.129052961604\\
70.25	0.11496	11.9030794242212	11.9030794242212\\
70.25	0.11862	12.7434958373008	12.7434958373008\\
70.25	0.12228	13.6503022008427	13.6503022008427\\
70.25	0.12594	14.623498514847	14.623498514847\\
70.25	0.1296	15.6630847793137	15.6630847793137\\
70.25	0.13326	16.7690609942427	16.7690609942427\\
70.25	0.13692	17.9414271596342	17.9414271596342\\
70.25	0.14058	19.180183275488	19.180183275488\\
70.25	0.14424	20.4853293418043	20.4853293418043\\
70.25	0.1479	21.8568653585828	21.8568653585828\\
70.25	0.15156	23.2947913258238	23.2947913258238\\
70.25	0.15522	24.7991072435272	24.7991072435272\\
70.25	0.15888	26.3698131116929	26.3698131116929\\
70.25	0.16254	28.006908930321	28.006908930321\\
70.25	0.1662	29.7103946994115	29.7103946994115\\
70.25	0.16986	31.4802704189644	31.4802704189644\\
70.25	0.17352	33.3165360889797	33.3165360889797\\
70.25	0.17718	35.2191917094573	35.2191917094573\\
70.25	0.18084	37.1882372803973	37.1882372803973\\
70.25	0.1845	39.2236728017998	39.2236728017998\\
70.25	0.18816	41.3254982736645	41.3254982736645\\
70.25	0.19182	43.4937136959917	43.4937136959917\\
70.25	0.19548	45.7283190687812	45.7283190687812\\
70.25	0.19914	48.0293143920332	48.0293143920332\\
70.25	0.2028	50.3966996657475	50.3966996657475\\
70.25	0.20646	52.8304748899242	52.8304748899242\\
70.25	0.21012	55.3306400645633	55.3306400645633\\
70.25	0.21378	57.8971951896647	57.8971951896647\\
70.25	0.21744	60.5301402652285	60.5301402652285\\
70.25	0.2211	63.2294752912547	63.2294752912547\\
70.25	0.22476	65.9952002677434	65.9952002677434\\
70.25	0.22842	68.8273151946943	68.8273151946943\\
70.25	0.23208	71.7258200721077	71.7258200721077\\
70.25	0.23574	74.6907148999835	74.6907148999835\\
70.25	0.2394	77.7219996783216	77.7219996783216\\
70.25	0.24306	80.8196744071221	80.8196744071221\\
70.25	0.24672	83.983739086385	83.983739086385\\
70.25	0.25038	87.2141937161103	87.2141937161103\\
70.25	0.25404	90.5110382962979	90.5110382962979\\
70.25	0.2577	93.8742728269479	93.8742728269479\\
70.25	0.26136	97.3038973080604	97.3038973080604\\
70.25	0.26502	100.799911739635	100.799911739635\\
70.25	0.26868	104.362316121672	104.362316121672\\
70.25	0.27234	107.991110454172	107.991110454172\\
70.25	0.276	111.686294737134	111.686294737134\\
70.625	0.093	8.39376302164971	8.39376302164971\\
70.625	0.09666	8.83625687128464	8.83625687128464\\
70.625	0.10032	9.34514067138194	9.34514067138194\\
70.625	0.10398	9.92041442194164	9.92041442194164\\
70.625	0.10764	10.5620781229637	10.5620781229637\\
70.625	0.1113	11.2701317744481	11.2701317744481\\
70.625	0.11496	12.044575376395	12.044575376395\\
70.625	0.11862	12.8854089288042	12.8854089288042\\
70.625	0.12228	13.7926324316758	13.7926324316758\\
70.625	0.12594	14.7662458850097	14.7662458850097\\
70.625	0.1296	15.8062492888061	15.8062492888061\\
70.625	0.13326	16.9126426430648	16.9126426430648\\
70.625	0.13692	18.0854259477859	18.0854259477859\\
70.625	0.14058	19.3245992029694	19.3245992029694\\
70.625	0.14424	20.6301624086153	20.6301624086153\\
70.625	0.1479	22.0021155647236	22.0021155647236\\
70.625	0.15156	23.4404586712942	23.4404586712942\\
70.625	0.15522	24.9451917283273	24.9451917283273\\
70.625	0.15888	26.5163147358227	26.5163147358227\\
70.625	0.16254	28.1538276937805	28.1538276937805\\
70.625	0.1662	29.8577306022006	29.8577306022006\\
70.625	0.16986	31.6280234610832	31.6280234610832\\
70.625	0.17352	33.4647062704281	33.4647062704281\\
70.625	0.17718	35.3677790302354	35.3677790302354\\
70.625	0.18084	37.3372417405051	37.3372417405051\\
70.625	0.1845	39.3730944012372	39.3730944012372\\
70.625	0.18816	41.4753370124316	41.4753370124316\\
70.625	0.19182	43.6439695740884	43.6439695740884\\
70.625	0.19548	45.8789920862077	45.8789920862077\\
70.625	0.19914	48.1804045487893	48.1804045487893\\
70.625	0.2028	50.5482069618333	50.5482069618333\\
70.625	0.20646	52.9823993253396	52.9823993253396\\
70.625	0.21012	55.4829816393084	55.4829816393084\\
70.625	0.21378	58.0499539037395	58.0499539037395\\
70.625	0.21744	60.683316118633	60.683316118633\\
70.625	0.2211	63.3830682839888	63.3830682839888\\
70.625	0.22476	66.1492103998071	66.1492103998071\\
70.625	0.22842	68.9817424660878	68.9817424660878\\
70.625	0.23208	71.8806644828308	71.8806644828308\\
70.625	0.23574	74.8459764500362	74.8459764500362\\
70.625	0.2394	77.877678367704	77.877678367704\\
70.625	0.24306	80.9757702358342	80.9757702358342\\
70.625	0.24672	84.1402520544267	84.1402520544267\\
70.625	0.25038	87.3711238234817	87.3711238234817\\
70.625	0.25404	90.668385542999	90.668385542999\\
70.625	0.2577	94.0320372129787	94.0320372129787\\
70.625	0.26136	97.4620788334208	97.4620788334208\\
70.625	0.26502	100.958510404325	100.958510404325\\
70.625	0.26868	104.521331925692	104.521331925692\\
70.625	0.27234	108.150543397521	108.150543397521\\
70.625	0.276	111.846144819813	111.846144819813\\
71	0.093	8.53649851303954	8.53649851303954\\
71	0.09666	8.97940950200414	8.97940950200414\\
71	0.10032	9.4887104414311	9.4887104414311\\
71	0.10398	10.0644013313204	10.0644013313204\\
71	0.10764	10.7064821716722	10.7064821716722\\
71	0.1113	11.4149529624863	11.4149529624863\\
71	0.11496	12.1898137037628	12.1898137037628\\
71	0.11862	13.0310643955017	13.0310643955017\\
71	0.12228	13.9387050377029	13.9387050377029\\
71	0.12594	14.9127356303666	14.9127356303666\\
71	0.1296	15.9531561734926	15.9531561734926\\
71	0.13326	17.059966667081	17.059966667081\\
71	0.13692	18.2331671111318	18.2331671111318\\
71	0.14058	19.4727575056449	19.4727575056449\\
71	0.14424	20.7787378506205	20.7787378506205\\
71	0.1479	22.1511081460584	22.1511081460584\\
71	0.15156	23.5898683919587	23.5898683919587\\
71	0.15522	25.0950185883214	25.0950185883214\\
71	0.15888	26.6665587351465	26.6665587351465\\
71	0.16254	28.3044888324339	28.3044888324339\\
71	0.1662	30.0088088801838	30.0088088801838\\
71	0.16986	31.779518878396	31.779518878396\\
71	0.17352	33.6166188270706	33.6166188270706\\
71	0.17718	35.5201087262076	35.5201087262076\\
71	0.18084	37.4899885758069	37.4899885758069\\
71	0.1845	39.5262583758687	39.5262583758687\\
71	0.18816	41.6289181263928	41.6289181263928\\
71	0.19182	43.7979678273793	43.7979678273793\\
71	0.19548	46.0334074788282	46.0334074788282\\
71	0.19914	48.3352370807394	48.3352370807394\\
71	0.2028	50.7034566331131	50.7034566331131\\
71	0.20646	53.1380661359491	53.1380661359491\\
71	0.21012	55.6390655892475	55.6390655892475\\
71	0.21378	58.2064549930083	58.2064549930083\\
71	0.21744	60.8402343472315	60.8402343472315\\
71	0.2211	63.540403651917	63.540403651917\\
71	0.22476	66.306962907065	66.306962907065\\
71	0.22842	69.1399121126753	69.1399121126753\\
71	0.23208	72.039251268748	72.039251268748\\
71	0.23574	75.0049803752831	75.0049803752831\\
71	0.2394	78.0370994322805	78.0370994322805\\
71	0.24306	81.1356084397404	81.1356084397404\\
71	0.24672	84.3005073976626	84.3005073976626\\
71	0.25038	87.5317963060472	87.5317963060472\\
71	0.25404	90.8294751648942	90.8294751648942\\
71	0.2577	94.1935439742035	94.1935439742035\\
71	0.26136	97.6240027339753	97.6240027339753\\
71	0.26502	101.120851444209	101.120851444209\\
71	0.26868	104.684090104906	104.684090104906\\
71	0.27234	108.313718716065	108.313718716065\\
71	0.276	112.009737277686	112.009737277686\\
71.375	0.093	8.68297637962346	8.68297637962346\\
71.375	0.09666	9.12630450791772	9.12630450791772\\
71.375	0.10032	9.63602258667436	9.63602258667436\\
71.375	0.10398	10.2121306158934	10.2121306158934\\
71.375	0.10764	10.8546285955748	10.8546285955748\\
71.375	0.1113	11.5635165257185	11.5635165257185\\
71.375	0.11496	12.3387944063247	12.3387944063247\\
71.375	0.11862	13.1804622373933	13.1804622373933\\
71.375	0.12228	14.0885200189242	14.0885200189242\\
71.375	0.12594	15.0629677509175	15.0629677509175\\
71.375	0.1296	16.1038054333732	16.1038054333732\\
71.375	0.13326	17.2110330662912	17.2110330662912\\
71.375	0.13692	18.3846506496717	18.3846506496717\\
71.375	0.14058	19.6246581835145	19.6246581835145\\
71.375	0.14424	20.9310556678198	20.9310556678198\\
71.375	0.1479	22.3038431025873	22.3038431025873\\
71.375	0.15156	23.7430204878173	23.7430204878173\\
71.375	0.15522	25.2485878235097	25.2485878235097\\
71.375	0.15888	26.8205451096644	26.8205451096644\\
71.375	0.16254	28.4588923462815	28.4588923462815\\
71.375	0.1662	30.163629533361	30.163629533361\\
71.375	0.16986	31.9347566709029	31.9347566709029\\
71.375	0.17352	33.7722737589072	33.7722737589072\\
71.375	0.17718	35.6761807973738	35.6761807973738\\
71.375	0.18084	37.6464777863028	37.6464777863028\\
71.375	0.1845	39.6831647256943	39.6831647256943\\
71.375	0.18816	41.786241615548	41.786241615548\\
71.375	0.19182	43.9557084558642	43.9557084558642\\
71.375	0.19548	46.1915652466428	46.1915652466428\\
71.375	0.19914	48.4938119878837	48.4938119878837\\
71.375	0.2028	50.862448679587	50.862448679587\\
71.375	0.20646	53.2974753217527	53.2974753217527\\
71.375	0.21012	55.7988919143808	55.7988919143808\\
71.375	0.21378	58.3666984574713	58.3666984574713\\
71.375	0.21744	61.0008949510241	61.0008949510241\\
71.375	0.2211	63.7014813950393	63.7014813950393\\
71.375	0.22476	66.4684577895169	66.4684577895169\\
71.375	0.22842	69.3018241344569	69.3018241344569\\
71.375	0.23208	72.2015804298592	72.2015804298592\\
71.375	0.23574	75.167726675724	75.167726675724\\
71.375	0.2394	78.2002628720511	78.2002628720511\\
71.375	0.24306	81.2991890188406	81.2991890188406\\
71.375	0.24672	84.4645051160925	84.4645051160925\\
71.375	0.25038	87.6962111638068	87.6962111638068\\
71.375	0.25404	90.9943071619834	90.9943071619834\\
71.375	0.2577	94.3587931106225	94.3587931106225\\
71.375	0.26136	97.7896690097239	97.7896690097239\\
71.375	0.26502	101.286934859288	101.286934859288\\
71.375	0.26868	104.850590659314	104.850590659314\\
71.375	0.27234	108.480636409802	108.480636409802\\
71.375	0.276	112.177072110753	112.177072110753\\
71.75	0.093	8.83319662140149	8.83319662140149\\
71.75	0.09666	9.27694188902542	9.27694188902542\\
71.75	0.10032	9.78707710711172	9.78707710711172\\
71.75	0.10398	10.3636022756604	10.3636022756604\\
71.75	0.10764	11.0065173946715	11.0065173946715\\
71.75	0.1113	11.7158224641449	11.7158224641449\\
71.75	0.11496	12.4915174840807	12.4915174840807\\
71.75	0.11862	13.333602454479	13.333602454479\\
71.75	0.12228	14.2420773753395	14.2420773753395\\
71.75	0.12594	15.2169422466625	15.2169422466625\\
71.75	0.1296	16.2581970684479	16.2581970684479\\
71.75	0.13326	17.3658418406956	17.3658418406956\\
71.75	0.13692	18.5398765634057	18.5398765634057\\
71.75	0.14058	19.7803012365782	19.7803012365782\\
71.75	0.14424	21.0871158602131	21.0871158602131\\
71.75	0.1479	22.4603204343104	22.4603204343104\\
71.75	0.15156	23.89991495887	23.89991495887\\
71.75	0.15522	25.405899433892	25.405899433892\\
71.75	0.15888	26.9782738593765	26.9782738593765\\
71.75	0.16254	28.6170382353232	28.6170382353232\\
71.75	0.1662	30.3221925617324	30.3221925617324\\
71.75	0.16986	32.093736838604	32.093736838604\\
71.75	0.17352	33.9316710659379	33.9316710659379\\
71.75	0.17718	35.8359952437342	35.8359952437342\\
71.75	0.18084	37.8067093719929	37.8067093719929\\
71.75	0.1845	39.843813450714	39.843813450714\\
71.75	0.18816	41.9473074798974	41.9473074798974\\
71.75	0.19182	44.1171914595432	44.1171914595432\\
71.75	0.19548	46.3534653896515	46.3534653896515\\
71.75	0.19914	48.6561292702221	48.6561292702221\\
71.75	0.2028	51.0251831012551	51.0251831012551\\
71.75	0.20646	53.4606268827504	53.4606268827504\\
71.75	0.21012	55.9624606147082	55.9624606147082\\
71.75	0.21378	58.5306842971283	58.5306842971283\\
71.75	0.21744	61.1652979300108	61.1652979300108\\
71.75	0.2211	63.8663015133557	63.8663015133557\\
71.75	0.22476	66.6336950471629	66.6336950471629\\
71.75	0.22842	69.4674785314326	69.4674785314326\\
71.75	0.23208	72.3676519661646	72.3676519661646\\
71.75	0.23574	75.334215351359	75.334215351359\\
71.75	0.2394	78.3671686870158	78.3671686870158\\
71.75	0.24306	81.466511973135	81.466511973135\\
71.75	0.24672	84.6322452097166	84.6322452097166\\
71.75	0.25038	87.8643683967605	87.8643683967605\\
71.75	0.25404	91.1628815342668	91.1628815342668\\
71.75	0.2577	94.5277846222355	94.5277846222355\\
71.75	0.26136	97.9590776606666	97.9590776606666\\
71.75	0.26502	101.45676064956	101.45676064956\\
71.75	0.26868	105.020833588916	105.020833588916\\
71.75	0.27234	108.651296478734	108.651296478734\\
71.75	0.276	112.348149319015	112.348149319015\\
72.125	0.093	8.98715923837361	8.98715923837361\\
72.125	0.09666	9.43132164532719	9.43132164532719\\
72.125	0.10032	9.94187400274317	9.94187400274317\\
72.125	0.10398	10.5188163106215	10.5188163106215\\
72.125	0.10764	11.1621485689623	11.1621485689623\\
72.125	0.1113	11.8718707777654	11.8718707777654\\
72.125	0.11496	12.6479829370309	12.6479829370309\\
72.125	0.11862	13.4904850467588	13.4904850467588\\
72.125	0.12228	14.399377106949	14.399377106949\\
72.125	0.12594	15.3746591176017	15.3746591176017\\
72.125	0.1296	16.4163310787167	16.4163310787167\\
72.125	0.13326	17.5243929902941	17.5243929902941\\
72.125	0.13692	18.6988448523338	18.6988448523338\\
72.125	0.14058	19.939686664836	19.939686664836\\
72.125	0.14424	21.2469184278006	21.2469184278006\\
72.125	0.1479	22.6205401412275	22.6205401412275\\
72.125	0.15156	24.0605518051168	24.0605518051168\\
72.125	0.15522	25.5669534194685	25.5669534194685\\
72.125	0.15888	27.1397449842826	27.1397449842826\\
72.125	0.16254	28.778926499559	28.778926499559\\
72.125	0.1662	30.4844979652979	30.4844979652979\\
72.125	0.16986	32.2564593814991	32.2564593814991\\
72.125	0.17352	34.0948107481627	34.0948107481627\\
72.125	0.17718	35.9995520652886	35.9995520652886\\
72.125	0.18084	37.970683332877	37.970683332877\\
72.125	0.1845	40.0082045509278	40.0082045509278\\
72.125	0.18816	42.1121157194409	42.1121157194409\\
72.125	0.19182	44.2824168384164	44.2824168384164\\
72.125	0.19548	46.5191079078543	46.5191079078543\\
72.125	0.19914	48.8221889277545	48.8221889277545\\
72.125	0.2028	51.1916598981172	51.1916598981172\\
72.125	0.20646	53.6275208189422	53.6275208189422\\
72.125	0.21012	56.1297716902296	56.1297716902296\\
72.125	0.21378	58.6984125119794	58.6984125119794\\
72.125	0.21744	61.3334432841916	61.3334432841916\\
72.125	0.2211	64.0348640068661	64.0348640068661\\
72.125	0.22476	66.8026746800031	66.8026746800031\\
72.125	0.22842	69.6368753036024	69.6368753036024\\
72.125	0.23208	72.5374658776641	72.5374658776641\\
72.125	0.23574	75.5044464021882	75.5044464021882\\
72.125	0.2394	78.5378168771746	78.5378168771746\\
72.125	0.24306	81.6375773026235	81.6375773026235\\
72.125	0.24672	84.8037276785347	84.8037276785347\\
72.125	0.25038	88.0362680049083	88.0362680049083\\
72.125	0.25404	91.3351982817443	91.3351982817443\\
72.125	0.2577	94.7005185090427	94.7005185090427\\
72.125	0.26136	98.1322286868034	98.1322286868034\\
72.125	0.26502	101.630328815027	101.630328815027\\
72.125	0.26868	105.194818893712	105.194818893712\\
72.125	0.27234	108.82569892286	108.82569892286\\
72.125	0.276	112.52296890247	112.52296890247\\
72.5	0.093	9.14486423053983	9.14486423053983\\
72.5	0.09666	9.5894437768231	9.5894437768231\\
72.5	0.10032	10.1004132735687	10.1004132735687\\
72.5	0.10398	10.6777727207768	10.6777727207768\\
72.5	0.10764	11.3215221184471	11.3215221184471\\
72.5	0.1113	12.0316614665799	12.0316614665799\\
72.5	0.11496	12.8081907651751	12.8081907651751\\
72.5	0.11862	13.6511100142326	13.6511100142326\\
72.5	0.12228	14.5604192137526	14.5604192137526\\
72.5	0.12594	15.5361183637349	15.5361183637349\\
72.5	0.1296	16.5782074641796	16.5782074641796\\
72.5	0.13326	17.6866865150866	17.6866865150866\\
72.5	0.13692	18.8615555164561	18.8615555164561\\
72.5	0.14058	20.1028144682879	20.1028144682879\\
72.5	0.14424	21.4104633705821	21.4104633705821\\
72.5	0.1479	22.7845022233387	22.7845022233387\\
72.5	0.15156	24.2249310265577	24.2249310265577\\
72.5	0.15522	25.7317497802391	25.7317497802391\\
72.5	0.15888	27.3049584843828	27.3049584843828\\
72.5	0.16254	28.9445571389889	28.9445571389889\\
72.5	0.1662	30.6505457440574	30.6505457440574\\
72.5	0.16986	32.4229242995883	32.4229242995883\\
72.5	0.17352	34.2616928055816	34.2616928055816\\
72.5	0.17718	36.1668512620372	36.1668512620372\\
72.5	0.18084	38.1383996689552	38.1383996689552\\
72.5	0.1845	40.1763380263357	40.1763380263357\\
72.5	0.18816	42.2806663341784	42.2806663341784\\
72.5	0.19182	44.4513845924836	44.4513845924836\\
72.5	0.19548	46.6884928012512	46.6884928012512\\
72.5	0.19914	48.9919909604811	48.9919909604811\\
72.5	0.2028	51.3618790701734	51.3618790701734\\
72.5	0.20646	53.7981571303281	53.7981571303281\\
72.5	0.21012	56.3008251409452	56.3008251409452\\
72.5	0.21378	58.8698831020246	58.8698831020246\\
72.5	0.21744	61.5053310135665	61.5053310135665\\
72.5	0.2211	64.2071688755707	64.2071688755707\\
72.5	0.22476	66.9753966880373	66.9753966880373\\
72.5	0.22842	69.8100144509663	69.8100144509663\\
72.5	0.23208	72.7110221643576	72.7110221643576\\
72.5	0.23574	75.6784198282114	75.6784198282114\\
72.5	0.2394	78.7122074425275	78.7122074425275\\
72.5	0.24306	81.812385007306	81.812385007306\\
72.5	0.24672	84.9789525225469	84.9789525225469\\
72.5	0.25038	88.2119099882502	88.2119099882502\\
72.5	0.25404	91.5112574044159	91.5112574044159\\
72.5	0.2577	94.8769947710439	94.8769947710439\\
72.5	0.26136	98.3091220881343	98.3091220881343\\
72.5	0.26502	101.807639355687	101.807639355687\\
72.5	0.26868	105.372546573702	105.372546573702\\
72.5	0.27234	109.00384374218	109.00384374218\\
72.5	0.276	112.70153086112	112.70153086112\\
72.875	0.093	9.30631159790014	9.30631159790014\\
72.875	0.09666	9.75130828351306	9.75130828351306\\
72.875	0.10032	10.2626949195884	10.2626949195884\\
72.875	0.10398	10.8404715061261	10.8404715061261\\
72.875	0.10764	11.4846380431261	11.4846380431261\\
72.875	0.1113	12.1951945305886	12.1951945305886\\
72.875	0.11496	12.9721409685134	12.9721409685134\\
72.875	0.11862	13.8154773569006	13.8154773569006\\
72.875	0.12228	14.7252036957502	14.7252036957502\\
72.875	0.12594	15.7013199850622	15.7013199850622\\
72.875	0.1296	16.7438262248365	16.7438262248365\\
72.875	0.13326	17.8527224150733	17.8527224150733\\
72.875	0.13692	19.0280085557724	19.0280085557724\\
72.875	0.14058	20.2696846469339	20.2696846469339\\
72.875	0.14424	21.5777506885578	21.5777506885578\\
72.875	0.1479	22.952206680644	22.952206680644\\
72.875	0.15156	24.3930526231927	24.3930526231927\\
72.875	0.15522	25.9002885162037	25.9002885162037\\
72.875	0.15888	27.4739143596771	27.4739143596771\\
72.875	0.16254	29.1139301536129	29.1139301536129\\
72.875	0.1662	30.8203358980111	30.8203358980111\\
72.875	0.16986	32.5931315928716	32.5931315928716\\
72.875	0.17352	34.4323172381946	34.4323172381946\\
72.875	0.17718	36.3378928339799	36.3378928339799\\
72.875	0.18084	38.3098583802276	38.3098583802276\\
72.875	0.1845	40.3482138769376	40.3482138769376\\
72.875	0.18816	42.4529593241101	42.4529593241101\\
72.875	0.19182	44.6240947217449	44.6240947217449\\
72.875	0.19548	46.8616200698421	46.8616200698421\\
72.875	0.19914	49.1655353684017	49.1655353684017\\
72.875	0.2028	51.5358406174237	51.5358406174237\\
72.875	0.20646	53.9725358169081	53.9725358169081\\
72.875	0.21012	56.4756209668548	56.4756209668548\\
72.875	0.21378	59.045096067264	59.045096067264\\
72.875	0.21744	61.6809611181355	61.6809611181355\\
72.875	0.2211	64.3832161194693	64.3832161194693\\
72.875	0.22476	67.1518610712656	67.1518610712656\\
72.875	0.22842	69.9868959735243	69.9868959735243\\
72.875	0.23208	72.8883208262453	72.8883208262453\\
72.875	0.23574	75.8561356294287	75.8561356294287\\
72.875	0.2394	78.8903403830745	78.8903403830745\\
72.875	0.24306	81.9909350871827	81.9909350871827\\
72.875	0.24672	85.1579197417533	85.1579197417533\\
72.875	0.25038	88.3912943467862	88.3912943467862\\
72.875	0.25404	91.6910589022815	91.6910589022815\\
72.875	0.2577	95.0572134082392	95.0572134082392\\
72.875	0.26136	98.4897578646593	98.4897578646593\\
72.875	0.26502	101.988692271542	101.988692271542\\
72.875	0.26868	105.554016628887	105.554016628887\\
72.875	0.27234	109.185730936694	109.185730936694\\
72.875	0.276	112.883835194963	112.883835194963\\
73.25	0.093	9.47150134045456	9.47150134045456\\
73.25	0.09666	9.91691516539715	9.91691516539715\\
73.25	0.10032	10.4287189408021	10.4287189408021\\
73.25	0.10398	11.0069126666695	11.0069126666695\\
73.25	0.10764	11.6514963429992	11.6514963429992\\
73.25	0.1113	12.3624699697913	12.3624699697913\\
73.25	0.11496	13.1398335470458	13.1398335470458\\
73.25	0.11862	13.9835870747627	13.9835870747627\\
73.25	0.12228	14.893730552942	14.893730552942\\
73.25	0.12594	15.8702639815836	15.8702639815836\\
73.25	0.1296	16.9131873606876	16.9131873606876\\
73.25	0.13326	18.022500690254	18.022500690254\\
73.25	0.13692	19.1982039702828	19.1982039702828\\
73.25	0.14058	20.440297200774	20.440297200774\\
73.25	0.14424	21.7487803817275	21.7487803817275\\
73.25	0.1479	23.1236535131435	23.1236535131435\\
73.25	0.15156	24.5649165950218	24.5649165950218\\
73.25	0.15522	26.0725696273625	26.0725696273625\\
73.25	0.15888	27.6466126101655	27.6466126101655\\
73.25	0.16254	29.287045543431	29.287045543431\\
73.25	0.1662	30.9938684271588	30.9938684271588\\
73.25	0.16986	32.767081261349	32.767081261349\\
73.25	0.17352	34.6066840460016	34.6066840460016\\
73.25	0.17718	36.5126767811166	36.5126767811166\\
73.25	0.18084	38.485059466694	38.485059466694\\
73.25	0.1845	40.5238321027337	40.5238321027337\\
73.25	0.18816	42.6289946892359	42.6289946892359\\
73.25	0.19182	44.8005472262004	44.8005472262004\\
73.25	0.19548	47.0384897136272	47.0384897136272\\
73.25	0.19914	49.3428221515165	49.3428221515165\\
73.25	0.2028	51.7135445398682	51.7135445398682\\
73.25	0.20646	54.1506568786822	54.1506568786822\\
73.25	0.21012	56.6541591679586	56.6541591679586\\
73.25	0.21378	59.2240514076974	59.2240514076974\\
73.25	0.21744	61.8603335978986	61.8603335978986\\
73.25	0.2211	64.5630057385621	64.5630057385621\\
73.25	0.22476	67.332067829688	67.332067829688\\
73.25	0.22842	70.1675198712764	70.1675198712764\\
73.25	0.23208	73.069361863327	73.069361863327\\
73.25	0.23574	76.0375938058402	76.0375938058402\\
73.25	0.2394	79.0722156988156	79.0722156988156\\
73.25	0.24306	82.1732275422535	82.1732275422535\\
73.25	0.24672	85.3406293361537	85.3406293361537\\
73.25	0.25038	88.5744210805163	88.5744210805163\\
73.25	0.25404	91.8746027753413	91.8746027753413\\
73.25	0.2577	95.2411744206287	95.2411744206287\\
73.25	0.26136	98.6741360163784	98.6741360163784\\
73.25	0.26502	102.173487562591	102.173487562591\\
73.25	0.26868	105.739229059265	105.739229059265\\
73.25	0.27234	109.371360506402	109.371360506402\\
73.25	0.276	113.069881904001	113.069881904001\\
73.625	0.093	9.64043345820307	9.64043345820307\\
73.625	0.09666	10.0862644224753	10.0862644224753\\
73.625	0.10032	10.59848533721	10.59848533721\\
73.625	0.10398	11.177096202407	11.177096202407\\
73.625	0.10764	11.8220970180664	11.8220970180664\\
73.625	0.1113	12.5334877841882	12.5334877841882\\
73.625	0.11496	13.3112685007723	13.3112685007723\\
73.625	0.11862	14.1554391678189	14.1554391678189\\
73.625	0.12228	15.0659997853278	15.0659997853278\\
73.625	0.12594	16.0429503532991	16.0429503532991\\
73.625	0.1296	17.0862908717328	17.0862908717328\\
73.625	0.13326	18.1960213406289	18.1960213406289\\
73.625	0.13692	19.3721417599873	19.3721417599873\\
73.625	0.14058	20.6146521298082	20.6146521298082\\
73.625	0.14424	21.9235524500914	21.9235524500914\\
73.625	0.1479	23.298842720837	23.298842720837\\
73.625	0.15156	24.7405229420449	24.7405229420449\\
73.625	0.15522	26.2485931137153	26.2485931137153\\
73.625	0.15888	27.8230532358481	27.8230532358481\\
73.625	0.16254	29.4639033084432	29.4639033084432\\
73.625	0.1662	31.1711433315007	31.1711433315007\\
73.625	0.16986	32.9447733050206	32.9447733050206\\
73.625	0.17352	34.7847932290028	34.7847932290028\\
73.625	0.17718	36.6912031034475	36.6912031034475\\
73.625	0.18084	38.6640029283545	38.6640029283545\\
73.625	0.1845	40.7031927037239	40.7031927037239\\
73.625	0.18816	42.8087724295557	42.8087724295557\\
73.625	0.19182	44.9807421058499	44.9807421058499\\
73.625	0.19548	47.2191017326064	47.2191017326064\\
73.625	0.19914	49.5238513098254	49.5238513098254\\
73.625	0.2028	51.8949908375067	51.8949908375067\\
73.625	0.20646	54.3325203156504	54.3325203156504\\
73.625	0.21012	56.8364397442565	56.8364397442565\\
73.625	0.21378	59.4067491233249	59.4067491233249\\
73.625	0.21744	62.0434484528557	62.0434484528557\\
73.625	0.2211	64.746537732849	64.746537732849\\
73.625	0.22476	67.5160169633046	67.5160169633046\\
73.625	0.22842	70.3518861442226	70.3518861442226\\
73.625	0.23208	73.2541452756029	73.2541452756029\\
73.625	0.23574	76.2227943574457	76.2227943574457\\
73.625	0.2394	79.2578333897508	79.2578333897508\\
73.625	0.24306	82.3592623725183	82.3592623725183\\
73.625	0.24672	85.5270813057482	85.5270813057482\\
73.625	0.25038	88.7612901894405	88.7612901894405\\
73.625	0.25404	92.0618890235951	92.0618890235951\\
73.625	0.2577	95.4288778082122	95.4288778082122\\
73.625	0.26136	98.8622565432916	98.8622565432916\\
73.625	0.26502	102.362025228833	102.362025228833\\
73.625	0.26868	105.928183864838	105.928183864838\\
73.625	0.27234	109.560732451304	109.560732451304\\
73.625	0.276	113.259670988233	113.259670988233\\
74	0.093	9.81310795114566	9.81310795114566\\
74	0.09666	10.2593560547476	10.2593560547476\\
74	0.10032	10.7719941088119	10.7719941088119\\
74	0.10398	11.3510221133386	11.3510221133386\\
74	0.10764	11.9964400683276	11.9964400683276\\
74	0.1113	12.7082479737791	12.7082479737791\\
74	0.11496	13.4864458296929	13.4864458296929\\
74	0.11862	14.3310336360691	14.3310336360691\\
74	0.12228	15.2420113929077	15.2420113929077\\
74	0.12594	16.2193791002087	16.2193791002087\\
74	0.1296	17.2631367579721	17.2631367579721\\
74	0.13326	18.3732843661978	18.3732843661978\\
74	0.13692	19.5498219248859	19.5498219248859\\
74	0.14058	20.7927494340364	20.7927494340364\\
74	0.14424	22.1020668936493	22.1020668936493\\
74	0.1479	23.4777743037246	23.4777743037246\\
74	0.15156	24.9198716642622	24.9198716642622\\
74	0.15522	26.4283589752623	26.4283589752623\\
74	0.15888	28.0032362367247	28.0032362367247\\
74	0.16254	29.6445034486495	29.6445034486495\\
74	0.1662	31.3521606110366	31.3521606110366\\
74	0.16986	33.1262077238862	33.1262077238862\\
74	0.17352	34.9666447871981	34.9666447871981\\
74	0.17718	36.8734718009724	36.8734718009724\\
74	0.18084	38.8466887652091	38.8466887652091\\
74	0.1845	40.8862956799082	40.8862956799082\\
74	0.18816	42.9922925450696	42.9922925450696\\
74	0.19182	45.1646793606935	45.1646793606935\\
74	0.19548	47.4034561267797	47.4034561267797\\
74	0.19914	49.7086228433283	49.7086228433283\\
74	0.2028	52.0801795103393	52.0801795103393\\
74	0.20646	54.5181261278127	54.5181261278127\\
74	0.21012	57.0224626957484	57.0224626957484\\
74	0.21378	59.5931892141465	59.5931892141465\\
74	0.21744	62.230305683007	62.230305683007\\
74	0.2211	64.9338121023299	64.9338121023299\\
74	0.22476	67.7037084721152	67.7037084721152\\
74	0.22842	70.5399947923628	70.5399947923628\\
74	0.23208	73.4426710630729	73.4426710630729\\
74	0.23574	76.4117372842453	76.4117372842453\\
74	0.2394	79.4471934558801	79.4471934558801\\
74	0.24306	82.5490395779773	82.5490395779773\\
74	0.24672	85.7172756505368	85.7172756505368\\
74	0.25038	88.9519016735588	88.9519016735588\\
74	0.25404	92.2529176470431	92.2529176470431\\
74	0.2577	95.6203235709898	95.6203235709898\\
74	0.26136	99.0541194453989	99.0541194453989\\
74	0.26502	102.55430527027	102.55430527027\\
74	0.26868	106.120881045604	106.120881045604\\
74	0.27234	109.7538467714	109.7538467714\\
74	0.276	113.453202447659	113.453202447659\\
};
\end{axis}
\end{tikzpicture}%
%	\caption{The quantitative relationship between external parameters ($L_{cut}, D_{rlx}$) and internal parameters $\theta$ is obtained by fitting a polynomial surface to the internal parameter values estimated for the sets \dataset1-\dataset5 (\ref{tikz:thetas1}) and sets \dataset6-\dataset10 (\ref{tikz:thetas2}). The fitted surfaces (\ref{tikz:theta_surf}) are then used to compute the internal model parameters corresponding to the settings of choice (\ref{tikz:thetaidentified}).}\label{fig:surfaces_all}
%\end{figure}
%\begin{figure}[!t]
%	\centering
%	\subfloat[\dataset3]{% This file was created by matlab2tikz.
%
\definecolor{mycolor1}{rgb}{0.00000,0.44700,0.74100}%
\definecolor{mycolor2}{rgb}{0.85000,0.32500,0.09800}%
%
\begin{tikzpicture}

\begin{axis}[%
width=11.411cm,
height=4cm,
at={(0cm,0cm)},
scale only axis,
xmin=1000,
xmax=1500,
ymin=-45,
ymax=0,
axis background/.style={fill=white},
legend style={legend cell align=left, align=left, draw=white!15!black}
]
\addplot [color=mycolor1]
  table[row sep=crcr]{%
1001	-17.09\\
1002	-19.531\\
1003	-14.648\\
1004	-14.648\\
1005	-19.531\\
1006	-18.311\\
1007	-23.193\\
1008	-18.311\\
1009	-9.766\\
1010	-15.869\\
1011	-12.207\\
1012	-14.648\\
1013	-10.986\\
1014	-3.662\\
1015	-2.441\\
1016	-4.883\\
1017	-6.104\\
1018	-14.648\\
1019	-17.09\\
1020	-17.09\\
1021	-13.428\\
1022	-9.766\\
1023	-18.311\\
1024	-14.648\\
1025	-10.986\\
1026	-13.428\\
1027	-12.207\\
1028	-10.986\\
1029	-20.752\\
1030	-19.531\\
1031	-12.207\\
1032	-20.752\\
1033	-20.752\\
1034	-15.869\\
1035	-13.428\\
1036	-9.766\\
1037	-14.648\\
1038	-14.648\\
1039	-14.648\\
1040	-14.648\\
1041	-14.648\\
1042	-15.869\\
1043	-21.973\\
1044	-20.752\\
1045	-13.428\\
1046	-13.428\\
1047	-8.545\\
1048	-12.207\\
1049	-12.207\\
1050	-13.428\\
1051	-10.986\\
1052	-13.428\\
1053	-14.648\\
1054	-13.428\\
1055	-20.752\\
1056	-13.428\\
1057	-9.766\\
1058	-7.324\\
1059	-8.545\\
1060	-10.986\\
1061	-6.104\\
1062	-9.766\\
1063	-10.986\\
1064	-6.104\\
1065	-6.104\\
1066	-9.766\\
1067	-9.766\\
1068	-15.869\\
1069	-14.648\\
1070	-17.09\\
1071	-15.869\\
1072	-18.311\\
1073	-13.428\\
1074	-14.648\\
1075	-12.207\\
1076	-10.986\\
1077	-12.207\\
1078	-23.193\\
1079	-28.076\\
1080	-30.518\\
1081	-29.297\\
1082	-18.311\\
1083	-28.076\\
1084	-32.959\\
1085	-31.738\\
1086	-20.752\\
1087	-34.18\\
1088	-40.283\\
1089	-29.297\\
1090	-24.414\\
1091	-19.531\\
1092	-15.869\\
1093	-12.207\\
1094	-14.648\\
1095	-9.766\\
1096	-7.324\\
1097	-7.324\\
1098	-10.986\\
1099	-13.428\\
1100	-12.207\\
1101	-12.207\\
1102	-15.869\\
1103	-18.311\\
1104	-19.531\\
1105	-15.869\\
1106	-21.973\\
1107	-15.869\\
1108	-15.869\\
1109	-17.09\\
1110	-12.207\\
1111	-12.207\\
1112	-15.869\\
1113	-14.648\\
1114	-8.545\\
1115	-12.207\\
1116	-8.545\\
1117	-9.766\\
1118	-9.766\\
1119	-7.324\\
1120	-9.766\\
1121	-12.207\\
1122	-17.09\\
1123	-17.09\\
1124	-17.09\\
1125	-10.986\\
1126	-12.207\\
1127	-23.193\\
1128	-15.869\\
1129	-20.752\\
1130	-21.973\\
1131	-12.207\\
1132	-9.766\\
1133	-12.207\\
1134	-13.428\\
1135	-21.973\\
1136	-25.635\\
1137	-26.855\\
1138	-19.531\\
1139	-20.752\\
1140	-18.311\\
1141	-17.09\\
1142	-18.311\\
1143	-13.428\\
1144	-12.207\\
1145	-9.766\\
1146	-9.766\\
1147	-10.986\\
1148	-15.869\\
1149	-23.193\\
1150	-17.09\\
1151	-13.428\\
1152	-10.986\\
1153	-10.986\\
1154	-7.324\\
1155	-6.104\\
1156	-6.104\\
1157	-12.207\\
1158	-7.324\\
1159	-8.545\\
1160	-8.545\\
1161	-8.545\\
1162	-7.324\\
1163	-4.883\\
1164	-3.662\\
1165	-8.545\\
1166	-15.869\\
1167	-18.311\\
1168	-20.752\\
1169	-15.869\\
1170	-10.986\\
1171	-8.545\\
1172	-4.883\\
1173	-9.766\\
1174	-8.545\\
1175	-15.869\\
1176	-17.09\\
1177	-29.297\\
1178	-32.959\\
1179	-25.635\\
1180	-25.635\\
1181	-18.311\\
1182	-23.193\\
1183	-21.973\\
1184	-24.414\\
1185	-15.869\\
1186	-17.09\\
1187	-17.09\\
1188	-15.869\\
1189	-15.869\\
1190	-13.428\\
1191	-23.193\\
1192	-25.635\\
1193	-19.531\\
1194	-13.428\\
1195	-14.648\\
1196	-23.193\\
1197	-24.414\\
1198	-28.076\\
1199	-30.518\\
1200	-29.297\\
1201	-23.193\\
1202	-21.973\\
1203	-25.635\\
1204	-28.076\\
1205	-18.311\\
1206	-12.207\\
1207	-19.531\\
1208	-14.648\\
1209	-9.766\\
1210	-13.428\\
1211	-14.648\\
1212	-10.986\\
1213	-10.986\\
1214	-10.986\\
1215	-8.545\\
1216	-10.986\\
1217	-18.311\\
1218	-17.09\\
1219	-12.207\\
1220	-12.207\\
1221	-18.311\\
1222	-26.855\\
1223	-17.09\\
1224	-14.648\\
1225	-10.986\\
1226	-13.428\\
1227	-10.986\\
1228	-8.545\\
1229	-8.545\\
1230	-10.986\\
1231	-13.428\\
1232	-14.648\\
1233	-12.207\\
1234	-15.869\\
1235	-20.752\\
1236	-17.09\\
1237	-12.207\\
1238	-15.869\\
1239	-20.752\\
1240	-13.428\\
1241	-7.324\\
1242	-13.428\\
1243	-10.986\\
1244	-12.207\\
1245	-9.766\\
1246	-8.545\\
1247	-13.428\\
1248	-13.428\\
1249	-9.766\\
1250	-12.207\\
1251	-9.766\\
1252	-7.324\\
1253	-12.207\\
1254	-8.545\\
1255	-9.766\\
1256	-7.324\\
1257	-4.883\\
1258	-12.207\\
1259	-15.869\\
1260	-17.09\\
1261	-23.193\\
1262	-15.869\\
1263	-9.766\\
1264	-8.545\\
1265	-8.545\\
1266	-10.986\\
1267	-8.545\\
1268	-4.883\\
1269	-10.986\\
1270	-15.869\\
1271	-13.428\\
1272	-20.752\\
1273	-15.869\\
1274	-14.648\\
1275	-8.545\\
1276	-10.986\\
1277	-4.883\\
1278	-3.662\\
1279	-4.883\\
1280	-9.766\\
1281	-10.986\\
1282	-12.207\\
1283	-13.428\\
1284	-20.752\\
1285	-17.09\\
1286	-14.648\\
1287	-17.09\\
1288	-19.531\\
1289	-20.752\\
1290	-17.09\\
1291	-13.428\\
1292	-7.324\\
1293	-6.104\\
1294	-4.883\\
1295	-7.324\\
1296	-7.324\\
1297	-4.883\\
1298	-8.545\\
1299	-12.207\\
1300	-10.986\\
1301	-12.207\\
1302	-12.207\\
1303	-8.545\\
1304	-4.883\\
1305	-9.766\\
1306	-15.869\\
1307	-13.428\\
1308	-15.869\\
1309	-13.428\\
1310	-9.766\\
1311	-14.648\\
1312	-18.311\\
1313	-18.311\\
1314	-13.428\\
1315	-20.752\\
1316	-15.869\\
1317	-17.09\\
1318	-18.311\\
1319	-19.531\\
1320	-13.428\\
1321	-13.428\\
1322	-18.311\\
1323	-26.855\\
1324	-21.973\\
1325	-13.428\\
1326	-13.428\\
1327	-13.428\\
1328	-15.869\\
1329	-17.09\\
1330	-12.207\\
1331	-14.648\\
1332	-18.311\\
1333	-20.752\\
1334	-13.428\\
1335	-12.207\\
1336	-15.869\\
1337	-23.193\\
1338	-20.752\\
1339	-21.973\\
1340	-14.648\\
1341	-14.648\\
1342	-12.207\\
1343	-9.766\\
1344	-4.883\\
1345	-3.662\\
1346	-3.662\\
1347	-7.324\\
1348	-13.428\\
1349	-14.648\\
1350	-9.766\\
1351	-12.207\\
1352	-13.428\\
1353	-9.766\\
1354	-8.545\\
1355	-8.545\\
1356	-6.104\\
1357	-9.766\\
1358	-14.648\\
1359	-15.869\\
1360	-10.986\\
1361	-8.545\\
1362	-8.545\\
1363	-8.545\\
1364	-7.324\\
1365	-10.986\\
1366	-20.752\\
1367	-18.311\\
1368	-18.311\\
1369	-24.414\\
1370	-23.193\\
1371	-18.311\\
1372	-14.648\\
1373	-14.648\\
1374	-14.648\\
1375	-17.09\\
1376	-19.531\\
1377	-19.531\\
1378	-25.635\\
1379	-26.855\\
1380	-31.738\\
1381	-24.414\\
1382	-25.635\\
1383	-26.855\\
1384	-21.973\\
1385	-24.414\\
1386	-20.752\\
1387	-13.428\\
1388	-9.766\\
1389	-9.766\\
1390	-12.207\\
1391	-9.766\\
1392	-7.324\\
1393	-10.986\\
1394	-8.545\\
1395	-6.104\\
1396	-7.324\\
1397	-9.766\\
1398	-6.104\\
1399	-12.207\\
1400	-14.648\\
1401	-10.986\\
1402	-17.09\\
1403	-24.414\\
1404	-24.414\\
1405	-28.076\\
1406	-23.193\\
1407	-21.973\\
1408	-17.09\\
1409	-9.766\\
1410	-10.986\\
1411	-10.986\\
1412	-7.324\\
1413	-10.986\\
1414	-9.766\\
1415	-2.441\\
1416	-6.104\\
1417	-9.766\\
1418	-12.207\\
1419	-14.648\\
1420	-17.09\\
1421	-14.648\\
1422	-17.09\\
1423	-14.648\\
1424	-13.428\\
1425	-12.207\\
1426	-10.986\\
1427	-14.648\\
1428	-13.428\\
1429	-10.986\\
1430	-13.428\\
1431	-18.311\\
1432	-13.428\\
1433	-10.986\\
1434	-10.986\\
1435	-10.986\\
1436	-8.545\\
1437	-7.324\\
1438	-10.986\\
1439	-13.428\\
1440	-13.428\\
1441	-10.986\\
1442	-13.428\\
1443	-13.428\\
1444	-8.545\\
1445	-7.324\\
1446	-15.869\\
1447	-19.531\\
1448	-18.311\\
1449	-18.311\\
1450	-15.869\\
1451	-13.428\\
1452	-10.986\\
1453	-9.766\\
1454	-13.428\\
1455	-13.428\\
1456	-8.545\\
1457	-12.207\\
1458	-14.648\\
1459	-14.648\\
1460	-17.09\\
1461	-17.09\\
1462	-20.752\\
1463	-28.076\\
1464	-30.518\\
1465	-20.752\\
1466	-13.428\\
1467	-8.545\\
1468	-6.104\\
1469	-8.545\\
1470	-7.324\\
1471	-10.986\\
1472	-12.207\\
1473	-12.207\\
1474	-13.428\\
1475	-15.869\\
1476	-19.531\\
1477	-19.531\\
1478	-28.076\\
1479	-23.193\\
1480	-18.311\\
1481	-14.648\\
1482	-14.648\\
1483	-14.648\\
1484	-17.09\\
1485	-14.648\\
1486	-12.207\\
1487	-13.428\\
1488	-8.545\\
1489	-7.324\\
1490	-17.09\\
1491	-14.648\\
1492	-12.207\\
1493	-23.193\\
1494	-18.311\\
1495	-15.869\\
1496	-21.973\\
1497	-23.193\\
1498	-14.648\\
1499	-8.545\\
1500	-9.766\\
};
\addlegendentry{True output}

\addplot [color=mycolor2, dashed]
  table[row sep=crcr]{%
1001	-16.254527795835\\
1002	-19.6785535483227\\
1003	-18.4695238735799\\
1004	-16.8085595785455\\
1005	-20.6967288010317\\
1006	-20.9229581833036\\
1007	-22.1588523355377\\
1008	-18.9522242268178\\
1009	-11.852930282059\\
1010	-12.7097929583578\\
1011	-13.8176911214572\\
1012	-15.6680960627051\\
1013	-14.3387146598432\\
1014	-7.44412900453769\\
1015	-4.8418820233322\\
1016	-4.54558709548009\\
1017	-10.2805005439112\\
1018	-18.6656315953563\\
1019	-18.1550065451774\\
1020	-17.4193008050967\\
1021	-13.4253527706572\\
1022	-9.17876081794417\\
1023	-16.2819251515142\\
1024	-13.7572983830717\\
1025	-10.5808002198238\\
1026	-14.2600863261157\\
1027	-15.1542363761452\\
1028	-12.4175435697077\\
1029	-17.5324658075764\\
1030	-17.7337281700071\\
1031	-13.9936533067158\\
1032	-18.0981665740989\\
1033	-18.7879458582808\\
1034	-15.4981038856351\\
1035	-13.5604273513669\\
1036	-12.1798403398373\\
1037	-13.1052897796164\\
1038	-15.729572443885\\
1039	-15.6325009114338\\
1040	-14.7766484584953\\
1041	-15.2801081420831\\
1042	-16.1268641691837\\
1043	-20.7296060663079\\
1044	-20.1844053248746\\
1045	-14.5952861338106\\
1046	-13.2928458118398\\
1047	-9.39819069122964\\
1048	-8.85449698520292\\
1049	-11.7984229473124\\
1050	-11.0473327882388\\
1051	-10.7574669425996\\
1052	-13.1991465980444\\
1053	-14.7951825921551\\
1054	-13.9364492543648\\
1055	-19.082990888115\\
1056	-16.7251096070586\\
1057	-10.227987990023\\
1058	-8.6944446207486\\
1059	-11.1021082020429\\
1060	-12.987407754281\\
1061	-9.34960794134911\\
1062	-7.73616153788004\\
1063	-10.3744371753598\\
1064	-10.0456299877943\\
1065	-8.48840928991694\\
1066	-9.6316171234278\\
1067	-10.9790178347557\\
1068	-15.0190079150973\\
1069	-15.2807843502467\\
1070	-16.1021289661051\\
1071	-15.1503437204151\\
1072	-16.8886526779588\\
1073	-15.2239566275856\\
1074	-13.9144134600621\\
1075	-13.4039923990204\\
1076	-13.3909125715513\\
1077	-13.2245011419285\\
1078	-19.4481189159192\\
1079	-28.091645619237\\
1080	-28.0501288434693\\
1081	-26.3405135189048\\
1082	-17.8013324181827\\
1083	-23.906540522589\\
1084	-29.4196670314917\\
1085	-29.4309146016869\\
1086	-24.5335176563917\\
1087	-29.1049706221812\\
1088	-36.7577763759706\\
1089	-31.0429113667165\\
1090	-25.9574621226679\\
1091	-19.3997407738875\\
1092	-16.3208382956374\\
1093	-14.9976353638659\\
1094	-16.8288067247173\\
1095	-15.2467666705987\\
1096	-10.2245672999938\\
1097	-8.8740427225416\\
1098	-10.9635938146415\\
1099	-15.1540628805987\\
1100	-14.9940159324954\\
1101	-14.3166688145998\\
1102	-16.4750779984649\\
1103	-17.2239579442626\\
1104	-22.8154402334777\\
1105	-20.4912215099301\\
1106	-21.1855093503453\\
1107	-18.2179026934775\\
1108	-15.5838902605412\\
1109	-18.355664073965\\
1110	-15.501712454729\\
1111	-13.3586533476621\\
1112	-15.8308478142802\\
1113	-16.2283199398048\\
1114	-13.6861082792868\\
1115	-12.9045301721076\\
1116	-11.3626974195098\\
1117	-10.9783638949845\\
1118	-12.7443803010989\\
1119	-9.68145238691905\\
1120	-10.3058880847157\\
1121	-12.9002524636774\\
1122	-17.4204453936312\\
1123	-17.6193672838978\\
1124	-17.4694960644632\\
1125	-12.9323145396955\\
1126	-15.7641869846631\\
1127	-23.3624231155263\\
1128	-18.9656345302036\\
1129	-20.012980140676\\
1130	-20.0481065704999\\
1131	-14.4373939232766\\
1132	-10.8743990493386\\
1133	-13.082178709204\\
1134	-15.4735497451192\\
1135	-22.1783709924051\\
1136	-26.5876104811415\\
1137	-24.8768382978155\\
1138	-20.5084016374023\\
1139	-19.9682365619472\\
1140	-19.9427406522857\\
1141	-19.1195046734225\\
1142	-19.0103087108665\\
1143	-14.9406316584238\\
1144	-12.899175481349\\
1145	-12.8556498626275\\
1146	-12.3783206506346\\
1147	-13.0025409567306\\
1148	-17.002186423881\\
1149	-23.167838836912\\
1150	-20.3952811327998\\
1151	-14.3353286428408\\
1152	-12.3102166181048\\
1153	-12.7581966598952\\
1154	-9.2196826444159\\
1155	-7.03589110955713\\
1156	-7.63726237002204\\
1157	-12.0325255948571\\
1158	-11.0240397076226\\
1159	-10.3259686383801\\
1160	-10.9526523320839\\
1161	-10.4953721786074\\
1162	-8.89803779510969\\
1163	-6.66330239103735\\
1164	-5.69840034993383\\
1165	-7.31957075641177\\
1166	-14.1187053213359\\
1167	-19.9866815805115\\
1168	-20.8178976007142\\
1169	-14.3109217006899\\
1170	-10.4343188026074\\
1171	-8.17000945054156\\
1172	-6.2465499929547\\
1173	-11.1074007793417\\
1174	-12.2888153858427\\
1175	-14.9268597597601\\
1176	-18.3691731175147\\
1177	-27.8513269533019\\
1178	-32.4578740558389\\
1179	-26.7152328147948\\
1180	-25.5304894902503\\
1181	-20.259450484194\\
1182	-20.1583961130355\\
1183	-22.3871671558667\\
1184	-23.5833452090276\\
1185	-19.9505573548944\\
1186	-18.4519290179529\\
1187	-18.8049251837313\\
1188	-17.5501866153547\\
1189	-19.9453873305425\\
1190	-16.7210750892428\\
1191	-21.6428881321012\\
1192	-26.220111931263\\
1193	-20.2356679425326\\
1194	-13.7333816837161\\
1195	-14.3875038416724\\
1196	-21.705050301303\\
1197	-25.7403160454778\\
1198	-28.9326860356227\\
1199	-30.0813993969761\\
1200	-29.7381542813311\\
1201	-24.4603578932273\\
1202	-22.9668464998021\\
1203	-25.5049417203235\\
1204	-28.1972277516781\\
1205	-21.0503513126442\\
1206	-14.0731618692229\\
1207	-18.1000201335959\\
1208	-17.4539263490992\\
1209	-12.5129380674449\\
1210	-14.6042131013128\\
1211	-16.9681001059637\\
1212	-14.5590786761075\\
1213	-11.7891039889261\\
1214	-13.6318372330255\\
1215	-13.3120183067239\\
1216	-15.3281518331053\\
1217	-20.0030473902354\\
1218	-18.1718972602771\\
1219	-14.1277157405455\\
1220	-13.5271316909107\\
1221	-18.8822301962699\\
1222	-28.2617253947708\\
1223	-23.592769336167\\
1224	-15.299850257314\\
1225	-12.6924792244913\\
1226	-13.6120441448873\\
1227	-14.5224185894347\\
1228	-11.7059717562675\\
1229	-12.158241059464\\
1230	-13.4091608908626\\
1231	-16.1410094942187\\
1232	-17.2231901419689\\
1233	-12.6126641344204\\
1234	-16.2248855191873\\
1235	-19.8500997959637\\
1236	-18.9720796162841\\
1237	-16.1615203138505\\
1238	-16.6115883716647\\
1239	-21.3277290294522\\
1240	-16.067235998004\\
1241	-8.27973024702612\\
1242	-9.36537116083202\\
1243	-11.7601257947719\\
1244	-15.3979061430704\\
1245	-14.3126363561289\\
1246	-11.4079194223811\\
1247	-14.5272282595769\\
1248	-14.9396312644322\\
1249	-11.9386321191322\\
1250	-13.2743821968853\\
1251	-12.8287581134105\\
1252	-12.1373731367485\\
1253	-12.8002555314779\\
1254	-9.75457553358341\\
1255	-9.78250893382767\\
1256	-8.6581591269492\\
1257	-8.9275421278963\\
1258	-13.6388183209548\\
1259	-15.5373757232236\\
1260	-18.5672623307148\\
1261	-23.5023433971908\\
1262	-17.3025232320516\\
1263	-10.9772017414028\\
1264	-8.85163409732329\\
1265	-9.10450981702048\\
1266	-12.3758765311307\\
1267	-9.62063661681329\\
1268	-9.36224065893933\\
1269	-11.4390492147414\\
1270	-16.7926063803233\\
1271	-16.8554309936768\\
1272	-19.2792427915847\\
1273	-15.1183768158966\\
1274	-13.1446291398401\\
1275	-8.76776715046061\\
1276	-7.95348582306552\\
1277	-6.24843965216333\\
1278	-6.30508598808701\\
1279	-7.65721823512418\\
1280	-12.5553901779878\\
1281	-12.7646688523679\\
1282	-13.6707945274289\\
1283	-13.9275803281834\\
1284	-20.2787501270059\\
1285	-20.0041757101421\\
1286	-14.4991900359391\\
1287	-16.2774395222776\\
1288	-17.2476013330773\\
1289	-21.5795066573139\\
1290	-18.7869449244937\\
1291	-13.5857972315294\\
1292	-8.3878088311381\\
1293	-6.13792391714111\\
1294	-6.72686302117015\\
1295	-8.00948052364981\\
1296	-8.37711807869147\\
1297	-8.15579287248692\\
1298	-9.27479097513148\\
1299	-12.8271829493702\\
1300	-12.8269327510388\\
1301	-11.9872358127937\\
1302	-13.2602639198292\\
1303	-9.79893019224206\\
1304	-6.1853120244679\\
1305	-9.675668477782\\
1306	-14.3218418123058\\
1307	-14.6440547537416\\
1308	-14.3704148412383\\
1309	-13.2574504010815\\
1310	-10.8235106869072\\
1311	-13.0789023398639\\
1312	-17.6017018440589\\
1313	-18.1385340801341\\
1314	-13.6087009677457\\
1315	-19.9853081693773\\
1316	-17.7789728664422\\
1317	-16.0428111563599\\
1318	-18.4954923331756\\
1319	-17.9580684429222\\
1320	-15.2503556948337\\
1321	-13.5020601212662\\
1322	-18.5732809355128\\
1323	-26.7350471694415\\
1324	-21.4408447226086\\
1325	-13.5667059798754\\
1326	-12.6952514290382\\
1327	-14.6815799019769\\
1328	-17.068341169465\\
1329	-19.2059227517558\\
1330	-15.5598299909801\\
1331	-16.0453525269692\\
1332	-19.4234522494844\\
1333	-20.7797578699979\\
1334	-15.937117413586\\
1335	-13.3223849922964\\
1336	-16.0125489432341\\
1337	-24.1148100099088\\
1338	-21.9831162094355\\
1339	-21.1318656350575\\
1340	-14.6802707914842\\
1341	-13.0009519660553\\
1342	-13.971223732964\\
1343	-9.67469731275744\\
1344	-7.41623229586748\\
1345	-5.93007691785131\\
1346	-5.45582507700952\\
1347	-9.28718366490923\\
1348	-13.5515890616187\\
1349	-15.2525475006181\\
1350	-11.9154953951825\\
1351	-11.7335282946571\\
1352	-12.9447763382902\\
1353	-10.0644345857967\\
1354	-9.53258524621465\\
1355	-10.015489529515\\
1356	-8.71148726825974\\
1357	-9.39301877148121\\
1358	-13.1628195012754\\
1359	-16.5860743970687\\
1360	-11.8954665532997\\
1361	-9.1421813809755\\
1362	-8.31749497293008\\
1363	-9.76715218928246\\
1364	-7.95353919690366\\
1365	-13.6081705172173\\
1366	-21.5595723163141\\
1367	-17.5464657775623\\
1368	-20.4379762514503\\
1369	-24.341148174161\\
1370	-23.4699707777061\\
1371	-19.4499760845079\\
1372	-15.1142519270206\\
1373	-15.2265647768365\\
1374	-16.1809074874407\\
1375	-18.0950330067858\\
1376	-19.8949602244743\\
1377	-20.2228871319075\\
1378	-23.3212541231441\\
1379	-26.5669211777671\\
1380	-30.4315367225362\\
1381	-23.775419929132\\
1382	-23.3015105469787\\
1383	-27.3418601373673\\
1384	-23.5783881054257\\
1385	-24.3340120500237\\
1386	-23.0673247624779\\
1387	-14.184429351209\\
1388	-10.3090176960213\\
1389	-11.1355103039195\\
1390	-15.7144261899986\\
1391	-14.25914050165\\
1392	-10.5566995707071\\
1393	-12.2620722202176\\
1394	-10.543572180114\\
1395	-8.42730554003665\\
1396	-9.50826378345298\\
1397	-10.5181865419145\\
1398	-8.90316586150725\\
1399	-12.2868492331555\\
1400	-16.3023310144022\\
1401	-13.3159135428293\\
1402	-17.305066541887\\
1403	-25.0988159381525\\
1404	-23.5152499234897\\
1405	-26.6860650707936\\
1406	-21.3247851197159\\
1407	-20.7891086896782\\
1408	-17.1262336553421\\
1409	-11.5255754137825\\
1410	-13.8255411949341\\
1411	-12.3155582832819\\
1412	-10.0010259361444\\
1413	-12.2036150425358\\
1414	-8.78843567813996\\
1415	-6.14879677121817\\
1416	-6.77118555399161\\
1417	-11.0260177346679\\
1418	-15.4909779483148\\
1419	-17.2159964813981\\
1420	-16.7825400598738\\
1421	-15.7355500312245\\
1422	-16.9711750893266\\
1423	-15.425405139149\\
1424	-13.2519091976085\\
1425	-13.9450276484476\\
1426	-12.4106816766141\\
1427	-16.3078971182963\\
1428	-13.1633888023794\\
1429	-10.7083193505805\\
1430	-14.0985565673601\\
1431	-19.2114843717261\\
1432	-16.3238202334337\\
1433	-12.3393575505876\\
1434	-11.1157565422425\\
1435	-12.4837610687157\\
1436	-10.0528221524678\\
1437	-9.03558130101\\
1438	-11.499184511203\\
1439	-13.3357172133222\\
1440	-14.8521272968782\\
1441	-12.6347769127954\\
1442	-13.9802424646758\\
1443	-14.0723875880144\\
1444	-9.91747179431477\\
1445	-9.37127282430209\\
1446	-14.9939434726051\\
1447	-21.0316215843534\\
1448	-19.9531149521046\\
1449	-16.9508001862369\\
1450	-16.4829143846543\\
1451	-14.2275409423853\\
1452	-13.631215740707\\
1453	-12.0478524217997\\
1454	-14.7132221435115\\
1455	-14.4416298909729\\
1456	-10.2631108537023\\
1457	-12.851309902905\\
1458	-14.5653629895465\\
1459	-16.9708127493046\\
1460	-17.3674837997607\\
1461	-17.0743807602671\\
1462	-21.7845132103158\\
1463	-26.3724203450924\\
1464	-29.3748731887452\\
1465	-21.4248514285985\\
1466	-13.4369519171937\\
1467	-9.79249352914372\\
1468	-7.23445340804922\\
1469	-9.58503481369203\\
1470	-10.6836604050061\\
1471	-14.8814284126221\\
1472	-14.5220824611656\\
1473	-12.1271810994553\\
1474	-14.5019473200892\\
1475	-16.3747886762306\\
1476	-18.0655033650769\\
1477	-19.1206578141706\\
1478	-26.869766440414\\
1479	-24.8689926895787\\
1480	-19.9413257879833\\
1481	-15.0995932212746\\
1482	-14.4371602819976\\
1483	-16.2573487568386\\
1484	-18.6633593431921\\
1485	-15.4379615859187\\
1486	-14.5316315181258\\
1487	-14.6254876485864\\
1488	-9.92892991027907\\
1489	-12.9615144123974\\
1490	-18.3167599441571\\
1491	-14.7783357958607\\
1492	-14.3303168789501\\
1493	-22.6430306650768\\
1494	-19.4062633963547\\
1495	-17.3111761839129\\
1496	-22.3021896506457\\
1497	-22.0856899954143\\
1498	-16.260267648596\\
1499	-10.4113296930648\\
1500	-11.3791239579802\\
};
\addlegendentry{Generated output}

\end{axis}
\end{tikzpicture}%\label{fig:c3all}}\\
%	\subfloat[\dataset8]{% This file was created by matlab2tikz.
%
\definecolor{mycolor1}{rgb}{0.00000,0.44700,0.74100}%
\definecolor{mycolor2}{rgb}{0.85000,0.32500,0.09800}%
%
\begin{tikzpicture}

\begin{axis}[%
width=11.411cm,
height=6cm,
at={(0cm,0cm)},
scale only axis,
xmin=1000,
xmax=1500,
ymin=-300,
ymax=0,
axis background/.style={fill=white},
legend style={legend cell align=left, align=left, draw=white!15!black}
]
\addplot [color=mycolor1]
  table[row sep=crcr]{%
1001	-96.436\\
1002	-122.07\\
1003	-100.098\\
1004	-93.994\\
1005	-130.615\\
1006	-125.732\\
1007	-153.809\\
1008	-115.967\\
1009	-61.035\\
1010	-74.463\\
1011	-73.242\\
1012	-86.67\\
1013	-73.242\\
1014	-34.18\\
1015	-20.752\\
1016	-23.193\\
1017	-58.594\\
1018	-100.098\\
1019	-109.863\\
1020	-114.746\\
1021	-83.008\\
1022	-52.49\\
1023	-104.98\\
1024	-81.787\\
1025	-59.814\\
1026	-97.656\\
1027	-83.008\\
1028	-72.021\\
1029	-118.408\\
1030	-107.422\\
1031	-83.008\\
1032	-119.629\\
1033	-123.291\\
1034	-95.215\\
1035	-80.566\\
1036	-62.256\\
1037	-75.684\\
1038	-95.215\\
1039	-87.891\\
1040	-91.553\\
1041	-96.436\\
1042	-102.539\\
1043	-141.602\\
1044	-128.174\\
1045	-85.449\\
1046	-79.346\\
1047	-47.607\\
1048	-48.828\\
1049	-68.359\\
1050	-52.49\\
1051	-57.373\\
1052	-75.684\\
1053	-89.111\\
1054	-81.787\\
1055	-128.174\\
1056	-98.877\\
1057	-57.373\\
1058	-45.166\\
1059	-62.256\\
1060	-72.021\\
1061	-40.283\\
1062	-42.725\\
1063	-56.152\\
1064	-46.387\\
1065	-40.283\\
1066	-52.49\\
1067	-62.256\\
1068	-92.773\\
1069	-90.332\\
1070	-107.422\\
1071	-98.877\\
1072	-115.967\\
1073	-91.553\\
1074	-91.553\\
1075	-79.346\\
1076	-75.684\\
1077	-76.904\\
1078	-133.057\\
1079	-189.209\\
1080	-194.092\\
1081	-189.209\\
1082	-119.629\\
1083	-170.898\\
1084	-213.623\\
1085	-219.727\\
1086	-169.678\\
1087	-219.727\\
1088	-279.541\\
1089	-197.754\\
1090	-161.133\\
1091	-109.863\\
1092	-85.449\\
1093	-73.242\\
1094	-91.553\\
1095	-62.256\\
1096	-46.387\\
1097	-42.725\\
1098	-54.932\\
1099	-79.346\\
1100	-74.463\\
1101	-75.684\\
1102	-100.098\\
1103	-103.76\\
1104	-141.602\\
1105	-114.746\\
1106	-142.822\\
1107	-104.98\\
1108	-90.332\\
1109	-103.76\\
1110	-76.904\\
1111	-69.58\\
1112	-84.229\\
1113	-86.67\\
1114	-59.814\\
1115	-64.697\\
1116	-48.828\\
1117	-53.711\\
1118	-57.373\\
1119	-41.504\\
1120	-52.49\\
1121	-65.918\\
1122	-101.318\\
1123	-104.98\\
1124	-114.746\\
1125	-68.359\\
1126	-93.994\\
1127	-150.146\\
1128	-114.746\\
1129	-125.732\\
1130	-128.174\\
1131	-80.566\\
1132	-53.711\\
1133	-75.684\\
1134	-85.449\\
1135	-137.939\\
1136	-172.119\\
1137	-170.898\\
1138	-123.291\\
1139	-130.615\\
1140	-115.967\\
1141	-113.525\\
1142	-111.084\\
1143	-76.904\\
1144	-68.359\\
1145	-61.035\\
1146	-57.373\\
1147	-62.256\\
1148	-91.553\\
1149	-140.381\\
1150	-111.084\\
1151	-79.346\\
1152	-64.697\\
1153	-62.256\\
1154	-40.283\\
1155	-29.297\\
1156	-36.621\\
1157	-63.477\\
1158	-51.27\\
1159	-52.49\\
1160	-58.594\\
1161	-54.932\\
1162	-37.842\\
1163	-28.076\\
1164	-23.193\\
1165	-39.063\\
1166	-83.008\\
1167	-117.188\\
1168	-130.615\\
1169	-86.67\\
1170	-58.594\\
1171	-45.166\\
1172	-32.959\\
1173	-67.139\\
1174	-65.918\\
1175	-91.553\\
1176	-117.188\\
1177	-186.768\\
1178	-216.064\\
1179	-177.002\\
1180	-168.457\\
1181	-109.863\\
1182	-122.07\\
1183	-131.836\\
1184	-146.484\\
1185	-108.643\\
1186	-100.098\\
1187	-98.877\\
1188	-89.111\\
1189	-106.201\\
1190	-78.125\\
1191	-130.615\\
1192	-159.912\\
1193	-113.525\\
1194	-70.801\\
1195	-73.242\\
1196	-123.291\\
1197	-153.809\\
1198	-189.209\\
1199	-202.637\\
1200	-202.637\\
1201	-153.809\\
1202	-137.939\\
1203	-158.691\\
1204	-187.988\\
1205	-109.863\\
1206	-72.021\\
1207	-101.318\\
1208	-86.67\\
1209	-53.711\\
1210	-74.463\\
1211	-84.229\\
1212	-67.139\\
1213	-51.27\\
1214	-70.801\\
1215	-58.594\\
1216	-79.346\\
1217	-107.422\\
1218	-100.098\\
1219	-69.58\\
1220	-70.801\\
1221	-107.422\\
1222	-177.002\\
1223	-128.174\\
1224	-79.346\\
1225	-61.035\\
1226	-70.801\\
1227	-64.697\\
1228	-48.828\\
1229	-53.711\\
1230	-65.918\\
1231	-83.008\\
1232	-91.553\\
1233	-63.477\\
1234	-104.98\\
1235	-124.512\\
1236	-118.408\\
1237	-91.553\\
1238	-100.098\\
1239	-135.498\\
1240	-87.891\\
1241	-42.725\\
1242	-57.373\\
1243	-62.256\\
1244	-81.787\\
1245	-72.021\\
1246	-52.49\\
1247	-79.346\\
1248	-80.566\\
1249	-57.373\\
1250	-70.801\\
1251	-63.477\\
1252	-56.152\\
1253	-67.139\\
1254	-41.504\\
1255	-48.828\\
1256	-36.621\\
1257	-40.283\\
1258	-75.684\\
1259	-84.229\\
1260	-113.525\\
1261	-146.484\\
1262	-98.877\\
1263	-58.594\\
1264	-46.387\\
1265	-43.945\\
1266	-63.477\\
1267	-42.725\\
1268	-47.607\\
1269	-61.035\\
1270	-102.539\\
1271	-93.994\\
1272	-134.277\\
1273	-89.111\\
1274	-81.787\\
1275	-46.387\\
1276	-45.166\\
1277	-28.076\\
1278	-35.4\\
1279	-37.842\\
1280	-67.139\\
1281	-70.801\\
1282	-79.346\\
1283	-85.449\\
1284	-125.732\\
1285	-112.305\\
1286	-84.229\\
1287	-102.539\\
1288	-109.863\\
1289	-144.043\\
1290	-115.967\\
1291	-75.684\\
1292	-40.283\\
1293	-29.297\\
1294	-29.297\\
1295	-36.621\\
1296	-39.063\\
1297	-37.842\\
1298	-50.049\\
1299	-74.463\\
1300	-68.359\\
1301	-70.801\\
1302	-83.008\\
1303	-51.27\\
1304	-30.518\\
1305	-58.594\\
1306	-90.332\\
1307	-87.891\\
1308	-97.656\\
1309	-83.008\\
1310	-61.035\\
1311	-85.449\\
1312	-118.408\\
1313	-118.408\\
1314	-84.229\\
1315	-136.719\\
1316	-100.098\\
1317	-101.318\\
1318	-117.188\\
1319	-115.967\\
1320	-86.67\\
1321	-76.904\\
1322	-128.174\\
1323	-178.223\\
1324	-139.16\\
1325	-79.346\\
1326	-76.904\\
1327	-79.346\\
1328	-98.877\\
1329	-114.746\\
1330	-85.449\\
1331	-90.332\\
1332	-120.85\\
1333	-131.836\\
1334	-87.891\\
1335	-68.359\\
1336	-91.553\\
1337	-156.25\\
1338	-137.939\\
1339	-140.381\\
1340	-84.229\\
1341	-76.904\\
1342	-72.021\\
1343	-47.607\\
1344	-30.518\\
1345	-26.855\\
1346	-23.193\\
1347	-54.932\\
1348	-70.801\\
1349	-87.891\\
1350	-65.918\\
1351	-73.242\\
1352	-83.008\\
1353	-53.711\\
1354	-56.152\\
1355	-53.711\\
1356	-40.283\\
1357	-51.27\\
1358	-84.229\\
1359	-106.201\\
1360	-63.477\\
1361	-48.828\\
1362	-40.283\\
1363	-50.049\\
1364	-35.4\\
1365	-76.904\\
1366	-130.615\\
1367	-104.98\\
1368	-139.16\\
1369	-162.354\\
1370	-164.795\\
1371	-124.512\\
1372	-90.332\\
1373	-89.111\\
1374	-89.111\\
1375	-106.201\\
1376	-125.732\\
1377	-125.732\\
1378	-168.457\\
1379	-183.105\\
1380	-219.727\\
1381	-147.705\\
1382	-166.016\\
1383	-184.326\\
1384	-140.381\\
1385	-162.354\\
1386	-139.16\\
1387	-80.566\\
1388	-52.49\\
1389	-56.152\\
1390	-81.787\\
1391	-65.918\\
1392	-43.945\\
1393	-62.256\\
1394	-47.607\\
1395	-36.621\\
1396	-46.387\\
1397	-52.49\\
1398	-36.621\\
1399	-70.801\\
1400	-92.773\\
1401	-68.359\\
1402	-114.746\\
1403	-164.795\\
1404	-150.146\\
1405	-196.533\\
1406	-131.836\\
1407	-144.043\\
1408	-96.436\\
1409	-63.477\\
1410	-76.904\\
1411	-59.814\\
1412	-45.166\\
1413	-64.697\\
1414	-36.621\\
1415	-28.076\\
1416	-34.18\\
1417	-70.801\\
1418	-86.67\\
1419	-107.422\\
1420	-103.76\\
1421	-100.098\\
1422	-114.746\\
1423	-93.994\\
1424	-76.904\\
1425	-76.904\\
1426	-62.256\\
1427	-97.656\\
1428	-68.359\\
1429	-56.152\\
1430	-81.787\\
1431	-113.525\\
1432	-86.67\\
1433	-65.918\\
1434	-56.152\\
1435	-65.918\\
1436	-45.166\\
1437	-45.166\\
1438	-59.814\\
1439	-75.684\\
1440	-84.229\\
1441	-65.918\\
1442	-86.67\\
1443	-79.346\\
1444	-47.607\\
1445	-50.049\\
1446	-95.215\\
1447	-131.836\\
1448	-131.836\\
1449	-107.422\\
1450	-103.76\\
1451	-79.346\\
1452	-76.904\\
1453	-61.035\\
1454	-89.111\\
1455	-75.684\\
1456	-50.049\\
1457	-74.463\\
1458	-85.449\\
1459	-101.318\\
1460	-109.863\\
1461	-109.863\\
1462	-148.926\\
1463	-181.885\\
1464	-205.078\\
1465	-129.395\\
1466	-73.242\\
1467	-47.607\\
1468	-34.18\\
1469	-54.932\\
1470	-46.387\\
1471	-80.566\\
1472	-72.021\\
1473	-64.697\\
1474	-85.449\\
1475	-98.877\\
1476	-117.188\\
1477	-123.291\\
1478	-175.781\\
1479	-150.146\\
1480	-117.188\\
1481	-80.566\\
1482	-80.566\\
1483	-83.008\\
1484	-106.201\\
1485	-75.684\\
1486	-79.346\\
1487	-74.463\\
1488	-42.725\\
1489	-74.463\\
1490	-107.422\\
1491	-73.242\\
1492	-80.566\\
1493	-147.705\\
1494	-109.863\\
1495	-106.201\\
1496	-147.705\\
1497	-140.381\\
1498	-87.891\\
1499	-56.152\\
1500	-58.594\\
};
\addlegendentry{True output}

\addplot [color=mycolor2, dashed]
  table[row sep=crcr]{%
1001	-85.8756471558183\\
1002	-108.781975897838\\
1003	-93.9974712580271\\
1004	-89.4772939880649\\
1005	-119.592201601983\\
1006	-105.370464520536\\
1007	-132.748351280165\\
1008	-104.931762756019\\
1009	-50.7605166987932\\
1010	-64.7981708146356\\
1011	-63.0915843052091\\
1012	-84.4129891825257\\
1013	-69.2122289037765\\
1014	-26.7702783148575\\
1015	-18.330887815544\\
1016	-15.8811324664409\\
1017	-53.0750856773175\\
1018	-97.3231810491477\\
1019	-102.86396249747\\
1020	-98.012687664862\\
1021	-69.5591603474269\\
1022	-38.1340493794365\\
1023	-92.1827780839704\\
1024	-61.2327889727107\\
1025	-51.6239957421089\\
1026	-87.1615955457352\\
1027	-71.4207821629431\\
1028	-63.5639501835796\\
1029	-100.656932405948\\
1030	-87.9960556079452\\
1031	-70.9656747799672\\
1032	-98.0233763731248\\
1033	-94.939874638527\\
1034	-80.752029344413\\
1035	-68.4948585985856\\
1036	-54.3485939604692\\
1037	-66.2225762139784\\
1038	-77.4629737823706\\
1039	-78.0912498278969\\
1040	-77.8276936973815\\
1041	-79.7049346251984\\
1042	-80.8794046746441\\
1043	-114.384922924191\\
1044	-105.335106603163\\
1045	-75.0889341430586\\
1046	-67.6859302134342\\
1047	-36.4191846509842\\
1048	-39.5086324797758\\
1049	-59.4444915089967\\
1050	-46.775582839451\\
1051	-51.5217532613501\\
1052	-68.641422517927\\
1053	-74.7721428773677\\
1054	-71.5515352011179\\
1055	-109.227122446292\\
1056	-79.7405793146656\\
1057	-48.111800036796\\
1058	-39.4822982607182\\
1059	-49.3079252339567\\
1060	-63.3152175355686\\
1061	-34.8496634270999\\
1062	-38.5340080664833\\
1063	-49.2208994835417\\
1064	-40.8319707841164\\
1065	-38.4988284743211\\
1066	-47.8755317930123\\
1067	-48.9109484243007\\
1068	-86.5104040109645\\
1069	-73.0384082533741\\
1070	-91.127798514347\\
1071	-80.5813627921775\\
1072	-87.0683229749219\\
1073	-74.7062858384942\\
1074	-71.5424063502072\\
1075	-66.2452288527443\\
1076	-66.3632111139621\\
1077	-64.9169035346241\\
1078	-105.871987366112\\
1079	-156.124622644753\\
1080	-161.950716519448\\
1081	-152.348807514668\\
1082	-97.4793216870099\\
1083	-138.238908462946\\
1084	-167.527896347524\\
1085	-175.70596256708\\
1086	-145.926551753494\\
1087	-173.466196576734\\
1088	-230.097844876827\\
1089	-177.48688171353\\
1090	-151.995568904892\\
1091	-100.855255766391\\
1092	-78.4157507840204\\
1093	-71.2112690532629\\
1094	-89.2219883239786\\
1095	-64.7731948291869\\
1096	-41.5762137311027\\
1097	-38.1976692003346\\
1098	-51.9331652671483\\
1099	-78.1824194842909\\
1100	-72.7748189646537\\
1101	-73.9527041323413\\
1102	-90.0840866965444\\
1103	-94.0698962690143\\
1104	-126.594176860882\\
1105	-111.533874231513\\
1106	-120.376813174784\\
1107	-94.8674922357301\\
1108	-79.6203973396839\\
1109	-95.1114844859548\\
1110	-72.3359268753987\\
1111	-67.8749189805355\\
1112	-83.0984459460096\\
1113	-81.8298272552157\\
1114	-64.2417485673695\\
1115	-66.0850897165648\\
1116	-48.3034192436949\\
1117	-50.7107738202193\\
1118	-58.4496895906325\\
1119	-44.3352942891684\\
1120	-53.1781351682474\\
1121	-63.3660758196293\\
1122	-98.3250286173859\\
1123	-91.4235983371455\\
1124	-104.603077537049\\
1125	-57.7555129035452\\
1126	-84.0872272647323\\
1127	-127.746436319712\\
1128	-103.125595576166\\
1129	-114.128078556118\\
1130	-108.023883095345\\
1131	-68.5556811028296\\
1132	-44.7650426369743\\
1133	-67.8571081248201\\
1134	-75.5626528091764\\
1135	-128.194063423056\\
1136	-153.481066618615\\
1137	-147.864349551485\\
1138	-106.517986880385\\
1139	-116.335406321578\\
1140	-102.681839004007\\
1141	-102.062316477502\\
1142	-105.155739136035\\
1143	-70.2749130666348\\
1144	-66.5958306265896\\
1145	-58.9902785598199\\
1146	-55.1201483838191\\
1147	-65.1696302739873\\
1148	-90.9770976354053\\
1149	-127.914728648067\\
1150	-111.906356807174\\
1151	-72.9946464124578\\
1152	-63.8775619633677\\
1153	-58.6783003597715\\
1154	-35.8639992019743\\
1155	-26.6158544787682\\
1156	-33.2295453682164\\
1157	-56.3501679575868\\
1158	-47.3306539114569\\
1159	-53.8166028582985\\
1160	-53.1880151769902\\
1161	-51.209649876813\\
1162	-36.8133711526278\\
1163	-29.1913287008618\\
1164	-21.5784139097438\\
1165	-32.4736398419213\\
1166	-75.1281829591923\\
1167	-110.077291061382\\
1168	-116.120052987912\\
1169	-78.3149851711529\\
1170	-48.7263578175166\\
1171	-35.3462100821797\\
1172	-24.1734263727574\\
1173	-57.6065085652929\\
1174	-55.4799217260177\\
1175	-80.5909021921064\\
1176	-101.087142277794\\
1177	-167.281172858719\\
1178	-194.111560602298\\
1179	-160.480985585159\\
1180	-142.3964177489\\
1181	-94.7777875540258\\
1182	-106.060930175999\\
1183	-120.715599585085\\
1184	-128.484864583248\\
1185	-108.391927096851\\
1186	-90.930285192826\\
1187	-99.4741307308238\\
1188	-90.7320252418332\\
1189	-101.89046119946\\
1190	-79.7289216784303\\
1191	-123.158016374611\\
1192	-150.711815539327\\
1193	-107.698527017693\\
1194	-72.6268787537307\\
1195	-67.8930808923659\\
1196	-113.050519763141\\
1197	-142.891314858754\\
1198	-163.723789497519\\
1199	-172.263877927729\\
1200	-173.84029981327\\
1201	-133.20839631612\\
1202	-126.158570328342\\
1203	-140.597118237057\\
1204	-163.350770713792\\
1205	-101.403548379271\\
1206	-65.2394580900068\\
1207	-100.684042198925\\
1208	-84.1819585630751\\
1209	-55.9673254717629\\
1210	-75.4745577663282\\
1211	-83.2561289181423\\
1212	-69.6611766822554\\
1213	-61.6790005874582\\
1214	-71.446116784998\\
1215	-61.9728889934971\\
1216	-74.7905104883132\\
1217	-110.025164831449\\
1218	-95.8660943225174\\
1219	-72.9686695127974\\
1220	-69.0698483611327\\
1221	-98.9452931398126\\
1222	-162.62935285924\\
1223	-131.064085672889\\
1224	-76.2726607590403\\
1225	-61.1110337253837\\
1226	-65.7494772530113\\
1227	-67.4707303343973\\
1228	-49.184487901371\\
1229	-57.8857255419981\\
1230	-64.8107963805425\\
1231	-80.2593388202381\\
1232	-91.5658197586818\\
1233	-63.9236791661498\\
1234	-97.3876536518374\\
1235	-115.409400335629\\
1236	-104.741272318663\\
1237	-85.7316778262958\\
1238	-89.7698398821794\\
1239	-115.925294977797\\
1240	-88.8139853045268\\
1241	-33.2363764279707\\
1242	-46.5751911987759\\
1243	-54.8038908593412\\
1244	-80.0853166822502\\
1245	-73.198581916334\\
1246	-56.4011035268316\\
1247	-80.0465405844137\\
1248	-71.2055409660861\\
1249	-54.4111288873932\\
1250	-70.809456924989\\
1251	-59.4729137075886\\
1252	-54.6788293961354\\
1253	-66.8546137233178\\
1254	-39.9647406143725\\
1255	-48.7845733275513\\
1256	-34.7757703239027\\
1257	-39.2536325530202\\
1258	-73.0581654781187\\
1259	-74.7946611004563\\
1260	-100.500590551516\\
1261	-130.69484428458\\
1262	-87.8099115799422\\
1263	-51.7337369055579\\
1264	-37.2439088948131\\
1265	-36.4284481152178\\
1266	-61.9997521243629\\
1267	-40.6450368856659\\
1268	-46.4611474593169\\
1269	-55.6030436347039\\
1270	-94.0905369557472\\
1271	-86.1146952295014\\
1272	-119.317270306922\\
1273	-81.2771577675406\\
1274	-66.8840394056158\\
1275	-34.9827539411686\\
1276	-34.4921217703222\\
1277	-23.0453047554055\\
1278	-29.7138966761695\\
1279	-34.8109649282721\\
1280	-59.4233191219673\\
1281	-64.4157765704903\\
1282	-69.0598776304082\\
1283	-70.8608433854086\\
1284	-105.817506673668\\
1285	-104.228341995503\\
1286	-73.7343425264116\\
1287	-84.8250239504844\\
1288	-91.8183609307161\\
1289	-117.469356706803\\
1290	-96.8508574023282\\
1291	-65.3883836588961\\
1292	-27.4953272699391\\
1293	-19.5863117798302\\
1294	-22.9012764922793\\
1295	-31.5126277299564\\
1296	-36.397220443073\\
1297	-36.5727948055756\\
1298	-42.3790852711318\\
1299	-68.7833360996199\\
1300	-62.3213105521933\\
1301	-60.3271378712103\\
1302	-67.475031441762\\
1303	-41.8627120561024\\
1304	-26.1540209508996\\
1305	-47.4369012882117\\
1306	-73.4422586009234\\
1307	-73.4571870983283\\
1308	-84.7306936354444\\
1309	-66.3034122085303\\
1310	-49.7077760912864\\
1311	-67.6607393203343\\
1312	-92.9700130417137\\
1313	-96.5345194540836\\
1314	-71.0875502000962\\
1315	-111.01206409922\\
1316	-85.6637373933219\\
1317	-80.2079491417663\\
1318	-100.801126898798\\
1319	-94.050725115293\\
1320	-78.2906044390477\\
1321	-70.7631626569724\\
1322	-108.254861544063\\
1323	-155.32202100177\\
1324	-121.966031174252\\
1325	-71.3431775721947\\
1326	-63.3984558161095\\
1327	-72.7782671031525\\
1328	-86.1915770469237\\
1329	-100.446714167238\\
1330	-79.182330001914\\
1331	-86.385562735571\\
1332	-105.628028690026\\
1333	-108.872978846652\\
1334	-78.568494676037\\
1335	-68.3452069759672\\
1336	-83.6253378013735\\
1337	-142.326385609468\\
1338	-125.26319396611\\
1339	-121.78644473948\\
1340	-75.6878088906359\\
1341	-63.9988657197975\\
1342	-66.3264825351296\\
1343	-41.4795963396261\\
1344	-26.8361437550982\\
1345	-21.6075153293021\\
1346	-19.6457621343238\\
1347	-45.6766490248392\\
1348	-66.0599086651134\\
1349	-81.9621864662466\\
1350	-61.6584650112253\\
1351	-60.1548117950183\\
1352	-67.4130260577116\\
1353	-41.5371716935127\\
1354	-46.8778444321015\\
1355	-44.6308342480773\\
1356	-34.5444255582418\\
1357	-46.6416918428564\\
1358	-69.8962584487476\\
1359	-90.4882583194646\\
1360	-54.7842905200869\\
1361	-42.5561624596549\\
1362	-35.7678840465475\\
1363	-44.3542092314954\\
1364	-30.8607919089673\\
1365	-71.5955780653889\\
1366	-117.823030255056\\
1367	-96.6477299291345\\
1368	-117.085886058351\\
1369	-131.279840981503\\
1370	-134.935040377696\\
1371	-101.027942970684\\
1372	-74.8852086083915\\
1373	-78.6810661094642\\
1374	-77.6426376099907\\
1375	-94.62684790852\\
1376	-109.768206918528\\
1377	-106.469685728954\\
1378	-140.67909660764\\
1379	-155.272568428374\\
1380	-179.080960225594\\
1381	-130.118275265813\\
1382	-143.109207960522\\
1383	-151.229815627344\\
1384	-127.7735433281\\
1385	-143.473200277865\\
1386	-127.514563467725\\
1387	-74.2189351874832\\
1388	-44.1386176769837\\
1389	-48.6829986222211\\
1390	-75.0993367114458\\
1391	-65.0783137454596\\
1392	-48.0033602277612\\
1393	-64.4305265797044\\
1394	-46.1489302920758\\
1395	-38.6028092105506\\
1396	-46.6238538488804\\
1397	-45.9894745233666\\
1398	-40.5272628437784\\
1399	-64.5952865619058\\
1400	-84.9274073270748\\
1401	-64.7306704351326\\
1402	-107.983297362985\\
1403	-154.572808763304\\
1404	-139.632048006407\\
1405	-169.689905543426\\
1406	-115.742525519512\\
1407	-125.923530701896\\
1408	-86.2481819449752\\
1409	-50.6562361987688\\
1410	-68.6041923654971\\
1411	-57.7798792489252\\
1412	-44.3090320719579\\
1413	-61.0547392391141\\
1414	-40.0094952324102\\
1415	-24.2627975862001\\
1416	-31.0052601561606\\
1417	-61.1039119975748\\
1418	-84.1614784784435\\
1419	-98.3167177316726\\
1420	-94.8527524964739\\
1421	-83.1201272937762\\
1422	-93.3749262581091\\
1423	-78.6580837617323\\
1424	-67.1801869245947\\
1425	-69.3659691597715\\
1426	-57.0802958073391\\
1427	-88.6940222564625\\
1428	-59.3587490352998\\
1429	-52.3273445808056\\
1430	-77.1928808209879\\
1431	-97.5083724354539\\
1432	-78.243073278266\\
1433	-61.2223618500198\\
1434	-53.7295847177309\\
1435	-60.7584881634399\\
1436	-42.3136247001052\\
1437	-41.969765006024\\
1438	-56.4874704385603\\
1439	-64.9891970392833\\
1440	-75.0558110957581\\
1441	-62.9354389468223\\
1442	-74.2987443150698\\
1443	-66.7046221421842\\
1444	-41.9109777609738\\
1445	-45.8929603089634\\
1446	-82.522618495864\\
1447	-122.290796633796\\
1448	-114.61174257422\\
1449	-92.929295583135\\
1450	-85.7555605676239\\
1451	-71.501159182665\\
1452	-66.0037305441524\\
1453	-58.9969904786343\\
1454	-74.9227343805362\\
1455	-73.6597455674379\\
1456	-46.2222933708766\\
1457	-70.5869418771113\\
1458	-72.0296214269847\\
1459	-86.7305910509578\\
1460	-97.9148631674249\\
1461	-88.8822818039363\\
1462	-130.04880794556\\
1463	-153.802720848001\\
1464	-171.910300059338\\
1465	-119.781430066221\\
1466	-59.5550303619634\\
1467	-34.5462721671724\\
1468	-23.6585844618198\\
1469	-43.9700992560787\\
1470	-41.4967819951212\\
1471	-77.2748572125817\\
1472	-67.5502438328214\\
1473	-61.1456686455546\\
1474	-76.3723721801872\\
1475	-82.7228933580437\\
1476	-100.454321559855\\
1477	-104.927116168167\\
1478	-150.565523516917\\
1479	-137.570189035985\\
1480	-103.062479689649\\
1481	-75.9491345705617\\
1482	-69.1772775993997\\
1483	-81.0941223821902\\
1484	-96.9778918666842\\
1485	-74.4545346071711\\
1486	-74.1614206884306\\
1487	-77.8984003096345\\
1488	-36.4628772571252\\
1489	-68.8615862616538\\
1490	-96.6519930458516\\
1491	-71.7693255688265\\
1492	-71.9605221742408\\
1493	-138.916057225027\\
1494	-95.8535280172065\\
1495	-98.0005547531333\\
1496	-126.559534249726\\
1497	-118.62926555078\\
1498	-83.1282598517199\\
1499	-51.5170785587445\\
1500	-51.4928349932362\\
};
\addlegendentry{Generated output}

\end{axis}
\end{tikzpicture}%\label{fig:c8all}}
%	\caption{Samples of the output obtained experimentally and the output generated by the identified model.}\label{fig:Callout}
%\end{figure}
%
%\begin{figure}[!t]
%	\centering
%	% This file was created by matlab2tikz.
%
\definecolor{mycolor1}{rgb}{0.00000,0.44700,0.74100}%
\definecolor{mycolor2}{rgb}{0.85000,0.32500,0.09800}%
%
\begin{tikzpicture}

\begin{axis}[%
width=4.927cm,
height=3.484cm,
at={(0cm,14.516cm)},
scale only axis,
xmin=-60,
xmax=0,
xlabel style={font=\color{white!15!black}},
xlabel={y(t-1)},
ymin=-58.594,
ymax=0,
ylabel style={font=\color{white!15!black}},
ylabel={y(t)},
axis background/.style={fill=white},
title={C1, R = 0.7745},
axis x line*=bottom,
axis y line*=left,
legend style={legend cell align=left, align=left, draw=white!15!black}
]
\addplot[only marks, mark=*, mark options={}, mark size=1.5000pt, color=mycolor1, fill=mycolor1] table[row sep=crcr]{%
x	y\\
-19.531	-23.193\\
-23.193	-28.076\\
-28.076	-26.855\\
-26.855	-20.752\\
-20.752	-29.297\\
-29.297	-28.076\\
-28.076	-32.959\\
-32.959	-25.635\\
-25.635	-14.648\\
-14.648	-20.752\\
-20.752	-17.09\\
-17.09	-18.311\\
-18.311	-20.752\\
-20.752	-7.324\\
-7.324	-4.883\\
-4.883	-4.883\\
-4.883	-13.428\\
-13.428	-23.193\\
-23.193	-24.414\\
-24.414	-25.635\\
-25.635	-19.531\\
-19.531	-12.207\\
-12.207	-24.414\\
-24.414	-19.531\\
-19.531	-14.648\\
-14.648	-20.752\\
-20.752	-18.311\\
-18.311	-17.09\\
-17.09	-29.297\\
-29.297	-25.635\\
-25.635	-18.311\\
-18.311	-28.076\\
-28.076	-28.076\\
-28.076	-21.973\\
-21.973	-18.311\\
-18.311	-15.869\\
-15.869	-15.869\\
-15.869	-21.973\\
-21.973	-21.973\\
-21.973	-21.973\\
-21.973	-21.973\\
-21.973	-23.193\\
-23.193	-30.518\\
-30.518	-29.297\\
-29.297	-19.531\\
-19.531	-18.311\\
-18.311	-10.986\\
-10.986	-14.648\\
-14.648	-15.869\\
-15.869	-12.207\\
-12.207	-13.428\\
-13.428	-18.311\\
-18.311	-19.531\\
-19.531	-18.311\\
-18.311	-29.297\\
-29.297	-21.973\\
-21.973	-12.207\\
-12.207	-12.207\\
-12.207	-13.428\\
-13.428	-17.09\\
-17.09	-9.766\\
-9.766	-10.986\\
-10.986	-14.648\\
-14.648	-9.766\\
-9.766	-7.324\\
-7.324	-13.428\\
-13.428	-13.428\\
-13.428	-21.973\\
-21.973	-20.752\\
-20.752	-25.635\\
-25.635	-21.973\\
-21.973	-25.635\\
-25.635	-20.752\\
-20.752	-20.752\\
-20.752	-18.311\\
-18.311	-18.311\\
-18.311	-17.09\\
-17.09	-30.518\\
-30.518	-41.504\\
-41.504	-41.504\\
-41.504	-42.725\\
-42.725	-28.076\\
-28.076	-37.842\\
-37.842	-46.387\\
-46.387	-46.387\\
-46.387	-35.4\\
-35.4	-47.607\\
-47.607	-58.594\\
-58.594	-43.945\\
-43.945	-34.18\\
-34.18	-24.414\\
-24.414	-20.752\\
-20.752	-17.09\\
-17.09	-21.973\\
-21.973	-17.09\\
-17.09	-12.207\\
-12.207	-9.766\\
-9.766	-12.207\\
-12.207	-18.311\\
-18.311	-18.311\\
-18.311	-18.311\\
-18.311	-21.973\\
-21.973	-23.193\\
-23.193	-30.518\\
-30.518	-28.076\\
-28.076	-30.518\\
-30.518	-25.635\\
-25.635	-20.752\\
-20.752	-24.414\\
-24.414	-18.311\\
-18.311	-13.428\\
-13.428	-19.531\\
-19.531	-20.752\\
-20.752	-12.207\\
-12.207	-15.869\\
-15.869	-12.207\\
-12.207	-13.428\\
-13.428	-13.428\\
-13.428	-9.766\\
-9.766	-10.986\\
-10.986	-17.09\\
-17.09	-23.193\\
-23.193	-25.635\\
-25.635	-26.855\\
-26.855	-15.869\\
-15.869	-20.752\\
-20.752	-32.959\\
-32.959	-24.414\\
-24.414	-28.076\\
-28.076	-29.297\\
-29.297	-18.311\\
-18.311	-12.207\\
-12.207	-17.09\\
-17.09	-18.311\\
-18.311	-30.518\\
-30.518	-36.621\\
-36.621	-36.621\\
-36.621	-28.076\\
-28.076	-28.076\\
-28.076	-25.635\\
-25.635	-25.635\\
-25.635	-25.635\\
-25.635	-20.752\\
-20.752	-15.869\\
-15.869	-14.648\\
-14.648	-13.428\\
-13.428	-13.428\\
-13.428	-20.752\\
-20.752	-31.738\\
-31.738	-26.855\\
-26.855	-17.09\\
-17.09	-15.869\\
-15.869	-15.869\\
-15.869	-9.766\\
-9.766	-7.324\\
-7.324	-7.324\\
-7.324	-15.869\\
-15.869	-10.986\\
-10.986	-14.648\\
-14.648	-14.648\\
-14.648	-13.428\\
-13.428	-9.766\\
-9.766	-6.104\\
-6.104	-4.883\\
-4.883	-7.324\\
-7.324	-18.311\\
-18.311	-26.855\\
-26.855	-28.076\\
-28.076	-20.752\\
-20.752	-13.428\\
-13.428	-10.986\\
-10.986	-6.104\\
-6.104	-12.207\\
-12.207	-14.648\\
-14.648	-18.311\\
-18.311	-26.855\\
-26.855	-39.063\\
-39.063	-47.607\\
-47.607	-42.725\\
-42.725	-37.842\\
-37.842	-28.076\\
-28.076	-30.518\\
-30.518	-31.738\\
-31.738	-34.18\\
-34.18	-24.414\\
-24.414	-23.193\\
-23.193	-23.193\\
-23.193	-21.973\\
-21.973	-23.193\\
-23.193	-18.311\\
-18.311	-30.518\\
-30.518	-37.842\\
-37.842	-28.076\\
-28.076	-18.311\\
-18.311	-18.311\\
-18.311	-30.518\\
-30.518	-35.4\\
-35.4	-40.283\\
-40.283	-45.166\\
-45.166	-45.166\\
-45.166	-34.18\\
-34.18	-32.959\\
-32.959	-36.621\\
-36.621	-41.504\\
-41.504	-29.297\\
-29.297	-17.09\\
-17.09	-23.193\\
-23.193	-20.752\\
-20.752	-13.428\\
-13.428	-18.311\\
-18.311	-19.531\\
-19.531	-14.648\\
-14.648	-13.428\\
-13.428	-15.869\\
-15.869	-13.428\\
-13.428	-17.09\\
-17.09	-25.635\\
-25.635	-25.635\\
-25.635	-15.869\\
-15.869	-14.648\\
-14.648	-24.414\\
-24.414	-40.283\\
-40.283	-29.297\\
-29.297	-20.752\\
-20.752	-14.648\\
-14.648	-18.311\\
-18.311	-15.869\\
-15.869	-12.207\\
-12.207	-13.428\\
-13.428	-15.869\\
-15.869	-18.311\\
-18.311	-20.752\\
-20.752	-15.869\\
-15.869	-23.193\\
-23.193	-29.297\\
-29.297	-25.635\\
-25.635	-20.752\\
-20.752	-23.193\\
-23.193	-30.518\\
-30.518	-21.973\\
-21.973	-8.545\\
-8.545	-13.428\\
-13.428	-13.428\\
-13.428	-17.09\\
-17.09	-17.09\\
-17.09	-13.428\\
-13.428	-17.09\\
-17.09	-19.531\\
-19.531	-13.428\\
-13.428	-15.869\\
-15.869	-15.869\\
-15.869	-12.207\\
-12.207	-15.869\\
-15.869	-9.766\\
-9.766	-10.986\\
-10.986	-8.545\\
-8.545	-8.545\\
-8.545	-14.648\\
-14.648	-20.752\\
-20.752	-24.414\\
-24.414	-35.4\\
-35.4	-24.414\\
-24.414	-13.428\\
-13.428	-12.207\\
-12.207	-10.986\\
-10.986	-13.428\\
-13.428	-10.986\\
-10.986	-12.207\\
-12.207	-14.648\\
-14.648	-24.414\\
-24.414	-21.973\\
-21.973	-26.855\\
-26.855	-23.193\\
-23.193	-17.09\\
-17.09	-13.428\\
-13.428	-10.986\\
-10.986	-8.545\\
-8.545	-6.104\\
-6.104	-9.766\\
-9.766	-14.648\\
-14.648	-18.311\\
-18.311	-18.311\\
-18.311	-19.531\\
-19.531	-29.297\\
-29.297	-25.635\\
-25.635	-18.311\\
-18.311	-24.414\\
-24.414	-24.414\\
-24.414	-31.738\\
-31.738	-26.855\\
-26.855	-17.09\\
-17.09	-9.766\\
-9.766	-6.104\\
-6.104	-7.324\\
-7.324	-9.766\\
-9.766	-8.545\\
-8.545	-8.545\\
-8.545	-12.207\\
-12.207	-17.09\\
-17.09	-15.869\\
-15.869	-15.869\\
-15.869	-19.531\\
-19.531	-13.428\\
-13.428	-4.883\\
-4.883	-9.766\\
-9.766	-21.973\\
-21.973	-19.531\\
-19.531	-21.973\\
-21.973	-18.311\\
-18.311	-14.648\\
-14.648	-19.531\\
-19.531	-25.635\\
-25.635	-25.635\\
-25.635	-19.531\\
-19.531	-26.855\\
-26.855	-26.855\\
-26.855	-23.193\\
-23.193	-28.076\\
-28.076	-26.855\\
-26.855	-20.752\\
-20.752	-19.531\\
-19.531	-29.297\\
-29.297	-40.283\\
-40.283	-30.518\\
-30.518	-18.311\\
-18.311	-17.09\\
-17.09	-18.311\\
-18.311	-21.973\\
-21.973	-26.855\\
-26.855	-19.531\\
-19.531	-19.531\\
-19.531	-26.855\\
-26.855	-28.076\\
-28.076	-20.752\\
-20.752	-14.648\\
-14.648	-20.752\\
-20.752	-34.18\\
-34.18	-30.518\\
-30.518	-31.738\\
-31.738	-21.973\\
-21.973	-17.09\\
-17.09	-17.09\\
-17.09	-10.986\\
-10.986	-6.104\\
-6.104	-6.104\\
-6.104	-4.883\\
-4.883	-10.986\\
-10.986	-19.531\\
-19.531	-18.311\\
-18.311	-15.869\\
-15.869	-14.648\\
-14.648	-20.752\\
-20.752	-14.648\\
-14.648	-9.766\\
-9.766	-13.428\\
-13.428	-8.545\\
-8.545	-10.986\\
-10.986	-18.311\\
-18.311	-23.193\\
-23.193	-17.09\\
-17.09	-9.766\\
-9.766	-10.986\\
-10.986	-12.207\\
-12.207	-9.766\\
-9.766	-13.428\\
-13.428	-31.738\\
-31.738	-25.635\\
-25.635	-31.738\\
-31.738	-37.842\\
-37.842	-36.621\\
-36.621	-26.855\\
-26.855	-20.752\\
-20.752	-18.311\\
-18.311	-21.973\\
-21.973	-23.193\\
-23.193	-29.297\\
-29.297	-29.297\\
-29.297	-36.621\\
-36.621	-42.725\\
-42.725	-46.387\\
-46.387	-37.842\\
-37.842	-35.4\\
-35.4	-40.283\\
-40.283	-31.738\\
-31.738	-34.18\\
-34.18	-31.738\\
-31.738	-18.311\\
-18.311	-12.207\\
-12.207	-13.428\\
-13.428	-18.311\\
-18.311	-17.09\\
-17.09	-8.545\\
-8.545	-14.648\\
-14.648	-13.428\\
-13.428	-6.104\\
-6.104	-10.986\\
-10.986	-12.207\\
-12.207	-7.324\\
-7.324	-18.311\\
-18.311	-20.752\\
-20.752	-14.648\\
-14.648	-23.193\\
-23.193	-37.842\\
-37.842	-32.959\\
-32.959	-37.842\\
-37.842	-32.959\\
-32.959	-29.297\\
-29.297	-20.752\\
-20.752	-14.648\\
-14.648	-17.09\\
-17.09	-13.428\\
-13.428	-10.986\\
-10.986	-14.648\\
-14.648	-13.428\\
-13.428	-3.662\\
-3.662	-7.324\\
-7.324	-15.869\\
-15.869	-20.752\\
-20.752	-21.973\\
-21.973	-23.193\\
-23.193	-21.973\\
-21.973	-26.855\\
-26.855	-21.973\\
-21.973	-17.09\\
-17.09	-17.09\\
-17.09	-14.648\\
-14.648	-20.752\\
-20.752	-18.311\\
-18.311	-13.428\\
-13.428	-18.311\\
-18.311	-26.855\\
-26.855	-20.752\\
-20.752	-15.869\\
-15.869	-13.428\\
-13.428	-14.648\\
-14.648	-12.207\\
-12.207	-8.545\\
-8.545	-14.648\\
-14.648	-19.531\\
-19.531	-18.311\\
-18.311	-15.869\\
-15.869	-19.531\\
-19.531	-18.311\\
-18.311	-10.986\\
-10.986	-10.986\\
-10.986	-21.973\\
-21.973	-30.518\\
-30.518	-29.297\\
-29.297	-23.193\\
-23.193	-21.973\\
-21.973	-19.531\\
-19.531	-18.311\\
-18.311	-15.869\\
-15.869	-19.531\\
-19.531	-17.09\\
-17.09	-13.428\\
-13.428	-17.09\\
-17.09	-19.531\\
-19.531	-21.973\\
-21.973	-25.635\\
-25.635	-25.635\\
-25.635	-32.959\\
-32.959	-41.504\\
-41.504	-45.166\\
-45.166	-31.738\\
-31.738	-17.09\\
-17.09	-13.428\\
-13.428	-8.545\\
-8.545	-10.986\\
-10.986	-13.428\\
-13.428	-17.09\\
-17.09	-18.311\\
-18.311	-13.428\\
-13.428	-19.531\\
-19.531	-24.414\\
-24.414	-25.635\\
-25.635	-26.855\\
-26.855	-39.063\\
-39.063	-35.4\\
-35.4	-25.635\\
-25.635	-20.752\\
-20.752	-20.752\\
-20.752	-19.531\\
-19.531	-23.193\\
-23.193	-19.531\\
-19.531	-15.869\\
-15.869	-17.09\\
-17.09	-10.986\\
-10.986	-14.648\\
-14.648	-28.076\\
-28.076	-19.531\\
-19.531	-15.869\\
-15.869	-34.18\\
-34.18	-28.076\\
-28.076	-21.973\\
-21.973	-34.18\\
-34.18	-31.738\\
-31.738	-20.752\\
-20.752	-15.869\\
-15.869	-13.428\\
-13.428	-23.193\\
-23.193	-35.4\\
-35.4	-37.842\\
-37.842	-36.621\\
-36.621	-29.297\\
-29.297	-20.752\\
-20.752	-15.869\\
-15.869	-14.648\\
-14.648	-10.986\\
-10.986	-10.986\\
-10.986	-13.428\\
-13.428	-15.869\\
-15.869	-10.986\\
-10.986	-13.428\\
-13.428	-21.973\\
-21.973	-21.973\\
-21.973	-26.855\\
-26.855	-35.4\\
-35.4	-43.945\\
-43.945	-28.076\\
-28.076	-28.076\\
-28.076	-28.076\\
-28.076	-24.414\\
-24.414	-18.311\\
-18.311	-20.752\\
-20.752	-34.18\\
-34.18	-37.842\\
-37.842	-24.414\\
-24.414	-12.207\\
-12.207	-10.986\\
-10.986	-3.662\\
-3.662	-8.545\\
-8.545	-9.766\\
-9.766	-12.207\\
-12.207	-13.428\\
-13.428	-13.428\\
-13.428	-7.324\\
-7.324	-8.545\\
-8.545	-10.986\\
-10.986	-8.545\\
-8.545	-12.207\\
-12.207	-14.648\\
-14.648	-24.414\\
-24.414	-28.076\\
-28.076	-24.414\\
-24.414	-28.076\\
-28.076	-35.4\\
-35.4	-35.4\\
-35.4	-40.283\\
-40.283	-40.283\\
-40.283	-29.297\\
-29.297	-19.531\\
-19.531	-18.311\\
-18.311	-30.518\\
-30.518	-37.842\\
-37.842	-26.855\\
-26.855	-15.869\\
-15.869	-9.766\\
-9.766	-4.883\\
-4.883	-6.104\\
-6.104	-6.104\\
-6.104	-6.104\\
-6.104	-12.207\\
-12.207	-14.648\\
-14.648	-12.207\\
-12.207	-9.766\\
-9.766	-4.883\\
-4.883	-8.545\\
-8.545	-8.545\\
-8.545	-8.545\\
-8.545	-7.324\\
-7.324	-4.883\\
-4.883	-12.207\\
-12.207	-25.635\\
-25.635	-24.414\\
-24.414	-31.738\\
-31.738	-25.635\\
-25.635	-32.959\\
-32.959	-46.387\\
-46.387	-42.725\\
-42.725	-39.063\\
-39.063	-34.18\\
-34.18	-31.738\\
-31.738	-32.959\\
-32.959	-20.752\\
-20.752	-20.752\\
-20.752	-13.428\\
-13.428	-14.648\\
-14.648	-23.193\\
-23.193	-20.752\\
-20.752	-15.869\\
-15.869	-19.531\\
-19.531	-17.09\\
-17.09	-17.09\\
-17.09	-20.752\\
-20.752	-19.531\\
-19.531	-24.414\\
-24.414	-20.752\\
-20.752	-15.869\\
-15.869	-18.311\\
-18.311	-14.648\\
-14.648	-2.441\\
-2.441	-9.766\\
-9.766	-13.428\\
-13.428	-8.545\\
-8.545	-4.883\\
-4.883	-8.545\\
-8.545	-12.207\\
-12.207	-12.207\\
-12.207	-14.648\\
-14.648	-13.428\\
-13.428	-19.531\\
-19.531	-19.531\\
-19.531	-12.207\\
-12.207	-19.531\\
-19.531	-17.09\\
-17.09	-7.324\\
-7.324	-7.324\\
-7.324	-3.662\\
-3.662	-10.986\\
-10.986	-8.545\\
-8.545	-7.324\\
-7.324	-19.531\\
-19.531	-20.752\\
-20.752	-10.986\\
-10.986	-14.648\\
-14.648	-12.207\\
-12.207	-13.428\\
-13.428	-14.648\\
-14.648	-8.545\\
-8.545	-10.986\\
-10.986	-7.324\\
-7.324	-7.324\\
-7.324	-9.766\\
-9.766	-14.648\\
-14.648	-14.648\\
-14.648	-9.766\\
-9.766	-9.766\\
-9.766	-9.766\\
-9.766	-9.766\\
-9.766	-18.311\\
-18.311	-26.855\\
-26.855	-18.311\\
-18.311	-17.09\\
-17.09	-15.869\\
-15.869	-21.973\\
-21.973	-21.973\\
-21.973	-13.428\\
-13.428	-14.648\\
-14.648	-13.428\\
-13.428	-17.09\\
-17.09	-12.207\\
-12.207	-8.545\\
-8.545	-10.986\\
-10.986	-15.869\\
-15.869	-12.207\\
-12.207	-14.648\\
-14.648	-14.648\\
-14.648	-21.973\\
-21.973	-18.311\\
-18.311	-25.635\\
-25.635	-35.4\\
-35.4	-28.076\\
-28.076	-18.311\\
-18.311	-15.869\\
-15.869	-23.193\\
-23.193	-28.076\\
-28.076	-20.752\\
-20.752	-23.193\\
-23.193	-18.311\\
-18.311	-7.324\\
-7.324	-8.545\\
-8.545	-20.752\\
-20.752	-18.311\\
-18.311	-18.311\\
-18.311	-23.193\\
-23.193	-23.193\\
-23.193	-19.531\\
-19.531	-15.869\\
-15.869	-18.311\\
-18.311	-23.193\\
-23.193	-19.531\\
-19.531	-18.311\\
-18.311	-19.531\\
-19.531	-13.428\\
-13.428	-14.648\\
-14.648	-13.428\\
-13.428	-13.428\\
-13.428	-20.752\\
-20.752	-25.635\\
-25.635	-23.193\\
-23.193	-23.193\\
-23.193	-17.09\\
-17.09	-12.207\\
-12.207	-15.869\\
-15.869	-15.869\\
-15.869	-21.973\\
-21.973	-24.414\\
-24.414	-28.076\\
-28.076	-23.193\\
-23.193	-13.428\\
-13.428	-24.414\\
-24.414	-36.621\\
-36.621	-24.414\\
-24.414	-23.193\\
-23.193	-29.297\\
-29.297	-21.973\\
-21.973	-20.752\\
-20.752	-20.752\\
-20.752	-18.311\\
-18.311	-20.752\\
-20.752	-25.635\\
-25.635	-19.531\\
-19.531	-21.973\\
-21.973	-21.973\\
-21.973	-20.752\\
-20.752	-19.531\\
-19.531	-21.973\\
-21.973	-17.09\\
-17.09	-18.311\\
-18.311	-17.09\\
-17.09	-17.09\\
-17.09	-8.545\\
-8.545	-2.441\\
-2.441	-2.441\\
-2.441	-8.545\\
-8.545	-19.531\\
-19.531	-25.635\\
-25.635	-23.193\\
-23.193	-17.09\\
-17.09	-20.752\\
-20.752	-19.531\\
-19.531	-15.869\\
-15.869	-12.207\\
-12.207	-14.648\\
-14.648	-14.648\\
-14.648	-12.207\\
-12.207	-15.869\\
-15.869	-13.428\\
-13.428	-10.986\\
-10.986	-8.545\\
-8.545	-14.648\\
-14.648	-21.973\\
-21.973	-17.09\\
-17.09	-17.09\\
-17.09	-17.09\\
-17.09	-23.193\\
-23.193	-20.752\\
-20.752	-28.076\\
-28.076	-21.973\\
-21.973	-23.193\\
-23.193	-23.193\\
-23.193	-15.869\\
-15.869	-20.752\\
-20.752	-19.531\\
-19.531	-20.752\\
-20.752	-21.973\\
-21.973	-25.635\\
-25.635	-19.531\\
-19.531	-24.414\\
-24.414	-30.518\\
-30.518	-25.635\\
-25.635	-23.193\\
-23.193	-18.311\\
-18.311	-21.973\\
-21.973	-19.531\\
-19.531	-14.648\\
-14.648	-13.428\\
-13.428	-15.869\\
-15.869	-13.428\\
-13.428	-26.855\\
-26.855	-35.4\\
-35.4	-29.297\\
-29.297	-29.297\\
-29.297	-29.297\\
-29.297	-18.311\\
-18.311	-14.648\\
-14.648	-12.207\\
-12.207	-10.986\\
-10.986	-10.986\\
-10.986	-18.311\\
-18.311	-23.193\\
-23.193	-25.635\\
-25.635	-21.973\\
-21.973	-15.869\\
-15.869	-25.635\\
-25.635	-30.518\\
-30.518	-31.738\\
-31.738	-25.635\\
-25.635	-42.725\\
-42.725	-32.959\\
-32.959	-18.311\\
-18.311	-13.428\\
-13.428	-13.428\\
-13.428	-19.531\\
-19.531	-26.855\\
-26.855	-31.738\\
-31.738	-26.855\\
-26.855	-19.531\\
-19.531	-13.428\\
-13.428	-13.428\\
-13.428	-13.428\\
-13.428	-15.869\\
-15.869	-9.766\\
-9.766	-9.766\\
-9.766	-8.545\\
-8.545	-4.883\\
-4.883	-10.986\\
-10.986	-14.648\\
-14.648	-15.869\\
-15.869	-15.869\\
-15.869	-17.09\\
-17.09	-18.311\\
-18.311	-23.193\\
-23.193	-14.648\\
-14.648	-21.973\\
-21.973	-26.855\\
-26.855	-21.973\\
-21.973	-23.193\\
-23.193	-21.973\\
-21.973	-20.752\\
-20.752	-29.297\\
-29.297	-23.193\\
-23.193	-17.09\\
-17.09	-20.752\\
-20.752	-20.752\\
-20.752	-13.428\\
-13.428	-20.752\\
-20.752	-26.855\\
-26.855	-28.076\\
-28.076	-29.297\\
-29.297	-45.166\\
-45.166	-30.518\\
-30.518	-26.855\\
-26.855	-18.311\\
-18.311	-26.855\\
-26.855	-35.4\\
-35.4	-24.414\\
-24.414	-19.531\\
-19.531	-26.855\\
-26.855	-20.752\\
-20.752	-10.986\\
-10.986	-15.869\\
-15.869	-13.428\\
-13.428	-8.545\\
-8.545	-8.545\\
-8.545	-8.545\\
-8.545	-7.324\\
-7.324	-10.986\\
-10.986	-14.648\\
-14.648	-17.09\\
-17.09	-23.193\\
-23.193	-21.973\\
-21.973	-25.635\\
-25.635	-28.076\\
-28.076	-28.076\\
-28.076	-19.531\\
-19.531	-21.973\\
-21.973	-19.531\\
-19.531	-20.752\\
-20.752	-29.297\\
-29.297	-23.193\\
-23.193	-26.855\\
-26.855	-23.193\\
-23.193	-19.531\\
-19.531	-29.297\\
-29.297	-25.635\\
-25.635	-26.855\\
-26.855	-24.414\\
-24.414	-15.869\\
-15.869	-9.766\\
-9.766	-8.545\\
-8.545	-10.986\\
-10.986	-12.207\\
-12.207	-18.311\\
-18.311	-13.428\\
-13.428	-21.973\\
-21.973	-31.738\\
-31.738	-36.621\\
-36.621	-26.855\\
-26.855	-29.297\\
-29.297	-26.855\\
-26.855	-18.311\\
-18.311	-21.973\\
-21.973	-23.193\\
-23.193	-19.531\\
-19.531	-23.193\\
-23.193	-26.855\\
-26.855	-39.063\\
-39.063	-34.18\\
-34.18	-31.738\\
-31.738	-37.842\\
-37.842	-28.076\\
-28.076	-30.518\\
-30.518	-24.414\\
-24.414	-23.193\\
-23.193	-30.518\\
-30.518	-34.18\\
-34.18	-9.766\\
-9.766	-20.752\\
-20.752	-14.648\\
-14.648	-23.193\\
-23.193	-20.752\\
-20.752	-13.428\\
-13.428	-17.09\\
-17.09	-26.855\\
-26.855	-43.945\\
-43.945	-42.725\\
-42.725	-37.842\\
-37.842	-32.959\\
-32.959	-39.063\\
-39.063	-45.166\\
-45.166	-32.959\\
-32.959	-34.18\\
-34.18	-36.621\\
-36.621	-25.635\\
-25.635	-21.973\\
-21.973	-26.855\\
-26.855	-19.531\\
-19.531	-13.428\\
-13.428	-19.531\\
-19.531	-13.428\\
-13.428	-19.531\\
-19.531	-15.869\\
-15.869	-19.531\\
-19.531	-24.414\\
-24.414	-21.973\\
-21.973	-15.869\\
-15.869	-19.531\\
-19.531	-21.973\\
-21.973	-23.193\\
-23.193	-19.531\\
-19.531	-15.869\\
-15.869	-26.855\\
-26.855	-26.855\\
-26.855	-18.311\\
-18.311	-24.414\\
-24.414	-21.973\\
-21.973	-18.311\\
-18.311	-13.428\\
-13.428	-8.545\\
-8.545	-4.883\\
-4.883	-18.311\\
-18.311	-13.428\\
-13.428	-12.207\\
-12.207	-7.324\\
-7.324	-12.207\\
-12.207	-17.09\\
-17.09	-20.752\\
-20.752	-23.193\\
-23.193	-28.076\\
-28.076	-26.855\\
-26.855	-26.855\\
-26.855	-18.311\\
-18.311	-7.324\\
-7.324	-15.869\\
-15.869	-10.986\\
-10.986	-14.648\\
-14.648	-20.752\\
-20.752	-20.752\\
-20.752	-36.621\\
-36.621	-28.076\\
-28.076	-25.635\\
-25.635	-34.18\\
-34.18	-23.193\\
-23.193	-13.428\\
-13.428	-13.428\\
-13.428	-17.09\\
-17.09	-20.752\\
-20.752	-30.518\\
-30.518	-21.973\\
-21.973	-19.531\\
-19.531	-17.09\\
-17.09	-20.752\\
-20.752	-25.635\\
-25.635	-40.283\\
-40.283	-32.959\\
-32.959	-32.959\\
-32.959	-21.973\\
-21.973	-14.648\\
-14.648	-14.648\\
-14.648	-7.324\\
-7.324	-9.766\\
-9.766	-8.545\\
-8.545	-13.428\\
-13.428	-23.193\\
-23.193	-31.738\\
-31.738	-31.738\\
-31.738	-43.945\\
-43.945	-35.4\\
-35.4	-20.752\\
-20.752	-21.973\\
-21.973	-19.531\\
-19.531	-21.973\\
-21.973	-29.297\\
-29.297	-36.621\\
-36.621	-26.855\\
-26.855	-19.531\\
-19.531	-21.973\\
-21.973	-29.297\\
-29.297	-19.531\\
-19.531	-14.648\\
-14.648	-13.428\\
-13.428	-9.766\\
-9.766	-7.324\\
-7.324	-6.104\\
-6.104	-12.207\\
-12.207	-17.09\\
-17.09	-18.311\\
-18.311	-10.986\\
-10.986	-12.207\\
-12.207	-6.104\\
-6.104	-2.441\\
-2.441	-6.104\\
-6.104	-10.986\\
-10.986	-13.428\\
-13.428	-17.09\\
-17.09	-19.531\\
-19.531	-23.193\\
-23.193	-29.297\\
-29.297	-40.283\\
-40.283	-40.283\\
-40.283	-42.725\\
-42.725	-37.842\\
-37.842	-21.973\\
-21.973	-20.752\\
-20.752	-21.973\\
-21.973	-28.076\\
-28.076	-21.973\\
-21.973	-17.09\\
-17.09	-12.207\\
-12.207	-12.207\\
-12.207	-19.531\\
-19.531	-17.09\\
-17.09	-8.545\\
-8.545	-6.104\\
-6.104	-10.986\\
-10.986	-14.648\\
-14.648	-9.766\\
-9.766	-18.311\\
-18.311	-12.207\\
-12.207	-21.973\\
-21.973	-29.297\\
-29.297	-21.973\\
-21.973	-29.297\\
-29.297	-40.283\\
-40.283	-42.725\\
-42.725	-31.738\\
-31.738	-24.414\\
-24.414	-17.09\\
-17.09	-10.986\\
-10.986	-18.311\\
-18.311	-9.766\\
-9.766	-20.752\\
-20.752	-14.648\\
-14.648	-14.648\\
-14.648	-19.531\\
-19.531	-12.207\\
-12.207	-20.752\\
-20.752	-13.428\\
-13.428	-9.766\\
-9.766	-10.986\\
-10.986	-7.324\\
-7.324	-8.545\\
-8.545	-6.104\\
-6.104	-7.324\\
-7.324	-6.104\\
-6.104	-9.766\\
-9.766	-9.766\\
-9.766	-9.766\\
-9.766	-14.648\\
-14.648	-20.752\\
-20.752	-15.869\\
-15.869	-15.869\\
-15.869	-25.635\\
-25.635	-30.518\\
-30.518	-21.973\\
-21.973	-25.635\\
-25.635	-9.766\\
-9.766	-7.324\\
-7.324	-12.207\\
-12.207	-20.752\\
-20.752	-20.752\\
-20.752	-30.518\\
-30.518	-29.297\\
-29.297	-19.531\\
-19.531	-14.648\\
-14.648	-18.311\\
-18.311	-17.09\\
-17.09	-14.648\\
-14.648	-7.324\\
-7.324	-13.428\\
-13.428	-10.986\\
-10.986	-7.324\\
-7.324	-7.324\\
-7.324	-12.207\\
-12.207	-14.648\\
-14.648	-15.869\\
-15.869	-8.545\\
-8.545	-19.531\\
-19.531	-25.635\\
-25.635	-17.09\\
-17.09	-25.635\\
-25.635	-25.635\\
-25.635	-19.531\\
-19.531	-12.207\\
-12.207	-19.531\\
-19.531	-17.09\\
-17.09	-9.766\\
-9.766	-14.648\\
-14.648	-7.324\\
-7.324	-4.883\\
-4.883	-4.883\\
-4.883	-8.545\\
-8.545	-13.428\\
-13.428	-12.207\\
-12.207	-13.428\\
-13.428	-15.869\\
-15.869	-9.766\\
-9.766	-7.324\\
-7.324	-6.104\\
-6.104	-7.324\\
-7.324	-9.766\\
-9.766	-9.766\\
-9.766	-15.869\\
-15.869	-18.311\\
-18.311	-15.869\\
-15.869	-17.09\\
-17.09	-21.973\\
-21.973	-28.076\\
-28.076	-30.518\\
-30.518	-30.518\\
-30.518	-21.973\\
-21.973	-24.414\\
-24.414	-24.414\\
-24.414	-25.635\\
-25.635	-29.297\\
-29.297	-37.842\\
-37.842	-32.959\\
-32.959	-29.297\\
-29.297	-24.414\\
-24.414	-24.414\\
-24.414	-23.193\\
-23.193	-29.297\\
-29.297	-17.09\\
-17.09	-21.973\\
-21.973	-23.193\\
-23.193	-26.855\\
-26.855	-20.752\\
-20.752	-25.635\\
-25.635	-18.311\\
-18.311	-18.311\\
-18.311	-12.207\\
-12.207	-20.752\\
-20.752	-28.076\\
-28.076	-20.752\\
-20.752	-13.428\\
-13.428	-13.428\\
-13.428	-9.766\\
-9.766	-7.324\\
-7.324	-9.766\\
-9.766	-4.883\\
-4.883	-7.324\\
-7.324	-8.545\\
-8.545	-4.883\\
-4.883	-9.766\\
-9.766	-7.324\\
-7.324	-4.883\\
-4.883	-7.324\\
-7.324	-12.207\\
-12.207	-10.986\\
-10.986	-13.428\\
-13.428	-13.428\\
-13.428	-15.869\\
-15.869	-13.428\\
-13.428	-17.09\\
-17.09	-29.297\\
-29.297	-20.752\\
-20.752	-13.428\\
-13.428	-19.531\\
-19.531	-13.428\\
-13.428	-8.545\\
-8.545	-15.869\\
-15.869	-8.545\\
-8.545	-6.104\\
-6.104	-12.207\\
-12.207	-15.869\\
-15.869	-15.869\\
-15.869	-9.766\\
-9.766	-6.104\\
-6.104	-6.104\\
-6.104	-6.104\\
-6.104	-12.207\\
-12.207	-12.207\\
-12.207	-12.207\\
-12.207	-18.311\\
-18.311	-24.414\\
-24.414	-15.869\\
-15.869	-23.193\\
-23.193	-28.076\\
-28.076	-25.635\\
-25.635	-21.973\\
-21.973	-36.621\\
-36.621	-30.518\\
-30.518	-32.959\\
-32.959	-28.076\\
-28.076	-35.4\\
-35.4	-37.842\\
-37.842	-25.635\\
-25.635	-13.428\\
-13.428	-13.428\\
-13.428	-18.311\\
-18.311	-21.973\\
-21.973	-23.193\\
-23.193	-14.648\\
-14.648	-8.545\\
-8.545	-13.428\\
-13.428	-21.973\\
-21.973	-28.076\\
-28.076	-25.635\\
-25.635	-15.869\\
-15.869	-13.428\\
-13.428	-17.09\\
-17.09	-26.855\\
-26.855	-30.518\\
-30.518	-30.518\\
-30.518	-23.193\\
-23.193	-31.738\\
-31.738	-34.18\\
-34.18	-30.518\\
-30.518	-42.725\\
-42.725	-36.621\\
-36.621	-30.518\\
-30.518	-37.842\\
-37.842	-36.621\\
-36.621	-18.311\\
-18.311	-17.09\\
-17.09	-20.752\\
-20.752	-25.635\\
-25.635	-26.855\\
-26.855	-18.311\\
-18.311	-24.414\\
-24.414	-21.973\\
-21.973	-14.648\\
-14.648	-19.531\\
-19.531	-18.311\\
-18.311	-26.855\\
-26.855	-30.518\\
-30.518	-23.193\\
-23.193	-17.09\\
-17.09	-10.986\\
-10.986	-15.869\\
-15.869	-12.207\\
-12.207	-12.207\\
-12.207	-7.324\\
-7.324	-15.869\\
-15.869	-19.531\\
-19.531	-23.193\\
-23.193	-21.973\\
-21.973	-13.428\\
-13.428	-9.766\\
-9.766	-7.324\\
-7.324	-9.766\\
-9.766	-14.648\\
-14.648	-17.09\\
-17.09	-14.648\\
-14.648	-24.414\\
-24.414	-26.855\\
-26.855	-30.518\\
-30.518	-30.518\\
-30.518	-35.4\\
-35.4	-23.193\\
-23.193	-19.531\\
-19.531	-14.648\\
-14.648	-17.09\\
-17.09	-20.752\\
-20.752	-15.869\\
-15.869	-10.986\\
-10.986	-10.986\\
-10.986	-18.311\\
-18.311	-23.193\\
-23.193	-20.752\\
-20.752	-31.738\\
-31.738	-36.621\\
-36.621	-25.635\\
-25.635	-18.311\\
-18.311	-13.428\\
-13.428	-10.986\\
-10.986	-20.752\\
-20.752	-14.648\\
-14.648	-17.09\\
-17.09	-23.193\\
-23.193	-25.635\\
-25.635	-24.414\\
-24.414	-28.076\\
-28.076	-18.311\\
-18.311	-21.973\\
-21.973	-15.869\\
-15.869	-14.648\\
-14.648	-20.752\\
-20.752	-28.076\\
-28.076	-30.518\\
-30.518	-18.311\\
-18.311	-29.297\\
-29.297	-30.518\\
-30.518	-21.973\\
-21.973	-14.648\\
-14.648	-21.973\\
-21.973	-17.09\\
-17.09	-17.09\\
-17.09	-19.531\\
-19.531	-13.428\\
-13.428	-18.311\\
-18.311	-34.18\\
-34.18	-36.621\\
-36.621	-25.635\\
-25.635	-28.076\\
-28.076	-28.076\\
-28.076	-28.076\\
-28.076	-20.752\\
-20.752	-20.752\\
-20.752	-14.648\\
-14.648	-20.752\\
-20.752	-20.752\\
-20.752	-32.959\\
-32.959	-37.842\\
-37.842	-47.607\\
-47.607	-37.842\\
-37.842	-30.518\\
-30.518	-23.193\\
-23.193	-25.635\\
-25.635	-19.531\\
-19.531	-14.648\\
-14.648	-13.428\\
-13.428	-15.869\\
-15.869	-10.986\\
-10.986	-7.324\\
-7.324	-12.207\\
-12.207	-17.09\\
-17.09	-12.207\\
-12.207	-7.324\\
-7.324	-7.324\\
-7.324	-18.311\\
-18.311	-25.635\\
-25.635	-12.207\\
-12.207	-12.207\\
-12.207	-15.869\\
-15.869	-14.648\\
-14.648	-18.311\\
-18.311	-17.09\\
-17.09	-10.986\\
-10.986	-8.545\\
-8.545	-7.324\\
-7.324	-9.766\\
-9.766	-18.311\\
-18.311	-18.311\\
-18.311	-15.869\\
-15.869	-10.986\\
-10.986	-7.324\\
-7.324	-8.545\\
-8.545	-17.09\\
-17.09	-19.531\\
-19.531	-19.531\\
-19.531	-15.869\\
-15.869	-13.428\\
-13.428	-14.648\\
-14.648	-20.752\\
-20.752	-28.076\\
-28.076	-32.959\\
-32.959	-24.414\\
-24.414	-17.09\\
-17.09	-19.531\\
-19.531	-18.311\\
-18.311	-17.09\\
-17.09	-12.207\\
-12.207	-13.428\\
-13.428	-15.869\\
-15.869	-14.648\\
-14.648	-10.986\\
-10.986	-6.104\\
-6.104	-8.545\\
-8.545	-13.428\\
-13.428	-20.752\\
-20.752	-24.414\\
-24.414	-28.076\\
-28.076	-25.635\\
-25.635	-29.297\\
-29.297	-36.621\\
-36.621	-47.607\\
-47.607	-37.842\\
-37.842	-28.076\\
-28.076	-31.738\\
-31.738	-36.621\\
-36.621	-43.945\\
-43.945	-30.518\\
-30.518	-13.428\\
-13.428	-9.766\\
-9.766	-10.986\\
-10.986	-4.883\\
-4.883	-4.883\\
-4.883	-6.104\\
-6.104	-9.766\\
-9.766	-13.428\\
-13.428	-14.648\\
-14.648	-10.986\\
-10.986	-8.545\\
-8.545	-13.428\\
-13.428	-15.869\\
-15.869	-17.09\\
-17.09	-15.869\\
-15.869	-13.428\\
-13.428	-8.545\\
-8.545	-8.545\\
-8.545	-10.986\\
-10.986	-14.648\\
-14.648	-10.986\\
-10.986	-7.324\\
-7.324	-10.986\\
-10.986	-10.986\\
-10.986	-8.545\\
-8.545	-13.428\\
-13.428	-18.311\\
-18.311	-18.311\\
-18.311	-13.428\\
-13.428	-19.531\\
-19.531	-20.752\\
-20.752	-17.09\\
-17.09	-14.648\\
-14.648	-21.973\\
-21.973	-26.855\\
-26.855	-29.297\\
-29.297	-29.297\\
-29.297	-24.414\\
-24.414	-21.973\\
-21.973	-20.752\\
-20.752	-28.076\\
-28.076	-21.973\\
-21.973	-26.855\\
-26.855	-28.076\\
-28.076	-26.855\\
-26.855	-47.607\\
-47.607	-37.842\\
-37.842	-20.752\\
-20.752	-14.648\\
-14.648	-14.648\\
-14.648	-18.311\\
-18.311	-24.414\\
-24.414	-26.855\\
-26.855	-29.297\\
-29.297	-31.738\\
-31.738	-20.752\\
-20.752	-19.531\\
-19.531	-20.752\\
-20.752	-20.752\\
-20.752	-14.648\\
-14.648	-12.207\\
-12.207	-17.09\\
-17.09	-17.09\\
-17.09	-21.973\\
-21.973	-28.076\\
-28.076	-25.635\\
-25.635	-18.311\\
-18.311	-14.648\\
-14.648	-21.973\\
-21.973	-28.076\\
-28.076	-34.18\\
-34.18	-37.842\\
-37.842	-42.725\\
-42.725	-35.4\\
-35.4	-32.959\\
-32.959	-20.752\\
-20.752	-12.207\\
-12.207	-24.414\\
-24.414	-32.959\\
-32.959	-19.531\\
-19.531	-25.635\\
-25.635	-36.621\\
-36.621	-45.166\\
-45.166	-31.738\\
-31.738	-18.311\\
-18.311	-12.207\\
-12.207	-15.869\\
-15.869	-14.648\\
-14.648	-13.428\\
-13.428	-10.986\\
-10.986	-12.207\\
-12.207	-9.766\\
-9.766	-12.207\\
-12.207	-10.986\\
-10.986	-13.428\\
-13.428	-20.752\\
-20.752	-20.752\\
-20.752	-24.414\\
-24.414	-30.518\\
-30.518	-31.738\\
-31.738	-19.531\\
-19.531	-17.09\\
-17.09	-21.973\\
-21.973	-14.648\\
-14.648	-13.428\\
-13.428	-10.986\\
-10.986	-14.648\\
-14.648	-14.648\\
-14.648	-8.545\\
-8.545	-4.883\\
-4.883	-6.104\\
-6.104	-8.545\\
-8.545	-14.648\\
-14.648	-14.648\\
-14.648	-9.766\\
-9.766	-13.428\\
-13.428	-17.09\\
-17.09	-20.752\\
-20.752	-17.09\\
-17.09	-14.648\\
-14.648	-17.09\\
-17.09	-20.752\\
-20.752	-18.311\\
-18.311	-20.752\\
-20.752	-20.752\\
-20.752	-13.428\\
-13.428	-12.207\\
-12.207	-15.869\\
-15.869	-13.428\\
-13.428	-14.648\\
-14.648	-20.752\\
-20.752	-30.518\\
-30.518	-24.414\\
-24.414	-20.752\\
-20.752	-31.738\\
-31.738	-23.193\\
-23.193	-15.869\\
-15.869	-14.648\\
-14.648	-9.766\\
-9.766	-9.766\\
-9.766	-7.324\\
-7.324	-7.324\\
-7.324	-17.09\\
-17.09	-13.428\\
-13.428	-7.324\\
-7.324	-13.428\\
-13.428	-14.648\\
-14.648	-10.986\\
-10.986	-9.766\\
-9.766	-4.883\\
-4.883	-6.104\\
-6.104	-17.09\\
-17.09	-20.752\\
-20.752	-15.869\\
-15.869	-12.207\\
-12.207	-17.09\\
-17.09	-18.311\\
-18.311	-20.752\\
-20.752	-21.973\\
-21.973	-26.855\\
-26.855	-26.855\\
-26.855	-21.973\\
-21.973	-15.869\\
-15.869	-12.207\\
-12.207	-13.428\\
-13.428	-15.869\\
-15.869	-13.428\\
-13.428	-8.545\\
-8.545	-18.311\\
-18.311	-23.193\\
-23.193	-20.752\\
-20.752	-24.414\\
-24.414	-30.518\\
-30.518	-20.752\\
-20.752	-23.193\\
-23.193	-30.518\\
-30.518	-24.414\\
-24.414	-29.297\\
-29.297	-29.297\\
-29.297	-45.166\\
-45.166	-42.725\\
-42.725	-30.518\\
-30.518	-32.959\\
-32.959	-42.725\\
-42.725	-41.504\\
-41.504	-46.387\\
-46.387	-43.945\\
-43.945	-53.711\\
-53.711	-46.387\\
-46.387	-29.297\\
-29.297	-19.531\\
-19.531	-20.752\\
-20.752	-14.648\\
-14.648	-14.648\\
-14.648	-19.531\\
-19.531	-29.297\\
-29.297	-20.752\\
-20.752	-14.648\\
-14.648	-17.09\\
-17.09	-12.207\\
-12.207	-12.207\\
-12.207	-3.662\\
-3.662	-9.766\\
-9.766	-9.766\\
-9.766	-9.766\\
-9.766	-10.986\\
-10.986	-6.104\\
-6.104	-4.883\\
-4.883	-3.662\\
-3.662	-7.324\\
-7.324	-7.324\\
-7.324	-6.104\\
-6.104	-13.428\\
-13.428	-18.311\\
-18.311	-17.09\\
-17.09	-13.428\\
-13.428	-15.869\\
-15.869	-26.855\\
-26.855	-18.311\\
-18.311	-4.883\\
-4.883	-7.324\\
-7.324	-4.883\\
-4.883	-4.883\\
-4.883	-15.869\\
-15.869	-18.311\\
-18.311	-12.207\\
-12.207	-18.311\\
-18.311	-29.297\\
-29.297	-25.635\\
-25.635	-18.311\\
-18.311	-13.428\\
-13.428	-14.648\\
-14.648	-23.193\\
-23.193	-26.855\\
-26.855	-18.311\\
-18.311	-10.986\\
-10.986	-12.207\\
-12.207	-12.207\\
-12.207	-4.883\\
-4.883	-3.662\\
-3.662	-7.324\\
-7.324	-20.752\\
-20.752	-20.752\\
-20.752	-13.428\\
-13.428	-10.986\\
-10.986	-10.986\\
-10.986	-10.986\\
-10.986	-4.883\\
-4.883	-2.441\\
-2.441	-7.324\\
-7.324	-15.869\\
-15.869	-21.973\\
-21.973	-23.193\\
-23.193	-21.973\\
-21.973	-23.193\\
-23.193	-23.193\\
-23.193	-24.414\\
-24.414	-19.531\\
-19.531	-25.635\\
-25.635	-35.4\\
-35.4	-34.18\\
-34.18	-18.311\\
-18.311	-8.545\\
-8.545	-14.648\\
-14.648	-28.076\\
-28.076	-21.973\\
-21.973	-15.869\\
-15.869	-13.428\\
-13.428	-24.414\\
-24.414	-34.18\\
-34.18	-37.842\\
-37.842	-37.842\\
-37.842	-39.063\\
-39.063	-29.297\\
-29.297	-28.076\\
-28.076	-18.311\\
-18.311	-12.207\\
-12.207	-18.311\\
-18.311	-19.531\\
-19.531	-20.752\\
-20.752	-23.193\\
-23.193	-15.869\\
-15.869	-17.09\\
-17.09	-20.752\\
-20.752	-9.766\\
-9.766	-6.104\\
-6.104	-6.104\\
-6.104	-8.545\\
-8.545	-6.104\\
-6.104	-1.221\\
-1.221	-4.883\\
-4.883	-17.09\\
-17.09	-13.428\\
-13.428	-10.986\\
-10.986	-15.869\\
-15.869	-24.414\\
-24.414	-20.752\\
-20.752	-7.324\\
-7.324	-6.104\\
-6.104	-8.545\\
-8.545	-26.855\\
-26.855	-31.738\\
-31.738	-41.504\\
-41.504	-50.049\\
-50.049	-47.607\\
-47.607	-34.18\\
-34.18	-34.18\\
-34.18	-43.945\\
-43.945	-36.621\\
-36.621	-32.959\\
-32.959	-37.842\\
-37.842	-41.504\\
-41.504	-41.504\\
-41.504	-56.152\\
-56.152	-41.504\\
-41.504	-21.973\\
-21.973	-13.428\\
-13.428	-9.766\\
-9.766	-12.207\\
-12.207	-12.207\\
-12.207	-10.986\\
-10.986	-20.752\\
-20.752	-18.311\\
-18.311	-19.531\\
-19.531	-20.752\\
-20.752	-14.648\\
-14.648	-13.428\\
-13.428	-21.973\\
-21.973	-23.193\\
-23.193	-13.428\\
-13.428	-7.324\\
-7.324	-9.766\\
-9.766	-12.207\\
-12.207	-29.297\\
-29.297	-35.4\\
-35.4	-32.959\\
-32.959	-35.4\\
-35.4	-43.945\\
-43.945	-52.49\\
-52.49	-46.387\\
-46.387	-30.518\\
-30.518	-19.531\\
-19.531	-13.428\\
-13.428	-13.428\\
-13.428	-14.648\\
-14.648	-17.09\\
-17.09	-9.766\\
-9.766	-12.207\\
-12.207	-13.428\\
-13.428	-15.869\\
-15.869	-19.531\\
-19.531	-20.752\\
-20.752	-18.311\\
-18.311	-8.545\\
-8.545	-4.883\\
-4.883	-4.883\\
-4.883	-3.662\\
-3.662	-6.104\\
-6.104	-10.986\\
-10.986	-14.648\\
-14.648	-13.428\\
-13.428	-23.193\\
-23.193	-18.311\\
-18.311	-23.193\\
-23.193	-40.283\\
-40.283	-30.518\\
-30.518	-25.635\\
-25.635	-32.959\\
-32.959	-36.621\\
-36.621	-26.855\\
-26.855	-21.973\\
-21.973	-15.869\\
-15.869	-15.869\\
-15.869	-30.518\\
-30.518	-48.828\\
-48.828	-57.373\\
-57.373	-46.387\\
-46.387	-39.063\\
-39.063	-30.518\\
-30.518	-30.518\\
-30.518	-37.842\\
-37.842	-28.076\\
-28.076	-19.531\\
-19.531	-19.531\\
-19.531	-25.635\\
-25.635	-29.297\\
-29.297	-15.869\\
-15.869	-14.648\\
-14.648	-14.648\\
-14.648	-24.414\\
-24.414	-28.076\\
-28.076	-29.297\\
-29.297	-20.752\\
-20.752	-17.09\\
-17.09	-17.09\\
-17.09	-15.869\\
-15.869	-10.986\\
-10.986	-9.766\\
-9.766	-8.545\\
-8.545	-6.104\\
-6.104	-7.324\\
-7.324	-12.207\\
-12.207	-18.311\\
-18.311	-18.311\\
-18.311	-13.428\\
-13.428	-9.766\\
-9.766	-8.545\\
-8.545	-10.986\\
-10.986	-15.869\\
-15.869	-18.311\\
-18.311	-15.869\\
-15.869	-10.986\\
-10.986	-24.414\\
-24.414	-26.855\\
-26.855	-20.752\\
-20.752	-8.545\\
-8.545	-13.428\\
-13.428	-17.09\\
-17.09	-17.09\\
-17.09	-12.207\\
-12.207	-8.545\\
-8.545	-12.207\\
-12.207	-26.855\\
-26.855	-13.428\\
-13.428	-7.324\\
-7.324	-9.766\\
-9.766	-18.311\\
-18.311	-21.973\\
-21.973	-12.207\\
-12.207	-10.986\\
-10.986	-20.752\\
-20.752	-29.297\\
-29.297	-25.635\\
-25.635	-37.842\\
-37.842	-40.283\\
-40.283	-30.518\\
-30.518	-23.193\\
-23.193	-30.518\\
-30.518	-25.635\\
-25.635	-26.855\\
-26.855	-36.621\\
-36.621	-28.076\\
-28.076	-14.648\\
-14.648	-25.635\\
-25.635	-36.621\\
-36.621	-41.504\\
-41.504	-31.738\\
-31.738	-28.076\\
-28.076	-34.18\\
-34.18	-30.518\\
-30.518	-17.09\\
-17.09	-17.09\\
-17.09	-21.973\\
-21.973	-23.193\\
-23.193	-20.752\\
-20.752	-24.414\\
-24.414	-15.869\\
-15.869	-19.531\\
-19.531	-28.076\\
-28.076	-19.531\\
-19.531	-13.428\\
-13.428	-10.986\\
-10.986	-8.545\\
-8.545	-8.545\\
-8.545	-17.09\\
-17.09	-18.311\\
-18.311	-19.531\\
-19.531	-26.855\\
-26.855	-31.738\\
-31.738	-28.076\\
-28.076	-21.973\\
-21.973	-13.428\\
-13.428	-15.869\\
-15.869	-29.297\\
-29.297	-20.752\\
-20.752	-7.324\\
-7.324	-17.09\\
-17.09	-25.635\\
-25.635	-28.076\\
-28.076	-35.4\\
-35.4	-46.387\\
-46.387	-36.621\\
-36.621	-29.297\\
-29.297	-34.18\\
-34.18	-25.635\\
-25.635	-18.311\\
-18.311	-21.973\\
-21.973	-29.297\\
-29.297	-28.076\\
-28.076	-28.076\\
-28.076	-18.311\\
-18.311	-14.648\\
-14.648	-12.207\\
-12.207	-17.09\\
-17.09	-18.311\\
-18.311	-13.428\\
-13.428	-24.414\\
-24.414	-30.518\\
-30.518	-18.311\\
-18.311	-13.428\\
-13.428	-23.193\\
-23.193	-14.648\\
-14.648	-6.104\\
-6.104	-12.207\\
-12.207	-13.428\\
-13.428	-10.986\\
-10.986	-7.324\\
-7.324	-6.104\\
-6.104	-7.324\\
-7.324	-10.986\\
-10.986	-20.752\\
-20.752	-25.635\\
-25.635	-29.297\\
-29.297	-32.959\\
-32.959	-18.311\\
-18.311	-12.207\\
-12.207	-28.076\\
-28.076	-41.504\\
-41.504	-31.738\\
-31.738	-29.297\\
-29.297	-45.166\\
-45.166	-34.18\\
-34.18	-30.518\\
-30.518	-25.635\\
-25.635	-23.193\\
-23.193	-31.738\\
-31.738	-24.414\\
-24.414	-20.752\\
-20.752	-30.518\\
-30.518	-40.283\\
-40.283	-40.283\\
-40.283	-15.869\\
-15.869	-9.766\\
-9.766	-28.076\\
-28.076	-20.752\\
-20.752	-8.545\\
-8.545	-8.545\\
-8.545	-14.648\\
-14.648	-19.531\\
-19.531	-28.076\\
-28.076	-18.311\\
-18.311	-13.428\\
-13.428	-10.986\\
-10.986	-6.104\\
-6.104	-8.545\\
-8.545	-6.104\\
-6.104	-7.324\\
-7.324	-17.09\\
-17.09	-14.648\\
-14.648	-6.104\\
-6.104	-6.104\\
-6.104	-8.545\\
-8.545	-10.986\\
-10.986	-8.545\\
-8.545	-9.766\\
-9.766	-9.766\\
-9.766	-2.441\\
-2.441	-12.207\\
-12.207	-18.311\\
-18.311	-24.414\\
-24.414	-34.18\\
-34.18	-30.518\\
-30.518	-14.648\\
-14.648	-12.207\\
-12.207	-12.207\\
-12.207	-3.662\\
-3.662	-4.883\\
-4.883	-13.428\\
-13.428	-18.311\\
-18.311	-14.648\\
-14.648	-14.648\\
-14.648	-18.311\\
-18.311	-19.531\\
-19.531	-19.531\\
-19.531	-25.635\\
-25.635	-20.752\\
-20.752	-34.18\\
-34.18	-18.311\\
-18.311	-12.207\\
-12.207	-8.545\\
-8.545	-15.869\\
-15.869	-13.428\\
-13.428	-9.766\\
-9.766	-3.662\\
-3.662	-10.986\\
-10.986	-8.545\\
-8.545	-2.441\\
-2.441	-7.324\\
-7.324	-9.766\\
-9.766	-14.648\\
-14.648	-19.531\\
-19.531	-26.855\\
-26.855	-21.973\\
-21.973	-7.324\\
-7.324	-13.428\\
-13.428	-21.973\\
-21.973	-10.986\\
-10.986	-8.545\\
-8.545	-21.973\\
-21.973	-18.311\\
-18.311	-28.076\\
-28.076	-36.621\\
-36.621	-23.193\\
-23.193	-18.311\\
-18.311	-15.869\\
};
\addlegendentry{data1}

\addplot [color=mycolor2, line width=2.0pt]
  table[row sep=crcr]{%
-19.531	-18.7469676958217\\
-23.193	-22.2619641477237\\
-28.076	-26.9489460359372\\
-26.855	-25.7769605996258\\
-20.752	-19.9189531321331\\
-29.297	-28.1209314722486\\
-28.076	-26.9489460359372\\
-32.959	-31.6359279241507\\
-25.635	-24.6059350203466\\
-14.648	-14.0599858076082\\
-20.752	-19.9189531321331\\
-17.09	-16.4039566802311\\
-18.311	-17.5759421165425\\
-20.752	-19.9189531321331\\
-7.324	-7.02999290380411\\
-4.883	-4.68698188821347\\
-13.428	-12.888960228329\\
-23.193	-22.2619641477237\\
-24.414	-23.4339495840352\\
-25.635	-24.6059350203466\\
-19.531	-18.7469676958217\\
-12.207	-11.7169747920176\\
-24.414	-23.4339495840352\\
-19.531	-18.7469676958217\\
-14.648	-14.0599858076082\\
-20.752	-19.9189531321331\\
-18.311	-17.5759421165425\\
-17.09	-16.4039566802311\\
-29.297	-28.1209314722486\\
-25.635	-24.6059350203466\\
-18.311	-17.5759421165425\\
-28.076	-26.9489460359372\\
-21.973	-21.0909385684445\\
-18.311	-17.5759421165425\\
-15.869	-15.2319712439196\\
-21.973	-21.0909385684445\\
-23.193	-22.2619641477237\\
-30.518	-29.2929169085601\\
-29.297	-28.1209314722486\\
-19.531	-18.7469676958217\\
-18.311	-17.5759421165425\\
-10.986	-10.5449893557062\\
-14.648	-14.0599858076082\\
-15.869	-15.2319712439196\\
-12.207	-11.7169747920176\\
-13.428	-12.888960228329\\
-18.311	-17.5759421165425\\
-19.531	-18.7469676958217\\
-18.311	-17.5759421165425\\
-29.297	-28.1209314722486\\
-21.973	-21.0909385684445\\
-12.207	-11.7169747920176\\
-13.428	-12.888960228329\\
-17.09	-16.4039566802311\\
-9.766	-9.37396377642694\\
-10.986	-10.5449893557062\\
-14.648	-14.0599858076082\\
-9.766	-9.37396377642694\\
-7.324	-7.02999290380411\\
-13.428	-12.888960228329\\
-21.973	-21.0909385684445\\
-20.752	-19.9189531321331\\
-25.635	-24.6059350203466\\
-21.973	-21.0909385684445\\
-25.635	-24.6059350203466\\
-20.752	-19.9189531321331\\
-18.311	-17.5759421165425\\
-17.09	-16.4039566802311\\
-30.518	-29.2929169085601\\
-41.504	-39.8379062642662\\
-42.725	-41.0098917005776\\
-28.076	-26.9489460359372\\
-37.842	-36.3229098123642\\
-46.387	-44.5248881524797\\
-35.4	-33.9789389397413\\
-47.607	-45.6959137317589\\
-58.594	-56.2418629444973\\
-43.945	-42.1809172798568\\
-34.18	-32.8079133604621\\
-24.414	-23.4339495840352\\
-20.752	-19.9189531321331\\
-17.09	-16.4039566802311\\
-21.973	-21.0909385684445\\
-17.09	-16.4039566802311\\
-12.207	-11.7169747920176\\
-9.766	-9.37396377642694\\
-12.207	-11.7169747920176\\
-18.311	-17.5759421165425\\
-21.973	-21.0909385684445\\
-23.193	-22.2619641477237\\
-30.518	-29.2929169085601\\
-28.076	-26.9489460359372\\
-30.518	-29.2929169085601\\
-25.635	-24.6059350203466\\
-20.752	-19.9189531321331\\
-24.414	-23.4339495840352\\
-18.311	-17.5759421165425\\
-13.428	-12.888960228329\\
-19.531	-18.7469676958217\\
-20.752	-19.9189531321331\\
-12.207	-11.7169747920176\\
-15.869	-15.2319712439196\\
-12.207	-11.7169747920176\\
-13.428	-12.888960228329\\
-9.766	-9.37396377642694\\
-10.986	-10.5449893557062\\
-17.09	-16.4039566802311\\
-23.193	-22.2619641477237\\
-25.635	-24.6059350203466\\
-26.855	-25.7769605996258\\
-15.869	-15.2319712439196\\
-20.752	-19.9189531321331\\
-32.959	-31.6359279241507\\
-24.414	-23.4339495840352\\
-28.076	-26.9489460359372\\
-29.297	-28.1209314722486\\
-18.311	-17.5759421165425\\
-12.207	-11.7169747920176\\
-17.09	-16.4039566802311\\
-18.311	-17.5759421165425\\
-30.518	-29.2929169085601\\
-36.621	-35.1509243760527\\
-28.076	-26.9489460359372\\
-25.635	-24.6059350203466\\
-20.752	-19.9189531321331\\
-15.869	-15.2319712439196\\
-14.648	-14.0599858076082\\
-13.428	-12.888960228329\\
-20.752	-19.9189531321331\\
-31.738	-30.4639424878393\\
-26.855	-25.7769605996258\\
-17.09	-16.4039566802311\\
-15.869	-15.2319712439196\\
-9.766	-9.37396377642694\\
-7.324	-7.02999290380411\\
-15.869	-15.2319712439196\\
-10.986	-10.5449893557062\\
-14.648	-14.0599858076082\\
-13.428	-12.888960228329\\
-9.766	-9.37396377642694\\
-6.104	-5.85896732452489\\
-4.883	-4.68698188821347\\
-7.324	-7.02999290380411\\
-18.311	-17.5759421165425\\
-26.855	-25.7769605996258\\
-28.076	-26.9489460359372\\
-20.752	-19.9189531321331\\
-13.428	-12.888960228329\\
-10.986	-10.5449893557062\\
-6.104	-5.85896732452489\\
-12.207	-11.7169747920176\\
-14.648	-14.0599858076082\\
-18.311	-17.5759421165425\\
-26.855	-25.7769605996258\\
-39.063	-37.4948952486756\\
-47.607	-45.6959137317589\\
-42.725	-41.0098917005776\\
-37.842	-36.3229098123642\\
-28.076	-26.9489460359372\\
-30.518	-29.2929169085601\\
-31.738	-30.4639424878393\\
-34.18	-32.8079133604621\\
-24.414	-23.4339495840352\\
-23.193	-22.2619641477237\\
-21.973	-21.0909385684445\\
-23.193	-22.2619641477237\\
-18.311	-17.5759421165425\\
-30.518	-29.2929169085601\\
-37.842	-36.3229098123642\\
-28.076	-26.9489460359372\\
-18.311	-17.5759421165425\\
-30.518	-29.2929169085601\\
-35.4	-33.9789389397413\\
-40.283	-38.6659208279548\\
-45.166	-43.3529027161683\\
-34.18	-32.8079133604621\\
-32.959	-31.6359279241507\\
-36.621	-35.1509243760527\\
-41.504	-39.8379062642662\\
-29.297	-28.1209314722486\\
-17.09	-16.4039566802311\\
-23.193	-22.2619641477237\\
-20.752	-19.9189531321331\\
-13.428	-12.888960228329\\
-18.311	-17.5759421165425\\
-19.531	-18.7469676958217\\
-14.648	-14.0599858076082\\
-13.428	-12.888960228329\\
-15.869	-15.2319712439196\\
-13.428	-12.888960228329\\
-17.09	-16.4039566802311\\
-25.635	-24.6059350203466\\
-15.869	-15.2319712439196\\
-14.648	-14.0599858076082\\
-24.414	-23.4339495840352\\
-40.283	-38.6659208279548\\
-29.297	-28.1209314722486\\
-20.752	-19.9189531321331\\
-14.648	-14.0599858076082\\
-18.311	-17.5759421165425\\
-15.869	-15.2319712439196\\
-12.207	-11.7169747920176\\
-13.428	-12.888960228329\\
-15.869	-15.2319712439196\\
-18.311	-17.5759421165425\\
-20.752	-19.9189531321331\\
-15.869	-15.2319712439196\\
-23.193	-22.2619641477237\\
-29.297	-28.1209314722486\\
-25.635	-24.6059350203466\\
-20.752	-19.9189531321331\\
-23.193	-22.2619641477237\\
-30.518	-29.2929169085601\\
-21.973	-21.0909385684445\\
-8.545	-8.20197834011553\\
-13.428	-12.888960228329\\
-17.09	-16.4039566802311\\
-13.428	-12.888960228329\\
-17.09	-16.4039566802311\\
-19.531	-18.7469676958217\\
-13.428	-12.888960228329\\
-15.869	-15.2319712439196\\
-12.207	-11.7169747920176\\
-15.869	-15.2319712439196\\
-9.766	-9.37396377642694\\
-10.986	-10.5449893557062\\
-8.545	-8.20197834011553\\
-14.648	-14.0599858076082\\
-20.752	-19.9189531321331\\
-24.414	-23.4339495840352\\
-35.4	-33.9789389397413\\
-24.414	-23.4339495840352\\
-13.428	-12.888960228329\\
-12.207	-11.7169747920176\\
-10.986	-10.5449893557062\\
-13.428	-12.888960228329\\
-10.986	-10.5449893557062\\
-12.207	-11.7169747920176\\
-14.648	-14.0599858076082\\
-24.414	-23.4339495840352\\
-21.973	-21.0909385684445\\
-26.855	-25.7769605996258\\
-23.193	-22.2619641477237\\
-17.09	-16.4039566802311\\
-13.428	-12.888960228329\\
-10.986	-10.5449893557062\\
-8.545	-8.20197834011553\\
-6.104	-5.85896732452489\\
-9.766	-9.37396377642694\\
-14.648	-14.0599858076082\\
-18.311	-17.5759421165425\\
-19.531	-18.7469676958217\\
-29.297	-28.1209314722486\\
-25.635	-24.6059350203466\\
-18.311	-17.5759421165425\\
-24.414	-23.4339495840352\\
-31.738	-30.4639424878393\\
-26.855	-25.7769605996258\\
-17.09	-16.4039566802311\\
-9.766	-9.37396377642694\\
-6.104	-5.85896732452489\\
-7.324	-7.02999290380411\\
-9.766	-9.37396377642694\\
-8.545	-8.20197834011553\\
-12.207	-11.7169747920176\\
-17.09	-16.4039566802311\\
-15.869	-15.2319712439196\\
-19.531	-18.7469676958217\\
-13.428	-12.888960228329\\
-4.883	-4.68698188821347\\
-9.766	-9.37396377642694\\
-21.973	-21.0909385684445\\
-19.531	-18.7469676958217\\
-21.973	-21.0909385684445\\
-18.311	-17.5759421165425\\
-14.648	-14.0599858076082\\
-19.531	-18.7469676958217\\
-25.635	-24.6059350203466\\
-19.531	-18.7469676958217\\
-26.855	-25.7769605996258\\
-23.193	-22.2619641477237\\
-28.076	-26.9489460359372\\
-26.855	-25.7769605996258\\
-20.752	-19.9189531321331\\
-19.531	-18.7469676958217\\
-29.297	-28.1209314722486\\
-40.283	-38.6659208279548\\
-30.518	-29.2929169085601\\
-18.311	-17.5759421165425\\
-17.09	-16.4039566802311\\
-18.311	-17.5759421165425\\
-21.973	-21.0909385684445\\
-26.855	-25.7769605996258\\
-19.531	-18.7469676958217\\
-26.855	-25.7769605996258\\
-28.076	-26.9489460359372\\
-20.752	-19.9189531321331\\
-14.648	-14.0599858076082\\
-20.752	-19.9189531321331\\
-34.18	-32.8079133604621\\
-30.518	-29.2929169085601\\
-31.738	-30.4639424878393\\
-21.973	-21.0909385684445\\
-17.09	-16.4039566802311\\
-10.986	-10.5449893557062\\
-6.104	-5.85896732452489\\
-4.883	-4.68698188821347\\
-10.986	-10.5449893557062\\
-19.531	-18.7469676958217\\
-18.311	-17.5759421165425\\
-15.869	-15.2319712439196\\
-14.648	-14.0599858076082\\
-20.752	-19.9189531321331\\
-14.648	-14.0599858076082\\
-9.766	-9.37396377642694\\
-13.428	-12.888960228329\\
-8.545	-8.20197834011553\\
-10.986	-10.5449893557062\\
-18.311	-17.5759421165425\\
-23.193	-22.2619641477237\\
-17.09	-16.4039566802311\\
-9.766	-9.37396377642694\\
-10.986	-10.5449893557062\\
-12.207	-11.7169747920176\\
-9.766	-9.37396377642694\\
-13.428	-12.888960228329\\
-31.738	-30.4639424878393\\
-25.635	-24.6059350203466\\
-31.738	-30.4639424878393\\
-37.842	-36.3229098123642\\
-36.621	-35.1509243760527\\
-26.855	-25.7769605996258\\
-20.752	-19.9189531321331\\
-18.311	-17.5759421165425\\
-21.973	-21.0909385684445\\
-23.193	-22.2619641477237\\
-29.297	-28.1209314722486\\
-36.621	-35.1509243760527\\
-42.725	-41.0098917005776\\
-46.387	-44.5248881524797\\
-37.842	-36.3229098123642\\
-35.4	-33.9789389397413\\
-40.283	-38.6659208279548\\
-31.738	-30.4639424878393\\
-34.18	-32.8079133604621\\
-31.738	-30.4639424878393\\
-18.311	-17.5759421165425\\
-12.207	-11.7169747920176\\
-13.428	-12.888960228329\\
-18.311	-17.5759421165425\\
-17.09	-16.4039566802311\\
-8.545	-8.20197834011553\\
-14.648	-14.0599858076082\\
-13.428	-12.888960228329\\
-6.104	-5.85896732452489\\
-10.986	-10.5449893557062\\
-12.207	-11.7169747920176\\
-7.324	-7.02999290380411\\
-18.311	-17.5759421165425\\
-20.752	-19.9189531321331\\
-14.648	-14.0599858076082\\
-23.193	-22.2619641477237\\
-37.842	-36.3229098123642\\
-32.959	-31.6359279241507\\
-37.842	-36.3229098123642\\
-32.959	-31.6359279241507\\
-29.297	-28.1209314722486\\
-20.752	-19.9189531321331\\
-14.648	-14.0599858076082\\
-17.09	-16.4039566802311\\
-13.428	-12.888960228329\\
-10.986	-10.5449893557062\\
-14.648	-14.0599858076082\\
-13.428	-12.888960228329\\
-3.662	-3.51499645190205\\
-7.324	-7.02999290380411\\
-15.869	-15.2319712439196\\
-20.752	-19.9189531321331\\
-21.973	-21.0909385684445\\
-23.193	-22.2619641477237\\
-21.973	-21.0909385684445\\
-26.855	-25.7769605996258\\
-21.973	-21.0909385684445\\
-17.09	-16.4039566802311\\
-14.648	-14.0599858076082\\
-20.752	-19.9189531321331\\
-18.311	-17.5759421165425\\
-13.428	-12.888960228329\\
-18.311	-17.5759421165425\\
-26.855	-25.7769605996258\\
-20.752	-19.9189531321331\\
-15.869	-15.2319712439196\\
-13.428	-12.888960228329\\
-14.648	-14.0599858076082\\
-12.207	-11.7169747920176\\
-8.545	-8.20197834011553\\
-14.648	-14.0599858076082\\
-19.531	-18.7469676958217\\
-18.311	-17.5759421165425\\
-15.869	-15.2319712439196\\
-19.531	-18.7469676958217\\
-18.311	-17.5759421165425\\
-10.986	-10.5449893557062\\
-21.973	-21.0909385684445\\
-30.518	-29.2929169085601\\
-29.297	-28.1209314722486\\
-23.193	-22.2619641477237\\
-21.973	-21.0909385684445\\
-19.531	-18.7469676958217\\
-18.311	-17.5759421165425\\
-15.869	-15.2319712439196\\
-19.531	-18.7469676958217\\
-17.09	-16.4039566802311\\
-13.428	-12.888960228329\\
-17.09	-16.4039566802311\\
-19.531	-18.7469676958217\\
-21.973	-21.0909385684445\\
-25.635	-24.6059350203466\\
-32.959	-31.6359279241507\\
-41.504	-39.8379062642662\\
-45.166	-43.3529027161683\\
-31.738	-30.4639424878393\\
-17.09	-16.4039566802311\\
-13.428	-12.888960228329\\
-8.545	-8.20197834011553\\
-10.986	-10.5449893557062\\
-13.428	-12.888960228329\\
-17.09	-16.4039566802311\\
-18.311	-17.5759421165425\\
-13.428	-12.888960228329\\
-19.531	-18.7469676958217\\
-24.414	-23.4339495840352\\
-25.635	-24.6059350203466\\
-26.855	-25.7769605996258\\
-39.063	-37.4948952486756\\
-35.4	-33.9789389397413\\
-25.635	-24.6059350203466\\
-20.752	-19.9189531321331\\
-19.531	-18.7469676958217\\
-23.193	-22.2619641477237\\
-19.531	-18.7469676958217\\
-15.869	-15.2319712439196\\
-17.09	-16.4039566802311\\
-10.986	-10.5449893557062\\
-14.648	-14.0599858076082\\
-28.076	-26.9489460359372\\
-19.531	-18.7469676958217\\
-15.869	-15.2319712439196\\
-34.18	-32.8079133604621\\
-28.076	-26.9489460359372\\
-21.973	-21.0909385684445\\
-34.18	-32.8079133604621\\
-31.738	-30.4639424878393\\
-20.752	-19.9189531321331\\
-15.869	-15.2319712439196\\
-13.428	-12.888960228329\\
-23.193	-22.2619641477237\\
-35.4	-33.9789389397413\\
-37.842	-36.3229098123642\\
-36.621	-35.1509243760527\\
-29.297	-28.1209314722486\\
-20.752	-19.9189531321331\\
-15.869	-15.2319712439196\\
-14.648	-14.0599858076082\\
-10.986	-10.5449893557062\\
-13.428	-12.888960228329\\
-15.869	-15.2319712439196\\
-10.986	-10.5449893557062\\
-13.428	-12.888960228329\\
-21.973	-21.0909385684445\\
-26.855	-25.7769605996258\\
-35.4	-33.9789389397413\\
-43.945	-42.1809172798568\\
-28.076	-26.9489460359372\\
-24.414	-23.4339495840352\\
-18.311	-17.5759421165425\\
-20.752	-19.9189531321331\\
-34.18	-32.8079133604621\\
-37.842	-36.3229098123642\\
-24.414	-23.4339495840352\\
-12.207	-11.7169747920176\\
-10.986	-10.5449893557062\\
-3.662	-3.51499645190205\\
-8.545	-8.20197834011553\\
-9.766	-9.37396377642694\\
-12.207	-11.7169747920176\\
-13.428	-12.888960228329\\
-7.324	-7.02999290380411\\
-8.545	-8.20197834011553\\
-10.986	-10.5449893557062\\
-8.545	-8.20197834011553\\
-12.207	-11.7169747920176\\
-14.648	-14.0599858076082\\
-24.414	-23.4339495840352\\
-28.076	-26.9489460359372\\
-24.414	-23.4339495840352\\
-28.076	-26.9489460359372\\
-35.4	-33.9789389397413\\
-40.283	-38.6659208279548\\
-29.297	-28.1209314722486\\
-19.531	-18.7469676958217\\
-18.311	-17.5759421165425\\
-30.518	-29.2929169085601\\
-37.842	-36.3229098123642\\
-26.855	-25.7769605996258\\
-15.869	-15.2319712439196\\
-9.766	-9.37396377642694\\
-4.883	-4.68698188821347\\
-6.104	-5.85896732452489\\
-12.207	-11.7169747920176\\
-14.648	-14.0599858076082\\
-12.207	-11.7169747920176\\
-9.766	-9.37396377642694\\
-4.883	-4.68698188821347\\
-8.545	-8.20197834011553\\
-7.324	-7.02999290380411\\
-4.883	-4.68698188821347\\
-12.207	-11.7169747920176\\
-25.635	-24.6059350203466\\
-24.414	-23.4339495840352\\
-31.738	-30.4639424878393\\
-25.635	-24.6059350203466\\
-32.959	-31.6359279241507\\
-46.387	-44.5248881524797\\
-42.725	-41.0098917005776\\
-39.063	-37.4948952486756\\
-34.18	-32.8079133604621\\
-31.738	-30.4639424878393\\
-32.959	-31.6359279241507\\
-20.752	-19.9189531321331\\
-13.428	-12.888960228329\\
-14.648	-14.0599858076082\\
-23.193	-22.2619641477237\\
-20.752	-19.9189531321331\\
-15.869	-15.2319712439196\\
-19.531	-18.7469676958217\\
-17.09	-16.4039566802311\\
-20.752	-19.9189531321331\\
-19.531	-18.7469676958217\\
-24.414	-23.4339495840352\\
-20.752	-19.9189531321331\\
-15.869	-15.2319712439196\\
-18.311	-17.5759421165425\\
-14.648	-14.0599858076082\\
-2.441	-2.34301101559064\\
-9.766	-9.37396377642694\\
-13.428	-12.888960228329\\
-8.545	-8.20197834011553\\
-4.883	-4.68698188821347\\
-8.545	-8.20197834011553\\
-12.207	-11.7169747920176\\
-14.648	-14.0599858076082\\
-13.428	-12.888960228329\\
-19.531	-18.7469676958217\\
-12.207	-11.7169747920176\\
-19.531	-18.7469676958217\\
-17.09	-16.4039566802311\\
-7.324	-7.02999290380411\\
-3.662	-3.51499645190205\\
-10.986	-10.5449893557062\\
-8.545	-8.20197834011553\\
-7.324	-7.02999290380411\\
-19.531	-18.7469676958217\\
-20.752	-19.9189531321331\\
-10.986	-10.5449893557062\\
-14.648	-14.0599858076082\\
-12.207	-11.7169747920176\\
-13.428	-12.888960228329\\
-14.648	-14.0599858076082\\
-8.545	-8.20197834011553\\
-10.986	-10.5449893557062\\
-7.324	-7.02999290380411\\
-9.766	-9.37396377642694\\
-14.648	-14.0599858076082\\
-9.766	-9.37396377642694\\
-18.311	-17.5759421165425\\
-26.855	-25.7769605996258\\
-18.311	-17.5759421165425\\
-17.09	-16.4039566802311\\
-15.869	-15.2319712439196\\
-21.973	-21.0909385684445\\
-13.428	-12.888960228329\\
-14.648	-14.0599858076082\\
-13.428	-12.888960228329\\
-17.09	-16.4039566802311\\
-12.207	-11.7169747920176\\
-8.545	-8.20197834011553\\
-10.986	-10.5449893557062\\
-15.869	-15.2319712439196\\
-12.207	-11.7169747920176\\
-14.648	-14.0599858076082\\
-21.973	-21.0909385684445\\
-18.311	-17.5759421165425\\
-25.635	-24.6059350203466\\
-35.4	-33.9789389397413\\
-28.076	-26.9489460359372\\
-18.311	-17.5759421165425\\
-15.869	-15.2319712439196\\
-23.193	-22.2619641477237\\
-28.076	-26.9489460359372\\
-20.752	-19.9189531321331\\
-23.193	-22.2619641477237\\
-18.311	-17.5759421165425\\
-7.324	-7.02999290380411\\
-8.545	-8.20197834011553\\
-20.752	-19.9189531321331\\
-18.311	-17.5759421165425\\
-23.193	-22.2619641477237\\
-19.531	-18.7469676958217\\
-15.869	-15.2319712439196\\
-18.311	-17.5759421165425\\
-23.193	-22.2619641477237\\
-19.531	-18.7469676958217\\
-18.311	-17.5759421165425\\
-19.531	-18.7469676958217\\
-13.428	-12.888960228329\\
-14.648	-14.0599858076082\\
-13.428	-12.888960228329\\
-20.752	-19.9189531321331\\
-25.635	-24.6059350203466\\
-23.193	-22.2619641477237\\
-17.09	-16.4039566802311\\
-12.207	-11.7169747920176\\
-15.869	-15.2319712439196\\
-21.973	-21.0909385684445\\
-24.414	-23.4339495840352\\
-28.076	-26.9489460359372\\
-23.193	-22.2619641477237\\
-13.428	-12.888960228329\\
-24.414	-23.4339495840352\\
-36.621	-35.1509243760527\\
-24.414	-23.4339495840352\\
-23.193	-22.2619641477237\\
-29.297	-28.1209314722486\\
-21.973	-21.0909385684445\\
-20.752	-19.9189531321331\\
-18.311	-17.5759421165425\\
-20.752	-19.9189531321331\\
-25.635	-24.6059350203466\\
-19.531	-18.7469676958217\\
-21.973	-21.0909385684445\\
-20.752	-19.9189531321331\\
-19.531	-18.7469676958217\\
-21.973	-21.0909385684445\\
-17.09	-16.4039566802311\\
-18.311	-17.5759421165425\\
-17.09	-16.4039566802311\\
-8.545	-8.20197834011553\\
-2.441	-2.34301101559064\\
-8.545	-8.20197834011553\\
-19.531	-18.7469676958217\\
-25.635	-24.6059350203466\\
-23.193	-22.2619641477237\\
-17.09	-16.4039566802311\\
-20.752	-19.9189531321331\\
-19.531	-18.7469676958217\\
-15.869	-15.2319712439196\\
-12.207	-11.7169747920176\\
-14.648	-14.0599858076082\\
-12.207	-11.7169747920176\\
-15.869	-15.2319712439196\\
-13.428	-12.888960228329\\
-10.986	-10.5449893557062\\
-8.545	-8.20197834011553\\
-14.648	-14.0599858076082\\
-21.973	-21.0909385684445\\
-17.09	-16.4039566802311\\
-23.193	-22.2619641477237\\
-20.752	-19.9189531321331\\
-28.076	-26.9489460359372\\
-21.973	-21.0909385684445\\
-23.193	-22.2619641477237\\
-15.869	-15.2319712439196\\
-20.752	-19.9189531321331\\
-19.531	-18.7469676958217\\
-20.752	-19.9189531321331\\
-21.973	-21.0909385684445\\
-25.635	-24.6059350203466\\
-19.531	-18.7469676958217\\
-24.414	-23.4339495840352\\
-30.518	-29.2929169085601\\
-25.635	-24.6059350203466\\
-23.193	-22.2619641477237\\
-18.311	-17.5759421165425\\
-21.973	-21.0909385684445\\
-19.531	-18.7469676958217\\
-14.648	-14.0599858076082\\
-13.428	-12.888960228329\\
-15.869	-15.2319712439196\\
-13.428	-12.888960228329\\
-26.855	-25.7769605996258\\
-35.4	-33.9789389397413\\
-29.297	-28.1209314722486\\
-18.311	-17.5759421165425\\
-14.648	-14.0599858076082\\
-12.207	-11.7169747920176\\
-10.986	-10.5449893557062\\
-18.311	-17.5759421165425\\
-23.193	-22.2619641477237\\
-25.635	-24.6059350203466\\
-21.973	-21.0909385684445\\
-15.869	-15.2319712439196\\
-25.635	-24.6059350203466\\
-30.518	-29.2929169085601\\
-31.738	-30.4639424878393\\
-25.635	-24.6059350203466\\
-42.725	-41.0098917005776\\
-32.959	-31.6359279241507\\
-18.311	-17.5759421165425\\
-13.428	-12.888960228329\\
-19.531	-18.7469676958217\\
-26.855	-25.7769605996258\\
-31.738	-30.4639424878393\\
-26.855	-25.7769605996258\\
-19.531	-18.7469676958217\\
-13.428	-12.888960228329\\
-15.869	-15.2319712439196\\
-9.766	-9.37396377642694\\
-8.545	-8.20197834011553\\
-4.883	-4.68698188821347\\
-10.986	-10.5449893557062\\
-14.648	-14.0599858076082\\
-15.869	-15.2319712439196\\
-17.09	-16.4039566802311\\
-18.311	-17.5759421165425\\
-23.193	-22.2619641477237\\
-14.648	-14.0599858076082\\
-21.973	-21.0909385684445\\
-26.855	-25.7769605996258\\
-21.973	-21.0909385684445\\
-23.193	-22.2619641477237\\
-21.973	-21.0909385684445\\
-20.752	-19.9189531321331\\
-29.297	-28.1209314722486\\
-23.193	-22.2619641477237\\
-17.09	-16.4039566802311\\
-20.752	-19.9189531321331\\
-13.428	-12.888960228329\\
-20.752	-19.9189531321331\\
-26.855	-25.7769605996258\\
-28.076	-26.9489460359372\\
-29.297	-28.1209314722486\\
-45.166	-43.3529027161683\\
-30.518	-29.2929169085601\\
-26.855	-25.7769605996258\\
-18.311	-17.5759421165425\\
-26.855	-25.7769605996258\\
-35.4	-33.9789389397413\\
-24.414	-23.4339495840352\\
-19.531	-18.7469676958217\\
-26.855	-25.7769605996258\\
-20.752	-19.9189531321331\\
-10.986	-10.5449893557062\\
-15.869	-15.2319712439196\\
-13.428	-12.888960228329\\
-8.545	-8.20197834011553\\
-7.324	-7.02999290380411\\
-10.986	-10.5449893557062\\
-14.648	-14.0599858076082\\
-17.09	-16.4039566802311\\
-23.193	-22.2619641477237\\
-21.973	-21.0909385684445\\
-25.635	-24.6059350203466\\
-28.076	-26.9489460359372\\
-19.531	-18.7469676958217\\
-21.973	-21.0909385684445\\
-19.531	-18.7469676958217\\
-20.752	-19.9189531321331\\
-29.297	-28.1209314722486\\
-23.193	-22.2619641477237\\
-26.855	-25.7769605996258\\
-23.193	-22.2619641477237\\
-19.531	-18.7469676958217\\
-29.297	-28.1209314722486\\
-25.635	-24.6059350203466\\
-26.855	-25.7769605996258\\
-24.414	-23.4339495840352\\
-15.869	-15.2319712439196\\
-9.766	-9.37396377642694\\
-8.545	-8.20197834011553\\
-10.986	-10.5449893557062\\
-12.207	-11.7169747920176\\
-18.311	-17.5759421165425\\
-13.428	-12.888960228329\\
-21.973	-21.0909385684445\\
-31.738	-30.4639424878393\\
-36.621	-35.1509243760527\\
-26.855	-25.7769605996258\\
-29.297	-28.1209314722486\\
-26.855	-25.7769605996258\\
-18.311	-17.5759421165425\\
-21.973	-21.0909385684445\\
-23.193	-22.2619641477237\\
-19.531	-18.7469676958217\\
-23.193	-22.2619641477237\\
-26.855	-25.7769605996258\\
-39.063	-37.4948952486756\\
-34.18	-32.8079133604621\\
-31.738	-30.4639424878393\\
-37.842	-36.3229098123642\\
-28.076	-26.9489460359372\\
-30.518	-29.2929169085601\\
-24.414	-23.4339495840352\\
-23.193	-22.2619641477237\\
-30.518	-29.2929169085601\\
-34.18	-32.8079133604621\\
-9.766	-9.37396377642694\\
-20.752	-19.9189531321331\\
-14.648	-14.0599858076082\\
-23.193	-22.2619641477237\\
-20.752	-19.9189531321331\\
-13.428	-12.888960228329\\
-17.09	-16.4039566802311\\
-26.855	-25.7769605996258\\
-43.945	-42.1809172798568\\
-42.725	-41.0098917005776\\
-37.842	-36.3229098123642\\
-32.959	-31.6359279241507\\
-39.063	-37.4948952486756\\
-45.166	-43.3529027161683\\
-32.959	-31.6359279241507\\
-34.18	-32.8079133604621\\
-36.621	-35.1509243760527\\
-25.635	-24.6059350203466\\
-21.973	-21.0909385684445\\
-26.855	-25.7769605996258\\
-19.531	-18.7469676958217\\
-13.428	-12.888960228329\\
-19.531	-18.7469676958217\\
-13.428	-12.888960228329\\
-19.531	-18.7469676958217\\
-15.869	-15.2319712439196\\
-19.531	-18.7469676958217\\
-24.414	-23.4339495840352\\
-21.973	-21.0909385684445\\
-15.869	-15.2319712439196\\
-19.531	-18.7469676958217\\
-21.973	-21.0909385684445\\
-23.193	-22.2619641477237\\
-19.531	-18.7469676958217\\
-15.869	-15.2319712439196\\
-26.855	-25.7769605996258\\
-18.311	-17.5759421165425\\
-24.414	-23.4339495840352\\
-21.973	-21.0909385684445\\
-18.311	-17.5759421165425\\
-13.428	-12.888960228329\\
-8.545	-8.20197834011553\\
-4.883	-4.68698188821347\\
-18.311	-17.5759421165425\\
-13.428	-12.888960228329\\
-12.207	-11.7169747920176\\
-7.324	-7.02999290380411\\
-12.207	-11.7169747920176\\
-17.09	-16.4039566802311\\
-20.752	-19.9189531321331\\
-23.193	-22.2619641477237\\
-28.076	-26.9489460359372\\
-26.855	-25.7769605996258\\
-18.311	-17.5759421165425\\
-7.324	-7.02999290380411\\
-15.869	-15.2319712439196\\
-10.986	-10.5449893557062\\
-14.648	-14.0599858076082\\
-20.752	-19.9189531321331\\
-36.621	-35.1509243760527\\
-28.076	-26.9489460359372\\
-25.635	-24.6059350203466\\
-34.18	-32.8079133604621\\
-23.193	-22.2619641477237\\
-13.428	-12.888960228329\\
-17.09	-16.4039566802311\\
-20.752	-19.9189531321331\\
-30.518	-29.2929169085601\\
-21.973	-21.0909385684445\\
-19.531	-18.7469676958217\\
-17.09	-16.4039566802311\\
-20.752	-19.9189531321331\\
-25.635	-24.6059350203466\\
-40.283	-38.6659208279548\\
-32.959	-31.6359279241507\\
-21.973	-21.0909385684445\\
-14.648	-14.0599858076082\\
-7.324	-7.02999290380411\\
-9.766	-9.37396377642694\\
-8.545	-8.20197834011553\\
-13.428	-12.888960228329\\
-23.193	-22.2619641477237\\
-31.738	-30.4639424878393\\
-43.945	-42.1809172798568\\
-35.4	-33.9789389397413\\
-20.752	-19.9189531321331\\
-21.973	-21.0909385684445\\
-19.531	-18.7469676958217\\
-21.973	-21.0909385684445\\
-29.297	-28.1209314722486\\
-36.621	-35.1509243760527\\
-26.855	-25.7769605996258\\
-19.531	-18.7469676958217\\
-21.973	-21.0909385684445\\
-29.297	-28.1209314722486\\
-19.531	-18.7469676958217\\
-14.648	-14.0599858076082\\
-13.428	-12.888960228329\\
-9.766	-9.37396377642694\\
-7.324	-7.02999290380411\\
-6.104	-5.85896732452489\\
-12.207	-11.7169747920176\\
-17.09	-16.4039566802311\\
-18.311	-17.5759421165425\\
-10.986	-10.5449893557062\\
-12.207	-11.7169747920176\\
-6.104	-5.85896732452489\\
-2.441	-2.34301101559064\\
-6.104	-5.85896732452489\\
-10.986	-10.5449893557062\\
-13.428	-12.888960228329\\
-17.09	-16.4039566802311\\
-19.531	-18.7469676958217\\
-23.193	-22.2619641477237\\
-29.297	-28.1209314722486\\
-40.283	-38.6659208279548\\
-42.725	-41.0098917005776\\
-37.842	-36.3229098123642\\
-21.973	-21.0909385684445\\
-20.752	-19.9189531321331\\
-21.973	-21.0909385684445\\
-28.076	-26.9489460359372\\
-21.973	-21.0909385684445\\
-17.09	-16.4039566802311\\
-12.207	-11.7169747920176\\
-19.531	-18.7469676958217\\
-17.09	-16.4039566802311\\
-8.545	-8.20197834011553\\
-6.104	-5.85896732452489\\
-10.986	-10.5449893557062\\
-14.648	-14.0599858076082\\
-9.766	-9.37396377642694\\
-18.311	-17.5759421165425\\
-12.207	-11.7169747920176\\
-21.973	-21.0909385684445\\
-29.297	-28.1209314722486\\
-21.973	-21.0909385684445\\
-29.297	-28.1209314722486\\
-40.283	-38.6659208279548\\
-42.725	-41.0098917005776\\
-31.738	-30.4639424878393\\
-24.414	-23.4339495840352\\
-17.09	-16.4039566802311\\
-10.986	-10.5449893557062\\
-18.311	-17.5759421165425\\
-9.766	-9.37396377642694\\
-20.752	-19.9189531321331\\
-14.648	-14.0599858076082\\
-19.531	-18.7469676958217\\
-12.207	-11.7169747920176\\
-20.752	-19.9189531321331\\
-13.428	-12.888960228329\\
-9.766	-9.37396377642694\\
-10.986	-10.5449893557062\\
-7.324	-7.02999290380411\\
-8.545	-8.20197834011553\\
-6.104	-5.85896732452489\\
-7.324	-7.02999290380411\\
-6.104	-5.85896732452489\\
-9.766	-9.37396377642694\\
-14.648	-14.0599858076082\\
-20.752	-19.9189531321331\\
-15.869	-15.2319712439196\\
-25.635	-24.6059350203466\\
-30.518	-29.2929169085601\\
-21.973	-21.0909385684445\\
-25.635	-24.6059350203466\\
-9.766	-9.37396377642694\\
-7.324	-7.02999290380411\\
-12.207	-11.7169747920176\\
-20.752	-19.9189531321331\\
-30.518	-29.2929169085601\\
-29.297	-28.1209314722486\\
-19.531	-18.7469676958217\\
-14.648	-14.0599858076082\\
-18.311	-17.5759421165425\\
-17.09	-16.4039566802311\\
-14.648	-14.0599858076082\\
-7.324	-7.02999290380411\\
-13.428	-12.888960228329\\
-10.986	-10.5449893557062\\
-7.324	-7.02999290380411\\
-12.207	-11.7169747920176\\
-14.648	-14.0599858076082\\
-15.869	-15.2319712439196\\
-8.545	-8.20197834011553\\
-19.531	-18.7469676958217\\
-25.635	-24.6059350203466\\
-17.09	-16.4039566802311\\
-25.635	-24.6059350203466\\
-19.531	-18.7469676958217\\
-12.207	-11.7169747920176\\
-19.531	-18.7469676958217\\
-17.09	-16.4039566802311\\
-9.766	-9.37396377642694\\
-14.648	-14.0599858076082\\
-7.324	-7.02999290380411\\
-4.883	-4.68698188821347\\
-8.545	-8.20197834011553\\
-13.428	-12.888960228329\\
-12.207	-11.7169747920176\\
-13.428	-12.888960228329\\
-15.869	-15.2319712439196\\
-9.766	-9.37396377642694\\
-7.324	-7.02999290380411\\
-6.104	-5.85896732452489\\
-7.324	-7.02999290380411\\
-9.766	-9.37396377642694\\
-15.869	-15.2319712439196\\
-18.311	-17.5759421165425\\
-15.869	-15.2319712439196\\
-17.09	-16.4039566802311\\
-21.973	-21.0909385684445\\
-28.076	-26.9489460359372\\
-30.518	-29.2929169085601\\
-21.973	-21.0909385684445\\
-24.414	-23.4339495840352\\
-25.635	-24.6059350203466\\
-29.297	-28.1209314722486\\
-37.842	-36.3229098123642\\
-32.959	-31.6359279241507\\
-29.297	-28.1209314722486\\
-24.414	-23.4339495840352\\
-23.193	-22.2619641477237\\
-29.297	-28.1209314722486\\
-17.09	-16.4039566802311\\
-21.973	-21.0909385684445\\
-23.193	-22.2619641477237\\
-26.855	-25.7769605996258\\
-20.752	-19.9189531321331\\
-25.635	-24.6059350203466\\
-18.311	-17.5759421165425\\
-12.207	-11.7169747920176\\
-20.752	-19.9189531321331\\
-28.076	-26.9489460359372\\
-20.752	-19.9189531321331\\
-13.428	-12.888960228329\\
-9.766	-9.37396377642694\\
-7.324	-7.02999290380411\\
-9.766	-9.37396377642694\\
-4.883	-4.68698188821347\\
-7.324	-7.02999290380411\\
-8.545	-8.20197834011553\\
-4.883	-4.68698188821347\\
-9.766	-9.37396377642694\\
-7.324	-7.02999290380411\\
-4.883	-4.68698188821347\\
-7.324	-7.02999290380411\\
-12.207	-11.7169747920176\\
-10.986	-10.5449893557062\\
-13.428	-12.888960228329\\
-15.869	-15.2319712439196\\
-13.428	-12.888960228329\\
-17.09	-16.4039566802311\\
-29.297	-28.1209314722486\\
-20.752	-19.9189531321331\\
-13.428	-12.888960228329\\
-19.531	-18.7469676958217\\
-13.428	-12.888960228329\\
-8.545	-8.20197834011553\\
-15.869	-15.2319712439196\\
-8.545	-8.20197834011553\\
-6.104	-5.85896732452489\\
-12.207	-11.7169747920176\\
-15.869	-15.2319712439196\\
-9.766	-9.37396377642694\\
-6.104	-5.85896732452489\\
-12.207	-11.7169747920176\\
-18.311	-17.5759421165425\\
-24.414	-23.4339495840352\\
-15.869	-15.2319712439196\\
-23.193	-22.2619641477237\\
-28.076	-26.9489460359372\\
-25.635	-24.6059350203466\\
-21.973	-21.0909385684445\\
-36.621	-35.1509243760527\\
-30.518	-29.2929169085601\\
-32.959	-31.6359279241507\\
-28.076	-26.9489460359372\\
-35.4	-33.9789389397413\\
-37.842	-36.3229098123642\\
-25.635	-24.6059350203466\\
-13.428	-12.888960228329\\
-18.311	-17.5759421165425\\
-21.973	-21.0909385684445\\
-23.193	-22.2619641477237\\
-14.648	-14.0599858076082\\
-8.545	-8.20197834011553\\
-13.428	-12.888960228329\\
-21.973	-21.0909385684445\\
-28.076	-26.9489460359372\\
-25.635	-24.6059350203466\\
-15.869	-15.2319712439196\\
-13.428	-12.888960228329\\
-17.09	-16.4039566802311\\
-26.855	-25.7769605996258\\
-30.518	-29.2929169085601\\
-23.193	-22.2619641477237\\
-31.738	-30.4639424878393\\
-34.18	-32.8079133604621\\
-30.518	-29.2929169085601\\
-42.725	-41.0098917005776\\
-36.621	-35.1509243760527\\
-30.518	-29.2929169085601\\
-37.842	-36.3229098123642\\
-36.621	-35.1509243760527\\
-18.311	-17.5759421165425\\
-17.09	-16.4039566802311\\
-20.752	-19.9189531321331\\
-25.635	-24.6059350203466\\
-26.855	-25.7769605996258\\
-18.311	-17.5759421165425\\
-24.414	-23.4339495840352\\
-21.973	-21.0909385684445\\
-14.648	-14.0599858076082\\
-19.531	-18.7469676958217\\
-18.311	-17.5759421165425\\
-26.855	-25.7769605996258\\
-30.518	-29.2929169085601\\
-23.193	-22.2619641477237\\
-17.09	-16.4039566802311\\
-10.986	-10.5449893557062\\
-15.869	-15.2319712439196\\
-12.207	-11.7169747920176\\
-7.324	-7.02999290380411\\
-15.869	-15.2319712439196\\
-19.531	-18.7469676958217\\
-23.193	-22.2619641477237\\
-21.973	-21.0909385684445\\
-13.428	-12.888960228329\\
-9.766	-9.37396377642694\\
-7.324	-7.02999290380411\\
-9.766	-9.37396377642694\\
-14.648	-14.0599858076082\\
-17.09	-16.4039566802311\\
-14.648	-14.0599858076082\\
-24.414	-23.4339495840352\\
-26.855	-25.7769605996258\\
-30.518	-29.2929169085601\\
-35.4	-33.9789389397413\\
-23.193	-22.2619641477237\\
-19.531	-18.7469676958217\\
-14.648	-14.0599858076082\\
-17.09	-16.4039566802311\\
-20.752	-19.9189531321331\\
-15.869	-15.2319712439196\\
-10.986	-10.5449893557062\\
-18.311	-17.5759421165425\\
-23.193	-22.2619641477237\\
-20.752	-19.9189531321331\\
-31.738	-30.4639424878393\\
-36.621	-35.1509243760527\\
-25.635	-24.6059350203466\\
-18.311	-17.5759421165425\\
-13.428	-12.888960228329\\
-10.986	-10.5449893557062\\
-20.752	-19.9189531321331\\
-14.648	-14.0599858076082\\
-17.09	-16.4039566802311\\
-23.193	-22.2619641477237\\
-25.635	-24.6059350203466\\
-24.414	-23.4339495840352\\
-28.076	-26.9489460359372\\
-18.311	-17.5759421165425\\
-21.973	-21.0909385684445\\
-15.869	-15.2319712439196\\
-14.648	-14.0599858076082\\
-20.752	-19.9189531321331\\
-28.076	-26.9489460359372\\
-30.518	-29.2929169085601\\
-18.311	-17.5759421165425\\
-29.297	-28.1209314722486\\
-30.518	-29.2929169085601\\
-21.973	-21.0909385684445\\
-14.648	-14.0599858076082\\
-21.973	-21.0909385684445\\
-17.09	-16.4039566802311\\
-19.531	-18.7469676958217\\
-13.428	-12.888960228329\\
-18.311	-17.5759421165425\\
-34.18	-32.8079133604621\\
-36.621	-35.1509243760527\\
-25.635	-24.6059350203466\\
-28.076	-26.9489460359372\\
-20.752	-19.9189531321331\\
-14.648	-14.0599858076082\\
-20.752	-19.9189531321331\\
-32.959	-31.6359279241507\\
-37.842	-36.3229098123642\\
-47.607	-45.6959137317589\\
-37.842	-36.3229098123642\\
-30.518	-29.2929169085601\\
-23.193	-22.2619641477237\\
-25.635	-24.6059350203466\\
-19.531	-18.7469676958217\\
-14.648	-14.0599858076082\\
-13.428	-12.888960228329\\
-15.869	-15.2319712439196\\
-10.986	-10.5449893557062\\
-7.324	-7.02999290380411\\
-12.207	-11.7169747920176\\
-17.09	-16.4039566802311\\
-12.207	-11.7169747920176\\
-7.324	-7.02999290380411\\
-18.311	-17.5759421165425\\
-25.635	-24.6059350203466\\
-12.207	-11.7169747920176\\
-15.869	-15.2319712439196\\
-14.648	-14.0599858076082\\
-18.311	-17.5759421165425\\
-17.09	-16.4039566802311\\
-10.986	-10.5449893557062\\
-8.545	-8.20197834011553\\
-7.324	-7.02999290380411\\
-9.766	-9.37396377642694\\
-18.311	-17.5759421165425\\
-15.869	-15.2319712439196\\
-10.986	-10.5449893557062\\
-7.324	-7.02999290380411\\
-8.545	-8.20197834011553\\
-17.09	-16.4039566802311\\
-19.531	-18.7469676958217\\
-15.869	-15.2319712439196\\
-13.428	-12.888960228329\\
-14.648	-14.0599858076082\\
-20.752	-19.9189531321331\\
-28.076	-26.9489460359372\\
-32.959	-31.6359279241507\\
-24.414	-23.4339495840352\\
-17.09	-16.4039566802311\\
-19.531	-18.7469676958217\\
-18.311	-17.5759421165425\\
-17.09	-16.4039566802311\\
-12.207	-11.7169747920176\\
-13.428	-12.888960228329\\
-15.869	-15.2319712439196\\
-14.648	-14.0599858076082\\
-10.986	-10.5449893557062\\
-6.104	-5.85896732452489\\
-8.545	-8.20197834011553\\
-13.428	-12.888960228329\\
-20.752	-19.9189531321331\\
-24.414	-23.4339495840352\\
-28.076	-26.9489460359372\\
-25.635	-24.6059350203466\\
-29.297	-28.1209314722486\\
-36.621	-35.1509243760527\\
-47.607	-45.6959137317589\\
-37.842	-36.3229098123642\\
-28.076	-26.9489460359372\\
-31.738	-30.4639424878393\\
-36.621	-35.1509243760527\\
-43.945	-42.1809172798568\\
-30.518	-29.2929169085601\\
-13.428	-12.888960228329\\
-9.766	-9.37396377642694\\
-10.986	-10.5449893557062\\
-4.883	-4.68698188821347\\
-6.104	-5.85896732452489\\
-9.766	-9.37396377642694\\
-13.428	-12.888960228329\\
-14.648	-14.0599858076082\\
-10.986	-10.5449893557062\\
-8.545	-8.20197834011553\\
-13.428	-12.888960228329\\
-15.869	-15.2319712439196\\
-17.09	-16.4039566802311\\
-15.869	-15.2319712439196\\
-13.428	-12.888960228329\\
-8.545	-8.20197834011553\\
-10.986	-10.5449893557062\\
-14.648	-14.0599858076082\\
-10.986	-10.5449893557062\\
-7.324	-7.02999290380411\\
-10.986	-10.5449893557062\\
-8.545	-8.20197834011553\\
-13.428	-12.888960228329\\
-18.311	-17.5759421165425\\
-13.428	-12.888960228329\\
-19.531	-18.7469676958217\\
-20.752	-19.9189531321331\\
-17.09	-16.4039566802311\\
-14.648	-14.0599858076082\\
-21.973	-21.0909385684445\\
-26.855	-25.7769605996258\\
-29.297	-28.1209314722486\\
-24.414	-23.4339495840352\\
-21.973	-21.0909385684445\\
-20.752	-19.9189531321331\\
-28.076	-26.9489460359372\\
-21.973	-21.0909385684445\\
-26.855	-25.7769605996258\\
-28.076	-26.9489460359372\\
-26.855	-25.7769605996258\\
-47.607	-45.6959137317589\\
-37.842	-36.3229098123642\\
-20.752	-19.9189531321331\\
-14.648	-14.0599858076082\\
-18.311	-17.5759421165425\\
-24.414	-23.4339495840352\\
-26.855	-25.7769605996258\\
-29.297	-28.1209314722486\\
-31.738	-30.4639424878393\\
-20.752	-19.9189531321331\\
-19.531	-18.7469676958217\\
-20.752	-19.9189531321331\\
-14.648	-14.0599858076082\\
-12.207	-11.7169747920176\\
-17.09	-16.4039566802311\\
-21.973	-21.0909385684445\\
-28.076	-26.9489460359372\\
-25.635	-24.6059350203466\\
-18.311	-17.5759421165425\\
-14.648	-14.0599858076082\\
-21.973	-21.0909385684445\\
-28.076	-26.9489460359372\\
-34.18	-32.8079133604621\\
-37.842	-36.3229098123642\\
-42.725	-41.0098917005776\\
-35.4	-33.9789389397413\\
-32.959	-31.6359279241507\\
-20.752	-19.9189531321331\\
-12.207	-11.7169747920176\\
-24.414	-23.4339495840352\\
-32.959	-31.6359279241507\\
-19.531	-18.7469676958217\\
-25.635	-24.6059350203466\\
-36.621	-35.1509243760527\\
-45.166	-43.3529027161683\\
-31.738	-30.4639424878393\\
-18.311	-17.5759421165425\\
-12.207	-11.7169747920176\\
-15.869	-15.2319712439196\\
-14.648	-14.0599858076082\\
-13.428	-12.888960228329\\
-10.986	-10.5449893557062\\
-12.207	-11.7169747920176\\
-9.766	-9.37396377642694\\
-12.207	-11.7169747920176\\
-10.986	-10.5449893557062\\
-13.428	-12.888960228329\\
-20.752	-19.9189531321331\\
-24.414	-23.4339495840352\\
-30.518	-29.2929169085601\\
-31.738	-30.4639424878393\\
-19.531	-18.7469676958217\\
-17.09	-16.4039566802311\\
-21.973	-21.0909385684445\\
-14.648	-14.0599858076082\\
-13.428	-12.888960228329\\
-10.986	-10.5449893557062\\
-14.648	-14.0599858076082\\
-8.545	-8.20197834011553\\
-4.883	-4.68698188821347\\
-6.104	-5.85896732452489\\
-8.545	-8.20197834011553\\
-14.648	-14.0599858076082\\
-9.766	-9.37396377642694\\
-13.428	-12.888960228329\\
-17.09	-16.4039566802311\\
-20.752	-19.9189531321331\\
-17.09	-16.4039566802311\\
-14.648	-14.0599858076082\\
-17.09	-16.4039566802311\\
-20.752	-19.9189531321331\\
-18.311	-17.5759421165425\\
-20.752	-19.9189531321331\\
-13.428	-12.888960228329\\
-12.207	-11.7169747920176\\
-15.869	-15.2319712439196\\
-13.428	-12.888960228329\\
-14.648	-14.0599858076082\\
-20.752	-19.9189531321331\\
-30.518	-29.2929169085601\\
-24.414	-23.4339495840352\\
-20.752	-19.9189531321331\\
-31.738	-30.4639424878393\\
-23.193	-22.2619641477237\\
-15.869	-15.2319712439196\\
-14.648	-14.0599858076082\\
-9.766	-9.37396377642694\\
-7.324	-7.02999290380411\\
-17.09	-16.4039566802311\\
-13.428	-12.888960228329\\
-7.324	-7.02999290380411\\
-13.428	-12.888960228329\\
-14.648	-14.0599858076082\\
-10.986	-10.5449893557062\\
-9.766	-9.37396377642694\\
-4.883	-4.68698188821347\\
-6.104	-5.85896732452489\\
-17.09	-16.4039566802311\\
-20.752	-19.9189531321331\\
-15.869	-15.2319712439196\\
-12.207	-11.7169747920176\\
-17.09	-16.4039566802311\\
-18.311	-17.5759421165425\\
-20.752	-19.9189531321331\\
-21.973	-21.0909385684445\\
-26.855	-25.7769605996258\\
-21.973	-21.0909385684445\\
-15.869	-15.2319712439196\\
-12.207	-11.7169747920176\\
-13.428	-12.888960228329\\
-15.869	-15.2319712439196\\
-13.428	-12.888960228329\\
-8.545	-8.20197834011553\\
-18.311	-17.5759421165425\\
-23.193	-22.2619641477237\\
-20.752	-19.9189531321331\\
-24.414	-23.4339495840352\\
-30.518	-29.2929169085601\\
-20.752	-19.9189531321331\\
-23.193	-22.2619641477237\\
-30.518	-29.2929169085601\\
-24.414	-23.4339495840352\\
-29.297	-28.1209314722486\\
-45.166	-43.3529027161683\\
-42.725	-41.0098917005776\\
-30.518	-29.2929169085601\\
-32.959	-31.6359279241507\\
-42.725	-41.0098917005776\\
-41.504	-39.8379062642662\\
-46.387	-44.5248881524797\\
-43.945	-42.1809172798568\\
-53.711	-51.5548810562838\\
-46.387	-44.5248881524797\\
-29.297	-28.1209314722486\\
-19.531	-18.7469676958217\\
-20.752	-19.9189531321331\\
-14.648	-14.0599858076082\\
-19.531	-18.7469676958217\\
-29.297	-28.1209314722486\\
-20.752	-19.9189531321331\\
-14.648	-14.0599858076082\\
-17.09	-16.4039566802311\\
-12.207	-11.7169747920176\\
-3.662	-3.51499645190205\\
-9.766	-9.37396377642694\\
-10.986	-10.5449893557062\\
-6.104	-5.85896732452489\\
-4.883	-4.68698188821347\\
-3.662	-3.51499645190205\\
-7.324	-7.02999290380411\\
-6.104	-5.85896732452489\\
-13.428	-12.888960228329\\
-18.311	-17.5759421165425\\
-17.09	-16.4039566802311\\
-13.428	-12.888960228329\\
-15.869	-15.2319712439196\\
-26.855	-25.7769605996258\\
-18.311	-17.5759421165425\\
-4.883	-4.68698188821347\\
-7.324	-7.02999290380411\\
-4.883	-4.68698188821347\\
-15.869	-15.2319712439196\\
-18.311	-17.5759421165425\\
-12.207	-11.7169747920176\\
-18.311	-17.5759421165425\\
-29.297	-28.1209314722486\\
-25.635	-24.6059350203466\\
-18.311	-17.5759421165425\\
-13.428	-12.888960228329\\
-14.648	-14.0599858076082\\
-23.193	-22.2619641477237\\
-26.855	-25.7769605996258\\
-18.311	-17.5759421165425\\
-10.986	-10.5449893557062\\
-12.207	-11.7169747920176\\
-4.883	-4.68698188821347\\
-3.662	-3.51499645190205\\
-7.324	-7.02999290380411\\
-20.752	-19.9189531321331\\
-13.428	-12.888960228329\\
-10.986	-10.5449893557062\\
-4.883	-4.68698188821347\\
-2.441	-2.34301101559064\\
-7.324	-7.02999290380411\\
-15.869	-15.2319712439196\\
-21.973	-21.0909385684445\\
-23.193	-22.2619641477237\\
-21.973	-21.0909385684445\\
-23.193	-22.2619641477237\\
-24.414	-23.4339495840352\\
-19.531	-18.7469676958217\\
-25.635	-24.6059350203466\\
-35.4	-33.9789389397413\\
-34.18	-32.8079133604621\\
-18.311	-17.5759421165425\\
-8.545	-8.20197834011553\\
-14.648	-14.0599858076082\\
-28.076	-26.9489460359372\\
-21.973	-21.0909385684445\\
-15.869	-15.2319712439196\\
-13.428	-12.888960228329\\
-24.414	-23.4339495840352\\
-34.18	-32.8079133604621\\
-37.842	-36.3229098123642\\
-39.063	-37.4948952486756\\
-29.297	-28.1209314722486\\
-28.076	-26.9489460359372\\
-18.311	-17.5759421165425\\
-12.207	-11.7169747920176\\
-18.311	-17.5759421165425\\
-19.531	-18.7469676958217\\
-20.752	-19.9189531321331\\
-23.193	-22.2619641477237\\
-15.869	-15.2319712439196\\
-17.09	-16.4039566802311\\
-20.752	-19.9189531321331\\
-9.766	-9.37396377642694\\
-6.104	-5.85896732452489\\
-8.545	-8.20197834011553\\
-6.104	-5.85896732452489\\
-1.221	-1.17198543631142\\
-4.883	-4.68698188821347\\
-17.09	-16.4039566802311\\
-13.428	-12.888960228329\\
-10.986	-10.5449893557062\\
-15.869	-15.2319712439196\\
-24.414	-23.4339495840352\\
-20.752	-19.9189531321331\\
-7.324	-7.02999290380411\\
-6.104	-5.85896732452489\\
-8.545	-8.20197834011553\\
-26.855	-25.7769605996258\\
-31.738	-30.4639424878393\\
-41.504	-39.8379062642662\\
-50.049	-48.0398846043817\\
-47.607	-45.6959137317589\\
-34.18	-32.8079133604621\\
-43.945	-42.1809172798568\\
-36.621	-35.1509243760527\\
-32.959	-31.6359279241507\\
-37.842	-36.3229098123642\\
-41.504	-39.8379062642662\\
-56.152	-53.8978920718744\\
-41.504	-39.8379062642662\\
-21.973	-21.0909385684445\\
-13.428	-12.888960228329\\
-9.766	-9.37396377642694\\
-12.207	-11.7169747920176\\
-10.986	-10.5449893557062\\
-20.752	-19.9189531321331\\
-18.311	-17.5759421165425\\
-19.531	-18.7469676958217\\
-20.752	-19.9189531321331\\
-14.648	-14.0599858076082\\
-13.428	-12.888960228329\\
-21.973	-21.0909385684445\\
-23.193	-22.2619641477237\\
-13.428	-12.888960228329\\
-7.324	-7.02999290380411\\
-9.766	-9.37396377642694\\
-12.207	-11.7169747920176\\
-29.297	-28.1209314722486\\
-35.4	-33.9789389397413\\
-32.959	-31.6359279241507\\
-35.4	-33.9789389397413\\
-43.945	-42.1809172798568\\
-52.49	-50.3828956199724\\
-46.387	-44.5248881524797\\
-30.518	-29.2929169085601\\
-19.531	-18.7469676958217\\
-13.428	-12.888960228329\\
-14.648	-14.0599858076082\\
-17.09	-16.4039566802311\\
-9.766	-9.37396377642694\\
-12.207	-11.7169747920176\\
-13.428	-12.888960228329\\
-15.869	-15.2319712439196\\
-19.531	-18.7469676958217\\
-20.752	-19.9189531321331\\
-18.311	-17.5759421165425\\
-8.545	-8.20197834011553\\
-4.883	-4.68698188821347\\
-3.662	-3.51499645190205\\
-6.104	-5.85896732452489\\
-10.986	-10.5449893557062\\
-14.648	-14.0599858076082\\
-13.428	-12.888960228329\\
-23.193	-22.2619641477237\\
-18.311	-17.5759421165425\\
-23.193	-22.2619641477237\\
-40.283	-38.6659208279548\\
-30.518	-29.2929169085601\\
-25.635	-24.6059350203466\\
-32.959	-31.6359279241507\\
-36.621	-35.1509243760527\\
-26.855	-25.7769605996258\\
-21.973	-21.0909385684445\\
-15.869	-15.2319712439196\\
-30.518	-29.2929169085601\\
-48.828	-46.8678991680703\\
-57.373	-55.0698775081858\\
-46.387	-44.5248881524797\\
-39.063	-37.4948952486756\\
-30.518	-29.2929169085601\\
-37.842	-36.3229098123642\\
-28.076	-26.9489460359372\\
-19.531	-18.7469676958217\\
-25.635	-24.6059350203466\\
-29.297	-28.1209314722486\\
-15.869	-15.2319712439196\\
-14.648	-14.0599858076082\\
-24.414	-23.4339495840352\\
-28.076	-26.9489460359372\\
-29.297	-28.1209314722486\\
-20.752	-19.9189531321331\\
-17.09	-16.4039566802311\\
-15.869	-15.2319712439196\\
-10.986	-10.5449893557062\\
-9.766	-9.37396377642694\\
-8.545	-8.20197834011553\\
-6.104	-5.85896732452489\\
-7.324	-7.02999290380411\\
-12.207	-11.7169747920176\\
-18.311	-17.5759421165425\\
-13.428	-12.888960228329\\
-9.766	-9.37396377642694\\
-8.545	-8.20197834011553\\
-10.986	-10.5449893557062\\
-15.869	-15.2319712439196\\
-18.311	-17.5759421165425\\
-15.869	-15.2319712439196\\
-10.986	-10.5449893557062\\
-24.414	-23.4339495840352\\
-26.855	-25.7769605996258\\
-20.752	-19.9189531321331\\
-8.545	-8.20197834011553\\
-13.428	-12.888960228329\\
-17.09	-16.4039566802311\\
-12.207	-11.7169747920176\\
-8.545	-8.20197834011553\\
-12.207	-11.7169747920176\\
-26.855	-25.7769605996258\\
-13.428	-12.888960228329\\
-7.324	-7.02999290380411\\
-9.766	-9.37396377642694\\
-18.311	-17.5759421165425\\
-21.973	-21.0909385684445\\
-12.207	-11.7169747920176\\
-10.986	-10.5449893557062\\
-20.752	-19.9189531321331\\
-29.297	-28.1209314722486\\
-25.635	-24.6059350203466\\
-37.842	-36.3229098123642\\
-40.283	-38.6659208279548\\
-30.518	-29.2929169085601\\
-23.193	-22.2619641477237\\
-30.518	-29.2929169085601\\
-25.635	-24.6059350203466\\
-26.855	-25.7769605996258\\
-36.621	-35.1509243760527\\
-28.076	-26.9489460359372\\
-14.648	-14.0599858076082\\
-25.635	-24.6059350203466\\
-36.621	-35.1509243760527\\
-41.504	-39.8379062642662\\
-31.738	-30.4639424878393\\
-28.076	-26.9489460359372\\
-34.18	-32.8079133604621\\
-30.518	-29.2929169085601\\
-17.09	-16.4039566802311\\
-21.973	-21.0909385684445\\
-23.193	-22.2619641477237\\
-20.752	-19.9189531321331\\
-24.414	-23.4339495840352\\
-15.869	-15.2319712439196\\
-19.531	-18.7469676958217\\
-28.076	-26.9489460359372\\
-19.531	-18.7469676958217\\
-13.428	-12.888960228329\\
-10.986	-10.5449893557062\\
-8.545	-8.20197834011553\\
-17.09	-16.4039566802311\\
-18.311	-17.5759421165425\\
-19.531	-18.7469676958217\\
-26.855	-25.7769605996258\\
-31.738	-30.4639424878393\\
-28.076	-26.9489460359372\\
-21.973	-21.0909385684445\\
-13.428	-12.888960228329\\
-15.869	-15.2319712439196\\
-29.297	-28.1209314722486\\
-20.752	-19.9189531321331\\
-7.324	-7.02999290380411\\
-17.09	-16.4039566802311\\
-25.635	-24.6059350203466\\
-28.076	-26.9489460359372\\
-35.4	-33.9789389397413\\
-46.387	-44.5248881524797\\
-36.621	-35.1509243760527\\
-29.297	-28.1209314722486\\
-34.18	-32.8079133604621\\
-25.635	-24.6059350203466\\
-18.311	-17.5759421165425\\
-21.973	-21.0909385684445\\
-29.297	-28.1209314722486\\
-28.076	-26.9489460359372\\
-18.311	-17.5759421165425\\
-14.648	-14.0599858076082\\
-12.207	-11.7169747920176\\
-17.09	-16.4039566802311\\
-18.311	-17.5759421165425\\
-13.428	-12.888960228329\\
-24.414	-23.4339495840352\\
-30.518	-29.2929169085601\\
-18.311	-17.5759421165425\\
-13.428	-12.888960228329\\
-23.193	-22.2619641477237\\
-14.648	-14.0599858076082\\
-6.104	-5.85896732452489\\
-12.207	-11.7169747920176\\
-13.428	-12.888960228329\\
-10.986	-10.5449893557062\\
-7.324	-7.02999290380411\\
-6.104	-5.85896732452489\\
-7.324	-7.02999290380411\\
-10.986	-10.5449893557062\\
-20.752	-19.9189531321331\\
-25.635	-24.6059350203466\\
-29.297	-28.1209314722486\\
-32.959	-31.6359279241507\\
-18.311	-17.5759421165425\\
-12.207	-11.7169747920176\\
-28.076	-26.9489460359372\\
-41.504	-39.8379062642662\\
-31.738	-30.4639424878393\\
-29.297	-28.1209314722486\\
-45.166	-43.3529027161683\\
-34.18	-32.8079133604621\\
-30.518	-29.2929169085601\\
-25.635	-24.6059350203466\\
-23.193	-22.2619641477237\\
-31.738	-30.4639424878393\\
-24.414	-23.4339495840352\\
-20.752	-19.9189531321331\\
-30.518	-29.2929169085601\\
-40.283	-38.6659208279548\\
-15.869	-15.2319712439196\\
-9.766	-9.37396377642694\\
-28.076	-26.9489460359372\\
-20.752	-19.9189531321331\\
-8.545	-8.20197834011553\\
-14.648	-14.0599858076082\\
-19.531	-18.7469676958217\\
-28.076	-26.9489460359372\\
-18.311	-17.5759421165425\\
-13.428	-12.888960228329\\
-10.986	-10.5449893557062\\
-6.104	-5.85896732452489\\
-8.545	-8.20197834011553\\
-6.104	-5.85896732452489\\
-7.324	-7.02999290380411\\
-17.09	-16.4039566802311\\
-14.648	-14.0599858076082\\
-6.104	-5.85896732452489\\
-8.545	-8.20197834011553\\
-10.986	-10.5449893557062\\
-8.545	-8.20197834011553\\
-9.766	-9.37396377642694\\
-2.441	-2.34301101559064\\
-12.207	-11.7169747920176\\
-18.311	-17.5759421165425\\
-24.414	-23.4339495840352\\
-34.18	-32.8079133604621\\
-30.518	-29.2929169085601\\
-14.648	-14.0599858076082\\
-12.207	-11.7169747920176\\
-3.662	-3.51499645190205\\
-4.883	-4.68698188821347\\
-13.428	-12.888960228329\\
-18.311	-17.5759421165425\\
-14.648	-14.0599858076082\\
-18.311	-17.5759421165425\\
-19.531	-18.7469676958217\\
-25.635	-24.6059350203466\\
-20.752	-19.9189531321331\\
-34.18	-32.8079133604621\\
-18.311	-17.5759421165425\\
-12.207	-11.7169747920176\\
-8.545	-8.20197834011553\\
-15.869	-15.2319712439196\\
-13.428	-12.888960228329\\
-9.766	-9.37396377642694\\
-3.662	-3.51499645190205\\
-10.986	-10.5449893557062\\
-8.545	-8.20197834011553\\
-2.441	-2.34301101559064\\
-7.324	-7.02999290380411\\
-9.766	-9.37396377642694\\
-14.648	-14.0599858076082\\
-19.531	-18.7469676958217\\
-26.855	-25.7769605996258\\
-21.973	-21.0909385684445\\
-7.324	-7.02999290380411\\
-13.428	-12.888960228329\\
-21.973	-21.0909385684445\\
-10.986	-10.5449893557062\\
-8.545	-8.20197834011553\\
-21.973	-21.0909385684445\\
-18.311	-17.5759421165425\\
-28.076	-26.9489460359372\\
-36.621	-35.1509243760527\\
-23.193	-22.2619641477237\\
-18.311	-17.5759421165425\\
};
\addlegendentry{data2}

\end{axis}

\begin{axis}[%
width=4.927cm,
height=3.484cm,
at={(6.484cm,14.516cm)},
scale only axis,
xmin=-60,
xmax=0,
xlabel style={font=\color{white!15!black}},
xlabel={y(t-1)},
ymin=-50,
ymax=0,
ylabel style={font=\color{white!15!black}},
ylabel={y(t)},
axis background/.style={fill=white},
title={C2, R = 0.7625},
axis x line*=bottom,
axis y line*=left,
legend style={legend cell align=left, align=left, draw=white!15!black}
]
\addplot[only marks, mark=*, mark options={}, mark size=1.5000pt, color=mycolor1, fill=mycolor1] table[row sep=crcr]{%
x	y\\
-18.311	-19.531\\
-19.531	-24.414\\
-24.414	-19.531\\
-19.531	-20.752\\
-20.752	-25.635\\
-25.635	-24.414\\
-24.414	-28.076\\
-28.076	-23.193\\
-23.193	-13.428\\
-13.428	-17.09\\
-17.09	-17.09\\
-17.09	-18.311\\
-18.311	-15.869\\
-15.869	-8.545\\
-8.545	-6.104\\
-6.104	-7.324\\
-7.324	-9.766\\
-9.766	-21.973\\
-21.973	-23.193\\
-23.193	-23.193\\
-23.193	-19.531\\
-19.531	-10.986\\
-10.986	-17.09\\
-17.09	-19.531\\
-19.531	-13.428\\
-13.428	-18.311\\
-18.311	-19.531\\
-19.531	-14.648\\
-14.648	-24.414\\
-24.414	-21.973\\
-21.973	-15.869\\
-15.869	-23.193\\
-23.193	-25.635\\
-25.635	-19.531\\
-19.531	-17.09\\
-17.09	-14.648\\
-14.648	-17.09\\
-17.09	-20.752\\
-20.752	-18.311\\
-18.311	-18.311\\
-18.311	-19.531\\
-19.531	-20.752\\
-20.752	-28.076\\
-28.076	-25.635\\
-25.635	-17.09\\
-17.09	-15.869\\
-15.869	-12.207\\
-12.207	-10.986\\
-10.986	-15.869\\
-15.869	-12.207\\
-12.207	-12.207\\
-12.207	-15.869\\
-15.869	-18.311\\
-18.311	-15.869\\
-15.869	-25.635\\
-25.635	-21.973\\
-21.973	-12.207\\
-12.207	-9.766\\
-9.766	-13.428\\
-13.428	-15.869\\
-15.869	-10.986\\
-10.986	-10.986\\
-10.986	-12.207\\
-12.207	-9.766\\
-9.766	-8.545\\
-8.545	-12.207\\
-12.207	-14.648\\
-14.648	-17.09\\
-17.09	-18.311\\
-18.311	-23.193\\
-23.193	-21.973\\
-21.973	-21.973\\
-21.973	-17.09\\
-17.09	-17.09\\
-17.09	-17.09\\
-17.09	-15.869\\
-15.869	-17.09\\
-17.09	-24.414\\
-24.414	-35.4\\
-35.4	-36.621\\
-36.621	-34.18\\
-34.18	-24.414\\
-24.414	-29.297\\
-29.297	-36.621\\
-36.621	-40.283\\
-40.283	-31.738\\
-31.738	-37.842\\
-37.842	-48.828\\
-48.828	-39.063\\
-39.063	-30.518\\
-30.518	-25.635\\
-25.635	-19.531\\
-19.531	-14.648\\
-14.648	-19.531\\
-19.531	-15.869\\
-15.869	-9.766\\
-9.766	-9.766\\
-9.766	-12.207\\
-12.207	-17.09\\
-17.09	-15.869\\
-15.869	-15.869\\
-15.869	-19.531\\
-19.531	-19.531\\
-19.531	-28.076\\
-28.076	-23.193\\
-23.193	-25.635\\
-25.635	-23.193\\
-23.193	-18.311\\
-18.311	-20.752\\
-20.752	-17.09\\
-17.09	-14.648\\
-14.648	-18.311\\
-18.311	-17.09\\
-17.09	-15.869\\
-15.869	-13.428\\
-13.428	-10.986\\
-10.986	-12.207\\
-12.207	-13.428\\
-13.428	-10.986\\
-10.986	-9.766\\
-9.766	-14.648\\
-14.648	-19.531\\
-19.531	-23.193\\
-23.193	-23.193\\
-23.193	-17.09\\
-17.09	-18.311\\
-18.311	-30.518\\
-30.518	-23.193\\
-23.193	-24.414\\
-24.414	-26.855\\
-26.855	-17.09\\
-17.09	-10.986\\
-10.986	-17.09\\
-17.09	-18.311\\
-18.311	-26.855\\
-26.855	-34.18\\
-34.18	-31.738\\
-31.738	-25.635\\
-25.635	-24.414\\
-24.414	-23.193\\
-23.193	-23.193\\
-23.193	-21.973\\
-21.973	-18.311\\
-18.311	-14.648\\
-14.648	-13.428\\
-13.428	-12.207\\
-12.207	-13.428\\
-13.428	-19.531\\
-19.531	-25.635\\
-25.635	-21.973\\
-21.973	-14.648\\
-14.648	-13.428\\
-13.428	-14.648\\
-14.648	-12.207\\
-12.207	-8.545\\
-8.545	-8.545\\
-8.545	-13.428\\
-13.428	-12.207\\
-12.207	-12.207\\
-12.207	-12.207\\
-12.207	-12.207\\
-12.207	-8.545\\
-8.545	-7.324\\
-7.324	-4.883\\
-4.883	-7.324\\
-7.324	-17.09\\
-17.09	-24.414\\
-24.414	-24.414\\
-24.414	-19.531\\
-19.531	-12.207\\
-12.207	-12.207\\
-12.207	-8.545\\
-8.545	-10.986\\
-10.986	-14.648\\
-14.648	-14.648\\
-14.648	-24.414\\
-24.414	-34.18\\
-34.18	-40.283\\
-40.283	-32.959\\
-32.959	-31.738\\
-31.738	-23.193\\
-23.193	-25.635\\
-25.635	-26.855\\
-26.855	-28.076\\
-28.076	-21.973\\
-21.973	-20.752\\
-20.752	-19.531\\
-19.531	-19.531\\
-19.531	-21.973\\
-21.973	-19.531\\
-19.531	-23.193\\
-23.193	-29.297\\
-29.297	-24.414\\
-24.414	-14.648\\
-14.648	-15.869\\
-15.869	-25.635\\
-25.635	-32.959\\
-32.959	-35.4\\
-35.4	-39.063\\
-39.063	-37.842\\
-37.842	-30.518\\
-30.518	-26.855\\
-26.855	-30.518\\
-30.518	-35.4\\
-35.4	-25.635\\
-25.635	-14.648\\
-14.648	-19.531\\
-19.531	-20.752\\
-20.752	-12.207\\
-12.207	-13.428\\
-13.428	-18.311\\
-18.311	-14.648\\
-14.648	-12.207\\
-12.207	-18.311\\
-18.311	-10.986\\
-10.986	-14.648\\
-14.648	-24.414\\
-24.414	-21.973\\
-21.973	-14.648\\
-14.648	-14.648\\
-14.648	-20.752\\
-20.752	-34.18\\
-34.18	-28.076\\
-28.076	-15.869\\
-15.869	-14.648\\
-14.648	-14.648\\
-14.648	-12.207\\
-12.207	-10.986\\
-10.986	-10.986\\
-10.986	-13.428\\
-13.428	-17.09\\
-17.09	-19.531\\
-19.531	-14.648\\
-14.648	-19.531\\
-19.531	-26.855\\
-26.855	-23.193\\
-23.193	-17.09\\
-17.09	-21.973\\
-21.973	-25.635\\
-25.635	-21.973\\
-21.973	-8.545\\
-8.545	-12.207\\
-12.207	-13.428\\
-13.428	-15.869\\
-15.869	-15.869\\
-15.869	-10.986\\
-10.986	-17.09\\
-17.09	-15.869\\
-15.869	-10.986\\
-10.986	-14.648\\
-14.648	-14.648\\
-14.648	-12.207\\
-12.207	-14.648\\
-14.648	-10.986\\
-10.986	-9.766\\
-9.766	-12.207\\
-12.207	-8.545\\
-8.545	-17.09\\
-17.09	-18.311\\
-18.311	-20.752\\
-20.752	-29.297\\
-29.297	-20.752\\
-20.752	-10.986\\
-10.986	-10.986\\
-10.986	-8.545\\
-8.545	-13.428\\
-13.428	-12.207\\
-12.207	-9.766\\
-9.766	-15.869\\
-15.869	-19.531\\
-19.531	-20.752\\
-20.752	-23.193\\
-23.193	-23.193\\
-23.193	-17.09\\
-17.09	-15.869\\
-15.869	-10.986\\
-10.986	-10.986\\
-10.986	-7.324\\
-7.324	-8.545\\
-8.545	-10.986\\
-10.986	-17.09\\
-17.09	-18.311\\
-18.311	-17.09\\
-17.09	-24.414\\
-24.414	-25.635\\
-25.635	-15.869\\
-15.869	-19.531\\
-19.531	-23.193\\
-23.193	-26.855\\
-26.855	-25.635\\
-25.635	-17.09\\
-17.09	-10.986\\
-10.986	-7.324\\
-7.324	-6.104\\
-6.104	-8.545\\
-8.545	-9.766\\
-9.766	-8.545\\
-8.545	-10.986\\
-10.986	-17.09\\
-17.09	-15.869\\
-15.869	-13.428\\
-13.428	-15.869\\
-15.869	-12.207\\
-12.207	-6.104\\
-6.104	-9.766\\
-9.766	-20.752\\
-20.752	-17.09\\
-17.09	-18.311\\
-18.311	-15.869\\
-15.869	-10.986\\
-10.986	-17.09\\
-17.09	-23.193\\
-23.193	-23.193\\
-23.193	-17.09\\
-17.09	-26.855\\
-26.855	-25.635\\
-25.635	-18.311\\
-18.311	-23.193\\
-23.193	-25.635\\
-25.635	-19.531\\
-19.531	-15.869\\
-15.869	-24.414\\
-24.414	-35.4\\
-35.4	-29.297\\
-29.297	-17.09\\
-17.09	-17.09\\
-17.09	-18.311\\
-18.311	-20.752\\
-20.752	-23.193\\
-23.193	-19.531\\
-19.531	-17.09\\
-17.09	-24.414\\
-24.414	-25.635\\
-25.635	-18.311\\
-18.311	-15.869\\
-15.869	-18.311\\
-18.311	-28.076\\
-28.076	-26.855\\
-26.855	-24.414\\
-24.414	-19.531\\
-19.531	-17.09\\
-17.09	-18.311\\
-18.311	-13.428\\
-13.428	-6.104\\
-6.104	-6.104\\
-6.104	-6.104\\
-6.104	-9.766\\
-9.766	-19.531\\
-19.531	-15.869\\
-15.869	-14.648\\
-14.648	-13.428\\
-13.428	-17.09\\
-17.09	-13.428\\
-13.428	-9.766\\
-9.766	-12.207\\
-12.207	-9.766\\
-9.766	-8.545\\
-8.545	-17.09\\
-17.09	-20.752\\
-20.752	-17.09\\
-17.09	-10.986\\
-10.986	-10.986\\
-10.986	-13.428\\
-13.428	-9.766\\
-9.766	-10.986\\
-10.986	-28.076\\
-28.076	-20.752\\
-20.752	-25.635\\
-25.635	-35.4\\
-35.4	-30.518\\
-30.518	-24.414\\
-24.414	-18.311\\
-18.311	-18.311\\
-18.311	-19.531\\
-19.531	-21.973\\
-21.973	-25.635\\
-25.635	-25.635\\
-25.635	-31.738\\
-31.738	-34.18\\
-34.18	-41.504\\
-41.504	-32.959\\
-32.959	-28.076\\
-28.076	-35.4\\
-35.4	-26.855\\
-26.855	-29.297\\
-29.297	-28.076\\
-28.076	-17.09\\
-17.09	-10.986\\
-10.986	-12.207\\
-12.207	-17.09\\
-17.09	-14.648\\
-14.648	-9.766\\
-9.766	-13.428\\
-13.428	-13.428\\
-13.428	-7.324\\
-7.324	-8.545\\
-8.545	-12.207\\
-12.207	-7.324\\
-7.324	-14.648\\
-14.648	-20.752\\
-20.752	-13.428\\
-13.428	-21.973\\
-21.973	-30.518\\
-30.518	-29.297\\
-29.297	-35.4\\
-35.4	-29.297\\
-29.297	-28.076\\
-28.076	-23.193\\
-23.193	-12.207\\
-12.207	-14.648\\
-14.648	-15.869\\
-15.869	-10.986\\
-10.986	-12.207\\
-12.207	-13.428\\
-13.428	-3.662\\
-3.662	-8.545\\
-8.545	-17.09\\
-17.09	-19.531\\
-19.531	-20.752\\
-20.752	-23.193\\
-23.193	-20.752\\
-20.752	-23.193\\
-23.193	-18.311\\
-18.311	-15.869\\
-15.869	-15.869\\
-15.869	-12.207\\
-12.207	-18.311\\
-18.311	-17.09\\
-17.09	-12.207\\
-12.207	-17.09\\
-17.09	-23.193\\
-23.193	-17.09\\
-17.09	-13.428\\
-13.428	-12.207\\
-12.207	-14.648\\
-14.648	-7.324\\
-7.324	-8.545\\
-8.545	-12.207\\
-12.207	-17.09\\
-17.09	-17.09\\
-17.09	-14.648\\
-14.648	-15.869\\
-15.869	-18.311\\
-18.311	-12.207\\
-12.207	-9.766\\
-9.766	-18.311\\
-18.311	-26.855\\
-26.855	-26.855\\
-26.855	-21.973\\
-21.973	-19.531\\
-19.531	-17.09\\
-17.09	-14.648\\
-14.648	-13.428\\
-13.428	-17.09\\
-17.09	-17.09\\
-17.09	-10.986\\
-10.986	-14.648\\
-14.648	-18.311\\
-18.311	-19.531\\
-19.531	-21.973\\
-21.973	-21.973\\
-21.973	-29.297\\
-29.297	-35.4\\
-35.4	-36.621\\
-36.621	-26.855\\
-26.855	-14.648\\
-14.648	-10.986\\
-10.986	-7.324\\
-7.324	-10.986\\
-10.986	-10.986\\
-10.986	-14.648\\
-14.648	-13.428\\
-13.428	-12.207\\
-12.207	-15.869\\
-15.869	-20.752\\
-20.752	-20.752\\
-20.752	-24.414\\
-24.414	-31.738\\
-31.738	-29.297\\
-29.297	-26.855\\
-26.855	-18.311\\
-18.311	-17.09\\
-17.09	-20.752\\
-20.752	-20.752\\
-20.752	-17.09\\
-17.09	-14.648\\
-14.648	-18.311\\
-18.311	-12.207\\
-12.207	-13.428\\
-13.428	-20.752\\
-20.752	-17.09\\
-17.09	-18.311\\
-18.311	-32.959\\
-32.959	-25.635\\
-25.635	-20.752\\
-20.752	-28.076\\
-28.076	-28.076\\
-28.076	-19.531\\
-19.531	-13.428\\
-13.428	-13.428\\
-13.428	-19.531\\
-19.531	-30.518\\
-30.518	-34.18\\
-34.18	-31.738\\
-31.738	-25.635\\
-25.635	-17.09\\
-17.09	-14.648\\
-14.648	-13.428\\
-13.428	-10.986\\
-10.986	-8.545\\
-8.545	-12.207\\
-12.207	-14.648\\
-14.648	-10.986\\
-10.986	-13.428\\
-13.428	-19.531\\
-19.531	-20.752\\
-20.752	-21.973\\
-21.973	-29.297\\
-29.297	-34.18\\
-34.18	-25.635\\
-25.635	-23.193\\
-23.193	-24.414\\
-24.414	-20.752\\
-20.752	-14.648\\
-14.648	-18.311\\
-18.311	-29.297\\
-29.297	-31.738\\
-31.738	-19.531\\
-19.531	-12.207\\
-12.207	-8.545\\
-8.545	-7.324\\
-7.324	-9.766\\
-9.766	-8.545\\
-8.545	-12.207\\
-12.207	-14.648\\
-14.648	-13.428\\
-13.428	-9.766\\
-9.766	-10.986\\
-10.986	-10.986\\
-10.986	-9.766\\
-9.766	-10.986\\
-10.986	-14.648\\
-14.648	-21.973\\
-21.973	-18.311\\
-18.311	-20.752\\
-20.752	-21.973\\
-21.973	-30.518\\
-30.518	-31.738\\
-31.738	-34.18\\
-34.18	-34.18\\
-34.18	-24.414\\
-24.414	-17.09\\
-17.09	-15.869\\
-15.869	-26.855\\
-26.855	-31.738\\
-31.738	-23.193\\
-23.193	-18.311\\
-18.311	-9.766\\
-9.766	-4.883\\
-4.883	-6.104\\
-6.104	-6.104\\
-6.104	-7.324\\
-7.324	-14.648\\
-14.648	-10.986\\
-10.986	-12.207\\
-12.207	-8.545\\
-8.545	-4.883\\
-4.883	-8.545\\
-8.545	-8.545\\
-8.545	-8.545\\
-8.545	-9.766\\
-9.766	-7.324\\
-7.324	-10.986\\
-10.986	-24.414\\
-24.414	-26.855\\
-26.855	-25.635\\
-25.635	-21.973\\
-21.973	-30.518\\
-30.518	-40.283\\
-40.283	-34.18\\
-34.18	-31.738\\
-31.738	-26.855\\
-26.855	-28.076\\
-28.076	-29.297\\
-29.297	-20.752\\
-20.752	-17.09\\
-17.09	-14.648\\
-14.648	-12.207\\
-12.207	-20.752\\
-20.752	-18.311\\
-18.311	-15.869\\
-15.869	-18.311\\
-18.311	-18.311\\
-18.311	-14.648\\
-14.648	-17.09\\
-17.09	-17.09\\
-17.09	-20.752\\
-20.752	-19.531\\
-19.531	-13.428\\
-13.428	-17.09\\
-17.09	-9.766\\
-9.766	-4.883\\
-4.883	-10.986\\
-10.986	-13.428\\
-13.428	-8.545\\
-8.545	-2.441\\
-2.441	-8.545\\
-8.545	-12.207\\
-12.207	-12.207\\
-12.207	-15.869\\
-15.869	-13.428\\
-13.428	-18.311\\
-18.311	-17.09\\
-17.09	-10.986\\
-10.986	-17.09\\
-17.09	-14.648\\
-14.648	-9.766\\
-9.766	-7.324\\
-7.324	-6.104\\
-6.104	-7.324\\
-7.324	-8.545\\
-8.545	-6.104\\
-6.104	-13.428\\
-13.428	-17.09\\
-17.09	-10.986\\
-10.986	-14.648\\
-14.648	-10.986\\
-10.986	-14.648\\
-14.648	-10.986\\
-10.986	-9.766\\
-9.766	-9.766\\
-9.766	-7.324\\
-7.324	-8.545\\
-8.545	-10.986\\
-10.986	-14.648\\
-14.648	-12.207\\
-12.207	-8.545\\
-8.545	-8.545\\
-8.545	-8.545\\
-8.545	-10.986\\
-10.986	-21.973\\
-21.973	-24.414\\
-24.414	-18.311\\
-18.311	-17.09\\
-17.09	-13.428\\
-13.428	-19.531\\
-19.531	-17.09\\
-17.09	-9.766\\
-9.766	-14.648\\
-14.648	-13.428\\
-13.428	-15.869\\
-15.869	-12.207\\
-12.207	-14.648\\
-14.648	-12.207\\
-12.207	-15.869\\
-15.869	-12.207\\
-12.207	-13.428\\
-13.428	-14.648\\
-14.648	-19.531\\
-19.531	-18.311\\
-18.311	-21.973\\
-21.973	-30.518\\
-30.518	-24.414\\
-24.414	-14.648\\
-14.648	-14.648\\
-14.648	-15.869\\
-15.869	-23.193\\
-23.193	-18.311\\
-18.311	-23.193\\
-23.193	-17.09\\
-17.09	-8.545\\
-8.545	-9.766\\
-9.766	-19.531\\
-19.531	-14.648\\
-14.648	-13.428\\
-13.428	-20.752\\
-20.752	-21.973\\
-21.973	-18.311\\
-18.311	-14.648\\
-14.648	-18.311\\
-18.311	-20.752\\
-20.752	-18.311\\
-18.311	-15.869\\
-15.869	-15.869\\
-15.869	-12.207\\
-12.207	-14.648\\
-14.648	-12.207\\
-12.207	-13.428\\
-13.428	-19.531\\
-19.531	-20.752\\
-20.752	-19.531\\
-19.531	-19.531\\
-19.531	-14.648\\
-14.648	-10.986\\
-10.986	-13.428\\
-13.428	-14.648\\
-14.648	-18.311\\
-18.311	-20.752\\
-20.752	-24.414\\
-24.414	-18.311\\
-18.311	-12.207\\
-12.207	-21.973\\
-21.973	-30.518\\
-30.518	-20.752\\
-20.752	-20.752\\
-20.752	-25.635\\
-25.635	-18.311\\
-18.311	-17.09\\
-17.09	-18.311\\
-18.311	-15.869\\
-15.869	-18.311\\
-18.311	-21.973\\
-21.973	-17.09\\
-17.09	-20.752\\
-20.752	-20.752\\
-20.752	-18.311\\
-18.311	-14.648\\
-14.648	-21.973\\
-21.973	-14.648\\
-14.648	-17.09\\
-17.09	-17.09\\
-17.09	-17.09\\
-17.09	-10.986\\
-10.986	-6.104\\
-6.104	-4.883\\
-4.883	-10.986\\
-10.986	-20.752\\
-20.752	-21.973\\
-21.973	-19.531\\
-19.531	-13.428\\
-13.428	-17.09\\
-17.09	-14.648\\
-14.648	-14.648\\
-14.648	-9.766\\
-9.766	-15.869\\
-15.869	-14.648\\
-14.648	-12.207\\
-12.207	-13.428\\
-13.428	-12.207\\
-12.207	-9.766\\
-9.766	-7.324\\
-7.324	-8.545\\
-8.545	-18.311\\
-18.311	-13.428\\
-13.428	-18.311\\
-18.311	-15.869\\
-15.869	-21.973\\
-21.973	-19.531\\
-19.531	-25.635\\
-25.635	-15.869\\
-15.869	-23.193\\
-23.193	-21.973\\
-21.973	-13.428\\
-13.428	-17.09\\
-17.09	-14.648\\
-14.648	-19.531\\
-19.531	-20.752\\
-20.752	-24.414\\
-24.414	-17.09\\
-17.09	-24.414\\
-24.414	-25.635\\
-25.635	-23.193\\
-23.193	-18.311\\
-18.311	-14.648\\
-14.648	-19.531\\
-19.531	-17.09\\
-17.09	-14.648\\
-14.648	-12.207\\
-12.207	-13.428\\
-13.428	-12.207\\
-12.207	-25.635\\
-25.635	-30.518\\
-30.518	-25.635\\
-25.635	-23.193\\
-23.193	-25.635\\
-25.635	-14.648\\
-14.648	-13.428\\
-13.428	-9.766\\
-9.766	-9.766\\
-9.766	-12.207\\
-12.207	-18.311\\
-18.311	-19.531\\
-19.531	-23.193\\
-23.193	-18.311\\
-18.311	-15.869\\
-15.869	-19.531\\
-19.531	-28.076\\
-28.076	-28.076\\
-28.076	-23.193\\
-23.193	-37.842\\
-37.842	-25.635\\
-25.635	-15.869\\
-15.869	-13.428\\
-13.428	-12.207\\
-12.207	-19.531\\
-19.531	-23.193\\
-23.193	-28.076\\
-28.076	-24.414\\
-24.414	-18.311\\
-18.311	-10.986\\
-10.986	-14.648\\
-14.648	-10.986\\
-10.986	-13.428\\
-13.428	-8.545\\
-8.545	-10.986\\
-10.986	-8.545\\
-8.545	-7.324\\
-7.324	-9.766\\
-9.766	-13.428\\
-13.428	-15.869\\
-15.869	-17.09\\
-17.09	-17.09\\
-17.09	-15.869\\
-15.869	-19.531\\
-19.531	-12.207\\
-12.207	-21.973\\
-21.973	-23.193\\
-23.193	-20.752\\
-20.752	-18.311\\
-18.311	-18.311\\
-18.311	-19.531\\
-19.531	-24.414\\
-24.414	-19.531\\
-19.531	-15.869\\
-15.869	-18.311\\
-18.311	-17.09\\
-17.09	-13.428\\
-13.428	-20.752\\
-20.752	-25.635\\
-25.635	-23.193\\
-23.193	-25.635\\
-25.635	-32.959\\
-32.959	-24.414\\
-24.414	-23.193\\
-23.193	-19.531\\
-19.531	-23.193\\
-23.193	-29.297\\
-29.297	-20.752\\
-20.752	-19.531\\
-19.531	-24.414\\
-24.414	-17.09\\
-17.09	-10.986\\
-10.986	-15.869\\
-15.869	-8.545\\
-8.545	-8.545\\
-8.545	-8.545\\
-8.545	-8.545\\
-8.545	-9.766\\
-9.766	-10.986\\
-10.986	-14.648\\
-14.648	-17.09\\
-17.09	-19.531\\
-19.531	-20.752\\
-20.752	-21.973\\
-21.973	-25.635\\
-25.635	-24.414\\
-24.414	-17.09\\
-17.09	-20.752\\
-20.752	-17.09\\
-17.09	-19.531\\
-19.531	-24.414\\
-24.414	-21.973\\
-21.973	-23.193\\
-23.193	-20.752\\
-20.752	-19.531\\
-19.531	-26.855\\
-26.855	-21.973\\
-21.973	-23.193\\
-23.193	-21.973\\
-21.973	-14.648\\
-14.648	-9.766\\
-9.766	-8.545\\
-8.545	-13.428\\
-13.428	-12.207\\
-12.207	-17.09\\
-17.09	-14.648\\
-14.648	-18.311\\
-18.311	-25.635\\
-25.635	-30.518\\
-30.518	-21.973\\
-21.973	-26.855\\
-26.855	-23.193\\
-23.193	-18.311\\
-18.311	-19.531\\
-19.531	-19.531\\
-19.531	-18.311\\
-18.311	-20.752\\
-20.752	-25.635\\
-25.635	-32.959\\
-32.959	-29.297\\
-29.297	-29.297\\
-29.297	-30.518\\
-30.518	-23.193\\
-23.193	-25.635\\
-25.635	-19.531\\
-19.531	-18.311\\
-18.311	-24.414\\
-24.414	-29.297\\
-29.297	-9.766\\
-9.766	-19.531\\
-19.531	-12.207\\
-12.207	-18.311\\
-18.311	-18.311\\
-18.311	-10.986\\
-10.986	-12.207\\
-12.207	-23.193\\
-23.193	-37.842\\
-37.842	-35.4\\
-35.4	-34.18\\
-34.18	-26.855\\
-26.855	-31.738\\
-31.738	-36.621\\
-36.621	-28.076\\
-28.076	-30.518\\
-30.518	-31.738\\
-31.738	-23.193\\
-23.193	-18.311\\
-18.311	-24.414\\
-24.414	-17.09\\
-17.09	-13.428\\
-13.428	-18.311\\
-18.311	-12.207\\
-12.207	-17.09\\
-17.09	-14.648\\
-14.648	-18.311\\
-18.311	-20.752\\
-20.752	-15.869\\
-15.869	-12.207\\
-12.207	-15.869\\
-15.869	-18.311\\
-18.311	-20.752\\
-20.752	-18.311\\
-18.311	-18.311\\
-18.311	-24.414\\
-24.414	-23.193\\
-23.193	-15.869\\
-15.869	-25.635\\
-25.635	-20.752\\
-20.752	-17.09\\
-17.09	-14.648\\
-14.648	-9.766\\
-9.766	-7.324\\
-7.324	-15.869\\
-15.869	-13.428\\
-13.428	-10.986\\
-10.986	-7.324\\
-7.324	-12.207\\
-12.207	-14.648\\
-14.648	-15.869\\
-15.869	-19.531\\
-19.531	-23.193\\
-23.193	-21.973\\
-21.973	-24.414\\
-24.414	-17.09\\
-17.09	-8.545\\
-8.545	-14.648\\
-14.648	-9.766\\
-9.766	-13.428\\
-13.428	-17.09\\
-17.09	-18.311\\
-18.311	-28.076\\
-28.076	-18.311\\
-18.311	-21.973\\
-21.973	-23.193\\
-23.193	-18.311\\
-18.311	-14.648\\
-14.648	-13.428\\
-13.428	-17.09\\
-17.09	-18.311\\
-18.311	-26.855\\
-26.855	-15.869\\
-15.869	-15.869\\
-15.869	-15.869\\
-15.869	-17.09\\
-17.09	-20.752\\
-20.752	-32.959\\
-32.959	-28.076\\
-28.076	-28.076\\
-28.076	-19.531\\
-19.531	-13.428\\
-13.428	-15.869\\
-15.869	-9.766\\
-9.766	-6.104\\
-6.104	-17.09\\
-17.09	-17.09\\
-17.09	-19.531\\
-19.531	-25.635\\
-25.635	-28.076\\
-28.076	-35.4\\
-35.4	-31.738\\
-31.738	-18.311\\
-18.311	-18.311\\
-18.311	-13.428\\
-13.428	-18.311\\
-18.311	-24.414\\
-24.414	-29.297\\
-29.297	-23.193\\
-23.193	-18.311\\
-18.311	-20.752\\
-20.752	-25.635\\
-25.635	-18.311\\
-18.311	-13.428\\
-13.428	-12.207\\
-12.207	-7.324\\
-7.324	-8.545\\
-8.545	-4.883\\
-4.883	-12.207\\
-12.207	-15.869\\
-15.869	-15.869\\
-15.869	-12.207\\
-12.207	-13.428\\
-13.428	-7.324\\
-7.324	-3.662\\
-3.662	-7.324\\
-7.324	-10.986\\
-10.986	-12.207\\
-12.207	-14.648\\
-14.648	-18.311\\
-18.311	-20.752\\
-20.752	-24.414\\
-24.414	-34.18\\
-34.18	-32.959\\
-32.959	-36.621\\
-36.621	-31.738\\
-31.738	-20.752\\
-20.752	-18.311\\
-18.311	-18.311\\
-18.311	-21.973\\
-21.973	-17.09\\
-17.09	-15.869\\
-15.869	-12.207\\
-12.207	-12.207\\
-12.207	-17.09\\
-17.09	-14.648\\
-14.648	-8.545\\
-8.545	-7.324\\
-7.324	-10.986\\
-10.986	-14.648\\
-14.648	-13.428\\
-13.428	-14.648\\
-14.648	-12.207\\
-12.207	-15.869\\
-15.869	-23.193\\
-23.193	-17.09\\
-17.09	-24.414\\
-24.414	-32.959\\
-32.959	-35.4\\
-35.4	-26.855\\
-26.855	-19.531\\
-19.531	-14.648\\
-14.648	-9.766\\
-9.766	-15.869\\
-15.869	-10.986\\
-10.986	-17.09\\
-17.09	-14.648\\
-14.648	-12.207\\
-12.207	-17.09\\
-17.09	-12.207\\
-12.207	-12.207\\
-12.207	-13.428\\
-13.428	-9.766\\
-9.766	-9.766\\
-9.766	-8.545\\
-8.545	-8.545\\
-8.545	-7.324\\
-7.324	-6.104\\
-6.104	-6.104\\
-6.104	-9.766\\
-9.766	-10.986\\
-10.986	-8.545\\
-8.545	-15.869\\
-15.869	-17.09\\
-17.09	-14.648\\
-14.648	-14.648\\
-14.648	-20.752\\
-20.752	-25.635\\
-25.635	-18.311\\
-18.311	-21.973\\
-21.973	-9.766\\
-9.766	-6.104\\
-6.104	-14.648\\
-14.648	-18.311\\
-18.311	-19.531\\
-19.531	-26.855\\
-26.855	-25.635\\
-25.635	-17.09\\
-17.09	-13.428\\
-13.428	-15.869\\
-15.869	-14.648\\
-14.648	-10.986\\
-10.986	-7.324\\
-7.324	-12.207\\
-12.207	-7.324\\
-7.324	-6.104\\
-6.104	-7.324\\
-7.324	-9.766\\
-9.766	-13.428\\
-13.428	-13.428\\
-13.428	-9.766\\
-9.766	-17.09\\
-17.09	-23.193\\
-23.193	-14.648\\
-14.648	-20.752\\
-20.752	-23.193\\
-23.193	-15.869\\
-15.869	-10.986\\
-10.986	-14.648\\
-14.648	-14.648\\
-14.648	-9.766\\
-9.766	-12.207\\
-12.207	-7.324\\
-7.324	-6.104\\
-6.104	-6.104\\
-6.104	-8.545\\
-8.545	-12.207\\
-12.207	-10.986\\
-10.986	-12.207\\
-12.207	-14.648\\
-14.648	-9.766\\
-9.766	-7.324\\
-7.324	-6.104\\
-6.104	-8.545\\
-8.545	-8.545\\
-8.545	-9.766\\
-9.766	-18.311\\
-18.311	-18.311\\
-18.311	-14.648\\
-14.648	-14.648\\
-14.648	-19.531\\
-19.531	-24.414\\
-24.414	-25.635\\
-25.635	-26.855\\
-26.855	-21.973\\
-21.973	-21.973\\
-21.973	-21.973\\
-21.973	-23.193\\
-23.193	-25.635\\
-25.635	-30.518\\
-30.518	-26.855\\
-26.855	-24.414\\
-24.414	-20.752\\
-20.752	-19.531\\
-19.531	-19.531\\
-19.531	-24.414\\
-24.414	-15.869\\
-15.869	-19.531\\
-19.531	-21.973\\
-21.973	-24.414\\
-24.414	-17.09\\
-17.09	-20.752\\
-20.752	-18.311\\
-18.311	-17.09\\
-17.09	-12.207\\
-12.207	-17.09\\
-17.09	-24.414\\
-24.414	-19.531\\
-19.531	-10.986\\
-10.986	-14.648\\
-14.648	-12.207\\
-12.207	-9.766\\
-9.766	-10.986\\
-10.986	-7.324\\
-7.324	-9.766\\
-9.766	-8.545\\
-8.545	-6.104\\
-6.104	-7.324\\
-7.324	-9.766\\
-9.766	-7.324\\
-7.324	-8.545\\
-8.545	-9.766\\
-9.766	-10.986\\
-10.986	-13.428\\
-13.428	-12.207\\
-12.207	-15.869\\
-15.869	-13.428\\
-13.428	-14.648\\
-14.648	-24.414\\
-24.414	-17.09\\
-17.09	-15.869\\
-15.869	-18.311\\
-18.311	-13.428\\
-13.428	-8.545\\
-8.545	-19.531\\
-19.531	-10.986\\
-10.986	-8.545\\
-8.545	-10.986\\
-10.986	-13.428\\
-13.428	-12.207\\
-12.207	-10.986\\
-10.986	-6.104\\
-6.104	-6.104\\
-6.104	-6.104\\
-6.104	-12.207\\
-12.207	-10.986\\
-10.986	-12.207\\
-12.207	-15.869\\
-15.869	-19.531\\
-19.531	-15.869\\
-15.869	-20.752\\
-20.752	-24.414\\
-24.414	-20.752\\
-20.752	-20.752\\
-20.752	-31.738\\
-31.738	-23.193\\
-23.193	-30.518\\
-30.518	-24.414\\
-24.414	-26.855\\
-26.855	-30.518\\
-30.518	-20.752\\
-20.752	-10.986\\
-10.986	-12.207\\
-12.207	-15.869\\
-15.869	-18.311\\
-18.311	-18.311\\
-18.311	-15.869\\
-15.869	-9.766\\
-9.766	-10.986\\
-10.986	-18.311\\
-18.311	-24.414\\
-24.414	-23.193\\
-23.193	-17.09\\
-17.09	-12.207\\
-12.207	-14.648\\
-14.648	-24.414\\
-24.414	-25.635\\
-25.635	-26.855\\
-26.855	-20.752\\
-20.752	-26.855\\
-26.855	-29.297\\
-29.297	-26.855\\
-26.855	-35.4\\
-35.4	-31.738\\
-31.738	-26.855\\
-26.855	-31.738\\
-31.738	-26.855\\
-26.855	-17.09\\
-17.09	-14.648\\
-14.648	-18.311\\
-18.311	-20.752\\
-20.752	-21.973\\
-21.973	-18.311\\
-18.311	-20.752\\
-20.752	-18.311\\
-18.311	-14.648\\
-14.648	-17.09\\
-17.09	-17.09\\
-17.09	-25.635\\
-25.635	-24.414\\
-24.414	-20.752\\
-20.752	-14.648\\
-14.648	-12.207\\
-12.207	-13.428\\
-13.428	-14.648\\
-14.648	-9.766\\
-9.766	-8.545\\
-8.545	-14.648\\
-14.648	-17.09\\
-17.09	-20.752\\
-20.752	-19.531\\
-19.531	-12.207\\
-12.207	-8.545\\
-8.545	-9.766\\
-9.766	-8.545\\
-8.545	-13.428\\
-13.428	-14.648\\
-14.648	-13.428\\
-13.428	-19.531\\
-19.531	-25.635\\
-25.635	-24.414\\
-24.414	-26.855\\
-26.855	-30.518\\
-30.518	-17.09\\
-17.09	-17.09\\
-17.09	-14.648\\
-14.648	-15.869\\
-15.869	-18.311\\
-18.311	-15.869\\
-15.869	-10.986\\
-10.986	-9.766\\
-9.766	-17.09\\
-17.09	-20.752\\
-20.752	-18.311\\
-18.311	-26.855\\
-26.855	-31.738\\
-31.738	-21.973\\
-21.973	-14.648\\
-14.648	-13.428\\
-13.428	-10.986\\
-10.986	-18.311\\
-18.311	-17.09\\
-17.09	-15.869\\
-15.869	-18.311\\
-18.311	-23.193\\
-23.193	-20.752\\
-20.752	-24.414\\
-24.414	-17.09\\
-17.09	-18.311\\
-18.311	-14.648\\
-14.648	-12.207\\
-12.207	-17.09\\
-17.09	-24.414\\
-24.414	-25.635\\
-25.635	-19.531\\
-19.531	-35.4\\
-35.4	-26.855\\
-26.855	-20.752\\
-20.752	-13.428\\
-13.428	-17.09\\
-17.09	-17.09\\
-17.09	-15.869\\
-15.869	-17.09\\
-17.09	-14.648\\
-14.648	-13.428\\
-13.428	-28.076\\
-28.076	-30.518\\
-30.518	-18.311\\
-18.311	-23.193\\
-23.193	-24.414\\
-24.414	-24.414\\
-24.414	-18.311\\
-18.311	-18.311\\
-18.311	-15.869\\
-15.869	-18.311\\
-18.311	-21.973\\
-21.973	-28.076\\
-28.076	-32.959\\
-32.959	-42.725\\
-42.725	-35.4\\
-35.4	-24.414\\
-24.414	-21.973\\
-21.973	-21.973\\
-21.973	-18.311\\
-18.311	-13.428\\
-13.428	-12.207\\
-12.207	-13.428\\
-13.428	-8.545\\
-8.545	-8.545\\
-8.545	-9.766\\
-9.766	-14.648\\
-14.648	-12.207\\
-12.207	-8.545\\
-8.545	-7.324\\
-7.324	-14.648\\
-14.648	-23.193\\
-23.193	-12.207\\
-12.207	-12.207\\
-12.207	-17.09\\
-17.09	-14.648\\
-14.648	-17.09\\
-17.09	-15.869\\
-15.869	-10.986\\
-10.986	-9.766\\
-9.766	-6.104\\
-6.104	-8.545\\
-8.545	-15.869\\
-15.869	-18.311\\
-18.311	-13.428\\
-13.428	-9.766\\
-9.766	-7.324\\
-7.324	-12.207\\
-12.207	-13.428\\
-13.428	-18.311\\
-18.311	-18.311\\
-18.311	-14.648\\
-14.648	-12.207\\
-12.207	-13.428\\
-13.428	-18.311\\
-18.311	-25.635\\
-25.635	-28.076\\
-28.076	-23.193\\
-23.193	-14.648\\
-14.648	-14.648\\
-14.648	-15.869\\
-15.869	-14.648\\
-14.648	-12.207\\
-12.207	-12.207\\
-12.207	-14.648\\
-14.648	-14.648\\
-14.648	-10.986\\
-10.986	-8.545\\
-8.545	-8.545\\
-8.545	-13.428\\
-13.428	-17.09\\
-17.09	-21.973\\
-21.973	-23.193\\
-23.193	-23.193\\
-23.193	-23.193\\
-23.193	-30.518\\
-30.518	-40.283\\
-40.283	-34.18\\
-34.18	-23.193\\
-23.193	-26.855\\
-26.855	-29.297\\
-29.297	-36.621\\
-36.621	-28.076\\
-28.076	-14.648\\
-14.648	-9.766\\
-9.766	-12.207\\
-12.207	-9.766\\
-9.766	-6.104\\
-6.104	-7.324\\
-7.324	-9.766\\
-9.766	-12.207\\
-12.207	-12.207\\
-12.207	-10.986\\
-10.986	-9.766\\
-9.766	-10.986\\
-10.986	-14.648\\
-14.648	-14.648\\
-14.648	-13.428\\
-13.428	-10.986\\
-10.986	-7.324\\
-7.324	-7.324\\
-7.324	-10.986\\
-10.986	-12.207\\
-12.207	-9.766\\
-9.766	-7.324\\
-7.324	-10.986\\
-10.986	-13.428\\
-13.428	-9.766\\
-9.766	-13.428\\
-13.428	-14.648\\
-14.648	-19.531\\
-19.531	-13.428\\
-13.428	-15.869\\
-15.869	-17.09\\
-17.09	-15.869\\
-15.869	-13.428\\
-13.428	-18.311\\
-18.311	-25.635\\
-25.635	-23.193\\
-23.193	-25.635\\
-25.635	-21.973\\
-21.973	-19.531\\
-19.531	-18.311\\
-18.311	-21.973\\
-21.973	-19.531\\
-19.531	-20.752\\
-20.752	-23.193\\
-23.193	-23.193\\
-23.193	-40.283\\
-40.283	-34.18\\
-34.18	-17.09\\
-17.09	-12.207\\
-12.207	-14.648\\
-14.648	-17.09\\
-17.09	-20.752\\
-20.752	-23.193\\
-23.193	-24.414\\
-24.414	-26.855\\
-26.855	-19.531\\
-19.531	-15.869\\
-15.869	-19.531\\
-19.531	-20.752\\
-20.752	-12.207\\
-12.207	-10.986\\
-10.986	-13.428\\
-13.428	-13.428\\
-13.428	-17.09\\
-17.09	-23.193\\
-23.193	-21.973\\
-21.973	-14.648\\
-14.648	-12.207\\
-12.207	-18.311\\
-18.311	-23.193\\
-23.193	-26.855\\
-26.855	-31.738\\
-31.738	-36.621\\
-36.621	-30.518\\
-30.518	-29.297\\
-29.297	-19.531\\
-19.531	-13.428\\
-13.428	-21.973\\
-21.973	-28.076\\
-28.076	-18.311\\
-18.311	-20.752\\
-20.752	-35.4\\
-35.4	-37.842\\
-37.842	-25.635\\
-25.635	-17.09\\
-17.09	-9.766\\
-9.766	-12.207\\
-12.207	-14.648\\
-14.648	-12.207\\
-12.207	-10.986\\
-10.986	-10.986\\
-10.986	-12.207\\
-12.207	-8.545\\
-8.545	-12.207\\
-12.207	-15.869\\
-15.869	-18.311\\
-18.311	-17.09\\
-17.09	-19.531\\
-19.531	-28.076\\
-28.076	-24.414\\
-24.414	-17.09\\
-17.09	-15.869\\
-15.869	-19.531\\
-19.531	-14.648\\
-14.648	-10.986\\
-10.986	-10.986\\
-10.986	-13.428\\
-13.428	-12.207\\
-12.207	-9.766\\
-9.766	-6.104\\
-6.104	-6.104\\
-6.104	-8.545\\
-8.545	-14.648\\
-14.648	-13.428\\
-13.428	-10.986\\
-10.986	-8.545\\
-8.545	-15.869\\
-15.869	-18.311\\
-18.311	-10.986\\
-10.986	-13.428\\
-13.428	-15.869\\
-15.869	-17.09\\
-17.09	-17.09\\
-17.09	-19.531\\
-19.531	-19.531\\
-19.531	-13.428\\
-13.428	-9.766\\
-9.766	-14.648\\
-14.648	-10.986\\
-10.986	-13.428\\
-13.428	-19.531\\
-19.531	-25.635\\
-25.635	-21.973\\
-21.973	-18.311\\
-18.311	-26.855\\
-26.855	-24.414\\
-24.414	-14.648\\
-14.648	-10.986\\
-10.986	-10.986\\
-10.986	-9.766\\
-9.766	-8.545\\
-8.545	-7.324\\
-7.324	-14.648\\
-14.648	-13.428\\
-13.428	-8.545\\
-8.545	-12.207\\
-12.207	-13.428\\
-13.428	-9.766\\
-9.766	-8.545\\
-8.545	-4.883\\
-4.883	-7.324\\
-7.324	-14.648\\
-14.648	-19.531\\
-19.531	-17.09\\
-17.09	-12.207\\
-12.207	-14.648\\
-14.648	-14.648\\
-14.648	-15.869\\
-15.869	-20.752\\
-20.752	-23.193\\
-23.193	-23.193\\
-23.193	-19.531\\
-19.531	-10.986\\
-10.986	-10.986\\
-10.986	-12.207\\
-12.207	-15.869\\
-15.869	-14.648\\
-14.648	-8.545\\
-8.545	-17.09\\
-17.09	-20.752\\
-20.752	-18.311\\
-18.311	-20.752\\
-20.752	-25.635\\
-25.635	-19.531\\
-19.531	-18.311\\
-18.311	-26.855\\
-26.855	-21.973\\
-21.973	-24.414\\
-24.414	-25.635\\
-25.635	-37.842\\
-37.842	-39.063\\
-39.063	-25.635\\
-25.635	-32.959\\
-32.959	-36.621\\
-36.621	-35.4\\
-35.4	-37.842\\
-37.842	-36.621\\
-36.621	-43.945\\
-43.945	-41.504\\
-41.504	-24.414\\
-24.414	-19.531\\
-19.531	-18.311\\
-18.311	-15.869\\
-15.869	-12.207\\
-12.207	-17.09\\
-17.09	-23.193\\
-23.193	-18.311\\
-18.311	-13.428\\
-13.428	-15.869\\
-15.869	-12.207\\
-12.207	-10.986\\
-10.986	-10.986\\
-10.986	-7.324\\
-7.324	-8.545\\
-8.545	-15.869\\
-15.869	-8.545\\
-8.545	-8.545\\
-8.545	-6.104\\
-6.104	-3.662\\
-3.662	-7.324\\
-7.324	-8.545\\
-8.545	-6.104\\
-6.104	-12.207\\
-12.207	-14.648\\
-14.648	-14.648\\
-14.648	-12.207\\
-12.207	-15.869\\
-15.869	-21.973\\
-21.973	-17.09\\
-17.09	-6.104\\
-6.104	-10.986\\
-10.986	-6.104\\
-6.104	-7.324\\
-7.324	-13.428\\
-13.428	-15.869\\
-15.869	-15.869\\
-15.869	-12.207\\
-12.207	-28.076\\
-28.076	-23.193\\
-23.193	-17.09\\
-17.09	-13.428\\
-13.428	-14.648\\
-14.648	-20.752\\
-20.752	-23.193\\
-23.193	-18.311\\
-18.311	-6.104\\
-6.104	-12.207\\
-12.207	-9.766\\
-9.766	-6.104\\
-6.104	-3.662\\
-3.662	-7.324\\
-7.324	-19.531\\
-19.531	-19.531\\
-19.531	-10.986\\
-10.986	-8.545\\
-8.545	-8.545\\
-8.545	-9.766\\
-9.766	-6.104\\
-6.104	-3.662\\
-3.662	-9.766\\
-9.766	-14.648\\
-14.648	-18.311\\
-18.311	-20.752\\
-20.752	-19.531\\
-19.531	-20.752\\
-20.752	-20.752\\
-20.752	-19.531\\
-19.531	-18.311\\
-18.311	-21.973\\
-21.973	-30.518\\
-30.518	-26.855\\
-26.855	-17.09\\
-17.09	-8.545\\
-8.545	-12.207\\
-12.207	-26.855\\
-26.855	-21.973\\
-21.973	-13.428\\
-13.428	-15.869\\
-15.869	-20.752\\
-20.752	-29.297\\
-29.297	-29.297\\
-29.297	-31.738\\
-31.738	-29.297\\
-29.297	-26.855\\
-26.855	-24.414\\
-24.414	-17.09\\
-17.09	-13.428\\
-13.428	-15.869\\
-15.869	-19.531\\
-19.531	-17.09\\
-17.09	-19.531\\
-19.531	-13.428\\
-13.428	-18.311\\
-18.311	-17.09\\
-17.09	-7.324\\
-7.324	-4.883\\
-4.883	-4.883\\
-4.883	-7.324\\
-7.324	-8.545\\
-8.545	-3.662\\
-3.662	-7.324\\
-7.324	-17.09\\
-17.09	-3.662\\
-3.662	-10.986\\
-10.986	-14.648\\
-14.648	-20.752\\
-20.752	-18.311\\
-18.311	-7.324\\
-7.324	-7.324\\
-7.324	-7.324\\
-7.324	-24.414\\
-24.414	-26.855\\
-26.855	-34.18\\
-34.18	-42.725\\
-42.725	-40.283\\
-40.283	-29.297\\
-29.297	-29.297\\
-29.297	-36.621\\
-36.621	-32.959\\
-32.959	-29.297\\
-29.297	-30.518\\
-30.518	-35.4\\
-35.4	-35.4\\
-35.4	-46.387\\
-46.387	-36.621\\
-36.621	-18.311\\
-18.311	-10.986\\
-10.986	-9.766\\
-9.766	-10.986\\
-10.986	-12.207\\
-12.207	-10.986\\
-10.986	-17.09\\
-17.09	-18.311\\
-18.311	-14.648\\
-14.648	-19.531\\
-19.531	-12.207\\
-12.207	-17.09\\
-17.09	-18.311\\
-18.311	-19.531\\
-19.531	-10.986\\
-10.986	-6.104\\
-6.104	-10.986\\
-10.986	-13.428\\
-13.428	-17.09\\
-17.09	-30.518\\
-30.518	-30.518\\
-30.518	-32.959\\
-32.959	-35.4\\
-35.4	-45.166\\
-45.166	-40.283\\
-40.283	-26.855\\
-26.855	-17.09\\
-17.09	-13.428\\
-13.428	-13.428\\
-13.428	-14.648\\
-14.648	-15.869\\
-15.869	-8.545\\
-8.545	-12.207\\
-12.207	-12.207\\
-12.207	-14.648\\
-14.648	-18.311\\
-18.311	-18.311\\
-18.311	-15.869\\
-15.869	-7.324\\
-7.324	-4.883\\
-4.883	-7.324\\
-7.324	-4.883\\
-4.883	-6.104\\
-6.104	-10.986\\
-10.986	-14.648\\
-14.648	-15.869\\
-15.869	-19.531\\
-19.531	-17.09\\
-17.09	-19.531\\
-19.531	-34.18\\
-34.18	-26.855\\
-26.855	-21.973\\
-21.973	-28.076\\
-28.076	-31.738\\
-31.738	-20.752\\
-20.752	-18.311\\
-18.311	-13.428\\
-13.428	-15.869\\
-15.869	-28.076\\
-28.076	-41.504\\
-41.504	-46.387\\
-46.387	-36.621\\
-36.621	-31.738\\
-31.738	-26.855\\
-26.855	-28.076\\
-28.076	-34.18\\
-34.18	-23.193\\
-23.193	-18.311\\
-18.311	-18.311\\
-18.311	-24.414\\
-24.414	-25.635\\
-25.635	-14.648\\
-14.648	-14.648\\
-14.648	-13.428\\
-13.428	-17.09\\
-17.09	-26.855\\
-26.855	-25.635\\
-25.635	-20.752\\
-20.752	-18.311\\
-18.311	-17.09\\
-17.09	-14.648\\
-14.648	-8.545\\
-8.545	-8.545\\
-8.545	-7.324\\
-7.324	-7.324\\
-7.324	-6.104\\
-6.104	-10.986\\
-10.986	-17.09\\
-17.09	-18.311\\
-18.311	-13.428\\
-13.428	-9.766\\
-9.766	-8.545\\
-8.545	-10.986\\
-10.986	-14.648\\
-14.648	-17.09\\
-17.09	-14.648\\
-14.648	-9.766\\
-9.766	-20.752\\
-20.752	-21.973\\
-21.973	-15.869\\
-15.869	-4.883\\
-4.883	-10.986\\
-10.986	-13.428\\
-13.428	-14.648\\
-14.648	-9.766\\
-9.766	-7.324\\
-7.324	-13.428\\
-13.428	-21.973\\
-21.973	-13.428\\
-13.428	-3.662\\
-3.662	-9.766\\
-9.766	-14.648\\
-14.648	-20.752\\
-20.752	-14.648\\
-14.648	-12.207\\
-12.207	-20.752\\
-20.752	-26.855\\
-26.855	-23.193\\
-23.193	-29.297\\
-29.297	-35.4\\
-35.4	-24.414\\
-24.414	-19.531\\
-19.531	-26.855\\
-26.855	-19.531\\
-19.531	-23.193\\
-23.193	-29.297\\
-29.297	-21.973\\
-21.973	-10.986\\
-10.986	-20.752\\
-20.752	-34.18\\
-34.18	-37.842\\
-37.842	-29.297\\
-29.297	-24.414\\
-24.414	-31.738\\
-31.738	-26.855\\
-26.855	-17.09\\
-17.09	-17.09\\
-17.09	-19.531\\
-19.531	-21.973\\
-21.973	-19.531\\
-19.531	-21.973\\
-21.973	-14.648\\
-14.648	-19.531\\
-19.531	-24.414\\
-24.414	-18.311\\
-18.311	-14.648\\
-14.648	-12.207\\
-12.207	-8.545\\
-8.545	-9.766\\
-9.766	-17.09\\
-17.09	-15.869\\
-15.869	-19.531\\
-19.531	-25.635\\
-25.635	-26.855\\
-26.855	-20.752\\
-20.752	-18.311\\
-18.311	-10.986\\
-10.986	-14.648\\
-14.648	-24.414\\
-24.414	-18.311\\
-18.311	-8.545\\
-8.545	-15.869\\
-15.869	-25.635\\
-25.635	-23.193\\
-23.193	-30.518\\
-30.518	-39.063\\
-39.063	-30.518\\
-30.518	-26.855\\
-26.855	-29.297\\
-29.297	-20.752\\
-20.752	-14.648\\
-14.648	-17.09\\
-17.09	-24.414\\
-24.414	-24.414\\
-24.414	-25.635\\
-25.635	-13.428\\
-13.428	-14.648\\
-14.648	-12.207\\
-12.207	-17.09\\
-17.09	-14.648\\
-14.648	-12.207\\
-12.207	-20.752\\
-20.752	-24.414\\
-24.414	-15.869\\
-15.869	-12.207\\
-12.207	-20.752\\
-20.752	-13.428\\
-13.428	-9.766\\
-9.766	-10.986\\
-10.986	-12.207\\
-12.207	-10.986\\
-10.986	-6.104\\
-6.104	-6.104\\
-6.104	-10.986\\
-10.986	-12.207\\
-12.207	-20.752\\
-20.752	-19.531\\
-19.531	-21.973\\
-21.973	-28.076\\
-28.076	-14.648\\
-14.648	-14.648\\
-14.648	-25.635\\
-25.635	-36.621\\
-36.621	-26.855\\
-26.855	-25.635\\
-25.635	-39.063\\
-39.063	-31.738\\
-31.738	-25.635\\
-25.635	-20.752\\
-20.752	-20.752\\
-20.752	-25.635\\
-25.635	-18.311\\
-18.311	-17.09\\
-17.09	-28.076\\
-28.076	-31.738\\
-31.738	-31.738\\
-31.738	-18.311\\
-18.311	-8.545\\
-8.545	-15.869\\
-15.869	-15.869\\
-15.869	-8.545\\
-8.545	-7.324\\
-7.324	-10.986\\
-10.986	-13.428\\
-13.428	-23.193\\
-23.193	-17.09\\
-17.09	-10.986\\
-10.986	-9.766\\
-9.766	-7.324\\
-7.324	-8.545\\
-8.545	-7.324\\
-7.324	-7.324\\
-7.324	-14.648\\
-14.648	-12.207\\
-12.207	-7.324\\
-7.324	-8.545\\
-8.545	-9.766\\
-9.766	-10.986\\
-10.986	-8.545\\
-8.545	-9.766\\
-9.766	-8.545\\
-8.545	-4.883\\
-4.883	-13.428\\
-13.428	-17.09\\
-17.09	-23.193\\
-23.193	-30.518\\
-30.518	-26.855\\
-26.855	-13.428\\
-13.428	-10.986\\
-10.986	-12.207\\
-12.207	-7.324\\
-7.324	-9.766\\
-9.766	-10.986\\
-10.986	-14.648\\
-14.648	-13.428\\
-13.428	-13.428\\
-13.428	-15.869\\
-15.869	-18.311\\
-18.311	-17.09\\
-17.09	-21.973\\
-21.973	-21.973\\
-21.973	-30.518\\
-30.518	-19.531\\
-19.531	-9.766\\
-9.766	-8.545\\
-8.545	-15.869\\
-15.869	-10.986\\
-10.986	-9.766\\
-9.766	-4.883\\
-4.883	-10.986\\
-10.986	-7.324\\
-7.324	-3.662\\
-3.662	-6.104\\
-6.104	-10.986\\
-10.986	-13.428\\
-13.428	-15.869\\
-15.869	-24.414\\
-24.414	-17.09\\
-17.09	-7.324\\
-7.324	-13.428\\
-13.428	-18.311\\
-18.311	-10.986\\
-10.986	-13.428\\
-13.428	-20.752\\
-20.752	-17.09\\
-17.09	-25.635\\
-25.635	-30.518\\
-30.518	-20.752\\
-20.752	-17.09\\
-17.09	-15.869\\
};
\addlegendentry{data1}

\addplot [color=mycolor2, line width=2.0pt]
  table[row sep=crcr]{%
-18.311	-17.6579837260104\\
-19.531	-18.8344754602539\\
-24.414	-23.5433354096891\\
-19.531	-18.8344754602539\\
-20.752	-20.0119315319844\\
-25.635	-24.7207914814197\\
-24.414	-23.5433354096891\\
-28.076	-27.0747392873938\\
-23.193	-22.3658793379585\\
-13.428	-12.9491237765751\\
-17.09	-16.4805276542798\\
-18.311	-17.6579837260104\\
-15.869	-15.3030715825492\\
-8.545	-8.2402638271399\\
-6.104	-5.88631602116582\\
-7.324	-7.06280775540931\\
-9.766	-9.41771989887048\\
-21.973	-21.189387603715\\
-23.193	-22.3658793379585\\
-19.531	-18.8344754602539\\
-10.986	-10.594211633114\\
-17.09	-16.4805276542798\\
-19.531	-18.8344754602539\\
-13.428	-12.9491237765751\\
-18.311	-17.6579837260104\\
-19.531	-18.8344754602539\\
-14.648	-14.1256155108186\\
-24.414	-23.5433354096891\\
-21.973	-21.189387603715\\
-15.869	-15.3030715825492\\
-23.193	-22.3658793379585\\
-25.635	-24.7207914814197\\
-19.531	-18.8344754602539\\
-17.09	-16.4805276542798\\
-14.648	-14.1256155108186\\
-17.09	-16.4805276542798\\
-20.752	-20.0119315319844\\
-18.311	-17.6579837260104\\
-19.531	-18.8344754602539\\
-20.752	-20.0119315319844\\
-28.076	-27.0747392873938\\
-25.635	-24.7207914814197\\
-17.09	-16.4805276542798\\
-15.869	-15.3030715825492\\
-12.207	-11.7716677048446\\
-10.986	-10.594211633114\\
-15.869	-15.3030715825492\\
-12.207	-11.7716677048446\\
-15.869	-15.3030715825492\\
-18.311	-17.6579837260104\\
-15.869	-15.3030715825492\\
-25.635	-24.7207914814197\\
-21.973	-21.189387603715\\
-12.207	-11.7716677048446\\
-9.766	-9.41771989887048\\
-13.428	-12.9491237765751\\
-15.869	-15.3030715825492\\
-10.986	-10.594211633114\\
-12.207	-11.7716677048446\\
-9.766	-9.41771989887048\\
-8.545	-8.2402638271399\\
-12.207	-11.7716677048446\\
-14.648	-14.1256155108186\\
-17.09	-16.4805276542798\\
-18.311	-17.6579837260104\\
-23.193	-22.3658793379585\\
-21.973	-21.189387603715\\
-17.09	-16.4805276542798\\
-15.869	-15.3030715825492\\
-17.09	-16.4805276542798\\
-24.414	-23.5433354096891\\
-35.4	-34.1375470428031\\
-36.621	-35.3150031145337\\
-34.18	-32.9610553085596\\
-24.414	-23.5433354096891\\
-29.297	-28.2521953591243\\
-36.621	-35.3150031145337\\
-40.283	-38.8464069922383\\
-31.738	-30.6061431650984\\
-37.842	-36.4924591862642\\
-48.828	-47.0866708193782\\
-39.063	-37.6699152579948\\
-30.518	-29.4296514308549\\
-25.635	-24.7207914814197\\
-19.531	-18.8344754602539\\
-14.648	-14.1256155108186\\
-19.531	-18.8344754602539\\
-15.869	-15.3030715825492\\
-9.766	-9.41771989887048\\
-12.207	-11.7716677048446\\
-17.09	-16.4805276542798\\
-15.869	-15.3030715825492\\
-19.531	-18.8344754602539\\
-28.076	-27.0747392873938\\
-23.193	-22.3658793379585\\
-25.635	-24.7207914814197\\
-23.193	-22.3658793379585\\
-18.311	-17.6579837260104\\
-20.752	-20.0119315319844\\
-17.09	-16.4805276542798\\
-14.648	-14.1256155108186\\
-18.311	-17.6579837260104\\
-17.09	-16.4805276542798\\
-15.869	-15.3030715825492\\
-13.428	-12.9491237765751\\
-10.986	-10.594211633114\\
-12.207	-11.7716677048446\\
-13.428	-12.9491237765751\\
-10.986	-10.594211633114\\
-9.766	-9.41771989887048\\
-14.648	-14.1256155108186\\
-19.531	-18.8344754602539\\
-23.193	-22.3658793379585\\
-17.09	-16.4805276542798\\
-18.311	-17.6579837260104\\
-30.518	-29.4296514308549\\
-23.193	-22.3658793379585\\
-24.414	-23.5433354096891\\
-26.855	-25.8972832156632\\
-17.09	-16.4805276542798\\
-10.986	-10.594211633114\\
-17.09	-16.4805276542798\\
-18.311	-17.6579837260104\\
-26.855	-25.8972832156632\\
-34.18	-32.9610553085596\\
-31.738	-30.6061431650984\\
-25.635	-24.7207914814197\\
-24.414	-23.5433354096891\\
-23.193	-22.3658793379585\\
-21.973	-21.189387603715\\
-18.311	-17.6579837260104\\
-14.648	-14.1256155108186\\
-13.428	-12.9491237765751\\
-12.207	-11.7716677048446\\
-13.428	-12.9491237765751\\
-19.531	-18.8344754602539\\
-25.635	-24.7207914814197\\
-21.973	-21.189387603715\\
-14.648	-14.1256155108186\\
-13.428	-12.9491237765751\\
-14.648	-14.1256155108186\\
-12.207	-11.7716677048446\\
-8.545	-8.2402638271399\\
-13.428	-12.9491237765751\\
-12.207	-11.7716677048446\\
-8.545	-8.2402638271399\\
-7.324	-7.06280775540931\\
-4.883	-4.70885994943524\\
-7.324	-7.06280775540931\\
-17.09	-16.4805276542798\\
-24.414	-23.5433354096891\\
-19.531	-18.8344754602539\\
-12.207	-11.7716677048446\\
-8.545	-8.2402638271399\\
-10.986	-10.594211633114\\
-14.648	-14.1256155108186\\
-24.414	-23.5433354096891\\
-34.18	-32.9610553085596\\
-40.283	-38.8464069922383\\
-32.959	-31.783599236829\\
-31.738	-30.6061431650984\\
-23.193	-22.3658793379585\\
-25.635	-24.7207914814197\\
-26.855	-25.8972832156632\\
-28.076	-27.0747392873938\\
-21.973	-21.189387603715\\
-20.752	-20.0119315319844\\
-19.531	-18.8344754602539\\
-21.973	-21.189387603715\\
-19.531	-18.8344754602539\\
-23.193	-22.3658793379585\\
-29.297	-28.2521953591243\\
-24.414	-23.5433354096891\\
-14.648	-14.1256155108186\\
-15.869	-15.3030715825492\\
-25.635	-24.7207914814197\\
-32.959	-31.783599236829\\
-35.4	-34.1375470428031\\
-39.063	-37.6699152579948\\
-37.842	-36.4924591862642\\
-30.518	-29.4296514308549\\
-26.855	-25.8972832156632\\
-30.518	-29.4296514308549\\
-35.4	-34.1375470428031\\
-25.635	-24.7207914814197\\
-14.648	-14.1256155108186\\
-19.531	-18.8344754602539\\
-20.752	-20.0119315319844\\
-12.207	-11.7716677048446\\
-13.428	-12.9491237765751\\
-18.311	-17.6579837260104\\
-14.648	-14.1256155108186\\
-12.207	-11.7716677048446\\
-18.311	-17.6579837260104\\
-10.986	-10.594211633114\\
-14.648	-14.1256155108186\\
-24.414	-23.5433354096891\\
-21.973	-21.189387603715\\
-14.648	-14.1256155108186\\
-20.752	-20.0119315319844\\
-34.18	-32.9610553085596\\
-28.076	-27.0747392873938\\
-15.869	-15.3030715825492\\
-14.648	-14.1256155108186\\
-12.207	-11.7716677048446\\
-10.986	-10.594211633114\\
-13.428	-12.9491237765751\\
-17.09	-16.4805276542798\\
-19.531	-18.8344754602539\\
-14.648	-14.1256155108186\\
-19.531	-18.8344754602539\\
-26.855	-25.8972832156632\\
-23.193	-22.3658793379585\\
-17.09	-16.4805276542798\\
-21.973	-21.189387603715\\
-25.635	-24.7207914814197\\
-21.973	-21.189387603715\\
-8.545	-8.2402638271399\\
-12.207	-11.7716677048446\\
-13.428	-12.9491237765751\\
-15.869	-15.3030715825492\\
-10.986	-10.594211633114\\
-17.09	-16.4805276542798\\
-15.869	-15.3030715825492\\
-10.986	-10.594211633114\\
-14.648	-14.1256155108186\\
-12.207	-11.7716677048446\\
-14.648	-14.1256155108186\\
-10.986	-10.594211633114\\
-9.766	-9.41771989887048\\
-12.207	-11.7716677048446\\
-8.545	-8.2402638271399\\
-17.09	-16.4805276542798\\
-18.311	-17.6579837260104\\
-20.752	-20.0119315319844\\
-29.297	-28.2521953591243\\
-20.752	-20.0119315319844\\
-10.986	-10.594211633114\\
-8.545	-8.2402638271399\\
-13.428	-12.9491237765751\\
-12.207	-11.7716677048446\\
-9.766	-9.41771989887048\\
-15.869	-15.3030715825492\\
-19.531	-18.8344754602539\\
-20.752	-20.0119315319844\\
-23.193	-22.3658793379585\\
-17.09	-16.4805276542798\\
-15.869	-15.3030715825492\\
-10.986	-10.594211633114\\
-7.324	-7.06280775540931\\
-8.545	-8.2402638271399\\
-10.986	-10.594211633114\\
-17.09	-16.4805276542798\\
-18.311	-17.6579837260104\\
-17.09	-16.4805276542798\\
-24.414	-23.5433354096891\\
-25.635	-24.7207914814197\\
-15.869	-15.3030715825492\\
-19.531	-18.8344754602539\\
-23.193	-22.3658793379585\\
-26.855	-25.8972832156632\\
-25.635	-24.7207914814197\\
-17.09	-16.4805276542798\\
-10.986	-10.594211633114\\
-7.324	-7.06280775540931\\
-6.104	-5.88631602116582\\
-8.545	-8.2402638271399\\
-9.766	-9.41771989887048\\
-8.545	-8.2402638271399\\
-10.986	-10.594211633114\\
-17.09	-16.4805276542798\\
-15.869	-15.3030715825492\\
-13.428	-12.9491237765751\\
-15.869	-15.3030715825492\\
-12.207	-11.7716677048446\\
-6.104	-5.88631602116582\\
-9.766	-9.41771989887048\\
-20.752	-20.0119315319844\\
-17.09	-16.4805276542798\\
-18.311	-17.6579837260104\\
-15.869	-15.3030715825492\\
-10.986	-10.594211633114\\
-17.09	-16.4805276542798\\
-23.193	-22.3658793379585\\
-17.09	-16.4805276542798\\
-26.855	-25.8972832156632\\
-25.635	-24.7207914814197\\
-18.311	-17.6579837260104\\
-23.193	-22.3658793379585\\
-25.635	-24.7207914814197\\
-19.531	-18.8344754602539\\
-15.869	-15.3030715825492\\
-24.414	-23.5433354096891\\
-35.4	-34.1375470428031\\
-29.297	-28.2521953591243\\
-17.09	-16.4805276542798\\
-18.311	-17.6579837260104\\
-20.752	-20.0119315319844\\
-23.193	-22.3658793379585\\
-19.531	-18.8344754602539\\
-17.09	-16.4805276542798\\
-24.414	-23.5433354096891\\
-25.635	-24.7207914814197\\
-18.311	-17.6579837260104\\
-15.869	-15.3030715825492\\
-18.311	-17.6579837260104\\
-28.076	-27.0747392873938\\
-26.855	-25.8972832156632\\
-24.414	-23.5433354096891\\
-19.531	-18.8344754602539\\
-17.09	-16.4805276542798\\
-18.311	-17.6579837260104\\
-13.428	-12.9491237765751\\
-6.104	-5.88631602116582\\
-9.766	-9.41771989887048\\
-19.531	-18.8344754602539\\
-15.869	-15.3030715825492\\
-14.648	-14.1256155108186\\
-13.428	-12.9491237765751\\
-17.09	-16.4805276542798\\
-13.428	-12.9491237765751\\
-9.766	-9.41771989887048\\
-12.207	-11.7716677048446\\
-9.766	-9.41771989887048\\
-8.545	-8.2402638271399\\
-17.09	-16.4805276542798\\
-20.752	-20.0119315319844\\
-17.09	-16.4805276542798\\
-10.986	-10.594211633114\\
-13.428	-12.9491237765751\\
-9.766	-9.41771989887048\\
-10.986	-10.594211633114\\
-28.076	-27.0747392873938\\
-20.752	-20.0119315319844\\
-25.635	-24.7207914814197\\
-35.4	-34.1375470428031\\
-30.518	-29.4296514308549\\
-24.414	-23.5433354096891\\
-18.311	-17.6579837260104\\
-19.531	-18.8344754602539\\
-21.973	-21.189387603715\\
-25.635	-24.7207914814197\\
-31.738	-30.6061431650984\\
-34.18	-32.9610553085596\\
-41.504	-40.0238630639689\\
-32.959	-31.783599236829\\
-28.076	-27.0747392873938\\
-35.4	-34.1375470428031\\
-26.855	-25.8972832156632\\
-29.297	-28.2521953591243\\
-28.076	-27.0747392873938\\
-17.09	-16.4805276542798\\
-10.986	-10.594211633114\\
-12.207	-11.7716677048446\\
-17.09	-16.4805276542798\\
-14.648	-14.1256155108186\\
-9.766	-9.41771989887048\\
-13.428	-12.9491237765751\\
-7.324	-7.06280775540931\\
-8.545	-8.2402638271399\\
-12.207	-11.7716677048446\\
-7.324	-7.06280775540931\\
-14.648	-14.1256155108186\\
-20.752	-20.0119315319844\\
-13.428	-12.9491237765751\\
-21.973	-21.189387603715\\
-30.518	-29.4296514308549\\
-29.297	-28.2521953591243\\
-35.4	-34.1375470428031\\
-29.297	-28.2521953591243\\
-28.076	-27.0747392873938\\
-23.193	-22.3658793379585\\
-12.207	-11.7716677048446\\
-14.648	-14.1256155108186\\
-15.869	-15.3030715825492\\
-10.986	-10.594211633114\\
-12.207	-11.7716677048446\\
-13.428	-12.9491237765751\\
-3.662	-3.53140387770466\\
-8.545	-8.2402638271399\\
-17.09	-16.4805276542798\\
-19.531	-18.8344754602539\\
-20.752	-20.0119315319844\\
-23.193	-22.3658793379585\\
-20.752	-20.0119315319844\\
-23.193	-22.3658793379585\\
-18.311	-17.6579837260104\\
-15.869	-15.3030715825492\\
-12.207	-11.7716677048446\\
-18.311	-17.6579837260104\\
-17.09	-16.4805276542798\\
-12.207	-11.7716677048446\\
-17.09	-16.4805276542798\\
-23.193	-22.3658793379585\\
-17.09	-16.4805276542798\\
-13.428	-12.9491237765751\\
-12.207	-11.7716677048446\\
-14.648	-14.1256155108186\\
-7.324	-7.06280775540931\\
-8.545	-8.2402638271399\\
-12.207	-11.7716677048446\\
-17.09	-16.4805276542798\\
-14.648	-14.1256155108186\\
-15.869	-15.3030715825492\\
-18.311	-17.6579837260104\\
-12.207	-11.7716677048446\\
-9.766	-9.41771989887048\\
-18.311	-17.6579837260104\\
-26.855	-25.8972832156632\\
-21.973	-21.189387603715\\
-19.531	-18.8344754602539\\
-17.09	-16.4805276542798\\
-14.648	-14.1256155108186\\
-13.428	-12.9491237765751\\
-17.09	-16.4805276542798\\
-10.986	-10.594211633114\\
-14.648	-14.1256155108186\\
-18.311	-17.6579837260104\\
-19.531	-18.8344754602539\\
-21.973	-21.189387603715\\
-29.297	-28.2521953591243\\
-35.4	-34.1375470428031\\
-36.621	-35.3150031145337\\
-26.855	-25.8972832156632\\
-14.648	-14.1256155108186\\
-10.986	-10.594211633114\\
-7.324	-7.06280775540931\\
-10.986	-10.594211633114\\
-14.648	-14.1256155108186\\
-13.428	-12.9491237765751\\
-12.207	-11.7716677048446\\
-15.869	-15.3030715825492\\
-20.752	-20.0119315319844\\
-24.414	-23.5433354096891\\
-31.738	-30.6061431650984\\
-29.297	-28.2521953591243\\
-26.855	-25.8972832156632\\
-18.311	-17.6579837260104\\
-17.09	-16.4805276542798\\
-20.752	-20.0119315319844\\
-17.09	-16.4805276542798\\
-14.648	-14.1256155108186\\
-18.311	-17.6579837260104\\
-12.207	-11.7716677048446\\
-13.428	-12.9491237765751\\
-20.752	-20.0119315319844\\
-17.09	-16.4805276542798\\
-18.311	-17.6579837260104\\
-32.959	-31.783599236829\\
-25.635	-24.7207914814197\\
-20.752	-20.0119315319844\\
-28.076	-27.0747392873938\\
-19.531	-18.8344754602539\\
-13.428	-12.9491237765751\\
-19.531	-18.8344754602539\\
-30.518	-29.4296514308549\\
-34.18	-32.9610553085596\\
-31.738	-30.6061431650984\\
-25.635	-24.7207914814197\\
-17.09	-16.4805276542798\\
-14.648	-14.1256155108186\\
-13.428	-12.9491237765751\\
-10.986	-10.594211633114\\
-8.545	-8.2402638271399\\
-12.207	-11.7716677048446\\
-14.648	-14.1256155108186\\
-10.986	-10.594211633114\\
-13.428	-12.9491237765751\\
-19.531	-18.8344754602539\\
-20.752	-20.0119315319844\\
-21.973	-21.189387603715\\
-29.297	-28.2521953591243\\
-34.18	-32.9610553085596\\
-25.635	-24.7207914814197\\
-23.193	-22.3658793379585\\
-24.414	-23.5433354096891\\
-20.752	-20.0119315319844\\
-14.648	-14.1256155108186\\
-18.311	-17.6579837260104\\
-29.297	-28.2521953591243\\
-31.738	-30.6061431650984\\
-19.531	-18.8344754602539\\
-12.207	-11.7716677048446\\
-8.545	-8.2402638271399\\
-7.324	-7.06280775540931\\
-9.766	-9.41771989887048\\
-8.545	-8.2402638271399\\
-12.207	-11.7716677048446\\
-14.648	-14.1256155108186\\
-13.428	-12.9491237765751\\
-9.766	-9.41771989887048\\
-10.986	-10.594211633114\\
-9.766	-9.41771989887048\\
-10.986	-10.594211633114\\
-14.648	-14.1256155108186\\
-21.973	-21.189387603715\\
-18.311	-17.6579837260104\\
-20.752	-20.0119315319844\\
-21.973	-21.189387603715\\
-30.518	-29.4296514308549\\
-31.738	-30.6061431650984\\
-34.18	-32.9610553085596\\
-24.414	-23.5433354096891\\
-17.09	-16.4805276542798\\
-15.869	-15.3030715825492\\
-26.855	-25.8972832156632\\
-31.738	-30.6061431650984\\
-23.193	-22.3658793379585\\
-18.311	-17.6579837260104\\
-9.766	-9.41771989887048\\
-4.883	-4.70885994943524\\
-6.104	-5.88631602116582\\
-7.324	-7.06280775540931\\
-14.648	-14.1256155108186\\
-10.986	-10.594211633114\\
-12.207	-11.7716677048446\\
-8.545	-8.2402638271399\\
-4.883	-4.70885994943524\\
-8.545	-8.2402638271399\\
-9.766	-9.41771989887048\\
-7.324	-7.06280775540931\\
-10.986	-10.594211633114\\
-24.414	-23.5433354096891\\
-26.855	-25.8972832156632\\
-25.635	-24.7207914814197\\
-21.973	-21.189387603715\\
-30.518	-29.4296514308549\\
-40.283	-38.8464069922383\\
-34.18	-32.9610553085596\\
-31.738	-30.6061431650984\\
-26.855	-25.8972832156632\\
-28.076	-27.0747392873938\\
-29.297	-28.2521953591243\\
-20.752	-20.0119315319844\\
-17.09	-16.4805276542798\\
-14.648	-14.1256155108186\\
-12.207	-11.7716677048446\\
-20.752	-20.0119315319844\\
-18.311	-17.6579837260104\\
-15.869	-15.3030715825492\\
-18.311	-17.6579837260104\\
-14.648	-14.1256155108186\\
-17.09	-16.4805276542798\\
-20.752	-20.0119315319844\\
-19.531	-18.8344754602539\\
-13.428	-12.9491237765751\\
-17.09	-16.4805276542798\\
-9.766	-9.41771989887048\\
-4.883	-4.70885994943524\\
-10.986	-10.594211633114\\
-13.428	-12.9491237765751\\
-8.545	-8.2402638271399\\
-2.441	-2.35394780597408\\
-8.545	-8.2402638271399\\
-12.207	-11.7716677048446\\
-15.869	-15.3030715825492\\
-13.428	-12.9491237765751\\
-18.311	-17.6579837260104\\
-17.09	-16.4805276542798\\
-10.986	-10.594211633114\\
-17.09	-16.4805276542798\\
-14.648	-14.1256155108186\\
-9.766	-9.41771989887048\\
-7.324	-7.06280775540931\\
-6.104	-5.88631602116582\\
-7.324	-7.06280775540931\\
-8.545	-8.2402638271399\\
-6.104	-5.88631602116582\\
-13.428	-12.9491237765751\\
-17.09	-16.4805276542798\\
-10.986	-10.594211633114\\
-14.648	-14.1256155108186\\
-10.986	-10.594211633114\\
-14.648	-14.1256155108186\\
-10.986	-10.594211633114\\
-9.766	-9.41771989887048\\
-7.324	-7.06280775540931\\
-8.545	-8.2402638271399\\
-10.986	-10.594211633114\\
-14.648	-14.1256155108186\\
-12.207	-11.7716677048446\\
-8.545	-8.2402638271399\\
-10.986	-10.594211633114\\
-21.973	-21.189387603715\\
-24.414	-23.5433354096891\\
-18.311	-17.6579837260104\\
-17.09	-16.4805276542798\\
-13.428	-12.9491237765751\\
-19.531	-18.8344754602539\\
-17.09	-16.4805276542798\\
-9.766	-9.41771989887048\\
-14.648	-14.1256155108186\\
-13.428	-12.9491237765751\\
-15.869	-15.3030715825492\\
-12.207	-11.7716677048446\\
-14.648	-14.1256155108186\\
-12.207	-11.7716677048446\\
-15.869	-15.3030715825492\\
-12.207	-11.7716677048446\\
-13.428	-12.9491237765751\\
-14.648	-14.1256155108186\\
-19.531	-18.8344754602539\\
-18.311	-17.6579837260104\\
-21.973	-21.189387603715\\
-30.518	-29.4296514308549\\
-24.414	-23.5433354096891\\
-14.648	-14.1256155108186\\
-15.869	-15.3030715825492\\
-23.193	-22.3658793379585\\
-18.311	-17.6579837260104\\
-23.193	-22.3658793379585\\
-17.09	-16.4805276542798\\
-8.545	-8.2402638271399\\
-9.766	-9.41771989887048\\
-19.531	-18.8344754602539\\
-14.648	-14.1256155108186\\
-13.428	-12.9491237765751\\
-20.752	-20.0119315319844\\
-21.973	-21.189387603715\\
-18.311	-17.6579837260104\\
-14.648	-14.1256155108186\\
-18.311	-17.6579837260104\\
-20.752	-20.0119315319844\\
-18.311	-17.6579837260104\\
-15.869	-15.3030715825492\\
-12.207	-11.7716677048446\\
-14.648	-14.1256155108186\\
-12.207	-11.7716677048446\\
-13.428	-12.9491237765751\\
-19.531	-18.8344754602539\\
-20.752	-20.0119315319844\\
-19.531	-18.8344754602539\\
-14.648	-14.1256155108186\\
-10.986	-10.594211633114\\
-13.428	-12.9491237765751\\
-14.648	-14.1256155108186\\
-18.311	-17.6579837260104\\
-20.752	-20.0119315319844\\
-24.414	-23.5433354096891\\
-18.311	-17.6579837260104\\
-12.207	-11.7716677048446\\
-21.973	-21.189387603715\\
-30.518	-29.4296514308549\\
-20.752	-20.0119315319844\\
-25.635	-24.7207914814197\\
-18.311	-17.6579837260104\\
-17.09	-16.4805276542798\\
-18.311	-17.6579837260104\\
-15.869	-15.3030715825492\\
-18.311	-17.6579837260104\\
-21.973	-21.189387603715\\
-17.09	-16.4805276542798\\
-20.752	-20.0119315319844\\
-18.311	-17.6579837260104\\
-14.648	-14.1256155108186\\
-21.973	-21.189387603715\\
-14.648	-14.1256155108186\\
-17.09	-16.4805276542798\\
-10.986	-10.594211633114\\
-6.104	-5.88631602116582\\
-4.883	-4.70885994943524\\
-10.986	-10.594211633114\\
-20.752	-20.0119315319844\\
-21.973	-21.189387603715\\
-19.531	-18.8344754602539\\
-13.428	-12.9491237765751\\
-17.09	-16.4805276542798\\
-14.648	-14.1256155108186\\
-9.766	-9.41771989887048\\
-15.869	-15.3030715825492\\
-14.648	-14.1256155108186\\
-12.207	-11.7716677048446\\
-13.428	-12.9491237765751\\
-12.207	-11.7716677048446\\
-9.766	-9.41771989887048\\
-7.324	-7.06280775540931\\
-8.545	-8.2402638271399\\
-18.311	-17.6579837260104\\
-13.428	-12.9491237765751\\
-18.311	-17.6579837260104\\
-15.869	-15.3030715825492\\
-21.973	-21.189387603715\\
-19.531	-18.8344754602539\\
-25.635	-24.7207914814197\\
-15.869	-15.3030715825492\\
-23.193	-22.3658793379585\\
-21.973	-21.189387603715\\
-13.428	-12.9491237765751\\
-17.09	-16.4805276542798\\
-14.648	-14.1256155108186\\
-19.531	-18.8344754602539\\
-20.752	-20.0119315319844\\
-24.414	-23.5433354096891\\
-17.09	-16.4805276542798\\
-24.414	-23.5433354096891\\
-25.635	-24.7207914814197\\
-23.193	-22.3658793379585\\
-18.311	-17.6579837260104\\
-14.648	-14.1256155108186\\
-19.531	-18.8344754602539\\
-17.09	-16.4805276542798\\
-14.648	-14.1256155108186\\
-12.207	-11.7716677048446\\
-13.428	-12.9491237765751\\
-12.207	-11.7716677048446\\
-25.635	-24.7207914814197\\
-30.518	-29.4296514308549\\
-25.635	-24.7207914814197\\
-23.193	-22.3658793379585\\
-25.635	-24.7207914814197\\
-14.648	-14.1256155108186\\
-13.428	-12.9491237765751\\
-9.766	-9.41771989887048\\
-12.207	-11.7716677048446\\
-18.311	-17.6579837260104\\
-19.531	-18.8344754602539\\
-23.193	-22.3658793379585\\
-18.311	-17.6579837260104\\
-15.869	-15.3030715825492\\
-19.531	-18.8344754602539\\
-28.076	-27.0747392873938\\
-23.193	-22.3658793379585\\
-37.842	-36.4924591862642\\
-25.635	-24.7207914814197\\
-15.869	-15.3030715825492\\
-13.428	-12.9491237765751\\
-12.207	-11.7716677048446\\
-19.531	-18.8344754602539\\
-23.193	-22.3658793379585\\
-28.076	-27.0747392873938\\
-24.414	-23.5433354096891\\
-18.311	-17.6579837260104\\
-10.986	-10.594211633114\\
-14.648	-14.1256155108186\\
-10.986	-10.594211633114\\
-13.428	-12.9491237765751\\
-8.545	-8.2402638271399\\
-10.986	-10.594211633114\\
-8.545	-8.2402638271399\\
-7.324	-7.06280775540931\\
-9.766	-9.41771989887048\\
-13.428	-12.9491237765751\\
-15.869	-15.3030715825492\\
-17.09	-16.4805276542798\\
-15.869	-15.3030715825492\\
-19.531	-18.8344754602539\\
-12.207	-11.7716677048446\\
-21.973	-21.189387603715\\
-23.193	-22.3658793379585\\
-20.752	-20.0119315319844\\
-18.311	-17.6579837260104\\
-19.531	-18.8344754602539\\
-24.414	-23.5433354096891\\
-19.531	-18.8344754602539\\
-15.869	-15.3030715825492\\
-18.311	-17.6579837260104\\
-17.09	-16.4805276542798\\
-13.428	-12.9491237765751\\
-20.752	-20.0119315319844\\
-25.635	-24.7207914814197\\
-23.193	-22.3658793379585\\
-25.635	-24.7207914814197\\
-32.959	-31.783599236829\\
-24.414	-23.5433354096891\\
-23.193	-22.3658793379585\\
-19.531	-18.8344754602539\\
-23.193	-22.3658793379585\\
-29.297	-28.2521953591243\\
-20.752	-20.0119315319844\\
-19.531	-18.8344754602539\\
-24.414	-23.5433354096891\\
-17.09	-16.4805276542798\\
-10.986	-10.594211633114\\
-15.869	-15.3030715825492\\
-8.545	-8.2402638271399\\
-9.766	-9.41771989887048\\
-10.986	-10.594211633114\\
-14.648	-14.1256155108186\\
-17.09	-16.4805276542798\\
-19.531	-18.8344754602539\\
-20.752	-20.0119315319844\\
-21.973	-21.189387603715\\
-25.635	-24.7207914814197\\
-24.414	-23.5433354096891\\
-17.09	-16.4805276542798\\
-20.752	-20.0119315319844\\
-17.09	-16.4805276542798\\
-19.531	-18.8344754602539\\
-24.414	-23.5433354096891\\
-21.973	-21.189387603715\\
-23.193	-22.3658793379585\\
-20.752	-20.0119315319844\\
-19.531	-18.8344754602539\\
-26.855	-25.8972832156632\\
-21.973	-21.189387603715\\
-23.193	-22.3658793379585\\
-21.973	-21.189387603715\\
-14.648	-14.1256155108186\\
-9.766	-9.41771989887048\\
-8.545	-8.2402638271399\\
-13.428	-12.9491237765751\\
-12.207	-11.7716677048446\\
-17.09	-16.4805276542798\\
-14.648	-14.1256155108186\\
-18.311	-17.6579837260104\\
-25.635	-24.7207914814197\\
-30.518	-29.4296514308549\\
-21.973	-21.189387603715\\
-26.855	-25.8972832156632\\
-23.193	-22.3658793379585\\
-18.311	-17.6579837260104\\
-19.531	-18.8344754602539\\
-18.311	-17.6579837260104\\
-20.752	-20.0119315319844\\
-25.635	-24.7207914814197\\
-32.959	-31.783599236829\\
-29.297	-28.2521953591243\\
-30.518	-29.4296514308549\\
-23.193	-22.3658793379585\\
-25.635	-24.7207914814197\\
-19.531	-18.8344754602539\\
-18.311	-17.6579837260104\\
-24.414	-23.5433354096891\\
-29.297	-28.2521953591243\\
-9.766	-9.41771989887048\\
-19.531	-18.8344754602539\\
-12.207	-11.7716677048446\\
-18.311	-17.6579837260104\\
-10.986	-10.594211633114\\
-12.207	-11.7716677048446\\
-23.193	-22.3658793379585\\
-37.842	-36.4924591862642\\
-35.4	-34.1375470428031\\
-34.18	-32.9610553085596\\
-26.855	-25.8972832156632\\
-31.738	-30.6061431650984\\
-36.621	-35.3150031145337\\
-28.076	-27.0747392873938\\
-30.518	-29.4296514308549\\
-31.738	-30.6061431650984\\
-23.193	-22.3658793379585\\
-18.311	-17.6579837260104\\
-24.414	-23.5433354096891\\
-17.09	-16.4805276542798\\
-13.428	-12.9491237765751\\
-18.311	-17.6579837260104\\
-12.207	-11.7716677048446\\
-17.09	-16.4805276542798\\
-14.648	-14.1256155108186\\
-18.311	-17.6579837260104\\
-20.752	-20.0119315319844\\
-15.869	-15.3030715825492\\
-12.207	-11.7716677048446\\
-15.869	-15.3030715825492\\
-18.311	-17.6579837260104\\
-20.752	-20.0119315319844\\
-18.311	-17.6579837260104\\
-24.414	-23.5433354096891\\
-23.193	-22.3658793379585\\
-15.869	-15.3030715825492\\
-25.635	-24.7207914814197\\
-20.752	-20.0119315319844\\
-17.09	-16.4805276542798\\
-14.648	-14.1256155108186\\
-9.766	-9.41771989887048\\
-7.324	-7.06280775540931\\
-15.869	-15.3030715825492\\
-13.428	-12.9491237765751\\
-10.986	-10.594211633114\\
-7.324	-7.06280775540931\\
-12.207	-11.7716677048446\\
-14.648	-14.1256155108186\\
-15.869	-15.3030715825492\\
-19.531	-18.8344754602539\\
-23.193	-22.3658793379585\\
-21.973	-21.189387603715\\
-24.414	-23.5433354096891\\
-17.09	-16.4805276542798\\
-8.545	-8.2402638271399\\
-14.648	-14.1256155108186\\
-9.766	-9.41771989887048\\
-13.428	-12.9491237765751\\
-17.09	-16.4805276542798\\
-18.311	-17.6579837260104\\
-28.076	-27.0747392873938\\
-18.311	-17.6579837260104\\
-21.973	-21.189387603715\\
-23.193	-22.3658793379585\\
-18.311	-17.6579837260104\\
-14.648	-14.1256155108186\\
-13.428	-12.9491237765751\\
-17.09	-16.4805276542798\\
-18.311	-17.6579837260104\\
-26.855	-25.8972832156632\\
-15.869	-15.3030715825492\\
-17.09	-16.4805276542798\\
-20.752	-20.0119315319844\\
-32.959	-31.783599236829\\
-28.076	-27.0747392873938\\
-19.531	-18.8344754602539\\
-13.428	-12.9491237765751\\
-15.869	-15.3030715825492\\
-9.766	-9.41771989887048\\
-6.104	-5.88631602116582\\
-17.09	-16.4805276542798\\
-19.531	-18.8344754602539\\
-25.635	-24.7207914814197\\
-28.076	-27.0747392873938\\
-35.4	-34.1375470428031\\
-31.738	-30.6061431650984\\
-18.311	-17.6579837260104\\
-13.428	-12.9491237765751\\
-18.311	-17.6579837260104\\
-24.414	-23.5433354096891\\
-29.297	-28.2521953591243\\
-23.193	-22.3658793379585\\
-18.311	-17.6579837260104\\
-20.752	-20.0119315319844\\
-25.635	-24.7207914814197\\
-18.311	-17.6579837260104\\
-13.428	-12.9491237765751\\
-12.207	-11.7716677048446\\
-7.324	-7.06280775540931\\
-8.545	-8.2402638271399\\
-4.883	-4.70885994943524\\
-12.207	-11.7716677048446\\
-15.869	-15.3030715825492\\
-12.207	-11.7716677048446\\
-13.428	-12.9491237765751\\
-7.324	-7.06280775540931\\
-3.662	-3.53140387770466\\
-7.324	-7.06280775540931\\
-10.986	-10.594211633114\\
-12.207	-11.7716677048446\\
-14.648	-14.1256155108186\\
-18.311	-17.6579837260104\\
-20.752	-20.0119315319844\\
-24.414	-23.5433354096891\\
-34.18	-32.9610553085596\\
-32.959	-31.783599236829\\
-36.621	-35.3150031145337\\
-31.738	-30.6061431650984\\
-20.752	-20.0119315319844\\
-18.311	-17.6579837260104\\
-21.973	-21.189387603715\\
-17.09	-16.4805276542798\\
-15.869	-15.3030715825492\\
-12.207	-11.7716677048446\\
-17.09	-16.4805276542798\\
-14.648	-14.1256155108186\\
-8.545	-8.2402638271399\\
-7.324	-7.06280775540931\\
-10.986	-10.594211633114\\
-14.648	-14.1256155108186\\
-13.428	-12.9491237765751\\
-14.648	-14.1256155108186\\
-12.207	-11.7716677048446\\
-15.869	-15.3030715825492\\
-23.193	-22.3658793379585\\
-17.09	-16.4805276542798\\
-24.414	-23.5433354096891\\
-32.959	-31.783599236829\\
-35.4	-34.1375470428031\\
-26.855	-25.8972832156632\\
-19.531	-18.8344754602539\\
-14.648	-14.1256155108186\\
-9.766	-9.41771989887048\\
-15.869	-15.3030715825492\\
-10.986	-10.594211633114\\
-17.09	-16.4805276542798\\
-14.648	-14.1256155108186\\
-12.207	-11.7716677048446\\
-17.09	-16.4805276542798\\
-12.207	-11.7716677048446\\
-13.428	-12.9491237765751\\
-9.766	-9.41771989887048\\
-8.545	-8.2402638271399\\
-7.324	-7.06280775540931\\
-6.104	-5.88631602116582\\
-9.766	-9.41771989887048\\
-10.986	-10.594211633114\\
-8.545	-8.2402638271399\\
-15.869	-15.3030715825492\\
-17.09	-16.4805276542798\\
-14.648	-14.1256155108186\\
-20.752	-20.0119315319844\\
-25.635	-24.7207914814197\\
-18.311	-17.6579837260104\\
-21.973	-21.189387603715\\
-9.766	-9.41771989887048\\
-6.104	-5.88631602116582\\
-14.648	-14.1256155108186\\
-18.311	-17.6579837260104\\
-19.531	-18.8344754602539\\
-26.855	-25.8972832156632\\
-25.635	-24.7207914814197\\
-17.09	-16.4805276542798\\
-13.428	-12.9491237765751\\
-15.869	-15.3030715825492\\
-14.648	-14.1256155108186\\
-10.986	-10.594211633114\\
-7.324	-7.06280775540931\\
-12.207	-11.7716677048446\\
-7.324	-7.06280775540931\\
-6.104	-5.88631602116582\\
-7.324	-7.06280775540931\\
-9.766	-9.41771989887048\\
-13.428	-12.9491237765751\\
-9.766	-9.41771989887048\\
-17.09	-16.4805276542798\\
-23.193	-22.3658793379585\\
-14.648	-14.1256155108186\\
-20.752	-20.0119315319844\\
-23.193	-22.3658793379585\\
-15.869	-15.3030715825492\\
-10.986	-10.594211633114\\
-14.648	-14.1256155108186\\
-9.766	-9.41771989887048\\
-12.207	-11.7716677048446\\
-7.324	-7.06280775540931\\
-6.104	-5.88631602116582\\
-8.545	-8.2402638271399\\
-12.207	-11.7716677048446\\
-10.986	-10.594211633114\\
-12.207	-11.7716677048446\\
-14.648	-14.1256155108186\\
-9.766	-9.41771989887048\\
-7.324	-7.06280775540931\\
-6.104	-5.88631602116582\\
-8.545	-8.2402638271399\\
-9.766	-9.41771989887048\\
-18.311	-17.6579837260104\\
-14.648	-14.1256155108186\\
-19.531	-18.8344754602539\\
-24.414	-23.5433354096891\\
-25.635	-24.7207914814197\\
-26.855	-25.8972832156632\\
-21.973	-21.189387603715\\
-23.193	-22.3658793379585\\
-25.635	-24.7207914814197\\
-30.518	-29.4296514308549\\
-26.855	-25.8972832156632\\
-24.414	-23.5433354096891\\
-20.752	-20.0119315319844\\
-19.531	-18.8344754602539\\
-24.414	-23.5433354096891\\
-15.869	-15.3030715825492\\
-19.531	-18.8344754602539\\
-21.973	-21.189387603715\\
-24.414	-23.5433354096891\\
-17.09	-16.4805276542798\\
-20.752	-20.0119315319844\\
-18.311	-17.6579837260104\\
-17.09	-16.4805276542798\\
-12.207	-11.7716677048446\\
-17.09	-16.4805276542798\\
-24.414	-23.5433354096891\\
-19.531	-18.8344754602539\\
-10.986	-10.594211633114\\
-14.648	-14.1256155108186\\
-12.207	-11.7716677048446\\
-9.766	-9.41771989887048\\
-10.986	-10.594211633114\\
-7.324	-7.06280775540931\\
-9.766	-9.41771989887048\\
-8.545	-8.2402638271399\\
-6.104	-5.88631602116582\\
-7.324	-7.06280775540931\\
-9.766	-9.41771989887048\\
-7.324	-7.06280775540931\\
-8.545	-8.2402638271399\\
-9.766	-9.41771989887048\\
-10.986	-10.594211633114\\
-13.428	-12.9491237765751\\
-12.207	-11.7716677048446\\
-15.869	-15.3030715825492\\
-13.428	-12.9491237765751\\
-14.648	-14.1256155108186\\
-24.414	-23.5433354096891\\
-17.09	-16.4805276542798\\
-15.869	-15.3030715825492\\
-18.311	-17.6579837260104\\
-13.428	-12.9491237765751\\
-8.545	-8.2402638271399\\
-19.531	-18.8344754602539\\
-10.986	-10.594211633114\\
-8.545	-8.2402638271399\\
-10.986	-10.594211633114\\
-13.428	-12.9491237765751\\
-12.207	-11.7716677048446\\
-10.986	-10.594211633114\\
-6.104	-5.88631602116582\\
-12.207	-11.7716677048446\\
-10.986	-10.594211633114\\
-12.207	-11.7716677048446\\
-15.869	-15.3030715825492\\
-19.531	-18.8344754602539\\
-15.869	-15.3030715825492\\
-20.752	-20.0119315319844\\
-24.414	-23.5433354096891\\
-20.752	-20.0119315319844\\
-31.738	-30.6061431650984\\
-23.193	-22.3658793379585\\
-30.518	-29.4296514308549\\
-24.414	-23.5433354096891\\
-26.855	-25.8972832156632\\
-30.518	-29.4296514308549\\
-20.752	-20.0119315319844\\
-10.986	-10.594211633114\\
-12.207	-11.7716677048446\\
-15.869	-15.3030715825492\\
-18.311	-17.6579837260104\\
-15.869	-15.3030715825492\\
-9.766	-9.41771989887048\\
-10.986	-10.594211633114\\
-18.311	-17.6579837260104\\
-24.414	-23.5433354096891\\
-23.193	-22.3658793379585\\
-17.09	-16.4805276542798\\
-12.207	-11.7716677048446\\
-14.648	-14.1256155108186\\
-24.414	-23.5433354096891\\
-25.635	-24.7207914814197\\
-26.855	-25.8972832156632\\
-20.752	-20.0119315319844\\
-26.855	-25.8972832156632\\
-29.297	-28.2521953591243\\
-26.855	-25.8972832156632\\
-35.4	-34.1375470428031\\
-31.738	-30.6061431650984\\
-26.855	-25.8972832156632\\
-31.738	-30.6061431650984\\
-26.855	-25.8972832156632\\
-17.09	-16.4805276542798\\
-14.648	-14.1256155108186\\
-18.311	-17.6579837260104\\
-20.752	-20.0119315319844\\
-21.973	-21.189387603715\\
-18.311	-17.6579837260104\\
-20.752	-20.0119315319844\\
-18.311	-17.6579837260104\\
-14.648	-14.1256155108186\\
-17.09	-16.4805276542798\\
-25.635	-24.7207914814197\\
-24.414	-23.5433354096891\\
-20.752	-20.0119315319844\\
-14.648	-14.1256155108186\\
-12.207	-11.7716677048446\\
-13.428	-12.9491237765751\\
-14.648	-14.1256155108186\\
-9.766	-9.41771989887048\\
-8.545	-8.2402638271399\\
-14.648	-14.1256155108186\\
-17.09	-16.4805276542798\\
-20.752	-20.0119315319844\\
-19.531	-18.8344754602539\\
-12.207	-11.7716677048446\\
-8.545	-8.2402638271399\\
-9.766	-9.41771989887048\\
-8.545	-8.2402638271399\\
-13.428	-12.9491237765751\\
-14.648	-14.1256155108186\\
-13.428	-12.9491237765751\\
-19.531	-18.8344754602539\\
-25.635	-24.7207914814197\\
-24.414	-23.5433354096891\\
-26.855	-25.8972832156632\\
-30.518	-29.4296514308549\\
-17.09	-16.4805276542798\\
-14.648	-14.1256155108186\\
-15.869	-15.3030715825492\\
-18.311	-17.6579837260104\\
-15.869	-15.3030715825492\\
-10.986	-10.594211633114\\
-9.766	-9.41771989887048\\
-17.09	-16.4805276542798\\
-20.752	-20.0119315319844\\
-18.311	-17.6579837260104\\
-26.855	-25.8972832156632\\
-31.738	-30.6061431650984\\
-21.973	-21.189387603715\\
-14.648	-14.1256155108186\\
-13.428	-12.9491237765751\\
-10.986	-10.594211633114\\
-18.311	-17.6579837260104\\
-17.09	-16.4805276542798\\
-15.869	-15.3030715825492\\
-18.311	-17.6579837260104\\
-23.193	-22.3658793379585\\
-20.752	-20.0119315319844\\
-24.414	-23.5433354096891\\
-17.09	-16.4805276542798\\
-18.311	-17.6579837260104\\
-14.648	-14.1256155108186\\
-12.207	-11.7716677048446\\
-17.09	-16.4805276542798\\
-24.414	-23.5433354096891\\
-25.635	-24.7207914814197\\
-19.531	-18.8344754602539\\
-35.4	-34.1375470428031\\
-26.855	-25.8972832156632\\
-20.752	-20.0119315319844\\
-13.428	-12.9491237765751\\
-17.09	-16.4805276542798\\
-15.869	-15.3030715825492\\
-17.09	-16.4805276542798\\
-14.648	-14.1256155108186\\
-13.428	-12.9491237765751\\
-28.076	-27.0747392873938\\
-30.518	-29.4296514308549\\
-18.311	-17.6579837260104\\
-23.193	-22.3658793379585\\
-24.414	-23.5433354096891\\
-18.311	-17.6579837260104\\
-15.869	-15.3030715825492\\
-18.311	-17.6579837260104\\
-21.973	-21.189387603715\\
-28.076	-27.0747392873938\\
-32.959	-31.783599236829\\
-42.725	-41.2013191356995\\
-35.4	-34.1375470428031\\
-24.414	-23.5433354096891\\
-21.973	-21.189387603715\\
-18.311	-17.6579837260104\\
-13.428	-12.9491237765751\\
-12.207	-11.7716677048446\\
-13.428	-12.9491237765751\\
-8.545	-8.2402638271399\\
-9.766	-9.41771989887048\\
-14.648	-14.1256155108186\\
-12.207	-11.7716677048446\\
-8.545	-8.2402638271399\\
-7.324	-7.06280775540931\\
-14.648	-14.1256155108186\\
-23.193	-22.3658793379585\\
-12.207	-11.7716677048446\\
-17.09	-16.4805276542798\\
-14.648	-14.1256155108186\\
-17.09	-16.4805276542798\\
-15.869	-15.3030715825492\\
-10.986	-10.594211633114\\
-9.766	-9.41771989887048\\
-6.104	-5.88631602116582\\
-8.545	-8.2402638271399\\
-15.869	-15.3030715825492\\
-18.311	-17.6579837260104\\
-13.428	-12.9491237765751\\
-9.766	-9.41771989887048\\
-7.324	-7.06280775540931\\
-12.207	-11.7716677048446\\
-13.428	-12.9491237765751\\
-18.311	-17.6579837260104\\
-14.648	-14.1256155108186\\
-12.207	-11.7716677048446\\
-13.428	-12.9491237765751\\
-18.311	-17.6579837260104\\
-25.635	-24.7207914814197\\
-28.076	-27.0747392873938\\
-23.193	-22.3658793379585\\
-14.648	-14.1256155108186\\
-15.869	-15.3030715825492\\
-14.648	-14.1256155108186\\
-12.207	-11.7716677048446\\
-14.648	-14.1256155108186\\
-10.986	-10.594211633114\\
-8.545	-8.2402638271399\\
-13.428	-12.9491237765751\\
-17.09	-16.4805276542798\\
-21.973	-21.189387603715\\
-23.193	-22.3658793379585\\
-30.518	-29.4296514308549\\
-40.283	-38.8464069922383\\
-34.18	-32.9610553085596\\
-23.193	-22.3658793379585\\
-26.855	-25.8972832156632\\
-29.297	-28.2521953591243\\
-36.621	-35.3150031145337\\
-28.076	-27.0747392873938\\
-14.648	-14.1256155108186\\
-9.766	-9.41771989887048\\
-12.207	-11.7716677048446\\
-9.766	-9.41771989887048\\
-6.104	-5.88631602116582\\
-7.324	-7.06280775540931\\
-9.766	-9.41771989887048\\
-12.207	-11.7716677048446\\
-10.986	-10.594211633114\\
-9.766	-9.41771989887048\\
-10.986	-10.594211633114\\
-14.648	-14.1256155108186\\
-13.428	-12.9491237765751\\
-10.986	-10.594211633114\\
-7.324	-7.06280775540931\\
-10.986	-10.594211633114\\
-12.207	-11.7716677048446\\
-9.766	-9.41771989887048\\
-7.324	-7.06280775540931\\
-10.986	-10.594211633114\\
-13.428	-12.9491237765751\\
-9.766	-9.41771989887048\\
-13.428	-12.9491237765751\\
-14.648	-14.1256155108186\\
-19.531	-18.8344754602539\\
-13.428	-12.9491237765751\\
-15.869	-15.3030715825492\\
-17.09	-16.4805276542798\\
-15.869	-15.3030715825492\\
-13.428	-12.9491237765751\\
-18.311	-17.6579837260104\\
-25.635	-24.7207914814197\\
-23.193	-22.3658793379585\\
-25.635	-24.7207914814197\\
-21.973	-21.189387603715\\
-19.531	-18.8344754602539\\
-18.311	-17.6579837260104\\
-21.973	-21.189387603715\\
-19.531	-18.8344754602539\\
-20.752	-20.0119315319844\\
-23.193	-22.3658793379585\\
-40.283	-38.8464069922383\\
-34.18	-32.9610553085596\\
-17.09	-16.4805276542798\\
-12.207	-11.7716677048446\\
-14.648	-14.1256155108186\\
-17.09	-16.4805276542798\\
-20.752	-20.0119315319844\\
-23.193	-22.3658793379585\\
-24.414	-23.5433354096891\\
-26.855	-25.8972832156632\\
-19.531	-18.8344754602539\\
-15.869	-15.3030715825492\\
-19.531	-18.8344754602539\\
-20.752	-20.0119315319844\\
-12.207	-11.7716677048446\\
-10.986	-10.594211633114\\
-13.428	-12.9491237765751\\
-17.09	-16.4805276542798\\
-23.193	-22.3658793379585\\
-21.973	-21.189387603715\\
-14.648	-14.1256155108186\\
-12.207	-11.7716677048446\\
-18.311	-17.6579837260104\\
-23.193	-22.3658793379585\\
-26.855	-25.8972832156632\\
-31.738	-30.6061431650984\\
-36.621	-35.3150031145337\\
-30.518	-29.4296514308549\\
-29.297	-28.2521953591243\\
-19.531	-18.8344754602539\\
-13.428	-12.9491237765751\\
-21.973	-21.189387603715\\
-28.076	-27.0747392873938\\
-18.311	-17.6579837260104\\
-20.752	-20.0119315319844\\
-35.4	-34.1375470428031\\
-37.842	-36.4924591862642\\
-25.635	-24.7207914814197\\
-17.09	-16.4805276542798\\
-9.766	-9.41771989887048\\
-12.207	-11.7716677048446\\
-14.648	-14.1256155108186\\
-12.207	-11.7716677048446\\
-10.986	-10.594211633114\\
-12.207	-11.7716677048446\\
-8.545	-8.2402638271399\\
-12.207	-11.7716677048446\\
-15.869	-15.3030715825492\\
-18.311	-17.6579837260104\\
-17.09	-16.4805276542798\\
-19.531	-18.8344754602539\\
-28.076	-27.0747392873938\\
-24.414	-23.5433354096891\\
-17.09	-16.4805276542798\\
-15.869	-15.3030715825492\\
-19.531	-18.8344754602539\\
-14.648	-14.1256155108186\\
-10.986	-10.594211633114\\
-13.428	-12.9491237765751\\
-12.207	-11.7716677048446\\
-9.766	-9.41771989887048\\
-6.104	-5.88631602116582\\
-8.545	-8.2402638271399\\
-14.648	-14.1256155108186\\
-13.428	-12.9491237765751\\
-10.986	-10.594211633114\\
-8.545	-8.2402638271399\\
-15.869	-15.3030715825492\\
-18.311	-17.6579837260104\\
-10.986	-10.594211633114\\
-13.428	-12.9491237765751\\
-15.869	-15.3030715825492\\
-17.09	-16.4805276542798\\
-19.531	-18.8344754602539\\
-13.428	-12.9491237765751\\
-9.766	-9.41771989887048\\
-14.648	-14.1256155108186\\
-10.986	-10.594211633114\\
-13.428	-12.9491237765751\\
-19.531	-18.8344754602539\\
-25.635	-24.7207914814197\\
-21.973	-21.189387603715\\
-18.311	-17.6579837260104\\
-26.855	-25.8972832156632\\
-24.414	-23.5433354096891\\
-14.648	-14.1256155108186\\
-10.986	-10.594211633114\\
-9.766	-9.41771989887048\\
-8.545	-8.2402638271399\\
-7.324	-7.06280775540931\\
-14.648	-14.1256155108186\\
-13.428	-12.9491237765751\\
-8.545	-8.2402638271399\\
-12.207	-11.7716677048446\\
-13.428	-12.9491237765751\\
-9.766	-9.41771989887048\\
-8.545	-8.2402638271399\\
-4.883	-4.70885994943524\\
-7.324	-7.06280775540931\\
-14.648	-14.1256155108186\\
-19.531	-18.8344754602539\\
-17.09	-16.4805276542798\\
-12.207	-11.7716677048446\\
-14.648	-14.1256155108186\\
-15.869	-15.3030715825492\\
-20.752	-20.0119315319844\\
-23.193	-22.3658793379585\\
-19.531	-18.8344754602539\\
-10.986	-10.594211633114\\
-12.207	-11.7716677048446\\
-15.869	-15.3030715825492\\
-14.648	-14.1256155108186\\
-8.545	-8.2402638271399\\
-17.09	-16.4805276542798\\
-20.752	-20.0119315319844\\
-18.311	-17.6579837260104\\
-20.752	-20.0119315319844\\
-25.635	-24.7207914814197\\
-19.531	-18.8344754602539\\
-18.311	-17.6579837260104\\
-26.855	-25.8972832156632\\
-21.973	-21.189387603715\\
-24.414	-23.5433354096891\\
-25.635	-24.7207914814197\\
-37.842	-36.4924591862642\\
-39.063	-37.6699152579948\\
-25.635	-24.7207914814197\\
-32.959	-31.783599236829\\
-36.621	-35.3150031145337\\
-35.4	-34.1375470428031\\
-37.842	-36.4924591862642\\
-36.621	-35.3150031145337\\
-43.945	-42.377810869943\\
-41.504	-40.0238630639689\\
-24.414	-23.5433354096891\\
-19.531	-18.8344754602539\\
-18.311	-17.6579837260104\\
-15.869	-15.3030715825492\\
-12.207	-11.7716677048446\\
-17.09	-16.4805276542798\\
-23.193	-22.3658793379585\\
-18.311	-17.6579837260104\\
-13.428	-12.9491237765751\\
-15.869	-15.3030715825492\\
-12.207	-11.7716677048446\\
-10.986	-10.594211633114\\
-7.324	-7.06280775540931\\
-8.545	-8.2402638271399\\
-15.869	-15.3030715825492\\
-8.545	-8.2402638271399\\
-6.104	-5.88631602116582\\
-3.662	-3.53140387770466\\
-7.324	-7.06280775540931\\
-8.545	-8.2402638271399\\
-6.104	-5.88631602116582\\
-12.207	-11.7716677048446\\
-14.648	-14.1256155108186\\
-12.207	-11.7716677048446\\
-15.869	-15.3030715825492\\
-21.973	-21.189387603715\\
-17.09	-16.4805276542798\\
-6.104	-5.88631602116582\\
-10.986	-10.594211633114\\
-6.104	-5.88631602116582\\
-7.324	-7.06280775540931\\
-13.428	-12.9491237765751\\
-15.869	-15.3030715825492\\
-12.207	-11.7716677048446\\
-28.076	-27.0747392873938\\
-23.193	-22.3658793379585\\
-17.09	-16.4805276542798\\
-13.428	-12.9491237765751\\
-14.648	-14.1256155108186\\
-20.752	-20.0119315319844\\
-23.193	-22.3658793379585\\
-18.311	-17.6579837260104\\
-6.104	-5.88631602116582\\
-12.207	-11.7716677048446\\
-9.766	-9.41771989887048\\
-6.104	-5.88631602116582\\
-3.662	-3.53140387770466\\
-7.324	-7.06280775540931\\
-19.531	-18.8344754602539\\
-10.986	-10.594211633114\\
-8.545	-8.2402638271399\\
-9.766	-9.41771989887048\\
-6.104	-5.88631602116582\\
-3.662	-3.53140387770466\\
-9.766	-9.41771989887048\\
-14.648	-14.1256155108186\\
-18.311	-17.6579837260104\\
-20.752	-20.0119315319844\\
-19.531	-18.8344754602539\\
-20.752	-20.0119315319844\\
-19.531	-18.8344754602539\\
-18.311	-17.6579837260104\\
-21.973	-21.189387603715\\
-30.518	-29.4296514308549\\
-26.855	-25.8972832156632\\
-17.09	-16.4805276542798\\
-8.545	-8.2402638271399\\
-12.207	-11.7716677048446\\
-26.855	-25.8972832156632\\
-21.973	-21.189387603715\\
-13.428	-12.9491237765751\\
-15.869	-15.3030715825492\\
-20.752	-20.0119315319844\\
-29.297	-28.2521953591243\\
-31.738	-30.6061431650984\\
-29.297	-28.2521953591243\\
-26.855	-25.8972832156632\\
-24.414	-23.5433354096891\\
-17.09	-16.4805276542798\\
-13.428	-12.9491237765751\\
-15.869	-15.3030715825492\\
-19.531	-18.8344754602539\\
-17.09	-16.4805276542798\\
-19.531	-18.8344754602539\\
-13.428	-12.9491237765751\\
-18.311	-17.6579837260104\\
-17.09	-16.4805276542798\\
-7.324	-7.06280775540931\\
-4.883	-4.70885994943524\\
-7.324	-7.06280775540931\\
-8.545	-8.2402638271399\\
-3.662	-3.53140387770466\\
-7.324	-7.06280775540931\\
-17.09	-16.4805276542798\\
-3.662	-3.53140387770466\\
-10.986	-10.594211633114\\
-14.648	-14.1256155108186\\
-20.752	-20.0119315319844\\
-18.311	-17.6579837260104\\
-7.324	-7.06280775540931\\
-24.414	-23.5433354096891\\
-26.855	-25.8972832156632\\
-34.18	-32.9610553085596\\
-42.725	-41.2013191356995\\
-40.283	-38.8464069922383\\
-29.297	-28.2521953591243\\
-36.621	-35.3150031145337\\
-32.959	-31.783599236829\\
-29.297	-28.2521953591243\\
-30.518	-29.4296514308549\\
-35.4	-34.1375470428031\\
-46.387	-44.7327230134041\\
-36.621	-35.3150031145337\\
-18.311	-17.6579837260104\\
-10.986	-10.594211633114\\
-9.766	-9.41771989887048\\
-10.986	-10.594211633114\\
-12.207	-11.7716677048446\\
-10.986	-10.594211633114\\
-17.09	-16.4805276542798\\
-18.311	-17.6579837260104\\
-14.648	-14.1256155108186\\
-19.531	-18.8344754602539\\
-12.207	-11.7716677048446\\
-17.09	-16.4805276542798\\
-18.311	-17.6579837260104\\
-19.531	-18.8344754602539\\
-10.986	-10.594211633114\\
-6.104	-5.88631602116582\\
-10.986	-10.594211633114\\
-13.428	-12.9491237765751\\
-17.09	-16.4805276542798\\
-30.518	-29.4296514308549\\
-32.959	-31.783599236829\\
-35.4	-34.1375470428031\\
-45.166	-43.5552669416736\\
-40.283	-38.8464069922383\\
-26.855	-25.8972832156632\\
-17.09	-16.4805276542798\\
-13.428	-12.9491237765751\\
-14.648	-14.1256155108186\\
-15.869	-15.3030715825492\\
-8.545	-8.2402638271399\\
-12.207	-11.7716677048446\\
-14.648	-14.1256155108186\\
-18.311	-17.6579837260104\\
-15.869	-15.3030715825492\\
-7.324	-7.06280775540931\\
-4.883	-4.70885994943524\\
-7.324	-7.06280775540931\\
-4.883	-4.70885994943524\\
-6.104	-5.88631602116582\\
-10.986	-10.594211633114\\
-14.648	-14.1256155108186\\
-15.869	-15.3030715825492\\
-19.531	-18.8344754602539\\
-17.09	-16.4805276542798\\
-19.531	-18.8344754602539\\
-34.18	-32.9610553085596\\
-26.855	-25.8972832156632\\
-21.973	-21.189387603715\\
-28.076	-27.0747392873938\\
-31.738	-30.6061431650984\\
-20.752	-20.0119315319844\\
-18.311	-17.6579837260104\\
-13.428	-12.9491237765751\\
-15.869	-15.3030715825492\\
-28.076	-27.0747392873938\\
-41.504	-40.0238630639689\\
-46.387	-44.7327230134041\\
-36.621	-35.3150031145337\\
-31.738	-30.6061431650984\\
-26.855	-25.8972832156632\\
-28.076	-27.0747392873938\\
-34.18	-32.9610553085596\\
-23.193	-22.3658793379585\\
-18.311	-17.6579837260104\\
-24.414	-23.5433354096891\\
-25.635	-24.7207914814197\\
-14.648	-14.1256155108186\\
-13.428	-12.9491237765751\\
-17.09	-16.4805276542798\\
-26.855	-25.8972832156632\\
-25.635	-24.7207914814197\\
-20.752	-20.0119315319844\\
-18.311	-17.6579837260104\\
-17.09	-16.4805276542798\\
-14.648	-14.1256155108186\\
-8.545	-8.2402638271399\\
-7.324	-7.06280775540931\\
-6.104	-5.88631602116582\\
-10.986	-10.594211633114\\
-17.09	-16.4805276542798\\
-18.311	-17.6579837260104\\
-13.428	-12.9491237765751\\
-9.766	-9.41771989887048\\
-8.545	-8.2402638271399\\
-10.986	-10.594211633114\\
-14.648	-14.1256155108186\\
-17.09	-16.4805276542798\\
-14.648	-14.1256155108186\\
-9.766	-9.41771989887048\\
-20.752	-20.0119315319844\\
-21.973	-21.189387603715\\
-15.869	-15.3030715825492\\
-4.883	-4.70885994943524\\
-10.986	-10.594211633114\\
-13.428	-12.9491237765751\\
-14.648	-14.1256155108186\\
-9.766	-9.41771989887048\\
-7.324	-7.06280775540931\\
-13.428	-12.9491237765751\\
-21.973	-21.189387603715\\
-13.428	-12.9491237765751\\
-3.662	-3.53140387770466\\
-9.766	-9.41771989887048\\
-14.648	-14.1256155108186\\
-20.752	-20.0119315319844\\
-14.648	-14.1256155108186\\
-12.207	-11.7716677048446\\
-20.752	-20.0119315319844\\
-26.855	-25.8972832156632\\
-23.193	-22.3658793379585\\
-29.297	-28.2521953591243\\
-35.4	-34.1375470428031\\
-24.414	-23.5433354096891\\
-19.531	-18.8344754602539\\
-26.855	-25.8972832156632\\
-19.531	-18.8344754602539\\
-23.193	-22.3658793379585\\
-29.297	-28.2521953591243\\
-21.973	-21.189387603715\\
-10.986	-10.594211633114\\
-20.752	-20.0119315319844\\
-34.18	-32.9610553085596\\
-37.842	-36.4924591862642\\
-29.297	-28.2521953591243\\
-24.414	-23.5433354096891\\
-31.738	-30.6061431650984\\
-26.855	-25.8972832156632\\
-17.09	-16.4805276542798\\
-19.531	-18.8344754602539\\
-21.973	-21.189387603715\\
-19.531	-18.8344754602539\\
-21.973	-21.189387603715\\
-14.648	-14.1256155108186\\
-19.531	-18.8344754602539\\
-24.414	-23.5433354096891\\
-18.311	-17.6579837260104\\
-14.648	-14.1256155108186\\
-12.207	-11.7716677048446\\
-8.545	-8.2402638271399\\
-9.766	-9.41771989887048\\
-17.09	-16.4805276542798\\
-15.869	-15.3030715825492\\
-19.531	-18.8344754602539\\
-25.635	-24.7207914814197\\
-26.855	-25.8972832156632\\
-20.752	-20.0119315319844\\
-18.311	-17.6579837260104\\
-10.986	-10.594211633114\\
-14.648	-14.1256155108186\\
-24.414	-23.5433354096891\\
-18.311	-17.6579837260104\\
-8.545	-8.2402638271399\\
-15.869	-15.3030715825492\\
-25.635	-24.7207914814197\\
-23.193	-22.3658793379585\\
-30.518	-29.4296514308549\\
-39.063	-37.6699152579948\\
-30.518	-29.4296514308549\\
-26.855	-25.8972832156632\\
-29.297	-28.2521953591243\\
-20.752	-20.0119315319844\\
-14.648	-14.1256155108186\\
-17.09	-16.4805276542798\\
-24.414	-23.5433354096891\\
-25.635	-24.7207914814197\\
-13.428	-12.9491237765751\\
-14.648	-14.1256155108186\\
-12.207	-11.7716677048446\\
-17.09	-16.4805276542798\\
-14.648	-14.1256155108186\\
-12.207	-11.7716677048446\\
-20.752	-20.0119315319844\\
-24.414	-23.5433354096891\\
-15.869	-15.3030715825492\\
-12.207	-11.7716677048446\\
-20.752	-20.0119315319844\\
-13.428	-12.9491237765751\\
-9.766	-9.41771989887048\\
-10.986	-10.594211633114\\
-12.207	-11.7716677048446\\
-10.986	-10.594211633114\\
-6.104	-5.88631602116582\\
-10.986	-10.594211633114\\
-12.207	-11.7716677048446\\
-20.752	-20.0119315319844\\
-19.531	-18.8344754602539\\
-21.973	-21.189387603715\\
-28.076	-27.0747392873938\\
-14.648	-14.1256155108186\\
-25.635	-24.7207914814197\\
-36.621	-35.3150031145337\\
-26.855	-25.8972832156632\\
-25.635	-24.7207914814197\\
-39.063	-37.6699152579948\\
-31.738	-30.6061431650984\\
-25.635	-24.7207914814197\\
-20.752	-20.0119315319844\\
-25.635	-24.7207914814197\\
-18.311	-17.6579837260104\\
-17.09	-16.4805276542798\\
-28.076	-27.0747392873938\\
-31.738	-30.6061431650984\\
-18.311	-17.6579837260104\\
-8.545	-8.2402638271399\\
-15.869	-15.3030715825492\\
-8.545	-8.2402638271399\\
-7.324	-7.06280775540931\\
-10.986	-10.594211633114\\
-13.428	-12.9491237765751\\
-23.193	-22.3658793379585\\
-17.09	-16.4805276542798\\
-10.986	-10.594211633114\\
-9.766	-9.41771989887048\\
-7.324	-7.06280775540931\\
-8.545	-8.2402638271399\\
-7.324	-7.06280775540931\\
-14.648	-14.1256155108186\\
-12.207	-11.7716677048446\\
-7.324	-7.06280775540931\\
-8.545	-8.2402638271399\\
-9.766	-9.41771989887048\\
-10.986	-10.594211633114\\
-8.545	-8.2402638271399\\
-9.766	-9.41771989887048\\
-8.545	-8.2402638271399\\
-4.883	-4.70885994943524\\
-13.428	-12.9491237765751\\
-17.09	-16.4805276542798\\
-23.193	-22.3658793379585\\
-30.518	-29.4296514308549\\
-26.855	-25.8972832156632\\
-13.428	-12.9491237765751\\
-10.986	-10.594211633114\\
-12.207	-11.7716677048446\\
-7.324	-7.06280775540931\\
-9.766	-9.41771989887048\\
-10.986	-10.594211633114\\
-14.648	-14.1256155108186\\
-13.428	-12.9491237765751\\
-15.869	-15.3030715825492\\
-18.311	-17.6579837260104\\
-17.09	-16.4805276542798\\
-21.973	-21.189387603715\\
-30.518	-29.4296514308549\\
-19.531	-18.8344754602539\\
-9.766	-9.41771989887048\\
-8.545	-8.2402638271399\\
-15.869	-15.3030715825492\\
-10.986	-10.594211633114\\
-9.766	-9.41771989887048\\
-4.883	-4.70885994943524\\
-10.986	-10.594211633114\\
-7.324	-7.06280775540931\\
-3.662	-3.53140387770466\\
-6.104	-5.88631602116582\\
-10.986	-10.594211633114\\
-13.428	-12.9491237765751\\
-15.869	-15.3030715825492\\
-24.414	-23.5433354096891\\
-17.09	-16.4805276542798\\
-7.324	-7.06280775540931\\
-13.428	-12.9491237765751\\
-18.311	-17.6579837260104\\
-10.986	-10.594211633114\\
-13.428	-12.9491237765751\\
-20.752	-20.0119315319844\\
-17.09	-16.4805276542798\\
-25.635	-24.7207914814197\\
-30.518	-29.4296514308549\\
-20.752	-20.0119315319844\\
-17.09	-16.4805276542798\\
};
\addlegendentry{data2}

\end{axis}

\begin{axis}[%
width=4.927cm,
height=3.484cm,
at={(0cm,9.677cm)},
scale only axis,
xmin=-26.855,
xmax=1.221,
xlabel style={font=\color{white!15!black}},
xlabel={y(t-1)},
ymin=-26.855,
ymax=1.221,
ylabel style={font=\color{white!15!black}},
ylabel={y(t)},
axis background/.style={fill=white},
title={C4, R = 0.6952},
axis x line*=bottom,
axis y line*=left,
legend style={legend cell align=left, align=left, draw=white!15!black}
]
\addplot[only marks, mark=*, mark options={}, mark size=1.5000pt, color=mycolor1, fill=mycolor1] table[row sep=crcr]{%
x	y\\
-10.986	-10.986\\
-10.986	-15.869\\
-15.869	-10.986\\
-10.986	-10.986\\
-10.986	-14.648\\
-14.648	-14.648\\
-14.648	-17.09\\
-17.09	-10.986\\
-10.986	-7.324\\
-7.324	-14.648\\
-14.648	-8.545\\
-8.545	-9.766\\
-9.766	-6.104\\
-6.104	-3.662\\
-3.662	-3.662\\
-3.662	-3.662\\
-3.662	-2.441\\
-2.441	-13.428\\
-13.428	-13.428\\
-13.428	-13.428\\
-13.428	-9.766\\
-9.766	-7.324\\
-7.324	-10.986\\
-10.986	-12.207\\
-12.207	-6.104\\
-6.104	-9.766\\
-9.766	-13.428\\
-13.428	-7.324\\
-7.324	-15.869\\
-15.869	-13.428\\
-13.428	-10.986\\
-10.986	-17.09\\
-17.09	-13.428\\
-13.428	-10.986\\
-10.986	-8.545\\
-8.545	-8.545\\
-8.545	-10.986\\
-10.986	-12.207\\
-12.207	-10.986\\
-10.986	-9.766\\
-9.766	-10.986\\
-10.986	-10.986\\
-10.986	-15.869\\
-15.869	-14.648\\
-14.648	-10.986\\
-10.986	-10.986\\
-10.986	-6.104\\
-6.104	-4.883\\
-4.883	-8.545\\
-8.545	-7.324\\
-7.324	-7.324\\
-7.324	-10.986\\
-10.986	-9.766\\
-9.766	-9.766\\
-9.766	-15.869\\
-15.869	-8.545\\
-8.545	-6.104\\
-6.104	-7.324\\
-7.324	-8.545\\
-8.545	-9.766\\
-9.766	-4.883\\
-4.883	-7.324\\
-7.324	-7.324\\
-7.324	-7.324\\
-7.324	-6.104\\
-6.104	-7.324\\
-7.324	-8.545\\
-8.545	-9.766\\
-9.766	-10.986\\
-10.986	-10.986\\
-10.986	-12.207\\
-12.207	-13.428\\
-13.428	-10.986\\
-10.986	-12.207\\
-12.207	-9.766\\
-9.766	-9.766\\
-9.766	-9.766\\
-9.766	-17.09\\
-17.09	-20.752\\
-20.752	-19.531\\
-19.531	-19.531\\
-19.531	-15.869\\
-15.869	-20.752\\
-20.752	-23.193\\
-23.193	-23.193\\
-23.193	-14.648\\
-14.648	-23.193\\
-23.193	-26.855\\
-26.855	-21.973\\
-21.973	-15.869\\
-15.869	-14.648\\
-14.648	-10.986\\
-10.986	-8.545\\
-8.545	-10.986\\
-10.986	-8.545\\
-8.545	-6.104\\
-6.104	-6.104\\
-6.104	-8.545\\
-8.545	-9.766\\
-9.766	-10.986\\
-10.986	-9.766\\
-9.766	-12.207\\
-12.207	-10.986\\
-10.986	-15.869\\
-15.869	-13.428\\
-13.428	-18.311\\
-18.311	-12.207\\
-12.207	-10.986\\
-10.986	-12.207\\
-12.207	-9.766\\
-9.766	-8.545\\
-8.545	-10.986\\
-10.986	-10.986\\
-10.986	-8.545\\
-8.545	-8.545\\
-8.545	-7.324\\
-7.324	-7.324\\
-7.324	-7.324\\
-7.324	-4.883\\
-4.883	-6.104\\
-6.104	-8.545\\
-8.545	-13.428\\
-13.428	-12.207\\
-12.207	-13.428\\
-13.428	-8.545\\
-8.545	-15.869\\
-15.869	-17.09\\
-17.09	-13.428\\
-13.428	-14.648\\
-14.648	-14.648\\
-14.648	-10.986\\
-10.986	-7.324\\
-7.324	-8.545\\
-8.545	-8.545\\
-8.545	-17.09\\
-17.09	-20.752\\
-20.752	-19.531\\
-19.531	-14.648\\
-14.648	-15.869\\
-15.869	-13.428\\
-13.428	-12.207\\
-12.207	-12.207\\
-12.207	-9.766\\
-9.766	-8.545\\
-8.545	-8.545\\
-8.545	-9.766\\
-9.766	-8.545\\
-8.545	-13.428\\
-13.428	-15.869\\
-15.869	-10.986\\
-10.986	-8.545\\
-8.545	-9.766\\
-9.766	-8.545\\
-8.545	-6.104\\
-6.104	-3.662\\
-3.662	-4.883\\
-4.883	-8.545\\
-8.545	-9.766\\
-9.766	-4.883\\
-4.883	-7.324\\
-7.324	-7.324\\
-7.324	-3.662\\
-3.662	-3.662\\
-3.662	-2.441\\
-2.441	-4.883\\
-4.883	-10.986\\
-10.986	-12.207\\
-12.207	-14.648\\
-14.648	-13.428\\
-13.428	-8.545\\
-8.545	-7.324\\
-7.324	-4.883\\
-4.883	-7.324\\
-7.324	-6.104\\
-6.104	-10.986\\
-10.986	-12.207\\
-12.207	-21.973\\
-21.973	-23.193\\
-23.193	-21.973\\
-21.973	-18.311\\
-18.311	-13.428\\
-13.428	-17.09\\
-17.09	-15.869\\
-15.869	-19.531\\
-19.531	-12.207\\
-12.207	-13.428\\
-13.428	-12.207\\
-12.207	-13.428\\
-13.428	-13.428\\
-13.428	-10.986\\
-10.986	-18.311\\
-18.311	-17.09\\
-17.09	-13.428\\
-13.428	-7.324\\
-7.324	-12.207\\
-12.207	-15.869\\
-15.869	-17.09\\
-17.09	-19.531\\
-19.531	-21.973\\
-21.973	-20.752\\
-20.752	-15.869\\
-15.869	-14.648\\
-14.648	-18.311\\
-18.311	-19.531\\
-19.531	-13.428\\
-13.428	-9.766\\
-9.766	-13.428\\
-13.428	-9.766\\
-9.766	-7.324\\
-7.324	-9.766\\
-9.766	-10.986\\
-10.986	-7.324\\
-7.324	-7.324\\
-7.324	-9.766\\
-9.766	-6.104\\
-6.104	-10.986\\
-10.986	-13.428\\
-13.428	-13.428\\
-13.428	-8.545\\
-8.545	-9.766\\
-9.766	-12.207\\
-12.207	-18.311\\
-18.311	-15.869\\
-15.869	-9.766\\
-9.766	-8.545\\
-8.545	-9.766\\
-9.766	-7.324\\
-7.324	-8.545\\
-8.545	-7.324\\
-7.324	-8.545\\
-8.545	-10.986\\
-10.986	-10.986\\
-10.986	-6.104\\
-6.104	-10.986\\
-10.986	-13.428\\
-13.428	-13.428\\
-13.428	-9.766\\
-9.766	-12.207\\
-12.207	-13.428\\
-13.428	-10.986\\
-10.986	-4.883\\
-4.883	-10.986\\
-10.986	-8.545\\
-8.545	-8.545\\
-8.545	-6.104\\
-6.104	-6.104\\
-6.104	-10.986\\
-10.986	-10.986\\
-10.986	-6.104\\
-6.104	-8.545\\
-8.545	-8.545\\
-8.545	-4.883\\
-4.883	-8.545\\
-8.545	-4.883\\
-4.883	-7.324\\
-7.324	-4.883\\
-4.883	-4.883\\
-4.883	-10.986\\
-10.986	-12.207\\
-12.207	-12.207\\
-12.207	-15.869\\
-15.869	-10.986\\
-10.986	-6.104\\
-6.104	-7.324\\
-7.324	-4.883\\
-4.883	-7.324\\
-7.324	-6.104\\
-6.104	-3.662\\
-3.662	-8.545\\
-8.545	-9.766\\
-9.766	-9.766\\
-9.766	-17.09\\
-17.09	-10.986\\
-10.986	-13.428\\
-13.428	-7.324\\
-7.324	-8.545\\
-8.545	-3.662\\
-3.662	-1.221\\
-1.221	-6.104\\
-6.104	-9.766\\
-9.766	-8.545\\
-8.545	-9.766\\
-9.766	-9.766\\
-9.766	-17.09\\
-17.09	-10.986\\
-10.986	-10.986\\
-10.986	-10.986\\
-10.986	-13.428\\
-13.428	-15.869\\
-15.869	-15.869\\
-15.869	-10.986\\
-10.986	-6.104\\
-6.104	-3.662\\
-3.662	-4.883\\
-4.883	-6.104\\
-6.104	-4.883\\
-4.883	-6.104\\
-6.104	-6.104\\
-6.104	-9.766\\
-9.766	-9.766\\
-9.766	-8.545\\
-8.545	-9.766\\
-9.766	-6.104\\
-6.104	-3.662\\
-3.662	-7.324\\
-7.324	-10.986\\
-10.986	-8.545\\
-8.545	-12.207\\
-12.207	-8.545\\
-8.545	-7.324\\
-7.324	-8.545\\
-8.545	-12.207\\
-12.207	-14.648\\
-14.648	-10.986\\
-10.986	-17.09\\
-17.09	-9.766\\
-9.766	-12.207\\
-12.207	-13.428\\
-13.428	-13.428\\
-13.428	-9.766\\
-9.766	-9.766\\
-9.766	-12.207\\
-12.207	-18.311\\
-18.311	-15.869\\
-15.869	-8.545\\
-8.545	-12.207\\
-12.207	-9.766\\
-9.766	-12.207\\
-12.207	-13.428\\
-13.428	-9.766\\
-9.766	-9.766\\
-9.766	-14.648\\
-14.648	-13.428\\
-13.428	-12.207\\
-12.207	-8.545\\
-8.545	-9.766\\
-9.766	-15.869\\
-15.869	-14.648\\
-14.648	-15.869\\
-15.869	-12.207\\
-12.207	-10.986\\
-10.986	-9.766\\
-9.766	-8.545\\
-8.545	-4.883\\
-4.883	-3.662\\
-3.662	-3.662\\
-3.662	-7.324\\
-7.324	-10.986\\
-10.986	-9.766\\
-9.766	-8.545\\
-8.545	-10.986\\
-10.986	-9.766\\
-9.766	-8.545\\
-8.545	-3.662\\
-3.662	-6.104\\
-6.104	-6.104\\
-6.104	-6.104\\
-6.104	-8.545\\
-8.545	-10.986\\
-10.986	-7.324\\
-7.324	-6.104\\
-6.104	-7.324\\
-7.324	-7.324\\
-7.324	-6.104\\
-6.104	-8.545\\
-8.545	-15.869\\
-15.869	-14.648\\
-14.648	-17.09\\
-17.09	-17.09\\
-17.09	-17.09\\
-17.09	-13.428\\
-13.428	-10.986\\
-10.986	-9.766\\
-9.766	-10.986\\
-10.986	-12.207\\
-12.207	-14.648\\
-14.648	-14.648\\
-14.648	-17.09\\
-17.09	-19.531\\
-19.531	-21.973\\
-21.973	-18.311\\
-18.311	-21.973\\
-21.973	-19.531\\
-19.531	-14.648\\
-14.648	-17.09\\
-17.09	-14.648\\
-14.648	-10.986\\
-10.986	-8.545\\
-8.545	-8.545\\
-8.545	-10.986\\
-10.986	-7.324\\
-7.324	-7.324\\
-7.324	-7.324\\
-7.324	-7.324\\
-7.324	-3.662\\
-3.662	-7.324\\
-7.324	-7.324\\
-7.324	-6.104\\
-6.104	-8.545\\
-8.545	-12.207\\
-12.207	-8.545\\
-8.545	-12.207\\
-12.207	-18.311\\
-18.311	-17.09\\
-17.09	-19.531\\
-19.531	-17.09\\
-17.09	-14.648\\
-14.648	-14.648\\
-14.648	-7.324\\
-7.324	-9.766\\
-9.766	-8.545\\
-8.545	-6.104\\
-6.104	-7.324\\
-7.324	-8.545\\
-8.545	-2.441\\
-2.441	-4.883\\
-4.883	-8.545\\
-8.545	-9.766\\
-9.766	-12.207\\
-12.207	-12.207\\
-12.207	-10.986\\
-10.986	-12.207\\
-12.207	-14.648\\
-14.648	-10.986\\
-10.986	-9.766\\
-9.766	-7.324\\
-7.324	-10.986\\
-10.986	-10.986\\
-10.986	-7.324\\
-7.324	-10.986\\
-10.986	-13.428\\
-13.428	-10.986\\
-10.986	-8.545\\
-8.545	-8.545\\
-8.545	-8.545\\
-8.545	-6.104\\
-6.104	-7.324\\
-7.324	-6.104\\
-6.104	-9.766\\
-9.766	-8.545\\
-8.545	-10.986\\
-10.986	-10.986\\
-10.986	-8.545\\
-8.545	-6.104\\
-6.104	-6.104\\
-6.104	-10.986\\
-10.986	-14.648\\
-14.648	-14.648\\
-14.648	-12.207\\
-12.207	-10.986\\
-10.986	-9.766\\
-9.766	-8.545\\
-8.545	-8.545\\
-8.545	-9.766\\
-9.766	-12.207\\
-12.207	-7.324\\
-7.324	-7.324\\
-7.324	-10.986\\
-10.986	-10.986\\
-10.986	-13.428\\
-13.428	-13.428\\
-13.428	-14.648\\
-14.648	-20.752\\
-20.752	-21.973\\
-21.973	-14.648\\
-14.648	-9.766\\
-9.766	-7.324\\
-7.324	-6.104\\
-6.104	-7.324\\
-7.324	-4.883\\
-4.883	-7.324\\
-7.324	-7.324\\
-7.324	-7.324\\
-7.324	-10.986\\
-10.986	-12.207\\
-12.207	-13.428\\
-13.428	-13.428\\
-13.428	-18.311\\
-18.311	-17.09\\
-17.09	-12.207\\
-12.207	-12.207\\
-12.207	-9.766\\
-9.766	-9.766\\
-9.766	-13.428\\
-13.428	-9.766\\
-9.766	-8.545\\
-8.545	-10.986\\
-10.986	-6.104\\
-6.104	-10.986\\
-10.986	-14.648\\
-14.648	-8.545\\
-8.545	-8.545\\
-8.545	-15.869\\
-15.869	-13.428\\
-13.428	-10.986\\
-10.986	-17.09\\
-17.09	-17.09\\
-17.09	-10.986\\
-10.986	-8.545\\
-8.545	-9.766\\
-9.766	-12.207\\
-12.207	-18.311\\
-18.311	-19.531\\
-19.531	-18.311\\
-18.311	-15.869\\
-15.869	-9.766\\
-9.766	-9.766\\
-9.766	-7.324\\
-7.324	-6.104\\
-6.104	-6.104\\
-6.104	-7.324\\
-7.324	-9.766\\
-9.766	-8.545\\
-8.545	-6.104\\
-6.104	-12.207\\
-12.207	-10.986\\
-10.986	-13.428\\
-13.428	-17.09\\
-17.09	-19.531\\
-19.531	-15.869\\
-15.869	-12.207\\
-12.207	-15.869\\
-15.869	-12.207\\
-12.207	-9.766\\
-9.766	-10.986\\
-10.986	-15.869\\
-15.869	-18.311\\
-18.311	-12.207\\
-12.207	-8.545\\
-8.545	-4.883\\
-4.883	-2.441\\
-2.441	0\\
0	-6.104\\
-6.104	-8.545\\
-8.545	-8.545\\
-8.545	-7.324\\
-7.324	-6.104\\
-6.104	-6.104\\
-6.104	-7.324\\
-7.324	-4.883\\
-4.883	-6.104\\
-6.104	-8.545\\
-8.545	-10.986\\
-10.986	-13.428\\
-13.428	-10.986\\
-10.986	-12.207\\
-12.207	-17.09\\
-17.09	-17.09\\
-17.09	-19.531\\
-19.531	-20.752\\
-20.752	-13.428\\
-13.428	-10.986\\
-10.986	-10.986\\
-10.986	-14.648\\
-14.648	-17.09\\
-17.09	-13.428\\
-13.428	-9.766\\
-9.766	-7.324\\
-7.324	-3.662\\
-3.662	-4.883\\
-4.883	-6.104\\
-6.104	-1.221\\
-1.221	-9.766\\
-9.766	-7.324\\
-7.324	-6.104\\
-6.104	-4.883\\
-4.883	-3.662\\
-3.662	-6.104\\
-6.104	-6.104\\
-6.104	-4.883\\
-4.883	-3.662\\
-3.662	-3.662\\
-3.662	-4.883\\
-4.883	-12.207\\
-12.207	-13.428\\
-13.428	-15.869\\
-15.869	-14.648\\
-14.648	-14.648\\
-14.648	-23.193\\
-23.193	-20.752\\
-20.752	-18.311\\
-18.311	-18.311\\
-18.311	-15.869\\
-15.869	-15.869\\
-15.869	-14.648\\
-14.648	-10.986\\
-10.986	-8.545\\
-8.545	-8.545\\
-8.545	-12.207\\
-12.207	-14.648\\
-14.648	-8.545\\
-8.545	-9.766\\
-9.766	-9.766\\
-9.766	-9.766\\
-9.766	-10.986\\
-10.986	-10.986\\
-10.986	-12.207\\
-12.207	-10.986\\
-10.986	-9.766\\
-9.766	-9.766\\
-9.766	-10.986\\
-10.986	-2.441\\
-2.441	-4.883\\
-4.883	-8.545\\
-8.545	-7.324\\
-7.324	1.221\\
1.221	-2.441\\
-2.441	-7.324\\
-7.324	-7.324\\
-7.324	-8.545\\
-8.545	-8.545\\
-8.545	-8.545\\
-8.545	-10.986\\
-10.986	-7.324\\
-7.324	-8.545\\
-8.545	-10.986\\
-10.986	-4.883\\
-4.883	-4.883\\
-4.883	-3.662\\
-3.662	-3.662\\
-3.662	-2.441\\
-2.441	-3.662\\
-3.662	-10.986\\
-10.986	-12.207\\
-12.207	-8.545\\
-8.545	-8.545\\
-8.545	-7.324\\
-7.324	-6.104\\
-6.104	-8.545\\
-8.545	-6.104\\
-6.104	-6.104\\
-6.104	-6.104\\
-6.104	-3.662\\
-3.662	-4.883\\
-4.883	-7.324\\
-7.324	-7.324\\
-7.324	-3.662\\
-3.662	-4.883\\
-4.883	-6.104\\
-6.104	-6.104\\
-6.104	-9.766\\
-9.766	-13.428\\
-13.428	-6.104\\
-6.104	-7.324\\
-7.324	-9.766\\
-9.766	-8.545\\
-8.545	-12.207\\
-12.207	-7.324\\
-7.324	-7.324\\
-7.324	-10.986\\
-10.986	-7.324\\
-7.324	-9.766\\
-9.766	-6.104\\
-6.104	-7.324\\
-7.324	-8.545\\
-8.545	-8.545\\
-8.545	-7.324\\
-7.324	-8.545\\
-8.545	-8.545\\
-8.545	-10.986\\
-10.986	-10.986\\
-10.986	-18.311\\
-18.311	-13.428\\
-13.428	-10.986\\
-10.986	-9.766\\
-9.766	-9.766\\
-9.766	-13.428\\
-13.428	-9.766\\
-9.766	-10.986\\
-10.986	-14.648\\
-14.648	-4.883\\
-4.883	-3.662\\
-3.662	-10.986\\
-10.986	-7.324\\
-7.324	-9.766\\
-9.766	-10.986\\
-10.986	-12.207\\
-12.207	-10.986\\
-10.986	-9.766\\
-9.766	-8.545\\
-8.545	-12.207\\
-12.207	-8.545\\
-8.545	-10.986\\
-10.986	-10.986\\
-10.986	-7.324\\
-7.324	-8.545\\
-8.545	-4.883\\
-4.883	-6.104\\
-6.104	-10.986\\
-10.986	-13.428\\
-13.428	-10.986\\
-10.986	-13.428\\
-13.428	-9.766\\
-9.766	-7.324\\
-7.324	-7.324\\
-7.324	-7.324\\
-7.324	-9.766\\
-9.766	-12.207\\
-12.207	-13.428\\
-13.428	-12.207\\
-12.207	-7.324\\
-7.324	-8.545\\
-8.545	-19.531\\
-19.531	-12.207\\
-12.207	-13.428\\
-13.428	-15.869\\
-15.869	-10.986\\
-10.986	-9.766\\
-9.766	-9.766\\
-9.766	-9.766\\
-9.766	-8.545\\
-8.545	-13.428\\
-13.428	-10.986\\
-10.986	-9.766\\
-9.766	-10.986\\
-10.986	-10.986\\
-10.986	-8.545\\
-8.545	-9.766\\
-9.766	-4.883\\
-4.883	-9.766\\
-9.766	-12.207\\
-12.207	-9.766\\
-9.766	-7.324\\
-7.324	-3.662\\
-3.662	-2.441\\
-2.441	-3.662\\
-3.662	-9.766\\
-9.766	-13.428\\
-13.428	-10.986\\
-10.986	-10.986\\
-10.986	-13.428\\
-13.428	-12.207\\
-12.207	-7.324\\
-7.324	-6.104\\
-6.104	-3.662\\
-3.662	-8.545\\
-8.545	-4.883\\
-4.883	-8.545\\
-8.545	-7.324\\
-7.324	-7.324\\
-7.324	-6.104\\
-6.104	-4.883\\
-4.883	-9.766\\
-9.766	-10.986\\
-10.986	-6.104\\
-6.104	-10.986\\
-10.986	-8.545\\
-8.545	-9.766\\
-9.766	-14.648\\
-14.648	-8.545\\
-8.545	-12.207\\
-12.207	-12.207\\
-12.207	-3.662\\
-3.662	-12.207\\
-12.207	-13.428\\
-13.428	-9.766\\
-9.766	-15.869\\
-15.869	-13.428\\
-13.428	-13.428\\
-13.428	-10.986\\
-10.986	-15.869\\
-15.869	-13.428\\
-13.428	-12.207\\
-12.207	-10.986\\
-10.986	-8.545\\
-8.545	-9.766\\
-9.766	-8.545\\
-8.545	-6.104\\
-6.104	-8.545\\
-8.545	-7.324\\
-7.324	-10.986\\
-10.986	-15.869\\
-15.869	-13.428\\
-13.428	-14.648\\
-14.648	-14.648\\
-14.648	-10.986\\
-10.986	-7.324\\
-7.324	-7.324\\
-7.324	-4.883\\
-4.883	-6.104\\
-6.104	-9.766\\
-9.766	-12.207\\
-12.207	-12.207\\
-12.207	-12.207\\
-12.207	-8.545\\
-8.545	-9.766\\
-9.766	-17.09\\
-17.09	-14.648\\
-14.648	-13.428\\
-13.428	-17.09\\
-17.09	-23.193\\
-23.193	-7.324\\
-7.324	-7.324\\
-7.324	-8.545\\
-8.545	-8.545\\
-8.545	-12.207\\
-12.207	-17.09\\
-17.09	-14.648\\
-14.648	-10.986\\
-10.986	-9.766\\
-9.766	-6.104\\
-6.104	-6.104\\
-6.104	-8.545\\
-8.545	-6.104\\
-6.104	-3.662\\
-3.662	-6.104\\
-6.104	-3.662\\
-3.662	-4.883\\
-4.883	-12.207\\
-12.207	-7.324\\
-7.324	-9.766\\
-9.766	-7.324\\
-7.324	-9.766\\
-9.766	-9.766\\
-9.766	-10.986\\
-10.986	-13.428\\
-13.428	-10.986\\
-10.986	-13.428\\
-13.428	-12.207\\
-12.207	-13.428\\
-13.428	-9.766\\
-9.766	-14.648\\
-14.648	-13.428\\
-13.428	-7.324\\
-7.324	-9.766\\
-9.766	-10.986\\
-10.986	-8.545\\
-8.545	-13.428\\
-13.428	-13.428\\
-13.428	-14.648\\
-14.648	-13.428\\
-13.428	-20.752\\
-20.752	-20.752\\
-20.752	-10.986\\
-10.986	-12.207\\
-12.207	-10.986\\
-10.986	-15.869\\
-15.869	-7.324\\
-7.324	-10.986\\
-10.986	-15.869\\
-15.869	-4.883\\
-4.883	-7.324\\
-7.324	-9.766\\
-9.766	-10.986\\
-10.986	-6.104\\
-6.104	-6.104\\
-6.104	-4.883\\
-4.883	-4.883\\
-4.883	-6.104\\
-6.104	-7.324\\
-7.324	-7.324\\
-7.324	-9.766\\
-9.766	-13.428\\
-13.428	-10.986\\
-10.986	-14.648\\
-14.648	-13.428\\
-13.428	-12.207\\
-12.207	-8.545\\
-8.545	-10.986\\
-10.986	-9.766\\
-9.766	-15.869\\
-15.869	-13.428\\
-13.428	-13.428\\
-13.428	-12.207\\
-12.207	-7.324\\
-7.324	-12.207\\
-12.207	-14.648\\
-14.648	-10.986\\
-10.986	-13.428\\
-13.428	-9.766\\
-9.766	-4.883\\
-4.883	-6.104\\
-6.104	-6.104\\
-6.104	-9.766\\
-9.766	-6.104\\
-6.104	-8.545\\
-8.545	-8.545\\
-8.545	-14.648\\
-14.648	-20.752\\
-20.752	-12.207\\
-12.207	-14.648\\
-14.648	-14.648\\
-14.648	-10.986\\
-10.986	-9.766\\
-9.766	-10.986\\
-10.986	-10.986\\
-10.986	-12.207\\
-12.207	-14.648\\
-14.648	-18.311\\
-18.311	-19.531\\
-19.531	-14.648\\
-14.648	-19.531\\
-19.531	-15.869\\
-15.869	-14.648\\
-14.648	-12.207\\
-12.207	-10.986\\
-10.986	-13.428\\
-13.428	-18.311\\
-18.311	-7.324\\
-7.324	-8.545\\
-8.545	-12.207\\
-12.207	-9.766\\
-9.766	-8.545\\
-8.545	-3.662\\
-3.662	-7.324\\
-7.324	-13.428\\
-13.428	-20.752\\
-20.752	-21.973\\
-21.973	-19.531\\
-19.531	-14.648\\
-14.648	-17.09\\
-17.09	-21.973\\
-21.973	-17.09\\
-17.09	-14.648\\
-14.648	-18.311\\
-18.311	-13.428\\
-13.428	-13.428\\
-13.428	-15.869\\
-15.869	-12.207\\
-12.207	-7.324\\
-7.324	-10.986\\
-10.986	-10.986\\
-10.986	-6.104\\
-6.104	-9.766\\
-9.766	-7.324\\
-7.324	-13.428\\
-13.428	-9.766\\
-9.766	-8.545\\
-8.545	-9.766\\
-9.766	-12.207\\
-12.207	-12.207\\
-12.207	-10.986\\
-10.986	-7.324\\
-7.324	-13.428\\
-13.428	-14.648\\
-14.648	-9.766\\
-9.766	-13.428\\
-13.428	-12.207\\
-12.207	-9.766\\
-9.766	-9.766\\
-9.766	-6.104\\
-6.104	-4.883\\
-4.883	-6.104\\
-6.104	-9.766\\
-9.766	-4.883\\
-4.883	-6.104\\
-6.104	-4.883\\
-4.883	-12.207\\
-12.207	-10.986\\
-10.986	-12.207\\
-12.207	-14.648\\
-14.648	-13.428\\
-13.428	-14.648\\
-14.648	-10.986\\
-10.986	-4.883\\
-4.883	-8.545\\
-8.545	-7.324\\
-7.324	-7.324\\
-7.324	-10.986\\
-10.986	-10.986\\
-10.986	-15.869\\
-15.869	-15.869\\
-15.869	-9.766\\
-9.766	-15.869\\
-15.869	-17.09\\
-17.09	-6.104\\
-6.104	-7.324\\
-7.324	-9.766\\
-9.766	-10.986\\
-10.986	-14.648\\
-14.648	-4.883\\
-4.883	-10.986\\
-10.986	-9.766\\
-9.766	-10.986\\
-10.986	-13.428\\
-13.428	-18.311\\
-18.311	-15.869\\
-15.869	-14.648\\
-14.648	-13.428\\
-13.428	-8.545\\
-8.545	-9.766\\
-9.766	0\\
0	-7.324\\
-7.324	-10.986\\
-10.986	-7.324\\
-7.324	-8.545\\
-8.545	-17.09\\
-17.09	-15.869\\
-15.869	-18.311\\
-18.311	-18.311\\
-18.311	-10.986\\
-10.986	-10.986\\
-10.986	-8.545\\
-8.545	-8.545\\
-8.545	-15.869\\
-15.869	-17.09\\
-17.09	-14.648\\
-14.648	-9.766\\
-9.766	-10.986\\
-10.986	-15.869\\
-15.869	-13.428\\
-13.428	-7.324\\
-7.324	-7.324\\
-7.324	-7.324\\
-7.324	-7.324\\
-7.324	-4.883\\
-4.883	-6.104\\
-6.104	-10.986\\
-10.986	-9.766\\
-9.766	-6.104\\
-6.104	-7.324\\
-7.324	-6.104\\
-6.104	-1.221\\
-1.221	-1.221\\
-1.221	-7.324\\
-7.324	-7.324\\
-7.324	-7.324\\
-7.324	-10.986\\
-10.986	-10.986\\
-10.986	-13.428\\
-13.428	-18.311\\
-18.311	-20.752\\
-20.752	-18.311\\
-18.311	-17.09\\
-17.09	-12.207\\
-12.207	-10.986\\
-10.986	-12.207\\
-12.207	-12.207\\
-12.207	-12.207\\
-12.207	-8.545\\
-8.545	-7.324\\
-7.324	-6.104\\
-6.104	-9.766\\
-9.766	-10.986\\
-10.986	-7.324\\
-7.324	-3.662\\
-3.662	-6.104\\
-6.104	-7.324\\
-7.324	-4.883\\
-4.883	-8.545\\
-8.545	-8.545\\
-8.545	-8.545\\
-8.545	-14.648\\
-14.648	-10.986\\
-10.986	-14.648\\
-14.648	-19.531\\
-19.531	-21.973\\
-21.973	-14.648\\
-14.648	-10.986\\
-10.986	-8.545\\
-8.545	-7.324\\
-7.324	-9.766\\
-9.766	-8.545\\
-8.545	-8.545\\
-8.545	-9.766\\
-9.766	-7.324\\
-7.324	-9.766\\
-9.766	-8.545\\
-8.545	-7.324\\
-7.324	-9.766\\
-9.766	-4.883\\
-4.883	-2.441\\
-2.441	-1.221\\
-1.221	-3.662\\
-3.662	-6.104\\
-6.104	-2.441\\
-2.441	-2.441\\
-2.441	-7.324\\
-7.324	-6.104\\
-6.104	-6.104\\
-6.104	-9.766\\
-9.766	-10.986\\
-10.986	-8.545\\
-8.545	-7.324\\
-7.324	-13.428\\
-13.428	-19.531\\
-19.531	-13.428\\
-13.428	-10.986\\
-10.986	-9.766\\
-9.766	-3.662\\
-3.662	-7.324\\
-7.324	-8.545\\
-8.545	-9.766\\
-9.766	-13.428\\
-13.428	-13.428\\
-13.428	-10.986\\
-10.986	-9.766\\
-9.766	-9.766\\
-9.766	-8.545\\
-8.545	-7.324\\
-7.324	-4.883\\
-4.883	-4.883\\
-4.883	-4.883\\
-4.883	-3.662\\
-3.662	-6.104\\
-6.104	-7.324\\
-7.324	-10.986\\
-10.986	-9.766\\
-9.766	-4.883\\
-4.883	-9.766\\
-9.766	-12.207\\
-12.207	-6.104\\
-6.104	-13.428\\
-13.428	-12.207\\
-12.207	-7.324\\
-7.324	-8.545\\
-8.545	-12.207\\
-12.207	-9.766\\
-9.766	-6.104\\
-6.104	-9.766\\
-9.766	-4.883\\
-4.883	-3.662\\
-3.662	-2.441\\
-2.441	-4.883\\
-4.883	-7.324\\
-7.324	-7.324\\
-7.324	-8.545\\
-8.545	-8.545\\
-8.545	-4.883\\
-4.883	-3.662\\
-3.662	-2.441\\
-2.441	-4.883\\
-4.883	-4.883\\
-4.883	-4.883\\
-4.883	-8.545\\
-8.545	-10.986\\
-10.986	-6.104\\
-6.104	-8.545\\
-8.545	-10.986\\
-10.986	-13.428\\
-13.428	-13.428\\
-13.428	-13.428\\
-13.428	-10.986\\
-10.986	-12.207\\
-12.207	-13.428\\
-13.428	-14.648\\
-14.648	-13.428\\
-13.428	-18.311\\
-18.311	-15.869\\
-15.869	-15.869\\
-15.869	-12.207\\
-12.207	-13.428\\
-13.428	-13.428\\
-13.428	-13.428\\
-13.428	-9.766\\
-9.766	-7.324\\
-7.324	-10.986\\
-10.986	-13.428\\
-13.428	-12.207\\
-12.207	-10.986\\
-10.986	-8.545\\
-8.545	-12.207\\
-12.207	-8.545\\
-8.545	-14.648\\
-14.648	-15.869\\
-15.869	-10.986\\
-10.986	-6.104\\
-6.104	-9.766\\
-9.766	-2.441\\
-2.441	-3.662\\
-3.662	-7.324\\
-7.324	-4.883\\
-4.883	-3.662\\
-3.662	-8.545\\
-8.545	-4.883\\
-4.883	-7.324\\
-7.324	-6.104\\
-6.104	-4.883\\
-4.883	-4.883\\
-4.883	-6.104\\
-6.104	-6.104\\
-6.104	-7.324\\
-7.324	-6.104\\
-6.104	-7.324\\
-7.324	-7.324\\
-7.324	-7.324\\
-7.324	-14.648\\
-14.648	-9.766\\
-9.766	-8.545\\
-8.545	-10.986\\
-10.986	-7.324\\
-7.324	-6.104\\
-6.104	-10.986\\
-10.986	-8.545\\
-8.545	-4.883\\
-4.883	-10.986\\
-10.986	-8.545\\
-8.545	-8.545\\
-8.545	-7.324\\
-7.324	-2.441\\
-2.441	-3.662\\
-3.662	-4.883\\
-4.883	-4.883\\
-4.883	-7.324\\
-7.324	-8.545\\
-8.545	-8.545\\
-8.545	-9.766\\
-9.766	-7.324\\
-7.324	-10.986\\
-10.986	-14.648\\
-14.648	-12.207\\
-12.207	-10.986\\
-10.986	-14.648\\
-14.648	-12.207\\
-12.207	-17.09\\
-17.09	-15.869\\
-15.869	-19.531\\
-19.531	-20.752\\
-20.752	-13.428\\
-13.428	-6.104\\
-6.104	-6.104\\
-6.104	-8.545\\
-8.545	-12.207\\
-12.207	-12.207\\
-12.207	-8.545\\
-8.545	-7.324\\
-7.324	-10.986\\
-10.986	-13.428\\
-13.428	-14.648\\
-14.648	-13.428\\
-13.428	-9.766\\
-9.766	-6.104\\
-6.104	-10.986\\
-10.986	-12.207\\
-12.207	-14.648\\
-14.648	-14.648\\
-14.648	-12.207\\
-12.207	-19.531\\
-19.531	-19.531\\
-19.531	-17.09\\
-17.09	-20.752\\
-20.752	-18.311\\
-18.311	-15.869\\
-15.869	-17.09\\
-17.09	-15.869\\
-15.869	-10.986\\
-10.986	-8.545\\
-8.545	-9.766\\
-9.766	-12.207\\
-12.207	-12.207\\
-12.207	-8.545\\
-8.545	-13.428\\
-13.428	-12.207\\
-12.207	-8.545\\
-8.545	-9.766\\
-9.766	-12.207\\
-12.207	-13.428\\
-13.428	-14.648\\
-14.648	-10.986\\
-10.986	-9.766\\
-9.766	-8.545\\
-8.545	-12.207\\
-12.207	-8.545\\
-8.545	-7.324\\
-7.324	-2.441\\
-2.441	-7.324\\
-7.324	-9.766\\
-9.766	-12.207\\
-12.207	-12.207\\
-12.207	-8.545\\
-8.545	-6.104\\
-6.104	-6.104\\
-6.104	-7.324\\
-7.324	-8.545\\
-8.545	-9.766\\
-9.766	-6.104\\
-6.104	-13.428\\
-13.428	-13.428\\
-13.428	-14.648\\
-14.648	-15.869\\
-15.869	-17.09\\
-17.09	-12.207\\
-12.207	-12.207\\
-12.207	-7.324\\
-7.324	-8.545\\
-8.545	-12.207\\
-12.207	-10.986\\
-10.986	-6.104\\
-6.104	-6.104\\
-6.104	-8.545\\
-8.545	-12.207\\
-12.207	-10.986\\
-10.986	-18.311\\
-18.311	-18.311\\
-18.311	-12.207\\
-12.207	-8.545\\
-8.545	-7.324\\
-7.324	-8.545\\
-8.545	-13.428\\
-13.428	-12.207\\
-12.207	-13.428\\
-13.428	-14.648\\
-14.648	-12.207\\
-12.207	-10.986\\
-10.986	-13.428\\
-13.428	-9.766\\
-9.766	-12.207\\
-12.207	-7.324\\
-7.324	-7.324\\
-7.324	-10.986\\
-10.986	-13.428\\
-13.428	-14.648\\
-14.648	-8.545\\
-8.545	-14.648\\
-14.648	-13.428\\
-13.428	-12.207\\
-12.207	-7.324\\
-7.324	-17.09\\
-17.09	-8.545\\
-8.545	-10.986\\
-10.986	-12.207\\
-12.207	-8.545\\
-8.545	-6.104\\
-6.104	-17.09\\
-17.09	-20.752\\
-20.752	-12.207\\
-12.207	-15.869\\
-15.869	-15.869\\
-15.869	-13.428\\
-13.428	-12.207\\
-12.207	-10.986\\
-10.986	-8.545\\
-8.545	-7.324\\
-7.324	-10.986\\
-10.986	-17.09\\
-17.09	-17.09\\
-17.09	-20.752\\
-20.752	-15.869\\
-15.869	-14.648\\
-14.648	-10.986\\
-10.986	-13.428\\
-13.428	-9.766\\
-9.766	-8.545\\
-8.545	-8.545\\
-8.545	-7.324\\
-7.324	-7.324\\
-7.324	-3.662\\
-3.662	-8.545\\
-8.545	-10.986\\
-10.986	-6.104\\
-6.104	-6.104\\
-6.104	-6.104\\
-6.104	-6.104\\
-6.104	-14.648\\
-14.648	-6.104\\
-6.104	-3.662\\
-3.662	-7.324\\
-7.324	-8.545\\
-8.545	-9.766\\
-9.766	-8.545\\
-8.545	-7.324\\
-7.324	-6.104\\
-6.104	-4.883\\
-4.883	-4.883\\
-4.883	-10.986\\
-10.986	-10.986\\
-10.986	-8.545\\
-8.545	-6.104\\
-6.104	-4.883\\
-4.883	-8.545\\
-8.545	-10.986\\
-10.986	-10.986\\
-10.986	-9.766\\
-9.766	-6.104\\
-6.104	-7.324\\
-7.324	-7.324\\
-7.324	-10.986\\
-10.986	-14.648\\
-14.648	-15.869\\
-15.869	-13.428\\
-13.428	-7.324\\
-7.324	-12.207\\
-12.207	-9.766\\
-9.766	-9.766\\
-9.766	-6.104\\
-6.104	-9.766\\
-9.766	-8.545\\
-8.545	-8.545\\
-8.545	-6.104\\
-6.104	-4.883\\
-4.883	-7.324\\
-7.324	-9.766\\
-9.766	-13.428\\
-13.428	-13.428\\
-13.428	-12.207\\
-12.207	-13.428\\
-13.428	-13.428\\
-13.428	-17.09\\
-17.09	-21.973\\
-21.973	-23.193\\
-23.193	-13.428\\
-13.428	-17.09\\
-17.09	-15.869\\
-15.869	-20.752\\
-20.752	-14.648\\
-14.648	-7.324\\
-7.324	-6.104\\
-6.104	-7.324\\
-7.324	-6.104\\
-6.104	-3.662\\
-3.662	-2.441\\
-2.441	-4.883\\
-4.883	-8.545\\
-8.545	-9.766\\
-9.766	-4.883\\
-4.883	-7.324\\
-7.324	-6.104\\
-6.104	-8.545\\
-8.545	-8.545\\
-8.545	-7.324\\
-7.324	-4.883\\
-4.883	-3.662\\
-3.662	-3.662\\
-3.662	-6.104\\
-6.104	-7.324\\
-7.324	-3.662\\
-3.662	-3.662\\
-3.662	-9.766\\
-9.766	-7.324\\
-7.324	-9.766\\
-9.766	-7.324\\
-7.324	-8.545\\
-8.545	-12.207\\
-12.207	-9.766\\
-9.766	-12.207\\
-12.207	-10.986\\
-10.986	-7.324\\
-7.324	-6.104\\
-6.104	-13.428\\
-13.428	-13.428\\
-13.428	-13.428\\
-13.428	-14.648\\
-14.648	-10.986\\
-10.986	-10.986\\
-10.986	-10.986\\
-10.986	-12.207\\
-12.207	-8.545\\
-8.545	-17.09\\
-17.09	-13.428\\
-13.428	-14.648\\
-14.648	-23.193\\
-23.193	-15.869\\
-15.869	-8.545\\
-8.545	-7.324\\
-7.324	-6.104\\
-6.104	-9.766\\
-9.766	-13.428\\
-13.428	-15.869\\
-15.869	-14.648\\
-14.648	-15.869\\
-15.869	-9.766\\
-9.766	-9.766\\
-9.766	-10.986\\
-10.986	-12.207\\
-12.207	-6.104\\
-6.104	-4.883\\
-4.883	-7.324\\
-7.324	-8.545\\
-8.545	-10.986\\
-10.986	-13.428\\
-13.428	-10.986\\
-10.986	-8.545\\
-8.545	-7.324\\
-7.324	-10.986\\
-10.986	-14.648\\
-14.648	-15.869\\
-15.869	-17.09\\
-17.09	-20.752\\
-20.752	-14.648\\
-14.648	-18.311\\
-18.311	-12.207\\
-12.207	-8.545\\
-8.545	-14.648\\
-14.648	-15.869\\
-15.869	-10.986\\
-10.986	-10.986\\
-10.986	-15.869\\
-15.869	-19.531\\
-19.531	-17.09\\
-17.09	-7.324\\
-7.324	-6.104\\
-6.104	-12.207\\
-12.207	-6.104\\
-6.104	-8.545\\
-8.545	-6.104\\
-6.104	-9.766\\
-9.766	-3.662\\
-3.662	-4.883\\
-4.883	-7.324\\
-7.324	-9.766\\
-9.766	-10.986\\
-10.986	-12.207\\
-12.207	-13.428\\
-13.428	-15.869\\
-15.869	-13.428\\
-13.428	-8.545\\
-8.545	-7.324\\
-7.324	-10.986\\
-10.986	-8.545\\
-8.545	-7.324\\
-7.324	-4.883\\
-4.883	-4.883\\
-4.883	-6.104\\
-6.104	-4.883\\
-4.883	-3.662\\
-3.662	-4.883\\
-4.883	-4.883\\
-4.883	-8.545\\
-8.545	-7.324\\
-7.324	-6.104\\
-6.104	-4.883\\
-4.883	-9.766\\
-9.766	-9.766\\
-9.766	-7.324\\
-7.324	-9.766\\
-9.766	-8.545\\
-8.545	-10.986\\
-10.986	-9.766\\
-9.766	-10.986\\
-10.986	-9.766\\
-9.766	-7.324\\
-7.324	-7.324\\
-7.324	-7.324\\
-7.324	-8.545\\
-8.545	-8.545\\
-8.545	-9.766\\
-9.766	-14.648\\
-14.648	-9.766\\
-9.766	-10.986\\
-10.986	-15.869\\
-15.869	-10.986\\
-10.986	-9.766\\
-9.766	-8.545\\
-8.545	-7.324\\
-7.324	-4.883\\
-4.883	-3.662\\
-3.662	-6.104\\
-6.104	-10.986\\
-10.986	-7.324\\
-7.324	-9.766\\
-9.766	-7.324\\
-7.324	-8.545\\
-8.545	-2.441\\
-2.441	-4.883\\
-4.883	-2.441\\
-2.441	-2.441\\
-2.441	-8.545\\
-8.545	-10.986\\
-10.986	-7.324\\
-7.324	-7.324\\
-7.324	-9.766\\
-9.766	-9.766\\
-9.766	-9.766\\
-9.766	-12.207\\
-12.207	-12.207\\
-12.207	-13.428\\
-13.428	-12.207\\
-12.207	-8.545\\
-8.545	-7.324\\
-7.324	-7.324\\
-7.324	-9.766\\
-9.766	-8.545\\
-8.545	-8.545\\
-8.545	-8.545\\
-8.545	-12.207\\
-12.207	-10.986\\
-10.986	-12.207\\
-12.207	-15.869\\
-15.869	-8.545\\
-8.545	-8.545\\
-8.545	-13.428\\
-13.428	-13.428\\
-13.428	-15.869\\
-15.869	-14.648\\
-14.648	-21.973\\
-21.973	-25.635\\
-25.635	-15.869\\
-15.869	-18.311\\
-18.311	-21.973\\
-21.973	-20.752\\
-20.752	-24.414\\
-24.414	-23.193\\
-23.193	-25.635\\
-25.635	-20.752\\
-20.752	-14.648\\
-14.648	-10.986\\
-10.986	-13.428\\
-13.428	-9.766\\
-9.766	-8.545\\
-8.545	-9.766\\
-9.766	-13.428\\
-13.428	-10.986\\
-10.986	-8.545\\
-8.545	-8.545\\
-8.545	-8.545\\
-8.545	-8.545\\
-8.545	-6.104\\
-6.104	-10.986\\
-10.986	-7.324\\
-7.324	-7.324\\
-7.324	-6.104\\
-6.104	-7.324\\
-7.324	-3.662\\
-3.662	-3.662\\
-3.662	-3.662\\
-3.662	-3.662\\
-3.662	-6.104\\
-6.104	-6.104\\
-6.104	-9.766\\
-9.766	-10.986\\
-10.986	-8.545\\
-8.545	-8.545\\
-8.545	-12.207\\
-12.207	-9.766\\
-9.766	-4.883\\
-4.883	-7.324\\
-7.324	-4.883\\
-4.883	-6.104\\
-6.104	-8.545\\
-8.545	-8.545\\
-8.545	-4.883\\
-4.883	-9.766\\
-9.766	-14.648\\
-14.648	-15.869\\
-15.869	-10.986\\
-10.986	-7.324\\
-7.324	-8.545\\
-8.545	-12.207\\
-12.207	-13.428\\
-13.428	-7.324\\
-7.324	-6.104\\
-6.104	-6.104\\
-6.104	-8.545\\
-8.545	-2.441\\
-2.441	-2.441\\
-2.441	-7.324\\
-7.324	-10.986\\
-10.986	-10.986\\
-10.986	-7.324\\
-7.324	-6.104\\
-6.104	-6.104\\
-6.104	-3.662\\
-3.662	-7.324\\
-7.324	-1.221\\
-1.221	-3.662\\
-3.662	-7.324\\
-7.324	-10.986\\
-10.986	-10.986\\
-10.986	-10.986\\
-10.986	-12.207\\
-12.207	-10.986\\
-10.986	-10.986\\
-10.986	-8.545\\
-8.545	-10.986\\
-10.986	-17.09\\
-17.09	-18.311\\
-18.311	-8.545\\
-8.545	-4.883\\
-4.883	-4.883\\
-4.883	-14.648\\
-14.648	-10.986\\
-10.986	-8.545\\
-8.545	-9.766\\
-9.766	-10.986\\
-10.986	-17.09\\
-17.09	-18.311\\
-18.311	-18.311\\
-18.311	-19.531\\
-19.531	-14.648\\
-14.648	-13.428\\
-13.428	-9.766\\
-9.766	-9.766\\
-9.766	-8.545\\
-8.545	-10.986\\
-10.986	-9.766\\
-9.766	-12.207\\
-12.207	-8.545\\
-8.545	-9.766\\
-9.766	-8.545\\
-8.545	-6.104\\
-6.104	-4.883\\
-4.883	-4.883\\
-4.883	-4.883\\
-4.883	-4.883\\
-4.883	-2.441\\
-2.441	-4.883\\
-4.883	-7.324\\
-7.324	-6.104\\
-6.104	-6.104\\
-6.104	-8.545\\
-8.545	-12.207\\
-12.207	-8.545\\
-8.545	-4.883\\
-4.883	-1.221\\
-1.221	-7.324\\
-7.324	-13.428\\
-13.428	-14.648\\
-14.648	-20.752\\
-20.752	-24.414\\
-24.414	-24.414\\
-24.414	-15.869\\
-15.869	-17.09\\
-17.09	-21.973\\
-21.973	-18.311\\
-18.311	-15.869\\
-15.869	-18.311\\
-18.311	-20.752\\
-20.752	-19.531\\
-19.531	-26.855\\
-26.855	-20.752\\
-20.752	-12.207\\
-12.207	-7.324\\
-7.324	-6.104\\
-6.104	-6.104\\
-6.104	-6.104\\
-6.104	-6.104\\
-6.104	-10.986\\
-10.986	-12.207\\
-12.207	-8.545\\
-8.545	-10.986\\
-10.986	-6.104\\
-6.104	-3.662\\
-3.662	-9.766\\
-9.766	-12.207\\
-12.207	-9.766\\
-9.766	-6.104\\
-6.104	-6.104\\
-6.104	-7.324\\
-7.324	-10.986\\
-10.986	-17.09\\
-17.09	-15.869\\
-15.869	-17.09\\
-17.09	-20.752\\
-20.752	-25.635\\
-25.635	-21.973\\
-21.973	-17.09\\
-17.09	-10.986\\
-10.986	-8.545\\
-8.545	-4.883\\
-4.883	-7.324\\
-7.324	-9.766\\
-9.766	-8.545\\
-8.545	-6.104\\
-6.104	-7.324\\
-7.324	-8.545\\
-8.545	-8.545\\
-8.545	-10.986\\
-10.986	-9.766\\
-9.766	-4.883\\
-4.883	-4.883\\
-4.883	-2.441\\
-2.441	-2.441\\
-2.441	-3.662\\
-3.662	-7.324\\
-7.324	-8.545\\
-8.545	-7.324\\
-7.324	-10.986\\
-10.986	-8.545\\
-8.545	-8.545\\
-8.545	-19.531\\
-19.531	-14.648\\
-14.648	-12.207\\
-12.207	-17.09\\
-17.09	-18.311\\
-18.311	-13.428\\
-13.428	-12.207\\
-12.207	-10.986\\
-10.986	-9.766\\
-9.766	-14.648\\
-14.648	-24.414\\
-24.414	-25.635\\
-25.635	-20.752\\
-20.752	-19.531\\
-19.531	-14.648\\
-14.648	-17.09\\
-17.09	-19.531\\
-19.531	-14.648\\
-14.648	-10.986\\
-10.986	-9.766\\
-9.766	-12.207\\
-12.207	-13.428\\
-13.428	-8.545\\
-8.545	-7.324\\
-7.324	-7.324\\
-7.324	-8.545\\
-8.545	-14.648\\
-14.648	-17.09\\
-17.09	-12.207\\
-12.207	-10.986\\
-10.986	-9.766\\
-9.766	-9.766\\
-9.766	-6.104\\
-6.104	-4.883\\
-4.883	-6.104\\
-6.104	-6.104\\
-6.104	-4.883\\
-4.883	-7.324\\
-7.324	-9.766\\
-9.766	-10.986\\
-10.986	-8.545\\
-8.545	-3.662\\
-3.662	-3.662\\
-3.662	-6.104\\
-6.104	-8.545\\
-8.545	-9.766\\
-9.766	-8.545\\
-8.545	-4.883\\
-4.883	-10.986\\
-10.986	-14.648\\
-14.648	-10.986\\
-10.986	-7.324\\
-7.324	-9.766\\
-9.766	-8.545\\
-8.545	-8.545\\
-8.545	-6.104\\
-6.104	-6.104\\
-6.104	-4.883\\
-4.883	-10.986\\
-10.986	-9.766\\
-9.766	-3.662\\
-3.662	-6.104\\
-6.104	-9.766\\
-9.766	-12.207\\
-12.207	-7.324\\
-7.324	-6.104\\
-6.104	-10.986\\
-10.986	-13.428\\
-13.428	-12.207\\
-12.207	-17.09\\
-17.09	-19.531\\
-19.531	-12.207\\
-12.207	-12.207\\
-12.207	-17.09\\
-17.09	-13.428\\
-13.428	-10.986\\
-10.986	-18.311\\
-18.311	-14.648\\
-14.648	-7.324\\
-7.324	-8.545\\
-8.545	-19.531\\
-19.531	-21.973\\
-21.973	-17.09\\
-17.09	-12.207\\
-12.207	-17.09\\
-17.09	-14.648\\
-14.648	-9.766\\
-9.766	-8.545\\
-8.545	-12.207\\
-12.207	-12.207\\
-12.207	-12.207\\
-12.207	-12.207\\
-12.207	-9.766\\
-9.766	-9.766\\
-9.766	-12.207\\
-12.207	-10.986\\
-10.986	-6.104\\
-6.104	-6.104\\
-6.104	-4.883\\
-4.883	-6.104\\
-6.104	-10.986\\
-10.986	-8.545\\
-8.545	-13.428\\
-13.428	-13.428\\
-13.428	-15.869\\
-15.869	-13.428\\
-13.428	-9.766\\
-9.766	-9.766\\
-9.766	-6.104\\
-6.104	-14.648\\
-14.648	-10.986\\
-10.986	-7.324\\
-7.324	-13.428\\
-13.428	-10.986\\
-10.986	-13.428\\
-13.428	-15.869\\
-15.869	-20.752\\
-20.752	-17.09\\
-17.09	-14.648\\
-14.648	-15.869\\
-15.869	-14.648\\
-14.648	-7.324\\
-7.324	-12.207\\
-12.207	-13.428\\
-13.428	-13.428\\
-13.428	-15.869\\
-15.869	-12.207\\
-12.207	-8.545\\
-8.545	-8.545\\
-8.545	-8.545\\
-8.545	-8.545\\
-8.545	-6.104\\
-6.104	-10.986\\
-10.986	-13.428\\
-13.428	-7.324\\
-7.324	-7.324\\
-7.324	-10.986\\
-10.986	-9.766\\
-9.766	-3.662\\
-3.662	-6.104\\
-6.104	-9.766\\
-9.766	-6.104\\
-6.104	-4.883\\
-4.883	-4.883\\
-4.883	-2.441\\
-2.441	-8.545\\
-8.545	-8.545\\
-8.545	-9.766\\
-9.766	-13.428\\
-13.428	-17.09\\
-17.09	-13.428\\
-13.428	-6.104\\
-6.104	-12.207\\
-12.207	-20.752\\
-20.752	-15.869\\
-15.869	-14.648\\
-14.648	-19.531\\
-19.531	-17.09\\
-17.09	-13.428\\
-13.428	-13.428\\
-13.428	-12.207\\
-12.207	-15.869\\
-15.869	-14.648\\
-14.648	-10.986\\
-10.986	-14.648\\
-14.648	-20.752\\
-20.752	-18.311\\
-18.311	-9.766\\
-9.766	-7.324\\
-7.324	-6.104\\
-6.104	-14.648\\
-14.648	-2.441\\
-2.441	-3.662\\
-3.662	-7.324\\
-7.324	-10.986\\
-10.986	-12.207\\
-12.207	-12.207\\
-12.207	-9.766\\
-9.766	-3.662\\
-3.662	-4.883\\
-4.883	-3.662\\
-3.662	-7.324\\
-7.324	-3.662\\
-3.662	-8.545\\
-8.545	-8.545\\
-8.545	-7.324\\
-7.324	-6.104\\
-6.104	-6.104\\
-6.104	-6.104\\
-6.104	-7.324\\
-7.324	-3.662\\
-3.662	0\\
0	-2.441\\
-2.441	-6.104\\
-6.104	-10.986\\
-10.986	-10.986\\
-10.986	-18.311\\
-18.311	-14.648\\
-14.648	-9.766\\
-9.766	-6.104\\
-6.104	-8.545\\
-8.545	-1.221\\
-1.221	-3.662\\
-3.662	-6.104\\
-6.104	-12.207\\
-12.207	-7.324\\
-7.324	-7.324\\
-7.324	-9.766\\
-9.766	-10.986\\
-10.986	-9.766\\
-9.766	-10.986\\
-10.986	-13.428\\
-13.428	-17.09\\
-17.09	-14.648\\
-14.648	-2.441\\
-2.441	-3.662\\
-3.662	-8.545\\
-8.545	-8.545\\
-8.545	-4.883\\
-4.883	-6.104\\
-6.104	-3.662\\
-3.662	-1.221\\
-1.221	-1.221\\
-1.221	-2.441\\
-2.441	-4.883\\
-4.883	-4.883\\
-4.883	-8.545\\
-8.545	-13.428\\
-13.428	-13.428\\
-13.428	-6.104\\
-6.104	-6.104\\
-6.104	-13.428\\
-13.428	-7.324\\
-7.324	-3.662\\
-3.662	-12.207\\
-12.207	-8.545\\
-8.545	-13.428\\
-13.428	-21.973\\
-21.973	-10.986\\
-10.986	-8.545\\
-8.545	-10.986\\
};
\addlegendentry{data1}

\addplot [color=mycolor2, line width=2.0pt]
  table[row sep=crcr]{%
-10.986	-10.4803982980153\\
-15.869	-15.1386710896782\\
-10.986	-10.4803982980153\\
-14.648	-13.9738643973538\\
-17.09	-16.3034777820027\\
-10.986	-10.4803982980153\\
-7.324	-6.98693219867688\\
-14.648	-13.9738643973538\\
-8.545	-8.15173889100136\\
-9.766	-9.31654558332584\\
-6.104	-5.82307948398739\\
-3.662	-3.49346609933844\\
-2.441	-2.32865940701396\\
-13.428	-12.8100116826643\\
-9.766	-9.31654558332584\\
-7.324	-6.98693219867688\\
-10.986	-10.4803982980153\\
-12.207	-11.6452049903398\\
-6.104	-5.82307948398739\\
-9.766	-9.31654558332584\\
-13.428	-12.8100116826643\\
-7.324	-6.98693219867688\\
-15.869	-15.1386710896782\\
-13.428	-12.8100116826643\\
-10.986	-10.4803982980153\\
-17.09	-16.3034777820027\\
-13.428	-12.8100116826643\\
-10.986	-10.4803982980153\\
-8.545	-8.15173889100136\\
-10.986	-10.4803982980153\\
-12.207	-11.6452049903398\\
-10.986	-10.4803982980153\\
-9.766	-9.31654558332584\\
-10.986	-10.4803982980153\\
-15.869	-15.1386710896782\\
-14.648	-13.9738643973538\\
-10.986	-10.4803982980153\\
-6.104	-5.82307948398739\\
-4.883	-4.65827279166292\\
-8.545	-8.15173889100136\\
-7.324	-6.98693219867688\\
-10.986	-10.4803982980153\\
-9.766	-9.31654558332584\\
-15.869	-15.1386710896782\\
-8.545	-8.15173889100136\\
-6.104	-5.82307948398739\\
-7.324	-6.98693219867688\\
-8.545	-8.15173889100136\\
-9.766	-9.31654558332584\\
-4.883	-4.65827279166292\\
-7.324	-6.98693219867688\\
-6.104	-5.82307948398739\\
-7.324	-6.98693219867688\\
-8.545	-8.15173889100136\\
-9.766	-9.31654558332584\\
-10.986	-10.4803982980153\\
-12.207	-11.6452049903398\\
-13.428	-12.8100116826643\\
-10.986	-10.4803982980153\\
-12.207	-11.6452049903398\\
-9.766	-9.31654558332584\\
-17.09	-16.3034777820027\\
-20.752	-19.7969438813412\\
-19.531	-18.6321371890167\\
-15.869	-15.1386710896782\\
-20.752	-19.7969438813412\\
-23.193	-22.1256032883551\\
-14.648	-13.9738643973538\\
-23.193	-22.1256032883551\\
-26.855	-25.6190693876936\\
-21.973	-20.9617505736656\\
-15.869	-15.1386710896782\\
-14.648	-13.9738643973538\\
-10.986	-10.4803982980153\\
-8.545	-8.15173889100136\\
-10.986	-10.4803982980153\\
-8.545	-8.15173889100136\\
-6.104	-5.82307948398739\\
-8.545	-8.15173889100136\\
-9.766	-9.31654558332584\\
-10.986	-10.4803982980153\\
-9.766	-9.31654558332584\\
-12.207	-11.6452049903398\\
-10.986	-10.4803982980153\\
-15.869	-15.1386710896782\\
-13.428	-12.8100116826643\\
-18.311	-17.4682844743272\\
-12.207	-11.6452049903398\\
-10.986	-10.4803982980153\\
-12.207	-11.6452049903398\\
-9.766	-9.31654558332584\\
-8.545	-8.15173889100136\\
-10.986	-10.4803982980153\\
-8.545	-8.15173889100136\\
-7.324	-6.98693219867688\\
-4.883	-4.65827279166292\\
-6.104	-5.82307948398739\\
-8.545	-8.15173889100136\\
-13.428	-12.8100116826643\\
-12.207	-11.6452049903398\\
-13.428	-12.8100116826643\\
-8.545	-8.15173889100136\\
-15.869	-15.1386710896782\\
-17.09	-16.3034777820027\\
-13.428	-12.8100116826643\\
-14.648	-13.9738643973538\\
-10.986	-10.4803982980153\\
-7.324	-6.98693219867688\\
-8.545	-8.15173889100136\\
-17.09	-16.3034777820027\\
-20.752	-19.7969438813412\\
-19.531	-18.6321371890167\\
-14.648	-13.9738643973538\\
-15.869	-15.1386710896782\\
-13.428	-12.8100116826643\\
-12.207	-11.6452049903398\\
-9.766	-9.31654558332584\\
-8.545	-8.15173889100136\\
-9.766	-9.31654558332584\\
-8.545	-8.15173889100136\\
-13.428	-12.8100116826643\\
-15.869	-15.1386710896782\\
-10.986	-10.4803982980153\\
-8.545	-8.15173889100136\\
-9.766	-9.31654558332584\\
-8.545	-8.15173889100136\\
-6.104	-5.82307948398739\\
-3.662	-3.49346609933844\\
-4.883	-4.65827279166292\\
-8.545	-8.15173889100136\\
-9.766	-9.31654558332584\\
-4.883	-4.65827279166292\\
-7.324	-6.98693219867688\\
-3.662	-3.49346609933844\\
-2.441	-2.32865940701396\\
-4.883	-4.65827279166292\\
-10.986	-10.4803982980153\\
-12.207	-11.6452049903398\\
-14.648	-13.9738643973538\\
-13.428	-12.8100116826643\\
-8.545	-8.15173889100136\\
-7.324	-6.98693219867688\\
-4.883	-4.65827279166292\\
-7.324	-6.98693219867688\\
-6.104	-5.82307948398739\\
-10.986	-10.4803982980153\\
-12.207	-11.6452049903398\\
-21.973	-20.9617505736656\\
-23.193	-22.1256032883551\\
-21.973	-20.9617505736656\\
-18.311	-17.4682844743272\\
-13.428	-12.8100116826643\\
-17.09	-16.3034777820027\\
-15.869	-15.1386710896782\\
-19.531	-18.6321371890167\\
-12.207	-11.6452049903398\\
-13.428	-12.8100116826643\\
-12.207	-11.6452049903398\\
-13.428	-12.8100116826643\\
-10.986	-10.4803982980153\\
-18.311	-17.4682844743272\\
-17.09	-16.3034777820027\\
-13.428	-12.8100116826643\\
-7.324	-6.98693219867688\\
-12.207	-11.6452049903398\\
-15.869	-15.1386710896782\\
-17.09	-16.3034777820027\\
-19.531	-18.6321371890167\\
-21.973	-20.9617505736656\\
-20.752	-19.7969438813412\\
-15.869	-15.1386710896782\\
-14.648	-13.9738643973538\\
-18.311	-17.4682844743272\\
-19.531	-18.6321371890167\\
-13.428	-12.8100116826643\\
-9.766	-9.31654558332584\\
-13.428	-12.8100116826643\\
-9.766	-9.31654558332584\\
-7.324	-6.98693219867688\\
-9.766	-9.31654558332584\\
-10.986	-10.4803982980153\\
-7.324	-6.98693219867688\\
-9.766	-9.31654558332584\\
-6.104	-5.82307948398739\\
-10.986	-10.4803982980153\\
-13.428	-12.8100116826643\\
-8.545	-8.15173889100136\\
-9.766	-9.31654558332584\\
-12.207	-11.6452049903398\\
-18.311	-17.4682844743272\\
-15.869	-15.1386710896782\\
-9.766	-9.31654558332584\\
-8.545	-8.15173889100136\\
-9.766	-9.31654558332584\\
-7.324	-6.98693219867688\\
-8.545	-8.15173889100136\\
-7.324	-6.98693219867688\\
-8.545	-8.15173889100136\\
-10.986	-10.4803982980153\\
-6.104	-5.82307948398739\\
-10.986	-10.4803982980153\\
-13.428	-12.8100116826643\\
-9.766	-9.31654558332584\\
-12.207	-11.6452049903398\\
-13.428	-12.8100116826643\\
-10.986	-10.4803982980153\\
-4.883	-4.65827279166292\\
-10.986	-10.4803982980153\\
-8.545	-8.15173889100136\\
-6.104	-5.82307948398739\\
-10.986	-10.4803982980153\\
-6.104	-5.82307948398739\\
-8.545	-8.15173889100136\\
-4.883	-4.65827279166292\\
-8.545	-8.15173889100136\\
-4.883	-4.65827279166292\\
-7.324	-6.98693219867688\\
-4.883	-4.65827279166292\\
-10.986	-10.4803982980153\\
-12.207	-11.6452049903398\\
-15.869	-15.1386710896782\\
-10.986	-10.4803982980153\\
-6.104	-5.82307948398739\\
-7.324	-6.98693219867688\\
-4.883	-4.65827279166292\\
-7.324	-6.98693219867688\\
-6.104	-5.82307948398739\\
-3.662	-3.49346609933844\\
-8.545	-8.15173889100136\\
-9.766	-9.31654558332584\\
-17.09	-16.3034777820027\\
-10.986	-10.4803982980153\\
-13.428	-12.8100116826643\\
-7.324	-6.98693219867688\\
-8.545	-8.15173889100136\\
-3.662	-3.49346609933844\\
-1.221	-1.16480669232448\\
-6.104	-5.82307948398739\\
-9.766	-9.31654558332584\\
-8.545	-8.15173889100136\\
-9.766	-9.31654558332584\\
-17.09	-16.3034777820027\\
-10.986	-10.4803982980153\\
-13.428	-12.8100116826643\\
-15.869	-15.1386710896782\\
-10.986	-10.4803982980153\\
-6.104	-5.82307948398739\\
-3.662	-3.49346609933844\\
-4.883	-4.65827279166292\\
-6.104	-5.82307948398739\\
-4.883	-4.65827279166292\\
-6.104	-5.82307948398739\\
-9.766	-9.31654558332584\\
-8.545	-8.15173889100136\\
-9.766	-9.31654558332584\\
-6.104	-5.82307948398739\\
-3.662	-3.49346609933844\\
-7.324	-6.98693219867688\\
-10.986	-10.4803982980153\\
-8.545	-8.15173889100136\\
-12.207	-11.6452049903398\\
-8.545	-8.15173889100136\\
-7.324	-6.98693219867688\\
-8.545	-8.15173889100136\\
-12.207	-11.6452049903398\\
-14.648	-13.9738643973538\\
-10.986	-10.4803982980153\\
-17.09	-16.3034777820027\\
-9.766	-9.31654558332584\\
-12.207	-11.6452049903398\\
-13.428	-12.8100116826643\\
-9.766	-9.31654558332584\\
-12.207	-11.6452049903398\\
-18.311	-17.4682844743272\\
-15.869	-15.1386710896782\\
-8.545	-8.15173889100136\\
-12.207	-11.6452049903398\\
-9.766	-9.31654558332584\\
-12.207	-11.6452049903398\\
-13.428	-12.8100116826643\\
-9.766	-9.31654558332584\\
-14.648	-13.9738643973538\\
-13.428	-12.8100116826643\\
-12.207	-11.6452049903398\\
-8.545	-8.15173889100136\\
-9.766	-9.31654558332584\\
-15.869	-15.1386710896782\\
-14.648	-13.9738643973538\\
-15.869	-15.1386710896782\\
-12.207	-11.6452049903398\\
-10.986	-10.4803982980153\\
-9.766	-9.31654558332584\\
-8.545	-8.15173889100136\\
-4.883	-4.65827279166292\\
-3.662	-3.49346609933844\\
-7.324	-6.98693219867688\\
-10.986	-10.4803982980153\\
-9.766	-9.31654558332584\\
-8.545	-8.15173889100136\\
-10.986	-10.4803982980153\\
-9.766	-9.31654558332584\\
-8.545	-8.15173889100136\\
-3.662	-3.49346609933844\\
-6.104	-5.82307948398739\\
-8.545	-8.15173889100136\\
-10.986	-10.4803982980153\\
-7.324	-6.98693219867688\\
-6.104	-5.82307948398739\\
-7.324	-6.98693219867688\\
-6.104	-5.82307948398739\\
-8.545	-8.15173889100136\\
-15.869	-15.1386710896782\\
-14.648	-13.9738643973538\\
-17.09	-16.3034777820027\\
-13.428	-12.8100116826643\\
-10.986	-10.4803982980153\\
-9.766	-9.31654558332584\\
-10.986	-10.4803982980153\\
-12.207	-11.6452049903398\\
-14.648	-13.9738643973538\\
-17.09	-16.3034777820027\\
-19.531	-18.6321371890167\\
-21.973	-20.9617505736656\\
-18.311	-17.4682844743272\\
-21.973	-20.9617505736656\\
-19.531	-18.6321371890167\\
-14.648	-13.9738643973538\\
-17.09	-16.3034777820027\\
-14.648	-13.9738643973538\\
-10.986	-10.4803982980153\\
-8.545	-8.15173889100136\\
-10.986	-10.4803982980153\\
-7.324	-6.98693219867688\\
-3.662	-3.49346609933844\\
-7.324	-6.98693219867688\\
-6.104	-5.82307948398739\\
-8.545	-8.15173889100136\\
-12.207	-11.6452049903398\\
-8.545	-8.15173889100136\\
-12.207	-11.6452049903398\\
-18.311	-17.4682844743272\\
-17.09	-16.3034777820027\\
-19.531	-18.6321371890167\\
-17.09	-16.3034777820027\\
-14.648	-13.9738643973538\\
-7.324	-6.98693219867688\\
-9.766	-9.31654558332584\\
-8.545	-8.15173889100136\\
-6.104	-5.82307948398739\\
-7.324	-6.98693219867688\\
-8.545	-8.15173889100136\\
-2.441	-2.32865940701396\\
-4.883	-4.65827279166292\\
-8.545	-8.15173889100136\\
-9.766	-9.31654558332584\\
-12.207	-11.6452049903398\\
-10.986	-10.4803982980153\\
-12.207	-11.6452049903398\\
-14.648	-13.9738643973538\\
-10.986	-10.4803982980153\\
-9.766	-9.31654558332584\\
-7.324	-6.98693219867688\\
-10.986	-10.4803982980153\\
-7.324	-6.98693219867688\\
-10.986	-10.4803982980153\\
-13.428	-12.8100116826643\\
-10.986	-10.4803982980153\\
-8.545	-8.15173889100136\\
-6.104	-5.82307948398739\\
-7.324	-6.98693219867688\\
-6.104	-5.82307948398739\\
-9.766	-9.31654558332584\\
-8.545	-8.15173889100136\\
-10.986	-10.4803982980153\\
-8.545	-8.15173889100136\\
-6.104	-5.82307948398739\\
-10.986	-10.4803982980153\\
-14.648	-13.9738643973538\\
-12.207	-11.6452049903398\\
-10.986	-10.4803982980153\\
-9.766	-9.31654558332584\\
-8.545	-8.15173889100136\\
-9.766	-9.31654558332584\\
-12.207	-11.6452049903398\\
-7.324	-6.98693219867688\\
-10.986	-10.4803982980153\\
-13.428	-12.8100116826643\\
-14.648	-13.9738643973538\\
-20.752	-19.7969438813412\\
-21.973	-20.9617505736656\\
-14.648	-13.9738643973538\\
-9.766	-9.31654558332584\\
-7.324	-6.98693219867688\\
-6.104	-5.82307948398739\\
-7.324	-6.98693219867688\\
-4.883	-4.65827279166292\\
-7.324	-6.98693219867688\\
-10.986	-10.4803982980153\\
-12.207	-11.6452049903398\\
-13.428	-12.8100116826643\\
-18.311	-17.4682844743272\\
-17.09	-16.3034777820027\\
-12.207	-11.6452049903398\\
-9.766	-9.31654558332584\\
-13.428	-12.8100116826643\\
-9.766	-9.31654558332584\\
-8.545	-8.15173889100136\\
-10.986	-10.4803982980153\\
-6.104	-5.82307948398739\\
-10.986	-10.4803982980153\\
-14.648	-13.9738643973538\\
-8.545	-8.15173889100136\\
-15.869	-15.1386710896782\\
-13.428	-12.8100116826643\\
-10.986	-10.4803982980153\\
-17.09	-16.3034777820027\\
-10.986	-10.4803982980153\\
-8.545	-8.15173889100136\\
-9.766	-9.31654558332584\\
-12.207	-11.6452049903398\\
-18.311	-17.4682844743272\\
-19.531	-18.6321371890167\\
-18.311	-17.4682844743272\\
-15.869	-15.1386710896782\\
-9.766	-9.31654558332584\\
-7.324	-6.98693219867688\\
-6.104	-5.82307948398739\\
-7.324	-6.98693219867688\\
-9.766	-9.31654558332584\\
-8.545	-8.15173889100136\\
-6.104	-5.82307948398739\\
-12.207	-11.6452049903398\\
-10.986	-10.4803982980153\\
-13.428	-12.8100116826643\\
-17.09	-16.3034777820027\\
-19.531	-18.6321371890167\\
-15.869	-15.1386710896782\\
-12.207	-11.6452049903398\\
-15.869	-15.1386710896782\\
-12.207	-11.6452049903398\\
-9.766	-9.31654558332584\\
-10.986	-10.4803982980153\\
-15.869	-15.1386710896782\\
-18.311	-17.4682844743272\\
-12.207	-11.6452049903398\\
-8.545	-8.15173889100136\\
-4.883	-4.65827279166292\\
-2.441	-2.32865940701396\\
0	0\\
-6.104	-5.82307948398739\\
-8.545	-8.15173889100136\\
-7.324	-6.98693219867688\\
-6.104	-5.82307948398739\\
-7.324	-6.98693219867688\\
-4.883	-4.65827279166292\\
-6.104	-5.82307948398739\\
-8.545	-8.15173889100136\\
-10.986	-10.4803982980153\\
-13.428	-12.8100116826643\\
-10.986	-10.4803982980153\\
-12.207	-11.6452049903398\\
-17.09	-16.3034777820027\\
-19.531	-18.6321371890167\\
-20.752	-19.7969438813412\\
-13.428	-12.8100116826643\\
-10.986	-10.4803982980153\\
-14.648	-13.9738643973538\\
-17.09	-16.3034777820027\\
-13.428	-12.8100116826643\\
-9.766	-9.31654558332584\\
-7.324	-6.98693219867688\\
-3.662	-3.49346609933844\\
-4.883	-4.65827279166292\\
-6.104	-5.82307948398739\\
-1.221	-1.16480669232448\\
-9.766	-9.31654558332584\\
-7.324	-6.98693219867688\\
-6.104	-5.82307948398739\\
-4.883	-4.65827279166292\\
-3.662	-3.49346609933844\\
-6.104	-5.82307948398739\\
-4.883	-4.65827279166292\\
-3.662	-3.49346609933844\\
-4.883	-4.65827279166292\\
-12.207	-11.6452049903398\\
-13.428	-12.8100116826643\\
-15.869	-15.1386710896782\\
-14.648	-13.9738643973538\\
-23.193	-22.1256032883551\\
-20.752	-19.7969438813412\\
-18.311	-17.4682844743272\\
-15.869	-15.1386710896782\\
-14.648	-13.9738643973538\\
-10.986	-10.4803982980153\\
-8.545	-8.15173889100136\\
-12.207	-11.6452049903398\\
-14.648	-13.9738643973538\\
-8.545	-8.15173889100136\\
-9.766	-9.31654558332584\\
-10.986	-10.4803982980153\\
-12.207	-11.6452049903398\\
-10.986	-10.4803982980153\\
-9.766	-9.31654558332584\\
-10.986	-10.4803982980153\\
-2.441	-2.32865940701396\\
-4.883	-4.65827279166292\\
-8.545	-8.15173889100136\\
-7.324	-6.98693219867688\\
1.221	1.16480669232448\\
-2.441	-2.32865940701396\\
-7.324	-6.98693219867688\\
-8.545	-8.15173889100136\\
-10.986	-10.4803982980153\\
-7.324	-6.98693219867688\\
-8.545	-8.15173889100136\\
-10.986	-10.4803982980153\\
-4.883	-4.65827279166292\\
-3.662	-3.49346609933844\\
-2.441	-2.32865940701396\\
-3.662	-3.49346609933844\\
-10.986	-10.4803982980153\\
-12.207	-11.6452049903398\\
-8.545	-8.15173889100136\\
-7.324	-6.98693219867688\\
-6.104	-5.82307948398739\\
-8.545	-8.15173889100136\\
-6.104	-5.82307948398739\\
-3.662	-3.49346609933844\\
-4.883	-4.65827279166292\\
-7.324	-6.98693219867688\\
-3.662	-3.49346609933844\\
-4.883	-4.65827279166292\\
-6.104	-5.82307948398739\\
-9.766	-9.31654558332584\\
-13.428	-12.8100116826643\\
-6.104	-5.82307948398739\\
-7.324	-6.98693219867688\\
-9.766	-9.31654558332584\\
-8.545	-8.15173889100136\\
-12.207	-11.6452049903398\\
-7.324	-6.98693219867688\\
-10.986	-10.4803982980153\\
-7.324	-6.98693219867688\\
-9.766	-9.31654558332584\\
-6.104	-5.82307948398739\\
-7.324	-6.98693219867688\\
-8.545	-8.15173889100136\\
-7.324	-6.98693219867688\\
-8.545	-8.15173889100136\\
-10.986	-10.4803982980153\\
-18.311	-17.4682844743272\\
-13.428	-12.8100116826643\\
-10.986	-10.4803982980153\\
-9.766	-9.31654558332584\\
-13.428	-12.8100116826643\\
-9.766	-9.31654558332584\\
-10.986	-10.4803982980153\\
-14.648	-13.9738643973538\\
-4.883	-4.65827279166292\\
-3.662	-3.49346609933844\\
-10.986	-10.4803982980153\\
-7.324	-6.98693219867688\\
-9.766	-9.31654558332584\\
-10.986	-10.4803982980153\\
-12.207	-11.6452049903398\\
-10.986	-10.4803982980153\\
-9.766	-9.31654558332584\\
-8.545	-8.15173889100136\\
-12.207	-11.6452049903398\\
-8.545	-8.15173889100136\\
-10.986	-10.4803982980153\\
-7.324	-6.98693219867688\\
-8.545	-8.15173889100136\\
-4.883	-4.65827279166292\\
-6.104	-5.82307948398739\\
-10.986	-10.4803982980153\\
-13.428	-12.8100116826643\\
-10.986	-10.4803982980153\\
-13.428	-12.8100116826643\\
-9.766	-9.31654558332584\\
-7.324	-6.98693219867688\\
-9.766	-9.31654558332584\\
-12.207	-11.6452049903398\\
-13.428	-12.8100116826643\\
-12.207	-11.6452049903398\\
-7.324	-6.98693219867688\\
-8.545	-8.15173889100136\\
-19.531	-18.6321371890167\\
-12.207	-11.6452049903398\\
-13.428	-12.8100116826643\\
-15.869	-15.1386710896782\\
-10.986	-10.4803982980153\\
-9.766	-9.31654558332584\\
-8.545	-8.15173889100136\\
-13.428	-12.8100116826643\\
-10.986	-10.4803982980153\\
-9.766	-9.31654558332584\\
-10.986	-10.4803982980153\\
-8.545	-8.15173889100136\\
-9.766	-9.31654558332584\\
-4.883	-4.65827279166292\\
-9.766	-9.31654558332584\\
-12.207	-11.6452049903398\\
-9.766	-9.31654558332584\\
-7.324	-6.98693219867688\\
-3.662	-3.49346609933844\\
-2.441	-2.32865940701396\\
-3.662	-3.49346609933844\\
-9.766	-9.31654558332584\\
-13.428	-12.8100116826643\\
-10.986	-10.4803982980153\\
-13.428	-12.8100116826643\\
-12.207	-11.6452049903398\\
-7.324	-6.98693219867688\\
-6.104	-5.82307948398739\\
-3.662	-3.49346609933844\\
-8.545	-8.15173889100136\\
-4.883	-4.65827279166292\\
-8.545	-8.15173889100136\\
-7.324	-6.98693219867688\\
-6.104	-5.82307948398739\\
-4.883	-4.65827279166292\\
-9.766	-9.31654558332584\\
-10.986	-10.4803982980153\\
-6.104	-5.82307948398739\\
-10.986	-10.4803982980153\\
-8.545	-8.15173889100136\\
-9.766	-9.31654558332584\\
-14.648	-13.9738643973538\\
-8.545	-8.15173889100136\\
-12.207	-11.6452049903398\\
-3.662	-3.49346609933844\\
-12.207	-11.6452049903398\\
-13.428	-12.8100116826643\\
-9.766	-9.31654558332584\\
-15.869	-15.1386710896782\\
-13.428	-12.8100116826643\\
-10.986	-10.4803982980153\\
-15.869	-15.1386710896782\\
-13.428	-12.8100116826643\\
-12.207	-11.6452049903398\\
-10.986	-10.4803982980153\\
-8.545	-8.15173889100136\\
-9.766	-9.31654558332584\\
-8.545	-8.15173889100136\\
-6.104	-5.82307948398739\\
-8.545	-8.15173889100136\\
-7.324	-6.98693219867688\\
-10.986	-10.4803982980153\\
-15.869	-15.1386710896782\\
-13.428	-12.8100116826643\\
-14.648	-13.9738643973538\\
-10.986	-10.4803982980153\\
-7.324	-6.98693219867688\\
-4.883	-4.65827279166292\\
-6.104	-5.82307948398739\\
-9.766	-9.31654558332584\\
-12.207	-11.6452049903398\\
-8.545	-8.15173889100136\\
-9.766	-9.31654558332584\\
-17.09	-16.3034777820027\\
-14.648	-13.9738643973538\\
-13.428	-12.8100116826643\\
-17.09	-16.3034777820027\\
-23.193	-22.1256032883551\\
-7.324	-6.98693219867688\\
-8.545	-8.15173889100136\\
-12.207	-11.6452049903398\\
-17.09	-16.3034777820027\\
-14.648	-13.9738643973538\\
-10.986	-10.4803982980153\\
-9.766	-9.31654558332584\\
-6.104	-5.82307948398739\\
-8.545	-8.15173889100136\\
-6.104	-5.82307948398739\\
-3.662	-3.49346609933844\\
-6.104	-5.82307948398739\\
-3.662	-3.49346609933844\\
-4.883	-4.65827279166292\\
-12.207	-11.6452049903398\\
-7.324	-6.98693219867688\\
-9.766	-9.31654558332584\\
-7.324	-6.98693219867688\\
-9.766	-9.31654558332584\\
-10.986	-10.4803982980153\\
-13.428	-12.8100116826643\\
-10.986	-10.4803982980153\\
-13.428	-12.8100116826643\\
-12.207	-11.6452049903398\\
-13.428	-12.8100116826643\\
-9.766	-9.31654558332584\\
-14.648	-13.9738643973538\\
-13.428	-12.8100116826643\\
-7.324	-6.98693219867688\\
-9.766	-9.31654558332584\\
-10.986	-10.4803982980153\\
-8.545	-8.15173889100136\\
-13.428	-12.8100116826643\\
-14.648	-13.9738643973538\\
-13.428	-12.8100116826643\\
-20.752	-19.7969438813412\\
-10.986	-10.4803982980153\\
-12.207	-11.6452049903398\\
-10.986	-10.4803982980153\\
-15.869	-15.1386710896782\\
-7.324	-6.98693219867688\\
-10.986	-10.4803982980153\\
-15.869	-15.1386710896782\\
-4.883	-4.65827279166292\\
-7.324	-6.98693219867688\\
-9.766	-9.31654558332584\\
-10.986	-10.4803982980153\\
-6.104	-5.82307948398739\\
-4.883	-4.65827279166292\\
-6.104	-5.82307948398739\\
-7.324	-6.98693219867688\\
-9.766	-9.31654558332584\\
-13.428	-12.8100116826643\\
-10.986	-10.4803982980153\\
-14.648	-13.9738643973538\\
-13.428	-12.8100116826643\\
-12.207	-11.6452049903398\\
-8.545	-8.15173889100136\\
-10.986	-10.4803982980153\\
-9.766	-9.31654558332584\\
-15.869	-15.1386710896782\\
-13.428	-12.8100116826643\\
-12.207	-11.6452049903398\\
-7.324	-6.98693219867688\\
-12.207	-11.6452049903398\\
-14.648	-13.9738643973538\\
-10.986	-10.4803982980153\\
-13.428	-12.8100116826643\\
-9.766	-9.31654558332584\\
-4.883	-4.65827279166292\\
-6.104	-5.82307948398739\\
-9.766	-9.31654558332584\\
-6.104	-5.82307948398739\\
-8.545	-8.15173889100136\\
-14.648	-13.9738643973538\\
-20.752	-19.7969438813412\\
-12.207	-11.6452049903398\\
-14.648	-13.9738643973538\\
-10.986	-10.4803982980153\\
-9.766	-9.31654558332584\\
-10.986	-10.4803982980153\\
-12.207	-11.6452049903398\\
-14.648	-13.9738643973538\\
-18.311	-17.4682844743272\\
-19.531	-18.6321371890167\\
-14.648	-13.9738643973538\\
-19.531	-18.6321371890167\\
-15.869	-15.1386710896782\\
-14.648	-13.9738643973538\\
-12.207	-11.6452049903398\\
-10.986	-10.4803982980153\\
-13.428	-12.8100116826643\\
-18.311	-17.4682844743272\\
-7.324	-6.98693219867688\\
-8.545	-8.15173889100136\\
-12.207	-11.6452049903398\\
-9.766	-9.31654558332584\\
-8.545	-8.15173889100136\\
-3.662	-3.49346609933844\\
-7.324	-6.98693219867688\\
-13.428	-12.8100116826643\\
-20.752	-19.7969438813412\\
-21.973	-20.9617505736656\\
-19.531	-18.6321371890167\\
-14.648	-13.9738643973538\\
-17.09	-16.3034777820027\\
-21.973	-20.9617505736656\\
-17.09	-16.3034777820027\\
-14.648	-13.9738643973538\\
-18.311	-17.4682844743272\\
-13.428	-12.8100116826643\\
-15.869	-15.1386710896782\\
-12.207	-11.6452049903398\\
-7.324	-6.98693219867688\\
-10.986	-10.4803982980153\\
-6.104	-5.82307948398739\\
-9.766	-9.31654558332584\\
-7.324	-6.98693219867688\\
-13.428	-12.8100116826643\\
-9.766	-9.31654558332584\\
-8.545	-8.15173889100136\\
-9.766	-9.31654558332584\\
-12.207	-11.6452049903398\\
-10.986	-10.4803982980153\\
-7.324	-6.98693219867688\\
-13.428	-12.8100116826643\\
-14.648	-13.9738643973538\\
-9.766	-9.31654558332584\\
-13.428	-12.8100116826643\\
-12.207	-11.6452049903398\\
-9.766	-9.31654558332584\\
-6.104	-5.82307948398739\\
-4.883	-4.65827279166292\\
-6.104	-5.82307948398739\\
-9.766	-9.31654558332584\\
-4.883	-4.65827279166292\\
-6.104	-5.82307948398739\\
-4.883	-4.65827279166292\\
-12.207	-11.6452049903398\\
-10.986	-10.4803982980153\\
-12.207	-11.6452049903398\\
-14.648	-13.9738643973538\\
-13.428	-12.8100116826643\\
-14.648	-13.9738643973538\\
-10.986	-10.4803982980153\\
-4.883	-4.65827279166292\\
-8.545	-8.15173889100136\\
-7.324	-6.98693219867688\\
-10.986	-10.4803982980153\\
-15.869	-15.1386710896782\\
-9.766	-9.31654558332584\\
-15.869	-15.1386710896782\\
-17.09	-16.3034777820027\\
-6.104	-5.82307948398739\\
-7.324	-6.98693219867688\\
-9.766	-9.31654558332584\\
-10.986	-10.4803982980153\\
-14.648	-13.9738643973538\\
-4.883	-4.65827279166292\\
-10.986	-10.4803982980153\\
-9.766	-9.31654558332584\\
-10.986	-10.4803982980153\\
-13.428	-12.8100116826643\\
-18.311	-17.4682844743272\\
-15.869	-15.1386710896782\\
-14.648	-13.9738643973538\\
-13.428	-12.8100116826643\\
-8.545	-8.15173889100136\\
-9.766	-9.31654558332584\\
0	0\\
-7.324	-6.98693219867688\\
-10.986	-10.4803982980153\\
-7.324	-6.98693219867688\\
-8.545	-8.15173889100136\\
-17.09	-16.3034777820027\\
-15.869	-15.1386710896782\\
-18.311	-17.4682844743272\\
-10.986	-10.4803982980153\\
-8.545	-8.15173889100136\\
-15.869	-15.1386710896782\\
-17.09	-16.3034777820027\\
-14.648	-13.9738643973538\\
-9.766	-9.31654558332584\\
-10.986	-10.4803982980153\\
-15.869	-15.1386710896782\\
-13.428	-12.8100116826643\\
-7.324	-6.98693219867688\\
-4.883	-4.65827279166292\\
-6.104	-5.82307948398739\\
-10.986	-10.4803982980153\\
-9.766	-9.31654558332584\\
-6.104	-5.82307948398739\\
-7.324	-6.98693219867688\\
-6.104	-5.82307948398739\\
-1.221	-1.16480669232448\\
-7.324	-6.98693219867688\\
-10.986	-10.4803982980153\\
-13.428	-12.8100116826643\\
-18.311	-17.4682844743272\\
-20.752	-19.7969438813412\\
-18.311	-17.4682844743272\\
-17.09	-16.3034777820027\\
-12.207	-11.6452049903398\\
-10.986	-10.4803982980153\\
-12.207	-11.6452049903398\\
-8.545	-8.15173889100136\\
-7.324	-6.98693219867688\\
-6.104	-5.82307948398739\\
-9.766	-9.31654558332584\\
-10.986	-10.4803982980153\\
-7.324	-6.98693219867688\\
-3.662	-3.49346609933844\\
-6.104	-5.82307948398739\\
-7.324	-6.98693219867688\\
-4.883	-4.65827279166292\\
-8.545	-8.15173889100136\\
-14.648	-13.9738643973538\\
-10.986	-10.4803982980153\\
-14.648	-13.9738643973538\\
-19.531	-18.6321371890167\\
-21.973	-20.9617505736656\\
-14.648	-13.9738643973538\\
-10.986	-10.4803982980153\\
-8.545	-8.15173889100136\\
-7.324	-6.98693219867688\\
-9.766	-9.31654558332584\\
-8.545	-8.15173889100136\\
-9.766	-9.31654558332584\\
-7.324	-6.98693219867688\\
-9.766	-9.31654558332584\\
-8.545	-8.15173889100136\\
-7.324	-6.98693219867688\\
-9.766	-9.31654558332584\\
-4.883	-4.65827279166292\\
-2.441	-2.32865940701396\\
-1.221	-1.16480669232448\\
-3.662	-3.49346609933844\\
-6.104	-5.82307948398739\\
-2.441	-2.32865940701396\\
-7.324	-6.98693219867688\\
-6.104	-5.82307948398739\\
-9.766	-9.31654558332584\\
-10.986	-10.4803982980153\\
-8.545	-8.15173889100136\\
-7.324	-6.98693219867688\\
-13.428	-12.8100116826643\\
-19.531	-18.6321371890167\\
-13.428	-12.8100116826643\\
-10.986	-10.4803982980153\\
-9.766	-9.31654558332584\\
-3.662	-3.49346609933844\\
-7.324	-6.98693219867688\\
-8.545	-8.15173889100136\\
-9.766	-9.31654558332584\\
-13.428	-12.8100116826643\\
-10.986	-10.4803982980153\\
-9.766	-9.31654558332584\\
-8.545	-8.15173889100136\\
-7.324	-6.98693219867688\\
-4.883	-4.65827279166292\\
-3.662	-3.49346609933844\\
-6.104	-5.82307948398739\\
-7.324	-6.98693219867688\\
-10.986	-10.4803982980153\\
-9.766	-9.31654558332584\\
-4.883	-4.65827279166292\\
-9.766	-9.31654558332584\\
-12.207	-11.6452049903398\\
-6.104	-5.82307948398739\\
-13.428	-12.8100116826643\\
-12.207	-11.6452049903398\\
-7.324	-6.98693219867688\\
-8.545	-8.15173889100136\\
-12.207	-11.6452049903398\\
-9.766	-9.31654558332584\\
-6.104	-5.82307948398739\\
-9.766	-9.31654558332584\\
-4.883	-4.65827279166292\\
-3.662	-3.49346609933844\\
-2.441	-2.32865940701396\\
-4.883	-4.65827279166292\\
-7.324	-6.98693219867688\\
-8.545	-8.15173889100136\\
-4.883	-4.65827279166292\\
-3.662	-3.49346609933844\\
-2.441	-2.32865940701396\\
-4.883	-4.65827279166292\\
-8.545	-8.15173889100136\\
-10.986	-10.4803982980153\\
-6.104	-5.82307948398739\\
-8.545	-8.15173889100136\\
-10.986	-10.4803982980153\\
-13.428	-12.8100116826643\\
-10.986	-10.4803982980153\\
-12.207	-11.6452049903398\\
-13.428	-12.8100116826643\\
-14.648	-13.9738643973538\\
-13.428	-12.8100116826643\\
-18.311	-17.4682844743272\\
-15.869	-15.1386710896782\\
-12.207	-11.6452049903398\\
-13.428	-12.8100116826643\\
-9.766	-9.31654558332584\\
-7.324	-6.98693219867688\\
-10.986	-10.4803982980153\\
-13.428	-12.8100116826643\\
-12.207	-11.6452049903398\\
-10.986	-10.4803982980153\\
-8.545	-8.15173889100136\\
-12.207	-11.6452049903398\\
-8.545	-8.15173889100136\\
-14.648	-13.9738643973538\\
-15.869	-15.1386710896782\\
-10.986	-10.4803982980153\\
-6.104	-5.82307948398739\\
-9.766	-9.31654558332584\\
-2.441	-2.32865940701396\\
-3.662	-3.49346609933844\\
-7.324	-6.98693219867688\\
-4.883	-4.65827279166292\\
-3.662	-3.49346609933844\\
-8.545	-8.15173889100136\\
-4.883	-4.65827279166292\\
-7.324	-6.98693219867688\\
-6.104	-5.82307948398739\\
-4.883	-4.65827279166292\\
-6.104	-5.82307948398739\\
-7.324	-6.98693219867688\\
-6.104	-5.82307948398739\\
-7.324	-6.98693219867688\\
-14.648	-13.9738643973538\\
-9.766	-9.31654558332584\\
-8.545	-8.15173889100136\\
-10.986	-10.4803982980153\\
-7.324	-6.98693219867688\\
-6.104	-5.82307948398739\\
-10.986	-10.4803982980153\\
-8.545	-8.15173889100136\\
-4.883	-4.65827279166292\\
-10.986	-10.4803982980153\\
-8.545	-8.15173889100136\\
-7.324	-6.98693219867688\\
-2.441	-2.32865940701396\\
-3.662	-3.49346609933844\\
-4.883	-4.65827279166292\\
-7.324	-6.98693219867688\\
-8.545	-8.15173889100136\\
-9.766	-9.31654558332584\\
-7.324	-6.98693219867688\\
-10.986	-10.4803982980153\\
-14.648	-13.9738643973538\\
-12.207	-11.6452049903398\\
-10.986	-10.4803982980153\\
-14.648	-13.9738643973538\\
-12.207	-11.6452049903398\\
-17.09	-16.3034777820027\\
-15.869	-15.1386710896782\\
-19.531	-18.6321371890167\\
-20.752	-19.7969438813412\\
-13.428	-12.8100116826643\\
-6.104	-5.82307948398739\\
-8.545	-8.15173889100136\\
-12.207	-11.6452049903398\\
-8.545	-8.15173889100136\\
-7.324	-6.98693219867688\\
-10.986	-10.4803982980153\\
-13.428	-12.8100116826643\\
-14.648	-13.9738643973538\\
-13.428	-12.8100116826643\\
-9.766	-9.31654558332584\\
-6.104	-5.82307948398739\\
-10.986	-10.4803982980153\\
-12.207	-11.6452049903398\\
-14.648	-13.9738643973538\\
-12.207	-11.6452049903398\\
-19.531	-18.6321371890167\\
-17.09	-16.3034777820027\\
-20.752	-19.7969438813412\\
-18.311	-17.4682844743272\\
-15.869	-15.1386710896782\\
-17.09	-16.3034777820027\\
-15.869	-15.1386710896782\\
-10.986	-10.4803982980153\\
-8.545	-8.15173889100136\\
-9.766	-9.31654558332584\\
-12.207	-11.6452049903398\\
-8.545	-8.15173889100136\\
-13.428	-12.8100116826643\\
-12.207	-11.6452049903398\\
-8.545	-8.15173889100136\\
-9.766	-9.31654558332584\\
-12.207	-11.6452049903398\\
-13.428	-12.8100116826643\\
-14.648	-13.9738643973538\\
-10.986	-10.4803982980153\\
-9.766	-9.31654558332584\\
-8.545	-8.15173889100136\\
-12.207	-11.6452049903398\\
-8.545	-8.15173889100136\\
-7.324	-6.98693219867688\\
-2.441	-2.32865940701396\\
-7.324	-6.98693219867688\\
-9.766	-9.31654558332584\\
-12.207	-11.6452049903398\\
-8.545	-8.15173889100136\\
-6.104	-5.82307948398739\\
-7.324	-6.98693219867688\\
-8.545	-8.15173889100136\\
-9.766	-9.31654558332584\\
-6.104	-5.82307948398739\\
-13.428	-12.8100116826643\\
-14.648	-13.9738643973538\\
-15.869	-15.1386710896782\\
-17.09	-16.3034777820027\\
-12.207	-11.6452049903398\\
-7.324	-6.98693219867688\\
-8.545	-8.15173889100136\\
-12.207	-11.6452049903398\\
-10.986	-10.4803982980153\\
-6.104	-5.82307948398739\\
-8.545	-8.15173889100136\\
-12.207	-11.6452049903398\\
-10.986	-10.4803982980153\\
-18.311	-17.4682844743272\\
-12.207	-11.6452049903398\\
-8.545	-8.15173889100136\\
-7.324	-6.98693219867688\\
-8.545	-8.15173889100136\\
-13.428	-12.8100116826643\\
-12.207	-11.6452049903398\\
-13.428	-12.8100116826643\\
-14.648	-13.9738643973538\\
-12.207	-11.6452049903398\\
-10.986	-10.4803982980153\\
-13.428	-12.8100116826643\\
-9.766	-9.31654558332584\\
-12.207	-11.6452049903398\\
-7.324	-6.98693219867688\\
-10.986	-10.4803982980153\\
-13.428	-12.8100116826643\\
-14.648	-13.9738643973538\\
-8.545	-8.15173889100136\\
-14.648	-13.9738643973538\\
-13.428	-12.8100116826643\\
-12.207	-11.6452049903398\\
-7.324	-6.98693219867688\\
-17.09	-16.3034777820027\\
-8.545	-8.15173889100136\\
-10.986	-10.4803982980153\\
-12.207	-11.6452049903398\\
-8.545	-8.15173889100136\\
-6.104	-5.82307948398739\\
-17.09	-16.3034777820027\\
-20.752	-19.7969438813412\\
-12.207	-11.6452049903398\\
-15.869	-15.1386710896782\\
-13.428	-12.8100116826643\\
-12.207	-11.6452049903398\\
-10.986	-10.4803982980153\\
-8.545	-8.15173889100136\\
-7.324	-6.98693219867688\\
-10.986	-10.4803982980153\\
-17.09	-16.3034777820027\\
-20.752	-19.7969438813412\\
-15.869	-15.1386710896782\\
-14.648	-13.9738643973538\\
-10.986	-10.4803982980153\\
-13.428	-12.8100116826643\\
-9.766	-9.31654558332584\\
-8.545	-8.15173889100136\\
-7.324	-6.98693219867688\\
-3.662	-3.49346609933844\\
-8.545	-8.15173889100136\\
-10.986	-10.4803982980153\\
-6.104	-5.82307948398739\\
-14.648	-13.9738643973538\\
-6.104	-5.82307948398739\\
-3.662	-3.49346609933844\\
-7.324	-6.98693219867688\\
-8.545	-8.15173889100136\\
-9.766	-9.31654558332584\\
-8.545	-8.15173889100136\\
-7.324	-6.98693219867688\\
-6.104	-5.82307948398739\\
-4.883	-4.65827279166292\\
-10.986	-10.4803982980153\\
-8.545	-8.15173889100136\\
-6.104	-5.82307948398739\\
-4.883	-4.65827279166292\\
-8.545	-8.15173889100136\\
-10.986	-10.4803982980153\\
-9.766	-9.31654558332584\\
-6.104	-5.82307948398739\\
-7.324	-6.98693219867688\\
-10.986	-10.4803982980153\\
-14.648	-13.9738643973538\\
-15.869	-15.1386710896782\\
-13.428	-12.8100116826643\\
-7.324	-6.98693219867688\\
-12.207	-11.6452049903398\\
-9.766	-9.31654558332584\\
-6.104	-5.82307948398739\\
-9.766	-9.31654558332584\\
-8.545	-8.15173889100136\\
-6.104	-5.82307948398739\\
-4.883	-4.65827279166292\\
-7.324	-6.98693219867688\\
-9.766	-9.31654558332584\\
-13.428	-12.8100116826643\\
-12.207	-11.6452049903398\\
-13.428	-12.8100116826643\\
-17.09	-16.3034777820027\\
-21.973	-20.9617505736656\\
-23.193	-22.1256032883551\\
-13.428	-12.8100116826643\\
-17.09	-16.3034777820027\\
-15.869	-15.1386710896782\\
-20.752	-19.7969438813412\\
-14.648	-13.9738643973538\\
-7.324	-6.98693219867688\\
-6.104	-5.82307948398739\\
-7.324	-6.98693219867688\\
-6.104	-5.82307948398739\\
-3.662	-3.49346609933844\\
-2.441	-2.32865940701396\\
-4.883	-4.65827279166292\\
-8.545	-8.15173889100136\\
-9.766	-9.31654558332584\\
-4.883	-4.65827279166292\\
-7.324	-6.98693219867688\\
-6.104	-5.82307948398739\\
-8.545	-8.15173889100136\\
-7.324	-6.98693219867688\\
-4.883	-4.65827279166292\\
-3.662	-3.49346609933844\\
-6.104	-5.82307948398739\\
-7.324	-6.98693219867688\\
-3.662	-3.49346609933844\\
-9.766	-9.31654558332584\\
-7.324	-6.98693219867688\\
-9.766	-9.31654558332584\\
-7.324	-6.98693219867688\\
-8.545	-8.15173889100136\\
-12.207	-11.6452049903398\\
-9.766	-9.31654558332584\\
-12.207	-11.6452049903398\\
-10.986	-10.4803982980153\\
-7.324	-6.98693219867688\\
-6.104	-5.82307948398739\\
-13.428	-12.8100116826643\\
-14.648	-13.9738643973538\\
-10.986	-10.4803982980153\\
-12.207	-11.6452049903398\\
-8.545	-8.15173889100136\\
-17.09	-16.3034777820027\\
-13.428	-12.8100116826643\\
-14.648	-13.9738643973538\\
-23.193	-22.1256032883551\\
-15.869	-15.1386710896782\\
-8.545	-8.15173889100136\\
-7.324	-6.98693219867688\\
-6.104	-5.82307948398739\\
-9.766	-9.31654558332584\\
-13.428	-12.8100116826643\\
-15.869	-15.1386710896782\\
-14.648	-13.9738643973538\\
-15.869	-15.1386710896782\\
-9.766	-9.31654558332584\\
-10.986	-10.4803982980153\\
-12.207	-11.6452049903398\\
-6.104	-5.82307948398739\\
-4.883	-4.65827279166292\\
-7.324	-6.98693219867688\\
-8.545	-8.15173889100136\\
-10.986	-10.4803982980153\\
-13.428	-12.8100116826643\\
-10.986	-10.4803982980153\\
-8.545	-8.15173889100136\\
-7.324	-6.98693219867688\\
-10.986	-10.4803982980153\\
-14.648	-13.9738643973538\\
-15.869	-15.1386710896782\\
-17.09	-16.3034777820027\\
-20.752	-19.7969438813412\\
-14.648	-13.9738643973538\\
-18.311	-17.4682844743272\\
-12.207	-11.6452049903398\\
-8.545	-8.15173889100136\\
-14.648	-13.9738643973538\\
-15.869	-15.1386710896782\\
-10.986	-10.4803982980153\\
-15.869	-15.1386710896782\\
-19.531	-18.6321371890167\\
-17.09	-16.3034777820027\\
-7.324	-6.98693219867688\\
-6.104	-5.82307948398739\\
-12.207	-11.6452049903398\\
-6.104	-5.82307948398739\\
-8.545	-8.15173889100136\\
-6.104	-5.82307948398739\\
-9.766	-9.31654558332584\\
-3.662	-3.49346609933844\\
-4.883	-4.65827279166292\\
-7.324	-6.98693219867688\\
-9.766	-9.31654558332584\\
-10.986	-10.4803982980153\\
-12.207	-11.6452049903398\\
-13.428	-12.8100116826643\\
-15.869	-15.1386710896782\\
-13.428	-12.8100116826643\\
-8.545	-8.15173889100136\\
-7.324	-6.98693219867688\\
-10.986	-10.4803982980153\\
-8.545	-8.15173889100136\\
-7.324	-6.98693219867688\\
-4.883	-4.65827279166292\\
-6.104	-5.82307948398739\\
-4.883	-4.65827279166292\\
-3.662	-3.49346609933844\\
-4.883	-4.65827279166292\\
-8.545	-8.15173889100136\\
-7.324	-6.98693219867688\\
-6.104	-5.82307948398739\\
-4.883	-4.65827279166292\\
-9.766	-9.31654558332584\\
-7.324	-6.98693219867688\\
-9.766	-9.31654558332584\\
-8.545	-8.15173889100136\\
-10.986	-10.4803982980153\\
-9.766	-9.31654558332584\\
-10.986	-10.4803982980153\\
-9.766	-9.31654558332584\\
-7.324	-6.98693219867688\\
-8.545	-8.15173889100136\\
-9.766	-9.31654558332584\\
-14.648	-13.9738643973538\\
-9.766	-9.31654558332584\\
-10.986	-10.4803982980153\\
-15.869	-15.1386710896782\\
-10.986	-10.4803982980153\\
-9.766	-9.31654558332584\\
-8.545	-8.15173889100136\\
-7.324	-6.98693219867688\\
-4.883	-4.65827279166292\\
-3.662	-3.49346609933844\\
-6.104	-5.82307948398739\\
-10.986	-10.4803982980153\\
-7.324	-6.98693219867688\\
-9.766	-9.31654558332584\\
-7.324	-6.98693219867688\\
-8.545	-8.15173889100136\\
-2.441	-2.32865940701396\\
-4.883	-4.65827279166292\\
-2.441	-2.32865940701396\\
-8.545	-8.15173889100136\\
-10.986	-10.4803982980153\\
-7.324	-6.98693219867688\\
-9.766	-9.31654558332584\\
-12.207	-11.6452049903398\\
-13.428	-12.8100116826643\\
-12.207	-11.6452049903398\\
-8.545	-8.15173889100136\\
-7.324	-6.98693219867688\\
-9.766	-9.31654558332584\\
-8.545	-8.15173889100136\\
-12.207	-11.6452049903398\\
-10.986	-10.4803982980153\\
-12.207	-11.6452049903398\\
-15.869	-15.1386710896782\\
-8.545	-8.15173889100136\\
-13.428	-12.8100116826643\\
-15.869	-15.1386710896782\\
-14.648	-13.9738643973538\\
-21.973	-20.9617505736656\\
-25.635	-24.4552166730041\\
-15.869	-15.1386710896782\\
-18.311	-17.4682844743272\\
-21.973	-20.9617505736656\\
-20.752	-19.7969438813412\\
-24.414	-23.2904099806796\\
-23.193	-22.1256032883551\\
-25.635	-24.4552166730041\\
-20.752	-19.7969438813412\\
-14.648	-13.9738643973538\\
-10.986	-10.4803982980153\\
-13.428	-12.8100116826643\\
-9.766	-9.31654558332584\\
-8.545	-8.15173889100136\\
-9.766	-9.31654558332584\\
-13.428	-12.8100116826643\\
-10.986	-10.4803982980153\\
-8.545	-8.15173889100136\\
-6.104	-5.82307948398739\\
-10.986	-10.4803982980153\\
-7.324	-6.98693219867688\\
-6.104	-5.82307948398739\\
-7.324	-6.98693219867688\\
-3.662	-3.49346609933844\\
-6.104	-5.82307948398739\\
-9.766	-9.31654558332584\\
-10.986	-10.4803982980153\\
-8.545	-8.15173889100136\\
-12.207	-11.6452049903398\\
-9.766	-9.31654558332584\\
-4.883	-4.65827279166292\\
-7.324	-6.98693219867688\\
-4.883	-4.65827279166292\\
-6.104	-5.82307948398739\\
-8.545	-8.15173889100136\\
-4.883	-4.65827279166292\\
-9.766	-9.31654558332584\\
-14.648	-13.9738643973538\\
-15.869	-15.1386710896782\\
-10.986	-10.4803982980153\\
-7.324	-6.98693219867688\\
-8.545	-8.15173889100136\\
-12.207	-11.6452049903398\\
-13.428	-12.8100116826643\\
-7.324	-6.98693219867688\\
-6.104	-5.82307948398739\\
-8.545	-8.15173889100136\\
-2.441	-2.32865940701396\\
-7.324	-6.98693219867688\\
-10.986	-10.4803982980153\\
-7.324	-6.98693219867688\\
-6.104	-5.82307948398739\\
-3.662	-3.49346609933844\\
-7.324	-6.98693219867688\\
-1.221	-1.16480669232448\\
-3.662	-3.49346609933844\\
-7.324	-6.98693219867688\\
-10.986	-10.4803982980153\\
-12.207	-11.6452049903398\\
-10.986	-10.4803982980153\\
-8.545	-8.15173889100136\\
-10.986	-10.4803982980153\\
-17.09	-16.3034777820027\\
-18.311	-17.4682844743272\\
-8.545	-8.15173889100136\\
-4.883	-4.65827279166292\\
-14.648	-13.9738643973538\\
-10.986	-10.4803982980153\\
-8.545	-8.15173889100136\\
-9.766	-9.31654558332584\\
-10.986	-10.4803982980153\\
-17.09	-16.3034777820027\\
-18.311	-17.4682844743272\\
-19.531	-18.6321371890167\\
-14.648	-13.9738643973538\\
-13.428	-12.8100116826643\\
-9.766	-9.31654558332584\\
-8.545	-8.15173889100136\\
-10.986	-10.4803982980153\\
-9.766	-9.31654558332584\\
-12.207	-11.6452049903398\\
-8.545	-8.15173889100136\\
-9.766	-9.31654558332584\\
-8.545	-8.15173889100136\\
-6.104	-5.82307948398739\\
-4.883	-4.65827279166292\\
-2.441	-2.32865940701396\\
-4.883	-4.65827279166292\\
-7.324	-6.98693219867688\\
-6.104	-5.82307948398739\\
-8.545	-8.15173889100136\\
-12.207	-11.6452049903398\\
-8.545	-8.15173889100136\\
-4.883	-4.65827279166292\\
-1.221	-1.16480669232448\\
-7.324	-6.98693219867688\\
-13.428	-12.8100116826643\\
-14.648	-13.9738643973538\\
-20.752	-19.7969438813412\\
-24.414	-23.2904099806796\\
-15.869	-15.1386710896782\\
-17.09	-16.3034777820027\\
-21.973	-20.9617505736656\\
-18.311	-17.4682844743272\\
-15.869	-15.1386710896782\\
-18.311	-17.4682844743272\\
-20.752	-19.7969438813412\\
-19.531	-18.6321371890167\\
-26.855	-25.6190693876936\\
-20.752	-19.7969438813412\\
-12.207	-11.6452049903398\\
-7.324	-6.98693219867688\\
-6.104	-5.82307948398739\\
-10.986	-10.4803982980153\\
-12.207	-11.6452049903398\\
-8.545	-8.15173889100136\\
-10.986	-10.4803982980153\\
-6.104	-5.82307948398739\\
-3.662	-3.49346609933844\\
-9.766	-9.31654558332584\\
-12.207	-11.6452049903398\\
-9.766	-9.31654558332584\\
-6.104	-5.82307948398739\\
-7.324	-6.98693219867688\\
-10.986	-10.4803982980153\\
-17.09	-16.3034777820027\\
-15.869	-15.1386710896782\\
-17.09	-16.3034777820027\\
-20.752	-19.7969438813412\\
-25.635	-24.4552166730041\\
-21.973	-20.9617505736656\\
-17.09	-16.3034777820027\\
-10.986	-10.4803982980153\\
-8.545	-8.15173889100136\\
-4.883	-4.65827279166292\\
-7.324	-6.98693219867688\\
-9.766	-9.31654558332584\\
-8.545	-8.15173889100136\\
-6.104	-5.82307948398739\\
-7.324	-6.98693219867688\\
-8.545	-8.15173889100136\\
-10.986	-10.4803982980153\\
-9.766	-9.31654558332584\\
-4.883	-4.65827279166292\\
-2.441	-2.32865940701396\\
-3.662	-3.49346609933844\\
-7.324	-6.98693219867688\\
-8.545	-8.15173889100136\\
-7.324	-6.98693219867688\\
-10.986	-10.4803982980153\\
-8.545	-8.15173889100136\\
-19.531	-18.6321371890167\\
-14.648	-13.9738643973538\\
-12.207	-11.6452049903398\\
-17.09	-16.3034777820027\\
-18.311	-17.4682844743272\\
-13.428	-12.8100116826643\\
-12.207	-11.6452049903398\\
-10.986	-10.4803982980153\\
-9.766	-9.31654558332584\\
-14.648	-13.9738643973538\\
-24.414	-23.2904099806796\\
-25.635	-24.4552166730041\\
-20.752	-19.7969438813412\\
-19.531	-18.6321371890167\\
-14.648	-13.9738643973538\\
-17.09	-16.3034777820027\\
-19.531	-18.6321371890167\\
-14.648	-13.9738643973538\\
-10.986	-10.4803982980153\\
-9.766	-9.31654558332584\\
-12.207	-11.6452049903398\\
-13.428	-12.8100116826643\\
-8.545	-8.15173889100136\\
-7.324	-6.98693219867688\\
-8.545	-8.15173889100136\\
-14.648	-13.9738643973538\\
-17.09	-16.3034777820027\\
-12.207	-11.6452049903398\\
-10.986	-10.4803982980153\\
-9.766	-9.31654558332584\\
-6.104	-5.82307948398739\\
-4.883	-4.65827279166292\\
-6.104	-5.82307948398739\\
-4.883	-4.65827279166292\\
-7.324	-6.98693219867688\\
-9.766	-9.31654558332584\\
-10.986	-10.4803982980153\\
-8.545	-8.15173889100136\\
-3.662	-3.49346609933844\\
-6.104	-5.82307948398739\\
-8.545	-8.15173889100136\\
-9.766	-9.31654558332584\\
-8.545	-8.15173889100136\\
-4.883	-4.65827279166292\\
-10.986	-10.4803982980153\\
-14.648	-13.9738643973538\\
-10.986	-10.4803982980153\\
-7.324	-6.98693219867688\\
-9.766	-9.31654558332584\\
-8.545	-8.15173889100136\\
-6.104	-5.82307948398739\\
-4.883	-4.65827279166292\\
-10.986	-10.4803982980153\\
-9.766	-9.31654558332584\\
-3.662	-3.49346609933844\\
-6.104	-5.82307948398739\\
-9.766	-9.31654558332584\\
-12.207	-11.6452049903398\\
-7.324	-6.98693219867688\\
-6.104	-5.82307948398739\\
-10.986	-10.4803982980153\\
-13.428	-12.8100116826643\\
-12.207	-11.6452049903398\\
-17.09	-16.3034777820027\\
-19.531	-18.6321371890167\\
-12.207	-11.6452049903398\\
-17.09	-16.3034777820027\\
-13.428	-12.8100116826643\\
-10.986	-10.4803982980153\\
-18.311	-17.4682844743272\\
-14.648	-13.9738643973538\\
-7.324	-6.98693219867688\\
-8.545	-8.15173889100136\\
-19.531	-18.6321371890167\\
-21.973	-20.9617505736656\\
-17.09	-16.3034777820027\\
-12.207	-11.6452049903398\\
-17.09	-16.3034777820027\\
-14.648	-13.9738643973538\\
-9.766	-9.31654558332584\\
-8.545	-8.15173889100136\\
-12.207	-11.6452049903398\\
-9.766	-9.31654558332584\\
-12.207	-11.6452049903398\\
-10.986	-10.4803982980153\\
-6.104	-5.82307948398739\\
-4.883	-4.65827279166292\\
-6.104	-5.82307948398739\\
-10.986	-10.4803982980153\\
-8.545	-8.15173889100136\\
-13.428	-12.8100116826643\\
-15.869	-15.1386710896782\\
-13.428	-12.8100116826643\\
-9.766	-9.31654558332584\\
-6.104	-5.82307948398739\\
-14.648	-13.9738643973538\\
-10.986	-10.4803982980153\\
-7.324	-6.98693219867688\\
-13.428	-12.8100116826643\\
-10.986	-10.4803982980153\\
-13.428	-12.8100116826643\\
-15.869	-15.1386710896782\\
-20.752	-19.7969438813412\\
-17.09	-16.3034777820027\\
-14.648	-13.9738643973538\\
-15.869	-15.1386710896782\\
-14.648	-13.9738643973538\\
-7.324	-6.98693219867688\\
-12.207	-11.6452049903398\\
-13.428	-12.8100116826643\\
-15.869	-15.1386710896782\\
-12.207	-11.6452049903398\\
-8.545	-8.15173889100136\\
-6.104	-5.82307948398739\\
-10.986	-10.4803982980153\\
-13.428	-12.8100116826643\\
-7.324	-6.98693219867688\\
-10.986	-10.4803982980153\\
-9.766	-9.31654558332584\\
-3.662	-3.49346609933844\\
-6.104	-5.82307948398739\\
-9.766	-9.31654558332584\\
-6.104	-5.82307948398739\\
-4.883	-4.65827279166292\\
-2.441	-2.32865940701396\\
-8.545	-8.15173889100136\\
-9.766	-9.31654558332584\\
-13.428	-12.8100116826643\\
-17.09	-16.3034777820027\\
-13.428	-12.8100116826643\\
-6.104	-5.82307948398739\\
-12.207	-11.6452049903398\\
-20.752	-19.7969438813412\\
-15.869	-15.1386710896782\\
-14.648	-13.9738643973538\\
-19.531	-18.6321371890167\\
-17.09	-16.3034777820027\\
-13.428	-12.8100116826643\\
-12.207	-11.6452049903398\\
-15.869	-15.1386710896782\\
-14.648	-13.9738643973538\\
-10.986	-10.4803982980153\\
-14.648	-13.9738643973538\\
-20.752	-19.7969438813412\\
-18.311	-17.4682844743272\\
-9.766	-9.31654558332584\\
-7.324	-6.98693219867688\\
-6.104	-5.82307948398739\\
-14.648	-13.9738643973538\\
-2.441	-2.32865940701396\\
-3.662	-3.49346609933844\\
-7.324	-6.98693219867688\\
-10.986	-10.4803982980153\\
-12.207	-11.6452049903398\\
-9.766	-9.31654558332584\\
-3.662	-3.49346609933844\\
-4.883	-4.65827279166292\\
-3.662	-3.49346609933844\\
-7.324	-6.98693219867688\\
-3.662	-3.49346609933844\\
-8.545	-8.15173889100136\\
-7.324	-6.98693219867688\\
-6.104	-5.82307948398739\\
-7.324	-6.98693219867688\\
-3.662	-3.49346609933844\\
0	0\\
-2.441	-2.32865940701396\\
-6.104	-5.82307948398739\\
-10.986	-10.4803982980153\\
-18.311	-17.4682844743272\\
-14.648	-13.9738643973538\\
-9.766	-9.31654558332584\\
-6.104	-5.82307948398739\\
-8.545	-8.15173889100136\\
-1.221	-1.16480669232448\\
-3.662	-3.49346609933844\\
-6.104	-5.82307948398739\\
-12.207	-11.6452049903398\\
-7.324	-6.98693219867688\\
-9.766	-9.31654558332584\\
-10.986	-10.4803982980153\\
-9.766	-9.31654558332584\\
-10.986	-10.4803982980153\\
-13.428	-12.8100116826643\\
-17.09	-16.3034777820027\\
-14.648	-13.9738643973538\\
-2.441	-2.32865940701396\\
-3.662	-3.49346609933844\\
-8.545	-8.15173889100136\\
-4.883	-4.65827279166292\\
-6.104	-5.82307948398739\\
-3.662	-3.49346609933844\\
-1.221	-1.16480669232448\\
-2.441	-2.32865940701396\\
-4.883	-4.65827279166292\\
-8.545	-8.15173889100136\\
-13.428	-12.8100116826643\\
-6.104	-5.82307948398739\\
-13.428	-12.8100116826643\\
-7.324	-6.98693219867688\\
-3.662	-3.49346609933844\\
-12.207	-11.6452049903398\\
-8.545	-8.15173889100136\\
-13.428	-12.8100116826643\\
-21.973	-20.9617505736656\\
-10.986	-10.4803982980153\\
-8.545	-8.15173889100136\\
};
\addlegendentry{data2}

\end{axis}

\begin{axis}[%
width=4.927cm,
height=3.484cm,
at={(6.484cm,9.677cm)},
scale only axis,
xmin=-29.297,
xmax=2.441,
xlabel style={font=\color{white!15!black}},
xlabel={y(t-1)},
ymin=-29.297,
ymax=2.441,
ylabel style={font=\color{white!15!black}},
ylabel={y(t)},
axis background/.style={fill=white},
title={C5, R = 0.7172},
axis x line*=bottom,
axis y line*=left,
legend style={legend cell align=left, align=left, draw=white!15!black}
]
\addplot[only marks, mark=*, mark options={}, mark size=1.5000pt, color=mycolor1, fill=mycolor1] table[row sep=crcr]{%
x	y\\
-12.207	-10.986\\
-10.986	-15.869\\
-15.869	-10.986\\
-10.986	-12.207\\
-12.207	-13.428\\
-13.428	-15.869\\
-15.869	-14.648\\
-14.648	-14.648\\
-14.648	-7.324\\
-7.324	-8.545\\
-8.545	-10.986\\
-10.986	-9.766\\
-9.766	-9.766\\
-9.766	-6.104\\
-6.104	-2.441\\
-2.441	-3.662\\
-3.662	-2.441\\
-2.441	-12.207\\
-12.207	-13.428\\
-13.428	-12.207\\
-12.207	-10.986\\
-10.986	-6.104\\
-6.104	-6.104\\
-6.104	-1.221\\
-1.221	-6.104\\
-6.104	-10.986\\
-10.986	-12.207\\
-12.207	-8.545\\
-8.545	-14.648\\
-14.648	-14.648\\
-14.648	-9.766\\
-9.766	-13.428\\
-13.428	-12.207\\
-12.207	-9.766\\
-9.766	-9.766\\
-9.766	-8.545\\
-8.545	-8.545\\
-8.545	-12.207\\
-12.207	-10.986\\
-10.986	-10.986\\
-10.986	-10.986\\
-10.986	-10.986\\
-10.986	-14.648\\
-14.648	-15.869\\
-15.869	-9.766\\
-9.766	-8.545\\
-8.545	-8.545\\
-8.545	-6.104\\
-6.104	-7.324\\
-7.324	-7.324\\
-7.324	-4.883\\
-4.883	-9.766\\
-9.766	-10.986\\
-10.986	-8.545\\
-8.545	-13.428\\
-13.428	-15.869\\
-15.869	-8.545\\
-8.545	-6.104\\
-6.104	-6.104\\
-6.104	-9.766\\
-9.766	-6.104\\
-6.104	-4.883\\
-4.883	-7.324\\
-7.324	-7.324\\
-7.324	-3.662\\
-3.662	-6.104\\
-6.104	-8.545\\
-8.545	-10.986\\
-10.986	-10.986\\
-10.986	-9.766\\
-9.766	-13.428\\
-13.428	-10.986\\
-10.986	-12.207\\
-12.207	-10.986\\
-10.986	-10.986\\
-10.986	-7.324\\
-7.324	-8.545\\
-8.545	-14.648\\
-14.648	-20.752\\
-20.752	-20.752\\
-20.752	-19.531\\
-19.531	-13.428\\
-13.428	-15.869\\
-15.869	-24.414\\
-24.414	-23.193\\
-23.193	-15.869\\
-15.869	-21.973\\
-21.973	-28.076\\
-28.076	-23.193\\
-23.193	-17.09\\
-17.09	-14.648\\
-14.648	-13.428\\
-13.428	-9.766\\
-9.766	-10.986\\
-10.986	-12.207\\
-12.207	-4.883\\
-4.883	-4.883\\
-4.883	-7.324\\
-7.324	-10.986\\
-10.986	-9.766\\
-9.766	-9.766\\
-9.766	-10.986\\
-10.986	-13.428\\
-13.428	-14.648\\
-14.648	-15.869\\
-15.869	-14.648\\
-14.648	-17.09\\
-17.09	-12.207\\
-12.207	-12.207\\
-12.207	-10.986\\
-10.986	-8.545\\
-8.545	-9.766\\
-9.766	-12.207\\
-12.207	-8.545\\
-8.545	-9.766\\
-9.766	-7.324\\
-7.324	-6.104\\
-6.104	-8.545\\
-8.545	-6.104\\
-6.104	-4.883\\
-4.883	-8.545\\
-8.545	-13.428\\
-13.428	-12.207\\
-12.207	-10.986\\
-10.986	-7.324\\
-7.324	-8.545\\
-8.545	-19.531\\
-19.531	-14.648\\
-14.648	-10.986\\
-10.986	-17.09\\
-17.09	-12.207\\
-12.207	-6.104\\
-6.104	-9.766\\
-9.766	-8.545\\
-8.545	-14.648\\
-14.648	-15.869\\
-15.869	-19.531\\
-19.531	-14.648\\
-14.648	-14.648\\
-14.648	-13.428\\
-13.428	-13.428\\
-13.428	-14.648\\
-14.648	-10.986\\
-10.986	-7.324\\
-7.324	-7.324\\
-7.324	-6.104\\
-6.104	-8.545\\
-8.545	-10.986\\
-10.986	-15.869\\
-15.869	-14.648\\
-14.648	-8.545\\
-8.545	-8.545\\
-8.545	-9.766\\
-9.766	-6.104\\
-6.104	-2.441\\
-2.441	-4.883\\
-4.883	-7.324\\
-7.324	-8.545\\
-8.545	-6.104\\
-6.104	-7.324\\
-7.324	-6.104\\
-6.104	-4.883\\
-4.883	-4.883\\
-4.883	-1.221\\
-1.221	-3.662\\
-3.662	-12.207\\
-12.207	-14.648\\
-14.648	-15.869\\
-15.869	-12.207\\
-12.207	-7.324\\
-7.324	-6.104\\
-6.104	-3.662\\
-3.662	-7.324\\
-7.324	-8.545\\
-8.545	-9.766\\
-9.766	-12.207\\
-12.207	-18.311\\
-18.311	-18.311\\
-18.311	-19.531\\
-19.531	-18.311\\
-18.311	-15.869\\
-15.869	-15.869\\
-15.869	-17.09\\
-17.09	-18.311\\
-18.311	-12.207\\
-12.207	-10.986\\
-10.986	-12.207\\
-12.207	-10.986\\
-10.986	-12.207\\
-12.207	-9.766\\
-9.766	-15.869\\
-15.869	-18.311\\
-18.311	-12.207\\
-12.207	-7.324\\
-7.324	-9.766\\
-9.766	-17.09\\
-17.09	-18.311\\
-18.311	-18.311\\
-18.311	-21.973\\
-21.973	-20.752\\
-20.752	-17.09\\
-17.09	-15.869\\
-15.869	-18.311\\
-18.311	-20.752\\
-20.752	-17.09\\
-17.09	-8.545\\
-8.545	-14.648\\
-14.648	-12.207\\
-12.207	-7.324\\
-7.324	-10.986\\
-10.986	-9.766\\
-9.766	-8.545\\
-8.545	-7.324\\
-7.324	-8.545\\
-8.545	-8.545\\
-8.545	-9.766\\
-9.766	-13.428\\
-13.428	-14.648\\
-14.648	-7.324\\
-7.324	-8.545\\
-8.545	-12.207\\
-12.207	-17.09\\
-17.09	-15.869\\
-15.869	-8.545\\
-8.545	-8.545\\
-8.545	-9.766\\
-9.766	-8.545\\
-8.545	-7.324\\
-7.324	-6.104\\
-6.104	-8.545\\
-8.545	-9.766\\
-9.766	-9.766\\
-9.766	-9.766\\
-9.766	-12.207\\
-12.207	-14.648\\
-14.648	-15.869\\
-15.869	-10.986\\
-10.986	-13.428\\
-13.428	-17.09\\
-17.09	-12.207\\
-12.207	-3.662\\
-3.662	-9.766\\
-9.766	-8.545\\
-8.545	-9.766\\
-9.766	-8.545\\
-8.545	-8.545\\
-8.545	-8.545\\
-8.545	-10.986\\
-10.986	-7.324\\
-7.324	-7.324\\
-7.324	-9.766\\
-9.766	-6.104\\
-6.104	-8.545\\
-8.545	-6.104\\
-6.104	-3.662\\
-3.662	-6.104\\
-6.104	-4.883\\
-4.883	-10.986\\
-10.986	-12.207\\
-12.207	-12.207\\
-12.207	-18.311\\
-18.311	-10.986\\
-10.986	-4.883\\
-4.883	-6.104\\
-6.104	-6.104\\
-6.104	-8.545\\
-8.545	-6.104\\
-6.104	-8.545\\
-8.545	-7.324\\
-7.324	-13.428\\
-13.428	-9.766\\
-9.766	-14.648\\
-14.648	-10.986\\
-10.986	-9.766\\
-9.766	-6.104\\
-6.104	-3.662\\
-3.662	-3.662\\
-3.662	-1.221\\
-1.221	-2.441\\
-2.441	-8.545\\
-8.545	-8.545\\
-8.545	-7.324\\
-7.324	-9.766\\
-9.766	-15.869\\
-15.869	-10.986\\
-10.986	-9.766\\
-9.766	-9.766\\
-9.766	-13.428\\
-13.428	-14.648\\
-14.648	-12.207\\
-12.207	-12.207\\
-12.207	-6.104\\
-6.104	-2.441\\
-2.441	-4.883\\
-4.883	-4.883\\
-4.883	-6.104\\
-6.104	-3.662\\
-3.662	-4.883\\
-4.883	-9.766\\
-9.766	-7.324\\
-7.324	-7.324\\
-7.324	-10.986\\
-10.986	-7.324\\
-7.324	-4.883\\
-4.883	-9.766\\
-9.766	-12.207\\
-12.207	-9.766\\
-9.766	-10.986\\
-10.986	-7.324\\
-7.324	-7.324\\
-7.324	-10.986\\
-10.986	-14.648\\
-14.648	-13.428\\
-13.428	-8.545\\
-8.545	-14.648\\
-14.648	-13.428\\
-13.428	-13.428\\
-13.428	-13.428\\
-13.428	-13.428\\
-13.428	-10.986\\
-10.986	-10.986\\
-10.986	-12.207\\
-12.207	-18.311\\
-18.311	-17.09\\
-17.09	-10.986\\
-10.986	-10.986\\
-10.986	-10.986\\
-10.986	-13.428\\
-13.428	-13.428\\
-13.428	-9.766\\
-9.766	-12.207\\
-12.207	-14.648\\
-14.648	-14.648\\
-14.648	-9.766\\
-9.766	-8.545\\
-8.545	-10.986\\
-10.986	-18.311\\
-18.311	-14.648\\
-14.648	-13.428\\
-13.428	-9.766\\
-9.766	-10.986\\
-10.986	-9.766\\
-9.766	-6.104\\
-6.104	-4.883\\
-4.883	-3.662\\
-3.662	-3.662\\
-3.662	-7.324\\
-7.324	-10.986\\
-10.986	-9.766\\
-9.766	-6.104\\
-6.104	-7.324\\
-7.324	-9.766\\
-9.766	-6.104\\
-6.104	-6.104\\
-6.104	-9.766\\
-9.766	-6.104\\
-6.104	-7.324\\
-7.324	-12.207\\
-12.207	-12.207\\
-12.207	-8.545\\
-8.545	-4.883\\
-4.883	-6.104\\
-6.104	-7.324\\
-7.324	-6.104\\
-6.104	-7.324\\
-7.324	-14.648\\
-14.648	-9.766\\
-9.766	-17.09\\
-17.09	-20.752\\
-20.752	-18.311\\
-18.311	-13.428\\
-13.428	-10.986\\
-10.986	-10.986\\
-10.986	-12.207\\
-12.207	-13.428\\
-13.428	-14.648\\
-14.648	-15.869\\
-15.869	-20.752\\
-20.752	-20.752\\
-20.752	-24.414\\
-24.414	-15.869\\
-15.869	-20.752\\
-20.752	-20.752\\
-20.752	-14.648\\
-14.648	-17.09\\
-17.09	-18.311\\
-18.311	-9.766\\
-9.766	-7.324\\
-7.324	-8.545\\
-8.545	-9.766\\
-9.766	-7.324\\
-7.324	-6.104\\
-6.104	-7.324\\
-7.324	-6.104\\
-6.104	-3.662\\
-3.662	-4.883\\
-4.883	-7.324\\
-7.324	-2.441\\
-2.441	-10.986\\
-10.986	-8.545\\
-8.545	-7.324\\
-7.324	-15.869\\
-15.869	-19.531\\
-19.531	-19.531\\
-19.531	-19.531\\
-19.531	-14.648\\
-14.648	-17.09\\
-17.09	-12.207\\
-12.207	-7.324\\
-7.324	-10.986\\
-10.986	-7.324\\
-7.324	-3.662\\
-3.662	-6.104\\
-6.104	-4.883\\
-4.883	-2.441\\
-2.441	-4.883\\
-4.883	-10.986\\
-10.986	-9.766\\
-9.766	-10.986\\
-10.986	-10.986\\
-10.986	-12.207\\
-12.207	-14.648\\
-14.648	-10.986\\
-10.986	-10.986\\
-10.986	-8.545\\
-8.545	-7.324\\
-7.324	-13.428\\
-13.428	-9.766\\
-9.766	-6.104\\
-6.104	-10.986\\
-10.986	-13.428\\
-13.428	-10.986\\
-10.986	-7.324\\
-7.324	-7.324\\
-7.324	-7.324\\
-7.324	-4.883\\
-4.883	-6.104\\
-6.104	-7.324\\
-7.324	-10.986\\
-10.986	-8.545\\
-8.545	-7.324\\
-7.324	-10.986\\
-10.986	-8.545\\
-8.545	-6.104\\
-6.104	-7.324\\
-7.324	-13.428\\
-13.428	-14.648\\
-14.648	-13.428\\
-13.428	-12.207\\
-12.207	-12.207\\
-12.207	-8.545\\
-8.545	-9.766\\
-9.766	-8.545\\
-8.545	-12.207\\
-12.207	-7.324\\
-7.324	-4.883\\
-4.883	-9.766\\
-9.766	-9.766\\
-9.766	-10.986\\
-10.986	-12.207\\
-12.207	-10.986\\
-10.986	-17.09\\
-17.09	-20.752\\
-20.752	-23.193\\
-23.193	-15.869\\
-15.869	-9.766\\
-9.766	-6.104\\
-6.104	-6.104\\
-6.104	-9.766\\
-9.766	-6.104\\
-6.104	-9.766\\
-9.766	-6.104\\
-6.104	-7.324\\
-7.324	-8.545\\
-8.545	-10.986\\
-10.986	-13.428\\
-13.428	-14.648\\
-14.648	-19.531\\
-19.531	-14.648\\
-14.648	-13.428\\
-13.428	-10.986\\
-10.986	-10.986\\
-10.986	-10.986\\
-10.986	-12.207\\
-12.207	-9.766\\
-9.766	-10.986\\
-10.986	-9.766\\
-9.766	-7.324\\
-7.324	-12.207\\
-12.207	-13.428\\
-13.428	-8.545\\
-8.545	-9.766\\
-9.766	-18.311\\
-18.311	-12.207\\
-12.207	-12.207\\
-12.207	-17.09\\
-17.09	-14.648\\
-14.648	-10.986\\
-10.986	-7.324\\
-7.324	-7.324\\
-7.324	-12.207\\
-12.207	-17.09\\
-17.09	-20.752\\
-20.752	-18.311\\
-18.311	-13.428\\
-13.428	-9.766\\
-9.766	-8.545\\
-8.545	-6.104\\
-6.104	-7.324\\
-7.324	-4.883\\
-4.883	-8.545\\
-8.545	-8.545\\
-8.545	-6.104\\
-6.104	-9.766\\
-9.766	-10.986\\
-10.986	-12.207\\
-12.207	-14.648\\
-14.648	-17.09\\
-17.09	-19.531\\
-19.531	-18.311\\
-18.311	-13.428\\
-13.428	-14.648\\
-14.648	-12.207\\
-12.207	-7.324\\
-7.324	-13.428\\
-13.428	-19.531\\
-19.531	-18.311\\
-18.311	-12.207\\
-12.207	-6.104\\
-6.104	-4.883\\
-4.883	-1.221\\
-1.221	2.441\\
2.441	-4.883\\
-4.883	-6.104\\
-6.104	-8.545\\
-8.545	-6.104\\
-6.104	-4.883\\
-4.883	-4.883\\
-4.883	-7.324\\
-7.324	-4.883\\
-4.883	-4.883\\
-4.883	-8.545\\
-8.545	-12.207\\
-12.207	-13.428\\
-13.428	-10.986\\
-10.986	-14.648\\
-14.648	-17.09\\
-17.09	-17.09\\
-17.09	-20.752\\
-20.752	-20.752\\
-20.752	-14.648\\
-14.648	-10.986\\
-10.986	-10.986\\
-10.986	-18.311\\
-18.311	-18.311\\
-18.311	-12.207\\
-12.207	-8.545\\
-8.545	-4.883\\
-4.883	-4.883\\
-4.883	-4.883\\
-4.883	-2.441\\
-2.441	-6.104\\
-6.104	-8.545\\
-8.545	-7.324\\
-7.324	-7.324\\
-7.324	-6.104\\
-6.104	-3.662\\
-3.662	-4.883\\
-4.883	-4.883\\
-4.883	-6.104\\
-6.104	-2.441\\
-2.441	-2.441\\
-2.441	-7.324\\
-7.324	-9.766\\
-9.766	-13.428\\
-13.428	-15.869\\
-15.869	-10.986\\
-10.986	-18.311\\
-18.311	-23.193\\
-23.193	-21.973\\
-21.973	-19.531\\
-19.531	-14.648\\
-14.648	-14.648\\
-14.648	-17.09\\
-17.09	-13.428\\
-13.428	-9.766\\
-9.766	-7.324\\
-7.324	-7.324\\
-7.324	-13.428\\
-13.428	-8.545\\
-8.545	-12.207\\
-12.207	-10.986\\
-10.986	-8.545\\
-8.545	-8.545\\
-8.545	-10.986\\
-10.986	-9.766\\
-9.766	-12.207\\
-12.207	-10.986\\
-10.986	-8.545\\
-8.545	-12.207\\
-12.207	-6.104\\
-6.104	-2.441\\
-2.441	-7.324\\
-7.324	-7.324\\
-7.324	-2.441\\
-2.441	-1.221\\
-1.221	-2.441\\
-2.441	-6.104\\
-6.104	-6.104\\
-6.104	-8.545\\
-8.545	-7.324\\
-7.324	-13.428\\
-13.428	-8.545\\
-8.545	-6.104\\
-6.104	-10.986\\
-10.986	-7.324\\
-7.324	-4.883\\
-4.883	-4.883\\
-4.883	-2.441\\
-2.441	-3.662\\
-3.662	-3.662\\
-3.662	-6.104\\
-6.104	-9.766\\
-9.766	-8.545\\
-8.545	-4.883\\
-4.883	-7.324\\
-7.324	-7.324\\
-7.324	-7.324\\
-7.324	-6.104\\
-6.104	-2.441\\
-2.441	-7.324\\
-7.324	-3.662\\
-3.662	-4.883\\
-4.883	-4.883\\
-4.883	-7.324\\
-7.324	-6.104\\
-6.104	-4.883\\
-4.883	-3.662\\
-3.662	-4.883\\
-4.883	-4.883\\
-4.883	-13.428\\
-13.428	-13.428\\
-13.428	-7.324\\
-7.324	-12.207\\
-12.207	-7.324\\
-7.324	-12.207\\
-12.207	-8.545\\
-8.545	-7.324\\
-7.324	-8.545\\
-8.545	-7.324\\
-7.324	-4.883\\
-4.883	-6.104\\
-6.104	-9.766\\
-9.766	-7.324\\
-7.324	-12.207\\
-12.207	-6.104\\
-6.104	-8.545\\
-8.545	-6.104\\
-6.104	-12.207\\
-12.207	-7.324\\
-7.324	-14.648\\
-14.648	-17.09\\
-17.09	-13.428\\
-13.428	-9.766\\
-9.766	-9.766\\
-9.766	-12.207\\
-12.207	-13.428\\
-13.428	-8.545\\
-8.545	-13.428\\
-13.428	-10.986\\
-10.986	-4.883\\
-4.883	-3.662\\
-3.662	-12.207\\
-12.207	-10.986\\
-10.986	-8.545\\
-8.545	-10.986\\
-10.986	-12.207\\
-12.207	-9.766\\
-9.766	-7.324\\
-7.324	-9.766\\
-9.766	-12.207\\
-12.207	-8.545\\
-8.545	-10.986\\
-10.986	-9.766\\
-9.766	-6.104\\
-6.104	-8.545\\
-8.545	-4.883\\
-4.883	-8.545\\
-8.545	-10.986\\
-10.986	-14.648\\
-14.648	-13.428\\
-13.428	-13.428\\
-13.428	-8.545\\
-8.545	-6.104\\
-6.104	-7.324\\
-7.324	-8.545\\
-8.545	-12.207\\
-12.207	-12.207\\
-12.207	-14.648\\
-14.648	-10.986\\
-10.986	-8.545\\
-8.545	-12.207\\
-12.207	-18.311\\
-18.311	-12.207\\
-12.207	-14.648\\
-14.648	-14.648\\
-14.648	-9.766\\
-9.766	-9.766\\
-9.766	-12.207\\
-12.207	-9.766\\
-9.766	-10.986\\
-10.986	-14.648\\
-14.648	-9.766\\
-9.766	-12.207\\
-12.207	-10.986\\
-10.986	-10.986\\
-10.986	-6.104\\
-6.104	-9.766\\
-9.766	-8.545\\
-8.545	-9.766\\
-9.766	-9.766\\
-9.766	-9.766\\
-9.766	-6.104\\
-6.104	-3.662\\
-3.662	-2.441\\
-2.441	-6.104\\
-6.104	-10.986\\
-10.986	-12.207\\
-12.207	-10.986\\
-10.986	-8.545\\
-8.545	-12.207\\
-12.207	-8.545\\
-8.545	-9.766\\
-9.766	-6.104\\
-6.104	-8.545\\
-8.545	-7.324\\
-7.324	-7.324\\
-7.324	-7.324\\
-7.324	-7.324\\
-7.324	-6.104\\
-6.104	-6.104\\
-6.104	-4.883\\
-4.883	-10.986\\
-10.986	-7.324\\
-7.324	-9.766\\
-9.766	-7.324\\
-7.324	-12.207\\
-12.207	-9.766\\
-9.766	-14.648\\
-14.648	-8.545\\
-8.545	-13.428\\
-13.428	-12.207\\
-12.207	-7.324\\
-7.324	-10.986\\
-10.986	-9.766\\
-9.766	-10.986\\
-10.986	-12.207\\
-12.207	-13.428\\
-13.428	-7.324\\
-7.324	-13.428\\
-13.428	-14.648\\
-14.648	-14.648\\
-14.648	-9.766\\
-9.766	-9.766\\
-9.766	-10.986\\
-10.986	-7.324\\
-7.324	-9.766\\
-9.766	-7.324\\
-7.324	-8.545\\
-8.545	-7.324\\
-7.324	-13.428\\
-13.428	-15.869\\
-15.869	-15.869\\
-15.869	-14.648\\
-14.648	-15.869\\
-15.869	-8.545\\
-8.545	-7.324\\
-7.324	-6.104\\
-6.104	-4.883\\
-4.883	-4.883\\
-4.883	-8.545\\
-8.545	-10.986\\
-10.986	-13.428\\
-13.428	-12.207\\
-12.207	-7.324\\
-7.324	-12.207\\
-12.207	-15.869\\
-15.869	-15.869\\
-15.869	-13.428\\
-13.428	-20.752\\
-20.752	-17.09\\
-17.09	-10.986\\
-10.986	-6.104\\
-6.104	-7.324\\
-7.324	-10.986\\
-10.986	-13.428\\
-13.428	-17.09\\
-17.09	-12.207\\
-12.207	-10.986\\
-10.986	-6.104\\
-6.104	-9.766\\
-9.766	-7.324\\
-7.324	-7.324\\
-7.324	-4.883\\
-4.883	-4.883\\
-4.883	-3.662\\
-3.662	-3.662\\
-3.662	-6.104\\
-6.104	-10.986\\
-10.986	-8.545\\
-8.545	-9.766\\
-9.766	-8.545\\
-8.545	-8.545\\
-8.545	-12.207\\
-12.207	-7.324\\
-7.324	-12.207\\
-12.207	-13.428\\
-13.428	-8.545\\
-8.545	-10.986\\
-10.986	-8.545\\
-8.545	-10.986\\
-10.986	-15.869\\
-15.869	-10.986\\
-10.986	-8.545\\
-8.545	-9.766\\
-9.766	-10.986\\
-10.986	-7.324\\
-7.324	-8.545\\
-8.545	-13.428\\
-13.428	-14.648\\
-14.648	-15.869\\
-15.869	-20.752\\
-20.752	-15.869\\
-15.869	-14.648\\
-14.648	-10.986\\
-10.986	-10.986\\
-10.986	-17.09\\
-17.09	-13.428\\
-13.428	-10.986\\
-10.986	-14.648\\
-14.648	-9.766\\
-9.766	-7.324\\
-7.324	-9.766\\
-9.766	-6.104\\
-6.104	-3.662\\
-3.662	-6.104\\
-6.104	-4.883\\
-4.883	-4.883\\
-4.883	-4.883\\
-4.883	-8.545\\
-8.545	-9.766\\
-9.766	-12.207\\
-12.207	-12.207\\
-12.207	-12.207\\
-12.207	-14.648\\
-14.648	-14.648\\
-14.648	-10.986\\
-10.986	-10.986\\
-10.986	-9.766\\
-9.766	-10.986\\
-10.986	-14.648\\
-14.648	-10.986\\
-10.986	-10.986\\
-10.986	-9.766\\
-9.766	-10.986\\
-10.986	-14.648\\
-14.648	-12.207\\
-12.207	-12.207\\
-12.207	-13.428\\
-13.428	-9.766\\
-9.766	-6.104\\
-6.104	-4.883\\
-4.883	-7.324\\
-7.324	-7.324\\
-7.324	-7.324\\
-7.324	-6.104\\
-6.104	-8.545\\
-8.545	-14.648\\
-14.648	-18.311\\
-18.311	-12.207\\
-12.207	-14.648\\
-14.648	-13.428\\
-13.428	-10.986\\
-10.986	-10.986\\
-10.986	-10.986\\
-10.986	-10.986\\
-10.986	-13.428\\
-13.428	-14.648\\
-14.648	-19.531\\
-19.531	-17.09\\
-17.09	-15.869\\
-15.869	-19.531\\
-19.531	-13.428\\
-13.428	-15.869\\
-15.869	-13.428\\
-13.428	-10.986\\
-10.986	-13.428\\
-13.428	-17.09\\
-17.09	-7.324\\
-7.324	-10.986\\
-10.986	-7.324\\
-7.324	-8.545\\
-8.545	-10.986\\
-10.986	-7.324\\
-7.324	-8.545\\
-8.545	-15.869\\
-15.869	-20.752\\
-20.752	-21.973\\
-21.973	-19.531\\
-19.531	-15.869\\
-15.869	-19.531\\
-19.531	-21.973\\
-21.973	-17.09\\
-17.09	-18.311\\
-18.311	-17.09\\
-17.09	-12.207\\
-12.207	-9.766\\
-9.766	-13.428\\
-13.428	-10.986\\
-10.986	-6.104\\
-6.104	-9.766\\
-9.766	-8.545\\
-8.545	-8.545\\
-8.545	-8.545\\
-8.545	-9.766\\
-9.766	-12.207\\
-12.207	-8.545\\
-8.545	-7.324\\
-7.324	-8.545\\
-8.545	-10.986\\
-10.986	-10.986\\
-10.986	-9.766\\
-9.766	-9.766\\
-9.766	-14.648\\
-14.648	-13.428\\
-13.428	-9.766\\
-9.766	-13.428\\
-13.428	-13.428\\
-13.428	-10.986\\
-10.986	-8.545\\
-8.545	-7.324\\
-7.324	-4.883\\
-4.883	-6.104\\
-6.104	-8.545\\
-8.545	-6.104\\
-6.104	-4.883\\
-4.883	-4.883\\
-4.883	-7.324\\
-7.324	-7.324\\
-7.324	-12.207\\
-12.207	-14.648\\
-14.648	-13.428\\
-13.428	-13.428\\
-13.428	-10.986\\
-10.986	-4.883\\
-4.883	-6.104\\
-6.104	-7.324\\
-7.324	-8.545\\
-8.545	-8.545\\
-8.545	-10.986\\
-10.986	-14.648\\
-14.648	-14.648\\
-14.648	-9.766\\
-9.766	-14.648\\
-14.648	-13.428\\
-13.428	-6.104\\
-6.104	-8.545\\
-8.545	-7.324\\
-7.324	-10.986\\
-10.986	-17.09\\
-17.09	-12.207\\
-12.207	-9.766\\
-9.766	-8.545\\
-8.545	-10.986\\
-10.986	-12.207\\
-12.207	-19.531\\
-19.531	-15.869\\
-15.869	-17.09\\
-17.09	-12.207\\
-12.207	-7.324\\
-7.324	-8.545\\
-8.545	-4.883\\
-4.883	-6.104\\
-6.104	-8.545\\
-8.545	-3.662\\
-3.662	-8.545\\
-8.545	-13.428\\
-13.428	-15.869\\
-15.869	-20.752\\
-20.752	-18.311\\
-18.311	-9.766\\
-9.766	-10.986\\
-10.986	-8.545\\
-8.545	-13.428\\
-13.428	-15.869\\
-15.869	-15.869\\
-15.869	-15.869\\
-15.869	-9.766\\
-9.766	-10.986\\
-10.986	-14.648\\
-14.648	-13.428\\
-13.428	-7.324\\
-7.324	-6.104\\
-6.104	-4.883\\
-4.883	-3.662\\
-3.662	-4.883\\
-4.883	-3.662\\
-3.662	-8.545\\
-8.545	-8.545\\
-8.545	-6.104\\
-6.104	-8.545\\
-8.545	-6.104\\
-6.104	-3.662\\
-3.662	-2.441\\
-2.441	-6.104\\
-6.104	-8.545\\
-8.545	-8.545\\
-8.545	-7.324\\
-7.324	-10.986\\
-10.986	-14.648\\
-14.648	-19.531\\
-19.531	-21.973\\
-21.973	-20.752\\
-20.752	-19.531\\
-19.531	-13.428\\
-13.428	-10.986\\
-10.986	-10.986\\
-10.986	-12.207\\
-12.207	-10.986\\
-10.986	-7.324\\
-7.324	-7.324\\
-7.324	-7.324\\
-7.324	-8.545\\
-8.545	-10.986\\
-10.986	-7.324\\
-7.324	-2.441\\
-2.441	-4.883\\
-4.883	-6.104\\
-6.104	-7.324\\
-7.324	-6.104\\
-6.104	-8.545\\
-8.545	-6.104\\
-6.104	-14.648\\
-14.648	-10.986\\
-10.986	-12.207\\
-12.207	-20.752\\
-20.752	-20.752\\
-20.752	-15.869\\
-15.869	-13.428\\
-13.428	-8.545\\
-8.545	-7.324\\
-7.324	-6.104\\
-6.104	-10.986\\
-10.986	-7.324\\
-7.324	-10.986\\
-10.986	-8.545\\
-8.545	-10.986\\
-10.986	-4.883\\
-4.883	-7.324\\
-7.324	-9.766\\
-9.766	-3.662\\
-3.662	-3.662\\
-3.662	-6.104\\
-6.104	-2.441\\
-2.441	-3.662\\
-3.662	-3.662\\
-3.662	-2.441\\
-2.441	-6.104\\
-6.104	-6.104\\
-6.104	-6.104\\
-6.104	-7.324\\
-7.324	-10.986\\
-10.986	-7.324\\
-7.324	-7.324\\
-7.324	-12.207\\
-12.207	-14.648\\
-14.648	-9.766\\
-9.766	-10.986\\
-10.986	-9.766\\
-9.766	-2.441\\
-2.441	-9.766\\
-9.766	-8.545\\
-8.545	-12.207\\
-12.207	-13.428\\
-13.428	-15.869\\
-15.869	-10.986\\
-10.986	-8.545\\
-8.545	-7.324\\
-7.324	-8.545\\
-8.545	-6.104\\
-6.104	-4.883\\
-4.883	-4.883\\
-4.883	-4.883\\
-4.883	-3.662\\
-3.662	-4.883\\
-4.883	-3.662\\
-3.662	-8.545\\
-8.545	-8.545\\
-8.545	-4.883\\
-4.883	-7.324\\
-7.324	-14.648\\
-14.648	-6.104\\
-6.104	-9.766\\
-9.766	-17.09\\
-17.09	-9.766\\
-9.766	-6.104\\
-6.104	-9.766\\
-9.766	-10.986\\
-10.986	-8.545\\
-8.545	-3.662\\
-3.662	-6.104\\
-6.104	-1.221\\
-1.221	-2.441\\
-2.441	-6.104\\
-6.104	-6.104\\
-6.104	-7.324\\
-7.324	-6.104\\
-6.104	-8.545\\
-8.545	-7.324\\
-7.324	-3.662\\
-3.662	-1.221\\
-1.221	-6.104\\
-6.104	-4.883\\
-4.883	-7.324\\
-7.324	-10.986\\
-10.986	-7.324\\
-7.324	-8.545\\
-8.545	-9.766\\
-9.766	-12.207\\
-12.207	-14.648\\
-14.648	-17.09\\
-17.09	-17.09\\
-17.09	-12.207\\
-12.207	-10.986\\
-10.986	-12.207\\
-12.207	-12.207\\
-12.207	-13.428\\
-13.428	-18.311\\
-18.311	-17.09\\
-17.09	-13.428\\
-13.428	-12.207\\
-12.207	-12.207\\
-12.207	-12.207\\
-12.207	-13.428\\
-13.428	-10.986\\
-10.986	-10.986\\
-10.986	-12.207\\
-12.207	-13.428\\
-13.428	-12.207\\
-12.207	-10.986\\
-10.986	-10.986\\
-10.986	-9.766\\
-9.766	-9.766\\
-9.766	-7.324\\
-7.324	-12.207\\
-12.207	-8.545\\
-8.545	-6.104\\
-6.104	-8.545\\
-8.545	-8.545\\
-8.545	-3.662\\
-3.662	-7.324\\
-7.324	-7.324\\
-7.324	-3.662\\
-3.662	-6.104\\
-6.104	-4.883\\
-4.883	-2.441\\
-2.441	-6.104\\
-6.104	-3.662\\
-3.662	-3.662\\
-3.662	-8.545\\
-8.545	-3.662\\
-3.662	-8.545\\
-8.545	-4.883\\
-4.883	-7.324\\
-7.324	-8.545\\
-8.545	-7.324\\
-7.324	-13.428\\
-13.428	-4.883\\
-4.883	-8.545\\
-8.545	-10.986\\
-10.986	-10.986\\
-10.986	-6.104\\
-6.104	-7.324\\
-7.324	-12.207\\
-12.207	-3.662\\
-3.662	-6.104\\
-6.104	-8.545\\
-8.545	-6.104\\
-6.104	-9.766\\
-9.766	-1.221\\
-1.221	-3.662\\
-3.662	-4.883\\
-4.883	-4.883\\
-4.883	-7.324\\
-7.324	-3.662\\
-3.662	-8.545\\
-8.545	-10.986\\
-10.986	-9.766\\
-9.766	-8.545\\
-8.545	-14.648\\
-14.648	-10.986\\
-10.986	-10.986\\
-10.986	-18.311\\
-18.311	-18.311\\
-18.311	-13.428\\
-13.428	-17.09\\
-17.09	-13.428\\
-13.428	-17.09\\
-17.09	-13.428\\
-13.428	-4.883\\
-4.883	-7.324\\
-7.324	-12.207\\
-12.207	-10.986\\
-10.986	-10.986\\
-10.986	-8.545\\
-8.545	-4.883\\
-4.883	-6.104\\
-6.104	-9.766\\
-9.766	-15.869\\
-15.869	-13.428\\
-13.428	-9.766\\
-9.766	-9.766\\
-9.766	-7.324\\
-7.324	-12.207\\
-12.207	-15.869\\
-15.869	-14.648\\
-14.648	-13.428\\
-13.428	-13.428\\
-13.428	-19.531\\
-19.531	-17.09\\
-17.09	-19.531\\
-19.531	-20.752\\
-20.752	-15.869\\
-15.869	-18.311\\
-18.311	-15.869\\
-15.869	-8.545\\
-8.545	-8.545\\
-8.545	-10.986\\
-10.986	-13.428\\
-13.428	-13.428\\
-13.428	-10.986\\
-10.986	-9.766\\
-9.766	-13.428\\
-13.428	-9.766\\
-9.766	-12.207\\
-12.207	-9.766\\
-9.766	-14.648\\
-14.648	-18.311\\
-18.311	-10.986\\
-10.986	-8.545\\
-8.545	-7.324\\
-7.324	-7.324\\
-7.324	-12.207\\
-12.207	-3.662\\
-3.662	-4.883\\
-4.883	-13.428\\
-13.428	-10.986\\
-10.986	-10.986\\
-10.986	-9.766\\
-9.766	-8.545\\
-8.545	-3.662\\
-3.662	-6.104\\
-6.104	-4.883\\
-4.883	-8.545\\
-8.545	-7.324\\
-7.324	-7.324\\
-7.324	-13.428\\
-13.428	-13.428\\
-13.428	-15.869\\
-15.869	-14.648\\
-14.648	-17.09\\
-17.09	-14.648\\
-14.648	-9.766\\
-9.766	-8.545\\
-8.545	-8.545\\
-8.545	-9.766\\
-9.766	-9.766\\
-9.766	-6.104\\
-6.104	-4.883\\
-4.883	-10.986\\
-10.986	-12.207\\
-12.207	-9.766\\
-9.766	-14.648\\
-14.648	-17.09\\
-17.09	-12.207\\
-12.207	-8.545\\
-8.545	-7.324\\
-7.324	-7.324\\
-7.324	-9.766\\
-9.766	-8.545\\
-8.545	-9.766\\
-9.766	-15.869\\
-15.869	-12.207\\
-12.207	-12.207\\
-12.207	-12.207\\
-12.207	-10.986\\
-10.986	-10.986\\
-10.986	-10.986\\
-10.986	-7.324\\
-7.324	-12.207\\
-12.207	-14.648\\
-14.648	-14.648\\
-14.648	-12.207\\
-12.207	-10.986\\
-10.986	-17.09\\
-17.09	-10.986\\
-10.986	-8.545\\
-8.545	-10.986\\
-10.986	-7.324\\
-7.324	-9.766\\
-9.766	-10.986\\
-10.986	-8.545\\
-8.545	-9.766\\
-9.766	-15.869\\
-15.869	-18.311\\
-18.311	-10.986\\
-10.986	-12.207\\
-12.207	-13.428\\
-13.428	-14.648\\
-14.648	-9.766\\
-9.766	-9.766\\
-9.766	-8.545\\
-8.545	-10.986\\
-10.986	-8.545\\
-8.545	-15.869\\
-15.869	-20.752\\
-20.752	-23.193\\
-23.193	-21.973\\
-21.973	-14.648\\
-14.648	-13.428\\
-13.428	-12.207\\
-12.207	-12.207\\
-12.207	-8.545\\
-8.545	-6.104\\
-6.104	-7.324\\
-7.324	-7.324\\
-7.324	-3.662\\
-3.662	-4.883\\
-4.883	-10.986\\
-10.986	-7.324\\
-7.324	-3.662\\
-3.662	-3.662\\
-3.662	-8.545\\
-8.545	-10.986\\
-10.986	-9.766\\
-9.766	-6.104\\
-6.104	-12.207\\
-12.207	-6.104\\
-6.104	-7.324\\
-7.324	-8.545\\
-8.545	-6.104\\
-6.104	-6.104\\
-6.104	-4.883\\
-4.883	-6.104\\
-6.104	-8.545\\
-8.545	-9.766\\
-9.766	-8.545\\
-8.545	-6.104\\
-6.104	-4.883\\
-4.883	-4.883\\
-4.883	-8.545\\
-8.545	-9.766\\
-9.766	-8.545\\
-8.545	-8.545\\
-8.545	-6.104\\
-6.104	-7.324\\
-7.324	-8.545\\
-8.545	-14.648\\
-14.648	-17.09\\
-17.09	-13.428\\
-13.428	-7.324\\
-7.324	-9.766\\
-9.766	-9.766\\
-9.766	-8.545\\
-8.545	-7.324\\
-7.324	-8.545\\
-8.545	-8.545\\
-8.545	-7.324\\
-7.324	-3.662\\
-3.662	-3.662\\
-3.662	-4.883\\
-4.883	-7.324\\
-7.324	-9.766\\
-9.766	-12.207\\
-12.207	-10.986\\
-10.986	-14.648\\
-14.648	-13.428\\
-13.428	-17.09\\
-17.09	-23.193\\
-23.193	-17.09\\
-17.09	-14.648\\
-14.648	-17.09\\
-17.09	-18.311\\
-18.311	-19.531\\
-19.531	-18.311\\
-18.311	-8.545\\
-8.545	-4.883\\
-4.883	-6.104\\
-6.104	-3.662\\
-3.662	-3.662\\
-3.662	-3.662\\
-3.662	-6.104\\
-6.104	-6.104\\
-6.104	-8.545\\
-8.545	-4.883\\
-4.883	-6.104\\
-6.104	-6.104\\
-6.104	-8.545\\
-8.545	-7.324\\
-7.324	-7.324\\
-7.324	-6.104\\
-6.104	-6.104\\
-6.104	-6.104\\
-6.104	-7.324\\
-7.324	-9.766\\
-9.766	-4.883\\
-4.883	-3.662\\
-3.662	-6.104\\
-6.104	-4.883\\
-4.883	-8.545\\
-8.545	-9.766\\
-9.766	-7.324\\
-7.324	-9.766\\
-9.766	-7.324\\
-7.324	-9.766\\
-9.766	-14.648\\
-14.648	-7.324\\
-7.324	-7.324\\
-7.324	-10.986\\
-10.986	-14.648\\
-14.648	-14.648\\
-14.648	-14.648\\
-14.648	-10.986\\
-10.986	-12.207\\
-12.207	-12.207\\
-12.207	-13.428\\
-13.428	-12.207\\
-12.207	-13.428\\
-13.428	-15.869\\
-15.869	-13.428\\
-13.428	-21.973\\
-21.973	-20.752\\
-20.752	-12.207\\
-12.207	-8.545\\
-8.545	-8.545\\
-8.545	-8.545\\
-8.545	-12.207\\
-12.207	-12.207\\
-12.207	-13.428\\
-13.428	-15.869\\
-15.869	-10.986\\
-10.986	-9.766\\
-9.766	-12.207\\
-12.207	-9.766\\
-9.766	-9.766\\
-9.766	-3.662\\
-3.662	-9.766\\
-9.766	-12.207\\
-12.207	-10.986\\
-10.986	-14.648\\
-14.648	-13.428\\
-13.428	-8.545\\
-8.545	-8.545\\
-8.545	-12.207\\
-12.207	-12.207\\
-12.207	-15.869\\
-15.869	-19.531\\
-19.531	-20.752\\
-20.752	-18.311\\
-18.311	-17.09\\
-17.09	-10.986\\
-10.986	-8.545\\
-8.545	-13.428\\
-13.428	-19.531\\
-19.531	-10.986\\
-10.986	-12.207\\
-12.207	-21.973\\
-21.973	-21.973\\
-21.973	-19.531\\
-19.531	-9.766\\
-9.766	-6.104\\
-6.104	-9.766\\
-9.766	-10.986\\
-10.986	-7.324\\
-7.324	-4.883\\
-4.883	-6.104\\
-6.104	-6.104\\
-6.104	-8.545\\
-8.545	-7.324\\
-7.324	-6.104\\
-6.104	-8.545\\
-8.545	-9.766\\
-9.766	-12.207\\
-12.207	-14.648\\
-14.648	-14.648\\
-14.648	-9.766\\
-9.766	-12.207\\
-12.207	-9.766\\
-9.766	-8.545\\
-8.545	-7.324\\
-7.324	-7.324\\
-7.324	-8.545\\
-8.545	-9.766\\
-9.766	-6.104\\
-6.104	-3.662\\
-3.662	-3.662\\
-3.662	-4.883\\
-4.883	-7.324\\
-7.324	-7.324\\
-7.324	-6.104\\
-6.104	-10.986\\
-10.986	-8.545\\
-8.545	-9.766\\
-9.766	-6.104\\
-6.104	-4.883\\
-4.883	-7.324\\
-7.324	-10.986\\
-10.986	-9.766\\
-9.766	-9.766\\
-9.766	-10.986\\
-10.986	-7.324\\
-7.324	-7.324\\
-7.324	-8.545\\
-8.545	-7.324\\
-7.324	-7.324\\
-7.324	-12.207\\
-12.207	-15.869\\
-15.869	-12.207\\
-12.207	-12.207\\
-12.207	-15.869\\
-15.869	-13.428\\
-13.428	-8.545\\
-8.545	-7.324\\
-7.324	-4.883\\
-4.883	-4.883\\
-4.883	-3.662\\
-3.662	-3.662\\
-3.662	-10.986\\
-10.986	-6.104\\
-6.104	-4.883\\
-4.883	-6.104\\
-6.104	-8.545\\
-8.545	-7.324\\
-7.324	-6.104\\
-6.104	-4.883\\
-4.883	-6.104\\
-6.104	-10.986\\
-10.986	-10.986\\
-10.986	-7.324\\
-7.324	-4.883\\
-4.883	-9.766\\
-9.766	-8.545\\
-8.545	-9.766\\
-9.766	-10.986\\
-10.986	-13.428\\
-13.428	-13.428\\
-13.428	-12.207\\
-12.207	-8.545\\
-8.545	-7.324\\
-7.324	-6.104\\
-6.104	-9.766\\
-9.766	-6.104\\
-6.104	-6.104\\
-6.104	-8.545\\
-8.545	-12.207\\
-12.207	-9.766\\
-9.766	-10.986\\
-10.986	-15.869\\
-15.869	-10.986\\
-10.986	-10.986\\
-10.986	-14.648\\
-14.648	-12.207\\
-12.207	-15.869\\
-15.869	-13.428\\
-13.428	-21.973\\
-21.973	-21.973\\
-21.973	-15.869\\
-15.869	-19.531\\
-19.531	-21.973\\
-21.973	-20.752\\
-20.752	-23.193\\
-23.193	-21.973\\
-21.973	-25.635\\
-25.635	-20.752\\
-20.752	-13.428\\
-13.428	-9.766\\
-9.766	-12.207\\
-12.207	-8.545\\
-8.545	-7.324\\
-7.324	-12.207\\
-12.207	-14.648\\
-14.648	-9.766\\
-9.766	-7.324\\
-7.324	-9.766\\
-9.766	-7.324\\
-7.324	-8.545\\
-8.545	-6.104\\
-6.104	-9.766\\
-9.766	-4.883\\
-4.883	-4.883\\
-4.883	-1.221\\
-1.221	-3.662\\
-3.662	-2.441\\
-2.441	-3.662\\
-3.662	-4.883\\
-4.883	-4.883\\
-4.883	-4.883\\
-4.883	-3.662\\
-3.662	-8.545\\
-8.545	-9.766\\
-9.766	-6.104\\
-6.104	-12.207\\
-12.207	-14.648\\
-14.648	-8.545\\
-8.545	-2.441\\
-2.441	-3.662\\
-3.662	-2.441\\
-2.441	-6.104\\
-6.104	-7.324\\
-7.324	-8.545\\
-8.545	-3.662\\
-3.662	-7.324\\
-7.324	-14.648\\
-14.648	-14.648\\
-14.648	-9.766\\
-9.766	-7.324\\
-7.324	-12.207\\
-12.207	-13.428\\
-13.428	-14.648\\
-14.648	-7.324\\
-7.324	-4.883\\
-4.883	-6.104\\
-6.104	-6.104\\
-6.104	-3.662\\
-3.662	-3.662\\
-3.662	-8.545\\
-8.545	-14.648\\
-14.648	-10.986\\
-10.986	-6.104\\
-6.104	-3.662\\
-3.662	-4.883\\
-4.883	-7.324\\
-7.324	-4.883\\
-4.883	-3.662\\
-3.662	-6.104\\
-6.104	-6.104\\
-6.104	-12.207\\
-12.207	-9.766\\
-9.766	-10.986\\
-10.986	-12.207\\
-12.207	-10.986\\
-10.986	-12.207\\
-12.207	-10.986\\
-10.986	-13.428\\
-13.428	-18.311\\
-18.311	-15.869\\
-15.869	-7.324\\
-7.324	-4.883\\
-4.883	-8.545\\
-8.545	-14.648\\
-14.648	-10.986\\
-10.986	-4.883\\
-4.883	-10.986\\
-10.986	-9.766\\
-9.766	-17.09\\
-17.09	-20.752\\
-20.752	-19.531\\
-19.531	-19.531\\
-19.531	-15.869\\
-15.869	-14.648\\
-14.648	-7.324\\
-7.324	-9.766\\
-9.766	-8.545\\
-8.545	-10.986\\
-10.986	-10.986\\
-10.986	-12.207\\
-12.207	-7.324\\
-7.324	-10.986\\
-10.986	-8.545\\
-8.545	-6.104\\
-6.104	-3.662\\
-3.662	-4.883\\
-4.883	-4.883\\
-4.883	-4.883\\
-4.883	-3.662\\
-3.662	-6.104\\
-6.104	-9.766\\
-9.766	-6.104\\
-6.104	-6.104\\
-6.104	-6.104\\
-6.104	-13.428\\
-13.428	-6.104\\
-6.104	-3.662\\
-3.662	-1.221\\
-1.221	-6.104\\
-6.104	-10.986\\
-10.986	-14.648\\
-14.648	-21.973\\
-21.973	-23.193\\
-23.193	-24.414\\
-24.414	-17.09\\
-17.09	-18.311\\
-18.311	-20.752\\
-20.752	-18.311\\
-18.311	-17.09\\
-17.09	-18.311\\
-18.311	-20.752\\
-20.752	-20.752\\
-20.752	-28.076\\
-28.076	-19.531\\
-19.531	-12.207\\
-12.207	-7.324\\
-7.324	-6.104\\
-6.104	-6.104\\
-6.104	-6.104\\
-6.104	-4.883\\
-4.883	-12.207\\
-12.207	-7.324\\
-7.324	-10.986\\
-10.986	-10.986\\
-10.986	-7.324\\
-7.324	-12.207\\
-12.207	-12.207\\
-12.207	-12.207\\
-12.207	-6.104\\
-6.104	-4.883\\
-4.883	-6.104\\
-6.104	-8.545\\
-8.545	-13.428\\
-13.428	-19.531\\
-19.531	-18.311\\
-18.311	-18.311\\
-18.311	-23.193\\
-23.193	-29.297\\
-29.297	-21.973\\
-21.973	-14.648\\
-14.648	-9.766\\
-9.766	-6.104\\
-6.104	-10.986\\
-10.986	-8.545\\
-8.545	-9.766\\
-9.766	-7.324\\
-7.324	-8.545\\
-8.545	-7.324\\
-7.324	-9.766\\
-9.766	-9.766\\
-9.766	-10.986\\
-10.986	-7.324\\
-7.324	-6.104\\
-6.104	-3.662\\
-3.662	-3.662\\
-3.662	-1.221\\
-1.221	-3.662\\
-3.662	-6.104\\
-6.104	-7.324\\
-7.324	-7.324\\
-7.324	-12.207\\
-12.207	-7.324\\
-7.324	-12.207\\
-12.207	-18.311\\
-18.311	-12.207\\
-12.207	-12.207\\
-12.207	-17.09\\
-17.09	-17.09\\
-17.09	-12.207\\
-12.207	-14.648\\
-14.648	-8.545\\
-8.545	-12.207\\
-12.207	-17.09\\
-17.09	-24.414\\
-24.414	-29.297\\
-29.297	-23.193\\
-23.193	-18.311\\
-18.311	-15.869\\
-15.869	-18.311\\
-18.311	-19.531\\
-19.531	-13.428\\
-13.428	-12.207\\
-12.207	-8.545\\
-8.545	-14.648\\
-14.648	-12.207\\
-12.207	-8.545\\
-8.545	-8.545\\
-8.545	-7.324\\
-7.324	-12.207\\
-12.207	-14.648\\
-14.648	-15.869\\
-15.869	-9.766\\
-9.766	-9.766\\
-9.766	-12.207\\
-12.207	-7.324\\
-7.324	-7.324\\
-7.324	-4.883\\
-4.883	-4.883\\
-4.883	-3.662\\
-3.662	-4.883\\
-4.883	-10.986\\
-10.986	-8.545\\
-8.545	-10.986\\
-10.986	-7.324\\
-7.324	-6.104\\
-6.104	-3.662\\
-3.662	-8.545\\
-8.545	-8.545\\
-8.545	-9.766\\
-9.766	-7.324\\
-7.324	-4.883\\
-4.883	-8.545\\
-8.545	-14.648\\
-14.648	-8.545\\
-8.545	-2.441\\
-2.441	-10.986\\
-10.986	-9.766\\
-9.766	-9.766\\
-9.766	-7.324\\
-7.324	-3.662\\
-3.662	-10.986\\
-10.986	-10.986\\
-10.986	-6.104\\
-6.104	-3.662\\
-3.662	-10.986\\
-10.986	-13.428\\
-13.428	-12.207\\
-12.207	-6.104\\
-6.104	-7.324\\
-7.324	-12.207\\
-12.207	-14.648\\
-14.648	-12.207\\
-12.207	-18.311\\
-18.311	-21.973\\
-21.973	-13.428\\
-13.428	-13.428\\
-13.428	-17.09\\
-17.09	-10.986\\
-10.986	-17.09\\
-17.09	-17.09\\
-17.09	-13.428\\
-13.428	-6.104\\
-6.104	-9.766\\
-9.766	-19.531\\
-19.531	-21.973\\
-21.973	-15.869\\
-15.869	-15.869\\
-15.869	-19.531\\
-19.531	-14.648\\
-14.648	-10.986\\
-10.986	-9.766\\
-9.766	-12.207\\
-12.207	-12.207\\
-12.207	-12.207\\
-12.207	-13.428\\
-13.428	-8.545\\
-8.545	-10.986\\
-10.986	-13.428\\
-13.428	-8.545\\
-8.545	-7.324\\
-7.324	-6.104\\
-6.104	-6.104\\
-6.104	-3.662\\
-3.662	-7.324\\
-7.324	-7.324\\
-7.324	-8.545\\
-8.545	-12.207\\
-12.207	-15.869\\
-15.869	-10.986\\
-10.986	-12.207\\
-12.207	-8.545\\
-8.545	-13.428\\
-13.428	-14.648\\
-14.648	-8.545\\
-8.545	-3.662\\
-3.662	-10.986\\
-10.986	-13.428\\
-13.428	-13.428\\
-13.428	-18.311\\
-18.311	-24.414\\
-24.414	-15.869\\
-15.869	-15.869\\
-15.869	-17.09\\
-17.09	-12.207\\
-12.207	-9.766\\
-9.766	-12.207\\
-12.207	-14.648\\
-14.648	-14.648\\
-14.648	-13.428\\
-13.428	-9.766\\
-9.766	-8.545\\
-8.545	-6.104\\
-6.104	-10.986\\
-10.986	-7.324\\
-7.324	-7.324\\
-7.324	-14.648\\
-14.648	-15.869\\
-15.869	-8.545\\
-8.545	-7.324\\
-7.324	-12.207\\
-12.207	-7.324\\
-7.324	-6.104\\
-6.104	-7.324\\
-7.324	-9.766\\
-9.766	-6.104\\
-6.104	-3.662\\
-3.662	-3.662\\
-3.662	-2.441\\
-2.441	-6.104\\
-6.104	-9.766\\
-9.766	-9.766\\
-9.766	-13.428\\
-13.428	-15.869\\
-15.869	-10.986\\
-10.986	-8.545\\
-8.545	-15.869\\
-15.869	-20.752\\
-20.752	-15.869\\
-15.869	-14.648\\
-14.648	-23.193\\
-23.193	-17.09\\
-17.09	-15.869\\
-15.869	-13.428\\
-13.428	-13.428\\
-13.428	-15.869\\
-15.869	-10.986\\
-10.986	-9.766\\
-9.766	-17.09\\
-17.09	-18.311\\
-18.311	-19.531\\
-19.531	-12.207\\
-12.207	-4.883\\
-4.883	-7.324\\
-7.324	-10.986\\
-10.986	-2.441\\
-2.441	-3.662\\
-3.662	-8.545\\
-8.545	-8.545\\
-8.545	-13.428\\
-13.428	-8.545\\
-8.545	-6.104\\
-6.104	-6.104\\
-6.104	-3.662\\
-3.662	-6.104\\
-6.104	-3.662\\
-3.662	-4.883\\
-4.883	-8.545\\
-8.545	-6.104\\
-6.104	-1.221\\
-1.221	-4.883\\
-4.883	-6.104\\
-6.104	-6.104\\
-6.104	-3.662\\
-3.662	-3.662\\
-3.662	-3.662\\
-3.662	-3.662\\
-3.662	-8.545\\
-8.545	-10.986\\
-10.986	-14.648\\
-14.648	-19.531\\
-19.531	-15.869\\
-15.869	-7.324\\
-7.324	-7.324\\
-7.324	-6.104\\
-6.104	-4.883\\
-4.883	-3.662\\
-3.662	-7.324\\
-7.324	-8.545\\
-8.545	-6.104\\
-6.104	-8.545\\
-8.545	-8.545\\
-8.545	-9.766\\
-9.766	-9.766\\
-9.766	-10.986\\
-10.986	-12.207\\
-12.207	-14.648\\
-14.648	-9.766\\
-9.766	-6.104\\
-6.104	-6.104\\
-6.104	-9.766\\
-9.766	-4.883\\
-4.883	-7.324\\
-7.324	-2.441\\
-2.441	-6.104\\
-6.104	-6.104\\
-6.104	-3.662\\
-3.662	-3.662\\
-3.662	-6.104\\
-6.104	-7.324\\
-7.324	-10.986\\
-10.986	-13.428\\
-13.428	-8.545\\
-8.545	-2.441\\
-2.441	-6.104\\
-6.104	-9.766\\
-9.766	-4.883\\
-4.883	-1.221\\
-1.221	-12.207\\
-12.207	-7.324\\
-7.324	-14.648\\
-14.648	-18.311\\
-18.311	-13.428\\
-13.428	-8.545\\
-8.545	-8.545\\
};
\addlegendentry{data1}

\addplot [color=mycolor2, line width=2.0pt]
  table[row sep=crcr]{%
-12.207	-11.6408886335848\\
-10.986	-10.4765136830149\\
-15.869	-15.1330598612565\\
-10.986	-10.4765136830149\\
-12.207	-11.6408886335848\\
-13.428	-12.8052635841548\\
-15.869	-15.1330598612565\\
-14.648	-13.9686849106865\\
-7.324	-6.98434245534327\\
-8.545	-8.1487174059132\\
-10.986	-10.4765136830149\\
-9.766	-9.31309235648312\\
-6.104	-5.82092112881149\\
-2.441	-2.32779627710171\\
-3.662	-3.49217122767164\\
-2.441	-2.32779627710171\\
-12.207	-11.6408886335848\\
-13.428	-12.8052635841548\\
-12.207	-11.6408886335848\\
-10.986	-10.4765136830149\\
-6.104	-5.82092112881149\\
-1.221	-1.16437495056993\\
-6.104	-5.82092112881149\\
-10.986	-10.4765136830149\\
-12.207	-11.6408886335848\\
-8.545	-8.1487174059132\\
-14.648	-13.9686849106865\\
-9.766	-9.31309235648312\\
-13.428	-12.8052635841548\\
-12.207	-11.6408886335848\\
-9.766	-9.31309235648312\\
-8.545	-8.1487174059132\\
-12.207	-11.6408886335848\\
-10.986	-10.4765136830149\\
-14.648	-13.9686849106865\\
-15.869	-15.1330598612565\\
-9.766	-9.31309235648312\\
-8.545	-8.1487174059132\\
-6.104	-5.82092112881149\\
-7.324	-6.98434245534327\\
-4.883	-4.65654617824156\\
-9.766	-9.31309235648312\\
-10.986	-10.4765136830149\\
-8.545	-8.1487174059132\\
-13.428	-12.8052635841548\\
-15.869	-15.1330598612565\\
-8.545	-8.1487174059132\\
-6.104	-5.82092112881149\\
-9.766	-9.31309235648312\\
-6.104	-5.82092112881149\\
-4.883	-4.65654617824156\\
-7.324	-6.98434245534327\\
-3.662	-3.49217122767164\\
-6.104	-5.82092112881149\\
-8.545	-8.1487174059132\\
-10.986	-10.4765136830149\\
-9.766	-9.31309235648312\\
-13.428	-12.8052635841548\\
-10.986	-10.4765136830149\\
-12.207	-11.6408886335848\\
-10.986	-10.4765136830149\\
-7.324	-6.98434245534327\\
-8.545	-8.1487174059132\\
-14.648	-13.9686849106865\\
-20.752	-19.789606039498\\
-19.531	-18.6252310889281\\
-13.428	-12.8052635841548\\
-15.869	-15.1330598612565\\
-24.414	-23.2817772671697\\
-23.193	-22.1174023165997\\
-15.869	-15.1330598612565\\
-21.973	-20.953980990068\\
-28.076	-26.7739484948413\\
-23.193	-22.1174023165997\\
-17.09	-16.2974348118264\\
-14.648	-13.9686849106865\\
-13.428	-12.8052635841548\\
-9.766	-9.31309235648312\\
-10.986	-10.4765136830149\\
-12.207	-11.6408886335848\\
-4.883	-4.65654617824156\\
-7.324	-6.98434245534327\\
-10.986	-10.4765136830149\\
-9.766	-9.31309235648312\\
-10.986	-10.4765136830149\\
-13.428	-12.8052635841548\\
-14.648	-13.9686849106865\\
-15.869	-15.1330598612565\\
-14.648	-13.9686849106865\\
-17.09	-16.2974348118264\\
-12.207	-11.6408886335848\\
-10.986	-10.4765136830149\\
-8.545	-8.1487174059132\\
-9.766	-9.31309235648312\\
-12.207	-11.6408886335848\\
-8.545	-8.1487174059132\\
-9.766	-9.31309235648312\\
-7.324	-6.98434245534327\\
-6.104	-5.82092112881149\\
-8.545	-8.1487174059132\\
-6.104	-5.82092112881149\\
-4.883	-4.65654617824156\\
-8.545	-8.1487174059132\\
-13.428	-12.8052635841548\\
-12.207	-11.6408886335848\\
-10.986	-10.4765136830149\\
-7.324	-6.98434245534327\\
-8.545	-8.1487174059132\\
-19.531	-18.6252310889281\\
-14.648	-13.9686849106865\\
-10.986	-10.4765136830149\\
-17.09	-16.2974348118264\\
-12.207	-11.6408886335848\\
-6.104	-5.82092112881149\\
-9.766	-9.31309235648312\\
-8.545	-8.1487174059132\\
-14.648	-13.9686849106865\\
-15.869	-15.1330598612565\\
-19.531	-18.6252310889281\\
-14.648	-13.9686849106865\\
-13.428	-12.8052635841548\\
-14.648	-13.9686849106865\\
-10.986	-10.4765136830149\\
-7.324	-6.98434245534327\\
-6.104	-5.82092112881149\\
-8.545	-8.1487174059132\\
-10.986	-10.4765136830149\\
-15.869	-15.1330598612565\\
-14.648	-13.9686849106865\\
-8.545	-8.1487174059132\\
-9.766	-9.31309235648312\\
-6.104	-5.82092112881149\\
-2.441	-2.32779627710171\\
-4.883	-4.65654617824156\\
-7.324	-6.98434245534327\\
-8.545	-8.1487174059132\\
-6.104	-5.82092112881149\\
-7.324	-6.98434245534327\\
-6.104	-5.82092112881149\\
-4.883	-4.65654617824156\\
-1.221	-1.16437495056993\\
-3.662	-3.49217122767164\\
-12.207	-11.6408886335848\\
-14.648	-13.9686849106865\\
-15.869	-15.1330598612565\\
-12.207	-11.6408886335848\\
-7.324	-6.98434245534327\\
-6.104	-5.82092112881149\\
-3.662	-3.49217122767164\\
-7.324	-6.98434245534327\\
-8.545	-8.1487174059132\\
-9.766	-9.31309235648312\\
-12.207	-11.6408886335848\\
-18.311	-17.4618097623963\\
-19.531	-18.6252310889281\\
-18.311	-17.4618097623963\\
-15.869	-15.1330598612565\\
-17.09	-16.2974348118264\\
-18.311	-17.4618097623963\\
-12.207	-11.6408886335848\\
-10.986	-10.4765136830149\\
-12.207	-11.6408886335848\\
-10.986	-10.4765136830149\\
-12.207	-11.6408886335848\\
-9.766	-9.31309235648312\\
-15.869	-15.1330598612565\\
-18.311	-17.4618097623963\\
-12.207	-11.6408886335848\\
-7.324	-6.98434245534327\\
-9.766	-9.31309235648312\\
-17.09	-16.2974348118264\\
-18.311	-17.4618097623963\\
-21.973	-20.953980990068\\
-20.752	-19.789606039498\\
-17.09	-16.2974348118264\\
-15.869	-15.1330598612565\\
-18.311	-17.4618097623963\\
-20.752	-19.789606039498\\
-17.09	-16.2974348118264\\
-8.545	-8.1487174059132\\
-14.648	-13.9686849106865\\
-12.207	-11.6408886335848\\
-7.324	-6.98434245534327\\
-10.986	-10.4765136830149\\
-9.766	-9.31309235648312\\
-8.545	-8.1487174059132\\
-7.324	-6.98434245534327\\
-8.545	-8.1487174059132\\
-9.766	-9.31309235648312\\
-13.428	-12.8052635841548\\
-14.648	-13.9686849106865\\
-7.324	-6.98434245534327\\
-8.545	-8.1487174059132\\
-12.207	-11.6408886335848\\
-17.09	-16.2974348118264\\
-15.869	-15.1330598612565\\
-8.545	-8.1487174059132\\
-9.766	-9.31309235648312\\
-8.545	-8.1487174059132\\
-7.324	-6.98434245534327\\
-6.104	-5.82092112881149\\
-8.545	-8.1487174059132\\
-9.766	-9.31309235648312\\
-12.207	-11.6408886335848\\
-14.648	-13.9686849106865\\
-15.869	-15.1330598612565\\
-10.986	-10.4765136830149\\
-13.428	-12.8052635841548\\
-17.09	-16.2974348118264\\
-12.207	-11.6408886335848\\
-3.662	-3.49217122767164\\
-9.766	-9.31309235648312\\
-8.545	-8.1487174059132\\
-9.766	-9.31309235648312\\
-8.545	-8.1487174059132\\
-10.986	-10.4765136830149\\
-7.324	-6.98434245534327\\
-9.766	-9.31309235648312\\
-6.104	-5.82092112881149\\
-8.545	-8.1487174059132\\
-6.104	-5.82092112881149\\
-3.662	-3.49217122767164\\
-6.104	-5.82092112881149\\
-4.883	-4.65654617824156\\
-10.986	-10.4765136830149\\
-12.207	-11.6408886335848\\
-18.311	-17.4618097623963\\
-10.986	-10.4765136830149\\
-4.883	-4.65654617824156\\
-6.104	-5.82092112881149\\
-8.545	-8.1487174059132\\
-6.104	-5.82092112881149\\
-8.545	-8.1487174059132\\
-7.324	-6.98434245534327\\
-13.428	-12.8052635841548\\
-9.766	-9.31309235648312\\
-14.648	-13.9686849106865\\
-10.986	-10.4765136830149\\
-9.766	-9.31309235648312\\
-6.104	-5.82092112881149\\
-3.662	-3.49217122767164\\
-1.221	-1.16437495056993\\
-2.441	-2.32779627710171\\
-8.545	-8.1487174059132\\
-7.324	-6.98434245534327\\
-9.766	-9.31309235648312\\
-15.869	-15.1330598612565\\
-10.986	-10.4765136830149\\
-9.766	-9.31309235648312\\
-13.428	-12.8052635841548\\
-14.648	-13.9686849106865\\
-12.207	-11.6408886335848\\
-6.104	-5.82092112881149\\
-2.441	-2.32779627710171\\
-4.883	-4.65654617824156\\
-6.104	-5.82092112881149\\
-3.662	-3.49217122767164\\
-4.883	-4.65654617824156\\
-9.766	-9.31309235648312\\
-7.324	-6.98434245534327\\
-10.986	-10.4765136830149\\
-7.324	-6.98434245534327\\
-4.883	-4.65654617824156\\
-9.766	-9.31309235648312\\
-12.207	-11.6408886335848\\
-9.766	-9.31309235648312\\
-10.986	-10.4765136830149\\
-7.324	-6.98434245534327\\
-10.986	-10.4765136830149\\
-14.648	-13.9686849106865\\
-13.428	-12.8052635841548\\
-8.545	-8.1487174059132\\
-14.648	-13.9686849106865\\
-13.428	-12.8052635841548\\
-10.986	-10.4765136830149\\
-12.207	-11.6408886335848\\
-18.311	-17.4618097623963\\
-17.09	-16.2974348118264\\
-10.986	-10.4765136830149\\
-13.428	-12.8052635841548\\
-9.766	-9.31309235648312\\
-12.207	-11.6408886335848\\
-14.648	-13.9686849106865\\
-9.766	-9.31309235648312\\
-8.545	-8.1487174059132\\
-10.986	-10.4765136830149\\
-18.311	-17.4618097623963\\
-14.648	-13.9686849106865\\
-13.428	-12.8052635841548\\
-9.766	-9.31309235648312\\
-10.986	-10.4765136830149\\
-9.766	-9.31309235648312\\
-6.104	-5.82092112881149\\
-4.883	-4.65654617824156\\
-3.662	-3.49217122767164\\
-7.324	-6.98434245534327\\
-10.986	-10.4765136830149\\
-9.766	-9.31309235648312\\
-6.104	-5.82092112881149\\
-7.324	-6.98434245534327\\
-9.766	-9.31309235648312\\
-6.104	-5.82092112881149\\
-9.766	-9.31309235648312\\
-6.104	-5.82092112881149\\
-7.324	-6.98434245534327\\
-12.207	-11.6408886335848\\
-8.545	-8.1487174059132\\
-4.883	-4.65654617824156\\
-6.104	-5.82092112881149\\
-7.324	-6.98434245534327\\
-6.104	-5.82092112881149\\
-7.324	-6.98434245534327\\
-14.648	-13.9686849106865\\
-9.766	-9.31309235648312\\
-17.09	-16.2974348118264\\
-20.752	-19.789606039498\\
-18.311	-17.4618097623963\\
-13.428	-12.8052635841548\\
-10.986	-10.4765136830149\\
-12.207	-11.6408886335848\\
-13.428	-12.8052635841548\\
-14.648	-13.9686849106865\\
-15.869	-15.1330598612565\\
-20.752	-19.789606039498\\
-24.414	-23.2817772671697\\
-15.869	-15.1330598612565\\
-20.752	-19.789606039498\\
-14.648	-13.9686849106865\\
-17.09	-16.2974348118264\\
-18.311	-17.4618097623963\\
-9.766	-9.31309235648312\\
-7.324	-6.98434245534327\\
-8.545	-8.1487174059132\\
-9.766	-9.31309235648312\\
-7.324	-6.98434245534327\\
-6.104	-5.82092112881149\\
-7.324	-6.98434245534327\\
-6.104	-5.82092112881149\\
-3.662	-3.49217122767164\\
-4.883	-4.65654617824156\\
-7.324	-6.98434245534327\\
-2.441	-2.32779627710171\\
-10.986	-10.4765136830149\\
-8.545	-8.1487174059132\\
-7.324	-6.98434245534327\\
-15.869	-15.1330598612565\\
-19.531	-18.6252310889281\\
-14.648	-13.9686849106865\\
-17.09	-16.2974348118264\\
-12.207	-11.6408886335848\\
-7.324	-6.98434245534327\\
-10.986	-10.4765136830149\\
-7.324	-6.98434245534327\\
-3.662	-3.49217122767164\\
-6.104	-5.82092112881149\\
-4.883	-4.65654617824156\\
-2.441	-2.32779627710171\\
-4.883	-4.65654617824156\\
-10.986	-10.4765136830149\\
-9.766	-9.31309235648312\\
-10.986	-10.4765136830149\\
-12.207	-11.6408886335848\\
-14.648	-13.9686849106865\\
-10.986	-10.4765136830149\\
-8.545	-8.1487174059132\\
-7.324	-6.98434245534327\\
-13.428	-12.8052635841548\\
-9.766	-9.31309235648312\\
-6.104	-5.82092112881149\\
-10.986	-10.4765136830149\\
-13.428	-12.8052635841548\\
-10.986	-10.4765136830149\\
-7.324	-6.98434245534327\\
-4.883	-4.65654617824156\\
-6.104	-5.82092112881149\\
-7.324	-6.98434245534327\\
-10.986	-10.4765136830149\\
-8.545	-8.1487174059132\\
-7.324	-6.98434245534327\\
-10.986	-10.4765136830149\\
-8.545	-8.1487174059132\\
-6.104	-5.82092112881149\\
-7.324	-6.98434245534327\\
-13.428	-12.8052635841548\\
-14.648	-13.9686849106865\\
-13.428	-12.8052635841548\\
-12.207	-11.6408886335848\\
-8.545	-8.1487174059132\\
-9.766	-9.31309235648312\\
-8.545	-8.1487174059132\\
-12.207	-11.6408886335848\\
-7.324	-6.98434245534327\\
-4.883	-4.65654617824156\\
-9.766	-9.31309235648312\\
-10.986	-10.4765136830149\\
-12.207	-11.6408886335848\\
-10.986	-10.4765136830149\\
-17.09	-16.2974348118264\\
-20.752	-19.789606039498\\
-23.193	-22.1174023165997\\
-15.869	-15.1330598612565\\
-9.766	-9.31309235648312\\
-6.104	-5.82092112881149\\
-9.766	-9.31309235648312\\
-6.104	-5.82092112881149\\
-9.766	-9.31309235648312\\
-6.104	-5.82092112881149\\
-7.324	-6.98434245534327\\
-8.545	-8.1487174059132\\
-10.986	-10.4765136830149\\
-13.428	-12.8052635841548\\
-14.648	-13.9686849106865\\
-19.531	-18.6252310889281\\
-14.648	-13.9686849106865\\
-13.428	-12.8052635841548\\
-10.986	-10.4765136830149\\
-12.207	-11.6408886335848\\
-9.766	-9.31309235648312\\
-10.986	-10.4765136830149\\
-9.766	-9.31309235648312\\
-7.324	-6.98434245534327\\
-12.207	-11.6408886335848\\
-13.428	-12.8052635841548\\
-8.545	-8.1487174059132\\
-9.766	-9.31309235648312\\
-18.311	-17.4618097623963\\
-12.207	-11.6408886335848\\
-17.09	-16.2974348118264\\
-14.648	-13.9686849106865\\
-10.986	-10.4765136830149\\
-7.324	-6.98434245534327\\
-12.207	-11.6408886335848\\
-17.09	-16.2974348118264\\
-20.752	-19.789606039498\\
-18.311	-17.4618097623963\\
-13.428	-12.8052635841548\\
-9.766	-9.31309235648312\\
-8.545	-8.1487174059132\\
-6.104	-5.82092112881149\\
-7.324	-6.98434245534327\\
-4.883	-4.65654617824156\\
-8.545	-8.1487174059132\\
-6.104	-5.82092112881149\\
-9.766	-9.31309235648312\\
-10.986	-10.4765136830149\\
-12.207	-11.6408886335848\\
-14.648	-13.9686849106865\\
-17.09	-16.2974348118264\\
-19.531	-18.6252310889281\\
-18.311	-17.4618097623963\\
-13.428	-12.8052635841548\\
-14.648	-13.9686849106865\\
-12.207	-11.6408886335848\\
-7.324	-6.98434245534327\\
-13.428	-12.8052635841548\\
-19.531	-18.6252310889281\\
-18.311	-17.4618097623963\\
-12.207	-11.6408886335848\\
-6.104	-5.82092112881149\\
-4.883	-4.65654617824156\\
-1.221	-1.16437495056993\\
2.441	2.32779627710171\\
-4.883	-4.65654617824156\\
-6.104	-5.82092112881149\\
-8.545	-8.1487174059132\\
-6.104	-5.82092112881149\\
-4.883	-4.65654617824156\\
-7.324	-6.98434245534327\\
-4.883	-4.65654617824156\\
-8.545	-8.1487174059132\\
-12.207	-11.6408886335848\\
-13.428	-12.8052635841548\\
-10.986	-10.4765136830149\\
-14.648	-13.9686849106865\\
-17.09	-16.2974348118264\\
-20.752	-19.789606039498\\
-14.648	-13.9686849106865\\
-10.986	-10.4765136830149\\
-18.311	-17.4618097623963\\
-12.207	-11.6408886335848\\
-8.545	-8.1487174059132\\
-4.883	-4.65654617824156\\
-2.441	-2.32779627710171\\
-6.104	-5.82092112881149\\
-8.545	-8.1487174059132\\
-7.324	-6.98434245534327\\
-6.104	-5.82092112881149\\
-3.662	-3.49217122767164\\
-4.883	-4.65654617824156\\
-6.104	-5.82092112881149\\
-2.441	-2.32779627710171\\
-7.324	-6.98434245534327\\
-9.766	-9.31309235648312\\
-13.428	-12.8052635841548\\
-15.869	-15.1330598612565\\
-10.986	-10.4765136830149\\
-18.311	-17.4618097623963\\
-23.193	-22.1174023165997\\
-21.973	-20.953980990068\\
-19.531	-18.6252310889281\\
-14.648	-13.9686849106865\\
-17.09	-16.2974348118264\\
-13.428	-12.8052635841548\\
-9.766	-9.31309235648312\\
-7.324	-6.98434245534327\\
-13.428	-12.8052635841548\\
-8.545	-8.1487174059132\\
-12.207	-11.6408886335848\\
-10.986	-10.4765136830149\\
-8.545	-8.1487174059132\\
-10.986	-10.4765136830149\\
-9.766	-9.31309235648312\\
-12.207	-11.6408886335848\\
-10.986	-10.4765136830149\\
-8.545	-8.1487174059132\\
-12.207	-11.6408886335848\\
-6.104	-5.82092112881149\\
-2.441	-2.32779627710171\\
-7.324	-6.98434245534327\\
-2.441	-2.32779627710171\\
-1.221	-1.16437495056993\\
-2.441	-2.32779627710171\\
-6.104	-5.82092112881149\\
-8.545	-8.1487174059132\\
-7.324	-6.98434245534327\\
-13.428	-12.8052635841548\\
-8.545	-8.1487174059132\\
-6.104	-5.82092112881149\\
-10.986	-10.4765136830149\\
-7.324	-6.98434245534327\\
-4.883	-4.65654617824156\\
-2.441	-2.32779627710171\\
-3.662	-3.49217122767164\\
-6.104	-5.82092112881149\\
-9.766	-9.31309235648312\\
-8.545	-8.1487174059132\\
-4.883	-4.65654617824156\\
-7.324	-6.98434245534327\\
-6.104	-5.82092112881149\\
-2.441	-2.32779627710171\\
-7.324	-6.98434245534327\\
-3.662	-3.49217122767164\\
-4.883	-4.65654617824156\\
-7.324	-6.98434245534327\\
-6.104	-5.82092112881149\\
-4.883	-4.65654617824156\\
-3.662	-3.49217122767164\\
-4.883	-4.65654617824156\\
-13.428	-12.8052635841548\\
-7.324	-6.98434245534327\\
-12.207	-11.6408886335848\\
-7.324	-6.98434245534327\\
-12.207	-11.6408886335848\\
-8.545	-8.1487174059132\\
-7.324	-6.98434245534327\\
-8.545	-8.1487174059132\\
-7.324	-6.98434245534327\\
-4.883	-4.65654617824156\\
-6.104	-5.82092112881149\\
-9.766	-9.31309235648312\\
-7.324	-6.98434245534327\\
-12.207	-11.6408886335848\\
-6.104	-5.82092112881149\\
-8.545	-8.1487174059132\\
-6.104	-5.82092112881149\\
-12.207	-11.6408886335848\\
-7.324	-6.98434245534327\\
-14.648	-13.9686849106865\\
-17.09	-16.2974348118264\\
-13.428	-12.8052635841548\\
-9.766	-9.31309235648312\\
-12.207	-11.6408886335848\\
-13.428	-12.8052635841548\\
-8.545	-8.1487174059132\\
-13.428	-12.8052635841548\\
-10.986	-10.4765136830149\\
-4.883	-4.65654617824156\\
-3.662	-3.49217122767164\\
-12.207	-11.6408886335848\\
-10.986	-10.4765136830149\\
-8.545	-8.1487174059132\\
-10.986	-10.4765136830149\\
-12.207	-11.6408886335848\\
-9.766	-9.31309235648312\\
-7.324	-6.98434245534327\\
-9.766	-9.31309235648312\\
-12.207	-11.6408886335848\\
-8.545	-8.1487174059132\\
-10.986	-10.4765136830149\\
-9.766	-9.31309235648312\\
-6.104	-5.82092112881149\\
-8.545	-8.1487174059132\\
-4.883	-4.65654617824156\\
-8.545	-8.1487174059132\\
-10.986	-10.4765136830149\\
-14.648	-13.9686849106865\\
-13.428	-12.8052635841548\\
-8.545	-8.1487174059132\\
-6.104	-5.82092112881149\\
-7.324	-6.98434245534327\\
-8.545	-8.1487174059132\\
-12.207	-11.6408886335848\\
-14.648	-13.9686849106865\\
-10.986	-10.4765136830149\\
-8.545	-8.1487174059132\\
-12.207	-11.6408886335848\\
-18.311	-17.4618097623963\\
-12.207	-11.6408886335848\\
-14.648	-13.9686849106865\\
-9.766	-9.31309235648312\\
-12.207	-11.6408886335848\\
-9.766	-9.31309235648312\\
-10.986	-10.4765136830149\\
-14.648	-13.9686849106865\\
-9.766	-9.31309235648312\\
-12.207	-11.6408886335848\\
-10.986	-10.4765136830149\\
-6.104	-5.82092112881149\\
-9.766	-9.31309235648312\\
-8.545	-8.1487174059132\\
-9.766	-9.31309235648312\\
-6.104	-5.82092112881149\\
-3.662	-3.49217122767164\\
-2.441	-2.32779627710171\\
-6.104	-5.82092112881149\\
-10.986	-10.4765136830149\\
-12.207	-11.6408886335848\\
-10.986	-10.4765136830149\\
-8.545	-8.1487174059132\\
-12.207	-11.6408886335848\\
-8.545	-8.1487174059132\\
-9.766	-9.31309235648312\\
-6.104	-5.82092112881149\\
-8.545	-8.1487174059132\\
-7.324	-6.98434245534327\\
-6.104	-5.82092112881149\\
-4.883	-4.65654617824156\\
-10.986	-10.4765136830149\\
-7.324	-6.98434245534327\\
-9.766	-9.31309235648312\\
-7.324	-6.98434245534327\\
-12.207	-11.6408886335848\\
-9.766	-9.31309235648312\\
-14.648	-13.9686849106865\\
-8.545	-8.1487174059132\\
-13.428	-12.8052635841548\\
-12.207	-11.6408886335848\\
-7.324	-6.98434245534327\\
-10.986	-10.4765136830149\\
-9.766	-9.31309235648312\\
-10.986	-10.4765136830149\\
-12.207	-11.6408886335848\\
-13.428	-12.8052635841548\\
-7.324	-6.98434245534327\\
-13.428	-12.8052635841548\\
-14.648	-13.9686849106865\\
-9.766	-9.31309235648312\\
-10.986	-10.4765136830149\\
-7.324	-6.98434245534327\\
-9.766	-9.31309235648312\\
-7.324	-6.98434245534327\\
-8.545	-8.1487174059132\\
-7.324	-6.98434245534327\\
-13.428	-12.8052635841548\\
-15.869	-15.1330598612565\\
-14.648	-13.9686849106865\\
-15.869	-15.1330598612565\\
-8.545	-8.1487174059132\\
-7.324	-6.98434245534327\\
-6.104	-5.82092112881149\\
-4.883	-4.65654617824156\\
-8.545	-8.1487174059132\\
-10.986	-10.4765136830149\\
-13.428	-12.8052635841548\\
-12.207	-11.6408886335848\\
-7.324	-6.98434245534327\\
-12.207	-11.6408886335848\\
-15.869	-15.1330598612565\\
-13.428	-12.8052635841548\\
-20.752	-19.789606039498\\
-17.09	-16.2974348118264\\
-10.986	-10.4765136830149\\
-6.104	-5.82092112881149\\
-7.324	-6.98434245534327\\
-10.986	-10.4765136830149\\
-13.428	-12.8052635841548\\
-17.09	-16.2974348118264\\
-12.207	-11.6408886335848\\
-10.986	-10.4765136830149\\
-6.104	-5.82092112881149\\
-9.766	-9.31309235648312\\
-7.324	-6.98434245534327\\
-4.883	-4.65654617824156\\
-3.662	-3.49217122767164\\
-6.104	-5.82092112881149\\
-10.986	-10.4765136830149\\
-8.545	-8.1487174059132\\
-9.766	-9.31309235648312\\
-8.545	-8.1487174059132\\
-12.207	-11.6408886335848\\
-7.324	-6.98434245534327\\
-12.207	-11.6408886335848\\
-13.428	-12.8052635841548\\
-8.545	-8.1487174059132\\
-10.986	-10.4765136830149\\
-8.545	-8.1487174059132\\
-10.986	-10.4765136830149\\
-15.869	-15.1330598612565\\
-10.986	-10.4765136830149\\
-8.545	-8.1487174059132\\
-9.766	-9.31309235648312\\
-10.986	-10.4765136830149\\
-7.324	-6.98434245534327\\
-8.545	-8.1487174059132\\
-13.428	-12.8052635841548\\
-14.648	-13.9686849106865\\
-15.869	-15.1330598612565\\
-20.752	-19.789606039498\\
-15.869	-15.1330598612565\\
-14.648	-13.9686849106865\\
-10.986	-10.4765136830149\\
-17.09	-16.2974348118264\\
-13.428	-12.8052635841548\\
-10.986	-10.4765136830149\\
-14.648	-13.9686849106865\\
-9.766	-9.31309235648312\\
-7.324	-6.98434245534327\\
-9.766	-9.31309235648312\\
-6.104	-5.82092112881149\\
-3.662	-3.49217122767164\\
-6.104	-5.82092112881149\\
-4.883	-4.65654617824156\\
-8.545	-8.1487174059132\\
-9.766	-9.31309235648312\\
-12.207	-11.6408886335848\\
-14.648	-13.9686849106865\\
-10.986	-10.4765136830149\\
-9.766	-9.31309235648312\\
-10.986	-10.4765136830149\\
-14.648	-13.9686849106865\\
-10.986	-10.4765136830149\\
-9.766	-9.31309235648312\\
-10.986	-10.4765136830149\\
-14.648	-13.9686849106865\\
-12.207	-11.6408886335848\\
-13.428	-12.8052635841548\\
-9.766	-9.31309235648312\\
-6.104	-5.82092112881149\\
-4.883	-4.65654617824156\\
-7.324	-6.98434245534327\\
-6.104	-5.82092112881149\\
-8.545	-8.1487174059132\\
-14.648	-13.9686849106865\\
-18.311	-17.4618097623963\\
-12.207	-11.6408886335848\\
-14.648	-13.9686849106865\\
-13.428	-12.8052635841548\\
-10.986	-10.4765136830149\\
-13.428	-12.8052635841548\\
-14.648	-13.9686849106865\\
-19.531	-18.6252310889281\\
-17.09	-16.2974348118264\\
-15.869	-15.1330598612565\\
-19.531	-18.6252310889281\\
-13.428	-12.8052635841548\\
-15.869	-15.1330598612565\\
-13.428	-12.8052635841548\\
-10.986	-10.4765136830149\\
-13.428	-12.8052635841548\\
-17.09	-16.2974348118264\\
-7.324	-6.98434245534327\\
-10.986	-10.4765136830149\\
-7.324	-6.98434245534327\\
-8.545	-8.1487174059132\\
-10.986	-10.4765136830149\\
-7.324	-6.98434245534327\\
-8.545	-8.1487174059132\\
-15.869	-15.1330598612565\\
-20.752	-19.789606039498\\
-21.973	-20.953980990068\\
-19.531	-18.6252310889281\\
-15.869	-15.1330598612565\\
-19.531	-18.6252310889281\\
-21.973	-20.953980990068\\
-17.09	-16.2974348118264\\
-18.311	-17.4618097623963\\
-17.09	-16.2974348118264\\
-12.207	-11.6408886335848\\
-9.766	-9.31309235648312\\
-13.428	-12.8052635841548\\
-10.986	-10.4765136830149\\
-6.104	-5.82092112881149\\
-9.766	-9.31309235648312\\
-8.545	-8.1487174059132\\
-9.766	-9.31309235648312\\
-12.207	-11.6408886335848\\
-8.545	-8.1487174059132\\
-7.324	-6.98434245534327\\
-8.545	-8.1487174059132\\
-10.986	-10.4765136830149\\
-9.766	-9.31309235648312\\
-14.648	-13.9686849106865\\
-13.428	-12.8052635841548\\
-9.766	-9.31309235648312\\
-13.428	-12.8052635841548\\
-10.986	-10.4765136830149\\
-8.545	-8.1487174059132\\
-7.324	-6.98434245534327\\
-4.883	-4.65654617824156\\
-6.104	-5.82092112881149\\
-8.545	-8.1487174059132\\
-6.104	-5.82092112881149\\
-4.883	-4.65654617824156\\
-7.324	-6.98434245534327\\
-12.207	-11.6408886335848\\
-14.648	-13.9686849106865\\
-13.428	-12.8052635841548\\
-10.986	-10.4765136830149\\
-4.883	-4.65654617824156\\
-6.104	-5.82092112881149\\
-7.324	-6.98434245534327\\
-8.545	-8.1487174059132\\
-10.986	-10.4765136830149\\
-14.648	-13.9686849106865\\
-9.766	-9.31309235648312\\
-14.648	-13.9686849106865\\
-13.428	-12.8052635841548\\
-6.104	-5.82092112881149\\
-8.545	-8.1487174059132\\
-7.324	-6.98434245534327\\
-10.986	-10.4765136830149\\
-17.09	-16.2974348118264\\
-12.207	-11.6408886335848\\
-9.766	-9.31309235648312\\
-8.545	-8.1487174059132\\
-10.986	-10.4765136830149\\
-12.207	-11.6408886335848\\
-19.531	-18.6252310889281\\
-15.869	-15.1330598612565\\
-17.09	-16.2974348118264\\
-12.207	-11.6408886335848\\
-7.324	-6.98434245534327\\
-8.545	-8.1487174059132\\
-4.883	-4.65654617824156\\
-6.104	-5.82092112881149\\
-8.545	-8.1487174059132\\
-3.662	-3.49217122767164\\
-8.545	-8.1487174059132\\
-13.428	-12.8052635841548\\
-15.869	-15.1330598612565\\
-20.752	-19.789606039498\\
-18.311	-17.4618097623963\\
-9.766	-9.31309235648312\\
-10.986	-10.4765136830149\\
-8.545	-8.1487174059132\\
-13.428	-12.8052635841548\\
-15.869	-15.1330598612565\\
-9.766	-9.31309235648312\\
-10.986	-10.4765136830149\\
-14.648	-13.9686849106865\\
-13.428	-12.8052635841548\\
-7.324	-6.98434245534327\\
-6.104	-5.82092112881149\\
-4.883	-4.65654617824156\\
-3.662	-3.49217122767164\\
-4.883	-4.65654617824156\\
-3.662	-3.49217122767164\\
-8.545	-8.1487174059132\\
-6.104	-5.82092112881149\\
-8.545	-8.1487174059132\\
-6.104	-5.82092112881149\\
-3.662	-3.49217122767164\\
-2.441	-2.32779627710171\\
-6.104	-5.82092112881149\\
-8.545	-8.1487174059132\\
-7.324	-6.98434245534327\\
-10.986	-10.4765136830149\\
-14.648	-13.9686849106865\\
-19.531	-18.6252310889281\\
-21.973	-20.953980990068\\
-20.752	-19.789606039498\\
-19.531	-18.6252310889281\\
-13.428	-12.8052635841548\\
-10.986	-10.4765136830149\\
-12.207	-11.6408886335848\\
-10.986	-10.4765136830149\\
-7.324	-6.98434245534327\\
-8.545	-8.1487174059132\\
-10.986	-10.4765136830149\\
-7.324	-6.98434245534327\\
-2.441	-2.32779627710171\\
-4.883	-4.65654617824156\\
-6.104	-5.82092112881149\\
-7.324	-6.98434245534327\\
-6.104	-5.82092112881149\\
-8.545	-8.1487174059132\\
-6.104	-5.82092112881149\\
-14.648	-13.9686849106865\\
-10.986	-10.4765136830149\\
-12.207	-11.6408886335848\\
-20.752	-19.789606039498\\
-15.869	-15.1330598612565\\
-13.428	-12.8052635841548\\
-8.545	-8.1487174059132\\
-7.324	-6.98434245534327\\
-6.104	-5.82092112881149\\
-10.986	-10.4765136830149\\
-7.324	-6.98434245534327\\
-10.986	-10.4765136830149\\
-8.545	-8.1487174059132\\
-10.986	-10.4765136830149\\
-4.883	-4.65654617824156\\
-7.324	-6.98434245534327\\
-9.766	-9.31309235648312\\
-3.662	-3.49217122767164\\
-6.104	-5.82092112881149\\
-2.441	-2.32779627710171\\
-3.662	-3.49217122767164\\
-2.441	-2.32779627710171\\
-6.104	-5.82092112881149\\
-7.324	-6.98434245534327\\
-10.986	-10.4765136830149\\
-7.324	-6.98434245534327\\
-12.207	-11.6408886335848\\
-14.648	-13.9686849106865\\
-9.766	-9.31309235648312\\
-10.986	-10.4765136830149\\
-9.766	-9.31309235648312\\
-2.441	-2.32779627710171\\
-9.766	-9.31309235648312\\
-8.545	-8.1487174059132\\
-12.207	-11.6408886335848\\
-13.428	-12.8052635841548\\
-15.869	-15.1330598612565\\
-10.986	-10.4765136830149\\
-8.545	-8.1487174059132\\
-7.324	-6.98434245534327\\
-8.545	-8.1487174059132\\
-6.104	-5.82092112881149\\
-4.883	-4.65654617824156\\
-3.662	-3.49217122767164\\
-4.883	-4.65654617824156\\
-3.662	-3.49217122767164\\
-8.545	-8.1487174059132\\
-4.883	-4.65654617824156\\
-7.324	-6.98434245534327\\
-14.648	-13.9686849106865\\
-6.104	-5.82092112881149\\
-9.766	-9.31309235648312\\
-17.09	-16.2974348118264\\
-9.766	-9.31309235648312\\
-6.104	-5.82092112881149\\
-9.766	-9.31309235648312\\
-10.986	-10.4765136830149\\
-8.545	-8.1487174059132\\
-3.662	-3.49217122767164\\
-6.104	-5.82092112881149\\
-1.221	-1.16437495056993\\
-2.441	-2.32779627710171\\
-6.104	-5.82092112881149\\
-7.324	-6.98434245534327\\
-6.104	-5.82092112881149\\
-8.545	-8.1487174059132\\
-7.324	-6.98434245534327\\
-3.662	-3.49217122767164\\
-1.221	-1.16437495056993\\
-6.104	-5.82092112881149\\
-4.883	-4.65654617824156\\
-7.324	-6.98434245534327\\
-10.986	-10.4765136830149\\
-7.324	-6.98434245534327\\
-8.545	-8.1487174059132\\
-9.766	-9.31309235648312\\
-12.207	-11.6408886335848\\
-14.648	-13.9686849106865\\
-17.09	-16.2974348118264\\
-12.207	-11.6408886335848\\
-10.986	-10.4765136830149\\
-12.207	-11.6408886335848\\
-13.428	-12.8052635841548\\
-18.311	-17.4618097623963\\
-17.09	-16.2974348118264\\
-13.428	-12.8052635841548\\
-12.207	-11.6408886335848\\
-13.428	-12.8052635841548\\
-10.986	-10.4765136830149\\
-12.207	-11.6408886335848\\
-13.428	-12.8052635841548\\
-12.207	-11.6408886335848\\
-10.986	-10.4765136830149\\
-9.766	-9.31309235648312\\
-7.324	-6.98434245534327\\
-12.207	-11.6408886335848\\
-8.545	-8.1487174059132\\
-6.104	-5.82092112881149\\
-8.545	-8.1487174059132\\
-3.662	-3.49217122767164\\
-7.324	-6.98434245534327\\
-3.662	-3.49217122767164\\
-6.104	-5.82092112881149\\
-4.883	-4.65654617824156\\
-2.441	-2.32779627710171\\
-6.104	-5.82092112881149\\
-3.662	-3.49217122767164\\
-8.545	-8.1487174059132\\
-3.662	-3.49217122767164\\
-8.545	-8.1487174059132\\
-4.883	-4.65654617824156\\
-7.324	-6.98434245534327\\
-8.545	-8.1487174059132\\
-7.324	-6.98434245534327\\
-13.428	-12.8052635841548\\
-4.883	-4.65654617824156\\
-8.545	-8.1487174059132\\
-10.986	-10.4765136830149\\
-6.104	-5.82092112881149\\
-7.324	-6.98434245534327\\
-12.207	-11.6408886335848\\
-3.662	-3.49217122767164\\
-6.104	-5.82092112881149\\
-8.545	-8.1487174059132\\
-6.104	-5.82092112881149\\
-9.766	-9.31309235648312\\
-1.221	-1.16437495056993\\
-3.662	-3.49217122767164\\
-4.883	-4.65654617824156\\
-7.324	-6.98434245534327\\
-3.662	-3.49217122767164\\
-8.545	-8.1487174059132\\
-10.986	-10.4765136830149\\
-9.766	-9.31309235648312\\
-8.545	-8.1487174059132\\
-14.648	-13.9686849106865\\
-10.986	-10.4765136830149\\
-18.311	-17.4618097623963\\
-13.428	-12.8052635841548\\
-17.09	-16.2974348118264\\
-13.428	-12.8052635841548\\
-17.09	-16.2974348118264\\
-13.428	-12.8052635841548\\
-4.883	-4.65654617824156\\
-7.324	-6.98434245534327\\
-12.207	-11.6408886335848\\
-10.986	-10.4765136830149\\
-8.545	-8.1487174059132\\
-4.883	-4.65654617824156\\
-6.104	-5.82092112881149\\
-9.766	-9.31309235648312\\
-15.869	-15.1330598612565\\
-13.428	-12.8052635841548\\
-9.766	-9.31309235648312\\
-7.324	-6.98434245534327\\
-12.207	-11.6408886335848\\
-15.869	-15.1330598612565\\
-14.648	-13.9686849106865\\
-13.428	-12.8052635841548\\
-19.531	-18.6252310889281\\
-17.09	-16.2974348118264\\
-19.531	-18.6252310889281\\
-20.752	-19.789606039498\\
-15.869	-15.1330598612565\\
-18.311	-17.4618097623963\\
-15.869	-15.1330598612565\\
-8.545	-8.1487174059132\\
-10.986	-10.4765136830149\\
-13.428	-12.8052635841548\\
-10.986	-10.4765136830149\\
-9.766	-9.31309235648312\\
-13.428	-12.8052635841548\\
-9.766	-9.31309235648312\\
-12.207	-11.6408886335848\\
-9.766	-9.31309235648312\\
-14.648	-13.9686849106865\\
-18.311	-17.4618097623963\\
-10.986	-10.4765136830149\\
-8.545	-8.1487174059132\\
-7.324	-6.98434245534327\\
-12.207	-11.6408886335848\\
-3.662	-3.49217122767164\\
-4.883	-4.65654617824156\\
-13.428	-12.8052635841548\\
-10.986	-10.4765136830149\\
-9.766	-9.31309235648312\\
-8.545	-8.1487174059132\\
-3.662	-3.49217122767164\\
-6.104	-5.82092112881149\\
-4.883	-4.65654617824156\\
-8.545	-8.1487174059132\\
-7.324	-6.98434245534327\\
-13.428	-12.8052635841548\\
-15.869	-15.1330598612565\\
-14.648	-13.9686849106865\\
-17.09	-16.2974348118264\\
-14.648	-13.9686849106865\\
-9.766	-9.31309235648312\\
-8.545	-8.1487174059132\\
-9.766	-9.31309235648312\\
-6.104	-5.82092112881149\\
-4.883	-4.65654617824156\\
-10.986	-10.4765136830149\\
-12.207	-11.6408886335848\\
-9.766	-9.31309235648312\\
-14.648	-13.9686849106865\\
-17.09	-16.2974348118264\\
-12.207	-11.6408886335848\\
-8.545	-8.1487174059132\\
-7.324	-6.98434245534327\\
-9.766	-9.31309235648312\\
-8.545	-8.1487174059132\\
-9.766	-9.31309235648312\\
-15.869	-15.1330598612565\\
-12.207	-11.6408886335848\\
-10.986	-10.4765136830149\\
-7.324	-6.98434245534327\\
-12.207	-11.6408886335848\\
-14.648	-13.9686849106865\\
-12.207	-11.6408886335848\\
-10.986	-10.4765136830149\\
-17.09	-16.2974348118264\\
-10.986	-10.4765136830149\\
-8.545	-8.1487174059132\\
-10.986	-10.4765136830149\\
-7.324	-6.98434245534327\\
-9.766	-9.31309235648312\\
-10.986	-10.4765136830149\\
-8.545	-8.1487174059132\\
-9.766	-9.31309235648312\\
-15.869	-15.1330598612565\\
-18.311	-17.4618097623963\\
-10.986	-10.4765136830149\\
-12.207	-11.6408886335848\\
-13.428	-12.8052635841548\\
-14.648	-13.9686849106865\\
-9.766	-9.31309235648312\\
-8.545	-8.1487174059132\\
-10.986	-10.4765136830149\\
-8.545	-8.1487174059132\\
-15.869	-15.1330598612565\\
-20.752	-19.789606039498\\
-23.193	-22.1174023165997\\
-21.973	-20.953980990068\\
-14.648	-13.9686849106865\\
-13.428	-12.8052635841548\\
-12.207	-11.6408886335848\\
-8.545	-8.1487174059132\\
-6.104	-5.82092112881149\\
-7.324	-6.98434245534327\\
-3.662	-3.49217122767164\\
-4.883	-4.65654617824156\\
-10.986	-10.4765136830149\\
-7.324	-6.98434245534327\\
-3.662	-3.49217122767164\\
-8.545	-8.1487174059132\\
-10.986	-10.4765136830149\\
-9.766	-9.31309235648312\\
-6.104	-5.82092112881149\\
-12.207	-11.6408886335848\\
-6.104	-5.82092112881149\\
-7.324	-6.98434245534327\\
-8.545	-8.1487174059132\\
-6.104	-5.82092112881149\\
-4.883	-4.65654617824156\\
-6.104	-5.82092112881149\\
-8.545	-8.1487174059132\\
-9.766	-9.31309235648312\\
-8.545	-8.1487174059132\\
-6.104	-5.82092112881149\\
-4.883	-4.65654617824156\\
-8.545	-8.1487174059132\\
-9.766	-9.31309235648312\\
-8.545	-8.1487174059132\\
-6.104	-5.82092112881149\\
-7.324	-6.98434245534327\\
-8.545	-8.1487174059132\\
-14.648	-13.9686849106865\\
-17.09	-16.2974348118264\\
-13.428	-12.8052635841548\\
-7.324	-6.98434245534327\\
-9.766	-9.31309235648312\\
-8.545	-8.1487174059132\\
-7.324	-6.98434245534327\\
-8.545	-8.1487174059132\\
-7.324	-6.98434245534327\\
-3.662	-3.49217122767164\\
-4.883	-4.65654617824156\\
-7.324	-6.98434245534327\\
-9.766	-9.31309235648312\\
-12.207	-11.6408886335848\\
-10.986	-10.4765136830149\\
-14.648	-13.9686849106865\\
-13.428	-12.8052635841548\\
-17.09	-16.2974348118264\\
-23.193	-22.1174023165997\\
-17.09	-16.2974348118264\\
-14.648	-13.9686849106865\\
-17.09	-16.2974348118264\\
-18.311	-17.4618097623963\\
-19.531	-18.6252310889281\\
-18.311	-17.4618097623963\\
-8.545	-8.1487174059132\\
-4.883	-4.65654617824156\\
-6.104	-5.82092112881149\\
-3.662	-3.49217122767164\\
-6.104	-5.82092112881149\\
-8.545	-8.1487174059132\\
-4.883	-4.65654617824156\\
-6.104	-5.82092112881149\\
-8.545	-8.1487174059132\\
-7.324	-6.98434245534327\\
-6.104	-5.82092112881149\\
-7.324	-6.98434245534327\\
-9.766	-9.31309235648312\\
-4.883	-4.65654617824156\\
-3.662	-3.49217122767164\\
-6.104	-5.82092112881149\\
-4.883	-4.65654617824156\\
-8.545	-8.1487174059132\\
-9.766	-9.31309235648312\\
-7.324	-6.98434245534327\\
-9.766	-9.31309235648312\\
-7.324	-6.98434245534327\\
-9.766	-9.31309235648312\\
-14.648	-13.9686849106865\\
-7.324	-6.98434245534327\\
-10.986	-10.4765136830149\\
-14.648	-13.9686849106865\\
-10.986	-10.4765136830149\\
-12.207	-11.6408886335848\\
-13.428	-12.8052635841548\\
-12.207	-11.6408886335848\\
-13.428	-12.8052635841548\\
-15.869	-15.1330598612565\\
-13.428	-12.8052635841548\\
-21.973	-20.953980990068\\
-20.752	-19.789606039498\\
-12.207	-11.6408886335848\\
-8.545	-8.1487174059132\\
-12.207	-11.6408886335848\\
-13.428	-12.8052635841548\\
-15.869	-15.1330598612565\\
-10.986	-10.4765136830149\\
-9.766	-9.31309235648312\\
-12.207	-11.6408886335848\\
-9.766	-9.31309235648312\\
-3.662	-3.49217122767164\\
-9.766	-9.31309235648312\\
-12.207	-11.6408886335848\\
-10.986	-10.4765136830149\\
-14.648	-13.9686849106865\\
-13.428	-12.8052635841548\\
-8.545	-8.1487174059132\\
-12.207	-11.6408886335848\\
-15.869	-15.1330598612565\\
-19.531	-18.6252310889281\\
-20.752	-19.789606039498\\
-18.311	-17.4618097623963\\
-17.09	-16.2974348118264\\
-10.986	-10.4765136830149\\
-8.545	-8.1487174059132\\
-13.428	-12.8052635841548\\
-19.531	-18.6252310889281\\
-10.986	-10.4765136830149\\
-12.207	-11.6408886335848\\
-21.973	-20.953980990068\\
-19.531	-18.6252310889281\\
-9.766	-9.31309235648312\\
-6.104	-5.82092112881149\\
-9.766	-9.31309235648312\\
-10.986	-10.4765136830149\\
-7.324	-6.98434245534327\\
-4.883	-4.65654617824156\\
-6.104	-5.82092112881149\\
-8.545	-8.1487174059132\\
-7.324	-6.98434245534327\\
-6.104	-5.82092112881149\\
-8.545	-8.1487174059132\\
-9.766	-9.31309235648312\\
-12.207	-11.6408886335848\\
-14.648	-13.9686849106865\\
-9.766	-9.31309235648312\\
-12.207	-11.6408886335848\\
-9.766	-9.31309235648312\\
-8.545	-8.1487174059132\\
-7.324	-6.98434245534327\\
-8.545	-8.1487174059132\\
-9.766	-9.31309235648312\\
-6.104	-5.82092112881149\\
-3.662	-3.49217122767164\\
-4.883	-4.65654617824156\\
-7.324	-6.98434245534327\\
-6.104	-5.82092112881149\\
-10.986	-10.4765136830149\\
-8.545	-8.1487174059132\\
-9.766	-9.31309235648312\\
-6.104	-5.82092112881149\\
-4.883	-4.65654617824156\\
-7.324	-6.98434245534327\\
-10.986	-10.4765136830149\\
-9.766	-9.31309235648312\\
-10.986	-10.4765136830149\\
-7.324	-6.98434245534327\\
-8.545	-8.1487174059132\\
-7.324	-6.98434245534327\\
-12.207	-11.6408886335848\\
-15.869	-15.1330598612565\\
-12.207	-11.6408886335848\\
-15.869	-15.1330598612565\\
-13.428	-12.8052635841548\\
-8.545	-8.1487174059132\\
-7.324	-6.98434245534327\\
-4.883	-4.65654617824156\\
-3.662	-3.49217122767164\\
-10.986	-10.4765136830149\\
-6.104	-5.82092112881149\\
-4.883	-4.65654617824156\\
-6.104	-5.82092112881149\\
-8.545	-8.1487174059132\\
-7.324	-6.98434245534327\\
-6.104	-5.82092112881149\\
-4.883	-4.65654617824156\\
-6.104	-5.82092112881149\\
-10.986	-10.4765136830149\\
-7.324	-6.98434245534327\\
-4.883	-4.65654617824156\\
-9.766	-9.31309235648312\\
-8.545	-8.1487174059132\\
-9.766	-9.31309235648312\\
-10.986	-10.4765136830149\\
-13.428	-12.8052635841548\\
-12.207	-11.6408886335848\\
-8.545	-8.1487174059132\\
-7.324	-6.98434245534327\\
-6.104	-5.82092112881149\\
-9.766	-9.31309235648312\\
-6.104	-5.82092112881149\\
-8.545	-8.1487174059132\\
-12.207	-11.6408886335848\\
-9.766	-9.31309235648312\\
-10.986	-10.4765136830149\\
-15.869	-15.1330598612565\\
-10.986	-10.4765136830149\\
-14.648	-13.9686849106865\\
-12.207	-11.6408886335848\\
-15.869	-15.1330598612565\\
-13.428	-12.8052635841548\\
-21.973	-20.953980990068\\
-15.869	-15.1330598612565\\
-19.531	-18.6252310889281\\
-21.973	-20.953980990068\\
-20.752	-19.789606039498\\
-23.193	-22.1174023165997\\
-21.973	-20.953980990068\\
-25.635	-24.4461522177396\\
-20.752	-19.789606039498\\
-13.428	-12.8052635841548\\
-9.766	-9.31309235648312\\
-12.207	-11.6408886335848\\
-8.545	-8.1487174059132\\
-7.324	-6.98434245534327\\
-12.207	-11.6408886335848\\
-14.648	-13.9686849106865\\
-9.766	-9.31309235648312\\
-7.324	-6.98434245534327\\
-9.766	-9.31309235648312\\
-7.324	-6.98434245534327\\
-8.545	-8.1487174059132\\
-6.104	-5.82092112881149\\
-9.766	-9.31309235648312\\
-4.883	-4.65654617824156\\
-1.221	-1.16437495056993\\
-3.662	-3.49217122767164\\
-2.441	-2.32779627710171\\
-3.662	-3.49217122767164\\
-4.883	-4.65654617824156\\
-3.662	-3.49217122767164\\
-8.545	-8.1487174059132\\
-9.766	-9.31309235648312\\
-6.104	-5.82092112881149\\
-12.207	-11.6408886335848\\
-14.648	-13.9686849106865\\
-8.545	-8.1487174059132\\
-2.441	-2.32779627710171\\
-3.662	-3.49217122767164\\
-2.441	-2.32779627710171\\
-6.104	-5.82092112881149\\
-7.324	-6.98434245534327\\
-8.545	-8.1487174059132\\
-3.662	-3.49217122767164\\
-7.324	-6.98434245534327\\
-14.648	-13.9686849106865\\
-9.766	-9.31309235648312\\
-7.324	-6.98434245534327\\
-12.207	-11.6408886335848\\
-13.428	-12.8052635841548\\
-14.648	-13.9686849106865\\
-7.324	-6.98434245534327\\
-4.883	-4.65654617824156\\
-6.104	-5.82092112881149\\
-3.662	-3.49217122767164\\
-8.545	-8.1487174059132\\
-14.648	-13.9686849106865\\
-10.986	-10.4765136830149\\
-6.104	-5.82092112881149\\
-3.662	-3.49217122767164\\
-4.883	-4.65654617824156\\
-7.324	-6.98434245534327\\
-4.883	-4.65654617824156\\
-3.662	-3.49217122767164\\
-6.104	-5.82092112881149\\
-12.207	-11.6408886335848\\
-9.766	-9.31309235648312\\
-10.986	-10.4765136830149\\
-12.207	-11.6408886335848\\
-10.986	-10.4765136830149\\
-12.207	-11.6408886335848\\
-10.986	-10.4765136830149\\
-13.428	-12.8052635841548\\
-18.311	-17.4618097623963\\
-15.869	-15.1330598612565\\
-7.324	-6.98434245534327\\
-4.883	-4.65654617824156\\
-8.545	-8.1487174059132\\
-14.648	-13.9686849106865\\
-10.986	-10.4765136830149\\
-4.883	-4.65654617824156\\
-10.986	-10.4765136830149\\
-9.766	-9.31309235648312\\
-17.09	-16.2974348118264\\
-20.752	-19.789606039498\\
-19.531	-18.6252310889281\\
-15.869	-15.1330598612565\\
-14.648	-13.9686849106865\\
-7.324	-6.98434245534327\\
-9.766	-9.31309235648312\\
-8.545	-8.1487174059132\\
-10.986	-10.4765136830149\\
-12.207	-11.6408886335848\\
-7.324	-6.98434245534327\\
-10.986	-10.4765136830149\\
-8.545	-8.1487174059132\\
-6.104	-5.82092112881149\\
-3.662	-3.49217122767164\\
-4.883	-4.65654617824156\\
-3.662	-3.49217122767164\\
-6.104	-5.82092112881149\\
-9.766	-9.31309235648312\\
-6.104	-5.82092112881149\\
-13.428	-12.8052635841548\\
-6.104	-5.82092112881149\\
-3.662	-3.49217122767164\\
-1.221	-1.16437495056993\\
-6.104	-5.82092112881149\\
-10.986	-10.4765136830149\\
-14.648	-13.9686849106865\\
-21.973	-20.953980990068\\
-23.193	-22.1174023165997\\
-24.414	-23.2817772671697\\
-17.09	-16.2974348118264\\
-18.311	-17.4618097623963\\
-20.752	-19.789606039498\\
-18.311	-17.4618097623963\\
-17.09	-16.2974348118264\\
-18.311	-17.4618097623963\\
-20.752	-19.789606039498\\
-28.076	-26.7739484948413\\
-19.531	-18.6252310889281\\
-12.207	-11.6408886335848\\
-7.324	-6.98434245534327\\
-6.104	-5.82092112881149\\
-4.883	-4.65654617824156\\
-12.207	-11.6408886335848\\
-7.324	-6.98434245534327\\
-10.986	-10.4765136830149\\
-7.324	-6.98434245534327\\
-12.207	-11.6408886335848\\
-6.104	-5.82092112881149\\
-4.883	-4.65654617824156\\
-6.104	-5.82092112881149\\
-8.545	-8.1487174059132\\
-13.428	-12.8052635841548\\
-19.531	-18.6252310889281\\
-18.311	-17.4618097623963\\
-23.193	-22.1174023165997\\
-29.297	-27.9383234454112\\
-21.973	-20.953980990068\\
-14.648	-13.9686849106865\\
-9.766	-9.31309235648312\\
-6.104	-5.82092112881149\\
-10.986	-10.4765136830149\\
-8.545	-8.1487174059132\\
-9.766	-9.31309235648312\\
-7.324	-6.98434245534327\\
-8.545	-8.1487174059132\\
-7.324	-6.98434245534327\\
-9.766	-9.31309235648312\\
-10.986	-10.4765136830149\\
-7.324	-6.98434245534327\\
-6.104	-5.82092112881149\\
-3.662	-3.49217122767164\\
-1.221	-1.16437495056993\\
-3.662	-3.49217122767164\\
-6.104	-5.82092112881149\\
-7.324	-6.98434245534327\\
-12.207	-11.6408886335848\\
-7.324	-6.98434245534327\\
-12.207	-11.6408886335848\\
-18.311	-17.4618097623963\\
-12.207	-11.6408886335848\\
-17.09	-16.2974348118264\\
-12.207	-11.6408886335848\\
-14.648	-13.9686849106865\\
-8.545	-8.1487174059132\\
-12.207	-11.6408886335848\\
-17.09	-16.2974348118264\\
-24.414	-23.2817772671697\\
-29.297	-27.9383234454112\\
-23.193	-22.1174023165997\\
-18.311	-17.4618097623963\\
-15.869	-15.1330598612565\\
-18.311	-17.4618097623963\\
-19.531	-18.6252310889281\\
-13.428	-12.8052635841548\\
-12.207	-11.6408886335848\\
-8.545	-8.1487174059132\\
-14.648	-13.9686849106865\\
-12.207	-11.6408886335848\\
-8.545	-8.1487174059132\\
-7.324	-6.98434245534327\\
-12.207	-11.6408886335848\\
-14.648	-13.9686849106865\\
-15.869	-15.1330598612565\\
-9.766	-9.31309235648312\\
-12.207	-11.6408886335848\\
-7.324	-6.98434245534327\\
-4.883	-4.65654617824156\\
-3.662	-3.49217122767164\\
-4.883	-4.65654617824156\\
-10.986	-10.4765136830149\\
-8.545	-8.1487174059132\\
-10.986	-10.4765136830149\\
-7.324	-6.98434245534327\\
-6.104	-5.82092112881149\\
-3.662	-3.49217122767164\\
-8.545	-8.1487174059132\\
-9.766	-9.31309235648312\\
-7.324	-6.98434245534327\\
-4.883	-4.65654617824156\\
-8.545	-8.1487174059132\\
-14.648	-13.9686849106865\\
-8.545	-8.1487174059132\\
-2.441	-2.32779627710171\\
-10.986	-10.4765136830149\\
-9.766	-9.31309235648312\\
-7.324	-6.98434245534327\\
-3.662	-3.49217122767164\\
-10.986	-10.4765136830149\\
-6.104	-5.82092112881149\\
-3.662	-3.49217122767164\\
-10.986	-10.4765136830149\\
-13.428	-12.8052635841548\\
-12.207	-11.6408886335848\\
-6.104	-5.82092112881149\\
-7.324	-6.98434245534327\\
-12.207	-11.6408886335848\\
-14.648	-13.9686849106865\\
-12.207	-11.6408886335848\\
-18.311	-17.4618097623963\\
-21.973	-20.953980990068\\
-13.428	-12.8052635841548\\
-17.09	-16.2974348118264\\
-10.986	-10.4765136830149\\
-17.09	-16.2974348118264\\
-13.428	-12.8052635841548\\
-6.104	-5.82092112881149\\
-9.766	-9.31309235648312\\
-19.531	-18.6252310889281\\
-21.973	-20.953980990068\\
-15.869	-15.1330598612565\\
-19.531	-18.6252310889281\\
-14.648	-13.9686849106865\\
-10.986	-10.4765136830149\\
-9.766	-9.31309235648312\\
-12.207	-11.6408886335848\\
-13.428	-12.8052635841548\\
-8.545	-8.1487174059132\\
-10.986	-10.4765136830149\\
-13.428	-12.8052635841548\\
-8.545	-8.1487174059132\\
-7.324	-6.98434245534327\\
-6.104	-5.82092112881149\\
-3.662	-3.49217122767164\\
-7.324	-6.98434245534327\\
-8.545	-8.1487174059132\\
-12.207	-11.6408886335848\\
-15.869	-15.1330598612565\\
-10.986	-10.4765136830149\\
-12.207	-11.6408886335848\\
-8.545	-8.1487174059132\\
-13.428	-12.8052635841548\\
-14.648	-13.9686849106865\\
-8.545	-8.1487174059132\\
-3.662	-3.49217122767164\\
-10.986	-10.4765136830149\\
-13.428	-12.8052635841548\\
-18.311	-17.4618097623963\\
-24.414	-23.2817772671697\\
-15.869	-15.1330598612565\\
-17.09	-16.2974348118264\\
-12.207	-11.6408886335848\\
-9.766	-9.31309235648312\\
-12.207	-11.6408886335848\\
-14.648	-13.9686849106865\\
-13.428	-12.8052635841548\\
-9.766	-9.31309235648312\\
-8.545	-8.1487174059132\\
-6.104	-5.82092112881149\\
-10.986	-10.4765136830149\\
-7.324	-6.98434245534327\\
-14.648	-13.9686849106865\\
-15.869	-15.1330598612565\\
-8.545	-8.1487174059132\\
-7.324	-6.98434245534327\\
-12.207	-11.6408886335848\\
-7.324	-6.98434245534327\\
-6.104	-5.82092112881149\\
-7.324	-6.98434245534327\\
-9.766	-9.31309235648312\\
-6.104	-5.82092112881149\\
-3.662	-3.49217122767164\\
-2.441	-2.32779627710171\\
-6.104	-5.82092112881149\\
-9.766	-9.31309235648312\\
-13.428	-12.8052635841548\\
-15.869	-15.1330598612565\\
-10.986	-10.4765136830149\\
-8.545	-8.1487174059132\\
-15.869	-15.1330598612565\\
-20.752	-19.789606039498\\
-15.869	-15.1330598612565\\
-14.648	-13.9686849106865\\
-23.193	-22.1174023165997\\
-17.09	-16.2974348118264\\
-15.869	-15.1330598612565\\
-13.428	-12.8052635841548\\
-15.869	-15.1330598612565\\
-10.986	-10.4765136830149\\
-9.766	-9.31309235648312\\
-17.09	-16.2974348118264\\
-18.311	-17.4618097623963\\
-19.531	-18.6252310889281\\
-12.207	-11.6408886335848\\
-4.883	-4.65654617824156\\
-7.324	-6.98434245534327\\
-10.986	-10.4765136830149\\
-2.441	-2.32779627710171\\
-3.662	-3.49217122767164\\
-8.545	-8.1487174059132\\
-13.428	-12.8052635841548\\
-8.545	-8.1487174059132\\
-6.104	-5.82092112881149\\
-3.662	-3.49217122767164\\
-6.104	-5.82092112881149\\
-3.662	-3.49217122767164\\
-4.883	-4.65654617824156\\
-8.545	-8.1487174059132\\
-6.104	-5.82092112881149\\
-1.221	-1.16437495056993\\
-4.883	-4.65654617824156\\
-6.104	-5.82092112881149\\
-3.662	-3.49217122767164\\
-8.545	-8.1487174059132\\
-10.986	-10.4765136830149\\
-14.648	-13.9686849106865\\
-19.531	-18.6252310889281\\
-15.869	-15.1330598612565\\
-7.324	-6.98434245534327\\
-6.104	-5.82092112881149\\
-4.883	-4.65654617824156\\
-3.662	-3.49217122767164\\
-7.324	-6.98434245534327\\
-8.545	-8.1487174059132\\
-6.104	-5.82092112881149\\
-8.545	-8.1487174059132\\
-9.766	-9.31309235648312\\
-10.986	-10.4765136830149\\
-12.207	-11.6408886335848\\
-14.648	-13.9686849106865\\
-9.766	-9.31309235648312\\
-6.104	-5.82092112881149\\
-9.766	-9.31309235648312\\
-4.883	-4.65654617824156\\
-7.324	-6.98434245534327\\
-2.441	-2.32779627710171\\
-6.104	-5.82092112881149\\
-3.662	-3.49217122767164\\
-6.104	-5.82092112881149\\
-7.324	-6.98434245534327\\
-10.986	-10.4765136830149\\
-13.428	-12.8052635841548\\
-8.545	-8.1487174059132\\
-2.441	-2.32779627710171\\
-6.104	-5.82092112881149\\
-9.766	-9.31309235648312\\
-4.883	-4.65654617824156\\
-1.221	-1.16437495056993\\
-12.207	-11.6408886335848\\
-7.324	-6.98434245534327\\
-14.648	-13.9686849106865\\
-18.311	-17.4618097623963\\
-13.428	-12.8052635841548\\
-8.545	-8.1487174059132\\
};
\addlegendentry{data2}

\end{axis}

\begin{axis}[%
width=4.927cm,
height=3.484cm,
at={(0cm,4.839cm)},
scale only axis,
xmin=-404.053,
xmax=0,
xlabel style={font=\color{white!15!black}},
xlabel={y(t-1)},
ymin=-500,
ymax=0,
ylabel style={font=\color{white!15!black}},
ylabel={y(t)},
axis background/.style={fill=white},
title={C6, R = 0.7809},
axis x line*=bottom,
axis y line*=left,
legend style={legend cell align=left, align=left, draw=white!15!black}
]
\addplot[only marks, mark=*, mark options={}, mark size=1.5000pt, color=mycolor1, fill=mycolor1] table[row sep=crcr]{%
x	y\\
-122.07	-140.381\\
-140.381	-178.223\\
-178.223	-147.705\\
-147.705	-139.16\\
-139.16	-192.871\\
-192.871	-185.547\\
-185.547	-219.727\\
-219.727	-169.678\\
-169.678	-86.67\\
-86.67	-107.422\\
-107.422	-102.539\\
-102.539	-126.953\\
-126.953	-103.76\\
-103.76	-41.504\\
-41.504	-31.738\\
-31.738	-28.076\\
-28.076	-89.111\\
-89.111	-157.471\\
-157.471	-167.236\\
-167.236	-174.561\\
-174.561	-124.512\\
-124.512	-74.463\\
-74.463	-158.691\\
-158.691	-111.084\\
-111.084	-83.008\\
-83.008	-142.822\\
-142.822	-123.291\\
-123.291	-103.76\\
-103.76	-173.34\\
-173.34	-164.795\\
-164.795	-124.512\\
-124.512	-180.664\\
-180.664	-179.443\\
-179.443	-142.822\\
-142.822	-117.188\\
-117.188	-89.111\\
-89.111	-108.643\\
-108.643	-137.939\\
-137.939	-130.615\\
-130.615	-136.719\\
-136.719	-140.381\\
-140.381	-151.367\\
-151.367	-213.623\\
-213.623	-191.65\\
-191.65	-126.953\\
-126.953	-115.967\\
-115.967	-64.697\\
-64.697	-69.58\\
-69.58	-96.436\\
-96.436	-74.463\\
-74.463	-78.125\\
-78.125	-111.084\\
-111.084	-128.174\\
-128.174	-119.629\\
-119.629	-190.43\\
-190.43	-139.16\\
-139.16	-81.787\\
-81.787	-63.477\\
-63.477	-85.449\\
-85.449	-101.318\\
-101.318	-51.27\\
-51.27	-58.594\\
-58.594	-81.787\\
-81.787	-65.918\\
-65.918	-56.152\\
-56.152	-74.463\\
-74.463	-86.67\\
-86.67	-139.16\\
-139.16	-130.615\\
-130.615	-163.574\\
-163.574	-146.484\\
-146.484	-170.898\\
-170.898	-139.16\\
-139.16	-133.057\\
-133.057	-115.967\\
-115.967	-111.084\\
-111.084	-111.084\\
-111.084	-196.533\\
-196.533	-280.762\\
-280.762	-291.748\\
-291.748	-280.762\\
-280.762	-175.781\\
-175.781	-240.479\\
-240.479	-303.955\\
-303.955	-317.383\\
-317.383	-239.258\\
-239.258	-313.721\\
-313.721	-404.053\\
-404.053	-289.307\\
-289.307	-238.037\\
-238.037	-158.691\\
-158.691	-122.07\\
-122.07	-102.539\\
-102.539	-129.395\\
-129.395	-92.773\\
-92.773	-61.035\\
-61.035	-59.814\\
-59.814	-75.684\\
-75.684	-118.408\\
-118.408	-107.422\\
-107.422	-111.084\\
-111.084	-146.484\\
-146.484	-152.588\\
-152.588	-207.52\\
-207.52	-167.236\\
-167.236	-208.74\\
-208.74	-148.926\\
-148.926	-130.615\\
-130.615	-151.367\\
-151.367	-108.643\\
-108.643	-98.877\\
-98.877	-122.07\\
-122.07	-123.291\\
-123.291	-85.449\\
-85.449	-95.215\\
-95.215	-67.139\\
-67.139	-76.904\\
-76.904	-81.787\\
-81.787	-57.373\\
-57.373	-75.684\\
-75.684	-93.994\\
-93.994	-151.367\\
-151.367	-152.588\\
-152.588	-170.898\\
-170.898	-97.656\\
-97.656	-140.381\\
-140.381	-211.182\\
-211.182	-161.133\\
-161.133	-186.768\\
-186.768	-191.65\\
-191.65	-112.305\\
-112.305	-75.684\\
-75.684	-107.422\\
-107.422	-123.291\\
-123.291	-205.078\\
-205.078	-252.686\\
-252.686	-252.686\\
-252.686	-184.326\\
-184.326	-189.209\\
-189.209	-167.236\\
-167.236	-163.574\\
-163.574	-162.354\\
-162.354	-108.643\\
-108.643	-96.436\\
-96.436	-86.67\\
-86.67	-78.125\\
-78.125	-87.891\\
-87.891	-133.057\\
-133.057	-203.857\\
-203.857	-161.133\\
-161.133	-109.863\\
-109.863	-92.773\\
-92.773	-89.111\\
-89.111	-52.49\\
-52.49	-41.504\\
-41.504	-47.607\\
-47.607	-90.332\\
-90.332	-67.139\\
-67.139	-78.125\\
-78.125	-85.449\\
-85.449	-80.566\\
-80.566	-54.932\\
-54.932	-37.842\\
-37.842	-30.518\\
-30.518	-54.932\\
-54.932	-119.629\\
-119.629	-172.119\\
-172.119	-194.092\\
-194.092	-136.719\\
-136.719	-91.553\\
-91.553	-64.697\\
-64.697	-43.945\\
-43.945	-98.877\\
-98.877	-91.553\\
-91.553	-137.939\\
-137.939	-168.457\\
-168.457	-275.879\\
-275.879	-316.162\\
-316.162	-260.01\\
-260.01	-249.023\\
-249.023	-159.912\\
-159.912	-185.547\\
-185.547	-195.313\\
-195.313	-212.402\\
-212.402	-158.691\\
-158.691	-144.043\\
-144.043	-142.822\\
-142.822	-129.395\\
-129.395	-156.25\\
-156.25	-109.863\\
-109.863	-195.313\\
-195.313	-234.375\\
-234.375	-175.781\\
-175.781	-104.98\\
-104.98	-111.084\\
-111.084	-192.871\\
-192.871	-234.375\\
-234.375	-289.307\\
-289.307	-303.955\\
-303.955	-302.734\\
-302.734	-228.271\\
-228.271	-205.078\\
-205.078	-235.596\\
-235.596	-275.879\\
-275.879	-170.898\\
-170.898	-104.98\\
-104.98	-152.588\\
-152.588	-122.07\\
-122.07	-79.346\\
-79.346	-109.863\\
-109.863	-122.07\\
-122.07	-95.215\\
-95.215	-75.684\\
-75.684	-101.318\\
-101.318	-80.566\\
-80.566	-115.967\\
-115.967	-162.354\\
-162.354	-147.705\\
-147.705	-100.098\\
-100.098	-101.318\\
-101.318	-157.471\\
-157.471	-261.23\\
-261.23	-185.547\\
-185.547	-118.408\\
-118.408	-85.449\\
-85.449	-101.318\\
-101.318	-90.332\\
-90.332	-67.139\\
-67.139	-76.904\\
-76.904	-92.773\\
-92.773	-120.85\\
-120.85	-134.277\\
-134.277	-89.111\\
-89.111	-156.25\\
-156.25	-179.443\\
-179.443	-170.898\\
-170.898	-140.381\\
-140.381	-150.146\\
-150.146	-202.637\\
-202.637	-125.732\\
-125.732	-53.711\\
-53.711	-78.125\\
-78.125	-85.449\\
-85.449	-119.629\\
-119.629	-101.318\\
-101.318	-74.463\\
-74.463	-118.408\\
-118.408	-112.305\\
-112.305	-79.346\\
-79.346	-102.539\\
-102.539	-90.332\\
-90.332	-80.566\\
-80.566	-95.215\\
-95.215	-54.932\\
-54.932	-68.359\\
-68.359	-47.607\\
-47.607	-51.27\\
-51.27	-106.201\\
-106.201	-122.07\\
-122.07	-167.236\\
-167.236	-219.727\\
-219.727	-142.822\\
-142.822	-83.008\\
-83.008	-63.477\\
-63.477	-62.256\\
-62.256	-91.553\\
-91.553	-54.932\\
-54.932	-70.801\\
-70.801	-85.449\\
-85.449	-150.146\\
-150.146	-133.057\\
-133.057	-190.43\\
-190.43	-124.512\\
-124.512	-117.188\\
-117.188	-56.152\\
-56.152	-62.256\\
-62.256	-34.18\\
-34.18	-46.387\\
-46.387	-51.27\\
-51.27	-95.215\\
-95.215	-100.098\\
-100.098	-117.188\\
-117.188	-123.291\\
-123.291	-197.754\\
-197.754	-170.898\\
-170.898	-128.174\\
-128.174	-153.809\\
-153.809	-163.574\\
-163.574	-213.623\\
-213.623	-161.133\\
-161.133	-104.98\\
-104.98	-53.711\\
-53.711	-40.283\\
-40.283	-41.504\\
-41.504	-52.49\\
-52.49	-54.932\\
-54.932	-51.27\\
-51.27	-70.801\\
-70.801	-108.643\\
-108.643	-98.877\\
-98.877	-103.76\\
-103.76	-118.408\\
-118.408	-69.58\\
-69.58	-36.621\\
-36.621	-81.787\\
-81.787	-125.732\\
-125.732	-125.732\\
-125.732	-142.822\\
-142.822	-118.408\\
-118.408	-87.891\\
-87.891	-123.291\\
-123.291	-172.119\\
-172.119	-172.119\\
-172.119	-120.85\\
-120.85	-207.52\\
-207.52	-155.029\\
-155.029	-151.367\\
-151.367	-173.34\\
-173.34	-172.119\\
-172.119	-130.615\\
-130.615	-112.305\\
-112.305	-192.871\\
-192.871	-266.113\\
-266.113	-202.637\\
-202.637	-124.512\\
-124.512	-118.408\\
-118.408	-115.967\\
-115.967	-150.146\\
-150.146	-173.34\\
-173.34	-125.732\\
-125.732	-134.277\\
-134.277	-177.002\\
-177.002	-192.871\\
-192.871	-120.85\\
-120.85	-97.656\\
-97.656	-130.615\\
-130.615	-218.506\\
-218.506	-195.313\\
-195.313	-203.857\\
-203.857	-115.967\\
-115.967	-108.643\\
-108.643	-101.318\\
-101.318	-63.477\\
-63.477	-41.504\\
-41.504	-34.18\\
-34.18	-29.297\\
-29.297	-78.125\\
-78.125	-108.643\\
-108.643	-131.836\\
-131.836	-93.994\\
-93.994	-107.422\\
-107.422	-119.629\\
-119.629	-72.021\\
-72.021	-79.346\\
-79.346	-73.242\\
-73.242	-54.932\\
-54.932	-75.684\\
-75.684	-120.85\\
-120.85	-156.25\\
-156.25	-95.215\\
-95.215	-73.242\\
-73.242	-58.594\\
-58.594	-74.463\\
-74.463	-45.166\\
-45.166	-117.188\\
-117.188	-195.313\\
-195.313	-163.574\\
-163.574	-216.064\\
-216.064	-241.699\\
-241.699	-249.023\\
-249.023	-181.885\\
-181.885	-131.836\\
-131.836	-130.615\\
-130.615	-129.395\\
-129.395	-155.029\\
-155.029	-184.326\\
-184.326	-181.885\\
-181.885	-247.803\\
-247.803	-270.996\\
-270.996	-328.369\\
-328.369	-217.285\\
-217.285	-238.037\\
-238.037	-264.893\\
-264.893	-205.078\\
-205.078	-239.258\\
-239.258	-200.195\\
-200.195	-113.525\\
-113.525	-74.463\\
-74.463	-79.346\\
-79.346	-117.188\\
-117.188	-97.656\\
-97.656	-64.697\\
-64.697	-90.332\\
-90.332	-63.477\\
-63.477	-48.828\\
-48.828	-64.697\\
-64.697	-72.021\\
-72.021	-48.828\\
-48.828	-100.098\\
-100.098	-135.498\\
-135.498	-97.656\\
-97.656	-173.34\\
-173.34	-241.699\\
-241.699	-220.947\\
-220.947	-279.541\\
-279.541	-187.988\\
-187.988	-206.299\\
-206.299	-134.277\\
-134.277	-85.449\\
-85.449	-108.643\\
-108.643	-87.891\\
-87.891	-63.477\\
-63.477	-92.773\\
-92.773	-50.049\\
-50.049	-32.959\\
-32.959	-43.945\\
-43.945	-98.877\\
-98.877	-125.732\\
-125.732	-157.471\\
-157.471	-152.588\\
-152.588	-146.484\\
-146.484	-168.457\\
-168.457	-135.498\\
-135.498	-108.643\\
-108.643	-112.305\\
-112.305	-89.111\\
-89.111	-146.484\\
-146.484	-96.436\\
-96.436	-80.566\\
-80.566	-120.85\\
-120.85	-164.795\\
-164.795	-123.291\\
-123.291	-93.994\\
-93.994	-80.566\\
-80.566	-91.553\\
-91.553	-61.035\\
-61.035	-61.035\\
-61.035	-87.891\\
-87.891	-109.863\\
-109.863	-124.512\\
-124.512	-96.436\\
-96.436	-128.174\\
-128.174	-114.746\\
-114.746	-61.035\\
-61.035	-69.58\\
-69.58	-135.498\\
-135.498	-190.43\\
-190.43	-186.768\\
-186.768	-157.471\\
-157.471	-152.588\\
-152.588	-114.746\\
-114.746	-109.863\\
-109.863	-87.891\\
-87.891	-125.732\\
-125.732	-117.188\\
-117.188	-70.801\\
-70.801	-107.422\\
-107.422	-126.953\\
-126.953	-150.146\\
-150.146	-164.795\\
-164.795	-163.574\\
-163.574	-222.168\\
-222.168	-267.334\\
-267.334	-302.734\\
-302.734	-190.43\\
-190.43	-106.201\\
-106.201	-68.359\\
-68.359	-45.166\\
-45.166	-70.801\\
-70.801	-64.697\\
-64.697	-119.629\\
-119.629	-102.539\\
-102.539	-93.994\\
-93.994	-124.512\\
-124.512	-145.264\\
-145.264	-172.119\\
-172.119	-183.105\\
-183.105	-258.789\\
-258.789	-231.934\\
-231.934	-177.002\\
-177.002	-120.85\\
-120.85	-120.85\\
-120.85	-123.291\\
-123.291	-157.471\\
-157.471	-108.643\\
-108.643	-113.525\\
-113.525	-107.422\\
-107.422	-58.594\\
-58.594	-102.539\\
-102.539	-152.588\\
-152.588	-107.422\\
-107.422	-115.967\\
-115.967	-216.064\\
-216.064	-159.912\\
-159.912	-156.25\\
-156.25	-219.727\\
-219.727	-206.299\\
-206.299	-129.395\\
-129.395	-85.449\\
-85.449	-85.449\\
-85.449	-146.484\\
-146.484	-236.816\\
-236.816	-246.582\\
-246.582	-251.465\\
-251.465	-192.871\\
-192.871	-122.07\\
-122.07	-103.76\\
-103.76	-72.021\\
-72.021	-69.58\\
-69.58	-58.594\\
-58.594	-89.111\\
-89.111	-102.539\\
-102.539	-58.594\\
-58.594	-90.332\\
-90.332	-126.953\\
-126.953	-145.264\\
-145.264	-185.547\\
-185.547	-233.154\\
-233.154	-281.982\\
-281.982	-197.754\\
-197.754	-172.119\\
-172.119	-178.223\\
-178.223	-156.25\\
-156.25	-103.76\\
-103.76	-128.174\\
-128.174	-214.844\\
-214.844	-246.582\\
-246.582	-141.602\\
-141.602	-79.346\\
-79.346	-52.49\\
-52.49	-28.076\\
-28.076	-35.4\\
-35.4	-64.697\\
-64.697	-74.463\\
-74.463	-102.539\\
-102.539	-81.787\\
-81.787	-64.697\\
-64.697	-62.256\\
-62.256	-61.035\\
-61.035	-54.932\\
-54.932	-65.918\\
-65.918	-102.539\\
-102.539	-148.926\\
-148.926	-157.471\\
-157.471	-155.029\\
-155.029	-183.105\\
-183.105	-225.83\\
-225.83	-225.83\\
-225.83	-270.996\\
-270.996	-247.803\\
-247.803	-184.326\\
-184.326	-125.732\\
-125.732	-122.07\\
-122.07	-205.078\\
-205.078	-236.816\\
-236.816	-166.016\\
-166.016	-107.422\\
-107.422	-53.711\\
-53.711	-40.283\\
-40.283	-42.725\\
-42.725	-25.635\\
-25.635	-61.035\\
-61.035	-84.229\\
-84.229	-89.111\\
-89.111	-78.125\\
-78.125	-47.607\\
-47.607	-35.4\\
-35.4	-52.49\\
-52.49	-56.152\\
-56.152	-62.256\\
-62.256	-47.607\\
-47.607	-37.842\\
-37.842	-69.58\\
-69.58	-164.795\\
-164.795	-190.43\\
-190.43	-217.285\\
-217.285	-147.705\\
-147.705	-208.74\\
-208.74	-291.748\\
-291.748	-261.23\\
-261.23	-273.438\\
-273.438	-200.195\\
-200.195	-202.637\\
-202.637	-212.402\\
-212.402	-139.16\\
-139.16	-133.057\\
-133.057	-80.566\\
-80.566	-75.684\\
-75.684	-135.498\\
-135.498	-93.994\\
-93.994	-104.98\\
-104.98	-115.967\\
-115.967	-111.084\\
-111.084	-113.525\\
-113.525	-120.85\\
-120.85	-136.719\\
-136.719	-147.705\\
-147.705	-125.732\\
-125.732	-92.773\\
-92.773	-120.85\\
-120.85	-57.373\\
-57.373	-26.855\\
-26.855	-46.387\\
-46.387	-73.242\\
-73.242	-37.842\\
-37.842	-17.09\\
-17.09	-39.063\\
-39.063	-70.801\\
-70.801	-84.229\\
-84.229	-104.98\\
-104.98	-87.891\\
-87.891	-142.822\\
-142.822	-123.291\\
-123.291	-83.008\\
-83.008	-125.732\\
-125.732	-70.801\\
-70.801	-56.152\\
-56.152	-39.063\\
-39.063	-24.414\\
-24.414	-48.828\\
-48.828	-36.621\\
-36.621	-65.918\\
-65.918	-111.084\\
-111.084	-106.201\\
-106.201	-76.904\\
-76.904	-86.67\\
-86.67	-64.697\\
-64.697	-92.773\\
-92.773	-67.139\\
-67.139	-63.477\\
-63.477	-58.594\\
-58.594	-43.945\\
-43.945	-57.373\\
-57.373	-51.27\\
-51.27	-90.332\\
-90.332	-83.008\\
-83.008	-67.139\\
-67.139	-64.697\\
-64.697	-51.27\\
-51.27	-61.035\\
-61.035	-135.498\\
-135.498	-179.443\\
-179.443	-123.291\\
-123.291	-114.746\\
-114.746	-80.566\\
-80.566	-136.719\\
-136.719	-125.732\\
-125.732	-80.566\\
-80.566	-102.539\\
-102.539	-76.904\\
-76.904	-108.643\\
-108.643	-64.697\\
-64.697	-64.697\\
-64.697	-80.566\\
-80.566	-104.98\\
-104.98	-74.463\\
-74.463	-81.787\\
-81.787	-85.449\\
-85.449	-135.498\\
-135.498	-115.967\\
-115.967	-178.223\\
-178.223	-231.934\\
-231.934	-183.105\\
-183.105	-117.188\\
-117.188	-86.67\\
-86.67	-135.498\\
-135.498	-173.34\\
-173.34	-133.057\\
-133.057	-163.574\\
-163.574	-98.877\\
-98.877	-50.049\\
-50.049	-52.49\\
-52.49	-119.629\\
-119.629	-107.422\\
-107.422	-100.098\\
-100.098	-156.25\\
-156.25	-146.484\\
-146.484	-133.057\\
-133.057	-90.332\\
-90.332	-114.746\\
-114.746	-148.926\\
-148.926	-120.85\\
-120.85	-125.732\\
-125.732	-112.305\\
-112.305	-80.566\\
-80.566	-97.656\\
-97.656	-70.801\\
-70.801	-79.346\\
-79.346	-135.498\\
-135.498	-156.25\\
-156.25	-164.795\\
-164.795	-146.484\\
-146.484	-104.98\\
-104.98	-72.021\\
-72.021	-85.449\\
-85.449	-101.318\\
-101.318	-129.395\\
-129.395	-159.912\\
-159.912	-190.43\\
-190.43	-146.484\\
-146.484	-84.229\\
-84.229	-155.029\\
-155.029	-222.168\\
-222.168	-155.029\\
-155.029	-155.029\\
-155.029	-181.885\\
-181.885	-137.939\\
-137.939	-113.525\\
-113.525	-129.395\\
-129.395	-115.967\\
-115.967	-131.836\\
-131.836	-167.236\\
-167.236	-125.732\\
-125.732	-146.484\\
-146.484	-137.939\\
-137.939	-141.602\\
-141.602	-109.863\\
-109.863	-144.043\\
-144.043	-102.539\\
-102.539	-106.201\\
-106.201	-119.629\\
-119.629	-115.967\\
-115.967	-58.594\\
-58.594	-35.4\\
-35.4	-28.076\\
-28.076	-47.607\\
-47.607	-122.07\\
-122.07	-161.133\\
-161.133	-161.133\\
-161.133	-101.318\\
-101.318	-119.629\\
-119.629	-104.98\\
-104.98	-92.773\\
-92.773	-58.594\\
-58.594	-81.787\\
-81.787	-81.787\\
-81.787	-80.566\\
-80.566	-95.215\\
-95.215	-79.346\\
-79.346	-75.684\\
-75.684	-50.049\\
-50.049	-72.021\\
-72.021	-139.16\\
-139.16	-96.436\\
-96.436	-119.629\\
-119.629	-107.422\\
-107.422	-145.264\\
-145.264	-135.498\\
-135.498	-186.768\\
-186.768	-137.939\\
-137.939	-151.367\\
-151.367	-155.029\\
-155.029	-72.021\\
-72.021	-131.836\\
-131.836	-95.215\\
-95.215	-113.525\\
-113.525	-128.174\\
-128.174	-166.016\\
-166.016	-123.291\\
-123.291	-158.691\\
-158.691	-201.416\\
-201.416	-164.795\\
-164.795	-142.822\\
-142.822	-113.525\\
-113.525	-136.719\\
-136.719	-113.525\\
-113.525	-96.436\\
-96.436	-81.787\\
-81.787	-96.436\\
-96.436	-83.008\\
-83.008	-172.119\\
-172.119	-225.83\\
-225.83	-191.65\\
-191.65	-187.988\\
-187.988	-181.885\\
-181.885	-101.318\\
-101.318	-89.111\\
-89.111	-72.021\\
-72.021	-62.256\\
-62.256	-68.359\\
-68.359	-115.967\\
-115.967	-146.484\\
-146.484	-167.236\\
-167.236	-141.602\\
-141.602	-93.994\\
-93.994	-169.678\\
-169.678	-201.416\\
-201.416	-224.609\\
-224.609	-178.223\\
-178.223	-288.086\\
-288.086	-209.961\\
-209.961	-117.188\\
-117.188	-84.229\\
-84.229	-69.58\\
-69.58	-125.732\\
-125.732	-162.354\\
-162.354	-218.506\\
-218.506	-166.016\\
-166.016	-126.953\\
-126.953	-69.58\\
-69.58	-72.021\\
-72.021	-78.125\\
-78.125	-87.891\\
-87.891	-56.152\\
-56.152	-70.801\\
-70.801	-59.814\\
-59.814	-36.621\\
-36.621	-83.008\\
-83.008	-93.994\\
-93.994	-118.408\\
-118.408	-107.422\\
-107.422	-112.305\\
-112.305	-113.525\\
-113.525	-146.484\\
-146.484	-100.098\\
-100.098	-142.822\\
-142.822	-169.678\\
-169.678	-130.615\\
-130.615	-139.16\\
-139.16	-119.629\\
-119.629	-135.498\\
-135.498	-191.65\\
-191.65	-148.926\\
-148.926	-114.746\\
-114.746	-128.174\\
-128.174	-115.967\\
-115.967	-79.346\\
-79.346	-122.07\\
-122.07	-181.885\\
-181.885	-170.898\\
-170.898	-187.988\\
-187.988	-279.541\\
-279.541	-187.988\\
-187.988	-159.912\\
-159.912	-120.85\\
-120.85	-152.588\\
-152.588	-224.609\\
-224.609	-148.926\\
-148.926	-129.395\\
-129.395	-186.768\\
-186.768	-119.629\\
-119.629	-68.359\\
-68.359	-87.891\\
-87.891	-68.359\\
-68.359	-52.49\\
-52.49	-57.373\\
-57.373	-50.049\\
-50.049	-48.828\\
-48.828	-64.697\\
-64.697	-95.215\\
-95.215	-107.422\\
-107.422	-147.705\\
-147.705	-139.16\\
-139.16	-166.016\\
-166.016	-194.092\\
-194.092	-174.561\\
-174.561	-125.732\\
-125.732	-134.277\\
-134.277	-119.629\\
-119.629	-130.615\\
-130.615	-183.105\\
-183.105	-140.381\\
-140.381	-158.691\\
-158.691	-136.719\\
-136.719	-129.395\\
-129.395	-201.416\\
-201.416	-166.016\\
-166.016	-167.236\\
-167.236	-152.588\\
-152.588	-91.553\\
-91.553	-58.594\\
-58.594	-46.387\\
-46.387	-84.229\\
-84.229	-79.346\\
-79.346	-106.201\\
-106.201	-79.346\\
-79.346	-108.643\\
-108.643	-191.65\\
-191.65	-235.596\\
-235.596	-185.547\\
-185.547	-190.43\\
-190.43	-169.678\\
-169.678	-118.408\\
-118.408	-133.057\\
-133.057	-145.264\\
-145.264	-125.732\\
-125.732	-144.043\\
-144.043	-181.885\\
-181.885	-260.01\\
-260.01	-230.713\\
-230.713	-214.844\\
-214.844	-239.258\\
-239.258	-162.354\\
-162.354	-173.34\\
-173.34	-139.16\\
-139.16	-141.602\\
-141.602	-186.768\\
-186.768	-213.623\\
-213.623	-84.229\\
-84.229	-100.098\\
-100.098	-85.449\\
-85.449	-119.629\\
-119.629	-123.291\\
-123.291	-75.684\\
-75.684	-98.877\\
-98.877	-173.34\\
-173.34	-303.955\\
-303.955	-289.307\\
-289.307	-261.23\\
-261.23	-211.182\\
-211.182	-240.479\\
-240.479	-290.527\\
-290.527	-217.285\\
-217.285	-218.506\\
-218.506	-225.83\\
-225.83	-152.588\\
-152.588	-131.836\\
-131.836	-169.678\\
-169.678	-115.967\\
-115.967	-81.787\\
-81.787	-114.746\\
-114.746	-84.229\\
-84.229	-98.877\\
-98.877	-95.215\\
-95.215	-115.967\\
-115.967	-158.691\\
-158.691	-126.953\\
-126.953	-90.332\\
-90.332	-112.305\\
-112.305	-129.395\\
-129.395	-153.809\\
-153.809	-129.395\\
-129.395	-106.201\\
-106.201	-167.236\\
-167.236	-159.912\\
-159.912	-111.084\\
-111.084	-183.105\\
-183.105	-168.457\\
-168.457	-119.629\\
-119.629	-81.787\\
-81.787	-58.594\\
-58.594	-45.166\\
-45.166	-98.877\\
-98.877	-96.436\\
-96.436	-65.918\\
-65.918	-47.607\\
-47.607	-57.373\\
-57.373	-102.539\\
-102.539	-119.629\\
-119.629	-158.691\\
-158.691	-180.664\\
-180.664	-172.119\\
-172.119	-181.885\\
-181.885	-112.305\\
-112.305	-46.387\\
-46.387	-70.801\\
-70.801	-62.256\\
-62.256	-78.125\\
-78.125	-126.953\\
-126.953	-139.16\\
-139.16	-201.416\\
-201.416	-133.057\\
-133.057	-124.512\\
-124.512	-191.65\\
-191.65	-142.822\\
-142.822	-95.215\\
-95.215	-69.58\\
-69.58	-93.994\\
-93.994	-129.395\\
-129.395	-190.43\\
-190.43	-130.615\\
-130.615	-106.201\\
-106.201	-98.877\\
-98.877	-122.07\\
-122.07	-163.574\\
-163.574	-255.127\\
-255.127	-219.727\\
-219.727	-214.844\\
-214.844	-137.939\\
-137.939	-90.332\\
-90.332	-107.422\\
-107.422	-50.049\\
-50.049	-74.463\\
-74.463	-69.58\\
-69.58	-79.346\\
-79.346	-134.277\\
-134.277	-185.547\\
-185.547	-217.285\\
-217.285	-278.32\\
-278.32	-239.258\\
-239.258	-129.395\\
-129.395	-118.408\\
-118.408	-98.877\\
-98.877	-136.719\\
-136.719	-179.443\\
-179.443	-239.258\\
-239.258	-179.443\\
-179.443	-122.07\\
-122.07	-139.16\\
-139.16	-184.326\\
-184.326	-133.057\\
-133.057	-89.111\\
-89.111	-73.242\\
-73.242	-50.049\\
-50.049	-48.828\\
-48.828	-36.621\\
-36.621	-64.697\\
-64.697	-98.877\\
-98.877	-107.422\\
-107.422	-80.566\\
-80.566	-73.242\\
-73.242	-34.18\\
-34.18	-17.09\\
-17.09	-30.518\\
-30.518	-57.373\\
-57.373	-87.891\\
-87.891	-111.084\\
-111.084	-130.615\\
-130.615	-150.146\\
-150.146	-190.43\\
-190.43	-266.113\\
-266.113	-261.23\\
-261.23	-280.762\\
-280.762	-249.023\\
-249.023	-136.719\\
-136.719	-123.291\\
-123.291	-125.732\\
-125.732	-166.016\\
-166.016	-145.264\\
-145.264	-102.539\\
-102.539	-79.346\\
-79.346	-63.477\\
-63.477	-111.084\\
-111.084	-108.643\\
-108.643	-56.152\\
-56.152	-41.504\\
-41.504	-69.58\\
-69.58	-96.436\\
-96.436	-76.904\\
-76.904	-106.201\\
-106.201	-86.67\\
-86.67	-122.07\\
-122.07	-181.885\\
-181.885	-146.484\\
-146.484	-190.43\\
-190.43	-260.01\\
-260.01	-281.982\\
-281.982	-223.389\\
-223.389	-145.264\\
-145.264	-107.422\\
-107.422	-64.697\\
-64.697	-101.318\\
-101.318	-75.684\\
-75.684	-126.953\\
-126.953	-115.967\\
-115.967	-92.773\\
-92.773	-119.629\\
-119.629	-69.58\\
-69.58	-98.877\\
-98.877	-75.684\\
-75.684	-46.387\\
-46.387	-54.932\\
-54.932	-42.725\\
-42.725	-47.607\\
-47.607	-46.387\\
-46.387	-30.518\\
-30.518	-42.725\\
-42.725	-62.256\\
-62.256	-61.035\\
-61.035	-61.035\\
-61.035	-97.656\\
-97.656	-130.615\\
-130.615	-108.643\\
-108.643	-98.877\\
-98.877	-156.25\\
-156.25	-190.43\\
-190.43	-150.146\\
-150.146	-156.25\\
-156.25	-74.463\\
-74.463	-43.945\\
-43.945	-90.332\\
-90.332	-133.057\\
-133.057	-145.264\\
-145.264	-202.637\\
-202.637	-183.105\\
-183.105	-130.615\\
-130.615	-89.111\\
-89.111	-93.994\\
-93.994	-102.539\\
-102.539	-81.787\\
-81.787	-48.828\\
-48.828	-64.697\\
-64.697	-52.49\\
-52.49	-41.504\\
-41.504	-32.959\\
-32.959	-61.035\\
-61.035	-93.994\\
-93.994	-100.098\\
-100.098	-69.58\\
-69.58	-123.291\\
-123.291	-172.119\\
-172.119	-109.863\\
-109.863	-146.484\\
-146.484	-163.574\\
-163.574	-118.408\\
-118.408	-69.58\\
-69.58	-86.67\\
-86.67	-108.643\\
-108.643	-59.814\\
-59.814	-69.58\\
-69.58	-53.711\\
-53.711	-31.738\\
-31.738	-31.738\\
-31.738	-56.152\\
-56.152	-76.904\\
-76.904	-67.139\\
-67.139	-83.008\\
-83.008	-100.098\\
-100.098	-73.242\\
-73.242	-37.842\\
-37.842	-32.959\\
-32.959	-41.504\\
-41.504	-50.049\\
-50.049	-61.035\\
-61.035	-128.174\\
-128.174	-130.615\\
-130.615	-109.863\\
-109.863	-107.422\\
-107.422	-141.602\\
-141.602	-185.547\\
-185.547	-201.416\\
-201.416	-205.078\\
-205.078	-151.367\\
-151.367	-157.471\\
-157.471	-162.354\\
-162.354	-166.016\\
-166.016	-185.547\\
-185.547	-246.582\\
-246.582	-216.064\\
-216.064	-195.313\\
-195.313	-157.471\\
-157.471	-147.705\\
-147.705	-142.822\\
-142.822	-167.236\\
-167.236	-103.76\\
-103.76	-120.85\\
-120.85	-148.926\\
-148.926	-179.443\\
-179.443	-137.939\\
-137.939	-155.029\\
-155.029	-129.395\\
-129.395	-112.305\\
-112.305	-70.801\\
-70.801	-111.084\\
-111.084	-177.002\\
-177.002	-141.602\\
-141.602	-81.787\\
-81.787	-97.656\\
-97.656	-53.711\\
-53.711	-50.049\\
-50.049	-70.801\\
-70.801	-45.166\\
-45.166	-46.387\\
-46.387	-54.932\\
-54.932	-36.621\\
-36.621	-62.256\\
-62.256	-51.27\\
-51.27	-30.518\\
-30.518	-57.373\\
-57.373	-64.697\\
-64.697	-80.566\\
-80.566	-75.684\\
-75.684	-76.904\\
-76.904	-104.98\\
-104.98	-92.773\\
-92.773	-104.98\\
-104.98	-180.664\\
-180.664	-128.174\\
-128.174	-112.305\\
-112.305	-122.07\\
-122.07	-86.67\\
-86.67	-51.27\\
-51.27	-92.773\\
-92.773	-73.242\\
-73.242	-43.945\\
-43.945	-81.787\\
-81.787	-80.566\\
-80.566	-98.877\\
-98.877	-61.035\\
-61.035	-36.621\\
-36.621	-29.297\\
-29.297	-36.621\\
-36.621	-69.58\\
-69.58	-68.359\\
-68.359	-80.566\\
-80.566	-109.863\\
-109.863	-150.146\\
-150.146	-109.863\\
-109.863	-152.588\\
-152.588	-177.002\\
-177.002	-153.809\\
-153.809	-145.264\\
-145.264	-230.713\\
-230.713	-167.236\\
-167.236	-208.74\\
-208.74	-178.223\\
-178.223	-203.857\\
-203.857	-234.375\\
-234.375	-136.719\\
-136.719	-81.787\\
-81.787	-65.918\\
-65.918	-108.643\\
-108.643	-134.277\\
-134.277	-137.939\\
-137.939	-90.332\\
-90.332	-50.049\\
-50.049	-67.139\\
-67.139	-128.174\\
-128.174	-179.443\\
-179.443	-177.002\\
-177.002	-103.76\\
-103.76	-81.787\\
-81.787	-109.863\\
-109.863	-175.781\\
-175.781	-200.195\\
-200.195	-203.857\\
-203.857	-150.146\\
-150.146	-208.74\\
-208.74	-227.051\\
-227.051	-212.402\\
-212.402	-290.527\\
-290.527	-251.465\\
-251.465	-209.961\\
-209.961	-241.699\\
-241.699	-207.52\\
-207.52	-106.201\\
-106.201	-100.098\\
-100.098	-126.953\\
-126.953	-155.029\\
-155.029	-158.691\\
-158.691	-115.967\\
-115.967	-148.926\\
-148.926	-130.615\\
-130.615	-96.436\\
-96.436	-129.395\\
-129.395	-122.07\\
-122.07	-175.781\\
-175.781	-181.885\\
-181.885	-134.277\\
-134.277	-97.656\\
-97.656	-58.594\\
-58.594	-96.436\\
-96.436	-74.463\\
-74.463	-65.918\\
-65.918	-53.711\\
-53.711	-95.215\\
-95.215	-123.291\\
-123.291	-137.939\\
-137.939	-136.719\\
-136.719	-80.566\\
-80.566	-62.256\\
-62.256	-42.725\\
-42.725	-54.932\\
-54.932	-91.553\\
-91.553	-95.215\\
-95.215	-91.553\\
-91.553	-150.146\\
-150.146	-174.561\\
-174.561	-198.975\\
-198.975	-198.975\\
-198.975	-229.492\\
-229.492	-137.939\\
-137.939	-113.525\\
-113.525	-87.891\\
-87.891	-93.994\\
-93.994	-128.174\\
-128.174	-106.201\\
-106.201	-69.58\\
-69.58	-64.697\\
-64.697	-124.512\\
-124.512	-153.809\\
-153.809	-134.277\\
-134.277	-211.182\\
-211.182	-230.713\\
-230.713	-159.912\\
-159.912	-111.084\\
-111.084	-76.904\\
-76.904	-70.801\\
-70.801	-131.836\\
-131.836	-106.201\\
-106.201	-122.07\\
-122.07	-139.16\\
-139.16	-156.25\\
-156.25	-151.367\\
-151.367	-168.457\\
-168.457	-115.967\\
-115.967	-130.615\\
-130.615	-91.553\\
-91.553	-84.229\\
-84.229	-129.395\\
-129.395	-183.105\\
-183.105	-201.416\\
-201.416	-106.201\\
-106.201	-220.947\\
-220.947	-207.52\\
-207.52	-139.16\\
-139.16	-76.904\\
-76.904	-123.291\\
-123.291	-107.422\\
-107.422	-104.98\\
-104.98	-123.291\\
-123.291	-80.566\\
-80.566	-108.643\\
-108.643	-207.52\\
-207.52	-219.727\\
-219.727	-145.264\\
-145.264	-159.912\\
-159.912	-177.002\\
-177.002	-180.664\\
-180.664	-131.836\\
-131.836	-128.174\\
-128.174	-84.229\\
-84.229	-133.057\\
-133.057	-134.277\\
-134.277	-229.492\\
-229.492	-251.465\\
-251.465	-332.031\\
-332.031	-238.037\\
-238.037	-194.092\\
-194.092	-147.705\\
-147.705	-162.354\\
-162.354	-122.07\\
-122.07	-84.229\\
-84.229	-81.787\\
-81.787	-85.449\\
-85.449	-62.256\\
-62.256	-46.387\\
-46.387	-74.463\\
-74.463	-100.098\\
-100.098	-68.359\\
-68.359	-46.387\\
-46.387	-46.387\\
-46.387	-112.305\\
-112.305	-157.471\\
-157.471	-73.242\\
-73.242	-95.215\\
-95.215	-97.656\\
-97.656	-91.553\\
-91.553	-113.525\\
-113.525	-98.877\\
-98.877	-68.359\\
-68.359	-48.828\\
-48.828	-40.283\\
-40.283	-53.711\\
-53.711	-106.201\\
-106.201	-126.953\\
-126.953	-96.436\\
-96.436	-63.477\\
-63.477	-39.063\\
-39.063	-56.152\\
-56.152	-96.436\\
-96.436	-123.291\\
-123.291	-133.057\\
-133.057	-90.332\\
-90.332	-86.67\\
-86.67	-95.215\\
-95.215	-133.057\\
-133.057	-185.547\\
-185.547	-216.064\\
-216.064	-164.795\\
-164.795	-102.539\\
-102.539	-107.422\\
-107.422	-100.098\\
-100.098	-103.76\\
-103.76	-65.918\\
-65.918	-87.891\\
-87.891	-93.994\\
-93.994	-91.553\\
-91.553	-56.152\\
-56.152	-36.621\\
-36.621	-56.152\\
-56.152	-84.229\\
-84.229	-134.277\\
-134.277	-148.926\\
-148.926	-179.443\\
-179.443	-159.912\\
-159.912	-180.664\\
-180.664	-239.258\\
-239.258	-319.824\\
-319.824	-253.906\\
-253.906	-173.34\\
-173.34	-189.209\\
-189.209	-220.947\\
-220.947	-288.086\\
-288.086	-180.664\\
-180.664	-89.111\\
-89.111	-59.814\\
-59.814	-62.256\\
-62.256	-42.725\\
-42.725	-28.076\\
-28.076	-39.063\\
-39.063	-59.814\\
-59.814	-83.008\\
-83.008	-86.67\\
-86.67	-65.918\\
-65.918	-62.256\\
-62.256	-78.125\\
-78.125	-100.098\\
-100.098	-104.98\\
-104.98	-96.436\\
-96.436	-68.359\\
-68.359	-47.607\\
-47.607	-45.166\\
-45.166	-72.021\\
-72.021	-87.891\\
-87.891	-64.697\\
-64.697	-41.504\\
-41.504	-76.904\\
-76.904	-48.828\\
-48.828	-69.58\\
-69.58	-79.346\\
-79.346	-120.85\\
-120.85	-126.953\\
-126.953	-84.229\\
-84.229	-125.732\\
-125.732	-123.291\\
-123.291	-101.318\\
-101.318	-84.229\\
-84.229	-142.822\\
-142.822	-175.781\\
-175.781	-175.781\\
-175.781	-194.092\\
-194.092	-147.705\\
-147.705	-136.719\\
-136.719	-126.953\\
-126.953	-173.34\\
-173.34	-135.498\\
-135.498	-184.326\\
-184.326	-172.119\\
-172.119	-178.223\\
-178.223	-303.955\\
-303.955	-231.934\\
-231.934	-125.732\\
-125.732	-83.008\\
-83.008	-87.891\\
-87.891	-112.305\\
-112.305	-146.484\\
-146.484	-172.119\\
-172.119	-189.209\\
-189.209	-192.871\\
-192.871	-125.732\\
-125.732	-111.084\\
-111.084	-130.615\\
-130.615	-129.395\\
-129.395	-76.904\\
-76.904	-57.373\\
-57.373	-103.76\\
-103.76	-101.318\\
-101.318	-142.822\\
-142.822	-174.561\\
-174.561	-155.029\\
-155.029	-106.201\\
-106.201	-85.449\\
-85.449	-135.498\\
-135.498	-168.457\\
-168.457	-230.713\\
-230.713	-244.141\\
-244.141	-286.865\\
-286.865	-236.816\\
-236.816	-216.064\\
-216.064	-123.291\\
-123.291	-85.449\\
-85.449	-153.809\\
-153.809	-202.637\\
-202.637	-115.967\\
-115.967	-186.768\\
-186.768	-249.023\\
-249.023	-308.838\\
-308.838	-189.209\\
-189.209	-112.305\\
-112.305	-63.477\\
-63.477	-95.215\\
-95.215	-85.449\\
-85.449	-93.994\\
-93.994	-51.27\\
-51.27	-76.904\\
-76.904	-52.49\\
-52.49	-74.463\\
-74.463	-58.594\\
-58.594	-79.346\\
-79.346	-124.512\\
-124.512	-112.305\\
-112.305	-162.354\\
-162.354	-191.65\\
-191.65	-190.43\\
-190.43	-103.76\\
-103.76	-109.863\\
-109.863	-130.615\\
-130.615	-85.449\\
-85.449	-80.566\\
-80.566	-54.932\\
-54.932	-90.332\\
-90.332	-79.346\\
-79.346	-47.607\\
-47.607	-26.855\\
-26.855	-31.738\\
-31.738	-58.594\\
-58.594	-85.449\\
-85.449	-91.553\\
-91.553	-56.152\\
-56.152	-67.139\\
-67.139	-102.539\\
-102.539	-135.498\\
-135.498	-93.994\\
-93.994	-87.891\\
-87.891	-104.98\\
-104.98	-125.732\\
-125.732	-115.967\\
-115.967	-137.939\\
-137.939	-126.953\\
-126.953	-89.111\\
-89.111	-72.021\\
-72.021	-95.215\\
-95.215	-78.125\\
-78.125	-97.656\\
-97.656	-129.395\\
-129.395	-203.857\\
-203.857	-139.16\\
-139.16	-140.381\\
-140.381	-206.299\\
-206.299	-152.588\\
-152.588	-96.436\\
-96.436	-69.58\\
-69.58	-54.932\\
-54.932	-58.594\\
-58.594	-46.387\\
-46.387	-50.049\\
-50.049	-106.201\\
-106.201	-45.166\\
-45.166	-50.049\\
-50.049	-65.918\\
-65.918	-93.994\\
-93.994	-69.58\\
-69.58	-53.711\\
-53.711	-28.076\\
-28.076	-54.932\\
-54.932	-101.318\\
-101.318	-139.16\\
-139.16	-108.643\\
-108.643	-89.111\\
-89.111	-101.318\\
-101.318	-102.539\\
-102.539	-119.629\\
-119.629	-144.043\\
-144.043	-168.457\\
-168.457	-164.795\\
-164.795	-140.381\\
-140.381	-91.553\\
-91.553	-73.242\\
-73.242	-79.346\\
-79.346	-112.305\\
-112.305	-64.697\\
-64.697	-62.256\\
-62.256	-114.746\\
-114.746	-139.16\\
-139.16	-133.057\\
-133.057	-157.471\\
-157.471	-205.078\\
-205.078	-104.98\\
-104.98	-184.326\\
-184.326	-185.547\\
-185.547	-170.898\\
-170.898	-184.326\\
-184.326	-181.885\\
-181.885	-312.5\\
-312.5	-286.865\\
-286.865	-195.313\\
-195.313	-236.816\\
-236.816	-283.203\\
-283.203	-275.879\\
-275.879	-311.279\\
-311.279	-281.982\\
-281.982	-372.314\\
-372.314	-280.762\\
-280.762	-196.533\\
-196.533	-118.408\\
-118.408	-124.512\\
-124.512	-92.773\\
-92.773	-79.346\\
-79.346	-123.291\\
-123.291	-172.119\\
-172.119	-104.98\\
-104.98	-92.773\\
-92.773	-91.553\\
-91.553	-59.814\\
-59.814	-80.566\\
-80.566	-36.621\\
-36.621	-54.932\\
-54.932	-43.945\\
-43.945	-65.918\\
-65.918	-45.166\\
-45.166	-40.283\\
-40.283	-23.193\\
-23.193	-23.193\\
-23.193	-46.387\\
-46.387	-30.518\\
-30.518	-37.842\\
-37.842	-81.787\\
-81.787	-112.305\\
-112.305	-115.967\\
-115.967	-81.787\\
-81.787	-107.422\\
-107.422	-166.016\\
-166.016	-80.566\\
-80.566	-48.828\\
-48.828	-37.842\\
-37.842	-28.076\\
-28.076	-48.828\\
-48.828	-81.787\\
-81.787	-123.291\\
-123.291	-72.021\\
-72.021	-129.395\\
-129.395	-173.34\\
-173.34	-175.781\\
-175.781	-130.615\\
-130.615	-75.684\\
-75.684	-100.098\\
-100.098	-135.498\\
-135.498	-178.223\\
-178.223	-93.994\\
-93.994	-53.711\\
-53.711	-84.229\\
-84.229	-63.477\\
-63.477	-31.738\\
-31.738	-21.973\\
-21.973	-54.932\\
-54.932	-126.953\\
-126.953	-126.953\\
-126.953	-92.773\\
-92.773	-57.373\\
-57.373	-61.035\\
-61.035	-18.311\\
-18.311	-35.4\\
-35.4	-30.518\\
-30.518	-58.594\\
-58.594	-84.229\\
-84.229	-131.836\\
-131.836	-135.498\\
-135.498	-148.926\\
-148.926	-153.809\\
-153.809	-148.926\\
-148.926	-145.264\\
-145.264	-125.732\\
-125.732	-155.029\\
-155.029	-236.816\\
-236.816	-213.623\\
-213.623	-111.084\\
-111.084	-58.594\\
-58.594	-135.498\\
-135.498	-158.691\\
-158.691	-142.822\\
-142.822	-78.125\\
-78.125	-83.008\\
-83.008	-142.822\\
-142.822	-217.285\\
-217.285	-224.609\\
-224.609	-250.244\\
-250.244	-239.258\\
-239.258	-178.223\\
-178.223	-183.105\\
-183.105	-84.229\\
-84.229	-79.346\\
-79.346	-101.318\\
-101.318	-126.953\\
-126.953	-134.277\\
-134.277	-142.822\\
-142.822	-86.67\\
-86.67	-124.512\\
-124.512	-102.539\\
-102.539	-59.814\\
-59.814	-39.063\\
-39.063	-32.959\\
-32.959	-53.711\\
-53.711	-40.283\\
-40.283	-23.193\\
-23.193	-48.828\\
-48.828	-100.098\\
-100.098	-64.697\\
-64.697	-65.918\\
-65.918	-86.67\\
-86.67	-146.484\\
-146.484	-97.656\\
-97.656	-52.49\\
-52.49	-29.297\\
-29.297	-81.787\\
-81.787	-159.912\\
-159.912	-201.416\\
-201.416	-280.762\\
-280.762	-333.252\\
-333.252	-323.486\\
-323.486	-207.52\\
-207.52	-212.402\\
-212.402	-277.1\\
-277.1	-231.934\\
-231.934	-214.844\\
-214.844	-242.92\\
-242.92	-263.672\\
-263.672	-270.996\\
-270.996	-357.666\\
-357.666	-238.037\\
-238.037	-135.498\\
-135.498	-70.801\\
-70.801	-59.814\\
-59.814	-67.139\\
-67.139	-67.139\\
-67.139	-69.58\\
-69.58	-126.953\\
-126.953	-93.994\\
-93.994	-107.422\\
-107.422	-136.719\\
-136.719	-70.801\\
-70.801	-85.449\\
-85.449	-117.188\\
-117.188	-140.381\\
-140.381	-76.904\\
-76.904	-51.27\\
-51.27	-68.359\\
-68.359	-64.697\\
-64.697	-167.236\\
-167.236	-227.051\\
-227.051	-214.844\\
-214.844	-249.023\\
-249.023	-274.658\\
-274.658	-360.107\\
-360.107	-306.396\\
-306.396	-190.43\\
-190.43	-125.732\\
-125.732	-69.58\\
-69.58	-78.125\\
-78.125	-81.787\\
-81.787	-109.863\\
-109.863	-72.021\\
-72.021	-68.359\\
-68.359	-75.684\\
-75.684	-95.215\\
-95.215	-107.422\\
-107.422	-135.498\\
-135.498	-95.215\\
-95.215	-45.166\\
-45.166	-34.18\\
-34.18	-28.076\\
-28.076	-32.959\\
-32.959	-39.063\\
-39.063	-72.021\\
-72.021	-73.242\\
-73.242	-95.215\\
-95.215	-140.381\\
-140.381	-100.098\\
-100.098	-168.457\\
-168.457	-244.141\\
-244.141	-190.43\\
-190.43	-172.119\\
-172.119	-207.52\\
-207.52	-239.258\\
-239.258	-148.926\\
-148.926	-130.615\\
-130.615	-80.566\\
-80.566	-92.773\\
-92.773	-187.988\\
-187.988	-314.941\\
-314.941	-371.094\\
-371.094	-294.189\\
-294.189	-247.803\\
-247.803	-175.781\\
-175.781	-191.65\\
-191.65	-256.348\\
-256.348	-147.705\\
-147.705	-128.174\\
-128.174	-96.436\\
-96.436	-185.547\\
-185.547	-173.34\\
-173.34	-92.773\\
-92.773	-90.332\\
-90.332	-65.918\\
-65.918	-119.629\\
-119.629	-180.664\\
-180.664	-186.768\\
-186.768	-141.602\\
-141.602	-104.98\\
-104.98	-115.967\\
-115.967	-89.111\\
-89.111	-64.697\\
-64.697	-52.49\\
-52.49	-45.166\\
-45.166	-36.621\\
-36.621	-46.387\\
-46.387	-76.904\\
-76.904	-113.525\\
-113.525	-113.525\\
-113.525	-72.021\\
-72.021	-62.256\\
-62.256	-51.27\\
-51.27	-68.359\\
-68.359	-95.215\\
-95.215	-103.76\\
-103.76	-100.098\\
-100.098	-54.932\\
-54.932	-123.291\\
-123.291	-172.119\\
-172.119	-124.512\\
-124.512	-50.049\\
-50.049	-61.035\\
-61.035	-95.215\\
-95.215	-95.215\\
-95.215	-84.229\\
-84.229	-51.27\\
-51.27	-93.994\\
-93.994	-144.043\\
-144.043	-86.67\\
-86.67	-32.959\\
-32.959	-42.725\\
-42.725	-111.084\\
-111.084	-139.16\\
-139.16	-81.787\\
-81.787	-68.359\\
-68.359	-133.057\\
-133.057	-179.443\\
-179.443	-157.471\\
-157.471	-222.168\\
-222.168	-268.555\\
-268.555	-173.34\\
-173.34	-139.16\\
-139.16	-197.754\\
-197.754	-133.057\\
-133.057	-179.443\\
-179.443	-220.947\\
-220.947	-172.119\\
-172.119	-84.229\\
-84.229	-142.822\\
-142.822	-238.037\\
-238.037	-280.762\\
-280.762	-205.078\\
-205.078	-175.781\\
-175.781	-229.492\\
-229.492	-175.781\\
-175.781	-112.305\\
-112.305	-101.318\\
-101.318	-130.615\\
-130.615	-136.719\\
-136.719	-136.719\\
-136.719	-155.029\\
-155.029	-106.201\\
-106.201	-114.746\\
-114.746	-163.574\\
-163.574	-92.773\\
-92.773	-76.904\\
-76.904	-61.035\\
-61.035	-48.828\\
-48.828	-57.373\\
-57.373	-102.539\\
-102.539	-93.994\\
-93.994	-133.057\\
-133.057	-169.678\\
-169.678	-202.637\\
-202.637	-157.471\\
-157.471	-136.719\\
-136.719	-61.035\\
-61.035	-95.215\\
-95.215	-167.236\\
-167.236	-111.084\\
-111.084	-56.152\\
-56.152	-113.525\\
-113.525	-172.119\\
-172.119	-178.223\\
-178.223	-233.154\\
-233.154	-305.176\\
-305.176	-213.623\\
-213.623	-181.885\\
-181.885	-216.064\\
-216.064	-140.381\\
-140.381	-113.525\\
-113.525	-129.395\\
-129.395	-164.795\\
-164.795	-174.561\\
-174.561	-177.002\\
-177.002	-97.656\\
-97.656	-86.67\\
-86.67	-69.58\\
-69.58	-101.318\\
-101.318	-92.773\\
-92.773	-74.463\\
-74.463	-151.367\\
-151.367	-162.354\\
-162.354	-112.305\\
-112.305	-93.994\\
-93.994	-136.719\\
-136.719	-68.359\\
-68.359	-43.945\\
-43.945	-54.932\\
-54.932	-78.125\\
-78.125	-68.359\\
-68.359	-47.607\\
-47.607	-26.855\\
-26.855	-48.828\\
-48.828	-78.125\\
-78.125	-124.512\\
-124.512	-142.822\\
-142.822	-187.988\\
-187.988	-209.961\\
-209.961	-107.422\\
-107.422	-83.008\\
-83.008	-178.223\\
-178.223	-261.23\\
-261.23	-202.637\\
-202.637	-189.209\\
-189.209	-283.203\\
-283.203	-227.051\\
-227.051	-185.547\\
-185.547	-133.057\\
-133.057	-140.381\\
-140.381	-185.547\\
-185.547	-129.395\\
-129.395	-112.305\\
-112.305	-189.209\\
-189.209	-236.816\\
-236.816	-247.803\\
-247.803	-111.084\\
-111.084	-52.49\\
-52.49	-141.602\\
-141.602	-111.084\\
-111.084	-52.49\\
-52.49	-37.842\\
-37.842	-68.359\\
-68.359	-103.76\\
-103.76	-167.236\\
-167.236	-117.188\\
-117.188	-86.67\\
-86.67	-52.49\\
-52.49	-34.18\\
-34.18	-48.828\\
-48.828	-40.283\\
-40.283	-56.152\\
-56.152	-96.436\\
-96.436	-87.891\\
-87.891	-43.945\\
-43.945	-39.063\\
-39.063	-52.49\\
-52.49	-67.139\\
-67.139	-39.063\\
-39.063	-62.256\\
-62.256	-45.166\\
-45.166	-34.18\\
-34.18	-79.346\\
-79.346	-120.85\\
-120.85	-175.781\\
-175.781	-239.258\\
-239.258	-207.52\\
-207.52	-100.098\\
-100.098	-70.801\\
-70.801	-74.463\\
-74.463	-41.504\\
-41.504	-36.621\\
-36.621	-85.449\\
-85.449	-91.553\\
-91.553	-87.891\\
-87.891	-103.76\\
-103.76	-122.07\\
-122.07	-130.615\\
-130.615	-135.498\\
-135.498	-166.016\\
-166.016	-152.588\\
-152.588	-223.389\\
-223.389	-115.967\\
-115.967	-51.27\\
-51.27	-50.049\\
-50.049	-95.215\\
-95.215	-74.463\\
-74.463	-67.139\\
-67.139	-31.738\\
-31.738	-48.828\\
-48.828	-37.842\\
-37.842	-18.311\\
-18.311	-32.959\\
-32.959	-69.58\\
-69.58	-87.891\\
-87.891	-130.615\\
-130.615	-183.105\\
-183.105	-128.174\\
-128.174	-51.27\\
-51.27	-93.994\\
-93.994	-106.201\\
-106.201	-50.049\\
-50.049	-68.359\\
-68.359	-139.16\\
-139.16	-128.174\\
-128.174	-207.52\\
-207.52	-240.479\\
-240.479	-148.926\\
-148.926	-114.746\\
-114.746	-100.098\\
};
\addlegendentry{data1}

\addplot [color=mycolor2, line width=2.0pt]
  table[row sep=crcr]{%
-122.07	-116.783530728682\\
-140.381	-134.301538684551\\
-178.223	-170.504720218382\\
-147.705	-141.308359189645\\
-139.16	-133.133416369324\\
-192.871	-184.518361228571\\
-185.547	-177.511540723477\\
-219.727	-210.211312004761\\
-169.678	-162.329777398061\\
-86.67	-82.9165938253043\\
-107.422	-102.769889718494\\
-102.539	-98.0983571507197\\
-126.953	-121.455063296456\\
-103.76	-99.2664794659464\\
-41.504	-39.7065917863785\\
-31.738	-30.3635266508308\\
-28.076	-26.8601163982836\\
-89.111	-85.2518817626248\\
-157.471	-150.651424325193\\
-167.236	-159.993532767608\\
-174.561	-167.001309965835\\
-124.512	-119.119775359136\\
-74.463	-71.238240752436\\
-158.691	-151.818589947287\\
-111.084	-106.273299971041\\
-83.008	-79.4131835727571\\
-142.822	-136.636826621872\\
-123.291	-117.951653043909\\
-103.76	-99.2664794659464\\
-173.34	-165.833187650609\\
-164.795	-157.658244830287\\
-124.512	-119.119775359136\\
-180.664	-172.840008155703\\
-179.443	-171.671885840476\\
-142.822	-136.636826621872\\
-117.188	-112.112954854041\\
-89.111	-85.2518817626248\\
-108.643	-103.93801203372\\
-137.939	-131.965294054098\\
-130.615	-124.958473549003\\
-136.719	-130.798128432004\\
-140.381	-134.301538684551\\
-151.367	-144.811769442193\\
-213.623	-204.37165712176\\
-191.65	-183.350238913344\\
-126.953	-121.455063296456\\
-115.967	-110.944832538815\\
-64.697	-61.8951756168883\\
-69.58	-66.5667081846622\\
-96.436	-92.259658960852\\
-74.463	-71.238240752436\\
-78.125	-74.7416510049832\\
-111.084	-106.273299971041\\
-128.174	-122.623185611683\\
-119.629	-114.448242791362\\
-190.43	-182.183073291251\\
-139.16	-133.133416369324\\
-81.787	-78.2450612575304\\
-63.477	-60.7280099947945\\
-85.449	-81.7484715100776\\
-101.318	-96.930234835493\\
-51.27	-49.0496569219263\\
-58.594	-56.0564774270206\\
-81.787	-78.2450612575304\\
-65.918	-63.063297932115\\
-56.152	-53.7202327965673\\
-74.463	-71.238240752436\\
-86.67	-82.9165938253043\\
-139.16	-133.133416369324\\
-130.615	-124.958473549003\\
-163.574	-156.490122515061\\
-146.484	-140.140236874419\\
-170.898	-163.496943020155\\
-139.16	-133.133416369324\\
-133.057	-127.294718179457\\
-115.967	-110.944832538815\\
-111.084	-106.273299971041\\
-196.533	-188.021771481118\\
-280.762	-268.603077369102\\
-291.748	-279.113308126744\\
-280.762	-268.603077369102\\
-175.781	-168.168475587929\\
-240.479	-230.06460789795\\
-303.955	-290.791661199612\\
-317.383	-303.638136587707\\
-239.258	-228.896485582724\\
-313.721	-300.13472633516\\
-404.053	-386.554730413011\\
-289.307	-276.778020189423\\
-238.037	-227.728363267497\\
-158.691	-151.818589947287\\
-122.07	-116.783530728682\\
-102.539	-98.0983571507197\\
-129.395	-123.791307926909\\
-92.773	-88.7552920151719\\
-61.035	-58.3917653643411\\
-59.814	-57.2236430491145\\
-75.684	-72.4063630676627\\
-118.408	-113.280120476135\\
-107.422	-102.769889718494\\
-111.084	-106.273299971041\\
-146.484	-140.140236874419\\
-152.588	-145.979891757419\\
-207.52	-198.532958931893\\
-167.236	-159.993532767608\\
-208.74	-199.700124553987\\
-148.926	-142.476481504872\\
-130.615	-124.958473549003\\
-151.367	-144.811769442193\\
-108.643	-103.93801203372\\
-98.877	-94.5949468981725\\
-122.07	-116.783530728682\\
-123.291	-117.951653043909\\
-85.449	-81.7484715100776\\
-95.215	-91.0915366456253\\
-67.139	-64.2314202473417\\
-76.904	-73.5735286897565\\
-81.787	-78.2450612575304\\
-57.373	-54.8883551117939\\
-75.684	-72.4063630676627\\
-93.994	-89.9234143303986\\
-151.367	-144.811769442193\\
-152.588	-145.979891757419\\
-170.898	-163.496943020155\\
-97.656	-93.4268245829458\\
-140.381	-134.301538684551\\
-211.182	-202.03636918444\\
-161.133	-154.15483457774\\
-186.768	-178.679663038703\\
-191.65	-183.350238913344\\
-112.305	-107.441422286267\\
-75.684	-72.4063630676627\\
-107.422	-102.769889718494\\
-123.291	-117.951653043909\\
-205.078	-196.196714301439\\
-252.686	-241.742960970818\\
-184.326	-176.34341840825\\
-189.209	-181.014950976024\\
-167.236	-159.993532767608\\
-163.574	-156.490122515061\\
-162.354	-155.322956892967\\
-108.643	-103.93801203372\\
-96.436	-92.259658960852\\
-86.67	-82.9165938253043\\
-78.125	-74.7416510049832\\
-87.891	-84.084716140531\\
-133.057	-127.294718179457\\
-203.857	-195.028591986213\\
-161.133	-154.15483457774\\
-109.863	-105.105177655814\\
-92.773	-88.7552920151719\\
-89.111	-85.2518817626248\\
-52.49	-50.2168225440201\\
-41.504	-39.7065917863785\\
-47.607	-45.5452899762462\\
-90.332	-86.4200040778514\\
-67.139	-64.2314202473417\\
-78.125	-74.7416510049832\\
-85.449	-81.7484715100776\\
-80.566	-77.0769389423037\\
-54.932	-52.5530671744735\\
-37.842	-36.2031815338314\\
-30.518	-29.196361028737\\
-54.932	-52.5530671744735\\
-119.629	-114.448242791362\\
-172.119	-164.665065335382\\
-194.092	-185.686483543798\\
-136.719	-130.798128432004\\
-91.553	-87.5881263930781\\
-64.697	-61.8951756168883\\
-43.945	-42.041879723699\\
-98.877	-94.5949468981725\\
-91.553	-87.5881263930781\\
-137.939	-131.965294054098\\
-168.457	-161.161655082835\\
-275.879	-263.931544801328\\
-316.162	-302.47001427248\\
-260.01	-248.749781475913\\
-249.023	-238.238594025138\\
-159.912	-152.986712262514\\
-185.547	-177.511540723477\\
-195.313	-186.854605859024\\
-212.402	-203.203534806534\\
-158.691	-151.818589947287\\
-144.043	-137.804948937098\\
-142.822	-136.636826621872\\
-129.395	-123.791307926909\\
-156.25	-149.483302009966\\
-109.863	-105.105177655814\\
-195.313	-186.854605859024\\
-234.375	-224.22495301495\\
-175.781	-168.168475587929\\
-104.98	-100.43364508804\\
-111.084	-106.273299971041\\
-192.871	-184.518361228571\\
-234.375	-224.22495301495\\
-289.307	-276.778020189423\\
-303.955	-290.791661199612\\
-302.734	-289.623538884385\\
-228.271	-218.385298131949\\
-205.078	-196.196714301439\\
-235.596	-225.393075330176\\
-275.879	-263.931544801328\\
-170.898	-163.496943020155\\
-104.98	-100.43364508804\\
-152.588	-145.979891757419\\
-122.07	-116.783530728682\\
-79.346	-75.9097733202099\\
-109.863	-105.105177655814\\
-122.07	-116.783530728682\\
-95.215	-91.0915366456253\\
-75.684	-72.4063630676627\\
-101.318	-96.930234835493\\
-80.566	-77.0769389423037\\
-115.967	-110.944832538815\\
-162.354	-155.322956892967\\
-147.705	-141.308359189645\\
-100.098	-95.7630692133992\\
-101.318	-96.930234835493\\
-157.471	-150.651424325193\\
-261.23	-249.916947098007\\
-185.547	-177.511540723477\\
-118.408	-113.280120476135\\
-85.449	-81.7484715100776\\
-101.318	-96.930234835493\\
-90.332	-86.4200040778514\\
-67.139	-64.2314202473417\\
-76.904	-73.5735286897565\\
-92.773	-88.7552920151719\\
-120.85	-115.616365106588\\
-134.277	-128.46188380155\\
-89.111	-85.2518817626248\\
-156.25	-149.483302009966\\
-179.443	-171.671885840476\\
-170.898	-163.496943020155\\
-140.381	-134.301538684551\\
-150.146	-143.643647126966\\
-202.637	-193.861426364119\\
-125.732	-120.286940981229\\
-53.711	-51.3849448592468\\
-78.125	-74.7416510049832\\
-85.449	-81.7484715100776\\
-119.629	-114.448242791362\\
-101.318	-96.930234835493\\
-74.463	-71.238240752436\\
-118.408	-113.280120476135\\
-112.305	-107.441422286267\\
-79.346	-75.9097733202099\\
-102.539	-98.0983571507197\\
-90.332	-86.4200040778514\\
-80.566	-77.0769389423037\\
-95.215	-91.0915366456253\\
-54.932	-52.5530671744735\\
-68.359	-65.3985858694355\\
-47.607	-45.5452899762462\\
-51.27	-49.0496569219263\\
-106.201	-101.601767403267\\
-122.07	-116.783530728682\\
-167.236	-159.993532767608\\
-219.727	-210.211312004761\\
-142.822	-136.636826621872\\
-83.008	-79.4131835727571\\
-63.477	-60.7280099947945\\
-62.256	-59.5598876795678\\
-91.553	-87.5881263930781\\
-54.932	-52.5530671744735\\
-70.801	-67.7348304998889\\
-85.449	-81.7484715100776\\
-150.146	-143.643647126966\\
-133.057	-127.294718179457\\
-190.43	-182.183073291251\\
-124.512	-119.119775359136\\
-117.188	-112.112954854041\\
-56.152	-53.7202327965673\\
-62.256	-59.5598876795678\\
-34.18	-32.6997712812842\\
-46.387	-44.3781243541524\\
-51.27	-49.0496569219263\\
-95.215	-91.0915366456253\\
-100.098	-95.7630692133992\\
-117.188	-112.112954854041\\
-123.291	-117.951653043909\\
-197.754	-189.189893796345\\
-170.898	-163.496943020155\\
-128.174	-122.623185611683\\
-153.809	-147.148014072646\\
-163.574	-156.490122515061\\
-213.623	-204.37165712176\\
-161.133	-154.15483457774\\
-104.98	-100.43364508804\\
-53.711	-51.3849448592468\\
-40.283	-38.5384694711519\\
-41.504	-39.7065917863785\\
-52.49	-50.2168225440201\\
-54.932	-52.5530671744735\\
-51.27	-49.0496569219263\\
-70.801	-67.7348304998889\\
-108.643	-103.93801203372\\
-98.877	-94.5949468981725\\
-103.76	-99.2664794659464\\
-118.408	-113.280120476135\\
-69.58	-66.5667081846622\\
-36.621	-35.0350592186047\\
-81.787	-78.2450612575304\\
-125.732	-120.286940981229\\
-142.822	-136.636826621872\\
-118.408	-113.280120476135\\
-87.891	-84.084716140531\\
-123.291	-117.951653043909\\
-172.119	-164.665065335382\\
-120.85	-115.616365106588\\
-207.52	-198.532958931893\\
-155.029	-148.31517969474\\
-151.367	-144.811769442193\\
-173.34	-165.833187650609\\
-172.119	-164.665065335382\\
-130.615	-124.958473549003\\
-112.305	-107.441422286267\\
-192.871	-184.518361228571\\
-266.113	-254.58847966578\\
-202.637	-193.861426364119\\
-124.512	-119.119775359136\\
-118.408	-113.280120476135\\
-115.967	-110.944832538815\\
-150.146	-143.643647126966\\
-173.34	-165.833187650609\\
-125.732	-120.286940981229\\
-134.277	-128.46188380155\\
-177.002	-169.336597903156\\
-192.871	-184.518361228571\\
-120.85	-115.616365106588\\
-97.656	-93.4268245829458\\
-130.615	-124.958473549003\\
-218.506	-209.043189689534\\
-195.313	-186.854605859024\\
-203.857	-195.028591986213\\
-115.967	-110.944832538815\\
-108.643	-103.93801203372\\
-101.318	-96.930234835493\\
-63.477	-60.7280099947945\\
-41.504	-39.7065917863785\\
-34.18	-32.6997712812842\\
-29.297	-28.0282387135103\\
-78.125	-74.7416510049832\\
-108.643	-103.93801203372\\
-131.836	-126.12659586423\\
-93.994	-89.9234143303986\\
-107.422	-102.769889718494\\
-119.629	-114.448242791362\\
-72.021	-68.9019961219827\\
-79.346	-75.9097733202099\\
-73.242	-70.0701184372094\\
-54.932	-52.5530671744735\\
-75.684	-72.4063630676627\\
-120.85	-115.616365106588\\
-156.25	-149.483302009966\\
-95.215	-91.0915366456253\\
-73.242	-70.0701184372094\\
-58.594	-56.0564774270206\\
-74.463	-71.238240752436\\
-45.166	-43.2100020389257\\
-117.188	-112.112954854041\\
-195.313	-186.854605859024\\
-163.574	-156.490122515061\\
-216.064	-206.706945059081\\
-241.699	-231.231773520044\\
-249.023	-238.238594025138\\
-181.885	-174.00813047093\\
-131.836	-126.12659586423\\
-130.615	-124.958473549003\\
-129.395	-123.791307926909\\
-155.029	-148.31517969474\\
-184.326	-176.34341840825\\
-181.885	-174.00813047093\\
-247.803	-237.071428403045\\
-270.996	-259.260012233554\\
-328.369	-314.148367345348\\
-217.285	-207.875067374308\\
-238.037	-227.728363267497\\
-264.893	-253.421314043687\\
-205.078	-196.196714301439\\
-239.258	-228.896485582724\\
-200.195	-191.525181733665\\
-113.525	-108.608587908361\\
-74.463	-71.238240752436\\
-79.346	-75.9097733202099\\
-117.188	-112.112954854041\\
-97.656	-93.4268245829458\\
-64.697	-61.8951756168883\\
-90.332	-86.4200040778514\\
-63.477	-60.7280099947945\\
-48.828	-46.7134122914729\\
-64.697	-61.8951756168883\\
-72.021	-68.9019961219827\\
-48.828	-46.7134122914729\\
-100.098	-95.7630692133992\\
-135.498	-129.630006116777\\
-97.656	-93.4268245829458\\
-173.34	-165.833187650609\\
-241.699	-231.231773520044\\
-220.947	-211.378477626855\\
-279.541	-267.434955053875\\
-187.988	-179.846828660797\\
-206.299	-197.364836616666\\
-134.277	-128.46188380155\\
-85.449	-81.7484715100776\\
-108.643	-103.93801203372\\
-87.891	-84.084716140531\\
-63.477	-60.7280099947945\\
-92.773	-88.7552920151719\\
-50.049	-47.8815346066996\\
-32.959	-31.5316489660575\\
-43.945	-42.041879723699\\
-98.877	-94.5949468981725\\
-125.732	-120.286940981229\\
-157.471	-150.651424325193\\
-152.588	-145.979891757419\\
-146.484	-140.140236874419\\
-168.457	-161.161655082835\\
-135.498	-129.630006116777\\
-108.643	-103.93801203372\\
-112.305	-107.441422286267\\
-89.111	-85.2518817626248\\
-146.484	-140.140236874419\\
-96.436	-92.259658960852\\
-80.566	-77.0769389423037\\
-120.85	-115.616365106588\\
-164.795	-157.658244830287\\
-123.291	-117.951653043909\\
-93.994	-89.9234143303986\\
-80.566	-77.0769389423037\\
-91.553	-87.5881263930781\\
-61.035	-58.3917653643411\\
-87.891	-84.084716140531\\
-109.863	-105.105177655814\\
-124.512	-119.119775359136\\
-96.436	-92.259658960852\\
-128.174	-122.623185611683\\
-114.746	-109.776710223588\\
-61.035	-58.3917653643411\\
-69.58	-66.5667081846622\\
-135.498	-129.630006116777\\
-190.43	-182.183073291251\\
-186.768	-178.679663038703\\
-157.471	-150.651424325193\\
-152.588	-145.979891757419\\
-114.746	-109.776710223588\\
-109.863	-105.105177655814\\
-87.891	-84.084716140531\\
-125.732	-120.286940981229\\
-117.188	-112.112954854041\\
-70.801	-67.7348304998889\\
-107.422	-102.769889718494\\
-126.953	-121.455063296456\\
-150.146	-143.643647126966\\
-164.795	-157.658244830287\\
-163.574	-156.490122515061\\
-222.168	-212.546599942081\\
-267.334	-255.756601981007\\
-302.734	-289.623538884385\\
-190.43	-182.183073291251\\
-106.201	-101.601767403267\\
-68.359	-65.3985858694355\\
-45.166	-43.2100020389257\\
-70.801	-67.7348304998889\\
-64.697	-61.8951756168883\\
-119.629	-114.448242791362\\
-102.539	-98.0983571507197\\
-93.994	-89.9234143303986\\
-124.512	-119.119775359136\\
-145.264	-138.973071252325\\
-172.119	-164.665065335382\\
-183.105	-175.175296093023\\
-258.789	-247.581659160686\\
-231.934	-221.889665077629\\
-177.002	-169.336597903156\\
-120.85	-115.616365106588\\
-123.291	-117.951653043909\\
-157.471	-150.651424325193\\
-108.643	-103.93801203372\\
-113.525	-108.608587908361\\
-107.422	-102.769889718494\\
-58.594	-56.0564774270206\\
-102.539	-98.0983571507197\\
-152.588	-145.979891757419\\
-107.422	-102.769889718494\\
-115.967	-110.944832538815\\
-216.064	-206.706945059081\\
-159.912	-152.986712262514\\
-156.25	-149.483302009966\\
-219.727	-210.211312004761\\
-206.299	-197.364836616666\\
-129.395	-123.791307926909\\
-85.449	-81.7484715100776\\
-146.484	-140.140236874419\\
-236.816	-226.56024095227\\
-246.582	-235.903306087818\\
-251.465	-240.574838655592\\
-192.871	-184.518361228571\\
-122.07	-116.783530728682\\
-103.76	-99.2664794659464\\
-72.021	-68.9019961219827\\
-69.58	-66.5667081846622\\
-58.594	-56.0564774270206\\
-89.111	-85.2518817626248\\
-102.539	-98.0983571507197\\
-58.594	-56.0564774270206\\
-90.332	-86.4200040778514\\
-126.953	-121.455063296456\\
-145.264	-138.973071252325\\
-185.547	-177.511540723477\\
-233.154	-223.056830699723\\
-281.982	-269.770242991196\\
-197.754	-189.189893796345\\
-172.119	-164.665065335382\\
-178.223	-170.504720218382\\
-156.25	-149.483302009966\\
-103.76	-99.2664794659464\\
-128.174	-122.623185611683\\
-214.844	-205.539779436987\\
-246.582	-235.903306087818\\
-141.602	-135.469660999778\\
-79.346	-75.9097733202099\\
-52.49	-50.2168225440201\\
-28.076	-26.8601163982836\\
-35.4	-33.866936903378\\
-64.697	-61.8951756168883\\
-74.463	-71.238240752436\\
-102.539	-98.0983571507197\\
-81.787	-78.2450612575304\\
-64.697	-61.8951756168883\\
-62.256	-59.5598876795678\\
-61.035	-58.3917653643411\\
-54.932	-52.5530671744735\\
-65.918	-63.063297932115\\
-102.539	-98.0983571507197\\
-148.926	-142.476481504872\\
-157.471	-150.651424325193\\
-155.029	-148.31517969474\\
-183.105	-175.175296093023\\
-225.83	-216.050010194629\\
-270.996	-259.260012233554\\
-247.803	-237.071428403045\\
-184.326	-176.34341840825\\
-125.732	-120.286940981229\\
-122.07	-116.783530728682\\
-205.078	-196.196714301439\\
-236.816	-226.56024095227\\
-166.016	-158.826367145514\\
-107.422	-102.769889718494\\
-53.711	-51.3849448592468\\
-40.283	-38.5384694711519\\
-42.725	-40.8747141016052\\
-25.635	-24.5248284609631\\
-61.035	-58.3917653643411\\
-84.229	-80.5813058879838\\
-89.111	-85.2518817626248\\
-78.125	-74.7416510049832\\
-47.607	-45.5452899762462\\
-35.4	-33.866936903378\\
-52.49	-50.2168225440201\\
-56.152	-53.7202327965673\\
-62.256	-59.5598876795678\\
-47.607	-45.5452899762462\\
-37.842	-36.2031815338314\\
-69.58	-66.5667081846622\\
-164.795	-157.658244830287\\
-190.43	-182.183073291251\\
-217.285	-207.875067374308\\
-147.705	-141.308359189645\\
-208.74	-199.700124553987\\
-291.748	-279.113308126744\\
-261.23	-249.916947098007\\
-273.438	-261.596256864008\\
-200.195	-191.525181733665\\
-202.637	-193.861426364119\\
-212.402	-203.203534806534\\
-139.16	-133.133416369324\\
-133.057	-127.294718179457\\
-80.566	-77.0769389423037\\
-75.684	-72.4063630676627\\
-135.498	-129.630006116777\\
-93.994	-89.9234143303986\\
-104.98	-100.43364508804\\
-115.967	-110.944832538815\\
-111.084	-106.273299971041\\
-113.525	-108.608587908361\\
-120.85	-115.616365106588\\
-136.719	-130.798128432004\\
-147.705	-141.308359189645\\
-125.732	-120.286940981229\\
-92.773	-88.7552920151719\\
-120.85	-115.616365106588\\
-57.373	-54.8883551117939\\
-26.855	-25.691994083057\\
-46.387	-44.3781243541524\\
-73.242	-70.0701184372094\\
-37.842	-36.2031815338314\\
-17.09	-16.3498856406421\\
-39.063	-37.371303849058\\
-70.801	-67.7348304998889\\
-84.229	-80.5813058879838\\
-104.98	-100.43364508804\\
-87.891	-84.084716140531\\
-142.822	-136.636826621872\\
-123.291	-117.951653043909\\
-83.008	-79.4131835727571\\
-125.732	-120.286940981229\\
-70.801	-67.7348304998889\\
-56.152	-53.7202327965673\\
-39.063	-37.371303849058\\
-24.414	-23.3567061457365\\
-48.828	-46.7134122914729\\
-36.621	-35.0350592186047\\
-65.918	-63.063297932115\\
-111.084	-106.273299971041\\
-106.201	-101.601767403267\\
-76.904	-73.5735286897565\\
-86.67	-82.9165938253043\\
-64.697	-61.8951756168883\\
-92.773	-88.7552920151719\\
-67.139	-64.2314202473417\\
-63.477	-60.7280099947945\\
-58.594	-56.0564774270206\\
-43.945	-42.041879723699\\
-57.373	-54.8883551117939\\
-51.27	-49.0496569219263\\
-90.332	-86.4200040778514\\
-83.008	-79.4131835727571\\
-67.139	-64.2314202473417\\
-64.697	-61.8951756168883\\
-51.27	-49.0496569219263\\
-61.035	-58.3917653643411\\
-135.498	-129.630006116777\\
-179.443	-171.671885840476\\
-123.291	-117.951653043909\\
-114.746	-109.776710223588\\
-80.566	-77.0769389423037\\
-136.719	-130.798128432004\\
-125.732	-120.286940981229\\
-80.566	-77.0769389423037\\
-102.539	-98.0983571507197\\
-76.904	-73.5735286897565\\
-108.643	-103.93801203372\\
-64.697	-61.8951756168883\\
-80.566	-77.0769389423037\\
-104.98	-100.43364508804\\
-74.463	-71.238240752436\\
-81.787	-78.2450612575304\\
-85.449	-81.7484715100776\\
-135.498	-129.630006116777\\
-115.967	-110.944832538815\\
-178.223	-170.504720218382\\
-231.934	-221.889665077629\\
-183.105	-175.175296093023\\
-117.188	-112.112954854041\\
-86.67	-82.9165938253043\\
-135.498	-129.630006116777\\
-173.34	-165.833187650609\\
-133.057	-127.294718179457\\
-163.574	-156.490122515061\\
-98.877	-94.5949468981725\\
-50.049	-47.8815346066996\\
-52.49	-50.2168225440201\\
-119.629	-114.448242791362\\
-107.422	-102.769889718494\\
-100.098	-95.7630692133992\\
-156.25	-149.483302009966\\
-146.484	-140.140236874419\\
-133.057	-127.294718179457\\
-90.332	-86.4200040778514\\
-114.746	-109.776710223588\\
-148.926	-142.476481504872\\
-120.85	-115.616365106588\\
-125.732	-120.286940981229\\
-112.305	-107.441422286267\\
-80.566	-77.0769389423037\\
-97.656	-93.4268245829458\\
-70.801	-67.7348304998889\\
-79.346	-75.9097733202099\\
-135.498	-129.630006116777\\
-156.25	-149.483302009966\\
-164.795	-157.658244830287\\
-146.484	-140.140236874419\\
-104.98	-100.43364508804\\
-72.021	-68.9019961219827\\
-85.449	-81.7484715100776\\
-101.318	-96.930234835493\\
-129.395	-123.791307926909\\
-159.912	-152.986712262514\\
-190.43	-182.183073291251\\
-146.484	-140.140236874419\\
-84.229	-80.5813058879838\\
-155.029	-148.31517969474\\
-222.168	-212.546599942081\\
-155.029	-148.31517969474\\
-181.885	-174.00813047093\\
-137.939	-131.965294054098\\
-113.525	-108.608587908361\\
-129.395	-123.791307926909\\
-115.967	-110.944832538815\\
-131.836	-126.12659586423\\
-167.236	-159.993532767608\\
-125.732	-120.286940981229\\
-146.484	-140.140236874419\\
-137.939	-131.965294054098\\
-141.602	-135.469660999778\\
-109.863	-105.105177655814\\
-144.043	-137.804948937098\\
-102.539	-98.0983571507197\\
-106.201	-101.601767403267\\
-119.629	-114.448242791362\\
-115.967	-110.944832538815\\
-58.594	-56.0564774270206\\
-35.4	-33.866936903378\\
-28.076	-26.8601163982836\\
-47.607	-45.5452899762462\\
-122.07	-116.783530728682\\
-161.133	-154.15483457774\\
-101.318	-96.930234835493\\
-119.629	-114.448242791362\\
-104.98	-100.43364508804\\
-92.773	-88.7552920151719\\
-58.594	-56.0564774270206\\
-81.787	-78.2450612575304\\
-80.566	-77.0769389423037\\
-95.215	-91.0915366456253\\
-79.346	-75.9097733202099\\
-75.684	-72.4063630676627\\
-50.049	-47.8815346066996\\
-72.021	-68.9019961219827\\
-139.16	-133.133416369324\\
-96.436	-92.259658960852\\
-119.629	-114.448242791362\\
-107.422	-102.769889718494\\
-145.264	-138.973071252325\\
-135.498	-129.630006116777\\
-186.768	-178.679663038703\\
-137.939	-131.965294054098\\
-151.367	-144.811769442193\\
-155.029	-148.31517969474\\
-72.021	-68.9019961219827\\
-131.836	-126.12659586423\\
-95.215	-91.0915366456253\\
-113.525	-108.608587908361\\
-128.174	-122.623185611683\\
-166.016	-158.826367145514\\
-123.291	-117.951653043909\\
-158.691	-151.818589947287\\
-201.416	-192.693304048892\\
-164.795	-157.658244830287\\
-142.822	-136.636826621872\\
-113.525	-108.608587908361\\
-136.719	-130.798128432004\\
-113.525	-108.608587908361\\
-96.436	-92.259658960852\\
-81.787	-78.2450612575304\\
-96.436	-92.259658960852\\
-83.008	-79.4131835727571\\
-172.119	-164.665065335382\\
-225.83	-216.050010194629\\
-191.65	-183.350238913344\\
-187.988	-179.846828660797\\
-181.885	-174.00813047093\\
-101.318	-96.930234835493\\
-89.111	-85.2518817626248\\
-72.021	-68.9019961219827\\
-62.256	-59.5598876795678\\
-68.359	-65.3985858694355\\
-115.967	-110.944832538815\\
-146.484	-140.140236874419\\
-167.236	-159.993532767608\\
-141.602	-135.469660999778\\
-93.994	-89.9234143303986\\
-169.678	-162.329777398061\\
-201.416	-192.693304048892\\
-224.609	-214.881887879402\\
-178.223	-170.504720218382\\
-288.086	-275.609897874196\\
-209.961	-200.868246869213\\
-117.188	-112.112954854041\\
-84.229	-80.5813058879838\\
-69.58	-66.5667081846622\\
-125.732	-120.286940981229\\
-162.354	-155.322956892967\\
-218.506	-209.043189689534\\
-166.016	-158.826367145514\\
-126.953	-121.455063296456\\
-69.58	-66.5667081846622\\
-72.021	-68.9019961219827\\
-78.125	-74.7416510049832\\
-87.891	-84.084716140531\\
-56.152	-53.7202327965673\\
-70.801	-67.7348304998889\\
-59.814	-57.2236430491145\\
-36.621	-35.0350592186047\\
-83.008	-79.4131835727571\\
-93.994	-89.9234143303986\\
-118.408	-113.280120476135\\
-107.422	-102.769889718494\\
-112.305	-107.441422286267\\
-113.525	-108.608587908361\\
-146.484	-140.140236874419\\
-100.098	-95.7630692133992\\
-142.822	-136.636826621872\\
-169.678	-162.329777398061\\
-130.615	-124.958473549003\\
-139.16	-133.133416369324\\
-119.629	-114.448242791362\\
-135.498	-129.630006116777\\
-191.65	-183.350238913344\\
-148.926	-142.476481504872\\
-114.746	-109.776710223588\\
-128.174	-122.623185611683\\
-115.967	-110.944832538815\\
-79.346	-75.9097733202099\\
-122.07	-116.783530728682\\
-181.885	-174.00813047093\\
-170.898	-163.496943020155\\
-187.988	-179.846828660797\\
-279.541	-267.434955053875\\
-187.988	-179.846828660797\\
-159.912	-152.986712262514\\
-120.85	-115.616365106588\\
-152.588	-145.979891757419\\
-224.609	-214.881887879402\\
-148.926	-142.476481504872\\
-129.395	-123.791307926909\\
-186.768	-178.679663038703\\
-119.629	-114.448242791362\\
-68.359	-65.3985858694355\\
-87.891	-84.084716140531\\
-68.359	-65.3985858694355\\
-52.49	-50.2168225440201\\
-57.373	-54.8883551117939\\
-50.049	-47.8815346066996\\
-48.828	-46.7134122914729\\
-64.697	-61.8951756168883\\
-95.215	-91.0915366456253\\
-107.422	-102.769889718494\\
-147.705	-141.308359189645\\
-139.16	-133.133416369324\\
-166.016	-158.826367145514\\
-194.092	-185.686483543798\\
-174.561	-167.001309965835\\
-125.732	-120.286940981229\\
-134.277	-128.46188380155\\
-119.629	-114.448242791362\\
-130.615	-124.958473549003\\
-183.105	-175.175296093023\\
-140.381	-134.301538684551\\
-158.691	-151.818589947287\\
-136.719	-130.798128432004\\
-129.395	-123.791307926909\\
-201.416	-192.693304048892\\
-166.016	-158.826367145514\\
-167.236	-159.993532767608\\
-152.588	-145.979891757419\\
-91.553	-87.5881263930781\\
-58.594	-56.0564774270206\\
-46.387	-44.3781243541524\\
-84.229	-80.5813058879838\\
-79.346	-75.9097733202099\\
-106.201	-101.601767403267\\
-79.346	-75.9097733202099\\
-108.643	-103.93801203372\\
-191.65	-183.350238913344\\
-235.596	-225.393075330176\\
-185.547	-177.511540723477\\
-190.43	-182.183073291251\\
-169.678	-162.329777398061\\
-118.408	-113.280120476135\\
-133.057	-127.294718179457\\
-145.264	-138.973071252325\\
-125.732	-120.286940981229\\
-144.043	-137.804948937098\\
-181.885	-174.00813047093\\
-260.01	-248.749781475913\\
-230.713	-220.721542762402\\
-214.844	-205.539779436987\\
-239.258	-228.896485582724\\
-162.354	-155.322956892967\\
-173.34	-165.833187650609\\
-139.16	-133.133416369324\\
-141.602	-135.469660999778\\
-186.768	-178.679663038703\\
-213.623	-204.37165712176\\
-84.229	-80.5813058879838\\
-100.098	-95.7630692133992\\
-85.449	-81.7484715100776\\
-119.629	-114.448242791362\\
-123.291	-117.951653043909\\
-75.684	-72.4063630676627\\
-98.877	-94.5949468981725\\
-173.34	-165.833187650609\\
-303.955	-290.791661199612\\
-289.307	-276.778020189423\\
-261.23	-249.916947098007\\
-211.182	-202.03636918444\\
-240.479	-230.06460789795\\
-290.527	-277.945185811517\\
-217.285	-207.875067374308\\
-218.506	-209.043189689534\\
-225.83	-216.050010194629\\
-152.588	-145.979891757419\\
-131.836	-126.12659586423\\
-169.678	-162.329777398061\\
-115.967	-110.944832538815\\
-81.787	-78.2450612575304\\
-114.746	-109.776710223588\\
-84.229	-80.5813058879838\\
-98.877	-94.5949468981725\\
-95.215	-91.0915366456253\\
-115.967	-110.944832538815\\
-158.691	-151.818589947287\\
-126.953	-121.455063296456\\
-90.332	-86.4200040778514\\
-112.305	-107.441422286267\\
-129.395	-123.791307926909\\
-153.809	-147.148014072646\\
-129.395	-123.791307926909\\
-106.201	-101.601767403267\\
-167.236	-159.993532767608\\
-159.912	-152.986712262514\\
-111.084	-106.273299971041\\
-183.105	-175.175296093023\\
-168.457	-161.161655082835\\
-119.629	-114.448242791362\\
-81.787	-78.2450612575304\\
-58.594	-56.0564774270206\\
-45.166	-43.2100020389257\\
-98.877	-94.5949468981725\\
-96.436	-92.259658960852\\
-65.918	-63.063297932115\\
-47.607	-45.5452899762462\\
-57.373	-54.8883551117939\\
-102.539	-98.0983571507197\\
-119.629	-114.448242791362\\
-158.691	-151.818589947287\\
-180.664	-172.840008155703\\
-172.119	-164.665065335382\\
-181.885	-174.00813047093\\
-112.305	-107.441422286267\\
-46.387	-44.3781243541524\\
-70.801	-67.7348304998889\\
-62.256	-59.5598876795678\\
-78.125	-74.7416510049832\\
-126.953	-121.455063296456\\
-139.16	-133.133416369324\\
-201.416	-192.693304048892\\
-133.057	-127.294718179457\\
-124.512	-119.119775359136\\
-191.65	-183.350238913344\\
-142.822	-136.636826621872\\
-95.215	-91.0915366456253\\
-69.58	-66.5667081846622\\
-93.994	-89.9234143303986\\
-129.395	-123.791307926909\\
-190.43	-182.183073291251\\
-130.615	-124.958473549003\\
-106.201	-101.601767403267\\
-98.877	-94.5949468981725\\
-122.07	-116.783530728682\\
-163.574	-156.490122515061\\
-255.127	-244.078248908139\\
-219.727	-210.211312004761\\
-214.844	-205.539779436987\\
-137.939	-131.965294054098\\
-90.332	-86.4200040778514\\
-107.422	-102.769889718494\\
-50.049	-47.8815346066996\\
-74.463	-71.238240752436\\
-69.58	-66.5667081846622\\
-79.346	-75.9097733202099\\
-134.277	-128.46188380155\\
-185.547	-177.511540723477\\
-217.285	-207.875067374308\\
-278.32	-266.266832738649\\
-239.258	-228.896485582724\\
-129.395	-123.791307926909\\
-118.408	-113.280120476135\\
-98.877	-94.5949468981725\\
-136.719	-130.798128432004\\
-179.443	-171.671885840476\\
-239.258	-228.896485582724\\
-179.443	-171.671885840476\\
-122.07	-116.783530728682\\
-139.16	-133.133416369324\\
-184.326	-176.34341840825\\
-133.057	-127.294718179457\\
-89.111	-85.2518817626248\\
-73.242	-70.0701184372094\\
-50.049	-47.8815346066996\\
-48.828	-46.7134122914729\\
-36.621	-35.0350592186047\\
-64.697	-61.8951756168883\\
-98.877	-94.5949468981725\\
-107.422	-102.769889718494\\
-80.566	-77.0769389423037\\
-73.242	-70.0701184372094\\
-34.18	-32.6997712812842\\
-17.09	-16.3498856406421\\
-30.518	-29.196361028737\\
-57.373	-54.8883551117939\\
-87.891	-84.084716140531\\
-111.084	-106.273299971041\\
-130.615	-124.958473549003\\
-150.146	-143.643647126966\\
-190.43	-182.183073291251\\
-266.113	-254.58847966578\\
-261.23	-249.916947098007\\
-280.762	-268.603077369102\\
-249.023	-238.238594025138\\
-136.719	-130.798128432004\\
-123.291	-117.951653043909\\
-125.732	-120.286940981229\\
-166.016	-158.826367145514\\
-145.264	-138.973071252325\\
-102.539	-98.0983571507197\\
-79.346	-75.9097733202099\\
-63.477	-60.7280099947945\\
-111.084	-106.273299971041\\
-108.643	-103.93801203372\\
-56.152	-53.7202327965673\\
-41.504	-39.7065917863785\\
-69.58	-66.5667081846622\\
-96.436	-92.259658960852\\
-76.904	-73.5735286897565\\
-106.201	-101.601767403267\\
-86.67	-82.9165938253043\\
-122.07	-116.783530728682\\
-181.885	-174.00813047093\\
-146.484	-140.140236874419\\
-190.43	-182.183073291251\\
-260.01	-248.749781475913\\
-281.982	-269.770242991196\\
-223.389	-213.714722257308\\
-145.264	-138.973071252325\\
-107.422	-102.769889718494\\
-64.697	-61.8951756168883\\
-101.318	-96.930234835493\\
-75.684	-72.4063630676627\\
-126.953	-121.455063296456\\
-115.967	-110.944832538815\\
-92.773	-88.7552920151719\\
-119.629	-114.448242791362\\
-69.58	-66.5667081846622\\
-98.877	-94.5949468981725\\
-75.684	-72.4063630676627\\
-46.387	-44.3781243541524\\
-54.932	-52.5530671744735\\
-42.725	-40.8747141016052\\
-47.607	-45.5452899762462\\
-46.387	-44.3781243541524\\
-30.518	-29.196361028737\\
-42.725	-40.8747141016052\\
-62.256	-59.5598876795678\\
-61.035	-58.3917653643411\\
-97.656	-93.4268245829458\\
-130.615	-124.958473549003\\
-108.643	-103.93801203372\\
-98.877	-94.5949468981725\\
-156.25	-149.483302009966\\
-190.43	-182.183073291251\\
-150.146	-143.643647126966\\
-156.25	-149.483302009966\\
-74.463	-71.238240752436\\
-43.945	-42.041879723699\\
-90.332	-86.4200040778514\\
-133.057	-127.294718179457\\
-145.264	-138.973071252325\\
-202.637	-193.861426364119\\
-183.105	-175.175296093023\\
-130.615	-124.958473549003\\
-89.111	-85.2518817626248\\
-93.994	-89.9234143303986\\
-102.539	-98.0983571507197\\
-81.787	-78.2450612575304\\
-48.828	-46.7134122914729\\
-64.697	-61.8951756168883\\
-52.49	-50.2168225440201\\
-41.504	-39.7065917863785\\
-32.959	-31.5316489660575\\
-61.035	-58.3917653643411\\
-93.994	-89.9234143303986\\
-100.098	-95.7630692133992\\
-69.58	-66.5667081846622\\
-123.291	-117.951653043909\\
-172.119	-164.665065335382\\
-109.863	-105.105177655814\\
-146.484	-140.140236874419\\
-163.574	-156.490122515061\\
-118.408	-113.280120476135\\
-69.58	-66.5667081846622\\
-86.67	-82.9165938253043\\
-108.643	-103.93801203372\\
-59.814	-57.2236430491145\\
-69.58	-66.5667081846622\\
-53.711	-51.3849448592468\\
-31.738	-30.3635266508308\\
-56.152	-53.7202327965673\\
-76.904	-73.5735286897565\\
-67.139	-64.2314202473417\\
-83.008	-79.4131835727571\\
-100.098	-95.7630692133992\\
-73.242	-70.0701184372094\\
-37.842	-36.2031815338314\\
-32.959	-31.5316489660575\\
-41.504	-39.7065917863785\\
-50.049	-47.8815346066996\\
-61.035	-58.3917653643411\\
-128.174	-122.623185611683\\
-130.615	-124.958473549003\\
-109.863	-105.105177655814\\
-107.422	-102.769889718494\\
-141.602	-135.469660999778\\
-185.547	-177.511540723477\\
-201.416	-192.693304048892\\
-205.078	-196.196714301439\\
-151.367	-144.811769442193\\
-157.471	-150.651424325193\\
-162.354	-155.322956892967\\
-166.016	-158.826367145514\\
-185.547	-177.511540723477\\
-246.582	-235.903306087818\\
-216.064	-206.706945059081\\
-195.313	-186.854605859024\\
-157.471	-150.651424325193\\
-147.705	-141.308359189645\\
-142.822	-136.636826621872\\
-167.236	-159.993532767608\\
-103.76	-99.2664794659464\\
-120.85	-115.616365106588\\
-148.926	-142.476481504872\\
-179.443	-171.671885840476\\
-137.939	-131.965294054098\\
-155.029	-148.31517969474\\
-129.395	-123.791307926909\\
-112.305	-107.441422286267\\
-70.801	-67.7348304998889\\
-111.084	-106.273299971041\\
-177.002	-169.336597903156\\
-141.602	-135.469660999778\\
-81.787	-78.2450612575304\\
-97.656	-93.4268245829458\\
-53.711	-51.3849448592468\\
-50.049	-47.8815346066996\\
-70.801	-67.7348304998889\\
-45.166	-43.2100020389257\\
-46.387	-44.3781243541524\\
-54.932	-52.5530671744735\\
-36.621	-35.0350592186047\\
-62.256	-59.5598876795678\\
-51.27	-49.0496569219263\\
-30.518	-29.196361028737\\
-57.373	-54.8883551117939\\
-64.697	-61.8951756168883\\
-80.566	-77.0769389423037\\
-75.684	-72.4063630676627\\
-76.904	-73.5735286897565\\
-104.98	-100.43364508804\\
-92.773	-88.7552920151719\\
-104.98	-100.43364508804\\
-180.664	-172.840008155703\\
-128.174	-122.623185611683\\
-112.305	-107.441422286267\\
-122.07	-116.783530728682\\
-86.67	-82.9165938253043\\
-51.27	-49.0496569219263\\
-92.773	-88.7552920151719\\
-73.242	-70.0701184372094\\
-43.945	-42.041879723699\\
-81.787	-78.2450612575304\\
-80.566	-77.0769389423037\\
-98.877	-94.5949468981725\\
-61.035	-58.3917653643411\\
-36.621	-35.0350592186047\\
-29.297	-28.0282387135103\\
-36.621	-35.0350592186047\\
-69.58	-66.5667081846622\\
-68.359	-65.3985858694355\\
-80.566	-77.0769389423037\\
-109.863	-105.105177655814\\
-150.146	-143.643647126966\\
-109.863	-105.105177655814\\
-152.588	-145.979891757419\\
-177.002	-169.336597903156\\
-153.809	-147.148014072646\\
-145.264	-138.973071252325\\
-230.713	-220.721542762402\\
-167.236	-159.993532767608\\
-208.74	-199.700124553987\\
-178.223	-170.504720218382\\
-203.857	-195.028591986213\\
-234.375	-224.22495301495\\
-136.719	-130.798128432004\\
-81.787	-78.2450612575304\\
-65.918	-63.063297932115\\
-108.643	-103.93801203372\\
-134.277	-128.46188380155\\
-137.939	-131.965294054098\\
-90.332	-86.4200040778514\\
-50.049	-47.8815346066996\\
-67.139	-64.2314202473417\\
-128.174	-122.623185611683\\
-179.443	-171.671885840476\\
-177.002	-169.336597903156\\
-103.76	-99.2664794659464\\
-81.787	-78.2450612575304\\
-109.863	-105.105177655814\\
-175.781	-168.168475587929\\
-200.195	-191.525181733665\\
-203.857	-195.028591986213\\
-150.146	-143.643647126966\\
-208.74	-199.700124553987\\
-227.051	-217.218132509855\\
-212.402	-203.203534806534\\
-290.527	-277.945185811517\\
-251.465	-240.574838655592\\
-209.961	-200.868246869213\\
-241.699	-231.231773520044\\
-207.52	-198.532958931893\\
-106.201	-101.601767403267\\
-100.098	-95.7630692133992\\
-126.953	-121.455063296456\\
-155.029	-148.31517969474\\
-158.691	-151.818589947287\\
-115.967	-110.944832538815\\
-148.926	-142.476481504872\\
-130.615	-124.958473549003\\
-96.436	-92.259658960852\\
-129.395	-123.791307926909\\
-122.07	-116.783530728682\\
-175.781	-168.168475587929\\
-181.885	-174.00813047093\\
-134.277	-128.46188380155\\
-97.656	-93.4268245829458\\
-58.594	-56.0564774270206\\
-96.436	-92.259658960852\\
-74.463	-71.238240752436\\
-65.918	-63.063297932115\\
-53.711	-51.3849448592468\\
-95.215	-91.0915366456253\\
-123.291	-117.951653043909\\
-137.939	-131.965294054098\\
-136.719	-130.798128432004\\
-80.566	-77.0769389423037\\
-62.256	-59.5598876795678\\
-42.725	-40.8747141016052\\
-54.932	-52.5530671744735\\
-91.553	-87.5881263930781\\
-95.215	-91.0915366456253\\
-91.553	-87.5881263930781\\
-150.146	-143.643647126966\\
-174.561	-167.001309965835\\
-198.975	-190.358016111572\\
-229.492	-219.553420447176\\
-137.939	-131.965294054098\\
-113.525	-108.608587908361\\
-87.891	-84.084716140531\\
-93.994	-89.9234143303986\\
-128.174	-122.623185611683\\
-106.201	-101.601767403267\\
-69.58	-66.5667081846622\\
-64.697	-61.8951756168883\\
-124.512	-119.119775359136\\
-153.809	-147.148014072646\\
-134.277	-128.46188380155\\
-211.182	-202.03636918444\\
-230.713	-220.721542762402\\
-159.912	-152.986712262514\\
-111.084	-106.273299971041\\
-76.904	-73.5735286897565\\
-70.801	-67.7348304998889\\
-131.836	-126.12659586423\\
-106.201	-101.601767403267\\
-122.07	-116.783530728682\\
-139.16	-133.133416369324\\
-156.25	-149.483302009966\\
-151.367	-144.811769442193\\
-168.457	-161.161655082835\\
-115.967	-110.944832538815\\
-130.615	-124.958473549003\\
-91.553	-87.5881263930781\\
-84.229	-80.5813058879838\\
-129.395	-123.791307926909\\
-183.105	-175.175296093023\\
-201.416	-192.693304048892\\
-106.201	-101.601767403267\\
-220.947	-211.378477626855\\
-207.52	-198.532958931893\\
-139.16	-133.133416369324\\
-76.904	-73.5735286897565\\
-123.291	-117.951653043909\\
-107.422	-102.769889718494\\
-104.98	-100.43364508804\\
-123.291	-117.951653043909\\
-80.566	-77.0769389423037\\
-108.643	-103.93801203372\\
-207.52	-198.532958931893\\
-219.727	-210.211312004761\\
-145.264	-138.973071252325\\
-159.912	-152.986712262514\\
-177.002	-169.336597903156\\
-180.664	-172.840008155703\\
-131.836	-126.12659586423\\
-128.174	-122.623185611683\\
-84.229	-80.5813058879838\\
-133.057	-127.294718179457\\
-134.277	-128.46188380155\\
-229.492	-219.553420447176\\
-251.465	-240.574838655592\\
-332.031	-317.651777597895\\
-238.037	-227.728363267497\\
-194.092	-185.686483543798\\
-147.705	-141.308359189645\\
-162.354	-155.322956892967\\
-122.07	-116.783530728682\\
-84.229	-80.5813058879838\\
-81.787	-78.2450612575304\\
-85.449	-81.7484715100776\\
-62.256	-59.5598876795678\\
-46.387	-44.3781243541524\\
-74.463	-71.238240752436\\
-100.098	-95.7630692133992\\
-68.359	-65.3985858694355\\
-46.387	-44.3781243541524\\
-112.305	-107.441422286267\\
-157.471	-150.651424325193\\
-73.242	-70.0701184372094\\
-95.215	-91.0915366456253\\
-97.656	-93.4268245829458\\
-91.553	-87.5881263930781\\
-113.525	-108.608587908361\\
-98.877	-94.5949468981725\\
-68.359	-65.3985858694355\\
-48.828	-46.7134122914729\\
-40.283	-38.5384694711519\\
-53.711	-51.3849448592468\\
-106.201	-101.601767403267\\
-126.953	-121.455063296456\\
-96.436	-92.259658960852\\
-63.477	-60.7280099947945\\
-39.063	-37.371303849058\\
-56.152	-53.7202327965673\\
-96.436	-92.259658960852\\
-123.291	-117.951653043909\\
-133.057	-127.294718179457\\
-90.332	-86.4200040778514\\
-86.67	-82.9165938253043\\
-95.215	-91.0915366456253\\
-133.057	-127.294718179457\\
-185.547	-177.511540723477\\
-216.064	-206.706945059081\\
-164.795	-157.658244830287\\
-102.539	-98.0983571507197\\
-107.422	-102.769889718494\\
-100.098	-95.7630692133992\\
-103.76	-99.2664794659464\\
-65.918	-63.063297932115\\
-87.891	-84.084716140531\\
-93.994	-89.9234143303986\\
-91.553	-87.5881263930781\\
-56.152	-53.7202327965673\\
-36.621	-35.0350592186047\\
-56.152	-53.7202327965673\\
-84.229	-80.5813058879838\\
-134.277	-128.46188380155\\
-148.926	-142.476481504872\\
-179.443	-171.671885840476\\
-159.912	-152.986712262514\\
-180.664	-172.840008155703\\
-239.258	-228.896485582724\\
-319.824	-305.973424525027\\
-253.906	-242.910126592912\\
-173.34	-165.833187650609\\
-189.209	-181.014950976024\\
-220.947	-211.378477626855\\
-288.086	-275.609897874196\\
-180.664	-172.840008155703\\
-89.111	-85.2518817626248\\
-59.814	-57.2236430491145\\
-62.256	-59.5598876795678\\
-42.725	-40.8747141016052\\
-28.076	-26.8601163982836\\
-39.063	-37.371303849058\\
-59.814	-57.2236430491145\\
-83.008	-79.4131835727571\\
-86.67	-82.9165938253043\\
-65.918	-63.063297932115\\
-62.256	-59.5598876795678\\
-78.125	-74.7416510049832\\
-100.098	-95.7630692133992\\
-104.98	-100.43364508804\\
-96.436	-92.259658960852\\
-68.359	-65.3985858694355\\
-47.607	-45.5452899762462\\
-45.166	-43.2100020389257\\
-72.021	-68.9019961219827\\
-87.891	-84.084716140531\\
-64.697	-61.8951756168883\\
-41.504	-39.7065917863785\\
-76.904	-73.5735286897565\\
-48.828	-46.7134122914729\\
-69.58	-66.5667081846622\\
-79.346	-75.9097733202099\\
-120.85	-115.616365106588\\
-126.953	-121.455063296456\\
-84.229	-80.5813058879838\\
-125.732	-120.286940981229\\
-123.291	-117.951653043909\\
-101.318	-96.930234835493\\
-84.229	-80.5813058879838\\
-142.822	-136.636826621872\\
-175.781	-168.168475587929\\
-194.092	-185.686483543798\\
-147.705	-141.308359189645\\
-136.719	-130.798128432004\\
-126.953	-121.455063296456\\
-173.34	-165.833187650609\\
-135.498	-129.630006116777\\
-184.326	-176.34341840825\\
-172.119	-164.665065335382\\
-178.223	-170.504720218382\\
-303.955	-290.791661199612\\
-231.934	-221.889665077629\\
-125.732	-120.286940981229\\
-83.008	-79.4131835727571\\
-87.891	-84.084716140531\\
-112.305	-107.441422286267\\
-146.484	-140.140236874419\\
-172.119	-164.665065335382\\
-189.209	-181.014950976024\\
-192.871	-184.518361228571\\
-125.732	-120.286940981229\\
-111.084	-106.273299971041\\
-130.615	-124.958473549003\\
-129.395	-123.791307926909\\
-76.904	-73.5735286897565\\
-57.373	-54.8883551117939\\
-103.76	-99.2664794659464\\
-101.318	-96.930234835493\\
-142.822	-136.636826621872\\
-174.561	-167.001309965835\\
-155.029	-148.31517969474\\
-106.201	-101.601767403267\\
-85.449	-81.7484715100776\\
-135.498	-129.630006116777\\
-168.457	-161.161655082835\\
-230.713	-220.721542762402\\
-244.141	-233.568018150497\\
-286.865	-274.44177555897\\
-236.816	-226.56024095227\\
-216.064	-206.706945059081\\
-123.291	-117.951653043909\\
-85.449	-81.7484715100776\\
-153.809	-147.148014072646\\
-202.637	-193.861426364119\\
-115.967	-110.944832538815\\
-186.768	-178.679663038703\\
-249.023	-238.238594025138\\
-308.838	-295.463193767386\\
-189.209	-181.014950976024\\
-112.305	-107.441422286267\\
-63.477	-60.7280099947945\\
-95.215	-91.0915366456253\\
-85.449	-81.7484715100776\\
-93.994	-89.9234143303986\\
-51.27	-49.0496569219263\\
-76.904	-73.5735286897565\\
-52.49	-50.2168225440201\\
-74.463	-71.238240752436\\
-58.594	-56.0564774270206\\
-79.346	-75.9097733202099\\
-124.512	-119.119775359136\\
-112.305	-107.441422286267\\
-162.354	-155.322956892967\\
-191.65	-183.350238913344\\
-190.43	-182.183073291251\\
-103.76	-99.2664794659464\\
-109.863	-105.105177655814\\
-130.615	-124.958473549003\\
-85.449	-81.7484715100776\\
-80.566	-77.0769389423037\\
-54.932	-52.5530671744735\\
-90.332	-86.4200040778514\\
-79.346	-75.9097733202099\\
-47.607	-45.5452899762462\\
-26.855	-25.691994083057\\
-31.738	-30.3635266508308\\
-58.594	-56.0564774270206\\
-85.449	-81.7484715100776\\
-91.553	-87.5881263930781\\
-56.152	-53.7202327965673\\
-67.139	-64.2314202473417\\
-102.539	-98.0983571507197\\
-135.498	-129.630006116777\\
-93.994	-89.9234143303986\\
-87.891	-84.084716140531\\
-104.98	-100.43364508804\\
-125.732	-120.286940981229\\
-115.967	-110.944832538815\\
-137.939	-131.965294054098\\
-126.953	-121.455063296456\\
-89.111	-85.2518817626248\\
-72.021	-68.9019961219827\\
-95.215	-91.0915366456253\\
-78.125	-74.7416510049832\\
-97.656	-93.4268245829458\\
-129.395	-123.791307926909\\
-203.857	-195.028591986213\\
-139.16	-133.133416369324\\
-140.381	-134.301538684551\\
-206.299	-197.364836616666\\
-152.588	-145.979891757419\\
-96.436	-92.259658960852\\
-69.58	-66.5667081846622\\
-54.932	-52.5530671744735\\
-58.594	-56.0564774270206\\
-46.387	-44.3781243541524\\
-50.049	-47.8815346066996\\
-106.201	-101.601767403267\\
-45.166	-43.2100020389257\\
-50.049	-47.8815346066996\\
-65.918	-63.063297932115\\
-93.994	-89.9234143303986\\
-69.58	-66.5667081846622\\
-53.711	-51.3849448592468\\
-28.076	-26.8601163982836\\
-54.932	-52.5530671744735\\
-101.318	-96.930234835493\\
-139.16	-133.133416369324\\
-108.643	-103.93801203372\\
-89.111	-85.2518817626248\\
-101.318	-96.930234835493\\
-102.539	-98.0983571507197\\
-119.629	-114.448242791362\\
-144.043	-137.804948937098\\
-168.457	-161.161655082835\\
-164.795	-157.658244830287\\
-140.381	-134.301538684551\\
-91.553	-87.5881263930781\\
-73.242	-70.0701184372094\\
-79.346	-75.9097733202099\\
-112.305	-107.441422286267\\
-64.697	-61.8951756168883\\
-62.256	-59.5598876795678\\
-114.746	-109.776710223588\\
-139.16	-133.133416369324\\
-133.057	-127.294718179457\\
-157.471	-150.651424325193\\
-205.078	-196.196714301439\\
-104.98	-100.43364508804\\
-184.326	-176.34341840825\\
-185.547	-177.511540723477\\
-170.898	-163.496943020155\\
-184.326	-176.34341840825\\
-181.885	-174.00813047093\\
-312.5	-298.966604019933\\
-286.865	-274.44177555897\\
-195.313	-186.854605859024\\
-236.816	-226.56024095227\\
-283.203	-270.938365306423\\
-275.879	-263.931544801328\\
-311.279	-297.798481704706\\
-281.982	-269.770242991196\\
-372.314	-356.190247069047\\
-280.762	-268.603077369102\\
-196.533	-188.021771481118\\
-118.408	-113.280120476135\\
-124.512	-119.119775359136\\
-92.773	-88.7552920151719\\
-79.346	-75.9097733202099\\
-123.291	-117.951653043909\\
-172.119	-164.665065335382\\
-104.98	-100.43364508804\\
-92.773	-88.7552920151719\\
-91.553	-87.5881263930781\\
-59.814	-57.2236430491145\\
-80.566	-77.0769389423037\\
-36.621	-35.0350592186047\\
-54.932	-52.5530671744735\\
-43.945	-42.041879723699\\
-65.918	-63.063297932115\\
-45.166	-43.2100020389257\\
-40.283	-38.5384694711519\\
-23.193	-22.1885838305098\\
-46.387	-44.3781243541524\\
-30.518	-29.196361028737\\
-37.842	-36.2031815338314\\
-81.787	-78.2450612575304\\
-112.305	-107.441422286267\\
-115.967	-110.944832538815\\
-81.787	-78.2450612575304\\
-107.422	-102.769889718494\\
-166.016	-158.826367145514\\
-80.566	-77.0769389423037\\
-48.828	-46.7134122914729\\
-37.842	-36.2031815338314\\
-28.076	-26.8601163982836\\
-48.828	-46.7134122914729\\
-81.787	-78.2450612575304\\
-123.291	-117.951653043909\\
-72.021	-68.9019961219827\\
-129.395	-123.791307926909\\
-173.34	-165.833187650609\\
-175.781	-168.168475587929\\
-130.615	-124.958473549003\\
-75.684	-72.4063630676627\\
-100.098	-95.7630692133992\\
-135.498	-129.630006116777\\
-178.223	-170.504720218382\\
-93.994	-89.9234143303986\\
-53.711	-51.3849448592468\\
-84.229	-80.5813058879838\\
-63.477	-60.7280099947945\\
-31.738	-30.3635266508308\\
-21.973	-21.021418208416\\
-54.932	-52.5530671744735\\
-126.953	-121.455063296456\\
-92.773	-88.7552920151719\\
-57.373	-54.8883551117939\\
-61.035	-58.3917653643411\\
-18.311	-17.5180079558688\\
-35.4	-33.866936903378\\
-30.518	-29.196361028737\\
-58.594	-56.0564774270206\\
-84.229	-80.5813058879838\\
-131.836	-126.12659586423\\
-135.498	-129.630006116777\\
-148.926	-142.476481504872\\
-153.809	-147.148014072646\\
-148.926	-142.476481504872\\
-145.264	-138.973071252325\\
-125.732	-120.286940981229\\
-155.029	-148.31517969474\\
-236.816	-226.56024095227\\
-213.623	-204.37165712176\\
-111.084	-106.273299971041\\
-58.594	-56.0564774270206\\
-135.498	-129.630006116777\\
-158.691	-151.818589947287\\
-142.822	-136.636826621872\\
-78.125	-74.7416510049832\\
-83.008	-79.4131835727571\\
-142.822	-136.636826621872\\
-217.285	-207.875067374308\\
-224.609	-214.881887879402\\
-250.244	-239.406716340365\\
-239.258	-228.896485582724\\
-178.223	-170.504720218382\\
-183.105	-175.175296093023\\
-84.229	-80.5813058879838\\
-79.346	-75.9097733202099\\
-101.318	-96.930234835493\\
-126.953	-121.455063296456\\
-134.277	-128.46188380155\\
-142.822	-136.636826621872\\
-86.67	-82.9165938253043\\
-124.512	-119.119775359136\\
-102.539	-98.0983571507197\\
-59.814	-57.2236430491145\\
-39.063	-37.371303849058\\
-32.959	-31.5316489660575\\
-53.711	-51.3849448592468\\
-40.283	-38.5384694711519\\
-23.193	-22.1885838305098\\
-48.828	-46.7134122914729\\
-100.098	-95.7630692133992\\
-64.697	-61.8951756168883\\
-65.918	-63.063297932115\\
-86.67	-82.9165938253043\\
-146.484	-140.140236874419\\
-97.656	-93.4268245829458\\
-52.49	-50.2168225440201\\
-29.297	-28.0282387135103\\
-81.787	-78.2450612575304\\
-159.912	-152.986712262514\\
-201.416	-192.693304048892\\
-280.762	-268.603077369102\\
-333.252	-318.819899913122\\
-323.486	-309.476834777574\\
-207.52	-198.532958931893\\
-212.402	-203.203534806534\\
-277.1	-265.099667116555\\
-231.934	-221.889665077629\\
-214.844	-205.539779436987\\
-242.92	-232.399895835271\\
-263.672	-252.25319172846\\
-270.996	-259.260012233554\\
-357.666	-342.176606058859\\
-238.037	-227.728363267497\\
-135.498	-129.630006116777\\
-70.801	-67.7348304998889\\
-59.814	-57.2236430491145\\
-67.139	-64.2314202473417\\
-69.58	-66.5667081846622\\
-126.953	-121.455063296456\\
-93.994	-89.9234143303986\\
-107.422	-102.769889718494\\
-136.719	-130.798128432004\\
-70.801	-67.7348304998889\\
-85.449	-81.7484715100776\\
-117.188	-112.112954854041\\
-140.381	-134.301538684551\\
-76.904	-73.5735286897565\\
-51.27	-49.0496569219263\\
-68.359	-65.3985858694355\\
-64.697	-61.8951756168883\\
-167.236	-159.993532767608\\
-227.051	-217.218132509855\\
-214.844	-205.539779436987\\
-249.023	-238.238594025138\\
-274.658	-262.763422486102\\
-360.107	-344.511893996179\\
-306.396	-293.126949136932\\
-190.43	-182.183073291251\\
-125.732	-120.286940981229\\
-69.58	-66.5667081846622\\
-78.125	-74.7416510049832\\
-81.787	-78.2450612575304\\
-109.863	-105.105177655814\\
-72.021	-68.9019961219827\\
-68.359	-65.3985858694355\\
-75.684	-72.4063630676627\\
-95.215	-91.0915366456253\\
-107.422	-102.769889718494\\
-135.498	-129.630006116777\\
-95.215	-91.0915366456253\\
-45.166	-43.2100020389257\\
-34.18	-32.6997712812842\\
-28.076	-26.8601163982836\\
-32.959	-31.5316489660575\\
-39.063	-37.371303849058\\
-72.021	-68.9019961219827\\
-73.242	-70.0701184372094\\
-95.215	-91.0915366456253\\
-140.381	-134.301538684551\\
-100.098	-95.7630692133992\\
-168.457	-161.161655082835\\
-244.141	-233.568018150497\\
-190.43	-182.183073291251\\
-172.119	-164.665065335382\\
-207.52	-198.532958931893\\
-239.258	-228.896485582724\\
-148.926	-142.476481504872\\
-130.615	-124.958473549003\\
-80.566	-77.0769389423037\\
-92.773	-88.7552920151719\\
-187.988	-179.846828660797\\
-314.941	-301.301891957253\\
-371.094	-355.023081446953\\
-294.189	-281.448596064064\\
-247.803	-237.071428403045\\
-175.781	-168.168475587929\\
-191.65	-183.350238913344\\
-256.348	-245.246371223366\\
-147.705	-141.308359189645\\
-128.174	-122.623185611683\\
-96.436	-92.259658960852\\
-185.547	-177.511540723477\\
-173.34	-165.833187650609\\
-92.773	-88.7552920151719\\
-90.332	-86.4200040778514\\
-65.918	-63.063297932115\\
-119.629	-114.448242791362\\
-180.664	-172.840008155703\\
-186.768	-178.679663038703\\
-141.602	-135.469660999778\\
-104.98	-100.43364508804\\
-115.967	-110.944832538815\\
-89.111	-85.2518817626248\\
-64.697	-61.8951756168883\\
-52.49	-50.2168225440201\\
-45.166	-43.2100020389257\\
-36.621	-35.0350592186047\\
-46.387	-44.3781243541524\\
-76.904	-73.5735286897565\\
-113.525	-108.608587908361\\
-72.021	-68.9019961219827\\
-62.256	-59.5598876795678\\
-51.27	-49.0496569219263\\
-68.359	-65.3985858694355\\
-95.215	-91.0915366456253\\
-103.76	-99.2664794659464\\
-100.098	-95.7630692133992\\
-54.932	-52.5530671744735\\
-123.291	-117.951653043909\\
-172.119	-164.665065335382\\
-124.512	-119.119775359136\\
-50.049	-47.8815346066996\\
-61.035	-58.3917653643411\\
-95.215	-91.0915366456253\\
-84.229	-80.5813058879838\\
-51.27	-49.0496569219263\\
-93.994	-89.9234143303986\\
-144.043	-137.804948937098\\
-86.67	-82.9165938253043\\
-32.959	-31.5316489660575\\
-42.725	-40.8747141016052\\
-111.084	-106.273299971041\\
-139.16	-133.133416369324\\
-81.787	-78.2450612575304\\
-68.359	-65.3985858694355\\
-133.057	-127.294718179457\\
-179.443	-171.671885840476\\
-157.471	-150.651424325193\\
-222.168	-212.546599942081\\
-268.555	-256.924724296234\\
-173.34	-165.833187650609\\
-139.16	-133.133416369324\\
-197.754	-189.189893796345\\
-133.057	-127.294718179457\\
-179.443	-171.671885840476\\
-220.947	-211.378477626855\\
-172.119	-164.665065335382\\
-84.229	-80.5813058879838\\
-142.822	-136.636826621872\\
-238.037	-227.728363267497\\
-280.762	-268.603077369102\\
-205.078	-196.196714301439\\
-175.781	-168.168475587929\\
-229.492	-219.553420447176\\
-175.781	-168.168475587929\\
-112.305	-107.441422286267\\
-101.318	-96.930234835493\\
-130.615	-124.958473549003\\
-136.719	-130.798128432004\\
-155.029	-148.31517969474\\
-106.201	-101.601767403267\\
-114.746	-109.776710223588\\
-163.574	-156.490122515061\\
-92.773	-88.7552920151719\\
-76.904	-73.5735286897565\\
-61.035	-58.3917653643411\\
-48.828	-46.7134122914729\\
-57.373	-54.8883551117939\\
-102.539	-98.0983571507197\\
-93.994	-89.9234143303986\\
-133.057	-127.294718179457\\
-169.678	-162.329777398061\\
-202.637	-193.861426364119\\
-157.471	-150.651424325193\\
-136.719	-130.798128432004\\
-61.035	-58.3917653643411\\
-95.215	-91.0915366456253\\
-167.236	-159.993532767608\\
-111.084	-106.273299971041\\
-56.152	-53.7202327965673\\
-113.525	-108.608587908361\\
-172.119	-164.665065335382\\
-178.223	-170.504720218382\\
-233.154	-223.056830699723\\
-305.176	-291.959783514838\\
-213.623	-204.37165712176\\
-181.885	-174.00813047093\\
-216.064	-206.706945059081\\
-140.381	-134.301538684551\\
-113.525	-108.608587908361\\
-129.395	-123.791307926909\\
-164.795	-157.658244830287\\
-174.561	-167.001309965835\\
-177.002	-169.336597903156\\
-97.656	-93.4268245829458\\
-86.67	-82.9165938253043\\
-69.58	-66.5667081846622\\
-101.318	-96.930234835493\\
-92.773	-88.7552920151719\\
-74.463	-71.238240752436\\
-151.367	-144.811769442193\\
-162.354	-155.322956892967\\
-112.305	-107.441422286267\\
-93.994	-89.9234143303986\\
-136.719	-130.798128432004\\
-68.359	-65.3985858694355\\
-43.945	-42.041879723699\\
-54.932	-52.5530671744735\\
-78.125	-74.7416510049832\\
-68.359	-65.3985858694355\\
-47.607	-45.5452899762462\\
-26.855	-25.691994083057\\
-48.828	-46.7134122914729\\
-78.125	-74.7416510049832\\
-124.512	-119.119775359136\\
-142.822	-136.636826621872\\
-187.988	-179.846828660797\\
-209.961	-200.868246869213\\
-107.422	-102.769889718494\\
-83.008	-79.4131835727571\\
-178.223	-170.504720218382\\
-261.23	-249.916947098007\\
-202.637	-193.861426364119\\
-189.209	-181.014950976024\\
-283.203	-270.938365306423\\
-227.051	-217.218132509855\\
-185.547	-177.511540723477\\
-133.057	-127.294718179457\\
-140.381	-134.301538684551\\
-185.547	-177.511540723477\\
-129.395	-123.791307926909\\
-112.305	-107.441422286267\\
-189.209	-181.014950976024\\
-236.816	-226.56024095227\\
-247.803	-237.071428403045\\
-111.084	-106.273299971041\\
-52.49	-50.2168225440201\\
-141.602	-135.469660999778\\
-111.084	-106.273299971041\\
-52.49	-50.2168225440201\\
-37.842	-36.2031815338314\\
-68.359	-65.3985858694355\\
-103.76	-99.2664794659464\\
-167.236	-159.993532767608\\
-117.188	-112.112954854041\\
-86.67	-82.9165938253043\\
-52.49	-50.2168225440201\\
-34.18	-32.6997712812842\\
-48.828	-46.7134122914729\\
-40.283	-38.5384694711519\\
-56.152	-53.7202327965673\\
-96.436	-92.259658960852\\
-87.891	-84.084716140531\\
-43.945	-42.041879723699\\
-39.063	-37.371303849058\\
-52.49	-50.2168225440201\\
-67.139	-64.2314202473417\\
-39.063	-37.371303849058\\
-62.256	-59.5598876795678\\
-45.166	-43.2100020389257\\
-34.18	-32.6997712812842\\
-79.346	-75.9097733202099\\
-120.85	-115.616365106588\\
-175.781	-168.168475587929\\
-239.258	-228.896485582724\\
-207.52	-198.532958931893\\
-100.098	-95.7630692133992\\
-70.801	-67.7348304998889\\
-74.463	-71.238240752436\\
-41.504	-39.7065917863785\\
-36.621	-35.0350592186047\\
-85.449	-81.7484715100776\\
-91.553	-87.5881263930781\\
-87.891	-84.084716140531\\
-103.76	-99.2664794659464\\
-122.07	-116.783530728682\\
-130.615	-124.958473549003\\
-135.498	-129.630006116777\\
-166.016	-158.826367145514\\
-152.588	-145.979891757419\\
-223.389	-213.714722257308\\
-115.967	-110.944832538815\\
-51.27	-49.0496569219263\\
-50.049	-47.8815346066996\\
-95.215	-91.0915366456253\\
-74.463	-71.238240752436\\
-67.139	-64.2314202473417\\
-31.738	-30.3635266508308\\
-48.828	-46.7134122914729\\
-37.842	-36.2031815338314\\
-18.311	-17.5180079558688\\
-32.959	-31.5316489660575\\
-69.58	-66.5667081846622\\
-87.891	-84.084716140531\\
-130.615	-124.958473549003\\
-183.105	-175.175296093023\\
-128.174	-122.623185611683\\
-51.27	-49.0496569219263\\
-93.994	-89.9234143303986\\
-106.201	-101.601767403267\\
-50.049	-47.8815346066996\\
-68.359	-65.3985858694355\\
-139.16	-133.133416369324\\
-128.174	-122.623185611683\\
-207.52	-198.532958931893\\
-240.479	-230.06460789795\\
-148.926	-142.476481504872\\
-114.746	-109.776710223588\\
};
\addlegendentry{data2}

\end{axis}

\begin{axis}[%
width=4.927cm,
height=3.484cm,
at={(6.484cm,4.839cm)},
scale only axis,
xmin=-400,
xmax=0,
xlabel style={font=\color{white!15!black}},
xlabel={y(t-1)},
ymin=-400,
ymax=0,
ylabel style={font=\color{white!15!black}},
ylabel={y(t)},
axis background/.style={fill=white},
title={C7, R = 0.7818},
axis x line*=bottom,
axis y line*=left,
legend style={legend cell align=left, align=left, draw=white!15!black}
]
\addplot[only marks, mark=*, mark options={}, mark size=1.5000pt, color=mycolor1, fill=mycolor1] table[row sep=crcr]{%
x	y\\
-97.656	-112.305\\
-112.305	-144.043\\
-144.043	-117.188\\
-117.188	-111.084\\
-111.084	-156.25\\
-156.25	-147.705\\
-147.705	-175.781\\
-175.781	-134.277\\
-134.277	-68.359\\
-68.359	-80.566\\
-80.566	-80.566\\
-80.566	-100.098\\
-100.098	-81.787\\
-81.787	-34.18\\
-34.18	-24.414\\
-24.414	-23.193\\
-23.193	-73.242\\
-73.242	-128.174\\
-128.174	-133.057\\
-133.057	-137.939\\
-137.939	-92.773\\
-92.773	-56.152\\
-56.152	-125.732\\
-125.732	-86.67\\
-86.67	-68.359\\
-68.359	-114.746\\
-114.746	-100.098\\
-100.098	-85.449\\
-85.449	-142.822\\
-142.822	-133.057\\
-133.057	-102.539\\
-102.539	-150.146\\
-150.146	-145.264\\
-145.264	-113.525\\
-113.525	-92.773\\
-92.773	-72.021\\
-72.021	-86.67\\
-86.67	-112.305\\
-112.305	-103.76\\
-103.76	-109.863\\
-109.863	-112.305\\
-112.305	-122.07\\
-122.07	-175.781\\
-175.781	-155.029\\
-155.029	-102.539\\
-102.539	-92.773\\
-92.773	-53.711\\
-53.711	-58.594\\
-58.594	-79.346\\
-79.346	-59.814\\
-59.814	-62.256\\
-62.256	-89.111\\
-89.111	-102.539\\
-102.539	-95.215\\
-95.215	-151.367\\
-151.367	-111.084\\
-111.084	-64.697\\
-64.697	-51.27\\
-51.27	-69.58\\
-69.58	-81.787\\
-81.787	-43.945\\
-43.945	-45.166\\
-45.166	-67.139\\
-67.139	-51.27\\
-51.27	-42.725\\
-42.725	-61.035\\
-61.035	-70.801\\
-70.801	-109.863\\
-109.863	-104.98\\
-104.98	-129.395\\
-129.395	-120.85\\
-120.85	-137.939\\
-137.939	-109.863\\
-109.863	-107.422\\
-107.422	-92.773\\
-92.773	-91.553\\
-91.553	-87.891\\
-87.891	-157.471\\
-157.471	-224.609\\
-224.609	-231.934\\
-231.934	-225.83\\
-225.83	-140.381\\
-140.381	-205.078\\
-205.078	-252.686\\
-252.686	-261.23\\
-261.23	-200.195\\
-200.195	-270.996\\
-270.996	-335.693\\
-335.693	-235.596\\
-235.596	-191.65\\
-191.65	-128.174\\
-128.174	-98.877\\
-98.877	-84.229\\
-84.229	-104.98\\
-104.98	-74.463\\
-74.463	-50.049\\
-50.049	-48.828\\
-48.828	-63.477\\
-63.477	-95.215\\
-95.215	-86.67\\
-86.67	-89.111\\
-89.111	-119.629\\
-119.629	-123.291\\
-123.291	-167.236\\
-167.236	-134.277\\
-134.277	-169.678\\
-169.678	-122.07\\
-122.07	-108.643\\
-108.643	-125.732\\
-125.732	-89.111\\
-89.111	-80.566\\
-80.566	-97.656\\
-97.656	-98.877\\
-98.877	-68.359\\
-68.359	-73.242\\
-73.242	-54.932\\
-54.932	-61.035\\
-61.035	-64.697\\
-64.697	-45.166\\
-45.166	-58.594\\
-58.594	-72.021\\
-72.021	-117.188\\
-117.188	-122.07\\
-122.07	-136.719\\
-136.719	-76.904\\
-76.904	-109.863\\
-109.863	-173.34\\
-173.34	-131.836\\
-131.836	-157.471\\
-157.471	-157.471\\
-157.471	-85.449\\
-85.449	-59.814\\
-59.814	-85.449\\
-85.449	-98.877\\
-98.877	-161.133\\
-161.133	-202.637\\
-202.637	-198.975\\
-198.975	-145.264\\
-145.264	-150.146\\
-150.146	-135.498\\
-135.498	-131.836\\
-131.836	-126.953\\
-126.953	-85.449\\
-85.449	-76.904\\
-76.904	-69.58\\
-69.58	-62.256\\
-62.256	-70.801\\
-70.801	-107.422\\
-107.422	-164.795\\
-164.795	-125.732\\
-125.732	-86.67\\
-86.67	-73.242\\
-73.242	-70.801\\
-70.801	-42.725\\
-42.725	-31.738\\
-31.738	-39.063\\
-39.063	-72.021\\
-72.021	-57.373\\
-57.373	-59.814\\
-59.814	-67.139\\
-67.139	-63.477\\
-63.477	-43.945\\
-43.945	-29.297\\
-29.297	-25.635\\
-25.635	-43.945\\
-43.945	-96.436\\
-96.436	-140.381\\
-140.381	-155.029\\
-155.029	-101.318\\
-101.318	-69.58\\
-69.58	-50.049\\
-50.049	-34.18\\
-34.18	-83.008\\
-83.008	-81.787\\
-81.787	-119.629\\
-119.629	-145.264\\
-145.264	-230.713\\
-230.713	-262.451\\
-262.451	-211.182\\
-211.182	-202.637\\
-202.637	-136.719\\
-136.719	-151.367\\
-151.367	-159.912\\
-159.912	-172.119\\
-172.119	-129.395\\
-129.395	-118.408\\
-118.408	-115.967\\
-115.967	-103.76\\
-103.76	-123.291\\
-123.291	-87.891\\
-87.891	-153.809\\
-153.809	-189.209\\
-189.209	-133.057\\
-133.057	-78.125\\
-78.125	-85.449\\
-85.449	-146.484\\
-146.484	-181.885\\
-181.885	-229.492\\
-229.492	-241.699\\
-241.699	-240.479\\
-240.479	-180.664\\
-180.664	-163.574\\
-163.574	-187.988\\
-187.988	-222.168\\
-222.168	-139.16\\
-139.16	-81.787\\
-81.787	-122.07\\
-122.07	-102.539\\
-102.539	-65.918\\
-65.918	-87.891\\
-87.891	-98.877\\
-98.877	-76.904\\
-76.904	-58.594\\
-58.594	-80.566\\
-80.566	-65.918\\
-65.918	-91.553\\
-91.553	-133.057\\
-133.057	-122.07\\
-122.07	-79.346\\
-79.346	-83.008\\
-83.008	-126.953\\
-126.953	-209.961\\
-209.961	-150.146\\
-150.146	-92.773\\
-92.773	-69.58\\
-69.58	-83.008\\
-83.008	-74.463\\
-74.463	-56.152\\
-56.152	-62.256\\
-62.256	-74.463\\
-74.463	-96.436\\
-96.436	-107.422\\
-107.422	-72.021\\
-72.021	-122.07\\
-122.07	-144.043\\
-144.043	-137.939\\
-137.939	-106.201\\
-106.201	-118.408\\
-118.408	-158.691\\
-158.691	-101.318\\
-101.318	-41.504\\
-41.504	-58.594\\
-58.594	-65.918\\
-65.918	-91.553\\
-91.553	-81.787\\
-81.787	-59.814\\
-59.814	-96.436\\
-96.436	-91.553\\
-91.553	-65.918\\
-65.918	-83.008\\
-83.008	-76.904\\
-76.904	-67.139\\
-67.139	-79.346\\
-79.346	-46.387\\
-46.387	-53.711\\
-53.711	-40.283\\
-40.283	-43.945\\
-43.945	-86.67\\
-86.67	-100.098\\
-100.098	-137.939\\
-137.939	-177.002\\
-177.002	-115.967\\
-115.967	-68.359\\
-68.359	-48.828\\
-48.828	-48.828\\
-48.828	-73.242\\
-73.242	-46.387\\
-46.387	-52.49\\
-52.49	-70.801\\
-70.801	-119.629\\
-119.629	-107.422\\
-107.422	-150.146\\
-150.146	-102.539\\
-102.539	-91.553\\
-91.553	-47.607\\
-47.607	-47.607\\
-47.607	-31.738\\
-31.738	-34.18\\
-34.18	-41.504\\
-41.504	-75.684\\
-75.684	-83.008\\
-83.008	-92.773\\
-92.773	-97.656\\
-97.656	-152.588\\
-152.588	-130.615\\
-130.615	-97.656\\
-97.656	-119.629\\
-119.629	-129.395\\
-129.395	-167.236\\
-167.236	-133.057\\
-133.057	-84.229\\
-84.229	-43.945\\
-43.945	-31.738\\
-31.738	-34.18\\
-34.18	-41.504\\
-41.504	-43.945\\
-43.945	-40.283\\
-40.283	-56.152\\
-56.152	-87.891\\
-87.891	-79.346\\
-79.346	-81.787\\
-81.787	-96.436\\
-96.436	-58.594\\
-58.594	-29.297\\
-29.297	-64.697\\
-64.697	-97.656\\
-97.656	-96.436\\
-96.436	-109.863\\
-109.863	-93.994\\
-93.994	-69.58\\
-69.58	-97.656\\
-97.656	-137.939\\
-137.939	-139.16\\
-139.16	-97.656\\
-97.656	-162.354\\
-162.354	-128.174\\
-128.174	-118.408\\
-118.408	-137.939\\
-137.939	-137.939\\
-137.939	-103.76\\
-103.76	-90.332\\
-90.332	-153.809\\
-153.809	-214.844\\
-214.844	-164.795\\
-164.795	-92.773\\
-92.773	-92.773\\
-92.773	-91.553\\
-91.553	-113.525\\
-113.525	-136.719\\
-136.719	-100.098\\
-100.098	-104.98\\
-104.98	-140.381\\
-140.381	-153.809\\
-153.809	-103.76\\
-103.76	-79.346\\
-79.346	-104.98\\
-104.98	-175.781\\
-175.781	-159.912\\
-159.912	-162.354\\
-162.354	-95.215\\
-95.215	-84.229\\
-84.229	-83.008\\
-83.008	-52.49\\
-52.49	-34.18\\
-34.18	-28.076\\
-28.076	-23.193\\
-23.193	-59.814\\
-59.814	-86.67\\
-86.67	-104.98\\
-104.98	-76.904\\
-76.904	-87.891\\
-87.891	-97.656\\
-97.656	-59.814\\
-59.814	-63.477\\
-63.477	-61.035\\
-61.035	-45.166\\
-45.166	-57.373\\
-57.373	-97.656\\
-97.656	-124.512\\
-124.512	-78.125\\
-78.125	-56.152\\
-56.152	-46.387\\
-46.387	-59.814\\
-59.814	-41.504\\
-41.504	-91.553\\
-91.553	-158.691\\
-158.691	-122.07\\
-122.07	-167.236\\
-167.236	-194.092\\
-194.092	-197.754\\
-197.754	-140.381\\
-140.381	-106.201\\
-106.201	-106.201\\
-106.201	-104.98\\
-104.98	-120.85\\
-120.85	-150.146\\
-150.146	-147.705\\
-147.705	-200.195\\
-200.195	-220.947\\
-220.947	-263.672\\
-263.672	-178.223\\
-178.223	-190.43\\
-190.43	-212.402\\
-212.402	-163.574\\
-163.574	-189.209\\
-189.209	-163.574\\
-163.574	-91.553\\
-91.553	-58.594\\
-58.594	-62.256\\
-62.256	-95.215\\
-95.215	-75.684\\
-75.684	-51.27\\
-51.27	-72.021\\
-72.021	-52.49\\
-52.49	-40.283\\
-40.283	-48.828\\
-48.828	-56.152\\
-56.152	-40.283\\
-40.283	-76.904\\
-76.904	-109.863\\
-109.863	-78.125\\
-78.125	-134.277\\
-134.277	-195.313\\
-195.313	-178.223\\
-178.223	-222.168\\
-222.168	-150.146\\
-150.146	-161.133\\
-161.133	-111.084\\
-111.084	-67.139\\
-67.139	-85.449\\
-85.449	-67.139\\
-67.139	-48.828\\
-48.828	-72.021\\
-72.021	-41.504\\
-41.504	-25.635\\
-25.635	-36.621\\
-36.621	-79.346\\
-79.346	-104.98\\
-104.98	-128.174\\
-128.174	-124.512\\
-124.512	-117.188\\
-117.188	-136.719\\
-136.719	-111.084\\
-111.084	-90.332\\
-90.332	-90.332\\
-90.332	-73.242\\
-73.242	-115.967\\
-115.967	-78.125\\
-78.125	-67.139\\
-67.139	-100.098\\
-100.098	-137.939\\
-137.939	-103.76\\
-103.76	-76.904\\
-76.904	-64.697\\
-64.697	-74.463\\
-74.463	-51.27\\
-51.27	-50.049\\
-50.049	-72.021\\
-72.021	-86.67\\
-86.67	-98.877\\
-98.877	-79.346\\
-79.346	-101.318\\
-101.318	-92.773\\
-92.773	-52.49\\
-52.49	-54.932\\
-54.932	-112.305\\
-112.305	-158.691\\
-158.691	-153.809\\
-153.809	-129.395\\
-129.395	-122.07\\
-122.07	-92.773\\
-92.773	-89.111\\
-89.111	-70.801\\
-70.801	-102.539\\
-102.539	-90.332\\
-90.332	-54.932\\
-54.932	-84.229\\
-84.229	-101.318\\
-101.318	-118.408\\
-118.408	-129.395\\
-129.395	-129.395\\
-129.395	-175.781\\
-175.781	-216.064\\
-216.064	-244.141\\
-244.141	-153.809\\
-153.809	-85.449\\
-85.449	-54.932\\
-54.932	-37.842\\
-37.842	-57.373\\
-57.373	-53.711\\
-53.711	-95.215\\
-95.215	-85.449\\
-85.449	-75.684\\
-75.684	-102.539\\
-102.539	-118.408\\
-118.408	-141.602\\
-141.602	-148.926\\
-148.926	-208.74\\
-208.74	-181.885\\
-181.885	-136.719\\
-136.719	-90.332\\
-90.332	-92.773\\
-92.773	-95.215\\
-95.215	-123.291\\
-123.291	-87.891\\
-87.891	-89.111\\
-89.111	-87.891\\
-87.891	-47.607\\
-47.607	-81.787\\
-81.787	-128.174\\
-128.174	-90.332\\
-90.332	-96.436\\
-96.436	-170.898\\
-170.898	-129.395\\
-129.395	-122.07\\
-122.07	-177.002\\
-177.002	-166.016\\
-166.016	-102.539\\
-102.539	-64.697\\
-64.697	-65.918\\
-65.918	-114.746\\
-114.746	-190.43\\
-190.43	-197.754\\
-197.754	-206.299\\
-206.299	-150.146\\
-150.146	-97.656\\
-97.656	-83.008\\
-83.008	-58.594\\
-58.594	-56.152\\
-56.152	-47.607\\
-47.607	-68.359\\
-68.359	-81.787\\
-81.787	-46.387\\
-46.387	-70.801\\
-70.801	-103.76\\
-103.76	-114.746\\
-114.746	-145.264\\
-145.264	-185.547\\
-185.547	-223.389\\
-223.389	-161.133\\
-161.133	-137.939\\
-137.939	-144.043\\
-144.043	-120.85\\
-120.85	-84.229\\
-84.229	-103.76\\
-103.76	-167.236\\
-167.236	-194.092\\
-194.092	-115.967\\
-115.967	-62.256\\
-62.256	-41.504\\
-41.504	-23.193\\
-23.193	-31.738\\
-31.738	-51.27\\
-51.27	-61.035\\
-61.035	-80.566\\
-80.566	-67.139\\
-67.139	-51.27\\
-51.27	-50.049\\
-50.049	-48.828\\
-48.828	-45.166\\
-45.166	-52.49\\
-52.49	-81.787\\
-81.787	-123.291\\
-123.291	-133.057\\
-133.057	-125.732\\
-125.732	-152.588\\
-152.588	-181.885\\
-181.885	-183.105\\
-183.105	-223.389\\
-223.389	-201.416\\
-201.416	-146.484\\
-146.484	-101.318\\
-101.318	-96.436\\
-96.436	-163.574\\
-163.574	-191.65\\
-191.65	-131.836\\
-131.836	-85.449\\
-85.449	-45.166\\
-45.166	-31.738\\
-31.738	-35.4\\
-35.4	-21.973\\
-21.973	-46.387\\
-46.387	-68.359\\
-68.359	-72.021\\
-72.021	-62.256\\
-62.256	-36.621\\
-36.621	-28.076\\
-28.076	-41.504\\
-41.504	-43.945\\
-43.945	-48.828\\
-48.828	-36.621\\
-36.621	-30.518\\
-30.518	-54.932\\
-54.932	-128.174\\
-128.174	-153.809\\
-153.809	-170.898\\
-170.898	-124.512\\
-124.512	-172.119\\
-172.119	-241.699\\
-241.699	-222.168\\
-222.168	-224.609\\
-224.609	-164.795\\
-164.795	-163.574\\
-163.574	-172.119\\
-172.119	-113.525\\
-113.525	-106.201\\
-106.201	-65.918\\
-65.918	-59.814\\
-59.814	-117.188\\
-117.188	-79.346\\
-79.346	-81.787\\
-81.787	-96.436\\
-96.436	-86.67\\
-86.67	-87.891\\
-87.891	-93.994\\
-93.994	-107.422\\
-107.422	-119.629\\
-119.629	-103.76\\
-103.76	-74.463\\
-74.463	-96.436\\
-96.436	-48.828\\
-48.828	-20.752\\
-20.752	-34.18\\
-34.18	-58.594\\
-58.594	-30.518\\
-30.518	-13.428\\
-13.428	-32.959\\
-32.959	-58.594\\
-58.594	-69.58\\
-69.58	-83.008\\
-83.008	-72.021\\
-72.021	-117.188\\
-117.188	-104.98\\
-104.98	-70.801\\
-70.801	-107.422\\
-107.422	-59.814\\
-59.814	-46.387\\
-46.387	-32.959\\
-32.959	-21.973\\
-21.973	-41.504\\
-41.504	-34.18\\
-34.18	-53.711\\
-53.711	-91.553\\
-91.553	-87.891\\
-87.891	-59.814\\
-59.814	-68.359\\
-68.359	-52.49\\
-52.49	-74.463\\
-74.463	-53.711\\
-53.711	-50.049\\
-50.049	-46.387\\
-46.387	-36.621\\
-36.621	-45.166\\
-45.166	-41.504\\
-41.504	-75.684\\
-75.684	-73.242\\
-73.242	-54.932\\
-54.932	-51.27\\
-51.27	-40.283\\
-40.283	-48.828\\
-48.828	-108.643\\
-108.643	-144.043\\
-144.043	-96.436\\
-96.436	-90.332\\
-90.332	-63.477\\
-63.477	-111.084\\
-111.084	-101.318\\
-101.318	-67.139\\
-67.139	-83.008\\
-83.008	-64.697\\
-64.697	-89.111\\
-89.111	-52.49\\
-52.49	-54.932\\
-54.932	-65.918\\
-65.918	-84.229\\
-84.229	-62.256\\
-62.256	-65.918\\
-65.918	-68.359\\
-68.359	-112.305\\
-112.305	-95.215\\
-95.215	-146.484\\
-146.484	-190.43\\
-190.43	-151.367\\
-151.367	-95.215\\
-95.215	-72.021\\
-72.021	-113.525\\
-113.525	-141.602\\
-141.602	-108.643\\
-108.643	-129.395\\
-129.395	-81.787\\
-81.787	-42.725\\
-42.725	-42.725\\
-42.725	-98.877\\
-98.877	-89.111\\
-89.111	-79.346\\
-79.346	-124.512\\
-124.512	-118.408\\
-118.408	-103.76\\
-103.76	-67.139\\
-67.139	-84.229\\
-84.229	-115.967\\
-115.967	-95.215\\
-95.215	-97.656\\
-97.656	-89.111\\
-89.111	-62.256\\
-62.256	-75.684\\
-75.684	-54.932\\
-54.932	-62.256\\
-62.256	-106.201\\
-106.201	-123.291\\
-123.291	-131.836\\
-131.836	-117.188\\
-117.188	-83.008\\
-83.008	-58.594\\
-58.594	-67.139\\
-67.139	-80.566\\
-80.566	-104.98\\
-104.98	-130.615\\
-130.615	-155.029\\
-155.029	-114.746\\
-114.746	-69.58\\
-69.58	-126.953\\
-126.953	-178.223\\
-178.223	-128.174\\
-128.174	-125.732\\
-125.732	-147.705\\
-147.705	-111.084\\
-111.084	-90.332\\
-90.332	-100.098\\
-100.098	-92.773\\
-92.773	-103.76\\
-103.76	-131.836\\
-131.836	-101.318\\
-101.318	-112.305\\
-112.305	-106.201\\
-106.201	-108.643\\
-108.643	-86.67\\
-86.67	-114.746\\
-114.746	-81.787\\
-81.787	-86.67\\
-86.67	-96.436\\
-96.436	-93.994\\
-93.994	-42.725\\
-42.725	-26.855\\
-26.855	-18.311\\
-18.311	-39.063\\
-39.063	-96.436\\
-96.436	-125.732\\
-125.732	-125.732\\
-125.732	-83.008\\
-83.008	-97.656\\
-97.656	-85.449\\
-85.449	-75.684\\
-75.684	-46.387\\
-46.387	-65.918\\
-65.918	-65.918\\
-65.918	-64.697\\
-64.697	-76.904\\
-76.904	-65.918\\
-65.918	-59.814\\
-59.814	-40.283\\
-40.283	-58.594\\
-58.594	-109.863\\
-109.863	-78.125\\
-78.125	-92.773\\
-92.773	-83.008\\
-83.008	-118.408\\
-118.408	-109.863\\
-109.863	-152.588\\
-152.588	-111.084\\
-111.084	-124.512\\
-124.512	-124.512\\
-124.512	-63.477\\
-63.477	-113.525\\
-113.525	-76.904\\
-76.904	-96.436\\
-96.436	-102.539\\
-102.539	-129.395\\
-129.395	-97.656\\
-97.656	-125.732\\
-125.732	-158.691\\
-158.691	-131.836\\
-131.836	-114.746\\
-114.746	-90.332\\
-90.332	-107.422\\
-107.422	-89.111\\
-89.111	-76.904\\
-76.904	-68.359\\
-68.359	-81.787\\
-81.787	-69.58\\
-69.58	-141.602\\
-141.602	-181.885\\
-181.885	-157.471\\
-157.471	-150.146\\
-150.146	-147.705\\
-147.705	-81.787\\
-81.787	-72.021\\
-72.021	-54.932\\
-54.932	-50.049\\
-50.049	-54.932\\
-54.932	-93.994\\
-93.994	-118.408\\
-118.408	-134.277\\
-134.277	-114.746\\
-114.746	-76.904\\
-76.904	-140.381\\
-140.381	-163.574\\
-163.574	-183.105\\
-183.105	-150.146\\
-150.146	-236.816\\
-236.816	-158.691\\
-158.691	-93.994\\
-93.994	-68.359\\
-68.359	-56.152\\
-56.152	-101.318\\
-101.318	-130.615\\
-130.615	-172.119\\
-172.119	-129.395\\
-129.395	-102.539\\
-102.539	-53.711\\
-53.711	-53.711\\
-53.711	-64.697\\
-64.697	-70.801\\
-70.801	-43.945\\
-43.945	-61.035\\
-61.035	-45.166\\
-45.166	-29.297\\
-29.297	-65.918\\
-65.918	-75.684\\
-75.684	-96.436\\
-96.436	-89.111\\
-89.111	-93.994\\
-93.994	-93.994\\
-93.994	-120.85\\
-120.85	-79.346\\
-79.346	-119.629\\
-119.629	-136.719\\
-136.719	-103.76\\
-103.76	-108.643\\
-108.643	-93.994\\
-93.994	-108.643\\
-108.643	-151.367\\
-151.367	-117.188\\
-117.188	-93.994\\
-93.994	-102.539\\
-102.539	-93.994\\
-93.994	-62.256\\
-62.256	-100.098\\
-100.098	-145.264\\
-145.264	-139.16\\
-139.16	-151.367\\
-151.367	-227.051\\
-227.051	-139.16\\
-139.16	-122.07\\
-122.07	-91.553\\
-91.553	-119.629\\
-119.629	-172.119\\
-172.119	-114.746\\
-114.746	-102.539\\
-102.539	-142.822\\
-142.822	-91.553\\
-91.553	-53.711\\
-53.711	-70.801\\
-70.801	-52.49\\
-52.49	-42.725\\
-42.725	-46.387\\
-46.387	-40.283\\
-40.283	-39.063\\
-39.063	-53.711\\
-53.711	-76.904\\
-76.904	-90.332\\
-90.332	-119.629\\
-119.629	-112.305\\
-112.305	-135.498\\
-135.498	-157.471\\
-157.471	-141.602\\
-141.602	-96.436\\
-96.436	-106.201\\
-106.201	-95.215\\
-95.215	-103.76\\
-103.76	-147.705\\
-147.705	-109.863\\
-109.863	-122.07\\
-122.07	-106.201\\
-106.201	-102.539\\
-102.539	-158.691\\
-158.691	-130.615\\
-130.615	-129.395\\
-129.395	-122.07\\
-122.07	-73.242\\
-73.242	-47.607\\
-47.607	-37.842\\
-37.842	-69.58\\
-69.58	-62.256\\
-62.256	-86.67\\
-86.67	-65.918\\
-65.918	-93.994\\
-93.994	-152.588\\
-152.588	-189.209\\
-189.209	-150.146\\
-150.146	-153.809\\
-153.809	-137.939\\
-137.939	-96.436\\
-96.436	-103.76\\
-103.76	-117.188\\
-117.188	-100.098\\
-100.098	-114.746\\
-114.746	-145.264\\
-145.264	-202.637\\
-202.637	-181.885\\
-181.885	-170.898\\
-170.898	-191.65\\
-191.65	-131.836\\
-131.836	-144.043\\
-144.043	-108.643\\
-108.643	-112.305\\
-112.305	-147.705\\
-147.705	-172.119\\
-172.119	-79.346\\
-79.346	-89.111\\
-89.111	-67.139\\
-67.139	-98.877\\
-98.877	-100.098\\
-100.098	-61.035\\
-61.035	-81.787\\
-81.787	-139.16\\
-139.16	-234.375\\
-234.375	-227.051\\
-227.051	-206.299\\
-206.299	-166.016\\
-166.016	-194.092\\
-194.092	-233.154\\
-233.154	-170.898\\
-170.898	-177.002\\
-177.002	-184.326\\
-184.326	-125.732\\
-125.732	-106.201\\
-106.201	-137.939\\
-137.939	-91.553\\
-91.553	-67.139\\
-67.139	-97.656\\
-97.656	-73.242\\
-73.242	-86.67\\
-86.67	-81.787\\
-81.787	-100.098\\
-100.098	-133.057\\
-133.057	-102.539\\
-102.539	-75.684\\
-75.684	-89.111\\
-89.111	-104.98\\
-104.98	-124.512\\
-124.512	-107.422\\
-107.422	-86.67\\
-86.67	-136.719\\
-136.719	-129.395\\
-129.395	-90.332\\
-90.332	-148.926\\
-148.926	-134.277\\
-134.277	-97.656\\
-97.656	-64.697\\
-64.697	-46.387\\
-46.387	-32.959\\
-32.959	-75.684\\
-75.684	-72.021\\
-72.021	-52.49\\
-52.49	-36.621\\
-36.621	-47.607\\
-47.607	-81.787\\
-81.787	-93.994\\
-93.994	-124.512\\
-124.512	-142.822\\
-142.822	-137.939\\
-137.939	-145.264\\
-145.264	-87.891\\
-87.891	-36.621\\
-36.621	-47.607\\
-47.607	-46.387\\
-46.387	-64.697\\
-64.697	-100.098\\
-100.098	-111.084\\
-111.084	-168.457\\
-168.457	-106.201\\
-106.201	-104.98\\
-104.98	-155.029\\
-155.029	-106.201\\
-106.201	-74.463\\
-74.463	-53.711\\
-53.711	-76.904\\
-76.904	-100.098\\
-100.098	-151.367\\
-151.367	-98.877\\
-98.877	-85.449\\
-85.449	-80.566\\
-80.566	-101.318\\
-101.318	-133.057\\
-133.057	-207.52\\
-207.52	-184.326\\
-184.326	-180.664\\
-180.664	-112.305\\
-112.305	-75.684\\
-75.684	-87.891\\
-87.891	-36.621\\
-36.621	-67.139\\
-67.139	-52.49\\
-52.49	-65.918\\
-65.918	-117.188\\
-117.188	-153.809\\
-153.809	-177.002\\
-177.002	-234.375\\
-234.375	-196.533\\
-196.533	-106.201\\
-106.201	-98.877\\
-98.877	-79.346\\
-79.346	-108.643\\
-108.643	-141.602\\
-141.602	-191.65\\
-191.65	-140.381\\
-140.381	-97.656\\
-97.656	-108.643\\
-108.643	-144.043\\
-144.043	-102.539\\
-102.539	-73.242\\
-73.242	-57.373\\
-57.373	-40.283\\
-40.283	-36.621\\
-36.621	-29.297\\
-29.297	-54.932\\
-54.932	-80.566\\
-80.566	-86.67\\
-86.67	-64.697\\
-64.697	-58.594\\
-58.594	-26.855\\
-26.855	-15.869\\
-15.869	-24.414\\
-24.414	-46.387\\
-46.387	-70.801\\
-70.801	-87.891\\
-87.891	-102.539\\
-102.539	-119.629\\
-119.629	-153.809\\
-153.809	-212.402\\
-212.402	-213.623\\
-213.623	-230.713\\
-230.713	-201.416\\
-201.416	-111.084\\
-111.084	-102.539\\
-102.539	-101.318\\
-101.318	-135.498\\
-135.498	-113.525\\
-113.525	-80.566\\
-80.566	-61.035\\
-61.035	-50.049\\
-50.049	-93.994\\
-93.994	-90.332\\
-90.332	-47.607\\
-47.607	-37.842\\
-37.842	-58.594\\
-58.594	-78.125\\
-78.125	-59.814\\
-59.814	-86.67\\
-86.67	-67.139\\
-67.139	-98.877\\
-98.877	-141.602\\
-141.602	-112.305\\
-112.305	-153.809\\
-153.809	-206.299\\
-206.299	-224.609\\
-224.609	-175.781\\
-175.781	-111.084\\
-111.084	-83.008\\
-83.008	-48.828\\
-48.828	-81.787\\
-81.787	-57.373\\
-57.373	-109.863\\
-109.863	-92.773\\
-92.773	-74.463\\
-74.463	-96.436\\
-96.436	-52.49\\
-52.49	-84.229\\
-84.229	-58.594\\
-58.594	-37.842\\
-37.842	-45.166\\
-45.166	-32.959\\
-32.959	-40.283\\
-40.283	-35.4\\
-35.4	-26.855\\
-26.855	-36.621\\
-36.621	-48.828\\
-48.828	-50.049\\
-50.049	-48.828\\
-48.828	-79.346\\
-79.346	-103.76\\
-103.76	-84.229\\
-84.229	-78.125\\
-78.125	-128.174\\
-128.174	-153.809\\
-153.809	-119.629\\
-119.629	-125.732\\
-125.732	-56.152\\
-56.152	-36.621\\
-36.621	-73.242\\
-73.242	-106.201\\
-106.201	-115.967\\
-115.967	-162.354\\
-162.354	-146.484\\
-146.484	-102.539\\
-102.539	-73.242\\
-73.242	-79.346\\
-79.346	-84.229\\
-84.229	-64.697\\
-64.697	-39.063\\
-39.063	-52.49\\
-52.49	-39.063\\
-39.063	-32.959\\
-32.959	-24.414\\
-24.414	-53.711\\
-53.711	-74.463\\
-74.463	-79.346\\
-79.346	-56.152\\
-56.152	-100.098\\
-100.098	-136.719\\
-136.719	-86.67\\
-86.67	-120.85\\
-120.85	-130.615\\
-130.615	-98.877\\
-98.877	-56.152\\
-56.152	-72.021\\
-72.021	-87.891\\
-87.891	-47.607\\
-47.607	-50.049\\
-50.049	-41.504\\
-41.504	-24.414\\
-24.414	-26.855\\
-26.855	-46.387\\
-46.387	-64.697\\
-64.697	-54.932\\
-54.932	-72.021\\
-72.021	-79.346\\
-79.346	-57.373\\
-57.373	-31.738\\
-31.738	-25.635\\
-25.635	-34.18\\
-34.18	-39.063\\
-39.063	-48.828\\
-48.828	-104.98\\
-104.98	-101.318\\
-101.318	-89.111\\
-89.111	-85.449\\
-85.449	-117.188\\
-117.188	-150.146\\
-150.146	-159.912\\
-159.912	-166.016\\
-166.016	-120.85\\
-120.85	-126.953\\
-126.953	-131.836\\
-131.836	-133.057\\
-133.057	-151.367\\
-151.367	-200.195\\
-200.195	-169.678\\
-169.678	-156.25\\
-156.25	-128.174\\
-128.174	-119.629\\
-119.629	-114.746\\
-114.746	-136.719\\
-136.719	-83.008\\
-83.008	-98.877\\
-98.877	-119.629\\
-119.629	-140.381\\
-140.381	-107.422\\
-107.422	-124.512\\
-124.512	-98.877\\
-98.877	-90.332\\
-90.332	-54.932\\
-54.932	-85.449\\
-85.449	-137.939\\
-137.939	-109.863\\
-109.863	-64.697\\
-64.697	-80.566\\
-80.566	-41.504\\
-41.504	-40.283\\
-40.283	-56.152\\
-56.152	-32.959\\
-32.959	-36.621\\
-36.621	-43.945\\
-43.945	-26.855\\
-26.855	-50.049\\
-50.049	-40.283\\
-40.283	-23.193\\
-23.193	-43.945\\
-43.945	-51.27\\
-51.27	-63.477\\
-63.477	-62.256\\
-62.256	-63.477\\
-63.477	-84.229\\
-84.229	-76.904\\
-76.904	-86.67\\
-86.67	-146.484\\
-146.484	-101.318\\
-101.318	-92.773\\
-92.773	-100.098\\
-100.098	-69.58\\
-69.58	-40.283\\
-40.283	-79.346\\
-79.346	-61.035\\
-61.035	-37.842\\
-37.842	-67.139\\
-67.139	-62.256\\
-62.256	-80.566\\
-80.566	-48.828\\
-48.828	-29.297\\
-29.297	-23.193\\
-23.193	-29.297\\
-29.297	-56.152\\
-56.152	-54.932\\
-54.932	-65.918\\
-65.918	-87.891\\
-87.891	-117.188\\
-117.188	-84.229\\
-84.229	-124.512\\
-124.512	-141.602\\
-141.602	-124.512\\
-124.512	-117.188\\
-117.188	-195.313\\
-195.313	-135.498\\
-135.498	-173.34\\
-173.34	-141.602\\
-141.602	-164.795\\
-164.795	-184.326\\
-184.326	-101.318\\
-101.318	-63.477\\
-63.477	-51.27\\
-51.27	-84.229\\
-84.229	-104.98\\
-104.98	-108.643\\
-108.643	-74.463\\
-74.463	-40.283\\
-40.283	-52.49\\
-52.49	-104.98\\
-104.98	-145.264\\
-145.264	-142.822\\
-142.822	-84.229\\
-84.229	-64.697\\
-64.697	-89.111\\
-89.111	-141.602\\
-141.602	-161.133\\
-161.133	-164.795\\
-164.795	-115.967\\
-115.967	-166.016\\
-166.016	-178.223\\
-178.223	-170.898\\
-170.898	-240.479\\
-240.479	-203.857\\
-203.857	-169.678\\
-169.678	-196.533\\
-196.533	-164.795\\
-164.795	-91.553\\
-91.553	-84.229\\
-84.229	-104.98\\
-104.98	-125.732\\
-125.732	-128.174\\
-128.174	-96.436\\
-96.436	-124.512\\
-124.512	-106.201\\
-106.201	-78.125\\
-78.125	-106.201\\
-106.201	-95.215\\
-95.215	-150.146\\
-150.146	-151.367\\
-151.367	-108.643\\
-108.643	-79.346\\
-79.346	-45.166\\
-45.166	-78.125\\
-78.125	-53.711\\
-53.711	-52.49\\
-52.49	-42.725\\
-42.725	-76.904\\
-76.904	-98.877\\
-98.877	-112.305\\
-112.305	-109.863\\
-109.863	-64.697\\
-64.697	-48.828\\
-48.828	-35.4\\
-35.4	-43.945\\
-43.945	-70.801\\
-70.801	-75.684\\
-75.684	-74.463\\
-74.463	-120.85\\
-120.85	-140.381\\
-140.381	-162.354\\
-162.354	-159.912\\
-159.912	-187.988\\
-187.988	-109.863\\
-109.863	-91.553\\
-91.553	-72.021\\
-72.021	-78.125\\
-78.125	-104.98\\
-104.98	-86.67\\
-86.67	-57.373\\
-57.373	-50.049\\
-50.049	-97.656\\
-97.656	-117.188\\
-117.188	-107.422\\
-107.422	-168.457\\
-168.457	-184.326\\
-184.326	-130.615\\
-130.615	-91.553\\
-91.553	-62.256\\
-62.256	-57.373\\
-57.373	-107.422\\
-107.422	-85.449\\
-85.449	-103.76\\
-103.76	-113.525\\
-113.525	-129.395\\
-129.395	-122.07\\
-122.07	-135.498\\
-135.498	-91.553\\
-91.553	-109.863\\
-109.863	-75.684\\
-75.684	-68.359\\
-68.359	-107.422\\
-107.422	-147.705\\
-147.705	-162.354\\
-162.354	-91.553\\
-91.553	-190.43\\
-190.43	-170.898\\
-170.898	-113.525\\
-113.525	-65.918\\
-65.918	-96.436\\
-96.436	-83.008\\
-83.008	-80.566\\
-80.566	-98.877\\
-98.877	-64.697\\
-64.697	-93.994\\
-93.994	-170.898\\
-170.898	-179.443\\
-179.443	-117.188\\
-117.188	-134.277\\
-134.277	-146.484\\
-146.484	-148.926\\
-148.926	-106.201\\
-106.201	-106.201\\
-106.201	-69.58\\
-69.58	-106.201\\
-106.201	-109.863\\
-109.863	-175.781\\
-175.781	-195.313\\
-195.313	-263.672\\
-263.672	-190.43\\
-190.43	-155.029\\
-155.029	-118.408\\
-118.408	-130.615\\
-130.615	-97.656\\
-97.656	-67.139\\
-67.139	-68.359\\
-68.359	-68.359\\
-68.359	-48.828\\
-48.828	-35.4\\
-35.4	-59.814\\
-59.814	-81.787\\
-81.787	-54.932\\
-54.932	-39.063\\
-39.063	-37.842\\
-37.842	-90.332\\
-90.332	-128.174\\
-128.174	-59.814\\
-59.814	-74.463\\
-74.463	-79.346\\
-79.346	-73.242\\
-73.242	-91.553\\
-91.553	-79.346\\
-79.346	-54.932\\
-54.932	-41.504\\
-41.504	-34.18\\
-34.18	-43.945\\
-43.945	-86.67\\
-86.67	-103.76\\
-103.76	-72.021\\
-72.021	-48.828\\
-48.828	-29.297\\
-29.297	-42.725\\
-42.725	-80.566\\
-80.566	-101.318\\
-101.318	-107.422\\
-107.422	-76.904\\
-76.904	-70.801\\
-70.801	-78.125\\
-78.125	-108.643\\
-108.643	-151.367\\
-151.367	-175.781\\
-175.781	-131.836\\
-131.836	-83.008\\
-83.008	-86.67\\
-86.67	-81.787\\
-81.787	-84.229\\
-84.229	-54.932\\
-54.932	-69.58\\
-69.58	-76.904\\
-76.904	-76.904\\
-76.904	-45.166\\
-45.166	-29.297\\
-29.297	-47.607\\
-47.607	-70.801\\
-70.801	-109.863\\
-109.863	-118.408\\
-118.408	-148.926\\
-148.926	-130.615\\
-130.615	-146.484\\
-146.484	-195.313\\
-195.313	-260.01\\
-260.01	-212.402\\
-212.402	-142.822\\
-142.822	-155.029\\
-155.029	-180.664\\
-180.664	-236.816\\
-236.816	-155.029\\
-155.029	-76.904\\
-76.904	-50.049\\
-50.049	-52.49\\
-52.49	-37.842\\
-37.842	-23.193\\
-23.193	-30.518\\
-30.518	-48.828\\
-48.828	-68.359\\
-68.359	-70.801\\
-70.801	-54.932\\
-54.932	-48.828\\
-48.828	-67.139\\
-67.139	-81.787\\
-81.787	-85.449\\
-85.449	-79.346\\
-79.346	-57.373\\
-57.373	-37.842\\
-37.842	-37.842\\
-37.842	-57.373\\
-57.373	-73.242\\
-73.242	-53.711\\
-53.711	-35.4\\
-35.4	-58.594\\
-58.594	-39.063\\
-39.063	-50.049\\
-50.049	-63.477\\
-63.477	-92.773\\
-92.773	-100.098\\
-100.098	-68.359\\
-68.359	-106.201\\
-106.201	-102.539\\
-102.539	-83.008\\
-83.008	-70.801\\
-70.801	-119.629\\
-119.629	-146.484\\
-146.484	-144.043\\
-144.043	-158.691\\
-158.691	-120.85\\
-120.85	-111.084\\
-111.084	-102.539\\
-102.539	-139.16\\
-139.16	-108.643\\
-108.643	-147.705\\
-147.705	-141.602\\
-141.602	-146.484\\
-146.484	-249.023\\
-249.023	-190.43\\
-190.43	-102.539\\
-102.539	-69.58\\
-69.58	-73.242\\
-73.242	-91.553\\
-91.553	-119.629\\
-119.629	-139.16\\
-139.16	-153.809\\
-153.809	-157.471\\
-157.471	-96.436\\
-96.436	-86.67\\
-86.67	-101.318\\
-101.318	-102.539\\
-102.539	-61.035\\
-61.035	-43.945\\
-43.945	-79.346\\
-79.346	-80.566\\
-80.566	-109.863\\
-109.863	-139.16\\
-139.16	-123.291\\
-123.291	-87.891\\
-87.891	-70.801\\
-70.801	-107.422\\
-107.422	-139.16\\
-139.16	-185.547\\
-185.547	-198.975\\
-198.975	-231.934\\
-231.934	-184.326\\
-184.326	-168.457\\
-168.457	-89.111\\
-89.111	-67.139\\
-67.139	-126.953\\
-126.953	-163.574\\
-163.574	-93.994\\
-93.994	-151.367\\
-151.367	-203.857\\
-203.857	-246.582\\
-246.582	-153.809\\
-153.809	-86.67\\
-86.67	-50.049\\
-50.049	-75.684\\
-75.684	-69.58\\
-69.58	-75.684\\
-75.684	-42.725\\
-42.725	-61.035\\
-61.035	-45.166\\
-45.166	-58.594\\
-58.594	-48.828\\
-48.828	-62.256\\
-62.256	-101.318\\
-101.318	-91.553\\
-91.553	-129.395\\
-129.395	-156.25\\
-156.25	-155.029\\
-155.029	-85.449\\
-85.449	-86.67\\
-86.67	-106.201\\
-106.201	-70.801\\
-70.801	-63.477\\
-63.477	-45.166\\
-45.166	-73.242\\
-73.242	-69.58\\
-69.58	-39.063\\
-39.063	-21.973\\
-21.973	-25.635\\
-25.635	-46.387\\
-46.387	-72.021\\
-72.021	-75.684\\
-75.684	-45.166\\
-45.166	-57.373\\
-57.373	-84.229\\
-84.229	-109.863\\
-109.863	-76.904\\
-76.904	-72.021\\
-72.021	-84.229\\
-84.229	-101.318\\
-101.318	-92.773\\
-92.773	-109.863\\
-109.863	-103.76\\
-103.76	-70.801\\
-70.801	-56.152\\
-56.152	-72.021\\
-72.021	-59.814\\
-59.814	-75.684\\
-75.684	-102.539\\
-102.539	-157.471\\
-157.471	-109.863\\
-109.863	-107.422\\
-107.422	-166.016\\
-166.016	-125.732\\
-125.732	-75.684\\
-75.684	-54.932\\
-54.932	-43.945\\
-43.945	-46.387\\
-46.387	-39.063\\
-39.063	-40.283\\
-40.283	-84.229\\
-84.229	-26.855\\
-26.855	-37.842\\
-37.842	-53.711\\
-53.711	-72.021\\
-72.021	-56.152\\
-56.152	-40.283\\
-40.283	-24.414\\
-24.414	-42.725\\
-42.725	-84.229\\
-84.229	-111.084\\
-111.084	-87.891\\
-87.891	-73.242\\
-73.242	-83.008\\
-83.008	-84.229\\
-84.229	-96.436\\
-96.436	-117.188\\
-117.188	-139.16\\
-139.16	-133.057\\
-133.057	-114.746\\
-114.746	-74.463\\
-74.463	-58.594\\
-58.594	-61.035\\
-61.035	-91.553\\
-91.553	-57.373\\
-57.373	-50.049\\
-50.049	-92.773\\
-92.773	-113.525\\
-113.525	-108.643\\
-108.643	-124.512\\
-124.512	-164.795\\
-164.795	-85.449\\
-85.449	-145.264\\
-145.264	-152.588\\
-152.588	-140.381\\
-140.381	-152.588\\
-152.588	-148.926\\
-148.926	-253.906\\
-253.906	-235.596\\
-235.596	-158.691\\
-158.691	-189.209\\
-189.209	-233.154\\
-233.154	-223.389\\
-223.389	-253.906\\
-253.906	-227.051\\
-227.051	-311.279\\
-311.279	-234.375\\
-234.375	-159.912\\
-159.912	-98.877\\
-98.877	-100.098\\
-100.098	-79.346\\
-79.346	-64.697\\
-64.697	-100.098\\
-100.098	-142.822\\
-142.822	-87.891\\
-87.891	-76.904\\
-76.904	-73.242\\
-73.242	-48.828\\
-48.828	-63.477\\
-63.477	-34.18\\
-34.18	-40.283\\
-40.283	-35.4\\
-35.4	-48.828\\
-48.828	-37.842\\
-37.842	-31.738\\
-31.738	-17.09\\
-17.09	-18.311\\
-18.311	-40.283\\
-40.283	-28.076\\
-28.076	-29.297\\
-29.297	-65.918\\
-65.918	-93.994\\
-93.994	-93.994\\
-93.994	-68.359\\
-68.359	-86.67\\
-86.67	-135.498\\
-135.498	-70.801\\
-70.801	-37.842\\
-37.842	-29.297\\
-29.297	-24.414\\
-24.414	-40.283\\
-40.283	-67.139\\
-67.139	-100.098\\
-100.098	-62.256\\
-62.256	-100.098\\
-100.098	-144.043\\
-144.043	-145.264\\
-145.264	-103.76\\
-103.76	-62.256\\
-62.256	-83.008\\
-83.008	-113.525\\
-113.525	-148.926\\
-148.926	-83.008\\
-83.008	-42.725\\
-42.725	-67.139\\
-67.139	-53.711\\
-53.711	-24.414\\
-24.414	-15.869\\
-15.869	-42.725\\
-42.725	-92.773\\
-92.773	-98.877\\
-98.877	-70.801\\
-70.801	-43.945\\
-43.945	-47.607\\
-47.607	-20.752\\
-20.752	-28.076\\
-28.076	-29.297\\
-29.297	-46.387\\
-46.387	-69.58\\
-69.58	-103.76\\
-103.76	-107.422\\
-107.422	-120.85\\
-120.85	-124.512\\
-124.512	-120.85\\
-120.85	-115.967\\
-115.967	-101.318\\
-101.318	-123.291\\
-123.291	-187.988\\
-187.988	-169.678\\
-169.678	-87.891\\
-87.891	-45.166\\
-45.166	-103.76\\
-103.76	-122.07\\
-122.07	-114.746\\
-114.746	-63.477\\
-63.477	-67.139\\
-67.139	-113.525\\
-113.525	-175.781\\
-175.781	-180.664\\
-180.664	-203.857\\
-203.857	-195.313\\
-195.313	-139.16\\
-139.16	-144.043\\
-144.043	-59.814\\
-59.814	-61.035\\
-61.035	-76.904\\
-76.904	-98.877\\
-98.877	-104.98\\
-104.98	-111.084\\
-111.084	-70.801\\
-70.801	-98.877\\
-98.877	-86.67\\
-86.67	-47.607\\
-47.607	-32.959\\
-32.959	-26.855\\
-26.855	-42.725\\
-42.725	-35.4\\
-35.4	-19.531\\
-19.531	-40.283\\
-40.283	-85.449\\
-85.449	-56.152\\
-56.152	-51.27\\
-51.27	-72.021\\
-72.021	-119.629\\
-119.629	-81.787\\
-81.787	-42.725\\
-42.725	-21.973\\
-21.973	-57.373\\
-57.373	-126.953\\
-126.953	-161.133\\
-161.133	-223.389\\
-223.389	-266.113\\
-266.113	-261.23\\
-261.23	-168.457\\
-168.457	-170.898\\
-170.898	-222.168\\
-222.168	-195.313\\
-195.313	-179.443\\
-179.443	-202.637\\
-202.637	-220.947\\
-220.947	-223.389\\
-223.389	-301.514\\
-301.514	-203.857\\
-203.857	-109.863\\
-109.863	-59.814\\
-59.814	-47.607\\
-47.607	-54.932\\
-54.932	-53.711\\
-53.711	-56.152\\
-56.152	-101.318\\
-101.318	-79.346\\
-79.346	-84.229\\
-84.229	-111.084\\
-111.084	-63.477\\
-63.477	-70.801\\
-70.801	-101.318\\
-101.318	-115.967\\
-115.967	-64.697\\
-64.697	-46.387\\
-46.387	-58.594\\
-58.594	-54.932\\
-54.932	-129.395\\
-129.395	-186.768\\
-186.768	-173.34\\
-173.34	-198.975\\
-198.975	-225.83\\
-225.83	-297.852\\
-297.852	-257.568\\
-257.568	-157.471\\
-157.471	-100.098\\
-100.098	-54.932\\
-54.932	-61.035\\
-61.035	-67.139\\
-67.139	-86.67\\
-86.67	-62.256\\
-62.256	-56.152\\
-56.152	-63.477\\
-63.477	-78.125\\
-78.125	-89.111\\
-89.111	-111.084\\
-111.084	-79.346\\
-79.346	-40.283\\
-40.283	-28.076\\
-28.076	-23.193\\
-23.193	-24.414\\
-24.414	-34.18\\
-34.18	-57.373\\
-57.373	-59.814\\
-59.814	-75.684\\
-75.684	-114.746\\
-114.746	-81.787\\
-81.787	-150.146\\
-150.146	-208.74\\
-208.74	-162.354\\
-162.354	-144.043\\
-144.043	-170.898\\
-170.898	-198.975\\
-198.975	-131.836\\
-131.836	-111.084\\
-111.084	-72.021\\
-72.021	-79.346\\
-79.346	-142.822\\
-142.822	-247.803\\
-247.803	-300.293\\
-300.293	-239.258\\
-239.258	-200.195\\
-200.195	-145.264\\
-145.264	-153.809\\
-153.809	-208.74\\
-208.74	-115.967\\
-115.967	-98.877\\
-98.877	-75.684\\
-75.684	-148.926\\
-148.926	-136.719\\
-136.719	-76.904\\
-76.904	-72.021\\
-72.021	-53.711\\
-53.711	-95.215\\
-95.215	-145.264\\
-145.264	-148.926\\
-148.926	-113.525\\
-113.525	-83.008\\
-83.008	-93.994\\
-93.994	-74.463\\
-74.463	-53.711\\
-53.711	-43.945\\
-43.945	-39.063\\
-39.063	-29.297\\
-29.297	-35.4\\
-35.4	-65.918\\
-65.918	-95.215\\
-95.215	-96.436\\
-96.436	-58.594\\
-58.594	-53.711\\
-53.711	-41.504\\
-41.504	-54.932\\
-54.932	-78.125\\
-78.125	-86.67\\
-86.67	-80.566\\
-80.566	-46.387\\
-46.387	-96.436\\
-96.436	-142.822\\
-142.822	-101.318\\
-101.318	-40.283\\
-40.283	-50.049\\
-50.049	-76.904\\
-76.904	-73.242\\
-73.242	-67.139\\
-67.139	-41.504\\
-41.504	-75.684\\
-75.684	-117.188\\
-117.188	-70.801\\
-70.801	-25.635\\
-25.635	-39.063\\
-39.063	-91.553\\
-91.553	-111.084\\
-111.084	-65.918\\
-65.918	-59.814\\
-59.814	-113.525\\
-113.525	-151.367\\
-151.367	-131.836\\
-131.836	-179.443\\
-179.443	-219.727\\
-219.727	-140.381\\
-140.381	-111.084\\
-111.084	-159.912\\
-159.912	-109.863\\
-109.863	-142.822\\
-142.822	-174.561\\
-174.561	-137.939\\
-137.939	-69.58\\
-69.58	-111.084\\
-111.084	-202.637\\
-202.637	-233.154\\
-233.154	-169.678\\
-169.678	-142.822\\
-142.822	-191.65\\
-191.65	-153.809\\
-153.809	-96.436\\
-96.436	-85.449\\
-85.449	-107.422\\
-107.422	-113.525\\
-113.525	-109.863\\
-109.863	-125.732\\
-125.732	-87.891\\
-87.891	-90.332\\
-90.332	-130.615\\
-130.615	-78.125\\
-78.125	-62.256\\
-62.256	-50.049\\
-50.049	-40.283\\
-40.283	-46.387\\
-46.387	-85.449\\
-85.449	-79.346\\
-79.346	-107.422\\
-107.422	-139.16\\
-139.16	-164.795\\
-164.795	-123.291\\
-123.291	-107.422\\
-107.422	-41.504\\
-41.504	-67.139\\
-67.139	-129.395\\
-129.395	-85.449\\
-85.449	-41.504\\
-41.504	-87.891\\
-87.891	-137.939\\
-137.939	-142.822\\
-142.822	-185.547\\
-185.547	-245.361\\
-245.361	-173.34\\
-173.34	-147.705\\
-147.705	-170.898\\
-170.898	-113.525\\
-113.525	-90.332\\
-90.332	-102.539\\
-102.539	-131.836\\
-131.836	-141.602\\
-141.602	-142.822\\
-142.822	-79.346\\
-79.346	-70.801\\
-70.801	-54.932\\
-54.932	-79.346\\
-79.346	-75.684\\
-75.684	-59.814\\
-59.814	-119.629\\
-119.629	-131.836\\
-131.836	-89.111\\
-89.111	-74.463\\
-74.463	-108.643\\
-108.643	-63.477\\
-63.477	-35.4\\
-35.4	-47.607\\
-47.607	-64.697\\
-64.697	-56.152\\
-56.152	-41.504\\
-41.504	-23.193\\
-23.193	-42.725\\
-42.725	-64.697\\
-64.697	-102.539\\
-102.539	-118.408\\
-118.408	-152.588\\
-152.588	-173.34\\
-173.34	-91.553\\
-91.553	-69.58\\
-69.58	-136.719\\
-136.719	-211.182\\
-211.182	-164.795\\
-164.795	-151.367\\
-151.367	-230.713\\
-230.713	-185.547\\
-185.547	-148.926\\
-148.926	-112.305\\
-112.305	-114.746\\
-114.746	-155.029\\
-155.029	-107.422\\
-107.422	-90.332\\
-90.332	-156.25\\
-156.25	-196.533\\
-196.533	-205.078\\
-205.078	-92.773\\
-92.773	-43.945\\
-43.945	-111.084\\
-111.084	-87.891\\
-87.891	-37.842\\
-37.842	-29.297\\
-29.297	-53.711\\
-53.711	-81.787\\
-81.787	-135.498\\
-135.498	-96.436\\
-96.436	-70.801\\
-70.801	-42.725\\
-42.725	-28.076\\
-28.076	-39.063\\
-39.063	-35.4\\
-35.4	-45.166\\
-45.166	-79.346\\
-79.346	-72.021\\
-72.021	-32.959\\
-32.959	-30.518\\
-30.518	-41.504\\
-41.504	-53.711\\
-53.711	-32.959\\
-32.959	-51.27\\
-51.27	-37.842\\
-37.842	-25.635\\
-25.635	-63.477\\
-63.477	-96.436\\
-96.436	-145.264\\
-145.264	-190.43\\
-190.43	-161.133\\
-161.133	-80.566\\
-80.566	-57.373\\
-57.373	-58.594\\
-58.594	-32.959\\
-32.959	-30.518\\
-30.518	-64.697\\
-64.697	-73.242\\
-73.242	-67.139\\
-67.139	-73.242\\
-73.242	-89.111\\
-89.111	-93.994\\
-93.994	-97.656\\
-97.656	-122.07\\
-122.07	-118.408\\
-118.408	-170.898\\
-170.898	-89.111\\
-89.111	-41.504\\
-41.504	-40.283\\
-40.283	-76.904\\
-76.904	-63.477\\
-63.477	-57.373\\
-57.373	-26.855\\
-26.855	-45.166\\
-45.166	-28.076\\
-28.076	-13.428\\
-13.428	-25.635\\
-25.635	-52.49\\
-52.49	-64.697\\
-64.697	-96.436\\
-96.436	-136.719\\
-136.719	-98.877\\
-98.877	-43.945\\
-43.945	-80.566\\
-80.566	-90.332\\
-90.332	-42.725\\
-42.725	-57.373\\
-57.373	-112.305\\
-112.305	-97.656\\
-97.656	-162.354\\
-162.354	-189.209\\
-189.209	-117.188\\
-117.188	-92.773\\
-92.773	-80.566\\
};
\addlegendentry{data1}

\addplot [color=mycolor2, line width=2.0pt]
  table[row sep=crcr]{%
-97.656	-93.4078599971614\\
-112.305	-107.419612896097\\
-144.043	-137.776976095387\\
-117.188	-112.090197195742\\
-111.084	-106.251727696452\\
-156.25	-149.452958595032\\
-147.705	-141.279675195387\\
-175.781	-168.134339294677\\
-134.277	-128.435807496097\\
-68.359	-65.385310698226\\
-80.566	-77.0612931978711\\
-100.098	-95.7436303964514\\
-81.787	-78.2291783975162\\
-34.18	-32.6931335985805\\
-24.414	-23.3519649992903\\
-23.193	-22.1840797996453\\
-73.242	-70.055894997871\\
-128.174	-122.598294495742\\
-133.057	-127.268878795387\\
-137.939	-131.938506596097\\
-92.773	-88.7372756975163\\
-56.152	-53.7093281985808\\
-125.732	-120.262524096452\\
-86.67	-82.8997626971612\\
-68.359	-65.385310698226\\
-114.746	-109.754426796452\\
-100.098	-95.7436303964514\\
-85.449	-81.7318774975162\\
-142.822	-136.609090895742\\
-133.057	-127.268878795387\\
-102.539	-98.0784442968064\\
-150.146	-143.614489095742\\
-145.264	-138.944861295032\\
-113.525	-108.586541596807\\
-92.773	-88.7372756975163\\
-72.021	-68.888009798226\\
-86.67	-82.8997626971612\\
-112.305	-107.419612896097\\
-103.76	-99.2463294964515\\
-109.863	-105.083842496807\\
-112.305	-107.419612896097\\
-122.07	-116.759824996452\\
-175.781	-168.134339294677\\
-155.029	-148.285073395387\\
-102.539	-98.0784442968064\\
-92.773	-88.7372756975163\\
-53.711	-51.3745142982258\\
-58.594	-56.0450985978708\\
-79.346	-75.8943644971611\\
-59.814	-57.2120272985808\\
-62.256	-59.5477976978709\\
-89.111	-85.2345765975163\\
-102.539	-98.0784442968064\\
-95.215	-91.0730460968063\\
-151.367	-144.782374295387\\
-111.084	-106.251727696452\\
-64.697	-61.8826115982259\\
-51.27	-49.0397003978707\\
-69.58	-66.553195897871\\
-81.787	-78.2291783975162\\
-43.945	-42.0333456989356\\
-45.166	-43.2012308985806\\
-67.139	-64.2183819975159\\
-51.27	-49.0397003978707\\
-42.725	-40.8664169982256\\
-61.035	-58.3799124982258\\
-70.801	-67.721081097516\\
-109.863	-105.083842496807\\
-104.98	-100.413258197161\\
-129.395	-123.766179695387\\
-120.85	-115.592896295742\\
-137.939	-131.938506596097\\
-109.863	-105.083842496807\\
-107.422	-102.749028596452\\
-92.773	-88.7372756975163\\
-91.553	-87.5703469968063\\
-87.891	-84.0676478968062\\
-157.471	-150.620843794677\\
-224.609	-214.838269293258\\
-231.934	-221.844623992193\\
-225.83	-216.006154492903\\
-140.381	-134.274276995387\\
-205.078	-196.156888593613\\
-252.686	-241.693889891484\\
-261.23	-249.866216792194\\
-200.195	-191.486304293968\\
-270.996	-259.207385391484\\
-335.693	-321.08999698971\\
-235.596	-225.347323092193\\
-191.65	-183.313020894323\\
-128.174	-122.598294495742\\
-98.877	-94.5757451968064\\
-84.229	-80.5649487968062\\
-104.98	-100.413258197161\\
-74.463	-71.223780197516\\
-50.049	-47.8718151982257\\
-48.828	-46.7039299985807\\
-63.477	-60.7156828975159\\
-95.215	-91.0730460968063\\
-86.67	-82.8997626971612\\
-89.111	-85.2345765975163\\
-119.629	-114.425011096097\\
-123.291	-117.927710196097\\
-167.236	-159.961055895032\\
-134.277	-128.435807496097\\
-169.678	-162.296826294322\\
-122.07	-116.759824996452\\
-108.643	-103.916913796097\\
-125.732	-120.262524096452\\
-89.111	-85.2345765975163\\
-80.566	-77.0612931978711\\
-97.656	-93.4078599971614\\
-98.877	-94.5757451968064\\
-68.359	-65.385310698226\\
-73.242	-70.055894997871\\
-54.932	-52.5423994978708\\
-61.035	-58.3799124982258\\
-64.697	-61.8826115982259\\
-45.166	-43.2012308985806\\
-58.594	-56.0450985978708\\
-72.021	-68.888009798226\\
-117.188	-112.090197195742\\
-122.07	-116.759824996452\\
-136.719	-130.771577895387\\
-76.904	-73.5585940978711\\
-109.863	-105.083842496807\\
-173.34	-165.799525394322\\
-131.836	-126.100993595742\\
-157.471	-150.620843794677\\
-85.449	-81.7318774975162\\
-59.814	-57.2120272985808\\
-85.449	-81.7318774975162\\
-98.877	-94.5757451968064\\
-161.133	-154.123542894677\\
-202.637	-193.822074693258\\
-198.975	-190.319375593258\\
-145.264	-138.944861295032\\
-150.146	-143.614489095742\\
-135.498	-129.603692695742\\
-131.836	-126.100993595742\\
-126.953	-121.430409296097\\
-85.449	-81.7318774975162\\
-76.904	-73.5585940978711\\
-69.58	-66.553195897871\\
-62.256	-59.5477976978709\\
-70.801	-67.721081097516\\
-107.422	-102.749028596452\\
-164.795	-157.626241994677\\
-125.732	-120.262524096452\\
-86.67	-82.8997626971612\\
-73.242	-70.055894997871\\
-70.801	-67.721081097516\\
-42.725	-40.8664169982256\\
-31.738	-30.3573631992904\\
-39.063	-37.3637178982256\\
-72.021	-68.888009798226\\
-57.373	-54.8772133982258\\
-59.814	-57.2120272985808\\
-67.139	-64.2183819975159\\
-63.477	-60.7156828975159\\
-43.945	-42.0333456989356\\
-29.297	-28.0225492989354\\
-25.635	-24.5198501989354\\
-43.945	-42.0333456989356\\
-96.436	-92.2409312964514\\
-140.381	-134.274276995387\\
-155.029	-148.285073395387\\
-101.318	-96.9105590971614\\
-69.58	-66.553195897871\\
-50.049	-47.8718151982257\\
-34.18	-32.6931335985805\\
-83.008	-79.3970635971612\\
-81.787	-78.2291783975162\\
-119.629	-114.425011096097\\
-145.264	-138.944861295032\\
-230.713	-220.676738792548\\
-262.451	-251.034101991839\\
-211.182	-201.995358092903\\
-202.637	-193.822074693258\\
-136.719	-130.771577895387\\
-151.367	-144.782374295387\\
-159.912	-152.955657695032\\
-172.119	-164.631640194677\\
-129.395	-123.766179695387\\
-118.408	-113.257125896452\\
-115.967	-110.922311996097\\
-103.76	-99.2463294964515\\
-123.291	-117.927710196097\\
-87.891	-84.0676478968062\\
-153.809	-147.118144694677\\
-189.209	-180.978206993968\\
-133.057	-127.268878795387\\
-78.125	-74.7264792975161\\
-85.449	-81.7318774975162\\
-146.484	-140.111789995742\\
-181.885	-173.972808793968\\
-229.492	-219.508853592903\\
-241.699	-231.184836092548\\
-240.479	-230.017907391838\\
-180.664	-172.804923594323\\
-163.574	-156.458356795032\\
-187.988	-179.810321794323\\
-222.168	-212.503455392903\\
-139.16	-133.106391795742\\
-81.787	-78.2291783975162\\
-122.07	-116.759824996452\\
-102.539	-98.0784442968064\\
-65.918	-63.0504967978709\\
-87.891	-84.0676478968062\\
-98.877	-94.5757451968064\\
-76.904	-73.5585940978711\\
-58.594	-56.0450985978708\\
-80.566	-77.0612931978711\\
-65.918	-63.0504967978709\\
-91.553	-87.5703469968063\\
-133.057	-127.268878795387\\
-122.07	-116.759824996452\\
-79.346	-75.8943644971611\\
-83.008	-79.3970635971612\\
-126.953	-121.430409296097\\
-209.961	-200.827472893258\\
-150.146	-143.614489095742\\
-92.773	-88.7372756975163\\
-69.58	-66.553195897871\\
-83.008	-79.3970635971612\\
-74.463	-71.223780197516\\
-56.152	-53.7093281985808\\
-62.256	-59.5477976978709\\
-74.463	-71.223780197516\\
-96.436	-92.2409312964514\\
-107.422	-102.749028596452\\
-72.021	-68.888009798226\\
-122.07	-116.759824996452\\
-144.043	-137.776976095387\\
-137.939	-131.938506596097\\
-106.201	-101.581143396806\\
-118.408	-113.257125896452\\
-158.691	-151.787772495387\\
-101.318	-96.9105590971614\\
-41.504	-39.6985317985806\\
-58.594	-56.0450985978708\\
-65.918	-63.0504967978709\\
-91.553	-87.5703469968063\\
-81.787	-78.2291783975162\\
-59.814	-57.2120272985808\\
-96.436	-92.2409312964514\\
-91.553	-87.5703469968063\\
-65.918	-63.0504967978709\\
-83.008	-79.3970635971612\\
-76.904	-73.5585940978711\\
-67.139	-64.2183819975159\\
-79.346	-75.8943644971611\\
-46.387	-44.3691160982257\\
-53.711	-51.3745142982258\\
-40.283	-38.5306465989356\\
-43.945	-42.0333456989356\\
-86.67	-82.8997626971612\\
-100.098	-95.7436303964514\\
-137.939	-131.938506596097\\
-177.002	-169.302224494323\\
-115.967	-110.922311996097\\
-68.359	-65.385310698226\\
-48.828	-46.7039299985807\\
-73.242	-70.055894997871\\
-46.387	-44.3691160982257\\
-52.49	-50.2066290985807\\
-70.801	-67.721081097516\\
-119.629	-114.425011096097\\
-107.422	-102.749028596452\\
-150.146	-143.614489095742\\
-102.539	-98.0784442968064\\
-91.553	-87.5703469968063\\
-47.607	-45.5360447989357\\
-31.738	-30.3573631992904\\
-34.18	-32.6931335985805\\
-41.504	-39.6985317985806\\
-75.684	-72.3916653971611\\
-83.008	-79.3970635971612\\
-92.773	-88.7372756975163\\
-97.656	-93.4078599971614\\
-152.588	-145.950259495032\\
-130.615	-124.933108396097\\
-97.656	-93.4078599971614\\
-119.629	-114.425011096097\\
-129.395	-123.766179695387\\
-167.236	-159.961055895032\\
-133.057	-127.268878795387\\
-84.229	-80.5649487968062\\
-43.945	-42.0333456989356\\
-31.738	-30.3573631992904\\
-34.18	-32.6931335985805\\
-41.504	-39.6985317985806\\
-43.945	-42.0333456989356\\
-40.283	-38.5306465989356\\
-56.152	-53.7093281985808\\
-87.891	-84.0676478968062\\
-79.346	-75.8943644971611\\
-81.787	-78.2291783975162\\
-96.436	-92.2409312964514\\
-58.594	-56.0450985978708\\
-29.297	-28.0225492989354\\
-64.697	-61.8826115982259\\
-97.656	-93.4078599971614\\
-96.436	-92.2409312964514\\
-109.863	-105.083842496807\\
-93.994	-89.9051608971613\\
-69.58	-66.553195897871\\
-97.656	-93.4078599971614\\
-137.939	-131.938506596097\\
-139.16	-133.106391795742\\
-97.656	-93.4078599971614\\
-162.354	-155.291428094322\\
-128.174	-122.598294495742\\
-118.408	-113.257125896452\\
-137.939	-131.938506596097\\
-103.76	-99.2463294964515\\
-90.332	-86.4024617971613\\
-153.809	-147.118144694677\\
-214.844	-205.498057192903\\
-164.795	-157.626241994677\\
-92.773	-88.7372756975163\\
-91.553	-87.5703469968063\\
-113.525	-108.586541596807\\
-136.719	-130.771577895387\\
-100.098	-95.7436303964514\\
-104.98	-100.413258197161\\
-140.381	-134.274276995387\\
-153.809	-147.118144694677\\
-103.76	-99.2463294964515\\
-79.346	-75.8943644971611\\
-104.98	-100.413258197161\\
-175.781	-168.134339294677\\
-159.912	-152.955657695032\\
-162.354	-155.291428094322\\
-95.215	-91.0730460968063\\
-84.229	-80.5649487968062\\
-83.008	-79.3970635971612\\
-52.49	-50.2066290985807\\
-34.18	-32.6931335985805\\
-28.076	-26.8546640992904\\
-23.193	-22.1840797996453\\
-59.814	-57.2120272985808\\
-86.67	-82.8997626971612\\
-104.98	-100.413258197161\\
-76.904	-73.5585940978711\\
-87.891	-84.0676478968062\\
-97.656	-93.4078599971614\\
-59.814	-57.2120272985808\\
-63.477	-60.7156828975159\\
-61.035	-58.3799124982258\\
-45.166	-43.2012308985806\\
-57.373	-54.8772133982258\\
-97.656	-93.4078599971614\\
-124.512	-119.095595395742\\
-78.125	-74.7264792975161\\
-56.152	-53.7093281985808\\
-46.387	-44.3691160982257\\
-59.814	-57.2120272985808\\
-41.504	-39.6985317985806\\
-91.553	-87.5703469968063\\
-158.691	-151.787772495387\\
-122.07	-116.759824996452\\
-167.236	-159.961055895032\\
-194.092	-185.648791293613\\
-197.754	-189.151490393613\\
-140.381	-134.274276995387\\
-106.201	-101.581143396806\\
-104.98	-100.413258197161\\
-120.85	-115.592896295742\\
-150.146	-143.614489095742\\
-147.705	-141.279675195387\\
-200.195	-191.486304293968\\
-220.947	-211.335570193258\\
-263.672	-252.201987191484\\
-178.223	-170.470109693968\\
-190.43	-182.146092193613\\
-212.402	-203.162286793613\\
-163.574	-156.458356795032\\
-189.209	-180.978206993968\\
-163.574	-156.458356795032\\
-91.553	-87.5703469968063\\
-58.594	-56.0450985978708\\
-62.256	-59.5477976978709\\
-95.215	-91.0730460968063\\
-75.684	-72.3916653971611\\
-51.27	-49.0397003978707\\
-72.021	-68.888009798226\\
-52.49	-50.2066290985807\\
-40.283	-38.5306465989356\\
-48.828	-46.7039299985807\\
-56.152	-53.7093281985808\\
-40.283	-38.5306465989356\\
-76.904	-73.5585940978711\\
-109.863	-105.083842496807\\
-78.125	-74.7264792975161\\
-134.277	-128.435807496097\\
-195.313	-186.816676493258\\
-178.223	-170.470109693968\\
-222.168	-212.503455392903\\
-150.146	-143.614489095742\\
-161.133	-154.123542894677\\
-111.084	-106.251727696452\\
-67.139	-64.2183819975159\\
-85.449	-81.7318774975162\\
-67.139	-64.2183819975159\\
-48.828	-46.7039299985807\\
-72.021	-68.888009798226\\
-41.504	-39.6985317985806\\
-25.635	-24.5198501989354\\
-36.621	-35.0279474989355\\
-79.346	-75.8943644971611\\
-104.98	-100.413258197161\\
-128.174	-122.598294495742\\
-124.512	-119.095595395742\\
-117.188	-112.090197195742\\
-136.719	-130.771577895387\\
-111.084	-106.251727696452\\
-90.332	-86.4024617971613\\
-73.242	-70.055894997871\\
-115.967	-110.922311996097\\
-78.125	-74.7264792975161\\
-67.139	-64.2183819975159\\
-100.098	-95.7436303964514\\
-137.939	-131.938506596097\\
-103.76	-99.2463294964515\\
-76.904	-73.5585940978711\\
-64.697	-61.8826115982259\\
-74.463	-71.223780197516\\
-51.27	-49.0397003978707\\
-50.049	-47.8718151982257\\
-72.021	-68.888009798226\\
-86.67	-82.8997626971612\\
-98.877	-94.5757451968064\\
-79.346	-75.8943644971611\\
-101.318	-96.9105590971614\\
-92.773	-88.7372756975163\\
-52.49	-50.2066290985807\\
-54.932	-52.5423994978708\\
-112.305	-107.419612896097\\
-158.691	-151.787772495387\\
-153.809	-147.118144694677\\
-129.395	-123.766179695387\\
-122.07	-116.759824996452\\
-92.773	-88.7372756975163\\
-89.111	-85.2345765975163\\
-70.801	-67.721081097516\\
-102.539	-98.0784442968064\\
-90.332	-86.4024617971613\\
-54.932	-52.5423994978708\\
-84.229	-80.5649487968062\\
-101.318	-96.9105590971614\\
-118.408	-113.257125896452\\
-129.395	-123.766179695387\\
-175.781	-168.134339294677\\
-216.064	-206.664985893613\\
-244.141	-233.520606491838\\
-153.809	-147.118144694677\\
-85.449	-81.7318774975162\\
-54.932	-52.5423994978708\\
-37.842	-36.1958326985805\\
-57.373	-54.8772133982258\\
-53.711	-51.3745142982258\\
-95.215	-91.0730460968063\\
-85.449	-81.7318774975162\\
-75.684	-72.3916653971611\\
-102.539	-98.0784442968064\\
-118.408	-113.257125896452\\
-141.602	-135.442162195032\\
-148.926	-142.447560395032\\
-208.74	-199.659587693613\\
-181.885	-173.972808793968\\
-136.719	-130.771577895387\\
-90.332	-86.4024617971613\\
-92.773	-88.7372756975163\\
-95.215	-91.0730460968063\\
-123.291	-117.927710196097\\
-87.891	-84.0676478968062\\
-89.111	-85.2345765975163\\
-87.891	-84.0676478968062\\
-47.607	-45.5360447989357\\
-81.787	-78.2291783975162\\
-128.174	-122.598294495742\\
-90.332	-86.4024617971613\\
-96.436	-92.2409312964514\\
-170.898	-163.463754995032\\
-129.395	-123.766179695387\\
-122.07	-116.759824996452\\
-177.002	-169.302224494323\\
-166.016	-158.794127194322\\
-102.539	-98.0784442968064\\
-64.697	-61.8826115982259\\
-65.918	-63.0504967978709\\
-114.746	-109.754426796452\\
-190.43	-182.146092193613\\
-197.754	-189.151490393613\\
-206.299	-197.324773793258\\
-150.146	-143.614489095742\\
-97.656	-93.4078599971614\\
-83.008	-79.3970635971612\\
-58.594	-56.0450985978708\\
-56.152	-53.7093281985808\\
-47.607	-45.5360447989357\\
-68.359	-65.385310698226\\
-81.787	-78.2291783975162\\
-46.387	-44.3691160982257\\
-70.801	-67.721081097516\\
-103.76	-99.2463294964515\\
-114.746	-109.754426796452\\
-145.264	-138.944861295032\\
-185.547	-177.475507893968\\
-223.389	-213.671340592548\\
-161.133	-154.123542894677\\
-137.939	-131.938506596097\\
-144.043	-137.776976095387\\
-120.85	-115.592896295742\\
-84.229	-80.5649487968062\\
-103.76	-99.2463294964515\\
-167.236	-159.961055895032\\
-194.092	-185.648791293613\\
-115.967	-110.922311996097\\
-62.256	-59.5477976978709\\
-41.504	-39.6985317985806\\
-23.193	-22.1840797996453\\
-31.738	-30.3573631992904\\
-51.27	-49.0397003978707\\
-61.035	-58.3799124982258\\
-80.566	-77.0612931978711\\
-67.139	-64.2183819975159\\
-51.27	-49.0397003978707\\
-50.049	-47.8718151982257\\
-48.828	-46.7039299985807\\
-45.166	-43.2012308985806\\
-52.49	-50.2066290985807\\
-81.787	-78.2291783975162\\
-123.291	-117.927710196097\\
-133.057	-127.268878795387\\
-125.732	-120.262524096452\\
-152.588	-145.950259495032\\
-181.885	-173.972808793968\\
-183.105	-175.139737494678\\
-223.389	-213.671340592548\\
-201.416	-192.654189493613\\
-146.484	-140.111789995742\\
-101.318	-96.9105590971614\\
-96.436	-92.2409312964514\\
-163.574	-156.458356795032\\
-191.65	-183.313020894323\\
-131.836	-126.100993595742\\
-85.449	-81.7318774975162\\
-45.166	-43.2012308985806\\
-31.738	-30.3573631992904\\
-35.4	-33.8600622992905\\
-21.973	-21.0171510989353\\
-46.387	-44.3691160982257\\
-68.359	-65.385310698226\\
-72.021	-68.888009798226\\
-62.256	-59.5477976978709\\
-36.621	-35.0279474989355\\
-28.076	-26.8546640992904\\
-41.504	-39.6985317985806\\
-43.945	-42.0333456989356\\
-48.828	-46.7039299985807\\
-36.621	-35.0279474989355\\
-30.518	-29.1904344985804\\
-54.932	-52.5423994978708\\
-128.174	-122.598294495742\\
-153.809	-147.118144694677\\
-170.898	-163.463754995032\\
-124.512	-119.095595395742\\
-172.119	-164.631640194677\\
-241.699	-231.184836092548\\
-222.168	-212.503455392903\\
-224.609	-214.838269293258\\
-164.795	-157.626241994677\\
-163.574	-156.458356795032\\
-172.119	-164.631640194677\\
-113.525	-108.586541596807\\
-106.201	-101.581143396806\\
-65.918	-63.0504967978709\\
-59.814	-57.2120272985808\\
-117.188	-112.090197195742\\
-79.346	-75.8943644971611\\
-81.787	-78.2291783975162\\
-96.436	-92.2409312964514\\
-86.67	-82.8997626971612\\
-87.891	-84.0676478968062\\
-93.994	-89.9051608971613\\
-107.422	-102.749028596452\\
-119.629	-114.425011096097\\
-103.76	-99.2463294964515\\
-74.463	-71.223780197516\\
-96.436	-92.2409312964514\\
-48.828	-46.7039299985807\\
-20.752	-19.8492658992903\\
-34.18	-32.6931335985805\\
-58.594	-56.0450985978708\\
-30.518	-29.1904344985804\\
-13.428	-12.8438676992902\\
-32.959	-31.5252483989355\\
-58.594	-56.0450985978708\\
-69.58	-66.553195897871\\
-83.008	-79.3970635971612\\
-72.021	-68.888009798226\\
-117.188	-112.090197195742\\
-104.98	-100.413258197161\\
-70.801	-67.721081097516\\
-107.422	-102.749028596452\\
-59.814	-57.2120272985808\\
-46.387	-44.3691160982257\\
-32.959	-31.5252483989355\\
-21.973	-21.0171510989353\\
-41.504	-39.6985317985806\\
-34.18	-32.6931335985805\\
-53.711	-51.3745142982258\\
-91.553	-87.5703469968063\\
-87.891	-84.0676478968062\\
-59.814	-57.2120272985808\\
-68.359	-65.385310698226\\
-52.49	-50.2066290985807\\
-74.463	-71.223780197516\\
-53.711	-51.3745142982258\\
-50.049	-47.8718151982257\\
-46.387	-44.3691160982257\\
-36.621	-35.0279474989355\\
-45.166	-43.2012308985806\\
-41.504	-39.6985317985806\\
-75.684	-72.3916653971611\\
-73.242	-70.055894997871\\
-54.932	-52.5423994978708\\
-51.27	-49.0397003978707\\
-40.283	-38.5306465989356\\
-48.828	-46.7039299985807\\
-108.643	-103.916913796097\\
-144.043	-137.776976095387\\
-96.436	-92.2409312964514\\
-90.332	-86.4024617971613\\
-63.477	-60.7156828975159\\
-111.084	-106.251727696452\\
-101.318	-96.9105590971614\\
-67.139	-64.2183819975159\\
-83.008	-79.3970635971612\\
-64.697	-61.8826115982259\\
-89.111	-85.2345765975163\\
-52.49	-50.2066290985807\\
-54.932	-52.5423994978708\\
-65.918	-63.0504967978709\\
-84.229	-80.5649487968062\\
-62.256	-59.5477976978709\\
-65.918	-63.0504967978709\\
-68.359	-65.385310698226\\
-112.305	-107.419612896097\\
-95.215	-91.0730460968063\\
-146.484	-140.111789995742\\
-190.43	-182.146092193613\\
-151.367	-144.782374295387\\
-95.215	-91.0730460968063\\
-72.021	-68.888009798226\\
-113.525	-108.586541596807\\
-141.602	-135.442162195032\\
-108.643	-103.916913796097\\
-129.395	-123.766179695387\\
-81.787	-78.2291783975162\\
-42.725	-40.8664169982256\\
-98.877	-94.5757451968064\\
-89.111	-85.2345765975163\\
-79.346	-75.8943644971611\\
-124.512	-119.095595395742\\
-118.408	-113.257125896452\\
-103.76	-99.2463294964515\\
-67.139	-64.2183819975159\\
-84.229	-80.5649487968062\\
-115.967	-110.922311996097\\
-95.215	-91.0730460968063\\
-97.656	-93.4078599971614\\
-89.111	-85.2345765975163\\
-62.256	-59.5477976978709\\
-75.684	-72.3916653971611\\
-54.932	-52.5423994978708\\
-62.256	-59.5477976978709\\
-106.201	-101.581143396806\\
-123.291	-117.927710196097\\
-131.836	-126.100993595742\\
-117.188	-112.090197195742\\
-83.008	-79.3970635971612\\
-58.594	-56.0450985978708\\
-67.139	-64.2183819975159\\
-80.566	-77.0612931978711\\
-104.98	-100.413258197161\\
-130.615	-124.933108396097\\
-155.029	-148.285073395387\\
-114.746	-109.754426796452\\
-69.58	-66.553195897871\\
-126.953	-121.430409296097\\
-178.223	-170.470109693968\\
-128.174	-122.598294495742\\
-125.732	-120.262524096452\\
-147.705	-141.279675195387\\
-111.084	-106.251727696452\\
-90.332	-86.4024617971613\\
-100.098	-95.7436303964514\\
-92.773	-88.7372756975163\\
-103.76	-99.2463294964515\\
-131.836	-126.100993595742\\
-101.318	-96.9105590971614\\
-112.305	-107.419612896097\\
-106.201	-101.581143396806\\
-108.643	-103.916913796097\\
-86.67	-82.8997626971612\\
-114.746	-109.754426796452\\
-81.787	-78.2291783975162\\
-86.67	-82.8997626971612\\
-96.436	-92.2409312964514\\
-93.994	-89.9051608971613\\
-42.725	-40.8664169982256\\
-26.855	-25.6867788996454\\
-18.311	-17.5144519989353\\
-39.063	-37.3637178982256\\
-96.436	-92.2409312964514\\
-125.732	-120.262524096452\\
-83.008	-79.3970635971612\\
-97.656	-93.4078599971614\\
-85.449	-81.7318774975162\\
-75.684	-72.3916653971611\\
-46.387	-44.3691160982257\\
-65.918	-63.0504967978709\\
-64.697	-61.8826115982259\\
-76.904	-73.5585940978711\\
-65.918	-63.0504967978709\\
-59.814	-57.2120272985808\\
-40.283	-38.5306465989356\\
-58.594	-56.0450985978708\\
-109.863	-105.083842496807\\
-78.125	-74.7264792975161\\
-92.773	-88.7372756975163\\
-83.008	-79.3970635971612\\
-118.408	-113.257125896452\\
-109.863	-105.083842496807\\
-152.588	-145.950259495032\\
-111.084	-106.251727696452\\
-124.512	-119.095595395742\\
-63.477	-60.7156828975159\\
-113.525	-108.586541596807\\
-76.904	-73.5585940978711\\
-96.436	-92.2409312964514\\
-102.539	-98.0784442968064\\
-129.395	-123.766179695387\\
-97.656	-93.4078599971614\\
-125.732	-120.262524096452\\
-158.691	-151.787772495387\\
-131.836	-126.100993595742\\
-114.746	-109.754426796452\\
-90.332	-86.4024617971613\\
-107.422	-102.749028596452\\
-89.111	-85.2345765975163\\
-76.904	-73.5585940978711\\
-68.359	-65.385310698226\\
-81.787	-78.2291783975162\\
-69.58	-66.553195897871\\
-141.602	-135.442162195032\\
-181.885	-173.972808793968\\
-157.471	-150.620843794677\\
-150.146	-143.614489095742\\
-147.705	-141.279675195387\\
-81.787	-78.2291783975162\\
-72.021	-68.888009798226\\
-54.932	-52.5423994978708\\
-50.049	-47.8718151982257\\
-54.932	-52.5423994978708\\
-93.994	-89.9051608971613\\
-118.408	-113.257125896452\\
-134.277	-128.435807496097\\
-114.746	-109.754426796452\\
-76.904	-73.5585940978711\\
-140.381	-134.274276995387\\
-163.574	-156.458356795032\\
-183.105	-175.139737494678\\
-150.146	-143.614489095742\\
-236.816	-226.514251792903\\
-158.691	-151.787772495387\\
-93.994	-89.9051608971613\\
-68.359	-65.385310698226\\
-56.152	-53.7093281985808\\
-101.318	-96.9105590971614\\
-130.615	-124.933108396097\\
-172.119	-164.631640194677\\
-129.395	-123.766179695387\\
-102.539	-98.0784442968064\\
-53.711	-51.3745142982258\\
-64.697	-61.8826115982259\\
-70.801	-67.721081097516\\
-43.945	-42.0333456989356\\
-61.035	-58.3799124982258\\
-45.166	-43.2012308985806\\
-29.297	-28.0225492989354\\
-65.918	-63.0504967978709\\
-75.684	-72.3916653971611\\
-96.436	-92.2409312964514\\
-89.111	-85.2345765975163\\
-93.994	-89.9051608971613\\
-120.85	-115.592896295742\\
-79.346	-75.8943644971611\\
-119.629	-114.425011096097\\
-136.719	-130.771577895387\\
-103.76	-99.2463294964515\\
-108.643	-103.916913796097\\
-93.994	-89.9051608971613\\
-108.643	-103.916913796097\\
-151.367	-144.782374295387\\
-117.188	-112.090197195742\\
-93.994	-89.9051608971613\\
-102.539	-98.0784442968064\\
-93.994	-89.9051608971613\\
-62.256	-59.5477976978709\\
-100.098	-95.7436303964514\\
-145.264	-138.944861295032\\
-139.16	-133.106391795742\\
-151.367	-144.782374295387\\
-227.051	-217.174039692548\\
-139.16	-133.106391795742\\
-122.07	-116.759824996452\\
-91.553	-87.5703469968063\\
-119.629	-114.425011096097\\
-172.119	-164.631640194677\\
-114.746	-109.754426796452\\
-102.539	-98.0784442968064\\
-142.822	-136.609090895742\\
-91.553	-87.5703469968063\\
-53.711	-51.3745142982258\\
-70.801	-67.721081097516\\
-52.49	-50.2066290985807\\
-42.725	-40.8664169982256\\
-46.387	-44.3691160982257\\
-40.283	-38.5306465989356\\
-39.063	-37.3637178982256\\
-53.711	-51.3745142982258\\
-76.904	-73.5585940978711\\
-90.332	-86.4024617971613\\
-119.629	-114.425011096097\\
-112.305	-107.419612896097\\
-135.498	-129.603692695742\\
-157.471	-150.620843794677\\
-141.602	-135.442162195032\\
-96.436	-92.2409312964514\\
-106.201	-101.581143396806\\
-95.215	-91.0730460968063\\
-103.76	-99.2463294964515\\
-147.705	-141.279675195387\\
-109.863	-105.083842496807\\
-122.07	-116.759824996452\\
-106.201	-101.581143396806\\
-102.539	-98.0784442968064\\
-158.691	-151.787772495387\\
-130.615	-124.933108396097\\
-129.395	-123.766179695387\\
-122.07	-116.759824996452\\
-73.242	-70.055894997871\\
-47.607	-45.5360447989357\\
-37.842	-36.1958326985805\\
-69.58	-66.553195897871\\
-62.256	-59.5477976978709\\
-86.67	-82.8997626971612\\
-65.918	-63.0504967978709\\
-93.994	-89.9051608971613\\
-152.588	-145.950259495032\\
-189.209	-180.978206993968\\
-150.146	-143.614489095742\\
-153.809	-147.118144694677\\
-137.939	-131.938506596097\\
-96.436	-92.2409312964514\\
-103.76	-99.2463294964515\\
-117.188	-112.090197195742\\
-100.098	-95.7436303964514\\
-114.746	-109.754426796452\\
-145.264	-138.944861295032\\
-202.637	-193.822074693258\\
-181.885	-173.972808793968\\
-170.898	-163.463754995032\\
-191.65	-183.313020894323\\
-131.836	-126.100993595742\\
-144.043	-137.776976095387\\
-108.643	-103.916913796097\\
-112.305	-107.419612896097\\
-147.705	-141.279675195387\\
-172.119	-164.631640194677\\
-79.346	-75.8943644971611\\
-89.111	-85.2345765975163\\
-67.139	-64.2183819975159\\
-98.877	-94.5757451968064\\
-100.098	-95.7436303964514\\
-61.035	-58.3799124982258\\
-81.787	-78.2291783975162\\
-139.16	-133.106391795742\\
-234.375	-224.179437892548\\
-227.051	-217.174039692548\\
-206.299	-197.324773793258\\
-166.016	-158.794127194322\\
-194.092	-185.648791293613\\
-233.154	-223.011552692903\\
-170.898	-163.463754995032\\
-177.002	-169.302224494323\\
-184.326	-176.307622694323\\
-125.732	-120.262524096452\\
-106.201	-101.581143396806\\
-137.939	-131.938506596097\\
-91.553	-87.5703469968063\\
-67.139	-64.2183819975159\\
-97.656	-93.4078599971614\\
-73.242	-70.055894997871\\
-86.67	-82.8997626971612\\
-81.787	-78.2291783975162\\
-100.098	-95.7436303964514\\
-133.057	-127.268878795387\\
-102.539	-98.0784442968064\\
-75.684	-72.3916653971611\\
-89.111	-85.2345765975163\\
-104.98	-100.413258197161\\
-124.512	-119.095595395742\\
-107.422	-102.749028596452\\
-86.67	-82.8997626971612\\
-136.719	-130.771577895387\\
-129.395	-123.766179695387\\
-90.332	-86.4024617971613\\
-148.926	-142.447560395032\\
-134.277	-128.435807496097\\
-97.656	-93.4078599971614\\
-64.697	-61.8826115982259\\
-46.387	-44.3691160982257\\
-32.959	-31.5252483989355\\
-75.684	-72.3916653971611\\
-72.021	-68.888009798226\\
-52.49	-50.2066290985807\\
-36.621	-35.0279474989355\\
-47.607	-45.5360447989357\\
-81.787	-78.2291783975162\\
-93.994	-89.9051608971613\\
-124.512	-119.095595395742\\
-142.822	-136.609090895742\\
-137.939	-131.938506596097\\
-145.264	-138.944861295032\\
-87.891	-84.0676478968062\\
-36.621	-35.0279474989355\\
-47.607	-45.5360447989357\\
-46.387	-44.3691160982257\\
-64.697	-61.8826115982259\\
-100.098	-95.7436303964514\\
-111.084	-106.251727696452\\
-168.457	-161.128941094677\\
-106.201	-101.581143396806\\
-104.98	-100.413258197161\\
-155.029	-148.285073395387\\
-106.201	-101.581143396806\\
-74.463	-71.223780197516\\
-53.711	-51.3745142982258\\
-76.904	-73.5585940978711\\
-100.098	-95.7436303964514\\
-151.367	-144.782374295387\\
-98.877	-94.5757451968064\\
-85.449	-81.7318774975162\\
-80.566	-77.0612931978711\\
-101.318	-96.9105590971614\\
-133.057	-127.268878795387\\
-207.52	-198.492658992903\\
-184.326	-176.307622694323\\
-180.664	-172.804923594323\\
-112.305	-107.419612896097\\
-75.684	-72.3916653971611\\
-87.891	-84.0676478968062\\
-36.621	-35.0279474989355\\
-67.139	-64.2183819975159\\
-52.49	-50.2066290985807\\
-65.918	-63.0504967978709\\
-117.188	-112.090197195742\\
-153.809	-147.118144694677\\
-177.002	-169.302224494323\\
-234.375	-224.179437892548\\
-196.533	-187.983605193968\\
-106.201	-101.581143396806\\
-98.877	-94.5757451968064\\
-79.346	-75.8943644971611\\
-108.643	-103.916913796097\\
-141.602	-135.442162195032\\
-191.65	-183.313020894323\\
-140.381	-134.274276995387\\
-97.656	-93.4078599971614\\
-108.643	-103.916913796097\\
-144.043	-137.776976095387\\
-102.539	-98.0784442968064\\
-73.242	-70.055894997871\\
-57.373	-54.8772133982258\\
-40.283	-38.5306465989356\\
-36.621	-35.0279474989355\\
-29.297	-28.0225492989354\\
-54.932	-52.5423994978708\\
-80.566	-77.0612931978711\\
-86.67	-82.8997626971612\\
-64.697	-61.8826115982259\\
-58.594	-56.0450985978708\\
-26.855	-25.6867788996454\\
-15.869	-15.1786815996452\\
-24.414	-23.3519649992903\\
-46.387	-44.3691160982257\\
-70.801	-67.721081097516\\
-87.891	-84.0676478968062\\
-102.539	-98.0784442968064\\
-119.629	-114.425011096097\\
-153.809	-147.118144694677\\
-212.402	-203.162286793613\\
-213.623	-204.330171993258\\
-230.713	-220.676738792548\\
-201.416	-192.654189493613\\
-111.084	-106.251727696452\\
-102.539	-98.0784442968064\\
-101.318	-96.9105590971614\\
-135.498	-129.603692695742\\
-113.525	-108.586541596807\\
-80.566	-77.0612931978711\\
-61.035	-58.3799124982258\\
-50.049	-47.8718151982257\\
-93.994	-89.9051608971613\\
-90.332	-86.4024617971613\\
-47.607	-45.5360447989357\\
-37.842	-36.1958326985805\\
-58.594	-56.0450985978708\\
-78.125	-74.7264792975161\\
-59.814	-57.2120272985808\\
-86.67	-82.8997626971612\\
-67.139	-64.2183819975159\\
-98.877	-94.5757451968064\\
-141.602	-135.442162195032\\
-112.305	-107.419612896097\\
-153.809	-147.118144694677\\
-206.299	-197.324773793258\\
-224.609	-214.838269293258\\
-175.781	-168.134339294677\\
-111.084	-106.251727696452\\
-83.008	-79.3970635971612\\
-48.828	-46.7039299985807\\
-81.787	-78.2291783975162\\
-57.373	-54.8772133982258\\
-109.863	-105.083842496807\\
-92.773	-88.7372756975163\\
-74.463	-71.223780197516\\
-96.436	-92.2409312964514\\
-52.49	-50.2066290985807\\
-84.229	-80.5649487968062\\
-58.594	-56.0450985978708\\
-37.842	-36.1958326985805\\
-45.166	-43.2012308985806\\
-32.959	-31.5252483989355\\
-40.283	-38.5306465989356\\
-35.4	-33.8600622992905\\
-26.855	-25.6867788996454\\
-36.621	-35.0279474989355\\
-48.828	-46.7039299985807\\
-50.049	-47.8718151982257\\
-48.828	-46.7039299985807\\
-79.346	-75.8943644971611\\
-103.76	-99.2463294964515\\
-84.229	-80.5649487968062\\
-78.125	-74.7264792975161\\
-128.174	-122.598294495742\\
-153.809	-147.118144694677\\
-119.629	-114.425011096097\\
-125.732	-120.262524096452\\
-56.152	-53.7093281985808\\
-36.621	-35.0279474989355\\
-73.242	-70.055894997871\\
-106.201	-101.581143396806\\
-115.967	-110.922311996097\\
-162.354	-155.291428094322\\
-146.484	-140.111789995742\\
-102.539	-98.0784442968064\\
-73.242	-70.055894997871\\
-79.346	-75.8943644971611\\
-84.229	-80.5649487968062\\
-64.697	-61.8826115982259\\
-39.063	-37.3637178982256\\
-52.49	-50.2066290985807\\
-39.063	-37.3637178982256\\
-32.959	-31.5252483989355\\
-24.414	-23.3519649992903\\
-53.711	-51.3745142982258\\
-74.463	-71.223780197516\\
-79.346	-75.8943644971611\\
-56.152	-53.7093281985808\\
-100.098	-95.7436303964514\\
-136.719	-130.771577895387\\
-86.67	-82.8997626971612\\
-120.85	-115.592896295742\\
-130.615	-124.933108396097\\
-98.877	-94.5757451968064\\
-56.152	-53.7093281985808\\
-72.021	-68.888009798226\\
-87.891	-84.0676478968062\\
-47.607	-45.5360447989357\\
-50.049	-47.8718151982257\\
-41.504	-39.6985317985806\\
-24.414	-23.3519649992903\\
-26.855	-25.6867788996454\\
-46.387	-44.3691160982257\\
-64.697	-61.8826115982259\\
-54.932	-52.5423994978708\\
-72.021	-68.888009798226\\
-79.346	-75.8943644971611\\
-57.373	-54.8772133982258\\
-31.738	-30.3573631992904\\
-25.635	-24.5198501989354\\
-34.18	-32.6931335985805\\
-39.063	-37.3637178982256\\
-48.828	-46.7039299985807\\
-104.98	-100.413258197161\\
-101.318	-96.9105590971614\\
-89.111	-85.2345765975163\\
-85.449	-81.7318774975162\\
-117.188	-112.090197195742\\
-150.146	-143.614489095742\\
-159.912	-152.955657695032\\
-166.016	-158.794127194322\\
-120.85	-115.592896295742\\
-126.953	-121.430409296097\\
-131.836	-126.100993595742\\
-133.057	-127.268878795387\\
-151.367	-144.782374295387\\
-200.195	-191.486304293968\\
-169.678	-162.296826294322\\
-156.25	-149.452958595032\\
-128.174	-122.598294495742\\
-119.629	-114.425011096097\\
-114.746	-109.754426796452\\
-136.719	-130.771577895387\\
-83.008	-79.3970635971612\\
-98.877	-94.5757451968064\\
-119.629	-114.425011096097\\
-140.381	-134.274276995387\\
-107.422	-102.749028596452\\
-124.512	-119.095595395742\\
-98.877	-94.5757451968064\\
-90.332	-86.4024617971613\\
-54.932	-52.5423994978708\\
-85.449	-81.7318774975162\\
-137.939	-131.938506596097\\
-109.863	-105.083842496807\\
-64.697	-61.8826115982259\\
-80.566	-77.0612931978711\\
-41.504	-39.6985317985806\\
-40.283	-38.5306465989356\\
-56.152	-53.7093281985808\\
-32.959	-31.5252483989355\\
-36.621	-35.0279474989355\\
-43.945	-42.0333456989356\\
-26.855	-25.6867788996454\\
-50.049	-47.8718151982257\\
-40.283	-38.5306465989356\\
-23.193	-22.1840797996453\\
-43.945	-42.0333456989356\\
-51.27	-49.0397003978707\\
-63.477	-60.7156828975159\\
-62.256	-59.5477976978709\\
-63.477	-60.7156828975159\\
-84.229	-80.5649487968062\\
-76.904	-73.5585940978711\\
-86.67	-82.8997626971612\\
-146.484	-140.111789995742\\
-101.318	-96.9105590971614\\
-92.773	-88.7372756975163\\
-100.098	-95.7436303964514\\
-69.58	-66.553195897871\\
-40.283	-38.5306465989356\\
-79.346	-75.8943644971611\\
-61.035	-58.3799124982258\\
-37.842	-36.1958326985805\\
-67.139	-64.2183819975159\\
-62.256	-59.5477976978709\\
-80.566	-77.0612931978711\\
-48.828	-46.7039299985807\\
-29.297	-28.0225492989354\\
-23.193	-22.1840797996453\\
-29.297	-28.0225492989354\\
-56.152	-53.7093281985808\\
-54.932	-52.5423994978708\\
-65.918	-63.0504967978709\\
-87.891	-84.0676478968062\\
-117.188	-112.090197195742\\
-84.229	-80.5649487968062\\
-124.512	-119.095595395742\\
-141.602	-135.442162195032\\
-124.512	-119.095595395742\\
-117.188	-112.090197195742\\
-195.313	-186.816676493258\\
-135.498	-129.603692695742\\
-173.34	-165.799525394322\\
-141.602	-135.442162195032\\
-164.795	-157.626241994677\\
-184.326	-176.307622694323\\
-101.318	-96.9105590971614\\
-63.477	-60.7156828975159\\
-51.27	-49.0397003978707\\
-84.229	-80.5649487968062\\
-104.98	-100.413258197161\\
-108.643	-103.916913796097\\
-74.463	-71.223780197516\\
-40.283	-38.5306465989356\\
-52.49	-50.2066290985807\\
-104.98	-100.413258197161\\
-145.264	-138.944861295032\\
-142.822	-136.609090895742\\
-84.229	-80.5649487968062\\
-64.697	-61.8826115982259\\
-89.111	-85.2345765975163\\
-141.602	-135.442162195032\\
-161.133	-154.123542894677\\
-164.795	-157.626241994677\\
-115.967	-110.922311996097\\
-166.016	-158.794127194322\\
-178.223	-170.470109693968\\
-170.898	-163.463754995032\\
-240.479	-230.017907391838\\
-203.857	-194.989003393968\\
-169.678	-162.296826294322\\
-196.533	-187.983605193968\\
-164.795	-157.626241994677\\
-91.553	-87.5703469968063\\
-84.229	-80.5649487968062\\
-104.98	-100.413258197161\\
-125.732	-120.262524096452\\
-128.174	-122.598294495742\\
-96.436	-92.2409312964514\\
-124.512	-119.095595395742\\
-106.201	-101.581143396806\\
-78.125	-74.7264792975161\\
-106.201	-101.581143396806\\
-95.215	-91.0730460968063\\
-150.146	-143.614489095742\\
-151.367	-144.782374295387\\
-108.643	-103.916913796097\\
-79.346	-75.8943644971611\\
-45.166	-43.2012308985806\\
-78.125	-74.7264792975161\\
-53.711	-51.3745142982258\\
-52.49	-50.2066290985807\\
-42.725	-40.8664169982256\\
-76.904	-73.5585940978711\\
-98.877	-94.5757451968064\\
-112.305	-107.419612896097\\
-109.863	-105.083842496807\\
-64.697	-61.8826115982259\\
-48.828	-46.7039299985807\\
-35.4	-33.8600622992905\\
-43.945	-42.0333456989356\\
-70.801	-67.721081097516\\
-75.684	-72.3916653971611\\
-74.463	-71.223780197516\\
-120.85	-115.592896295742\\
-140.381	-134.274276995387\\
-162.354	-155.291428094322\\
-159.912	-152.955657695032\\
-187.988	-179.810321794323\\
-109.863	-105.083842496807\\
-91.553	-87.5703469968063\\
-72.021	-68.888009798226\\
-78.125	-74.7264792975161\\
-104.98	-100.413258197161\\
-86.67	-82.8997626971612\\
-57.373	-54.8772133982258\\
-50.049	-47.8718151982257\\
-97.656	-93.4078599971614\\
-117.188	-112.090197195742\\
-107.422	-102.749028596452\\
-168.457	-161.128941094677\\
-184.326	-176.307622694323\\
-130.615	-124.933108396097\\
-91.553	-87.5703469968063\\
-62.256	-59.5477976978709\\
-57.373	-54.8772133982258\\
-107.422	-102.749028596452\\
-85.449	-81.7318774975162\\
-103.76	-99.2463294964515\\
-113.525	-108.586541596807\\
-129.395	-123.766179695387\\
-122.07	-116.759824996452\\
-135.498	-129.603692695742\\
-91.553	-87.5703469968063\\
-109.863	-105.083842496807\\
-75.684	-72.3916653971611\\
-68.359	-65.385310698226\\
-107.422	-102.749028596452\\
-147.705	-141.279675195387\\
-162.354	-155.291428094322\\
-91.553	-87.5703469968063\\
-190.43	-182.146092193613\\
-170.898	-163.463754995032\\
-113.525	-108.586541596807\\
-65.918	-63.0504967978709\\
-96.436	-92.2409312964514\\
-83.008	-79.3970635971612\\
-80.566	-77.0612931978711\\
-98.877	-94.5757451968064\\
-64.697	-61.8826115982259\\
-93.994	-89.9051608971613\\
-170.898	-163.463754995032\\
-179.443	-171.637038394678\\
-117.188	-112.090197195742\\
-134.277	-128.435807496097\\
-146.484	-140.111789995742\\
-148.926	-142.447560395032\\
-106.201	-101.581143396806\\
-69.58	-66.553195897871\\
-106.201	-101.581143396806\\
-109.863	-105.083842496807\\
-175.781	-168.134339294677\\
-195.313	-186.816676493258\\
-263.672	-252.201987191484\\
-190.43	-182.146092193613\\
-155.029	-148.285073395387\\
-118.408	-113.257125896452\\
-130.615	-124.933108396097\\
-97.656	-93.4078599971614\\
-67.139	-64.2183819975159\\
-68.359	-65.385310698226\\
-48.828	-46.7039299985807\\
-35.4	-33.8600622992905\\
-59.814	-57.2120272985808\\
-81.787	-78.2291783975162\\
-54.932	-52.5423994978708\\
-39.063	-37.3637178982256\\
-37.842	-36.1958326985805\\
-90.332	-86.4024617971613\\
-128.174	-122.598294495742\\
-59.814	-57.2120272985808\\
-74.463	-71.223780197516\\
-79.346	-75.8943644971611\\
-73.242	-70.055894997871\\
-91.553	-87.5703469968063\\
-79.346	-75.8943644971611\\
-54.932	-52.5423994978708\\
-41.504	-39.6985317985806\\
-34.18	-32.6931335985805\\
-43.945	-42.0333456989356\\
-86.67	-82.8997626971612\\
-103.76	-99.2463294964515\\
-72.021	-68.888009798226\\
-48.828	-46.7039299985807\\
-29.297	-28.0225492989354\\
-42.725	-40.8664169982256\\
-80.566	-77.0612931978711\\
-101.318	-96.9105590971614\\
-107.422	-102.749028596452\\
-76.904	-73.5585940978711\\
-70.801	-67.721081097516\\
-78.125	-74.7264792975161\\
-108.643	-103.916913796097\\
-151.367	-144.782374295387\\
-175.781	-168.134339294677\\
-131.836	-126.100993595742\\
-83.008	-79.3970635971612\\
-86.67	-82.8997626971612\\
-81.787	-78.2291783975162\\
-84.229	-80.5649487968062\\
-54.932	-52.5423994978708\\
-69.58	-66.553195897871\\
-76.904	-73.5585940978711\\
-45.166	-43.2012308985806\\
-29.297	-28.0225492989354\\
-47.607	-45.5360447989357\\
-70.801	-67.721081097516\\
-109.863	-105.083842496807\\
-118.408	-113.257125896452\\
-148.926	-142.447560395032\\
-130.615	-124.933108396097\\
-146.484	-140.111789995742\\
-195.313	-186.816676493258\\
-260.01	-248.699288091484\\
-212.402	-203.162286793613\\
-142.822	-136.609090895742\\
-155.029	-148.285073395387\\
-180.664	-172.804923594323\\
-236.816	-226.514251792903\\
-155.029	-148.285073395387\\
-76.904	-73.5585940978711\\
-50.049	-47.8718151982257\\
-52.49	-50.2066290985807\\
-37.842	-36.1958326985805\\
-23.193	-22.1840797996453\\
-30.518	-29.1904344985804\\
-48.828	-46.7039299985807\\
-68.359	-65.385310698226\\
-70.801	-67.721081097516\\
-54.932	-52.5423994978708\\
-48.828	-46.7039299985807\\
-67.139	-64.2183819975159\\
-81.787	-78.2291783975162\\
-85.449	-81.7318774975162\\
-79.346	-75.8943644971611\\
-57.373	-54.8772133982258\\
-37.842	-36.1958326985805\\
-57.373	-54.8772133982258\\
-73.242	-70.055894997871\\
-53.711	-51.3745142982258\\
-35.4	-33.8600622992905\\
-58.594	-56.0450985978708\\
-39.063	-37.3637178982256\\
-50.049	-47.8718151982257\\
-63.477	-60.7156828975159\\
-92.773	-88.7372756975163\\
-100.098	-95.7436303964514\\
-68.359	-65.385310698226\\
-106.201	-101.581143396806\\
-102.539	-98.0784442968064\\
-83.008	-79.3970635971612\\
-70.801	-67.721081097516\\
-119.629	-114.425011096097\\
-146.484	-140.111789995742\\
-144.043	-137.776976095387\\
-158.691	-151.787772495387\\
-120.85	-115.592896295742\\
-111.084	-106.251727696452\\
-102.539	-98.0784442968064\\
-139.16	-133.106391795742\\
-108.643	-103.916913796097\\
-147.705	-141.279675195387\\
-141.602	-135.442162195032\\
-146.484	-140.111789995742\\
-249.023	-238.190234292548\\
-190.43	-182.146092193613\\
-102.539	-98.0784442968064\\
-69.58	-66.553195897871\\
-73.242	-70.055894997871\\
-91.553	-87.5703469968063\\
-119.629	-114.425011096097\\
-139.16	-133.106391795742\\
-153.809	-147.118144694677\\
-157.471	-150.620843794677\\
-96.436	-92.2409312964514\\
-86.67	-82.8997626971612\\
-101.318	-96.9105590971614\\
-102.539	-98.0784442968064\\
-61.035	-58.3799124982258\\
-43.945	-42.0333456989356\\
-79.346	-75.8943644971611\\
-80.566	-77.0612931978711\\
-109.863	-105.083842496807\\
-139.16	-133.106391795742\\
-123.291	-117.927710196097\\
-87.891	-84.0676478968062\\
-70.801	-67.721081097516\\
-107.422	-102.749028596452\\
-139.16	-133.106391795742\\
-185.547	-177.475507893968\\
-198.975	-190.319375593258\\
-231.934	-221.844623992193\\
-184.326	-176.307622694323\\
-168.457	-161.128941094677\\
-89.111	-85.2345765975163\\
-67.139	-64.2183819975159\\
-126.953	-121.430409296097\\
-163.574	-156.458356795032\\
-93.994	-89.9051608971613\\
-151.367	-144.782374295387\\
-203.857	-194.989003393968\\
-246.582	-235.855420392193\\
-153.809	-147.118144694677\\
-86.67	-82.8997626971612\\
-50.049	-47.8718151982257\\
-75.684	-72.3916653971611\\
-69.58	-66.553195897871\\
-75.684	-72.3916653971611\\
-42.725	-40.8664169982256\\
-61.035	-58.3799124982258\\
-45.166	-43.2012308985806\\
-58.594	-56.0450985978708\\
-48.828	-46.7039299985807\\
-62.256	-59.5477976978709\\
-101.318	-96.9105590971614\\
-91.553	-87.5703469968063\\
-129.395	-123.766179695387\\
-156.25	-149.452958595032\\
-155.029	-148.285073395387\\
-85.449	-81.7318774975162\\
-86.67	-82.8997626971612\\
-106.201	-101.581143396806\\
-70.801	-67.721081097516\\
-63.477	-60.7156828975159\\
-45.166	-43.2012308985806\\
-73.242	-70.055894997871\\
-69.58	-66.553195897871\\
-39.063	-37.3637178982256\\
-21.973	-21.0171510989353\\
-25.635	-24.5198501989354\\
-46.387	-44.3691160982257\\
-72.021	-68.888009798226\\
-75.684	-72.3916653971611\\
-45.166	-43.2012308985806\\
-57.373	-54.8772133982258\\
-84.229	-80.5649487968062\\
-109.863	-105.083842496807\\
-76.904	-73.5585940978711\\
-72.021	-68.888009798226\\
-84.229	-80.5649487968062\\
-101.318	-96.9105590971614\\
-92.773	-88.7372756975163\\
-109.863	-105.083842496807\\
-103.76	-99.2463294964515\\
-70.801	-67.721081097516\\
-56.152	-53.7093281985808\\
-72.021	-68.888009798226\\
-59.814	-57.2120272985808\\
-75.684	-72.3916653971611\\
-102.539	-98.0784442968064\\
-157.471	-150.620843794677\\
-109.863	-105.083842496807\\
-107.422	-102.749028596452\\
-166.016	-158.794127194322\\
-125.732	-120.262524096452\\
-75.684	-72.3916653971611\\
-54.932	-52.5423994978708\\
-43.945	-42.0333456989356\\
-46.387	-44.3691160982257\\
-39.063	-37.3637178982256\\
-40.283	-38.5306465989356\\
-84.229	-80.5649487968062\\
-26.855	-25.6867788996454\\
-37.842	-36.1958326985805\\
-53.711	-51.3745142982258\\
-72.021	-68.888009798226\\
-56.152	-53.7093281985808\\
-40.283	-38.5306465989356\\
-24.414	-23.3519649992903\\
-42.725	-40.8664169982256\\
-84.229	-80.5649487968062\\
-111.084	-106.251727696452\\
-87.891	-84.0676478968062\\
-73.242	-70.055894997871\\
-83.008	-79.3970635971612\\
-84.229	-80.5649487968062\\
-96.436	-92.2409312964514\\
-117.188	-112.090197195742\\
-139.16	-133.106391795742\\
-133.057	-127.268878795387\\
-114.746	-109.754426796452\\
-74.463	-71.223780197516\\
-58.594	-56.0450985978708\\
-61.035	-58.3799124982258\\
-91.553	-87.5703469968063\\
-57.373	-54.8772133982258\\
-50.049	-47.8718151982257\\
-92.773	-88.7372756975163\\
-113.525	-108.586541596807\\
-108.643	-103.916913796097\\
-124.512	-119.095595395742\\
-164.795	-157.626241994677\\
-85.449	-81.7318774975162\\
-145.264	-138.944861295032\\
-152.588	-145.950259495032\\
-140.381	-134.274276995387\\
-152.588	-145.950259495032\\
-148.926	-142.447560395032\\
-253.906	-242.860818592194\\
-235.596	-225.347323092193\\
-158.691	-151.787772495387\\
-189.209	-180.978206993968\\
-233.154	-223.011552692903\\
-223.389	-213.671340592548\\
-253.906	-242.860818592194\\
-227.051	-217.174039692548\\
-311.279	-297.738031990419\\
-234.375	-224.179437892548\\
-159.912	-152.955657695032\\
-98.877	-94.5757451968064\\
-100.098	-95.7436303964514\\
-79.346	-75.8943644971611\\
-64.697	-61.8826115982259\\
-100.098	-95.7436303964514\\
-142.822	-136.609090895742\\
-87.891	-84.0676478968062\\
-76.904	-73.5585940978711\\
-73.242	-70.055894997871\\
-48.828	-46.7039299985807\\
-63.477	-60.7156828975159\\
-34.18	-32.6931335985805\\
-40.283	-38.5306465989356\\
-35.4	-33.8600622992905\\
-48.828	-46.7039299985807\\
-37.842	-36.1958326985805\\
-31.738	-30.3573631992904\\
-17.09	-16.3465667992902\\
-18.311	-17.5144519989353\\
-40.283	-38.5306465989356\\
-28.076	-26.8546640992904\\
-29.297	-28.0225492989354\\
-65.918	-63.0504967978709\\
-93.994	-89.9051608971613\\
-68.359	-65.385310698226\\
-86.67	-82.8997626971612\\
-135.498	-129.603692695742\\
-70.801	-67.721081097516\\
-37.842	-36.1958326985805\\
-29.297	-28.0225492989354\\
-24.414	-23.3519649992903\\
-40.283	-38.5306465989356\\
-67.139	-64.2183819975159\\
-100.098	-95.7436303964514\\
-62.256	-59.5477976978709\\
-100.098	-95.7436303964514\\
-144.043	-137.776976095387\\
-145.264	-138.944861295032\\
-103.76	-99.2463294964515\\
-62.256	-59.5477976978709\\
-83.008	-79.3970635971612\\
-113.525	-108.586541596807\\
-148.926	-142.447560395032\\
-83.008	-79.3970635971612\\
-42.725	-40.8664169982256\\
-67.139	-64.2183819975159\\
-53.711	-51.3745142982258\\
-24.414	-23.3519649992903\\
-15.869	-15.1786815996452\\
-42.725	-40.8664169982256\\
-92.773	-88.7372756975163\\
-98.877	-94.5757451968064\\
-70.801	-67.721081097516\\
-43.945	-42.0333456989356\\
-47.607	-45.5360447989357\\
-20.752	-19.8492658992903\\
-28.076	-26.8546640992904\\
-29.297	-28.0225492989354\\
-46.387	-44.3691160982257\\
-69.58	-66.553195897871\\
-103.76	-99.2463294964515\\
-107.422	-102.749028596452\\
-120.85	-115.592896295742\\
-124.512	-119.095595395742\\
-120.85	-115.592896295742\\
-115.967	-110.922311996097\\
-101.318	-96.9105590971614\\
-123.291	-117.927710196097\\
-187.988	-179.810321794323\\
-169.678	-162.296826294322\\
-87.891	-84.0676478968062\\
-45.166	-43.2012308985806\\
-103.76	-99.2463294964515\\
-122.07	-116.759824996452\\
-114.746	-109.754426796452\\
-63.477	-60.7156828975159\\
-67.139	-64.2183819975159\\
-113.525	-108.586541596807\\
-175.781	-168.134339294677\\
-180.664	-172.804923594323\\
-203.857	-194.989003393968\\
-195.313	-186.816676493258\\
-139.16	-133.106391795742\\
-144.043	-137.776976095387\\
-59.814	-57.2120272985808\\
-61.035	-58.3799124982258\\
-76.904	-73.5585940978711\\
-98.877	-94.5757451968064\\
-104.98	-100.413258197161\\
-111.084	-106.251727696452\\
-70.801	-67.721081097516\\
-98.877	-94.5757451968064\\
-86.67	-82.8997626971612\\
-47.607	-45.5360447989357\\
-32.959	-31.5252483989355\\
-26.855	-25.6867788996454\\
-42.725	-40.8664169982256\\
-35.4	-33.8600622992905\\
-19.531	-18.6813806996453\\
-40.283	-38.5306465989356\\
-85.449	-81.7318774975162\\
-56.152	-53.7093281985808\\
-51.27	-49.0397003978707\\
-72.021	-68.888009798226\\
-119.629	-114.425011096097\\
-81.787	-78.2291783975162\\
-42.725	-40.8664169982256\\
-21.973	-21.0171510989353\\
-57.373	-54.8772133982258\\
-126.953	-121.430409296097\\
-161.133	-154.123542894677\\
-223.389	-213.671340592548\\
-266.113	-254.536801091839\\
-261.23	-249.866216792194\\
-168.457	-161.128941094677\\
-170.898	-163.463754995032\\
-222.168	-212.503455392903\\
-195.313	-186.816676493258\\
-179.443	-171.637038394678\\
-202.637	-193.822074693258\\
-220.947	-211.335570193258\\
-223.389	-213.671340592548\\
-301.514	-288.397819890064\\
-203.857	-194.989003393968\\
-109.863	-105.083842496807\\
-59.814	-57.2120272985808\\
-47.607	-45.5360447989357\\
-54.932	-52.5423994978708\\
-53.711	-51.3745142982258\\
-56.152	-53.7093281985808\\
-101.318	-96.9105590971614\\
-79.346	-75.8943644971611\\
-84.229	-80.5649487968062\\
-111.084	-106.251727696452\\
-63.477	-60.7156828975159\\
-70.801	-67.721081097516\\
-101.318	-96.9105590971614\\
-115.967	-110.922311996097\\
-64.697	-61.8826115982259\\
-46.387	-44.3691160982257\\
-58.594	-56.0450985978708\\
-54.932	-52.5423994978708\\
-129.395	-123.766179695387\\
-186.768	-178.643393093613\\
-173.34	-165.799525394322\\
-198.975	-190.319375593258\\
-225.83	-216.006154492903\\
-297.852	-284.895120790064\\
-257.568	-246.363517692194\\
-157.471	-150.620843794677\\
-100.098	-95.7436303964514\\
-54.932	-52.5423994978708\\
-61.035	-58.3799124982258\\
-67.139	-64.2183819975159\\
-86.67	-82.8997626971612\\
-62.256	-59.5477976978709\\
-56.152	-53.7093281985808\\
-63.477	-60.7156828975159\\
-78.125	-74.7264792975161\\
-89.111	-85.2345765975163\\
-111.084	-106.251727696452\\
-79.346	-75.8943644971611\\
-40.283	-38.5306465989356\\
-28.076	-26.8546640992904\\
-23.193	-22.1840797996453\\
-24.414	-23.3519649992903\\
-34.18	-32.6931335985805\\
-57.373	-54.8772133982258\\
-59.814	-57.2120272985808\\
-75.684	-72.3916653971611\\
-114.746	-109.754426796452\\
-81.787	-78.2291783975162\\
-150.146	-143.614489095742\\
-208.74	-199.659587693613\\
-162.354	-155.291428094322\\
-144.043	-137.776976095387\\
-170.898	-163.463754995032\\
-198.975	-190.319375593258\\
-131.836	-126.100993595742\\
-111.084	-106.251727696452\\
-72.021	-68.888009798226\\
-79.346	-75.8943644971611\\
-142.822	-136.609090895742\\
-247.803	-237.023305591838\\
-300.293	-287.229934690419\\
-239.258	-228.850022192193\\
-200.195	-191.486304293968\\
-145.264	-138.944861295032\\
-153.809	-147.118144694677\\
-208.74	-199.659587693613\\
-115.967	-110.922311996097\\
-98.877	-94.5757451968064\\
-75.684	-72.3916653971611\\
-148.926	-142.447560395032\\
-136.719	-130.771577895387\\
-76.904	-73.5585940978711\\
-72.021	-68.888009798226\\
-53.711	-51.3745142982258\\
-95.215	-91.0730460968063\\
-145.264	-138.944861295032\\
-148.926	-142.447560395032\\
-113.525	-108.586541596807\\
-83.008	-79.3970635971612\\
-93.994	-89.9051608971613\\
-74.463	-71.223780197516\\
-53.711	-51.3745142982258\\
-43.945	-42.0333456989356\\
-39.063	-37.3637178982256\\
-29.297	-28.0225492989354\\
-35.4	-33.8600622992905\\
-65.918	-63.0504967978709\\
-95.215	-91.0730460968063\\
-96.436	-92.2409312964514\\
-58.594	-56.0450985978708\\
-53.711	-51.3745142982258\\
-41.504	-39.6985317985806\\
-54.932	-52.5423994978708\\
-78.125	-74.7264792975161\\
-86.67	-82.8997626971612\\
-80.566	-77.0612931978711\\
-46.387	-44.3691160982257\\
-96.436	-92.2409312964514\\
-142.822	-136.609090895742\\
-101.318	-96.9105590971614\\
-40.283	-38.5306465989356\\
-50.049	-47.8718151982257\\
-76.904	-73.5585940978711\\
-73.242	-70.055894997871\\
-67.139	-64.2183819975159\\
-41.504	-39.6985317985806\\
-75.684	-72.3916653971611\\
-117.188	-112.090197195742\\
-70.801	-67.721081097516\\
-25.635	-24.5198501989354\\
-39.063	-37.3637178982256\\
-91.553	-87.5703469968063\\
-111.084	-106.251727696452\\
-65.918	-63.0504967978709\\
-59.814	-57.2120272985808\\
-113.525	-108.586541596807\\
-151.367	-144.782374295387\\
-131.836	-126.100993595742\\
-179.443	-171.637038394678\\
-219.727	-210.168641492548\\
-140.381	-134.274276995387\\
-111.084	-106.251727696452\\
-159.912	-152.955657695032\\
-109.863	-105.083842496807\\
-142.822	-136.609090895742\\
-174.561	-166.967410593967\\
-137.939	-131.938506596097\\
-69.58	-66.553195897871\\
-111.084	-106.251727696452\\
-202.637	-193.822074693258\\
-233.154	-223.011552692903\\
-169.678	-162.296826294322\\
-142.822	-136.609090895742\\
-191.65	-183.313020894323\\
-153.809	-147.118144694677\\
-96.436	-92.2409312964514\\
-85.449	-81.7318774975162\\
-107.422	-102.749028596452\\
-113.525	-108.586541596807\\
-109.863	-105.083842496807\\
-125.732	-120.262524096452\\
-87.891	-84.0676478968062\\
-90.332	-86.4024617971613\\
-130.615	-124.933108396097\\
-78.125	-74.7264792975161\\
-62.256	-59.5477976978709\\
-50.049	-47.8718151982257\\
-40.283	-38.5306465989356\\
-46.387	-44.3691160982257\\
-85.449	-81.7318774975162\\
-79.346	-75.8943644971611\\
-107.422	-102.749028596452\\
-139.16	-133.106391795742\\
-164.795	-157.626241994677\\
-123.291	-117.927710196097\\
-107.422	-102.749028596452\\
-41.504	-39.6985317985806\\
-67.139	-64.2183819975159\\
-129.395	-123.766179695387\\
-85.449	-81.7318774975162\\
-41.504	-39.6985317985806\\
-87.891	-84.0676478968062\\
-137.939	-131.938506596097\\
-142.822	-136.609090895742\\
-185.547	-177.475507893968\\
-245.361	-234.687535192548\\
-173.34	-165.799525394322\\
-147.705	-141.279675195387\\
-170.898	-163.463754995032\\
-113.525	-108.586541596807\\
-90.332	-86.4024617971613\\
-102.539	-98.0784442968064\\
-131.836	-126.100993595742\\
-141.602	-135.442162195032\\
-142.822	-136.609090895742\\
-79.346	-75.8943644971611\\
-70.801	-67.721081097516\\
-54.932	-52.5423994978708\\
-79.346	-75.8943644971611\\
-75.684	-72.3916653971611\\
-59.814	-57.2120272985808\\
-119.629	-114.425011096097\\
-131.836	-126.100993595742\\
-89.111	-85.2345765975163\\
-74.463	-71.223780197516\\
-108.643	-103.916913796097\\
-63.477	-60.7156828975159\\
-35.4	-33.8600622992905\\
-47.607	-45.5360447989357\\
-64.697	-61.8826115982259\\
-56.152	-53.7093281985808\\
-41.504	-39.6985317985806\\
-23.193	-22.1840797996453\\
-42.725	-40.8664169982256\\
-64.697	-61.8826115982259\\
-102.539	-98.0784442968064\\
-118.408	-113.257125896452\\
-152.588	-145.950259495032\\
-173.34	-165.799525394322\\
-91.553	-87.5703469968063\\
-69.58	-66.553195897871\\
-136.719	-130.771577895387\\
-211.182	-201.995358092903\\
-164.795	-157.626241994677\\
-151.367	-144.782374295387\\
-230.713	-220.676738792548\\
-185.547	-177.475507893968\\
-148.926	-142.447560395032\\
-112.305	-107.419612896097\\
-114.746	-109.754426796452\\
-155.029	-148.285073395387\\
-107.422	-102.749028596452\\
-90.332	-86.4024617971613\\
-156.25	-149.452958595032\\
-196.533	-187.983605193968\\
-205.078	-196.156888593613\\
-92.773	-88.7372756975163\\
-43.945	-42.0333456989356\\
-111.084	-106.251727696452\\
-87.891	-84.0676478968062\\
-37.842	-36.1958326985805\\
-29.297	-28.0225492989354\\
-53.711	-51.3745142982258\\
-81.787	-78.2291783975162\\
-135.498	-129.603692695742\\
-96.436	-92.2409312964514\\
-70.801	-67.721081097516\\
-42.725	-40.8664169982256\\
-28.076	-26.8546640992904\\
-39.063	-37.3637178982256\\
-35.4	-33.8600622992905\\
-45.166	-43.2012308985806\\
-79.346	-75.8943644971611\\
-72.021	-68.888009798226\\
-32.959	-31.5252483989355\\
-30.518	-29.1904344985804\\
-41.504	-39.6985317985806\\
-53.711	-51.3745142982258\\
-32.959	-31.5252483989355\\
-51.27	-49.0397003978707\\
-37.842	-36.1958326985805\\
-25.635	-24.5198501989354\\
-63.477	-60.7156828975159\\
-96.436	-92.2409312964514\\
-145.264	-138.944861295032\\
-190.43	-182.146092193613\\
-161.133	-154.123542894677\\
-80.566	-77.0612931978711\\
-57.373	-54.8772133982258\\
-58.594	-56.0450985978708\\
-32.959	-31.5252483989355\\
-30.518	-29.1904344985804\\
-64.697	-61.8826115982259\\
-73.242	-70.055894997871\\
-67.139	-64.2183819975159\\
-73.242	-70.055894997871\\
-89.111	-85.2345765975163\\
-93.994	-89.9051608971613\\
-97.656	-93.4078599971614\\
-122.07	-116.759824996452\\
-118.408	-113.257125896452\\
-170.898	-163.463754995032\\
-89.111	-85.2345765975163\\
-41.504	-39.6985317985806\\
-40.283	-38.5306465989356\\
-76.904	-73.5585940978711\\
-63.477	-60.7156828975159\\
-57.373	-54.8772133982258\\
-26.855	-25.6867788996454\\
-45.166	-43.2012308985806\\
-28.076	-26.8546640992904\\
-13.428	-12.8438676992902\\
-25.635	-24.5198501989354\\
-52.49	-50.2066290985807\\
-64.697	-61.8826115982259\\
-96.436	-92.2409312964514\\
-136.719	-130.771577895387\\
-98.877	-94.5757451968064\\
-43.945	-42.0333456989356\\
-80.566	-77.0612931978711\\
-90.332	-86.4024617971613\\
-42.725	-40.8664169982256\\
-57.373	-54.8772133982258\\
-112.305	-107.419612896097\\
-97.656	-93.4078599971614\\
-162.354	-155.291428094322\\
-189.209	-180.978206993968\\
-117.188	-112.090197195742\\
-92.773	-88.7372756975163\\
};
\addlegendentry{data2}

\end{axis}

\begin{axis}[%
width=4.927cm,
height=3.484cm,
at={(0cm,0cm)},
scale only axis,
xmin=-300,
xmax=0,
xlabel style={font=\color{white!15!black}},
xlabel={y(t-1)},
ymin=-279.541,
ymax=0,
ylabel style={font=\color{white!15!black}},
ylabel={y(t)},
axis background/.style={fill=white},
title={C8, R = 0.7799},
axis x line*=bottom,
axis y line*=left,
legend style={legend cell align=left, align=left, draw=white!15!black}
]
\addplot[only marks, mark=*, mark options={}, mark size=1.5000pt, color=mycolor1, fill=mycolor1] table[row sep=crcr]{%
x	y\\
-84.229	-96.436\\
-96.436	-122.07\\
-122.07	-100.098\\
-100.098	-93.994\\
-93.994	-130.615\\
-130.615	-125.732\\
-125.732	-153.809\\
-153.809	-115.967\\
-115.967	-61.035\\
-61.035	-74.463\\
-74.463	-73.242\\
-73.242	-86.67\\
-86.67	-73.242\\
-73.242	-34.18\\
-34.18	-20.752\\
-20.752	-23.193\\
-23.193	-58.594\\
-58.594	-100.098\\
-100.098	-109.863\\
-109.863	-114.746\\
-114.746	-83.008\\
-83.008	-52.49\\
-52.49	-104.98\\
-104.98	-81.787\\
-81.787	-59.814\\
-59.814	-97.656\\
-97.656	-83.008\\
-83.008	-72.021\\
-72.021	-118.408\\
-118.408	-107.422\\
-107.422	-83.008\\
-83.008	-119.629\\
-119.629	-123.291\\
-123.291	-95.215\\
-95.215	-80.566\\
-80.566	-62.256\\
-62.256	-75.684\\
-75.684	-95.215\\
-95.215	-87.891\\
-87.891	-91.553\\
-91.553	-96.436\\
-96.436	-102.539\\
-102.539	-141.602\\
-141.602	-128.174\\
-128.174	-85.449\\
-85.449	-79.346\\
-79.346	-47.607\\
-47.607	-48.828\\
-48.828	-68.359\\
-68.359	-52.49\\
-52.49	-57.373\\
-57.373	-75.684\\
-75.684	-89.111\\
-89.111	-81.787\\
-81.787	-128.174\\
-128.174	-98.877\\
-98.877	-57.373\\
-57.373	-45.166\\
-45.166	-62.256\\
-62.256	-72.021\\
-72.021	-40.283\\
-40.283	-42.725\\
-42.725	-56.152\\
-56.152	-46.387\\
-46.387	-40.283\\
-40.283	-52.49\\
-52.49	-62.256\\
-62.256	-92.773\\
-92.773	-90.332\\
-90.332	-107.422\\
-107.422	-98.877\\
-98.877	-115.967\\
-115.967	-91.553\\
-91.553	-91.553\\
-91.553	-79.346\\
-79.346	-75.684\\
-75.684	-76.904\\
-76.904	-133.057\\
-133.057	-189.209\\
-189.209	-194.092\\
-194.092	-189.209\\
-189.209	-119.629\\
-119.629	-170.898\\
-170.898	-213.623\\
-213.623	-219.727\\
-219.727	-169.678\\
-169.678	-219.727\\
-219.727	-279.541\\
-279.541	-197.754\\
-197.754	-161.133\\
-161.133	-109.863\\
-109.863	-85.449\\
-85.449	-73.242\\
-73.242	-91.553\\
-91.553	-62.256\\
-62.256	-46.387\\
-46.387	-42.725\\
-42.725	-54.932\\
-54.932	-79.346\\
-79.346	-74.463\\
-74.463	-75.684\\
-75.684	-100.098\\
-100.098	-103.76\\
-103.76	-141.602\\
-141.602	-114.746\\
-114.746	-142.822\\
-142.822	-104.98\\
-104.98	-90.332\\
-90.332	-103.76\\
-103.76	-76.904\\
-76.904	-69.58\\
-69.58	-84.229\\
-84.229	-86.67\\
-86.67	-59.814\\
-59.814	-64.697\\
-64.697	-48.828\\
-48.828	-53.711\\
-53.711	-57.373\\
-57.373	-41.504\\
-41.504	-52.49\\
-52.49	-65.918\\
-65.918	-101.318\\
-101.318	-104.98\\
-104.98	-114.746\\
-114.746	-68.359\\
-68.359	-93.994\\
-93.994	-150.146\\
-150.146	-114.746\\
-114.746	-125.732\\
-125.732	-128.174\\
-128.174	-80.566\\
-80.566	-53.711\\
-53.711	-75.684\\
-75.684	-85.449\\
-85.449	-137.939\\
-137.939	-172.119\\
-172.119	-170.898\\
-170.898	-123.291\\
-123.291	-130.615\\
-130.615	-115.967\\
-115.967	-113.525\\
-113.525	-111.084\\
-111.084	-76.904\\
-76.904	-68.359\\
-68.359	-61.035\\
-61.035	-57.373\\
-57.373	-62.256\\
-62.256	-91.553\\
-91.553	-140.381\\
-140.381	-111.084\\
-111.084	-79.346\\
-79.346	-64.697\\
-64.697	-62.256\\
-62.256	-40.283\\
-40.283	-29.297\\
-29.297	-36.621\\
-36.621	-63.477\\
-63.477	-51.27\\
-51.27	-52.49\\
-52.49	-58.594\\
-58.594	-54.932\\
-54.932	-37.842\\
-37.842	-28.076\\
-28.076	-23.193\\
-23.193	-39.063\\
-39.063	-83.008\\
-83.008	-117.188\\
-117.188	-130.615\\
-130.615	-86.67\\
-86.67	-58.594\\
-58.594	-45.166\\
-45.166	-32.959\\
-32.959	-67.139\\
-67.139	-65.918\\
-65.918	-91.553\\
-91.553	-117.188\\
-117.188	-186.768\\
-186.768	-216.064\\
-216.064	-177.002\\
-177.002	-168.457\\
-168.457	-109.863\\
-109.863	-122.07\\
-122.07	-131.836\\
-131.836	-146.484\\
-146.484	-108.643\\
-108.643	-100.098\\
-100.098	-98.877\\
-98.877	-89.111\\
-89.111	-106.201\\
-106.201	-78.125\\
-78.125	-130.615\\
-130.615	-159.912\\
-159.912	-113.525\\
-113.525	-70.801\\
-70.801	-73.242\\
-73.242	-123.291\\
-123.291	-153.809\\
-153.809	-189.209\\
-189.209	-202.637\\
-202.637	-202.637\\
-202.637	-153.809\\
-153.809	-137.939\\
-137.939	-158.691\\
-158.691	-187.988\\
-187.988	-109.863\\
-109.863	-72.021\\
-72.021	-101.318\\
-101.318	-86.67\\
-86.67	-53.711\\
-53.711	-74.463\\
-74.463	-84.229\\
-84.229	-67.139\\
-67.139	-51.27\\
-51.27	-70.801\\
-70.801	-58.594\\
-58.594	-79.346\\
-79.346	-107.422\\
-107.422	-100.098\\
-100.098	-69.58\\
-69.58	-70.801\\
-70.801	-107.422\\
-107.422	-177.002\\
-177.002	-128.174\\
-128.174	-79.346\\
-79.346	-61.035\\
-61.035	-70.801\\
-70.801	-64.697\\
-64.697	-48.828\\
-48.828	-53.711\\
-53.711	-65.918\\
-65.918	-83.008\\
-83.008	-91.553\\
-91.553	-63.477\\
-63.477	-104.98\\
-104.98	-124.512\\
-124.512	-118.408\\
-118.408	-91.553\\
-91.553	-100.098\\
-100.098	-135.498\\
-135.498	-87.891\\
-87.891	-42.725\\
-42.725	-57.373\\
-57.373	-62.256\\
-62.256	-81.787\\
-81.787	-72.021\\
-72.021	-52.49\\
-52.49	-79.346\\
-79.346	-80.566\\
-80.566	-57.373\\
-57.373	-70.801\\
-70.801	-63.477\\
-63.477	-56.152\\
-56.152	-67.139\\
-67.139	-41.504\\
-41.504	-48.828\\
-48.828	-36.621\\
-36.621	-40.283\\
-40.283	-75.684\\
-75.684	-84.229\\
-84.229	-113.525\\
-113.525	-146.484\\
-146.484	-98.877\\
-98.877	-58.594\\
-58.594	-46.387\\
-46.387	-43.945\\
-43.945	-63.477\\
-63.477	-42.725\\
-42.725	-47.607\\
-47.607	-61.035\\
-61.035	-102.539\\
-102.539	-93.994\\
-93.994	-134.277\\
-134.277	-89.111\\
-89.111	-81.787\\
-81.787	-46.387\\
-46.387	-45.166\\
-45.166	-28.076\\
-28.076	-35.4\\
-35.4	-37.842\\
-37.842	-67.139\\
-67.139	-70.801\\
-70.801	-79.346\\
-79.346	-85.449\\
-85.449	-125.732\\
-125.732	-112.305\\
-112.305	-84.229\\
-84.229	-102.539\\
-102.539	-109.863\\
-109.863	-144.043\\
-144.043	-115.967\\
-115.967	-75.684\\
-75.684	-40.283\\
-40.283	-29.297\\
-29.297	-29.297\\
-29.297	-36.621\\
-36.621	-39.063\\
-39.063	-37.842\\
-37.842	-50.049\\
-50.049	-74.463\\
-74.463	-68.359\\
-68.359	-70.801\\
-70.801	-83.008\\
-83.008	-51.27\\
-51.27	-30.518\\
-30.518	-58.594\\
-58.594	-90.332\\
-90.332	-87.891\\
-87.891	-97.656\\
-97.656	-83.008\\
-83.008	-61.035\\
-61.035	-85.449\\
-85.449	-118.408\\
-118.408	-118.408\\
-118.408	-84.229\\
-84.229	-136.719\\
-136.719	-100.098\\
-100.098	-101.318\\
-101.318	-117.188\\
-117.188	-115.967\\
-115.967	-86.67\\
-86.67	-76.904\\
-76.904	-128.174\\
-128.174	-178.223\\
-178.223	-139.16\\
-139.16	-79.346\\
-79.346	-76.904\\
-76.904	-79.346\\
-79.346	-98.877\\
-98.877	-114.746\\
-114.746	-85.449\\
-85.449	-90.332\\
-90.332	-120.85\\
-120.85	-131.836\\
-131.836	-87.891\\
-87.891	-68.359\\
-68.359	-91.553\\
-91.553	-156.25\\
-156.25	-137.939\\
-137.939	-140.381\\
-140.381	-84.229\\
-84.229	-76.904\\
-76.904	-72.021\\
-72.021	-47.607\\
-47.607	-30.518\\
-30.518	-26.855\\
-26.855	-23.193\\
-23.193	-54.932\\
-54.932	-70.801\\
-70.801	-87.891\\
-87.891	-65.918\\
-65.918	-73.242\\
-73.242	-83.008\\
-83.008	-53.711\\
-53.711	-56.152\\
-56.152	-53.711\\
-53.711	-40.283\\
-40.283	-51.27\\
-51.27	-84.229\\
-84.229	-106.201\\
-106.201	-63.477\\
-63.477	-48.828\\
-48.828	-40.283\\
-40.283	-50.049\\
-50.049	-35.4\\
-35.4	-76.904\\
-76.904	-130.615\\
-130.615	-104.98\\
-104.98	-139.16\\
-139.16	-162.354\\
-162.354	-164.795\\
-164.795	-124.512\\
-124.512	-90.332\\
-90.332	-89.111\\
-89.111	-89.111\\
-89.111	-106.201\\
-106.201	-125.732\\
-125.732	-125.732\\
-125.732	-168.457\\
-168.457	-183.105\\
-183.105	-219.727\\
-219.727	-147.705\\
-147.705	-166.016\\
-166.016	-184.326\\
-184.326	-140.381\\
-140.381	-162.354\\
-162.354	-139.16\\
-139.16	-80.566\\
-80.566	-52.49\\
-52.49	-56.152\\
-56.152	-81.787\\
-81.787	-65.918\\
-65.918	-43.945\\
-43.945	-62.256\\
-62.256	-47.607\\
-47.607	-36.621\\
-36.621	-46.387\\
-46.387	-52.49\\
-52.49	-36.621\\
-36.621	-70.801\\
-70.801	-92.773\\
-92.773	-68.359\\
-68.359	-114.746\\
-114.746	-164.795\\
-164.795	-150.146\\
-150.146	-196.533\\
-196.533	-131.836\\
-131.836	-144.043\\
-144.043	-96.436\\
-96.436	-63.477\\
-63.477	-76.904\\
-76.904	-59.814\\
-59.814	-45.166\\
-45.166	-64.697\\
-64.697	-36.621\\
-36.621	-28.076\\
-28.076	-34.18\\
-34.18	-70.801\\
-70.801	-86.67\\
-86.67	-107.422\\
-107.422	-103.76\\
-103.76	-100.098\\
-100.098	-114.746\\
-114.746	-93.994\\
-93.994	-76.904\\
-76.904	-76.904\\
-76.904	-62.256\\
-62.256	-97.656\\
-97.656	-68.359\\
-68.359	-56.152\\
-56.152	-81.787\\
-81.787	-113.525\\
-113.525	-86.67\\
-86.67	-65.918\\
-65.918	-56.152\\
-56.152	-65.918\\
-65.918	-45.166\\
-45.166	-45.166\\
-45.166	-59.814\\
-59.814	-75.684\\
-75.684	-84.229\\
-84.229	-65.918\\
-65.918	-86.67\\
-86.67	-79.346\\
-79.346	-47.607\\
-47.607	-50.049\\
-50.049	-95.215\\
-95.215	-131.836\\
-131.836	-131.836\\
-131.836	-107.422\\
-107.422	-103.76\\
-103.76	-79.346\\
-79.346	-76.904\\
-76.904	-61.035\\
-61.035	-89.111\\
-89.111	-75.684\\
-75.684	-50.049\\
-50.049	-74.463\\
-74.463	-85.449\\
-85.449	-101.318\\
-101.318	-109.863\\
-109.863	-109.863\\
-109.863	-148.926\\
-148.926	-181.885\\
-181.885	-205.078\\
-205.078	-129.395\\
-129.395	-73.242\\
-73.242	-47.607\\
-47.607	-34.18\\
-34.18	-54.932\\
-54.932	-46.387\\
-46.387	-80.566\\
-80.566	-72.021\\
-72.021	-64.697\\
-64.697	-85.449\\
-85.449	-98.877\\
-98.877	-117.188\\
-117.188	-123.291\\
-123.291	-175.781\\
-175.781	-150.146\\
-150.146	-117.188\\
-117.188	-80.566\\
-80.566	-80.566\\
-80.566	-83.008\\
-83.008	-106.201\\
-106.201	-75.684\\
-75.684	-79.346\\
-79.346	-74.463\\
-74.463	-42.725\\
-42.725	-74.463\\
-74.463	-107.422\\
-107.422	-73.242\\
-73.242	-80.566\\
-80.566	-147.705\\
-147.705	-109.863\\
-109.863	-106.201\\
-106.201	-147.705\\
-147.705	-140.381\\
-140.381	-87.891\\
-87.891	-56.152\\
-56.152	-58.594\\
-58.594	-97.656\\
-97.656	-151.367\\
-151.367	-162.354\\
-162.354	-167.236\\
-167.236	-129.395\\
-129.395	-83.008\\
-83.008	-72.021\\
-72.021	-53.711\\
-53.711	-50.049\\
-50.049	-42.725\\
-42.725	-61.035\\
-61.035	-72.021\\
-72.021	-41.504\\
-41.504	-65.918\\
-65.918	-89.111\\
-89.111	-101.318\\
-101.318	-128.174\\
-128.174	-161.133\\
-161.133	-192.871\\
-192.871	-136.719\\
-136.719	-118.408\\
-118.408	-124.512\\
-124.512	-102.539\\
-102.539	-73.242\\
-73.242	-89.111\\
-89.111	-136.719\\
-136.719	-162.354\\
-162.354	-93.994\\
-93.994	-54.932\\
-54.932	-37.842\\
-37.842	-23.193\\
-23.193	-32.959\\
-32.959	-45.166\\
-45.166	-52.49\\
-52.49	-73.242\\
-73.242	-58.594\\
-58.594	-45.166\\
-45.166	-45.166\\
-45.166	-43.945\\
-43.945	-40.283\\
-40.283	-47.607\\
-47.607	-70.801\\
-70.801	-104.98\\
-104.98	-111.084\\
-111.084	-107.422\\
-107.422	-123.291\\
-123.291	-152.588\\
-152.588	-155.029\\
-155.029	-183.105\\
-183.105	-167.236\\
-167.236	-124.512\\
-124.512	-85.449\\
-85.449	-83.008\\
-83.008	-141.602\\
-141.602	-158.691\\
-158.691	-115.967\\
-115.967	-76.904\\
-76.904	-39.063\\
-39.063	-30.518\\
-30.518	-29.297\\
-29.297	-19.531\\
-19.531	-45.166\\
-45.166	-57.373\\
-57.373	-62.256\\
-62.256	-53.711\\
-53.711	-34.18\\
-34.18	-26.855\\
-26.855	-36.621\\
-36.621	-41.504\\
-41.504	-43.945\\
-43.945	-35.4\\
-35.4	-28.076\\
-28.076	-50.049\\
-50.049	-112.305\\
-112.305	-128.174\\
-128.174	-145.264\\
-145.264	-101.318\\
-101.318	-133.057\\
-133.057	-195.313\\
-195.313	-174.561\\
-174.561	-184.326\\
-184.326	-135.498\\
-135.498	-137.939\\
-137.939	-145.264\\
-145.264	-95.215\\
-95.215	-90.332\\
-90.332	-56.152\\
-56.152	-54.932\\
-54.932	-102.539\\
-102.539	-68.359\\
-68.359	-79.346\\
-79.346	-84.229\\
-84.229	-79.346\\
-79.346	-79.346\\
-79.346	-84.229\\
-84.229	-93.994\\
-93.994	-102.539\\
-102.539	-89.111\\
-89.111	-67.139\\
-67.139	-84.229\\
-84.229	-39.063\\
-39.063	-20.752\\
-20.752	-36.621\\
-36.621	-51.27\\
-51.27	-26.855\\
-26.855	-13.428\\
-13.428	-31.738\\
-31.738	-50.049\\
-50.049	-58.594\\
-58.594	-72.021\\
-72.021	-62.256\\
-62.256	-91.553\\
-91.553	-80.566\\
-80.566	-57.373\\
-57.373	-86.67\\
-86.67	-48.828\\
-48.828	-41.504\\
-41.504	-28.076\\
-28.076	-21.973\\
-21.973	-37.842\\
-37.842	-28.076\\
-28.076	-50.049\\
-50.049	-75.684\\
-75.684	-73.242\\
-73.242	-54.932\\
-54.932	-61.035\\
-61.035	-45.166\\
-45.166	-65.918\\
-65.918	-47.607\\
-47.607	-46.387\\
-46.387	-42.725\\
-42.725	-32.959\\
-32.959	-42.725\\
-42.725	-37.842\\
-37.842	-64.697\\
-64.697	-63.477\\
-63.477	-50.049\\
-50.049	-46.387\\
-46.387	-36.621\\
-36.621	-43.945\\
-43.945	-93.994\\
-93.994	-120.85\\
-120.85	-81.787\\
-81.787	-80.566\\
-80.566	-54.932\\
-54.932	-93.994\\
-93.994	-81.787\\
-81.787	-54.932\\
-54.932	-70.801\\
-70.801	-52.49\\
-52.49	-75.684\\
-75.684	-42.725\\
-42.725	-50.049\\
-50.049	-57.373\\
-57.373	-74.463\\
-74.463	-58.594\\
-58.594	-61.035\\
-61.035	-61.035\\
-61.035	-95.215\\
-95.215	-79.346\\
-79.346	-125.732\\
-125.732	-157.471\\
-157.471	-125.732\\
-125.732	-81.787\\
-81.787	-62.256\\
-62.256	-89.111\\
-89.111	-112.305\\
-112.305	-87.891\\
-87.891	-107.422\\
-107.422	-64.697\\
-64.697	-34.18\\
-34.18	-37.842\\
-37.842	-84.229\\
-84.229	-70.801\\
-70.801	-69.58\\
-69.58	-104.98\\
-104.98	-97.656\\
-97.656	-87.891\\
-87.891	-62.256\\
-62.256	-73.242\\
-73.242	-100.098\\
-100.098	-81.787\\
-81.787	-87.891\\
-87.891	-78.125\\
-78.125	-58.594\\
-58.594	-67.139\\
-67.139	-48.828\\
-48.828	-56.152\\
-56.152	-95.215\\
-95.215	-109.863\\
-109.863	-115.967\\
-115.967	-102.539\\
-102.539	-73.242\\
-73.242	-51.27\\
-51.27	-62.256\\
-62.256	-70.801\\
-70.801	-91.553\\
-91.553	-111.084\\
-111.084	-131.836\\
-131.836	-107.422\\
-107.422	-62.256\\
-62.256	-109.863\\
-109.863	-151.367\\
-151.367	-107.422\\
-107.422	-108.643\\
-108.643	-124.512\\
-124.512	-93.994\\
-93.994	-79.346\\
-79.346	-89.111\\
-89.111	-78.125\\
-78.125	-91.553\\
-91.553	-114.746\\
-114.746	-86.67\\
-86.67	-102.539\\
-102.539	-92.773\\
-92.773	-96.436\\
-96.436	-76.904\\
-76.904	-100.098\\
-100.098	-69.58\\
-69.58	-76.904\\
-76.904	-84.229\\
-84.229	-80.566\\
-80.566	-41.504\\
-41.504	-28.076\\
-28.076	-21.973\\
-21.973	-37.842\\
-37.842	-84.229\\
-84.229	-108.643\\
-108.643	-111.084\\
-111.084	-74.463\\
-74.463	-86.67\\
-86.67	-72.021\\
-72.021	-65.918\\
-65.918	-42.725\\
-42.725	-61.035\\
-61.035	-58.594\\
-58.594	-57.373\\
-57.373	-67.139\\
-67.139	-57.373\\
-57.373	-53.711\\
-53.711	-37.842\\
-37.842	-50.049\\
-50.049	-91.553\\
-91.553	-70.801\\
-70.801	-86.67\\
-86.67	-75.684\\
-75.684	-101.318\\
-101.318	-92.773\\
-92.773	-129.395\\
-129.395	-91.553\\
-91.553	-106.201\\
-106.201	-103.76\\
-103.76	-54.932\\
-54.932	-96.436\\
-96.436	-67.139\\
-67.139	-84.229\\
-84.229	-90.332\\
-90.332	-115.967\\
-115.967	-85.449\\
-85.449	-111.084\\
-111.084	-136.719\\
-136.719	-111.084\\
-111.084	-95.215\\
-95.215	-75.684\\
-75.684	-91.553\\
-91.553	-80.566\\
-80.566	-69.58\\
-69.58	-56.152\\
-56.152	-70.801\\
-70.801	-58.594\\
-58.594	-119.629\\
-119.629	-151.367\\
-151.367	-130.615\\
-130.615	-126.953\\
-126.953	-124.512\\
-124.512	-69.58\\
-69.58	-63.477\\
-63.477	-50.049\\
-50.049	-42.725\\
-42.725	-48.828\\
-48.828	-80.566\\
-80.566	-100.098\\
-100.098	-114.746\\
-114.746	-96.436\\
-96.436	-67.139\\
-67.139	-117.188\\
-117.188	-137.939\\
-137.939	-152.588\\
-152.588	-118.408\\
-118.408	-187.988\\
-187.988	-133.057\\
-133.057	-78.125\\
-78.125	-57.373\\
-57.373	-48.828\\
-48.828	-87.891\\
-87.891	-111.084\\
-111.084	-147.705\\
-147.705	-109.863\\
-109.863	-87.891\\
-87.891	-47.607\\
-47.607	-53.711\\
-53.711	-56.152\\
-56.152	-63.477\\
-63.477	-40.283\\
-40.283	-52.49\\
-52.49	-40.283\\
-40.283	-26.855\\
-26.855	-58.594\\
-58.594	-63.477\\
-63.477	-80.566\\
-80.566	-72.021\\
-72.021	-76.904\\
-76.904	-76.904\\
-76.904	-98.877\\
-98.877	-67.139\\
-67.139	-101.318\\
-101.318	-117.188\\
-117.188	-89.111\\
-89.111	-93.994\\
-93.994	-80.566\\
-80.566	-93.994\\
-93.994	-133.057\\
-133.057	-100.098\\
-100.098	-79.346\\
-79.346	-90.332\\
-90.332	-81.787\\
-81.787	-56.152\\
-56.152	-85.449\\
-85.449	-123.291\\
-123.291	-117.188\\
-117.188	-130.615\\
-130.615	-192.871\\
-192.871	-125.732\\
-125.732	-107.422\\
-107.422	-83.008\\
-83.008	-106.201\\
-106.201	-152.588\\
-152.588	-101.318\\
-101.318	-90.332\\
-90.332	-124.512\\
-124.512	-79.346\\
-79.346	-47.607\\
-47.607	-62.256\\
-62.256	-46.387\\
-46.387	-39.063\\
-39.063	-42.725\\
-42.725	-37.842\\
-37.842	-37.842\\
-37.842	-47.607\\
-47.607	-65.918\\
-65.918	-78.125\\
-78.125	-102.539\\
-102.539	-96.436\\
-96.436	-112.305\\
-112.305	-130.615\\
-130.615	-119.629\\
-119.629	-86.67\\
-86.67	-92.773\\
-92.773	-83.008\\
-83.008	-91.553\\
-91.553	-126.953\\
-126.953	-93.994\\
-93.994	-106.201\\
-106.201	-91.553\\
-91.553	-87.891\\
-87.891	-140.381\\
-140.381	-113.525\\
-113.525	-115.967\\
-115.967	-103.76\\
-103.76	-64.697\\
-64.697	-42.725\\
-42.725	-34.18\\
-34.18	-61.035\\
-61.035	-54.932\\
-54.932	-74.463\\
-74.463	-56.152\\
-56.152	-79.346\\
-79.346	-124.512\\
-124.512	-155.029\\
-155.029	-122.07\\
-122.07	-128.174\\
-128.174	-113.525\\
-113.525	-83.008\\
-83.008	-87.891\\
-87.891	-98.877\\
-98.877	-84.229\\
-84.229	-101.318\\
-101.318	-126.953\\
-126.953	-178.223\\
-178.223	-156.25\\
-156.25	-147.705\\
-147.705	-164.795\\
-164.795	-109.863\\
-109.863	-120.85\\
-120.85	-93.994\\
-93.994	-95.215\\
-95.215	-126.953\\
-126.953	-144.043\\
-144.043	-56.152\\
-56.152	-74.463\\
-74.463	-57.373\\
-57.373	-83.008\\
-83.008	-84.229\\
-84.229	-51.27\\
-51.27	-68.359\\
-68.359	-119.629\\
-119.629	-196.533\\
-196.533	-190.43\\
-190.43	-174.561\\
-174.561	-137.939\\
-137.939	-161.133\\
-161.133	-196.533\\
-196.533	-144.043\\
-144.043	-148.926\\
-148.926	-153.809\\
-153.809	-104.98\\
-104.98	-91.553\\
-91.553	-118.408\\
-118.408	-79.346\\
-79.346	-59.814\\
-59.814	-80.566\\
-80.566	-59.814\\
-59.814	-73.242\\
-73.242	-67.139\\
-67.139	-84.229\\
-84.229	-111.084\\
-111.084	-85.449\\
-85.449	-64.697\\
-64.697	-76.904\\
-76.904	-89.111\\
-89.111	-107.422\\
-107.422	-89.111\\
-89.111	-74.463\\
-74.463	-115.967\\
-115.967	-108.643\\
-108.643	-75.684\\
-75.684	-126.953\\
-126.953	-113.525\\
-113.525	-83.008\\
-83.008	-57.373\\
-57.373	-42.725\\
-42.725	-31.738\\
-31.738	-69.58\\
-69.58	-63.477\\
-63.477	-47.607\\
-47.607	-34.18\\
-34.18	-42.725\\
-42.725	-70.801\\
-70.801	-81.787\\
-81.787	-104.98\\
-104.98	-122.07\\
-122.07	-117.188\\
-117.188	-122.07\\
-122.07	-72.021\\
-72.021	-34.18\\
-34.18	-50.049\\
-50.049	-41.504\\
-41.504	-53.711\\
-53.711	-86.67\\
-86.67	-95.215\\
-95.215	-141.602\\
-141.602	-92.773\\
-92.773	-87.891\\
-87.891	-128.174\\
-128.174	-93.994\\
-93.994	-65.918\\
-65.918	-50.049\\
-50.049	-65.918\\
-65.918	-87.891\\
-87.891	-129.395\\
-129.395	-86.67\\
-86.67	-74.463\\
-74.463	-72.021\\
-72.021	-86.67\\
-86.67	-113.525\\
-113.525	-179.443\\
-179.443	-151.367\\
-151.367	-151.367\\
-151.367	-97.656\\
-97.656	-64.697\\
-64.697	-76.904\\
-76.904	-36.621\\
-36.621	-54.932\\
-54.932	-51.27\\
-51.27	-54.932\\
-54.932	-93.994\\
-93.994	-128.174\\
-128.174	-146.484\\
-146.484	-189.209\\
-189.209	-161.133\\
-161.133	-86.67\\
-86.67	-81.787\\
-81.787	-68.359\\
-68.359	-91.553\\
-91.553	-120.85\\
-120.85	-161.133\\
-161.133	-113.525\\
-113.525	-80.566\\
-80.566	-92.773\\
-92.773	-124.512\\
-124.512	-87.891\\
-87.891	-63.477\\
-63.477	-51.27\\
-51.27	-37.842\\
-37.842	-34.18\\
-34.18	-29.297\\
-29.297	-47.607\\
-47.607	-64.697\\
-64.697	-73.242\\
-73.242	-56.152\\
-56.152	-52.49\\
-52.49	-26.855\\
-26.855	-15.869\\
-15.869	-23.193\\
-23.193	-41.504\\
-41.504	-62.256\\
-62.256	-76.904\\
-76.904	-86.67\\
-86.67	-101.318\\
-101.318	-131.836\\
-131.836	-181.885\\
-181.885	-179.443\\
-179.443	-194.092\\
-194.092	-169.678\\
-169.678	-93.994\\
-93.994	-86.67\\
-86.67	-85.449\\
-85.449	-113.525\\
-113.525	-96.436\\
-96.436	-70.801\\
-70.801	-56.152\\
-56.152	-46.387\\
-46.387	-76.904\\
-76.904	-76.904\\
-76.904	-42.725\\
-42.725	-32.959\\
-32.959	-50.049\\
-50.049	-67.139\\
-67.139	-51.27\\
-51.27	-73.242\\
-73.242	-54.932\\
-54.932	-81.787\\
-81.787	-122.07\\
-122.07	-96.436\\
-96.436	-128.174\\
-128.174	-177.002\\
-177.002	-189.209\\
-189.209	-148.926\\
-148.926	-95.215\\
-95.215	-73.242\\
-73.242	-45.166\\
-45.166	-68.359\\
-68.359	-47.607\\
-47.607	-85.449\\
-85.449	-79.346\\
-79.346	-62.256\\
-62.256	-81.787\\
-81.787	-47.607\\
-47.607	-69.58\\
-69.58	-54.932\\
-54.932	-34.18\\
-34.18	-39.063\\
-39.063	-32.959\\
-32.959	-37.842\\
-37.842	-32.959\\
-32.959	-24.414\\
-24.414	-31.738\\
-31.738	-45.166\\
-45.166	-43.945\\
-43.945	-43.945\\
-43.945	-68.359\\
-68.359	-90.332\\
-90.332	-72.021\\
-72.021	-67.139\\
-67.139	-104.98\\
-104.98	-125.732\\
-125.732	-100.098\\
-100.098	-104.98\\
-104.98	-51.27\\
-51.27	-32.959\\
-32.959	-62.256\\
-62.256	-91.553\\
-91.553	-96.436\\
-96.436	-135.498\\
-135.498	-120.85\\
-120.85	-86.67\\
-86.67	-61.035\\
-61.035	-64.697\\
-64.697	-73.242\\
-73.242	-56.152\\
-56.152	-35.4\\
-35.4	-45.166\\
-45.166	-37.842\\
-37.842	-31.738\\
-31.738	-25.635\\
-25.635	-45.166\\
-45.166	-62.256\\
-62.256	-68.359\\
-68.359	-50.049\\
-50.049	-81.787\\
-81.787	-115.967\\
-115.967	-74.463\\
-74.463	-100.098\\
-100.098	-112.305\\
-112.305	-83.008\\
-83.008	-48.828\\
-48.828	-62.256\\
-62.256	-74.463\\
-74.463	-45.166\\
-45.166	-48.828\\
-48.828	-41.504\\
-41.504	-24.414\\
-24.414	-24.414\\
-24.414	-40.283\\
-40.283	-54.932\\
-54.932	-48.828\\
-48.828	-59.814\\
-59.814	-69.58\\
-69.58	-50.049\\
-50.049	-30.518\\
-30.518	-24.414\\
-24.414	-31.738\\
-31.738	-36.621\\
-36.621	-43.945\\
-43.945	-86.67\\
-86.67	-89.111\\
-89.111	-76.904\\
-76.904	-74.463\\
-74.463	-101.318\\
-101.318	-128.174\\
-128.174	-137.939\\
-137.939	-140.381\\
-140.381	-103.76\\
-103.76	-108.643\\
-108.643	-111.084\\
-111.084	-112.305\\
-112.305	-126.953\\
-126.953	-169.678\\
-169.678	-144.043\\
-144.043	-131.836\\
-131.836	-104.98\\
-104.98	-101.318\\
-101.318	-96.436\\
-96.436	-114.746\\
-114.746	-72.021\\
-72.021	-83.008\\
-83.008	-102.539\\
-102.539	-119.629\\
-119.629	-91.553\\
-91.553	-106.201\\
-106.201	-85.449\\
-85.449	-75.684\\
-75.684	-50.049\\
-50.049	-74.463\\
-74.463	-122.07\\
-122.07	-96.436\\
-96.436	-56.152\\
-56.152	-69.58\\
-69.58	-41.504\\
-41.504	-35.4\\
-35.4	-50.049\\
-50.049	-31.738\\
-31.738	-35.4\\
-35.4	-42.725\\
-42.725	-28.076\\
-28.076	-43.945\\
-43.945	-40.283\\
-40.283	-24.414\\
-24.414	-39.063\\
-39.063	-46.387\\
-46.387	-53.711\\
-53.711	-53.711\\
-53.711	-53.711\\
-53.711	-74.463\\
-74.463	-64.697\\
-64.697	-72.021\\
-72.021	-124.512\\
-124.512	-89.111\\
-89.111	-76.904\\
-76.904	-85.449\\
-85.449	-61.035\\
-61.035	-37.842\\
-37.842	-65.918\\
-65.918	-57.373\\
-57.373	-32.959\\
-32.959	-58.594\\
-58.594	-54.932\\
-54.932	-67.139\\
-67.139	-45.166\\
-45.166	-28.076\\
-28.076	-23.193\\
-23.193	-26.855\\
-26.855	-48.828\\
-48.828	-48.828\\
-48.828	-59.814\\
-59.814	-76.904\\
-76.904	-102.539\\
-102.539	-76.904\\
-76.904	-102.539\\
-102.539	-122.07\\
-122.07	-104.98\\
-104.98	-98.877\\
-98.877	-162.354\\
-162.354	-118.408\\
-118.408	-142.822\\
-142.822	-123.291\\
-123.291	-145.264\\
-145.264	-162.354\\
-162.354	-96.436\\
-96.436	-58.594\\
-58.594	-48.828\\
-48.828	-74.463\\
-74.463	-93.994\\
-93.994	-95.215\\
-95.215	-65.918\\
-65.918	-36.621\\
-36.621	-46.387\\
-46.387	-92.773\\
-92.773	-125.732\\
-125.732	-124.512\\
-124.512	-74.463\\
-74.463	-57.373\\
-57.373	-76.904\\
-76.904	-122.07\\
-122.07	-136.719\\
-136.719	-139.16\\
-139.16	-97.656\\
-97.656	-137.939\\
-137.939	-152.588\\
-152.588	-142.822\\
-142.822	-197.754\\
-197.754	-169.678\\
-169.678	-141.602\\
-141.602	-164.795\\
-164.795	-140.381\\
-140.381	-78.125\\
-78.125	-72.021\\
-72.021	-89.111\\
-89.111	-108.643\\
-108.643	-111.084\\
-111.084	-81.787\\
-81.787	-102.539\\
-102.539	-91.553\\
-91.553	-65.918\\
-65.918	-89.111\\
-89.111	-84.229\\
-84.229	-124.512\\
-124.512	-128.174\\
-128.174	-91.553\\
-91.553	-70.801\\
-70.801	-42.725\\
-42.725	-64.697\\
-64.697	-54.932\\
-54.932	-46.387\\
-46.387	-39.063\\
-39.063	-65.918\\
-65.918	-86.67\\
-86.67	-95.215\\
-95.215	-95.215\\
-95.215	-57.373\\
-57.373	-42.725\\
-42.725	-29.297\\
-29.297	-37.842\\
-37.842	-63.477\\
-63.477	-65.918\\
-65.918	-64.697\\
-64.697	-102.539\\
-102.539	-122.07\\
-122.07	-134.277\\
-134.277	-135.498\\
-135.498	-157.471\\
-157.471	-97.656\\
-97.656	-80.566\\
-80.566	-61.035\\
-61.035	-67.139\\
-67.139	-90.332\\
-90.332	-75.684\\
-75.684	-51.27\\
-51.27	-46.387\\
-46.387	-83.008\\
-83.008	-102.539\\
-102.539	-90.332\\
-90.332	-147.705\\
-147.705	-161.133\\
-161.133	-112.305\\
-112.305	-79.346\\
-79.346	-54.932\\
-54.932	-50.049\\
-50.049	-90.332\\
-90.332	-76.904\\
-76.904	-86.67\\
-86.67	-98.877\\
-98.877	-107.422\\
-107.422	-103.76\\
-103.76	-117.188\\
-117.188	-81.787\\
-81.787	-89.111\\
-89.111	-65.918\\
-65.918	-58.594\\
-58.594	-90.332\\
-90.332	-125.732\\
-125.732	-135.498\\
-135.498	-80.566\\
-80.566	-155.029\\
-155.029	-145.264\\
-145.264	-96.436\\
-96.436	-56.152\\
-56.152	-83.008\\
-83.008	-75.684\\
-75.684	-70.801\\
-70.801	-85.449\\
-85.449	-58.594\\
-58.594	-79.346\\
-79.346	-145.264\\
-145.264	-151.367\\
-151.367	-98.877\\
-98.877	-108.643\\
-108.643	-122.07\\
-122.07	-123.291\\
-123.291	-90.332\\
-90.332	-87.891\\
-87.891	-61.035\\
-61.035	-89.111\\
-89.111	-95.215\\
-95.215	-146.484\\
-146.484	-166.016\\
-166.016	-228.271\\
-228.271	-163.574\\
-163.574	-131.836\\
-131.836	-103.76\\
-103.76	-109.863\\
-109.863	-85.449\\
-85.449	-59.814\\
-59.814	-57.373\\
-57.373	-59.814\\
-59.814	-46.387\\
-46.387	-32.959\\
-32.959	-51.27\\
-51.27	-67.139\\
-67.139	-47.607\\
-47.607	-34.18\\
-34.18	-32.959\\
-32.959	-76.904\\
-76.904	-109.863\\
-109.863	-52.49\\
-52.49	-62.256\\
-62.256	-68.359\\
-68.359	-61.035\\
-61.035	-76.904\\
-76.904	-67.139\\
-67.139	-47.607\\
-47.607	-34.18\\
-34.18	-29.297\\
-29.297	-39.063\\
-39.063	-73.242\\
-73.242	-87.891\\
-87.891	-62.256\\
-62.256	-43.945\\
-43.945	-29.297\\
-29.297	-41.504\\
-41.504	-64.697\\
-64.697	-84.229\\
-84.229	-89.111\\
-89.111	-63.477\\
-63.477	-59.814\\
-59.814	-67.139\\
-67.139	-91.553\\
-91.553	-126.953\\
-126.953	-146.484\\
-146.484	-111.084\\
-111.084	-69.58\\
-69.58	-75.684\\
-75.684	-69.58\\
-69.58	-72.021\\
-72.021	-50.049\\
-50.049	-59.814\\
-59.814	-67.139\\
-67.139	-63.477\\
-63.477	-41.504\\
-41.504	-29.297\\
-29.297	-41.504\\
-41.504	-63.477\\
-63.477	-93.994\\
-93.994	-100.098\\
-100.098	-122.07\\
-122.07	-109.863\\
-109.863	-123.291\\
-123.291	-163.574\\
-163.574	-217.285\\
-217.285	-179.443\\
-179.443	-119.629\\
-119.629	-130.615\\
-130.615	-153.809\\
-153.809	-197.754\\
-197.754	-128.174\\
-128.174	-62.256\\
-62.256	-43.945\\
-43.945	-45.166\\
-45.166	-32.959\\
-32.959	-21.973\\
-21.973	-28.076\\
-28.076	-42.725\\
-42.725	-61.035\\
-61.035	-59.814\\
-59.814	-47.607\\
-47.607	-45.166\\
-45.166	-56.152\\
-56.152	-69.58\\
-69.58	-73.242\\
-73.242	-69.58\\
-69.58	-48.828\\
-48.828	-35.4\\
-35.4	-34.18\\
-34.18	-50.049\\
-50.049	-63.477\\
-63.477	-46.387\\
-46.387	-32.959\\
-32.959	-52.49\\
-52.49	-40.283\\
-40.283	-43.945\\
-43.945	-56.152\\
-56.152	-79.346\\
-79.346	-86.67\\
-86.67	-59.814\\
-59.814	-90.332\\
-90.332	-87.891\\
-87.891	-69.58\\
-69.58	-61.035\\
-61.035	-97.656\\
-97.656	-123.291\\
-123.291	-119.629\\
-119.629	-133.057\\
-133.057	-102.539\\
-102.539	-92.773\\
-92.773	-87.891\\
-87.891	-118.408\\
-118.408	-86.67\\
-86.67	-120.85\\
-120.85	-115.967\\
-115.967	-118.408\\
-118.408	-203.857\\
-203.857	-161.133\\
-161.133	-85.449\\
-85.449	-58.594\\
-58.594	-62.256\\
-62.256	-79.346\\
-79.346	-101.318\\
-101.318	-114.746\\
-114.746	-128.174\\
-128.174	-131.836\\
-131.836	-86.67\\
-86.67	-76.904\\
-76.904	-89.111\\
-89.111	-89.111\\
-89.111	-56.152\\
-56.152	-40.283\\
-40.283	-70.801\\
-70.801	-70.801\\
-70.801	-93.994\\
-93.994	-115.967\\
-115.967	-107.422\\
-107.422	-74.463\\
-74.463	-61.035\\
-61.035	-90.332\\
-90.332	-118.408\\
-118.408	-153.809\\
-153.809	-169.678\\
-169.678	-194.092\\
-194.092	-157.471\\
-157.471	-141.602\\
-141.602	-84.229\\
-84.229	-62.256\\
-62.256	-104.98\\
-104.98	-139.16\\
-139.16	-81.787\\
-81.787	-125.732\\
-125.732	-172.119\\
-172.119	-206.299\\
-206.299	-129.395\\
-129.395	-74.463\\
-74.463	-42.725\\
-42.725	-64.697\\
-64.697	-59.814\\
-59.814	-62.256\\
-62.256	-37.842\\
-37.842	-52.49\\
-52.49	-41.504\\
-41.504	-52.49\\
-52.49	-43.945\\
-43.945	-54.932\\
-54.932	-85.449\\
-85.449	-79.346\\
-79.346	-112.305\\
-112.305	-133.057\\
-133.057	-133.057\\
-133.057	-74.463\\
-74.463	-76.904\\
-76.904	-91.553\\
-91.553	-62.256\\
-62.256	-56.152\\
-56.152	-42.725\\
-42.725	-63.477\\
-63.477	-59.814\\
-59.814	-34.18\\
-34.18	-24.414\\
-24.414	-24.414\\
-24.414	-41.504\\
-41.504	-61.035\\
-61.035	-65.918\\
-65.918	-41.504\\
-41.504	-51.27\\
-51.27	-73.242\\
-73.242	-93.994\\
-93.994	-65.918\\
-65.918	-62.256\\
-62.256	-74.463\\
-74.463	-87.891\\
-87.891	-80.566\\
-80.566	-92.773\\
-92.773	-86.67\\
-86.67	-62.256\\
-62.256	-48.828\\
-48.828	-62.256\\
-62.256	-53.711\\
-53.711	-64.697\\
-64.697	-87.891\\
-87.891	-139.16\\
-139.16	-98.877\\
-98.877	-95.215\\
-95.215	-140.381\\
-140.381	-108.643\\
-108.643	-67.139\\
-67.139	-48.828\\
-48.828	-40.283\\
-40.283	-42.725\\
-42.725	-35.4\\
-35.4	-36.621\\
-36.621	-67.139\\
-67.139	-35.4\\
-35.4	-34.18\\
-34.18	-47.607\\
-47.607	-65.918\\
-65.918	-50.049\\
-50.049	-37.842\\
-37.842	-23.193\\
-23.193	-39.063\\
-39.063	-72.021\\
-72.021	-96.436\\
-96.436	-79.346\\
-79.346	-63.477\\
-63.477	-72.021\\
-72.021	-72.021\\
-72.021	-80.566\\
-80.566	-98.877\\
-98.877	-117.188\\
-117.188	-114.746\\
-114.746	-95.215\\
-95.215	-64.697\\
-64.697	-52.49\\
-52.49	-56.152\\
-56.152	-76.904\\
-76.904	-47.607\\
-47.607	-43.945\\
-43.945	-80.566\\
-80.566	-96.436\\
-96.436	-91.553\\
-91.553	-104.98\\
-104.98	-137.939\\
-137.939	-74.463\\
-74.463	-139.16\\
-139.16	-129.395\\
-129.395	-120.85\\
-120.85	-128.174\\
-128.174	-125.732\\
-125.732	-212.402\\
-212.402	-198.975\\
-198.975	-134.277\\
-134.277	-159.912\\
-159.912	-194.092\\
-194.092	-190.43\\
-190.43	-212.402\\
-212.402	-194.092\\
-194.092	-253.906\\
-253.906	-195.313\\
-195.313	-135.498\\
-135.498	-81.787\\
-81.787	-84.229\\
-84.229	-68.359\\
-68.359	-57.373\\
-57.373	-85.449\\
-85.449	-120.85\\
-120.85	-76.904\\
-76.904	-67.139\\
-67.139	-65.918\\
-65.918	-46.387\\
-46.387	-56.152\\
-56.152	-29.297\\
-29.297	-39.063\\
-39.063	-29.297\\
-29.297	-46.387\\
-46.387	-32.959\\
-32.959	-30.518\\
-30.518	-19.531\\
-19.531	-18.311\\
-18.311	-34.18\\
-34.18	-25.635\\
-25.635	-28.076\\
-28.076	-57.373\\
-57.373	-78.125\\
-78.125	-80.566\\
-80.566	-58.594\\
-58.594	-74.463\\
-74.463	-114.746\\
-114.746	-53.711\\
-53.711	-37.842\\
-37.842	-28.076\\
-28.076	-20.752\\
-20.752	-34.18\\
-34.18	-56.152\\
-56.152	-84.229\\
-84.229	-50.049\\
-50.049	-85.449\\
-85.449	-118.408\\
-118.408	-120.85\\
-120.85	-89.111\\
-89.111	-57.373\\
-57.373	-72.021\\
-72.021	-96.436\\
-96.436	-124.512\\
-124.512	-68.359\\
-68.359	-37.842\\
-37.842	-57.373\\
-57.373	-47.607\\
-47.607	-23.193\\
-23.193	-18.311\\
-18.311	-39.063\\
-39.063	-85.449\\
-85.449	-85.449\\
-85.449	-63.477\\
-63.477	-41.504\\
-41.504	-43.945\\
-43.945	-15.869\\
-15.869	-30.518\\
-30.518	-25.635\\
-25.635	-42.725\\
-42.725	-58.594\\
-58.594	-89.111\\
-89.111	-93.994\\
-93.994	-100.098\\
-100.098	-104.98\\
-104.98	-102.539\\
-102.539	-100.098\\
-100.098	-86.67\\
-86.67	-104.98\\
-104.98	-157.471\\
-157.471	-144.043\\
-144.043	-74.463\\
-74.463	-40.283\\
-40.283	-86.67\\
-86.67	-106.201\\
-106.201	-96.436\\
-96.436	-56.152\\
-56.152	-57.373\\
-57.373	-97.656\\
-97.656	-146.484\\
-146.484	-150.146\\
-150.146	-169.678\\
-169.678	-164.795\\
-164.795	-120.85\\
-120.85	-123.291\\
-123.291	-58.594\\
-58.594	-56.152\\
-56.152	-72.021\\
-72.021	-87.891\\
-87.891	-91.553\\
-91.553	-96.436\\
-96.436	-63.477\\
-63.477	-86.67\\
-86.67	-69.58\\
-69.58	-41.504\\
-41.504	-29.297\\
-29.297	-24.414\\
-24.414	-37.842\\
-37.842	-30.518\\
-30.518	-18.311\\
-18.311	-39.063\\
-39.063	-74.463\\
-74.463	-48.828\\
-48.828	-45.166\\
-45.166	-61.035\\
-61.035	-100.098\\
-100.098	-68.359\\
-68.359	-37.842\\
-37.842	-24.414\\
-24.414	-53.711\\
-53.711	-108.643\\
-108.643	-136.719\\
-136.719	-189.209\\
-189.209	-223.389\\
-223.389	-220.947\\
-220.947	-142.822\\
-142.822	-147.705\\
-147.705	-191.65\\
-191.65	-159.912\\
-159.912	-147.705\\
-147.705	-168.457\\
-168.457	-183.105\\
-183.105	-185.547\\
-185.547	-253.906\\
-253.906	-168.457\\
-168.457	-95.215\\
-95.215	-56.152\\
-56.152	-45.166\\
-45.166	-50.049\\
-50.049	-48.828\\
-48.828	-48.828\\
-48.828	-86.67\\
-86.67	-65.918\\
-65.918	-73.242\\
-73.242	-92.773\\
-92.773	-53.711\\
-53.711	-64.697\\
-64.697	-84.229\\
-84.229	-98.877\\
-98.877	-56.152\\
-56.152	-39.063\\
-39.063	-50.049\\
-50.049	-48.828\\
-48.828	-113.525\\
-113.525	-155.029\\
-155.029	-145.264\\
-145.264	-168.457\\
-168.457	-189.209\\
-189.209	-246.582\\
-246.582	-208.74\\
-208.74	-135.498\\
-135.498	-89.111\\
-89.111	-51.27\\
-51.27	-58.594\\
-58.594	-58.594\\
-58.594	-76.904\\
-76.904	-53.711\\
-53.711	-50.049\\
-50.049	-53.711\\
-53.711	-69.58\\
-69.58	-76.904\\
-76.904	-95.215\\
-95.215	-68.359\\
-68.359	-32.959\\
-32.959	-25.635\\
-25.635	-23.193\\
-23.193	-25.635\\
-25.635	-30.518\\
-30.518	-53.711\\
-53.711	-52.49\\
-52.49	-65.918\\
-65.918	-95.215\\
-95.215	-68.359\\
-68.359	-113.525\\
-113.525	-164.795\\
-164.795	-129.395\\
-129.395	-115.967\\
-115.967	-140.381\\
-140.381	-162.354\\
-162.354	-108.643\\
-108.643	-93.994\\
-93.994	-59.814\\
-59.814	-65.918\\
-65.918	-131.836\\
-131.836	-214.844\\
-214.844	-256.348\\
-256.348	-202.637\\
-202.637	-168.457\\
-168.457	-123.291\\
-123.291	-133.057\\
-133.057	-175.781\\
-175.781	-93.994\\
-93.994	-83.008\\
-83.008	-63.477\\
-63.477	-115.967\\
-115.967	-112.305\\
-112.305	-61.035\\
-61.035	-62.256\\
-62.256	-46.387\\
-46.387	-84.229\\
-84.229	-119.629\\
-119.629	-126.953\\
-126.953	-95.215\\
-95.215	-69.58\\
-69.58	-80.566\\
-80.566	-61.035\\
-61.035	-46.387\\
-46.387	-36.621\\
-36.621	-32.959\\
-32.959	-29.297\\
-29.297	-34.18\\
-34.18	-59.814\\
-59.814	-84.229\\
-84.229	-83.008\\
-83.008	-52.49\\
-52.49	-47.607\\
-47.607	-40.283\\
-40.283	-52.49\\
-52.49	-68.359\\
-68.359	-73.242\\
-73.242	-69.58\\
-69.58	-40.283\\
-40.283	-83.008\\
-83.008	-120.85\\
-120.85	-84.229\\
-84.229	-35.4\\
-35.4	-40.283\\
-40.283	-67.139\\
-67.139	-67.139\\
-67.139	-58.594\\
-58.594	-37.842\\
-37.842	-68.359\\
-68.359	-97.656\\
-97.656	-59.814\\
-59.814	-24.414\\
-24.414	-37.842\\
-37.842	-81.787\\
-81.787	-96.436\\
-96.436	-59.814\\
-59.814	-53.711\\
-53.711	-97.656\\
-97.656	-125.732\\
-125.732	-108.643\\
-108.643	-152.588\\
-152.588	-184.326\\
-184.326	-115.967\\
-115.967	-95.215\\
-95.215	-139.16\\
-139.16	-93.994\\
-93.994	-126.953\\
-126.953	-152.588\\
-152.588	-119.629\\
-119.629	-57.373\\
-57.373	-101.318\\
-101.318	-172.119\\
-172.119	-195.313\\
-195.313	-142.822\\
-142.822	-120.85\\
-120.85	-151.367\\
-151.367	-118.408\\
-118.408	-76.904\\
-76.904	-73.242\\
-73.242	-91.553\\
-91.553	-95.215\\
-95.215	-93.994\\
-93.994	-104.98\\
-104.98	-72.021\\
-72.021	-79.346\\
-79.346	-109.863\\
-109.863	-64.697\\
-64.697	-53.711\\
-53.711	-43.945\\
-43.945	-35.4\\
-35.4	-42.725\\
-42.725	-73.242\\
-73.242	-65.918\\
-65.918	-93.994\\
-93.994	-114.746\\
-114.746	-139.16\\
-139.16	-102.539\\
-102.539	-92.773\\
-92.773	-43.945\\
-43.945	-56.152\\
-56.152	-107.422\\
-107.422	-73.242\\
-73.242	-42.725\\
-42.725	-78.125\\
-78.125	-115.967\\
-115.967	-120.85\\
-120.85	-157.471\\
-157.471	-205.078\\
-205.078	-136.719\\
-136.719	-118.408\\
-118.408	-142.822\\
-142.822	-92.773\\
-92.773	-76.904\\
-76.904	-85.449\\
-85.449	-111.084\\
-111.084	-118.408\\
-118.408	-119.629\\
-119.629	-67.139\\
-67.139	-59.814\\
-59.814	-52.49\\
-52.49	-73.242\\
-73.242	-65.918\\
-65.918	-52.49\\
-52.49	-104.98\\
-104.98	-111.084\\
-111.084	-78.125\\
-78.125	-64.697\\
-64.697	-96.436\\
-96.436	-53.711\\
-53.711	-32.959\\
-32.959	-41.504\\
-41.504	-57.373\\
-57.373	-48.828\\
-48.828	-35.4\\
-35.4	-21.973\\
-21.973	-40.283\\
-40.283	-54.932\\
-54.932	-86.67\\
-86.67	-96.436\\
-96.436	-126.953\\
-126.953	-142.822\\
-142.822	-81.787\\
-81.787	-62.256\\
-62.256	-128.174\\
-128.174	-181.885\\
-181.885	-139.16\\
-139.16	-130.615\\
-130.615	-196.533\\
-196.533	-155.029\\
-155.029	-128.174\\
-128.174	-93.994\\
-93.994	-98.877\\
-98.877	-130.615\\
-130.615	-85.449\\
-85.449	-74.463\\
-74.463	-128.174\\
-128.174	-158.691\\
-158.691	-169.678\\
-169.678	-73.242\\
-73.242	-39.063\\
-39.063	-93.994\\
-93.994	-73.242\\
-73.242	-35.4\\
-35.4	-29.297\\
-29.297	-50.049\\
-50.049	-67.139\\
-67.139	-109.863\\
-109.863	-76.904\\
-76.904	-61.035\\
-61.035	-36.621\\
-36.621	-26.855\\
-26.855	-36.621\\
-36.621	-30.518\\
-30.518	-40.283\\
-40.283	-68.359\\
-68.359	-61.035\\
-61.035	-32.959\\
-32.959	-30.518\\
-30.518	-39.063\\
-39.063	-47.607\\
-47.607	-29.297\\
-29.297	-47.607\\
-47.607	-32.959\\
-32.959	-24.414\\
-24.414	-56.152\\
-56.152	-78.125\\
-78.125	-115.967\\
-115.967	-156.25\\
-156.25	-137.939\\
-137.939	-67.139\\
-67.139	-51.27\\
-51.27	-51.27\\
-51.27	-29.297\\
-29.297	-30.518\\
-30.518	-57.373\\
-57.373	-62.256\\
-62.256	-57.373\\
-57.373	-65.918\\
-65.918	-76.904\\
-76.904	-80.566\\
-80.566	-83.008\\
-83.008	-103.76\\
-103.76	-100.098\\
-100.098	-145.264\\
-145.264	-76.904\\
-76.904	-37.842\\
-37.842	-36.621\\
-36.621	-67.139\\
-67.139	-50.049\\
-50.049	-47.607\\
-47.607	-23.193\\
-23.193	-41.504\\
-41.504	-25.635\\
-25.635	-15.869\\
-15.869	-25.635\\
-25.635	-47.607\\
-47.607	-58.594\\
-58.594	-84.229\\
-84.229	-118.408\\
-118.408	-84.229\\
-84.229	-35.4\\
-35.4	-65.918\\
-65.918	-74.463\\
-74.463	-39.063\\
-39.063	-51.27\\
-51.27	-92.773\\
-92.773	-81.787\\
-81.787	-137.939\\
-137.939	-157.471\\
-157.471	-107.422\\
-107.422	-83.008\\
-83.008	-74.463\\
};
\addlegendentry{data1}

\addplot [color=mycolor2, line width=2.0pt]
  table[row sep=crcr]{%
-84.229	-80.7789024124196\\
-96.436	-92.485892424748\\
-122.07	-117.069900123284\\
-100.098	-95.9978935245388\\
-93.994	-90.1439189988362\\
-130.615	-125.264889035821\\
-125.732	-120.581901223075\\
-153.809	-147.508841386599\\
-115.967	-111.216884636658\\
-61.035	-58.5349500616419\\
-74.463	-71.4129267869262\\
-73.242	-70.2419400739702\\
-86.67	-83.1199167992545\\
-73.242	-70.2419400739702\\
-34.18	-32.7799556501502\\
-20.752	-19.9019789248659\\
-23.193	-22.2429933117008\\
-58.594	-56.193935674807\\
-100.098	-95.9978935245388\\
-109.863	-105.362910110955\\
-114.746	-110.045897923702\\
-83.008	-79.6079156994637\\
-52.49	-50.3399611491043\\
-104.98	-100.679922298209\\
-81.787	-78.4369289865078\\
-59.814	-57.3639633486859\\
-97.656	-93.655920098627\\
-83.008	-79.6079156994637\\
-72.021	-69.0709533610143\\
-118.408	-113.557899023493\\
-107.422	-103.02189572412\\
-83.008	-79.6079156994637\\
-119.629	-114.728885736449\\
-123.291	-118.24088683624\\
-95.215	-91.3149057117921\\
-80.566	-77.2659422735519\\
-62.256	-59.7059367745978\\
-75.684	-72.5839134998821\\
-95.215	-91.3149057117921\\
-87.891	-84.2909035122105\\
-91.553	-87.8029046120013\\
-96.436	-92.485892424748\\
-102.539	-98.3389079113737\\
-141.602	-135.801851374271\\
-128.174	-122.923874648986\\
-85.449	-81.9489300862986\\
-79.346	-76.0959145996729\\
-47.607	-45.6569733363576\\
-48.828	-46.8279600493135\\
-68.359	-65.5589522612235\\
-52.49	-50.3399611491043\\
-57.373	-55.022948961851\\
-75.684	-72.5839134998821\\
-89.111	-85.4609311860894\\
-81.787	-78.4369289865078\\
-128.174	-122.923874648986\\
-98.877	-94.8269068115829\\
-57.373	-55.022948961851\\
-45.166	-43.3159589495227\\
-62.256	-59.7059367745978\\
-72.021	-69.0709533610143\\
-40.283	-38.6329711367759\\
-42.725	-40.9749445626878\\
-56.152	-53.8519622488951\\
-46.387	-44.4869456624786\\
-40.283	-38.6329711367759\\
-52.49	-50.3399611491043\\
-62.256	-59.7059367745978\\
-92.773	-88.9729322858802\\
-90.332	-86.6319178990453\\
-107.422	-103.02189572412\\
-98.877	-94.8269068115829\\
-115.967	-111.216884636658\\
-91.553	-87.8029046120013\\
-79.346	-76.0959145996729\\
-75.684	-72.5839134998821\\
-76.904	-73.753941173761\\
-133.057	-127.606862461733\\
-189.209	-181.458824710628\\
-194.092	-186.141812523375\\
-189.209	-181.458824710628\\
-119.629	-114.728885736449\\
-170.898	-163.897860172597\\
-213.623	-204.872804735285\\
-219.727	-210.726779260988\\
-169.678	-162.727832498718\\
-219.727	-210.726779260988\\
-279.541	-268.090742609674\\
-197.754	-189.653813623166\\
-161.133	-154.532843586181\\
-109.863	-105.362910110955\\
-85.449	-81.9489300862986\\
-73.242	-70.2419400739702\\
-91.553	-87.8029046120013\\
-62.256	-59.7059367745978\\
-46.387	-44.4869456624786\\
-42.725	-40.9749445626878\\
-54.932	-52.6819345750162\\
-79.346	-76.0959145996729\\
-74.463	-71.4129267869262\\
-75.684	-72.5839134998821\\
-100.098	-95.9978935245388\\
-103.76	-99.5098946243297\\
-141.602	-135.801851374271\\
-114.746	-110.045897923702\\
-142.822	-136.97187904815\\
-104.98	-100.679922298209\\
-90.332	-86.6319178990453\\
-103.76	-99.5098946243297\\
-76.904	-73.753941173761\\
-69.58	-66.7299389741794\\
-84.229	-80.7789024124196\\
-86.67	-83.1199167992545\\
-59.814	-57.3639633486859\\
-64.697	-62.0469511614327\\
-48.828	-46.8279600493135\\
-53.711	-51.5109478620602\\
-57.373	-55.022948961851\\
-41.504	-39.8039578497319\\
-52.49	-50.3399611491043\\
-65.918	-63.2179378743886\\
-101.318	-97.1679211984178\\
-104.98	-100.679922298209\\
-114.746	-110.045897923702\\
-68.359	-65.5589522612235\\
-93.994	-90.1439189988362\\
-150.146	-143.995881247731\\
-114.746	-110.045897923702\\
-125.732	-120.581901223075\\
-128.174	-122.923874648986\\
-80.566	-77.2659422735519\\
-53.711	-51.5109478620602\\
-75.684	-72.5839134998821\\
-85.449	-81.9489300862986\\
-137.939	-132.288891235403\\
-172.119	-165.068846885553\\
-170.898	-163.897860172597\\
-123.291	-118.24088683624\\
-130.615	-125.264889035821\\
-115.967	-111.216884636658\\
-113.525	-108.874911210746\\
-111.084	-106.533896823911\\
-76.904	-73.753941173761\\
-68.359	-65.5589522612235\\
-61.035	-58.5349500616419\\
-57.373	-55.022948961851\\
-62.256	-59.7059367745978\\
-91.553	-87.8029046120013\\
-140.381	-134.630864661315\\
-111.084	-106.533896823911\\
-79.346	-76.0959145996729\\
-64.697	-62.0469511614327\\
-62.256	-59.7059367745978\\
-40.283	-38.6329711367759\\
-29.297	-28.0969678374035\\
-36.621	-35.1209700369851\\
-63.477	-60.8769234875537\\
-51.27	-49.1699334752253\\
-52.49	-50.3399611491043\\
-58.594	-56.193935674807\\
-54.932	-52.6819345750162\\
-37.842	-36.291956749941\\
-28.076	-26.9259811244476\\
-23.193	-22.2429933117008\\
-39.063	-37.462943462897\\
-83.008	-79.6079156994637\\
-117.188	-112.387871349614\\
-130.615	-125.264889035821\\
-86.67	-83.1199167992545\\
-58.594	-56.193935674807\\
-45.166	-43.3159589495227\\
-32.959	-31.6089689371943\\
-67.139	-64.3889245873445\\
-65.918	-63.2179378743886\\
-91.553	-87.8029046120013\\
-117.188	-112.387871349614\\
-186.768	-179.117810323793\\
-216.064	-207.21381912212\\
-177.002	-169.7518346983\\
-168.457	-161.556845785762\\
-109.863	-105.362910110955\\
-122.07	-117.069900123284\\
-131.836	-126.435875748777\\
-146.484	-140.48388014794\\
-108.643	-104.192882437076\\
-100.098	-95.9978935245388\\
-98.877	-94.8269068115829\\
-89.111	-85.4609311860894\\
-106.201	-101.850909011165\\
-78.125	-74.924927886717\\
-130.615	-125.264889035821\\
-159.912	-153.361856873225\\
-113.525	-108.874911210746\\
-70.801	-67.9009256871354\\
-73.242	-70.2419400739702\\
-123.291	-118.24088683624\\
-153.809	-147.508841386599\\
-189.209	-181.458824710628\\
-202.637	-194.336801435913\\
-153.809	-147.508841386599\\
-137.939	-132.288891235403\\
-158.691	-152.190870160269\\
-187.988	-180.287837997672\\
-109.863	-105.362910110955\\
-72.021	-69.0709533610143\\
-101.318	-97.1679211984178\\
-86.67	-83.1199167992545\\
-53.711	-51.5109478620602\\
-74.463	-71.4129267869262\\
-84.229	-80.7789024124196\\
-67.139	-64.3889245873445\\
-51.27	-49.1699334752253\\
-70.801	-67.9009256871354\\
-58.594	-56.193935674807\\
-79.346	-76.0959145996729\\
-107.422	-103.02189572412\\
-100.098	-95.9978935245388\\
-69.58	-66.7299389741794\\
-70.801	-67.9009256871354\\
-107.422	-103.02189572412\\
-177.002	-169.7518346983\\
-128.174	-122.923874648986\\
-79.346	-76.0959145996729\\
-61.035	-58.5349500616419\\
-70.801	-67.9009256871354\\
-64.697	-62.0469511614327\\
-48.828	-46.8279600493135\\
-53.711	-51.5109478620602\\
-65.918	-63.2179378743886\\
-83.008	-79.6079156994637\\
-91.553	-87.8029046120013\\
-63.477	-60.8769234875537\\
-104.98	-100.679922298209\\
-124.512	-119.411873549196\\
-118.408	-113.557899023493\\
-91.553	-87.8029046120013\\
-100.098	-95.9978935245388\\
-135.498	-129.947876848568\\
-87.891	-84.2909035122105\\
-42.725	-40.9749445626878\\
-57.373	-55.022948961851\\
-62.256	-59.7059367745978\\
-81.787	-78.4369289865078\\
-72.021	-69.0709533610143\\
-52.49	-50.3399611491043\\
-79.346	-76.0959145996729\\
-80.566	-77.2659422735519\\
-57.373	-55.022948961851\\
-70.801	-67.9009256871354\\
-63.477	-60.8769234875537\\
-56.152	-53.8519622488951\\
-67.139	-64.3889245873445\\
-41.504	-39.8039578497319\\
-48.828	-46.8279600493135\\
-36.621	-35.1209700369851\\
-40.283	-38.6329711367759\\
-75.684	-72.5839134998821\\
-84.229	-80.7789024124196\\
-113.525	-108.874911210746\\
-146.484	-140.48388014794\\
-98.877	-94.8269068115829\\
-58.594	-56.193935674807\\
-46.387	-44.4869456624786\\
-43.945	-42.1449722365668\\
-63.477	-60.8769234875537\\
-42.725	-40.9749445626878\\
-47.607	-45.6569733363576\\
-61.035	-58.5349500616419\\
-102.539	-98.3389079113737\\
-93.994	-90.1439189988362\\
-134.277	-128.776890135612\\
-89.111	-85.4609311860894\\
-81.787	-78.4369289865078\\
-46.387	-44.4869456624786\\
-45.166	-43.3159589495227\\
-28.076	-26.9259811244476\\
-35.4	-33.9499833240292\\
-37.842	-36.291956749941\\
-67.139	-64.3889245873445\\
-70.801	-67.9009256871354\\
-79.346	-76.0959145996729\\
-85.449	-81.9489300862986\\
-125.732	-120.581901223075\\
-112.305	-107.704883536867\\
-84.229	-80.7789024124196\\
-102.539	-98.3389079113737\\
-109.863	-105.362910110955\\
-144.043	-138.142865761106\\
-115.967	-111.216884636658\\
-75.684	-72.5839134998821\\
-40.283	-38.6329711367759\\
-29.297	-28.0969678374035\\
-36.621	-35.1209700369851\\
-39.063	-37.462943462897\\
-37.842	-36.291956749941\\
-50.049	-47.9989467622694\\
-74.463	-71.4129267869262\\
-68.359	-65.5589522612235\\
-70.801	-67.9009256871354\\
-83.008	-79.6079156994637\\
-51.27	-49.1699334752253\\
-30.518	-29.2679545503594\\
-58.594	-56.193935674807\\
-90.332	-86.6319178990453\\
-87.891	-84.2909035122105\\
-97.656	-93.655920098627\\
-83.008	-79.6079156994637\\
-61.035	-58.5349500616419\\
-85.449	-81.9489300862986\\
-118.408	-113.557899023493\\
-84.229	-80.7789024124196\\
-136.719	-131.118863561524\\
-100.098	-95.9978935245388\\
-101.318	-97.1679211984178\\
-117.188	-112.387871349614\\
-115.967	-111.216884636658\\
-86.67	-83.1199167992545\\
-76.904	-73.753941173761\\
-128.174	-122.923874648986\\
-178.223	-170.922821411256\\
-139.16	-133.459877948359\\
-79.346	-76.0959145996729\\
-76.904	-73.753941173761\\
-79.346	-76.0959145996729\\
-98.877	-94.8269068115829\\
-114.746	-110.045897923702\\
-85.449	-81.9489300862986\\
-90.332	-86.6319178990453\\
-120.85	-115.899872449405\\
-131.836	-126.435875748777\\
-87.891	-84.2909035122105\\
-68.359	-65.5589522612235\\
-91.553	-87.8029046120013\\
-156.25	-149.849855773434\\
-137.939	-132.288891235403\\
-140.381	-134.630864661315\\
-84.229	-80.7789024124196\\
-76.904	-73.753941173761\\
-72.021	-69.0709533610143\\
-47.607	-45.6569733363576\\
-30.518	-29.2679545503594\\
-26.855	-25.7549944114916\\
-23.193	-22.2429933117008\\
-54.932	-52.6819345750162\\
-70.801	-67.9009256871354\\
-87.891	-84.2909035122105\\
-65.918	-63.2179378743886\\
-73.242	-70.2419400739702\\
-83.008	-79.6079156994637\\
-53.711	-51.5109478620602\\
-56.152	-53.8519622488951\\
-53.711	-51.5109478620602\\
-40.283	-38.6329711367759\\
-51.27	-49.1699334752253\\
-84.229	-80.7789024124196\\
-106.201	-101.850909011165\\
-63.477	-60.8769234875537\\
-48.828	-46.8279600493135\\
-40.283	-38.6329711367759\\
-50.049	-47.9989467622694\\
-35.4	-33.9499833240292\\
-76.904	-73.753941173761\\
-130.615	-125.264889035821\\
-104.98	-100.679922298209\\
-139.16	-133.459877948359\\
-162.354	-155.703830299137\\
-164.795	-158.044844685971\\
-124.512	-119.411873549196\\
-90.332	-86.6319178990453\\
-89.111	-85.4609311860894\\
-106.201	-101.850909011165\\
-125.732	-120.581901223075\\
-168.457	-161.556845785762\\
-183.105	-175.604850184926\\
-219.727	-210.726779260988\\
-147.705	-141.654866860896\\
-166.016	-159.215831398927\\
-184.326	-176.775836897882\\
-140.381	-134.630864661315\\
-162.354	-155.703830299137\\
-139.16	-133.459877948359\\
-80.566	-77.2659422735519\\
-52.49	-50.3399611491043\\
-56.152	-53.8519622488951\\
-81.787	-78.4369289865078\\
-65.918	-63.2179378743886\\
-43.945	-42.1449722365668\\
-62.256	-59.7059367745978\\
-47.607	-45.6569733363576\\
-36.621	-35.1209700369851\\
-46.387	-44.4869456624786\\
-52.49	-50.3399611491043\\
-36.621	-35.1209700369851\\
-70.801	-67.9009256871354\\
-92.773	-88.9729322858802\\
-68.359	-65.5589522612235\\
-114.746	-110.045897923702\\
-164.795	-158.044844685971\\
-150.146	-143.995881247731\\
-196.533	-188.48282691021\\
-131.836	-126.435875748777\\
-144.043	-138.142865761106\\
-96.436	-92.485892424748\\
-63.477	-60.8769234875537\\
-76.904	-73.753941173761\\
-59.814	-57.3639633486859\\
-45.166	-43.3159589495227\\
-64.697	-62.0469511614327\\
-36.621	-35.1209700369851\\
-28.076	-26.9259811244476\\
-34.18	-32.7799556501502\\
-70.801	-67.9009256871354\\
-86.67	-83.1199167992545\\
-107.422	-103.02189572412\\
-103.76	-99.5098946243297\\
-100.098	-95.9978935245388\\
-114.746	-110.045897923702\\
-93.994	-90.1439189988362\\
-76.904	-73.753941173761\\
-62.256	-59.7059367745978\\
-97.656	-93.655920098627\\
-68.359	-65.5589522612235\\
-56.152	-53.8519622488951\\
-81.787	-78.4369289865078\\
-113.525	-108.874911210746\\
-86.67	-83.1199167992545\\
-65.918	-63.2179378743886\\
-56.152	-53.8519622488951\\
-65.918	-63.2179378743886\\
-45.166	-43.3159589495227\\
-59.814	-57.3639633486859\\
-75.684	-72.5839134998821\\
-84.229	-80.7789024124196\\
-65.918	-63.2179378743886\\
-86.67	-83.1199167992545\\
-79.346	-76.0959145996729\\
-47.607	-45.6569733363576\\
-50.049	-47.9989467622694\\
-95.215	-91.3149057117921\\
-131.836	-126.435875748777\\
-107.422	-103.02189572412\\
-103.76	-99.5098946243297\\
-79.346	-76.0959145996729\\
-76.904	-73.753941173761\\
-61.035	-58.5349500616419\\
-89.111	-85.4609311860894\\
-75.684	-72.5839134998821\\
-50.049	-47.9989467622694\\
-74.463	-71.4129267869262\\
-85.449	-81.9489300862986\\
-101.318	-97.1679211984178\\
-109.863	-105.362910110955\\
-148.926	-142.825853573852\\
-181.885	-174.434822511047\\
-205.078	-196.677815822747\\
-129.395	-124.094861361942\\
-73.242	-70.2419400739702\\
-47.607	-45.6569733363576\\
-34.18	-32.7799556501502\\
-54.932	-52.6819345750162\\
-46.387	-44.4869456624786\\
-80.566	-77.2659422735519\\
-72.021	-69.0709533610143\\
-64.697	-62.0469511614327\\
-85.449	-81.9489300862986\\
-98.877	-94.8269068115829\\
-117.188	-112.387871349614\\
-123.291	-118.24088683624\\
-175.781	-168.580847985344\\
-150.146	-143.995881247731\\
-117.188	-112.387871349614\\
-80.566	-77.2659422735519\\
-83.008	-79.6079156994637\\
-106.201	-101.850909011165\\
-75.684	-72.5839134998821\\
-79.346	-76.0959145996729\\
-74.463	-71.4129267869262\\
-42.725	-40.9749445626878\\
-74.463	-71.4129267869262\\
-107.422	-103.02189572412\\
-73.242	-70.2419400739702\\
-80.566	-77.2659422735519\\
-147.705	-141.654866860896\\
-109.863	-105.362910110955\\
-106.201	-101.850909011165\\
-147.705	-141.654866860896\\
-140.381	-134.630864661315\\
-87.891	-84.2909035122105\\
-56.152	-53.8519622488951\\
-58.594	-56.193935674807\\
-97.656	-93.655920098627\\
-151.367	-145.166867960687\\
-162.354	-155.703830299137\\
-167.236	-160.385859072806\\
-129.395	-124.094861361942\\
-83.008	-79.6079156994637\\
-72.021	-69.0709533610143\\
-53.711	-51.5109478620602\\
-50.049	-47.9989467622694\\
-42.725	-40.9749445626878\\
-61.035	-58.5349500616419\\
-72.021	-69.0709533610143\\
-41.504	-39.8039578497319\\
-65.918	-63.2179378743886\\
-89.111	-85.4609311860894\\
-101.318	-97.1679211984178\\
-128.174	-122.923874648986\\
-161.133	-154.532843586181\\
-192.871	-184.970825810419\\
-136.719	-131.118863561524\\
-118.408	-113.557899023493\\
-124.512	-119.411873549196\\
-102.539	-98.3389079113737\\
-73.242	-70.2419400739702\\
-89.111	-85.4609311860894\\
-136.719	-131.118863561524\\
-162.354	-155.703830299137\\
-93.994	-90.1439189988362\\
-54.932	-52.6819345750162\\
-37.842	-36.291956749941\\
-23.193	-22.2429933117008\\
-32.959	-31.6089689371943\\
-45.166	-43.3159589495227\\
-52.49	-50.3399611491043\\
-73.242	-70.2419400739702\\
-58.594	-56.193935674807\\
-45.166	-43.3159589495227\\
-43.945	-42.1449722365668\\
-40.283	-38.6329711367759\\
-47.607	-45.6569733363576\\
-70.801	-67.9009256871354\\
-104.98	-100.679922298209\\
-111.084	-106.533896823911\\
-107.422	-103.02189572412\\
-123.291	-118.24088683624\\
-152.588	-146.337854673643\\
-155.029	-148.678869060478\\
-183.105	-175.604850184926\\
-167.236	-160.385859072806\\
-124.512	-119.411873549196\\
-85.449	-81.9489300862986\\
-83.008	-79.6079156994637\\
-141.602	-135.801851374271\\
-158.691	-152.190870160269\\
-115.967	-111.216884636658\\
-76.904	-73.753941173761\\
-39.063	-37.462943462897\\
-30.518	-29.2679545503594\\
-29.297	-28.0969678374035\\
-19.531	-18.73099221191\\
-45.166	-43.3159589495227\\
-57.373	-55.022948961851\\
-62.256	-59.7059367745978\\
-53.711	-51.5109478620602\\
-34.18	-32.7799556501502\\
-26.855	-25.7549944114916\\
-36.621	-35.1209700369851\\
-41.504	-39.8039578497319\\
-43.945	-42.1449722365668\\
-35.4	-33.9499833240292\\
-28.076	-26.9259811244476\\
-50.049	-47.9989467622694\\
-112.305	-107.704883536867\\
-128.174	-122.923874648986\\
-145.264	-139.313852474062\\
-101.318	-97.1679211984178\\
-133.057	-127.606862461733\\
-195.313	-187.312799236331\\
-174.561	-167.410820311465\\
-184.326	-176.775836897882\\
-135.498	-129.947876848568\\
-137.939	-132.288891235403\\
-145.264	-139.313852474062\\
-95.215	-91.3149057117921\\
-90.332	-86.6319178990453\\
-56.152	-53.8519622488951\\
-54.932	-52.6819345750162\\
-102.539	-98.3389079113737\\
-68.359	-65.5589522612235\\
-79.346	-76.0959145996729\\
-84.229	-80.7789024124196\\
-79.346	-76.0959145996729\\
-84.229	-80.7789024124196\\
-93.994	-90.1439189988362\\
-102.539	-98.3389079113737\\
-89.111	-85.4609311860894\\
-67.139	-64.3889245873445\\
-84.229	-80.7789024124196\\
-39.063	-37.462943462897\\
-20.752	-19.9019789248659\\
-36.621	-35.1209700369851\\
-51.27	-49.1699334752253\\
-26.855	-25.7549944114916\\
-13.428	-12.8779767252843\\
-31.738	-30.4379822242384\\
-50.049	-47.9989467622694\\
-58.594	-56.193935674807\\
-72.021	-69.0709533610143\\
-62.256	-59.7059367745978\\
-91.553	-87.8029046120013\\
-80.566	-77.2659422735519\\
-57.373	-55.022948961851\\
-86.67	-83.1199167992545\\
-48.828	-46.8279600493135\\
-41.504	-39.8039578497319\\
-28.076	-26.9259811244476\\
-21.973	-21.0729656378219\\
-37.842	-36.291956749941\\
-28.076	-26.9259811244476\\
-50.049	-47.9989467622694\\
-75.684	-72.5839134998821\\
-73.242	-70.2419400739702\\
-54.932	-52.6819345750162\\
-61.035	-58.5349500616419\\
-45.166	-43.3159589495227\\
-65.918	-63.2179378743886\\
-47.607	-45.6569733363576\\
-46.387	-44.4869456624786\\
-42.725	-40.9749445626878\\
-32.959	-31.6089689371943\\
-42.725	-40.9749445626878\\
-37.842	-36.291956749941\\
-64.697	-62.0469511614327\\
-63.477	-60.8769234875537\\
-50.049	-47.9989467622694\\
-46.387	-44.4869456624786\\
-36.621	-35.1209700369851\\
-43.945	-42.1449722365668\\
-93.994	-90.1439189988362\\
-120.85	-115.899872449405\\
-81.787	-78.4369289865078\\
-80.566	-77.2659422735519\\
-54.932	-52.6819345750162\\
-93.994	-90.1439189988362\\
-81.787	-78.4369289865078\\
-54.932	-52.6819345750162\\
-70.801	-67.9009256871354\\
-52.49	-50.3399611491043\\
-75.684	-72.5839134998821\\
-42.725	-40.9749445626878\\
-50.049	-47.9989467622694\\
-57.373	-55.022948961851\\
-74.463	-71.4129267869262\\
-58.594	-56.193935674807\\
-61.035	-58.5349500616419\\
-95.215	-91.3149057117921\\
-79.346	-76.0959145996729\\
-125.732	-120.581901223075\\
-157.471	-151.02084248639\\
-125.732	-120.581901223075\\
-81.787	-78.4369289865078\\
-62.256	-59.7059367745978\\
-89.111	-85.4609311860894\\
-112.305	-107.704883536867\\
-87.891	-84.2909035122105\\
-107.422	-103.02189572412\\
-64.697	-62.0469511614327\\
-34.18	-32.7799556501502\\
-37.842	-36.291956749941\\
-84.229	-80.7789024124196\\
-70.801	-67.9009256871354\\
-69.58	-66.7299389741794\\
-104.98	-100.679922298209\\
-97.656	-93.655920098627\\
-87.891	-84.2909035122105\\
-62.256	-59.7059367745978\\
-73.242	-70.2419400739702\\
-100.098	-95.9978935245388\\
-81.787	-78.4369289865078\\
-87.891	-84.2909035122105\\
-78.125	-74.924927886717\\
-58.594	-56.193935674807\\
-67.139	-64.3889245873445\\
-48.828	-46.8279600493135\\
-56.152	-53.8519622488951\\
-95.215	-91.3149057117921\\
-109.863	-105.362910110955\\
-115.967	-111.216884636658\\
-102.539	-98.3389079113737\\
-73.242	-70.2419400739702\\
-51.27	-49.1699334752253\\
-62.256	-59.7059367745978\\
-70.801	-67.9009256871354\\
-91.553	-87.8029046120013\\
-111.084	-106.533896823911\\
-131.836	-126.435875748777\\
-107.422	-103.02189572412\\
-62.256	-59.7059367745978\\
-109.863	-105.362910110955\\
-151.367	-145.166867960687\\
-107.422	-103.02189572412\\
-108.643	-104.192882437076\\
-124.512	-119.411873549196\\
-93.994	-90.1439189988362\\
-79.346	-76.0959145996729\\
-89.111	-85.4609311860894\\
-78.125	-74.924927886717\\
-91.553	-87.8029046120013\\
-114.746	-110.045897923702\\
-86.67	-83.1199167992545\\
-102.539	-98.3389079113737\\
-92.773	-88.9729322858802\\
-96.436	-92.485892424748\\
-76.904	-73.753941173761\\
-100.098	-95.9978935245388\\
-69.58	-66.7299389741794\\
-76.904	-73.753941173761\\
-84.229	-80.7789024124196\\
-80.566	-77.2659422735519\\
-41.504	-39.8039578497319\\
-28.076	-26.9259811244476\\
-21.973	-21.0729656378219\\
-37.842	-36.291956749941\\
-84.229	-80.7789024124196\\
-108.643	-104.192882437076\\
-111.084	-106.533896823911\\
-74.463	-71.4129267869262\\
-86.67	-83.1199167992545\\
-72.021	-69.0709533610143\\
-65.918	-63.2179378743886\\
-42.725	-40.9749445626878\\
-61.035	-58.5349500616419\\
-58.594	-56.193935674807\\
-57.373	-55.022948961851\\
-67.139	-64.3889245873445\\
-57.373	-55.022948961851\\
-53.711	-51.5109478620602\\
-37.842	-36.291956749941\\
-50.049	-47.9989467622694\\
-91.553	-87.8029046120013\\
-70.801	-67.9009256871354\\
-86.67	-83.1199167992545\\
-75.684	-72.5839134998821\\
-101.318	-97.1679211984178\\
-92.773	-88.9729322858802\\
-129.395	-124.094861361942\\
-91.553	-87.8029046120013\\
-106.201	-101.850909011165\\
-103.76	-99.5098946243297\\
-54.932	-52.6819345750162\\
-96.436	-92.485892424748\\
-67.139	-64.3889245873445\\
-84.229	-80.7789024124196\\
-90.332	-86.6319178990453\\
-115.967	-111.216884636658\\
-85.449	-81.9489300862986\\
-111.084	-106.533896823911\\
-136.719	-131.118863561524\\
-111.084	-106.533896823911\\
-95.215	-91.3149057117921\\
-75.684	-72.5839134998821\\
-91.553	-87.8029046120013\\
-80.566	-77.2659422735519\\
-69.58	-66.7299389741794\\
-56.152	-53.8519622488951\\
-70.801	-67.9009256871354\\
-58.594	-56.193935674807\\
-119.629	-114.728885736449\\
-151.367	-145.166867960687\\
-130.615	-125.264889035821\\
-126.953	-121.75288793603\\
-124.512	-119.411873549196\\
-69.58	-66.7299389741794\\
-63.477	-60.8769234875537\\
-50.049	-47.9989467622694\\
-42.725	-40.9749445626878\\
-48.828	-46.8279600493135\\
-80.566	-77.2659422735519\\
-100.098	-95.9978935245388\\
-114.746	-110.045897923702\\
-96.436	-92.485892424748\\
-67.139	-64.3889245873445\\
-117.188	-112.387871349614\\
-137.939	-132.288891235403\\
-152.588	-146.337854673643\\
-118.408	-113.557899023493\\
-187.988	-180.287837997672\\
-133.057	-127.606862461733\\
-78.125	-74.924927886717\\
-57.373	-55.022948961851\\
-48.828	-46.8279600493135\\
-87.891	-84.2909035122105\\
-111.084	-106.533896823911\\
-147.705	-141.654866860896\\
-109.863	-105.362910110955\\
-87.891	-84.2909035122105\\
-47.607	-45.6569733363576\\
-53.711	-51.5109478620602\\
-56.152	-53.8519622488951\\
-63.477	-60.8769234875537\\
-40.283	-38.6329711367759\\
-52.49	-50.3399611491043\\
-40.283	-38.6329711367759\\
-26.855	-25.7549944114916\\
-58.594	-56.193935674807\\
-63.477	-60.8769234875537\\
-80.566	-77.2659422735519\\
-72.021	-69.0709533610143\\
-76.904	-73.753941173761\\
-98.877	-94.8269068115829\\
-67.139	-64.3889245873445\\
-101.318	-97.1679211984178\\
-117.188	-112.387871349614\\
-89.111	-85.4609311860894\\
-93.994	-90.1439189988362\\
-80.566	-77.2659422735519\\
-93.994	-90.1439189988362\\
-133.057	-127.606862461733\\
-100.098	-95.9978935245388\\
-79.346	-76.0959145996729\\
-90.332	-86.6319178990453\\
-81.787	-78.4369289865078\\
-56.152	-53.8519622488951\\
-85.449	-81.9489300862986\\
-123.291	-118.24088683624\\
-117.188	-112.387871349614\\
-130.615	-125.264889035821\\
-192.871	-184.970825810419\\
-125.732	-120.581901223075\\
-107.422	-103.02189572412\\
-83.008	-79.6079156994637\\
-106.201	-101.850909011165\\
-152.588	-146.337854673643\\
-101.318	-97.1679211984178\\
-90.332	-86.6319178990453\\
-124.512	-119.411873549196\\
-79.346	-76.0959145996729\\
-47.607	-45.6569733363576\\
-62.256	-59.7059367745978\\
-46.387	-44.4869456624786\\
-39.063	-37.462943462897\\
-42.725	-40.9749445626878\\
-37.842	-36.291956749941\\
-47.607	-45.6569733363576\\
-65.918	-63.2179378743886\\
-78.125	-74.924927886717\\
-102.539	-98.3389079113737\\
-96.436	-92.485892424748\\
-112.305	-107.704883536867\\
-130.615	-125.264889035821\\
-119.629	-114.728885736449\\
-86.67	-83.1199167992545\\
-92.773	-88.9729322858802\\
-83.008	-79.6079156994637\\
-91.553	-87.8029046120013\\
-126.953	-121.75288793603\\
-93.994	-90.1439189988362\\
-106.201	-101.850909011165\\
-91.553	-87.8029046120013\\
-87.891	-84.2909035122105\\
-140.381	-134.630864661315\\
-113.525	-108.874911210746\\
-115.967	-111.216884636658\\
-103.76	-99.5098946243297\\
-64.697	-62.0469511614327\\
-42.725	-40.9749445626878\\
-34.18	-32.7799556501502\\
-61.035	-58.5349500616419\\
-54.932	-52.6819345750162\\
-74.463	-71.4129267869262\\
-56.152	-53.8519622488951\\
-79.346	-76.0959145996729\\
-124.512	-119.411873549196\\
-155.029	-148.678869060478\\
-122.07	-117.069900123284\\
-128.174	-122.923874648986\\
-113.525	-108.874911210746\\
-83.008	-79.6079156994637\\
-87.891	-84.2909035122105\\
-98.877	-94.8269068115829\\
-84.229	-80.7789024124196\\
-101.318	-97.1679211984178\\
-126.953	-121.75288793603\\
-178.223	-170.922821411256\\
-156.25	-149.849855773434\\
-147.705	-141.654866860896\\
-164.795	-158.044844685971\\
-109.863	-105.362910110955\\
-120.85	-115.899872449405\\
-93.994	-90.1439189988362\\
-95.215	-91.3149057117921\\
-126.953	-121.75288793603\\
-144.043	-138.142865761106\\
-56.152	-53.8519622488951\\
-74.463	-71.4129267869262\\
-57.373	-55.022948961851\\
-83.008	-79.6079156994637\\
-84.229	-80.7789024124196\\
-51.27	-49.1699334752253\\
-68.359	-65.5589522612235\\
-119.629	-114.728885736449\\
-196.533	-188.48282691021\\
-190.43	-182.629811423584\\
-174.561	-167.410820311465\\
-137.939	-132.288891235403\\
-161.133	-154.532843586181\\
-196.533	-188.48282691021\\
-144.043	-138.142865761106\\
-148.926	-142.825853573852\\
-153.809	-147.508841386599\\
-104.98	-100.679922298209\\
-91.553	-87.8029046120013\\
-118.408	-113.557899023493\\
-79.346	-76.0959145996729\\
-59.814	-57.3639633486859\\
-80.566	-77.2659422735519\\
-59.814	-57.3639633486859\\
-73.242	-70.2419400739702\\
-67.139	-64.3889245873445\\
-84.229	-80.7789024124196\\
-111.084	-106.533896823911\\
-85.449	-81.9489300862986\\
-64.697	-62.0469511614327\\
-76.904	-73.753941173761\\
-89.111	-85.4609311860894\\
-107.422	-103.02189572412\\
-89.111	-85.4609311860894\\
-74.463	-71.4129267869262\\
-115.967	-111.216884636658\\
-108.643	-104.192882437076\\
-75.684	-72.5839134998821\\
-126.953	-121.75288793603\\
-113.525	-108.874911210746\\
-83.008	-79.6079156994637\\
-57.373	-55.022948961851\\
-42.725	-40.9749445626878\\
-31.738	-30.4379822242384\\
-69.58	-66.7299389741794\\
-63.477	-60.8769234875537\\
-47.607	-45.6569733363576\\
-34.18	-32.7799556501502\\
-42.725	-40.9749445626878\\
-70.801	-67.9009256871354\\
-81.787	-78.4369289865078\\
-104.98	-100.679922298209\\
-122.07	-117.069900123284\\
-117.188	-112.387871349614\\
-122.07	-117.069900123284\\
-72.021	-69.0709533610143\\
-34.18	-32.7799556501502\\
-50.049	-47.9989467622694\\
-41.504	-39.8039578497319\\
-53.711	-51.5109478620602\\
-86.67	-83.1199167992545\\
-95.215	-91.3149057117921\\
-141.602	-135.801851374271\\
-92.773	-88.9729322858802\\
-87.891	-84.2909035122105\\
-128.174	-122.923874648986\\
-93.994	-90.1439189988362\\
-65.918	-63.2179378743886\\
-50.049	-47.9989467622694\\
-65.918	-63.2179378743886\\
-87.891	-84.2909035122105\\
-129.395	-124.094861361942\\
-86.67	-83.1199167992545\\
-74.463	-71.4129267869262\\
-72.021	-69.0709533610143\\
-86.67	-83.1199167992545\\
-113.525	-108.874911210746\\
-179.443	-172.092849085135\\
-151.367	-145.166867960687\\
-97.656	-93.655920098627\\
-64.697	-62.0469511614327\\
-76.904	-73.753941173761\\
-36.621	-35.1209700369851\\
-54.932	-52.6819345750162\\
-51.27	-49.1699334752253\\
-54.932	-52.6819345750162\\
-93.994	-90.1439189988362\\
-128.174	-122.923874648986\\
-146.484	-140.48388014794\\
-189.209	-181.458824710628\\
-161.133	-154.532843586181\\
-86.67	-83.1199167992545\\
-81.787	-78.4369289865078\\
-68.359	-65.5589522612235\\
-91.553	-87.8029046120013\\
-120.85	-115.899872449405\\
-161.133	-154.532843586181\\
-113.525	-108.874911210746\\
-80.566	-77.2659422735519\\
-92.773	-88.9729322858802\\
-124.512	-119.411873549196\\
-87.891	-84.2909035122105\\
-63.477	-60.8769234875537\\
-51.27	-49.1699334752253\\
-37.842	-36.291956749941\\
-34.18	-32.7799556501502\\
-29.297	-28.0969678374035\\
-47.607	-45.6569733363576\\
-64.697	-62.0469511614327\\
-73.242	-70.2419400739702\\
-56.152	-53.8519622488951\\
-52.49	-50.3399611491043\\
-26.855	-25.7549944114916\\
-15.869	-15.2189911121192\\
-23.193	-22.2429933117008\\
-41.504	-39.8039578497319\\
-62.256	-59.7059367745978\\
-76.904	-73.753941173761\\
-86.67	-83.1199167992545\\
-101.318	-97.1679211984178\\
-131.836	-126.435875748777\\
-181.885	-174.434822511047\\
-179.443	-172.092849085135\\
-194.092	-186.141812523375\\
-169.678	-162.727832498718\\
-93.994	-90.1439189988362\\
-86.67	-83.1199167992545\\
-85.449	-81.9489300862986\\
-113.525	-108.874911210746\\
-96.436	-92.485892424748\\
-70.801	-67.9009256871354\\
-56.152	-53.8519622488951\\
-46.387	-44.4869456624786\\
-76.904	-73.753941173761\\
-42.725	-40.9749445626878\\
-32.959	-31.6089689371943\\
-50.049	-47.9989467622694\\
-67.139	-64.3889245873445\\
-51.27	-49.1699334752253\\
-73.242	-70.2419400739702\\
-54.932	-52.6819345750162\\
-81.787	-78.4369289865078\\
-122.07	-117.069900123284\\
-96.436	-92.485892424748\\
-128.174	-122.923874648986\\
-177.002	-169.7518346983\\
-189.209	-181.458824710628\\
-148.926	-142.825853573852\\
-95.215	-91.3149057117921\\
-73.242	-70.2419400739702\\
-45.166	-43.3159589495227\\
-68.359	-65.5589522612235\\
-47.607	-45.6569733363576\\
-85.449	-81.9489300862986\\
-79.346	-76.0959145996729\\
-62.256	-59.7059367745978\\
-81.787	-78.4369289865078\\
-47.607	-45.6569733363576\\
-69.58	-66.7299389741794\\
-54.932	-52.6819345750162\\
-34.18	-32.7799556501502\\
-39.063	-37.462943462897\\
-32.959	-31.6089689371943\\
-37.842	-36.291956749941\\
-32.959	-31.6089689371943\\
-24.414	-23.4139800246567\\
-31.738	-30.4379822242384\\
-45.166	-43.3159589495227\\
-43.945	-42.1449722365668\\
-68.359	-65.5589522612235\\
-90.332	-86.6319178990453\\
-72.021	-69.0709533610143\\
-67.139	-64.3889245873445\\
-104.98	-100.679922298209\\
-125.732	-120.581901223075\\
-100.098	-95.9978935245388\\
-104.98	-100.679922298209\\
-51.27	-49.1699334752253\\
-32.959	-31.6089689371943\\
-62.256	-59.7059367745978\\
-91.553	-87.8029046120013\\
-96.436	-92.485892424748\\
-135.498	-129.947876848568\\
-120.85	-115.899872449405\\
-86.67	-83.1199167992545\\
-61.035	-58.5349500616419\\
-64.697	-62.0469511614327\\
-73.242	-70.2419400739702\\
-56.152	-53.8519622488951\\
-35.4	-33.9499833240292\\
-45.166	-43.3159589495227\\
-37.842	-36.291956749941\\
-31.738	-30.4379822242384\\
-25.635	-24.5849667376127\\
-45.166	-43.3159589495227\\
-62.256	-59.7059367745978\\
-68.359	-65.5589522612235\\
-50.049	-47.9989467622694\\
-81.787	-78.4369289865078\\
-115.967	-111.216884636658\\
-74.463	-71.4129267869262\\
-100.098	-95.9978935245388\\
-112.305	-107.704883536867\\
-83.008	-79.6079156994637\\
-48.828	-46.8279600493135\\
-62.256	-59.7059367745978\\
-74.463	-71.4129267869262\\
-45.166	-43.3159589495227\\
-48.828	-46.8279600493135\\
-41.504	-39.8039578497319\\
-24.414	-23.4139800246567\\
-40.283	-38.6329711367759\\
-54.932	-52.6819345750162\\
-48.828	-46.8279600493135\\
-59.814	-57.3639633486859\\
-69.58	-66.7299389741794\\
-50.049	-47.9989467622694\\
-30.518	-29.2679545503594\\
-24.414	-23.4139800246567\\
-31.738	-30.4379822242384\\
-36.621	-35.1209700369851\\
-43.945	-42.1449722365668\\
-86.67	-83.1199167992545\\
-89.111	-85.4609311860894\\
-76.904	-73.753941173761\\
-74.463	-71.4129267869262\\
-101.318	-97.1679211984178\\
-128.174	-122.923874648986\\
-137.939	-132.288891235403\\
-140.381	-134.630864661315\\
-103.76	-99.5098946243297\\
-108.643	-104.192882437076\\
-111.084	-106.533896823911\\
-112.305	-107.704883536867\\
-126.953	-121.75288793603\\
-169.678	-162.727832498718\\
-144.043	-138.142865761106\\
-131.836	-126.435875748777\\
-104.98	-100.679922298209\\
-101.318	-97.1679211984178\\
-96.436	-92.485892424748\\
-114.746	-110.045897923702\\
-72.021	-69.0709533610143\\
-83.008	-79.6079156994637\\
-102.539	-98.3389079113737\\
-119.629	-114.728885736449\\
-91.553	-87.8029046120013\\
-106.201	-101.850909011165\\
-85.449	-81.9489300862986\\
-75.684	-72.5839134998821\\
-50.049	-47.9989467622694\\
-74.463	-71.4129267869262\\
-122.07	-117.069900123284\\
-96.436	-92.485892424748\\
-56.152	-53.8519622488951\\
-69.58	-66.7299389741794\\
-41.504	-39.8039578497319\\
-35.4	-33.9499833240292\\
-50.049	-47.9989467622694\\
-31.738	-30.4379822242384\\
-35.4	-33.9499833240292\\
-42.725	-40.9749445626878\\
-28.076	-26.9259811244476\\
-43.945	-42.1449722365668\\
-40.283	-38.6329711367759\\
-24.414	-23.4139800246567\\
-39.063	-37.462943462897\\
-46.387	-44.4869456624786\\
-53.711	-51.5109478620602\\
-74.463	-71.4129267869262\\
-64.697	-62.0469511614327\\
-72.021	-69.0709533610143\\
-124.512	-119.411873549196\\
-89.111	-85.4609311860894\\
-76.904	-73.753941173761\\
-85.449	-81.9489300862986\\
-61.035	-58.5349500616419\\
-37.842	-36.291956749941\\
-65.918	-63.2179378743886\\
-57.373	-55.022948961851\\
-32.959	-31.6089689371943\\
-58.594	-56.193935674807\\
-54.932	-52.6819345750162\\
-67.139	-64.3889245873445\\
-45.166	-43.3159589495227\\
-28.076	-26.9259811244476\\
-23.193	-22.2429933117008\\
-26.855	-25.7549944114916\\
-48.828	-46.8279600493135\\
-59.814	-57.3639633486859\\
-76.904	-73.753941173761\\
-102.539	-98.3389079113737\\
-76.904	-73.753941173761\\
-102.539	-98.3389079113737\\
-122.07	-117.069900123284\\
-104.98	-100.679922298209\\
-98.877	-94.8269068115829\\
-162.354	-155.703830299137\\
-118.408	-113.557899023493\\
-142.822	-136.97187904815\\
-123.291	-118.24088683624\\
-145.264	-139.313852474062\\
-162.354	-155.703830299137\\
-96.436	-92.485892424748\\
-58.594	-56.193935674807\\
-48.828	-46.8279600493135\\
-74.463	-71.4129267869262\\
-93.994	-90.1439189988362\\
-95.215	-91.3149057117921\\
-65.918	-63.2179378743886\\
-36.621	-35.1209700369851\\
-46.387	-44.4869456624786\\
-92.773	-88.9729322858802\\
-125.732	-120.581901223075\\
-124.512	-119.411873549196\\
-74.463	-71.4129267869262\\
-57.373	-55.022948961851\\
-76.904	-73.753941173761\\
-122.07	-117.069900123284\\
-136.719	-131.118863561524\\
-139.16	-133.459877948359\\
-97.656	-93.655920098627\\
-137.939	-132.288891235403\\
-152.588	-146.337854673643\\
-142.822	-136.97187904815\\
-197.754	-189.653813623166\\
-169.678	-162.727832498718\\
-141.602	-135.801851374271\\
-164.795	-158.044844685971\\
-140.381	-134.630864661315\\
-78.125	-74.924927886717\\
-72.021	-69.0709533610143\\
-89.111	-85.4609311860894\\
-108.643	-104.192882437076\\
-111.084	-106.533896823911\\
-81.787	-78.4369289865078\\
-102.539	-98.3389079113737\\
-91.553	-87.8029046120013\\
-65.918	-63.2179378743886\\
-89.111	-85.4609311860894\\
-84.229	-80.7789024124196\\
-124.512	-119.411873549196\\
-128.174	-122.923874648986\\
-91.553	-87.8029046120013\\
-70.801	-67.9009256871354\\
-42.725	-40.9749445626878\\
-64.697	-62.0469511614327\\
-54.932	-52.6819345750162\\
-46.387	-44.4869456624786\\
-39.063	-37.462943462897\\
-65.918	-63.2179378743886\\
-86.67	-83.1199167992545\\
-95.215	-91.3149057117921\\
-57.373	-55.022948961851\\
-42.725	-40.9749445626878\\
-29.297	-28.0969678374035\\
-37.842	-36.291956749941\\
-63.477	-60.8769234875537\\
-65.918	-63.2179378743886\\
-64.697	-62.0469511614327\\
-102.539	-98.3389079113737\\
-122.07	-117.069900123284\\
-134.277	-128.776890135612\\
-135.498	-129.947876848568\\
-157.471	-151.02084248639\\
-97.656	-93.655920098627\\
-80.566	-77.2659422735519\\
-61.035	-58.5349500616419\\
-67.139	-64.3889245873445\\
-90.332	-86.6319178990453\\
-75.684	-72.5839134998821\\
-51.27	-49.1699334752253\\
-46.387	-44.4869456624786\\
-83.008	-79.6079156994637\\
-102.539	-98.3389079113737\\
-90.332	-86.6319178990453\\
-147.705	-141.654866860896\\
-161.133	-154.532843586181\\
-112.305	-107.704883536867\\
-79.346	-76.0959145996729\\
-54.932	-52.6819345750162\\
-50.049	-47.9989467622694\\
-90.332	-86.6319178990453\\
-76.904	-73.753941173761\\
-86.67	-83.1199167992545\\
-98.877	-94.8269068115829\\
-107.422	-103.02189572412\\
-103.76	-99.5098946243297\\
-117.188	-112.387871349614\\
-81.787	-78.4369289865078\\
-89.111	-85.4609311860894\\
-65.918	-63.2179378743886\\
-58.594	-56.193935674807\\
-90.332	-86.6319178990453\\
-125.732	-120.581901223075\\
-135.498	-129.947876848568\\
-80.566	-77.2659422735519\\
-155.029	-148.678869060478\\
-145.264	-139.313852474062\\
-96.436	-92.485892424748\\
-56.152	-53.8519622488951\\
-83.008	-79.6079156994637\\
-75.684	-72.5839134998821\\
-70.801	-67.9009256871354\\
-85.449	-81.9489300862986\\
-58.594	-56.193935674807\\
-79.346	-76.0959145996729\\
-145.264	-139.313852474062\\
-151.367	-145.166867960687\\
-98.877	-94.8269068115829\\
-108.643	-104.192882437076\\
-122.07	-117.069900123284\\
-123.291	-118.24088683624\\
-90.332	-86.6319178990453\\
-87.891	-84.2909035122105\\
-61.035	-58.5349500616419\\
-89.111	-85.4609311860894\\
-95.215	-91.3149057117921\\
-146.484	-140.48388014794\\
-166.016	-159.215831398927\\
-228.271	-218.920809134448\\
-163.574	-156.873857973016\\
-131.836	-126.435875748777\\
-103.76	-99.5098946243297\\
-109.863	-105.362910110955\\
-85.449	-81.9489300862986\\
-59.814	-57.3639633486859\\
-57.373	-55.022948961851\\
-59.814	-57.3639633486859\\
-46.387	-44.4869456624786\\
-32.959	-31.6089689371943\\
-51.27	-49.1699334752253\\
-67.139	-64.3889245873445\\
-47.607	-45.6569733363576\\
-34.18	-32.7799556501502\\
-32.959	-31.6089689371943\\
-76.904	-73.753941173761\\
-109.863	-105.362910110955\\
-52.49	-50.3399611491043\\
-62.256	-59.7059367745978\\
-68.359	-65.5589522612235\\
-61.035	-58.5349500616419\\
-76.904	-73.753941173761\\
-67.139	-64.3889245873445\\
-47.607	-45.6569733363576\\
-34.18	-32.7799556501502\\
-29.297	-28.0969678374035\\
-39.063	-37.462943462897\\
-73.242	-70.2419400739702\\
-87.891	-84.2909035122105\\
-62.256	-59.7059367745978\\
-43.945	-42.1449722365668\\
-29.297	-28.0969678374035\\
-41.504	-39.8039578497319\\
-64.697	-62.0469511614327\\
-84.229	-80.7789024124196\\
-89.111	-85.4609311860894\\
-63.477	-60.8769234875537\\
-59.814	-57.3639633486859\\
-67.139	-64.3889245873445\\
-91.553	-87.8029046120013\\
-126.953	-121.75288793603\\
-146.484	-140.48388014794\\
-111.084	-106.533896823911\\
-69.58	-66.7299389741794\\
-75.684	-72.5839134998821\\
-69.58	-66.7299389741794\\
-72.021	-69.0709533610143\\
-50.049	-47.9989467622694\\
-59.814	-57.3639633486859\\
-67.139	-64.3889245873445\\
-63.477	-60.8769234875537\\
-41.504	-39.8039578497319\\
-29.297	-28.0969678374035\\
-41.504	-39.8039578497319\\
-63.477	-60.8769234875537\\
-93.994	-90.1439189988362\\
-100.098	-95.9978935245388\\
-122.07	-117.069900123284\\
-109.863	-105.362910110955\\
-123.291	-118.24088683624\\
-163.574	-156.873857973016\\
-217.285	-208.384805835076\\
-179.443	-172.092849085135\\
-119.629	-114.728885736449\\
-130.615	-125.264889035821\\
-153.809	-147.508841386599\\
-197.754	-189.653813623166\\
-128.174	-122.923874648986\\
-62.256	-59.7059367745978\\
-43.945	-42.1449722365668\\
-45.166	-43.3159589495227\\
-32.959	-31.6089689371943\\
-21.973	-21.0729656378219\\
-28.076	-26.9259811244476\\
-42.725	-40.9749445626878\\
-61.035	-58.5349500616419\\
-59.814	-57.3639633486859\\
-47.607	-45.6569733363576\\
-45.166	-43.3159589495227\\
-56.152	-53.8519622488951\\
-69.58	-66.7299389741794\\
-73.242	-70.2419400739702\\
-69.58	-66.7299389741794\\
-48.828	-46.8279600493135\\
-35.4	-33.9499833240292\\
-34.18	-32.7799556501502\\
-50.049	-47.9989467622694\\
-63.477	-60.8769234875537\\
-46.387	-44.4869456624786\\
-32.959	-31.6089689371943\\
-52.49	-50.3399611491043\\
-40.283	-38.6329711367759\\
-43.945	-42.1449722365668\\
-56.152	-53.8519622488951\\
-79.346	-76.0959145996729\\
-86.67	-83.1199167992545\\
-59.814	-57.3639633486859\\
-90.332	-86.6319178990453\\
-87.891	-84.2909035122105\\
-69.58	-66.7299389741794\\
-61.035	-58.5349500616419\\
-97.656	-93.655920098627\\
-123.291	-118.24088683624\\
-119.629	-114.728885736449\\
-133.057	-127.606862461733\\
-102.539	-98.3389079113737\\
-92.773	-88.9729322858802\\
-87.891	-84.2909035122105\\
-118.408	-113.557899023493\\
-86.67	-83.1199167992545\\
-120.85	-115.899872449405\\
-115.967	-111.216884636658\\
-118.408	-113.557899023493\\
-203.857	-195.506829109792\\
-161.133	-154.532843586181\\
-85.449	-81.9489300862986\\
-58.594	-56.193935674807\\
-62.256	-59.7059367745978\\
-79.346	-76.0959145996729\\
-101.318	-97.1679211984178\\
-114.746	-110.045897923702\\
-128.174	-122.923874648986\\
-131.836	-126.435875748777\\
-86.67	-83.1199167992545\\
-76.904	-73.753941173761\\
-89.111	-85.4609311860894\\
-56.152	-53.8519622488951\\
-40.283	-38.6329711367759\\
-70.801	-67.9009256871354\\
-93.994	-90.1439189988362\\
-115.967	-111.216884636658\\
-107.422	-103.02189572412\\
-74.463	-71.4129267869262\\
-61.035	-58.5349500616419\\
-90.332	-86.6319178990453\\
-118.408	-113.557899023493\\
-153.809	-147.508841386599\\
-169.678	-162.727832498718\\
-194.092	-186.141812523375\\
-157.471	-151.02084248639\\
-141.602	-135.801851374271\\
-84.229	-80.7789024124196\\
-62.256	-59.7059367745978\\
-104.98	-100.679922298209\\
-139.16	-133.459877948359\\
-81.787	-78.4369289865078\\
-125.732	-120.581901223075\\
-172.119	-165.068846885553\\
-206.299	-197.848802535703\\
-129.395	-124.094861361942\\
-74.463	-71.4129267869262\\
-42.725	-40.9749445626878\\
-64.697	-62.0469511614327\\
-59.814	-57.3639633486859\\
-62.256	-59.7059367745978\\
-37.842	-36.291956749941\\
-52.49	-50.3399611491043\\
-41.504	-39.8039578497319\\
-52.49	-50.3399611491043\\
-43.945	-42.1449722365668\\
-54.932	-52.6819345750162\\
-85.449	-81.9489300862986\\
-79.346	-76.0959145996729\\
-112.305	-107.704883536867\\
-133.057	-127.606862461733\\
-74.463	-71.4129267869262\\
-76.904	-73.753941173761\\
-91.553	-87.8029046120013\\
-62.256	-59.7059367745978\\
-56.152	-53.8519622488951\\
-42.725	-40.9749445626878\\
-63.477	-60.8769234875537\\
-59.814	-57.3639633486859\\
-34.18	-32.7799556501502\\
-24.414	-23.4139800246567\\
-41.504	-39.8039578497319\\
-61.035	-58.5349500616419\\
-65.918	-63.2179378743886\\
-41.504	-39.8039578497319\\
-51.27	-49.1699334752253\\
-73.242	-70.2419400739702\\
-93.994	-90.1439189988362\\
-65.918	-63.2179378743886\\
-62.256	-59.7059367745978\\
-74.463	-71.4129267869262\\
-87.891	-84.2909035122105\\
-80.566	-77.2659422735519\\
-92.773	-88.9729322858802\\
-86.67	-83.1199167992545\\
-62.256	-59.7059367745978\\
-48.828	-46.8279600493135\\
-62.256	-59.7059367745978\\
-53.711	-51.5109478620602\\
-64.697	-62.0469511614327\\
-87.891	-84.2909035122105\\
-139.16	-133.459877948359\\
-98.877	-94.8269068115829\\
-95.215	-91.3149057117921\\
-140.381	-134.630864661315\\
-108.643	-104.192882437076\\
-67.139	-64.3889245873445\\
-48.828	-46.8279600493135\\
-40.283	-38.6329711367759\\
-42.725	-40.9749445626878\\
-35.4	-33.9499833240292\\
-36.621	-35.1209700369851\\
-67.139	-64.3889245873445\\
-35.4	-33.9499833240292\\
-34.18	-32.7799556501502\\
-47.607	-45.6569733363576\\
-65.918	-63.2179378743886\\
-50.049	-47.9989467622694\\
-37.842	-36.291956749941\\
-23.193	-22.2429933117008\\
-39.063	-37.462943462897\\
-72.021	-69.0709533610143\\
-96.436	-92.485892424748\\
-79.346	-76.0959145996729\\
-63.477	-60.8769234875537\\
-72.021	-69.0709533610143\\
-80.566	-77.2659422735519\\
-98.877	-94.8269068115829\\
-117.188	-112.387871349614\\
-114.746	-110.045897923702\\
-95.215	-91.3149057117921\\
-64.697	-62.0469511614327\\
-52.49	-50.3399611491043\\
-56.152	-53.8519622488951\\
-76.904	-73.753941173761\\
-47.607	-45.6569733363576\\
-43.945	-42.1449722365668\\
-80.566	-77.2659422735519\\
-96.436	-92.485892424748\\
-91.553	-87.8029046120013\\
-104.98	-100.679922298209\\
-137.939	-132.288891235403\\
-74.463	-71.4129267869262\\
-139.16	-133.459877948359\\
-129.395	-124.094861361942\\
-120.85	-115.899872449405\\
-128.174	-122.923874648986\\
-125.732	-120.581901223075\\
-212.402	-203.701818022329\\
-198.975	-190.824800336122\\
-134.277	-128.776890135612\\
-159.912	-153.361856873225\\
-194.092	-186.141812523375\\
-190.43	-182.629811423584\\
-212.402	-203.701818022329\\
-194.092	-186.141812523375\\
-253.906	-243.505775872061\\
-195.313	-187.312799236331\\
-135.498	-129.947876848568\\
-81.787	-78.4369289865078\\
-84.229	-80.7789024124196\\
-68.359	-65.5589522612235\\
-57.373	-55.022948961851\\
-85.449	-81.9489300862986\\
-120.85	-115.899872449405\\
-76.904	-73.753941173761\\
-67.139	-64.3889245873445\\
-65.918	-63.2179378743886\\
-46.387	-44.4869456624786\\
-56.152	-53.8519622488951\\
-29.297	-28.0969678374035\\
-39.063	-37.462943462897\\
-29.297	-28.0969678374035\\
-46.387	-44.4869456624786\\
-32.959	-31.6089689371943\\
-30.518	-29.2679545503594\\
-19.531	-18.73099221191\\
-18.311	-17.560964538031\\
-34.18	-32.7799556501502\\
-25.635	-24.5849667376127\\
-28.076	-26.9259811244476\\
-57.373	-55.022948961851\\
-78.125	-74.924927886717\\
-80.566	-77.2659422735519\\
-58.594	-56.193935674807\\
-74.463	-71.4129267869262\\
-114.746	-110.045897923702\\
-53.711	-51.5109478620602\\
-37.842	-36.291956749941\\
-28.076	-26.9259811244476\\
-20.752	-19.9019789248659\\
-34.18	-32.7799556501502\\
-56.152	-53.8519622488951\\
-84.229	-80.7789024124196\\
-50.049	-47.9989467622694\\
-85.449	-81.9489300862986\\
-118.408	-113.557899023493\\
-120.85	-115.899872449405\\
-89.111	-85.4609311860894\\
-57.373	-55.022948961851\\
-72.021	-69.0709533610143\\
-96.436	-92.485892424748\\
-124.512	-119.411873549196\\
-68.359	-65.5589522612235\\
-37.842	-36.291956749941\\
-57.373	-55.022948961851\\
-47.607	-45.6569733363576\\
-23.193	-22.2429933117008\\
-18.311	-17.560964538031\\
-39.063	-37.462943462897\\
-85.449	-81.9489300862986\\
-63.477	-60.8769234875537\\
-41.504	-39.8039578497319\\
-43.945	-42.1449722365668\\
-15.869	-15.2189911121192\\
-30.518	-29.2679545503594\\
-25.635	-24.5849667376127\\
-42.725	-40.9749445626878\\
-58.594	-56.193935674807\\
-89.111	-85.4609311860894\\
-93.994	-90.1439189988362\\
-100.098	-95.9978935245388\\
-104.98	-100.679922298209\\
-102.539	-98.3389079113737\\
-100.098	-95.9978935245388\\
-86.67	-83.1199167992545\\
-104.98	-100.679922298209\\
-157.471	-151.02084248639\\
-144.043	-138.142865761106\\
-74.463	-71.4129267869262\\
-40.283	-38.6329711367759\\
-86.67	-83.1199167992545\\
-106.201	-101.850909011165\\
-96.436	-92.485892424748\\
-56.152	-53.8519622488951\\
-57.373	-55.022948961851\\
-97.656	-93.655920098627\\
-146.484	-140.48388014794\\
-150.146	-143.995881247731\\
-169.678	-162.727832498718\\
-164.795	-158.044844685971\\
-120.85	-115.899872449405\\
-123.291	-118.24088683624\\
-58.594	-56.193935674807\\
-56.152	-53.8519622488951\\
-72.021	-69.0709533610143\\
-87.891	-84.2909035122105\\
-91.553	-87.8029046120013\\
-96.436	-92.485892424748\\
-63.477	-60.8769234875537\\
-86.67	-83.1199167992545\\
-69.58	-66.7299389741794\\
-41.504	-39.8039578497319\\
-29.297	-28.0969678374035\\
-24.414	-23.4139800246567\\
-37.842	-36.291956749941\\
-30.518	-29.2679545503594\\
-18.311	-17.560964538031\\
-39.063	-37.462943462897\\
-74.463	-71.4129267869262\\
-48.828	-46.8279600493135\\
-45.166	-43.3159589495227\\
-61.035	-58.5349500616419\\
-100.098	-95.9978935245388\\
-68.359	-65.5589522612235\\
-37.842	-36.291956749941\\
-24.414	-23.4139800246567\\
-53.711	-51.5109478620602\\
-108.643	-104.192882437076\\
-136.719	-131.118863561524\\
-189.209	-181.458824710628\\
-223.389	-214.238780360778\\
-220.947	-211.896806934867\\
-142.822	-136.97187904815\\
-147.705	-141.654866860896\\
-191.65	-183.799839097463\\
-159.912	-153.361856873225\\
-147.705	-141.654866860896\\
-168.457	-161.556845785762\\
-183.105	-175.604850184926\\
-185.547	-177.946823610837\\
-253.906	-243.505775872061\\
-168.457	-161.556845785762\\
-95.215	-91.3149057117921\\
-56.152	-53.8519622488951\\
-45.166	-43.3159589495227\\
-50.049	-47.9989467622694\\
-48.828	-46.8279600493135\\
-86.67	-83.1199167992545\\
-65.918	-63.2179378743886\\
-73.242	-70.2419400739702\\
-92.773	-88.9729322858802\\
-53.711	-51.5109478620602\\
-64.697	-62.0469511614327\\
-84.229	-80.7789024124196\\
-98.877	-94.8269068115829\\
-56.152	-53.8519622488951\\
-39.063	-37.462943462897\\
-50.049	-47.9989467622694\\
-48.828	-46.8279600493135\\
-113.525	-108.874911210746\\
-155.029	-148.678869060478\\
-145.264	-139.313852474062\\
-168.457	-161.556845785762\\
-189.209	-181.458824710628\\
-246.582	-236.481773672479\\
-208.74	-200.189816922538\\
-135.498	-129.947876848568\\
-89.111	-85.4609311860894\\
-51.27	-49.1699334752253\\
-58.594	-56.193935674807\\
-76.904	-73.753941173761\\
-53.711	-51.5109478620602\\
-50.049	-47.9989467622694\\
-53.711	-51.5109478620602\\
-69.58	-66.7299389741794\\
-76.904	-73.753941173761\\
-95.215	-91.3149057117921\\
-68.359	-65.5589522612235\\
-32.959	-31.6089689371943\\
-25.635	-24.5849667376127\\
-23.193	-22.2429933117008\\
-25.635	-24.5849667376127\\
-30.518	-29.2679545503594\\
-53.711	-51.5109478620602\\
-52.49	-50.3399611491043\\
-65.918	-63.2179378743886\\
-95.215	-91.3149057117921\\
-68.359	-65.5589522612235\\
-113.525	-108.874911210746\\
-164.795	-158.044844685971\\
-129.395	-124.094861361942\\
-115.967	-111.216884636658\\
-140.381	-134.630864661315\\
-162.354	-155.703830299137\\
-108.643	-104.192882437076\\
-93.994	-90.1439189988362\\
-59.814	-57.3639633486859\\
-65.918	-63.2179378743886\\
-131.836	-126.435875748777\\
-214.844	-206.043791448241\\
-256.348	-245.847749297973\\
-202.637	-194.336801435913\\
-168.457	-161.556845785762\\
-123.291	-118.24088683624\\
-133.057	-127.606862461733\\
-175.781	-168.580847985344\\
-93.994	-90.1439189988362\\
-83.008	-79.6079156994637\\
-63.477	-60.8769234875537\\
-115.967	-111.216884636658\\
-112.305	-107.704883536867\\
-61.035	-58.5349500616419\\
-62.256	-59.7059367745978\\
-46.387	-44.4869456624786\\
-84.229	-80.7789024124196\\
-119.629	-114.728885736449\\
-126.953	-121.75288793603\\
-95.215	-91.3149057117921\\
-69.58	-66.7299389741794\\
-80.566	-77.2659422735519\\
-61.035	-58.5349500616419\\
-46.387	-44.4869456624786\\
-36.621	-35.1209700369851\\
-32.959	-31.6089689371943\\
-29.297	-28.0969678374035\\
-34.18	-32.7799556501502\\
-59.814	-57.3639633486859\\
-84.229	-80.7789024124196\\
-83.008	-79.6079156994637\\
-52.49	-50.3399611491043\\
-47.607	-45.6569733363576\\
-40.283	-38.6329711367759\\
-52.49	-50.3399611491043\\
-68.359	-65.5589522612235\\
-73.242	-70.2419400739702\\
-69.58	-66.7299389741794\\
-40.283	-38.6329711367759\\
-83.008	-79.6079156994637\\
-120.85	-115.899872449405\\
-84.229	-80.7789024124196\\
-35.4	-33.9499833240292\\
-40.283	-38.6329711367759\\
-67.139	-64.3889245873445\\
-58.594	-56.193935674807\\
-37.842	-36.291956749941\\
-68.359	-65.5589522612235\\
-97.656	-93.655920098627\\
-59.814	-57.3639633486859\\
-24.414	-23.4139800246567\\
-37.842	-36.291956749941\\
-81.787	-78.4369289865078\\
-96.436	-92.485892424748\\
-59.814	-57.3639633486859\\
-53.711	-51.5109478620602\\
-97.656	-93.655920098627\\
-125.732	-120.581901223075\\
-108.643	-104.192882437076\\
-152.588	-146.337854673643\\
-184.326	-176.775836897882\\
-115.967	-111.216884636658\\
-95.215	-91.3149057117921\\
-139.16	-133.459877948359\\
-93.994	-90.1439189988362\\
-126.953	-121.75288793603\\
-152.588	-146.337854673643\\
-119.629	-114.728885736449\\
-57.373	-55.022948961851\\
-101.318	-97.1679211984178\\
-172.119	-165.068846885553\\
-195.313	-187.312799236331\\
-142.822	-136.97187904815\\
-120.85	-115.899872449405\\
-151.367	-145.166867960687\\
-118.408	-113.557899023493\\
-76.904	-73.753941173761\\
-73.242	-70.2419400739702\\
-91.553	-87.8029046120013\\
-95.215	-91.3149057117921\\
-93.994	-90.1439189988362\\
-104.98	-100.679922298209\\
-72.021	-69.0709533610143\\
-79.346	-76.0959145996729\\
-109.863	-105.362910110955\\
-64.697	-62.0469511614327\\
-53.711	-51.5109478620602\\
-43.945	-42.1449722365668\\
-35.4	-33.9499833240292\\
-42.725	-40.9749445626878\\
-73.242	-70.2419400739702\\
-65.918	-63.2179378743886\\
-93.994	-90.1439189988362\\
-114.746	-110.045897923702\\
-139.16	-133.459877948359\\
-102.539	-98.3389079113737\\
-92.773	-88.9729322858802\\
-43.945	-42.1449722365668\\
-56.152	-53.8519622488951\\
-107.422	-103.02189572412\\
-73.242	-70.2419400739702\\
-42.725	-40.9749445626878\\
-78.125	-74.924927886717\\
-115.967	-111.216884636658\\
-120.85	-115.899872449405\\
-157.471	-151.02084248639\\
-205.078	-196.677815822747\\
-136.719	-131.118863561524\\
-118.408	-113.557899023493\\
-142.822	-136.97187904815\\
-92.773	-88.9729322858802\\
-76.904	-73.753941173761\\
-85.449	-81.9489300862986\\
-111.084	-106.533896823911\\
-118.408	-113.557899023493\\
-119.629	-114.728885736449\\
-67.139	-64.3889245873445\\
-59.814	-57.3639633486859\\
-52.49	-50.3399611491043\\
-73.242	-70.2419400739702\\
-65.918	-63.2179378743886\\
-52.49	-50.3399611491043\\
-104.98	-100.679922298209\\
-111.084	-106.533896823911\\
-78.125	-74.924927886717\\
-64.697	-62.0469511614327\\
-96.436	-92.485892424748\\
-53.711	-51.5109478620602\\
-32.959	-31.6089689371943\\
-41.504	-39.8039578497319\\
-57.373	-55.022948961851\\
-48.828	-46.8279600493135\\
-35.4	-33.9499833240292\\
-21.973	-21.0729656378219\\
-40.283	-38.6329711367759\\
-54.932	-52.6819345750162\\
-86.67	-83.1199167992545\\
-96.436	-92.485892424748\\
-126.953	-121.75288793603\\
-142.822	-136.97187904815\\
-81.787	-78.4369289865078\\
-62.256	-59.7059367745978\\
-128.174	-122.923874648986\\
-181.885	-174.434822511047\\
-139.16	-133.459877948359\\
-130.615	-125.264889035821\\
-196.533	-188.48282691021\\
-155.029	-148.678869060478\\
-128.174	-122.923874648986\\
-93.994	-90.1439189988362\\
-98.877	-94.8269068115829\\
-130.615	-125.264889035821\\
-85.449	-81.9489300862986\\
-74.463	-71.4129267869262\\
-128.174	-122.923874648986\\
-158.691	-152.190870160269\\
-169.678	-162.727832498718\\
-73.242	-70.2419400739702\\
-39.063	-37.462943462897\\
-93.994	-90.1439189988362\\
-73.242	-70.2419400739702\\
-35.4	-33.9499833240292\\
-29.297	-28.0969678374035\\
-50.049	-47.9989467622694\\
-67.139	-64.3889245873445\\
-109.863	-105.362910110955\\
-76.904	-73.753941173761\\
-61.035	-58.5349500616419\\
-36.621	-35.1209700369851\\
-26.855	-25.7549944114916\\
-36.621	-35.1209700369851\\
-30.518	-29.2679545503594\\
-40.283	-38.6329711367759\\
-68.359	-65.5589522612235\\
-61.035	-58.5349500616419\\
-32.959	-31.6089689371943\\
-30.518	-29.2679545503594\\
-39.063	-37.462943462897\\
-47.607	-45.6569733363576\\
-29.297	-28.0969678374035\\
-47.607	-45.6569733363576\\
-32.959	-31.6089689371943\\
-24.414	-23.4139800246567\\
-56.152	-53.8519622488951\\
-78.125	-74.924927886717\\
-115.967	-111.216884636658\\
-156.25	-149.849855773434\\
-137.939	-132.288891235403\\
-67.139	-64.3889245873445\\
-51.27	-49.1699334752253\\
-29.297	-28.0969678374035\\
-30.518	-29.2679545503594\\
-57.373	-55.022948961851\\
-62.256	-59.7059367745978\\
-57.373	-55.022948961851\\
-65.918	-63.2179378743886\\
-76.904	-73.753941173761\\
-80.566	-77.2659422735519\\
-83.008	-79.6079156994637\\
-103.76	-99.5098946243297\\
-100.098	-95.9978935245388\\
-145.264	-139.313852474062\\
-76.904	-73.753941173761\\
-37.842	-36.291956749941\\
-36.621	-35.1209700369851\\
-67.139	-64.3889245873445\\
-50.049	-47.9989467622694\\
-47.607	-45.6569733363576\\
-23.193	-22.2429933117008\\
-41.504	-39.8039578497319\\
-25.635	-24.5849667376127\\
-15.869	-15.2189911121192\\
-25.635	-24.5849667376127\\
-47.607	-45.6569733363576\\
-58.594	-56.193935674807\\
-84.229	-80.7789024124196\\
-118.408	-113.557899023493\\
-84.229	-80.7789024124196\\
-35.4	-33.9499833240292\\
-65.918	-63.2179378743886\\
-74.463	-71.4129267869262\\
-39.063	-37.462943462897\\
-51.27	-49.1699334752253\\
-92.773	-88.9729322858802\\
-81.787	-78.4369289865078\\
-137.939	-132.288891235403\\
-157.471	-151.02084248639\\
-107.422	-103.02189572412\\
-83.008	-79.6079156994637\\
};
\addlegendentry{data2}

\end{axis}

\begin{axis}[%
width=4.927cm,
height=3.484cm,
at={(6.484cm,0cm)},
scale only axis,
xmin=-158.691,
xmax=0,
xlabel style={font=\color{white!15!black}},
xlabel={y(t-1)},
ymin=-158.691,
ymax=-9.43152579599367,
ylabel style={font=\color{white!15!black}},
ylabel={y(t)},
axis background/.style={fill=white},
title={C10, R = 0.7824},
axis x line*=bottom,
axis y line*=left,
legend style={legend cell align=left, align=left, draw=white!15!black}
]
\addplot[only marks, mark=*, mark options={}, mark size=1.5000pt, color=mycolor1, fill=mycolor1] table[row sep=crcr]{%
x	y\\
-54.932	-63.477\\
-63.477	-75.684\\
-75.684	-62.256\\
-62.256	-59.814\\
-59.814	-80.566\\
-80.566	-76.904\\
-76.904	-90.332\\
-90.332	-68.359\\
-68.359	-40.283\\
-40.283	-48.828\\
-48.828	-46.387\\
-46.387	-54.932\\
-54.932	-45.166\\
-45.166	-21.973\\
-21.973	-17.09\\
-17.09	-18.311\\
-18.311	-40.283\\
-40.283	-62.256\\
-62.256	-67.139\\
-67.139	-69.58\\
-69.58	-52.49\\
-52.49	-34.18\\
-34.18	-64.697\\
-64.697	-52.49\\
-52.49	-39.063\\
-39.063	-61.035\\
-61.035	-53.711\\
-53.711	-45.166\\
-45.166	-73.242\\
-73.242	-67.139\\
-67.139	-52.49\\
-52.49	-75.684\\
-75.684	-74.463\\
-74.463	-58.594\\
-58.594	-50.049\\
-50.049	-41.504\\
-41.504	-47.607\\
-47.607	-58.594\\
-58.594	-54.932\\
-54.932	-59.814\\
-59.814	-59.814\\
-59.814	-65.918\\
-65.918	-85.449\\
-85.449	-76.904\\
-76.904	-54.932\\
-54.932	-50.049\\
-50.049	-35.4\\
-35.4	-34.18\\
-34.18	-45.166\\
-45.166	-34.18\\
-34.18	-36.621\\
-36.621	-51.27\\
-51.27	-54.932\\
-54.932	-51.27\\
-51.27	-78.125\\
-78.125	-62.256\\
-62.256	-36.621\\
-36.621	-30.518\\
-30.518	-40.283\\
-40.283	-46.387\\
-46.387	-29.297\\
-29.297	-29.297\\
-29.297	-36.621\\
-36.621	-31.738\\
-31.738	-26.855\\
-26.855	-35.4\\
-35.4	-40.283\\
-40.283	-58.594\\
-58.594	-56.152\\
-56.152	-65.918\\
-65.918	-61.035\\
-61.035	-70.801\\
-70.801	-57.373\\
-57.373	-58.594\\
-58.594	-51.27\\
-51.27	-50.049\\
-50.049	-48.828\\
-48.828	-80.566\\
-80.566	-111.084\\
-111.084	-114.746\\
-114.746	-112.305\\
-112.305	-74.463\\
-74.463	-100.098\\
-100.098	-125.732\\
-125.732	-128.174\\
-128.174	-96.436\\
-96.436	-124.512\\
-124.512	-158.691\\
-158.691	-115.967\\
-115.967	-97.656\\
-97.656	-68.359\\
-68.359	-53.711\\
-53.711	-47.607\\
-47.607	-56.152\\
-56.152	-45.166\\
-45.166	-30.518\\
-30.518	-30.518\\
-30.518	-36.621\\
-36.621	-50.049\\
-50.049	-48.828\\
-48.828	-47.607\\
-47.607	-62.256\\
-62.256	-64.697\\
-64.697	-85.449\\
-85.449	-72.021\\
-72.021	-85.449\\
-85.449	-63.477\\
-63.477	-54.932\\
-54.932	-64.697\\
-64.697	-48.828\\
-48.828	-43.945\\
-43.945	-53.711\\
-53.711	-54.932\\
-54.932	-40.283\\
-40.283	-43.945\\
-43.945	-32.959\\
-32.959	-36.621\\
-36.621	-40.283\\
-40.283	-30.518\\
-30.518	-34.18\\
-34.18	-43.945\\
-43.945	-63.477\\
-63.477	-65.918\\
-65.918	-72.021\\
-72.021	-45.166\\
-45.166	-56.152\\
-56.152	-89.111\\
-89.111	-70.801\\
-70.801	-81.787\\
-81.787	-80.566\\
-80.566	-50.049\\
-50.049	-36.621\\
-36.621	-47.607\\
-47.607	-52.49\\
-52.49	-81.787\\
-81.787	-103.76\\
-103.76	-102.539\\
-102.539	-76.904\\
-76.904	-80.566\\
-80.566	-72.021\\
-72.021	-69.58\\
-69.58	-68.359\\
-68.359	-51.27\\
-51.27	-45.166\\
-45.166	-39.063\\
-39.063	-37.842\\
-37.842	-41.504\\
-41.504	-56.152\\
-56.152	-84.229\\
-84.229	-70.801\\
-70.801	-50.049\\
-50.049	-42.725\\
-42.725	-40.283\\
-40.283	-28.076\\
-28.076	-23.193\\
-23.193	-24.414\\
-24.414	-41.504\\
-41.504	-32.959\\
-32.959	-34.18\\
-34.18	-37.842\\
-37.842	-35.4\\
-35.4	-26.855\\
-26.855	-21.973\\
-21.973	-18.311\\
-18.311	-25.635\\
-25.635	-51.27\\
-51.27	-73.242\\
-73.242	-78.125\\
-78.125	-54.932\\
-54.932	-39.063\\
-39.063	-31.738\\
-31.738	-24.414\\
-24.414	-41.504\\
-41.504	-42.725\\
-42.725	-54.932\\
-54.932	-73.242\\
-73.242	-108.643\\
-108.643	-125.732\\
-125.732	-103.76\\
-103.76	-101.318\\
-101.318	-67.139\\
-67.139	-75.684\\
-75.684	-79.346\\
-79.346	-86.67\\
-86.67	-69.58\\
-69.58	-62.256\\
-62.256	-61.035\\
-61.035	-56.152\\
-56.152	-67.139\\
-67.139	-52.49\\
-52.49	-76.904\\
-76.904	-97.656\\
-97.656	-72.021\\
-72.021	-45.166\\
-45.166	-47.607\\
-47.607	-78.125\\
-78.125	-93.994\\
-93.994	-112.305\\
-112.305	-119.629\\
-119.629	-119.629\\
-119.629	-91.553\\
-91.553	-84.229\\
-84.229	-96.436\\
-96.436	-111.084\\
-111.084	-74.463\\
-74.463	-46.387\\
-46.387	-62.256\\
-62.256	-54.932\\
-54.932	-34.18\\
-34.18	-47.607\\
-47.607	-53.711\\
-53.711	-42.725\\
-42.725	-34.18\\
-34.18	-43.945\\
-43.945	-40.283\\
-40.283	-52.49\\
-52.49	-65.918\\
-65.918	-62.256\\
-62.256	-45.166\\
-45.166	-45.166\\
-45.166	-67.139\\
-67.139	-104.98\\
-104.98	-79.346\\
-79.346	-51.27\\
-51.27	-40.283\\
-40.283	-47.607\\
-47.607	-40.283\\
-40.283	-31.738\\
-31.738	-35.4\\
-35.4	-42.725\\
-42.725	-51.27\\
-51.27	-57.373\\
-57.373	-42.725\\
-42.725	-63.477\\
-63.477	-75.684\\
-75.684	-70.801\\
-70.801	-57.373\\
-57.373	-62.256\\
-62.256	-81.787\\
-81.787	-52.49\\
-52.49	-28.076\\
-28.076	-36.621\\
-36.621	-42.725\\
-42.725	-51.27\\
-51.27	-46.387\\
-46.387	-36.621\\
-36.621	-52.49\\
-52.49	-52.49\\
-52.49	-37.842\\
-37.842	-47.607\\
-47.607	-41.504\\
-41.504	-37.842\\
-37.842	-45.166\\
-45.166	-29.297\\
-29.297	-31.738\\
-31.738	-29.297\\
-29.297	-26.855\\
-26.855	-47.607\\
-47.607	-53.711\\
-53.711	-69.58\\
-69.58	-89.111\\
-89.111	-62.256\\
-62.256	-36.621\\
-36.621	-31.738\\
-31.738	-29.297\\
-29.297	-40.283\\
-40.283	-31.738\\
-31.738	-30.518\\
-30.518	-39.063\\
-39.063	-61.035\\
-61.035	-53.711\\
-53.711	-79.346\\
-79.346	-54.932\\
-54.932	-48.828\\
-48.828	-32.959\\
-32.959	-29.297\\
-29.297	-23.193\\
-23.193	-23.193\\
-23.193	-25.635\\
-25.635	-40.283\\
-40.283	-46.387\\
-46.387	-48.828\\
-48.828	-51.27\\
-51.27	-79.346\\
-79.346	-73.242\\
-73.242	-53.711\\
-53.711	-63.477\\
-63.477	-69.58\\
-69.58	-85.449\\
-85.449	-72.021\\
-72.021	-47.607\\
-47.607	-28.076\\
-28.076	-21.973\\
-21.973	-21.973\\
-21.973	-26.855\\
-26.855	-26.855\\
-26.855	-25.635\\
-25.635	-32.959\\
-32.959	-47.607\\
-47.607	-43.945\\
-43.945	-45.166\\
-45.166	-52.49\\
-52.49	-35.4\\
-35.4	-20.752\\
-20.752	-35.4\\
-35.4	-56.152\\
-56.152	-54.932\\
-54.932	-59.814\\
-59.814	-53.711\\
-53.711	-40.283\\
-40.283	-51.27\\
-51.27	-72.021\\
-72.021	-70.801\\
-70.801	-51.27\\
-51.27	-81.787\\
-81.787	-62.256\\
-62.256	-63.477\\
-63.477	-73.242\\
-73.242	-70.801\\
-70.801	-54.932\\
-54.932	-47.607\\
-47.607	-78.125\\
-78.125	-108.643\\
-108.643	-85.449\\
-85.449	-50.049\\
-50.049	-51.27\\
-51.27	-51.27\\
-51.27	-61.035\\
-61.035	-72.021\\
-72.021	-54.932\\
-54.932	-56.152\\
-56.152	-73.242\\
-73.242	-79.346\\
-79.346	-56.152\\
-56.152	-45.166\\
-45.166	-57.373\\
-57.373	-87.891\\
-87.891	-78.125\\
-78.125	-80.566\\
-80.566	-54.932\\
-54.932	-45.166\\
-45.166	-47.607\\
-47.607	-31.738\\
-31.738	-20.752\\
-20.752	-20.752\\
-20.752	-17.09\\
-17.09	-31.738\\
-31.738	-48.828\\
-48.828	-54.932\\
-54.932	-40.283\\
-40.283	-45.166\\
-45.166	-52.49\\
-52.49	-35.4\\
-35.4	-35.4\\
-35.4	-35.4\\
-35.4	-29.297\\
-29.297	-31.738\\
-31.738	-51.27\\
-51.27	-64.697\\
-64.697	-41.504\\
-41.504	-32.959\\
-32.959	-28.076\\
-28.076	-32.959\\
-32.959	-26.855\\
-26.855	-45.166\\
-45.166	-80.566\\
-80.566	-63.477\\
-63.477	-81.787\\
-81.787	-98.877\\
-98.877	-97.656\\
-97.656	-76.904\\
-76.904	-58.594\\
-58.594	-54.932\\
-54.932	-57.373\\
-57.373	-65.918\\
-65.918	-76.904\\
-76.904	-76.904\\
-76.904	-100.098\\
-100.098	-108.643\\
-108.643	-130.615\\
-130.615	-91.553\\
-91.553	-98.877\\
-98.877	-108.643\\
-108.643	-85.449\\
-85.449	-96.436\\
-96.436	-85.449\\
-85.449	-51.27\\
-51.27	-34.18\\
-34.18	-36.621\\
-36.621	-51.27\\
-51.27	-43.945\\
-43.945	-31.738\\
-31.738	-41.504\\
-41.504	-32.959\\
-32.959	-23.193\\
-23.193	-31.738\\
-31.738	-34.18\\
-34.18	-25.635\\
-25.635	-42.725\\
-42.725	-58.594\\
-58.594	-45.166\\
-45.166	-68.359\\
-68.359	-100.098\\
-100.098	-87.891\\
-87.891	-109.863\\
-109.863	-81.787\\
-81.787	-81.787\\
-81.787	-62.256\\
-62.256	-37.842\\
-37.842	-46.387\\
-46.387	-42.725\\
-42.725	-29.297\\
-29.297	-40.283\\
-40.283	-21.973\\
-21.973	-19.531\\
-19.531	-24.414\\
-24.414	-45.166\\
-45.166	-57.373\\
-57.373	-65.918\\
-65.918	-65.918\\
-65.918	-62.256\\
-62.256	-69.58\\
-69.58	-57.373\\
-57.373	-48.828\\
-48.828	-50.049\\
-50.049	-43.945\\
-43.945	-59.814\\
-59.814	-46.387\\
-46.387	-36.621\\
-36.621	-51.27\\
-51.27	-69.58\\
-69.58	-54.932\\
-54.932	-41.504\\
-41.504	-37.842\\
-37.842	-41.504\\
-41.504	-32.959\\
-32.959	-30.518\\
-30.518	-41.504\\
-41.504	-47.607\\
-47.607	-52.49\\
-52.49	-42.725\\
-42.725	-53.711\\
-53.711	-51.27\\
-51.27	-32.959\\
-32.959	-32.959\\
-32.959	-58.594\\
-58.594	-80.566\\
-80.566	-78.125\\
-78.125	-67.139\\
-67.139	-62.256\\
-62.256	-50.049\\
-50.049	-48.828\\
-48.828	-41.504\\
-41.504	-54.932\\
-54.932	-50.049\\
-50.049	-32.959\\
-32.959	-45.166\\
-45.166	-52.49\\
-52.49	-61.035\\
-61.035	-68.359\\
-68.359	-68.359\\
-68.359	-87.891\\
-87.891	-107.422\\
-107.422	-119.629\\
-119.629	-80.566\\
-80.566	-46.387\\
-46.387	-32.959\\
-32.959	-24.414\\
-24.414	-34.18\\
-34.18	-29.297\\
-29.297	-51.27\\
-51.27	-48.828\\
-48.828	-40.283\\
-40.283	-52.49\\
-52.49	-63.477\\
-63.477	-70.801\\
-70.801	-75.684\\
-75.684	-103.76\\
-103.76	-92.773\\
-92.773	-72.021\\
-72.021	-51.27\\
-51.27	-52.49\\
-52.49	-54.932\\
-54.932	-65.918\\
-65.918	-51.27\\
-51.27	-50.049\\
-50.049	-48.828\\
-48.828	-30.518\\
-30.518	-45.166\\
-45.166	-68.359\\
-68.359	-48.828\\
-48.828	-52.49\\
-52.49	-86.67\\
-86.67	-63.477\\
-63.477	-63.477\\
-63.477	-85.449\\
-85.449	-81.787\\
-81.787	-53.711\\
-53.711	-35.4\\
-35.4	-36.621\\
-36.621	-61.035\\
-61.035	-90.332\\
-90.332	-97.656\\
-97.656	-100.098\\
-100.098	-79.346\\
-79.346	-53.711\\
-53.711	-46.387\\
-46.387	-37.842\\
-37.842	-34.18\\
-34.18	-29.297\\
-29.297	-40.283\\
-40.283	-42.725\\
-42.725	-30.518\\
-30.518	-42.725\\
-42.725	-56.152\\
-56.152	-62.256\\
-62.256	-78.125\\
-78.125	-96.436\\
-96.436	-114.746\\
-114.746	-84.229\\
-84.229	-73.242\\
-73.242	-76.904\\
-76.904	-63.477\\
-63.477	-46.387\\
-46.387	-57.373\\
-57.373	-90.332\\
-90.332	-101.318\\
-101.318	-61.035\\
-61.035	-36.621\\
-36.621	-26.855\\
-26.855	-18.311\\
-18.311	-21.973\\
-21.973	-30.518\\
-30.518	-31.738\\
-31.738	-42.725\\
-42.725	-37.842\\
-37.842	-30.518\\
-30.518	-29.297\\
-29.297	-29.297\\
-29.297	-26.855\\
-26.855	-31.738\\
-31.738	-43.945\\
-43.945	-63.477\\
-63.477	-68.359\\
-68.359	-65.918\\
-65.918	-74.463\\
-74.463	-91.553\\
-91.553	-92.773\\
-92.773	-106.201\\
-106.201	-100.098\\
-100.098	-74.463\\
-74.463	-54.932\\
-54.932	-51.27\\
-51.27	-84.229\\
-84.229	-97.656\\
-97.656	-69.58\\
-69.58	-46.387\\
-46.387	-26.855\\
-26.855	-21.973\\
-21.973	-20.752\\
-20.752	-14.648\\
-14.648	-25.635\\
-25.635	-40.283\\
-40.283	-40.283\\
-40.283	-35.4\\
-35.4	-24.414\\
-24.414	-19.531\\
-19.531	-28.076\\
-28.076	-28.076\\
-28.076	-30.518\\
-30.518	-24.414\\
-24.414	-21.973\\
-21.973	-32.959\\
-32.959	-69.58\\
-69.58	-79.346\\
-79.346	-86.67\\
-86.67	-63.477\\
-63.477	-80.566\\
-80.566	-114.746\\
-114.746	-103.76\\
-103.76	-106.201\\
-106.201	-81.787\\
-81.787	-81.787\\
-81.787	-86.67\\
-86.67	-59.814\\
-59.814	-56.152\\
-56.152	-40.283\\
-40.283	-35.4\\
-35.4	-62.256\\
-62.256	-47.607\\
-47.607	-48.828\\
-48.828	-53.711\\
-53.711	-50.049\\
-50.049	-50.049\\
-50.049	-52.49\\
-52.49	-57.373\\
-57.373	-64.697\\
-64.697	-56.152\\
-56.152	-41.504\\
-41.504	-52.49\\
-52.49	-31.738\\
-31.738	-18.311\\
-18.311	-25.635\\
-25.635	-34.18\\
-34.18	-21.973\\
-21.973	-9.766\\
-9.766	-21.973\\
-21.973	-32.959\\
-32.959	-37.842\\
-37.842	-43.945\\
-43.945	-39.063\\
-39.063	-58.594\\
-58.594	-48.828\\
-48.828	-36.621\\
-36.621	-54.932\\
-54.932	-39.063\\
-39.063	-26.855\\
-26.855	-23.193\\
-23.193	-17.09\\
-17.09	-25.635\\
-25.635	-19.531\\
-19.531	-29.297\\
-29.297	-43.945\\
-43.945	-42.725\\
-42.725	-32.959\\
-32.959	-40.283\\
-40.283	-29.297\\
-29.297	-43.945\\
-43.945	-31.738\\
-31.738	-31.738\\
-31.738	-30.518\\
-30.518	-24.414\\
-24.414	-26.855\\
-26.855	-26.855\\
-26.855	-41.504\\
-41.504	-36.621\\
-36.621	-30.518\\
-30.518	-32.959\\
-32.959	-26.855\\
-26.855	-31.738\\
-31.738	-58.594\\
-58.594	-75.684\\
-75.684	-51.27\\
-51.27	-48.828\\
-48.828	-39.063\\
-39.063	-58.594\\
-58.594	-58.594\\
-58.594	-37.842\\
-37.842	-46.387\\
-46.387	-39.063\\
-39.063	-51.27\\
-51.27	-32.959\\
-32.959	-35.4\\
-35.4	-37.842\\
-37.842	-47.607\\
-47.607	-36.621\\
-36.621	-41.504\\
-41.504	-41.504\\
-41.504	-57.373\\
-57.373	-51.27\\
-51.27	-72.021\\
-72.021	-93.994\\
-93.994	-78.125\\
-78.125	-50.049\\
-50.049	-40.283\\
-40.283	-54.932\\
-54.932	-69.58\\
-69.58	-56.152\\
-56.152	-63.477\\
-63.477	-43.945\\
-43.945	-23.193\\
-23.193	-24.414\\
-24.414	-52.49\\
-52.49	-45.166\\
-45.166	-42.725\\
-42.725	-64.697\\
-64.697	-61.035\\
-61.035	-56.152\\
-56.152	-39.063\\
-39.063	-53.711\\
-53.711	-65.918\\
-65.918	-52.49\\
-52.49	-53.711\\
-53.711	-48.828\\
-48.828	-37.842\\
-37.842	-42.725\\
-42.725	-35.4\\
-35.4	-36.621\\
-36.621	-57.373\\
-57.373	-67.139\\
-67.139	-70.801\\
-70.801	-62.256\\
-62.256	-45.166\\
-45.166	-35.4\\
-35.4	-39.063\\
-39.063	-46.387\\
-46.387	-57.373\\
-57.373	-68.359\\
-68.359	-79.346\\
-79.346	-65.918\\
-65.918	-40.283\\
-40.283	-68.359\\
-68.359	-91.553\\
-91.553	-65.918\\
-65.918	-64.697\\
-64.697	-75.684\\
-75.684	-59.814\\
-59.814	-50.049\\
-50.049	-56.152\\
-56.152	-51.27\\
-51.27	-57.373\\
-57.373	-70.801\\
-70.801	-56.152\\
-56.152	-63.477\\
-63.477	-58.594\\
-58.594	-59.814\\
-59.814	-48.828\\
-48.828	-63.477\\
-63.477	-45.166\\
-45.166	-48.828\\
-48.828	-52.49\\
-52.49	-50.049\\
-50.049	-30.518\\
-30.518	-19.531\\
-19.531	-15.869\\
-15.869	-28.076\\
-28.076	-53.711\\
-53.711	-67.139\\
-67.139	-65.918\\
-65.918	-46.387\\
-46.387	-54.932\\
-54.932	-50.049\\
-50.049	-43.945\\
-43.945	-30.518\\
-30.518	-42.725\\
-42.725	-37.842\\
-37.842	-36.621\\
-36.621	-41.504\\
-41.504	-37.842\\
-37.842	-34.18\\
-34.18	-25.635\\
-25.635	-35.4\\
-35.4	-58.594\\
-58.594	-42.725\\
-42.725	-52.49\\
-52.49	-45.166\\
-45.166	-59.814\\
-59.814	-57.373\\
-57.373	-76.904\\
-76.904	-58.594\\
-58.594	-64.697\\
-64.697	-65.918\\
-65.918	-37.842\\
-37.842	-61.035\\
-61.035	-46.387\\
-46.387	-51.27\\
-51.27	-57.373\\
-57.373	-69.58\\
-69.58	-53.711\\
-53.711	-70.801\\
-70.801	-81.787\\
-81.787	-69.58\\
-69.58	-58.594\\
-58.594	-50.049\\
-50.049	-58.594\\
-58.594	-52.49\\
-52.49	-43.945\\
-43.945	-37.842\\
-37.842	-45.166\\
-45.166	-39.063\\
-39.063	-72.021\\
-72.021	-90.332\\
-90.332	-79.346\\
-79.346	-75.684\\
-75.684	-74.463\\
-74.463	-46.387\\
-46.387	-42.725\\
-42.725	-35.4\\
-35.4	-30.518\\
-30.518	-32.959\\
-32.959	-52.49\\
-52.49	-59.814\\
-59.814	-70.801\\
-70.801	-59.814\\
-59.814	-42.725\\
-42.725	-72.021\\
-72.021	-84.229\\
-84.229	-90.332\\
-90.332	-73.242\\
-73.242	-115.967\\
-115.967	-85.449\\
-85.449	-51.27\\
-51.27	-39.063\\
-39.063	-34.18\\
-34.18	-56.152\\
-56.152	-69.58\\
-69.58	-89.111\\
-89.111	-68.359\\
-68.359	-53.711\\
-53.711	-31.738\\
-31.738	-36.621\\
-36.621	-34.18\\
-34.18	-39.063\\
-39.063	-25.635\\
-25.635	-35.4\\
-35.4	-26.855\\
-26.855	-19.531\\
-19.531	-39.063\\
-39.063	-42.725\\
-42.725	-48.828\\
-48.828	-46.387\\
-46.387	-48.828\\
-48.828	-48.828\\
-48.828	-62.256\\
-62.256	-43.945\\
-43.945	-62.256\\
-62.256	-69.58\\
-69.58	-56.152\\
-56.152	-58.594\\
-58.594	-54.932\\
-54.932	-61.035\\
-61.035	-80.566\\
-80.566	-62.256\\
-62.256	-50.049\\
-50.049	-57.373\\
-57.373	-51.27\\
-51.27	-36.621\\
-36.621	-56.152\\
-56.152	-74.463\\
-74.463	-69.58\\
-69.58	-78.125\\
-78.125	-113.525\\
-113.525	-79.346\\
-79.346	-72.021\\
-72.021	-56.152\\
-56.152	-68.359\\
-68.359	-92.773\\
-92.773	-64.697\\
-64.697	-57.373\\
-57.373	-76.904\\
-76.904	-51.27\\
-51.27	-32.959\\
-32.959	-42.725\\
-42.725	-32.959\\
-32.959	-25.635\\
-25.635	-28.076\\
-28.076	-25.635\\
-25.635	-25.635\\
-25.635	-31.738\\
-31.738	-42.725\\
-42.725	-48.828\\
-48.828	-63.477\\
-63.477	-59.814\\
-59.814	-67.139\\
-67.139	-79.346\\
-79.346	-73.242\\
-73.242	-54.932\\
-54.932	-58.594\\
-58.594	-53.711\\
-53.711	-57.373\\
-57.373	-76.904\\
-76.904	-59.814\\
-59.814	-63.477\\
-63.477	-62.256\\
-62.256	-57.373\\
-57.373	-84.229\\
-84.229	-69.58\\
-69.58	-73.242\\
-73.242	-65.918\\
-65.918	-42.725\\
-42.725	-29.297\\
-29.297	-24.414\\
-24.414	-41.504\\
-41.504	-34.18\\
-34.18	-47.607\\
-47.607	-35.4\\
-35.4	-45.166\\
-45.166	-79.346\\
-79.346	-93.994\\
-93.994	-75.684\\
-75.684	-79.346\\
-79.346	-70.801\\
-70.801	-52.49\\
-52.49	-54.932\\
-54.932	-59.814\\
-59.814	-53.711\\
-53.711	-62.256\\
-62.256	-76.904\\
-76.904	-106.201\\
-106.201	-93.994\\
-93.994	-87.891\\
-87.891	-97.656\\
-97.656	-68.359\\
-68.359	-74.463\\
-74.463	-59.814\\
-59.814	-59.814\\
-59.814	-75.684\\
-75.684	-85.449\\
-85.449	-37.842\\
-37.842	-48.828\\
-48.828	-37.842\\
-37.842	-53.711\\
-53.711	-53.711\\
-53.711	-35.4\\
-35.4	-46.387\\
-46.387	-74.463\\
-74.463	-118.408\\
-118.408	-118.408\\
-118.408	-106.201\\
-106.201	-85.449\\
-85.449	-97.656\\
-97.656	-115.967\\
-115.967	-86.67\\
-86.67	-89.111\\
-89.111	-91.553\\
-91.553	-64.697\\
-64.697	-56.152\\
-56.152	-72.021\\
-72.021	-48.828\\
-48.828	-37.842\\
-37.842	-52.49\\
-52.49	-39.063\\
-39.063	-47.607\\
-47.607	-41.504\\
-41.504	-51.27\\
-51.27	-65.918\\
-65.918	-52.49\\
-52.49	-41.504\\
-41.504	-50.049\\
-50.049	-57.373\\
-57.373	-65.918\\
-65.918	-56.152\\
-56.152	-46.387\\
-46.387	-70.801\\
-70.801	-65.918\\
-65.918	-47.607\\
-47.607	-78.125\\
-78.125	-67.139\\
-67.139	-53.711\\
-53.711	-37.842\\
-37.842	-29.297\\
-29.297	-23.193\\
-23.193	-41.504\\
-41.504	-36.621\\
-36.621	-31.738\\
-31.738	-24.414\\
-24.414	-31.738\\
-31.738	-46.387\\
-46.387	-51.27\\
-51.27	-63.477\\
-63.477	-73.242\\
-73.242	-72.021\\
-72.021	-74.463\\
-74.463	-47.607\\
-47.607	-23.193\\
-23.193	-35.4\\
-35.4	-26.855\\
-26.855	-37.842\\
-37.842	-52.49\\
-52.49	-58.594\\
-58.594	-85.449\\
-85.449	-56.152\\
-56.152	-57.373\\
-57.373	-74.463\\
-74.463	-57.373\\
-57.373	-41.504\\
-41.504	-34.18\\
-34.18	-45.166\\
-45.166	-54.932\\
-54.932	-79.346\\
-79.346	-53.711\\
-53.711	-47.607\\
-47.607	-45.166\\
-45.166	-50.049\\
-50.049	-68.359\\
-68.359	-103.76\\
-103.76	-89.111\\
-89.111	-90.332\\
-90.332	-59.814\\
-59.814	-43.945\\
-43.945	-48.828\\
-48.828	-21.973\\
-21.973	-39.063\\
-39.063	-30.518\\
-30.518	-36.621\\
-36.621	-58.594\\
-58.594	-76.904\\
-76.904	-86.67\\
-86.67	-111.084\\
-111.084	-93.994\\
-93.994	-53.711\\
-53.711	-53.711\\
-53.711	-42.725\\
-42.725	-57.373\\
-57.373	-73.242\\
-73.242	-95.215\\
-95.215	-72.021\\
-72.021	-52.49\\
-52.49	-59.814\\
-59.814	-75.684\\
-75.684	-54.932\\
-54.932	-40.283\\
-40.283	-35.4\\
-35.4	-25.635\\
-25.635	-24.414\\
-24.414	-19.531\\
-19.531	-32.959\\
-32.959	-41.504\\
-41.504	-46.387\\
-46.387	-36.621\\
-36.621	-35.4\\
-35.4	-19.531\\
-19.531	-13.428\\
-13.428	-20.752\\
-20.752	-29.297\\
-29.297	-39.063\\
-39.063	-48.828\\
-48.828	-52.49\\
-52.49	-62.256\\
-62.256	-79.346\\
-79.346	-107.422\\
-107.422	-103.76\\
-103.76	-112.305\\
-112.305	-98.877\\
-98.877	-61.035\\
-61.035	-54.932\\
-54.932	-53.711\\
-53.711	-69.58\\
-69.58	-59.814\\
-59.814	-45.166\\
-45.166	-36.621\\
-36.621	-31.738\\
-31.738	-50.049\\
-50.049	-47.607\\
-47.607	-28.076\\
-28.076	-21.973\\
-21.973	-32.959\\
-32.959	-43.945\\
-43.945	-32.959\\
-32.959	-46.387\\
-46.387	-36.621\\
-36.621	-51.27\\
-51.27	-76.904\\
-76.904	-61.035\\
-61.035	-80.566\\
-80.566	-104.98\\
-104.98	-113.525\\
-113.525	-87.891\\
-87.891	-59.814\\
-59.814	-45.166\\
-45.166	-31.738\\
-31.738	-40.283\\
-40.283	-30.518\\
-30.518	-54.932\\
-54.932	-47.607\\
-47.607	-40.283\\
-40.283	-51.27\\
-51.27	-32.959\\
-32.959	-45.166\\
-45.166	-34.18\\
-34.18	-23.193\\
-23.193	-25.635\\
-25.635	-21.973\\
-21.973	-23.193\\
-23.193	-24.414\\
-24.414	-17.09\\
-17.09	-21.973\\
-21.973	-29.297\\
-29.297	-30.518\\
-30.518	-29.297\\
-29.297	-43.945\\
-43.945	-54.932\\
-54.932	-45.166\\
-45.166	-42.725\\
-42.725	-62.256\\
-62.256	-74.463\\
-74.463	-59.814\\
-59.814	-65.918\\
-65.918	-32.959\\
-32.959	-21.973\\
-21.973	-40.283\\
-40.283	-56.152\\
-56.152	-58.594\\
-58.594	-83.008\\
-83.008	-74.463\\
-74.463	-53.711\\
-53.711	-40.283\\
-40.283	-42.725\\
-42.725	-46.387\\
-46.387	-36.621\\
-36.621	-24.414\\
-24.414	-32.959\\
-32.959	-25.635\\
-25.635	-21.973\\
-21.973	-18.311\\
-18.311	-28.076\\
-28.076	-40.283\\
-40.283	-43.945\\
-43.945	-31.738\\
-31.738	-52.49\\
-52.49	-70.801\\
-70.801	-47.607\\
-47.607	-64.697\\
-64.697	-70.801\\
-70.801	-51.27\\
-51.27	-34.18\\
-34.18	-42.725\\
-42.725	-47.607\\
-47.607	-29.297\\
-29.297	-34.18\\
-34.18	-28.076\\
-28.076	-19.531\\
-19.531	-19.531\\
-19.531	-29.297\\
-29.297	-36.621\\
-36.621	-32.959\\
-32.959	-39.063\\
-39.063	-42.725\\
-42.725	-32.959\\
-32.959	-20.752\\
-20.752	-18.311\\
-18.311	-21.973\\
-21.973	-25.635\\
-25.635	-29.297\\
-29.297	-54.932\\
-54.932	-53.711\\
-53.711	-47.607\\
-47.607	-47.607\\
-47.607	-61.035\\
-61.035	-76.904\\
-76.904	-83.008\\
-83.008	-85.449\\
-85.449	-64.697\\
-64.697	-65.918\\
-65.918	-67.139\\
-67.139	-69.58\\
-69.58	-76.904\\
-76.904	-101.318\\
-101.318	-86.67\\
-86.67	-79.346\\
-79.346	-63.477\\
-63.477	-63.477\\
-63.477	-59.814\\
-59.814	-70.801\\
-70.801	-46.387\\
-46.387	-53.711\\
-53.711	-62.256\\
-62.256	-72.021\\
-72.021	-56.152\\
-56.152	-65.918\\
-65.918	-52.49\\
-52.49	-48.828\\
-48.828	-32.959\\
-32.959	-51.27\\
-51.27	-73.242\\
-73.242	-57.373\\
-57.373	-36.621\\
-36.621	-43.945\\
-43.945	-28.076\\
-28.076	-25.635\\
-25.635	-32.959\\
-32.959	-23.193\\
-23.193	-25.635\\
-25.635	-28.076\\
-28.076	-19.531\\
-19.531	-29.297\\
-29.297	-25.635\\
-25.635	-19.531\\
-19.531	-28.076\\
-28.076	-31.738\\
-31.738	-35.4\\
-35.4	-36.621\\
-36.621	-35.4\\
-35.4	-46.387\\
-46.387	-40.283\\
-40.283	-47.607\\
-47.607	-75.684\\
-75.684	-56.152\\
-56.152	-50.049\\
-50.049	-52.49\\
-52.49	-40.283\\
-40.283	-26.855\\
-26.855	-42.725\\
-42.725	-35.4\\
-35.4	-25.635\\
-25.635	-37.842\\
-37.842	-36.621\\
-36.621	-42.725\\
-42.725	-30.518\\
-30.518	-21.973\\
-21.973	-18.311\\
-18.311	-20.752\\
-20.752	-34.18\\
-34.18	-31.738\\
-31.738	-35.4\\
-35.4	-47.607\\
-47.607	-63.477\\
-63.477	-48.828\\
-48.828	-63.477\\
-63.477	-74.463\\
-74.463	-64.697\\
-64.697	-59.814\\
-59.814	-97.656\\
-97.656	-70.801\\
-70.801	-86.67\\
-86.67	-76.904\\
-76.904	-86.67\\
-86.67	-95.215\\
-95.215	-61.035\\
-61.035	-39.063\\
-39.063	-32.959\\
-32.959	-47.607\\
-47.607	-57.373\\
-57.373	-58.594\\
-58.594	-41.504\\
-41.504	-25.635\\
-25.635	-34.18\\
-34.18	-57.373\\
-57.373	-76.904\\
-76.904	-75.684\\
-75.684	-48.828\\
-48.828	-39.063\\
-39.063	-48.828\\
-48.828	-72.021\\
-72.021	-81.787\\
-81.787	-83.008\\
-83.008	-61.035\\
-61.035	-83.008\\
-83.008	-90.332\\
-90.332	-86.67\\
-86.67	-114.746\\
-114.746	-98.877\\
-98.877	-85.449\\
-85.449	-98.877\\
-98.877	-84.229\\
-84.229	-51.27\\
-51.27	-47.607\\
-47.607	-56.152\\
-56.152	-67.139\\
-67.139	-67.139\\
-67.139	-51.27\\
-51.27	-63.477\\
-63.477	-56.152\\
-56.152	-43.945\\
-43.945	-54.932\\
-54.932	-52.49\\
-52.49	-75.684\\
-75.684	-76.904\\
-76.904	-57.373\\
-57.373	-45.166\\
-45.166	-30.518\\
-30.518	-43.945\\
-43.945	-35.4\\
-35.4	-31.738\\
-31.738	-26.855\\
-26.855	-42.725\\
-42.725	-53.711\\
-53.711	-58.594\\
-58.594	-58.594\\
-58.594	-37.842\\
-37.842	-30.518\\
-30.518	-23.193\\
-23.193	-28.076\\
-28.076	-40.283\\
-40.283	-42.725\\
-42.725	-40.283\\
-40.283	-62.256\\
-62.256	-73.242\\
-73.242	-80.566\\
-80.566	-81.787\\
-81.787	-95.215\\
-95.215	-59.814\\
-59.814	-47.607\\
-47.607	-40.283\\
-40.283	-43.945\\
-43.945	-54.932\\
-54.932	-46.387\\
-46.387	-34.18\\
-34.18	-30.518\\
-30.518	-52.49\\
-52.49	-62.256\\
-62.256	-57.373\\
-57.373	-85.449\\
-85.449	-91.553\\
-91.553	-65.918\\
-65.918	-47.607\\
-47.607	-36.621\\
-36.621	-32.959\\
-32.959	-56.152\\
-56.152	-48.828\\
-48.828	-53.711\\
-53.711	-59.814\\
-59.814	-67.139\\
-67.139	-63.477\\
-63.477	-72.021\\
-72.021	-51.27\\
-51.27	-56.152\\
-56.152	-42.725\\
-42.725	-39.063\\
-39.063	-54.932\\
-54.932	-75.684\\
-75.684	-81.787\\
-81.787	-47.607\\
-47.607	-91.553\\
-91.553	-78.125\\
-78.125	-53.711\\
-53.711	-34.18\\
-34.18	-50.049\\
-50.049	-45.166\\
-45.166	-45.166\\
-45.166	-53.711\\
-53.711	-37.842\\
-37.842	-48.828\\
-48.828	-86.67\\
-86.67	-91.553\\
-91.553	-63.477\\
-63.477	-70.801\\
-70.801	-78.125\\
-78.125	-76.904\\
-76.904	-57.373\\
-57.373	-56.152\\
-56.152	-41.504\\
-41.504	-54.932\\
-54.932	-59.814\\
-59.814	-92.773\\
-92.773	-101.318\\
-101.318	-131.836\\
-131.836	-90.332\\
-90.332	-76.904\\
-76.904	-62.256\\
-62.256	-67.139\\
-67.139	-52.49\\
-52.49	-37.842\\
-37.842	-37.842\\
-37.842	-39.063\\
-39.063	-31.738\\
-31.738	-23.193\\
-23.193	-34.18\\
-34.18	-45.166\\
-45.166	-29.297\\
-29.297	-25.635\\
-25.635	-24.414\\
-24.414	-50.049\\
-50.049	-67.139\\
-67.139	-35.4\\
-35.4	-41.504\\
-41.504	-43.945\\
-43.945	-40.283\\
-40.283	-47.607\\
-47.607	-45.166\\
-45.166	-32.959\\
-32.959	-25.635\\
-25.635	-23.193\\
-23.193	-26.855\\
-26.855	-47.607\\
-47.607	-54.932\\
-54.932	-41.504\\
-41.504	-30.518\\
-30.518	-23.193\\
-23.193	-28.076\\
-28.076	-42.725\\
-42.725	-51.27\\
-51.27	-56.152\\
-56.152	-40.283\\
-40.283	-39.063\\
-39.063	-42.725\\
-42.725	-57.373\\
-57.373	-76.904\\
-76.904	-89.111\\
-89.111	-69.58\\
-69.58	-46.387\\
-46.387	-48.828\\
-48.828	-46.387\\
-46.387	-46.387\\
-46.387	-35.4\\
-35.4	-39.063\\
-39.063	-43.945\\
-43.945	-41.504\\
-41.504	-29.297\\
-29.297	-20.752\\
-20.752	-29.297\\
-29.297	-40.283\\
-40.283	-57.373\\
-57.373	-61.035\\
-61.035	-74.463\\
-74.463	-65.918\\
-65.918	-73.242\\
-73.242	-96.436\\
-96.436	-126.953\\
-126.953	-106.201\\
-106.201	-73.242\\
-73.242	-80.566\\
-80.566	-91.553\\
-91.553	-122.07\\
-122.07	-80.566\\
-80.566	-43.945\\
-43.945	-30.518\\
-30.518	-31.738\\
-31.738	-25.635\\
-25.635	-18.311\\
-18.311	-21.973\\
-21.973	-29.297\\
-29.297	-37.842\\
-37.842	-37.842\\
-37.842	-30.518\\
-30.518	-29.297\\
-29.297	-36.621\\
-36.621	-43.945\\
-43.945	-45.166\\
-45.166	-42.725\\
-42.725	-32.959\\
-32.959	-24.414\\
-24.414	-23.193\\
-23.193	-35.4\\
-35.4	-39.063\\
-39.063	-30.518\\
-30.518	-23.193\\
-23.193	-35.4\\
-35.4	-28.076\\
-28.076	-30.518\\
-30.518	-37.842\\
-37.842	-50.049\\
-50.049	-54.932\\
-54.932	-40.283\\
-40.283	-56.152\\
-56.152	-54.932\\
-54.932	-46.387\\
-46.387	-37.842\\
-37.842	-58.594\\
-58.594	-73.242\\
-73.242	-73.242\\
-73.242	-79.346\\
-79.346	-62.256\\
-62.256	-57.373\\
-57.373	-54.932\\
-54.932	-72.021\\
-72.021	-59.814\\
-59.814	-78.125\\
-78.125	-70.801\\
-70.801	-74.463\\
-74.463	-124.512\\
-124.512	-98.877\\
-98.877	-53.711\\
-53.711	-37.842\\
-37.842	-41.504\\
-41.504	-51.27\\
-51.27	-63.477\\
-63.477	-70.801\\
-70.801	-78.125\\
-78.125	-80.566\\
-80.566	-54.932\\
-54.932	-47.607\\
-47.607	-56.152\\
-56.152	-56.152\\
-56.152	-36.621\\
-36.621	-28.076\\
-28.076	-45.166\\
-45.166	-46.387\\
-46.387	-59.814\\
-59.814	-73.242\\
-73.242	-64.697\\
-64.697	-47.607\\
-47.607	-40.283\\
-40.283	-57.373\\
-57.373	-70.801\\
-70.801	-91.553\\
-91.553	-100.098\\
-100.098	-113.525\\
-113.525	-92.773\\
-92.773	-85.449\\
-85.449	-52.49\\
-52.49	-40.283\\
-40.283	-65.918\\
-65.918	-85.449\\
-85.449	-52.49\\
-52.49	-74.463\\
-74.463	-104.98\\
-104.98	-119.629\\
-119.629	-79.346\\
-79.346	-46.387\\
-46.387	-30.518\\
-30.518	-40.283\\
-40.283	-37.842\\
-37.842	-41.504\\
-41.504	-29.297\\
-29.297	-35.4\\
-35.4	-30.518\\
-30.518	-35.4\\
-35.4	-31.738\\
-31.738	-34.18\\
-34.18	-53.711\\
-53.711	-50.049\\
-50.049	-67.139\\
-67.139	-80.566\\
-80.566	-79.346\\
-79.346	-46.387\\
-46.387	-47.607\\
-47.607	-57.373\\
-57.373	-40.283\\
-40.283	-35.4\\
-35.4	-28.076\\
-28.076	-41.504\\
-41.504	-35.4\\
-35.4	-23.193\\
-23.193	-14.648\\
-14.648	-17.09\\
-17.09	-26.855\\
-26.855	-37.842\\
-37.842	-40.283\\
-40.283	-28.076\\
-28.076	-31.738\\
-31.738	-46.387\\
-46.387	-57.373\\
-57.373	-41.504\\
-41.504	-41.504\\
-41.504	-48.828\\
-48.828	-54.932\\
-54.932	-50.049\\
-50.049	-57.373\\
-57.373	-53.711\\
-53.711	-40.283\\
-40.283	-31.738\\
-31.738	-42.725\\
-42.725	-36.621\\
-36.621	-42.725\\
-42.725	-57.373\\
-57.373	-83.008\\
-83.008	-63.477\\
-63.477	-57.373\\
-57.373	-85.449\\
-85.449	-68.359\\
-68.359	-41.504\\
-41.504	-34.18\\
-34.18	-28.076\\
-28.076	-28.076\\
-28.076	-26.855\\
-26.855	-25.635\\
-25.635	-45.166\\
-45.166	-28.076\\
-28.076	-23.193\\
-23.193	-32.959\\
-32.959	-41.504\\
-41.504	-32.959\\
-32.959	-25.635\\
-25.635	-17.09\\
-17.09	-25.635\\
-25.635	-46.387\\
-46.387	-59.814\\
-59.814	-50.049\\
-50.049	-40.283\\
-40.283	-45.166\\
-45.166	-46.387\\
-46.387	-50.049\\
-50.049	-61.035\\
-61.035	-70.801\\
-70.801	-68.359\\
-68.359	-58.594\\
-58.594	-41.504\\
-41.504	-34.18\\
-34.18	-36.621\\
-36.621	-48.828\\
-48.828	-35.4\\
-35.4	-28.076\\
-28.076	-50.049\\
-50.049	-61.035\\
-61.035	-57.373\\
-57.373	-65.918\\
-65.918	-83.008\\
-83.008	-48.828\\
-48.828	-81.787\\
-81.787	-76.904\\
-76.904	-73.242\\
-73.242	-76.904\\
-76.904	-76.904\\
-76.904	-122.07\\
-122.07	-119.629\\
-119.629	-79.346\\
-79.346	-93.994\\
-93.994	-115.967\\
-115.967	-112.305\\
-112.305	-124.512\\
-124.512	-113.525\\
-113.525	-142.822\\
-142.822	-115.967\\
-115.967	-79.346\\
-79.346	-53.711\\
-53.711	-52.49\\
-52.49	-45.166\\
-45.166	-37.842\\
-37.842	-52.49\\
-52.49	-73.242\\
-73.242	-45.166\\
-45.166	-41.504\\
-41.504	-42.725\\
-42.725	-31.738\\
-31.738	-34.18\\
-34.18	-25.635\\
-25.635	-24.414\\
-24.414	-23.193\\
-23.193	-29.297\\
-29.297	-24.414\\
-24.414	-20.752\\
-20.752	-17.09\\
-17.09	-13.428\\
-13.428	-24.414\\
-24.414	-20.752\\
-20.752	-18.311\\
-18.311	-37.842\\
-37.842	-51.27\\
-51.27	-50.049\\
-50.049	-40.283\\
-40.283	-48.828\\
-48.828	-67.139\\
-67.139	-34.18\\
-34.18	-24.414\\
-24.414	-19.531\\
-19.531	-14.648\\
-14.648	-20.752\\
-20.752	-36.621\\
-36.621	-50.049\\
-50.049	-36.621\\
-36.621	-47.607\\
-47.607	-74.463\\
-74.463	-73.242\\
-73.242	-53.711\\
-53.711	-37.842\\
-37.842	-43.945\\
-43.945	-59.814\\
-59.814	-73.242\\
-73.242	-46.387\\
-46.387	-25.635\\
-25.635	-36.621\\
-36.621	-34.18\\
-34.18	-17.09\\
-17.09	-12.207\\
-12.207	-26.855\\
-26.855	-53.711\\
-53.711	-53.711\\
-53.711	-43.945\\
-43.945	-29.297\\
-29.297	-28.076\\
-28.076	-17.09\\
-17.09	-18.311\\
-18.311	-21.973\\
-21.973	-26.855\\
-26.855	-36.621\\
-36.621	-54.932\\
-54.932	-57.373\\
-57.373	-59.814\\
-59.814	-63.477\\
-63.477	-62.256\\
-62.256	-59.814\\
-59.814	-53.711\\
-53.711	-64.697\\
-64.697	-93.994\\
-93.994	-86.67\\
-86.67	-50.049\\
-50.049	-29.297\\
-29.297	-52.49\\
-52.49	-67.139\\
-67.139	-59.814\\
-59.814	-37.842\\
-37.842	-37.842\\
-37.842	-59.814\\
-59.814	-89.111\\
-89.111	-89.111\\
-89.111	-98.877\\
-98.877	-97.656\\
-97.656	-73.242\\
-73.242	-76.904\\
-76.904	-43.945\\
-43.945	-36.621\\
-36.621	-45.166\\
-45.166	-54.932\\
-54.932	-57.373\\
-57.373	-61.035\\
-61.035	-42.725\\
-42.725	-51.27\\
-51.27	-48.828\\
-48.828	-29.297\\
-29.297	-21.973\\
-21.973	-17.09\\
-17.09	-25.635\\
-25.635	-23.193\\
-23.193	-13.428\\
-13.428	-25.635\\
-25.635	-46.387\\
-46.387	-35.4\\
-35.4	-30.518\\
-30.518	-40.283\\
-40.283	-61.035\\
-61.035	-48.828\\
-48.828	-24.414\\
-24.414	-18.311\\
-18.311	-32.959\\
-32.959	-68.359\\
-68.359	-81.787\\
-81.787	-109.863\\
-109.863	-130.615\\
-130.615	-126.953\\
-126.953	-87.891\\
-87.891	-87.891\\
-87.891	-109.863\\
-109.863	-95.215\\
-95.215	-89.111\\
-89.111	-98.877\\
-98.877	-107.422\\
-107.422	-108.643\\
-108.643	-145.264\\
-145.264	-102.539\\
-102.539	-57.373\\
-57.373	-34.18\\
-34.18	-29.297\\
-29.297	-32.959\\
-32.959	-32.959\\
-32.959	-31.738\\
-31.738	-53.711\\
-53.711	-45.166\\
-45.166	-45.166\\
-45.166	-59.814\\
-59.814	-34.18\\
-34.18	-37.842\\
-37.842	-51.27\\
-51.27	-57.373\\
-57.373	-35.4\\
-35.4	-24.414\\
-24.414	-31.738\\
-31.738	-31.738\\
-31.738	-58.594\\
-58.594	-91.553\\
-91.553	-86.67\\
-86.67	-96.436\\
-96.436	-111.084\\
-111.084	-140.381\\
-140.381	-120.85\\
-120.85	-80.566\\
-80.566	-54.932\\
-54.932	-34.18\\
-34.18	-39.063\\
-39.063	-39.063\\
-39.063	-47.607\\
-47.607	-36.621\\
-36.621	-32.959\\
-32.959	-36.621\\
-36.621	-43.945\\
-43.945	-47.607\\
-47.607	-58.594\\
-58.594	-45.166\\
-45.166	-23.193\\
-23.193	-18.311\\
-18.311	-18.311\\
-18.311	-17.09\\
-17.09	-21.973\\
-21.973	-32.959\\
-32.959	-35.4\\
-35.4	-42.725\\
-42.725	-57.373\\
-57.373	-45.166\\
-45.166	-63.477\\
-63.477	-97.656\\
-97.656	-78.125\\
-78.125	-68.359\\
-68.359	-84.229\\
-84.229	-96.436\\
-96.436	-62.256\\
-62.256	-58.594\\
-58.594	-40.283\\
-40.283	-41.504\\
-41.504	-76.904\\
-76.904	-124.512\\
-124.512	-147.705\\
-147.705	-115.967\\
-115.967	-98.877\\
-98.877	-75.684\\
-75.684	-80.566\\
-80.566	-103.76\\
-103.76	-63.477\\
-63.477	-51.27\\
-51.27	-41.504\\
-41.504	-72.021\\
-72.021	-72.021\\
-72.021	-40.283\\
-40.283	-41.504\\
-41.504	-34.18\\
-34.18	-50.049\\
-50.049	-76.904\\
-76.904	-75.684\\
-75.684	-58.594\\
-58.594	-46.387\\
-46.387	-52.49\\
-52.49	-41.504\\
-41.504	-30.518\\
-30.518	-26.855\\
-26.855	-24.414\\
-24.414	-20.752\\
-20.752	-23.193\\
-23.193	-37.842\\
-37.842	-51.27\\
-51.27	-51.27\\
-51.27	-31.738\\
-31.738	-29.297\\
-29.297	-26.855\\
-26.855	-32.959\\
-32.959	-45.166\\
-45.166	-47.607\\
-47.607	-45.166\\
-45.166	-28.076\\
-28.076	-47.607\\
-47.607	-69.58\\
-69.58	-53.711\\
-53.711	-26.855\\
-26.855	-26.855\\
-26.855	-41.504\\
-41.504	-42.725\\
-42.725	-37.842\\
-37.842	-26.855\\
-26.855	-41.504\\
-41.504	-62.256\\
-62.256	-41.504\\
-41.504	-17.09\\
-17.09	-25.635\\
-25.635	-51.27\\
-51.27	-54.932\\
-54.932	-39.063\\
-39.063	-32.959\\
-32.959	-58.594\\
-58.594	-76.904\\
-76.904	-64.697\\
-64.697	-87.891\\
-87.891	-108.643\\
-108.643	-72.021\\
-72.021	-59.814\\
-59.814	-80.566\\
-80.566	-54.932\\
-54.932	-73.242\\
-73.242	-87.891\\
-87.891	-69.58\\
-69.58	-36.621\\
-36.621	-58.594\\
-58.594	-100.098\\
-100.098	-114.746\\
-114.746	-83.008\\
-83.008	-72.021\\
-72.021	-92.773\\
-92.773	-74.463\\
-74.463	-48.828\\
-48.828	-45.166\\
-45.166	-57.373\\
-57.373	-58.594\\
-58.594	-57.373\\
-57.373	-64.697\\
-64.697	-48.828\\
-48.828	-50.049\\
-50.049	-68.359\\
-68.359	-43.945\\
-43.945	-35.4\\
-35.4	-30.518\\
-30.518	-23.193\\
-23.193	-26.855\\
-26.855	-46.387\\
-46.387	-41.504\\
-41.504	-57.373\\
-57.373	-69.58\\
-69.58	-83.008\\
-83.008	-63.477\\
-63.477	-57.373\\
-57.373	-25.635\\
-25.635	-47.607\\
-47.607	-72.021\\
-72.021	-51.27\\
-51.27	-26.855\\
-26.855	-51.27\\
-51.27	-74.463\\
-74.463	-74.463\\
-74.463	-92.773\\
-92.773	-120.85\\
-120.85	-87.891\\
-87.891	-74.463\\
-74.463	-86.67\\
-86.67	-61.035\\
-61.035	-47.607\\
-47.607	-53.711\\
-53.711	-68.359\\
-68.359	-72.021\\
-72.021	-72.021\\
-72.021	-45.166\\
-45.166	-39.063\\
-39.063	-32.959\\
-32.959	-46.387\\
-46.387	-43.945\\
-43.945	-34.18\\
-34.18	-63.477\\
-63.477	-68.359\\
-68.359	-50.049\\
-50.049	-40.283\\
-40.283	-58.594\\
-58.594	-32.959\\
-32.959	-21.973\\
-21.973	-28.076\\
-28.076	-35.4\\
-35.4	-34.18\\
-34.18	-25.635\\
-25.635	-18.311\\
-18.311	-28.076\\
-28.076	-37.842\\
-37.842	-53.711\\
-53.711	-59.814\\
-59.814	-75.684\\
-75.684	-86.67\\
-86.67	-48.828\\
-48.828	-39.063\\
-39.063	-76.904\\
-76.904	-109.863\\
-109.863	-83.008\\
-83.008	-78.125\\
-78.125	-113.525\\
-113.525	-92.773\\
-92.773	-74.463\\
-74.463	-57.373\\
-57.373	-58.594\\
-58.594	-75.684\\
-75.684	-52.49\\
-52.49	-45.166\\
-45.166	-75.684\\
-75.684	-95.215\\
-95.215	-96.436\\
-96.436	-48.828\\
-48.828	-25.635\\
-25.635	-56.152\\
-56.152	-47.607\\
-47.607	-21.973\\
-21.973	-20.752\\
-20.752	-31.738\\
-31.738	-41.504\\
-41.504	-67.139\\
-67.139	-50.049\\
-50.049	-39.063\\
-39.063	-28.076\\
-28.076	-19.531\\
-19.531	-25.635\\
-25.635	-23.193\\
-23.193	-26.855\\
-26.855	-42.725\\
-42.725	-39.063\\
-39.063	-24.414\\
-24.414	-20.752\\
-20.752	-26.855\\
-26.855	-30.518\\
-30.518	-20.752\\
-20.752	-31.738\\
-31.738	-24.414\\
-24.414	-19.531\\
-19.531	-36.621\\
-36.621	-47.607\\
-47.607	-69.58\\
-69.58	-92.773\\
-92.773	-81.787\\
-81.787	-42.725\\
-42.725	-32.959\\
-32.959	-34.18\\
-34.18	-21.973\\
-21.973	-21.973\\
-21.973	-35.4\\
-35.4	-39.063\\
-39.063	-37.842\\
-37.842	-42.725\\
-42.725	-48.828\\
-48.828	-52.49\\
-52.49	-52.49\\
-52.49	-62.256\\
-62.256	-62.256\\
-62.256	-86.67\\
-86.67	-47.607\\
-47.607	-25.635\\
-25.635	-24.414\\
-24.414	-42.725\\
-42.725	-31.738\\
-31.738	-34.18\\
-34.18	-18.311\\
-18.311	-28.076\\
-28.076	-18.311\\
-18.311	-13.428\\
-13.428	-18.311\\
-18.311	-31.738\\
-31.738	-36.621\\
-36.621	-51.27\\
-51.27	-69.58\\
-69.58	-52.49\\
-52.49	-25.635\\
-25.635	-39.063\\
-39.063	-45.166\\
-45.166	-25.635\\
-25.635	-32.959\\
-32.959	-58.594\\
-58.594	-50.049\\
-50.049	-79.346\\
-79.346	-92.773\\
-92.773	-64.697\\
-64.697	-51.27\\
-51.27	-46.387\\
};
\addlegendentry{data1}

\addplot [color=mycolor2, line width=2.0pt]
  table[row sep=crcr]{%
-54.932	-53.0506425379402\\
-63.477	-61.3029861716455\\
-75.684	-73.0919105410592\\
-62.256	-60.1238040093571\\
-59.814	-57.7654396847804\\
-80.566	-77.8067076878994\\
-76.904	-74.2701269521909\\
-90.332	-87.2382334838931\\
-68.359	-66.0177833184857\\
-40.283	-38.9033538439497\\
-48.828	-47.155697477655\\
-46.387	-44.7982989042349\\
-54.932	-53.0506425379402\\
-45.166	-43.6191167419465\\
-21.973	-21.2204501654074\\
-17.09	-16.5046872674106\\
-18.311	-17.683869429699\\
-40.283	-38.9033538439497\\
-62.256	-60.1238040093571\\
-67.139	-64.839566907354\\
-69.58	-67.196965480774\\
-52.49	-50.6922782133635\\
-34.18	-33.0093745348212\\
-64.697	-62.4812025827772\\
-52.49	-50.6922782133635\\
-39.063	-37.725137432818\\
-61.035	-58.9446218470687\\
-53.711	-51.8714603756518\\
-45.166	-43.6191167419465\\
-73.242	-70.7335462164825\\
-67.139	-64.839566907354\\
-52.49	-50.6922782133635\\
-75.684	-73.0919105410592\\
-74.463	-71.9127283787709\\
-58.594	-56.5872232736487\\
-50.049	-48.3348796399434\\
-41.504	-40.0825360062381\\
-47.607	-45.9765153153666\\
-58.594	-56.5872232736487\\
-54.932	-53.0506425379402\\
-59.814	-57.7654396847804\\
-65.918	-63.6603847450656\\
-85.449	-82.5224705858962\\
-76.904	-74.2701269521909\\
-54.932	-53.0506425379402\\
-50.049	-48.3348796399434\\
-35.4	-34.1875909459529\\
-34.18	-33.0093745348212\\
-45.166	-43.6191167419465\\
-34.18	-33.0093745348212\\
-36.621	-35.3667731082412\\
-51.27	-49.5140618022318\\
-54.932	-53.0506425379402\\
-51.27	-49.5140618022318\\
-78.125	-75.4493091144793\\
-62.256	-60.1238040093571\\
-36.621	-35.3667731082412\\
-30.518	-29.4727937991127\\
-40.283	-38.9033538439497\\
-46.387	-44.7982989042349\\
-29.297	-28.2936116368243\\
-36.621	-35.3667731082412\\
-31.738	-30.6510102102444\\
-26.855	-25.9352473122476\\
-35.4	-34.1875909459529\\
-40.283	-38.9033538439497\\
-58.594	-56.5872232736487\\
-56.152	-54.2288589490719\\
-65.918	-63.6603847450656\\
-61.035	-58.9446218470687\\
-70.801	-68.3761476430624\\
-57.373	-55.4080411113603\\
-58.594	-56.5872232736487\\
-51.27	-49.5140618022318\\
-50.049	-48.3348796399434\\
-48.828	-47.155697477655\\
-80.566	-77.8067076878994\\
-111.084	-107.279501487012\\
-114.746	-110.816082222721\\
-112.305	-108.4586836493\\
-74.463	-71.9127283787709\\
-100.098	-96.6697592798867\\
-125.732	-121.425824429846\\
-128.174	-123.784188754423\\
-96.436	-93.1331785441783\\
-124.512	-120.247608018714\\
-158.691	-153.256016802379\\
-115.967	-111.995264385009\\
-97.656	-94.31139495531\\
-68.359	-66.0177833184857\\
-53.711	-51.8714603756518\\
-47.607	-45.9765153153666\\
-56.152	-54.2288589490719\\
-45.166	-43.6191167419465\\
-30.518	-29.4727937991127\\
-36.621	-35.3667731082412\\
-50.049	-48.3348796399434\\
-48.828	-47.155697477655\\
-47.607	-45.9765153153666\\
-62.256	-60.1238040093571\\
-64.697	-62.4812025827772\\
-85.449	-82.5224705858962\\
-72.021	-69.5543640541941\\
-85.449	-82.5224705858962\\
-63.477	-61.3029861716455\\
-54.932	-53.0506425379402\\
-64.697	-62.4812025827772\\
-48.828	-47.155697477655\\
-43.945	-42.4399345796582\\
-53.711	-51.8714603756518\\
-54.932	-53.0506425379402\\
-40.283	-38.9033538439497\\
-43.945	-42.4399345796582\\
-32.959	-31.8301923725328\\
-36.621	-35.3667731082412\\
-40.283	-38.9033538439497\\
-30.518	-29.4727937991127\\
-34.18	-33.0093745348212\\
-43.945	-42.4399345796582\\
-63.477	-61.3029861716455\\
-65.918	-63.6603847450656\\
-72.021	-69.5543640541941\\
-45.166	-43.6191167419465\\
-56.152	-54.2288589490719\\
-89.111	-86.0590513216047\\
-70.801	-68.3761476430624\\
-81.787	-78.9858898501878\\
-80.566	-77.8067076878994\\
-50.049	-48.3348796399434\\
-36.621	-35.3667731082412\\
-47.607	-45.9765153153666\\
-52.49	-50.6922782133635\\
-81.787	-78.9858898501878\\
-103.76	-100.206340015595\\
-102.539	-99.0271578533068\\
-76.904	-74.2701269521909\\
-80.566	-77.8067076878994\\
-72.021	-69.5543640541941\\
-69.58	-67.196965480774\\
-68.359	-66.0177833184857\\
-51.27	-49.5140618022318\\
-45.166	-43.6191167419465\\
-39.063	-37.725137432818\\
-37.842	-36.5459552705296\\
-41.504	-40.0825360062381\\
-56.152	-54.2288589490719\\
-84.229	-81.3442541747645\\
-70.801	-68.3761476430624\\
-50.049	-48.3348796399434\\
-42.725	-41.2617181685265\\
-40.283	-38.9033538439497\\
-28.076	-27.114429474536\\
-23.193	-22.3986665765391\\
-24.414	-23.5778487388275\\
-41.504	-40.0825360062381\\
-32.959	-31.8301923725328\\
-34.18	-33.0093745348212\\
-37.842	-36.5459552705296\\
-35.4	-34.1875909459529\\
-26.855	-25.9352473122476\\
-21.973	-21.2204501654074\\
-18.311	-17.683869429699\\
-25.635	-24.7570309011159\\
-51.27	-49.5140618022318\\
-73.242	-70.7335462164825\\
-78.125	-75.4493091144793\\
-54.932	-53.0506425379402\\
-39.063	-37.725137432818\\
-31.738	-30.6510102102444\\
-24.414	-23.5778487388275\\
-41.504	-40.0825360062381\\
-42.725	-41.2617181685265\\
-54.932	-53.0506425379402\\
-73.242	-70.7335462164825\\
-108.643	-104.922102913592\\
-125.732	-121.425824429846\\
-103.76	-100.206340015595\\
-101.318	-97.8479756910184\\
-67.139	-64.839566907354\\
-75.684	-73.0919105410592\\
-79.346	-76.6284912767677\\
-86.67	-83.7016527481846\\
-69.58	-67.196965480774\\
-62.256	-60.1238040093571\\
-61.035	-58.9446218470687\\
-56.152	-54.2288589490719\\
-67.139	-64.839566907354\\
-52.49	-50.6922782133635\\
-76.904	-74.2701269521909\\
-97.656	-94.31139495531\\
-72.021	-69.5543640541941\\
-45.166	-43.6191167419465\\
-47.607	-45.9765153153666\\
-78.125	-75.4493091144793\\
-93.994	-90.7748142196015\\
-112.305	-108.4586836493\\
-119.629	-115.531845120717\\
-91.553	-88.4174156461815\\
-84.229	-81.3442541747645\\
-96.436	-93.1331785441783\\
-111.084	-107.279501487012\\
-74.463	-71.9127283787709\\
-46.387	-44.7982989042349\\
-62.256	-60.1238040093571\\
-54.932	-53.0506425379402\\
-34.18	-33.0093745348212\\
-47.607	-45.9765153153666\\
-53.711	-51.8714603756518\\
-42.725	-41.2617181685265\\
-34.18	-33.0093745348212\\
-43.945	-42.4399345796582\\
-40.283	-38.9033538439497\\
-52.49	-50.6922782133635\\
-65.918	-63.6603847450656\\
-62.256	-60.1238040093571\\
-45.166	-43.6191167419465\\
-67.139	-64.839566907354\\
-104.98	-101.384556426727\\
-79.346	-76.6284912767677\\
-51.27	-49.5140618022318\\
-40.283	-38.9033538439497\\
-47.607	-45.9765153153666\\
-40.283	-38.9033538439497\\
-31.738	-30.6510102102444\\
-35.4	-34.1875909459529\\
-42.725	-41.2617181685265\\
-51.27	-49.5140618022318\\
-57.373	-55.4080411113603\\
-42.725	-41.2617181685265\\
-63.477	-61.3029861716455\\
-75.684	-73.0919105410592\\
-70.801	-68.3761476430624\\
-57.373	-55.4080411113603\\
-62.256	-60.1238040093571\\
-81.787	-78.9858898501878\\
-52.49	-50.6922782133635\\
-28.076	-27.114429474536\\
-36.621	-35.3667731082412\\
-42.725	-41.2617181685265\\
-51.27	-49.5140618022318\\
-46.387	-44.7982989042349\\
-36.621	-35.3667731082412\\
-52.49	-50.6922782133635\\
-37.842	-36.5459552705296\\
-47.607	-45.9765153153666\\
-41.504	-40.0825360062381\\
-37.842	-36.5459552705296\\
-45.166	-43.6191167419465\\
-29.297	-28.2936116368243\\
-31.738	-30.6510102102444\\
-29.297	-28.2936116368243\\
-26.855	-25.9352473122476\\
-47.607	-45.9765153153666\\
-53.711	-51.8714603756518\\
-69.58	-67.196965480774\\
-89.111	-86.0590513216047\\
-62.256	-60.1238040093571\\
-36.621	-35.3667731082412\\
-31.738	-30.6510102102444\\
-29.297	-28.2936116368243\\
-40.283	-38.9033538439497\\
-31.738	-30.6510102102444\\
-30.518	-29.4727937991127\\
-39.063	-37.725137432818\\
-61.035	-58.9446218470687\\
-53.711	-51.8714603756518\\
-79.346	-76.6284912767677\\
-54.932	-53.0506425379402\\
-48.828	-47.155697477655\\
-32.959	-31.8301923725328\\
-29.297	-28.2936116368243\\
-23.193	-22.3986665765391\\
-25.635	-24.7570309011159\\
-40.283	-38.9033538439497\\
-46.387	-44.7982989042349\\
-48.828	-47.155697477655\\
-51.27	-49.5140618022318\\
-79.346	-76.6284912767677\\
-73.242	-70.7335462164825\\
-53.711	-51.8714603756518\\
-63.477	-61.3029861716455\\
-69.58	-67.196965480774\\
-85.449	-82.5224705858962\\
-72.021	-69.5543640541941\\
-47.607	-45.9765153153666\\
-28.076	-27.114429474536\\
-21.973	-21.2204501654074\\
-26.855	-25.9352473122476\\
-25.635	-24.7570309011159\\
-32.959	-31.8301923725328\\
-47.607	-45.9765153153666\\
-43.945	-42.4399345796582\\
-45.166	-43.6191167419465\\
-52.49	-50.6922782133635\\
-35.4	-34.1875909459529\\
-20.752	-20.041268003119\\
-35.4	-34.1875909459529\\
-56.152	-54.2288589490719\\
-54.932	-53.0506425379402\\
-59.814	-57.7654396847804\\
-53.711	-51.8714603756518\\
-40.283	-38.9033538439497\\
-51.27	-49.5140618022318\\
-72.021	-69.5543640541941\\
-70.801	-68.3761476430624\\
-51.27	-49.5140618022318\\
-81.787	-78.9858898501878\\
-62.256	-60.1238040093571\\
-63.477	-61.3029861716455\\
-73.242	-70.7335462164825\\
-70.801	-68.3761476430624\\
-54.932	-53.0506425379402\\
-47.607	-45.9765153153666\\
-78.125	-75.4493091144793\\
-108.643	-104.922102913592\\
-85.449	-82.5224705858962\\
-50.049	-48.3348796399434\\
-51.27	-49.5140618022318\\
-61.035	-58.9446218470687\\
-72.021	-69.5543640541941\\
-54.932	-53.0506425379402\\
-56.152	-54.2288589490719\\
-73.242	-70.7335462164825\\
-79.346	-76.6284912767677\\
-56.152	-54.2288589490719\\
-45.166	-43.6191167419465\\
-57.373	-55.4080411113603\\
-87.891	-84.880834910473\\
-78.125	-75.4493091144793\\
-80.566	-77.8067076878994\\
-54.932	-53.0506425379402\\
-45.166	-43.6191167419465\\
-47.607	-45.9765153153666\\
-31.738	-30.6510102102444\\
-20.752	-20.041268003119\\
-17.09	-16.5046872674106\\
-31.738	-30.6510102102444\\
-48.828	-47.155697477655\\
-54.932	-53.0506425379402\\
-40.283	-38.9033538439497\\
-45.166	-43.6191167419465\\
-52.49	-50.6922782133635\\
-35.4	-34.1875909459529\\
-29.297	-28.2936116368243\\
-31.738	-30.6510102102444\\
-51.27	-49.5140618022318\\
-64.697	-62.4812025827772\\
-41.504	-40.0825360062381\\
-32.959	-31.8301923725328\\
-28.076	-27.114429474536\\
-32.959	-31.8301923725328\\
-26.855	-25.9352473122476\\
-45.166	-43.6191167419465\\
-80.566	-77.8067076878994\\
-63.477	-61.3029861716455\\
-81.787	-78.9858898501878\\
-98.877	-95.4905771175984\\
-97.656	-94.31139495531\\
-76.904	-74.2701269521909\\
-58.594	-56.5872232736487\\
-54.932	-53.0506425379402\\
-57.373	-55.4080411113603\\
-65.918	-63.6603847450656\\
-76.904	-74.2701269521909\\
-100.098	-96.6697592798867\\
-108.643	-104.922102913592\\
-130.615	-126.141587327843\\
-91.553	-88.4174156461815\\
-98.877	-95.4905771175984\\
-108.643	-104.922102913592\\
-85.449	-82.5224705858962\\
-96.436	-93.1331785441783\\
-85.449	-82.5224705858962\\
-51.27	-49.5140618022318\\
-34.18	-33.0093745348212\\
-36.621	-35.3667731082412\\
-51.27	-49.5140618022318\\
-43.945	-42.4399345796582\\
-31.738	-30.6510102102444\\
-41.504	-40.0825360062381\\
-32.959	-31.8301923725328\\
-23.193	-22.3986665765391\\
-31.738	-30.6510102102444\\
-34.18	-33.0093745348212\\
-25.635	-24.7570309011159\\
-42.725	-41.2617181685265\\
-58.594	-56.5872232736487\\
-45.166	-43.6191167419465\\
-68.359	-66.0177833184857\\
-100.098	-96.6697592798867\\
-87.891	-84.880834910473\\
-109.863	-106.100319324724\\
-81.787	-78.9858898501878\\
-62.256	-60.1238040093571\\
-37.842	-36.5459552705296\\
-46.387	-44.7982989042349\\
-42.725	-41.2617181685265\\
-29.297	-28.2936116368243\\
-40.283	-38.9033538439497\\
-21.973	-21.2204501654074\\
-19.531	-18.8620858408307\\
-24.414	-23.5778487388275\\
-45.166	-43.6191167419465\\
-57.373	-55.4080411113603\\
-65.918	-63.6603847450656\\
-62.256	-60.1238040093571\\
-69.58	-67.196965480774\\
-57.373	-55.4080411113603\\
-48.828	-47.155697477655\\
-50.049	-48.3348796399434\\
-43.945	-42.4399345796582\\
-59.814	-57.7654396847804\\
-46.387	-44.7982989042349\\
-36.621	-35.3667731082412\\
-51.27	-49.5140618022318\\
-69.58	-67.196965480774\\
-54.932	-53.0506425379402\\
-41.504	-40.0825360062381\\
-37.842	-36.5459552705296\\
-41.504	-40.0825360062381\\
-32.959	-31.8301923725328\\
-30.518	-29.4727937991127\\
-41.504	-40.0825360062381\\
-47.607	-45.9765153153666\\
-52.49	-50.6922782133635\\
-42.725	-41.2617181685265\\
-53.711	-51.8714603756518\\
-51.27	-49.5140618022318\\
-32.959	-31.8301923725328\\
-58.594	-56.5872232736487\\
-80.566	-77.8067076878994\\
-78.125	-75.4493091144793\\
-67.139	-64.839566907354\\
-62.256	-60.1238040093571\\
-50.049	-48.3348796399434\\
-48.828	-47.155697477655\\
-41.504	-40.0825360062381\\
-54.932	-53.0506425379402\\
-50.049	-48.3348796399434\\
-32.959	-31.8301923725328\\
-45.166	-43.6191167419465\\
-52.49	-50.6922782133635\\
-61.035	-58.9446218470687\\
-68.359	-66.0177833184857\\
-87.891	-84.880834910473\\
-107.422	-103.742920751304\\
-119.629	-115.531845120717\\
-80.566	-77.8067076878994\\
-46.387	-44.7982989042349\\
-32.959	-31.8301923725328\\
-24.414	-23.5778487388275\\
-34.18	-33.0093745348212\\
-29.297	-28.2936116368243\\
-51.27	-49.5140618022318\\
-48.828	-47.155697477655\\
-40.283	-38.9033538439497\\
-52.49	-50.6922782133635\\
-63.477	-61.3029861716455\\
-70.801	-68.3761476430624\\
-75.684	-73.0919105410592\\
-103.76	-100.206340015595\\
-92.773	-89.5956320573132\\
-72.021	-69.5543640541941\\
-51.27	-49.5140618022318\\
-52.49	-50.6922782133635\\
-54.932	-53.0506425379402\\
-65.918	-63.6603847450656\\
-51.27	-49.5140618022318\\
-50.049	-48.3348796399434\\
-48.828	-47.155697477655\\
-30.518	-29.4727937991127\\
-45.166	-43.6191167419465\\
-68.359	-66.0177833184857\\
-48.828	-47.155697477655\\
-52.49	-50.6922782133635\\
-86.67	-83.7016527481846\\
-63.477	-61.3029861716455\\
-85.449	-82.5224705858962\\
-81.787	-78.9858898501878\\
-53.711	-51.8714603756518\\
-35.4	-34.1875909459529\\
-36.621	-35.3667731082412\\
-61.035	-58.9446218470687\\
-90.332	-87.2382334838931\\
-97.656	-94.31139495531\\
-100.098	-96.6697592798867\\
-79.346	-76.6284912767677\\
-53.711	-51.8714603756518\\
-46.387	-44.7982989042349\\
-37.842	-36.5459552705296\\
-34.18	-33.0093745348212\\
-29.297	-28.2936116368243\\
-40.283	-38.9033538439497\\
-42.725	-41.2617181685265\\
-30.518	-29.4727937991127\\
-42.725	-41.2617181685265\\
-56.152	-54.2288589490719\\
-62.256	-60.1238040093571\\
-78.125	-75.4493091144793\\
-96.436	-93.1331785441783\\
-114.746	-110.816082222721\\
-84.229	-81.3442541747645\\
-73.242	-70.7335462164825\\
-76.904	-74.2701269521909\\
-63.477	-61.3029861716455\\
-46.387	-44.7982989042349\\
-57.373	-55.4080411113603\\
-90.332	-87.2382334838931\\
-101.318	-97.8479756910184\\
-61.035	-58.9446218470687\\
-36.621	-35.3667731082412\\
-26.855	-25.9352473122476\\
-18.311	-17.683869429699\\
-21.973	-21.2204501654074\\
-30.518	-29.4727937991127\\
-31.738	-30.6510102102444\\
-42.725	-41.2617181685265\\
-37.842	-36.5459552705296\\
-30.518	-29.4727937991127\\
-29.297	-28.2936116368243\\
-26.855	-25.9352473122476\\
-31.738	-30.6510102102444\\
-43.945	-42.4399345796582\\
-63.477	-61.3029861716455\\
-68.359	-66.0177833184857\\
-65.918	-63.6603847450656\\
-74.463	-71.9127283787709\\
-91.553	-88.4174156461815\\
-92.773	-89.5956320573132\\
-106.201	-102.563738589015\\
-100.098	-96.6697592798867\\
-74.463	-71.9127283787709\\
-54.932	-53.0506425379402\\
-51.27	-49.5140618022318\\
-84.229	-81.3442541747645\\
-97.656	-94.31139495531\\
-69.58	-67.196965480774\\
-46.387	-44.7982989042349\\
-26.855	-25.9352473122476\\
-21.973	-21.2204501654074\\
-20.752	-20.041268003119\\
-14.648	-14.1463229428338\\
-25.635	-24.7570309011159\\
-40.283	-38.9033538439497\\
-35.4	-34.1875909459529\\
-24.414	-23.5778487388275\\
-19.531	-18.8620858408307\\
-28.076	-27.114429474536\\
-30.518	-29.4727937991127\\
-24.414	-23.5778487388275\\
-21.973	-21.2204501654074\\
-32.959	-31.8301923725328\\
-69.58	-67.196965480774\\
-79.346	-76.6284912767677\\
-86.67	-83.7016527481846\\
-63.477	-61.3029861716455\\
-80.566	-77.8067076878994\\
-114.746	-110.816082222721\\
-103.76	-100.206340015595\\
-106.201	-102.563738589015\\
-81.787	-78.9858898501878\\
-86.67	-83.7016527481846\\
-59.814	-57.7654396847804\\
-56.152	-54.2288589490719\\
-40.283	-38.9033538439497\\
-35.4	-34.1875909459529\\
-62.256	-60.1238040093571\\
-47.607	-45.9765153153666\\
-48.828	-47.155697477655\\
-53.711	-51.8714603756518\\
-50.049	-48.3348796399434\\
-52.49	-50.6922782133635\\
-57.373	-55.4080411113603\\
-64.697	-62.4812025827772\\
-56.152	-54.2288589490719\\
-41.504	-40.0825360062381\\
-52.49	-50.6922782133635\\
-31.738	-30.6510102102444\\
-18.311	-17.683869429699\\
-25.635	-24.7570309011159\\
-34.18	-33.0093745348212\\
-21.973	-21.2204501654074\\
-9.766	-9.43152579599367\\
-21.973	-21.2204501654074\\
-32.959	-31.8301923725328\\
-37.842	-36.5459552705296\\
-43.945	-42.4399345796582\\
-39.063	-37.725137432818\\
-58.594	-56.5872232736487\\
-48.828	-47.155697477655\\
-36.621	-35.3667731082412\\
-54.932	-53.0506425379402\\
-39.063	-37.725137432818\\
-26.855	-25.9352473122476\\
-23.193	-22.3986665765391\\
-17.09	-16.5046872674106\\
-25.635	-24.7570309011159\\
-19.531	-18.8620858408307\\
-29.297	-28.2936116368243\\
-43.945	-42.4399345796582\\
-42.725	-41.2617181685265\\
-32.959	-31.8301923725328\\
-40.283	-38.9033538439497\\
-29.297	-28.2936116368243\\
-43.945	-42.4399345796582\\
-31.738	-30.6510102102444\\
-30.518	-29.4727937991127\\
-24.414	-23.5778487388275\\
-26.855	-25.9352473122476\\
-41.504	-40.0825360062381\\
-36.621	-35.3667731082412\\
-30.518	-29.4727937991127\\
-32.959	-31.8301923725328\\
-26.855	-25.9352473122476\\
-31.738	-30.6510102102444\\
-58.594	-56.5872232736487\\
-75.684	-73.0919105410592\\
-51.27	-49.5140618022318\\
-48.828	-47.155697477655\\
-39.063	-37.725137432818\\
-58.594	-56.5872232736487\\
-37.842	-36.5459552705296\\
-46.387	-44.7982989042349\\
-39.063	-37.725137432818\\
-51.27	-49.5140618022318\\
-32.959	-31.8301923725328\\
-35.4	-34.1875909459529\\
-37.842	-36.5459552705296\\
-47.607	-45.9765153153666\\
-36.621	-35.3667731082412\\
-41.504	-40.0825360062381\\
-57.373	-55.4080411113603\\
-51.27	-49.5140618022318\\
-72.021	-69.5543640541941\\
-93.994	-90.7748142196015\\
-78.125	-75.4493091144793\\
-50.049	-48.3348796399434\\
-40.283	-38.9033538439497\\
-54.932	-53.0506425379402\\
-69.58	-67.196965480774\\
-56.152	-54.2288589490719\\
-63.477	-61.3029861716455\\
-43.945	-42.4399345796582\\
-23.193	-22.3986665765391\\
-24.414	-23.5778487388275\\
-52.49	-50.6922782133635\\
-45.166	-43.6191167419465\\
-42.725	-41.2617181685265\\
-64.697	-62.4812025827772\\
-61.035	-58.9446218470687\\
-56.152	-54.2288589490719\\
-39.063	-37.725137432818\\
-53.711	-51.8714603756518\\
-65.918	-63.6603847450656\\
-52.49	-50.6922782133635\\
-53.711	-51.8714603756518\\
-48.828	-47.155697477655\\
-37.842	-36.5459552705296\\
-42.725	-41.2617181685265\\
-35.4	-34.1875909459529\\
-36.621	-35.3667731082412\\
-57.373	-55.4080411113603\\
-67.139	-64.839566907354\\
-70.801	-68.3761476430624\\
-62.256	-60.1238040093571\\
-45.166	-43.6191167419465\\
-35.4	-34.1875909459529\\
-39.063	-37.725137432818\\
-46.387	-44.7982989042349\\
-57.373	-55.4080411113603\\
-68.359	-66.0177833184857\\
-79.346	-76.6284912767677\\
-65.918	-63.6603847450656\\
-40.283	-38.9033538439497\\
-68.359	-66.0177833184857\\
-91.553	-88.4174156461815\\
-65.918	-63.6603847450656\\
-64.697	-62.4812025827772\\
-75.684	-73.0919105410592\\
-59.814	-57.7654396847804\\
-50.049	-48.3348796399434\\
-56.152	-54.2288589490719\\
-51.27	-49.5140618022318\\
-57.373	-55.4080411113603\\
-70.801	-68.3761476430624\\
-56.152	-54.2288589490719\\
-63.477	-61.3029861716455\\
-58.594	-56.5872232736487\\
-59.814	-57.7654396847804\\
-48.828	-47.155697477655\\
-63.477	-61.3029861716455\\
-45.166	-43.6191167419465\\
-48.828	-47.155697477655\\
-52.49	-50.6922782133635\\
-50.049	-48.3348796399434\\
-30.518	-29.4727937991127\\
-19.531	-18.8620858408307\\
-15.869	-15.3255051051222\\
-28.076	-27.114429474536\\
-53.711	-51.8714603756518\\
-67.139	-64.839566907354\\
-65.918	-63.6603847450656\\
-46.387	-44.7982989042349\\
-54.932	-53.0506425379402\\
-50.049	-48.3348796399434\\
-43.945	-42.4399345796582\\
-30.518	-29.4727937991127\\
-42.725	-41.2617181685265\\
-37.842	-36.5459552705296\\
-36.621	-35.3667731082412\\
-41.504	-40.0825360062381\\
-37.842	-36.5459552705296\\
-34.18	-33.0093745348212\\
-25.635	-24.7570309011159\\
-35.4	-34.1875909459529\\
-58.594	-56.5872232736487\\
-42.725	-41.2617181685265\\
-52.49	-50.6922782133635\\
-45.166	-43.6191167419465\\
-59.814	-57.7654396847804\\
-57.373	-55.4080411113603\\
-76.904	-74.2701269521909\\
-58.594	-56.5872232736487\\
-64.697	-62.4812025827772\\
-65.918	-63.6603847450656\\
-37.842	-36.5459552705296\\
-61.035	-58.9446218470687\\
-46.387	-44.7982989042349\\
-51.27	-49.5140618022318\\
-57.373	-55.4080411113603\\
-69.58	-67.196965480774\\
-53.711	-51.8714603756518\\
-70.801	-68.3761476430624\\
-81.787	-78.9858898501878\\
-69.58	-67.196965480774\\
-58.594	-56.5872232736487\\
-50.049	-48.3348796399434\\
-58.594	-56.5872232736487\\
-52.49	-50.6922782133635\\
-43.945	-42.4399345796582\\
-37.842	-36.5459552705296\\
-45.166	-43.6191167419465\\
-39.063	-37.725137432818\\
-72.021	-69.5543640541941\\
-90.332	-87.2382334838931\\
-79.346	-76.6284912767677\\
-75.684	-73.0919105410592\\
-74.463	-71.9127283787709\\
-46.387	-44.7982989042349\\
-42.725	-41.2617181685265\\
-35.4	-34.1875909459529\\
-30.518	-29.4727937991127\\
-32.959	-31.8301923725328\\
-52.49	-50.6922782133635\\
-59.814	-57.7654396847804\\
-70.801	-68.3761476430624\\
-59.814	-57.7654396847804\\
-42.725	-41.2617181685265\\
-72.021	-69.5543640541941\\
-84.229	-81.3442541747645\\
-90.332	-87.2382334838931\\
-73.242	-70.7335462164825\\
-115.967	-111.995264385009\\
-85.449	-82.5224705858962\\
-51.27	-49.5140618022318\\
-39.063	-37.725137432818\\
-34.18	-33.0093745348212\\
-56.152	-54.2288589490719\\
-69.58	-67.196965480774\\
-89.111	-86.0590513216047\\
-68.359	-66.0177833184857\\
-53.711	-51.8714603756518\\
-31.738	-30.6510102102444\\
-36.621	-35.3667731082412\\
-34.18	-33.0093745348212\\
-39.063	-37.725137432818\\
-25.635	-24.7570309011159\\
-35.4	-34.1875909459529\\
-26.855	-25.9352473122476\\
-19.531	-18.8620858408307\\
-39.063	-37.725137432818\\
-42.725	-41.2617181685265\\
-48.828	-47.155697477655\\
-46.387	-44.7982989042349\\
-48.828	-47.155697477655\\
-62.256	-60.1238040093571\\
-43.945	-42.4399345796582\\
-62.256	-60.1238040093571\\
-69.58	-67.196965480774\\
-56.152	-54.2288589490719\\
-58.594	-56.5872232736487\\
-54.932	-53.0506425379402\\
-61.035	-58.9446218470687\\
-80.566	-77.8067076878994\\
-62.256	-60.1238040093571\\
-50.049	-48.3348796399434\\
-57.373	-55.4080411113603\\
-51.27	-49.5140618022318\\
-36.621	-35.3667731082412\\
-56.152	-54.2288589490719\\
-74.463	-71.9127283787709\\
-69.58	-67.196965480774\\
-78.125	-75.4493091144793\\
-113.525	-109.636900060432\\
-79.346	-76.6284912767677\\
-72.021	-69.5543640541941\\
-56.152	-54.2288589490719\\
-68.359	-66.0177833184857\\
-92.773	-89.5956320573132\\
-64.697	-62.4812025827772\\
-57.373	-55.4080411113603\\
-76.904	-74.2701269521909\\
-51.27	-49.5140618022318\\
-32.959	-31.8301923725328\\
-42.725	-41.2617181685265\\
-32.959	-31.8301923725328\\
-25.635	-24.7570309011159\\
-28.076	-27.114429474536\\
-25.635	-24.7570309011159\\
-31.738	-30.6510102102444\\
-42.725	-41.2617181685265\\
-48.828	-47.155697477655\\
-63.477	-61.3029861716455\\
-59.814	-57.7654396847804\\
-67.139	-64.839566907354\\
-79.346	-76.6284912767677\\
-73.242	-70.7335462164825\\
-54.932	-53.0506425379402\\
-58.594	-56.5872232736487\\
-53.711	-51.8714603756518\\
-57.373	-55.4080411113603\\
-76.904	-74.2701269521909\\
-59.814	-57.7654396847804\\
-63.477	-61.3029861716455\\
-62.256	-60.1238040093571\\
-57.373	-55.4080411113603\\
-84.229	-81.3442541747645\\
-69.58	-67.196965480774\\
-73.242	-70.7335462164825\\
-65.918	-63.6603847450656\\
-42.725	-41.2617181685265\\
-29.297	-28.2936116368243\\
-24.414	-23.5778487388275\\
-41.504	-40.0825360062381\\
-34.18	-33.0093745348212\\
-47.607	-45.9765153153666\\
-35.4	-34.1875909459529\\
-45.166	-43.6191167419465\\
-79.346	-76.6284912767677\\
-93.994	-90.7748142196015\\
-75.684	-73.0919105410592\\
-79.346	-76.6284912767677\\
-70.801	-68.3761476430624\\
-52.49	-50.6922782133635\\
-54.932	-53.0506425379402\\
-59.814	-57.7654396847804\\
-53.711	-51.8714603756518\\
-62.256	-60.1238040093571\\
-76.904	-74.2701269521909\\
-106.201	-102.563738589015\\
-93.994	-90.7748142196015\\
-87.891	-84.880834910473\\
-97.656	-94.31139495531\\
-68.359	-66.0177833184857\\
-74.463	-71.9127283787709\\
-59.814	-57.7654396847804\\
-75.684	-73.0919105410592\\
-85.449	-82.5224705858962\\
-37.842	-36.5459552705296\\
-48.828	-47.155697477655\\
-37.842	-36.5459552705296\\
-53.711	-51.8714603756518\\
-35.4	-34.1875909459529\\
-46.387	-44.7982989042349\\
-74.463	-71.9127283787709\\
-118.408	-114.352662958429\\
-106.201	-102.563738589015\\
-85.449	-82.5224705858962\\
-97.656	-94.31139495531\\
-115.967	-111.995264385009\\
-86.67	-83.7016527481846\\
-89.111	-86.0590513216047\\
-91.553	-88.4174156461815\\
-64.697	-62.4812025827772\\
-56.152	-54.2288589490719\\
-72.021	-69.5543640541941\\
-48.828	-47.155697477655\\
-37.842	-36.5459552705296\\
-52.49	-50.6922782133635\\
-39.063	-37.725137432818\\
-47.607	-45.9765153153666\\
-41.504	-40.0825360062381\\
-51.27	-49.5140618022318\\
-65.918	-63.6603847450656\\
-52.49	-50.6922782133635\\
-41.504	-40.0825360062381\\
-50.049	-48.3348796399434\\
-57.373	-55.4080411113603\\
-65.918	-63.6603847450656\\
-56.152	-54.2288589490719\\
-46.387	-44.7982989042349\\
-70.801	-68.3761476430624\\
-65.918	-63.6603847450656\\
-47.607	-45.9765153153666\\
-78.125	-75.4493091144793\\
-67.139	-64.839566907354\\
-53.711	-51.8714603756518\\
-37.842	-36.5459552705296\\
-29.297	-28.2936116368243\\
-23.193	-22.3986665765391\\
-41.504	-40.0825360062381\\
-36.621	-35.3667731082412\\
-31.738	-30.6510102102444\\
-24.414	-23.5778487388275\\
-31.738	-30.6510102102444\\
-46.387	-44.7982989042349\\
-51.27	-49.5140618022318\\
-63.477	-61.3029861716455\\
-73.242	-70.7335462164825\\
-72.021	-69.5543640541941\\
-74.463	-71.9127283787709\\
-47.607	-45.9765153153666\\
-23.193	-22.3986665765391\\
-35.4	-34.1875909459529\\
-26.855	-25.9352473122476\\
-37.842	-36.5459552705296\\
-52.49	-50.6922782133635\\
-58.594	-56.5872232736487\\
-85.449	-82.5224705858962\\
-56.152	-54.2288589490719\\
-57.373	-55.4080411113603\\
-74.463	-71.9127283787709\\
-57.373	-55.4080411113603\\
-41.504	-40.0825360062381\\
-34.18	-33.0093745348212\\
-45.166	-43.6191167419465\\
-54.932	-53.0506425379402\\
-79.346	-76.6284912767677\\
-53.711	-51.8714603756518\\
-47.607	-45.9765153153666\\
-45.166	-43.6191167419465\\
-50.049	-48.3348796399434\\
-68.359	-66.0177833184857\\
-103.76	-100.206340015595\\
-89.111	-86.0590513216047\\
-90.332	-87.2382334838931\\
-59.814	-57.7654396847804\\
-43.945	-42.4399345796582\\
-48.828	-47.155697477655\\
-21.973	-21.2204501654074\\
-39.063	-37.725137432818\\
-30.518	-29.4727937991127\\
-36.621	-35.3667731082412\\
-58.594	-56.5872232736487\\
-76.904	-74.2701269521909\\
-86.67	-83.7016527481846\\
-111.084	-107.279501487012\\
-93.994	-90.7748142196015\\
-53.711	-51.8714603756518\\
-42.725	-41.2617181685265\\
-57.373	-55.4080411113603\\
-73.242	-70.7335462164825\\
-95.215	-91.9539963818899\\
-72.021	-69.5543640541941\\
-52.49	-50.6922782133635\\
-59.814	-57.7654396847804\\
-75.684	-73.0919105410592\\
-54.932	-53.0506425379402\\
-40.283	-38.9033538439497\\
-35.4	-34.1875909459529\\
-25.635	-24.7570309011159\\
-24.414	-23.5778487388275\\
-19.531	-18.8620858408307\\
-32.959	-31.8301923725328\\
-41.504	-40.0825360062381\\
-46.387	-44.7982989042349\\
-36.621	-35.3667731082412\\
-35.4	-34.1875909459529\\
-19.531	-18.8620858408307\\
-13.428	-12.9681065317021\\
-20.752	-20.041268003119\\
-29.297	-28.2936116368243\\
-39.063	-37.725137432818\\
-48.828	-47.155697477655\\
-52.49	-50.6922782133635\\
-62.256	-60.1238040093571\\
-79.346	-76.6284912767677\\
-107.422	-103.742920751304\\
-103.76	-100.206340015595\\
-112.305	-108.4586836493\\
-98.877	-95.4905771175984\\
-61.035	-58.9446218470687\\
-54.932	-53.0506425379402\\
-53.711	-51.8714603756518\\
-69.58	-67.196965480774\\
-59.814	-57.7654396847804\\
-45.166	-43.6191167419465\\
-36.621	-35.3667731082412\\
-31.738	-30.6510102102444\\
-50.049	-48.3348796399434\\
-47.607	-45.9765153153666\\
-28.076	-27.114429474536\\
-21.973	-21.2204501654074\\
-32.959	-31.8301923725328\\
-43.945	-42.4399345796582\\
-32.959	-31.8301923725328\\
-46.387	-44.7982989042349\\
-36.621	-35.3667731082412\\
-51.27	-49.5140618022318\\
-76.904	-74.2701269521909\\
-61.035	-58.9446218470687\\
-80.566	-77.8067076878994\\
-104.98	-101.384556426727\\
-113.525	-109.636900060432\\
-87.891	-84.880834910473\\
-59.814	-57.7654396847804\\
-45.166	-43.6191167419465\\
-31.738	-30.6510102102444\\
-40.283	-38.9033538439497\\
-30.518	-29.4727937991127\\
-54.932	-53.0506425379402\\
-47.607	-45.9765153153666\\
-40.283	-38.9033538439497\\
-51.27	-49.5140618022318\\
-32.959	-31.8301923725328\\
-45.166	-43.6191167419465\\
-34.18	-33.0093745348212\\
-23.193	-22.3986665765391\\
-25.635	-24.7570309011159\\
-21.973	-21.2204501654074\\
-23.193	-22.3986665765391\\
-24.414	-23.5778487388275\\
-17.09	-16.5046872674106\\
-21.973	-21.2204501654074\\
-29.297	-28.2936116368243\\
-30.518	-29.4727937991127\\
-29.297	-28.2936116368243\\
-43.945	-42.4399345796582\\
-54.932	-53.0506425379402\\
-45.166	-43.6191167419465\\
-42.725	-41.2617181685265\\
-62.256	-60.1238040093571\\
-74.463	-71.9127283787709\\
-59.814	-57.7654396847804\\
-65.918	-63.6603847450656\\
-32.959	-31.8301923725328\\
-21.973	-21.2204501654074\\
-40.283	-38.9033538439497\\
-56.152	-54.2288589490719\\
-58.594	-56.5872232736487\\
-83.008	-80.1650720124762\\
-74.463	-71.9127283787709\\
-53.711	-51.8714603756518\\
-40.283	-38.9033538439497\\
-42.725	-41.2617181685265\\
-46.387	-44.7982989042349\\
-36.621	-35.3667731082412\\
-24.414	-23.5778487388275\\
-32.959	-31.8301923725328\\
-25.635	-24.7570309011159\\
-21.973	-21.2204501654074\\
-18.311	-17.683869429699\\
-28.076	-27.114429474536\\
-40.283	-38.9033538439497\\
-43.945	-42.4399345796582\\
-31.738	-30.6510102102444\\
-52.49	-50.6922782133635\\
-70.801	-68.3761476430624\\
-47.607	-45.9765153153666\\
-64.697	-62.4812025827772\\
-70.801	-68.3761476430624\\
-51.27	-49.5140618022318\\
-34.18	-33.0093745348212\\
-42.725	-41.2617181685265\\
-47.607	-45.9765153153666\\
-29.297	-28.2936116368243\\
-34.18	-33.0093745348212\\
-28.076	-27.114429474536\\
-19.531	-18.8620858408307\\
-29.297	-28.2936116368243\\
-36.621	-35.3667731082412\\
-32.959	-31.8301923725328\\
-39.063	-37.725137432818\\
-42.725	-41.2617181685265\\
-32.959	-31.8301923725328\\
-20.752	-20.041268003119\\
-18.311	-17.683869429699\\
-21.973	-21.2204501654074\\
-25.635	-24.7570309011159\\
-29.297	-28.2936116368243\\
-54.932	-53.0506425379402\\
-53.711	-51.8714603756518\\
-47.607	-45.9765153153666\\
-61.035	-58.9446218470687\\
-76.904	-74.2701269521909\\
-83.008	-80.1650720124762\\
-85.449	-82.5224705858962\\
-64.697	-62.4812025827772\\
-65.918	-63.6603847450656\\
-67.139	-64.839566907354\\
-69.58	-67.196965480774\\
-76.904	-74.2701269521909\\
-101.318	-97.8479756910184\\
-86.67	-83.7016527481846\\
-79.346	-76.6284912767677\\
-63.477	-61.3029861716455\\
-59.814	-57.7654396847804\\
-70.801	-68.3761476430624\\
-46.387	-44.7982989042349\\
-53.711	-51.8714603756518\\
-62.256	-60.1238040093571\\
-72.021	-69.5543640541941\\
-56.152	-54.2288589490719\\
-65.918	-63.6603847450656\\
-52.49	-50.6922782133635\\
-48.828	-47.155697477655\\
-32.959	-31.8301923725328\\
-51.27	-49.5140618022318\\
-73.242	-70.7335462164825\\
-57.373	-55.4080411113603\\
-36.621	-35.3667731082412\\
-43.945	-42.4399345796582\\
-28.076	-27.114429474536\\
-25.635	-24.7570309011159\\
-32.959	-31.8301923725328\\
-23.193	-22.3986665765391\\
-25.635	-24.7570309011159\\
-28.076	-27.114429474536\\
-19.531	-18.8620858408307\\
-29.297	-28.2936116368243\\
-25.635	-24.7570309011159\\
-19.531	-18.8620858408307\\
-28.076	-27.114429474536\\
-31.738	-30.6510102102444\\
-35.4	-34.1875909459529\\
-36.621	-35.3667731082412\\
-35.4	-34.1875909459529\\
-46.387	-44.7982989042349\\
-40.283	-38.9033538439497\\
-47.607	-45.9765153153666\\
-75.684	-73.0919105410592\\
-56.152	-54.2288589490719\\
-50.049	-48.3348796399434\\
-52.49	-50.6922782133635\\
-40.283	-38.9033538439497\\
-26.855	-25.9352473122476\\
-42.725	-41.2617181685265\\
-35.4	-34.1875909459529\\
-25.635	-24.7570309011159\\
-37.842	-36.5459552705296\\
-36.621	-35.3667731082412\\
-42.725	-41.2617181685265\\
-30.518	-29.4727937991127\\
-21.973	-21.2204501654074\\
-18.311	-17.683869429699\\
-20.752	-20.041268003119\\
-34.18	-33.0093745348212\\
-31.738	-30.6510102102444\\
-35.4	-34.1875909459529\\
-47.607	-45.9765153153666\\
-63.477	-61.3029861716455\\
-48.828	-47.155697477655\\
-63.477	-61.3029861716455\\
-74.463	-71.9127283787709\\
-64.697	-62.4812025827772\\
-59.814	-57.7654396847804\\
-97.656	-94.31139495531\\
-70.801	-68.3761476430624\\
-86.67	-83.7016527481846\\
-76.904	-74.2701269521909\\
-86.67	-83.7016527481846\\
-95.215	-91.9539963818899\\
-61.035	-58.9446218470687\\
-39.063	-37.725137432818\\
-32.959	-31.8301923725328\\
-47.607	-45.9765153153666\\
-57.373	-55.4080411113603\\
-58.594	-56.5872232736487\\
-41.504	-40.0825360062381\\
-25.635	-24.7570309011159\\
-34.18	-33.0093745348212\\
-57.373	-55.4080411113603\\
-76.904	-74.2701269521909\\
-75.684	-73.0919105410592\\
-48.828	-47.155697477655\\
-39.063	-37.725137432818\\
-48.828	-47.155697477655\\
-72.021	-69.5543640541941\\
-81.787	-78.9858898501878\\
-83.008	-80.1650720124762\\
-61.035	-58.9446218470687\\
-83.008	-80.1650720124762\\
-90.332	-87.2382334838931\\
-86.67	-83.7016527481846\\
-114.746	-110.816082222721\\
-98.877	-95.4905771175984\\
-85.449	-82.5224705858962\\
-98.877	-95.4905771175984\\
-84.229	-81.3442541747645\\
-51.27	-49.5140618022318\\
-47.607	-45.9765153153666\\
-56.152	-54.2288589490719\\
-67.139	-64.839566907354\\
-51.27	-49.5140618022318\\
-63.477	-61.3029861716455\\
-56.152	-54.2288589490719\\
-43.945	-42.4399345796582\\
-54.932	-53.0506425379402\\
-52.49	-50.6922782133635\\
-75.684	-73.0919105410592\\
-76.904	-74.2701269521909\\
-57.373	-55.4080411113603\\
-45.166	-43.6191167419465\\
-30.518	-29.4727937991127\\
-43.945	-42.4399345796582\\
-35.4	-34.1875909459529\\
-31.738	-30.6510102102444\\
-26.855	-25.9352473122476\\
-42.725	-41.2617181685265\\
-53.711	-51.8714603756518\\
-58.594	-56.5872232736487\\
-37.842	-36.5459552705296\\
-30.518	-29.4727937991127\\
-23.193	-22.3986665765391\\
-28.076	-27.114429474536\\
-40.283	-38.9033538439497\\
-42.725	-41.2617181685265\\
-40.283	-38.9033538439497\\
-62.256	-60.1238040093571\\
-73.242	-70.7335462164825\\
-80.566	-77.8067076878994\\
-81.787	-78.9858898501878\\
-95.215	-91.9539963818899\\
-59.814	-57.7654396847804\\
-47.607	-45.9765153153666\\
-40.283	-38.9033538439497\\
-43.945	-42.4399345796582\\
-54.932	-53.0506425379402\\
-46.387	-44.7982989042349\\
-34.18	-33.0093745348212\\
-30.518	-29.4727937991127\\
-52.49	-50.6922782133635\\
-62.256	-60.1238040093571\\
-57.373	-55.4080411113603\\
-85.449	-82.5224705858962\\
-91.553	-88.4174156461815\\
-65.918	-63.6603847450656\\
-47.607	-45.9765153153666\\
-36.621	-35.3667731082412\\
-32.959	-31.8301923725328\\
-56.152	-54.2288589490719\\
-48.828	-47.155697477655\\
-53.711	-51.8714603756518\\
-59.814	-57.7654396847804\\
-67.139	-64.839566907354\\
-63.477	-61.3029861716455\\
-72.021	-69.5543640541941\\
-51.27	-49.5140618022318\\
-56.152	-54.2288589490719\\
-42.725	-41.2617181685265\\
-39.063	-37.725137432818\\
-54.932	-53.0506425379402\\
-75.684	-73.0919105410592\\
-81.787	-78.9858898501878\\
-47.607	-45.9765153153666\\
-91.553	-88.4174156461815\\
-78.125	-75.4493091144793\\
-53.711	-51.8714603756518\\
-34.18	-33.0093745348212\\
-50.049	-48.3348796399434\\
-45.166	-43.6191167419465\\
-53.711	-51.8714603756518\\
-37.842	-36.5459552705296\\
-48.828	-47.155697477655\\
-86.67	-83.7016527481846\\
-91.553	-88.4174156461815\\
-63.477	-61.3029861716455\\
-70.801	-68.3761476430624\\
-78.125	-75.4493091144793\\
-76.904	-74.2701269521909\\
-57.373	-55.4080411113603\\
-56.152	-54.2288589490719\\
-41.504	-40.0825360062381\\
-54.932	-53.0506425379402\\
-59.814	-57.7654396847804\\
-92.773	-89.5956320573132\\
-101.318	-97.8479756910184\\
-131.836	-127.320769490131\\
-90.332	-87.2382334838931\\
-76.904	-74.2701269521909\\
-62.256	-60.1238040093571\\
-67.139	-64.839566907354\\
-52.49	-50.6922782133635\\
-37.842	-36.5459552705296\\
-39.063	-37.725137432818\\
-31.738	-30.6510102102444\\
-23.193	-22.3986665765391\\
-34.18	-33.0093745348212\\
-45.166	-43.6191167419465\\
-29.297	-28.2936116368243\\
-25.635	-24.7570309011159\\
-24.414	-23.5778487388275\\
-50.049	-48.3348796399434\\
-67.139	-64.839566907354\\
-35.4	-34.1875909459529\\
-41.504	-40.0825360062381\\
-43.945	-42.4399345796582\\
-40.283	-38.9033538439497\\
-47.607	-45.9765153153666\\
-45.166	-43.6191167419465\\
-32.959	-31.8301923725328\\
-25.635	-24.7570309011159\\
-23.193	-22.3986665765391\\
-26.855	-25.9352473122476\\
-47.607	-45.9765153153666\\
-54.932	-53.0506425379402\\
-41.504	-40.0825360062381\\
-30.518	-29.4727937991127\\
-23.193	-22.3986665765391\\
-28.076	-27.114429474536\\
-42.725	-41.2617181685265\\
-51.27	-49.5140618022318\\
-56.152	-54.2288589490719\\
-40.283	-38.9033538439497\\
-39.063	-37.725137432818\\
-42.725	-41.2617181685265\\
-57.373	-55.4080411113603\\
-76.904	-74.2701269521909\\
-89.111	-86.0590513216047\\
-69.58	-67.196965480774\\
-46.387	-44.7982989042349\\
-48.828	-47.155697477655\\
-46.387	-44.7982989042349\\
-35.4	-34.1875909459529\\
-39.063	-37.725137432818\\
-43.945	-42.4399345796582\\
-41.504	-40.0825360062381\\
-29.297	-28.2936116368243\\
-20.752	-20.041268003119\\
-29.297	-28.2936116368243\\
-40.283	-38.9033538439497\\
-57.373	-55.4080411113603\\
-61.035	-58.9446218470687\\
-74.463	-71.9127283787709\\
-65.918	-63.6603847450656\\
-73.242	-70.7335462164825\\
-96.436	-93.1331785441783\\
-126.953	-122.605006592134\\
-106.201	-102.563738589015\\
-73.242	-70.7335462164825\\
-80.566	-77.8067076878994\\
-91.553	-88.4174156461815\\
-122.07	-117.889243694137\\
-80.566	-77.8067076878994\\
-43.945	-42.4399345796582\\
-30.518	-29.4727937991127\\
-31.738	-30.6510102102444\\
-25.635	-24.7570309011159\\
-18.311	-17.683869429699\\
-21.973	-21.2204501654074\\
-29.297	-28.2936116368243\\
-37.842	-36.5459552705296\\
-30.518	-29.4727937991127\\
-29.297	-28.2936116368243\\
-36.621	-35.3667731082412\\
-43.945	-42.4399345796582\\
-45.166	-43.6191167419465\\
-42.725	-41.2617181685265\\
-32.959	-31.8301923725328\\
-24.414	-23.5778487388275\\
-23.193	-22.3986665765391\\
-35.4	-34.1875909459529\\
-39.063	-37.725137432818\\
-30.518	-29.4727937991127\\
-23.193	-22.3986665765391\\
-35.4	-34.1875909459529\\
-28.076	-27.114429474536\\
-30.518	-29.4727937991127\\
-37.842	-36.5459552705296\\
-50.049	-48.3348796399434\\
-54.932	-53.0506425379402\\
-40.283	-38.9033538439497\\
-56.152	-54.2288589490719\\
-54.932	-53.0506425379402\\
-46.387	-44.7982989042349\\
-37.842	-36.5459552705296\\
-58.594	-56.5872232736487\\
-73.242	-70.7335462164825\\
-79.346	-76.6284912767677\\
-62.256	-60.1238040093571\\
-57.373	-55.4080411113603\\
-54.932	-53.0506425379402\\
-72.021	-69.5543640541941\\
-59.814	-57.7654396847804\\
-78.125	-75.4493091144793\\
-70.801	-68.3761476430624\\
-74.463	-71.9127283787709\\
-124.512	-120.247608018714\\
-98.877	-95.4905771175984\\
-53.711	-51.8714603756518\\
-37.842	-36.5459552705296\\
-41.504	-40.0825360062381\\
-51.27	-49.5140618022318\\
-63.477	-61.3029861716455\\
-70.801	-68.3761476430624\\
-78.125	-75.4493091144793\\
-80.566	-77.8067076878994\\
-54.932	-53.0506425379402\\
-47.607	-45.9765153153666\\
-56.152	-54.2288589490719\\
-36.621	-35.3667731082412\\
-28.076	-27.114429474536\\
-45.166	-43.6191167419465\\
-46.387	-44.7982989042349\\
-59.814	-57.7654396847804\\
-73.242	-70.7335462164825\\
-64.697	-62.4812025827772\\
-47.607	-45.9765153153666\\
-40.283	-38.9033538439497\\
-57.373	-55.4080411113603\\
-70.801	-68.3761476430624\\
-91.553	-88.4174156461815\\
-100.098	-96.6697592798867\\
-113.525	-109.636900060432\\
-92.773	-89.5956320573132\\
-85.449	-82.5224705858962\\
-52.49	-50.6922782133635\\
-40.283	-38.9033538439497\\
-65.918	-63.6603847450656\\
-85.449	-82.5224705858962\\
-52.49	-50.6922782133635\\
-74.463	-71.9127283787709\\
-104.98	-101.384556426727\\
-119.629	-115.531845120717\\
-79.346	-76.6284912767677\\
-46.387	-44.7982989042349\\
-30.518	-29.4727937991127\\
-40.283	-38.9033538439497\\
-37.842	-36.5459552705296\\
-41.504	-40.0825360062381\\
-29.297	-28.2936116368243\\
-35.4	-34.1875909459529\\
-30.518	-29.4727937991127\\
-35.4	-34.1875909459529\\
-31.738	-30.6510102102444\\
-34.18	-33.0093745348212\\
-53.711	-51.8714603756518\\
-50.049	-48.3348796399434\\
-67.139	-64.839566907354\\
-80.566	-77.8067076878994\\
-79.346	-76.6284912767677\\
-46.387	-44.7982989042349\\
-47.607	-45.9765153153666\\
-57.373	-55.4080411113603\\
-40.283	-38.9033538439497\\
-35.4	-34.1875909459529\\
-28.076	-27.114429474536\\
-41.504	-40.0825360062381\\
-35.4	-34.1875909459529\\
-23.193	-22.3986665765391\\
-14.648	-14.1463229428338\\
-17.09	-16.5046872674106\\
-26.855	-25.9352473122476\\
-37.842	-36.5459552705296\\
-40.283	-38.9033538439497\\
-28.076	-27.114429474536\\
-31.738	-30.6510102102444\\
-46.387	-44.7982989042349\\
-57.373	-55.4080411113603\\
-41.504	-40.0825360062381\\
-48.828	-47.155697477655\\
-54.932	-53.0506425379402\\
-50.049	-48.3348796399434\\
-57.373	-55.4080411113603\\
-53.711	-51.8714603756518\\
-40.283	-38.9033538439497\\
-31.738	-30.6510102102444\\
-42.725	-41.2617181685265\\
-36.621	-35.3667731082412\\
-42.725	-41.2617181685265\\
-57.373	-55.4080411113603\\
-83.008	-80.1650720124762\\
-63.477	-61.3029861716455\\
-57.373	-55.4080411113603\\
-85.449	-82.5224705858962\\
-68.359	-66.0177833184857\\
-41.504	-40.0825360062381\\
-34.18	-33.0093745348212\\
-28.076	-27.114429474536\\
-26.855	-25.9352473122476\\
-25.635	-24.7570309011159\\
-45.166	-43.6191167419465\\
-28.076	-27.114429474536\\
-23.193	-22.3986665765391\\
-32.959	-31.8301923725328\\
-41.504	-40.0825360062381\\
-32.959	-31.8301923725328\\
-25.635	-24.7570309011159\\
-17.09	-16.5046872674106\\
-25.635	-24.7570309011159\\
-46.387	-44.7982989042349\\
-59.814	-57.7654396847804\\
-50.049	-48.3348796399434\\
-40.283	-38.9033538439497\\
-45.166	-43.6191167419465\\
-46.387	-44.7982989042349\\
-50.049	-48.3348796399434\\
-61.035	-58.9446218470687\\
-70.801	-68.3761476430624\\
-68.359	-66.0177833184857\\
-58.594	-56.5872232736487\\
-41.504	-40.0825360062381\\
-34.18	-33.0093745348212\\
-36.621	-35.3667731082412\\
-48.828	-47.155697477655\\
-35.4	-34.1875909459529\\
-28.076	-27.114429474536\\
-50.049	-48.3348796399434\\
-61.035	-58.9446218470687\\
-57.373	-55.4080411113603\\
-65.918	-63.6603847450656\\
-83.008	-80.1650720124762\\
-48.828	-47.155697477655\\
-81.787	-78.9858898501878\\
-76.904	-74.2701269521909\\
-73.242	-70.7335462164825\\
-76.904	-74.2701269521909\\
-122.07	-117.889243694137\\
-119.629	-115.531845120717\\
-79.346	-76.6284912767677\\
-93.994	-90.7748142196015\\
-115.967	-111.995264385009\\
-112.305	-108.4586836493\\
-124.512	-120.247608018714\\
-113.525	-109.636900060432\\
-142.822	-137.930511697257\\
-115.967	-111.995264385009\\
-79.346	-76.6284912767677\\
-53.711	-51.8714603756518\\
-52.49	-50.6922782133635\\
-45.166	-43.6191167419465\\
-37.842	-36.5459552705296\\
-52.49	-50.6922782133635\\
-73.242	-70.7335462164825\\
-45.166	-43.6191167419465\\
-41.504	-40.0825360062381\\
-42.725	-41.2617181685265\\
-31.738	-30.6510102102444\\
-34.18	-33.0093745348212\\
-25.635	-24.7570309011159\\
-24.414	-23.5778487388275\\
-23.193	-22.3986665765391\\
-29.297	-28.2936116368243\\
-24.414	-23.5778487388275\\
-20.752	-20.041268003119\\
-17.09	-16.5046872674106\\
-13.428	-12.9681065317021\\
-24.414	-23.5778487388275\\
-20.752	-20.041268003119\\
-18.311	-17.683869429699\\
-37.842	-36.5459552705296\\
-51.27	-49.5140618022318\\
-50.049	-48.3348796399434\\
-40.283	-38.9033538439497\\
-48.828	-47.155697477655\\
-67.139	-64.839566907354\\
-34.18	-33.0093745348212\\
-24.414	-23.5778487388275\\
-19.531	-18.8620858408307\\
-14.648	-14.1463229428338\\
-20.752	-20.041268003119\\
-36.621	-35.3667731082412\\
-50.049	-48.3348796399434\\
-36.621	-35.3667731082412\\
-47.607	-45.9765153153666\\
-74.463	-71.9127283787709\\
-73.242	-70.7335462164825\\
-53.711	-51.8714603756518\\
-37.842	-36.5459552705296\\
-43.945	-42.4399345796582\\
-59.814	-57.7654396847804\\
-73.242	-70.7335462164825\\
-46.387	-44.7982989042349\\
-25.635	-24.7570309011159\\
-36.621	-35.3667731082412\\
-34.18	-33.0093745348212\\
-17.09	-16.5046872674106\\
-12.207	-11.7889243694137\\
-26.855	-25.9352473122476\\
-53.711	-51.8714603756518\\
-43.945	-42.4399345796582\\
-29.297	-28.2936116368243\\
-28.076	-27.114429474536\\
-17.09	-16.5046872674106\\
-18.311	-17.683869429699\\
-21.973	-21.2204501654074\\
-26.855	-25.9352473122476\\
-36.621	-35.3667731082412\\
-54.932	-53.0506425379402\\
-57.373	-55.4080411113603\\
-59.814	-57.7654396847804\\
-63.477	-61.3029861716455\\
-62.256	-60.1238040093571\\
-59.814	-57.7654396847804\\
-53.711	-51.8714603756518\\
-64.697	-62.4812025827772\\
-93.994	-90.7748142196015\\
-86.67	-83.7016527481846\\
-50.049	-48.3348796399434\\
-29.297	-28.2936116368243\\
-52.49	-50.6922782133635\\
-67.139	-64.839566907354\\
-59.814	-57.7654396847804\\
-37.842	-36.5459552705296\\
-59.814	-57.7654396847804\\
-89.111	-86.0590513216047\\
-98.877	-95.4905771175984\\
-97.656	-94.31139495531\\
-73.242	-70.7335462164825\\
-76.904	-74.2701269521909\\
-43.945	-42.4399345796582\\
-36.621	-35.3667731082412\\
-45.166	-43.6191167419465\\
-54.932	-53.0506425379402\\
-57.373	-55.4080411113603\\
-61.035	-58.9446218470687\\
-42.725	-41.2617181685265\\
-51.27	-49.5140618022318\\
-48.828	-47.155697477655\\
-29.297	-28.2936116368243\\
-21.973	-21.2204501654074\\
-17.09	-16.5046872674106\\
-25.635	-24.7570309011159\\
-23.193	-22.3986665765391\\
-13.428	-12.9681065317021\\
-25.635	-24.7570309011159\\
-46.387	-44.7982989042349\\
-35.4	-34.1875909459529\\
-30.518	-29.4727937991127\\
-40.283	-38.9033538439497\\
-61.035	-58.9446218470687\\
-48.828	-47.155697477655\\
-24.414	-23.5778487388275\\
-18.311	-17.683869429699\\
-32.959	-31.8301923725328\\
-68.359	-66.0177833184857\\
-81.787	-78.9858898501878\\
-109.863	-106.100319324724\\
-130.615	-126.141587327843\\
-126.953	-122.605006592134\\
-87.891	-84.880834910473\\
-109.863	-106.100319324724\\
-95.215	-91.9539963818899\\
-89.111	-86.0590513216047\\
-98.877	-95.4905771175984\\
-107.422	-103.742920751304\\
-108.643	-104.922102913592\\
-145.264	-140.288876021833\\
-102.539	-99.0271578533068\\
-57.373	-55.4080411113603\\
-34.18	-33.0093745348212\\
-29.297	-28.2936116368243\\
-32.959	-31.8301923725328\\
-31.738	-30.6510102102444\\
-53.711	-51.8714603756518\\
-45.166	-43.6191167419465\\
-59.814	-57.7654396847804\\
-34.18	-33.0093745348212\\
-37.842	-36.5459552705296\\
-51.27	-49.5140618022318\\
-57.373	-55.4080411113603\\
-35.4	-34.1875909459529\\
-24.414	-23.5778487388275\\
-31.738	-30.6510102102444\\
-58.594	-56.5872232736487\\
-91.553	-88.4174156461815\\
-86.67	-83.7016527481846\\
-96.436	-93.1331785441783\\
-111.084	-107.279501487012\\
-140.381	-135.573113123836\\
-120.85	-116.711027283006\\
-80.566	-77.8067076878994\\
-54.932	-53.0506425379402\\
-34.18	-33.0093745348212\\
-39.063	-37.725137432818\\
-47.607	-45.9765153153666\\
-36.621	-35.3667731082412\\
-32.959	-31.8301923725328\\
-36.621	-35.3667731082412\\
-43.945	-42.4399345796582\\
-47.607	-45.9765153153666\\
-58.594	-56.5872232736487\\
-45.166	-43.6191167419465\\
-23.193	-22.3986665765391\\
-18.311	-17.683869429699\\
-17.09	-16.5046872674106\\
-21.973	-21.2204501654074\\
-32.959	-31.8301923725328\\
-35.4	-34.1875909459529\\
-42.725	-41.2617181685265\\
-57.373	-55.4080411113603\\
-45.166	-43.6191167419465\\
-63.477	-61.3029861716455\\
-97.656	-94.31139495531\\
-78.125	-75.4493091144793\\
-68.359	-66.0177833184857\\
-84.229	-81.3442541747645\\
-96.436	-93.1331785441783\\
-62.256	-60.1238040093571\\
-58.594	-56.5872232736487\\
-40.283	-38.9033538439497\\
-41.504	-40.0825360062381\\
-76.904	-74.2701269521909\\
-124.512	-120.247608018714\\
-147.705	-142.646274595253\\
-115.967	-111.995264385009\\
-98.877	-95.4905771175984\\
-75.684	-73.0919105410592\\
-80.566	-77.8067076878994\\
-103.76	-100.206340015595\\
-63.477	-61.3029861716455\\
-51.27	-49.5140618022318\\
-41.504	-40.0825360062381\\
-72.021	-69.5543640541941\\
-40.283	-38.9033538439497\\
-41.504	-40.0825360062381\\
-34.18	-33.0093745348212\\
-50.049	-48.3348796399434\\
-76.904	-74.2701269521909\\
-75.684	-73.0919105410592\\
-58.594	-56.5872232736487\\
-46.387	-44.7982989042349\\
-52.49	-50.6922782133635\\
-41.504	-40.0825360062381\\
-30.518	-29.4727937991127\\
-26.855	-25.9352473122476\\
-24.414	-23.5778487388275\\
-20.752	-20.041268003119\\
-23.193	-22.3986665765391\\
-37.842	-36.5459552705296\\
-51.27	-49.5140618022318\\
-31.738	-30.6510102102444\\
-29.297	-28.2936116368243\\
-26.855	-25.9352473122476\\
-32.959	-31.8301923725328\\
-45.166	-43.6191167419465\\
-47.607	-45.9765153153666\\
-45.166	-43.6191167419465\\
-28.076	-27.114429474536\\
-47.607	-45.9765153153666\\
-69.58	-67.196965480774\\
-53.711	-51.8714603756518\\
-26.855	-25.9352473122476\\
-41.504	-40.0825360062381\\
-42.725	-41.2617181685265\\
-37.842	-36.5459552705296\\
-26.855	-25.9352473122476\\
-41.504	-40.0825360062381\\
-62.256	-60.1238040093571\\
-41.504	-40.0825360062381\\
-17.09	-16.5046872674106\\
-25.635	-24.7570309011159\\
-51.27	-49.5140618022318\\
-54.932	-53.0506425379402\\
-39.063	-37.725137432818\\
-32.959	-31.8301923725328\\
-58.594	-56.5872232736487\\
-76.904	-74.2701269521909\\
-64.697	-62.4812025827772\\
-87.891	-84.880834910473\\
-108.643	-104.922102913592\\
-72.021	-69.5543640541941\\
-59.814	-57.7654396847804\\
-80.566	-77.8067076878994\\
-54.932	-53.0506425379402\\
-73.242	-70.7335462164825\\
-87.891	-84.880834910473\\
-69.58	-67.196965480774\\
-36.621	-35.3667731082412\\
-58.594	-56.5872232736487\\
-100.098	-96.6697592798867\\
-114.746	-110.816082222721\\
-83.008	-80.1650720124762\\
-72.021	-69.5543640541941\\
-92.773	-89.5956320573132\\
-74.463	-71.9127283787709\\
-48.828	-47.155697477655\\
-45.166	-43.6191167419465\\
-57.373	-55.4080411113603\\
-58.594	-56.5872232736487\\
-57.373	-55.4080411113603\\
-64.697	-62.4812025827772\\
-48.828	-47.155697477655\\
-50.049	-48.3348796399434\\
-68.359	-66.0177833184857\\
-43.945	-42.4399345796582\\
-35.4	-34.1875909459529\\
-30.518	-29.4727937991127\\
-23.193	-22.3986665765391\\
-26.855	-25.9352473122476\\
-46.387	-44.7982989042349\\
-41.504	-40.0825360062381\\
-57.373	-55.4080411113603\\
-69.58	-67.196965480774\\
-83.008	-80.1650720124762\\
-63.477	-61.3029861716455\\
-57.373	-55.4080411113603\\
-25.635	-24.7570309011159\\
-47.607	-45.9765153153666\\
-72.021	-69.5543640541941\\
-51.27	-49.5140618022318\\
-26.855	-25.9352473122476\\
-51.27	-49.5140618022318\\
-74.463	-71.9127283787709\\
-92.773	-89.5956320573132\\
-120.85	-116.711027283006\\
-87.891	-84.880834910473\\
-74.463	-71.9127283787709\\
-86.67	-83.7016527481846\\
-61.035	-58.9446218470687\\
-47.607	-45.9765153153666\\
-53.711	-51.8714603756518\\
-68.359	-66.0177833184857\\
-72.021	-69.5543640541941\\
-45.166	-43.6191167419465\\
-39.063	-37.725137432818\\
-32.959	-31.8301923725328\\
-46.387	-44.7982989042349\\
-43.945	-42.4399345796582\\
-34.18	-33.0093745348212\\
-63.477	-61.3029861716455\\
-68.359	-66.0177833184857\\
-50.049	-48.3348796399434\\
-40.283	-38.9033538439497\\
-58.594	-56.5872232736487\\
-32.959	-31.8301923725328\\
-21.973	-21.2204501654074\\
-28.076	-27.114429474536\\
-35.4	-34.1875909459529\\
-34.18	-33.0093745348212\\
-25.635	-24.7570309011159\\
-18.311	-17.683869429699\\
-28.076	-27.114429474536\\
-37.842	-36.5459552705296\\
-53.711	-51.8714603756518\\
-59.814	-57.7654396847804\\
-75.684	-73.0919105410592\\
-86.67	-83.7016527481846\\
-48.828	-47.155697477655\\
-39.063	-37.725137432818\\
-76.904	-74.2701269521909\\
-109.863	-106.100319324724\\
-83.008	-80.1650720124762\\
-78.125	-75.4493091144793\\
-113.525	-109.636900060432\\
-92.773	-89.5956320573132\\
-74.463	-71.9127283787709\\
-57.373	-55.4080411113603\\
-58.594	-56.5872232736487\\
-75.684	-73.0919105410592\\
-52.49	-50.6922782133635\\
-45.166	-43.6191167419465\\
-75.684	-73.0919105410592\\
-95.215	-91.9539963818899\\
-96.436	-93.1331785441783\\
-48.828	-47.155697477655\\
-25.635	-24.7570309011159\\
-56.152	-54.2288589490719\\
-47.607	-45.9765153153666\\
-21.973	-21.2204501654074\\
-20.752	-20.041268003119\\
-31.738	-30.6510102102444\\
-41.504	-40.0825360062381\\
-67.139	-64.839566907354\\
-50.049	-48.3348796399434\\
-39.063	-37.725137432818\\
-28.076	-27.114429474536\\
-19.531	-18.8620858408307\\
-25.635	-24.7570309011159\\
-23.193	-22.3986665765391\\
-26.855	-25.9352473122476\\
-42.725	-41.2617181685265\\
-39.063	-37.725137432818\\
-24.414	-23.5778487388275\\
-20.752	-20.041268003119\\
-26.855	-25.9352473122476\\
-30.518	-29.4727937991127\\
-20.752	-20.041268003119\\
-31.738	-30.6510102102444\\
-24.414	-23.5778487388275\\
-19.531	-18.8620858408307\\
-36.621	-35.3667731082412\\
-47.607	-45.9765153153666\\
-69.58	-67.196965480774\\
-92.773	-89.5956320573132\\
-81.787	-78.9858898501878\\
-42.725	-41.2617181685265\\
-32.959	-31.8301923725328\\
-34.18	-33.0093745348212\\
-21.973	-21.2204501654074\\
-35.4	-34.1875909459529\\
-39.063	-37.725137432818\\
-37.842	-36.5459552705296\\
-42.725	-41.2617181685265\\
-48.828	-47.155697477655\\
-52.49	-50.6922782133635\\
-62.256	-60.1238040093571\\
-86.67	-83.7016527481846\\
-47.607	-45.9765153153666\\
-25.635	-24.7570309011159\\
-24.414	-23.5778487388275\\
-42.725	-41.2617181685265\\
-31.738	-30.6510102102444\\
-34.18	-33.0093745348212\\
-18.311	-17.683869429699\\
-28.076	-27.114429474536\\
-18.311	-17.683869429699\\
-13.428	-12.9681065317021\\
-18.311	-17.683869429699\\
-31.738	-30.6510102102444\\
-36.621	-35.3667731082412\\
-51.27	-49.5140618022318\\
-69.58	-67.196965480774\\
-52.49	-50.6922782133635\\
-25.635	-24.7570309011159\\
-39.063	-37.725137432818\\
-45.166	-43.6191167419465\\
-25.635	-24.7570309011159\\
-32.959	-31.8301923725328\\
-58.594	-56.5872232736487\\
-50.049	-48.3348796399434\\
-79.346	-76.6284912767677\\
-92.773	-89.5956320573132\\
-64.697	-62.4812025827772\\
-51.27	-49.5140618022318\\
};
\addlegendentry{data2}

\end{axis}
\end{tikzpicture}%
%	\caption{Scatter plots of the AR(0) term $y(t-1)$ for training datasets. Correlation coefficient with the output signal is shown for each figure.}\label{fig:regr_y}
%\end{figure}
%\begin{figure}[!t]
%	\centering
%	% This file was created by matlab2tikz.
%
\definecolor{mycolor1}{rgb}{0.00000,0.44700,0.74100}%
\definecolor{mycolor2}{rgb}{0.85000,0.32500,0.09800}%
%
\begin{tikzpicture}

\begin{axis}[%
width=4.927cm,
height=3.484cm,
at={(0cm,14.516cm)},
scale only axis,
xmin=-1000000,
xmax=1025000,
xlabel style={font=\color{white!15!black}},
xlabel={$\delta^3 u(t)$},
ymin=-8788900000,
ymax=11596600000,
ylabel style={font=\color{white!15!black}},
ylabel={y(t)},
axis background/.style={fill=white},
title={C1, R = 0.5155},
axis x line*=bottom,
axis y line*=left
]
\addplot[only marks, mark=*, mark options={}, mark size=1.5000pt, color=mycolor1, fill=mycolor1] table[row sep=crcr]{%
x	y\\
-201000	-1830800000\\
255999.999999999	4394400000\\
-125999.999999999	-4028400000\\
143000	3418100000\\
149999.999999999	-2685500000\\
-534000	-1221000000\\
239999.999999999	4761100000\\
-18999.9999999992	-4150600000\\
129000.000000001	1831200000\\
254999.999999997	1342600000\\
-348000	-4394500000\\
-145000	3540200000\\
-384999.999999998	-1465000000\\
530000.000000001	1342900000\\
311999.999999995	243899999.999999\\
-237999.999999999	-1830800000\\
109999.999999998	1586900000\\
-36999.9999999981	-1343000000\\
-623000.000000002	-1464600000\\
807000.000000005	5737200000\\
-184000.000000003	-5371100000\\
-457000	610300000.000002\\
437999.999999999	3051900000\\
2000.00000000067	-2929700000\\
-368000.000000001	-244300000\\
402999.999999999	4150500000\\
-17999.9999999989	-4150400000\\
-327999.999999998	-854399999.999999\\
253999.999999997	4760500000\\
73999.9999999989	-3051600000\\
-110000	-488200000.000003\\
20000.000000004	976500000.000001\\
-204000.000000005	-244300000\\
130000	-244000000\\
109000	1587000000\\
-73000.0000000013	-1831200000\\
-36999.9999999999	610400000\\
56000	-122000000\\
-184000.000000002	-366499999.999999\\
347999.999999999	1953600000\\
-36999.9999999999	-1465200000\\
-218999.999999998	-1709000000\\
202000.000000003	3174200000\\
-203000.000000006	-3174300000\\
-163999.999999997	3052000000\\
420999.999999998	-1098600000\\
-236999.999999997	-1220800000\\
-75000.0000000019	1098700000\\
146999.999999998	610399999.999999\\
74000.0000000034	-854800000\\
-275000	-1342300000\\
402999.999999999	4516200000\\
-37000.0000000008	-4638400000\\
-257000.000000001	366000000.000002\\
-144999.999999999	2075300000\\
236999.999999998	-976500000\\
237999.999999999	1464700000\\
-382999.999999997	-3295700000\\
-38999.9999999979	2563200000\\
294999.999999997	488400000\\
-74999.9999999966	-2197000000\\
-218000.000000003	487799999.999999\\
182000.000000001	2075600000\\
-201000.000000002	-2929900000\\
366000.000000001	3296000000\\
-201000.000000002	-3418100000\\
72000.0000000018	3051900000\\
-70999.9999999988	-3051800000\\
88999.9999999977	3173800000\\
-70999.9999999971	-2929700000\\
-20000.000000004	2075200000\\
56000.0000000018	-1220600000\\
-110000	854399999.999999\\
-110000.000000003	-1953200000\\
72000.0000000027	3296100000\\
313000	-854700000\\
-91999.9999999978	-2075100000\\
91999.9999999996	2929800000\\
-514000.000000001	-5737600000\\
238999.999999999	6592100000\\
201000.000000004	-1831200000\\
18999.9999999983	-488200000.000001\\
-313000	-3662300000\\
21000.0000000017	5859500000\\
436999.999999998	200000.000000955\\
-218999.999999998	-5493600000\\
-53999.9999999985	3540500000\\
17000.0000000003	-1099000000\\
-108000.000000002	1220900000\\
-75999.9999999996	-1464900000\\
277000.000000002	2685600000\\
-1000.00000000211	-2807700000\\
-237999.999999999	732400000.000001\\
-54000.0000000003	199999.999999534\\
52999.9999999973	365900000\\
240000.000000002	854799999.999999\\
-111000.000000002	-1587100000\\
-128000	244200000\\
165000.000000002	976600000\\
-183000.000000001	-1465100000\\
311000	2441900000\\
-219999.999999999	-3052300000\\
148000	2686000000\\
-75000.0000000028	-1953400000\\
-201000	-122000000.000001\\
329999.999999999	2685500000\\
-202000.000000003	-2929500000\\
-72999.9999999977	121900000\\
129000	2563400000\\
89999.999999999	-1098400000\\
-163000	-2685700000\\
127000	4150400000\\
-147000.000000002	-3173800000\\
2000.00000000333	1831100000\\
217999.999999996	-366300000\\
-255999.999999998	-1098500000\\
35999.9999999996	854200000\\
-70999.9999999997	488700000\\
272000	-122499999.999999\\
-89999.9999999998	-609900000.000001\\
237999.999999999	1342300000\\
-550000.000000001	-3905900000\\
36999.999999999	3662000000\\
586000.000000001	1953100000\\
-421000	-6103500000\\
53999.9999999985	4760700000\\
221000	-488200000\\
-129000.000000002	-2685500000\\
-366000	1098400000\\
274999.999999998	2075400000\\
-110999.999999998	-2929700000\\
112000.000000001	3173800000\\
180999.999999996	-1709100000\\
19000.0000000028	244300000.000002\\
-293000	-1953200000\\
129000.000000001	2807600000\\
-38000.0000000038	-1586800000\\
1000.00000000211	732300000\\
182999.999999999	244200000.000001\\
-202000.000000002	-976600000\\
-54000.0000000003	122100000.000001\\
126999.999999998	732300000\\
-126999.999999998	-488000000\\
-20000.0000000013	-488500000\\
-33999.9999999963	976599999.999999\\
454999.999999996	1586900000\\
-161999.999999996	-3051800000\\
-223000.000000003	-243900000\\
3000.000000001	2074900000\\
217999.999999998	100000.000000122\\
-109000	-1708800000\\
-221000.000000001	1098300000\\
-53000.0000000017	-732200000\\
456000.000000001	2807600000\\
-218000.000000003	-4394600000\\
-56999.9999999995	3418000000\\
111000	-1464900000\\
18999.9999999983	366400000\\
-20000.0000000005	-366400000.000001\\
-34999.9999999993	100000.000000122\\
-257000	122000000\\
-74000.0000000007	-366300000\\
386000.000000001	1587300000\\
90999.9999999984	-610900000.000001\\
145000	-365799999.999999\\
-289999.999999997	-976700000.000002\\
-40000.0000000036	366300000\\
75999.9999999996	1220400000\\
-477000	-2074700000\\
548000.000000001	2807200000\\
-199000	-1953100000\\
34999.9999999984	122499999.999999\\
-109999.999999999	487600000\\
203000.000000002	611100000.000001\\
254000	243399999.999999\\
-309000	-2318700000\\
126999.999999999	1830700000\\
-257000.000000001	-2197300000\\
57000.0000000022	3052100000\\
88999.9999999977	-1587400000\\
76000.0000000014	1587400000\\
-76000.0000000014	-3418300000\\
-108000	2807700000\\
164000.000000001	-488300000\\
-181999.999999999	-610100000.000001\\
235999.999999997	1220200000\\
-419000	-3173300000\\
346000.000000001	4516300000\\
203000.000000003	-976500000\\
-110000.000000003	-2929800000\\
-349000	732699999.999997\\
-55000.0000000024	732200000\\
294000.000000003	2197400000\\
-35999.9999999996	-2685800000\\
89000.0000000004	732700000.000003\\
-51999.9999999987	488199999.999998\\
88999.9999999986	122000000.000001\\
-145000	-2685400000\\
-90999.9999999975	2563300000\\
71999.9999999965	-122000000\\
274000.000000001	1464900000\\
-107999.999999997	-3540100000\\
-495000.000000001	-122000000.000002\\
493999.999999998	4516400000\\
36999.9999999999	-3051500000\\
-440000.000000001	-1342900000\\
274999.999999999	3296000000\\
127999.999999999	-1343000000\\
-72999.9999999977	-1220600000\\
-219000	976799999.999999\\
235999.999999997	854099999.999999\\
-162999.999999997	-1952800000\\
-55000.0000000015	1220500000\\
365000.000000002	1220800000\\
-109000.000000002	-1220700000\\
-200999.999999997	-1953200000\\
-129000.000000005	1586900000\\
110000.000000004	732600000\\
439999.999999997	2807400000\\
-166000	-6225400000\\
-181999.999999999	2929600000\\
-128000.000000001	-732599999.999999\\
126999.999999999	2319800000\\
128999.999999998	-2075700000\\
-218999.999999997	-121799999.999999\\
15999.9999999973	976600000\\
76000.0000000005	-244400000.000001\\
52999.999999999	-121699999.999999\\
146999.999999998	732099999.999999\\
-476000	-2685400000\\
312000	3295800000\\
126999.999999998	-488100000.000002\\
-54999.9999999997	-1709100000\\
-200000	-3.5527136788005e-07\\
-999.999999998557	1098500000\\
383999.999999997	1831400000\\
-53999.9999999985	-3174100000\\
-549000	-1220700000\\
236999.999999998	4638800000\\
36999.999999999	-3173900000\\
129000.000000002	1586900000\\
89999.9999999972	-732399999.999999\\
-402000.000000001	-1098600000\\
311000.000000002	1953100000\\
73999.9999999998	-122199999.999999\\
-292999.999999998	-2441100000\\
217999.999999996	2807300000\\
-70999.9999999962	-976400000\\
-93000.0000000053	-1220700000\\
348000.000000002	2807500000\\
-402000.000000003	-3417700000\\
165000.000000002	2807200000\\
-74999.9999999984	-1708600000\\
-200000.000000002	244000000.000001\\
365999.999999999	976400000.000001\\
-164999.999999999	-365900000.000001\\
35999.9999999996	-1220900000\\
331000.000000001	3906200000\\
-165999.999999999	-5126800000\\
-237000.000000002	1220700000\\
-1000.00000000122	1953000000\\
-72000.0000000009	-1342800000\\
364999.999999999	1221000000\\
-274999.999999999	-1709400000\\
-53000.0000000008	1099000000\\
-19999.9999999996	-854800000\\
310999.999999999	2563700000\\
-235999.999999995	-3906200000\\
327999.999999998	3539700000\\
-293000.000000002	-2197000000\\
219999.999999999	122099999.999999\\
-309999.999999996	610300000.000001\\
161999.999999998	-200000.00000131\\
-144000.000000002	-121799999.999999\\
36000.0000000005	-610400000\\
-147000.000000001	1098600000\\
347999.999999998	-244400000\\
-90999.9999999975	499999.999999456\\
-1000.000000003	-732700000\\
-272999.999999997	-244300000\\
436999.999999999	2930000000\\
21000.0000000008	-3174000000\\
-423000.000000003	-732300000\\
220000	3661900000\\
-90999.9999999984	-3295700000\\
238000.000000002	3295800000\\
-54999.9999999997	-2685600000\\
54999.9999999988	199999.999999534\\
-239000.000000002	610200000.000002\\
-109000.000000002	99999.9999994117\\
74000.0000000034	488000000\\
125999.999999999	122500000\\
20999.9999999999	-976900000\\
-166999.999999999	244300000\\
73999.9999999972	488200000.000001\\
183000.000000001	488399999.999999\\
-126999.999999999	-1465000000\\
-20000.0000000022	488400000\\
238999.999999999	1586800000\\
-146999.999999997	-2075000000\\
-458000	-854700000.000001\\
276000	2197400000\\
327999.999999997	1586900000\\
-236999.999999996	-4150600000\\
72999.9999999986	3052100000\\
73000.0000000004	-1709100000\\
-329000.000000001	-244400000.000001\\
91000.0000000011	1587200000\\
219999.999999999	7.105427357601e-07\\
90999.9999999975	-732499999.999999\\
-530000.000000001	-1953200000\\
548000.000000001	4028400000\\
-163999.999999999	-2441400000\\
-201000.000000001	-854499999.999999\\
219000.000000001	2685600000\\
0	-1587100000\\
-54000.000000002	-854199999.999999\\
-257999.999999999	365900000\\
93000	1587200000\\
476000.000000002	1220400000\\
-92000.0000000023	-4028000000\\
-422000	488100000\\
111000.000000001	2197200000\\
-35999.9999999987	-854299999.999999\\
107999.999999997	121999999.999999\\
203000.000000004	1220500000\\
-312000.000000002	-3295600000\\
8.88178419700125e-10	1953000000\\
147000.000000002	1342700000\\
182999.999999996	-1098500000\\
-239000	-1220700000\\
-164000.000000001	-122300000\\
-8.88178419700125e-10	1587200000\\
347999.999999998	1953000000\\
-55999.9999999983	-4638600000\\
38000.0000000029	3783900000\\
-202000.000000003	-3173400000\\
-53999.9999999976	1586600000\\
254999.999999997	1098800000\\
-55000.0000000006	-1831300000\\
-146000.000000002	366600000\\
1000.000000003	976300000\\
-330999.999999998	-1464800000\\
365999.999999999	1342700000\\
148000.000000001	732500000\\
53999.9999999994	-2075000000\\
-220000.000000002	610000000\\
-53999.9999999985	-366099999.999999\\
329000	2563700000\\
-366000.000000001	-3296300000\\
90999.9999999993	610799999.999999\\
127999.999999999	2441100000\\
-199999.999999999	-3295800000\\
-93000.0000000017	1830900000\\
239000	488700000\\
292999.999999999	121500000.000001\\
-331000	-1830600000\\
-34999.9999999993	-99999.9999983459\\
-111000.000000002	1830800000\\
294000.000000001	-487999999.999999\\
-495000.000000003	-1342800000\\
202000.000000001	122000000\\
566000.000000001	4760600000\\
-566000	-7568000000\\
92000.0000000014	4882400000\\
180999.999999997	-487900000\\
-16999.9999999986	-610599999.999999\\
2.66453525910038e-09	-1342800000\\
-256000.000000004	1220900000\\
108999.999999999	-244200000\\
-18000.0000000007	1098600000\\
37000.0000000017	-1587100000\\
71999.9999999983	1831400000\\
-180999.999999997	-2441600000\\
125999.999999999	2197200000\\
20000.0000000022	-732199999.999999\\
219000.000000001	854300000\\
-311000.000000002	-2807500000\\
-55999.9999999991	1708800000\\
331000	2197500000\\
-348999.999999998	-4516800000\\
221000.000000001	4028600000\\
145000	-977000000\\
-182000	-2440900000\\
-238000.000000002	1830600000\\
-19000.000000001	366500000\\
403000	610299999.999999\\
-74000.0000000025	-854600000\\
-308999.999999997	-2319200000\\
290999.999999997	4394300000\\
19000.0000000019	-2319000000\\
-293999.999999999	-1709100000\\
166999.999999998	3417700000\\
126000	-1342400000\\
-310000.000000001	-2441700000\\
146000.000000001	4639000000\\
273999.999999999	-2441700000\\
-437999.999999997	-2319300000\\
90999.9999999993	3173900000\\
384000	1709100000\\
-329000.000000001	-5493400000\\
311000	4883000000\\
-330000.000000002	-3051900000\\
239000.000000001	1709100000\\
-74000.0000000025	-1342800000\\
-292999.999999999	121899999.999999\\
293999.999999998	2075500000\\
-20000.0000000005	-2197400000\\
-272999.999999999	243999999.999999\\
474999.999999998	1587000000\\
-201000	-732200000\\
-255999.999999998	-2563800000\\
-110999.999999997	3051900000\\
348999.999999995	0\\
73000.0000000013	-854500000\\
-37000.0000000008	-366100000\\
-8.88178419700125e-10	610000000.000001\\
-145999.999999998	-1098100000\\
146999.999999999	2440800000\\
15999.9999999982	-2562900000\\
-198999.999999998	487900000\\
183000	1220900000\\
-277000.000000001	-1831300000\\
424000.000000003	2807900000\\
-185000.000000004	-2319400000\\
-292000.000000001	-610500000.000001\\
201000	1831300000\\
329000	1220300000\\
-254999.999999996	-3417500000\\
-55000.0000000024	1464500000\\
-57000.0000000013	300000.000000367\\
222000.000000001	854099999.999999\\
-129000.000000001	-976200000\\
-220000.000000002	-854600000.000001\\
239000.000000002	2197100000\\
-2000.00000000067	-610200000.000001\\
94000.0000000003	-976399999.999999\\
-185000	-244500000\\
75000.000000001	1831200000\\
235999.999999996	-976300000\\
-382999.999999999	-1465300000\\
-147000	976800000.000001\\
274999.999999998	1709100000\\
237000.000000001	-610499999.999999\\
-145000	-1220700000\\
-56000.0000000018	-488400000\\
0	1587300000\\
-18000.0000000016	-855000000\\
-18000.0000000016	488800000\\
-110000	-977000000\\
237000.000000001	1953300000\\
19999.9999999987	-1708900000\\
-367000.000000002	-366300000\\
219000.000000001	1709000000\\
37999.9999999994	-976699999.999999\\
-18999.9999999992	299999.999999656\\
72999.9999999968	610100000\\
-238000	-1586800000\\
147000.000000001	1708900000\\
36000.0000000005	99999.9999979906\\
312000.000000001	610300000\\
-202000.000000002	-2807700000\\
-164999.999999998	366400000.000001\\
37999.9999999985	2441300000\\
-277000.000000001	-2075200000\\
294999.999999998	1586800000\\
-220000.000000001	-854200000\\
365000.000000001	487999999.999999\\
-72000.0000000001	200000.00000131\\
-220000	-2075300000\\
108999.999999998	2929600000\\
20000.0000000031	-976399999.999999\\
16999.999999995	-610300000.000001\\
-164999.999999998	-732800000\\
294000	3784800000\\
54000.0000000011	-3662800000\\
-128000	-121499999.999996\\
-219000.000000001	1098300000\\
91000.0000000011	610500000\\
-36000.0000000014	-1220800000\\
292000	1831100000\\
-275000.000000002	-1953100000\\
56000.0000000036	244100000.000001\\
220000.000000001	1709100000\\
-495000.000000004	-2929900000\\
165000.000000002	1709100000\\
401999.999999997	3173900000\\
-328999.999999998	-5859500000\\
-256000.000000002	976600000.000002\\
622000.000000002	6347800000\\
-273999.999999998	-7080400000\\
-274999.999999999	610699999.999997\\
403000	5126800000\\
53999.9999999994	-3906400000\\
-144999.999999999	-854200000\\
-312000.000000003	1830800000\\
35999.9999999987	-1342500000\\
56000	1952800000\\
346999.999999999	244299999.999998\\
-127999.999999997	-1830900000\\
56000	854200000.000001\\
-2000.00000000244	-732200000\\
-216999.999999996	-100000.000000477\\
124999.999999996	488300000\\
-88999.9999999977	610299999.999999\\
-19000.0000000019	-1220600000\\
-19000.0000000001	732300000.000001\\
73999.9999999989	122300000.000001\\
255000.000000001	487999999.999998\\
-436999.999999998	-2197200000\\
52999.9999999973	1587100000\\
238000	1342600000\\
-126999.999999997	-2807500000\\
-19000.0000000001	1464600000\\
35999.9999999996	244400000\\
276000.000000002	2075100000\\
-201999.999999998	-6469700000\\
-165000.000000001	5615200000\\
183000.000000002	-1220700000\\
18999.9999999992	-488300000.000002\\
-238000.000000002	-976400000\\
-92999.9999999991	854200000\\
311999.999999999	2319600000\\
276000.000000002	-1342900000\\
-240000.000000006	-2563500000\\
-164000	854599999.999998\\
92000.0000000014	2685400000\\
-183000.000000001	-3539900000\\
-165999.999999999	3417900000\\
366999.999999996	-2075100000\\
-90999.9999999984	732200000\\
181999.999999999	-243900000\\
-8.88178419700125e-10	488200000\\
-292000	-1831200000\\
201000	1953400000\\
-92000.0000000032	-299999.999999656\\
-16999.9999999986	-1708700000\\
-57000.0000000004	1830900000\\
-16000.0000000009	-1587000000\\
273000	2197500000\\
-17999.9999999998	-1220900000\\
-220000.000000001	-1586800000\\
37000.0000000044	1831000000\\
110999.999999999	732399999.999998\\
-57000.0000000004	-2319300000\\
57000.000000003	2197300000\\
125999.999999998	-366300000\\
-144999.999999998	-1830900000\\
-128000	487999999.999998\\
-184000.000000004	244500000.000001\\
329999.999999999	2319100000\\
182000.000000001	-488200000.000001\\
-143999.999999999	-3174000000\\
70999.9999999971	1343000000\\
-274000	854500000\\
-91000.0000000019	-854700000.000001\\
348000.000000002	1220900000\\
-495999.999999999	-854600000\\
294999.999999999	-488200000\\
236999.999999999	1586800000\\
-128999.999999999	-854500000\\
74999.9999999993	-610200000\\
-166000.000000001	732399999.999999\\
-145999.999999998	-1342900000\\
201999.999999998	2319400000\\
-37999.9999999967	-1586900000\\
164999.999999999	488300000.000001\\
-164000.000000001	-122200000\\
-218999.999999998	-1098400000\\
-74999.9999999993	1464600000\\
512999.999999997	1709200000\\
-35999.9999999969	-4394700000\\
36999.9999999972	4516600000\\
-385999.999999999	-4882600000\\
-16999.9999999995	3417700000\\
458000.000000001	1587100000\\
-239999.999999998	-4028400000\\
129999.999999998	1831100000\\
-237999.999999996	-488300000.000001\\
16999.999999995	244000000.000002\\
258000.000000004	1831300000\\
-221000.000000004	-4272600000\\
182999.999999998	4516600000\\
-237000	-3539900000\\
-312000.000000002	1708700000\\
714000.000000001	1709200000\\
-347000.000000001	-2685500000\\
-75000.0000000002	-244300000\\
222000.000000004	2563500000\\
-148000.000000005	-2319200000\\
8.88178419700125e-10	732300000.000001\\
38000.0000000029	976600000\\
-2000.00000000333	-1953200000\\
57000.0000000013	2563600000\\
52999.9999999982	-2197300000\\
-292000.000000002	-122200000\\
476000.000000003	2197600000\\
-128000.000000002	-1099100000\\
-512999.999999995	-3051500000\\
108999.999999996	5127100000\\
570000.000000001	-1831300000\\
-129999.999999999	-1464800000\\
-567000.000000001	366300000.000001\\
109000.000000003	1342700000\\
349000	-366199999.999999\\
-36999.999999999	-976500000\\
73000.0000000013	1220500000\\
-348000	-1708600000\\
384999.999999995	2441000000\\
36000.0000000022	-1220500000\\
-437999.999999999	-2319300000\\
492999.999999996	4638500000\\
-237999.999999996	-2685300000\\
-54999.9999999997	-1465100000\\
110999.999999999	3051900000\\
-366999.999999998	-2807600000\\
365999.999999998	3539900000\\
-201000	-3173600000\\
-36999.999999999	-122299999.999999\\
384000	3662200000\\
-53000.0000000026	-2441300000\\
-258000	-2441600000\\
146999.999999999	4394600000\\
-182999.999999998	-2929600000\\
274000.000000001	1342800000\\
-146000.000000001	365899999.999999\\
-36000.0000000005	-2318900000\\
109000.000000004	3051400000\\
-238000.000000001	-2441200000\\
238000.000000004	1098500000\\
-238000.000000003	-121800000\\
258000	732000000\\
34000.0000000034	-732200000\\
-182000.000000001	-976400000\\
128999.999999999	1464600000\\
-165999.999999996	-488200000\\
-220000.000000001	-854500000\\
330999.999999997	1709100000\\
348000.000000002	854100000.000001\\
-386000.000000002	-4149800000\\
109999.999999999	3173400000\\
-382999.999999999	-1464800000\\
363999.999999999	2075400000\\
185000.000000001	-1098800000\\
-475999.999999999	-2075100000\\
364999.999999997	3051500000\\
-310999.999999998	-1952700000\\
347999.999999999	2074900000\\
-238000	-2319300000\\
-109000	488400000\\
163000.000000002	854300000.000001\\
165999.999999999	732600000\\
-219000	-2563500000\\
-1999.999999998	2319200000\\
-90000.0000000025	-1831000000\\
310999.999999999	2807800000\\
-330999.999999998	-4028500000\\
112000.000000004	3051800000\\
310999.999999998	1098500000\\
-111000.000000002	-3417800000\\
-128000.000000001	488399999.999998\\
-293000	732100000\\
202000	1465000000\\
312000.000000001	-244100000.000001\\
-332000.000000002	-3173800000\\
259000.000000002	3906000000\\
15999.9999999965	-1830600000\\
-400999.999999998	-1709500000\\
-166000.000000002	1953500000\\
569000.000000002	2441200000\\
-20000.0000000013	-4272300000\\
-402000.000000001	1464800000\\
367000	1220500000\\
-56999.9999999995	-1098400000\\
20999.999999999	-244200000.000001\\
-241000	-244199999.999998\\
21000.0000000008	976800000\\
308999.999999999	732000000.000001\\
-199000	-2197000000\\
33999.999999998	1098800000\\
57000.0000000022	976099999.999999\\
-184000.000000003	-2440900000\\
202000	2440900000\\
-166000.000000001	-1342300000\\
-144999.999999999	-244400000.000001\\
290999.999999999	1586900000\\
21000.0000000035	-488099999.999999\\
-39000.0000000032	-1465100000\\
56000.0000000018	1831200000\\
-55000.0000000006	-1586800000\\
-218999.999999999	-199999.999998468\\
52999.9999999973	1953200000\\
93000	-2197300000\\
18000.0000000025	1953300000\\
-17999.999999998	-1465100000\\
182999.999999996	1465000000\\
35999.9999999987	-1342900000\\
-585999.999999998	-2197000000\\
294000.000000003	4516300000\\
402999.999999998	610500000\\
-331000.000000002	-6103500000\\
-109999.999999999	3906100000\\
277000	1709200000\\
-94000.0000000021	-4028400000\\
-219000	2441300000\\
220000.000000001	-122000000\\
-72999.9999999977	-1098500000\\
-91000.0000000011	976299999.999999\\
290999.999999998	1098900000\\
-273000.000000001	-3296200000\\
108999.999999999	3052100000\\
-17999.9999999998	-1220900000\\
90999.9999999993	7.105427357601e-07\\
-218999.999999999	-244200000.000001\\
257000.000000003	1465100000\\
-184000.000000004	-2441700000\\
-19000.000000001	2197500000\\
93000.0000000008	-1220900000\\
236999.999999998	1342900000\\
-182999.999999999	-2075200000\\
-201000	732300000\\
-91999.999999997	366300000\\
-164000.000000001	122200000\\
493999.999999998	854199999.999999\\
164000.000000001	-609999999.999998\\
-107999.999999996	-854900000\\
-295000.000000004	-854100000\\
129000	2807400000\\
73000.0000000004	-1709000000\\
36999.999999999	99999.9999994117\\
-330000	-366200000\\
238999.999999999	1464700000\\
-999.999999999446	-854400000\\
-73000.0000000013	-854400000\\
146999.999999999	2075000000\\
-93000.0000000008	-1830800000\\
130000	609999999.999999\\
-312000.000000001	-854100000\\
-91999.9999999978	1586500000\\
603999.999999997	610799999.999999\\
-400999.999999996	-3052100000\\
107999.999999997	2197400000\\
-128000	-1098600000\\
129000.000000003	2075000000\\
-110000.000000001	-3295600000\\
256000	4150100000\\
-219999.999999999	-4394200000\\
-34999.9999999984	2929300000\\
363999.999999995	-243900000.000001\\
-639999.999999998	-2563500000\\
530999.999999997	3784200000\\
-273999.999999997	-2685700000\\
53999.9999999985	1098800000\\
-8.88178419700125e-10	-488300000.000001\\
166000.000000002	1464800000\\
-221000.000000002	-3296000000\\
0	3051900000\\
294000.000000003	244199999.999998\\
-202000.000000003	-2563600000\\
36999.9999999999	1830900000\\
-129000.000000001	-1586500000\\
128000	2563200000\\
-17000.0000000003	-1831100000\\
8.88178419700125e-10	-244100000\\
-93000.0000000017	610599999.999999\\
92000.0000000005	854099999.999999\\
-274000.000000003	-2929300000\\
274000.000000003	4150000000\\
202000	-1098400000\\
-201999.999999999	-3051700000\\
-16999.9999999995	2685400000\\
198999.999999998	488300000\\
-289999.999999996	-2929500000\\
51999.9999999952	2441000000\\
56999.9999999995	-609900000.000001\\
-129000	-300000.000000722\\
-55000.0000000006	-610299999.999999\\
184000	1587200000\\
34999.9999999993	-977100000\\
74999.9999999993	366699999.999999\\
-1000.00000000033	-732600000\\
-458000.000000002	-1465000000\\
330000.000000001	3906500000\\
73000.0000000022	-2197300000\\
74000.0000000007	488000000\\
-420999.999999999	-3417600000\\
511000	8056500000\\
37999.9999999976	-7202300000\\
-347999.999999997	732699999.999999\\
-1000.00000000477	1952900000\\
-199999.999999999	-610200000.000001\\
237000	610199999.999998\\
19000.000000001	-121999999.999999\\
237999.999999999	366299999.999999\\
-129000.000000001	-1465000000\\
18999.9999999983	366300000\\
-201999.999999999	-121999999.999998\\
-89999.999999999	1098500000\\
198999.999999997	-854400000\\
149000.000000003	1342600000\\
-295000.000000003	-2563200000\\
130000.000000003	2197100000\\
199000	-610400000\\
-582999.999999999	-1342600000\\
418000	2441200000\\
-34000.0000000034	-1220600000\\
127000	-122000000\\
-164000.000000001	-122200000\\
52999.9999999999	366300000\\
-88999.9999999977	-488199999.999999\\
235999.999999995	2074900000\\
-309999.999999997	-4638500000\\
71999.9999999974	4761000000\\
276000.000000002	-1099200000\\
-238999.999999999	-2318700000\\
56000.0000000009	2440800000\\
-74000.0000000052	-1098100000\\
-91999.9999999978	-732799999.999999\\
367000	3418200000\\
-165000.000000004	-3906500000\\
-183999.999999997	488599999.999999\\
185000.000000001	2319100000\\
34999.9999999957	-976500000\\
-291999.999999997	-2563400000\\
53999.9999999976	3784100000\\
330999.999999999	-1220800000\\
-166999.999999999	-854300000\\
-271999.999999998	-976600000.000001\\
601999.999999999	5981300000\\
-274000	-8666700000\\
-108999.999999998	5736800000\\
-92000.0000000023	-3783500000\\
-54999.9999999988	3905600000\\
492999.999999998	244500000\\
-289999.999999999	-4516600000\\
-278000	1953000000\\
515999.999999999	3173800000\\
-131000.000000001	-3539800000\\
-346000.000000001	-854799999.999999\\
236000	4028500000\\
76000.0000000014	-2685500000\\
-222000.000000003	-244300000.000002\\
166000	1220800000\\
-36999.999999999	-366200000\\
-91000.0000000019	-732499999.999999\\
-37999.9999999958	1098700000\\
201999.999999996	-366300000\\
-146000	-610100000\\
202000	1586500000\\
-185000	-2318900000\\
38000.0000000002	1830800000\\
128000.000000003	-488300000\\
19000.0000000001	488399999.999999\\
-240000.000000005	-2563600000\\
111000.000000002	3540300000\\
-35999.9999999996	-2441800000\\
-74000.0000000025	488600000\\
274000.000000001	2563400000\\
-181000.000000002	-4638800000\\
16000.0000000009	4150500000\\
19999.9999999978	-2441400000\\
-203000	-610400000.000002\\
331000.000000002	4028400000\\
-129000	-4516600000\\
1000.00000000122	2685300000\\
145000	-609999999.999998\\
-145000.000000002	-1098900000\\
-73999.9999999998	610600000.000001\\
-238000.000000001	366000000\\
366000	366200000.000001\\
-73000.0000000004	-1098500000\\
90999.9999999993	2197300000\\
-199999.999999997	-4028500000\\
-110999.999999999	3662300000\\
311999.999999998	-610600000\\
145000.000000003	366500000\\
-254000.000000001	-3662400000\\
15999.9999999982	4394900000\\
130000	-1587400000\\
-258000	-1952600000\\
92999.9999999964	3295600000\\
90000.0000000025	-1220800000\\
-108000.000000001	-1464600000\\
-38000.0000000011	1953000000\\
1000.00000000122	-1587000000\\
254999.999999998	3418300000\\
-89999.9999999998	-4516900000\\
-166000.000000001	1342700000\\
257000.000000001	3052100000\\
-239000.000000002	-5249400000\\
148000.000000003	4883200000\\
-147000.000000003	-3418300000\\
-56000	976600000\\
129999.999999999	1587200000\\
400999.999999999	1220400000\\
-585000.000000001	-8788900000\\
72999.9999999995	11596600000\\
19000.000000001	-8422900000\\
108999.999999997	5737400000\\
202000.000000002	-3173800000\\
-403000.000000001	-976599999.999999\\
-19000.0000000019	2075200000\\
56000	-610499999.999999\\
365000.000000004	2685700000\\
-35000.000000001	-4028200000\\
-130000.000000001	1098400000\\
-180999.999999996	-732399999.999999\\
52999.9999999973	2197500000\\
275000.000000002	732099999.999998\\
-254999.999999999	-5004700000\\
-20000.0000000005	4394600000\\
258000.000000001	243900000.000004\\
-203000.000000002	-3539800000\\
-182000	1953100000\\
420000.000000002	2197000000\\
-181999.999999999	-3417700000\\
-256000	244200000.000001\\
327999.999999997	3539700000\\
-164000.000000001	-4882400000\\
37000.0000000017	4638300000\\
-36999.9999999999	-3906000000\\
-17999.9999999971	2319200000\\
219999.999999995	244200000.000001\\
-37999.9999999994	-1220600000\\
-256000	-976799999.999999\\
112000.000000003	2441500000\\
14999.9999999961	-1098400000\\
113000	365800000.000002\\
-20999.999999999	-854200000.000002\\
-327999.999999999	-976599999.999999\\
403000.000000001	4028200000\\
17999.9999999989	-2807600000\\
-441000.000000002	-2074900000\\
369000.000000003	4638200000\\
52999.9999999955	-3051400000\\
-72999.9999999977	732300000.000001\\
-73000.0000000031	-122100000\\
8.88178419700125e-10	-1.4210854715202e-06\\
-366000	-1464800000\\
475999.999999999	5127000000\\
72000.0000000009	-5737400000\\
-161999.999999997	2929700000\\
-113000.000000001	-2075200000\\
-144000	2319400000\\
309999.999999998	-854500000\\
-55000.0000000006	-122099999.999999\\
-54999.9999999979	-366300000\\
165999.999999999	1220900000\\
-147999.999999997	-1587100000\\
202999.999999999	1709000000\\
71999.9999999974	-1586600000\\
-640999.999999999	-1587400000\\
384999.999999997	5493500000\\
-53999.9999999985	-5493300000\\
-167000	2807600000\\
386999.999999999	244300000\\
-386999.999999999	-3051900000\\
588000.000000001	6225600000\\
-311999.999999998	-7080100000\\
-311000.000000003	2563600000\\
456999.999999998	3540000000\\
-54999.9999999997	-5127200000\\
-163999.999999999	1221100000\\
-201000	1464600000\\
125999.999999998	-244100000.000001\\
21000.0000000043	-976599999.999999\\
345999.999999998	3051900000\\
-219000	-4882900000\\
-145999.999999997	3051600000\\
16999.9999999959	-1220400000\\
-16999.999999995	1098400000\\
-55000.0000000041	-1342600000\\
383000	4028100000\\
-181999.999999999	-6103300000\\
110000.000000002	4760600000\\
-91000.0000000011	-3295700000\\
-313000.000000001	1098300000\\
513999.999999997	1831300000\\
-585999.999999998	-3173900000\\
309999.999999998	3051900000\\
20000.0000000022	-2319600000\\
-238000	1098900000\\
180999.999999997	487999999.999999\\
167000.000000005	122299999.999999\\
-183999.999999999	-2807700000\\
201000	5371100000\\
184000.000000002	-4760800000\\
-476000.000000002	-732299999.999998\\
35999.9999999978	4150400000\\
17999.999999998	-2807900000\\
37000.0000000017	854900000\\
-18000.0000000025	488000000.000001\\
329000	1220800000\\
-163999.999999999	-3662200000\\
-257000.000000002	1220800000\\
54999.9999999997	1220800000\\
329999.999999997	1708800000\\
-165999.999999998	-4394500000\\
-70999.9999999988	2319300000\\
52999.9999999964	488499999.999999\\
-91000.0000000002	-976700000.000001\\
54999.9999999988	366000000\\
-219000	-609900000\\
108000	1464400000\\
222000.000000002	-610100000\\
52999.9999999955	244200000.000001\\
-181999.999999999	-2197600000\\
238000.000000001	3296200000\\
-165999.999999999	-2563400000\\
-328000	487800000\\
-1000.00000000211	1099300000\\
276000.000000002	-244800000\\
34999.9999999993	-731900000\\
-8.88178419700125e-10	610099999.999999\\
999.999999998557	-488299999.999999\\
-127999.999999998	122100000.000001\\
-1000.00000000211	-121900000.000002\\
275000.000000002	1830800000\\
-127999.999999997	-2929600000\\
0	2319500000\\
220000	-610600000.000004\\
-404000.000000003	-2929500000\\
1999.99999999978	3784000000\\
16000.0000000009	-1464600000\\
221000	1952800000\\
19000.000000001	-3051400000\\
-130000.000000003	1464600000\\
2000.00000000156	-610299999.999999\\
-276000.000000002	244200000\\
348000.000000001	1953000000\\
200999.999999999	-2075000000\\
-290999.999999997	-854700000\\
-222000.000000003	1098900000\\
93000	976200000\\
383999.999999998	-121900000\\
-384999.999999999	-2929500000\\
275999.999999999	5004700000\\
-276000.000000001	-5859500000\\
1000.00000000122	4883100000\\
419999.999999997	-610599999.999999\\
-401999.999999997	-4150200000\\
-17999.999999998	4028200000\\
273999.999999997	122000000\\
73000.0000000013	-732100000\\
-17999.9999999971	-2197700000\\
-274000.000000001	2075500000\\
201000	-488300000.000002\\
-422000.000000003	-1098900000\\
422000.000000004	4150800000\\
-347000.000000001	-6470000000\\
326999.999999998	7202200000\\
94000.0000000047	-5981500000\\
-349000.000000002	2441500000\\
403000	1586900000\\
-475000	-4516600000\\
326999.999999998	5981400000\\
76000.0000000005	-5126900000\\
-349999.999999998	1831100000\\
238999.999999998	1098400000\\
-91000.0000000002	-1952900000\\
54000.0000000038	1831000000\\
55000.0000000015	-1586800000\\
-238000	1342400000\\
110999.999999998	-1342300000\\
126000	1586600000\\
-53000.0000000026	-1220600000\\
-238000	-122000000.000001\\
255000	854200000\\
165000	854899999.999999\\
-183000.000000002	-2807900000\\
-236999.999999997	1098700000\\
237999.999999998	1953200000\\
163000	-610400000\\
-182000.000000002	-3418000000\\
257000.000000002	5737300000\\
-239000.000000003	-6469600000\\
-438999.999999998	3906000000\\
328999.999999999	-243900000\\
220000.000000001	854400000\\
-218999.999999999	-3051800000\\
219000	3906400000\\
37000.0000000026	-2319500000\\
-92000.0000000023	-1098600000\\
-181999.999999999	976500000\\
-2000.00000000244	1709300000\\
203000.000000001	-1709300000\\
53999.9999999967	732400000.000001\\
-309999.999999998	-2197100000\\
199999.999999999	4028400000\\
-36000.0000000022	-2930000000\\
92000.0000000023	244400000.000001\\
-403000	366100000\\
274000.000000002	854600000\\
145999.999999999	-854700000.000001\\
38999.9999999988	854700000\\
-405999.999999998	-3418000000\\
185999.999999999	5004700000\\
438000	-1342500000\\
-531000.000000002	-4150600000\\
109999.999999999	5737400000\\
275000.000000001	-2807600000\\
-110000.000000002	-244299999.999999\\
-294000	-1098400000\\
39000.0000000024	4028100000\\
436999.999999998	-2929500000\\
-419999.999999998	-1220700000\\
128000.000000002	4150100000\\
146000.000000003	-4150100000\\
-274000.000000002	1953100000\\
-36000.0000000005	-366300000\\
108000.000000001	366200000\\
220999.999999995	488400000.000001\\
-255999.999999999	-1587100000\\
54000.0000000003	976800000\\
238999.999999998	854199999.999999\\
-73999.9999999989	-2196900000\\
-201999.999999997	1464400000\\
-109000	-365700000\\
184000	243600000\\
-57000.0000000004	244600000\\
-145000	-1220900000\\
329999.999999997	1830900000\\
89000.0000000013	-854100000\\
-271999.999999999	-977000000\\
-54999.9999999997	854800000\\
34999.9999999984	244100000\\
128000	243900000.000001\\
39000.0000000032	-610000000.000002\\
33999.9999999971	732200000\\
-199000.000000001	-2563400000\\
34000.0000000016	3295800000\\
38999.9999999979	-1708900000\\
-55999.9999999983	244200000.000001\\
-19000.0000000019	-122199999.999999\\
221000.000000004	2075300000\\
-91999.9999999996	-3296000000\\
-54999.9999999988	1709100000\\
17999.999999998	-854599999.999999\\
-90999.9999999966	1220800000\\
54999.9999999997	-1465000000\\
182999.999999999	3418200000\\
-258000.000000002	-6103700000\\
-51999.999999996	5615400000\\
162999.999999995	-2685800000\\
110000	1831200000\\
-183000.000000001	-3295800000\\
72999.9999999995	4394400000\\
54999.9999999988	-4272400000\\
20000.000000004	3295900000\\
-351000.000000003	-3418100000\\
94000.0000000021	3662300000\\
438999.999999998	-244299999.999999\\
-92000.0000000005	-2807500000\\
-367000.000000001	732400000.000001\\
368000.000000003	1831000000\\
-239000.000000003	-1586800000\\
-201999.999999998	121800000\\
459000	1587300000\\
-239000.000000003	-2685800000\\
-71999.9999999965	2319300000\\
255000.000000001	-488100000\\
-366000	-1709100000\\
330000.000000003	2929800000\\
53999.9999999985	-2319600000\\
-419000	244500000\\
290999.999999999	732100000\\
-73000.0000000013	610600000\\
111000	-1831300000\\
-74000.0000000016	1587200000\\
-109999.999999999	-1098800000\\
218999.999999999	1220600000\\
-108000.000000002	-1830800000\\
-330999.999999998	854299999.999999\\
770000	3173900000\\
-441000.000000003	-5127000000\\
-144999.999999996	976700000.000001\\
328999.999999999	3783900000\\
-71999.9999999992	-3905900000\\
-478999.999999999	243899999.999998\\
496999.999999996	3784200000\\
73000.0000000013	-4638600000\\
-531999.999999997	1587000000\\
421000	1464600000\\
-109000.000000004	-976400000\\
183000.000000002	122000000.000001\\
-74000.0000000034	-1098500000\\
-163999.999999999	732299999.999999\\
-35999.9999999996	488300000\\
-92999.9999999973	-976500000\\
330000.000000001	1830900000\\
-110000	-1830900000\\
-146000.000000003	-99999.9999998558\\
74000.0000000007	1220900000\\
308999.999999999	854200000.000001\\
-381999.999999997	-4516400000\\
70999.9999999962	4882700000\\
222000.000000003	-1342700000\\
-167000.000000003	-1098600000\\
-309999.999999999	-1342800000\\
622000.000000002	5859200000\\
-383999.999999998	-6835600000\\
164999.999999999	4516300000\\
-147000.000000001	-3539900000\\
-19000.000000001	3662000000\\
367000.000000001	-732199999.999999\\
-200999.999999998	-2441600000\\
-220000.000000002	244200000\\
-128000.000000001	1953100000\\
236999.999999998	-122000000.000001\\
110000.000000002	-488300000\\
20000.0000000013	610200000.000001\\
71999.9999999965	-1953000000\\
-422000	366300000\\
-71000.0000000006	1586800000\\
327000	-122000000.000002\\
204000	-199999.999999356\\
16000	-732100000.000002\\
-273000.000000001	-1343100000\\
-202000.000000001	1587200000\\
91000.0000000002	-122200000.000002\\
220000.000000001	1220500000\\
38000.000000002	-1464400000\\
70999.9999999971	610100000\\
-437999.999999998	-2685700000\\
257000.000000001	4516800000\\
125999.999999998	-2197200000\\
-327999.999999998	-2197400000\\
383999.999999999	5615300000\\
-109999.999999999	-5249200000\\
-219000.000000001	488600000.000002\\
255000.000000001	3539800000\\
148000.000000003	-1342800000\\
-403000.000000002	-4272200000\\
-74000.0000000034	4638400000\\
219000.000000002	-854400000\\
57000.0000000013	122200000\\
71999.9999999983	121700000\\
-311000.000000003	-3051200000\\
292000.000000002	4760200000\\
-17000.0000000003	-2685100000\\
-273999.999999996	-1343200000\\
253999.999999994	3540300000\\
-291999.999999997	-3417800000\\
367000.000000003	3051200000\\
17999.9999999998	-853799999.999999\\
-127999.999999999	-1831700000\\
71999.9999999965	1343300000\\
-401000.000000001	-1221100000\\
364000	3052000000\\
-16000.0000000009	-3173900000\\
-93000	2075200000\\
-146999.999999999	-2685600000\\
149000.000000001	3662200000\\
143999.999999997	-2319400000\\
-72999.9999999968	976500000.000001\\
148000.000000001	-243900000.000002\\
-184000.000000002	-1465100000\\
-37000.0000000017	1587100000\\
-54000.0000000003	-732700000\\
-129000	610800000\\
348000	243599999.999999\\
-92000.0000000005	-243600000\\
-199999.999999997	-1953600000\\
182999.999999998	3662500000\\
35000.0000000001	-2808000000\\
18999.9999999983	1343200000\\
-145999.999999998	-1343000000\\
348000.000000002	3051600000\\
-275000.000000002	-4760500000\\
-55000.0000000006	3540000000\\
-19000.0000000019	-1831200000\\
-52999.9999999982	1465100000\\
235999.999999999	366000000\\
37999.9999999985	-1831000000\\
-221000.000000001	366200000\\
-199999.999999997	244100000.000001\\
310999.999999998	1221000000\\
126999.999999999	-488799999.999999\\
-382999.999999999	-2563000000\\
272999.999999999	4028000000\\
203000.000000003	-976400000\\
-165000.000000002	-2929700000\\
-93000.0000000026	2075200000\\
-126999.999999999	-122100000\\
-73000.0000000022	-976700000.000001\\
311000.000000001	3784500000\\
-147000.000000003	-5249400000\\
37000.0000000017	2930100000\\
-1.77635683940025e-09	243700000.000001\\
17999.9999999998	-732000000.000001\\
-17000.0000000003	-854799999.999998\\
162999.999999999	2685600000\\
-274000.000000001	-4516400000\\
165000.000000002	5004700000\\
-18000.0000000016	-3784200000\\
-292999.999999999	1220700000\\
218999.999999998	854699999.999999\\
128000	-300000.000000011\\
184000	366499999.999997\\
-713999.999999999	-4760900000\\
603000.000000001	7080000000\\
111000.000000001	-3295700000\\
-164999.999999999	-1098700000\\
-476000.000000002	-244399999.999999\\
457000.000000002	4028800000\\
-71999.9999999992	-4394900000\\
-111000.000000004	1953300000\\
239000.000000003	854300000.000001\\
-294000.000000005	-3051500000\\
-109999.999999996	1953000000\\
459999.999999999	2441400000\\
51999.9999999987	-2441500000\\
-364000	-2685300000\\
16999.9999999986	4272200000\\
146999.999999998	-1830900000\\
54999.9999999988	976499999.999999\\
-183999.999999998	-2197200000\\
202999.999999998	2807600000\\
-349000.000000001	-3174000000\\
220000.000000002	3662400000\\
-165000	-3662300000\\
257000.000000001	3784200000\\
-183000	-3173700000\\
383000	3661800000\\
-237000.000000001	-4638300000\\
-73000.0000000004	2441300000\\
-73999.9999999981	-1221000000\\
202000.000000002	2808100000\\
-54999.9999999997	-2808000000\\
-146000	732500000.000001\\
-19000.0000000028	244399999.999999\\
273999.999999998	1098300000\\
-127000	-1953000000\\
-275000	122100000\\
128999.999999999	1586900000\\
382000.000000001	122099999.999999\\
-252999.999999997	-1953200000\\
-76000.0000000005	488300000\\
-310000	-122100000\\
420999.999999997	2075400000\\
421000.000000003	243900000\\
-678000.000000002	-5126900000\\
148000.000000002	4394600000\\
71999.9999999974	-122100000\\
-91000.0000000011	-1831200000\\
164000.000000001	1953500000\\
1000.00000000033	-976900000\\
-55000.0000000006	-854499999.999999\\
-37999.9999999985	1098900000\\
-182000.000000003	-488599999.999999\\
-36000.0000000005	299999.999999656\\
329000.000000001	1708800000\\
126999.999999998	-2075100000\\
-162999.999999999	610200000\\
-38000.000000002	-366099999.999999\\
-254999.999999998	0\\
16999.9999999977	122100000\\
258000	1586900000\\
34000.0000000007	-1709100000\\
94000.0000000003	488400000\\
-239000.000000003	-610400000\\
-17999.9999999989	244300000\\
-37000.0000000017	121699999.999999\\
17999.9999999998	488699999.999999\\
147000	-300000.000000367\\
183000.000000002	732599999.999999\\
-90999.9999999984	-2563600000\\
-312000.000000005	610500000.000001\\
110000	2197000000\\
54999.9999999988	-1708600000\\
146000	732099999.999999\\
-327999.999999998	-1342700000\\
180999.999999996	1465000000\\
38000.0000000038	-200000.0000006\\
145999.999999999	-610300000\\
-164000.000000001	-3.5527136788005e-07\\
-293999.999999997	-732200000\\
219000	1342400000\\
-53000.0000000035	-487999999.999999\\
219000.000000002	610199999.999998\\
-91999.9999999996	-976499999.999998\\
91999.9999999987	976499999.999998\\
-164999.999999999	-1830900000\\
-53999.9999999985	1464700000\\
34999.9999999966	-244100000\\
293000.000000004	2441300000\\
1999.999999998	-4516300000\\
-405000.000000001	732100000\\
93000.0000000008	2563600000\\
17000.0000000021	-1342700000\\
367999.999999998	2441200000\\
-56999.9999999986	-4028000000\\
-328000.000000002	-300000.000000722\\
-110000.000000001	2563700000\\
219000.000000001	365900000\\
91000.0000000002	-2563100000\\
-273000	1830800000\\
-1000.00000000033	-610200000\\
111000.000000002	366100000\\
163000	100000.000000389\\
18999.9999999992	-200000.000001754\\
-164000.000000001	-854300000\\
-74999.9999999975	-3.5527136788005e-07\\
111999.999999997	1586900000\\
-999.999999998557	-1098800000\\
72999.9999999977	244400000\\
36999.9999999999	-244399999.999999\\
-73999.9999999981	244400000\\
-109000.000000004	-854700000.000001\\
-108999.999999996	976700000.000001\\
199000.000000002	-122199999.999999\\
129999.999999996	732500000\\
-36999.9999999981	-1586900000\\
-402999.999999997	3.5527136788005e-07\\
494000.000000001	1831000000\\
-293000.000000002	-1220700000\\
19999.9999999987	-854400000\\
15999.9999999991	1708900000\\
94000.0000000021	-244100000.000001\\
198999.999999998	-488299999.999999\\
-493000	-1586900000\\
310000	3173700000\\
94000.0000000012	-1586700000\\
-149000.000000001	-610400000\\
-219000.000000001	-244400000\\
184000.000000003	2075700000\\
182999.999999998	-1221300000\\
-184999.999999999	500000.000000611\\
167000.000000002	243899999.999999\\
-110000.000000002	-976600000\\
-75000.0000000019	854699999.999999\\
-53999.9999999985	-854700000\\
238999.999999999	2929700000\\
-258000	-4638400000\\
185000.000000002	3905800000\\
-2000.00000000333	-1586500000\\
-419999.999999997	-2319600000\\
732000	7690500000\\
-90999.9999999975	-7568200000\\
-311000.000000003	488099999.999999\\
-166000	2319300000\\
91999.9999999961	-244100000.000001\\
55000.0000000024	-121800000.000001\\
54999.9999999979	487900000\\
37000.0000000017	-976500000\\
35999.9999999969	366500000.000001\\
54999.9999999997	1342400000\\
-163999.999999997	-3661800000\\
-93000.0000000044	3051500000\\
148000	-366000000\\
182000.000000002	122000000\\
-183000.000000004	-1465000000\\
-328999.999999998	610599999.999999\\
366000.000000001	1586800000\\
-110000	-2197300000\\
1000.00000000122	1342900000\\
253999.999999998	610100000\\
-108000	-1342500000\\
-109999.999999997	-488300000\\
-202000.000000002	121700000.000001\\
166000.000000001	1953700000\\
70999.9999999979	-1343300000\\
2000.00000000333	366599999.999998\\
-1000.00000000211	-610599999.999998\\
129000.000000002	1709200000\\
-129000.000000001	-3052100000\\
165000	3174200000\\
-165000.000000003	-2807800000\\
-439000	-366199999.999998\\
403000.000000001	4150400000\\
366000.000000002	-610300000\\
-623000.000000001	-5981600000\\
164999.999999998	5615500000\\
202000.000000005	-732700000\\
182999.999999996	1220900000\\
-110999.999999999	-4150600000\\
-182000.000000001	1465200000\\
-293000.000000001	487900000\\
309999.999999999	1709200000\\
39000.0000000024	-1953300000\\
106999.999999998	610699999.999999\\
-326999.999999998	-610800000.000001\\
254999.999999997	1221000000\\
-164999.999999999	-1586900000\\
184000	1708800000\\
-183000.000000002	-1586900000\\
-74999.9999999993	610600000.000001\\
350000	1342500000\\
-312000.000000001	-2319200000\\
108999.999999998	1220600000\\
147000.000000001	610599999.999999\\
129000	121700000\\
-404000.000000001	-3173600000\\
999.999999997669	2563500000\\
329000.000000001	1709000000\\
-202000.000000003	-3784500000\\
56000.0000000036	2564000000\\
-220000.000000001	-1465300000\\
219999.999999999	1709200000\\
145999.999999999	-732500000.000001\\
-92000.0000000023	-1098500000\\
-199999.999999997	610200000\\
-202999.999999999	610500000\\
275999.999999999	-610500000\\
145000	1220700000\\
130000.000000001	-1098600000\\
-405000	-1220500000\\
19999.9999999987	2197000000\\
128000.000000001	-854400000.000001\\
293000	732400000\\
-294000.000000004	-1586800000\\
-108999.999999998	488000000\\
91999.9999999987	732800000\\
127000	365900000.000001\\
-128000.000000001	-1830800000\\
56000.0000000009	1830800000\\
108999.999999998	-244000000\\
-128999.999999999	-1831000000\\
-143999.999999998	1464700000\\
216999.999999997	976600000\\
-162999.999999997	-2075000000\\
17999.9999999989	854099999.999999\\
-54999.9999999997	244499999.999999\\
439000	1831000000\\
-293000.000000003	-3784400000\\
-328999.999999997	610599999.999999\\
585999.999999998	4638500000\\
-93000	-5493100000\\
-180999.999999997	1587000000\\
-93000.0000000044	1464600000\\
1000.00000000033	-1830700000\\
146000	1586700000\\
-110999.999999996	-1220800000\\
-218000.000000005	-244000000\\
732000.000000002	3051800000\\
-532000.000000002	-3418000000\\
-199999.999999998	-366400000.000002\\
255999.999999997	3174200000\\
182000.000000003	-1709400000\\
-54000.000000002	-732200000\\
-54999.9999999997	1342900000\\
-292999.999999997	-1587200000\\
19000.0000000001	610600000\\
272999.999999998	2075000000\\
184000	-1586800000\\
-145999.999999999	-1098700000\\
-239000.000000002	244200000\\
184000.000000001	1953100000\\
-110000	-1708900000\\
55000.0000000015	732200000\\
-19000.0000000037	-732100000.000001\\
128000	1342400000\\
999.999999999446	-854300000\\
18000.0000000007	-366000000\\
-147000.000000001	-399999.999998713\\
-55000.0000000006	122300000\\
-71999.9999999974	610400000\\
439000	243899999.999999\\
-312000.000000001	-854199999.999999\\
-274999.999999999	-1465100000\\
367999.999999998	3662400000\\
89999.999999999	-1709400000\\
-220000	-1586500000\\
-15999.9999999965	1708700000\\
454999.999999996	1465100000\\
-731000.000000001	-4638900000\\
403000.000000001	3540000000\\
146000	1099000000\\
-256000	-4272900000\\
71999.9999999992	4028600000\\
-235999.999999997	-3662200000\\
365000.000000001	5493100000\\
183000.000000001	-4272300000\\
-531000.000000003	-1587000000\\
129000.000000006	3173700000\\
200999.999999997	1098900000\\
-146999.999999997	-3540300000\\
91999.9999999996	3052000000\\
-202000.000000001	-3296200000\\
330999.999999999	4883100000\\
-37999.9999999958	-3662200000\\
-36000.0000000022	-976600000\\
-310999.999999998	1342700000\\
199999.999999997	2197500000\\
38000.0000000011	-3174100000\\
-237999.999999997	1465000000\\
72999.9999999968	-122100000.000001\\
474999.999999999	2319400000\\
-364999.999999998	-4394600000\\
-90999.9999999984	1464700000\\
254999.999999997	2197600000\\
-311000.000000001	-2807900000\\
366000	2563600000\\
-310000	-3662200000\\
108000.000000004	4394700000\\
-163000.000000004	-2685700000\\
273000.000000001	488400000\\
-16000	854200000\\
-93999.9999999959	-1708500000\\
18999.9999999992	1342400000\\
-292000.000000001	-854400000\\
439000.000000001	1342800000\\
-147000.000000002	-1098700000\\
-385000	-732200000\\
440999.999999999	2074900000\\
53999.9999999985	-732200000.000002\\
92000.0000000014	-732599999.999999\\
-403000.000000003	-488099999.999999\\
0	610199999.999998\\
640000	3051700000\\
-383000.000000002	-4272100000\\
-183999.999999998	-610700000\\
91000.0000000019	4150400000\\
-182000	-2807400000\\
36000.0000000005	-122200000\\
128999.999999998	2807500000\\
439000.000000002	-1952800000\\
-714999.999999999	-2075600000\\
201999.999999997	2808000000\\
312000.000000002	1220400000\\
-56000.0000000027	-3051600000\\
37000.0000000026	488300000\\
-439000.000000001	244100000.000001\\
109999.999999999	243999999.999999\\
126999.999999999	1343000000\\
367000.000000001	-488500000.000001\\
-255999.999999999	-2074800000\\
-404000.000000003	609800000.000001\\
386000.000000001	1709400000\\
182000	-366400000\\
-292999.999999998	-1953000000\\
-383999.999999998	1464700000\\
109999.999999999	-610300000\\
641000	2807700000\\
-74000.0000000007	-2929800000\\
-146000.000000001	-610200000.000001\\
-238999.999999996	1464600000\\
422999.999999996	200000.0000006\\
-697999.999999998	366100000\\
404999.999999999	-1586700000\\
-110999.999999999	610000000\\
18000.0000000007	732700000\\
18999.9999999957	244000000\\
220000.000000001	-366000000\\
-56000.0000000009	-488699999.999999\\
-73000.0000000013	-243600000.000002\\
999.999999999446	854000000.000002\\
-999.999999999446	-610100000.000002\\
36999.9999999999	976599999.999999\\
-165000.000000004	-2441600000\\
-91999.9999999961	2441700000\\
386000.000000001	731999999.999998\\
164000.000000001	-1098200000\\
-239000.000000003	-2441600000\\
-731000.000000001	1098600000\\
659000.000000002	1831000000\\
200999.999999998	1831200000\\
-37000.0000000008	-4638600000\\
-365999.999999999	1586600000\\
-110000.000000001	-610000000\\
166000.000000002	2441100000\\
328999.999999999	-976300000\\
-221000	-732599999.999998\\
74999.9999999993	-244100000\\
35000.0000000028	1709100000\\
-163000	-3174000000\\
364999.999999998	3662100000\\
-458000.000000001	-2929400000\\
-164000.000000001	365800000.000001\\
330000.000000001	2563900000\\
34999.9999999993	-2197700000\\
-52999.999999999	366600000\\
180999.999999996	1220400000\\
-382999.999999999	-2929500000\\
292999.999999999	2441400000\\
71999.9999999983	1098500000\\
-108999.999999999	-3906100000\\
-146000.000000001	2563400000\\
-146999.999999999	-244100000\\
383999.999999999	610200000\\
-54000.0000000003	-976299999.999999\\
-366999.999999999	-854700000\\
-17999.9999999989	1098700000\\
769999.999999999	2441400000\\
-459999.999999997	-4150300000\\
-199000.000000002	1098400000\\
-38999.999999997	976800000\\
569999.999999998	1220600000\\
-93000.0000000008	-1831000000\\
-238000.000000001	-1953300000\\
-566999.999999997	3052100000\\
365999.999999996	-2075500000\\
329000.000000002	4150400000\\
-108999.999999996	-4760600000\\
-37000.0000000017	2441400000\\
165000.000000004	366200000\\
72999.9999999977	-976800000\\
-330000	-2441000000\\
-108999.999999997	2807400000\\
347000.000000001	2319200000\\
-109000.000000001	-4760500000\\
-146999.999999998	1586800000\\
128999.999999999	1464900000\\
-19999.9999999987	-732499999.999998\\
-162999.999999999	-2075100000\\
473999.999999998	6225400000\\
-106999.999999998	-6835700000\\
-94000.0000000012	854400000\\
-274000.000000003	2197200000\\
-54000.0000000003	-732300000.000001\\
181999.999999999	976400000\\
54999.9999999988	-976400000.000001\\
-273999.999999999	-1098800000\\
530999.999999999	3540200000\\
-276000.000000001	-3906200000\\
-162999.999999998	1952800000\\
456999.999999999	366600000.000001\\
-421000.000000001	-1953500000\\
-38000.0000000002	732800000\\
204000	2197100000\\
218000	-1343000000\\
-202000.000000002	-1830600000\\
-291000.000000001	976100000\\
292000.000000002	1343200000\\
-422000.000000001	-2319700000\\
348999.999999999	4028600000\\
330000.000000004	-2807900000\\
-368000.000000003	-1098300000\\
38000.0000000011	1220400000\\
-109999.999999999	732600000.000003\\
329000.000000001	854399999.999998\\
74999.9999999966	-1953000000\\
-258999.999999999	-976600000\\
74999.9999999984	1464600000\\
-294000	-121699999.999998\\
38000.0000000002	610199999.999999\\
89999.999999999	-488500000.000001\\
221000	1099000000\\
-109999.999999999	-3051900000\\
-166000.000000004	3051600000\\
130000.000000004	-1342500000\\
-20000.0000000031	243899999.999999\\
-36000.0000000014	366300000\\
238000.000000001	-244099999.999999\\
54999.9999999979	244199999.999999\\
-329000	-1709200000\\
17999.9999999998	1587100000\\
-999.999999998557	244100000\\
-35999.9999999978	-976700000\\
-54000.000000002	610700000\\
272999.999999999	243700000\\
-89999.9999999998	200000.000000511\\
-202000.000000003	-1952900000\\
475000	4149900000\\
-510999.999999997	-5004300000\\
35999.9999999996	2196700000\\
567000	4150700000\\
-346999.999999998	-7080000000\\
-146999.999999998	2441200000\\
73999.9999999989	2319400000\\
255000	-610300000.000001\\
-219000	-2807800000\\
73000.0000000004	2441800000\\
-181999.999999999	-1343200000\\
-313000.000000002	-121900000\\
331000.000000002	1953300000\\
274000.000000002	243800000.000002\\
109999.999999999	-366000000.000003\\
-274999.999999999	-3295900000\\
110999.999999999	2807600000\\
-313000.000000001	-1464900000\\
38000.000000002	1098700000\\
494000	2319300000\\
-385000.000000001	-4272500000\\
-18000.0000000033	1098700000\\
-273999.999999997	976500000\\
437999.999999998	610500000.000001\\
276000	610200000\\
-567999.999999999	-4394500000\\
218999.999999998	4028300000\\
-291999.999999998	-1953100000\\
128000.000000003	2441500000\\
274999.999999999	-1953300000\\
17000.0000000003	1098800000\\
-72000.0000000018	-2197400000\\
-257000.000000001	1587000000\\
219999.999999999	366200000\\
55000.0000000006	-244200000\\
-111000.000000002	-976599999.999999\\
-33999.9999999963	1098900000\\
14999.9999999988	-244500000\\
-89000.0000000022	-610000000\\
-74000.0000000007	487900000\\
146999.999999999	244400000\\
217999.999999999	488299999.999999\\
1000.00000000033	-854599999.999999\\
-218999.999999999	-488299999.999999\\
-8.88178419700125e-10	488399999.999999\\
-74999.9999999975	-99999.9999997669\\
19999.9999999969	244100000.000001\\
146000	366300000\\
18000.0000000007	-244000000\\
109999.999999999	-488600000.000001\\
-530999.999999999	-1830900000\\
219999.999999997	5005000000\\
494000.000000001	-3174100000\\
-53999.9999999985	300000.000000367\\
-587000.000000001	-2075400000\\
-18999.9999999983	4150500000\\
423000	-1587000000\\
-55000.0000000006	-121999999.999999\\
71999.9999999992	-732499999.999999\\
-439999.999999998	99999.9999994117\\
185000	244100000.000001\\
493999.999999999	4272300000\\
-148000.000000003	-7445900000\\
-565999.999999999	3417500000\\
-92000.0000000005	366600000\\
548999.999999998	854300000\\
292999.999999998	-244100000\\
-494000.000000001	-3051800000\\
-256999.999999999	1953100000\\
312000	1465000000\\
237999.999999999	-122200000\\
-329999.999999998	-3906200000\\
17999.9999999971	5371100000\\
440000.000000005	-2319500000\\
-311000.000000001	-1708600000\\
-348000.000000002	243600000.000001\\
566000	3906800000\\
-418999.999999997	-4516900000\\
72999.9999999995	1586800000\\
310000	2930000000\\
93000.0000000008	-4028500000\\
-770000.000000001	-1587000000\\
182999.999999997	5371400000\\
421000.000000002	-1831400000\\
148000.000000001	244400000\\
-294999.999999999	-2929900000\\
-162999.999999999	1709100000\\
274000.000000001	2319400000\\
90999.9999999984	-1953200000\\
-329000.000000003	-2319400000\\
-36999.999999999	3173900000\\
182999.999999999	100000.000000477\\
19000.000000001	-854799999.999998\\
-73999.9999999989	-976200000.000003\\
148000	2807400000\\
-149000.000000001	-4272400000\\
-108000	3173800000\\
403000.000000001	1464900000\\
-239000.000000003	-4150500000\\
-73000.0000000013	1831300000\\
109000.000000001	487899999.999998\\
-199000.000000002	-609999999.999999\\
-93999.9999999994	-366399999.999999\\
422000	2197300000\\
-255999.999999998	-2319200000\\
2.66453525910038e-09	121799999.999999\\
128000	1465000000\\
128000.000000001	-244100000\\
-146000.000000003	-1220800000\\
181999.999999999	610499999.999998\\
-493000	-1342900000\\
-75000.0000000002	1342700000\\
716000	3296100000\\
-129000	-5005000000\\
-659000.000000002	-1098700000\\
311000	5249200000\\
256000.000000003	-1953200000\\
-218000.000000001	-1587000000\\
-2000.00000000244	1220700000\\
385000.000000002	2319600000\\
-274000.000000002	-4761100000\\
-183999.999999997	1343000000\\
330999.999999999	3540000000\\
-185000.000000002	-4028400000\\
-144999.999999999	488400000\\
72999.9999999995	1708900000\\
90999.9999999975	488300000\\
19000.0000000019	-2197300000\\
181999.999999998	2075200000\\
-273999.999999997	-2685300000\\
19000.000000001	2074700000\\
-147000.000000002	-1098200000\\
255999.999999998	1708800000\\
0	-854399999.999999\\
-364999.999999999	-2441500000\\
400999.999999999	4272400000\\
91999.9999999987	-732200000.000001\\
-163999.999999999	-3906400000\\
-255999.999999998	1831100000\\
455999.999999996	4028200000\\
-108000	-5126900000\\
-348999.999999999	366500000\\
36999.9999999972	3417500000\\
329000	-2074800000\\
-35999.9999999996	-122400000\\
19000.0000000001	-121899999.999999\\
-440999.999999996	366400000\\
164999.999999998	-400000.000000489\\
110999.999999998	-366000000\\
109000.000000003	1464900000\\
-72000.0000000001	-1464900000\\
15999.9999999982	244099999.999999\\
350000.000000002	1953100000\\
-386000.000000004	-4516400000\\
-455999.999999999	1342500000\\
436999.999999998	3784300000\\
369000.000000003	-366100000.000002\\
-295000.000000003	-5127200000\\
-274000.000000003	1953400000\\
402000.000000004	5615000000\\
-35000.000000001	-7934400000\\
-110999.999999997	4272400000\\
-72999.9999999995	-1220700000\\
-146000	-488399999.999999\\
457000.000000002	3540200000\\
-255000.000000002	-4638700000\\
-313000.000000001	976499999.999998\\
329999.999999999	2319500000\\
74000.0000000007	-366500000.000001\\
237000.000000001	488500000\\
-146000.000000001	-5737400000\\
-914999.999999999	3662300000\\
1025000	5614900000\\
73999.9999999998	-7079800000\\
-478000.000000002	366099999.999999\\
-237000.000000001	2319400000\\
238999.999999999	121900000\\
35000.0000000019	-1220500000\\
385999.999999998	2685400000\\
-166000.000000001	-4516400000\\
-128000	2563300000\\
38000.0000000029	243999999.999999\\
-295000.000000001	-1464400000\\
238999.999999997	2196800000\\
-53999.9999999976	-2074800000\\
-130000.000000001	365800000\\
331000	2563900000\\
181999.999999998	-2686000000\\
-457000.000000002	-854000000.000001\\
-54999.9999999979	2074900000\\
129000	-366200000\\
327000.000000001	244100000.000001\\
-492000.000000002	-1342600000\\
348000	1342700000\\
-93999.9999999985	122099999.999999\\
-363000	-3052000000\\
254000	4516900000\\
73999.9999999972	-2441400000\\
8.88178419700125e-10	-299999.999999034\\
238000.000000001	2075500000\\
146999.999999999	-1831100000\\
-477000.000000002	-2441700000\\
-72999.9999999986	3662500000\\
257000	-200000.000000244\\
-73999.9999999963	-2929700000\\
-403000	2075300000\\
476999.999999998	854400000\\
37000.0000000017	-610200000.000001\\
-185000.000000003	-1709300000\\
20000.0000000013	1220900000\\
53999.9999999967	610599999.999999\\
2.66453525910038e-09	-732900000\\
-109000	-610100000\\
164000.000000001	2441500000\\
-219000.000000001	-4638900000\\
584999.999999997	7690600000\\
-273999.999999999	-8667000000\\
-440000	4638500000\\
-36000.0000000005	-1586700000\\
475000	2929500000\\
-70999.9999999979	-2929500000\\
88999.9999999986	976499999.999999\\
-401000	-1709100000\\
237000.000000004	3906300000\\
145999.999999997	-2929500000\\
-382999.999999998	-854799999.999999\\
-19000.0000000001	2807800000\\
292999.999999999	-1830900000\\
-92000.0000000023	732000000\\
-35999.999999996	-487900000.000001\\
403000	1708700000\\
17999.9999999971	-1708600000\\
-787999.999999997	-2808100000\\
347999.999999998	4883200000\\
513999.999999998	366099999.999998\\
-678999.999999998	-5005100000\\
-72000.0000000018	2075500000\\
567000	4028200000\\
-421999.999999999	-6347600000\\
222000	4516400000\\
308999.999999999	610599999.999999\\
-310000	-5127200000\\
-129000.000000002	3662500000\\
-53999.9999999976	-1465300000\\
34999.9999999984	2075500000\\
350000.000000004	243999999.999999\\
-55999.9999999991	-2929700000\\
};
\addplot [color=mycolor2, line width=2.0pt, forget plot]
  table[row sep=crcr]{%
-201000	-201000\\
255999.999999999	255999.999999999\\
-125999.999999999	-125999.999999999\\
143000	143000\\
149999.999999999	149999.999999999\\
-534000	-534000\\
239999.999999999	239999.999999999\\
-18999.9999999992	-18999.9999999992\\
129000.000000001	129000.000000001\\
254999.999999997	254999.999999997\\
-348000	-348000\\
-145000	-145000\\
-384999.999999998	-384999.999999998\\
530000.000000001	530000.000000001\\
311999.999999995	311999.999999995\\
-237999.999999999	-237999.999999999\\
109999.999999998	109999.999999998\\
-36999.9999999981	-36999.9999999981\\
-623000.000000002	-623000.000000002\\
807000.000000005	807000.000000005\\
-184000.000000003	-184000.000000003\\
-457000	-457000\\
437999.999999999	437999.999999999\\
2000.00000000067	2000.00000000067\\
-368000.000000001	-368000.000000001\\
402999.999999999	402999.999999999\\
-17999.9999999989	-17999.9999999989\\
-327999.999999998	-327999.999999998\\
253999.999999997	253999.999999997\\
73999.9999999989	73999.9999999989\\
-110000	-110000\\
20000.000000004	20000.000000004\\
-204000.000000005	-204000.000000005\\
130000	130000\\
109000	109000\\
-73000.0000000013	-73000.0000000013\\
-36999.9999999999	-36999.9999999999\\
56000	56000\\
-184000.000000002	-184000.000000002\\
347999.999999999	347999.999999999\\
-36999.9999999999	-36999.9999999999\\
-218999.999999998	-218999.999999998\\
202000.000000003	202000.000000003\\
-203000.000000006	-203000.000000006\\
-163999.999999997	-163999.999999997\\
420999.999999998	420999.999999998\\
-236999.999999997	-236999.999999997\\
-75000.0000000019	-75000.0000000019\\
146999.999999998	146999.999999998\\
74000.0000000034	74000.0000000034\\
-275000	-275000\\
402999.999999999	402999.999999999\\
-37000.0000000008	-37000.0000000008\\
-257000.000000001	-257000.000000001\\
-144999.999999999	-144999.999999999\\
236999.999999998	236999.999999998\\
237999.999999999	237999.999999999\\
-382999.999999997	-382999.999999997\\
-38999.9999999979	-38999.9999999979\\
294999.999999997	294999.999999997\\
-74999.9999999966	-74999.9999999966\\
-218000.000000003	-218000.000000003\\
182000.000000001	182000.000000001\\
-201000.000000002	-201000.000000002\\
366000.000000001	366000.000000001\\
-201000.000000002	-201000.000000002\\
72000.0000000018	72000.0000000018\\
-70999.9999999988	-70999.9999999988\\
88999.9999999977	88999.9999999977\\
-70999.9999999971	-70999.9999999971\\
-20000.000000004	-20000.000000004\\
56000.0000000018	56000.0000000018\\
-110000	-110000\\
-110000.000000003	-110000.000000003\\
72000.0000000027	72000.0000000027\\
313000	313000\\
-91999.9999999978	-91999.9999999978\\
91999.9999999996	91999.9999999996\\
-514000.000000001	-514000.000000001\\
238999.999999999	238999.999999999\\
201000.000000004	201000.000000004\\
18999.9999999983	18999.9999999983\\
-313000	-313000\\
21000.0000000017	21000.0000000017\\
436999.999999998	436999.999999998\\
-218999.999999998	-218999.999999998\\
-53999.9999999985	-53999.9999999985\\
17000.0000000003	17000.0000000003\\
-108000.000000002	-108000.000000002\\
-75999.9999999996	-75999.9999999996\\
277000.000000002	277000.000000002\\
-1000.00000000211	-1000.00000000211\\
-237999.999999999	-237999.999999999\\
-54000.0000000003	-54000.0000000003\\
52999.9999999973	52999.9999999973\\
240000.000000002	240000.000000002\\
-111000.000000002	-111000.000000002\\
-128000	-128000\\
165000.000000002	165000.000000002\\
-183000.000000001	-183000.000000001\\
311000	311000\\
-219999.999999999	-219999.999999999\\
148000	148000\\
-75000.0000000028	-75000.0000000028\\
-201000	-201000\\
329999.999999999	329999.999999999\\
-202000.000000003	-202000.000000003\\
-72999.9999999977	-72999.9999999977\\
129000	129000\\
89999.999999999	89999.999999999\\
-163000	-163000\\
127000	127000\\
-147000.000000002	-147000.000000002\\
2000.00000000333	2000.00000000333\\
217999.999999996	217999.999999996\\
-255999.999999998	-255999.999999998\\
35999.9999999996	35999.9999999996\\
-70999.9999999997	-70999.9999999997\\
272000	272000\\
-89999.9999999998	-89999.9999999998\\
237999.999999999	237999.999999999\\
-550000.000000001	-550000.000000001\\
36999.999999999	36999.999999999\\
586000.000000001	586000.000000001\\
-421000	-421000\\
53999.9999999985	53999.9999999985\\
221000	221000\\
-129000.000000002	-129000.000000002\\
-366000	-366000\\
274999.999999998	274999.999999998\\
-110999.999999998	-110999.999999998\\
112000.000000001	112000.000000001\\
180999.999999996	180999.999999996\\
19000.0000000028	19000.0000000028\\
-293000	-293000\\
129000.000000001	129000.000000001\\
-38000.0000000038	-38000.0000000038\\
1000.00000000211	1000.00000000211\\
182999.999999999	182999.999999999\\
-202000.000000002	-202000.000000002\\
-54000.0000000003	-54000.0000000003\\
126999.999999998	126999.999999998\\
-126999.999999998	-126999.999999998\\
-20000.0000000013	-20000.0000000013\\
-33999.9999999963	-33999.9999999963\\
454999.999999996	454999.999999996\\
-161999.999999996	-161999.999999996\\
-223000.000000003	-223000.000000003\\
3000.000000001	3000.000000001\\
217999.999999998	217999.999999998\\
-109000	-109000\\
-221000.000000001	-221000.000000001\\
-53000.0000000017	-53000.0000000017\\
456000.000000001	456000.000000001\\
-218000.000000003	-218000.000000003\\
-56999.9999999995	-56999.9999999995\\
111000	111000\\
18999.9999999983	18999.9999999983\\
-20000.0000000005	-20000.0000000005\\
-34999.9999999993	-34999.9999999993\\
-257000	-257000\\
-74000.0000000007	-74000.0000000007\\
386000.000000001	386000.000000001\\
90999.9999999984	90999.9999999984\\
145000	145000\\
-289999.999999997	-289999.999999997\\
-40000.0000000036	-40000.0000000036\\
75999.9999999996	75999.9999999996\\
-477000	-477000\\
548000.000000001	548000.000000001\\
-199000	-199000\\
34999.9999999984	34999.9999999984\\
-109999.999999999	-109999.999999999\\
203000.000000002	203000.000000002\\
254000	254000\\
-309000	-309000\\
126999.999999999	126999.999999999\\
-257000.000000001	-257000.000000001\\
57000.0000000022	57000.0000000022\\
88999.9999999977	88999.9999999977\\
76000.0000000014	76000.0000000014\\
-76000.0000000014	-76000.0000000014\\
-108000	-108000\\
164000.000000001	164000.000000001\\
-181999.999999999	-181999.999999999\\
235999.999999997	235999.999999997\\
-419000	-419000\\
346000.000000001	346000.000000001\\
203000.000000003	203000.000000003\\
-110000.000000003	-110000.000000003\\
-349000	-349000\\
-55000.0000000024	-55000.0000000024\\
294000.000000003	294000.000000003\\
-35999.9999999996	-35999.9999999996\\
89000.0000000004	89000.0000000004\\
-51999.9999999987	-51999.9999999987\\
88999.9999999986	88999.9999999986\\
-145000	-145000\\
-90999.9999999975	-90999.9999999975\\
71999.9999999965	71999.9999999965\\
274000.000000001	274000.000000001\\
-107999.999999997	-107999.999999997\\
-495000.000000001	-495000.000000001\\
493999.999999998	493999.999999998\\
36999.9999999999	36999.9999999999\\
-440000.000000001	-440000.000000001\\
274999.999999999	274999.999999999\\
127999.999999999	127999.999999999\\
-72999.9999999977	-72999.9999999977\\
-219000	-219000\\
235999.999999997	235999.999999997\\
-162999.999999997	-162999.999999997\\
-55000.0000000015	-55000.0000000015\\
365000.000000002	365000.000000002\\
-109000.000000002	-109000.000000002\\
-200999.999999997	-200999.999999997\\
-129000.000000005	-129000.000000005\\
110000.000000004	110000.000000004\\
439999.999999997	439999.999999997\\
-166000	-166000\\
-181999.999999999	-181999.999999999\\
-128000.000000001	-128000.000000001\\
126999.999999999	126999.999999999\\
128999.999999998	128999.999999998\\
-218999.999999997	-218999.999999997\\
15999.9999999973	15999.9999999973\\
76000.0000000005	76000.0000000005\\
52999.999999999	52999.999999999\\
146999.999999998	146999.999999998\\
-476000	-476000\\
312000	312000\\
126999.999999998	126999.999999998\\
-54999.9999999997	-54999.9999999997\\
-200000	-200000\\
-999.999999998557	-999.999999998557\\
383999.999999997	383999.999999997\\
-53999.9999999985	-53999.9999999985\\
-549000	-549000\\
236999.999999998	236999.999999998\\
36999.999999999	36999.999999999\\
129000.000000002	129000.000000002\\
89999.9999999972	89999.9999999972\\
-402000.000000001	-402000.000000001\\
311000.000000002	311000.000000002\\
73999.9999999998	73999.9999999998\\
-292999.999999998	-292999.999999998\\
217999.999999996	217999.999999996\\
-70999.9999999962	-70999.9999999962\\
-93000.0000000053	-93000.0000000053\\
348000.000000002	348000.000000002\\
-402000.000000003	-402000.000000003\\
165000.000000002	165000.000000002\\
-74999.9999999984	-74999.9999999984\\
-200000.000000002	-200000.000000002\\
365999.999999999	365999.999999999\\
-164999.999999999	-164999.999999999\\
35999.9999999996	35999.9999999996\\
331000.000000001	331000.000000001\\
-165999.999999999	-165999.999999999\\
-237000.000000002	-237000.000000002\\
-1000.00000000122	-1000.00000000122\\
-72000.0000000009	-72000.0000000009\\
364999.999999999	364999.999999999\\
-274999.999999999	-274999.999999999\\
-53000.0000000008	-53000.0000000008\\
-19999.9999999996	-19999.9999999996\\
310999.999999999	310999.999999999\\
-235999.999999995	-235999.999999995\\
327999.999999998	327999.999999998\\
-293000.000000002	-293000.000000002\\
219999.999999999	219999.999999999\\
-309999.999999996	-309999.999999996\\
161999.999999998	161999.999999998\\
-144000.000000002	-144000.000000002\\
36000.0000000005	36000.0000000005\\
-147000.000000001	-147000.000000001\\
347999.999999998	347999.999999998\\
-90999.9999999975	-90999.9999999975\\
-1000.000000003	-1000.000000003\\
-272999.999999997	-272999.999999997\\
436999.999999999	436999.999999999\\
21000.0000000008	21000.0000000008\\
-423000.000000003	-423000.000000003\\
220000	220000\\
-90999.9999999984	-90999.9999999984\\
238000.000000002	238000.000000002\\
-54999.9999999997	-54999.9999999997\\
54999.9999999988	54999.9999999988\\
-239000.000000002	-239000.000000002\\
-109000.000000002	-109000.000000002\\
74000.0000000034	74000.0000000034\\
125999.999999999	125999.999999999\\
20999.9999999999	20999.9999999999\\
-166999.999999999	-166999.999999999\\
73999.9999999972	73999.9999999972\\
183000.000000001	183000.000000001\\
-126999.999999999	-126999.999999999\\
-20000.0000000022	-20000.0000000022\\
238999.999999999	238999.999999999\\
-146999.999999997	-146999.999999997\\
-458000	-458000\\
276000	276000\\
327999.999999997	327999.999999997\\
-236999.999999996	-236999.999999996\\
72999.9999999986	72999.9999999986\\
73000.0000000004	73000.0000000004\\
-329000.000000001	-329000.000000001\\
91000.0000000011	91000.0000000011\\
219999.999999999	219999.999999999\\
90999.9999999975	90999.9999999975\\
-530000.000000001	-530000.000000001\\
548000.000000001	548000.000000001\\
-163999.999999999	-163999.999999999\\
-201000.000000001	-201000.000000001\\
219000.000000001	219000.000000001\\
0	0\\
-54000.000000002	-54000.000000002\\
-257999.999999999	-257999.999999999\\
93000	93000\\
476000.000000002	476000.000000002\\
-92000.0000000023	-92000.0000000023\\
-422000	-422000\\
111000.000000001	111000.000000001\\
-35999.9999999987	-35999.9999999987\\
107999.999999997	107999.999999997\\
203000.000000004	203000.000000004\\
-312000.000000002	-312000.000000002\\
8.88178419700125e-10	8.88178419700125e-10\\
147000.000000002	147000.000000002\\
182999.999999996	182999.999999996\\
-239000	-239000\\
-164000.000000001	-164000.000000001\\
-8.88178419700125e-10	-8.88178419700125e-10\\
347999.999999998	347999.999999998\\
-55999.9999999983	-55999.9999999983\\
38000.0000000029	38000.0000000029\\
-202000.000000003	-202000.000000003\\
-53999.9999999976	-53999.9999999976\\
254999.999999997	254999.999999997\\
-55000.0000000006	-55000.0000000006\\
-146000.000000002	-146000.000000002\\
1000.000000003	1000.000000003\\
-330999.999999998	-330999.999999998\\
365999.999999999	365999.999999999\\
148000.000000001	148000.000000001\\
53999.9999999994	53999.9999999994\\
-220000.000000002	-220000.000000002\\
-53999.9999999985	-53999.9999999985\\
329000	329000\\
-366000.000000001	-366000.000000001\\
90999.9999999993	90999.9999999993\\
127999.999999999	127999.999999999\\
-199999.999999999	-199999.999999999\\
-93000.0000000017	-93000.0000000017\\
239000	239000\\
292999.999999999	292999.999999999\\
-331000	-331000\\
-34999.9999999993	-34999.9999999993\\
-111000.000000002	-111000.000000002\\
294000.000000001	294000.000000001\\
-495000.000000003	-495000.000000003\\
202000.000000001	202000.000000001\\
566000.000000001	566000.000000001\\
-566000	-566000\\
92000.0000000014	92000.0000000014\\
180999.999999997	180999.999999997\\
-16999.9999999986	-16999.9999999986\\
2.66453525910038e-09	2.66453525910038e-09\\
-256000.000000004	-256000.000000004\\
108999.999999999	108999.999999999\\
-18000.0000000007	-18000.0000000007\\
37000.0000000017	37000.0000000017\\
71999.9999999983	71999.9999999983\\
-180999.999999997	-180999.999999997\\
125999.999999999	125999.999999999\\
20000.0000000022	20000.0000000022\\
219000.000000001	219000.000000001\\
-311000.000000002	-311000.000000002\\
-55999.9999999991	-55999.9999999991\\
331000	331000\\
-348999.999999998	-348999.999999998\\
221000.000000001	221000.000000001\\
145000	145000\\
-182000	-182000\\
-238000.000000002	-238000.000000002\\
-19000.000000001	-19000.000000001\\
403000	403000\\
-74000.0000000025	-74000.0000000025\\
-308999.999999997	-308999.999999997\\
290999.999999997	290999.999999997\\
19000.0000000019	19000.0000000019\\
-293999.999999999	-293999.999999999\\
166999.999999998	166999.999999998\\
126000	126000\\
-310000.000000001	-310000.000000001\\
146000.000000001	146000.000000001\\
273999.999999999	273999.999999999\\
-437999.999999997	-437999.999999997\\
90999.9999999993	90999.9999999993\\
384000	384000\\
-329000.000000001	-329000.000000001\\
311000	311000\\
-330000.000000002	-330000.000000002\\
239000.000000001	239000.000000001\\
-74000.0000000025	-74000.0000000025\\
-292999.999999999	-292999.999999999\\
293999.999999998	293999.999999998\\
-20000.0000000005	-20000.0000000005\\
-272999.999999999	-272999.999999999\\
474999.999999998	474999.999999998\\
-201000	-201000\\
-255999.999999998	-255999.999999998\\
-110999.999999997	-110999.999999997\\
348999.999999995	348999.999999995\\
73000.0000000013	73000.0000000013\\
-37000.0000000008	-37000.0000000008\\
-8.88178419700125e-10	-8.88178419700125e-10\\
-145999.999999998	-145999.999999998\\
146999.999999999	146999.999999999\\
15999.9999999982	15999.9999999982\\
-198999.999999998	-198999.999999998\\
183000	183000\\
-277000.000000001	-277000.000000001\\
424000.000000003	424000.000000003\\
-185000.000000004	-185000.000000004\\
-292000.000000001	-292000.000000001\\
201000	201000\\
329000	329000\\
-254999.999999996	-254999.999999996\\
-55000.0000000024	-55000.0000000024\\
-57000.0000000013	-57000.0000000013\\
222000.000000001	222000.000000001\\
-129000.000000001	-129000.000000001\\
-220000.000000002	-220000.000000002\\
239000.000000002	239000.000000002\\
-2000.00000000067	-2000.00000000067\\
94000.0000000003	94000.0000000003\\
-185000	-185000\\
75000.000000001	75000.000000001\\
235999.999999996	235999.999999996\\
-382999.999999999	-382999.999999999\\
-147000	-147000\\
274999.999999998	274999.999999998\\
237000.000000001	237000.000000001\\
-145000	-145000\\
-56000.0000000018	-56000.0000000018\\
0	0\\
-18000.0000000016	-18000.0000000016\\
-110000	-110000\\
237000.000000001	237000.000000001\\
19999.9999999987	19999.9999999987\\
-367000.000000002	-367000.000000002\\
219000.000000001	219000.000000001\\
37999.9999999994	37999.9999999994\\
-18999.9999999992	-18999.9999999992\\
72999.9999999968	72999.9999999968\\
-238000	-238000\\
147000.000000001	147000.000000001\\
36000.0000000005	36000.0000000005\\
312000.000000001	312000.000000001\\
-202000.000000002	-202000.000000002\\
-164999.999999998	-164999.999999998\\
37999.9999999985	37999.9999999985\\
-277000.000000001	-277000.000000001\\
294999.999999998	294999.999999998\\
-220000.000000001	-220000.000000001\\
365000.000000001	365000.000000001\\
-72000.0000000001	-72000.0000000001\\
-220000	-220000\\
108999.999999998	108999.999999998\\
20000.0000000031	20000.0000000031\\
16999.999999995	16999.999999995\\
-164999.999999998	-164999.999999998\\
294000	294000\\
54000.0000000011	54000.0000000011\\
-128000	-128000\\
-219000.000000001	-219000.000000001\\
91000.0000000011	91000.0000000011\\
-36000.0000000014	-36000.0000000014\\
292000	292000\\
-275000.000000002	-275000.000000002\\
56000.0000000036	56000.0000000036\\
220000.000000001	220000.000000001\\
-495000.000000004	-495000.000000004\\
165000.000000002	165000.000000002\\
401999.999999997	401999.999999997\\
-328999.999999998	-328999.999999998\\
-256000.000000002	-256000.000000002\\
622000.000000002	622000.000000002\\
-273999.999999998	-273999.999999998\\
-274999.999999999	-274999.999999999\\
403000	403000\\
53999.9999999994	53999.9999999994\\
-144999.999999999	-144999.999999999\\
-312000.000000003	-312000.000000003\\
35999.9999999987	35999.9999999987\\
56000	56000\\
346999.999999999	346999.999999999\\
-127999.999999997	-127999.999999997\\
56000	56000\\
-2000.00000000244	-2000.00000000244\\
-216999.999999996	-216999.999999996\\
124999.999999996	124999.999999996\\
-88999.9999999977	-88999.9999999977\\
-19000.0000000019	-19000.0000000019\\
-19000.0000000001	-19000.0000000001\\
73999.9999999989	73999.9999999989\\
255000.000000001	255000.000000001\\
-436999.999999998	-436999.999999998\\
52999.9999999973	52999.9999999973\\
238000	238000\\
-126999.999999997	-126999.999999997\\
-19000.0000000001	-19000.0000000001\\
35999.9999999996	35999.9999999996\\
276000.000000002	276000.000000002\\
-201999.999999998	-201999.999999998\\
-165000.000000001	-165000.000000001\\
183000.000000002	183000.000000002\\
18999.9999999992	18999.9999999992\\
-238000.000000002	-238000.000000002\\
-92999.9999999991	-92999.9999999991\\
311999.999999999	311999.999999999\\
276000.000000002	276000.000000002\\
-240000.000000006	-240000.000000006\\
-164000	-164000\\
92000.0000000014	92000.0000000014\\
-183000.000000001	-183000.000000001\\
-165999.999999999	-165999.999999999\\
366999.999999996	366999.999999996\\
-90999.9999999984	-90999.9999999984\\
181999.999999999	181999.999999999\\
-8.88178419700125e-10	-8.88178419700125e-10\\
-292000	-292000\\
201000	201000\\
-92000.0000000032	-92000.0000000032\\
-16999.9999999986	-16999.9999999986\\
-57000.0000000004	-57000.0000000004\\
-16000.0000000009	-16000.0000000009\\
273000	273000\\
-17999.9999999998	-17999.9999999998\\
-220000.000000001	-220000.000000001\\
37000.0000000044	37000.0000000044\\
110999.999999999	110999.999999999\\
-57000.0000000004	-57000.0000000004\\
57000.000000003	57000.000000003\\
125999.999999998	125999.999999998\\
-144999.999999998	-144999.999999998\\
-128000	-128000\\
-184000.000000004	-184000.000000004\\
329999.999999999	329999.999999999\\
182000.000000001	182000.000000001\\
-143999.999999999	-143999.999999999\\
70999.9999999971	70999.9999999971\\
-274000	-274000\\
-91000.0000000019	-91000.0000000019\\
348000.000000002	348000.000000002\\
-495999.999999999	-495999.999999999\\
294999.999999999	294999.999999999\\
236999.999999999	236999.999999999\\
-128999.999999999	-128999.999999999\\
74999.9999999993	74999.9999999993\\
-166000.000000001	-166000.000000001\\
-145999.999999998	-145999.999999998\\
201999.999999998	201999.999999998\\
-37999.9999999967	-37999.9999999967\\
164999.999999999	164999.999999999\\
-164000.000000001	-164000.000000001\\
-218999.999999998	-218999.999999998\\
-74999.9999999993	-74999.9999999993\\
512999.999999997	512999.999999997\\
-35999.9999999969	-35999.9999999969\\
36999.9999999972	36999.9999999972\\
-385999.999999999	-385999.999999999\\
-16999.9999999995	-16999.9999999995\\
458000.000000001	458000.000000001\\
-239999.999999998	-239999.999999998\\
129999.999999998	129999.999999998\\
-237999.999999996	-237999.999999996\\
16999.999999995	16999.999999995\\
258000.000000004	258000.000000004\\
-221000.000000004	-221000.000000004\\
182999.999999998	182999.999999998\\
-237000	-237000\\
-312000.000000002	-312000.000000002\\
714000.000000001	714000.000000001\\
-347000.000000001	-347000.000000001\\
-75000.0000000002	-75000.0000000002\\
222000.000000004	222000.000000004\\
-148000.000000005	-148000.000000005\\
8.88178419700125e-10	8.88178419700125e-10\\
38000.0000000029	38000.0000000029\\
-2000.00000000333	-2000.00000000333\\
57000.0000000013	57000.0000000013\\
52999.9999999982	52999.9999999982\\
-292000.000000002	-292000.000000002\\
476000.000000003	476000.000000003\\
-128000.000000002	-128000.000000002\\
-512999.999999995	-512999.999999995\\
108999.999999996	108999.999999996\\
570000.000000001	570000.000000001\\
-129999.999999999	-129999.999999999\\
-567000.000000001	-567000.000000001\\
109000.000000003	109000.000000003\\
349000	349000\\
-36999.999999999	-36999.999999999\\
73000.0000000013	73000.0000000013\\
-348000	-348000\\
384999.999999995	384999.999999995\\
36000.0000000022	36000.0000000022\\
-437999.999999999	-437999.999999999\\
492999.999999996	492999.999999996\\
-237999.999999996	-237999.999999996\\
-54999.9999999997	-54999.9999999997\\
110999.999999999	110999.999999999\\
-366999.999999998	-366999.999999998\\
365999.999999998	365999.999999998\\
-201000	-201000\\
-36999.999999999	-36999.999999999\\
384000	384000\\
-53000.0000000026	-53000.0000000026\\
-258000	-258000\\
146999.999999999	146999.999999999\\
-182999.999999998	-182999.999999998\\
274000.000000001	274000.000000001\\
-146000.000000001	-146000.000000001\\
-36000.0000000005	-36000.0000000005\\
109000.000000004	109000.000000004\\
-238000.000000001	-238000.000000001\\
238000.000000004	238000.000000004\\
-238000.000000003	-238000.000000003\\
258000	258000\\
34000.0000000034	34000.0000000034\\
-182000.000000001	-182000.000000001\\
128999.999999999	128999.999999999\\
-165999.999999996	-165999.999999996\\
-220000.000000001	-220000.000000001\\
330999.999999997	330999.999999997\\
348000.000000002	348000.000000002\\
-386000.000000002	-386000.000000002\\
109999.999999999	109999.999999999\\
-382999.999999999	-382999.999999999\\
363999.999999999	363999.999999999\\
185000.000000001	185000.000000001\\
-475999.999999999	-475999.999999999\\
364999.999999997	364999.999999997\\
-310999.999999998	-310999.999999998\\
347999.999999999	347999.999999999\\
-238000	-238000\\
-109000	-109000\\
163000.000000002	163000.000000002\\
165999.999999999	165999.999999999\\
-219000	-219000\\
-1999.999999998	-1999.999999998\\
-90000.0000000025	-90000.0000000025\\
310999.999999999	310999.999999999\\
-330999.999999998	-330999.999999998\\
112000.000000004	112000.000000004\\
310999.999999998	310999.999999998\\
-111000.000000002	-111000.000000002\\
-128000.000000001	-128000.000000001\\
-293000	-293000\\
202000	202000\\
312000.000000001	312000.000000001\\
-332000.000000002	-332000.000000002\\
259000.000000002	259000.000000002\\
15999.9999999965	15999.9999999965\\
-400999.999999998	-400999.999999998\\
-166000.000000002	-166000.000000002\\
569000.000000002	569000.000000002\\
-20000.0000000013	-20000.0000000013\\
-402000.000000001	-402000.000000001\\
367000	367000\\
-56999.9999999995	-56999.9999999995\\
20999.999999999	20999.999999999\\
-241000	-241000\\
21000.0000000008	21000.0000000008\\
308999.999999999	308999.999999999\\
-199000	-199000\\
33999.999999998	33999.999999998\\
57000.0000000022	57000.0000000022\\
-184000.000000003	-184000.000000003\\
202000	202000\\
-166000.000000001	-166000.000000001\\
-144999.999999999	-144999.999999999\\
290999.999999999	290999.999999999\\
21000.0000000035	21000.0000000035\\
-39000.0000000032	-39000.0000000032\\
56000.0000000018	56000.0000000018\\
-55000.0000000006	-55000.0000000006\\
-218999.999999999	-218999.999999999\\
52999.9999999973	52999.9999999973\\
93000	93000\\
18000.0000000025	18000.0000000025\\
-17999.999999998	-17999.999999998\\
182999.999999996	182999.999999996\\
35999.9999999987	35999.9999999987\\
-585999.999999998	-585999.999999998\\
294000.000000003	294000.000000003\\
402999.999999998	402999.999999998\\
-331000.000000002	-331000.000000002\\
-109999.999999999	-109999.999999999\\
277000	277000\\
-94000.0000000021	-94000.0000000021\\
-219000	-219000\\
220000.000000001	220000.000000001\\
-72999.9999999977	-72999.9999999977\\
-91000.0000000011	-91000.0000000011\\
290999.999999998	290999.999999998\\
-273000.000000001	-273000.000000001\\
108999.999999999	108999.999999999\\
-17999.9999999998	-17999.9999999998\\
90999.9999999993	90999.9999999993\\
-218999.999999999	-218999.999999999\\
257000.000000003	257000.000000003\\
-184000.000000004	-184000.000000004\\
-19000.000000001	-19000.000000001\\
93000.0000000008	93000.0000000008\\
236999.999999998	236999.999999998\\
-182999.999999999	-182999.999999999\\
-201000	-201000\\
-91999.999999997	-91999.999999997\\
-164000.000000001	-164000.000000001\\
493999.999999998	493999.999999998\\
164000.000000001	164000.000000001\\
-107999.999999996	-107999.999999996\\
-295000.000000004	-295000.000000004\\
129000	129000\\
73000.0000000004	73000.0000000004\\
36999.999999999	36999.999999999\\
-330000	-330000\\
238999.999999999	238999.999999999\\
-999.999999999446	-999.999999999446\\
-73000.0000000013	-73000.0000000013\\
146999.999999999	146999.999999999\\
-93000.0000000008	-93000.0000000008\\
130000	130000\\
-312000.000000001	-312000.000000001\\
-91999.9999999978	-91999.9999999978\\
603999.999999997	603999.999999997\\
-400999.999999996	-400999.999999996\\
107999.999999997	107999.999999997\\
-128000	-128000\\
129000.000000003	129000.000000003\\
-110000.000000001	-110000.000000001\\
256000	256000\\
-219999.999999999	-219999.999999999\\
-34999.9999999984	-34999.9999999984\\
363999.999999995	363999.999999995\\
-639999.999999998	-639999.999999998\\
530999.999999997	530999.999999997\\
-273999.999999997	-273999.999999997\\
53999.9999999985	53999.9999999985\\
-8.88178419700125e-10	-8.88178419700125e-10\\
166000.000000002	166000.000000002\\
-221000.000000002	-221000.000000002\\
0	0\\
294000.000000003	294000.000000003\\
-202000.000000003	-202000.000000003\\
36999.9999999999	36999.9999999999\\
-129000.000000001	-129000.000000001\\
128000	128000\\
-17000.0000000003	-17000.0000000003\\
8.88178419700125e-10	8.88178419700125e-10\\
-93000.0000000017	-93000.0000000017\\
92000.0000000005	92000.0000000005\\
-274000.000000003	-274000.000000003\\
274000.000000003	274000.000000003\\
202000	202000\\
-201999.999999999	-201999.999999999\\
-16999.9999999995	-16999.9999999995\\
198999.999999998	198999.999999998\\
-289999.999999996	-289999.999999996\\
51999.9999999952	51999.9999999952\\
56999.9999999995	56999.9999999995\\
-129000	-129000\\
-55000.0000000006	-55000.0000000006\\
184000	184000\\
34999.9999999993	34999.9999999993\\
74999.9999999993	74999.9999999993\\
-1000.00000000033	-1000.00000000033\\
-458000.000000002	-458000.000000002\\
330000.000000001	330000.000000001\\
73000.0000000022	73000.0000000022\\
74000.0000000007	74000.0000000007\\
-420999.999999999	-420999.999999999\\
511000	511000\\
37999.9999999976	37999.9999999976\\
-347999.999999997	-347999.999999997\\
-1000.00000000477	-1000.00000000477\\
-199999.999999999	-199999.999999999\\
237000	237000\\
19000.000000001	19000.000000001\\
237999.999999999	237999.999999999\\
-129000.000000001	-129000.000000001\\
18999.9999999983	18999.9999999983\\
-201999.999999999	-201999.999999999\\
-89999.999999999	-89999.999999999\\
198999.999999997	198999.999999997\\
149000.000000003	149000.000000003\\
-295000.000000003	-295000.000000003\\
130000.000000003	130000.000000003\\
199000	199000\\
-582999.999999999	-582999.999999999\\
418000	418000\\
-34000.0000000034	-34000.0000000034\\
127000	127000\\
-164000.000000001	-164000.000000001\\
52999.9999999999	52999.9999999999\\
-88999.9999999977	-88999.9999999977\\
235999.999999995	235999.999999995\\
-309999.999999997	-309999.999999997\\
71999.9999999974	71999.9999999974\\
276000.000000002	276000.000000002\\
-238999.999999999	-238999.999999999\\
56000.0000000009	56000.0000000009\\
-74000.0000000052	-74000.0000000052\\
-91999.9999999978	-91999.9999999978\\
367000	367000\\
-165000.000000004	-165000.000000004\\
-183999.999999997	-183999.999999997\\
185000.000000001	185000.000000001\\
34999.9999999957	34999.9999999957\\
-291999.999999997	-291999.999999997\\
53999.9999999976	53999.9999999976\\
330999.999999999	330999.999999999\\
-166999.999999999	-166999.999999999\\
-271999.999999998	-271999.999999998\\
601999.999999999	601999.999999999\\
-274000	-274000\\
-108999.999999998	-108999.999999998\\
-92000.0000000023	-92000.0000000023\\
-54999.9999999988	-54999.9999999988\\
492999.999999998	492999.999999998\\
-289999.999999999	-289999.999999999\\
-278000	-278000\\
515999.999999999	515999.999999999\\
-131000.000000001	-131000.000000001\\
-346000.000000001	-346000.000000001\\
236000	236000\\
76000.0000000014	76000.0000000014\\
-222000.000000003	-222000.000000003\\
166000	166000\\
-36999.999999999	-36999.999999999\\
-91000.0000000019	-91000.0000000019\\
-37999.9999999958	-37999.9999999958\\
201999.999999996	201999.999999996\\
-146000	-146000\\
202000	202000\\
-185000	-185000\\
38000.0000000002	38000.0000000002\\
128000.000000003	128000.000000003\\
19000.0000000001	19000.0000000001\\
-240000.000000005	-240000.000000005\\
111000.000000002	111000.000000002\\
-35999.9999999996	-35999.9999999996\\
-74000.0000000025	-74000.0000000025\\
274000.000000001	274000.000000001\\
-181000.000000002	-181000.000000002\\
16000.0000000009	16000.0000000009\\
19999.9999999978	19999.9999999978\\
-203000	-203000\\
331000.000000002	331000.000000002\\
-129000	-129000\\
1000.00000000122	1000.00000000122\\
145000	145000\\
-145000.000000002	-145000.000000002\\
-73999.9999999998	-73999.9999999998\\
-238000.000000001	-238000.000000001\\
366000	366000\\
-73000.0000000004	-73000.0000000004\\
90999.9999999993	90999.9999999993\\
-199999.999999997	-199999.999999997\\
-110999.999999999	-110999.999999999\\
311999.999999998	311999.999999998\\
145000.000000003	145000.000000003\\
-254000.000000001	-254000.000000001\\
15999.9999999982	15999.9999999982\\
130000	130000\\
-258000	-258000\\
92999.9999999964	92999.9999999964\\
90000.0000000025	90000.0000000025\\
-108000.000000001	-108000.000000001\\
-38000.0000000011	-38000.0000000011\\
1000.00000000122	1000.00000000122\\
254999.999999998	254999.999999998\\
-89999.9999999998	-89999.9999999998\\
-166000.000000001	-166000.000000001\\
257000.000000001	257000.000000001\\
-239000.000000002	-239000.000000002\\
148000.000000003	148000.000000003\\
-147000.000000003	-147000.000000003\\
-56000	-56000\\
129999.999999999	129999.999999999\\
400999.999999999	400999.999999999\\
-585000.000000001	-585000.000000001\\
72999.9999999995	72999.9999999995\\
19000.000000001	19000.000000001\\
108999.999999997	108999.999999997\\
202000.000000002	202000.000000002\\
-403000.000000001	-403000.000000001\\
-19000.0000000019	-19000.0000000019\\
56000	56000\\
365000.000000004	365000.000000004\\
-35000.000000001	-35000.000000001\\
-130000.000000001	-130000.000000001\\
-180999.999999996	-180999.999999996\\
52999.9999999973	52999.9999999973\\
275000.000000002	275000.000000002\\
-254999.999999999	-254999.999999999\\
-20000.0000000005	-20000.0000000005\\
258000.000000001	258000.000000001\\
-203000.000000002	-203000.000000002\\
-182000	-182000\\
420000.000000002	420000.000000002\\
-181999.999999999	-181999.999999999\\
-256000	-256000\\
327999.999999997	327999.999999997\\
-164000.000000001	-164000.000000001\\
37000.0000000017	37000.0000000017\\
-36999.9999999999	-36999.9999999999\\
-17999.9999999971	-17999.9999999971\\
219999.999999995	219999.999999995\\
-37999.9999999994	-37999.9999999994\\
-256000	-256000\\
112000.000000003	112000.000000003\\
14999.9999999961	14999.9999999961\\
113000	113000\\
-20999.999999999	-20999.999999999\\
-327999.999999999	-327999.999999999\\
403000.000000001	403000.000000001\\
17999.9999999989	17999.9999999989\\
-441000.000000002	-441000.000000002\\
369000.000000003	369000.000000003\\
52999.9999999955	52999.9999999955\\
-72999.9999999977	-72999.9999999977\\
-73000.0000000031	-73000.0000000031\\
8.88178419700125e-10	8.88178419700125e-10\\
-366000	-366000\\
475999.999999999	475999.999999999\\
72000.0000000009	72000.0000000009\\
-161999.999999997	-161999.999999997\\
-113000.000000001	-113000.000000001\\
-144000	-144000\\
309999.999999998	309999.999999998\\
-55000.0000000006	-55000.0000000006\\
-54999.9999999979	-54999.9999999979\\
165999.999999999	165999.999999999\\
-147999.999999997	-147999.999999997\\
202999.999999999	202999.999999999\\
71999.9999999974	71999.9999999974\\
-640999.999999999	-640999.999999999\\
384999.999999997	384999.999999997\\
-53999.9999999985	-53999.9999999985\\
-167000	-167000\\
386999.999999999	386999.999999999\\
-386999.999999999	-386999.999999999\\
588000.000000001	588000.000000001\\
-311999.999999998	-311999.999999998\\
-311000.000000003	-311000.000000003\\
456999.999999998	456999.999999998\\
-54999.9999999997	-54999.9999999997\\
-163999.999999999	-163999.999999999\\
-201000	-201000\\
125999.999999998	125999.999999998\\
21000.0000000043	21000.0000000043\\
345999.999999998	345999.999999998\\
-219000	-219000\\
-145999.999999997	-145999.999999997\\
16999.9999999959	16999.9999999959\\
-16999.999999995	-16999.999999995\\
-55000.0000000041	-55000.0000000041\\
383000	383000\\
-181999.999999999	-181999.999999999\\
110000.000000002	110000.000000002\\
-91000.0000000011	-91000.0000000011\\
-313000.000000001	-313000.000000001\\
513999.999999997	513999.999999997\\
-585999.999999998	-585999.999999998\\
309999.999999998	309999.999999998\\
20000.0000000022	20000.0000000022\\
-238000	-238000\\
180999.999999997	180999.999999997\\
167000.000000005	167000.000000005\\
-183999.999999999	-183999.999999999\\
201000	201000\\
184000.000000002	184000.000000002\\
-476000.000000002	-476000.000000002\\
35999.9999999978	35999.9999999978\\
17999.999999998	17999.999999998\\
37000.0000000017	37000.0000000017\\
-18000.0000000025	-18000.0000000025\\
329000	329000\\
-163999.999999999	-163999.999999999\\
-257000.000000002	-257000.000000002\\
54999.9999999997	54999.9999999997\\
329999.999999997	329999.999999997\\
-165999.999999998	-165999.999999998\\
-70999.9999999988	-70999.9999999988\\
52999.9999999964	52999.9999999964\\
-91000.0000000002	-91000.0000000002\\
54999.9999999988	54999.9999999988\\
-219000	-219000\\
108000	108000\\
222000.000000002	222000.000000002\\
52999.9999999955	52999.9999999955\\
-181999.999999999	-181999.999999999\\
238000.000000001	238000.000000001\\
-165999.999999999	-165999.999999999\\
-328000	-328000\\
-1000.00000000211	-1000.00000000211\\
276000.000000002	276000.000000002\\
34999.9999999993	34999.9999999993\\
-8.88178419700125e-10	-8.88178419700125e-10\\
999.999999998557	999.999999998557\\
-127999.999999998	-127999.999999998\\
-1000.00000000211	-1000.00000000211\\
275000.000000002	275000.000000002\\
-127999.999999997	-127999.999999997\\
0	0\\
220000	220000\\
-404000.000000003	-404000.000000003\\
1999.99999999978	1999.99999999978\\
16000.0000000009	16000.0000000009\\
221000	221000\\
19000.000000001	19000.000000001\\
-130000.000000003	-130000.000000003\\
2000.00000000156	2000.00000000156\\
-276000.000000002	-276000.000000002\\
348000.000000001	348000.000000001\\
200999.999999999	200999.999999999\\
-290999.999999997	-290999.999999997\\
-222000.000000003	-222000.000000003\\
93000	93000\\
383999.999999998	383999.999999998\\
-384999.999999999	-384999.999999999\\
275999.999999999	275999.999999999\\
-276000.000000001	-276000.000000001\\
1000.00000000122	1000.00000000122\\
419999.999999997	419999.999999997\\
-401999.999999997	-401999.999999997\\
-17999.999999998	-17999.999999998\\
273999.999999997	273999.999999997\\
73000.0000000013	73000.0000000013\\
-17999.9999999971	-17999.9999999971\\
-274000.000000001	-274000.000000001\\
201000	201000\\
-422000.000000003	-422000.000000003\\
422000.000000004	422000.000000004\\
-347000.000000001	-347000.000000001\\
326999.999999998	326999.999999998\\
94000.0000000047	94000.0000000047\\
-349000.000000002	-349000.000000002\\
403000	403000\\
-475000	-475000\\
326999.999999998	326999.999999998\\
76000.0000000005	76000.0000000005\\
-349999.999999998	-349999.999999998\\
238999.999999998	238999.999999998\\
-91000.0000000002	-91000.0000000002\\
54000.0000000038	54000.0000000038\\
55000.0000000015	55000.0000000015\\
-238000	-238000\\
110999.999999998	110999.999999998\\
126000	126000\\
-53000.0000000026	-53000.0000000026\\
-238000	-238000\\
255000	255000\\
165000	165000\\
-183000.000000002	-183000.000000002\\
-236999.999999997	-236999.999999997\\
237999.999999998	237999.999999998\\
163000	163000\\
-182000.000000002	-182000.000000002\\
257000.000000002	257000.000000002\\
-239000.000000003	-239000.000000003\\
-438999.999999998	-438999.999999998\\
328999.999999999	328999.999999999\\
220000.000000001	220000.000000001\\
-218999.999999999	-218999.999999999\\
219000	219000\\
37000.0000000026	37000.0000000026\\
-92000.0000000023	-92000.0000000023\\
-181999.999999999	-181999.999999999\\
-2000.00000000244	-2000.00000000244\\
203000.000000001	203000.000000001\\
53999.9999999967	53999.9999999967\\
-309999.999999998	-309999.999999998\\
199999.999999999	199999.999999999\\
-36000.0000000022	-36000.0000000022\\
92000.0000000023	92000.0000000023\\
-403000	-403000\\
274000.000000002	274000.000000002\\
145999.999999999	145999.999999999\\
38999.9999999988	38999.9999999988\\
-405999.999999998	-405999.999999998\\
185999.999999999	185999.999999999\\
438000	438000\\
-531000.000000002	-531000.000000002\\
109999.999999999	109999.999999999\\
275000.000000001	275000.000000001\\
-110000.000000002	-110000.000000002\\
-294000	-294000\\
39000.0000000024	39000.0000000024\\
436999.999999998	436999.999999998\\
-419999.999999998	-419999.999999998\\
128000.000000002	128000.000000002\\
146000.000000003	146000.000000003\\
-274000.000000002	-274000.000000002\\
-36000.0000000005	-36000.0000000005\\
108000.000000001	108000.000000001\\
220999.999999995	220999.999999995\\
-255999.999999999	-255999.999999999\\
54000.0000000003	54000.0000000003\\
238999.999999998	238999.999999998\\
-73999.9999999989	-73999.9999999989\\
-201999.999999997	-201999.999999997\\
-109000	-109000\\
184000	184000\\
-57000.0000000004	-57000.0000000004\\
-145000	-145000\\
329999.999999997	329999.999999997\\
89000.0000000013	89000.0000000013\\
-271999.999999999	-271999.999999999\\
-54999.9999999997	-54999.9999999997\\
34999.9999999984	34999.9999999984\\
128000	128000\\
39000.0000000032	39000.0000000032\\
33999.9999999971	33999.9999999971\\
-199000.000000001	-199000.000000001\\
34000.0000000016	34000.0000000016\\
38999.9999999979	38999.9999999979\\
-55999.9999999983	-55999.9999999983\\
-19000.0000000019	-19000.0000000019\\
221000.000000004	221000.000000004\\
-91999.9999999996	-91999.9999999996\\
-54999.9999999988	-54999.9999999988\\
17999.999999998	17999.999999998\\
-90999.9999999966	-90999.9999999966\\
54999.9999999997	54999.9999999997\\
182999.999999999	182999.999999999\\
-258000.000000002	-258000.000000002\\
-51999.999999996	-51999.999999996\\
162999.999999995	162999.999999995\\
110000	110000\\
-183000.000000001	-183000.000000001\\
72999.9999999995	72999.9999999995\\
54999.9999999988	54999.9999999988\\
20000.000000004	20000.000000004\\
-351000.000000003	-351000.000000003\\
94000.0000000021	94000.0000000021\\
438999.999999998	438999.999999998\\
-92000.0000000005	-92000.0000000005\\
-367000.000000001	-367000.000000001\\
368000.000000003	368000.000000003\\
-239000.000000003	-239000.000000003\\
-201999.999999998	-201999.999999998\\
459000	459000\\
-239000.000000003	-239000.000000003\\
-71999.9999999965	-71999.9999999965\\
255000.000000001	255000.000000001\\
-366000	-366000\\
330000.000000003	330000.000000003\\
53999.9999999985	53999.9999999985\\
-419000	-419000\\
290999.999999999	290999.999999999\\
-73000.0000000013	-73000.0000000013\\
111000	111000\\
-74000.0000000016	-74000.0000000016\\
-109999.999999999	-109999.999999999\\
218999.999999999	218999.999999999\\
-108000.000000002	-108000.000000002\\
-330999.999999998	-330999.999999998\\
770000	770000\\
-441000.000000003	-441000.000000003\\
-144999.999999996	-144999.999999996\\
328999.999999999	328999.999999999\\
-71999.9999999992	-71999.9999999992\\
-478999.999999999	-478999.999999999\\
496999.999999996	496999.999999996\\
73000.0000000013	73000.0000000013\\
-531999.999999997	-531999.999999997\\
421000	421000\\
-109000.000000004	-109000.000000004\\
183000.000000002	183000.000000002\\
-74000.0000000034	-74000.0000000034\\
-163999.999999999	-163999.999999999\\
-35999.9999999996	-35999.9999999996\\
-92999.9999999973	-92999.9999999973\\
330000.000000001	330000.000000001\\
-110000	-110000\\
-146000.000000003	-146000.000000003\\
74000.0000000007	74000.0000000007\\
308999.999999999	308999.999999999\\
-381999.999999997	-381999.999999997\\
70999.9999999962	70999.9999999962\\
222000.000000003	222000.000000003\\
-167000.000000003	-167000.000000003\\
-309999.999999999	-309999.999999999\\
622000.000000002	622000.000000002\\
-383999.999999998	-383999.999999998\\
164999.999999999	164999.999999999\\
-147000.000000001	-147000.000000001\\
-19000.000000001	-19000.000000001\\
367000.000000001	367000.000000001\\
-200999.999999998	-200999.999999998\\
-220000.000000002	-220000.000000002\\
-128000.000000001	-128000.000000001\\
236999.999999998	236999.999999998\\
110000.000000002	110000.000000002\\
20000.0000000013	20000.0000000013\\
71999.9999999965	71999.9999999965\\
-422000	-422000\\
-71000.0000000006	-71000.0000000006\\
327000	327000\\
204000	204000\\
16000	16000\\
-273000.000000001	-273000.000000001\\
-202000.000000001	-202000.000000001\\
91000.0000000002	91000.0000000002\\
220000.000000001	220000.000000001\\
38000.000000002	38000.000000002\\
70999.9999999971	70999.9999999971\\
-437999.999999998	-437999.999999998\\
257000.000000001	257000.000000001\\
125999.999999998	125999.999999998\\
-327999.999999998	-327999.999999998\\
383999.999999999	383999.999999999\\
-109999.999999999	-109999.999999999\\
-219000.000000001	-219000.000000001\\
255000.000000001	255000.000000001\\
148000.000000003	148000.000000003\\
-403000.000000002	-403000.000000002\\
-74000.0000000034	-74000.0000000034\\
219000.000000002	219000.000000002\\
57000.0000000013	57000.0000000013\\
71999.9999999983	71999.9999999983\\
-311000.000000003	-311000.000000003\\
292000.000000002	292000.000000002\\
-17000.0000000003	-17000.0000000003\\
-273999.999999996	-273999.999999996\\
253999.999999994	253999.999999994\\
-291999.999999997	-291999.999999997\\
367000.000000003	367000.000000003\\
17999.9999999998	17999.9999999998\\
-127999.999999999	-127999.999999999\\
71999.9999999965	71999.9999999965\\
-401000.000000001	-401000.000000001\\
364000	364000\\
-16000.0000000009	-16000.0000000009\\
-93000	-93000\\
-146999.999999999	-146999.999999999\\
149000.000000001	149000.000000001\\
143999.999999997	143999.999999997\\
-72999.9999999968	-72999.9999999968\\
148000.000000001	148000.000000001\\
-184000.000000002	-184000.000000002\\
-37000.0000000017	-37000.0000000017\\
-54000.0000000003	-54000.0000000003\\
-129000	-129000\\
348000	348000\\
-92000.0000000005	-92000.0000000005\\
-199999.999999997	-199999.999999997\\
182999.999999998	182999.999999998\\
35000.0000000001	35000.0000000001\\
18999.9999999983	18999.9999999983\\
-145999.999999998	-145999.999999998\\
348000.000000002	348000.000000002\\
-275000.000000002	-275000.000000002\\
-55000.0000000006	-55000.0000000006\\
-19000.0000000019	-19000.0000000019\\
-52999.9999999982	-52999.9999999982\\
235999.999999999	235999.999999999\\
37999.9999999985	37999.9999999985\\
-221000.000000001	-221000.000000001\\
-199999.999999997	-199999.999999997\\
310999.999999998	310999.999999998\\
126999.999999999	126999.999999999\\
-382999.999999999	-382999.999999999\\
272999.999999999	272999.999999999\\
203000.000000003	203000.000000003\\
-165000.000000002	-165000.000000002\\
-93000.0000000026	-93000.0000000026\\
-126999.999999999	-126999.999999999\\
-73000.0000000022	-73000.0000000022\\
311000.000000001	311000.000000001\\
-147000.000000003	-147000.000000003\\
37000.0000000017	37000.0000000017\\
-1.77635683940025e-09	-1.77635683940025e-09\\
17999.9999999998	17999.9999999998\\
-17000.0000000003	-17000.0000000003\\
162999.999999999	162999.999999999\\
-274000.000000001	-274000.000000001\\
165000.000000002	165000.000000002\\
-18000.0000000016	-18000.0000000016\\
-292999.999999999	-292999.999999999\\
218999.999999998	218999.999999998\\
128000	128000\\
184000	184000\\
-713999.999999999	-713999.999999999\\
603000.000000001	603000.000000001\\
111000.000000001	111000.000000001\\
-164999.999999999	-164999.999999999\\
-476000.000000002	-476000.000000002\\
457000.000000002	457000.000000002\\
-71999.9999999992	-71999.9999999992\\
-111000.000000004	-111000.000000004\\
239000.000000003	239000.000000003\\
-294000.000000005	-294000.000000005\\
-109999.999999996	-109999.999999996\\
459999.999999999	459999.999999999\\
51999.9999999987	51999.9999999987\\
-364000	-364000\\
16999.9999999986	16999.9999999986\\
146999.999999998	146999.999999998\\
54999.9999999988	54999.9999999988\\
-183999.999999998	-183999.999999998\\
202999.999999998	202999.999999998\\
-349000.000000001	-349000.000000001\\
220000.000000002	220000.000000002\\
-165000	-165000\\
257000.000000001	257000.000000001\\
-183000	-183000\\
383000	383000\\
-237000.000000001	-237000.000000001\\
-73000.0000000004	-73000.0000000004\\
-73999.9999999981	-73999.9999999981\\
202000.000000002	202000.000000002\\
-54999.9999999997	-54999.9999999997\\
-146000	-146000\\
-19000.0000000028	-19000.0000000028\\
273999.999999998	273999.999999998\\
-127000	-127000\\
-275000	-275000\\
128999.999999999	128999.999999999\\
382000.000000001	382000.000000001\\
-252999.999999997	-252999.999999997\\
-76000.0000000005	-76000.0000000005\\
-310000	-310000\\
420999.999999997	420999.999999997\\
421000.000000003	421000.000000003\\
-678000.000000002	-678000.000000002\\
148000.000000002	148000.000000002\\
71999.9999999974	71999.9999999974\\
-91000.0000000011	-91000.0000000011\\
164000.000000001	164000.000000001\\
1000.00000000033	1000.00000000033\\
-55000.0000000006	-55000.0000000006\\
-37999.9999999985	-37999.9999999985\\
-182000.000000003	-182000.000000003\\
-36000.0000000005	-36000.0000000005\\
329000.000000001	329000.000000001\\
126999.999999998	126999.999999998\\
-162999.999999999	-162999.999999999\\
-38000.000000002	-38000.000000002\\
-254999.999999998	-254999.999999998\\
16999.9999999977	16999.9999999977\\
258000	258000\\
34000.0000000007	34000.0000000007\\
94000.0000000003	94000.0000000003\\
-239000.000000003	-239000.000000003\\
-17999.9999999989	-17999.9999999989\\
-37000.0000000017	-37000.0000000017\\
17999.9999999998	17999.9999999998\\
147000	147000\\
183000.000000002	183000.000000002\\
-90999.9999999984	-90999.9999999984\\
-312000.000000005	-312000.000000005\\
110000	110000\\
54999.9999999988	54999.9999999988\\
146000	146000\\
-327999.999999998	-327999.999999998\\
180999.999999996	180999.999999996\\
38000.0000000038	38000.0000000038\\
145999.999999999	145999.999999999\\
-164000.000000001	-164000.000000001\\
-293999.999999997	-293999.999999997\\
219000	219000\\
-53000.0000000035	-53000.0000000035\\
219000.000000002	219000.000000002\\
-91999.9999999996	-91999.9999999996\\
91999.9999999987	91999.9999999987\\
-164999.999999999	-164999.999999999\\
-53999.9999999985	-53999.9999999985\\
34999.9999999966	34999.9999999966\\
293000.000000004	293000.000000004\\
1999.999999998	1999.999999998\\
-405000.000000001	-405000.000000001\\
93000.0000000008	93000.0000000008\\
17000.0000000021	17000.0000000021\\
367999.999999998	367999.999999998\\
-56999.9999999986	-56999.9999999986\\
-328000.000000002	-328000.000000002\\
-110000.000000001	-110000.000000001\\
219000.000000001	219000.000000001\\
91000.0000000002	91000.0000000002\\
-273000	-273000\\
-1000.00000000033	-1000.00000000033\\
111000.000000002	111000.000000002\\
163000	163000\\
18999.9999999992	18999.9999999992\\
-164000.000000001	-164000.000000001\\
-74999.9999999975	-74999.9999999975\\
111999.999999997	111999.999999997\\
-999.999999998557	-999.999999998557\\
72999.9999999977	72999.9999999977\\
36999.9999999999	36999.9999999999\\
-73999.9999999981	-73999.9999999981\\
-109000.000000004	-109000.000000004\\
-108999.999999996	-108999.999999996\\
199000.000000002	199000.000000002\\
129999.999999996	129999.999999996\\
-36999.9999999981	-36999.9999999981\\
-402999.999999997	-402999.999999997\\
494000.000000001	494000.000000001\\
-293000.000000002	-293000.000000002\\
19999.9999999987	19999.9999999987\\
15999.9999999991	15999.9999999991\\
94000.0000000021	94000.0000000021\\
198999.999999998	198999.999999998\\
-493000	-493000\\
310000	310000\\
94000.0000000012	94000.0000000012\\
-149000.000000001	-149000.000000001\\
-219000.000000001	-219000.000000001\\
184000.000000003	184000.000000003\\
182999.999999998	182999.999999998\\
-184999.999999999	-184999.999999999\\
167000.000000002	167000.000000002\\
-110000.000000002	-110000.000000002\\
-75000.0000000019	-75000.0000000019\\
-53999.9999999985	-53999.9999999985\\
238999.999999999	238999.999999999\\
-258000	-258000\\
185000.000000002	185000.000000002\\
-2000.00000000333	-2000.00000000333\\
-419999.999999997	-419999.999999997\\
732000	732000\\
-90999.9999999975	-90999.9999999975\\
-311000.000000003	-311000.000000003\\
-166000	-166000\\
91999.9999999961	91999.9999999961\\
55000.0000000024	55000.0000000024\\
54999.9999999979	54999.9999999979\\
37000.0000000017	37000.0000000017\\
35999.9999999969	35999.9999999969\\
54999.9999999997	54999.9999999997\\
-163999.999999997	-163999.999999997\\
-93000.0000000044	-93000.0000000044\\
148000	148000\\
182000.000000002	182000.000000002\\
-183000.000000004	-183000.000000004\\
-328999.999999998	-328999.999999998\\
366000.000000001	366000.000000001\\
-110000	-110000\\
1000.00000000122	1000.00000000122\\
253999.999999998	253999.999999998\\
-108000	-108000\\
-109999.999999997	-109999.999999997\\
-202000.000000002	-202000.000000002\\
166000.000000001	166000.000000001\\
70999.9999999979	70999.9999999979\\
2000.00000000333	2000.00000000333\\
-1000.00000000211	-1000.00000000211\\
129000.000000002	129000.000000002\\
-129000.000000001	-129000.000000001\\
165000	165000\\
-165000.000000003	-165000.000000003\\
-439000	-439000\\
403000.000000001	403000.000000001\\
366000.000000002	366000.000000002\\
-623000.000000001	-623000.000000001\\
164999.999999998	164999.999999998\\
202000.000000005	202000.000000005\\
182999.999999996	182999.999999996\\
-110999.999999999	-110999.999999999\\
-182000.000000001	-182000.000000001\\
-293000.000000001	-293000.000000001\\
309999.999999999	309999.999999999\\
39000.0000000024	39000.0000000024\\
106999.999999998	106999.999999998\\
-326999.999999998	-326999.999999998\\
254999.999999997	254999.999999997\\
-164999.999999999	-164999.999999999\\
184000	184000\\
-183000.000000002	-183000.000000002\\
-74999.9999999993	-74999.9999999993\\
350000	350000\\
-312000.000000001	-312000.000000001\\
108999.999999998	108999.999999998\\
147000.000000001	147000.000000001\\
129000	129000\\
-404000.000000001	-404000.000000001\\
999.999999997669	999.999999997669\\
329000.000000001	329000.000000001\\
-202000.000000003	-202000.000000003\\
56000.0000000036	56000.0000000036\\
-220000.000000001	-220000.000000001\\
219999.999999999	219999.999999999\\
145999.999999999	145999.999999999\\
-92000.0000000023	-92000.0000000023\\
-199999.999999997	-199999.999999997\\
-202999.999999999	-202999.999999999\\
275999.999999999	275999.999999999\\
145000	145000\\
130000.000000001	130000.000000001\\
-405000	-405000\\
19999.9999999987	19999.9999999987\\
128000.000000001	128000.000000001\\
293000	293000\\
-294000.000000004	-294000.000000004\\
-108999.999999998	-108999.999999998\\
91999.9999999987	91999.9999999987\\
127000	127000\\
-128000.000000001	-128000.000000001\\
56000.0000000009	56000.0000000009\\
108999.999999998	108999.999999998\\
-128999.999999999	-128999.999999999\\
-143999.999999998	-143999.999999998\\
216999.999999997	216999.999999997\\
-162999.999999997	-162999.999999997\\
17999.9999999989	17999.9999999989\\
-54999.9999999997	-54999.9999999997\\
439000	439000\\
-293000.000000003	-293000.000000003\\
-328999.999999997	-328999.999999997\\
585999.999999998	585999.999999998\\
-93000	-93000\\
-180999.999999997	-180999.999999997\\
-93000.0000000044	-93000.0000000044\\
1000.00000000033	1000.00000000033\\
146000	146000\\
-110999.999999996	-110999.999999996\\
-218000.000000005	-218000.000000005\\
732000.000000002	732000.000000002\\
-532000.000000002	-532000.000000002\\
-199999.999999998	-199999.999999998\\
255999.999999997	255999.999999997\\
182000.000000003	182000.000000003\\
-54000.000000002	-54000.000000002\\
-54999.9999999997	-54999.9999999997\\
-292999.999999997	-292999.999999997\\
19000.0000000001	19000.0000000001\\
272999.999999998	272999.999999998\\
184000	184000\\
-145999.999999999	-145999.999999999\\
-239000.000000002	-239000.000000002\\
184000.000000001	184000.000000001\\
-110000	-110000\\
55000.0000000015	55000.0000000015\\
-19000.0000000037	-19000.0000000037\\
128000	128000\\
999.999999999446	999.999999999446\\
18000.0000000007	18000.0000000007\\
-147000.000000001	-147000.000000001\\
-55000.0000000006	-55000.0000000006\\
-71999.9999999974	-71999.9999999974\\
439000	439000\\
-312000.000000001	-312000.000000001\\
-274999.999999999	-274999.999999999\\
367999.999999998	367999.999999998\\
89999.999999999	89999.999999999\\
-220000	-220000\\
-15999.9999999965	-15999.9999999965\\
454999.999999996	454999.999999996\\
-731000.000000001	-731000.000000001\\
403000.000000001	403000.000000001\\
146000	146000\\
-256000	-256000\\
71999.9999999992	71999.9999999992\\
-235999.999999997	-235999.999999997\\
365000.000000001	365000.000000001\\
183000.000000001	183000.000000001\\
-531000.000000003	-531000.000000003\\
129000.000000006	129000.000000006\\
200999.999999997	200999.999999997\\
-146999.999999997	-146999.999999997\\
91999.9999999996	91999.9999999996\\
-202000.000000001	-202000.000000001\\
330999.999999999	330999.999999999\\
-37999.9999999958	-37999.9999999958\\
-36000.0000000022	-36000.0000000022\\
-310999.999999998	-310999.999999998\\
199999.999999997	199999.999999997\\
38000.0000000011	38000.0000000011\\
-237999.999999997	-237999.999999997\\
72999.9999999968	72999.9999999968\\
474999.999999999	474999.999999999\\
-364999.999999998	-364999.999999998\\
-90999.9999999984	-90999.9999999984\\
254999.999999997	254999.999999997\\
-311000.000000001	-311000.000000001\\
366000	366000\\
-310000	-310000\\
108000.000000004	108000.000000004\\
-163000.000000004	-163000.000000004\\
273000.000000001	273000.000000001\\
-16000	-16000\\
-93999.9999999959	-93999.9999999959\\
18999.9999999992	18999.9999999992\\
-292000.000000001	-292000.000000001\\
439000.000000001	439000.000000001\\
-147000.000000002	-147000.000000002\\
-385000	-385000\\
440999.999999999	440999.999999999\\
53999.9999999985	53999.9999999985\\
92000.0000000014	92000.0000000014\\
-403000.000000003	-403000.000000003\\
0	0\\
640000	640000\\
-383000.000000002	-383000.000000002\\
-183999.999999998	-183999.999999998\\
91000.0000000019	91000.0000000019\\
-182000	-182000\\
36000.0000000005	36000.0000000005\\
128999.999999998	128999.999999998\\
439000.000000002	439000.000000002\\
-714999.999999999	-714999.999999999\\
201999.999999997	201999.999999997\\
312000.000000002	312000.000000002\\
-56000.0000000027	-56000.0000000027\\
37000.0000000026	37000.0000000026\\
-439000.000000001	-439000.000000001\\
109999.999999999	109999.999999999\\
126999.999999999	126999.999999999\\
367000.000000001	367000.000000001\\
-255999.999999999	-255999.999999999\\
-404000.000000003	-404000.000000003\\
386000.000000001	386000.000000001\\
182000	182000\\
-292999.999999998	-292999.999999998\\
-383999.999999998	-383999.999999998\\
109999.999999999	109999.999999999\\
641000	641000\\
-74000.0000000007	-74000.0000000007\\
-146000.000000001	-146000.000000001\\
-238999.999999996	-238999.999999996\\
422999.999999996	422999.999999996\\
-697999.999999998	-697999.999999998\\
404999.999999999	404999.999999999\\
-110999.999999999	-110999.999999999\\
18000.0000000007	18000.0000000007\\
18999.9999999957	18999.9999999957\\
220000.000000001	220000.000000001\\
-56000.0000000009	-56000.0000000009\\
-73000.0000000013	-73000.0000000013\\
999.999999999446	999.999999999446\\
-999.999999999446	-999.999999999446\\
36999.9999999999	36999.9999999999\\
-165000.000000004	-165000.000000004\\
-91999.9999999961	-91999.9999999961\\
386000.000000001	386000.000000001\\
164000.000000001	164000.000000001\\
-239000.000000003	-239000.000000003\\
-731000.000000001	-731000.000000001\\
659000.000000002	659000.000000002\\
200999.999999998	200999.999999998\\
-37000.0000000008	-37000.0000000008\\
-365999.999999999	-365999.999999999\\
-110000.000000001	-110000.000000001\\
166000.000000002	166000.000000002\\
328999.999999999	328999.999999999\\
-221000	-221000\\
74999.9999999993	74999.9999999993\\
35000.0000000028	35000.0000000028\\
-163000	-163000\\
364999.999999998	364999.999999998\\
-458000.000000001	-458000.000000001\\
-164000.000000001	-164000.000000001\\
330000.000000001	330000.000000001\\
34999.9999999993	34999.9999999993\\
-52999.999999999	-52999.999999999\\
180999.999999996	180999.999999996\\
-382999.999999999	-382999.999999999\\
292999.999999999	292999.999999999\\
71999.9999999983	71999.9999999983\\
-108999.999999999	-108999.999999999\\
-146000.000000001	-146000.000000001\\
-146999.999999999	-146999.999999999\\
383999.999999999	383999.999999999\\
-54000.0000000003	-54000.0000000003\\
-366999.999999999	-366999.999999999\\
-17999.9999999989	-17999.9999999989\\
769999.999999999	769999.999999999\\
-459999.999999997	-459999.999999997\\
-199000.000000002	-199000.000000002\\
-38999.999999997	-38999.999999997\\
569999.999999998	569999.999999998\\
-93000.0000000008	-93000.0000000008\\
-238000.000000001	-238000.000000001\\
-566999.999999997	-566999.999999997\\
365999.999999996	365999.999999996\\
329000.000000002	329000.000000002\\
-108999.999999996	-108999.999999996\\
-37000.0000000017	-37000.0000000017\\
165000.000000004	165000.000000004\\
72999.9999999977	72999.9999999977\\
-330000	-330000\\
-108999.999999997	-108999.999999997\\
347000.000000001	347000.000000001\\
-109000.000000001	-109000.000000001\\
-146999.999999998	-146999.999999998\\
128999.999999999	128999.999999999\\
-19999.9999999987	-19999.9999999987\\
-162999.999999999	-162999.999999999\\
473999.999999998	473999.999999998\\
-106999.999999998	-106999.999999998\\
-94000.0000000012	-94000.0000000012\\
-274000.000000003	-274000.000000003\\
-54000.0000000003	-54000.0000000003\\
181999.999999999	181999.999999999\\
54999.9999999988	54999.9999999988\\
-273999.999999999	-273999.999999999\\
530999.999999999	530999.999999999\\
-276000.000000001	-276000.000000001\\
-162999.999999998	-162999.999999998\\
456999.999999999	456999.999999999\\
-421000.000000001	-421000.000000001\\
-38000.0000000002	-38000.0000000002\\
204000	204000\\
218000	218000\\
-202000.000000002	-202000.000000002\\
-291000.000000001	-291000.000000001\\
292000.000000002	292000.000000002\\
-422000.000000001	-422000.000000001\\
348999.999999999	348999.999999999\\
330000.000000004	330000.000000004\\
-368000.000000003	-368000.000000003\\
38000.0000000011	38000.0000000011\\
-109999.999999999	-109999.999999999\\
329000.000000001	329000.000000001\\
74999.9999999966	74999.9999999966\\
-258999.999999999	-258999.999999999\\
74999.9999999984	74999.9999999984\\
-294000	-294000\\
38000.0000000002	38000.0000000002\\
89999.999999999	89999.999999999\\
221000	221000\\
-109999.999999999	-109999.999999999\\
-166000.000000004	-166000.000000004\\
130000.000000004	130000.000000004\\
-20000.0000000031	-20000.0000000031\\
-36000.0000000014	-36000.0000000014\\
238000.000000001	238000.000000001\\
54999.9999999979	54999.9999999979\\
-329000	-329000\\
17999.9999999998	17999.9999999998\\
-999.999999998557	-999.999999998557\\
-35999.9999999978	-35999.9999999978\\
-54000.000000002	-54000.000000002\\
272999.999999999	272999.999999999\\
-89999.9999999998	-89999.9999999998\\
-202000.000000003	-202000.000000003\\
475000	475000\\
-510999.999999997	-510999.999999997\\
35999.9999999996	35999.9999999996\\
567000	567000\\
-346999.999999998	-346999.999999998\\
-146999.999999998	-146999.999999998\\
73999.9999999989	73999.9999999989\\
255000	255000\\
-219000	-219000\\
73000.0000000004	73000.0000000004\\
-181999.999999999	-181999.999999999\\
-313000.000000002	-313000.000000002\\
331000.000000002	331000.000000002\\
274000.000000002	274000.000000002\\
109999.999999999	109999.999999999\\
-274999.999999999	-274999.999999999\\
110999.999999999	110999.999999999\\
-313000.000000001	-313000.000000001\\
38000.000000002	38000.000000002\\
494000	494000\\
-385000.000000001	-385000.000000001\\
-18000.0000000033	-18000.0000000033\\
-273999.999999997	-273999.999999997\\
437999.999999998	437999.999999998\\
276000	276000\\
-567999.999999999	-567999.999999999\\
218999.999999998	218999.999999998\\
-291999.999999998	-291999.999999998\\
128000.000000003	128000.000000003\\
274999.999999999	274999.999999999\\
17000.0000000003	17000.0000000003\\
-72000.0000000018	-72000.0000000018\\
-257000.000000001	-257000.000000001\\
219999.999999999	219999.999999999\\
55000.0000000006	55000.0000000006\\
-111000.000000002	-111000.000000002\\
-33999.9999999963	-33999.9999999963\\
14999.9999999988	14999.9999999988\\
-89000.0000000022	-89000.0000000022\\
-74000.0000000007	-74000.0000000007\\
146999.999999999	146999.999999999\\
217999.999999999	217999.999999999\\
1000.00000000033	1000.00000000033\\
-218999.999999999	-218999.999999999\\
-8.88178419700125e-10	-8.88178419700125e-10\\
-74999.9999999975	-74999.9999999975\\
19999.9999999969	19999.9999999969\\
146000	146000\\
18000.0000000007	18000.0000000007\\
109999.999999999	109999.999999999\\
-530999.999999999	-530999.999999999\\
219999.999999997	219999.999999997\\
494000.000000001	494000.000000001\\
-53999.9999999985	-53999.9999999985\\
-587000.000000001	-587000.000000001\\
-18999.9999999983	-18999.9999999983\\
423000	423000\\
-55000.0000000006	-55000.0000000006\\
71999.9999999992	71999.9999999992\\
-439999.999999998	-439999.999999998\\
185000	185000\\
493999.999999999	493999.999999999\\
-148000.000000003	-148000.000000003\\
-565999.999999999	-565999.999999999\\
-92000.0000000005	-92000.0000000005\\
548999.999999998	548999.999999998\\
292999.999999998	292999.999999998\\
-494000.000000001	-494000.000000001\\
-256999.999999999	-256999.999999999\\
312000	312000\\
237999.999999999	237999.999999999\\
-329999.999999998	-329999.999999998\\
17999.9999999971	17999.9999999971\\
440000.000000005	440000.000000005\\
-311000.000000001	-311000.000000001\\
-348000.000000002	-348000.000000002\\
566000	566000\\
-418999.999999997	-418999.999999997\\
72999.9999999995	72999.9999999995\\
310000	310000\\
93000.0000000008	93000.0000000008\\
-770000.000000001	-770000.000000001\\
182999.999999997	182999.999999997\\
421000.000000002	421000.000000002\\
148000.000000001	148000.000000001\\
-294999.999999999	-294999.999999999\\
-162999.999999999	-162999.999999999\\
274000.000000001	274000.000000001\\
90999.9999999984	90999.9999999984\\
-329000.000000003	-329000.000000003\\
-36999.999999999	-36999.999999999\\
182999.999999999	182999.999999999\\
19000.000000001	19000.000000001\\
-73999.9999999989	-73999.9999999989\\
148000	148000\\
-149000.000000001	-149000.000000001\\
-108000	-108000\\
403000.000000001	403000.000000001\\
-239000.000000003	-239000.000000003\\
-73000.0000000013	-73000.0000000013\\
109000.000000001	109000.000000001\\
-199000.000000002	-199000.000000002\\
-93999.9999999994	-93999.9999999994\\
422000	422000\\
-255999.999999998	-255999.999999998\\
2.66453525910038e-09	2.66453525910038e-09\\
128000	128000\\
128000.000000001	128000.000000001\\
-146000.000000003	-146000.000000003\\
181999.999999999	181999.999999999\\
-493000	-493000\\
-75000.0000000002	-75000.0000000002\\
716000	716000\\
-129000	-129000\\
-659000.000000002	-659000.000000002\\
311000	311000\\
256000.000000003	256000.000000003\\
-218000.000000001	-218000.000000001\\
-2000.00000000244	-2000.00000000244\\
385000.000000002	385000.000000002\\
-274000.000000002	-274000.000000002\\
-183999.999999997	-183999.999999997\\
330999.999999999	330999.999999999\\
-185000.000000002	-185000.000000002\\
-144999.999999999	-144999.999999999\\
72999.9999999995	72999.9999999995\\
90999.9999999975	90999.9999999975\\
19000.0000000019	19000.0000000019\\
181999.999999998	181999.999999998\\
-273999.999999997	-273999.999999997\\
19000.000000001	19000.000000001\\
-147000.000000002	-147000.000000002\\
255999.999999998	255999.999999998\\
0	0\\
-364999.999999999	-364999.999999999\\
400999.999999999	400999.999999999\\
91999.9999999987	91999.9999999987\\
-163999.999999999	-163999.999999999\\
-255999.999999998	-255999.999999998\\
455999.999999996	455999.999999996\\
-108000	-108000\\
-348999.999999999	-348999.999999999\\
36999.9999999972	36999.9999999972\\
329000	329000\\
-35999.9999999996	-35999.9999999996\\
19000.0000000001	19000.0000000001\\
-440999.999999996	-440999.999999996\\
164999.999999998	164999.999999998\\
110999.999999998	110999.999999998\\
109000.000000003	109000.000000003\\
-72000.0000000001	-72000.0000000001\\
15999.9999999982	15999.9999999982\\
350000.000000002	350000.000000002\\
-386000.000000004	-386000.000000004\\
-455999.999999999	-455999.999999999\\
436999.999999998	436999.999999998\\
369000.000000003	369000.000000003\\
-295000.000000003	-295000.000000003\\
-274000.000000003	-274000.000000003\\
402000.000000004	402000.000000004\\
-35000.000000001	-35000.000000001\\
-110999.999999997	-110999.999999997\\
-72999.9999999995	-72999.9999999995\\
-146000	-146000\\
457000.000000002	457000.000000002\\
-255000.000000002	-255000.000000002\\
-313000.000000001	-313000.000000001\\
329999.999999999	329999.999999999\\
74000.0000000007	74000.0000000007\\
237000.000000001	237000.000000001\\
-146000.000000001	-146000.000000001\\
-914999.999999999	-914999.999999999\\
1025000	1025000\\
73999.9999999998	73999.9999999998\\
-478000.000000002	-478000.000000002\\
-237000.000000001	-237000.000000001\\
238999.999999999	238999.999999999\\
35000.0000000019	35000.0000000019\\
385999.999999998	385999.999999998\\
-166000.000000001	-166000.000000001\\
-128000	-128000\\
38000.0000000029	38000.0000000029\\
-295000.000000001	-295000.000000001\\
238999.999999997	238999.999999997\\
-53999.9999999976	-53999.9999999976\\
-130000.000000001	-130000.000000001\\
331000	331000\\
181999.999999998	181999.999999998\\
-457000.000000002	-457000.000000002\\
-54999.9999999979	-54999.9999999979\\
129000	129000\\
327000.000000001	327000.000000001\\
-492000.000000002	-492000.000000002\\
348000	348000\\
-93999.9999999985	-93999.9999999985\\
-363000	-363000\\
254000	254000\\
73999.9999999972	73999.9999999972\\
8.88178419700125e-10	8.88178419700125e-10\\
238000.000000001	238000.000000001\\
146999.999999999	146999.999999999\\
-477000.000000002	-477000.000000002\\
-72999.9999999986	-72999.9999999986\\
257000	257000\\
-73999.9999999963	-73999.9999999963\\
-403000	-403000\\
476999.999999998	476999.999999998\\
37000.0000000017	37000.0000000017\\
-185000.000000003	-185000.000000003\\
20000.0000000013	20000.0000000013\\
53999.9999999967	53999.9999999967\\
2.66453525910038e-09	2.66453525910038e-09\\
-109000	-109000\\
164000.000000001	164000.000000001\\
-219000.000000001	-219000.000000001\\
584999.999999997	584999.999999997\\
-273999.999999999	-273999.999999999\\
-440000	-440000\\
-36000.0000000005	-36000.0000000005\\
475000	475000\\
-70999.9999999979	-70999.9999999979\\
88999.9999999986	88999.9999999986\\
-401000	-401000\\
237000.000000004	237000.000000004\\
145999.999999997	145999.999999997\\
-382999.999999998	-382999.999999998\\
-19000.0000000001	-19000.0000000001\\
292999.999999999	292999.999999999\\
-92000.0000000023	-92000.0000000023\\
-35999.999999996	-35999.999999996\\
403000	403000\\
17999.9999999971	17999.9999999971\\
-787999.999999997	-787999.999999997\\
347999.999999998	347999.999999998\\
513999.999999998	513999.999999998\\
-678999.999999998	-678999.999999998\\
-72000.0000000018	-72000.0000000018\\
567000	567000\\
-421999.999999999	-421999.999999999\\
222000	222000\\
308999.999999999	308999.999999999\\
-310000	-310000\\
-129000.000000002	-129000.000000002\\
-53999.9999999976	-53999.9999999976\\
34999.9999999984	34999.9999999984\\
350000.000000004	350000.000000004\\
-55999.9999999991	-55999.9999999991\\
};
\end{axis}

\begin{axis}[%
width=4.927cm,
height=3.484cm,
at={(6.484cm,14.516cm)},
scale only axis,
xmin=-1000000,
xmax=1000000,
xlabel style={font=\color{white!15!black}},
xlabel={$\delta^3 u(t)$},
ymin=-7690500000,
ymax=10009500000,
ylabel style={font=\color{white!15!black}},
ylabel={y(t)},
axis background/.style={fill=white},
title={C2, R = 0.4793},
axis x line*=bottom,
axis y line*=left
]
\addplot[only marks, mark=*, mark options={}, mark size=1.5000pt, color=mycolor1, fill=mycolor1] table[row sep=crcr]{%
x	y\\
-111000.000000001	1831200000\\
238999.999999998	732400000\\
-129000.000000001	-2075300000\\
147000.000000002	2441500000\\
127999.999999996	-1709100000\\
-494999.999999997	-1464600000\\
202999.999999998	3539800000\\
53999.9999999976	-2197200000\\
36000.0000000014	976700000.000001\\
293999.999999998	-366500000\\
-348000	-1098400000\\
-183999.999999998	1098700000\\
-292000	121700000\\
457000.000000002	-1098200000\\
329999.999999995	2929500000\\
-219999.999999998	-3051900000\\
73999.9999999998	1220900000\\
-37000.0000000017	-122100000.000002\\
-587000	-2075300000\\
789000.000000004	3784400000\\
-184000.000000003	-1343100000\\
-493999.999999999	-2441100000\\
530999.999999999	3417900000\\
-74000.0000000016	-1220900000\\
-327999.999999998	-2319200000\\
326999.999999997	4760800000\\
112000.000000002	-3540000000\\
-457999.999999997	-854699999.999999\\
347999.999999995	3540100000\\
35000.000000001	-1464600000\\
-126000.000000001	-1587300000\\
17000.0000000003	1587300000\\
-147000.000000002	-854900000\\
57000.0000000013	854900000\\
162999.999999998	365900000.000001\\
-109999.999999999	-1586700000\\
38000.0000000029	976500000\\
-55000.0000000006	-199999.999999889\\
-93000.0000000026	-732100000\\
276000.000000001	2197000000\\
35999.9999999987	-1952900000\\
-275000.000000002	-976700000\\
238000.000000002	2319300000\\
-255000.000000001	-1464700000\\
-111000.000000002	121900000\\
385000.000000002	1831200000\\
-183000.000000001	-2685600000\\
-147000	1220700000\\
200999.999999997	488200000\\
56000.0000000036	-121800000\\
-275000	-2075600000\\
385000	4272800000\\
16999.9999999977	-3296000000\\
-311000	-610499999.999997\\
-109000	1465100000\\
200999.999999999	610200000\\
237999.999999999	-122100000.000002\\
-329000	-1831000000\\
-111000	1586900000\\
329999.999999997	122100000.000001\\
-35999.9999999978	-976500000\\
-275000.000000002	121900000\\
183000.000000001	976700000.000001\\
-129000	-732600000.000001\\
258000.000000001	244400000\\
-111000.000000002	-610399999.999999\\
18999.9999999983	1464500000\\
-110999.999999998	-1708500000\\
202999.999999998	1342500000\\
-93000.0000000008	-1586900000\\
-72000.0000000009	1464900000\\
16999.9999999986	-366199999.999999\\
19000.000000001	-488400000\\
-237000	199999.999999889\\
162999.999999999	610200000.000001\\
256999.999999998	1098600000\\
-109999.999999997	-1953000000\\
165999.999999999	976599999.999999\\
-533000.000000001	-2563700000\\
221000.000000001	3418200000\\
183000.000000001	-610500000\\
37000.0000000026	244200000.000001\\
-331000.000000004	-3540100000\\
57000.0000000039	3662300000\\
436999.999999996	1586600000\\
-290999.999999995	-4760400000\\
90999.9999999975	1952900000\\
-129000	732500000.000002\\
-54000.0000000011	-732500000\\
-73000.0000000004	-610299999.999999\\
237000.000000002	2685600000\\
92999.9999999982	-2441500000\\
-313000.000000001	-244000000\\
-36000.0000000014	1220600000\\
37000.0000000008	-366300000\\
275000	854700000\\
-129000.000000002	-1587100000\\
-128000	488400000\\
147000.000000001	976500000\\
-111000.000000002	-1953100000\\
221000.000000004	3418000000\\
-201999.999999999	-4272600000\\
165000	3296200000\\
-54999.9999999988	-1465300000\\
-201000.000000003	-731900000.000001\\
273000.000000001	2318900000\\
-126000.000000001	-2074900000\\
-148000.000000001	243800000\\
201999.999999998	1587400000\\
55000.0000000006	-1587300000\\
-184000.000000002	610400000.000001\\
148000.000000002	-243900000\\
-166000.000000002	-244500000.000001\\
74000.0000000016	732700000.000001\\
127999.999999999	-3.5527136788005e-07\\
-238000.000000003	-854800000\\
54000.000000002	500000.000000256\\
-35000.000000001	1098100000\\
218999.999999998	-487900000.000001\\
-74000.0000000016	121900000.000001\\
183000.000000002	-1.77635683940025e-06\\
-456000.000000001	-1586800000\\
72000.0000000018	976500000.000001\\
420999.999999999	3418000000\\
-291999.999999998	-5859600000\\
-1000.00000000122	3540400000\\
330000.000000003	609999999.999999\\
-329000.000000002	-2929300000\\
-203000.000000002	732000000.000001\\
202999.999999999	2563800000\\
-147999.999999998	-2929700000\\
258000.000000002	2074800000\\
53999.9999999994	600000.000000378\\
19000.0000000028	-1465400000\\
-202000.000000004	-243699999.999999\\
72000.0000000009	1342500000\\
-51999.9999999996	-610299999.999999\\
15999.9999999973	366200000\\
165000	-121900000.000002\\
-163999.999999999	-366299999.999999\\
-35999.9999999969	-300000.000000367\\
52999.9999999937	488800000\\
-89999.9999999972	-488700000.000001\\
-20000.0000000013	300000.000000367\\
2000.00000000511	732099999.999998\\
365999.999999998	488600000.000002\\
-111000.000000002	-1587000000\\
-220000.000000001	-366499999.999999\\
37999.9999999994	1343300000\\
182000	243599999.999999\\
-126999.999999998	-854100000\\
-202000.000000003	-244300000.000001\\
-55999.9999999974	366200000.000001\\
459999.999999997	1220800000\\
-239999.999999999	-1831200000\\
-8.88178419700125e-10	854600000\\
37999.9999999985	-122100000\\
90999.9999999993	366200000\\
-72999.9999999977	-976500000\\
-20000.0000000005	976400000.000001\\
-271999.999999999	-976300000\\
-39000.0000000006	365900000\\
368000.000000001	1221000000\\
108999.999999999	-488499999.999999\\
54999.9999999997	-732300000.000001\\
-147000.000000002	-100000.000000122\\
-108999.999999998	-732299999.999999\\
36999.9999999981	2075100000\\
-440999.999999999	-2075100000\\
586999.999999999	1464700000\\
-238000.000000002	100000.000000477\\
17999.9999999998	-1831100000\\
-110000	2319400000\\
275000.000000001	-610300000.000001\\
146000.000000003	610100000.000002\\
-255000.000000003	-2929400000\\
199000.000000002	3295700000\\
-419000	-3173800000\\
201000	3052000000\\
-1000.00000000033	-1343200000\\
130000	854800000.000001\\
-94000.0000000012	-1953100000\\
-89999.999999999	1708800000\\
110000.000000002	-610300000.000002\\
-128000	0\\
217999.999999997	854699999.999999\\
-382999.999999998	-1831400000\\
257000.000000001	1465000000\\
273000	976699999.999998\\
-146000.000000003	-1953300000\\
-292999.999999999	-976600000.000001\\
-128000.000000001	1831200000\\
348000.000000002	854500000\\
-35999.9999999987	-854600000\\
34999.9999999984	-854600000.000002\\
2000.00000000333	1221100000\\
89999.9999999963	-488700000.000003\\
-182999.999999998	-1098300000\\
-73000.0000000004	609899999.999999\\
56000.0000000009	977200000.000001\\
308999.999999999	975899999.999999\\
-125999.999999997	-2929100000\\
-477000	-366700000\\
439999.999999999	3662400000\\
72000.0000000001	-1342800000\\
-364999.999999997	-2563600000\\
128000	2563600000\\
218999.999999998	610400000\\
-108999.999999997	-2197600000\\
-165000.000000003	244500000\\
163999.999999999	2929700000\\
-108999.999999998	-4639000000\\
-55000.0000000006	2929900000\\
292999.999999999	1342800000\\
-18999.9999999983	-2563400000\\
-219000.000000001	-488599999.999999\\
-128000.000000004	1343000000\\
-999.999999997669	-244100000.000001\\
604999.999999999	2807600000\\
-220000.000000003	-4028500000\\
-238000	-366000000.000001\\
-146999.999999999	2685400000\\
258000.000000002	-610300000.000001\\
16999.999999995	-732300000.000001\\
-145999.999999998	365999999.999999\\
-19000.000000001	-121999999.999999\\
54999.9999999988	244300000\\
38000.0000000002	121899999.999999\\
218000	366200000.000001\\
-494000.000000001	-2319300000\\
276000.000000002	2441500000\\
162999.999999998	610200000.000001\\
-126999.999999999	-2197200000\\
-73999.9999999989	-488200000.000001\\
-127000.000000002	2563400000\\
437999.999999999	-610399999.999999\\
-17999.9999999971	-366100000.000001\\
-585000.000000002	-2929800000\\
201000	4638700000\\
72999.9999999968	-2319200000\\
183000	732199999.999999\\
-17999.9999999998	-121900000\\
-348000.000000002	-1831200000\\
329000	3418200000\\
55999.9999999991	-2197500000\\
-274999.999999999	-854399999.999999\\
145999.999999998	2441400000\\
55000.0000000006	-1342800000\\
-183000.000000002	-610200000\\
294000.000000004	1830800000\\
-295000.000000003	-1953000000\\
167000	732600000\\
-130000.000000001	1098300000\\
-127000	-2807400000\\
255999.999999998	3784100000\\
-91999.9999999996	-2807500000\\
-35999.9999999987	366000000\\
402999.999999998	2807800000\\
-148000	-3906300000\\
-291000	488200000\\
16999.9999999986	2319400000\\
-91000.0000000011	-2197200000\\
422000.000000001	2319300000\\
-332000.000000002	-1831200000\\
-34000.0000000007	-488000000\\
-75000.000000001	2074900000\\
440000	-1098500000\\
-311000.000000002	-366100000\\
292000.000000001	732200000\\
-218000.000000001	99999.9999983459\\
162999.999999999	-1464700000\\
-273000	1952900000\\
108000.000000001	-1708900000\\
-72000.0000000036	1709000000\\
-54999.9999999988	-1709000000\\
-1.77635683940025e-09	1220800000\\
165000	-610599999.999999\\
71999.9999999983	1098900000\\
-127000	-1098700000\\
-200999.999999999	-854600000\\
437999.999999998	2563500000\\
-36000.0000000014	-976400000\\
-348000	-2929900000\\
165999.999999999	3784300000\\
-111999.999999999	-1342800000\\
312999.999999999	488199999.999998\\
-73999.9999999989	-243899999.999999\\
19000.0000000001	-1220900000\\
-221000.000000002	976500000.000002\\
-127000.000000001	100000.000000122\\
128000.000000002	-121900000\\
72000.0000000009	610000000\\
38999.9999999997	-365900000\\
-130999.999999999	-732600000\\
-52000.0000000023	610300000.000001\\
326999.999999999	1098900000\\
-162999.999999999	-1709300000\\
-56000.0000000009	300000.000000367\\
237999.999999999	1708700000\\
-71999.9999999992	-1464700000\\
-549999.999999999	-854400000.000001\\
310000.000000001	1464700000\\
312999.999999997	1953100000\\
-201999.999999997	-4150300000\\
73999.9999999989	2807700000\\
15999.9999999982	-976800000\\
-289999.999999999	-1220600000\\
125999.999999999	2441600000\\
166000	-488599999.999998\\
126999.999999998	-610200000.000001\\
-547999.999999998	-2197100000\\
493999.999999999	4882400000\\
-73999.9999999981	-3173400000\\
-237000.000000002	-1342900000\\
201000	3295600000\\
53999.9999999985	-854000000\\
-108000.000000001	-1709400000\\
-294999.999999999	122300000.000001\\
258000.000000003	1953100000\\
310999.999999999	976399999.999999\\
-56000	-3051500000\\
-383000.000000002	-732600000.000002\\
89999.9999999998	2929700000\\
-16999.9999999995	-1098500000\\
71999.9999999983	121900000\\
203000.000000003	488300000.000001\\
-258000.000000004	-1342700000\\
-53999.9999999976	-121999999.999999\\
165000.000000002	2441200000\\
163999.999999998	-1342600000\\
-182000.000000001	-1586900000\\
-221000	1342500000\\
-18000.0000000007	-243700000\\
477000.000000003	2074800000\\
-221000.000000001	-2807500000\\
111000.000000002	1098800000\\
-166000	-610600000.000002\\
-89999.999999999	366400000\\
236999.999999997	1098600000\\
-37000.0000000017	-1342900000\\
-163999.999999999	-610200000.000001\\
0	1708900000\\
-256999.999999999	-1098600000\\
330999.999999999	122100000\\
52999.999999999	2197100000\\
184000	-3539800000\\
-311000.000000002	1830800000\\
-8.88178419700125e-10	-732100000\\
311000.000000001	1708700000\\
-348000.000000002	-1953000000\\
55000.0000000006	122099999.999999\\
165000.000000001	1708800000\\
-274999.999999997	-1708700000\\
0	-244400000\\
219999.999999996	2319500000\\
238000	-1220800000\\
-255999.999999998	-732300000.000001\\
-129000.000000002	-366400000\\
18999.9999999983	1220800000\\
183000.000000001	488400000.000001\\
-458000.000000003	-1953200000\\
202000.000000001	-199999.999999889\\
548000	5127200000\\
-530000.000000002	-7690500000\\
92000.0000000014	4394600000\\
126999.999999998	1220400000\\
54999.9999999997	-3295400000\\
-71999.9999999965	1220200000\\
-184000.000000002	-487900000.000001\\
54999.999999997	1098500000\\
0	-488500000\\
37000.0000000017	399999.999999778\\
71999.9999999983	488000000\\
-180999.999999997	-1464700000\\
125999.999999999	1952900000\\
20000.0000000022	-1830700000\\
219000.000000001	2929400000\\
-311000.000000002	-4028200000\\
-37000.0000000008	1098600000\\
256000.000000001	3662100000\\
-238000.000000001	-5493200000\\
148000.000000003	4150600000\\
181999.999999999	-854800000.000001\\
-220000.000000002	-2074900000\\
-220000.000000001	1220400000\\
1000.00000000389	610600000\\
347999.999999997	732400000\\
-19000.0000000001	-1587200000\\
-348000.000000001	-609900000.000001\\
347999.999999998	2319000000\\
-53999.9999999967	-976400000.000001\\
-240000	-1587100000\\
148999.999999998	1831300000\\
126000	610200000.000001\\
-329000.000000001	-3173800000\\
166000.000000002	3417900000\\
327999.999999998	-121900000.000001\\
-584999.999999997	-4150500000\\
237999.999999999	4516600000\\
366000	-610299999.999998\\
-385000.000000002	-2685600000\\
275000.000000002	3662000000\\
-219000.000000002	-3661800000\\
126999.999999999	2563200000\\
54999.9999999997	-610300000.000002\\
-400999.999999997	-2197100000\\
308999.999999997	3417700000\\
999.999999998557	-976300000.000001\\
-219000	-1709200000\\
383000	1831200000\\
-183000.000000001	488300000\\
-200000.000000001	-3662300000\\
-183999.999999999	3662300000\\
439999.999999997	-122100000\\
-18999.9999999992	-1465000000\\
-8.88178419700125e-10	244400000\\
-18000.0000000016	854200000\\
-110000	-1586600000\\
146999.999999999	2196900000\\
-20000.0000000005	-2196800000\\
-180999.999999998	976100000.000001\\
183000	610600000\\
-277000.000000001	-1953200000\\
424000.000000003	3051900000\\
-203000.000000003	-2075400000\\
-220000.000000001	-976499999.999999\\
111000.000000003	2197400000\\
328999.999999999	488000000.000001\\
-164999.999999999	-2807300000\\
-127000.000000001	1464700000\\
-39000.0000000015	-122100000.000001\\
222000.000000001	1464800000\\
-147000.000000002	-3173700000\\
-183999.999999999	2441400000\\
219999.999999998	-488399999.999999\\
19000.0000000037	488399999.999999\\
72999.9999999986	-854499999.999999\\
-238000	-366399999.999999\\
182999.999999998	854699999.999999\\
163999.999999998	732500000\\
-364999.999999999	-2197600000\\
-164999.999999999	488600000\\
328999.999999996	1831000000\\
183000.000000002	-244400000\\
-109000	-1220500000\\
-110000	-366000000\\
53999.9999999985	976100000\\
-36000.0000000014	-243699999.999999\\
-8.88178419700125e-10	-122499999.999999\\
-127999.999999997	-243700000\\
182999.999999999	1220400000\\
92000.0000000005	-610199999.999999\\
-367000.000000003	-1831200000\\
219000.000000001	2563500000\\
2000.00000000067	-732100000\\
-999.999999999446	-610900000\\
54999.9999999962	732900000\\
-183999.999999997	-1343000000\\
74999.9999999984	1831100000\\
108000.000000003	-488400000.000003\\
294999.999999998	244400000.000003\\
-277000.000000001	-1465100000\\
-71000.0000000006	-243999999.999999\\
-999.999999998557	1953100000\\
-311999.999999999	-1586900000\\
348999.999999997	1831000000\\
-257000.000000001	-1831000000\\
404000.000000003	1586800000\\
-93000.0000000035	-1342500000\\
-238000	-300000.000001077\\
165999.999999999	854600000\\
-55999.9999999991	244200000\\
73999.9999999972	-1464900000\\
-183999.999999998	854500000\\
312000	1342700000\\
17999.9999999998	-2319100000\\
-145999.999999997	1586500000\\
-129000.000000001	-1952600000\\
18999.9999999992	1586500000\\
0	610500000.000001\\
219000.000000001	-854500000.000001\\
-146000.000000001	-488200000\\
-53999.9999999985	-299999.999999656\\
235999.999999997	2075700000\\
-493000.000000001	-3296400000\\
220000.000000001	1831400000\\
364999.999999997	1586700000\\
-309999.999999999	-3295800000\\
-294000.000000001	732500000\\
640999.999999998	4394300000\\
-255999.999999998	-5981200000\\
-273999.999999998	1464700000\\
327999.999999999	2929700000\\
130000	-1831000000\\
-203000.000000001	-1220800000\\
-201000.000000003	732600000.000001\\
-90999.9999999984	366099999.999999\\
145999.999999999	121899999.999999\\
293000.000000002	1099000000\\
-110000.000000002	-1343000000\\
38000.0000000038	-122199999.999999\\
33999.9999999954	122400000\\
-216999.999999997	-732600000\\
70999.9999999971	1342600000\\
-34999.9999999993	-243700000.000001\\
-19000.0000000028	-366700000.000001\\
-54999.9999999997	-487899999.999999\\
109999.999999999	1342600000\\
183000.000000002	-244200000.000001\\
-347000	-1708900000\\
16999.9999999977	1465000000\\
256000	609999999.999999\\
-198999.999999996	-1342500000\\
70999.9999999962	-122100000.000001\\
36999.9999999999	1464700000\\
238000.000000002	244299999.999998\\
-236999.999999999	-3051800000\\
-146999.999999998	2197100000\\
200999.999999998	610600000\\
17999.999999998	-1098700000\\
-218000	-976799999.999999\\
-130000	1465300000\\
365999.999999998	1342400000\\
203000.000000006	-976500000.000002\\
-165000.000000004	-2563400000\\
-239000.000000003	2075200000\\
202000.000000001	-7.105427357601e-07\\
-330000.000000001	-244299999.999999\\
-90999.9999999984	854799999.999999\\
402999.999999999	-1587200000\\
-165999.999999997	1465000000\\
256999.999999996	-366400000\\
-90999.9999999984	-365900000\\
-184000.000000003	-610500000.000001\\
111000.000000002	1342600000\\
-20000.0000000013	-610199999.999999\\
-52999.9999999999	-365999999.999999\\
-93000.0000000017	365800000\\
38000.0000000011	-121900000\\
272999.999999998	1587100000\\
-71999.9999999983	-3174000000\\
-148000.000000001	2441300000\\
-34999.9999999975	-1586700000\\
183000.000000002	2319300000\\
-129000	-2319600000\\
111000.000000002	1221100000\\
107999.999999998	366000000.000001\\
-162999.999999998	-1953200000\\
-74000.0000000016	854699999.999999\\
-275000.000000001	-244299999.999999\\
422000.000000004	2441400000\\
181999.999999998	-1098500000\\
-217999.999999999	-2441600000\\
89999.9999999981	2441700000\\
-220000.000000001	-1465100000\\
-125999.999999999	488299999.999999\\
327000.000000002	976700000\\
-474000.000000001	-976600000\\
310000.000000001	-244300000\\
165000.000000001	2197400000\\
-55000.0000000006	-3295900000\\
73999.9999999989	2563500000\\
-165999.999999998	-1464900000\\
-182000	-244099999.999999\\
219999.999999998	1831000000\\
-19999.9999999978	-1464800000\\
110999.999999997	244100000.000001\\
-127999.999999999	610500000.000001\\
-164999.999999999	-1465100000\\
-165000.000000003	610500000\\
531000	2441500000\\
18000.0000000025	-2807900000\\
1000.00000000033	610700000.000002\\
-422000.000000001	-1343000000\\
56000.0000000018	2563600000\\
382999.999999999	610099999.999999\\
-182000.000000001	-3661700000\\
72000.0000000027	2563100000\\
-126999.999999999	-1464700000\\
-91000.0000000011	1464900000\\
309000.000000002	366199999.999999\\
-254000	-2441500000\\
146000.000000001	1831200000\\
-148000.000000002	-244400000\\
-290999.999999999	-1220400000\\
602999.999999997	3295700000\\
-274999.999999999	-3295700000\\
-72000.0000000001	610000000\\
165000.000000002	1221100000\\
-93000.0000000035	-610500000\\
8.88178419700125e-10	-1098900000\\
-15999.9999999973	1831500000\\
33999.9999999945	-1465100000\\
111000.000000003	1464900000\\
-56000.0000000027	-854600000\\
-235999.999999998	-1464600000\\
493000	3539800000\\
-128000.000000004	-3417800000\\
-548999.999999998	488199999.999999\\
109999.999999999	2319200000\\
640000.000000001	-1098300000\\
-237000	-976800000\\
-531999.999999998	-732499999.999999\\
165000	2807900000\\
347999.999999999	-1343000000\\
-109000.000000002	-854400000\\
108999.999999999	1708900000\\
-329000	-2319200000\\
328999.999999998	2685500000\\
109000.000000001	-1464900000\\
-509999.999999998	-1587000000\\
564999.999999998	3784500000\\
-311000	-2686000000\\
19999.9999999996	122600000\\
54000.0000000003	609800000\\
-366999.999999998	-243600000\\
440999.999999997	365699999.999999\\
-348999.999999997	122400000\\
127999.999999998	-1709000000\\
293999.999999998	2685400000\\
-18999.9999999992	-732300000.000002\\
-293000.000000002	-2563600000\\
183000.000000001	3662200000\\
-219000.000000001	-3173800000\\
274000.000000001	2929600000\\
-109999.999999999	-2441400000\\
-73000.0000000022	1098800000\\
147000.000000002	244000000\\
-238000.000000001	-976700000.000001\\
200000.000000004	854800000\\
-255000.000000003	-244399999.999999\\
346999.999999999	732500000\\
-36000.0000000005	-1220600000\\
-164000	-100000.000000833\\
126000	854500000\\
-163000.000000001	-610300000\\
-238000	-366399999.999999\\
364999.999999997	2319700000\\
367000.000000002	-1709400000\\
-440000.000000001	-1342400000\\
91999.9999999978	2074900000\\
-328999.999999997	-1952900000\\
345999.999999998	3051500000\\
185000.000000001	-2197000000\\
-458000	-1342800000\\
291999.999999998	3539700000\\
-255000.000000001	-2807100000\\
383000	1952700000\\
-219000	-2197000000\\
-201000	2319100000\\
164000	-2197000000\\
202000.000000002	2441200000\\
-202000.000000003	-2563400000\\
1000.00000000211	1709100000\\
-111000.000000002	-854800000.000001\\
276000.000000001	1343000000\\
-258000.000000001	-2075100000\\
37000.0000000017	1098400000\\
368000	1953300000\\
-94000.0000000021	-3051900000\\
-217999.999999999	-244100000\\
-257000.000000002	2197300000\\
256000	-1342700000\\
294000.000000001	2319100000\\
-386000.000000004	-4027900000\\
294000	4271900000\\
54999.9999999997	-2929200000\\
-422000.000000001	-366499999.999999\\
-236999.999999999	1343000000\\
659000.000000001	2197000000\\
-91999.9999999996	-4150300000\\
-366000.000000003	1343000000\\
402999.999999999	1952800000\\
-110999.999999998	-1586700000\\
38999.9999999988	-366200000\\
-259000.000000001	-244500000.000001\\
74999.9999999993	1587500000\\
274000.000000001	-488800000\\
-238000.000000002	-854100000\\
72999.9999999995	243800000\\
74000.0000000016	854700000\\
-184000.000000003	-1586900000\\
184000	2075000000\\
-183999.999999999	-1952900000\\
-145000.000000002	732400000\\
345999.999999999	1098400000\\
-53999.9999999994	-1220400000\\
17999.9999999989	-122300000.000002\\
37000.0000000017	976700000\\
-73000.0000000013	-1220800000\\
-202000.000000002	122100000\\
109999.999999999	1220900000\\
-8.88178419700125e-10	-1099100000\\
55000.0000000024	733000000\\
-17999.999999998	-610800000.000001\\
200999.999999997	1342900000\\
-17999.9999999998	-2075000000\\
-532000	-610699999.999999\\
276000.000000003	3296200000\\
420999.999999998	-99999.9999983459\\
-385000.000000001	-4516700000\\
-37000.0000000017	3296000000\\
201999.999999998	1220700000\\
-36999.9999999999	-3540000000\\
-238000	2197100000\\
239000	244400000.000001\\
-130000	-1465200000\\
-33999.9999999989	1221100000\\
271999.999999998	610099999.999999\\
-273000.000000001	-2685500000\\
108999.999999999	2929700000\\
-17999.9999999998	-1342800000\\
54000.0000000003	199999.999999889\\
-143999.999999997	-1099100000\\
216999.999999997	3784800000\\
-181000	-5005500000\\
-38000.0000000011	3662600000\\
164999.999999999	-1465100000\\
147000.000000001	854600000\\
-145999.999999998	-1343000000\\
-185000	488700000\\
-126999.999999999	-122400000\\
-164000.000000002	732400000\\
548000	854700000\\
91999.9999999978	-1709000000\\
-54999.999999997	488000000\\
-311000.000000001	-1342400000\\
128000	2929500000\\
73999.9999999998	-2441500000\\
15999.9999999982	1587000000\\
-254000.000000001	-2319100000\\
145999.999999999	3417600000\\
36000.0000000005	-2441300000\\
-73000.0000000013	122200000\\
164999.999999998	1098600000\\
-127999.999999999	-732600000\\
108999.999999999	300000.000000011\\
-236999.999999999	-244500000\\
-110999.999999997	-121800000\\
567999.999999996	2807600000\\
-382999.999999996	-4760900000\\
107999.999999997	4150600000\\
-128000	-3296200000\\
148000.000000003	3296300000\\
-148000.000000003	-3418400000\\
256000	4150800000\\
-199999.999999997	-5737600000\\
17999.9999999989	5859400000\\
255999.999999998	-2685300000\\
-587000	-1831300000\\
533000	3784300000\\
-238999.999999999	-3174000000\\
0	2441600000\\
18999.9999999983	-1709000000\\
199999.999999999	1953000000\\
-292000.000000002	-3906100000\\
18000.0000000033	4638500000\\
312000	-2319100000\\
-147000	121700000.000002\\
-56000.0000000036	-243600000.000001\\
-126999.999999999	-366800000\\
184000.000000003	2319700000\\
15999.9999999973	-2319300000\\
-88999.9999999995	732100000\\
-76000.0000000014	-365900000\\
131000.000000002	976500000\\
-277000.000000003	-2319500000\\
239000.000000001	4028500000\\
202000	-2197300000\\
-148000.000000001	-1342800000\\
-70999.9999999979	976400000\\
216999.999999998	2075600000\\
-271999.999999998	-4150900000\\
15999.9999999973	3540500000\\
56999.9999999995	-1831300000\\
-75000.0000000002	732500000\\
-127000.000000002	-244300000\\
238000.000000002	976900000\\
35999.9999999987	-1587300000\\
17999.9999999971	1831200000\\
37000.0000000017	-2197000000\\
-420000	732000000.000001\\
272999.999999999	488499999.999999\\
111000.000000001	1220700000\\
35999.9999999969	-1709000000\\
-402999.999999999	-2075300000\\
495000	7080300000\\
91000.0000000002	-7568500000\\
-402000.000000001	2441300000\\
-1000.00000000122	1098900000\\
-201000	-1343000000\\
294000	1953200000\\
-20999.9999999981	-1709000000\\
295999.999999999	1464900000\\
-258000	-1587000000\\
127999.999999997	488499999.999999\\
-273999.999999997	-1099000000\\
-18000.0000000016	3052000000\\
126999.999999999	-3173900000\\
185000	2685700000\\
-314000.000000001	-2807800000\\
221999.999999998	2685500000\\
54000.0000000003	-1830800000\\
-419999.999999999	365900000.000001\\
272000.000000001	488600000\\
20999.9999999964	-200000.000000067\\
109000.000000001	-244200000.000002\\
-92000.0000000014	200000.000000067\\
-55000.0000000006	-99999.9999994117\\
1000.000000003	-610399999.999999\\
163999.999999996	2197300000\\
-255999.999999998	-4394500000\\
53999.9999999976	5371200000\\
294000.000000003	-3052100000\\
-292999.999999997	122399999.999998\\
128000	122000000\\
-146000.000000003	366200000\\
-19999.9999999987	-366400000\\
294999.999999998	1587200000\\
-111000.000000002	-2441600000\\
-201999.999999998	610400000.000002\\
167000.000000001	1465000000\\
88999.9999999977	-1098900000\\
-345999.999999999	-1220500000\\
108999.999999997	2685400000\\
256000	-854299999.999999\\
-163999.999999999	-1709300000\\
-145999.999999998	1221100000\\
455999.999999996	2074900000\\
-254999.999999998	-4394400000\\
-37000.0000000035	3295800000\\
-146999.999999998	-1953000000\\
1000.00000000033	1464700000\\
457000	1220900000\\
-329000.000000001	-3906400000\\
-220000.000000002	2319300000\\
512000	1709100000\\
-144999.999999999	-3173800000\\
-404000.000000003	366000000.000002\\
347999.999999999	3296100000\\
-17999.9999999989	-4272500000\\
-164000	2685500000\\
126000	-732400000\\
19999.9999999987	-122099999.999999\\
-183000.000000002	244300000\\
73000.0000000013	-366500000\\
126999.999999998	610500000\\
-90000.0000000007	-488100000\\
109000	243800000.000001\\
-109000.000000001	-243900000\\
16999.9999999986	-122100000.000001\\
110000.000000002	976500000.000001\\
57000.0000000022	-610400000\\
-259000.000000005	-1830900000\\
129000.000000002	3539900000\\
-71999.9999999974	-3173700000\\
-74000.0000000025	1708800000\\
310000	610500000.000001\\
-218000	-2075100000\\
72999.9999999995	1830700000\\
-19000.0000000019	-1220300000\\
-238000	-244400000\\
385000.000000001	2807600000\\
-146999.999999999	-3905900000\\
1000.00000000122	2685000000\\
126999.999999999	-487700000.000002\\
-91000.0000000037	-1221300000\\
-109999.999999999	732999999.999999\\
-274000	-122500000.000001\\
455999.999999999	1465100000\\
-198999.999999997	-2441600000\\
180999.999999996	2563700000\\
-181999.999999999	-2685900000\\
-182999.999999998	1587400000\\
365999.999999999	365800000\\
126999.999999999	488500000\\
-235999.999999997	-3784200000\\
-2000.00000000333	4882600000\\
112000.000000003	-2929500000\\
-221000.000000003	199999.999998113\\
17999.999999998	1464400000\\
202000.000000003	-732200000\\
-202000.000000003	-488099999.999999\\
19000.000000001	609999999.999999\\
-17999.9999999989	-121800000\\
236999.999999997	1342600000\\
-35999.9999999978	-2807500000\\
-202000.000000003	1708900000\\
257000.000000001	732599999.999999\\
-274999.999999996	-2808000000\\
165999.999999997	3662600000\\
-75000.0000000019	-3174300000\\
-146000	1099100000\\
201999.999999997	1098200000\\
365000.000000001	1465100000\\
-620999.999999999	-7690400000\\
126999.999999998	10009500000\\
1000.00000000122	-7690200000\\
126999.999999998	5004900000\\
148000.000000003	-1831100000\\
-386000.000000004	-1709200000\\
36999.9999999999	1465200000\\
36999.9999999999	732100000\\
367000.000000003	1465200000\\
-112000.000000002	-3906700000\\
-54000.000000002	2564000000\\
-218999.999999996	-2564000000\\
91999.9999999987	3052100000\\
271999.999999998	122000000.000002\\
-289999.999999997	-3784300000\\
16999.9999999995	3662400000\\
218999.999999999	-366600000\\
-199999.999999998	-2197100000\\
-129000.000000001	610599999.999998\\
348000.000000003	3173400000\\
-128000	-4150100000\\
-291999.999999999	1220600000\\
381999.999999995	2441500000\\
-217999.999999999	-4150600000\\
37000.0000000008	4028600000\\
16999.9999999986	-3174200000\\
-89999.9999999963	2075700000\\
309999.999999998	-122500000\\
-146000.000000003	-1464700000\\
-183999.999999997	244200000\\
94000.0000000021	1464700000\\
-3000.00000000455	-976299999.999999\\
185000.000000003	610000000\\
-129000.000000003	-1220400000\\
-255999.999999997	366099999.999999\\
385000.000000001	1708900000\\
-1000.00000000122	-1464800000\\
-438999.999999999	-2197200000\\
422000.000000001	5493200000\\
52999.9999999973	-4760900000\\
-162999.999999999	1587100000\\
-19000.000000001	365900000\\
35999.9999999978	-854000000\\
-419999.999999999	-366700000.000001\\
493999.999999998	3052000000\\
91000.0000000002	-3295800000\\
-181999.999999999	1220400000\\
-147999.999999999	-1098400000\\
-91000.0000000019	2075200000\\
294000.000000001	-1220900000\\
-73999.9999999998	-243900000\\
0	610200000\\
109999.999999999	244100000.000001\\
-129000.000000001	-1342500000\\
239000.000000001	2196900000\\
-36999.9999999999	-2197000000\\
-512000.000000002	-732499999.999999\\
273999.999999999	4150200000\\
73000.0000000013	-4516200000\\
-256000	2807300000\\
383999.999999998	-610300000.000001\\
-292000	-1342600000\\
457000.000000002	3905900000\\
-292000.000000002	-6103100000\\
-257999.999999999	4882600000\\
441000.000000002	-1220900000\\
-74000.0000000025	-1098100000\\
-146000	609700000.000001\\
-183000.000000002	-121500000.000001\\
108000.000000001	976199999.999999\\
2999.99999999923	-1708700000\\
382000	3661700000\\
-219000.000000001	-5736900000\\
-218999.999999998	4150200000\\
91000.0000000002	-1220700000\\
1000.00000000122	100000.000000477\\
-112000.000000002	-488400000\\
404000.000000001	3173900000\\
-181999.999999999	-4760800000\\
88999.9999999986	3540100000\\
-70999.9999999971	-2441500000\\
-293000	488500000.000001\\
455999.999999997	2319000000\\
-493000.000000001	-2807300000\\
238000	-122200000\\
35999.9999999987	3784100000\\
-256000.000000001	-3906100000\\
238000.000000001	1220500000\\
109999.999999998	854800000\\
-165999.999999999	-1587200000\\
240000.000000003	2441500000\\
91000.0000000002	-1709000000\\
-403000.000000004	-2197100000\\
36000.0000000014	4150200000\\
-18000.0000000042	-3295900000\\
55000.0000000024	2319500000\\
-37000.0000000008	-610599999.999999\\
349000.000000003	732699999.999999\\
-110000	-2197600000\\
-368000.000000001	610799999.999999\\
149000.000000002	1098200000\\
253999.999999997	976800000.000001\\
-107999.999999996	-2929700000\\
-92000.0000000014	1342700000\\
71999.9999999965	854499999.999999\\
-144999.999999998	-1709000000\\
108999.999999998	2075300000\\
-236999.999999999	-2685600000\\
108000.000000002	3051700000\\
239999.999999998	-1464800000\\
-1000.00000000211	2.66453525910038e-07\\
-110000	-854499999.999999\\
184000.000000003	2075300000\\
-148000	-2197500000\\
-346000.000000001	488500000\\
52999.9999999999	1220600000\\
184999.999999999	-488300000\\
126999.999999996	-610199999.999999\\
2.66453525910038e-09	365900000\\
-109000.000000002	244600000\\
-1000.00000000211	-488700000.000001\\
-91999.9999999961	-243999999.999999\\
368000	2197400000\\
-220999.999999997	-3296100000\\
36999.9999999999	2929800000\\
220000	-1587000000\\
-404000.000000003	-1220600000\\
-16999.9999999986	2075200000\\
72999.9999999995	-732500000\\
146000.000000001	1342800000\\
73999.9999999998	-2441400000\\
-166000.000000002	1831000000\\
54999.9999999997	-1220600000\\
-327999.999999998	488199999.999999\\
365000	1342900000\\
219999.999999999	-1587200000\\
-328999.999999999	-487900000\\
-239000.000000001	854200000\\
201000	488400000\\
275999.999999998	-99999.9999995893\\
-348999.999999999	-1220400000\\
258000	1342300000\\
-240000.000000003	-1586500000\\
-16999.9999999995	1220500000\\
438000.000000001	1464800000\\
-438000	-4394300000\\
18000.0000000033	3906000000\\
255999.999999998	-488100000\\
37000.0000000017	-244300000.000002\\
8.88178419700125e-10	-1708900000\\
-202000.000000003	1098700000\\
92000.0000000005	366000000.000002\\
-311000	-1342400000\\
346999.999999998	3295500000\\
-310000	-4394400000\\
309999.999999998	4150600000\\
74000.0000000025	-2808000000\\
-367000.000000003	122399999.999999\\
476999.999999999	2441300000\\
-548999.999999999	-3173900000\\
419999.999999999	1831100000\\
17999.9999999998	244200000\\
-363999.999999999	-1464900000\\
272000.000000002	1342800000\\
-145000.000000004	-732500000.000001\\
92000.0000000014	488400000.000001\\
90999.9999999993	-366400000\\
-275999.999999999	299999.999999834\\
93999.9999999994	-122300000\\
144000	854600000\\
-53000.0000000026	-488500000\\
-202000.000000001	-1464500000\\
146999.999999999	2929400000\\
290999.999999999	-1830800000\\
-254000	-366499999.999999\\
-221000	244300000\\
219999.999999998	1098700000\\
201000.000000002	366100000.000001\\
-201000.000000002	-3417900000\\
257000.000000002	5004800000\\
-239000.000000003	-5126900000\\
-438999.999999998	2075300000\\
292000	2074800000\\
313000.000000001	-1952500000\\
-312000.000000003	-610900000.000001\\
292000.000000002	2319500000\\
1000.00000000033	-1586700000\\
-110000	-1098900000\\
-183000	1098800000\\
37000.0000000008	1098500000\\
199999.999999997	-1098600000\\
38000.0000000002	-122000000\\
-330999.999999998	-610400000\\
184999.999999999	2685600000\\
34999.9999999993	-3174000000\\
56000	1465200000\\
-384999.999999996	-500000.000000256\\
218999.999999999	-121600000\\
220999.999999997	488000000\\
-17999.9999999998	-488200000.000001\\
-386999.999999998	-1464800000\\
185999.999999999	2685500000\\
438000	122000000.000002\\
-531000.000000002	-4272400000\\
109999.999999999	4760900000\\
257000.000000001	-1221000000\\
-20000.0000000005	-1830800000\\
-436999.999999999	610200000.000001\\
125999.999999999	1831100000\\
421999.999999999	-1098700000\\
-420999.999999999	-1342500000\\
110000.000000001	2685200000\\
200000.000000003	-2563400000\\
-309000.000000003	1343000000\\
-111999.999999999	-366400000\\
257999.999999999	244100000\\
108999.999999997	488400000\\
-183000.000000001	-1342900000\\
19000.0000000019	854700000.000001\\
199999.999999999	732100000\\
-17000.0000000003	-1830600000\\
-238999.999999995	1098100000\\
-35999.9999999996	-365700000\\
73000.0000000013	854100000\\
36999.9999999999	-976399999.999999\\
-256999.999999999	-244100000\\
475999.999999996	2197200000\\
-16999.9999999968	-2075100000\\
-203000.000000005	-244400000\\
-108999.999999998	610599999.999999\\
74000.0000000016	610299999.999999\\
88999.9999999977	-122099999.999999\\
39000.000000005	-732300000.000001\\
90999.9999999993	976200000.000001\\
-256000.000000003	-1708500000\\
35000.000000001	1586600000\\
111000.000000001	-610200000\\
-128000.000000001	121800000\\
-35999.9999999978	-121700000.000001\\
272999.999999996	1220600000\\
-127999.999999999	-2075500000\\
20000.0000000022	1221100000\\
-93000.0000000044	-610500000\\
8.88178419700125e-10	488199999.999999\\
-35999.9999999987	-488200000\\
292999.999999998	2197300000\\
-349000	-4394600000\\
2000.00000000244	3906200000\\
144999.999999995	-1464600000\\
110000	1098300000\\
-220000	-3051500000\\
184000	3784000000\\
-74000.0000000025	-2441200000\\
92000.0000000023	1220500000\\
-348000	-1831000000\\
54999.9999999997	2075300000\\
439999.999999998	732300000.000001\\
-56999.9999999986	-2319300000\\
-419000	-732399999.999998\\
438999.999999999	3417900000\\
-275000	-2441300000\\
-183000	244200000\\
384000.000000001	1220400000\\
-128000.000000002	-1952900000\\
-127000	2075300000\\
218999.999999999	-1221000000\\
-293999.999999996	-243800000\\
312999.999999997	732000000\\
-19999.9999999969	366700000\\
-346000.000000002	-1465300000\\
273000.000000002	1221000000\\
-109000	-366200000\\
219000	-122400000\\
-201000.000000003	610800000.000001\\
1000.00000000033	-1343100000\\
125999.999999999	1953200000\\
-89000.0000000013	-2075000000\\
-258999.999999999	487899999.999998\\
625000.000000003	3052100000\\
-313000.000000004	-4882900000\\
-144999.999999997	2563300000\\
235999.999999996	854800000\\
39000.000000005	-1831300000\\
-515000.000000002	-854400000.000001\\
459000	5126900000\\
54999.999999997	-6103500000\\
-456999.999999998	2685700000\\
346000.000000001	365900000.000001\\
-54000.0000000011	-121700000\\
165000	-732700000.000001\\
-73999.9999999981	732400000\\
-127000.000000001	-1220400000\\
-110999.999999998	1342500000\\
-54000.0000000003	-1098500000\\
328999.999999998	1953000000\\
-110000.000000001	-2319300000\\
-145000	976700000.000001\\
70999.9999999997	244000000\\
294000	488300000\\
-346999.999999998	-2319300000\\
52999.9999999964	2563500000\\
222000.000000003	-366299999.999999\\
-149000.000000003	-1708900000\\
-308999.999999997	366200000\\
565999.999999997	3784100000\\
-346999.999999998	-6591800000\\
146000.000000003	6470000000\\
-127000.000000002	-5127300000\\
17000.0000000003	2929700000\\
310999.999999997	732800000.000002\\
-181999.999999999	-2808000000\\
-239000.000000002	244200000\\
-89999.9999999998	1953300000\\
199999.999999998	-488500000\\
146000.000000001	-243900000\\
20000.0000000022	-122200000\\
53999.9999999976	121900000\\
-439000.000000001	-1220300000\\
-56000.0000000009	1220200000\\
349000.000000003	610900000\\
127999.999999998	-122499999.999999\\
109000.000000002	-854400000.000001\\
-310000.000000002	-366000000.000002\\
-220000.000000001	-199999.999998113\\
126999.999999999	610299999.999999\\
220999.999999999	1587100000\\
-19000.0000000001	-2441400000\\
109999.999999999	1586600000\\
-383999.999999998	-2685100000\\
163999.999999997	3539600000\\
147000.000000002	-1464400000\\
-274999.999999999	-1709400000\\
312000.000000001	3906500000\\
-56000.0000000009	-3418000000\\
-219000	-100000.000000122\\
220000.000000002	3051900000\\
163999.999999998	-2441600000\\
-403000.000000001	-732099999.999999\\
-71999.9999999992	1342300000\\
199999.999999998	610900000\\
111000.000000004	-732900000\\
-8.88178419700125e-10	366499999.999999\\
-239000.000000003	-1464900000\\
202000.000000003	2197100000\\
72999.9999999995	-1464500000\\
-311000	-366700000.000001\\
238000	1587400000\\
-201000	-1953400000\\
218000	2929800000\\
112000.000000002	-2563600000\\
-146999.999999999	732600000\\
71999.9999999965	-610600000\\
-401000.000000001	610599999.999999\\
364000	366200000\\
-16000.0000000009	243799999.999999\\
-93000	-1586400000\\
-127999.999999999	609999999.999999\\
109999.999999998	1464800000\\
147000	-1586600000\\
-55999.9999999991	1098400000\\
221000.000000001	-488300000\\
-349000.000000005	-1098500000\\
111000.000000004	854399999.999999\\
-148000	854600000.000001\\
-54000.0000000029	-1587100000\\
275000.000000001	1831300000\\
-19000.0000000019	-1099000000\\
-293000	-854000000\\
293999.999999999	1708500000\\
-19999.9999999978	400000.000001555\\
-35000.000000001	-1831300000\\
-38000.0000000011	1342700000\\
258000.000000003	1587300000\\
-221000.000000003	-4883200000\\
-74000.0000000007	4638900000\\
-15999.9999999991	-2197500000\\
-57000.0000000004	854699999.999999\\
240000.000000002	366300000\\
34999.9999999975	-854800000\\
-220000.000000001	-366099999.999999\\
-181999.999999999	122300000.000001\\
275000	1708700000\\
126999.999999998	-976499999.999999\\
-347999.999999999	-1952900000\\
276000.000000002	3173400000\\
164000.000000001	-365900000\\
-146000.000000002	-2807500000\\
-111000.000000002	1342300000\\
-54999.9999999997	1099200000\\
-217999.999999998	-1831600000\\
475000	2930200000\\
-219000.000000001	-2685900000\\
-999.999999999446	488300000.000001\\
72999.9999999968	488600000\\
-17000.0000000003	854000000.000001\\
-37999.9999999985	-2318900000\\
165999.999999999	3051500000\\
-257000.000000002	-3662000000\\
201000	3296000000\\
-90000.0000000007	-1953500000\\
-221000.000000001	399999.999999423\\
164999.999999999	1098500000\\
146000	366100000.000001\\
203000	-732200000\\
-826000.000000001	-3052000000\\
770000.000000001	7568500000\\
56000.0000000027	-7324300000\\
-222000.000000005	3051900000\\
-363999.999999995	-1587000000\\
291999.999999997	2685500000\\
72999.9999999977	-1708900000\\
-183000	-122200000\\
292999.999999998	976800000\\
-366000	-1099000000\\
-73000.0000000013	-976100000\\
438999.999999999	4272000000\\
91000.0000000002	-2563100000\\
-401999.999999998	-3418100000\\
74000.0000000007	5248800000\\
70999.9999999979	-2319000000\\
112000.000000003	732200000\\
-165000	-1708800000\\
107999.999999998	2075100000\\
-291000.000000003	-1587100000\\
237000.000000001	1099000000\\
-201000.000000001	-488500000\\
274000.000000002	488200000\\
-126999.999999998	-976400000\\
256000	2807700000\\
-111000.000000001	-3540400000\\
-107999.999999999	122400000.000002\\
-130000.000000002	1831000000\\
203000.000000001	-99999.9999997669\\
52999.9999999973	-854400000\\
-271999.999999996	-244199999.999999\\
88999.9999999942	610299999.999999\\
130000.000000002	732600000\\
35000.0000000001	-1953300000\\
-383000.000000002	1464900000\\
219000.000000001	-610200000\\
273999.999999999	1342400000\\
-162999.999999998	-1708600000\\
-147999.999999998	244000000\\
-237999.999999999	-244200000\\
366999.999999995	1342800000\\
421000.000000002	1342800000\\
-642000.000000001	-5126900000\\
111999.999999999	3662000000\\
126000.000000001	610500000\\
-127000.000000003	-2441700000\\
128000.000000002	2075600000\\
72999.9999999986	-854800000\\
-91000.0000000002	-732399999.999999\\
-91999.9999999996	1343000000\\
-91999.9999999978	-1465000000\\
-54000.0000000038	976500000\\
293000.000000002	854500000\\
144999.999999998	-732200000\\
-181000	-1098900000\\
-19999.9999999978	854699999.999999\\
-183000.000000001	-610600000\\
-127000.000000001	1709200000\\
347999.999999998	-1831100000\\
16000.0000000009	1586900000\\
94000.0000000003	-976499999.999999\\
-257000.000000001	-366500000\\
35999.9999999996	244500000\\
-110000	244000000\\
93000.0000000017	122099999.999999\\
71999.9999999974	610300000\\
238000	-488300000\\
-127999.999999997	-610300000\\
-273999.999999998	-854500000.000001\\
109999.999999998	1953100000\\
34999.9999999975	-366100000\\
129000.000000001	-488500000\\
-254999.999999999	-243900000\\
52999.9999999982	366100000\\
129000	488200000\\
90999.9999999993	-366100000.000001\\
-109000	-610400000.000001\\
-256000	366300000\\
108000	-122199999.999999\\
19999.9999999987	854600000\\
201000.000000002	-854600000.000001\\
-91999.9999999996	732600000.000001\\
91999.9999999987	-732699999.999999\\
-146000.000000001	122299999.999999\\
-111000.000000001	-610500000\\
73000.0000000004	1221000000\\
331000.000000002	1342300000\\
-73999.9999999989	-2929200000\\
-347000	-854899999.999998\\
52999.9999999982	3540200000\\
75000.0000000037	-2197100000\\
310000	2685300000\\
-16999.9999999986	-3173700000\\
-385000.000000002	-244300000.000002\\
-56000.0000000027	1465200000\\
240000.000000004	1098200000\\
52999.9999999973	-1586600000\\
-345999.999999998	-244200000\\
125999.999999999	976300000\\
19999.9999999996	-243700000\\
236999.999999998	121700000\\
-54000.0000000003	-366000000.000001\\
-109999.999999999	-199999.999999712\\
-74000.0000000007	-121699999.999999\\
54999.9999999988	243700000\\
55999.9999999991	610600000\\
54000.0000000003	-854600000\\
54999.9999999988	244399999.999999\\
-128000.000000001	-400000.000000844\\
-53999.9999999994	-487999999.999999\\
-166999.999999999	488200000\\
276999.999999998	610300000\\
108000	-488199999.999999\\
-72000.0000000001	-488200000\\
-366999.999999997	-244400000\\
439999.999999999	1342900000\\
-219999.999999998	-243999999.999999\\
-37000.0000000008	-1831200000\\
-17000.0000000003	2319400000\\
217999.999999997	-1587100000\\
111000.000000003	2075400000\\
-476000.000000001	-3417900000\\
293000.000000001	2929400000\\
144999.999999996	-854199999.999999\\
-180999.999999996	-244399999.999999\\
-240000.000000004	-732200000\\
203000.000000003	1342700000\\
164000	732400000.000001\\
-128000	-2685700000\\
129000.000000001	2563800000\\
-148000.000000001	-1831200000\\
-2.66453525910038e-09	732200000\\
-107999.999999998	-365800000\\
254000	1464600000\\
-254000.000000001	-2075300000\\
181000	1221000000\\
-16999.9999999995	121800000\\
-367000	-2319200000\\
642000.000000003	5981400000\\
16999.9999999995	-5248900000\\
-383000.000000001	-1098800000\\
-130000.000000002	2807700000\\
55999.9999999983	243999999.999999\\
72999.9999999995	-854200000\\
74000.0000000007	366000000\\
-20000.0000000005	-244200000\\
74999.9999999993	-243900000.000002\\
89999.999999999	1342500000\\
-235999.999999999	-2441200000\\
-21000.0000000017	976500000.000001\\
94000.0000000012	1342700000\\
236999.999999998	-244000000\\
-275000	-2441500000\\
-274000	2075100000\\
384000	244400000.000001\\
-110000.000000001	-1220900000\\
-35999.9999999996	732499999.999999\\
238000.000000001	854300000.000002\\
-36999.9999999999	-1098200000\\
-145999.999999997	-977100000\\
-202000.000000002	732799999.999999\\
202000.000000003	1342800000\\
-55000.0000000033	-976899999.999999\\
146000.000000002	-243900000.000001\\
-54999.9999999997	366200000\\
127999.999999998	976500000.000001\\
-126999.999999997	-2685400000\\
183000	2929500000\\
-203000.000000005	-2563500000\\
-420000	122300000.000001\\
385000.000000001	2807500000\\
438000.000000001	-366399999.999999\\
-731000.000000002	-4150000000\\
237000	2807300000\\
166000.000000004	2441400000\\
273000	-2197000000\\
-255000.000000001	-2075400000\\
-92000.0000000014	2075200000\\
-310999.999999999	-1098500000\\
328999.999999997	1830900000\\
-18000.0000000007	-488300000.000001\\
165000	-1098400000\\
-330000.000000001	610099999.999999\\
238000	100000.000000833\\
-201000.000000002	610400000.000001\\
238000.000000003	-1831100000\\
-202000.000000003	1953100000\\
-89999.999999999	-610400000.000001\\
363999.999999999	122300000\\
-273000.000000001	-976800000\\
54000.0000000003	488300000\\
147000.000000001	2075300000\\
164999.999999998	-2685600000\\
-422000.000000001	-122000000.000001\\
18999.9999999983	1098500000\\
293000.000000002	1220800000\\
-202000.000000003	-2319400000\\
111000.000000002	732499999.999999\\
-331000.000000002	366100000.000001\\
366999.999999999	488500000.000001\\
73000.0000000004	-854800000\\
-110000.000000001	244400000\\
-218999.999999998	-488400000\\
-110000.000000001	610400000\\
219000.000000002	-244200000\\
145999.999999999	1220600000\\
93999.9999999985	-1708600000\\
-350999.999999998	487799999.999999\\
1999.999999998	-854099999.999999\\
147000.000000001	2441100000\\
198999.999999997	-976200000\\
-143999.999999997	-2441900000\\
-203000.000000003	2930200000\\
109999.999999999	-854900000.000001\\
148000.000000003	-121800000\\
-167000.000000003	-366299999.999999\\
75000.000000001	854400000\\
126999.999999999	-122000000\\
-164000.000000001	-1220600000\\
-146999.999999998	244099999.999998\\
237999.999999998	2319100000\\
-108999.999999998	-3173600000\\
-74000.0000000016	1709100000\\
-36999.9999999999	121699999.999998\\
459000.000000001	610700000.000002\\
-294000	-1953300000\\
-238000.000000003	-243999999.999999\\
421000.000000002	3539700000\\
-34999.9999999984	-2685100000\\
-167000.000000002	-976900000.000001\\
-53000.0000000017	1587100000\\
-56999.9999999995	244000000\\
149000	-610100000\\
-75999.9999999969	121800000\\
-236000.000000005	-854400000\\
732000.000000002	2563400000\\
-532000.000000002	-2197000000\\
-181999.999999999	-732700000.000001\\
201999.999999998	2319400000\\
236000.000000002	-854400000\\
-54000.0000000029	-976600000.000001\\
-71999.9999999974	1220600000\\
-350000	-1342600000\\
92999.9999999991	976400000\\
274999.999999998	610400000\\
162999.999999999	-244100000.000001\\
-144999.999999999	-976500000\\
-221000.000000002	-488399999.999999\\
130000.000000002	1953100000\\
-38000.000000002	-1342700000\\
1000.000000003	122100000\\
-1000.00000000389	854500000\\
128000	-610499999.999999\\
999.999999999446	122199999.999999\\
-999.999999999446	-1.77635683940025e-06\\
-108999.999999999	-1464900000\\
-73000.0000000013	2075200000\\
-56000.0000000009	-854399999.999999\\
404000.000000002	854400000\\
-294000.000000001	-732500000\\
-274999.999999999	-1952900000\\
367999.999999998	3906100000\\
89999.999999999	-1831100000\\
-238000	-1220500000\\
1000.000000003	1342500000\\
493999.999999998	1098900000\\
-751000.000000001	-2930000000\\
311000	1099100000\\
292999.999999998	2807000000\\
-401999.999999995	-4393900000\\
201000	2929200000\\
-365999.999999999	-2074900000\\
494000.000000001	3417800000\\
146000	-1830900000\\
-565999.999999998	-3906400000\\
162999.999999998	5981500000\\
166000.000000002	-2319300000\\
-148000.000000001	-976699999.999999\\
130000.000000002	1587200000\\
-184000.000000001	-1953400000\\
238999.999999999	3051800000\\
53000.0000000026	-1342600000\\
-108000.000000001	-3174000000\\
-239000	3540000000\\
145999.999999998	-365899999.999998\\
74999.9999999993	-488700000.000001\\
-331000	-976300000\\
238000	1709000000\\
329999.999999997	488000000.000002\\
-310999.999999997	-2318900000\\
-56000.0000000018	365900000.000001\\
203000.000000003	2075200000\\
-257000.000000001	-2197100000\\
310999.999999997	976399999.999999\\
-293999.999999997	366300000.000001\\
56999.9999999968	-1342800000\\
-56000	732500000\\
201999.999999999	2197100000\\
-18999.9999999992	-4272300000\\
-36999.9999999981	3173700000\\
-53000.0000000008	-1220500000\\
-240999.999999999	-366500000\\
441999.999999998	1465000000\\
-219999.999999999	-732400000\\
-276000	-1342700000\\
312000.000000001	2441200000\\
218999.999999998	-1342700000\\
-70999.9999999988	122100000\\
-277000.000000001	-854400000\\
-91000.0000000011	1220500000\\
697000.000000003	976800000\\
-386000.000000001	-1831300000\\
-200999.999999997	-1708700000\\
73000.0000000004	4760300000\\
-164000.000000001	-4149800000\\
17999.9999999998	1708400000\\
182999.999999996	732900000\\
330000.000000002	-976900000.000001\\
-569000.000000001	244300000\\
111999.999999999	-2441400000\\
345999.999999998	6347600000\\
-90999.9999999984	-5981500000\\
54999.9999999979	1587100000\\
-421000	122000000.000002\\
37000.0000000017	243899999.999999\\
255999.999999999	854900000\\
238000	-488699999.999999\\
-147000.000000002	-365800000.000002\\
-493000	-2563700000\\
419000	5248900000\\
185000	-3417700000\\
-311999.999999999	488200000\\
-329000	-244300000\\
54000.0000000003	244300000\\
658999.999999997	2319300000\\
-89999.9999999972	-2441400000\\
-130000.000000004	-1098700000\\
-236999.999999998	1831200000\\
439999.999999999	-244299999.999999\\
-641999.999999997	488400000\\
220000.000000001	-1220700000\\
36999.9999999999	-122300000\\
-36999.9999999981	1709400000\\
90999.9999999984	-977100000\\
130000	599999.999999934\\
-20000.0000000022	243700000\\
-91000.0000000011	-854400000\\
36999.9999999981	976700000\\
-18999.9999999992	-366300000.000001\\
36999.9999999999	-122200000.000002\\
-165000.000000004	-365900000\\
-91999.9999999961	487999999.999999\\
404000.000000002	1709200000\\
109999.999999999	-2319700000\\
-185000.000000001	-121599999.999999\\
-749000.000000001	-366499999.999999\\
696000.000000002	1220800000\\
89999.9999999972	2929500000\\
54999.9999999997	-4638300000\\
-327999.999999998	121800000.000001\\
-221000.000000002	2319300000\\
257000.000000001	-976400000\\
257000.000000001	1342700000\\
-149000.000000001	-2319300000\\
-15999.9999999991	1830900000\\
146000.000000003	-1220400000\\
-238000.000000002	487899999.999999\\
365000	488600000.000001\\
-400999.999999999	-1342900000\\
-202000.000000003	610400000.000002\\
310000	732299999.999999\\
37999.9999999994	244200000\\
1.77635683940025e-09	-1830900000\\
126999.999999998	2441100000\\
-345999.999999998	-3295600000\\
198999.999999997	3662000000\\
222000.000000001	-1464900000\\
-203000.000000001	-1831100000\\
-108000.000000001	2075400000\\
-186000.000000001	-488400000\\
369000.000000002	366100000\\
35000.0000000001	122300000\\
-438999.999999999	-1953300000\\
55000.0000000006	1709000000\\
604999.999999999	2685700000\\
-294999.999999998	-7324400000\\
-254000.000000002	6836000000\\
-19999.9999999969	-3051800000\\
512999.999999996	1709100000\\
-17999.9999999998	-1098600000\\
-311000	-1953400000\\
-494999.999999998	3052000000\\
293999.999999998	-2807700000\\
401999.999999999	4882900000\\
-220000	-5127200000\\
92000.0000000014	2319700000\\
108999.999999999	854300000.000003\\
94000.0000000012	-1465000000\\
-369000	-1708700000\\
-90000.0000000016	2319200000\\
347000.000000001	1464800000\\
-108999.999999998	-2929600000\\
-129000.000000001	610299999.999999\\
93000.0000000008	610500000\\
16000	732099999.999998\\
-198999.999999997	-2441200000\\
492999.999999998	4760900000\\
-109999.999999998	-4394900000\\
-146000.000000003	-732000000.000001\\
-145999.999999998	2440900000\\
-166000.000000004	-121500000\\
257000.000000001	-122600000\\
1000.00000000122	400000.000000134\\
-222000.000000002	-1221000000\\
442000.000000002	2197500000\\
-222000.000000003	-1220700000\\
-145000	-1343100000\\
475000	3418300000\\
-474999.999999999	-4516700000\\
-55000.0000000024	4028300000\\
254999.999999999	-1953000000\\
203000.000000002	1342500000\\
-202000.000000002	-2319200000\\
-294000	732700000\\
257999.999999998	1830500000\\
-403000	-1586400000\\
438000.000000001	-488600000.000001\\
240000.000000001	3174000000\\
-313000.000000003	-3906300000\\
-53999.9999999976	1831000000\\
-18999.9999999992	-976600000.000002\\
275000	2929900000\\
92000.0000000023	-2807800000\\
-238000.000000002	-610400000.000001\\
35999.9999999996	976800000.000001\\
-275000.000000001	488100000.000001\\
38000.0000000002	99999.9999994117\\
71999.9999999983	-122300000.000001\\
256999.999999998	732699999.999999\\
-109999.999999999	-2807700000\\
-202000.000000003	3417900000\\
129000.000000004	-2075100000\\
36999.999999999	732200000\\
-93000.0000000035	366600000\\
239000.000000001	-610600000\\
90999.9999999993	488200000\\
-329000	-1586800000\\
-8.88178419700125e-10	1342900000\\
-36999.9999999964	854200000\\
18000.0000000007	-1830800000\\
-72000.0000000018	854500000\\
292000.000000001	488000000\\
-146999.999999998	-366000000\\
-127000.000000005	-610300000\\
421000.000000002	1342600000\\
-496000	-1952900000\\
39000.0000000041	365900000\\
548000	4150800000\\
-292999.999999998	-5859900000\\
-200000.000000002	1587500000\\
70999.9999999988	2196800000\\
330000.000000003	-121900000.000002\\
-328000	-3540000000\\
164000.000000001	3418000000\\
-257000.000000004	-2075300000\\
-254999.999999998	732399999.999999\\
309999.999999998	1098700000\\
275000.000000002	122099999.999997\\
90999.9999999984	-366200000.000001\\
-217999.999999996	-2563600000\\
16999.9999999995	2441500000\\
-184000.000000001	-1098700000\\
-70999.9999999988	732500000\\
530000.000000001	2075300000\\
-385000.000000003	-4517000000\\
-35999.9999999996	2441900000\\
-238000	-244400000\\
419999.999999998	1220600000\\
257999.999999999	-365900000\\
-514000	-3052100000\\
164999.999999999	3540200000\\
-310000	-1830900000\\
199999.999999998	488099999.999998\\
274999.999999999	1830900000\\
-36999.9999999964	-2441000000\\
-90000.0000000007	121700000\\
-203000.000000003	732700000\\
201999.999999999	121900000\\
55000.0000000006	-99999.9999994117\\
-129000.000000002	-1220400000\\
-16999.9999999986	1708800000\\
71999.9999999974	-976599999.999999\\
-144999.999999999	488300000\\
-129000.000000001	-976300000\\
237000.000000001	1220200000\\
148000.000000001	122500000\\
54999.9999999997	-488399999.999999\\
-258000.000000003	-854599999.999999\\
57000.0000000004	610499999.999999\\
-148000	-99999.9999997669\\
73999.9999999989	366200000.000001\\
146000.000000001	-1.77635683940025e-07\\
-18000.0000000007	122300000\\
128000	-610799999.999999\\
-548999.999999999	-1586400000\\
273999.999999999	4394100000\\
459000.000000001	-2807300000\\
-93000	-300000.0000009\\
-512000.000000002	-1952800000\\
-91000.0000000002	4272100000\\
456000.000000001	-2318900000\\
-16000	731899999.999999\\
-999.999999999446	-1342200000\\
-385999.999999999	243599999.999999\\
167000.000000001	1221100000\\
456999.999999998	1342600000\\
-72999.9999999968	-3540000000\\
-587000.000000001	-122200000.000001\\
-72000.0000000009	3418300000\\
457000	-1953600000\\
328999.999999999	1587400000\\
-436999.999999998	-2930100000\\
-332000.000000001	854800000\\
366999.999999997	2075100000\\
219000.000000003	-610499999.999999\\
-365000	-2685400000\\
110000.000000002	2929800000\\
384000	732099999.999999\\
-349000.000000003	-4028000000\\
-254999.999999998	1708800000\\
458000.000000001	3295900000\\
-349000.000000004	-5248900000\\
55000.0000000006	3417800000\\
294000.000000003	732599999.999998\\
72999.9999999995	-2563500000\\
-678000.000000003	-1465100000\\
128000.000000001	4150700000\\
440000	-366300000\\
91000.0000000002	-1098700000\\
-219000.000000001	-1831000000\\
-220999.999999998	732399999.999999\\
332000.000000001	3295900000\\
-3000.00000000011	-3173900000\\
-254000	-732200000\\
-37000.0000000026	2197100000\\
127000	-488400000\\
73999.9999999981	244500000\\
-91999.9999999996	-1465300000\\
147000.000000001	2441900000\\
-127999.999999997	-3662600000\\
-183000.000000003	3418300000\\
492999.999999999	-122199999.999999\\
-254999.999999999	-2441200000\\
-128000.000000001	1464400000\\
164000	122499999.999999\\
-202000.000000002	-854700000\\
-90000.0000000007	488400000.000001\\
329000.000000002	1586800000\\
-129000.000000002	-2807600000\\
-52999.9999999973	1586900000\\
88999.9999999977	488500000.000001\\
185000.000000003	-488700000\\
-147000.000000003	-1342400000\\
146000.000000002	1953100000\\
-493000	-2441700000\\
-39000.0000000015	2075400000\\
717000	1709000000\\
-186000.000000002	-3417900000\\
-602000	-854700000\\
273999.999999999	3540100000\\
292000.000000001	200000.000000955\\
-255000	-3662500000\\
36000.0000000005	3052200000\\
366000.000000003	976300000.000001\\
-237000.000000002	-4028200000\\
-294999.999999999	2074900000\\
460000.000000002	1831500000\\
-239000.000000001	-3052000000\\
-165000.000000001	732300000\\
73000.0000000013	976899999.999999\\
129999.999999999	854200000.000001\\
15999.9999999982	-2075100000\\
91999.9999999996	2319400000\\
-127000	-4150400000\\
-74000.0000000007	4394300000\\
-110000.000000003	-2807300000\\
220000.000000001	2563400000\\
35999.9999999996	-2197500000\\
-419999.999999999	-365900000\\
438999.999999999	2685400000\\
164999.999999999	-854499999.999999\\
-330999.999999999	-2441400000\\
-126000.000000001	976600000.000002\\
419000	3540000000\\
-72000.0000000027	-4760700000\\
-457000	1831100000\\
163000.000000001	610100000\\
256999.999999997	-243800000\\
1000.000000003	-122400000\\
-37999.9999999985	-976200000\\
-365999.999999999	854300000\\
110999.999999998	854300000\\
108000.000000001	-1952800000\\
148000.000000001	2807500000\\
-109999.999999999	-3051900000\\
36999.999999999	1343100000\\
345999.999999997	2319000000\\
-345999.999999997	-5615100000\\
-514000.000000004	3540000000\\
459000.000000003	854700000\\
365000	976300000\\
-291999.999999998	-5004900000\\
-257000.000000001	2319600000\\
385000.000000002	4150300000\\
-37000.0000000008	-5737600000\\
-108999.999999997	2197700000\\
-111000.000000003	-366500000.000001\\
-91000.0000000028	366300000\\
420000.000000002	1709000000\\
-217000	-3540000000\\
-314000.000000001	1220600000\\
276000.000000001	2563500000\\
146000	-2319300000\\
256000	1342700000\\
-309999.999999999	-2685300000\\
-715000	365899999.999999\\
823999.999999999	3784300000\\
238000	-2441400000\\
-547999.999999998	-1342700000\\
-203999.999999999	1464700000\\
202999.999999996	488300000\\
92000.0000000032	-1464600000\\
309999.999999997	3173400000\\
-126999.999999999	-3905800000\\
-92000.0000000014	1098200000\\
-36999.999999999	1099100000\\
-237999.999999999	-1099100000\\
219999.999999997	1099000000\\
-72999.9999999995	-976800000\\
-110000.000000001	-244000000\\
385000.000000002	2319200000\\
89999.9999999972	-2441200000\\
-438000.000000001	-122300000.000001\\
-18999.9999999983	1465000000\\
146999.999999998	-610300000\\
201000.000000002	365900000\\
-347999.999999997	-1098300000\\
255999.999999998	1342700000\\
20000.0000000005	-610500000\\
-495999.999999999	-1464700000\\
402999.999999998	3173800000\\
-127000	-2441400000\\
144999.999999998	854300000.000002\\
220000.000000001	1099100000\\
92000.0000000005	-1343400000\\
-421000	-1952500000\\
-91999.9999999987	2807100000\\
293999.999999999	244500000.000001\\
-148000.000000001	-2319600000\\
-401999.999999999	2197600000\\
566999.999999998	-1221100000\\
-34999.9999999966	1098800000\\
-167000.000000003	-1342600000\\
20000.0000000013	488099999.999999\\
53999.9999999967	366099999.999999\\
2.66453525910038e-09	122400000\\
-90999.9999999993	-1343100000\\
110000.000000002	2075400000\\
-147000.000000002	-2441500000\\
495000	4150500000\\
-221000.000000002	-4883100000\\
-401999.999999999	1343200000\\
-55000.0000000015	732099999.999999\\
440000.000000001	2075300000\\
-38000.000000002	-3662200000\\
57000.0000000013	2319600000\\
-404999.999999997	-2197500000\\
275999.999999999	3540000000\\
164000.000000001	-3051600000\\
-475000	366100000\\
36000.0000000005	976800000\\
329000.000000001	121600000\\
-183000.000000002	-731899999.999999\\
57000.0000000013	-366600000.000001\\
399999.999999997	2807800000\\
-53999.9999999985	-3540000000\\
-750000.000000002	-488499999.999999\\
348000	3540300000\\
492999.999999998	-610400000.000002\\
-675999.999999997	-3296200000\\
-1000.000000003	2686000000\\
495000.000000002	1098300000\\
-330000.000000004	-3906100000\\
55000.0000000024	3906200000\\
438999.999999998	-488199999.999999\\
-382999.999999998	-3174000000\\
-94000.0000000021	2441600000\\
-33999.9999999954	-732599999.999999\\
34999.9999999984	732600000\\
331000	1342600000\\
-112000.000000002	-3051600000\\
};
\addplot [color=mycolor2, line width=2.0pt, forget plot]
  table[row sep=crcr]{%
-111000.000000001	-111000.000000001\\
238999.999999998	238999.999999998\\
-129000.000000001	-129000.000000001\\
147000.000000002	147000.000000002\\
127999.999999996	127999.999999996\\
-494999.999999997	-494999.999999997\\
202999.999999998	202999.999999998\\
53999.9999999976	53999.9999999976\\
36000.0000000014	36000.0000000014\\
293999.999999998	293999.999999998\\
-348000	-348000\\
-183999.999999998	-183999.999999998\\
-292000	-292000\\
457000.000000002	457000.000000002\\
329999.999999995	329999.999999995\\
-219999.999999998	-219999.999999998\\
73999.9999999998	73999.9999999998\\
-37000.0000000017	-37000.0000000017\\
-587000	-587000\\
789000.000000004	789000.000000004\\
-184000.000000003	-184000.000000003\\
-493999.999999999	-493999.999999999\\
530999.999999999	530999.999999999\\
-74000.0000000016	-74000.0000000016\\
-327999.999999998	-327999.999999998\\
326999.999999997	326999.999999997\\
112000.000000002	112000.000000002\\
-457999.999999997	-457999.999999997\\
347999.999999995	347999.999999995\\
35000.000000001	35000.000000001\\
-126000.000000001	-126000.000000001\\
17000.0000000003	17000.0000000003\\
-147000.000000002	-147000.000000002\\
57000.0000000013	57000.0000000013\\
162999.999999998	162999.999999998\\
-109999.999999999	-109999.999999999\\
38000.0000000029	38000.0000000029\\
-55000.0000000006	-55000.0000000006\\
-93000.0000000026	-93000.0000000026\\
276000.000000001	276000.000000001\\
35999.9999999987	35999.9999999987\\
-275000.000000002	-275000.000000002\\
238000.000000002	238000.000000002\\
-255000.000000001	-255000.000000001\\
-111000.000000002	-111000.000000002\\
385000.000000002	385000.000000002\\
-183000.000000001	-183000.000000001\\
-147000	-147000\\
200999.999999997	200999.999999997\\
56000.0000000036	56000.0000000036\\
-275000	-275000\\
385000	385000\\
16999.9999999977	16999.9999999977\\
-311000	-311000\\
-109000	-109000\\
200999.999999999	200999.999999999\\
237999.999999999	237999.999999999\\
-329000	-329000\\
-111000	-111000\\
329999.999999997	329999.999999997\\
-35999.9999999978	-35999.9999999978\\
-275000.000000002	-275000.000000002\\
183000.000000001	183000.000000001\\
-129000	-129000\\
258000.000000001	258000.000000001\\
-111000.000000002	-111000.000000002\\
18999.9999999983	18999.9999999983\\
-110999.999999998	-110999.999999998\\
202999.999999998	202999.999999998\\
-93000.0000000008	-93000.0000000008\\
-72000.0000000009	-72000.0000000009\\
16999.9999999986	16999.9999999986\\
19000.000000001	19000.000000001\\
-237000	-237000\\
162999.999999999	162999.999999999\\
256999.999999998	256999.999999998\\
-109999.999999997	-109999.999999997\\
165999.999999999	165999.999999999\\
-533000.000000001	-533000.000000001\\
221000.000000001	221000.000000001\\
183000.000000001	183000.000000001\\
37000.0000000026	37000.0000000026\\
-331000.000000004	-331000.000000004\\
57000.0000000039	57000.0000000039\\
436999.999999996	436999.999999996\\
-290999.999999995	-290999.999999995\\
90999.9999999975	90999.9999999975\\
-129000	-129000\\
-54000.0000000011	-54000.0000000011\\
-73000.0000000004	-73000.0000000004\\
237000.000000002	237000.000000002\\
92999.9999999982	92999.9999999982\\
-313000.000000001	-313000.000000001\\
-36000.0000000014	-36000.0000000014\\
37000.0000000008	37000.0000000008\\
275000	275000\\
-129000.000000002	-129000.000000002\\
-128000	-128000\\
147000.000000001	147000.000000001\\
-111000.000000002	-111000.000000002\\
221000.000000004	221000.000000004\\
-201999.999999999	-201999.999999999\\
165000	165000\\
-54999.9999999988	-54999.9999999988\\
-201000.000000003	-201000.000000003\\
273000.000000001	273000.000000001\\
-126000.000000001	-126000.000000001\\
-148000.000000001	-148000.000000001\\
201999.999999998	201999.999999998\\
55000.0000000006	55000.0000000006\\
-184000.000000002	-184000.000000002\\
148000.000000002	148000.000000002\\
-166000.000000002	-166000.000000002\\
74000.0000000016	74000.0000000016\\
127999.999999999	127999.999999999\\
-238000.000000003	-238000.000000003\\
54000.000000002	54000.000000002\\
-35000.000000001	-35000.000000001\\
218999.999999998	218999.999999998\\
-74000.0000000016	-74000.0000000016\\
183000.000000002	183000.000000002\\
-456000.000000001	-456000.000000001\\
72000.0000000018	72000.0000000018\\
420999.999999999	420999.999999999\\
-291999.999999998	-291999.999999998\\
-1000.00000000122	-1000.00000000122\\
330000.000000003	330000.000000003\\
-329000.000000002	-329000.000000002\\
-203000.000000002	-203000.000000002\\
202999.999999999	202999.999999999\\
-147999.999999998	-147999.999999998\\
258000.000000002	258000.000000002\\
53999.9999999994	53999.9999999994\\
19000.0000000028	19000.0000000028\\
-202000.000000004	-202000.000000004\\
72000.0000000009	72000.0000000009\\
-51999.9999999996	-51999.9999999996\\
15999.9999999973	15999.9999999973\\
165000	165000\\
-163999.999999999	-163999.999999999\\
-35999.9999999969	-35999.9999999969\\
52999.9999999937	52999.9999999937\\
-89999.9999999972	-89999.9999999972\\
-20000.0000000013	-20000.0000000013\\
2000.00000000511	2000.00000000511\\
365999.999999998	365999.999999998\\
-111000.000000002	-111000.000000002\\
-220000.000000001	-220000.000000001\\
37999.9999999994	37999.9999999994\\
182000	182000\\
-126999.999999998	-126999.999999998\\
-202000.000000003	-202000.000000003\\
-55999.9999999974	-55999.9999999974\\
459999.999999997	459999.999999997\\
-239999.999999999	-239999.999999999\\
-8.88178419700125e-10	-8.88178419700125e-10\\
37999.9999999985	37999.9999999985\\
90999.9999999993	90999.9999999993\\
-72999.9999999977	-72999.9999999977\\
-20000.0000000005	-20000.0000000005\\
-271999.999999999	-271999.999999999\\
-39000.0000000006	-39000.0000000006\\
368000.000000001	368000.000000001\\
108999.999999999	108999.999999999\\
54999.9999999997	54999.9999999997\\
-147000.000000002	-147000.000000002\\
-108999.999999998	-108999.999999998\\
36999.9999999981	36999.9999999981\\
-440999.999999999	-440999.999999999\\
586999.999999999	586999.999999999\\
-238000.000000002	-238000.000000002\\
17999.9999999998	17999.9999999998\\
-110000	-110000\\
275000.000000001	275000.000000001\\
146000.000000003	146000.000000003\\
-255000.000000003	-255000.000000003\\
199000.000000002	199000.000000002\\
-419000	-419000\\
201000	201000\\
-1000.00000000033	-1000.00000000033\\
130000	130000\\
-94000.0000000012	-94000.0000000012\\
-89999.999999999	-89999.999999999\\
110000.000000002	110000.000000002\\
-128000	-128000\\
217999.999999997	217999.999999997\\
-382999.999999998	-382999.999999998\\
257000.000000001	257000.000000001\\
273000	273000\\
-146000.000000003	-146000.000000003\\
-292999.999999999	-292999.999999999\\
-128000.000000001	-128000.000000001\\
348000.000000002	348000.000000002\\
-35999.9999999987	-35999.9999999987\\
34999.9999999984	34999.9999999984\\
2000.00000000333	2000.00000000333\\
89999.9999999963	89999.9999999963\\
-182999.999999998	-182999.999999998\\
-73000.0000000004	-73000.0000000004\\
56000.0000000009	56000.0000000009\\
308999.999999999	308999.999999999\\
-125999.999999997	-125999.999999997\\
-477000	-477000\\
439999.999999999	439999.999999999\\
72000.0000000001	72000.0000000001\\
-364999.999999997	-364999.999999997\\
128000	128000\\
218999.999999998	218999.999999998\\
-108999.999999997	-108999.999999997\\
-165000.000000003	-165000.000000003\\
163999.999999999	163999.999999999\\
-108999.999999998	-108999.999999998\\
-55000.0000000006	-55000.0000000006\\
292999.999999999	292999.999999999\\
-18999.9999999983	-18999.9999999983\\
-219000.000000001	-219000.000000001\\
-128000.000000004	-128000.000000004\\
-999.999999997669	-999.999999997669\\
604999.999999999	604999.999999999\\
-220000.000000003	-220000.000000003\\
-238000	-238000\\
-146999.999999999	-146999.999999999\\
258000.000000002	258000.000000002\\
16999.999999995	16999.999999995\\
-145999.999999998	-145999.999999998\\
-19000.000000001	-19000.000000001\\
54999.9999999988	54999.9999999988\\
38000.0000000002	38000.0000000002\\
218000	218000\\
-494000.000000001	-494000.000000001\\
276000.000000002	276000.000000002\\
162999.999999998	162999.999999998\\
-126999.999999999	-126999.999999999\\
-73999.9999999989	-73999.9999999989\\
-127000.000000002	-127000.000000002\\
437999.999999999	437999.999999999\\
-17999.9999999971	-17999.9999999971\\
-585000.000000002	-585000.000000002\\
201000	201000\\
72999.9999999968	72999.9999999968\\
183000	183000\\
-17999.9999999998	-17999.9999999998\\
-348000.000000002	-348000.000000002\\
329000	329000\\
55999.9999999991	55999.9999999991\\
-274999.999999999	-274999.999999999\\
145999.999999998	145999.999999998\\
55000.0000000006	55000.0000000006\\
-183000.000000002	-183000.000000002\\
294000.000000004	294000.000000004\\
-295000.000000003	-295000.000000003\\
167000	167000\\
-130000.000000001	-130000.000000001\\
-127000	-127000\\
255999.999999998	255999.999999998\\
-91999.9999999996	-91999.9999999996\\
-35999.9999999987	-35999.9999999987\\
402999.999999998	402999.999999998\\
-148000	-148000\\
-291000	-291000\\
16999.9999999986	16999.9999999986\\
-91000.0000000011	-91000.0000000011\\
422000.000000001	422000.000000001\\
-332000.000000002	-332000.000000002\\
-34000.0000000007	-34000.0000000007\\
-75000.000000001	-75000.000000001\\
440000	440000\\
-311000.000000002	-311000.000000002\\
292000.000000001	292000.000000001\\
-218000.000000001	-218000.000000001\\
162999.999999999	162999.999999999\\
-273000	-273000\\
108000.000000001	108000.000000001\\
-72000.0000000036	-72000.0000000036\\
-54999.9999999988	-54999.9999999988\\
-1.77635683940025e-09	-1.77635683940025e-09\\
165000	165000\\
71999.9999999983	71999.9999999983\\
-127000	-127000\\
-200999.999999999	-200999.999999999\\
437999.999999998	437999.999999998\\
-36000.0000000014	-36000.0000000014\\
-348000	-348000\\
165999.999999999	165999.999999999\\
-111999.999999999	-111999.999999999\\
312999.999999999	312999.999999999\\
-73999.9999999989	-73999.9999999989\\
19000.0000000001	19000.0000000001\\
-221000.000000002	-221000.000000002\\
-127000.000000001	-127000.000000001\\
128000.000000002	128000.000000002\\
72000.0000000009	72000.0000000009\\
38999.9999999997	38999.9999999997\\
-130999.999999999	-130999.999999999\\
-52000.0000000023	-52000.0000000023\\
326999.999999999	326999.999999999\\
-162999.999999999	-162999.999999999\\
-56000.0000000009	-56000.0000000009\\
237999.999999999	237999.999999999\\
-71999.9999999992	-71999.9999999992\\
-549999.999999999	-549999.999999999\\
310000.000000001	310000.000000001\\
312999.999999997	312999.999999997\\
-201999.999999997	-201999.999999997\\
73999.9999999989	73999.9999999989\\
15999.9999999982	15999.9999999982\\
-289999.999999999	-289999.999999999\\
125999.999999999	125999.999999999\\
166000	166000\\
126999.999999998	126999.999999998\\
-547999.999999998	-547999.999999998\\
493999.999999999	493999.999999999\\
-73999.9999999981	-73999.9999999981\\
-237000.000000002	-237000.000000002\\
201000	201000\\
53999.9999999985	53999.9999999985\\
-108000.000000001	-108000.000000001\\
-294999.999999999	-294999.999999999\\
258000.000000003	258000.000000003\\
310999.999999999	310999.999999999\\
-56000	-56000\\
-383000.000000002	-383000.000000002\\
89999.9999999998	89999.9999999998\\
-16999.9999999995	-16999.9999999995\\
71999.9999999983	71999.9999999983\\
203000.000000003	203000.000000003\\
-258000.000000004	-258000.000000004\\
-53999.9999999976	-53999.9999999976\\
165000.000000002	165000.000000002\\
163999.999999998	163999.999999998\\
-182000.000000001	-182000.000000001\\
-221000	-221000\\
-18000.0000000007	-18000.0000000007\\
477000.000000003	477000.000000003\\
-221000.000000001	-221000.000000001\\
111000.000000002	111000.000000002\\
-166000	-166000\\
-89999.999999999	-89999.999999999\\
236999.999999997	236999.999999997\\
-37000.0000000017	-37000.0000000017\\
-163999.999999999	-163999.999999999\\
0	0\\
-256999.999999999	-256999.999999999\\
330999.999999999	330999.999999999\\
52999.999999999	52999.999999999\\
184000	184000\\
-311000.000000002	-311000.000000002\\
-8.88178419700125e-10	-8.88178419700125e-10\\
311000.000000001	311000.000000001\\
-348000.000000002	-348000.000000002\\
55000.0000000006	55000.0000000006\\
165000.000000001	165000.000000001\\
-274999.999999997	-274999.999999997\\
0	0\\
219999.999999996	219999.999999996\\
238000	238000\\
-255999.999999998	-255999.999999998\\
-129000.000000002	-129000.000000002\\
18999.9999999983	18999.9999999983\\
183000.000000001	183000.000000001\\
-458000.000000003	-458000.000000003\\
202000.000000001	202000.000000001\\
548000	548000\\
-530000.000000002	-530000.000000002\\
92000.0000000014	92000.0000000014\\
126999.999999998	126999.999999998\\
54999.9999999997	54999.9999999997\\
-71999.9999999965	-71999.9999999965\\
-184000.000000002	-184000.000000002\\
54999.999999997	54999.999999997\\
0	0\\
37000.0000000017	37000.0000000017\\
71999.9999999983	71999.9999999983\\
-180999.999999997	-180999.999999997\\
125999.999999999	125999.999999999\\
20000.0000000022	20000.0000000022\\
219000.000000001	219000.000000001\\
-311000.000000002	-311000.000000002\\
-37000.0000000008	-37000.0000000008\\
256000.000000001	256000.000000001\\
-238000.000000001	-238000.000000001\\
148000.000000003	148000.000000003\\
181999.999999999	181999.999999999\\
-220000.000000002	-220000.000000002\\
-220000.000000001	-220000.000000001\\
1000.00000000389	1000.00000000389\\
347999.999999997	347999.999999997\\
-19000.0000000001	-19000.0000000001\\
-348000.000000001	-348000.000000001\\
347999.999999998	347999.999999998\\
-53999.9999999967	-53999.9999999967\\
-240000	-240000\\
148999.999999998	148999.999999998\\
126000	126000\\
-329000.000000001	-329000.000000001\\
166000.000000002	166000.000000002\\
327999.999999998	327999.999999998\\
-584999.999999997	-584999.999999997\\
237999.999999999	237999.999999999\\
366000	366000\\
-385000.000000002	-385000.000000002\\
275000.000000002	275000.000000002\\
-219000.000000002	-219000.000000002\\
126999.999999999	126999.999999999\\
54999.9999999997	54999.9999999997\\
-400999.999999997	-400999.999999997\\
308999.999999997	308999.999999997\\
999.999999998557	999.999999998557\\
-219000	-219000\\
383000	383000\\
-183000.000000001	-183000.000000001\\
-200000.000000001	-200000.000000001\\
-183999.999999999	-183999.999999999\\
439999.999999997	439999.999999997\\
-18999.9999999992	-18999.9999999992\\
-8.88178419700125e-10	-8.88178419700125e-10\\
-18000.0000000016	-18000.0000000016\\
-110000	-110000\\
146999.999999999	146999.999999999\\
-20000.0000000005	-20000.0000000005\\
-180999.999999998	-180999.999999998\\
183000	183000\\
-277000.000000001	-277000.000000001\\
424000.000000003	424000.000000003\\
-203000.000000003	-203000.000000003\\
-220000.000000001	-220000.000000001\\
111000.000000003	111000.000000003\\
328999.999999999	328999.999999999\\
-164999.999999999	-164999.999999999\\
-127000.000000001	-127000.000000001\\
-39000.0000000015	-39000.0000000015\\
222000.000000001	222000.000000001\\
-147000.000000002	-147000.000000002\\
-183999.999999999	-183999.999999999\\
219999.999999998	219999.999999998\\
19000.0000000037	19000.0000000037\\
72999.9999999986	72999.9999999986\\
-238000	-238000\\
182999.999999998	182999.999999998\\
163999.999999998	163999.999999998\\
-364999.999999999	-364999.999999999\\
-164999.999999999	-164999.999999999\\
328999.999999996	328999.999999996\\
183000.000000002	183000.000000002\\
-109000	-109000\\
-110000	-110000\\
53999.9999999985	53999.9999999985\\
-36000.0000000014	-36000.0000000014\\
-8.88178419700125e-10	-8.88178419700125e-10\\
-127999.999999997	-127999.999999997\\
182999.999999999	182999.999999999\\
92000.0000000005	92000.0000000005\\
-367000.000000003	-367000.000000003\\
219000.000000001	219000.000000001\\
2000.00000000067	2000.00000000067\\
-999.999999999446	-999.999999999446\\
54999.9999999962	54999.9999999962\\
-183999.999999997	-183999.999999997\\
74999.9999999984	74999.9999999984\\
108000.000000003	108000.000000003\\
294999.999999998	294999.999999998\\
-277000.000000001	-277000.000000001\\
-71000.0000000006	-71000.0000000006\\
-999.999999998557	-999.999999998557\\
-311999.999999999	-311999.999999999\\
348999.999999997	348999.999999997\\
-257000.000000001	-257000.000000001\\
404000.000000003	404000.000000003\\
-93000.0000000035	-93000.0000000035\\
-238000	-238000\\
165999.999999999	165999.999999999\\
-55999.9999999991	-55999.9999999991\\
73999.9999999972	73999.9999999972\\
-183999.999999998	-183999.999999998\\
312000	312000\\
17999.9999999998	17999.9999999998\\
-145999.999999997	-145999.999999997\\
-129000.000000001	-129000.000000001\\
18999.9999999992	18999.9999999992\\
0	0\\
219000.000000001	219000.000000001\\
-146000.000000001	-146000.000000001\\
-53999.9999999985	-53999.9999999985\\
235999.999999997	235999.999999997\\
-493000.000000001	-493000.000000001\\
220000.000000001	220000.000000001\\
364999.999999997	364999.999999997\\
-309999.999999999	-309999.999999999\\
-294000.000000001	-294000.000000001\\
640999.999999998	640999.999999998\\
-255999.999999998	-255999.999999998\\
-273999.999999998	-273999.999999998\\
327999.999999999	327999.999999999\\
130000	130000\\
-203000.000000001	-203000.000000001\\
-201000.000000003	-201000.000000003\\
-90999.9999999984	-90999.9999999984\\
145999.999999999	145999.999999999\\
293000.000000002	293000.000000002\\
-110000.000000002	-110000.000000002\\
38000.0000000038	38000.0000000038\\
33999.9999999954	33999.9999999954\\
-216999.999999997	-216999.999999997\\
70999.9999999971	70999.9999999971\\
-34999.9999999993	-34999.9999999993\\
-19000.0000000028	-19000.0000000028\\
-54999.9999999997	-54999.9999999997\\
109999.999999999	109999.999999999\\
183000.000000002	183000.000000002\\
-347000	-347000\\
16999.9999999977	16999.9999999977\\
256000	256000\\
-198999.999999996	-198999.999999996\\
70999.9999999962	70999.9999999962\\
36999.9999999999	36999.9999999999\\
238000.000000002	238000.000000002\\
-236999.999999999	-236999.999999999\\
-146999.999999998	-146999.999999998\\
200999.999999998	200999.999999998\\
17999.999999998	17999.999999998\\
-218000	-218000\\
-130000	-130000\\
365999.999999998	365999.999999998\\
203000.000000006	203000.000000006\\
-165000.000000004	-165000.000000004\\
-239000.000000003	-239000.000000003\\
202000.000000001	202000.000000001\\
-330000.000000001	-330000.000000001\\
-90999.9999999984	-90999.9999999984\\
402999.999999999	402999.999999999\\
-165999.999999997	-165999.999999997\\
256999.999999996	256999.999999996\\
-90999.9999999984	-90999.9999999984\\
-184000.000000003	-184000.000000003\\
111000.000000002	111000.000000002\\
-20000.0000000013	-20000.0000000013\\
-52999.9999999999	-52999.9999999999\\
-93000.0000000017	-93000.0000000017\\
38000.0000000011	38000.0000000011\\
272999.999999998	272999.999999998\\
-71999.9999999983	-71999.9999999983\\
-148000.000000001	-148000.000000001\\
-34999.9999999975	-34999.9999999975\\
183000.000000002	183000.000000002\\
-129000	-129000\\
111000.000000002	111000.000000002\\
107999.999999998	107999.999999998\\
-162999.999999998	-162999.999999998\\
-74000.0000000016	-74000.0000000016\\
-275000.000000001	-275000.000000001\\
422000.000000004	422000.000000004\\
181999.999999998	181999.999999998\\
-217999.999999999	-217999.999999999\\
89999.9999999981	89999.9999999981\\
-220000.000000001	-220000.000000001\\
-125999.999999999	-125999.999999999\\
327000.000000002	327000.000000002\\
-474000.000000001	-474000.000000001\\
310000.000000001	310000.000000001\\
165000.000000001	165000.000000001\\
-55000.0000000006	-55000.0000000006\\
73999.9999999989	73999.9999999989\\
-165999.999999998	-165999.999999998\\
-182000	-182000\\
219999.999999998	219999.999999998\\
-19999.9999999978	-19999.9999999978\\
110999.999999997	110999.999999997\\
-127999.999999999	-127999.999999999\\
-164999.999999999	-164999.999999999\\
-165000.000000003	-165000.000000003\\
531000	531000\\
18000.0000000025	18000.0000000025\\
1000.00000000033	1000.00000000033\\
-422000.000000001	-422000.000000001\\
56000.0000000018	56000.0000000018\\
382999.999999999	382999.999999999\\
-182000.000000001	-182000.000000001\\
72000.0000000027	72000.0000000027\\
-126999.999999999	-126999.999999999\\
-91000.0000000011	-91000.0000000011\\
309000.000000002	309000.000000002\\
-254000	-254000\\
146000.000000001	146000.000000001\\
-148000.000000002	-148000.000000002\\
-290999.999999999	-290999.999999999\\
602999.999999997	602999.999999997\\
-274999.999999999	-274999.999999999\\
-72000.0000000001	-72000.0000000001\\
165000.000000002	165000.000000002\\
-93000.0000000035	-93000.0000000035\\
8.88178419700125e-10	8.88178419700125e-10\\
-15999.9999999973	-15999.9999999973\\
33999.9999999945	33999.9999999945\\
111000.000000003	111000.000000003\\
-56000.0000000027	-56000.0000000027\\
-235999.999999998	-235999.999999998\\
493000	493000\\
-128000.000000004	-128000.000000004\\
-548999.999999998	-548999.999999998\\
109999.999999999	109999.999999999\\
640000.000000001	640000.000000001\\
-237000	-237000\\
-531999.999999998	-531999.999999998\\
165000	165000\\
347999.999999999	347999.999999999\\
-109000.000000002	-109000.000000002\\
108999.999999999	108999.999999999\\
-329000	-329000\\
328999.999999998	328999.999999998\\
109000.000000001	109000.000000001\\
-509999.999999998	-509999.999999998\\
564999.999999998	564999.999999998\\
-311000	-311000\\
19999.9999999996	19999.9999999996\\
54000.0000000003	54000.0000000003\\
-366999.999999998	-366999.999999998\\
440999.999999997	440999.999999997\\
-348999.999999997	-348999.999999997\\
127999.999999998	127999.999999998\\
293999.999999998	293999.999999998\\
-18999.9999999992	-18999.9999999992\\
-293000.000000002	-293000.000000002\\
183000.000000001	183000.000000001\\
-219000.000000001	-219000.000000001\\
274000.000000001	274000.000000001\\
-109999.999999999	-109999.999999999\\
-73000.0000000022	-73000.0000000022\\
147000.000000002	147000.000000002\\
-238000.000000001	-238000.000000001\\
200000.000000004	200000.000000004\\
-255000.000000003	-255000.000000003\\
346999.999999999	346999.999999999\\
-36000.0000000005	-36000.0000000005\\
-164000	-164000\\
126000	126000\\
-163000.000000001	-163000.000000001\\
-238000	-238000\\
364999.999999997	364999.999999997\\
367000.000000002	367000.000000002\\
-440000.000000001	-440000.000000001\\
91999.9999999978	91999.9999999978\\
-328999.999999997	-328999.999999997\\
345999.999999998	345999.999999998\\
185000.000000001	185000.000000001\\
-458000	-458000\\
291999.999999998	291999.999999998\\
-255000.000000001	-255000.000000001\\
383000	383000\\
-219000	-219000\\
-201000	-201000\\
164000	164000\\
202000.000000002	202000.000000002\\
-202000.000000003	-202000.000000003\\
1000.00000000211	1000.00000000211\\
-111000.000000002	-111000.000000002\\
276000.000000001	276000.000000001\\
-258000.000000001	-258000.000000001\\
37000.0000000017	37000.0000000017\\
368000	368000\\
-94000.0000000021	-94000.0000000021\\
-217999.999999999	-217999.999999999\\
-257000.000000002	-257000.000000002\\
256000	256000\\
294000.000000001	294000.000000001\\
-386000.000000004	-386000.000000004\\
294000	294000\\
54999.9999999997	54999.9999999997\\
-422000.000000001	-422000.000000001\\
-236999.999999999	-236999.999999999\\
659000.000000001	659000.000000001\\
-91999.9999999996	-91999.9999999996\\
-366000.000000003	-366000.000000003\\
402999.999999999	402999.999999999\\
-110999.999999998	-110999.999999998\\
38999.9999999988	38999.9999999988\\
-259000.000000001	-259000.000000001\\
74999.9999999993	74999.9999999993\\
274000.000000001	274000.000000001\\
-238000.000000002	-238000.000000002\\
72999.9999999995	72999.9999999995\\
74000.0000000016	74000.0000000016\\
-184000.000000003	-184000.000000003\\
184000	184000\\
-183999.999999999	-183999.999999999\\
-145000.000000002	-145000.000000002\\
345999.999999999	345999.999999999\\
-53999.9999999994	-53999.9999999994\\
17999.9999999989	17999.9999999989\\
37000.0000000017	37000.0000000017\\
-73000.0000000013	-73000.0000000013\\
-202000.000000002	-202000.000000002\\
109999.999999999	109999.999999999\\
-8.88178419700125e-10	-8.88178419700125e-10\\
55000.0000000024	55000.0000000024\\
-17999.999999998	-17999.999999998\\
200999.999999997	200999.999999997\\
-17999.9999999998	-17999.9999999998\\
-532000	-532000\\
276000.000000003	276000.000000003\\
420999.999999998	420999.999999998\\
-385000.000000001	-385000.000000001\\
-37000.0000000017	-37000.0000000017\\
201999.999999998	201999.999999998\\
-36999.9999999999	-36999.9999999999\\
-238000	-238000\\
239000	239000\\
-130000	-130000\\
-33999.9999999989	-33999.9999999989\\
271999.999999998	271999.999999998\\
-273000.000000001	-273000.000000001\\
108999.999999999	108999.999999999\\
-17999.9999999998	-17999.9999999998\\
54000.0000000003	54000.0000000003\\
-143999.999999997	-143999.999999997\\
216999.999999997	216999.999999997\\
-181000	-181000\\
-38000.0000000011	-38000.0000000011\\
164999.999999999	164999.999999999\\
147000.000000001	147000.000000001\\
-145999.999999998	-145999.999999998\\
-185000	-185000\\
-126999.999999999	-126999.999999999\\
-164000.000000002	-164000.000000002\\
548000	548000\\
91999.9999999978	91999.9999999978\\
-54999.999999997	-54999.999999997\\
-311000.000000001	-311000.000000001\\
128000	128000\\
73999.9999999998	73999.9999999998\\
15999.9999999982	15999.9999999982\\
-254000.000000001	-254000.000000001\\
145999.999999999	145999.999999999\\
36000.0000000005	36000.0000000005\\
-73000.0000000013	-73000.0000000013\\
164999.999999998	164999.999999998\\
-127999.999999999	-127999.999999999\\
108999.999999999	108999.999999999\\
-236999.999999999	-236999.999999999\\
-110999.999999997	-110999.999999997\\
567999.999999996	567999.999999996\\
-382999.999999996	-382999.999999996\\
107999.999999997	107999.999999997\\
-128000	-128000\\
148000.000000003	148000.000000003\\
-148000.000000003	-148000.000000003\\
256000	256000\\
-199999.999999997	-199999.999999997\\
17999.9999999989	17999.9999999989\\
255999.999999998	255999.999999998\\
-587000	-587000\\
533000	533000\\
-238999.999999999	-238999.999999999\\
0	0\\
18999.9999999983	18999.9999999983\\
199999.999999999	199999.999999999\\
-292000.000000002	-292000.000000002\\
18000.0000000033	18000.0000000033\\
312000	312000\\
-147000	-147000\\
-56000.0000000036	-56000.0000000036\\
-126999.999999999	-126999.999999999\\
184000.000000003	184000.000000003\\
15999.9999999973	15999.9999999973\\
-88999.9999999995	-88999.9999999995\\
-76000.0000000014	-76000.0000000014\\
131000.000000002	131000.000000002\\
-277000.000000003	-277000.000000003\\
239000.000000001	239000.000000001\\
202000	202000\\
-148000.000000001	-148000.000000001\\
-70999.9999999979	-70999.9999999979\\
216999.999999998	216999.999999998\\
-271999.999999998	-271999.999999998\\
15999.9999999973	15999.9999999973\\
56999.9999999995	56999.9999999995\\
-75000.0000000002	-75000.0000000002\\
-127000.000000002	-127000.000000002\\
238000.000000002	238000.000000002\\
35999.9999999987	35999.9999999987\\
17999.9999999971	17999.9999999971\\
37000.0000000017	37000.0000000017\\
-420000	-420000\\
272999.999999999	272999.999999999\\
111000.000000001	111000.000000001\\
35999.9999999969	35999.9999999969\\
-402999.999999999	-402999.999999999\\
495000	495000\\
91000.0000000002	91000.0000000002\\
-402000.000000001	-402000.000000001\\
-1000.00000000122	-1000.00000000122\\
-201000	-201000\\
294000	294000\\
-20999.9999999981	-20999.9999999981\\
295999.999999999	295999.999999999\\
-258000	-258000\\
127999.999999997	127999.999999997\\
-273999.999999997	-273999.999999997\\
-18000.0000000016	-18000.0000000016\\
126999.999999999	126999.999999999\\
185000	185000\\
-314000.000000001	-314000.000000001\\
221999.999999998	221999.999999998\\
54000.0000000003	54000.0000000003\\
-419999.999999999	-419999.999999999\\
272000.000000001	272000.000000001\\
20999.9999999964	20999.9999999964\\
109000.000000001	109000.000000001\\
-92000.0000000014	-92000.0000000014\\
-55000.0000000006	-55000.0000000006\\
1000.000000003	1000.000000003\\
163999.999999996	163999.999999996\\
-255999.999999998	-255999.999999998\\
53999.9999999976	53999.9999999976\\
294000.000000003	294000.000000003\\
-292999.999999997	-292999.999999997\\
128000	128000\\
-146000.000000003	-146000.000000003\\
-19999.9999999987	-19999.9999999987\\
294999.999999998	294999.999999998\\
-111000.000000002	-111000.000000002\\
-201999.999999998	-201999.999999998\\
167000.000000001	167000.000000001\\
88999.9999999977	88999.9999999977\\
-345999.999999999	-345999.999999999\\
108999.999999997	108999.999999997\\
256000	256000\\
-163999.999999999	-163999.999999999\\
-145999.999999998	-145999.999999998\\
455999.999999996	455999.999999996\\
-254999.999999998	-254999.999999998\\
-37000.0000000035	-37000.0000000035\\
-146999.999999998	-146999.999999998\\
1000.00000000033	1000.00000000033\\
457000	457000\\
-329000.000000001	-329000.000000001\\
-220000.000000002	-220000.000000002\\
512000	512000\\
-144999.999999999	-144999.999999999\\
-404000.000000003	-404000.000000003\\
347999.999999999	347999.999999999\\
-17999.9999999989	-17999.9999999989\\
-164000	-164000\\
126000	126000\\
19999.9999999987	19999.9999999987\\
-183000.000000002	-183000.000000002\\
73000.0000000013	73000.0000000013\\
126999.999999998	126999.999999998\\
-90000.0000000007	-90000.0000000007\\
109000	109000\\
-109000.000000001	-109000.000000001\\
16999.9999999986	16999.9999999986\\
110000.000000002	110000.000000002\\
57000.0000000022	57000.0000000022\\
-259000.000000005	-259000.000000005\\
129000.000000002	129000.000000002\\
-71999.9999999974	-71999.9999999974\\
-74000.0000000025	-74000.0000000025\\
310000	310000\\
-218000	-218000\\
72999.9999999995	72999.9999999995\\
-19000.0000000019	-19000.0000000019\\
-238000	-238000\\
385000.000000001	385000.000000001\\
-146999.999999999	-146999.999999999\\
1000.00000000122	1000.00000000122\\
126999.999999999	126999.999999999\\
-91000.0000000037	-91000.0000000037\\
-109999.999999999	-109999.999999999\\
-274000	-274000\\
455999.999999999	455999.999999999\\
-198999.999999997	-198999.999999997\\
180999.999999996	180999.999999996\\
-181999.999999999	-181999.999999999\\
-182999.999999998	-182999.999999998\\
365999.999999999	365999.999999999\\
126999.999999999	126999.999999999\\
-235999.999999997	-235999.999999997\\
-2000.00000000333	-2000.00000000333\\
112000.000000003	112000.000000003\\
-221000.000000003	-221000.000000003\\
17999.999999998	17999.999999998\\
202000.000000003	202000.000000003\\
-202000.000000003	-202000.000000003\\
19000.000000001	19000.000000001\\
-17999.9999999989	-17999.9999999989\\
236999.999999997	236999.999999997\\
-35999.9999999978	-35999.9999999978\\
-202000.000000003	-202000.000000003\\
257000.000000001	257000.000000001\\
-274999.999999996	-274999.999999996\\
165999.999999997	165999.999999997\\
-75000.0000000019	-75000.0000000019\\
-146000	-146000\\
201999.999999997	201999.999999997\\
365000.000000001	365000.000000001\\
-620999.999999999	-620999.999999999\\
126999.999999998	126999.999999998\\
1000.00000000122	1000.00000000122\\
126999.999999998	126999.999999998\\
148000.000000003	148000.000000003\\
-386000.000000004	-386000.000000004\\
36999.9999999999	36999.9999999999\\
367000.000000003	367000.000000003\\
-112000.000000002	-112000.000000002\\
-54000.000000002	-54000.000000002\\
-218999.999999996	-218999.999999996\\
91999.9999999987	91999.9999999987\\
271999.999999998	271999.999999998\\
-289999.999999997	-289999.999999997\\
16999.9999999995	16999.9999999995\\
218999.999999999	218999.999999999\\
-199999.999999998	-199999.999999998\\
-129000.000000001	-129000.000000001\\
348000.000000003	348000.000000003\\
-128000	-128000\\
-291999.999999999	-291999.999999999\\
381999.999999995	381999.999999995\\
-217999.999999999	-217999.999999999\\
37000.0000000008	37000.0000000008\\
16999.9999999986	16999.9999999986\\
-89999.9999999963	-89999.9999999963\\
309999.999999998	309999.999999998\\
-146000.000000003	-146000.000000003\\
-183999.999999997	-183999.999999997\\
94000.0000000021	94000.0000000021\\
-3000.00000000455	-3000.00000000455\\
185000.000000003	185000.000000003\\
-129000.000000003	-129000.000000003\\
-255999.999999997	-255999.999999997\\
385000.000000001	385000.000000001\\
-1000.00000000122	-1000.00000000122\\
-438999.999999999	-438999.999999999\\
422000.000000001	422000.000000001\\
52999.9999999973	52999.9999999973\\
-162999.999999999	-162999.999999999\\
-19000.000000001	-19000.000000001\\
35999.9999999978	35999.9999999978\\
-419999.999999999	-419999.999999999\\
493999.999999998	493999.999999998\\
91000.0000000002	91000.0000000002\\
-181999.999999999	-181999.999999999\\
-147999.999999999	-147999.999999999\\
-91000.0000000019	-91000.0000000019\\
294000.000000001	294000.000000001\\
-73999.9999999998	-73999.9999999998\\
0	0\\
109999.999999999	109999.999999999\\
-129000.000000001	-129000.000000001\\
239000.000000001	239000.000000001\\
-36999.9999999999	-36999.9999999999\\
-512000.000000002	-512000.000000002\\
273999.999999999	273999.999999999\\
73000.0000000013	73000.0000000013\\
-256000	-256000\\
383999.999999998	383999.999999998\\
-292000	-292000\\
457000.000000002	457000.000000002\\
-292000.000000002	-292000.000000002\\
-257999.999999999	-257999.999999999\\
441000.000000002	441000.000000002\\
-74000.0000000025	-74000.0000000025\\
-146000	-146000\\
-183000.000000002	-183000.000000002\\
108000.000000001	108000.000000001\\
2999.99999999923	2999.99999999923\\
382000	382000\\
-219000.000000001	-219000.000000001\\
-218999.999999998	-218999.999999998\\
91000.0000000002	91000.0000000002\\
1000.00000000122	1000.00000000122\\
-112000.000000002	-112000.000000002\\
404000.000000001	404000.000000001\\
-181999.999999999	-181999.999999999\\
88999.9999999986	88999.9999999986\\
-70999.9999999971	-70999.9999999971\\
-293000	-293000\\
455999.999999997	455999.999999997\\
-493000.000000001	-493000.000000001\\
238000	238000\\
35999.9999999987	35999.9999999987\\
-256000.000000001	-256000.000000001\\
238000.000000001	238000.000000001\\
109999.999999998	109999.999999998\\
-165999.999999999	-165999.999999999\\
240000.000000003	240000.000000003\\
91000.0000000002	91000.0000000002\\
-403000.000000004	-403000.000000004\\
36000.0000000014	36000.0000000014\\
-18000.0000000042	-18000.0000000042\\
55000.0000000024	55000.0000000024\\
-37000.0000000008	-37000.0000000008\\
349000.000000003	349000.000000003\\
-110000	-110000\\
-368000.000000001	-368000.000000001\\
149000.000000002	149000.000000002\\
253999.999999997	253999.999999997\\
-107999.999999996	-107999.999999996\\
-92000.0000000014	-92000.0000000014\\
71999.9999999965	71999.9999999965\\
-144999.999999998	-144999.999999998\\
108999.999999998	108999.999999998\\
-236999.999999999	-236999.999999999\\
108000.000000002	108000.000000002\\
239999.999999998	239999.999999998\\
-1000.00000000211	-1000.00000000211\\
-110000	-110000\\
184000.000000003	184000.000000003\\
-148000	-148000\\
-346000.000000001	-346000.000000001\\
52999.9999999999	52999.9999999999\\
184999.999999999	184999.999999999\\
126999.999999996	126999.999999996\\
2.66453525910038e-09	2.66453525910038e-09\\
-109000.000000002	-109000.000000002\\
-1000.00000000211	-1000.00000000211\\
-91999.9999999961	-91999.9999999961\\
368000	368000\\
-220999.999999997	-220999.999999997\\
36999.9999999999	36999.9999999999\\
220000	220000\\
-404000.000000003	-404000.000000003\\
-16999.9999999986	-16999.9999999986\\
72999.9999999995	72999.9999999995\\
146000.000000001	146000.000000001\\
73999.9999999998	73999.9999999998\\
-166000.000000002	-166000.000000002\\
54999.9999999997	54999.9999999997\\
-327999.999999998	-327999.999999998\\
365000	365000\\
219999.999999999	219999.999999999\\
-328999.999999999	-328999.999999999\\
-239000.000000001	-239000.000000001\\
201000	201000\\
275999.999999998	275999.999999998\\
-348999.999999999	-348999.999999999\\
258000	258000\\
-240000.000000003	-240000.000000003\\
-16999.9999999995	-16999.9999999995\\
438000.000000001	438000.000000001\\
-438000	-438000\\
18000.0000000033	18000.0000000033\\
255999.999999998	255999.999999998\\
37000.0000000017	37000.0000000017\\
8.88178419700125e-10	8.88178419700125e-10\\
-202000.000000003	-202000.000000003\\
92000.0000000005	92000.0000000005\\
-311000	-311000\\
346999.999999998	346999.999999998\\
-310000	-310000\\
309999.999999998	309999.999999998\\
74000.0000000025	74000.0000000025\\
-367000.000000003	-367000.000000003\\
476999.999999999	476999.999999999\\
-548999.999999999	-548999.999999999\\
419999.999999999	419999.999999999\\
17999.9999999998	17999.9999999998\\
-363999.999999999	-363999.999999999\\
272000.000000002	272000.000000002\\
-145000.000000004	-145000.000000004\\
92000.0000000014	92000.0000000014\\
90999.9999999993	90999.9999999993\\
-275999.999999999	-275999.999999999\\
93999.9999999994	93999.9999999994\\
144000	144000\\
-53000.0000000026	-53000.0000000026\\
-202000.000000001	-202000.000000001\\
146999.999999999	146999.999999999\\
290999.999999999	290999.999999999\\
-254000	-254000\\
-221000	-221000\\
219999.999999998	219999.999999998\\
201000.000000002	201000.000000002\\
-201000.000000002	-201000.000000002\\
257000.000000002	257000.000000002\\
-239000.000000003	-239000.000000003\\
-438999.999999998	-438999.999999998\\
292000	292000\\
313000.000000001	313000.000000001\\
-312000.000000003	-312000.000000003\\
292000.000000002	292000.000000002\\
1000.00000000033	1000.00000000033\\
-110000	-110000\\
-183000	-183000\\
37000.0000000008	37000.0000000008\\
199999.999999997	199999.999999997\\
38000.0000000002	38000.0000000002\\
-330999.999999998	-330999.999999998\\
184999.999999999	184999.999999999\\
34999.9999999993	34999.9999999993\\
56000	56000\\
-384999.999999996	-384999.999999996\\
218999.999999999	218999.999999999\\
220999.999999997	220999.999999997\\
-17999.9999999998	-17999.9999999998\\
-386999.999999998	-386999.999999998\\
185999.999999999	185999.999999999\\
438000	438000\\
-531000.000000002	-531000.000000002\\
109999.999999999	109999.999999999\\
257000.000000001	257000.000000001\\
-20000.0000000005	-20000.0000000005\\
-436999.999999999	-436999.999999999\\
125999.999999999	125999.999999999\\
421999.999999999	421999.999999999\\
-420999.999999999	-420999.999999999\\
110000.000000001	110000.000000001\\
200000.000000003	200000.000000003\\
-309000.000000003	-309000.000000003\\
-111999.999999999	-111999.999999999\\
257999.999999999	257999.999999999\\
108999.999999997	108999.999999997\\
-183000.000000001	-183000.000000001\\
19000.0000000019	19000.0000000019\\
199999.999999999	199999.999999999\\
-17000.0000000003	-17000.0000000003\\
-238999.999999995	-238999.999999995\\
-35999.9999999996	-35999.9999999996\\
73000.0000000013	73000.0000000013\\
36999.9999999999	36999.9999999999\\
-256999.999999999	-256999.999999999\\
475999.999999996	475999.999999996\\
-16999.9999999968	-16999.9999999968\\
-203000.000000005	-203000.000000005\\
-108999.999999998	-108999.999999998\\
74000.0000000016	74000.0000000016\\
88999.9999999977	88999.9999999977\\
39000.000000005	39000.000000005\\
90999.9999999993	90999.9999999993\\
-256000.000000003	-256000.000000003\\
35000.000000001	35000.000000001\\
111000.000000001	111000.000000001\\
-128000.000000001	-128000.000000001\\
-35999.9999999978	-35999.9999999978\\
272999.999999996	272999.999999996\\
-127999.999999999	-127999.999999999\\
20000.0000000022	20000.0000000022\\
-93000.0000000044	-93000.0000000044\\
8.88178419700125e-10	8.88178419700125e-10\\
-35999.9999999987	-35999.9999999987\\
292999.999999998	292999.999999998\\
-349000	-349000\\
2000.00000000244	2000.00000000244\\
144999.999999995	144999.999999995\\
110000	110000\\
-220000	-220000\\
184000	184000\\
-74000.0000000025	-74000.0000000025\\
92000.0000000023	92000.0000000023\\
-348000	-348000\\
54999.9999999997	54999.9999999997\\
439999.999999998	439999.999999998\\
-56999.9999999986	-56999.9999999986\\
-419000	-419000\\
438999.999999999	438999.999999999\\
-275000	-275000\\
-183000	-183000\\
384000.000000001	384000.000000001\\
-128000.000000002	-128000.000000002\\
-127000	-127000\\
218999.999999999	218999.999999999\\
-293999.999999996	-293999.999999996\\
312999.999999997	312999.999999997\\
-19999.9999999969	-19999.9999999969\\
-346000.000000002	-346000.000000002\\
273000.000000002	273000.000000002\\
-109000	-109000\\
219000	219000\\
-201000.000000003	-201000.000000003\\
1000.00000000033	1000.00000000033\\
125999.999999999	125999.999999999\\
-89000.0000000013	-89000.0000000013\\
-258999.999999999	-258999.999999999\\
625000.000000003	625000.000000003\\
-313000.000000004	-313000.000000004\\
-144999.999999997	-144999.999999997\\
235999.999999996	235999.999999996\\
39000.000000005	39000.000000005\\
-515000.000000002	-515000.000000002\\
459000	459000\\
54999.999999997	54999.999999997\\
-456999.999999998	-456999.999999998\\
346000.000000001	346000.000000001\\
-54000.0000000011	-54000.0000000011\\
165000	165000\\
-73999.9999999981	-73999.9999999981\\
-127000.000000001	-127000.000000001\\
-110999.999999998	-110999.999999998\\
-54000.0000000003	-54000.0000000003\\
328999.999999998	328999.999999998\\
-110000.000000001	-110000.000000001\\
-145000	-145000\\
70999.9999999997	70999.9999999997\\
294000	294000\\
-346999.999999998	-346999.999999998\\
52999.9999999964	52999.9999999964\\
222000.000000003	222000.000000003\\
-149000.000000003	-149000.000000003\\
-308999.999999997	-308999.999999997\\
565999.999999997	565999.999999997\\
-346999.999999998	-346999.999999998\\
146000.000000003	146000.000000003\\
-127000.000000002	-127000.000000002\\
17000.0000000003	17000.0000000003\\
310999.999999997	310999.999999997\\
-181999.999999999	-181999.999999999\\
-239000.000000002	-239000.000000002\\
-89999.9999999998	-89999.9999999998\\
199999.999999998	199999.999999998\\
146000.000000001	146000.000000001\\
20000.0000000022	20000.0000000022\\
53999.9999999976	53999.9999999976\\
-439000.000000001	-439000.000000001\\
-56000.0000000009	-56000.0000000009\\
349000.000000003	349000.000000003\\
127999.999999998	127999.999999998\\
109000.000000002	109000.000000002\\
-310000.000000002	-310000.000000002\\
-220000.000000001	-220000.000000001\\
126999.999999999	126999.999999999\\
220999.999999999	220999.999999999\\
-19000.0000000001	-19000.0000000001\\
109999.999999999	109999.999999999\\
-383999.999999998	-383999.999999998\\
163999.999999997	163999.999999997\\
147000.000000002	147000.000000002\\
-274999.999999999	-274999.999999999\\
312000.000000001	312000.000000001\\
-56000.0000000009	-56000.0000000009\\
-219000	-219000\\
220000.000000002	220000.000000002\\
163999.999999998	163999.999999998\\
-403000.000000001	-403000.000000001\\
-71999.9999999992	-71999.9999999992\\
199999.999999998	199999.999999998\\
111000.000000004	111000.000000004\\
-8.88178419700125e-10	-8.88178419700125e-10\\
-239000.000000003	-239000.000000003\\
202000.000000003	202000.000000003\\
72999.9999999995	72999.9999999995\\
-311000	-311000\\
238000	238000\\
-201000	-201000\\
218000	218000\\
112000.000000002	112000.000000002\\
-146999.999999999	-146999.999999999\\
71999.9999999965	71999.9999999965\\
-401000.000000001	-401000.000000001\\
364000	364000\\
-16000.0000000009	-16000.0000000009\\
-93000	-93000\\
-127999.999999999	-127999.999999999\\
109999.999999998	109999.999999998\\
147000	147000\\
-55999.9999999991	-55999.9999999991\\
221000.000000001	221000.000000001\\
-349000.000000005	-349000.000000005\\
111000.000000004	111000.000000004\\
-148000	-148000\\
-54000.0000000029	-54000.0000000029\\
275000.000000001	275000.000000001\\
-19000.0000000019	-19000.0000000019\\
-293000	-293000\\
293999.999999999	293999.999999999\\
-19999.9999999978	-19999.9999999978\\
-35000.000000001	-35000.000000001\\
-38000.0000000011	-38000.0000000011\\
258000.000000003	258000.000000003\\
-221000.000000003	-221000.000000003\\
-74000.0000000007	-74000.0000000007\\
-15999.9999999991	-15999.9999999991\\
-57000.0000000004	-57000.0000000004\\
240000.000000002	240000.000000002\\
34999.9999999975	34999.9999999975\\
-220000.000000001	-220000.000000001\\
-181999.999999999	-181999.999999999\\
275000	275000\\
126999.999999998	126999.999999998\\
-347999.999999999	-347999.999999999\\
276000.000000002	276000.000000002\\
164000.000000001	164000.000000001\\
-146000.000000002	-146000.000000002\\
-111000.000000002	-111000.000000002\\
-54999.9999999997	-54999.9999999997\\
-217999.999999998	-217999.999999998\\
475000	475000\\
-219000.000000001	-219000.000000001\\
-999.999999999446	-999.999999999446\\
72999.9999999968	72999.9999999968\\
-17000.0000000003	-17000.0000000003\\
-37999.9999999985	-37999.9999999985\\
165999.999999999	165999.999999999\\
-257000.000000002	-257000.000000002\\
201000	201000\\
-90000.0000000007	-90000.0000000007\\
-221000.000000001	-221000.000000001\\
164999.999999999	164999.999999999\\
146000	146000\\
203000	203000\\
-826000.000000001	-826000.000000001\\
770000.000000001	770000.000000001\\
56000.0000000027	56000.0000000027\\
-222000.000000005	-222000.000000005\\
-363999.999999995	-363999.999999995\\
291999.999999997	291999.999999997\\
72999.9999999977	72999.9999999977\\
-183000	-183000\\
292999.999999998	292999.999999998\\
-366000	-366000\\
-73000.0000000013	-73000.0000000013\\
438999.999999999	438999.999999999\\
91000.0000000002	91000.0000000002\\
-401999.999999998	-401999.999999998\\
74000.0000000007	74000.0000000007\\
70999.9999999979	70999.9999999979\\
112000.000000003	112000.000000003\\
-165000	-165000\\
107999.999999998	107999.999999998\\
-291000.000000003	-291000.000000003\\
237000.000000001	237000.000000001\\
-201000.000000001	-201000.000000001\\
274000.000000002	274000.000000002\\
-126999.999999998	-126999.999999998\\
256000	256000\\
-111000.000000001	-111000.000000001\\
-107999.999999999	-107999.999999999\\
-130000.000000002	-130000.000000002\\
203000.000000001	203000.000000001\\
52999.9999999973	52999.9999999973\\
-271999.999999996	-271999.999999996\\
88999.9999999942	88999.9999999942\\
130000.000000002	130000.000000002\\
35000.0000000001	35000.0000000001\\
-383000.000000002	-383000.000000002\\
219000.000000001	219000.000000001\\
273999.999999999	273999.999999999\\
-162999.999999998	-162999.999999998\\
-147999.999999998	-147999.999999998\\
-237999.999999999	-237999.999999999\\
366999.999999995	366999.999999995\\
421000.000000002	421000.000000002\\
-642000.000000001	-642000.000000001\\
111999.999999999	111999.999999999\\
126000.000000001	126000.000000001\\
-127000.000000003	-127000.000000003\\
128000.000000002	128000.000000002\\
72999.9999999986	72999.9999999986\\
-91000.0000000002	-91000.0000000002\\
-91999.9999999996	-91999.9999999996\\
-91999.9999999978	-91999.9999999978\\
-54000.0000000038	-54000.0000000038\\
293000.000000002	293000.000000002\\
144999.999999998	144999.999999998\\
-181000	-181000\\
-19999.9999999978	-19999.9999999978\\
-183000.000000001	-183000.000000001\\
-127000.000000001	-127000.000000001\\
347999.999999998	347999.999999998\\
16000.0000000009	16000.0000000009\\
94000.0000000003	94000.0000000003\\
-257000.000000001	-257000.000000001\\
35999.9999999996	35999.9999999996\\
-110000	-110000\\
93000.0000000017	93000.0000000017\\
71999.9999999974	71999.9999999974\\
238000	238000\\
-127999.999999997	-127999.999999997\\
-273999.999999998	-273999.999999998\\
109999.999999998	109999.999999998\\
34999.9999999975	34999.9999999975\\
129000.000000001	129000.000000001\\
-254999.999999999	-254999.999999999\\
52999.9999999982	52999.9999999982\\
129000	129000\\
90999.9999999993	90999.9999999993\\
-109000	-109000\\
-256000	-256000\\
108000	108000\\
19999.9999999987	19999.9999999987\\
201000.000000002	201000.000000002\\
-91999.9999999996	-91999.9999999996\\
91999.9999999987	91999.9999999987\\
-146000.000000001	-146000.000000001\\
-111000.000000001	-111000.000000001\\
73000.0000000004	73000.0000000004\\
331000.000000002	331000.000000002\\
-73999.9999999989	-73999.9999999989\\
-347000	-347000\\
52999.9999999982	52999.9999999982\\
75000.0000000037	75000.0000000037\\
310000	310000\\
-16999.9999999986	-16999.9999999986\\
-385000.000000002	-385000.000000002\\
-56000.0000000027	-56000.0000000027\\
240000.000000004	240000.000000004\\
52999.9999999973	52999.9999999973\\
-345999.999999998	-345999.999999998\\
125999.999999999	125999.999999999\\
19999.9999999996	19999.9999999996\\
236999.999999998	236999.999999998\\
-54000.0000000003	-54000.0000000003\\
-109999.999999999	-109999.999999999\\
-74000.0000000007	-74000.0000000007\\
54999.9999999988	54999.9999999988\\
55999.9999999991	55999.9999999991\\
54000.0000000003	54000.0000000003\\
54999.9999999988	54999.9999999988\\
-128000.000000001	-128000.000000001\\
-53999.9999999994	-53999.9999999994\\
-166999.999999999	-166999.999999999\\
276999.999999998	276999.999999998\\
108000	108000\\
-72000.0000000001	-72000.0000000001\\
-366999.999999997	-366999.999999997\\
439999.999999999	439999.999999999\\
-219999.999999998	-219999.999999998\\
-37000.0000000008	-37000.0000000008\\
-17000.0000000003	-17000.0000000003\\
217999.999999997	217999.999999997\\
111000.000000003	111000.000000003\\
-476000.000000001	-476000.000000001\\
293000.000000001	293000.000000001\\
144999.999999996	144999.999999996\\
-180999.999999996	-180999.999999996\\
-240000.000000004	-240000.000000004\\
203000.000000003	203000.000000003\\
164000	164000\\
-128000	-128000\\
129000.000000001	129000.000000001\\
-148000.000000001	-148000.000000001\\
-2.66453525910038e-09	-2.66453525910038e-09\\
-107999.999999998	-107999.999999998\\
254000	254000\\
-254000.000000001	-254000.000000001\\
181000	181000\\
-16999.9999999995	-16999.9999999995\\
-367000	-367000\\
642000.000000003	642000.000000003\\
16999.9999999995	16999.9999999995\\
-383000.000000001	-383000.000000001\\
-130000.000000002	-130000.000000002\\
55999.9999999983	55999.9999999983\\
72999.9999999995	72999.9999999995\\
74000.0000000007	74000.0000000007\\
-20000.0000000005	-20000.0000000005\\
74999.9999999993	74999.9999999993\\
89999.999999999	89999.999999999\\
-235999.999999999	-235999.999999999\\
-21000.0000000017	-21000.0000000017\\
94000.0000000012	94000.0000000012\\
236999.999999998	236999.999999998\\
-275000	-275000\\
-274000	-274000\\
384000	384000\\
-110000.000000001	-110000.000000001\\
-35999.9999999996	-35999.9999999996\\
238000.000000001	238000.000000001\\
-36999.9999999999	-36999.9999999999\\
-145999.999999997	-145999.999999997\\
-202000.000000002	-202000.000000002\\
202000.000000003	202000.000000003\\
-55000.0000000033	-55000.0000000033\\
146000.000000002	146000.000000002\\
-54999.9999999997	-54999.9999999997\\
127999.999999998	127999.999999998\\
-126999.999999997	-126999.999999997\\
183000	183000\\
-203000.000000005	-203000.000000005\\
-420000	-420000\\
385000.000000001	385000.000000001\\
438000.000000001	438000.000000001\\
-731000.000000002	-731000.000000002\\
237000	237000\\
166000.000000004	166000.000000004\\
273000	273000\\
-255000.000000001	-255000.000000001\\
-92000.0000000014	-92000.0000000014\\
-310999.999999999	-310999.999999999\\
328999.999999997	328999.999999997\\
-18000.0000000007	-18000.0000000007\\
165000	165000\\
-330000.000000001	-330000.000000001\\
238000	238000\\
-201000.000000002	-201000.000000002\\
238000.000000003	238000.000000003\\
-202000.000000003	-202000.000000003\\
-89999.999999999	-89999.999999999\\
363999.999999999	363999.999999999\\
-273000.000000001	-273000.000000001\\
54000.0000000003	54000.0000000003\\
147000.000000001	147000.000000001\\
164999.999999998	164999.999999998\\
-422000.000000001	-422000.000000001\\
18999.9999999983	18999.9999999983\\
293000.000000002	293000.000000002\\
-202000.000000003	-202000.000000003\\
111000.000000002	111000.000000002\\
-331000.000000002	-331000.000000002\\
366999.999999999	366999.999999999\\
73000.0000000004	73000.0000000004\\
-110000.000000001	-110000.000000001\\
-218999.999999998	-218999.999999998\\
-110000.000000001	-110000.000000001\\
219000.000000002	219000.000000002\\
145999.999999999	145999.999999999\\
93999.9999999985	93999.9999999985\\
-350999.999999998	-350999.999999998\\
1999.999999998	1999.999999998\\
147000.000000001	147000.000000001\\
198999.999999997	198999.999999997\\
-143999.999999997	-143999.999999997\\
-203000.000000003	-203000.000000003\\
109999.999999999	109999.999999999\\
148000.000000003	148000.000000003\\
-167000.000000003	-167000.000000003\\
75000.000000001	75000.000000001\\
126999.999999999	126999.999999999\\
-164000.000000001	-164000.000000001\\
-146999.999999998	-146999.999999998\\
237999.999999998	237999.999999998\\
-108999.999999998	-108999.999999998\\
-74000.0000000016	-74000.0000000016\\
-36999.9999999999	-36999.9999999999\\
459000.000000001	459000.000000001\\
-294000	-294000\\
-238000.000000003	-238000.000000003\\
421000.000000002	421000.000000002\\
-34999.9999999984	-34999.9999999984\\
-167000.000000002	-167000.000000002\\
-53000.0000000017	-53000.0000000017\\
-56999.9999999995	-56999.9999999995\\
149000	149000\\
-75999.9999999969	-75999.9999999969\\
-236000.000000005	-236000.000000005\\
732000.000000002	732000.000000002\\
-532000.000000002	-532000.000000002\\
-181999.999999999	-181999.999999999\\
201999.999999998	201999.999999998\\
236000.000000002	236000.000000002\\
-54000.0000000029	-54000.0000000029\\
-71999.9999999974	-71999.9999999974\\
-350000	-350000\\
92999.9999999991	92999.9999999991\\
274999.999999998	274999.999999998\\
162999.999999999	162999.999999999\\
-144999.999999999	-144999.999999999\\
-221000.000000002	-221000.000000002\\
130000.000000002	130000.000000002\\
-38000.000000002	-38000.000000002\\
1000.000000003	1000.000000003\\
-1000.00000000389	-1000.00000000389\\
128000	128000\\
999.999999999446	999.999999999446\\
-999.999999999446	-999.999999999446\\
-108999.999999999	-108999.999999999\\
-73000.0000000013	-73000.0000000013\\
-56000.0000000009	-56000.0000000009\\
404000.000000002	404000.000000002\\
-294000.000000001	-294000.000000001\\
-274999.999999999	-274999.999999999\\
367999.999999998	367999.999999998\\
89999.999999999	89999.999999999\\
-238000	-238000\\
1000.000000003	1000.000000003\\
493999.999999998	493999.999999998\\
-751000.000000001	-751000.000000001\\
311000	311000\\
292999.999999998	292999.999999998\\
-401999.999999995	-401999.999999995\\
201000	201000\\
-365999.999999999	-365999.999999999\\
494000.000000001	494000.000000001\\
146000	146000\\
-565999.999999998	-565999.999999998\\
162999.999999998	162999.999999998\\
166000.000000002	166000.000000002\\
-148000.000000001	-148000.000000001\\
130000.000000002	130000.000000002\\
-184000.000000001	-184000.000000001\\
238999.999999999	238999.999999999\\
53000.0000000026	53000.0000000026\\
-108000.000000001	-108000.000000001\\
-239000	-239000\\
145999.999999998	145999.999999998\\
74999.9999999993	74999.9999999993\\
-331000	-331000\\
238000	238000\\
329999.999999997	329999.999999997\\
-310999.999999997	-310999.999999997\\
-56000.0000000018	-56000.0000000018\\
203000.000000003	203000.000000003\\
-257000.000000001	-257000.000000001\\
310999.999999997	310999.999999997\\
-293999.999999997	-293999.999999997\\
56999.9999999968	56999.9999999968\\
-56000	-56000\\
201999.999999999	201999.999999999\\
-18999.9999999992	-18999.9999999992\\
-36999.9999999981	-36999.9999999981\\
-53000.0000000008	-53000.0000000008\\
-240999.999999999	-240999.999999999\\
441999.999999998	441999.999999998\\
-219999.999999999	-219999.999999999\\
-276000	-276000\\
312000.000000001	312000.000000001\\
218999.999999998	218999.999999998\\
-70999.9999999988	-70999.9999999988\\
-277000.000000001	-277000.000000001\\
-91000.0000000011	-91000.0000000011\\
697000.000000003	697000.000000003\\
-386000.000000001	-386000.000000001\\
-200999.999999997	-200999.999999997\\
73000.0000000004	73000.0000000004\\
-164000.000000001	-164000.000000001\\
17999.9999999998	17999.9999999998\\
182999.999999996	182999.999999996\\
330000.000000002	330000.000000002\\
-569000.000000001	-569000.000000001\\
111999.999999999	111999.999999999\\
345999.999999998	345999.999999998\\
-90999.9999999984	-90999.9999999984\\
54999.9999999979	54999.9999999979\\
-421000	-421000\\
37000.0000000017	37000.0000000017\\
255999.999999999	255999.999999999\\
238000	238000\\
-147000.000000002	-147000.000000002\\
-493000	-493000\\
419000	419000\\
185000	185000\\
-311999.999999999	-311999.999999999\\
-329000	-329000\\
54000.0000000003	54000.0000000003\\
658999.999999997	658999.999999997\\
-89999.9999999972	-89999.9999999972\\
-130000.000000004	-130000.000000004\\
-236999.999999998	-236999.999999998\\
439999.999999999	439999.999999999\\
-641999.999999997	-641999.999999997\\
220000.000000001	220000.000000001\\
36999.9999999999	36999.9999999999\\
-36999.9999999981	-36999.9999999981\\
90999.9999999984	90999.9999999984\\
130000	130000\\
-20000.0000000022	-20000.0000000022\\
-91000.0000000011	-91000.0000000011\\
36999.9999999981	36999.9999999981\\
-18999.9999999992	-18999.9999999992\\
36999.9999999999	36999.9999999999\\
-165000.000000004	-165000.000000004\\
-91999.9999999961	-91999.9999999961\\
404000.000000002	404000.000000002\\
109999.999999999	109999.999999999\\
-185000.000000001	-185000.000000001\\
-749000.000000001	-749000.000000001\\
696000.000000002	696000.000000002\\
89999.9999999972	89999.9999999972\\
54999.9999999997	54999.9999999997\\
-327999.999999998	-327999.999999998\\
-221000.000000002	-221000.000000002\\
257000.000000001	257000.000000001\\
257000.000000001	257000.000000001\\
-149000.000000001	-149000.000000001\\
-15999.9999999991	-15999.9999999991\\
146000.000000003	146000.000000003\\
-238000.000000002	-238000.000000002\\
365000	365000\\
-400999.999999999	-400999.999999999\\
-202000.000000003	-202000.000000003\\
310000	310000\\
37999.9999999994	37999.9999999994\\
1.77635683940025e-09	1.77635683940025e-09\\
126999.999999998	126999.999999998\\
-345999.999999998	-345999.999999998\\
198999.999999997	198999.999999997\\
222000.000000001	222000.000000001\\
-203000.000000001	-203000.000000001\\
-108000.000000001	-108000.000000001\\
-186000.000000001	-186000.000000001\\
369000.000000002	369000.000000002\\
35000.0000000001	35000.0000000001\\
-438999.999999999	-438999.999999999\\
55000.0000000006	55000.0000000006\\
604999.999999999	604999.999999999\\
-294999.999999998	-294999.999999998\\
-254000.000000002	-254000.000000002\\
-19999.9999999969	-19999.9999999969\\
512999.999999996	512999.999999996\\
-17999.9999999998	-17999.9999999998\\
-311000	-311000\\
-494999.999999998	-494999.999999998\\
293999.999999998	293999.999999998\\
401999.999999999	401999.999999999\\
-220000	-220000\\
92000.0000000014	92000.0000000014\\
108999.999999999	108999.999999999\\
94000.0000000012	94000.0000000012\\
-369000	-369000\\
-90000.0000000016	-90000.0000000016\\
347000.000000001	347000.000000001\\
-108999.999999998	-108999.999999998\\
-129000.000000001	-129000.000000001\\
93000.0000000008	93000.0000000008\\
16000	16000\\
-198999.999999997	-198999.999999997\\
492999.999999998	492999.999999998\\
-109999.999999998	-109999.999999998\\
-146000.000000003	-146000.000000003\\
-145999.999999998	-145999.999999998\\
-166000.000000004	-166000.000000004\\
257000.000000001	257000.000000001\\
1000.00000000122	1000.00000000122\\
-222000.000000002	-222000.000000002\\
442000.000000002	442000.000000002\\
-222000.000000003	-222000.000000003\\
-145000	-145000\\
475000	475000\\
-474999.999999999	-474999.999999999\\
-55000.0000000024	-55000.0000000024\\
254999.999999999	254999.999999999\\
203000.000000002	203000.000000002\\
-202000.000000002	-202000.000000002\\
-294000	-294000\\
257999.999999998	257999.999999998\\
-403000	-403000\\
438000.000000001	438000.000000001\\
240000.000000001	240000.000000001\\
-313000.000000003	-313000.000000003\\
-53999.9999999976	-53999.9999999976\\
-18999.9999999992	-18999.9999999992\\
275000	275000\\
92000.0000000023	92000.0000000023\\
-238000.000000002	-238000.000000002\\
35999.9999999996	35999.9999999996\\
-275000.000000001	-275000.000000001\\
38000.0000000002	38000.0000000002\\
71999.9999999983	71999.9999999983\\
256999.999999998	256999.999999998\\
-109999.999999999	-109999.999999999\\
-202000.000000003	-202000.000000003\\
129000.000000004	129000.000000004\\
36999.999999999	36999.999999999\\
-93000.0000000035	-93000.0000000035\\
239000.000000001	239000.000000001\\
90999.9999999993	90999.9999999993\\
-329000	-329000\\
-8.88178419700125e-10	-8.88178419700125e-10\\
-36999.9999999964	-36999.9999999964\\
18000.0000000007	18000.0000000007\\
-72000.0000000018	-72000.0000000018\\
292000.000000001	292000.000000001\\
-146999.999999998	-146999.999999998\\
-127000.000000005	-127000.000000005\\
421000.000000002	421000.000000002\\
-496000	-496000\\
39000.0000000041	39000.0000000041\\
548000	548000\\
-292999.999999998	-292999.999999998\\
-200000.000000002	-200000.000000002\\
70999.9999999988	70999.9999999988\\
330000.000000003	330000.000000003\\
-328000	-328000\\
164000.000000001	164000.000000001\\
-257000.000000004	-257000.000000004\\
-254999.999999998	-254999.999999998\\
309999.999999998	309999.999999998\\
275000.000000002	275000.000000002\\
90999.9999999984	90999.9999999984\\
-217999.999999996	-217999.999999996\\
16999.9999999995	16999.9999999995\\
-184000.000000001	-184000.000000001\\
-70999.9999999988	-70999.9999999988\\
530000.000000001	530000.000000001\\
-385000.000000003	-385000.000000003\\
-35999.9999999996	-35999.9999999996\\
-238000	-238000\\
419999.999999998	419999.999999998\\
257999.999999999	257999.999999999\\
-514000	-514000\\
164999.999999999	164999.999999999\\
-310000	-310000\\
199999.999999998	199999.999999998\\
274999.999999999	274999.999999999\\
-36999.9999999964	-36999.9999999964\\
-90000.0000000007	-90000.0000000007\\
-203000.000000003	-203000.000000003\\
201999.999999999	201999.999999999\\
55000.0000000006	55000.0000000006\\
-129000.000000002	-129000.000000002\\
-16999.9999999986	-16999.9999999986\\
71999.9999999974	71999.9999999974\\
-144999.999999999	-144999.999999999\\
-129000.000000001	-129000.000000001\\
237000.000000001	237000.000000001\\
148000.000000001	148000.000000001\\
54999.9999999997	54999.9999999997\\
-258000.000000003	-258000.000000003\\
57000.0000000004	57000.0000000004\\
-148000	-148000\\
73999.9999999989	73999.9999999989\\
146000.000000001	146000.000000001\\
-18000.0000000007	-18000.0000000007\\
128000	128000\\
-548999.999999999	-548999.999999999\\
273999.999999999	273999.999999999\\
459000.000000001	459000.000000001\\
-93000	-93000\\
-512000.000000002	-512000.000000002\\
-91000.0000000002	-91000.0000000002\\
456000.000000001	456000.000000001\\
-16000	-16000\\
-999.999999999446	-999.999999999446\\
-385999.999999999	-385999.999999999\\
167000.000000001	167000.000000001\\
456999.999999998	456999.999999998\\
-72999.9999999968	-72999.9999999968\\
-587000.000000001	-587000.000000001\\
-72000.0000000009	-72000.0000000009\\
457000	457000\\
328999.999999999	328999.999999999\\
-436999.999999998	-436999.999999998\\
-332000.000000001	-332000.000000001\\
366999.999999997	366999.999999997\\
219000.000000003	219000.000000003\\
-365000	-365000\\
110000.000000002	110000.000000002\\
384000	384000\\
-349000.000000003	-349000.000000003\\
-254999.999999998	-254999.999999998\\
458000.000000001	458000.000000001\\
-349000.000000004	-349000.000000004\\
55000.0000000006	55000.0000000006\\
294000.000000003	294000.000000003\\
72999.9999999995	72999.9999999995\\
-678000.000000003	-678000.000000003\\
128000.000000001	128000.000000001\\
440000	440000\\
91000.0000000002	91000.0000000002\\
-219000.000000001	-219000.000000001\\
-220999.999999998	-220999.999999998\\
332000.000000001	332000.000000001\\
-3000.00000000011	-3000.00000000011\\
-254000	-254000\\
-37000.0000000026	-37000.0000000026\\
127000	127000\\
73999.9999999981	73999.9999999981\\
-91999.9999999996	-91999.9999999996\\
147000.000000001	147000.000000001\\
-127999.999999997	-127999.999999997\\
-183000.000000003	-183000.000000003\\
492999.999999999	492999.999999999\\
-254999.999999999	-254999.999999999\\
-128000.000000001	-128000.000000001\\
164000	164000\\
-202000.000000002	-202000.000000002\\
-90000.0000000007	-90000.0000000007\\
329000.000000002	329000.000000002\\
-129000.000000002	-129000.000000002\\
-52999.9999999973	-52999.9999999973\\
88999.9999999977	88999.9999999977\\
185000.000000003	185000.000000003\\
-147000.000000003	-147000.000000003\\
146000.000000002	146000.000000002\\
-493000	-493000\\
-39000.0000000015	-39000.0000000015\\
717000	717000\\
-186000.000000002	-186000.000000002\\
-602000	-602000\\
273999.999999999	273999.999999999\\
292000.000000001	292000.000000001\\
-255000	-255000\\
36000.0000000005	36000.0000000005\\
366000.000000003	366000.000000003\\
-237000.000000002	-237000.000000002\\
-294999.999999999	-294999.999999999\\
460000.000000002	460000.000000002\\
-239000.000000001	-239000.000000001\\
-165000.000000001	-165000.000000001\\
73000.0000000013	73000.0000000013\\
129999.999999999	129999.999999999\\
15999.9999999982	15999.9999999982\\
91999.9999999996	91999.9999999996\\
-127000	-127000\\
-74000.0000000007	-74000.0000000007\\
-110000.000000003	-110000.000000003\\
220000.000000001	220000.000000001\\
35999.9999999996	35999.9999999996\\
-419999.999999999	-419999.999999999\\
438999.999999999	438999.999999999\\
164999.999999999	164999.999999999\\
-330999.999999999	-330999.999999999\\
-126000.000000001	-126000.000000001\\
419000	419000\\
-72000.0000000027	-72000.0000000027\\
-457000	-457000\\
163000.000000001	163000.000000001\\
256999.999999997	256999.999999997\\
1000.000000003	1000.000000003\\
-37999.9999999985	-37999.9999999985\\
-365999.999999999	-365999.999999999\\
110999.999999998	110999.999999998\\
108000.000000001	108000.000000001\\
148000.000000001	148000.000000001\\
-109999.999999999	-109999.999999999\\
36999.999999999	36999.999999999\\
345999.999999997	345999.999999997\\
-345999.999999997	-345999.999999997\\
-514000.000000004	-514000.000000004\\
459000.000000003	459000.000000003\\
365000	365000\\
-291999.999999998	-291999.999999998\\
-257000.000000001	-257000.000000001\\
385000.000000002	385000.000000002\\
-37000.0000000008	-37000.0000000008\\
-108999.999999997	-108999.999999997\\
-111000.000000003	-111000.000000003\\
-91000.0000000028	-91000.0000000028\\
420000.000000002	420000.000000002\\
-217000	-217000\\
-314000.000000001	-314000.000000001\\
276000.000000001	276000.000000001\\
146000	146000\\
256000	256000\\
-309999.999999999	-309999.999999999\\
-715000	-715000\\
823999.999999999	823999.999999999\\
238000	238000\\
-547999.999999998	-547999.999999998\\
-203999.999999999	-203999.999999999\\
202999.999999996	202999.999999996\\
92000.0000000032	92000.0000000032\\
309999.999999997	309999.999999997\\
-126999.999999999	-126999.999999999\\
-92000.0000000014	-92000.0000000014\\
-36999.999999999	-36999.999999999\\
-237999.999999999	-237999.999999999\\
219999.999999997	219999.999999997\\
-72999.9999999995	-72999.9999999995\\
-110000.000000001	-110000.000000001\\
385000.000000002	385000.000000002\\
89999.9999999972	89999.9999999972\\
-438000.000000001	-438000.000000001\\
-18999.9999999983	-18999.9999999983\\
146999.999999998	146999.999999998\\
201000.000000002	201000.000000002\\
-347999.999999997	-347999.999999997\\
255999.999999998	255999.999999998\\
20000.0000000005	20000.0000000005\\
-495999.999999999	-495999.999999999\\
402999.999999998	402999.999999998\\
-127000	-127000\\
144999.999999998	144999.999999998\\
220000.000000001	220000.000000001\\
92000.0000000005	92000.0000000005\\
-421000	-421000\\
-91999.9999999987	-91999.9999999987\\
293999.999999999	293999.999999999\\
-148000.000000001	-148000.000000001\\
-401999.999999999	-401999.999999999\\
566999.999999998	566999.999999998\\
-34999.9999999966	-34999.9999999966\\
-167000.000000003	-167000.000000003\\
20000.0000000013	20000.0000000013\\
53999.9999999967	53999.9999999967\\
2.66453525910038e-09	2.66453525910038e-09\\
-90999.9999999993	-90999.9999999993\\
110000.000000002	110000.000000002\\
-147000.000000002	-147000.000000002\\
495000	495000\\
-221000.000000002	-221000.000000002\\
-401999.999999999	-401999.999999999\\
-55000.0000000015	-55000.0000000015\\
440000.000000001	440000.000000001\\
-38000.000000002	-38000.000000002\\
57000.0000000013	57000.0000000013\\
-404999.999999997	-404999.999999997\\
275999.999999999	275999.999999999\\
164000.000000001	164000.000000001\\
-475000	-475000\\
36000.0000000005	36000.0000000005\\
329000.000000001	329000.000000001\\
-183000.000000002	-183000.000000002\\
57000.0000000013	57000.0000000013\\
399999.999999997	399999.999999997\\
-53999.9999999985	-53999.9999999985\\
-750000.000000002	-750000.000000002\\
348000	348000\\
492999.999999998	492999.999999998\\
-675999.999999997	-675999.999999997\\
-1000.000000003	-1000.000000003\\
495000.000000002	495000.000000002\\
-330000.000000004	-330000.000000004\\
55000.0000000024	55000.0000000024\\
438999.999999998	438999.999999998\\
-382999.999999998	-382999.999999998\\
-94000.0000000021	-94000.0000000021\\
-33999.9999999954	-33999.9999999954\\
34999.9999999984	34999.9999999984\\
331000	331000\\
-112000.000000002	-112000.000000002\\
};
\end{axis}

\begin{axis}[%
width=4.927cm,
height=3.484cm,
at={(0cm,9.677cm)},
scale only axis,
xmin=-1000000,
xmax=1000000,
xlabel style={font=\color{white!15!black}},
xlabel={$\delta^3 u(t)$},
ymin=-6469400000,
ymax=5248900000,
ylabel style={font=\color{white!15!black}},
ylabel={y(t)},
axis background/.style={fill=white},
title={C4, R = 0.3621},
axis x line*=bottom,
axis y line*=left
]
\addplot[only marks, mark=*, mark options={}, mark size=1.5000pt, color=mycolor1, fill=mycolor1] table[row sep=crcr]{%
x	y\\
-164999.999999998	1587000000\\
202999.999999999	610300000.000001\\
-57000.0000000022	-1342800000\\
93000.0000000044	1709200000\\
145999.999999996	-2197600000\\
-494000.000000001	244400000\\
218999.999999999	3295700000\\
-36999.9999999999	-4516400000\\
203000.000000005	3295800000\\
145999.999999997	-1831000000\\
-312000	488100000\\
-90999.9999999975	366400000\\
-403000	-122100000\\
494000.000000002	-1465000000\\
329999.999999995	3662400000\\
-219999.999999998	-3418200000\\
55999.9999999991	1464900000\\
-1999.99999999978	-854400000.000001\\
-565000.000000001	-199999.999999889\\
729999.999999999	1342900000\\
-200000.000000001	-366199999.999999\\
-459000	-2197200000\\
533000	2685400000\\
-57000.0000000004	100000.000000122\\
-347000.000000001	-3418100000\\
367000	5005000000\\
16000.0000000009	-3662000000\\
-381999.999999998	243799999.999999\\
328999.999999996	2685900000\\
17000.0000000003	-2929800000\\
-72000.0000000027	1220500000\\
-55999.9999999983	-365900000\\
-72000.0000000001	244000000\\
-8.88178419700125e-10	366100000\\
181999.999999999	-243900000\\
-109999.999999999	-366500000\\
20000.0000000022	400000.000000844\\
-1000.00000000211	609900000\\
-147000.000000001	-976299999.999999\\
294000	1709000000\\
35999.9999999987	-1465000000\\
-257000.000000001	-244000000.000001\\
203000	1464700000\\
-258000.000000001	-1708700000\\
-52999.9999999999	732100000\\
327999.999999999	1098800000\\
-165000	-1587000000\\
-145000.000000001	366300000\\
201000	1098500000\\
53999.9999999994	-1464600000\\
-293000	121900000\\
423000.000000002	2441300000\\
-2000.00000000244	-3784000000\\
-311000	1953200000\\
-71999.9999999992	243800000\\
107999.999999999	-365900000\\
295000.000000001	610300000\\
-294000	-1953200000\\
-183000.000000002	2319300000\\
403000.000000001	-1220600000\\
-110999.999999997	366100000\\
-200000.000000005	-488000000\\
128000.000000003	609900000\\
-165000.000000003	-243800000\\
384000.000000001	2.66453525910038e-07\\
-238000.000000001	121799999.999999\\
111000.000000002	-365999999.999999\\
-111000.000000002	366200000.000001\\
146999.999999998	244200000\\
-128999.999999999	-1098900000\\
20000.0000000022	1465100000\\
16999.999999995	-1342800000\\
-72999.9999999995	854400000\\
-146000.000000002	-976500000\\
109000.000000001	1831000000\\
255999.999999998	-976500000\\
-52999.9999999946	-732500000.000001\\
108999.999999997	1098700000\\
-532000.000000002	-1709000000\\
257000.000000002	2319400000\\
183000.000000001	-1098800000\\
-35999.9999999978	610499999.999999\\
-202000.000000003	-3173900000\\
-53999.9999999967	4760800000\\
491999.999999996	-1831200000\\
-344999.999999997	-1098300000\\
145000	121700000.000001\\
-129000	1342900000\\
-89000.0000000013	-1098600000\\
-94000.0000000012	100000.000000477\\
348999.999999999	1342500000\\
-37000.0000000017	-1464600000\\
-256999.999999999	244100000\\
2000.00000000067	244100000\\
-20000.0000000013	366100000\\
276000.000000002	-488000000\\
-75000.0000000037	365800000\\
-199999.999999999	-854000000\\
237000	1342400000\\
-236999.999999999	-1708900000\\
347000	2319400000\\
-291999.999999998	-2807600000\\
256000	3295900000\\
-166000.000000003	-3418100000\\
-126000	1831100000\\
273000.000000001	366300000.000001\\
-183000.000000003	-1098600000\\
-90999.9999999984	244000000\\
164999.999999999	854500000\\
126999.999999998	-610300000\\
-291999.999999999	-488200000\\
238000.000000001	854400000.000001\\
-201000.000000001	-610400000\\
16999.9999999995	366300000\\
221000	122000000\\
-275000.000000003	-854400000\\
73000.0000000022	854500000\\
-92000.0000000032	-366400000\\
257000.000000001	976800000.000001\\
-54999.9999999979	-1709200000\\
201999.999999999	1709200000\\
-514000.000000003	-2685700000\\
37000.0000000026	3662100000\\
531999.999999998	-1953000000\\
-349000.000000001	-854499999.999999\\
-36999.999999999	1586700000\\
368000.000000001	-366000000.000001\\
-312000	-610400000\\
-239000.000000002	-122100000\\
238000	1098700000\\
-125999.999999996	-1587000000\\
143999.999999997	2319400000\\
166000.000000001	-1342800000\\
37000.0000000017	-122100000\\
-331000.000000003	-854500000\\
129000.000000002	1953200000\\
19000.000000001	-1464800000\\
-38000.0000000038	488099999.999999\\
166000.000000001	366300000\\
-184000.000000002	-732300000\\
-36000.0000000014	366000000\\
72999.9999999995	100000.000000477\\
-72999.9999999995	366300000\\
-38000.0000000011	-1220900000\\
2000.00000000511	1709200000\\
365999.999999998	-366400000.000002\\
-111000.000000002	-1464800000\\
-202000.000000001	854600000\\
-15999.9999999991	732400000\\
235999.999999999	-732600000\\
-144999.999999998	300000.000000544\\
-202000.000000003	-244500000\\
-55999.9999999974	488600000\\
495999.999999997	366000000\\
-292999.999999998	-121900000\\
-37999.9999999994	-1709100000\\
128999.999999999	2319300000\\
35999.9999999987	-854400000\\
-16999.9999999995	-854500000\\
-73999.9999999998	1220700000\\
-256999.999999997	-976700000\\
-17000.0000000012	488600000\\
310000.000000001	854100000\\
148000.000000001	-1464500000\\
71999.9999999983	1098300000\\
-183000.000000002	-487900000\\
-73000.0000000004	-732700000.000001\\
17999.9999999998	1220700000\\
-383000.000000002	-1098400000\\
419000.000000001	1464500000\\
-72000.0000000018	-1830600000\\
-19000.0000000001	1952600000\\
-145999.999999999	-2196900000\\
275000	2929700000\\
182000.000000001	-2319700000\\
-273999.999999997	610799999.999997\\
183999.999999997	-122299999.999998\\
-368000.000000001	-854500000\\
111000.000000002	2319400000\\
127999.999999999	-2319400000\\
36999.999999999	2563500000\\
-111000.000000001	-3540000000\\
-54000.000000002	3051800000\\
146000	-1587100000\\
-182999.999999997	854700000\\
220999.999999999	-244200000.000001\\
-332000.000000001	-1343000000\\
203000.000000002	3052200000\\
312000	-2441800000\\
-185000.000000005	610600000.000001\\
-291999.999999998	-1343000000\\
-92000.0000000032	2563700000\\
348000.000000002	-1098800000\\
-89999.9999999981	-488099999.999999\\
107000	488000000\\
-51999.9999999987	244500000.000001\\
107999.999999997	-366500000\\
-182999.999999998	-732300000\\
-73000.0000000004	610200000\\
74000.0000000016	854900000\\
255000.000000001	-244700000\\
-71999.9999999983	-1464400000\\
-495000	488099999.999998\\
439999.999999999	1953100000\\
72000.0000000001	-2319200000\\
-364999.999999997	488000000\\
128000	977000000\\
237999.999999997	-244600000\\
-165999.999999999	-1220400000\\
-125999.999999999	976400000\\
198999.999999997	732600000\\
-180999.999999997	-2319400000\\
-1000.000000003	2563200000\\
275000.000000002	-1098200000\\
-1000.00000000211	243899999.999999\\
-254999.999999999	-1342800000\\
-129000.000000002	1587100000\\
37000.0000000008	-732700000\\
550000.000000003	1465200000\\
-130000.000000003	-1709400000\\
-328000.000000001	-365799999.999999\\
-56000.0000000009	1098300000\\
147000.000000002	366500000.000001\\
111000.000000002	-1343100000\\
-204000.000000004	1343100000\\
2000.00000000333	-1098900000\\
54999.9999999979	610600000\\
71999.9999999992	243900000\\
109999.999999999	-121999999.999999\\
-345999.999999999	-1708700000\\
161999.999999998	2685000000\\
185000.000000002	-1220200000\\
-73999.9999999998	121800000\\
-164000.000000001	-1098500000\\
-999.999999997669	1708800000\\
329999.999999998	-488000000.000001\\
0	-244500000.000001\\
-548999.999999999	-1586500000\\
201000	3661700000\\
36999.999999999	-3173500000\\
201000	1586700000\\
0	-976400000\\
-384000	244100000\\
348000.000000001	1220500000\\
53999.9999999967	-976399999.999999\\
-292999.999999999	-1220500000\\
202000	2196900000\\
-35999.9999999978	-854300000\\
-93000.0000000044	-1220700000\\
258000.000000004	2563400000\\
-295000.000000003	-2807500000\\
167000	2441200000\\
-93000	-1830800000\\
-255999.999999999	366100000\\
420999.999999998	1464700000\\
-200999.999999999	-1464600000\\
35999.9999999996	-122200000\\
367000	1709000000\\
-183999.999999998	-2075300000\\
-237000.000000002	244500000\\
-1000.00000000122	1586400000\\
-109000.000000002	-1830600000\\
438999.999999999	1708600000\\
-329999.999999999	-1098200000\\
18999.9999999975	-610800000\\
-164999.999999996	1953400000\\
477000.000000001	-1342800000\\
-314000.000000004	-610400000\\
314000.000000004	2929700000\\
-257000.000000003	-4272600000\\
181999.999999999	3906600000\\
-292000.000000002	-3296300000\\
165000	2930000000\\
-129000.000000002	-2197500000\\
-16999.9999999968	366400000\\
-74999.9999999993	1342700000\\
311999.999999997	-488300000\\
-108999.999999997	-1098700000\\
-1000.000000003	1098800000\\
-236999.999999998	-1220800000\\
437999.999999998	2929700000\\
-36000.0000000014	-4028400000\\
-366000	2563600000\\
201000	-854599999.999999\\
-90999.9999999984	488500000.000001\\
220000.000000001	-300000.000001077\\
-19000.0000000019	199999.999999889\\
72999.9999999977	-732600000\\
-328999.999999999	244500000.000001\\
-37000.0000000035	121600000\\
74000.0000000025	488600000\\
71999.9999999974	-122100000\\
75000.0000000019	-732600000\\
-166999.999999998	854700000\\
1999.99999999623	-854600000\\
273000.000000001	1220700000\\
-127000.000000001	-976500000\\
-109999.999999999	-122200000\\
310999.999999997	1098800000\\
-146999.999999998	-1342800000\\
-512000.000000002	121900000\\
348000.000000001	1098800000\\
255999.999999998	-100000.000000211\\
-182999.999999997	-1830900000\\
54999.9999999988	2563300000\\
92000.0000000005	-2319200000\\
-367000.000000003	976400000\\
55000.0000000015	199999.999999534\\
365999.999999998	366100000.000001\\
-54000.0000000003	121999999.999999\\
-421999.999999999	-2075100000\\
440000	3906300000\\
-110000	-4638700000\\
-165000.000000003	3417800000\\
147000.000000002	-1098400000\\
71999.9999999983	244000000\\
-126999.999999999	-976500000.000001\\
-200999.999999998	854500000\\
126999.999999999	-244300000\\
366999.999999998	1343100000\\
-74000.0000000016	-1587300000\\
-347999.999999999	-1220400000\\
74000.0000000007	3295700000\\
-35999.9999999987	-2807400000\\
126999.999999995	1708700000\\
146000.000000002	-243900000\\
-255000	-1220800000\\
-8.88178419700125e-10	732500000\\
108000.000000001	1220400000\\
222000.000000001	-1708500000\\
-294000.000000001	854099999.999999\\
-165000.000000002	-976400000\\
110000.000000003	732500000.000001\\
294000	1220500000\\
-166000	-2197200000\\
218999.999999998	1709100000\\
-327999.999999999	-1464900000\\
-36999.9999999964	976400000\\
330000	-243800000\\
-166000.000000005	243800000\\
-72999.9999999995	-732199999.999999\\
-35999.9999999996	610200000\\
-293000.000000001	-366100000\\
365999.999999999	610300000\\
92000.0000000014	122000000\\
126999.999999998	-976300000\\
-274000.000000001	121800000\\
-35999.9999999987	1098600000\\
347000	-1098300000\\
-402000.000000003	732100000\\
109000	-1464800000\\
146999.999999999	2075400000\\
-257000	-1220900000\\
-35999.9999999996	100000.000000033\\
220000.000000001	488200000\\
274999.999999999	366200000.000001\\
-295000.000000002	-1464800000\\
-52999.999999999	854700000\\
-92000.0000000005	365800000\\
255000.000000002	-366000000\\
-492000.000000003	-488100000\\
255000	365900000\\
457999.999999997	1465000000\\
-439999.999999999	-2563600000\\
54999.9999999988	1831300000\\
129000.000000004	-854700000\\
72999.9999999995	610400000.000001\\
-109999.999999999	-854399999.999999\\
-165000.000000002	487999999.999999\\
54999.999999997	-121599999.999999\\
0	365700000\\
37000.0000000017	-365799999.999999\\
53999.9999999976	488000000\\
-163999.999999998	-854400000.000001\\
165000.000000002	732599999.999999\\
-54999.9999999997	-244500000\\
273999.999999998	610700000\\
-329999.999999997	-1953300000\\
-35000.0000000001	2685600000\\
310000.000000002	-1709100000\\
-366000	-610300000\\
239000.000000002	2197500000\\
145000	-1587300000\\
-200000.000000001	122300000\\
-184000.000000004	122100000\\
-72999.9999999995	122000000.000001\\
402000	854400000.000001\\
-35000.000000001	-1830900000\\
-330000.000000001	1342700000\\
310000	-366200000\\
-36000.0000000022	366200000\\
-218999.999999999	-1464800000\\
109999.999999999	2197200000\\
127000.000000002	-1342800000\\
-293000	-243900000\\
165999.999999999	732100000\\
273999.999999998	610500000.000001\\
-512999.999999999	-2319300000\\
220999.999999999	1953000000\\
272999.999999999	488499999.999999\\
-293000.000000001	-2075400000\\
313000.000000005	1953100000\\
-330000.000000003	-1342500000\\
235999.999999999	243800000.000001\\
-51999.9999999987	1220900000\\
-350000.000000002	-2685600000\\
312000	3051900000\\
53999.9999999976	-1587200000\\
-345999.999999997	-243800000\\
493000	854100000\\
-182000.000000002	366600000\\
-312999.999999997	-2319700000\\
-54000.0000000029	2319700000\\
366999.999999998	-366500000\\
-19999.9999999978	-732200000\\
57000.0000000013	732200000\\
-75000.0000000037	-488100000.000001\\
-109000.000000001	-244300000\\
182000.000000001	488500000.000001\\
-35000.000000001	610099999.999999\\
-166000.000000001	-1586800000\\
129000.000000001	1220900000\\
-220000	-1099000000\\
420999.999999999	1709200000\\
-238000.000000001	-976599999.999999\\
-202000.000000002	-1098600000\\
93000.0000000026	1953000000\\
364999.999999997	-488000000\\
-183000	-854900000.000001\\
-109000.000000001	244500000\\
-75000.0000000002	488100000\\
258000.000000002	3.5527136788005e-07\\
-183000	-854300000\\
-148000.000000003	1220300000\\
167000.000000003	-1342300000\\
51999.9999999978	1708700000\\
113000.000000001	-1831000000\\
-277000.000000001	1464800000\\
164999.999999999	-610300000\\
203000.000000001	-244099999.999999\\
-422000.000000001	3.5527136788005e-07\\
-92000.0000000005	-3.5527136788005e-07\\
255999.999999999	854300000\\
221000.000000002	-365999999.999999\\
-91999.999999997	-366300000\\
-128000.000000004	-244000000.000001\\
53999.9999999994	488000000\\
-35999.9999999978	-121700000\\
999.999999998557	-122400000\\
-149000.000000001	122200000\\
204000.000000001	100000.0000003\\
144999.999999998	854300000\\
-457000.000000001	-2075100000\\
219000	1342800000\\
91999.9999999978	610299999.999999\\
-54999.999999997	-1342800000\\
18999.9999999975	1098800000\\
-148000	-854600000\\
110999.999999997	-200000.0000006\\
54000.0000000047	1343100000\\
238999.999999998	-610400000\\
-54999.9999999962	-1465200000\\
-311000.000000004	1099200000\\
107999.999999999	121500000\\
-327000	-243600000\\
346000	731900000\\
-238000	-1464400000\\
368000.000000002	1586600000\\
-75000.0000000028	-976400000\\
-238000	-122100000\\
147000	976500000\\
1000.000000003	-854399999.999999\\
16999.999999995	366200000\\
-182999.999999999	-732500000.000001\\
365999.999999999	1709100000\\
-35999.9999999987	-1342900000\\
-109999.999999997	-610299999.999998\\
-164000.000000003	1586900000\\
15999.9999999991	-1220600000\\
39000.0000000015	366099999.999999\\
237000	1220700000\\
-238000.000000003	-2075100000\\
0	854400000\\
310999.999999999	1220600000\\
-567000.000000001	-2807200000\\
127999.999999999	2807100000\\
494000.000000002	-243900000\\
-384000.000000001	-2441300000\\
-219999.999999999	1464700000\\
586000.000000002	1831000000\\
-238000	-2685300000\\
-237999.999999999	121699999.999999\\
274999.999999999	2319700000\\
89999.9999999972	-1465000000\\
-71999.9999999983	-976699999.999999\\
-292000.000000001	976800000\\
-75000.000000001	244100000.000001\\
146999.999999998	-488500000\\
274000.000000001	1099000000\\
-52999.9999999955	-1099100000\\
-1000.00000000211	122600000.000001\\
16999.9999999977	365700000\\
-216999.999999996	-1220300000\\
124999.999999996	1830900000\\
-124999.999999997	-1220900000\\
106999.999999998	366600000\\
-163000	-200000.000000422\\
127999.999999998	-200000.000000067\\
256000.000000001	488700000\\
-458000.000000001	-732800000\\
109999.999999999	-732099999.999999\\
165000.000000001	2563200000\\
-72999.9999999986	-2685500000\\
-36999.9999999999	1343000000\\
54999.9999999997	-200000.000000955\\
256000	244100000.000001\\
-219000.000000001	-1098500000\\
-146999.999999998	-122100000\\
164999.999999999	2197200000\\
72999.9999999968	-2319300000\\
-311000	610500000\\
-36999.9999999999	243800000\\
294000.000000003	610600000\\
254999.999999998	100000.0000003\\
-201000.000000003	-1709300000\\
-164000	1343000000\\
54000.0000000011	-366199999.999999\\
-182999.999999997	243900000\\
-128000	-976300000\\
366999.999999998	2075200000\\
-129999.999999998	-1343000000\\
202999.999999998	200000.0000006\\
-19000.000000001	-199999.999999623\\
-236999.999999999	299999.999999834\\
108000.000000001	121900000\\
999.999999998557	488100000\\
-18000.0000000016	-1220400000\\
-147000	976500000\\
73999.9999999998	-122200000\\
237000	-122099999.999999\\
-35999.9999999987	488600000\\
-220000	-1343200000\\
54999.9999999997	854800000\\
166000	854399999.999999\\
-186000.000000002	-1586900000\\
186000.000000004	1098500000\\
34999.9999999993	366400000\\
-90999.9999999993	-2075200000\\
-164000.000000001	1586700000\\
-203000.000000002	-365999999.999999\\
439999.999999999	610200000\\
128000.000000003	199999.999999889\\
-200000	-1098800000\\
107999.999999999	488399999.999999\\
-256000.000000003	366000000\\
-107999.999999998	-854300000\\
327000.000000002	1098600000\\
-511000	122100000\\
383999.999999998	-2563600000\\
145999.999999999	4394700000\\
-109000	-3662400000\\
165000.000000001	1343200000\\
-275999.999999997	-122400000\\
-91000.0000000002	-366200000\\
146999.999999999	976800000\\
55000.0000000006	-732600000\\
35999.9999999996	-8.88178419700125e-08\\
-54999.9999999988	-8.88178419700125e-08\\
-218999.999999999	122100000\\
-239000.000000002	-488200000\\
770000.000000001	1708800000\\
-201999.999999996	-1952900000\\
73999.9999999981	1220500000\\
-385999.999999998	-976499999.999999\\
-16000.0000000009	-244100000.000001\\
437000.000000001	2685500000\\
-164000.000000001	-3051700000\\
37000.0000000026	854500000\\
-201000	732399999.999999\\
34999.9999999993	-976700000\\
240000.000000003	854700000\\
-239000.000000001	-244300000\\
145999.999999999	-488200000\\
-107999.999999998	244200000\\
-405000.000000003	-99999.9999994117\\
732999.999999998	610400000.000001\\
-347999.999999997	243999999.999999\\
-54000.0000000003	-2319100000\\
165000.000000002	2441300000\\
-93000.0000000035	-976600000\\
18999.9999999992	99999.9999995893\\
-55000.0000000006	365999999.999999\\
54999.9999999988	-488099999.999999\\
109999.999999999	610400000\\
-54999.9999999988	-610600000\\
-202000.000000003	122400000.000001\\
422000	121900000\\
-127999.999999998	1098500000\\
-476999.999999998	-3173700000\\
72999.9999999977	3051900000\\
641999.999999998	-366400000\\
-255999.999999998	-366200000\\
-514999.999999999	-2197200000\\
222000	3051700000\\
255999.999999998	-488200000\\
-93000	-1220800000\\
147999.999999998	854600000\\
-366000	-366300000\\
400999.999999999	-122000000\\
1000.00000000033	1098500000\\
-437999.999999999	-1953000000\\
546999.999999998	1464900000\\
-291999.999999998	610100000\\
1.77635683940025e-09	-2441100000\\
1.77635683940025e-09	2197100000\\
-310999.999999999	-976600000\\
457000	488400000\\
-347000	-610500000\\
127999.999999997	199999.999999978\\
274000.000000001	1586700000\\
-55000.0000000015	-1342600000\\
-217999.999999998	-732500000\\
108000.000000002	1342800000\\
-145999.999999998	-610500000\\
237999.999999999	-243800000\\
-146000	1220300000\\
-1000.00000000033	-1586600000\\
93000	976400000\\
-239000	99999.9999999446\\
257000.000000002	-854700000\\
-294000	854800000\\
311999.999999997	121800000\\
18000.0000000016	-244000000\\
-182000	-976600000\\
163000.000000003	1342800000\\
-236999.999999999	-366200000\\
-238000	-610400000\\
456999.999999998	854500000\\
220999.999999999	732400000\\
-276000.000000002	-3051600000\\
37000.0000000017	2685200000\\
-382999.999999999	-243700000\\
436999.999999997	-1343100000\\
93000	2197400000\\
-420999.999999999	-2685600000\\
291999.999999997	1464900000\\
-255999.999999998	976500000\\
421999.999999999	-2441400000\\
-330000.000000001	2563600000\\
-91999.9999999987	-2319400000\\
128000.000000001	1586700000\\
219999.999999999	-365900000\\
-219000	-122200000\\
-37999.9999999967	-366300000\\
-72000.0000000027	732600000\\
329000	-732500000\\
-366000.000000001	854400000\\
73000.0000000022	-1464800000\\
420999.999999998	2929900000\\
-90999.9999999984	-3418200000\\
-310999.999999999	1586800000\\
-166000.000000004	-121699999.999999\\
220000.000000002	-244400000.000001\\
238999.999999999	1342800000\\
-293000	-2319200000\\
310000	1464600000\\
-91000.0000000028	1342900000\\
-310999.999999999	-3784000000\\
-220000.000000001	2197000000\\
587000.000000001	1953200000\\
-38000.0000000002	-3662100000\\
-348000.000000002	2441600000\\
276000.000000002	-854900000\\
72000.0000000009	366600000\\
-90000.0000000016	-488600000\\
-204000.000000001	244500000\\
21000.0000000008	-488600000\\
327999.999999999	1709100000\\
-256000.000000002	-2563400000\\
72999.9999999995	2197100000\\
74000.0000000016	-732299999.999999\\
-202000.000000002	-976600000\\
182999.999999999	1831100000\\
-127999.999999998	-1953200000\\
-145999.999999999	1098800000\\
255999.999999999	487900000\\
37000.0000000008	-365700000\\
-38000.0000000038	-1221200000\\
56000.0000000009	2075600000\\
-73000.0000000004	-1831200000\\
-202000.000000001	610200000\\
110000	366400000\\
17999.9999999989	-488400000\\
1000.000000003	488500000\\
35999.9999999996	-122400000\\
201000	299999.999999834\\
-54000.000000002	-200000.000000422\\
-532000	-1098600000\\
330000.000000002	610500000\\
367000	3173600000\\
-367000.000000001	-5493000000\\
-56000	3418000000\\
277000.000000001	121899999.999999\\
-148000.000000004	-1953100000\\
-164999.999999998	1343000000\\
220999.999999999	-122300000\\
-148000.000000001	99999.9999994117\\
19999.9999999996	-854600000.000001\\
217999.999999999	2075400000\\
-255000.000000001	-2197600000\\
89999.999999999	732900000.000001\\
39000.0000000024	487800000\\
15999.9999999965	-243900000\\
-181999.999999999	-732399999.999999\\
255999.999999998	1586900000\\
-164000.000000001	-2563600000\\
-38000.0000000011	2807800000\\
111000	-976800000\\
218999.999999999	-732100000\\
-218999.999999998	610000000\\
-110000.000000002	-488000000\\
-219999.999999999	366000000\\
-55000.0000000006	-244000000\\
475999.999999997	976600000\\
128000.000000001	-366300000\\
-71999.9999999992	-1220800000\\
-313000.000000003	854600000.000001\\
129000.000000001	610500000.000001\\
37000.0000000008	-610600000\\
90999.9999999966	-732400000\\
-346999.999999999	1221000000\\
236000	-1343200000\\
1999.999999998	2441700000\\
-73999.9999999998	-3173900000\\
129000.000000002	2807600000\\
-38000.0000000029	-1831100000\\
55000.0000000015	854500000\\
-292000.000000003	-366000000\\
-17999.9999999971	-488600000\\
546999.999999999	1587200000\\
-417999.999999998	-732800000\\
125999.999999998	-1830500000\\
-146000.000000001	3295300000\\
202000.000000002	-2807200000\\
-239000.000000003	1098600000\\
330000.000000001	1464500000\\
-181999.999999999	-3539600000\\
-39000.0000000024	3417700000\\
277000.000000001	-854400000\\
-606000	-3051800000\\
588000.000000003	5004900000\\
-295000.000000004	-2685500000\\
36999.9999999999	-1220700000\\
-16999.9999999995	3295700000\\
199000	-2929400000\\
-234999.999999997	1586800000\\
-21000.0000000026	-1465000000\\
277000.000000002	2441600000\\
-94000.0000000021	-2319300000\\
-71000.0000000006	976399999.999998\\
-130000	0\\
221000	-610199999.999999\\
-74000.0000000025	1098600000\\
-16999.9999999977	-732600000\\
-75000.0000000019	-487999999.999999\\
109999.999999999	1464600000\\
-328000.000000001	-1708900000\\
310000	1220700000\\
237999.999999999	488300000\\
-255999.999999998	-1953000000\\
1000.00000000033	1586600000\\
198999.999999998	-243900000.000001\\
-289999.999999996	-610400000\\
51999.9999999952	3.5527136788005e-07\\
56999.9999999995	976500000\\
-129000	-1220600000\\
-55000.0000000006	732400000\\
184000	244100000\\
34999.9999999993	-244200000\\
57000.0000000004	-366100000\\
15999.9999999973	610300000\\
-437999.999999999	-1220700000\\
311000.000000002	732500000.000001\\
110000.000000002	1708900000\\
-18000.0000000007	-2685700000\\
-330000.000000001	732799999.999999\\
457000.000000002	610099999.999999\\
55999.9999999974	2197200000\\
-329999.999999998	-6225400000\\
-55000.0000000033	5248900000\\
-183000	-1220600000\\
275000.000000001	-732500000\\
91000.0000000002	732400000.000001\\
74000.0000000016	610500000\\
-56000.0000000009	-1465100000\\
37999.9999999994	244300000\\
-277000.000000001	854600000\\
-15999.9999999982	-1098800000\\
218999.999999999	732500000\\
54999.9999999997	610200000\\
-220000.000000002	-1220400000\\
164999.999999999	-400000.000000222\\
55000.0000000006	1465300000\\
-440000.000000002	-1831500000\\
404000.000000001	610700000\\
-93000	2075000000\\
110999.999999998	-3784200000\\
-73999.9999999981	3174100000\\
-37000.0000000008	-2197700000\\
-71999.9999999974	1709400000\\
292999.999999997	-1098800000\\
-404000	366000000\\
165999.999999997	610800000\\
291000.000000003	-1587400000\\
-345000	1831500000\\
88999.9999999977	-1465200000\\
-17000.0000000021	1343000000\\
-146999.999999999	-2075200000\\
385000.000000001	2807300000\\
-183000.000000001	-1586400000\\
-148000	-1221200000\\
167000.000000002	2319800000\\
53999.9999999958	-732900000\\
-330999.999999999	-1342400000\\
94000.0000000003	2319200000\\
309999.999999997	-1831000000\\
-238999.999999997	976299999.999999\\
-125999.999999999	-1464400000\\
529999.999999997	2685200000\\
-330999.999999999	-1342600000\\
3000.00000000011	-2319500000\\
-186000.000000001	3418200000\\
1999.99999999978	-2197500000\\
455999.999999999	2807800000\\
-273000	-4516700000\\
-256999.999999999	3662100000\\
475999.999999995	610400000\\
-73999.9999999981	-4638600000\\
-401000.000000001	4272200000\\
272999.999999999	-1220300000\\
37999.9999999985	365700000\\
-185000	-1586400000\\
129999.999999998	1708700000\\
-36999.999999999	-854500000\\
-74000.0000000007	244200000\\
18999.9999999992	122200000\\
109000	-300000.0000001\\
-109000	-488100000\\
183000.000000002	488400000\\
-108999.999999997	610200000\\
-76000.000000004	-1953200000\\
204000.000000005	2319400000\\
0	-1586700000\\
-222000.000000004	732099999.999999\\
75000.0000000037	-1098400000\\
-17999.9999999989	1830800000\\
-74000.0000000016	-2074800000\\
330000.000000001	2685100000\\
-312000.000000003	-2685200000\\
92000.0000000005	1464700000\\
35999.9999999996	-122100000\\
-236999.999999997	-1586900000\\
366000.000000001	2563600000\\
-146000.000000001	-854700000.000001\\
-2000.00000000156	-1586800000\\
130000	2441500000\\
-91999.9999999996	-1709100000\\
-145999.999999999	-244200000\\
-184000.000000003	1465000000\\
384000	-1220800000\\
-180999.999999997	1586900000\\
198999.999999996	-2441300000\\
-216999.999999997	2197100000\\
-168000.000000001	-1708800000\\
350999.999999999	1464600000\\
180000.000000001	854799999.999998\\
-308000	-4028500000\\
33999.999999998	3906200000\\
130000	-1220600000\\
-239999.999999999	-732500000\\
19999.9999999996	610600000\\
201000	365800000\\
-164999.999999999	-610100000\\
-54999.9999999997	244200000.000001\\
1000.00000000122	-300000.000000367\\
272999.999999996	366700000\\
-53999.9999999976	-500000.000000256\\
-202000.000000003	-1952900000\\
257000.000000001	3418000000\\
-255999.999999998	-2929700000\\
108999.999999995	1464700000\\
-17999.9999999998	-610100000.000001\\
-165000	-300000.000000011\\
201999.999999997	366500000.000001\\
365000.000000001	1708900000\\
-620999.999999999	-4638900000\\
145999.999999998	3784500000\\
-56000.0000000009	-122299999.999999\\
184000	-1586700000\\
111000.000000004	1220500000\\
-295000.000000004	-1708900000\\
-146000	1831000000\\
202000.000000003	-488099999.999999\\
311000.000000002	610099999.999998\\
-73000.0000000004	-976300000.000001\\
-164999.999999998	121800000\\
-91999.9999999996	-854400000\\
38000.0000000011	1465000000\\
236000	732300000\\
-235999.999999999	-2441400000\\
-1000.00000000033	854300000\\
218999.999999999	1831500000\\
-181999.999999998	-2808000000\\
-183000	1587100000\\
384000.000000001	610199999.999999\\
-92000.0000000014	-1342600000\\
-327999.999999998	-488400000\\
345999.999999996	2197300000\\
-164000.000000001	-1098700000\\
37000.0000000017	-1464600000\\
-36999.9999999999	2807400000\\
-35999.9999999978	-2929800000\\
254999.999999999	3296200000\\
-35000.0000000028	-3051900000\\
-276000	1220600000\\
54999.9999999997	122300000\\
129000	243900000\\
36999.9999999981	-488100000.000001\\
-19999.9999999996	244000000\\
-255000.000000003	-1342700000\\
310000.000000001	2685700000\\
20000.0000000022	-1343200000\\
-367000.000000002	-1586400000\\
293000	2807300000\\
163999.999999998	-1709000000\\
-199999.999999999	200000.000000067\\
16999.9999999986	976400000\\
-52999.9999999973	-1220600000\\
-351000	610200000\\
460999.999999998	199999.999999889\\
143999.999999999	1098500000\\
-235999.999999999	-2563500000\\
-148000.000000002	2319500000\\
-35999.9999999996	-1953300000\\
255999.999999998	2807700000\\
-127999.999999998	-2807700000\\
92000.0000000005	1220900000\\
72999.9999999995	365900000\\
-111000.000000001	-1098200000\\
167000.000000002	1342300000\\
70999.9999999971	-976300000\\
-566000.000000002	-976500000\\
274000	2685400000\\
-8.88178419700125e-10	-2075200000\\
-109000.000000001	366300000\\
272999.999999999	976500000\\
-181999.999999999	-1586900000\\
328999.999999999	1831100000\\
-256000.000000001	-854600000\\
-255999.999999998	-1952900000\\
511999.999999999	3539700000\\
-164000.000000001	-976300000\\
-92000.0000000005	-3173800000\\
-183000.000000002	3539700000\\
54000.0000000011	-854000000\\
55999.9999999991	-610800000\\
348000.000000004	1953300000\\
-220000.000000003	-4516400000\\
-109999.999999999	5248600000\\
-55000.0000000033	-3295400000\\
37000.0000000026	1098100000\\
-55000.0000000006	-243699999.999999\\
346999.999999998	1098500000\\
-163999.999999998	-1831200000\\
110000.000000002	976600000\\
-128000	244400000\\
-239000	-1343100000\\
459000	2685800000\\
-514000.000000003	-4516800000\\
239000.000000002	4882900000\\
17999.9999999998	-1709000000\\
-256000	-1586900000\\
291999.999999998	976600000\\
38000.0000000038	1953100000\\
-130000.000000002	-3051900000\\
277000.000000002	1953400000\\
70999.9999999979	-122200000\\
-475000	-1953300000\\
128000	2441600000\\
-90999.9999999975	-1464800000\\
90999.9999999975	-99999.9999994117\\
-17999.9999999998	1831000000\\
309999.999999999	-1586700000\\
-70999.9999999979	121700000.000001\\
-368000.000000003	-731900000\\
73000.0000000013	1098100000\\
368000.000000001	854800000\\
-165999.999999998	-1464800000\\
-109000.000000001	-610600000.000001\\
71999.9999999965	1587100000\\
-91000.0000000002	-610400000\\
36999.9999999999	244100000\\
-165000.000000002	-854400000\\
36000.0000000005	610400000\\
311999.999999999	976200000\\
-55000.0000000015	-1342300000\\
-91999.9999999996	-366400000\\
146999.999999998	1464600000\\
-54999.9999999988	-609900000\\
-421000	-976900000\\
109000.000000001	732600000\\
55999.9999999991	1342600000\\
309999.999999998	-1830900000\\
-145000.000000001	244100000\\
-19000.0000000001	1098600000\\
-73000.0000000013	-1342800000\\
-999.999999998557	610500000.000001\\
239000	488199999.999999\\
-109999.999999997	-244299999.999999\\
0	-854200000\\
220000	1098400000\\
-385000.000000004	-1220600000\\
-54999.9999999988	854400000\\
54999.9999999988	244300000.000001\\
255999.999999999	-488400000\\
-34999.9999999975	488300000\\
-112000.000000003	-976500000\\
-16999.9999999986	854300000.000001\\
-219000	-732100000\\
271999.999999998	1220500000\\
314000	-488400000\\
-440999.999999999	-732200000\\
-182000.000000002	-122200000\\
200000.000000004	1343000000\\
293999.999999997	-488700000\\
-348999.999999999	-1220300000\\
222000.000000001	1952900000\\
-222000.000000003	-1342700000\\
-16999.9999999995	-244100000\\
401999.999999999	2197100000\\
-329999.999999999	-3295700000\\
-89999.9999999972	2319200000\\
291999.999999999	-244100000.000001\\
19000.0000000028	366400000\\
17000.0000000003	-2075600000\\
-181000.000000003	1587200000\\
126000	-244100000\\
-439000.000000003	-244399999.999999\\
422000.000000001	976899999.999999\\
-294000.000000002	-1221000000\\
294000.000000003	488400000.000001\\
73000.0000000022	488400000.000001\\
-330000.000000002	-1343100000\\
403000	1709400000\\
-439000	-1221000000\\
310999.999999998	5.32907051820075e-07\\
-19000.0000000001	1465100000\\
-274000	-2075400000\\
256000	1098500000\\
-128000.000000002	-365800000\\
56000.0000000027	609900000\\
108000.000000002	244600000\\
-236999.999999999	-1587400000\\
17999.9999999998	854800000\\
182999.999999997	1220600000\\
-16999.9999999986	-1830900000\\
-276999.999999999	488100000\\
221999.999999999	854600000\\
217999.999999997	-488500000\\
-199999.999999998	-610000000\\
-221000.000000002	-122400000\\
165999.999999999	1343100000\\
273000.000000001	487899999.999999\\
-273000.000000001	-2807200000\\
347000	1830600000\\
-346999.999999999	366700000.000001\\
-367000.000000003	-2075600000\\
256000	2685700000\\
367000.000000003	-1464800000\\
-403000	2.66453525910038e-07\\
366000.000000003	854400000\\
36999.999999999	-732300000.000001\\
-202000.000000001	-244400000\\
-164000.000000001	366600000.000001\\
90999.9999999993	243900000\\
127999.999999998	-366200000\\
91000.0000000002	244100000\\
-328000.000000001	-488100000\\
146000.000000001	610200000\\
71999.9999999983	-122000000\\
2000.00000000156	-610500000\\
-312999.999999999	976900000\\
146999.999999998	-854900000\\
274999.999999999	1098800000\\
-55000.0000000015	-854300000\\
-329999.999999999	-1221000000\\
111000	2563700000\\
455999.999999999	-610600000\\
-457000.000000001	-2807300000\\
56000.0000000036	4394300000\\
235999.999999996	-2685500000\\
-15999.9999999956	-488300000\\
-387000.000000005	1342900000\\
94000.0000000038	488100000.000001\\
400999.999999996	-1342600000\\
-419999.999999998	-366300000\\
164000.000000003	2441400000\\
110000.000000001	-2807600000\\
-202000.000000001	1586900000\\
-144000.000000001	-732500000\\
198000.000000001	732700000\\
130999.999999998	-122400000\\
-148000.000000002	-610200000\\
-17999.999999998	610400000\\
256999.999999998	-99999.9999995893\\
-73999.9999999989	-854399999.999999\\
-219999.999999996	854400000\\
-35999.9999999996	-610400000\\
55000.0000000006	976800000\\
90999.9999999984	-854700000\\
-255999.999999998	-122000000\\
365999.999999998	854500000\\
72999.9999999977	121900000.000001\\
-217999.999999996	-2074800000\\
-112000.000000004	2196900000\\
38000.000000002	-732399999.999999\\
181999.999999997	244400000.000001\\
-35999.9999999987	-488500000\\
90999.9999999993	488399999.999999\\
-218000.000000001	-854700000.000001\\
34000.0000000016	976800000.000001\\
56999.9999999986	-366200000\\
-73999.9999999989	243799999.999999\\
-72999.9999999968	-1098200000\\
292999.999999999	2197100000\\
-92000.0000000014	-2319500000\\
-36999.9999999981	1587100000\\
-53000.0000000008	-1464900000\\
-39000.0000000015	1464900000\\
-16999.9999999986	-732500000\\
329000	488300000\\
-419999.999999998	-854400000.000001\\
35999.999999996	-199999.999999889\\
108999.999999999	1220800000\\
184000	-488099999.999999\\
-257000	-610600000.000001\\
184000	488300000\\
-74000.0000000025	-854300000\\
92000.0000000023	2075000000\\
-348000	-3051600000\\
54999.9999999997	3173600000\\
420999.999999999	-1342500000\\
0	-732700000\\
-456999.999999999	-243800000\\
345999.999999996	2807400000\\
-89999.9999999998	-3906400000\\
-292999.999999999	2563800000\\
456999.999999998	243900000\\
-236999.999999999	-1586700000\\
-92999.9999999964	243900000\\
257999.999999997	1953300000\\
-294999.999999997	-2929700000\\
258999.999999999	2441200000\\
52000.0000000005	-1342400000\\
-399999.999999999	243800000\\
290999.999999999	122200000\\
-36000.0000000005	244200000\\
17999.999999998	-488300000\\
1000.00000000122	610100000\\
-129000.000000001	-854000000\\
220000	854000000\\
-129000	-488000000\\
-255000.000000001	-610400000\\
584999.999999999	2685400000\\
-220000	-3539700000\\
-200000	1586600000\\
236999.999999999	976600000.000001\\
-1.77635683940025e-09	-1831000000\\
-476000	488500000\\
458000	1708500000\\
73000.0000000004	-1952700000\\
-457000.000000002	-488399999.999999\\
310000.000000003	2929500000\\
-55000.0000000015	-2929400000\\
277000.000000002	1464700000\\
-278000	-122100000\\
3000.00000000011	-1220700000\\
-111999.999999999	1587000000\\
-53999.9999999994	-488300000\\
312000	-488300000\\
-147999.999999998	732300000\\
-109000.000000001	-366000000\\
90999.9999999984	-244300000\\
294000	732600000.000001\\
-368000	-1465100000\\
21000.0000000017	1587100000\\
271999.999999997	-99999.9999999446\\
-163999.999999998	-1342600000\\
-272999.999999998	366000000\\
491999.999999996	1464900000\\
-309999.999999998	-2441300000\\
127999.999999999	2685500000\\
-73000.0000000004	-2441500000\\
-56000.0000000009	1831100000\\
349000.000000004	-122000000\\
-183000.000000001	-1464900000\\
-202000.000000003	122099999.999998\\
-183000.000000001	1220700000\\
329000	-366300000\\
999.999999999446	366300000.000001\\
110000.000000003	-488300000\\
71999.9999999965	-610300000\\
-492999.999999998	366100000.000001\\
-37999.9999999985	854500000\\
348999.999999998	-610099999.999999\\
145999.999999998	121599999.999999\\
73000.0000000031	-121600000\\
-310000.000000001	-244400000.000001\\
-148000.000000003	-610200000\\
37000.0000000026	2074900000\\
275000	-1708600000\\
-54999.999999997	854200000\\
109999.999999999	-366100000\\
-366000.000000002	-1220600000\\
163999.999999999	2929500000\\
183000	-2197200000\\
-328999.999999998	-366099999.999999\\
311000.000000001	2075000000\\
1000.00000000122	-1830800000\\
-258000.000000004	243799999.999999\\
203000.000000005	976900000\\
199999.999999996	-488500000\\
-456999.999999999	-732400000\\
1000.000000003	488400000\\
143999.999999995	366200000.000001\\
75000.0000000037	121900000\\
127999.999999998	-244000000\\
-312000.000000001	-1342800000\\
184000.000000001	2685600000\\
91000.0000000002	-1831200000\\
-309999.999999997	-366100000\\
216999.999999995	1098700000\\
-125999.999999997	-122300000\\
110000.000000001	-365900000\\
199999.999999996	609999999.999999\\
-218999.999999997	-1220500000\\
127999.999999999	976700000\\
-438000.000000001	-732700000\\
417999.999999999	1709100000\\
-87999.9999999992	-2197200000\\
-39000.0000000015	1586800000\\
-183000	-1953100000\\
202999.999999999	2563500000\\
70999.9999999997	-1464700000\\
-33999.9999999963	488000000\\
198999.999999996	-121900000\\
-290999.999999998	-610400000\\
52999.9999999964	366300000\\
-107999.999999999	244100000\\
-111000.000000001	-200000.000000422\\
348000	-121800000\\
-73000.0000000004	488200000\\
-257000	-2075200000\\
240000	3417900000\\
34000.0000000007	-2685400000\\
-35000.0000000001	976299999.999999\\
-91999.9999999996	-121800000.000001\\
348000.000000002	610300000\\
-329000.000000001	-1709100000\\
-1000.00000000211	2075300000\\
-37000.0000000017	-2075300000\\
-52999.9999999982	1465000000\\
254000	366100000\\
-34000.0000000007	-854600000\\
-113000	-732100000\\
-307999.999999998	1098400000\\
418999.999999997	-122100000\\
54999.9999999997	488400000\\
-365999.999999998	-1953300000\\
276000.000000002	2930000000\\
217999.999999999	-1709200000\\
-219000	-732500000\\
-18000.0000000007	854700000\\
-166000	-199999.999999534\\
-163000	-121900000\\
492999.999999998	1098600000\\
-238000	-1831200000\\
20000.000000004	1098900000\\
15999.9999999938	121699999.999999\\
56000.0000000018	-854100000\\
-92000.0000000005	243800000\\
184000	1221000000\\
-257000.000000002	-2197400000\\
201000	2563400000\\
-90000.0000000007	-2563400000\\
-203000	1342800000\\
128999.999999998	366100000\\
146000.000000003	-487999999.999999\\
219999.999999998	609899999.999999\\
-713999.999999998	-2563000000\\
531000.000000002	3905800000\\
218999.999999998	-2685100000\\
-236999.999999997	1098300000\\
-440000.000000001	-2197200000\\
385000.000000002	5127100000\\
53999.9999999985	-6225700000\\
-219000	4150300000\\
293000.000000003	-854300000\\
-311000.000000002	-976700000\\
-130000.000000002	-610199999.999999\\
460000.000000004	3295700000\\
109000	-1586800000\\
-458000.000000001	-2929700000\\
128000.000000001	4028300000\\
38000.0000000002	-1709000000\\
144999.999999996	-244000000\\
-219999.999999998	488099999.999999\\
239999.999999999	0\\
-423000.000000003	-365999999.999999\\
239000.000000003	-122299999.999999\\
-129000.000000003	610400000\\
239000	610500000\\
-165999.999999999	-1831200000\\
313000.000000001	2197300000\\
-130000	-2441400000\\
-162999.999999999	1831000000\\
-1999.99999999889	-1464800000\\
93999.9999999985	2075300000\\
106999.999999999	-2075300000\\
-271999.999999998	976500000\\
34999.9999999957	122200000\\
220000	-488399999.999999\\
-73000.0000000013	732500000\\
-293000	-1709000000\\
146999.999999999	2319400000\\
309999.999999998	-610599999.999999\\
-126999.999999998	-1708600000\\
-202000.000000002	1708700000\\
-255999.999999999	-488200000\\
439000	-854400000\\
402999.999999999	3417600000\\
-713999.999999998	-4882200000\\
218999.999999999	2318800000\\
38000.000000002	854700000\\
-73999.9999999998	-1098600000\\
129000	488300000\\
52999.9999999973	-488400000\\
-125999.999999998	244100000\\
-38999.9999999997	299999.999999834\\
-89000.0000000013	-122400000\\
-111000.000000002	-366000000\\
347000	1708800000\\
112000.000000002	-1586800000\\
-149000.000000003	122100000.000001\\
-16999.9999999995	122100000.000001\\
-256000.000000002	-244300000\\
-8.88178419700125e-10	976700000\\
237000.000000001	-488400000\\
74000.0000000007	-244099999.999999\\
36999.9999999999	244300000.000001\\
-166000.000000004	-854600000\\
-108999.999999998	1342600000\\
91999.9999999987	-1098400000\\
-111000.000000002	854400000\\
238000.000000002	-122100000\\
110999.999999999	-122000000\\
-18999.9999999992	-122000000\\
-366000.000000003	-1465100000\\
146000.000000001	3296100000\\
1000.00000000033	-2807600000\\
199999.999999998	1586800000\\
-309999.999999999	-1708900000\\
110000	2319300000\\
71999.9999999974	-1831100000\\
91999.9999999987	976600000\\
-35999.999999996	-732300000\\
-367000.000000001	121900000\\
182999.999999999	610300000\\
999.999999998557	-488000000\\
219000.000000003	610100000\\
-145999.999999998	-732300000.000001\\
145999.999999997	-122200000\\
-182999.999999999	732600000\\
-53999.9999999985	-854600000.000001\\
15999.9999999982	732399999.999999\\
369000.000000001	244300000.000001\\
-111999.999999999	243800000.000001\\
-329000.000000003	-3173400000\\
74000.0000000016	4272200000\\
17000.0000000021	-2929700000\\
367999.999999998	2807800000\\
-56999.9999999986	-2685800000\\
-328000.000000002	244300000\\
-129000.000000001	1098800000\\
294000.000000002	121600000\\
16999.9999999986	-609800000\\
-255999.999999998	-122500000\\
-54000.0000000003	99999.9999999446\\
217999.999999999	488500000\\
56999.9999999986	121800000\\
108000	200000.000000156\\
-182000.000000001	-1709100000\\
-109999.999999998	2441300000\\
91000.0000000011	-1830700000\\
109999.999999998	1342400000\\
-36000.0000000005	-732199999.999999\\
54999.9999999997	121900000\\
-20000.0000000005	-243900000\\
-162000	243900000\\
-93999.9999999976	-122000000\\
183999.999999998	488500000\\
164999.999999998	-400000.000000222\\
-54999.9999999979	-1220400000\\
-402999.999999997	610200000\\
494000.000000001	1709200000\\
-274000	-2808000000\\
-19000.000000001	2319800000\\
37999.9999999994	-1831500000\\
89999.999999999	976900000\\
183999.999999998	732200000.000001\\
-439999.999999998	-1952900000\\
237999.999999998	1952900000\\
148000.000000003	-976500000\\
-167000.000000002	-366199999.999999\\
-219000.000000001	-121900000\\
184000.000000003	2197000000\\
182999.999999998	-2319200000\\
-166000	610400000\\
128999.999999998	732199999.999998\\
-109999.999999999	-1464600000\\
-37000.0000000035	1220600000\\
-108999.999999999	-488299999.999999\\
346999.999999999	732500000.000001\\
-366000	-2319400000\\
221000.000000003	4150400000\\
-2000.00000000333	-4150300000\\
-419999.999999997	1464600000\\
714000.000000001	2563700000\\
-36999.999999999	-3906300000\\
-347000	976600000\\
-166000.000000003	1220500000\\
56000.0000000018	-1098300000\\
108999.999999997	976300000\\
37000.0000000026	-366100000\\
37000.0000000017	121999999.999999\\
-1000.00000000122	-854500000\\
148000.000000001	1587000000\\
-238999.999999999	-2319300000\\
-74000.0000000043	1831000000\\
166000	-366400000\\
128000	610600000.000001\\
-129000.000000002	-1953100000\\
-327999.999999998	1342600000\\
345999.999999998	366200000\\
-127000	-732200000\\
17999.9999999989	365900000\\
220000.000000001	366600000\\
-18000.0000000016	-977000000.000001\\
-202999.999999999	366600000.000001\\
-145000	-244399999.999999\\
165000.000000002	854600000\\
16999.9999999995	-244200000\\
56000.0000000018	-488200000.000001\\
-1000.000000003	3.5527136788005e-07\\
93000.0000000044	1464800000\\
-111000.000000001	-3174000000\\
146999.999999997	3906700000\\
-129000	-3174300000\\
-438000.000000002	488599999.999999\\
346000.000000003	2197000000\\
478000.000000001	-1342500000\\
-734000.000000003	-1220900000\\
185000.000000003	1098700000\\
143999.999999997	610400000.000001\\
313000.000000001	-122200000\\
-127999.999999997	-365999999.999999\\
-239000.000000003	-1709300000\\
-256000.000000002	1709400000\\
275000	1830600000\\
53999.9999999976	-4027900000\\
111000.000000002	3417600000\\
-312000.000000001	-2441100000\\
220000	2685400000\\
-109999.999999999	-3296000000\\
73000.0000000004	2319600000\\
-91000.0000000028	-488600000\\
-72999.9999999968	399999.999999689\\
309999.999999997	-244600000\\
-272999.999999999	122400000.000001\\
89999.9999999981	-122100000\\
147000.000000001	732199999.999999\\
111000.000000001	-854199999.999999\\
-332000.000000002	-366400000.000001\\
-124999.999999998	488300000.000001\\
472999.999999997	1220700000\\
-327999.999999999	-1830900000\\
128000.000000002	976300000.000001\\
-292000.000000001	-610100000\\
326999.999999998	488100000\\
131000.000000002	244300000\\
-167000.000000002	-610500000\\
-182000	-8.88178419700125e-08\\
-164999.999999997	610500000\\
274999.999999999	-854600000\\
127999.999999998	1342800000\\
91000.0000000002	-1342900000\\
-310000	488600000\\
-74999.9999999984	-610700000\\
183999.999999997	1709200000\\
220000.000000001	-1342800000\\
-184000.000000003	-488500000\\
-199999.999999997	1587300000\\
163999.999999996	-1587200000\\
73000.0000000004	1464800000\\
-128000.000000001	-1342400000\\
109999.999999999	1098100000\\
55000.0000000006	-609800000\\
-110000.000000002	-244600000\\
-146999.999999999	610600000\\
221000	-366300000.000001\\
-148000.000000001	366299999.999999\\
-34999.9999999993	-488399999.999999\\
-2000.00000000067	200000.000000422\\
423000.000000001	1586500000\\
-311999.999999999	-2929100000\\
-238000	1830500000\\
457999.999999998	1099000000\\
-56000.0000000009	-2685800000\\
-200000.000000001	1709300000\\
16999.9999999986	-366500000\\
-72000.0000000001	122100000\\
90999.9999999993	-366000000\\
8.88178419700125e-10	-300000.000000189\\
-311000.000000003	366600000\\
787000.000000002	976100000\\
-531999.999999998	-2563200000\\
-218000.000000002	2563600000\\
237999.999999999	-1953500000\\
199999.999999998	1953500000\\
-53999.9999999994	-2685900000\\
-54000.0000000003	2930100000\\
-332000	-2075600000\\
92999.9999999991	366400000\\
219999.999999999	1342900000\\
238000.000000001	-732699999.999999\\
-202000.000000001	-1220500000\\
-202000.000000002	1098500000\\
130000.000000002	366400000\\
-38000.000000002	-732600000\\
1000.000000003	99999.9999995893\\
34999.9999999957	732300000\\
38000.0000000038	-854400000\\
72999.9999999986	732500000\\
-36999.999999999	-366400000.000001\\
-18000.0000000007	-488100000\\
-238000	610200000.000001\\
73999.9999999998	-244100000\\
345999.999999998	732600000\\
-254999.999999999	-1098900000\\
-311999.999999998	610500000\\
402999.999999997	-488300000\\
56000.0000000018	1220700000\\
-148000.000000001	-1587000000\\
-126999.999999995	732600000.000001\\
511999.999999999	1342600000\\
-713000	-3173700000\\
418999.999999998	2075100000\\
75000.0000000002	732500000.000001\\
-220000	-1709000000\\
71999.9999999992	1342700000\\
-308999.999999997	-1831100000\\
510999.999999998	2441700000\\
110000	-244399999.999999\\
-530999.999999999	-3540100000\\
147000	3662400000\\
165000.000000003	-488500000\\
-129000	-1586900000\\
109999.999999999	1953200000\\
-236999.999999997	-1831200000\\
364999.999999999	1953400000\\
-54999.9999999979	-1709200000\\
-109000.000000001	244100000\\
-146999.999999998	99999.9999994117\\
73999.9999999989	1587000000\\
90000.0000000007	-2075300000\\
-309000	854400000\\
180000	200000.000000244\\
406000	854400000\\
-349999.999999999	-1465000000\\
-109000.000000001	366500000\\
330000	488100000\\
-386000.000000001	-244100000\\
403999.999999998	244100000\\
-366999.999999999	-1220500000\\
93000	2563100000\\
-56000.0000000009	-2807300000\\
238000.000000002	1708800000\\
-72999.9999999986	-854200000\\
-18999.9999999983	1098200000\\
-35000.000000001	-1586600000\\
-295000.000000001	1220600000\\
496000	-366200000\\
-220000.000000001	-244200000\\
-349000	732600000\\
458999.999999997	-1098800000\\
54000.000000002	1220800000\\
19999.9999999996	-488500000\\
-314000.000000001	-732100000\\
-15999.9999999982	488100000.000001\\
586000.000000002	1098600000\\
-313000.000000002	-1342600000\\
-237999.999999999	-610500000\\
129999.999999999	2197200000\\
-256999.999999999	-2075000000\\
181999.999999999	1098600000\\
-17999.9999999989	121900000\\
459000	-243999999.999999\\
-642000.000000002	-1342800000\\
166000	2075300000\\
327999.999999998	-488600000\\
-91000.0000000011	-121700000.000001\\
55000.0000000024	-976800000\\
-457000.000000002	366300000\\
145999.999999999	610400000\\
108999.999999998	244000000.000001\\
312000	200000.000000067\\
-91000.0000000011	-1709300000\\
-605999.999999999	1587300000\\
515000	-488500000\\
162999.999999998	1220700000\\
-309999.999999999	-2563500000\\
-403999.999999999	1587000000\\
129000	488300000\\
640999.999999999	-366300000\\
-56000.0000000009	-244099999.999999\\
-201000.000000001	-610399999.999999\\
-218999.999999998	732600000\\
511999.999999997	244000000\\
-731999.999999997	-1220800000\\
255999.999999998	2441500000\\
37000.0000000008	-3417800000\\
-8.88178419700125e-10	2563200000\\
-38000.0000000002	-610200000\\
295000	244100000\\
-129000	-732400000\\
-19000.0000000019	244100000.000001\\
-35000.0000000001	488400000\\
52999.9999999999	-732600000\\
-35000.0000000019	732500000.000001\\
-130000	-1098500000\\
-88999.999999996	854200000.000001\\
400999.999999994	732700000\\
165000.000000002	-244199999.999999\\
-275000.000000003	-2319500000\\
-712999.999999998	1953300000\\
676999.999999998	-854500000\\
165000	2929500000\\
-19000.000000001	-3784000000\\
-348000	1220700000\\
-145000	610400000.000001\\
199999.999999997	-854800000\\
275000.000000001	1465200000\\
-145999.999999999	-1342900000\\
-19000.0000000028	122100000\\
128000.000000004	976400000\\
-200000	-1708900000\\
346000	1587100000\\
-400999.999999999	-1220900000\\
-202000.000000003	1098700000\\
310000	-976600000\\
56000	1220600000\\
-53999.9999999967	-1464500000\\
162999.999999996	1708600000\\
-329000	-2075000000\\
202000.000000001	1831100000\\
182999.999999999	-854700000\\
-184000.000000001	-121799999.999999\\
-127000.000000001	243900000\\
-146999.999999999	122200000\\
348000	-122100000\\
72000.0000000027	244200000\\
-511000.000000001	-976800000\\
110000	1221100000\\
565999.999999997	-122500000\\
-272999.999999997	-854100000\\
-258000.000000001	366000000\\
-17000.0000000003	244100000\\
531000.000000001	732500000\\
-92000.0000000032	-1586900000\\
-238999.999999995	732400000.000001\\
-547000.000000002	-976500000\\
419000.000000001	1952900000\\
293999.999999995	-487900000\\
-256999.999999997	-1465300000\\
202000.000000002	1709400000\\
55000.0000000006	-610600000\\
73999.9999999989	366299999.999999\\
-386999.999999999	-2319400000\\
20999.999999999	2441500000\\
236000.000000001	610299999.999999\\
-71999.9999999983	-2197200000\\
-147000.000000002	610100000\\
128000.000000003	854900000.000001\\
1000.00000000122	-122400000\\
-148000.000000002	-1586800000\\
368000	3417900000\\
-1000.00000000033	-3295700000\\
-184000.000000002	488100000.000001\\
-181999.999999998	610299999.999999\\
-91000.0000000019	244400000\\
181999.999999999	-122300000\\
54999.9999999997	-122000000\\
-273999.999999998	-488200000\\
548999.999999998	1342500000\\
-328999.999999999	-732100000\\
-112000.000000002	-1220800000\\
424000.000000003	2441200000\\
-423000.000000004	-2318900000\\
36999.9999999999	365700000\\
109999.999999999	1831500000\\
312000.000000002	-1099000000\\
-295000.000000003	-488000000\\
-254000.000000001	-122200000\\
255000.000000001	732500000\\
-402000	-366400000\\
455999.999999999	122200000\\
222000.000000004	976700000\\
-368000.000000002	-1953400000\\
38000.0000000047	976800000\\
-37000.0000000026	121900000\\
237000.000000002	854600000.000001\\
111000.000000002	-1709000000\\
-238000.000000002	732400000.000001\\
54000.0000000003	-488400000\\
-347000.000000004	976800000\\
128000.000000002	-1220800000\\
71999.9999999983	1342600000\\
240000.000000001	-243800000\\
-220999.999999998	-610700000\\
-72000.0000000009	-243800000\\
89999.9999999963	854100000\\
19000.0000000019	-243800000\\
-73000.0000000022	-488400000\\
292000.000000001	976400000.000001\\
-35000.000000001	-610000000\\
-203000	-854800000\\
-89999.9999999998	1587100000\\
17000.0000000021	-1220900000\\
17999.9999999998	610500000\\
-108000.000000001	-244100000\\
291999.999999999	610200000\\
-110999.999999996	-488100000\\
-163000.000000004	-732599999.999999\\
456000	1831100000\\
-510999.999999997	-1953000000\\
17999.9999999989	-99999.999999234\\
620999.999999999	3540000000\\
-400999.999999996	-4516600000\\
-110000	1342900000\\
16999.9999999968	1586800000\\
294000.000000001	-854399999.999999\\
-202000.000000003	-1220800000\\
91999.9999999987	1342800000\\
-292999.999999998	-366300000\\
-184000.000000002	-610000000\\
277000.000000004	731900000\\
254000	1221100000\\
109999.999999999	-1587000000\\
-217999.999999996	-732499999.999999\\
16999.9999999995	1709000000\\
-202000	-1831100000\\
-34999.9999999993	1831300000\\
512000	-299999.999999301\\
-403999.999999999	-1586800000\\
20999.9999999981	732399999.999999\\
-295999.999999999	200000.000000244\\
478000.000000002	610000000\\
146000	299999.999999656\\
-383999.999999999	-1465000000\\
53999.9999999976	1220700000\\
-201000.000000001	-244100000.000001\\
127000.000000002	-366100000\\
277000.000000003	1220400000\\
17000.0000000012	-487900000.000001\\
-165000.000000002	-1465100000\\
-127000.000000002	1464800000\\
145000.000000001	-487999999.999999\\
91999.999999997	610100000\\
-165000	-1098500000\\
19000.0000000001	610200000\\
17999.9999999989	366400000\\
-108999.999999999	-366300000\\
-56999.9999999995	-488300000\\
74999.9999999984	854400000\\
274000.000000002	-243800000\\
18999.9999999992	121600000\\
-239000.000000003	-365800000\\
-17999.9999999971	-610599999.999999\\
-19000.0000000001	976600000\\
-53000.0000000008	199999.999999889\\
199999.999999998	-122400000\\
18000.0000000016	300000.000000278\\
36999.9999999972	-122300000\\
-566999.999999998	-1220500000\\
494000.000000001	2441200000\\
293000.000000002	-732300000\\
-73000.0000000022	-1220700000\\
-550000	122000000\\
1.77635683940025e-09	1587100000\\
402999.999999997	-1465100000\\
-72999.9999999995	854600000\\
111000.000000002	-854400000\\
-497000.000000001	854400000\\
257999.999999999	-1220700000\\
403000	2319200000\\
-73999.9999999989	-1708600000\\
-530000.000000003	-1099100000\\
-183999.999999999	2075600000\\
621999.999999999	-488500000\\
203000.000000001	366200000\\
-404000.000000002	-1709000000\\
-329000.000000001	854800000\\
348000.000000002	1098100000\\
255999.999999999	-731899999.999999\\
-367000.000000001	-1099000000\\
38000.0000000038	1831300000\\
457000	-122299999.999999\\
-347000	-2441200000\\
-276000.000000002	1953000000\\
477000.000000001	1098700000\\
-367000.000000002	-2319300000\\
56000.0000000045	121799999.999999\\
291999.999999997	2930200000\\
38000.0000000029	-2808200000\\
-625000.000000003	-487900000\\
94000.0000000021	1098600000\\
512000.000000002	1952900000\\
16999.9999999995	-1952800000\\
-217999.999999999	-610600000.000001\\
-221000.000000002	-244100000\\
368000.000000003	2685700000\\
-21000.0000000008	-2197600000\\
-235000.000000001	-121600000.000001\\
-94000.0000000012	487899999.999999\\
183999.999999998	976700000.000001\\
72999.9999999995	-1220700000\\
-145999.999999998	366200000\\
274999.999999998	244100000\\
-330999.999999999	-732300000\\
-17000.000000003	488200000\\
420000.000000001	610299999.999999\\
-255000	-610400000\\
-74000.0000000007	-854200000\\
72999.9999999968	1464600000\\
-108999.999999997	-976600000\\
-129000.000000001	244400000\\
329000	1220300000\\
-107999.999999999	-2563100000\\
-111000.000000002	2685400000\\
146000.000000004	-1953100000\\
165999.999999999	1464700000\\
-164999.999999999	-1098400000\\
201000	-122200000.000001\\
-587000.000000002	1220700000\\
75000.0000000028	-2319200000\\
640999.999999998	4028000000\\
-130000.000000001	-3661800000\\
-658000.000000001	244000000\\
256000.000000002	2807800000\\
366999.999999999	-3174200000\\
-293999.999999998	1831500000\\
1000.00000000122	-732800000.000001\\
420000.000000002	1343000000\\
-292000.000000001	-2075200000\\
-202000.000000001	732200000\\
367000.000000002	854800000\\
-185000.000000002	-244400000\\
-198999.999999997	-2197100000\\
108000	3417900000\\
111999.999999999	-1831000000\\
15999.9999999982	-122100000.000001\\
146999.999999999	1220600000\\
-255999.999999998	-1464700000\\
1000.00000000033	244099999.999999\\
-93000.0000000035	732400000\\
201999.999999999	-366200000\\
17999.9999999998	244100000\\
-383999.999999998	-1220500000\\
420999.999999998	1952700000\\
146000.000000001	-365699999.999999\\
-292000.000000002	-2075600000\\
-166000	1709200000\\
477000	610200000\\
-147000.000000001	-854200000\\
-385000.000000002	-1343200000\\
130000	2075600000\\
235999.999999999	121800000\\
-16999.9999999986	-1830900000\\
53999.9999999967	1098500000\\
-437999.999999996	244300000\\
125999.999999999	-1465100000\\
111999.999999997	2563800000\\
163000.000000001	-2197500000\\
-144999.999999997	610500000\\
72999.9999999995	366100000\\
346999.999999999	488299999.999999\\
-421000.000000002	-1098600000\\
-474999.999999998	-1342700000\\
530000	2807400000\\
312000.000000001	488499999.999999\\
-312000.000000001	-3296000000\\
-311000.000000003	1464800000\\
549000.000000003	1587000000\\
-145999.999999998	-1953100000\\
-109999.999999999	122000000.000001\\
-54999.9999999988	976599999.999999\\
-127000.000000002	-1098700000\\
399999.999999999	1587000000\\
-197999.999999999	-1220800000\\
-351000	-732300000\\
368000.000000001	1464700000\\
74000.0000000007	610500000\\
271999.999999999	-1342800000\\
-327000.000000002	-976600000.000001\\
-698000	1708800000\\
917999.999999999	-1342300000\\
52000.0000000014	3905700000\\
-436000	-6469400000\\
-222999.999999998	4516600000\\
221999.999999996	-854600000\\
17000.0000000003	-8.88178419700125e-08\\
440000.000000001	-366100000\\
-257000.000000003	244000000\\
-35999.9999999987	199999.999999889\\
-17999.9999999989	-1221000000\\
-275000.000000003	2075500000\\
275000	-1709200000\\
-147000.000000001	1953200000\\
-36999.9999999999	-2807600000\\
349000	2929700000\\
53999.9999999985	-1709000000\\
-365999.999999999	244000000\\
-73000.0000000004	299999.999999834\\
147000	243900000\\
255000.000000004	-244000000\\
-401999.999999999	732200000\\
238000	-1098400000\\
110000	-122100000\\
-550000.000000002	1098400000\\
329999.999999999	-487800000\\
55999.9999999991	609800000\\
-38999.9999999997	-1830800000\\
314000.000000003	3052000000\\
88999.9999999968	-2808200000\\
-455999.999999999	732999999.999999\\
-91999.9999999996	-244400000\\
329999.999999998	2075100000\\
-165999.999999997	-3539800000\\
-366000	2929400000\\
476999.999999998	-1342400000\\
37000.0000000017	1830700000\\
-185000.000000003	-3051600000\\
20000.0000000013	1831000000\\
53999.9999999967	122300000\\
2.66453525910038e-09	-244600000\\
-109000	-609799999.999999\\
183000.000000001	609800000\\
-221000.000000004	-121600000\\
477000	732199999.999999\\
-147000	-366299999.999999\\
-456000	-2685400000\\
-57999.9999999972	3295900000\\
515999.999999999	-122100000\\
-167000.000000002	-976599999.999999\\
165000.000000001	-732400000\\
-439000.000000002	1709100000\\
312000.000000003	-1221000000\\
54000.0000000003	122400000\\
-384999.999999999	366100000\\
38000.0000000011	-122300000\\
310000	366600000\\
-201000.000000003	-976800000\\
128999.999999999	854500000\\
273000.000000001	366300000.000001\\
94000.0000000038	-366300000.000001\\
-900000.000000003	-1708900000\\
478000.000000001	1464800000\\
401999.999999997	2075200000\\
-585999.999999999	-3662200000\\
-90999.9999999993	610500000\\
513000.000000002	3417900000\\
-349000.000000005	-4516600000\\
147000.000000002	2563500000\\
401999.999999999	1831100000\\
-400999.999999997	-5127200000\\
-93000.0000000008	3174200000\\
-53999.9999999976	487900000.000001\\
35999.9999999978	-1952700000\\
384999.999999999	2685100000\\
-148000	-2196900000\\
};
\addplot [color=mycolor2, line width=2.0pt, forget plot]
  table[row sep=crcr]{%
-164999.999999998	-164999.999999998\\
202999.999999999	202999.999999999\\
-57000.0000000022	-57000.0000000022\\
93000.0000000044	93000.0000000044\\
145999.999999996	145999.999999996\\
-494000.000000001	-494000.000000001\\
218999.999999999	218999.999999999\\
-36999.9999999999	-36999.9999999999\\
203000.000000005	203000.000000005\\
145999.999999997	145999.999999997\\
-312000	-312000\\
-90999.9999999975	-90999.9999999975\\
-403000	-403000\\
494000.000000002	494000.000000002\\
329999.999999995	329999.999999995\\
-219999.999999998	-219999.999999998\\
55999.9999999991	55999.9999999991\\
-1999.99999999978	-1999.99999999978\\
-565000.000000001	-565000.000000001\\
729999.999999999	729999.999999999\\
-200000.000000001	-200000.000000001\\
-459000	-459000\\
533000	533000\\
-57000.0000000004	-57000.0000000004\\
-347000.000000001	-347000.000000001\\
367000	367000\\
16000.0000000009	16000.0000000009\\
-381999.999999998	-381999.999999998\\
328999.999999996	328999.999999996\\
17000.0000000003	17000.0000000003\\
-72000.0000000027	-72000.0000000027\\
-55999.9999999983	-55999.9999999983\\
-72000.0000000001	-72000.0000000001\\
-8.88178419700125e-10	-8.88178419700125e-10\\
181999.999999999	181999.999999999\\
-109999.999999999	-109999.999999999\\
20000.0000000022	20000.0000000022\\
-1000.00000000211	-1000.00000000211\\
-147000.000000001	-147000.000000001\\
294000	294000\\
35999.9999999987	35999.9999999987\\
-257000.000000001	-257000.000000001\\
203000	203000\\
-258000.000000001	-258000.000000001\\
-52999.9999999999	-52999.9999999999\\
327999.999999999	327999.999999999\\
-165000	-165000\\
-145000.000000001	-145000.000000001\\
201000	201000\\
53999.9999999994	53999.9999999994\\
-293000	-293000\\
423000.000000002	423000.000000002\\
-2000.00000000244	-2000.00000000244\\
-311000	-311000\\
-71999.9999999992	-71999.9999999992\\
107999.999999999	107999.999999999\\
295000.000000001	295000.000000001\\
-294000	-294000\\
-183000.000000002	-183000.000000002\\
403000.000000001	403000.000000001\\
-110999.999999997	-110999.999999997\\
-200000.000000005	-200000.000000005\\
128000.000000003	128000.000000003\\
-165000.000000003	-165000.000000003\\
384000.000000001	384000.000000001\\
-238000.000000001	-238000.000000001\\
111000.000000002	111000.000000002\\
-111000.000000002	-111000.000000002\\
146999.999999998	146999.999999998\\
-128999.999999999	-128999.999999999\\
20000.0000000022	20000.0000000022\\
16999.999999995	16999.999999995\\
-72999.9999999995	-72999.9999999995\\
-146000.000000002	-146000.000000002\\
109000.000000001	109000.000000001\\
255999.999999998	255999.999999998\\
-52999.9999999946	-52999.9999999946\\
108999.999999997	108999.999999997\\
-532000.000000002	-532000.000000002\\
257000.000000002	257000.000000002\\
183000.000000001	183000.000000001\\
-35999.9999999978	-35999.9999999978\\
-202000.000000003	-202000.000000003\\
-53999.9999999967	-53999.9999999967\\
491999.999999996	491999.999999996\\
-344999.999999997	-344999.999999997\\
145000	145000\\
-129000	-129000\\
-89000.0000000013	-89000.0000000013\\
-94000.0000000012	-94000.0000000012\\
348999.999999999	348999.999999999\\
-37000.0000000017	-37000.0000000017\\
-256999.999999999	-256999.999999999\\
2000.00000000067	2000.00000000067\\
-20000.0000000013	-20000.0000000013\\
276000.000000002	276000.000000002\\
-75000.0000000037	-75000.0000000037\\
-199999.999999999	-199999.999999999\\
237000	237000\\
-236999.999999999	-236999.999999999\\
347000	347000\\
-291999.999999998	-291999.999999998\\
256000	256000\\
-166000.000000003	-166000.000000003\\
-126000	-126000\\
273000.000000001	273000.000000001\\
-183000.000000003	-183000.000000003\\
-90999.9999999984	-90999.9999999984\\
164999.999999999	164999.999999999\\
126999.999999998	126999.999999998\\
-291999.999999999	-291999.999999999\\
238000.000000001	238000.000000001\\
-201000.000000001	-201000.000000001\\
16999.9999999995	16999.9999999995\\
221000	221000\\
-275000.000000003	-275000.000000003\\
73000.0000000022	73000.0000000022\\
-92000.0000000032	-92000.0000000032\\
257000.000000001	257000.000000001\\
-54999.9999999979	-54999.9999999979\\
201999.999999999	201999.999999999\\
-514000.000000003	-514000.000000003\\
37000.0000000026	37000.0000000026\\
531999.999999998	531999.999999998\\
-349000.000000001	-349000.000000001\\
-36999.999999999	-36999.999999999\\
368000.000000001	368000.000000001\\
-312000	-312000\\
-239000.000000002	-239000.000000002\\
238000	238000\\
-125999.999999996	-125999.999999996\\
143999.999999997	143999.999999997\\
166000.000000001	166000.000000001\\
37000.0000000017	37000.0000000017\\
-331000.000000003	-331000.000000003\\
129000.000000002	129000.000000002\\
19000.000000001	19000.000000001\\
-38000.0000000038	-38000.0000000038\\
166000.000000001	166000.000000001\\
-184000.000000002	-184000.000000002\\
-36000.0000000014	-36000.0000000014\\
72999.9999999995	72999.9999999995\\
-72999.9999999995	-72999.9999999995\\
-38000.0000000011	-38000.0000000011\\
2000.00000000511	2000.00000000511\\
365999.999999998	365999.999999998\\
-111000.000000002	-111000.000000002\\
-202000.000000001	-202000.000000001\\
-15999.9999999991	-15999.9999999991\\
235999.999999999	235999.999999999\\
-144999.999999998	-144999.999999998\\
-202000.000000003	-202000.000000003\\
-55999.9999999974	-55999.9999999974\\
495999.999999997	495999.999999997\\
-292999.999999998	-292999.999999998\\
-37999.9999999994	-37999.9999999994\\
128999.999999999	128999.999999999\\
35999.9999999987	35999.9999999987\\
-16999.9999999995	-16999.9999999995\\
-73999.9999999998	-73999.9999999998\\
-256999.999999997	-256999.999999997\\
-17000.0000000012	-17000.0000000012\\
310000.000000001	310000.000000001\\
148000.000000001	148000.000000001\\
71999.9999999983	71999.9999999983\\
-183000.000000002	-183000.000000002\\
-73000.0000000004	-73000.0000000004\\
17999.9999999998	17999.9999999998\\
-383000.000000002	-383000.000000002\\
419000.000000001	419000.000000001\\
-72000.0000000018	-72000.0000000018\\
-19000.0000000001	-19000.0000000001\\
-145999.999999999	-145999.999999999\\
275000	275000\\
182000.000000001	182000.000000001\\
-273999.999999997	-273999.999999997\\
183999.999999997	183999.999999997\\
-368000.000000001	-368000.000000001\\
111000.000000002	111000.000000002\\
127999.999999999	127999.999999999\\
36999.999999999	36999.999999999\\
-111000.000000001	-111000.000000001\\
-54000.000000002	-54000.000000002\\
146000	146000\\
-182999.999999997	-182999.999999997\\
220999.999999999	220999.999999999\\
-332000.000000001	-332000.000000001\\
203000.000000002	203000.000000002\\
312000	312000\\
-185000.000000005	-185000.000000005\\
-291999.999999998	-291999.999999998\\
-92000.0000000032	-92000.0000000032\\
348000.000000002	348000.000000002\\
-89999.9999999981	-89999.9999999981\\
107000	107000\\
-51999.9999999987	-51999.9999999987\\
107999.999999997	107999.999999997\\
-182999.999999998	-182999.999999998\\
-73000.0000000004	-73000.0000000004\\
74000.0000000016	74000.0000000016\\
255000.000000001	255000.000000001\\
-71999.9999999983	-71999.9999999983\\
-495000	-495000\\
439999.999999999	439999.999999999\\
72000.0000000001	72000.0000000001\\
-364999.999999997	-364999.999999997\\
128000	128000\\
237999.999999997	237999.999999997\\
-165999.999999999	-165999.999999999\\
-125999.999999999	-125999.999999999\\
198999.999999997	198999.999999997\\
-180999.999999997	-180999.999999997\\
-1000.000000003	-1000.000000003\\
275000.000000002	275000.000000002\\
-1000.00000000211	-1000.00000000211\\
-254999.999999999	-254999.999999999\\
-129000.000000002	-129000.000000002\\
37000.0000000008	37000.0000000008\\
550000.000000003	550000.000000003\\
-130000.000000003	-130000.000000003\\
-328000.000000001	-328000.000000001\\
-56000.0000000009	-56000.0000000009\\
147000.000000002	147000.000000002\\
111000.000000002	111000.000000002\\
-204000.000000004	-204000.000000004\\
2000.00000000333	2000.00000000333\\
54999.9999999979	54999.9999999979\\
71999.9999999992	71999.9999999992\\
109999.999999999	109999.999999999\\
-345999.999999999	-345999.999999999\\
161999.999999998	161999.999999998\\
185000.000000002	185000.000000002\\
-73999.9999999998	-73999.9999999998\\
-164000.000000001	-164000.000000001\\
-999.999999997669	-999.999999997669\\
329999.999999998	329999.999999998\\
0	0\\
-548999.999999999	-548999.999999999\\
201000	201000\\
36999.999999999	36999.999999999\\
201000	201000\\
0	0\\
-384000	-384000\\
348000.000000001	348000.000000001\\
53999.9999999967	53999.9999999967\\
-292999.999999999	-292999.999999999\\
202000	202000\\
-35999.9999999978	-35999.9999999978\\
-93000.0000000044	-93000.0000000044\\
258000.000000004	258000.000000004\\
-295000.000000003	-295000.000000003\\
167000	167000\\
-93000	-93000\\
-255999.999999999	-255999.999999999\\
420999.999999998	420999.999999998\\
-200999.999999999	-200999.999999999\\
35999.9999999996	35999.9999999996\\
367000	367000\\
-183999.999999998	-183999.999999998\\
-237000.000000002	-237000.000000002\\
-1000.00000000122	-1000.00000000122\\
-109000.000000002	-109000.000000002\\
438999.999999999	438999.999999999\\
-329999.999999999	-329999.999999999\\
18999.9999999975	18999.9999999975\\
-164999.999999996	-164999.999999996\\
477000.000000001	477000.000000001\\
-314000.000000004	-314000.000000004\\
314000.000000004	314000.000000004\\
-257000.000000003	-257000.000000003\\
181999.999999999	181999.999999999\\
-292000.000000002	-292000.000000002\\
165000	165000\\
-129000.000000002	-129000.000000002\\
-16999.9999999968	-16999.9999999968\\
-74999.9999999993	-74999.9999999993\\
311999.999999997	311999.999999997\\
-108999.999999997	-108999.999999997\\
-1000.000000003	-1000.000000003\\
-236999.999999998	-236999.999999998\\
437999.999999998	437999.999999998\\
-36000.0000000014	-36000.0000000014\\
-366000	-366000\\
201000	201000\\
-90999.9999999984	-90999.9999999984\\
220000.000000001	220000.000000001\\
-19000.0000000019	-19000.0000000019\\
72999.9999999977	72999.9999999977\\
-328999.999999999	-328999.999999999\\
-37000.0000000035	-37000.0000000035\\
74000.0000000025	74000.0000000025\\
71999.9999999974	71999.9999999974\\
75000.0000000019	75000.0000000019\\
-166999.999999998	-166999.999999998\\
1999.99999999623	1999.99999999623\\
273000.000000001	273000.000000001\\
-127000.000000001	-127000.000000001\\
-109999.999999999	-109999.999999999\\
310999.999999997	310999.999999997\\
-146999.999999998	-146999.999999998\\
-512000.000000002	-512000.000000002\\
348000.000000001	348000.000000001\\
255999.999999998	255999.999999998\\
-182999.999999997	-182999.999999997\\
54999.9999999988	54999.9999999988\\
92000.0000000005	92000.0000000005\\
-367000.000000003	-367000.000000003\\
55000.0000000015	55000.0000000015\\
365999.999999998	365999.999999998\\
-54000.0000000003	-54000.0000000003\\
-421999.999999999	-421999.999999999\\
440000	440000\\
-110000	-110000\\
-165000.000000003	-165000.000000003\\
147000.000000002	147000.000000002\\
71999.9999999983	71999.9999999983\\
-126999.999999999	-126999.999999999\\
-200999.999999998	-200999.999999998\\
126999.999999999	126999.999999999\\
366999.999999998	366999.999999998\\
-74000.0000000016	-74000.0000000016\\
-347999.999999999	-347999.999999999\\
74000.0000000007	74000.0000000007\\
-35999.9999999987	-35999.9999999987\\
126999.999999995	126999.999999995\\
146000.000000002	146000.000000002\\
-255000	-255000\\
-8.88178419700125e-10	-8.88178419700125e-10\\
108000.000000001	108000.000000001\\
222000.000000001	222000.000000001\\
-294000.000000001	-294000.000000001\\
-165000.000000002	-165000.000000002\\
110000.000000003	110000.000000003\\
294000	294000\\
-166000	-166000\\
218999.999999998	218999.999999998\\
-327999.999999999	-327999.999999999\\
-36999.9999999964	-36999.9999999964\\
330000	330000\\
-166000.000000005	-166000.000000005\\
-72999.9999999995	-72999.9999999995\\
-35999.9999999996	-35999.9999999996\\
-293000.000000001	-293000.000000001\\
365999.999999999	365999.999999999\\
92000.0000000014	92000.0000000014\\
126999.999999998	126999.999999998\\
-274000.000000001	-274000.000000001\\
-35999.9999999987	-35999.9999999987\\
347000	347000\\
-402000.000000003	-402000.000000003\\
109000	109000\\
146999.999999999	146999.999999999\\
-257000	-257000\\
-35999.9999999996	-35999.9999999996\\
220000.000000001	220000.000000001\\
274999.999999999	274999.999999999\\
-295000.000000002	-295000.000000002\\
-52999.999999999	-52999.999999999\\
-92000.0000000005	-92000.0000000005\\
255000.000000002	255000.000000002\\
-492000.000000003	-492000.000000003\\
255000	255000\\
457999.999999997	457999.999999997\\
-439999.999999999	-439999.999999999\\
54999.9999999988	54999.9999999988\\
129000.000000004	129000.000000004\\
72999.9999999995	72999.9999999995\\
-109999.999999999	-109999.999999999\\
-165000.000000002	-165000.000000002\\
54999.999999997	54999.999999997\\
0	0\\
37000.0000000017	37000.0000000017\\
53999.9999999976	53999.9999999976\\
-163999.999999998	-163999.999999998\\
165000.000000002	165000.000000002\\
-54999.9999999997	-54999.9999999997\\
273999.999999998	273999.999999998\\
-329999.999999997	-329999.999999997\\
-35000.0000000001	-35000.0000000001\\
310000.000000002	310000.000000002\\
-366000	-366000\\
239000.000000002	239000.000000002\\
145000	145000\\
-200000.000000001	-200000.000000001\\
-184000.000000004	-184000.000000004\\
-72999.9999999995	-72999.9999999995\\
402000	402000\\
-35000.000000001	-35000.000000001\\
-330000.000000001	-330000.000000001\\
310000	310000\\
-36000.0000000022	-36000.0000000022\\
-218999.999999999	-218999.999999999\\
109999.999999999	109999.999999999\\
127000.000000002	127000.000000002\\
-293000	-293000\\
165999.999999999	165999.999999999\\
273999.999999998	273999.999999998\\
-512999.999999999	-512999.999999999\\
220999.999999999	220999.999999999\\
272999.999999999	272999.999999999\\
-293000.000000001	-293000.000000001\\
313000.000000005	313000.000000005\\
-330000.000000003	-330000.000000003\\
235999.999999999	235999.999999999\\
-51999.9999999987	-51999.9999999987\\
-350000.000000002	-350000.000000002\\
312000	312000\\
53999.9999999976	53999.9999999976\\
-345999.999999997	-345999.999999997\\
493000	493000\\
-182000.000000002	-182000.000000002\\
-312999.999999997	-312999.999999997\\
-54000.0000000029	-54000.0000000029\\
366999.999999998	366999.999999998\\
-19999.9999999978	-19999.9999999978\\
57000.0000000013	57000.0000000013\\
-75000.0000000037	-75000.0000000037\\
-109000.000000001	-109000.000000001\\
182000.000000001	182000.000000001\\
-35000.000000001	-35000.000000001\\
-166000.000000001	-166000.000000001\\
129000.000000001	129000.000000001\\
-220000	-220000\\
420999.999999999	420999.999999999\\
-238000.000000001	-238000.000000001\\
-202000.000000002	-202000.000000002\\
93000.0000000026	93000.0000000026\\
364999.999999997	364999.999999997\\
-183000	-183000\\
-109000.000000001	-109000.000000001\\
-75000.0000000002	-75000.0000000002\\
258000.000000002	258000.000000002\\
-183000	-183000\\
-148000.000000003	-148000.000000003\\
167000.000000003	167000.000000003\\
51999.9999999978	51999.9999999978\\
113000.000000001	113000.000000001\\
-277000.000000001	-277000.000000001\\
164999.999999999	164999.999999999\\
203000.000000001	203000.000000001\\
-422000.000000001	-422000.000000001\\
-92000.0000000005	-92000.0000000005\\
255999.999999999	255999.999999999\\
221000.000000002	221000.000000002\\
-91999.999999997	-91999.999999997\\
-128000.000000004	-128000.000000004\\
53999.9999999994	53999.9999999994\\
-35999.9999999978	-35999.9999999978\\
999.999999998557	999.999999998557\\
-149000.000000001	-149000.000000001\\
204000.000000001	204000.000000001\\
144999.999999998	144999.999999998\\
-457000.000000001	-457000.000000001\\
219000	219000\\
91999.9999999978	91999.9999999978\\
-54999.999999997	-54999.999999997\\
18999.9999999975	18999.9999999975\\
-148000	-148000\\
110999.999999997	110999.999999997\\
54000.0000000047	54000.0000000047\\
238999.999999998	238999.999999998\\
-54999.9999999962	-54999.9999999962\\
-311000.000000004	-311000.000000004\\
107999.999999999	107999.999999999\\
-327000	-327000\\
346000	346000\\
-238000	-238000\\
368000.000000002	368000.000000002\\
-75000.0000000028	-75000.0000000028\\
-238000	-238000\\
147000	147000\\
1000.000000003	1000.000000003\\
16999.999999995	16999.999999995\\
-182999.999999999	-182999.999999999\\
365999.999999999	365999.999999999\\
-35999.9999999987	-35999.9999999987\\
-109999.999999997	-109999.999999997\\
-164000.000000003	-164000.000000003\\
15999.9999999991	15999.9999999991\\
39000.0000000015	39000.0000000015\\
237000	237000\\
-238000.000000003	-238000.000000003\\
0	0\\
310999.999999999	310999.999999999\\
-567000.000000001	-567000.000000001\\
127999.999999999	127999.999999999\\
494000.000000002	494000.000000002\\
-384000.000000001	-384000.000000001\\
-219999.999999999	-219999.999999999\\
586000.000000002	586000.000000002\\
-238000	-238000\\
-237999.999999999	-237999.999999999\\
274999.999999999	274999.999999999\\
89999.9999999972	89999.9999999972\\
-71999.9999999983	-71999.9999999983\\
-292000.000000001	-292000.000000001\\
-75000.000000001	-75000.000000001\\
146999.999999998	146999.999999998\\
274000.000000001	274000.000000001\\
-52999.9999999955	-52999.9999999955\\
-1000.00000000211	-1000.00000000211\\
16999.9999999977	16999.9999999977\\
-216999.999999996	-216999.999999996\\
124999.999999996	124999.999999996\\
-124999.999999997	-124999.999999997\\
106999.999999998	106999.999999998\\
-163000	-163000\\
127999.999999998	127999.999999998\\
256000.000000001	256000.000000001\\
-458000.000000001	-458000.000000001\\
109999.999999999	109999.999999999\\
165000.000000001	165000.000000001\\
-72999.9999999986	-72999.9999999986\\
-36999.9999999999	-36999.9999999999\\
54999.9999999997	54999.9999999997\\
256000	256000\\
-219000.000000001	-219000.000000001\\
-146999.999999998	-146999.999999998\\
164999.999999999	164999.999999999\\
72999.9999999968	72999.9999999968\\
-311000	-311000\\
-36999.9999999999	-36999.9999999999\\
294000.000000003	294000.000000003\\
254999.999999998	254999.999999998\\
-201000.000000003	-201000.000000003\\
-164000	-164000\\
54000.0000000011	54000.0000000011\\
-182999.999999997	-182999.999999997\\
-128000	-128000\\
366999.999999998	366999.999999998\\
-129999.999999998	-129999.999999998\\
202999.999999998	202999.999999998\\
-19000.000000001	-19000.000000001\\
-236999.999999999	-236999.999999999\\
108000.000000001	108000.000000001\\
999.999999998557	999.999999998557\\
-18000.0000000016	-18000.0000000016\\
-147000	-147000\\
73999.9999999998	73999.9999999998\\
237000	237000\\
-35999.9999999987	-35999.9999999987\\
-220000	-220000\\
54999.9999999997	54999.9999999997\\
166000	166000\\
-186000.000000002	-186000.000000002\\
186000.000000004	186000.000000004\\
34999.9999999993	34999.9999999993\\
-90999.9999999993	-90999.9999999993\\
-164000.000000001	-164000.000000001\\
-203000.000000002	-203000.000000002\\
439999.999999999	439999.999999999\\
128000.000000003	128000.000000003\\
-200000	-200000\\
107999.999999999	107999.999999999\\
-256000.000000003	-256000.000000003\\
-107999.999999998	-107999.999999998\\
327000.000000002	327000.000000002\\
-511000	-511000\\
383999.999999998	383999.999999998\\
145999.999999999	145999.999999999\\
-109000	-109000\\
165000.000000001	165000.000000001\\
-275999.999999997	-275999.999999997\\
-91000.0000000002	-91000.0000000002\\
146999.999999999	146999.999999999\\
55000.0000000006	55000.0000000006\\
35999.9999999996	35999.9999999996\\
-54999.9999999988	-54999.9999999988\\
-218999.999999999	-218999.999999999\\
-239000.000000002	-239000.000000002\\
770000.000000001	770000.000000001\\
-201999.999999996	-201999.999999996\\
73999.9999999981	73999.9999999981\\
-385999.999999998	-385999.999999998\\
-16000.0000000009	-16000.0000000009\\
437000.000000001	437000.000000001\\
-164000.000000001	-164000.000000001\\
37000.0000000026	37000.0000000026\\
-201000	-201000\\
34999.9999999993	34999.9999999993\\
240000.000000003	240000.000000003\\
-239000.000000001	-239000.000000001\\
145999.999999999	145999.999999999\\
-107999.999999998	-107999.999999998\\
-405000.000000003	-405000.000000003\\
732999.999999998	732999.999999998\\
-347999.999999997	-347999.999999997\\
-54000.0000000003	-54000.0000000003\\
165000.000000002	165000.000000002\\
-93000.0000000035	-93000.0000000035\\
18999.9999999992	18999.9999999992\\
-55000.0000000006	-55000.0000000006\\
54999.9999999988	54999.9999999988\\
109999.999999999	109999.999999999\\
-54999.9999999988	-54999.9999999988\\
-202000.000000003	-202000.000000003\\
422000	422000\\
-127999.999999998	-127999.999999998\\
-476999.999999998	-476999.999999998\\
72999.9999999977	72999.9999999977\\
641999.999999998	641999.999999998\\
-255999.999999998	-255999.999999998\\
-514999.999999999	-514999.999999999\\
222000	222000\\
255999.999999998	255999.999999998\\
-93000	-93000\\
147999.999999998	147999.999999998\\
-366000	-366000\\
400999.999999999	400999.999999999\\
1000.00000000033	1000.00000000033\\
-437999.999999999	-437999.999999999\\
546999.999999998	546999.999999998\\
-291999.999999998	-291999.999999998\\
1.77635683940025e-09	1.77635683940025e-09\\
-310999.999999999	-310999.999999999\\
457000	457000\\
-347000	-347000\\
127999.999999997	127999.999999997\\
274000.000000001	274000.000000001\\
-55000.0000000015	-55000.0000000015\\
-217999.999999998	-217999.999999998\\
108000.000000002	108000.000000002\\
-145999.999999998	-145999.999999998\\
237999.999999999	237999.999999999\\
-146000	-146000\\
-1000.00000000033	-1000.00000000033\\
93000	93000\\
-239000	-239000\\
257000.000000002	257000.000000002\\
-294000	-294000\\
311999.999999997	311999.999999997\\
18000.0000000016	18000.0000000016\\
-182000	-182000\\
163000.000000003	163000.000000003\\
-236999.999999999	-236999.999999999\\
-238000	-238000\\
456999.999999998	456999.999999998\\
220999.999999999	220999.999999999\\
-276000.000000002	-276000.000000002\\
37000.0000000017	37000.0000000017\\
-382999.999999999	-382999.999999999\\
436999.999999997	436999.999999997\\
93000	93000\\
-420999.999999999	-420999.999999999\\
291999.999999997	291999.999999997\\
-255999.999999998	-255999.999999998\\
421999.999999999	421999.999999999\\
-330000.000000001	-330000.000000001\\
-91999.9999999987	-91999.9999999987\\
128000.000000001	128000.000000001\\
219999.999999999	219999.999999999\\
-219000	-219000\\
-37999.9999999967	-37999.9999999967\\
-72000.0000000027	-72000.0000000027\\
329000	329000\\
-366000.000000001	-366000.000000001\\
73000.0000000022	73000.0000000022\\
420999.999999998	420999.999999998\\
-90999.9999999984	-90999.9999999984\\
-310999.999999999	-310999.999999999\\
-166000.000000004	-166000.000000004\\
220000.000000002	220000.000000002\\
238999.999999999	238999.999999999\\
-293000	-293000\\
310000	310000\\
-91000.0000000028	-91000.0000000028\\
-310999.999999999	-310999.999999999\\
-220000.000000001	-220000.000000001\\
587000.000000001	587000.000000001\\
-38000.0000000002	-38000.0000000002\\
-348000.000000002	-348000.000000002\\
276000.000000002	276000.000000002\\
72000.0000000009	72000.0000000009\\
-90000.0000000016	-90000.0000000016\\
-204000.000000001	-204000.000000001\\
21000.0000000008	21000.0000000008\\
327999.999999999	327999.999999999\\
-256000.000000002	-256000.000000002\\
72999.9999999995	72999.9999999995\\
74000.0000000016	74000.0000000016\\
-202000.000000002	-202000.000000002\\
182999.999999999	182999.999999999\\
-127999.999999998	-127999.999999998\\
-145999.999999999	-145999.999999999\\
255999.999999999	255999.999999999\\
37000.0000000008	37000.0000000008\\
-38000.0000000038	-38000.0000000038\\
56000.0000000009	56000.0000000009\\
-73000.0000000004	-73000.0000000004\\
-202000.000000001	-202000.000000001\\
110000	110000\\
17999.9999999989	17999.9999999989\\
1000.000000003	1000.000000003\\
35999.9999999996	35999.9999999996\\
201000	201000\\
-54000.000000002	-54000.000000002\\
-532000	-532000\\
330000.000000002	330000.000000002\\
367000	367000\\
-367000.000000001	-367000.000000001\\
-56000	-56000\\
277000.000000001	277000.000000001\\
-148000.000000004	-148000.000000004\\
-164999.999999998	-164999.999999998\\
220999.999999999	220999.999999999\\
-148000.000000001	-148000.000000001\\
19999.9999999996	19999.9999999996\\
217999.999999999	217999.999999999\\
-255000.000000001	-255000.000000001\\
89999.999999999	89999.999999999\\
39000.0000000024	39000.0000000024\\
15999.9999999965	15999.9999999965\\
-181999.999999999	-181999.999999999\\
255999.999999998	255999.999999998\\
-164000.000000001	-164000.000000001\\
-38000.0000000011	-38000.0000000011\\
111000	111000\\
218999.999999999	218999.999999999\\
-218999.999999998	-218999.999999998\\
-110000.000000002	-110000.000000002\\
-219999.999999999	-219999.999999999\\
-55000.0000000006	-55000.0000000006\\
475999.999999997	475999.999999997\\
128000.000000001	128000.000000001\\
-71999.9999999992	-71999.9999999992\\
-313000.000000003	-313000.000000003\\
129000.000000001	129000.000000001\\
37000.0000000008	37000.0000000008\\
90999.9999999966	90999.9999999966\\
-346999.999999999	-346999.999999999\\
236000	236000\\
1999.999999998	1999.999999998\\
-73999.9999999998	-73999.9999999998\\
129000.000000002	129000.000000002\\
-38000.0000000029	-38000.0000000029\\
55000.0000000015	55000.0000000015\\
-292000.000000003	-292000.000000003\\
-17999.9999999971	-17999.9999999971\\
546999.999999999	546999.999999999\\
-417999.999999998	-417999.999999998\\
125999.999999998	125999.999999998\\
-146000.000000001	-146000.000000001\\
202000.000000002	202000.000000002\\
-239000.000000003	-239000.000000003\\
330000.000000001	330000.000000001\\
-181999.999999999	-181999.999999999\\
-39000.0000000024	-39000.0000000024\\
277000.000000001	277000.000000001\\
-606000	-606000\\
588000.000000003	588000.000000003\\
-295000.000000004	-295000.000000004\\
36999.9999999999	36999.9999999999\\
-16999.9999999995	-16999.9999999995\\
199000	199000\\
-234999.999999997	-234999.999999997\\
-21000.0000000026	-21000.0000000026\\
277000.000000002	277000.000000002\\
-94000.0000000021	-94000.0000000021\\
-71000.0000000006	-71000.0000000006\\
-130000	-130000\\
221000	221000\\
-74000.0000000025	-74000.0000000025\\
-16999.9999999977	-16999.9999999977\\
-75000.0000000019	-75000.0000000019\\
109999.999999999	109999.999999999\\
-328000.000000001	-328000.000000001\\
310000	310000\\
237999.999999999	237999.999999999\\
-255999.999999998	-255999.999999998\\
1000.00000000033	1000.00000000033\\
198999.999999998	198999.999999998\\
-289999.999999996	-289999.999999996\\
51999.9999999952	51999.9999999952\\
56999.9999999995	56999.9999999995\\
-129000	-129000\\
-55000.0000000006	-55000.0000000006\\
184000	184000\\
34999.9999999993	34999.9999999993\\
57000.0000000004	57000.0000000004\\
15999.9999999973	15999.9999999973\\
-437999.999999999	-437999.999999999\\
311000.000000002	311000.000000002\\
110000.000000002	110000.000000002\\
-18000.0000000007	-18000.0000000007\\
-330000.000000001	-330000.000000001\\
457000.000000002	457000.000000002\\
55999.9999999974	55999.9999999974\\
-329999.999999998	-329999.999999998\\
-55000.0000000033	-55000.0000000033\\
-183000	-183000\\
275000.000000001	275000.000000001\\
91000.0000000002	91000.0000000002\\
74000.0000000016	74000.0000000016\\
-56000.0000000009	-56000.0000000009\\
37999.9999999994	37999.9999999994\\
-277000.000000001	-277000.000000001\\
-15999.9999999982	-15999.9999999982\\
218999.999999999	218999.999999999\\
54999.9999999997	54999.9999999997\\
-220000.000000002	-220000.000000002\\
164999.999999999	164999.999999999\\
55000.0000000006	55000.0000000006\\
-440000.000000002	-440000.000000002\\
404000.000000001	404000.000000001\\
-93000	-93000\\
110999.999999998	110999.999999998\\
-73999.9999999981	-73999.9999999981\\
-37000.0000000008	-37000.0000000008\\
-71999.9999999974	-71999.9999999974\\
292999.999999997	292999.999999997\\
-404000	-404000\\
165999.999999997	165999.999999997\\
291000.000000003	291000.000000003\\
-345000	-345000\\
88999.9999999977	88999.9999999977\\
-17000.0000000021	-17000.0000000021\\
-146999.999999999	-146999.999999999\\
385000.000000001	385000.000000001\\
-183000.000000001	-183000.000000001\\
-148000	-148000\\
167000.000000002	167000.000000002\\
53999.9999999958	53999.9999999958\\
-330999.999999999	-330999.999999999\\
94000.0000000003	94000.0000000003\\
309999.999999997	309999.999999997\\
-238999.999999997	-238999.999999997\\
-125999.999999999	-125999.999999999\\
529999.999999997	529999.999999997\\
-330999.999999999	-330999.999999999\\
3000.00000000011	3000.00000000011\\
-186000.000000001	-186000.000000001\\
1999.99999999978	1999.99999999978\\
455999.999999999	455999.999999999\\
-273000	-273000\\
-256999.999999999	-256999.999999999\\
475999.999999995	475999.999999995\\
-73999.9999999981	-73999.9999999981\\
-401000.000000001	-401000.000000001\\
272999.999999999	272999.999999999\\
37999.9999999985	37999.9999999985\\
-185000	-185000\\
129999.999999998	129999.999999998\\
-36999.999999999	-36999.999999999\\
-74000.0000000007	-74000.0000000007\\
18999.9999999992	18999.9999999992\\
109000	109000\\
-109000	-109000\\
183000.000000002	183000.000000002\\
-108999.999999997	-108999.999999997\\
-76000.000000004	-76000.000000004\\
204000.000000005	204000.000000005\\
0	0\\
-222000.000000004	-222000.000000004\\
75000.0000000037	75000.0000000037\\
-17999.9999999989	-17999.9999999989\\
-74000.0000000016	-74000.0000000016\\
330000.000000001	330000.000000001\\
-312000.000000003	-312000.000000003\\
92000.0000000005	92000.0000000005\\
35999.9999999996	35999.9999999996\\
-236999.999999997	-236999.999999997\\
366000.000000001	366000.000000001\\
-146000.000000001	-146000.000000001\\
-2000.00000000156	-2000.00000000156\\
130000	130000\\
-91999.9999999996	-91999.9999999996\\
-145999.999999999	-145999.999999999\\
-184000.000000003	-184000.000000003\\
384000	384000\\
-180999.999999997	-180999.999999997\\
198999.999999996	198999.999999996\\
-216999.999999997	-216999.999999997\\
-168000.000000001	-168000.000000001\\
350999.999999999	350999.999999999\\
180000.000000001	180000.000000001\\
-308000	-308000\\
33999.999999998	33999.999999998\\
130000	130000\\
-239999.999999999	-239999.999999999\\
19999.9999999996	19999.9999999996\\
201000	201000\\
-164999.999999999	-164999.999999999\\
-54999.9999999997	-54999.9999999997\\
1000.00000000122	1000.00000000122\\
272999.999999996	272999.999999996\\
-53999.9999999976	-53999.9999999976\\
-202000.000000003	-202000.000000003\\
257000.000000001	257000.000000001\\
-255999.999999998	-255999.999999998\\
108999.999999995	108999.999999995\\
-17999.9999999998	-17999.9999999998\\
-165000	-165000\\
201999.999999997	201999.999999997\\
365000.000000001	365000.000000001\\
-620999.999999999	-620999.999999999\\
145999.999999998	145999.999999998\\
-56000.0000000009	-56000.0000000009\\
184000	184000\\
111000.000000004	111000.000000004\\
-295000.000000004	-295000.000000004\\
-146000	-146000\\
202000.000000003	202000.000000003\\
311000.000000002	311000.000000002\\
-73000.0000000004	-73000.0000000004\\
-164999.999999998	-164999.999999998\\
-91999.9999999996	-91999.9999999996\\
38000.0000000011	38000.0000000011\\
236000	236000\\
-235999.999999999	-235999.999999999\\
-1000.00000000033	-1000.00000000033\\
218999.999999999	218999.999999999\\
-181999.999999998	-181999.999999998\\
-183000	-183000\\
384000.000000001	384000.000000001\\
-92000.0000000014	-92000.0000000014\\
-327999.999999998	-327999.999999998\\
345999.999999996	345999.999999996\\
-164000.000000001	-164000.000000001\\
37000.0000000017	37000.0000000017\\
-36999.9999999999	-36999.9999999999\\
-35999.9999999978	-35999.9999999978\\
254999.999999999	254999.999999999\\
-35000.0000000028	-35000.0000000028\\
-276000	-276000\\
54999.9999999997	54999.9999999997\\
129000	129000\\
36999.9999999981	36999.9999999981\\
-19999.9999999996	-19999.9999999996\\
-255000.000000003	-255000.000000003\\
310000.000000001	310000.000000001\\
20000.0000000022	20000.0000000022\\
-367000.000000002	-367000.000000002\\
293000	293000\\
163999.999999998	163999.999999998\\
-199999.999999999	-199999.999999999\\
16999.9999999986	16999.9999999986\\
-52999.9999999973	-52999.9999999973\\
-351000	-351000\\
460999.999999998	460999.999999998\\
143999.999999999	143999.999999999\\
-235999.999999999	-235999.999999999\\
-148000.000000002	-148000.000000002\\
-35999.9999999996	-35999.9999999996\\
255999.999999998	255999.999999998\\
-127999.999999998	-127999.999999998\\
92000.0000000005	92000.0000000005\\
72999.9999999995	72999.9999999995\\
-111000.000000001	-111000.000000001\\
167000.000000002	167000.000000002\\
70999.9999999971	70999.9999999971\\
-566000.000000002	-566000.000000002\\
274000	274000\\
-8.88178419700125e-10	-8.88178419700125e-10\\
-109000.000000001	-109000.000000001\\
272999.999999999	272999.999999999\\
-181999.999999999	-181999.999999999\\
328999.999999999	328999.999999999\\
-256000.000000001	-256000.000000001\\
-255999.999999998	-255999.999999998\\
511999.999999999	511999.999999999\\
-164000.000000001	-164000.000000001\\
-92000.0000000005	-92000.0000000005\\
-183000.000000002	-183000.000000002\\
54000.0000000011	54000.0000000011\\
55999.9999999991	55999.9999999991\\
348000.000000004	348000.000000004\\
-220000.000000003	-220000.000000003\\
-109999.999999999	-109999.999999999\\
-55000.0000000033	-55000.0000000033\\
37000.0000000026	37000.0000000026\\
-55000.0000000006	-55000.0000000006\\
346999.999999998	346999.999999998\\
-163999.999999998	-163999.999999998\\
110000.000000002	110000.000000002\\
-128000	-128000\\
-239000	-239000\\
459000	459000\\
-514000.000000003	-514000.000000003\\
239000.000000002	239000.000000002\\
17999.9999999998	17999.9999999998\\
-256000	-256000\\
291999.999999998	291999.999999998\\
38000.0000000038	38000.0000000038\\
-130000.000000002	-130000.000000002\\
277000.000000002	277000.000000002\\
70999.9999999979	70999.9999999979\\
-475000	-475000\\
128000	128000\\
-90999.9999999975	-90999.9999999975\\
90999.9999999975	90999.9999999975\\
-17999.9999999998	-17999.9999999998\\
309999.999999999	309999.999999999\\
-70999.9999999979	-70999.9999999979\\
-368000.000000003	-368000.000000003\\
73000.0000000013	73000.0000000013\\
368000.000000001	368000.000000001\\
-165999.999999998	-165999.999999998\\
-109000.000000001	-109000.000000001\\
71999.9999999965	71999.9999999965\\
-91000.0000000002	-91000.0000000002\\
36999.9999999999	36999.9999999999\\
-165000.000000002	-165000.000000002\\
36000.0000000005	36000.0000000005\\
311999.999999999	311999.999999999\\
-55000.0000000015	-55000.0000000015\\
-91999.9999999996	-91999.9999999996\\
146999.999999998	146999.999999998\\
-54999.9999999988	-54999.9999999988\\
-421000	-421000\\
109000.000000001	109000.000000001\\
55999.9999999991	55999.9999999991\\
309999.999999998	309999.999999998\\
-145000.000000001	-145000.000000001\\
-19000.0000000001	-19000.0000000001\\
-73000.0000000013	-73000.0000000013\\
-999.999999998557	-999.999999998557\\
239000	239000\\
-109999.999999997	-109999.999999997\\
0	0\\
220000	220000\\
-385000.000000004	-385000.000000004\\
-54999.9999999988	-54999.9999999988\\
54999.9999999988	54999.9999999988\\
255999.999999999	255999.999999999\\
-34999.9999999975	-34999.9999999975\\
-112000.000000003	-112000.000000003\\
-16999.9999999986	-16999.9999999986\\
-219000	-219000\\
271999.999999998	271999.999999998\\
314000	314000\\
-440999.999999999	-440999.999999999\\
-182000.000000002	-182000.000000002\\
200000.000000004	200000.000000004\\
293999.999999997	293999.999999997\\
-348999.999999999	-348999.999999999\\
222000.000000001	222000.000000001\\
-222000.000000003	-222000.000000003\\
-16999.9999999995	-16999.9999999995\\
401999.999999999	401999.999999999\\
-329999.999999999	-329999.999999999\\
-89999.9999999972	-89999.9999999972\\
291999.999999999	291999.999999999\\
19000.0000000028	19000.0000000028\\
17000.0000000003	17000.0000000003\\
-181000.000000003	-181000.000000003\\
126000	126000\\
-439000.000000003	-439000.000000003\\
422000.000000001	422000.000000001\\
-294000.000000002	-294000.000000002\\
294000.000000003	294000.000000003\\
73000.0000000022	73000.0000000022\\
-330000.000000002	-330000.000000002\\
403000	403000\\
-439000	-439000\\
310999.999999998	310999.999999998\\
-19000.0000000001	-19000.0000000001\\
-274000	-274000\\
256000	256000\\
-128000.000000002	-128000.000000002\\
56000.0000000027	56000.0000000027\\
108000.000000002	108000.000000002\\
-236999.999999999	-236999.999999999\\
17999.9999999998	17999.9999999998\\
182999.999999997	182999.999999997\\
-16999.9999999986	-16999.9999999986\\
-276999.999999999	-276999.999999999\\
221999.999999999	221999.999999999\\
217999.999999997	217999.999999997\\
-199999.999999998	-199999.999999998\\
-221000.000000002	-221000.000000002\\
165999.999999999	165999.999999999\\
273000.000000001	273000.000000001\\
-273000.000000001	-273000.000000001\\
347000	347000\\
-346999.999999999	-346999.999999999\\
-367000.000000003	-367000.000000003\\
256000	256000\\
367000.000000003	367000.000000003\\
-403000	-403000\\
366000.000000003	366000.000000003\\
36999.999999999	36999.999999999\\
-202000.000000001	-202000.000000001\\
-164000.000000001	-164000.000000001\\
90999.9999999993	90999.9999999993\\
127999.999999998	127999.999999998\\
91000.0000000002	91000.0000000002\\
-328000.000000001	-328000.000000001\\
146000.000000001	146000.000000001\\
71999.9999999983	71999.9999999983\\
2000.00000000156	2000.00000000156\\
-312999.999999999	-312999.999999999\\
146999.999999998	146999.999999998\\
274999.999999999	274999.999999999\\
-55000.0000000015	-55000.0000000015\\
-329999.999999999	-329999.999999999\\
111000	111000\\
455999.999999999	455999.999999999\\
-457000.000000001	-457000.000000001\\
56000.0000000036	56000.0000000036\\
235999.999999996	235999.999999996\\
-15999.9999999956	-15999.9999999956\\
-387000.000000005	-387000.000000005\\
94000.0000000038	94000.0000000038\\
400999.999999996	400999.999999996\\
-419999.999999998	-419999.999999998\\
164000.000000003	164000.000000003\\
110000.000000001	110000.000000001\\
-202000.000000001	-202000.000000001\\
-144000.000000001	-144000.000000001\\
198000.000000001	198000.000000001\\
130999.999999998	130999.999999998\\
-148000.000000002	-148000.000000002\\
-17999.999999998	-17999.999999998\\
256999.999999998	256999.999999998\\
-73999.9999999989	-73999.9999999989\\
-219999.999999996	-219999.999999996\\
-35999.9999999996	-35999.9999999996\\
55000.0000000006	55000.0000000006\\
90999.9999999984	90999.9999999984\\
-255999.999999998	-255999.999999998\\
365999.999999998	365999.999999998\\
72999.9999999977	72999.9999999977\\
-217999.999999996	-217999.999999996\\
-112000.000000004	-112000.000000004\\
38000.000000002	38000.000000002\\
181999.999999997	181999.999999997\\
-35999.9999999987	-35999.9999999987\\
90999.9999999993	90999.9999999993\\
-218000.000000001	-218000.000000001\\
34000.0000000016	34000.0000000016\\
56999.9999999986	56999.9999999986\\
-73999.9999999989	-73999.9999999989\\
-72999.9999999968	-72999.9999999968\\
292999.999999999	292999.999999999\\
-92000.0000000014	-92000.0000000014\\
-36999.9999999981	-36999.9999999981\\
-53000.0000000008	-53000.0000000008\\
-39000.0000000015	-39000.0000000015\\
-16999.9999999986	-16999.9999999986\\
329000	329000\\
-419999.999999998	-419999.999999998\\
35999.999999996	35999.999999996\\
108999.999999999	108999.999999999\\
184000	184000\\
-257000	-257000\\
184000	184000\\
-74000.0000000025	-74000.0000000025\\
92000.0000000023	92000.0000000023\\
-348000	-348000\\
54999.9999999997	54999.9999999997\\
420999.999999999	420999.999999999\\
0	0\\
-456999.999999999	-456999.999999999\\
345999.999999996	345999.999999996\\
-89999.9999999998	-89999.9999999998\\
-292999.999999999	-292999.999999999\\
456999.999999998	456999.999999998\\
-236999.999999999	-236999.999999999\\
-92999.9999999964	-92999.9999999964\\
257999.999999997	257999.999999997\\
-294999.999999997	-294999.999999997\\
258999.999999999	258999.999999999\\
52000.0000000005	52000.0000000005\\
-399999.999999999	-399999.999999999\\
290999.999999999	290999.999999999\\
-36000.0000000005	-36000.0000000005\\
17999.999999998	17999.999999998\\
1000.00000000122	1000.00000000122\\
-129000.000000001	-129000.000000001\\
220000	220000\\
-129000	-129000\\
-255000.000000001	-255000.000000001\\
584999.999999999	584999.999999999\\
-220000	-220000\\
-200000	-200000\\
236999.999999999	236999.999999999\\
-1.77635683940025e-09	-1.77635683940025e-09\\
-476000	-476000\\
458000	458000\\
73000.0000000004	73000.0000000004\\
-457000.000000002	-457000.000000002\\
310000.000000003	310000.000000003\\
-55000.0000000015	-55000.0000000015\\
277000.000000002	277000.000000002\\
-278000	-278000\\
3000.00000000011	3000.00000000011\\
-111999.999999999	-111999.999999999\\
-53999.9999999994	-53999.9999999994\\
312000	312000\\
-147999.999999998	-147999.999999998\\
-109000.000000001	-109000.000000001\\
90999.9999999984	90999.9999999984\\
294000	294000\\
-368000	-368000\\
21000.0000000017	21000.0000000017\\
271999.999999997	271999.999999997\\
-163999.999999998	-163999.999999998\\
-272999.999999998	-272999.999999998\\
491999.999999996	491999.999999996\\
-309999.999999998	-309999.999999998\\
127999.999999999	127999.999999999\\
-73000.0000000004	-73000.0000000004\\
-56000.0000000009	-56000.0000000009\\
349000.000000004	349000.000000004\\
-183000.000000001	-183000.000000001\\
-202000.000000003	-202000.000000003\\
-183000.000000001	-183000.000000001\\
329000	329000\\
999.999999999446	999.999999999446\\
110000.000000003	110000.000000003\\
71999.9999999965	71999.9999999965\\
-492999.999999998	-492999.999999998\\
-37999.9999999985	-37999.9999999985\\
348999.999999998	348999.999999998\\
145999.999999998	145999.999999998\\
73000.0000000031	73000.0000000031\\
-310000.000000001	-310000.000000001\\
-148000.000000003	-148000.000000003\\
37000.0000000026	37000.0000000026\\
275000	275000\\
-54999.999999997	-54999.999999997\\
109999.999999999	109999.999999999\\
-366000.000000002	-366000.000000002\\
163999.999999999	163999.999999999\\
183000	183000\\
-328999.999999998	-328999.999999998\\
311000.000000001	311000.000000001\\
1000.00000000122	1000.00000000122\\
-258000.000000004	-258000.000000004\\
203000.000000005	203000.000000005\\
199999.999999996	199999.999999996\\
-456999.999999999	-456999.999999999\\
1000.000000003	1000.000000003\\
143999.999999995	143999.999999995\\
75000.0000000037	75000.0000000037\\
127999.999999998	127999.999999998\\
-312000.000000001	-312000.000000001\\
184000.000000001	184000.000000001\\
91000.0000000002	91000.0000000002\\
-309999.999999997	-309999.999999997\\
216999.999999995	216999.999999995\\
-125999.999999997	-125999.999999997\\
110000.000000001	110000.000000001\\
199999.999999996	199999.999999996\\
-218999.999999997	-218999.999999997\\
127999.999999999	127999.999999999\\
-438000.000000001	-438000.000000001\\
417999.999999999	417999.999999999\\
-87999.9999999992	-87999.9999999992\\
-39000.0000000015	-39000.0000000015\\
-183000	-183000\\
202999.999999999	202999.999999999\\
70999.9999999997	70999.9999999997\\
-33999.9999999963	-33999.9999999963\\
198999.999999996	198999.999999996\\
-290999.999999998	-290999.999999998\\
52999.9999999964	52999.9999999964\\
-107999.999999999	-107999.999999999\\
-111000.000000001	-111000.000000001\\
348000	348000\\
-73000.0000000004	-73000.0000000004\\
-257000	-257000\\
240000	240000\\
34000.0000000007	34000.0000000007\\
-35000.0000000001	-35000.0000000001\\
-91999.9999999996	-91999.9999999996\\
348000.000000002	348000.000000002\\
-329000.000000001	-329000.000000001\\
-1000.00000000211	-1000.00000000211\\
-37000.0000000017	-37000.0000000017\\
-52999.9999999982	-52999.9999999982\\
254000	254000\\
-34000.0000000007	-34000.0000000007\\
-113000	-113000\\
-307999.999999998	-307999.999999998\\
418999.999999997	418999.999999997\\
54999.9999999997	54999.9999999997\\
-365999.999999998	-365999.999999998\\
276000.000000002	276000.000000002\\
217999.999999999	217999.999999999\\
-219000	-219000\\
-18000.0000000007	-18000.0000000007\\
-166000	-166000\\
-163000	-163000\\
492999.999999998	492999.999999998\\
-238000	-238000\\
20000.000000004	20000.000000004\\
15999.9999999938	15999.9999999938\\
56000.0000000018	56000.0000000018\\
-92000.0000000005	-92000.0000000005\\
184000	184000\\
-257000.000000002	-257000.000000002\\
201000	201000\\
-90000.0000000007	-90000.0000000007\\
-203000	-203000\\
128999.999999998	128999.999999998\\
146000.000000003	146000.000000003\\
219999.999999998	219999.999999998\\
-713999.999999998	-713999.999999998\\
531000.000000002	531000.000000002\\
218999.999999998	218999.999999998\\
-236999.999999997	-236999.999999997\\
-440000.000000001	-440000.000000001\\
385000.000000002	385000.000000002\\
53999.9999999985	53999.9999999985\\
-219000	-219000\\
293000.000000003	293000.000000003\\
-311000.000000002	-311000.000000002\\
-130000.000000002	-130000.000000002\\
460000.000000004	460000.000000004\\
109000	109000\\
-458000.000000001	-458000.000000001\\
128000.000000001	128000.000000001\\
38000.0000000002	38000.0000000002\\
144999.999999996	144999.999999996\\
-219999.999999998	-219999.999999998\\
239999.999999999	239999.999999999\\
-423000.000000003	-423000.000000003\\
239000.000000003	239000.000000003\\
-129000.000000003	-129000.000000003\\
239000	239000\\
-165999.999999999	-165999.999999999\\
313000.000000001	313000.000000001\\
-130000	-130000\\
-162999.999999999	-162999.999999999\\
-1999.99999999889	-1999.99999999889\\
93999.9999999985	93999.9999999985\\
106999.999999999	106999.999999999\\
-271999.999999998	-271999.999999998\\
34999.9999999957	34999.9999999957\\
220000	220000\\
-73000.0000000013	-73000.0000000013\\
-293000	-293000\\
146999.999999999	146999.999999999\\
309999.999999998	309999.999999998\\
-126999.999999998	-126999.999999998\\
-202000.000000002	-202000.000000002\\
-255999.999999999	-255999.999999999\\
439000	439000\\
402999.999999999	402999.999999999\\
-713999.999999998	-713999.999999998\\
218999.999999999	218999.999999999\\
38000.000000002	38000.000000002\\
-73999.9999999998	-73999.9999999998\\
129000	129000\\
52999.9999999973	52999.9999999973\\
-125999.999999998	-125999.999999998\\
-38999.9999999997	-38999.9999999997\\
-89000.0000000013	-89000.0000000013\\
-111000.000000002	-111000.000000002\\
347000	347000\\
112000.000000002	112000.000000002\\
-149000.000000003	-149000.000000003\\
-16999.9999999995	-16999.9999999995\\
-256000.000000002	-256000.000000002\\
-8.88178419700125e-10	-8.88178419700125e-10\\
237000.000000001	237000.000000001\\
74000.0000000007	74000.0000000007\\
36999.9999999999	36999.9999999999\\
-166000.000000004	-166000.000000004\\
-108999.999999998	-108999.999999998\\
91999.9999999987	91999.9999999987\\
-111000.000000002	-111000.000000002\\
238000.000000002	238000.000000002\\
110999.999999999	110999.999999999\\
-18999.9999999992	-18999.9999999992\\
-366000.000000003	-366000.000000003\\
146000.000000001	146000.000000001\\
1000.00000000033	1000.00000000033\\
199999.999999998	199999.999999998\\
-309999.999999999	-309999.999999999\\
110000	110000\\
71999.9999999974	71999.9999999974\\
91999.9999999987	91999.9999999987\\
-35999.999999996	-35999.999999996\\
-367000.000000001	-367000.000000001\\
182999.999999999	182999.999999999\\
999.999999998557	999.999999998557\\
219000.000000003	219000.000000003\\
-145999.999999998	-145999.999999998\\
145999.999999997	145999.999999997\\
-182999.999999999	-182999.999999999\\
-53999.9999999985	-53999.9999999985\\
15999.9999999982	15999.9999999982\\
369000.000000001	369000.000000001\\
-111999.999999999	-111999.999999999\\
-329000.000000003	-329000.000000003\\
74000.0000000016	74000.0000000016\\
17000.0000000021	17000.0000000021\\
367999.999999998	367999.999999998\\
-56999.9999999986	-56999.9999999986\\
-328000.000000002	-328000.000000002\\
-129000.000000001	-129000.000000001\\
294000.000000002	294000.000000002\\
16999.9999999986	16999.9999999986\\
-255999.999999998	-255999.999999998\\
-54000.0000000003	-54000.0000000003\\
217999.999999999	217999.999999999\\
56999.9999999986	56999.9999999986\\
108000	108000\\
-182000.000000001	-182000.000000001\\
-109999.999999998	-109999.999999998\\
91000.0000000011	91000.0000000011\\
109999.999999998	109999.999999998\\
-36000.0000000005	-36000.0000000005\\
54999.9999999997	54999.9999999997\\
-20000.0000000005	-20000.0000000005\\
-162000	-162000\\
-93999.9999999976	-93999.9999999976\\
183999.999999998	183999.999999998\\
164999.999999998	164999.999999998\\
-54999.9999999979	-54999.9999999979\\
-402999.999999997	-402999.999999997\\
494000.000000001	494000.000000001\\
-274000	-274000\\
-19000.000000001	-19000.000000001\\
37999.9999999994	37999.9999999994\\
89999.999999999	89999.999999999\\
183999.999999998	183999.999999998\\
-439999.999999998	-439999.999999998\\
237999.999999998	237999.999999998\\
148000.000000003	148000.000000003\\
-167000.000000002	-167000.000000002\\
-219000.000000001	-219000.000000001\\
184000.000000003	184000.000000003\\
182999.999999998	182999.999999998\\
-166000	-166000\\
128999.999999998	128999.999999998\\
-109999.999999999	-109999.999999999\\
-37000.0000000035	-37000.0000000035\\
-108999.999999999	-108999.999999999\\
346999.999999999	346999.999999999\\
-366000	-366000\\
221000.000000003	221000.000000003\\
-2000.00000000333	-2000.00000000333\\
-419999.999999997	-419999.999999997\\
714000.000000001	714000.000000001\\
-36999.999999999	-36999.999999999\\
-347000	-347000\\
-166000.000000003	-166000.000000003\\
56000.0000000018	56000.0000000018\\
108999.999999997	108999.999999997\\
37000.0000000026	37000.0000000026\\
37000.0000000017	37000.0000000017\\
-1000.00000000122	-1000.00000000122\\
148000.000000001	148000.000000001\\
-238999.999999999	-238999.999999999\\
-74000.0000000043	-74000.0000000043\\
166000	166000\\
128000	128000\\
-129000.000000002	-129000.000000002\\
-327999.999999998	-327999.999999998\\
345999.999999998	345999.999999998\\
-127000	-127000\\
17999.9999999989	17999.9999999989\\
220000.000000001	220000.000000001\\
-18000.0000000016	-18000.0000000016\\
-202999.999999999	-202999.999999999\\
-145000	-145000\\
165000.000000002	165000.000000002\\
16999.9999999995	16999.9999999995\\
56000.0000000018	56000.0000000018\\
-1000.000000003	-1000.000000003\\
93000.0000000044	93000.0000000044\\
-111000.000000001	-111000.000000001\\
146999.999999997	146999.999999997\\
-129000	-129000\\
-438000.000000002	-438000.000000002\\
346000.000000003	346000.000000003\\
478000.000000001	478000.000000001\\
-734000.000000003	-734000.000000003\\
185000.000000003	185000.000000003\\
143999.999999997	143999.999999997\\
313000.000000001	313000.000000001\\
-127999.999999997	-127999.999999997\\
-239000.000000003	-239000.000000003\\
-256000.000000002	-256000.000000002\\
275000	275000\\
53999.9999999976	53999.9999999976\\
111000.000000002	111000.000000002\\
-312000.000000001	-312000.000000001\\
220000	220000\\
-109999.999999999	-109999.999999999\\
73000.0000000004	73000.0000000004\\
-91000.0000000028	-91000.0000000028\\
-72999.9999999968	-72999.9999999968\\
309999.999999997	309999.999999997\\
-272999.999999999	-272999.999999999\\
89999.9999999981	89999.9999999981\\
147000.000000001	147000.000000001\\
111000.000000001	111000.000000001\\
-332000.000000002	-332000.000000002\\
-124999.999999998	-124999.999999998\\
472999.999999997	472999.999999997\\
-327999.999999999	-327999.999999999\\
128000.000000002	128000.000000002\\
-292000.000000001	-292000.000000001\\
326999.999999998	326999.999999998\\
131000.000000002	131000.000000002\\
-167000.000000002	-167000.000000002\\
-182000	-182000\\
-164999.999999997	-164999.999999997\\
274999.999999999	274999.999999999\\
127999.999999998	127999.999999998\\
91000.0000000002	91000.0000000002\\
-310000	-310000\\
-74999.9999999984	-74999.9999999984\\
183999.999999997	183999.999999997\\
220000.000000001	220000.000000001\\
-184000.000000003	-184000.000000003\\
-199999.999999997	-199999.999999997\\
163999.999999996	163999.999999996\\
73000.0000000004	73000.0000000004\\
-128000.000000001	-128000.000000001\\
109999.999999999	109999.999999999\\
55000.0000000006	55000.0000000006\\
-110000.000000002	-110000.000000002\\
-146999.999999999	-146999.999999999\\
221000	221000\\
-148000.000000001	-148000.000000001\\
-34999.9999999993	-34999.9999999993\\
-2000.00000000067	-2000.00000000067\\
423000.000000001	423000.000000001\\
-311999.999999999	-311999.999999999\\
-238000	-238000\\
457999.999999998	457999.999999998\\
-56000.0000000009	-56000.0000000009\\
-200000.000000001	-200000.000000001\\
16999.9999999986	16999.9999999986\\
-72000.0000000001	-72000.0000000001\\
90999.9999999993	90999.9999999993\\
8.88178419700125e-10	8.88178419700125e-10\\
-311000.000000003	-311000.000000003\\
787000.000000002	787000.000000002\\
-531999.999999998	-531999.999999998\\
-218000.000000002	-218000.000000002\\
237999.999999999	237999.999999999\\
199999.999999998	199999.999999998\\
-53999.9999999994	-53999.9999999994\\
-54000.0000000003	-54000.0000000003\\
-332000	-332000\\
92999.9999999991	92999.9999999991\\
219999.999999999	219999.999999999\\
238000.000000001	238000.000000001\\
-202000.000000001	-202000.000000001\\
-202000.000000002	-202000.000000002\\
130000.000000002	130000.000000002\\
-38000.000000002	-38000.000000002\\
1000.000000003	1000.000000003\\
34999.9999999957	34999.9999999957\\
38000.0000000038	38000.0000000038\\
72999.9999999986	72999.9999999986\\
-36999.999999999	-36999.999999999\\
-18000.0000000007	-18000.0000000007\\
-238000	-238000\\
73999.9999999998	73999.9999999998\\
345999.999999998	345999.999999998\\
-254999.999999999	-254999.999999999\\
-311999.999999998	-311999.999999998\\
402999.999999997	402999.999999997\\
56000.0000000018	56000.0000000018\\
-148000.000000001	-148000.000000001\\
-126999.999999995	-126999.999999995\\
511999.999999999	511999.999999999\\
-713000	-713000\\
418999.999999998	418999.999999998\\
75000.0000000002	75000.0000000002\\
-220000	-220000\\
71999.9999999992	71999.9999999992\\
-308999.999999997	-308999.999999997\\
510999.999999998	510999.999999998\\
110000	110000\\
-530999.999999999	-530999.999999999\\
147000	147000\\
165000.000000003	165000.000000003\\
-129000	-129000\\
109999.999999999	109999.999999999\\
-236999.999999997	-236999.999999997\\
364999.999999999	364999.999999999\\
-54999.9999999979	-54999.9999999979\\
-109000.000000001	-109000.000000001\\
-146999.999999998	-146999.999999998\\
73999.9999999989	73999.9999999989\\
90000.0000000007	90000.0000000007\\
-309000	-309000\\
180000	180000\\
406000	406000\\
-349999.999999999	-349999.999999999\\
-109000.000000001	-109000.000000001\\
330000	330000\\
-386000.000000001	-386000.000000001\\
403999.999999998	403999.999999998\\
-366999.999999999	-366999.999999999\\
93000	93000\\
-56000.0000000009	-56000.0000000009\\
238000.000000002	238000.000000002\\
-72999.9999999986	-72999.9999999986\\
-18999.9999999983	-18999.9999999983\\
-35000.000000001	-35000.000000001\\
-295000.000000001	-295000.000000001\\
496000	496000\\
-220000.000000001	-220000.000000001\\
-349000	-349000\\
458999.999999997	458999.999999997\\
54000.000000002	54000.000000002\\
19999.9999999996	19999.9999999996\\
-314000.000000001	-314000.000000001\\
-15999.9999999982	-15999.9999999982\\
586000.000000002	586000.000000002\\
-313000.000000002	-313000.000000002\\
-237999.999999999	-237999.999999999\\
129999.999999999	129999.999999999\\
-256999.999999999	-256999.999999999\\
181999.999999999	181999.999999999\\
-17999.9999999989	-17999.9999999989\\
459000	459000\\
-642000.000000002	-642000.000000002\\
166000	166000\\
327999.999999998	327999.999999998\\
-91000.0000000011	-91000.0000000011\\
55000.0000000024	55000.0000000024\\
-457000.000000002	-457000.000000002\\
145999.999999999	145999.999999999\\
108999.999999998	108999.999999998\\
312000	312000\\
-91000.0000000011	-91000.0000000011\\
-605999.999999999	-605999.999999999\\
515000	515000\\
162999.999999998	162999.999999998\\
-309999.999999999	-309999.999999999\\
-403999.999999999	-403999.999999999\\
129000	129000\\
640999.999999999	640999.999999999\\
-56000.0000000009	-56000.0000000009\\
-201000.000000001	-201000.000000001\\
-218999.999999998	-218999.999999998\\
511999.999999997	511999.999999997\\
-731999.999999997	-731999.999999997\\
255999.999999998	255999.999999998\\
37000.0000000008	37000.0000000008\\
-8.88178419700125e-10	-8.88178419700125e-10\\
-38000.0000000002	-38000.0000000002\\
295000	295000\\
-129000	-129000\\
-19000.0000000019	-19000.0000000019\\
-35000.0000000001	-35000.0000000001\\
52999.9999999999	52999.9999999999\\
-35000.0000000019	-35000.0000000019\\
-130000	-130000\\
-88999.999999996	-88999.999999996\\
400999.999999994	400999.999999994\\
165000.000000002	165000.000000002\\
-275000.000000003	-275000.000000003\\
-712999.999999998	-712999.999999998\\
676999.999999998	676999.999999998\\
165000	165000\\
-19000.000000001	-19000.000000001\\
-348000	-348000\\
-145000	-145000\\
199999.999999997	199999.999999997\\
275000.000000001	275000.000000001\\
-145999.999999999	-145999.999999999\\
-19000.0000000028	-19000.0000000028\\
128000.000000004	128000.000000004\\
-200000	-200000\\
346000	346000\\
-400999.999999999	-400999.999999999\\
-202000.000000003	-202000.000000003\\
310000	310000\\
56000	56000\\
-53999.9999999967	-53999.9999999967\\
162999.999999996	162999.999999996\\
-329000	-329000\\
202000.000000001	202000.000000001\\
182999.999999999	182999.999999999\\
-184000.000000001	-184000.000000001\\
-127000.000000001	-127000.000000001\\
-146999.999999999	-146999.999999999\\
348000	348000\\
72000.0000000027	72000.0000000027\\
-511000.000000001	-511000.000000001\\
110000	110000\\
565999.999999997	565999.999999997\\
-272999.999999997	-272999.999999997\\
-258000.000000001	-258000.000000001\\
-17000.0000000003	-17000.0000000003\\
531000.000000001	531000.000000001\\
-92000.0000000032	-92000.0000000032\\
-238999.999999995	-238999.999999995\\
-547000.000000002	-547000.000000002\\
419000.000000001	419000.000000001\\
293999.999999995	293999.999999995\\
-256999.999999997	-256999.999999997\\
202000.000000002	202000.000000002\\
55000.0000000006	55000.0000000006\\
73999.9999999989	73999.9999999989\\
-386999.999999999	-386999.999999999\\
20999.999999999	20999.999999999\\
236000.000000001	236000.000000001\\
-71999.9999999983	-71999.9999999983\\
-147000.000000002	-147000.000000002\\
128000.000000003	128000.000000003\\
1000.00000000122	1000.00000000122\\
-148000.000000002	-148000.000000002\\
368000	368000\\
-1000.00000000033	-1000.00000000033\\
-184000.000000002	-184000.000000002\\
-181999.999999998	-181999.999999998\\
-91000.0000000019	-91000.0000000019\\
181999.999999999	181999.999999999\\
54999.9999999997	54999.9999999997\\
-273999.999999998	-273999.999999998\\
548999.999999998	548999.999999998\\
-328999.999999999	-328999.999999999\\
-112000.000000002	-112000.000000002\\
424000.000000003	424000.000000003\\
-423000.000000004	-423000.000000004\\
36999.9999999999	36999.9999999999\\
109999.999999999	109999.999999999\\
312000.000000002	312000.000000002\\
-295000.000000003	-295000.000000003\\
-254000.000000001	-254000.000000001\\
255000.000000001	255000.000000001\\
-402000	-402000\\
455999.999999999	455999.999999999\\
222000.000000004	222000.000000004\\
-368000.000000002	-368000.000000002\\
38000.0000000047	38000.0000000047\\
-37000.0000000026	-37000.0000000026\\
237000.000000002	237000.000000002\\
111000.000000002	111000.000000002\\
-238000.000000002	-238000.000000002\\
54000.0000000003	54000.0000000003\\
-347000.000000004	-347000.000000004\\
128000.000000002	128000.000000002\\
71999.9999999983	71999.9999999983\\
240000.000000001	240000.000000001\\
-220999.999999998	-220999.999999998\\
-72000.0000000009	-72000.0000000009\\
89999.9999999963	89999.9999999963\\
19000.0000000019	19000.0000000019\\
-73000.0000000022	-73000.0000000022\\
292000.000000001	292000.000000001\\
-35000.000000001	-35000.000000001\\
-203000	-203000\\
-89999.9999999998	-89999.9999999998\\
17000.0000000021	17000.0000000021\\
17999.9999999998	17999.9999999998\\
-108000.000000001	-108000.000000001\\
291999.999999999	291999.999999999\\
-110999.999999996	-110999.999999996\\
-163000.000000004	-163000.000000004\\
456000	456000\\
-510999.999999997	-510999.999999997\\
17999.9999999989	17999.9999999989\\
620999.999999999	620999.999999999\\
-400999.999999996	-400999.999999996\\
-110000	-110000\\
16999.9999999968	16999.9999999968\\
294000.000000001	294000.000000001\\
-202000.000000003	-202000.000000003\\
91999.9999999987	91999.9999999987\\
-292999.999999998	-292999.999999998\\
-184000.000000002	-184000.000000002\\
277000.000000004	277000.000000004\\
254000	254000\\
109999.999999999	109999.999999999\\
-217999.999999996	-217999.999999996\\
16999.9999999995	16999.9999999995\\
-202000	-202000\\
-34999.9999999993	-34999.9999999993\\
512000	512000\\
-403999.999999999	-403999.999999999\\
20999.9999999981	20999.9999999981\\
-295999.999999999	-295999.999999999\\
478000.000000002	478000.000000002\\
146000	146000\\
-383999.999999999	-383999.999999999\\
53999.9999999976	53999.9999999976\\
-201000.000000001	-201000.000000001\\
127000.000000002	127000.000000002\\
277000.000000003	277000.000000003\\
17000.0000000012	17000.0000000012\\
-165000.000000002	-165000.000000002\\
-127000.000000002	-127000.000000002\\
145000.000000001	145000.000000001\\
91999.999999997	91999.999999997\\
-165000	-165000\\
19000.0000000001	19000.0000000001\\
17999.9999999989	17999.9999999989\\
-108999.999999999	-108999.999999999\\
-56999.9999999995	-56999.9999999995\\
74999.9999999984	74999.9999999984\\
274000.000000002	274000.000000002\\
18999.9999999992	18999.9999999992\\
-239000.000000003	-239000.000000003\\
-17999.9999999971	-17999.9999999971\\
-19000.0000000001	-19000.0000000001\\
-53000.0000000008	-53000.0000000008\\
199999.999999998	199999.999999998\\
18000.0000000016	18000.0000000016\\
36999.9999999972	36999.9999999972\\
-566999.999999998	-566999.999999998\\
494000.000000001	494000.000000001\\
293000.000000002	293000.000000002\\
-73000.0000000022	-73000.0000000022\\
-550000	-550000\\
1.77635683940025e-09	1.77635683940025e-09\\
402999.999999997	402999.999999997\\
-72999.9999999995	-72999.9999999995\\
111000.000000002	111000.000000002\\
-497000.000000001	-497000.000000001\\
257999.999999999	257999.999999999\\
403000	403000\\
-73999.9999999989	-73999.9999999989\\
-530000.000000003	-530000.000000003\\
-183999.999999999	-183999.999999999\\
621999.999999999	621999.999999999\\
203000.000000001	203000.000000001\\
-404000.000000002	-404000.000000002\\
-329000.000000001	-329000.000000001\\
348000.000000002	348000.000000002\\
255999.999999999	255999.999999999\\
-367000.000000001	-367000.000000001\\
38000.0000000038	38000.0000000038\\
457000	457000\\
-347000	-347000\\
-276000.000000002	-276000.000000002\\
477000.000000001	477000.000000001\\
-367000.000000002	-367000.000000002\\
56000.0000000045	56000.0000000045\\
291999.999999997	291999.999999997\\
38000.0000000029	38000.0000000029\\
-625000.000000003	-625000.000000003\\
94000.0000000021	94000.0000000021\\
512000.000000002	512000.000000002\\
16999.9999999995	16999.9999999995\\
-217999.999999999	-217999.999999999\\
-221000.000000002	-221000.000000002\\
368000.000000003	368000.000000003\\
-21000.0000000008	-21000.0000000008\\
-235000.000000001	-235000.000000001\\
-94000.0000000012	-94000.0000000012\\
183999.999999998	183999.999999998\\
72999.9999999995	72999.9999999995\\
-145999.999999998	-145999.999999998\\
274999.999999998	274999.999999998\\
-330999.999999999	-330999.999999999\\
-17000.000000003	-17000.000000003\\
420000.000000001	420000.000000001\\
-255000	-255000\\
-74000.0000000007	-74000.0000000007\\
72999.9999999968	72999.9999999968\\
-108999.999999997	-108999.999999997\\
-129000.000000001	-129000.000000001\\
329000	329000\\
-107999.999999999	-107999.999999999\\
-111000.000000002	-111000.000000002\\
146000.000000004	146000.000000004\\
165999.999999999	165999.999999999\\
-164999.999999999	-164999.999999999\\
201000	201000\\
-587000.000000002	-587000.000000002\\
75000.0000000028	75000.0000000028\\
640999.999999998	640999.999999998\\
-130000.000000001	-130000.000000001\\
-658000.000000001	-658000.000000001\\
256000.000000002	256000.000000002\\
366999.999999999	366999.999999999\\
-293999.999999998	-293999.999999998\\
1000.00000000122	1000.00000000122\\
420000.000000002	420000.000000002\\
-292000.000000001	-292000.000000001\\
-202000.000000001	-202000.000000001\\
367000.000000002	367000.000000002\\
-185000.000000002	-185000.000000002\\
-198999.999999997	-198999.999999997\\
108000	108000\\
111999.999999999	111999.999999999\\
15999.9999999982	15999.9999999982\\
146999.999999999	146999.999999999\\
-255999.999999998	-255999.999999998\\
1000.00000000033	1000.00000000033\\
-93000.0000000035	-93000.0000000035\\
201999.999999999	201999.999999999\\
17999.9999999998	17999.9999999998\\
-383999.999999998	-383999.999999998\\
420999.999999998	420999.999999998\\
146000.000000001	146000.000000001\\
-292000.000000002	-292000.000000002\\
-166000	-166000\\
477000	477000\\
-147000.000000001	-147000.000000001\\
-385000.000000002	-385000.000000002\\
130000	130000\\
235999.999999999	235999.999999999\\
-16999.9999999986	-16999.9999999986\\
53999.9999999967	53999.9999999967\\
-437999.999999996	-437999.999999996\\
125999.999999999	125999.999999999\\
111999.999999997	111999.999999997\\
163000.000000001	163000.000000001\\
-144999.999999997	-144999.999999997\\
72999.9999999995	72999.9999999995\\
346999.999999999	346999.999999999\\
-421000.000000002	-421000.000000002\\
-474999.999999998	-474999.999999998\\
530000	530000\\
312000.000000001	312000.000000001\\
-312000.000000001	-312000.000000001\\
-311000.000000003	-311000.000000003\\
549000.000000003	549000.000000003\\
-145999.999999998	-145999.999999998\\
-109999.999999999	-109999.999999999\\
-54999.9999999988	-54999.9999999988\\
-127000.000000002	-127000.000000002\\
399999.999999999	399999.999999999\\
-197999.999999999	-197999.999999999\\
-351000	-351000\\
368000.000000001	368000.000000001\\
74000.0000000007	74000.0000000007\\
271999.999999999	271999.999999999\\
-327000.000000002	-327000.000000002\\
-698000	-698000\\
917999.999999999	917999.999999999\\
52000.0000000014	52000.0000000014\\
-436000	-436000\\
-222999.999999998	-222999.999999998\\
221999.999999996	221999.999999996\\
17000.0000000003	17000.0000000003\\
440000.000000001	440000.000000001\\
-257000.000000003	-257000.000000003\\
-35999.9999999987	-35999.9999999987\\
-17999.9999999989	-17999.9999999989\\
-275000.000000003	-275000.000000003\\
275000	275000\\
-147000.000000001	-147000.000000001\\
-36999.9999999999	-36999.9999999999\\
349000	349000\\
53999.9999999985	53999.9999999985\\
-365999.999999999	-365999.999999999\\
-73000.0000000004	-73000.0000000004\\
147000	147000\\
255000.000000004	255000.000000004\\
-401999.999999999	-401999.999999999\\
238000	238000\\
110000	110000\\
-550000.000000002	-550000.000000002\\
329999.999999999	329999.999999999\\
55999.9999999991	55999.9999999991\\
-38999.9999999997	-38999.9999999997\\
314000.000000003	314000.000000003\\
88999.9999999968	88999.9999999968\\
-455999.999999999	-455999.999999999\\
-91999.9999999996	-91999.9999999996\\
329999.999999998	329999.999999998\\
-165999.999999997	-165999.999999997\\
-366000	-366000\\
476999.999999998	476999.999999998\\
37000.0000000017	37000.0000000017\\
-185000.000000003	-185000.000000003\\
20000.0000000013	20000.0000000013\\
53999.9999999967	53999.9999999967\\
2.66453525910038e-09	2.66453525910038e-09\\
-109000	-109000\\
183000.000000001	183000.000000001\\
-221000.000000004	-221000.000000004\\
477000	477000\\
-147000	-147000\\
-456000	-456000\\
-57999.9999999972	-57999.9999999972\\
515999.999999999	515999.999999999\\
-167000.000000002	-167000.000000002\\
165000.000000001	165000.000000001\\
-439000.000000002	-439000.000000002\\
312000.000000003	312000.000000003\\
54000.0000000003	54000.0000000003\\
-384999.999999999	-384999.999999999\\
38000.0000000011	38000.0000000011\\
310000	310000\\
-201000.000000003	-201000.000000003\\
128999.999999999	128999.999999999\\
273000.000000001	273000.000000001\\
94000.0000000038	94000.0000000038\\
-900000.000000003	-900000.000000003\\
478000.000000001	478000.000000001\\
401999.999999997	401999.999999997\\
-585999.999999999	-585999.999999999\\
-90999.9999999993	-90999.9999999993\\
513000.000000002	513000.000000002\\
-349000.000000005	-349000.000000005\\
147000.000000002	147000.000000002\\
401999.999999999	401999.999999999\\
-400999.999999997	-400999.999999997\\
-93000.0000000008	-93000.0000000008\\
-53999.9999999976	-53999.9999999976\\
35999.9999999978	35999.9999999978\\
384999.999999999	384999.999999999\\
-148000	-148000\\
};
\end{axis}

\begin{axis}[%
width=4.927cm,
height=3.484cm,
at={(6.484cm,9.677cm)},
scale only axis,
xmin=-1000000,
xmax=1000000,
xlabel style={font=\color{white!15!black}},
xlabel={$\delta^3 u(t)$},
ymin=-5859300000,
ymax=4760800000,
ylabel style={font=\color{white!15!black}},
ylabel={y(t)},
axis background/.style={fill=white},
title={C5, R = 0.3706},
axis x line*=bottom,
axis y line*=left
]
\addplot[only marks, mark=*, mark options={}, mark size=1.5000pt, color=mycolor1, fill=mycolor1] table[row sep=crcr]{%
x	y\\
-201000	2197400000\\
310999.999999996	-732400000\\
-201000	610199999.999999\\
219000	-976499999.999999\\
54999.9999999988	1342800000\\
-419999.999999995	-2441400000\\
145999.999999996	2319400000\\
53999.9999999994	-244399999.999999\\
111000.000000004	-976200000\\
200999.999999996	976300000\\
-329000	-854300000\\
-129999.999999999	-122400000\\
-345999.999999999	1221100000\\
456999.999999999	-2075500000\\
365999.999999997	3296100000\\
-237999.999999998	-2563500000\\
55999.9999999991	366099999.999999\\
17000.0000000003	610300000\\
-604000.000000003	-1220400000\\
678000	1830800000\\
-37000.0000000008	-2441400000\\
-567000.000000001	2441600000\\
585000	-610700000\\
-128000.000000001	-243800000\\
-311000.000000001	-1587000000\\
367000	3051600000\\
16000.0000000009	-1708900000\\
-381999.999999998	-1220500000\\
328999.999999996	2685300000\\
17000.0000000003	-1709000000\\
-91000.0000000011	200000.000000067\\
1000.00000000389	732300000\\
-129000.000000002	-610400000.000001\\
36999.9999999999	100000.0000003\\
145999.999999997	1098600000\\
-91000.0000000011	-1464900000\\
17000.0000000021	732500000\\
-33999.999999998	-122100000\\
-58000.0000000043	-366199999.999999\\
222000.000000002	976500000\\
53999.9999999985	-122000000\\
-275000.000000002	-1708900000\\
257000.000000002	1586700000\\
-330000.000000004	100000.000000477\\
19000.0000000028	-976400000\\
254999.999999997	1098300000\\
-107999.999999998	-366000000\\
-148000.000000001	-1098600000\\
165999.999999999	2075200000\\
72000.0000000001	-1098900000\\
-293000	-1098300000\\
423000.000000002	2075100000\\
15999.9999999982	-244300000.000002\\
-328000.000000001	-2197100000\\
-147999.999999998	1709000000\\
258000.000000001	-366300000\\
200999.999999999	1220700000\\
-293000.000000001	-2075100000\\
-129000	854400000\\
366999.999999998	732400000\\
-164999.999999999	-488200000\\
-109999.999999999	-1098700000\\
109999.999999999	1587100000\\
-128000	-610600000\\
218999.999999999	244200000\\
-71999.9999999974	-366199999.999999\\
16999.9999999977	-488100000\\
-52999.9999999955	1708800000\\
124999.999999996	-2075300000\\
-124999.999999996	1587200000\\
16999.9999999959	-976800000\\
-19000.000000001	854600000.000001\\
19000.000000001	-1342800000\\
-200999.999999999	854600000\\
72999.9999999995	488000000.000001\\
329000.000000001	122399999.999999\\
-108999.999999998	-1098800000\\
108999.999999998	854399999.999999\\
-493999.999999999	-1708700000\\
238000	1586600000\\
165000.000000001	1343000000\\
37000.0000000026	-1953300000\\
-331000.000000004	-1586800000\\
57000.0000000039	3296000000\\
418999.999999997	-244399999.999999\\
-218999.999999998	-2075100000\\
-35999.9999999996	488499999.999999\\
-17999.9999999998	732000000.000001\\
-111000.000000006	122500000\\
-17999.9999999989	-1098800000\\
201000	1220500000\\
56000	366500000\\
-256999.999999999	-2441500000\\
-54000.0000000003	2075200000\\
52999.9999999973	-366300000\\
222000.000000003	488300000\\
-75000.0000000028	-1220500000\\
-128000	610200000\\
111000.000000002	-200000.000000244\\
-111000.000000002	244600000.000001\\
257000.000000001	-366700000.000001\\
-201999.999999999	366600000\\
129000.000000001	-854800000\\
-36999.999999999	1709300000\\
-201000.000000003	-2319600000\\
273000.000000001	1831200000\\
-126000.000000001	-610500000\\
-130000	-488099999.999999\\
148000	732400000\\
126999.999999998	488100000\\
-256000	-1830900000\\
202000.000000001	1953200000\\
-184000.000000002	-1343100000\\
56000.0000000009	244600000\\
163999.999999997	1098200000\\
-238000	-1464500000\\
-8.88178419700125e-10	243900000\\
54999.9999999988	732500000\\
111000.000000002	366300000.000001\\
-21000.0000000017	-1342900000\\
222000.000000001	854500000\\
-494999.999999999	-976500000\\
37000.0000000026	244200000\\
493999.999999996	3051600000\\
-367000	-4272400000\\
74000.0000000016	854499999.999999\\
257000.000000001	2929800000\\
-275000	-3052000000\\
-239000.000000002	-121800000.000001\\
256999.999999998	2563300000\\
-182999.999999998	-2685500000\\
218999.999999999	2441300000\\
130000.000000002	-1952900000\\
-20000.0000000005	1830900000\\
-220000.000000001	-2441400000\\
38000.0000000038	1953100000\\
91000.0000000002	-854300000\\
-110000.000000002	243999999.999999\\
201999.999999999	610199999.999999\\
-184999.999999999	-1098400000\\
3000.00000000189	122000000\\
33999.9999999945	854400000\\
-89999.9999999972	-976300000\\
-38000.000000002	854200000\\
38000.0000000029	-610299999.999999\\
346999.999999999	1098800000\\
-89999.9999999981	-976800000\\
-241000.000000003	-976300000.000001\\
3000.000000001	1586700000\\
235999.999999999	122200000\\
-144999.999999998	-1098600000\\
-202000.000000003	-122400000\\
-55999.9999999974	1221200000\\
459999.999999997	-488700000\\
-221999.999999999	122300000\\
-72000.0000000036	-976500000\\
165000.000000003	1342400000\\
-38000.000000002	-854000000\\
20000.0000000005	121700000\\
-111000.000000001	610500000\\
-163999.999999999	-1464800000\\
-92000.0000000005	976400000\\
346999.999999999	1220900000\\
94000.0000000021	-1709200000\\
161999.999999997	854700000\\
-290999.999999999	-732500000\\
-1000.00000000211	-122200000\\
19000.0000000019	976900000\\
-440000	-1221100000\\
493999.999999999	1587100000\\
-127000.000000002	-1098600000\\
-19000.0000000001	122100000\\
-110000	-122300000\\
275000.000000001	1221000000\\
165000.000000002	-1709100000\\
-294000	1098400000\\
165999.999999998	-487799999.999999\\
-276000	-244600000.000001\\
57000.0000000022	488500000\\
88999.9999999977	0\\
76000.0000000014	610400000.000001\\
-76000.0000000014	-1953300000\\
-89999.999999999	1464900000\\
74000.0000000007	244300000\\
-39000.0000000015	-976800000\\
186000.000000001	1098800000\\
-424000	-1831000000\\
239999.999999999	2441100000\\
329000	-732000000\\
-183000	-1465200000\\
-239000.000000001	366300000\\
-181999.999999999	854700000\\
348000.000000002	854199999.999999\\
-18000.0000000007	-1586700000\\
71000.0000000006	-99999.9999999446\\
-33999.999999998	1342800000\\
53999.9999999985	-1098700000\\
-110999.999999999	-244000000\\
-144999.999999999	366000000\\
109999.999999999	488600000.000001\\
310000	243799999.999999\\
-201000	-732200000\\
-347000.000000001	-1831100000\\
291999.999999998	4272300000\\
219999.999999999	-2929400000\\
-494999.999999998	-488500000\\
203000.000000002	2441400000\\
199999.999999998	-1830800000\\
-89999.9999999972	488000000\\
-240000.000000004	-244100000\\
274999.999999999	610500000\\
-199999.999999997	-610500000\\
16999.9999999977	122200000\\
239000.000000001	610300000\\
-999.999999998557	121899999.999999\\
-237000.000000002	-2319100000\\
-93000	2319300000\\
38000.0000000002	-488399999.999999\\
476000.000000002	610500000\\
-129000.000000001	-732599999.999999\\
-255999.999999999	-1342600000\\
-146000.000000004	1953000000\\
237000.000000002	-244000000\\
92999.9999999973	-610500000\\
-257999.999999998	244100000\\
37999.9999999976	-365900000\\
54000.0000000003	854100000\\
92000.0000000005	-487999999.999999\\
127999.999999998	-122200000\\
-512000.000000001	-122000000\\
383000.000000002	488200000\\
111000	-122100000.000001\\
-108999.999999999	366399999.999999\\
-130000.000000003	-1831300000\\
-36000.0000000005	1953400000\\
347999.999999998	365999999.999999\\
2.66453525910038e-09	-1464800000\\
-567000.000000003	-1342800000\\
254999.999999999	4028500000\\
-17000.0000000003	-3174100000\\
218999.999999999	1465100000\\
-17999.9999999998	-854700000.000001\\
-330000.000000003	488400000\\
275000.000000001	-366200000\\
127999.999999998	1098500000\\
-346999.999999997	-1830900000\\
217999.999999995	1098500000\\
-16999.9999999977	732600000\\
-129000.000000003	-2075300000\\
276000.000000004	2319200000\\
-313000.000000004	-1586600000\\
203000.000000003	-400000.000000222\\
-130000.000000001	1343200000\\
-200000	-1953400000\\
348000	2319300000\\
-127999.999999999	-1586700000\\
-20000.0000000031	-366400000\\
385999.999999999	2685800000\\
-183999.999999999	-3418400000\\
-219000.000000002	854900000\\
-55000.0000000006	1464700000\\
-18999.9999999992	-1220700000\\
367999.999999999	1098500000\\
-294999.999999999	-1708700000\\
-90999.9999999993	1830800000\\
36999.999999999	-1953100000\\
273999.999999998	2807800000\\
-181999.999999997	-3540100000\\
256000	3539800000\\
-221000.000000004	-2807400000\\
221000	1587000000\\
-421000.000000001	-854600000\\
256000.000000001	244000000\\
-165000	610500000\\
8.88178419700125e-10	-1098500000\\
-35999.9999999996	487900000\\
200000	1221100000\\
37999.9999999985	-1587100000\\
-110000.000000002	-99999.9999995005\\
-146999.999999999	488600000\\
329999.999999997	1464500000\\
18000.0000000016	-2929600000\\
-312000.000000002	1709200000\\
93000.0000000026	-488500000\\
-19000.0000000001	854600000\\
147000.000000002	-488500000.000001\\
145999.999999997	-732100000\\
-147000.000000001	1464600000\\
-146000.000000001	-1708700000\\
-109000.000000001	487800000\\
34999.9999999993	1221200000\\
148000.000000001	-1221000000\\
16999.9999999977	854700000\\
-126999.999999998	-1221000000\\
-56000.0000000027	732700000\\
331000.000000004	1098600000\\
-167000.000000003	-2075400000\\
-51999.9999999978	854700000\\
234999.999999998	1220600000\\
-70999.9999999979	-1953100000\\
-551000	244200000\\
330999.999999999	1586900000\\
255999.999999998	-732600000\\
-146999.999999999	-1098300000\\
73999.9999999989	1708600000\\
-20000.0000000005	-1708700000\\
-253999.999999999	854400000\\
54000.0000000003	366200000\\
273999.999999997	122000000\\
18999.9999999983	-610100000\\
-457999.999999999	-1343000000\\
475999.999999998	3295800000\\
-110000	-2685200000\\
-219000.000000002	976299999.999999\\
219000	-122000000\\
-8.88178419700125e-10	244199999.999999\\
-55000.0000000006	-732600000.000001\\
-327999.999999997	610500000.000001\\
290999.999999996	-488299999.999999\\
312000.000000002	1587000000\\
-92000.0000000032	-1465000000\\
-346999.999999998	-854499999.999999\\
35999.9999999969	1709100000\\
36999.999999999	-854599999.999999\\
72000.0000000018	732600000.000001\\
149000	-366400000\\
-186000.000000001	-1098500000\\
-126000	1586800000\\
255000.000000002	-366200000\\
19999.9999999996	-1.77635683940025e-07\\
-2000.00000000244	-1098400000\\
-364999.999999999	854200000\\
89999.999999999	-122099999.999999\\
351000.000000002	1709400000\\
-113000.000000001	-2930300000\\
93000.0000000008	1831600000\\
-201999.999999999	-1220900000\\
-108999.999999998	1464600000\\
366000	-732000000\\
-165000.000000004	-488500000\\
-164999.999999998	732400000\\
73000.0000000031	-366200000\\
-273999.999999999	-122000000\\
218999.999999998	610300000\\
220999.999999999	122000000\\
52999.9999999982	-732200000\\
-200000	-488400000\\
-72999.9999999986	1098400000\\
329000	366600000\\
-348000.000000001	-1709200000\\
54999.9999999979	976600000\\
147000.000000001	1098600000\\
-220999.999999997	-2319200000\\
-36000.0000000031	1342500000\\
184000.000000002	732700000\\
292999.999999999	-976700000\\
-295000.000000002	-244100000\\
-70999.9999999979	-122100000\\
-37999.9999999985	976700000\\
183000.000000001	-244500000\\
-402000.000000004	-731900000\\
129000.000000002	121600000\\
583999.999999998	2197400000\\
-512000	-4272200000\\
91999.9999999978	4028000000\\
72000.0000000018	-1342700000\\
112000.000000001	-610200000\\
-93000.0000000008	-122200000\\
-165000.000000002	488199999.999999\\
18999.9999999983	122200000.000001\\
36000.0000000005	1.06581410364015e-06\\
1000.00000000033	-122000000\\
72000.0000000018	-300000.000001077\\
-145000	-365900000\\
126999.999999999	1220600000\\
-36999.9999999999	-1709000000\\
276000.000000003	2441400000\\
-330000.000000002	-4150400000\\
-37000.0000000008	4394600000\\
256000.000000001	-1709000000\\
-238000.000000001	-1587100000\\
130000.000000003	2441700000\\
198999.999999998	-122199999.999999\\
-163000	-2441400000\\
-313000.000000002	1830900000\\
20000.0000000031	122300000\\
383999.999999997	5.32907051820075e-07\\
-38000.0000000002	-854800000\\
-344999.999999998	366700000\\
344999.999999999	609800000\\
-35000.0000000001	-609800000\\
-275999.999999998	-366700000\\
166999.999999998	732800000\\
126000	610100000\\
-329000.000000001	-2929600000\\
184000.000000003	4516600000\\
273999.999999999	-3662000000\\
-494000	366000000.000001\\
163999.999999998	2319500000\\
330000.000000002	-1587000000\\
-311000.000000001	-244100000\\
311000	854500000\\
-330000.000000002	-1709100000\\
220000.000000002	2685800000\\
1000.00000000033	-2197500000\\
-404000.000000003	-122000000\\
348000	2441400000\\
1000.00000000211	-2319300000\\
-276000.000000002	122000000\\
421999.999999999	1587100000\\
-128999.999999998	-1220900000\\
-329000.000000002	-366300000\\
-35999.9999999978	732800000\\
309999.999999997	976100000\\
38000.0000000029	-2074700000\\
-1000.00000000122	1342300000\\
-1.77635683940025e-09	-610100000\\
-110000	244200000.000001\\
92000.0000000005	732199999.999999\\
37000.0000000008	-1708800000\\
-184000.000000002	1586800000\\
129000.000000001	-976399999.999999\\
-183000.000000001	-244400000\\
309999.999999999	2319600000\\
-109000.000000002	-2685700000\\
-293000	122200000\\
184000.000000002	1952800000\\
290999.999999996	-854000000\\
-199999.999999997	-610800000\\
-92000.0000000023	-121799999.999999\\
8.88178419700125e-10	854400000\\
164000	-122100000\\
-107999.999999999	-854400000\\
-221000.000000002	976600000\\
221000.000000002	-610600000\\
33999.999999998	1098800000\\
94000.0000000012	-1586800000\\
-238999.999999999	366000000.000001\\
146999.999999998	1464900000\\
126999.999999999	-1708900000\\
-218000	244200000.000001\\
-294000	243800000\\
383999.999999997	1099100000\\
129000.000000004	-1221200000\\
-55000.0000000015	500000.000000078\\
-128000	121699999.999999\\
71999.9999999992	610500000\\
-89999.9999999998	-1342800000\\
72999.9999999977	1587000000\\
-202999.999999999	-1465000000\\
259000.000000001	2075300000\\
51999.9999999978	-2441500000\\
-381999.999999998	610500000\\
200000.000000001	1708900000\\
91999.9999999987	-1831000000\\
-90999.9999999975	732200000\\
126999.999999995	122400000\\
-255999.999999999	-1221000000\\
129000	1953400000\\
127000.000000002	-1098800000\\
164999.999999999	976500000\\
-89999.9999999998	-1953000000\\
-222000.000000001	976600000.000001\\
55999.9999999983	-100000.000000833\\
-274000	122100000\\
273999.999999998	1098600000\\
-183999.999999998	-2563400000\\
350000.000000001	2929600000\\
-75000.0000000028	-2685400000\\
-219999.999999999	1708700000\\
112000	-610000000\\
-20000.0000000013	243800000.000001\\
109999.999999999	400000.000001022\\
-256000.000000001	-610800000.000001\\
383999.999999999	1831400000\\
-53999.9999999994	-2685800000\\
-55999.9999999983	1831400000\\
-237000.000000002	-854900000.000001\\
91000.0000000011	610600000\\
-36000.0000000014	-366299999.999999\\
292000	610400000.000001\\
-257000.000000001	-1220600000\\
2000.00000000511	1342400000\\
256000	-731900000\\
-477000.000000005	-732900000\\
128000.000000001	1953400000\\
439999.999999999	-854500000\\
-310999.999999998	-1465000000\\
-330000.000000003	1098800000\\
641000.000000002	2319300000\\
-220000	-4272600000\\
-291999.999999998	2197400000\\
327999.999999999	1098700000\\
111000.000000002	-1831300000\\
-145999.999999999	488500000\\
-258000.000000005	-244200000\\
-71999.9999999983	244100000\\
127999.999999998	610400000\\
347000	-366200000\\
-164000.000000001	366100000.000001\\
56000.0000000036	-854300000\\
33999.9999999954	-199999.999999179\\
-234999.999999996	244299999.999999\\
124999.999999996	488100000\\
-106999.999999997	-854200000\\
52999.9999999955	1220300000\\
-108999.999999998	-1708600000\\
91999.9999999987	1952900000\\
310000	-1098600000\\
-531000	-732300000\\
167000.000000002	1708900000\\
179999.999999998	-1098800000\\
-198999.999999994	122400000.000001\\
127999.999999998	243799999.999999\\
-39000.0000000024	-121700000\\
333000.000000003	365800000\\
-351000.000000003	-365799999.999999\\
-52999.9999999955	-976800000\\
146000	1953000000\\
54999.999999997	-1098300000\\
-257000.000000001	-1221000000\\
-108999.999999998	2441700000\\
348000.000000002	-366600000.000002\\
254999.999999999	-976099999.999999\\
-255000.000000002	-244600000\\
-129000.000000001	399999.999999778\\
129000.000000001	1220400000\\
-312000.000000002	-976400000\\
-71999.9999999992	-854500000\\
328000.000000002	2807500000\\
-36999.9999999999	-2441200000\\
129999.999999996	1342500000\\
-36999.9999999999	-1220400000\\
-146999.999999999	610100000\\
36000.0000000022	-121900000\\
18999.9999999983	854300000\\
-36000.0000000022	-1464600000\\
-110999.999999999	610200000\\
73999.9999999998	488300000\\
182999.999999998	-122099999.999999\\
8.88178419700125e-10	-121900000.000001\\
-166000.000000001	-1098900000\\
-34999.9999999975	1709100000\\
183000.000000002	-610200000.000001\\
-146999.999999998	-732599999.999999\\
165000	1342800000\\
35999.9999999987	-488200000\\
-109000.000000001	-1098800000\\
-92000.0000000023	732599999.999999\\
-256999.999999998	-244299999.999999\\
386000.000000002	1831300000\\
163999.999999997	-1587100000\\
-146999.999999998	-732500000\\
73999.9999999981	1098800000\\
-238000	-610399999.999999\\
-165000.000000003	732400000\\
385000.000000002	-122000000\\
-495999.999999999	-1098900000\\
312999.999999998	1587400000\\
163999.999999999	-488700000\\
-73000.0000000013	-732299999.999999\\
127999.999999999	732400000\\
-218999.999999998	-243900000\\
-165999.999999998	-488700000\\
239000	976900000\\
-73999.9999999972	-732600000\\
146999.999999999	854700000\\
-73999.9999999989	-1465200000\\
-255000.000000002	732700000\\
-92000.0000000014	854400000\\
475000	-1098500000\\
-17000.0000000012	610200000\\
72000.0000000009	366199999.999998\\
-419999.999999999	-2563500000\\
37000.0000000017	3418300000\\
364999.999999998	-1099200000\\
-127999.999999999	-853900000.000003\\
35999.9999999996	609900000.000001\\
-198999.999999998	-854300000\\
70999.9999999979	976499999.999999\\
183000.000000002	610500000\\
-236000.000000001	-1465000000\\
199999.999999999	488399999.999999\\
-183000.000000001	-199999.999999889\\
-348000.000000001	-244000000\\
713999.999999998	2075300000\\
-365999.999999998	-3662300000\\
-18000.0000000025	3296000000\\
146000.000000003	-1709100000\\
-72000.0000000001	200000.000000422\\
-20000.0000000031	366100000\\
-36000.0000000005	610200000.000001\\
54999.9999999988	-1342400000\\
72999.9999999995	1464500000\\
19000.000000001	-976500000\\
-220000.000000003	-488099999.999999\\
365000.000000002	2319100000\\
-33999.9999999998	-2807300000\\
-588999.999999998	609900000\\
166999.999999998	1953500000\\
548999.999999998	-1342900000\\
-128999.999999996	-854600000\\
-641000	976900000\\
294000.000000001	-122600000\\
218999.999999999	610900000\\
-36000.0000000022	-1221000000\\
91000.0000000002	1220700000\\
-329000.000000001	-1709000000\\
328000.000000001	2929900000\\
112000	-3174100000\\
-495000.000000001	854800000.000001\\
529999.999999999	2074800000\\
-273999.999999998	-2563200000\\
-54999.9999999979	854500000\\
128000	610300000\\
-365000.000000001	-1098800000\\
401000.000000001	1098900000\\
-255000.000000001	-854700000\\
-1000.00000000211	488500000\\
348999.999999998	488100000\\
-36999.9999999981	-854500000\\
-239000	-610200000\\
128999.999999999	1708800000\\
-183000.000000002	-1098500000\\
256000.000000002	366100000\\
-164000.000000001	299999.999999834\\
36000.0000000005	-1221200000\\
73000.0000000031	2808000000\\
-238000.000000001	-3051900000\\
238000.000000004	1953200000\\
-274000.000000003	-976600000\\
329000	976400000\\
36999.9999999999	-976200000\\
-256999.999999998	365900000\\
165999.999999999	-244100000\\
-165999.999999996	610500000\\
-275000	-1342900000\\
477999.999999999	2685600000\\
236999.999999997	-1953100000\\
-385000	-1465000000\\
127999.999999999	3784400000\\
-401999.999999999	-4028500000\\
421000.000000001	3784300000\\
127999.999999999	-2929700000\\
-439000	1098500000\\
273999.999999997	488500000\\
-219999.999999999	-610600000\\
386000	-366000000\\
-294000.000000001	610300000\\
-110999.999999997	732400000\\
148999.999999997	-2197400000\\
181000	3174000000\\
-237000.000000001	-3784100000\\
54999.9999999997	3295600000\\
-128999.999999998	-2685200000\\
293999.999999997	3295600000\\
-293999.999999999	-4272300000\\
55000.0000000015	4028200000\\
368000	-1586700000\\
-76000.0000000014	-732599999.999999\\
-253999.999999997	244199999.999999\\
-239000.000000002	488300000\\
274999.999999996	244000000\\
237000.000000002	122300000\\
-329000.000000001	-2075400000\\
294000.000000001	3296100000\\
52999.9999999964	-2075500000\\
-528999.999999996	-487899999.999999\\
-57000.0000000004	365900000\\
515000.000000001	2441600000\\
-20000.000000004	-2807800000\\
-383999.999999998	244400000.000001\\
365999.999999998	1220400000\\
-17999.9999999971	-365999999.999999\\
-55000.0000000033	-610300000\\
-183999.999999998	-122400000\\
20000.0000000013	977000000\\
273999.999999997	121700000\\
-184000.000000001	-1830800000\\
19000.000000001	2197000000\\
92000.0000000005	-1098300000\\
-184000.000000002	-732600000\\
148000.000000001	2075100000\\
-94000.0000000021	-2563300000\\
-217000	2197200000\\
363999.999999999	-1098700000\\
-53999.9999999994	854500000\\
17999.9999999989	-1220500000\\
37000.0000000017	1220500000\\
-73000.0000000013	-1342800000\\
-202000.000000002	610600000\\
92000.0000000005	487899999.999999\\
53999.9999999976	-610000000\\
-34999.9999999975	854300000.000001\\
107999.999999999	-1220600000\\
93000	1464700000\\
36000.0000000005	-1586800000\\
-585999.999999998	244200000\\
330000.000000002	854300000.000001\\
402999.999999999	1098900000\\
-367000	-3540300000\\
-91000.0000000037	3173900000\\
274000.000000001	-854500000\\
-128000.000000001	-1220500000\\
-108999.999999996	1220500000\\
109999.999999999	488200000\\
-75000.0000000019	-1586600000\\
1999.99999999978	976200000\\
217999.999999999	976700000\\
-255000.000000001	-2685300000\\
126999.999999998	2685200000\\
-53999.9999999994	-1586800000\\
72999.9999999986	1098600000\\
-182999.999999999	-1952900000\\
291999.999999997	2685300000\\
-200000	-2075200000\\
-73999.9999999998	1098800000\\
164999.999999998	-488400000\\
201000	488300000\\
-199999.999999997	-854400000\\
-149000	488100000\\
-199000.000000002	-366200000\\
-92999.9999999991	732800000\\
568999.999999999	121500000\\
52999.999999999	-609900000\\
-89999.9999999998	-300000.0000001\\
-257000.000000002	-610100000\\
128000	2075000000\\
38000.0000000011	-2563400000\\
52999.9999999982	2197300000\\
-311000	-2075200000\\
202999.999999998	2075100000\\
17000.0000000012	-1464800000\\
-55000.0000000006	610400000\\
93000.0000000008	-122100000\\
-39000.0000000024	122000000\\
93999.9999999994	-366000000\\
-275999.999999998	488100000\\
-109999.999999999	-1098600000\\
603999.999999997	2563400000\\
-383000	-3295800000\\
71999.9999999992	2685700000\\
-110000	-2319700000\\
129000.000000003	2685800000\\
-73000.0000000022	-2929600000\\
163000	3295500000\\
-180999.999999997	-4027900000\\
53999.9999999976	3906100000\\
255999.999999999	-1953200000\\
-604000.000000003	-976500000\\
513000.000000001	2563400000\\
-220000.000000001	-2074900000\\
-17999.9999999998	976100000\\
35999.9999999969	-243799999.999999\\
202000.000000001	732400000\\
-257000	-2685800000\\
-17000.0000000003	3662500000\\
290999.999999999	-2075600000\\
-126000.000000001	732600000\\
-75000.0000000002	-1342600000\\
-72000.0000000027	1342600000\\
127000.000000002	244000000\\
37999.9999999976	-1708800000\\
-92999.9999999991	2197400000\\
-54000.0000000011	-1953500000\\
73000.0000000004	1465200000\\
-238000.000000001	-1587200000\\
256000.000000002	2075500000\\
220000.000000001	-1221000000\\
-255999.999999998	200000.000000777\\
36999.999999999	-244300000\\
162999.999999998	1465000000\\
-217999.999999999	-2563500000\\
-38000.0000000011	2075000000\\
93000	-610000000\\
-147000.000000003	-122400000\\
-54999.9999999979	-121900000\\
273999.999999996	732400000\\
-72999.9999999977	-610499999.999999\\
110999.999999999	488600000.000001\\
-1000.00000000033	-366500000\\
-495000.000000001	-1342700000\\
423000.000000001	2441500000\\
-2000.00000000067	-854600000.000001\\
111000.000000004	-366199999.999999\\
-457000.000000002	-1098500000\\
529000.000000001	3295700000\\
37999.9999999976	-2929500000\\
-347999.999999997	488000000.000001\\
-19000.0000000037	-121600000\\
-183000	854000000\\
258000.000000001	199999.999999889\\
16000	-610200000\\
220999.999999997	1220500000\\
-165000	-2197200000\\
128000.000000001	1953000000\\
-367000.000000002	-1952800000\\
39000.0000000024	2685300000\\
143999.999999997	-2319400000\\
166000	1342900000\\
-293000.000000001	-976500000\\
165000	854400000\\
126999.999999998	-610400000\\
-528999.999999997	122100000\\
419000	122300000\\
-72000.0000000018	976100000\\
146000	-2074800000\\
-147000	1708900000\\
20000.0000000013	-976700000.000001\\
-75000.000000001	122200000\\
257999.999999999	1464800000\\
-331000	-3051800000\\
55999.9999999991	3173900000\\
309999.999999998	-1098600000\\
-291999.999999999	-1587000000\\
90999.9999999993	2563400000\\
-36000.0000000022	-2197000000\\
-146999.999999999	1220400000\\
385000.000000001	976800000\\
-183000.000000001	-2441600000\\
-148000	1098800000\\
149000.000000001	488300000.000001\\
88999.9999999977	121800000\\
-382999.999999998	-1464600000\\
164999.999999998	1098600000\\
292000.000000002	610400000.000001\\
-256000	-1098900000\\
-146000.000000001	300000.000000189\\
548000.000000002	1708900000\\
-309999.999999999	-2685600000\\
-36000.0000000014	1953100000\\
-185000.000000001	-1220600000\\
55999.9999999983	366100000\\
438999.999999999	1831200000\\
-348000	-2685600000\\
-145000	610199999.999999\\
400999.999999999	1953200000\\
-72000.0000000009	-2563200000\\
-403999.999999999	854000000\\
274999.999999998	1343200000\\
93000.0000000035	-1831200000\\
-221000.000000002	366000000\\
109000.000000002	1221100000\\
19999.9999999996	-1343100000\\
-147000.000000003	610500000\\
73000.0000000013	-488300000\\
73000.0000000004	976500000\\
-55000.0000000015	-976400000\\
129000.000000003	732200000\\
-109999.999999998	-610200000\\
-19000.0000000019	5.32907051820075e-07\\
147000.000000003	732300000\\
0	-366100000.000001\\
-183000.000000002	-854500000.000001\\
53999.9999999994	1220600000\\
1000.00000000122	-854199999.999999\\
-110000	365800000\\
311000	976800000\\
-238000.000000003	-2075200000\\
73000.0000000004	1586800000\\
1000.00000000033	-854199999.999999\\
-239000.000000002	365800000\\
420999.999999999	854700000\\
-200999.999999998	-1708900000\\
19000.000000001	976400000\\
126999.999999999	488400000\\
-109000.000000003	-1098700000\\
-56000.0000000009	244200000\\
-310000	122000000\\
439000	732400000\\
-184000.000000002	-854400000\\
203000	366100000\\
-258000.000000001	-610100000\\
-71999.9999999965	488000000\\
292999.999999997	610300000\\
164000.000000001	122499999.999999\\
-293000.000000003	-2563900000\\
37000.0000000026	3051800000\\
147000.000000001	-1464500000\\
-293000.000000001	-122500000.000002\\
89999.9999999972	610600000\\
111000.000000001	-244200000\\
-72999.9999999986	-244199999.999999\\
-110000	610600000.000001\\
54999.9999999997	-854900000.000001\\
236999.999999998	1587200000\\
-108999.999999998	-1953100000\\
-108999.999999999	488100000\\
273000	1831100000\\
-384000.000000003	-3295700000\\
220000.000000002	3173500000\\
-90999.9999999984	-1830700000\\
-111000.000000001	-400000.000000844\\
183999.999999998	854900000.000001\\
347000.000000001	1098400000\\
-585000.000000001	-4150400000\\
127999.999999998	4760800000\\
-56000.0000000009	-3295900000\\
184000	1587000000\\
111000.000000004	366000000.000001\\
-314000.000000004	-1830900000\\
-70999.9999999971	976600000\\
72999.9999999986	976400000.000001\\
420000.000000002	-732300000\\
-126999.999999999	-122000000.000001\\
-93000	-244400000\\
-181999.999999999	-610099999.999999\\
74000.0000000016	1708800000\\
255000	-243900000\\
-275000.000000002	-1953400000\\
2000.00000000333	2197500000\\
272999.999999996	-732600000\\
-238000.000000001	-732399999.999999\\
-164999.999999999	244299999.999999\\
405000	1586800000\\
-149000.000000002	-1587100000\\
-292000.000000001	-732000000\\
383999.999999999	2441100000\\
-217999.999999999	-1953100000\\
52999.9999999982	732500000\\
-36000.0000000014	-244199999.999999\\
-54999.9999999979	122200000\\
330000	732200000\\
-184000.000000002	-1586700000\\
-126999.999999995	854300000\\
17999.9999999998	122300000\\
72999.9999999977	243900000\\
110000	-488199999.999999\\
-56000.0000000009	-121899999.999999\\
-291000	-122200000\\
345999.999999999	1464600000\\
56000.0000000009	-1464400000\\
-476000	-610600000.000002\\
475999.999999999	2075200000\\
-19000.0000000019	-1220600000\\
-108999.999999998	-122300000.000001\\
16999.9999999986	122400000.000001\\
-108000.000000001	365900000.000001\\
-294000.000000001	-732200000\\
475999.999999999	732400000\\
37000.0000000017	366000000\\
-110000.000000002	-1220400000\\
-220000.000000001	610100000\\
-8.88178419700125e-10	-121900000\\
202000	610200000\\
-37000.0000000008	-1220600000\\
-18999.9999999992	1464900000\\
147999.999999999	-610600000\\
-147999.999999997	-610000000\\
221000	854299999.999998\\
35999.9999999996	-244300000.000001\\
-623000.000000004	-1220400000\\
386000.000000003	1831000000\\
-56999.9999999986	-732700000\\
-145000.000000004	122400000\\
329000.000000001	-488400000\\
-238000.000000001	488200000.000001\\
404000.000000003	366300000.000001\\
-259000.000000001	-366300000\\
-344999.999999998	-1586600000\\
583999.999999999	3051200000\\
-182000.000000001	-1586400000\\
-92000.0000000005	-1587100000\\
-202000.000000002	2929600000\\
93000.0000000026	-2197200000\\
89999.9999999981	1098600000\\
257000.000000001	1098800000\\
-183000.000000001	-2685800000\\
-164000	1465100000\\
52999.9999999973	-366400000\\
-53000.0000000008	732400000\\
-19999.9999999978	-1220500000\\
367000	2441200000\\
-181999.999999999	-3295800000\\
125999.999999998	2685600000\\
-162999.999999999	-1709100000\\
-239000.000000002	-7.105427357601e-07\\
603999.999999999	1709100000\\
-768000.000000001	-2075300000\\
348000	1342900000\\
52999.9999999999	488100000\\
-254000.000000002	-2563400000\\
273000.000000001	2685600000\\
37999.9999999994	-732399999.999999\\
-130000.000000002	-732600000.000001\\
258000.000000004	1465000000\\
54999.9999999988	-1098600000\\
-368000.000000004	-1464900000\\
20000.0000000013	2929500000\\
17999.9999999998	-2441100000\\
-74000.0000000016	2075100000\\
129000.000000001	-976700000.000001\\
274000	-244000000\\
-145999.999999997	854400000\\
-311000.000000003	-1952900000\\
109000.000000003	1830700000\\
293999.999999997	244300000\\
-111000.000000001	-732200000\\
-128000	-977000000.000001\\
56000	1465300000\\
-36999.9999999999	-488600000\\
-999.999999997669	-244100000\\
-236999.999999999	732600000\\
181999.999999999	-1343000000\\
220999.999999997	1953300000\\
-19000.000000001	-1342900000\\
-145999.999999998	-488099999.999999\\
237999.999999997	1708700000\\
-164999.999999999	-1464500000\\
-348999.999999997	365900000\\
37999.9999999985	-244100000\\
238000	976900000\\
54999.999999997	-488700000\\
54000.0000000011	-243899999.999999\\
-109000.000000003	-488399999.999999\\
-36999.9999999999	1098700000\\
-73000.0000000013	-610400000\\
347000	488299999.999999\\
-199999.999999997	-243999999.999999\\
71999.9999999983	-488499999.999999\\
147000.000000003	854499999.999999\\
-347000.000000001	-1342500000\\
-19999.9999999987	976199999.999998\\
19999.9999999969	200000.000000955\\
218000	244200000\\
20000.0000000013	-366400000\\
-112000.000000003	-610199999.999999\\
-33999.9999999971	976500000\\
-258000.000000003	-488300000.000001\\
348000.000000001	122200000\\
182999.999999998	732200000.000001\\
-236999.999999998	-1220500000\\
-311999.999999999	-366400000\\
219000	1709200000\\
275999.999999998	-976600000\\
-348999.999999999	365900000\\
258000	-854000000\\
-240000.000000003	1464400000\\
1000.00000000122	-2441000000\\
383999.999999998	3905800000\\
-383999.999999998	-4028000000\\
2.66453525910038e-09	1464800000\\
238000	1831000000\\
73000.0000000013	-1953100000\\
8.88178419700125e-10	-366300000\\
-238000.000000002	1220900000\\
129000	-1098800000\\
-368000.000000002	976500000.000001\\
404000	-976200000\\
-348000.000000001	2074700000\\
367000	-3051400000\\
17000.0000000003	2929500000\\
-330000.000000002	-2441200000\\
387000.000000001	2441100000\\
-369000.000000002	-3051400000\\
258000.000000001	2563100000\\
72999.9999999986	399999.999999778\\
-366999.999999996	-2319700000\\
274999.999999997	1831200000\\
-127000	488500000\\
16999.9999999995	-1953600000\\
184000	1709400000\\
-312999.999999999	-610500000\\
93999.9999999994	-610500000\\
144000	1465200000\\
-53000.0000000026	-1221000000\\
-202000.000000001	244300000\\
165000	-200000.000000333\\
218999.999999998	1098800000\\
-128000.000000001	-2075200000\\
-346999.999999998	976500000\\
291999.999999996	854600000\\
201000.000000003	-244400000\\
-273000.000000001	-1830600000\\
364999.999999999	2196700000\\
-293000.000000001	-487700000\\
-475000	-2442000000\\
383000.000000001	4395100000\\
202000	-3662500000\\
-218999.999999999	2075300000\\
201000	-1098500000\\
110000.000000001	1220500000\\
-184000.000000003	-1831000000\\
-183000	1098800000\\
75000.0000000028	-244400000\\
180999.999999997	732600000\\
73999.9999999981	-1098600000\\
-419999.999999996	488100000\\
236000.000000001	122200000\\
92999.9999999991	-8.88178419700125e-08\\
-73999.9999999972	-488400000\\
-256000	854700000\\
127999.999999997	-1343000000\\
274999.999999999	1953300000\\
-55000.0000000015	-1220800000\\
-329999.999999999	-854400000\\
129000	1098500000\\
439000.000000001	1953100000\\
-514000.000000003	-4882500000\\
167000.000000005	3661800000\\
144999.999999998	976600000\\
19000.0000000028	-3662000000\\
-366000.000000003	1464800000\\
72999.9999999995	1342800000\\
401999.999999999	-854700000\\
-401999.999999998	-243800000\\
146000.000000001	-854800000\\
92000.0000000005	2441700000\\
-202000.000000001	-2807800000\\
-127000.000000001	1708800000\\
183000.000000001	244600000\\
218000.000000001	-1098900000\\
-255000.000000002	854300000\\
54999.9999999988	-976100000\\
164000.000000001	1342400000\\
-18000.0000000025	-854400000\\
-181999.999999997	-244100000\\
-92999.9999999982	-244100000\\
74000.0000000007	1953100000\\
90999.9999999984	-2319400000\\
-310999.999999997	1220700000\\
475999.999999995	610400000\\
37000.0000000017	-2075200000\\
-257000.000000003	1709000000\\
-90999.9999999984	-610300000\\
55000.0000000015	244000000\\
146000	-122100000.000001\\
-17999.9999999971	244399999.999999\\
109999.999999999	-199999.999999534\\
-256000.000000003	-1098600000\\
53000.0000000017	976500000\\
56999.9999999986	244300000.000001\\
-91999.9999999996	-610500000\\
0	100000.000000122\\
199999.999999999	1220700000\\
-70999.9999999979	-1342900000\\
-1999.99999999978	-121899999.999999\\
-71000.0000000006	610199999.999999\\
-57000.0000000022	100000.000000122\\
18999.9999999992	-244200000\\
274999.999999999	610500000\\
-312000.000000002	-1098900000\\
-90999.9999999984	732600000.000001\\
219999.999999997	1.06581410364015e-06\\
91000.0000000002	122100000\\
-201999.999999999	-488400000.000001\\
129999.999999998	122100000\\
-1999.99999999978	366200000.000001\\
20000.0000000031	-488099999.999999\\
-294000.000000001	610200000\\
999.999999998557	-1342900000\\
529000	2563700000\\
-144000	-2563600000\\
-350000.000000002	610500000.000001\\
367000.000000001	1098400000\\
-199999.999999997	-488099999.999999\\
-241000	-1587000000\\
459999.999999997	2563500000\\
-275000.000000001	-1220700000\\
8.88178419700125e-10	-976600000\\
182999.999999999	1953300000\\
-293999.999999996	-1220900000\\
312999.999999997	-488400000\\
-19999.9999999969	1953600000\\
-365000	-2075700000\\
330000.000000001	610600000\\
-129999.999999999	1709000000\\
129999.999999996	-3418100000\\
-73999.9999999972	3784300000\\
-91999.9999999987	-3295900000\\
182999.999999998	2197100000\\
-127000.000000002	-610100000\\
-237999.999999998	-1098900000\\
602999.999999997	3174100000\\
-238000	-4883000000\\
-200000	4028400000\\
254999.999999998	-1220800000\\
-16999.9999999959	122100000.000001\\
-496000.000000001	-1342500000\\
476999.999999998	1342300000\\
72999.9999999995	1465200000\\
-438999.999999998	-4150600000\\
255000.000000001	3540300000\\
38999.9999999979	-610700000\\
126000	-1586600000\\
-91000.0000000019	2929500000\\
-163999.999999998	-4150300000\\
35999.9999999996	3539900000\\
-182999.999999999	-1220500000\\
365999.999999998	-366300000\\
-146000.000000001	1220600000\\
-55999.9999999991	-2319200000\\
19999.9999999978	2563500000\\
292000.000000002	-976799999.999999\\
-385000.000000001	-487900000\\
128999.999999999	-366500000\\
164000	2441400000\\
-127999.999999998	-3051600000\\
-327999.999999998	976400000\\
638999.999999995	1831300000\\
-438999.999999999	-1709200000\\
182999.999999999	-1098600000\\
-127999.999999997	2929700000\\
1000.00000000033	-3051700000\\
346000	2929600000\\
-217999.999999999	-1708900000\\
-184000.000000003	-1342800000\\
-128000.000000001	2441300000\\
164000.000000001	199999.999999534\\
202999.999999998	-1587100000\\
-999.999999997669	1098700000\\
35999.9999999969	-488200000\\
-400999.999999999	-488400000\\
-93000.0000000017	854600000\\
385000.000000001	-244200000\\
109999.999999998	1098500000\\
146000.000000002	-1830800000\\
-403000.000000003	243999999.999999\\
-127000	1098700000\\
35999.9999999996	-1587100000\\
310000.000000001	1831300000\\
-33999.9999999998	-488400000\\
70999.9999999971	-854599999.999999\\
-401999.999999997	366500000\\
256999.999999997	-366400000.000001\\
71999.9999999992	1953000000\\
-256000	-2807300000\\
311000.000000001	1952800000\\
-52999.9999999982	-121800000\\
-204000.000000001	-1831300000\\
185000	2563800000\\
126999.999999999	-1221100000\\
-293000.000000002	-1220400000\\
-89999.9999999998	1708900000\\
126000	-244299999.999999\\
111000.000000001	300000.0000009\\
111000	-244300000.000001\\
-350000.000000001	-366400000\\
238999.999999998	400000.000000134\\
91000.0000000002	1586600000\\
-364999.999999999	-2563300000\\
291999.999999997	2441200000\\
-201000	-2319000000\\
182000.000000001	2074700000\\
130000.000000002	122699999.999999\\
-146999.999999999	-2564100000\\
53999.9999999976	1953600000\\
-347000.000000002	-366499999.999999\\
310000.000000001	-366100000\\
1999.99999999889	2197300000\\
-112000	-4150500000\\
-53000.0000000017	2563600000\\
17000.0000000012	1586900000\\
202000.000000001	-3174000000\\
-110000	1709100000\\
239000.000000003	-488100000\\
-277000.000000005	488000000\\
39000.0000000024	-1464800000\\
-129999.999999999	2197500000\\
-54000.0000000029	-1953400000\\
293000	1831200000\\
-73000.0000000004	-1587000000\\
-221000.000000001	122100000\\
222000	1709100000\\
16000	-2075300000\\
999.999999997669	1464800000\\
-73000.0000000013	-1342800000\\
237000.000000001	1587200000\\
-200000	-1099100000\\
-74000.0000000007	-365700000.000001\\
-19000.0000000019	854099999.999999\\
-71999.9999999965	-243999999.999999\\
275000	244200000.000001\\
34999.9999999984	-100000.0000003\\
-256000.000000003	-854400000\\
-162999.999999996	122000000\\
271999.999999997	1708900000\\
148000.000000003	-1342600000\\
-383999.999999998	-976500000\\
273999.999999997	2074800000\\
181999.999999999	-487800000.000001\\
-125999.999999999	-1343100000\\
-112000.000000003	732600000\\
-126999.999999999	244000000.000001\\
-146000.000000003	-244100000\\
475000	732600000\\
-201000.000000001	-1221000000\\
-72999.9999999995	366500000\\
90999.9999999966	1708700000\\
18000.0000000033	-2807400000\\
-35000.0000000019	1708900000\\
109000.000000003	-244100000\\
-184000.000000005	-366300000\\
166000.000000002	366300000\\
-111000.000000002	244100000.000001\\
-199999.999999997	-1586900000\\
163999.999999996	2319400000\\
146000.000000004	-1098800000\\
185000	99999.9999995893\\
-699000.000000002	-366100000.000001\\
534000.000000003	-244400000.000001\\
237000	2563800000\\
-312000	-3540400000\\
-328000.000000002	1465200000\\
346000	1220400000\\
-35000.0000000019	-2319200000\\
-130000	1953300000\\
332000.000000003	-488700000\\
-423000.000000004	-976200000\\
-72999.9999999995	610300000.000001\\
477000.000000003	976299999.999999\\
72999.9999999995	-243700000\\
-386000.000000002	-2441900000\\
2000.00000000333	2685900000\\
181999.999999998	-854499999.999999\\
18999.9999999992	609999999.999999\\
-111000.000000001	-1708500000\\
91999.9999999987	1708700000\\
-291999.999999999	-1098600000\\
201000	1342700000\\
-74999.9999999993	-2319100000\\
131000	2685300000\\
-93000.0000000008	-1220500000\\
292000.000000002	121800000.000001\\
-108000.000000001	122499999.999998\\
-166000	-1465400000\\
-90999.9999999966	1831600000\\
219999.999999999	-732800000.000001\\
53999.9999999958	610500000\\
-292999.999999998	-976600000\\
37999.9999999994	244300000\\
254999.999999998	732100000\\
-91000.0000000011	-243900000\\
-293000	-1098700000\\
146999.999999999	854600000\\
327999.999999998	1464600000\\
-180999.999999996	-2441200000\\
-130000.000000003	610300000\\
-292000.000000002	244100000\\
421000.000000001	854600000\\
421000.000000001	-610599999.999999\\
-714000.000000003	-243800000\\
220000.000000001	-1098800000\\
35999.9999999987	3417800000\\
-73000.0000000013	-4150000000\\
128000.000000002	2685100000\\
37000.0000000008	-365900000\\
-108999.999999999	-976600000\\
16000	976500000\\
-181000.000000001	-732500000\\
-74999.9999999984	488500000\\
332000.000000001	121800000\\
161999.999999996	-121800000\\
-199000	-244400000\\
-19999.9999999978	-121799999.999999\\
-218999.999999998	243900000\\
-19000.0000000001	-244000000\\
220999.999999997	854400000\\
91000.0000000019	-610200000\\
72999.9999999977	-366400000.000001\\
-220000.000000002	732500000\\
-72999.9999999977	-976400000\\
19000.0000000001	976200000\\
-8.88178419700125e-10	-854100000\\
163000	1342400000\\
129999.999999999	-610000000\\
-36999.9999999999	-610499999.999999\\
-330000	-732600000.000001\\
127999.999999997	2197600000\\
37000.0000000017	-1220900000\\
129000.000000001	-1.77635683940025e-07\\
-258000.000000004	-122100000\\
56000.0000000009	610500000\\
127999.999999998	-366299999.999999\\
72999.9999999995	122000000\\
-36000.0000000005	-732300000\\
-349000.000000001	854400000\\
130000	-244000000\\
53000.0000000008	121800000\\
182999.999999999	-121700000\\
-107999.999999995	365899999.999999\\
145999.999999995	-1220600000\\
-202999.999999998	1831000000\\
-89999.9999999998	-1952900000\\
54000.0000000011	1220500000\\
330999.999999999	1220600000\\
-1999.99999999889	-3051400000\\
-401000.000000002	1464400000\\
52999.999999999	732799999.999999\\
21000.0000000043	-732499999.999999\\
436999.999999997	365900000\\
-145999.999999999	366700000.000001\\
-310000	-2075600000\\
-75000.0000000028	1587100000\\
258000.000000003	732500000\\
-1000.00000000122	-1465100000\\
-256000	854700000\\
-17999.9999999998	-488400000\\
182000	732600000\\
128999.999999998	-976700000\\
17999.999999998	1342700000\\
-163999.999999998	-1953000000\\
-73999.9999999998	1709000000\\
73000.0000000013	-976600000\\
109999.999999998	976500000\\
-53999.9999999994	-1098600000\\
108999.999999998	732400000\\
-91999.9999999978	-488100000\\
-90000.0000000016	366000000\\
-147999.999999996	-244000000\\
201999.999999998	121800000\\
164999.999999998	854900000\\
-54999.9999999979	-1953400000\\
-384999.999999998	1098600000\\
475999.999999997	732700000\\
-292000.000000001	-1587200000\\
0	1587000000\\
-56000	-610199999.999999\\
292999.999999998	-976900000\\
74000.0000000034	1831500000\\
-476000.000000001	-1953600000\\
329000.000000001	1221200000\\
36999.999999999	1220200000\\
-72999.9999999986	-3417600000\\
-276000.000000003	2319200000\\
203000.000000003	0\\
182999.999999998	5.32907051820075e-07\\
-184999.999999999	-732400000\\
167000.000000002	732399999.999999\\
-110000.000000002	-1220700000\\
-57000.0000000013	1464900000\\
-107999.999999997	-854600000\\
292999.999999997	610500000\\
-257000.000000001	-854700000\\
146000.000000001	610600000\\
1000.00000000033	487999999.999999\\
-384999.999999999	-2197000000\\
696000.000000001	3662000000\\
0	-2319400000\\
-421000	-976500000\\
-147000.000000003	1342800000\\
73000.0000000004	244100000.000001\\
129000	-732400000\\
18000.0000000025	1098600000\\
37000.0000000017	-1220700000\\
-1000.00000000122	488400000\\
148000.000000001	854299999.999998\\
-238999.999999999	-1953100000\\
-56000.0000000036	1098900000\\
94000.0000000012	854100000\\
254000	-1586600000\\
-237000.000000002	1586800000\\
-328999.999999998	-2685700000\\
421000	3418300000\\
-185000.000000002	-1587200000\\
76000.000000004	-854400000\\
180999.999999999	1831000000\\
-72000.0000000018	-1098400000\\
-73999.9999999981	-732700000.000002\\
-219000.000000001	976700000.000001\\
108999.999999998	610300000\\
38000.0000000002	-1464800000\\
162999.999999999	1098600000\\
-108999.999999998	-122100000\\
145999.999999998	-121999999.999999\\
-107999.999999997	-610300000.000001\\
106999.999999999	1098500000\\
-89000.0000000004	-1464900000\\
-514000.000000003	488500000\\
476000.000000003	976500000.000001\\
330000.000000001	976399999.999998\\
-659000.000000002	-4028200000\\
238000	2563500000\\
127000.000000002	1709000000\\
239999.999999998	-2563500000\\
-129999.999999999	1220500000\\
-200000	-1830700000\\
-221000.000000004	1220500000\\
221000.000000002	1098700000\\
71999.9999999974	-732600000\\
92000.0000000014	-854300000\\
-254999.999999999	366200000\\
198999.999999997	732400000\\
-199000	-854500000\\
181000.000000001	976500000\\
-127000.000000003	-976599999.999999\\
-91999.999999997	299999.999999923\\
366999.999999999	854100000\\
-312000.000000002	-732100000\\
54999.9999999997	366000000\\
201000	122100000\\
111000	-1.77635683940025e-07\\
-386000	-1464600000\\
-35000.0000000001	2441000000\\
347000	-1830700000\\
-220000.000000002	732200000\\
111000.000000002	-244100000\\
-295000.000000003	122099999.999999\\
277000.000000001	122100000\\
144999.999999998	366200000\\
-146000.000000001	-1098600000\\
-165000	488100000\\
-181999.999999999	244300000\\
273000.000000002	-122000000\\
109999.999999999	366000000\\
165999.999999999	-488200000\\
-422000.000000001	-610100000\\
-17999.9999999998	2074700000\\
183000	-2441000000\\
256000	1953000000\\
-238000.000000001	-1586900000\\
-181999.999999997	610300000\\
163999.999999996	366200000\\
73000.0000000004	366200000\\
-146000	-1220500000\\
145999.999999998	610200000\\
36999.9999999999	610200000\\
-91999.9999999987	-1464600000\\
-201000	1098500000\\
273999.999999999	122200000\\
-162999.999999997	-732600000\\
-20000.0000000031	99999.9999995893\\
-54999.9999999997	976600000\\
459000.000000001	-100000.000000655\\
-258000.000000001	-1708900000\\
-346000	1098600000\\
565999.999999998	976500000\\
-145999.999999997	-1342600000\\
-146000.000000002	-244300000.000001\\
-17999.9999999989	1098600000\\
-74000.0000000016	-854300000\\
128000.000000004	732300000\\
-19000.0000000001	-610400000\\
-291000.000000003	-366100000\\
732000.000000002	2563300000\\
-477000.000000001	-3539700000\\
-256999.999999998	1708600000\\
276999.999999997	299999.999999834\\
200000	366000000\\
-91999.9999999961	-854499999.999999\\
-54000.0000000003	366500000\\
-239000	-244500000\\
-72000.0000000001	122400000\\
364999.999999998	976200000\\
146999.999999999	-976299999.999999\\
-163999.999999998	-366300000\\
-185000.000000003	-122000000\\
112000.000000002	1953100000\\
-19000.0000000019	-2197400000\\
-55999.9999999991	1098900000\\
55999.9999999983	-366600000\\
109000	488700000.000001\\
999.999999999446	-488500000.000001\\
18000.0000000007	244099999.999999\\
-147000.000000001	-610200000\\
-55000.0000000006	732200000.000001\\
-52999.9999999973	-732100000\\
381999.999999998	1708700000\\
-254999.999999999	-2319200000\\
-275999.999999998	1220700000\\
331999.999999999	-100000.0000003\\
89999.999999999	610400000.000001\\
-165000.000000003	-1708800000\\
-90999.9999999984	976200000\\
511999.999999999	1343100000\\
-713000.000000002	-2807800000\\
310000.000000001	1587000000\\
221000.000000003	854400000\\
-293000	-2197100000\\
126999.999999996	2441200000\\
-365999.999999999	-2929500000\\
569000.000000002	3662000000\\
15999.9999999991	-2197200000\\
-454999.999999997	-1342900000\\
107999.999999997	2685600000\\
201000	-854299999.999999\\
-125999.999999998	-976800000\\
33999.9999999971	1464800000\\
-125999.999999998	-1586600000\\
274000.000000001	2197000000\\
-999.999999999446	-1953100000\\
-90000.0000000016	100000.000000477\\
-239000	366199999.999998\\
109999.999999997	1464700000\\
164000.000000001	-2075000000\\
-363999.999999998	488100000.000001\\
143999.999999997	1220900000\\
442000.000000001	-366500000\\
-296000	-1464400000\\
-179999.999999999	731900000\\
290999.999999998	1221200000\\
-311000.000000001	-1831500000\\
458000.000000001	1587200000\\
-494999.999999997	-1709000000\\
165999.999999997	2441300000\\
-111000	-2807600000\\
276000	2197300000\\
-73999.9999999998	-1831000000\\
17000.0000000039	1953000000\\
-107000.000000003	-1586900000\\
-222999.999999998	854600000\\
441999.999999998	-122100000\\
-219999.999999999	-244200000\\
-240000.000000002	244200000\\
277000.000000001	-854600000\\
144999.999999996	1709100000\\
37000.0000000017	-854499999.999999\\
-292000.000000002	-1586900000\\
-130000	2807500000\\
696999.999999998	-854500000.000001\\
-347999.999999999	-1342500000\\
-256000.000000001	121700000\\
163000.000000005	1709300000\\
-217000.000000004	-1709300000\\
16000.0000000027	1465300000\\
110999.999999995	-977100000\\
458000.000000001	854900000\\
-624000	-2075400000\\
111999.999999999	1953200000\\
345999.999999998	610300000\\
-72999.9999999977	-1342800000\\
999.999999999446	-488000000\\
-367000.000000002	243699999.999999\\
38000.000000002	1587200000\\
180999.999999996	-1464800000\\
349000	1220400000\\
-220000.000000001	-2197000000\\
-457000	1464800000\\
364999.999999997	366200000\\
239000.000000002	-366200000\\
-311000	-610500000\\
-367999.999999999	244300000\\
75000.000000001	610400000\\
639999.999999999	732200000\\
-53999.9999999985	-1953000000\\
-129000.000000002	488500000.000001\\
-311999.999999996	243700000\\
550999.999999997	366600000\\
-732999.999999996	365900000\\
309999.999999999	-1220400000\\
-52999.9999999999	365900000\\
-999.999999999446	854800000\\
72999.9999999968	-1465000000\\
166000.000000002	2319200000\\
-38000.000000002	-2685200000\\
-91000.0000000011	1586500000\\
36999.9999999981	-121700000\\
-999.999999998557	-732700000\\
-36000.0000000022	976800000\\
-53999.9999999994	-1098900000\\
-185000	732700000.000001\\
440999.999999999	854399999.999999\\
164000.000000001	-1098800000\\
-293000.000000003	-1098500000\\
-658999.999999999	1220800000\\
623000	366100000\\
165000	854400000\\
-20000.0000000031	-1953000000\\
-272999.999999998	-732300000\\
-276000.000000002	3417600000\\
331000.000000006	-3539600000\\
255999.999999998	2807300000\\
-237999.999999999	-1098500000\\
72000.0000000001	-732500000.000001\\
111000.000000002	1098700000\\
-274000.000000001	-1098600000\\
418999.999999998	1464700000\\
-418999.999999998	-2441300000\\
-184000.000000004	2929800000\\
256000	-2075400000\\
110000.000000001	1342800000\\
-71999.9999999992	-976500000\\
180999.999999996	1098700000\\
-364999.999999998	-2197400000\\
220000.000000001	2929700000\\
182999.999999999	-2075100000\\
-202000	610400000.000001\\
-73000.0000000022	-366500000\\
-200999.999999998	854800000\\
366000	-610500000\\
72000.0000000027	244200000\\
-511000.000000001	-610500000\\
92000.0000000014	732700000\\
619999.999999999	610100000\\
-326999.999999999	-1953000000\\
-258000	1464800000\\
18999.9999999975	-1098600000\\
513000.000000001	2929600000\\
-73999.9999999998	-4150200000\\
-274999.999999998	2441100000\\
-456000.000000001	-1220400000\\
255000	1464800000\\
365999.999999998	-610600000\\
-145999.999999996	-610200000\\
19000.0000000001	1465100000\\
182000.000000002	-1587400000\\
18999.9999999992	1465200000\\
-312000	-2563600000\\
-126999.999999998	2441500000\\
401000	-122300000\\
-181000	-1220400000\\
-56000.0000000009	488000000\\
18000.0000000016	244400000.000001\\
54999.9999999997	243900000\\
-145999.999999999	-1342600000\\
403000.000000002	3295800000\\
-74000.0000000007	-4028300000\\
-109000	1587000000\\
-257000.000000001	-200000.000000955\\
-36000.0000000022	366500000\\
164000	-122300000\\
54999.9999999988	99999.9999999446\\
-236999.999999998	-1098700000\\
419999.999999998	3051800000\\
-183000.000000001	-4150400000\\
-145999.999999999	3295900000\\
403000	-1220700000\\
-385000.000000001	-1220700000\\
-73000.0000000022	2563500000\\
183000.000000001	-1831100000\\
311000	1098600000\\
-256000	-1708800000\\
-330000	1342500000\\
330000	-121800000\\
-512000.000000002	-244400000\\
566000.000000002	244400000.000001\\
203000.000000002	732099999.999998\\
-367000.000000002	-1708600000\\
18000.0000000016	488100000\\
-54999.9999999997	732200000.000001\\
257999.999999999	1099000000\\
108000.000000001	-2807700000\\
-236999.999999999	1098300000\\
53999.9999999958	366699999.999999\\
-329000	-854700000\\
91999.9999999996	2319100000\\
71999.9999999983	-2685200000\\
258000	1831000000\\
-185000	-1465100000\\
-108000	1343100000\\
53999.9999999985	-1221000000\\
73000.0000000013	1221000000\\
-55000.0000000015	-976700000\\
166000.000000002	976400000\\
162999.999999998	-1342600000\\
-401000.000000002	1098800000\\
72000.0000000001	-732800000\\
-108999.999999999	854700000\\
89999.9999999998	-1220600000\\
-144000	1220400000\\
346000.000000001	-365800000\\
-164999.999999998	-122500000\\
-127000.000000005	-610100000\\
421000.000000002	2075200000\\
-532000.000000002	-3418100000\\
129000	2807700000\\
494000.000000001	488399999.999999\\
-310999.999999997	-3174100000\\
-219000.000000001	1953300000\\
162999.999999999	854499999.999999\\
240000.000000003	-976599999.999999\\
-258000.000000003	-1220700000\\
92000.0000000005	2807500000\\
-181999.999999999	-3417700000\\
-313000.000000002	2685300000\\
331000.000000002	-976400000.000001\\
274000.000000002	610200000.000002\\
73999.9999999989	366399999.999999\\
-185000.000000002	-2075500000\\
57000.0000000039	1099100000\\
-313000.000000004	-122599999.999999\\
2000.00000000422	855000000\\
547999.999999998	-500000.000000256\\
-385000	-1830600000\\
-72000.0000000018	1952800000\\
-219999.999999999	-1952900000\\
437999.999999999	3051500000\\
222000.000000001	-2563200000\\
-514000	243999999.999999\\
200999.999999998	976600000.000001\\
-309999.999999999	-1220800000\\
164000.000000001	1587100000\\
274999.999999999	-976800000.000001\\
-999.999999997669	732600000\\
-126000.000000001	-1953100000\\
-167000.000000002	1708900000\\
201999.999999999	610300000\\
36999.9999999981	-2197200000\\
-146999.999999999	1953100000\\
1000.00000000033	-1220600000\\
72999.9999999995	854400000\\
-165999.999999999	-732500000\\
-72000.0000000018	122300000\\
145999.999999998	1586600000\\
238000.000000001	-2685200000\\
36999.999999999	2441100000\\
-312000.000000002	-1953000000\\
111000.000000003	1220900000\\
-148000	-1221100000\\
37999.9999999976	2075500000\\
146000	-1831200000\\
18000.0000000007	1098800000\\
109999.999999999	-854800000.000001\\
-603999.999999999	-243800000\\
401999.999999999	976400000\\
331000.000000001	1098500000\\
35999.9999999996	-2685200000\\
-658999.999999999	-244500000\\
71999.9999999992	3906400000\\
313000	-3539900000\\
53999.9999999976	1464700000\\
-54999.9999999988	-488400000\\
-311000.000000002	-1098400000\\
92000.0000000005	3051600000\\
529999.999999999	-2075200000\\
-146000	-487999999.999999\\
-512000.000000001	-400000.000000134\\
-184999.999999996	2197400000\\
624999.999999998	-1586700000\\
199999.999999998	243799999.999999\\
-403000.000000002	-1342400000\\
-328999.999999998	1586500000\\
365999.999999998	244500000\\
220000.000000001	-366500000.000001\\
-349000.000000001	-1586700000\\
93000.0000000035	2441400000\\
347000	-122200000\\
-292000.000000001	-3051700000\\
-294000.000000003	2563500000\\
512000.000000001	854499999.999999\\
-381999.999999999	-3540200000\\
15999.9999999973	4028600000\\
367000.000000002	-2075400000\\
19000.000000001	244200000.000001\\
-715000.000000003	-1464800000\\
238000	1953100000\\
421000	854299999.999999\\
38000.0000000029	-1220300000\\
-221000.000000004	-1587300000\\
-219999.999999997	1709200000\\
385999.999999999	976500000\\
-74999.9999999993	-2197300000\\
-182000.000000001	854499999.999999\\
-109999.999999999	200000.000000244\\
202000.000000002	732100000\\
34999.9999999966	-854300000\\
-108999.999999997	122000000\\
184000	854600000\\
-184000.000000001	-2075300000\\
-111000	2075100000\\
442000	300000.000000367\\
-258000.000000002	-1831300000\\
-73000.0000000013	1464800000\\
109000.000000001	-488000000\\
-199000.000000002	488100000\\
-75999.9999999987	-1220800000\\
387000.000000001	1831200000\\
-259000.000000002	-1464900000\\
39000.000000005	366300000\\
109000	366100000.000001\\
110000	610399999.999999\\
-92000.0000000049	-2319400000\\
146000.000000002	2563600000\\
-547999.999999999	-2441500000\\
36000.0000000005	2563600000\\
640999.999999998	-854800000.000002\\
-91999.9999999987	-1220300000\\
-713000.000000001	-244400000\\
346999.999999998	2807600000\\
292000.000000002	-1952900000\\
-271999.999999999	-488500000.000001\\
15999.9999999973	1098800000\\
367000.000000002	1220500000\\
-220000	-3906100000\\
-256000	3051700000\\
384000	99999.9999997669\\
-163999.999999998	-1587100000\\
-238000	610599999.999999\\
108999.999999999	732200000.000001\\
147000.000000003	-244100000\\
18999.9999999983	-366200000\\
89999.9999999972	244300000\\
-201000	-610500000.000001\\
1000.00000000033	854400000\\
-129000.000000002	-1220400000\\
219999.999999999	2441000000\\
17999.9999999998	-2807300000\\
-364999.999999999	854400000\\
382000.000000001	1708900000\\
131000	-1098500000\\
-185000	-1709000000\\
-274000	1464700000\\
512999.999999998	1587100000\\
-165000.000000001	-2929900000\\
-330000	1465200000\\
55000.0000000006	-500000.000000256\\
292999.999999997	610800000\\
-35999.9999999978	-1465000000\\
90999.9999999993	610200000\\
-493999.999999997	488500000\\
127000.000000001	-976800000\\
74999.9999999966	1099000000\\
273000.000000001	-122400000\\
-199999.999999997	-1098500000\\
72999.9999999995	1220700000\\
328999.999999998	121999999.999998\\
-367000	-1586900000\\
-510999.999999999	244300000\\
494000.000000002	1952900000\\
366000	-488100000\\
-348999.999999999	-2075300000\\
-199000.000000001	732400000\\
364000.000000001	3051800000\\
-55000.0000000006	-4394400000\\
-70999.9999999979	2563200000\\
-149000.000000001	-976299999.999999\\
-34000.0000000007	366099999.999999\\
362999.999999999	976499999.999999\\
-197999.999999999	-2075200000\\
-351000	610600000\\
368000.000000001	1952800000\\
74000.0000000007	-2074900000\\
326999.999999999	1464500000\\
-419000	-1708700000\\
-678000.000000002	-122100000.000001\\
915000.000000001	1830900000\\
90999.9999999993	488400000\\
-491999.999999997	-3540100000\\
-112000.000000002	2807700000\\
73999.9999999998	244100000\\
127999.999999998	-1831100000\\
385000.000000002	2441500000\\
-203000.000000001	-2685700000\\
-125999.999999999	1220900000\\
89999.9999999981	488200000\\
-327999.999999999	-1220900000\\
236000.000000002	1709400000\\
-54000.0000000029	-1831500000\\
-146000	976900000\\
403000.000000002	732200000\\
71999.9999999965	-1220500000\\
-383999.999999998	-732600000\\
-73000.0000000004	2197300000\\
147000	-1220600000\\
273000.000000003	244100000\\
-493000.000000002	-610500000\\
420999.999999998	732600000\\
-72999.9999999986	-244200000\\
-459000	-488300000\\
330000	1220800000\\
19999.9999999996	-1098800000\\
-20000.0000000013	366300000\\
292999.999999999	976599999.999999\\
109999.999999999	-1342800000\\
-402000	-976599999.999999\\
-185000	2319300000\\
386999.999999999	-1098400000\\
-221000	121800000\\
-311000	-488200000\\
457000.000000002	1220700000\\
36999.9999999972	-610300000\\
-182999.999999999	-976499999.999999\\
1000.00000000122	1586700000\\
89999.9999999954	-1098500000\\
-17999.9999999971	610400000\\
-90999.9999999993	-488300000\\
147000.000000001	366000000\\
-295000.000000002	-243799999.999999\\
698000.000000001	976200000\\
-312000	-1708600000\\
-404000.000000003	610099999.999999\\
-107999.999999999	244200000\\
547999.999999998	1220700000\\
-110000	-2807600000\\
75000.000000001	3051700000\\
-404999.999999997	-3051800000\\
293999.999999998	2807900000\\
91999.9999999996	-1343000000\\
-366999.999999997	-366300000\\
0	488400000\\
274999.999999999	366400000\\
-128000.000000001	-732800000\\
37000.0000000008	732600000\\
327999.999999998	244300000\\
110999.999999997	-1220900000\\
-951999.999999999	-488400000\\
566999.999999999	2075600000\\
311999.999999998	-122400000.000001\\
-550000	-1831000000\\
-73000.0000000013	-366100000\\
513000.000000002	4394400000\\
-385000.000000003	-5859300000\\
183000.000000002	4394400000\\
385000	-1098300000\\
-457999.999999999	-1343100000\\
0	366299999.999999\\
-71999.9999999974	732500000.000001\\
-21000.0000000043	-122199999.999999\\
460000.000000002	854500000\\
-238999.999999999	-2197000000\\
};
\addplot [color=mycolor2, line width=2.0pt, forget plot]
  table[row sep=crcr]{%
-201000	-201000\\
310999.999999996	310999.999999996\\
-201000	-201000\\
219000	219000\\
54999.9999999988	54999.9999999988\\
-419999.999999995	-419999.999999995\\
145999.999999996	145999.999999996\\
53999.9999999994	53999.9999999994\\
111000.000000004	111000.000000004\\
200999.999999996	200999.999999996\\
-329000	-329000\\
-129999.999999999	-129999.999999999\\
-345999.999999999	-345999.999999999\\
456999.999999999	456999.999999999\\
365999.999999997	365999.999999997\\
-237999.999999998	-237999.999999998\\
55999.9999999991	55999.9999999991\\
17000.0000000003	17000.0000000003\\
-604000.000000003	-604000.000000003\\
678000	678000\\
-37000.0000000008	-37000.0000000008\\
-567000.000000001	-567000.000000001\\
585000	585000\\
-128000.000000001	-128000.000000001\\
-311000.000000001	-311000.000000001\\
367000	367000\\
16000.0000000009	16000.0000000009\\
-381999.999999998	-381999.999999998\\
328999.999999996	328999.999999996\\
17000.0000000003	17000.0000000003\\
-91000.0000000011	-91000.0000000011\\
1000.00000000389	1000.00000000389\\
-129000.000000002	-129000.000000002\\
36999.9999999999	36999.9999999999\\
145999.999999997	145999.999999997\\
-91000.0000000011	-91000.0000000011\\
17000.0000000021	17000.0000000021\\
-33999.999999998	-33999.999999998\\
-58000.0000000043	-58000.0000000043\\
222000.000000002	222000.000000002\\
53999.9999999985	53999.9999999985\\
-275000.000000002	-275000.000000002\\
257000.000000002	257000.000000002\\
-330000.000000004	-330000.000000004\\
19000.0000000028	19000.0000000028\\
254999.999999997	254999.999999997\\
-107999.999999998	-107999.999999998\\
-148000.000000001	-148000.000000001\\
165999.999999999	165999.999999999\\
72000.0000000001	72000.0000000001\\
-293000	-293000\\
423000.000000002	423000.000000002\\
15999.9999999982	15999.9999999982\\
-328000.000000001	-328000.000000001\\
-147999.999999998	-147999.999999998\\
258000.000000001	258000.000000001\\
200999.999999999	200999.999999999\\
-293000.000000001	-293000.000000001\\
-129000	-129000\\
366999.999999998	366999.999999998\\
-164999.999999999	-164999.999999999\\
-109999.999999999	-109999.999999999\\
109999.999999999	109999.999999999\\
-128000	-128000\\
218999.999999999	218999.999999999\\
-71999.9999999974	-71999.9999999974\\
16999.9999999977	16999.9999999977\\
-52999.9999999955	-52999.9999999955\\
124999.999999996	124999.999999996\\
-124999.999999996	-124999.999999996\\
16999.9999999959	16999.9999999959\\
-19000.000000001	-19000.000000001\\
19000.000000001	19000.000000001\\
-200999.999999999	-200999.999999999\\
72999.9999999995	72999.9999999995\\
329000.000000001	329000.000000001\\
-108999.999999998	-108999.999999998\\
108999.999999998	108999.999999998\\
-493999.999999999	-493999.999999999\\
238000	238000\\
165000.000000001	165000.000000001\\
37000.0000000026	37000.0000000026\\
-331000.000000004	-331000.000000004\\
57000.0000000039	57000.0000000039\\
418999.999999997	418999.999999997\\
-218999.999999998	-218999.999999998\\
-35999.9999999996	-35999.9999999996\\
-17999.9999999998	-17999.9999999998\\
-111000.000000006	-111000.000000006\\
-17999.9999999989	-17999.9999999989\\
201000	201000\\
56000	56000\\
-256999.999999999	-256999.999999999\\
-54000.0000000003	-54000.0000000003\\
52999.9999999973	52999.9999999973\\
222000.000000003	222000.000000003\\
-75000.0000000028	-75000.0000000028\\
-128000	-128000\\
111000.000000002	111000.000000002\\
-111000.000000002	-111000.000000002\\
257000.000000001	257000.000000001\\
-201999.999999999	-201999.999999999\\
129000.000000001	129000.000000001\\
-36999.999999999	-36999.999999999\\
-201000.000000003	-201000.000000003\\
273000.000000001	273000.000000001\\
-126000.000000001	-126000.000000001\\
-130000	-130000\\
148000	148000\\
126999.999999998	126999.999999998\\
-256000	-256000\\
202000.000000001	202000.000000001\\
-184000.000000002	-184000.000000002\\
56000.0000000009	56000.0000000009\\
163999.999999997	163999.999999997\\
-238000	-238000\\
-8.88178419700125e-10	-8.88178419700125e-10\\
54999.9999999988	54999.9999999988\\
111000.000000002	111000.000000002\\
-21000.0000000017	-21000.0000000017\\
222000.000000001	222000.000000001\\
-494999.999999999	-494999.999999999\\
37000.0000000026	37000.0000000026\\
493999.999999996	493999.999999996\\
-367000	-367000\\
74000.0000000016	74000.0000000016\\
257000.000000001	257000.000000001\\
-275000	-275000\\
-239000.000000002	-239000.000000002\\
256999.999999998	256999.999999998\\
-182999.999999998	-182999.999999998\\
218999.999999999	218999.999999999\\
130000.000000002	130000.000000002\\
-20000.0000000005	-20000.0000000005\\
-220000.000000001	-220000.000000001\\
38000.0000000038	38000.0000000038\\
91000.0000000002	91000.0000000002\\
-110000.000000002	-110000.000000002\\
201999.999999999	201999.999999999\\
-184999.999999999	-184999.999999999\\
3000.00000000189	3000.00000000189\\
33999.9999999945	33999.9999999945\\
-89999.9999999972	-89999.9999999972\\
-38000.000000002	-38000.000000002\\
38000.0000000029	38000.0000000029\\
346999.999999999	346999.999999999\\
-89999.9999999981	-89999.9999999981\\
-241000.000000003	-241000.000000003\\
3000.000000001	3000.000000001\\
235999.999999999	235999.999999999\\
-144999.999999998	-144999.999999998\\
-202000.000000003	-202000.000000003\\
-55999.9999999974	-55999.9999999974\\
459999.999999997	459999.999999997\\
-221999.999999999	-221999.999999999\\
-72000.0000000036	-72000.0000000036\\
165000.000000003	165000.000000003\\
-38000.000000002	-38000.000000002\\
20000.0000000005	20000.0000000005\\
-111000.000000001	-111000.000000001\\
-163999.999999999	-163999.999999999\\
-92000.0000000005	-92000.0000000005\\
346999.999999999	346999.999999999\\
94000.0000000021	94000.0000000021\\
161999.999999997	161999.999999997\\
-290999.999999999	-290999.999999999\\
-1000.00000000211	-1000.00000000211\\
19000.0000000019	19000.0000000019\\
-440000	-440000\\
493999.999999999	493999.999999999\\
-127000.000000002	-127000.000000002\\
-19000.0000000001	-19000.0000000001\\
-110000	-110000\\
275000.000000001	275000.000000001\\
165000.000000002	165000.000000002\\
-294000	-294000\\
165999.999999998	165999.999999998\\
-276000	-276000\\
57000.0000000022	57000.0000000022\\
88999.9999999977	88999.9999999977\\
76000.0000000014	76000.0000000014\\
-76000.0000000014	-76000.0000000014\\
-89999.999999999	-89999.999999999\\
74000.0000000007	74000.0000000007\\
-39000.0000000015	-39000.0000000015\\
186000.000000001	186000.000000001\\
-424000	-424000\\
239999.999999999	239999.999999999\\
329000	329000\\
-183000	-183000\\
-239000.000000001	-239000.000000001\\
-181999.999999999	-181999.999999999\\
348000.000000002	348000.000000002\\
-18000.0000000007	-18000.0000000007\\
71000.0000000006	71000.0000000006\\
-33999.999999998	-33999.999999998\\
53999.9999999985	53999.9999999985\\
-110999.999999999	-110999.999999999\\
-144999.999999999	-144999.999999999\\
109999.999999999	109999.999999999\\
310000	310000\\
-201000	-201000\\
-347000.000000001	-347000.000000001\\
291999.999999998	291999.999999998\\
219999.999999999	219999.999999999\\
-494999.999999998	-494999.999999998\\
203000.000000002	203000.000000002\\
199999.999999998	199999.999999998\\
-89999.9999999972	-89999.9999999972\\
-240000.000000004	-240000.000000004\\
274999.999999999	274999.999999999\\
-199999.999999997	-199999.999999997\\
16999.9999999977	16999.9999999977\\
239000.000000001	239000.000000001\\
-999.999999998557	-999.999999998557\\
-237000.000000002	-237000.000000002\\
-93000	-93000\\
38000.0000000002	38000.0000000002\\
476000.000000002	476000.000000002\\
-129000.000000001	-129000.000000001\\
-255999.999999999	-255999.999999999\\
-146000.000000004	-146000.000000004\\
237000.000000002	237000.000000002\\
92999.9999999973	92999.9999999973\\
-257999.999999998	-257999.999999998\\
37999.9999999976	37999.9999999976\\
54000.0000000003	54000.0000000003\\
92000.0000000005	92000.0000000005\\
127999.999999998	127999.999999998\\
-512000.000000001	-512000.000000001\\
383000.000000002	383000.000000002\\
111000	111000\\
-108999.999999999	-108999.999999999\\
-130000.000000003	-130000.000000003\\
-36000.0000000005	-36000.0000000005\\
347999.999999998	347999.999999998\\
2.66453525910038e-09	2.66453525910038e-09\\
-567000.000000003	-567000.000000003\\
254999.999999999	254999.999999999\\
-17000.0000000003	-17000.0000000003\\
218999.999999999	218999.999999999\\
-17999.9999999998	-17999.9999999998\\
-330000.000000003	-330000.000000003\\
275000.000000001	275000.000000001\\
127999.999999998	127999.999999998\\
-346999.999999997	-346999.999999997\\
217999.999999995	217999.999999995\\
-16999.9999999977	-16999.9999999977\\
-129000.000000003	-129000.000000003\\
276000.000000004	276000.000000004\\
-313000.000000004	-313000.000000004\\
203000.000000003	203000.000000003\\
-130000.000000001	-130000.000000001\\
-200000	-200000\\
348000	348000\\
-127999.999999999	-127999.999999999\\
-20000.0000000031	-20000.0000000031\\
385999.999999999	385999.999999999\\
-183999.999999999	-183999.999999999\\
-219000.000000002	-219000.000000002\\
-55000.0000000006	-55000.0000000006\\
-18999.9999999992	-18999.9999999992\\
367999.999999999	367999.999999999\\
-294999.999999999	-294999.999999999\\
-90999.9999999993	-90999.9999999993\\
36999.999999999	36999.999999999\\
273999.999999998	273999.999999998\\
-181999.999999997	-181999.999999997\\
256000	256000\\
-221000.000000004	-221000.000000004\\
221000	221000\\
-421000.000000001	-421000.000000001\\
256000.000000001	256000.000000001\\
-165000	-165000\\
8.88178419700125e-10	8.88178419700125e-10\\
-35999.9999999996	-35999.9999999996\\
200000	200000\\
37999.9999999985	37999.9999999985\\
-110000.000000002	-110000.000000002\\
-146999.999999999	-146999.999999999\\
329999.999999997	329999.999999997\\
18000.0000000016	18000.0000000016\\
-312000.000000002	-312000.000000002\\
93000.0000000026	93000.0000000026\\
-19000.0000000001	-19000.0000000001\\
147000.000000002	147000.000000002\\
145999.999999997	145999.999999997\\
-147000.000000001	-147000.000000001\\
-146000.000000001	-146000.000000001\\
-109000.000000001	-109000.000000001\\
34999.9999999993	34999.9999999993\\
148000.000000001	148000.000000001\\
16999.9999999977	16999.9999999977\\
-126999.999999998	-126999.999999998\\
-56000.0000000027	-56000.0000000027\\
331000.000000004	331000.000000004\\
-167000.000000003	-167000.000000003\\
-51999.9999999978	-51999.9999999978\\
234999.999999998	234999.999999998\\
-70999.9999999979	-70999.9999999979\\
-551000	-551000\\
330999.999999999	330999.999999999\\
255999.999999998	255999.999999998\\
-146999.999999999	-146999.999999999\\
73999.9999999989	73999.9999999989\\
-20000.0000000005	-20000.0000000005\\
-253999.999999999	-253999.999999999\\
54000.0000000003	54000.0000000003\\
273999.999999997	273999.999999997\\
18999.9999999983	18999.9999999983\\
-457999.999999999	-457999.999999999\\
475999.999999998	475999.999999998\\
-110000	-110000\\
-219000.000000002	-219000.000000002\\
219000	219000\\
-8.88178419700125e-10	-8.88178419700125e-10\\
-55000.0000000006	-55000.0000000006\\
-327999.999999997	-327999.999999997\\
290999.999999996	290999.999999996\\
312000.000000002	312000.000000002\\
-92000.0000000032	-92000.0000000032\\
-346999.999999998	-346999.999999998\\
35999.9999999969	35999.9999999969\\
36999.999999999	36999.999999999\\
72000.0000000018	72000.0000000018\\
149000	149000\\
-186000.000000001	-186000.000000001\\
-126000	-126000\\
255000.000000002	255000.000000002\\
19999.9999999996	19999.9999999996\\
-2000.00000000244	-2000.00000000244\\
-364999.999999999	-364999.999999999\\
89999.999999999	89999.999999999\\
351000.000000002	351000.000000002\\
-113000.000000001	-113000.000000001\\
93000.0000000008	93000.0000000008\\
-201999.999999999	-201999.999999999\\
-108999.999999998	-108999.999999998\\
366000	366000\\
-165000.000000004	-165000.000000004\\
-164999.999999998	-164999.999999998\\
73000.0000000031	73000.0000000031\\
-273999.999999999	-273999.999999999\\
218999.999999998	218999.999999998\\
220999.999999999	220999.999999999\\
52999.9999999982	52999.9999999982\\
-200000	-200000\\
-72999.9999999986	-72999.9999999986\\
329000	329000\\
-348000.000000001	-348000.000000001\\
54999.9999999979	54999.9999999979\\
147000.000000001	147000.000000001\\
-220999.999999997	-220999.999999997\\
-36000.0000000031	-36000.0000000031\\
184000.000000002	184000.000000002\\
292999.999999999	292999.999999999\\
-295000.000000002	-295000.000000002\\
-70999.9999999979	-70999.9999999979\\
-37999.9999999985	-37999.9999999985\\
183000.000000001	183000.000000001\\
-402000.000000004	-402000.000000004\\
129000.000000002	129000.000000002\\
583999.999999998	583999.999999998\\
-512000	-512000\\
91999.9999999978	91999.9999999978\\
72000.0000000018	72000.0000000018\\
112000.000000001	112000.000000001\\
-93000.0000000008	-93000.0000000008\\
-165000.000000002	-165000.000000002\\
18999.9999999983	18999.9999999983\\
36000.0000000005	36000.0000000005\\
1000.00000000033	1000.00000000033\\
72000.0000000018	72000.0000000018\\
-145000	-145000\\
126999.999999999	126999.999999999\\
-36999.9999999999	-36999.9999999999\\
276000.000000003	276000.000000003\\
-330000.000000002	-330000.000000002\\
-37000.0000000008	-37000.0000000008\\
256000.000000001	256000.000000001\\
-238000.000000001	-238000.000000001\\
130000.000000003	130000.000000003\\
198999.999999998	198999.999999998\\
-163000	-163000\\
-313000.000000002	-313000.000000002\\
20000.0000000031	20000.0000000031\\
383999.999999997	383999.999999997\\
-38000.0000000002	-38000.0000000002\\
-344999.999999998	-344999.999999998\\
344999.999999999	344999.999999999\\
-35000.0000000001	-35000.0000000001\\
-275999.999999998	-275999.999999998\\
166999.999999998	166999.999999998\\
126000	126000\\
-329000.000000001	-329000.000000001\\
184000.000000003	184000.000000003\\
273999.999999999	273999.999999999\\
-494000	-494000\\
163999.999999998	163999.999999998\\
330000.000000002	330000.000000002\\
-311000.000000001	-311000.000000001\\
311000	311000\\
-330000.000000002	-330000.000000002\\
220000.000000002	220000.000000002\\
1000.00000000033	1000.00000000033\\
-404000.000000003	-404000.000000003\\
348000	348000\\
1000.00000000211	1000.00000000211\\
-276000.000000002	-276000.000000002\\
421999.999999999	421999.999999999\\
-128999.999999998	-128999.999999998\\
-329000.000000002	-329000.000000002\\
-35999.9999999978	-35999.9999999978\\
309999.999999997	309999.999999997\\
38000.0000000029	38000.0000000029\\
-1000.00000000122	-1000.00000000122\\
-1.77635683940025e-09	-1.77635683940025e-09\\
-110000	-110000\\
92000.0000000005	92000.0000000005\\
37000.0000000008	37000.0000000008\\
-184000.000000002	-184000.000000002\\
129000.000000001	129000.000000001\\
-183000.000000001	-183000.000000001\\
309999.999999999	309999.999999999\\
-109000.000000002	-109000.000000002\\
-293000	-293000\\
184000.000000002	184000.000000002\\
290999.999999996	290999.999999996\\
-199999.999999997	-199999.999999997\\
-92000.0000000023	-92000.0000000023\\
8.88178419700125e-10	8.88178419700125e-10\\
164000	164000\\
-107999.999999999	-107999.999999999\\
-221000.000000002	-221000.000000002\\
221000.000000002	221000.000000002\\
33999.999999998	33999.999999998\\
94000.0000000012	94000.0000000012\\
-238999.999999999	-238999.999999999\\
146999.999999998	146999.999999998\\
126999.999999999	126999.999999999\\
-218000	-218000\\
-294000	-294000\\
383999.999999997	383999.999999997\\
129000.000000004	129000.000000004\\
-55000.0000000015	-55000.0000000015\\
-128000	-128000\\
71999.9999999992	71999.9999999992\\
-89999.9999999998	-89999.9999999998\\
72999.9999999977	72999.9999999977\\
-202999.999999999	-202999.999999999\\
259000.000000001	259000.000000001\\
51999.9999999978	51999.9999999978\\
-381999.999999998	-381999.999999998\\
200000.000000001	200000.000000001\\
91999.9999999987	91999.9999999987\\
-90999.9999999975	-90999.9999999975\\
126999.999999995	126999.999999995\\
-255999.999999999	-255999.999999999\\
129000	129000\\
127000.000000002	127000.000000002\\
164999.999999999	164999.999999999\\
-89999.9999999998	-89999.9999999998\\
-222000.000000001	-222000.000000001\\
55999.9999999983	55999.9999999983\\
-274000	-274000\\
273999.999999998	273999.999999998\\
-183999.999999998	-183999.999999998\\
350000.000000001	350000.000000001\\
-75000.0000000028	-75000.0000000028\\
-219999.999999999	-219999.999999999\\
112000	112000\\
-20000.0000000013	-20000.0000000013\\
109999.999999999	109999.999999999\\
-256000.000000001	-256000.000000001\\
383999.999999999	383999.999999999\\
-53999.9999999994	-53999.9999999994\\
-55999.9999999983	-55999.9999999983\\
-237000.000000002	-237000.000000002\\
91000.0000000011	91000.0000000011\\
-36000.0000000014	-36000.0000000014\\
292000	292000\\
-257000.000000001	-257000.000000001\\
2000.00000000511	2000.00000000511\\
256000	256000\\
-477000.000000005	-477000.000000005\\
128000.000000001	128000.000000001\\
439999.999999999	439999.999999999\\
-310999.999999998	-310999.999999998\\
-330000.000000003	-330000.000000003\\
641000.000000002	641000.000000002\\
-220000	-220000\\
-291999.999999998	-291999.999999998\\
327999.999999999	327999.999999999\\
111000.000000002	111000.000000002\\
-145999.999999999	-145999.999999999\\
-258000.000000005	-258000.000000005\\
-71999.9999999983	-71999.9999999983\\
127999.999999998	127999.999999998\\
347000	347000\\
-164000.000000001	-164000.000000001\\
56000.0000000036	56000.0000000036\\
33999.9999999954	33999.9999999954\\
-234999.999999996	-234999.999999996\\
124999.999999996	124999.999999996\\
-106999.999999997	-106999.999999997\\
52999.9999999955	52999.9999999955\\
-108999.999999998	-108999.999999998\\
91999.9999999987	91999.9999999987\\
310000	310000\\
-531000	-531000\\
167000.000000002	167000.000000002\\
179999.999999998	179999.999999998\\
-198999.999999994	-198999.999999994\\
127999.999999998	127999.999999998\\
-39000.0000000024	-39000.0000000024\\
333000.000000003	333000.000000003\\
-351000.000000003	-351000.000000003\\
-52999.9999999955	-52999.9999999955\\
146000	146000\\
54999.999999997	54999.999999997\\
-257000.000000001	-257000.000000001\\
-108999.999999998	-108999.999999998\\
348000.000000002	348000.000000002\\
254999.999999999	254999.999999999\\
-255000.000000002	-255000.000000002\\
-129000.000000001	-129000.000000001\\
129000.000000001	129000.000000001\\
-312000.000000002	-312000.000000002\\
-71999.9999999992	-71999.9999999992\\
328000.000000002	328000.000000002\\
-36999.9999999999	-36999.9999999999\\
129999.999999996	129999.999999996\\
-36999.9999999999	-36999.9999999999\\
-146999.999999999	-146999.999999999\\
36000.0000000022	36000.0000000022\\
18999.9999999983	18999.9999999983\\
-36000.0000000022	-36000.0000000022\\
-110999.999999999	-110999.999999999\\
73999.9999999998	73999.9999999998\\
182999.999999998	182999.999999998\\
8.88178419700125e-10	8.88178419700125e-10\\
-166000.000000001	-166000.000000001\\
-34999.9999999975	-34999.9999999975\\
183000.000000002	183000.000000002\\
-146999.999999998	-146999.999999998\\
165000	165000\\
35999.9999999987	35999.9999999987\\
-109000.000000001	-109000.000000001\\
-92000.0000000023	-92000.0000000023\\
-256999.999999998	-256999.999999998\\
386000.000000002	386000.000000002\\
163999.999999997	163999.999999997\\
-146999.999999998	-146999.999999998\\
73999.9999999981	73999.9999999981\\
-238000	-238000\\
-165000.000000003	-165000.000000003\\
385000.000000002	385000.000000002\\
-495999.999999999	-495999.999999999\\
312999.999999998	312999.999999998\\
163999.999999999	163999.999999999\\
-73000.0000000013	-73000.0000000013\\
127999.999999999	127999.999999999\\
-218999.999999998	-218999.999999998\\
-165999.999999998	-165999.999999998\\
239000	239000\\
-73999.9999999972	-73999.9999999972\\
146999.999999999	146999.999999999\\
-73999.9999999989	-73999.9999999989\\
-255000.000000002	-255000.000000002\\
-92000.0000000014	-92000.0000000014\\
475000	475000\\
-17000.0000000012	-17000.0000000012\\
72000.0000000009	72000.0000000009\\
-419999.999999999	-419999.999999999\\
37000.0000000017	37000.0000000017\\
364999.999999998	364999.999999998\\
-127999.999999999	-127999.999999999\\
35999.9999999996	35999.9999999996\\
-198999.999999998	-198999.999999998\\
70999.9999999979	70999.9999999979\\
183000.000000002	183000.000000002\\
-236000.000000001	-236000.000000001\\
199999.999999999	199999.999999999\\
-183000.000000001	-183000.000000001\\
-348000.000000001	-348000.000000001\\
713999.999999998	713999.999999998\\
-365999.999999998	-365999.999999998\\
-18000.0000000025	-18000.0000000025\\
146000.000000003	146000.000000003\\
-72000.0000000001	-72000.0000000001\\
-20000.0000000031	-20000.0000000031\\
-36000.0000000005	-36000.0000000005\\
54999.9999999988	54999.9999999988\\
72999.9999999995	72999.9999999995\\
19000.000000001	19000.000000001\\
-220000.000000003	-220000.000000003\\
365000.000000002	365000.000000002\\
-33999.9999999998	-33999.9999999998\\
-588999.999999998	-588999.999999998\\
166999.999999998	166999.999999998\\
548999.999999998	548999.999999998\\
-128999.999999996	-128999.999999996\\
-641000	-641000\\
294000.000000001	294000.000000001\\
218999.999999999	218999.999999999\\
-36000.0000000022	-36000.0000000022\\
91000.0000000002	91000.0000000002\\
-329000.000000001	-329000.000000001\\
328000.000000001	328000.000000001\\
112000	112000\\
-495000.000000001	-495000.000000001\\
529999.999999999	529999.999999999\\
-273999.999999998	-273999.999999998\\
-54999.9999999979	-54999.9999999979\\
128000	128000\\
-365000.000000001	-365000.000000001\\
401000.000000001	401000.000000001\\
-255000.000000001	-255000.000000001\\
-1000.00000000211	-1000.00000000211\\
348999.999999998	348999.999999998\\
-36999.9999999981	-36999.9999999981\\
-239000	-239000\\
128999.999999999	128999.999999999\\
-183000.000000002	-183000.000000002\\
256000.000000002	256000.000000002\\
-164000.000000001	-164000.000000001\\
36000.0000000005	36000.0000000005\\
73000.0000000031	73000.0000000031\\
-238000.000000001	-238000.000000001\\
238000.000000004	238000.000000004\\
-274000.000000003	-274000.000000003\\
329000	329000\\
36999.9999999999	36999.9999999999\\
-256999.999999998	-256999.999999998\\
165999.999999999	165999.999999999\\
-165999.999999996	-165999.999999996\\
-275000	-275000\\
477999.999999999	477999.999999999\\
236999.999999997	236999.999999997\\
-385000	-385000\\
127999.999999999	127999.999999999\\
-401999.999999999	-401999.999999999\\
421000.000000001	421000.000000001\\
127999.999999999	127999.999999999\\
-439000	-439000\\
273999.999999997	273999.999999997\\
-219999.999999999	-219999.999999999\\
386000	386000\\
-294000.000000001	-294000.000000001\\
-110999.999999997	-110999.999999997\\
148999.999999997	148999.999999997\\
181000	181000\\
-237000.000000001	-237000.000000001\\
54999.9999999997	54999.9999999997\\
-128999.999999998	-128999.999999998\\
293999.999999997	293999.999999997\\
-293999.999999999	-293999.999999999\\
55000.0000000015	55000.0000000015\\
368000	368000\\
-76000.0000000014	-76000.0000000014\\
-253999.999999997	-253999.999999997\\
-239000.000000002	-239000.000000002\\
274999.999999996	274999.999999996\\
237000.000000002	237000.000000002\\
-329000.000000001	-329000.000000001\\
294000.000000001	294000.000000001\\
52999.9999999964	52999.9999999964\\
-528999.999999996	-528999.999999996\\
-57000.0000000004	-57000.0000000004\\
515000.000000001	515000.000000001\\
-20000.000000004	-20000.000000004\\
-383999.999999998	-383999.999999998\\
365999.999999998	365999.999999998\\
-17999.9999999971	-17999.9999999971\\
-55000.0000000033	-55000.0000000033\\
-183999.999999998	-183999.999999998\\
20000.0000000013	20000.0000000013\\
273999.999999997	273999.999999997\\
-184000.000000001	-184000.000000001\\
19000.000000001	19000.000000001\\
92000.0000000005	92000.0000000005\\
-184000.000000002	-184000.000000002\\
148000.000000001	148000.000000001\\
-94000.0000000021	-94000.0000000021\\
-217000	-217000\\
363999.999999999	363999.999999999\\
-53999.9999999994	-53999.9999999994\\
17999.9999999989	17999.9999999989\\
37000.0000000017	37000.0000000017\\
-73000.0000000013	-73000.0000000013\\
-202000.000000002	-202000.000000002\\
92000.0000000005	92000.0000000005\\
53999.9999999976	53999.9999999976\\
-34999.9999999975	-34999.9999999975\\
107999.999999999	107999.999999999\\
93000	93000\\
36000.0000000005	36000.0000000005\\
-585999.999999998	-585999.999999998\\
330000.000000002	330000.000000002\\
402999.999999999	402999.999999999\\
-367000	-367000\\
-91000.0000000037	-91000.0000000037\\
274000.000000001	274000.000000001\\
-128000.000000001	-128000.000000001\\
-108999.999999996	-108999.999999996\\
109999.999999999	109999.999999999\\
-75000.0000000019	-75000.0000000019\\
1999.99999999978	1999.99999999978\\
217999.999999999	217999.999999999\\
-255000.000000001	-255000.000000001\\
126999.999999998	126999.999999998\\
-53999.9999999994	-53999.9999999994\\
72999.9999999986	72999.9999999986\\
-182999.999999999	-182999.999999999\\
291999.999999997	291999.999999997\\
-200000	-200000\\
-73999.9999999998	-73999.9999999998\\
164999.999999998	164999.999999998\\
201000	201000\\
-199999.999999997	-199999.999999997\\
-149000	-149000\\
-199000.000000002	-199000.000000002\\
-92999.9999999991	-92999.9999999991\\
568999.999999999	568999.999999999\\
52999.999999999	52999.999999999\\
-89999.9999999998	-89999.9999999998\\
-257000.000000002	-257000.000000002\\
128000	128000\\
38000.0000000011	38000.0000000011\\
52999.9999999982	52999.9999999982\\
-311000	-311000\\
202999.999999998	202999.999999998\\
17000.0000000012	17000.0000000012\\
-55000.0000000006	-55000.0000000006\\
93000.0000000008	93000.0000000008\\
-39000.0000000024	-39000.0000000024\\
93999.9999999994	93999.9999999994\\
-275999.999999998	-275999.999999998\\
-109999.999999999	-109999.999999999\\
603999.999999997	603999.999999997\\
-383000	-383000\\
71999.9999999992	71999.9999999992\\
-110000	-110000\\
129000.000000003	129000.000000003\\
-73000.0000000022	-73000.0000000022\\
163000	163000\\
-180999.999999997	-180999.999999997\\
53999.9999999976	53999.9999999976\\
255999.999999999	255999.999999999\\
-604000.000000003	-604000.000000003\\
513000.000000001	513000.000000001\\
-220000.000000001	-220000.000000001\\
-17999.9999999998	-17999.9999999998\\
35999.9999999969	35999.9999999969\\
202000.000000001	202000.000000001\\
-257000	-257000\\
-17000.0000000003	-17000.0000000003\\
290999.999999999	290999.999999999\\
-126000.000000001	-126000.000000001\\
-75000.0000000002	-75000.0000000002\\
-72000.0000000027	-72000.0000000027\\
127000.000000002	127000.000000002\\
37999.9999999976	37999.9999999976\\
-92999.9999999991	-92999.9999999991\\
-54000.0000000011	-54000.0000000011\\
73000.0000000004	73000.0000000004\\
-238000.000000001	-238000.000000001\\
256000.000000002	256000.000000002\\
220000.000000001	220000.000000001\\
-255999.999999998	-255999.999999998\\
36999.999999999	36999.999999999\\
162999.999999998	162999.999999998\\
-217999.999999999	-217999.999999999\\
-38000.0000000011	-38000.0000000011\\
93000	93000\\
-147000.000000003	-147000.000000003\\
-54999.9999999979	-54999.9999999979\\
273999.999999996	273999.999999996\\
-72999.9999999977	-72999.9999999977\\
110999.999999999	110999.999999999\\
-1000.00000000033	-1000.00000000033\\
-495000.000000001	-495000.000000001\\
423000.000000001	423000.000000001\\
-2000.00000000067	-2000.00000000067\\
111000.000000004	111000.000000004\\
-457000.000000002	-457000.000000002\\
529000.000000001	529000.000000001\\
37999.9999999976	37999.9999999976\\
-347999.999999997	-347999.999999997\\
-19000.0000000037	-19000.0000000037\\
-183000	-183000\\
258000.000000001	258000.000000001\\
16000	16000\\
220999.999999997	220999.999999997\\
-165000	-165000\\
128000.000000001	128000.000000001\\
-367000.000000002	-367000.000000002\\
39000.0000000024	39000.0000000024\\
143999.999999997	143999.999999997\\
166000	166000\\
-293000.000000001	-293000.000000001\\
165000	165000\\
126999.999999998	126999.999999998\\
-528999.999999997	-528999.999999997\\
419000	419000\\
-72000.0000000018	-72000.0000000018\\
146000	146000\\
-147000	-147000\\
20000.0000000013	20000.0000000013\\
-75000.000000001	-75000.000000001\\
257999.999999999	257999.999999999\\
-331000	-331000\\
55999.9999999991	55999.9999999991\\
309999.999999998	309999.999999998\\
-291999.999999999	-291999.999999999\\
90999.9999999993	90999.9999999993\\
-36000.0000000022	-36000.0000000022\\
-146999.999999999	-146999.999999999\\
385000.000000001	385000.000000001\\
-183000.000000001	-183000.000000001\\
-148000	-148000\\
149000.000000001	149000.000000001\\
88999.9999999977	88999.9999999977\\
-382999.999999998	-382999.999999998\\
164999.999999998	164999.999999998\\
292000.000000002	292000.000000002\\
-256000	-256000\\
-146000.000000001	-146000.000000001\\
548000.000000002	548000.000000002\\
-309999.999999999	-309999.999999999\\
-36000.0000000014	-36000.0000000014\\
-185000.000000001	-185000.000000001\\
55999.9999999983	55999.9999999983\\
438999.999999999	438999.999999999\\
-348000	-348000\\
-145000	-145000\\
400999.999999999	400999.999999999\\
-72000.0000000009	-72000.0000000009\\
-403999.999999999	-403999.999999999\\
274999.999999998	274999.999999998\\
93000.0000000035	93000.0000000035\\
-221000.000000002	-221000.000000002\\
109000.000000002	109000.000000002\\
19999.9999999996	19999.9999999996\\
-147000.000000003	-147000.000000003\\
73000.0000000013	73000.0000000013\\
73000.0000000004	73000.0000000004\\
-55000.0000000015	-55000.0000000015\\
129000.000000003	129000.000000003\\
-109999.999999998	-109999.999999998\\
-19000.0000000019	-19000.0000000019\\
147000.000000003	147000.000000003\\
0	0\\
-183000.000000002	-183000.000000002\\
53999.9999999994	53999.9999999994\\
1000.00000000122	1000.00000000122\\
-110000	-110000\\
311000	311000\\
-238000.000000003	-238000.000000003\\
73000.0000000004	73000.0000000004\\
1000.00000000033	1000.00000000033\\
-239000.000000002	-239000.000000002\\
420999.999999999	420999.999999999\\
-200999.999999998	-200999.999999998\\
19000.000000001	19000.000000001\\
126999.999999999	126999.999999999\\
-109000.000000003	-109000.000000003\\
-56000.0000000009	-56000.0000000009\\
-310000	-310000\\
439000	439000\\
-184000.000000002	-184000.000000002\\
203000	203000\\
-258000.000000001	-258000.000000001\\
-71999.9999999965	-71999.9999999965\\
292999.999999997	292999.999999997\\
164000.000000001	164000.000000001\\
-293000.000000003	-293000.000000003\\
37000.0000000026	37000.0000000026\\
147000.000000001	147000.000000001\\
-293000.000000001	-293000.000000001\\
89999.9999999972	89999.9999999972\\
111000.000000001	111000.000000001\\
-72999.9999999986	-72999.9999999986\\
-110000	-110000\\
54999.9999999997	54999.9999999997\\
236999.999999998	236999.999999998\\
-108999.999999998	-108999.999999998\\
-108999.999999999	-108999.999999999\\
273000	273000\\
-384000.000000003	-384000.000000003\\
220000.000000002	220000.000000002\\
-90999.9999999984	-90999.9999999984\\
-111000.000000001	-111000.000000001\\
183999.999999998	183999.999999998\\
347000.000000001	347000.000000001\\
-585000.000000001	-585000.000000001\\
127999.999999998	127999.999999998\\
-56000.0000000009	-56000.0000000009\\
184000	184000\\
111000.000000004	111000.000000004\\
-314000.000000004	-314000.000000004\\
-70999.9999999971	-70999.9999999971\\
72999.9999999986	72999.9999999986\\
420000.000000002	420000.000000002\\
-126999.999999999	-126999.999999999\\
-93000	-93000\\
-181999.999999999	-181999.999999999\\
74000.0000000016	74000.0000000016\\
255000	255000\\
-275000.000000002	-275000.000000002\\
2000.00000000333	2000.00000000333\\
272999.999999996	272999.999999996\\
-238000.000000001	-238000.000000001\\
-164999.999999999	-164999.999999999\\
405000	405000\\
-149000.000000002	-149000.000000002\\
-292000.000000001	-292000.000000001\\
383999.999999999	383999.999999999\\
-217999.999999999	-217999.999999999\\
52999.9999999982	52999.9999999982\\
-36000.0000000014	-36000.0000000014\\
-54999.9999999979	-54999.9999999979\\
330000	330000\\
-184000.000000002	-184000.000000002\\
-126999.999999995	-126999.999999995\\
17999.9999999998	17999.9999999998\\
72999.9999999977	72999.9999999977\\
110000	110000\\
-56000.0000000009	-56000.0000000009\\
-291000	-291000\\
345999.999999999	345999.999999999\\
56000.0000000009	56000.0000000009\\
-476000	-476000\\
475999.999999999	475999.999999999\\
-19000.0000000019	-19000.0000000019\\
-108999.999999998	-108999.999999998\\
16999.9999999986	16999.9999999986\\
-108000.000000001	-108000.000000001\\
-294000.000000001	-294000.000000001\\
475999.999999999	475999.999999999\\
37000.0000000017	37000.0000000017\\
-110000.000000002	-110000.000000002\\
-220000.000000001	-220000.000000001\\
-8.88178419700125e-10	-8.88178419700125e-10\\
202000	202000\\
-37000.0000000008	-37000.0000000008\\
-18999.9999999992	-18999.9999999992\\
147999.999999999	147999.999999999\\
-147999.999999997	-147999.999999997\\
221000	221000\\
35999.9999999996	35999.9999999996\\
-623000.000000004	-623000.000000004\\
386000.000000003	386000.000000003\\
-56999.9999999986	-56999.9999999986\\
-145000.000000004	-145000.000000004\\
329000.000000001	329000.000000001\\
-238000.000000001	-238000.000000001\\
404000.000000003	404000.000000003\\
-259000.000000001	-259000.000000001\\
-344999.999999998	-344999.999999998\\
583999.999999999	583999.999999999\\
-182000.000000001	-182000.000000001\\
-92000.0000000005	-92000.0000000005\\
-202000.000000002	-202000.000000002\\
93000.0000000026	93000.0000000026\\
89999.9999999981	89999.9999999981\\
257000.000000001	257000.000000001\\
-183000.000000001	-183000.000000001\\
-164000	-164000\\
52999.9999999973	52999.9999999973\\
-53000.0000000008	-53000.0000000008\\
-19999.9999999978	-19999.9999999978\\
367000	367000\\
-181999.999999999	-181999.999999999\\
125999.999999998	125999.999999998\\
-162999.999999999	-162999.999999999\\
-239000.000000002	-239000.000000002\\
603999.999999999	603999.999999999\\
-768000.000000001	-768000.000000001\\
348000	348000\\
52999.9999999999	52999.9999999999\\
-254000.000000002	-254000.000000002\\
273000.000000001	273000.000000001\\
37999.9999999994	37999.9999999994\\
-130000.000000002	-130000.000000002\\
258000.000000004	258000.000000004\\
54999.9999999988	54999.9999999988\\
-368000.000000004	-368000.000000004\\
20000.0000000013	20000.0000000013\\
17999.9999999998	17999.9999999998\\
-74000.0000000016	-74000.0000000016\\
129000.000000001	129000.000000001\\
274000	274000\\
-145999.999999997	-145999.999999997\\
-311000.000000003	-311000.000000003\\
109000.000000003	109000.000000003\\
293999.999999997	293999.999999997\\
-111000.000000001	-111000.000000001\\
-128000	-128000\\
56000	56000\\
-36999.9999999999	-36999.9999999999\\
-999.999999997669	-999.999999997669\\
-236999.999999999	-236999.999999999\\
181999.999999999	181999.999999999\\
220999.999999997	220999.999999997\\
-19000.000000001	-19000.000000001\\
-145999.999999998	-145999.999999998\\
237999.999999997	237999.999999997\\
-164999.999999999	-164999.999999999\\
-348999.999999997	-348999.999999997\\
37999.9999999985	37999.9999999985\\
238000	238000\\
54999.999999997	54999.999999997\\
54000.0000000011	54000.0000000011\\
-109000.000000003	-109000.000000003\\
-36999.9999999999	-36999.9999999999\\
-73000.0000000013	-73000.0000000013\\
347000	347000\\
-199999.999999997	-199999.999999997\\
71999.9999999983	71999.9999999983\\
147000.000000003	147000.000000003\\
-347000.000000001	-347000.000000001\\
-19999.9999999987	-19999.9999999987\\
19999.9999999969	19999.9999999969\\
218000	218000\\
20000.0000000013	20000.0000000013\\
-112000.000000003	-112000.000000003\\
-33999.9999999971	-33999.9999999971\\
-258000.000000003	-258000.000000003\\
348000.000000001	348000.000000001\\
182999.999999998	182999.999999998\\
-236999.999999998	-236999.999999998\\
-311999.999999999	-311999.999999999\\
219000	219000\\
275999.999999998	275999.999999998\\
-348999.999999999	-348999.999999999\\
258000	258000\\
-240000.000000003	-240000.000000003\\
1000.00000000122	1000.00000000122\\
383999.999999998	383999.999999998\\
-383999.999999998	-383999.999999998\\
2.66453525910038e-09	2.66453525910038e-09\\
238000	238000\\
73000.0000000013	73000.0000000013\\
8.88178419700125e-10	8.88178419700125e-10\\
-238000.000000002	-238000.000000002\\
129000	129000\\
-368000.000000002	-368000.000000002\\
404000	404000\\
-348000.000000001	-348000.000000001\\
367000	367000\\
17000.0000000003	17000.0000000003\\
-330000.000000002	-330000.000000002\\
387000.000000001	387000.000000001\\
-369000.000000002	-369000.000000002\\
258000.000000001	258000.000000001\\
72999.9999999986	72999.9999999986\\
-366999.999999996	-366999.999999996\\
274999.999999997	274999.999999997\\
-127000	-127000\\
16999.9999999995	16999.9999999995\\
184000	184000\\
-312999.999999999	-312999.999999999\\
93999.9999999994	93999.9999999994\\
144000	144000\\
-53000.0000000026	-53000.0000000026\\
-202000.000000001	-202000.000000001\\
165000	165000\\
218999.999999998	218999.999999998\\
-128000.000000001	-128000.000000001\\
-346999.999999998	-346999.999999998\\
291999.999999996	291999.999999996\\
201000.000000003	201000.000000003\\
-273000.000000001	-273000.000000001\\
364999.999999999	364999.999999999\\
-293000.000000001	-293000.000000001\\
-475000	-475000\\
383000.000000001	383000.000000001\\
202000	202000\\
-218999.999999999	-218999.999999999\\
201000	201000\\
110000.000000001	110000.000000001\\
-184000.000000003	-184000.000000003\\
-183000	-183000\\
75000.0000000028	75000.0000000028\\
180999.999999997	180999.999999997\\
73999.9999999981	73999.9999999981\\
-419999.999999996	-419999.999999996\\
236000.000000001	236000.000000001\\
92999.9999999991	92999.9999999991\\
-73999.9999999972	-73999.9999999972\\
-256000	-256000\\
127999.999999997	127999.999999997\\
274999.999999999	274999.999999999\\
-55000.0000000015	-55000.0000000015\\
-329999.999999999	-329999.999999999\\
129000	129000\\
439000.000000001	439000.000000001\\
-514000.000000003	-514000.000000003\\
167000.000000005	167000.000000005\\
144999.999999998	144999.999999998\\
19000.0000000028	19000.0000000028\\
-366000.000000003	-366000.000000003\\
72999.9999999995	72999.9999999995\\
401999.999999999	401999.999999999\\
-401999.999999998	-401999.999999998\\
146000.000000001	146000.000000001\\
92000.0000000005	92000.0000000005\\
-202000.000000001	-202000.000000001\\
-127000.000000001	-127000.000000001\\
183000.000000001	183000.000000001\\
218000.000000001	218000.000000001\\
-255000.000000002	-255000.000000002\\
54999.9999999988	54999.9999999988\\
164000.000000001	164000.000000001\\
-18000.0000000025	-18000.0000000025\\
-181999.999999997	-181999.999999997\\
-92999.9999999982	-92999.9999999982\\
74000.0000000007	74000.0000000007\\
90999.9999999984	90999.9999999984\\
-310999.999999997	-310999.999999997\\
475999.999999995	475999.999999995\\
37000.0000000017	37000.0000000017\\
-257000.000000003	-257000.000000003\\
-90999.9999999984	-90999.9999999984\\
55000.0000000015	55000.0000000015\\
146000	146000\\
-17999.9999999971	-17999.9999999971\\
109999.999999999	109999.999999999\\
-256000.000000003	-256000.000000003\\
53000.0000000017	53000.0000000017\\
56999.9999999986	56999.9999999986\\
-91999.9999999996	-91999.9999999996\\
0	0\\
199999.999999999	199999.999999999\\
-70999.9999999979	-70999.9999999979\\
-1999.99999999978	-1999.99999999978\\
-71000.0000000006	-71000.0000000006\\
-57000.0000000022	-57000.0000000022\\
18999.9999999992	18999.9999999992\\
274999.999999999	274999.999999999\\
-312000.000000002	-312000.000000002\\
-90999.9999999984	-90999.9999999984\\
219999.999999997	219999.999999997\\
91000.0000000002	91000.0000000002\\
-201999.999999999	-201999.999999999\\
129999.999999998	129999.999999998\\
-1999.99999999978	-1999.99999999978\\
20000.0000000031	20000.0000000031\\
-294000.000000001	-294000.000000001\\
999.999999998557	999.999999998557\\
529000	529000\\
-144000	-144000\\
-350000.000000002	-350000.000000002\\
367000.000000001	367000.000000001\\
-199999.999999997	-199999.999999997\\
-241000	-241000\\
459999.999999997	459999.999999997\\
-275000.000000001	-275000.000000001\\
8.88178419700125e-10	8.88178419700125e-10\\
182999.999999999	182999.999999999\\
-293999.999999996	-293999.999999996\\
312999.999999997	312999.999999997\\
-19999.9999999969	-19999.9999999969\\
-365000	-365000\\
330000.000000001	330000.000000001\\
-129999.999999999	-129999.999999999\\
129999.999999996	129999.999999996\\
-73999.9999999972	-73999.9999999972\\
-91999.9999999987	-91999.9999999987\\
182999.999999998	182999.999999998\\
-127000.000000002	-127000.000000002\\
-237999.999999998	-237999.999999998\\
602999.999999997	602999.999999997\\
-238000	-238000\\
-200000	-200000\\
254999.999999998	254999.999999998\\
-16999.9999999959	-16999.9999999959\\
-496000.000000001	-496000.000000001\\
476999.999999998	476999.999999998\\
72999.9999999995	72999.9999999995\\
-438999.999999998	-438999.999999998\\
255000.000000001	255000.000000001\\
38999.9999999979	38999.9999999979\\
126000	126000\\
-91000.0000000019	-91000.0000000019\\
-163999.999999998	-163999.999999998\\
35999.9999999996	35999.9999999996\\
-182999.999999999	-182999.999999999\\
365999.999999998	365999.999999998\\
-146000.000000001	-146000.000000001\\
-55999.9999999991	-55999.9999999991\\
19999.9999999978	19999.9999999978\\
292000.000000002	292000.000000002\\
-385000.000000001	-385000.000000001\\
128999.999999999	128999.999999999\\
164000	164000\\
-127999.999999998	-127999.999999998\\
-327999.999999998	-327999.999999998\\
638999.999999995	638999.999999995\\
-438999.999999999	-438999.999999999\\
182999.999999999	182999.999999999\\
-127999.999999997	-127999.999999997\\
1000.00000000033	1000.00000000033\\
346000	346000\\
-217999.999999999	-217999.999999999\\
-184000.000000003	-184000.000000003\\
-128000.000000001	-128000.000000001\\
164000.000000001	164000.000000001\\
202999.999999998	202999.999999998\\
-999.999999997669	-999.999999997669\\
35999.9999999969	35999.9999999969\\
-400999.999999999	-400999.999999999\\
-93000.0000000017	-93000.0000000017\\
385000.000000001	385000.000000001\\
109999.999999998	109999.999999998\\
146000.000000002	146000.000000002\\
-403000.000000003	-403000.000000003\\
-127000	-127000\\
35999.9999999996	35999.9999999996\\
310000.000000001	310000.000000001\\
-33999.9999999998	-33999.9999999998\\
70999.9999999971	70999.9999999971\\
-401999.999999997	-401999.999999997\\
256999.999999997	256999.999999997\\
71999.9999999992	71999.9999999992\\
-256000	-256000\\
311000.000000001	311000.000000001\\
-52999.9999999982	-52999.9999999982\\
-204000.000000001	-204000.000000001\\
185000	185000\\
126999.999999999	126999.999999999\\
-293000.000000002	-293000.000000002\\
-89999.9999999998	-89999.9999999998\\
126000	126000\\
111000.000000001	111000.000000001\\
111000	111000\\
-350000.000000001	-350000.000000001\\
238999.999999998	238999.999999998\\
91000.0000000002	91000.0000000002\\
-364999.999999999	-364999.999999999\\
291999.999999997	291999.999999997\\
-201000	-201000\\
182000.000000001	182000.000000001\\
130000.000000002	130000.000000002\\
-146999.999999999	-146999.999999999\\
53999.9999999976	53999.9999999976\\
-347000.000000002	-347000.000000002\\
310000.000000001	310000.000000001\\
1999.99999999889	1999.99999999889\\
-112000	-112000\\
-53000.0000000017	-53000.0000000017\\
17000.0000000012	17000.0000000012\\
202000.000000001	202000.000000001\\
-110000	-110000\\
239000.000000003	239000.000000003\\
-277000.000000005	-277000.000000005\\
39000.0000000024	39000.0000000024\\
-129999.999999999	-129999.999999999\\
-54000.0000000029	-54000.0000000029\\
293000	293000\\
-73000.0000000004	-73000.0000000004\\
-221000.000000001	-221000.000000001\\
222000	222000\\
16000	16000\\
999.999999997669	999.999999997669\\
-73000.0000000013	-73000.0000000013\\
237000.000000001	237000.000000001\\
-200000	-200000\\
-74000.0000000007	-74000.0000000007\\
-19000.0000000019	-19000.0000000019\\
-71999.9999999965	-71999.9999999965\\
275000	275000\\
34999.9999999984	34999.9999999984\\
-256000.000000003	-256000.000000003\\
-162999.999999996	-162999.999999996\\
271999.999999997	271999.999999997\\
148000.000000003	148000.000000003\\
-383999.999999998	-383999.999999998\\
273999.999999997	273999.999999997\\
181999.999999999	181999.999999999\\
-125999.999999999	-125999.999999999\\
-112000.000000003	-112000.000000003\\
-126999.999999999	-126999.999999999\\
-146000.000000003	-146000.000000003\\
475000	475000\\
-201000.000000001	-201000.000000001\\
-72999.9999999995	-72999.9999999995\\
90999.9999999966	90999.9999999966\\
18000.0000000033	18000.0000000033\\
-35000.0000000019	-35000.0000000019\\
109000.000000003	109000.000000003\\
-184000.000000005	-184000.000000005\\
166000.000000002	166000.000000002\\
-111000.000000002	-111000.000000002\\
-199999.999999997	-199999.999999997\\
163999.999999996	163999.999999996\\
146000.000000004	146000.000000004\\
185000	185000\\
-699000.000000002	-699000.000000002\\
534000.000000003	534000.000000003\\
237000	237000\\
-312000	-312000\\
-328000.000000002	-328000.000000002\\
346000	346000\\
-35000.0000000019	-35000.0000000019\\
-130000	-130000\\
332000.000000003	332000.000000003\\
-423000.000000004	-423000.000000004\\
-72999.9999999995	-72999.9999999995\\
477000.000000003	477000.000000003\\
72999.9999999995	72999.9999999995\\
-386000.000000002	-386000.000000002\\
2000.00000000333	2000.00000000333\\
181999.999999998	181999.999999998\\
18999.9999999992	18999.9999999992\\
-111000.000000001	-111000.000000001\\
91999.9999999987	91999.9999999987\\
-291999.999999999	-291999.999999999\\
201000	201000\\
-74999.9999999993	-74999.9999999993\\
131000	131000\\
-93000.0000000008	-93000.0000000008\\
292000.000000002	292000.000000002\\
-108000.000000001	-108000.000000001\\
-166000	-166000\\
-90999.9999999966	-90999.9999999966\\
219999.999999999	219999.999999999\\
53999.9999999958	53999.9999999958\\
-292999.999999998	-292999.999999998\\
37999.9999999994	37999.9999999994\\
254999.999999998	254999.999999998\\
-91000.0000000011	-91000.0000000011\\
-293000	-293000\\
146999.999999999	146999.999999999\\
327999.999999998	327999.999999998\\
-180999.999999996	-180999.999999996\\
-130000.000000003	-130000.000000003\\
-292000.000000002	-292000.000000002\\
421000.000000001	421000.000000001\\
-714000.000000003	-714000.000000003\\
220000.000000001	220000.000000001\\
35999.9999999987	35999.9999999987\\
-73000.0000000013	-73000.0000000013\\
128000.000000002	128000.000000002\\
37000.0000000008	37000.0000000008\\
-108999.999999999	-108999.999999999\\
16000	16000\\
-181000.000000001	-181000.000000001\\
-74999.9999999984	-74999.9999999984\\
332000.000000001	332000.000000001\\
161999.999999996	161999.999999996\\
-199000	-199000\\
-19999.9999999978	-19999.9999999978\\
-218999.999999998	-218999.999999998\\
-19000.0000000001	-19000.0000000001\\
220999.999999997	220999.999999997\\
91000.0000000019	91000.0000000019\\
72999.9999999977	72999.9999999977\\
-220000.000000002	-220000.000000002\\
-72999.9999999977	-72999.9999999977\\
19000.0000000001	19000.0000000001\\
-8.88178419700125e-10	-8.88178419700125e-10\\
163000	163000\\
129999.999999999	129999.999999999\\
-36999.9999999999	-36999.9999999999\\
-330000	-330000\\
127999.999999997	127999.999999997\\
37000.0000000017	37000.0000000017\\
129000.000000001	129000.000000001\\
-258000.000000004	-258000.000000004\\
56000.0000000009	56000.0000000009\\
127999.999999998	127999.999999998\\
72999.9999999995	72999.9999999995\\
-36000.0000000005	-36000.0000000005\\
-349000.000000001	-349000.000000001\\
130000	130000\\
53000.0000000008	53000.0000000008\\
182999.999999999	182999.999999999\\
-107999.999999995	-107999.999999995\\
145999.999999995	145999.999999995\\
-202999.999999998	-202999.999999998\\
-89999.9999999998	-89999.9999999998\\
54000.0000000011	54000.0000000011\\
330999.999999999	330999.999999999\\
-1999.99999999889	-1999.99999999889\\
-401000.000000002	-401000.000000002\\
52999.999999999	52999.999999999\\
21000.0000000043	21000.0000000043\\
436999.999999997	436999.999999997\\
-145999.999999999	-145999.999999999\\
-310000	-310000\\
-75000.0000000028	-75000.0000000028\\
258000.000000003	258000.000000003\\
-1000.00000000122	-1000.00000000122\\
-256000	-256000\\
-17999.9999999998	-17999.9999999998\\
182000	182000\\
128999.999999998	128999.999999998\\
17999.999999998	17999.999999998\\
-163999.999999998	-163999.999999998\\
-73999.9999999998	-73999.9999999998\\
73000.0000000013	73000.0000000013\\
109999.999999998	109999.999999998\\
-53999.9999999994	-53999.9999999994\\
108999.999999998	108999.999999998\\
-91999.9999999978	-91999.9999999978\\
-90000.0000000016	-90000.0000000016\\
-147999.999999996	-147999.999999996\\
201999.999999998	201999.999999998\\
164999.999999998	164999.999999998\\
-54999.9999999979	-54999.9999999979\\
-384999.999999998	-384999.999999998\\
475999.999999997	475999.999999997\\
-292000.000000001	-292000.000000001\\
0	0\\
-56000	-56000\\
292999.999999998	292999.999999998\\
74000.0000000034	74000.0000000034\\
-476000.000000001	-476000.000000001\\
329000.000000001	329000.000000001\\
36999.999999999	36999.999999999\\
-72999.9999999986	-72999.9999999986\\
-276000.000000003	-276000.000000003\\
203000.000000003	203000.000000003\\
182999.999999998	182999.999999998\\
-184999.999999999	-184999.999999999\\
167000.000000002	167000.000000002\\
-110000.000000002	-110000.000000002\\
-57000.0000000013	-57000.0000000013\\
-107999.999999997	-107999.999999997\\
292999.999999997	292999.999999997\\
-257000.000000001	-257000.000000001\\
146000.000000001	146000.000000001\\
1000.00000000033	1000.00000000033\\
-384999.999999999	-384999.999999999\\
696000.000000001	696000.000000001\\
0	0\\
-421000	-421000\\
-147000.000000003	-147000.000000003\\
73000.0000000004	73000.0000000004\\
129000	129000\\
18000.0000000025	18000.0000000025\\
37000.0000000017	37000.0000000017\\
-1000.00000000122	-1000.00000000122\\
148000.000000001	148000.000000001\\
-238999.999999999	-238999.999999999\\
-56000.0000000036	-56000.0000000036\\
94000.0000000012	94000.0000000012\\
254000	254000\\
-237000.000000002	-237000.000000002\\
-328999.999999998	-328999.999999998\\
421000	421000\\
-185000.000000002	-185000.000000002\\
76000.000000004	76000.000000004\\
180999.999999999	180999.999999999\\
-72000.0000000018	-72000.0000000018\\
-73999.9999999981	-73999.9999999981\\
-219000.000000001	-219000.000000001\\
108999.999999998	108999.999999998\\
38000.0000000002	38000.0000000002\\
162999.999999999	162999.999999999\\
-108999.999999998	-108999.999999998\\
145999.999999998	145999.999999998\\
-107999.999999997	-107999.999999997\\
106999.999999999	106999.999999999\\
-89000.0000000004	-89000.0000000004\\
-514000.000000003	-514000.000000003\\
476000.000000003	476000.000000003\\
330000.000000001	330000.000000001\\
-659000.000000002	-659000.000000002\\
238000	238000\\
127000.000000002	127000.000000002\\
239999.999999998	239999.999999998\\
-129999.999999999	-129999.999999999\\
-200000	-200000\\
-221000.000000004	-221000.000000004\\
221000.000000002	221000.000000002\\
71999.9999999974	71999.9999999974\\
92000.0000000014	92000.0000000014\\
-254999.999999999	-254999.999999999\\
198999.999999997	198999.999999997\\
-199000	-199000\\
181000.000000001	181000.000000001\\
-127000.000000003	-127000.000000003\\
-91999.999999997	-91999.999999997\\
366999.999999999	366999.999999999\\
-312000.000000002	-312000.000000002\\
54999.9999999997	54999.9999999997\\
201000	201000\\
111000	111000\\
-386000	-386000\\
-35000.0000000001	-35000.0000000001\\
347000	347000\\
-220000.000000002	-220000.000000002\\
111000.000000002	111000.000000002\\
-295000.000000003	-295000.000000003\\
277000.000000001	277000.000000001\\
144999.999999998	144999.999999998\\
-146000.000000001	-146000.000000001\\
-165000	-165000\\
-181999.999999999	-181999.999999999\\
273000.000000002	273000.000000002\\
109999.999999999	109999.999999999\\
165999.999999999	165999.999999999\\
-422000.000000001	-422000.000000001\\
-17999.9999999998	-17999.9999999998\\
183000	183000\\
256000	256000\\
-238000.000000001	-238000.000000001\\
-181999.999999997	-181999.999999997\\
163999.999999996	163999.999999996\\
73000.0000000004	73000.0000000004\\
-146000	-146000\\
145999.999999998	145999.999999998\\
36999.9999999999	36999.9999999999\\
-91999.9999999987	-91999.9999999987\\
-201000	-201000\\
273999.999999999	273999.999999999\\
-162999.999999997	-162999.999999997\\
-20000.0000000031	-20000.0000000031\\
-54999.9999999997	-54999.9999999997\\
459000.000000001	459000.000000001\\
-258000.000000001	-258000.000000001\\
-346000	-346000\\
565999.999999998	565999.999999998\\
-145999.999999997	-145999.999999997\\
-146000.000000002	-146000.000000002\\
-17999.9999999989	-17999.9999999989\\
-74000.0000000016	-74000.0000000016\\
128000.000000004	128000.000000004\\
-19000.0000000001	-19000.0000000001\\
-291000.000000003	-291000.000000003\\
732000.000000002	732000.000000002\\
-477000.000000001	-477000.000000001\\
-256999.999999998	-256999.999999998\\
276999.999999997	276999.999999997\\
200000	200000\\
-91999.9999999961	-91999.9999999961\\
-54000.0000000003	-54000.0000000003\\
-239000	-239000\\
-72000.0000000001	-72000.0000000001\\
364999.999999998	364999.999999998\\
146999.999999999	146999.999999999\\
-163999.999999998	-163999.999999998\\
-185000.000000003	-185000.000000003\\
112000.000000002	112000.000000002\\
-19000.0000000019	-19000.0000000019\\
-55999.9999999991	-55999.9999999991\\
55999.9999999983	55999.9999999983\\
109000	109000\\
999.999999999446	999.999999999446\\
18000.0000000007	18000.0000000007\\
-147000.000000001	-147000.000000001\\
-55000.0000000006	-55000.0000000006\\
-52999.9999999973	-52999.9999999973\\
381999.999999998	381999.999999998\\
-254999.999999999	-254999.999999999\\
-275999.999999998	-275999.999999998\\
331999.999999999	331999.999999999\\
89999.999999999	89999.999999999\\
-165000.000000003	-165000.000000003\\
-90999.9999999984	-90999.9999999984\\
511999.999999999	511999.999999999\\
-713000.000000002	-713000.000000002\\
310000.000000001	310000.000000001\\
221000.000000003	221000.000000003\\
-293000	-293000\\
126999.999999996	126999.999999996\\
-365999.999999999	-365999.999999999\\
569000.000000002	569000.000000002\\
15999.9999999991	15999.9999999991\\
-454999.999999997	-454999.999999997\\
107999.999999997	107999.999999997\\
201000	201000\\
-125999.999999998	-125999.999999998\\
33999.9999999971	33999.9999999971\\
-125999.999999998	-125999.999999998\\
274000.000000001	274000.000000001\\
-999.999999999446	-999.999999999446\\
-90000.0000000016	-90000.0000000016\\
-239000	-239000\\
109999.999999997	109999.999999997\\
164000.000000001	164000.000000001\\
-363999.999999998	-363999.999999998\\
143999.999999997	143999.999999997\\
442000.000000001	442000.000000001\\
-296000	-296000\\
-179999.999999999	-179999.999999999\\
290999.999999998	290999.999999998\\
-311000.000000001	-311000.000000001\\
458000.000000001	458000.000000001\\
-494999.999999997	-494999.999999997\\
165999.999999997	165999.999999997\\
-111000	-111000\\
276000	276000\\
-73999.9999999998	-73999.9999999998\\
17000.0000000039	17000.0000000039\\
-107000.000000003	-107000.000000003\\
-222999.999999998	-222999.999999998\\
441999.999999998	441999.999999998\\
-219999.999999999	-219999.999999999\\
-240000.000000002	-240000.000000002\\
277000.000000001	277000.000000001\\
144999.999999996	144999.999999996\\
37000.0000000017	37000.0000000017\\
-292000.000000002	-292000.000000002\\
-130000	-130000\\
696999.999999998	696999.999999998\\
-347999.999999999	-347999.999999999\\
-256000.000000001	-256000.000000001\\
163000.000000005	163000.000000005\\
-217000.000000004	-217000.000000004\\
16000.0000000027	16000.0000000027\\
110999.999999995	110999.999999995\\
458000.000000001	458000.000000001\\
-624000	-624000\\
111999.999999999	111999.999999999\\
345999.999999998	345999.999999998\\
-72999.9999999977	-72999.9999999977\\
999.999999999446	999.999999999446\\
-367000.000000002	-367000.000000002\\
38000.000000002	38000.000000002\\
180999.999999996	180999.999999996\\
349000	349000\\
-220000.000000001	-220000.000000001\\
-457000	-457000\\
364999.999999997	364999.999999997\\
239000.000000002	239000.000000002\\
-311000	-311000\\
-367999.999999999	-367999.999999999\\
75000.000000001	75000.000000001\\
639999.999999999	639999.999999999\\
-53999.9999999985	-53999.9999999985\\
-129000.000000002	-129000.000000002\\
-311999.999999996	-311999.999999996\\
550999.999999997	550999.999999997\\
-732999.999999996	-732999.999999996\\
309999.999999999	309999.999999999\\
-52999.9999999999	-52999.9999999999\\
-999.999999999446	-999.999999999446\\
72999.9999999968	72999.9999999968\\
166000.000000002	166000.000000002\\
-38000.000000002	-38000.000000002\\
-91000.0000000011	-91000.0000000011\\
36999.9999999981	36999.9999999981\\
-999.999999998557	-999.999999998557\\
-36000.0000000022	-36000.0000000022\\
-53999.9999999994	-53999.9999999994\\
-185000	-185000\\
440999.999999999	440999.999999999\\
164000.000000001	164000.000000001\\
-293000.000000003	-293000.000000003\\
-658999.999999999	-658999.999999999\\
623000	623000\\
165000	165000\\
-20000.0000000031	-20000.0000000031\\
-272999.999999998	-272999.999999998\\
-276000.000000002	-276000.000000002\\
331000.000000006	331000.000000006\\
255999.999999998	255999.999999998\\
-237999.999999999	-237999.999999999\\
72000.0000000001	72000.0000000001\\
111000.000000002	111000.000000002\\
-274000.000000001	-274000.000000001\\
418999.999999998	418999.999999998\\
-418999.999999998	-418999.999999998\\
-184000.000000004	-184000.000000004\\
256000	256000\\
110000.000000001	110000.000000001\\
-71999.9999999992	-71999.9999999992\\
180999.999999996	180999.999999996\\
-364999.999999998	-364999.999999998\\
220000.000000001	220000.000000001\\
182999.999999999	182999.999999999\\
-202000	-202000\\
-73000.0000000022	-73000.0000000022\\
-200999.999999998	-200999.999999998\\
366000	366000\\
72000.0000000027	72000.0000000027\\
-511000.000000001	-511000.000000001\\
92000.0000000014	92000.0000000014\\
619999.999999999	619999.999999999\\
-326999.999999999	-326999.999999999\\
-258000	-258000\\
18999.9999999975	18999.9999999975\\
513000.000000001	513000.000000001\\
-73999.9999999998	-73999.9999999998\\
-274999.999999998	-274999.999999998\\
-456000.000000001	-456000.000000001\\
255000	255000\\
365999.999999998	365999.999999998\\
-145999.999999996	-145999.999999996\\
19000.0000000001	19000.0000000001\\
182000.000000002	182000.000000002\\
18999.9999999992	18999.9999999992\\
-312000	-312000\\
-126999.999999998	-126999.999999998\\
401000	401000\\
-181000	-181000\\
-56000.0000000009	-56000.0000000009\\
18000.0000000016	18000.0000000016\\
54999.9999999997	54999.9999999997\\
-145999.999999999	-145999.999999999\\
403000.000000002	403000.000000002\\
-74000.0000000007	-74000.0000000007\\
-109000	-109000\\
-257000.000000001	-257000.000000001\\
-36000.0000000022	-36000.0000000022\\
164000	164000\\
54999.9999999988	54999.9999999988\\
-236999.999999998	-236999.999999998\\
419999.999999998	419999.999999998\\
-183000.000000001	-183000.000000001\\
-145999.999999999	-145999.999999999\\
403000	403000\\
-385000.000000001	-385000.000000001\\
-73000.0000000022	-73000.0000000022\\
183000.000000001	183000.000000001\\
311000	311000\\
-256000	-256000\\
-330000	-330000\\
330000	330000\\
-512000.000000002	-512000.000000002\\
566000.000000002	566000.000000002\\
203000.000000002	203000.000000002\\
-367000.000000002	-367000.000000002\\
18000.0000000016	18000.0000000016\\
-54999.9999999997	-54999.9999999997\\
257999.999999999	257999.999999999\\
108000.000000001	108000.000000001\\
-236999.999999999	-236999.999999999\\
53999.9999999958	53999.9999999958\\
-329000	-329000\\
91999.9999999996	91999.9999999996\\
71999.9999999983	71999.9999999983\\
258000	258000\\
-185000	-185000\\
-108000	-108000\\
53999.9999999985	53999.9999999985\\
73000.0000000013	73000.0000000013\\
-55000.0000000015	-55000.0000000015\\
166000.000000002	166000.000000002\\
162999.999999998	162999.999999998\\
-401000.000000002	-401000.000000002\\
72000.0000000001	72000.0000000001\\
-108999.999999999	-108999.999999999\\
89999.9999999998	89999.9999999998\\
-144000	-144000\\
346000.000000001	346000.000000001\\
-164999.999999998	-164999.999999998\\
-127000.000000005	-127000.000000005\\
421000.000000002	421000.000000002\\
-532000.000000002	-532000.000000002\\
129000	129000\\
494000.000000001	494000.000000001\\
-310999.999999997	-310999.999999997\\
-219000.000000001	-219000.000000001\\
162999.999999999	162999.999999999\\
240000.000000003	240000.000000003\\
-258000.000000003	-258000.000000003\\
92000.0000000005	92000.0000000005\\
-181999.999999999	-181999.999999999\\
-313000.000000002	-313000.000000002\\
331000.000000002	331000.000000002\\
274000.000000002	274000.000000002\\
73999.9999999989	73999.9999999989\\
-185000.000000002	-185000.000000002\\
57000.0000000039	57000.0000000039\\
-313000.000000004	-313000.000000004\\
2000.00000000422	2000.00000000422\\
547999.999999998	547999.999999998\\
-385000	-385000\\
-72000.0000000018	-72000.0000000018\\
-219999.999999999	-219999.999999999\\
437999.999999999	437999.999999999\\
222000.000000001	222000.000000001\\
-514000	-514000\\
200999.999999998	200999.999999998\\
-309999.999999999	-309999.999999999\\
164000.000000001	164000.000000001\\
274999.999999999	274999.999999999\\
-999.999999997669	-999.999999997669\\
-126000.000000001	-126000.000000001\\
-167000.000000002	-167000.000000002\\
201999.999999999	201999.999999999\\
36999.9999999981	36999.9999999981\\
-146999.999999999	-146999.999999999\\
1000.00000000033	1000.00000000033\\
72999.9999999995	72999.9999999995\\
-165999.999999999	-165999.999999999\\
-72000.0000000018	-72000.0000000018\\
145999.999999998	145999.999999998\\
238000.000000001	238000.000000001\\
36999.999999999	36999.999999999\\
-312000.000000002	-312000.000000002\\
111000.000000003	111000.000000003\\
-148000	-148000\\
37999.9999999976	37999.9999999976\\
146000	146000\\
18000.0000000007	18000.0000000007\\
109999.999999999	109999.999999999\\
-603999.999999999	-603999.999999999\\
401999.999999999	401999.999999999\\
331000.000000001	331000.000000001\\
35999.9999999996	35999.9999999996\\
-658999.999999999	-658999.999999999\\
71999.9999999992	71999.9999999992\\
313000	313000\\
53999.9999999976	53999.9999999976\\
-54999.9999999988	-54999.9999999988\\
-311000.000000002	-311000.000000002\\
92000.0000000005	92000.0000000005\\
529999.999999999	529999.999999999\\
-146000	-146000\\
-512000.000000001	-512000.000000001\\
-184999.999999996	-184999.999999996\\
624999.999999998	624999.999999998\\
199999.999999998	199999.999999998\\
-403000.000000002	-403000.000000002\\
-328999.999999998	-328999.999999998\\
365999.999999998	365999.999999998\\
220000.000000001	220000.000000001\\
-349000.000000001	-349000.000000001\\
93000.0000000035	93000.0000000035\\
347000	347000\\
-292000.000000001	-292000.000000001\\
-294000.000000003	-294000.000000003\\
512000.000000001	512000.000000001\\
-381999.999999999	-381999.999999999\\
15999.9999999973	15999.9999999973\\
367000.000000002	367000.000000002\\
19000.000000001	19000.000000001\\
-715000.000000003	-715000.000000003\\
238000	238000\\
421000	421000\\
38000.0000000029	38000.0000000029\\
-221000.000000004	-221000.000000004\\
-219999.999999997	-219999.999999997\\
385999.999999999	385999.999999999\\
-74999.9999999993	-74999.9999999993\\
-182000.000000001	-182000.000000001\\
-109999.999999999	-109999.999999999\\
202000.000000002	202000.000000002\\
34999.9999999966	34999.9999999966\\
-108999.999999997	-108999.999999997\\
184000	184000\\
-184000.000000001	-184000.000000001\\
-111000	-111000\\
442000	442000\\
-258000.000000002	-258000.000000002\\
-73000.0000000013	-73000.0000000013\\
109000.000000001	109000.000000001\\
-199000.000000002	-199000.000000002\\
-75999.9999999987	-75999.9999999987\\
387000.000000001	387000.000000001\\
-259000.000000002	-259000.000000002\\
39000.000000005	39000.000000005\\
109000	109000\\
110000	110000\\
-92000.0000000049	-92000.0000000049\\
146000.000000002	146000.000000002\\
-547999.999999999	-547999.999999999\\
36000.0000000005	36000.0000000005\\
640999.999999998	640999.999999998\\
-91999.9999999987	-91999.9999999987\\
-713000.000000001	-713000.000000001\\
346999.999999998	346999.999999998\\
292000.000000002	292000.000000002\\
-271999.999999999	-271999.999999999\\
15999.9999999973	15999.9999999973\\
367000.000000002	367000.000000002\\
-220000	-220000\\
-256000	-256000\\
384000	384000\\
-163999.999999998	-163999.999999998\\
-238000	-238000\\
108999.999999999	108999.999999999\\
147000.000000003	147000.000000003\\
18999.9999999983	18999.9999999983\\
89999.9999999972	89999.9999999972\\
-201000	-201000\\
1000.00000000033	1000.00000000033\\
-129000.000000002	-129000.000000002\\
219999.999999999	219999.999999999\\
17999.9999999998	17999.9999999998\\
-364999.999999999	-364999.999999999\\
382000.000000001	382000.000000001\\
131000	131000\\
-185000	-185000\\
-274000	-274000\\
512999.999999998	512999.999999998\\
-165000.000000001	-165000.000000001\\
-330000	-330000\\
55000.0000000006	55000.0000000006\\
292999.999999997	292999.999999997\\
-35999.9999999978	-35999.9999999978\\
90999.9999999993	90999.9999999993\\
-493999.999999997	-493999.999999997\\
127000.000000001	127000.000000001\\
74999.9999999966	74999.9999999966\\
273000.000000001	273000.000000001\\
-199999.999999997	-199999.999999997\\
72999.9999999995	72999.9999999995\\
328999.999999998	328999.999999998\\
-367000	-367000\\
-510999.999999999	-510999.999999999\\
494000.000000002	494000.000000002\\
366000	366000\\
-348999.999999999	-348999.999999999\\
-199000.000000001	-199000.000000001\\
364000.000000001	364000.000000001\\
-55000.0000000006	-55000.0000000006\\
-70999.9999999979	-70999.9999999979\\
-149000.000000001	-149000.000000001\\
-34000.0000000007	-34000.0000000007\\
362999.999999999	362999.999999999\\
-197999.999999999	-197999.999999999\\
-351000	-351000\\
368000.000000001	368000.000000001\\
74000.0000000007	74000.0000000007\\
326999.999999999	326999.999999999\\
-419000	-419000\\
-678000.000000002	-678000.000000002\\
915000.000000001	915000.000000001\\
90999.9999999993	90999.9999999993\\
-491999.999999997	-491999.999999997\\
-112000.000000002	-112000.000000002\\
73999.9999999998	73999.9999999998\\
127999.999999998	127999.999999998\\
385000.000000002	385000.000000002\\
-203000.000000001	-203000.000000001\\
-125999.999999999	-125999.999999999\\
89999.9999999981	89999.9999999981\\
-327999.999999999	-327999.999999999\\
236000.000000002	236000.000000002\\
-54000.0000000029	-54000.0000000029\\
-146000	-146000\\
403000.000000002	403000.000000002\\
71999.9999999965	71999.9999999965\\
-383999.999999998	-383999.999999998\\
-73000.0000000004	-73000.0000000004\\
147000	147000\\
273000.000000003	273000.000000003\\
-493000.000000002	-493000.000000002\\
420999.999999998	420999.999999998\\
-72999.9999999986	-72999.9999999986\\
-459000	-459000\\
330000	330000\\
19999.9999999996	19999.9999999996\\
-20000.0000000013	-20000.0000000013\\
292999.999999999	292999.999999999\\
109999.999999999	109999.999999999\\
-402000	-402000\\
-185000	-185000\\
386999.999999999	386999.999999999\\
-221000	-221000\\
-311000	-311000\\
457000.000000002	457000.000000002\\
36999.9999999972	36999.9999999972\\
-182999.999999999	-182999.999999999\\
1000.00000000122	1000.00000000122\\
89999.9999999954	89999.9999999954\\
-17999.9999999971	-17999.9999999971\\
-90999.9999999993	-90999.9999999993\\
147000.000000001	147000.000000001\\
-295000.000000002	-295000.000000002\\
698000.000000001	698000.000000001\\
-312000	-312000\\
-404000.000000003	-404000.000000003\\
-107999.999999999	-107999.999999999\\
547999.999999998	547999.999999998\\
-110000	-110000\\
75000.000000001	75000.000000001\\
-404999.999999997	-404999.999999997\\
293999.999999998	293999.999999998\\
91999.9999999996	91999.9999999996\\
-366999.999999997	-366999.999999997\\
0	0\\
274999.999999999	274999.999999999\\
-128000.000000001	-128000.000000001\\
37000.0000000008	37000.0000000008\\
327999.999999998	327999.999999998\\
110999.999999997	110999.999999997\\
-951999.999999999	-951999.999999999\\
566999.999999999	566999.999999999\\
311999.999999998	311999.999999998\\
-550000	-550000\\
-73000.0000000013	-73000.0000000013\\
513000.000000002	513000.000000002\\
-385000.000000003	-385000.000000003\\
183000.000000002	183000.000000002\\
385000	385000\\
-457999.999999999	-457999.999999999\\
0	0\\
-71999.9999999974	-71999.9999999974\\
-21000.0000000043	-21000.0000000043\\
460000.000000002	460000.000000002\\
-238999.999999999	-238999.999999999\\
};
\end{axis}

\begin{axis}[%
width=4.927cm,
height=3.484cm,
at={(0cm,4.839cm)},
scale only axis,
xmin=-1000000,
xmax=1000000,
xlabel style={font=\color{white!15!black}},
xlabel={$\delta^3 u(t)$},
ymin=-51757700000,
ymax=66162100000,
ylabel style={font=\color{white!15!black}},
ylabel={y(t)},
axis background/.style={fill=white},
title={C6, R = 0.5352},
axis x line*=bottom,
axis y line*=left
]
\addplot[only marks, mark=*, mark options={}, mark size=1.5000pt, color=mycolor1, fill=mycolor1] table[row sep=crcr]{%
x	y\\
-220000	5005000000\\
310999.999999996	16357400000\\
-201000	-22583000000\\
200000.000000002	22827200000\\
130999.999999999	-17700300000\\
-534000	-8544899999.99999\\
221999.999999999	26611400000\\
16999.9999999986	-18432700000\\
147000.000000002	13183600000\\
182999.999999997	-8544800000.00001\\
-347999.999999999	-8300900000\\
-91999.9999999978	13793900000\\
-346000.000000002	-10497900000\\
437000.000000002	11596500000\\
330999.999999996	854799999.999997\\
-165999.999999999	-12207500000\\
2000.00000000067	11108900000\\
35000.0000000001	-11230800000\\
-622000.000000003	-7690300000\\
750000.000000003	40038900000\\
-127000.000000001	-41747800000\\
-512999.999999999	8300700000\\
548999.999999999	23559400000\\
-109000	-24658000000\\
-331000.000000001	-976599999.999992\\
384999.999999998	25634700000\\
55000.0000000015	-21362300000\\
-439000.000000002	-8178600000\\
329000	28198100000\\
72999.9999999995	-17578100000\\
-145000	-2441400000.00002\\
17000.0000000003	5981700000.00001\\
-128000.000000002	-6348200000\\
17999.9999999998	8789700000\\
183999.999999999	853900000.000003\\
-110000.000000001	-9643200000\\
-999.999999998557	6591799999.99999\\
20000.0000000005	-2563599999.99999\\
-112000.000000003	-3418000000.00001\\
240000.000000002	17944500000\\
53999.9999999985	-17700400000\\
-275000.000000002	-5493000000.00001\\
238000.000000002	19043000000\\
-273000.000000001	-19043200000\\
-56999.9999999995	13061700000\\
330999.999999999	3662200000.00001\\
-165000	-14526600000\\
-129000.000000001	7080200000.00001\\
146999.999999998	4882800000\\
110000.000000002	-3540000000\\
-293000	-11474700000\\
385000	30639800000\\
16999.9999999977	-31738500000\\
-311000	7080200000\\
-90999.9999999993	4394699999.99999\\
147000	4760400000.00001\\
291999.999999997	1342900000\\
-328999.999999999	-18310300000\\
-165000.000000002	16479200000\\
403000.000000001	1342800000\\
-110999.999999997	-10009600000\\
-218000.000000003	2319100000\\
164000.000000001	5615499999.99999\\
-109999.999999999	-8056800000\\
200999.999999999	14770500000\\
-72000.0000000009	-20385700000\\
54000.0000000003	19409200000\\
-146000	-18310600000\\
199999.999999998	18920900000\\
-161999.999999996	-17944300000\\
70999.9999999979	11840900000\\
-55000.0000000024	-5981599999.99999\\
-16999.9999999968	3051799999.99999\\
-202000	-8789000000\\
183000	16723500000\\
238000.000000001	-1464599999.99999\\
-72999.9999999986	-12329400000\\
73999.9999999989	12329400000\\
-442000.000000002	-33569700000\\
149000.000000003	43457500000\\
236999.999999999	-12207500000\\
19000.0000000028	-732100000.000008\\
-312000.000000004	-28564600000\\
18000.0000000025	38086000000\\
422000.000000001	8422800000\\
-201999.999999998	-48950100000\\
18999.9999999992	36010600000\\
-111000	-16235300000\\
-72000.0000000018	9643600000\\
-18999.9999999983	-5493200000\\
202000	13916200000\\
34999.9999999966	-17822700000\\
-198999.999999996	4272900000.00001\\
-112000.000000004	3905900000\\
56000.0000000009	-2318900000\\
257000.000000001	9032700000.00001\\
-93000.0000000026	-14892200000\\
-146000.000000001	5126800000.00001\\
165000.000000001	7812400000\\
-165000.000000001	-13915800000\\
275000.000000001	22216800000\\
-201999.999999999	-32104800000\\
148000	36011000000\\
-75000.0000000028	-32592700000\\
-201000	14526100000\\
329999.999999999	10009900000\\
-202000.000000003	-19897300000\\
-72999.9999999977	9643300000\\
111000.000000001	5493200000\\
162999.999999998	-3784000000\\
-274000.000000001	-10376200000\\
202000.000000002	17212100000\\
-184000.000000003	-16113300000\\
56000.0000000018	11840600000\\
163999.999999998	-1830800000\\
-220000	-9643700000.00001\\
-17999.9999999989	11474800000\\
-17000.0000000012	-8179000000\\
254000	13427900000\\
-89999.9999999998	-16845600000\\
201000	18188200000\\
-457000.000000001	-31616000000\\
-1999.99999999889	29541000000\\
515000	6103499999.99999\\
-367999.999999999	-34546000000\\
75000.0000000002	29297100000\\
235999.999999997	-3296299999.99998\\
-235999.999999998	-19042500000\\
-240000.000000003	10131600000\\
202999.999999999	10986300000\\
-129000	-16601500000\\
201000	18188400000\\
111000.000000002	-8666800000.00001\\
18000.0000000033	732300000.000004\\
-238000.000000002	-16235500000\\
54000.0000000003	24170200000\\
19000.0000000028	-14526600000\\
999.999999997669	6103600000.00001\\
144999.999999997	3906399999.99999\\
-182999.999999999	-14892800000\\
-36000.0000000005	13305800000\\
72999.9999999977	-3784300000\\
-54999.9999999988	-1831000000\\
-73999.9999999998	100000.00000332\\
2000.00000000422	2685500000\\
400999.999999998	12939200000\\
-107999.999999998	-24413600000\\
-259000.000000004	6225200000\\
56999.9999999995	6347800000\\
182000	2563500000.00001\\
-144999.999999999	-10498100000\\
-148000.000000001	6714000000\\
-109999.999999999	-2807900000\\
458999.999999998	12207300000\\
-202000.000000001	-20263700000\\
-18000.0000000016	13793800000\\
54000.0000000011	-2929600000\\
37999.9999999976	-99999.9999962142\\
-999.999999998557	-3783900000\\
-73999.9999999989	2807300000\\
-254000	-2075000000\\
-20000.0000000005	1342700000\\
329000	6103500000.00001\\
112000.000000003	-3418000000\\
71999.9999999992	3051900000.00002\\
-129000.000000004	-14038200000\\
-144999.999999998	8545000000.00001\\
89999.9999999981	1830900000\\
-456000	-8178600000\\
511999.999999999	20752000000\\
-165000	-25390600000\\
-17999.9999999998	18554400000\\
-56000.0000000018	-16235000000\\
203000.000000002	23681500000\\
218000.000000002	-11474700000\\
-291000	-17089600000\\
145000	26488900000\\
-274000	-31615900000\\
0	32348500000\\
181999.999999999	-15380700000\\
57000.0000000022	10131500000\\
-111999.999999999	-18798600000\\
-89999.999999999	13549900000\\
146000.000000001	-299999.99999859\\
-146000	-7812100000\\
217999.999999997	16601200000\\
-382999.999999998	-31860300000\\
275000.000000001	38330400000\\
199999.999999999	-12695700000\\
-53999.9999999976	-13671700000\\
-329000.000000002	-366300000.000012\\
-130000.000000002	9033400000.00001\\
349000	11474400000\\
-17999.9999999989	-16967700000\\
-1000.00000000033	10742300000\\
20000.0000000031	-7812700000.00001\\
89999.9999999963	8178800000\\
-182999.999999998	-18188500000\\
-73000.0000000004	12207100000\\
56000.0000000009	4638699999.99999\\
308999.999999999	11108300000\\
-125999.999999997	-33935600000\\
-477000	10986400000\\
458000	26611500000\\
18000.0000000016	-25757200000\\
-310999.999999999	-1952699999.99999\\
110000	17699800000\\
218999.999999998	-7079900000\\
-108999.999999997	-6713799999.99999\\
-183000.000000002	854500000.000002\\
217999.999999997	12939200000\\
-180999.999999997	-19409000000\\
-19000.0000000037	14770600000\\
347000.000000001	2685500000\\
-108999.999999999	-10010000000\\
-201000.000000001	-5370699999.99999\\
-93000	7567900000.00001\\
19999.9999999996	1343300000\\
511999.999999999	21972100000\\
-147000.000000002	-41503400000\\
-256000	16235000000\\
-110000	1098800000\\
146999.999999998	9033100000\\
146000	-9033100000.00001\\
-218999.999999997	-3051700000\\
-19000.0000000046	7201900000\\
91000.0000000011	-3295800000\\
37999.9999999985	3296200000.00001\\
182000	1708500000\\
-494000.000000001	-21484100000\\
312000	32714900000\\
144999.999999999	-16845900000\\
-145000	244199999.999988\\
-38000.0000000011	-5248799999.99998\\
-145999.999999997	5980999999.99999\\
458999.999999999	17456600000\\
-56999.9999999986	-30640200000\\
-584000.000000002	4272900000\\
254999.999999999	20507600000\\
18999.9999999983	-15747100000\\
201000	12329300000\\
-17999.9999999998	-12329400000\\
-330000.000000003	-3539700000\\
293000.000000002	20019200000\\
73999.9999999998	-14404000000\\
-274999.999999999	-5981600000\\
163999.999999998	17456000000\\
-16999.9999999977	-12939300000\\
-93000.0000000044	1586700000\\
276000.000000003	10132100000\\
-349000.000000002	-18798900000\\
220999.999999998	19653100000\\
-128999.999999999	-14648300000\\
-220999.999999997	3174100000\\
405999.999999999	11718300000\\
-204000.000000001	-15868900000\\
54999.9999999997	9033100000\\
313000.000000001	11474900000\\
-130000.000000001	-28320800000\\
-219000.000000001	12329500000\\
-108999.999999999	4516500000\\
34999.9999999993	-3418100000\\
350000	10864400000\\
-314000.000000001	-21484400000\\
-34000.0000000007	17211900000\\
-38999.9999999997	-10498100000\\
368000.000000001	18310500000\\
-292999.999999997	-28808300000\\
346999.999999998	35400100000\\
-293000.000000002	-37963800000\\
219999.999999999	29419100000\\
-309999.999999996	-23315800000\\
161999.999999998	22217200000\\
-144000.000000002	-17578300000\\
36000.0000000005	12207000000\\
-129000.000000002	-9399300000\\
312000	12451000000\\
-72999.9999999986	-12939300000\\
-37000.0000000035	7446300000\\
-164999.999999999	-10254100000\\
310999.999999999	24902600000\\
75000.0000000037	-25513000000\\
-369000.000000004	122400000.00001\\
112000.000000003	16845600000\\
-37000.0000000008	-14038300000\\
256000	19897700000\\
-54999.9999999997	-24169900000\\
0	9032900000\\
-181999.999999999	-2441000000\\
-112000.000000002	5614900000.00001\\
39000.0000000015	-1830800000\\
126000	1342500000\\
74999.9999999984	-2074900000\\
-184999.999999998	-2685700000\\
-34000.0000000025	3417900000\\
309000	6103700000\\
-91000.0000000019	-12817600000\\
-163999.999999998	6714100000.00001\\
309999.999999997	6835699999.99999\\
-89999.999999999	-15258600000\\
-569000.000000001	1708900000\\
349000.000000001	14160200000\\
309999.999999997	-3662200000\\
-236999.999999996	-10375900000\\
72999.9999999986	11962900000\\
73000.0000000004	-9399500000\\
-329000.000000001	-3661899999.99999\\
73000.0000000004	12450900000\\
273999.999999997	976700000.000001\\
18999.9999999983	-5981500000\\
-457999.999999999	-19164900000\\
493999.999999999	46630800000\\
-163999.999999999	-46509000000\\
-165000.000000004	21118400000\\
201000	2563500000\\
17999.9999999998	-3174000000\\
-108999.999999999	-8056599999.99999\\
-218999.999999998	-1220500000\\
143999.999999998	18188200000\\
387000.000000001	2319400000\\
-93000.0000000008	-25146300000\\
-346999.999999999	3539899999.99999\\
71999.9999999956	15502800000\\
-16999.9999999995	-10131500000\\
90000.0000000016	8056199999.99999\\
149000	1221200000.00001\\
-204000.000000001	-18677200000\\
-72000.0000000018	14892800000\\
183000.000000002	3906300000.00001\\
146999.999999998	-200000.00002085\\
-221000	-19775100000\\
-182000	12939100000\\
53999.9999999976	854700000\\
312000	16479500000\\
-109999.999999998	-30883700000\\
146000	27099200000\\
-274000.000000001	-30517100000\\
-54999.9999999962	25756700000\\
330000	-5005199999.99999\\
-148000.000000006	-7689900000\\
-126999.999999998	4760300000\\
17999.9999999989	1098900000\\
-311000.000000001	-6347800000\\
365999.999999999	12329100000\\
110000.000000002	-8300600000\\
72999.9999999995	6469500000.00001\\
-238000.000000003	-16601500000\\
0	16479600000\\
257000.000000002	610300000.000007\\
-294000.000000003	-17334200000\\
37000.0000000008	18311000000\\
165000.000000001	-6958500000\\
-274999.999999997	-5004500000\\
35999.9999999996	6591600000\\
130000	1953200000\\
310999.999999999	5248900000\\
-313000.000000001	-22216600000\\
-16999.9999999995	16723400000\\
-110999.999999999	-5492900000\\
257999.999999999	9887500000\\
-495000.000000002	-22216800000\\
274000	24170100000\\
421999.999999999	2075099999.99998\\
-421999.999999999	-31006000000\\
54999.9999999988	30517700000\\
129000.000000004	-11962800000\\
54999.9999999988	6469500000\\
-74000.0000000007	-14770200000\\
-165000.000000002	5981200000.00001\\
18999.9999999983	8056599999.99999\\
17999.9999999998	-7568000000\\
37000.0000000017	5004399999.99999\\
71999.9999999983	1221000000\\
-180999.999999997	-13549800000\\
144999.999999999	21118100000\\
-36999.9999999999	-18798900000\\
294000.000000002	27954200000\\
-384000	-50293000000\\
-2000.00000000067	42602500000\\
313000.000000001	-3295700000.00001\\
-348999.999999998	-27344100000\\
239000	34790400000\\
110000	-19287400000\\
-166000.000000004	-6957900000\\
-238000.000000001	9887799999.99998\\
-34999.9999999975	732300000.000009\\
400999.999999998	7934700000\\
-17000.0000000012	-13428000000\\
-330000.000000001	-2807400000\\
274000.000000002	18310500000\\
999.999999999446	-17578000000\\
-293999.999999997	4638400000\\
220999.999999997	5737500000\\
72000.0000000018	-1709100000\\
-311000.000000002	-12695200000\\
148000.000000003	19531300000\\
363999.999999995	-3296100000\\
-547999.999999998	-24414000000\\
127999.999999998	30761900000\\
383999.999999999	-3906499999.99999\\
-330000.000000001	-25024300000\\
313000.000000005	39795000000\\
-312000.000000002	-48950400000\\
164000	46020800000\\
73999.9999999989	-31372400000\\
-458000.000000002	6469900000.00001\\
330000	16479600000\\
89999.999999999	-15625200000\\
-344999.999999998	-1708800000\\
435999.999999998	18310200000\\
-144000.000000002	-22338400000\\
-257000	9521200000\\
-165000	-1342800000\\
439999.999999997	8789299999.99999\\
-37999.9999999976	-10498400000\\
39000.0000000024	7446700000\\
-39000.0000000041	-7690700000.00001\\
-91000.0000000002	610300000.000004\\
111000.000000001	11230700000\\
-2000.00000000067	-14404500000\\
-161999.999999998	3662300000\\
125999.999999998	8178600000.00001\\
-219999.999999999	-16479600000\\
405000.000000003	29541100000\\
-185000.000000004	-32958700000\\
-274000	11962300000\\
165000.000000002	7446900000\\
329000	3661700000\\
-218999.999999998	-18676600000\\
-73000.0000000022	9399399999.99999\\
-38000.0000000011	-488399999.999993\\
164999.999999999	7446600000\\
-71999.9999999992	-13794300000\\
-221000.000000002	7568500000\\
185000.000000003	2807800000\\
70999.9999999979	-2930100000\\
18999.9999999983	3296300000\\
-127999.999999998	-13794100000\\
73999.9999999998	20751900000\\
236999.999999997	-10986300000\\
-438999.999999998	-9765599999.99999\\
-147000.000000003	10742200000\\
367000.000000003	6347599999.99999\\
164999.999999997	-2075099999.99999\\
-129999.999999999	-8056700000\\
-89000.0000000022	-1708999999.99999\\
107000.000000002	10742200000\\
-143999.999999998	-12329100000\\
72999.9999999977	11596600000\\
-167000.000000001	-12695000000\\
204000.000000001	18310000000\\
144999.999999998	-11474000000\\
-475000	-11230900000\\
310000	22094900000\\
-73000.0000000013	-12085000000\\
56000.0000000027	3295799999.99999\\
35999.999999996	-488000000.000002\\
-183999.999999997	-8301100000\\
92999.9999999991	14892800000\\
54000.0000000011	-7690499999.99999\\
312000.000000001	14160000000\\
-184000.000000001	-31371700000\\
-164000.000000001	15746700000\\
-17999.9999999971	5005000000.00001\\
-276000.000000002	-6591700000\\
385000	11474600000\\
-292000.000000002	-17334200000\\
383000.000000001	22583300000\\
-89999.9999999998	-21362500000\\
-203000.000000001	5004900000\\
129000	7934700000\\
-16999.9999999977	-6469800000\\
51999.9999999987	3784099999.99999\\
-179999.999999999	-10253900000\\
308999.999999997	24780400000\\
19000.0000000028	-24169900000\\
-128000.000000001	4760499999.99999\\
-164999.999999999	-3051500000.00002\\
36999.999999999	11108300000\\
-18000.0000000007	-8300900000\\
273000	14404500000\\
-199999.999999999	-25146500000\\
-54999.999999997	20141300000\\
347999.999999999	-3295500000.00001\\
-641000.000000003	-16723800000\\
236999.999999999	22216700000\\
403999.999999999	1464999999.99999\\
-293999.999999999	-25024500000\\
-310000	11108500000\\
641000.000000001	28564200000\\
-276000.000000003	-45654000000\\
-237999.999999999	19409000000\\
294000.000000003	15869300000\\
164999.999999998	-15747300000\\
-240000.000000001	-8300500000\\
-162000	8544599999.99999\\
-112000.000000001	-610099999.999991\\
146999.999999998	4882700000\\
293000.000000001	7812500000\\
-109999.999999998	-18554600000\\
56000	13427700000\\
16999.9999999959	-10986400000\\
-273999.999999998	-1342800000.00001\\
181999.999999998	13061800000\\
-125999.999999997	-10864700000\\
70999.9999999962	8057000000\\
-180999.999999996	-8789100000\\
199999.999999996	10864000000\\
219000.000000001	-1830800000\\
-455999.999999998	-17334000000\\
109999.999999999	20385600000\\
180999.999999997	-4760700000\\
-125999.999999998	-6347500000.00001\\
34999.9999999984	5493000000\\
-17999.9999999989	-854499999.999987\\
349000.000000003	12817400000\\
-312000.000000001	-32592600000\\
-127999.999999998	21850300000\\
181999.999999999	3296200000\\
93999.9999999985	-5737600000.00002\\
-314000	-10986100000\\
-90000.0000000025	12206900000\\
366000.000000002	10254000000\\
219000	-3540199999.99998\\
-237000.000000002	-26122800000\\
-147000.000000002	18676600000\\
183000	2563400000\\
-383999.999999999	-6225400000\\
-18999.9999999992	3906100000\\
348999.999999999	3173900000\\
-148000.000000001	-7934500000\\
239000	10497900000\\
-37000.0000000008	-11962800000\\
-236999.999999999	4150300000\\
126000.000000002	2441600000\\
-53000.0000000035	-732700000.000003\\
17999.9999999998	-2807300000\\
-110999.999999999	1342500000\\
37999.9999999985	2441500000\\
219000.000000003	3173899999.99999\\
-53999.9999999994	-7446299999.99999\\
-111000.000000004	-1465000000.00001\\
-54999.9999999988	5737400000\\
164999.999999998	4150500000.00001\\
-90999.9999999966	-14526500000\\
110000.000000002	20141600000\\
35999.9999999987	-14160000000\\
-55000.0000000015	-1709200000.00001\\
-164000.000000001	-488199999.999995\\
-202999.999999999	1831099999.99999\\
368000.000000002	16967800000\\
181999.999999998	-8667200000.00001\\
-181999.999999998	-16601200000\\
70999.9999999979	12206700000\\
-199000.000000001	-4272300000\\
-184000.000000002	5981300000\\
385000.000000002	1098900000\\
-495999.999999999	-10742400000\\
294999.999999999	13671800000\\
199999.999999998	-5859000000\\
-73000.0000000013	-855000000.000002\\
92000.0000000005	610799999.999997\\
-200999.999999999	-4150700000\\
-147999.999999999	2685700000\\
165999.999999999	5371100000\\
37000.0000000017	-5859500000\\
90000.0000000016	3906500000\\
-91000.0000000019	-4883200000\\
-272999.999999999	-1098200000\\
-56999.9999999995	1464500000\\
494999.999999996	15503100000\\
36000.0000000022	-20385700000\\
-52999.999999999	16845500000\\
-367999.999999999	-32470500000\\
73999.9999999989	33569200000\\
311000.000000003	2685699999.99999\\
-146000.000000001	-29175100000\\
90999.9999999984	28442900000\\
-219999.999999998	-28931300000\\
73999.9999999981	22949800000\\
164000.000000001	2196799999.99998\\
-181999.999999999	-24047600000\\
146000.000000001	26367300000\\
-165000.000000004	-20752400000\\
-330000.000000001	7690999999.99999\\
678000.000000003	18310100000\\
-329000.000000002	-31982200000\\
-56000.0000000009	20629700000\\
146999.999999998	-3661800000\\
-36999.9999999999	-3906500000\\
-72000.0000000001	2563400000\\
34999.9999999993	-610000000.000009\\
36999.999999999	1708700000\\
56000	1464900000.00001\\
-1000.00000000033	-5004900000.00001\\
-238000.000000003	-5004900000\\
458000.000000003	22461200000\\
-128000.000000002	-27710300000\\
-512999.999999995	10742200000\\
90999.9999999975	5981800000\\
624000	2685100000\\
-220999.999999997	-14648100000\\
-512000	4882600000\\
165000	6103700000\\
329000	-488499999.999999\\
-147000	-5370800000.00001\\
165999.999999998	7079600000\\
-366000	-15502300000\\
383000	25634200000\\
73999.9999999989	-20019200000\\
-548999.999999999	-5004900000\\
640000	28442100000\\
-384000.000000002	-31859900000\\
74000.0000000016	18065900000\\
18000.0000000007	-4760200000\\
-331000	-3174300000\\
459000.000000001	11230700000\\
-402000.000000001	-15380900000\\
108000.000000001	10376000000\\
348000.000000001	4028300000\\
-53000.0000000026	-9155300000\\
-258000	-3784199999.99999\\
165000	13427900000\\
-237000	-15259000000\\
346000.000000004	18554700000\\
-200000.000000005	-17944100000\\
-35999.9999999969	9887500000\\
108000.000000002	-1464800000\\
-127000	-4638800000\\
91000.0000000002	8545100000\\
-237000.000000001	-11230400000\\
382999.999999998	15624700000\\
-54000.0000000003	-12939200000\\
-128000.000000002	1586899999.99999\\
37000.0000000017	4638400000\\
-111999.999999999	-5858900000\\
-236000.000000002	-732899999.999999\\
384000.000000001	13672200000\\
330000.000000001	-2563700000\\
-404000.000000003	-21728300000\\
93000.0000000026	22094600000\\
-404000.000000003	-18921000000\\
475999.999999999	27344100000\\
73999.9999999998	-19043400000\\
-420999.999999999	-6835700000.00001\\
291999.999999997	21606500000\\
-236999.999999997	-21972900000\\
364999.999999997	23804100000\\
-254999.999999998	-25269000000\\
-165000.000000001	14770800000\\
200000.000000002	-2075300000\\
148000	5615200000\\
-165000.000000002	-15624800000\\
-55999.9999999965	13427500000\\
-72000.0000000027	-9155200000.00001\\
310999.999999999	16601600000\\
-330999.999999998	-26733400000\\
94000.000000003	24169900000\\
345999.999999998	366299999.999978\\
-71999.9999999983	-17944700000\\
-220000.000000001	3296500000.00001\\
-313000.000000001	853999999.999993\\
314000	13427900000\\
235999.999999999	-2319299999.99999\\
-346999.999999998	-21606400000\\
347999.999999998	31493900000\\
-54999.9999999997	-27709700000\\
-385999.999999999	7568300000.00001\\
-181000.000000002	2197100000\\
622000.000000001	15747300000\\
-91999.9999999996	-22827300000\\
-348000.000000002	2563600000.00001\\
330000.000000002	18798700000\\
2.66453525910038e-09	-19165100000\\
-36000.0000000031	8789400000\\
-203000	-12207400000\\
-18000.0000000007	15381000000\\
348999.999999998	1464899999.99999\\
-274999.999999999	-16723600000\\
126999.999999998	14648100000\\
20000.0000000022	-5126400000.00001\\
-202000.000000002	-6714400000\\
255999.999999998	15991500000\\
-220000	-17211900000\\
-109000	6713800000\\
237000	9521399999.99999\\
57000.000000003	-10620000000\\
-3000.00000000455	3784199999.99999\\
2000.00000000333	-1831200000\\
-55000.0000000015	-2807500000\\
-184000.000000001	-610399999.999996\\
56000.0000000009	8178800000\\
72999.9999999977	-5371300000\\
-19999.9999999996	1953500000\\
39000.0000000041	-732900000.000011\\
181999.999999997	7202600000\\
-17999.9999999998	-13062000000\\
-532000	-9520900000\\
258000.000000003	28808000000\\
456999.999999996	-609899999.999996\\
-385000.000000001	-33203400000\\
-91999.9999999987	24170000000\\
277000	5737500000\\
-94000.0000000021	-18799200000\\
-219000	6958200000\\
239000	9033399999.99999\\
-130000	-12817700000\\
-33999.9999999989	6836100000.00001\\
271999.999999998	8666899999.99999\\
-273000.000000001	-23559500000\\
108999.999999999	23071300000\\
-17999.9999999998	-13305800000\\
72999.9999999986	8911500000\\
-200999.999999999	-14893100000\\
255999.999999998	24292400000\\
-146000.000000002	-26245300000\\
-91999.9999999996	15625000000\\
182999.999999999	-854399999.999987\\
147000.000000001	976499999.999997\\
-145999.999999998	-12451100000\\
-167000	10619900000\\
-199000.000000002	-2929400000\\
-73999.9999999989	-1709200000\\
529999.999999997	11840900000\\
55999.9999999991	-8666900000\\
1.77635683940025e-09	1708900000\\
-385000.000000002	-15869300000\\
164999.999999999	24902700000\\
55000.0000000006	-14648800000\\
72999.9999999977	5981600000\\
-348000	-10375800000\\
238999.999999999	15990900000\\
17000.0000000012	-10253700000\\
-109000.000000003	488100000.000002\\
165000.000000002	6347900000\\
-92000.0000000014	-8911299999.99999\\
73000.0000000004	7690500000\\
-201999.999999999	-10376000000\\
-126000.000000001	7202000000\\
564999.999999998	15259000000\\
-382000.000000001	-33081000000\\
126000.000000001	27709700000\\
-200000.000000002	-18676600000\\
255000.000000001	18310600000\\
-217999.999999998	-20630100000\\
273999.999999999	26977900000\\
-184000.000000001	-32349100000\\
-34999.9999999984	23437900000\\
363999.999999996	488100000\\
-712000	-30639700000\\
603000	46875200000\\
-291999.999999999	-39062600000\\
53999.9999999976	20995900000\\
-18000.0000000007	-8544600000\\
220000.000000001	13061400000\\
-293000.000000004	-26245200000\\
73000.0000000004	22949200000\\
201000	1587100000.00001\\
-71999.9999999983	-18066500000\\
-93000.0000000044	11596600000\\
-72999.9999999968	-8178700000\\
166000.000000002	15869400000\\
-74000.0000000025	-15137200000\\
37000.0000000017	5615799999.99999\\
-166000.000000004	-3052300000\\
203000.000000005	8423300000\\
-349000.000000005	-18799100000\\
293000.000000003	26855500000\\
202000	-8544799999.99999\\
-183000.000000001	-17090000000\\
-93000	15136800000\\
313000.000000002	3906400000.00001\\
-365999.999999999	-21484700000\\
88999.999999996	21606700000\\
21000.0000000017	-8545100000.00001\\
-129000.000000003	366500000.000006\\
-36999.999999999	-1709400000\\
219999.999999998	8423300000\\
-18999.9999999992	-6592200000\\
92999.9999999991	4394700000\\
-20000.0000000005	-6103299999.99999\\
-437999.999999999	-12085400000\\
365999.999999999	31250400000\\
16999.9999999995	-20263900000\\
111000.000000004	9643499999.99999\\
-476000.000000002	-28686200000\\
568000.000000002	57006500000\\
34999.9999999975	-51757700000\\
-382999.999999999	9887800000\\
16999.9999999959	11596600000\\
-236999.999999998	-9399500000.00001\\
311999.999999999	14282200000\\
-20999.9999999981	-12939100000\\
241000	16723200000\\
-111000.000000002	-25024100000\\
17000.0000000003	15380600000\\
-291000	-10986100000\\
35999.9999999987	13427500000\\
108999.999999998	-5615000000\\
203000	4516600000\\
-350000	-13306000000\\
239999.999999997	15991700000\\
36000.0000000014	-8545400000\\
-421000	-6835600000\\
328000.000000002	18676700000\\
21000.0000000008	-15381000000\\
34999.9999999957	9765700000.00001\\
-54999.9999999979	-10009700000\\
-55000.0000000015	7080100000.00001\\
1000.00000000122	-5493400000.00001\\
181999.999999997	14648600000\\
-310999.999999997	-27953900000\\
129000	27343300000\\
237000.000000001	-5492700000\\
-273999.999999997	-16357800000\\
109999.999999999	18921100000\\
-92000.0000000049	-13916000000\\
-91999.9999999978	5859300000\\
367000	14404200000\\
-165000.000000004	-24657900000\\
-183999.999999997	6835600000\\
167000.000000001	11230700000\\
107999.999999998	-7446400000.00001\\
-403000.000000001	-10253800000\\
129000	16601300000\\
330000	2563900000\\
-201999.999999999	-18677200000\\
-257000.000000001	5249300000.00001\\
623999.999999999	30395500000\\
-349999.999999999	-50415200000\\
3000.00000000011	32104600000\\
-186000.000000001	-15624900000\\
20000.0000000005	11230300000\\
438999.999999999	15747000000\\
-293000.000000003	-39184300000\\
-238999.999999997	18310400000\\
496999.999999999	22216800000\\
-131000.000000001	-34179700000\\
-346000.000000001	8544800000\\
236000	16479900000\\
76000.0000000014	-15259300000\\
-222000.000000003	2563800000\\
148000.000000001	5004800000\\
16999.9999999995	-5126900000\\
-182000.000000001	732300000.000001\\
91000.0000000011	1342800000\\
90999.9999999993	3051900000\\
-109000	-7934700000\\
202000	12329100000\\
-185000	-16113300000\\
38000.0000000002	11841000000\\
110000.000000002	1464600000\\
72999.9999999986	-6713600000.00001\\
-294000.000000003	-6836299999.99999\\
129000.000000001	16723900000\\
-17999.9999999989	-12939400000\\
-128000.000000001	3295700000\\
364999.999999998	15258800000\\
-292000.000000002	-29296600000\\
72999.9999999995	25756400000\\
54999.9999999988	-15624600000\\
-274999.999999999	-976699999.999991\\
385000.000000001	25146300000\\
-146999.999999999	-33080700000\\
1000.00000000122	19652900000\\
126999.999999999	-2196900000\\
-91000.0000000037	-10498200000\\
-127999.999999998	8178700000\\
-202000.000000003	-3662100000\\
330000.000000001	12207100000\\
-37000.0000000008	-16723700000\\
38000.0000000011	15991100000\\
-130000.000000002	-19531000000\\
-201000	11230400000\\
404000.000000002	9032900000.00001\\
108000.000000002	-3783700000.00001\\
-236000.000000001	-20386100000\\
-38000.0000000002	22949400000\\
201999.999999999	-7568400000\\
-312000.000000001	-10132000000\\
93000	16479800000\\
126000	-3906399999.99998\\
-126000.000000001	-9887900000.00001\\
-38000.0000000011	8789400000\\
-17999.9999999971	-3906400000.00001\\
294000	16845700000\\
-93000	-26855500000\\
-201000.000000003	9399499999.99999\\
311000.000000003	16845600000\\
-328999.999999998	-33080900000\\
183999.999999997	32226400000\\
-94000.0000000003	-21484400000\\
-107999.999999996	7568600000\\
182999.999999999	6713700000\\
366000	7690300000\\
-623000.000000002	-43945000000\\
145999.999999998	47729300000\\
-52999.9999999982	-25512800000\\
180999.999999999	15869400000\\
148000	-5859600000\\
-404000.000000001	-14282000000\\
1000.00000000122	14159900000\\
35999.9999999987	-2441199999.99999\\
459000.000000003	20629700000\\
-130000.000000003	-33324900000\\
-145999.999999999	14037600000\\
-128000	-10985800000\\
56000.0000000009	15991000000\\
236000.000000001	8544699999.99999\\
-236000	-34179400000\\
-2000.00000000067	26611300000\\
222000.000000002	1830900000\\
-185000.000000002	-21972500000\\
-182000	12695200000\\
402000.000000001	15625100000\\
-128000	-26123100000\\
-346999.999999998	6347600000\\
456999.999999997	17822300000\\
-293000.000000002	-23925600000\\
109999.999999999	17211600000\\
-55000.0000000006	-10619900000\\
-17999.999999998	4516600000\\
237999.999999999	9399200000\\
-74000.0000000016	-16601300000\\
-256000	610199999.999985\\
166000.000000001	12695400000\\
-39000.0000000024	-7568400000\\
131000	6835899999.99999\\
-20999.999999999	-10620000000\\
-309000.000000001	-3296100000.00002\\
345999.999999999	23559700000\\
56000.0000000009	-17944300000\\
-439000.000000001	-13549800000\\
382999.999999999	36987100000\\
56000.0000000009	-26000700000\\
-128000.000000001	732300000.000004\\
-19000.0000000019	4150300000\\
-18000.0000000007	854699999.999997\\
-366000	-6225800000\\
475999.999999998	18066500000\\
73000.0000000013	-15136600000\\
-128000.000000001	-1220899999.99999\\
-201999.999999997	2441400000\\
-73000.0000000022	854700000\\
313000.000000001	7079900000\\
-149000.000000003	-11352400000\\
111000.000000001	8910900000\\
37000.0000000008	-2563200000\\
-111000.000000002	-6225799999.99999\\
203000.000000001	14648600000\\
70999.9999999971	-18066500000\\
-619999.999999999	-366200000.000012\\
327999.999999999	20996100000\\
-8.88178419700125e-10	-18066400000\\
-145000	4882800000\\
290999.999999999	7812500000\\
-200999.999999998	-15625000000\\
368000	26733400000\\
-241000	-37109300000\\
-344999.999999998	17456000000\\
583999.999999999	20751800000\\
-182000.000000001	-30883600000\\
-92000.0000000005	9643600000\\
-202000.000000002	-732599999.999994\\
129000.000000004	6713900000\\
-55000.0000000015	-5370899999.99999\\
458000.000000002	16113100000\\
-294000.000000004	-30273500000\\
-146000	17456200000\\
56000	-3173799999.99999\\
-38000.0000000011	2563300000\\
-54999.9999999997	-4394399999.99999\\
385000.000000001	20874000000\\
-181999.999999999	-33447200000\\
107999.999999997	26000900000\\
-109000.000000001	-20385900000\\
-293000	6592099999.99999\\
548999.999999998	17455900000\\
-622000.000000001	-29541000000\\
310999.999999999	26733400000\\
17999.9999999998	-15503000000\\
-310000.000000001	1343000000\\
364000	7934199999.99999\\
-34999.9999999975	-3295499999.99999\\
-54999.9999999997	-6470000000\\
183000.000000003	17822300000\\
127999.999999997	-15869000000\\
-440000.000000003	-14038100000\\
91999.9999999996	27709700000\\
-90999.9999999984	-17333700000\\
90999.9999999966	11840800000\\
0	-6470000000\\
292999.999999998	14893000000\\
-110000	-25879300000\\
-311000.000000001	5005099999.99999\\
91000.0000000002	11840800000\\
275999.999999999	7812399999.99999\\
-130000	-22826900000\\
-52999.999999999	8300400000.00001\\
52999.9999999964	5615499999.99999\\
-126999.999999999	-6469700000\\
72999.9999999986	6225400000\\
-236999.999999999	-8422700000\\
162000.000000002	8544800000\\
185999.999999999	-244000000\\
16999.9999999968	-2197300000\\
-128000	-6469900000\\
220000.000000001	10620300000\\
-165999.999999999	-10498000000\\
-328000	4516400000\\
-1000.00000000211	2563700000\\
239000.000000001	-732700000\\
108999.999999996	122400000.000002\\
-17999.9999999962	-1465100000\\
-55000.0000000033	100000.000008293\\
-55000.0000000006	-1709100000.00001\\
-54999.9999999979	610700000.000008\\
310999.999999998	13061100000\\
-163999.999999995	-22094600000\\
35999.9999999987	18066700000\\
166000.000000001	-4639200000\\
-368000.000000005	-20873500000\\
19000.0000000037	26244800000\\
999.999999996781	-10498100000\\
199999.999999999	12085300000\\
73999.9999999998	-13794200000\\
-166000.000000001	-244200000.000015\\
-52999.9999999982	5371300000\\
-166000.000000002	-6836000000.00001\\
294000.000000002	16967600000\\
162999.999999996	-11352200000\\
-235999.999999997	-8789499999.99999\\
-276999.999999999	8301199999.99999\\
203999.999999998	4882500000\\
272999.999999999	122400000\\
-348000	-14038500000\\
257000	19287400000\\
-218999.999999999	-20141600000\\
-56000.0000000018	13427500000\\
420999.999999999	8911500000\\
-329999.999999999	-29419400000\\
-89999.9999999972	22827600000\\
274000	1952900000.00001\\
91000.0000000002	-4028499999.99998\\
-90999.9999999966	-9399000000.00003\\
-91000.0000000037	121800000.000007\\
34999.9999999984	10498100000\\
-345999.999999997	-12939500000\\
400999.999999997	22583000000\\
-347000.000000001	-28075900000\\
366000.000000001	27831600000\\
19000.0000000037	-18920500000\\
-312000.000000002	-1221100000\\
367000	18921300000\\
-386000.000000003	-28320600000\\
276000.000000003	28808700000\\
72999.9999999968	-17822200000\\
-347999.999999999	243999999.999998\\
219000.000000004	10254000000\\
-91000.0000000011	-9643500000\\
73999.9999999989	6103200000\\
72000.0000000036	-1464400000\\
-218999.999999999	-5127200000\\
17999.9999999998	6347700000\\
182999.999999997	732399999.999998\\
-36000.0000000005	-5004900000\\
-220000.000000001	-1342700000\\
165000	7568300000\\
236999.999999997	1098600000\\
-199999.999999998	-11840600000\\
-258000.000000001	1220400000.00001\\
258000.000000002	14526500000\\
200999.999999999	-3906099999.99999\\
-239000.000000001	-17212300000\\
294000	25513100000\\
-275000.000000001	-27343900000\\
-439000.000000002	11352400000\\
328000.000000002	10620300000\\
276999.999999999	-5371100000\\
-258000.000000001	-10254000000\\
219999.999999999	19775500000\\
72999.9999999995	-16601800000\\
-163999.999999999	300000.000015643\\
-165000	854299999.999995\\
37000.0000000008	7812600000\\
199999.999999997	-976599999.999991\\
18999.9999999983	-5004900000\\
-291999.999999998	-4394500000\\
164000	13793900000\\
72000.0000000018	-10620100000\\
-52000.0000000014	2807700000\\
-258999.999999999	-3296000000\\
147000.000000001	6591800000\\
220999.999999997	2.1316282072803e-06\\
17000.0000000003	-2197099999.99999\\
-365000.000000001	-13061800000\\
89999.999999999	20996300000\\
532000.000000003	1708900000.00001\\
-604000.000000001	-31616200000\\
182999.999999998	32836900000\\
199999.999999999	-7568300000\\
-35000.000000001	-10131900000\\
-366999.999999999	-1098600000\\
111000.000000002	13061500000\\
382999.999999996	1465000000\\
-439000	-20508200000\\
202000	21362700000\\
110000.000000001	-10376100000\\
-220999.999999997	-854599999.999997\\
-181999.999999999	2563600000\\
255000	3051700000\\
111999.999999997	-122100000.000006\\
-148000.000000002	-8300600000\\
1000.00000000211	8056400000\\
199999.999999999	2075400000\\
1000.00000000033	-8056899999.99999\\
-274999.999999999	-365900000.000001\\
-35999.9999999996	5615000000\\
109000.000000001	-366100000.000004\\
19000.0000000001	-1586900000\\
-256999.999999999	-5127100000\\
457999.999999996	17456300000\\
55000.0000000006	-16235600000\\
-311000.000000002	99999.9999947931\\
-17999.9999999998	2319400000\\
-20000.0000000005	4516600000.00001\\
203000.000000002	1098499999.99998\\
-36999.9999999972	-5370999999.99999\\
109999.999999999	6103499999.99999\\
-256000.000000003	-16235400000\\
53000.0000000017	17822400000\\
56999.9999999986	-6103600000\\
-91999.9999999996	-1709000000.00001\\
0	-854499999.999996\\
181999.999999999	15869200000\\
-16999.9999999995	-23437700000\\
-55999.9999999983	12817800000\\
-71999.9999999992	-7202500000\\
0	6836000000\\
-20000.0000000022	-4760700000\\
240000.000000001	14160100000\\
-277000.000000002	-28564300000\\
-51999.999999996	23803600000\\
143999.999999997	-6103499999.99999\\
148000.000000004	6591700000.00001\\
-220000.000000003	-20507700000\\
146000.000000002	23193300000\\
-55000.0000000033	-15258600000\\
92000.0000000023	8422599999.99999\\
-366000	-13915900000\\
108999.999999998	16235300000\\
367000.000000001	7080100000\\
17999.9999999998	-20385600000\\
-439000	-2319600000.00001\\
291999.999999998	23559700000\\
-54000.0000000003	-23559500000\\
-257000.000000001	11596600000\\
384999.999999999	5493200000\\
-183999.999999999	-14404400000\\
-107999.999999999	9277500000\\
273000.000000001	1464800000\\
-329999.999999999	-10498200000\\
257999.999999996	15136900000\\
91000.0000000002	-10742200000\\
-421000	-3051800000\\
292000.000000002	12451100000\\
-91000.0000000011	-9521400000\\
146999.999999999	5737200000\\
-55999.9999999983	-5614900000\\
-182000.000000002	609899999.999999\\
236999.999999999	8789300000.00001\\
-71999.9999999992	-13183600000\\
-313000.000000001	2563400000.00001\\
625000.000000003	23071400000\\
-240000.000000002	-35644600000\\
-238000.000000002	17578200000\\
294000.000000001	5981200000\\
-37000.0000000008	-11596400000\\
-384999.999999997	-3173800000\\
310999.999999998	21484000000\\
185000	-18920500000\\
-532999.999999998	-2563700000\\
366999.999999998	18310700000\\
-73000.0000000013	-16479700000\\
219000.000000001	13428000000\\
-128000.000000002	-14526600000\\
-126999.999999997	6591900000\\
-74999.9999999993	610400000\\
-72000.0000000001	-1342900000\\
310999.999999999	7080200000\\
-73999.9999999981	-10742300000\\
-164000.000000003	4394600000\\
74000.0000000007	976600000\\
327999.999999999	8544800000\\
-438999.999999999	-25512600000\\
127999.999999998	26489300000\\
221000.000000002	-7202299999.99999\\
-203000.000000001	-9155100000.00002\\
-199999.999999999	-1709099999.99999\\
419999.999999997	32226600000\\
-238000	-49682700000\\
74000.0000000007	43090900000\\
-55000.0000000006	-30517400000\\
-74000.0000000016	17943900000\\
422000.000000001	8179099999.99999\\
-294000.000000001	-30395600000\\
-127000.000000001	17455900000\\
-166000.000000001	-2319200000\\
203000.000000002	9521600000\\
180999.999999999	-7080399999.99999\\
21000.0000000026	2441600000.00001\\
-21000.0000000035	-8789000000.00001\\
-345999.999999999	854400000.000004\\
-92999.9999999982	6347700000\\
330999.999999998	4028300000.00001\\
200000	-976700000.000013\\
1000.00000000033	-2685300000.00001\\
-256000.000000003	-13916100000\\
-184999.999999999	12329000000\\
75999.9999999996	1098700000\\
235000.000000001	6713900000\\
3000.00000000011	-10009800000\\
33999.999999998	5737300000\\
-272999.999999995	-20629900000\\
72999.9999999995	32226600000\\
201000.000000001	-15991100000\\
-274000	-11841100000\\
255000	33569500000\\
20000.0000000005	-32470600000\\
-257999.999999999	3906100000\\
184000	21484300000\\
257000.000000001	-13793600000\\
-496000.000000001	-16357900000\\
-15999.9999999982	22461400000\\
143999.999999995	-3052100000\\
129000.000000002	-610200000.000012\\
55999.9999999991	-366299999.999994\\
-239999.999999999	-14404100000\\
129999.999999999	24902200000\\
109000.000000001	-16235500000\\
-309999.999999997	-4760400000.00001\\
216999.999999995	19042800000\\
-125999.999999997	-20874200000\\
128000.000000002	20996300000\\
145999.999999997	-10253800000\\
-164999.999999998	-7080399999.99999\\
109999.999999999	7812700000.00001\\
-438000.000000001	-9277300000.00001\\
417999.999999999	21606400000\\
-87999.9999999992	-20996200000\\
-21000.0000000008	9033300000.00001\\
-219000.000000001	-7446300000\\
239000.000000003	12451200000\\
17999.9999999989	-6713900000\\
16999.9999999995	244000000.000001\\
166000	3662499999.99999\\
-238000	-13184100000\\
-19000.0000000019	13184000000\\
-54000.0000000003	-7202300000\\
-111000.000000002	4028300000\\
312000	5004900000\\
-55000.0000000015	-8300800000\\
-274999.999999999	-4394399999.99999\\
257000.000000001	16601200000\\
54999.9999999988	-13061000000\\
-38000.0000000002	5858999999.99999\\
-108000.000000001	-7934399999.99998\\
309000	20751800000\\
-271999.999999998	-34179600000\\
-20000.0000000022	25756800000\\
-18000.0000000016	-10131600000\\
-110000	3661700000\\
293000.000000001	8057000000.00001\\
18999.9999999983	-12573500000\\
-221000.000000001	-488099999.999989\\
-217999.999999997	1342600000\\
346999.999999998	12817600000\\
126999.999999999	-7690500000\\
-418999.999999998	-16357700000\\
290999.999999999	29907700000\\
203000.000000003	-12085300000\\
-165000.000000002	-14526300000\\
-111000.000000002	11963000000\\
-73000.0000000004	-2075400000\\
-182000	-2563200000\\
456999.999999999	19286900000\\
-183000.000000001	-28198200000\\
-73000.0000000013	16845700000\\
109000.000000001	-3906200000\\
-17000.0000000003	2075200000\\
-56000.0000000018	-6591900000\\
165000.000000003	13549900000\\
-164000.000000002	-22827100000\\
89999.9999999998	25756600000\\
-35000.0000000001	-20629600000\\
-274999.999999999	6469600000\\
218999.999999998	6469800000\\
128000	-300000.000021328\\
220999.999999999	3418400000.00001\\
-752000	-40161400000\\
513999.999999999	66162100000\\
272999.999999999	-41137400000\\
-272999.999999998	1220299999.99999\\
-404000.000000004	-4150200000\\
366000.000000003	27343800000\\
38000.000000002	-24658200000\\
-202000.000000003	6835700000.00001\\
329000.000000001	8911499999.99999\\
-365000	-21362700000\\
-93000.0000000035	13184000000\\
403000.000000002	15746800000\\
166000.000000002	-15747000000\\
-477000.000000001	-17578100000\\
128000.000000001	26245000000\\
56000.0000000009	-7079899999.99999\\
90999.9999999975	2319199999.99998\\
-147000.000000001	-13671800000\\
128000	18310500000\\
-292000.000000002	-21850500000\\
183000.000000002	27343700000\\
-147000.000000003	-28198400000\\
275000.000000002	30884000000\\
-184000	-29907200000\\
331000.000000003	36498800000\\
-165000.000000003	-45776200000\\
-110999.999999999	27710000000\\
-72999.9999999995	-11596900000\\
129999.999999999	17944700000\\
88999.9999999986	-17334500000\\
-217999.999999999	2442000000.00001\\
-19000.0000000028	6225099999.99999\\
220000.000000001	366399999.999993\\
-55000.0000000024	-6713799999.99999\\
-274999.999999998	-244200000.000004\\
146999.999999998	8300700000\\
292000	854700000.000005\\
-162999.999999998	-12207400000\\
-129999.999999999	5493600000\\
-274000.000000002	-3174100000\\
366999.999999998	13061600000\\
457000.000000001	2197300000\\
-660000.000000001	-34424000000\\
74999.9999999984	36133100000\\
219000.000000001	-13672200000\\
-202000.000000001	-2563100000\\
164999.999999998	10131400000\\
37000.0000000008	-8544500000.00001\\
-90999.9999999993	-610699999.999997\\
-56999.9999999995	2685800000\\
-89000.0000000013	-1098800000\\
-93000.0000000026	-610199999.999999\\
312000.000000002	8788900000\\
108999.999999998	-5126899999.99999\\
-91999.9999999996	-6835800000\\
-70999.9999999997	3784000000\\
-239999.999999997	-2197099999.99999\\
-36000.0000000005	5126600000\\
328999.999999998	1831600000\\
1000.00000000211	-3296400000\\
54999.9999999997	3174200000\\
-184000.000000003	-12695600000\\
-36999.999999999	11841000000\\
-34999.9999999984	-4394600000\\
35999.9999999996	3173900000\\
90999.9999999975	2197200000.00002\\
219999.999999999	2319200000\\
-73999.9999999989	-13061200000\\
-308999.999999997	-732699999.999997\\
88999.9999999942	15747200000\\
20000.0000000022	-10253900000\\
199999.999999997	7568300000\\
-310000	-15380900000\\
128000	17700400000\\
17999.999999998	-8301100000\\
164000.000000001	3174200000.00001\\
-89999.9999999981	-7324600000\\
-313000.000000001	2563799999.99999\\
93999.9999999985	5370800000\\
88999.9999999995	-4394100000\\
147999.999999999	7079499999.99999\\
-73999.9999999998	-10863700000\\
54999.9999999988	11718300000\\
-72000.0000000001	-15624700000\\
-148000.000000001	9277200000\\
73000.0000000004	1342900000.00002\\
331000.000000002	15258600000\\
-73999.9999999989	-30029200000\\
-347000	2075300000\\
89999.9999999972	19164900000\\
-17999.9999999962	-10009800000\\
366999.999999998	22949400000\\
3.5527136788005e-09	-40039200000\\
-403000.000000003	14404300000\\
-37000.0000000026	7690399999.99999\\
183000.000000002	2319500000\\
73999.9999999998	-8056700000\\
-256999.999999996	610099999.999996\\
-19000.0000000037	3662600000\\
167000.000000001	-855100000.000002\\
107999.999999999	1465400000\\
18999.9999999992	-1709300000\\
-109999.999999999	-4638599999.99999\\
-109999.999999999	3906300000\\
54999.9999999979	1586800000\\
110000.000000001	976799999.999997\\
-1.77635683940025e-09	-2685999999.99999\\
55000.0000000006	977200000.000005\\
-74000.0000000025	-3296500000\\
-107999.999999998	1587200000\\
-111999.999999998	99999.9999947931\\
183999.999999998	5126600000\\
164999.999999998	-1220200000\\
-54999.9999999979	-6714400000\\
-402999.999999997	-1952700000\\
494000.000000001	18066100000\\
-293000.000000002	-23437300000\\
37999.9999999976	17211800000\\
-56000	-10253800000\\
182999.999999999	10986300000\\
165000.000000002	-5371299999.99999\\
-457000	-14648000000\\
236999.999999997	26122500000\\
110000.000000001	-15258300000\\
-92000.0000000005	-399999.999997647\\
-292000	-4638200000\\
238000.000000001	17211400000\\
164999.999999998	-9399099999.99999\\
-184999.999999999	-5859500000.00001\\
167000.000000002	13427900000\\
-129000.000000002	-18310800000\\
-17999.9999999998	13428000000\\
-111000.000000001	-8911399999.99999\\
276000	19531500000\\
-293000	-31128100000\\
199999.999999999	31860400000\\
-16999.9999999995	-22705100000\\
-367000	-2197099999.99998\\
624000.000000002	41869800000\\
89999.9999999963	-48095300000\\
-476000.000000001	6591300000\\
-91000.0000000011	11353000000\\
37000.0000000017	1220500000\\
108999.999999997	-1831000000\\
37000.0000000026	854299999.999999\\
37000.0000000017	-1830800000\\
-1000.00000000122	488200000.000012\\
148000.000000001	5249000000.00002\\
-238999.999999999	-18066500000\\
-74000.0000000043	14160400000\\
184000.000000001	3661800000\\
92000.0000000005	-2441000000\\
-111000.000000002	-11475100000\\
-364999.999999998	5127300000\\
402000	14770500000\\
-146000.000000001	-20752200000\\
54999.9999999997	14648600000\\
181999.999999999	-1220500000\\
-35999.9999999978	-6348100000\\
-163999.999999999	-3539699999.99999\\
-184000.000000001	1464700000.00001\\
184000.000000002	13061600000\\
-1000.00000000122	-13427800000\\
91999.9999999996	12451200000\\
-36999.999999999	-15624900000\\
127999.999999998	20019300000\\
-126999.999999997	-24413900000\\
183000	22338800000\\
-221000.000000005	-22827000000\\
-383999.999999998	7568100000\\
385000.000000001	17700500000\\
383999.999999999	-976800000.000003\\
-658999.999999999	-40893500000\\
201999.999999998	45898600000\\
109000.000000002	-17212300000\\
311000	18311000000\\
-181999.999999999	-39917300000\\
-110000.000000001	23681800000\\
-386000.000000002	-6713900000.00001\\
368000.000000002	17456000000\\
-19000.0000000001	-18188500000\\
183000	12939500000\\
-366000	-18920700000\\
257000	23803300000\\
-167000.000000002	-21484100000\\
130999.999999998	18066400000\\
-111999.999999998	-15869200000\\
-73000.0000000013	8667000000\\
313000.000000002	6958000000\\
-240000.000000004	-20141600000\\
19000.0000000019	20263800000\\
183000.000000002	-7324599999.99999\\
129000	4517100000\\
-404000	-23315700000\\
-17000.0000000021	25634700000\\
401000	244300000.000004\\
-308999.999999999	-18676800000\\
161999.999999998	16723500000\\
-345000.000000001	-14281900000\\
327000	18920500000\\
112000.000000001	-13305300000\\
-111000.000000002	-610700000.000007\\
-183000	1709200000\\
-164000.000000001	1831000000\\
219000	1831200000\\
182000.000000001	-122400000.000003\\
75999.9999999987	400000.000004752\\
-331999.999999996	-10864700000\\
-18000.0000000016	10986800000\\
110999.999999998	487900000.000008\\
291999.999999999	4516799999.99999\\
-256000.000000001	-18188600000\\
-127999.999999997	12207300000\\
91999.9999999978	731999999.999999\\
109000	1465100000\\
-92000.0000000023	-9643400000.00001\\
38000.0000000011	12694900000\\
126999.999999999	-7079700000\\
-164000.000000001	-4150600000.00001\\
-146999.999999998	2807600000\\
255999.999999998	10010000000\\
-162999.999999997	-15747300000\\
-39000.0000000032	10131800000\\
20000.0000000031	-5492800000\\
347999.999999997	21239700000\\
-202999.999999999	-38695900000\\
-310000.000000002	20629800000\\
475000	18310500000\\
-34999.9999999975	-30151400000\\
-239000.000000003	8545100000\\
54999.9999999988	4882500000\\
-128999.999999999	-2319000000\\
147999.999999998	4028100000\\
-18999.9999999983	-6591700000\\
-293000.000000004	-488300000\\
751000.000000002	20629800000\\
-532000.000000002	-35278200000\\
-199999.999999998	23803700000\\
255999.999999997	-5615300000\\
182000.000000003	6591800000\\
-36000.0000000031	-12573200000\\
-89999.9999999981	7934600000\\
-276999.999999999	-8056800000\\
19000.0000000028	9521800000\\
256999.999999994	-488700000.000001\\
238000.000000002	3174099999.99999\\
-221000.000000002	-13915900000\\
-181999.999999999	5859000000.00001\\
147000	6347900000\\
-56000.0000000018	-6958000000\\
1000.000000003	3540000000\\
34999.9999999957	-122099999.999996\\
38000.0000000038	2075200000\\
54999.9999999979	-3540000000\\
35999.9999999996	1098600000\\
-165000.000000001	-5859300000.00001\\
-55000.0000000006	6103300000\\
-52999.9999999973	-854200000\\
381999.999999998	10986200000\\
-254999.999999999	-23315600000\\
-293999.999999999	11597000000\\
367999.999999998	9277100000\\
52999.9999999982	-8056600000\\
-108999.999999999	-6347500000.00001\\
-145999.999999995	6835800000.00002\\
548999.999999998	16357400000\\
-788000.000000003	-49804700000\\
422000.000000001	58471800000\\
146000	-31982400000\\
-256000	1830800000\\
71999.9999999992	8789300000.00001\\
-290999.999999997	-19287100000\\
475000	43823100000\\
128000.000000001	-37963900000\\
-511999.999999999	-10863900000\\
108999.999999999	32714300000\\
184000.000000002	-6957600000.00001\\
-148000.000000001	-15503000000\\
130000.000000002	20385600000\\
-184000.000000001	-29174700000\\
238999.999999999	48583900000\\
53000.0000000026	-49072000000\\
-108000.000000001	19042600000\\
-239000	-7934399999.99999\\
164999.999999996	20019700000\\
36000.0000000014	-17822700000\\
-292000.000000002	1709500000.00001\\
183000.000000002	9154899999.99999\\
402999.999999998	6836100000\\
-349000	-29174900000\\
-109000.000000002	21484400000\\
293000	-243899999.999994\\
-312000	-12451600000\\
366999.999999998	20019800000\\
-366999.999999999	-24414100000\\
93000	21850700000\\
-56000.0000000009	-15381200000\\
238000.000000002	13794300000\\
-72999.9999999986	-13427900000\\
-999.999999997669	8667000000\\
-108000.000000003	-5737300000\\
-183999.999999998	2319300000\\
420999.999999999	6836100000\\
-220000.000000001	-12451300000\\
-255000.000000001	4882800000\\
309000	6347600000\\
148999.999999997	-3661900000\\
15999.9999999991	-244300000\\
-328000	-10864300000\\
-72999.9999999986	12451300000\\
711999.999999999	15014700000\\
-436999.999999999	-37475900000\\
-184000.000000002	23071600000\\
164000.000000005	-1342800000\\
-311000.000000003	-4883000000\\
147000.000000003	4760900000\\
91999.999999997	-1464900000\\
402000.000000001	9765699999.99999\\
-641000.000000003	-30273700000\\
184000	32349100000\\
273999.999999999	-9399800000.00002\\
-18999.9999999992	-2197199999.99999\\
1000.00000000033	-4394500000.00001\\
-421000	-5126800000.00001\\
109999.999999999	15746900000\\
90999.9999999975	-6469800000\\
457000.000000001	13061800000\\
-292000.000000002	-30517900000\\
-476000	14404500000\\
457999.999999999	14892600000\\
146000	-16235400000\\
-238999.999999996	732200000.000002\\
-438000.000000001	1221100000\\
146999.999999999	2441200000\\
583999.999999998	10742100000\\
-16999.9999999986	-14892400000\\
-184000.000000003	488099999.999993\\
-236999.999999997	-732199999.999992\\
493999.999999998	12573000000\\
-695999.999999995	-19164700000\\
256000	18798300000\\
999.999999998557	-13671300000\\
-999.999999998557	9032800000\\
18999.9999999983	-5981200000\\
238000.000000001	9033000000\\
-91999.9999999996	-11962800000\\
-91000.0000000011	7202200000.00001\\
72999.9999999968	-1709000000\\
-18999.9999999983	-1220800000.00001\\
-17000.0000000021	2807800000.00001\\
-130000	-8179000000.00001\\
-88999.999999996	6103800000\\
400999.999999994	16113100000\\
165000.000000002	-18310400000\\
-293000.000000002	-10376100000\\
-676999.999999999	5005000000\\
639999.999999999	26245000000\\
221000.000000002	-19775400000\\
-73999.9999999998	2441500000\\
-311000	-12817400000\\
-238000.000000003	13305700000\\
347000.000000001	2563300000\\
202999.999999998	4150599999.99999\\
-185000	-16723800000\\
55999.9999999991	14038200000\\
111000.000000003	-4150399999.99999\\
-277000.000000005	-12939400000\\
422000.000000002	28564100000\\
-437999.999999998	-36742600000\\
-148000.000000005	26488900000\\
256000.000000001	-4394600000.00001\\
55999.9999999991	-121799999.999998\\
-17999.9999999971	-4150600000.00001\\
162999.999999996	8545000000.00001\\
-382999.999999999	-22460900000\\
273999.999999999	31250000000\\
129000	-19287200000\\
-166000.000000001	-366300000.000005\\
-127000.000000001	5005300000\\
-165000	-1953600000\\
420000.000000003	7324500000\\
-54000.0000000003	-9155400000\\
-385000.000000001	-1586900000\\
20000.0000000005	6347700000\\
657000.000000001	9521600000\\
-365999.999999999	-23559900000\\
-237000.000000005	14038400000\\
18000.0000000016	-3662200000\\
513000.000000002	16723500000\\
-92000.0000000032	-26000800000\\
-238999.999999995	9399300000\\
-493000	-3539900000\\
274999.999999998	10375800000\\
401999.999999999	1220800000\\
-220000	-13671900000\\
92000.0000000014	13916100000\\
127999.999999999	-2929800000.00001\\
37000.0000000026	854399999.999981\\
-312000.000000001	-27099200000\\
-109000.000000001	28808000000\\
347000.000000001	10864800000\\
-90999.9999999975	-30762000000\\
-183000	12085000000\\
127999.999999999	6958000000\\
37000.0000000035	-4638599999.99999\\
-166000.000000003	-9887800000.00002\\
368000	37841900000\\
-1000.00000000033	-50903400000\\
-184000.000000002	20263700000\\
-181999.999999998	488400000.000007\\
-109000.000000001	5126599999.99999\\
235999.999999997	-976100000.000002\\
-16999.9999999995	-3540299999.99999\\
-202000	-4272500000\\
459000	19775500000\\
-240000.000000003	-28198300000\\
-126999.999999999	16723700000\\
440000.000000001	8056600000.00001\\
-496000.000000001	-28686500000\\
37999.9999999985	23925600000\\
182000	-3783800000\\
221000.000000002	5248700000\\
-202000.000000002	-20263700000\\
-294000	11963300000\\
257999.999999998	6835400000\\
-420999.999999999	-19042600000\\
492000	27587700000\\
186000.000000003	-11962700000\\
-314000.000000004	-14770600000\\
-15999.9999999938	17333800000\\
-19000.0000000028	-12328800000\\
255000.000000001	26733200000\\
75999.9999999978	-27587900000\\
-240999.999999999	-3661800000.00001\\
92999.9999999991	15624500000\\
-348000.000000002	-9887300000.00001\\
74000.0000000034	12573100000\\
125999.999999997	-9887700000\\
204000.000000001	11962900000\\
-148000	-19043000000\\
-165000.000000003	12329100000\\
129000	-2441200000\\
1.77635683940025e-09	2074900000\\
-37000.0000000035	-4272300000\\
274000.000000001	10742100000\\
-17000.0000000003	-14282100000\\
-257000.000000002	976399999.999998\\
-17999.9999999989	8301000000\\
-55000.0000000006	-4028600000\\
72000.0000000001	1587100000\\
-126000.000000001	-3539900000\\
309999.999999998	8422400000\\
-182999.999999999	-11108000000\\
-37000.0000000008	5004800000\\
330000	11108300000\\
-438999.999999998	-30273300000\\
17999.9999999998	29540800000\\
567000	3540300000\\
-346999.999999998	-30151500000\\
-146999.999999998	14648300000\\
73999.9999999989	7568700000.00001\\
291000.000000001	6103199999.99999\\
-327000.000000001	-31249800000\\
199999.999999999	29785000000\\
-293000	-19775300000\\
-202000.000000002	7324200000\\
277000.000000004	7202200000.00001\\
254000	5126799999.99999\\
109999.999999999	-4027999999.99998\\
-199999.999999996	-22583500000\\
-37000.0000000026	21973200000\\
-147999.999999998	-16968200000\\
-52999.9999999999	15258900000\\
493000	18310800000\\
-346999.999999997	-48462300000\\
-55000.0000000024	36377200000\\
-219999.999999997	-23437500000\\
403000	35522300000\\
237999.999999997	-25512500000\\
-495000	-11352800000\\
146999.999999998	24658500000\\
-256000.000000001	-20019700000\\
127000.000000002	17089900000\\
332000	-854600000.000005\\
-94999.9999999997	-6591599999.99999\\
21999.9999999993	-5615300000\\
-332999.999999999	2074899999.99999\\
313999.999999997	12451700000\\
-37999.9999999994	-12573700000\\
-92000.0000000023	3051999999.99999\\
-16999.9999999986	1708900000\\
71999.9999999974	-122000000.000001\\
-162999.999999998	-2563600000\\
-56999.9999999986	1709200000\\
110999.999999996	1708700000\\
256000.000000001	2807800000.00001\\
36999.999999999	-3784200000.00001\\
-294000.000000001	-7812600000.00001\\
37999.9999999994	10620300000\\
-19000.0000000001	-6225600000\\
-110000	4760400000\\
294000.000000001	977000000.000004\\
-111000.000000003	-3418200000\\
201000.000000002	3540099999.99999\\
-676000.000000002	-18432600000\\
438000.000000001	28808500000\\
386000.000000003	-5615200000\\
-75000.0000000046	-14648300000\\
-565999.999999997	-4272600000.00001\\
-19000.0000000019	17456000000\\
438999.999999999	-488100000\\
-71999.9999999992	-8056800000\\
54000.0000000003	3418100000.00001\\
-421999.999999998	-10864300000\\
148999.999999999	16601400000\\
546999.999999998	4638900000.00001\\
-126999.999999999	-22583100000\\
-604999.999999999	5126900000\\
-109000.000000001	6469900000\\
602999.999999999	9399200000\\
203000.000000001	-5371000000.00001\\
-404000.000000002	-17456000000\\
-329000.000000001	9521299999.99999\\
348000.000000002	13061900000\\
274000	-4639199999.99999\\
-402999.999999999	-20507300000\\
55999.9999999991	26000600000\\
476000.000000003	1831300000\\
-386000.000000002	-32592900000\\
-254000	17089800000\\
472999.999999999	24780400000\\
-399999.999999998	-45043900000\\
109000.000000001	35033900000\\
310000	-3051500000.00002\\
-15999.9999999991	-13672000000\\
-607000.000000003	-13427500000\\
149000.000000002	29540600000\\
383000.000000001	-2074900000.00001\\
92000.0000000005	-2319299999.99999\\
-218999.999999998	-23071500000\\
-203000.000000004	12817500000\\
368000.000000004	22705100000\\
-56000.0000000045	-28808700000\\
-257000	3540299999.99998\\
-16999.9999999995	7445900000\\
183000.000000002	5127200000\\
-19000.0000000019	-8056600000.00001\\
-72999.9999999995	-732499999.999995\\
238000	10986200000\\
-311000	-20995900000\\
18000.0000000007	14160100000\\
385000.000000002	14282200000\\
-293000.000000001	-33447300000\\
-1000.00000000211	22949300000\\
18999.9999999983	-5859400000\\
-109999.999999998	-1342800000\\
-72000.0000000027	122099999.999995\\
272000.000000001	10620100000\\
-125999.999999999	-19165100000\\
-36999.9999999972	15136900000\\
128000	-4882999999.99999\\
110000	7324300000\\
-111000.000000003	-17700200000\\
185000.000000003	18188500000\\
-569000.000000003	-24414200000\\
36999.9999999999	23681900000\\
678000.000000002	9399099999.99999\\
-185000.000000003	-29540700000\\
-637999.999999998	1830799999.99999\\
327999.999999998	22216900000\\
274000.000000001	-5737300000.00001\\
-218000.000000001	-15502800000\\
-38000.0000000011	13305300000\\
403000.000000002	14893000000\\
-256000.000000001	-40405600000\\
-256000	21728800000\\
420000.000000002	18188100000\\
-218000	-33446800000\\
-130000.000000003	16479000000\\
1000.00000000211	1709500000\\
202000.000000001	2196799999.99998\\
-38000.0000000038	-6347300000\\
165000	9277100000\\
-255999.999999998	-22460900000\\
19000.000000001	22461100000\\
-147000.000000002	-12939600000\\
237999.999999997	14404300000\\
54000.000000002	-11962900000\\
-419000.000000001	-7446099999.99999\\
400000.000000001	26611000000\\
149000.000000001	-16601200000\\
-239000.000000002	-8789300000.00001\\
-201000	6347600000\\
475000	20141900000\\
-183000.000000003	-32715200000\\
-327999.999999998	16357600000\\
89999.9999999981	1465000000\\
275000.000000001	2197000000\\
-16999.9999999977	-6713800000\\
-21000.0000000017	1098700000\\
-327000	-3173900000\\
53999.9999999994	7812600000\\
127000.000000002	-4516700000\\
148000.000000001	5493300000\\
-109999.999999999	-10010000000\\
54999.9999999997	10498200000\\
346999.999999999	5127000000.00002\\
-421000.000000002	-30395600000\\
-474999.999999998	16113300000\\
530000	17334100000\\
312000.000000001	-244500000.000008\\
-312000.000000001	-31615700000\\
-275000.000000001	12450800000\\
478000.000000001	31982500000\\
-111999.999999999	-42236200000\\
-91000.0000000002	19042800000\\
-126999.999999999	-9643400000\\
-38000.0000000038	9277200000.00001\\
367000.000000002	11718800000\\
-202000.000000003	-27954000000\\
-365999.999999996	8544799999.99998\\
422000	17822300000\\
-1000.00000000122	-11596800000\\
347000.000000002	10376300000\\
-382000.000000001	-33691700000\\
-679000.000000002	15625000000\\
896999.999999999	33691700000\\
72999.9999999995	-35889000000\\
-420000.000000003	1953200000\\
-237999.999999999	7080300000\\
217999.999999999	4150000000\\
39000.0000000006	-6347200000\\
382000	16479000000\\
-181000.000000001	-27465300000\\
-74000.0000000007	15624600000\\
-36999.999999999	-4272400000\\
-255999.999999998	244400000\\
256000	7323900000\\
-54000.0000000011	-10375800000\\
-183999.999999998	4760600000\\
402999.999999999	7324500000\\
127999.999999999	-8667200000\\
-457000.000000001	-6103700000\\
-18999.9999999983	9522000000\\
146999.999999998	-366799999.999999\\
201000.000000002	2686000000\\
-347999.999999997	-13794100000\\
220000.000000001	18554600000\\
91000.0000000002	-13793900000\\
-511000.000000002	-366099999.999999\\
327000.000000001	10986200000\\
1999.99999999711	-7690300000\\
18000.0000000016	2197000000.00001\\
292000.000000001	9887999999.99999\\
111000.000000001	-12329200000\\
-495000.000000001	-13427800000\\
-72999.9999999995	19897500000\\
365999.999999996	2441399999.99999\\
-255999.999999998	-13427700000\\
-348000.000000002	3906200000\\
585000	12207000000\\
-72000.0000000009	-12939300000\\
-129000	366100000.000004\\
-17000.0000000003	4638700000\\
52999.9999999999	-488400000.000007\\
56999.9999999986	-1830899999.99999\\
-130000	-2319400000.00001\\
146999.999999998	9887800000.00001\\
-236999.999999999	-19775600000\\
585000	39062700000\\
-220000.000000001	-48340000000\\
-494000.000000001	20019700000\\
-17999.9999999998	3784000000\\
511999.999999999	9521600000.00001\\
-145999.999999999	-19165100000\\
146999.999999998	12085100000\\
-476999.999999999	-12207300000\\
367000	16113500000\\
35000.0000000001	-10009800000\\
-363999.999999999	-2319300000\\
-38000.0000000002	5493000000\\
402999.999999997	2807700000\\
-218999.999999998	-8300600000\\
71999.9999999974	5737000000\\
331000.000000002	10254000000\\
89999.9999999981	-20263500000\\
-932000	-5615300000.00001\\
492999.999999999	29174600000\\
421000	-11230300000\\
-622000.000000002	-18066300000\\
-108999.999999998	16479200000\\
620999.999999999	11230700000\\
-476000	-30639700000\\
221000.000000005	30883800000\\
383999.999999998	-5859400000.00001\\
-385000.000000001	-26001000000\\
-109000.000000001	21972600000\\
-93000	-6957800000.00001\\
93000	6469600000\\
383999.999999999	9765500000.00001\\
-201999.999999998	-25024300000\\
};
\addplot [color=mycolor2, line width=2.0pt, forget plot]
  table[row sep=crcr]{%
-220000	-220000\\
310999.999999996	310999.999999996\\
-201000	-201000\\
200000.000000002	200000.000000002\\
130999.999999999	130999.999999999\\
-534000	-534000\\
221999.999999999	221999.999999999\\
16999.9999999986	16999.9999999986\\
147000.000000002	147000.000000002\\
182999.999999997	182999.999999997\\
-347999.999999999	-347999.999999999\\
-91999.9999999978	-91999.9999999978\\
-346000.000000002	-346000.000000002\\
437000.000000002	437000.000000002\\
330999.999999996	330999.999999996\\
-165999.999999999	-165999.999999999\\
2000.00000000067	2000.00000000067\\
35000.0000000001	35000.0000000001\\
-622000.000000003	-622000.000000003\\
750000.000000003	750000.000000003\\
-127000.000000001	-127000.000000001\\
-512999.999999999	-512999.999999999\\
548999.999999999	548999.999999999\\
-109000	-109000\\
-331000.000000001	-331000.000000001\\
384999.999999998	384999.999999998\\
55000.0000000015	55000.0000000015\\
-439000.000000002	-439000.000000002\\
329000	329000\\
72999.9999999995	72999.9999999995\\
-145000	-145000\\
17000.0000000003	17000.0000000003\\
-128000.000000002	-128000.000000002\\
17999.9999999998	17999.9999999998\\
183999.999999999	183999.999999999\\
-110000.000000001	-110000.000000001\\
-999.999999998557	-999.999999998557\\
20000.0000000005	20000.0000000005\\
-112000.000000003	-112000.000000003\\
240000.000000002	240000.000000002\\
53999.9999999985	53999.9999999985\\
-275000.000000002	-275000.000000002\\
238000.000000002	238000.000000002\\
-273000.000000001	-273000.000000001\\
-56999.9999999995	-56999.9999999995\\
330999.999999999	330999.999999999\\
-165000	-165000\\
-129000.000000001	-129000.000000001\\
146999.999999998	146999.999999998\\
110000.000000002	110000.000000002\\
-293000	-293000\\
385000	385000\\
16999.9999999977	16999.9999999977\\
-311000	-311000\\
-90999.9999999993	-90999.9999999993\\
147000	147000\\
291999.999999997	291999.999999997\\
-328999.999999999	-328999.999999999\\
-165000.000000002	-165000.000000002\\
403000.000000001	403000.000000001\\
-110999.999999997	-110999.999999997\\
-218000.000000003	-218000.000000003\\
164000.000000001	164000.000000001\\
-109999.999999999	-109999.999999999\\
200999.999999999	200999.999999999\\
-72000.0000000009	-72000.0000000009\\
54000.0000000003	54000.0000000003\\
-146000	-146000\\
199999.999999998	199999.999999998\\
-161999.999999996	-161999.999999996\\
70999.9999999979	70999.9999999979\\
-55000.0000000024	-55000.0000000024\\
-16999.9999999968	-16999.9999999968\\
-202000	-202000\\
183000	183000\\
238000.000000001	238000.000000001\\
-72999.9999999986	-72999.9999999986\\
73999.9999999989	73999.9999999989\\
-442000.000000002	-442000.000000002\\
149000.000000003	149000.000000003\\
236999.999999999	236999.999999999\\
19000.0000000028	19000.0000000028\\
-312000.000000004	-312000.000000004\\
18000.0000000025	18000.0000000025\\
422000.000000001	422000.000000001\\
-201999.999999998	-201999.999999998\\
18999.9999999992	18999.9999999992\\
-111000	-111000\\
-72000.0000000018	-72000.0000000018\\
-18999.9999999983	-18999.9999999983\\
202000	202000\\
34999.9999999966	34999.9999999966\\
-198999.999999996	-198999.999999996\\
-112000.000000004	-112000.000000004\\
56000.0000000009	56000.0000000009\\
257000.000000001	257000.000000001\\
-93000.0000000026	-93000.0000000026\\
-146000.000000001	-146000.000000001\\
165000.000000001	165000.000000001\\
-165000.000000001	-165000.000000001\\
275000.000000001	275000.000000001\\
-201999.999999999	-201999.999999999\\
148000	148000\\
-75000.0000000028	-75000.0000000028\\
-201000	-201000\\
329999.999999999	329999.999999999\\
-202000.000000003	-202000.000000003\\
-72999.9999999977	-72999.9999999977\\
111000.000000001	111000.000000001\\
162999.999999998	162999.999999998\\
-274000.000000001	-274000.000000001\\
202000.000000002	202000.000000002\\
-184000.000000003	-184000.000000003\\
56000.0000000018	56000.0000000018\\
163999.999999998	163999.999999998\\
-220000	-220000\\
-17999.9999999989	-17999.9999999989\\
-17000.0000000012	-17000.0000000012\\
254000	254000\\
-89999.9999999998	-89999.9999999998\\
201000	201000\\
-457000.000000001	-457000.000000001\\
-1999.99999999889	-1999.99999999889\\
515000	515000\\
-367999.999999999	-367999.999999999\\
75000.0000000002	75000.0000000002\\
235999.999999997	235999.999999997\\
-235999.999999998	-235999.999999998\\
-240000.000000003	-240000.000000003\\
202999.999999999	202999.999999999\\
-129000	-129000\\
201000	201000\\
111000.000000002	111000.000000002\\
18000.0000000033	18000.0000000033\\
-238000.000000002	-238000.000000002\\
54000.0000000003	54000.0000000003\\
19000.0000000028	19000.0000000028\\
999.999999997669	999.999999997669\\
144999.999999997	144999.999999997\\
-182999.999999999	-182999.999999999\\
-36000.0000000005	-36000.0000000005\\
72999.9999999977	72999.9999999977\\
-54999.9999999988	-54999.9999999988\\
-73999.9999999998	-73999.9999999998\\
2000.00000000422	2000.00000000422\\
400999.999999998	400999.999999998\\
-107999.999999998	-107999.999999998\\
-259000.000000004	-259000.000000004\\
56999.9999999995	56999.9999999995\\
182000	182000\\
-144999.999999999	-144999.999999999\\
-148000.000000001	-148000.000000001\\
-109999.999999999	-109999.999999999\\
458999.999999998	458999.999999998\\
-202000.000000001	-202000.000000001\\
-18000.0000000016	-18000.0000000016\\
54000.0000000011	54000.0000000011\\
37999.9999999976	37999.9999999976\\
-999.999999998557	-999.999999998557\\
-73999.9999999989	-73999.9999999989\\
-254000	-254000\\
-20000.0000000005	-20000.0000000005\\
329000	329000\\
112000.000000003	112000.000000003\\
71999.9999999992	71999.9999999992\\
-129000.000000004	-129000.000000004\\
-144999.999999998	-144999.999999998\\
89999.9999999981	89999.9999999981\\
-456000	-456000\\
511999.999999999	511999.999999999\\
-165000	-165000\\
-17999.9999999998	-17999.9999999998\\
-56000.0000000018	-56000.0000000018\\
203000.000000002	203000.000000002\\
218000.000000002	218000.000000002\\
-291000	-291000\\
145000	145000\\
-274000	-274000\\
0	0\\
181999.999999999	181999.999999999\\
57000.0000000022	57000.0000000022\\
-111999.999999999	-111999.999999999\\
-89999.999999999	-89999.999999999\\
146000.000000001	146000.000000001\\
-146000	-146000\\
217999.999999997	217999.999999997\\
-382999.999999998	-382999.999999998\\
275000.000000001	275000.000000001\\
199999.999999999	199999.999999999\\
-53999.9999999976	-53999.9999999976\\
-329000.000000002	-329000.000000002\\
-130000.000000002	-130000.000000002\\
349000	349000\\
-17999.9999999989	-17999.9999999989\\
-1000.00000000033	-1000.00000000033\\
20000.0000000031	20000.0000000031\\
89999.9999999963	89999.9999999963\\
-182999.999999998	-182999.999999998\\
-73000.0000000004	-73000.0000000004\\
56000.0000000009	56000.0000000009\\
308999.999999999	308999.999999999\\
-125999.999999997	-125999.999999997\\
-477000	-477000\\
458000	458000\\
18000.0000000016	18000.0000000016\\
-310999.999999999	-310999.999999999\\
110000	110000\\
218999.999999998	218999.999999998\\
-108999.999999997	-108999.999999997\\
-183000.000000002	-183000.000000002\\
217999.999999997	217999.999999997\\
-180999.999999997	-180999.999999997\\
-19000.0000000037	-19000.0000000037\\
347000.000000001	347000.000000001\\
-108999.999999999	-108999.999999999\\
-201000.000000001	-201000.000000001\\
-93000	-93000\\
19999.9999999996	19999.9999999996\\
511999.999999999	511999.999999999\\
-147000.000000002	-147000.000000002\\
-256000	-256000\\
-110000	-110000\\
146999.999999998	146999.999999998\\
146000	146000\\
-218999.999999997	-218999.999999997\\
-19000.0000000046	-19000.0000000046\\
91000.0000000011	91000.0000000011\\
37999.9999999985	37999.9999999985\\
182000	182000\\
-494000.000000001	-494000.000000001\\
312000	312000\\
144999.999999999	144999.999999999\\
-145000	-145000\\
-38000.0000000011	-38000.0000000011\\
-145999.999999997	-145999.999999997\\
458999.999999999	458999.999999999\\
-56999.9999999986	-56999.9999999986\\
-584000.000000002	-584000.000000002\\
254999.999999999	254999.999999999\\
18999.9999999983	18999.9999999983\\
201000	201000\\
-17999.9999999998	-17999.9999999998\\
-330000.000000003	-330000.000000003\\
293000.000000002	293000.000000002\\
73999.9999999998	73999.9999999998\\
-274999.999999999	-274999.999999999\\
163999.999999998	163999.999999998\\
-16999.9999999977	-16999.9999999977\\
-93000.0000000044	-93000.0000000044\\
276000.000000003	276000.000000003\\
-349000.000000002	-349000.000000002\\
220999.999999998	220999.999999998\\
-128999.999999999	-128999.999999999\\
-220999.999999997	-220999.999999997\\
405999.999999999	405999.999999999\\
-204000.000000001	-204000.000000001\\
54999.9999999997	54999.9999999997\\
313000.000000001	313000.000000001\\
-130000.000000001	-130000.000000001\\
-219000.000000001	-219000.000000001\\
-108999.999999999	-108999.999999999\\
34999.9999999993	34999.9999999993\\
350000	350000\\
-314000.000000001	-314000.000000001\\
-34000.0000000007	-34000.0000000007\\
-38999.9999999997	-38999.9999999997\\
368000.000000001	368000.000000001\\
-292999.999999997	-292999.999999997\\
346999.999999998	346999.999999998\\
-293000.000000002	-293000.000000002\\
219999.999999999	219999.999999999\\
-309999.999999996	-309999.999999996\\
161999.999999998	161999.999999998\\
-144000.000000002	-144000.000000002\\
36000.0000000005	36000.0000000005\\
-129000.000000002	-129000.000000002\\
312000	312000\\
-72999.9999999986	-72999.9999999986\\
-37000.0000000035	-37000.0000000035\\
-164999.999999999	-164999.999999999\\
310999.999999999	310999.999999999\\
75000.0000000037	75000.0000000037\\
-369000.000000004	-369000.000000004\\
112000.000000003	112000.000000003\\
-37000.0000000008	-37000.0000000008\\
256000	256000\\
-54999.9999999997	-54999.9999999997\\
0	0\\
-181999.999999999	-181999.999999999\\
-112000.000000002	-112000.000000002\\
39000.0000000015	39000.0000000015\\
126000	126000\\
74999.9999999984	74999.9999999984\\
-184999.999999998	-184999.999999998\\
-34000.0000000025	-34000.0000000025\\
309000	309000\\
-91000.0000000019	-91000.0000000019\\
-163999.999999998	-163999.999999998\\
309999.999999997	309999.999999997\\
-89999.999999999	-89999.999999999\\
-569000.000000001	-569000.000000001\\
349000.000000001	349000.000000001\\
309999.999999997	309999.999999997\\
-236999.999999996	-236999.999999996\\
72999.9999999986	72999.9999999986\\
73000.0000000004	73000.0000000004\\
-329000.000000001	-329000.000000001\\
73000.0000000004	73000.0000000004\\
273999.999999997	273999.999999997\\
18999.9999999983	18999.9999999983\\
-457999.999999999	-457999.999999999\\
493999.999999999	493999.999999999\\
-163999.999999999	-163999.999999999\\
-165000.000000004	-165000.000000004\\
201000	201000\\
17999.9999999998	17999.9999999998\\
-108999.999999999	-108999.999999999\\
-218999.999999998	-218999.999999998\\
143999.999999998	143999.999999998\\
387000.000000001	387000.000000001\\
-93000.0000000008	-93000.0000000008\\
-346999.999999999	-346999.999999999\\
71999.9999999956	71999.9999999956\\
-16999.9999999995	-16999.9999999995\\
90000.0000000016	90000.0000000016\\
149000	149000\\
-204000.000000001	-204000.000000001\\
-72000.0000000018	-72000.0000000018\\
183000.000000002	183000.000000002\\
146999.999999998	146999.999999998\\
-221000	-221000\\
-182000	-182000\\
53999.9999999976	53999.9999999976\\
312000	312000\\
-109999.999999998	-109999.999999998\\
146000	146000\\
-274000.000000001	-274000.000000001\\
-54999.9999999962	-54999.9999999962\\
330000	330000\\
-148000.000000006	-148000.000000006\\
-126999.999999998	-126999.999999998\\
17999.9999999989	17999.9999999989\\
-311000.000000001	-311000.000000001\\
365999.999999999	365999.999999999\\
110000.000000002	110000.000000002\\
72999.9999999995	72999.9999999995\\
-238000.000000003	-238000.000000003\\
0	0\\
257000.000000002	257000.000000002\\
-294000.000000003	-294000.000000003\\
37000.0000000008	37000.0000000008\\
165000.000000001	165000.000000001\\
-274999.999999997	-274999.999999997\\
35999.9999999996	35999.9999999996\\
130000	130000\\
310999.999999999	310999.999999999\\
-313000.000000001	-313000.000000001\\
-16999.9999999995	-16999.9999999995\\
-110999.999999999	-110999.999999999\\
257999.999999999	257999.999999999\\
-495000.000000002	-495000.000000002\\
274000	274000\\
421999.999999999	421999.999999999\\
-421999.999999999	-421999.999999999\\
54999.9999999988	54999.9999999988\\
129000.000000004	129000.000000004\\
54999.9999999988	54999.9999999988\\
-74000.0000000007	-74000.0000000007\\
-165000.000000002	-165000.000000002\\
18999.9999999983	18999.9999999983\\
17999.9999999998	17999.9999999998\\
37000.0000000017	37000.0000000017\\
71999.9999999983	71999.9999999983\\
-180999.999999997	-180999.999999997\\
144999.999999999	144999.999999999\\
-36999.9999999999	-36999.9999999999\\
294000.000000002	294000.000000002\\
-384000	-384000\\
-2000.00000000067	-2000.00000000067\\
313000.000000001	313000.000000001\\
-348999.999999998	-348999.999999998\\
239000	239000\\
110000	110000\\
-166000.000000004	-166000.000000004\\
-238000.000000001	-238000.000000001\\
-34999.9999999975	-34999.9999999975\\
400999.999999998	400999.999999998\\
-17000.0000000012	-17000.0000000012\\
-330000.000000001	-330000.000000001\\
274000.000000002	274000.000000002\\
999.999999999446	999.999999999446\\
-293999.999999997	-293999.999999997\\
220999.999999997	220999.999999997\\
72000.0000000018	72000.0000000018\\
-311000.000000002	-311000.000000002\\
148000.000000003	148000.000000003\\
363999.999999995	363999.999999995\\
-547999.999999998	-547999.999999998\\
127999.999999998	127999.999999998\\
383999.999999999	383999.999999999\\
-330000.000000001	-330000.000000001\\
313000.000000005	313000.000000005\\
-312000.000000002	-312000.000000002\\
164000	164000\\
73999.9999999989	73999.9999999989\\
-458000.000000002	-458000.000000002\\
330000	330000\\
89999.999999999	89999.999999999\\
-344999.999999998	-344999.999999998\\
435999.999999998	435999.999999998\\
-144000.000000002	-144000.000000002\\
-257000	-257000\\
-165000	-165000\\
439999.999999997	439999.999999997\\
-37999.9999999976	-37999.9999999976\\
39000.0000000024	39000.0000000024\\
-39000.0000000041	-39000.0000000041\\
-91000.0000000002	-91000.0000000002\\
111000.000000001	111000.000000001\\
-2000.00000000067	-2000.00000000067\\
-161999.999999998	-161999.999999998\\
125999.999999998	125999.999999998\\
-219999.999999999	-219999.999999999\\
405000.000000003	405000.000000003\\
-185000.000000004	-185000.000000004\\
-274000	-274000\\
165000.000000002	165000.000000002\\
329000	329000\\
-218999.999999998	-218999.999999998\\
-73000.0000000022	-73000.0000000022\\
-38000.0000000011	-38000.0000000011\\
164999.999999999	164999.999999999\\
-71999.9999999992	-71999.9999999992\\
-221000.000000002	-221000.000000002\\
185000.000000003	185000.000000003\\
70999.9999999979	70999.9999999979\\
18999.9999999983	18999.9999999983\\
-127999.999999998	-127999.999999998\\
73999.9999999998	73999.9999999998\\
236999.999999997	236999.999999997\\
-438999.999999998	-438999.999999998\\
-147000.000000003	-147000.000000003\\
367000.000000003	367000.000000003\\
164999.999999997	164999.999999997\\
-129999.999999999	-129999.999999999\\
-89000.0000000022	-89000.0000000022\\
107000.000000002	107000.000000002\\
-143999.999999998	-143999.999999998\\
72999.9999999977	72999.9999999977\\
-167000.000000001	-167000.000000001\\
204000.000000001	204000.000000001\\
144999.999999998	144999.999999998\\
-475000	-475000\\
310000	310000\\
-73000.0000000013	-73000.0000000013\\
56000.0000000027	56000.0000000027\\
35999.999999996	35999.999999996\\
-183999.999999997	-183999.999999997\\
92999.9999999991	92999.9999999991\\
54000.0000000011	54000.0000000011\\
312000.000000001	312000.000000001\\
-184000.000000001	-184000.000000001\\
-164000.000000001	-164000.000000001\\
-17999.9999999971	-17999.9999999971\\
-276000.000000002	-276000.000000002\\
385000	385000\\
-292000.000000002	-292000.000000002\\
383000.000000001	383000.000000001\\
-89999.9999999998	-89999.9999999998\\
-203000.000000001	-203000.000000001\\
129000	129000\\
-16999.9999999977	-16999.9999999977\\
51999.9999999987	51999.9999999987\\
-179999.999999999	-179999.999999999\\
308999.999999997	308999.999999997\\
19000.0000000028	19000.0000000028\\
-128000.000000001	-128000.000000001\\
-164999.999999999	-164999.999999999\\
36999.999999999	36999.999999999\\
-18000.0000000007	-18000.0000000007\\
273000	273000\\
-199999.999999999	-199999.999999999\\
-54999.999999997	-54999.999999997\\
347999.999999999	347999.999999999\\
-641000.000000003	-641000.000000003\\
236999.999999999	236999.999999999\\
403999.999999999	403999.999999999\\
-293999.999999999	-293999.999999999\\
-310000	-310000\\
641000.000000001	641000.000000001\\
-276000.000000003	-276000.000000003\\
-237999.999999999	-237999.999999999\\
294000.000000003	294000.000000003\\
164999.999999998	164999.999999998\\
-240000.000000001	-240000.000000001\\
-162000	-162000\\
-112000.000000001	-112000.000000001\\
146999.999999998	146999.999999998\\
293000.000000001	293000.000000001\\
-109999.999999998	-109999.999999998\\
56000	56000\\
16999.9999999959	16999.9999999959\\
-273999.999999998	-273999.999999998\\
181999.999999998	181999.999999998\\
-125999.999999997	-125999.999999997\\
70999.9999999962	70999.9999999962\\
-180999.999999996	-180999.999999996\\
199999.999999996	199999.999999996\\
219000.000000001	219000.000000001\\
-455999.999999998	-455999.999999998\\
109999.999999999	109999.999999999\\
180999.999999997	180999.999999997\\
-125999.999999998	-125999.999999998\\
34999.9999999984	34999.9999999984\\
-17999.9999999989	-17999.9999999989\\
349000.000000003	349000.000000003\\
-312000.000000001	-312000.000000001\\
-127999.999999998	-127999.999999998\\
181999.999999999	181999.999999999\\
93999.9999999985	93999.9999999985\\
-314000	-314000\\
-90000.0000000025	-90000.0000000025\\
366000.000000002	366000.000000002\\
219000	219000\\
-237000.000000002	-237000.000000002\\
-147000.000000002	-147000.000000002\\
183000	183000\\
-383999.999999999	-383999.999999999\\
-18999.9999999992	-18999.9999999992\\
348999.999999999	348999.999999999\\
-148000.000000001	-148000.000000001\\
239000	239000\\
-37000.0000000008	-37000.0000000008\\
-236999.999999999	-236999.999999999\\
126000.000000002	126000.000000002\\
-53000.0000000035	-53000.0000000035\\
17999.9999999998	17999.9999999998\\
-110999.999999999	-110999.999999999\\
37999.9999999985	37999.9999999985\\
219000.000000003	219000.000000003\\
-53999.9999999994	-53999.9999999994\\
-111000.000000004	-111000.000000004\\
-54999.9999999988	-54999.9999999988\\
164999.999999998	164999.999999998\\
-90999.9999999966	-90999.9999999966\\
110000.000000002	110000.000000002\\
35999.9999999987	35999.9999999987\\
-55000.0000000015	-55000.0000000015\\
-164000.000000001	-164000.000000001\\
-202999.999999999	-202999.999999999\\
368000.000000002	368000.000000002\\
181999.999999998	181999.999999998\\
-181999.999999998	-181999.999999998\\
70999.9999999979	70999.9999999979\\
-199000.000000001	-199000.000000001\\
-184000.000000002	-184000.000000002\\
385000.000000002	385000.000000002\\
-495999.999999999	-495999.999999999\\
294999.999999999	294999.999999999\\
199999.999999998	199999.999999998\\
-73000.0000000013	-73000.0000000013\\
92000.0000000005	92000.0000000005\\
-200999.999999999	-200999.999999999\\
-147999.999999999	-147999.999999999\\
165999.999999999	165999.999999999\\
37000.0000000017	37000.0000000017\\
90000.0000000016	90000.0000000016\\
-91000.0000000019	-91000.0000000019\\
-272999.999999999	-272999.999999999\\
-56999.9999999995	-56999.9999999995\\
494999.999999996	494999.999999996\\
36000.0000000022	36000.0000000022\\
-52999.999999999	-52999.999999999\\
-367999.999999999	-367999.999999999\\
73999.9999999989	73999.9999999989\\
311000.000000003	311000.000000003\\
-146000.000000001	-146000.000000001\\
90999.9999999984	90999.9999999984\\
-219999.999999998	-219999.999999998\\
73999.9999999981	73999.9999999981\\
164000.000000001	164000.000000001\\
-181999.999999999	-181999.999999999\\
146000.000000001	146000.000000001\\
-165000.000000004	-165000.000000004\\
-330000.000000001	-330000.000000001\\
678000.000000003	678000.000000003\\
-329000.000000002	-329000.000000002\\
-56000.0000000009	-56000.0000000009\\
146999.999999998	146999.999999998\\
-36999.9999999999	-36999.9999999999\\
-72000.0000000001	-72000.0000000001\\
34999.9999999993	34999.9999999993\\
36999.999999999	36999.999999999\\
56000	56000\\
-1000.00000000033	-1000.00000000033\\
-238000.000000003	-238000.000000003\\
458000.000000003	458000.000000003\\
-128000.000000002	-128000.000000002\\
-512999.999999995	-512999.999999995\\
90999.9999999975	90999.9999999975\\
624000	624000\\
-220999.999999997	-220999.999999997\\
-512000	-512000\\
165000	165000\\
329000	329000\\
-147000	-147000\\
165999.999999998	165999.999999998\\
-366000	-366000\\
383000	383000\\
73999.9999999989	73999.9999999989\\
-548999.999999999	-548999.999999999\\
640000	640000\\
-384000.000000002	-384000.000000002\\
74000.0000000016	74000.0000000016\\
18000.0000000007	18000.0000000007\\
-331000	-331000\\
459000.000000001	459000.000000001\\
-402000.000000001	-402000.000000001\\
108000.000000001	108000.000000001\\
348000.000000001	348000.000000001\\
-53000.0000000026	-53000.0000000026\\
-258000	-258000\\
165000	165000\\
-237000	-237000\\
346000.000000004	346000.000000004\\
-200000.000000005	-200000.000000005\\
-35999.9999999969	-35999.9999999969\\
108000.000000002	108000.000000002\\
-127000	-127000\\
91000.0000000002	91000.0000000002\\
-237000.000000001	-237000.000000001\\
382999.999999998	382999.999999998\\
-54000.0000000003	-54000.0000000003\\
-128000.000000002	-128000.000000002\\
37000.0000000017	37000.0000000017\\
-111999.999999999	-111999.999999999\\
-236000.000000002	-236000.000000002\\
384000.000000001	384000.000000001\\
330000.000000001	330000.000000001\\
-404000.000000003	-404000.000000003\\
93000.0000000026	93000.0000000026\\
-404000.000000003	-404000.000000003\\
475999.999999999	475999.999999999\\
73999.9999999998	73999.9999999998\\
-420999.999999999	-420999.999999999\\
291999.999999997	291999.999999997\\
-236999.999999997	-236999.999999997\\
364999.999999997	364999.999999997\\
-254999.999999998	-254999.999999998\\
-165000.000000001	-165000.000000001\\
200000.000000002	200000.000000002\\
148000	148000\\
-165000.000000002	-165000.000000002\\
-55999.9999999965	-55999.9999999965\\
-72000.0000000027	-72000.0000000027\\
310999.999999999	310999.999999999\\
-330999.999999998	-330999.999999998\\
94000.000000003	94000.000000003\\
345999.999999998	345999.999999998\\
-71999.9999999983	-71999.9999999983\\
-220000.000000001	-220000.000000001\\
-313000.000000001	-313000.000000001\\
314000	314000\\
235999.999999999	235999.999999999\\
-346999.999999998	-346999.999999998\\
347999.999999998	347999.999999998\\
-54999.9999999997	-54999.9999999997\\
-385999.999999999	-385999.999999999\\
-181000.000000002	-181000.000000002\\
622000.000000001	622000.000000001\\
-91999.9999999996	-91999.9999999996\\
-348000.000000002	-348000.000000002\\
330000.000000002	330000.000000002\\
2.66453525910038e-09	2.66453525910038e-09\\
-36000.0000000031	-36000.0000000031\\
-203000	-203000\\
-18000.0000000007	-18000.0000000007\\
348999.999999998	348999.999999998\\
-274999.999999999	-274999.999999999\\
126999.999999998	126999.999999998\\
20000.0000000022	20000.0000000022\\
-202000.000000002	-202000.000000002\\
255999.999999998	255999.999999998\\
-220000	-220000\\
-109000	-109000\\
237000	237000\\
57000.000000003	57000.000000003\\
-3000.00000000455	-3000.00000000455\\
2000.00000000333	2000.00000000333\\
-55000.0000000015	-55000.0000000015\\
-184000.000000001	-184000.000000001\\
56000.0000000009	56000.0000000009\\
72999.9999999977	72999.9999999977\\
-19999.9999999996	-19999.9999999996\\
39000.0000000041	39000.0000000041\\
181999.999999997	181999.999999997\\
-17999.9999999998	-17999.9999999998\\
-532000	-532000\\
258000.000000003	258000.000000003\\
456999.999999996	456999.999999996\\
-385000.000000001	-385000.000000001\\
-91999.9999999987	-91999.9999999987\\
277000	277000\\
-94000.0000000021	-94000.0000000021\\
-219000	-219000\\
239000	239000\\
-130000	-130000\\
-33999.9999999989	-33999.9999999989\\
271999.999999998	271999.999999998\\
-273000.000000001	-273000.000000001\\
108999.999999999	108999.999999999\\
-17999.9999999998	-17999.9999999998\\
72999.9999999986	72999.9999999986\\
-200999.999999999	-200999.999999999\\
255999.999999998	255999.999999998\\
-146000.000000002	-146000.000000002\\
-91999.9999999996	-91999.9999999996\\
182999.999999999	182999.999999999\\
147000.000000001	147000.000000001\\
-145999.999999998	-145999.999999998\\
-167000	-167000\\
-199000.000000002	-199000.000000002\\
-73999.9999999989	-73999.9999999989\\
529999.999999997	529999.999999997\\
55999.9999999991	55999.9999999991\\
1.77635683940025e-09	1.77635683940025e-09\\
-385000.000000002	-385000.000000002\\
164999.999999999	164999.999999999\\
55000.0000000006	55000.0000000006\\
72999.9999999977	72999.9999999977\\
-348000	-348000\\
238999.999999999	238999.999999999\\
17000.0000000012	17000.0000000012\\
-109000.000000003	-109000.000000003\\
165000.000000002	165000.000000002\\
-92000.0000000014	-92000.0000000014\\
73000.0000000004	73000.0000000004\\
-201999.999999999	-201999.999999999\\
-126000.000000001	-126000.000000001\\
564999.999999998	564999.999999998\\
-382000.000000001	-382000.000000001\\
126000.000000001	126000.000000001\\
-200000.000000002	-200000.000000002\\
255000.000000001	255000.000000001\\
-217999.999999998	-217999.999999998\\
273999.999999999	273999.999999999\\
-184000.000000001	-184000.000000001\\
-34999.9999999984	-34999.9999999984\\
363999.999999996	363999.999999996\\
-712000	-712000\\
603000	603000\\
-291999.999999999	-291999.999999999\\
53999.9999999976	53999.9999999976\\
-18000.0000000007	-18000.0000000007\\
220000.000000001	220000.000000001\\
-293000.000000004	-293000.000000004\\
73000.0000000004	73000.0000000004\\
201000	201000\\
-71999.9999999983	-71999.9999999983\\
-93000.0000000044	-93000.0000000044\\
-72999.9999999968	-72999.9999999968\\
166000.000000002	166000.000000002\\
-74000.0000000025	-74000.0000000025\\
37000.0000000017	37000.0000000017\\
-166000.000000004	-166000.000000004\\
203000.000000005	203000.000000005\\
-349000.000000005	-349000.000000005\\
293000.000000003	293000.000000003\\
202000	202000\\
-183000.000000001	-183000.000000001\\
-93000	-93000\\
313000.000000002	313000.000000002\\
-365999.999999999	-365999.999999999\\
88999.999999996	88999.999999996\\
21000.0000000017	21000.0000000017\\
-129000.000000003	-129000.000000003\\
-36999.999999999	-36999.999999999\\
219999.999999998	219999.999999998\\
-18999.9999999992	-18999.9999999992\\
92999.9999999991	92999.9999999991\\
-20000.0000000005	-20000.0000000005\\
-437999.999999999	-437999.999999999\\
365999.999999999	365999.999999999\\
16999.9999999995	16999.9999999995\\
111000.000000004	111000.000000004\\
-476000.000000002	-476000.000000002\\
568000.000000002	568000.000000002\\
34999.9999999975	34999.9999999975\\
-382999.999999999	-382999.999999999\\
16999.9999999959	16999.9999999959\\
-236999.999999998	-236999.999999998\\
311999.999999999	311999.999999999\\
-20999.9999999981	-20999.9999999981\\
241000	241000\\
-111000.000000002	-111000.000000002\\
17000.0000000003	17000.0000000003\\
-291000	-291000\\
35999.9999999987	35999.9999999987\\
108999.999999998	108999.999999998\\
203000	203000\\
-350000	-350000\\
239999.999999997	239999.999999997\\
36000.0000000014	36000.0000000014\\
-421000	-421000\\
328000.000000002	328000.000000002\\
21000.0000000008	21000.0000000008\\
34999.9999999957	34999.9999999957\\
-54999.9999999979	-54999.9999999979\\
-55000.0000000015	-55000.0000000015\\
1000.00000000122	1000.00000000122\\
181999.999999997	181999.999999997\\
-310999.999999997	-310999.999999997\\
129000	129000\\
237000.000000001	237000.000000001\\
-273999.999999997	-273999.999999997\\
109999.999999999	109999.999999999\\
-92000.0000000049	-92000.0000000049\\
-91999.9999999978	-91999.9999999978\\
367000	367000\\
-165000.000000004	-165000.000000004\\
-183999.999999997	-183999.999999997\\
167000.000000001	167000.000000001\\
107999.999999998	107999.999999998\\
-403000.000000001	-403000.000000001\\
129000	129000\\
330000	330000\\
-201999.999999999	-201999.999999999\\
-257000.000000001	-257000.000000001\\
623999.999999999	623999.999999999\\
-349999.999999999	-349999.999999999\\
3000.00000000011	3000.00000000011\\
-186000.000000001	-186000.000000001\\
20000.0000000005	20000.0000000005\\
438999.999999999	438999.999999999\\
-293000.000000003	-293000.000000003\\
-238999.999999997	-238999.999999997\\
496999.999999999	496999.999999999\\
-131000.000000001	-131000.000000001\\
-346000.000000001	-346000.000000001\\
236000	236000\\
76000.0000000014	76000.0000000014\\
-222000.000000003	-222000.000000003\\
148000.000000001	148000.000000001\\
16999.9999999995	16999.9999999995\\
-182000.000000001	-182000.000000001\\
91000.0000000011	91000.0000000011\\
90999.9999999993	90999.9999999993\\
-109000	-109000\\
202000	202000\\
-185000	-185000\\
38000.0000000002	38000.0000000002\\
110000.000000002	110000.000000002\\
72999.9999999986	72999.9999999986\\
-294000.000000003	-294000.000000003\\
129000.000000001	129000.000000001\\
-17999.9999999989	-17999.9999999989\\
-128000.000000001	-128000.000000001\\
364999.999999998	364999.999999998\\
-292000.000000002	-292000.000000002\\
72999.9999999995	72999.9999999995\\
54999.9999999988	54999.9999999988\\
-274999.999999999	-274999.999999999\\
385000.000000001	385000.000000001\\
-146999.999999999	-146999.999999999\\
1000.00000000122	1000.00000000122\\
126999.999999999	126999.999999999\\
-91000.0000000037	-91000.0000000037\\
-127999.999999998	-127999.999999998\\
-202000.000000003	-202000.000000003\\
330000.000000001	330000.000000001\\
-37000.0000000008	-37000.0000000008\\
38000.0000000011	38000.0000000011\\
-130000.000000002	-130000.000000002\\
-201000	-201000\\
404000.000000002	404000.000000002\\
108000.000000002	108000.000000002\\
-236000.000000001	-236000.000000001\\
-38000.0000000002	-38000.0000000002\\
201999.999999999	201999.999999999\\
-312000.000000001	-312000.000000001\\
93000	93000\\
126000	126000\\
-126000.000000001	-126000.000000001\\
-38000.0000000011	-38000.0000000011\\
-17999.9999999971	-17999.9999999971\\
294000	294000\\
-93000	-93000\\
-201000.000000003	-201000.000000003\\
311000.000000003	311000.000000003\\
-328999.999999998	-328999.999999998\\
183999.999999997	183999.999999997\\
-94000.0000000003	-94000.0000000003\\
-107999.999999996	-107999.999999996\\
182999.999999999	182999.999999999\\
366000	366000\\
-623000.000000002	-623000.000000002\\
145999.999999998	145999.999999998\\
-52999.9999999982	-52999.9999999982\\
180999.999999999	180999.999999999\\
148000	148000\\
-404000.000000001	-404000.000000001\\
1000.00000000122	1000.00000000122\\
35999.9999999987	35999.9999999987\\
459000.000000003	459000.000000003\\
-130000.000000003	-130000.000000003\\
-145999.999999999	-145999.999999999\\
-128000	-128000\\
56000.0000000009	56000.0000000009\\
236000.000000001	236000.000000001\\
-236000	-236000\\
-2000.00000000067	-2000.00000000067\\
222000.000000002	222000.000000002\\
-185000.000000002	-185000.000000002\\
-182000	-182000\\
402000.000000001	402000.000000001\\
-128000	-128000\\
-346999.999999998	-346999.999999998\\
456999.999999997	456999.999999997\\
-293000.000000002	-293000.000000002\\
109999.999999999	109999.999999999\\
-55000.0000000006	-55000.0000000006\\
-17999.999999998	-17999.999999998\\
237999.999999999	237999.999999999\\
-74000.0000000016	-74000.0000000016\\
-256000	-256000\\
166000.000000001	166000.000000001\\
-39000.0000000024	-39000.0000000024\\
131000	131000\\
-20999.999999999	-20999.999999999\\
-309000.000000001	-309000.000000001\\
345999.999999999	345999.999999999\\
56000.0000000009	56000.0000000009\\
-439000.000000001	-439000.000000001\\
382999.999999999	382999.999999999\\
56000.0000000009	56000.0000000009\\
-128000.000000001	-128000.000000001\\
-19000.0000000019	-19000.0000000019\\
-18000.0000000007	-18000.0000000007\\
-366000	-366000\\
475999.999999998	475999.999999998\\
73000.0000000013	73000.0000000013\\
-128000.000000001	-128000.000000001\\
-201999.999999997	-201999.999999997\\
-73000.0000000022	-73000.0000000022\\
313000.000000001	313000.000000001\\
-149000.000000003	-149000.000000003\\
111000.000000001	111000.000000001\\
37000.0000000008	37000.0000000008\\
-111000.000000002	-111000.000000002\\
203000.000000001	203000.000000001\\
70999.9999999971	70999.9999999971\\
-619999.999999999	-619999.999999999\\
327999.999999999	327999.999999999\\
-8.88178419700125e-10	-8.88178419700125e-10\\
-145000	-145000\\
290999.999999999	290999.999999999\\
-200999.999999998	-200999.999999998\\
368000	368000\\
-241000	-241000\\
-344999.999999998	-344999.999999998\\
583999.999999999	583999.999999999\\
-182000.000000001	-182000.000000001\\
-92000.0000000005	-92000.0000000005\\
-202000.000000002	-202000.000000002\\
129000.000000004	129000.000000004\\
-55000.0000000015	-55000.0000000015\\
458000.000000002	458000.000000002\\
-294000.000000004	-294000.000000004\\
-146000	-146000\\
56000	56000\\
-38000.0000000011	-38000.0000000011\\
-54999.9999999997	-54999.9999999997\\
385000.000000001	385000.000000001\\
-181999.999999999	-181999.999999999\\
107999.999999997	107999.999999997\\
-109000.000000001	-109000.000000001\\
-293000	-293000\\
548999.999999998	548999.999999998\\
-622000.000000001	-622000.000000001\\
310999.999999999	310999.999999999\\
17999.9999999998	17999.9999999998\\
-310000.000000001	-310000.000000001\\
364000	364000\\
-34999.9999999975	-34999.9999999975\\
-54999.9999999997	-54999.9999999997\\
183000.000000003	183000.000000003\\
127999.999999997	127999.999999997\\
-440000.000000003	-440000.000000003\\
91999.9999999996	91999.9999999996\\
-90999.9999999984	-90999.9999999984\\
90999.9999999966	90999.9999999966\\
0	0\\
292999.999999998	292999.999999998\\
-110000	-110000\\
-311000.000000001	-311000.000000001\\
91000.0000000002	91000.0000000002\\
275999.999999999	275999.999999999\\
-130000	-130000\\
-52999.999999999	-52999.999999999\\
52999.9999999964	52999.9999999964\\
-126999.999999999	-126999.999999999\\
72999.9999999986	72999.9999999986\\
-236999.999999999	-236999.999999999\\
162000.000000002	162000.000000002\\
185999.999999999	185999.999999999\\
16999.9999999968	16999.9999999968\\
-128000	-128000\\
220000.000000001	220000.000000001\\
-165999.999999999	-165999.999999999\\
-328000	-328000\\
-1000.00000000211	-1000.00000000211\\
239000.000000001	239000.000000001\\
108999.999999996	108999.999999996\\
-17999.9999999962	-17999.9999999962\\
-55000.0000000033	-55000.0000000033\\
-55000.0000000006	-55000.0000000006\\
-54999.9999999979	-54999.9999999979\\
310999.999999998	310999.999999998\\
-163999.999999995	-163999.999999995\\
35999.9999999987	35999.9999999987\\
166000.000000001	166000.000000001\\
-368000.000000005	-368000.000000005\\
19000.0000000037	19000.0000000037\\
999.999999996781	999.999999996781\\
199999.999999999	199999.999999999\\
73999.9999999998	73999.9999999998\\
-166000.000000001	-166000.000000001\\
-52999.9999999982	-52999.9999999982\\
-166000.000000002	-166000.000000002\\
294000.000000002	294000.000000002\\
162999.999999996	162999.999999996\\
-235999.999999997	-235999.999999997\\
-276999.999999999	-276999.999999999\\
203999.999999998	203999.999999998\\
272999.999999999	272999.999999999\\
-348000	-348000\\
257000	257000\\
-218999.999999999	-218999.999999999\\
-56000.0000000018	-56000.0000000018\\
420999.999999999	420999.999999999\\
-329999.999999999	-329999.999999999\\
-89999.9999999972	-89999.9999999972\\
274000	274000\\
91000.0000000002	91000.0000000002\\
-90999.9999999966	-90999.9999999966\\
-91000.0000000037	-91000.0000000037\\
34999.9999999984	34999.9999999984\\
-345999.999999997	-345999.999999997\\
400999.999999997	400999.999999997\\
-347000.000000001	-347000.000000001\\
366000.000000001	366000.000000001\\
19000.0000000037	19000.0000000037\\
-312000.000000002	-312000.000000002\\
367000	367000\\
-386000.000000003	-386000.000000003\\
276000.000000003	276000.000000003\\
72999.9999999968	72999.9999999968\\
-347999.999999999	-347999.999999999\\
219000.000000004	219000.000000004\\
-91000.0000000011	-91000.0000000011\\
73999.9999999989	73999.9999999989\\
72000.0000000036	72000.0000000036\\
-218999.999999999	-218999.999999999\\
17999.9999999998	17999.9999999998\\
182999.999999997	182999.999999997\\
-36000.0000000005	-36000.0000000005\\
-220000.000000001	-220000.000000001\\
165000	165000\\
236999.999999997	236999.999999997\\
-199999.999999998	-199999.999999998\\
-258000.000000001	-258000.000000001\\
258000.000000002	258000.000000002\\
200999.999999999	200999.999999999\\
-239000.000000001	-239000.000000001\\
294000	294000\\
-275000.000000001	-275000.000000001\\
-439000.000000002	-439000.000000002\\
328000.000000002	328000.000000002\\
276999.999999999	276999.999999999\\
-258000.000000001	-258000.000000001\\
219999.999999999	219999.999999999\\
72999.9999999995	72999.9999999995\\
-163999.999999999	-163999.999999999\\
-165000	-165000\\
37000.0000000008	37000.0000000008\\
199999.999999997	199999.999999997\\
18999.9999999983	18999.9999999983\\
-291999.999999998	-291999.999999998\\
164000	164000\\
72000.0000000018	72000.0000000018\\
-52000.0000000014	-52000.0000000014\\
-258999.999999999	-258999.999999999\\
147000.000000001	147000.000000001\\
220999.999999997	220999.999999997\\
17000.0000000003	17000.0000000003\\
-365000.000000001	-365000.000000001\\
89999.999999999	89999.999999999\\
532000.000000003	532000.000000003\\
-604000.000000001	-604000.000000001\\
182999.999999998	182999.999999998\\
199999.999999999	199999.999999999\\
-35000.000000001	-35000.000000001\\
-366999.999999999	-366999.999999999\\
111000.000000002	111000.000000002\\
382999.999999996	382999.999999996\\
-439000	-439000\\
202000	202000\\
110000.000000001	110000.000000001\\
-220999.999999997	-220999.999999997\\
-181999.999999999	-181999.999999999\\
255000	255000\\
111999.999999997	111999.999999997\\
-148000.000000002	-148000.000000002\\
1000.00000000211	1000.00000000211\\
199999.999999999	199999.999999999\\
1000.00000000033	1000.00000000033\\
-274999.999999999	-274999.999999999\\
-35999.9999999996	-35999.9999999996\\
109000.000000001	109000.000000001\\
19000.0000000001	19000.0000000001\\
-256999.999999999	-256999.999999999\\
457999.999999996	457999.999999996\\
55000.0000000006	55000.0000000006\\
-311000.000000002	-311000.000000002\\
-17999.9999999998	-17999.9999999998\\
-20000.0000000005	-20000.0000000005\\
203000.000000002	203000.000000002\\
-36999.9999999972	-36999.9999999972\\
109999.999999999	109999.999999999\\
-256000.000000003	-256000.000000003\\
53000.0000000017	53000.0000000017\\
56999.9999999986	56999.9999999986\\
-91999.9999999996	-91999.9999999996\\
0	0\\
181999.999999999	181999.999999999\\
-16999.9999999995	-16999.9999999995\\
-55999.9999999983	-55999.9999999983\\
-71999.9999999992	-71999.9999999992\\
0	0\\
-20000.0000000022	-20000.0000000022\\
240000.000000001	240000.000000001\\
-277000.000000002	-277000.000000002\\
-51999.999999996	-51999.999999996\\
143999.999999997	143999.999999997\\
148000.000000004	148000.000000004\\
-220000.000000003	-220000.000000003\\
146000.000000002	146000.000000002\\
-55000.0000000033	-55000.0000000033\\
92000.0000000023	92000.0000000023\\
-366000	-366000\\
108999.999999998	108999.999999998\\
367000.000000001	367000.000000001\\
17999.9999999998	17999.9999999998\\
-439000	-439000\\
291999.999999998	291999.999999998\\
-54000.0000000003	-54000.0000000003\\
-257000.000000001	-257000.000000001\\
384999.999999999	384999.999999999\\
-183999.999999999	-183999.999999999\\
-107999.999999999	-107999.999999999\\
273000.000000001	273000.000000001\\
-329999.999999999	-329999.999999999\\
257999.999999996	257999.999999996\\
91000.0000000002	91000.0000000002\\
-421000	-421000\\
292000.000000002	292000.000000002\\
-91000.0000000011	-91000.0000000011\\
146999.999999999	146999.999999999\\
-55999.9999999983	-55999.9999999983\\
-182000.000000002	-182000.000000002\\
236999.999999999	236999.999999999\\
-71999.9999999992	-71999.9999999992\\
-313000.000000001	-313000.000000001\\
625000.000000003	625000.000000003\\
-240000.000000002	-240000.000000002\\
-238000.000000002	-238000.000000002\\
294000.000000001	294000.000000001\\
-37000.0000000008	-37000.0000000008\\
-384999.999999997	-384999.999999997\\
310999.999999998	310999.999999998\\
185000	185000\\
-532999.999999998	-532999.999999998\\
366999.999999998	366999.999999998\\
-73000.0000000013	-73000.0000000013\\
219000.000000001	219000.000000001\\
-128000.000000002	-128000.000000002\\
-126999.999999997	-126999.999999997\\
-74999.9999999993	-74999.9999999993\\
-72000.0000000001	-72000.0000000001\\
310999.999999999	310999.999999999\\
-73999.9999999981	-73999.9999999981\\
-164000.000000003	-164000.000000003\\
74000.0000000007	74000.0000000007\\
327999.999999999	327999.999999999\\
-438999.999999999	-438999.999999999\\
127999.999999998	127999.999999998\\
221000.000000002	221000.000000002\\
-203000.000000001	-203000.000000001\\
-199999.999999999	-199999.999999999\\
419999.999999997	419999.999999997\\
-238000	-238000\\
74000.0000000007	74000.0000000007\\
-55000.0000000006	-55000.0000000006\\
-74000.0000000016	-74000.0000000016\\
422000.000000001	422000.000000001\\
-294000.000000001	-294000.000000001\\
-127000.000000001	-127000.000000001\\
-166000.000000001	-166000.000000001\\
203000.000000002	203000.000000002\\
180999.999999999	180999.999999999\\
21000.0000000026	21000.0000000026\\
-21000.0000000035	-21000.0000000035\\
-345999.999999999	-345999.999999999\\
-92999.9999999982	-92999.9999999982\\
330999.999999998	330999.999999998\\
200000	200000\\
1000.00000000033	1000.00000000033\\
-256000.000000003	-256000.000000003\\
-184999.999999999	-184999.999999999\\
75999.9999999996	75999.9999999996\\
235000.000000001	235000.000000001\\
3000.00000000011	3000.00000000011\\
33999.999999998	33999.999999998\\
-272999.999999995	-272999.999999995\\
72999.9999999995	72999.9999999995\\
201000.000000001	201000.000000001\\
-274000	-274000\\
255000	255000\\
20000.0000000005	20000.0000000005\\
-257999.999999999	-257999.999999999\\
184000	184000\\
257000.000000001	257000.000000001\\
-496000.000000001	-496000.000000001\\
-15999.9999999982	-15999.9999999982\\
143999.999999995	143999.999999995\\
129000.000000002	129000.000000002\\
55999.9999999991	55999.9999999991\\
-239999.999999999	-239999.999999999\\
129999.999999999	129999.999999999\\
109000.000000001	109000.000000001\\
-309999.999999997	-309999.999999997\\
216999.999999995	216999.999999995\\
-125999.999999997	-125999.999999997\\
128000.000000002	128000.000000002\\
145999.999999997	145999.999999997\\
-164999.999999998	-164999.999999998\\
109999.999999999	109999.999999999\\
-438000.000000001	-438000.000000001\\
417999.999999999	417999.999999999\\
-87999.9999999992	-87999.9999999992\\
-21000.0000000008	-21000.0000000008\\
-219000.000000001	-219000.000000001\\
239000.000000003	239000.000000003\\
17999.9999999989	17999.9999999989\\
16999.9999999995	16999.9999999995\\
166000	166000\\
-238000	-238000\\
-19000.0000000019	-19000.0000000019\\
-54000.0000000003	-54000.0000000003\\
-111000.000000002	-111000.000000002\\
312000	312000\\
-55000.0000000015	-55000.0000000015\\
-274999.999999999	-274999.999999999\\
257000.000000001	257000.000000001\\
54999.9999999988	54999.9999999988\\
-38000.0000000002	-38000.0000000002\\
-108000.000000001	-108000.000000001\\
309000	309000\\
-271999.999999998	-271999.999999998\\
-20000.0000000022	-20000.0000000022\\
-18000.0000000016	-18000.0000000016\\
-110000	-110000\\
293000.000000001	293000.000000001\\
18999.9999999983	18999.9999999983\\
-221000.000000001	-221000.000000001\\
-217999.999999997	-217999.999999997\\
346999.999999998	346999.999999998\\
126999.999999999	126999.999999999\\
-418999.999999998	-418999.999999998\\
290999.999999999	290999.999999999\\
203000.000000003	203000.000000003\\
-165000.000000002	-165000.000000002\\
-111000.000000002	-111000.000000002\\
-73000.0000000004	-73000.0000000004\\
-182000	-182000\\
456999.999999999	456999.999999999\\
-183000.000000001	-183000.000000001\\
-73000.0000000013	-73000.0000000013\\
109000.000000001	109000.000000001\\
-17000.0000000003	-17000.0000000003\\
-56000.0000000018	-56000.0000000018\\
165000.000000003	165000.000000003\\
-164000.000000002	-164000.000000002\\
89999.9999999998	89999.9999999998\\
-35000.0000000001	-35000.0000000001\\
-274999.999999999	-274999.999999999\\
218999.999999998	218999.999999998\\
128000	128000\\
220999.999999999	220999.999999999\\
-752000	-752000\\
513999.999999999	513999.999999999\\
272999.999999999	272999.999999999\\
-272999.999999998	-272999.999999998\\
-404000.000000004	-404000.000000004\\
366000.000000003	366000.000000003\\
38000.000000002	38000.000000002\\
-202000.000000003	-202000.000000003\\
329000.000000001	329000.000000001\\
-365000	-365000\\
-93000.0000000035	-93000.0000000035\\
403000.000000002	403000.000000002\\
166000.000000002	166000.000000002\\
-477000.000000001	-477000.000000001\\
128000.000000001	128000.000000001\\
56000.0000000009	56000.0000000009\\
90999.9999999975	90999.9999999975\\
-147000.000000001	-147000.000000001\\
128000	128000\\
-292000.000000002	-292000.000000002\\
183000.000000002	183000.000000002\\
-147000.000000003	-147000.000000003\\
275000.000000002	275000.000000002\\
-184000	-184000\\
331000.000000003	331000.000000003\\
-165000.000000003	-165000.000000003\\
-110999.999999999	-110999.999999999\\
-72999.9999999995	-72999.9999999995\\
129999.999999999	129999.999999999\\
88999.9999999986	88999.9999999986\\
-217999.999999999	-217999.999999999\\
-19000.0000000028	-19000.0000000028\\
220000.000000001	220000.000000001\\
-55000.0000000024	-55000.0000000024\\
-274999.999999998	-274999.999999998\\
146999.999999998	146999.999999998\\
292000	292000\\
-162999.999999998	-162999.999999998\\
-129999.999999999	-129999.999999999\\
-274000.000000002	-274000.000000002\\
366999.999999998	366999.999999998\\
457000.000000001	457000.000000001\\
-660000.000000001	-660000.000000001\\
74999.9999999984	74999.9999999984\\
219000.000000001	219000.000000001\\
-202000.000000001	-202000.000000001\\
164999.999999998	164999.999999998\\
37000.0000000008	37000.0000000008\\
-90999.9999999993	-90999.9999999993\\
-56999.9999999995	-56999.9999999995\\
-89000.0000000013	-89000.0000000013\\
-93000.0000000026	-93000.0000000026\\
312000.000000002	312000.000000002\\
108999.999999998	108999.999999998\\
-91999.9999999996	-91999.9999999996\\
-70999.9999999997	-70999.9999999997\\
-239999.999999997	-239999.999999997\\
-36000.0000000005	-36000.0000000005\\
328999.999999998	328999.999999998\\
1000.00000000211	1000.00000000211\\
54999.9999999997	54999.9999999997\\
-184000.000000003	-184000.000000003\\
-36999.999999999	-36999.999999999\\
-34999.9999999984	-34999.9999999984\\
35999.9999999996	35999.9999999996\\
90999.9999999975	90999.9999999975\\
219999.999999999	219999.999999999\\
-73999.9999999989	-73999.9999999989\\
-308999.999999997	-308999.999999997\\
88999.9999999942	88999.9999999942\\
20000.0000000022	20000.0000000022\\
199999.999999997	199999.999999997\\
-310000	-310000\\
128000	128000\\
17999.999999998	17999.999999998\\
164000.000000001	164000.000000001\\
-89999.9999999981	-89999.9999999981\\
-313000.000000001	-313000.000000001\\
93999.9999999985	93999.9999999985\\
88999.9999999995	88999.9999999995\\
147999.999999999	147999.999999999\\
-73999.9999999998	-73999.9999999998\\
54999.9999999988	54999.9999999988\\
-72000.0000000001	-72000.0000000001\\
-148000.000000001	-148000.000000001\\
73000.0000000004	73000.0000000004\\
331000.000000002	331000.000000002\\
-73999.9999999989	-73999.9999999989\\
-347000	-347000\\
89999.9999999972	89999.9999999972\\
-17999.9999999962	-17999.9999999962\\
366999.999999998	366999.999999998\\
3.5527136788005e-09	3.5527136788005e-09\\
-403000.000000003	-403000.000000003\\
-37000.0000000026	-37000.0000000026\\
183000.000000002	183000.000000002\\
73999.9999999998	73999.9999999998\\
-256999.999999996	-256999.999999996\\
-19000.0000000037	-19000.0000000037\\
167000.000000001	167000.000000001\\
107999.999999999	107999.999999999\\
18999.9999999992	18999.9999999992\\
-109999.999999999	-109999.999999999\\
54999.9999999979	54999.9999999979\\
110000.000000001	110000.000000001\\
-1.77635683940025e-09	-1.77635683940025e-09\\
55000.0000000006	55000.0000000006\\
-74000.0000000025	-74000.0000000025\\
-107999.999999998	-107999.999999998\\
-111999.999999998	-111999.999999998\\
183999.999999998	183999.999999998\\
164999.999999998	164999.999999998\\
-54999.9999999979	-54999.9999999979\\
-402999.999999997	-402999.999999997\\
494000.000000001	494000.000000001\\
-293000.000000002	-293000.000000002\\
37999.9999999976	37999.9999999976\\
-56000	-56000\\
182999.999999999	182999.999999999\\
165000.000000002	165000.000000002\\
-457000	-457000\\
236999.999999997	236999.999999997\\
110000.000000001	110000.000000001\\
-92000.0000000005	-92000.0000000005\\
-292000	-292000\\
238000.000000001	238000.000000001\\
164999.999999998	164999.999999998\\
-184999.999999999	-184999.999999999\\
167000.000000002	167000.000000002\\
-129000.000000002	-129000.000000002\\
-17999.9999999998	-17999.9999999998\\
-111000.000000001	-111000.000000001\\
276000	276000\\
-293000	-293000\\
199999.999999999	199999.999999999\\
-16999.9999999995	-16999.9999999995\\
-367000	-367000\\
624000.000000002	624000.000000002\\
89999.9999999963	89999.9999999963\\
-476000.000000001	-476000.000000001\\
-91000.0000000011	-91000.0000000011\\
37000.0000000017	37000.0000000017\\
108999.999999997	108999.999999997\\
37000.0000000026	37000.0000000026\\
37000.0000000017	37000.0000000017\\
-1000.00000000122	-1000.00000000122\\
148000.000000001	148000.000000001\\
-238999.999999999	-238999.999999999\\
-74000.0000000043	-74000.0000000043\\
184000.000000001	184000.000000001\\
92000.0000000005	92000.0000000005\\
-111000.000000002	-111000.000000002\\
-364999.999999998	-364999.999999998\\
402000	402000\\
-146000.000000001	-146000.000000001\\
54999.9999999997	54999.9999999997\\
181999.999999999	181999.999999999\\
-35999.9999999978	-35999.9999999978\\
-163999.999999999	-163999.999999999\\
-184000.000000001	-184000.000000001\\
184000.000000002	184000.000000002\\
-1000.00000000122	-1000.00000000122\\
91999.9999999996	91999.9999999996\\
-36999.999999999	-36999.999999999\\
127999.999999998	127999.999999998\\
-126999.999999997	-126999.999999997\\
183000	183000\\
-221000.000000005	-221000.000000005\\
-383999.999999998	-383999.999999998\\
385000.000000001	385000.000000001\\
383999.999999999	383999.999999999\\
-658999.999999999	-658999.999999999\\
201999.999999998	201999.999999998\\
109000.000000002	109000.000000002\\
311000	311000\\
-181999.999999999	-181999.999999999\\
-110000.000000001	-110000.000000001\\
-386000.000000002	-386000.000000002\\
368000.000000002	368000.000000002\\
-19000.0000000001	-19000.0000000001\\
183000	183000\\
-366000	-366000\\
257000	257000\\
-167000.000000002	-167000.000000002\\
130999.999999998	130999.999999998\\
-111999.999999998	-111999.999999998\\
-73000.0000000013	-73000.0000000013\\
313000.000000002	313000.000000002\\
-240000.000000004	-240000.000000004\\
19000.0000000019	19000.0000000019\\
183000.000000002	183000.000000002\\
129000	129000\\
-404000	-404000\\
-17000.0000000021	-17000.0000000021\\
401000	401000\\
-308999.999999999	-308999.999999999\\
161999.999999998	161999.999999998\\
-345000.000000001	-345000.000000001\\
327000	327000\\
112000.000000001	112000.000000001\\
-111000.000000002	-111000.000000002\\
-183000	-183000\\
-164000.000000001	-164000.000000001\\
219000	219000\\
182000.000000001	182000.000000001\\
75999.9999999987	75999.9999999987\\
-331999.999999996	-331999.999999996\\
-18000.0000000016	-18000.0000000016\\
110999.999999998	110999.999999998\\
291999.999999999	291999.999999999\\
-256000.000000001	-256000.000000001\\
-127999.999999997	-127999.999999997\\
91999.9999999978	91999.9999999978\\
109000	109000\\
-92000.0000000023	-92000.0000000023\\
38000.0000000011	38000.0000000011\\
126999.999999999	126999.999999999\\
-164000.000000001	-164000.000000001\\
-146999.999999998	-146999.999999998\\
255999.999999998	255999.999999998\\
-162999.999999997	-162999.999999997\\
-39000.0000000032	-39000.0000000032\\
20000.0000000031	20000.0000000031\\
347999.999999997	347999.999999997\\
-202999.999999999	-202999.999999999\\
-310000.000000002	-310000.000000002\\
475000	475000\\
-34999.9999999975	-34999.9999999975\\
-239000.000000003	-239000.000000003\\
54999.9999999988	54999.9999999988\\
-128999.999999999	-128999.999999999\\
147999.999999998	147999.999999998\\
-18999.9999999983	-18999.9999999983\\
-293000.000000004	-293000.000000004\\
751000.000000002	751000.000000002\\
-532000.000000002	-532000.000000002\\
-199999.999999998	-199999.999999998\\
255999.999999997	255999.999999997\\
182000.000000003	182000.000000003\\
-36000.0000000031	-36000.0000000031\\
-89999.9999999981	-89999.9999999981\\
-276999.999999999	-276999.999999999\\
19000.0000000028	19000.0000000028\\
256999.999999994	256999.999999994\\
238000.000000002	238000.000000002\\
-221000.000000002	-221000.000000002\\
-181999.999999999	-181999.999999999\\
147000	147000\\
-56000.0000000018	-56000.0000000018\\
1000.000000003	1000.000000003\\
34999.9999999957	34999.9999999957\\
38000.0000000038	38000.0000000038\\
54999.9999999979	54999.9999999979\\
35999.9999999996	35999.9999999996\\
-165000.000000001	-165000.000000001\\
-55000.0000000006	-55000.0000000006\\
-52999.9999999973	-52999.9999999973\\
381999.999999998	381999.999999998\\
-254999.999999999	-254999.999999999\\
-293999.999999999	-293999.999999999\\
367999.999999998	367999.999999998\\
52999.9999999982	52999.9999999982\\
-108999.999999999	-108999.999999999\\
-145999.999999995	-145999.999999995\\
548999.999999998	548999.999999998\\
-788000.000000003	-788000.000000003\\
422000.000000001	422000.000000001\\
146000	146000\\
-256000	-256000\\
71999.9999999992	71999.9999999992\\
-290999.999999997	-290999.999999997\\
475000	475000\\
128000.000000001	128000.000000001\\
-511999.999999999	-511999.999999999\\
108999.999999999	108999.999999999\\
184000.000000002	184000.000000002\\
-148000.000000001	-148000.000000001\\
130000.000000002	130000.000000002\\
-184000.000000001	-184000.000000001\\
238999.999999999	238999.999999999\\
53000.0000000026	53000.0000000026\\
-108000.000000001	-108000.000000001\\
-239000	-239000\\
164999.999999996	164999.999999996\\
36000.0000000014	36000.0000000014\\
-292000.000000002	-292000.000000002\\
183000.000000002	183000.000000002\\
402999.999999998	402999.999999998\\
-349000	-349000\\
-109000.000000002	-109000.000000002\\
293000	293000\\
-312000	-312000\\
366999.999999998	366999.999999998\\
-366999.999999999	-366999.999999999\\
93000	93000\\
-56000.0000000009	-56000.0000000009\\
238000.000000002	238000.000000002\\
-72999.9999999986	-72999.9999999986\\
-999.999999997669	-999.999999997669\\
-108000.000000003	-108000.000000003\\
-183999.999999998	-183999.999999998\\
420999.999999999	420999.999999999\\
-220000.000000001	-220000.000000001\\
-255000.000000001	-255000.000000001\\
309000	309000\\
148999.999999997	148999.999999997\\
15999.9999999991	15999.9999999991\\
-328000	-328000\\
-72999.9999999986	-72999.9999999986\\
711999.999999999	711999.999999999\\
-436999.999999999	-436999.999999999\\
-184000.000000002	-184000.000000002\\
164000.000000005	164000.000000005\\
-311000.000000003	-311000.000000003\\
147000.000000003	147000.000000003\\
91999.999999997	91999.999999997\\
402000.000000001	402000.000000001\\
-641000.000000003	-641000.000000003\\
184000	184000\\
273999.999999999	273999.999999999\\
-18999.9999999992	-18999.9999999992\\
1000.00000000033	1000.00000000033\\
-421000	-421000\\
109999.999999999	109999.999999999\\
90999.9999999975	90999.9999999975\\
457000.000000001	457000.000000001\\
-292000.000000002	-292000.000000002\\
-476000	-476000\\
457999.999999999	457999.999999999\\
146000	146000\\
-238999.999999996	-238999.999999996\\
-438000.000000001	-438000.000000001\\
146999.999999999	146999.999999999\\
583999.999999998	583999.999999998\\
-16999.9999999986	-16999.9999999986\\
-184000.000000003	-184000.000000003\\
-236999.999999997	-236999.999999997\\
493999.999999998	493999.999999998\\
-695999.999999995	-695999.999999995\\
256000	256000\\
999.999999998557	999.999999998557\\
-999.999999998557	-999.999999998557\\
18999.9999999983	18999.9999999983\\
238000.000000001	238000.000000001\\
-91999.9999999996	-91999.9999999996\\
-91000.0000000011	-91000.0000000011\\
72999.9999999968	72999.9999999968\\
-18999.9999999983	-18999.9999999983\\
-17000.0000000021	-17000.0000000021\\
-130000	-130000\\
-88999.999999996	-88999.999999996\\
400999.999999994	400999.999999994\\
165000.000000002	165000.000000002\\
-293000.000000002	-293000.000000002\\
-676999.999999999	-676999.999999999\\
639999.999999999	639999.999999999\\
221000.000000002	221000.000000002\\
-73999.9999999998	-73999.9999999998\\
-311000	-311000\\
-238000.000000003	-238000.000000003\\
347000.000000001	347000.000000001\\
202999.999999998	202999.999999998\\
-185000	-185000\\
55999.9999999991	55999.9999999991\\
111000.000000003	111000.000000003\\
-277000.000000005	-277000.000000005\\
422000.000000002	422000.000000002\\
-437999.999999998	-437999.999999998\\
-148000.000000005	-148000.000000005\\
256000.000000001	256000.000000001\\
55999.9999999991	55999.9999999991\\
-17999.9999999971	-17999.9999999971\\
162999.999999996	162999.999999996\\
-382999.999999999	-382999.999999999\\
273999.999999999	273999.999999999\\
129000	129000\\
-166000.000000001	-166000.000000001\\
-127000.000000001	-127000.000000001\\
-165000	-165000\\
420000.000000003	420000.000000003\\
-54000.0000000003	-54000.0000000003\\
-385000.000000001	-385000.000000001\\
20000.0000000005	20000.0000000005\\
657000.000000001	657000.000000001\\
-365999.999999999	-365999.999999999\\
-237000.000000005	-237000.000000005\\
18000.0000000016	18000.0000000016\\
513000.000000002	513000.000000002\\
-92000.0000000032	-92000.0000000032\\
-238999.999999995	-238999.999999995\\
-493000	-493000\\
274999.999999998	274999.999999998\\
401999.999999999	401999.999999999\\
-220000	-220000\\
92000.0000000014	92000.0000000014\\
127999.999999999	127999.999999999\\
37000.0000000026	37000.0000000026\\
-312000.000000001	-312000.000000001\\
-109000.000000001	-109000.000000001\\
347000.000000001	347000.000000001\\
-90999.9999999975	-90999.9999999975\\
-183000	-183000\\
127999.999999999	127999.999999999\\
37000.0000000035	37000.0000000035\\
-166000.000000003	-166000.000000003\\
368000	368000\\
-1000.00000000033	-1000.00000000033\\
-184000.000000002	-184000.000000002\\
-181999.999999998	-181999.999999998\\
-109000.000000001	-109000.000000001\\
235999.999999997	235999.999999997\\
-16999.9999999995	-16999.9999999995\\
-202000	-202000\\
459000	459000\\
-240000.000000003	-240000.000000003\\
-126999.999999999	-126999.999999999\\
440000.000000001	440000.000000001\\
-496000.000000001	-496000.000000001\\
37999.9999999985	37999.9999999985\\
182000	182000\\
221000.000000002	221000.000000002\\
-202000.000000002	-202000.000000002\\
-294000	-294000\\
257999.999999998	257999.999999998\\
-420999.999999999	-420999.999999999\\
492000	492000\\
186000.000000003	186000.000000003\\
-314000.000000004	-314000.000000004\\
-15999.9999999938	-15999.9999999938\\
-19000.0000000028	-19000.0000000028\\
255000.000000001	255000.000000001\\
75999.9999999978	75999.9999999978\\
-240999.999999999	-240999.999999999\\
92999.9999999991	92999.9999999991\\
-348000.000000002	-348000.000000002\\
74000.0000000034	74000.0000000034\\
125999.999999997	125999.999999997\\
204000.000000001	204000.000000001\\
-148000	-148000\\
-165000.000000003	-165000.000000003\\
129000	129000\\
1.77635683940025e-09	1.77635683940025e-09\\
-37000.0000000035	-37000.0000000035\\
274000.000000001	274000.000000001\\
-17000.0000000003	-17000.0000000003\\
-257000.000000002	-257000.000000002\\
-17999.9999999989	-17999.9999999989\\
-55000.0000000006	-55000.0000000006\\
72000.0000000001	72000.0000000001\\
-126000.000000001	-126000.000000001\\
309999.999999998	309999.999999998\\
-182999.999999999	-182999.999999999\\
-37000.0000000008	-37000.0000000008\\
330000	330000\\
-438999.999999998	-438999.999999998\\
17999.9999999998	17999.9999999998\\
567000	567000\\
-346999.999999998	-346999.999999998\\
-146999.999999998	-146999.999999998\\
73999.9999999989	73999.9999999989\\
291000.000000001	291000.000000001\\
-327000.000000001	-327000.000000001\\
199999.999999999	199999.999999999\\
-293000	-293000\\
-202000.000000002	-202000.000000002\\
277000.000000004	277000.000000004\\
254000	254000\\
109999.999999999	109999.999999999\\
-199999.999999996	-199999.999999996\\
-37000.0000000026	-37000.0000000026\\
-147999.999999998	-147999.999999998\\
-52999.9999999999	-52999.9999999999\\
493000	493000\\
-346999.999999997	-346999.999999997\\
-55000.0000000024	-55000.0000000024\\
-219999.999999997	-219999.999999997\\
403000	403000\\
237999.999999997	237999.999999997\\
-495000	-495000\\
146999.999999998	146999.999999998\\
-256000.000000001	-256000.000000001\\
127000.000000002	127000.000000002\\
332000	332000\\
-94999.9999999997	-94999.9999999997\\
21999.9999999993	21999.9999999993\\
-332999.999999999	-332999.999999999\\
313999.999999997	313999.999999997\\
-37999.9999999994	-37999.9999999994\\
-92000.0000000023	-92000.0000000023\\
-16999.9999999986	-16999.9999999986\\
71999.9999999974	71999.9999999974\\
-162999.999999998	-162999.999999998\\
-56999.9999999986	-56999.9999999986\\
110999.999999996	110999.999999996\\
256000.000000001	256000.000000001\\
36999.999999999	36999.999999999\\
-294000.000000001	-294000.000000001\\
37999.9999999994	37999.9999999994\\
-19000.0000000001	-19000.0000000001\\
-110000	-110000\\
294000.000000001	294000.000000001\\
-111000.000000003	-111000.000000003\\
201000.000000002	201000.000000002\\
-676000.000000002	-676000.000000002\\
438000.000000001	438000.000000001\\
386000.000000003	386000.000000003\\
-75000.0000000046	-75000.0000000046\\
-565999.999999997	-565999.999999997\\
-19000.0000000019	-19000.0000000019\\
438999.999999999	438999.999999999\\
-71999.9999999992	-71999.9999999992\\
54000.0000000003	54000.0000000003\\
-421999.999999998	-421999.999999998\\
148999.999999999	148999.999999999\\
546999.999999998	546999.999999998\\
-126999.999999999	-126999.999999999\\
-604999.999999999	-604999.999999999\\
-109000.000000001	-109000.000000001\\
602999.999999999	602999.999999999\\
203000.000000001	203000.000000001\\
-404000.000000002	-404000.000000002\\
-329000.000000001	-329000.000000001\\
348000.000000002	348000.000000002\\
274000	274000\\
-402999.999999999	-402999.999999999\\
55999.9999999991	55999.9999999991\\
476000.000000003	476000.000000003\\
-386000.000000002	-386000.000000002\\
-254000	-254000\\
472999.999999999	472999.999999999\\
-399999.999999998	-399999.999999998\\
109000.000000001	109000.000000001\\
310000	310000\\
-15999.9999999991	-15999.9999999991\\
-607000.000000003	-607000.000000003\\
149000.000000002	149000.000000002\\
383000.000000001	383000.000000001\\
92000.0000000005	92000.0000000005\\
-218999.999999998	-218999.999999998\\
-203000.000000004	-203000.000000004\\
368000.000000004	368000.000000004\\
-56000.0000000045	-56000.0000000045\\
-257000	-257000\\
-16999.9999999995	-16999.9999999995\\
183000.000000002	183000.000000002\\
-19000.0000000019	-19000.0000000019\\
-72999.9999999995	-72999.9999999995\\
238000	238000\\
-311000	-311000\\
18000.0000000007	18000.0000000007\\
385000.000000002	385000.000000002\\
-293000.000000001	-293000.000000001\\
-1000.00000000211	-1000.00000000211\\
18999.9999999983	18999.9999999983\\
-109999.999999998	-109999.999999998\\
-72000.0000000027	-72000.0000000027\\
272000.000000001	272000.000000001\\
-125999.999999999	-125999.999999999\\
-36999.9999999972	-36999.9999999972\\
128000	128000\\
110000	110000\\
-111000.000000003	-111000.000000003\\
185000.000000003	185000.000000003\\
-569000.000000003	-569000.000000003\\
36999.9999999999	36999.9999999999\\
678000.000000002	678000.000000002\\
-185000.000000003	-185000.000000003\\
-637999.999999998	-637999.999999998\\
327999.999999998	327999.999999998\\
274000.000000001	274000.000000001\\
-218000.000000001	-218000.000000001\\
-38000.0000000011	-38000.0000000011\\
403000.000000002	403000.000000002\\
-256000.000000001	-256000.000000001\\
-256000	-256000\\
420000.000000002	420000.000000002\\
-218000	-218000\\
-130000.000000003	-130000.000000003\\
1000.00000000211	1000.00000000211\\
202000.000000001	202000.000000001\\
-38000.0000000038	-38000.0000000038\\
165000	165000\\
-255999.999999998	-255999.999999998\\
19000.000000001	19000.000000001\\
-147000.000000002	-147000.000000002\\
237999.999999997	237999.999999997\\
54000.000000002	54000.000000002\\
-419000.000000001	-419000.000000001\\
400000.000000001	400000.000000001\\
149000.000000001	149000.000000001\\
-239000.000000002	-239000.000000002\\
-201000	-201000\\
475000	475000\\
-183000.000000003	-183000.000000003\\
-327999.999999998	-327999.999999998\\
89999.9999999981	89999.9999999981\\
275000.000000001	275000.000000001\\
-16999.9999999977	-16999.9999999977\\
-21000.0000000017	-21000.0000000017\\
-327000	-327000\\
53999.9999999994	53999.9999999994\\
127000.000000002	127000.000000002\\
148000.000000001	148000.000000001\\
-109999.999999999	-109999.999999999\\
54999.9999999997	54999.9999999997\\
346999.999999999	346999.999999999\\
-421000.000000002	-421000.000000002\\
-474999.999999998	-474999.999999998\\
530000	530000\\
312000.000000001	312000.000000001\\
-312000.000000001	-312000.000000001\\
-275000.000000001	-275000.000000001\\
478000.000000001	478000.000000001\\
-111999.999999999	-111999.999999999\\
-91000.0000000002	-91000.0000000002\\
-126999.999999999	-126999.999999999\\
-38000.0000000038	-38000.0000000038\\
367000.000000002	367000.000000002\\
-202000.000000003	-202000.000000003\\
-365999.999999996	-365999.999999996\\
422000	422000\\
-1000.00000000122	-1000.00000000122\\
347000.000000002	347000.000000002\\
-382000.000000001	-382000.000000001\\
-679000.000000002	-679000.000000002\\
896999.999999999	896999.999999999\\
72999.9999999995	72999.9999999995\\
-420000.000000003	-420000.000000003\\
-237999.999999999	-237999.999999999\\
217999.999999999	217999.999999999\\
39000.0000000006	39000.0000000006\\
382000	382000\\
-181000.000000001	-181000.000000001\\
-74000.0000000007	-74000.0000000007\\
-36999.999999999	-36999.999999999\\
-255999.999999998	-255999.999999998\\
256000	256000\\
-54000.0000000011	-54000.0000000011\\
-183999.999999998	-183999.999999998\\
402999.999999999	402999.999999999\\
127999.999999999	127999.999999999\\
-457000.000000001	-457000.000000001\\
-18999.9999999983	-18999.9999999983\\
146999.999999998	146999.999999998\\
201000.000000002	201000.000000002\\
-347999.999999997	-347999.999999997\\
220000.000000001	220000.000000001\\
91000.0000000002	91000.0000000002\\
-511000.000000002	-511000.000000002\\
327000.000000001	327000.000000001\\
1999.99999999711	1999.99999999711\\
18000.0000000016	18000.0000000016\\
292000.000000001	292000.000000001\\
111000.000000001	111000.000000001\\
-495000.000000001	-495000.000000001\\
-72999.9999999995	-72999.9999999995\\
365999.999999996	365999.999999996\\
-255999.999999998	-255999.999999998\\
-348000.000000002	-348000.000000002\\
585000	585000\\
-72000.0000000009	-72000.0000000009\\
-129000	-129000\\
-17000.0000000003	-17000.0000000003\\
52999.9999999999	52999.9999999999\\
56999.9999999986	56999.9999999986\\
-130000	-130000\\
146999.999999998	146999.999999998\\
-236999.999999999	-236999.999999999\\
585000	585000\\
-220000.000000001	-220000.000000001\\
-494000.000000001	-494000.000000001\\
-17999.9999999998	-17999.9999999998\\
511999.999999999	511999.999999999\\
-145999.999999999	-145999.999999999\\
146999.999999998	146999.999999998\\
-476999.999999999	-476999.999999999\\
367000	367000\\
35000.0000000001	35000.0000000001\\
-363999.999999999	-363999.999999999\\
-38000.0000000002	-38000.0000000002\\
402999.999999997	402999.999999997\\
-218999.999999998	-218999.999999998\\
71999.9999999974	71999.9999999974\\
331000.000000002	331000.000000002\\
89999.9999999981	89999.9999999981\\
-932000	-932000\\
492999.999999999	492999.999999999\\
421000	421000\\
-622000.000000002	-622000.000000002\\
-108999.999999998	-108999.999999998\\
620999.999999999	620999.999999999\\
-476000	-476000\\
221000.000000005	221000.000000005\\
383999.999999998	383999.999999998\\
-385000.000000001	-385000.000000001\\
-109000.000000001	-109000.000000001\\
-93000	-93000\\
93000	93000\\
383999.999999999	383999.999999999\\
-201999.999999998	-201999.999999998\\
};
\end{axis}

\begin{axis}[%
width=4.927cm,
height=3.484cm,
at={(6.484cm,4.839cm)},
scale only axis,
xmin=-1000000,
xmax=1000000,
xlabel style={font=\color{white!15!black}},
xlabel={$\delta^3 u(t)$},
ymin=-46264700000,
ymax=54321500000,
ylabel style={font=\color{white!15!black}},
ylabel={y(t)},
axis background/.style={fill=white},
title={C7, R = 0.5231},
axis x line*=bottom,
axis y line*=left
]
\addplot[only marks, mark=*, mark options={}, mark size=1.5000pt, color=mycolor1, fill=mycolor1] table[row sep=crcr]{%
x	y\\
-75000.0000000019	4882500000\\
166000.000000001	13550000000\\
-72000.0000000001	-19531300000\\
143999.999999998	19653300000\\
92999.9999999973	-15136700000\\
-457999.999999997	-5737299999.99999\\
163999.999999997	19287100000\\
38000.0000000029	-12207100000\\
128000.000000002	8911400000\\
218999.999999996	-6592199999.99999\\
-329000	-5859000000.00001\\
-165999.999999998	9643300000.00001\\
-345999.999999998	-7202100000\\
511000.000000001	8911200000\\
329999.999999996	854500000.000002\\
-291999.999999997	-10498000000\\
199999.999999997	10009500000\\
-145000	-10864000000\\
-550000.000000002	-3906300000\\
787000.000000001	31249900000\\
-202000.000000003	-34423600000\\
-436999.999999997	8544600000\\
472999.999999997	16968000000\\
-70999.9999999979	-18676700000\\
-292999.999999999	-1098900000\\
308999.999999997	21118300000\\
112000.000000002	-18554500000\\
-457999.999999997	-5249299999.99999\\
328999.999999997	22949200000\\
74000.0000000025	-15624600000\\
-148000.000000003	-1221200000.00001\\
39000.0000000041	4883099999.99999\\
-186000.000000004	-4638799999.99999\\
76000.0000000014	5981600000\\
180999.999999999	2075100000.00001\\
-145000.000000003	-9399400000\\
35000.0000000028	6713700000\\
-15999.9999999991	-2929299999.99999\\
-130000.000000003	-2563900000.00001\\
311000.000000001	15503200000\\
2000.00000000156	-16113400000\\
-222000.000000002	-3173699999.99999\\
203000.000000004	14648200000\\
-293000.000000004	-14526100000\\
-37999.9999999994	10131700000\\
312999.999999999	2807700000\\
-111000.000000002	-11841100000\\
-200999.999999999	5981900000\\
218999.999999997	4027900000\\
56000.0000000036	-3051500000\\
-293000.000000001	-9155300000\\
438999.999999998	24413900000\\
-37000.0000000008	-25024200000\\
-293000	5126700000\\
-90999.9999999993	4028700000\\
129000.000000001	3661700000\\
327999.999999997	610600000.000002\\
-346999.999999999	-13305800000\\
-146999.999999998	10742300000\\
366999.999999999	4028300000\\
-92999.9999999964	-10376000000\\
-200000.000000005	2563500000\\
128000.000000003	5493000000\\
-129000.000000003	-7323900000\\
312000.000000004	11108100000\\
-201000.000000002	-14648400000\\
108000.000000001	13550100000\\
-106999.999999998	-12085200000\\
124999.999999996	12939300000\\
-70999.9999999971	-14159900000\\
-56000.0000000027	10864300000\\
56000.0000000009	-6347999999.99999\\
-38000.0000000011	4150800000\\
-236999.999999998	-9155500000\\
201000	15136800000\\
238000.000000001	-1831299999.99999\\
-72999.9999999986	-10375500000\\
73999.9999999989	11230000000\\
-460000.000000003	-29540700000\\
185000	39672600000\\
219000.000000004	-14526000000\\
0	854000000.000016\\
-292000.000000001	-23193000000\\
35000.000000001	33935500000\\
439999.999999999	2075000000\\
-273000	-37963500000\\
51999.9999999987	29662600000\\
-106999.999999997	-12939000000\\
-57000.0000000057	7323899999.99999\\
-90999.9999999957	-4027999999.99999\\
312000	10741700000\\
-37000.0000000026	-14403800000\\
-202000.000000001	4028100000\\
-54000.0000000003	2441300000\\
16999.9999999986	-854200000.000003\\
240000.000000003	5859100000\\
-57000.0000000039	-10864100000\\
-199999.999999999	3417800000\\
237000	7202400000\\
-255000	-12207200000\\
400999.999999999	18432600000\\
-345999.999999996	-26245100000\\
255000.000000002	29663300000\\
-127000.000000002	-26855900000\\
-147999.999999999	12085500000\\
258000	8056100000\\
-130000.000000003	-16601200000\\
-125999.999999997	8422699999.99999\\
144999.999999997	3906300000\\
127999.999999999	-2563400000\\
-238000	-8301000000\\
148000.000000003	12573400000\\
-112000.000000004	-10620000000\\
2000.00000000244	7446000000\\
163999.999999998	-610199999.999996\\
-184000.000000001	-7690400000\\
-35999.9999999987	8911200000.00001\\
-17000.0000000012	-6470100000\\
273000.000000001	10376600000\\
-128000.000000004	-12207700000\\
220000.000000003	13428300000\\
-495000.000000003	-25146900000\\
75000.0000000055	22949400000\\
474999.999999996	7324300000\\
-385000.000000001	-30761900000\\
128000	26489400000\\
203000.000000002	-4638699999.99999\\
-257000.000000001	-16479600000\\
-203000.000000002	11352600000\\
149000.000000001	6836000000\\
-75000.000000001	-12451200000\\
183000	13061500000\\
130000.000000002	-4516600000\\
-20000.0000000005	-1953100000.00001\\
-220000.000000001	-11352500000\\
56000.0000000045	18676500000\\
36999.9999999981	-10863900000\\
-74000.0000000007	4272300000.00001\\
237999.999999997	2319299999.99999\\
-219999.999999999	-10498000000\\
-36000.0000000005	10131800000\\
90999.9999999966	-3051700000\\
-108999.999999997	-1709000000\\
-20000.0000000013	366200000.000001\\
-15999.9999999956	1953100000\\
419999.999999996	10986400000\\
-165000	-21362500000\\
-184000.000000001	7080399999.99999\\
2000.00000000156	4028000000.00001\\
182000	2197500000\\
-90999.9999999993	-7934600000\\
-220000.000000002	4150100000\\
-73999.9999999981	-609799999.999998\\
513999.999999999	8056000000\\
-276000.000000001	-13793400000\\
-54000.0000000029	7690000000\\
146000.000000003	366599999.999995\\
18999.9999999966	-1098799999.99999\\
-54999.999999997	-2563700000\\
-37999.9999999994	1465200000\\
-218000.000000001	-488400000.000001\\
-93000.0000000008	-122300000.000001\\
404000.000000002	5493600000\\
55000.0000000006	-2197600000\\
108999.999999998	1831100000\\
-165000.000000002	-12939400000\\
-126999.999999999	10009800000\\
91000.0000000002	-122100000.000003\\
-513000	-6958000000\\
603999.999999998	17578100000\\
-200000	-20385800000\\
34999.9999999984	14038200000\\
-145999.999999998	-12329100000\\
257000.000000001	18554600000\\
183000.000000002	-8422899999.99999\\
-257000.000000001	-15502700000\\
91999.999999997	22582800000\\
-257000.000000001	-23803600000\\
92000.0000000023	22460800000\\
56000	-9643400000\\
126000.000000001	6835800000\\
-144999.999999999	-14526100000\\
-18000.0000000007	10985900000\\
35999.9999999987	-487899999.999993\\
-56000	-5981600000\\
167000.000000002	12817300000\\
-331000	-24291800000\\
164999.999999996	28808500000\\
366000.000000003	-7080200000\\
-219000.000000002	-15380600000\\
-202999.999999999	3173600000.00001\\
-253999.999999998	6958100000\\
474999.999999997	7080000000.00001\\
-129000	-11718500000\\
111000.000000001	8544600000.00001\\
-55999.9999999983	-6957900000\\
93000	6714100000.00001\\
-129999.999999999	-14648800000\\
-162999.999999999	10254100000\\
181999.999999999	3051699999.99999\\
220000.000000003	9521600000.00002\\
-165000.000000003	-26977700000\\
-364999.999999997	7080200000\\
327999.999999996	22949100000\\
183000.000000001	-20019400000\\
-455999.999999997	-3296000000.00001\\
199999.999999998	14526500000\\
165000	-4760900000\\
-90999.9999999975	-5859399999.99999\\
-183000.000000002	300000.000000011\\
217999.999999997	11352100000\\
-180999.999999997	-15380500000\\
-1000.000000003	10131700000\\
256000.000000004	4394600000\\
74000.0000000007	-8911399999.99999\\
-384000	-5736900000\\
16999.9999999968	8422599999.99999\\
-72999.9999999995	-488299999.999998\\
604000.000000002	18066600000\\
-181999.999999999	-32715100000\\
-256999.999999998	11352700000\\
-109000.000000001	2929700000\\
162999.999999998	6103499999.99999\\
92999.9999999973	-7080100000\\
-146999.999999998	-2197399999.99999\\
-73000.0000000031	5249300000\\
91000.0000000002	-2197500000\\
110000.000000001	2441600000\\
73999.9999999998	1464700000\\
-440000.000000002	-16723700000\\
347999.999999999	24536300000\\
91000.0000000002	-11352500000\\
-90999.9999999984	-244400000.000012\\
-146000.000000002	-6713600000\\
-18999.9999999983	8544800000.00001\\
383999.999999997	10986300000\\
-16999.9999999995	-22094700000\\
-623000	1587000000\\
254999.999999999	16601500000\\
57000.0000000013	-11474700000\\
182000	8178899999.99999\\
-18000.0000000007	-7690599999.99999\\
-348000.000000002	-4760800000.00001\\
348000.000000001	17090200000\\
17999.999999998	-12085300000\\
-274999.999999999	-4272399999.99999\\
183999.999999999	12939600000\\
18000.0000000007	-8545200000\\
-147000.000000003	-610000000.000002\\
276000.000000004	9277100000.00001\\
-276000.000000003	-15258700000\\
92000.0000000023	14648400000\\
-19000.0000000001	-9887699999.99999\\
-255000.000000001	1586900000\\
401999.999999998	9033300000\\
-200999.999999998	-12207000000\\
54999.9999999988	7690100000\\
328999.999999997	7812900000\\
-127999.999999998	-21484500000\\
-312000.000000002	9887500000\\
20000.0000000022	2319600000\\
-56000.0000000027	-1342900000\\
404000.000000002	8056600000\\
-332000.000000002	-15991000000\\
-34000.0000000007	10497700000\\
-57000.0000000004	-3905900000\\
422000	10986100000\\
-328999.999999995	-20751800000\\
292999.999999997	26122800000\\
-167000.000000001	-27221400000\\
94000.0000000012	19653300000\\
-237999.999999998	-14648700000\\
143999.999999998	13672100000\\
-144000.000000002	-9399500000\\
-1000.00000000033	4760900000\\
-54000.0000000011	-3540300000\\
309000	7568600000\\
-161999.999999998	-8300900000\\
52999.9999999955	3662000000\\
-200999.999999997	-6225399999.99999\\
330000	18188500000\\
-8.88178419700125e-10	-19287300000\\
-310999.999999998	99999.999991951\\
144999.999999996	13305700000\\
-89999.9999999998	-10742100000\\
237000	14037700000\\
19000.000000001	-15746600000\\
-17999.9999999998	3417800000.00001\\
-239000.000000001	365999999.999997\\
-72999.9999999986	3296100000\\
73999.9999999998	-366099999.999999\\
90000.0000000007	-199999.999998823\\
38999.9999999997	-854499999.999999\\
-130999.999999999	-2685400000\\
-52000.0000000023	2929500000\\
326999.999999999	5249300000\\
-145000	-10742400000\\
-128000	5004800000\\
345999.999999999	6592100000\\
-126000.000000002	-12573500000\\
-549999.999999997	488399999.999999\\
292999.999999997	12329000000\\
329000.000000001	-3540000000\\
-219000.000000001	-8056400000\\
109000	9276900000\\
19000.0000000001	-6469400000\\
-329000.000000001	-4028400000\\
108000	10131800000\\
240000.000000001	1098600000\\
71999.9999999974	-4760600000\\
-512000	-15259000000\\
494000.000000002	35400700000\\
-128000.000000001	-32837200000\\
-165000.000000003	11840900000\\
165000.000000003	5371100000\\
17999.9999999998	-3418000000\\
-54000.000000002	-6957800000\\
-312999.999999999	-122400000.000007\\
258000.000000003	13550100000\\
310999.999999999	2929499999.99999\\
-56000	-19775300000\\
-365000.000000001	-488400000.000004\\
36000.0000000014	16723700000\\
36999.999999999	-9765399999.99999\\
53999.9999999985	4638200000\\
203000.000000003	3906699999.99999\\
-258000.000000004	-16235500000\\
-53999.9999999976	11230200000\\
183000.000000002	4150800000\\
128999.999999998	-1098800000\\
-167000.000000001	-13061600000\\
-254000.000000002	6469899999.99999\\
89000.0000000013	2929400000\\
333000.000000001	12695600000\\
-113000.000000001	-23681800000\\
111000.000000002	19287300000\\
-237000.000000003	-21362600000\\
-111999.999999998	17212200000\\
405000.000000001	-732600000.000005\\
-221000.000000004	-8056700000\\
-53999.9999999976	4150700000\\
-37999.9999999976	1098300000\\
-291999.999999999	-5126800000\\
328999.999999998	9155200000\\
201999.999999999	-5248800000\\
18000.0000000016	3905900000\\
-274000.000000002	-12328900000\\
35999.9999999978	12573400000\\
274000.000000001	609999999.999998\\
-290999.999999999	-13549700000\\
15999.9999999991	13672200000\\
129999.999999999	-4028800000\\
-202999.999999996	-4882500000\\
-36000.0000000031	4150300000\\
184000.000000002	4150300000\\
273999.999999999	1831300000\\
-219999.999999999	-15747300000\\
-201000.000000004	10986300000\\
73000.0000000004	-2319000000\\
165000	6591300000\\
-458000.000000003	-15502500000\\
202000.000000001	15136600000\\
548000	6957800000\\
-530000.000000002	-30639400000\\
92000.0000000014	28564300000\\
108000	-9521299999.99999\\
130000.000000003	3295699999.99998\\
-183000.000000001	-12206900000\\
-129000	7324099999.99998\\
72999.9999999977	4638800000.00001\\
0	-5371299999.99999\\
1000.00000000033	2197700000\\
71999.9999999983	4149800000\\
-126999.999999999	-13183100000\\
91000.0000000002	17333700000\\
-19000.0000000001	-14038000000\\
258000.000000003	20385800000\\
-276000.000000003	-37597700000\\
-109999.999999998	31372100000\\
349000	-732600000.000002\\
-348999.999999998	-22582800000\\
221000.000000001	27099600000\\
163999.999999999	-13061700000\\
-220000.000000002	-8056400000\\
-220000.000000001	8788900000.00001\\
-16999.9999999968	488299999.999995\\
383999.999999997	7446300000\\
-38000.0000000002	-12939400000\\
-308999.999999998	-244099999.999995\\
236999.999999998	13549500000\\
90999.9999999993	-13305300000\\
-346999.999999999	3417800000\\
182000.000000002	3540100000\\
147999.999999998	-100000.000002609\\
-404999.999999997	-9765500000\\
276999.999999999	13183500000\\
236999.999999999	488300000.000002\\
-494000	-21362200000\\
163999.999999998	23559300000\\
330000.000000002	400000.000007594\\
-275000.000000001	-22217100000\\
221000.000000004	31616300000\\
-258000.000000003	-37597800000\\
166000.000000002	34302100000\\
90999.9999999975	-21118500000\\
-476000.000000002	1098900000\\
366000	15502600000\\
-17999.999999998	-13549400000\\
-219000	-488599999.999995\\
382999.999999997	13671900000\\
-164000	-16357200000\\
-255999.999999998	5615100000\\
-165999.999999999	732299999.999999\\
476999.999999998	5371400000\\
-17999.9999999998	-6348100000\\
8.88178419700125e-10	3906700000\\
-38000.0000000038	-4761000000.00001\\
-145000	-732299999.999996\\
219000.000000001	10253800000\\
-92000.0000000032	-12207000000\\
-127000	3417999999.99999\\
164000	5371100000\\
-255999.999999998	-11474700000\\
348000.000000002	21728700000\\
-73000.0000000022	-24780500000\\
-367000.000000002	9033400000\\
238000	5615200000\\
239000.000000001	3783900000\\
-93000.0000000017	-15624500000\\
-181999.999999999	7201699999.99999\\
19000.0000000019	200000.000009481\\
125999.999999997	6225599999.99999\\
-70999.9999999997	-10986300000\\
-167000	5371000000\\
131000.000000001	3173700000\\
69999.9999999985	-3539700000\\
94000.0000000012	3417700000\\
-238999.999999999	-10253700000\\
146999.999999998	14526100000\\
181999.999999999	-7079900000\\
-364999.999999998	-7568399999.99999\\
-165000.000000004	6225700000\\
329000.000000001	7812400000\\
165000.000000002	-2563700000\\
-73000.0000000013	-7201699999.99999\\
-146999.999999998	-488500000.000005\\
110999.999999997	7568199999.99998\\
-93000	-8666799999.99999\\
18999.9999999983	8789000000.00001\\
-127999.999999997	-10497900000\\
164999.999999998	15868900000\\
163999.999999998	-11474500000\\
-439000	-6713799999.99999\\
183000.000000002	16479500000\\
109999.999999999	-8911400000\\
-90999.9999999975	1831299999.99999\\
126999.999999995	-121999999.999993\\
-255999.999999999	-6225700000.00001\\
147000.000000001	10986200000\\
54999.9999999988	-4638600000.00001\\
237000.000000002	10010000000\\
-54000.000000002	-24658400000\\
-330000	12451000000\\
127999.999999998	4028700000.00001\\
-292000.000000001	-4761000000\\
292000.000000002	8300800000\\
-201000	-12817300000\\
292000	16479500000\\
20000.0000000013	-14770700000\\
-274999.999999999	1465200000.00001\\
144999.999999996	8422500000\\
2000.00000000244	-6591700000.00001\\
16000	4150600000.00001\\
-161999.999999999	-9155500000.00001\\
308999.999999997	20751900000\\
19000.0000000028	-20751700000\\
-110000	5126800000.00001\\
-200000.000000001	-3295900000.00001\\
52999.999999999	9887600000\\
0	-7445999999.99999\\
240000	11474300000\\
-222000	-18920600000\\
39000.0000000015	13915600000\\
198999.999999997	-243600000.000004\\
-456000.000000002	-15015200000\\
146000.000000002	17578500000\\
382999.999999998	3417900000\\
-271999.999999996	-22461100000\\
-313000.000000004	10376300000\\
641000.000000003	20873500000\\
-255000	-33446600000\\
-277000.000000001	12206400000\\
332000.000000002	15625400000\\
88999.9999999977	-14160200000\\
-106999.999999997	-6469899999.99999\\
-277000.000000005	6469799999.99999\\
-89999.9999999972	488400000.000004\\
181999.999999997	2929499999.99999\\
256000.000000001	7446500000\\
-52999.9999999955	-16479700000\\
16999.9999999986	13550000000\\
-37000.0000000008	-13428000000\\
-144999.999999999	3418200000\\
52999.9999999973	8178699999.99999\\
-34999.9999999993	-7934600000\\
-19000.0000000028	5981300000\\
-54999.9999999997	-6347500000\\
91999.9999999978	7202100000\\
256000.000000001	488300000.000002\\
-477000.000000001	-15014600000\\
167000.000000001	15991100000\\
125999.999999999	-2075100000\\
-107999.999999997	-7202300000.00001\\
16999.9999999986	5127200000\\
36999.9999999999	243899999.999989\\
256000	8545100000.00001\\
-219000.000000001	-23681700000\\
-146999.999999998	14892400000\\
164999.999999999	4883100000\\
72999.9999999968	-7446400000.00001\\
-329000.000000001	-5371099999.99999\\
35000.0000000028	8178700000\\
204000	6835799999.99999\\
290999.999999999	-1220399999.99999\\
-201000	-19775600000\\
-201000	12085000000\\
202000.000000002	3906300000\\
-404999.999999999	-5493300000\\
38999.9999999979	4028400000\\
273000.000000001	488500000.000003\\
-73000.0000000022	-4028700000\\
183999.999999998	6225700000\\
-37000.0000000008	-7324000000\\
-200999.999999998	1342600000.00001\\
90000.0000000007	3173800000\\
18999.9999999975	-1220699999.99999\\
-72000.0000000001	-1586800000\\
-111000.000000002	244000000.000003\\
110000.000000002	2075300000\\
164999.999999997	3417899999.99999\\
8.88178419700125e-10	-5859200000\\
-184000.000000002	-3662499999.99999\\
38000.0000000029	8301199999.99999\\
90999.9999999957	-122200000.000007\\
-110999.999999999	-9765900000\\
130000.000000003	16846200000\\
127999.999999999	-13061900000\\
-148000.000000001	-1342700000\\
-162999.999999998	1220700000\\
-203000.000000003	-121799999.999996\\
404000.000000004	14281800000\\
145999.999999996	-6225400000\\
-127999.999999999	-15014500000\\
16999.9999999995	10864000000\\
-199000	-2807399999.99999\\
-130000.000000003	3051600000\\
331000.000000003	2441400000\\
-458999.999999998	-8910900000\\
238000	9521300000\\
274999.999999996	-2441500000\\
-126999.999999998	-2075100000\\
108000.000000002	732599999.999999\\
-218999.999999998	-3540300000\\
-110000	2807700000\\
146999.999999999	3784300000\\
55000.0000000006	-4638900000\\
53999.9999999985	3296100000\\
-72000.0000000001	-4272600000\\
-275999.999999998	-121900000\\
-54000.0000000038	610200000\\
512000.000000001	11474600000\\
-35999.9999999996	-13549600000\\
19000.000000001	9398999999.99999\\
-350000.000000001	-21239700000\\
-17000.0000000003	22948800000\\
404000.000000003	3906400000.00001\\
-167000.000000002	-22216700000\\
37000.0000000008	19531000000\\
-126999.999999999	-20507500000\\
-37000.0000000026	16967500000\\
255000.000000003	2807799999.99999\\
-254000.000000001	-19531400000\\
199999.999999999	20263800000\\
-202000.000000001	-15136700000\\
-273000	3783899999.99999\\
620999.999999998	18799300000\\
-292000.000000002	-29419300000\\
-110000	16357400000\\
219000	854799999.999999\\
-91000.0000000019	-7202500000\\
-35999.9999999987	4150700000.00001\\
-1000.00000000211	-854800000.000003\\
37000.0000000026	1098900000\\
73999.9999999989	1830900000.00001\\
15999.9999999982	-4150300000\\
-216999.999999999	-5005000000.00001\\
345999.999999998	18554900000\\
2000.00000000244	-20996400000\\
-587999.999999998	6714100000\\
127999.999999997	5249000000\\
588000	3295800000\\
-221000	-12695200000\\
-457000	3784100000\\
108000.000000001	5615200000\\
331999.999999998	-976400000.000002\\
-129999.999999999	-3784400000\\
183999.999999997	4394800000\\
-385000.000000001	-10742600000\\
385000.000000003	19409800000\\
54999.9999999979	-14893300000\\
-513000.000000002	-5736699999.99999\\
603999.999999999	24780000000\\
-310999.999999999	-27343900000\\
-36000.0000000022	15259200000\\
90000.0000000061	-3662500000\\
-328000.000000001	-2563200000\\
383000.000000002	8544700000\\
-291000.000000003	-11108200000\\
52999.9999999999	6225400000\\
330999.999999998	5127100000\\
-36999.9999999981	-7690400000\\
-239000	-4394800000\\
128999.999999999	12207300000\\
-201000.000000001	-12329200000\\
292000.000000002	14282300000\\
-146000.000000003	-14038200000\\
-53999.9999999967	7690500000\\
126000.000000001	-1098600000\\
-199000.000000002	-3051900000\\
163000.000000003	5493300000\\
-218999.999999999	-8056700000\\
311000.000000001	12451300000\\
17999.9999999998	-9521800000\\
-182000.000000003	-976200000.000004\\
53999.9999999994	5248900000.00001\\
-108999.999999998	-4883000000\\
-202000.000000002	-488099999.999999\\
310999.999999999	10742300000\\
366000.000000001	-1709299999.99999\\
-420000	-18310200000\\
128000.000000001	18676400000\\
-386000.000000003	-15746700000\\
368000.000000002	22704800000\\
181999.999999997	-16479500000\\
-456999.999999999	-4150100000\\
291999.999999997	15868900000\\
-236999.999999997	-16113300000\\
346999.999999998	18066500000\\
-219000	-20385800000\\
-183999.999999999	13061700000\\
202999.999999999	-2929999999.99999\\
163000	4638999999.99999\\
-200000	-11352800000\\
-55999.9999999974	9277500000\\
-36000.0000000014	-6958200000.00001\\
273999.999999998	14526700000\\
-273999.999999999	-23193600000\\
37000.0000000008	20507700000\\
382999.999999998	400000.000007594\\
-125999.999999998	-14160600000\\
-185000.000000003	1587300000\\
-274000	1830700000\\
256999.999999997	10986500000\\
237000.000000003	-3051600000.00001\\
-293000.000000004	-16235600000\\
257000.000000001	23681800000\\
54999.9999999997	-19897700000\\
-458000.000000001	4639000000\\
-146999.999999999	1342600000\\
586999.999999998	13916000000\\
-55999.9999999991	-18798900000\\
-402000.000000002	1098899999.99999\\
402000	16113100000\\
-71999.9999999965	-14892700000\\
53999.9999999958	5615500000\\
-293000	-8911300000.00001\\
-8.88178419700125e-10	11474700000\\
367000.000000001	2807500000\\
-221000.000000002	-14282100000\\
19000.000000001	10986200000\\
110000.000000001	-2685500000\\
-238000	-6591700000\\
219999.999999999	13305600000\\
-184000.000000001	-13671900000\\
-91000.0000000011	5371100000\\
237999.999999999	7202100000\\
36000.0000000005	-8178599999.99999\\
-53999.9999999994	3295800000.00001\\
91000.0000000002	-1830900000.00001\\
-91000.0000000011	-2563700000\\
-184000.000000001	610500000.000002\\
56000.0000000009	5126999999.99999\\
53999.9999999976	-3418199999.99999\\
19000.0000000019	1587100000\\
36000.0000000005	-732399999.999994\\
165999.999999999	6103400000\\
-57000.0000000022	-12329000000\\
-474999.999999998	-4760800000.00001\\
221000.000000002	21606400000\\
491999.999999996	-1342600000\\
-381999.999999997	-24414200000\\
-131000.000000001	17211800000\\
277000.000000002	5981700000\\
-56000.0000000018	-15747200000\\
-218999.999999997	5981400000\\
200999.999999997	6225800000.00001\\
-111000	-8301200000\\
-33999.9999999989	3662600000.00001\\
271999.999999998	7446000000\\
-255000.000000002	-17578200000\\
55000.0000000006	15869500000\\
53999.9999999976	-8423300000.00001\\
1000.00000000033	5859800000.00001\\
-147000	-10742500000\\
255999.999999999	18554800000\\
-200000	-20996100000\\
-38000.0000000011	13183600000\\
164999.999999999	-1586800000\\
184000	1952800000\\
-256999.999999999	-12084500000\\
-36999.999999999	11229900000\\
-275000.000000003	-5004300000\\
-34999.9999999993	1464500000\\
474999.999999998	7202300000\\
110000.000000002	-6347900000\\
-53999.9999999976	1220900000\\
-313000.000000004	-11352400000\\
146999.999999999	18432300000\\
19000.0000000019	-11352400000\\
72999.9999999968	5127100000.00001\\
-274999.999999998	-9033400000\\
145999.999999999	13671900000\\
56999.9999999986	-8666900000\\
-111999.999999998	366100000\\
146999.999999998	5127000000\\
-53999.9999999985	-6469599999.99999\\
89999.9999999998	4638400000\\
-291000.000000001	-6957800000\\
-74999.9999999957	5615300000\\
603999.999999997	11108100000\\
-436999.999999998	-24535800000\\
125999.999999998	20019200000\\
-146000.000000001	-14037700000\\
165000.000000001	15868800000\\
-110000.000000001	-18432500000\\
202000.000000001	23071400000\\
-166000.000000001	-27466000000\\
-34999.9999999984	20752100000\\
327999.999999997	-2075300000\\
-603999.999999999	-21972500000\\
459000	36987000000\\
-166000	-34057400000\\
-8.88178419700125e-10	21240400000\\
-17999.9999999989	-10376400000\\
202000.000000003	11353000000\\
-275000.000000003	-19775800000\\
73000.0000000013	17334200000\\
183000	976499999.999996\\
-35999.9999999996	-13427600000\\
-111000.000000004	8666799999.99999\\
-72999.9999999968	-6591700000\\
166000.000000002	12573300000\\
-55000.0000000024	-11841000000\\
-38000.0000000011	4394699999.99999\\
-54999.9999999997	-2075300000\\
112000.000000002	6591900000\\
-277000.000000003	-15747200000\\
221000	22583200000\\
273999.999999999	-8301000000.00001\\
-255999.999999998	-11474400000\\
-17999.999999998	9399099999.99998\\
256000	5615600000.00001\\
-329000	-18799000000\\
16999.9999999968	18310400000\\
92999.9999999991	-8300400000\\
-93000.0000000008	2196900000\\
-108999.999999999	-2685300000\\
202000	7324000000\\
34999.9999999993	-5492999999.99999\\
57000.0000000004	3295800000\\
33999.999999998	-4394399999.99999\\
-472999.999999997	-10254100000\\
307999.999999998	26123300000\\
131000.000000004	-17822500000\\
-999.999999999446	8544999999.99999\\
-349000.000000005	-22094700000\\
496000.000000007	45654300000\\
-19000.0000000028	-46264700000\\
-311000.000000002	15258900000\\
-19000.000000001	5126800000.00001\\
-182000	-6957900000\\
236999.999999997	11718700000\\
55000.0000000024	-10131800000\\
201999.999999996	12451100000\\
-146999.999999999	-19653100000\\
36999.9999999981	13793600000\\
-201999.999999999	-10864000000\\
-89999.999999999	10864200000\\
180999.999999996	-2197399999.99999\\
166000	1221000000\\
-274000.000000001	-10498400000\\
108000.000000001	15381100000\\
202999.999999998	-10986400000\\
-586999.999999996	-1953100000\\
439999.999999996	13183500000\\
-90999.9999999984	-11718600000\\
162999.999999998	7690400000.00001\\
-107999.999999998	-7934800000\\
-19000.0000000001	5737599999.99999\\
-90999.9999999993	-4883000000\\
254999.999999995	12695500000\\
-309999.999999997	-24536300000\\
71999.9999999974	25512700000\\
294000.000000003	-7812400000\\
-273999.999999999	-11474700000\\
70999.9999999979	14526500000\\
-89000.0000000013	-10620400000\\
-38999.9999999988	5005300000.00001\\
294999.999999998	10375500000\\
-130000	-19286600000\\
-144999.999999996	6713399999.99999\\
127999.999999999	6958300000.00001\\
53999.9999999958	-4272600000\\
-309999.999999996	-9887599999.99999\\
126999.999999994	15502900000\\
238000.000000003	-366199999.999989\\
-145999.999999999	-12817500000\\
-236999.999999999	2441500000\\
638999.999999999	27221800000\\
-402000	-46142800000\\
0	31860400000\\
-164000.000000001	-15624800000\\
-1999.99999999889	10619900000\\
478000.000000001	10009800000\\
-312000.000000002	-28930600000\\
-238000.000000002	14770500000\\
494000	15136600000\\
-128000.000000001	-24902100000\\
-346999.999999997	6347400000\\
217999.999999995	13183800000\\
94000.000000003	-13428000000\\
-167000	3906600000\\
54999.9999999997	2807400000\\
38000.0000000002	-3784300000\\
-147000.000000003	366600000.000002\\
91000.0000000002	1830700000\\
19000.0000000019	1098700000\\
-1000.000000003	-4394400000\\
130000.000000002	7812400000\\
-148000	-11962800000\\
-19000.0000000019	9887499999.99999\\
185000.000000005	488500000.000012\\
0	-4516700000.00001\\
-222000.000000004	-7568300000\\
75000.0000000037	15991000000\\
1.77635683940025e-09	-11596400000\\
-128000	2441300000\\
328999.999999998	13305600000\\
-202000.000000001	-24902300000\\
999.999999997669	20996100000\\
72999.9999999995	-11840800000\\
-274999.999999999	-732399999.99999\\
403000.000000001	19164900000\\
-181999.999999999	-25512600000\\
-2000.00000000244	14404500000\\
184000.000000002	243699999.999987\\
-146000.000000002	-10009400000\\
-109999.999999999	7201900000\\
-238000.000000001	-3295600000\\
437999.999999999	10619800000\\
-180999.999999998	-15136500000\\
163999.999999998	14770400000\\
-257000.000000001	-17089800000\\
19000.000000001	11230400000\\
127999.999999998	3418100000.00001\\
274000.000000004	122000000.000003\\
-291000.000000001	-17212100000\\
15999.9999999982	18066900000\\
148000.000000001	-5615900000.00002\\
-294000.000000001	-8056000000\\
111000.000000001	11718300000\\
90999.9999999984	-610100000.000003\\
-129000.000000001	-9887800000\\
1000.00000000033	7812400000\\
1.77635683940025e-09	-2685300000\\
219000	11596500000\\
-89999.9999999998	-19287000000\\
-112000.000000002	6591600000\\
221000.000000001	13427900000\\
-255999.999999998	-26489200000\\
126999.999999995	27221500000\\
-71999.9999999983	-20629700000\\
-111000.000000002	9399300000.00001\\
183999.999999997	3540000000\\
310000.000000001	6347700000\\
-492999.999999999	-32592600000\\
54999.9999999988	35400000000\\
-1000.00000000122	-21972200000\\
73999.9999999981	16967400000\\
274000	-7446000000\\
-457000	-10986600000\\
-38000.0000000011	12329300000\\
130000.000000002	-2441500000.00001\\
365000.000000004	14160200000\\
-127000.000000002	-22949200000\\
-56999.9999999977	9521400000\\
-181000	-9399300000.00002\\
16999.9999999995	14526300000\\
293000.000000002	5493099999.99998\\
-273000	-28198200000\\
52999.9999999982	23681800000\\
165000.000000001	-200000.000018008\\
-181999.999999999	-17211900000\\
-129000.000000001	9277499999.99999\\
330000.000000003	14159900000\\
-74000.0000000016	-22948900000\\
-345999.999999997	6713700000\\
362999.999999996	14282100000\\
-142999.999999996	-20263500000\\
-3000.00000000189	14892600000\\
21000.0000000017	-9765800000\\
-75000.0000000037	5005100000\\
256000.000000001	6957900000\\
-53000.0000000017	-14526500000\\
-276999.999999998	3052099999.99999\\
112000.000000002	7445900000\\
53999.9999999976	-3906100000\\
91000.0000000019	4150600000\\
999.999999997669	-7324500000\\
-347999.999999999	-4150299999.99999\\
364999.999999998	20263700000\\
56000.0000000009	-15380800000\\
-439000.000000001	-10376200000\\
382999.999999999	30029600000\\
56000.0000000009	-22217100000\\
-110000	2563700000.00001\\
-73000.0000000004	1464699999.99999\\
72999.9999999986	2075400000\\
-458000	-6103800000\\
512999.999999999	15381200000\\
91000.0000000011	-13306100000\\
-218999.999999999	1099000000\\
-55999.9999999991	-366299999.999998\\
-163000	2685400000\\
327999.999999997	4150600000\\
-127999.999999997	-8545100000\\
56000.0000000027	7080300000.00001\\
72999.9999999959	-1953400000\\
-74999.9999999984	-4638600000\\
149000.000000001	11230700000\\
70999.9999999971	-14770700000\\
-547000	1464800000.00001\\
217000.000000001	13061500000\\
75999.9999999978	-10619800000\\
-167000.000000001	3417500000\\
294000	3906700000\\
-238000.000000001	-11230800000\\
458000.000000001	23681800000\\
-331000	-34668000000\\
-291000	19042900000\\
583999.999999999	14038200000\\
-218000	-26611400000\\
-56000.0000000018	12207100000\\
-218999.999999999	-3906300000\\
71999.9999999983	7690300000\\
57000.0000000013	-7201800000\\
382000	15990800000\\
-273000	-27465500000\\
-165000	17333800000\\
91999.9999999987	-4760699999.99999\\
-56000.0000000009	3173800000\\
-54999.9999999997	-4638500000\\
367000	17211800000\\
-128000.000000001	-25757000000\\
35000.0000000001	20141800000\\
-52999.9999999973	-18066400000\\
-293000	7934499999.99999\\
510999.999999997	12939500000\\
-603000.000000001	-25757000000\\
293000	27222000000\\
35999.9999999987	-20019900000\\
-311000.000000001	6347900000\\
384000.000000002	6225600000\\
-16999.9999999977	-5371200000\\
-165999.999999999	-4638700000\\
294000.000000001	17700300000\\
108999.999999999	-17212000000\\
-457000.000000002	-9277300000\\
72000.0000000001	23071300000\\
-36000.0000000014	-15625000000\\
1000.00000000033	10620100000\\
71999.9999999983	-5859299999.99999\\
292999.999999999	13183300000\\
-145999.999999998	-22826700000\\
-293000	6469300000\\
54999.9999999997	7446700000\\
365999.999999996	7201800000.00001\\
-183999.999999997	-19042800000\\
-70999.9999999988	8789000000\\
70999.9999999971	1587000000\\
-163000.000000001	-2929800000\\
182000.000000001	3173900000\\
-347999.999999998	-5371100000\\
182999.999999998	6958100000\\
221000.000000001	-1343100000\\
-1000.000000003	-1098200000\\
-128000	-5249400000\\
220000.000000001	8545300000\\
-165999.999999999	-8789500000\\
-328000	4761100000\\
-19000.000000001	488100000.000001\\
275000	488399999.999999\\
126999.999999998	-122200000.000001\\
-90000.0000000016	-1464700000\\
-19000.0000000001	-200000.000000955\\
-55000.0000000006	-976400000.000002\\
-55000.0000000006	732500000\\
275000.000000001	8910799999.99999\\
-73999.9999999989	-15502600000\\
-53999.9999999985	13549700000\\
238000	-4760800000.00001\\
-404000.000000003	-15747000000\\
-16999.9999999986	21728500000\\
72999.9999999995	-10254000000\\
164000.000000001	11963100000\\
20000.0000000013	-13672100000\\
-76000.000000004	2075299999.99999\\
-70999.9999999971	2929700000\\
-238000.000000001	-5126900000\\
382999.999999999	14892400000\\
148000	-11108200000\\
-274999.999999999	-6347900000\\
-257999.999999999	7446600000\\
203999.999999998	2929500000\\
253999.999999999	488299999.999998\\
-290999.999999998	-11963000000\\
199999.999999997	17456300000\\
-181999.999999999	-18921000000\\
-91999.9999999996	13793800000\\
438999.999999996	4272700000\\
-347999.999999999	-22583200000\\
-34999.9999999975	20263800000\\
216999.999999996	-1464900000\\
76000.0000000049	-1220800000.00001\\
34999.9999999993	-8422700000\\
-237000.000000002	-122100000.000015\\
71999.9999999983	9521499999.99999\\
-328999.999999999	-11596800000\\
402999.999999999	19775500000\\
-311999.999999999	-25878900000\\
330999.999999998	28076100000\\
-999.999999997669	-21484400000\\
-312000.000000002	2685700000\\
405000.000000002	14770500000\\
-405000.000000003	-24780600000\\
294000.000000004	27466300000\\
18999.9999999983	-19531600000\\
-313000.000000002	3906400000\\
239000.000000001	7080000000\\
-91000.0000000011	-8666900000\\
16999.9999999986	7080000000\\
111000.000000002	-4028300000\\
-202000	-1342800000\\
18999.9999999992	3784300000\\
145000.000000001	-244299999.999999\\
-17000.000000003	-2197100000\\
-220000	-2563700000\\
164999.999999999	7202400000\\
237000.000000002	-199999.999997402\\
-182000.000000002	-9521300000.00001\\
-311999.999999999	1464600000\\
330000	12329300000\\
165000.000000001	-4516600000.00001\\
-274999.999999999	-13549900000\\
347999.999999999	21606400000\\
-293000.000000001	-24169800000\\
-439000.000000002	11962900000\\
328000.000000002	6591700000\\
313000.000000001	-4028300000\\
-366000.000000001	-7934499999.99999\\
346000	15869200000\\
2000.00000000067	-13306000000\\
-185000.000000003	-854099999.999995\\
-107999.999999999	2197000000\\
52999.999999999	5737500000\\
149000.000000002	-1342800000\\
51999.9999999978	-4150700000\\
-327000	-2685000000\\
182000	11107800000\\
91000.0000000037	-10009200000\\
-90999.9999999993	4394100000\\
-275000.000000001	-5004700000\\
238999.999999998	8667000000\\
127999.999999999	-3906300000\\
127000	488400000.000006\\
-530000	-10742500000\\
237000	16968200000\\
441000.000000004	488000000\\
-550000.000000002	-25024400000\\
145999.999999998	27954300000\\
257000.000000001	-9155599999.99999\\
-110000.000000002	-4760400000.00001\\
-294000	-3906499999.99999\\
57000.000000003	12817400000\\
382999.999999996	-243799999.999997\\
-365999.999999996	-15503500000\\
128000	15259300000\\
92000.0000000005	-5615500000\\
-202000.000000001	-2563400000\\
-107999.999999999	3051600000\\
162000.000000002	1587300000\\
184999.999999996	853999999.999999\\
-220000	-8178200000\\
73000.0000000004	9154700000\\
164000	-1708399999.99999\\
-52999.999999999	-4517000000\\
-184999.999999997	244200000.000002\\
-53999.9999999994	2807800000\\
55000.0000000006	1342600000\\
108999.999999999	-2685400000\\
-364999.999999999	-3296100000\\
547999.999999998	14770600000\\
-16999.9999999986	-15747000000\\
-257000.000000003	3417900000\\
-35999.9999999996	-976600000.000006\\
-1000.00000000033	6103800000\\
146000	-977100000.000014\\
39000.0000000032	-4394099999.99999\\
52999.9999999955	6713799999.99999\\
-237999.999999999	-15014700000\\
36999.9999999981	15502800000\\
74000.0000000007	-5004700000.00001\\
-37999.9999999976	-2319300000.00001\\
-126999.999999999	732200000.000014\\
329000.000000001	12329200000\\
-127999.999999999	-20629700000\\
-36999.9999999981	12817100000\\
-36000.0000000005	-6591600000\\
8.88178419700125e-10	5004799999.99999\\
-74000.0000000007	-3906300000\\
274999.999999999	12573400000\\
-256000	-24780400000\\
-54999.9999999997	20996100000\\
108999.999999998	-5981400000\\
185000.000000002	4882800000\\
-259000.000000001	-15747100000\\
148999.999999998	19653400000\\
-20000.0000000005	-15258900000\\
74000.0000000016	10376000000\\
-329000	-13671700000\\
16999.9999999959	13671600000\\
458000.000000002	5859500000\\
-54000.000000002	-16601500000\\
-404000.000000001	-1464900000\\
385000.000000001	19409100000\\
-237000.000000002	-20873800000\\
-165999.999999998	11352300000\\
402000	3540100000\\
-217000	-12206900000\\
-75999.9999999978	8911000000\\
275999.999999997	488300000.000002\\
-365999.999999999	-9277400000\\
256000	13794200000\\
127000.000000001	-9888000000\\
-436999.999999999	-1953000000\\
273000.000000001	9643500000\\
-36999.9999999981	-6957800000\\
92999.9999999955	3661900000\\
-73999.9999999972	-3418000000\\
-110999.999999999	-121900000.000004\\
239999.999999997	6469700000\\
-184000	-9277600000.00001\\
-182999.999999997	1221100000\\
550000	18798500000\\
-240000.000000002	-29662900000\\
-181000	16235200000\\
236999.999999999	3296200000.00001\\
-1.77635683940025e-09	-9277700000\\
-493999.999999999	-2807500000\\
529999.999999999	19287300000\\
-17000.0000000012	-17822600000\\
-438999.999999998	-487999999.999998\\
364000.000000002	14404200000\\
-90000.0000000043	-14404300000\\
201000.000000001	13061400000\\
-128000.000000002	-13549600000\\
-90999.9999999984	6103500000\\
-74999.9999999984	243900000.000001\\
-126000.000000002	-976200000.000001\\
383000	5736900000\\
-146000.000000001	-8910700000\\
-110000.000000001	4150000000\\
111000.000000001	244400000\\
181000.000000001	6591700000\\
-292000.000000002	-20507800000\\
38000.0000000038	23193300000\\
253999.999999996	-8544800000\\
-181999.999999999	-5493300000.00001\\
-255999.999999998	-3173700000\\
511999.999999999	29907200000\\
-309999.999999999	-45898600000\\
107999.999999997	40283400000\\
-89999.999999999	-29174800000\\
-56000.0000000018	18310400000\\
440000.000000001	4028400000\\
-329999.999999999	-24658300000\\
-91000.0000000011	16723900000\\
-238000	-3906500000\\
311000	7690600000\\
109999.999999996	-5249400000\\
18000.0000000025	1587500000\\
17999.999999998	-5859800000\\
-364999.999999999	-854399999.999997\\
-110999.999999997	5249100000\\
384999.999999996	4638500000\\
128000.000000001	-2196900000\\
90999.9999999993	-1709500000\\
-345999.999999999	-10863700000\\
-148000.000000002	9032700000\\
73000.0000000004	2075400000\\
238999.999999999	4516799999.99999\\
-999.999999999446	-7812800000\\
56000	5371200000\\
-366999.999999998	-18798800000\\
219999.999999999	28808600000\\
147000.000000002	-15502900000\\
-386000.000000001	-7812800000.00003\\
423000.000000003	27954700000\\
-74000.0000000016	-29175500000\\
-257000	5005500000\\
237999.999999999	17821800000\\
166000.000000004	-13671600000\\
-385000.000000002	-9033300000.00001\\
-110000.000000001	14526500000\\
218999.999999998	-976900000.000003\\
93000.0000000035	-976299999.999998\\
35999.9999999969	-244100000.000005\\
-257000.000000001	-10986400000\\
238000	20019500000\\
-17000.0000000003	-14282300000\\
-238999.999999998	-2929499999.99999\\
219999.999999999	16113100000\\
-201000	-20019300000\\
236999.999999998	22460600000\\
54999.9999999997	-12939200000\\
-89999.9999999972	-4760700000.00001\\
52999.9999999964	7568200000\\
-420000.000000001	-9033200000\\
439000.000000003	19653400000\\
-109000.000000003	-20507800000\\
-55999.9999999991	11230300000\\
-146000.000000003	-8422500000\\
146000	10863800000\\
129000	-5981100000\\
-37999.9999999985	1098600000.00001\\
130000.000000002	1952899999.99999\\
-166000.000000001	-9887499999.99999\\
-72000.0000000018	9887700000\\
-38999.9999999961	-4638800000\\
-89000.0000000048	2319400000\\
236000.000000002	3662200000\\
37999.9999999985	-5615300000\\
-311999.999999999	-3784300000\\
275999.999999999	12817600000\\
-1999.99999999978	-10376200000\\
18999.9999999983	5615499999.99999\\
-127000	-8178999999.99999\\
327999.999999998	19165200000\\
-329000.000000001	-30273500000\\
17999.9999999998	22705300000\\
-17000.0000000003	-8789500000\\
-75000.0000000002	3174300000\\
258000.000000003	6103100000\\
-19000.0000000046	-10009400000\\
-165999.999999999	121800000.000002\\
-200000	199999.999999534\\
257000.000000001	11596400000\\
198999.999999997	-8178300000\\
-436999.999999998	-10132200000\\
309999.999999997	21606600000\\
146000.000000001	-9155300000\\
-108000	-10864300000\\
-130000.000000002	8911299999.99999\\
-54999.9999999997	-1953299999.99999\\
-235999.999999999	-1586900000\\
510999.999999997	15747200000\\
-238000	-23926000000\\
38000.0000000029	16113600000\\
-20000.000000004	-6348100000\\
55999.9999999983	4395100000\\
-55999.9999999983	-7324800000.00001\\
165999.999999999	12207400000\\
-257000.000000002	-19775400000\\
220000.000000001	23437200000\\
-147000.000000003	-19408700000\\
-163999.999999999	5980900000\\
145999.999999999	6470199999.99999\\
146000	-1831399999.99999\\
184000	3296200000\\
-695999.999999999	-31494400000\\
549000.000000003	54321500000\\
164999.999999999	-36865500000\\
-182999.999999999	3296100000.00001\\
-421000.000000002	-2075200000.00001\\
310999.999999997	19043000000\\
73000.0000000022	-17700300000\\
-202000.000000003	4516500000\\
313000.000000005	8301099999.99999\\
-349000.000000002	-18921200000\\
-91000.0000000002	13183800000\\
439000	10009600000\\
109999.999999999	-11352500000\\
-458000.000000002	-15258500000\\
128000	23437000000\\
56000	-7934299999.99999\\
89999.999999999	3051899999.99999\\
-125999.999999998	-12329400000\\
88999.9999999977	16723800000\\
-291000.000000003	-18920900000\\
237000.000000001	21606400000\\
-183000.000000001	-20141600000\\
239000.000000003	20385700000\\
-111000.000000002	-20385500000\\
274999.999999999	28564100000\\
-146999.999999998	-36987000000\\
-127000.000000002	21850300000\\
-55999.9999999974	-8910899999.99999\\
184000.000000001	14404200000\\
-19000.0000000019	-14160200000\\
-146000	1831300000\\
-55000.0000000015	6225200000.00001\\
291999.999999998	-1464600000\\
-144999.999999999	-4394500000\\
-239000.000000003	-610500000.000003\\
147000.000000003	7202200000\\
273999.999999996	610399999.999998\\
-90999.9999999984	-10620100000\\
-238000	5615200000.00001\\
-220000.000000003	-3540100000\\
385000.000000001	10742200000\\
457000.000000001	2319500000\\
-714000.000000003	-28076500000\\
166000.000000001	28198600000\\
125999.999999999	-9155400000\\
-181000.000000002	-3662300000\\
200000.000000001	9033500000\\
19000.000000001	-7324400000\\
-128000.000000001	-488199999.999999\\
35999.9999999978	2807500000\\
-164000	-1586700000\\
-74000.0000000025	-488600000\\
293000.000000002	7446600000\\
184000.000000001	-3540100000\\
-184000.000000004	-8056900000.00001\\
-54999.9999999962	6225900000\\
-200999.999999999	-3418100000\\
-37000.0000000008	3784300000\\
274999.999999999	3295600000\\
36999.9999999999	-4394300000\\
90999.9999999984	2441500000\\
-238000.000000001	-8301099999.99999\\
-18999.9999999992	7202500000\\
-34999.9999999984	-2075500000.00001\\
17999.9999999989	2075500000.00001\\
163000	1952800000\\
129999.999999999	1953300000\\
-36999.9999999999	-11352500000\\
-330000	610299999.999992\\
109999.999999998	11840800000\\
72999.9999999995	-7690500000\\
110000.000000001	5493400000\\
-292000	-11474800000\\
163999.999999999	12695300000\\
-1.77635683940025e-09	-5126900000\\
182000.000000001	2441400000\\
-144000	-7202100000\\
-276999.999999997	2929700000\\
92999.9999999955	4760500000\\
146000.000000002	-4027900000\\
35999.9999999996	5736899999.99999\\
2000.00000000156	-9887500000\\
51999.9999999987	12329200000\\
-107000.000000002	-15381100000\\
-148999.999999998	8422899999.99999\\
130000.000000003	1587200000\\
255999.999999999	11108100000\\
-999.999999996781	-21850600000\\
-420000.000000001	-1342599999.99998\\
165000	17211800000\\
-74999.9999999984	-8544800000\\
349999.999999999	18554500000\\
53000.0000000008	-31005700000\\
-364000.000000002	9399300000.00001\\
-149000.000000001	6958200000\\
277000.000000001	2441100000\\
53000.0000000026	-6347300000\\
-328000	-488699999.999999\\
52999.9999999982	3296400000\\
149000	-122499999.999998\\
89999.9999999981	854599999.999997\\
55000.0000000024	-1708800000\\
-109999.999999999	-2929800000.00001\\
-164000	1342600000\\
109000.000000001	4272800000\\
91999.9999999987	-2075499999.99999\\
-8.88178419700125e-10	-854400000.000007\\
73000.0000000004	732600000.000009\\
-128000	-2441700000.00001\\
-55000.0000000015	122300000.000003\\
-126999.999999997	1708900000\\
199000.000000002	2319300000\\
147999.999999995	854500000\\
-90999.9999999966	-6835800000\\
-330999.999999999	-366499999.999994\\
439999.999999999	12451500000\\
-256000.000000001	-15747200000\\
-37000.0000000008	10131700000\\
55000.0000000015	-4150100000\\
92000.0000000005	5126500000\\
238000	-2074599999.99999\\
-587000.000000003	-12573800000\\
440000.000000001	21973000000\\
-16999.9999999995	-13672000000\\
-112000.000000001	244199999.999994\\
-255000.000000001	-3051800000\\
273999.999999999	13671900000\\
92000.0000000005	-7568500000\\
-91999.999999997	-5370800000\\
74000.0000000007	11596300000\\
-92000.0000000023	-15014200000\\
-17999.9999999998	10741800000\\
-111000.000000001	-7079900000.00001\\
257999.999999999	15624900000\\
-239000.000000002	-24902100000\\
146000.000000001	25146100000\\
1000.00000000033	-17089400000\\
-384999.999999999	-3052200000\\
696000.000000001	34546100000\\
-36999.999999999	-39062300000\\
-347000	4760400000.00001\\
-184000.000000004	10254100000\\
91999.9999999996	366100000\\
91000.0000000002	-1708800000\\
36999.9999999981	1342600000\\
37000.0000000017	-2197300000\\
16999.9999999986	976799999.999997\\
130000.000000003	4760499999.99999\\
-239000.000000002	-16967600000\\
-109999.999999999	14282100000\\
219999.999999999	1098600000\\
128000	-854300000.000003\\
-183000	-9643700000.00001\\
-292000	3906200000.00001\\
345999.999999998	11474900000\\
-109000.000000002	-14893000000\\
1.77635683940025e-09	9033500000\\
220000.000000001	1708900000\\
-36999.9999999999	-7080100000.00001\\
-128000	-1220600000\\
-238000	244000000.000003\\
202000.000000002	9399500000\\
-1000.00000000122	-7812600000\\
110000	6713999999.99999\\
-73000.0000000004	-10009800000\\
147000.000000002	15258800000\\
-129000.000000001	-21240400000\\
220000	20752200000\\
-312000.000000001	-21606700000\\
-346000	9643899999.99999\\
438000.000000001	12939100000\\
347999.999999999	-2197100000.00001\\
-658999.999999999	-31616200000\\
182999.999999998	36499000000\\
184000.000000005	-12695400000\\
236999.999999998	12085100000\\
-183000.000000002	-28686500000\\
-128000.000000001	15624800000\\
-293000.000000001	-2685400000\\
274000	12573300000\\
38000.000000002	-13794200000\\
164000	9521799999.99999\\
-346999.999999999	-14160300000\\
199999.999999997	17578000000\\
-109999.999999999	-14892400000\\
92999.9999999982	11596700000\\
-54999.9999999997	-9887900000\\
-130000	4394800000\\
350000	7690100000\\
-294000.000000002	-17089500000\\
54999.9999999997	15502800000\\
219000	-4150500000.00001\\
57000.0000000013	2319400000.00001\\
-332000.000000002	-17944300000\\
-52999.9999999999	19165100000\\
347000	2075000000\\
-220000.000000002	-15624800000\\
111000.000000002	12207000000\\
-295000.000000003	-9643700000\\
259000.000000002	13549900000\\
180999.999999997	-8300800000\\
-146000.000000001	-3539900000\\
-200999.999999999	3295700000\\
-146000.000000001	1098700000\\
219000	854599999.999999\\
182000.000000001	1464500000\\
75999.9999999987	-1464300000\\
-350999.999999998	-8911600000\\
38999.999999997	10498200000\\
36000.0000000005	-1220500000\\
346999.999999999	4150000000\\
-256000.000000001	-14404000000\\
-181999.999999997	9765400000\\
145999.999999997	122500000.000002\\
109000	1830500000\\
-128000.000000001	-8178400000\\
73999.9999999998	10009700000\\
73000.0000000004	-4516500000\\
-92000.0000000032	-4882900000\\
-218999.999999997	3295800000\\
328999.999999997	7080200000\\
-220000	-11474700000\\
-16999.9999999995	7324500000\\
-2000.00000000067	-3418400000\\
423000.000000001	14770900000\\
-311999.999999999	-27832400000\\
-238000	13183900000\\
457999.999999998	17578100000\\
-56000.0000000009	-24902700000\\
-182000.000000002	5005399999.99999\\
8.88178419700125e-10	5859100000\\
-129000.000000002	-2319500000\\
221000.000000001	2685900000\\
-110999.999999999	-4150500000\\
-293000	-1587200000\\
880000.000000001	17822800000\\
-678999.999999999	-31372700000\\
-145000	23316000000\\
218999.999999998	-6103800000\\
275000	3417900000\\
-128999.999999996	-7079900000\\
-35999.9999999996	3417900000\\
-292999.999999998	-3418000000\\
-8.88178419700125e-10	4516700000\\
292999.999999997	2685500000\\
220000.000000001	-244300000.000003\\
-221000.000000002	-9399100000\\
-181999.999999999	4272100000\\
165000.000000001	4883100000\\
-91000.0000000002	-5249100000\\
-2000.00000000067	2197200000\\
37999.9999999985	488400000\\
109000	2197000000\\
999.999999999446	-4516200000.00001\\
18000.0000000007	2563100000.00001\\
-129000.000000002	-5615000000.00001\\
-108999.999999999	5249000000\\
1000.00000000122	-1587100000.00001\\
363999.999999998	10254200000\\
-254999.999999999	-18432900000\\
-293999.999999999	6836200000\\
348999.999999998	9521200000\\
147000.000000001	-6835799999.99999\\
-295000.000000003	-5004700000.00001\\
38000.0000000038	4272200000.00001\\
458000	14770600000\\
-770000.000000003	-40283300000\\
422000.000000001	45044200000\\
127999.999999999	-22461200000\\
-221000	-1098499999.99999\\
57000.0000000022	8422799999.99999\\
-276000.000000001	-16479400000\\
440000.000000002	35644300000\\
146000.000000001	-29662700000\\
-511999.999999999	-10132300000\\
90999.9999999984	26001400000\\
219000	-2685899999.99999\\
-144999.999999998	-16112900000\\
90999.9999999993	19164600000\\
-200999.999999999	-26610900000\\
327999.999999998	44067000000\\
-16000	-43578800000\\
-92999.9999999991	15258600000\\
-220000.000000003	-3784100000\\
91999.9999999996	13305700000\\
147000.000000001	-11230500000\\
-367000.000000002	-1587100000\\
201999.999999998	8667400000\\
402000	6225100000\\
-364999.999999999	-24657700000\\
-55999.9999999983	17821800000\\
239000.000000001	-854199999.999995\\
-275000	-8789200000\\
328999.999999996	14282400000\\
-328999.999999997	-16235500000\\
54000.0000000003	12573200000\\
-35000.0000000001	-7568300000\\
218999.999999998	7202200000\\
-36999.9999999964	-7202100000\\
-54999.9999999979	4272200000\\
-8.88178419700125e-10	-3783900000\\
-292000.000000001	1953100000\\
439000.000000001	5981200000\\
-166000.000000002	-10253700000\\
-327999.999999998	2563500000\\
365999.999999998	6591700000\\
127000.000000001	-2441400000.00001\\
1999.99999999978	-2197199999.99999\\
-277000.000000002	-6714000000\\
-126999.999999999	8301000000\\
733000.000000002	13061300000\\
-440000.000000004	-28930500000\\
-201999.999999999	15258700000\\
184000	1342800000\\
-256999.999999996	-3784200000\\
72999.9999999986	2685500000\\
92999.9999999973	-488100000.000003\\
402000	7202000000\\
-604999.999999999	-22338900000\\
146999.999999998	21606600000\\
349000.000000002	-2075399999.99999\\
-112000	-4882600000\\
2000.00000000067	-4272600000\\
-367000.000000003	-1953100000\\
127000.000000002	11474700000\\
75999.9999999987	-4761000000\\
400000	10132200000\\
-255000.000000003	-23315700000\\
-494000.000000001	8789200000\\
512000.000000001	14160100000\\
109999.999999999	-12451200000\\
-274999.999999998	-1464800000\\
-365000	2197200000\\
89999.999999999	2685800000\\
623999.999999999	5492700000\\
-75000.0000000019	-7689999999.99999\\
-108000	-2563599999.99999\\
-293999.999999999	610199999.999995\\
494000.000000001	9033300000\\
-676000	-12573100000\\
273000.000000001	10497800000\\
-17000.0000000012	-6225400000\\
-18999.9999999992	3173800000\\
91999.999999997	-1465000000\\
127000.000000001	4638900000\\
-16999.9999999977	-8178900000.00001\\
-92000.0000000005	5981600000.00001\\
36999.999999999	-2197399999.99999\\
-19000.0000000019	-366100000.000003\\
19000.0000000001	1464800000.00001\\
-148000.000000001	-5493300000\\
-70999.9999999953	4028600000\\
364999.999999997	13183300000\\
165000.000000002	-14526100000\\
-239000.000000004	-8300900000\\
-712999.999999997	4028200000\\
622999.999999999	20386000000\\
199999.999999998	-15625300000\\
20000.0000000013	3296100000.00001\\
-405000.000000002	-11718700000\\
-145000	11108300000\\
238000.000000001	1464700000.00001\\
257000.000000001	4638899999.99999\\
-167000.000000002	-14892600000\\
37999.9999999994	12573000000\\
129000.000000004	-3417500000.00001\\
-295000.000000002	-12451700000\\
442000	25879300000\\
-497000.000000003	-32471000000\\
-125999.999999999	24536400000\\
273000	-6225800000\\
73999.9999999972	1343000000\\
-54999.9999999979	-3784499999.99999\\
219999.999999999	6225900000\\
-459000	-16113400000\\
332000.000000002	22338800000\\
88999.9999999968	-12206900000\\
-144999.999999998	-3784400000\\
-92000.0000000023	6714200000\\
-218999.999999998	-2930100000\\
420000.000000002	5859900000\\
1.77635683940025e-09	-5982000000\\
-475000	-3051400000\\
127000	5737200000\\
568999.999999998	8667000000\\
-313000.000000001	-19775500000\\
-273000.000000001	9766000000\\
90999.9999999993	-600000.000001444\\
421000	11353100000\\
-1000.00000000033	-19653700000\\
-346000.000000001	6470000000\\
-458999.999999994	-1831200000\\
310999.999999998	5981400000\\
386000	4760800000\\
-220999.999999999	-13305600000\\
91999.9999999987	11108400000\\
110000.000000003	-1953300000\\
73000.0000000004	1220800000.00002\\
-312000	-22338700000\\
-181999.999999999	22948900000\\
476000.000000003	8056900000.00001\\
-238000.000000002	-21606400000\\
-55999.9999999991	6103100000.00001\\
56000	7202699999.99998\\
55000.0000000033	-3296400000.00001\\
-202000.000000001	-10253500000\\
458000	34301600000\\
-109999.999999999	-43091000000\\
-55000.0000000024	13916300000\\
-275000.000000001	4638500000\\
-89999.9999999998	1220700000\\
199999.999999998	976800000\\
54999.9999999997	-4028600000\\
-273999.999999998	-2685499999.99999\\
512999.999999999	14892600000\\
-258000.000000003	-20385600000\\
-126999.999999999	9887600000\\
403000	9155100000\\
-403000.000000001	-22582700000\\
-38000.0000000002	16113100000\\
186000.000000001	732299999.999995\\
252999.999999998	1099000000\\
-216999.999999999	-14893000000\\
-294000	10132200000\\
255999.999999998	4394300000\\
-403000	-14038000000\\
441000.000000001	18920900000\\
254000	-4150400000.00001\\
-345999.999999998	-16357500000\\
-36999.9999999964	14770700000\\
-37000.0000000035	-8178999999.99998\\
312000.000000003	20142000000\\
89999.9999999972	-20996600000\\
-272999.999999998	-5004400000\\
89999.9999999963	13305400000\\
-365000.000000001	-6957900000\\
110000.000000002	9033000000\\
90999.9999999984	-6469400000\\
182999.999999998	7079800000\\
-73999.9999999989	-11962700000\\
-198999.999999998	6713600000\\
125999.999999996	122499999.999995\\
1000.000000003	487900000.000005\\
-19000.0000000037	-2563400000\\
220000.000000001	7934700000\\
36999.9999999972	-11108400000\\
-274999.999999998	1220500000\\
-36000.0000000022	5371300000\\
17000.0000000021	-1831200000\\
-35999.9999999978	-366100000.000002\\
-36000.0000000014	-244100000\\
237999.999999999	3906100000\\
-92999.9999999964	-6836000000\\
-182000.000000004	2441800000\\
458000.000000003	10497600000\\
-476000.000000001	-26855200000\\
53999.9999999985	28442200000\\
495000.000000001	-1586800000\\
-256000.000000001	-22827000000\\
-239000.000000001	11596400000\\
109999.999999999	6103500000\\
257000.000000004	5249400000.00001\\
-220000	-23804100000\\
91999.9999999996	20630100000\\
-220000	-12939700000\\
-239000.000000003	5493600000\\
221000.000000004	2440900000\\
329000.000000002	7935000000\\
90999.9999999975	-3296200000\\
-217999.999999999	-19653100000\\
-1000.00000000033	17333700000\\
-147999.999999999	-11718400000\\
-106999.999999998	9643399999.99999\\
565999.999999999	17700000000\\
-385000	-41747700000\\
-90000.0000000025	30517300000\\
-166000	-18432400000\\
366000	28442200000\\
312000	-21972500000\\
-622000.000000001	-6469900000.00001\\
254999.999999999	17089900000\\
-274000.000000001	-14160000000\\
92000.0000000014	12451000000\\
346999.999999997	366300000.000008\\
-90999.9999999975	-6225600000.00001\\
-19000.0000000028	-3662299999.99999\\
-218000	732799999.999986\\
163000.000000001	10863900000\\
72999.9999999968	-10131600000\\
-125999.999999997	1708900000\\
-2000.00000000244	1830900000\\
19000.0000000001	366600000\\
-90999.9999999993	-3052100000\\
-128999.999999998	1220700000\\
164999.999999999	3418200000\\
255999.999999998	121899999.999993\\
1000.00000000122	-1586799999.99999\\
-276000.000000002	-8300900000\\
57000.0000000013	11230500000\\
-111999.999999998	-7324200000\\
54999.9999999988	4882900000\\
149000	854299999.999999\\
-40000.0000000009	-2441200000\\
186000.000000002	1342500000\\
-661000.000000001	-12572900000\\
420999.999999999	20019400000\\
350000.000000001	-366399999.999996\\
-21000.0000000026	-15258600000\\
-583999.999999998	-2197300000\\
-19000.0000000019	14404400000\\
421000	-610600000.000001\\
-18000.0000000007	-7568200000\\
-19000.0000000001	4516700000\\
-346999.999999999	-9643799999.99999\\
128000.000000002	13183800000\\
475999.999999998	4272400000\\
-72999.9999999968	-18432700000\\
-587000.000000001	3173899999.99999\\
-127000.000000001	7690500000\\
584999.999999998	5248900000\\
221000	-4028300000\\
-385999.999999999	-13549700000\\
-347000	8300600000\\
383999.999999998	9643700000.00001\\
184000.000000003	-3418000000.00001\\
-331000.000000001	-16601500000\\
75000.0000000028	19897200000\\
364999.999999997	3784600000\\
-257000.000000001	-28198600000\\
-326999.999999998	14160300000\\
472999.999999998	20507800000\\
-309000	-35888700000\\
-20000.0000000005	26611300000\\
350000.000000001	-1708700000\\
15999.9999999973	-10376500000\\
-621000.000000001	-10497600000\\
109000.000000001	20141400000\\
440000	5127100000\\
109000.000000001	-7812800000.00001\\
-254999.999999999	-16357000000\\
-239000.000000001	9154900000\\
385000.000000002	20141700000\\
18000.0000000007	-22949000000\\
-293000.000000002	121900000.000005\\
-72999.9999999995	7934400000\\
202000.000000002	3540400000.00001\\
16999.9999999959	-5493499999.99999\\
-72999.9999999995	-2319100000.00001\\
184000	10253700000\\
-220000	-16723300000\\
-93000	9643200000\\
442000	12817500000\\
-258000.000000002	-26000900000\\
-55000.0000000006	16235300000\\
18999.9999999992	-3173800000\\
-36999.999999999	-1465000000\\
-202000	-365899999.999999\\
404000.000000002	9521100000\\
-220000.000000001	-15746700000\\
54999.9999999997	10986100000\\
-1000.00000000122	-2075199999.99999\\
257000.000000001	5127099999.99998\\
-219000.000000002	-15381000000\\
273000	16845800000\\
-658000.000000001	-21728600000\\
19000.000000001	19653400000\\
730999.999999997	8789100000.00001\\
-164999.999999999	-24902600000\\
-620999.999999999	1587200000.00001\\
255999.999999999	17700200000\\
346000	-3784500000\\
-289999.999999999	-13183300000\\
15999.9999999973	10376000000\\
385000.000000002	12817100000\\
-256000.000000001	-32714500000\\
-238000	17577900000\\
385000.000000002	13183600000\\
-203000.000000003	-24414000000\\
-144999.999999999	11352600000\\
72999.9999999995	1953000000.00001\\
72999.9999999968	1831100000\\
73000.0000000004	-4760600000\\
127999.999999999	6713500000\\
-255999.999999998	-17577700000\\
1000.00000000033	18188200000\\
-75000.0000000028	-10986200000\\
148000.000000001	11596600000\\
90999.9999999984	-8422700000\\
-422000.000000001	-7202500000\\
403999.999999999	21118600000\\
145999.999999999	-11596900000\\
-238999.999999998	-9033300000.00001\\
-198999.999999999	6225799999.99999\\
418999.999999996	14892500000\\
-72000.0000000001	-22460900000\\
-367000	7324000000\\
72999.9999999995	5859600000\\
238999.999999999	-488299999.999997\\
18000.0000000007	-5005000000\\
17999.999999998	1709200000\\
-365999.999999999	-3906600000\\
17999.9999999989	7690900000\\
183999.999999999	-4883300000\\
127000.000000001	5127300000\\
-90999.9999999984	-7812700000\\
19000.000000001	7202300000\\
364999.999999998	5737200000\\
-421000.000000002	-25146400000\\
-401999.999999997	13305500000\\
328999.999999998	11108600000\\
477000.000000004	4638600000\\
-313000.000000002	-28198300000\\
-328000.000000001	9399400000\\
475000.000000001	27710100000\\
-92000.0000000005	-35034300000\\
-70999.9999999979	14160200000\\
-130000.000000003	-4760699999.99999\\
-73000.0000000022	4028199999.99999\\
366000.000000003	12451200000\\
-163000.000000001	-24413900000\\
-350000.000000001	6591600000\\
310999.999999999	16113400000\\
131000.000000003	-10254000000\\
252999.999999999	8300900000.00001\\
-291000.000000002	-27343900000\\
-750999.999999998	13183700000\\
860999.999999997	25878900000\\
181000.000000001	-26977500000\\
-473000.000000001	-488399999.999996\\
-239999.999999999	7690500000\\
201000	2075200000\\
1999.99999999711	-5126999999.99999\\
492000.000000002	14038100000\\
-254000	-22460800000\\
-20000.0000000022	12206800000\\
-72999.9999999986	-3173600000\\
-292000	365899999.999996\\
328000.000000001	5249500000\\
-125999.999999999	-6836500000\\
-130000	1709400000\\
384999.999999999	7690400000\\
127999.999999999	-7568700000\\
-457000.000000001	-5859000000\\
-18999.9999999983	9155200000\\
129000	-1098800000\\
255000.000000004	2197400000\\
-421000.000000001	-10620200000\\
330999.999999997	14282400000\\
-36999.9999999981	-10376200000\\
-476999.999999999	-1586800000\\
365999.999999999	10376000000\\
-52999.9999999999	-7568399999.99999\\
144999.999999998	4028300000\\
146000.000000001	5127000000\\
129000	-9399399999.99999\\
-421000	-8545099999.99999\\
-91999.9999999987	14160400000\\
311999.999999998	1831000000\\
-202000	-10132000000\\
-293000.000000001	3662400000\\
403000	7568000000\\
90999.9999999993	-7324000000\\
-218999.999999999	-1586800000\\
72999.9999999986	2929400000.00001\\
-8.88178419700125e-10	1831200000\\
2.66453525910038e-09	-3051700000\\
-55000.0000000006	-1220799999.99999\\
129000.000000001	7080099999.99999\\
-240000.000000002	-13305600000\\
533000.000000002	27465700000\\
-129000.000000002	-35888600000\\
-513000.000000001	15625100000\\
-73000.0000000004	2075000000\\
587000.000000001	7934600000\\
-185000.000000003	-14526100000\\
184000.000000001	8910800000\\
-529999.999999999	-10498000000\\
328000.000000002	15747300000\\
146999.999999998	-12207300000\\
-439999.999999999	1343000000\\
19999.9999999996	3662000000\\
327999.999999999	1708900000\\
-182000.000000002	-6347600000\\
90000.0000000007	4516800000\\
295000.000000003	7568100000\\
16999.9999999959	-14770400000\\
-732000	-4760800000\\
366000	22705100000\\
401999.999999999	-8789000000.00001\\
-546999.999999999	-15014600000\\
-93999.9999999994	14159900000\\
495999.999999999	8789400000\\
-349000.000000003	-25879300000\\
111000.000000003	26611800000\\
492999.999999997	-5615700000\\
-529999.999999997	-20751500000\\
17999.9999999998	18188000000\\
-126999.999999999	-7201899999.99999\\
143999.999999998	7934800000\\
258000	5492700000\\
-54999.999999997	-19897100000\\
};
\addplot [color=mycolor2, line width=2.0pt, forget plot]
  table[row sep=crcr]{%
-75000.0000000019	-75000.0000000019\\
166000.000000001	166000.000000001\\
-72000.0000000001	-72000.0000000001\\
143999.999999998	143999.999999998\\
92999.9999999973	92999.9999999973\\
-457999.999999997	-457999.999999997\\
163999.999999997	163999.999999997\\
38000.0000000029	38000.0000000029\\
128000.000000002	128000.000000002\\
218999.999999996	218999.999999996\\
-329000	-329000\\
-165999.999999998	-165999.999999998\\
-345999.999999998	-345999.999999998\\
511000.000000001	511000.000000001\\
329999.999999996	329999.999999996\\
-291999.999999997	-291999.999999997\\
199999.999999997	199999.999999997\\
-145000	-145000\\
-550000.000000002	-550000.000000002\\
787000.000000001	787000.000000001\\
-202000.000000003	-202000.000000003\\
-436999.999999997	-436999.999999997\\
472999.999999997	472999.999999997\\
-70999.9999999979	-70999.9999999979\\
-292999.999999999	-292999.999999999\\
308999.999999997	308999.999999997\\
112000.000000002	112000.000000002\\
-457999.999999997	-457999.999999997\\
328999.999999997	328999.999999997\\
74000.0000000025	74000.0000000025\\
-148000.000000003	-148000.000000003\\
39000.0000000041	39000.0000000041\\
-186000.000000004	-186000.000000004\\
76000.0000000014	76000.0000000014\\
180999.999999999	180999.999999999\\
-145000.000000003	-145000.000000003\\
35000.0000000028	35000.0000000028\\
-15999.9999999991	-15999.9999999991\\
-130000.000000003	-130000.000000003\\
311000.000000001	311000.000000001\\
2000.00000000156	2000.00000000156\\
-222000.000000002	-222000.000000002\\
203000.000000004	203000.000000004\\
-293000.000000004	-293000.000000004\\
-37999.9999999994	-37999.9999999994\\
312999.999999999	312999.999999999\\
-111000.000000002	-111000.000000002\\
-200999.999999999	-200999.999999999\\
218999.999999997	218999.999999997\\
56000.0000000036	56000.0000000036\\
-293000.000000001	-293000.000000001\\
438999.999999998	438999.999999998\\
-37000.0000000008	-37000.0000000008\\
-293000	-293000\\
-90999.9999999993	-90999.9999999993\\
129000.000000001	129000.000000001\\
327999.999999997	327999.999999997\\
-346999.999999999	-346999.999999999\\
-146999.999999998	-146999.999999998\\
366999.999999999	366999.999999999\\
-92999.9999999964	-92999.9999999964\\
-200000.000000005	-200000.000000005\\
128000.000000003	128000.000000003\\
-129000.000000003	-129000.000000003\\
312000.000000004	312000.000000004\\
-201000.000000002	-201000.000000002\\
108000.000000001	108000.000000001\\
-106999.999999998	-106999.999999998\\
124999.999999996	124999.999999996\\
-70999.9999999971	-70999.9999999971\\
-56000.0000000027	-56000.0000000027\\
56000.0000000009	56000.0000000009\\
-38000.0000000011	-38000.0000000011\\
-236999.999999998	-236999.999999998\\
201000	201000\\
238000.000000001	238000.000000001\\
-72999.9999999986	-72999.9999999986\\
73999.9999999989	73999.9999999989\\
-460000.000000003	-460000.000000003\\
185000	185000\\
219000.000000004	219000.000000004\\
0	0\\
-292000.000000001	-292000.000000001\\
35000.000000001	35000.000000001\\
439999.999999999	439999.999999999\\
-273000	-273000\\
51999.9999999987	51999.9999999987\\
-106999.999999997	-106999.999999997\\
-57000.0000000057	-57000.0000000057\\
-90999.9999999957	-90999.9999999957\\
312000	312000\\
-37000.0000000026	-37000.0000000026\\
-202000.000000001	-202000.000000001\\
-54000.0000000003	-54000.0000000003\\
16999.9999999986	16999.9999999986\\
240000.000000003	240000.000000003\\
-57000.0000000039	-57000.0000000039\\
-199999.999999999	-199999.999999999\\
237000	237000\\
-255000	-255000\\
400999.999999999	400999.999999999\\
-345999.999999996	-345999.999999996\\
255000.000000002	255000.000000002\\
-127000.000000002	-127000.000000002\\
-147999.999999999	-147999.999999999\\
258000	258000\\
-130000.000000003	-130000.000000003\\
-125999.999999997	-125999.999999997\\
144999.999999997	144999.999999997\\
127999.999999999	127999.999999999\\
-238000	-238000\\
148000.000000003	148000.000000003\\
-112000.000000004	-112000.000000004\\
2000.00000000244	2000.00000000244\\
163999.999999998	163999.999999998\\
-184000.000000001	-184000.000000001\\
-35999.9999999987	-35999.9999999987\\
-17000.0000000012	-17000.0000000012\\
273000.000000001	273000.000000001\\
-128000.000000004	-128000.000000004\\
220000.000000003	220000.000000003\\
-495000.000000003	-495000.000000003\\
75000.0000000055	75000.0000000055\\
474999.999999996	474999.999999996\\
-385000.000000001	-385000.000000001\\
128000	128000\\
203000.000000002	203000.000000002\\
-257000.000000001	-257000.000000001\\
-203000.000000002	-203000.000000002\\
149000.000000001	149000.000000001\\
-75000.000000001	-75000.000000001\\
183000	183000\\
130000.000000002	130000.000000002\\
-20000.0000000005	-20000.0000000005\\
-220000.000000001	-220000.000000001\\
56000.0000000045	56000.0000000045\\
36999.9999999981	36999.9999999981\\
-74000.0000000007	-74000.0000000007\\
237999.999999997	237999.999999997\\
-219999.999999999	-219999.999999999\\
-36000.0000000005	-36000.0000000005\\
90999.9999999966	90999.9999999966\\
-108999.999999997	-108999.999999997\\
-20000.0000000013	-20000.0000000013\\
-15999.9999999956	-15999.9999999956\\
419999.999999996	419999.999999996\\
-165000	-165000\\
-184000.000000001	-184000.000000001\\
2000.00000000156	2000.00000000156\\
182000	182000\\
-90999.9999999993	-90999.9999999993\\
-220000.000000002	-220000.000000002\\
-73999.9999999981	-73999.9999999981\\
513999.999999999	513999.999999999\\
-276000.000000001	-276000.000000001\\
-54000.0000000029	-54000.0000000029\\
146000.000000003	146000.000000003\\
18999.9999999966	18999.9999999966\\
-54999.999999997	-54999.999999997\\
-37999.9999999994	-37999.9999999994\\
-218000.000000001	-218000.000000001\\
-93000.0000000008	-93000.0000000008\\
404000.000000002	404000.000000002\\
55000.0000000006	55000.0000000006\\
108999.999999998	108999.999999998\\
-165000.000000002	-165000.000000002\\
-126999.999999999	-126999.999999999\\
91000.0000000002	91000.0000000002\\
-513000	-513000\\
603999.999999998	603999.999999998\\
-200000	-200000\\
34999.9999999984	34999.9999999984\\
-145999.999999998	-145999.999999998\\
257000.000000001	257000.000000001\\
183000.000000002	183000.000000002\\
-257000.000000001	-257000.000000001\\
91999.999999997	91999.999999997\\
-257000.000000001	-257000.000000001\\
92000.0000000023	92000.0000000023\\
56000	56000\\
126000.000000001	126000.000000001\\
-144999.999999999	-144999.999999999\\
-18000.0000000007	-18000.0000000007\\
35999.9999999987	35999.9999999987\\
-56000	-56000\\
167000.000000002	167000.000000002\\
-331000	-331000\\
164999.999999996	164999.999999996\\
366000.000000003	366000.000000003\\
-219000.000000002	-219000.000000002\\
-202999.999999999	-202999.999999999\\
-253999.999999998	-253999.999999998\\
474999.999999997	474999.999999997\\
-129000	-129000\\
111000.000000001	111000.000000001\\
-55999.9999999983	-55999.9999999983\\
93000	93000\\
-129999.999999999	-129999.999999999\\
-162999.999999999	-162999.999999999\\
181999.999999999	181999.999999999\\
220000.000000003	220000.000000003\\
-165000.000000003	-165000.000000003\\
-364999.999999997	-364999.999999997\\
327999.999999996	327999.999999996\\
183000.000000001	183000.000000001\\
-455999.999999997	-455999.999999997\\
199999.999999998	199999.999999998\\
165000	165000\\
-90999.9999999975	-90999.9999999975\\
-183000.000000002	-183000.000000002\\
217999.999999997	217999.999999997\\
-180999.999999997	-180999.999999997\\
-1000.000000003	-1000.000000003\\
256000.000000004	256000.000000004\\
74000.0000000007	74000.0000000007\\
-384000	-384000\\
16999.9999999968	16999.9999999968\\
-72999.9999999995	-72999.9999999995\\
604000.000000002	604000.000000002\\
-181999.999999999	-181999.999999999\\
-256999.999999998	-256999.999999998\\
-109000.000000001	-109000.000000001\\
162999.999999998	162999.999999998\\
92999.9999999973	92999.9999999973\\
-146999.999999998	-146999.999999998\\
-73000.0000000031	-73000.0000000031\\
91000.0000000002	91000.0000000002\\
110000.000000001	110000.000000001\\
73999.9999999998	73999.9999999998\\
-440000.000000002	-440000.000000002\\
347999.999999999	347999.999999999\\
91000.0000000002	91000.0000000002\\
-90999.9999999984	-90999.9999999984\\
-146000.000000002	-146000.000000002\\
-18999.9999999983	-18999.9999999983\\
383999.999999997	383999.999999997\\
-16999.9999999995	-16999.9999999995\\
-623000	-623000\\
254999.999999999	254999.999999999\\
57000.0000000013	57000.0000000013\\
182000	182000\\
-18000.0000000007	-18000.0000000007\\
-348000.000000002	-348000.000000002\\
348000.000000001	348000.000000001\\
17999.999999998	17999.999999998\\
-274999.999999999	-274999.999999999\\
183999.999999999	183999.999999999\\
18000.0000000007	18000.0000000007\\
-147000.000000003	-147000.000000003\\
276000.000000004	276000.000000004\\
-276000.000000003	-276000.000000003\\
92000.0000000023	92000.0000000023\\
-19000.0000000001	-19000.0000000001\\
-255000.000000001	-255000.000000001\\
401999.999999998	401999.999999998\\
-200999.999999998	-200999.999999998\\
54999.9999999988	54999.9999999988\\
328999.999999997	328999.999999997\\
-127999.999999998	-127999.999999998\\
-312000.000000002	-312000.000000002\\
20000.0000000022	20000.0000000022\\
-56000.0000000027	-56000.0000000027\\
404000.000000002	404000.000000002\\
-332000.000000002	-332000.000000002\\
-34000.0000000007	-34000.0000000007\\
-57000.0000000004	-57000.0000000004\\
422000	422000\\
-328999.999999995	-328999.999999995\\
292999.999999997	292999.999999997\\
-167000.000000001	-167000.000000001\\
94000.0000000012	94000.0000000012\\
-237999.999999998	-237999.999999998\\
143999.999999998	143999.999999998\\
-144000.000000002	-144000.000000002\\
-1000.00000000033	-1000.00000000033\\
-54000.0000000011	-54000.0000000011\\
309000	309000\\
-161999.999999998	-161999.999999998\\
52999.9999999955	52999.9999999955\\
-200999.999999997	-200999.999999997\\
330000	330000\\
-8.88178419700125e-10	-8.88178419700125e-10\\
-310999.999999998	-310999.999999998\\
144999.999999996	144999.999999996\\
-89999.9999999998	-89999.9999999998\\
237000	237000\\
19000.000000001	19000.000000001\\
-17999.9999999998	-17999.9999999998\\
-239000.000000001	-239000.000000001\\
-72999.9999999986	-72999.9999999986\\
73999.9999999998	73999.9999999998\\
90000.0000000007	90000.0000000007\\
38999.9999999997	38999.9999999997\\
-130999.999999999	-130999.999999999\\
-52000.0000000023	-52000.0000000023\\
326999.999999999	326999.999999999\\
-145000	-145000\\
-128000	-128000\\
345999.999999999	345999.999999999\\
-126000.000000002	-126000.000000002\\
-549999.999999997	-549999.999999997\\
292999.999999997	292999.999999997\\
329000.000000001	329000.000000001\\
-219000.000000001	-219000.000000001\\
109000	109000\\
19000.0000000001	19000.0000000001\\
-329000.000000001	-329000.000000001\\
108000	108000\\
240000.000000001	240000.000000001\\
71999.9999999974	71999.9999999974\\
-512000	-512000\\
494000.000000002	494000.000000002\\
-128000.000000001	-128000.000000001\\
-165000.000000003	-165000.000000003\\
165000.000000003	165000.000000003\\
17999.9999999998	17999.9999999998\\
-54000.000000002	-54000.000000002\\
-312999.999999999	-312999.999999999\\
258000.000000003	258000.000000003\\
310999.999999999	310999.999999999\\
-56000	-56000\\
-365000.000000001	-365000.000000001\\
36000.0000000014	36000.0000000014\\
36999.999999999	36999.999999999\\
53999.9999999985	53999.9999999985\\
203000.000000003	203000.000000003\\
-258000.000000004	-258000.000000004\\
-53999.9999999976	-53999.9999999976\\
183000.000000002	183000.000000002\\
128999.999999998	128999.999999998\\
-167000.000000001	-167000.000000001\\
-254000.000000002	-254000.000000002\\
89000.0000000013	89000.0000000013\\
333000.000000001	333000.000000001\\
-113000.000000001	-113000.000000001\\
111000.000000002	111000.000000002\\
-237000.000000003	-237000.000000003\\
-111999.999999998	-111999.999999998\\
405000.000000001	405000.000000001\\
-221000.000000004	-221000.000000004\\
-53999.9999999976	-53999.9999999976\\
-37999.9999999976	-37999.9999999976\\
-291999.999999999	-291999.999999999\\
328999.999999998	328999.999999998\\
201999.999999999	201999.999999999\\
18000.0000000016	18000.0000000016\\
-274000.000000002	-274000.000000002\\
35999.9999999978	35999.9999999978\\
274000.000000001	274000.000000001\\
-290999.999999999	-290999.999999999\\
15999.9999999991	15999.9999999991\\
129999.999999999	129999.999999999\\
-202999.999999996	-202999.999999996\\
-36000.0000000031	-36000.0000000031\\
184000.000000002	184000.000000002\\
273999.999999999	273999.999999999\\
-219999.999999999	-219999.999999999\\
-201000.000000004	-201000.000000004\\
73000.0000000004	73000.0000000004\\
165000	165000\\
-458000.000000003	-458000.000000003\\
202000.000000001	202000.000000001\\
548000	548000\\
-530000.000000002	-530000.000000002\\
92000.0000000014	92000.0000000014\\
108000	108000\\
130000.000000003	130000.000000003\\
-183000.000000001	-183000.000000001\\
-129000	-129000\\
72999.9999999977	72999.9999999977\\
0	0\\
1000.00000000033	1000.00000000033\\
71999.9999999983	71999.9999999983\\
-126999.999999999	-126999.999999999\\
91000.0000000002	91000.0000000002\\
-19000.0000000001	-19000.0000000001\\
258000.000000003	258000.000000003\\
-276000.000000003	-276000.000000003\\
-109999.999999998	-109999.999999998\\
349000	349000\\
-348999.999999998	-348999.999999998\\
221000.000000001	221000.000000001\\
163999.999999999	163999.999999999\\
-220000.000000002	-220000.000000002\\
-220000.000000001	-220000.000000001\\
-16999.9999999968	-16999.9999999968\\
383999.999999997	383999.999999997\\
-38000.0000000002	-38000.0000000002\\
-308999.999999998	-308999.999999998\\
236999.999999998	236999.999999998\\
90999.9999999993	90999.9999999993\\
-346999.999999999	-346999.999999999\\
182000.000000002	182000.000000002\\
147999.999999998	147999.999999998\\
-404999.999999997	-404999.999999997\\
276999.999999999	276999.999999999\\
236999.999999999	236999.999999999\\
-494000	-494000\\
163999.999999998	163999.999999998\\
330000.000000002	330000.000000002\\
-275000.000000001	-275000.000000001\\
221000.000000004	221000.000000004\\
-258000.000000003	-258000.000000003\\
166000.000000002	166000.000000002\\
90999.9999999975	90999.9999999975\\
-476000.000000002	-476000.000000002\\
366000	366000\\
-17999.999999998	-17999.999999998\\
-219000	-219000\\
382999.999999997	382999.999999997\\
-164000	-164000\\
-255999.999999998	-255999.999999998\\
-165999.999999999	-165999.999999999\\
476999.999999998	476999.999999998\\
-17999.9999999998	-17999.9999999998\\
8.88178419700125e-10	8.88178419700125e-10\\
-38000.0000000038	-38000.0000000038\\
-145000	-145000\\
219000.000000001	219000.000000001\\
-92000.0000000032	-92000.0000000032\\
-127000	-127000\\
164000	164000\\
-255999.999999998	-255999.999999998\\
348000.000000002	348000.000000002\\
-73000.0000000022	-73000.0000000022\\
-367000.000000002	-367000.000000002\\
238000	238000\\
239000.000000001	239000.000000001\\
-93000.0000000017	-93000.0000000017\\
-181999.999999999	-181999.999999999\\
19000.0000000019	19000.0000000019\\
125999.999999997	125999.999999997\\
-70999.9999999997	-70999.9999999997\\
-167000	-167000\\
131000.000000001	131000.000000001\\
69999.9999999985	69999.9999999985\\
94000.0000000012	94000.0000000012\\
-238999.999999999	-238999.999999999\\
146999.999999998	146999.999999998\\
181999.999999999	181999.999999999\\
-364999.999999998	-364999.999999998\\
-165000.000000004	-165000.000000004\\
329000.000000001	329000.000000001\\
165000.000000002	165000.000000002\\
-73000.0000000013	-73000.0000000013\\
-146999.999999998	-146999.999999998\\
110999.999999997	110999.999999997\\
-93000	-93000\\
18999.9999999983	18999.9999999983\\
-127999.999999997	-127999.999999997\\
164999.999999998	164999.999999998\\
163999.999999998	163999.999999998\\
-439000	-439000\\
183000.000000002	183000.000000002\\
109999.999999999	109999.999999999\\
-90999.9999999975	-90999.9999999975\\
126999.999999995	126999.999999995\\
-255999.999999999	-255999.999999999\\
147000.000000001	147000.000000001\\
54999.9999999988	54999.9999999988\\
237000.000000002	237000.000000002\\
-54000.000000002	-54000.000000002\\
-330000	-330000\\
127999.999999998	127999.999999998\\
-292000.000000001	-292000.000000001\\
292000.000000002	292000.000000002\\
-201000	-201000\\
292000	292000\\
20000.0000000013	20000.0000000013\\
-274999.999999999	-274999.999999999\\
144999.999999996	144999.999999996\\
2000.00000000244	2000.00000000244\\
16000	16000\\
-161999.999999999	-161999.999999999\\
308999.999999997	308999.999999997\\
19000.0000000028	19000.0000000028\\
-110000	-110000\\
-200000.000000001	-200000.000000001\\
52999.999999999	52999.999999999\\
0	0\\
240000	240000\\
-222000	-222000\\
39000.0000000015	39000.0000000015\\
198999.999999997	198999.999999997\\
-456000.000000002	-456000.000000002\\
146000.000000002	146000.000000002\\
382999.999999998	382999.999999998\\
-271999.999999996	-271999.999999996\\
-313000.000000004	-313000.000000004\\
641000.000000003	641000.000000003\\
-255000	-255000\\
-277000.000000001	-277000.000000001\\
332000.000000002	332000.000000002\\
88999.9999999977	88999.9999999977\\
-106999.999999997	-106999.999999997\\
-277000.000000005	-277000.000000005\\
-89999.9999999972	-89999.9999999972\\
181999.999999997	181999.999999997\\
256000.000000001	256000.000000001\\
-52999.9999999955	-52999.9999999955\\
16999.9999999986	16999.9999999986\\
-37000.0000000008	-37000.0000000008\\
-144999.999999999	-144999.999999999\\
52999.9999999973	52999.9999999973\\
-34999.9999999993	-34999.9999999993\\
-19000.0000000028	-19000.0000000028\\
-54999.9999999997	-54999.9999999997\\
91999.9999999978	91999.9999999978\\
256000.000000001	256000.000000001\\
-477000.000000001	-477000.000000001\\
167000.000000001	167000.000000001\\
125999.999999999	125999.999999999\\
-107999.999999997	-107999.999999997\\
16999.9999999986	16999.9999999986\\
36999.9999999999	36999.9999999999\\
256000	256000\\
-219000.000000001	-219000.000000001\\
-146999.999999998	-146999.999999998\\
164999.999999999	164999.999999999\\
72999.9999999968	72999.9999999968\\
-329000.000000001	-329000.000000001\\
35000.0000000028	35000.0000000028\\
204000	204000\\
290999.999999999	290999.999999999\\
-201000	-201000\\
-201000	-201000\\
202000.000000002	202000.000000002\\
-404999.999999999	-404999.999999999\\
38999.9999999979	38999.9999999979\\
273000.000000001	273000.000000001\\
-73000.0000000022	-73000.0000000022\\
183999.999999998	183999.999999998\\
-37000.0000000008	-37000.0000000008\\
-200999.999999998	-200999.999999998\\
90000.0000000007	90000.0000000007\\
18999.9999999975	18999.9999999975\\
-72000.0000000001	-72000.0000000001\\
-111000.000000002	-111000.000000002\\
110000.000000002	110000.000000002\\
164999.999999997	164999.999999997\\
8.88178419700125e-10	8.88178419700125e-10\\
-184000.000000002	-184000.000000002\\
38000.0000000029	38000.0000000029\\
90999.9999999957	90999.9999999957\\
-110999.999999999	-110999.999999999\\
130000.000000003	130000.000000003\\
127999.999999999	127999.999999999\\
-148000.000000001	-148000.000000001\\
-162999.999999998	-162999.999999998\\
-203000.000000003	-203000.000000003\\
404000.000000004	404000.000000004\\
145999.999999996	145999.999999996\\
-127999.999999999	-127999.999999999\\
16999.9999999995	16999.9999999995\\
-199000	-199000\\
-130000.000000003	-130000.000000003\\
331000.000000003	331000.000000003\\
-458999.999999998	-458999.999999998\\
238000	238000\\
274999.999999996	274999.999999996\\
-126999.999999998	-126999.999999998\\
108000.000000002	108000.000000002\\
-218999.999999998	-218999.999999998\\
-110000	-110000\\
146999.999999999	146999.999999999\\
55000.0000000006	55000.0000000006\\
53999.9999999985	53999.9999999985\\
-72000.0000000001	-72000.0000000001\\
-275999.999999998	-275999.999999998\\
-54000.0000000038	-54000.0000000038\\
512000.000000001	512000.000000001\\
-35999.9999999996	-35999.9999999996\\
19000.000000001	19000.000000001\\
-350000.000000001	-350000.000000001\\
-17000.0000000003	-17000.0000000003\\
404000.000000003	404000.000000003\\
-167000.000000002	-167000.000000002\\
37000.0000000008	37000.0000000008\\
-126999.999999999	-126999.999999999\\
-37000.0000000026	-37000.0000000026\\
255000.000000003	255000.000000003\\
-254000.000000001	-254000.000000001\\
199999.999999999	199999.999999999\\
-202000.000000001	-202000.000000001\\
-273000	-273000\\
620999.999999998	620999.999999998\\
-292000.000000002	-292000.000000002\\
-110000	-110000\\
219000	219000\\
-91000.0000000019	-91000.0000000019\\
-35999.9999999987	-35999.9999999987\\
-1000.00000000211	-1000.00000000211\\
37000.0000000026	37000.0000000026\\
73999.9999999989	73999.9999999989\\
15999.9999999982	15999.9999999982\\
-216999.999999999	-216999.999999999\\
345999.999999998	345999.999999998\\
2000.00000000244	2000.00000000244\\
-587999.999999998	-587999.999999998\\
127999.999999997	127999.999999997\\
588000	588000\\
-221000	-221000\\
-457000	-457000\\
108000.000000001	108000.000000001\\
331999.999999998	331999.999999998\\
-129999.999999999	-129999.999999999\\
183999.999999997	183999.999999997\\
-385000.000000001	-385000.000000001\\
385000.000000003	385000.000000003\\
54999.9999999979	54999.9999999979\\
-513000.000000002	-513000.000000002\\
603999.999999999	603999.999999999\\
-310999.999999999	-310999.999999999\\
-36000.0000000022	-36000.0000000022\\
90000.0000000061	90000.0000000061\\
-328000.000000001	-328000.000000001\\
383000.000000002	383000.000000002\\
-291000.000000003	-291000.000000003\\
52999.9999999999	52999.9999999999\\
330999.999999998	330999.999999998\\
-36999.9999999981	-36999.9999999981\\
-239000	-239000\\
128999.999999999	128999.999999999\\
-201000.000000001	-201000.000000001\\
292000.000000002	292000.000000002\\
-146000.000000003	-146000.000000003\\
-53999.9999999967	-53999.9999999967\\
126000.000000001	126000.000000001\\
-199000.000000002	-199000.000000002\\
163000.000000003	163000.000000003\\
-218999.999999999	-218999.999999999\\
311000.000000001	311000.000000001\\
17999.9999999998	17999.9999999998\\
-182000.000000003	-182000.000000003\\
53999.9999999994	53999.9999999994\\
-108999.999999998	-108999.999999998\\
-202000.000000002	-202000.000000002\\
310999.999999999	310999.999999999\\
366000.000000001	366000.000000001\\
-420000	-420000\\
128000.000000001	128000.000000001\\
-386000.000000003	-386000.000000003\\
368000.000000002	368000.000000002\\
181999.999999997	181999.999999997\\
-456999.999999999	-456999.999999999\\
291999.999999997	291999.999999997\\
-236999.999999997	-236999.999999997\\
346999.999999998	346999.999999998\\
-219000	-219000\\
-183999.999999999	-183999.999999999\\
202999.999999999	202999.999999999\\
163000	163000\\
-200000	-200000\\
-55999.9999999974	-55999.9999999974\\
-36000.0000000014	-36000.0000000014\\
273999.999999998	273999.999999998\\
-273999.999999999	-273999.999999999\\
37000.0000000008	37000.0000000008\\
382999.999999998	382999.999999998\\
-125999.999999998	-125999.999999998\\
-185000.000000003	-185000.000000003\\
-274000	-274000\\
256999.999999997	256999.999999997\\
237000.000000003	237000.000000003\\
-293000.000000004	-293000.000000004\\
257000.000000001	257000.000000001\\
54999.9999999997	54999.9999999997\\
-458000.000000001	-458000.000000001\\
-146999.999999999	-146999.999999999\\
586999.999999998	586999.999999998\\
-55999.9999999991	-55999.9999999991\\
-402000.000000002	-402000.000000002\\
402000	402000\\
-71999.9999999965	-71999.9999999965\\
53999.9999999958	53999.9999999958\\
-293000	-293000\\
-8.88178419700125e-10	-8.88178419700125e-10\\
367000.000000001	367000.000000001\\
-221000.000000002	-221000.000000002\\
19000.000000001	19000.000000001\\
110000.000000001	110000.000000001\\
-238000	-238000\\
219999.999999999	219999.999999999\\
-184000.000000001	-184000.000000001\\
-91000.0000000011	-91000.0000000011\\
237999.999999999	237999.999999999\\
36000.0000000005	36000.0000000005\\
-53999.9999999994	-53999.9999999994\\
91000.0000000002	91000.0000000002\\
-91000.0000000011	-91000.0000000011\\
-184000.000000001	-184000.000000001\\
56000.0000000009	56000.0000000009\\
53999.9999999976	53999.9999999976\\
19000.0000000019	19000.0000000019\\
36000.0000000005	36000.0000000005\\
165999.999999999	165999.999999999\\
-57000.0000000022	-57000.0000000022\\
-474999.999999998	-474999.999999998\\
221000.000000002	221000.000000002\\
491999.999999996	491999.999999996\\
-381999.999999997	-381999.999999997\\
-131000.000000001	-131000.000000001\\
277000.000000002	277000.000000002\\
-56000.0000000018	-56000.0000000018\\
-218999.999999997	-218999.999999997\\
200999.999999997	200999.999999997\\
-111000	-111000\\
-33999.9999999989	-33999.9999999989\\
271999.999999998	271999.999999998\\
-255000.000000002	-255000.000000002\\
55000.0000000006	55000.0000000006\\
53999.9999999976	53999.9999999976\\
1000.00000000033	1000.00000000033\\
-147000	-147000\\
255999.999999999	255999.999999999\\
-200000	-200000\\
-38000.0000000011	-38000.0000000011\\
164999.999999999	164999.999999999\\
184000	184000\\
-256999.999999999	-256999.999999999\\
-36999.999999999	-36999.999999999\\
-275000.000000003	-275000.000000003\\
-34999.9999999993	-34999.9999999993\\
474999.999999998	474999.999999998\\
110000.000000002	110000.000000002\\
-53999.9999999976	-53999.9999999976\\
-313000.000000004	-313000.000000004\\
146999.999999999	146999.999999999\\
19000.0000000019	19000.0000000019\\
72999.9999999968	72999.9999999968\\
-274999.999999998	-274999.999999998\\
145999.999999999	145999.999999999\\
56999.9999999986	56999.9999999986\\
-111999.999999998	-111999.999999998\\
146999.999999998	146999.999999998\\
-53999.9999999985	-53999.9999999985\\
89999.9999999998	89999.9999999998\\
-291000.000000001	-291000.000000001\\
-74999.9999999957	-74999.9999999957\\
603999.999999997	603999.999999997\\
-436999.999999998	-436999.999999998\\
125999.999999998	125999.999999998\\
-146000.000000001	-146000.000000001\\
165000.000000001	165000.000000001\\
-110000.000000001	-110000.000000001\\
202000.000000001	202000.000000001\\
-166000.000000001	-166000.000000001\\
-34999.9999999984	-34999.9999999984\\
327999.999999997	327999.999999997\\
-603999.999999999	-603999.999999999\\
459000	459000\\
-166000	-166000\\
-8.88178419700125e-10	-8.88178419700125e-10\\
-17999.9999999989	-17999.9999999989\\
202000.000000003	202000.000000003\\
-275000.000000003	-275000.000000003\\
73000.0000000013	73000.0000000013\\
183000	183000\\
-35999.9999999996	-35999.9999999996\\
-111000.000000004	-111000.000000004\\
-72999.9999999968	-72999.9999999968\\
166000.000000002	166000.000000002\\
-55000.0000000024	-55000.0000000024\\
-38000.0000000011	-38000.0000000011\\
-54999.9999999997	-54999.9999999997\\
112000.000000002	112000.000000002\\
-277000.000000003	-277000.000000003\\
221000	221000\\
273999.999999999	273999.999999999\\
-255999.999999998	-255999.999999998\\
-17999.999999998	-17999.999999998\\
256000	256000\\
-329000	-329000\\
16999.9999999968	16999.9999999968\\
92999.9999999991	92999.9999999991\\
-93000.0000000008	-93000.0000000008\\
-108999.999999999	-108999.999999999\\
202000	202000\\
34999.9999999993	34999.9999999993\\
57000.0000000004	57000.0000000004\\
33999.999999998	33999.999999998\\
-472999.999999997	-472999.999999997\\
307999.999999998	307999.999999998\\
131000.000000004	131000.000000004\\
-999.999999999446	-999.999999999446\\
-349000.000000005	-349000.000000005\\
496000.000000007	496000.000000007\\
-19000.0000000028	-19000.0000000028\\
-311000.000000002	-311000.000000002\\
-19000.000000001	-19000.000000001\\
-182000	-182000\\
236999.999999997	236999.999999997\\
55000.0000000024	55000.0000000024\\
201999.999999996	201999.999999996\\
-146999.999999999	-146999.999999999\\
36999.9999999981	36999.9999999981\\
-201999.999999999	-201999.999999999\\
-89999.999999999	-89999.999999999\\
180999.999999996	180999.999999996\\
166000	166000\\
-274000.000000001	-274000.000000001\\
108000.000000001	108000.000000001\\
202999.999999998	202999.999999998\\
-586999.999999996	-586999.999999996\\
439999.999999996	439999.999999996\\
-90999.9999999984	-90999.9999999984\\
162999.999999998	162999.999999998\\
-107999.999999998	-107999.999999998\\
-19000.0000000001	-19000.0000000001\\
-90999.9999999993	-90999.9999999993\\
254999.999999995	254999.999999995\\
-309999.999999997	-309999.999999997\\
71999.9999999974	71999.9999999974\\
294000.000000003	294000.000000003\\
-273999.999999999	-273999.999999999\\
70999.9999999979	70999.9999999979\\
-89000.0000000013	-89000.0000000013\\
-38999.9999999988	-38999.9999999988\\
294999.999999998	294999.999999998\\
-130000	-130000\\
-144999.999999996	-144999.999999996\\
127999.999999999	127999.999999999\\
53999.9999999958	53999.9999999958\\
-309999.999999996	-309999.999999996\\
126999.999999994	126999.999999994\\
238000.000000003	238000.000000003\\
-145999.999999999	-145999.999999999\\
-236999.999999999	-236999.999999999\\
638999.999999999	638999.999999999\\
-402000	-402000\\
0	0\\
-164000.000000001	-164000.000000001\\
-1999.99999999889	-1999.99999999889\\
478000.000000001	478000.000000001\\
-312000.000000002	-312000.000000002\\
-238000.000000002	-238000.000000002\\
494000	494000\\
-128000.000000001	-128000.000000001\\
-346999.999999997	-346999.999999997\\
217999.999999995	217999.999999995\\
94000.000000003	94000.000000003\\
-167000	-167000\\
54999.9999999997	54999.9999999997\\
38000.0000000002	38000.0000000002\\
-147000.000000003	-147000.000000003\\
91000.0000000002	91000.0000000002\\
19000.0000000019	19000.0000000019\\
-1000.000000003	-1000.000000003\\
130000.000000002	130000.000000002\\
-148000	-148000\\
-19000.0000000019	-19000.0000000019\\
185000.000000005	185000.000000005\\
0	0\\
-222000.000000004	-222000.000000004\\
75000.0000000037	75000.0000000037\\
1.77635683940025e-09	1.77635683940025e-09\\
-128000	-128000\\
328999.999999998	328999.999999998\\
-202000.000000001	-202000.000000001\\
999.999999997669	999.999999997669\\
72999.9999999995	72999.9999999995\\
-274999.999999999	-274999.999999999\\
403000.000000001	403000.000000001\\
-181999.999999999	-181999.999999999\\
-2000.00000000244	-2000.00000000244\\
184000.000000002	184000.000000002\\
-146000.000000002	-146000.000000002\\
-109999.999999999	-109999.999999999\\
-238000.000000001	-238000.000000001\\
437999.999999999	437999.999999999\\
-180999.999999998	-180999.999999998\\
163999.999999998	163999.999999998\\
-257000.000000001	-257000.000000001\\
19000.000000001	19000.000000001\\
127999.999999998	127999.999999998\\
274000.000000004	274000.000000004\\
-291000.000000001	-291000.000000001\\
15999.9999999982	15999.9999999982\\
148000.000000001	148000.000000001\\
-294000.000000001	-294000.000000001\\
111000.000000001	111000.000000001\\
90999.9999999984	90999.9999999984\\
-129000.000000001	-129000.000000001\\
1000.00000000033	1000.00000000033\\
1.77635683940025e-09	1.77635683940025e-09\\
219000	219000\\
-89999.9999999998	-89999.9999999998\\
-112000.000000002	-112000.000000002\\
221000.000000001	221000.000000001\\
-255999.999999998	-255999.999999998\\
126999.999999995	126999.999999995\\
-71999.9999999983	-71999.9999999983\\
-111000.000000002	-111000.000000002\\
183999.999999997	183999.999999997\\
310000.000000001	310000.000000001\\
-492999.999999999	-492999.999999999\\
54999.9999999988	54999.9999999988\\
-1000.00000000122	-1000.00000000122\\
73999.9999999981	73999.9999999981\\
274000	274000\\
-457000	-457000\\
-38000.0000000011	-38000.0000000011\\
130000.000000002	130000.000000002\\
365000.000000004	365000.000000004\\
-127000.000000002	-127000.000000002\\
-56999.9999999977	-56999.9999999977\\
-181000	-181000\\
16999.9999999995	16999.9999999995\\
293000.000000002	293000.000000002\\
-273000	-273000\\
52999.9999999982	52999.9999999982\\
165000.000000001	165000.000000001\\
-181999.999999999	-181999.999999999\\
-129000.000000001	-129000.000000001\\
330000.000000003	330000.000000003\\
-74000.0000000016	-74000.0000000016\\
-345999.999999997	-345999.999999997\\
362999.999999996	362999.999999996\\
-142999.999999996	-142999.999999996\\
-3000.00000000189	-3000.00000000189\\
21000.0000000017	21000.0000000017\\
-75000.0000000037	-75000.0000000037\\
256000.000000001	256000.000000001\\
-53000.0000000017	-53000.0000000017\\
-276999.999999998	-276999.999999998\\
112000.000000002	112000.000000002\\
53999.9999999976	53999.9999999976\\
91000.0000000019	91000.0000000019\\
999.999999997669	999.999999997669\\
-347999.999999999	-347999.999999999\\
364999.999999998	364999.999999998\\
56000.0000000009	56000.0000000009\\
-439000.000000001	-439000.000000001\\
382999.999999999	382999.999999999\\
56000.0000000009	56000.0000000009\\
-110000	-110000\\
-73000.0000000004	-73000.0000000004\\
72999.9999999986	72999.9999999986\\
-458000	-458000\\
512999.999999999	512999.999999999\\
91000.0000000011	91000.0000000011\\
-218999.999999999	-218999.999999999\\
-55999.9999999991	-55999.9999999991\\
-163000	-163000\\
327999.999999997	327999.999999997\\
-127999.999999997	-127999.999999997\\
56000.0000000027	56000.0000000027\\
72999.9999999959	72999.9999999959\\
-74999.9999999984	-74999.9999999984\\
149000.000000001	149000.000000001\\
70999.9999999971	70999.9999999971\\
-547000	-547000\\
217000.000000001	217000.000000001\\
75999.9999999978	75999.9999999978\\
-167000.000000001	-167000.000000001\\
294000	294000\\
-238000.000000001	-238000.000000001\\
458000.000000001	458000.000000001\\
-331000	-331000\\
-291000	-291000\\
583999.999999999	583999.999999999\\
-218000	-218000\\
-56000.0000000018	-56000.0000000018\\
-218999.999999999	-218999.999999999\\
71999.9999999983	71999.9999999983\\
57000.0000000013	57000.0000000013\\
382000	382000\\
-273000	-273000\\
-165000	-165000\\
91999.9999999987	91999.9999999987\\
-56000.0000000009	-56000.0000000009\\
-54999.9999999997	-54999.9999999997\\
367000	367000\\
-128000.000000001	-128000.000000001\\
35000.0000000001	35000.0000000001\\
-52999.9999999973	-52999.9999999973\\
-293000	-293000\\
510999.999999997	510999.999999997\\
-603000.000000001	-603000.000000001\\
293000	293000\\
35999.9999999987	35999.9999999987\\
-311000.000000001	-311000.000000001\\
384000.000000002	384000.000000002\\
-16999.9999999977	-16999.9999999977\\
-165999.999999999	-165999.999999999\\
294000.000000001	294000.000000001\\
108999.999999999	108999.999999999\\
-457000.000000002	-457000.000000002\\
72000.0000000001	72000.0000000001\\
-36000.0000000014	-36000.0000000014\\
1000.00000000033	1000.00000000033\\
71999.9999999983	71999.9999999983\\
292999.999999999	292999.999999999\\
-145999.999999998	-145999.999999998\\
-293000	-293000\\
54999.9999999997	54999.9999999997\\
365999.999999996	365999.999999996\\
-183999.999999997	-183999.999999997\\
-70999.9999999988	-70999.9999999988\\
70999.9999999971	70999.9999999971\\
-163000.000000001	-163000.000000001\\
182000.000000001	182000.000000001\\
-347999.999999998	-347999.999999998\\
182999.999999998	182999.999999998\\
221000.000000001	221000.000000001\\
-1000.000000003	-1000.000000003\\
-128000	-128000\\
220000.000000001	220000.000000001\\
-165999.999999999	-165999.999999999\\
-328000	-328000\\
-19000.000000001	-19000.000000001\\
275000	275000\\
126999.999999998	126999.999999998\\
-90000.0000000016	-90000.0000000016\\
-19000.0000000001	-19000.0000000001\\
-55000.0000000006	-55000.0000000006\\
275000.000000001	275000.000000001\\
-73999.9999999989	-73999.9999999989\\
-53999.9999999985	-53999.9999999985\\
238000	238000\\
-404000.000000003	-404000.000000003\\
-16999.9999999986	-16999.9999999986\\
72999.9999999995	72999.9999999995\\
164000.000000001	164000.000000001\\
20000.0000000013	20000.0000000013\\
-76000.000000004	-76000.000000004\\
-70999.9999999971	-70999.9999999971\\
-238000.000000001	-238000.000000001\\
382999.999999999	382999.999999999\\
148000	148000\\
-274999.999999999	-274999.999999999\\
-257999.999999999	-257999.999999999\\
203999.999999998	203999.999999998\\
253999.999999999	253999.999999999\\
-290999.999999998	-290999.999999998\\
199999.999999997	199999.999999997\\
-181999.999999999	-181999.999999999\\
-91999.9999999996	-91999.9999999996\\
438999.999999996	438999.999999996\\
-347999.999999999	-347999.999999999\\
-34999.9999999975	-34999.9999999975\\
216999.999999996	216999.999999996\\
76000.0000000049	76000.0000000049\\
34999.9999999993	34999.9999999993\\
-237000.000000002	-237000.000000002\\
71999.9999999983	71999.9999999983\\
-328999.999999999	-328999.999999999\\
402999.999999999	402999.999999999\\
-311999.999999999	-311999.999999999\\
330999.999999998	330999.999999998\\
-999.999999997669	-999.999999997669\\
-312000.000000002	-312000.000000002\\
405000.000000002	405000.000000002\\
-405000.000000003	-405000.000000003\\
294000.000000004	294000.000000004\\
18999.9999999983	18999.9999999983\\
-313000.000000002	-313000.000000002\\
239000.000000001	239000.000000001\\
-91000.0000000011	-91000.0000000011\\
16999.9999999986	16999.9999999986\\
111000.000000002	111000.000000002\\
-202000	-202000\\
18999.9999999992	18999.9999999992\\
145000.000000001	145000.000000001\\
-17000.000000003	-17000.000000003\\
-220000	-220000\\
164999.999999999	164999.999999999\\
237000.000000002	237000.000000002\\
-182000.000000002	-182000.000000002\\
-311999.999999999	-311999.999999999\\
330000	330000\\
165000.000000001	165000.000000001\\
-274999.999999999	-274999.999999999\\
347999.999999999	347999.999999999\\
-293000.000000001	-293000.000000001\\
-439000.000000002	-439000.000000002\\
328000.000000002	328000.000000002\\
313000.000000001	313000.000000001\\
-366000.000000001	-366000.000000001\\
346000	346000\\
2000.00000000067	2000.00000000067\\
-185000.000000003	-185000.000000003\\
-107999.999999999	-107999.999999999\\
52999.999999999	52999.999999999\\
149000.000000002	149000.000000002\\
51999.9999999978	51999.9999999978\\
-327000	-327000\\
182000	182000\\
91000.0000000037	91000.0000000037\\
-90999.9999999993	-90999.9999999993\\
-275000.000000001	-275000.000000001\\
238999.999999998	238999.999999998\\
127999.999999999	127999.999999999\\
127000	127000\\
-530000	-530000\\
237000	237000\\
441000.000000004	441000.000000004\\
-550000.000000002	-550000.000000002\\
145999.999999998	145999.999999998\\
257000.000000001	257000.000000001\\
-110000.000000002	-110000.000000002\\
-294000	-294000\\
57000.000000003	57000.000000003\\
382999.999999996	382999.999999996\\
-365999.999999996	-365999.999999996\\
128000	128000\\
92000.0000000005	92000.0000000005\\
-202000.000000001	-202000.000000001\\
-107999.999999999	-107999.999999999\\
162000.000000002	162000.000000002\\
184999.999999996	184999.999999996\\
-220000	-220000\\
73000.0000000004	73000.0000000004\\
164000	164000\\
-52999.999999999	-52999.999999999\\
-184999.999999997	-184999.999999997\\
-53999.9999999994	-53999.9999999994\\
55000.0000000006	55000.0000000006\\
108999.999999999	108999.999999999\\
-364999.999999999	-364999.999999999\\
547999.999999998	547999.999999998\\
-16999.9999999986	-16999.9999999986\\
-257000.000000003	-257000.000000003\\
-35999.9999999996	-35999.9999999996\\
-1000.00000000033	-1000.00000000033\\
146000	146000\\
39000.0000000032	39000.0000000032\\
52999.9999999955	52999.9999999955\\
-237999.999999999	-237999.999999999\\
36999.9999999981	36999.9999999981\\
74000.0000000007	74000.0000000007\\
-37999.9999999976	-37999.9999999976\\
-126999.999999999	-126999.999999999\\
329000.000000001	329000.000000001\\
-127999.999999999	-127999.999999999\\
-36999.9999999981	-36999.9999999981\\
-36000.0000000005	-36000.0000000005\\
8.88178419700125e-10	8.88178419700125e-10\\
-74000.0000000007	-74000.0000000007\\
274999.999999999	274999.999999999\\
-256000	-256000\\
-54999.9999999997	-54999.9999999997\\
108999.999999998	108999.999999998\\
185000.000000002	185000.000000002\\
-259000.000000001	-259000.000000001\\
148999.999999998	148999.999999998\\
-20000.0000000005	-20000.0000000005\\
74000.0000000016	74000.0000000016\\
-329000	-329000\\
16999.9999999959	16999.9999999959\\
458000.000000002	458000.000000002\\
-54000.000000002	-54000.000000002\\
-404000.000000001	-404000.000000001\\
385000.000000001	385000.000000001\\
-237000.000000002	-237000.000000002\\
-165999.999999998	-165999.999999998\\
402000	402000\\
-217000	-217000\\
-75999.9999999978	-75999.9999999978\\
275999.999999997	275999.999999997\\
-365999.999999999	-365999.999999999\\
256000	256000\\
127000.000000001	127000.000000001\\
-436999.999999999	-436999.999999999\\
273000.000000001	273000.000000001\\
-36999.9999999981	-36999.9999999981\\
92999.9999999955	92999.9999999955\\
-73999.9999999972	-73999.9999999972\\
-110999.999999999	-110999.999999999\\
239999.999999997	239999.999999997\\
-184000	-184000\\
-182999.999999997	-182999.999999997\\
550000	550000\\
-240000.000000002	-240000.000000002\\
-181000	-181000\\
236999.999999999	236999.999999999\\
-1.77635683940025e-09	-1.77635683940025e-09\\
-493999.999999999	-493999.999999999\\
529999.999999999	529999.999999999\\
-17000.0000000012	-17000.0000000012\\
-438999.999999998	-438999.999999998\\
364000.000000002	364000.000000002\\
-90000.0000000043	-90000.0000000043\\
201000.000000001	201000.000000001\\
-128000.000000002	-128000.000000002\\
-90999.9999999984	-90999.9999999984\\
-74999.9999999984	-74999.9999999984\\
-126000.000000002	-126000.000000002\\
383000	383000\\
-146000.000000001	-146000.000000001\\
-110000.000000001	-110000.000000001\\
111000.000000001	111000.000000001\\
181000.000000001	181000.000000001\\
-292000.000000002	-292000.000000002\\
38000.0000000038	38000.0000000038\\
253999.999999996	253999.999999996\\
-181999.999999999	-181999.999999999\\
-255999.999999998	-255999.999999998\\
511999.999999999	511999.999999999\\
-309999.999999999	-309999.999999999\\
107999.999999997	107999.999999997\\
-89999.999999999	-89999.999999999\\
-56000.0000000018	-56000.0000000018\\
440000.000000001	440000.000000001\\
-329999.999999999	-329999.999999999\\
-91000.0000000011	-91000.0000000011\\
-238000	-238000\\
311000	311000\\
109999.999999996	109999.999999996\\
18000.0000000025	18000.0000000025\\
17999.999999998	17999.999999998\\
-364999.999999999	-364999.999999999\\
-110999.999999997	-110999.999999997\\
384999.999999996	384999.999999996\\
128000.000000001	128000.000000001\\
90999.9999999993	90999.9999999993\\
-345999.999999999	-345999.999999999\\
-148000.000000002	-148000.000000002\\
73000.0000000004	73000.0000000004\\
238999.999999999	238999.999999999\\
-999.999999999446	-999.999999999446\\
56000	56000\\
-366999.999999998	-366999.999999998\\
219999.999999999	219999.999999999\\
147000.000000002	147000.000000002\\
-386000.000000001	-386000.000000001\\
423000.000000003	423000.000000003\\
-74000.0000000016	-74000.0000000016\\
-257000	-257000\\
237999.999999999	237999.999999999\\
166000.000000004	166000.000000004\\
-385000.000000002	-385000.000000002\\
-110000.000000001	-110000.000000001\\
218999.999999998	218999.999999998\\
93000.0000000035	93000.0000000035\\
35999.9999999969	35999.9999999969\\
-257000.000000001	-257000.000000001\\
238000	238000\\
-17000.0000000003	-17000.0000000003\\
-238999.999999998	-238999.999999998\\
219999.999999999	219999.999999999\\
-201000	-201000\\
236999.999999998	236999.999999998\\
54999.9999999997	54999.9999999997\\
-89999.9999999972	-89999.9999999972\\
52999.9999999964	52999.9999999964\\
-420000.000000001	-420000.000000001\\
439000.000000003	439000.000000003\\
-109000.000000003	-109000.000000003\\
-55999.9999999991	-55999.9999999991\\
-146000.000000003	-146000.000000003\\
146000	146000\\
129000	129000\\
-37999.9999999985	-37999.9999999985\\
130000.000000002	130000.000000002\\
-166000.000000001	-166000.000000001\\
-72000.0000000018	-72000.0000000018\\
-38999.9999999961	-38999.9999999961\\
-89000.0000000048	-89000.0000000048\\
236000.000000002	236000.000000002\\
37999.9999999985	37999.9999999985\\
-311999.999999999	-311999.999999999\\
275999.999999999	275999.999999999\\
-1999.99999999978	-1999.99999999978\\
18999.9999999983	18999.9999999983\\
-127000	-127000\\
327999.999999998	327999.999999998\\
-329000.000000001	-329000.000000001\\
17999.9999999998	17999.9999999998\\
-17000.0000000003	-17000.0000000003\\
-75000.0000000002	-75000.0000000002\\
258000.000000003	258000.000000003\\
-19000.0000000046	-19000.0000000046\\
-165999.999999999	-165999.999999999\\
-200000	-200000\\
257000.000000001	257000.000000001\\
198999.999999997	198999.999999997\\
-436999.999999998	-436999.999999998\\
309999.999999997	309999.999999997\\
146000.000000001	146000.000000001\\
-108000	-108000\\
-130000.000000002	-130000.000000002\\
-54999.9999999997	-54999.9999999997\\
-235999.999999999	-235999.999999999\\
510999.999999997	510999.999999997\\
-238000	-238000\\
38000.0000000029	38000.0000000029\\
-20000.000000004	-20000.000000004\\
55999.9999999983	55999.9999999983\\
-55999.9999999983	-55999.9999999983\\
165999.999999999	165999.999999999\\
-257000.000000002	-257000.000000002\\
220000.000000001	220000.000000001\\
-147000.000000003	-147000.000000003\\
-163999.999999999	-163999.999999999\\
145999.999999999	145999.999999999\\
146000	146000\\
184000	184000\\
-695999.999999999	-695999.999999999\\
549000.000000003	549000.000000003\\
164999.999999999	164999.999999999\\
-182999.999999999	-182999.999999999\\
-421000.000000002	-421000.000000002\\
310999.999999997	310999.999999997\\
73000.0000000022	73000.0000000022\\
-202000.000000003	-202000.000000003\\
313000.000000005	313000.000000005\\
-349000.000000002	-349000.000000002\\
-91000.0000000002	-91000.0000000002\\
439000	439000\\
109999.999999999	109999.999999999\\
-458000.000000002	-458000.000000002\\
128000	128000\\
56000	56000\\
89999.999999999	89999.999999999\\
-125999.999999998	-125999.999999998\\
88999.9999999977	88999.9999999977\\
-291000.000000003	-291000.000000003\\
237000.000000001	237000.000000001\\
-183000.000000001	-183000.000000001\\
239000.000000003	239000.000000003\\
-111000.000000002	-111000.000000002\\
274999.999999999	274999.999999999\\
-146999.999999998	-146999.999999998\\
-127000.000000002	-127000.000000002\\
-55999.9999999974	-55999.9999999974\\
184000.000000001	184000.000000001\\
-19000.0000000019	-19000.0000000019\\
-146000	-146000\\
-55000.0000000015	-55000.0000000015\\
291999.999999998	291999.999999998\\
-144999.999999999	-144999.999999999\\
-239000.000000003	-239000.000000003\\
147000.000000003	147000.000000003\\
273999.999999996	273999.999999996\\
-90999.9999999984	-90999.9999999984\\
-238000	-238000\\
-220000.000000003	-220000.000000003\\
385000.000000001	385000.000000001\\
457000.000000001	457000.000000001\\
-714000.000000003	-714000.000000003\\
166000.000000001	166000.000000001\\
125999.999999999	125999.999999999\\
-181000.000000002	-181000.000000002\\
200000.000000001	200000.000000001\\
19000.000000001	19000.000000001\\
-128000.000000001	-128000.000000001\\
35999.9999999978	35999.9999999978\\
-164000	-164000\\
-74000.0000000025	-74000.0000000025\\
293000.000000002	293000.000000002\\
184000.000000001	184000.000000001\\
-184000.000000004	-184000.000000004\\
-54999.9999999962	-54999.9999999962\\
-200999.999999999	-200999.999999999\\
-37000.0000000008	-37000.0000000008\\
274999.999999999	274999.999999999\\
36999.9999999999	36999.9999999999\\
90999.9999999984	90999.9999999984\\
-238000.000000001	-238000.000000001\\
-18999.9999999992	-18999.9999999992\\
-34999.9999999984	-34999.9999999984\\
17999.9999999989	17999.9999999989\\
163000	163000\\
129999.999999999	129999.999999999\\
-36999.9999999999	-36999.9999999999\\
-330000	-330000\\
109999.999999998	109999.999999998\\
72999.9999999995	72999.9999999995\\
110000.000000001	110000.000000001\\
-292000	-292000\\
163999.999999999	163999.999999999\\
-1.77635683940025e-09	-1.77635683940025e-09\\
182000.000000001	182000.000000001\\
-144000	-144000\\
-276999.999999997	-276999.999999997\\
92999.9999999955	92999.9999999955\\
146000.000000002	146000.000000002\\
35999.9999999996	35999.9999999996\\
2000.00000000156	2000.00000000156\\
51999.9999999987	51999.9999999987\\
-107000.000000002	-107000.000000002\\
-148999.999999998	-148999.999999998\\
130000.000000003	130000.000000003\\
255999.999999999	255999.999999999\\
-999.999999996781	-999.999999996781\\
-420000.000000001	-420000.000000001\\
165000	165000\\
-74999.9999999984	-74999.9999999984\\
349999.999999999	349999.999999999\\
53000.0000000008	53000.0000000008\\
-364000.000000002	-364000.000000002\\
-149000.000000001	-149000.000000001\\
277000.000000001	277000.000000001\\
53000.0000000026	53000.0000000026\\
-328000	-328000\\
52999.9999999982	52999.9999999982\\
149000	149000\\
89999.9999999981	89999.9999999981\\
55000.0000000024	55000.0000000024\\
-109999.999999999	-109999.999999999\\
-164000	-164000\\
109000.000000001	109000.000000001\\
91999.9999999987	91999.9999999987\\
-8.88178419700125e-10	-8.88178419700125e-10\\
73000.0000000004	73000.0000000004\\
-128000	-128000\\
-55000.0000000015	-55000.0000000015\\
-126999.999999997	-126999.999999997\\
199000.000000002	199000.000000002\\
147999.999999995	147999.999999995\\
-90999.9999999966	-90999.9999999966\\
-330999.999999999	-330999.999999999\\
439999.999999999	439999.999999999\\
-256000.000000001	-256000.000000001\\
-37000.0000000008	-37000.0000000008\\
55000.0000000015	55000.0000000015\\
92000.0000000005	92000.0000000005\\
238000	238000\\
-587000.000000003	-587000.000000003\\
440000.000000001	440000.000000001\\
-16999.9999999995	-16999.9999999995\\
-112000.000000001	-112000.000000001\\
-255000.000000001	-255000.000000001\\
273999.999999999	273999.999999999\\
92000.0000000005	92000.0000000005\\
-91999.999999997	-91999.999999997\\
74000.0000000007	74000.0000000007\\
-92000.0000000023	-92000.0000000023\\
-17999.9999999998	-17999.9999999998\\
-111000.000000001	-111000.000000001\\
257999.999999999	257999.999999999\\
-239000.000000002	-239000.000000002\\
146000.000000001	146000.000000001\\
1000.00000000033	1000.00000000033\\
-384999.999999999	-384999.999999999\\
696000.000000001	696000.000000001\\
-36999.999999999	-36999.999999999\\
-347000	-347000\\
-184000.000000004	-184000.000000004\\
91999.9999999996	91999.9999999996\\
91000.0000000002	91000.0000000002\\
36999.9999999981	36999.9999999981\\
37000.0000000017	37000.0000000017\\
16999.9999999986	16999.9999999986\\
130000.000000003	130000.000000003\\
-239000.000000002	-239000.000000002\\
-109999.999999999	-109999.999999999\\
219999.999999999	219999.999999999\\
128000	128000\\
-183000	-183000\\
-292000	-292000\\
345999.999999998	345999.999999998\\
-109000.000000002	-109000.000000002\\
1.77635683940025e-09	1.77635683940025e-09\\
220000.000000001	220000.000000001\\
-36999.9999999999	-36999.9999999999\\
-128000	-128000\\
-238000	-238000\\
202000.000000002	202000.000000002\\
-1000.00000000122	-1000.00000000122\\
110000	110000\\
-73000.0000000004	-73000.0000000004\\
147000.000000002	147000.000000002\\
-129000.000000001	-129000.000000001\\
220000	220000\\
-312000.000000001	-312000.000000001\\
-346000	-346000\\
438000.000000001	438000.000000001\\
347999.999999999	347999.999999999\\
-658999.999999999	-658999.999999999\\
182999.999999998	182999.999999998\\
184000.000000005	184000.000000005\\
236999.999999998	236999.999999998\\
-183000.000000002	-183000.000000002\\
-128000.000000001	-128000.000000001\\
-293000.000000001	-293000.000000001\\
274000	274000\\
38000.000000002	38000.000000002\\
164000	164000\\
-346999.999999999	-346999.999999999\\
199999.999999997	199999.999999997\\
-109999.999999999	-109999.999999999\\
92999.9999999982	92999.9999999982\\
-54999.9999999997	-54999.9999999997\\
-130000	-130000\\
350000	350000\\
-294000.000000002	-294000.000000002\\
54999.9999999997	54999.9999999997\\
219000	219000\\
57000.0000000013	57000.0000000013\\
-332000.000000002	-332000.000000002\\
-52999.9999999999	-52999.9999999999\\
347000	347000\\
-220000.000000002	-220000.000000002\\
111000.000000002	111000.000000002\\
-295000.000000003	-295000.000000003\\
259000.000000002	259000.000000002\\
180999.999999997	180999.999999997\\
-146000.000000001	-146000.000000001\\
-200999.999999999	-200999.999999999\\
-146000.000000001	-146000.000000001\\
219000	219000\\
182000.000000001	182000.000000001\\
75999.9999999987	75999.9999999987\\
-350999.999999998	-350999.999999998\\
38999.999999997	38999.999999997\\
36000.0000000005	36000.0000000005\\
346999.999999999	346999.999999999\\
-256000.000000001	-256000.000000001\\
-181999.999999997	-181999.999999997\\
145999.999999997	145999.999999997\\
109000	109000\\
-128000.000000001	-128000.000000001\\
73999.9999999998	73999.9999999998\\
73000.0000000004	73000.0000000004\\
-92000.0000000032	-92000.0000000032\\
-218999.999999997	-218999.999999997\\
328999.999999997	328999.999999997\\
-220000	-220000\\
-16999.9999999995	-16999.9999999995\\
-2000.00000000067	-2000.00000000067\\
423000.000000001	423000.000000001\\
-311999.999999999	-311999.999999999\\
-238000	-238000\\
457999.999999998	457999.999999998\\
-56000.0000000009	-56000.0000000009\\
-182000.000000002	-182000.000000002\\
8.88178419700125e-10	8.88178419700125e-10\\
-129000.000000002	-129000.000000002\\
221000.000000001	221000.000000001\\
-110999.999999999	-110999.999999999\\
-293000	-293000\\
880000.000000001	880000.000000001\\
-678999.999999999	-678999.999999999\\
-145000	-145000\\
218999.999999998	218999.999999998\\
275000	275000\\
-128999.999999996	-128999.999999996\\
-35999.9999999996	-35999.9999999996\\
-292999.999999998	-292999.999999998\\
-8.88178419700125e-10	-8.88178419700125e-10\\
292999.999999997	292999.999999997\\
220000.000000001	220000.000000001\\
-221000.000000002	-221000.000000002\\
-181999.999999999	-181999.999999999\\
165000.000000001	165000.000000001\\
-91000.0000000002	-91000.0000000002\\
-2000.00000000067	-2000.00000000067\\
37999.9999999985	37999.9999999985\\
109000	109000\\
999.999999999446	999.999999999446\\
18000.0000000007	18000.0000000007\\
-129000.000000002	-129000.000000002\\
-108999.999999999	-108999.999999999\\
1000.00000000122	1000.00000000122\\
363999.999999998	363999.999999998\\
-254999.999999999	-254999.999999999\\
-293999.999999999	-293999.999999999\\
348999.999999998	348999.999999998\\
147000.000000001	147000.000000001\\
-295000.000000003	-295000.000000003\\
38000.0000000038	38000.0000000038\\
458000	458000\\
-770000.000000003	-770000.000000003\\
422000.000000001	422000.000000001\\
127999.999999999	127999.999999999\\
-221000	-221000\\
57000.0000000022	57000.0000000022\\
-276000.000000001	-276000.000000001\\
440000.000000002	440000.000000002\\
146000.000000001	146000.000000001\\
-511999.999999999	-511999.999999999\\
90999.9999999984	90999.9999999984\\
219000	219000\\
-144999.999999998	-144999.999999998\\
90999.9999999993	90999.9999999993\\
-200999.999999999	-200999.999999999\\
327999.999999998	327999.999999998\\
-16000	-16000\\
-92999.9999999991	-92999.9999999991\\
-220000.000000003	-220000.000000003\\
91999.9999999996	91999.9999999996\\
147000.000000001	147000.000000001\\
-367000.000000002	-367000.000000002\\
201999.999999998	201999.999999998\\
402000	402000\\
-364999.999999999	-364999.999999999\\
-55999.9999999983	-55999.9999999983\\
239000.000000001	239000.000000001\\
-275000	-275000\\
328999.999999996	328999.999999996\\
-328999.999999997	-328999.999999997\\
54000.0000000003	54000.0000000003\\
-35000.0000000001	-35000.0000000001\\
218999.999999998	218999.999999998\\
-36999.9999999964	-36999.9999999964\\
-54999.9999999979	-54999.9999999979\\
-8.88178419700125e-10	-8.88178419700125e-10\\
-292000.000000001	-292000.000000001\\
439000.000000001	439000.000000001\\
-166000.000000002	-166000.000000002\\
-327999.999999998	-327999.999999998\\
365999.999999998	365999.999999998\\
127000.000000001	127000.000000001\\
1999.99999999978	1999.99999999978\\
-277000.000000002	-277000.000000002\\
-126999.999999999	-126999.999999999\\
733000.000000002	733000.000000002\\
-440000.000000004	-440000.000000004\\
-201999.999999999	-201999.999999999\\
184000	184000\\
-256999.999999996	-256999.999999996\\
72999.9999999986	72999.9999999986\\
92999.9999999973	92999.9999999973\\
402000	402000\\
-604999.999999999	-604999.999999999\\
146999.999999998	146999.999999998\\
349000.000000002	349000.000000002\\
-112000	-112000\\
2000.00000000067	2000.00000000067\\
-367000.000000003	-367000.000000003\\
127000.000000002	127000.000000002\\
75999.9999999987	75999.9999999987\\
400000	400000\\
-255000.000000003	-255000.000000003\\
-494000.000000001	-494000.000000001\\
512000.000000001	512000.000000001\\
109999.999999999	109999.999999999\\
-274999.999999998	-274999.999999998\\
-365000	-365000\\
89999.999999999	89999.999999999\\
623999.999999999	623999.999999999\\
-75000.0000000019	-75000.0000000019\\
-108000	-108000\\
-293999.999999999	-293999.999999999\\
494000.000000001	494000.000000001\\
-676000	-676000\\
273000.000000001	273000.000000001\\
-17000.0000000012	-17000.0000000012\\
-18999.9999999992	-18999.9999999992\\
91999.999999997	91999.999999997\\
127000.000000001	127000.000000001\\
-16999.9999999977	-16999.9999999977\\
-92000.0000000005	-92000.0000000005\\
36999.999999999	36999.999999999\\
-19000.0000000019	-19000.0000000019\\
19000.0000000001	19000.0000000001\\
-148000.000000001	-148000.000000001\\
-70999.9999999953	-70999.9999999953\\
364999.999999997	364999.999999997\\
165000.000000002	165000.000000002\\
-239000.000000004	-239000.000000004\\
-712999.999999997	-712999.999999997\\
622999.999999999	622999.999999999\\
199999.999999998	199999.999999998\\
20000.0000000013	20000.0000000013\\
-405000.000000002	-405000.000000002\\
-145000	-145000\\
238000.000000001	238000.000000001\\
257000.000000001	257000.000000001\\
-167000.000000002	-167000.000000002\\
37999.9999999994	37999.9999999994\\
129000.000000004	129000.000000004\\
-295000.000000002	-295000.000000002\\
442000	442000\\
-497000.000000003	-497000.000000003\\
-125999.999999999	-125999.999999999\\
273000	273000\\
73999.9999999972	73999.9999999972\\
-54999.9999999979	-54999.9999999979\\
219999.999999999	219999.999999999\\
-459000	-459000\\
332000.000000002	332000.000000002\\
88999.9999999968	88999.9999999968\\
-144999.999999998	-144999.999999998\\
-92000.0000000023	-92000.0000000023\\
-218999.999999998	-218999.999999998\\
420000.000000002	420000.000000002\\
1.77635683940025e-09	1.77635683940025e-09\\
-475000	-475000\\
127000	127000\\
568999.999999998	568999.999999998\\
-313000.000000001	-313000.000000001\\
-273000.000000001	-273000.000000001\\
90999.9999999993	90999.9999999993\\
421000	421000\\
-1000.00000000033	-1000.00000000033\\
-346000.000000001	-346000.000000001\\
-458999.999999994	-458999.999999994\\
310999.999999998	310999.999999998\\
386000	386000\\
-220999.999999999	-220999.999999999\\
91999.9999999987	91999.9999999987\\
110000.000000003	110000.000000003\\
73000.0000000004	73000.0000000004\\
-312000	-312000\\
-181999.999999999	-181999.999999999\\
476000.000000003	476000.000000003\\
-238000.000000002	-238000.000000002\\
-55999.9999999991	-55999.9999999991\\
56000	56000\\
55000.0000000033	55000.0000000033\\
-202000.000000001	-202000.000000001\\
458000	458000\\
-109999.999999999	-109999.999999999\\
-55000.0000000024	-55000.0000000024\\
-275000.000000001	-275000.000000001\\
-89999.9999999998	-89999.9999999998\\
199999.999999998	199999.999999998\\
54999.9999999997	54999.9999999997\\
-273999.999999998	-273999.999999998\\
512999.999999999	512999.999999999\\
-258000.000000003	-258000.000000003\\
-126999.999999999	-126999.999999999\\
403000	403000\\
-403000.000000001	-403000.000000001\\
-38000.0000000002	-38000.0000000002\\
186000.000000001	186000.000000001\\
252999.999999998	252999.999999998\\
-216999.999999999	-216999.999999999\\
-294000	-294000\\
255999.999999998	255999.999999998\\
-403000	-403000\\
441000.000000001	441000.000000001\\
254000	254000\\
-345999.999999998	-345999.999999998\\
-36999.9999999964	-36999.9999999964\\
-37000.0000000035	-37000.0000000035\\
312000.000000003	312000.000000003\\
89999.9999999972	89999.9999999972\\
-272999.999999998	-272999.999999998\\
89999.9999999963	89999.9999999963\\
-365000.000000001	-365000.000000001\\
110000.000000002	110000.000000002\\
90999.9999999984	90999.9999999984\\
182999.999999998	182999.999999998\\
-73999.9999999989	-73999.9999999989\\
-198999.999999998	-198999.999999998\\
125999.999999996	125999.999999996\\
1000.000000003	1000.000000003\\
-19000.0000000037	-19000.0000000037\\
220000.000000001	220000.000000001\\
36999.9999999972	36999.9999999972\\
-274999.999999998	-274999.999999998\\
-36000.0000000022	-36000.0000000022\\
17000.0000000021	17000.0000000021\\
-35999.9999999978	-35999.9999999978\\
-36000.0000000014	-36000.0000000014\\
237999.999999999	237999.999999999\\
-92999.9999999964	-92999.9999999964\\
-182000.000000004	-182000.000000004\\
458000.000000003	458000.000000003\\
-476000.000000001	-476000.000000001\\
53999.9999999985	53999.9999999985\\
495000.000000001	495000.000000001\\
-256000.000000001	-256000.000000001\\
-239000.000000001	-239000.000000001\\
109999.999999999	109999.999999999\\
257000.000000004	257000.000000004\\
-220000	-220000\\
91999.9999999996	91999.9999999996\\
-220000	-220000\\
-239000.000000003	-239000.000000003\\
221000.000000004	221000.000000004\\
329000.000000002	329000.000000002\\
90999.9999999975	90999.9999999975\\
-217999.999999999	-217999.999999999\\
-1000.00000000033	-1000.00000000033\\
-147999.999999999	-147999.999999999\\
-106999.999999998	-106999.999999998\\
565999.999999999	565999.999999999\\
-385000	-385000\\
-90000.0000000025	-90000.0000000025\\
-166000	-166000\\
366000	366000\\
312000	312000\\
-622000.000000001	-622000.000000001\\
254999.999999999	254999.999999999\\
-274000.000000001	-274000.000000001\\
92000.0000000014	92000.0000000014\\
346999.999999997	346999.999999997\\
-90999.9999999975	-90999.9999999975\\
-19000.0000000028	-19000.0000000028\\
-218000	-218000\\
163000.000000001	163000.000000001\\
72999.9999999968	72999.9999999968\\
-125999.999999997	-125999.999999997\\
-2000.00000000244	-2000.00000000244\\
19000.0000000001	19000.0000000001\\
-90999.9999999993	-90999.9999999993\\
-128999.999999998	-128999.999999998\\
164999.999999999	164999.999999999\\
255999.999999998	255999.999999998\\
1000.00000000122	1000.00000000122\\
-276000.000000002	-276000.000000002\\
57000.0000000013	57000.0000000013\\
-111999.999999998	-111999.999999998\\
54999.9999999988	54999.9999999988\\
149000	149000\\
-40000.0000000009	-40000.0000000009\\
186000.000000002	186000.000000002\\
-661000.000000001	-661000.000000001\\
420999.999999999	420999.999999999\\
350000.000000001	350000.000000001\\
-21000.0000000026	-21000.0000000026\\
-583999.999999998	-583999.999999998\\
-19000.0000000019	-19000.0000000019\\
421000	421000\\
-18000.0000000007	-18000.0000000007\\
-19000.0000000001	-19000.0000000001\\
-346999.999999999	-346999.999999999\\
128000.000000002	128000.000000002\\
475999.999999998	475999.999999998\\
-72999.9999999968	-72999.9999999968\\
-587000.000000001	-587000.000000001\\
-127000.000000001	-127000.000000001\\
584999.999999998	584999.999999998\\
221000	221000\\
-385999.999999999	-385999.999999999\\
-347000	-347000\\
383999.999999998	383999.999999998\\
184000.000000003	184000.000000003\\
-331000.000000001	-331000.000000001\\
75000.0000000028	75000.0000000028\\
364999.999999997	364999.999999997\\
-257000.000000001	-257000.000000001\\
-326999.999999998	-326999.999999998\\
472999.999999998	472999.999999998\\
-309000	-309000\\
-20000.0000000005	-20000.0000000005\\
350000.000000001	350000.000000001\\
15999.9999999973	15999.9999999973\\
-621000.000000001	-621000.000000001\\
109000.000000001	109000.000000001\\
440000	440000\\
109000.000000001	109000.000000001\\
-254999.999999999	-254999.999999999\\
-239000.000000001	-239000.000000001\\
385000.000000002	385000.000000002\\
18000.0000000007	18000.0000000007\\
-293000.000000002	-293000.000000002\\
-72999.9999999995	-72999.9999999995\\
202000.000000002	202000.000000002\\
16999.9999999959	16999.9999999959\\
-72999.9999999995	-72999.9999999995\\
184000	184000\\
-220000	-220000\\
-93000	-93000\\
442000	442000\\
-258000.000000002	-258000.000000002\\
-55000.0000000006	-55000.0000000006\\
18999.9999999992	18999.9999999992\\
-36999.999999999	-36999.999999999\\
-202000	-202000\\
404000.000000002	404000.000000002\\
-220000.000000001	-220000.000000001\\
54999.9999999997	54999.9999999997\\
-1000.00000000122	-1000.00000000122\\
257000.000000001	257000.000000001\\
-219000.000000002	-219000.000000002\\
273000	273000\\
-658000.000000001	-658000.000000001\\
19000.000000001	19000.000000001\\
730999.999999997	730999.999999997\\
-164999.999999999	-164999.999999999\\
-620999.999999999	-620999.999999999\\
255999.999999999	255999.999999999\\
346000	346000\\
-289999.999999999	-289999.999999999\\
15999.9999999973	15999.9999999973\\
385000.000000002	385000.000000002\\
-256000.000000001	-256000.000000001\\
-238000	-238000\\
385000.000000002	385000.000000002\\
-203000.000000003	-203000.000000003\\
-144999.999999999	-144999.999999999\\
72999.9999999995	72999.9999999995\\
72999.9999999968	72999.9999999968\\
73000.0000000004	73000.0000000004\\
127999.999999999	127999.999999999\\
-255999.999999998	-255999.999999998\\
1000.00000000033	1000.00000000033\\
-75000.0000000028	-75000.0000000028\\
148000.000000001	148000.000000001\\
90999.9999999984	90999.9999999984\\
-422000.000000001	-422000.000000001\\
403999.999999999	403999.999999999\\
145999.999999999	145999.999999999\\
-238999.999999998	-238999.999999998\\
-198999.999999999	-198999.999999999\\
418999.999999996	418999.999999996\\
-72000.0000000001	-72000.0000000001\\
-367000	-367000\\
72999.9999999995	72999.9999999995\\
238999.999999999	238999.999999999\\
18000.0000000007	18000.0000000007\\
17999.999999998	17999.999999998\\
-365999.999999999	-365999.999999999\\
17999.9999999989	17999.9999999989\\
183999.999999999	183999.999999999\\
127000.000000001	127000.000000001\\
-90999.9999999984	-90999.9999999984\\
19000.000000001	19000.000000001\\
364999.999999998	364999.999999998\\
-421000.000000002	-421000.000000002\\
-401999.999999997	-401999.999999997\\
328999.999999998	328999.999999998\\
477000.000000004	477000.000000004\\
-313000.000000002	-313000.000000002\\
-328000.000000001	-328000.000000001\\
475000.000000001	475000.000000001\\
-92000.0000000005	-92000.0000000005\\
-70999.9999999979	-70999.9999999979\\
-130000.000000003	-130000.000000003\\
-73000.0000000022	-73000.0000000022\\
366000.000000003	366000.000000003\\
-163000.000000001	-163000.000000001\\
-350000.000000001	-350000.000000001\\
310999.999999999	310999.999999999\\
131000.000000003	131000.000000003\\
252999.999999999	252999.999999999\\
-291000.000000002	-291000.000000002\\
-750999.999999998	-750999.999999998\\
860999.999999997	860999.999999997\\
181000.000000001	181000.000000001\\
-473000.000000001	-473000.000000001\\
-239999.999999999	-239999.999999999\\
201000	201000\\
1999.99999999711	1999.99999999711\\
492000.000000002	492000.000000002\\
-254000	-254000\\
-20000.0000000022	-20000.0000000022\\
-72999.9999999986	-72999.9999999986\\
-292000	-292000\\
328000.000000001	328000.000000001\\
-125999.999999999	-125999.999999999\\
-130000	-130000\\
384999.999999999	384999.999999999\\
127999.999999999	127999.999999999\\
-457000.000000001	-457000.000000001\\
-18999.9999999983	-18999.9999999983\\
129000	129000\\
255000.000000004	255000.000000004\\
-421000.000000001	-421000.000000001\\
330999.999999997	330999.999999997\\
-36999.9999999981	-36999.9999999981\\
-476999.999999999	-476999.999999999\\
365999.999999999	365999.999999999\\
-52999.9999999999	-52999.9999999999\\
144999.999999998	144999.999999998\\
146000.000000001	146000.000000001\\
129000	129000\\
-421000	-421000\\
-91999.9999999987	-91999.9999999987\\
311999.999999998	311999.999999998\\
-202000	-202000\\
-293000.000000001	-293000.000000001\\
403000	403000\\
90999.9999999993	90999.9999999993\\
-218999.999999999	-218999.999999999\\
72999.9999999986	72999.9999999986\\
-8.88178419700125e-10	-8.88178419700125e-10\\
2.66453525910038e-09	2.66453525910038e-09\\
-55000.0000000006	-55000.0000000006\\
129000.000000001	129000.000000001\\
-240000.000000002	-240000.000000002\\
533000.000000002	533000.000000002\\
-129000.000000002	-129000.000000002\\
-513000.000000001	-513000.000000001\\
-73000.0000000004	-73000.0000000004\\
587000.000000001	587000.000000001\\
-185000.000000003	-185000.000000003\\
184000.000000001	184000.000000001\\
-529999.999999999	-529999.999999999\\
328000.000000002	328000.000000002\\
146999.999999998	146999.999999998\\
-439999.999999999	-439999.999999999\\
19999.9999999996	19999.9999999996\\
327999.999999999	327999.999999999\\
-182000.000000002	-182000.000000002\\
90000.0000000007	90000.0000000007\\
295000.000000003	295000.000000003\\
16999.9999999959	16999.9999999959\\
-732000	-732000\\
366000	366000\\
401999.999999999	401999.999999999\\
-546999.999999999	-546999.999999999\\
-93999.9999999994	-93999.9999999994\\
495999.999999999	495999.999999999\\
-349000.000000003	-349000.000000003\\
111000.000000003	111000.000000003\\
492999.999999997	492999.999999997\\
-529999.999999997	-529999.999999997\\
17999.9999999998	17999.9999999998\\
-126999.999999999	-126999.999999999\\
143999.999999998	143999.999999998\\
258000	258000\\
-54999.999999997	-54999.999999997\\
};
\end{axis}

\begin{axis}[%
width=4.927cm,
height=3.484cm,
at={(0cm,0cm)},
scale only axis,
xmin=-1000000,
xmax=1000000,
xlabel style={font=\color{white!15!black}},
xlabel={$\delta^3 u(t)$},
ymin=-26001000000,
ymax=29419000000,
ylabel style={font=\color{white!15!black}},
ylabel={y(t)},
axis background/.style={fill=white},
title={C9, R = 0.5298},
axis x line*=bottom,
axis y line*=left
]
\addplot[only marks, mark=*, mark options={}, mark size=1.5000pt, color=mycolor1, fill=mycolor1] table[row sep=crcr]{%
x	y\\
-111000	2929700000\\
185000	7202099999.99999\\
-112000.000000002	-10498000000\\
221000.000000004	11108400000\\
54999.9999999988	-9399400000\\
-457999.999999997	-2441500000\\
182999.999999997	12085000000\\
8.88178419700125e-10	-10131700000\\
129000.000000001	8178600000\\
272999.999999998	-6225699999.99999\\
-383000.000000002	-976400000.000003\\
-147999.999999997	3662100000\\
-345999.999999998	-2929700000\\
547000	4394400000\\
221999.999999999	366400000\\
-146999.999999999	-5249200000\\
34999.9999999966	4638900000.00001\\
19999.9999999996	-4394700000\\
-622999.999999999	-4150400000\\
750999.999999998	18920900000\\
-148000.000000001	-20263500000\\
-454999.999999997	5492900000\\
472999.999999997	9155500000\\
-51999.9999999978	-10009900000\\
-369000.000000002	-2075200000\\
423000.000000002	13916200000\\
35999.9999999996	-11352700000\\
-419999.999999998	-3662300000\\
271999.999999998	14038600000\\
149000.000000002	-9277900000\\
-221000.000000001	-731999999.999997\\
93000.0000000026	2441200000\\
-204000.000000004	-1708900000\\
94000.0000000021	3417800000\\
108999.999999996	-1220500000\\
-36999.999999999	-1831200000\\
-36999.9999999999	1465100000\\
-15999.9999999973	-854899999.999998\\
-58000.0000000043	-1342500000\\
203000.000000004	7934600000\\
111000.000000001	-8544999999.99999\\
-351000.000000003	-1220800000\\
314000.000000004	7568400000\\
-312000.000000003	-7812300000.00001\\
-73000.0000000013	4638500000\\
346999.999999999	3295800000\\
-201000	-7812300000\\
-37000.0000000008	3418000000\\
73999.9999999998	3051400000\\
128000.000000002	-2807200000\\
-293000	-4394700000\\
385000	13793900000\\
-2000.00000000244	-15258800000\\
-271999.999999997	4638800000\\
-94000.000000003	1098600000\\
112000.000000002	2563300000\\
309999.999999997	122400000.000004\\
-364999.999999998	-7446700000\\
-56999.9999999977	5249400000\\
294999.999999997	3906000000\\
-74999.9999999966	-6713700000\\
-218000.000000003	1220600000\\
164000.000000001	3662100000\\
-129000	-5126900000\\
258000.000000001	7812400000\\
-129000.000000003	-10009500000\\
54000.000000002	9643300000\\
-70999.9999999988	-9155300000\\
107999.999999996	9033400000\\
-109999.999999999	-8545000000\\
19999.9999999987	6103400000\\
-20000.0000000005	-3295700000\\
1000.00000000122	1708800000\\
-201000.000000001	-4150300000\\
145000	7690500000\\
256999.999999998	-1220900000\\
-91999.9999999978	-5126800000\\
112000.000000001	5981500000\\
-461000.000000002	-15869300000\\
149000.000000003	20507800000\\
236999.999999999	-6713600000\\
37000.0000000035	609999999.999995\\
-385000.000000002	-13671600000\\
129000.000000003	18310300000\\
346999.999999999	2807900000\\
-164999.999999999	-22339100000\\
-34999.9999999993	17822400000\\
-57000.0000000013	-9033200000\\
-90000.0000000016	5493000000\\
-36999.999999999	-3051600000\\
273999.999999997	6591700000\\
-35999.9999999996	-8544700000\\
-238000.000000002	2319000000\\
-999.999999998557	1709200000\\
-34999.9999999984	-610399999.999998\\
328999.999999999	3051800000\\
-165000.000000001	-5859500000\\
-74000.0000000016	1709200000\\
111000.000000003	4638400000\\
-147000	-7934400000\\
239000.000000001	11352500000\\
-130000.000000003	-15380700000\\
93000.0000000044	17577800000\\
-19000.0000000028	-16479300000\\
-255999.999999999	7934600000\\
348000.000000002	3662000000\\
-183000.000000003	-8422700000\\
-129000	4394500000\\
183999.999999999	1586600000\\
126999.999999998	-365700000\\
-291999.999999999	-5737700000\\
202000.000000002	8178800000\\
-130000.000000003	-7568100000\\
2000.00000000244	5981000000\\
181999.999999997	-1220300000\\
-238000	-4638900000\\
17999.9999999998	5615400000\\
-35000.000000001	-3662400000\\
254000	5371300000\\
-71000.0000000015	-6469600000\\
126000	7201900000.00001\\
-365000.000000002	-14038000000\\
-53999.9999999958	14160300000\\
546999.999999996	976199999.999997\\
-381999.999999997	-13061000000\\
88999.9999999977	11352000000\\
204000.000000001	-1708600000\\
-203000.000000002	-7812600000\\
-293000	4760600000\\
255999.999999999	5005000000\\
-125999.999999996	-8422800000\\
161999.999999997	8911000000\\
149000.000000002	-3662000000\\
-56999.9999999995	-854399999.999995\\
-145000.000000002	-5859700000\\
16999.9999999995	10376200000\\
37000.0000000026	-7079999999.99999\\
-35000.000000001	3417699999.99999\\
162999.999999997	1343100000\\
-219999.999999998	-6469900000\\
56000	5737200000\\
18999.9999999992	-1220500000\\
-94000.0000000012	-1465000000\\
2000.00000000244	610499999.999996\\
-36999.9999999972	732200000.000003\\
384999.999999998	6348000000\\
-111000.000000002	-11963200000\\
-202000.000000001	3906300000.00001\\
-15999.9999999991	2075300000\\
235999.999999999	2075200000\\
-144999.999999998	-5615300000\\
-202000.000000003	3051700000\\
-55999.9999999974	-732400000.000001\\
477999.999999998	5249300000\\
-276000.000000001	-9399900000\\
18999.9999999992	7202600000\\
36000.0000000014	-3174200000\\
54999.9999999997	1709400000\\
19000.0000000001	-2441900000\\
-92999.9999999982	1831500000\\
-254000	-1953400000\\
-20000.0000000005	1587000000\\
329000	2441500000\\
130000.000000003	-2075500000\\
71999.9999999974	2808100000\\
-199999.999999997	-9033699999.99999\\
-76000.000000004	6958300000\\
38999.9999999988	-732399999.999998\\
-385000	-3662300000\\
475999.999999998	9277400000\\
-165000	-10742000000\\
-8.88178419700125e-10	7568200000\\
-110000	-6958099999.99999\\
275000.000000001	11108600000\\
165000.000000002	-6103700000\\
-294000	-7446100000\\
201999.999999999	12451100000\\
-346999.999999998	-13672000000\\
89999.9999999998	11474700000\\
75000.0000000002	-2807500000.00001\\
107999.999999997	2197100000\\
-90999.9999999966	-8422799999.99999\\
-90000.0000000007	6835900000\\
88999.9999999986	100000.000001899\\
-88999.9999999977	-4394600000\\
198999.999999997	8545000000\\
-364000	-15381000000\\
217999.999999999	18554700000\\
276000	-7079800000\\
-111000.000000002	-5859899999.99999\\
-328999.999999999	733000000.000001\\
-74000.0000000025	3295600000\\
294000.000000003	5615100000\\
-35999.9999999996	-8300499999.99999\\
89000.0000000004	5370900000\\
-51999.9999999987	-3783999999.99999\\
88999.9999999986	4028199999.99999\\
-107999.999999999	-8667100000.00001\\
-183999.999999997	5859700000.00001\\
146999.999999999	2074800000\\
292000.000000002	5249400000.00001\\
-218000.000000003	-16601900000\\
-350000	6470000000\\
349000.000000001	11840600000\\
129000	-12695200000\\
-405000.000000001	-121899999.999996\\
186000.000000003	8544500000\\
142999.999999996	-4028000000\\
-32999.999999995	-3295900000\\
-259000.000000004	1098500000\\
274999.999999999	5859400000\\
-162999.999999997	-9277200000\\
-75000.0000000046	6591600000\\
311000.000000003	1953200000\\
-17000.0000000012	-4882900000\\
-237999.999999997	-2441100000\\
-92000.0000000058	3051500000\\
-999.999999996781	1464800000\\
531999.999999999	9155400000\\
-108999.999999999	-18798800000\\
-295000.000000002	7568200000\\
-145000.000000002	1342900000\\
238000	2563500000\\
35999.9999999987	-3174000000\\
-164999.999999998	-1342500000\\
18999.9999999966	2563300000\\
36000.0000000031	-366299999.999999\\
55999.9999999991	610500000\\
164000.000000001	610299999.999999\\
-458000.000000002	-8544800000\\
294000	13305500000\\
126999.999999998	-6713900000\\
-73000.0000000004	488499999.999999\\
-127999.999999997	-4028500000\\
-109000.000000003	4760700000\\
419999.999999998	6592100000.00001\\
18000.0000000007	-13672300000\\
-603000.000000002	2075500000.00001\\
219000	9887700000\\
18999.9999999992	-8301100000\\
254999.999999998	6226000000\\
-53999.9999999994	-5859600000\\
-348000.000000001	-732499999.999997\\
312000.000000002	7934800000\\
89999.9999999972	-5981600000\\
-347000.000000001	-2807500000\\
256000.000000002	7812300000\\
-53999.9999999985	-5493100000\\
-74000.0000000043	-121800000\\
219000.000000001	6347300000\\
-273000	-10742000000\\
162999.999999998	10498000000\\
-125999.999999999	-7690500000\\
-166999.999999998	2929800000\\
330999.999999997	3418000000\\
-111000	-5493300000\\
-53999.9999999967	2319400000\\
385000	6713800000\\
-148000	-13549600000\\
-237000.000000001	6347500000.00001\\
-35999.9999999996	1098600000\\
-75000.000000001	-1464800000\\
403999.999999998	5859300000\\
-293999.999999999	-10619800000\\
-53000.0000000008	7812100000\\
-2000.00000000067	-3539900000\\
294000	5371200000\\
-256000.000000002	-9399500000.00001\\
328000.000000002	13305700000\\
-236000	-15381000000\\
144999.999999997	11841000000\\
-238000	-9033300000\\
93000.0000000017	8545000000\\
-74999.9999999975	-6836000000\\
-54000.0000000011	4638700000\\
-17999.9999999998	-3662200000\\
218000.000000001	5859500000\\
2000.00000000067	-6836000000\\
-92000.0000000014	4272500000\\
-164999.999999999	-4882900000\\
365999.999999996	11474700000\\
18000.0000000007	-12573200000\\
-366000	1586800000\\
165999.999999999	6347600000\\
-93999.9999999985	-5370800000\\
241000	7934200000\\
33999.999999998	-9399200000\\
-52999.9999999982	2197200000\\
-221000.000000001	610500000\\
-72999.9999999986	1098500000\\
73999.9999999998	121900000.000002\\
90000.0000000007	610699999.999998\\
38999.9999999997	-1465000000\\
-130999.999999999	-1098800000\\
-52000.0000000023	1953400000\\
326999.999999999	2563400000\\
-145000	-6347900000\\
-109999.999999999	4028800000\\
273999.999999996	2074700000\\
-72999.9999999977	-5614900000\\
-512000.000000002	-854700000.000002\\
256000.000000001	7690600000\\
328999.999999996	-1831100000\\
-200999.999999998	-5615400000\\
72999.9999999986	6469900000\\
37000.0000000017	-4760700000\\
-329000.000000001	-1342900000\\
109000	4882700000\\
255999.999999997	1465300000\\
37000.0000000017	-2930200000\\
-531000	-9887500000\\
586000.000000001	23193500000\\
-200999.999999998	-22827400000\\
-147000.000000002	10254100000\\
147000.000000003	1220700000\\
90999.9999999975	-1587100000\\
-165000.000000002	-3661900000\\
-200999.999999997	-366299999.999997\\
147000.000000001	7812400000\\
401999.999999997	1709200000\\
-109999.999999999	-11718900000\\
-347000.000000001	1220700000.00001\\
54000.000000002	7202200000\\
-16999.9999999995	-4028200000\\
107999.999999997	3295500000\\
203000.000000004	610899999.999999\\
-330000.000000003	-9033700000\\
35999.9999999987	7690700000\\
165000.000000003	976599999.999998\\
92999.9999999991	121900000.000002\\
-149000.000000001	-8056600000\\
-236000.000000003	4882900000\\
35000.0000000028	244200000.000001\\
386999.999999999	8544600000.00001\\
-131000.000000001	-15014400000\\
93000.0000000008	12207100000\\
-201999.999999999	-12451300000\\
-108999.999999998	10009700000\\
348000.000000001	-1220500000\\
-148000.000000006	-3784300000\\
-126999.999999998	1830900000\\
17999.9999999989	1221000000\\
-311000.000000001	-3784300000\\
365999.999999999	6347500000\\
110000.000000002	-3661800000\\
72999.9999999995	1586700000\\
-220000.000000002	-5615300000\\
-53999.9999999985	6103700000\\
311000.000000001	488199999.999998\\
-312000.000000003	-6958000000\\
37000.0000000008	6713900000\\
145999.999999999	-1220700000\\
-182000	-4272700000\\
-129000	4761200000\\
257000.000000001	-488700000.000001\\
274999.999999999	2807800000\\
-277000.000000001	-8911100000\\
-107000.000000001	5248900000\\
-57000.0000000004	100000.000001899\\
294000	2197300000\\
-549000.000000001	-7568600000\\
255999.999999999	7935000000\\
530000	4516200000\\
-530000.000000002	-18066300000\\
92000.0000000014	16601700000\\
126999.999999998	-5249200000\\
54999.9999999997	1098799999.99999\\
-71999.9999999965	-5371300000\\
-184000.000000002	2319600000\\
54999.999999997	3661900000\\
0	-3417900000\\
37000.0000000017	2319400000\\
53999.9999999976	610100000.000001\\
-145000	-6469400000\\
144999.999999999	9887600000\\
-53999.9999999985	-8545100000\\
274000	12695400000\\
-348000.000000001	-23803400000\\
-16999.9999999959	21361800000\\
254999.999999997	-2807300000\\
-254999.999999997	-12573400000\\
199999.999999997	15869400000\\
110000.000000001	-7935000000\\
-164000.000000001	-3905799999.99999\\
-257000.000000001	5004700000\\
37000.0000000026	-366399999.999999\\
311999.999999998	4638900000\\
-2000.00000000244	-8422800000\\
-290999.999999999	1952900000\\
236999.999999998	6225700000\\
37000.0000000017	-7202000000\\
-293999.999999999	1586500000\\
166999.999999998	2930200000\\
126000	-610699999.999999\\
-329000.000000001	-6347500000\\
166000.000000002	9399300000\\
291999.999999998	-1586900000\\
-477000	-11474400000\\
130000.000000002	14281900000\\
383000.000000001	-1220400000\\
-327999.999999999	-12329400000\\
236000.000000001	18433000000\\
-236000	-21606900000\\
145000	19043300000\\
55999.9999999983	-11352600000\\
-440999.999999999	244000000.000003\\
405000.000000001	8911300000\\
-38000.0000000029	-7934600000\\
-311000.000000001	610299999.999997\\
457000	6591800000\\
-126999.999999999	-9033300000\\
-293000.000000001	4272700000\\
-92999.9999999973	-976699999.999998\\
312999.999999997	4150400000\\
91000.0000000019	-5249100000\\
-17999.9999999998	4638800000\\
-56000.0000000027	-4638700000\\
-91000.0000000011	244100000.000003\\
183000.000000002	5981400000\\
-128000.000000002	-7446200000\\
-73000.0000000013	2441400000\\
165000	3173900000.00001\\
-313000	-7324500000\\
442000.000000003	13061800000\\
-203000.000000003	-13671900000\\
-220000.000000001	3661900000\\
111000.000000003	4272800000\\
328999.999999999	2196900000\\
-164999.999999999	-9643400000\\
-127000.000000001	4883000000\\
-39000.0000000015	-122300000.000001\\
240000.000000002	3417900000\\
-183000	-6591500000\\
-148000.000000003	3417700000\\
167000.000000003	1587000000\\
51999.9999999978	-1464700000\\
94000.0000000012	1220400000\\
-238999.999999999	-5370800000\\
146999.999999998	7934400000\\
200999.999999999	-3540000000\\
-422000.000000001	-5004800000\\
-90000.0000000007	4882700000\\
238000.000000001	2441400000\\
272999.999999999	-244000000.000005\\
-145000	-3784400000\\
-110000	-1220400000\\
71999.9999999992	5126600000\\
-108000.000000001	-5126700000\\
108000	4394500000\\
-218000	-5127100000\\
254999.999999998	8300900000\\
74000.0000000016	-6103400000\\
-439000.000000001	-3540300000\\
273000	8789200000\\
19999.9999999996	-5004900000\\
-999.999999998557	1342900000\\
999.999999997669	-244400000.000007\\
-129999.999999999	-3417800000\\
56999.9999999986	6103500000\\
127000.000000002	-2685500000\\
219000.000000001	5615100000\\
-144000.000000002	-13305500000\\
-204000.000000001	6347400000\\
55999.9999999983	2197600000\\
-311000.000000001	-2075400000\\
347999.999999998	4394500000\\
-201999.999999999	-7812400000\\
310999.999999999	10376000000\\
-34999.9999999993	-8789200000\\
-277000.000000002	244300000.000004\\
166000	5493100000\\
-16999.9999999977	-3540200000\\
51999.9999999987	1098900000\\
-161999.999999998	-4028400000\\
254999.999999998	12206900000\\
73000.0000000013	-14404200000\\
-128000	6225600000\\
-219000.000000001	-3173800000.00001\\
91000.0000000011	4882800000\\
-18000.0000000007	-3540300000\\
238000.000000001	6470200000\\
-203000.000000003	-10986600000\\
-15999.9999999947	8422700000\\
311000	-854200000.000003\\
-569000.000000003	-8545000000\\
128999.999999999	12451100000\\
531000.000000002	-2807600000\\
-477000.000000002	-8911100000\\
-144999.999999997	4272400000\\
604000.000000001	13183700000\\
-312000.000000001	-20507800000\\
-237999.999999999	7812200000\\
349000	8179100000\\
89999.9999999963	-6958200000.00001\\
-163999.999999999	-4638600000\\
-238000.000000002	3784000000\\
-18999.9999999992	854800000.000002\\
20000.0000000013	1830800000\\
400999.999999998	2319399999.99999\\
-127999.999999998	-7079999999.99999\\
20000.0000000022	5859299999.99999\\
-2000.00000000244	-5249000000\\
-162999.999999998	-366300000.000001\\
52999.9999999973	5737400000\\
-52999.9999999982	-4516400000\\
52999.9999999964	2563000000\\
-144999.999999996	-2685200000\\
127999.999999998	4272400000\\
274000.000000001	-976599999.999998\\
-512000.000000004	-7202100000\\
164000.000000001	8178700000\\
165999.999999999	-854600000.000001\\
-148000.000000001	-3662100000\\
74000.0000000016	2075500000\\
-17999.999999998	854099999.999997\\
274000.000000001	4150600000.00001\\
-219000	-12451200000\\
-164999.999999998	7568300000.00001\\
218999.999999998	3173899999.99999\\
-36000.0000000005	-4516699999.99999\\
-146000.000000001	-2563400000\\
-184000.000000001	3906400000\\
366000.000000001	5126600000.00001\\
276000.000000002	-1952800000\\
-294000.000000004	-11108600000\\
-128000.000000002	7568400000\\
183000	2075300000\\
-383999.999999999	-3784300000\\
-36999.9999999999	2929800000\\
365999.999999997	244000000.000001\\
-127999.999999998	-3295800000\\
239000.000000001	5615200000\\
-93000.0000000008	-5737300000\\
-145999.999999997	244099999.999998\\
1.77635683940025e-09	3784400000\\
108999.999999999	-2563900000\\
-144000.000000003	244600000\\
-39000.0000000006	-488600000.000002\\
38000.0000000011	1465000000\\
218999.999999998	2441400000\\
8.88178419700125e-10	-5127200000\\
-202000	610799999.999998\\
-16999.9999999977	1708600000\\
183000.000000002	2807800000\\
-129000	-7202199999.99999\\
92000.0000000014	9521500000.00001\\
165000	-6958000000.00001\\
-202000	-488300000.000001\\
-109000	122000000\\
-239000.000000003	200000.000005218\\
458000.000000003	9032999999.99999\\
91999.9999999978	-5493099999.99999\\
-91999.9999999987	-6713800000\\
-37000.0000000026	5981400000\\
-126999.999999999	-2685600000\\
-183000.000000001	2319400000\\
346000.000000002	1708900000\\
-474000.000000001	-4882600000\\
290999.999999999	4028100000\\
202999.999999998	610299999.999999\\
-72999.9999999995	-3295600000\\
90000.0000000025	1953000000\\
-201000	-2075500000\\
-164000.000000001	610800000.000001\\
219999.999999998	3173600000\\
-19999.9999999978	-3417900000\\
74999.9999999975	2319200000\\
-19999.9999999969	-2685400000\\
-273000.000000001	488299999.999998\\
-147000.000000001	-732499999.999999\\
584999.999999999	7934500000\\
-35999.999999996	-8788900000\\
999.999999999446	5859300000\\
-349000	-12695300000\\
-74000.0000000025	13427600000\\
497000.000000003	3052000000\\
-204000.000000001	-13794100000\\
37000.0000000008	10986400000\\
-126999.999999999	-10132000000\\
-37000.0000000026	7690700000\\
255000.000000003	3295700000.00001\\
-254000.000000001	-12329100000\\
199999.999999999	12085000000\\
-202000.000000001	-8788900000\\
-273000	3051500000\\
620999.999999998	8178800000\\
-310000	-13061500000\\
-75000.0000000002	6225700000.00001\\
204000.000000003	1220400000\\
-112000.000000004	-2807300000\\
18999.9999999992	854400000.000002\\
-55000.0000000006	610100000.000002\\
54999.9999999988	-243700000\\
109999.999999999	1586500000\\
-73999.9999999989	-2929400000\\
-163000.000000001	-2319400000\\
456000	11352400000\\
-219000.000000001	-15014600000\\
-494999.999999998	8301100000\\
165000.000000001	-732999999.999999\\
605999.999999998	3540600000\\
-258999.999999999	-7080500000\\
-530000	1098800000\\
257000.000000001	4028500000\\
237999.999999998	-732799999.999998\\
-93000	-2319000000\\
147999.999999998	2929600000\\
-330000.000000001	-6592000000\\
310999.999999999	11474800000\\
92000.0000000014	-9643500000\\
-495000.000000002	-1220900000\\
549999.999999999	11230700000\\
-257000.000000001	-11108600000\\
-53999.9999999994	3906400000\\
90000.0000000016	854299999.999998\\
-364999.999999998	-2563300000\\
475999.999999999	5127000000\\
-330000.000000001	-7080300000\\
-35999.9999999996	5615300000\\
419999.999999997	610600000.000001\\
-35000.0000000001	-2930000000\\
-258000.000000001	-3539899999.99999\\
111000.000000002	8178600000\\
-165000.000000003	-7812200000\\
274000.000000002	7323800000\\
-182000	-5614900000\\
35000.0000000019	1708800000\\
57000.0000000004	1342800000\\
-165999.999999999	-2929500000\\
146000.000000003	4028000000\\
-237000.000000003	-5126700000\\
346999.999999999	6469600000\\
-16999.9999999986	-4150400000\\
-202999.999999999	-1220500000\\
128999.999999997	3173600000\\
-127999.999999999	-2807700000\\
-256000	-609899999.999998\\
364999.999999997	6591200000\\
330000.000000001	-1220100000\\
-366000.000000001	-10864800000\\
74000.0000000025	12207300000\\
-349000.000000003	-11108300000\\
329000.000000001	14282000000\\
203000.000000001	-9033100000\\
-458000	-4028199999.99999\\
273999.999999997	10253600000\\
-200999.999999999	-8422600000\\
365999.999999999	7568400000\\
-329999.999999997	-7690700000\\
-54000.0000000003	3174100000\\
109000.000000001	2075000000\\
201999.999999998	366299999.999999\\
-184000	-6347600000\\
-53000.0000000008	6469700000\\
-74999.9999999993	-4150500000\\
310999.999999999	6225600000\\
-309999.999999997	-9765400000\\
55000.0000000015	7812200000\\
383999.999999997	3662300000\\
-129000	-10864300000\\
-162999.999999998	2807600000\\
-313000.000000001	-99999.9999976353\\
256999.999999998	6714200000\\
293000.000000001	-2563999999.99999\\
-346999.999999998	-8056100000\\
290999.999999996	12572900000\\
2000.00000000244	-10131700000\\
-404999.999999999	732399999.999999\\
-181000.000000002	2563300000\\
622000.000000001	6592100000\\
-73999.9999999989	-9643700000\\
-383000.000000001	122000000.000005\\
345999.999999996	9399600000\\
-17999.9999999971	-9277600000\\
19999.9999999996	3784400000\\
-295000.000000003	-4150500000\\
93000.0000000035	5127100000\\
255999.999999997	2319100000\\
-202000	-8544800000\\
72999.9999999995	6225600000\\
56000.0000000027	-1098700000\\
-238000.000000001	-3295700000\\
273999.999999997	5737100000\\
-238000	-5737300000\\
-110000	1709000000\\
330000.000000001	4883000000\\
-53999.9999999976	-4638899999.99999\\
33999.9999999954	1220700000\\
-15999.9999999973	-1220500000\\
-19000.0000000019	-122300000.000001\\
-202000.000000002	-854300000.000001\\
56000.0000000018	3295800000\\
90999.9999999975	-1586900000\\
-73999.9999999989	244199999.999999\\
93000.0000000017	-122299999.999996\\
182000	3052100000\\
-36000.0000000014	-5249399999.99999\\
-567999.999999998	-5004400000\\
312000.000000001	12816900000\\
456999.999999997	854700000.000001\\
-438999.999999999	-15624800000\\
-19000.0000000019	9887200000\\
201999.999999998	4517100000\\
-36999.9999999999	-9888000000\\
-220000	3540200000\\
185000.000000001	4394500000\\
-94000.0000000021	-6225700000\\
20000.0000000005	3418000000\\
145999.999999998	3296100000\\
-147000	-9643800000\\
36999.9999999999	9521600000\\
-19000.0000000001	-5371100000.00001\\
130000.000000001	3173800000\\
-258000.000000001	-5493099999.99999\\
292999.999999999	9765500000\\
-181999.999999999	-10742100000\\
-91999.9999999996	5859400000\\
218999.999999998	732300000.000002\\
129000.000000001	-366000000.000002\\
-165000	-5127200000\\
-128000.000000002	4638900000\\
-201999.999999998	-1709200000\\
-109000.000000001	-121800000\\
547999.999999998	5004500000\\
55999.9999999991	-3539700000\\
-36999.9999999972	-488300000.000002\\
-311000.000000001	-5371400000\\
128000	10132200000\\
73999.9999999998	-5737500000\\
34999.9999999984	854399999.999999\\
-311000	-2441100000\\
165999.999999999	5493000000\\
128000.000000002	-3296100000\\
-166000.000000001	-732100000.000002\\
148000	2929600000\\
-93000.0000000008	-3296100000\\
147999.999999998	2441700000\\
-293999.999999998	-3418200000\\
-73999.9999999989	2807800000\\
532999.999999998	5126900000\\
-368000.000000001	-11841100000\\
93000.0000000008	9399900000\\
-93000.0000000017	-5005200000\\
130000.000000003	4638800000\\
-148000.000000003	-6469899999.99999\\
274000.000000001	10009900000\\
-235999.999999995	-12451000000\\
17999.9999999989	8910700000\\
309999.999999998	244500000.000004\\
-659000.000000002	-11718800000\\
661000.000000002	18798600000\\
-442000.000000002	-18310200000\\
166000	13183300000\\
-90999.9999999993	-6835700000\\
274000	5859100000\\
-293000.000000003	-10253800000\\
19000.0000000019	9399700000\\
272999.999999999	487800000.000001\\
-144000	-6835499999.99999\\
-39000.0000000024	2685200000\\
-90999.9999999975	-366100000\\
148000.000000001	4028599999.99999\\
-20000.000000004	-4395000000\\
-53999.9999999976	976799999.999996\\
-17999.9999999998	-732300000\\
34999.9999999984	3661900000\\
-255000.000000003	-7812500000\\
274000.000000003	10010000000\\
220000.000000001	-1098900000\\
-255999.999999998	-9277199999.99999\\
36999.999999999	6591800000\\
144999.999999997	2807500000\\
-201000	-10620000000\\
-34999.9999999993	10620100000\\
125999.999999999	-4638700000\\
-182000	488300000.000002\\
-19000.0000000001	-854600000.000002\\
148000.000000001	3784400000\\
52999.999999999	-2319500000\\
57000.0000000004	854500000.000002\\
33999.999999998	-2197200000\\
-472999.999999997	-5493300000\\
307999.999999998	15015000000\\
149000.000000004	-11108900000\\
-19000.0000000001	5615600000.00001\\
-383999.999999999	-12817400000\\
511000	25146200000\\
37999.9999999976	-22338600000\\
-347999.999999997	2685400000\\
-1000.00000000477	6836200000\\
-236999.999999998	-4761100000\\
311999.999999999	5981600000\\
-1999.99999999978	-4516500000\\
220999.999999997	5248900000\\
-146999.999999999	-9033199999.99999\\
36999.9999999981	6347800000\\
-220000	-5859700000\\
-55000.0000000006	6958300000\\
202000.000000001	-2929600000\\
72999.9999999995	2196800000\\
-201000.000000002	-5858800000\\
90000.0000000016	6713400000\\
202999.999999998	-3173600000\\
-586999.999999996	-3295900000\\
439999.999999996	7080100000\\
-36999.9999999981	-4394700000\\
38000.0000000002	2929700000\\
-20000.0000000022	-5126600000\\
-73000.0000000004	4150000000\\
-16999.9999999959	-2197199999.99999\\
217999.999999995	5371199999.99999\\
-328999.999999997	-10986200000\\
129000	10497600000\\
237000.000000001	-365700000.000003\\
-273999.999999997	-9888100000.00001\\
109999.999999999	10498400000\\
-111000.000000003	-6592100000\\
-34999.9999999957	2441500000\\
291999.999999997	5615400000\\
-111000	-9765900000\\
-180999.999999997	2197400000\\
127999.999999999	5737500000\\
126999.999999998	-3906600000\\
-403000.000000001	-4516400000\\
147999.999999999	7690300000\\
291000.000000002	976800000.000002\\
-218000.000000001	-8423100000\\
-164999.999999996	2441500000\\
529999.999999997	13916100000\\
-273999.999999996	-22949300000\\
-73000.0000000004	14038000000\\
-128000.000000003	-6347600000\\
17000.0000000003	5005099999.99999\\
404000	6225200000\\
-294000.000000001	-16113000000\\
-199999.999999999	6713800000\\
457000	10742200000\\
-72999.9999999986	-15380900000\\
-440000.000000003	3906200000\\
366000	7080200000\\
-35000.0000000001	-6591900000\\
-184999.999999999	1098700000\\
165999.999999998	1953200000\\
-18999.9999999983	-1709200000\\
-146000.000000003	-244100000.000001\\
37000.0000000026	854800000\\
126999.999999999	1464400000\\
-91000.0000000019	-3417600000\\
202000	4638500000.00001\\
-201999.999999998	-6103600000.00001\\
17999.9999999962	5127100000\\
128000.000000003	-732500000\\
75000.000000001	-976400000.000008\\
-313000.000000003	-4272700000\\
183000.000000001	7568500000\\
-71999.9999999965	-5005000000\\
-128000.000000001	244399999.999999\\
400999.999999997	7323900000\\
-328000.000000002	-12206800000\\
109000.000000001	10009600000\\
17999.9999999998	-5615100000\\
-218000.000000001	-1098700000\\
345999.999999999	10620200000\\
-163999.999999998	-13428000000\\
1000.00000000033	7568800000\\
180999.999999997	-244600000\\
-145000.000000002	-4760400000\\
-91999.9999999996	3051700000\\
-274000	-854800000\\
455999.999999999	5859800000\\
-180999.999999998	-10010000000\\
144999.999999998	10009800000\\
-162999.999999998	-9155300000\\
-186000.000000001	3174000000\\
350999.999999999	6103200000\\
199000	-2807299999.99999\\
-347000.000000001	-9155400000\\
55000.0000000024	10742100000\\
129000	-3540000000\\
-239000.000000003	-4150200000\\
17999.999999998	5981200000\\
202000.000000003	366400000.000002\\
-202000.000000003	-6470000000\\
19000.000000001	5371400000\\
-17999.9999999989	-2563600000\\
254999.999999998	8056600000\\
-108999.999999998	-11840800000\\
-108999.999999999	2807700000\\
236000.000000001	9033100000.00001\\
-327999.999999998	-15014600000\\
238000	14038100000\\
-129000.000000004	-10376100000\\
-127999.999999999	5615399999.99999\\
201999.999999997	488199999.999999\\
365000.000000001	5127000000\\
-620999.999999999	-20385800000\\
126999.999999998	21728600000\\
1000.00000000122	-11596800000\\
108999.999999997	6958000000\\
202000.000000002	-2441300000\\
-403000.000000001	-6347499999.99999\\
-56000.0000000009	5614700000\\
148000.000000001	1343400000\\
293000.000000005	4760399999.99999\\
-1000.00000000122	-10742300000\\
-165000.000000003	5127299999.99998\\
-162999.999999996	-6225799999.99998\\
52999.9999999973	8911200000\\
257000.000000003	2685400000\\
-201000	-15136600000\\
-73999.9999999989	12329200000\\
276000.000000001	854399999.999998\\
-203000.000000002	-10254000000\\
-164000	5127100000.00001\\
348000.000000002	8422700000\\
-74000.0000000016	-12695100000\\
-345999.999999997	3539800000\\
399999.999999995	7202400000\\
-217999.999999999	-10376300000\\
37000.0000000008	7568700000\\
16999.9999999986	-4028500000\\
-89999.9999999963	854500000.000001\\
309999.999999998	5126900000\\
-146000.000000003	-7690200000\\
-183999.999999997	-366500000.000003\\
94000.0000000021	6836100000\\
-3000.00000000455	-4028400000\\
167000.000000002	3174000000\\
-75000.000000001	-5127100000\\
-291000	-854499999.999999\\
345999.999999999	9887600000\\
56000.0000000009	-6835600000\\
-439000.000000001	-8057000000\\
401000	18555000000\\
2000.00000000244	-12817800000\\
-56000.0000000018	855000000.000003\\
-91000.0000000002	1098200000\\
35999.9999999978	2075500000\\
-384000	-5249300000\\
475999.999999998	10498400000\\
91000.0000000002	-8423300000\\
-164000.000000001	244599999.999998\\
-184000.000000002	610100000.000003\\
-71999.9999999983	1098500000\\
291999.999999998	2319800000\\
-91999.9999999996	-4028900000\\
38000.000000002	2197700000\\
109000	854300000.000001\\
-183000	-4028299999.99999\\
274999.999999999	7324400000\\
-1000.00000000122	-8667400000\\
-566000.000000001	-243700000.000004\\
310000	10009500000\\
-8.88178419700125e-10	-8422700000\\
-164000	1952900000\\
348000.000000001	3906500000\\
-294000.000000001	-7568400000\\
457999.999999998	12817200000\\
-273999.999999998	-16845600000\\
-312000.000000002	7324400000\\
513000.000000002	9032900000\\
-146000.000000002	-12695200000\\
-92000.0000000005	3296100000\\
-202000.000000002	365899999.999998\\
111000.000000003	3051900000\\
17999.9999999954	-3417900000\\
347000.000000001	8422700000\\
-237000.000000003	-14526200000\\
-91999.9999999978	8544700000\\
-55000.0000000041	-1586600000\\
37000.0000000017	732000000.000002\\
-55999.9999999991	-1342300000\\
331000.000000002	8788700000\\
-128000.000000001	-14648200000\\
70999.9999999988	11352300000\\
-70999.9999999971	-8422700000\\
-293000	1831100000\\
510999.999999997	8911100000\\
-585000	-14404500000\\
238999.999999998	14648800000\\
108000	-12573600000\\
-310000	7446600000\\
255999.999999998	-732600000.000003\\
73000.0000000031	122000000.000002\\
-90999.9999999984	-4516400000\\
183000	10009700000\\
145999.999999998	-8423000000.00001\\
-476000.000000002	-5859200000\\
91999.9999999987	12695300000\\
-36999.9999999999	-8056800000\\
54999.9999999988	4394700000\\
-17999.999999998	-1098600000\\
347000	5981200000\\
-199999.999999999	-13305400000\\
-257000.000000001	5248900000\\
91000.0000000019	3173800000\\
293999.999999997	4394599999.99999\\
-129999.999999999	-10864300000\\
-88999.9999999986	4150500000\\
52999.9999999964	2563300000\\
-91000.0000000002	-2807600000\\
54999.9999999988	1831200000\\
-219000	-2197300000\\
127000	3051700000\\
182999.999999999	-1342800000\\
73999.9999999998	1342800000\\
-182999.999999998	-4394500000\\
219999.999999998	5004900000\\
-129999.999999997	-4516700000\\
-346000.000000001	1953200000\\
-1000.00000000211	976699999.999999\\
239000.000000001	243800000.000001\\
108999.999999996	-854300000\\
2.66453525910038e-09	488500000\\
-91000.0000000028	-1831500000\\
-55000.0000000006	1587200000\\
-18999.9999999992	-1587000000\\
292999.999999998	6836100000\\
-163999.999999995	-10132100000\\
17999.9999999998	8056800000.00001\\
201000.000000001	-2319300000.00001\\
-347000	-9277499999.99999\\
-74000.0000000007	12085000000\\
73999.9999999989	-4638500000\\
199999.999999999	4760600000\\
38000.0000000002	-5371100000\\
-202000.000000001	-976500000.000001\\
108999.999999998	3051600000\\
-345999.999999997	-3051500000\\
384000	7079900000\\
144999.999999996	-4272400000\\
-217999.999999998	-5004900000\\
-311999.999999999	4516500000\\
219000	2075400000\\
257999.999999999	121900000.000003\\
-313000.000000001	-6957900000\\
275999.999999999	9521500000\\
-311999.999999999	-9887800000\\
19000.000000001	6957900000\\
419999.999999997	3906700000\\
-401999.999999997	-14526900000\\
2.66453525910038e-09	11841100000\\
238000	-122000000.000007\\
73000.0000000013	-854699999.999993\\
8.88178419700125e-10	-5737200000\\
-238000.000000002	1831000000\\
147000.000000001	3174000000\\
-385000.000000003	-4150600000\\
365000.000000002	7934700000\\
-272000	-10742300000\\
290999.999999999	11840900000\\
73999.9999999981	-8300800000\\
-347999.999999998	-1220600000\\
401999.999999997	9643300000\\
-400999.999999999	-13915800000\\
272000	14160200000\\
75999.9999999996	-9399600000\\
-366999.999999999	976600000.000003\\
273000.000000001	4760900000\\
-145000.000000004	-4883100000\\
92000.0000000014	3296300000\\
54000.0000000003	-1465300000\\
-182999.999999997	-1464400000\\
17999.9999999989	2441000000\\
148000.000000001	610700000\\
-19999.9999999987	-2685800000\\
-218999.999999998	-610300000\\
163999.999999999	4761000000\\
238999.999999998	-1831500000\\
-201000	-4027900000\\
-239000.000000002	854100000\\
201999.999999998	6226000000\\
255000.000000002	-2197600000\\
-273000.000000001	-6957900000.00001\\
311000.000000001	10620200000\\
-221000.000000004	-11840900000\\
-546999.999999999	5249000000\\
419000	4638800000\\
219999.999999998	-2197400000\\
-237000	-5493200000.00001\\
219000.000000003	9887800000\\
54999.9999999988	-8178600000\\
-109000	732199999.999999\\
-240000.000000002	-244100000\\
112000.000000002	4028300000\\
126999.999999999	-732099999.999997\\
91999.9999999987	-2686000000\\
-348999.999999997	-1098500000\\
167000	5981700000\\
89000.0000000013	-5981700000\\
-35000.0000000028	3540000000\\
-255999.999999999	-3906000000\\
90000.0000000016	5126800000\\
293999.999999995	-1465000000\\
-17999.9999999962	-610099999.999999\\
-404000	-6103500000\\
146999.999999998	10375700000\\
495000.000000002	122399999.999999\\
-550000.000000002	-15014900000\\
109999.999999999	16601600000\\
257000.000000001	-5126800000\\
-56000	-3540100000\\
-384000.000000002	-732600000.000001\\
165000	5249300000\\
310000	2075000000\\
-364000.000000001	-11230400000\\
162999.999999998	12085000000\\
93000.0000000026	-7080200000\\
-239999.999999998	1464900000\\
-71000.0000000024	-488100000\\
126000.000000002	3051400000\\
184999.999999995	-1342500000\\
-147999.999999999	-3051800000\\
-35000.0000000001	3784100000\\
200000.000000001	-244100000.000002\\
37000.0000000017	-2685600000\\
-292999.999999997	366400000.000001\\
-37000.0000000017	1220600000\\
111999.999999999	854300000\\
33999.9999999998	-1342600000\\
-291999.999999997	-1831000000\\
493999.999999996	7812500000\\
1000.00000000211	-7934700000\\
-257000.000000003	610299999.999998\\
-54999.9999999997	610700000.000001\\
37000.0000000017	3051300000\\
146000	-854099999.999996\\
-35999.9999999978	-1709200000\\
163999.999999998	2807600000.00001\\
-310000.000000001	-7690300000\\
53000.0000000008	8544800000\\
93000	-3295800000\\
-91999.9999999996	-488300000.000002\\
-54999.9999999997	-732400000.000001\\
239000.000000001	8300600000\\
-2000.00000000244	-11962700000\\
-144999.999999995	6103500000\\
-8.88178419700125e-10	-2685500000\\
-17999.9999999998	2929400000\\
-20000.0000000022	-2929400000\\
258000	7812400000\\
-331000	-14648300000\\
-15999.9999999982	12817200000\\
161999.999999998	-4638800000\\
148000.000000001	3662600000\\
-256000.000000001	-9155800000\\
164000.000000001	11230900000\\
-37000.0000000026	-9033699999.99999\\
57000.0000000039	6226200000\\
-351000.000000003	-8057200000\\
94000.0000000021	8789300000.00001\\
438999.999999998	1953299999.99999\\
-129000.000000001	-8789300000.00001\\
-292000	-1098700000\\
309999.999999998	11474900000\\
-199999.999999998	-12085200000\\
-202000	6103500000\\
384000.000000001	2563800000\\
-109000	-6592300000\\
-183999.999999998	3662600000\\
293999.999999999	1342500000\\
-348999.999999999	-5615300000\\
293999.999999999	8301100000\\
72999.9999999995	-6226000000\\
-403000	-854100000.000002\\
238000	5615000000\\
-36999.999999999	-4028300000\\
128999.999999999	1709100000\\
-92999.9999999991	-1587100000\\
-90000.0000000016	-243899999.999997\\
200999.999999998	4760500000\\
-146000	-7568200000\\
-239999.999999999	2563400000\\
607000.000000002	10742200000\\
-259000.000000002	-18310500000\\
-198999.999999999	10742200000\\
272999.999999997	488099999.999999\\
1.77635683940025e-09	-3783900000\\
-494000.000000001	-2075400000\\
439999.999999999	9399400000\\
110000.000000001	-8544700000\\
-513999.999999998	243900000.000001\\
384999.999999997	6836100000\\
-127000	-7202300000\\
273000	6347800000\\
-146000.000000002	-6714000000\\
-126999.999999997	3296100000\\
-74999.9999999993	-732699999.999998\\
-35000.000000001	976799999.999999\\
236000	1464700000\\
-71000.0000000015	-3417800000\\
-93000.0000000026	1342500000\\
56000	610599999.999999\\
255000.000000001	3662000000\\
-310000	-12084900000\\
-1000.00000000211	14038000000\\
276000.000000002	-5615100000\\
-185000.000000003	-3174100000\\
-254999.999999999	-976199999.999997\\
529999.999999999	16235100000\\
-345999.999999997	-26001000000\\
88999.9999999977	22949500000\\
-15999.9999999973	-15503200000\\
-73999.9999999989	8544900000\\
346999.999999995	3174100000\\
-163999.999999999	-13183800000\\
-220000.000000001	7690400000.00001\\
-165000	-1342600000\\
221000.000000002	5859200000\\
180999.999999999	-5737300000\\
21000.0000000026	3052000000\\
-58000.0000000043	-4883099999.99999\\
-271999.999999999	1098700000\\
-111000.000000001	1587100000\\
310999.999999999	3906000000\\
146000.000000003	-2685400000\\
130000.000000001	366200000.000004\\
-367000.000000003	-7080200000\\
-147999.999999999	5615400000\\
75999.9999999996	854500000\\
254000	3295700000\\
-36000.0000000014	-5859300000\\
74000.0000000007	4150600000.00001\\
-366999.999999998	-10620400000\\
238999.999999997	15869300000\\
53000.0000000008	-7934600000\\
-198999.999999998	-5859400000\\
253999.999999999	16235300000\\
-15999.9999999973	-15380600000\\
-258000.000000003	1586500000\\
239000.000000003	10986600000\\
145000	-8544799999.99999\\
-364000	-4517000000\\
-75000.0000000028	8056900000\\
165000	-976400000.000001\\
111000.000000003	243700000.000003\\
35999.9999999969	-609999999.999999\\
-239000	-6469800000\\
184000.000000001	11840700000\\
36999.9999999981	-8544700000\\
-256999.999999998	-1098900000\\
219999.999999999	8789200000\\
-201000	-11596600000\\
236999.999999998	13305500000\\
73999.9999999981	-7812400000\\
-146999.999999999	-2807700000\\
109999.999999999	5249300000\\
-439000.000000001	-6226100000\\
421000.000000002	11963500000\\
-73000.0000000031	-12695900000\\
-73999.9999999998	7935000000\\
-146999.999999999	-6714100000\\
167000	8056800000\\
70999.9999999997	-4272600000\\
38000.0000000038	122199999.999997\\
36999.9999999999	1830900000.00001\\
-75000.0000000037	-5615100000\\
-91000.0000000002	5493100000.00001\\
-71999.9999999992	-3173800000\\
-111000.000000001	2441300000\\
348000	1098900000\\
-73000.0000000004	-2563800000\\
-275000	-3173600000\\
275000	8666800000.00001\\
37000.0000000008	-6469400000\\
-73999.9999999989	2441000000\\
-54000.000000002	-3417699999.99999\\
291000	10131700000\\
-253999.999999998	-16967600000\\
-74000.0000000007	13061300000\\
16999.9999999968	-5371000000\\
-70999.9999999979	1831200000\\
254000	3784000000\\
-16000	-5981500000\\
-167000.000000002	488499999.999999\\
-235999.999999997	-366299999.999998\\
364999.999999997	6591700000\\
73000.0000000004	-4394500000\\
-401999.999999998	-6469600000\\
365999.999999999	13183400000\\
128000	-5004600000\\
-147000.000000002	-7690700000\\
-108999.999999998	6469700000\\
-55000.0000000006	-1586500000\\
-220000.000000002	-1099200000\\
475999.999999999	9521800000\\
-183000.000000001	-14648300000\\
-73000.0000000013	11230100000\\
72999.9999999995	-6103200000\\
72999.9999999995	3539900000\\
-110000	-3784300000\\
184000	6836200000\\
-239000.000000002	-11230600000\\
147000.000000002	12573200000\\
-18000.0000000016	-10742200000\\
-292999.999999999	4272700000\\
236999.999999998	3173500000\\
74000.0000000016	-1708800000\\
237999.999999999	2441400000.00001\\
-750000	-17212000000\\
657000	29419000000\\
57000.0000000022	-19897400000\\
-146999.999999999	1464800000\\
-458000.000000001	-1465000000\\
385000.000000002	12451400000\\
35999.9999999978	-12695300000\\
-165000.000000002	5126700000\\
221000.000000004	2685700000\\
-258000.000000004	-9643500000\\
-144999.999999999	7446200000\\
457000	5127100000\\
91000.0000000011	-5859600000\\
-382999.999999999	-8788999999.99999\\
-1000.00000000122	13305800000\\
182999.999999997	-4882900000\\
18000.0000000007	1953100000\\
-107999.999999998	-6958000000\\
70999.9999999971	9765700000\\
-237000.000000001	-11474700000\\
164999.999999998	13794000000\\
-146999.999999999	-13794000000\\
255999.999999999	13916000000\\
-144999.999999997	-13427600000\\
309999.999999998	18066300000\\
-164999.999999998	-22949200000\\
-89999.9999999998	13305700000\\
-130000.000000003	-4394699999.99999\\
221000.000000001	7324500000\\
16999.9999999986	-8423000000\\
-253999.999999996	2563300000\\
88999.9999999942	1953500000\\
148000.000000002	-200000.000001665\\
-19000.0000000019	-2685700000\\
-329000	-488000000\\
201000	4882600000\\
256000	-976400000.000001\\
-109000	-5249100000\\
-183999.999999997	3295900000\\
-256000.000000002	-2563500000\\
366999.999999998	6958000000\\
475000.000000001	488299999.999997\\
-714000.000000003	-15380700000\\
166000.000000001	15136400000\\
108000.000000001	-3906000000\\
-164000.000000002	-3051800000\\
221000.000000004	5493000000\\
15999.9999999965	-4028000000.00001\\
-125999.999999998	-732799999.999997\\
-19999.9999999978	2075500000\\
-146000	-1465100000\\
-36000.000000004	610699999.999999\\
274000.000000002	3539700000\\
184000.000000001	-2197200000\\
-184000.000000004	-4394300000\\
-36999.9999999972	4516300000\\
-237000.000000002	-3051600000\\
-19000.0000000001	1953200000\\
276000	2929400000\\
53000.0000000008	-3661800000\\
36999.9999999981	2563400000\\
-182000	-6103700000\\
-36999.999999999	6103800000\\
-56000.0000000018	-3296200000\\
57000.0000000013	2807900000\\
125999.999999999	-610500000.000002\\
165999.999999997	2319300000.00001\\
-55999.9999999983	-5248900000\\
-308999.999999997	-2807700000\\
70999.9999999953	9399400000\\
91999.9999999996	-5737200000\\
92000.0000000005	4150200000\\
-220000.000000001	-8300600000\\
56000.0000000018	10131800000\\
89999.9999999972	-5981500000\\
73999.9999999998	3173900000\\
-999.999999999446	-4883000000\\
-364000.000000001	2319700000\\
144000	3173400000\\
19999.9999999987	-4394300000\\
219000.000000003	5371200000\\
-145999.999999998	-6347899999.99999\\
127999.999999997	5981699999.99999\\
-91999.9999999978	-6714200000\\
-238000	3296200000\\
184000.000000001	1586700000.00001\\
292000.000000002	5737599999.99999\\
-54999.9999999988	-11963300000\\
-365000.000000001	-1220300000\\
34999.9999999993	11108100000\\
93000.0000000044	-6713699999.99999\\
291999.999999996	12084800000\\
19000.0000000037	-19531000000\\
-403000.000000003	7446100000\\
-37000.0000000026	3173800000\\
201000	1098800000\\
38000.000000002	-3540200000\\
-238999.999999999	100000.000000477\\
-36999.9999999999	1831100000\\
185000	-366499999.999999\\
107000	854900000.000001\\
75999.9999999978	-1465100000\\
-220999.999999999	-1220600000\\
-36999.9999999964	854300000.000003\\
54999.9999999979	1953400000\\
73999.9999999989	-976699999.999999\\
17999.999999998	-122100000\\
55000.0000000006	100000.000002609\\
-74000.0000000025	-1587000000\\
-107999.999999998	976500000\\
-93999.9999999976	-243899999.999999\\
129999.999999996	2197000000\\
236999.999999999	244300000.000003\\
-126999.999999997	-4272500000\\
-330999.999999999	488200000.000001\\
422000	6836200000\\
-201999.999999999	-9644000000\\
-72000.0000000009	7813000000\\
-1999.999999998	-5493600000\\
221999.999999998	5737700000\\
88999.9999999995	-3174300000\\
-417999.999999998	-6225000000\\
252999.999999998	13793400000\\
113000.000000001	-11230200000\\
-149000.000000001	3906200000\\
-219000.000000001	-4150200000\\
184000.000000003	8422400000\\
182999.999999998	-4272000000\\
-184999.999999999	-2807800000\\
186000	6835800000\\
-167000.000000001	-9887500000\\
-2.66453525910038e-09	7446200000\\
-126999.999999996	-4272400000\\
329999.999999998	8911200000\\
-350000.000000001	-14892900000\\
221000.000000004	15259200000\\
-17999.9999999989	-10742600000\\
-385000	-854000000.000002\\
678000.000000002	20019100000\\
-1000.00000000033	-24413900000\\
-365000	5615200000.00001\\
-184000.000000004	3906400000\\
110000	1098500000\\
36999.9999999981	-1587000000\\
109000.000000001	1220800000\\
1000.00000000389	-1586799999.99999\\
-36999.9999999999	610099999.999998\\
220000	2563800000\\
-293000.000000002	-8545300000\\
-56000	6714200000\\
148000	1220600000\\
181999.999999998	-244399999.999993\\
-183000	-6469200000\\
-364999.999999999	3661500000\\
437999.999999998	5371700000\\
-164000.000000001	-8057200000\\
72999.9999999995	5615700000\\
146000	-400000.000006884\\
-35999.9999999987	-3295500000\\
-110000	-1343100000\\
-238000	976700000.000001\\
202000.000000002	5127000000\\
-1000.00000000122	-5371300000\\
91999.9999999996	5005200000\\
-18999.9999999983	-6225899999.99999\\
93000	8667299999.99999\\
-130000.000000001	-12329400000\\
222000.000000002	12695400000\\
-222000.000000005	-12451000000\\
-420000	4516500000\\
403000.000000001	7446100000\\
366000.000000002	-121800000\\
-623000.000000001	-18921000000\\
183999.999999998	21240200000\\
127000.000000002	-7568300000\\
275000	7202100000\\
-163999.999999998	-16967700000\\
-92000.0000000014	10009600000\\
-403000	-2563300000\\
329999.999999999	8056500000\\
17999.9999999989	-9155100000\\
200999.999999999	6957900000\\
-383999.999999999	-9765600000\\
256000	12451100000\\
-183000.000000002	-11840800000\\
166000	10376100000\\
-112000	-8667200000\\
-127000	3906600000\\
385000	4027900000\\
-330000.000000001	-9521300000\\
108999.999999998	9033300000\\
147000.000000001	-3540200000\\
129000	3051900000\\
-386000	-12207100000\\
-35000.0000000001	13671700000\\
329000.000000001	-1830600000\\
-166000.000000004	-7080600000\\
38000.0000000038	6714200000\\
-238000.000000001	-6469800000\\
255999.999999998	9521400000\\
163999.999999998	-6469600000\\
-181999.999999999	-1220900000\\
-147000	2197600000\\
-164000.000000001	-366500000\\
219000	1587000000\\
182000.000000001	-366200000\\
75999.9999999987	244099999.999998\\
-350999.999999998	-5248900000\\
20999.9999999981	5370900000\\
89999.999999999	100000.000000477\\
274999.999999999	1464900000\\
-219999.999999999	-6713900000\\
-146000.000000001	3906200000\\
109999.999999999	1098600000\\
91000.0000000011	854700000\\
-110000.000000001	-5127300000\\
56000.0000000009	6714299999.99999\\
126999.999999999	-4150800000\\
-164000.000000001	-854200000.000002\\
-146999.999999998	121800000\\
237999.999999998	4883200000\\
-145999.999999999	-5737800000\\
17999.9999999989	2686100000\\
-90999.9999999984	-1831600000\\
438999.999999999	10132200000\\
-275000	-18554700000\\
-257000.000000001	10253600000\\
478000.000000004	8179100000\\
-92000.0000000005	-14282500000\\
-184000.000000002	4638800000\\
35999.9999999987	1586900000\\
-126000	-366300000\\
162999.999999998	1342900000\\
-53999.9999999976	-3173800000\\
-275000.000000004	121800000.000002\\
733000.000000003	9155600000\\
-478000	-15381100000\\
-254000	9887900000\\
273999.999999996	-1831300000\\
200000.000000002	2319700000\\
-108000.000000001	-5127400000\\
-1000.00000000122	3418300000\\
-310999.999999997	-3906400000\\
0	4638800000\\
311999.999999998	121899999.999999\\
181000	488299999.999997\\
-198999.999999997	-5370800000\\
-167000.000000003	1464400000\\
112000.000000002	4761000000\\
-19000.0000000019	-5249000000\\
-55999.9999999991	3051700000\\
55999.9999999983	-732500000.000004\\
127000.000000001	1464900000\\
-70999.9999999979	-2197100000\\
106999.999999998	732099999.999998\\
-180999.999999997	-2197100000\\
-55000.0000000015	1953300000\\
-56000.0000000009	-299999.99999859\\
404000.000000002	5249100000\\
-313000.000000003	-10986100000\\
-218000	5492900000\\
311000	4028300000\\
89999.999999999	-3539700000\\
-180999.999999998	-2685999999.99999\\
-19000.0000000001	2563799999.99999\\
474999.999999998	8300800000.00001\\
-823000	-23437900000\\
458000.000000001	25879400000\\
127999.999999999	-12451400000\\
-239000.000000001	-488300000.000001\\
93000	4272500000\\
-313000.000000001	-8667100000\\
478000.000000003	18799100000\\
109000.000000001	-15381100000\\
-458000	-6347599999.99999\\
37000.0000000008	15869200000\\
238000	-4272600000\\
-166000.000000001	-6225300000\\
149000.000000002	9276899999.99999\\
-259000.000000002	-13793500000\\
368000.000000002	22094400000\\
-91999.9999999987	-21239900000\\
17999.9999999989	7079600000.00001\\
-311000.000000002	-2318800000\\
182999.999999997	8178300000\\
18000.0000000007	-7202000000\\
-238000	244299999.999998\\
92000.0000000014	3661800000\\
495000	4394700000\\
-385999.999999999	-14404200000\\
-72000.0000000018	10131600000\\
199999.999999999	244300000\\
-217999.999999998	-6225600000\\
327999.999999995	8911100000\\
-401999.999999997	-9399400000\\
164999.999999997	7690500000\\
-127999.999999999	-5737500000\\
292000.000000001	5737500000\\
-90999.9999999984	-5737400000\\
-18999.9999999983	4150400000\\
-54000.0000000011	-3051700000\\
-220000	1220600000\\
402999.999999999	2929800000\\
-219999.999999999	-5371100000\\
-219000.000000001	1830900000\\
273000	3418100000\\
183999.999999999	-2075100000\\
18999.9999999992	-244499999.999998\\
-348000.000000002	-3783700000\\
-111000.000000001	3417600000\\
733000.000000001	8911300000\\
-384000.000000002	-17211900000\\
-183999.999999995	8544800000\\
73000.0000000004	2075200000\\
-218000.000000003	-4272200000\\
90000.0000000025	3173400000\\
110999.999999997	-854100000\\
384000	3784000000\\
-569000	-12207000000\\
94000.0000000003	13061400000\\
310000	-3906000000\\
-37000.0000000026	-122300000.000003\\
37000.0000000026	-2563300000\\
-439000.000000001	-2685700000\\
92000.0000000005	7812600000\\
162999.999999997	-3418000000\\
367000.000000001	6103499999.99999\\
-273999.999999999	-13671900000\\
-458000.000000003	6225700000\\
476000.000000001	6713700000\\
146000	-7324000000\\
-274999.999999998	976400000\\
-420000	-122000000\\
127999.999999999	976399999.999999\\
641000	5249200000\\
-93000.0000000008	-6469700000\\
-108000.000000001	-732599999.999997\\
-276000.000000001	610599999.999996\\
476999.999999998	5126700000\\
-713999.999999998	-7690400000\\
328000	6225900000\\
-34999.9999999984	-2930100000\\
16999.9999999995	1220900000\\
-17000.0000000021	-1953000000\\
238000	5736900000\\
-38000.0000000011	-7934000000\\
-126999.999999999	5492600000\\
54999.9999999988	-2807300000\\
-19000.0000000019	1586900000.00001\\
19000.0000000001	-610400000\\
-129000.000000002	-2685600000\\
-164999.999999996	2563600000\\
496000	6957900000\\
146000.000000002	-7202100000\\
-295000.000000004	-6591900000\\
-748000	4516900000\\
731000	9276900000\\
146999.999999999	-6225100000\\
-19000.000000001	-855000000.000001\\
-348000	-4516200000\\
-164000	5126900000\\
238000.000000001	1220400000\\
257000.000000001	2441800000\\
-167000.000000002	-8545300000\\
56999.9999999995	8301200000\\
35000.0000000028	-4272799999.99999\\
-163000	-4638500000.00001\\
346000	12939400000\\
-400999.999999999	-16235400000\\
-202000.000000003	10620100000\\
291999.999999999	-488000000.000001\\
92000.0000000014	-1221100000\\
-71999.9999999992	-1098399999.99999\\
162999.999999998	3662000000\\
-347000.000000001	-10864100000\\
274000.000000003	15869000000\\
92999.9999999982	-10864200000\\
-148000.000000001	1708900000\\
-127000.000000001	976699999.999997\\
-146999.999999999	-366199999.999998\\
366000	2929400000\\
18000.0000000007	-3173500000\\
-493000	-1831200000\\
182000.000000001	3417900000\\
494000	4394900000\\
-256000.000000003	-10986900000\\
-255000	6958400000\\
-20000.0000000013	-3051700000\\
532000.000000001	9154800000\\
-73999.9999999998	-12084300000\\
-274999.999999998	3295300000\\
-493000	-487900000\\
328999.999999996	3173700000\\
348000.000000001	2929600000\\
-238999.999999999	-8178500000\\
203000.000000002	7446100000\\
16999.9999999986	-2929599999.99998\\
92000.0000000032	3417999999.99999\\
-348000.000000002	-16479700000\\
-90999.9999999966	16968100000\\
347000.000000001	2685300000\\
-109000.000000001	-13305600000\\
-146999.999999998	5371200000.00001\\
109999.999999999	3539800000\\
37000.0000000035	-2441100000\\
-202000.000000001	-5249300000\\
476000.000000001	19287200000\\
-146000	-24658000000\\
-92000.0000000014	8422500000\\
-181999.999999999	1953400000\\
-112000.000000002	1342600000\\
222000	244299999.999996\\
-20000.0000000013	-2319500000\\
-236999.999999999	-1708900000\\
548999.999999998	8911200000\\
-348000	-11963100000\\
-55000.0000000006	5737600000\\
422000.000000001	5737100000\\
-478000.000000002	-14160200000\\
-16000	10620300000\\
235999.999999999	-732599999.999998\\
203000.000000002	1343000000\\
-202000.000000002	-7934899999.99999\\
-312000.000000001	3662500000\\
312000	4638300000\\
-494000.000000001	-8178500000\\
567000.000000003	8911100000\\
129000	-488200000\\
-295000.000000003	-9765899999.99999\\
3000.00000000455	8301099999.99999\\
-76000.0000000014	-4882900000\\
294000	11840600000\\
92000.0000000023	-13183200000\\
-238000.000000002	-1099100000.00001\\
54000.0000000003	7446700000\\
-347000.000000004	-4883000000\\
128000.000000002	5737300000\\
53999.9999999976	-4760700000\\
258000	5493100000\\
-167000	-8056499999.99999\\
-161999.999999998	5737200000\\
107999.999999997	-3051800000\\
55000.0000000015	3540100000\\
-73000.0000000022	-3540000000\\
238000.000000001	5126900000\\
36999.9999999972	-5737300000\\
-274999.999999998	-854500000\\
-18000.0000000016	4760700000\\
-36999.9999999964	-1708800000\\
18000.0000000007	-854700000\\
-54000.0000000011	854500000.000001\\
237999.999999999	1953300000\\
-74999.9999999975	-4150400000\\
-199000.000000003	1952800000\\
475000	4272800000\\
-514000	-11718900000\\
39000.0000000041	10498200000\\
548000	5126800000\\
-292999.999999998	-16723700000\\
-219000	8301000000\\
109000.000000001	2807400000\\
330000.000000003	2930000000\\
-384000.000000003	-14038600000\\
255000.000000001	13428300000\\
-365000.000000001	-9643800000\\
-164999.999999999	5004600000\\
256000	1465400000\\
275000.000000002	2929300000\\
126999.999999997	-1220700000\\
-235999.999999996	-11108200000\\
16999.9999999995	10498000000\\
-184000.000000001	-7446500000.00001\\
-51999.9999999987	6225900000\\
472999.999999999	8544700000\\
-328000.000000001	-21850500000\\
-54999.9999999997	15991100000\\
-238000	-9033000000\\
419999.999999998	13305500000\\
276000	-8300700000\\
-567999.999999999	-7568500000\\
200999.999999997	12085300000\\
-274000.000000001	-8911600000\\
127000.000000002	7080400000\\
332000	-244100000\\
-58000.0000000007	-1831300000\\
-89000.0000000013	-4150300000\\
-185000	2075300000\\
183999.999999997	4760600000\\
37000.0000000017	-4638500000\\
-129000.000000002	121900000\\
36999.999999999	1709000000\\
-8.88178419700125e-10	-488099999.999999\\
-91000.0000000002	-122300000\\
-147000.000000001	-1708900000\\
219000	3662200000\\
166000	-366299999.999999\\
71999.9999999974	-976500000.000003\\
-272999.999999999	-5005100000\\
52999.999999999	6714200000\\
-126000	-3784400000\\
34000.0000000007	2075200000\\
168000	854600000.000002\\
-21000.0000000008	-1342800000\\
109999.999999999	244000000.000006\\
-566000.000000001	-6225300000\\
384000.000000002	10375600000\\
347000	488699999.999997\\
18999.9999999983	-8911500000.00001\\
-640000	-1220600000\\
-38999.9999999979	8178900000\\
495999.999999998	-244400000\\
-108999.999999999	-4272200000\\
89999.999999999	1708700000\\
-457999.999999998	-3661900000\\
220999.999999998	5126800000\\
456999.999999998	4761000000\\
-145999.999999996	-11841200000\\
-476000.000000003	1221100000\\
-202999.999999996	7323900000\\
624999.999999998	-1098500000\\
199999.999999998	854499999.999996\\
-403000.000000002	-8911000000\\
-310999.999999997	4272100000\\
293999.999999999	5981700000\\
345999.999999999	-1342700000\\
-474999.999999998	-10498200000\\
128000.000000001	11962900000\\
385000.000000002	2441499999.99999\\
-275000.000000001	-17456300000\\
-348000.000000001	9521900000\\
531000	11718400000\\
-403000.000000002	-21362200000\\
92000.0000000023	15014700000\\
274000	610300000.000002\\
36999.999999999	-7812500000\\
-603999.999999999	-5248900000\\
54999.9999999997	12206700000\\
512000.000000002	1953500000\\
74000.0000000016	-3906400000.00001\\
-275000.000000001	-9765699999.99999\\
-183000.000000002	5981600000\\
329000.000000001	10253700000\\
-18000.0000000007	-12573000000\\
-218999.999999998	854299999.999994\\
-74000.0000000016	4028400000\\
202000.000000003	1831100000\\
-19000.0000000019	-3173900000\\
-72999.9999999995	-1098699999.99999\\
219999.999999999	5859599999.99999\\
-238000	-10010000000\\
-74000.0000000016	6591900000\\
439999.999999998	6469700000\\
-329999.999999999	-15869000000\\
1000.00000000033	11230200000\\
72999.9999999977	-1952900000\\
-165999.999999999	-2441500000\\
-90000.0000000025	976499999.999999\\
365000.000000002	5249300000\\
-201000.000000002	-8667400000\\
-36999.9999999972	5127300000\\
184000.000000003	366000000.000003\\
72999.9999999995	1464900000\\
-127999.999999999	-7934499999.99999\\
181999.999999999	8789099999.99999\\
-511000	-9521800000\\
-57000.0000000022	6958400000\\
771000.000000002	7690200000\\
-240000.000000004	-14404200000\\
-584000	-488399999.999996\\
273999.999999999	11352800000\\
310000.000000001	-1831400000\\
-271999.999999999	-8910899999.99999\\
15999.9999999973	6957900000\\
421000.000000002	7812600000\\
-345999.999999997	-21240300000\\
-184000.000000001	12695300000\\
421000.000000001	6958100000\\
-257000.000000001	-13916200000\\
-126999.999999998	5249199999.99999\\
54999.9999999997	2563400000\\
126999.999999996	610400000.000003\\
-17999.9999999962	-3418000000\\
220999.999999999	5248900000\\
-277000.000000001	-11962600000\\
-33999.999999998	12450900000\\
-75000.0000000037	-7446200000\\
183999.999999998	6591900000\\
72000.0000000018	-4150599999.99999\\
-419000.000000001	-5004600000\\
382000.000000001	12328700000\\
221999.999999998	-5248600000\\
-349999.999999999	-7446500000\\
-145000	4882800000\\
438999.999999999	8667100000\\
-53999.9999999967	-13672100000\\
-478000.000000005	5249300000\\
184000.000000002	2197200000\\
221000	366000000\\
15999.9999999973	-2563200000\\
-35000.0000000001	121799999.999999\\
-365999.999999998	-1220400000\\
127000.000000002	2929500000\\
72999.9999999995	-732400000.000001\\
166000.000000001	976500000\\
-127999.999999999	-3173600000\\
90999.9999999975	3295600000\\
310000	3906399999.99999\\
-382000	-15014600000\\
-478000.000000002	7690400000\\
495000.000000001	8056600000\\
329000.000000001	488300000.000002\\
-309999.999999996	-15991300000\\
-275000.000000002	6592000000\\
475000.000000003	14770300000\\
-109000.000000003	-18920800000\\
-108999.999999997	7080100000\\
-57000.0000000048	-2929600000\\
-127000.000000001	3783800000\\
384000.000000003	5981900000\\
-181999.999999999	-14038300000\\
-311999.999999998	4516600000\\
310999.999999999	8422900000\\
111000.000000002	-5249100000\\
217999.999999997	4028400000\\
-237000	-14526400000\\
-768999.999999998	6469700000\\
860999.999999997	15869200000\\
163000	-18310500000\\
-437000.000000001	4882600000\\
-276999.999999997	-243900000.000002\\
220999.999999996	3539800000\\
74000.0000000025	-3539700000\\
363999.999999999	7323900000\\
-181000.000000001	-11352400000\\
-74000.0000000007	5126900000\\
-36999.999999999	488300000.000001\\
-255999.999999998	-2075100000\\
256000	4394400000\\
-54000.0000000011	-4028300000\\
-238999.999999998	-7.105427357601e-07\\
548999.999999998	5859500000\\
-15999.9999999973	-5737500000\\
-350999.999999999	-1952900000\\
-108000.000000002	4028000000\\
165000	400000.000000489\\
255000.000000004	976199999.999999\\
-421000.000000001	-5737100000\\
312999.999999998	7202100000\\
17000.0000000003	-3784200000\\
-530999.999999998	-3662000000\\
383999.999999998	8544700000\\
1999.99999999978	-6835700000\\
-2000.00000000067	3784000000.00001\\
275000.000000002	3173899999.99999\\
74000.0000000007	-5615099999.99999\\
-348000.000000001	-5493500000\\
-221000.000000003	8301200000\\
442000.000000004	1098200000\\
-278000	-4760400000\\
-345000.000000001	366099999.999998\\
583999.999999997	5615100000\\
-71999.9999999965	-5492900000\\
-110000.000000001	732299999.999999\\
-55000.0000000015	1098400000\\
109000	-121699999.999999\\
-8.88178419700125e-10	121900000.000004\\
-144999.999999998	-2197200000\\
218999.999999999	4516400000\\
-312000.000000001	-6713700000\\
587000	14038200000\\
-147000.000000001	-19653600000\\
-513000	9888000000\\
-36000.0000000005	-488600000\\
511999.999999999	5127100000\\
-145999.999999999	-6957800000\\
129000	1952800000\\
-422999.999999997	-2319200000\\
293999.999999998	5615200000\\
91999.9999999996	-3540000000\\
-384999.999999998	-1953100000\\
-1000.000000003	2685500000\\
350000	1831100000\\
-185000.000000002	-4028400000\\
-17000.0000000003	1709000000\\
438000.000000001	5615300000\\
39000.0000000041	-8178700000\\
-918000.000000002	-4882900000\\
531999.999999999	14282200000\\
347999.999999998	-4150200000\\
-585999.999999998	-9155500000\\
-55000.0000000015	6836100000\\
531000	6713800000\\
-439000.000000002	-15258800000\\
237000.000000001	13671900000\\
330000.000000001	-488199999.999996\\
-383000	-13550100000\\
-21000.0000000008	10376300000\\
-126000	-3051800000\\
71999.9999999983	2441300000\\
349000	5249000000.00001\\
-148000.000000001	-11718700000\\
};
\addplot [color=mycolor2, line width=2.0pt, forget plot]
  table[row sep=crcr]{%
-111000	-111000\\
185000	185000\\
-112000.000000002	-112000.000000002\\
221000.000000004	221000.000000004\\
54999.9999999988	54999.9999999988\\
-457999.999999997	-457999.999999997\\
182999.999999997	182999.999999997\\
8.88178419700125e-10	8.88178419700125e-10\\
129000.000000001	129000.000000001\\
272999.999999998	272999.999999998\\
-383000.000000002	-383000.000000002\\
-147999.999999997	-147999.999999997\\
-345999.999999998	-345999.999999998\\
547000	547000\\
221999.999999999	221999.999999999\\
-146999.999999999	-146999.999999999\\
34999.9999999966	34999.9999999966\\
19999.9999999996	19999.9999999996\\
-622999.999999999	-622999.999999999\\
750999.999999998	750999.999999998\\
-148000.000000001	-148000.000000001\\
-454999.999999997	-454999.999999997\\
472999.999999997	472999.999999997\\
-51999.9999999978	-51999.9999999978\\
-369000.000000002	-369000.000000002\\
423000.000000002	423000.000000002\\
35999.9999999996	35999.9999999996\\
-419999.999999998	-419999.999999998\\
271999.999999998	271999.999999998\\
149000.000000002	149000.000000002\\
-221000.000000001	-221000.000000001\\
93000.0000000026	93000.0000000026\\
-204000.000000004	-204000.000000004\\
94000.0000000021	94000.0000000021\\
108999.999999996	108999.999999996\\
-36999.999999999	-36999.999999999\\
-36999.9999999999	-36999.9999999999\\
-15999.9999999973	-15999.9999999973\\
-58000.0000000043	-58000.0000000043\\
203000.000000004	203000.000000004\\
111000.000000001	111000.000000001\\
-351000.000000003	-351000.000000003\\
314000.000000004	314000.000000004\\
-312000.000000003	-312000.000000003\\
-73000.0000000013	-73000.0000000013\\
346999.999999999	346999.999999999\\
-201000	-201000\\
-37000.0000000008	-37000.0000000008\\
73999.9999999998	73999.9999999998\\
128000.000000002	128000.000000002\\
-293000	-293000\\
385000	385000\\
-2000.00000000244	-2000.00000000244\\
-271999.999999997	-271999.999999997\\
-94000.000000003	-94000.000000003\\
112000.000000002	112000.000000002\\
309999.999999997	309999.999999997\\
-364999.999999998	-364999.999999998\\
-56999.9999999977	-56999.9999999977\\
294999.999999997	294999.999999997\\
-74999.9999999966	-74999.9999999966\\
-218000.000000003	-218000.000000003\\
164000.000000001	164000.000000001\\
-129000	-129000\\
258000.000000001	258000.000000001\\
-129000.000000003	-129000.000000003\\
54000.000000002	54000.000000002\\
-70999.9999999988	-70999.9999999988\\
107999.999999996	107999.999999996\\
-109999.999999999	-109999.999999999\\
19999.9999999987	19999.9999999987\\
-20000.0000000005	-20000.0000000005\\
1000.00000000122	1000.00000000122\\
-201000.000000001	-201000.000000001\\
145000	145000\\
256999.999999998	256999.999999998\\
-91999.9999999978	-91999.9999999978\\
112000.000000001	112000.000000001\\
-461000.000000002	-461000.000000002\\
149000.000000003	149000.000000003\\
236999.999999999	236999.999999999\\
37000.0000000035	37000.0000000035\\
-385000.000000002	-385000.000000002\\
129000.000000003	129000.000000003\\
346999.999999999	346999.999999999\\
-164999.999999999	-164999.999999999\\
-34999.9999999993	-34999.9999999993\\
-57000.0000000013	-57000.0000000013\\
-90000.0000000016	-90000.0000000016\\
-36999.999999999	-36999.999999999\\
273999.999999997	273999.999999997\\
-35999.9999999996	-35999.9999999996\\
-238000.000000002	-238000.000000002\\
-999.999999998557	-999.999999998557\\
-34999.9999999984	-34999.9999999984\\
328999.999999999	328999.999999999\\
-165000.000000001	-165000.000000001\\
-74000.0000000016	-74000.0000000016\\
111000.000000003	111000.000000003\\
-147000	-147000\\
239000.000000001	239000.000000001\\
-130000.000000003	-130000.000000003\\
93000.0000000044	93000.0000000044\\
-19000.0000000028	-19000.0000000028\\
-255999.999999999	-255999.999999999\\
348000.000000002	348000.000000002\\
-183000.000000003	-183000.000000003\\
-129000	-129000\\
183999.999999999	183999.999999999\\
126999.999999998	126999.999999998\\
-291999.999999999	-291999.999999999\\
202000.000000002	202000.000000002\\
-130000.000000003	-130000.000000003\\
2000.00000000244	2000.00000000244\\
181999.999999997	181999.999999997\\
-238000	-238000\\
17999.9999999998	17999.9999999998\\
-35000.000000001	-35000.000000001\\
254000	254000\\
-71000.0000000015	-71000.0000000015\\
126000	126000\\
-365000.000000002	-365000.000000002\\
-53999.9999999958	-53999.9999999958\\
546999.999999996	546999.999999996\\
-381999.999999997	-381999.999999997\\
88999.9999999977	88999.9999999977\\
204000.000000001	204000.000000001\\
-203000.000000002	-203000.000000002\\
-293000	-293000\\
255999.999999999	255999.999999999\\
-125999.999999996	-125999.999999996\\
161999.999999997	161999.999999997\\
149000.000000002	149000.000000002\\
-56999.9999999995	-56999.9999999995\\
-145000.000000002	-145000.000000002\\
16999.9999999995	16999.9999999995\\
37000.0000000026	37000.0000000026\\
-35000.000000001	-35000.000000001\\
162999.999999997	162999.999999997\\
-219999.999999998	-219999.999999998\\
56000	56000\\
18999.9999999992	18999.9999999992\\
-94000.0000000012	-94000.0000000012\\
2000.00000000244	2000.00000000244\\
-36999.9999999972	-36999.9999999972\\
384999.999999998	384999.999999998\\
-111000.000000002	-111000.000000002\\
-202000.000000001	-202000.000000001\\
-15999.9999999991	-15999.9999999991\\
235999.999999999	235999.999999999\\
-144999.999999998	-144999.999999998\\
-202000.000000003	-202000.000000003\\
-55999.9999999974	-55999.9999999974\\
477999.999999998	477999.999999998\\
-276000.000000001	-276000.000000001\\
18999.9999999992	18999.9999999992\\
36000.0000000014	36000.0000000014\\
54999.9999999997	54999.9999999997\\
19000.0000000001	19000.0000000001\\
-92999.9999999982	-92999.9999999982\\
-254000	-254000\\
-20000.0000000005	-20000.0000000005\\
329000	329000\\
130000.000000003	130000.000000003\\
71999.9999999974	71999.9999999974\\
-199999.999999997	-199999.999999997\\
-76000.000000004	-76000.000000004\\
38999.9999999988	38999.9999999988\\
-385000	-385000\\
475999.999999998	475999.999999998\\
-165000	-165000\\
-8.88178419700125e-10	-8.88178419700125e-10\\
-110000	-110000\\
275000.000000001	275000.000000001\\
165000.000000002	165000.000000002\\
-294000	-294000\\
201999.999999999	201999.999999999\\
-346999.999999998	-346999.999999998\\
89999.9999999998	89999.9999999998\\
75000.0000000002	75000.0000000002\\
107999.999999997	107999.999999997\\
-90999.9999999966	-90999.9999999966\\
-90000.0000000007	-90000.0000000007\\
88999.9999999986	88999.9999999986\\
-88999.9999999977	-88999.9999999977\\
198999.999999997	198999.999999997\\
-364000	-364000\\
217999.999999999	217999.999999999\\
276000	276000\\
-111000.000000002	-111000.000000002\\
-328999.999999999	-328999.999999999\\
-74000.0000000025	-74000.0000000025\\
294000.000000003	294000.000000003\\
-35999.9999999996	-35999.9999999996\\
89000.0000000004	89000.0000000004\\
-51999.9999999987	-51999.9999999987\\
88999.9999999986	88999.9999999986\\
-107999.999999999	-107999.999999999\\
-183999.999999997	-183999.999999997\\
146999.999999999	146999.999999999\\
292000.000000002	292000.000000002\\
-218000.000000003	-218000.000000003\\
-350000	-350000\\
349000.000000001	349000.000000001\\
129000	129000\\
-405000.000000001	-405000.000000001\\
186000.000000003	186000.000000003\\
142999.999999996	142999.999999996\\
-32999.999999995	-32999.999999995\\
-259000.000000004	-259000.000000004\\
274999.999999999	274999.999999999\\
-162999.999999997	-162999.999999997\\
-75000.0000000046	-75000.0000000046\\
311000.000000003	311000.000000003\\
-17000.0000000012	-17000.0000000012\\
-237999.999999997	-237999.999999997\\
-92000.0000000058	-92000.0000000058\\
-999.999999996781	-999.999999996781\\
531999.999999999	531999.999999999\\
-108999.999999999	-108999.999999999\\
-295000.000000002	-295000.000000002\\
-145000.000000002	-145000.000000002\\
238000	238000\\
35999.9999999987	35999.9999999987\\
-164999.999999998	-164999.999999998\\
18999.9999999966	18999.9999999966\\
36000.0000000031	36000.0000000031\\
55999.9999999991	55999.9999999991\\
164000.000000001	164000.000000001\\
-458000.000000002	-458000.000000002\\
294000	294000\\
126999.999999998	126999.999999998\\
-73000.0000000004	-73000.0000000004\\
-127999.999999997	-127999.999999997\\
-109000.000000003	-109000.000000003\\
419999.999999998	419999.999999998\\
18000.0000000007	18000.0000000007\\
-603000.000000002	-603000.000000002\\
219000	219000\\
18999.9999999992	18999.9999999992\\
254999.999999998	254999.999999998\\
-53999.9999999994	-53999.9999999994\\
-348000.000000001	-348000.000000001\\
312000.000000002	312000.000000002\\
89999.9999999972	89999.9999999972\\
-347000.000000001	-347000.000000001\\
256000.000000002	256000.000000002\\
-53999.9999999985	-53999.9999999985\\
-74000.0000000043	-74000.0000000043\\
219000.000000001	219000.000000001\\
-273000	-273000\\
162999.999999998	162999.999999998\\
-125999.999999999	-125999.999999999\\
-166999.999999998	-166999.999999998\\
330999.999999997	330999.999999997\\
-111000	-111000\\
-53999.9999999967	-53999.9999999967\\
385000	385000\\
-148000	-148000\\
-237000.000000001	-237000.000000001\\
-35999.9999999996	-35999.9999999996\\
-75000.000000001	-75000.000000001\\
403999.999999998	403999.999999998\\
-293999.999999999	-293999.999999999\\
-53000.0000000008	-53000.0000000008\\
-2000.00000000067	-2000.00000000067\\
294000	294000\\
-256000.000000002	-256000.000000002\\
328000.000000002	328000.000000002\\
-236000	-236000\\
144999.999999997	144999.999999997\\
-238000	-238000\\
93000.0000000017	93000.0000000017\\
-74999.9999999975	-74999.9999999975\\
-54000.0000000011	-54000.0000000011\\
-17999.9999999998	-17999.9999999998\\
218000.000000001	218000.000000001\\
2000.00000000067	2000.00000000067\\
-92000.0000000014	-92000.0000000014\\
-164999.999999999	-164999.999999999\\
365999.999999996	365999.999999996\\
18000.0000000007	18000.0000000007\\
-366000	-366000\\
165999.999999999	165999.999999999\\
-93999.9999999985	-93999.9999999985\\
241000	241000\\
33999.999999998	33999.999999998\\
-52999.9999999982	-52999.9999999982\\
-221000.000000001	-221000.000000001\\
-72999.9999999986	-72999.9999999986\\
73999.9999999998	73999.9999999998\\
90000.0000000007	90000.0000000007\\
38999.9999999997	38999.9999999997\\
-130999.999999999	-130999.999999999\\
-52000.0000000023	-52000.0000000023\\
326999.999999999	326999.999999999\\
-145000	-145000\\
-109999.999999999	-109999.999999999\\
273999.999999996	273999.999999996\\
-72999.9999999977	-72999.9999999977\\
-512000.000000002	-512000.000000002\\
256000.000000001	256000.000000001\\
328999.999999996	328999.999999996\\
-200999.999999998	-200999.999999998\\
72999.9999999986	72999.9999999986\\
37000.0000000017	37000.0000000017\\
-329000.000000001	-329000.000000001\\
109000	109000\\
255999.999999997	255999.999999997\\
37000.0000000017	37000.0000000017\\
-531000	-531000\\
586000.000000001	586000.000000001\\
-200999.999999998	-200999.999999998\\
-147000.000000002	-147000.000000002\\
147000.000000003	147000.000000003\\
90999.9999999975	90999.9999999975\\
-165000.000000002	-165000.000000002\\
-200999.999999997	-200999.999999997\\
147000.000000001	147000.000000001\\
401999.999999997	401999.999999997\\
-109999.999999999	-109999.999999999\\
-347000.000000001	-347000.000000001\\
54000.000000002	54000.000000002\\
-16999.9999999995	-16999.9999999995\\
107999.999999997	107999.999999997\\
203000.000000004	203000.000000004\\
-330000.000000003	-330000.000000003\\
35999.9999999987	35999.9999999987\\
165000.000000003	165000.000000003\\
92999.9999999991	92999.9999999991\\
-149000.000000001	-149000.000000001\\
-236000.000000003	-236000.000000003\\
35000.0000000028	35000.0000000028\\
386999.999999999	386999.999999999\\
-131000.000000001	-131000.000000001\\
93000.0000000008	93000.0000000008\\
-201999.999999999	-201999.999999999\\
-108999.999999998	-108999.999999998\\
348000.000000001	348000.000000001\\
-148000.000000006	-148000.000000006\\
-126999.999999998	-126999.999999998\\
17999.9999999989	17999.9999999989\\
-311000.000000001	-311000.000000001\\
365999.999999999	365999.999999999\\
110000.000000002	110000.000000002\\
72999.9999999995	72999.9999999995\\
-220000.000000002	-220000.000000002\\
-53999.9999999985	-53999.9999999985\\
311000.000000001	311000.000000001\\
-312000.000000003	-312000.000000003\\
37000.0000000008	37000.0000000008\\
145999.999999999	145999.999999999\\
-182000	-182000\\
-129000	-129000\\
257000.000000001	257000.000000001\\
274999.999999999	274999.999999999\\
-277000.000000001	-277000.000000001\\
-107000.000000001	-107000.000000001\\
-57000.0000000004	-57000.0000000004\\
294000	294000\\
-549000.000000001	-549000.000000001\\
255999.999999999	255999.999999999\\
530000	530000\\
-530000.000000002	-530000.000000002\\
92000.0000000014	92000.0000000014\\
126999.999999998	126999.999999998\\
54999.9999999997	54999.9999999997\\
-71999.9999999965	-71999.9999999965\\
-184000.000000002	-184000.000000002\\
54999.999999997	54999.999999997\\
0	0\\
37000.0000000017	37000.0000000017\\
53999.9999999976	53999.9999999976\\
-145000	-145000\\
144999.999999999	144999.999999999\\
-53999.9999999985	-53999.9999999985\\
274000	274000\\
-348000.000000001	-348000.000000001\\
-16999.9999999959	-16999.9999999959\\
254999.999999997	254999.999999997\\
-254999.999999997	-254999.999999997\\
199999.999999997	199999.999999997\\
110000.000000001	110000.000000001\\
-164000.000000001	-164000.000000001\\
-257000.000000001	-257000.000000001\\
37000.0000000026	37000.0000000026\\
311999.999999998	311999.999999998\\
-2000.00000000244	-2000.00000000244\\
-290999.999999999	-290999.999999999\\
236999.999999998	236999.999999998\\
37000.0000000017	37000.0000000017\\
-293999.999999999	-293999.999999999\\
166999.999999998	166999.999999998\\
126000	126000\\
-329000.000000001	-329000.000000001\\
166000.000000002	166000.000000002\\
291999.999999998	291999.999999998\\
-477000	-477000\\
130000.000000002	130000.000000002\\
383000.000000001	383000.000000001\\
-327999.999999999	-327999.999999999\\
236000.000000001	236000.000000001\\
-236000	-236000\\
145000	145000\\
55999.9999999983	55999.9999999983\\
-440999.999999999	-440999.999999999\\
405000.000000001	405000.000000001\\
-38000.0000000029	-38000.0000000029\\
-311000.000000001	-311000.000000001\\
457000	457000\\
-126999.999999999	-126999.999999999\\
-293000.000000001	-293000.000000001\\
-92999.9999999973	-92999.9999999973\\
312999.999999997	312999.999999997\\
91000.0000000019	91000.0000000019\\
-17999.9999999998	-17999.9999999998\\
-56000.0000000027	-56000.0000000027\\
-91000.0000000011	-91000.0000000011\\
183000.000000002	183000.000000002\\
-128000.000000002	-128000.000000002\\
-73000.0000000013	-73000.0000000013\\
165000	165000\\
-313000	-313000\\
442000.000000003	442000.000000003\\
-203000.000000003	-203000.000000003\\
-220000.000000001	-220000.000000001\\
111000.000000003	111000.000000003\\
328999.999999999	328999.999999999\\
-164999.999999999	-164999.999999999\\
-127000.000000001	-127000.000000001\\
-39000.0000000015	-39000.0000000015\\
240000.000000002	240000.000000002\\
-183000	-183000\\
-148000.000000003	-148000.000000003\\
167000.000000003	167000.000000003\\
51999.9999999978	51999.9999999978\\
94000.0000000012	94000.0000000012\\
-238999.999999999	-238999.999999999\\
146999.999999998	146999.999999998\\
200999.999999999	200999.999999999\\
-422000.000000001	-422000.000000001\\
-90000.0000000007	-90000.0000000007\\
238000.000000001	238000.000000001\\
272999.999999999	272999.999999999\\
-145000	-145000\\
-110000	-110000\\
71999.9999999992	71999.9999999992\\
-108000.000000001	-108000.000000001\\
108000	108000\\
-218000	-218000\\
254999.999999998	254999.999999998\\
74000.0000000016	74000.0000000016\\
-439000.000000001	-439000.000000001\\
273000	273000\\
19999.9999999996	19999.9999999996\\
-999.999999998557	-999.999999998557\\
999.999999997669	999.999999997669\\
-129999.999999999	-129999.999999999\\
56999.9999999986	56999.9999999986\\
127000.000000002	127000.000000002\\
219000.000000001	219000.000000001\\
-144000.000000002	-144000.000000002\\
-204000.000000001	-204000.000000001\\
55999.9999999983	55999.9999999983\\
-311000.000000001	-311000.000000001\\
347999.999999998	347999.999999998\\
-201999.999999999	-201999.999999999\\
310999.999999999	310999.999999999\\
-34999.9999999993	-34999.9999999993\\
-277000.000000002	-277000.000000002\\
166000	166000\\
-16999.9999999977	-16999.9999999977\\
51999.9999999987	51999.9999999987\\
-161999.999999998	-161999.999999998\\
254999.999999998	254999.999999998\\
73000.0000000013	73000.0000000013\\
-128000	-128000\\
-219000.000000001	-219000.000000001\\
91000.0000000011	91000.0000000011\\
-18000.0000000007	-18000.0000000007\\
238000.000000001	238000.000000001\\
-203000.000000003	-203000.000000003\\
-15999.9999999947	-15999.9999999947\\
311000	311000\\
-569000.000000003	-569000.000000003\\
128999.999999999	128999.999999999\\
531000.000000002	531000.000000002\\
-477000.000000002	-477000.000000002\\
-144999.999999997	-144999.999999997\\
604000.000000001	604000.000000001\\
-312000.000000001	-312000.000000001\\
-237999.999999999	-237999.999999999\\
349000	349000\\
89999.9999999963	89999.9999999963\\
-163999.999999999	-163999.999999999\\
-238000.000000002	-238000.000000002\\
-18999.9999999992	-18999.9999999992\\
20000.0000000013	20000.0000000013\\
400999.999999998	400999.999999998\\
-127999.999999998	-127999.999999998\\
20000.0000000022	20000.0000000022\\
-2000.00000000244	-2000.00000000244\\
-162999.999999998	-162999.999999998\\
52999.9999999973	52999.9999999973\\
-52999.9999999982	-52999.9999999982\\
52999.9999999964	52999.9999999964\\
-144999.999999996	-144999.999999996\\
127999.999999998	127999.999999998\\
274000.000000001	274000.000000001\\
-512000.000000004	-512000.000000004\\
164000.000000001	164000.000000001\\
165999.999999999	165999.999999999\\
-148000.000000001	-148000.000000001\\
74000.0000000016	74000.0000000016\\
-17999.999999998	-17999.999999998\\
274000.000000001	274000.000000001\\
-219000	-219000\\
-164999.999999998	-164999.999999998\\
218999.999999998	218999.999999998\\
-36000.0000000005	-36000.0000000005\\
-146000.000000001	-146000.000000001\\
-184000.000000001	-184000.000000001\\
366000.000000001	366000.000000001\\
276000.000000002	276000.000000002\\
-294000.000000004	-294000.000000004\\
-128000.000000002	-128000.000000002\\
183000	183000\\
-383999.999999999	-383999.999999999\\
-36999.9999999999	-36999.9999999999\\
365999.999999997	365999.999999997\\
-127999.999999998	-127999.999999998\\
239000.000000001	239000.000000001\\
-93000.0000000008	-93000.0000000008\\
-145999.999999997	-145999.999999997\\
1.77635683940025e-09	1.77635683940025e-09\\
108999.999999999	108999.999999999\\
-144000.000000003	-144000.000000003\\
-39000.0000000006	-39000.0000000006\\
38000.0000000011	38000.0000000011\\
218999.999999998	218999.999999998\\
8.88178419700125e-10	8.88178419700125e-10\\
-202000	-202000\\
-16999.9999999977	-16999.9999999977\\
183000.000000002	183000.000000002\\
-129000	-129000\\
92000.0000000014	92000.0000000014\\
165000	165000\\
-202000	-202000\\
-109000	-109000\\
-239000.000000003	-239000.000000003\\
458000.000000003	458000.000000003\\
91999.9999999978	91999.9999999978\\
-91999.9999999987	-91999.9999999987\\
-37000.0000000026	-37000.0000000026\\
-126999.999999999	-126999.999999999\\
-183000.000000001	-183000.000000001\\
346000.000000002	346000.000000002\\
-474000.000000001	-474000.000000001\\
290999.999999999	290999.999999999\\
202999.999999998	202999.999999998\\
-72999.9999999995	-72999.9999999995\\
90000.0000000025	90000.0000000025\\
-201000	-201000\\
-164000.000000001	-164000.000000001\\
219999.999999998	219999.999999998\\
-19999.9999999978	-19999.9999999978\\
74999.9999999975	74999.9999999975\\
-19999.9999999969	-19999.9999999969\\
-273000.000000001	-273000.000000001\\
-147000.000000001	-147000.000000001\\
584999.999999999	584999.999999999\\
-35999.999999996	-35999.999999996\\
999.999999999446	999.999999999446\\
-349000	-349000\\
-74000.0000000025	-74000.0000000025\\
497000.000000003	497000.000000003\\
-204000.000000001	-204000.000000001\\
37000.0000000008	37000.0000000008\\
-126999.999999999	-126999.999999999\\
-37000.0000000026	-37000.0000000026\\
255000.000000003	255000.000000003\\
-254000.000000001	-254000.000000001\\
199999.999999999	199999.999999999\\
-202000.000000001	-202000.000000001\\
-273000	-273000\\
620999.999999998	620999.999999998\\
-310000	-310000\\
-75000.0000000002	-75000.0000000002\\
204000.000000003	204000.000000003\\
-112000.000000004	-112000.000000004\\
18999.9999999992	18999.9999999992\\
-55000.0000000006	-55000.0000000006\\
54999.9999999988	54999.9999999988\\
109999.999999999	109999.999999999\\
-73999.9999999989	-73999.9999999989\\
-163000.000000001	-163000.000000001\\
456000	456000\\
-219000.000000001	-219000.000000001\\
-494999.999999998	-494999.999999998\\
165000.000000001	165000.000000001\\
605999.999999998	605999.999999998\\
-258999.999999999	-258999.999999999\\
-530000	-530000\\
257000.000000001	257000.000000001\\
237999.999999998	237999.999999998\\
-93000	-93000\\
147999.999999998	147999.999999998\\
-330000.000000001	-330000.000000001\\
310999.999999999	310999.999999999\\
92000.0000000014	92000.0000000014\\
-495000.000000002	-495000.000000002\\
549999.999999999	549999.999999999\\
-257000.000000001	-257000.000000001\\
-53999.9999999994	-53999.9999999994\\
90000.0000000016	90000.0000000016\\
-364999.999999998	-364999.999999998\\
475999.999999999	475999.999999999\\
-330000.000000001	-330000.000000001\\
-35999.9999999996	-35999.9999999996\\
419999.999999997	419999.999999997\\
-35000.0000000001	-35000.0000000001\\
-258000.000000001	-258000.000000001\\
111000.000000002	111000.000000002\\
-165000.000000003	-165000.000000003\\
274000.000000002	274000.000000002\\
-182000	-182000\\
35000.0000000019	35000.0000000019\\
57000.0000000004	57000.0000000004\\
-165999.999999999	-165999.999999999\\
146000.000000003	146000.000000003\\
-237000.000000003	-237000.000000003\\
346999.999999999	346999.999999999\\
-16999.9999999986	-16999.9999999986\\
-202999.999999999	-202999.999999999\\
128999.999999997	128999.999999997\\
-127999.999999999	-127999.999999999\\
-256000	-256000\\
364999.999999997	364999.999999997\\
330000.000000001	330000.000000001\\
-366000.000000001	-366000.000000001\\
74000.0000000025	74000.0000000025\\
-349000.000000003	-349000.000000003\\
329000.000000001	329000.000000001\\
203000.000000001	203000.000000001\\
-458000	-458000\\
273999.999999997	273999.999999997\\
-200999.999999999	-200999.999999999\\
365999.999999999	365999.999999999\\
-329999.999999997	-329999.999999997\\
-54000.0000000003	-54000.0000000003\\
109000.000000001	109000.000000001\\
201999.999999998	201999.999999998\\
-184000	-184000\\
-53000.0000000008	-53000.0000000008\\
-74999.9999999993	-74999.9999999993\\
310999.999999999	310999.999999999\\
-309999.999999997	-309999.999999997\\
55000.0000000015	55000.0000000015\\
383999.999999997	383999.999999997\\
-129000	-129000\\
-162999.999999998	-162999.999999998\\
-313000.000000001	-313000.000000001\\
256999.999999998	256999.999999998\\
293000.000000001	293000.000000001\\
-346999.999999998	-346999.999999998\\
290999.999999996	290999.999999996\\
2000.00000000244	2000.00000000244\\
-404999.999999999	-404999.999999999\\
-181000.000000002	-181000.000000002\\
622000.000000001	622000.000000001\\
-73999.9999999989	-73999.9999999989\\
-383000.000000001	-383000.000000001\\
345999.999999996	345999.999999996\\
-17999.9999999971	-17999.9999999971\\
19999.9999999996	19999.9999999996\\
-295000.000000003	-295000.000000003\\
93000.0000000035	93000.0000000035\\
255999.999999997	255999.999999997\\
-202000	-202000\\
72999.9999999995	72999.9999999995\\
56000.0000000027	56000.0000000027\\
-238000.000000001	-238000.000000001\\
273999.999999997	273999.999999997\\
-238000	-238000\\
-110000	-110000\\
330000.000000001	330000.000000001\\
-53999.9999999976	-53999.9999999976\\
33999.9999999954	33999.9999999954\\
-15999.9999999973	-15999.9999999973\\
-19000.0000000019	-19000.0000000019\\
-202000.000000002	-202000.000000002\\
56000.0000000018	56000.0000000018\\
90999.9999999975	90999.9999999975\\
-73999.9999999989	-73999.9999999989\\
93000.0000000017	93000.0000000017\\
182000	182000\\
-36000.0000000014	-36000.0000000014\\
-567999.999999998	-567999.999999998\\
312000.000000001	312000.000000001\\
456999.999999997	456999.999999997\\
-438999.999999999	-438999.999999999\\
-19000.0000000019	-19000.0000000019\\
201999.999999998	201999.999999998\\
-36999.9999999999	-36999.9999999999\\
-220000	-220000\\
185000.000000001	185000.000000001\\
-94000.0000000021	-94000.0000000021\\
20000.0000000005	20000.0000000005\\
145999.999999998	145999.999999998\\
-147000	-147000\\
36999.9999999999	36999.9999999999\\
-19000.0000000001	-19000.0000000001\\
130000.000000001	130000.000000001\\
-258000.000000001	-258000.000000001\\
292999.999999999	292999.999999999\\
-181999.999999999	-181999.999999999\\
-91999.9999999996	-91999.9999999996\\
218999.999999998	218999.999999998\\
129000.000000001	129000.000000001\\
-165000	-165000\\
-128000.000000002	-128000.000000002\\
-201999.999999998	-201999.999999998\\
-109000.000000001	-109000.000000001\\
547999.999999998	547999.999999998\\
55999.9999999991	55999.9999999991\\
-36999.9999999972	-36999.9999999972\\
-311000.000000001	-311000.000000001\\
128000	128000\\
73999.9999999998	73999.9999999998\\
34999.9999999984	34999.9999999984\\
-311000	-311000\\
165999.999999999	165999.999999999\\
128000.000000002	128000.000000002\\
-166000.000000001	-166000.000000001\\
148000	148000\\
-93000.0000000008	-93000.0000000008\\
147999.999999998	147999.999999998\\
-293999.999999998	-293999.999999998\\
-73999.9999999989	-73999.9999999989\\
532999.999999998	532999.999999998\\
-368000.000000001	-368000.000000001\\
93000.0000000008	93000.0000000008\\
-93000.0000000017	-93000.0000000017\\
130000.000000003	130000.000000003\\
-148000.000000003	-148000.000000003\\
274000.000000001	274000.000000001\\
-235999.999999995	-235999.999999995\\
17999.9999999989	17999.9999999989\\
309999.999999998	309999.999999998\\
-659000.000000002	-659000.000000002\\
661000.000000002	661000.000000002\\
-442000.000000002	-442000.000000002\\
166000	166000\\
-90999.9999999993	-90999.9999999993\\
274000	274000\\
-293000.000000003	-293000.000000003\\
19000.0000000019	19000.0000000019\\
272999.999999999	272999.999999999\\
-144000	-144000\\
-39000.0000000024	-39000.0000000024\\
-90999.9999999975	-90999.9999999975\\
148000.000000001	148000.000000001\\
-20000.000000004	-20000.000000004\\
-53999.9999999976	-53999.9999999976\\
-17999.9999999998	-17999.9999999998\\
34999.9999999984	34999.9999999984\\
-255000.000000003	-255000.000000003\\
274000.000000003	274000.000000003\\
220000.000000001	220000.000000001\\
-255999.999999998	-255999.999999998\\
36999.999999999	36999.999999999\\
144999.999999997	144999.999999997\\
-201000	-201000\\
-34999.9999999993	-34999.9999999993\\
125999.999999999	125999.999999999\\
-182000	-182000\\
-19000.0000000001	-19000.0000000001\\
148000.000000001	148000.000000001\\
52999.999999999	52999.999999999\\
57000.0000000004	57000.0000000004\\
33999.999999998	33999.999999998\\
-472999.999999997	-472999.999999997\\
307999.999999998	307999.999999998\\
149000.000000004	149000.000000004\\
-19000.0000000001	-19000.0000000001\\
-383999.999999999	-383999.999999999\\
511000	511000\\
37999.9999999976	37999.9999999976\\
-347999.999999997	-347999.999999997\\
-1000.00000000477	-1000.00000000477\\
-236999.999999998	-236999.999999998\\
311999.999999999	311999.999999999\\
-1999.99999999978	-1999.99999999978\\
220999.999999997	220999.999999997\\
-146999.999999999	-146999.999999999\\
36999.9999999981	36999.9999999981\\
-220000	-220000\\
-55000.0000000006	-55000.0000000006\\
202000.000000001	202000.000000001\\
72999.9999999995	72999.9999999995\\
-201000.000000002	-201000.000000002\\
90000.0000000016	90000.0000000016\\
202999.999999998	202999.999999998\\
-586999.999999996	-586999.999999996\\
439999.999999996	439999.999999996\\
-36999.9999999981	-36999.9999999981\\
38000.0000000002	38000.0000000002\\
-20000.0000000022	-20000.0000000022\\
-73000.0000000004	-73000.0000000004\\
-16999.9999999959	-16999.9999999959\\
217999.999999995	217999.999999995\\
-328999.999999997	-328999.999999997\\
129000	129000\\
237000.000000001	237000.000000001\\
-273999.999999997	-273999.999999997\\
109999.999999999	109999.999999999\\
-111000.000000003	-111000.000000003\\
-34999.9999999957	-34999.9999999957\\
291999.999999997	291999.999999997\\
-111000	-111000\\
-180999.999999997	-180999.999999997\\
127999.999999999	127999.999999999\\
126999.999999998	126999.999999998\\
-403000.000000001	-403000.000000001\\
147999.999999999	147999.999999999\\
291000.000000002	291000.000000002\\
-218000.000000001	-218000.000000001\\
-164999.999999996	-164999.999999996\\
529999.999999997	529999.999999997\\
-273999.999999996	-273999.999999996\\
-73000.0000000004	-73000.0000000004\\
-128000.000000003	-128000.000000003\\
17000.0000000003	17000.0000000003\\
404000	404000\\
-294000.000000001	-294000.000000001\\
-199999.999999999	-199999.999999999\\
457000	457000\\
-72999.9999999986	-72999.9999999986\\
-440000.000000003	-440000.000000003\\
366000	366000\\
-35000.0000000001	-35000.0000000001\\
-184999.999999999	-184999.999999999\\
165999.999999998	165999.999999998\\
-18999.9999999983	-18999.9999999983\\
-146000.000000003	-146000.000000003\\
37000.0000000026	37000.0000000026\\
126999.999999999	126999.999999999\\
-91000.0000000019	-91000.0000000019\\
202000	202000\\
-201999.999999998	-201999.999999998\\
17999.9999999962	17999.9999999962\\
128000.000000003	128000.000000003\\
75000.000000001	75000.000000001\\
-313000.000000003	-313000.000000003\\
183000.000000001	183000.000000001\\
-71999.9999999965	-71999.9999999965\\
-128000.000000001	-128000.000000001\\
400999.999999997	400999.999999997\\
-328000.000000002	-328000.000000002\\
109000.000000001	109000.000000001\\
17999.9999999998	17999.9999999998\\
-218000.000000001	-218000.000000001\\
345999.999999999	345999.999999999\\
-163999.999999998	-163999.999999998\\
1000.00000000033	1000.00000000033\\
180999.999999997	180999.999999997\\
-145000.000000002	-145000.000000002\\
-91999.9999999996	-91999.9999999996\\
-274000	-274000\\
455999.999999999	455999.999999999\\
-180999.999999998	-180999.999999998\\
144999.999999998	144999.999999998\\
-162999.999999998	-162999.999999998\\
-186000.000000001	-186000.000000001\\
350999.999999999	350999.999999999\\
199000	199000\\
-347000.000000001	-347000.000000001\\
55000.0000000024	55000.0000000024\\
129000	129000\\
-239000.000000003	-239000.000000003\\
17999.999999998	17999.999999998\\
202000.000000003	202000.000000003\\
-202000.000000003	-202000.000000003\\
19000.000000001	19000.000000001\\
-17999.9999999989	-17999.9999999989\\
254999.999999998	254999.999999998\\
-108999.999999998	-108999.999999998\\
-108999.999999999	-108999.999999999\\
236000.000000001	236000.000000001\\
-327999.999999998	-327999.999999998\\
238000	238000\\
-129000.000000004	-129000.000000004\\
-127999.999999999	-127999.999999999\\
201999.999999997	201999.999999997\\
365000.000000001	365000.000000001\\
-620999.999999999	-620999.999999999\\
126999.999999998	126999.999999998\\
1000.00000000122	1000.00000000122\\
108999.999999997	108999.999999997\\
202000.000000002	202000.000000002\\
-403000.000000001	-403000.000000001\\
-56000.0000000009	-56000.0000000009\\
148000.000000001	148000.000000001\\
293000.000000005	293000.000000005\\
-1000.00000000122	-1000.00000000122\\
-165000.000000003	-165000.000000003\\
-162999.999999996	-162999.999999996\\
52999.9999999973	52999.9999999973\\
257000.000000003	257000.000000003\\
-201000	-201000\\
-73999.9999999989	-73999.9999999989\\
276000.000000001	276000.000000001\\
-203000.000000002	-203000.000000002\\
-164000	-164000\\
348000.000000002	348000.000000002\\
-74000.0000000016	-74000.0000000016\\
-345999.999999997	-345999.999999997\\
399999.999999995	399999.999999995\\
-217999.999999999	-217999.999999999\\
37000.0000000008	37000.0000000008\\
16999.9999999986	16999.9999999986\\
-89999.9999999963	-89999.9999999963\\
309999.999999998	309999.999999998\\
-146000.000000003	-146000.000000003\\
-183999.999999997	-183999.999999997\\
94000.0000000021	94000.0000000021\\
-3000.00000000455	-3000.00000000455\\
167000.000000002	167000.000000002\\
-75000.000000001	-75000.000000001\\
-291000	-291000\\
345999.999999999	345999.999999999\\
56000.0000000009	56000.0000000009\\
-439000.000000001	-439000.000000001\\
401000	401000\\
2000.00000000244	2000.00000000244\\
-56000.0000000018	-56000.0000000018\\
-91000.0000000002	-91000.0000000002\\
35999.9999999978	35999.9999999978\\
-384000	-384000\\
475999.999999998	475999.999999998\\
91000.0000000002	91000.0000000002\\
-164000.000000001	-164000.000000001\\
-184000.000000002	-184000.000000002\\
-71999.9999999983	-71999.9999999983\\
291999.999999998	291999.999999998\\
-91999.9999999996	-91999.9999999996\\
38000.000000002	38000.000000002\\
109000	109000\\
-183000	-183000\\
274999.999999999	274999.999999999\\
-1000.00000000122	-1000.00000000122\\
-566000.000000001	-566000.000000001\\
310000	310000\\
-8.88178419700125e-10	-8.88178419700125e-10\\
-164000	-164000\\
348000.000000001	348000.000000001\\
-294000.000000001	-294000.000000001\\
457999.999999998	457999.999999998\\
-273999.999999998	-273999.999999998\\
-312000.000000002	-312000.000000002\\
513000.000000002	513000.000000002\\
-146000.000000002	-146000.000000002\\
-92000.0000000005	-92000.0000000005\\
-202000.000000002	-202000.000000002\\
111000.000000003	111000.000000003\\
17999.9999999954	17999.9999999954\\
347000.000000001	347000.000000001\\
-237000.000000003	-237000.000000003\\
-91999.9999999978	-91999.9999999978\\
-55000.0000000041	-55000.0000000041\\
37000.0000000017	37000.0000000017\\
-55999.9999999991	-55999.9999999991\\
331000.000000002	331000.000000002\\
-128000.000000001	-128000.000000001\\
70999.9999999988	70999.9999999988\\
-70999.9999999971	-70999.9999999971\\
-293000	-293000\\
510999.999999997	510999.999999997\\
-585000	-585000\\
238999.999999998	238999.999999998\\
108000	108000\\
-310000	-310000\\
255999.999999998	255999.999999998\\
73000.0000000031	73000.0000000031\\
-90999.9999999984	-90999.9999999984\\
183000	183000\\
145999.999999998	145999.999999998\\
-476000.000000002	-476000.000000002\\
91999.9999999987	91999.9999999987\\
-36999.9999999999	-36999.9999999999\\
54999.9999999988	54999.9999999988\\
-17999.999999998	-17999.999999998\\
347000	347000\\
-199999.999999999	-199999.999999999\\
-257000.000000001	-257000.000000001\\
91000.0000000019	91000.0000000019\\
293999.999999997	293999.999999997\\
-129999.999999999	-129999.999999999\\
-88999.9999999986	-88999.9999999986\\
52999.9999999964	52999.9999999964\\
-91000.0000000002	-91000.0000000002\\
54999.9999999988	54999.9999999988\\
-219000	-219000\\
127000	127000\\
182999.999999999	182999.999999999\\
73999.9999999998	73999.9999999998\\
-182999.999999998	-182999.999999998\\
219999.999999998	219999.999999998\\
-129999.999999997	-129999.999999997\\
-346000.000000001	-346000.000000001\\
-1000.00000000211	-1000.00000000211\\
239000.000000001	239000.000000001\\
108999.999999996	108999.999999996\\
2.66453525910038e-09	2.66453525910038e-09\\
-91000.0000000028	-91000.0000000028\\
-55000.0000000006	-55000.0000000006\\
-18999.9999999992	-18999.9999999992\\
292999.999999998	292999.999999998\\
-163999.999999995	-163999.999999995\\
17999.9999999998	17999.9999999998\\
201000.000000001	201000.000000001\\
-347000	-347000\\
-74000.0000000007	-74000.0000000007\\
73999.9999999989	73999.9999999989\\
199999.999999999	199999.999999999\\
38000.0000000002	38000.0000000002\\
-202000.000000001	-202000.000000001\\
108999.999999998	108999.999999998\\
-345999.999999997	-345999.999999997\\
384000	384000\\
144999.999999996	144999.999999996\\
-217999.999999998	-217999.999999998\\
-311999.999999999	-311999.999999999\\
219000	219000\\
257999.999999999	257999.999999999\\
-313000.000000001	-313000.000000001\\
275999.999999999	275999.999999999\\
-311999.999999999	-311999.999999999\\
19000.000000001	19000.000000001\\
419999.999999997	419999.999999997\\
-401999.999999997	-401999.999999997\\
2.66453525910038e-09	2.66453525910038e-09\\
238000	238000\\
73000.0000000013	73000.0000000013\\
8.88178419700125e-10	8.88178419700125e-10\\
-238000.000000002	-238000.000000002\\
147000.000000001	147000.000000001\\
-385000.000000003	-385000.000000003\\
365000.000000002	365000.000000002\\
-272000	-272000\\
290999.999999999	290999.999999999\\
73999.9999999981	73999.9999999981\\
-347999.999999998	-347999.999999998\\
401999.999999997	401999.999999997\\
-400999.999999999	-400999.999999999\\
272000	272000\\
75999.9999999996	75999.9999999996\\
-366999.999999999	-366999.999999999\\
273000.000000001	273000.000000001\\
-145000.000000004	-145000.000000004\\
92000.0000000014	92000.0000000014\\
54000.0000000003	54000.0000000003\\
-182999.999999997	-182999.999999997\\
17999.9999999989	17999.9999999989\\
148000.000000001	148000.000000001\\
-19999.9999999987	-19999.9999999987\\
-218999.999999998	-218999.999999998\\
163999.999999999	163999.999999999\\
238999.999999998	238999.999999998\\
-201000	-201000\\
-239000.000000002	-239000.000000002\\
201999.999999998	201999.999999998\\
255000.000000002	255000.000000002\\
-273000.000000001	-273000.000000001\\
311000.000000001	311000.000000001\\
-221000.000000004	-221000.000000004\\
-546999.999999999	-546999.999999999\\
419000	419000\\
219999.999999998	219999.999999998\\
-237000	-237000\\
219000.000000003	219000.000000003\\
54999.9999999988	54999.9999999988\\
-109000	-109000\\
-240000.000000002	-240000.000000002\\
112000.000000002	112000.000000002\\
126999.999999999	126999.999999999\\
91999.9999999987	91999.9999999987\\
-348999.999999997	-348999.999999997\\
167000	167000\\
89000.0000000013	89000.0000000013\\
-35000.0000000028	-35000.0000000028\\
-255999.999999999	-255999.999999999\\
90000.0000000016	90000.0000000016\\
293999.999999995	293999.999999995\\
-17999.9999999962	-17999.9999999962\\
-404000	-404000\\
146999.999999998	146999.999999998\\
495000.000000002	495000.000000002\\
-550000.000000002	-550000.000000002\\
109999.999999999	109999.999999999\\
257000.000000001	257000.000000001\\
-56000	-56000\\
-384000.000000002	-384000.000000002\\
165000	165000\\
310000	310000\\
-364000.000000001	-364000.000000001\\
162999.999999998	162999.999999998\\
93000.0000000026	93000.0000000026\\
-239999.999999998	-239999.999999998\\
-71000.0000000024	-71000.0000000024\\
126000.000000002	126000.000000002\\
184999.999999995	184999.999999995\\
-147999.999999999	-147999.999999999\\
-35000.0000000001	-35000.0000000001\\
200000.000000001	200000.000000001\\
37000.0000000017	37000.0000000017\\
-292999.999999997	-292999.999999997\\
-37000.0000000017	-37000.0000000017\\
111999.999999999	111999.999999999\\
33999.9999999998	33999.9999999998\\
-291999.999999997	-291999.999999997\\
493999.999999996	493999.999999996\\
1000.00000000211	1000.00000000211\\
-257000.000000003	-257000.000000003\\
-54999.9999999997	-54999.9999999997\\
37000.0000000017	37000.0000000017\\
146000	146000\\
-35999.9999999978	-35999.9999999978\\
163999.999999998	163999.999999998\\
-310000.000000001	-310000.000000001\\
53000.0000000008	53000.0000000008\\
93000	93000\\
-91999.9999999996	-91999.9999999996\\
-54999.9999999997	-54999.9999999997\\
239000.000000001	239000.000000001\\
-2000.00000000244	-2000.00000000244\\
-144999.999999995	-144999.999999995\\
-8.88178419700125e-10	-8.88178419700125e-10\\
-17999.9999999998	-17999.9999999998\\
-20000.0000000022	-20000.0000000022\\
258000	258000\\
-331000	-331000\\
-15999.9999999982	-15999.9999999982\\
161999.999999998	161999.999999998\\
148000.000000001	148000.000000001\\
-256000.000000001	-256000.000000001\\
164000.000000001	164000.000000001\\
-37000.0000000026	-37000.0000000026\\
57000.0000000039	57000.0000000039\\
-351000.000000003	-351000.000000003\\
94000.0000000021	94000.0000000021\\
438999.999999998	438999.999999998\\
-129000.000000001	-129000.000000001\\
-292000	-292000\\
309999.999999998	309999.999999998\\
-199999.999999998	-199999.999999998\\
-202000	-202000\\
384000.000000001	384000.000000001\\
-109000	-109000\\
-183999.999999998	-183999.999999998\\
293999.999999999	293999.999999999\\
-348999.999999999	-348999.999999999\\
293999.999999999	293999.999999999\\
72999.9999999995	72999.9999999995\\
-403000	-403000\\
238000	238000\\
-36999.999999999	-36999.999999999\\
128999.999999999	128999.999999999\\
-92999.9999999991	-92999.9999999991\\
-90000.0000000016	-90000.0000000016\\
200999.999999998	200999.999999998\\
-146000	-146000\\
-239999.999999999	-239999.999999999\\
607000.000000002	607000.000000002\\
-259000.000000002	-259000.000000002\\
-198999.999999999	-198999.999999999\\
272999.999999997	272999.999999997\\
1.77635683940025e-09	1.77635683940025e-09\\
-494000.000000001	-494000.000000001\\
439999.999999999	439999.999999999\\
110000.000000001	110000.000000001\\
-513999.999999998	-513999.999999998\\
384999.999999997	384999.999999997\\
-127000	-127000\\
273000	273000\\
-146000.000000002	-146000.000000002\\
-126999.999999997	-126999.999999997\\
-74999.9999999993	-74999.9999999993\\
-35000.000000001	-35000.000000001\\
236000	236000\\
-71000.0000000015	-71000.0000000015\\
-93000.0000000026	-93000.0000000026\\
56000	56000\\
255000.000000001	255000.000000001\\
-310000	-310000\\
-1000.00000000211	-1000.00000000211\\
276000.000000002	276000.000000002\\
-185000.000000003	-185000.000000003\\
-254999.999999999	-254999.999999999\\
529999.999999999	529999.999999999\\
-345999.999999997	-345999.999999997\\
88999.9999999977	88999.9999999977\\
-15999.9999999973	-15999.9999999973\\
-73999.9999999989	-73999.9999999989\\
346999.999999995	346999.999999995\\
-163999.999999999	-163999.999999999\\
-220000.000000001	-220000.000000001\\
-165000	-165000\\
221000.000000002	221000.000000002\\
180999.999999999	180999.999999999\\
21000.0000000026	21000.0000000026\\
-58000.0000000043	-58000.0000000043\\
-271999.999999999	-271999.999999999\\
-111000.000000001	-111000.000000001\\
310999.999999999	310999.999999999\\
146000.000000003	146000.000000003\\
130000.000000001	130000.000000001\\
-367000.000000003	-367000.000000003\\
-147999.999999999	-147999.999999999\\
75999.9999999996	75999.9999999996\\
254000	254000\\
-36000.0000000014	-36000.0000000014\\
74000.0000000007	74000.0000000007\\
-366999.999999998	-366999.999999998\\
238999.999999997	238999.999999997\\
53000.0000000008	53000.0000000008\\
-198999.999999998	-198999.999999998\\
253999.999999999	253999.999999999\\
-15999.9999999973	-15999.9999999973\\
-258000.000000003	-258000.000000003\\
239000.000000003	239000.000000003\\
145000	145000\\
-364000	-364000\\
-75000.0000000028	-75000.0000000028\\
165000	165000\\
111000.000000003	111000.000000003\\
35999.9999999969	35999.9999999969\\
-239000	-239000\\
184000.000000001	184000.000000001\\
36999.9999999981	36999.9999999981\\
-256999.999999998	-256999.999999998\\
219999.999999999	219999.999999999\\
-201000	-201000\\
236999.999999998	236999.999999998\\
73999.9999999981	73999.9999999981\\
-146999.999999999	-146999.999999999\\
109999.999999999	109999.999999999\\
-439000.000000001	-439000.000000001\\
421000.000000002	421000.000000002\\
-73000.0000000031	-73000.0000000031\\
-73999.9999999998	-73999.9999999998\\
-146999.999999999	-146999.999999999\\
167000	167000\\
70999.9999999997	70999.9999999997\\
38000.0000000038	38000.0000000038\\
36999.9999999999	36999.9999999999\\
-75000.0000000037	-75000.0000000037\\
-91000.0000000002	-91000.0000000002\\
-71999.9999999992	-71999.9999999992\\
-111000.000000001	-111000.000000001\\
348000	348000\\
-73000.0000000004	-73000.0000000004\\
-275000	-275000\\
275000	275000\\
37000.0000000008	37000.0000000008\\
-73999.9999999989	-73999.9999999989\\
-54000.000000002	-54000.000000002\\
291000	291000\\
-253999.999999998	-253999.999999998\\
-74000.0000000007	-74000.0000000007\\
16999.9999999968	16999.9999999968\\
-70999.9999999979	-70999.9999999979\\
254000	254000\\
-16000	-16000\\
-167000.000000002	-167000.000000002\\
-235999.999999997	-235999.999999997\\
364999.999999997	364999.999999997\\
73000.0000000004	73000.0000000004\\
-401999.999999998	-401999.999999998\\
365999.999999999	365999.999999999\\
128000	128000\\
-147000.000000002	-147000.000000002\\
-108999.999999998	-108999.999999998\\
-55000.0000000006	-55000.0000000006\\
-220000.000000002	-220000.000000002\\
475999.999999999	475999.999999999\\
-183000.000000001	-183000.000000001\\
-73000.0000000013	-73000.0000000013\\
72999.9999999995	72999.9999999995\\
-110000	-110000\\
184000	184000\\
-239000.000000002	-239000.000000002\\
147000.000000002	147000.000000002\\
-18000.0000000016	-18000.0000000016\\
-292999.999999999	-292999.999999999\\
236999.999999998	236999.999999998\\
74000.0000000016	74000.0000000016\\
237999.999999999	237999.999999999\\
-750000	-750000\\
657000	657000\\
57000.0000000022	57000.0000000022\\
-146999.999999999	-146999.999999999\\
-458000.000000001	-458000.000000001\\
385000.000000002	385000.000000002\\
35999.9999999978	35999.9999999978\\
-165000.000000002	-165000.000000002\\
221000.000000004	221000.000000004\\
-258000.000000004	-258000.000000004\\
-144999.999999999	-144999.999999999\\
457000	457000\\
91000.0000000011	91000.0000000011\\
-382999.999999999	-382999.999999999\\
-1000.00000000122	-1000.00000000122\\
182999.999999997	182999.999999997\\
18000.0000000007	18000.0000000007\\
-107999.999999998	-107999.999999998\\
70999.9999999971	70999.9999999971\\
-237000.000000001	-237000.000000001\\
164999.999999998	164999.999999998\\
-146999.999999999	-146999.999999999\\
255999.999999999	255999.999999999\\
-144999.999999997	-144999.999999997\\
309999.999999998	309999.999999998\\
-164999.999999998	-164999.999999998\\
-89999.9999999998	-89999.9999999998\\
-130000.000000003	-130000.000000003\\
221000.000000001	221000.000000001\\
16999.9999999986	16999.9999999986\\
-253999.999999996	-253999.999999996\\
88999.9999999942	88999.9999999942\\
148000.000000002	148000.000000002\\
-19000.0000000019	-19000.0000000019\\
-329000	-329000\\
201000	201000\\
256000	256000\\
-109000	-109000\\
-183999.999999997	-183999.999999997\\
-256000.000000002	-256000.000000002\\
366999.999999998	366999.999999998\\
475000.000000001	475000.000000001\\
-714000.000000003	-714000.000000003\\
166000.000000001	166000.000000001\\
108000.000000001	108000.000000001\\
-164000.000000002	-164000.000000002\\
221000.000000004	221000.000000004\\
15999.9999999965	15999.9999999965\\
-125999.999999998	-125999.999999998\\
-19999.9999999978	-19999.9999999978\\
-146000	-146000\\
-36000.000000004	-36000.000000004\\
274000.000000002	274000.000000002\\
184000.000000001	184000.000000001\\
-184000.000000004	-184000.000000004\\
-36999.9999999972	-36999.9999999972\\
-237000.000000002	-237000.000000002\\
-19000.0000000001	-19000.0000000001\\
276000	276000\\
53000.0000000008	53000.0000000008\\
36999.9999999981	36999.9999999981\\
-182000	-182000\\
-36999.999999999	-36999.999999999\\
-56000.0000000018	-56000.0000000018\\
57000.0000000013	57000.0000000013\\
125999.999999999	125999.999999999\\
165999.999999997	165999.999999997\\
-55999.9999999983	-55999.9999999983\\
-308999.999999997	-308999.999999997\\
70999.9999999953	70999.9999999953\\
91999.9999999996	91999.9999999996\\
92000.0000000005	92000.0000000005\\
-220000.000000001	-220000.000000001\\
56000.0000000018	56000.0000000018\\
89999.9999999972	89999.9999999972\\
73999.9999999998	73999.9999999998\\
-999.999999999446	-999.999999999446\\
-364000.000000001	-364000.000000001\\
144000	144000\\
19999.9999999987	19999.9999999987\\
219000.000000003	219000.000000003\\
-145999.999999998	-145999.999999998\\
127999.999999997	127999.999999997\\
-91999.9999999978	-91999.9999999978\\
-238000	-238000\\
184000.000000001	184000.000000001\\
292000.000000002	292000.000000002\\
-54999.9999999988	-54999.9999999988\\
-365000.000000001	-365000.000000001\\
34999.9999999993	34999.9999999993\\
93000.0000000044	93000.0000000044\\
291999.999999996	291999.999999996\\
19000.0000000037	19000.0000000037\\
-403000.000000003	-403000.000000003\\
-37000.0000000026	-37000.0000000026\\
201000	201000\\
38000.000000002	38000.000000002\\
-238999.999999999	-238999.999999999\\
-36999.9999999999	-36999.9999999999\\
185000	185000\\
107000	107000\\
75999.9999999978	75999.9999999978\\
-220999.999999999	-220999.999999999\\
-36999.9999999964	-36999.9999999964\\
54999.9999999979	54999.9999999979\\
73999.9999999989	73999.9999999989\\
17999.999999998	17999.999999998\\
55000.0000000006	55000.0000000006\\
-74000.0000000025	-74000.0000000025\\
-107999.999999998	-107999.999999998\\
-93999.9999999976	-93999.9999999976\\
129999.999999996	129999.999999996\\
236999.999999999	236999.999999999\\
-126999.999999997	-126999.999999997\\
-330999.999999999	-330999.999999999\\
422000	422000\\
-201999.999999999	-201999.999999999\\
-72000.0000000009	-72000.0000000009\\
-1999.999999998	-1999.999999998\\
221999.999999998	221999.999999998\\
88999.9999999995	88999.9999999995\\
-417999.999999998	-417999.999999998\\
252999.999999998	252999.999999998\\
113000.000000001	113000.000000001\\
-149000.000000001	-149000.000000001\\
-219000.000000001	-219000.000000001\\
184000.000000003	184000.000000003\\
182999.999999998	182999.999999998\\
-184999.999999999	-184999.999999999\\
186000	186000\\
-167000.000000001	-167000.000000001\\
-2.66453525910038e-09	-2.66453525910038e-09\\
-126999.999999996	-126999.999999996\\
329999.999999998	329999.999999998\\
-350000.000000001	-350000.000000001\\
221000.000000004	221000.000000004\\
-17999.9999999989	-17999.9999999989\\
-385000	-385000\\
678000.000000002	678000.000000002\\
-1000.00000000033	-1000.00000000033\\
-365000	-365000\\
-184000.000000004	-184000.000000004\\
110000	110000\\
36999.9999999981	36999.9999999981\\
109000.000000001	109000.000000001\\
1000.00000000389	1000.00000000389\\
-36999.9999999999	-36999.9999999999\\
220000	220000\\
-293000.000000002	-293000.000000002\\
-56000	-56000\\
148000	148000\\
181999.999999998	181999.999999998\\
-183000	-183000\\
-364999.999999999	-364999.999999999\\
437999.999999998	437999.999999998\\
-164000.000000001	-164000.000000001\\
72999.9999999995	72999.9999999995\\
146000	146000\\
-35999.9999999987	-35999.9999999987\\
-110000	-110000\\
-238000	-238000\\
202000.000000002	202000.000000002\\
-1000.00000000122	-1000.00000000122\\
91999.9999999996	91999.9999999996\\
-18999.9999999983	-18999.9999999983\\
93000	93000\\
-130000.000000001	-130000.000000001\\
222000.000000002	222000.000000002\\
-222000.000000005	-222000.000000005\\
-420000	-420000\\
403000.000000001	403000.000000001\\
366000.000000002	366000.000000002\\
-623000.000000001	-623000.000000001\\
183999.999999998	183999.999999998\\
127000.000000002	127000.000000002\\
275000	275000\\
-163999.999999998	-163999.999999998\\
-92000.0000000014	-92000.0000000014\\
-403000	-403000\\
329999.999999999	329999.999999999\\
17999.9999999989	17999.9999999989\\
200999.999999999	200999.999999999\\
-383999.999999999	-383999.999999999\\
256000	256000\\
-183000.000000002	-183000.000000002\\
166000	166000\\
-112000	-112000\\
-127000	-127000\\
385000	385000\\
-330000.000000001	-330000.000000001\\
108999.999999998	108999.999999998\\
147000.000000001	147000.000000001\\
129000	129000\\
-386000	-386000\\
-35000.0000000001	-35000.0000000001\\
329000.000000001	329000.000000001\\
-166000.000000004	-166000.000000004\\
38000.0000000038	38000.0000000038\\
-238000.000000001	-238000.000000001\\
255999.999999998	255999.999999998\\
163999.999999998	163999.999999998\\
-181999.999999999	-181999.999999999\\
-147000	-147000\\
-164000.000000001	-164000.000000001\\
219000	219000\\
182000.000000001	182000.000000001\\
75999.9999999987	75999.9999999987\\
-350999.999999998	-350999.999999998\\
20999.9999999981	20999.9999999981\\
89999.999999999	89999.999999999\\
274999.999999999	274999.999999999\\
-219999.999999999	-219999.999999999\\
-146000.000000001	-146000.000000001\\
109999.999999999	109999.999999999\\
91000.0000000011	91000.0000000011\\
-110000.000000001	-110000.000000001\\
56000.0000000009	56000.0000000009\\
126999.999999999	126999.999999999\\
-164000.000000001	-164000.000000001\\
-146999.999999998	-146999.999999998\\
237999.999999998	237999.999999998\\
-145999.999999999	-145999.999999999\\
17999.9999999989	17999.9999999989\\
-90999.9999999984	-90999.9999999984\\
438999.999999999	438999.999999999\\
-275000	-275000\\
-257000.000000001	-257000.000000001\\
478000.000000004	478000.000000004\\
-92000.0000000005	-92000.0000000005\\
-184000.000000002	-184000.000000002\\
35999.9999999987	35999.9999999987\\
-126000	-126000\\
162999.999999998	162999.999999998\\
-53999.9999999976	-53999.9999999976\\
-275000.000000004	-275000.000000004\\
733000.000000003	733000.000000003\\
-478000	-478000\\
-254000	-254000\\
273999.999999996	273999.999999996\\
200000.000000002	200000.000000002\\
-108000.000000001	-108000.000000001\\
-1000.00000000122	-1000.00000000122\\
-310999.999999997	-310999.999999997\\
0	0\\
311999.999999998	311999.999999998\\
181000	181000\\
-198999.999999997	-198999.999999997\\
-167000.000000003	-167000.000000003\\
112000.000000002	112000.000000002\\
-19000.0000000019	-19000.0000000019\\
-55999.9999999991	-55999.9999999991\\
55999.9999999983	55999.9999999983\\
127000.000000001	127000.000000001\\
-70999.9999999979	-70999.9999999979\\
106999.999999998	106999.999999998\\
-180999.999999997	-180999.999999997\\
-55000.0000000015	-55000.0000000015\\
-56000.0000000009	-56000.0000000009\\
404000.000000002	404000.000000002\\
-313000.000000003	-313000.000000003\\
-218000	-218000\\
311000	311000\\
89999.999999999	89999.999999999\\
-180999.999999998	-180999.999999998\\
-19000.0000000001	-19000.0000000001\\
474999.999999998	474999.999999998\\
-823000	-823000\\
458000.000000001	458000.000000001\\
127999.999999999	127999.999999999\\
-239000.000000001	-239000.000000001\\
93000	93000\\
-313000.000000001	-313000.000000001\\
478000.000000003	478000.000000003\\
109000.000000001	109000.000000001\\
-458000	-458000\\
37000.0000000008	37000.0000000008\\
238000	238000\\
-166000.000000001	-166000.000000001\\
149000.000000002	149000.000000002\\
-259000.000000002	-259000.000000002\\
368000.000000002	368000.000000002\\
-91999.9999999987	-91999.9999999987\\
17999.9999999989	17999.9999999989\\
-311000.000000002	-311000.000000002\\
182999.999999997	182999.999999997\\
18000.0000000007	18000.0000000007\\
-238000	-238000\\
92000.0000000014	92000.0000000014\\
495000	495000\\
-385999.999999999	-385999.999999999\\
-72000.0000000018	-72000.0000000018\\
199999.999999999	199999.999999999\\
-217999.999999998	-217999.999999998\\
327999.999999995	327999.999999995\\
-401999.999999997	-401999.999999997\\
164999.999999997	164999.999999997\\
-127999.999999999	-127999.999999999\\
292000.000000001	292000.000000001\\
-90999.9999999984	-90999.9999999984\\
-18999.9999999983	-18999.9999999983\\
-54000.0000000011	-54000.0000000011\\
-220000	-220000\\
402999.999999999	402999.999999999\\
-219999.999999999	-219999.999999999\\
-219000.000000001	-219000.000000001\\
273000	273000\\
183999.999999999	183999.999999999\\
18999.9999999992	18999.9999999992\\
-348000.000000002	-348000.000000002\\
-111000.000000001	-111000.000000001\\
733000.000000001	733000.000000001\\
-384000.000000002	-384000.000000002\\
-183999.999999995	-183999.999999995\\
73000.0000000004	73000.0000000004\\
-218000.000000003	-218000.000000003\\
90000.0000000025	90000.0000000025\\
110999.999999997	110999.999999997\\
384000	384000\\
-569000	-569000\\
94000.0000000003	94000.0000000003\\
310000	310000\\
-37000.0000000026	-37000.0000000026\\
37000.0000000026	37000.0000000026\\
-439000.000000001	-439000.000000001\\
92000.0000000005	92000.0000000005\\
162999.999999997	162999.999999997\\
367000.000000001	367000.000000001\\
-273999.999999999	-273999.999999999\\
-458000.000000003	-458000.000000003\\
476000.000000001	476000.000000001\\
146000	146000\\
-274999.999999998	-274999.999999998\\
-420000	-420000\\
127999.999999999	127999.999999999\\
641000	641000\\
-93000.0000000008	-93000.0000000008\\
-108000.000000001	-108000.000000001\\
-276000.000000001	-276000.000000001\\
476999.999999998	476999.999999998\\
-713999.999999998	-713999.999999998\\
328000	328000\\
-34999.9999999984	-34999.9999999984\\
16999.9999999995	16999.9999999995\\
-17000.0000000021	-17000.0000000021\\
238000	238000\\
-38000.0000000011	-38000.0000000011\\
-126999.999999999	-126999.999999999\\
54999.9999999988	54999.9999999988\\
-19000.0000000019	-19000.0000000019\\
19000.0000000001	19000.0000000001\\
-129000.000000002	-129000.000000002\\
-164999.999999996	-164999.999999996\\
496000	496000\\
146000.000000002	146000.000000002\\
-295000.000000004	-295000.000000004\\
-748000	-748000\\
731000	731000\\
146999.999999999	146999.999999999\\
-19000.000000001	-19000.000000001\\
-348000	-348000\\
-164000	-164000\\
238000.000000001	238000.000000001\\
257000.000000001	257000.000000001\\
-167000.000000002	-167000.000000002\\
56999.9999999995	56999.9999999995\\
35000.0000000028	35000.0000000028\\
-163000	-163000\\
346000	346000\\
-400999.999999999	-400999.999999999\\
-202000.000000003	-202000.000000003\\
291999.999999999	291999.999999999\\
92000.0000000014	92000.0000000014\\
-71999.9999999992	-71999.9999999992\\
162999.999999998	162999.999999998\\
-347000.000000001	-347000.000000001\\
274000.000000003	274000.000000003\\
92999.9999999982	92999.9999999982\\
-148000.000000001	-148000.000000001\\
-127000.000000001	-127000.000000001\\
-146999.999999999	-146999.999999999\\
366000	366000\\
18000.0000000007	18000.0000000007\\
-493000	-493000\\
182000.000000001	182000.000000001\\
494000	494000\\
-256000.000000003	-256000.000000003\\
-255000	-255000\\
-20000.0000000013	-20000.0000000013\\
532000.000000001	532000.000000001\\
-73999.9999999998	-73999.9999999998\\
-274999.999999998	-274999.999999998\\
-493000	-493000\\
328999.999999996	328999.999999996\\
348000.000000001	348000.000000001\\
-238999.999999999	-238999.999999999\\
203000.000000002	203000.000000002\\
16999.9999999986	16999.9999999986\\
92000.0000000032	92000.0000000032\\
-348000.000000002	-348000.000000002\\
-90999.9999999966	-90999.9999999966\\
347000.000000001	347000.000000001\\
-109000.000000001	-109000.000000001\\
-146999.999999998	-146999.999999998\\
109999.999999999	109999.999999999\\
37000.0000000035	37000.0000000035\\
-202000.000000001	-202000.000000001\\
476000.000000001	476000.000000001\\
-146000	-146000\\
-92000.0000000014	-92000.0000000014\\
-181999.999999999	-181999.999999999\\
-112000.000000002	-112000.000000002\\
222000	222000\\
-20000.0000000013	-20000.0000000013\\
-236999.999999999	-236999.999999999\\
548999.999999998	548999.999999998\\
-348000	-348000\\
-55000.0000000006	-55000.0000000006\\
422000.000000001	422000.000000001\\
-478000.000000002	-478000.000000002\\
-16000	-16000\\
235999.999999999	235999.999999999\\
203000.000000002	203000.000000002\\
-202000.000000002	-202000.000000002\\
-312000.000000001	-312000.000000001\\
312000	312000\\
-494000.000000001	-494000.000000001\\
567000.000000003	567000.000000003\\
129000	129000\\
-295000.000000003	-295000.000000003\\
3000.00000000455	3000.00000000455\\
-76000.0000000014	-76000.0000000014\\
294000	294000\\
92000.0000000023	92000.0000000023\\
-238000.000000002	-238000.000000002\\
54000.0000000003	54000.0000000003\\
-347000.000000004	-347000.000000004\\
128000.000000002	128000.000000002\\
53999.9999999976	53999.9999999976\\
258000	258000\\
-167000	-167000\\
-161999.999999998	-161999.999999998\\
107999.999999997	107999.999999997\\
55000.0000000015	55000.0000000015\\
-73000.0000000022	-73000.0000000022\\
238000.000000001	238000.000000001\\
36999.9999999972	36999.9999999972\\
-274999.999999998	-274999.999999998\\
-18000.0000000016	-18000.0000000016\\
-36999.9999999964	-36999.9999999964\\
18000.0000000007	18000.0000000007\\
-54000.0000000011	-54000.0000000011\\
237999.999999999	237999.999999999\\
-74999.9999999975	-74999.9999999975\\
-199000.000000003	-199000.000000003\\
475000	475000\\
-514000	-514000\\
39000.0000000041	39000.0000000041\\
548000	548000\\
-292999.999999998	-292999.999999998\\
-219000	-219000\\
109000.000000001	109000.000000001\\
330000.000000003	330000.000000003\\
-384000.000000003	-384000.000000003\\
255000.000000001	255000.000000001\\
-365000.000000001	-365000.000000001\\
-164999.999999999	-164999.999999999\\
256000	256000\\
275000.000000002	275000.000000002\\
126999.999999997	126999.999999997\\
-235999.999999996	-235999.999999996\\
16999.9999999995	16999.9999999995\\
-184000.000000001	-184000.000000001\\
-51999.9999999987	-51999.9999999987\\
472999.999999999	472999.999999999\\
-328000.000000001	-328000.000000001\\
-54999.9999999997	-54999.9999999997\\
-238000	-238000\\
419999.999999998	419999.999999998\\
276000	276000\\
-567999.999999999	-567999.999999999\\
200999.999999997	200999.999999997\\
-274000.000000001	-274000.000000001\\
127000.000000002	127000.000000002\\
332000	332000\\
-58000.0000000007	-58000.0000000007\\
-89000.0000000013	-89000.0000000013\\
-185000	-185000\\
183999.999999997	183999.999999997\\
37000.0000000017	37000.0000000017\\
-129000.000000002	-129000.000000002\\
36999.999999999	36999.999999999\\
-8.88178419700125e-10	-8.88178419700125e-10\\
-91000.0000000002	-91000.0000000002\\
-147000.000000001	-147000.000000001\\
219000	219000\\
166000	166000\\
71999.9999999974	71999.9999999974\\
-272999.999999999	-272999.999999999\\
52999.999999999	52999.999999999\\
-126000	-126000\\
34000.0000000007	34000.0000000007\\
168000	168000\\
-21000.0000000008	-21000.0000000008\\
109999.999999999	109999.999999999\\
-566000.000000001	-566000.000000001\\
384000.000000002	384000.000000002\\
347000	347000\\
18999.9999999983	18999.9999999983\\
-640000	-640000\\
-38999.9999999979	-38999.9999999979\\
495999.999999998	495999.999999998\\
-108999.999999999	-108999.999999999\\
89999.999999999	89999.999999999\\
-457999.999999998	-457999.999999998\\
220999.999999998	220999.999999998\\
456999.999999998	456999.999999998\\
-145999.999999996	-145999.999999996\\
-476000.000000003	-476000.000000003\\
-202999.999999996	-202999.999999996\\
624999.999999998	624999.999999998\\
199999.999999998	199999.999999998\\
-403000.000000002	-403000.000000002\\
-310999.999999997	-310999.999999997\\
293999.999999999	293999.999999999\\
345999.999999999	345999.999999999\\
-474999.999999998	-474999.999999998\\
128000.000000001	128000.000000001\\
385000.000000002	385000.000000002\\
-275000.000000001	-275000.000000001\\
-348000.000000001	-348000.000000001\\
531000	531000\\
-403000.000000002	-403000.000000002\\
92000.0000000023	92000.0000000023\\
274000	274000\\
36999.999999999	36999.999999999\\
-603999.999999999	-603999.999999999\\
54999.9999999997	54999.9999999997\\
512000.000000002	512000.000000002\\
74000.0000000016	74000.0000000016\\
-275000.000000001	-275000.000000001\\
-183000.000000002	-183000.000000002\\
329000.000000001	329000.000000001\\
-18000.0000000007	-18000.0000000007\\
-218999.999999998	-218999.999999998\\
-74000.0000000016	-74000.0000000016\\
202000.000000003	202000.000000003\\
-19000.0000000019	-19000.0000000019\\
-72999.9999999995	-72999.9999999995\\
219999.999999999	219999.999999999\\
-238000	-238000\\
-74000.0000000016	-74000.0000000016\\
439999.999999998	439999.999999998\\
-329999.999999999	-329999.999999999\\
1000.00000000033	1000.00000000033\\
72999.9999999977	72999.9999999977\\
-165999.999999999	-165999.999999999\\
-90000.0000000025	-90000.0000000025\\
365000.000000002	365000.000000002\\
-201000.000000002	-201000.000000002\\
-36999.9999999972	-36999.9999999972\\
184000.000000003	184000.000000003\\
72999.9999999995	72999.9999999995\\
-127999.999999999	-127999.999999999\\
181999.999999999	181999.999999999\\
-511000	-511000\\
-57000.0000000022	-57000.0000000022\\
771000.000000002	771000.000000002\\
-240000.000000004	-240000.000000004\\
-584000	-584000\\
273999.999999999	273999.999999999\\
310000.000000001	310000.000000001\\
-271999.999999999	-271999.999999999\\
15999.9999999973	15999.9999999973\\
421000.000000002	421000.000000002\\
-345999.999999997	-345999.999999997\\
-184000.000000001	-184000.000000001\\
421000.000000001	421000.000000001\\
-257000.000000001	-257000.000000001\\
-126999.999999998	-126999.999999998\\
54999.9999999997	54999.9999999997\\
126999.999999996	126999.999999996\\
-17999.9999999962	-17999.9999999962\\
220999.999999999	220999.999999999\\
-277000.000000001	-277000.000000001\\
-33999.999999998	-33999.999999998\\
-75000.0000000037	-75000.0000000037\\
183999.999999998	183999.999999998\\
72000.0000000018	72000.0000000018\\
-419000.000000001	-419000.000000001\\
382000.000000001	382000.000000001\\
221999.999999998	221999.999999998\\
-349999.999999999	-349999.999999999\\
-145000	-145000\\
438999.999999999	438999.999999999\\
-53999.9999999967	-53999.9999999967\\
-478000.000000005	-478000.000000005\\
184000.000000002	184000.000000002\\
221000	221000\\
15999.9999999973	15999.9999999973\\
-35000.0000000001	-35000.0000000001\\
-365999.999999998	-365999.999999998\\
127000.000000002	127000.000000002\\
72999.9999999995	72999.9999999995\\
166000.000000001	166000.000000001\\
-127999.999999999	-127999.999999999\\
90999.9999999975	90999.9999999975\\
310000	310000\\
-382000	-382000\\
-478000.000000002	-478000.000000002\\
495000.000000001	495000.000000001\\
329000.000000001	329000.000000001\\
-309999.999999996	-309999.999999996\\
-275000.000000002	-275000.000000002\\
475000.000000003	475000.000000003\\
-109000.000000003	-109000.000000003\\
-108999.999999997	-108999.999999997\\
-57000.0000000048	-57000.0000000048\\
-127000.000000001	-127000.000000001\\
384000.000000003	384000.000000003\\
-181999.999999999	-181999.999999999\\
-311999.999999998	-311999.999999998\\
310999.999999999	310999.999999999\\
111000.000000002	111000.000000002\\
217999.999999997	217999.999999997\\
-237000	-237000\\
-768999.999999998	-768999.999999998\\
860999.999999997	860999.999999997\\
163000	163000\\
-437000.000000001	-437000.000000001\\
-276999.999999997	-276999.999999997\\
220999.999999996	220999.999999996\\
74000.0000000025	74000.0000000025\\
363999.999999999	363999.999999999\\
-181000.000000001	-181000.000000001\\
-74000.0000000007	-74000.0000000007\\
-36999.999999999	-36999.999999999\\
-255999.999999998	-255999.999999998\\
256000	256000\\
-54000.0000000011	-54000.0000000011\\
-238999.999999998	-238999.999999998\\
548999.999999998	548999.999999998\\
-15999.9999999973	-15999.9999999973\\
-350999.999999999	-350999.999999999\\
-108000.000000002	-108000.000000002\\
165000	165000\\
255000.000000004	255000.000000004\\
-421000.000000001	-421000.000000001\\
312999.999999998	312999.999999998\\
17000.0000000003	17000.0000000003\\
-530999.999999998	-530999.999999998\\
383999.999999998	383999.999999998\\
1999.99999999978	1999.99999999978\\
-2000.00000000067	-2000.00000000067\\
275000.000000002	275000.000000002\\
74000.0000000007	74000.0000000007\\
-348000.000000001	-348000.000000001\\
-221000.000000003	-221000.000000003\\
442000.000000004	442000.000000004\\
-278000	-278000\\
-345000.000000001	-345000.000000001\\
583999.999999997	583999.999999997\\
-71999.9999999965	-71999.9999999965\\
-110000.000000001	-110000.000000001\\
-55000.0000000015	-55000.0000000015\\
109000	109000\\
-8.88178419700125e-10	-8.88178419700125e-10\\
-144999.999999998	-144999.999999998\\
218999.999999999	218999.999999999\\
-312000.000000001	-312000.000000001\\
587000	587000\\
-147000.000000001	-147000.000000001\\
-513000	-513000\\
-36000.0000000005	-36000.0000000005\\
511999.999999999	511999.999999999\\
-145999.999999999	-145999.999999999\\
129000	129000\\
-422999.999999997	-422999.999999997\\
293999.999999998	293999.999999998\\
91999.9999999996	91999.9999999996\\
-384999.999999998	-384999.999999998\\
-1000.000000003	-1000.000000003\\
350000	350000\\
-185000.000000002	-185000.000000002\\
-17000.0000000003	-17000.0000000003\\
438000.000000001	438000.000000001\\
39000.0000000041	39000.0000000041\\
-918000.000000002	-918000.000000002\\
531999.999999999	531999.999999999\\
347999.999999998	347999.999999998\\
-585999.999999998	-585999.999999998\\
-55000.0000000015	-55000.0000000015\\
531000	531000\\
-439000.000000002	-439000.000000002\\
237000.000000001	237000.000000001\\
330000.000000001	330000.000000001\\
-383000	-383000\\
-21000.0000000008	-21000.0000000008\\
-126000	-126000\\
71999.9999999983	71999.9999999983\\
349000	349000\\
-148000.000000001	-148000.000000001\\
};
\end{axis}

\begin{axis}[%
width=4.927cm,
height=3.484cm,
at={(6.484cm,0cm)},
scale only axis,
xmin=-1000000,
xmax=1000000,
xlabel style={font=\color{white!15!black}},
xlabel={$\delta^3 u(t)$},
ymin=-21484300000,
ymax=25390900000,
ylabel style={font=\color{white!15!black}},
ylabel={y(t)},
axis background/.style={fill=white},
title={C10, R = 0.5438},
axis x line*=bottom,
axis y line*=left
]
\addplot[only marks, mark=*, mark options={}, mark size=1.5000pt, color=mycolor1, fill=mycolor1] table[row sep=crcr]{%
x	y\\
-129000.000000001	2441300000\\
273999.999999998	5981600000\\
-143999.999999998	-8911200000\\
179999.999999999	9399500000\\
56999.9999999995	-8178900000\\
-458000.000000001	-1342600000\\
181999.999999999	9033100000\\
92999.9999999991	-6957900000\\
17000.0000000012	5126900000\\
311999.999999997	-3418100000\\
-401999.999999998	-2685300000\\
-130000	4394300000\\
-292000.000000001	-2685300000\\
420999.999999999	3539700000\\
347999.999999999	-365900000.000002\\
-237999.999999998	-3173900000\\
91999.9999999978	3173700000\\
-999.999999999446	-3540000000\\
-641000.000000002	-3173600000\\
826000.000000005	14159800000\\
-241000.000000005	-13305500000\\
-436999.999999997	488500000.000003\\
548999.999999999	10131400000\\
-185000.000000002	-9277000000\\
-216999.999999997	-976700000.000001\\
326999.999999996	10864200000\\
19999.9999999996	-9643300000\\
-366999.999999999	-2075599999.99999\\
274999.999999998	10864700000\\
73000.0000000031	-7202500000\\
-109000.000000003	-1220500000\\
-19999.9999999996	2929600000\\
-89999.9999999998	-2197200000\\
-8.88178419700125e-10	2441200000\\
181999.999999999	976900000.000001\\
-91000.0000000011	-4272600000\\
-18999.9999999992	3661900000\\
20000.0000000022	-2441200000\\
-166000.000000001	366300000\\
329999.999999999	4882600000\\
17999.9999999989	-5615200000\\
-275000.000000002	-1586700000\\
238000.000000002	5737100000\\
-273000.000000001	-5005000000\\
-37999.9999999994	2441800000\\
292999.999999997	3295500000\\
-165000	-6957700000\\
-126999.999999999	3661800000\\
235999.999999999	2197600000\\
21000.0000000017	-2685800000\\
-332000.000000002	-3417800000\\
496000	11108200000\\
-56000.0000000009	-10619900000\\
-293000	366000000\\
-90999.9999999993	3296200000\\
129000.000000001	1586500000\\
327999.999999997	399999.999999778\\
-346999.999999999	-5981700000\\
-146999.999999998	5005000000\\
366999.999999999	976500000\\
-110999.999999997	-3173800000\\
-164000.000000001	-122099999.999998\\
127999.999999999	3051800000\\
-183000.000000002	-3418000000\\
366000.000000001	5127100000\\
-201000.000000002	-6714200000\\
54000.0000000011	5981800000\\
-17000.0000000003	-5615500000\\
34999.9999999993	6714100000.00001\\
-52999.9999999973	-7568600000.00001\\
-1000.00000000389	6103700000\\
-1000.00000000033	-3784200000\\
-52999.9999999982	2075100000\\
-148000.000000001	-3906200000\\
148000	6713800000\\
198999.999999999	-854299999.999997\\
-15999.9999999965	-4638900000\\
54999.9999999988	5005100000.00001\\
-478000.000000003	-12817600000\\
239000.000000003	16235600000\\
165000.000000002	-4028700000\\
54999.9999999997	-1220300000.00001\\
-385000.000000001	-10498200000\\
129000.000000002	14770500000\\
364999.999999998	2929500000\\
-218999.999999997	-18432200000\\
17999.9999999989	13671500000\\
-53999.9999999976	-6103400000.00001\\
-129000.000000004	3174000000\\
-17999.9999999989	-1221000000\\
255999.999999998	4028500000\\
17999.9999999989	-5004900000\\
-274000.000000002	-244099999.999997\\
-36999.9999999972	2685500000\\
19000.0000000001	-976700000\\
310999.999999999	2319600000\\
-147000.000000002	-3662300000\\
-128000	-122100000.000002\\
165000.000000002	4394800000\\
-165000	-5859700000\\
257000.000000001	8301000000\\
-166000	-11352700000\\
148000.000000001	12329200000\\
-129000.000000001	-11108300000\\
-129000	4394400000\\
258000.000000001	4882700000\\
-148000.000000004	-8056400000\\
-90999.9999999975	3295700000\\
129000	2685700000\\
108999.999999999	-1586900000\\
-202000.000000002	-4150600000\\
112000.000000003	6714000000\\
-76000.0000000023	-6225500000\\
1999.99999999889	4394400000\\
128000	-122099999.999998\\
-184000	-4028100000\\
-8.88178419700125e-10	3417800000\\
-17000.0000000012	-1098800000\\
218000.000000002	3052200000\\
-72000.0000000001	-4761100000\\
201000	5737300000\\
-475999.999999999	-10741900000\\
19000.0000000019	8666700000\\
528999.999999996	5737400000\\
-363999.999999996	-15380700000\\
34999.9999999993	12206800000\\
276000.000000001	-2441400000\\
-275000	-6347400000\\
-221000.000000003	3906000000\\
202999.999999999	3784200000\\
-147999.999999998	-6103400000\\
258000.000000002	6225500000\\
53999.9999999994	-1586800000\\
37000.0000000035	-1465000000\\
-256000.000000003	-5493100000.00001\\
125999.999999999	9521500000\\
-69999.9999999994	-5981500000\\
15999.9999999973	2319500000\\
183999.999999998	1220399999.99999\\
-201999.999999999	-4394100000\\
-36000.0000000005	3783700000\\
90999.9999999966	-1586499999.99999\\
-108999.999999997	487999999.999997\\
-38000.000000002	-610199999.999999\\
38000.0000000029	366000000.000001\\
365999.999999998	5737700000\\
-166000	-8911500000\\
-145000	1342900000\\
-19000.0000000028	2929800000\\
201000	610099999.999998\\
-126999.999999998	-3173600000\\
-221000.000000004	1830900000\\
-16999.9999999995	-1098500000\\
438000.000000001	5126900000\\
-218000.000000003	-7690500000\\
-56999.9999999995	4272600000\\
111000	122000000.000002\\
36999.9999999972	-854600000.000001\\
-36999.9999999981	-976499999.999998\\
-74000.0000000025	1220900000\\
-217999.999999997	-1220900000\\
-38000.0000000002	244100000\\
329000	2929900000\\
112000.000000003	-854799999.999998\\
107999.999999999	-243900000\\
-200000	-4638700000\\
-112000.000000002	3418000000\\
76000.0000000005	976299999.999998\\
-441999.999999998	-3295600000\\
532999.999999997	6469600000\\
-184000	-6713800000\\
-8.88178419700125e-10	3173800000\\
-110000	-1587100000\\
275000.000000001	4639100000\\
165000.000000002	-1465399999.99999\\
-294000	-7934000000\\
201999.999999999	10985800000\\
-364999.999999998	-12572800000\\
143999.999999998	12206800000\\
-15999.9999999973	-5615200000\\
236999.999999997	3662100000\\
-219999.999999998	-6225600000\\
-18000.0000000007	3784300000\\
92000.0000000023	610199999.999999\\
-128000	-2929700000\\
217999.999999997	6103799999.99999\\
-400999.999999999	-10620500000\\
329000.000000003	10742400000\\
182999.999999996	4.2632564145606e-06\\
-109999.999999999	-8789200000.00001\\
-293000.000000003	1465100000.00001\\
-127999.999999997	3173500000\\
348999.999999999	4150700000\\
-39000.0000000015	-5981700000\\
39000.0000000024	3052000000.00001\\
-20000.0000000005	-1709200000\\
146999.999999999	2441500000\\
-256000	-6958000000\\
-18999.9999999983	5004900000\\
56000.0000000009	1586900000\\
274000.000000001	3662000000.00001\\
-110000.000000002	-11352400000\\
-458000.000000001	2441400000\\
385000.000000002	10253800000\\
147000.000000001	-7690300000\\
-440000.000000005	-3784200000\\
201000.000000001	8910900000\\
146999.999999998	-3173500000\\
-54999.9999999988	-2929800000\\
-202000.000000003	366199999.999998\\
202000.000000001	4760600000\\
-108999.999999998	-6103300000\\
-112000.000000003	4394400000\\
350000.000000001	366300000.000003\\
-37999.9999999985	-2197300000\\
-237000.000000002	-2685600000\\
-74000.0000000016	2563500000\\
-36999.999999999	1098800000\\
568999.999999998	7323800000.00001\\
-166000.000000001	-14037600000\\
-238000	4150200000\\
-164000.000000002	1830900000\\
219000	3418100000\\
93000.0000000008	-4638600000\\
-204000.000000001	-100000.000001899\\
-34000.0000000007	2197200000\\
144000	-610099999.999998\\
-15999.9999999991	121899999.999999\\
200000.000000001	1464700000\\
-493000	-7446000000\\
290999.999999997	10009600000\\
166000	-3540000000\\
-92000.0000000005	-1709000000\\
-145999.999999998	-1831100000.00001\\
-73999.9999999989	3051900000\\
422999.999999998	5981300000\\
-38999.9999999997	-11718700000\\
-548000.000000003	2563500000\\
219000	6347600000\\
999.999999999446	-4028200000\\
218999.999999999	2075100000\\
-17999.9999999998	-2441400000\\
-330000.000000003	-2197300000\\
293000.000000002	7202200000\\
73999.9999999998	-4272500000\\
-311999.999999998	-3784000000\\
238999.999999999	7934200000\\
-38000.000000002	-5859000000\\
-109999.999999999	976400000\\
240000.000000002	4272400000\\
-295000.000000003	-7568200000\\
221000.000000002	6469500000\\
-165000.000000001	-2807300000\\
-220999.999999997	-1831400000\\
440999.999999997	6103700000\\
-182999.999999999	-6225500000\\
-38000.0000000011	3051600000\\
385999.999999999	4394500000\\
-130000	-9765400000\\
-273000	2807400000.00001\\
-17999.9999999998	3784200000\\
-94000.0000000012	-2929500000\\
461000	4394300000\\
-351000.000000001	-5981400000\\
-34000.0000000007	2441600000\\
-2000.00000000067	-122199999.999999\\
275999.999999999	4638500000\\
-238999.999999999	-10497800000\\
349000.000000001	14526300000\\
-258000.000000004	-15136700000\\
148000	9643400000\\
-237999.999999998	-5004700000\\
90000.0000000007	3662100000\\
-53000.0000000017	-2319500000\\
-92999.9999999955	1220800000\\
-18000.0000000007	-1342600000\\
292999.999999997	3051400000\\
-108999.999999997	-2563100000\\
-1000.000000003	121700000\\
-236999.999999998	-2196900000\\
437999.999999998	8544700000\\
-36000.0000000014	-8056700000\\
-348000	-2197100000\\
147000.000000001	7568400000\\
-55000.0000000006	-4638900000\\
219000.000000001	5249199999.99999\\
56000.0000000009	-5737400000\\
-111000.000000001	244199999.999999\\
-164000.000000001	732400000.000003\\
-91000.0000000011	1587000000\\
71000.0000000015	-610400000\\
93999.9999999985	854300000\\
53999.9999999994	-1342600000\\
-165999.999999998	-610199999.999999\\
-34000.0000000025	1098400000\\
309000	2441400000\\
-91000.0000000019	-4882700000\\
-163999.999999998	2197300000\\
309999.999999997	3173700000\\
-127000	-5737300000\\
-458000	200000.000003797\\
220000	5004600000\\
382999.999999998	488400000.000001\\
-272999.999999999	-5615000000\\
144999.999999998	4516100000\\
-16999.9999999968	-2074700000\\
-293000.000000001	-2808000000\\
73000.0000000004	4639100000\\
255999.999999997	1708400000\\
54000.000000002	-3539500000\\
-454999.999999999	-6469900000\\
454999.999999997	16845500000\\
-126999.999999998	-17089600000\\
-201000.000000001	8300800000\\
201000	854200000.000006\\
53999.9999999985	-1952800000\\
-126999.999999999	-2319400000\\
-237999.999999997	-732699999.999998\\
201000	6714200000\\
311999.999999998	1586900000\\
-2000.00000000156	-9521799999.99999\\
-419000	-732099999.999997\\
54000.0000000011	8666900000\\
36999.999999999	-4882800000\\
72999.9999999968	1953000000\\
146000.000000001	2075200000\\
-182000.000000001	-7568000000\\
-130000	4882300000\\
222000.000000004	2441700000\\
144999.999999998	-854399999.999997\\
-182000.000000001	-5981800000\\
-221000	3052100000\\
18999.9999999983	1586700000\\
383999.999999999	5371300000\\
-163999.999999995	-11108600000\\
164999.999999998	9277300000\\
-295000.000000001	-8422500000\\
3000.000000001	4760400000\\
271999.999999998	2685600000\\
-126000	-5371000000\\
-129999.999999996	1709000000\\
19000.000000001	2075100000\\
-310999.999999999	-3662000000\\
365999.999999999	3784000000\\
90999.9999999984	-244000000\\
111000.000000002	-366100000.000003\\
-257000.000000003	-5005200000\\
-18000.0000000007	5737600000\\
311000.000000001	976400000.000002\\
-348000.000000002	-6835900000\\
55000.0000000006	5859400000\\
165000.000000001	-1098700000\\
-256999.999999998	-2075000000\\
-36000.0000000031	609999999.999997\\
238000	3174300000\\
275000	731900000.000001\\
-348000.000000002	-8178300000\\
-55999.9999999991	6225400000\\
-36000.0000000022	-1709000000\\
275000	2685700000\\
-531000	-5615500000\\
220000.000000001	4272800000\\
547999.999999997	6225200000\\
-547999.999999998	-15746600000\\
127999.999999999	12450700000\\
108999.999999998	-1952800000.00001\\
54999.9999999997	-1587099999.99999\\
-71999.9999999965	-2319300000.00001\\
-202000.000000003	976700000.000007\\
127000	2075100000\\
-89999.9999999998	-854599999.999999\\
54999.9999999988	366400000.000004\\
108000.000000001	976399999.999994\\
-216999.999999999	-4760700000\\
179999.999999997	7202300000\\
-14999.9999999961	-6591900000\\
197999.999999998	10253700000\\
-326999.999999998	-18188100000\\
36000.0000000014	15136400000\\
255999.999999998	-854199999.999997\\
-329999.999999998	-10254300000\\
221000.000000001	12329600000\\
163999.999999999	-5493700000\\
-239000.000000003	-4149900000\\
-162999.999999999	3783900000\\
-91999.9999999996	976500000.000003\\
456999.999999999	2685900000\\
-92000.0000000023	-5127400000\\
-290999.999999998	-976300000\\
254999.999999997	6713900000\\
19000.0000000019	-5737400000\\
-238999.999999998	-244200000.000004\\
91999.9999999996	4394600000\\
200999.999999998	-1953000000\\
-420000.000000001	-4150600000\\
273999.999999999	6347800000\\
255999.999999999	121999999.999999\\
-529999.999999998	-9399400000\\
163999.999999997	9399300000\\
402000.000000001	2441700000\\
-418999.999999998	-13061700000\\
382999.999999998	16235200000\\
-348000.000000002	-16235100000\\
220000.000000002	12573100000\\
1000.00000000033	-6225500000\\
-422000.000000004	-2319400000\\
402000.000000002	8300800000\\
-52999.9999999999	-4760700000\\
-258000.000000002	-3173900000\\
366999.999999999	8789000000\\
-18999.9999999983	-9887400000\\
-347000.000000001	5370700000\\
-129000	-1708700000\\
385999.999999999	3295800000\\
34000.0000000007	-2929700000\\
-14999.9999999961	976600000\\
-3000.00000000544	-976600000.000002\\
-162999.999999999	-976499999.999999\\
237000	4516500000\\
-146000.000000002	-5371000000\\
-55000.0000000006	1708900000\\
129000	2319500000\\
-313000	-4638900000\\
460000.000000003	8056700000\\
-149000.000000005	-8422600000\\
-327999.999999998	1220300000\\
183000.000000002	4150800000\\
310999.999999999	1586500000\\
-183000	-7079700000\\
-73000.0000000022	2563200000\\
-93000	1098800000\\
240000.000000001	1708900000\\
-129000.000000001	-3784200000\\
-183000.000000002	1098800000\\
128000.000000001	2563300000\\
109000	-2197300000\\
75000.000000001	1709100000\\
-238999.999999999	-4882700000\\
164999.999999999	6957700000\\
128000	-3173499999.99999\\
-311000	-3662400000\\
-183000.000000003	2685700000\\
329000.000000001	3662200000\\
147000.000000001	-854800000\\
-19000.0000000028	-3661800000\\
-200999.999999997	121999999.999996\\
146999.999999998	2807500000\\
-146999.999999998	-3173700000\\
91999.999999997	3539900000\\
-221000	-4394400000\\
294000	6591800000\\
37000.0000000008	-4516700000\\
-385000.000000002	-3540000000\\
238000.000000001	7568400000\\
-19000.0000000028	-4028400000\\
38000.0000000029	854599999.999999\\
35999.999999996	366100000.000003\\
-201999.999999998	-3295900000.00001\\
128999.999999997	4638900000\\
54000.0000000047	-1221000000.00001\\
275999.999999999	3662300000.00001\\
-147000.000000002	-10010000000\\
-220999.999999999	4028700000\\
2000.00000000067	3173500000\\
-166000.000000003	-2929600000\\
202000	4638800000\\
-165999.999999999	-7446500000\\
368000	9277600000\\
-111000.000000001	-6958300000\\
-238000.000000003	-854300000.000001\\
202000.000000001	4882700000\\
-73999.9999999998	-1952900000\\
73999.9999999972	-366499999.999995\\
-183999.999999998	-2441200000.00001\\
330000.000000001	8789000000.00001\\
-53999.9999999994	-9155400000.00001\\
-56000.0000000009	1953200000.00001\\
-147000.000000001	-1220400000\\
-16999.9999999995	4271900000\\
-8.88178419700125e-10	-2807100000\\
273000	4150000000\\
-218000	-7323900000\\
-999.999999998557	5248800000\\
257000.000000001	366200000.000003\\
-531000.000000002	-6713600000\\
146000.000000002	7446000000\\
494999.999999997	2685600000\\
-422000.000000001	-11718600000\\
-146000.000000001	5859200000\\
493999.999999999	9521600000.00001\\
-218999.999999998	-16845700000\\
-256999.999999999	8178700000\\
385000.000000002	4638500000\\
0	-4882600000\\
-110000	-3295900000\\
-184000.000000004	2441200000\\
-144999.999999996	610600000.000002\\
182000	2197100000\\
221000	854600000\\
-1000.00000000122	-4394700000\\
-18999.9999999975	3540299999.99999\\
1999.99999999889	-3662299999.99999\\
-185000.000000002	-488300000.000004\\
111000	4272600000.00001\\
-111000.000000002	-2563600000\\
37999.9999999994	1220800000\\
-91999.9999999987	-2319400000\\
145999.999999996	4150300000\\
183000.000000002	-1830800000\\
-437999.999999999	-4516800000\\
180999.999999996	6225700000\\
112000.000000002	-1465100000\\
-93000.0000000017	-2563100000\\
-36999.999999999	2441100000\\
111000.000000002	-487999999.999995\\
220000.000000001	4638300000\\
-257000.000000004	-11718300000\\
-73999.9999999998	7323800000\\
111000.000000001	2685700000\\
53999.9999999976	-4516400000\\
-218000	-1831300000\\
-130000	3784200000\\
365999.999999998	3784300000\\
203000.000000006	-1465000000\\
-183000.000000004	-9643400000\\
-167000	6835900000\\
111999.999999999	1220500000\\
-312000	-2441000000\\
-54999.9999999997	1830700000\\
365999.999999997	488500000.000001\\
-90999.9999999993	-2930000000\\
164000	4272900000\\
-54000.0000000029	-3906600000\\
-164999.999999998	488500000.000002\\
35999.9999999987	1342600000\\
55000.0000000006	-121900000.000001\\
-54000.000000002	-1343000000\\
-92999.9999999982	976800000.000001\\
19999.9999999978	-199999.999999889\\
255000.000000001	2197600000\\
-53999.9999999994	-2930200000\\
-166000.000000001	-1098200000\\
19999.9999999996	2074999999.99999\\
108000	2197400000\\
-71999.9999999974	-5249300000\\
92000.0000000014	5981700000\\
89999.9999999972	-3173799999.99999\\
-90999.9999999993	-2563700000\\
-199999.999999999	1587100000.00001\\
-131000	-1098700000\\
296000	7690500000\\
236000	-3418200000\\
-218999.999999997	-6835700000.00001\\
164999.999999996	4760800000\\
-347999.999999998	-1221100000\\
-73999.9999999998	2197800000\\
311999.999999998	-244499999.999999\\
-438999.999999999	-3051800000\\
292000.000000002	3540400000\\
165000.000000001	487900000\\
-55000.0000000006	-2807400000\\
73999.9999999989	1098500000\\
-183999.999999997	-1342600000\\
-109000	488099999.999999\\
127000	2929800000\\
1000.00000000033	-3296000000\\
128000	2197500000\\
-74000.0000000007	-2319700000\\
-273000	244499999.999999\\
-111000.000000002	-244399999.999997\\
566999.999999999	6469800000\\
-35999.999999996	-7690300000\\
999.999999999446	5248800000\\
-349000	-9887400000\\
-37000.0000000017	9399000000.00001\\
422000.000000004	3906600000\\
-183000.000000001	-12085000000\\
90999.9999999984	9887499999.99998\\
-201999.999999997	-9155099999.99999\\
1999.99999999889	7079999999.99999\\
272000.000000002	1709100000\\
-254000	-9155500000\\
164000.000000001	9033400000\\
-165000.000000004	-5859400000\\
-366000	243999999.999999\\
749999.999999999	9399700000\\
-346999.999999998	-13061900000\\
-92000.0000000041	6958300000\\
201000.000000003	-100000.00000332\\
-145000.000000001	-2441400000\\
53999.9999999976	1342800000\\
-36999.999999999	-122199999.999997\\
54999.9999999988	199999.999998823\\
38000.0000000011	1830900000\\
52999.9999999982	-2807600000\\
-292000.000000002	-2197100000\\
476000.000000003	8910900000\\
-92000.0000000023	-9643500000\\
-602999.999999999	2563700000\\
182000.000000002	3295600000\\
548999.999999999	244300000\\
-163000.000000001	-4272500000\\
-532999.999999998	-366199999.999999\\
164999.999999999	5004900000\\
312999.999999999	-2075300000\\
-112000	-1220500000\\
129999.999999999	1952800000\\
-258000.000000002	-4760300000\\
203000.000000002	8910800000\\
146000.000000001	-8056600000\\
-459000.000000003	-610299999.999999\\
496000	9765700000\\
-239000.000000001	-10253900000\\
-53999.9999999994	3295600000\\
72000.0000000009	1587200000\\
-292999.999999999	-2807600000\\
367999.999999998	4638600000\\
-240000	-5981600000\\
-90000.0000000016	4150700000\\
457000	976199999.999999\\
-92000.0000000032	-2807200000\\
-218999.999999998	-1831400000\\
128000	6103600000\\
-165000.000000004	-7934400000\\
220000.000000003	9643300000\\
-110000.000000003	-9155100000\\
-17999.999999998	5248900000\\
72000.0000000027	-976300000\\
-181000.000000003	-1709300000\\
181000.000000004	2441500000\\
-200000.000000001	-2807600000\\
201000.000000001	5127100000\\
109000.000000001	-5249300000\\
-237000.000000005	854799999.999999\\
147000.000000003	2685300000\\
-183999.999999999	-3662100000\\
-219999.999999998	854600000.000002\\
384999.999999995	4272500000\\
329000.000000001	-100000.000004385\\
-401000	-9521400000\\
70999.9999999988	9277100000\\
-365000	-6591400000\\
439000	8544600000\\
109999.999999999	-4760600000\\
-439999.999999999	-5127000000\\
313000.000000001	9521500000\\
-276000	-8056600000\\
365999.999999998	8544900000\\
-219000	-10131900000\\
-183999.999999999	7202100000\\
202999.999999999	-2807300000\\
125999.999999999	3539600000\\
-126000	-6469400000\\
-56000.0000000009	5737200000\\
-109999.999999999	-4150400000\\
328999.999999999	5859300000\\
-309999.999999997	-8666700000\\
37000.0000000008	7445800000\\
418999.999999997	1343200000\\
-143999.999999997	-6469899999.99999\\
-167000.000000002	-488199999.999999\\
-290999.999999998	2441200000\\
217999.999999995	3052100000\\
330000.000000002	365900000.000003\\
-400999.999999996	-7690400000\\
344999.999999994	9643899999.99999\\
-15999.9999999973	-7324700000\\
-385999.999999997	244399999.999997\\
-201000.000000001	1831100000\\
567999.999999999	6713700000\\
18000.0000000016	-10253800000\\
-439000.000000001	2075300000\\
402999.999999999	6957700000\\
-55999.9999999991	-7446000000\\
-55000.0000000033	3539800000\\
-126999.999999996	-5492800000.00001\\
-37000.0000000008	7812100000\\
273999.999999997	-1098399999.99999\\
-144999.999999998	-6347800000\\
-2000.00000000333	6103700000\\
111000.000000001	-2075399999.99999\\
-256000	-2197100000\\
273999.999999998	5004900000\\
-220000	-4883100000\\
-145999.999999999	976900000.000005\\
348000.000000001	4150200000\\
-35999.9999999987	-3539900000\\
-38000.0000000038	1098500000.00001\\
110000	-976500000.000005\\
-127000	-1220700000\\
-166000.000000001	976400000\\
56000.0000000009	1587300000\\
53999.9999999976	-976900000\\
19000.0000000019	366299999.999999\\
17999.9999999998	-366300000.000003\\
183000	2441700000\\
-18000.0000000025	-3662400000\\
-532000	-5371000000\\
294000.000000003	12451100000\\
366999.999999999	-1464600000\\
-313000.000000002	-11719000000\\
-109999.999999999	8544900000\\
277000	2685900000\\
-113000	-7202700000\\
-161999.999999998	2319900000\\
181999.999999998	3661600000\\
-129000	-4882300000\\
1000.00000000122	2563000000\\
256999.999999999	3174200000\\
-277000.000000001	-8545300000.00001\\
112000.000000002	8423300000\\
0	-5249300000\\
34999.9999999966	3662000000\\
-181999.999999999	-5615000000\\
255999.999999998	9643600000\\
-146000.000000002	-11352800000\\
-91999.9999999996	7690600000\\
182999.999999999	-1587000000\\
147000.000000001	488400000.000001\\
-145999.999999998	-3662100000\\
-185000	2685300000\\
-126999.999999999	-976300000\\
-164000.000000002	1098500000\\
548000	2319400000\\
73999.9999999972	-2319300000\\
-999.999999998557	121900000.000007\\
-364999.999999999	-5004700000.00001\\
146000	8789000000\\
92000.0000000005	-5371100000\\
-1000.000000003	1830900000\\
-292000	-3905900000\\
164000.000000001	7568100000\\
127999.999999998	-6347600000\\
-182000	1831000000\\
183000.000000002	1709100000\\
-92000.0000000014	-2319400000\\
109000	1342800000\\
-291000.000000001	-2807600000\\
-74999.9999999957	2807400000\\
603999.999999997	4761100000\\
-419000	-11718900000\\
89999.9999999981	10742000000\\
-128000	-8178400000\\
184000.000000002	7812200000\\
-166000.000000003	-7812200000\\
256000	9887400000\\
-199999.999999997	-12206700000\\
17999.9999999989	9154900000\\
255999.999999998	-487999999.999997\\
-550000.000000001	-10498100000\\
458999.999999999	16967600000\\
-219999.999999999	-14648200000\\
35999.9999999987	7568300000\\
-18000.0000000016	-2319500000\\
220000.000000001	3906400000\\
-275000	-9521500000\\
36999.9999999981	10009800000\\
219000	-2197400000\\
-90999.9999999984	-4150300000\\
-36000.0000000022	2319400000\\
-129999.999999999	-1342900000\\
167000.000000002	4638800000\\
-20000.000000004	-4394700000\\
-34999.9999999975	732500000\\
-94000.0000000021	-610200000.000001\\
131000.000000002	3783900000\\
-295000.000000002	-7934200000\\
256000	10619600000\\
259000.000000002	-3905800000\\
-222000.000000003	-5127099999.99999\\
-91000.0000000002	4150399999.99999\\
237999.999999999	2441300000\\
-220000	-8056499999.99999\\
1000.00000000033	7934600000\\
35999.9999999969	-3418300000\\
-109999.999999999	122600000.000003\\
-55000.0000000006	-488700000.000001\\
184000	3906400000\\
34999.9999999993	-4516700000\\
74999.9999999993	4150700000\\
-20000.0000000005	-4150900000\\
-437999.999999999	-3661500000\\
365999.999999999	11596000000\\
16999.9999999995	-7445600000\\
75000.0000000028	2807099999.99999\\
-349000.000000005	-10009600000\\
403000.000000003	21606600000\\
127999.999999999	-20264000000\\
-401999.999999999	4394900000\\
-1000.000000003	4028100000\\
-183000	-3417900000\\
258000.000000001	5493000000\\
16000	-5004600000\\
201999.999999998	6103300000\\
-107999.999999998	-9277300000.00001\\
70999.9999999988	5981600000\\
-329000.000000001	-4761000000\\
17999.9999999998	6836100000\\
111000.000000002	-4882800000\\
219999.999999999	4028300000\\
-313000.000000001	-6713900000\\
164999.999999999	8300700000\\
130000.000000001	-6103400000\\
-532999.999999998	-610400000.000001\\
421999.999999996	6836100000\\
-54999.9999999988	-6103700000\\
91999.9999999987	2929600000\\
-92000.0000000014	-2441100000\\
-999.999999998557	2074900000\\
-52999.999999999	-2319200000\\
199999.999999997	6103600000\\
-309999.999999997	-11352800000\\
107999.999999996	11597000000\\
276000.000000003	-3784400000\\
-273999.999999999	-4638700000\\
52999.9999999973	5859600000\\
-35000.0000000028	-3784300000\\
-128999.999999999	1220600000\\
402999.999999998	5493200000.00001\\
-165000	-9521300000.00001\\
-220000	3051600000\\
185000.000000001	4638600000\\
107999.999999998	-3783900000\\
-403000.000000001	-3784500000\\
129000	7812600000\\
330000	-1342600000\\
-220999.999999997	-5859600000\\
-199999.999999999	2319500000\\
511999.999999997	10986100000\\
-184000	-19286700000\\
-145000	13183100000\\
-91999.9999999996	-7201799999.99999\\
-37000.0000000026	5248899999.99999\\
457000.000000001	4882800000\\
-290999.999999998	-13793900000\\
-239000.000000002	6713900000\\
512999.999999999	7812300000\\
-148000	-12450800000\\
-382000	3173400000\\
326999.999999997	6836300000\\
-52999.999999999	-6958300000\\
-92000.0000000005	1465100000\\
91000.0000000037	2197000000\\
-19000.0000000001	-2197000000\\
-126000.000000004	366100000\\
72000.0000000001	488100000.000001\\
72999.9999999995	854900000.000001\\
-73000.0000000013	-2319800000\\
184000	4028800000\\
-165999.999999999	-5615799999.99999\\
-19000.0000000019	3540599999.99999\\
204000.000000003	1708700000.00001\\
-57000.0000000022	-2929800000\\
-183000.000000002	-2807300000\\
92000.0000000014	6469500000\\
-17999.9999999989	-4760700000\\
-128000.000000001	976600000.000003\\
364999.999999998	5981400000\\
-310000.000000002	-10986400000\\
109000.000000001	8301100000\\
36999.9999999999	-2685900000\\
-294000.000000001	-3417900000\\
423000.000000004	10864500000\\
-147000	-13306000000\\
-37000.0000000017	8911300000\\
145999.999999998	-2441400000\\
-90999.9999999993	-3051700000\\
-110000.000000002	2685400000\\
-255999.999999999	-1464800000\\
420000	5981500000\\
-162999.999999998	-9155200000\\
144999.999999998	9155000000\\
-162999.999999998	-9399200000\\
-149000.000000002	4516600000\\
257999.999999999	4638700000\\
256000.000000002	-3052000000\\
-330000.000000003	-6835599999.99999\\
19000.0000000019	8910900000\\
182999.999999999	-3662000000\\
-293000.000000001	-2807800000\\
35999.9999999978	4883200000\\
202000.000000003	-488799999.999996\\
-202000.000000003	-3905800000\\
19000.000000001	3417800000\\
-54999.9999999979	-1709100000\\
365999.999999999	6469900000\\
-219999.999999999	-10376100000\\
-71999.9999999992	3784400000.00001\\
236000.000000001	6469400000\\
-309999.999999998	-12939300000\\
183999.999999997	13061700000\\
-94000.0000000003	-9155600000\\
-88999.9999999977	3418100000\\
144999.999999995	2319600000\\
384000.000000001	2929200000\\
-638999.999999998	-16723200000\\
199999.999999997	19653000000\\
-109999.999999999	-12939200000\\
183999.999999999	9155100000\\
201000.000000002	-4028200000\\
-476000	-5005100000\\
73999.9999999981	5981800000\\
-38999.9999999997	-1098900000\\
516000.000000002	5859400000\\
-168000	-9155200000\\
-70999.9999999997	2807600000\\
-220999.999999998	-3784199999.99999\\
93000.0000000008	6835900000\\
218000.000000001	2685699999.99999\\
-200000.000000001	-13305800000\\
-20000.0000000005	11108300000\\
222000.000000002	-243800000.000006\\
-185000.000000002	-7690799999.99999\\
-164000	4150600000\\
348000.000000002	6957900000\\
-92000.0000000023	-11474500000\\
-291999.999999999	3784200000\\
345999.999999996	6713600000\\
-199999.999999999	-10375500000\\
55000.0000000015	8666400000\\
-55000.0000000006	-6713400000\\
18000.0000000007	4150300000\\
219999.999999999	2197100000\\
-56000.0000000009	-6347600000\\
-292000.000000002	1342900000\\
166000.000000005	3784100000\\
-3000.00000000455	-2319400000\\
131000.000000001	2197400000\\
-75000.000000001	-3784400000\\
-254999.999999999	-1586600000\\
327999.999999998	9765400000\\
75000.000000001	-7934500000\\
-496000.000000003	-4638800000\\
440000.000000001	15259000000\\
37000.0000000008	-12939500000\\
-128000.000000001	3906100000.00001\\
-1000.00000000122	-976399999.999998\\
-52999.9999999973	1464800000\\
-314000.000000001	-2685700000\\
404999.999999999	6958300000\\
89999.9999999972	-7080300000\\
-89999.9999999972	2563500000\\
-239000.000000003	-1953000000\\
-35999.9999999996	2441200000\\
273999.999999998	976800000\\
-73999.9999999989	-3418100000\\
2000.00000000422	2685600000\\
108999.999999996	-122199999.999995\\
-128999.999999997	-2319300000\\
257999.999999999	4761000000\\
-93000.0000000017	-6470100000\\
-458000.000000001	-243900000.000001\\
275999.999999999	9155200000\\
0	-9765700000\\
-148000.000000001	5615500000\\
294000	-366599999.999998\\
-238000.000000001	-4150000000\\
458000.000000001	10619800000\\
-349000.000000001	-16357300000\\
-201000	10131900000\\
441000.000000002	3540000000\\
-131000.000000001	-8545000000\\
-107000	2807700000\\
-148000.000000002	-244100000\\
36000.0000000014	2929500000\\
74999.9999999975	-3539800000\\
346000.000000002	8056500000\\
-255000	-13427700000\\
-165000	8544800000\\
91999.9999999987	-1952900000\\
-56000.0000000009	-244199999.999999\\
-54999.9999999997	243899999.999999\\
348999.999999999	7080500000\\
-110999.999999998	-13306100000\\
73999.9999999981	11352900000\\
-91999.9999999987	-9399699999.99999\\
-328999.999999997	4028500000\\
568000	5859300000\\
-606000.000000004	-12817300000\\
258000.000000002	14526300000\\
53999.9999999985	-10986300000\\
-238000	3906100000\\
202000	2075500000\\
109000.000000004	-1465200000\\
-145000	-2807300000\\
237000.000000002	7934400000\\
127999.999999997	-7446300000\\
-458000.000000001	-4516500000\\
56000.0000000009	11474500000\\
8.88178419700125e-10	-8788899999.99999\\
-38000.0000000038	6103300000\\
129000.000000001	-2929600000\\
220000.000000001	5615400000\\
-73999.9999999989	-10010100000\\
-365000.000000002	2563800000\\
127000.000000002	4150100000\\
293999.999999997	2685900000\\
-129999.999999999	-8789300000\\
-88999.9999999986	3906200000\\
70999.9999999971	1831100000\\
-163000.000000001	-2807400000\\
182000.000000001	2563200000\\
-347999.999999998	-3417900000\\
201999.999999998	4516700000\\
163999.999999999	-2441500000\\
73999.9999999998	1220800000\\
-182999.999999998	-3418100000\\
237999.999999997	4638700000\\
-164999.999999999	-4760700000\\
-348999.999999997	2075300000\\
19999.9999999996	1586700000\\
274000.000000001	-1220600000\\
54999.9999999953	122199999.999999\\
3.5527136788005e-09	487999999.999999\\
-37000.0000000044	-1830900000\\
-108999.999999999	1098700000\\
-19999.9999999978	-244199999.999998\\
332000	4638600000\\
-185000	-8666899999.99999\\
18999.9999999992	7812499999.99999\\
238000	-3173900000.00001\\
-440000.000000001	-5859400000\\
1000.00000000122	8301000000\\
54999.9999999988	-3906500000\\
218000	5493300000\\
-33999.9999999971	-6347800000\\
-94000.0000000038	976799999.999996\\
18999.9999999992	1342600000\\
-273999.999999999	-2197300000\\
311000.000000002	6347900000\\
237999.999999999	-4761100000\\
-346999.999999998	-2685300000\\
-204000.000000001	2685600000\\
203999.999999997	2075000000\\
219000	488299999.999997\\
-294000.000000001	-6835800000\\
258000	9399400000\\
-294000	-9521700000\\
36000.0000000005	6103900000\\
423000	3905800000\\
-441000.000000001	-12939100000\\
19000.0000000019	10742000000\\
255999.999999999	-976500000.000004\\
55000.0000000015	-244199999.999998\\
-35999.9999999978	-5004600000.00001\\
-184000.000000004	1586400000\\
129000.000000001	2808100000\\
-404000.000000003	-3296200000\\
404000.000000003	6103600000\\
-293000.000000003	-9277200000\\
309999.999999999	11840700000\\
20000.0000000005	-9765800000\\
-295000	1343000000\\
423000.000000001	6591900000\\
-494000	-10742500000\\
344999.999999999	11352700000\\
57999.9999999998	-7690300000\\
-366999.999999999	976200000\\
273000.000000001	3296300000\\
-163000.000000003	-3051900000\\
128000.000000001	1586700000\\
36000.0000000031	366499999.999998\\
-183000.000000002	-2929800000\\
17999.9999999998	3051800000\\
146999.999999998	-122199999.999999\\
-18000.0000000007	-1220500000\\
-220000.000000001	-1465000000\\
165000	3784100000\\
236999.999999997	-243800000\\
-199999.999999998	-4517000000\\
-221000.000000002	1343100000\\
165999.999999999	4394300000\\
292000.000000002	-976400000.000001\\
-311000.000000003	-6714099999.99999\\
329000	10742500000\\
-254999.999999999	-12085200000\\
-440000.000000001	5371300000\\
293000	3906000000\\
310000	-2075100000\\
-308999.999999999	-4638500000\\
273000.000000001	9032999999.99999\\
55999.9999999991	-7568300000\\
-166000.000000002	122100000\\
-164000	1098500000\\
35999.9999999996	2319600000\\
184000.000000001	-200000.000002376\\
72999.9999999986	-2563500000\\
-348999.999999997	-1220600000\\
202999.999999999	5981400000\\
-1000.00000000389	-5615200000\\
56000.0000000036	2319300000\\
-313000.000000001	-1708900000\\
130000	2441200000\\
217999.999999997	199999.999999534\\
75000.0000000002	-366300000\\
-478000	-5615200000\\
258000.000000002	8422800000\\
384000.000000002	366400000.000001\\
-531000.000000002	-12085300000\\
164999.999999998	13305900000\\
199999.999999999	-3662100000\\
-72000.0000000001	-4272500000\\
-291999.999999998	488200000\\
70999.9999999971	5249200000\\
405000.000000003	-976900000.000003\\
-460000.000000003	-6591400000\\
204000.000000001	8056500000\\
89000.0000000039	-4272600000\\
-255000.000000002	-244000000.000003\\
-36000.0000000005	976500000.000001\\
90000.0000000007	1342900000\\
256999.999999998	-366400000.000001\\
-274000.000000002	-2929600000\\
91000.0000000011	3296000000\\
128000.000000002	-122200000\\
55000.0000000015	-2197300000\\
-274999.999999999	-122000000.000001\\
-91000.0000000002	1587000000\\
184000	243999999.999998\\
-37999.9999999985	-610299999.999998\\
-256999.999999999	-2197300000\\
532999.999999998	7080200000\\
-56000	-7080200000\\
-238000.000000003	1098600000\\
-35999.9999999996	366300000\\
-38000.0000000011	1831100000\\
220999.999999999	121900000.000002\\
17000.0000000012	-1830800000\\
2000.00000000156	2563200000\\
-185000	-6469600000\\
19999.9999999969	6713900000\\
108000.000000001	-2319300000\\
-126999.999999999	-244299999.999993\\
-36999.9999999999	-854400000.000004\\
257000.000000001	6835900000\\
-74000.0000000016	-10253800000\\
-36999.9999999981	6225500000\\
-71999.9999999992	-4028300000\\
0	4394600000\\
-20000.0000000022	-3784500000\\
258000	6836400000\\
-349000.000000001	-11719000000\\
38000.0000000002	9765600000\\
126000	-3051600000\\
131000	2685300000\\
-241000	-7812300000\\
148999.999999998	10009800000\\
-20000.0000000005	-8178900000\\
74000.0000000016	5493300000\\
-348000	-6836000000\\
54999.9999999997	7690600000\\
439999.999999998	1098300000\\
-56999.9999999986	-7446000000\\
-419000	-99999.9999969248\\
457000	8422800000\\
-347000.000000001	-8789000000\\
-74999.9999999957	4028400000\\
329999.999999999	2319200000\\
-145000.000000002	-5615300000\\
-129999.999999996	4150700000\\
257999.999999997	-122400000\\
-312999.999999996	-4028200000\\
312999.999999997	6103600000\\
-19999.9999999969	-4272500000\\
-328000.000000001	-610500000\\
199999.999999998	3662300000\\
19999.9999999996	-2441500000\\
72000.0000000001	732399999.999999\\
-54999.999999997	-243999999.999998\\
-165000.000000001	-1465100000\\
293999.999999998	4394900000\\
-184000	-5981800000\\
-236999.999999999	2319400000\\
621999.999999999	7568700000\\
-294000	-12940000000\\
-163000	6592299999.99999\\
236999.999999999	1830700000\\
-1.77635683940025e-09	-3661900000\\
-493999.999999999	-1709200000\\
511999.999999999	8301200000\\
19000.0000000019	-7324800000\\
-458000.000000004	-365699999.999996\\
367000.000000002	5981200000\\
-112000	-5615300000\\
241000.000000002	4638900000\\
-131000	-4760900000\\
-127000.000000002	2075200000\\
-53999.9999999994	100000.000001188\\
-129000	-366400000.000001\\
384000.000000001	3174100000\\
-146000.000000001	-4883100000\\
-110000.000000002	1953300000\\
74000.0000000007	732300000.000001\\
272999.999999998	2930000000\\
-382999.999999996	-9399900000\\
127999.999999998	9277800000\\
183000	-1587200000\\
-184000.000000001	-4272400000\\
-254999.999999999	-1220700000\\
547999.999999997	14526400000\\
-328999.999999997	-21484300000\\
92000.0000000014	17578000000\\
-54999.9999999997	-11352600000\\
-54999.9999999988	6592000000\\
364999.999999996	2075100000\\
-199999.999999998	-9643700000\\
-239000.000000002	5127300000\\
-71999.9999999992	-122400000.000002\\
146000	3051800000\\
199999.999999999	-2197100000\\
2000.00000000244	610300000.000001\\
17999.999999998	-2929800000\\
-386000	-366099999.999998\\
-72000.0000000027	3295900000\\
366000.000000001	854400000.000002\\
146000.000000003	-122100000.000001\\
37999.9999999994	-1220400000\\
-276000.000000002	-4761200000\\
-183000.000000001	4028800000\\
36999.9999999981	854100000.000001\\
311000.000000002	2075300000\\
-55000.0000000015	-3173699999.99999\\
74000.0000000007	1953099999.99999\\
-348999.999999997	-8178900000\\
184999.999999999	12573500000\\
125999.999999998	-6225800000\\
-274000	-3906099999.99999\\
311000.000000001	11840700000\\
-71999.9999999983	-12206900000\\
-146999.999999998	2197100000\\
127999.999999998	7934700000\\
182000	-6469699999.99999\\
-364000	-3784200000\\
-75000.0000000028	6469500000\\
183999.999999998	-732199999.999999\\
54000.0000000011	366299999.999993\\
111000	-854699999.999997\\
-312000.000000002	-4882700000\\
238000.000000003	9155300000\\
18999.9999999975	-6225799999.99999\\
-256999.999999998	-1342600000\\
238999.999999999	6469900000\\
-258000.000000003	-7568800000\\
275000.000000002	8667500000\\
93000.0000000017	-4883300000\\
-146999.999999999	-2685200000\\
71999.9999999965	3784000000\\
-420000.000000001	-4028100000\\
403000.000000003	8056300000\\
-38000.0000000047	-7690200000\\
-89999.9999999963	3295900000\\
-129000.000000003	-2807800000\\
129000.000000001	4761100000\\
108999.999999997	-2441800000\\
-18999.9999999983	-243899999.999995\\
94000.0000000021	1708900000\\
-113000.000000004	-5004900000\\
3000.00000000278	4760900000\\
-240000.000000002	-2563800000\\
19000.0000000001	1709300000\\
311000	1220500000\\
-55000.0000000015	-2197000000\\
-310999.999999998	-2441800000\\
293999.999999999	6470100000\\
-1999.99999999978	-4272700000\\
18999.9999999983	976600000\\
-127000	-2075100000\\
290999.999999999	7934600000\\
-218000	-13305900000\\
-74000.0000000007	9033400000\\
-19000.0000000019	-2441400000\\
-52999.9999999982	976399999.999999\\
235999.999999999	2319600000\\
37999.9999999985	-4272700000\\
-221000.000000001	366300000\\
-217999.999999997	-488199999.999999\\
364999.999999997	5492900000\\
73000.0000000004	-3539700000\\
-364999.999999999	-5005100000\\
272999.999999999	10253900000\\
203000.000000003	-4516400000\\
-184000.000000002	-4882999999.99999\\
-53999.9999999994	3906199999.99999\\
-130000.000000003	200000.000003797\\
-145000	-1953200000\\
421000.000000001	7690300000\\
-201000.000000001	-10009600000\\
-999.999999999446	5371100000\\
72999.9999999968	-1098899999.99999\\
-35999.9999999987	1221100000\\
19000.0000000001	-3540199999.99999\\
72000.0000000018	6469400000\\
-127000.000000002	-9642900000\\
109000	9887000000\\
-92000.0000000023	-7201600000\\
-217999.999999997	1830800000\\
180999.999999999	2441400000\\
185000.000000002	488400000.000001\\
181999.999999998	610200000.000008\\
-749999.999999999	-14404300000\\
492999.999999998	25390900000\\
313000.000000002	-18188800000\\
-331000	3051900000\\
-365000.000000001	-1464800000\\
401999.999999997	8666900000\\
-35999.9999999996	-8178700000\\
-129000.000000002	2197300000\\
257000.000000002	3662100000\\
-349000.000000002	-8422800000\\
-34999.9999999975	5126800000\\
384000.000000002	5981600000\\
91000.0000000011	-5981500000\\
-365000.000000001	-6835900000\\
52999.9999999982	10375900000\\
56000.0000000018	-2685500000\\
165999.999999999	122000000.000001\\
-259000	-4638500000\\
203999.999999998	6835700000\\
-330999.999999999	-7324000000\\
255999.999999997	7812500000\\
-236999.999999999	-7324500000\\
292000.000000002	8911400000\\
-126999.999999998	-9887799999.99999\\
292999.999999999	14038200000\\
-185000	-19409300000\\
-90000.0000000016	12939400000\\
-91999.9999999952	-5004700000\\
165999.999999997	5981400000\\
88999.9999999995	-5859599999.99999\\
-271999.999999998	488600000\\
34999.9999999957	2807400000\\
220000	-365999999.999999\\
-73000.0000000013	-1709300000\\
-311000.000000001	-1342600000\\
201000.000000001	4028500000\\
255999.999999996	732100000.000001\\
-126999.999999996	-6591600000\\
-148000	4882800000\\
-274000.000000002	-3418100000\\
366999.999999998	5981600000\\
475000.000000001	488300000\\
-696000.000000002	-12695600000\\
112000.000000003	12817800000\\
179999.999999998	-3906600000\\
-217000	-1952900000\\
254000	3784000000\\
-34999.9999999975	-2075000000\\
-90999.9999999993	-1465000000\\
-1999.99999999889	1465000000\\
-146000.000000001	-122299999.999998\\
-90000.0000000025	-976400000.000001\\
347000	4150300000\\
108999.999999999	-2319100000\\
-127000.000000002	-3052100000\\
-55999.9999999974	2197600000\\
-255000.000000001	-732800000.000002\\
35000.0000000019	1098900000\\
202999.999999998	1342800000\\
128000.000000003	-1831100000\\
-38000.0000000029	1952900000\\
-109000.000000002	-5248800000\\
-109999.999999998	4516700000\\
19000.0000000001	-1587100000\\
17999.9999999998	1220700000\\
90999.9999999975	610400000\\
219999.999999999	1220700000\\
-73999.9999999989	-5249000000.00001\\
-345999.999999997	-121999999.999997\\
181999.999999996	5981200000\\
-19000.000000001	-3783900000\\
129000	2075100000\\
-238000.000000001	-3906500000\\
91999.9999999996	3906800000\\
90000.0000000007	-488900000.000001\\
55999.9999999965	-609900000.000001\\
-17999.9999999954	-1587100000\\
-349000	-2.1316282072803e-06\\
128999.999999997	2807700000\\
74000.0000000007	-1831200000\\
162999.999999998	2319500000\\
-127000	-4272600000\\
145999.999999999	5493300000\\
-183000.000000001	-6958100000\\
-73000.0000000013	3784100000\\
54000.0000000011	854800000.000009\\
331000.000000002	5004500000\\
-73999.9999999989	-9765400000.00001\\
-329000	-1342799999.99999\\
35999.9999999987	8911000000\\
-18999.9999999992	-5248700000\\
459000.000000001	10741800000\\
-73000.0000000004	-16845500000\\
-349000.000000002	5859300000\\
-91000.0000000011	2685800000\\
218999.999999998	1342300000\\
75000.0000000019	-2807200000\\
-312999.999999997	-610500000\\
17999.9999999962	1953100000\\
130000	-488299999.999999\\
200999.999999999	732499999.999998\\
-37999.9999999994	-1098700000\\
-164000.000000002	-1220600000\\
-17999.9999999989	1098500000\\
36999.999999999	1098700000\\
72000.0000000001	-244199999.999999\\
36999.9999999981	-854400000.000001\\
19000.0000000019	610400000.000002\\
-56000.0000000027	-1220900000\\
-107999.999999998	244300000\\
-93999.9999999976	-99999.9999983459\\
147999.999999995	2807600000\\
202000.000000003	-1830800000\\
-129999.999999997	-1709200000\\
-329000.000000001	-488399999.999998\\
514000	5737500000\\
-350000	-6836000000\\
20999.9999999999	3418100000\\
34000.0000000016	-488500000.000001\\
128999.999999997	1220800000\\
147000.000000003	100000.000000477\\
-476000.000000001	-6225800000\\
329000.000000001	9765700000\\
36999.999999999	-5737100000\\
-91999.9999999987	243900000.000004\\
-237000.000000002	-2197200000\\
181999.999999999	6469800000\\
183000	-2685700000\\
-181999.999999999	-2929600000\\
201000	5005000000\\
-202000	-6469900000\\
-3.5527136788005e-09	4516600000\\
-72999.9999999977	-2685300000\\
238999.999999999	6591400000\\
-258000	-10864000000\\
185000.000000002	11596700000\\
-2000.00000000333	-9277400000\\
-401999.999999998	121999999.999993\\
642000.000000002	15747200000\\
70999.9999999979	-17822400000\\
-436999.999999999	732500000.000007\\
-94000.0000000012	5859400000\\
1999.99999999978	366099999.999999\\
108999.999999997	-976400000\\
91000.0000000011	366099999.999996\\
1000.00000000389	-1220699999.99999\\
-36999.9999999999	976599999.999994\\
220000	1830900000\\
-311000.000000001	-6957600000\\
-20000.0000000022	4882300000\\
148000	2197600000\\
145999.999999999	-1342900000\\
-165000	-4150300000\\
-382999.999999998	1586800000\\
491999.999999996	5615300000\\
-199999.999999999	-6957900000\\
18000.0000000025	4028000000\\
239000	976900000\\
-39000.0000000015	-3540200000\\
-198999.999999997	-488300000.000002\\
-166000.000000001	366300000.000003\\
184000.000000002	4272400000\\
-19000.0000000019	-3906200000\\
163999.999999999	3051699999.99999\\
-126999.999999999	-3662000000\\
146000.000000002	5615000000\\
-71999.9999999992	-8666800000.00001\\
52999.9999999982	8667000000.00001\\
-18000.0000000025	-8545000000\\
-510999.999999996	2929700000\\
382999.999999998	6103600000\\
403000	243999999.999995\\
-677000	-15380800000\\
236999.999999997	15381000000\\
112000.000000005	-2197600000\\
289999.999999996	1465200000\\
-198999.999999996	-10132000000\\
-110000.000000001	5249000000\\
-350000.000000003	122199999.999999\\
332000.000000002	4638400000\\
34999.9999999984	-5614900000\\
111000.000000001	4028100000\\
-294000.000000001	-5615100000\\
203000.000000001	6347400000\\
-167000.000000002	-5004400000\\
166000	3905700000\\
-109000.000000002	-3295600000\\
-130000	366300000\\
386000	5126700000\\
-312000.000000001	-8422699999.99999\\
72999.9999999995	6836000000\\
147000	-1343099999.99999\\
164999.999999999	610799999.999998\\
-404000.000000001	-8301000000\\
-52999.9999999999	9155100000\\
401000	976900000\\
-256000.000000001	-7446500000\\
93000.0000000035	5371100000\\
-258000.000000003	-3784100000\\
238000	6347700000\\
167000.000000003	-5371300000\\
-130999.999999998	366400000.000004\\
-200000.000000001	243999999.999998\\
-164999.999999998	1098900000\\
256999.999999999	243700000.000001\\
163999.999999998	366700000\\
55000.0000000024	-366599999.999998\\
-274999.999999999	-3661900000\\
-109000.000000002	3539900000\\
218999.999999998	976800000.000001\\
184000.000000001	854199999.999999\\
-148000.000000001	-6591600000\\
-200000	5126900000\\
90999.9999999993	-100000.000001188\\
184000.000000002	122299999.999998\\
-203000.000000002	-3296100000\\
129000	4638700000\\
73000.0000000004	-2441300000\\
-146000.000000001	-1342900000\\
-146999.999999998	-7.105427357601e-07\\
237999.999999998	5127200000\\
-127999.999999998	-6592200000\\
-54000.0000000003	3296300000\\
-1000.00000000211	-610700000.000003\\
420999.999999999	5859600000\\
-310999.999999999	-11474600000\\
-239000.000000002	3784000000\\
478000.000000004	10009900000\\
-92000.0000000005	-11474700000\\
-148000.000000002	610500000.000001\\
-53000.0000000017	4760700000\\
-56999.9999999995	-2319500000\\
149000	1220900000\\
-75999.9999999969	-854700000.000002\\
-236000.000000005	-1952800000\\
750000.000000003	7812200000\\
-567000.000000002	-10620000000\\
-184999.999999996	4638600000\\
240999.999999996	1831200000\\
254000.000000002	-100000.000002609\\
-165000	-3418000000\\
20999.999999999	2075300000\\
-295999.999999998	-2075300000\\
-52999.9999999999	2319400000\\
346999.999999999	1464900000\\
200999.999999998	-366500000\\
-217999.999999996	-3905800000\\
-167000.000000003	854100000.000003\\
130000.000000003	3296100000\\
-73000.0000000004	-2441400000\\
-2000.00000000067	121999999.999998\\
37999.9999999985	1342700000\\
109000	244400000.000005\\
999.999999999446	-1587300000.00001\\
-999.999999999446	488700000.000003\\
-89999.999999999	-1709300000.00001\\
-130000.000000003	1709200000\\
1000.00000000122	-199999.999998823\\
385000.000000002	3540200000\\
-294000.000000001	-6714000000\\
-274999.999999999	854600000.000003\\
367999.999999998	6347700000\\
70999.9999999988	-3662300000\\
-162999.999999998	-3051600000\\
-91999.999999997	3051700000\\
512999.999999998	5615299999.99999\\
-770000.000000002	-17822400000\\
457999.999999999	22339000000\\
91999.9999999978	-14404400000\\
-256999.999999999	3296000000.00001\\
147000.000000001	1708900000\\
-383999.999999998	-5981400000\\
494000	14160100000\\
181999.999999999	-10253800000\\
-602999.999999997	-8300800000\\
219999.999999999	14037900000\\
108000	-1464599999.99999\\
-107999.999999998	-7446400000\\
90999.9999999993	8056700000\\
-183000	-10254100000\\
293000.000000001	15991400000\\
-19000.0000000001	-14282200000\\
-35999.9999999996	2563400000\\
-293000.000000002	732399999.999995\\
145999.999999998	4394500000.00001\\
129000.000000001	-3662000000\\
-367000.000000002	-1586900000\\
201999.999999998	3784000000\\
420999.999999998	3906400000\\
-384999.999999999	-12817400000\\
-73000.0000000013	9277300000\\
256999.999999999	-244000000.000001\\
-312000	-4272800000\\
403000	5005300000\\
-365999.999999998	-4272700000\\
36000.0000000014	2563500000\\
18999.9999999983	-1464900000\\
165000	2563700000\\
-18999.9999999966	-3052000000\\
-18999.9999999983	1342900000\\
-108000.000000003	-122099999.999999\\
-165999.999999998	-1464800000\\
366999.999999997	4394400000\\
-165999.999999999	-4516500000\\
-273000.000000001	-488199999.999999\\
309000	4882600000\\
148999.999999997	-1952900000\\
34999.9999999993	-1465000000\\
-365999.999999999	-2075200000\\
-37000.0000000008	3540100000\\
679000	5249100000\\
-404000	-13549900000\\
-238000.000000001	9277300000\\
202000.000000001	-1342700000\\
-294000	-1587000000\\
130000.000000001	1220900000\\
108000	1098400000\\
347999.999999998	1220900000\\
-585000	-7568500000\\
165000.000000001	5981400000\\
255999.999999999	3540300000\\
-1000.00000000122	-5371400000\\
18999.9999999983	-1220500000\\
-439000.000000001	366200000.000006\\
91000.0000000002	3051600000\\
184000	99999.9999990564\\
309999.999999998	2563500000\\
-201000	-8422800000\\
-494000.000000001	2075100000\\
475999.999999998	7080000000\\
92000.0000000014	-4394300000\\
-165999.999999998	-2807800000\\
-549000	1953200000\\
329999.999999999	1464700000\\
420999.999999999	3174100000\\
19000.000000001	-5615400000\\
-148000.000000004	1220600000\\
-217999.999999996	-1342500000\\
383000.000000001	4150100000\\
-584999.999999998	-4516400000\\
201000.000000001	3173800000\\
54999.9999999988	-854500000.000001\\
-36000.0000000005	-488499999.999999\\
-20000.0000000005	300000.000000722\\
276999.999999999	2807600000\\
-75000.0000000019	-4028500000\\
-91000.0000000011	1464800000\\
54999.9999999988	732799999.999996\\
-73000.0000000013	-976899999.999996\\
109000.000000001	610300000\\
-237000.000000003	-2319000000\\
-73999.9999999989	1952800000\\
420999.999999998	5615400000\\
203000.000000004	-6225599999.99999\\
-332000.000000005	-3784200000\\
-656999.999999997	1709000000\\
583999.999999998	8056500000\\
240000.000000001	-3905900000\\
-56000.0000000009	-2075700000\\
-294000	-2929200000\\
-235999.999999998	3661900000\\
292000	1464700000\\
239000.000000001	2197500000\\
-167000.000000002	-7568500000\\
37999.9999999994	5981599999.99999\\
74000.0000000043	-854699999.999993\\
-166000.000000004	-6347500000\\
293000.000000001	11596600000\\
-365000	-12695300000\\
-167000.000000002	7202200000\\
277000.000000001	488199999.999999\\
16999.9999999977	-610199999.999995\\
1000.000000003	-1709200000\\
162999.999999996	3173900000\\
-382999.999999999	-7202000000\\
255999.999999998	8666900000\\
201000	-3174000000\\
-254999.999999996	-3539800000\\
-93000.0000000008	3906200000\\
-147000.000000002	-2075300000\\
350000.000000002	3540200000\\
52999.999999999	-2807900000\\
-474999.999999999	-2563100000\\
74000.0000000007	4272200000\\
619999.999999999	2685700000\\
-326999.999999999	-7812800000\\
-258000	2930200000\\
36999.9999999981	1220200000\\
459000.000000002	4028600000\\
-20000.0000000013	-6469800000\\
-292999.999999997	-976600000.000003\\
-475000	2807800000\\
292999.999999998	243900000.000002\\
366000.000000001	4272500000\\
-220000	-7934400000\\
128000	5859199999.99999\\
90999.9999999993	-488199999.999991\\
75000.0000000028	-610400000.000007\\
-331000.000000001	-8544800000\\
-89999.9999999981	9155200000\\
308999.999999998	4150200000\\
-89999.9999999981	-10375600000\\
-147000.000000002	3783800000\\
147000.000000003	2441700000\\
-56000.0000000009	-1098800000\\
-126999.999999998	-4882700000\\
420000	15747000000\\
-34999.9999999993	-19165100000\\
-203000.000000002	5249100000\\
-145000	2807700000\\
-112000.000000002	610200000\\
222000	244200000\\
-38000.0000000002	-1464800000\\
-165000	-2197400000\\
404000.000000002	7812700000\\
-165000.000000001	-9277500000\\
-184000.000000001	3296000000\\
459000.000000001	6103300000\\
-478000.000000002	-12450800000\\
-16000	8910800000\\
217999.999999998	-243899999.999997\\
239000.000000002	365999999.999997\\
-184000.000000001	-5981400000\\
-402000.000000002	3174000000\\
384000.000000002	3295700000\\
-475000.000000003	-5981400000\\
438000.000000001	5493300000\\
312000	2319200000\\
-440000	-9643599999.99999\\
56000.0000000045	6225800000\\
-56000.0000000045	-1953399999.99999\\
276000.000000002	7324400000\\
90000.0000000007	-9155200000.00001\\
-255999.999999998	-732799999.999992\\
110999.999999998	4517100000\\
-386000.000000002	-3052100000\\
111000	5127100000\\
71999.9999999983	-4394500000\\
239000	4150100000\\
-127999.999999998	-5492800000\\
-165000.000000002	2685400000\\
72999.9999999986	366200000.000001\\
54000.0000000011	366200000.000001\\
-33999.9999999989	-1831100000\\
252999.999999998	4272700000\\
-69999.9999999985	-4761000000\\
-149000.000000001	-976599999.999996\\
-108000.000000001	3784500000\\
35000.0000000019	-610599999.999999\\
-17999.9999999989	-1342800000\\
-108000.000000001	732600000\\
346000.000000001	1464700000\\
-164999.999999998	-2807700000\\
-127000.000000005	1099000000.00001\\
421000.000000002	3661700000\\
-496000	-9155100000\\
39000.0000000041	7202300000\\
529999.999999999	5492800000\\
-239000	-13305300000\\
-290999.999999999	4760400000\\
180999.999999999	4517000000\\
258000.000000004	1342500000\\
-294000	-11962900000\\
129000	12207200000\\
-257000.000000004	-7934800000\\
-201000	1953400000\\
255999.999999999	3661800000\\
256000.000000003	1465200000\\
184000	-610699999.999995\\
-274999.999999999	-10009500000\\
8.88178419700125e-10	9033000000\\
-166000.000000001	-5492900000\\
-88999.9999999977	4394100000\\
565999.999999999	7202600000\\
-385000	-17334200000\\
-72000.0000000018	11718800000\\
-201999.999999998	-6347700000\\
366000	10864200000\\
311999.999999998	-6957900000\\
-513999.999999999	-6591799999.99999\\
111000.000000001	10620100000\\
-238000	-7324200000\\
128000	4394500000\\
310999.999999996	2685400000\\
-18999.9999999983	-5126600000\\
-108000.000000001	-854800000.000002\\
-203000	732599999.999996\\
202000	4882600000\\
72999.9999999977	-5248800000\\
-182999.999999999	976599999.999999\\
54999.9999999997	1342400000\\
0	-365799999.999998\\
-90999.9999999993	-976700000.000001\\
-110999.999999997	121900000.000001\\
110999.999999997	1953400000\\
292000.000000001	-122199999.999998\\
-18000.0000000007	-610300000\\
-183000.000000002	-4272700000\\
-90999.9999999975	5371500000\\
71999.9999999992	-2563900000\\
-146000.000000002	1099000000\\
258000.000000003	1342600000\\
-58000.0000000016	-2075300000\\
186000.000000002	1465100000\\
-643000.000000002	-6103700000\\
349000	8544900000\\
439000.000000002	610499999.999999\\
-54000.0000000011	-6713900000\\
-587000	-1098800000\\
-36999.999999999	5005000000\\
477999.999999998	1587000000\\
-75999.9999999996	-3540100000\\
40000.0000000018	732399999.999995\\
-461000	-3174000000\\
239999.999999998	5127300000\\
456999.999999998	2807400000\\
-127999.999999997	-8544900000\\
-512000.000000003	122100000\\
-183999.999999999	5249000000\\
603999.999999998	2319400000\\
257000	-4150500000\\
-476000.000000001	-2685400000\\
-256999.999999999	732200000.000001\\
294000	6103800000\\
274000	-1587200000\\
-349000	-8911000000\\
21000.0000000008	10376100000\\
437000	1708900000\\
-310000.000000001	-13672100000\\
-349000	7324500000\\
606000.000000002	8788900000\\
-496000.000000003	-16967500000\\
110000	13793500000\\
294000.000000003	-1830600000\\
72999.9999999995	-4761100000\\
-678000.000000003	-5126800000\\
128000.000000001	10498100000\\
440000	1098600000\\
109000.000000001	-2685700000\\
-254999.999999999	-8666700000\\
-202000.000000002	5614900000\\
311000.000000001	8178900000\\
54999.9999999997	-10253800000\\
-330000.000000002	243900000.000004\\
19999.9999999996	3540200000\\
126000	2075100000\\
19999.9999999996	-3539900000\\
-20000.0000000005	-244300000.000004\\
130000.000000003	4272500000\\
-220000.000000003	-7202100000\\
-37999.9999999994	4028400000\\
366999.999999997	5981200000\\
-183000.000000001	-11840600000\\
-146000	7079900000\\
144999.999999999	-609999999.999997\\
-162999.999999999	-1953600000\\
-130000.000000001	854699999.999999\\
405000.000000002	4516800000\\
-259000.000000002	-8545200000\\
58000.0000000034	6958100000\\
51999.9999999978	-2929700000\\
203000.000000004	3906300000\\
-219000.000000002	-8056600000\\
198999.999999999	8544700000\\
-454999.999999999	-11840500000\\
-57000.000000003	13061200000\\
678000.000000001	-366099999.999999\\
-128000.000000002	-8910800000\\
-694999.999999999	-1099300000\\
345999.999999999	10254600000\\
295000.000000004	-2808100000\\
-276000.000000002	-6347400000\\
19000.000000001	5004600000\\
385000.000000002	6226000000\\
-258000.000000002	-15137000000\\
-237000	7446299999.99999\\
384000	6958100000\\
-163999.999999998	-11352600000\\
-238000	4272400000\\
108999.999999999	1831300000\\
166000.000000001	854200000\\
-57000.0000000022	-2685400000\\
222000.000000002	3051700000\\
-313000	-7080000000\\
2000.00000000156	6836000000\\
-57000.0000000057	-4028600000\\
184000.000000002	5493500000\\
35999.9999999996	-4394900000\\
-401000.000000001	-3783800000\\
400000.000000001	10986200000\\
167999.999999999	-6470000000\\
-278000	-3051300000\\
-197999.999999999	1220300000\\
509999.999999999	9155600000\\
-219000.000000004	-13061800000\\
-291999.999999997	5615500000\\
71999.9999999992	1830800000\\
256999.999999999	-610300000.000004\\
35999.9999999996	-1098400000\\
-34999.9999999984	-732699999.999998\\
-404999.999999997	-732200000\\
129999.999999998	3295600000\\
108000.000000001	-2319000000\\
130000.000000001	2197100000\\
-37000.0000000017	-3540200000\\
-56000.0000000009	3418400000\\
367000.000000001	2929300000\\
-310999.999999999	-12084900000\\
-550000.000000004	5737600000\\
494000.000000001	7201700000\\
368000.000000004	244400000.000005\\
-313000	-13671800000\\
-310000	6347500000\\
493000.000000003	11474600000\\
-109000.000000003	-15502900000\\
-72999.9999999959	5981600000.00001\\
-165000.000000002	-1831300000\\
-19000.0000000037	1953300000\\
385000.000000002	5371100000\\
-238000	-11230700000\\
-311000	3418200000\\
311000.000000003	7080100000\\
109999.999999999	-4150600000\\
291999.999999999	2319600000.00001\\
-308999.999999998	-10376300000\\
-772000.000000002	4394900000\\
918000	12206700000\\
144999.999999998	-11474500000\\
-529999.999999999	-1952900000\\
-146999.999999999	5370800000\\
164000.000000001	122100000\\
56999.9999999986	-3051600000\\
345999.999999999	7568300000\\
-128000.000000001	-10742300000\\
-54000.0000000003	5493400000\\
-91999.9999999987	-854800000.000002\\
-220000.000000003	-976400000.000002\\
221000.000000003	3540200000\\
-56999.9999999986	-3784500000\\
-145000.000000004	854599999.999998\\
402000.000000002	3784400000\\
73999.9999999998	-4028500000\\
-403000.000000003	-1342900000\\
-17999.9999999971	2319600000\\
72000.0000000001	1098300000\\
294000	-121599999.999999\\
-384999.999999997	-4517000000\\
275000	7324300000\\
-36999.9999999981	-5981300000\\
-421000	121899999.999998\\
311999.999999997	4760900000\\
-1000.00000000033	-4516800000\\
19000.0000000001	2685800000\\
292000	2563199999.99999\\
74000.0000000007	-4150199999.99999\\
-402000	-5126900000.00001\\
-167000.000000002	7568100000\\
442000	610599999.999998\\
-331999.999999997	-5005000000\\
-255000.000000003	2441500000\\
476000	2441100000\\
19000.0000000019	-2807100000\\
-203000.000000002	-610800000.000002\\
73999.9999999998	1587200000\\
-1.77635683940025e-09	-122300000.000003\\
1.77635683940025e-09	-244000000.000001\\
-55000.0000000015	-1464900000\\
129000.000000002	3296000000\\
-240000.000000003	-5371200000\\
551000.000000002	12207100000\\
-183000	-16845900000\\
-441000.000000002	7690800000\\
-89999.9999999998	487900000\\
511999.999999999	4761100000\\
-128000.000000001	-9155700000\\
165000.000000002	7446700000\\
-476999.999999999	-7568500000\\
311999.999999998	8910900000\\
91999.9999999996	-6957600000\\
-384999.999999998	1952800000\\
17999.9999999989	610599999.999999\\
293000.000000002	1586600000\\
-128000.000000004	-3539800000\\
-17999.9999999998	2441500000\\
421000.000000001	3295600000\\
-2.66453525910038e-09	-6469600000\\
-824000	-2441300000\\
402000	9765600000\\
458999.999999998	-2929900000\\
-622999.999999999	-7079800000\\
-55000.0000000006	6103300000\\
513000.000000001	4394700000\\
-385000.000000002	-12451300000\\
165000.000000003	12573400000\\
383999.999999999	-2807900000\\
-382999.999999999	-8178499999.99999\\
-75000.0000000028	6225700000\\
-71999.9999999983	-1587100000\\
35999.9999999969	2075100000\\
384000	3906400000\\
-126999.999999996	-9521400000\\
};
\addplot [color=mycolor2, line width=2.0pt, forget plot]
  table[row sep=crcr]{%
-129000.000000001	-129000.000000001\\
273999.999999998	273999.999999998\\
-143999.999999998	-143999.999999998\\
179999.999999999	179999.999999999\\
56999.9999999995	56999.9999999995\\
-458000.000000001	-458000.000000001\\
181999.999999999	181999.999999999\\
92999.9999999991	92999.9999999991\\
17000.0000000012	17000.0000000012\\
311999.999999997	311999.999999997\\
-401999.999999998	-401999.999999998\\
-130000	-130000\\
-292000.000000001	-292000.000000001\\
420999.999999999	420999.999999999\\
347999.999999999	347999.999999999\\
-237999.999999998	-237999.999999998\\
91999.9999999978	91999.9999999978\\
-999.999999999446	-999.999999999446\\
-641000.000000002	-641000.000000002\\
826000.000000005	826000.000000005\\
-241000.000000005	-241000.000000005\\
-436999.999999997	-436999.999999997\\
548999.999999999	548999.999999999\\
-185000.000000002	-185000.000000002\\
-216999.999999997	-216999.999999997\\
326999.999999996	326999.999999996\\
19999.9999999996	19999.9999999996\\
-366999.999999999	-366999.999999999\\
274999.999999998	274999.999999998\\
73000.0000000031	73000.0000000031\\
-109000.000000003	-109000.000000003\\
-19999.9999999996	-19999.9999999996\\
-89999.9999999998	-89999.9999999998\\
-8.88178419700125e-10	-8.88178419700125e-10\\
181999.999999999	181999.999999999\\
-91000.0000000011	-91000.0000000011\\
-18999.9999999992	-18999.9999999992\\
20000.0000000022	20000.0000000022\\
-166000.000000001	-166000.000000001\\
329999.999999999	329999.999999999\\
17999.9999999989	17999.9999999989\\
-275000.000000002	-275000.000000002\\
238000.000000002	238000.000000002\\
-273000.000000001	-273000.000000001\\
-37999.9999999994	-37999.9999999994\\
292999.999999997	292999.999999997\\
-165000	-165000\\
-126999.999999999	-126999.999999999\\
235999.999999999	235999.999999999\\
21000.0000000017	21000.0000000017\\
-332000.000000002	-332000.000000002\\
496000	496000\\
-56000.0000000009	-56000.0000000009\\
-293000	-293000\\
-90999.9999999993	-90999.9999999993\\
129000.000000001	129000.000000001\\
327999.999999997	327999.999999997\\
-346999.999999999	-346999.999999999\\
-146999.999999998	-146999.999999998\\
366999.999999999	366999.999999999\\
-110999.999999997	-110999.999999997\\
-164000.000000001	-164000.000000001\\
127999.999999999	127999.999999999\\
-183000.000000002	-183000.000000002\\
366000.000000001	366000.000000001\\
-201000.000000002	-201000.000000002\\
54000.0000000011	54000.0000000011\\
-17000.0000000003	-17000.0000000003\\
34999.9999999993	34999.9999999993\\
-52999.9999999973	-52999.9999999973\\
-1000.00000000389	-1000.00000000389\\
-1000.00000000033	-1000.00000000033\\
-52999.9999999982	-52999.9999999982\\
-148000.000000001	-148000.000000001\\
148000	148000\\
198999.999999999	198999.999999999\\
-15999.9999999965	-15999.9999999965\\
54999.9999999988	54999.9999999988\\
-478000.000000003	-478000.000000003\\
239000.000000003	239000.000000003\\
165000.000000002	165000.000000002\\
54999.9999999997	54999.9999999997\\
-385000.000000001	-385000.000000001\\
129000.000000002	129000.000000002\\
364999.999999998	364999.999999998\\
-218999.999999997	-218999.999999997\\
17999.9999999989	17999.9999999989\\
-53999.9999999976	-53999.9999999976\\
-129000.000000004	-129000.000000004\\
-17999.9999999989	-17999.9999999989\\
255999.999999998	255999.999999998\\
17999.9999999989	17999.9999999989\\
-274000.000000002	-274000.000000002\\
-36999.9999999972	-36999.9999999972\\
19000.0000000001	19000.0000000001\\
310999.999999999	310999.999999999\\
-147000.000000002	-147000.000000002\\
-128000	-128000\\
165000.000000002	165000.000000002\\
-165000	-165000\\
257000.000000001	257000.000000001\\
-166000	-166000\\
148000.000000001	148000.000000001\\
-129000.000000001	-129000.000000001\\
-129000	-129000\\
258000.000000001	258000.000000001\\
-148000.000000004	-148000.000000004\\
-90999.9999999975	-90999.9999999975\\
129000	129000\\
108999.999999999	108999.999999999\\
-202000.000000002	-202000.000000002\\
112000.000000003	112000.000000003\\
-76000.0000000023	-76000.0000000023\\
1999.99999999889	1999.99999999889\\
128000	128000\\
-184000	-184000\\
-8.88178419700125e-10	-8.88178419700125e-10\\
-17000.0000000012	-17000.0000000012\\
218000.000000002	218000.000000002\\
-72000.0000000001	-72000.0000000001\\
201000	201000\\
-475999.999999999	-475999.999999999\\
19000.0000000019	19000.0000000019\\
528999.999999996	528999.999999996\\
-363999.999999996	-363999.999999996\\
34999.9999999993	34999.9999999993\\
276000.000000001	276000.000000001\\
-275000	-275000\\
-221000.000000003	-221000.000000003\\
202999.999999999	202999.999999999\\
-147999.999999998	-147999.999999998\\
258000.000000002	258000.000000002\\
53999.9999999994	53999.9999999994\\
37000.0000000035	37000.0000000035\\
-256000.000000003	-256000.000000003\\
125999.999999999	125999.999999999\\
-69999.9999999994	-69999.9999999994\\
15999.9999999973	15999.9999999973\\
183999.999999998	183999.999999998\\
-201999.999999999	-201999.999999999\\
-36000.0000000005	-36000.0000000005\\
90999.9999999966	90999.9999999966\\
-108999.999999997	-108999.999999997\\
-38000.000000002	-38000.000000002\\
38000.0000000029	38000.0000000029\\
365999.999999998	365999.999999998\\
-166000	-166000\\
-145000	-145000\\
-19000.0000000028	-19000.0000000028\\
201000	201000\\
-126999.999999998	-126999.999999998\\
-221000.000000004	-221000.000000004\\
-16999.9999999995	-16999.9999999995\\
438000.000000001	438000.000000001\\
-218000.000000003	-218000.000000003\\
-56999.9999999995	-56999.9999999995\\
111000	111000\\
36999.9999999972	36999.9999999972\\
-36999.9999999981	-36999.9999999981\\
-74000.0000000025	-74000.0000000025\\
-217999.999999997	-217999.999999997\\
-38000.0000000002	-38000.0000000002\\
329000	329000\\
112000.000000003	112000.000000003\\
107999.999999999	107999.999999999\\
-200000	-200000\\
-112000.000000002	-112000.000000002\\
76000.0000000005	76000.0000000005\\
-441999.999999998	-441999.999999998\\
532999.999999997	532999.999999997\\
-184000	-184000\\
-8.88178419700125e-10	-8.88178419700125e-10\\
-110000	-110000\\
275000.000000001	275000.000000001\\
165000.000000002	165000.000000002\\
-294000	-294000\\
201999.999999999	201999.999999999\\
-364999.999999998	-364999.999999998\\
143999.999999998	143999.999999998\\
-15999.9999999973	-15999.9999999973\\
236999.999999997	236999.999999997\\
-219999.999999998	-219999.999999998\\
-18000.0000000007	-18000.0000000007\\
92000.0000000023	92000.0000000023\\
-128000	-128000\\
217999.999999997	217999.999999997\\
-400999.999999999	-400999.999999999\\
329000.000000003	329000.000000003\\
182999.999999996	182999.999999996\\
-109999.999999999	-109999.999999999\\
-293000.000000003	-293000.000000003\\
-127999.999999997	-127999.999999997\\
348999.999999999	348999.999999999\\
-39000.0000000015	-39000.0000000015\\
39000.0000000024	39000.0000000024\\
-20000.0000000005	-20000.0000000005\\
146999.999999999	146999.999999999\\
-256000	-256000\\
-18999.9999999983	-18999.9999999983\\
56000.0000000009	56000.0000000009\\
274000.000000001	274000.000000001\\
-110000.000000002	-110000.000000002\\
-458000.000000001	-458000.000000001\\
385000.000000002	385000.000000002\\
147000.000000001	147000.000000001\\
-440000.000000005	-440000.000000005\\
201000.000000001	201000.000000001\\
146999.999999998	146999.999999998\\
-54999.9999999988	-54999.9999999988\\
-202000.000000003	-202000.000000003\\
202000.000000001	202000.000000001\\
-108999.999999998	-108999.999999998\\
-112000.000000003	-112000.000000003\\
350000.000000001	350000.000000001\\
-37999.9999999985	-37999.9999999985\\
-237000.000000002	-237000.000000002\\
-74000.0000000016	-74000.0000000016\\
-36999.999999999	-36999.999999999\\
568999.999999998	568999.999999998\\
-166000.000000001	-166000.000000001\\
-238000	-238000\\
-164000.000000002	-164000.000000002\\
219000	219000\\
93000.0000000008	93000.0000000008\\
-204000.000000001	-204000.000000001\\
-34000.0000000007	-34000.0000000007\\
144000	144000\\
-15999.9999999991	-15999.9999999991\\
200000.000000001	200000.000000001\\
-493000	-493000\\
290999.999999997	290999.999999997\\
166000	166000\\
-92000.0000000005	-92000.0000000005\\
-145999.999999998	-145999.999999998\\
-73999.9999999989	-73999.9999999989\\
422999.999999998	422999.999999998\\
-38999.9999999997	-38999.9999999997\\
-548000.000000003	-548000.000000003\\
219000	219000\\
999.999999999446	999.999999999446\\
218999.999999999	218999.999999999\\
-17999.9999999998	-17999.9999999998\\
-330000.000000003	-330000.000000003\\
293000.000000002	293000.000000002\\
73999.9999999998	73999.9999999998\\
-311999.999999998	-311999.999999998\\
238999.999999999	238999.999999999\\
-38000.000000002	-38000.000000002\\
-109999.999999999	-109999.999999999\\
240000.000000002	240000.000000002\\
-295000.000000003	-295000.000000003\\
221000.000000002	221000.000000002\\
-165000.000000001	-165000.000000001\\
-220999.999999997	-220999.999999997\\
440999.999999997	440999.999999997\\
-182999.999999999	-182999.999999999\\
-38000.0000000011	-38000.0000000011\\
385999.999999999	385999.999999999\\
-130000	-130000\\
-273000	-273000\\
-17999.9999999998	-17999.9999999998\\
-94000.0000000012	-94000.0000000012\\
461000	461000\\
-351000.000000001	-351000.000000001\\
-34000.0000000007	-34000.0000000007\\
-2000.00000000067	-2000.00000000067\\
275999.999999999	275999.999999999\\
-238999.999999999	-238999.999999999\\
349000.000000001	349000.000000001\\
-258000.000000004	-258000.000000004\\
148000	148000\\
-237999.999999998	-237999.999999998\\
90000.0000000007	90000.0000000007\\
-53000.0000000017	-53000.0000000017\\
-92999.9999999955	-92999.9999999955\\
-18000.0000000007	-18000.0000000007\\
292999.999999997	292999.999999997\\
-108999.999999997	-108999.999999997\\
-1000.000000003	-1000.000000003\\
-236999.999999998	-236999.999999998\\
437999.999999998	437999.999999998\\
-36000.0000000014	-36000.0000000014\\
-348000	-348000\\
147000.000000001	147000.000000001\\
-55000.0000000006	-55000.0000000006\\
219000.000000001	219000.000000001\\
56000.0000000009	56000.0000000009\\
-111000.000000001	-111000.000000001\\
-164000.000000001	-164000.000000001\\
-91000.0000000011	-91000.0000000011\\
71000.0000000015	71000.0000000015\\
93999.9999999985	93999.9999999985\\
53999.9999999994	53999.9999999994\\
-165999.999999998	-165999.999999998\\
-34000.0000000025	-34000.0000000025\\
309000	309000\\
-91000.0000000019	-91000.0000000019\\
-163999.999999998	-163999.999999998\\
309999.999999997	309999.999999997\\
-127000	-127000\\
-458000	-458000\\
220000	220000\\
382999.999999998	382999.999999998\\
-272999.999999999	-272999.999999999\\
144999.999999998	144999.999999998\\
-16999.9999999968	-16999.9999999968\\
-293000.000000001	-293000.000000001\\
73000.0000000004	73000.0000000004\\
255999.999999997	255999.999999997\\
54000.000000002	54000.000000002\\
-454999.999999999	-454999.999999999\\
454999.999999997	454999.999999997\\
-126999.999999998	-126999.999999998\\
-201000.000000001	-201000.000000001\\
201000	201000\\
53999.9999999985	53999.9999999985\\
-126999.999999999	-126999.999999999\\
-237999.999999997	-237999.999999997\\
201000	201000\\
311999.999999998	311999.999999998\\
-2000.00000000156	-2000.00000000156\\
-419000	-419000\\
54000.0000000011	54000.0000000011\\
36999.999999999	36999.999999999\\
72999.9999999968	72999.9999999968\\
146000.000000001	146000.000000001\\
-182000.000000001	-182000.000000001\\
-130000	-130000\\
222000.000000004	222000.000000004\\
144999.999999998	144999.999999998\\
-182000.000000001	-182000.000000001\\
-221000	-221000\\
18999.9999999983	18999.9999999983\\
383999.999999999	383999.999999999\\
-163999.999999995	-163999.999999995\\
164999.999999998	164999.999999998\\
-295000.000000001	-295000.000000001\\
3000.000000001	3000.000000001\\
271999.999999998	271999.999999998\\
-126000	-126000\\
-129999.999999996	-129999.999999996\\
19000.000000001	19000.000000001\\
-310999.999999999	-310999.999999999\\
365999.999999999	365999.999999999\\
90999.9999999984	90999.9999999984\\
111000.000000002	111000.000000002\\
-257000.000000003	-257000.000000003\\
-18000.0000000007	-18000.0000000007\\
311000.000000001	311000.000000001\\
-348000.000000002	-348000.000000002\\
55000.0000000006	55000.0000000006\\
165000.000000001	165000.000000001\\
-256999.999999998	-256999.999999998\\
-36000.0000000031	-36000.0000000031\\
238000	238000\\
275000	275000\\
-348000.000000002	-348000.000000002\\
-55999.9999999991	-55999.9999999991\\
-36000.0000000022	-36000.0000000022\\
275000	275000\\
-531000	-531000\\
220000.000000001	220000.000000001\\
547999.999999997	547999.999999997\\
-547999.999999998	-547999.999999998\\
127999.999999999	127999.999999999\\
108999.999999998	108999.999999998\\
54999.9999999997	54999.9999999997\\
-71999.9999999965	-71999.9999999965\\
-202000.000000003	-202000.000000003\\
127000	127000\\
-89999.9999999998	-89999.9999999998\\
54999.9999999988	54999.9999999988\\
108000.000000001	108000.000000001\\
-216999.999999999	-216999.999999999\\
179999.999999997	179999.999999997\\
-14999.9999999961	-14999.9999999961\\
197999.999999998	197999.999999998\\
-326999.999999998	-326999.999999998\\
36000.0000000014	36000.0000000014\\
255999.999999998	255999.999999998\\
-329999.999999998	-329999.999999998\\
221000.000000001	221000.000000001\\
163999.999999999	163999.999999999\\
-239000.000000003	-239000.000000003\\
-162999.999999999	-162999.999999999\\
-91999.9999999996	-91999.9999999996\\
456999.999999999	456999.999999999\\
-92000.0000000023	-92000.0000000023\\
-290999.999999998	-290999.999999998\\
254999.999999997	254999.999999997\\
19000.0000000019	19000.0000000019\\
-238999.999999998	-238999.999999998\\
91999.9999999996	91999.9999999996\\
200999.999999998	200999.999999998\\
-420000.000000001	-420000.000000001\\
273999.999999999	273999.999999999\\
255999.999999999	255999.999999999\\
-529999.999999998	-529999.999999998\\
163999.999999997	163999.999999997\\
402000.000000001	402000.000000001\\
-418999.999999998	-418999.999999998\\
382999.999999998	382999.999999998\\
-348000.000000002	-348000.000000002\\
220000.000000002	220000.000000002\\
1000.00000000033	1000.00000000033\\
-422000.000000004	-422000.000000004\\
402000.000000002	402000.000000002\\
-52999.9999999999	-52999.9999999999\\
-258000.000000002	-258000.000000002\\
366999.999999999	366999.999999999\\
-18999.9999999983	-18999.9999999983\\
-347000.000000001	-347000.000000001\\
-129000	-129000\\
385999.999999999	385999.999999999\\
34000.0000000007	34000.0000000007\\
-14999.9999999961	-14999.9999999961\\
-3000.00000000544	-3000.00000000544\\
-162999.999999999	-162999.999999999\\
237000	237000\\
-146000.000000002	-146000.000000002\\
-55000.0000000006	-55000.0000000006\\
129000	129000\\
-313000	-313000\\
460000.000000003	460000.000000003\\
-149000.000000005	-149000.000000005\\
-327999.999999998	-327999.999999998\\
183000.000000002	183000.000000002\\
310999.999999999	310999.999999999\\
-183000	-183000\\
-73000.0000000022	-73000.0000000022\\
-93000	-93000\\
240000.000000001	240000.000000001\\
-129000.000000001	-129000.000000001\\
-183000.000000002	-183000.000000002\\
128000.000000001	128000.000000001\\
109000	109000\\
75000.000000001	75000.000000001\\
-238999.999999999	-238999.999999999\\
164999.999999999	164999.999999999\\
128000	128000\\
-311000	-311000\\
-183000.000000003	-183000.000000003\\
329000.000000001	329000.000000001\\
147000.000000001	147000.000000001\\
-19000.0000000028	-19000.0000000028\\
-200999.999999997	-200999.999999997\\
146999.999999998	146999.999999998\\
-146999.999999998	-146999.999999998\\
91999.999999997	91999.999999997\\
-221000	-221000\\
294000	294000\\
37000.0000000008	37000.0000000008\\
-385000.000000002	-385000.000000002\\
238000.000000001	238000.000000001\\
-19000.0000000028	-19000.0000000028\\
38000.0000000029	38000.0000000029\\
35999.999999996	35999.999999996\\
-201999.999999998	-201999.999999998\\
128999.999999997	128999.999999997\\
54000.0000000047	54000.0000000047\\
275999.999999999	275999.999999999\\
-147000.000000002	-147000.000000002\\
-220999.999999999	-220999.999999999\\
2000.00000000067	2000.00000000067\\
-166000.000000003	-166000.000000003\\
202000	202000\\
-165999.999999999	-165999.999999999\\
368000	368000\\
-111000.000000001	-111000.000000001\\
-238000.000000003	-238000.000000003\\
202000.000000001	202000.000000001\\
-73999.9999999998	-73999.9999999998\\
73999.9999999972	73999.9999999972\\
-183999.999999998	-183999.999999998\\
330000.000000001	330000.000000001\\
-53999.9999999994	-53999.9999999994\\
-56000.0000000009	-56000.0000000009\\
-147000.000000001	-147000.000000001\\
-16999.9999999995	-16999.9999999995\\
-8.88178419700125e-10	-8.88178419700125e-10\\
273000	273000\\
-218000	-218000\\
-999.999999998557	-999.999999998557\\
257000.000000001	257000.000000001\\
-531000.000000002	-531000.000000002\\
146000.000000002	146000.000000002\\
494999.999999997	494999.999999997\\
-422000.000000001	-422000.000000001\\
-146000.000000001	-146000.000000001\\
493999.999999999	493999.999999999\\
-218999.999999998	-218999.999999998\\
-256999.999999999	-256999.999999999\\
385000.000000002	385000.000000002\\
0	0\\
-110000	-110000\\
-184000.000000004	-184000.000000004\\
-144999.999999996	-144999.999999996\\
182000	182000\\
221000	221000\\
-1000.00000000122	-1000.00000000122\\
-18999.9999999975	-18999.9999999975\\
1999.99999999889	1999.99999999889\\
-185000.000000002	-185000.000000002\\
111000	111000\\
-111000.000000002	-111000.000000002\\
37999.9999999994	37999.9999999994\\
-91999.9999999987	-91999.9999999987\\
145999.999999996	145999.999999996\\
183000.000000002	183000.000000002\\
-437999.999999999	-437999.999999999\\
180999.999999996	180999.999999996\\
112000.000000002	112000.000000002\\
-93000.0000000017	-93000.0000000017\\
-36999.999999999	-36999.999999999\\
111000.000000002	111000.000000002\\
220000.000000001	220000.000000001\\
-257000.000000004	-257000.000000004\\
-73999.9999999998	-73999.9999999998\\
111000.000000001	111000.000000001\\
53999.9999999976	53999.9999999976\\
-218000	-218000\\
-130000	-130000\\
365999.999999998	365999.999999998\\
203000.000000006	203000.000000006\\
-183000.000000004	-183000.000000004\\
-167000	-167000\\
111999.999999999	111999.999999999\\
-312000	-312000\\
-54999.9999999997	-54999.9999999997\\
365999.999999997	365999.999999997\\
-90999.9999999993	-90999.9999999993\\
164000	164000\\
-54000.0000000029	-54000.0000000029\\
-164999.999999998	-164999.999999998\\
35999.9999999987	35999.9999999987\\
55000.0000000006	55000.0000000006\\
-54000.000000002	-54000.000000002\\
-92999.9999999982	-92999.9999999982\\
19999.9999999978	19999.9999999978\\
255000.000000001	255000.000000001\\
-53999.9999999994	-53999.9999999994\\
-166000.000000001	-166000.000000001\\
19999.9999999996	19999.9999999996\\
108000	108000\\
-71999.9999999974	-71999.9999999974\\
92000.0000000014	92000.0000000014\\
89999.9999999972	89999.9999999972\\
-90999.9999999993	-90999.9999999993\\
-199999.999999999	-199999.999999999\\
-131000	-131000\\
296000	296000\\
236000	236000\\
-218999.999999997	-218999.999999997\\
164999.999999996	164999.999999996\\
-347999.999999998	-347999.999999998\\
-73999.9999999998	-73999.9999999998\\
311999.999999998	311999.999999998\\
-438999.999999999	-438999.999999999\\
292000.000000002	292000.000000002\\
165000.000000001	165000.000000001\\
-55000.0000000006	-55000.0000000006\\
73999.9999999989	73999.9999999989\\
-183999.999999997	-183999.999999997\\
-109000	-109000\\
127000	127000\\
1000.00000000033	1000.00000000033\\
128000	128000\\
-74000.0000000007	-74000.0000000007\\
-273000	-273000\\
-111000.000000002	-111000.000000002\\
566999.999999999	566999.999999999\\
-35999.999999996	-35999.999999996\\
999.999999999446	999.999999999446\\
-349000	-349000\\
-37000.0000000017	-37000.0000000017\\
422000.000000004	422000.000000004\\
-183000.000000001	-183000.000000001\\
90999.9999999984	90999.9999999984\\
-201999.999999997	-201999.999999997\\
1999.99999999889	1999.99999999889\\
272000.000000002	272000.000000002\\
-254000	-254000\\
164000.000000001	164000.000000001\\
-165000.000000004	-165000.000000004\\
-366000	-366000\\
749999.999999999	749999.999999999\\
-346999.999999998	-346999.999999998\\
-92000.0000000041	-92000.0000000041\\
201000.000000003	201000.000000003\\
-145000.000000001	-145000.000000001\\
53999.9999999976	53999.9999999976\\
-36999.999999999	-36999.999999999\\
54999.9999999988	54999.9999999988\\
38000.0000000011	38000.0000000011\\
52999.9999999982	52999.9999999982\\
-292000.000000002	-292000.000000002\\
476000.000000003	476000.000000003\\
-92000.0000000023	-92000.0000000023\\
-602999.999999999	-602999.999999999\\
182000.000000002	182000.000000002\\
548999.999999999	548999.999999999\\
-163000.000000001	-163000.000000001\\
-532999.999999998	-532999.999999998\\
164999.999999999	164999.999999999\\
312999.999999999	312999.999999999\\
-112000	-112000\\
129999.999999999	129999.999999999\\
-258000.000000002	-258000.000000002\\
203000.000000002	203000.000000002\\
146000.000000001	146000.000000001\\
-459000.000000003	-459000.000000003\\
496000	496000\\
-239000.000000001	-239000.000000001\\
-53999.9999999994	-53999.9999999994\\
72000.0000000009	72000.0000000009\\
-292999.999999999	-292999.999999999\\
367999.999999998	367999.999999998\\
-240000	-240000\\
-90000.0000000016	-90000.0000000016\\
457000	457000\\
-92000.0000000032	-92000.0000000032\\
-218999.999999998	-218999.999999998\\
128000	128000\\
-165000.000000004	-165000.000000004\\
220000.000000003	220000.000000003\\
-110000.000000003	-110000.000000003\\
-17999.999999998	-17999.999999998\\
72000.0000000027	72000.0000000027\\
-181000.000000003	-181000.000000003\\
181000.000000004	181000.000000004\\
-200000.000000001	-200000.000000001\\
201000.000000001	201000.000000001\\
109000.000000001	109000.000000001\\
-237000.000000005	-237000.000000005\\
147000.000000003	147000.000000003\\
-183999.999999999	-183999.999999999\\
-219999.999999998	-219999.999999998\\
384999.999999995	384999.999999995\\
329000.000000001	329000.000000001\\
-401000	-401000\\
70999.9999999988	70999.9999999988\\
-365000	-365000\\
439000	439000\\
109999.999999999	109999.999999999\\
-439999.999999999	-439999.999999999\\
313000.000000001	313000.000000001\\
-276000	-276000\\
365999.999999998	365999.999999998\\
-219000	-219000\\
-183999.999999999	-183999.999999999\\
202999.999999999	202999.999999999\\
125999.999999999	125999.999999999\\
-126000	-126000\\
-56000.0000000009	-56000.0000000009\\
-109999.999999999	-109999.999999999\\
328999.999999999	328999.999999999\\
-309999.999999997	-309999.999999997\\
37000.0000000008	37000.0000000008\\
418999.999999997	418999.999999997\\
-143999.999999997	-143999.999999997\\
-167000.000000002	-167000.000000002\\
-290999.999999998	-290999.999999998\\
217999.999999995	217999.999999995\\
330000.000000002	330000.000000002\\
-400999.999999996	-400999.999999996\\
344999.999999994	344999.999999994\\
-15999.9999999973	-15999.9999999973\\
-385999.999999997	-385999.999999997\\
-201000.000000001	-201000.000000001\\
567999.999999999	567999.999999999\\
18000.0000000016	18000.0000000016\\
-439000.000000001	-439000.000000001\\
402999.999999999	402999.999999999\\
-55999.9999999991	-55999.9999999991\\
-55000.0000000033	-55000.0000000033\\
-126999.999999996	-126999.999999996\\
-37000.0000000008	-37000.0000000008\\
273999.999999997	273999.999999997\\
-144999.999999998	-144999.999999998\\
-2000.00000000333	-2000.00000000333\\
111000.000000001	111000.000000001\\
-256000	-256000\\
273999.999999998	273999.999999998\\
-220000	-220000\\
-145999.999999999	-145999.999999999\\
348000.000000001	348000.000000001\\
-35999.9999999987	-35999.9999999987\\
-38000.0000000038	-38000.0000000038\\
110000	110000\\
-127000	-127000\\
-166000.000000001	-166000.000000001\\
56000.0000000009	56000.0000000009\\
53999.9999999976	53999.9999999976\\
19000.0000000019	19000.0000000019\\
17999.9999999998	17999.9999999998\\
183000	183000\\
-18000.0000000025	-18000.0000000025\\
-532000	-532000\\
294000.000000003	294000.000000003\\
366999.999999999	366999.999999999\\
-313000.000000002	-313000.000000002\\
-109999.999999999	-109999.999999999\\
277000	277000\\
-113000	-113000\\
-161999.999999998	-161999.999999998\\
181999.999999998	181999.999999998\\
-129000	-129000\\
1000.00000000122	1000.00000000122\\
256999.999999999	256999.999999999\\
-277000.000000001	-277000.000000001\\
112000.000000002	112000.000000002\\
0	0\\
34999.9999999966	34999.9999999966\\
-181999.999999999	-181999.999999999\\
255999.999999998	255999.999999998\\
-146000.000000002	-146000.000000002\\
-91999.9999999996	-91999.9999999996\\
182999.999999999	182999.999999999\\
147000.000000001	147000.000000001\\
-145999.999999998	-145999.999999998\\
-185000	-185000\\
-126999.999999999	-126999.999999999\\
-164000.000000002	-164000.000000002\\
548000	548000\\
73999.9999999972	73999.9999999972\\
-999.999999998557	-999.999999998557\\
-364999.999999999	-364999.999999999\\
146000	146000\\
92000.0000000005	92000.0000000005\\
-1000.000000003	-1000.000000003\\
-292000	-292000\\
164000.000000001	164000.000000001\\
127999.999999998	127999.999999998\\
-182000	-182000\\
183000.000000002	183000.000000002\\
-92000.0000000014	-92000.0000000014\\
109000	109000\\
-291000.000000001	-291000.000000001\\
-74999.9999999957	-74999.9999999957\\
603999.999999997	603999.999999997\\
-419000	-419000\\
89999.9999999981	89999.9999999981\\
-128000	-128000\\
184000.000000002	184000.000000002\\
-166000.000000003	-166000.000000003\\
256000	256000\\
-199999.999999997	-199999.999999997\\
17999.9999999989	17999.9999999989\\
255999.999999998	255999.999999998\\
-550000.000000001	-550000.000000001\\
458999.999999999	458999.999999999\\
-219999.999999999	-219999.999999999\\
35999.9999999987	35999.9999999987\\
-18000.0000000016	-18000.0000000016\\
220000.000000001	220000.000000001\\
-275000	-275000\\
36999.9999999981	36999.9999999981\\
219000	219000\\
-90999.9999999984	-90999.9999999984\\
-36000.0000000022	-36000.0000000022\\
-129999.999999999	-129999.999999999\\
167000.000000002	167000.000000002\\
-20000.000000004	-20000.000000004\\
-34999.9999999975	-34999.9999999975\\
-94000.0000000021	-94000.0000000021\\
131000.000000002	131000.000000002\\
-295000.000000002	-295000.000000002\\
256000	256000\\
259000.000000002	259000.000000002\\
-222000.000000003	-222000.000000003\\
-91000.0000000002	-91000.0000000002\\
237999.999999999	237999.999999999\\
-220000	-220000\\
1000.00000000033	1000.00000000033\\
35999.9999999969	35999.9999999969\\
-109999.999999999	-109999.999999999\\
-55000.0000000006	-55000.0000000006\\
184000	184000\\
34999.9999999993	34999.9999999993\\
74999.9999999993	74999.9999999993\\
-20000.0000000005	-20000.0000000005\\
-437999.999999999	-437999.999999999\\
365999.999999999	365999.999999999\\
16999.9999999995	16999.9999999995\\
75000.0000000028	75000.0000000028\\
-349000.000000005	-349000.000000005\\
403000.000000003	403000.000000003\\
127999.999999999	127999.999999999\\
-401999.999999999	-401999.999999999\\
-1000.000000003	-1000.000000003\\
-183000	-183000\\
258000.000000001	258000.000000001\\
16000	16000\\
201999.999999998	201999.999999998\\
-107999.999999998	-107999.999999998\\
70999.9999999988	70999.9999999988\\
-329000.000000001	-329000.000000001\\
17999.9999999998	17999.9999999998\\
111000.000000002	111000.000000002\\
219999.999999999	219999.999999999\\
-313000.000000001	-313000.000000001\\
164999.999999999	164999.999999999\\
130000.000000001	130000.000000001\\
-532999.999999998	-532999.999999998\\
421999.999999996	421999.999999996\\
-54999.9999999988	-54999.9999999988\\
91999.9999999987	91999.9999999987\\
-92000.0000000014	-92000.0000000014\\
-999.999999998557	-999.999999998557\\
-52999.999999999	-52999.999999999\\
199999.999999997	199999.999999997\\
-309999.999999997	-309999.999999997\\
107999.999999996	107999.999999996\\
276000.000000003	276000.000000003\\
-273999.999999999	-273999.999999999\\
52999.9999999973	52999.9999999973\\
-35000.0000000028	-35000.0000000028\\
-128999.999999999	-128999.999999999\\
402999.999999998	402999.999999998\\
-165000	-165000\\
-220000	-220000\\
185000.000000001	185000.000000001\\
107999.999999998	107999.999999998\\
-403000.000000001	-403000.000000001\\
129000	129000\\
330000	330000\\
-220999.999999997	-220999.999999997\\
-199999.999999999	-199999.999999999\\
511999.999999997	511999.999999997\\
-184000	-184000\\
-145000	-145000\\
-91999.9999999996	-91999.9999999996\\
-37000.0000000026	-37000.0000000026\\
457000.000000001	457000.000000001\\
-290999.999999998	-290999.999999998\\
-239000.000000002	-239000.000000002\\
512999.999999999	512999.999999999\\
-148000	-148000\\
-382000	-382000\\
326999.999999997	326999.999999997\\
-52999.999999999	-52999.999999999\\
-92000.0000000005	-92000.0000000005\\
91000.0000000037	91000.0000000037\\
-19000.0000000001	-19000.0000000001\\
-126000.000000004	-126000.000000004\\
72000.0000000001	72000.0000000001\\
72999.9999999995	72999.9999999995\\
-73000.0000000013	-73000.0000000013\\
184000	184000\\
-165999.999999999	-165999.999999999\\
-19000.0000000019	-19000.0000000019\\
204000.000000003	204000.000000003\\
-57000.0000000022	-57000.0000000022\\
-183000.000000002	-183000.000000002\\
92000.0000000014	92000.0000000014\\
-17999.9999999989	-17999.9999999989\\
-128000.000000001	-128000.000000001\\
364999.999999998	364999.999999998\\
-310000.000000002	-310000.000000002\\
109000.000000001	109000.000000001\\
36999.9999999999	36999.9999999999\\
-294000.000000001	-294000.000000001\\
423000.000000004	423000.000000004\\
-147000	-147000\\
-37000.0000000017	-37000.0000000017\\
145999.999999998	145999.999999998\\
-90999.9999999993	-90999.9999999993\\
-110000.000000002	-110000.000000002\\
-255999.999999999	-255999.999999999\\
420000	420000\\
-162999.999999998	-162999.999999998\\
144999.999999998	144999.999999998\\
-162999.999999998	-162999.999999998\\
-149000.000000002	-149000.000000002\\
257999.999999999	257999.999999999\\
256000.000000002	256000.000000002\\
-330000.000000003	-330000.000000003\\
19000.0000000019	19000.0000000019\\
182999.999999999	182999.999999999\\
-293000.000000001	-293000.000000001\\
35999.9999999978	35999.9999999978\\
202000.000000003	202000.000000003\\
-202000.000000003	-202000.000000003\\
19000.000000001	19000.000000001\\
-54999.9999999979	-54999.9999999979\\
365999.999999999	365999.999999999\\
-219999.999999999	-219999.999999999\\
-71999.9999999992	-71999.9999999992\\
236000.000000001	236000.000000001\\
-309999.999999998	-309999.999999998\\
183999.999999997	183999.999999997\\
-94000.0000000003	-94000.0000000003\\
-88999.9999999977	-88999.9999999977\\
144999.999999995	144999.999999995\\
384000.000000001	384000.000000001\\
-638999.999999998	-638999.999999998\\
199999.999999997	199999.999999997\\
-109999.999999999	-109999.999999999\\
183999.999999999	183999.999999999\\
201000.000000002	201000.000000002\\
-476000	-476000\\
73999.9999999981	73999.9999999981\\
-38999.9999999997	-38999.9999999997\\
516000.000000002	516000.000000002\\
-168000	-168000\\
-70999.9999999997	-70999.9999999997\\
-220999.999999998	-220999.999999998\\
93000.0000000008	93000.0000000008\\
218000.000000001	218000.000000001\\
-200000.000000001	-200000.000000001\\
-20000.0000000005	-20000.0000000005\\
222000.000000002	222000.000000002\\
-185000.000000002	-185000.000000002\\
-164000	-164000\\
348000.000000002	348000.000000002\\
-92000.0000000023	-92000.0000000023\\
-291999.999999999	-291999.999999999\\
345999.999999996	345999.999999996\\
-199999.999999999	-199999.999999999\\
55000.0000000015	55000.0000000015\\
-55000.0000000006	-55000.0000000006\\
18000.0000000007	18000.0000000007\\
219999.999999999	219999.999999999\\
-56000.0000000009	-56000.0000000009\\
-292000.000000002	-292000.000000002\\
166000.000000005	166000.000000005\\
-3000.00000000455	-3000.00000000455\\
131000.000000001	131000.000000001\\
-75000.000000001	-75000.000000001\\
-254999.999999999	-254999.999999999\\
327999.999999998	327999.999999998\\
75000.000000001	75000.000000001\\
-496000.000000003	-496000.000000003\\
440000.000000001	440000.000000001\\
37000.0000000008	37000.0000000008\\
-128000.000000001	-128000.000000001\\
-1000.00000000122	-1000.00000000122\\
-52999.9999999973	-52999.9999999973\\
-314000.000000001	-314000.000000001\\
404999.999999999	404999.999999999\\
89999.9999999972	89999.9999999972\\
-89999.9999999972	-89999.9999999972\\
-239000.000000003	-239000.000000003\\
-35999.9999999996	-35999.9999999996\\
273999.999999998	273999.999999998\\
-73999.9999999989	-73999.9999999989\\
2000.00000000422	2000.00000000422\\
108999.999999996	108999.999999996\\
-128999.999999997	-128999.999999997\\
257999.999999999	257999.999999999\\
-93000.0000000017	-93000.0000000017\\
-458000.000000001	-458000.000000001\\
275999.999999999	275999.999999999\\
0	0\\
-148000.000000001	-148000.000000001\\
294000	294000\\
-238000.000000001	-238000.000000001\\
458000.000000001	458000.000000001\\
-349000.000000001	-349000.000000001\\
-201000	-201000\\
441000.000000002	441000.000000002\\
-131000.000000001	-131000.000000001\\
-107000	-107000\\
-148000.000000002	-148000.000000002\\
36000.0000000014	36000.0000000014\\
74999.9999999975	74999.9999999975\\
346000.000000002	346000.000000002\\
-255000	-255000\\
-165000	-165000\\
91999.9999999987	91999.9999999987\\
-56000.0000000009	-56000.0000000009\\
-54999.9999999997	-54999.9999999997\\
348999.999999999	348999.999999999\\
-110999.999999998	-110999.999999998\\
73999.9999999981	73999.9999999981\\
-91999.9999999987	-91999.9999999987\\
-328999.999999997	-328999.999999997\\
568000	568000\\
-606000.000000004	-606000.000000004\\
258000.000000002	258000.000000002\\
53999.9999999985	53999.9999999985\\
-238000	-238000\\
202000	202000\\
109000.000000004	109000.000000004\\
-145000	-145000\\
237000.000000002	237000.000000002\\
127999.999999997	127999.999999997\\
-458000.000000001	-458000.000000001\\
56000.0000000009	56000.0000000009\\
8.88178419700125e-10	8.88178419700125e-10\\
-38000.0000000038	-38000.0000000038\\
129000.000000001	129000.000000001\\
220000.000000001	220000.000000001\\
-73999.9999999989	-73999.9999999989\\
-365000.000000002	-365000.000000002\\
127000.000000002	127000.000000002\\
293999.999999997	293999.999999997\\
-129999.999999999	-129999.999999999\\
-88999.9999999986	-88999.9999999986\\
70999.9999999971	70999.9999999971\\
-163000.000000001	-163000.000000001\\
182000.000000001	182000.000000001\\
-347999.999999998	-347999.999999998\\
201999.999999998	201999.999999998\\
163999.999999999	163999.999999999\\
73999.9999999998	73999.9999999998\\
-182999.999999998	-182999.999999998\\
237999.999999997	237999.999999997\\
-164999.999999999	-164999.999999999\\
-348999.999999997	-348999.999999997\\
19999.9999999996	19999.9999999996\\
274000.000000001	274000.000000001\\
54999.9999999953	54999.9999999953\\
3.5527136788005e-09	3.5527136788005e-09\\
-37000.0000000044	-37000.0000000044\\
-108999.999999999	-108999.999999999\\
-19999.9999999978	-19999.9999999978\\
332000	332000\\
-185000	-185000\\
18999.9999999992	18999.9999999992\\
238000	238000\\
-440000.000000001	-440000.000000001\\
1000.00000000122	1000.00000000122\\
54999.9999999988	54999.9999999988\\
218000	218000\\
-33999.9999999971	-33999.9999999971\\
-94000.0000000038	-94000.0000000038\\
18999.9999999992	18999.9999999992\\
-273999.999999999	-273999.999999999\\
311000.000000002	311000.000000002\\
237999.999999999	237999.999999999\\
-346999.999999998	-346999.999999998\\
-204000.000000001	-204000.000000001\\
203999.999999997	203999.999999997\\
219000	219000\\
-294000.000000001	-294000.000000001\\
258000	258000\\
-294000	-294000\\
36000.0000000005	36000.0000000005\\
423000	423000\\
-441000.000000001	-441000.000000001\\
19000.0000000019	19000.0000000019\\
255999.999999999	255999.999999999\\
55000.0000000015	55000.0000000015\\
-35999.9999999978	-35999.9999999978\\
-184000.000000004	-184000.000000004\\
129000.000000001	129000.000000001\\
-404000.000000003	-404000.000000003\\
404000.000000003	404000.000000003\\
-293000.000000003	-293000.000000003\\
309999.999999999	309999.999999999\\
20000.0000000005	20000.0000000005\\
-295000	-295000\\
423000.000000001	423000.000000001\\
-494000	-494000\\
344999.999999999	344999.999999999\\
57999.9999999998	57999.9999999998\\
-366999.999999999	-366999.999999999\\
273000.000000001	273000.000000001\\
-163000.000000003	-163000.000000003\\
128000.000000001	128000.000000001\\
36000.0000000031	36000.0000000031\\
-183000.000000002	-183000.000000002\\
17999.9999999998	17999.9999999998\\
146999.999999998	146999.999999998\\
-18000.0000000007	-18000.0000000007\\
-220000.000000001	-220000.000000001\\
165000	165000\\
236999.999999997	236999.999999997\\
-199999.999999998	-199999.999999998\\
-221000.000000002	-221000.000000002\\
165999.999999999	165999.999999999\\
292000.000000002	292000.000000002\\
-311000.000000003	-311000.000000003\\
329000	329000\\
-254999.999999999	-254999.999999999\\
-440000.000000001	-440000.000000001\\
293000	293000\\
310000	310000\\
-308999.999999999	-308999.999999999\\
273000.000000001	273000.000000001\\
55999.9999999991	55999.9999999991\\
-166000.000000002	-166000.000000002\\
-164000	-164000\\
35999.9999999996	35999.9999999996\\
184000.000000001	184000.000000001\\
72999.9999999986	72999.9999999986\\
-348999.999999997	-348999.999999997\\
202999.999999999	202999.999999999\\
-1000.00000000389	-1000.00000000389\\
56000.0000000036	56000.0000000036\\
-313000.000000001	-313000.000000001\\
130000	130000\\
217999.999999997	217999.999999997\\
75000.0000000002	75000.0000000002\\
-478000	-478000\\
258000.000000002	258000.000000002\\
384000.000000002	384000.000000002\\
-531000.000000002	-531000.000000002\\
164999.999999998	164999.999999998\\
199999.999999999	199999.999999999\\
-72000.0000000001	-72000.0000000001\\
-291999.999999998	-291999.999999998\\
70999.9999999971	70999.9999999971\\
405000.000000003	405000.000000003\\
-460000.000000003	-460000.000000003\\
204000.000000001	204000.000000001\\
89000.0000000039	89000.0000000039\\
-255000.000000002	-255000.000000002\\
-36000.0000000005	-36000.0000000005\\
90000.0000000007	90000.0000000007\\
256999.999999998	256999.999999998\\
-274000.000000002	-274000.000000002\\
91000.0000000011	91000.0000000011\\
128000.000000002	128000.000000002\\
55000.0000000015	55000.0000000015\\
-274999.999999999	-274999.999999999\\
-91000.0000000002	-91000.0000000002\\
184000	184000\\
-37999.9999999985	-37999.9999999985\\
-256999.999999999	-256999.999999999\\
532999.999999998	532999.999999998\\
-56000	-56000\\
-238000.000000003	-238000.000000003\\
-35999.9999999996	-35999.9999999996\\
-38000.0000000011	-38000.0000000011\\
220999.999999999	220999.999999999\\
17000.0000000012	17000.0000000012\\
2000.00000000156	2000.00000000156\\
-185000	-185000\\
19999.9999999969	19999.9999999969\\
108000.000000001	108000.000000001\\
-126999.999999999	-126999.999999999\\
-36999.9999999999	-36999.9999999999\\
257000.000000001	257000.000000001\\
-74000.0000000016	-74000.0000000016\\
-36999.9999999981	-36999.9999999981\\
-71999.9999999992	-71999.9999999992\\
0	0\\
-20000.0000000022	-20000.0000000022\\
258000	258000\\
-349000.000000001	-349000.000000001\\
38000.0000000002	38000.0000000002\\
126000	126000\\
131000	131000\\
-241000	-241000\\
148999.999999998	148999.999999998\\
-20000.0000000005	-20000.0000000005\\
74000.0000000016	74000.0000000016\\
-348000	-348000\\
54999.9999999997	54999.9999999997\\
439999.999999998	439999.999999998\\
-56999.9999999986	-56999.9999999986\\
-419000	-419000\\
457000	457000\\
-347000.000000001	-347000.000000001\\
-74999.9999999957	-74999.9999999957\\
329999.999999999	329999.999999999\\
-145000.000000002	-145000.000000002\\
-129999.999999996	-129999.999999996\\
257999.999999997	257999.999999997\\
-312999.999999996	-312999.999999996\\
312999.999999997	312999.999999997\\
-19999.9999999969	-19999.9999999969\\
-328000.000000001	-328000.000000001\\
199999.999999998	199999.999999998\\
19999.9999999996	19999.9999999996\\
72000.0000000001	72000.0000000001\\
-54999.999999997	-54999.999999997\\
-165000.000000001	-165000.000000001\\
293999.999999998	293999.999999998\\
-184000	-184000\\
-236999.999999999	-236999.999999999\\
621999.999999999	621999.999999999\\
-294000	-294000\\
-163000	-163000\\
236999.999999999	236999.999999999\\
-1.77635683940025e-09	-1.77635683940025e-09\\
-493999.999999999	-493999.999999999\\
511999.999999999	511999.999999999\\
19000.0000000019	19000.0000000019\\
-458000.000000004	-458000.000000004\\
367000.000000002	367000.000000002\\
-112000	-112000\\
241000.000000002	241000.000000002\\
-131000	-131000\\
-127000.000000002	-127000.000000002\\
-53999.9999999994	-53999.9999999994\\
-129000	-129000\\
384000.000000001	384000.000000001\\
-146000.000000001	-146000.000000001\\
-110000.000000002	-110000.000000002\\
74000.0000000007	74000.0000000007\\
272999.999999998	272999.999999998\\
-382999.999999996	-382999.999999996\\
127999.999999998	127999.999999998\\
183000	183000\\
-184000.000000001	-184000.000000001\\
-254999.999999999	-254999.999999999\\
547999.999999997	547999.999999997\\
-328999.999999997	-328999.999999997\\
92000.0000000014	92000.0000000014\\
-54999.9999999997	-54999.9999999997\\
-54999.9999999988	-54999.9999999988\\
364999.999999996	364999.999999996\\
-199999.999999998	-199999.999999998\\
-239000.000000002	-239000.000000002\\
-71999.9999999992	-71999.9999999992\\
146000	146000\\
199999.999999999	199999.999999999\\
2000.00000000244	2000.00000000244\\
17999.999999998	17999.999999998\\
-386000	-386000\\
-72000.0000000027	-72000.0000000027\\
366000.000000001	366000.000000001\\
146000.000000003	146000.000000003\\
37999.9999999994	37999.9999999994\\
-276000.000000002	-276000.000000002\\
-183000.000000001	-183000.000000001\\
36999.9999999981	36999.9999999981\\
311000.000000002	311000.000000002\\
-55000.0000000015	-55000.0000000015\\
74000.0000000007	74000.0000000007\\
-348999.999999997	-348999.999999997\\
184999.999999999	184999.999999999\\
125999.999999998	125999.999999998\\
-274000	-274000\\
311000.000000001	311000.000000001\\
-71999.9999999983	-71999.9999999983\\
-146999.999999998	-146999.999999998\\
127999.999999998	127999.999999998\\
182000	182000\\
-364000	-364000\\
-75000.0000000028	-75000.0000000028\\
183999.999999998	183999.999999998\\
54000.0000000011	54000.0000000011\\
111000	111000\\
-312000.000000002	-312000.000000002\\
238000.000000003	238000.000000003\\
18999.9999999975	18999.9999999975\\
-256999.999999998	-256999.999999998\\
238999.999999999	238999.999999999\\
-258000.000000003	-258000.000000003\\
275000.000000002	275000.000000002\\
93000.0000000017	93000.0000000017\\
-146999.999999999	-146999.999999999\\
71999.9999999965	71999.9999999965\\
-420000.000000001	-420000.000000001\\
403000.000000003	403000.000000003\\
-38000.0000000047	-38000.0000000047\\
-89999.9999999963	-89999.9999999963\\
-129000.000000003	-129000.000000003\\
129000.000000001	129000.000000001\\
108999.999999997	108999.999999997\\
-18999.9999999983	-18999.9999999983\\
94000.0000000021	94000.0000000021\\
-113000.000000004	-113000.000000004\\
3000.00000000278	3000.00000000278\\
-240000.000000002	-240000.000000002\\
19000.0000000001	19000.0000000001\\
311000	311000\\
-55000.0000000015	-55000.0000000015\\
-310999.999999998	-310999.999999998\\
293999.999999999	293999.999999999\\
-1999.99999999978	-1999.99999999978\\
18999.9999999983	18999.9999999983\\
-127000	-127000\\
290999.999999999	290999.999999999\\
-218000	-218000\\
-74000.0000000007	-74000.0000000007\\
-19000.0000000019	-19000.0000000019\\
-52999.9999999982	-52999.9999999982\\
235999.999999999	235999.999999999\\
37999.9999999985	37999.9999999985\\
-221000.000000001	-221000.000000001\\
-217999.999999997	-217999.999999997\\
364999.999999997	364999.999999997\\
73000.0000000004	73000.0000000004\\
-364999.999999999	-364999.999999999\\
272999.999999999	272999.999999999\\
203000.000000003	203000.000000003\\
-184000.000000002	-184000.000000002\\
-53999.9999999994	-53999.9999999994\\
-130000.000000003	-130000.000000003\\
-145000	-145000\\
421000.000000001	421000.000000001\\
-201000.000000001	-201000.000000001\\
-999.999999999446	-999.999999999446\\
72999.9999999968	72999.9999999968\\
-35999.9999999987	-35999.9999999987\\
19000.0000000001	19000.0000000001\\
72000.0000000018	72000.0000000018\\
-127000.000000002	-127000.000000002\\
109000	109000\\
-92000.0000000023	-92000.0000000023\\
-217999.999999997	-217999.999999997\\
180999.999999999	180999.999999999\\
185000.000000002	185000.000000002\\
181999.999999998	181999.999999998\\
-749999.999999999	-749999.999999999\\
492999.999999998	492999.999999998\\
313000.000000002	313000.000000002\\
-331000	-331000\\
-365000.000000001	-365000.000000001\\
401999.999999997	401999.999999997\\
-35999.9999999996	-35999.9999999996\\
-129000.000000002	-129000.000000002\\
257000.000000002	257000.000000002\\
-349000.000000002	-349000.000000002\\
-34999.9999999975	-34999.9999999975\\
384000.000000002	384000.000000002\\
91000.0000000011	91000.0000000011\\
-365000.000000001	-365000.000000001\\
52999.9999999982	52999.9999999982\\
56000.0000000018	56000.0000000018\\
165999.999999999	165999.999999999\\
-259000	-259000\\
203999.999999998	203999.999999998\\
-330999.999999999	-330999.999999999\\
255999.999999997	255999.999999997\\
-236999.999999999	-236999.999999999\\
292000.000000002	292000.000000002\\
-126999.999999998	-126999.999999998\\
292999.999999999	292999.999999999\\
-185000	-185000\\
-90000.0000000016	-90000.0000000016\\
-91999.9999999952	-91999.9999999952\\
165999.999999997	165999.999999997\\
88999.9999999995	88999.9999999995\\
-271999.999999998	-271999.999999998\\
34999.9999999957	34999.9999999957\\
220000	220000\\
-73000.0000000013	-73000.0000000013\\
-311000.000000001	-311000.000000001\\
201000.000000001	201000.000000001\\
255999.999999996	255999.999999996\\
-126999.999999996	-126999.999999996\\
-148000	-148000\\
-274000.000000002	-274000.000000002\\
366999.999999998	366999.999999998\\
475000.000000001	475000.000000001\\
-696000.000000002	-696000.000000002\\
112000.000000003	112000.000000003\\
179999.999999998	179999.999999998\\
-217000	-217000\\
254000	254000\\
-34999.9999999975	-34999.9999999975\\
-90999.9999999993	-90999.9999999993\\
-1999.99999999889	-1999.99999999889\\
-146000.000000001	-146000.000000001\\
-90000.0000000025	-90000.0000000025\\
347000	347000\\
108999.999999999	108999.999999999\\
-127000.000000002	-127000.000000002\\
-55999.9999999974	-55999.9999999974\\
-255000.000000001	-255000.000000001\\
35000.0000000019	35000.0000000019\\
202999.999999998	202999.999999998\\
128000.000000003	128000.000000003\\
-38000.0000000029	-38000.0000000029\\
-109000.000000002	-109000.000000002\\
-109999.999999998	-109999.999999998\\
19000.0000000001	19000.0000000001\\
17999.9999999998	17999.9999999998\\
90999.9999999975	90999.9999999975\\
219999.999999999	219999.999999999\\
-73999.9999999989	-73999.9999999989\\
-345999.999999997	-345999.999999997\\
181999.999999996	181999.999999996\\
-19000.000000001	-19000.000000001\\
129000	129000\\
-238000.000000001	-238000.000000001\\
91999.9999999996	91999.9999999996\\
90000.0000000007	90000.0000000007\\
55999.9999999965	55999.9999999965\\
-17999.9999999954	-17999.9999999954\\
-349000	-349000\\
128999.999999997	128999.999999997\\
74000.0000000007	74000.0000000007\\
162999.999999998	162999.999999998\\
-127000	-127000\\
145999.999999999	145999.999999999\\
-183000.000000001	-183000.000000001\\
-73000.0000000013	-73000.0000000013\\
54000.0000000011	54000.0000000011\\
331000.000000002	331000.000000002\\
-73999.9999999989	-73999.9999999989\\
-329000	-329000\\
35999.9999999987	35999.9999999987\\
-18999.9999999992	-18999.9999999992\\
459000.000000001	459000.000000001\\
-73000.0000000004	-73000.0000000004\\
-349000.000000002	-349000.000000002\\
-91000.0000000011	-91000.0000000011\\
218999.999999998	218999.999999998\\
75000.0000000019	75000.0000000019\\
-312999.999999997	-312999.999999997\\
17999.9999999962	17999.9999999962\\
130000	130000\\
200999.999999999	200999.999999999\\
-37999.9999999994	-37999.9999999994\\
-164000.000000002	-164000.000000002\\
-17999.9999999989	-17999.9999999989\\
36999.999999999	36999.999999999\\
72000.0000000001	72000.0000000001\\
36999.9999999981	36999.9999999981\\
19000.0000000019	19000.0000000019\\
-56000.0000000027	-56000.0000000027\\
-107999.999999998	-107999.999999998\\
-93999.9999999976	-93999.9999999976\\
147999.999999995	147999.999999995\\
202000.000000003	202000.000000003\\
-129999.999999997	-129999.999999997\\
-329000.000000001	-329000.000000001\\
514000	514000\\
-350000	-350000\\
20999.9999999999	20999.9999999999\\
34000.0000000016	34000.0000000016\\
128999.999999997	128999.999999997\\
147000.000000003	147000.000000003\\
-476000.000000001	-476000.000000001\\
329000.000000001	329000.000000001\\
36999.999999999	36999.999999999\\
-91999.9999999987	-91999.9999999987\\
-237000.000000002	-237000.000000002\\
181999.999999999	181999.999999999\\
183000	183000\\
-181999.999999999	-181999.999999999\\
201000	201000\\
-202000	-202000\\
-3.5527136788005e-09	-3.5527136788005e-09\\
-72999.9999999977	-72999.9999999977\\
238999.999999999	238999.999999999\\
-258000	-258000\\
185000.000000002	185000.000000002\\
-2000.00000000333	-2000.00000000333\\
-401999.999999998	-401999.999999998\\
642000.000000002	642000.000000002\\
70999.9999999979	70999.9999999979\\
-436999.999999999	-436999.999999999\\
-94000.0000000012	-94000.0000000012\\
1999.99999999978	1999.99999999978\\
108999.999999997	108999.999999997\\
91000.0000000011	91000.0000000011\\
1000.00000000389	1000.00000000389\\
-36999.9999999999	-36999.9999999999\\
220000	220000\\
-311000.000000001	-311000.000000001\\
-20000.0000000022	-20000.0000000022\\
148000	148000\\
145999.999999999	145999.999999999\\
-165000	-165000\\
-382999.999999998	-382999.999999998\\
491999.999999996	491999.999999996\\
-199999.999999999	-199999.999999999\\
18000.0000000025	18000.0000000025\\
239000	239000\\
-39000.0000000015	-39000.0000000015\\
-198999.999999997	-198999.999999997\\
-166000.000000001	-166000.000000001\\
184000.000000002	184000.000000002\\
-19000.0000000019	-19000.0000000019\\
163999.999999999	163999.999999999\\
-126999.999999999	-126999.999999999\\
146000.000000002	146000.000000002\\
-71999.9999999992	-71999.9999999992\\
52999.9999999982	52999.9999999982\\
-18000.0000000025	-18000.0000000025\\
-510999.999999996	-510999.999999996\\
382999.999999998	382999.999999998\\
403000	403000\\
-677000	-677000\\
236999.999999997	236999.999999997\\
112000.000000005	112000.000000005\\
289999.999999996	289999.999999996\\
-198999.999999996	-198999.999999996\\
-110000.000000001	-110000.000000001\\
-350000.000000003	-350000.000000003\\
332000.000000002	332000.000000002\\
34999.9999999984	34999.9999999984\\
111000.000000001	111000.000000001\\
-294000.000000001	-294000.000000001\\
203000.000000001	203000.000000001\\
-167000.000000002	-167000.000000002\\
166000	166000\\
-109000.000000002	-109000.000000002\\
-130000	-130000\\
386000	386000\\
-312000.000000001	-312000.000000001\\
72999.9999999995	72999.9999999995\\
147000	147000\\
164999.999999999	164999.999999999\\
-404000.000000001	-404000.000000001\\
-52999.9999999999	-52999.9999999999\\
401000	401000\\
-256000.000000001	-256000.000000001\\
93000.0000000035	93000.0000000035\\
-258000.000000003	-258000.000000003\\
238000	238000\\
167000.000000003	167000.000000003\\
-130999.999999998	-130999.999999998\\
-200000.000000001	-200000.000000001\\
-164999.999999998	-164999.999999998\\
256999.999999999	256999.999999999\\
163999.999999998	163999.999999998\\
55000.0000000024	55000.0000000024\\
-274999.999999999	-274999.999999999\\
-109000.000000002	-109000.000000002\\
218999.999999998	218999.999999998\\
184000.000000001	184000.000000001\\
-148000.000000001	-148000.000000001\\
-200000	-200000\\
90999.9999999993	90999.9999999993\\
184000.000000002	184000.000000002\\
-203000.000000002	-203000.000000002\\
129000	129000\\
73000.0000000004	73000.0000000004\\
-146000.000000001	-146000.000000001\\
-146999.999999998	-146999.999999998\\
237999.999999998	237999.999999998\\
-127999.999999998	-127999.999999998\\
-54000.0000000003	-54000.0000000003\\
-1000.00000000211	-1000.00000000211\\
420999.999999999	420999.999999999\\
-310999.999999999	-310999.999999999\\
-239000.000000002	-239000.000000002\\
478000.000000004	478000.000000004\\
-92000.0000000005	-92000.0000000005\\
-148000.000000002	-148000.000000002\\
-53000.0000000017	-53000.0000000017\\
-56999.9999999995	-56999.9999999995\\
149000	149000\\
-75999.9999999969	-75999.9999999969\\
-236000.000000005	-236000.000000005\\
750000.000000003	750000.000000003\\
-567000.000000002	-567000.000000002\\
-184999.999999996	-184999.999999996\\
240999.999999996	240999.999999996\\
254000.000000002	254000.000000002\\
-165000	-165000\\
20999.999999999	20999.999999999\\
-295999.999999998	-295999.999999998\\
-52999.9999999999	-52999.9999999999\\
346999.999999999	346999.999999999\\
200999.999999998	200999.999999998\\
-217999.999999996	-217999.999999996\\
-167000.000000003	-167000.000000003\\
130000.000000003	130000.000000003\\
-73000.0000000004	-73000.0000000004\\
-2000.00000000067	-2000.00000000067\\
37999.9999999985	37999.9999999985\\
109000	109000\\
999.999999999446	999.999999999446\\
-999.999999999446	-999.999999999446\\
-89999.999999999	-89999.999999999\\
-130000.000000003	-130000.000000003\\
1000.00000000122	1000.00000000122\\
385000.000000002	385000.000000002\\
-294000.000000001	-294000.000000001\\
-274999.999999999	-274999.999999999\\
367999.999999998	367999.999999998\\
70999.9999999988	70999.9999999988\\
-162999.999999998	-162999.999999998\\
-91999.999999997	-91999.999999997\\
512999.999999998	512999.999999998\\
-770000.000000002	-770000.000000002\\
457999.999999999	457999.999999999\\
91999.9999999978	91999.9999999978\\
-256999.999999999	-256999.999999999\\
147000.000000001	147000.000000001\\
-383999.999999998	-383999.999999998\\
494000	494000\\
181999.999999999	181999.999999999\\
-602999.999999997	-602999.999999997\\
219999.999999999	219999.999999999\\
108000	108000\\
-107999.999999998	-107999.999999998\\
90999.9999999993	90999.9999999993\\
-183000	-183000\\
293000.000000001	293000.000000001\\
-19000.0000000001	-19000.0000000001\\
-35999.9999999996	-35999.9999999996\\
-293000.000000002	-293000.000000002\\
145999.999999998	145999.999999998\\
129000.000000001	129000.000000001\\
-367000.000000002	-367000.000000002\\
201999.999999998	201999.999999998\\
420999.999999998	420999.999999998\\
-384999.999999999	-384999.999999999\\
-73000.0000000013	-73000.0000000013\\
256999.999999999	256999.999999999\\
-312000	-312000\\
403000	403000\\
-365999.999999998	-365999.999999998\\
36000.0000000014	36000.0000000014\\
18999.9999999983	18999.9999999983\\
165000	165000\\
-18999.9999999966	-18999.9999999966\\
-18999.9999999983	-18999.9999999983\\
-108000.000000003	-108000.000000003\\
-165999.999999998	-165999.999999998\\
366999.999999997	366999.999999997\\
-165999.999999999	-165999.999999999\\
-273000.000000001	-273000.000000001\\
309000	309000\\
148999.999999997	148999.999999997\\
34999.9999999993	34999.9999999993\\
-365999.999999999	-365999.999999999\\
-37000.0000000008	-37000.0000000008\\
679000	679000\\
-404000	-404000\\
-238000.000000001	-238000.000000001\\
202000.000000001	202000.000000001\\
-294000	-294000\\
130000.000000001	130000.000000001\\
108000	108000\\
347999.999999998	347999.999999998\\
-585000	-585000\\
165000.000000001	165000.000000001\\
255999.999999999	255999.999999999\\
-1000.00000000122	-1000.00000000122\\
18999.9999999983	18999.9999999983\\
-439000.000000001	-439000.000000001\\
91000.0000000002	91000.0000000002\\
184000	184000\\
309999.999999998	309999.999999998\\
-201000	-201000\\
-494000.000000001	-494000.000000001\\
475999.999999998	475999.999999998\\
92000.0000000014	92000.0000000014\\
-165999.999999998	-165999.999999998\\
-549000	-549000\\
329999.999999999	329999.999999999\\
420999.999999999	420999.999999999\\
19000.000000001	19000.000000001\\
-148000.000000004	-148000.000000004\\
-217999.999999996	-217999.999999996\\
383000.000000001	383000.000000001\\
-584999.999999998	-584999.999999998\\
201000.000000001	201000.000000001\\
54999.9999999988	54999.9999999988\\
-36000.0000000005	-36000.0000000005\\
-20000.0000000005	-20000.0000000005\\
276999.999999999	276999.999999999\\
-75000.0000000019	-75000.0000000019\\
-91000.0000000011	-91000.0000000011\\
54999.9999999988	54999.9999999988\\
-73000.0000000013	-73000.0000000013\\
109000.000000001	109000.000000001\\
-237000.000000003	-237000.000000003\\
-73999.9999999989	-73999.9999999989\\
420999.999999998	420999.999999998\\
203000.000000004	203000.000000004\\
-332000.000000005	-332000.000000005\\
-656999.999999997	-656999.999999997\\
583999.999999998	583999.999999998\\
240000.000000001	240000.000000001\\
-56000.0000000009	-56000.0000000009\\
-294000	-294000\\
-235999.999999998	-235999.999999998\\
292000	292000\\
239000.000000001	239000.000000001\\
-167000.000000002	-167000.000000002\\
37999.9999999994	37999.9999999994\\
74000.0000000043	74000.0000000043\\
-166000.000000004	-166000.000000004\\
293000.000000001	293000.000000001\\
-365000	-365000\\
-167000.000000002	-167000.000000002\\
277000.000000001	277000.000000001\\
16999.9999999977	16999.9999999977\\
1000.000000003	1000.000000003\\
162999.999999996	162999.999999996\\
-382999.999999999	-382999.999999999\\
255999.999999998	255999.999999998\\
201000	201000\\
-254999.999999996	-254999.999999996\\
-93000.0000000008	-93000.0000000008\\
-147000.000000002	-147000.000000002\\
350000.000000002	350000.000000002\\
52999.999999999	52999.999999999\\
-474999.999999999	-474999.999999999\\
74000.0000000007	74000.0000000007\\
619999.999999999	619999.999999999\\
-326999.999999999	-326999.999999999\\
-258000	-258000\\
36999.9999999981	36999.9999999981\\
459000.000000002	459000.000000002\\
-20000.0000000013	-20000.0000000013\\
-292999.999999997	-292999.999999997\\
-475000	-475000\\
292999.999999998	292999.999999998\\
366000.000000001	366000.000000001\\
-220000	-220000\\
128000	128000\\
90999.9999999993	90999.9999999993\\
75000.0000000028	75000.0000000028\\
-331000.000000001	-331000.000000001\\
-89999.9999999981	-89999.9999999981\\
308999.999999998	308999.999999998\\
-89999.9999999981	-89999.9999999981\\
-147000.000000002	-147000.000000002\\
147000.000000003	147000.000000003\\
-56000.0000000009	-56000.0000000009\\
-126999.999999998	-126999.999999998\\
420000	420000\\
-34999.9999999993	-34999.9999999993\\
-203000.000000002	-203000.000000002\\
-145000	-145000\\
-112000.000000002	-112000.000000002\\
222000	222000\\
-38000.0000000002	-38000.0000000002\\
-165000	-165000\\
404000.000000002	404000.000000002\\
-165000.000000001	-165000.000000001\\
-184000.000000001	-184000.000000001\\
459000.000000001	459000.000000001\\
-478000.000000002	-478000.000000002\\
-16000	-16000\\
217999.999999998	217999.999999998\\
239000.000000002	239000.000000002\\
-184000.000000001	-184000.000000001\\
-402000.000000002	-402000.000000002\\
384000.000000002	384000.000000002\\
-475000.000000003	-475000.000000003\\
438000.000000001	438000.000000001\\
312000	312000\\
-440000	-440000\\
56000.0000000045	56000.0000000045\\
-56000.0000000045	-56000.0000000045\\
276000.000000002	276000.000000002\\
90000.0000000007	90000.0000000007\\
-255999.999999998	-255999.999999998\\
110999.999999998	110999.999999998\\
-386000.000000002	-386000.000000002\\
111000	111000\\
71999.9999999983	71999.9999999983\\
239000	239000\\
-127999.999999998	-127999.999999998\\
-165000.000000002	-165000.000000002\\
72999.9999999986	72999.9999999986\\
54000.0000000011	54000.0000000011\\
-33999.9999999989	-33999.9999999989\\
252999.999999998	252999.999999998\\
-69999.9999999985	-69999.9999999985\\
-149000.000000001	-149000.000000001\\
-108000.000000001	-108000.000000001\\
35000.0000000019	35000.0000000019\\
-17999.9999999989	-17999.9999999989\\
-108000.000000001	-108000.000000001\\
346000.000000001	346000.000000001\\
-164999.999999998	-164999.999999998\\
-127000.000000005	-127000.000000005\\
421000.000000002	421000.000000002\\
-496000	-496000\\
39000.0000000041	39000.0000000041\\
529999.999999999	529999.999999999\\
-239000	-239000\\
-290999.999999999	-290999.999999999\\
180999.999999999	180999.999999999\\
258000.000000004	258000.000000004\\
-294000	-294000\\
129000	129000\\
-257000.000000004	-257000.000000004\\
-201000	-201000\\
255999.999999999	255999.999999999\\
256000.000000003	256000.000000003\\
184000	184000\\
-274999.999999999	-274999.999999999\\
8.88178419700125e-10	8.88178419700125e-10\\
-166000.000000001	-166000.000000001\\
-88999.9999999977	-88999.9999999977\\
565999.999999999	565999.999999999\\
-385000	-385000\\
-72000.0000000018	-72000.0000000018\\
-201999.999999998	-201999.999999998\\
366000	366000\\
311999.999999998	311999.999999998\\
-513999.999999999	-513999.999999999\\
111000.000000001	111000.000000001\\
-238000	-238000\\
128000	128000\\
310999.999999996	310999.999999996\\
-18999.9999999983	-18999.9999999983\\
-108000.000000001	-108000.000000001\\
-203000	-203000\\
202000	202000\\
72999.9999999977	72999.9999999977\\
-182999.999999999	-182999.999999999\\
54999.9999999997	54999.9999999997\\
0	0\\
-90999.9999999993	-90999.9999999993\\
-110999.999999997	-110999.999999997\\
110999.999999997	110999.999999997\\
292000.000000001	292000.000000001\\
-18000.0000000007	-18000.0000000007\\
-183000.000000002	-183000.000000002\\
-90999.9999999975	-90999.9999999975\\
71999.9999999992	71999.9999999992\\
-146000.000000002	-146000.000000002\\
258000.000000003	258000.000000003\\
-58000.0000000016	-58000.0000000016\\
186000.000000002	186000.000000002\\
-643000.000000002	-643000.000000002\\
349000	349000\\
439000.000000002	439000.000000002\\
-54000.0000000011	-54000.0000000011\\
-587000	-587000\\
-36999.999999999	-36999.999999999\\
477999.999999998	477999.999999998\\
-75999.9999999996	-75999.9999999996\\
40000.0000000018	40000.0000000018\\
-461000	-461000\\
239999.999999998	239999.999999998\\
456999.999999998	456999.999999998\\
-127999.999999997	-127999.999999997\\
-512000.000000003	-512000.000000003\\
-183999.999999999	-183999.999999999\\
603999.999999998	603999.999999998\\
257000	257000\\
-476000.000000001	-476000.000000001\\
-256999.999999999	-256999.999999999\\
294000	294000\\
274000	274000\\
-349000	-349000\\
21000.0000000008	21000.0000000008\\
437000	437000\\
-310000.000000001	-310000.000000001\\
-349000	-349000\\
606000.000000002	606000.000000002\\
-496000.000000003	-496000.000000003\\
110000	110000\\
294000.000000003	294000.000000003\\
72999.9999999995	72999.9999999995\\
-678000.000000003	-678000.000000003\\
128000.000000001	128000.000000001\\
440000	440000\\
109000.000000001	109000.000000001\\
-254999.999999999	-254999.999999999\\
-202000.000000002	-202000.000000002\\
311000.000000001	311000.000000001\\
54999.9999999997	54999.9999999997\\
-330000.000000002	-330000.000000002\\
19999.9999999996	19999.9999999996\\
126000	126000\\
19999.9999999996	19999.9999999996\\
-20000.0000000005	-20000.0000000005\\
130000.000000003	130000.000000003\\
-220000.000000003	-220000.000000003\\
-37999.9999999994	-37999.9999999994\\
366999.999999997	366999.999999997\\
-183000.000000001	-183000.000000001\\
-146000	-146000\\
144999.999999999	144999.999999999\\
-162999.999999999	-162999.999999999\\
-130000.000000001	-130000.000000001\\
405000.000000002	405000.000000002\\
-259000.000000002	-259000.000000002\\
58000.0000000034	58000.0000000034\\
51999.9999999978	51999.9999999978\\
203000.000000004	203000.000000004\\
-219000.000000002	-219000.000000002\\
198999.999999999	198999.999999999\\
-454999.999999999	-454999.999999999\\
-57000.000000003	-57000.000000003\\
678000.000000001	678000.000000001\\
-128000.000000002	-128000.000000002\\
-694999.999999999	-694999.999999999\\
345999.999999999	345999.999999999\\
295000.000000004	295000.000000004\\
-276000.000000002	-276000.000000002\\
19000.000000001	19000.000000001\\
385000.000000002	385000.000000002\\
-258000.000000002	-258000.000000002\\
-237000	-237000\\
384000	384000\\
-163999.999999998	-163999.999999998\\
-238000	-238000\\
108999.999999999	108999.999999999\\
166000.000000001	166000.000000001\\
-57000.0000000022	-57000.0000000022\\
222000.000000002	222000.000000002\\
-313000	-313000\\
2000.00000000156	2000.00000000156\\
-57000.0000000057	-57000.0000000057\\
184000.000000002	184000.000000002\\
35999.9999999996	35999.9999999996\\
-401000.000000001	-401000.000000001\\
400000.000000001	400000.000000001\\
167999.999999999	167999.999999999\\
-278000	-278000\\
-197999.999999999	-197999.999999999\\
509999.999999999	509999.999999999\\
-219000.000000004	-219000.000000004\\
-291999.999999997	-291999.999999997\\
71999.9999999992	71999.9999999992\\
256999.999999999	256999.999999999\\
35999.9999999996	35999.9999999996\\
-34999.9999999984	-34999.9999999984\\
-404999.999999997	-404999.999999997\\
129999.999999998	129999.999999998\\
108000.000000001	108000.000000001\\
130000.000000001	130000.000000001\\
-37000.0000000017	-37000.0000000017\\
-56000.0000000009	-56000.0000000009\\
367000.000000001	367000.000000001\\
-310999.999999999	-310999.999999999\\
-550000.000000004	-550000.000000004\\
494000.000000001	494000.000000001\\
368000.000000004	368000.000000004\\
-313000	-313000\\
-310000	-310000\\
493000.000000003	493000.000000003\\
-109000.000000003	-109000.000000003\\
-72999.9999999959	-72999.9999999959\\
-165000.000000002	-165000.000000002\\
-19000.0000000037	-19000.0000000037\\
385000.000000002	385000.000000002\\
-238000	-238000\\
-311000	-311000\\
311000.000000003	311000.000000003\\
109999.999999999	109999.999999999\\
291999.999999999	291999.999999999\\
-308999.999999998	-308999.999999998\\
-772000.000000002	-772000.000000002\\
918000	918000\\
144999.999999998	144999.999999998\\
-529999.999999999	-529999.999999999\\
-146999.999999999	-146999.999999999\\
164000.000000001	164000.000000001\\
56999.9999999986	56999.9999999986\\
345999.999999999	345999.999999999\\
-128000.000000001	-128000.000000001\\
-54000.0000000003	-54000.0000000003\\
-91999.9999999987	-91999.9999999987\\
-220000.000000003	-220000.000000003\\
221000.000000003	221000.000000003\\
-56999.9999999986	-56999.9999999986\\
-145000.000000004	-145000.000000004\\
402000.000000002	402000.000000002\\
73999.9999999998	73999.9999999998\\
-403000.000000003	-403000.000000003\\
-17999.9999999971	-17999.9999999971\\
72000.0000000001	72000.0000000001\\
294000	294000\\
-384999.999999997	-384999.999999997\\
275000	275000\\
-36999.9999999981	-36999.9999999981\\
-421000	-421000\\
311999.999999997	311999.999999997\\
-1000.00000000033	-1000.00000000033\\
19000.0000000001	19000.0000000001\\
292000	292000\\
74000.0000000007	74000.0000000007\\
-402000	-402000\\
-167000.000000002	-167000.000000002\\
442000	442000\\
-331999.999999997	-331999.999999997\\
-255000.000000003	-255000.000000003\\
476000	476000\\
19000.0000000019	19000.0000000019\\
-203000.000000002	-203000.000000002\\
73999.9999999998	73999.9999999998\\
-1.77635683940025e-09	-1.77635683940025e-09\\
1.77635683940025e-09	1.77635683940025e-09\\
-55000.0000000015	-55000.0000000015\\
129000.000000002	129000.000000002\\
-240000.000000003	-240000.000000003\\
551000.000000002	551000.000000002\\
-183000	-183000\\
-441000.000000002	-441000.000000002\\
-89999.9999999998	-89999.9999999998\\
511999.999999999	511999.999999999\\
-128000.000000001	-128000.000000001\\
165000.000000002	165000.000000002\\
-476999.999999999	-476999.999999999\\
311999.999999998	311999.999999998\\
91999.9999999996	91999.9999999996\\
-384999.999999998	-384999.999999998\\
17999.9999999989	17999.9999999989\\
293000.000000002	293000.000000002\\
-128000.000000004	-128000.000000004\\
-17999.9999999998	-17999.9999999998\\
421000.000000001	421000.000000001\\
-2.66453525910038e-09	-2.66453525910038e-09\\
-824000	-824000\\
402000	402000\\
458999.999999998	458999.999999998\\
-622999.999999999	-622999.999999999\\
-55000.0000000006	-55000.0000000006\\
513000.000000001	513000.000000001\\
-385000.000000002	-385000.000000002\\
165000.000000003	165000.000000003\\
383999.999999999	383999.999999999\\
-382999.999999999	-382999.999999999\\
-75000.0000000028	-75000.0000000028\\
-71999.9999999983	-71999.9999999983\\
35999.9999999969	35999.9999999969\\
384000	384000\\
-126999.999999996	-126999.999999996\\
};
\end{axis}
\end{tikzpicture}%
%	\caption{Scatter plots of the input term $u(t)$ for training datasets. Correlation coefficient with the output signal is shown for each figure.}\label{fig:regr_u}
%\end{figure}
%\begin{figure}[!t]
%	\centering
%	\input{Regressions_first_sign_T_2000.tikz}
%	\caption{Scatter plots of the identified significant term $y(t-1)u(t)$ for training datasets. Correlation coefficient with the output signal is shown for each figure.}\label{fig:regr_first}
%\end{figure}
\end{document}