% !TeX TXS-program:compile = txs:///pdflatex/[--shell-escape]

\documentclass[a4paper,11pt,twoside]{article}
%%%%%%%%%%%%%%%%%%%%%%%%%%%%%%%%%%%%%%%%%%%%%%%%%%%%%%%%%%%%%%%%%%%%%%%%%%%
% A generic report compiler for publishing rables and plots for a partcular choise of lags in the NARX system
% A path to media corresponding to the setting is defined below, as well as chosen parameter settings
\makeatletter
\def\input@path{{../delta_results_cv_f_set_V_lambda_2/}}
\def\dataset{C}
\def\ny{4}
\def\nu{4}
\def\order{3}
\makeatother
%%%%%%%%%%%%%%%%%%%%%%%%%%%%%%%%%%%%%%%%%%%%%%%%%%%%%%%%%%%%%%%%%%%%%%%%%%%
% Packages
\usepackage[final]{pdfpages}
\usepackage{verbatim}
\usepackage{inputenc}
\usepackage{graphicx} 
\usepackage{amsmath,amssymb,mathrsfs,amsfonts}
\usepackage{mathtools}
\usepackage{amsthm}
\usepackage{mathtools}
\usepackage{calrsfs}
\usepackage{graphicx}
\usepackage{subfig}
\usepackage{eucal}    
\usepackage{amssymb}  
\usepackage{pifont}
\usepackage{color} 
\usepackage{cancel}
\usepackage[toc,page]{appendix}
\usepackage{pgfplots}
\pgfplotsset{compat=newest,every axis/.append style={line width=0.5pt},x label style={font={\small},at={(axis description cs:0.5,-0.15)},anchor=north},y label style={font={\small},at={(axis description cs:-0.15,0.5)},anchor=south},z label style={font={\small},at={(axis description cs:-0.25,0.5)},anchor=north},label style={font=\small},tick label style={font=\small},title style={font=\small}}
\usetikzlibrary{shapes,shadows,arrows,backgrounds,patterns,positioning,automata,calc,decorations.markings,decorations.pathreplacing,bayesnet,arrows.meta} 
\usepackage{varwidth}
\usepackage{lscape}
\usepackage{array} 
\usepackage[colorlinks=false,pdfborder={0 0 0}]{hyperref}
\usepackage{tabularx}
\usepackage{textcomp}
\usepackage{multicol} 
\usepackage{booktabs}
\usepackage{multirow}
\usepackage[font=small,labelfont=bf]{caption}                                                           
\usepackage{textcase}
\usepackage{bbm} 
\usepackage{fancyhdr}
\usepackage{enumitem}
\usepackage{soul}
\usepackage{wrapfig}
%%%%%%%%%%%%%%%%%%%%%%%%%%%%%%%%%%%%%%%%%%%%%%%%%%%%%%%%%%%%%%%%%%%%%%%%%%%
% Geometry
\setlength{\parindent}{2em}
\setlength{\parskip}{0.5em}
\renewcommand{\baselinestretch}{1.2}
\usepackage[left=2cm, right=2cm, top=2.5cm, bottom=3cm, headheight=13.6pt]{geometry}
\allowdisplaybreaks 
%%%%%%%%%%%%%%%%%%%%%%%%%%%%%%%%%%%%%%%%%%%%%%%%%%%%%%%%%%%%%%%%%%%%%%%%%%%
% Bibliography
\usepackage[backend=bibtex,style=ieee,sorting=none]{biblatex} 
\bibliography{bibliography_ridge}
\renewcommand*{\bibfont}{\scriptsize}
%%%%%%%%%%%%%%%%%%%%%%%%%%%%%%%%%%%%%%%%%%%%%%%%%%%%%%%%%%%%%%%%%%%%%%%%%%%
% Custom commands and operators
\makeatletter
\newcommand*{\rom}[1]{\expandafter\@slowromancap\romannumeral #1@}
\newcommand{\ie}{\textit{i.e.} }
\newcommand{\eg}{\textit{e.g.} }
\newcommand\id{\ensuremath{\mathbbm{1}}} 
\newcommand{\norm}[1]{\left\lVert#1\right\rVert}
\DeclareMathOperator{\E}{\mathbb{E}}
\DeclareMathOperator{\eye}{\mathbb{I}}
\DeclareMathOperator{\zeros}{\mathbb{O}}
\DeclareMathOperator{\tr}{\textrm{tr}}
\DeclareMathOperator{\vvec}{\textrm{vec}}
\DeclareMathOperator{\ik}{\mathrm{k}}
\DeclareMathOperator{\ip}{\mathrm{p}}
\DeclareMathOperator{\inn}{\mathrm{n}}
\DeclareMathOperator{\im}{\mathrm{m}}
\DeclareMathOperator{\td}{\mathrm{t}}
\DeclareMathOperator{\kd}{\mathrm{k}}
\DeclareMathOperator{\T}{\mathrm{T}}
\DeclareMathOperator{\K}{\mathrm{K}}
\DeclareMathOperator{\rk}{\mathrm{rk}}
\DeclareMathOperator{\vc}{\mathrm{vec}}
\DeclareSymbolFontAlphabet{\mathcal} {symbols}
\DeclareSymbolFont{symbols}{OMS}{cm}{m}{n}
\DeclareMathAlphabet{\mathbfit}{OML}{cmm}{b}{it}
\makeatother
% Number equations
%\numberwithin{equation}{section}

%%%%%%%%%%%%%%%%%%%%%%%%%%%%%%%%%%%%%%%%%%%%%%%%%%%%%%%%%%%%%%%%%%%%%%%%%%%
%Theorems
\newtheoremstyle{mytheoremstyle} % name
{.5em}                    % Space above
{.8em}                    % Space below
{\itshape}                % Body font
{1em}                           % Indent amount
{\bfseries}                   % Theorem head font
{:}                          % Punctuation after theorem head
{.5em}                       % Space after theorem head
{}  % Theorem head spec (can be left empty, meaning ‘normal’)

\theoremstyle{mytheoremstyle}
\newtheorem{theorem}{Theorem}[section]
\newtheorem{remark}{Remark}[section]
\newtheorem{assumption}{Assumption}[section]
\newtheorem{lemma}{Lemma}[section]
\newtheorem{condition}{Condition}[section]
\newtheorem{definition}{Definition}[section]
\newtheorem{property}{Property}[section]
\newtheorem{corollary}{Corollary}[section]
\renewcommand\qedsymbol{$\blacksquare$}
%%%%%%%%%%%%%%%%%%%%%%%%%%%%%%%%%%%%%%%%%%%%%%%%%%%%%%%%%%%%%%%%%%%%%%%%%%%
% Nomenclature
\usepackage[intoc]{nomencl}
\makenomenclature

\usepackage[ruled]{algorithm}
\usepackage{float}

\usepackage{algorithmic}
\algsetup{linenosize=\scriptsize}
\usepackage{etoolbox}
\AtBeginEnvironment{algorithmic}{\scriptsize}
\renewcommand{\thealgorithm}{\thechapter.\arabic{algorithm}} 
%\usepackage{chngcntr}
%\counterwithin{algorithm}{section}
% correct bad hyphenation here
\hyphenation{op-tical net-works semi-conduc-tor}
%%%%%%%%%%%%%%%%%%%%%%%%%%%%%%%%%%%%%%%%%%%%%%%%%%%%%%%%%%%%%%%%%%%%%%%%%%%%%
\usepackage[explicit]{titlesec}
\usepackage{titletoc}
\interfootnotelinepenalty=10000

\title{Constrained parameter estimation of $\delta$-domain models}

\begin{document}
	\maketitle
\par This report demonstrates the performance of the $\delta$-domain identification framework on a simulation example.
\section{Model structure identification}
\par Equivalence between the lag models and $\delta$-domain models was established in \cite{ANDERSON20071859}. For a input-output model with a defined lags
\begin{equation}
	\mathbfit{y}(t) = f\big( \{y(t - k)\}^{n_y}_{k=1}, \{u(t - k + n_y + 1)\}^{n_y + n_u}_{k= n_y + 1} \big),
\end{equation}
there can be established an equivalent representation with $\delta$-operator, where the output of the NARX model is the highest order derivative of the registered output, $\delta^{n} y(t)$, and input vector takes the following form:
\begin{equation}
\mathbfit{x}(t) = \left[\begin{array}{cccccccc}
\delta^{n-1} y(t) & \dots & \delta y(t) & y(t) & \delta^{n-1}u(t-1) & \dots  & \delta u(t) & u(t)
\end{array}\right]^{\top}.
\end{equation}
The unknown model is approximated with a sum of polynomial basis functions up to second degree ($\lambda = \order$), rendering the following structure
\begin{equation}\label{eq:narx}
	\mathbfit{y}(t) = \theta^0 + \sum_{i=1}^{d} \theta_i x_i(t) + \sum_{i=1}^{d} \sum_{j=i}^{d} \theta_{i,j} x_i(t) x_j(t) +  \sum_{i=1}^{d} \sum_{j=i}^{d} \sum_{k=j}^{d} \theta_{i,j} x_i(t) x_j(t) x_k(t) + e(t).
\end{equation}
\par The performance of $\delta$-domain identification framework is tested on a simulated data for Van-der-Pol oscillator (VDPO) with varying damping strength. The dynamics of VDPO is described by a non-linear second-order ODE:
\begin{equation}
\frac{\delta^2}{\delta t^2} y(t) = \mu(1 - y^2(t))\frac{\delta}{\delta t} y(t) - y(t) + u(t),
\end{equation} 
where $u(t)$ is an excitation signal (in this case, sum of sinusoids). The damping coefficient $\mu$ was assigned the following values for the MC simulations
\begin{equation*}
\mu = \left[\begin{array}{ccccccc}0.0625 & 0.125 & 0.25 & 0.3 & 0.5 & 0.8 & 1.
\end{array}\right]
\end{equation*}
\par The number and order of significant terms are identified within the EFOR-CMSS truncated using Bayesian information criteria 
\begin{figure}[!h]
	\definecolor{mycolor1}{rgb}{0.00000,0.44700,0.74100}%
	\definecolor{mycolor2}{rgb}{0.85000,0.32500,0.09800}%
	\centering
	% This file was created by matlab2tikz.
%
\definecolor{mycolor1}{rgb}{0.00000,0.44700,0.74100}%
\definecolor{mycolor2}{rgb}{0.85000,0.32500,0.09800}%
%
\begin{tikzpicture}

\begin{axis}[%
width=5.986cm,
height=6cm,
at={(0cm,0cm)},
scale only axis,
xmin=0,
xmax=25,
xlabel style={font=\color{white!15!black}},
xlabel={Terms},
ymin=-50000,
ymax=0,
ylabel style={font=\color{white!15!black}},
ylabel={BIC},
axis background/.style={fill=white}
]
\addplot [color=mycolor1, forget plot]
  table[row sep=crcr]{%
1	0\\
2	-46249.9293952455\\
3	-47313.1785545196\\
4	-47657.1819846709\\
5	-47687.2958577504\\
6	-47683.7289655208\\
7	-47678.2949149853\\
8	-47672.5571265516\\
9	-47665.9647782809\\
10	-47657.9136710176\\
11	-47649.5389669121\\
12	-47641.2313122236\\
13	-47632.7818833552\\
14	-47624.063866548\\
15	-47615.3848729482\\
16	-47607.2239116368\\
17	-47598.0647593016\\
18	-47589.39510097\\
19	-47580.5913948774\\
20	-47571.613448152\\
21	-47562.4542958168\\
22	-47553.4714882956\\
23	-47544.7153087765\\
};
\addplot [color=mycolor2, draw=none, mark=asterisk, mark options={solid, mycolor2}, forget plot]
  table[row sep=crcr]{%
5	-47687.2958577504\\
};
\end{axis}
\end{tikzpicture}%
	\resizebox{!}{0cm}{
		\begin{minipage}{\textwidth}
			\begin{tikzpicture}	
			\begin{axis}[width=2cm,height=2cm]
			\addplot [color=mycolor2,line width=5.0pt,only marks,mark=asterisk,mark options={solid},forget plot]
			table[row sep=crcr]{%
				8	-3.00104696019893\\
			};\label{tikz:nterms}
			\end{axis}
			\end{tikzpicture}%
	\end{minipage}}
	\caption{Evolution of BIC with growing number of parameters in the model}\label{fig:bic}
\end{figure}	
\par The significant terms identified by the algorithm are presented in Table \ref{tab:thetas_all}
\begin{table}[!h]
	\centering
	\caption{Significant terms and corresponding coefficients identified in EFOR-CMSS algorithm.}\label{tab:theetas_all}
	\small
	\begin{tabular}{rrrrrrrrrrrrr}
Step & Terms & V2 & V3 & V4 & V5 & V6 & V7 & V8 & V9 & V10 & AERR($\%$) & BIC \\ 
\hline 
1 & $y(t-1)$ & 2.01 & 2.02 & 2.03 & 2.04 & 2.05 & 2.06 & 2.07 & 2.08 & 2.09 & 99.018 & -65410.626 \\ 
2 & $y(t-2)$ & -1.02 & -1.03 & -1.04 & -1.05 & -1.06 & -1.07 & -1.08 & -1.09 & -1.1 & 0.974 & -89429.5694 \\ 
3 & $u(t-1)$ & 0.01 & 0.01 & 0.01 & 0.01 & 0.01 & 0.01 & 0.01 & 0.01 & 0.01 & 0.004 & -93125.8763 \\ 
4 & $y(t-2)y(t-2)y(t-2)$ & -0.01 & -0.01 & -0.02 & -0.02 & -0.03 & -0.04 & -0.04 & -0.05 & -0.05 & 0 & -93439.9396 \\ 
5 & $y(t-1)y(t-1)y(t-1)$ & -0.01 & -0.02 & -0.03 & -0.03 & -0.04 & -0.05 & -0.06 & -0.07 & -0.07 & 0.005 & -110422.6008 \\ 
6 & $y(t-2)y(t-2)y(t-1)$ & 0.02 & 0.03 & 0.04 & 0.05 & 0.07 & 0.09 & 0.1 & 0.12 & 0.12 & 0 & -112741.2725 \\ 
7 & $y(t-1)y(t-1)u(t-1)$ & 0 & 0 & 0 & 0 & 0 & 0 & 0 & 0 & 0 & 0 & -113216.4723 \\ 
8 & $u(t-2)$ & 0 & 0 & 0 & 0 & 0 & 0 & 0 & 0 & 0 & 0 & -113233.0844 \\ 
\hline 
\end{tabular}
\end{table} 
\section{Direct estimation of external model parameters}
\par In order to link the external and internal parameters, an arbitrary polynomial function is formed from either a single parameter vector or a pair of vectors. The number of unknown parameters is defined by the number of avaliable datasets. This section demonstrates the direct estimation procedure and the justification for selecting polynomial terms for curve fitting.
\par Model structure for $K$ datasets:
\begin{equation}\label{eq:batchtimeser}
\underbrace{\bar{\mathbf{Y}}}_{\T K\times 1} = \underbrace{\bar{\Phi}}_{\T K \times NK} \underbrace{\bar{\Theta}}_{NK \times 1},
\end{equation}
where the block matrices have the following structure:
\begin{equation}
\underbrace{\bar{\mathbf{Y}}}_{\T K\times 1} = \left[\begin{array}{c} 
\underbrace{\mathbfit{y}^1}_{\T\times 1} \\
\underbrace{\mathbfit{y}^2}_{\T\times 1} \\
\vdots \\
\underbrace{\mathbfit{y}^K}_{\T\times 1}
\end{array}\right]; \quad 
\underbrace{\bar{\Phi}}_{\T K \times NK} = \left[\begin{array}{cccc} 
\underbrace{\Phi^1}_{\T \times N} & \dots & \dots & \zeros \\
\zeros & \underbrace{\Phi^2}_{\T \times N} & \dots & \zeros \\
\vdots & \vdots & \vdots & \vdots  \\
\zeros & \dots & \dots & \underbrace{\Phi^K}_{\T \times N}
\end{array}\right]; \quad 
\underbrace{\bar{\Theta}}_{NK \times 1} = \left[\begin{array}{c} 
\underbrace{\theta^1}_{N \times 1} \\
\underbrace{\theta^2}_{N \times 1} \\
\vdots \\
\underbrace{\theta^K}_{N \times 1}
\end{array}\right].
\end{equation}
\par The relationship of the design parameters known from the experiments and the internal parameters of NARMAX model  is defined by the following linear function:
\begin{equation}
\underbrace{\Theta}_{N \times K} = \underbrace{B}_{N \times L} \underbrace{A}_{L \times K},
\end{equation}
where $A$ is the matrix where each row is a function of the vector of design parameters. The example structure is
\begin{equation}
A = \left[\begin{array}{cccccc}
\eye_{K \times 1} & L_{K \times 1} & D_{K \times 1} & L D_{K \times 1} &  L^{2}_{K \times 1} & D^{2}_{K \times 1} 
\end{array}\right]^{\top},
\end{equation} 
and where $B$ denotes the matrix of unknown coefficients of a hypersurface of order $L$ that maps a point in external parameter space, $\xi^k = (L_k, D_k)$, onto the point in the space of internal parameters, $\theta^k$.
In can be seen that $\bar{\Theta} = \text{vec}(\Theta)$, then
\begin{equation}
\underbrace{\bar{\Theta}}_{NK \times 1} = \text{vec}\left(\underbrace{B}_{N \times L} \underbrace{A}_{L \times K}\right).
\end{equation}
This vectorisation can be obtained using Kronecker product:
\begin{equation}
\text{vec}\left(\underbrace{B}_{N \times L} \underbrace{A}_{L \times K}\right) = (\underbrace{A^{\top}}_{K \times L} \otimes \underbrace{\eye}_{N \times N}) \underbrace{\text{vec}(B)}_{NL \times 1}.
\end{equation}
Denoting the result of Kronecker product as $\underbrace{\mathbf{Kr}}_{NK \times NL} \triangleq (\underbrace{A^{\top}}_{K \times L} \otimes \underbrace{\eye}_{N \times N})$ and vectorised coefficient matrix as $\underbrace{\bar{\mathbf{B}}}_{NL \times 1} \triangleq \text{vec}(B)$  yields the following:
\begin{equation}\label{eq:BtoTheta}
\underbrace{\bar{\Theta}}_{NK \times 1} = \underbrace{\mathbf{Kr}}_{KN \times NL} \underbrace{\bar{\mathbf{B}}}_{NL \times 1}.
\end{equation}
Substituting the above expression into \eqref{eq:batchtimeser} renders an expression that directly links the design parameters and the timeseries data
\begin{equation}\label{eq:BtoY}
\underbrace{\bar{\mathbf{Y}}}_{\T K\times 1} = \underbrace{\bar{\Phi}}_{\T K \times NK} \underbrace{\mathbf{Kr}}_{KN \times NL} \underbrace{\bar{\mathbf{B}}}_{NL \times 1},
\end{equation}
where $\bar{\mathbf{B}}$ is the unknown vector and all other factors are known form the experiments or defined prior to structure identification. 
\par The representation \eqref{eq:BtoY} allows estimating the coefficients in  $\bar{\mathbf{B}}$ directly from the timeseries data bypassing the intermediate estimation of the internal coefficients in NARX model.
\par The following condition must be satisfied:
\begin{equation}\label{eq:rankcond}
\rk(\bar{\Phi}\mathbf{Kr}) \geq NL.
\end{equation}
The rank of the linear system \eqref{eq:BtoY} satisfies the following:
\begin{equation}
\rk (\bar{\Phi}\mathbf{Kr}) \leq \min\left(\rk(\bar{\Phi}), \rk(\mathbf{Kr})\right),
\end{equation}
where the rank of Kronecker product can be found as
\begin{equation}
\rk(\mathbf{Kr}) = \rk(\underbrace{A^{\top}}_{K \times L})\rk(\underbrace{\eye}_{N \times N}),
\end{equation}
thus the matrix $A$ composed of by-element combinations of external parameter vector(s) must be of rank $K$.
\section{Constrained estimation}
\par A common treatment for models that suffer from bad generalisation is to introduce a constrained LS problem, where the constraint is normally posed on the unknown parameter vector:
\begin{equation}\label{eq:rls_const}
\hat{\mathbf{\beta}}^i = \arg \min \norm{ \biggl(\bar{\theta}^i - X \mathbf{\beta}^i \biggr) }^2, \quad f_R(\mathbf{\beta}^i) < \gamma.
\end{equation}
where $\lambda$ is a pre-specified parameter that defines the size of the constraint in the parameter space $\Xi$.
Lagrangian formulation of the constrained problem is called regularised Least Squares (RLS),
\begin{equation}
\mathbf{\beta}^i = \arg \min \Biggl \{\norm{ \biggl(\bar{\theta}^i - X \mathbf{\beta}^i \biggr) }^2 + \lambda f_R(\mathbf{\beta}^i)\Biggr\}
\end{equation}
where the regularisation coefficient $\lambda$ is directly linked to  $\gamma$ in \eqref{eq:rls_const}. Different types of the constraint function are be considered depending on the problem.
\subsection{Tikhonov regularisation}
Innovation regularisation  constrains the 2-norm of the parameter vector
\begin{equation}
f_{R}(\mathbf{\beta}^i) = \norm{\mathbf{\beta}^i}^{2}_{2}.
\end{equation}
This is the only RLS formulation that has a closed form solution that is usually obtained for the normalised data. 
\begin{equation}\label{eq:ridge}
\hat{\mathbf{\beta}}^{i}_{RLS} = (\mathbf{R}^{*}_{aa} + \lambda \eye_M)^{-1} (\mathbf{A}^{\star})^{\top}\bar{\theta}^i,  
\end{equation}
This solution is referred to as ridge regression, because increasing $\lambda$ shrinks the coefficients $\beta_j$. The shrinkage is shown is best interpreted via singular values decomposition (SVD) of the normalised data matrix. Denote the SVD of $\mathbf{A}^{\star}$ as
\begin{equation}
\underbrace{\mathbf{A}^{\star}}_{K \times M} = \underbrace{\mathbf{U}}_{K \times M}\underbrace{\mathbf{D}}_{M \times M}\underbrace{\mathbf{V}^{\top}}_{M \times M},
\end{equation}
where columns of $\mathbf{U}$ are principal components of $\mathbf{A}^{\star}$, diagonal elements of  $\mathbf{D}$ are the singular values, and where $\mathbf{V}$ is the rotation. All orthonormality assumption are the same as in the general case. The ridge regression \eqref{eq:ridge} then takes form
\begin{equation}
\hat{\mathbf{\beta}}^{i}_{RLS} = \left( \mathbf{V}\mathbf{D}\mathbf{U}^{\top}\mathbf{U}\mathbf{D}\mathbf{V}^{\top}  + \lambda \eye_M\right)^{-1} \left( \mathbf{U}\mathbf{D}\mathbf{V}^{\top}\right)^{\top}\bar{\theta}^i.
\end{equation}
Simple linear algebra yields the following
\begin{equation}
\hat{\mathbf{\beta}}^{i}_{RLS} =  \mathbf{V}\left(\mathbf{D}^2  + \lambda \eye_M\right)^{-1}\mathbf{V}^{\top} \mathbf{V}\mathbf{D}\mathbf{U}^{\top}\bar{\theta}^i.
\end{equation}
The finale expression is
\begin{equation}\label{eq:rls_svd}
\hat{\mathbf{\beta}}^{i}_{RLS} = \mathbf{V}\left(\mathbf{D}^2  + \lambda \eye_M\right)^{-1}\mathbf{D}\mathbf{U}^{\top}\bar{\theta}^i,
\end{equation}
which can be compared to the SVD of OLS regression:
\begin{equation}\label{eq:ols_svd}
\hat{\mathbf{\beta}}^{i}_{OLS} = \mathbf{V}\mathbf{D}^{-1}\mathbf{U}^{\top}\bar{\theta}^i \qquad \left( \approx (\mathbf{A}^{\star})^{-1}\bar{\theta}^i\right).
\end{equation}
It can be seen from \eqref{eq:rls_svd}-\eqref{eq:ols_svd} that in Tikhonov regularisation the inverse of the diagonal matrix is obtained as $\mathbf{D} / (\mathbf{D}^2  + \lambda \eye_M)$ where $\lambda$ is non-negative. Increasing regularisation coefficient thus leads to shrinkage of the singular values, and the estimates asymptotically approach zero. This shows that in Tikhonov regularisation $\lambda$ quantifies the trade-off between the bias and the variance in the estimates. Small regularisation coefficient leads to near-OLS solution that overfits the model to the training data, while large value leads to biased estimates and drives all coefficients to near-zero values.
\subsection{LASSO regularisation}
\par While in the ridge regression LSEs asymptotically approach zero, none of the parameters can be zeroed-out explicitly if the model structure is overly detailed. Another formulation of RLS, called Least absolute shrinkage and selection operator (LASSO) regression, performs variable selection and the regularisation simultaneously by imposing the $l_1$ penalty: 
\begin{equation}
f_{R}(\mathbf{\beta}^i) = \norm{\mathbf{\beta}^i}_{1}.
\end{equation}
The Lagrangian optimisation problem then takes for of basis pursuit de-noising that can be solved numerically using quadratic programming or convex techniques. This report uses the shooting algorithm proposed in \cite{Fu1998} because of its relative simplicity. The results of ridge estimation are used as the initial point in convex optimisation searching for the LASSO solution.
\subsection{Selection of the regularisation coefficient}
\par The important stage of solving RLS problem is selecting the regularisation parameter that will result into a interpretable but parsimonious model. The constraint $\gamma$ in \eqref{eq:rls_constr} is often selected arbitrary since the shape of the parameter space is unknown. As a result, finding $\lambda$ relies on iterative schemes most of which do not guarantee convergence [CITE MANY]. For the lack of universal approach for selecting the optimal value of $\lambda$, the choice of the method remains application-specific.
\par In this report, ridge estimation is applied to a fixed model structure, hence Aikaike’s information criterion (AIC) may be used
\begin{equation}
\text{AIC}_{\lambda} = 2 p - 2\log p(\bar{\theta}^i \mid \mathbf{\beta}^i, \lambda),
\end{equation}
where the first term quantifies model complexity and the second term is the log-likelihood of the selected model fitting the data. The model complexity is determined as simply the trace of the hat matrix of the ridge estimator
\begin{equation}
p = \tr(\mathbf{H}_{RLS}) = \tr((\mathbf{A}^{\top})(\mathbf{R}_{aa} + \lambda \eye_M)^{-1} \mathbf{A}^{\top})
\end{equation}.
Bayesian information criterion (BIC), assigns a larger penalty to the model complexity 
\begin{equation}
\text{BIC}_{\lambda} = 2\log(n) p - 2\log p(\bar{\theta}^i \mid \mathbf{\beta}^i, \lambda),
\end{equation}
where $n$ is the number of data points used for parameter estimation. Both AIC and BIC are aim to estimate 

\par When it comes to finding a parsimonious model, it may be more reasonable to access model's prediction performance instead of explicitly penalising its complexity. Cross-validation procedure  
This work 
\par Selcting regularisation coefficient for LASSO regression is more nuanced as different model structure may arise for different values of $\lambda$. The most popular approach described in the literature uses cross-validation. Both the data matrix and the response vector are partitioned into pairs. Then each pair is exuded from the 
\section{Results}
The estimated coefficients are presented in Table \ref{tab:betas_all}, and the surface fitting results for each internal parameter are illustrated in Figure \ref{fig:surfaces_all}

\begin{table}[!h]
	\centering
	\caption{Polynomial coefficients estimated via ordinary LS.}\label{tab:betas_all}
	\small
	\begin{tabular}{rrrrrrrrrrr}
Step & Terms & $\beta_{0}$ & $\beta_{1}$ & $\beta_{2}$ & $\beta_{3}$ & $\beta_{4}$ & $\beta_{5}$ & $\beta_{6}$ & $\beta_{7}$ & $\beta_{8}$ \\ 
\hline 
1 & $\delta^1 y(t)$ & 1175.87 & -10095.27 & 20091 & -50.49 & 395.03 & -686.33 & 0.45 & -3.58 & 5.77 \\ 
2 & $y(t)y(t)y(t)$ & -128886.14 & -20167.02 & -4694.76 & -7041 & 71024.93 & -31152.47 & 119.95 & -1018.17 & 509.13 \\ 
3 & $c$ & -962.4 & 1876.76 & -13254.42 & -84.69 & 1864.26 & -7013.18 & 1.31 & -26.02 & 100.2 \\ 
4 & $y(t)$ & 5843.59 & -85876.07 & -20912.33 & -874.68 & 10156.69 & -19925.03 & 10.76 & -128.35 & 295.04 \\ 
5 & $y(t)\delta^1 u(t)$ & 34257.29 & 4105.26 & -230.18 & -3735.84 & 21112.09 & -21037.45 & 41.81 & -264.05 & 240.53 \\ 
6 & $y(t)\delta^1 u(t)$ & 34257.27 & 4104.98 & -230.12 & -3735.84 & 21112.09 & -21037.45 & 41.81 & -264.05 & 240.53 \\ 
\hline 
\end{tabular}
\end{table} 
\begin{table}[!h]
	\centering
	\caption{Polynomial coefficients estimated via Tikhonov regularisation.}\label{tab:betas_tikh}
	\small
	\begin{tabular}{rrrrrrrrrrr}
Step & Terms & $\beta_{0}$ & $\beta_{1}$ & $\beta_{2}$ & $\beta_{3}$ & $\beta_{4}$ & $\beta_{5}$ & $\beta_{6}$ & $\beta_{7}$ & $\beta_{8}$ \\ 
\hline 
1 & $y(t-1)$ & 8.85 & -37.62 & -8.56 & -0.3 & 1.61 & -0.46 & 0 & -0.02 & 0.01 \\ 
2 & $c$ & -0.05 & 1.55 & -7.84 & -0.01 & 0.08 & -0.25 & 0 & 0 & 0 \\ 
3 & $y(t-2)y(t-2)$ & 17.08 & 7.97 & 1.21 & -0.68 & 2.88 & -13.8 & 0.01 & -0.04 & 0.19 \\ 
\hline 
\end{tabular}
\end{table}

\begin{table}[!h]
	\centering
	\caption{Polynomial coefficients estimated via LASSO regularisation.}\label{tab:betas_lass}
	\small
	\begin{tabular}{rrrrrrrrrrr}
Step & Terms & $\beta_{0}$ & $\beta_{1}$ & $\beta_{2}$ & $\beta_{3}$ & $\beta_{4}$ & $\beta_{5}$ & $\beta_{6}$ & $\beta_{7}$ & $\beta_{8}$ \\ 
\hline 
1 & $y(t-1)$ & 8.85 & -37.62 & -8.56 & -0.3 & 1.61 & -0.46 & 0 & -0.02 & 0.01 \\ 
2 & $c$ & -0.05 & 1.55 & -7.84 & -0.01 & 0.08 & -0.25 & 0 & 0 & 0 \\ 
3 & $y(t-2)y(t-2)$ & 17.08 & 7.97 & 1.21 & -0.68 & 2.88 & -13.8 & 0.01 & -0.04 & 0.19 \\ 
\hline 
\end{tabular}
\end{table}
%The figure also shows values of the internal parameters computed for the external settings of experiments \dataset3 and \dataset8. The obtained internal parameters are substituted in the modified version of model \eqref{eq:narx} that only includes the identified significant  polynomial terms to validate the identified model structure. The simulation results are compared with true system outputs for \dataset3 and \dataset8 in Figure \ref{fig:c3all} and Figure \ref{fig:c8all}, respectively. Computed RMSEs for both lengths of the sample are presented in Table \ref{tab:RMSEs}.
%\begin{table}[!h]
%	\centering
%	\caption{RMSE of the system output generated by the identified model.}\label{tab:RMSEs}
%		\begin{tabular}{cc}
%			Sample size 2000 & Sample size 4000 \\
%			\hline	
%			\begin{minipage}{2in} \verbatiminput{RMSEs_T_2000.txt}\end{minipage}& 
%			\begin{minipage}{2in} \vspace{0.5cm}\verbatiminput{RMSEs_T_4000.txt}\end{minipage}\\
%			\hline
%		\end{tabular}
%\end{table}

%\begin{figure}[!t]
%	\centering
%	\resizebox{!}{0cm}{
%		\begin{minipage}{\textwidth}
%		% This file was created by matlab2tikz.
%
\definecolor{mycolor1}{rgb}{0.00000,0.44700,0.74100}%
\definecolor{mycolor2}{rgb}{0.85000,0.32500,0.09800}%
%
\begin{tikzpicture}

\begin{axis}[%
width=3.159cm,
height=3.097cm,
at={(0cm,12.903cm)},
scale only axis,
xmin=56,
xmax=74,
tick align=outside,
axis background/.style={fill=white},
xmajorgrids,
ymajorgrids,
zmajorgrids
]
\addplot3[only marks, mark=*, mark options={}, mark size=1.5000pt, color=mycolor1, fill=mycolor1] table[row sep=crcr]{%
x	y	z\\
74	0.123	-26.0353957891804\\
72	0.113	-20.9879322169279\\
61	0.095	-10.6920630070547\\
56	0.093	-10.9569379219045\\
};\label{tikz:thetas1}
\addplot3[only marks, mark=*, mark options={}, mark size=1.5000pt, color=mycolor2, fill=mycolor2] table[row sep=crcr]{%
x	y	z\\
67	0.276	-191.779551108501\\
66	0.255	-157.643535964989\\
62	0.209	-87.4196213968237\\
57	0.193	-69.8013569503948\\
};\label{tikz:thetas2}
\addplot3[only marks, mark=*, mark options={}, mark size=1.5000pt, color=black, fill=black] table[row sep=crcr]{%
x	y	z\\
69	0.104	-15.57705861622\\
};\label{tikz:thetaidentified}
\addplot3[only marks, mark=*, mark options={}, mark size=1.5000pt, color=black, fill=black] table[row sep=crcr]{%
x	y	z\\
64	0.23	-116.694087150299\\
};
\addplot3[%
surf,
fill opacity=0.7, shader=interp, colormap={mymap}{[1pt] rgb(0pt)=(1,0.905882,0); rgb(1pt)=(1,0.901964,0); rgb(2pt)=(1,0.898051,0); rgb(3pt)=(1,0.894144,0); rgb(4pt)=(1,0.890243,0); rgb(5pt)=(1,0.886349,0); rgb(6pt)=(1,0.88246,0); rgb(7pt)=(1,0.878577,0); rgb(8pt)=(1,0.8747,0); rgb(9pt)=(1,0.870829,0); rgb(10pt)=(1,0.866964,0); rgb(11pt)=(1,0.863106,0); rgb(12pt)=(1,0.859253,0); rgb(13pt)=(1,0.855406,0); rgb(14pt)=(1,0.851566,0); rgb(15pt)=(1,0.847732,0); rgb(16pt)=(1,0.843903,0); rgb(17pt)=(1,0.840081,0); rgb(18pt)=(1,0.836265,0); rgb(19pt)=(1,0.832455,0); rgb(20pt)=(1,0.828652,0); rgb(21pt)=(1,0.824854,0); rgb(22pt)=(1,0.821063,0); rgb(23pt)=(1,0.817278,0); rgb(24pt)=(1,0.8135,0); rgb(25pt)=(1,0.809727,0); rgb(26pt)=(1,0.805961,0); rgb(27pt)=(1,0.8022,0); rgb(28pt)=(1,0.798445,0); rgb(29pt)=(1,0.794696,0); rgb(30pt)=(1,0.790953,0); rgb(31pt)=(1,0.787215,0); rgb(32pt)=(1,0.783484,0); rgb(33pt)=(1,0.779758,0); rgb(34pt)=(1,0.776038,0); rgb(35pt)=(1,0.772324,0); rgb(36pt)=(1,0.768615,0); rgb(37pt)=(1,0.764913,0); rgb(38pt)=(1,0.761217,0); rgb(39pt)=(1,0.757527,0); rgb(40pt)=(1,0.753843,0); rgb(41pt)=(1,0.750165,0); rgb(42pt)=(1,0.746493,0); rgb(43pt)=(1,0.742827,0); rgb(44pt)=(1,0.739167,0); rgb(45pt)=(1,0.735514,0); rgb(46pt)=(1,0.731867,0); rgb(47pt)=(1,0.728226,0); rgb(48pt)=(1,0.724591,0); rgb(49pt)=(1,0.720963,0); rgb(50pt)=(1,0.717341,0); rgb(51pt)=(1,0.713725,0); rgb(52pt)=(0.999994,0.710077,0); rgb(53pt)=(0.999974,0.706363,0); rgb(54pt)=(0.999942,0.702592,0); rgb(55pt)=(0.999898,0.698775,0); rgb(56pt)=(0.999841,0.694921,0); rgb(57pt)=(0.999771,0.691039,0); rgb(58pt)=(0.99969,0.687139,0); rgb(59pt)=(0.999596,0.68323,0); rgb(60pt)=(0.99949,0.679323,0); rgb(61pt)=(0.999372,0.675427,0); rgb(62pt)=(0.999242,0.67155,0); rgb(63pt)=(0.9991,0.667704,0); rgb(64pt)=(0.998946,0.663897,0); rgb(65pt)=(0.998781,0.660138,0); rgb(66pt)=(0.998605,0.656439,0); rgb(67pt)=(0.998416,0.652807,0); rgb(68pt)=(0.998217,0.649253,0); rgb(69pt)=(0.998006,0.645786,0); rgb(70pt)=(0.997785,0.642416,0); rgb(71pt)=(0.997552,0.639152,0); rgb(72pt)=(0.997308,0.636004,0); rgb(73pt)=(0.997053,0.632982,0); rgb(74pt)=(0.996788,0.630095,0); rgb(75pt)=(0.996512,0.627352,0); rgb(76pt)=(0.996226,0.624763,0); rgb(77pt)=(0.995851,0.622329,0); rgb(78pt)=(0.99494,0.619997,0); rgb(79pt)=(0.99345,0.617753,0); rgb(80pt)=(0.991419,0.61559,0); rgb(81pt)=(0.988885,0.613503,0); rgb(82pt)=(0.985886,0.611486,0); rgb(83pt)=(0.98246,0.609532,0); rgb(84pt)=(0.978643,0.607636,0); rgb(85pt)=(0.974475,0.605791,0); rgb(86pt)=(0.969992,0.603992,0); rgb(87pt)=(0.965232,0.602233,0); rgb(88pt)=(0.960233,0.600507,0); rgb(89pt)=(0.955033,0.598808,0); rgb(90pt)=(0.949669,0.59713,0); rgb(91pt)=(0.94418,0.595468,0); rgb(92pt)=(0.938602,0.593815,0); rgb(93pt)=(0.932974,0.592166,0); rgb(94pt)=(0.927333,0.590513,0); rgb(95pt)=(0.921717,0.588852,0); rgb(96pt)=(0.916164,0.587176,0); rgb(97pt)=(0.910711,0.585479,0); rgb(98pt)=(0.905397,0.583755,0); rgb(99pt)=(0.900258,0.581999,0); rgb(100pt)=(0.895333,0.580203,0); rgb(101pt)=(0.890659,0.578362,0); rgb(102pt)=(0.886275,0.576471,0); rgb(103pt)=(0.882047,0.574545,0); rgb(104pt)=(0.877819,0.572608,0); rgb(105pt)=(0.873592,0.57066,0); rgb(106pt)=(0.869366,0.568701,0); rgb(107pt)=(0.865143,0.566733,0); rgb(108pt)=(0.860924,0.564756,0); rgb(109pt)=(0.856708,0.562771,0); rgb(110pt)=(0.852497,0.560778,0); rgb(111pt)=(0.848292,0.558779,0); rgb(112pt)=(0.844092,0.556774,0); rgb(113pt)=(0.8399,0.554763,0); rgb(114pt)=(0.835716,0.552749,0); rgb(115pt)=(0.831541,0.55073,0); rgb(116pt)=(0.827374,0.548709,0); rgb(117pt)=(0.823219,0.546686,0); rgb(118pt)=(0.819074,0.54466,0); rgb(119pt)=(0.81494,0.542635,0); rgb(120pt)=(0.81082,0.540609,0); rgb(121pt)=(0.806712,0.538584,0); rgb(122pt)=(0.802619,0.53656,0); rgb(123pt)=(0.798541,0.534539,0); rgb(124pt)=(0.794478,0.532521,0); rgb(125pt)=(0.790431,0.530506,0); rgb(126pt)=(0.786402,0.528496,0); rgb(127pt)=(0.782391,0.526491,0); rgb(128pt)=(0.77841,0.524489,0); rgb(129pt)=(0.774523,0.522478,0); rgb(130pt)=(0.770731,0.520455,0); rgb(131pt)=(0.767022,0.518424,0); rgb(132pt)=(0.763384,0.516385,0); rgb(133pt)=(0.759804,0.514339,0); rgb(134pt)=(0.756272,0.51229,0); rgb(135pt)=(0.752775,0.510237,0); rgb(136pt)=(0.749302,0.508182,0); rgb(137pt)=(0.74584,0.506128,0); rgb(138pt)=(0.742378,0.504075,0); rgb(139pt)=(0.738904,0.502025,0); rgb(140pt)=(0.735406,0.499979,0); rgb(141pt)=(0.731872,0.49794,0); rgb(142pt)=(0.72829,0.495909,0); rgb(143pt)=(0.724649,0.493887,0); rgb(144pt)=(0.720936,0.491875,0); rgb(145pt)=(0.71714,0.489876,0); rgb(146pt)=(0.713249,0.487891,0); rgb(147pt)=(0.709251,0.485921,0); rgb(148pt)=(0.705134,0.483968,0); rgb(149pt)=(0.700887,0.482033,0); rgb(150pt)=(0.696497,0.480118,0); rgb(151pt)=(0.691952,0.478225,0); rgb(152pt)=(0.687242,0.476355,0); rgb(153pt)=(0.682353,0.47451,0); rgb(154pt)=(0.677195,0.472696,0); rgb(155pt)=(0.6717,0.470916,0); rgb(156pt)=(0.665891,0.469169,0); rgb(157pt)=(0.659791,0.46745,0); rgb(158pt)=(0.653423,0.465756,0); rgb(159pt)=(0.64681,0.464084,0); rgb(160pt)=(0.639976,0.462432,0); rgb(161pt)=(0.632943,0.460795,0); rgb(162pt)=(0.625734,0.459171,0); rgb(163pt)=(0.618373,0.457556,0); rgb(164pt)=(0.610882,0.455948,0); rgb(165pt)=(0.603284,0.454343,0); rgb(166pt)=(0.595604,0.452737,0); rgb(167pt)=(0.587863,0.451129,0); rgb(168pt)=(0.580084,0.449514,0); rgb(169pt)=(0.572292,0.447889,0); rgb(170pt)=(0.564508,0.446252,0); rgb(171pt)=(0.556756,0.444599,0); rgb(172pt)=(0.549059,0.442927,0); rgb(173pt)=(0.54144,0.441232,0); rgb(174pt)=(0.533922,0.439512,0); rgb(175pt)=(0.526529,0.437764,0); rgb(176pt)=(0.519282,0.435983,0); rgb(177pt)=(0.512206,0.434168,0); rgb(178pt)=(0.505323,0.432315,0); rgb(179pt)=(0.498628,0.430422,3.92506e-06); rgb(180pt)=(0.491973,0.428504,3.49981e-05); rgb(181pt)=(0.485331,0.426562,9.63073e-05); rgb(182pt)=(0.478704,0.424596,0.000186979); rgb(183pt)=(0.472096,0.422609,0.000306141); rgb(184pt)=(0.465508,0.420599,0.00045292); rgb(185pt)=(0.458942,0.418567,0.000626441); rgb(186pt)=(0.452401,0.416515,0.000825833); rgb(187pt)=(0.445885,0.414441,0.00105022); rgb(188pt)=(0.439399,0.412348,0.00129873); rgb(189pt)=(0.432942,0.410234,0.00157049); rgb(190pt)=(0.426518,0.408102,0.00186463); rgb(191pt)=(0.420129,0.40595,0.00218028); rgb(192pt)=(0.413777,0.40378,0.00251655); rgb(193pt)=(0.407464,0.401592,0.00287258); rgb(194pt)=(0.401191,0.399386,0.00324749); rgb(195pt)=(0.394962,0.397164,0.00364042); rgb(196pt)=(0.388777,0.394925,0.00405048); rgb(197pt)=(0.38264,0.39267,0.00447681); rgb(198pt)=(0.376552,0.390399,0.00491852); rgb(199pt)=(0.370516,0.388113,0.00537476); rgb(200pt)=(0.364532,0.385812,0.00584464); rgb(201pt)=(0.358605,0.383497,0.00632729); rgb(202pt)=(0.352735,0.381168,0.00682184); rgb(203pt)=(0.346925,0.378826,0.00732741); rgb(204pt)=(0.341176,0.376471,0.00784314); rgb(205pt)=(0.335485,0.374093,0.00847245); rgb(206pt)=(0.329843,0.371682,0.00930909); rgb(207pt)=(0.324249,0.369242,0.0103377); rgb(208pt)=(0.318701,0.366772,0.0115428); rgb(209pt)=(0.313198,0.364275,0.0129091); rgb(210pt)=(0.307739,0.361753,0.0144211); rgb(211pt)=(0.302322,0.359206,0.0160634); rgb(212pt)=(0.296945,0.356637,0.0178207); rgb(213pt)=(0.291607,0.354048,0.0196776); rgb(214pt)=(0.286307,0.35144,0.0216186); rgb(215pt)=(0.281043,0.348814,0.0236284); rgb(216pt)=(0.275813,0.346172,0.0256916); rgb(217pt)=(0.270616,0.343517,0.0277927); rgb(218pt)=(0.265451,0.340849,0.0299163); rgb(219pt)=(0.260317,0.33817,0.0320472); rgb(220pt)=(0.25521,0.335482,0.0341698); rgb(221pt)=(0.250131,0.332786,0.0362688); rgb(222pt)=(0.245078,0.330085,0.0383287); rgb(223pt)=(0.240048,0.327379,0.0403343); rgb(224pt)=(0.235042,0.324671,0.04227); rgb(225pt)=(0.230056,0.321962,0.0441205); rgb(226pt)=(0.22509,0.319254,0.0458704); rgb(227pt)=(0.220142,0.316548,0.0475043); rgb(228pt)=(0.215212,0.313846,0.0490067); rgb(229pt)=(0.210296,0.311149,0.0503624); rgb(230pt)=(0.205395,0.308459,0.0515759); rgb(231pt)=(0.200514,0.305763,0.052757); rgb(232pt)=(0.195655,0.303061,0.0539242); rgb(233pt)=(0.190817,0.300353,0.0550763); rgb(234pt)=(0.186001,0.297639,0.0562123); rgb(235pt)=(0.181207,0.294918,0.0573313); rgb(236pt)=(0.176434,0.292191,0.0584321); rgb(237pt)=(0.171685,0.289458,0.0595136); rgb(238pt)=(0.166957,0.286719,0.060575); rgb(239pt)=(0.162252,0.283973,0.0616151); rgb(240pt)=(0.15757,0.281221,0.0626328); rgb(241pt)=(0.152911,0.278463,0.0636271); rgb(242pt)=(0.148275,0.275699,0.0645971); rgb(243pt)=(0.143663,0.272929,0.0655416); rgb(244pt)=(0.139074,0.270152,0.0664596); rgb(245pt)=(0.134508,0.26737,0.06735); rgb(246pt)=(0.129967,0.264581,0.0682118); rgb(247pt)=(0.125449,0.261787,0.0690441); rgb(248pt)=(0.120956,0.258986,0.0698456); rgb(249pt)=(0.116487,0.25618,0.0706154); rgb(250pt)=(0.112043,0.253367,0.0713525); rgb(251pt)=(0.107623,0.250549,0.0720557); rgb(252pt)=(0.103229,0.247724,0.0727241); rgb(253pt)=(0.0988592,0.244894,0.0733566); rgb(254pt)=(0.0945149,0.242058,0.0739522); rgb(255pt)=(0.0901961,0.239216,0.0745098)}, mesh/rows=49]
table[row sep=crcr, point meta=\thisrow{c}] {%
%
x	y	z	c\\
56	0.093	-10.9049623108072	-10.9049623108072\\
56	0.09666	-11.4423373867092	-11.4423373867092\\
56	0.10032	-12.101944154261	-12.101944154261\\
56	0.10398	-12.8837826134625	-12.8837826134625\\
56	0.10764	-13.7878527643139	-13.7878527643139\\
56	0.1113	-14.814154606815	-14.814154606815\\
56	0.11496	-15.9626881409659	-15.9626881409659\\
56	0.11862	-17.2334533667667	-17.2334533667667\\
56	0.12228	-18.6264502842172	-18.6264502842172\\
56	0.12594	-20.1416788933175	-20.1416788933175\\
56	0.1296	-21.7791391940677	-21.7791391940677\\
56	0.13326	-23.5388311864676	-23.5388311864676\\
56	0.13692	-25.4207548705173	-25.4207548705173\\
56	0.14058	-27.4249102462168	-27.4249102462168\\
56	0.14424	-29.5512973135661	-29.5512973135661\\
56	0.1479	-31.7999160725652	-31.7999160725652\\
56	0.15156	-34.1707665232141	-34.1707665232141\\
56	0.15522	-36.6638486655127	-36.6638486655127\\
56	0.15888	-39.2791624994612	-39.2791624994612\\
56	0.16254	-42.0167080250595	-42.0167080250595\\
56	0.1662	-44.8764852423076	-44.8764852423076\\
56	0.16986	-47.8584941512054	-47.8584941512054\\
56	0.17352	-50.9627347517531	-50.9627347517531\\
56	0.17718	-54.1892070439505	-54.1892070439505\\
56	0.18084	-57.5379110277977	-57.5379110277977\\
56	0.1845	-61.0088467032948	-61.0088467032948\\
56	0.18816	-64.6020140704416	-64.6020140704416\\
56	0.19182	-68.3174131292382	-68.3174131292382\\
56	0.19548	-72.1550438796846	-72.1550438796846\\
56	0.19914	-76.1149063217809	-76.1149063217809\\
56	0.2028	-80.1970004555269	-80.1970004555269\\
56	0.20646	-84.4013262809227	-84.4013262809227\\
56	0.21012	-88.7278837979683	-88.7278837979683\\
56	0.21378	-93.1766730066637	-93.1766730066637\\
56	0.21744	-97.7476939070088	-97.7476939070088\\
56	0.2211	-102.440946499004	-102.440946499004\\
56	0.22476	-107.256430782649	-107.256430782649\\
56	0.22842	-112.194146757943	-112.194146757943\\
56	0.23208	-117.254094424887	-117.254094424887\\
56	0.23574	-122.436273783482	-122.436273783482\\
56	0.2394	-127.740684833726	-127.740684833726\\
56	0.24306	-133.167327575619	-133.167327575619\\
56	0.24672	-138.716202009163	-138.716202009163\\
56	0.25038	-144.387308134356	-144.387308134356\\
56	0.25404	-150.180645951199	-150.180645951199\\
56	0.2577	-156.096215459692	-156.096215459692\\
56	0.26136	-162.134016659835	-162.134016659835\\
56	0.26502	-168.294049551628	-168.294049551628\\
56	0.26868	-174.57631413507	-174.57631413507\\
56	0.27234	-180.980810410162	-180.980810410162\\
56	0.276	-187.507538376904	-187.507538376904\\
56.375	0.093	-10.8126294158988	-10.8126294158988\\
56.375	0.09666	-11.3527833115828	-11.3527833115828\\
56.375	0.10032	-12.0151688989166	-12.0151688989166\\
56.375	0.10398	-12.7997861779002	-12.7997861779002\\
56.375	0.10764	-13.7066351485335	-13.7066351485335\\
56.375	0.1113	-14.7357158108167	-14.7357158108167\\
56.375	0.11496	-15.8870281647496	-15.8870281647496\\
56.375	0.11862	-17.1605722103324	-17.1605722103324\\
56.375	0.12228	-18.5563479475649	-18.5563479475649\\
56.375	0.12594	-20.0743553764473	-20.0743553764473\\
56.375	0.1296	-21.7145944969793	-21.7145944969793\\
56.375	0.13326	-23.4770653091613	-23.4770653091613\\
56.375	0.13692	-25.361767812993	-25.361767812993\\
56.375	0.14058	-27.3687020084745	-27.3687020084745\\
56.375	0.14424	-29.4978678956059	-29.4978678956059\\
56.375	0.1479	-31.7492654743869	-31.7492654743869\\
56.375	0.15156	-34.1228947448178	-34.1228947448178\\
56.375	0.15522	-36.6187557068985	-36.6187557068985\\
56.375	0.15888	-39.236848360629	-39.236848360629\\
56.375	0.16254	-41.9771727060093	-41.9771727060093\\
56.375	0.1662	-44.8397287430394	-44.8397287430394\\
56.375	0.16986	-47.8245164717192	-47.8245164717192\\
56.375	0.17352	-50.9315358920489	-50.9315358920489\\
56.375	0.17718	-54.1607870040283	-54.1607870040283\\
56.375	0.18084	-57.5122698076576	-57.5122698076576\\
56.375	0.1845	-60.9859843029366	-60.9859843029366\\
56.375	0.18816	-64.5819304898655	-64.5819304898655\\
56.375	0.19182	-68.3001083684441	-68.3001083684441\\
56.375	0.19548	-72.1405179386725	-72.1405179386725\\
56.375	0.19914	-76.1031592005507	-76.1031592005507\\
56.375	0.2028	-80.1880321540788	-80.1880321540788\\
56.375	0.20646	-84.3951367992566	-84.3951367992566\\
56.375	0.21012	-88.7244731360842	-88.7244731360842\\
56.375	0.21378	-93.1760411645616	-93.1760411645616\\
56.375	0.21744	-97.7498408846888	-97.7498408846888\\
56.375	0.2211	-102.445872296466	-102.445872296466\\
56.375	0.22476	-107.264135399893	-107.264135399893\\
56.375	0.22842	-112.204630194969	-112.204630194969\\
56.375	0.23208	-117.267356681695	-117.267356681695\\
56.375	0.23574	-122.452314860072	-122.452314860072\\
56.375	0.2394	-127.759504730098	-127.759504730098\\
56.375	0.24306	-133.188926291773	-133.188926291773\\
56.375	0.24672	-138.740579545099	-138.740579545099\\
56.375	0.25038	-144.414464490074	-144.414464490074\\
56.375	0.25404	-150.210581126699	-150.210581126699\\
56.375	0.2577	-156.128929454974	-156.128929454974\\
56.375	0.26136	-162.169509474899	-162.169509474899\\
56.375	0.26502	-168.332321186474	-168.332321186474\\
56.375	0.26868	-174.617364589698	-174.617364589698\\
56.375	0.27234	-181.024639684572	-181.024639684572\\
56.375	0.276	-187.554146471096	-187.554146471096\\
56.75	0.093	-10.7298928959079	-10.7298928959079\\
56.75	0.09666	-11.2728256113739	-11.2728256113739\\
56.75	0.10032	-11.9379900184897	-11.9379900184897\\
56.75	0.10398	-12.7253861172552	-12.7253861172552\\
56.75	0.10764	-13.6350139076706	-13.6350139076706\\
56.75	0.1113	-14.6668733897358	-14.6668733897358\\
56.75	0.11496	-15.8209645634508	-15.8209645634508\\
56.75	0.11862	-17.0972874288155	-17.0972874288155\\
56.75	0.12228	-18.49584198583	-18.49584198583\\
56.75	0.12594	-20.0166282344944	-20.0166282344944\\
56.75	0.1296	-21.6596461748085	-21.6596461748085\\
56.75	0.13326	-23.4248958067724	-23.4248958067724\\
56.75	0.13692	-25.3123771303862	-25.3123771303862\\
56.75	0.14058	-27.3220901456497	-27.3220901456497\\
56.75	0.14424	-29.454034852563	-29.454034852563\\
56.75	0.1479	-31.7082112511262	-31.7082112511262\\
56.75	0.15156	-34.084619341339	-34.084619341339\\
56.75	0.15522	-36.5832591232017	-36.5832591232017\\
56.75	0.15888	-39.2041305967142	-39.2041305967142\\
56.75	0.16254	-41.9472337618765	-41.9472337618765\\
56.75	0.1662	-44.8125686186886	-44.8125686186886\\
56.75	0.16986	-47.8001351671505	-47.8001351671505\\
56.75	0.17352	-50.9099334072622	-50.9099334072622\\
56.75	0.17718	-54.1419633390236	-54.1419633390236\\
56.75	0.18084	-57.4962249624349	-57.4962249624349\\
56.75	0.1845	-60.9727182774959	-60.9727182774959\\
56.75	0.18816	-64.5714432842068	-64.5714432842068\\
56.75	0.19182	-68.2923999825674	-68.2923999825674\\
56.75	0.19548	-72.1355883725778	-72.1355883725778\\
56.75	0.19914	-76.1010084542381	-76.1010084542381\\
56.75	0.2028	-80.1886602275481	-80.1886602275481\\
56.75	0.20646	-84.3985436925079	-84.3985436925079\\
56.75	0.21012	-88.7306588491175	-88.7306588491175\\
56.75	0.21378	-93.185005697377	-93.185005697377\\
56.75	0.21744	-97.7615842372861	-97.7615842372861\\
56.75	0.2211	-102.460394468845	-102.460394468845\\
56.75	0.22476	-107.281436392054	-107.281436392054\\
56.75	0.22842	-112.224710006913	-112.224710006913\\
56.75	0.23208	-117.290215313421	-117.290215313421\\
56.75	0.23574	-122.477952311579	-122.477952311579\\
56.75	0.2394	-127.787921001387	-127.787921001387\\
56.75	0.24306	-133.220121382845	-133.220121382845\\
56.75	0.24672	-138.774553455952	-138.774553455952\\
56.75	0.25038	-144.45121722071	-144.45121722071\\
56.75	0.25404	-150.250112677117	-150.250112677117\\
56.75	0.2577	-156.171239825174	-156.171239825174\\
56.75	0.26136	-162.214598664881	-162.214598664881\\
56.75	0.26502	-168.380189196237	-168.380189196237\\
56.75	0.26868	-174.668011419243	-174.668011419243\\
56.75	0.27234	-181.078065333899	-181.078065333899\\
56.75	0.276	-187.610350940205	-187.610350940205\\
57.125	0.093	-10.6567527508345	-10.6567527508345\\
57.125	0.09666	-11.2024642860824	-11.2024642860824\\
57.125	0.10032	-11.8704075129803	-11.8704075129803\\
57.125	0.10398	-12.6605824315278	-12.6605824315278\\
57.125	0.10764	-13.5729890417252	-13.5729890417252\\
57.125	0.1113	-14.6076273435724	-14.6076273435724\\
57.125	0.11496	-15.7644973370694	-15.7644973370694\\
57.125	0.11862	-17.0435990222161	-17.0435990222161\\
57.125	0.12228	-18.4449323990127	-18.4449323990127\\
57.125	0.12594	-19.968497467459	-19.968497467459\\
57.125	0.1296	-21.6142942275552	-21.6142942275552\\
57.125	0.13326	-23.3823226793011	-23.3823226793011\\
57.125	0.13692	-25.2725828226968	-25.2725828226968\\
57.125	0.14058	-27.2850746577424	-27.2850746577424\\
57.125	0.14424	-29.4197981844377	-29.4197981844377\\
57.125	0.1479	-31.6767534027828	-31.6767534027828\\
57.125	0.15156	-34.0559403127778	-34.0559403127778\\
57.125	0.15522	-36.5573589144225	-36.5573589144225\\
57.125	0.15888	-39.181009207717	-39.181009207717\\
57.125	0.16254	-41.9268911926613	-41.9268911926613\\
57.125	0.1662	-44.7950048692553	-44.7950048692553\\
57.125	0.16986	-47.7853502374992	-47.7853502374992\\
57.125	0.17352	-50.8979272973929	-50.8979272973929\\
57.125	0.17718	-54.1327360489364	-54.1327360489364\\
57.125	0.18084	-57.4897764921296	-57.4897764921296\\
57.125	0.1845	-60.9690486269727	-60.9690486269727\\
57.125	0.18816	-64.5705524534656	-64.5705524534656\\
57.125	0.19182	-68.2942879716082	-68.2942879716082\\
57.125	0.19548	-72.1402551814006	-72.1402551814006\\
57.125	0.19914	-76.1084540828429	-76.1084540828429\\
57.125	0.2028	-80.1988846759349	-80.1988846759349\\
57.125	0.20646	-84.4115469606768	-84.4115469606768\\
57.125	0.21012	-88.7464409370684	-88.7464409370684\\
57.125	0.21378	-93.2035666051098	-93.2035666051098\\
57.125	0.21744	-97.782923964801	-97.782923964801\\
57.125	0.2211	-102.484513016142	-102.484513016142\\
57.125	0.22476	-107.308333759133	-107.308333759133\\
57.125	0.22842	-112.254386193773	-112.254386193773\\
57.125	0.23208	-117.322670320064	-117.322670320064\\
57.125	0.23574	-122.513186138004	-122.513186138004\\
57.125	0.2394	-127.825933647594	-127.825933647594\\
57.125	0.24306	-133.260912848834	-133.260912848834\\
57.125	0.24672	-138.818123741723	-138.818123741723\\
57.125	0.25038	-144.497566326263	-144.497566326263\\
57.125	0.25404	-150.299240602452	-150.299240602452\\
57.125	0.2577	-156.223146570291	-156.223146570291\\
57.125	0.26136	-162.269284229779	-162.269284229779\\
57.125	0.26502	-168.437653580918	-168.437653580918\\
57.125	0.26868	-174.728254623706	-174.728254623706\\
57.125	0.27234	-181.141087358144	-181.141087358144\\
57.125	0.276	-187.676151784232	-187.676151784232\\
57.5	0.093	-10.5932089806785	-10.5932089806785\\
57.5	0.09666	-11.1416993357085	-11.1416993357085\\
57.5	0.10032	-11.8124213823883	-11.8124213823883\\
57.5	0.10398	-12.6053751207179	-12.6053751207179\\
57.5	0.10764	-13.5205605506973	-13.5205605506973\\
57.5	0.1113	-14.5579776723265	-14.5579776723265\\
57.5	0.11496	-15.7176264856055	-15.7176264856055\\
57.5	0.11862	-16.9995069905342	-16.9995069905342\\
57.5	0.12228	-18.4036191871128	-18.4036191871128\\
57.5	0.12594	-19.9299630753412	-19.9299630753412\\
57.5	0.1296	-21.5785386552193	-21.5785386552193\\
57.5	0.13326	-23.3493459267472	-23.3493459267472\\
57.5	0.13692	-25.242384889925	-25.242384889925\\
57.5	0.14058	-27.2576555447525	-27.2576555447525\\
57.5	0.14424	-29.3951578912299	-29.3951578912299\\
57.5	0.1479	-31.654891929357	-31.654891929357\\
57.5	0.15156	-34.0368576591339	-34.0368576591339\\
57.5	0.15522	-36.5410550805606	-36.5410550805606\\
57.5	0.15888	-39.1674841936372	-39.1674841936372\\
57.5	0.16254	-41.9161449983635	-41.9161449983635\\
57.5	0.1662	-44.7870374947396	-44.7870374947396\\
57.5	0.16986	-47.7801616827655	-47.7801616827655\\
57.5	0.17352	-50.8955175624411	-50.8955175624411\\
57.5	0.17718	-54.1331051337666	-54.1331051337666\\
57.5	0.18084	-57.4929243967419	-57.4929243967419\\
57.5	0.1845	-60.974975351367	-60.974975351367\\
57.5	0.18816	-64.5792579976418	-64.5792579976418\\
57.5	0.19182	-68.3057723355665	-68.3057723355665\\
57.5	0.19548	-72.1545183651409	-72.1545183651409\\
57.5	0.19914	-76.1254960863652	-76.1254960863652\\
57.5	0.2028	-80.2187054992393	-80.2187054992393\\
57.5	0.20646	-84.4341466037631	-84.4341466037631\\
57.5	0.21012	-88.7718193999367	-88.7718193999367\\
57.5	0.21378	-93.2317238877601	-93.2317238877601\\
57.5	0.21744	-97.8138600672334	-97.8138600672334\\
57.5	0.2211	-102.518227938356	-102.518227938356\\
57.5	0.22476	-107.344827501129	-107.344827501129\\
57.5	0.22842	-112.293658755552	-112.293658755552\\
57.5	0.23208	-117.364721701624	-117.364721701624\\
57.5	0.23574	-122.558016339346	-122.558016339346\\
57.5	0.2394	-127.873542668718	-127.873542668718\\
57.5	0.24306	-133.31130068974	-133.31130068974\\
57.5	0.24672	-138.871290402412	-138.871290402412\\
57.5	0.25038	-144.553511806733	-144.553511806733\\
57.5	0.25404	-150.357964902704	-150.357964902704\\
57.5	0.2577	-156.284649690325	-156.284649690325\\
57.5	0.26136	-162.333566169596	-162.333566169596\\
57.5	0.26502	-168.504714340516	-168.504714340516\\
57.5	0.26868	-174.798094203087	-174.798094203087\\
57.5	0.27234	-181.213705757307	-181.213705757307\\
57.5	0.276	-187.751549003177	-187.751549003177\\
57.875	0.093	-10.5392615854401	-10.5392615854401\\
57.875	0.09666	-11.0905307602521	-11.0905307602521\\
57.875	0.10032	-11.7640316267139	-11.7640316267139\\
57.875	0.10398	-12.5597641848255	-12.5597641848255\\
57.875	0.10764	-13.4777284345869	-13.4777284345869\\
57.875	0.1113	-14.5179243759981	-14.5179243759981\\
57.875	0.11496	-15.680352009059	-15.680352009059\\
57.875	0.11862	-16.9650113337699	-16.9650113337699\\
57.875	0.12228	-18.3719023501304	-18.3719023501304\\
57.875	0.12594	-19.9010250581408	-19.9010250581408\\
57.875	0.1296	-21.5523794578009	-21.5523794578009\\
57.875	0.13326	-23.3259655491109	-23.3259655491109\\
57.875	0.13692	-25.2217833320706	-25.2217833320706\\
57.875	0.14058	-27.2398328066802	-27.2398328066802\\
57.875	0.14424	-29.3801139729396	-29.3801139729396\\
57.875	0.1479	-31.6426268308487	-31.6426268308487\\
57.875	0.15156	-34.0273713804076	-34.0273713804076\\
57.875	0.15522	-36.5343476216163	-36.5343476216163\\
57.875	0.15888	-39.1635555544749	-39.1635555544749\\
57.875	0.16254	-41.9149951789832	-41.9149951789832\\
57.875	0.1662	-44.7886664951413	-44.7886664951413\\
57.875	0.16986	-47.7845695029492	-47.7845695029492\\
57.875	0.17352	-50.9027042024069	-50.9027042024069\\
57.875	0.17718	-54.1430705935144	-54.1430705935144\\
57.875	0.18084	-57.5056686762716	-57.5056686762716\\
57.875	0.1845	-60.9904984506787	-60.9904984506787\\
57.875	0.18816	-64.5975599167356	-64.5975599167356\\
57.875	0.19182	-68.3268530744423	-68.3268530744423\\
57.875	0.19548	-72.1783779237987	-72.1783779237987\\
57.875	0.19914	-76.152134464805	-76.152134464805\\
57.875	0.2028	-80.2481226974611	-80.2481226974611\\
57.875	0.20646	-84.4663426217669	-84.4663426217669\\
57.875	0.21012	-88.8067942377226	-88.8067942377226\\
57.875	0.21378	-93.269477545328	-93.269477545328\\
57.875	0.21744	-97.8543925445832	-97.8543925445832\\
57.875	0.2211	-102.561539235488	-102.561539235488\\
57.875	0.22476	-107.390917618043	-107.390917618043\\
57.875	0.22842	-112.342527692248	-112.342527692248\\
57.875	0.23208	-117.416369458102	-117.416369458102\\
57.875	0.23574	-122.612442915606	-122.612442915606\\
57.875	0.2394	-127.93074806476	-127.93074806476\\
57.875	0.24306	-133.371284905564	-133.371284905564\\
57.875	0.24672	-138.934053438018	-138.934053438018\\
57.875	0.25038	-144.619053662121	-144.619053662121\\
57.875	0.25404	-150.426285577874	-150.426285577874\\
57.875	0.2577	-156.355749185277	-156.355749185277\\
57.875	0.26136	-162.40744448433	-162.40744448433\\
57.875	0.26502	-168.581371475032	-168.581371475032\\
57.875	0.26868	-174.877530157385	-174.877530157385\\
57.875	0.27234	-181.295920531387	-181.295920531387\\
57.875	0.276	-187.836542597039	-187.836542597039\\
58.25	0.093	-10.4949105651191	-10.4949105651191\\
58.25	0.09666	-11.0489585597131	-11.0489585597131\\
58.25	0.10032	-11.7252382459569	-11.7252382459569\\
58.25	0.10398	-12.5237496238505	-12.5237496238505\\
58.25	0.10764	-13.444492693394	-13.444492693394\\
58.25	0.1113	-14.4874674545871	-14.4874674545871\\
58.25	0.11496	-15.6526739074302	-15.6526739074302\\
58.25	0.11862	-16.9401120519229	-16.9401120519229\\
58.25	0.12228	-18.3497818880655	-18.3497818880655\\
58.25	0.12594	-19.8816834158579	-19.8816834158579\\
58.25	0.1296	-21.5358166353001	-21.5358166353001\\
58.25	0.13326	-23.312181546392	-23.312181546392\\
58.25	0.13692	-25.2107781491338	-25.2107781491338\\
58.25	0.14058	-27.2316064435254	-27.2316064435254\\
58.25	0.14424	-29.3746664295667	-29.3746664295667\\
58.25	0.1479	-31.6399581072579	-31.6399581072579\\
58.25	0.15156	-34.0274814765988	-34.0274814765988\\
58.25	0.15522	-36.5372365375895	-36.5372365375895\\
58.25	0.15888	-39.1692232902301	-39.1692232902301\\
58.25	0.16254	-41.9234417345204	-41.9234417345204\\
58.25	0.1662	-44.7998918704605	-44.7998918704605\\
58.25	0.16986	-47.7985736980504	-47.7985736980504\\
58.25	0.17352	-50.9194872172901	-50.9194872172901\\
58.25	0.17718	-54.1626324281796	-54.1626324281796\\
58.25	0.18084	-57.5280093307189	-57.5280093307189\\
58.25	0.1845	-61.015617924908	-61.015617924908\\
58.25	0.18816	-64.6254582107469	-64.6254582107469\\
58.25	0.19182	-68.3575301882356	-68.3575301882356\\
58.25	0.19548	-72.211833857374	-72.211833857374\\
58.25	0.19914	-76.1883692181623	-76.1883692181623\\
58.25	0.2028	-80.2871362706004	-80.2871362706004\\
58.25	0.20646	-84.5081350146882	-84.5081350146882\\
58.25	0.21012	-88.8513654504259	-88.8513654504259\\
58.25	0.21378	-93.3168275778133	-93.3168275778133\\
58.25	0.21744	-97.9045213968505	-97.9045213968505\\
58.25	0.2211	-102.614446907538	-102.614446907538\\
58.25	0.22476	-107.446604109874	-107.446604109874\\
58.25	0.22842	-112.400993003861	-112.400993003861\\
58.25	0.23208	-117.477613589497	-117.477613589497\\
58.25	0.23574	-122.676465866784	-122.676465866784\\
58.25	0.2394	-127.99754983572	-127.99754983572\\
58.25	0.24306	-133.440865496305	-133.440865496305\\
58.25	0.24672	-139.006412848541	-139.006412848541\\
58.25	0.25038	-144.694191892426	-144.694191892426\\
58.25	0.25404	-150.504202627962	-150.504202627962\\
58.25	0.2577	-156.436445055147	-156.436445055147\\
58.25	0.26136	-162.490919173981	-162.490919173981\\
58.25	0.26502	-168.667624984466	-168.667624984466\\
58.25	0.26868	-174.9665624866	-174.9665624866\\
58.25	0.27234	-181.387731680384	-181.387731680384\\
58.25	0.276	-187.931132565818	-187.931132565818\\
58.625	0.093	-10.4601559197156	-10.4601559197156\\
58.625	0.09666	-11.0169827340916	-11.0169827340916\\
58.625	0.10032	-11.6960412401175	-11.6960412401175\\
58.625	0.10398	-12.4973314377931	-12.4973314377931\\
58.625	0.10764	-13.4208533271185	-13.4208533271185\\
58.625	0.1113	-14.4666069080937	-14.4666069080937\\
58.625	0.11496	-15.6345921807187	-15.6345921807187\\
58.625	0.11862	-16.9248091449935	-16.9248091449935\\
58.625	0.12228	-18.3372578009181	-18.3372578009181\\
58.625	0.12594	-19.8719381484925	-19.8719381484925\\
58.625	0.1296	-21.5288501877167	-21.5288501877167\\
58.625	0.13326	-23.3079939185906	-23.3079939185906\\
58.625	0.13692	-25.2093693411144	-25.2093693411144\\
58.625	0.14058	-27.232976455288	-27.232976455288\\
58.625	0.14424	-29.3788152611114	-29.3788152611114\\
58.625	0.1479	-31.6468857585845	-31.6468857585845\\
58.625	0.15156	-34.0371879477075	-34.0371879477075\\
58.625	0.15522	-36.5497218284802	-36.5497218284802\\
58.625	0.15888	-39.1844874009027	-39.1844874009027\\
58.625	0.16254	-41.9414846649751	-41.9414846649751\\
58.625	0.1662	-44.8207136206972	-44.8207136206972\\
58.625	0.16986	-47.8221742680691	-47.8221742680691\\
58.625	0.17352	-50.9458666070908	-50.9458666070908\\
58.625	0.17718	-54.1917906377623	-54.1917906377623\\
58.625	0.18084	-57.5599463600836	-57.5599463600836\\
58.625	0.1845	-61.0503337740547	-61.0503337740547\\
58.625	0.18816	-64.6629528796756	-64.6629528796756\\
58.625	0.19182	-68.3978036769463	-68.3978036769463\\
58.625	0.19548	-72.2548861658668	-72.2548861658668\\
58.625	0.19914	-76.234200346437	-76.234200346437\\
58.625	0.2028	-80.3357462186571	-80.3357462186571\\
58.625	0.20646	-84.559523782527	-84.559523782527\\
58.625	0.21012	-88.9055330380467	-88.9055330380467\\
58.625	0.21378	-93.3737739852161	-93.3737739852161\\
58.625	0.21744	-97.9642466240353	-97.9642466240353\\
58.625	0.2211	-102.676950954504	-102.676950954504\\
58.625	0.22476	-107.511886976623	-107.511886976623\\
58.625	0.22842	-112.469054690392	-112.469054690392\\
58.625	0.23208	-117.54845409581	-117.54845409581\\
58.625	0.23574	-122.750085192879	-122.750085192879\\
58.625	0.2394	-128.073947981596	-128.073947981596\\
58.625	0.24306	-133.520042461964	-133.520042461964\\
58.625	0.24672	-139.088368633982	-139.088368633982\\
58.625	0.25038	-144.778926497649	-144.778926497649\\
58.625	0.25404	-150.591716052967	-150.591716052967\\
58.625	0.2577	-156.526737299933	-156.526737299933\\
58.625	0.26136	-162.58399023855	-162.58399023855\\
58.625	0.26502	-168.763474868817	-168.763474868817\\
58.625	0.26868	-175.065191190733	-175.065191190733\\
58.625	0.27234	-181.489139204299	-181.489139204299\\
58.625	0.276	-188.035318909515	-188.035318909515\\
59	0.093	-10.4349976492296	-10.4349976492296\\
59	0.09666	-10.9946032833877	-10.9946032833877\\
59	0.10032	-11.6764406091955	-11.6764406091955\\
59	0.10398	-12.4805096266531	-12.4805096266531\\
59	0.10764	-13.4068103357606	-13.4068103357606\\
59	0.1113	-14.4553427365177	-14.4553427365177\\
59	0.11496	-15.6261068289248	-15.6261068289248\\
59	0.11862	-16.9191026129816	-16.9191026129816\\
59	0.12228	-18.3343300886882	-18.3343300886882\\
59	0.12594	-19.8717892560446	-19.8717892560446\\
59	0.1296	-21.5314801150508	-21.5314801150508\\
59	0.13326	-23.3134026657067	-23.3134026657067\\
59	0.13692	-25.2175569080125	-25.2175569080125\\
59	0.14058	-27.2439428419681	-27.2439428419681\\
59	0.14424	-29.3925604675735	-29.3925604675735\\
59	0.1479	-31.6634097848287	-31.6634097848287\\
59	0.15156	-34.0564907937336	-34.0564907937336\\
59	0.15522	-36.5718034942883	-36.5718034942883\\
59	0.15888	-39.2093478864929	-39.2093478864929\\
59	0.16254	-41.9691239703473	-41.9691239703473\\
59	0.1662	-44.8511317458514	-44.8511317458514\\
59	0.16986	-47.8553712130053	-47.8553712130053\\
59	0.17352	-50.981842371809	-50.981842371809\\
59	0.17718	-54.2305452222625	-54.2305452222625\\
59	0.18084	-57.6014797643658	-57.6014797643658\\
59	0.1845	-61.0946459981189	-61.0946459981189\\
59	0.18816	-64.7100439235219	-64.7100439235219\\
59	0.19182	-68.4476735405746	-68.4476735405746\\
59	0.19548	-72.307534849277	-72.307534849277\\
59	0.19914	-76.2896278496293	-76.2896278496293\\
59	0.2028	-80.3939525416314	-80.3939525416314\\
59	0.20646	-84.6205089252833	-84.6205089252833\\
59	0.21012	-88.969297000585	-88.969297000585\\
59	0.21378	-93.4403167675364	-93.4403167675364\\
59	0.21744	-98.0335682261377	-98.0335682261377\\
59	0.2211	-102.749051376389	-102.749051376389\\
59	0.22476	-107.58676621829	-107.58676621829\\
59	0.22842	-112.54671275184	-112.54671275184\\
59	0.23208	-117.628890977041	-117.628890977041\\
59	0.23574	-122.833300893891	-122.833300893891\\
59	0.2394	-128.159942502391	-128.159942502391\\
59	0.24306	-133.608815802541	-133.608815802541\\
59	0.24672	-139.17992079434	-139.17992079434\\
59	0.25038	-144.87325747779	-144.87325747779\\
59	0.25404	-150.688825852889	-150.688825852889\\
59	0.2577	-156.626625919638	-156.626625919638\\
59	0.26136	-162.686657678037	-162.686657678037\\
59	0.26502	-168.868921128085	-168.868921128085\\
59	0.26868	-175.173416269784	-175.173416269784\\
59	0.27234	-181.600143103132	-181.600143103132\\
59	0.276	-188.14910162813	-188.14910162813\\
59.375	0.093	-10.4194357536611	-10.4194357536611\\
59.375	0.09666	-10.9818202076011	-10.9818202076011\\
59.375	0.10032	-11.666436353191	-11.666436353191\\
59.375	0.10398	-12.4732841904306	-12.4732841904306\\
59.375	0.10764	-13.40236371932	-13.40236371932\\
59.375	0.1113	-14.4536749398593	-14.4536749398593\\
59.375	0.11496	-15.6272178520483	-15.6272178520483\\
59.375	0.11862	-16.9229924558871	-16.9229924558871\\
59.375	0.12228	-18.3409987513758	-18.3409987513758\\
59.375	0.12594	-19.8812367385142	-19.8812367385142\\
59.375	0.1296	-21.5437064173023	-21.5437064173023\\
59.375	0.13326	-23.3284077877404	-23.3284077877404\\
59.375	0.13692	-25.2353408498281	-25.2353408498281\\
59.375	0.14058	-27.2645056035657	-27.2645056035657\\
59.375	0.14424	-29.4159020489531	-29.4159020489531\\
59.375	0.1479	-31.6895301859903	-31.6895301859903\\
59.375	0.15156	-34.0853900146772	-34.0853900146772\\
59.375	0.15522	-36.603481535014	-36.603481535014\\
59.375	0.15888	-39.2438047470006	-39.2438047470006\\
59.375	0.16254	-42.0063596506369	-42.0063596506369\\
59.375	0.1662	-44.8911462459231	-44.8911462459231\\
59.375	0.16986	-47.898164532859	-47.898164532859\\
59.375	0.17352	-51.0274145114447	-51.0274145114447\\
59.375	0.17718	-54.2788961816802	-54.2788961816802\\
59.375	0.18084	-57.6526095435655	-57.6526095435655\\
59.375	0.1845	-61.1485545971007	-61.1485545971007\\
59.375	0.18816	-64.7667313422856	-64.7667313422856\\
59.375	0.19182	-68.5071397791203	-68.5071397791203\\
59.375	0.19548	-72.3697799076048	-72.3697799076048\\
59.375	0.19914	-76.3546517277391	-76.3546517277391\\
59.375	0.2028	-80.4617552395232	-80.4617552395232\\
59.375	0.20646	-84.6910904429571	-84.6910904429571\\
59.375	0.21012	-89.0426573380407	-89.0426573380407\\
59.375	0.21378	-93.5164559247742	-93.5164559247742\\
59.375	0.21744	-98.1124862031574	-98.1124862031574\\
59.375	0.2211	-102.830748173191	-102.830748173191\\
59.375	0.22476	-107.671241834873	-107.671241834873\\
59.375	0.22842	-112.633967188206	-112.633967188206\\
59.375	0.23208	-117.718924233188	-117.718924233188\\
59.375	0.23574	-122.926112969821	-122.926112969821\\
59.375	0.2394	-128.255533398103	-128.255533398103\\
59.375	0.24306	-133.707185518035	-133.707185518035\\
59.375	0.24672	-139.281069329616	-139.281069329616\\
59.375	0.25038	-144.977184832848	-144.977184832848\\
59.375	0.25404	-150.795532027729	-150.795532027729\\
59.375	0.2577	-156.73611091426	-156.73611091426\\
59.375	0.26136	-162.798921492441	-162.798921492441\\
59.375	0.26502	-168.983963762271	-168.983963762271\\
59.375	0.26868	-175.291237723752	-175.291237723752\\
59.375	0.27234	-181.720743376882	-181.720743376882\\
59.375	0.276	-188.272480721662	-188.272480721662\\
59.75	0.093	-10.4134702330102	-10.4134702330102\\
59.75	0.09666	-10.9786335067322	-10.9786335067322\\
59.75	0.10032	-11.6660284721041	-11.6660284721041\\
59.75	0.10398	-12.4756551291257	-12.4756551291257\\
59.75	0.10764	-13.4075134777971	-13.4075134777971\\
59.75	0.1113	-14.4616035181184	-14.4616035181184\\
59.75	0.11496	-15.6379252500894	-15.6379252500894\\
59.75	0.11862	-16.9364786737102	-16.9364786737102\\
59.75	0.12228	-18.3572637889809	-18.3572637889809\\
59.75	0.12594	-19.9002805959013	-19.9002805959013\\
59.75	0.1296	-21.5655290944715	-21.5655290944715\\
59.75	0.13326	-23.3530092846915	-23.3530092846915\\
59.75	0.13692	-25.2627211665612	-25.2627211665612\\
59.75	0.14058	-27.2946647400808	-27.2946647400808\\
59.75	0.14424	-29.4488400052502	-29.4488400052502\\
59.75	0.1479	-31.7252469620694	-31.7252469620694\\
59.75	0.15156	-34.1238856105384	-34.1238856105384\\
59.75	0.15522	-36.6447559506572	-36.6447559506572\\
59.75	0.15888	-39.2878579824258	-39.2878579824258\\
59.75	0.16254	-42.0531917058441	-42.0531917058441\\
59.75	0.1662	-44.9407571209123	-44.9407571209123\\
59.75	0.16986	-47.9505542276302	-47.9505542276302\\
59.75	0.17352	-51.0825830259979	-51.0825830259979\\
59.75	0.17718	-54.3368435160154	-54.3368435160154\\
59.75	0.18084	-57.7133356976828	-57.7133356976828\\
59.75	0.1845	-61.2120595709999	-61.2120595709999\\
59.75	0.18816	-64.8330151359668	-64.8330151359668\\
59.75	0.19182	-68.5762023925836	-68.5762023925836\\
59.75	0.19548	-72.44162134085	-72.44162134085\\
59.75	0.19914	-76.4292719807663	-76.4292719807663\\
59.75	0.2028	-80.5391543123325	-80.5391543123325\\
59.75	0.20646	-84.7712683355484	-84.7712683355484\\
59.75	0.21012	-89.1256140504141	-89.1256140504141\\
59.75	0.21378	-93.6021914569295	-93.6021914569295\\
59.75	0.21744	-98.2010005550948	-98.2010005550948\\
59.75	0.2211	-102.92204134491	-102.92204134491\\
59.75	0.22476	-107.765313826375	-107.765313826375\\
59.75	0.22842	-112.730817999489	-112.730817999489\\
59.75	0.23208	-117.818553864254	-117.818553864254\\
59.75	0.23574	-123.028521420668	-123.028521420668\\
59.75	0.2394	-128.360720668732	-128.360720668732\\
59.75	0.24306	-133.815151608446	-133.815151608446\\
59.75	0.24672	-139.39181423981	-139.39181423981\\
59.75	0.25038	-145.090708562823	-145.090708562823\\
59.75	0.25404	-150.911834577486	-150.911834577486\\
59.75	0.2577	-156.855192283799	-156.855192283799\\
59.75	0.26136	-162.920781681762	-162.920781681762\\
59.75	0.26502	-169.108602771375	-169.108602771375\\
59.75	0.26868	-175.418655552637	-175.418655552637\\
59.75	0.27234	-181.850940025549	-181.850940025549\\
59.75	0.276	-188.405456190111	-188.405456190111\\
60.125	0.093	-10.4171010872766	-10.4171010872766\\
60.125	0.09666	-10.9850431807806	-10.9850431807806\\
60.125	0.10032	-11.6752169659345	-11.6752169659345\\
60.125	0.10398	-12.4876224427381	-12.4876224427381\\
60.125	0.10764	-13.4222596111916	-13.4222596111916\\
60.125	0.1113	-14.4791284712948	-14.4791284712948\\
60.125	0.11496	-15.6582290230479	-15.6582290230479\\
60.125	0.11862	-16.9595612664507	-16.9595612664507\\
60.125	0.12228	-18.3831252015034	-18.3831252015034\\
60.125	0.12594	-19.9289208282058	-19.9289208282058\\
60.125	0.1296	-21.596948146558	-21.596948146558\\
60.125	0.13326	-23.38720715656	-23.38720715656\\
60.125	0.13692	-25.2996978582118	-25.2996978582118\\
60.125	0.14058	-27.3344202515134	-27.3344202515134\\
60.125	0.14424	-29.4913743364648	-29.4913743364648\\
60.125	0.1479	-31.770560113066	-31.770560113066\\
60.125	0.15156	-34.171977581317	-34.171977581317\\
60.125	0.15522	-36.6956267412178	-36.6956267412178\\
60.125	0.15888	-39.3415075927683	-39.3415075927683\\
60.125	0.16254	-42.1096201359687	-42.1096201359687\\
60.125	0.1662	-44.9999643708188	-44.9999643708188\\
60.125	0.16986	-48.0125402973188	-48.0125402973188\\
60.125	0.17352	-51.1473479154686	-51.1473479154686\\
60.125	0.17718	-54.4043872252681	-54.4043872252681\\
60.125	0.18084	-57.7836582267174	-57.7836582267174\\
60.125	0.1845	-61.2851609198165	-61.2851609198165\\
60.125	0.18816	-64.9088953045655	-64.9088953045655\\
60.125	0.19182	-68.6548613809642	-68.6548613809642\\
60.125	0.19548	-72.5230591490127	-72.5230591490127\\
60.125	0.19914	-76.513488608711	-76.513488608711\\
60.125	0.2028	-80.6261497600591	-80.6261497600591\\
60.125	0.20646	-84.8610426030571	-84.8610426030571\\
60.125	0.21012	-89.2181671377048	-89.2181671377048\\
60.125	0.21378	-93.6975233640023	-93.6975233640023\\
60.125	0.21744	-98.2991112819495	-98.2991112819495\\
60.125	0.2211	-103.022930891547	-103.022930891547\\
60.125	0.22476	-107.868982192793	-107.868982192793\\
60.125	0.22842	-112.83726518569	-112.83726518569\\
60.125	0.23208	-117.927779870237	-117.927779870237\\
60.125	0.23574	-123.140526246433	-123.140526246433\\
60.125	0.2394	-128.475504314279	-128.475504314279\\
60.125	0.24306	-133.932714073775	-133.932714073775\\
60.125	0.24672	-139.51215552492	-139.51215552492\\
60.125	0.25038	-145.213828667716	-145.213828667716\\
60.125	0.25404	-151.037733502161	-151.037733502161\\
60.125	0.2577	-156.983870028256	-156.983870028256\\
60.125	0.26136	-163.052238246001	-163.052238246001\\
60.125	0.26502	-169.242838155395	-169.242838155395\\
60.125	0.26868	-175.55566975644	-175.55566975644\\
60.125	0.27234	-181.990733049134	-181.990733049134\\
60.125	0.276	-188.548028033478	-188.548028033478\\
60.5	0.093	-10.4303283164606	-10.4303283164606\\
60.5	0.09666	-11.0010492297466	-11.0010492297466\\
60.5	0.10032	-11.6940018346825	-11.6940018346825\\
60.5	0.10398	-12.5091861312682	-12.5091861312682\\
60.5	0.10764	-13.4466021195036	-13.4466021195036\\
60.5	0.1113	-14.5062497993889	-14.5062497993889\\
60.5	0.11496	-15.6881291709239	-15.6881291709239\\
60.5	0.11862	-16.9922402341087	-16.9922402341087\\
60.5	0.12228	-18.4185829889434	-18.4185829889434\\
60.5	0.12594	-19.9671574354278	-19.9671574354278\\
60.5	0.1296	-21.637963573562	-21.637963573562\\
60.5	0.13326	-23.4310014033461	-23.4310014033461\\
60.5	0.13692	-25.3462709247799	-25.3462709247799\\
60.5	0.14058	-27.3837721378635	-27.3837721378635\\
60.5	0.14424	-29.5435050425969	-29.5435050425969\\
60.5	0.1479	-31.8254696389801	-31.8254696389801\\
60.5	0.15156	-34.2296659270131	-34.2296659270131\\
60.5	0.15522	-36.7560939066959	-36.7560939066959\\
60.5	0.15888	-39.4047535780285	-39.4047535780285\\
60.5	0.16254	-42.1756449410109	-42.1756449410109\\
60.5	0.1662	-45.068767995643	-45.068767995643\\
60.5	0.16986	-48.084122741925	-48.084122741925\\
60.5	0.17352	-51.2217091798567	-51.2217091798567\\
60.5	0.17718	-54.4815273094383	-54.4815273094383\\
60.5	0.18084	-57.8635771306696	-57.8635771306696\\
60.5	0.1845	-61.3678586435508	-61.3678586435508\\
60.5	0.18816	-64.9943718480817	-64.9943718480817\\
60.5	0.19182	-68.7431167442625	-68.7431167442625\\
60.5	0.19548	-72.6140933320929	-72.6140933320929\\
60.5	0.19914	-76.6073016115733	-76.6073016115733\\
60.5	0.2028	-80.7227415827034	-80.7227415827034\\
60.5	0.20646	-84.9604132454834	-84.9604132454834\\
60.5	0.21012	-89.320316599913	-89.320316599913\\
60.5	0.21378	-93.8024516459925	-93.8024516459925\\
60.5	0.21744	-98.4068183837218	-98.4068183837218\\
60.5	0.2211	-103.133416813101	-103.133416813101\\
60.5	0.22476	-107.98224693413	-107.98224693413\\
60.5	0.22842	-112.953308746808	-112.953308746808\\
60.5	0.23208	-118.046602251137	-118.046602251137\\
60.5	0.23574	-123.262127447115	-123.262127447115\\
60.5	0.2394	-128.599884334743	-128.599884334743\\
60.5	0.24306	-134.059872914021	-134.059872914021\\
60.5	0.24672	-139.642093184949	-139.642093184949\\
60.5	0.25038	-145.346545147526	-145.346545147526\\
60.5	0.25404	-151.173228801753	-151.173228801753\\
60.5	0.2577	-157.12214414763	-157.12214414763\\
60.5	0.26136	-163.193291185157	-163.193291185157\\
60.5	0.26502	-169.386669914334	-169.386669914334\\
60.5	0.26868	-175.70228033516	-175.70228033516\\
60.5	0.27234	-182.140122447636	-182.140122447636\\
60.5	0.276	-188.700196251762	-188.700196251762\\
60.875	0.093	-10.453151920562	-10.453151920562\\
60.875	0.09666	-11.02665165363	-11.02665165363\\
60.875	0.10032	-11.722383078348	-11.722383078348\\
60.875	0.10398	-12.5403461947156	-12.5403461947156\\
60.875	0.10764	-13.4805410027331	-13.4805410027331\\
60.875	0.1113	-14.5429675024004	-14.5429675024004\\
60.875	0.11496	-15.7276256937174	-15.7276256937174\\
60.875	0.11862	-17.0345155766842	-17.0345155766842\\
60.875	0.12228	-18.4636371513009	-18.4636371513009\\
60.875	0.12594	-20.0149904175673	-20.0149904175673\\
60.875	0.1296	-21.6885753754836	-21.6885753754836\\
60.875	0.13326	-23.4843920250496	-23.4843920250496\\
60.875	0.13692	-25.4024403662654	-25.4024403662654\\
60.875	0.14058	-27.4427203991311	-27.4427203991311\\
60.875	0.14424	-29.6052321236465	-29.6052321236465\\
60.875	0.1479	-31.8899755398117	-31.8899755398117\\
60.875	0.15156	-34.2969506476267	-34.2969506476267\\
60.875	0.15522	-36.8261574470915	-36.8261574470915\\
60.875	0.15888	-39.4775959382061	-39.4775959382061\\
60.875	0.16254	-42.2512661209704	-42.2512661209704\\
60.875	0.1662	-45.1471679953846	-45.1471679953846\\
60.875	0.16986	-48.1653015614486	-48.1653015614486\\
60.875	0.17352	-51.3056668191624	-51.3056668191624\\
60.875	0.17718	-54.5682637685259	-54.5682637685259\\
60.875	0.18084	-57.9530924095392	-57.9530924095392\\
60.875	0.1845	-61.4601527422024	-61.4601527422024\\
60.875	0.18816	-65.0894447665154	-65.0894447665154\\
60.875	0.19182	-68.8409684824781	-68.8409684824781\\
60.875	0.19548	-72.7147238900906	-72.7147238900906\\
60.875	0.19914	-76.710710989353	-76.710710989353\\
60.875	0.2028	-80.8289297802651	-80.8289297802651\\
60.875	0.20646	-85.0693802628271	-85.0693802628271\\
60.875	0.21012	-89.4320624370387	-89.4320624370387\\
60.875	0.21378	-93.9169763029002	-93.9169763029002\\
60.875	0.21744	-98.5241218604115	-98.5241218604115\\
60.875	0.2211	-103.253499109573	-103.253499109573\\
60.875	0.22476	-108.105108050384	-108.105108050384\\
60.875	0.22842	-113.078948682844	-113.078948682844\\
60.875	0.23208	-118.175021006955	-118.175021006955\\
60.875	0.23574	-123.393325022715	-123.393325022715\\
60.875	0.2394	-128.733860730125	-128.733860730125\\
60.875	0.24306	-134.196628129185	-134.196628129185\\
60.875	0.24672	-139.781627219895	-139.781627219895\\
60.875	0.25038	-145.488858002254	-145.488858002254\\
60.875	0.25404	-151.318320476263	-151.318320476263\\
60.875	0.2577	-157.270014641922	-157.270014641922\\
60.875	0.26136	-163.343940499231	-163.343940499231\\
60.875	0.26502	-169.54009804819	-169.54009804819\\
60.875	0.26868	-175.858487288798	-175.858487288798\\
60.875	0.27234	-182.299108221056	-182.299108221056\\
60.875	0.276	-188.861960844964	-188.861960844964\\
61.25	0.093	-10.485571899581	-10.485571899581\\
61.25	0.09666	-11.061850452431	-11.061850452431\\
61.25	0.10032	-11.7603606969309	-11.7603606969309\\
61.25	0.10398	-12.5811026330806	-12.5811026330806\\
61.25	0.10764	-13.5240762608801	-13.5240762608801\\
61.25	0.1113	-14.5892815803294	-14.5892815803294\\
61.25	0.11496	-15.7767185914284	-15.7767185914284\\
61.25	0.11862	-17.0863872941773	-17.0863872941773\\
61.25	0.12228	-18.518287688576	-18.518287688576\\
61.25	0.12594	-20.0724197746244	-20.0724197746244\\
61.25	0.1296	-21.7487835523226	-21.7487835523226\\
61.25	0.13326	-23.5473790216707	-23.5473790216707\\
61.25	0.13692	-25.4682061826685	-25.4682061826685\\
61.25	0.14058	-27.5112650353161	-27.5112650353161\\
61.25	0.14424	-29.6765555796136	-29.6765555796136\\
61.25	0.1479	-31.9640778155608	-31.9640778155608\\
61.25	0.15156	-34.3738317431578	-34.3738317431578\\
61.25	0.15522	-36.9058173624046	-36.9058173624046\\
61.25	0.15888	-39.5600346733012	-39.5600346733012\\
61.25	0.16254	-42.3364836758476	-42.3364836758476\\
61.25	0.1662	-45.2351643700438	-45.2351643700438\\
61.25	0.16986	-48.2560767558898	-48.2560767558898\\
61.25	0.17352	-51.3992208333855	-51.3992208333855\\
61.25	0.17718	-54.6645966025311	-54.6645966025311\\
61.25	0.18084	-58.0522040633264	-58.0522040633264\\
61.25	0.1845	-61.5620432157716	-61.5620432157716\\
61.25	0.18816	-65.1941140598666	-65.1941140598666\\
61.25	0.19182	-68.9484165956113	-68.9484165956113\\
61.25	0.19548	-72.8249508230058	-72.8249508230058\\
61.25	0.19914	-76.8237167420502	-76.8237167420502\\
61.25	0.2028	-80.9447143527443	-80.9447143527443\\
61.25	0.20646	-85.1879436550883	-85.1879436550883\\
61.25	0.21012	-89.553404649082	-89.553404649082\\
61.25	0.21378	-94.0410973347255	-94.0410973347255\\
61.25	0.21744	-98.6510217120188	-98.6510217120188\\
61.25	0.2211	-103.383177780962	-103.383177780962\\
61.25	0.22476	-108.237565541555	-108.237565541555\\
61.25	0.22842	-113.214184993797	-113.214184993797\\
61.25	0.23208	-118.31303613769	-118.31303613769\\
61.25	0.23574	-123.534118973232	-123.534118973232\\
61.25	0.2394	-128.877433500424	-128.877433500424\\
61.25	0.24306	-134.342979719266	-134.342979719266\\
61.25	0.24672	-139.930757629758	-139.930757629758\\
61.25	0.25038	-145.640767231899	-145.640767231899\\
61.25	0.25404	-151.473008525691	-151.473008525691\\
61.25	0.2577	-157.427481511132	-157.427481511132\\
61.25	0.26136	-163.504186188222	-163.504186188222\\
61.25	0.26502	-169.703122556963	-169.703122556963\\
61.25	0.26868	-176.024290617354	-176.024290617354\\
61.25	0.27234	-182.467690369394	-182.467690369394\\
61.25	0.276	-189.033321813084	-189.033321813084\\
61.625	0.093	-10.5275882535174	-10.5275882535174\\
61.625	0.09666	-11.1066456261495	-11.1066456261495\\
61.625	0.10032	-11.8079346904314	-11.8079346904314\\
61.625	0.10398	-12.6314554463631	-12.6314554463631\\
61.625	0.10764	-13.5772078939446	-13.5772078939446\\
61.625	0.1113	-14.6451920331759	-14.6451920331759\\
61.625	0.11496	-15.8354078640569	-15.8354078640569\\
61.625	0.11862	-17.1478553865878	-17.1478553865878\\
61.625	0.12228	-18.5825346007685	-18.5825346007685\\
61.625	0.12594	-20.1394455065989	-20.1394455065989\\
61.625	0.1296	-21.8185881040792	-21.8185881040792\\
61.625	0.13326	-23.6199623932092	-23.6199623932092\\
61.625	0.13692	-25.543568373989	-25.543568373989\\
61.625	0.14058	-27.5894060464187	-27.5894060464187\\
61.625	0.14424	-29.7574754104982	-29.7574754104982\\
61.625	0.1479	-32.0477764662273	-32.0477764662273\\
61.625	0.15156	-34.4603092136064	-34.4603092136064\\
61.625	0.15522	-36.9950736526352	-36.9950736526352\\
61.625	0.15888	-39.6520697833138	-39.6520697833138\\
61.625	0.16254	-42.4312976056422	-42.4312976056422\\
61.625	0.1662	-45.3327571196204	-45.3327571196204\\
61.625	0.16986	-48.3564483252484	-48.3564483252484\\
61.625	0.17352	-51.5023712225262	-51.5023712225262\\
61.625	0.17718	-54.7705258114537	-54.7705258114537\\
61.625	0.18084	-58.1609120920311	-58.1609120920311\\
61.625	0.1845	-61.6735300642583	-61.6735300642583\\
61.625	0.18816	-65.3083797281353	-65.3083797281353\\
61.625	0.19182	-69.065461083662	-69.065461083662\\
61.625	0.19548	-72.9447741308386	-72.9447741308386\\
61.625	0.19914	-76.9463188696649	-76.9463188696649\\
61.625	0.2028	-81.0700953001411	-81.0700953001411\\
61.625	0.20646	-85.316103422267	-85.316103422267\\
61.625	0.21012	-89.6843432360427	-89.6843432360427\\
61.625	0.21378	-94.1748147414682	-94.1748147414682\\
61.625	0.21744	-98.7875179385435	-98.7875179385435\\
61.625	0.2211	-103.522452827269	-103.522452827269\\
61.625	0.22476	-108.379619407644	-108.379619407644\\
61.625	0.22842	-113.359017679668	-113.359017679668\\
61.625	0.23208	-118.460647643343	-118.460647643343\\
61.625	0.23574	-123.684509298667	-123.684509298667\\
61.625	0.2394	-129.030602645641	-129.030602645641\\
61.625	0.24306	-134.498927684265	-134.498927684265\\
61.625	0.24672	-140.089484414539	-140.089484414539\\
61.625	0.25038	-145.802272836462	-145.802272836462\\
61.625	0.25404	-151.637292950035	-151.637292950035\\
61.625	0.2577	-157.594544755259	-157.594544755259\\
61.625	0.26136	-163.674028252131	-163.674028252131\\
61.625	0.26502	-169.875743440654	-169.875743440654\\
61.625	0.26868	-176.199690320826	-176.199690320826\\
61.625	0.27234	-182.645868892649	-182.645868892649\\
61.625	0.276	-189.214279156121	-189.214279156121\\
62	0.093	-10.5792009823713	-10.5792009823713\\
62	0.09666	-11.1610371747854	-11.1610371747854\\
62	0.10032	-11.8651050588493	-11.8651050588493\\
62	0.10398	-12.691404634563	-12.691404634563\\
62	0.10764	-13.6399359019265	-13.6399359019265\\
62	0.1113	-14.7106988609398	-14.7106988609398\\
62	0.11496	-15.9036935116028	-15.9036935116028\\
62	0.11862	-17.2189198539157	-17.2189198539157\\
62	0.12228	-18.6563778878784	-18.6563778878784\\
62	0.12594	-20.2160676134909	-20.2160676134909\\
62	0.1296	-21.8979890307531	-21.8979890307531\\
62	0.13326	-23.7021421396652	-23.7021421396652\\
62	0.13692	-25.628526940227	-25.628526940227\\
62	0.14058	-27.6771434324387	-27.6771434324387\\
62	0.14424	-29.8479916163001	-29.8479916163001\\
62	0.1479	-32.1410714918114	-32.1410714918114\\
62	0.15156	-34.5563830589724	-34.5563830589724\\
62	0.15522	-37.0939263177832	-37.0939263177832\\
62	0.15888	-39.7537012682439	-39.7537012682439\\
62	0.16254	-42.5357079103543	-42.5357079103543\\
62	0.1662	-45.4399462441145	-45.4399462441145\\
62	0.16986	-48.4664162695245	-48.4664162695245\\
62	0.17352	-51.6151179865843	-51.6151179865843\\
62	0.17718	-54.8860513952938	-54.8860513952938\\
62	0.18084	-58.2792164956532	-58.2792164956532\\
62	0.1845	-61.7946132876624	-61.7946132876624\\
62	0.18816	-65.4322417713214	-65.4322417713214\\
62	0.19182	-69.1921019466301	-69.1921019466301\\
62	0.19548	-73.0741938135887	-73.0741938135887\\
62	0.19914	-77.078517372197	-77.078517372197\\
62	0.2028	-81.2050726224552	-81.2050726224552\\
62	0.20646	-85.4538595643632	-85.4538595643632\\
62	0.21012	-89.8248781979209	-89.8248781979209\\
62	0.21378	-94.3181285231284	-94.3181285231284\\
62	0.21744	-98.9336105399857	-98.9336105399857\\
62	0.2211	-103.671324248493	-103.671324248493\\
62	0.22476	-108.53126964865	-108.53126964865\\
62	0.22842	-113.513446740456	-113.513446740456\\
62	0.23208	-118.617855523913	-118.617855523913\\
62	0.23574	-123.844495999019	-123.844495999019\\
62	0.2394	-129.193368165775	-129.193368165775\\
62	0.24306	-134.664472024181	-134.664472024181\\
62	0.24672	-140.257807574237	-140.257807574237\\
62	0.25038	-145.973374815942	-145.973374815942\\
62	0.25404	-151.811173749298	-151.811173749298\\
62	0.2577	-157.771204374303	-157.771204374303\\
62	0.26136	-163.853466690958	-163.853466690958\\
62	0.26502	-170.057960699262	-170.057960699262\\
62	0.26868	-176.384686399217	-176.384686399217\\
62	0.27234	-182.833643790821	-182.833643790821\\
62	0.276	-189.404832874075	-189.404832874075\\
62.375	0.093	-10.6404100861427	-10.6404100861427\\
62.375	0.09666	-11.2250250983388	-11.2250250983388\\
62.375	0.10032	-11.9318718021848	-11.9318718021848\\
62.375	0.10398	-12.7609501976805	-12.7609501976805\\
62.375	0.10764	-13.712260284826	-13.712260284826\\
62.375	0.1113	-14.7858020636213	-14.7858020636213\\
62.375	0.11496	-15.9815755340664	-15.9815755340664\\
62.375	0.11862	-17.2995806961613	-17.2995806961613\\
62.375	0.12228	-18.739817549906	-18.739817549906\\
62.375	0.12594	-20.3022860953004	-20.3022860953004\\
62.375	0.1296	-21.9869863323447	-21.9869863323447\\
62.375	0.13326	-23.7939182610387	-23.7939182610387\\
62.375	0.13692	-25.7230818813826	-25.7230818813826\\
62.375	0.14058	-27.7744771933763	-27.7744771933763\\
62.375	0.14424	-29.9481041970197	-29.9481041970197\\
62.375	0.1479	-32.243962892313	-32.243962892313\\
62.375	0.15156	-34.662053279256	-34.662053279256\\
62.375	0.15522	-37.2023753578488	-37.2023753578488\\
62.375	0.15888	-39.8649291280915	-39.8649291280915\\
62.375	0.16254	-42.6497145899839	-42.6497145899839\\
62.375	0.1662	-45.5567317435261	-45.5567317435261\\
62.375	0.16986	-48.5859805887181	-48.5859805887181\\
62.375	0.17352	-51.7374611255599	-51.7374611255599\\
62.375	0.17718	-55.0111733540515	-55.0111733540515\\
62.375	0.18084	-58.4071172741929	-58.4071172741929\\
62.375	0.1845	-61.925292885984	-61.925292885984\\
62.375	0.18816	-65.5657001894251	-65.5657001894251\\
62.375	0.19182	-69.3283391845158	-69.3283391845158\\
62.375	0.19548	-73.2132098712564	-73.2132098712564\\
62.375	0.19914	-77.2203122496468	-77.2203122496468\\
62.375	0.2028	-81.3496463196869	-81.3496463196869\\
62.375	0.20646	-85.6012120813769	-85.6012120813769\\
62.375	0.21012	-89.9750095347167	-89.9750095347167\\
62.375	0.21378	-94.4710386797062	-94.4710386797062\\
62.375	0.21744	-99.0892995163455	-99.0892995163455\\
62.375	0.2211	-103.829792044635	-103.829792044635\\
62.375	0.22476	-108.692516264574	-108.692516264574\\
62.375	0.22842	-113.677472176162	-113.677472176162\\
62.375	0.23208	-118.784659779401	-118.784659779401\\
62.375	0.23574	-124.014079074289	-124.014079074289\\
62.375	0.2394	-129.365730060827	-129.365730060827\\
62.375	0.24306	-134.839612739015	-134.839612739015\\
62.375	0.24672	-140.435727108853	-140.435727108853\\
62.375	0.25038	-146.15407317034	-146.15407317034\\
62.375	0.25404	-151.994650923478	-151.994650923478\\
62.375	0.2577	-157.957460368265	-157.957460368265\\
62.375	0.26136	-164.042501504701	-164.042501504701\\
62.375	0.26502	-170.249774332788	-170.249774332788\\
62.375	0.26868	-176.579278852525	-176.579278852525\\
62.375	0.27234	-183.031015063911	-183.031015063911\\
62.375	0.276	-189.604982966947	-189.604982966947\\
62.75	0.093	-10.7112155648316	-10.7112155648316\\
62.75	0.09666	-11.2986093968097	-11.2986093968097\\
62.75	0.10032	-12.0082349204377	-12.0082349204377\\
62.75	0.10398	-12.8400921357154	-12.8400921357154\\
62.75	0.10764	-13.7941810426428	-13.7941810426428\\
62.75	0.1113	-14.8705016412202	-14.8705016412202\\
62.75	0.11496	-16.0690539314473	-16.0690539314473\\
62.75	0.11862	-17.3898379133242	-17.3898379133242\\
62.75	0.12228	-18.8328535868509	-18.8328535868509\\
62.75	0.12594	-20.3981009520273	-20.3981009520273\\
62.75	0.1296	-22.0855800088536	-22.0855800088536\\
62.75	0.13326	-23.8952907573297	-23.8952907573297\\
62.75	0.13692	-25.8272331974555	-25.8272331974555\\
62.75	0.14058	-27.8814073292312	-27.8814073292312\\
62.75	0.14424	-30.0578131526567	-30.0578131526567\\
62.75	0.1479	-32.3564506677319	-32.3564506677319\\
62.75	0.15156	-34.777319874457	-34.777319874457\\
62.75	0.15522	-37.3204207728318	-37.3204207728318\\
62.75	0.15888	-39.9857533628565	-39.9857533628565\\
62.75	0.16254	-42.7733176445309	-42.7733176445309\\
62.75	0.1662	-45.6831136178551	-45.6831136178551\\
62.75	0.16986	-48.7151412828292	-48.7151412828292\\
62.75	0.17352	-51.8694006394529	-51.8694006394529\\
62.75	0.17718	-55.1458916877265	-55.1458916877265\\
62.75	0.18084	-58.5446144276499	-58.5446144276499\\
62.75	0.1845	-62.0655688592231	-62.0655688592231\\
62.75	0.18816	-65.7087549824462	-65.7087549824462\\
62.75	0.19182	-69.474172797319	-69.474172797319\\
62.75	0.19548	-73.3618223038415	-73.3618223038415\\
62.75	0.19914	-77.3717035020139	-77.3717035020139\\
62.75	0.2028	-81.5038163918361	-81.5038163918361\\
62.75	0.20646	-85.758160973308	-85.758160973308\\
62.75	0.21012	-90.1347372464298	-90.1347372464298\\
62.75	0.21378	-94.6335452112014	-94.6335452112014\\
62.75	0.21744	-99.2545848676226	-99.2545848676226\\
62.75	0.2211	-103.997856215694	-103.997856215694\\
62.75	0.22476	-108.863359255415	-108.863359255415\\
62.75	0.22842	-113.851093986785	-113.851093986785\\
62.75	0.23208	-118.961060409806	-118.961060409806\\
62.75	0.23574	-124.193258524476	-124.193258524476\\
62.75	0.2394	-129.547688330796	-129.547688330796\\
62.75	0.24306	-135.024349828766	-135.024349828766\\
62.75	0.24672	-140.623243018386	-140.623243018386\\
62.75	0.25038	-146.344367899656	-146.344367899656\\
62.75	0.25404	-152.187724472575	-152.187724472575\\
62.75	0.2577	-158.153312737144	-158.153312737144\\
62.75	0.26136	-164.241132693363	-164.241132693363\\
62.75	0.26502	-170.451184341231	-170.451184341231\\
62.75	0.26868	-176.78346768075	-176.78346768075\\
62.75	0.27234	-183.237982711918	-183.237982711918\\
62.75	0.276	-189.814729434736	-189.814729434736\\
63.125	0.093	-10.791617418438	-10.791617418438\\
63.125	0.09666	-11.3817900701981	-11.3817900701981\\
63.125	0.10032	-12.0941944136081	-12.0941944136081\\
63.125	0.10398	-12.9288304486678	-12.9288304486678\\
63.125	0.10764	-13.8856981753773	-13.8856981753773\\
63.125	0.1113	-14.9647975937367	-14.9647975937367\\
63.125	0.11496	-16.1661287037457	-16.1661287037457\\
63.125	0.11862	-17.4896915054046	-17.4896915054046\\
63.125	0.12228	-18.9354859987134	-18.9354859987134\\
63.125	0.12594	-20.5035121836718	-20.5035121836718\\
63.125	0.1296	-22.1937700602801	-22.1937700602801\\
63.125	0.13326	-24.0062596285382	-24.0062596285382\\
63.125	0.13692	-25.9409808884461	-25.9409808884461\\
63.125	0.14058	-27.9979338400038	-27.9979338400038\\
63.125	0.14424	-30.1771184832112	-30.1771184832112\\
63.125	0.1479	-32.4785348180685	-32.4785348180685\\
63.125	0.15156	-34.9021828445755	-34.9021828445755\\
63.125	0.15522	-37.4480625627324	-37.4480625627324\\
63.125	0.15888	-40.116173972539	-40.116173972539\\
63.125	0.16254	-42.9065170739955	-42.9065170739955\\
63.125	0.1662	-45.8190918671017	-45.8190918671017\\
63.125	0.16986	-48.8538983518577	-48.8538983518577\\
63.125	0.17352	-52.0109365282636	-52.0109365282636\\
63.125	0.17718	-55.2902063963191	-55.2902063963191\\
63.125	0.18084	-58.6917079560245	-58.6917079560245\\
63.125	0.1845	-62.2154412073798	-62.2154412073798\\
63.125	0.18816	-65.8614061503848	-65.8614061503848\\
63.125	0.19182	-69.6296027850395	-69.6296027850395\\
63.125	0.19548	-73.5200311113441	-73.5200311113441\\
63.125	0.19914	-77.5326911292985	-77.5326911292985\\
63.125	0.2028	-81.6675828389027	-81.6675828389027\\
63.125	0.20646	-85.9247062401567	-85.9247062401567\\
63.125	0.21012	-90.3040613330604	-90.3040613330604\\
63.125	0.21378	-94.805648117614	-94.805648117614\\
63.125	0.21744	-99.4294665938174	-99.4294665938174\\
63.125	0.2211	-104.175516761671	-104.175516761671\\
63.125	0.22476	-109.043798621173	-109.043798621173\\
63.125	0.22842	-114.034312172326	-114.034312172326\\
63.125	0.23208	-119.147057415129	-119.147057415129\\
63.125	0.23574	-124.382034349581	-124.382034349581\\
63.125	0.2394	-129.739242975683	-129.739242975683\\
63.125	0.24306	-135.218683293435	-135.218683293435\\
63.125	0.24672	-140.820355302837	-140.820355302837\\
63.125	0.25038	-146.544259003888	-146.544259003888\\
63.125	0.25404	-152.39039439659	-152.39039439659\\
63.125	0.2577	-158.358761480941	-158.358761480941\\
63.125	0.26136	-164.449360256942	-164.449360256942\\
63.125	0.26502	-170.662190724592	-170.662190724592\\
63.125	0.26868	-176.997252883893	-176.997252883893\\
63.125	0.27234	-183.454546734843	-183.454546734843\\
63.125	0.276	-190.034072277443	-190.034072277443\\
63.5	0.093	-10.8816156469619	-10.8816156469619\\
63.5	0.09666	-11.4745671185039	-11.4745671185039\\
63.5	0.10032	-12.1897502816959	-12.1897502816959\\
63.5	0.10398	-13.0271651365377	-13.0271651365377\\
63.5	0.10764	-13.9868116830292	-13.9868116830292\\
63.5	0.1113	-15.0686899211705	-15.0686899211705\\
63.5	0.11496	-16.2727998509616	-16.2727998509616\\
63.5	0.11862	-17.5991414724026	-17.5991414724026\\
63.5	0.12228	-19.0477147854933	-19.0477147854933\\
63.5	0.12594	-20.6185197902338	-20.6185197902338\\
63.5	0.1296	-22.3115564866241	-22.3115564866241\\
63.5	0.13326	-24.1268248746641	-24.1268248746641\\
63.5	0.13692	-26.064324954354	-26.064324954354\\
63.5	0.14058	-28.1240567256937	-28.1240567256937\\
63.5	0.14424	-30.3060201886832	-30.3060201886832\\
63.5	0.1479	-32.6102153433225	-32.6102153433225\\
63.5	0.15156	-35.0366421896115	-35.0366421896115\\
63.5	0.15522	-37.5853007275504	-37.5853007275504\\
63.5	0.15888	-40.2561909571391	-40.2561909571391\\
63.5	0.16254	-43.0493128783775	-43.0493128783775\\
63.5	0.1662	-45.9646664912657	-45.9646664912657\\
63.5	0.16986	-49.0022517958038	-49.0022517958038\\
63.5	0.17352	-52.1620687919916	-52.1620687919916\\
63.5	0.17718	-55.4441174798292	-55.4441174798292\\
63.5	0.18084	-58.8483978593166	-58.8483978593166\\
63.5	0.1845	-62.3749099304538	-62.3749099304538\\
63.5	0.18816	-66.0236536932408	-66.0236536932408\\
63.5	0.19182	-69.7946291476776	-69.7946291476776\\
63.5	0.19548	-73.6878362937642	-73.6878362937642\\
63.5	0.19914	-77.7032751315006	-77.7032751315006\\
63.5	0.2028	-81.8409456608868	-81.8409456608868\\
63.5	0.20646	-86.1008478819228	-86.1008478819228\\
63.5	0.21012	-90.4829817946085	-90.4829817946085\\
63.5	0.21378	-94.9873473989441	-94.9873473989441\\
63.5	0.21744	-99.6139446949295	-99.6139446949295\\
63.5	0.2211	-104.362773682565	-104.362773682565\\
63.5	0.22476	-109.23383436185	-109.23383436185\\
63.5	0.22842	-114.227126732784	-114.227126732784\\
63.5	0.23208	-119.342650795369	-119.342650795369\\
63.5	0.23574	-124.580406549603	-124.580406549603\\
63.5	0.2394	-129.940393995487	-129.940393995487\\
63.5	0.24306	-135.422613133021	-135.422613133021\\
63.5	0.24672	-141.027063962205	-141.027063962205\\
63.5	0.25038	-146.753746483039	-146.753746483039\\
63.5	0.25404	-152.602660695522	-152.602660695522\\
63.5	0.2577	-158.573806599655	-158.573806599655\\
63.5	0.26136	-164.667184195438	-164.667184195438\\
63.5	0.26502	-170.88279348287	-170.88279348287\\
63.5	0.26868	-177.220634461953	-177.220634461953\\
63.5	0.27234	-183.680707132685	-183.680707132685\\
63.5	0.276	-190.263011495067	-190.263011495067\\
63.875	0.093	-10.9812102504032	-10.9812102504032\\
63.875	0.09666	-11.5769405417274	-11.5769405417274\\
63.875	0.10032	-12.2949025247013	-12.2949025247013\\
63.875	0.10398	-13.1350961993251	-13.1350961993251\\
63.875	0.10764	-14.0975215655986	-14.0975215655986\\
63.875	0.1113	-15.182178623522	-15.182178623522\\
63.875	0.11496	-16.3890673730951	-16.3890673730951\\
63.875	0.11862	-17.718187814318	-17.718187814318\\
63.875	0.12228	-19.1695399471907	-19.1695399471907\\
63.875	0.12594	-20.7431237717133	-20.7431237717133\\
63.875	0.1296	-22.4389392878855	-22.4389392878855\\
63.875	0.13326	-24.2569864957076	-24.2569864957076\\
63.875	0.13692	-26.1972653951795	-26.1972653951795\\
63.875	0.14058	-28.2597759863012	-28.2597759863012\\
63.875	0.14424	-30.4445182690727	-30.4445182690727\\
63.875	0.1479	-32.751492243494	-32.751492243494\\
63.875	0.15156	-35.1806979095651	-35.1806979095651\\
63.875	0.15522	-37.7321352672859	-37.7321352672859\\
63.875	0.15888	-40.4058043166566	-40.4058043166566\\
63.875	0.16254	-43.201705057677	-43.201705057677\\
63.875	0.1662	-46.1198374903473	-46.1198374903473\\
63.875	0.16986	-49.1602016146674	-49.1602016146674\\
63.875	0.17352	-52.3227974306372	-52.3227974306372\\
63.875	0.17718	-55.6076249382568	-55.6076249382568\\
63.875	0.18084	-59.0146841375262	-59.0146841375262\\
63.875	0.1845	-62.5439750284455	-62.5439750284455\\
63.875	0.18816	-66.1954976110145	-66.1954976110145\\
63.875	0.19182	-69.9692518852333	-69.9692518852333\\
63.875	0.19548	-73.8652378511019	-73.8652378511019\\
63.875	0.19914	-77.8834555086203	-77.8834555086203\\
63.875	0.2028	-82.0239048577885	-82.0239048577885\\
63.875	0.20646	-86.2865858986065	-86.2865858986065\\
63.875	0.21012	-90.6714986310743	-90.6714986310743\\
63.875	0.21378	-95.1786430551919	-95.1786430551919\\
63.875	0.21744	-99.8080191709592	-99.8080191709592\\
63.875	0.2211	-104.559626978376	-104.559626978376\\
63.875	0.22476	-109.433466477443	-109.433466477443\\
63.875	0.22842	-114.42953766816	-114.42953766816\\
63.875	0.23208	-119.547840550527	-119.547840550527\\
63.875	0.23574	-124.788375124543	-124.788375124543\\
63.875	0.2394	-130.151141390209	-130.151141390209\\
63.875	0.24306	-135.636139347525	-135.636139347525\\
63.875	0.24672	-141.243368996491	-141.243368996491\\
63.875	0.25038	-146.972830337106	-146.972830337106\\
63.875	0.25404	-152.824523369372	-152.824523369372\\
63.875	0.2577	-158.798448093287	-158.798448093287\\
63.875	0.26136	-164.894604508852	-164.894604508852\\
63.875	0.26502	-171.112992616066	-171.112992616066\\
63.875	0.26868	-177.453612414931	-177.453612414931\\
63.875	0.27234	-183.916463905445	-183.916463905445\\
63.875	0.276	-190.501547087609	-190.501547087609\\
64.25	0.093	-11.090401228762	-11.090401228762\\
64.25	0.09666	-11.6889103398682	-11.6889103398682\\
64.25	0.10032	-12.4096511426241	-12.4096511426241\\
64.25	0.10398	-13.2526236370299	-13.2526236370299\\
64.25	0.10764	-14.2178278230854	-14.2178278230854\\
64.25	0.1113	-15.3052637007908	-15.3052637007908\\
64.25	0.11496	-16.5149312701459	-16.5149312701459\\
64.25	0.11862	-17.8468305311509	-17.8468305311509\\
64.25	0.12228	-19.3009614838056	-19.3009614838056\\
64.25	0.12594	-20.8773241281101	-20.8773241281101\\
64.25	0.1296	-22.5759184640644	-22.5759184640644\\
64.25	0.13326	-24.3967444916685	-24.3967444916685\\
64.25	0.13692	-26.3398022109224	-26.3398022109224\\
64.25	0.14058	-28.4050916218261	-28.4050916218261\\
64.25	0.14424	-30.5926127243796	-30.5926127243796\\
64.25	0.1479	-32.9023655185829	-32.9023655185829\\
64.25	0.15156	-35.334350004436	-35.334350004436\\
64.25	0.15522	-37.8885661819389	-37.8885661819389\\
64.25	0.15888	-40.5650140510916	-40.5650140510916\\
64.25	0.16254	-43.363693611894	-43.363693611894\\
64.25	0.1662	-46.2846048643463	-46.2846048643463\\
64.25	0.16986	-49.3277478084484	-49.3277478084484\\
64.25	0.17352	-52.4931224442002	-52.4931224442002\\
64.25	0.17718	-55.7807287716018	-55.7807287716018\\
64.25	0.18084	-59.1905667906532	-59.1905667906532\\
64.25	0.1845	-62.7226365013544	-62.7226365013544\\
64.25	0.18816	-66.3769379037055	-66.3769379037055\\
64.25	0.19182	-70.1534709977064	-70.1534709977064\\
64.25	0.19548	-74.0522357833569	-74.0522357833569\\
64.25	0.19914	-78.0732322606573	-78.0732322606573\\
64.25	0.2028	-82.2164604296076	-82.2164604296076\\
64.25	0.20646	-86.4819202902076	-86.4819202902076\\
64.25	0.21012	-90.8696118424574	-90.8696118424574\\
64.25	0.21378	-95.379535086357	-95.379535086357\\
64.25	0.21744	-100.011690021906	-100.011690021906\\
64.25	0.2211	-104.766076649105	-104.766076649105\\
64.25	0.22476	-109.642694967954	-109.642694967954\\
64.25	0.22842	-114.641544978453	-114.641544978453\\
64.25	0.23208	-119.762626680602	-119.762626680602\\
64.25	0.23574	-125.0059400744	-125.0059400744\\
64.25	0.2394	-130.371485159848	-130.371485159848\\
64.25	0.24306	-135.859261936946	-135.859261936946\\
64.25	0.24672	-141.469270405694	-141.469270405694\\
64.25	0.25038	-147.201510566092	-147.201510566092\\
64.25	0.25404	-153.055982418139	-153.055982418139\\
64.25	0.2577	-159.032685961836	-159.032685961836\\
64.25	0.26136	-165.131621197183	-165.131621197183\\
64.25	0.26502	-171.352788124179	-171.352788124179\\
64.25	0.26868	-177.696186742826	-177.696186742826\\
64.25	0.27234	-184.161817053122	-184.161817053122\\
64.25	0.276	-190.749679055068	-190.749679055068\\
64.625	0.093	-11.2091885820384	-11.2091885820384\\
64.625	0.09666	-11.8104765129265	-11.8104765129265\\
64.625	0.10032	-12.5339961354646	-12.5339961354646\\
64.625	0.10398	-13.3797474496523	-13.3797474496523\\
64.625	0.10764	-14.3477304554898	-14.3477304554898\\
64.625	0.1113	-15.4379451529772	-15.4379451529772\\
64.625	0.11496	-16.6503915421143	-16.6503915421143\\
64.625	0.11862	-17.9850696229013	-17.9850696229013\\
64.625	0.12228	-19.4419793953381	-19.4419793953381\\
64.625	0.12594	-21.0211208594246	-21.0211208594246\\
64.625	0.1296	-22.7224940151609	-22.7224940151609\\
64.625	0.13326	-24.546098862547	-24.546098862547\\
64.625	0.13692	-26.4919354015829	-26.4919354015829\\
64.625	0.14058	-28.5600036322686	-28.5600036322686\\
64.625	0.14424	-30.7503035546041	-30.7503035546041\\
64.625	0.1479	-33.0628351685894	-33.0628351685894\\
64.625	0.15156	-35.4975984742245	-35.4975984742245\\
64.625	0.15522	-38.0545934715094	-38.0545934715094\\
64.625	0.15888	-40.7338201604441	-40.7338201604441\\
64.625	0.16254	-43.5352785410286	-43.5352785410286\\
64.625	0.1662	-46.4589686132628	-46.4589686132628\\
64.625	0.16986	-49.5048903771469	-49.5048903771469\\
64.625	0.17352	-52.6730438326807	-52.6730438326807\\
64.625	0.17718	-55.9634289798644	-55.9634289798644\\
64.625	0.18084	-59.3760458186978	-59.3760458186978\\
64.625	0.1845	-62.9108943491811	-62.9108943491811\\
64.625	0.18816	-66.5679745713141	-66.5679745713141\\
64.625	0.19182	-70.3472864850969	-70.3472864850969\\
64.625	0.19548	-74.2488300905295	-74.2488300905295\\
64.625	0.19914	-78.272605387612	-78.272605387612\\
64.625	0.2028	-82.4186123763442	-82.4186123763442\\
64.625	0.20646	-86.6868510567262	-86.6868510567262\\
64.625	0.21012	-91.077321428758	-91.077321428758\\
64.625	0.21378	-95.5900234924396	-95.5900234924396\\
64.625	0.21744	-100.224957247771	-100.224957247771\\
64.625	0.2211	-104.982122694752	-104.982122694752\\
64.625	0.22476	-109.861519833383	-109.861519833383\\
64.625	0.22842	-114.863148663664	-114.863148663664\\
64.625	0.23208	-119.987009185594	-119.987009185594\\
64.625	0.23574	-125.233101399175	-125.233101399175\\
64.625	0.2394	-130.601425304405	-130.601425304405\\
64.625	0.24306	-136.091980901285	-136.091980901285\\
64.625	0.24672	-141.704768189815	-141.704768189815\\
64.625	0.25038	-147.439787169994	-147.439787169994\\
64.625	0.25404	-153.297037841824	-153.297037841824\\
64.625	0.2577	-159.276520205303	-159.276520205303\\
64.625	0.26136	-165.378234260432	-165.378234260432\\
64.625	0.26502	-171.60218000721	-171.60218000721\\
64.625	0.26868	-177.948357445639	-177.948357445639\\
64.625	0.27234	-184.416766575717	-184.416766575717\\
64.625	0.276	-191.007407397445	-191.007407397445\\
65	0.093	-11.3375723102322	-11.3375723102322\\
65	0.09666	-11.9416390609023	-11.9416390609023\\
65	0.10032	-12.6679375032224	-12.6679375032224\\
65	0.10398	-13.5164676371921	-13.5164676371921\\
65	0.10764	-14.4872294628117	-14.4872294628117\\
65	0.1113	-15.5802229800811	-15.5802229800811\\
65	0.11496	-16.7954481890002	-16.7954481890002\\
65	0.11862	-18.1329050895691	-18.1329050895691\\
65	0.12228	-19.5925936817879	-19.5925936817879\\
65	0.12594	-21.1745139656565	-21.1745139656565\\
65	0.1296	-22.8786659411747	-22.8786659411747\\
65	0.13326	-24.7050496083429	-24.7050496083429\\
65	0.13692	-26.6536649671608	-26.6536649671608\\
65	0.14058	-28.7245120176285	-28.7245120176285\\
65	0.14424	-30.917590759746	-30.917590759746\\
65	0.1479	-33.2329011935133	-33.2329011935133\\
65	0.15156	-35.6704433189305	-35.6704433189305\\
65	0.15522	-38.2302171359973	-38.2302171359973\\
65	0.15888	-40.912222644714	-40.912222644714\\
65	0.16254	-43.7164598450805	-43.7164598450805\\
65	0.1662	-46.6429287370968	-46.6429287370968\\
65	0.16986	-49.6916293207629	-49.6916293207629\\
65	0.17352	-52.8625615960787	-52.8625615960787\\
65	0.17718	-56.1557255630444	-56.1557255630444\\
65	0.18084	-59.5711212216598	-59.5711212216598\\
65	0.1845	-63.1087485719251	-63.1087485719251\\
65	0.18816	-66.7686076138401	-66.7686076138401\\
65	0.19182	-70.5506983474049	-70.5506983474049\\
65	0.19548	-74.4550207726196	-74.4550207726196\\
65	0.19914	-78.481574889484	-78.481574889484\\
65	0.2028	-82.6303606979983	-82.6303606979983\\
65	0.20646	-86.9013781981623	-86.9013781981623\\
65	0.21012	-91.2946273899761	-91.2946273899761\\
65	0.21378	-95.8101082734396	-95.8101082734396\\
65	0.21744	-100.447820848553	-100.447820848553\\
65	0.2211	-105.207765115316	-105.207765115316\\
65	0.22476	-110.089941073729	-110.089941073729\\
65	0.22842	-115.094348723792	-115.094348723792\\
65	0.23208	-120.220988065505	-120.220988065505\\
65	0.23574	-125.469859098867	-125.469859098867\\
65	0.2394	-130.840961823879	-130.840961823879\\
65	0.24306	-136.334296240541	-136.334296240541\\
65	0.24672	-141.949862348853	-141.949862348853\\
65	0.25038	-147.687660148814	-147.687660148814\\
65	0.25404	-153.547689640426	-153.547689640426\\
65	0.2577	-159.529950823687	-159.529950823687\\
65	0.26136	-165.634443698598	-165.634443698598\\
65	0.26502	-171.861168265158	-171.861168265158\\
65	0.26868	-178.210124523369	-178.210124523369\\
65	0.27234	-184.681312473229	-184.681312473229\\
65	0.276	-191.274732114739	-191.274732114739\\
65.375	0.093	-11.4755524133436	-11.4755524133436\\
65.375	0.09666	-12.0823979837957	-12.0823979837957\\
65.375	0.10032	-12.8114752458977	-12.8114752458977\\
65.375	0.10398	-13.6627841996495	-13.6627841996495\\
65.375	0.10764	-14.6363248450511	-14.6363248450511\\
65.375	0.1113	-15.7320971821025	-15.7320971821025\\
65.375	0.11496	-16.9501012108036	-16.9501012108036\\
65.375	0.11862	-18.2903369311546	-18.2903369311546\\
65.375	0.12228	-19.7528043431554	-19.7528043431554\\
65.375	0.12594	-21.3375034468059	-21.3375034468059\\
65.375	0.1296	-23.0444342421062	-23.0444342421062\\
65.375	0.13326	-24.8735967290564	-24.8735967290564\\
65.375	0.13692	-26.8249909076563	-26.8249909076563\\
65.375	0.14058	-28.898616777906	-28.898616777906\\
65.375	0.14424	-31.0944743398055	-31.0944743398055\\
65.375	0.1479	-33.4125635933548	-33.4125635933548\\
65.375	0.15156	-35.852884538554	-35.852884538554\\
65.375	0.15522	-38.4154371754029	-38.4154371754029\\
65.375	0.15888	-41.1002215039016	-41.1002215039016\\
65.375	0.16254	-43.9072375240501	-43.9072375240501\\
65.375	0.1662	-46.8364852358483	-46.8364852358483\\
65.375	0.16986	-49.8879646392964	-49.8879646392964\\
65.375	0.17352	-53.0616757343943	-53.0616757343943\\
65.375	0.17718	-56.3576185211419	-56.3576185211419\\
65.375	0.18084	-59.7757929995394	-59.7757929995394\\
65.375	0.1845	-63.3161991695866	-63.3161991695866\\
65.375	0.18816	-66.9788370312837	-66.9788370312837\\
65.375	0.19182	-70.7637065846306	-70.7637065846306\\
65.375	0.19548	-74.6708078296272	-74.6708078296272\\
65.375	0.19914	-78.7001407662736	-78.7001407662736\\
65.375	0.2028	-82.8517053945699	-82.8517053945699\\
65.375	0.20646	-87.1255017145159	-87.1255017145159\\
65.375	0.21012	-91.5215297261118	-91.5215297261118\\
65.375	0.21378	-96.0397894293573	-96.0397894293573\\
65.375	0.21744	-100.680280824253	-100.680280824253\\
65.375	0.2211	-105.443003910798	-105.443003910798\\
65.375	0.22476	-110.327958688993	-110.327958688993\\
65.375	0.22842	-115.335145158838	-115.335145158838\\
65.375	0.23208	-120.464563320332	-120.464563320332\\
65.375	0.23574	-125.716213173477	-125.716213173477\\
65.375	0.2394	-131.090094718271	-131.090094718271\\
65.375	0.24306	-136.586207954715	-136.586207954715\\
65.375	0.24672	-142.204552882809	-142.204552882809\\
65.375	0.25038	-147.945129502552	-147.945129502552\\
65.375	0.25404	-153.807937813946	-153.807937813946\\
65.375	0.2577	-159.792977816989	-159.792977816989\\
65.375	0.26136	-165.900249511682	-165.900249511682\\
65.375	0.26502	-172.129752898024	-172.129752898024\\
65.375	0.26868	-178.481487976017	-178.481487976017\\
65.375	0.27234	-184.955454745659	-184.955454745659\\
65.375	0.276	-191.551653206951	-191.551653206951\\
65.75	0.093	-11.6231288913724	-11.6231288913724\\
65.75	0.09666	-12.2327532816065	-12.2327532816065\\
65.75	0.10032	-12.9646093634906	-12.9646093634906\\
65.75	0.10398	-13.8186971370244	-13.8186971370244\\
65.75	0.10764	-14.7950166022079	-14.7950166022079\\
65.75	0.1113	-15.8935677590413	-15.8935677590413\\
65.75	0.11496	-17.1143506075245	-17.1143506075245\\
65.75	0.11862	-18.4573651476575	-18.4573651476575\\
65.75	0.12228	-19.9226113794402	-19.9226113794402\\
65.75	0.12594	-21.5100893028728	-21.5100893028728\\
65.75	0.1296	-23.2197989179551	-23.2197989179551\\
65.75	0.13326	-25.0517402246873	-25.0517402246873\\
65.75	0.13692	-27.0059132230692	-27.0059132230692\\
65.75	0.14058	-29.082317913101	-29.082317913101\\
65.75	0.14424	-31.2809542947825	-31.2809542947825\\
65.75	0.1479	-33.6018223681138	-33.6018223681138\\
65.75	0.15156	-36.0449221330949	-36.0449221330949\\
65.75	0.15522	-38.6102535897258	-38.6102535897258\\
65.75	0.15888	-41.2978167380066	-41.2978167380066\\
65.75	0.16254	-44.107611577937	-44.107611577937\\
65.75	0.1662	-47.0396381095173	-47.0396381095173\\
65.75	0.16986	-50.0938963327474	-50.0938963327474\\
65.75	0.17352	-53.2703862476273	-53.2703862476273\\
65.75	0.17718	-56.569107854157	-56.569107854157\\
65.75	0.18084	-59.9900611523364	-59.9900611523364\\
65.75	0.1845	-63.5332461421657	-63.5332461421657\\
65.75	0.18816	-67.1986628236448	-67.1986628236448\\
65.75	0.19182	-70.9863111967736	-70.9863111967736\\
65.75	0.19548	-74.8961912615523	-74.8961912615523\\
65.75	0.19914	-78.9283030179807	-78.9283030179807\\
65.75	0.2028	-83.082646466059	-83.082646466059\\
65.75	0.20646	-87.359221605787	-87.359221605787\\
65.75	0.21012	-91.7580284371648	-91.7580284371648\\
65.75	0.21378	-96.2790669601924	-96.2790669601924\\
65.75	0.21744	-100.92233717487	-100.92233717487\\
65.75	0.2211	-105.687839081197	-105.687839081197\\
65.75	0.22476	-110.575572679174	-110.575572679174\\
65.75	0.22842	-115.585537968801	-115.585537968801\\
65.75	0.23208	-120.717734950077	-120.717734950077\\
65.75	0.23574	-125.972163623004	-125.972163623004\\
65.75	0.2394	-131.34882398758	-131.34882398758\\
65.75	0.24306	-136.847716043806	-136.847716043806\\
65.75	0.24672	-142.468839791682	-142.468839791682\\
65.75	0.25038	-148.212195231207	-148.212195231207\\
65.75	0.25404	-154.077782362383	-154.077782362383\\
65.75	0.2577	-160.065601185208	-160.065601185208\\
65.75	0.26136	-166.175651699683	-166.175651699683\\
65.75	0.26502	-172.407933905808	-172.407933905808\\
65.75	0.26868	-178.762447803582	-178.762447803582\\
65.75	0.27234	-185.239193393006	-185.239193393006\\
65.75	0.276	-191.83817067408	-191.83817067408\\
66.125	0.093	-11.7803017443187	-11.7803017443187\\
66.125	0.09666	-12.3927049543348	-12.3927049543348\\
66.125	0.10032	-13.1273398560009	-13.1273398560009\\
66.125	0.10398	-13.9842064493166	-13.9842064493166\\
66.125	0.10764	-14.9633047342822	-14.9633047342822\\
66.125	0.1113	-16.0646347108977	-16.0646347108977\\
66.125	0.11496	-17.2881963791628	-17.2881963791628\\
66.125	0.11862	-18.6339897390778	-18.6339897390778\\
66.125	0.12228	-20.1020147906426	-20.1020147906426\\
66.125	0.12594	-21.6922715338572	-21.6922715338572\\
66.125	0.1296	-23.4047599687215	-23.4047599687215\\
66.125	0.13326	-25.2394800952356	-25.2394800952356\\
66.125	0.13692	-27.1964319133996	-27.1964319133996\\
66.125	0.14058	-29.2756154232134	-29.2756154232134\\
66.125	0.14424	-31.4770306246769	-31.4770306246769\\
66.125	0.1479	-33.8006775177902	-33.8006775177902\\
66.125	0.15156	-36.2465561025533	-36.2465561025533\\
66.125	0.15522	-38.8146663789663	-38.8146663789663\\
66.125	0.15888	-41.505008347029	-41.505008347029\\
66.125	0.16254	-44.3175820067415	-44.3175820067415\\
66.125	0.1662	-47.2523873581038	-47.2523873581038\\
66.125	0.16986	-50.3094244011159	-50.3094244011159\\
66.125	0.17352	-53.4886931357777	-53.4886931357777\\
66.125	0.17718	-56.7901935620894	-56.7901935620894\\
66.125	0.18084	-60.2139256800509	-60.2139256800509\\
66.125	0.1845	-63.7598894896622	-63.7598894896622\\
66.125	0.18816	-67.4280849909233	-67.4280849909233\\
66.125	0.19182	-71.2185121838341	-71.2185121838341\\
66.125	0.19548	-75.1311710683948	-75.1311710683948\\
66.125	0.19914	-79.1660616446052	-79.1660616446052\\
66.125	0.2028	-83.3231839124655	-83.3231839124655\\
66.125	0.20646	-87.6025378719756	-87.6025378719756\\
66.125	0.21012	-92.0041235231354	-92.0041235231354\\
66.125	0.21378	-96.527940865945	-96.527940865945\\
66.125	0.21744	-101.173989900404	-101.173989900404\\
66.125	0.2211	-105.942270626514	-105.942270626514\\
66.125	0.22476	-110.832783044273	-110.832783044273\\
66.125	0.22842	-115.845527153681	-115.845527153681\\
66.125	0.23208	-120.98050295474	-120.98050295474\\
66.125	0.23574	-126.237710447448	-126.237710447448\\
66.125	0.2394	-131.617149631807	-131.617149631807\\
66.125	0.24306	-137.118820507815	-137.118820507815\\
66.125	0.24672	-142.742723075472	-142.742723075472\\
66.125	0.25038	-148.48885733478	-148.48885733478\\
66.125	0.25404	-154.357223285737	-154.357223285737\\
66.125	0.2577	-160.347820928345	-160.347820928345\\
66.125	0.26136	-166.460650262602	-166.460650262602\\
66.125	0.26502	-172.695711288508	-172.695711288508\\
66.125	0.26868	-179.053004006065	-179.053004006065\\
66.125	0.27234	-185.532528415271	-185.532528415271\\
66.125	0.276	-192.134284516127	-192.134284516127\\
66.5	0.093	-11.9470709721825	-11.9470709721825\\
66.5	0.09666	-12.5622530019806	-12.5622530019806\\
66.5	0.10032	-13.2996667234287	-13.2996667234287\\
66.5	0.10398	-14.1593121365265	-14.1593121365265\\
66.5	0.10764	-15.1411892412741	-15.1411892412741\\
66.5	0.1113	-16.2452980376715	-16.2452980376715\\
66.5	0.11496	-17.4716385257186	-17.4716385257186\\
66.5	0.11862	-18.8202107054156	-18.8202107054156\\
66.5	0.12228	-20.2910145767624	-20.2910145767624\\
66.5	0.12594	-21.884050139759	-21.884050139759\\
66.5	0.1296	-23.5993173944053	-23.5993173944053\\
66.5	0.13326	-25.4368163407015	-25.4368163407015\\
66.5	0.13692	-27.3965469786475	-27.3965469786475\\
66.5	0.14058	-29.4785093082432	-29.4785093082432\\
66.5	0.14424	-31.6827033294888	-31.6827033294888\\
66.5	0.1479	-34.0091290423841	-34.0091290423841\\
66.5	0.15156	-36.4577864469293	-36.4577864469293\\
66.5	0.15522	-39.0286755431242	-39.0286755431242\\
66.5	0.15888	-41.7217963309689	-41.7217963309689\\
66.5	0.16254	-44.5371488104634	-44.5371488104634\\
66.5	0.1662	-47.4747329816077	-47.4747329816077\\
66.5	0.16986	-50.5345488444019	-50.5345488444019\\
66.5	0.17352	-53.7165963988458	-53.7165963988458\\
66.5	0.17718	-57.0208756449394	-57.0208756449394\\
66.5	0.18084	-60.4473865826829	-60.4473865826829\\
66.5	0.1845	-63.9961292120762	-63.9961292120762\\
66.5	0.18816	-67.6671035331193	-67.6671035331193\\
66.5	0.19182	-71.4603095458122	-71.4603095458122\\
66.5	0.19548	-75.3757472501548	-75.3757472501548\\
66.5	0.19914	-79.4134166461473	-79.4134166461473\\
66.5	0.2028	-83.5733177337896	-83.5733177337896\\
66.5	0.20646	-87.8554505130816	-87.8554505130816\\
66.5	0.21012	-92.2598149840235	-92.2598149840235\\
66.5	0.21378	-96.7864111466151	-96.7864111466151\\
66.5	0.21744	-101.435239000856	-101.435239000856\\
66.5	0.2211	-106.206298546748	-106.206298546748\\
66.5	0.22476	-111.099589784289	-111.099589784289\\
66.5	0.22842	-116.11511271348	-116.11511271348\\
66.5	0.23208	-121.25286733432	-121.25286733432\\
66.5	0.23574	-126.512853646811	-126.512853646811\\
66.5	0.2394	-131.895071650951	-131.895071650951\\
66.5	0.24306	-137.399521346741	-137.399521346741\\
66.5	0.24672	-143.026202734181	-143.026202734181\\
66.5	0.25038	-148.77511581327	-148.77511581327\\
66.5	0.25404	-154.64626058401	-154.64626058401\\
66.5	0.2577	-160.639637046399	-160.639637046399\\
66.5	0.26136	-166.755245200438	-166.755245200438\\
66.5	0.26502	-172.993085046126	-172.993085046126\\
66.5	0.26868	-179.353156583465	-179.353156583465\\
66.5	0.27234	-185.835459812453	-185.835459812453\\
66.5	0.276	-192.439994733091	-192.439994733091\\
66.875	0.093	-12.1234365749637	-12.1234365749637\\
66.875	0.09666	-12.7413974245439	-12.7413974245439\\
66.875	0.10032	-13.4815899657739	-13.4815899657739\\
66.875	0.10398	-14.3440141986537	-14.3440141986537\\
66.875	0.10764	-15.3286701231833	-15.3286701231833\\
66.875	0.1113	-16.4355577393628	-16.4355577393628\\
66.875	0.11496	-17.6646770471919	-17.6646770471919\\
66.875	0.11862	-19.016028046671	-19.016028046671\\
66.875	0.12228	-20.4896107377998	-20.4896107377998\\
66.875	0.12594	-22.0854251205783	-22.0854251205783\\
66.875	0.1296	-23.8034711950067	-23.8034711950067\\
66.875	0.13326	-25.6437489610849	-25.6437489610849\\
66.875	0.13692	-27.6062584188128	-27.6062584188128\\
66.875	0.14058	-29.6909995681906	-29.6909995681906\\
66.875	0.14424	-31.8979724092182	-31.8979724092182\\
66.875	0.1479	-34.2271769418955	-34.2271769418955\\
66.875	0.15156	-36.6786131662226	-36.6786131662226\\
66.875	0.15522	-39.2522810821996	-39.2522810821996\\
66.875	0.15888	-41.9481806898263	-41.9481806898263\\
66.875	0.16254	-44.7663119891029	-44.7663119891029\\
66.875	0.1662	-47.7066749800292	-47.7066749800292\\
66.875	0.16986	-50.7692696626053	-50.7692696626053\\
66.875	0.17352	-53.9540960368312	-53.9540960368312\\
66.875	0.17718	-57.2611541027069	-57.2611541027069\\
66.875	0.18084	-60.6904438602324	-60.6904438602324\\
66.875	0.1845	-64.2419653094076	-64.2419653094076\\
66.875	0.18816	-67.9157184502328	-67.9157184502328\\
66.875	0.19182	-71.7117032827077	-71.7117032827077\\
66.875	0.19548	-75.6299198068323	-75.6299198068323\\
66.875	0.19914	-79.6703680226068	-79.6703680226068\\
66.875	0.2028	-83.8330479300311	-83.8330479300311\\
66.875	0.20646	-88.1179595291051	-88.1179595291051\\
66.875	0.21012	-92.525102819829	-92.525102819829\\
66.875	0.21378	-97.0544778022027	-97.0544778022027\\
66.875	0.21744	-101.706084476226	-101.706084476226\\
66.875	0.2211	-106.479922841899	-106.479922841899\\
66.875	0.22476	-111.375992899222	-111.375992899222\\
66.875	0.22842	-116.394294648195	-116.394294648195\\
66.875	0.23208	-121.534828088818	-121.534828088818\\
66.875	0.23574	-126.79759322109	-126.79759322109\\
66.875	0.2394	-132.182590045012	-132.182590045012\\
66.875	0.24306	-137.689818560584	-137.689818560584\\
66.875	0.24672	-143.319278767806	-143.319278767806\\
66.875	0.25038	-149.070970666678	-149.070970666678\\
66.875	0.25404	-154.944894257199	-154.944894257199\\
66.875	0.2577	-160.94104953937	-160.94104953937\\
66.875	0.26136	-167.059436513191	-167.059436513191\\
66.875	0.26502	-173.300055178662	-173.300055178662\\
66.875	0.26868	-179.662905535783	-179.662905535783\\
66.875	0.27234	-186.147987584553	-186.147987584553\\
66.875	0.276	-192.755301324973	-192.755301324973\\
67.25	0.093	-12.3093985526625	-12.3093985526625\\
67.25	0.09666	-12.9301382220247	-12.9301382220247\\
67.25	0.10032	-13.6731095830367	-13.6731095830367\\
67.25	0.10398	-14.5383126356986	-14.5383126356986\\
67.25	0.10764	-15.5257473800102	-15.5257473800102\\
67.25	0.1113	-16.6354138159716	-16.6354138159716\\
67.25	0.11496	-17.8673119435828	-17.8673119435828\\
67.25	0.11862	-19.2214417628438	-19.2214417628438\\
67.25	0.12228	-20.6978032737546	-20.6978032737546\\
67.25	0.12594	-22.2963964763152	-22.2963964763152\\
67.25	0.1296	-24.0172213705256	-24.0172213705256\\
67.25	0.13326	-25.8602779563858	-25.8602779563858\\
67.25	0.13692	-27.8255662338957	-27.8255662338957\\
67.25	0.14058	-29.9130862030555	-29.9130862030555\\
67.25	0.14424	-32.1228378638651	-32.1228378638651\\
67.25	0.1479	-34.4548212163244	-34.4548212163244\\
67.25	0.15156	-36.9090362604336	-36.9090362604336\\
67.25	0.15522	-39.4854829961925	-39.4854829961925\\
67.25	0.15888	-42.1841614236013	-42.1841614236013\\
67.25	0.16254	-45.0050715426598	-45.0050715426598\\
67.25	0.1662	-47.9482133533681	-47.9482133533681\\
67.25	0.16986	-51.0135868557263	-51.0135868557263\\
67.25	0.17352	-54.2011920497342	-54.2011920497342\\
67.25	0.17718	-57.5110289353919	-57.5110289353919\\
67.25	0.18084	-60.9430975126994	-60.9430975126994\\
67.25	0.1845	-64.4973977816567	-64.4973977816567\\
67.25	0.18816	-68.1739297422638	-68.1739297422638\\
67.25	0.19182	-71.9726933945207	-71.9726933945207\\
67.25	0.19548	-75.8936887384274	-75.8936887384274\\
67.25	0.19914	-79.9369157739839	-79.9369157739839\\
67.25	0.2028	-84.1023745011902	-84.1023745011902\\
67.25	0.20646	-88.3900649200462	-88.3900649200462\\
67.25	0.21012	-92.7999870305521	-92.7999870305521\\
67.25	0.21378	-97.3321408327077	-97.3321408327077\\
67.25	0.21744	-101.986526326513	-101.986526326513\\
67.25	0.2211	-106.763143511968	-106.763143511968\\
67.25	0.22476	-111.661992389073	-111.661992389073\\
67.25	0.22842	-116.683072957828	-116.683072957828\\
67.25	0.23208	-121.826385218233	-121.826385218233\\
67.25	0.23574	-127.091929170287	-127.091929170287\\
67.25	0.2394	-132.479704813992	-132.479704813992\\
67.25	0.24306	-137.989712149346	-137.989712149346\\
67.25	0.24672	-143.621951176349	-143.621951176349\\
67.25	0.25038	-149.376421895003	-149.376421895003\\
67.25	0.25404	-155.253124305306	-155.253124305306\\
67.25	0.2577	-161.25205840726	-161.25205840726\\
67.25	0.26136	-167.373224200863	-167.373224200863\\
67.25	0.26502	-173.616621686115	-173.616621686115\\
67.25	0.26868	-179.982250863018	-179.982250863018\\
67.25	0.27234	-186.47011173157	-186.47011173157\\
67.25	0.276	-193.080204291772	-193.080204291772\\
67.625	0.093	-12.5049569052787	-12.5049569052787\\
67.625	0.09666	-13.1284753944229	-13.1284753944229\\
67.625	0.10032	-13.874225575217	-13.874225575217\\
67.625	0.10398	-14.7422074476608	-14.7422074476608\\
67.625	0.10764	-15.7324210117544	-15.7324210117544\\
67.625	0.1113	-16.8448662674979	-16.8448662674979\\
67.625	0.11496	-18.079543214891	-18.079543214891\\
67.625	0.11862	-19.4364518539341	-19.4364518539341\\
67.625	0.12228	-20.9155921846269	-20.9155921846269\\
67.625	0.12594	-22.5169642069695	-22.5169642069695\\
67.625	0.1296	-24.2405679209619	-24.2405679209619\\
67.625	0.13326	-26.0864033266041	-26.0864033266041\\
67.625	0.13692	-28.054470423896	-28.054470423896\\
67.625	0.14058	-30.1447692128378	-30.1447692128378\\
67.625	0.14424	-32.3572996934294	-32.3572996934294\\
67.625	0.1479	-34.6920618656708	-34.6920618656708\\
67.625	0.15156	-37.149055729562	-37.149055729562\\
67.625	0.15522	-39.7282812851029	-39.7282812851029\\
67.625	0.15888	-42.4297385322937	-42.4297385322937\\
67.625	0.16254	-45.2534274711342	-45.2534274711342\\
67.625	0.1662	-48.1993481016245	-48.1993481016245\\
67.625	0.16986	-51.2675004237647	-51.2675004237647\\
67.625	0.17352	-54.4578844375546	-54.4578844375546\\
67.625	0.17718	-57.7705001429943	-57.7705001429943\\
67.625	0.18084	-61.2053475400838	-61.2053475400838\\
67.625	0.1845	-64.7624266288232	-64.7624266288232\\
67.625	0.18816	-68.4417374092122	-68.4417374092122\\
67.625	0.19182	-72.2432798812511	-72.2432798812511\\
67.625	0.19548	-76.1670540449398	-76.1670540449398\\
67.625	0.19914	-80.2130599002783	-80.2130599002783\\
67.625	0.2028	-84.3812974472666	-84.3812974472666\\
67.625	0.20646	-88.6717666859047	-88.6717666859047\\
67.625	0.21012	-93.0844676161925	-93.0844676161925\\
67.625	0.21378	-97.6194002381302	-97.6194002381302\\
67.625	0.21744	-102.276564551718	-102.276564551718\\
67.625	0.2211	-107.055960556955	-107.055960556955\\
67.625	0.22476	-111.957588253842	-111.957588253842\\
67.625	0.22842	-116.981447642379	-116.981447642379\\
67.625	0.23208	-122.127538722565	-122.127538722565\\
67.625	0.23574	-127.395861494402	-127.395861494402\\
67.625	0.2394	-132.786415957888	-132.786415957888\\
67.625	0.24306	-138.299202113024	-138.299202113024\\
67.625	0.24672	-143.93421995981	-143.93421995981\\
67.625	0.25038	-149.691469498246	-149.691469498246\\
67.625	0.25404	-155.570950728331	-155.570950728331\\
67.625	0.2577	-161.572663650066	-161.572663650066\\
67.625	0.26136	-167.696608263451	-167.696608263451\\
67.625	0.26502	-173.942784568486	-173.942784568486\\
67.625	0.26868	-180.311192565171	-180.311192565171\\
67.625	0.27234	-186.801832253505	-186.801832253505\\
67.625	0.276	-193.414703633489	-193.414703633489\\
68	0.093	-12.7101116328124	-12.7101116328124\\
68	0.09666	-13.3364089417387	-13.3364089417387\\
68	0.10032	-14.0849379423148	-14.0849379423148\\
68	0.10398	-14.9556986345406	-14.9556986345406\\
68	0.10764	-15.9486910184162	-15.9486910184162\\
68	0.1113	-17.0639150939417	-17.0639150939417\\
68	0.11496	-18.3013708611169	-18.3013708611169\\
68	0.11862	-19.6610583199419	-19.6610583199419\\
68	0.12228	-21.1429774704167	-21.1429774704167\\
68	0.12594	-22.7471283125413	-22.7471283125413\\
68	0.1296	-24.4735108463157	-24.4735108463157\\
68	0.13326	-26.3221250717399	-26.3221250717399\\
68	0.13692	-28.2929709888139	-28.2929709888139\\
68	0.14058	-30.3860485975377	-30.3860485975377\\
68	0.14424	-32.6013578979113	-32.6013578979113\\
68	0.1479	-34.9388988899347	-34.9388988899347\\
68	0.15156	-37.3986715736078	-37.3986715736078\\
68	0.15522	-39.9806759489308	-39.9806759489308\\
68	0.15888	-42.6849120159036	-42.6849120159036\\
68	0.16254	-45.5113797745261	-45.5113797745261\\
68	0.1662	-48.4600792247985	-48.4600792247985\\
68	0.16986	-51.5310103667206	-51.5310103667206\\
68	0.17352	-54.7241732002926	-54.7241732002926\\
68	0.17718	-58.0395677255142	-58.0395677255142\\
68	0.18084	-61.4771939423858	-61.4771939423858\\
68	0.1845	-65.0370518509071	-65.0370518509071\\
68	0.18816	-68.7191414510782	-68.7191414510782\\
68	0.19182	-72.5234627428992	-72.5234627428992\\
68	0.19548	-76.4500157263698	-76.4500157263698\\
68	0.19914	-80.4988004014903	-80.4988004014903\\
68	0.2028	-84.6698167682607	-84.6698167682607\\
68	0.20646	-88.9630648266807	-88.9630648266807\\
68	0.21012	-93.3785445767506	-93.3785445767506\\
68	0.21378	-97.9162560184703	-97.9162560184703\\
68	0.21744	-102.57619915184	-102.57619915184\\
68	0.2211	-107.358373976859	-107.358373976859\\
68	0.22476	-112.262780493528	-112.262780493528\\
68	0.22842	-117.289418701847	-117.289418701847\\
68	0.23208	-122.438288601816	-122.438288601816\\
68	0.23574	-127.709390193434	-127.709390193434\\
68	0.2394	-133.102723476702	-133.102723476702\\
68	0.24306	-138.61828845162	-138.61828845162\\
68	0.24672	-144.256085118188	-144.256085118188\\
68	0.25038	-150.016113476406	-150.016113476406\\
68	0.25404	-155.898373526273	-155.898373526273\\
68	0.2577	-161.90286526779	-161.90286526779\\
68	0.26136	-168.029588700957	-168.029588700957\\
68	0.26502	-174.278543825774	-174.278543825774\\
68	0.26868	-180.649730642241	-180.649730642241\\
68	0.27234	-187.143149150357	-187.143149150357\\
68	0.276	-193.758799350123	-193.758799350123\\
68.375	0.093	-12.9248627352637	-12.9248627352637\\
68.375	0.09666	-13.5539388639719	-13.5539388639719\\
68.375	0.10032	-14.30524668433	-14.30524668433\\
68.375	0.10398	-15.1787861963378	-15.1787861963378\\
68.375	0.10764	-16.1745573999954	-16.1745573999954\\
68.375	0.1113	-17.2925602953029	-17.2925602953029\\
68.375	0.11496	-18.5327948822601	-18.5327948822601\\
68.375	0.11862	-19.8952611608671	-19.8952611608671\\
68.375	0.12228	-21.379959131124	-21.379959131124\\
68.375	0.12594	-22.9868887930306	-22.9868887930306\\
68.375	0.1296	-24.716050146587	-24.716050146587\\
68.375	0.13326	-26.5674431917932	-26.5674431917932\\
68.375	0.13692	-28.5410679286492	-28.5410679286492\\
68.375	0.14058	-30.636924357155	-30.636924357155\\
68.375	0.14424	-32.8550124773106	-32.8550124773106\\
68.375	0.1479	-35.195332289116	-35.195332289116\\
68.375	0.15156	-37.6578837925712	-37.6578837925712\\
68.375	0.15522	-40.2426669876761	-40.2426669876761\\
68.375	0.15888	-42.9496818744309	-42.9496818744309\\
68.375	0.16254	-45.7789284528355	-45.7789284528355\\
68.375	0.1662	-48.7304067228898	-48.7304067228898\\
68.375	0.16986	-51.804116684594	-51.804116684594\\
68.375	0.17352	-55.000058337948	-55.000058337948\\
68.375	0.17718	-58.3182316829516	-58.3182316829516\\
68.375	0.18084	-61.7586367196052	-61.7586367196052\\
68.375	0.1845	-65.3212734479085	-65.3212734479085\\
68.375	0.18816	-69.0061418678617	-69.0061418678617\\
68.375	0.19182	-72.8132419794646	-72.8132419794646\\
68.375	0.19548	-76.7425737827173	-76.7425737827173\\
68.375	0.19914	-80.7941372776198	-80.7941372776198\\
68.375	0.2028	-84.9679324641721	-84.9679324641721\\
68.375	0.20646	-89.2639593423742	-89.2639593423742\\
68.375	0.21012	-93.6822179122261	-93.6822179122261\\
68.375	0.21378	-98.2227081737278	-98.2227081737278\\
68.375	0.21744	-102.885430126879	-102.885430126879\\
68.375	0.2211	-107.67038377168	-107.67038377168\\
68.375	0.22476	-112.577569108132	-112.577569108132\\
68.375	0.22842	-117.606986136232	-117.606986136232\\
68.375	0.23208	-122.758634855983	-122.758634855983\\
68.375	0.23574	-128.032515267384	-128.032515267384\\
68.375	0.2394	-133.428627370434	-133.428627370434\\
68.375	0.24306	-138.946971165134	-138.946971165134\\
68.375	0.24672	-144.587546651484	-144.587546651484\\
68.375	0.25038	-150.350353829483	-150.350353829483\\
68.375	0.25404	-156.235392699133	-156.235392699133\\
68.375	0.2577	-162.242663260432	-162.242663260432\\
68.375	0.26136	-168.372165513381	-168.372165513381\\
68.375	0.26502	-174.62389945798	-174.62389945798\\
68.375	0.26868	-180.997865094228	-180.997865094228\\
68.375	0.27234	-187.494062422127	-187.494062422127\\
68.375	0.276	-194.112491441675	-194.112491441675\\
68.75	0.093	-13.1492102126324	-13.1492102126324\\
68.75	0.09666	-13.7810651611226	-13.7810651611226\\
68.75	0.10032	-14.5351518012627	-14.5351518012627\\
68.75	0.10398	-15.4114701330526	-15.4114701330526\\
68.75	0.10764	-16.4100201564922	-16.4100201564922\\
68.75	0.1113	-17.5308018715817	-17.5308018715817\\
68.75	0.11496	-18.7738152783209	-18.7738152783209\\
68.75	0.11862	-20.1390603767099	-20.1390603767099\\
68.75	0.12228	-21.6265371667488	-21.6265371667488\\
68.75	0.12594	-23.2362456484374	-23.2362456484374\\
68.75	0.1296	-24.9681858217758	-24.9681858217758\\
68.75	0.13326	-26.8223576867641	-26.8223576867641\\
68.75	0.13692	-28.7987612434021	-28.7987612434021\\
68.75	0.14058	-30.8973964916899	-30.8973964916899\\
68.75	0.14424	-33.1182634316275	-33.1182634316275\\
68.75	0.1479	-35.4613620632148	-35.4613620632148\\
68.75	0.15156	-37.9266923864521	-37.9266923864521\\
68.75	0.15522	-40.514254401339	-40.514254401339\\
68.75	0.15888	-43.2240481078758	-43.2240481078758\\
68.75	0.16254	-46.0560735060624	-46.0560735060624\\
68.75	0.1662	-49.0103305958988	-49.0103305958988\\
68.75	0.16986	-52.0868193773849	-52.0868193773849\\
68.75	0.17352	-55.2855398505208	-55.2855398505208\\
68.75	0.17718	-58.6064920153066	-58.6064920153066\\
68.75	0.18084	-62.0496758717421	-62.0496758717421\\
68.75	0.1845	-65.6150914198275	-65.6150914198275\\
68.75	0.18816	-69.3027386595626	-69.3027386595626\\
68.75	0.19182	-73.1126175909475	-73.1126175909475\\
68.75	0.19548	-77.0447282139823	-77.0447282139823\\
68.75	0.19914	-81.0990705286668	-81.0990705286668\\
68.75	0.2028	-85.2756445350011	-85.2756445350011\\
68.75	0.20646	-89.5744502329852	-89.5744502329852\\
68.75	0.21012	-93.9954876226191	-93.9954876226191\\
68.75	0.21378	-98.5387567039028	-98.5387567039028\\
68.75	0.21744	-103.204257476836	-103.204257476836\\
68.75	0.2211	-107.99198994142	-107.99198994142\\
68.75	0.22476	-112.901954097653	-112.901954097653\\
68.75	0.22842	-117.934149945535	-117.934149945535\\
68.75	0.23208	-123.088577485068	-123.088577485068\\
68.75	0.23574	-128.365236716251	-128.365236716251\\
68.75	0.2394	-133.764127639083	-133.764127639083\\
68.75	0.24306	-139.285250253565	-139.285250253565\\
68.75	0.24672	-144.928604559697	-144.928604559697\\
68.75	0.25038	-150.694190557478	-150.694190557478\\
68.75	0.25404	-156.58200824691	-156.58200824691\\
68.75	0.2577	-162.592057627991	-162.592057627991\\
68.75	0.26136	-168.724338700722	-168.724338700722\\
68.75	0.26502	-174.978851465103	-174.978851465103\\
68.75	0.26868	-181.355595921133	-181.355595921133\\
68.75	0.27234	-187.854572068814	-187.854572068814\\
68.75	0.276	-194.475779908144	-194.475779908144\\
69.125	0.093	-13.3831540649186	-13.3831540649186\\
69.125	0.09666	-14.0177878331909	-14.0177878331909\\
69.125	0.10032	-14.774653293113	-14.774653293113\\
69.125	0.10398	-15.6537504446848	-15.6537504446848\\
69.125	0.10764	-16.6550792879065	-16.6550792879065\\
69.125	0.1113	-17.778639822778	-17.778639822778\\
69.125	0.11496	-19.0244320492992	-19.0244320492992\\
69.125	0.11862	-20.3924559674702	-20.3924559674702\\
69.125	0.12228	-21.8827115772911	-21.8827115772911\\
69.125	0.12594	-23.4951988787617	-23.4951988787617\\
69.125	0.1296	-25.2299178718822	-25.2299178718822\\
69.125	0.13326	-27.0868685566524	-27.0868685566524\\
69.125	0.13692	-29.0660509330724	-29.0660509330724\\
69.125	0.14058	-31.1674650011422	-31.1674650011422\\
69.125	0.14424	-33.3911107608618	-33.3911107608618\\
69.125	0.1479	-35.7369882122312	-35.7369882122312\\
69.125	0.15156	-38.2050973552504	-38.2050973552504\\
69.125	0.15522	-40.7954381899194	-40.7954381899194\\
69.125	0.15888	-43.5080107162382	-43.5080107162382\\
69.125	0.16254	-46.3428149342068	-46.3428149342068\\
69.125	0.1662	-49.2998508438252	-49.2998508438252\\
69.125	0.16986	-52.3791184450934	-52.3791184450934\\
69.125	0.17352	-55.5806177380113	-55.5806177380113\\
69.125	0.17718	-58.904348722579	-58.904348722579\\
69.125	0.18084	-62.3503113987966	-62.3503113987966\\
69.125	0.1845	-65.9185057666639	-65.9185057666639\\
69.125	0.18816	-69.6089318261811	-69.6089318261811\\
69.125	0.19182	-73.421589577348	-73.421589577348\\
69.125	0.19548	-77.3564790201648	-77.3564790201648\\
69.125	0.19914	-81.4136001546313	-81.4136001546313\\
69.125	0.2028	-85.5929529807476	-85.5929529807476\\
69.125	0.20646	-89.8945374985138	-89.8945374985138\\
69.125	0.21012	-94.3183537079296	-94.3183537079296\\
69.125	0.21378	-98.8644016089954	-98.8644016089954\\
69.125	0.21744	-103.532681201711	-103.532681201711\\
69.125	0.2211	-108.323192486076	-108.323192486076\\
69.125	0.22476	-113.235935462091	-113.235935462091\\
69.125	0.22842	-118.270910129756	-118.270910129756\\
69.125	0.23208	-123.428116489071	-123.428116489071\\
69.125	0.23574	-128.707554540035	-128.707554540035\\
69.125	0.2394	-134.109224282649	-134.109224282649\\
69.125	0.24306	-139.633125716913	-139.633125716913\\
69.125	0.24672	-145.279258842827	-145.279258842827\\
69.125	0.25038	-151.047623660391	-151.047623660391\\
69.125	0.25404	-156.938220169604	-156.938220169604\\
69.125	0.2577	-162.951048370468	-162.951048370468\\
69.125	0.26136	-169.086108262981	-169.086108262981\\
69.125	0.26502	-175.343399847144	-175.343399847144\\
69.125	0.26868	-181.722923122956	-181.722923122956\\
69.125	0.27234	-188.224678090419	-188.224678090419\\
69.125	0.276	-194.848664749531	-194.848664749531\\
69.5	0.093	-13.6266942921223	-13.6266942921223\\
69.5	0.09666	-14.2641068801765	-14.2641068801765\\
69.5	0.10032	-15.0237511598807	-15.0237511598807\\
69.5	0.10398	-15.9056271312345	-15.9056271312345\\
69.5	0.10764	-16.9097347942382	-16.9097347942382\\
69.5	0.1113	-18.0360741488917	-18.0360741488917\\
69.5	0.11496	-19.2846451951949	-19.2846451951949\\
69.5	0.11862	-20.655447933148	-20.655447933148\\
69.5	0.12228	-22.1484823627509	-22.1484823627509\\
69.5	0.12594	-23.7637484840035	-23.7637484840035\\
69.5	0.1296	-25.5012462969059	-25.5012462969059\\
69.5	0.13326	-27.3609758014581	-27.3609758014581\\
69.5	0.13692	-29.3429369976602	-29.3429369976602\\
69.5	0.14058	-31.447129885512	-31.447129885512\\
69.5	0.14424	-33.6735544650136	-33.6735544650136\\
69.5	0.1479	-36.022210736165	-36.022210736165\\
69.5	0.15156	-38.4930986989662	-38.4930986989662\\
69.5	0.15522	-41.0862183534172	-41.0862183534172\\
69.5	0.15888	-43.8015696995181	-43.8015696995181\\
69.5	0.16254	-46.6391527372686	-46.6391527372686\\
69.5	0.1662	-49.598967466669	-49.598967466669\\
69.5	0.16986	-52.6810138877192	-52.6810138877192\\
69.5	0.17352	-55.8852920004192	-55.8852920004192\\
69.5	0.17718	-59.2118018047689	-59.2118018047689\\
69.5	0.18084	-62.6605433007685	-62.6605433007685\\
69.5	0.1845	-66.2315164884178	-66.2315164884178\\
69.5	0.18816	-69.924721367717	-69.924721367717\\
69.5	0.19182	-73.740157938666	-73.740157938666\\
69.5	0.19548	-77.6778262012647	-77.6778262012647\\
69.5	0.19914	-81.7377261555132	-81.7377261555132\\
69.5	0.2028	-85.9198578014115	-85.9198578014115\\
69.5	0.20646	-90.2242211389597	-90.2242211389597\\
69.5	0.21012	-94.6508161681576	-94.6508161681576\\
69.5	0.21378	-99.1996428890053	-99.1996428890053\\
69.5	0.21744	-103.870701301503	-103.870701301503\\
69.5	0.2211	-108.66399140565	-108.66399140565\\
69.5	0.22476	-113.579513201447	-113.579513201447\\
69.5	0.22842	-118.617266688894	-118.617266688894\\
69.5	0.23208	-123.777251867991	-123.777251867991\\
69.5	0.23574	-129.059468738737	-129.059468738737\\
69.5	0.2394	-134.463917301133	-134.463917301133\\
69.5	0.24306	-139.99059755518	-139.99059755518\\
69.5	0.24672	-145.639509500875	-145.639509500875\\
69.5	0.25038	-151.410653138221	-151.410653138221\\
69.5	0.25404	-157.304028467217	-157.304028467217\\
69.5	0.2577	-163.319635487862	-163.319635487862\\
69.5	0.26136	-169.457474200157	-169.457474200157\\
69.5	0.26502	-175.717544604102	-175.717544604102\\
69.5	0.26868	-182.099846699696	-182.099846699696\\
69.5	0.27234	-188.604380486941	-188.604380486941\\
69.5	0.276	-195.231145965835	-195.231145965835\\
69.875	0.093	-13.8798308942435	-13.8798308942435\\
69.875	0.09666	-14.5200223020797	-14.5200223020797\\
69.875	0.10032	-15.2824454015659	-15.2824454015659\\
69.875	0.10398	-16.1671001927017	-16.1671001927017\\
69.875	0.10764	-17.1739866754874	-17.1739866754874\\
69.875	0.1113	-18.3031048499229	-18.3031048499229\\
69.875	0.11496	-19.5544547160082	-19.5544547160082\\
69.875	0.11862	-20.9280362737432	-20.9280362737432\\
69.875	0.12228	-22.4238495231281	-22.4238495231281\\
69.875	0.12594	-24.0418944641628	-24.0418944641628\\
69.875	0.1296	-25.7821710968472	-25.7821710968472\\
69.875	0.13326	-27.6446794211814	-27.6446794211814\\
69.875	0.13692	-29.6294194371655	-29.6294194371655\\
69.875	0.14058	-31.7363911447993	-31.7363911447993\\
69.875	0.14424	-33.965594544083	-33.965594544083\\
69.875	0.1479	-36.3170296350163	-36.3170296350163\\
69.875	0.15156	-38.7906964175996	-38.7906964175996\\
69.875	0.15522	-41.3865948918326	-41.3865948918326\\
69.875	0.15888	-44.1047250577154	-44.1047250577154\\
69.875	0.16254	-46.945086915248	-46.945086915248\\
69.875	0.1662	-49.9076804644304	-49.9076804644304\\
69.875	0.16986	-52.9925057052626	-52.9925057052626\\
69.875	0.17352	-56.1995626377445	-56.1995626377445\\
69.875	0.17718	-59.5288512618763	-59.5288512618763\\
69.875	0.18084	-62.9803715776579	-62.9803715776579\\
69.875	0.1845	-66.5541235850893	-66.5541235850893\\
69.875	0.18816	-70.2501072841704	-70.2501072841704\\
69.875	0.19182	-74.0683226749013	-74.0683226749013\\
69.875	0.19548	-78.0087697572821	-78.0087697572821\\
69.875	0.19914	-82.0714485313127	-82.0714485313127\\
69.875	0.2028	-86.256358996993	-86.256358996993\\
69.875	0.20646	-90.5635011543232	-90.5635011543232\\
69.875	0.21012	-94.992875003303	-94.992875003303\\
69.875	0.21378	-99.5444805439328	-99.5444805439328\\
69.875	0.21744	-104.218317776212	-104.218317776212\\
69.875	0.2211	-109.014386700142	-109.014386700142\\
69.875	0.22476	-113.932687315721	-113.932687315721\\
69.875	0.22842	-118.97321962295	-118.97321962295\\
69.875	0.23208	-124.135983621828	-124.135983621828\\
69.875	0.23574	-129.420979312357	-129.420979312357\\
69.875	0.2394	-134.828206694535	-134.828206694535\\
69.875	0.24306	-140.357665768363	-140.357665768363\\
69.875	0.24672	-146.009356533841	-146.009356533841\\
69.875	0.25038	-151.783278990969	-151.783278990969\\
69.875	0.25404	-157.679433139746	-157.679433139746\\
69.875	0.2577	-163.697818980173	-163.697818980173\\
69.875	0.26136	-169.83843651225	-169.83843651225\\
69.875	0.26502	-176.101285735977	-176.101285735977\\
69.875	0.26868	-182.486366651354	-182.486366651354\\
69.875	0.27234	-188.99367925838	-188.99367925838\\
69.875	0.276	-195.623223557056	-195.623223557056\\
70.25	0.093	-14.1425638712821	-14.1425638712821\\
70.25	0.09666	-14.7855340989004	-14.7855340989004\\
70.25	0.10032	-15.5507360181685	-15.5507360181685\\
70.25	0.10398	-16.4381696290864	-16.4381696290864\\
70.25	0.10764	-17.4478349316541	-17.4478349316541\\
70.25	0.1113	-18.5797319258716	-18.5797319258716\\
70.25	0.11496	-19.8338606117388	-19.8338606117388\\
70.25	0.11862	-21.2102209892559	-21.2102209892559\\
70.25	0.12228	-22.7088130584228	-22.7088130584228\\
70.25	0.12594	-24.3296368192395	-24.3296368192395\\
70.25	0.1296	-26.0726922717059	-26.0726922717059\\
70.25	0.13326	-27.9379794158222	-27.9379794158222\\
70.25	0.13692	-29.9254982515882	-29.9254982515882\\
70.25	0.14058	-32.035248779004	-32.035248779004\\
70.25	0.14424	-34.2672309980697	-34.2672309980697\\
70.25	0.1479	-36.6214449087851	-36.6214449087851\\
70.25	0.15156	-39.0978905111504	-39.0978905111504\\
70.25	0.15522	-41.6965678051653	-41.6965678051653\\
70.25	0.15888	-44.4174767908302	-44.4174767908302\\
70.25	0.16254	-47.2606174681448	-47.2606174681448\\
70.25	0.1662	-50.2259898371092	-50.2259898371092\\
70.25	0.16986	-53.3135938977234	-53.3135938977234\\
70.25	0.17352	-56.5234296499873	-56.5234296499873\\
70.25	0.17718	-59.8554970939011	-59.8554970939011\\
70.25	0.18084	-63.3097962294647	-63.3097962294647\\
70.25	0.1845	-66.8863270566781	-66.8863270566781\\
70.25	0.18816	-70.5850895755413	-70.5850895755413\\
70.25	0.19182	-74.4060837860542	-74.4060837860542\\
70.25	0.19548	-78.3493096882169	-78.3493096882169\\
70.25	0.19914	-82.4147672820295	-82.4147672820295\\
70.25	0.2028	-86.6024565674919	-86.6024565674919\\
70.25	0.20646	-90.9123775446041	-90.9123775446041\\
70.25	0.21012	-95.3445302133659	-95.3445302133659\\
70.25	0.21378	-99.8989145737776	-99.8989145737776\\
70.25	0.21744	-104.575530625839	-104.575530625839\\
70.25	0.2211	-109.374378369551	-109.374378369551\\
70.25	0.22476	-114.295457804912	-114.295457804912\\
70.25	0.22842	-119.338768931923	-119.338768931923\\
70.25	0.23208	-124.504311750583	-124.504311750583\\
70.25	0.23574	-129.792086260894	-129.792086260894\\
70.25	0.2394	-135.202092462854	-135.202092462854\\
70.25	0.24306	-140.734330356464	-140.734330356464\\
70.25	0.24672	-146.388799941724	-146.388799941724\\
70.25	0.25038	-152.165501218634	-152.165501218634\\
70.25	0.25404	-158.064434187193	-158.064434187193\\
70.25	0.2577	-164.085598847402	-164.085598847402\\
70.25	0.26136	-170.228995199262	-170.228995199262\\
70.25	0.26502	-176.49462324277	-176.49462324277\\
70.25	0.26868	-182.882482977929	-182.882482977929\\
70.25	0.27234	-189.392574404737	-189.392574404737\\
70.25	0.276	-196.024897523196	-196.024897523196\\
70.625	0.093	-14.4148932232383	-14.4148932232383\\
70.625	0.09666	-15.0606422706386	-15.0606422706386\\
70.625	0.10032	-15.8286230096887	-15.8286230096887\\
70.625	0.10398	-16.7188354403886	-16.7188354403886\\
70.625	0.10764	-17.7312795627383	-17.7312795627383\\
70.625	0.1113	-18.8659553767379	-18.8659553767379\\
70.625	0.11496	-20.1228628823871	-20.1228628823871\\
70.625	0.11862	-21.5020020796862	-21.5020020796862\\
70.625	0.12228	-23.0033729686351	-23.0033729686351\\
70.625	0.12594	-24.6269755492338	-24.6269755492338\\
70.625	0.1296	-26.3728098214822	-26.3728098214822\\
70.625	0.13326	-28.2408757853805	-28.2408757853805\\
70.625	0.13692	-30.2311734409285	-30.2311734409285\\
70.625	0.14058	-32.3437027881264	-32.3437027881264\\
70.625	0.14424	-34.578463826974	-34.578463826974\\
70.625	0.1479	-36.9354565574715	-36.9354565574715\\
70.625	0.15156	-39.4146809796187	-39.4146809796187\\
70.625	0.15522	-42.0161370934158	-42.0161370934158\\
70.625	0.15888	-44.7398248988626	-44.7398248988626\\
70.625	0.16254	-47.5857443959591	-47.5857443959591\\
70.625	0.1662	-50.5538955847056	-50.5538955847056\\
70.625	0.16986	-53.6442784651018	-53.6442784651018\\
70.625	0.17352	-56.8568930371478	-56.8568930371478\\
70.625	0.17718	-60.1917393008435	-60.1917393008435\\
70.625	0.18084	-63.6488172561891	-63.6488172561891\\
70.625	0.1845	-67.2281269031845	-67.2281269031845\\
70.625	0.18816	-70.9296682418297	-70.9296682418297\\
70.625	0.19182	-74.7534412721247	-74.7534412721247\\
70.625	0.19548	-78.6994459940694	-78.6994459940694\\
70.625	0.19914	-82.767682407664	-82.767682407664\\
70.625	0.2028	-86.9581505129084	-86.9581505129084\\
70.625	0.20646	-91.2708503098026	-91.2708503098026\\
70.625	0.21012	-95.7057817983465	-95.7057817983465\\
70.625	0.21378	-100.26294497854	-100.26294497854\\
70.625	0.21744	-104.942339850384	-104.942339850384\\
70.625	0.2211	-109.743966413877	-109.743966413877\\
70.625	0.22476	-114.66782466902	-114.66782466902\\
70.625	0.22842	-119.713914615813	-119.713914615813\\
70.625	0.23208	-124.882236254256	-124.882236254256\\
70.625	0.23574	-130.172789584348	-130.172789584348\\
70.625	0.2394	-135.585574606091	-135.585574606091\\
70.625	0.24306	-141.120591319483	-141.120591319483\\
70.625	0.24672	-146.777839724525	-146.777839724525\\
70.625	0.25038	-152.557319821216	-152.557319821216\\
70.625	0.25404	-158.459031609558	-158.459031609558\\
70.625	0.2577	-164.482975089549	-164.482975089549\\
70.625	0.26136	-170.62915026119	-170.62915026119\\
70.625	0.26502	-176.897557124481	-176.897557124481\\
70.625	0.26868	-183.288195679422	-183.288195679422\\
70.625	0.27234	-189.801065926012	-189.801065926012\\
70.625	0.276	-196.436167864252	-196.436167864252\\
71	0.093	-14.6968189501119	-14.6968189501119\\
71	0.09666	-15.3453468172942	-15.3453468172942\\
71	0.10032	-16.1161063761264	-16.1161063761264\\
71	0.10398	-17.0090976266083	-17.0090976266083\\
71	0.10764	-18.02432056874	-18.02432056874\\
71	0.1113	-19.1617752025215	-19.1617752025215\\
71	0.11496	-20.4214615279528	-20.4214615279528\\
71	0.11862	-21.8033795450339	-21.8033795450339\\
71	0.12228	-23.3075292537648	-23.3075292537648\\
71	0.12594	-24.9339106541455	-24.9339106541455\\
71	0.1296	-26.6825237461759	-26.6825237461759\\
71	0.13326	-28.5533685298562	-28.5533685298562\\
71	0.13692	-30.5464450051862	-30.5464450051862\\
71	0.14058	-32.6617531721661	-32.6617531721661\\
71	0.14424	-34.8992930307958	-34.8992930307958\\
71	0.1479	-37.2590645810752	-37.2590645810752\\
71	0.15156	-39.7410678230044	-39.7410678230044\\
71	0.15522	-42.3453027565835	-42.3453027565835\\
71	0.15888	-45.0717693818123	-45.0717693818123\\
71	0.16254	-47.9204676986909	-47.9204676986909\\
71	0.1662	-50.8913977072194	-50.8913977072194\\
71	0.16986	-53.9845594073976	-53.9845594073976\\
71	0.17352	-57.1999527992256	-57.1999527992256\\
71	0.17718	-60.5375778827033	-60.5375778827033\\
71	0.18084	-63.997434657831	-63.997434657831\\
71	0.1845	-67.5795231246083	-67.5795231246083\\
71	0.18816	-71.2838432830355	-71.2838432830355\\
71	0.19182	-75.1103951331126	-75.1103951331126\\
71	0.19548	-79.0591786748393	-79.0591786748393\\
71	0.19914	-83.1301939082159	-83.1301939082159\\
71	0.2028	-87.3234408332422	-87.3234408332422\\
71	0.20646	-91.6389194499184	-91.6389194499184\\
71	0.21012	-96.0766297582444	-96.0766297582444\\
71	0.21378	-100.63657175822	-100.63657175822\\
71	0.21744	-105.318745449846	-105.318745449846\\
71	0.2211	-110.123150833121	-110.123150833121\\
71	0.22476	-115.049787908046	-115.049787908046\\
71	0.22842	-120.098656674621	-120.098656674621\\
71	0.23208	-125.269757132846	-125.269757132846\\
71	0.23574	-130.56308928272	-130.56308928272\\
71	0.2394	-135.978653124245	-135.978653124245\\
71	0.24306	-141.516448657419	-141.516448657419\\
71	0.24672	-147.176475882243	-147.176475882243\\
71	0.25038	-152.958734798716	-152.958734798716\\
71	0.25404	-158.86322540684	-158.86322540684\\
71	0.2577	-164.889947706613	-164.889947706613\\
71	0.26136	-171.038901698036	-171.038901698036\\
71	0.26502	-177.310087381109	-177.310087381109\\
71	0.26868	-183.703504755832	-183.703504755832\\
71	0.27234	-190.219153822204	-190.219153822204\\
71	0.276	-196.857034580226	-196.857034580226\\
71.375	0.093	-14.9883410519031	-14.9883410519031\\
71.375	0.09666	-15.6396477388673	-15.6396477388673\\
71.375	0.10032	-16.4131861174815	-16.4131861174815\\
71.375	0.10398	-17.3089561877454	-17.3089561877454\\
71.375	0.10764	-18.3269579496591	-18.3269579496591\\
71.375	0.1113	-19.4671914032227	-19.4671914032227\\
71.375	0.11496	-20.729656548436	-20.729656548436\\
71.375	0.11862	-22.114353385299	-22.114353385299\\
71.375	0.12228	-23.621281913812	-23.621281913812\\
71.375	0.12594	-25.2504421339747	-25.2504421339747\\
71.375	0.1296	-27.0018340457871	-27.0018340457871\\
71.375	0.13326	-28.8754576492494	-28.8754576492494\\
71.375	0.13692	-30.8713129443615	-30.8713129443615\\
71.375	0.14058	-32.9893999311234	-32.9893999311234\\
71.375	0.14424	-35.229718609535	-35.229718609535\\
71.375	0.1479	-37.5922689795965	-37.5922689795965\\
71.375	0.15156	-40.0770510413077	-40.0770510413077\\
71.375	0.15522	-42.6840647946687	-42.6840647946687\\
71.375	0.15888	-45.4133102396796	-45.4133102396796\\
71.375	0.16254	-48.2647873763403	-48.2647873763403\\
71.375	0.1662	-51.2384962046507	-51.2384962046507\\
71.375	0.16986	-54.3344367246109	-54.3344367246109\\
71.375	0.17352	-57.5526089362209	-57.5526089362209\\
71.375	0.17718	-60.8930128394807	-60.8930128394807\\
71.375	0.18084	-64.3556484343903	-64.3556484343903\\
71.375	0.1845	-67.9405157209497	-67.9405157209497\\
71.375	0.18816	-71.6476146991589	-71.6476146991589\\
71.375	0.19182	-75.4769453690179	-75.4769453690179\\
71.375	0.19548	-79.4285077305267	-79.4285077305267\\
71.375	0.19914	-83.5023017836852	-83.5023017836852\\
71.375	0.2028	-87.6983275284936	-87.6983275284936\\
71.375	0.20646	-92.0165849649518	-92.0165849649518\\
71.375	0.21012	-96.4570740930598	-96.4570740930598\\
71.375	0.21378	-101.019794912817	-101.019794912817\\
71.375	0.21744	-105.704747424225	-105.704747424225\\
71.375	0.2211	-110.511931627282	-110.511931627282\\
71.375	0.22476	-115.44134752199	-115.44134752199\\
71.375	0.22842	-120.492995108346	-120.492995108346\\
71.375	0.23208	-125.666874386353	-125.666874386353\\
71.375	0.23574	-130.96298535601	-130.96298535601\\
71.375	0.2394	-136.381328017316	-136.381328017316\\
71.375	0.24306	-141.921902370272	-141.921902370272\\
71.375	0.24672	-147.584708414878	-147.584708414878\\
71.375	0.25038	-153.369746151134	-153.369746151134\\
71.375	0.25404	-159.277015579039	-159.277015579039\\
71.375	0.2577	-165.306516698595	-165.306516698595\\
71.375	0.26136	-171.4582495098	-171.4582495098\\
71.375	0.26502	-177.732214012654	-177.732214012654\\
71.375	0.26868	-184.128410207159	-184.128410207159\\
71.375	0.27234	-190.646838093314	-190.646838093314\\
71.375	0.276	-197.287497671118	-197.287497671118\\
71.75	0.093	-15.2894595286117	-15.2894595286117\\
71.75	0.09666	-15.943545035358	-15.943545035358\\
71.75	0.10032	-16.7198622337541	-16.7198622337541\\
71.75	0.10398	-17.6184111238001	-17.6184111238001\\
71.75	0.10764	-18.6391917054958	-18.6391917054958\\
71.75	0.1113	-19.7822039788413	-19.7822039788413\\
71.75	0.11496	-21.0474479438366	-21.0474479438366\\
71.75	0.11862	-22.4349236004817	-22.4349236004817\\
71.75	0.12228	-23.9446309487766	-23.9446309487766\\
71.75	0.12594	-25.5765699887213	-25.5765699887213\\
71.75	0.1296	-27.3307407203158	-27.3307407203158\\
71.75	0.13326	-29.2071431435601	-29.2071431435601\\
71.75	0.13692	-31.2057772584542	-31.2057772584542\\
71.75	0.14058	-33.3266430649981	-33.3266430649981\\
71.75	0.14424	-35.5697405631917	-35.5697405631917\\
71.75	0.1479	-37.9350697530352	-37.9350697530352\\
71.75	0.15156	-40.4226306345285	-40.4226306345285\\
71.75	0.15522	-43.0324232076715	-43.0324232076715\\
71.75	0.15888	-45.7644474724644	-45.7644474724644\\
71.75	0.16254	-48.618703428907	-48.618703428907\\
71.75	0.1662	-51.5951910769994	-51.5951910769994\\
71.75	0.16986	-54.6939104167417	-54.6939104167417\\
71.75	0.17352	-57.9148614481337	-57.9148614481337\\
71.75	0.17718	-61.2580441711755	-61.2580441711755\\
71.75	0.18084	-64.7234585858671	-64.7234585858671\\
71.75	0.1845	-68.3111046922086	-68.3111046922086\\
71.75	0.18816	-72.0209824901997	-72.0209824901997\\
71.75	0.19182	-75.8530919798407	-75.8530919798407\\
71.75	0.19548	-79.8074331611315	-79.8074331611315\\
71.75	0.19914	-83.8840060340721	-83.8840060340721\\
71.75	0.2028	-88.0828105986625	-88.0828105986625\\
71.75	0.20646	-92.4038468549027	-92.4038468549027\\
71.75	0.21012	-96.8471148027926	-96.8471148027926\\
71.75	0.21378	-101.412614442332	-101.412614442332\\
71.75	0.21744	-106.100345773522	-106.100345773522\\
71.75	0.2211	-110.910308796361	-110.910308796361\\
71.75	0.22476	-115.842503510851	-115.842503510851\\
71.75	0.22842	-120.896929916989	-120.896929916989\\
71.75	0.23208	-126.073588014778	-126.073588014778\\
71.75	0.23574	-131.372477804217	-131.372477804217\\
71.75	0.2394	-136.793599285305	-136.793599285305\\
71.75	0.24306	-142.336952458043	-142.336952458043\\
71.75	0.24672	-148.002537322431	-148.002537322431\\
71.75	0.25038	-153.790353878469	-153.790353878469\\
71.75	0.25404	-159.700402126156	-159.700402126156\\
71.75	0.2577	-165.732682065494	-165.732682065494\\
71.75	0.26136	-171.887193696481	-171.887193696481\\
71.75	0.26502	-178.163937019117	-178.163937019117\\
71.75	0.26868	-184.562912033404	-184.562912033404\\
71.75	0.27234	-191.084118739341	-191.084118739341\\
71.75	0.276	-197.727557136927	-197.727557136927\\
72.125	0.093	-15.6001743802378	-15.6001743802378\\
72.125	0.09666	-16.2570387067661	-16.2570387067661\\
72.125	0.10032	-17.0361347249443	-17.0361347249443\\
72.125	0.10398	-17.9374624347722	-17.9374624347722\\
72.125	0.10764	-18.9610218362499	-18.9610218362499\\
72.125	0.1113	-20.1068129293775	-20.1068129293775\\
72.125	0.11496	-21.3748357141548	-21.3748357141548\\
72.125	0.11862	-22.7650901905819	-22.7650901905819\\
72.125	0.12228	-24.2775763586589	-24.2775763586589\\
72.125	0.12594	-25.9122942183856	-25.9122942183856\\
72.125	0.1296	-27.669243769762	-27.669243769762\\
72.125	0.13326	-29.5484250127883	-29.5484250127883\\
72.125	0.13692	-31.5498379474644	-31.5498379474644\\
72.125	0.14058	-33.6734825737903	-33.6734825737903\\
72.125	0.14424	-35.919358891766	-35.919358891766\\
72.125	0.1479	-38.2874669013915	-38.2874669013915\\
72.125	0.15156	-40.7778066026668	-40.7778066026668\\
72.125	0.15522	-43.3903779955918	-43.3903779955918\\
72.125	0.15888	-46.1251810801667	-46.1251810801667\\
72.125	0.16254	-48.9822158563913	-48.9822158563913\\
72.125	0.1662	-51.9614823242658	-51.9614823242658\\
72.125	0.16986	-55.06298048379	-55.06298048379\\
72.125	0.17352	-58.2867103349641	-58.2867103349641\\
72.125	0.17718	-61.6326718777879	-61.6326718777879\\
72.125	0.18084	-65.1008651122615	-65.1008651122615\\
72.125	0.1845	-68.6912900383849	-68.6912900383849\\
72.125	0.18816	-72.4039466561581	-72.4039466561581\\
72.125	0.19182	-76.2388349655812	-76.2388349655812\\
72.125	0.19548	-80.1959549666539	-80.1959549666539\\
72.125	0.19914	-84.2753066593765	-84.2753066593765\\
72.125	0.2028	-88.4768900437489	-88.4768900437489\\
72.125	0.20646	-92.8007051197711	-92.8007051197711\\
72.125	0.21012	-97.2467518874431	-97.2467518874431\\
72.125	0.21378	-101.815030346765	-101.815030346765\\
72.125	0.21744	-106.505540497736	-106.505540497736\\
72.125	0.2211	-111.318282340358	-111.318282340358\\
72.125	0.22476	-116.253255874629	-116.253255874629\\
72.125	0.22842	-121.31046110055	-121.31046110055\\
72.125	0.23208	-126.489898018121	-126.489898018121\\
72.125	0.23574	-131.791566627341	-131.791566627341\\
72.125	0.2394	-137.215466928211	-137.215466928211\\
72.125	0.24306	-142.761598920732	-142.761598920732\\
72.125	0.24672	-148.429962604902	-148.429962604902\\
72.125	0.25038	-154.220557980721	-154.220557980721\\
72.125	0.25404	-160.133385048191	-160.133385048191\\
72.125	0.2577	-166.16844380731	-166.16844380731\\
72.125	0.26136	-172.325734258079	-172.325734258079\\
72.125	0.26502	-178.605256400498	-178.605256400498\\
72.125	0.26868	-185.007010234567	-185.007010234567\\
72.125	0.27234	-191.530995760285	-191.530995760285\\
72.125	0.276	-198.177212977653	-198.177212977653\\
72.5	0.093	-15.9204856067813	-15.9204856067813\\
72.5	0.09666	-16.5801287530917	-16.5801287530917\\
72.5	0.10032	-17.3620035910519	-17.3620035910519\\
72.5	0.10398	-18.2661101206618	-18.2661101206618\\
72.5	0.10764	-19.2924483419215	-19.2924483419215\\
72.5	0.1113	-20.4410182548311	-20.4410182548311\\
72.5	0.11496	-21.7118198593904	-21.7118198593904\\
72.5	0.11862	-23.1048531555996	-23.1048531555996\\
72.5	0.12228	-24.6201181434585	-24.6201181434585\\
72.5	0.12594	-26.2576148229672	-26.2576148229672\\
72.5	0.1296	-28.0173431941257	-28.0173431941257\\
72.5	0.13326	-29.899303256934	-29.899303256934\\
72.5	0.13692	-31.9034950113921	-31.9034950113921\\
72.5	0.14058	-34.0299184575	-34.0299184575\\
72.5	0.14424	-36.2785735952577	-36.2785735952577\\
72.5	0.1479	-38.6494604246652	-38.6494604246652\\
72.5	0.15156	-41.1425789457224	-41.1425789457224\\
72.5	0.15522	-43.7579291584295	-43.7579291584295\\
72.5	0.15888	-46.4955110627864	-46.4955110627864\\
72.5	0.16254	-49.355324658793	-49.355324658793\\
72.5	0.1662	-52.3373699464495	-52.3373699464495\\
72.5	0.16986	-55.4416469257558	-55.4416469257558\\
72.5	0.17352	-58.6681555967118	-58.6681555967118\\
72.5	0.17718	-62.0168959593176	-62.0168959593176\\
72.5	0.18084	-65.4878680135732	-65.4878680135732\\
72.5	0.1845	-69.0810717594787	-69.0810717594787\\
72.5	0.18816	-72.7965071970339	-72.7965071970339\\
72.5	0.19182	-76.6341743262389	-76.6341743262389\\
72.5	0.19548	-80.5940731470937	-80.5940731470937\\
72.5	0.19914	-84.6762036595983	-84.6762036595983\\
72.5	0.2028	-88.8805658637528	-88.8805658637528\\
72.5	0.20646	-93.207159759557	-93.207159759557\\
72.5	0.21012	-97.655985347011	-97.655985347011\\
72.5	0.21378	-102.227042626115	-102.227042626115\\
72.5	0.21744	-106.920331596868	-106.920331596868\\
72.5	0.2211	-111.735852259272	-111.735852259272\\
72.5	0.22476	-116.673604613325	-116.673604613325\\
72.5	0.22842	-121.733588659028	-121.733588659028\\
72.5	0.23208	-126.91580439638	-126.91580439638\\
72.5	0.23574	-132.220251825383	-132.220251825383\\
72.5	0.2394	-137.646930946035	-137.646930946035\\
72.5	0.24306	-143.195841758337	-143.195841758337\\
72.5	0.24672	-148.866984262289	-148.866984262289\\
72.5	0.25038	-154.660358457891	-154.660358457891\\
72.5	0.25404	-160.575964345143	-160.575964345143\\
72.5	0.2577	-166.613801924044	-166.613801924044\\
72.5	0.26136	-172.773871194595	-172.773871194595\\
72.5	0.26502	-179.056172156796	-179.056172156796\\
72.5	0.26868	-185.460704810647	-185.460704810647\\
72.5	0.27234	-191.987469156147	-191.987469156147\\
72.5	0.276	-198.636465193297	-198.636465193297\\
72.875	0.093	-16.2503932082425	-16.2503932082425\\
72.875	0.09666	-16.9128151743348	-16.9128151743348\\
72.875	0.10032	-17.697468832077	-17.697468832077\\
72.875	0.10398	-18.604354181469	-18.604354181469\\
72.875	0.10764	-19.6334712225107	-19.6334712225107\\
72.875	0.1113	-20.7848199552023	-20.7848199552023\\
72.875	0.11496	-22.0584003795436	-22.0584003795436\\
72.875	0.11862	-23.4542124955347	-23.4542124955347\\
72.875	0.12228	-24.9722563031757	-24.9722563031757\\
72.875	0.12594	-26.6125318024664	-26.6125318024664\\
72.875	0.1296	-28.3750389934069	-28.3750389934069\\
72.875	0.13326	-30.2597778759972	-30.2597778759972\\
72.875	0.13692	-32.2667484502373	-32.2667484502373\\
72.875	0.14058	-34.3959507161272	-34.3959507161272\\
72.875	0.14424	-36.647384673667	-36.647384673667\\
72.875	0.1479	-39.0210503228564	-39.0210503228564\\
72.875	0.15156	-41.5169476636957	-41.5169476636957\\
72.875	0.15522	-44.1350766961848	-44.1350766961848\\
72.875	0.15888	-46.8754374203237	-46.8754374203237\\
72.875	0.16254	-49.7380298361124	-49.7380298361124\\
72.875	0.1662	-52.7228539435508	-52.7228539435508\\
72.875	0.16986	-55.8299097426391	-55.8299097426391\\
72.875	0.17352	-59.0591972333771	-59.0591972333771\\
72.875	0.17718	-62.410716415765	-62.410716415765\\
72.875	0.18084	-65.8844672898026	-65.8844672898026\\
72.875	0.1845	-69.48044985549	-69.48044985549\\
72.875	0.18816	-73.1986641128273	-73.1986641128273\\
72.875	0.19182	-77.0391100618143	-77.0391100618143\\
72.875	0.19548	-81.0017877024511	-81.0017877024511\\
72.875	0.19914	-85.0866970347377	-85.0866970347377\\
72.875	0.2028	-89.2938380586742	-89.2938380586742\\
72.875	0.20646	-93.6232107742604	-93.6232107742604\\
72.875	0.21012	-98.0748151814963	-98.0748151814963\\
72.875	0.21378	-102.648651280382	-102.648651280382\\
72.875	0.21744	-107.344719070918	-107.344719070918\\
72.875	0.2211	-112.163018553103	-112.163018553103\\
72.875	0.22476	-117.103549726938	-117.103549726938\\
72.875	0.22842	-122.166312592423	-122.166312592423\\
72.875	0.23208	-127.351307149558	-127.351307149558\\
72.875	0.23574	-132.658533398343	-132.658533398343\\
72.875	0.2394	-138.087991338777	-138.087991338777\\
72.875	0.24306	-143.639680970861	-143.639680970861\\
72.875	0.24672	-149.313602294595	-149.313602294595\\
72.875	0.25038	-155.109755309979	-155.109755309979\\
72.875	0.25404	-161.028140017012	-161.028140017012\\
72.875	0.2577	-167.068756415696	-167.068756415696\\
72.875	0.26136	-173.231604506029	-173.231604506029\\
72.875	0.26502	-179.516684288012	-179.516684288012\\
72.875	0.26868	-185.923995761644	-185.923995761644\\
72.875	0.27234	-192.453538926927	-192.453538926927\\
72.875	0.276	-199.105313783859	-199.105313783859\\
73.25	0.093	-16.5898971846211	-16.5898971846211\\
73.25	0.09666	-17.2550979704954	-17.2550979704954\\
73.25	0.10032	-18.0425304480196	-18.0425304480196\\
73.25	0.10398	-18.9521946171936	-18.9521946171936\\
73.25	0.10764	-19.9840904780173	-19.9840904780173\\
73.25	0.1113	-21.1382180304909	-21.1382180304909\\
73.25	0.11496	-22.4145772746142	-22.4145772746142\\
73.25	0.11862	-23.8131682103874	-23.8131682103874\\
73.25	0.12228	-25.3339908378104	-25.3339908378104\\
73.25	0.12594	-26.9770451568831	-26.9770451568831\\
73.25	0.1296	-28.7423311676056	-28.7423311676056\\
73.25	0.13326	-30.6298488699779	-30.6298488699779\\
73.25	0.13692	-32.6395982640001	-32.6395982640001\\
73.25	0.14058	-34.771579349672	-34.771579349672\\
73.25	0.14424	-37.0257921269937	-37.0257921269937\\
73.25	0.1479	-39.4022365959652	-39.4022365959652\\
73.25	0.15156	-41.9009127565865	-41.9009127565865\\
73.25	0.15522	-44.5218206088576	-44.5218206088576\\
73.25	0.15888	-47.2649601527785	-47.2649601527785\\
73.25	0.16254	-50.1303313883491	-50.1303313883491\\
73.25	0.1662	-53.1179343155696	-53.1179343155696\\
73.25	0.16986	-56.2277689344399	-56.2277689344399\\
73.25	0.17352	-59.4598352449599	-59.4598352449599\\
73.25	0.17718	-62.8141332471298	-62.8141332471298\\
73.25	0.18084	-66.2906629409494	-66.2906629409494\\
73.25	0.1845	-69.8894243264188	-69.8894243264188\\
73.25	0.18816	-73.6104174035381	-73.6104174035381\\
73.25	0.19182	-77.4536421723072	-77.4536421723072\\
73.25	0.19548	-81.419098632726	-81.419098632726\\
73.25	0.19914	-85.5067867847946	-85.5067867847946\\
73.25	0.2028	-89.716706628513	-89.716706628513\\
73.25	0.20646	-94.0488581638812	-94.0488581638812\\
73.25	0.21012	-98.5032413908993	-98.5032413908993\\
73.25	0.21378	-103.079856309567	-103.079856309567\\
73.25	0.21744	-107.778702919885	-107.778702919885\\
73.25	0.2211	-112.599781221852	-112.599781221852\\
73.25	0.22476	-117.543091215469	-117.543091215469\\
73.25	0.22842	-122.608632900736	-122.608632900736\\
73.25	0.23208	-127.796406277653	-127.796406277653\\
73.25	0.23574	-133.10641134622	-133.10641134622\\
73.25	0.2394	-138.538648106436	-138.538648106436\\
73.25	0.24306	-144.093116558302	-144.093116558302\\
73.25	0.24672	-149.769816701818	-149.769816701818\\
73.25	0.25038	-155.568748536984	-155.568748536984\\
73.25	0.25404	-161.489912063799	-161.489912063799\\
73.25	0.2577	-167.533307282265	-167.533307282265\\
73.25	0.26136	-173.69893419238	-173.69893419238\\
73.25	0.26502	-179.986792794145	-179.986792794145\\
73.25	0.26868	-186.396883087559	-186.396883087559\\
73.25	0.27234	-192.929205072624	-192.929205072624\\
73.25	0.276	-199.583758749338	-199.583758749338\\
73.625	0.093	-16.9389975359171	-16.9389975359171\\
73.625	0.09666	-17.6069771415735	-17.6069771415735\\
73.625	0.10032	-18.3971884388797	-18.3971884388797\\
73.625	0.10398	-19.3096314278356	-19.3096314278356\\
73.625	0.10764	-20.3443061084414	-20.3443061084414\\
73.625	0.1113	-21.501212480697	-21.501212480697\\
73.625	0.11496	-22.7803505446023	-22.7803505446023\\
73.625	0.11862	-24.1817203001575	-24.1817203001575\\
73.625	0.12228	-25.7053217473625	-25.7053217473625\\
73.625	0.12594	-27.3511548862172	-27.3511548862172\\
73.625	0.1296	-29.1192197167217	-29.1192197167217\\
73.625	0.13326	-31.0095162388761	-31.0095162388761\\
73.625	0.13692	-33.0220444526802	-33.0220444526802\\
73.625	0.14058	-35.1568043581341	-35.1568043581341\\
73.625	0.14424	-37.4137959552378	-37.4137959552378\\
73.625	0.1479	-39.7930192439913	-39.7930192439913\\
73.625	0.15156	-42.2944742243947	-42.2944742243947\\
73.625	0.15522	-44.9181608964478	-44.9181608964478\\
73.625	0.15888	-47.6640792601506	-47.6640792601506\\
73.625	0.16254	-50.5322293155033	-50.5322293155033\\
73.625	0.1662	-53.5226110625058	-53.5226110625058\\
73.625	0.16986	-56.6352245011581	-56.6352245011581\\
73.625	0.17352	-59.8700696314602	-59.8700696314602\\
73.625	0.17718	-63.227146453412	-63.227146453412\\
73.625	0.18084	-66.7064549670137	-66.7064549670137\\
73.625	0.1845	-70.3079951722651	-70.3079951722651\\
73.625	0.18816	-74.0317670691664	-74.0317670691664\\
73.625	0.19182	-77.8777706577174	-77.8777706577174\\
73.625	0.19548	-81.8460059379182	-81.8460059379182\\
73.625	0.19914	-85.9364729097689	-85.9364729097689\\
73.625	0.2028	-90.1491715732693	-90.1491715732693\\
73.625	0.20646	-94.4841019284196	-94.4841019284196\\
73.625	0.21012	-98.9412639752196	-98.9412639752196\\
73.625	0.21378	-103.520657713669	-103.520657713669\\
73.625	0.21744	-108.222283143769	-108.222283143769\\
73.625	0.2211	-113.046140265518	-113.046140265518\\
73.625	0.22476	-117.992229078917	-117.992229078917\\
73.625	0.22842	-123.060549583967	-123.060549583967\\
73.625	0.23208	-128.251101780665	-128.251101780665\\
73.625	0.23574	-133.563885669014	-133.563885669014\\
73.625	0.2394	-138.998901249012	-138.998901249012\\
73.625	0.24306	-144.55614852066	-144.55614852066\\
73.625	0.24672	-150.235627483958	-150.235627483958\\
73.625	0.25038	-156.037338138906	-156.037338138906\\
73.625	0.25404	-161.961280485504	-161.961280485504\\
73.625	0.2577	-168.007454523751	-168.007454523751\\
73.625	0.26136	-174.175860253648	-174.175860253648\\
73.625	0.26502	-180.466497675195	-180.466497675195\\
73.625	0.26868	-186.879366788392	-186.879366788392\\
73.625	0.27234	-193.414467593238	-193.414467593238\\
73.625	0.276	-200.071800089735	-200.071800089735\\
74	0.093	-17.2976942621307	-17.2976942621307\\
74	0.09666	-17.9684526875691	-17.9684526875691\\
74	0.10032	-18.7614428046573	-18.7614428046573\\
74	0.10398	-19.6766646133953	-19.6766646133953\\
74	0.10764	-20.714118113783	-20.714118113783\\
74	0.1113	-21.8738033058206	-21.8738033058206\\
74	0.11496	-23.155720189508	-23.155720189508\\
74	0.11862	-24.5598687648451	-24.5598687648451\\
74	0.12228	-26.0862490318321	-26.0862490318321\\
74	0.12594	-27.7348609904689	-27.7348609904689\\
74	0.1296	-29.5057046407554	-29.5057046407554\\
74	0.13326	-31.3987799826917	-31.3987799826917\\
74	0.13692	-33.4140870162778	-33.4140870162778\\
74	0.14058	-35.5516257415138	-35.5516257415138\\
74	0.14424	-37.8113961583996	-37.8113961583996\\
74	0.1479	-40.193398266935	-40.193398266935\\
74	0.15156	-42.6976320671204	-42.6976320671204\\
74	0.15522	-45.3240975589554	-45.3240975589554\\
74	0.15888	-48.0727947424404	-48.0727947424404\\
74	0.16254	-50.9437236175751	-50.9437236175751\\
74	0.1662	-53.9368841843596	-53.9368841843596\\
74	0.16986	-57.0522764427938	-57.0522764427938\\
74	0.17352	-60.2899003928779	-60.2899003928779\\
74	0.17718	-63.6497560346118	-63.6497560346118\\
74	0.18084	-67.1318433679955	-67.1318433679955\\
74	0.1845	-70.7361623930289	-70.7361623930289\\
74	0.18816	-74.4627131097122	-74.4627131097122\\
74	0.19182	-78.3114955180452	-78.3114955180452\\
74	0.19548	-82.2825096180281	-82.2825096180281\\
74	0.19914	-86.3757554096607	-86.3757554096607\\
74	0.2028	-90.5912328929432	-90.5912328929432\\
74	0.20646	-94.9289420678754	-94.9289420678754\\
74	0.21012	-99.3888829344574	-99.3888829344574\\
74	0.21378	-103.971055492689	-103.971055492689\\
74	0.21744	-108.675459742571	-108.675459742571\\
74	0.2211	-113.502095684102	-113.502095684102\\
74	0.22476	-118.450963317283	-118.450963317283\\
74	0.22842	-123.522062642114	-123.522062642114\\
74	0.23208	-128.715393658595	-128.715393658595\\
74	0.23574	-134.030956366726	-134.030956366726\\
74	0.2394	-139.468750766506	-139.468750766506\\
74	0.24306	-145.028776857936	-145.028776857936\\
74	0.24672	-150.711034641016	-150.711034641016\\
74	0.25038	-156.515524115746	-156.515524115746\\
74	0.25404	-162.442245282126	-162.442245282126\\
74	0.2577	-168.491198140155	-168.491198140155\\
74	0.26136	-174.662382689834	-174.662382689834\\
74	0.26502	-180.955798931163	-180.955798931163\\
74	0.26868	-187.371446864142	-187.371446864142\\
74	0.27234	-193.90932648877	-193.90932648877\\
74	0.276	-200.569437805049	-200.569437805049\\
};\label{tikz:theta_surf}
\end{axis}
\end{tikzpicture}%
%	\end{minipage}}
%	% This file was created by matlab2tikz.
%
\definecolor{mycolor1}{rgb}{0.00000,0.44700,0.74100}%
\definecolor{mycolor2}{rgb}{0.85000,0.32500,0.09800}%
%
\begin{tikzpicture}

\begin{axis}[%
width=2.616cm,
height=2.517cm,
at={(0cm,17.483cm)},
scale only axis,
xmin=56,
xmax=74,
tick align=outside,
xlabel style={font=\color{white!15!black}},
xlabel={$L_{cut}$},
ymin=0.093,
ymax=0.276,
ylabel style={font=\color{white!15!black}},
ylabel={$D_{rlx}$},
zmin=-100,
zmax=8.49026272261312,
zlabel style={font=\color{white!15!black}},
zlabel={$x_4,x_4$},
view={-140}{50},
axis background/.style={fill=white},
xmajorgrids,
ymajorgrids,
zmajorgrids,
legend style={at={(1.03,1)}, anchor=north west, legend cell align=left, align=left, draw=white!15!black}
]
\addplot3[only marks, mark=*, mark options={}, mark size=1.5000pt, color=mycolor1, fill=mycolor1] table[row sep=crcr]{%
x	y	z\\
74	0.123	-6.26560503440869\\
72	0.113	-7.54955426410135\\
61	0.095	-0.0389494181245107\\
56	0.093	0.0537159588168658\\
};
\addlegendentry{data1}

\addplot3[only marks, mark=*, mark options={}, mark size=1.5000pt, color=mycolor2, fill=mycolor2] table[row sep=crcr]{%
x	y	z\\
67	0.276	-46.0378528205072\\
66	0.255	-26.4573722128259\\
62	0.209	-10.5966716152609\\
57	0.193	-10.1446483284923\\
};
\addlegendentry{data2}

\addplot3[only marks, mark=*, mark options={}, mark size=1.5000pt, color=black, fill=black] table[row sep=crcr]{%
x	y	z\\
69	0.104	-6.1861611350065\\
};
\addlegendentry{data3}

\addplot3[only marks, mark=*, mark options={}, mark size=1.5000pt, color=black, fill=black] table[row sep=crcr]{%
x	y	z\\
64	0.23	-16.4864581125197\\
};
\addlegendentry{data4}


\addplot3[%
surf,
fill opacity=0.7, shader=interp, colormap={mymap}{[1pt] rgb(0pt)=(1,0.905882,0); rgb(1pt)=(1,0.901964,0); rgb(2pt)=(1,0.898051,0); rgb(3pt)=(1,0.894144,0); rgb(4pt)=(1,0.890243,0); rgb(5pt)=(1,0.886349,0); rgb(6pt)=(1,0.88246,0); rgb(7pt)=(1,0.878577,0); rgb(8pt)=(1,0.8747,0); rgb(9pt)=(1,0.870829,0); rgb(10pt)=(1,0.866964,0); rgb(11pt)=(1,0.863106,0); rgb(12pt)=(1,0.859253,0); rgb(13pt)=(1,0.855406,0); rgb(14pt)=(1,0.851566,0); rgb(15pt)=(1,0.847732,0); rgb(16pt)=(1,0.843903,0); rgb(17pt)=(1,0.840081,0); rgb(18pt)=(1,0.836265,0); rgb(19pt)=(1,0.832455,0); rgb(20pt)=(1,0.828652,0); rgb(21pt)=(1,0.824854,0); rgb(22pt)=(1,0.821063,0); rgb(23pt)=(1,0.817278,0); rgb(24pt)=(1,0.8135,0); rgb(25pt)=(1,0.809727,0); rgb(26pt)=(1,0.805961,0); rgb(27pt)=(1,0.8022,0); rgb(28pt)=(1,0.798445,0); rgb(29pt)=(1,0.794696,0); rgb(30pt)=(1,0.790953,0); rgb(31pt)=(1,0.787215,0); rgb(32pt)=(1,0.783484,0); rgb(33pt)=(1,0.779758,0); rgb(34pt)=(1,0.776038,0); rgb(35pt)=(1,0.772324,0); rgb(36pt)=(1,0.768615,0); rgb(37pt)=(1,0.764913,0); rgb(38pt)=(1,0.761217,0); rgb(39pt)=(1,0.757527,0); rgb(40pt)=(1,0.753843,0); rgb(41pt)=(1,0.750165,0); rgb(42pt)=(1,0.746493,0); rgb(43pt)=(1,0.742827,0); rgb(44pt)=(1,0.739167,0); rgb(45pt)=(1,0.735514,0); rgb(46pt)=(1,0.731867,0); rgb(47pt)=(1,0.728226,0); rgb(48pt)=(1,0.724591,0); rgb(49pt)=(1,0.720963,0); rgb(50pt)=(1,0.717341,0); rgb(51pt)=(1,0.713725,0); rgb(52pt)=(0.999994,0.710077,0); rgb(53pt)=(0.999974,0.706363,0); rgb(54pt)=(0.999942,0.702592,0); rgb(55pt)=(0.999898,0.698775,0); rgb(56pt)=(0.999841,0.694921,0); rgb(57pt)=(0.999771,0.691039,0); rgb(58pt)=(0.99969,0.687139,0); rgb(59pt)=(0.999596,0.68323,0); rgb(60pt)=(0.99949,0.679323,0); rgb(61pt)=(0.999372,0.675427,0); rgb(62pt)=(0.999242,0.67155,0); rgb(63pt)=(0.9991,0.667704,0); rgb(64pt)=(0.998946,0.663897,0); rgb(65pt)=(0.998781,0.660138,0); rgb(66pt)=(0.998605,0.656439,0); rgb(67pt)=(0.998416,0.652807,0); rgb(68pt)=(0.998217,0.649253,0); rgb(69pt)=(0.998006,0.645786,0); rgb(70pt)=(0.997785,0.642416,0); rgb(71pt)=(0.997552,0.639152,0); rgb(72pt)=(0.997308,0.636004,0); rgb(73pt)=(0.997053,0.632982,0); rgb(74pt)=(0.996788,0.630095,0); rgb(75pt)=(0.996512,0.627352,0); rgb(76pt)=(0.996226,0.624763,0); rgb(77pt)=(0.995851,0.622329,0); rgb(78pt)=(0.99494,0.619997,0); rgb(79pt)=(0.99345,0.617753,0); rgb(80pt)=(0.991419,0.61559,0); rgb(81pt)=(0.988885,0.613503,0); rgb(82pt)=(0.985886,0.611486,0); rgb(83pt)=(0.98246,0.609532,0); rgb(84pt)=(0.978643,0.607636,0); rgb(85pt)=(0.974475,0.605791,0); rgb(86pt)=(0.969992,0.603992,0); rgb(87pt)=(0.965232,0.602233,0); rgb(88pt)=(0.960233,0.600507,0); rgb(89pt)=(0.955033,0.598808,0); rgb(90pt)=(0.949669,0.59713,0); rgb(91pt)=(0.94418,0.595468,0); rgb(92pt)=(0.938602,0.593815,0); rgb(93pt)=(0.932974,0.592166,0); rgb(94pt)=(0.927333,0.590513,0); rgb(95pt)=(0.921717,0.588852,0); rgb(96pt)=(0.916164,0.587176,0); rgb(97pt)=(0.910711,0.585479,0); rgb(98pt)=(0.905397,0.583755,0); rgb(99pt)=(0.900258,0.581999,0); rgb(100pt)=(0.895333,0.580203,0); rgb(101pt)=(0.890659,0.578362,0); rgb(102pt)=(0.886275,0.576471,0); rgb(103pt)=(0.882047,0.574545,0); rgb(104pt)=(0.877819,0.572608,0); rgb(105pt)=(0.873592,0.57066,0); rgb(106pt)=(0.869366,0.568701,0); rgb(107pt)=(0.865143,0.566733,0); rgb(108pt)=(0.860924,0.564756,0); rgb(109pt)=(0.856708,0.562771,0); rgb(110pt)=(0.852497,0.560778,0); rgb(111pt)=(0.848292,0.558779,0); rgb(112pt)=(0.844092,0.556774,0); rgb(113pt)=(0.8399,0.554763,0); rgb(114pt)=(0.835716,0.552749,0); rgb(115pt)=(0.831541,0.55073,0); rgb(116pt)=(0.827374,0.548709,0); rgb(117pt)=(0.823219,0.546686,0); rgb(118pt)=(0.819074,0.54466,0); rgb(119pt)=(0.81494,0.542635,0); rgb(120pt)=(0.81082,0.540609,0); rgb(121pt)=(0.806712,0.538584,0); rgb(122pt)=(0.802619,0.53656,0); rgb(123pt)=(0.798541,0.534539,0); rgb(124pt)=(0.794478,0.532521,0); rgb(125pt)=(0.790431,0.530506,0); rgb(126pt)=(0.786402,0.528496,0); rgb(127pt)=(0.782391,0.526491,0); rgb(128pt)=(0.77841,0.524489,0); rgb(129pt)=(0.774523,0.522478,0); rgb(130pt)=(0.770731,0.520455,0); rgb(131pt)=(0.767022,0.518424,0); rgb(132pt)=(0.763384,0.516385,0); rgb(133pt)=(0.759804,0.514339,0); rgb(134pt)=(0.756272,0.51229,0); rgb(135pt)=(0.752775,0.510237,0); rgb(136pt)=(0.749302,0.508182,0); rgb(137pt)=(0.74584,0.506128,0); rgb(138pt)=(0.742378,0.504075,0); rgb(139pt)=(0.738904,0.502025,0); rgb(140pt)=(0.735406,0.499979,0); rgb(141pt)=(0.731872,0.49794,0); rgb(142pt)=(0.72829,0.495909,0); rgb(143pt)=(0.724649,0.493887,0); rgb(144pt)=(0.720936,0.491875,0); rgb(145pt)=(0.71714,0.489876,0); rgb(146pt)=(0.713249,0.487891,0); rgb(147pt)=(0.709251,0.485921,0); rgb(148pt)=(0.705134,0.483968,0); rgb(149pt)=(0.700887,0.482033,0); rgb(150pt)=(0.696497,0.480118,0); rgb(151pt)=(0.691952,0.478225,0); rgb(152pt)=(0.687242,0.476355,0); rgb(153pt)=(0.682353,0.47451,0); rgb(154pt)=(0.677195,0.472696,0); rgb(155pt)=(0.6717,0.470916,0); rgb(156pt)=(0.665891,0.469169,0); rgb(157pt)=(0.659791,0.46745,0); rgb(158pt)=(0.653423,0.465756,0); rgb(159pt)=(0.64681,0.464084,0); rgb(160pt)=(0.639976,0.462432,0); rgb(161pt)=(0.632943,0.460795,0); rgb(162pt)=(0.625734,0.459171,0); rgb(163pt)=(0.618373,0.457556,0); rgb(164pt)=(0.610882,0.455948,0); rgb(165pt)=(0.603284,0.454343,0); rgb(166pt)=(0.595604,0.452737,0); rgb(167pt)=(0.587863,0.451129,0); rgb(168pt)=(0.580084,0.449514,0); rgb(169pt)=(0.572292,0.447889,0); rgb(170pt)=(0.564508,0.446252,0); rgb(171pt)=(0.556756,0.444599,0); rgb(172pt)=(0.549059,0.442927,0); rgb(173pt)=(0.54144,0.441232,0); rgb(174pt)=(0.533922,0.439512,0); rgb(175pt)=(0.526529,0.437764,0); rgb(176pt)=(0.519282,0.435983,0); rgb(177pt)=(0.512206,0.434168,0); rgb(178pt)=(0.505323,0.432315,0); rgb(179pt)=(0.498628,0.430422,3.92506e-06); rgb(180pt)=(0.491973,0.428504,3.49981e-05); rgb(181pt)=(0.485331,0.426562,9.63073e-05); rgb(182pt)=(0.478704,0.424596,0.000186979); rgb(183pt)=(0.472096,0.422609,0.000306141); rgb(184pt)=(0.465508,0.420599,0.00045292); rgb(185pt)=(0.458942,0.418567,0.000626441); rgb(186pt)=(0.452401,0.416515,0.000825833); rgb(187pt)=(0.445885,0.414441,0.00105022); rgb(188pt)=(0.439399,0.412348,0.00129873); rgb(189pt)=(0.432942,0.410234,0.00157049); rgb(190pt)=(0.426518,0.408102,0.00186463); rgb(191pt)=(0.420129,0.40595,0.00218028); rgb(192pt)=(0.413777,0.40378,0.00251655); rgb(193pt)=(0.407464,0.401592,0.00287258); rgb(194pt)=(0.401191,0.399386,0.00324749); rgb(195pt)=(0.394962,0.397164,0.00364042); rgb(196pt)=(0.388777,0.394925,0.00405048); rgb(197pt)=(0.38264,0.39267,0.00447681); rgb(198pt)=(0.376552,0.390399,0.00491852); rgb(199pt)=(0.370516,0.388113,0.00537476); rgb(200pt)=(0.364532,0.385812,0.00584464); rgb(201pt)=(0.358605,0.383497,0.00632729); rgb(202pt)=(0.352735,0.381168,0.00682184); rgb(203pt)=(0.346925,0.378826,0.00732741); rgb(204pt)=(0.341176,0.376471,0.00784314); rgb(205pt)=(0.335485,0.374093,0.00847245); rgb(206pt)=(0.329843,0.371682,0.00930909); rgb(207pt)=(0.324249,0.369242,0.0103377); rgb(208pt)=(0.318701,0.366772,0.0115428); rgb(209pt)=(0.313198,0.364275,0.0129091); rgb(210pt)=(0.307739,0.361753,0.0144211); rgb(211pt)=(0.302322,0.359206,0.0160634); rgb(212pt)=(0.296945,0.356637,0.0178207); rgb(213pt)=(0.291607,0.354048,0.0196776); rgb(214pt)=(0.286307,0.35144,0.0216186); rgb(215pt)=(0.281043,0.348814,0.0236284); rgb(216pt)=(0.275813,0.346172,0.0256916); rgb(217pt)=(0.270616,0.343517,0.0277927); rgb(218pt)=(0.265451,0.340849,0.0299163); rgb(219pt)=(0.260317,0.33817,0.0320472); rgb(220pt)=(0.25521,0.335482,0.0341698); rgb(221pt)=(0.250131,0.332786,0.0362688); rgb(222pt)=(0.245078,0.330085,0.0383287); rgb(223pt)=(0.240048,0.327379,0.0403343); rgb(224pt)=(0.235042,0.324671,0.04227); rgb(225pt)=(0.230056,0.321962,0.0441205); rgb(226pt)=(0.22509,0.319254,0.0458704); rgb(227pt)=(0.220142,0.316548,0.0475043); rgb(228pt)=(0.215212,0.313846,0.0490067); rgb(229pt)=(0.210296,0.311149,0.0503624); rgb(230pt)=(0.205395,0.308459,0.0515759); rgb(231pt)=(0.200514,0.305763,0.052757); rgb(232pt)=(0.195655,0.303061,0.0539242); rgb(233pt)=(0.190817,0.300353,0.0550763); rgb(234pt)=(0.186001,0.297639,0.0562123); rgb(235pt)=(0.181207,0.294918,0.0573313); rgb(236pt)=(0.176434,0.292191,0.0584321); rgb(237pt)=(0.171685,0.289458,0.0595136); rgb(238pt)=(0.166957,0.286719,0.060575); rgb(239pt)=(0.162252,0.283973,0.0616151); rgb(240pt)=(0.15757,0.281221,0.0626328); rgb(241pt)=(0.152911,0.278463,0.0636271); rgb(242pt)=(0.148275,0.275699,0.0645971); rgb(243pt)=(0.143663,0.272929,0.0655416); rgb(244pt)=(0.139074,0.270152,0.0664596); rgb(245pt)=(0.134508,0.26737,0.06735); rgb(246pt)=(0.129967,0.264581,0.0682118); rgb(247pt)=(0.125449,0.261787,0.0690441); rgb(248pt)=(0.120956,0.258986,0.0698456); rgb(249pt)=(0.116487,0.25618,0.0706154); rgb(250pt)=(0.112043,0.253367,0.0713525); rgb(251pt)=(0.107623,0.250549,0.0720557); rgb(252pt)=(0.103229,0.247724,0.0727241); rgb(253pt)=(0.0988592,0.244894,0.0733566); rgb(254pt)=(0.0945149,0.242058,0.0739522); rgb(255pt)=(0.0901961,0.239216,0.0745098)}, mesh/rows=49]
table[row sep=crcr, point meta=\thisrow{c}] {%
%
x	y	z	c\\
56	0.093	0.399980093782396	0.399980093782396\\
56	0.09666	1.26640918112157	1.26640918112157\\
56	0.10032	2.02944281793312	2.02944281793312\\
56	0.10398	2.6890810042171	2.6890810042171\\
56	0.10764	3.24532373997347	3.24532373997347\\
56	0.1113	3.6981710252023	3.6981710252023\\
56	0.11496	4.04762285990347	4.04762285990347\\
56	0.11862	4.29367924407705	4.29367924407705\\
56	0.12228	4.43634017772302	4.43634017772302\\
56	0.12594	4.4756056608414	4.4756056608414\\
56	0.1296	4.41147569343218	4.41147569343218\\
56	0.13326	4.24395027549537	4.24395027549537\\
56	0.13692	3.97302940703095	3.97302940703095\\
56	0.14058	3.59871308803891	3.59871308803891\\
56	0.14424	3.12100131851928	3.12100131851928\\
56	0.1479	2.53989409847206	2.53989409847206\\
56	0.15156	1.85539142789725	1.85539142789725\\
56	0.15522	1.06749330679489	1.06749330679489\\
56	0.15888	0.176199735164843	0.176199735164843\\
56	0.16254	-0.818489286992772	-0.818489286992772\\
56	0.1662	-1.91657375967797	-1.91657375967797\\
56	0.16986	-3.11805368289079	-3.11805368289079\\
56	0.17352	-4.42292905663113	-4.42292905663113\\
56	0.17718	-5.83119988089915	-5.83119988089915\\
56	0.18084	-7.34286615569479	-7.34286615569479\\
56	0.1845	-8.95792788101801	-8.95792788101801\\
56	0.18816	-10.6763850568688	-10.6763850568688\\
56	0.19182	-12.4982376832472	-12.4982376832472\\
56	0.19548	-14.4234857601532	-14.4234857601532\\
56	0.19914	-16.4521292875868	-16.4521292875868\\
56	0.2028	-18.584168265548	-18.584168265548\\
56	0.20646	-20.8196026940369	-20.8196026940369\\
56	0.21012	-23.1584325730532	-23.1584325730532\\
56	0.21378	-25.6006579025972	-25.6006579025972\\
56	0.21744	-28.1462786826688	-28.1462786826688\\
56	0.2211	-30.795294913268	-30.795294913268\\
56	0.22476	-33.5477065943948	-33.5477065943948\\
56	0.22842	-36.4035137260492	-36.4035137260492\\
56	0.23208	-39.3627163082311	-39.3627163082311\\
56	0.23574	-42.4253143409408	-42.4253143409408\\
56	0.2394	-45.5913078241779	-45.5913078241779\\
56	0.24306	-48.8606967579427	-48.8606967579427\\
56	0.24672	-52.2334811422351	-52.2334811422351\\
56	0.25038	-55.7096609770551	-55.7096609770551\\
56	0.25404	-59.2892362624027	-59.2892362624027\\
56	0.2577	-62.9722069982779	-62.9722069982779\\
56	0.26136	-66.7585731846807	-66.7585731846807\\
56	0.26502	-70.6483348216111	-70.6483348216111\\
56	0.26868	-74.6414919090691	-74.6414919090691\\
56	0.27234	-78.7380444470546	-78.7380444470546\\
56	0.276	-82.9379924355678	-82.9379924355678\\
56.375	0.093	0.396020512192244	0.396020512192244\\
56.375	0.09666	1.29471409341978	1.29471409341978\\
56.375	0.10032	2.0900122241197	2.0900122241197\\
56.375	0.10398	2.78191490429209	2.78191490429209\\
56.375	0.10764	3.37042213393688	3.37042213393688\\
56.375	0.1113	3.85553391305407	3.85553391305407\\
56.375	0.11496	4.23725024164366	4.23725024164366\\
56.375	0.11862	4.5155711197056	4.5155711197056\\
56.375	0.12228	4.69049654723994	4.69049654723994\\
56.375	0.12594	4.76202652424674	4.76202652424674\\
56.375	0.1296	4.73016105072594	4.73016105072594\\
56.375	0.13326	4.59490012667754	4.59490012667754\\
56.375	0.13692	4.35624375210149	4.35624375210149\\
56.375	0.14058	4.01419192699787	4.01419192699787\\
56.375	0.14424	3.5687446513666	3.5687446513666\\
56.375	0.1479	3.0199019252078	3.0199019252078\\
56.375	0.15156	2.36766374852141	2.36766374852141\\
56.375	0.15522	1.61203012130736	1.61203012130736\\
56.375	0.15888	0.753001043565732	0.753001043565732\\
56.375	0.16254	-0.209423484703464	-0.209423484703464\\
56.375	0.1662	-1.2752434635003	-1.2752434635003\\
56.375	0.16986	-2.4444588928247	-2.4444588928247\\
56.375	0.17352	-3.71706977267668	-3.71706977267668\\
56.375	0.17718	-5.09307610305628	-5.09307610305628\\
56.375	0.18084	-6.5724778839635	-6.5724778839635\\
56.375	0.1845	-8.15527511539835	-8.15527511539835\\
56.375	0.18816	-9.84146779736076	-9.84146779736076\\
56.375	0.19182	-11.6310559298508	-11.6310559298508\\
56.375	0.19548	-13.5240395128683	-13.5240395128683\\
56.375	0.19914	-15.5204185464136	-15.5204185464136\\
56.375	0.2028	-17.6201930304864	-17.6201930304864\\
56.375	0.20646	-19.8233629650868	-19.8233629650868\\
56.375	0.21012	-22.1299283502148	-22.1299283502148\\
56.375	0.21378	-24.5398891858704	-24.5398891858704\\
56.375	0.21744	-27.0532454720536	-27.0532454720536\\
56.375	0.2211	-29.6699972087644	-29.6699972087644\\
56.375	0.22476	-32.3901443960028	-32.3901443960028\\
56.375	0.22842	-35.2136870337688	-35.2136870337688\\
56.375	0.23208	-38.1406251220624	-38.1406251220624\\
56.375	0.23574	-41.1709586608836	-41.1709586608836\\
56.375	0.2394	-44.3046876502324	-44.3046876502324\\
56.375	0.24306	-47.5418120901087	-47.5418120901087\\
56.375	0.24672	-50.8823319805128	-50.8823319805128\\
56.375	0.25038	-54.3262473214444	-54.3262473214444\\
56.375	0.25404	-57.8735581129036	-57.8735581129036\\
56.375	0.2577	-61.5242643548904	-61.5242643548904\\
56.375	0.26136	-65.2783660474048	-65.2783660474048\\
56.375	0.26502	-69.1358631904468	-69.1358631904468\\
56.375	0.26868	-73.0967557840163	-73.0967557840163\\
56.375	0.27234	-77.1610438281135	-77.1610438281135\\
56.375	0.276	-81.3287273227383	-81.3287273227383\\
56.75	0.093	0.371320974545078	0.371320974545078\\
56.75	0.09666	1.30227904966103	1.30227904966103\\
56.75	0.10032	2.12984167424937	2.12984167424937\\
56.75	0.10398	2.85400884831013	2.85400884831013\\
56.75	0.10764	3.47478057184334	3.47478057184334\\
56.75	0.1113	3.99215684484889	3.99215684484889\\
56.75	0.11496	4.40613766732685	4.40613766732685\\
56.75	0.11862	4.71672303927726	4.71672303927726\\
56.75	0.12228	4.92391296069996	4.92391296069996\\
56.75	0.12594	5.02770743159518	5.02770743159518\\
56.75	0.1296	5.02810645196274	5.02810645196274\\
56.75	0.13326	4.92511002180271	4.92511002180271\\
56.75	0.13692	4.71871814111508	4.71871814111508\\
56.75	0.14058	4.40893080989981	4.40893080989981\\
56.75	0.14424	3.99574802815702	3.99574802815702\\
56.75	0.1479	3.47916979588659	3.47916979588659\\
56.75	0.15156	2.85919611308856	2.85919611308856\\
56.75	0.15522	2.13582697976292	2.13582697976292\\
56.75	0.15888	1.30906239590966	1.30906239590966\\
56.75	0.16254	0.378902361528887	0.378902361528887\\
56.75	0.1662	-0.654653123379532	-0.654653123379532\\
56.75	0.16986	-1.79160405881557	-1.79160405881557\\
56.75	0.17352	-3.03195044477918	-3.03195044477918\\
56.75	0.17718	-4.37569228127042	-4.37569228127042\\
56.75	0.18084	-5.82282956828922	-5.82282956828922\\
56.75	0.1845	-7.37336230583566	-7.37336230583566\\
56.75	0.18816	-9.02729049390965	-9.02729049390965\\
56.75	0.19182	-10.7846141325113	-10.7846141325113\\
56.75	0.19548	-12.6453332216405	-12.6453332216405\\
56.75	0.19914	-14.6094477612973	-14.6094477612973\\
56.75	0.2028	-16.6769577514817	-16.6769577514817\\
56.75	0.20646	-18.8478631921938	-18.8478631921938\\
56.75	0.21012	-21.1221640834333	-21.1221640834333\\
56.75	0.21378	-23.4998604252006	-23.4998604252006\\
56.75	0.21744	-25.9809522174953	-25.9809522174953\\
56.75	0.2211	-28.5654394603177	-28.5654394603177\\
56.75	0.22476	-31.2533221536678	-31.2533221536678\\
56.75	0.22842	-34.0446002975454	-34.0446002975454\\
56.75	0.23208	-36.9392738919506	-36.9392738919506\\
56.75	0.23574	-39.9373429368834	-39.9373429368834\\
56.75	0.2394	-43.0388074323438	-43.0388074323438\\
56.75	0.24306	-46.2436673783318	-46.2436673783318\\
56.75	0.24672	-49.5519227748474	-49.5519227748474\\
56.75	0.25038	-52.9635736218906	-52.9635736218906\\
56.75	0.25404	-56.4786199194614	-56.4786199194614\\
56.75	0.2577	-60.0970616675598	-60.0970616675598\\
56.75	0.26136	-63.8188988661858	-63.8188988661858\\
56.75	0.26502	-67.6441315153394	-67.6441315153394\\
56.75	0.26868	-71.5727596150206	-71.5727596150206\\
56.75	0.27234	-75.6047831652294	-75.6047831652294\\
56.75	0.276	-79.7402021659657	-79.7402021659657\\
57.125	0.093	0.32588148084087	0.32588148084087\\
57.125	0.09666	1.28910404984519	1.28910404984519\\
57.125	0.10032	2.14893116832189	2.14893116832189\\
57.125	0.10398	2.90536283627112	2.90536283627112\\
57.125	0.10764	3.55839905369264	3.55839905369264\\
57.125	0.1113	4.10803982058661	4.10803982058661\\
57.125	0.11496	4.55428513695298	4.55428513695298\\
57.125	0.11862	4.89713500279171	4.89713500279171\\
57.125	0.12228	5.13658941810294	5.13658941810294\\
57.125	0.12594	5.27264838288646	5.27264838288646\\
57.125	0.1296	5.30531189714239	5.30531189714239\\
57.125	0.13326	5.23457996087083	5.23457996087083\\
57.125	0.13692	5.06045257407156	5.06045257407156\\
57.125	0.14058	4.78292973674472	4.78292973674472\\
57.125	0.14424	4.40201144889029	4.40201144889029\\
57.125	0.1479	3.91769771050822	3.91769771050822\\
57.125	0.15156	3.32998852159861	3.32998852159861\\
57.125	0.15522	2.63888388216139	2.63888388216139\\
57.125	0.15888	1.8443837921965	1.8443837921965\\
57.125	0.16254	0.946488251704139	0.946488251704139\\
57.125	0.1662	-0.0548027393159174	-0.0548027393159174\\
57.125	0.16986	-1.15948918086353	-1.15948918086353\\
57.125	0.17352	-2.36757107293873	-2.36757107293873\\
57.125	0.17718	-3.6790484155416	-3.6790484155416\\
57.125	0.18084	-5.09392120867199	-5.09392120867199\\
57.125	0.1845	-6.61218945233	-6.61218945233\\
57.125	0.18816	-8.23385314651568	-8.23385314651568\\
57.125	0.19182	-9.95891229122884	-9.95891229122884\\
57.125	0.19548	-11.7873668864697	-11.7873668864697\\
57.125	0.19914	-13.7192169322382	-13.7192169322382\\
57.125	0.2028	-15.7544624285341	-15.7544624285341\\
57.125	0.20646	-17.8931033753578	-17.8931033753578\\
57.125	0.21012	-20.135139772709	-20.135139772709\\
57.125	0.21378	-22.4805716205878	-22.4805716205878\\
57.125	0.21744	-24.9293989189942	-24.9293989189942\\
57.125	0.2211	-27.4816216679282	-27.4816216679282\\
57.125	0.22476	-30.1372398673898	-30.1372398673898\\
57.125	0.22842	-32.896253517379	-32.896253517379\\
57.125	0.23208	-35.7586626178959	-35.7586626178959\\
57.125	0.23574	-38.7244671689403	-38.7244671689403\\
57.125	0.2394	-41.7936671705124	-41.7936671705124\\
57.125	0.24306	-44.9662626226119	-44.9662626226119\\
57.125	0.24672	-48.2422535252391	-48.2422535252391\\
57.125	0.25038	-51.6216398783939	-51.6216398783939\\
57.125	0.25404	-55.1044216820763	-55.1044216820763\\
57.125	0.2577	-58.6905989362863	-58.6905989362863\\
57.125	0.26136	-62.380171641024	-62.380171641024\\
57.125	0.26502	-66.1731397962891	-66.1731397962891\\
57.125	0.26868	-70.0695034020819	-70.0695034020819\\
57.125	0.27234	-74.0692624584023	-74.0692624584023\\
57.125	0.276	-78.1724169652503	-78.1724169652503\\
57.5	0.093	0.259702031079591	0.259702031079591\\
57.5	0.09666	1.25518909397238	1.25518909397238\\
57.5	0.10032	2.1472807063375	2.1472807063375\\
57.5	0.10398	2.93597686817505	2.93597686817505\\
57.5	0.10764	3.62127757948498	3.62127757948498\\
57.5	0.1113	4.20318284026732	4.20318284026732\\
57.5	0.11496	4.68169265052211	4.68169265052211\\
57.5	0.11862	5.05680701024931	5.05680701024931\\
57.5	0.12228	5.32852591944884	5.32852591944884\\
57.5	0.12594	5.49684937812079	5.49684937812079\\
57.5	0.1296	5.56177738626513	5.56177738626513\\
57.5	0.13326	5.52330994388188	5.52330994388188\\
57.5	0.13692	5.38144705097103	5.38144705097103\\
57.5	0.14058	5.13618870753261	5.13618870753261\\
57.5	0.14424	4.7875349135666	4.7875349135666\\
57.5	0.1479	4.33548566907295	4.33548566907295\\
57.5	0.15156	3.7800409740517	3.7800409740517\\
57.5	0.15522	3.12120082850285	3.12120082850285\\
57.5	0.15888	2.35896523242637	2.35896523242637\\
57.5	0.16254	1.49333418582238	1.49333418582238\\
57.5	0.1662	0.52430768869074	0.52430768869074\\
57.5	0.16986	-0.548114258968454	-0.548114258968454\\
57.5	0.17352	-1.72393165715529	-1.72393165715529\\
57.5	0.17718	-3.00314450586974	-3.00314450586974\\
57.5	0.18084	-4.38575280511176	-4.38575280511176\\
57.5	0.1845	-5.87175655488142	-5.87175655488142\\
57.5	0.18816	-7.46115575517862	-7.46115575517862\\
57.5	0.19182	-9.15395040600342	-9.15395040600342\\
57.5	0.19548	-10.9501405073559	-10.9501405073559\\
57.5	0.19914	-12.849726059236	-12.849726059236\\
57.5	0.2028	-14.8527070616436	-14.8527070616436\\
57.5	0.20646	-16.9590835145788	-16.9590835145788\\
57.5	0.21012	-19.1688554180416	-19.1688554180416\\
57.5	0.21378	-21.482022772032	-21.482022772032\\
57.5	0.21744	-23.89858557655	-23.89858557655\\
57.5	0.2211	-26.4185438315956	-26.4185438315956\\
57.5	0.22476	-29.0418975371689	-29.0418975371689\\
57.5	0.22842	-31.7686466932697	-31.7686466932697\\
57.5	0.23208	-34.5987912998981	-34.5987912998981\\
57.5	0.23574	-37.5323313570542	-37.5323313570542\\
57.5	0.2394	-40.5692668647377	-40.5692668647377\\
57.5	0.24306	-43.7095978229489	-43.7095978229489\\
57.5	0.24672	-46.9533242316878	-46.9533242316878\\
57.5	0.25038	-50.3004460909542	-50.3004460909542\\
57.5	0.25404	-53.7509634007482	-53.7509634007482\\
57.5	0.2577	-57.3048761610698	-57.3048761610698\\
57.5	0.26136	-60.9621843719191	-60.9621843719191\\
57.5	0.26502	-64.7228880332958	-64.7228880332958\\
57.5	0.26868	-68.5869871452002	-68.5869871452002\\
57.5	0.27234	-72.5544817076322	-72.5544817076322\\
57.5	0.276	-76.6253717205918	-76.6253717205918\\
57.875	0.093	0.172782625261299	0.172782625261299\\
57.875	0.09666	1.20053418204245	1.20053418204245\\
57.875	0.10032	2.12489028829594	2.12489028829594\\
57.875	0.10398	2.9458509440219	2.9458509440219\\
57.875	0.10764	3.66341614922025	3.66341614922025\\
57.875	0.1113	4.27758590389101	4.27758590389101\\
57.875	0.11496	4.78836020803411	4.78836020803411\\
57.875	0.11862	5.19573906164967	5.19573906164967\\
57.875	0.12228	5.49972246473762	5.49972246473762\\
57.875	0.12594	5.70031041729798	5.70031041729798\\
57.875	0.1296	5.79750291933075	5.79750291933075\\
57.875	0.13326	5.79129997083592	5.79129997083592\\
57.875	0.13692	5.68170157181349	5.68170157181349\\
57.875	0.14058	5.46870772226343	5.46870772226343\\
57.875	0.14424	5.15231842218579	5.15231842218579\\
57.875	0.1479	4.73253367158055	4.73253367158055\\
57.875	0.15156	4.20935347044767	4.20935347044767\\
57.875	0.15522	3.58277781878718	3.58277781878718\\
57.875	0.15888	2.85280671659918	2.85280671659918\\
57.875	0.16254	2.01944016388349	2.01944016388349\\
57.875	0.1662	1.08267816064027	1.08267816064027\\
57.875	0.16986	0.0425207068694391	0.0425207068694391\\
57.875	0.17352	-1.10103219742898	-1.10103219742898\\
57.875	0.17718	-2.34798055225501	-2.34798055225501\\
57.875	0.18084	-3.69832435760861	-3.69832435760861\\
57.875	0.1845	-5.1520636134899	-5.1520636134899\\
57.875	0.18816	-6.70919831989869	-6.70919831989869\\
57.875	0.19182	-8.36972847683512	-8.36972847683512\\
57.875	0.19548	-10.1336540842991	-10.1336540842991\\
57.875	0.19914	-12.0009751422908	-12.0009751422908\\
57.875	0.2028	-13.9716916508101	-13.9716916508101\\
57.875	0.20646	-16.0458036098569	-16.0458036098569\\
57.875	0.21012	-18.2233110194313	-18.2233110194313\\
57.875	0.21378	-20.5042138795334	-20.5042138795334\\
57.875	0.21744	-22.888512190163	-22.888512190163\\
57.875	0.2211	-25.3762059513202	-25.3762059513202\\
57.875	0.22476	-27.967295163005	-27.967295163005\\
57.875	0.22842	-30.6617798252174	-30.6617798252174\\
57.875	0.23208	-33.4596599379574	-33.4596599379574\\
57.875	0.23574	-36.3609355012251	-36.3609355012251\\
57.875	0.2394	-39.3656065150203	-39.3656065150203\\
57.875	0.24306	-42.4736729793431	-42.4736729793431\\
57.875	0.24672	-45.6851348941935	-45.6851348941935\\
57.875	0.25038	-48.9999922595716	-48.9999922595716\\
57.875	0.25404	-52.4182450754772	-52.4182450754772\\
57.875	0.2577	-55.9398933419104	-55.9398933419104\\
57.875	0.26136	-59.5649370588713	-59.5649370588713\\
57.875	0.26502	-63.2933762263597	-63.2933762263597\\
57.875	0.26868	-67.1252108443757	-67.1252108443757\\
57.875	0.27234	-71.0604409129193	-71.0604409129193\\
57.875	0.276	-75.0990664319904	-75.0990664319904\\
58.25	0.093	0.0651232633859635	0.0651232633859635\\
58.25	0.09666	1.12513931405548	1.12513931405548\\
58.25	0.10032	2.08175991419744	2.08175991419744\\
58.25	0.10398	2.93498506381177	2.93498506381177\\
58.25	0.10764	3.68481476289848	3.68481476289848\\
58.25	0.1113	4.33124901145766	4.33124901145766\\
58.25	0.11496	4.87428780948923	4.87428780948923\\
58.25	0.11862	5.31393115699316	5.31393115699316\\
58.25	0.12228	5.65017905396947	5.65017905396947\\
58.25	0.12594	5.88303150041826	5.88303150041826\\
58.25	0.1296	6.01248849633939	6.01248849633939\\
58.25	0.13326	6.03855004173292	6.03855004173292\\
58.25	0.13692	5.96121613659891	5.96121613659891\\
58.25	0.14058	5.78048678093721	5.78048678093721\\
58.25	0.14424	5.49636197474793	5.49636197474793\\
58.25	0.1479	5.10884171803112	5.10884171803112\\
58.25	0.15156	4.61792601078665	4.61792601078665\\
58.25	0.15522	4.02361485301458	4.02361485301458\\
58.25	0.15888	3.32590824471494	3.32590824471494\\
58.25	0.16254	2.52480618588773	2.52480618588773\\
58.25	0.1662	1.62030867653287	1.62030867653287\\
58.25	0.16986	0.612415716650403	0.612415716650403\\
58.25	0.17352	-0.498872693759594	-0.498872693759594\\
58.25	0.17718	-1.71355655469726	-1.71355655469726\\
58.25	0.18084	-3.0316358661625	-3.0316358661625\\
58.25	0.1845	-4.45311062815537	-4.45311062815537\\
58.25	0.18816	-5.97798084067574	-5.97798084067574\\
58.25	0.19182	-7.60624650372387	-7.60624650372387\\
58.25	0.19548	-9.33790761729946	-9.33790761729946\\
58.25	0.19914	-11.1729641814027	-11.1729641814027\\
58.25	0.2028	-13.1114161960336	-13.1114161960336\\
58.25	0.20646	-15.153263661192	-15.153263661192\\
58.25	0.21012	-17.298506576878	-17.298506576878\\
58.25	0.21378	-19.5471449430917	-19.5471449430917\\
58.25	0.21744	-21.8991787598329	-21.8991787598329\\
58.25	0.2211	-24.3546080271017	-24.3546080271017\\
58.25	0.22476	-26.9134327448982	-26.9134327448982\\
58.25	0.22842	-29.5756529132222	-29.5756529132222\\
58.25	0.23208	-32.3412685320738	-32.3412685320738\\
58.25	0.23574	-35.210279601453	-35.210279601453\\
58.25	0.2394	-38.1826861213599	-38.1826861213599\\
58.25	0.24306	-41.2584880917943	-41.2584880917943\\
58.25	0.24672	-44.4376855127563	-44.4376855127563\\
58.25	0.25038	-47.720278384246	-47.720278384246\\
58.25	0.25404	-51.1062667062632	-51.1062667062632\\
58.25	0.2577	-54.595650478808	-54.595650478808\\
58.25	0.26136	-58.1884297018805	-58.1884297018805\\
58.25	0.26502	-61.8846043754805	-61.8846043754805\\
58.25	0.26868	-65.684174499608	-65.684174499608\\
58.25	0.27234	-69.5871400742633	-69.5871400742633\\
58.25	0.276	-73.5935010994461	-73.5935010994461\\
58.625	0.093	-0.0632760545463853	-0.0632760545463853\\
58.625	0.09666	1.02900449001149	1.02900449001149\\
58.625	0.10032	2.01788958404182	2.01788958404182\\
58.625	0.10398	2.90337922754456	2.90337922754456\\
58.625	0.10764	3.6854734205197	3.6854734205197\\
58.625	0.1113	4.36417216296724	4.36417216296724\\
58.625	0.11496	4.93947545488717	4.93947545488717\\
58.625	0.11862	5.41138329627952	5.41138329627952\\
58.625	0.12228	5.77989568714425	5.77989568714425\\
58.625	0.12594	6.0450126274814	6.0450126274814\\
58.625	0.1296	6.20673411729095	6.20673411729095\\
58.625	0.13326	6.2650601565729	6.2650601565729\\
58.625	0.13692	6.21999074532719	6.21999074532719\\
58.625	0.14058	6.07152588355397	6.07152588355397\\
58.625	0.14424	5.81966557125305	5.81966557125305\\
58.625	0.1479	5.46440980842466	5.46440980842466\\
58.625	0.15156	5.00575859506856	5.00575859506856\\
58.625	0.15522	4.44371193118491	4.44371193118491\\
58.625	0.15888	3.77826981677363	3.77826981677363\\
58.625	0.16254	3.00943225183484	3.00943225183484\\
58.625	0.1662	2.13719923636835	2.13719923636835\\
58.625	0.16986	1.1615707703743	1.1615707703743\\
58.625	0.17352	0.0825468538526621	0.0825468538526621\\
58.625	0.17718	-1.09987251319659	-1.09987251319659\\
58.625	0.18084	-2.38568733077341	-2.38568733077341\\
58.625	0.1845	-3.77489759887791	-3.77489759887791\\
58.625	0.18816	-5.26750331750992	-5.26750331750992\\
58.625	0.19182	-6.86350448666963	-6.86350448666963\\
58.625	0.19548	-8.5629011063568	-8.5629011063568\\
58.625	0.19914	-10.3656931765717	-10.3656931765717\\
58.625	0.2028	-12.2718806973141	-12.2718806973141\\
58.625	0.20646	-14.2814636685842	-14.2814636685842\\
58.625	0.21012	-16.3944420903818	-16.3944420903818\\
58.625	0.21378	-18.6108159627071	-18.6108159627071\\
58.625	0.21744	-20.9305852855599	-20.9305852855599\\
58.625	0.2211	-23.3537500589403	-23.3537500589403\\
58.625	0.22476	-25.8803102828484	-25.8803102828484\\
58.625	0.22842	-28.510265957284	-28.510265957284\\
58.625	0.23208	-31.2436170822472	-31.2436170822472\\
58.625	0.23574	-34.0803636577381	-34.0803636577381\\
58.625	0.2394	-37.0205056837566	-37.0205056837566\\
58.625	0.24306	-40.0640431603025	-40.0640431603025\\
58.625	0.24672	-43.2109760873762	-43.2109760873762\\
58.625	0.25038	-46.4613044649774	-46.4613044649774\\
58.625	0.25404	-49.8150282931062	-49.8150282931062\\
58.625	0.2577	-53.2721475717627	-53.2721475717627\\
58.625	0.26136	-56.8326623009468	-56.8326623009468\\
58.625	0.26502	-60.4965724806584	-60.4965724806584\\
58.625	0.26868	-64.2638781108976	-64.2638781108976\\
58.625	0.27234	-68.1345791916644	-68.1345791916644\\
58.625	0.276	-72.1086757229588	-72.1086757229588\\
59	0.093	-0.212415328535805	-0.212415328535805\\
59	0.09666	0.912129709910552	0.912129709910552\\
59	0.10032	1.93327929782924	1.93327929782924\\
59	0.10398	2.85103343522034	2.85103343522034\\
59	0.10764	3.66539212208384	3.66539212208384\\
59	0.1113	4.37635535841986	4.37635535841986\\
59	0.11496	4.98392314422816	4.98392314422816\\
59	0.11862	5.48809547950886	5.48809547950886\\
59	0.12228	5.88887236426202	5.88887236426202\\
59	0.12594	6.18625379848753	6.18625379848753\\
59	0.1296	6.3802397821855	6.3802397821855\\
59	0.13326	6.47083031535581	6.47083031535581\\
59	0.13692	6.45802539799853	6.45802539799853\\
59	0.14058	6.34182503011367	6.34182503011367\\
59	0.14424	6.12222921170122	6.12222921170122\\
59	0.1479	5.79923794276114	5.79923794276114\\
59	0.15156	5.37285122329345	5.37285122329345\\
59	0.15522	4.84306905329817	4.84306905329817\\
59	0.15888	4.20989143277536	4.20989143277536\\
59	0.16254	3.47331836172488	3.47331836172488\\
59	0.1662	2.63334984014681	2.63334984014681\\
59	0.16986	1.68998586804118	1.68998586804118\\
59	0.17352	0.643226445407961	0.643226445407961\\
59	0.17718	-0.506928427752925	-0.506928427752925\\
59	0.18084	-1.76047875144138	-1.76047875144138\\
59	0.1845	-3.11742452565741	-3.11742452565741\\
59	0.18816	-4.57776575040106	-4.57776575040106\\
59	0.19182	-6.14150242567234	-6.14150242567234\\
59	0.19548	-7.80863455147121	-7.80863455147121\\
59	0.19914	-9.5791621277977	-9.5791621277977\\
59	0.2028	-11.4530851546517	-11.4530851546517\\
59	0.20646	-13.4304036320334	-13.4304036320334\\
59	0.21012	-15.5111175599426	-15.5111175599426\\
59	0.21378	-17.6952269383794	-17.6952269383794\\
59	0.21744	-19.9827317673439	-19.9827317673439\\
59	0.2211	-22.3736320468359	-22.3736320468359\\
59	0.22476	-24.8679277768556	-24.8679277768556\\
59	0.22842	-27.4656189574028	-27.4656189574028\\
59	0.23208	-30.1667055884776	-30.1667055884776\\
59	0.23574	-32.9711876700801	-32.9711876700801\\
59	0.2394	-35.8790652022102	-35.8790652022102\\
59	0.24306	-38.8903381848678	-38.8903381848678\\
59	0.24672	-42.0050066180531	-42.0050066180531\\
59	0.25038	-45.2230705017659	-45.2230705017659\\
59	0.25404	-48.5445298360063	-48.5445298360063\\
59	0.2577	-51.9693846207744	-51.9693846207744\\
59	0.26136	-55.49763485607	-55.49763485607\\
59	0.26502	-59.1292805418933	-59.1292805418933\\
59	0.26868	-62.864321678244	-62.864321678244\\
59	0.27234	-66.7027582651225	-66.7027582651225\\
59	0.276	-70.6445903025285	-70.6445903025285\\
59.375	0.093	-0.382294558582238	-0.382294558582238\\
59.375	0.09666	0.774514973752481	0.774514973752481\\
59.375	0.10032	1.82792905555959	1.82792905555959\\
59.375	0.10398	2.77794768683911	2.77794768683911\\
59.375	0.10764	3.62457086759103	3.62457086759103\\
59.375	0.1113	4.36779859781535	4.36779859781535\\
59.375	0.11496	5.00763087751207	5.00763087751207\\
59.375	0.11862	5.5440677066812	5.5440677066812\\
59.375	0.12228	5.97710908532272	5.97710908532272\\
59.375	0.12594	6.30675501343664	6.30675501343664\\
59.375	0.1296	6.53300549102298	6.53300549102298\\
59.375	0.13326	6.65586051808171	6.65586051808171\\
59.375	0.13692	6.67532009461284	6.67532009461284\\
59.375	0.14058	6.59138422061635	6.59138422061635\\
59.375	0.14424	6.40405289609227	6.40405289609227\\
59.375	0.1479	6.1133261210406	6.1133261210406\\
59.375	0.15156	5.71920389546133	5.71920389546133\\
59.375	0.15522	5.22168621935441	5.22168621935441\\
59.375	0.15888	4.62077309271997	4.62077309271997\\
59.375	0.16254	3.91646451555791	3.91646451555791\\
59.375	0.1662	3.10876048786825	3.10876048786825\\
59.375	0.16986	2.19766100965099	2.19766100965099\\
59.375	0.17352	1.18316608090613	1.18316608090613\\
59.375	0.17718	0.0652757016336665	0.0652757016336665\\
59.375	0.18084	-1.15601012816637	-1.15601012816637\\
59.375	0.1845	-2.48069140849404	-2.48069140849404\\
59.375	0.18816	-3.90876813934932	-3.90876813934932\\
59.375	0.19182	-5.44024032073219	-5.44024032073219\\
59.375	0.19548	-7.07510795264264	-7.07510795264264\\
59.375	0.19914	-8.81337103508076	-8.81337103508076\\
59.375	0.2028	-10.6550295680464	-10.6550295680464\\
59.375	0.20646	-12.6000835515397	-12.6000835515397\\
59.375	0.21012	-14.6485329855605	-14.6485329855605\\
59.375	0.21378	-16.8003778701089	-16.8003778701089\\
59.375	0.21744	-19.055618205185	-19.055618205185\\
59.375	0.2211	-21.4142539907886	-21.4142539907886\\
59.375	0.22476	-23.8762852269199	-23.8762852269199\\
59.375	0.22842	-26.4417119135787	-26.4417119135787\\
59.375	0.23208	-29.1105340507652	-29.1105340507652\\
59.375	0.23574	-31.8827516384793	-31.8827516384793\\
59.375	0.2394	-34.7583646767209	-34.7583646767209\\
59.375	0.24306	-37.7373731654901	-37.7373731654901\\
59.375	0.24672	-40.819777104787	-40.819777104787\\
59.375	0.25038	-44.0055764946115	-44.0055764946115\\
59.375	0.25404	-47.2947713349635	-47.2947713349635\\
59.375	0.2577	-50.6873616258432	-50.6873616258432\\
59.375	0.26136	-54.1833473672504	-54.1833473672504\\
59.375	0.26502	-57.7827285591852	-57.7827285591852\\
59.375	0.26868	-61.4855052016476	-61.4855052016476\\
59.375	0.27234	-65.2916772946377	-65.2916772946377\\
59.375	0.276	-69.2012448381553	-69.2012448381553\\
59.75	0.093	-0.572913744685742	-0.572913744685742\\
59.75	0.09666	0.61616028153734	0.61616028153734\\
59.75	0.10032	1.70183885723286	1.70183885723286\\
59.75	0.10398	2.68412198240081	2.68412198240081\\
59.75	0.10764	3.56300965704109	3.56300965704109\\
59.75	0.1113	4.33850188115383	4.33850188115383\\
59.75	0.11496	5.01059865473891	5.01059865473891\\
59.75	0.11862	5.57929997779646	5.57929997779646\\
59.75	0.12228	6.0446058503264	6.0446058503264\\
59.75	0.12594	6.40651627232869	6.40651627232869\\
59.75	0.1296	6.66503124380338	6.66503124380338\\
59.75	0.13326	6.82015076475054	6.82015076475054\\
59.75	0.13692	6.87187483517009	6.87187483517009\\
59.75	0.14058	6.82020345506201	6.82020345506201\\
59.75	0.14424	6.6651366244263	6.6651366244263\\
59.75	0.1479	6.40667434326299	6.40667434326299\\
59.75	0.15156	6.04481661157214	6.04481661157214\\
59.75	0.15522	5.57956342935364	5.57956342935364\\
59.75	0.15888	5.01091479660757	5.01091479660757\\
59.75	0.16254	4.33887071333386	4.33887071333386\\
59.75	0.1662	3.56343117953263	3.56343117953263\\
59.75	0.16986	2.68459619520378	2.68459619520378\\
59.75	0.17352	1.70236576034735	1.70236576034735\\
59.75	0.17718	0.616739874963244	0.616739874963244\\
59.75	0.18084	-0.572281460948432	-0.572281460948432\\
59.75	0.1845	-1.86469824738768	-1.86469824738768\\
59.75	0.18816	-3.26051048435454	-3.26051048435454\\
59.75	0.19182	-4.75971817184899	-4.75971817184899\\
59.75	0.19548	-6.36232130987113	-6.36232130987113\\
59.75	0.19914	-8.06831989842078	-8.06831989842078\\
59.75	0.2028	-9.87771393749804	-9.87771393749804\\
59.75	0.20646	-11.7905034271029	-11.7905034271029\\
59.75	0.21012	-13.8066883672354	-13.8066883672354\\
59.75	0.21378	-15.9262687578954	-15.9262687578954\\
59.75	0.21744	-18.1492445990831	-18.1492445990831\\
59.75	0.2211	-20.4756158907983	-20.4756158907983\\
59.75	0.22476	-22.9053826330412	-22.9053826330412\\
59.75	0.22842	-25.4385448258116	-25.4385448258116\\
59.75	0.23208	-28.0751024691097	-28.0751024691097\\
59.75	0.23574	-30.8150555629354	-30.8150555629354\\
59.75	0.2394	-33.6584041072886	-33.6584041072886\\
59.75	0.24306	-36.6051481021694	-36.6051481021694\\
59.75	0.24672	-39.6552875475779	-39.6552875475779\\
59.75	0.25038	-42.808822443514	-42.808822443514\\
59.75	0.25404	-46.0657527899776	-46.0657527899776\\
59.75	0.2577	-49.4260785869689	-49.4260785869689\\
59.75	0.26136	-52.8897998344877	-52.8897998344877\\
59.75	0.26502	-56.4569165325342	-56.4569165325342\\
59.75	0.26868	-60.1274286811082	-60.1274286811082\\
59.75	0.27234	-63.9013362802098	-63.9013362802098\\
59.75	0.276	-67.7786393298391	-67.7786393298391\\
60.125	0.093	-0.78427288684626	-0.78427288684626\\
60.125	0.09666	0.437065633265185	0.437065633265185\\
60.125	0.10032	1.55500870284913	1.55500870284913\\
60.125	0.10398	2.56955632190538	2.56955632190538\\
60.125	0.10764	3.48070849043414	3.48070849043414\\
60.125	0.1113	4.28846520843524	4.28846520843524\\
60.125	0.11496	4.99282647590874	4.99282647590874\\
60.125	0.11862	5.59379229285465	5.59379229285465\\
60.125	0.12228	6.09136265927295	6.09136265927295\\
60.125	0.12594	6.48553757516372	6.48553757516372\\
60.125	0.1296	6.77631704052678	6.77631704052678\\
60.125	0.13326	6.96370105536235	6.96370105536235\\
60.125	0.13692	7.04768961967021	7.04768961967021\\
60.125	0.14058	7.02828273345055	7.02828273345055\\
60.125	0.14424	6.90548039670325	6.90548039670325\\
60.125	0.1479	6.67928260942837	6.67928260942837\\
60.125	0.15156	6.34968937162589	6.34968937162589\\
60.125	0.15522	5.9167006832958	5.9167006832958\\
60.125	0.15888	5.38031654443814	5.38031654443814\\
60.125	0.16254	4.7405369550528	4.7405369550528\\
60.125	0.1662	3.99736191513999	3.99736191513999\\
60.125	0.16986	3.15079142469945	3.15079142469945\\
60.125	0.17352	2.20082548373144	2.20082548373144\\
60.125	0.17718	1.14746409223575	1.14746409223575\\
60.125	0.18084	-0.00929274978750527	-0.00929274978750527\\
60.125	0.1845	-1.26944504233839	-1.26944504233839\\
60.125	0.18816	-2.63299278541689	-2.63299278541689\\
60.125	0.19182	-4.09993597902292	-4.09993597902292\\
60.125	0.19548	-5.67027462315664	-5.67027462315664\\
60.125	0.19914	-7.34400871781793	-7.34400871781793\\
60.125	0.2028	-9.12113826300677	-9.12113826300677\\
60.125	0.20646	-11.0016632587233	-11.0016632587233\\
60.125	0.21012	-12.9855837049673	-12.9855837049673\\
60.125	0.21378	-15.072899601739	-15.072899601739\\
60.125	0.21744	-17.2636109490382	-17.2636109490382\\
60.125	0.2211	-19.5577177468651	-19.5577177468651\\
60.125	0.22476	-21.9552199952196	-21.9552199952196\\
60.125	0.22842	-24.4561176941016	-24.4561176941016\\
60.125	0.23208	-27.0604108435113	-27.0604108435113\\
60.125	0.23574	-29.7680994434486	-29.7680994434486\\
60.125	0.2394	-32.5791834939134	-32.5791834939134\\
60.125	0.24306	-35.4936629949059	-35.4936629949059\\
60.125	0.24672	-38.511537946426	-38.511537946426\\
60.125	0.25038	-41.6328083484736	-41.6328083484736\\
60.125	0.25404	-44.8574742010489	-44.8574742010489\\
60.125	0.2577	-48.1855355041517	-48.1855355041517\\
60.125	0.26136	-51.6169922577822	-51.6169922577822\\
60.125	0.26502	-55.1518444619402	-55.1518444619402\\
60.125	0.26868	-58.7900921166258	-58.7900921166258\\
60.125	0.27234	-62.5317352218391	-62.5317352218391\\
60.125	0.276	-66.3767737775799	-66.3767737775799\\
60.5	0.093	-1.01637198506385	-1.01637198506385\\
60.5	0.09666	0.237231028936073	0.237231028936073\\
60.5	0.10032	1.38743859240832	1.38743859240832\\
60.5	0.10398	2.43425070535299	2.43425070535299\\
60.5	0.10764	3.37766736777012	3.37766736777012\\
60.5	0.1113	4.21768857965964	4.21768857965964\\
60.5	0.11496	4.95431434102156	4.95431434102156\\
60.5	0.11862	5.58754465185589	5.58754465185589\\
60.5	0.12228	6.11737951216255	6.11737951216255\\
60.5	0.12594	6.54381892194162	6.54381892194162\\
60.5	0.1296	6.86686288119321	6.86686288119321\\
60.5	0.13326	7.08651138991709	7.08651138991709\\
60.5	0.13692	7.20276444811337	7.20276444811337\\
60.5	0.14058	7.21562205578213	7.21562205578213\\
60.5	0.14424	7.1250842129232	7.1250842129232\\
60.5	0.1479	6.93115091953668	6.93115091953668\\
60.5	0.15156	6.63382217562261	6.63382217562261\\
60.5	0.15522	6.23309798118095	6.23309798118095\\
60.5	0.15888	5.7289783362116	5.7289783362116\\
60.5	0.16254	5.12146324071479	5.12146324071479\\
60.5	0.1662	4.41055269469028	4.41055269469028\\
60.5	0.16986	3.59624669813816	3.59624669813816\\
60.5	0.17352	2.67854525105851	2.67854525105851\\
60.5	0.17718	1.65744835345124	1.65744835345124\\
60.5	0.18084	0.53295600531635	0.53295600531635\\
60.5	0.1845	-0.694931793346115	-0.694931793346115\\
60.5	0.18816	-2.02621504253619	-2.02621504253619\\
60.5	0.19182	-3.46089374225392	-3.46089374225392\\
60.5	0.19548	-4.99896789249911	-4.99896789249911\\
60.5	0.19914	-6.64043749327209	-6.64043749327209\\
60.5	0.2028	-8.38530254457251	-8.38530254457251\\
60.5	0.20646	-10.2335630464006	-10.2335630464006\\
60.5	0.21012	-12.1852189987563	-12.1852189987563\\
60.5	0.21378	-14.2402704016396	-14.2402704016396\\
60.5	0.21744	-16.3987172550504	-16.3987172550504\\
60.5	0.2211	-18.6605595589889	-18.6605595589889\\
60.5	0.22476	-21.025797313455	-21.025797313455\\
60.5	0.22842	-23.4944305184486	-23.4944305184486\\
60.5	0.23208	-26.0664591739698	-26.0664591739698\\
60.5	0.23574	-28.7418832800188	-28.7418832800188\\
60.5	0.2394	-31.5207028365953	-31.5207028365953\\
60.5	0.24306	-34.4029178436993	-34.4029178436993\\
60.5	0.24672	-37.388528301331	-37.388528301331\\
60.5	0.25038	-40.4775342094902	-40.4775342094902\\
60.5	0.25404	-43.6699355681771	-43.6699355681771\\
60.5	0.2577	-46.9657323773916	-46.9657323773916\\
60.5	0.26136	-50.3649246371336	-50.3649246371336\\
60.5	0.26502	-53.8675123474032	-53.8675123474032\\
60.5	0.26868	-57.4734955082005	-57.4734955082005\\
60.5	0.27234	-61.1828741195254	-61.1828741195254\\
60.5	0.276	-64.9956481813778	-64.9956481813778\\
60.875	0.093	-1.26921103933845	-1.26921103933845\\
60.875	0.09666	0.0166564685498329	0.0166564685498329\\
60.875	0.10032	1.1991285259105	1.1991285259105\\
60.875	0.10398	2.27820513274359	2.27820513274359\\
60.875	0.10764	3.25388628904908	3.25388628904908\\
60.875	0.1113	4.12617199482696	4.12617199482696\\
60.875	0.11496	4.89506225007725	4.89506225007725\\
60.875	0.11862	5.56055705479994	5.56055705479994\\
60.875	0.12228	6.12265640899508	6.12265640899508\\
60.875	0.12594	6.58136031266257	6.58136031266257\\
60.875	0.1296	6.93666876580247	6.93666876580247\\
60.875	0.13326	7.18858176841482	7.18858176841482\\
60.875	0.13692	7.33709932049952	7.33709932049952\\
60.875	0.14058	7.38222142205659	7.38222142205659\\
60.875	0.14424	7.32394807308607	7.32394807308607\\
60.875	0.1479	7.16227927358797	7.16227927358797\\
60.875	0.15156	6.89721502356227	6.89721502356227\\
60.875	0.15522	6.52875532300897	6.52875532300897\\
60.875	0.15888	6.05690017192809	6.05690017192809\\
60.875	0.16254	5.48164957031959	5.48164957031959\\
60.875	0.1662	4.8030035181835	4.8030035181835\\
60.875	0.16986	4.02096201551986	4.02096201551986\\
60.875	0.17352	3.13552506232857	3.13552506232857\\
60.875	0.17718	2.14669265860967	2.14669265860967\\
60.875	0.18084	1.05446480436319	1.05446480436319\\
60.875	0.1845	-0.14115850041091	-0.14115850041091\\
60.875	0.18816	-1.44017725571263	-1.44017725571263\\
60.875	0.19182	-2.84259146154193	-2.84259146154193\\
60.875	0.19548	-4.34840111789882	-4.34840111789882\\
60.875	0.19914	-5.95760622478332	-5.95760622478332\\
60.875	0.2028	-7.67020678219538	-7.67020678219538\\
60.875	0.20646	-9.48620279013508	-9.48620279013508\\
60.875	0.21012	-11.4055942486023	-11.4055942486023\\
60.875	0.21378	-13.4283811575972	-13.4283811575972\\
60.875	0.21744	-15.5545635171196	-15.5545635171196\\
60.875	0.2211	-17.7841413271697	-17.7841413271697\\
60.875	0.22476	-20.1171145877474	-20.1171145877474\\
60.875	0.22842	-22.5534832988527	-22.5534832988527\\
60.875	0.23208	-25.0932474604855	-25.0932474604855\\
60.875	0.23574	-27.736407072646	-27.736407072646\\
60.875	0.2394	-30.4829621353341	-30.4829621353341\\
60.875	0.24306	-33.3329126485498	-33.3329126485498\\
60.875	0.24672	-36.2862586122931	-36.2862586122931\\
60.875	0.25038	-39.3430000265639	-39.3430000265639\\
60.875	0.25404	-42.5031368913624	-42.5031368913624\\
60.875	0.2577	-45.7666692066885	-45.7666692066885\\
60.875	0.26136	-49.1335969725422	-49.1335969725422\\
60.875	0.26502	-52.6039201889234	-52.6039201889234\\
60.875	0.26868	-56.1776388558322	-56.1776388558322\\
60.875	0.27234	-59.8547529732687	-59.8547529732687\\
60.875	0.276	-63.6352625412327	-63.6352625412327\\
61.25	0.093	-1.54279004967007	-1.54279004967007\\
61.25	0.09666	-0.224658047893364	-0.224658047893364\\
61.25	0.10032	0.990078503355669	0.990078503355669\\
61.25	0.10398	2.10141960407718	2.10141960407718\\
61.25	0.10764	3.10936525427108	3.10936525427108\\
61.25	0.1113	4.01391545393739	4.01391545393739\\
61.25	0.11496	4.81507020307609	4.81507020307609\\
61.25	0.11862	5.5128295016872	5.5128295016872\\
61.25	0.12228	6.10719334977065	6.10719334977065\\
61.25	0.12594	6.59816174732656	6.59816174732656\\
61.25	0.1296	6.98573469435482	6.98573469435482\\
61.25	0.13326	7.26991219085554	7.26991219085554\\
61.25	0.13692	7.45069423682865	7.45069423682865\\
61.25	0.14058	7.52808083227414	7.52808083227414\\
61.25	0.14424	7.50207197719205	7.50207197719205\\
61.25	0.1479	7.37266767158236	7.37266767158236\\
61.25	0.15156	7.13986791544502	7.13986791544502\\
61.25	0.15522	6.80367270878008	6.80367270878008\\
61.25	0.15888	6.36408205158763	6.36408205158763\\
61.25	0.16254	5.82109594386749	5.82109594386749\\
61.25	0.1662	5.17471438561982	5.17471438561982\\
61.25	0.16986	4.42493737684454	4.42493737684454\\
61.25	0.17352	3.57176491754167	3.57176491754167\\
61.25	0.17718	2.61519700771119	2.61519700771119\\
61.25	0.18084	1.55523364735308	1.55523364735308\\
61.25	0.1845	0.391874836467395	0.391874836467395\\
61.25	0.18816	-0.874879424945959	-0.874879424945959\\
61.25	0.19182	-2.24502913688684	-2.24502913688684\\
61.25	0.19548	-3.71857429935531	-3.71857429935531\\
61.25	0.19914	-5.29551491235145	-5.29551491235145\\
61.25	0.2028	-6.97585097587509	-6.97585097587509\\
61.25	0.20646	-8.75958248992637	-8.75958248992637\\
61.25	0.21012	-10.6467094545053	-10.6467094545053\\
61.25	0.21378	-12.6372318696118	-12.6372318696118\\
61.25	0.21744	-14.7311497352458	-14.7311497352458\\
61.25	0.2211	-16.9284630514075	-16.9284630514075\\
61.25	0.22476	-19.2291718180968	-19.2291718180968\\
61.25	0.22842	-21.6332760353137	-21.6332760353137\\
61.25	0.23208	-24.1407757030581	-24.1407757030581\\
61.25	0.23574	-26.7516708213302	-26.7516708213302\\
61.25	0.2394	-29.4659613901299	-29.4659613901299\\
61.25	0.24306	-32.2836474094571	-32.2836474094571\\
61.25	0.24672	-35.2047288793121	-35.2047288793121\\
61.25	0.25038	-38.2292057996945	-38.2292057996945\\
61.25	0.25404	-41.3570781706046	-41.3570781706046\\
61.25	0.2577	-44.5883459920423	-44.5883459920423\\
61.25	0.26136	-47.9230092640076	-47.9230092640076\\
61.25	0.26502	-51.3610679865004	-51.3610679865004\\
61.25	0.26868	-54.9025221595209	-54.9025221595209\\
61.25	0.27234	-58.547371783069	-58.547371783069\\
61.25	0.276	-62.2956168571446	-62.2956168571446\\
61.625	0.093	-1.83710901605875	-1.83710901605875\\
61.625	0.09666	-0.486712520393688	-0.486712520393688\\
61.625	0.10032	0.760288524743764	0.760288524743764\\
61.625	0.10398	1.90389411935364	1.90389411935364\\
61.625	0.10764	2.9441042634359	2.9441042634359\\
61.625	0.1113	3.88091895699063	3.88091895699063\\
61.625	0.11496	4.71433820001775	4.71433820001775\\
61.625	0.11862	5.44436199251717	5.44436199251717\\
61.625	0.12228	6.07099033448909	6.07099033448909\\
61.625	0.12594	6.59422322593336	6.59422322593336\\
61.625	0.1296	7.01406066685004	7.01406066685004\\
61.625	0.13326	7.33050265723918	7.33050265723918\\
61.625	0.13692	7.54354919710066	7.54354919710066\\
61.625	0.14058	7.65320028643451	7.65320028643451\\
61.625	0.14424	7.65945592524083	7.65945592524083\\
61.625	0.1479	7.56231611351951	7.56231611351951\\
61.625	0.15156	7.36178085127059	7.36178085127059\\
61.625	0.15522	7.05785013849408	7.05785013849408\\
61.625	0.15888	6.65052397518998	6.65052397518998\\
61.625	0.16254	6.13980236135826	6.13980236135826\\
61.625	0.1662	5.52568529699901	5.52568529699901\\
61.625	0.16986	4.80817278211209	4.80817278211209\\
61.625	0.17352	3.98726481669759	3.98726481669759\\
61.625	0.17718	3.06296140075547	3.06296140075547\\
61.625	0.18084	2.03526253428583	2.03526253428583\\
61.625	0.1845	0.904168217288515	0.904168217288515\\
61.625	0.18816	-0.330321550236363	-0.330321550236363\\
61.625	0.19182	-1.66820676828894	-1.66820676828894\\
61.625	0.19548	-3.10948743686899	-3.10948743686899\\
61.625	0.19914	-4.65416355597671	-4.65416355597671\\
61.625	0.2028	-6.30223512561199	-6.30223512561199\\
61.625	0.20646	-8.05370214577491	-8.05370214577491\\
61.625	0.21012	-9.90856461646538	-9.90856461646538\\
61.625	0.21378	-11.8668225376835	-11.8668225376835\\
61.625	0.21744	-13.9284759094291	-13.9284759094291\\
61.625	0.2211	-16.0935247317024	-16.0935247317024\\
61.625	0.22476	-18.3619690045034	-18.3619690045034\\
61.625	0.22842	-20.7338087278318	-20.7338087278318\\
61.625	0.23208	-23.2090439016879	-23.2090439016879\\
61.625	0.23574	-25.7876745260716	-25.7876745260716\\
61.625	0.2394	-28.4697006009829	-28.4697006009829\\
61.625	0.24306	-31.2551221264218	-31.2551221264218\\
61.625	0.24672	-34.1439391023882	-34.1439391023882\\
61.625	0.25038	-37.1361515288823	-37.1361515288823\\
61.625	0.25404	-40.231759405904	-40.231759405904\\
61.625	0.2577	-43.4307627334533	-43.4307627334533\\
61.625	0.26136	-46.7331615115302	-46.7331615115302\\
61.625	0.26502	-50.1389557401346	-50.1389557401346\\
61.625	0.26868	-53.6481454192667	-53.6481454192667\\
61.625	0.27234	-57.2607305489264	-57.2607305489264\\
61.625	0.276	-60.9767111291137	-60.9767111291137\\
62	0.093	-2.15216793850451	-2.15216793850451\\
62	0.09666	-0.769506948951026	-0.769506948951026\\
62	0.10032	0.509758590074789	0.509758590074789\\
62	0.10398	1.68562867857314	1.68562867857314\\
62	0.10764	2.75810331654377	2.75810331654377\\
62	0.1113	3.72718250398691	3.72718250398691\\
62	0.11496	4.59286624090234	4.59286624090234\\
62	0.11862	5.35515452729023	5.35515452729023\\
62	0.12228	6.01404736315052	6.01404736315052\\
62	0.12594	6.56954474848315	6.56954474848315\\
62	0.1296	7.02164668328831	7.02164668328831\\
62	0.13326	7.37035316756575	7.37035316756575\\
62	0.13692	7.61566420131565	7.61566420131565\\
62	0.14058	7.75757978453792	7.75757978453792\\
62	0.14424	7.79609991723261	7.79609991723261\\
62	0.1479	7.73122459939971	7.73122459939971\\
62	0.15156	7.56295383103915	7.56295383103915\\
62	0.15522	7.29128761215105	7.29128761215105\\
62	0.15888	6.91622594273532	6.91622594273532\\
62	0.16254	6.43776882279202	6.43776882279202\\
62	0.1662	5.85591625232114	5.85591625232114\\
62	0.16986	5.17066823132258	5.17066823132258\\
62	0.17352	4.38202475979655	4.38202475979655\\
62	0.17718	3.48998583774279	3.48998583774279\\
62	0.18084	2.49455146516152	2.49455146516152\\
62	0.1845	1.39572164205262	1.39572164205262\\
62	0.18816	0.193496368416163	0.193496368416163\\
62	0.19182	-1.11212435574799	-1.11212435574799\\
62	0.19548	-2.52114053043974	-2.52114053043974\\
62	0.19914	-4.03355215565904	-4.03355215565904\\
62	0.2028	-5.64935923140595	-5.64935923140595\\
62	0.20646	-7.36856175768045	-7.36856175768045\\
62	0.21012	-9.19115973448251	-9.19115973448251\\
62	0.21378	-11.1171531618122	-11.1171531618122\\
62	0.21744	-13.1465420396695	-13.1465420396695\\
62	0.2211	-15.2793263680544	-15.2793263680544\\
62	0.22476	-17.5155061469669	-17.5155061469669\\
62	0.22842	-19.855081376407	-19.855081376407\\
62	0.23208	-22.2980520563746	-22.2980520563746\\
62	0.23574	-24.84441818687	-24.84441818687\\
62	0.2394	-27.4941797678929	-27.4941797678929\\
62	0.24306	-30.2473367994433	-30.2473367994433\\
62	0.24672	-33.1038892815214	-33.1038892815214\\
62	0.25038	-36.0638372141272	-36.0638372141272\\
62	0.25404	-39.1271805972604	-39.1271805972604\\
62	0.2577	-42.2939194309213	-42.2939194309213\\
62	0.26136	-45.5640537151098	-45.5640537151098\\
62	0.26502	-48.9375834498259	-48.9375834498259\\
62	0.26868	-52.4145086350695	-52.4145086350695\\
62	0.27234	-55.9948292708409	-55.9948292708409\\
62	0.276	-59.6785453571397	-59.6785453571397\\
62.375	0.093	-2.48796681700728	-2.48796681700728\\
62.375	0.09666	-1.07304133356543	-1.07304133356543\\
62.375	0.10032	0.2384886993488	0.2384886993488\\
62.375	0.10398	1.44662328173546	1.44662328173546\\
62.375	0.10764	2.5513624135945	2.5513624135945\\
62.375	0.1113	3.55270609492607	3.55270609492607\\
62.375	0.11496	4.45065432572991	4.45065432572991\\
62.375	0.11862	5.24520710600617	5.24520710600617\\
62.375	0.12228	5.93636443575482	5.93636443575482\\
62.375	0.12594	6.52412631497593	6.52412631497593\\
62.375	0.1296	7.00849274366939	7.00849274366939\\
62.375	0.13326	7.38946372183531	7.38946372183531\\
62.375	0.13692	7.66703924947358	7.66703924947358\\
62.375	0.14058	7.84121932658421	7.84121932658421\\
62.375	0.14424	7.91200395316731	7.91200395316731\\
62.375	0.1479	7.87939312922278	7.87939312922278\\
62.375	0.15156	7.7433868547507	7.7433868547507\\
62.375	0.15522	7.50398512975096	7.50398512975096\\
62.375	0.15888	7.16118795422359	7.16118795422359\\
62.375	0.16254	6.71499532816871	6.71499532816871\\
62.375	0.1662	6.16540725158619	6.16540725158619\\
62.375	0.16986	5.51242372447611	5.51242372447611\\
62.375	0.17352	4.75604474683838	4.75604474683838\\
62.375	0.17718	3.89627031867305	3.89627031867305\\
62.375	0.18084	2.93310043998019	2.93310043998019\\
62.375	0.1845	1.86653511075966	1.86653511075966\\
62.375	0.18816	0.696574331011561	0.696574331011561\\
62.375	0.19182	-0.576781899264176	-0.576781899264176\\
62.375	0.19548	-1.9535335800675	-1.9535335800675\\
62.375	0.19914	-3.43368071139844	-3.43368071139844\\
62.375	0.2028	-5.01722329325693	-5.01722329325693\\
62.375	0.20646	-6.70416132564301	-6.70416132564301\\
62.375	0.21012	-8.49449480855671	-8.49449480855671\\
62.375	0.21378	-10.388223741998	-10.388223741998\\
62.375	0.21744	-12.3853481259669	-12.3853481259669\\
62.375	0.2211	-14.4858679604634	-14.4858679604634\\
62.375	0.22476	-16.6897832454875	-16.6897832454875\\
62.375	0.22842	-18.9970939810392	-18.9970939810392\\
62.375	0.23208	-21.4078001671185	-21.4078001671185\\
62.375	0.23574	-23.9219018037255	-23.9219018037255\\
62.375	0.2394	-26.5393988908599	-26.5393988908599\\
62.375	0.24306	-29.260291428522	-29.260291428522\\
62.375	0.24672	-32.0845794167117	-32.0845794167117\\
62.375	0.25038	-35.012262855429	-35.012262855429\\
62.375	0.25404	-38.0433417446739	-38.0433417446739\\
62.375	0.2577	-41.1778160844464	-41.1778160844464\\
62.375	0.26136	-44.4156858747465	-44.4156858747465\\
62.375	0.26502	-47.7569511155742	-47.7569511155742\\
62.375	0.26868	-51.2016118069295	-51.2016118069295\\
62.375	0.27234	-54.7496679488124	-54.7496679488124\\
62.375	0.276	-58.4011195412228	-58.4011195412228\\
62.75	0.093	-2.84450565156707	-2.84450565156707\\
62.75	0.09666	-1.3973156742368	-1.3973156742368\\
62.75	0.10032	-0.0535211474342034	-0.0535211474342034\\
62.75	0.10398	1.18687792884087	1.18687792884087\\
62.75	0.10764	2.32388155458834	2.32388155458834\\
62.75	0.1113	3.35748972980821	3.35748972980821\\
62.75	0.11496	4.28770245450048	4.28770245450048\\
62.75	0.11862	5.11451972866515	5.11451972866515\\
62.75	0.12228	5.83794155230222	5.83794155230222\\
62.75	0.12594	6.45796792541169	6.45796792541169\\
62.75	0.1296	6.97459884799358	6.97459884799358\\
62.75	0.13326	7.38783432004786	7.38783432004786\\
62.75	0.13692	7.69767434157454	7.69767434157454\\
62.75	0.14058	7.90411891257359	7.90411891257359\\
62.75	0.14424	8.00716803304506	8.00716803304506\\
62.75	0.1479	8.00682170298894	8.00682170298894\\
62.75	0.15156	7.90307992240523	7.90307992240523\\
62.75	0.15522	7.69594269129385	7.69594269129385\\
62.75	0.15888	7.38541000965496	7.38541000965496\\
62.75	0.16254	6.97148187748844	6.97148187748844\\
62.75	0.1662	6.45415829479434	6.45415829479434\\
62.75	0.16986	5.83343926157262	5.83343926157262\\
62.75	0.17352	5.10932477782332	5.10932477782332\\
62.75	0.17718	4.28181484354634	4.28181484354634\\
62.75	0.18084	3.35090945874185	3.35090945874185\\
62.75	0.1845	2.31660862340973	2.31660862340973\\
62.75	0.18816	1.17891233755006	1.17891233755006\\
62.75	0.19182	-0.0621793988372588	-0.0621793988372588\\
62.75	0.19548	-1.40666658575216	-1.40666658575216\\
62.75	0.19914	-2.85454922319479	-2.85454922319479\\
62.75	0.2028	-4.40582731116487	-4.40582731116487\\
62.75	0.20646	-6.06050084966259	-6.06050084966259\\
62.75	0.21012	-7.81856983868786	-7.81856983868786\\
62.75	0.21378	-9.68003427824078	-9.68003427824078\\
62.75	0.21744	-11.6448941683213	-11.6448941683213\\
62.75	0.2211	-13.7131495089294	-13.7131495089294\\
62.75	0.22476	-15.8848003000651	-15.8848003000651\\
62.75	0.22842	-18.1598465417284	-18.1598465417284\\
62.75	0.23208	-20.5382882339193	-20.5382882339193\\
62.75	0.23574	-23.0201253766379	-23.0201253766379\\
62.75	0.2394	-25.605357969884	-25.605357969884\\
62.75	0.24306	-28.2939860136576	-28.2939860136576\\
62.75	0.24672	-31.0860095079589	-31.0860095079589\\
62.75	0.25038	-33.9814284527879	-33.9814284527879\\
62.75	0.25404	-36.9802428481443	-36.9802428481443\\
62.75	0.2577	-40.0824526940285	-40.0824526940285\\
62.75	0.26136	-43.2880579904402	-43.2880579904402\\
62.75	0.26502	-46.5970587373794	-46.5970587373794\\
62.75	0.26868	-50.0094549348463	-50.0094549348463\\
62.75	0.27234	-53.5252465828408	-53.5252465828408\\
62.75	0.276	-57.1444336813628	-57.1444336813628\\
63.125	0.093	-3.22178444218392	-3.22178444218392\\
63.125	0.09666	-1.74232997096529	-1.74232997096529\\
63.125	0.10032	-0.366270950274277	-0.366270950274277\\
63.125	0.10398	0.906392619889218	0.906392619889218\\
63.125	0.10764	2.07566073952505	2.07566073952505\\
63.125	0.1113	3.14153340863334	3.14153340863334\\
63.125	0.11496	4.10401062721397	4.10401062721397\\
63.125	0.11862	4.96309239526706	4.96309239526706\\
63.125	0.12228	5.71877871279249	5.71877871279249\\
63.125	0.12594	6.37106957979033	6.37106957979033\\
63.125	0.1296	6.91996499626063	6.91996499626063\\
63.125	0.13326	7.36546496220333	7.36546496220333\\
63.125	0.13692	7.70756947761838	7.70756947761838\\
63.125	0.14058	7.94627854250579	7.94627854250579\\
63.125	0.14424	8.08159215686568	8.08159215686568\\
63.125	0.1479	8.11351032069798	8.11351032069798\\
63.125	0.15156	8.04203303400263	8.04203303400263\\
63.125	0.15522	7.86716029677967	7.86716029677967\\
63.125	0.15888	7.58889210902915	7.58889210902915\\
63.125	0.16254	7.20722847075105	7.20722847075105\\
63.125	0.1662	6.72216938194531	6.72216938194531\\
63.125	0.16986	6.13371484261195	6.13371484261195\\
63.125	0.17352	5.44186485275107	5.44186485275107\\
63.125	0.17718	4.64661941236251	4.64661941236251\\
63.125	0.18084	3.74797852144644	3.74797852144644\\
63.125	0.1845	2.74594218000269	2.74594218000269\\
63.125	0.18816	1.64051038803137	1.64051038803137\\
63.125	0.19182	0.431683145532475	0.431683145532475\\
63.125	0.19548	-0.880539547494124	-0.880539547494124\\
63.125	0.19914	-2.29615769104822	-2.29615769104822\\
63.125	0.2028	-3.81517128512994	-3.81517128512994\\
63.125	0.20646	-5.43758032973929	-5.43758032973929\\
63.125	0.21012	-7.16338482487615	-7.16338482487615\\
63.125	0.21378	-8.9925847705407	-8.9925847705407\\
63.125	0.21744	-10.9251801667328	-10.9251801667328\\
63.125	0.2211	-12.9611710134525	-12.9611710134525\\
63.125	0.22476	-15.1005573106999	-15.1005573106999\\
63.125	0.22842	-17.3433390584747	-17.3433390584747\\
63.125	0.23208	-19.6895162567772	-19.6895162567772\\
63.125	0.23574	-22.1390889056073	-22.1390889056073\\
63.125	0.2394	-24.6920570049651	-24.6920570049651\\
63.125	0.24306	-27.3484205548504	-27.3484205548504\\
63.125	0.24672	-30.1081795552633	-30.1081795552633\\
63.125	0.25038	-32.9713340062038	-32.9713340062038\\
63.125	0.25404	-35.9378839076719	-35.9378839076719\\
63.125	0.2577	-39.0078292596676	-39.0078292596676\\
63.125	0.26136	-42.1811700621909	-42.1811700621909\\
63.125	0.26502	-45.4579063152419	-45.4579063152419\\
63.125	0.26868	-48.8380380188203	-48.8380380188203\\
63.125	0.27234	-52.3215651729264	-52.3215651729264\\
63.125	0.276	-55.9084877775601	-55.9084877775601\\
63.5	0.093	-3.61980318885779	-3.61980318885779\\
63.5	0.09666	-2.10808422375074	-2.10808422375074\\
63.5	0.10032	-0.699760709171365	-0.699760709171365\\
63.5	0.10398	0.60516735488055	0.60516735488055\\
63.5	0.10764	1.80669996840474	1.80669996840474\\
63.5	0.1113	2.90483713140145	2.90483713140145\\
63.5	0.11496	3.8995788438705	3.8995788438705\\
63.5	0.11862	4.79092510581196	4.79092510581196\\
63.5	0.12228	5.57887591722581	5.57887591722581\\
63.5	0.12594	6.26343127811207	6.26343127811207\\
63.5	0.1296	6.84459118847073	6.84459118847073\\
63.5	0.13326	7.32235564830179	7.32235564830179\\
63.5	0.13692	7.69672465760526	7.69672465760526\\
63.5	0.14058	7.96769821638109	7.96769821638109\\
63.5	0.14424	8.13527632462934	8.13527632462934\\
63.5	0.1479	8.19945898235001	8.19945898235001\\
63.5	0.15156	8.16024618954307	8.16024618954307\\
63.5	0.15522	8.01763794620854	8.01763794620854\\
63.5	0.15888	7.77163425234643	7.77163425234643\\
63.5	0.16254	7.42223510795664	7.42223510795664\\
63.5	0.1662	6.96944051303937	6.96944051303937\\
63.5	0.16986	6.41325046759438	6.41325046759438\\
63.5	0.17352	5.75366497162186	5.75366497162186\\
63.5	0.17718	4.99068402512172	4.99068402512172\\
63.5	0.18084	4.12430762809402	4.12430762809402\\
63.5	0.1845	3.15453578053868	3.15453578053868\\
63.5	0.18816	2.08136848245579	2.08136848245579\\
63.5	0.19182	0.904805733845194	0.904805733845194\\
63.5	0.19548	-0.375152465292871	-0.375152465292871\\
63.5	0.19914	-1.75850611495866	-1.75850611495866\\
63.5	0.2028	-3.24525521515196	-3.24525521515196\\
63.5	0.20646	-4.83539976587289	-4.83539976587289\\
63.5	0.21012	-6.52893976712139	-6.52893976712139\\
63.5	0.21378	-8.32587521889752	-8.32587521889752\\
63.5	0.21744	-10.2262061212012	-10.2262061212012\\
63.5	0.2211	-12.2299324740326	-12.2299324740326\\
63.5	0.22476	-14.3370542773915	-14.3370542773915\\
63.5	0.22842	-16.547571531278	-16.547571531278\\
63.5	0.23208	-18.8614842356921	-18.8614842356921\\
63.5	0.23574	-21.2787923906339	-21.2787923906339\\
63.5	0.2394	-23.7994959961032	-23.7994959961032\\
63.5	0.24306	-26.4235950521001	-26.4235950521001\\
63.5	0.24672	-29.1510895586246	-29.1510895586246\\
63.5	0.25038	-31.9819795156767	-31.9819795156767\\
63.5	0.25404	-34.9162649232564	-34.9162649232564\\
63.5	0.2577	-37.9539457813637	-37.9539457813637\\
63.5	0.26136	-41.0950220899987	-41.0950220899987\\
63.5	0.26502	-44.3394938491612	-44.3394938491612\\
63.5	0.26868	-47.6873610588513	-47.6873610588513\\
63.5	0.27234	-51.138623719069	-51.138623719069\\
63.5	0.276	-54.6932818298142	-54.6932818298142\\
63.875	0.093	-4.03856189158868	-4.03856189158868\\
63.875	0.09666	-2.49457843259326	-2.49457843259326\\
63.875	0.10032	-1.05399042412547	-1.05399042412547\\
63.875	0.10398	0.283202133814754	0.283202133814754\\
63.875	0.10764	1.51699924122748	1.51699924122748\\
63.875	0.1113	2.64740089811249	2.64740089811249\\
63.875	0.11496	3.67440710446991	3.67440710446991\\
63.875	0.11862	4.59801786029978	4.59801786029978\\
63.875	0.12228	5.41823316560205	5.41823316560205\\
63.875	0.12594	6.13505302037667	6.13505302037667\\
63.875	0.1296	6.74847742462376	6.74847742462376\\
63.875	0.13326	7.25850637834324	7.25850637834324\\
63.875	0.13692	7.66513988153507	7.66513988153507\\
63.875	0.14058	7.96837793419927	7.96837793419927\\
63.875	0.14424	8.16822053633594	8.16822053633594\\
63.875	0.1479	8.26466768794502	8.26466768794502\\
63.875	0.15156	8.25771938902645	8.25771938902645\\
63.875	0.15522	8.14737563958028	8.14737563958028\\
63.875	0.15888	7.93363643960659	7.93363643960659\\
63.875	0.16254	7.61650178910521	7.61650178910521\\
63.875	0.1662	7.19597168807626	7.19597168807626\\
63.875	0.16986	6.67204613651974	6.67204613651974\\
63.875	0.17352	6.04472513443564	6.04472513443564\\
63.875	0.17718	5.31400868182386	5.31400868182386\\
63.875	0.18084	4.47989677868458	4.47989677868458\\
63.875	0.1845	3.5423894250176	3.5423894250176\\
63.875	0.18816	2.50148662082307	2.50148662082307\\
63.875	0.19182	1.3571883661009	1.3571883661009\\
63.875	0.19548	0.109494660851254	0.109494660851254\\
63.875	0.19914	-1.24159449492618	-1.24159449492618\\
63.875	0.2028	-2.69607910123111	-2.69607910123111\\
63.875	0.20646	-4.25395915806362	-4.25395915806362\\
63.875	0.21012	-5.91523466542375	-5.91523466542375\\
63.875	0.21378	-7.67990562331147	-7.67990562331147\\
63.875	0.21744	-9.54797203172674	-9.54797203172674\\
63.875	0.2211	-11.5194338906697	-11.5194338906697\\
63.875	0.22476	-13.5942912001402	-13.5942912001402\\
63.875	0.22842	-15.7725439601384	-15.7725439601384\\
63.875	0.23208	-18.054192170664	-18.054192170664\\
63.875	0.23574	-20.4392358317174	-20.4392358317174\\
63.875	0.2394	-22.9276749432983	-22.9276749432983\\
63.875	0.24306	-25.5195095054069	-25.5195095054069\\
63.875	0.24672	-28.214739518043	-28.214739518043\\
63.875	0.25038	-31.0133649812067	-31.0133649812067\\
63.875	0.25404	-33.915385894898	-33.915385894898\\
63.875	0.2577	-36.9208022591169	-36.9208022591169\\
63.875	0.26136	-40.0296140738635	-40.0296140738635\\
63.875	0.26502	-43.2418213391376	-43.2418213391376\\
63.875	0.26868	-46.5574240549393	-46.5574240549393\\
63.875	0.27234	-49.9764222212686	-49.9764222212686\\
63.875	0.276	-53.4988158381254	-53.4988158381254\\
64.25	0.093	-4.47806055037669	-4.47806055037669\\
64.25	0.09666	-2.9018125974928	-2.9018125974928\\
64.25	0.10032	-1.42896009513664	-1.42896009513664\\
64.25	0.10398	-0.0595030433079415	-0.0595030433079415\\
64.25	0.10764	1.20655855799309	1.20655855799309\\
64.25	0.1113	2.36922470876658	2.36922470876658\\
64.25	0.11496	3.42849540901241	3.42849540901241\\
64.25	0.11862	4.38437065873065	4.38437065873065\\
64.25	0.12228	5.23685045792129	5.23685045792129\\
64.25	0.12594	5.98593480658432	5.98593480658432\\
64.25	0.1296	6.63162370471977	6.63162370471977\\
64.25	0.13326	7.17391715232762	7.17391715232762\\
64.25	0.13692	7.61281514940787	7.61281514940787\\
64.25	0.14058	7.94831769596048	7.94831769596048\\
64.25	0.14424	8.18042479198552	8.18042479198552\\
64.25	0.1479	8.30913643748296	8.30913643748296\\
64.25	0.15156	8.33445263245281	8.33445263245281\\
64.25	0.15522	8.25637337689506	8.25637337689506\\
64.25	0.15888	8.07489867080973	8.07489867080973\\
64.25	0.16254	7.79002851419678	7.79002851419678\\
64.25	0.1662	7.40176290705624	7.40176290705624\\
64.25	0.16986	6.91010184938808	6.91010184938808\\
64.25	0.17352	6.31504534119235	6.31504534119235\\
64.25	0.17718	5.61659338246899	5.61659338246899\\
64.25	0.18084	4.81474597321807	4.81474597321807\\
64.25	0.1845	3.90950311343951	3.90950311343951\\
64.25	0.18816	2.9008648031334	2.9008648031334\\
64.25	0.19182	1.78883104229965	1.78883104229965\\
64.25	0.19548	0.573401830938309	0.573401830938309\\
64.25	0.19914	-0.745422830950702	-0.745422830950702\\
64.25	0.2028	-2.16764294336721	-2.16764294336721\\
64.25	0.20646	-3.69325850631137	-3.69325850631137\\
64.25	0.21012	-5.32226951978308	-5.32226951978308\\
64.25	0.21378	-7.05467598378243	-7.05467598378243\\
64.25	0.21744	-8.89047789830934	-8.89047789830934\\
64.25	0.2211	-10.8296752633638	-10.8296752633638\\
64.25	0.22476	-12.872268078946	-12.872268078946\\
64.25	0.22842	-15.0182563450557	-15.0182563450557\\
64.25	0.23208	-17.267640061693	-17.267640061693\\
64.25	0.23574	-19.620419228858	-19.620419228858\\
64.25	0.2394	-22.0765938465505	-22.0765938465505\\
64.25	0.24306	-24.6361639147706	-24.6361639147706\\
64.25	0.24672	-27.2991294335184	-27.2991294335184\\
64.25	0.25038	-30.0654904027937	-30.0654904027937\\
64.25	0.25404	-32.9352468225966	-32.9352468225966\\
64.25	0.2577	-35.9083986929272	-35.9083986929272\\
64.25	0.26136	-38.9849460137853	-38.9849460137853\\
64.25	0.26502	-42.164888785171	-42.164888785171\\
64.25	0.26868	-45.4482270070843	-45.4482270070843\\
64.25	0.27234	-48.8349606795252	-48.8349606795252\\
64.25	0.276	-52.3250898024937	-52.3250898024937\\
64.625	0.093	-4.93829916522166	-4.93829916522166\\
64.625	0.09666	-3.32978671844946	-3.32978671844946\\
64.625	0.10032	-1.82466972220488	-1.82466972220488\\
64.625	0.10398	-0.422948176487822	-0.422948176487822\\
64.625	0.10764	0.875377918701631	0.875377918701631\\
64.625	0.1113	2.07030856336348	2.07030856336348\\
64.625	0.11496	3.16184375749773	3.16184375749773\\
64.625	0.11862	4.14998350110434	4.14998350110434\\
64.625	0.12228	5.03472779418339	5.03472779418339\\
64.625	0.12594	5.81607663673479	5.81607663673479\\
64.625	0.1296	6.49403002875866	6.49403002875866\\
64.625	0.13326	7.06858797025492	7.06858797025492\\
64.625	0.13692	7.53975046122359	7.53975046122359\\
64.625	0.14058	7.90751750166463	7.90751750166463\\
64.625	0.14424	8.17188909157802	8.17188909157802\\
64.625	0.1479	8.33286523096383	8.33286523096383\\
64.625	0.15156	8.3904459198221	8.3904459198221\\
64.625	0.15522	8.34463115815277	8.34463115815277\\
64.625	0.15888	8.1954209459558	8.1954209459558\\
64.625	0.16254	7.94281528323127	7.94281528323127\\
64.625	0.1662	7.58681416997909	7.58681416997909\\
64.625	0.16986	7.12741760619936	7.12741760619936\\
64.625	0.17352	6.56462559189198	6.56462559189198\\
64.625	0.17718	5.89843812705705	5.89843812705705\\
64.625	0.18084	5.12885521169449	5.12885521169449\\
64.625	0.1845	4.25587684580435	4.25587684580435\\
64.625	0.18816	3.27950302938655	3.27950302938655\\
64.625	0.19182	2.19973376244121	2.19973376244121\\
64.625	0.19548	1.01656904496829	1.01656904496829\\
64.625	0.19914	-0.269991123032298	-0.269991123032298\\
64.625	0.2028	-1.65994674156039	-1.65994674156039\\
64.625	0.20646	-3.15329781061618	-3.15329781061618\\
64.625	0.21012	-4.75004433019947	-4.75004433019947\\
64.625	0.21378	-6.45018630031041	-6.45018630031041\\
64.625	0.21744	-8.25372372094895	-8.25372372094895\\
64.625	0.2211	-10.1606565921151	-10.1606565921151\\
64.625	0.22476	-12.1709849138088	-12.1709849138088\\
64.625	0.22842	-14.2847086860302	-14.2847086860302\\
64.625	0.23208	-16.5018279087791	-16.5018279087791\\
64.625	0.23574	-18.8223425820557	-18.8223425820557\\
64.625	0.2394	-21.2462527058598	-21.2462527058598\\
64.625	0.24306	-23.7735582801915	-23.7735582801915\\
64.625	0.24672	-26.4042593050509	-26.4042593050509\\
64.625	0.25038	-29.1383557804378	-29.1383557804378\\
64.625	0.25404	-31.9758477063524	-31.9758477063524\\
64.625	0.2577	-34.9167350827945	-34.9167350827945\\
64.625	0.26136	-37.9610179097642	-37.9610179097642\\
64.625	0.26502	-41.1086961872615	-41.1086961872615\\
64.625	0.26868	-44.3597699152864	-44.3597699152864\\
64.625	0.27234	-47.7142390938389	-47.7142390938389\\
64.625	0.276	-51.172103722919	-51.172103722919\\
65	0.093	-5.4192777361237	-5.4192777361237\\
65	0.09666	-3.77850079546308	-3.77850079546308\\
65	0.10032	-2.24111930533008	-2.24111930533008\\
65	0.10398	-0.807133265724659	-0.807133265724659\\
65	0.10764	0.523457323353213	0.523457323353213\\
65	0.1113	1.75065246190343	1.75065246190343\\
65	0.11496	2.87445214992604	2.87445214992604\\
65	0.11862	3.89485638742112	3.89485638742112\\
65	0.12228	4.81186517438848	4.81186517438848\\
65	0.12594	5.62547851082836	5.62547851082836\\
65	0.1296	6.33569639674059	6.33569639674059\\
65	0.13326	6.94251883212522	6.94251883212522\\
65	0.13692	7.4459458169823	7.4459458169823\\
65	0.14058	7.8459773513117	7.8459773513117\\
65	0.14424	8.14261343511352	8.14261343511352\\
65	0.1479	8.33585406838775	8.33585406838775\\
65	0.15156	8.42569925113438	8.42569925113438\\
65	0.15522	8.4121489833534	8.4121489833534\\
65	0.15888	8.29520326504486	8.29520326504486\\
65	0.16254	8.07486209620869	8.07486209620869\\
65	0.1662	7.75112547684499	7.75112547684499\\
65	0.16986	7.32399340695356	7.32399340695356\\
65	0.17352	6.79346588653466	6.79346588653466\\
65	0.17718	6.15954291558809	6.15954291558809\\
65	0.18084	5.42222449411395	5.42222449411395\\
65	0.1845	4.58151062211218	4.58151062211218\\
65	0.18816	3.63740129958279	3.63740129958279\\
65	0.19182	2.58989652652588	2.58989652652588\\
65	0.19548	1.43899630294126	1.43899630294126\\
65	0.19914	0.184700628829148	0.184700628829148\\
65	0.2028	-1.17299049581058	-1.17299049581058\\
65	0.20646	-2.63407707097795	-2.63407707097795\\
65	0.21012	-4.19855909667288	-4.19855909667288\\
65	0.21378	-5.86643657289545	-5.86643657289545\\
65	0.21744	-7.63770949964558	-7.63770949964558\\
65	0.2211	-9.51237787692335	-9.51237787692335\\
65	0.22476	-11.4904417047287	-11.4904417047287\\
65	0.22842	-13.5719009830616	-13.5719009830616\\
65	0.23208	-15.7567557119221	-15.7567557119221\\
65	0.23574	-18.0450058913104	-18.0450058913104\\
65	0.2394	-20.436651521226	-20.436651521226\\
65	0.24306	-22.9316926016694	-22.9316926016694\\
65	0.24672	-25.5301291326404	-25.5301291326404\\
65	0.25038	-28.2319611141388	-28.2319611141388\\
65	0.25404	-31.037188546165	-31.037188546165\\
65	0.2577	-33.9458114287187	-33.9458114287187\\
65	0.26136	-36.9578297618	-36.9578297618\\
65	0.26502	-40.073243545409	-40.073243545409\\
65	0.26868	-43.2920527795456	-43.2920527795456\\
65	0.27234	-46.6142574642097	-46.6142574642097\\
65	0.276	-50.0398575994013	-50.0398575994013\\
65.375	0.093	-5.92099626308275	-5.92099626308275\\
65.375	0.09666	-4.24795482853377	-4.24795482853377\\
65.375	0.10032	-2.67830884451235	-2.67830884451235\\
65.375	0.10398	-1.21205831101851	-1.21205831101851\\
65.375	0.10764	0.150796771947668	0.150796771947668\\
65.375	0.1113	1.41025640438636	1.41025640438636\\
65.375	0.11496	2.56632058629734	2.56632058629734\\
65.375	0.11862	3.61898931768072	3.61898931768072\\
65.375	0.12228	4.56826259853661	4.56826259853661\\
65.375	0.12594	5.4141404288648	5.4141404288648\\
65.375	0.1296	6.15662280866545	6.15662280866545\\
65.375	0.13326	6.79570973793849	6.79570973793849\\
65.375	0.13692	7.33140121668389	7.33140121668389\\
65.375	0.14058	7.76369724490176	7.76369724490176\\
65.375	0.14424	8.09259782259194	8.09259782259194\\
65.375	0.1479	8.31810294975459	8.31810294975459\\
65.375	0.15156	8.44021262638964	8.44021262638964\\
65.375	0.15522	8.45892685249703	8.45892685249703\\
65.375	0.15888	8.37424562807685	8.37424562807685\\
65.375	0.16254	8.1861689531291	8.1861689531291\\
65.375	0.1662	7.8946968276537	7.8946968276537\\
65.375	0.16986	7.49982925165075	7.49982925165075\\
65.375	0.17352	7.00156622512021	7.00156622512021\\
65.375	0.17718	6.39990774806206	6.39990774806206\\
65.375	0.18084	5.69485382047634	5.69485382047634\\
65.375	0.1845	4.88640444236293	4.88640444236293\\
65.375	0.18816	3.97455961372202	3.97455961372202\\
65.375	0.19182	2.95931933455341	2.95931933455341\\
65.375	0.19548	1.84068360485722	1.84068360485722\\
65.375	0.19914	0.618652424633467	0.618652424633467\\
65.375	0.2028	-0.706774206117899	-0.706774206117899\\
65.375	0.20646	-2.13559628739685	-2.13559628739685\\
65.375	0.21012	-3.66781381920336	-3.66781381920336\\
65.375	0.21378	-5.30342680153751	-5.30342680153751\\
65.375	0.21744	-7.04243523439928	-7.04243523439928\\
65.375	0.2211	-8.88483911778863	-8.88483911778863\\
65.375	0.22476	-10.8306384517056	-10.8306384517056\\
65.375	0.22842	-12.8798332361501	-12.8798332361501\\
65.375	0.23208	-15.0324234711223	-15.0324234711223\\
65.375	0.23574	-17.288409156622	-17.288409156622\\
65.375	0.2394	-19.6477902926494	-19.6477902926494\\
65.375	0.24306	-22.1105668792044	-22.1105668792044\\
65.375	0.24672	-24.6767389162869	-24.6767389162869\\
65.375	0.25038	-27.346306403897	-27.346306403897\\
65.375	0.25404	-30.1192693420347	-30.1192693420347\\
65.375	0.2577	-32.9956277307001	-32.9956277307001\\
65.375	0.26136	-35.9753815698931	-35.9753815698931\\
65.375	0.26502	-39.0585308596135	-39.0585308596135\\
65.375	0.26868	-42.2450755998617	-42.2450755998617\\
65.375	0.27234	-45.5350157906375	-45.5350157906375\\
65.375	0.276	-48.9283514319407	-48.9283514319407\\
65.75	0.093	-6.44345474609887	-6.44345474609887\\
65.75	0.09666	-4.73814881766148	-4.73814881766148\\
65.75	0.10032	-3.13623833975164	-3.13623833975164\\
65.75	0.10398	-1.63772331236943	-1.63772331236943\\
65.75	0.10764	-0.242603735514834	-0.242603735514834\\
65.75	0.1113	1.04912039081216	1.04912039081216\\
65.75	0.11496	2.23744906661156	2.23744906661156\\
65.75	0.11862	3.32238229188342	3.32238229188342\\
65.75	0.12228	4.30392006662768	4.30392006662768\\
65.75	0.12594	5.18206239084428	5.18206239084428\\
65.75	0.1296	5.95680926453329	5.95680926453329\\
65.75	0.13326	6.6281606876947	6.6281606876947\\
65.75	0.13692	7.19611666032851	7.19611666032851\\
65.75	0.14058	7.66067718243475	7.66067718243475\\
65.75	0.14424	8.02184225401335	8.02184225401335\\
65.75	0.1479	8.27961187506442	8.27961187506442\\
65.75	0.15156	8.43398604558783	8.43398604558783\\
65.75	0.15522	8.48496476558364	8.48496476558364\\
65.75	0.15888	8.43254803505182	8.43254803505182\\
65.75	0.16254	8.27673585399243	8.27673585399243\\
65.75	0.1662	8.01752822240552	8.01752822240552\\
65.75	0.16986	7.65492514029093	7.65492514029093\\
65.75	0.17352	7.18892660764875	7.18892660764875\\
65.75	0.17718	6.61953262447902	6.61953262447902\\
65.75	0.18084	5.94674319078166	5.94674319078166\\
65.75	0.1845	5.17055830655667	5.17055830655667\\
65.75	0.18816	4.29097797180407	4.29097797180407\\
65.75	0.19182	3.30800218652399	3.30800218652399\\
65.75	0.19548	2.22163095071622	2.22163095071622\\
65.75	0.19914	1.03186426438077	1.03186426438077\\
65.75	0.2028	-0.261297872482174	-0.261297872482174\\
65.75	0.20646	-1.65785545987276	-1.65785545987276\\
65.75	0.21012	-3.15780849779085	-3.15780849779085\\
65.75	0.21378	-4.76115698623664	-4.76115698623664\\
65.75	0.21744	-6.46790092520999	-6.46790092520999\\
65.75	0.2211	-8.27804031471098	-8.27804031471098\\
65.75	0.22476	-10.1915751547395	-10.1915751547395\\
65.75	0.22842	-12.2085054452957	-12.2085054452957\\
65.75	0.23208	-14.3288311863794	-14.3288311863794\\
65.75	0.23574	-16.5525523779908	-16.5525523779908\\
65.75	0.2394	-18.8796690201297	-18.8796690201297\\
65.75	0.24306	-21.3101811127962	-21.3101811127962\\
65.75	0.24672	-23.8440886559905	-23.8440886559905\\
65.75	0.25038	-26.4813916497122	-26.4813916497122\\
65.75	0.25404	-29.2220900939616	-29.2220900939616\\
65.75	0.2577	-32.0661839887384	-32.0661839887384\\
65.75	0.26136	-35.013673334043	-35.013673334043\\
65.75	0.26502	-38.0645581298751	-38.0645581298751\\
65.75	0.26868	-41.2188383762349	-41.2188383762349\\
65.75	0.27234	-44.4765140731222	-44.4765140731222\\
65.75	0.276	-47.8375852205372	-47.8375852205372\\
66.125	0.093	-6.98665318517201	-6.98665318517201\\
66.125	0.09666	-5.24908276284625	-5.24908276284625\\
66.125	0.10032	-3.61490779104805	-3.61490779104805\\
66.125	0.10398	-2.08412826977748	-2.08412826977748\\
66.125	0.10764	-0.656744199034463	-0.656744199034463\\
66.125	0.1113	0.667244421180953	0.667244421180953\\
66.125	0.11496	1.88783759086877	1.88783759086877\\
66.125	0.11862	3.00503531002899	3.00503531002899\\
66.125	0.12228	4.01883757866161	4.01883757866161\\
66.125	0.12594	4.92924439676663	4.92924439676663\\
66.125	0.1296	5.73625576434407	5.73625576434407\\
66.125	0.13326	6.4398716813939	6.4398716813939\\
66.125	0.13692	7.04009214791613	7.04009214791613\\
66.125	0.14058	7.53691716391073	7.53691716391073\\
66.125	0.14424	7.93034672937769	7.93034672937769\\
66.125	0.1479	8.22038084431712	8.22038084431712\\
66.125	0.15156	8.40701950872895	8.40701950872895\\
66.125	0.15522	8.49026272261312	8.49026272261312\\
66.125	0.15888	8.47011048596973	8.47011048596973\\
66.125	0.16254	8.34656279879876	8.34656279879876\\
66.125	0.1662	8.1196196611002	8.1196196611002\\
66.125	0.16986	7.78928107287403	7.78928107287403\\
66.125	0.17352	7.35554703412028	7.35554703412028\\
66.125	0.17718	6.81841754483891	6.81841754483891\\
66.125	0.18084	6.17789260502997	6.17789260502997\\
66.125	0.1845	5.43397221469334	5.43397221469334\\
66.125	0.18816	4.58665637382921	4.58665637382921\\
66.125	0.19182	3.63594508243744	3.63594508243744\\
66.125	0.19548	2.58183834051809	2.58183834051809\\
66.125	0.19914	1.42433614807101	1.42433614807101\\
66.125	0.2028	0.16343850509648	0.16343850509648\\
66.125	0.20646	-1.20085458840569	-1.20085458840569\\
66.125	0.21012	-2.66854313243542	-2.66854313243542\\
66.125	0.21378	-4.23962712699284	-4.23962712699284\\
66.125	0.21744	-5.91410657207777	-5.91410657207777\\
66.125	0.2211	-7.69198146769034	-7.69198146769034\\
66.125	0.22476	-9.5732518138305	-9.5732518138305\\
66.125	0.22842	-11.5579176104982	-11.5579176104982\\
66.125	0.23208	-13.6459788576937	-13.6459788576937\\
66.125	0.23574	-15.8374355554166	-15.8374355554166\\
66.125	0.2394	-18.1322877036671	-18.1322877036671\\
66.125	0.24306	-20.5305353024453	-20.5305353024453\\
66.125	0.24672	-23.0321783517511	-23.0321783517511\\
66.125	0.25038	-25.6372168515844	-25.6372168515844\\
66.125	0.25404	-28.3456508019454	-28.3456508019454\\
66.125	0.2577	-31.1574802028339	-31.1574802028339\\
66.125	0.26136	-34.0727050542502	-34.0727050542502\\
66.125	0.26502	-37.0913253561939	-37.0913253561939\\
66.125	0.26868	-40.2133411086652	-40.2133411086652\\
66.125	0.27234	-43.4387523116641	-43.4387523116641\\
66.125	0.276	-46.7675589651906	-46.7675589651906\\
66.5	0.093	-7.55059158030222	-7.55059158030222\\
66.5	0.09666	-5.78075666408804	-5.78075666408804\\
66.5	0.10032	-4.11431719840142	-4.11431719840142\\
66.5	0.10398	-2.55127318324249	-2.55127318324249\\
66.5	0.10764	-1.09162461861105	-1.09162461861105\\
66.5	0.1113	0.264628495492786	0.264628495492786\\
66.5	0.11496	1.51748615906897	1.51748615906897\\
66.5	0.11862	2.66694837211755	2.66694837211755\\
66.5	0.12228	3.71301513463859	3.71301513463859\\
66.5	0.12594	4.65568644663198	4.65568644663198\\
66.5	0.1296	5.49496230809777	5.49496230809777\\
66.5	0.13326	6.23084271903602	6.23084271903602\\
66.5	0.13692	6.86332767944667	6.86332767944667\\
66.5	0.14058	7.39241718932963	7.39241718932963\\
66.5	0.14424	7.81811124868501	7.81811124868501\\
66.5	0.1479	8.14040985751286	8.14040985751286\\
66.5	0.15156	8.35931301581306	8.35931301581306\\
66.5	0.15522	8.47482072358565	8.47482072358565\\
66.5	0.15888	8.48693298083067	8.48693298083067\\
66.5	0.16254	8.39564978754812	8.39564978754812\\
66.5	0.1662	8.20097114373793	8.20097114373793\\
66.5	0.16986	7.90289704940012	7.90289704940012\\
66.5	0.17352	7.50142750453473	7.50142750453473\\
66.5	0.17718	6.99656250914178	6.99656250914178\\
66.5	0.18084	6.3883020632212	6.3883020632212\\
66.5	0.1845	5.676646166773	5.676646166773\\
66.5	0.18816	4.86159481979729	4.86159481979729\\
66.5	0.19182	3.94314802229383	3.94314802229383\\
66.5	0.19548	2.92130577426289	2.92130577426289\\
66.5	0.19914	1.79606807570428	1.79606807570428\\
66.5	0.2028	0.56743492661812	0.56743492661812\\
66.5	0.20646	-0.764593672995687	-0.764593672995687\\
66.5	0.21012	-2.20001772313705	-2.20001772313705\\
66.5	0.21378	-3.738837223806	-3.738837223806\\
66.5	0.21744	-5.38105217500257	-5.38105217500257\\
66.5	0.2211	-7.12666257672677	-7.12666257672677\\
66.5	0.22476	-8.97566842897857	-8.97566842897857\\
66.5	0.22842	-10.9280697317579	-10.9280697317579\\
66.5	0.23208	-12.9838664850648	-12.9838664850648\\
66.5	0.23574	-15.1430586888994	-15.1430586888994\\
66.5	0.2394	-17.4056463432616	-17.4056463432616\\
66.5	0.24306	-19.7716294481513	-19.7716294481513\\
66.5	0.24672	-22.2410080035688	-22.2410080035688\\
66.5	0.25038	-24.8137820095137	-24.8137820095137\\
66.5	0.25404	-27.4899514659863	-27.4899514659863\\
66.5	0.2577	-30.2695163729864	-30.2695163729864\\
66.5	0.26136	-33.1524767305141	-33.1524767305141\\
66.5	0.26502	-36.1388325385695	-36.1388325385695\\
66.5	0.26868	-39.2285837971525	-39.2285837971525\\
66.5	0.27234	-42.4217305062631	-42.4217305062631\\
66.5	0.276	-45.7182726659012	-45.7182726659012\\
66.875	0.093	-8.1352699314895	-8.1352699314895\\
66.875	0.09666	-6.3331705213869	-6.3331705213869\\
66.875	0.10032	-4.63446656181191	-4.63446656181191\\
66.875	0.10398	-3.03915805276451	-3.03915805276451\\
66.875	0.10764	-1.54724499424476	-1.54724499424476\\
66.875	0.1113	-0.158727386252508	-0.158727386252508\\
66.875	0.11496	1.12639477121209	1.12639477121209\\
66.875	0.11862	2.3081214781491	2.3081214781491\\
66.875	0.12228	3.38645273455844	3.38645273455844\\
66.875	0.12594	4.3613885404403	4.3613885404403\\
66.875	0.1296	5.23292889579446	5.23292889579446\\
66.875	0.13326	6.00107380062113	6.00107380062113\\
66.875	0.13692	6.66582325492008	6.66582325492008\\
66.875	0.14058	7.22717725869153	7.22717725869153\\
66.875	0.14424	7.68513581193527	7.68513581193527\\
66.875	0.1479	8.03969891465148	8.03969891465148\\
66.875	0.15156	8.29086656684009	8.29086656684009\\
66.875	0.15522	8.43863876850105	8.43863876850105\\
66.875	0.15888	8.48301551963449	8.48301551963449\\
66.875	0.16254	8.4239968202403	8.4239968202403\\
66.875	0.1662	8.26158267031853	8.26158267031853\\
66.875	0.16986	7.99577306986914	7.99577306986914\\
66.875	0.17352	7.62656801889217	7.62656801889217\\
66.875	0.17718	7.15396751738758	7.15396751738758\\
66.875	0.18084	6.57797156535543	6.57797156535543\\
66.875	0.1845	5.89858016279558	5.89858016279558\\
66.875	0.18816	5.11579330970824	5.11579330970824\\
66.875	0.19182	4.22961100609319	4.22961100609319\\
66.875	0.19548	3.24003325195068	3.24003325195068\\
66.875	0.19914	2.14706004728043	2.14706004728043\\
66.875	0.2028	0.950691392082632	0.950691392082632\\
66.875	0.20646	-0.349072713642755	-0.349072713642755\\
66.875	0.21012	-1.7522322698957	-1.7522322698957\\
66.875	0.21378	-3.25878727667629	-3.25878727667629\\
66.875	0.21744	-4.86873773398443	-4.86873773398443\\
66.875	0.2211	-6.58208364182022	-6.58208364182022\\
66.875	0.22476	-8.39882500018354	-8.39882500018354\\
66.875	0.22842	-10.3189618090746	-10.3189618090746\\
66.875	0.23208	-12.3424940684932	-12.3424940684932\\
66.875	0.23574	-14.4694217784393	-14.4694217784393\\
66.875	0.2394	-16.6997449389131	-16.6997449389131\\
66.875	0.24306	-19.0334635499144	-19.0334635499144\\
66.875	0.24672	-21.4705776114434	-21.4705776114434\\
66.875	0.25038	-24.0110871235	-24.0110871235\\
66.875	0.25404	-26.6549920860841	-26.6549920860841\\
66.875	0.2577	-29.4022924991959	-29.4022924991959\\
66.875	0.26136	-32.2529883628354	-32.2529883628354\\
66.875	0.26502	-35.2070796770024	-35.2070796770024\\
66.875	0.26868	-38.2645664416968	-38.2645664416968\\
66.875	0.27234	-41.4254486569189	-41.4254486569189\\
66.875	0.276	-44.6897263226687	-44.6897263226687\\
67.25	0.093	-8.74068823873373	-8.74068823873373\\
67.25	0.09666	-6.90632433474271	-6.90632433474271\\
67.25	0.10032	-5.17535588127931	-5.17535588127931\\
67.25	0.10398	-3.54778287834354	-3.54778287834354\\
67.25	0.10764	-2.02360532593538	-2.02360532593538\\
67.25	0.1113	-0.60282322405476	-0.60282322405476\\
67.25	0.11496	0.714563427298259	0.714563427298259\\
67.25	0.11862	1.92855462812363	1.92855462812363\\
67.25	0.12228	3.03915037842139	3.03915037842139\\
67.25	0.12594	4.04635067819162	4.04635067819162\\
67.25	0.1296	4.95015552743425	4.95015552743425\\
67.25	0.13326	5.75056492614922	5.75056492614922\\
67.25	0.13692	6.4475788743366	6.4475788743366\\
67.25	0.14058	7.0411973719964	7.0411973719964\\
67.25	0.14424	7.53142041912862	7.53142041912862\\
67.25	0.1479	7.9182480157332	7.9182480157332\\
67.25	0.15156	8.20168016181017	8.20168016181017\\
67.25	0.15522	8.38171685735955	8.38171685735955\\
67.25	0.15888	8.45835810238141	8.45835810238141\\
67.25	0.16254	8.43160389687559	8.43160389687559\\
67.25	0.1662	8.30145424084218	8.30145424084218\\
67.25	0.16986	8.06790913428121	8.06790913428121\\
67.25	0.17352	7.7309685771926	7.7309685771926\\
67.25	0.17718	7.29063256957643	7.29063256957643\\
67.25	0.18084	6.74690111143263	6.74690111143263\\
67.25	0.1845	6.09977420276127	6.09977420276127\\
67.25	0.18816	5.34925184356229	5.34925184356229\\
67.25	0.19182	4.49533403383566	4.49533403383566\\
67.25	0.19548	3.53802077358145	3.53802077358145\\
67.25	0.19914	2.47731206279963	2.47731206279963\\
67.25	0.2028	1.31320790149024	1.31320790149024\\
67.25	0.20646	0.0457082896532768	0.0457082896532768\\
67.25	0.21012	-1.32518677271131	-1.32518677271131\\
67.25	0.21378	-2.79947728560347	-2.79947728560347\\
67.25	0.21744	-4.37716324902325	-4.37716324902325\\
67.25	0.2211	-6.05824466297062	-6.05824466297062\\
67.25	0.22476	-7.84272152744563	-7.84272152744563\\
67.25	0.22842	-9.73059384244823	-9.73059384244823\\
67.25	0.23208	-11.7218616079783	-11.7218616079783\\
67.25	0.23574	-13.8165248240362	-13.8165248240362\\
67.25	0.2394	-16.0145834906216	-16.0145834906216\\
67.25	0.24306	-18.3160376077346	-18.3160376077346\\
67.25	0.24672	-20.7208871753751	-20.7208871753751\\
67.25	0.25038	-23.2291321935433	-23.2291321935433\\
67.25	0.25404	-25.8407726622391	-25.8407726622391\\
67.25	0.2577	-28.5558085814625	-28.5558085814625\\
67.25	0.26136	-31.3742399512134	-31.3742399512134\\
67.25	0.26502	-34.2960667714921	-34.2960667714921\\
67.25	0.26868	-37.3212890422981	-37.3212890422981\\
67.25	0.27234	-40.4499067636319	-40.4499067636319\\
67.25	0.276	-43.6819199354933	-43.6819199354933\\
67.625	0.093	-9.36684650203504	-9.36684650203504\\
67.625	0.09666	-7.50021810415566	-7.50021810415566\\
67.625	0.10032	-5.73698515680389	-5.73698515680389\\
67.625	0.10398	-4.0771476599797	-4.0771476599797\\
67.625	0.10764	-2.52070561368312	-2.52070561368312\\
67.625	0.1113	-1.06765901791408	-1.06765901791408\\
67.625	0.11496	0.2819921273273	0.2819921273273\\
67.625	0.11862	1.52824782204109	1.52824782204109\\
67.625	0.12228	2.67110806622727	2.67110806622727\\
67.625	0.12594	3.71057285988586	3.71057285988586\\
67.625	0.1296	4.64664220301685	4.64664220301685\\
67.625	0.13326	5.47931609562019	5.47931609562019\\
67.625	0.13692	6.20859453769599	6.20859453769599\\
67.625	0.14058	6.83447752924421	6.83447752924421\\
67.625	0.14424	7.35696507026479	7.35696507026479\\
67.625	0.1479	7.77605716075779	7.77605716075779\\
67.625	0.15156	8.09175380072318	8.09175380072318\\
67.625	0.15522	8.30405499016092	8.30405499016092\\
67.625	0.15888	8.41296072907114	8.41296072907114\\
67.625	0.16254	8.41847101745374	8.41847101745374\\
67.625	0.1662	8.32058585530875	8.32058585530875\\
67.625	0.16986	8.11930524263614	8.11930524263614\\
67.625	0.17352	7.81462917943595	7.81462917943595\\
67.625	0.17718	7.40655766570815	7.40655766570815\\
67.625	0.18084	6.89509070145277	6.89509070145277\\
67.625	0.1845	6.28022828666977	6.28022828666977\\
67.625	0.18816	5.56197042135915	5.56197042135915\\
67.625	0.19182	4.74031710552094	4.74031710552094\\
67.625	0.19548	3.81526833915515	3.81526833915515\\
67.625	0.19914	2.78682412226169	2.78682412226169\\
67.625	0.2028	1.65498445484073	1.65498445484073\\
67.625	0.20646	0.419749336892124	0.419749336892124\\
67.625	0.21012	-0.918881231584038	-0.918881231584038\\
67.625	0.21378	-2.36090725058784	-2.36090725058784\\
67.625	0.21744	-3.90632872011921	-3.90632872011921\\
67.625	0.2211	-5.55514564017821	-5.55514564017821\\
67.625	0.22476	-7.3073580107648	-7.3073580107648\\
67.625	0.22842	-9.16296583187898	-9.16296583187898\\
67.625	0.23208	-11.1219691035207	-11.1219691035207\\
67.625	0.23574	-13.1843678256902	-13.1843678256902\\
67.625	0.2394	-15.3501619983872	-15.3501619983872\\
67.625	0.24306	-17.6193516216117	-17.6193516216117\\
67.625	0.24672	-19.9919366953639	-19.9919366953639\\
67.625	0.25038	-22.4679172196437	-22.4679172196437\\
67.625	0.25404	-25.0472931944511	-25.0472931944511\\
67.625	0.2577	-27.7300646197861	-27.7300646197861\\
67.625	0.26136	-30.5162314956487	-30.5162314956487\\
67.625	0.26502	-33.4057938220388	-33.4057938220388\\
67.625	0.26868	-36.3987515989566	-36.3987515989566\\
67.625	0.27234	-39.495104826402	-39.495104826402\\
67.625	0.276	-42.6948535043749	-42.6948535043749\\
68	0.093	-10.0137447213934	-10.0137447213934\\
68	0.09666	-8.11485182962567	-8.11485182962567\\
68	0.10032	-6.31935438838554	-6.31935438838554\\
68	0.10398	-4.62725239767293	-4.62725239767293\\
68	0.10764	-3.03854585748793	-3.03854585748793\\
68	0.1113	-1.55323476783053	-1.55323476783053\\
68	0.11496	-0.171319128700787	-0.171319128700787\\
68	0.11862	1.10720105990136	1.10720105990136\\
68	0.12228	2.28232579797596	2.28232579797596\\
68	0.12594	3.35405508552297	3.35405508552297\\
68	0.1296	4.32238892254233	4.32238892254233\\
68	0.13326	5.18732730903409	5.18732730903409\\
68	0.13692	5.9488702449983	5.9488702449983\\
68	0.14058	6.60701773043489	6.60701773043489\\
68	0.14424	7.16176976534389	7.16176976534389\\
68	0.1479	7.61312634972525	7.61312634972525\\
68	0.15156	7.96108748357901	7.96108748357901\\
68	0.15522	8.20565316690522	8.20565316690522\\
68	0.15888	8.34682339970381	8.34682339970381\\
68	0.16254	8.38459818197477	8.38459818197477\\
68	0.1662	8.3189775137182	8.3189775137182\\
68	0.16986	8.14996139493401	8.14996139493401\\
68	0.17352	7.87754982562218	7.87754982562218\\
68	0.17718	7.5017428057828	7.5017428057828\\
68	0.18084	7.02254033541578	7.02254033541578\\
68	0.1845	6.4399424145212	6.4399424145212\\
68	0.18816	5.75394904309894	5.75394904309894\\
68	0.19182	4.96456022114916	4.96456022114916\\
68	0.19548	4.07177594867179	4.07177594867179\\
68	0.19914	3.07559622566674	3.07559622566674\\
68	0.2028	1.97602105213414	1.97602105213414\\
68	0.20646	0.773050428073901	0.773050428073901\\
68	0.21012	-0.533315646513842	-0.533315646513842\\
68	0.21378	-1.94307717162934	-1.94307717162934\\
68	0.21744	-3.45623414727217	-3.45623414727217\\
68	0.2211	-5.07278657344287	-5.07278657344287\\
68	0.22476	-6.79273445014104	-6.79273445014104\\
68	0.22842	-8.6160777773668	-8.6160777773668\\
68	0.23208	-10.5428165551203	-10.5428165551203\\
68	0.23574	-12.5729507834013	-12.5729507834013\\
68	0.2394	-14.7064804622098	-14.7064804622098\\
68	0.24306	-16.943405591546	-16.943405591546\\
68	0.24672	-19.2837261714097	-19.2837261714097\\
68	0.25038	-21.7274422018013	-21.7274422018013\\
68	0.25404	-24.2745536827201	-24.2745536827201\\
68	0.2577	-26.9250606141668	-26.9250606141668\\
68	0.26136	-29.678962996141	-29.678962996141\\
68	0.26502	-32.5362608286427	-32.5362608286427\\
68	0.26868	-35.4969541116722	-35.4969541116722\\
68	0.27234	-38.5610428452291	-38.5610428452291\\
68	0.276	-41.7285270293136	-41.7285270293136\\
68.375	0.093	-10.6813828968087	-10.6813828968087\\
68.375	0.09666	-8.75022551115264	-8.75022551115264\\
68.375	0.10032	-6.92246357602404	-6.92246357602404\\
68.375	0.10398	-5.19809709142312	-5.19809709142312\\
68.375	0.10764	-3.5771260573497	-3.5771260573497\\
68.375	0.1113	-2.05955047380394	-2.05955047380394\\
68.375	0.11496	-0.645370340785774	-0.645370340785774\\
68.375	0.11862	0.665414341704853	0.665414341704853\\
68.375	0.12228	1.87280357366776	1.87280357366776\\
68.375	0.12594	2.97679735510319	2.97679735510319\\
68.375	0.1296	3.97739568601091	3.97739568601091\\
68.375	0.13326	4.87459856639114	4.87459856639114\\
68.375	0.13692	5.66840599624366	5.66840599624366\\
68.375	0.14058	6.35881797556867	6.35881797556867\\
68.375	0.14424	6.94583450436603	6.94583450436603\\
68.375	0.1479	7.42945558263581	7.42945558263581\\
68.375	0.15156	7.80968121037805	7.80968121037805\\
68.375	0.15522	8.08651138759257	8.08651138759257\\
68.375	0.15888	8.25994611427957	8.25994611427957\\
68.375	0.16254	8.32998539043895	8.32998539043895\\
68.375	0.1662	8.29662921607074	8.29662921607074\\
68.375	0.16986	8.15987759117492	8.15987759117492\\
68.375	0.17352	7.91973051575151	7.91973051575151\\
68.375	0.17718	7.57618798980049	7.57618798980049\\
68.375	0.18084	7.12925001332189	7.12925001332189\\
68.375	0.1845	6.57891658631567	6.57891658631567\\
68.375	0.18816	5.92518770878183	5.92518770878183\\
68.375	0.19182	5.16806338072047	5.16806338072047\\
68.375	0.19548	4.3075436021314	4.3075436021314\\
68.375	0.19914	3.34362837301484	3.34362837301484\\
68.375	0.2028	2.2763176933706	2.2763176933706\\
68.375	0.20646	1.10561156319883	1.10561156319883\\
68.375	0.21012	-0.168490017500545	-0.168490017500545\\
68.375	0.21378	-1.54598704872762	-1.54598704872762\\
68.375	0.21744	-3.02687953048215	-3.02687953048215\\
68.375	0.2211	-4.61116746276431	-4.61116746276431\\
68.375	0.22476	-6.29885084557418	-6.29885084557418\\
68.375	0.22842	-8.08992967891163	-8.08992967891163\\
68.375	0.23208	-9.98440396277655	-9.98440396277655\\
68.375	0.23574	-11.9822736971692	-11.9822736971692\\
68.375	0.2394	-14.0835388820894	-14.0835388820894\\
68.375	0.24306	-16.2881995175372	-16.2881995175372\\
68.375	0.24672	-18.5962556035126	-18.5962556035126\\
68.375	0.25038	-21.0077071400156	-21.0077071400156\\
68.375	0.25404	-23.5225541270461	-23.5225541270461\\
68.375	0.2577	-26.1407965646044	-26.1407965646044\\
68.375	0.26136	-28.8624344526903	-28.8624344526903\\
68.375	0.26502	-31.6874677913036	-31.6874677913036\\
68.375	0.26868	-34.6158965804445	-34.6158965804445\\
68.375	0.27234	-37.647720820113	-37.647720820113\\
68.375	0.276	-40.7829405103092	-40.7829405103092\\
68.75	0.093	-11.3697610282812	-11.3697610282812\\
68.75	0.09666	-9.40633914873668	-9.40633914873668\\
68.75	0.10032	-7.54631271971972	-7.54631271971972\\
68.75	0.10398	-5.78968174123032	-5.78968174123032\\
68.75	0.10764	-4.1364462132686	-4.1364462132686\\
68.75	0.1113	-2.58660613583436	-2.58660613583436\\
68.75	0.11496	-1.14016150892783	-1.14016150892783\\
68.75	0.11862	0.202887667451101	0.202887667451101\\
68.75	0.12228	1.44254139330248	1.44254139330248\\
68.75	0.12594	2.57879966862627	2.57879966862627\\
68.75	0.1296	3.61166249342241	3.61166249342241\\
68.75	0.13326	4.54112986769101	4.54112986769101\\
68.75	0.13692	5.36720179143201	5.36720179143201\\
68.75	0.14058	6.08987826464532	6.08987826464532\\
68.75	0.14424	6.70915928733116	6.70915928733116\\
68.75	0.1479	7.2250448594893	7.2250448594893\\
68.75	0.15156	7.63753498111984	7.63753498111984\\
68.75	0.15522	7.94662965222278	7.94662965222278\\
68.75	0.15888	8.15232887279821	8.15232887279821\\
68.75	0.16254	8.254632642846	8.254632642846\\
68.75	0.1662	8.25354096236616	8.25354096236616\\
68.75	0.16986	8.14905383135876	8.14905383135876\\
68.75	0.17352	7.94117124982377	7.94117124982377\\
68.75	0.17718	7.62989321776111	7.62989321776111\\
68.75	0.18084	7.21521973517093	7.21521973517093\\
68.75	0.1845	6.69715080205307	6.69715080205307\\
68.75	0.18816	6.07568641840771	6.07568641840771\\
68.75	0.19182	5.35082658423465	5.35082658423465\\
68.75	0.19548	4.52257129953401	4.52257129953401\\
68.75	0.19914	3.5909205643058	3.5909205643058\\
68.75	0.2028	2.55587437854999	2.55587437854999\\
68.75	0.20646	1.41743274226653	1.41743274226653\\
68.75	0.21012	0.17559565545551	0.17559565545551\\
68.75	0.21378	-1.16963688188304	-1.16963688188304\\
68.75	0.21744	-2.61826486974914	-2.61826486974914\\
68.75	0.2211	-4.170288308143	-4.170288308143\\
68.75	0.22476	-5.82570719706445	-5.82570719706445\\
68.75	0.22842	-7.58452153651348	-7.58452153651348\\
68.75	0.23208	-9.4467313264901	-9.4467313264901\\
68.75	0.23574	-11.4123365669943	-11.4123365669943\\
68.75	0.2394	-13.4813372580261	-13.4813372580261\\
68.75	0.24306	-15.6537333995854	-15.6537333995854\\
68.75	0.24672	-17.9295249916725	-17.9295249916725\\
68.75	0.25038	-20.3087120342871	-20.3087120342871\\
68.75	0.25404	-22.7912945274292	-22.7912945274292\\
68.75	0.2577	-25.3772724710991	-25.3772724710991\\
68.75	0.26136	-28.0666458652964	-28.0666458652964\\
68.75	0.26502	-30.8594147100214	-30.8594147100214\\
68.75	0.26868	-33.7555790052741	-33.7555790052741\\
68.75	0.27234	-36.7551387510543	-36.7551387510543\\
68.75	0.276	-39.8580939473621	-39.8580939473621\\
69.125	0.093	-12.0788791158106	-12.0788791158106\\
69.125	0.09666	-10.0831927423777	-10.0831927423777\\
69.125	0.10032	-8.19090181947235	-8.19090181947235\\
69.125	0.10398	-6.4020063470946	-6.4020063470946\\
69.125	0.10764	-4.71650632524445	-4.71650632524445\\
69.125	0.1113	-3.13440175392191	-3.13440175392191\\
69.125	0.11496	-1.65569263312696	-1.65569263312696\\
69.125	0.11862	-0.280378962859551	-0.280378962859551\\
69.125	0.12228	0.991539256880195	0.991539256880195\\
69.125	0.12594	2.16006202609235	2.16006202609235\\
69.125	0.1296	3.22518934477696	3.22518934477696\\
69.125	0.13326	4.18692121293392	4.18692121293392\\
69.125	0.13692	5.04525763056328	5.04525763056328\\
69.125	0.14058	5.80019859766502	5.80019859766502\\
69.125	0.14424	6.45174411423916	6.45174411423916\\
69.125	0.1479	6.99989418028572	6.99989418028572\\
69.125	0.15156	7.44464879580468	7.44464879580468\\
69.125	0.15522	7.78600796079604	7.78600796079604\\
69.125	0.15888	8.02397167525983	8.02397167525983\\
69.125	0.16254	8.15853993919605	8.15853993919605\\
69.125	0.1662	8.18971275260462	8.18971275260462\\
69.125	0.16986	8.11749011548558	8.11749011548558\\
69.125	0.17352	7.94187202783895	7.94187202783895\\
69.125	0.17718	7.66285848966471	7.66285848966471\\
69.125	0.18084	7.2804495009629	7.2804495009629\\
69.125	0.1845	6.79464506173352	6.79464506173352\\
69.125	0.18816	6.20544517197652	6.20544517197652\\
69.125	0.19182	5.51284983169188	5.51284983169188\\
69.125	0.19548	4.71685904087965	4.71685904087965\\
69.125	0.19914	3.81747279953976	3.81747279953976\\
69.125	0.2028	2.81469110767236	2.81469110767236\\
69.125	0.20646	1.70851396527732	1.70851396527732\\
69.125	0.21012	0.498941372354722	0.498941372354722\\
69.125	0.21378	-0.814026671095519	-0.814026671095519\\
69.125	0.21744	-2.2303901650732	-2.2303901650732\\
69.125	0.2211	-3.75014910957876	-3.75014910957876\\
69.125	0.22476	-5.37330350461167	-5.37330350461167\\
69.125	0.22842	-7.0998533501724	-7.0998533501724\\
69.125	0.23208	-8.92979864626048	-8.92979864626048\\
69.125	0.23574	-10.8631393928764	-10.8631393928764\\
69.125	0.2394	-12.8998755900198	-12.8998755900198\\
69.125	0.24306	-15.0400072376908	-15.0400072376908\\
69.125	0.24672	-17.2835343358894	-17.2835343358894\\
69.125	0.25038	-19.6304568846155	-19.6304568846155\\
69.125	0.25404	-22.0807748838695	-22.0807748838695\\
69.125	0.2577	-24.6344883336508	-24.6344883336508\\
69.125	0.26136	-27.2915972339599	-27.2915972339599\\
69.125	0.26502	-30.0521015847964	-30.0521015847964\\
69.125	0.26868	-32.9160013861606	-32.9160013861606\\
69.125	0.27234	-35.8832966380524	-35.8832966380524\\
69.125	0.276	-38.9539873404718	-38.9539873404718\\
69.5	0.093	-12.8087371593972	-12.8087371593972\\
69.5	0.09666	-10.7807862920758	-10.7807862920758\\
69.5	0.10032	-8.85623087528206	-8.85623087528206\\
69.5	0.10398	-7.03507090901589	-7.03507090901589\\
69.5	0.10764	-5.31730639327732	-5.31730639327732\\
69.5	0.1113	-3.70293732806641	-3.70293732806641\\
69.5	0.11496	-2.19196371338305	-2.19196371338305\\
69.5	0.11862	-0.784385549227331	-0.784385549227331\\
69.5	0.12228	0.519797164400835	0.519797164400835\\
69.5	0.12594	1.72058442750141	1.72058442750141\\
69.5	0.1296	2.81797624007439	2.81797624007439\\
69.5	0.13326	3.81197260211971	3.81197260211971\\
69.5	0.13692	4.70257351363743	4.70257351363743\\
69.5	0.14058	5.48977897462764	5.48977897462764\\
69.5	0.14424	6.17358898509021	6.17358898509021\\
69.5	0.1479	6.75400354502513	6.75400354502513\\
69.5	0.15156	7.23102265443251	7.23102265443251\\
69.5	0.15522	7.60464631331229	7.60464631331229\\
69.5	0.15888	7.87487452166444	7.87487452166444\\
69.5	0.16254	8.04170727948902	8.04170727948902\\
69.5	0.1662	8.10514458678601	8.10514458678601\\
69.5	0.16986	8.06518644355539	8.06518644355539\\
69.5	0.17352	7.92183284979707	7.92183284979707\\
69.5	0.17718	7.67508380551125	7.67508380551125\\
69.5	0.18084	7.32493931069786	7.32493931069786\\
69.5	0.1845	6.87139936535684	6.87139936535684\\
69.5	0.18816	6.3144639694882	6.3144639694882\\
69.5	0.19182	5.65413312309198	5.65413312309198\\
69.5	0.19548	4.89040682616817	4.89040682616817\\
69.5	0.19914	4.02328507871675	4.02328507871675\\
69.5	0.2028	3.05276788073772	3.05276788073772\\
69.5	0.20646	1.9788552322311	1.9788552322311\\
69.5	0.21012	0.801547133196806	0.801547133196806\\
69.5	0.21378	-0.479156416365015	-0.479156416365015\\
69.5	0.21744	-1.86325541645439	-1.86325541645439\\
69.5	0.2211	-3.35074986707153	-3.35074986707153\\
69.5	0.22476	-4.94163976821613	-4.94163976821613\\
69.5	0.22842	-6.63592511988833	-6.63592511988833\\
69.5	0.23208	-8.43360592208811	-8.43360592208811\\
69.5	0.23574	-10.3346821748156	-10.3346821748156\\
69.5	0.2394	-12.3391538780706	-12.3391538780706\\
69.5	0.24306	-14.4470210318532	-14.4470210318532\\
69.5	0.24672	-16.6582836361633	-16.6582836361633\\
69.5	0.25038	-18.9729416910012	-18.9729416910012\\
69.5	0.25404	-21.3909951963666	-21.3909951963666\\
69.5	0.2577	-23.9124441522596	-23.9124441522596\\
69.5	0.26136	-26.5372885586802	-26.5372885586802\\
69.5	0.26502	-29.2655284156284	-29.2655284156284\\
69.5	0.26868	-32.0971637231042	-32.0971637231042\\
69.5	0.27234	-35.0321944811076	-35.0321944811076\\
69.5	0.276	-38.0706206896386	-38.0706206896386\\
69.875	0.093	-13.5593351590408	-13.5593351590408\\
69.875	0.09666	-11.499119797831	-11.499119797831\\
69.875	0.10032	-9.54229988714884	-9.54229988714884\\
69.875	0.10398	-7.6888754269943	-7.6888754269943\\
69.875	0.10764	-5.93884641736737	-5.93884641736737\\
69.875	0.1113	-4.29221285826799	-4.29221285826799\\
69.875	0.11496	-2.74897474969632	-2.74897474969632\\
69.875	0.11862	-1.30913209165212	-1.30913209165212\\
69.875	0.12228	0.027315115864404	0.027315115864404\\
69.875	0.12594	1.26036687285334	1.26036687285334\\
69.875	0.1296	2.39002317931474	2.39002317931474\\
69.875	0.13326	3.41628403524842	3.41628403524842\\
69.875	0.13692	4.33914944065462	4.33914944065462\\
69.875	0.14058	5.15861939553319	5.15861939553319\\
69.875	0.14424	5.87469389988412	5.87469389988412\\
69.875	0.1479	6.48737295370746	6.48737295370746\\
69.875	0.15156	6.99665655700326	6.99665655700326\\
69.875	0.15522	7.4025447097714	7.4025447097714\\
69.875	0.15888	7.70503741201192	7.70503741201192\\
69.875	0.16254	7.90413466372492	7.90413466372492\\
69.875	0.1662	7.99983646491022	7.99983646491022\\
69.875	0.16986	7.99214281556802	7.99214281556802\\
69.875	0.17352	7.88105371569823	7.88105371569823\\
69.875	0.17718	7.66656916530071	7.66656916530071\\
69.875	0.18084	7.34868916437568	7.34868916437568\\
69.875	0.1845	6.92741371292308	6.92741371292308\\
69.875	0.18816	6.40274281094287	6.40274281094287\\
69.875	0.19182	5.77467645843501	5.77467645843501\\
69.875	0.19548	5.04321465539951	5.04321465539951\\
69.875	0.19914	4.20835740183662	4.20835740183662\\
69.875	0.2028	3.27010469774589	3.27010469774589\\
69.875	0.20646	2.22845654312769	2.22845654312769\\
69.875	0.21012	1.08341293798182	1.08341293798182\\
69.875	0.21378	-0.165026117691582	-0.165026117691582\\
69.875	0.21744	-1.51686062389254	-1.51686062389254\\
69.875	0.2211	-2.97209058062126	-2.97209058062126\\
69.875	0.22476	-4.53071598787744	-4.53071598787744\\
69.875	0.22842	-6.19273684566133	-6.19273684566133\\
69.875	0.23208	-7.95815315397269	-7.95815315397269\\
69.875	0.23574	-9.82696491281175	-9.82696491281175\\
69.875	0.2394	-11.7991721221784	-11.7991721221784\\
69.875	0.24306	-13.8747747820726	-13.8747747820726\\
69.875	0.24672	-16.0537728924944	-16.0537728924944\\
69.875	0.25038	-18.3361664534438	-18.3361664534438\\
69.875	0.25404	-20.7219554649209	-20.7219554649209\\
69.875	0.2577	-23.2111399269255	-23.2111399269255\\
69.875	0.26136	-25.8037198394578	-25.8037198394578\\
69.875	0.26502	-28.4996952025175	-28.4996952025175\\
69.875	0.26868	-31.2990660161049	-31.2990660161049\\
69.875	0.27234	-34.2018322802199	-34.2018322802199\\
69.875	0.276	-37.2079939948626	-37.2079939948626\\
70.25	0.093	-14.3306731147414	-14.3306731147414\\
70.25	0.09666	-12.2381932596432	-12.2381932596432\\
70.25	0.10032	-10.2491088550727	-10.2491088550727\\
70.25	0.10398	-8.36341990102973	-8.36341990102973\\
70.25	0.10764	-6.58112639751438	-6.58112639751438\\
70.25	0.1113	-4.90222834452663	-4.90222834452663\\
70.25	0.11496	-3.32672574206654	-3.32672574206654\\
70.25	0.11862	-1.85461859013399	-1.85461859013399\\
70.25	0.12228	-0.485906888729041	-0.485906888729041\\
70.25	0.12594	0.779409362148257	0.779409362148257\\
70.25	0.1296	1.94133016249808	1.94133016249808\\
70.25	0.13326	2.99985551232018	2.99985551232018\\
70.25	0.13692	3.95498541161469	3.95498541161469\\
70.25	0.14058	4.80671986038168	4.80671986038168\\
70.25	0.14424	5.55505885862102	5.55505885862102\\
70.25	0.1479	6.20000240633279	6.20000240633279\\
70.25	0.15156	6.74155050351689	6.74155050351689\\
70.25	0.15522	7.17970315017345	7.17970315017345\\
70.25	0.15888	7.51446034630239	7.51446034630239\\
70.25	0.16254	7.7458220919038	7.7458220919038\\
70.25	0.1662	7.87378838697752	7.87378838697752\\
70.25	0.16986	7.89835923152374	7.89835923152374\\
70.25	0.17352	7.81953462554226	7.81953462554226\\
70.25	0.17718	7.63731456903317	7.63731456903317\\
70.25	0.18084	7.35169906199656	7.35169906199656\\
70.25	0.1845	6.96268810443232	6.96268810443232\\
70.25	0.18816	6.47028169634046	6.47028169634046\\
70.25	0.19182	5.87447983772103	5.87447983772103\\
70.25	0.19548	5.17528252857406	5.17528252857406\\
70.25	0.19914	4.37268976889936	4.37268976889936\\
70.25	0.2028	3.46670155869705	3.46670155869705\\
70.25	0.20646	2.45731789796727	2.45731789796727\\
70.25	0.21012	1.34453878670982	1.34453878670982\\
70.25	0.21378	0.128364224924837	0.128364224924837\\
70.25	0.21744	-1.19120578738782	-1.19120578738782\\
70.25	0.2211	-2.61417125022811	-2.61417125022811\\
70.25	0.22476	-4.14053216359599	-4.14053216359599\\
70.25	0.22842	-5.77028852749135	-5.77028852749135\\
70.25	0.23208	-7.50344034191428	-7.50344034191428\\
70.25	0.23574	-9.33998760686504	-9.33998760686504\\
70.25	0.2394	-11.2799303223433	-11.2799303223433\\
70.25	0.24306	-13.323268488349	-13.323268488349\\
70.25	0.24672	-15.4700021048824	-15.4700021048824\\
70.25	0.25038	-17.7201311719435	-17.7201311719435\\
70.25	0.25404	-20.0736556895322	-20.0736556895322\\
70.25	0.2577	-22.5305756576483	-22.5305756576483\\
70.25	0.26136	-25.0908910762922	-25.0908910762922\\
70.25	0.26502	-27.7546019454635	-27.7546019454635\\
70.25	0.26868	-30.5217082651626	-30.5217082651626\\
70.25	0.27234	-33.3922100353893	-33.3922100353893\\
70.25	0.276	-36.3661072561435	-36.3661072561435\\
70.625	0.093	-15.1227510264988	-15.1227510264988\\
70.625	0.09666	-12.9980066775123	-12.9980066775123\\
70.625	0.10032	-10.9766577790534	-10.9766577790534\\
70.625	0.10398	-9.05870433112206	-9.05870433112206\\
70.625	0.10764	-7.24414633371829	-7.24414633371829\\
70.625	0.1113	-5.53298378684218	-5.53298378684218\\
70.625	0.11496	-3.92521669049367	-3.92521669049367\\
70.625	0.11862	-2.42084504467275	-2.42084504467275\\
70.625	0.12228	-1.01986884937944	-1.01986884937944\\
70.625	0.12594	0.277711895386332	0.277711895386332\\
70.625	0.1296	1.47189718962451	1.47189718962451\\
70.625	0.13326	2.56268703333504	2.56268703333504\\
70.625	0.13692	3.55008142651796	3.55008142651796\\
70.625	0.14058	4.43408036917332	4.43408036917332\\
70.625	0.14424	5.21468386130103	5.21468386130103\\
70.625	0.1479	5.89189190290121	5.89189190290121\\
70.625	0.15156	6.46570449397373	6.46570449397373\\
70.625	0.15522	6.93612163451866	6.93612163451866\\
70.625	0.15888	7.30314332453595	7.30314332453595\\
70.625	0.16254	7.56676956402579	7.56676956402579\\
70.625	0.1662	7.72700035298793	7.72700035298793\\
70.625	0.16986	7.78383569142245	7.78383569142245\\
70.625	0.17352	7.73727557932939	7.73727557932939\\
70.625	0.17718	7.58732001670872	7.58732001670872\\
70.625	0.18084	7.33396900356047	7.33396900356047\\
70.625	0.1845	6.97722253988465	6.97722253988465\\
70.625	0.18816	6.51708062568122	6.51708062568122\\
70.625	0.19182	5.95354326095014	5.95354326095014\\
70.625	0.19548	5.28661044569159	5.28661044569159\\
70.625	0.19914	4.51628217990532	4.51628217990532\\
70.625	0.2028	3.64255846359143	3.64255846359143\\
70.625	0.20646	2.66543929674995	2.66543929674995\\
70.625	0.21012	1.58492467938092	1.58492467938092\\
70.625	0.21378	0.401014611484243	0.401014611484243\\
70.625	0.21744	-0.886290906939877	-0.886290906939877\\
70.625	0.2211	-2.27699187589187	-2.27699187589187\\
70.625	0.22476	-3.77108829537133	-3.77108829537133\\
70.625	0.22842	-5.36858016537826	-5.36858016537826\\
70.625	0.23208	-7.0694674859129	-7.0694674859129\\
70.625	0.23574	-8.87375025697511	-8.87375025697511\\
70.625	0.2394	-10.7814284785649	-10.7814284785649\\
70.625	0.24306	-12.7925021506824	-12.7925021506824\\
70.625	0.24672	-14.9069712733274	-14.9069712733274\\
70.625	0.25038	-17.1248358465001	-17.1248358465001\\
70.625	0.25404	-19.4460958702003	-19.4460958702003\\
70.625	0.2577	-21.8707513444281	-21.8707513444281\\
70.625	0.26136	-24.3988022691836	-24.3988022691836\\
70.625	0.26502	-27.0302486444666	-27.0302486444666\\
70.625	0.26868	-29.7650904702772	-29.7650904702772\\
70.625	0.27234	-32.6033277466155	-32.6033277466155\\
70.625	0.276	-35.5449604734812	-35.5449604734812\\
71	0.093	-15.9355688943136	-15.9355688943136\\
71	0.09666	-13.7785600514386	-13.7785600514386\\
71	0.10032	-11.7249466590913	-11.7249466590913\\
71	0.10398	-9.77472871727157	-9.77472871727157\\
71	0.10764	-7.92790622597944	-7.92790622597944\\
71	0.1113	-6.18447918521491	-6.18447918521491\\
71	0.11496	-4.54444759497798	-4.54444759497798\\
71	0.11862	-3.00781145526864	-3.00781145526864\\
71	0.12228	-1.57457076608691	-1.57457076608691\\
71	0.12594	-0.244725527432777	-0.244725527432777\\
71	0.1296	0.981724260693767	0.981724260693767\\
71	0.13326	2.1047785982926	2.1047785982926\\
71	0.13692	3.12443748536394	3.12443748536394\\
71	0.14058	4.04070092190777	4.04070092190777\\
71	0.14424	4.85356890792384	4.85356890792384\\
71	0.1479	5.56304144341239	5.56304144341239\\
71	0.15156	6.16911852837333	6.16911852837333\\
71	0.15522	6.67180016280668	6.67180016280668\\
71	0.15888	7.07108634671245	7.07108634671245\\
71	0.16254	7.36697708009048	7.36697708009048\\
71	0.1662	7.55947236294104	7.55947236294104\\
71	0.16986	7.64857219526398	7.64857219526398\\
71	0.17352	7.63427657705934	7.63427657705934\\
71	0.17718	7.51658550832708	7.51658550832708\\
71	0.18084	7.29549898906726	7.29549898906726\\
71	0.1845	6.9710170192798	6.9710170192798\\
71	0.18816	6.54313959896473	6.54313959896473\\
71	0.19182	6.01186672812207	6.01186672812207\\
71	0.19548	5.37719840675183	5.37719840675183\\
71	0.19914	4.63913463485397	4.63913463485397\\
71	0.2028	3.7976754124285	3.7976754124285\\
71	0.20646	2.85282073947545	2.85282073947545\\
71	0.21012	1.80457061599483	1.80457061599483\\
71	0.21378	0.652925041986578	0.652925041986578\\
71	0.21744	-0.602115982549236	-0.602115982549236\\
71	0.2211	-1.96055245761281	-1.96055245761281\\
71	0.22476	-3.42238438320373	-3.42238438320373\\
71	0.22842	-4.98761175932248	-4.98761175932248\\
71	0.23208	-6.65623458596858	-6.65623458596858\\
71	0.23574	-8.42825286314249	-8.42825286314249\\
71	0.2394	-10.3036665908439	-10.3036665908439\\
71	0.24306	-12.2824757690729	-12.2824757690729\\
71	0.24672	-14.3646803978295	-14.3646803978295\\
71	0.25038	-16.5502804771138	-16.5502804771138\\
71	0.25404	-18.8392760069256	-18.8392760069256\\
71	0.2577	-21.2316669872651	-21.2316669872651\\
71	0.26136	-23.7274534181321	-23.7274534181321\\
71	0.26502	-26.3266352995267	-26.3266352995267\\
71	0.26868	-29.029212631449	-29.029212631449\\
71	0.27234	-31.8351854138989	-31.8351854138989\\
71	0.276	-34.7445536468763	-34.7445536468763\\
71.375	0.093	-16.7691267181853	-16.7691267181853\\
71.375	0.09666	-14.579853381422	-14.579853381422\\
71.375	0.10032	-12.4939754951863	-12.4939754951863\\
71.375	0.10398	-10.5114930594781	-10.5114930594781\\
71.375	0.10764	-8.63240607429766	-8.63240607429766\\
71.375	0.1113	-6.85671453964471	-6.85671453964471\\
71.375	0.11496	-5.18441845551936	-5.18441845551936\\
71.375	0.11862	-3.61551782192166	-3.61551782192166\\
71.375	0.12228	-2.15001263885157	-2.15001263885157\\
71.375	0.12594	-0.787902906309014	-0.787902906309014\\
71.375	0.1296	0.470811375705892	0.470811375705892\\
71.375	0.13326	1.6261302071932	1.6261302071932\\
71.375	0.13692	2.67805358815296	2.67805358815296\\
71.375	0.14058	3.6265815185851	3.6265815185851\\
71.375	0.14424	4.47171399848965	4.47171399848965\\
71.375	0.1479	5.2134510278665	5.2134510278665\\
71.375	0.15156	5.85179260671586	5.85179260671586\\
71.375	0.15522	6.38673873503762	6.38673873503762\\
71.375	0.15888	6.81828941283176	6.81828941283176\\
71.375	0.16254	7.14644464009821	7.14644464009821\\
71.375	0.1662	7.37120441683719	7.37120441683719\\
71.375	0.16986	7.49256874304855	7.49256874304855\\
71.375	0.17352	7.51053761873233	7.51053761873233\\
71.375	0.17718	7.42511104388838	7.42511104388838\\
71.375	0.18084	7.23628901851691	7.23628901851691\\
71.375	0.1845	6.94407154261793	6.94407154261793\\
71.375	0.18816	6.54845861619128	6.54845861619128\\
71.375	0.19182	6.04945023923705	6.04945023923705\\
71.375	0.19548	5.44704641175511	5.44704641175511\\
71.375	0.19914	4.74124713374556	4.74124713374556\\
71.375	0.2028	3.93205240520862	3.93205240520862\\
71.375	0.20646	3.01946222614387	3.01946222614387\\
71.375	0.21012	2.00347659655168	2.00347659655168\\
71.375	0.21378	0.884095516431842	0.884095516431842\\
71.375	0.21744	-0.338681014215666	-0.338681014215666\\
71.375	0.2211	-1.6648529953907	-1.6648529953907\\
71.375	0.22476	-3.09442042709344	-3.09442042709344\\
71.375	0.22842	-4.62738330932365	-4.62738330932365\\
71.375	0.23208	-6.26374164208144	-6.26374164208144\\
71.375	0.23574	-8.00349542536694	-8.00349542536694\\
71.375	0.2394	-9.84664465917993	-9.84664465917993\\
71.375	0.24306	-11.7931893435205	-11.7931893435205\\
71.375	0.24672	-13.8431294783887	-13.8431294783887\\
71.375	0.25038	-15.9964650637846	-15.9964650637846\\
71.375	0.25404	-18.2531960997081	-18.2531960997081\\
71.375	0.2577	-20.6133225861591	-20.6133225861591\\
71.375	0.26136	-23.0768445231379	-23.0768445231379\\
71.375	0.26502	-25.6437619106441	-25.6437619106441\\
71.375	0.26868	-28.3140747486779	-28.3140747486779\\
71.375	0.27234	-31.0877830372393	-31.0877830372393\\
71.375	0.276	-33.9648867763283	-33.9648867763283\\
71.75	0.093	-17.623424498114	-17.623424498114\\
71.75	0.09666	-15.4018866674623	-15.4018866674623\\
71.75	0.10032	-13.2837442873382	-13.2837442873382\\
71.75	0.10398	-11.2689973577417	-11.2689973577417\\
71.75	0.10764	-9.35764587867273	-9.35764587867273\\
71.75	0.1113	-7.54968985013147	-7.54968985013147\\
71.75	0.11496	-5.84512927211776	-5.84512927211776\\
71.75	0.11862	-4.24396414463164	-4.24396414463164\\
71.75	0.12228	-2.74619446767313	-2.74619446767313\\
71.75	0.12594	-1.35182024124221	-1.35182024124221\\
71.75	0.1296	-0.060841465338882	-0.060841465338882\\
71.75	0.13326	1.12674186003684	1.12674186003684\\
71.75	0.13692	2.21092973488491	2.21092973488491\\
71.75	0.14058	3.19172215920547	3.19172215920547\\
71.75	0.14424	4.06911913299844	4.06911913299844\\
71.75	0.1479	4.84312065626371	4.84312065626371\\
71.75	0.15156	5.51372672900149	5.51372672900149\\
71.75	0.15522	6.08093735121156	6.08093735121156\\
71.75	0.15888	6.54475252289411	6.54475252289411\\
71.75	0.16254	6.90517224404898	6.90517224404898\\
71.75	0.1662	7.16219651467638	7.16219651467638\\
71.75	0.16986	7.31582533477605	7.31582533477605\\
71.75	0.17352	7.36605870434825	7.36605870434825\\
71.75	0.17718	7.31289662339272	7.31289662339272\\
71.75	0.18084	7.15633909190973	7.15633909190973\\
71.75	0.1845	6.896386109899	6.896386109899\\
71.75	0.18816	6.53303767736077	6.53303767736077\\
71.75	0.19182	6.06629379429495	6.06629379429495\\
71.75	0.19548	5.49615446070143	5.49615446070143\\
71.75	0.19914	4.8226196765803	4.8226196765803\\
71.75	0.2028	4.04568944193167	4.04568944193167\\
71.75	0.20646	3.16536375675534	3.16536375675534\\
71.75	0.21012	2.18164262105157	2.18164262105157\\
71.75	0.21378	1.09452603482003	1.09452603482003\\
71.75	0.21744	-0.0959860019389396	-0.0959860019389396\\
71.75	0.2211	-1.38989348922567	-1.38989348922567\\
71.75	0.22476	-2.78719642703987	-2.78719642703987\\
71.75	0.22842	-4.28789481538178	-4.28789481538178\\
71.75	0.23208	-5.89198865425126	-5.89198865425126\\
71.75	0.23574	-7.59947794364822	-7.59947794364822\\
71.75	0.2394	-9.41036268357291	-9.41036268357291\\
71.75	0.24306	-11.3246428740252	-11.3246428740252\\
71.75	0.24672	-13.342318515005	-13.342318515005\\
71.75	0.25038	-15.4633896065124	-15.4633896065124\\
71.75	0.25404	-17.6878561485475	-17.6878561485475\\
71.75	0.2577	-20.0157181411101	-20.0157181411101\\
71.75	0.26136	-22.4469755842005	-22.4469755842005\\
71.75	0.26502	-24.9816284778182	-24.9816284778182\\
71.75	0.26868	-27.6196768219637	-27.6196768219637\\
71.75	0.27234	-30.3611206166368	-30.3611206166368\\
71.75	0.276	-33.2059598618374	-33.2059598618374\\
72.125	0.093	-18.4984622340997	-18.4984622340997\\
72.125	0.09666	-16.2446599095597	-16.2446599095597\\
72.125	0.10032	-14.0942530355471	-14.0942530355471\\
72.125	0.10398	-12.0472416120622	-12.0472416120622\\
72.125	0.10764	-10.1036256391049	-10.1036256391049\\
72.125	0.1113	-8.26340511667519	-8.26340511667519\\
72.125	0.11496	-6.52658004477306	-6.52658004477306\\
72.125	0.11862	-4.89315042339857	-4.89315042339857\\
72.125	0.12228	-3.3631162525517	-3.3631162525517\\
72.125	0.12594	-1.93647753223236	-1.93647753223236\\
72.125	0.1296	-0.613234262440614	-0.613234262440614\\
72.125	0.13326	0.606613556823476	0.606613556823476\\
72.125	0.13692	1.72306592555996	1.72306592555996\\
72.125	0.14058	2.73612284376894	2.73612284376894\\
72.125	0.14424	3.64578431145021	3.64578431145021\\
72.125	0.1479	4.4520503286039	4.4520503286039\\
72.125	0.15156	5.15492089523005	5.15492089523005\\
72.125	0.15522	5.75439601132859	5.75439601132859\\
72.125	0.15888	6.25047567689946	6.25047567689946\\
72.125	0.16254	6.64315989194286	6.64315989194286\\
72.125	0.1662	6.93244865645856	6.93244865645856\\
72.125	0.16986	7.11834197044665	7.11834197044665\\
72.125	0.17352	7.20083983390715	7.20083983390715\\
72.125	0.17718	7.17994224684016	7.17994224684016\\
72.125	0.18084	7.05564920924536	7.05564920924536\\
72.125	0.1845	6.82796072112316	6.82796072112316\\
72.125	0.18816	6.49687678247335	6.49687678247335\\
72.125	0.19182	6.06239739329584	6.06239739329584\\
72.125	0.19548	5.52452255359074	5.52452255359074\\
72.125	0.19914	4.88325226335814	4.88325226335814\\
72.125	0.2028	4.13858652259782	4.13858652259782\\
72.125	0.20646	3.29052533130991	3.29052533130991\\
72.125	0.21012	2.33906868949444	2.33906868949444\\
72.125	0.21378	1.28421659715144	1.28421659715144\\
72.125	0.21744	0.125969054280773	0.125969054280773\\
72.125	0.2211	-1.13567393911754	-1.13567393911754\\
72.125	0.22476	-2.50071238304344	-2.50071238304344\\
72.125	0.22842	-3.9691462774968	-3.9691462774968\\
72.125	0.23208	-5.54097562247799	-5.54097562247799\\
72.125	0.23574	-7.21620041798664	-7.21620041798664\\
72.125	0.2394	-8.9948206640228	-8.9948206640228\\
72.125	0.24306	-10.8768363605867	-10.8768363605867\\
72.125	0.24672	-12.8622475076781	-12.8622475076781\\
72.125	0.25038	-14.9510541052972	-14.9510541052972\\
72.125	0.25404	-17.1432561534438	-17.1432561534438\\
72.125	0.2577	-19.4388536521182	-19.4388536521182\\
72.125	0.26136	-21.8378466013201	-21.8378466013201\\
72.125	0.26502	-24.3402350010494	-24.3402350010494\\
72.125	0.26868	-26.9460188513065	-26.9460188513065\\
72.125	0.27234	-29.6551981520911	-29.6551981520911\\
72.125	0.276	-32.4677729034034	-32.4677729034034\\
72.5	0.093	-19.3942399261426	-19.3942399261426\\
72.5	0.09666	-17.1081731077141	-17.1081731077141\\
72.5	0.10032	-14.9255017398132	-14.9255017398132\\
72.5	0.10398	-12.8462258224399	-12.8462258224399\\
72.5	0.10764	-10.8703453555942	-10.8703453555942\\
72.5	0.1113	-8.99786033927609	-8.99786033927609\\
72.5	0.11496	-7.22877077348559	-7.22877077348559\\
72.5	0.11862	-5.56307665822269	-5.56307665822269\\
72.5	0.12228	-4.00077799348734	-4.00077799348734\\
72.5	0.12594	-2.54187477927964	-2.54187477927964\\
72.5	0.1296	-1.18636701559953	-1.18636701559953\\
72.5	0.13326	0.0657452975529225	0.0657452975529225\\
72.5	0.13692	1.21446216017783	1.21446216017783\\
72.5	0.14058	2.25978357227517	2.25978357227517\\
72.5	0.14424	3.20170953384486	3.20170953384486\\
72.5	0.1479	4.04024004488697	4.04024004488697\\
72.5	0.15156	4.77537510540148	4.77537510540148\\
72.5	0.15522	5.40711471538839	5.40711471538839\\
72.5	0.15888	5.93545887484767	5.93545887484767\\
72.5	0.16254	6.36040758377938	6.36040758377938\\
72.5	0.1662	6.6819608421835	6.6819608421835\\
72.5	0.16986	6.90011865006001	6.90011865006001\\
72.5	0.17352	7.01488100740904	7.01488100740904\\
72.5	0.17718	7.02624791423035	7.02624791423035\\
72.5	0.18084	6.93421937052398	6.93421937052398\\
72.5	0.1845	6.73879537629008	6.73879537629008\\
72.5	0.18816	6.43997593152869	6.43997593152869\\
72.5	0.19182	6.0377610362396	6.0377610362396\\
72.5	0.19548	5.53215069042292	5.53215069042292\\
72.5	0.19914	4.92314489407863	4.92314489407863\\
72.5	0.2028	4.21074364720684	4.21074364720684\\
72.5	0.20646	3.39494694980723	3.39494694980723\\
72.5	0.21012	2.47575480188019	2.47575480188019\\
72.5	0.21378	1.45316720342549	1.45316720342549\\
72.5	0.21744	0.327184154443245	0.327184154443245\\
72.5	0.2211	-0.902194345066647	-0.902194345066647\\
72.5	0.22476	-2.23496829510412	-2.23496829510412\\
72.5	0.22842	-3.67113769566919	-3.67113769566919\\
72.5	0.23208	-5.21070254676172	-5.21070254676172\\
72.5	0.23574	-6.85366284838207	-6.85366284838207\\
72.5	0.2394	-8.60001860052992	-8.60001860052992\\
72.5	0.24306	-10.4497698032054	-10.4497698032054\\
72.5	0.24672	-12.4029164564084	-12.4029164564084\\
72.5	0.25038	-14.4594585601392	-14.4594585601392\\
72.5	0.25404	-16.6193961143974	-16.6193961143974\\
72.5	0.2577	-18.8827291191833	-18.8827291191833\\
72.5	0.26136	-21.2494575744968	-21.2494575744968\\
72.5	0.26502	-23.7195814803378	-23.7195814803378\\
72.5	0.26868	-26.2931008367065	-26.2931008367065\\
72.5	0.27234	-28.9700156436027	-28.9700156436027\\
72.5	0.276	-31.7503259010265	-31.7503259010265\\
72.875	0.093	-20.3107575742425	-20.3107575742425\\
72.875	0.09666	-17.9924262619256	-17.9924262619256\\
72.875	0.10032	-15.7774904001363	-15.7774904001363\\
72.875	0.10398	-13.6659499888746	-13.6659499888746\\
72.875	0.10764	-11.6578050281405	-11.6578050281405\\
72.875	0.1113	-9.753055517934	-9.753055517934\\
72.875	0.11496	-7.95170145825509	-7.95170145825509\\
72.875	0.11862	-6.25374284910382	-6.25374284910382\\
72.875	0.12228	-4.65917969048011	-4.65917969048011\\
72.875	0.12594	-3.16801198238399	-3.16801198238399\\
72.875	0.1296	-1.78023972481546	-1.78023972481546\\
72.875	0.13326	-0.495862917774645	-0.495862917774645\\
72.875	0.13692	0.685118438738684	0.685118438738684\\
72.875	0.14058	1.76270434472438	1.76270434472438\\
72.875	0.14424	2.7368948001825	2.7368948001825\\
72.875	0.1479	3.60768980511303	3.60768980511303\\
72.875	0.15156	4.3750893595159	4.3750893595159\\
72.875	0.15522	5.03909346339123	5.03909346339123\\
72.875	0.15888	5.59970211673887	5.59970211673887\\
72.875	0.16254	6.056915319559	6.056915319559\\
72.875	0.1662	6.41073307185154	6.41073307185154\\
72.875	0.16986	6.66115537361647	6.66115537361647\\
72.875	0.17352	6.80818222485381	6.80818222485381\\
72.875	0.17718	6.85181362556354	6.85181362556354\\
72.875	0.18084	6.79204957574558	6.79204957574558\\
72.875	0.1845	6.62889007540011	6.62889007540011\\
72.875	0.18816	6.36233512452702	6.36233512452702\\
72.875	0.19182	5.99238472312635	5.99238472312635\\
72.875	0.19548	5.51903887119809	5.51903887119809\\
72.875	0.19914	4.94229756874211	4.94229756874211\\
72.875	0.2028	4.26216081575873	4.26216081575873\\
72.875	0.20646	3.47862861224755	3.47862861224755\\
72.875	0.21012	2.59170095820892	2.59170095820892\\
72.875	0.21378	1.60137785364265	1.60137785364265\\
72.875	0.21744	0.507659298548816	0.507659298548816\\
72.875	0.2211	-0.689454707072656	-0.689454707072656\\
72.875	0.22476	-1.98996416322183	-1.98996416322183\\
72.875	0.22842	-3.39386906989847	-3.39386906989847\\
72.875	0.23208	-4.9011694271027	-4.9011694271027\\
72.875	0.23574	-6.51186523483463	-6.51186523483463\\
72.875	0.2394	-8.22595649309406	-8.22595649309406\\
72.875	0.24306	-10.0434432018812	-10.0434432018812\\
72.875	0.24672	-11.9643253611957	-11.9643253611957\\
72.875	0.25038	-13.9886029710381	-13.9886029710381\\
72.875	0.25404	-16.116276031408	-16.116276031408\\
72.875	0.2577	-18.3473445423054	-18.3473445423054\\
72.875	0.26136	-20.6818085037305	-20.6818085037305\\
72.875	0.26502	-23.1196679156831	-23.1196679156831\\
72.875	0.26868	-25.6609227781634	-25.6609227781634\\
72.875	0.27234	-28.3055730911713	-28.3055730911713\\
72.875	0.276	-31.0536188547067	-31.0536188547067\\
73.25	0.093	-21.2480151783994	-21.2480151783994\\
73.25	0.09666	-18.8974193721941	-18.8974193721941\\
73.25	0.10032	-16.6502190165164	-16.6502190165164\\
73.25	0.10398	-14.5064141113663	-14.5064141113663\\
73.25	0.10764	-12.4660046567438	-12.4660046567438\\
73.25	0.1113	-10.5289906526489	-10.5289906526489\\
73.25	0.11496	-8.69537209908165	-8.69537209908165\\
73.25	0.11862	-6.96514899604191	-6.96514899604191\\
73.25	0.12228	-5.33832134352983	-5.33832134352983\\
73.25	0.12594	-3.81488914154535	-3.81488914154535\\
73.25	0.1296	-2.39485239008846	-2.39485239008846\\
73.25	0.13326	-1.07821108915917	-1.07821108915917\\
73.25	0.13692	0.135034761242522	0.135034761242522\\
73.25	0.14058	1.24488516111664	1.24488516111664\\
73.25	0.14424	2.25134011046312	2.25134011046312\\
73.25	0.1479	3.15439960928207	3.15439960928207\\
73.25	0.15156	3.95406365757336	3.95406365757336\\
73.25	0.15522	4.65033225533705	4.65033225533705\\
73.25	0.15888	5.24320540257317	5.24320540257317\\
73.25	0.16254	5.7326830992816	5.7326830992816\\
73.25	0.1662	6.11876534546256	6.11876534546256\\
73.25	0.16986	6.4014521411158	6.4014521411158\\
73.25	0.17352	6.58074348624156	6.58074348624156\\
73.25	0.17718	6.65663938083971	6.65663938083971\\
73.25	0.18084	6.62913982491017	6.62913982491017\\
73.25	0.1845	6.49824481845312	6.49824481845312\\
73.25	0.18816	6.26395436146845	6.26395436146845\\
73.25	0.19182	5.92626845395608	5.92626845395608\\
73.25	0.19548	5.48518709591613	5.48518709591613\\
73.25	0.19914	4.94071028734868	4.94071028734868\\
73.25	0.2028	4.29283802825361	4.29283802825361\\
73.25	0.20646	3.54157031863096	3.54157031863096\\
73.25	0.21012	2.68690715848064	2.68690715848064\\
73.25	0.21378	1.72884854780278	1.72884854780278\\
73.25	0.21744	0.66739448659726	0.66739448659726\\
73.25	0.2211	-0.497455025135793	-0.497455025135793\\
73.25	0.22476	-1.76569998739654	-1.76569998739654\\
73.25	0.22842	-3.13734040018477	-3.13734040018477\\
73.25	0.23208	-4.61237626350069	-4.61237626350069\\
73.25	0.23574	-6.1908075773442	-6.1908075773442\\
73.25	0.2394	-7.87263434171521	-7.87263434171521\\
73.25	0.24306	-9.65785655661384	-9.65785655661384\\
73.25	0.24672	-11.5464742220402	-11.5464742220402\\
73.25	0.25038	-13.538487337994	-13.538487337994\\
73.25	0.25404	-15.6338959044755	-15.6338959044755\\
73.25	0.2577	-17.8326999214846	-17.8326999214846\\
73.25	0.26136	-20.1348993890213	-20.1348993890213\\
73.25	0.26502	-22.5404943070856	-22.5404943070856\\
73.25	0.26868	-25.0494846756774	-25.0494846756774\\
73.25	0.27234	-27.6618704947969	-27.6618704947969\\
73.25	0.276	-30.3776517644439	-30.3776517644439\\
73.625	0.093	-22.2060127386132	-22.2060127386132\\
73.625	0.09666	-19.8231524385195	-19.8231524385195\\
73.625	0.10032	-17.5436875889534	-17.5436875889534\\
73.625	0.10398	-15.3676181899149	-15.3676181899149\\
73.625	0.10764	-13.2949442414041	-13.2949442414041\\
73.625	0.1113	-11.3256657434208	-11.3256657434208\\
73.625	0.11496	-9.45978269596512	-9.45978269596512\\
73.625	0.11862	-7.69729509903701	-7.69729509903701\\
73.625	0.12228	-6.03820295263652	-6.03820295263652\\
73.625	0.12594	-4.48250625676367	-4.48250625676367\\
73.625	0.1296	-3.03020501141836	-3.03020501141836\\
73.625	0.13326	-1.68129921660065	-1.68129921660065\\
73.625	0.13692	-0.435788872310596	-0.435788872310596\\
73.625	0.14058	0.706326021451943	0.706326021451943\\
73.625	0.14424	1.74504546468678	1.74504546468678\\
73.625	0.1479	2.68036945739409	2.68036945739409\\
73.625	0.15156	3.5122979995738	3.5122979995738\\
73.625	0.15522	4.24083109122586	4.24083109122586\\
73.625	0.15888	4.8659687323504	4.8659687323504\\
73.625	0.16254	5.38771092294725	5.38771092294725\\
73.625	0.1662	5.80605766301652	5.80605766301652\\
73.625	0.16986	6.12100895255828	6.12100895255828\\
73.625	0.17352	6.33256479157235	6.33256479157235\\
73.625	0.17718	6.44072518005892	6.44072518005892\\
73.625	0.18084	6.44549011801769	6.44549011801769\\
73.625	0.1845	6.34685960544905	6.34685960544905\\
73.625	0.18816	6.14483364235281	6.14483364235281\\
73.625	0.19182	5.83941222872886	5.83941222872886\\
73.625	0.19548	5.43059536457733	5.43059536457733\\
73.625	0.19914	4.9183830498983	4.9183830498983\\
73.625	0.2028	4.30277528469165	4.30277528469165\\
73.625	0.20646	3.5837720689573	3.5837720689573\\
73.625	0.21012	2.7613734026954	2.7613734026954\\
73.625	0.21378	1.83557928590585	1.83557928590585\\
73.625	0.21744	0.806389718588861	0.806389718588861\\
73.625	0.2211	-0.326195299255886	-0.326195299255886\\
73.625	0.22476	-1.5621757676281	-1.5621757676281\\
73.625	0.22842	-2.90155168652802	-2.90155168652802\\
73.625	0.23208	-4.34432305595553	-4.34432305595553\\
73.625	0.23574	-5.89048987591073	-5.89048987591073\\
73.625	0.2394	-7.54005214639332	-7.54005214639332\\
73.625	0.24306	-9.29300986740353	-9.29300986740353\\
73.625	0.24672	-11.1493630389414	-11.1493630389414\\
73.625	0.25038	-13.109111661007	-13.109111661007\\
73.625	0.25404	-15.1722557336	-15.1722557336\\
73.625	0.2577	-17.3387952567207	-17.3387952567207\\
73.625	0.26136	-19.6087302303691	-19.6087302303691\\
73.625	0.26502	-21.9820606545449	-21.9820606545449\\
73.625	0.26868	-24.4587865292484	-24.4587865292484\\
73.625	0.27234	-27.0389078544795	-27.0389078544795\\
73.625	0.276	-29.7224246302382	-29.7224246302382\\
74	0.093	-23.1847502548842	-23.1847502548842\\
74	0.09666	-20.7696254609022	-20.7696254609022\\
74	0.10032	-18.4578961174477	-18.4578961174477\\
74	0.10398	-16.2495622245207	-16.2495622245207\\
74	0.10764	-14.1446237821215	-14.1446237821215\\
74	0.1113	-12.1430807902498	-12.1430807902498\\
74	0.11496	-10.2449332489058	-10.2449332489058\\
74	0.11862	-8.4501811580893	-8.4501811580893\\
74	0.12228	-6.75882451780044	-6.75882451780044\\
74	0.12594	-5.17086332803918	-5.17086332803918\\
74	0.1296	-3.68629758880545	-3.68629758880545\\
74	0.13326	-2.30512730009943	-2.30512730009943\\
74	0.13692	-1.0273524619209	-1.0273524619209\\
74	0.14058	0.147026925730003	0.147026925730003\\
74	0.14424	1.21801086285326	1.21801086285326\\
74	0.1479	2.18559934944899	2.18559934944899\\
74	0.15156	3.04979238551707	3.04979238551707\\
74	0.15522	3.81058997105754	3.81058997105754\\
74	0.15888	4.46799210607044	4.46799210607044\\
74	0.16254	5.02199879055571	5.02199879055571\\
74	0.1662	5.4726100245134	5.4726100245134\\
74	0.16986	5.81982580794347	5.81982580794347\\
74	0.17352	6.06364614084596	6.06364614084596\\
74	0.17718	6.20407102322095	6.20407102322095\\
74	0.18084	6.24110045506814	6.24110045506814\\
74	0.1845	6.17473443638792	6.17473443638792\\
74	0.18816	6.00497296717998	6.00497296717998\\
74	0.19182	5.73181604744445	5.73181604744445\\
74	0.19548	5.35526367718145	5.35526367718145\\
74	0.19914	4.87531585639061	4.87531585639061\\
74	0.2028	4.29197258507239	4.29197258507239\\
74	0.20646	3.60523386322646	3.60523386322646\\
74	0.21012	2.81509969085297	2.81509969085297\\
74	0.21378	1.92157006795185	1.92157006795185\\
74	0.21744	0.924644994523163	0.924644994523163\\
74	0.2211	-0.175675529433164	-0.175675529433164\\
74	0.22476	-1.37939150391708	-1.37939150391708\\
74	0.22842	-2.68650292892846	-2.68650292892846\\
74	0.23208	-4.09700980446766	-4.09700980446766\\
74	0.23574	-5.61091213053433	-5.61091213053433\\
74	0.2394	-7.22820990712862	-7.22820990712862\\
74	0.24306	-8.94890313425051	-8.94890313425051\\
74	0.24672	-10.7729918118999	-10.7729918118999\\
74	0.25038	-12.700475940077	-12.700475940077\\
74	0.25404	-14.7313555187817	-14.7313555187817\\
74	0.2577	-16.8656305480141	-16.8656305480141\\
74	0.26136	-19.103301027774	-19.103301027774\\
74	0.26502	-21.4443669580614	-21.4443669580614\\
74	0.26868	-23.8888283388765	-23.8888283388765\\
74	0.27234	-26.4366851702191	-26.4366851702191\\
74	0.276	-29.0879374520895	-29.0879374520895\\
};
\end{axis}

\begin{axis}[%
width=2.616cm,
height=2.517cm,
at={(6.484cm,17.483cm)},
scale only axis,
xmin=56,
xmax=74,
tick align=outside,
xlabel style={font=\color{white!15!black}},
xlabel={$L_{cut}$},
ymin=0.093,
ymax=0.276,
ylabel style={font=\color{white!15!black}},
ylabel={$D_{rlx}$},
zmin=-26.1684427873018,
zmax=200,
zlabel style={font=\color{white!15!black}},
zlabel={$x_3$},
view={-140}{50},
axis background/.style={fill=white},
xmajorgrids,
ymajorgrids,
zmajorgrids,
legend style={at={(1.03,1)}, anchor=north west, legend cell align=left, align=left, draw=white!15!black}
]
\addplot3[only marks, mark=*, mark options={}, mark size=1.5000pt, color=mycolor1, fill=mycolor1] table[row sep=crcr]{%
x	y	z\\
74	0.123	12.4705349967638\\
72	0.113	13.7429721452886\\
61	0.095	0.286855662152137\\
56	0.093	-0.357948243584596\\
};
\addlegendentry{data1}

\addplot3[only marks, mark=*, mark options={}, mark size=1.5000pt, color=mycolor2, fill=mycolor2] table[row sep=crcr]{%
x	y	z\\
67	0.276	105.068464580978\\
66	0.255	52.4873211182725\\
62	0.209	20.2412332065643\\
57	0.193	18.1546430683691\\
};
\addlegendentry{data2}

\addplot3[only marks, mark=*, mark options={}, mark size=1.5000pt, color=black, fill=black] table[row sep=crcr]{%
x	y	z\\
69	0.104	12.9676374890799\\
};
\addlegendentry{data3}

\addplot3[only marks, mark=*, mark options={}, mark size=1.5000pt, color=black, fill=black] table[row sep=crcr]{%
x	y	z\\
64	0.23	32.1293658085243\\
};
\addlegendentry{data4}


\addplot3[%
surf,
fill opacity=0.7, shader=interp, colormap={mymap}{[1pt] rgb(0pt)=(1,0.905882,0); rgb(1pt)=(1,0.901964,0); rgb(2pt)=(1,0.898051,0); rgb(3pt)=(1,0.894144,0); rgb(4pt)=(1,0.890243,0); rgb(5pt)=(1,0.886349,0); rgb(6pt)=(1,0.88246,0); rgb(7pt)=(1,0.878577,0); rgb(8pt)=(1,0.8747,0); rgb(9pt)=(1,0.870829,0); rgb(10pt)=(1,0.866964,0); rgb(11pt)=(1,0.863106,0); rgb(12pt)=(1,0.859253,0); rgb(13pt)=(1,0.855406,0); rgb(14pt)=(1,0.851566,0); rgb(15pt)=(1,0.847732,0); rgb(16pt)=(1,0.843903,0); rgb(17pt)=(1,0.840081,0); rgb(18pt)=(1,0.836265,0); rgb(19pt)=(1,0.832455,0); rgb(20pt)=(1,0.828652,0); rgb(21pt)=(1,0.824854,0); rgb(22pt)=(1,0.821063,0); rgb(23pt)=(1,0.817278,0); rgb(24pt)=(1,0.8135,0); rgb(25pt)=(1,0.809727,0); rgb(26pt)=(1,0.805961,0); rgb(27pt)=(1,0.8022,0); rgb(28pt)=(1,0.798445,0); rgb(29pt)=(1,0.794696,0); rgb(30pt)=(1,0.790953,0); rgb(31pt)=(1,0.787215,0); rgb(32pt)=(1,0.783484,0); rgb(33pt)=(1,0.779758,0); rgb(34pt)=(1,0.776038,0); rgb(35pt)=(1,0.772324,0); rgb(36pt)=(1,0.768615,0); rgb(37pt)=(1,0.764913,0); rgb(38pt)=(1,0.761217,0); rgb(39pt)=(1,0.757527,0); rgb(40pt)=(1,0.753843,0); rgb(41pt)=(1,0.750165,0); rgb(42pt)=(1,0.746493,0); rgb(43pt)=(1,0.742827,0); rgb(44pt)=(1,0.739167,0); rgb(45pt)=(1,0.735514,0); rgb(46pt)=(1,0.731867,0); rgb(47pt)=(1,0.728226,0); rgb(48pt)=(1,0.724591,0); rgb(49pt)=(1,0.720963,0); rgb(50pt)=(1,0.717341,0); rgb(51pt)=(1,0.713725,0); rgb(52pt)=(0.999994,0.710077,0); rgb(53pt)=(0.999974,0.706363,0); rgb(54pt)=(0.999942,0.702592,0); rgb(55pt)=(0.999898,0.698775,0); rgb(56pt)=(0.999841,0.694921,0); rgb(57pt)=(0.999771,0.691039,0); rgb(58pt)=(0.99969,0.687139,0); rgb(59pt)=(0.999596,0.68323,0); rgb(60pt)=(0.99949,0.679323,0); rgb(61pt)=(0.999372,0.675427,0); rgb(62pt)=(0.999242,0.67155,0); rgb(63pt)=(0.9991,0.667704,0); rgb(64pt)=(0.998946,0.663897,0); rgb(65pt)=(0.998781,0.660138,0); rgb(66pt)=(0.998605,0.656439,0); rgb(67pt)=(0.998416,0.652807,0); rgb(68pt)=(0.998217,0.649253,0); rgb(69pt)=(0.998006,0.645786,0); rgb(70pt)=(0.997785,0.642416,0); rgb(71pt)=(0.997552,0.639152,0); rgb(72pt)=(0.997308,0.636004,0); rgb(73pt)=(0.997053,0.632982,0); rgb(74pt)=(0.996788,0.630095,0); rgb(75pt)=(0.996512,0.627352,0); rgb(76pt)=(0.996226,0.624763,0); rgb(77pt)=(0.995851,0.622329,0); rgb(78pt)=(0.99494,0.619997,0); rgb(79pt)=(0.99345,0.617753,0); rgb(80pt)=(0.991419,0.61559,0); rgb(81pt)=(0.988885,0.613503,0); rgb(82pt)=(0.985886,0.611486,0); rgb(83pt)=(0.98246,0.609532,0); rgb(84pt)=(0.978643,0.607636,0); rgb(85pt)=(0.974475,0.605791,0); rgb(86pt)=(0.969992,0.603992,0); rgb(87pt)=(0.965232,0.602233,0); rgb(88pt)=(0.960233,0.600507,0); rgb(89pt)=(0.955033,0.598808,0); rgb(90pt)=(0.949669,0.59713,0); rgb(91pt)=(0.94418,0.595468,0); rgb(92pt)=(0.938602,0.593815,0); rgb(93pt)=(0.932974,0.592166,0); rgb(94pt)=(0.927333,0.590513,0); rgb(95pt)=(0.921717,0.588852,0); rgb(96pt)=(0.916164,0.587176,0); rgb(97pt)=(0.910711,0.585479,0); rgb(98pt)=(0.905397,0.583755,0); rgb(99pt)=(0.900258,0.581999,0); rgb(100pt)=(0.895333,0.580203,0); rgb(101pt)=(0.890659,0.578362,0); rgb(102pt)=(0.886275,0.576471,0); rgb(103pt)=(0.882047,0.574545,0); rgb(104pt)=(0.877819,0.572608,0); rgb(105pt)=(0.873592,0.57066,0); rgb(106pt)=(0.869366,0.568701,0); rgb(107pt)=(0.865143,0.566733,0); rgb(108pt)=(0.860924,0.564756,0); rgb(109pt)=(0.856708,0.562771,0); rgb(110pt)=(0.852497,0.560778,0); rgb(111pt)=(0.848292,0.558779,0); rgb(112pt)=(0.844092,0.556774,0); rgb(113pt)=(0.8399,0.554763,0); rgb(114pt)=(0.835716,0.552749,0); rgb(115pt)=(0.831541,0.55073,0); rgb(116pt)=(0.827374,0.548709,0); rgb(117pt)=(0.823219,0.546686,0); rgb(118pt)=(0.819074,0.54466,0); rgb(119pt)=(0.81494,0.542635,0); rgb(120pt)=(0.81082,0.540609,0); rgb(121pt)=(0.806712,0.538584,0); rgb(122pt)=(0.802619,0.53656,0); rgb(123pt)=(0.798541,0.534539,0); rgb(124pt)=(0.794478,0.532521,0); rgb(125pt)=(0.790431,0.530506,0); rgb(126pt)=(0.786402,0.528496,0); rgb(127pt)=(0.782391,0.526491,0); rgb(128pt)=(0.77841,0.524489,0); rgb(129pt)=(0.774523,0.522478,0); rgb(130pt)=(0.770731,0.520455,0); rgb(131pt)=(0.767022,0.518424,0); rgb(132pt)=(0.763384,0.516385,0); rgb(133pt)=(0.759804,0.514339,0); rgb(134pt)=(0.756272,0.51229,0); rgb(135pt)=(0.752775,0.510237,0); rgb(136pt)=(0.749302,0.508182,0); rgb(137pt)=(0.74584,0.506128,0); rgb(138pt)=(0.742378,0.504075,0); rgb(139pt)=(0.738904,0.502025,0); rgb(140pt)=(0.735406,0.499979,0); rgb(141pt)=(0.731872,0.49794,0); rgb(142pt)=(0.72829,0.495909,0); rgb(143pt)=(0.724649,0.493887,0); rgb(144pt)=(0.720936,0.491875,0); rgb(145pt)=(0.71714,0.489876,0); rgb(146pt)=(0.713249,0.487891,0); rgb(147pt)=(0.709251,0.485921,0); rgb(148pt)=(0.705134,0.483968,0); rgb(149pt)=(0.700887,0.482033,0); rgb(150pt)=(0.696497,0.480118,0); rgb(151pt)=(0.691952,0.478225,0); rgb(152pt)=(0.687242,0.476355,0); rgb(153pt)=(0.682353,0.47451,0); rgb(154pt)=(0.677195,0.472696,0); rgb(155pt)=(0.6717,0.470916,0); rgb(156pt)=(0.665891,0.469169,0); rgb(157pt)=(0.659791,0.46745,0); rgb(158pt)=(0.653423,0.465756,0); rgb(159pt)=(0.64681,0.464084,0); rgb(160pt)=(0.639976,0.462432,0); rgb(161pt)=(0.632943,0.460795,0); rgb(162pt)=(0.625734,0.459171,0); rgb(163pt)=(0.618373,0.457556,0); rgb(164pt)=(0.610882,0.455948,0); rgb(165pt)=(0.603284,0.454343,0); rgb(166pt)=(0.595604,0.452737,0); rgb(167pt)=(0.587863,0.451129,0); rgb(168pt)=(0.580084,0.449514,0); rgb(169pt)=(0.572292,0.447889,0); rgb(170pt)=(0.564508,0.446252,0); rgb(171pt)=(0.556756,0.444599,0); rgb(172pt)=(0.549059,0.442927,0); rgb(173pt)=(0.54144,0.441232,0); rgb(174pt)=(0.533922,0.439512,0); rgb(175pt)=(0.526529,0.437764,0); rgb(176pt)=(0.519282,0.435983,0); rgb(177pt)=(0.512206,0.434168,0); rgb(178pt)=(0.505323,0.432315,0); rgb(179pt)=(0.498628,0.430422,3.92506e-06); rgb(180pt)=(0.491973,0.428504,3.49981e-05); rgb(181pt)=(0.485331,0.426562,9.63073e-05); rgb(182pt)=(0.478704,0.424596,0.000186979); rgb(183pt)=(0.472096,0.422609,0.000306141); rgb(184pt)=(0.465508,0.420599,0.00045292); rgb(185pt)=(0.458942,0.418567,0.000626441); rgb(186pt)=(0.452401,0.416515,0.000825833); rgb(187pt)=(0.445885,0.414441,0.00105022); rgb(188pt)=(0.439399,0.412348,0.00129873); rgb(189pt)=(0.432942,0.410234,0.00157049); rgb(190pt)=(0.426518,0.408102,0.00186463); rgb(191pt)=(0.420129,0.40595,0.00218028); rgb(192pt)=(0.413777,0.40378,0.00251655); rgb(193pt)=(0.407464,0.401592,0.00287258); rgb(194pt)=(0.401191,0.399386,0.00324749); rgb(195pt)=(0.394962,0.397164,0.00364042); rgb(196pt)=(0.388777,0.394925,0.00405048); rgb(197pt)=(0.38264,0.39267,0.00447681); rgb(198pt)=(0.376552,0.390399,0.00491852); rgb(199pt)=(0.370516,0.388113,0.00537476); rgb(200pt)=(0.364532,0.385812,0.00584464); rgb(201pt)=(0.358605,0.383497,0.00632729); rgb(202pt)=(0.352735,0.381168,0.00682184); rgb(203pt)=(0.346925,0.378826,0.00732741); rgb(204pt)=(0.341176,0.376471,0.00784314); rgb(205pt)=(0.335485,0.374093,0.00847245); rgb(206pt)=(0.329843,0.371682,0.00930909); rgb(207pt)=(0.324249,0.369242,0.0103377); rgb(208pt)=(0.318701,0.366772,0.0115428); rgb(209pt)=(0.313198,0.364275,0.0129091); rgb(210pt)=(0.307739,0.361753,0.0144211); rgb(211pt)=(0.302322,0.359206,0.0160634); rgb(212pt)=(0.296945,0.356637,0.0178207); rgb(213pt)=(0.291607,0.354048,0.0196776); rgb(214pt)=(0.286307,0.35144,0.0216186); rgb(215pt)=(0.281043,0.348814,0.0236284); rgb(216pt)=(0.275813,0.346172,0.0256916); rgb(217pt)=(0.270616,0.343517,0.0277927); rgb(218pt)=(0.265451,0.340849,0.0299163); rgb(219pt)=(0.260317,0.33817,0.0320472); rgb(220pt)=(0.25521,0.335482,0.0341698); rgb(221pt)=(0.250131,0.332786,0.0362688); rgb(222pt)=(0.245078,0.330085,0.0383287); rgb(223pt)=(0.240048,0.327379,0.0403343); rgb(224pt)=(0.235042,0.324671,0.04227); rgb(225pt)=(0.230056,0.321962,0.0441205); rgb(226pt)=(0.22509,0.319254,0.0458704); rgb(227pt)=(0.220142,0.316548,0.0475043); rgb(228pt)=(0.215212,0.313846,0.0490067); rgb(229pt)=(0.210296,0.311149,0.0503624); rgb(230pt)=(0.205395,0.308459,0.0515759); rgb(231pt)=(0.200514,0.305763,0.052757); rgb(232pt)=(0.195655,0.303061,0.0539242); rgb(233pt)=(0.190817,0.300353,0.0550763); rgb(234pt)=(0.186001,0.297639,0.0562123); rgb(235pt)=(0.181207,0.294918,0.0573313); rgb(236pt)=(0.176434,0.292191,0.0584321); rgb(237pt)=(0.171685,0.289458,0.0595136); rgb(238pt)=(0.166957,0.286719,0.060575); rgb(239pt)=(0.162252,0.283973,0.0616151); rgb(240pt)=(0.15757,0.281221,0.0626328); rgb(241pt)=(0.152911,0.278463,0.0636271); rgb(242pt)=(0.148275,0.275699,0.0645971); rgb(243pt)=(0.143663,0.272929,0.0655416); rgb(244pt)=(0.139074,0.270152,0.0664596); rgb(245pt)=(0.134508,0.26737,0.06735); rgb(246pt)=(0.129967,0.264581,0.0682118); rgb(247pt)=(0.125449,0.261787,0.0690441); rgb(248pt)=(0.120956,0.258986,0.0698456); rgb(249pt)=(0.116487,0.25618,0.0706154); rgb(250pt)=(0.112043,0.253367,0.0713525); rgb(251pt)=(0.107623,0.250549,0.0720557); rgb(252pt)=(0.103229,0.247724,0.0727241); rgb(253pt)=(0.0988592,0.244894,0.0733566); rgb(254pt)=(0.0945149,0.242058,0.0739522); rgb(255pt)=(0.0901961,0.239216,0.0745098)}, mesh/rows=49]
table[row sep=crcr, point meta=\thisrow{c}] {%
%
x	y	z	c\\
56	0.093	-0.77223868939393	-0.77223868939393\\
56	0.09666	-3.1840895231493	-3.1840895231493\\
56	0.10032	-5.33886082333007	-5.33886082333007\\
56	0.10398	-7.23655258993583	-7.23655258993583\\
56	0.10764	-8.87716482296698	-8.87716482296698\\
56	0.1113	-10.2606975224231	-10.2606975224231\\
56	0.11496	-11.3871506883046	-11.3871506883046\\
56	0.11862	-12.2565243206111	-12.2565243206111\\
56	0.12228	-12.8688184193429	-12.8688184193429\\
56	0.12594	-13.2240329844999	-13.2240329844999\\
56	0.1296	-13.3221680160819	-13.3221680160819\\
56	0.13326	-13.1632235140894	-13.1632235140894\\
56	0.13692	-12.7471994785218	-12.7471994785218\\
56	0.14058	-12.0740959093796	-12.0740959093796\\
56	0.14424	-11.1439128066624	-11.1439128066624\\
56	0.1479	-9.95665017037052	-9.95665017037052\\
56	0.15156	-8.51230800050382	-8.51230800050382\\
56	0.15522	-6.81088629706218	-6.81088629706218\\
56	0.15888	-4.85238506004586	-4.85238506004586\\
56	0.16254	-2.6368042894546	-2.6368042894546\\
56	0.1662	-0.164143985288774	-0.164143985288774\\
56	0.16986	2.56559585245213	2.56559585245213\\
56	0.17352	5.55241522376758	5.55241522376758\\
56	0.17718	8.79631412865814	8.79631412865814\\
56	0.18084	12.2972925671233	12.2972925671233\\
56	0.1845	16.0553505391635	16.0553505391635\\
56	0.18816	20.0704880447783	20.0704880447783\\
56	0.19182	24.3427050839679	24.3427050839679\\
56	0.19548	28.8720016567324	28.8720016567324\\
56	0.19914	33.6583777630718	33.6583777630718\\
56	0.2028	38.701833402986	38.701833402986\\
56	0.20646	44.0023685764748	44.0023685764748\\
56	0.21012	49.5599832835387	49.5599832835387\\
56	0.21378	55.3746775241772	55.3746775241772\\
56	0.21744	61.4464512983906	61.4464512983906\\
56	0.2211	67.7753046061787	67.7753046061787\\
56	0.22476	74.3612374475416	74.3612374475416\\
56	0.22842	81.2042498224795	81.2042498224795\\
56	0.23208	88.3043417309921	88.3043417309921\\
56	0.23574	95.6615131730797	95.6615131730797\\
56	0.2394	103.275764148742	103.275764148742\\
56	0.24306	111.147094657979	111.147094657979\\
56	0.24672	119.275504700791	119.275504700791\\
56	0.25038	127.660994277178	127.660994277178\\
56	0.25404	136.303563387139	136.303563387139\\
56	0.2577	145.203212030675	145.203212030675\\
56	0.26136	154.359940207787	154.359940207787\\
56	0.26502	163.773747918472	163.773747918472\\
56	0.26868	173.444635162733	173.444635162733\\
56	0.27234	183.372601940569	183.372601940569\\
56	0.276	193.557648251979	193.557648251979\\
56.375	0.093	-0.800722358807661	-0.800722358807661\\
56.375	0.09666	-3.29072418411808	-3.29072418411808\\
56.375	0.10032	-5.52364647585367	-5.52364647585367\\
56.375	0.10398	-7.49948923401436	-7.49948923401436\\
56.375	0.10764	-9.21825245860033	-9.21825245860033\\
56.375	0.1113	-10.6799361496115	-10.6799361496115\\
56.375	0.11496	-11.8845403070478	-11.8845403070478\\
56.375	0.11862	-12.8320649309093	-12.8320649309093\\
56.375	0.12228	-13.522510021196	-13.522510021196\\
56.375	0.12594	-13.9558755779079	-13.9558755779079\\
56.375	0.1296	-14.132161601045	-14.132161601045\\
56.375	0.13326	-14.0513680906073	-14.0513680906073\\
56.375	0.13692	-13.7134950465947	-13.7134950465947\\
56.375	0.14058	-13.1185424690074	-13.1185424690074\\
56.375	0.14424	-12.2665103578452	-12.2665103578452\\
56.375	0.1479	-11.1573987131082	-11.1573987131082\\
56.375	0.15156	-9.7912075347964	-9.7912075347964\\
56.375	0.15522	-8.1679368229097	-8.1679368229097\\
56.375	0.15888	-6.2875865774482	-6.2875865774482\\
56.375	0.16254	-4.15015679841198	-4.15015679841198\\
56.375	0.1662	-1.75564748580098	-1.75564748580098\\
56.375	0.16986	0.895941360384882	0.895941360384882\\
56.375	0.17352	3.80460974014551	3.80460974014551\\
56.375	0.17718	6.97035765348102	6.97035765348102\\
56.375	0.18084	10.3931851003913	10.3931851003913\\
56.375	0.1845	14.0730920808765	14.0730920808765\\
56.375	0.18816	18.0100785949365	18.0100785949365\\
56.375	0.19182	22.2041446425713	22.2041446425713\\
56.375	0.19548	26.6552902237807	26.6552902237807\\
56.375	0.19914	31.3635153385652	31.3635153385652\\
56.375	0.2028	36.3288199869243	36.3288199869243\\
56.375	0.20646	41.5512041688584	41.5512041688584\\
56.375	0.21012	47.0306678843672	47.0306678843672\\
56.375	0.21378	52.7672111334508	52.7672111334508\\
56.375	0.21744	58.7608339161093	58.7608339161093\\
56.375	0.2211	65.0115362323424	65.0115362323424\\
56.375	0.22476	71.5193180821506	71.5193180821506\\
56.375	0.22842	78.2841794655334	78.2841794655334\\
56.375	0.23208	85.3061203824911	85.3061203824911\\
56.375	0.23574	92.5851408330238	92.5851408330238\\
56.375	0.2394	100.121240817131	100.121240817131\\
56.375	0.24306	107.914420334813	107.914420334813\\
56.375	0.24672	115.96467938607	115.96467938607\\
56.375	0.25038	124.272017970902	124.272017970902\\
56.375	0.25404	132.836436089308	132.836436089308\\
56.375	0.2577	141.65793374129	141.65793374129\\
56.375	0.26136	150.736510926846	150.736510926846\\
56.375	0.26502	160.072167645977	160.072167645977\\
56.375	0.26868	169.664903898683	169.664903898683\\
56.375	0.27234	179.514719684963	179.514719684963\\
56.375	0.276	189.621615004819	189.621615004819\\
56.75	0.093	-0.778921535146424	-0.778921535146424\\
56.75	0.09666	-3.34707435201166	-3.34707435201166\\
56.75	0.10032	-5.6581476353023	-5.6581476353023\\
56.75	0.10398	-7.71214138501793	-7.71214138501793\\
56.75	0.10764	-9.50905560115883	-9.50905560115883\\
56.75	0.1113	-11.0488902837249	-11.0488902837249\\
56.75	0.11496	-12.3316454327161	-12.3316454327161\\
56.75	0.11862	-13.3573210481326	-13.3573210481326\\
56.75	0.12228	-14.1259171299742	-14.1259171299742\\
56.75	0.12594	-14.637433678241	-14.637433678241\\
56.75	0.1296	-14.891870692933	-14.891870692933\\
56.75	0.13326	-14.8892281740503	-14.8892281740503\\
56.75	0.13692	-14.6295061215926	-14.6295061215926\\
56.75	0.14058	-14.1127045355602	-14.1127045355602\\
56.75	0.14424	-13.3388234159529	-13.3388234159529\\
56.75	0.1479	-12.3078627627708	-12.3078627627708\\
56.75	0.15156	-11.019822576014	-11.019822576014\\
56.75	0.15522	-9.47470285568224	-9.47470285568224\\
56.75	0.15888	-7.67250360177567	-7.67250360177567\\
56.75	0.16254	-5.6132248142944	-5.6132248142944\\
56.75	0.1662	-3.29686649323833	-3.29686649323833\\
56.75	0.16986	-0.723428638607402	-0.723428638607402\\
56.75	0.17352	2.1070887495984	2.1070887495984\\
56.75	0.17718	5.19468567137909	5.19468567137909\\
56.75	0.18084	8.53936212673426	8.53936212673426\\
56.75	0.1845	12.1411181156646	12.1411181156646\\
56.75	0.18816	15.9999536381696	15.9999536381696\\
56.75	0.19182	20.1158686942494	20.1158686942494\\
56.75	0.19548	24.488863283904	24.488863283904\\
56.75	0.19914	29.1189374071336	29.1189374071336\\
56.75	0.2028	34.0060910639377	34.0060910639377\\
56.75	0.20646	39.1503242543168	39.1503242543168\\
56.75	0.21012	44.5516369782708	44.5516369782708\\
56.75	0.21378	50.2100292357994	50.2100292357994\\
56.75	0.21744	56.125501026903	56.125501026903\\
56.75	0.2211	62.2980523515812	62.2980523515812\\
56.75	0.22476	68.7276832098344	68.7276832098344\\
56.75	0.22842	75.4143936016624	75.4143936016624\\
56.75	0.23208	82.3581835270651	82.3581835270651\\
56.75	0.23574	89.5590529860428	89.5590529860428\\
56.75	0.2394	97.0170019785952	97.0170019785952\\
56.75	0.24306	104.732030504722	104.732030504722\\
56.75	0.24672	112.704138564424	112.704138564424\\
56.75	0.25038	120.933326157701	120.933326157701\\
56.75	0.25404	129.419593284553	129.419593284553\\
56.75	0.2577	138.162939944979	138.162939944979\\
56.75	0.26136	147.163366138981	147.163366138981\\
56.75	0.26502	156.420871866557	156.420871866557\\
56.75	0.26868	165.935457127708	165.935457127708\\
56.75	0.27234	175.707121922433	175.707121922433\\
56.75	0.276	185.735866250734	185.735866250734\\
57.125	0.093	-0.706836218410444	-0.706836218410444\\
57.125	0.09666	-3.35314002683073	-3.35314002683073\\
57.125	0.10032	-5.74236430167619	-5.74236430167619\\
57.125	0.10398	-7.87450904294687	-7.87450904294687\\
57.125	0.10764	-9.7495742506426	-9.7495742506426\\
57.125	0.1113	-11.3675599247636	-11.3675599247636\\
57.125	0.11496	-12.7284660653098	-12.7284660653098\\
57.125	0.11862	-13.8322926722811	-13.8322926722811\\
57.125	0.12228	-14.6790397456777	-14.6790397456777\\
57.125	0.12594	-15.2687072854995	-15.2687072854995\\
57.125	0.1296	-15.6012952917464	-15.6012952917464\\
57.125	0.13326	-15.6768037644186	-15.6768037644186\\
57.125	0.13692	-15.4952327035158	-15.4952327035158\\
57.125	0.14058	-15.0565821090383	-15.0565821090383\\
57.125	0.14424	-14.360851980986	-14.360851980986\\
57.125	0.1479	-13.4080423193589	-13.4080423193589\\
57.125	0.15156	-12.198153124157	-12.198153124157\\
57.125	0.15522	-10.7311843953802	-10.7311843953802\\
57.125	0.15888	-9.00713613302852	-9.00713613302852\\
57.125	0.16254	-7.02600833710207	-7.02600833710207\\
57.125	0.1662	-4.78780100760105	-4.78780100760105\\
57.125	0.16986	-2.29251414452494	-2.29251414452494\\
57.125	0.17352	0.459852252125813	0.459852252125813\\
57.125	0.17718	3.46929818235145	3.46929818235145\\
57.125	0.18084	6.73582364615191	6.73582364615191\\
57.125	0.1845	10.2594286435273	10.2594286435273\\
57.125	0.18816	14.0401131744774	14.0401131744774\\
57.125	0.19182	18.0778772390022	18.0778772390022\\
57.125	0.19548	22.3727208371019	22.3727208371019\\
57.125	0.19914	26.9246439687765	26.9246439687765\\
57.125	0.2028	31.7336466340258	31.7336466340258\\
57.125	0.20646	36.7997288328499	36.7997288328499\\
57.125	0.21012	42.122890565249	42.122890565249\\
57.125	0.21378	47.7031318312227	47.7031318312227\\
57.125	0.21744	53.5404526307713	53.5404526307713\\
57.125	0.2211	59.6348529638946	59.6348529638946\\
57.125	0.22476	65.9863328305928	65.9863328305928\\
57.125	0.22842	72.5948922308659	72.5948922308659\\
57.125	0.23208	79.4605311647138	79.4605311647138\\
57.125	0.23574	86.5832496321366	86.5832496321366\\
57.125	0.2394	93.963047633134	93.963047633134\\
57.125	0.24306	101.599925167706	101.599925167706\\
57.125	0.24672	109.493882235853	109.493882235853\\
57.125	0.25038	117.644918837575	117.644918837575\\
57.125	0.25404	126.053034972872	126.053034972872\\
57.125	0.2577	134.718230641744	134.718230641744\\
57.125	0.26136	143.64050584419	143.64050584419\\
57.125	0.26502	152.819860580211	152.819860580211\\
57.125	0.26868	162.256294849807	162.256294849807\\
57.125	0.27234	171.949808652978	171.949808652978\\
57.125	0.276	181.900401989723	181.900401989723\\
57.5	0.093	-0.584466408599496	-0.584466408599496\\
57.5	0.09666	-3.30892120857472	-3.30892120857472\\
57.5	0.10032	-5.77629647497511	-5.77629647497511\\
57.5	0.10398	-7.98659220780061	-7.98659220780061\\
57.5	0.10764	-9.93980840705139	-9.93980840705139\\
57.5	0.1113	-11.6359450727273	-11.6359450727273\\
57.5	0.11496	-13.0750022048284	-13.0750022048284\\
57.5	0.11862	-14.2569798033546	-14.2569798033546\\
57.5	0.12228	-15.1818778683063	-15.1818778683063\\
57.5	0.12594	-15.8496963996829	-15.8496963996829\\
57.5	0.1296	-16.2604353974847	-16.2604353974847\\
57.5	0.13326	-16.4140948617119	-16.4140948617119\\
57.5	0.13692	-16.3106747923641	-16.3106747923641\\
57.5	0.14058	-15.9501751894414	-15.9501751894414\\
57.5	0.14424	-15.3325960529441	-15.3325960529441\\
57.5	0.1479	-14.4579373828718	-14.4579373828718\\
57.5	0.15156	-13.3261991792248	-13.3261991792248\\
57.5	0.15522	-11.9373814420029	-11.9373814420029\\
57.5	0.15888	-10.2914841712063	-10.2914841712063\\
57.5	0.16254	-8.38850736683477	-8.38850736683477\\
57.5	0.1662	-6.22845102888857	-6.22845102888857\\
57.5	0.16986	-3.81131515736752	-3.81131515736752\\
57.5	0.17352	-1.13709975227158	-1.13709975227158\\
57.5	0.17718	1.79419518639924	1.79419518639924\\
57.5	0.18084	4.98256965864465	4.98256965864465\\
57.5	0.1845	8.428023664465	8.428023664465\\
57.5	0.18816	12.1305572038602	12.1305572038602\\
57.5	0.19182	16.0901702768301	16.0901702768301\\
57.5	0.19548	20.3068628833749	20.3068628833749\\
57.5	0.19914	24.7806350234945	24.7806350234945\\
57.5	0.2028	29.5114866971889	29.5114866971889\\
57.5	0.20646	34.4994179044582	34.4994179044582\\
57.5	0.21012	39.7444286453022	39.7444286453022\\
57.5	0.21378	45.2465189197211	45.2465189197211\\
57.5	0.21744	51.0056887277147	51.0056887277147\\
57.5	0.2211	57.0219380692832	57.0219380692832\\
57.5	0.22476	63.2952669444264	63.2952669444264\\
57.5	0.22842	69.8256753531445	69.8256753531445\\
57.5	0.23208	76.6131632954374	76.6131632954374\\
57.5	0.23574	83.6577307713052	83.6577307713052\\
57.5	0.2394	90.9593777807478	90.9593777807478\\
57.5	0.24306	98.5181043237652	98.5181043237652\\
57.5	0.24672	106.333910400357	106.333910400357\\
57.5	0.25038	114.406796010524	114.406796010524\\
57.5	0.25404	122.736761154266	122.736761154266\\
57.5	0.2577	131.323805831583	131.323805831583\\
57.5	0.26136	140.167930042474	140.167930042474\\
57.5	0.26502	149.26913378694	149.26913378694\\
57.5	0.26868	158.627417064982	158.627417064982\\
57.5	0.27234	168.242779876597	168.242779876597\\
57.5	0.276	178.115222221788	178.115222221788\\
57.875	0.093	-0.411812105713693	-0.411812105713693\\
57.875	0.09666	-3.21441789724373	-3.21441789724373\\
57.875	0.10032	-5.75994415519907	-5.75994415519907\\
57.875	0.10398	-8.04839087957961	-8.04839087957961\\
57.875	0.10764	-10.0797580703852	-10.0797580703852\\
57.875	0.1113	-11.8540457276161	-11.8540457276161\\
57.875	0.11496	-13.3712538512721	-13.3712538512721\\
57.875	0.11862	-14.6313824413534	-14.6313824413534\\
57.875	0.12228	-15.6344314978598	-15.6344314978598\\
57.875	0.12594	-16.3804010207913	-16.3804010207913\\
57.875	0.1296	-16.8692910101481	-16.8692910101481\\
57.875	0.13326	-17.1011014659302	-17.1011014659302\\
57.875	0.13692	-17.0758323881374	-17.0758323881374\\
57.875	0.14058	-16.7934837767697	-16.7934837767697\\
57.875	0.14424	-16.2540556318272	-16.2540556318272\\
57.875	0.1479	-15.4575479533098	-15.4575479533098\\
57.875	0.15156	-14.4039607412178	-14.4039607412178\\
57.875	0.15522	-13.093293995551	-13.093293995551\\
57.875	0.15888	-11.5255477163091	-11.5255477163091\\
57.875	0.16254	-9.70072190349273	-9.70072190349273\\
57.875	0.1662	-7.61881655710135	-7.61881655710135\\
57.875	0.16986	-5.27983167713512	-5.27983167713512\\
57.875	0.17352	-2.68376726359423	-2.68376726359423\\
57.875	0.17718	0.169376683521762	0.169376683521762\\
57.875	0.18084	3.27960016421213	3.27960016421213\\
57.875	0.1845	6.64690317847766	6.64690317847766\\
57.875	0.18816	10.2712857263179	10.2712857263179\\
57.875	0.19182	14.1527478077329	14.1527478077329\\
57.875	0.19548	18.2912894227227	18.2912894227227\\
57.875	0.19914	22.6869105712874	22.6869105712874\\
57.875	0.2028	27.3396112534269	27.3396112534269\\
57.875	0.20646	32.2493914691412	32.2493914691412\\
57.875	0.21012	37.4162512184303	37.4162512184303\\
57.875	0.21378	42.8401905012943	42.8401905012943\\
57.875	0.21744	48.5212093177331	48.5212093177331\\
57.875	0.2211	54.4593076677465	54.4593076677465\\
57.875	0.22476	60.6544855513348	60.6544855513348\\
57.875	0.22842	67.1067429684982	67.1067429684982\\
57.875	0.23208	73.816079919236	73.816079919236\\
57.875	0.23574	80.782496403549	80.782496403549\\
57.875	0.2394	88.0059924214365	88.0059924214365\\
57.875	0.24306	95.4865679728989	95.4865679728989\\
57.875	0.24672	103.224223057936	103.224223057936\\
57.875	0.25038	111.218957676548	111.218957676548\\
57.875	0.25404	119.470771828735	119.470771828735\\
57.875	0.2577	127.979665514497	127.979665514497\\
57.875	0.26136	136.745638733833	136.745638733833\\
57.875	0.26502	145.768691486745	145.768691486745\\
57.875	0.26868	155.048823773231	155.048823773231\\
57.875	0.27234	164.586035593292	164.586035593292\\
57.875	0.276	174.380326946927	174.380326946927\\
58.25	0.093	-0.188873309752921	-0.188873309752921\\
58.25	0.09666	-3.0696300928379	-3.0696300928379\\
58.25	0.10032	-5.69330734234816	-5.69330734234816\\
58.25	0.10398	-8.05990505828353	-8.05990505828353\\
58.25	0.10764	-10.1694232406442	-10.1694232406442\\
58.25	0.1113	-12.02186188943	-12.02186188943\\
58.25	0.11496	-13.6172210046409	-13.6172210046409\\
58.25	0.11862	-14.955500586277	-14.955500586277\\
58.25	0.12228	-16.0367006343384	-16.0367006343384\\
58.25	0.12594	-16.8608211488249	-16.8608211488249\\
58.25	0.1296	-17.4278621297367	-17.4278621297367\\
58.25	0.13326	-17.7378235770736	-17.7378235770736\\
58.25	0.13692	-17.7907054908356	-17.7907054908356\\
58.25	0.14058	-17.586507871023	-17.586507871023\\
58.25	0.14424	-17.1252307176354	-17.1252307176354\\
58.25	0.1479	-16.406874030673	-16.406874030673\\
58.25	0.15156	-15.4314378101359	-15.4314378101359\\
58.25	0.15522	-14.1989220560239	-14.1989220560239\\
58.25	0.15888	-12.709326768337	-12.709326768337\\
58.25	0.16254	-10.9626519470754	-10.9626519470754\\
58.25	0.1662	-8.95889759223917	-8.95889759223917\\
58.25	0.16986	-6.69806370382787	-6.69806370382787\\
58.25	0.17352	-4.1801502818418	-4.1801502818418\\
58.25	0.17718	-1.40515732628086	-1.40515732628086\\
58.25	0.18084	1.62691516285469	1.62691516285469\\
58.25	0.1845	4.91606718556528	4.91606718556528\\
58.25	0.18816	8.46229874185059	8.46229874185059\\
58.25	0.19182	12.2656098317107	12.2656098317107\\
58.25	0.19548	16.3260004551454	16.3260004551454\\
58.25	0.19914	20.6434706121553	20.6434706121553\\
58.25	0.2028	25.2180203027398	25.2180203027398\\
58.25	0.20646	30.0496495268991	30.0496495268991\\
58.25	0.21012	35.1383582846334	35.1383582846334\\
58.25	0.21378	40.4841465759424	40.4841465759424\\
58.25	0.21744	46.0870144008263	46.0870144008263\\
58.25	0.2211	51.9469617592849	51.9469617592849\\
58.25	0.22476	58.0639886513181	58.0639886513181\\
58.25	0.22842	64.4380950769265	64.4380950769265\\
58.25	0.23208	71.0692810361095	71.0692810361095\\
58.25	0.23574	77.9575465288676	77.9575465288676\\
58.25	0.2394	85.1028915552001	85.1028915552001\\
58.25	0.24306	92.5053161151077	92.5053161151077\\
58.25	0.24672	100.16482020859	100.16482020859\\
58.25	0.25038	108.081403835647	108.081403835647\\
58.25	0.25404	116.255066996279	116.255066996279\\
58.25	0.2577	124.685809690486	124.685809690486\\
58.25	0.26136	133.373631918267	133.373631918267\\
58.25	0.26502	142.318533679624	142.318533679624\\
58.25	0.26868	151.520514974555	151.520514974555\\
58.25	0.27234	160.979575803061	160.979575803061\\
58.25	0.276	170.695716165142	170.695716165142\\
58.625	0.093	0.084349979282706	0.084349979282706\\
58.625	0.09666	-2.87455779535709	-2.87455779535709\\
58.625	0.10032	-5.57638603642229	-5.57638603642229\\
58.625	0.10398	-8.02113474391271	-8.02113474391271\\
58.625	0.10764	-10.2088039178282	-10.2088039178282\\
58.625	0.1113	-12.1393935581688	-12.1393935581688\\
58.625	0.11496	-13.8129036649348	-13.8129036649348\\
58.625	0.11862	-15.2293342381259	-15.2293342381259\\
58.625	0.12228	-16.3886852777422	-16.3886852777422\\
58.625	0.12594	-17.2909567837836	-17.2909567837836\\
58.625	0.1296	-17.9361487562503	-17.9361487562503\\
58.625	0.13326	-18.3242611951422	-18.3242611951422\\
58.625	0.13692	-18.4552941004591	-18.4552941004591\\
58.625	0.14058	-18.3292474722013	-18.3292474722013\\
58.625	0.14424	-17.9461213103687	-17.9461213103687\\
58.625	0.1479	-17.3059156149613	-17.3059156149613\\
58.625	0.15156	-16.408630385979	-16.408630385979\\
58.625	0.15522	-15.254265623422	-15.254265623422\\
58.625	0.15888	-13.8428213272901	-13.8428213272901\\
58.625	0.16254	-12.1742974975835	-12.1742974975835\\
58.625	0.1662	-10.248694134302	-10.248694134302\\
58.625	0.16986	-8.06601123744565	-8.06601123744565\\
58.625	0.17352	-5.62624880701463	-5.62624880701463\\
58.625	0.17718	-2.92940684300851	-2.92940684300851\\
58.625	0.18084	0.0245146545721013	0.0245146545721013\\
58.625	0.1845	3.23551568572776	3.23551568572776\\
58.625	0.18816	6.70359625045813	6.70359625045813\\
58.625	0.19182	10.4287563487633	10.4287563487633\\
58.625	0.19548	14.4109959806431	14.4109959806431\\
58.625	0.19914	18.650315146098	18.650315146098\\
58.625	0.2028	23.1467138451276	23.1467138451276\\
58.625	0.20646	27.900192077732	27.900192077732\\
58.625	0.21012	32.9107498439113	32.9107498439113\\
58.625	0.21378	38.1783871436654	38.1783871436654\\
58.625	0.21744	43.7031039769942	43.7031039769942\\
58.625	0.2211	49.484900343898	49.484900343898\\
58.625	0.22476	55.5237762443764	55.5237762443764\\
58.625	0.22842	61.8197316784297	61.8197316784297\\
58.625	0.23208	68.3727666460579	68.3727666460579\\
58.625	0.23574	75.182881147261	75.182881147261\\
58.625	0.2394	82.2500751820387	82.2500751820387\\
58.625	0.24306	89.5743487503912	89.5743487503912\\
58.625	0.24672	97.1557018523184	97.1557018523184\\
58.625	0.25038	104.994134487821	104.994134487821\\
58.625	0.25404	113.089646656898	113.089646656898\\
58.625	0.2577	121.44223835955	121.44223835955\\
58.625	0.26136	130.051909595776	130.051909595776\\
58.625	0.26502	138.918660365578	138.918660365578\\
58.625	0.26868	148.042490668954	148.042490668954\\
58.625	0.27234	157.423400505905	157.423400505905\\
58.625	0.276	167.061389876431	167.061389876431\\
59	0.093	0.407857761393302	0.407857761393302\\
59	0.09666	-2.62920100480154	-2.62920100480154\\
59	0.10032	-5.40918023742157	-5.40918023742157\\
59	0.10398	-7.93207993646681	-7.93207993646681\\
59	0.10764	-10.1979001019373	-10.1979001019373\\
59	0.1113	-12.206640733833	-12.206640733833\\
59	0.11496	-13.9583018321538	-13.9583018321538\\
59	0.11862	-15.4528833968998	-15.4528833968998\\
59	0.12228	-16.6903854280709	-16.6903854280709\\
59	0.12594	-17.6708079256673	-17.6708079256673\\
59	0.1296	-18.3941508896889	-18.3941508896889\\
59	0.13326	-18.8604143201358	-18.8604143201358\\
59	0.13692	-19.0695982170077	-19.0695982170077\\
59	0.14058	-19.0217025803047	-19.0217025803047\\
59	0.14424	-18.7167274100271	-18.7167274100271\\
59	0.1479	-18.1546727061746	-18.1546727061746\\
59	0.15156	-17.3355384687474	-17.3355384687474\\
59	0.15522	-16.2593246977452	-16.2593246977452\\
59	0.15888	-14.9260313931682	-14.9260313931682\\
59	0.16254	-13.3356585550166	-13.3356585550166\\
59	0.1662	-11.48820618329	-11.48820618329\\
59	0.16986	-9.38367427798858	-9.38367427798858\\
59	0.17352	-7.02206283911238	-7.02206283911238\\
59	0.17718	-4.4033718666613	-4.4033718666613\\
59	0.18084	-1.52760136063563	-1.52760136063563\\
59	0.1845	1.6052486789651	1.6052486789651\\
59	0.18816	4.99517825214053	4.99517825214053\\
59	0.19182	8.64218735889079	8.64218735889079\\
59	0.19548	12.5462759992158	12.5462759992158\\
59	0.19914	16.7074441731157	16.7074441731157\\
59	0.2028	21.1256918805904	21.1256918805904\\
59	0.20646	25.80101912164	25.80101912164\\
59	0.21012	30.7334258962642	30.7334258962642\\
59	0.21378	35.9229122044635	35.9229122044635\\
59	0.21744	41.3694780462375	41.3694780462375\\
59	0.2211	47.073123421586	47.073123421586\\
59	0.22476	53.0338483305096	53.0338483305096\\
59	0.22842	59.2516527730081	59.2516527730081\\
59	0.23208	65.7265367490812	65.7265367490812\\
59	0.23574	72.4585002587293	72.4585002587293\\
59	0.2394	79.4475433019521	79.4475433019521\\
59	0.24306	86.6936658787498	86.6936658787498\\
59	0.24672	94.1968679891222	94.1968679891222\\
59	0.25038	101.95714963307	101.95714963307\\
59	0.25404	109.974510810591	109.974510810591\\
59	0.2577	118.248951521689	118.248951521689\\
59	0.26136	126.78047176636	126.78047176636\\
59	0.26502	135.569071544607	135.569071544607\\
59	0.26868	144.614750856428	144.614750856428\\
59	0.27234	153.917509701824	153.917509701824\\
59	0.276	163.477348080795	163.477348080795\\
59.375	0.093	0.781650036578867	0.781650036578867\\
59.375	0.09666	-2.3335597211708	-2.3335597211708\\
59.375	0.10032	-5.19168994534587	-5.19168994534587\\
59.375	0.10398	-7.79274063594616	-7.79274063594616\\
59.375	0.10764	-10.1367117929715	-10.1367117929715\\
59.375	0.1113	-12.223603416422	-12.223603416422\\
59.375	0.11496	-14.0534155062978	-14.0534155062978\\
59.375	0.11862	-15.6261480625986	-15.6261480625986\\
59.375	0.12228	-16.9418010853248	-16.9418010853248\\
59.375	0.12594	-18.0003745744761	-18.0003745744761\\
59.375	0.1296	-18.8018685300526	-18.8018685300526\\
59.375	0.13326	-19.3462829520544	-19.3462829520544\\
59.375	0.13692	-19.6336178404812	-19.6336178404812\\
59.375	0.14058	-19.6638731953333	-19.6638731953333\\
59.375	0.14424	-19.4370490166105	-19.4370490166105\\
59.375	0.1479	-18.9531453043129	-18.9531453043129\\
59.375	0.15156	-18.2121620584406	-18.2121620584406\\
59.375	0.15522	-17.2140992789934	-17.2140992789934\\
59.375	0.15888	-15.9589569659713	-15.9589569659713\\
59.375	0.16254	-14.4467351193744	-14.4467351193744\\
59.375	0.1662	-12.6774337392029	-12.6774337392029\\
59.375	0.16986	-10.6510528254564	-10.6510528254564\\
59.375	0.17352	-8.36759237813516	-8.36759237813516\\
59.375	0.17718	-5.82705239723913	-5.82705239723913\\
59.375	0.18084	-3.02943288276828	-3.02943288276828\\
59.375	0.1845	0.0252661652775146	0.0252661652775146\\
59.375	0.18816	3.33704474689802	3.33704474689802\\
59.375	0.19182	6.90590286209323	6.90590286209323\\
59.375	0.19548	10.7318405108634	10.7318405108634\\
59.375	0.19914	14.8148576932084	14.8148576932084\\
59.375	0.2028	19.1549544091282	19.1549544091282\\
59.375	0.20646	23.7521306586227	23.7521306586227\\
59.375	0.21012	28.6063864416922	28.6063864416922\\
59.375	0.21378	33.7177217583363	33.7177217583363\\
59.375	0.21744	39.0861366085553	39.0861366085553\\
59.375	0.2211	44.7116309923492	44.7116309923492\\
59.375	0.22476	50.5942049097177	50.5942049097177\\
59.375	0.22842	56.7338583606614	56.7338583606614\\
59.375	0.23208	63.1305913451795	63.1305913451795\\
59.375	0.23574	69.7844038632727	69.7844038632727\\
59.375	0.2394	76.6952959149405	76.6952959149405\\
59.375	0.24306	83.8632675001834	83.8632675001834\\
59.375	0.24672	91.288318619001	91.288318619001\\
59.375	0.25038	98.9704492713934	98.9704492713934\\
59.375	0.25404	106.90965945736	106.90965945736\\
59.375	0.2577	115.105949176902	115.105949176902\\
59.375	0.26136	123.559318430019	123.559318430019\\
59.375	0.26502	132.269767216711	132.269767216711\\
59.375	0.26868	141.237295536977	141.237295536977\\
59.375	0.27234	150.461903390818	150.461903390818\\
59.375	0.276	159.943590778234	159.943590778234\\
59.75	0.093	1.20572680483929	1.20572680483929\\
59.75	0.09666	-1.98763394446543	-1.98763394446543\\
59.75	0.10032	-4.92391516019532	-4.92391516019532\\
59.75	0.10398	-7.60311684235043	-7.60311684235043\\
59.75	0.10764	-10.0252389909308	-10.0252389909308\\
59.75	0.1113	-12.1902816059364	-12.1902816059364\\
59.75	0.11496	-14.098244687367	-14.098244687367\\
59.75	0.11862	-15.7491282352229	-15.7491282352229\\
59.75	0.12228	-17.1429322495039	-17.1429322495039\\
59.75	0.12594	-18.2796567302101	-18.2796567302101\\
59.75	0.1296	-19.1593016773415	-19.1593016773415\\
59.75	0.13326	-19.7818670908983	-19.7818670908983\\
59.75	0.13692	-20.14735297088	-20.14735297088\\
59.75	0.14058	-20.255759317287	-20.255759317287\\
59.75	0.14424	-20.1070861301192	-20.1070861301192\\
59.75	0.1479	-19.7013334093765	-19.7013334093765\\
59.75	0.15156	-19.0385011550591	-19.0385011550591\\
59.75	0.15522	-18.1185893671668	-18.1185893671668\\
59.75	0.15888	-16.9415980456997	-16.9415980456997\\
59.75	0.16254	-15.5075271906579	-15.5075271906579\\
59.75	0.1662	-13.8163768020412	-13.8163768020412\\
59.75	0.16986	-11.8681468798496	-11.8681468798496\\
59.75	0.17352	-9.66283742408331	-9.66283742408331\\
59.75	0.17718	-7.2004484347421	-7.2004484347421\\
59.75	0.18084	-4.4809799118263	-4.4809799118263\\
59.75	0.1845	-1.50443185533533	-1.50443185533533\\
59.75	0.18816	1.72919573473024	1.72919573473024\\
59.75	0.19182	5.21990285837052	5.21990285837052\\
59.75	0.19548	8.96768951558573	8.96768951558573\\
59.75	0.19914	12.9725557063758	12.9725557063758\\
59.75	0.2028	17.2345014307406	17.2345014307406\\
59.75	0.20646	21.7535266886802	21.7535266886802\\
59.75	0.21012	26.5296314801947	26.5296314801947\\
59.75	0.21378	31.562815805284	31.562815805284\\
59.75	0.21744	36.8530796639483	36.8530796639483\\
59.75	0.2211	42.400423056187	42.400423056187\\
59.75	0.22476	48.2048459820007	48.2048459820007\\
59.75	0.22842	54.2663484413893	54.2663484413893\\
59.75	0.23208	60.5849304343526	60.5849304343526\\
59.75	0.23574	67.160591960891	67.160591960891\\
59.75	0.2394	73.9933330210038	73.9933330210038\\
59.75	0.24306	81.0831536146916	81.0831536146916\\
59.75	0.24672	88.4300537419541	88.4300537419541\\
59.75	0.25038	96.0340334027917	96.0340334027917\\
59.75	0.25404	103.895092597204	103.895092597204\\
59.75	0.2577	112.013231325191	112.013231325191\\
59.75	0.26136	120.388449586753	120.388449586753\\
59.75	0.26502	129.020747381889	129.020747381889\\
59.75	0.26868	137.910124710601	137.910124710601\\
59.75	0.27234	147.056581572887	147.056581572887\\
59.75	0.276	156.460117968748	156.460117968748\\
60.125	0.093	1.68008806617456	1.68008806617456\\
60.125	0.09666	-1.59142367468509	-1.59142367468509\\
60.125	0.10032	-4.60585588196992	-4.60585588196992\\
60.125	0.10398	-7.36320855568007	-7.36320855568007\\
60.125	0.10764	-9.86348169581528	-9.86348169581528\\
60.125	0.1113	-12.1066753023757	-12.1066753023757\\
60.125	0.11496	-14.0927893753614	-14.0927893753614\\
60.125	0.11862	-15.821823914772	-15.821823914772\\
60.125	0.12228	-17.2937789206081	-17.2937789206081\\
60.125	0.12594	-18.5086543928691	-18.5086543928691\\
60.125	0.1296	-19.4664503315556	-19.4664503315556\\
60.125	0.13326	-20.1671667366672	-20.1671667366672\\
60.125	0.13692	-20.6108036082039	-20.6108036082039\\
60.125	0.14058	-20.7973609461658	-20.7973609461658\\
60.125	0.14424	-20.7268387505529	-20.7268387505529\\
60.125	0.1479	-20.3992370213652	-20.3992370213652\\
60.125	0.15156	-19.8145557586026	-19.8145557586026\\
60.125	0.15522	-18.9727949622653	-18.9727949622653\\
60.125	0.15888	-17.8739546323531	-17.8739546323531\\
60.125	0.16254	-16.5180347688663	-16.5180347688663\\
60.125	0.1662	-14.9050353718045	-14.9050353718045\\
60.125	0.16986	-13.0349564411678	-13.0349564411678\\
60.125	0.17352	-10.9077979769564	-10.9077979769564\\
60.125	0.17718	-8.52355997917022	-8.52355997917022\\
60.125	0.18084	-5.88224244780923	-5.88224244780923\\
60.125	0.1845	-2.98384538287331	-2.98384538287331\\
60.125	0.18816	0.171631215637319	0.171631215637319\\
60.125	0.19182	3.58418734772289	3.58418734772289\\
60.125	0.19548	7.25382301338306	7.25382301338306\\
60.125	0.19914	11.1805382126182	11.1805382126182\\
60.125	0.2028	15.364332945428	15.364332945428\\
60.125	0.20646	19.8052072118128	19.8052072118128\\
60.125	0.21012	24.5031610117723	24.5031610117723\\
60.125	0.21378	29.4581943453068	29.4581943453068\\
60.125	0.21744	34.6703072124158	34.6703072124158\\
60.125	0.2211	40.1394996130999	40.1394996130999\\
60.125	0.22476	45.8657715473586	45.8657715473586\\
60.125	0.22842	51.8491230151921	51.8491230151921\\
60.125	0.23208	58.0895540166006	58.0895540166006\\
60.125	0.23574	64.5870645515841	64.5870645515841\\
60.125	0.2394	71.3416546201419	71.3416546201419\\
60.125	0.24306	78.3533242222749	78.3533242222749\\
60.125	0.24672	85.6220733579826	85.6220733579826\\
60.125	0.25038	93.1479020272649	93.1479020272649\\
60.125	0.25404	100.930810230122	100.930810230122\\
60.125	0.2577	108.970797966555	108.970797966555\\
60.125	0.26136	117.267865236561	117.267865236561\\
60.125	0.26502	125.822012040143	125.822012040143\\
60.125	0.26868	134.6332383773	134.6332383773\\
60.125	0.27234	143.701544248031	143.701544248031\\
60.125	0.276	153.026929652337	153.026929652337\\
60.5	0.093	2.20473382058481	2.20473382058481\\
60.5	0.09666	-1.14492891182978	-1.14492891182978\\
60.5	0.10032	-4.23751211066966	-4.23751211066966\\
60.5	0.10398	-7.07301577593452	-7.07301577593452\\
60.5	0.10764	-9.65143990762478	-9.65143990762478\\
60.5	0.1113	-11.9727845057401	-11.9727845057401\\
60.5	0.11496	-14.0370495702807	-14.0370495702807\\
60.5	0.11862	-15.8442351012465	-15.8442351012465\\
60.5	0.12228	-17.3943410986373	-17.3943410986373\\
60.5	0.12594	-18.6873675624534	-18.6873675624534\\
60.5	0.1296	-19.7233144926947	-19.7233144926947\\
60.5	0.13326	-20.5021818893612	-20.5021818893612\\
60.5	0.13692	-21.0239697524528	-21.0239697524528\\
60.5	0.14058	-21.2886780819696	-21.2886780819696\\
60.5	0.14424	-21.2963068779117	-21.2963068779117\\
60.5	0.1479	-21.0468561402788	-21.0468561402788\\
60.5	0.15156	-20.5403258690714	-20.5403258690714\\
60.5	0.15522	-19.7767160642889	-19.7767160642889\\
60.5	0.15888	-18.7560267259316	-18.7560267259316\\
60.5	0.16254	-17.4782578539997	-17.4782578539997\\
60.5	0.1662	-15.9434094484928	-15.9434094484928\\
60.5	0.16986	-14.1514815094112	-14.1514815094112\\
60.5	0.17352	-12.1024740367547	-12.1024740367547\\
60.5	0.17718	-9.79638703052336	-9.79638703052336\\
60.5	0.18084	-7.23322049071731	-7.23322049071731\\
60.5	0.1845	-4.41297441733633	-4.41297441733633\\
60.5	0.18816	-1.33564881038063	-1.33564881038063\\
60.5	0.19182	1.99875633014977	1.99875633014977\\
60.5	0.19548	5.59024100425523	5.59024100425523\\
60.5	0.19914	9.43880521193557	9.43880521193557\\
60.5	0.2028	13.5444489531904	13.5444489531904\\
60.5	0.20646	17.9071722280202	17.9071722280202\\
60.5	0.21012	22.5269750364249	22.5269750364249\\
60.5	0.21378	27.4038573784043	27.4038573784043\\
60.5	0.21744	32.5378192539585	32.5378192539585\\
60.5	0.2211	37.9288606630875	37.9288606630875\\
60.5	0.22476	43.5769816057912	43.5769816057912\\
60.5	0.22842	49.4821820820701	49.4821820820701\\
60.5	0.23208	55.6444620919235	55.6444620919235\\
60.5	0.23574	62.0638216353518	62.0638216353518\\
60.5	0.2394	68.7402607123549	68.7402607123549\\
60.5	0.24306	75.6737793229329	75.6737793229329\\
60.5	0.24672	82.8643774670855	82.8643774670855\\
60.5	0.25038	90.3120551448133	90.3120551448133\\
60.5	0.25404	98.0168123561156	98.0168123561156\\
60.5	0.2577	105.978649100993	105.978649100993\\
60.5	0.26136	114.197565379445	114.197565379445\\
60.5	0.26502	122.673561191472	122.673561191472\\
60.5	0.26868	131.406636537073	131.406636537073\\
60.5	0.27234	140.39679141625	140.39679141625\\
60.5	0.276	149.644025829001	149.644025829001\\
60.875	0.093	2.77966406806991	2.77966406806991\\
60.875	0.09666	-0.648149655899616	-0.648149655899616\\
60.875	0.10032	-3.81888384629443	-3.81888384629443\\
60.875	0.10398	-6.73253850311434	-6.73253850311434\\
60.875	0.10764	-9.38911362635942	-9.38911362635942\\
60.875	0.1113	-11.7886092160297	-11.7886092160297\\
60.875	0.11496	-13.9310252721251	-13.9310252721251\\
60.875	0.11862	-15.8163617946458	-15.8163617946458\\
60.875	0.12228	-17.4446187835916	-17.4446187835916\\
60.875	0.12594	-18.8157962389626	-18.8157962389626\\
60.875	0.1296	-19.9298941607589	-19.9298941607589\\
60.875	0.13326	-20.7869125489803	-20.7869125489803\\
60.875	0.13692	-21.3868514036268	-21.3868514036268\\
60.875	0.14058	-21.7297107246985	-21.7297107246985\\
60.875	0.14424	-21.8154905121956	-21.8154905121956\\
60.875	0.1479	-21.6441907661176	-21.6441907661176\\
60.875	0.15156	-21.2158114864651	-21.2158114864651\\
60.875	0.15522	-20.5303526732376	-20.5303526732376\\
60.875	0.15888	-19.5878143264352	-19.5878143264352\\
60.875	0.16254	-18.3881964460582	-18.3881964460582\\
60.875	0.1662	-16.9314990321063	-16.9314990321063\\
60.875	0.16986	-15.2177220845796	-15.2177220845796\\
60.875	0.17352	-13.2468656034779	-13.2468656034779\\
60.875	0.17718	-11.0189295888017	-11.0189295888017\\
60.875	0.18084	-8.53391404055054	-8.53391404055054\\
60.875	0.1845	-5.79181895872438	-5.79181895872438\\
60.875	0.18816	-2.79264434332362	-2.79264434332362\\
60.875	0.19182	0.46360980565197	0.46360980565197\\
60.875	0.19548	3.97694348820227	3.97694348820227\\
60.875	0.19914	7.74735670432767	7.74735670432767\\
60.875	0.2028	11.7748494540277	11.7748494540277\\
60.875	0.20646	16.0594217373024	16.0594217373024\\
60.875	0.21012	20.6010735541523	20.6010735541523\\
60.875	0.21378	25.3998049045769	25.3998049045769\\
60.875	0.21744	30.455615788576	30.455615788576\\
60.875	0.2211	35.76850620615	35.76850620615\\
60.875	0.22476	41.3384761572991	41.3384761572991\\
60.875	0.22842	47.1655256420229	47.1655256420229\\
60.875	0.23208	53.2496546603213	53.2496546603213\\
60.875	0.23574	59.5908632121948	59.5908632121948\\
60.875	0.2394	66.1891512976429	66.1891512976429\\
60.875	0.24306	73.044518916666	73.044518916666\\
60.875	0.24672	80.1569660692638	80.1569660692638\\
60.875	0.25038	87.5264927554365	87.5264927554365\\
60.875	0.25404	95.1530989751838	95.1530989751838\\
60.875	0.2577	103.036784728506	103.036784728506\\
60.875	0.26136	111.177550015403	111.177550015403\\
60.875	0.26502	119.575394835875	119.575394835875\\
60.875	0.26868	128.230319189922	128.230319189922\\
60.875	0.27234	137.142323077543	137.142323077543\\
60.875	0.276	146.31140649874	146.31140649874\\
61.25	0.093	3.40487880862986	3.40487880862986\\
61.25	0.09666	-0.101085906894482	-0.101085906894482\\
61.25	0.10032	-3.34997108884423	-3.34997108884423\\
61.25	0.10398	-6.34177673721908	-6.34177673721908\\
61.25	0.10764	-9.07650285201909	-9.07650285201909\\
61.25	0.1113	-11.5541494332443	-11.5541494332443\\
61.25	0.11496	-13.7747164808948	-13.7747164808948\\
61.25	0.11862	-15.7382039949703	-15.7382039949703\\
61.25	0.12228	-17.444611975471	-17.444611975471\\
61.25	0.12594	-18.893940422397	-18.893940422397\\
61.25	0.1296	-20.0861893357481	-20.0861893357481\\
61.25	0.13326	-21.0213587155246	-21.0213587155246\\
61.25	0.13692	-21.699448561726	-21.699448561726\\
61.25	0.14058	-22.1204588743527	-22.1204588743527\\
61.25	0.14424	-22.2843896534046	-22.2843896534046\\
61.25	0.1479	-22.1912408988817	-22.1912408988817\\
61.25	0.15156	-21.841012610784	-21.841012610784\\
61.25	0.15522	-21.2337047891113	-21.2337047891113\\
61.25	0.15888	-20.369317433864	-20.369317433864\\
61.25	0.16254	-19.2478505450418	-19.2478505450418\\
61.25	0.1662	-17.8693041226448	-17.8693041226448\\
61.25	0.16986	-16.233678166673	-16.233678166673\\
61.25	0.17352	-14.3409726771264	-14.3409726771264\\
61.25	0.17718	-12.191187654005	-12.191187654005\\
61.25	0.18084	-9.7843230973088	-9.7843230973088\\
61.25	0.1845	-7.12037900703768	-7.12037900703768\\
61.25	0.18816	-4.19935538319174	-4.19935538319174\\
61.25	0.19182	-1.02125222577109	-1.02125222577109\\
61.25	0.19548	2.41393046522415	2.41393046522415\\
61.25	0.19914	6.10619268979474	6.10619268979474\\
61.25	0.2028	10.0555344479397	10.0555344479397\\
61.25	0.20646	14.2619557396596	14.2619557396596\\
61.25	0.21012	18.7254565649544	18.7254565649544\\
61.25	0.21378	23.4460369238242	23.4460369238242\\
61.25	0.21744	28.4236968162683	28.4236968162683\\
61.25	0.2211	33.6584362422877	33.6584362422877\\
61.25	0.22476	39.1502552018815	39.1502552018815\\
61.25	0.22842	44.8991536950505	44.8991536950505\\
61.25	0.23208	50.9051317217941	50.9051317217941\\
61.25	0.23574	57.1681892821127	57.1681892821127\\
61.25	0.2394	63.6883263760058	63.6883263760058\\
61.25	0.24306	70.4655430034741	70.4655430034741\\
61.25	0.24672	77.4998391645166	77.4998391645166\\
61.25	0.25038	84.7912148591345	84.7912148591345\\
61.25	0.25404	92.3396700873269	92.3396700873269\\
61.25	0.2577	100.145204849094	100.145204849094\\
61.25	0.26136	108.207819144436	108.207819144436\\
61.25	0.26502	116.527512973354	116.527512973354\\
61.25	0.26868	125.104286335845	125.104286335845\\
61.25	0.27234	133.938139231912	133.938139231912\\
61.25	0.276	143.029071661553	143.029071661553\\
61.625	0.093	4.0803780422649	4.0803780422649\\
61.625	0.09666	0.496262335185506	0.496262335185506\\
61.625	0.10032	-2.83077383831906	-2.83077383831906\\
61.625	0.10398	-5.90073047824885	-5.90073047824885\\
61.625	0.10764	-8.71360758460379	-8.71360758460379\\
61.625	0.1113	-11.2694051573839	-11.2694051573839\\
61.625	0.11496	-13.5681231965892	-13.5681231965892\\
61.625	0.11862	-15.6097617022198	-15.6097617022198\\
61.625	0.12228	-17.3943206742753	-17.3943206742753\\
61.625	0.12594	-18.9218001127564	-18.9218001127564\\
61.625	0.1296	-20.1922000176623	-20.1922000176623\\
61.625	0.13326	-21.2055203889937	-21.2055203889937\\
61.625	0.13692	-21.9617612267501	-21.9617612267501\\
61.625	0.14058	-22.4609225309318	-22.4609225309318\\
61.625	0.14424	-22.7030043015386	-22.7030043015386\\
61.625	0.1479	-22.6880065385706	-22.6880065385706\\
61.625	0.15156	-22.4159292420278	-22.4159292420278\\
61.625	0.15522	-21.8867724119101	-21.8867724119101\\
61.625	0.15888	-21.1005360482177	-21.1005360482177\\
61.625	0.16254	-20.0572201509504	-20.0572201509504\\
61.625	0.1662	-18.7568247201084	-18.7568247201084\\
61.625	0.16986	-17.1993497556915	-17.1993497556915\\
61.625	0.17352	-15.3847952576997	-15.3847952576997\\
61.625	0.17718	-13.3131612261333	-13.3131612261333\\
61.625	0.18084	-10.984447660992	-10.984447660992\\
61.625	0.1845	-8.39865456227579	-8.39865456227579\\
61.625	0.18816	-5.55578192998479	-5.55578192998479\\
61.625	0.19182	-2.45582976411919	-2.45582976411919\\
61.625	0.19548	0.901201935321239	0.901201935321239\\
61.625	0.19914	4.51531316833677	4.51531316833677\\
61.625	0.2028	8.38650393492713	8.38650393492713\\
61.625	0.20646	12.514774235092	12.514774235092\\
61.625	0.21012	16.9001240688318	16.9001240688318\\
61.625	0.21378	21.5425534361465	21.5425534361465\\
61.625	0.21744	26.442062337036	26.442062337036\\
61.625	0.2211	31.5986507715001	31.5986507715001\\
61.625	0.22476	37.0123187395391	37.0123187395391\\
61.625	0.22842	42.6830662411531	42.6830662411531\\
61.625	0.23208	48.6108932763418	48.6108932763418\\
61.625	0.23574	54.7957998451056	54.7957998451056\\
61.625	0.2394	61.2377859474436	61.2377859474436\\
61.625	0.24306	67.9368515833569	67.9368515833569\\
61.625	0.24672	74.8929967528449	74.8929967528449\\
61.625	0.25038	82.1062214559076	82.1062214559076\\
61.625	0.25404	89.5765256925453	89.5765256925453\\
61.625	0.2577	97.3039094627577	97.3039094627577\\
61.625	0.26136	105.288372766545	105.288372766545\\
61.625	0.26502	113.529915603907	113.529915603907\\
61.625	0.26868	122.028537974844	122.028537974844\\
61.625	0.27234	130.784239879355	130.784239879355\\
61.625	0.276	139.797021317442	139.797021317442\\
62	0.093	4.80616176897479	4.80616176897479\\
62	0.09666	1.14389507034058	1.14389507034058\\
62	0.10032	-2.26129209471904	-2.26129209471904\\
62	0.10398	-5.40939972620365	-5.40939972620365\\
62	0.10764	-8.30042782411353	-8.30042782411353\\
62	0.1113	-10.9343763884486	-10.9343763884486\\
62	0.11496	-13.3112454192088	-13.3112454192088\\
62	0.11862	-15.4310349163943	-15.4310349163943\\
62	0.12228	-17.2937448800049	-17.2937448800049\\
62	0.12594	-18.8993753100408	-18.8993753100408\\
62	0.1296	-20.2479262065018	-20.2479262065018\\
62	0.13326	-21.339397569388	-21.339397569388\\
62	0.13692	-22.1737893986994	-22.1737893986994\\
62	0.14058	-22.7511016944359	-22.7511016944359\\
62	0.14424	-23.0713344565977	-23.0713344565977\\
62	0.1479	-23.1344876851846	-23.1344876851846\\
62	0.15156	-22.9405613801968	-22.9405613801968\\
62	0.15522	-22.489555541634	-22.489555541634\\
62	0.15888	-21.7814701694965	-21.7814701694965\\
62	0.16254	-20.8163052637842	-20.8163052637842\\
62	0.1662	-19.5940608244971	-19.5940608244971\\
62	0.16986	-18.114736851635	-18.114736851635\\
62	0.17352	-16.3783333451984	-16.3783333451984\\
62	0.17718	-14.3848503051866	-14.3848503051866\\
62	0.18084	-12.1342877316004	-12.1342877316004\\
62	0.1845	-9.62664562443916	-9.62664562443916\\
62	0.18816	-6.86192398370321	-6.86192398370321\\
62	0.19182	-3.8401228093922	-3.8401228093922\\
62	0.19548	-0.561242101506821	-0.561242101506821\\
62	0.19914	2.97471813995367	2.97471813995367\\
62	0.2028	6.76775791498898	6.76775791498898\\
62	0.20646	10.8178772235991	10.8178772235991\\
62	0.21012	15.125076065784	15.125076065784\\
62	0.21378	19.6893544415436	19.6893544415436\\
62	0.21744	24.5107123508784	24.5107123508784\\
62	0.2211	29.5891497937876	29.5891497937876\\
62	0.22476	34.9246667702718	34.9246667702718\\
62	0.22842	40.5172632803307	40.5172632803307\\
62	0.23208	46.3669393239644	46.3669393239644\\
62	0.23574	52.4736949011732	52.4736949011732\\
62	0.2394	58.8375300119563	58.8375300119563\\
62	0.24306	65.4584446563148	65.4584446563148\\
62	0.24672	72.3364388342477	72.3364388342477\\
62	0.25038	79.4715125457557	79.4715125457557\\
62	0.25404	86.8636657908382	86.8636657908382\\
62	0.2577	94.5128985694959	94.5128985694959\\
62	0.26136	102.419210881728	102.419210881728\\
62	0.26502	110.582602727535	110.582602727535\\
62	0.26868	119.003074106917	119.003074106917\\
62	0.27234	127.680625019874	127.680625019874\\
62	0.276	136.615255466405	136.615255466405\\
62.375	0.093	5.58222998875965	5.58222998875965\\
62.375	0.09666	1.84181229857039	1.84181229857039\\
62.375	0.10032	-1.64152585804405	-1.64152585804405\\
62.375	0.10398	-4.8677844810837	-4.8677844810837\\
62.375	0.10764	-7.83696357054852	-7.83696357054852\\
62.375	0.1113	-10.5490631264384	-10.5490631264384\\
62.375	0.11496	-13.0040831487537	-13.0040831487537\\
62.375	0.11862	-15.202023637494	-15.202023637494\\
62.375	0.12228	-17.1428845926596	-17.1428845926596\\
62.375	0.12594	-18.8266660142503	-18.8266660142503\\
62.375	0.1296	-20.2533679022663	-20.2533679022663\\
62.375	0.13326	-21.4229902567074	-21.4229902567074\\
62.375	0.13692	-22.3355330775736	-22.3355330775736\\
62.375	0.14058	-22.9909963648652	-22.9909963648652\\
62.375	0.14424	-23.3893801185818	-23.3893801185818\\
62.375	0.1479	-23.5306843387238	-23.5306843387238\\
62.375	0.15156	-23.4149090252907	-23.4149090252907\\
62.375	0.15522	-23.042054178283	-23.042054178283\\
62.375	0.15888	-22.4121197977003	-22.4121197977003\\
62.375	0.16254	-21.5251058835431	-21.5251058835431\\
62.375	0.1662	-20.3810124358108	-20.3810124358108\\
62.375	0.16986	-18.9798394545038	-18.9798394545038\\
62.375	0.17352	-17.321586939622	-17.321586939622\\
62.375	0.17718	-15.4062548911652	-15.4062548911652\\
62.375	0.18084	-13.2338433091339	-13.2338433091339\\
62.375	0.1845	-10.8043521935276	-10.8043521935276\\
62.375	0.18816	-8.11778154434631	-8.11778154434631\\
62.375	0.19182	-5.17413136159058	-5.17413136159058\\
62.375	0.19548	-1.97340164526003	-1.97340164526003\\
62.375	0.19914	1.48440760464564	1.48440760464564\\
62.375	0.2028	5.19929638812613	5.19929638812613\\
62.375	0.20646	9.17126470518116	9.17126470518116\\
62.375	0.21012	13.400312555811	13.400312555811\\
62.375	0.21378	17.8864399400161	17.8864399400161\\
62.375	0.21744	22.6296468577955	22.6296468577955\\
62.375	0.2211	27.6299333091497	27.6299333091497\\
62.375	0.22476	32.8872992940791	32.8872992940791\\
62.375	0.22842	38.4017448125832	38.4017448125832\\
62.375	0.23208	44.1732698646618	44.1732698646618\\
62.375	0.23574	50.2018744503158	50.2018744503158\\
62.375	0.2394	56.4875585695441	56.4875585695441\\
62.375	0.24306	63.0303222223475	63.0303222223475\\
62.375	0.24672	69.8301654087254	69.8301654087254\\
62.375	0.25038	76.8870881286786	76.8870881286786\\
62.375	0.25404	84.2010903822061	84.2010903822061\\
62.375	0.2577	91.7721721693089	91.7721721693089\\
62.375	0.26136	99.600333489986	99.600333489986\\
62.375	0.26502	107.685574344238	107.685574344238\\
62.375	0.26868	116.027894732065	116.027894732065\\
62.375	0.27234	124.627294653467	124.627294653467\\
62.375	0.276	133.483774108444	133.483774108444\\
62.75	0.093	6.40858270161937	6.40858270161937\\
62.75	0.09666	2.59001401987517	2.59001401987517\\
62.75	0.10032	-0.971475128294202	-0.971475128294202\\
62.75	0.10398	-4.27588474288879	-4.27588474288879\\
62.75	0.10764	-7.32321482390854	-7.32321482390854\\
62.75	0.1113	-10.1134653713535	-10.1134653713535\\
62.75	0.11496	-12.6466363852236	-12.6466363852236\\
62.75	0.11862	-14.9227278655188	-14.9227278655188\\
62.75	0.12228	-16.9417398122393	-16.9417398122393\\
62.75	0.12594	-18.7036722253849	-18.7036722253849\\
62.75	0.1296	-20.2085251049558	-20.2085251049558\\
62.75	0.13326	-21.4562984509519	-21.4562984509519\\
62.75	0.13692	-22.4469922633731	-22.4469922633731\\
62.75	0.14058	-23.1806065422195	-23.1806065422195\\
62.75	0.14424	-23.6571412874912	-23.6571412874912\\
62.75	0.1479	-23.876596499188	-23.876596499188\\
62.75	0.15156	-23.8389721773099	-23.8389721773099\\
62.75	0.15522	-23.5442683218571	-23.5442683218571\\
62.75	0.15888	-22.9924849328293	-22.9924849328293\\
62.75	0.16254	-22.1836220102268	-22.1836220102268\\
62.75	0.1662	-21.1176795540497	-21.1176795540497\\
62.75	0.16986	-19.7946575642975	-19.7946575642975\\
62.75	0.17352	-18.2145560409706	-18.2145560409706\\
62.75	0.17718	-16.3773749840689	-16.3773749840689\\
62.75	0.18084	-14.2831143935925	-14.2831143935925\\
62.75	0.1845	-11.9317742695409	-11.9317742695409\\
62.75	0.18816	-9.32335461191491	-9.32335461191491\\
62.75	0.19182	-6.457855420714	-6.457855420714\\
62.75	0.19548	-3.33527669593826	-3.33527669593826\\
62.75	0.19914	0.0443815624125818	0.0443815624125818\\
62.75	0.2028	3.6811193543378	3.6811193543378\\
62.75	0.20646	7.574936679838	7.574936679838\\
62.75	0.21012	11.725833538913	11.725833538913\\
62.75	0.21378	16.1338099315631	16.1338099315631\\
62.75	0.21744	20.7988658577877	20.7988658577877\\
62.75	0.2211	25.7210013175871	25.7210013175871\\
62.75	0.22476	30.9002163109614	30.9002163109614\\
62.75	0.22842	36.3365108379105	36.3365108379105\\
62.75	0.23208	42.0298848984345	42.0298848984345\\
62.75	0.23574	47.9803384925332	47.9803384925332\\
62.75	0.2394	54.1878716202069	54.1878716202069\\
62.75	0.24306	60.652484281455	60.652484281455\\
62.75	0.24672	67.3741764762783	67.3741764762783\\
62.75	0.25038	74.3529482046762	74.3529482046762\\
62.75	0.25404	81.5887994666491	81.5887994666491\\
62.75	0.2577	89.0817302621969	89.0817302621969\\
62.75	0.26136	96.8317405913192	96.8317405913192\\
62.75	0.26502	104.838830454017	104.838830454017\\
62.75	0.26868	113.102999850288	113.102999850288\\
62.75	0.27234	121.624248780135	121.624248780135\\
62.75	0.276	130.402577243557	130.402577243557\\
63.125	0.093	7.28521990755382	7.28521990755382\\
63.125	0.09666	3.38850023425481	3.38850023425481\\
63.125	0.10032	-0.2511399054695	-0.2511399054695\\
63.125	0.10398	-3.63370051161903	-3.63370051161903\\
63.125	0.10764	-6.75918158419371	-6.75918158419371\\
63.125	0.1113	-9.62758312319346	-9.62758312319346\\
63.125	0.11496	-12.2389051286186	-12.2389051286186\\
63.125	0.11862	-14.5931476004688	-14.5931476004688\\
63.125	0.12228	-16.6903105387442	-16.6903105387442\\
63.125	0.12594	-18.5303939434447	-18.5303939434447\\
63.125	0.1296	-20.1133978145706	-20.1133978145706\\
63.125	0.13326	-21.4393221521216	-21.4393221521216\\
63.125	0.13692	-22.5081669560978	-22.5081669560978\\
63.125	0.14058	-23.319932226499	-23.319932226499\\
63.125	0.14424	-23.8746179633257	-23.8746179633257\\
63.125	0.1479	-24.1722241665773	-24.1722241665773\\
63.125	0.15156	-24.2127508362543	-24.2127508362543\\
63.125	0.15522	-23.9961979723563	-23.9961979723563\\
63.125	0.15888	-23.5225655748835	-23.5225655748835\\
63.125	0.16254	-22.7918536438361	-22.7918536438361\\
63.125	0.1662	-21.8040621792137	-21.8040621792137\\
63.125	0.16986	-20.5591911810166	-20.5591911810166\\
63.125	0.17352	-19.0572406492446	-19.0572406492446\\
63.125	0.17718	-17.2982105838977	-17.2982105838977\\
63.125	0.18084	-15.2821009849761	-15.2821009849761\\
63.125	0.1845	-13.0089118524796	-13.0089118524796\\
63.125	0.18816	-10.4786431864086	-10.4786431864086\\
63.125	0.19182	-7.69129498676256	-7.69129498676256\\
63.125	0.19548	-4.64686725354164	-4.64686725354164\\
63.125	0.19914	-1.34535998674585	-1.34535998674585\\
63.125	0.2028	2.21322681362454	2.21322681362454\\
63.125	0.20646	6.02889314756993	6.02889314756993\\
63.125	0.21012	10.1016390150899	10.1016390150899\\
63.125	0.21378	14.4314644161849	14.4314644161849\\
63.125	0.21744	19.0183693508547	19.0183693508547\\
63.125	0.2211	23.862353819099	23.862353819099\\
63.125	0.22476	28.9634178209185	28.9634178209185\\
63.125	0.22842	34.3215613563125	34.3215613563125\\
63.125	0.23208	39.9367844252815	39.9367844252815\\
63.125	0.23574	45.8090870278256	45.8090870278256\\
63.125	0.2394	51.9384691639441	51.9384691639441\\
63.125	0.24306	58.3249308336376	58.3249308336376\\
63.125	0.24672	64.9684720369056	64.9684720369056\\
63.125	0.25038	71.8690927737489	71.8690927737489\\
63.125	0.25404	79.0267930441668	79.0267930441668\\
63.125	0.2577	86.4415728481595	86.4415728481595\\
63.125	0.26136	94.113432185727	94.113432185727\\
63.125	0.26502	102.042371056869	102.042371056869\\
63.125	0.26868	110.228389461586	110.228389461586\\
63.125	0.27234	118.671487399878	118.671487399878\\
63.125	0.276	127.371664871745	127.371664871745\\
63.5	0.093	8.21214160656336	8.21214160656336\\
63.5	0.09666	4.23727094170941	4.23727094170941\\
63.5	0.10032	0.51947981043017	0.51947981043017\\
63.5	0.10398	-2.94123178727429	-2.94123178727429\\
63.5	0.10764	-6.14486385140391	-6.14486385140391\\
63.5	0.1113	-9.09141638195871	-9.09141638195871\\
63.5	0.11496	-11.7808893789386	-11.7808893789386\\
63.5	0.11862	-14.2132828423437	-14.2132828423437\\
63.5	0.12228	-16.3885967721741	-16.3885967721741\\
63.5	0.12594	-18.3068311684296	-18.3068311684296\\
63.5	0.1296	-19.9679860311104	-19.9679860311104\\
63.5	0.13326	-21.3720613602163	-21.3720613602163\\
63.5	0.13692	-22.5190571557472	-22.5190571557472\\
63.5	0.14058	-23.4089734177036	-23.4089734177036\\
63.5	0.14424	-24.041810146085	-24.041810146085\\
63.5	0.1479	-24.4175673408917	-24.4175673408917\\
63.5	0.15156	-24.5362450021236	-24.5362450021236\\
63.5	0.15522	-24.3978431297805	-24.3978431297805\\
63.5	0.15888	-24.0023617238627	-24.0023617238627\\
63.5	0.16254	-23.3498007843702	-23.3498007843702\\
63.5	0.1662	-22.4401603113027	-22.4401603113027\\
63.5	0.16986	-21.2734403046604	-21.2734403046604\\
63.5	0.17352	-19.8496407644434	-19.8496407644434\\
63.5	0.17718	-18.1687616906516	-18.1687616906516\\
63.5	0.18084	-16.2308030832848	-16.2308030832848\\
63.5	0.1845	-14.0357649423433	-14.0357649423433\\
63.5	0.18816	-11.583647267827	-11.583647267827\\
63.5	0.19182	-8.87445005973592	-8.87445005973592\\
63.5	0.19548	-5.90817331807006	-5.90817331807006\\
63.5	0.19914	-2.68481704282931	-2.68481704282931\\
63.5	0.2028	0.795618765986262	0.795618765986262\\
63.5	0.20646	4.5331341083766	4.5331341083766\\
63.5	0.21012	8.52772898434176	8.52772898434176\\
63.5	0.21378	12.7794033938817	12.7794033938817\\
63.5	0.21744	17.2881573369967	17.2881573369967\\
63.5	0.2211	22.0539908136862	22.0539908136862\\
63.5	0.22476	27.0769038239506	27.0769038239506\\
63.5	0.22842	32.3568963677901	32.3568963677901\\
63.5	0.23208	37.893968445204	37.893968445204\\
63.5	0.23574	43.688120056193	43.688120056193\\
63.5	0.2394	49.7393512007567	49.7393512007567\\
63.5	0.24306	56.0476618788952	56.0476618788952\\
63.5	0.24672	62.6130520906083	62.6130520906083\\
63.5	0.25038	69.4355218358966	69.4355218358966\\
63.5	0.25404	76.5150711147594	76.5150711147594\\
63.5	0.2577	83.8516999271973	83.8516999271973\\
63.5	0.26136	91.4454082732098	91.4454082732098\\
63.5	0.26502	99.2961961527973	99.2961961527973\\
63.5	0.26868	107.404063565959	107.404063565959\\
63.5	0.27234	115.769010512696	115.769010512696\\
63.5	0.276	124.391036993008	124.391036993008\\
63.875	0.093	9.18934779864776	9.18934779864776\\
63.875	0.09666	5.13632614223876	5.13632614223876\\
63.875	0.10032	1.34038401940469	1.34038401940469\\
63.875	0.10398	-2.1984785698547	-2.1984785698547\\
63.875	0.10764	-5.48026162553926	-5.48026162553926\\
63.875	0.1113	-8.50496514764887	-8.50496514764887\\
63.875	0.11496	-11.2725891361839	-11.2725891361839\\
63.875	0.11862	-13.783133591144	-13.783133591144\\
63.875	0.12228	-16.0365985125291	-16.0365985125291\\
63.875	0.12594	-18.0329839003395	-18.0329839003395\\
63.875	0.1296	-19.7722897545754	-19.7722897545754\\
63.875	0.13326	-21.2545160752361	-21.2545160752361\\
63.875	0.13692	-22.4796628623221	-22.4796628623221\\
63.875	0.14058	-23.4477301158334	-23.4477301158334\\
63.875	0.14424	-24.1587178357697	-24.1587178357697\\
63.875	0.1479	-24.6126260221312	-24.6126260221312\\
63.875	0.15156	-24.809454674918	-24.809454674918\\
63.875	0.15522	-24.74920379413	-24.74920379413\\
63.875	0.15888	-24.4318733797671	-24.4318733797671\\
63.875	0.16254	-23.8574634318295	-23.8574634318295\\
63.875	0.1662	-23.0259739503169	-23.0259739503169\\
63.875	0.16986	-21.9374049352298	-21.9374049352298\\
63.875	0.17352	-20.5917563865676	-20.5917563865676\\
63.875	0.17718	-18.9890283043306	-18.9890283043306\\
63.875	0.18084	-17.1292206885191	-17.1292206885191\\
63.875	0.1845	-15.0123335391322	-15.0123335391322\\
63.875	0.18816	-12.6383668561709	-12.6383668561709\\
63.875	0.19182	-10.0073206396347	-10.0073206396347\\
63.875	0.19548	-7.11919488952384	-7.11919488952384\\
63.875	0.19914	-3.97398960583791	-3.97398960583791\\
63.875	0.2028	-0.571704788577392	-0.571704788577392\\
63.875	0.20646	3.08765956225812	3.08765956225812\\
63.875	0.21012	7.00410344666847	7.00410344666847\\
63.875	0.21378	11.1776268646533	11.1776268646533\\
63.875	0.21744	15.6082298162135	15.6082298162135\\
63.875	0.2211	20.295912301348	20.295912301348\\
63.875	0.22476	25.2406743200573	25.2406743200573\\
63.875	0.22842	30.442515872342	30.442515872342\\
63.875	0.23208	35.9014369582009	35.9014369582009\\
63.875	0.23574	41.6174375776351	41.6174375776351\\
63.875	0.2394	47.5905177306437	47.5905177306437\\
63.875	0.24306	53.8206774172273	53.8206774172273\\
63.875	0.24672	60.3079166373857	60.3079166373857\\
63.875	0.25038	67.0522353911191	67.0522353911191\\
63.875	0.25404	74.0536336784269	74.0536336784269\\
63.875	0.2577	81.3121114993097	81.3121114993097\\
63.875	0.26136	88.8276688537676	88.8276688537676\\
63.875	0.26502	96.6003057417998	96.6003057417998\\
63.875	0.26868	104.630022163407	104.630022163407\\
63.875	0.27234	112.916818118589	112.916818118589\\
63.875	0.276	121.460693607346	121.460693607346\\
64.25	0.093	10.2168384838069	10.2168384838069\\
64.25	0.09666	6.08566583584319	6.08566583584319\\
64.25	0.10032	2.21157272145408	2.21157272145408\\
64.25	0.10398	-1.40544085936025	-1.40544085936025\\
64.25	0.10764	-4.76537490659975	-4.76537490659975\\
64.25	0.1113	-7.8682294202643	-7.8682294202643\\
64.25	0.11496	-10.7140044003542	-10.7140044003542\\
64.25	0.11862	-13.3026998468691	-13.3026998468691\\
64.25	0.12228	-15.6343157598094	-15.6343157598094\\
64.25	0.12594	-17.7088521391747	-17.7088521391747\\
64.25	0.1296	-19.5263089849653	-19.5263089849653\\
64.25	0.13326	-21.086686297181	-21.086686297181\\
64.25	0.13692	-22.389984075822	-22.389984075822\\
64.25	0.14058	-23.4362023208882	-23.4362023208882\\
64.25	0.14424	-24.2253410323795	-24.2253410323795\\
64.25	0.1479	-24.7574002102959	-24.7574002102959\\
64.25	0.15156	-25.0323798546376	-25.0323798546376\\
64.25	0.15522	-25.0502799654045	-25.0502799654045\\
64.25	0.15888	-24.8111005425966	-24.8111005425966\\
64.25	0.16254	-24.3148415862139	-24.3148415862139\\
64.25	0.1662	-23.5615030962562	-23.5615030962562\\
64.25	0.16986	-22.5510850727238	-22.5510850727238\\
64.25	0.17352	-21.2835875156167	-21.2835875156167\\
64.25	0.17718	-19.7590104249347	-19.7590104249347\\
64.25	0.18084	-17.977353800678	-17.977353800678\\
64.25	0.1845	-15.9386176428462	-15.9386176428462\\
64.25	0.18816	-13.6428019514397	-13.6428019514397\\
64.25	0.19182	-11.0899067264585	-11.0899067264585\\
64.25	0.19548	-8.27993196790254	-8.27993196790254\\
64.25	0.19914	-5.21287767577167	-5.21287767577167\\
64.25	0.2028	-1.88874385006574	-1.88874385006574\\
64.25	0.20646	1.69246950921473	1.69246950921473\\
64.25	0.21012	5.53076240207002	5.53076240207002\\
64.25	0.21378	9.62613482850008	9.62613482850008\\
64.25	0.21744	13.978586788505	13.978586788505\\
64.25	0.2211	18.5881182820848	18.5881182820848\\
64.25	0.22476	23.4547293092394	23.4547293092394\\
64.25	0.22842	28.5784198699687	28.5784198699687\\
64.25	0.23208	33.959189964273	33.959189964273\\
64.25	0.23574	39.597039592152	39.597039592152\\
64.25	0.2394	45.491968753606	45.491968753606\\
64.25	0.24306	51.6439774486346	51.6439774486346\\
64.25	0.24672	58.0530656772379	58.0530656772379\\
64.25	0.25038	64.7192334394163	64.7192334394163\\
64.25	0.25404	71.6424807351692	71.6424807351692\\
64.25	0.2577	78.8228075644975	78.8228075644975\\
64.25	0.26136	86.2602139274001	86.2602139274001\\
64.25	0.26502	93.9546998238775	93.9546998238775\\
64.25	0.26868	101.90626525393	101.90626525393\\
64.25	0.27234	110.114910217557	110.114910217557\\
64.25	0.276	118.580634714759	118.580634714759\\
64.625	0.093	11.2946136620412	11.2946136620412\\
64.625	0.09666	7.08529002252247	7.08529002252247\\
64.625	0.10032	3.13304591657854	3.13304591657854\\
64.625	0.10398	-0.562118655790613	-0.562118655790613\\
64.625	0.10764	-4.00020369458515	-4.00020369458515\\
64.625	0.1113	-7.18120919980464	-7.18120919980464\\
64.625	0.11496	-10.1051351714493	-10.1051351714493\\
64.625	0.11862	-12.7719816095192	-12.7719816095192\\
64.625	0.12228	-15.1817485140145	-15.1817485140145\\
64.625	0.12594	-17.3344358849348	-17.3344358849348\\
64.625	0.1296	-19.2300437222802	-19.2300437222802\\
64.625	0.13326	-20.868572026051	-20.868572026051\\
64.625	0.13692	-22.2500207962468	-22.2500207962468\\
64.625	0.14058	-23.3743900328679	-23.3743900328679\\
64.625	0.14424	-24.2416797359141	-24.2416797359141\\
64.625	0.1479	-24.8518899053854	-24.8518899053854\\
64.625	0.15156	-25.2050205412821	-25.2050205412821\\
64.625	0.15522	-25.301071643604	-25.301071643604\\
64.625	0.15888	-25.1400432123509	-25.1400432123509\\
64.625	0.16254	-24.7219352475232	-24.7219352475232\\
64.625	0.1662	-24.0467477491205	-24.0467477491205\\
64.625	0.16986	-23.114480717143	-23.114480717143\\
64.625	0.17352	-21.9251341515907	-21.9251341515907\\
64.625	0.17718	-20.4787080524636	-20.4787080524636\\
64.625	0.18084	-18.7752024197619	-18.7752024197619\\
64.625	0.1845	-16.8146172534851	-16.8146172534851\\
64.625	0.18816	-14.5969525536335	-14.5969525536335\\
64.625	0.19182	-12.1222083202073	-12.1222083202073\\
64.625	0.19548	-9.39038455320616	-9.39038455320616\\
64.625	0.19914	-6.40148125263011	-6.40148125263011\\
64.625	0.2028	-3.15549841847945	-3.15549841847945\\
64.625	0.20646	0.347563949246194	0.347563949246194\\
64.625	0.21012	4.10770585054667	4.10770585054667\\
64.625	0.21378	8.1249272854219	8.1249272854219\\
64.625	0.21744	12.399228253872	12.399228253872\\
64.625	0.2211	16.9306087558966	16.9306087558966\\
64.625	0.22476	21.7190687914963	21.7190687914963\\
64.625	0.22842	26.7646083606708	26.7646083606708\\
64.625	0.23208	32.0672274634201	32.0672274634201\\
64.625	0.23574	37.6269260997442	37.6269260997442\\
64.625	0.2394	43.4437042696431	43.4437042696431\\
64.625	0.24306	49.5175619731169	49.5175619731169\\
64.625	0.24672	55.8484992101654	55.8484992101654\\
64.625	0.25038	62.4365159807887	62.4365159807887\\
64.625	0.25404	69.2816122849869	69.2816122849869\\
64.625	0.2577	76.3837881227598	76.3837881227598\\
64.625	0.26136	83.7430434941076	83.7430434941076\\
64.625	0.26502	91.3593783990302	91.3593783990302\\
64.625	0.26868	99.2327928375277	99.2327928375277\\
64.625	0.27234	107.3632868096	107.3632868096\\
64.625	0.276	115.750860315247	115.750860315247\\
65	0.093	12.4226733333503	12.4226733333503\\
65	0.09666	8.13519870227661	8.13519870227661\\
65	0.10032	4.10480360477763	4.10480360477763\\
65	0.10398	0.331488040853543	0.331488040853543\\
65	0.10764	-3.18474798949582	-3.18474798949582\\
65	0.1113	-6.44390448627024	-6.44390448627024\\
65	0.11496	-9.44598144946987	-9.44598144946987\\
65	0.11862	-12.1909788790948	-12.1909788790948\\
65	0.12228	-14.6788967751449	-14.6788967751449\\
65	0.12594	-16.90973513762	-16.90973513762\\
65	0.1296	-18.8834939665205	-18.8834939665205\\
65	0.13326	-20.6001732618462	-20.6001732618462\\
65	0.13692	-22.059773023597	-22.059773023597\\
65	0.14058	-23.2622932517729	-23.2622932517729\\
65	0.14424	-24.2077339463741	-24.2077339463741\\
65	0.1479	-24.8960951074005	-24.8960951074005\\
65	0.15156	-25.327376734852	-25.327376734852\\
65	0.15522	-25.5015788287287	-25.5015788287287\\
65	0.15888	-25.4187013890306	-25.4187013890306\\
65	0.16254	-25.0787444157578	-25.0787444157578\\
65	0.1662	-24.4817079089102	-24.4817079089102\\
65	0.16986	-23.6275918684875	-23.6275918684875\\
65	0.17352	-22.5163962944902	-22.5163962944902\\
65	0.17718	-21.1481211869181	-21.1481211869181\\
65	0.18084	-19.522766545771	-19.522766545771\\
65	0.1845	-17.6403323710493	-17.6403323710493\\
65	0.18816	-15.5008186627527	-15.5008186627527\\
65	0.19182	-13.1042254208814	-13.1042254208814\\
65	0.19548	-10.4505526454353	-10.4505526454353\\
65	0.19914	-7.53980033641403	-7.53980033641403\\
65	0.2028	-4.3719684938182	-4.3719684938182\\
65	0.20646	-0.947057117647603	-0.947057117647603\\
65	0.21012	2.73493379209805	2.73493379209805\\
65	0.21378	6.67400423541824	6.67400423541824\\
65	0.21744	10.8701542123133	10.8701542123133\\
65	0.2211	15.323383722783	15.323383722783\\
65	0.22476	20.0336927668277	20.0336927668277\\
65	0.22842	25.0010813444474	25.0010813444474\\
65	0.23208	30.2255494556418	30.2255494556418\\
65	0.23574	35.7070971004111	35.7070971004111\\
65	0.2394	41.4457242787551	41.4457242787551\\
65	0.24306	47.4414309906738	47.4414309906738\\
65	0.24672	53.6942172361672	53.6942172361672\\
65	0.25038	60.2040830152357	60.2040830152357\\
65	0.25404	66.971028327879	66.971028327879\\
65	0.2577	73.9950531740972	73.9950531740972\\
65	0.26136	81.2761575538899	81.2761575538899\\
65	0.26502	88.8143414672577	88.8143414672577\\
65	0.26868	96.6096049142001	96.6096049142001\\
65	0.27234	104.661947894717	104.661947894717\\
65	0.276	112.971370408809	112.971370408809\\
65.375	0.093	13.6010174977343	13.6010174977343\\
65.375	0.09666	9.23539187510572	9.23539187510572\\
65.375	0.10032	5.1268457860518	5.1268457860518\\
65.375	0.10398	1.27537923057278	1.27537923057278\\
65.375	0.10764	-2.31900779133152	-2.31900779133152\\
65.375	0.1113	-5.65631527966087	-5.65631527966087\\
65.375	0.11496	-8.73654323441544	-8.73654323441544\\
65.375	0.11862	-11.5596916555952	-11.5596916555952\\
65.375	0.12228	-14.1257605432002	-14.1257605432002\\
65.375	0.12594	-16.4347498972304	-16.4347498972304\\
65.375	0.1296	-18.4866597176858	-18.4866597176858\\
65.375	0.13326	-20.2814900045663	-20.2814900045663\\
65.375	0.13692	-21.819240757872	-21.819240757872\\
65.375	0.14058	-23.0999119776029	-23.0999119776029\\
65.375	0.14424	-24.1235036637591	-24.1235036637591\\
65.375	0.1479	-24.8900158163404	-24.8900158163404\\
65.375	0.15156	-25.3994484353468	-25.3994484353468\\
65.375	0.15522	-25.6518015207784	-25.6518015207784\\
65.375	0.15888	-25.6470750726354	-25.6470750726354\\
65.375	0.16254	-25.3852690909174	-25.3852690909174\\
65.375	0.1662	-24.8663835756246	-24.8663835756246\\
65.375	0.16986	-24.0904185267569	-24.0904185267569\\
65.375	0.17352	-23.0573739443145	-23.0573739443145\\
65.375	0.17718	-21.7672498282972	-21.7672498282972\\
65.375	0.18084	-20.2200461787052	-20.2200461787052\\
65.375	0.1845	-18.4157629955385	-18.4157629955385\\
65.375	0.18816	-16.3544002787967	-16.3544002787967\\
65.375	0.19182	-14.0359580284803	-14.0359580284803\\
65.375	0.19548	-11.4604362445892	-11.4604362445892\\
65.375	0.19914	-8.62783492712276	-8.62783492712276\\
65.375	0.2028	-5.53815407608198	-5.53815407608198\\
65.375	0.20646	-2.1913936914662	-2.1913936914662\\
65.375	0.21012	1.4124462267244	1.4124462267244\\
65.375	0.21378	5.27336567848954	5.27336567848954\\
65.375	0.21744	9.39136466382996	9.39136466382996\\
65.375	0.2211	13.7664431827447	13.7664431827447\\
65.375	0.22476	18.3986012352345	18.3986012352345\\
65.375	0.22842	23.2878388212992	23.2878388212992\\
65.375	0.23208	28.4341559409384	28.4341559409384\\
65.375	0.23574	33.8375525941531	33.8375525941531\\
65.375	0.2394	39.4980287809419	39.4980287809419\\
65.375	0.24306	45.4155845013056	45.4155845013056\\
65.375	0.24672	51.5902197552444	51.5902197552444\\
65.375	0.25038	58.0219345427579	58.0219345427579\\
65.375	0.25404	64.7107288638462	64.7107288638462\\
65.375	0.2577	71.6566027185095	71.6566027185095\\
65.375	0.26136	78.8595561067472	78.8595561067472\\
65.375	0.26502	86.3195890285601	86.3195890285601\\
65.375	0.26868	94.0367014839475	94.0367014839475\\
65.375	0.27234	102.01089347291	102.01089347291\\
65.375	0.276	110.242164995447	110.242164995447\\
65.75	0.093	14.8296461551934	14.8296461551934\\
65.75	0.09666	10.3858695410098	10.3858695410098\\
65.75	0.10032	6.19917246040094	6.19917246040094\\
65.75	0.10398	2.26955491336699	2.26955491336699\\
65.75	0.10764	-1.40298310009224	-1.40298310009224\\
65.75	0.1113	-4.81844157997654	-4.81844157997654\\
65.75	0.11496	-7.97682052628603	-7.97682052628603\\
65.75	0.11862	-10.8781199390208	-10.8781199390208\\
65.75	0.12228	-13.5223398181807	-13.5223398181807\\
65.75	0.12594	-15.9094801637658	-15.9094801637658\\
65.75	0.1296	-18.0395409757762	-18.0395409757762\\
65.75	0.13326	-19.9125222542116	-19.9125222542116\\
65.75	0.13692	-21.5284239990722	-21.5284239990722\\
65.75	0.14058	-22.8872462103581	-22.8872462103581\\
65.75	0.14424	-23.9889888880692	-23.9889888880692\\
65.75	0.1479	-24.8336520322053	-24.8336520322053\\
65.75	0.15156	-25.4212356427668	-25.4212356427668\\
65.75	0.15522	-25.7517397197532	-25.7517397197532\\
65.75	0.15888	-25.8251642631652	-25.8251642631652\\
65.75	0.16254	-25.641509273002	-25.641509273002\\
65.75	0.1662	-25.2007747492643	-25.2007747492643\\
65.75	0.16986	-24.5029606919517	-24.5029606919517\\
65.75	0.17352	-23.5480671010641	-23.5480671010641\\
65.75	0.17718	-22.3360939766016	-22.3360939766016\\
65.75	0.18084	-20.8670413185646	-20.8670413185646\\
65.75	0.1845	-19.1409091269527	-19.1409091269527\\
65.75	0.18816	-17.1576974017658	-17.1576974017658\\
65.75	0.19182	-14.9174061430044	-14.9174061430044\\
65.75	0.19548	-12.4200353506681	-12.4200353506681\\
65.75	0.19914	-9.66558502475675	-9.66558502475675\\
65.75	0.2028	-6.65405516527102	-6.65405516527102\\
65.75	0.20646	-3.38544577221006	-3.38544577221006\\
65.75	0.21012	0.140243154425491	0.140243154425491\\
65.75	0.21378	3.92301161463581	3.92301161463581\\
65.75	0.21744	7.96285960842118	7.96285960842118\\
65.75	0.2211	12.2597871357811	12.2597871357811\\
65.75	0.22476	16.8137941967159	16.8137941967159\\
65.75	0.22842	21.6248807912257	21.6248807912257\\
65.75	0.23208	26.6930469193103	26.6930469193103\\
65.75	0.23574	32.0182925809695	32.0182925809695\\
65.75	0.2394	37.6006177762038	37.6006177762038\\
65.75	0.24306	43.4400225050126	43.4400225050126\\
65.75	0.24672	49.5365067673962	49.5365067673962\\
65.75	0.25038	55.890070563355	55.890070563355\\
65.75	0.25404	62.5007138928883	62.5007138928883\\
65.75	0.2577	69.3684367559965	69.3684367559965\\
65.75	0.26136	76.4932391526796	76.4932391526796\\
65.75	0.26502	83.8751210829373	83.8751210829373\\
65.75	0.26868	91.5140825467698	91.5140825467698\\
65.75	0.27234	99.4101235441774	99.4101235441774\\
65.75	0.276	107.56324407516	107.56324407516\\
66.125	0.093	16.1085593057272	16.1085593057272\\
66.125	0.09666	11.5866316999887	11.5866316999887\\
66.125	0.10032	7.32178362782494	7.32178362782494\\
66.125	0.10398	3.31401508923605	3.31401508923605\\
66.125	0.10764	-0.436673915778115	-0.436673915778115\\
66.125	0.1113	-3.93028338721734	-3.93028338721734\\
66.125	0.11496	-7.16681332508178	-7.16681332508178\\
66.125	0.11862	-10.1462637293714	-10.1462637293714\\
66.125	0.12228	-12.8686346000862	-12.8686346000862\\
66.125	0.12594	-15.3339259372262	-15.3339259372262\\
66.125	0.1296	-17.5421377407915	-17.5421377407915\\
66.125	0.13326	-19.4932700107819	-19.4932700107819\\
66.125	0.13692	-21.1873227471975	-21.1873227471975\\
66.125	0.14058	-22.6242959500383	-22.6242959500383\\
66.125	0.14424	-23.8041896193042	-23.8041896193042\\
66.125	0.1479	-24.7270037549953	-24.7270037549953\\
66.125	0.15156	-25.3927383571116	-25.3927383571116\\
66.125	0.15522	-25.8013934256531	-25.8013934256531\\
66.125	0.15888	-25.9529689606197	-25.9529689606197\\
66.125	0.16254	-25.8474649620118	-25.8474649620118\\
66.125	0.1662	-25.4848814298289	-25.4848814298289\\
66.125	0.16986	-24.8652183640711	-24.8652183640711\\
66.125	0.17352	-23.9884757647386	-23.9884757647386\\
66.125	0.17718	-22.8546536318311	-22.8546536318311\\
66.125	0.18084	-21.4637519653492	-21.4637519653492\\
66.125	0.1845	-19.8157707652921	-19.8157707652921\\
66.125	0.18816	-17.9107100316602	-17.9107100316602\\
66.125	0.19182	-15.7485697644536	-15.7485697644536\\
66.125	0.19548	-13.3293499636722	-13.3293499636722\\
66.125	0.19914	-10.6530506293159	-10.6530506293159\\
66.125	0.2028	-7.71967176138475	-7.71967176138475\\
66.125	0.20646	-4.52921335987907	-4.52921335987907\\
66.125	0.21012	-1.08167542479833	-1.08167542479833\\
66.125	0.21378	2.62294204385717	2.62294204385717\\
66.125	0.21744	6.58463904608749	6.58463904608749\\
66.125	0.2211	10.8034155818926	10.8034155818926\\
66.125	0.22476	15.2792716512723	15.2792716512723\\
66.125	0.22842	20.0122072542271	20.0122072542271\\
66.125	0.23208	25.0022223907566	25.0022223907566\\
66.125	0.23574	30.2493170608612	30.2493170608612\\
66.125	0.2394	35.7534912645405	35.7534912645405\\
66.125	0.24306	41.5147450017945	41.5147450017945\\
66.125	0.24672	47.5330782726232	47.5330782726232\\
66.125	0.25038	53.808491077027	53.808491077027\\
66.125	0.25404	60.3409834150052	60.3409834150052\\
66.125	0.2577	67.1305552865587	67.1305552865587\\
66.125	0.26136	74.1772066916867	74.1772066916867\\
66.125	0.26502	81.4809376303896	81.4809376303896\\
66.125	0.26868	89.0417481026673	89.0417481026673\\
66.125	0.27234	96.8596381085198	96.8596381085198\\
66.125	0.276	104.934607647947	104.934607647947\\
66.5	0.093	17.4377569493359	17.4377569493359\\
66.5	0.09666	12.8376783520424	12.8376783520424\\
66.5	0.10032	8.49467928832368	8.49467928832368\\
66.5	0.10398	4.40875975817997	4.40875975817997\\
66.5	0.10764	0.579919761610867	0.579919761610867\\
66.5	0.1113	-2.9918407013833	-2.9918407013833\\
66.5	0.11496	-6.30652163080266	-6.30652163080266\\
66.5	0.11862	-9.36412302664723	-9.36412302664723\\
66.5	0.12228	-12.164644888917	-12.164644888917\\
66.5	0.12594	-14.708087217612	-14.708087217612\\
66.5	0.1296	-16.994450012732	-16.994450012732\\
66.5	0.13326	-19.0237332742775	-19.0237332742775\\
66.5	0.13692	-20.795937002248	-20.795937002248\\
66.5	0.14058	-22.3110611966437	-22.3110611966437\\
66.5	0.14424	-23.5691058574646	-23.5691058574646\\
66.5	0.1479	-24.5700709847105	-24.5700709847105\\
66.5	0.15156	-25.3139565783819	-25.3139565783819\\
66.5	0.15522	-25.8007626384782	-25.8007626384782\\
66.5	0.15888	-26.0304891649998	-26.0304891649998\\
66.5	0.16254	-26.0031361579467	-26.0031361579467\\
66.5	0.1662	-25.7187036173189	-25.7187036173189\\
66.5	0.16986	-25.1771915431161	-25.1771915431161\\
66.5	0.17352	-24.3785999353384	-24.3785999353384\\
66.5	0.17718	-23.3229287939858	-23.3229287939858\\
66.5	0.18084	-22.0101781190587	-22.0101781190587\\
66.5	0.1845	-20.4403479105567	-20.4403479105567\\
66.5	0.18816	-18.6134381684798	-18.6134381684798\\
66.5	0.19182	-16.529448892828	-16.529448892828\\
66.5	0.19548	-14.1883800836017	-14.1883800836017\\
66.5	0.19914	-11.5902317408002	-11.5902317408002\\
66.5	0.2028	-8.73500386442407	-8.73500386442407\\
66.5	0.20646	-5.62269645447321	-5.62269645447321\\
66.5	0.21012	-2.25330951094753	-2.25330951094753\\
66.5	0.21378	1.37315696615315	1.37315696615315\\
66.5	0.21744	5.25670297682865	5.25670297682865\\
66.5	0.2211	9.39732852107869	9.39732852107869\\
66.5	0.22476	13.7950335989036	13.7950335989036\\
66.5	0.22842	18.4498182103036	18.4498182103036\\
66.5	0.23208	23.3616823552783	23.3616823552783\\
66.5	0.23574	28.5306260338278	28.5306260338278\\
66.5	0.2394	33.956649245952	33.956649245952\\
66.5	0.24306	39.639751991651	39.639751991651\\
66.5	0.24672	45.5799342709247	45.5799342709247\\
66.5	0.25038	51.7771960837737	51.7771960837737\\
66.5	0.25404	58.231537430197	58.231537430197\\
66.5	0.2577	64.9429583101956	64.9429583101956\\
66.5	0.26136	71.9114587237686	71.9114587237686\\
66.5	0.26502	79.1370386709165	79.1370386709165\\
66.5	0.26868	86.6196981516393	86.6196981516393\\
66.5	0.27234	94.359437165937	94.359437165937\\
66.5	0.276	102.356255713809	102.356255713809\\
66.875	0.093	18.8172390860196	18.8172390860196\\
66.875	0.09666	14.1390094971713	14.1390094971713\\
66.875	0.10032	9.71785944189762	9.71785944189762\\
66.875	0.10398	5.55378892019885	5.55378892019885\\
66.875	0.10764	1.64679793207482	1.64679793207482\\
66.875	0.1113	-2.00311352247428	-2.00311352247428\\
66.875	0.11496	-5.39594544344858	-5.39594544344858\\
66.875	0.11862	-8.53169783084809	-8.53169783084809\\
66.875	0.12228	-11.4103706846728	-11.4103706846728\\
66.875	0.12594	-14.0319640049226	-14.0319640049226\\
66.875	0.1296	-16.3964777915977	-16.3964777915977\\
66.875	0.13326	-18.503912044698	-18.503912044698\\
66.875	0.13692	-20.3542667642234	-20.3542667642234\\
66.875	0.14058	-21.947541950174	-21.947541950174\\
66.875	0.14424	-23.28373760255	-23.28373760255\\
66.875	0.1479	-24.3628537213509	-24.3628537213509\\
66.875	0.15156	-25.1848903065771	-25.1848903065771\\
66.875	0.15522	-25.7498473582283	-25.7498473582283\\
66.875	0.15888	-26.0577248763049	-26.0577248763049\\
66.875	0.16254	-26.1085228608067	-26.1085228608067\\
66.875	0.1662	-25.9022413117336	-25.9022413117336\\
66.875	0.16986	-25.4388802290857	-25.4388802290857\\
66.875	0.17352	-24.7184396128631	-24.7184396128631\\
66.875	0.17718	-23.7409194630655	-23.7409194630655\\
66.875	0.18084	-22.5063197796932	-22.5063197796932\\
66.875	0.1845	-21.014640562746	-21.014640562746\\
66.875	0.18816	-19.265881812224	-19.265881812224\\
66.875	0.19182	-17.2600435281275	-17.2600435281275\\
66.875	0.19548	-14.9971257104559	-14.9971257104559\\
66.875	0.19914	-12.4771283592095	-12.4771283592095\\
66.875	0.2028	-9.7000514743882	-9.7000514743882\\
66.875	0.20646	-6.66589505599217	-6.66589505599217\\
66.875	0.21012	-3.37465910402153	-3.37465910402153\\
66.875	0.21378	0.173656381524324	0.173656381524324\\
66.875	0.21744	3.97905140064455	3.97905140064455\\
66.875	0.2211	8.04152595334	8.04152595334\\
66.875	0.22476	12.3610800396099	12.3610800396099\\
66.875	0.22842	16.937713659455	16.937713659455\\
66.875	0.23208	21.7714268128744	21.7714268128744\\
66.875	0.23574	26.8622194998694	26.8622194998694\\
66.875	0.2394	32.2100917204385	32.2100917204385\\
66.875	0.24306	37.8150434745827	37.8150434745827\\
66.875	0.24672	43.6770747623016	43.6770747623016\\
66.875	0.25038	49.7961855835953	49.7961855835953\\
66.875	0.25404	56.172375938464	56.172375938464\\
66.875	0.2577	62.8056458269074	62.8056458269074\\
66.875	0.26136	69.6959952489258	69.6959952489258\\
66.875	0.26502	76.8434242045186	76.8434242045186\\
66.875	0.26868	84.2479326936864	84.2479326936864\\
66.875	0.27234	91.909520716429	91.909520716429\\
66.875	0.276	99.8281882727465	99.8281882727465\\
67.25	0.093	20.2470057157783	20.2470057157783\\
67.25	0.09666	15.4906251353749	15.4906251353749\\
67.25	0.10032	10.9913240885464	10.9913240885464\\
67.25	0.10398	6.74910257529271	6.74910257529271\\
67.25	0.10764	2.76396059561385	2.76396059561385\\
67.25	0.1113	-0.964101850490181	-0.964101850490181\\
67.25	0.11496	-4.43508476301953	-4.43508476301953\\
67.25	0.11862	-7.64898814197386	-7.64898814197386\\
67.25	0.12228	-10.6058119873536	-10.6058119873536\\
67.25	0.12594	-13.3055562991583	-13.3055562991583\\
67.25	0.1296	-15.7482210773884	-15.7482210773884\\
67.25	0.13326	-17.9338063220435	-17.9338063220435\\
67.25	0.13692	-19.8623120331238	-19.8623120331238\\
67.25	0.14058	-21.5337382106294	-21.5337382106294\\
67.25	0.14424	-22.9480848545602	-22.9480848545602\\
67.25	0.1479	-24.1053519649162	-24.1053519649162\\
67.25	0.15156	-25.0055395416972	-25.0055395416972\\
67.25	0.15522	-25.6486475849034	-25.6486475849034\\
67.25	0.15888	-26.0346760945351	-26.0346760945351\\
67.25	0.16254	-26.1636250705917	-26.1636250705917\\
67.25	0.1662	-26.0354945130734	-26.0354945130734\\
67.25	0.16986	-25.6502844219806	-25.6502844219806\\
67.25	0.17352	-25.007994797313	-25.007994797313\\
67.25	0.17718	-24.1086256390702	-24.1086256390702\\
67.25	0.18084	-22.9521769472528	-22.9521769472528\\
67.25	0.1845	-21.5386487218606	-21.5386487218606\\
67.25	0.18816	-19.8680409628936	-19.8680409628936\\
67.25	0.19182	-17.9403536703517	-17.9403536703517\\
67.25	0.19548	-15.7555868442352	-15.7555868442352\\
67.25	0.19914	-13.3137404845438	-13.3137404845438\\
67.25	0.2028	-10.6148145912774	-10.6148145912774\\
67.25	0.20646	-7.65880916443638	-7.65880916443638\\
67.25	0.21012	-4.44572420402056	-4.44572420402056\\
67.25	0.21378	-0.975559710029756	-0.975559710029756\\
67.25	0.21744	2.75168431753588	2.75168431753588\\
67.25	0.2211	6.73600787867605	6.73600787867605\\
67.25	0.22476	10.9774109733911	10.9774109733911\\
67.25	0.22842	15.4758936016812	15.4758936016812\\
67.25	0.23208	20.231455763546	20.231455763546\\
67.25	0.23574	25.2440974589857	25.2440974589857\\
67.25	0.2394	30.513818688	30.513818688\\
67.25	0.24306	36.0406194505894	36.0406194505894\\
67.25	0.24672	41.8244997467534	41.8244997467534\\
67.25	0.25038	47.8654595764921	47.8654595764921\\
67.25	0.25404	54.1634989398058	54.1634989398058\\
67.25	0.2577	60.7186178366945	60.7186178366945\\
67.25	0.26136	67.5308162671577	67.5308162671577\\
67.25	0.26502	74.6000942311956	74.6000942311956\\
67.25	0.26868	81.9264517288086	81.9264517288086\\
67.25	0.27234	89.5098887599964	89.5098887599964\\
67.25	0.276	97.3504053247591	97.3504053247591\\
67.625	0.093	21.7270568386117	21.7270568386117\\
67.625	0.09666	16.8925252666535	16.8925252666535\\
67.625	0.10032	12.3150732282701	12.3150732282701\\
67.625	0.10398	7.99470072346131	7.99470072346131\\
67.625	0.10764	3.93140775222751	3.93140775222751\\
67.625	0.1113	0.125194314568546	0.125194314568546\\
67.625	0.11496	-3.42393958951563	-3.42393958951563\\
67.625	0.11862	-6.715993960025	-6.715993960025\\
67.625	0.12228	-9.75096879695954	-9.75096879695954\\
67.625	0.12594	-12.5288641003192	-12.5288641003192\\
67.625	0.1296	-15.0496798701042	-15.0496798701042\\
67.625	0.13326	-17.3134161063143	-17.3134161063143\\
67.625	0.13692	-19.3200728089495	-19.3200728089495\\
67.625	0.14058	-21.0696499780101	-21.0696499780101\\
67.625	0.14424	-22.5621476134957	-22.5621476134957\\
67.625	0.1479	-23.7975657154066	-23.7975657154066\\
67.625	0.15156	-24.7759042837426	-24.7759042837426\\
67.625	0.15522	-25.4971633185039	-25.4971633185039\\
67.625	0.15888	-25.9613428196902	-25.9613428196902\\
67.625	0.16254	-26.1684427873018	-26.1684427873018\\
67.625	0.1662	-26.1184632213389	-26.1184632213389\\
67.625	0.16986	-25.8114041218006	-25.8114041218006\\
67.625	0.17352	-25.2472654886878	-25.2472654886878\\
67.625	0.17718	-24.4260473220001	-24.4260473220001\\
67.625	0.18084	-23.3477496217377	-23.3477496217377\\
67.625	0.1845	-22.0123723879004	-22.0123723879004\\
67.625	0.18816	-20.4199156204884	-20.4199156204884\\
67.625	0.19182	-18.5703793195013	-18.5703793195013\\
67.625	0.19548	-16.4637634849399	-16.4637634849399\\
67.625	0.19914	-14.1000681168033	-14.1000681168033\\
67.625	0.2028	-11.4792932150919	-11.4792932150919\\
67.625	0.20646	-8.60143877980573	-8.60143877980573\\
67.625	0.21012	-5.46650481094474	-5.46650481094474\\
67.625	0.21378	-2.07449130850898	-2.07449130850898\\
67.625	0.21744	1.5746017275016	1.5746017275016\\
67.625	0.2211	5.48077429708695	5.48077429708695\\
67.625	0.22476	9.64402640024718	9.64402640024718\\
67.625	0.22842	14.0643580369822	14.0643580369822\\
67.625	0.23208	18.741769207292	18.741769207292\\
67.625	0.23574	23.6762599111769	23.6762599111769\\
67.625	0.2394	28.8678301486364	28.8678301486364\\
67.625	0.24306	34.3164799196707	34.3164799196707\\
67.625	0.24672	40.0222092242797	40.0222092242797\\
67.625	0.25038	45.9850180624637	45.9850180624637\\
67.625	0.25404	52.2049064342224	52.2049064342224\\
67.625	0.2577	58.6818743395561	58.6818743395561\\
67.625	0.26136	65.4159217784644	65.4159217784644\\
67.625	0.26502	72.4070487509475	72.4070487509475\\
67.625	0.26868	79.6552552570055	79.6552552570055\\
67.625	0.27234	87.1605412966383	87.1605412966383\\
67.625	0.276	94.9229068698459	94.9229068698459\\
68	0.093	23.2573924545201	23.2573924545201\\
68	0.09666	18.3447098910069	18.3447098910069\\
68	0.10032	13.6891068610686	13.6891068610686\\
68	0.10398	9.29058336470499	9.29058336470499\\
68	0.10764	5.14913940191626	5.14913940191626\\
68	0.1113	1.26477497270236	1.26477497270236\\
68	0.11496	-2.36250992293687	-2.36250992293687\\
68	0.11862	-5.73271528500106	-5.73271528500106\\
68	0.12228	-8.84584111349054	-8.84584111349054\\
68	0.12594	-11.7018874084052	-11.7018874084052\\
68	0.1296	-14.3008541697451	-14.3008541697451\\
68	0.13326	-16.6427413975102	-16.6427413975102\\
68	0.13692	-18.7275490917005	-18.7275490917005\\
68	0.14058	-20.5552772523157	-20.5552772523157\\
68	0.14424	-22.1259258793565	-22.1259258793565\\
68	0.1479	-23.4394949728221	-23.4394949728221\\
68	0.15156	-24.4959845327132	-24.4959845327132\\
68	0.15522	-25.2953945590293	-25.2953945590293\\
68	0.15888	-25.8377250517704	-25.8377250517704\\
68	0.16254	-26.1229760109371	-26.1229760109371\\
68	0.1662	-26.151147436529	-26.151147436529\\
68	0.16986	-25.9222393285457	-25.9222393285457\\
68	0.17352	-25.436251686988	-25.436251686988\\
68	0.17718	-24.6931845118551	-24.6931845118551\\
68	0.18084	-23.6930378031478	-23.6930378031478\\
68	0.1845	-22.435811560865	-22.435811560865\\
68	0.18816	-20.9215057850079	-20.9215057850079\\
68	0.19182	-19.1501204755761	-19.1501204755761\\
68	0.19548	-17.1216556325695	-17.1216556325695\\
68	0.19914	-14.8361112559877	-14.8361112559877\\
68	0.2028	-12.2934873458311	-12.2934873458311\\
68	0.20646	-9.49378390210001	-9.49378390210001\\
68	0.21012	-6.43700092479406	-6.43700092479406\\
68	0.21378	-3.12313841391313	-3.12313841391313\\
68	0.21744	0.447803630542637	0.447803630542637\\
68	0.2211	4.27582520857294	4.27582520857294\\
68	0.22476	8.36092632017812	8.36092632017812\\
68	0.22842	12.7031069653584	12.7031069653584\\
68	0.23208	17.3023671441133	17.3023671441133\\
68	0.23574	22.1587068564431	22.1587068564431\\
68	0.2394	27.2721261023476	27.2721261023476\\
68	0.24306	32.642624881827	32.642624881827\\
68	0.24672	38.2702031948812	38.2702031948812\\
68	0.25038	44.1548610415102	44.1548610415102\\
68	0.25404	50.2965984217138	50.2965984217138\\
68	0.2577	56.6954153354925	56.6954153354925\\
68	0.26136	63.3513117828462	63.3513117828462\\
68	0.26502	70.2642877637743	70.2642877637743\\
68	0.26868	77.4343432782774	77.4343432782774\\
68	0.27234	84.8614783263549	84.8614783263549\\
68	0.276	92.5456929080079	92.5456929080079\\
68.375	0.093	24.8380125635033	24.8380125635033\\
68.375	0.09666	19.8471790084353	19.8471790084353\\
68.375	0.10032	15.113424986942	15.113424986942\\
68.375	0.10398	10.6367504990234	10.6367504990234\\
68.375	0.10764	6.41715554467974	6.41715554467974\\
68.375	0.1113	2.45464012391091	2.45464012391091\\
68.375	0.11496	-1.25079576328302	-1.25079576328302\\
68.375	0.11862	-4.69915211690238	-4.69915211690238\\
68.375	0.12228	-7.89042893694668	-7.89042893694668\\
68.375	0.12594	-10.8246262234162	-10.8246262234162\\
68.375	0.1296	-13.501743976311	-13.501743976311\\
68.375	0.13326	-15.9217821956311	-15.9217821956311\\
68.375	0.13692	-18.0847408813761	-18.0847408813761\\
68.375	0.14058	-19.9906200335467	-19.9906200335467\\
68.375	0.14424	-21.639419652142	-21.639419652142\\
68.375	0.1479	-23.0311397371627	-23.0311397371627\\
68.375	0.15156	-24.1657802886086	-24.1657802886086\\
68.375	0.15522	-25.0433413064797	-25.0433413064797\\
68.375	0.15888	-25.6638227907759	-25.6638227907759\\
68.375	0.16254	-26.0272247414974	-26.0272247414974\\
68.375	0.1662	-26.1335471586441	-26.1335471586441\\
68.375	0.16986	-25.9827900422159	-25.9827900422159\\
68.375	0.17352	-25.574953392213	-25.574953392213\\
68.375	0.17718	-24.9100372086352	-24.9100372086352\\
68.375	0.18084	-23.9880414914826	-23.9880414914826\\
68.375	0.1845	-22.8089662407552	-22.8089662407552\\
68.375	0.18816	-21.3728114564529	-21.3728114564529\\
68.375	0.19182	-19.6795771385759	-19.6795771385759\\
68.375	0.19548	-17.7292632871241	-17.7292632871241\\
68.375	0.19914	-15.5218699020974	-15.5218699020974\\
68.375	0.2028	-13.0573969834958	-13.0573969834958\\
68.375	0.20646	-10.3358445313195	-10.3358445313195\\
68.375	0.21012	-7.35721254556842	-7.35721254556842\\
68.375	0.21378	-4.12150102624253	-4.12150102624253\\
68.375	0.21744	-0.628709973341813	-0.628709973341813\\
68.375	0.2211	3.12116061313367	3.12116061313367\\
68.375	0.22476	7.12811073318403	7.12811073318403\\
68.375	0.22842	11.3921403868094	11.3921403868094\\
68.375	0.23208	15.9132495740091	15.9132495740091\\
68.375	0.23574	20.6914382947843	20.6914382947843\\
68.375	0.2394	25.7267065491337	25.7267065491337\\
68.375	0.24306	31.0190543370582	31.0190543370582\\
68.375	0.24672	36.5684816585573	36.5684816585573\\
68.375	0.25038	42.3749885136315	42.3749885136315\\
68.375	0.25404	48.4385749022803	48.4385749022803\\
68.375	0.2577	54.7592408245041	54.7592408245041\\
68.375	0.26136	61.3369862803025	61.3369862803025\\
68.375	0.26502	68.171811269676	68.171811269676\\
68.375	0.26868	75.2637157926239	75.2637157926239\\
68.375	0.27234	82.6126998491468	82.6126998491468\\
68.375	0.276	90.2187634392448	90.2187634392448\\
68.75	0.093	26.4689171655615	26.4689171655615\\
68.75	0.09666	21.3999326189386	21.3999326189386\\
68.75	0.10032	16.5880276058903	16.5880276058903\\
68.75	0.10398	12.0332021264169	12.0332021264169\\
68.75	0.10764	7.73545618051831	7.73545618051831\\
68.75	0.1113	3.69478976819454	3.69478976819454\\
68.75	0.11496	-0.0887971105543244	-0.0887971105543244\\
68.75	0.11862	-3.61530445572862	-3.61530445572862\\
68.75	0.12228	-6.88473226732785	-6.88473226732785\\
68.75	0.12594	-9.8970805453524	-9.8970805453524\\
68.75	0.1296	-12.6523492898021	-12.6523492898021\\
68.75	0.13326	-15.150538500677	-15.150538500677\\
68.75	0.13692	-17.3916481779772	-17.3916481779772\\
68.75	0.14058	-19.3756783217025	-19.3756783217025\\
68.75	0.14424	-21.1026289318529	-21.1026289318529\\
68.75	0.1479	-22.5725000084284	-22.5725000084284\\
68.75	0.15156	-23.7852915514294	-23.7852915514294\\
68.75	0.15522	-24.7410035608553	-24.7410035608553\\
68.75	0.15888	-25.4396360367066	-25.4396360367066\\
68.75	0.16254	-25.8811889789829	-25.8811889789829\\
68.75	0.1662	-26.0656623876846	-26.0656623876846\\
68.75	0.16986	-25.9930562628113	-25.9930562628113\\
68.75	0.17352	-25.6633706043634	-25.6633706043634\\
68.75	0.17718	-25.0766054123404	-25.0766054123404\\
68.75	0.18084	-24.2327606867427	-24.2327606867427\\
68.75	0.1845	-23.13183642757	-23.13183642757\\
68.75	0.18816	-21.773832634823	-21.773832634823\\
68.75	0.19182	-20.1587493085008	-20.1587493085008\\
68.75	0.19548	-18.2865864486038	-18.2865864486038\\
68.75	0.19914	-16.1573440551319	-16.1573440551319\\
68.75	0.2028	-13.7710221280855	-13.7710221280855\\
68.75	0.20646	-11.1276206674642	-11.1276206674642\\
68.75	0.21012	-8.22713967326791	-8.22713967326791\\
68.75	0.21378	-5.06957914549685	-5.06957914549685\\
68.75	0.21744	-1.65493908415095	-1.65493908415095\\
68.75	0.2211	2.01678051076948	2.01678051076948\\
68.75	0.22476	5.94557963926479	5.94557963926479\\
68.75	0.22842	10.1314583013352	10.1314583013352\\
68.75	0.23208	14.5744164969802	14.5744164969802\\
68.75	0.23574	19.2744542262002	19.2744542262002\\
68.75	0.2394	24.2315714889947	24.2315714889947\\
68.75	0.24306	29.4457682853644	29.4457682853644\\
68.75	0.24672	34.9170446153087	34.9170446153087\\
68.75	0.25038	40.6454004788278	40.6454004788278\\
68.75	0.25404	46.6308358759218	46.6308358759218\\
68.75	0.2577	52.8733508065906	52.8733508065906\\
68.75	0.26136	59.3729452708342	59.3729452708342\\
68.75	0.26502	66.1296192686526	66.1296192686526\\
68.75	0.26868	73.1433728000457	73.1433728000457\\
68.75	0.27234	80.4142058650135	80.4142058650135\\
68.75	0.276	87.9421184635564	87.9421184635564\\
69.125	0.093	28.1501062606947	28.1501062606947\\
69.125	0.09666	23.0029707225167	23.0029707225167\\
69.125	0.10032	18.1129147179135	18.1129147179135\\
69.125	0.10398	13.4799382468852	13.4799382468852\\
69.125	0.10764	9.10404130943162	9.10404130943162\\
69.125	0.1113	4.98522390555291	4.98522390555291\\
69.125	0.11496	1.12348603524923	1.12348603524923\\
69.125	0.11862	-2.48117230148	-2.48117230148\\
69.125	0.12228	-5.82875110463416	-5.82875110463416\\
69.125	0.12594	-8.91925037421365	-8.91925037421365\\
69.125	0.1296	-11.7526701102183	-11.7526701102183\\
69.125	0.13326	-14.3290103126482	-14.3290103126482\\
69.125	0.13692	-16.6482709815032	-16.6482709815032\\
69.125	0.14058	-18.7104521167834	-18.7104521167834\\
69.125	0.14424	-20.5155537184888	-20.5155537184888\\
69.125	0.1479	-22.0635757866194	-22.0635757866194\\
69.125	0.15156	-23.3545183211752	-23.3545183211752\\
69.125	0.15522	-24.3883813221562	-24.3883813221562\\
69.125	0.15888	-25.1651647895622	-25.1651647895622\\
69.125	0.16254	-25.6848687233936	-25.6848687233936\\
69.125	0.1662	-25.9474931236501	-25.9474931236501\\
69.125	0.16986	-25.9530379903316	-25.9530379903316\\
69.125	0.17352	-25.7015033234385	-25.7015033234385\\
69.125	0.17718	-25.1928891229706	-25.1928891229706\\
69.125	0.18084	-24.4271953889279	-24.4271953889279\\
69.125	0.1845	-23.4044221213101	-23.4044221213101\\
69.125	0.18816	-22.1245693201179	-22.1245693201179\\
69.125	0.19182	-20.5876369853508	-20.5876369853508\\
69.125	0.19548	-18.7936251170088	-18.7936251170088\\
69.125	0.19914	-16.7425337150918	-16.7425337150918\\
69.125	0.2028	-14.4343627796001	-14.4343627796001\\
69.125	0.20646	-11.8691123105339	-11.8691123105339\\
69.125	0.21012	-9.04678230789244	-9.04678230789244\\
69.125	0.21378	-5.96737277167642	-5.96737277167642\\
69.125	0.21744	-2.63088370188558	-2.63088370188558\\
69.125	0.2211	0.962684901480031	0.962684901480031\\
69.125	0.22476	4.81333303842052	4.81333303842052\\
69.125	0.22842	8.92106070893584	8.92106070893584\\
69.125	0.23208	13.2858679130259	13.2858679130259\\
69.125	0.23574	17.907754650691	17.907754650691\\
69.125	0.2394	22.7867209219307	22.7867209219307\\
69.125	0.24306	27.9227667267453	27.9227667267453\\
69.125	0.24672	33.3158920651346	33.3158920651346\\
69.125	0.25038	38.9660969370989	38.9660969370989\\
69.125	0.25404	44.8733813426378	44.8733813426378\\
69.125	0.2577	51.0377452817518	51.0377452817518\\
69.125	0.26136	57.4591887544403	57.4591887544403\\
69.125	0.26502	64.1377117607037	64.1377117607037\\
69.125	0.26868	71.0733143005422	71.0733143005422\\
69.125	0.27234	78.2659963739552	78.2659963739552\\
69.125	0.276	85.7157579809431	85.7157579809431\\
69.5	0.093	29.8815798489028	29.8815798489028\\
69.5	0.09666	24.6562933191699	24.6562933191699\\
69.5	0.10032	19.6880863230119	19.6880863230119\\
69.5	0.10398	14.9769588604285	14.9769588604285\\
69.5	0.10764	10.5229109314201	10.5229109314201\\
69.5	0.1113	6.32594253598648	6.32594253598648\\
69.5	0.11496	2.38605367412764	2.38605367412764\\
69.5	0.11862	-1.29675565415641	-1.29675565415641\\
69.5	0.12228	-4.72248544886551	-4.72248544886551\\
69.5	0.12594	-7.89113570999993	-7.89113570999993\\
69.5	0.1296	-10.8027064375594	-10.8027064375594\\
69.5	0.13326	-13.4571976315441	-13.4571976315441\\
69.5	0.13692	-15.8546092919542	-15.8546092919542\\
69.5	0.14058	-17.9949414187894	-17.9949414187894\\
69.5	0.14424	-19.8781940120494	-19.8781940120494\\
69.5	0.1479	-21.5043670717351	-21.5043670717351\\
69.5	0.15156	-22.8734605978459	-22.8734605978459\\
69.5	0.15522	-23.9854745903817	-23.9854745903817\\
69.5	0.15888	-24.8404090493426	-24.8404090493426\\
69.5	0.16254	-25.438263974729	-25.438263974729\\
69.5	0.1662	-25.7790393665406	-25.7790393665406\\
69.5	0.16986	-25.8627352247769	-25.8627352247769\\
69.5	0.17352	-25.6893515494389	-25.6893515494389\\
69.5	0.17718	-25.2588883405257	-25.2588883405257\\
69.5	0.18084	-24.5713455980379	-24.5713455980379\\
69.5	0.1845	-23.6267233219754	-23.6267233219754\\
69.5	0.18816	-22.425021512338	-22.425021512338\\
69.5	0.19182	-20.9662401691257	-20.9662401691257\\
69.5	0.19548	-19.2503792923386	-19.2503792923386\\
69.5	0.19914	-17.2774388819768	-17.2774388819768\\
69.5	0.2028	-15.0474189380402	-15.0474189380402\\
69.5	0.20646	-12.5603194605285	-12.5603194605285\\
69.5	0.21012	-9.81614044944212	-9.81614044944212\\
69.5	0.21378	-6.81488190478115	-6.81488190478115\\
69.5	0.21744	-3.5565438265449	-3.5565438265449\\
69.5	0.2211	-0.0411262147343336	-0.0411262147343336\\
69.5	0.22476	3.73137093065111	3.73137093065111\\
69.5	0.22842	7.76094760961183	7.76094760961183\\
69.5	0.23208	12.0476038221468	12.0476038221468\\
69.5	0.23574	16.5913395682569	16.5913395682569\\
69.5	0.2394	21.3921548479418	21.3921548479418\\
69.5	0.24306	26.4500496612015	26.4500496612015\\
69.5	0.24672	31.7650240080358	31.7650240080358\\
69.5	0.25038	37.3370778884453	37.3370778884453\\
69.5	0.25404	43.1662113024291	43.1662113024291\\
69.5	0.2577	49.2524242499883	49.2524242499883\\
69.5	0.26136	55.5957167311218	55.5957167311218\\
69.5	0.26502	62.1960887458306	62.1960887458306\\
69.5	0.26868	69.0535402941138	69.0535402941138\\
69.5	0.27234	76.168071375972	76.168071375972\\
69.5	0.276	83.5396819914048	83.5396819914048\\
69.875	0.093	31.6633379301857	31.6633379301857\\
69.875	0.09666	26.3599004088978	26.3599004088978\\
69.875	0.10032	21.3135424211848	21.3135424211848\\
69.875	0.10398	16.5242639670466	16.5242639670466\\
69.875	0.10764	11.9920650464833	11.9920650464833\\
69.875	0.1113	7.7169456594948	7.7169456594948\\
69.875	0.11496	3.69890580608102	3.69890580608102\\
69.875	0.11862	-0.0620545137579711	-0.0620545137579711\\
69.875	0.12228	-3.56593530002212	-3.56593530002212\\
69.875	0.12594	-6.81273655271124	-6.81273655271124\\
69.875	0.1296	-9.8024582718258	-9.8024582718258\\
69.875	0.13326	-12.5351004573655	-12.5351004573655\\
69.875	0.13692	-15.0106631093304	-15.0106631093304\\
69.875	0.14058	-17.2291462277205	-17.2291462277205\\
69.875	0.14424	-19.1905498125357	-19.1905498125357\\
69.875	0.1479	-20.8948738637762	-20.8948738637762\\
69.875	0.15156	-22.3421183814417	-22.3421183814417\\
69.875	0.15522	-23.5322833655327	-23.5322833655327\\
69.875	0.15888	-24.4653688160484	-24.4653688160484\\
69.875	0.16254	-25.1413747329896	-25.1413747329896\\
69.875	0.1662	-25.5603011163561	-25.5603011163561\\
69.875	0.16986	-25.7221479661476	-25.7221479661476\\
69.875	0.17352	-25.6269152823645	-25.6269152823645\\
69.875	0.17718	-25.2746030650061	-25.2746030650061\\
69.875	0.18084	-24.6652113140736	-24.6652113140736\\
69.875	0.1845	-23.7987400295656	-23.7987400295656\\
69.875	0.18816	-22.6751892114831	-22.6751892114831\\
69.875	0.19182	-21.2945588598258	-21.2945588598258\\
69.875	0.19548	-19.6568489745937	-19.6568489745937\\
69.875	0.19914	-17.7620595557868	-17.7620595557868\\
69.875	0.2028	-15.610190603405	-15.610190603405\\
69.875	0.20646	-13.2012421174484	-13.2012421174484\\
69.875	0.21012	-10.5352140979168	-10.5352140979168\\
69.875	0.21378	-7.6121065448109	-7.6121065448109\\
69.875	0.21744	-4.4319194581297	-4.4319194581297\\
69.875	0.2211	-0.994652837873957	-0.994652837873957\\
69.875	0.22476	2.69969331595667	2.69969331595667\\
69.875	0.22842	6.65111900336234	6.65111900336234\\
69.875	0.23208	10.8596242243425	10.8596242243425\\
69.875	0.23574	15.3252089788978	15.3252089788978\\
69.875	0.2394	20.0478732670274	20.0478732670274\\
69.875	0.24306	25.0276170887323	25.0276170887323\\
69.875	0.24672	30.2644404440119	30.2644404440119\\
69.875	0.25038	35.7583433328662	35.7583433328662\\
69.875	0.25404	41.5093257552952	41.5093257552952\\
69.875	0.2577	47.5173877112993	47.5173877112993\\
69.875	0.26136	53.7825292008782	53.7825292008782\\
69.875	0.26502	60.3047502240315	60.3047502240315\\
69.875	0.26868	67.0840507807601	67.0840507807601\\
69.875	0.27234	74.1204308710633	74.1204308710633\\
69.875	0.276	81.4138904949415	81.4138904949415\\
70.25	0.093	33.4953805045435	33.4953805045435\\
70.25	0.09666	28.1137919917008	28.1137919917008\\
70.25	0.10032	22.989283012433	22.989283012433\\
70.25	0.10398	18.1218535667397	18.1218535667397\\
70.25	0.10764	13.5115036546215	13.5115036546215\\
70.25	0.1113	9.15823327607797	9.15823327607797\\
70.25	0.11496	5.06204243110936	5.06204243110936\\
70.25	0.11862	1.22293111971544	1.22293111971544\\
70.25	0.12228	-2.35910065810353	-2.35910065810353\\
70.25	0.12594	-5.6840529023477	-5.6840529023477\\
70.25	0.1296	-8.75192561301708	-8.75192561301708\\
70.25	0.13326	-11.5627187901121	-11.5627187901121\\
70.25	0.13692	-14.1164324336318	-14.1164324336318\\
70.25	0.14058	-16.4130665435767	-16.4130665435767\\
70.25	0.14424	-18.4526211199468	-18.4526211199468\\
70.25	0.1479	-20.2350961627423	-20.2350961627423\\
70.25	0.15156	-21.7604916719628	-21.7604916719628\\
70.25	0.15522	-23.0288076476084	-23.0288076476084\\
70.25	0.15888	-24.0400440896794	-24.0400440896794\\
70.25	0.16254	-24.7942009981755	-24.7942009981755\\
70.25	0.1662	-25.2912783730967	-25.2912783730967\\
70.25	0.16986	-25.5312762144431	-25.5312762144431\\
70.25	0.17352	-25.514194522215	-25.514194522215\\
70.25	0.17718	-25.2400332964117	-25.2400332964117\\
70.25	0.18084	-24.708792537034	-24.708792537034\\
70.25	0.1845	-23.9204722440811	-23.9204722440811\\
70.25	0.18816	-22.8750724175533	-22.8750724175533\\
70.25	0.19182	-21.5725930574511	-21.5725930574511\\
70.25	0.19548	-20.0130341637739	-20.0130341637739\\
70.25	0.19914	-18.1963957365217	-18.1963957365217\\
70.25	0.2028	-16.122677775695	-16.122677775695\\
70.25	0.20646	-13.7918802812932	-13.7918802812932\\
70.25	0.21012	-11.2040032533167	-11.2040032533167\\
70.25	0.21378	-8.35904669176534	-8.35904669176534\\
70.25	0.21744	-5.25701059663942	-5.25701059663942\\
70.25	0.2211	-1.8978949679385	-1.8978949679385\\
70.25	0.22476	1.71830019433708	1.71830019433708\\
70.25	0.22842	5.5915748901877	5.5915748901877\\
70.25	0.23208	9.72192911961304	9.72192911961304\\
70.25	0.23574	14.1093628826135	14.1093628826135\\
70.25	0.2394	18.7538761791883	18.7538761791883\\
70.25	0.24306	23.6554690093382	23.6554690093382\\
70.25	0.24672	28.8141413730625	28.8141413730625\\
70.25	0.25038	34.2298932703621	34.2298932703621\\
70.25	0.25404	39.9027247012364	39.9027247012364\\
70.25	0.2577	45.8326356656856	45.8326356656856\\
70.25	0.26136	52.0196261637095	52.0196261637095\\
70.25	0.26502	58.4636961953078	58.4636961953078\\
70.25	0.26868	65.1648457604813	65.1648457604813\\
70.25	0.27234	72.1230748592296	72.1230748592296\\
70.25	0.276	79.3383834915528	79.3383834915528\\
70.625	0.093	35.3777075719762	35.3777075719762\\
70.625	0.09666	29.9179680675785	29.9179680675785\\
70.625	0.10032	24.7153080967556	24.7153080967556\\
70.625	0.10398	19.7697276595075	19.7697276595075\\
70.625	0.10764	15.0812267558343	15.0812267558343\\
70.625	0.1113	10.649805385736	10.649805385736\\
70.625	0.11496	6.47546354921234	6.47546354921234\\
70.625	0.11862	2.55820124626348	2.55820124626348\\
70.625	0.12228	-1.10198152311054	-1.10198152311054\\
70.625	0.12594	-4.50508475890953	-4.50508475890953\\
70.625	0.1296	-7.65110846113396	-7.65110846113396\\
70.625	0.13326	-10.5400526297836	-10.5400526297836\\
70.625	0.13692	-13.1719172648581	-13.1719172648581\\
70.625	0.14058	-15.546702366358	-15.546702366358\\
70.625	0.14424	-17.6644079342832	-17.6644079342832\\
70.625	0.1479	-19.5250339686335	-19.5250339686335\\
70.625	0.15156	-21.128580469409	-21.128580469409\\
70.625	0.15522	-22.4750474366097	-22.4750474366097\\
70.625	0.15888	-23.5644348702355	-23.5644348702355\\
70.625	0.16254	-24.3967427702864	-24.3967427702864\\
70.625	0.1662	-24.9719711367629	-24.9719711367629\\
70.625	0.16986	-25.2901199696641	-25.2901199696641\\
70.625	0.17352	-25.3511892689908	-25.3511892689908\\
70.625	0.17718	-25.1551790347424	-25.1551790347424\\
70.625	0.18084	-24.7020892669195	-24.7020892669195\\
70.625	0.1845	-23.9919199655216	-23.9919199655216\\
70.625	0.18816	-23.0246711305489	-23.0246711305489\\
70.625	0.19182	-21.8003427620015	-21.8003427620015\\
70.625	0.19548	-20.3189348598793	-20.3189348598793\\
70.625	0.19914	-18.580447424182	-18.580447424182\\
70.625	0.2028	-16.5848804549103	-16.5848804549103\\
70.625	0.20646	-14.3322339520634	-14.3322339520634\\
70.625	0.21012	-11.8225079156419	-11.8225079156419\\
70.625	0.21378	-9.05570234564539	-9.05570234564539\\
70.625	0.21744	-6.03181724207428	-6.03181724207428\\
70.625	0.2211	-2.75085260492841	-2.75085260492841\\
70.625	0.22476	0.787191565792341	0.787191565792341\\
70.625	0.22842	4.58231527008792	4.58231527008792\\
70.625	0.23208	8.63451850795843	8.63451850795843\\
70.625	0.23574	12.9438012794038	12.9438012794038\\
70.625	0.2394	17.5101635844236	17.5101635844236\\
70.625	0.24306	22.3336054230186	22.3336054230186\\
70.625	0.24672	27.4141267951884	27.4141267951884\\
70.625	0.25038	32.751727700933	32.751727700933\\
70.625	0.25404	38.3464081402522	38.3464081402522\\
70.625	0.2577	44.1981681131464	44.1981681131464\\
70.625	0.26136	50.3070076196152	50.3070076196152\\
70.625	0.26502	56.6729266596589	56.6729266596589\\
70.625	0.26868	63.2959252332773	63.2959252332773\\
70.625	0.27234	70.1760033404709	70.1760033404709\\
70.625	0.276	77.313160981239	77.313160981239\\
71	0.093	37.310319132484	37.310319132484\\
71	0.09666	31.7724286365314	31.7724286365314\\
71	0.10032	26.4916176741536	26.4916176741536\\
71	0.10398	21.4678862453506	21.4678862453506\\
71	0.10764	16.7012343501225	16.7012343501225\\
71	0.1113	12.1916619884691	12.1916619884691\\
71	0.11496	7.93916916039062	7.93916916039062\\
71	0.11862	3.94375586588683	3.94375586588683\\
71	0.12228	0.205422104957989	0.205422104957989\\
71	0.12594	-3.27583212239605	-3.27583212239605\\
71	0.1296	-6.5000068161753	-6.5000068161753\\
71	0.13326	-9.46710197637995	-9.46710197637995\\
71	0.13692	-12.1771176030095	-12.1771176030095\\
71	0.14058	-14.6300536960643	-14.6300536960643\\
71	0.14424	-16.8259102555445	-16.8259102555445\\
71	0.1479	-18.7646872814496	-18.7646872814496\\
71	0.15156	-20.44638477378	-20.44638477378\\
71	0.15522	-21.8710027325357	-21.8710027325357\\
71	0.15888	-23.0385411577164	-23.0385411577164\\
71	0.16254	-23.9490000493223	-23.9490000493223\\
71	0.1662	-24.6023794073536	-24.6023794073536\\
71	0.16986	-24.9986792318097	-24.9986792318097\\
71	0.17352	-25.1378995226914	-25.1378995226914\\
71	0.17718	-25.020040279998	-25.020040279998\\
71	0.18084	-24.6451015037299	-24.6451015037299\\
71	0.1845	-24.0130831938871	-24.0130831938871\\
71	0.18816	-23.1239853504692	-23.1239853504692\\
71	0.19182	-21.9778079734767	-21.9778079734767\\
71	0.19548	-20.5745510629095	-20.5745510629095\\
71	0.19914	-18.914214618767	-18.914214618767\\
71	0.2028	-16.9967986410501	-16.9967986410501\\
71	0.20646	-14.8223031297585	-14.8223031297585\\
71	0.21012	-12.3907280848918	-12.3907280848918\\
71	0.21378	-9.70207350645035	-9.70207350645035\\
71	0.21744	-6.75633939443406	-6.75633939443406\\
71	0.2211	-3.55352574884301	-3.55352574884301\\
71	0.22476	-0.093632569677311	-0.093632569677311\\
71	0.22842	3.62334014306344	3.62334014306344\\
71	0.23208	7.59739238937891	7.59739238937891\\
71	0.23574	11.8285241692693	11.8285241692693\\
71	0.2394	16.3167354827344	16.3167354827344\\
71	0.24306	21.0620263297744	21.0620263297744\\
71	0.24672	26.0643967103891	26.0643967103891\\
71	0.25038	31.3238466245787	31.3238466245787\\
71	0.25404	36.840376072343	36.840376072343\\
71	0.2577	42.6139850536822	42.6139850536822\\
71	0.26136	48.6446735685964	48.6446735685964\\
71	0.26502	54.932441617085	54.932441617085\\
71	0.26868	61.4772891991487	61.4772891991487\\
71	0.27234	68.2792163147872	68.2792163147872\\
71	0.276	75.3382229640005	75.3382229640005\\
71.375	0.093	39.2932151860664	39.2932151860664\\
71.375	0.09666	33.6771736985589	33.6771736985589\\
71.375	0.10032	28.3182117446262	28.3182117446262\\
71.375	0.10398	23.2163293242682	23.2163293242682\\
71.375	0.10764	18.3715264374851	18.3715264374851\\
71.375	0.1113	13.7838030842769	13.7838030842769\\
71.375	0.11496	9.45315926464342	9.45315926464342\\
71.375	0.11862	5.37959497858469	5.37959497858469\\
71.375	0.12228	1.5631102261008	1.5631102261008\\
71.375	0.12594	-1.99629499280806	-1.99629499280806\\
71.375	0.1296	-5.29862067814213	-5.29862067814213\\
71.375	0.13326	-8.34386682990183	-8.34386682990183\\
71.375	0.13692	-11.1320334480862	-11.1320334480862\\
71.375	0.14058	-13.663120532696	-13.663120532696\\
71.375	0.14424	-15.937128083731	-15.937128083731\\
71.375	0.1479	-17.954056101191	-17.954056101191\\
71.375	0.15156	-19.7139045850764	-19.7139045850764\\
71.375	0.15522	-21.216673535387	-21.216673535387\\
71.375	0.15888	-22.4623629521227	-22.4623629521227\\
71.375	0.16254	-23.4509728352836	-23.4509728352836\\
71.375	0.1662	-24.1825031848696	-24.1825031848696\\
71.375	0.16986	-24.6569540008809	-24.6569540008809\\
71.375	0.17352	-24.8743252833174	-24.8743252833174\\
71.375	0.17718	-24.8346170321789	-24.8346170321789\\
71.375	0.18084	-24.5378292474658	-24.5378292474658\\
71.375	0.1845	-23.9839619291778	-23.9839619291778\\
71.375	0.18816	-23.1730150773148	-23.1730150773148\\
71.375	0.19182	-22.1049886918773	-22.1049886918773\\
71.375	0.19548	-20.7798827728649	-20.7798827728649\\
71.375	0.19914	-19.1976973202777	-19.1976973202777\\
71.375	0.2028	-17.3584323341157	-17.3584323341157\\
71.375	0.20646	-15.2620878143788	-15.2620878143788\\
71.375	0.21012	-12.9086637610672	-12.9086637610672\\
71.375	0.21378	-10.2981601741806	-10.2981601741806\\
71.375	0.21744	-7.43057705371911	-7.43057705371911\\
71.375	0.2211	-4.3059143996831	-4.3059143996831\\
71.375	0.22476	-0.924172212072222	-0.924172212072222\\
71.375	0.22842	2.71464950911371	2.71464950911371\\
71.375	0.23208	6.61055076387413	6.61055076387413\\
71.375	0.23574	10.7635315522097	10.7635315522097\\
71.375	0.2394	15.1735918741196	15.1735918741196\\
71.375	0.24306	19.8407317296047	19.8407317296047\\
71.375	0.24672	24.7649511186646	24.7649511186646\\
71.375	0.25038	29.9462500412991	29.9462500412991\\
71.375	0.25404	35.3846284975086	35.3846284975086\\
71.375	0.2577	41.0800864872932	41.0800864872932\\
71.375	0.26136	47.0326240106519	47.0326240106519\\
71.375	0.26502	53.242241067586	53.242241067586\\
71.375	0.26868	59.7089376580946	59.7089376580946\\
71.375	0.27234	66.432713782178	66.432713782178\\
71.375	0.276	73.4135694398365	73.4135694398365\\
71.75	0.093	41.326395732724	41.326395732724\\
71.75	0.09666	35.6322032536616	35.6322032536616\\
71.75	0.10032	30.1950903081739	30.1950903081739\\
71.75	0.10398	25.015056896261	25.015056896261\\
71.75	0.10764	20.092103017923	20.092103017923\\
71.75	0.1113	15.4262286731599	15.4262286731599\\
71.75	0.11496	11.0174338619715	11.0174338619715\\
71.75	0.11862	6.86571858435775	6.86571858435775\\
71.75	0.12228	2.97108284031904	2.97108284031904\\
71.75	0.12594	-0.666473370144871	-0.666473370144871\\
71.75	0.1296	-4.04695004703399	-4.04695004703399\\
71.75	0.13326	-7.17034719034828	-7.17034719034828\\
71.75	0.13692	-10.0366648000877	-10.0366648000877\\
71.75	0.14058	-12.6459028762526	-12.6459028762526\\
71.75	0.14424	-14.9980614188424	-14.9980614188424\\
71.75	0.1479	-17.0931404278577	-17.0931404278577\\
71.75	0.15156	-18.9311399032979	-18.9311399032979\\
71.75	0.15522	-20.5120598451633	-20.5120598451633\\
71.75	0.15888	-21.8359002534538	-21.8359002534538\\
71.75	0.16254	-22.9026611281696	-22.9026611281696\\
71.75	0.1662	-23.7123424693106	-23.7123424693106\\
71.75	0.16986	-24.2649442768767	-24.2649442768767\\
71.75	0.17352	-24.5604665508681	-24.5604665508681\\
71.75	0.17718	-24.5989092912848	-24.5989092912848\\
71.75	0.18084	-24.3802724981265	-24.3802724981265\\
71.75	0.1845	-23.9045561713934	-23.9045561713934\\
71.75	0.18816	-23.1717603110856	-23.1717603110856\\
71.75	0.19182	-22.1818849172029	-22.1818849172029\\
71.75	0.19548	-20.9349299897454	-20.9349299897454\\
71.75	0.19914	-19.430895528713	-19.430895528713\\
71.75	0.2028	-17.669781534106	-17.669781534106\\
71.75	0.20646	-15.651588005924	-15.651588005924\\
71.75	0.21012	-13.3763149441674	-13.3763149441674\\
71.75	0.21378	-10.8439623488358	-10.8439623488358\\
71.75	0.21744	-8.0545302199294	-8.0545302199294\\
71.75	0.2211	-5.00801855744822	-5.00801855744822\\
71.75	0.22476	-1.70442736139216	-1.70442736139216\\
71.75	0.22842	1.85624336823872	1.85624336823872\\
71.75	0.23208	5.67399363144432	5.67399363144432\\
71.75	0.23574	9.74882342822502	9.74882342822502\\
71.75	0.2394	14.0807327585801	14.0807327585801\\
71.75	0.24306	18.6697216225102	18.6697216225102\\
71.75	0.24672	23.5157900200151	23.5157900200151\\
71.75	0.25038	28.6189379510947	28.6189379510947\\
71.75	0.25404	33.9791654157492	33.9791654157492\\
71.75	0.2577	39.5964724139787	39.5964724139787\\
71.75	0.26136	45.4708589457829	45.4708589457829\\
71.75	0.26502	51.6023250111618	51.6023250111618\\
71.75	0.26868	57.9908706101154	57.9908706101154\\
71.75	0.27234	64.636495742644	64.636495742644\\
71.75	0.276	71.5392004087474	71.5392004087474\\
72.125	0.093	43.4098607724563	43.4098607724563\\
72.125	0.09666	37.637517301839	37.637517301839\\
72.125	0.10032	32.1222533647964	32.1222533647964\\
72.125	0.10398	26.8640689613285	26.8640689613285\\
72.125	0.10764	21.8629640914357	21.8629640914357\\
72.125	0.1113	17.1189387551175	17.1189387551175\\
72.125	0.11496	12.6319929523741	12.6319929523741\\
72.125	0.11862	8.40212668320555	8.40212668320555\\
72.125	0.12228	4.42933994761202	4.42933994761202\\
72.125	0.12594	0.713632745593287	0.713632745593287\\
72.125	0.1296	-2.74499492285088	-2.74499492285088\\
72.125	0.13326	-5.94654305772022	-5.94654305772022\\
72.125	0.13692	-8.89101165901474	-8.89101165901474\\
72.125	0.14058	-11.5784007267344	-11.5784007267344\\
72.125	0.14424	-14.008710260879	-14.008710260879\\
72.125	0.1479	-16.1819402614491	-16.1819402614491\\
72.125	0.15156	-18.0980907284442	-18.0980907284442\\
72.125	0.15522	-19.7571616618646	-19.7571616618646\\
72.125	0.15888	-21.1591530617102	-21.1591530617102\\
72.125	0.16254	-22.3040649279808	-22.3040649279808\\
72.125	0.1662	-23.1918972606768	-23.1918972606768\\
72.125	0.16986	-23.8226500597978	-23.8226500597978\\
72.125	0.17352	-24.1963233253442	-24.1963233253442\\
72.125	0.17718	-24.3129170573157	-24.3129170573157\\
72.125	0.18084	-24.1724312557125	-24.1724312557125\\
72.125	0.1845	-23.7748659205342	-23.7748659205342\\
72.125	0.18816	-23.1202210517812	-23.1202210517812\\
72.125	0.19182	-22.2084966494534	-22.2084966494534\\
72.125	0.19548	-21.0396927135509	-21.0396927135509\\
72.125	0.19914	-19.6138092440735	-19.6138092440735\\
72.125	0.2028	-17.9308462410214	-17.9308462410214\\
72.125	0.20646	-15.9908037043944	-15.9908037043944\\
72.125	0.21012	-13.7936816341926	-13.7936816341926\\
72.125	0.21378	-11.3394800304159	-11.3394800304159\\
72.125	0.21744	-8.62819889306428	-8.62819889306428\\
72.125	0.2211	-5.65983822213815	-5.65983822213815\\
72.125	0.22476	-2.43439801763714	-2.43439801763714\\
72.125	0.22842	1.0481217204387	1.0481217204387\\
72.125	0.23208	4.78772099208948	4.78772099208948\\
72.125	0.23574	8.78439979731513	8.78439979731513\\
72.125	0.2394	13.0381581361156	13.0381581361156\\
72.125	0.24306	17.5489960084907	17.5489960084907\\
72.125	0.24672	22.3169134144405	22.3169134144405\\
72.125	0.25038	27.3419103539653	27.3419103539653\\
72.125	0.25404	32.623986827065	32.623986827065\\
72.125	0.2577	38.1631428337392	38.1631428337392\\
72.125	0.26136	43.9593783739888	43.9593783739888\\
72.125	0.26502	50.0126934478127	50.0126934478127\\
72.125	0.26868	56.3230880552114	56.3230880552114\\
72.125	0.27234	62.890562196185	62.890562196185\\
72.125	0.276	69.7151158707336	69.7151158707336\\
72.5	0.093	45.5436103052638	45.5436103052638\\
72.5	0.09666	39.6931158430915	39.6931158430915\\
72.5	0.10032	34.0997009144939	34.0997009144939\\
72.5	0.10398	28.7633655194711	28.7633655194711\\
72.5	0.10764	23.6841096580233	23.6841096580233\\
72.5	0.1113	18.8619333301503	18.8619333301503\\
72.5	0.11496	14.2968365358521	14.2968365358521\\
72.5	0.11862	9.98881927512866	9.98881927512866\\
72.5	0.12228	5.93788154797986	5.93788154797986\\
72.5	0.12594	2.1440233544063	2.1440233544063\\
72.5	0.1296	-1.39275530559291	-1.39275530559291\\
72.5	0.13326	-4.67245443201708	-4.67245443201708\\
72.5	0.13692	-7.69507402486641	-7.69507402486641\\
72.5	0.14058	-10.4606140841409	-10.4606140841409\\
72.5	0.14424	-12.9690746098406	-12.9690746098406\\
72.5	0.1479	-15.2204556019657	-15.2204556019657\\
72.5	0.15156	-17.2147570605156	-17.2147570605156\\
72.5	0.15522	-18.9519789854911	-18.9519789854911\\
72.5	0.15888	-20.4321213768914	-20.4321213768914\\
72.5	0.16254	-21.6551842347171	-21.6551842347171\\
72.5	0.1662	-22.6211675589682	-22.6211675589682\\
72.5	0.16986	-23.330071349644	-23.330071349644\\
72.5	0.17352	-23.7818956067454	-23.7818956067454\\
72.5	0.17718	-23.9766403302716	-23.9766403302716\\
72.5	0.18084	-23.9143055202234	-23.9143055202234\\
72.5	0.1845	-23.5948911765999	-23.5948911765999\\
72.5	0.18816	-23.018397299402	-23.018397299402\\
72.5	0.19182	-22.1848238886292	-22.1848238886292\\
72.5	0.19548	-21.0941709442815	-21.0941709442815\\
72.5	0.19914	-19.7464384663592	-19.7464384663592\\
72.5	0.2028	-18.1416264548619	-18.1416264548619\\
72.5	0.20646	-16.2797349097897	-16.2797349097897\\
72.5	0.21012	-14.160763831143	-14.160763831143\\
72.5	0.21378	-11.7847132189211	-11.7847132189211\\
72.5	0.21744	-9.15158307312453	-9.15158307312453\\
72.5	0.2211	-6.26137339375322	-6.26137339375322\\
72.5	0.22476	-3.11408418080725	-3.11408418080725\\
72.5	0.22842	0.290284565713989	0.290284565713989\\
72.5	0.23208	3.95173284580949	3.95173284580949\\
72.5	0.23574	7.87026065948032	7.87026065948032\\
72.5	0.2394	12.0458680067258	12.0458680067258\\
72.5	0.24306	16.4785548875458	16.4785548875458\\
72.5	0.24672	21.168321301941	21.168321301941\\
72.5	0.25038	26.1151672499108	26.1151672499108\\
72.5	0.25404	31.3190927314554	31.3190927314554\\
72.5	0.2577	36.7800977465751	36.7800977465751\\
72.5	0.26136	42.4981822952691	42.4981822952691\\
72.5	0.26502	48.4733463775384	48.4733463775384\\
72.5	0.26868	54.7055899933821	54.7055899933821\\
72.5	0.27234	61.1949131428008	61.1949131428008\\
72.5	0.276	67.9413158257946	67.9413158257946\\
72.875	0.093	47.7276443311457	47.7276443311457\\
72.875	0.09666	41.7989988774185	41.7989988774185\\
72.875	0.10032	36.1274329572661	36.1274329572661\\
72.875	0.10398	30.7129465706884	30.7129465706884\\
72.875	0.10764	25.5555397176857	25.5555397176857\\
72.875	0.1113	20.6552123982577	20.6552123982577\\
72.875	0.11496	16.0119646124044	16.0119646124044\\
72.875	0.11862	11.6257963601262	11.6257963601262\\
72.875	0.12228	7.49670764142255	7.49670764142255\\
72.875	0.12594	3.62469845629371	3.62469845629371\\
72.875	0.1296	0.00976880473990605	0.00976880473990605\\
72.875	0.13326	-3.34808131323908	-3.34808131323908\\
72.875	0.13692	-6.44885189764346	-6.44885189764346\\
72.875	0.14058	-9.29254294847297	-9.29254294847297\\
72.875	0.14424	-11.8791544657275	-11.8791544657275\\
72.875	0.1479	-14.2086864494077	-14.2086864494077\\
72.875	0.15156	-16.2811388995126	-16.2811388995126\\
72.875	0.15522	-18.0965118160427	-18.0965118160427\\
72.875	0.15888	-19.6548051989981	-19.6548051989981\\
72.875	0.16254	-20.9560190483788	-20.9560190483788\\
72.875	0.1662	-22.0001533641845	-22.0001533641845\\
72.875	0.16986	-22.7872081464155	-22.7872081464155\\
72.875	0.17352	-23.3171833950718	-23.3171833950718\\
72.875	0.17718	-23.590079110153	-23.590079110153\\
72.875	0.18084	-23.6058952916597	-23.6058952916597\\
72.875	0.1845	-23.3646319395912	-23.3646319395912\\
72.875	0.18816	-22.8662890539482	-22.8662890539482\\
72.875	0.19182	-22.1108666347301	-22.1108666347301\\
72.875	0.19548	-21.0983646819378	-21.0983646819378\\
72.875	0.19914	-19.8287831955701	-19.8287831955701\\
72.875	0.2028	-18.3021221756277	-18.3021221756277\\
72.875	0.20646	-16.5183816221107	-16.5183816221107\\
72.875	0.21012	-14.4775615350185	-14.4775615350185\\
72.875	0.21378	-12.1796619143516	-12.1796619143516\\
72.875	0.21744	-9.62468276011015	-9.62468276011015\\
72.875	0.2211	-6.81262407229389	-6.81262407229389\\
72.875	0.22476	-3.74348585090252	-3.74348585090252\\
72.875	0.22842	-0.417268095936549	-0.417268095936549\\
72.875	0.23208	3.16602919260413	3.16602919260413\\
72.875	0.23574	7.00640601472014	7.00640601472014\\
72.875	0.2394	11.1038623704105	11.1038623704105\\
72.875	0.24306	15.4583982596757	15.4583982596757\\
72.875	0.24672	20.0700136825159	20.0700136825159\\
72.875	0.25038	24.9387086389308	24.9387086389308\\
72.875	0.25404	30.0644831289206	30.0644831289206\\
72.875	0.2577	35.4473371524853	35.4473371524853\\
72.875	0.26136	41.0872707096247	41.0872707096247\\
72.875	0.26502	46.9842838003385	46.9842838003385\\
72.875	0.26868	53.1383764246276	53.1383764246276\\
72.875	0.27234	59.5495485824913	59.5495485824913\\
72.875	0.276	66.21780027393	66.21780027393\\
73.25	0.093	49.9619628501029	49.9619628501029\\
73.25	0.09666	43.9551664048208	43.9551664048208\\
73.25	0.10032	38.2054494931134	38.2054494931134\\
73.25	0.10398	32.7128121149808	32.7128121149808\\
73.25	0.10764	27.4772542704231	27.4772542704231\\
73.25	0.1113	22.4987759594403	22.4987759594403\\
73.25	0.11496	17.7773771820322	17.7773771820322\\
73.25	0.11862	13.3130579381987	13.3130579381987\\
73.25	0.12228	9.1058182279402	9.1058182279402\\
73.25	0.12594	5.15565805125655	5.15565805125655\\
73.25	0.1296	1.46257740814792	1.46257740814792\\
73.25	0.13326	-1.97342370138634	-1.97342370138634\\
73.25	0.13692	-5.15234527734555	-5.15234527734555\\
73.25	0.14058	-8.07418731972987	-8.07418731972987\\
73.25	0.14424	-10.7389498285395	-10.7389498285395\\
73.25	0.1479	-13.1466328037742	-13.1466328037742\\
73.25	0.15156	-15.2972362454342	-15.2972362454342\\
73.25	0.15522	-17.1907601535193	-17.1907601535193\\
73.25	0.15888	-18.8272045280296	-18.8272045280296\\
73.25	0.16254	-20.2065693689651	-20.2065693689651\\
73.25	0.1662	-21.328854676326	-21.328854676326\\
73.25	0.16986	-22.1940604501117	-22.1940604501117\\
73.25	0.17352	-22.8021866903228	-22.8021866903228\\
73.25	0.17718	-23.1532333969593	-23.1532333969593\\
73.25	0.18084	-23.2472005700208	-23.2472005700208\\
73.25	0.1845	-23.0840882095073	-23.0840882095073\\
73.25	0.18816	-22.6638963154191	-22.6638963154191\\
73.25	0.19182	-21.9866248877561	-21.9866248877561\\
73.25	0.19548	-21.0522739265186	-21.0522739265186\\
73.25	0.19914	-19.8608434317059	-19.8608434317059\\
73.25	0.2028	-18.4123334033184	-18.4123334033184\\
73.25	0.20646	-16.7067438413562	-16.7067438413562\\
73.25	0.21012	-14.7440747458191	-14.7440747458191\\
73.25	0.21378	-12.5243261167072	-12.5243261167072\\
73.25	0.21744	-10.0474979540206	-10.0474979540206\\
73.25	0.2211	-7.31359025775913	-7.31359025775913\\
73.25	0.22476	-4.32260302792281	-4.32260302792281\\
73.25	0.22842	-1.07453626451166	-1.07453626451166\\
73.25	0.23208	2.43061003247419	2.43061003247419\\
73.25	0.23574	6.19283586303516	6.19283586303516\\
73.25	0.2394	10.2121412271705	10.2121412271705\\
73.25	0.24306	14.4885261248809	14.4885261248809\\
73.25	0.24672	19.021990556166	19.021990556166\\
73.25	0.25038	23.8125345210261	23.8125345210261\\
73.25	0.25404	28.8601580194609	28.8601580194609\\
73.25	0.2577	34.1648610514704	34.1648610514704\\
73.25	0.26136	39.7266436170548	39.7266436170548\\
73.25	0.26502	45.5455057162141	45.5455057162141\\
73.25	0.26868	51.6214473489481	51.6214473489481\\
73.25	0.27234	57.954468515257	57.954468515257\\
73.25	0.276	64.5445692151409	64.5445692151409\\
73.625	0.093	52.2465658621348	52.2465658621348\\
73.625	0.09666	46.1616184252978	46.1616184252978\\
73.625	0.10032	40.3337505220355	40.3337505220355\\
73.625	0.10398	34.7629621523479	34.7629621523479\\
73.625	0.10764	29.4492533162354	29.4492533162354\\
73.625	0.1113	24.3926240136975	24.3926240136975\\
73.625	0.11496	19.5930742447346	19.5930742447346\\
73.625	0.11862	15.0506040093462	15.0506040093462\\
73.625	0.12228	10.7652133075329	10.7652133075329\\
73.625	0.12594	6.73690213929424	6.73690213929424\\
73.625	0.1296	2.96567050463057	2.96567050463057\\
73.625	0.13326	-0.548481596458515	-0.548481596458515\\
73.625	0.13692	-3.80555416397254	-3.80555416397254\\
73.625	0.14058	-6.80554719791192	-6.80554719791192\\
73.625	0.14424	-9.54846069827656	-9.54846069827656\\
73.625	0.1479	-12.0342946650664	-12.0342946650664\\
73.625	0.15156	-14.2630490982812	-14.2630490982812\\
73.625	0.15522	-16.2347239979211	-16.2347239979211\\
73.625	0.15888	-17.9493193639864	-17.9493193639864\\
73.625	0.16254	-19.406835196477	-19.406835196477\\
73.625	0.1662	-20.6072714953928	-20.6072714953928\\
73.625	0.16986	-21.5506282607332	-21.5506282607332\\
73.625	0.17352	-22.2369054924994	-22.2369054924994\\
73.625	0.17718	-22.6661031906906	-22.6661031906906\\
73.625	0.18084	-22.8382213553072	-22.8382213553072\\
73.625	0.1845	-22.7532599863484	-22.7532599863484\\
73.625	0.18816	-22.4112190838152	-22.4112190838152\\
73.625	0.19182	-21.8120986477073	-21.8120986477073\\
73.625	0.19548	-20.9558986780245	-20.9558986780245\\
73.625	0.19914	-19.8426191747667	-19.8426191747667\\
73.625	0.2028	-18.4722601379343	-18.4722601379343\\
73.625	0.20646	-16.844821567527	-16.844821567527\\
73.625	0.21012	-14.9603034635448	-14.9603034635448\\
73.625	0.21378	-12.818705825988	-12.818705825988\\
73.625	0.21744	-10.4200286548561	-10.4200286548561\\
73.625	0.2211	-7.76427195014952	-7.76427195014952\\
73.625	0.22476	-4.85143571186825	-4.85143571186825\\
73.625	0.22842	-1.68151994001215	-1.68151994001215\\
73.625	0.23208	1.74547536541888	1.74547536541888\\
73.625	0.23574	5.4295502044248	5.4295502044248\\
73.625	0.2394	9.37070457700554	9.37070457700554\\
73.625	0.24306	13.5689384831609	13.5689384831609\\
73.625	0.24672	18.0242519228912	18.0242519228912\\
73.625	0.25038	22.7366448961963	22.7366448961963\\
73.625	0.25404	27.706117403076	27.706117403076\\
73.625	0.2577	32.9326694435309	32.9326694435309\\
73.625	0.26136	38.4163010175603	38.4163010175603\\
73.625	0.26502	44.1570121251644	44.1570121251644\\
73.625	0.26868	50.1548027663437	50.1548027663437\\
73.625	0.27234	56.4096729410977	56.4096729410977\\
73.625	0.276	62.9216226494264	62.9216226494264\\
74	0.093	54.5814533672418	54.5814533672418\\
74	0.09666	48.4183549388499	48.4183549388499\\
74	0.10032	42.5123360440326	42.5123360440326\\
74	0.10398	36.8633966827902	36.8633966827902\\
74	0.10764	31.4715368551226	31.4715368551226\\
74	0.1113	26.3367565610299	26.3367565610299\\
74	0.11496	21.4590558005119	21.4590558005119\\
74	0.11862	16.8384345735687	16.8384345735687\\
74	0.12228	12.4748928802004	12.4748928802004\\
74	0.12594	8.36843072040691	8.36843072040691\\
74	0.1296	4.51904809418841	4.51904809418841\\
74	0.13326	0.926745001544276	0.926745001544276\\
74	0.13692	-2.4084785575248	-2.4084785575248\\
74	0.14058	-5.486622583019	-5.486622583019\\
74	0.14424	-8.30768707493846	-8.30768707493846\\
74	0.1479	-10.8716720332831	-10.8716720332831\\
74	0.15156	-13.1785774580529	-13.1785774580529\\
74	0.15522	-15.2284033492479	-15.2284033492479\\
74	0.15888	-17.0211497068683	-17.0211497068683\\
74	0.16254	-18.5568165309134	-18.5568165309134\\
74	0.1662	-19.8354038213843	-19.8354038213843\\
74	0.16986	-20.8569115782798	-20.8569115782798\\
74	0.17352	-21.6213398016008	-21.6213398016008\\
74	0.17718	-22.1286884913468	-22.1286884913468\\
74	0.18084	-22.3789576475185	-22.3789576475185\\
74	0.1845	-22.3721472701147	-22.3721472701147\\
74	0.18816	-22.1082573591365	-22.1082573591365\\
74	0.19182	-21.5872879145832	-21.5872879145832\\
74	0.19548	-20.8092389364555	-20.8092389364555\\
74	0.19914	-19.7741104247525	-19.7741104247525\\
74	0.2028	-18.4819023794751	-18.4819023794751\\
74	0.20646	-16.932614800623	-16.932614800623\\
74	0.21012	-15.1262476881955	-15.1262476881955\\
74	0.21378	-13.0628010421935	-13.0628010421935\\
74	0.21744	-10.7422748626167	-10.7422748626167\\
74	0.2211	-8.16466914946494	-8.16466914946494\\
74	0.22476	-5.32998390273872	-5.32998390273872\\
74	0.22842	-2.23821912243744	-2.23821912243744\\
74	0.23208	1.11062519143854	1.11062519143854\\
74	0.23574	4.71654903888964	4.71654903888964\\
74	0.2394	8.57955241991533	8.57955241991533\\
74	0.24306	12.6996353345158	12.6996353345158\\
74	0.24672	17.0767977826911	17.0767977826911\\
74	0.25038	21.7110397644414	21.7110397644414\\
74	0.25404	26.6023612797662	26.6023612797662\\
74	0.2577	31.7507623286659	31.7507623286659\\
74	0.26136	37.1562429111407	37.1562429111407\\
74	0.26502	42.8188030271898	42.8188030271898\\
74	0.26868	48.738442676814	48.738442676814\\
74	0.27234	54.9151618600132	54.9151618600132\\
74	0.276	61.348960576787	61.348960576787\\
};
\end{axis}

\begin{axis}[%
width=2.616cm,
height=2.517cm,
at={(0cm,13.986cm)},
scale only axis,
xmin=56,
xmax=74,
tick align=outside,
xlabel style={font=\color{white!15!black}},
xlabel={$L_{cut}$},
ymin=0.093,
ymax=0.276,
ylabel style={font=\color{white!15!black}},
ylabel={$D_{rlx}$},
zmin=-6.20069940174891,
zmax=0.448724105722027,
zlabel style={font=\color{white!15!black}},
zlabel={$x_1,x_3$},
view={-140}{50},
axis background/.style={fill=white},
xmajorgrids,
ymajorgrids,
zmajorgrids,
legend style={at={(1.03,1)}, anchor=north west, legend cell align=left, align=left, draw=white!15!black}
]
\addplot3[only marks, mark=*, mark options={}, mark size=1.5000pt, color=mycolor1, fill=mycolor1] table[row sep=crcr]{%
x	y	z\\
74	0.123	-0.234663723939627\\
72	0.113	-0.974827529912944\\
61	0.095	0.0483904855395844\\
56	0.093	0.0442880574542841\\
};
\addlegendentry{data1}

\addplot3[only marks, mark=*, mark options={}, mark size=1.5000pt, color=mycolor2, fill=mycolor2] table[row sep=crcr]{%
x	y	z\\
67	0.276	-3.30596642126641\\
66	0.255	-2.47661651083367\\
62	0.209	-1.06616605268006\\
57	0.193	-0.988638371147925\\
};
\addlegendentry{data2}

\addplot3[only marks, mark=*, mark options={}, mark size=1.5000pt, color=black, fill=black] table[row sep=crcr]{%
x	y	z\\
69	0.104	-0.656552506480986\\
};
\addlegendentry{data3}

\addplot3[only marks, mark=*, mark options={}, mark size=1.5000pt, color=black, fill=black] table[row sep=crcr]{%
x	y	z\\
64	0.23	-1.56547503472912\\
};
\addlegendentry{data4}


\addplot3[%
surf,
fill opacity=0.7, shader=interp, colormap={mymap}{[1pt] rgb(0pt)=(1,0.905882,0); rgb(1pt)=(1,0.901964,0); rgb(2pt)=(1,0.898051,0); rgb(3pt)=(1,0.894144,0); rgb(4pt)=(1,0.890243,0); rgb(5pt)=(1,0.886349,0); rgb(6pt)=(1,0.88246,0); rgb(7pt)=(1,0.878577,0); rgb(8pt)=(1,0.8747,0); rgb(9pt)=(1,0.870829,0); rgb(10pt)=(1,0.866964,0); rgb(11pt)=(1,0.863106,0); rgb(12pt)=(1,0.859253,0); rgb(13pt)=(1,0.855406,0); rgb(14pt)=(1,0.851566,0); rgb(15pt)=(1,0.847732,0); rgb(16pt)=(1,0.843903,0); rgb(17pt)=(1,0.840081,0); rgb(18pt)=(1,0.836265,0); rgb(19pt)=(1,0.832455,0); rgb(20pt)=(1,0.828652,0); rgb(21pt)=(1,0.824854,0); rgb(22pt)=(1,0.821063,0); rgb(23pt)=(1,0.817278,0); rgb(24pt)=(1,0.8135,0); rgb(25pt)=(1,0.809727,0); rgb(26pt)=(1,0.805961,0); rgb(27pt)=(1,0.8022,0); rgb(28pt)=(1,0.798445,0); rgb(29pt)=(1,0.794696,0); rgb(30pt)=(1,0.790953,0); rgb(31pt)=(1,0.787215,0); rgb(32pt)=(1,0.783484,0); rgb(33pt)=(1,0.779758,0); rgb(34pt)=(1,0.776038,0); rgb(35pt)=(1,0.772324,0); rgb(36pt)=(1,0.768615,0); rgb(37pt)=(1,0.764913,0); rgb(38pt)=(1,0.761217,0); rgb(39pt)=(1,0.757527,0); rgb(40pt)=(1,0.753843,0); rgb(41pt)=(1,0.750165,0); rgb(42pt)=(1,0.746493,0); rgb(43pt)=(1,0.742827,0); rgb(44pt)=(1,0.739167,0); rgb(45pt)=(1,0.735514,0); rgb(46pt)=(1,0.731867,0); rgb(47pt)=(1,0.728226,0); rgb(48pt)=(1,0.724591,0); rgb(49pt)=(1,0.720963,0); rgb(50pt)=(1,0.717341,0); rgb(51pt)=(1,0.713725,0); rgb(52pt)=(0.999994,0.710077,0); rgb(53pt)=(0.999974,0.706363,0); rgb(54pt)=(0.999942,0.702592,0); rgb(55pt)=(0.999898,0.698775,0); rgb(56pt)=(0.999841,0.694921,0); rgb(57pt)=(0.999771,0.691039,0); rgb(58pt)=(0.99969,0.687139,0); rgb(59pt)=(0.999596,0.68323,0); rgb(60pt)=(0.99949,0.679323,0); rgb(61pt)=(0.999372,0.675427,0); rgb(62pt)=(0.999242,0.67155,0); rgb(63pt)=(0.9991,0.667704,0); rgb(64pt)=(0.998946,0.663897,0); rgb(65pt)=(0.998781,0.660138,0); rgb(66pt)=(0.998605,0.656439,0); rgb(67pt)=(0.998416,0.652807,0); rgb(68pt)=(0.998217,0.649253,0); rgb(69pt)=(0.998006,0.645786,0); rgb(70pt)=(0.997785,0.642416,0); rgb(71pt)=(0.997552,0.639152,0); rgb(72pt)=(0.997308,0.636004,0); rgb(73pt)=(0.997053,0.632982,0); rgb(74pt)=(0.996788,0.630095,0); rgb(75pt)=(0.996512,0.627352,0); rgb(76pt)=(0.996226,0.624763,0); rgb(77pt)=(0.995851,0.622329,0); rgb(78pt)=(0.99494,0.619997,0); rgb(79pt)=(0.99345,0.617753,0); rgb(80pt)=(0.991419,0.61559,0); rgb(81pt)=(0.988885,0.613503,0); rgb(82pt)=(0.985886,0.611486,0); rgb(83pt)=(0.98246,0.609532,0); rgb(84pt)=(0.978643,0.607636,0); rgb(85pt)=(0.974475,0.605791,0); rgb(86pt)=(0.969992,0.603992,0); rgb(87pt)=(0.965232,0.602233,0); rgb(88pt)=(0.960233,0.600507,0); rgb(89pt)=(0.955033,0.598808,0); rgb(90pt)=(0.949669,0.59713,0); rgb(91pt)=(0.94418,0.595468,0); rgb(92pt)=(0.938602,0.593815,0); rgb(93pt)=(0.932974,0.592166,0); rgb(94pt)=(0.927333,0.590513,0); rgb(95pt)=(0.921717,0.588852,0); rgb(96pt)=(0.916164,0.587176,0); rgb(97pt)=(0.910711,0.585479,0); rgb(98pt)=(0.905397,0.583755,0); rgb(99pt)=(0.900258,0.581999,0); rgb(100pt)=(0.895333,0.580203,0); rgb(101pt)=(0.890659,0.578362,0); rgb(102pt)=(0.886275,0.576471,0); rgb(103pt)=(0.882047,0.574545,0); rgb(104pt)=(0.877819,0.572608,0); rgb(105pt)=(0.873592,0.57066,0); rgb(106pt)=(0.869366,0.568701,0); rgb(107pt)=(0.865143,0.566733,0); rgb(108pt)=(0.860924,0.564756,0); rgb(109pt)=(0.856708,0.562771,0); rgb(110pt)=(0.852497,0.560778,0); rgb(111pt)=(0.848292,0.558779,0); rgb(112pt)=(0.844092,0.556774,0); rgb(113pt)=(0.8399,0.554763,0); rgb(114pt)=(0.835716,0.552749,0); rgb(115pt)=(0.831541,0.55073,0); rgb(116pt)=(0.827374,0.548709,0); rgb(117pt)=(0.823219,0.546686,0); rgb(118pt)=(0.819074,0.54466,0); rgb(119pt)=(0.81494,0.542635,0); rgb(120pt)=(0.81082,0.540609,0); rgb(121pt)=(0.806712,0.538584,0); rgb(122pt)=(0.802619,0.53656,0); rgb(123pt)=(0.798541,0.534539,0); rgb(124pt)=(0.794478,0.532521,0); rgb(125pt)=(0.790431,0.530506,0); rgb(126pt)=(0.786402,0.528496,0); rgb(127pt)=(0.782391,0.526491,0); rgb(128pt)=(0.77841,0.524489,0); rgb(129pt)=(0.774523,0.522478,0); rgb(130pt)=(0.770731,0.520455,0); rgb(131pt)=(0.767022,0.518424,0); rgb(132pt)=(0.763384,0.516385,0); rgb(133pt)=(0.759804,0.514339,0); rgb(134pt)=(0.756272,0.51229,0); rgb(135pt)=(0.752775,0.510237,0); rgb(136pt)=(0.749302,0.508182,0); rgb(137pt)=(0.74584,0.506128,0); rgb(138pt)=(0.742378,0.504075,0); rgb(139pt)=(0.738904,0.502025,0); rgb(140pt)=(0.735406,0.499979,0); rgb(141pt)=(0.731872,0.49794,0); rgb(142pt)=(0.72829,0.495909,0); rgb(143pt)=(0.724649,0.493887,0); rgb(144pt)=(0.720936,0.491875,0); rgb(145pt)=(0.71714,0.489876,0); rgb(146pt)=(0.713249,0.487891,0); rgb(147pt)=(0.709251,0.485921,0); rgb(148pt)=(0.705134,0.483968,0); rgb(149pt)=(0.700887,0.482033,0); rgb(150pt)=(0.696497,0.480118,0); rgb(151pt)=(0.691952,0.478225,0); rgb(152pt)=(0.687242,0.476355,0); rgb(153pt)=(0.682353,0.47451,0); rgb(154pt)=(0.677195,0.472696,0); rgb(155pt)=(0.6717,0.470916,0); rgb(156pt)=(0.665891,0.469169,0); rgb(157pt)=(0.659791,0.46745,0); rgb(158pt)=(0.653423,0.465756,0); rgb(159pt)=(0.64681,0.464084,0); rgb(160pt)=(0.639976,0.462432,0); rgb(161pt)=(0.632943,0.460795,0); rgb(162pt)=(0.625734,0.459171,0); rgb(163pt)=(0.618373,0.457556,0); rgb(164pt)=(0.610882,0.455948,0); rgb(165pt)=(0.603284,0.454343,0); rgb(166pt)=(0.595604,0.452737,0); rgb(167pt)=(0.587863,0.451129,0); rgb(168pt)=(0.580084,0.449514,0); rgb(169pt)=(0.572292,0.447889,0); rgb(170pt)=(0.564508,0.446252,0); rgb(171pt)=(0.556756,0.444599,0); rgb(172pt)=(0.549059,0.442927,0); rgb(173pt)=(0.54144,0.441232,0); rgb(174pt)=(0.533922,0.439512,0); rgb(175pt)=(0.526529,0.437764,0); rgb(176pt)=(0.519282,0.435983,0); rgb(177pt)=(0.512206,0.434168,0); rgb(178pt)=(0.505323,0.432315,0); rgb(179pt)=(0.498628,0.430422,3.92506e-06); rgb(180pt)=(0.491973,0.428504,3.49981e-05); rgb(181pt)=(0.485331,0.426562,9.63073e-05); rgb(182pt)=(0.478704,0.424596,0.000186979); rgb(183pt)=(0.472096,0.422609,0.000306141); rgb(184pt)=(0.465508,0.420599,0.00045292); rgb(185pt)=(0.458942,0.418567,0.000626441); rgb(186pt)=(0.452401,0.416515,0.000825833); rgb(187pt)=(0.445885,0.414441,0.00105022); rgb(188pt)=(0.439399,0.412348,0.00129873); rgb(189pt)=(0.432942,0.410234,0.00157049); rgb(190pt)=(0.426518,0.408102,0.00186463); rgb(191pt)=(0.420129,0.40595,0.00218028); rgb(192pt)=(0.413777,0.40378,0.00251655); rgb(193pt)=(0.407464,0.401592,0.00287258); rgb(194pt)=(0.401191,0.399386,0.00324749); rgb(195pt)=(0.394962,0.397164,0.00364042); rgb(196pt)=(0.388777,0.394925,0.00405048); rgb(197pt)=(0.38264,0.39267,0.00447681); rgb(198pt)=(0.376552,0.390399,0.00491852); rgb(199pt)=(0.370516,0.388113,0.00537476); rgb(200pt)=(0.364532,0.385812,0.00584464); rgb(201pt)=(0.358605,0.383497,0.00632729); rgb(202pt)=(0.352735,0.381168,0.00682184); rgb(203pt)=(0.346925,0.378826,0.00732741); rgb(204pt)=(0.341176,0.376471,0.00784314); rgb(205pt)=(0.335485,0.374093,0.00847245); rgb(206pt)=(0.329843,0.371682,0.00930909); rgb(207pt)=(0.324249,0.369242,0.0103377); rgb(208pt)=(0.318701,0.366772,0.0115428); rgb(209pt)=(0.313198,0.364275,0.0129091); rgb(210pt)=(0.307739,0.361753,0.0144211); rgb(211pt)=(0.302322,0.359206,0.0160634); rgb(212pt)=(0.296945,0.356637,0.0178207); rgb(213pt)=(0.291607,0.354048,0.0196776); rgb(214pt)=(0.286307,0.35144,0.0216186); rgb(215pt)=(0.281043,0.348814,0.0236284); rgb(216pt)=(0.275813,0.346172,0.0256916); rgb(217pt)=(0.270616,0.343517,0.0277927); rgb(218pt)=(0.265451,0.340849,0.0299163); rgb(219pt)=(0.260317,0.33817,0.0320472); rgb(220pt)=(0.25521,0.335482,0.0341698); rgb(221pt)=(0.250131,0.332786,0.0362688); rgb(222pt)=(0.245078,0.330085,0.0383287); rgb(223pt)=(0.240048,0.327379,0.0403343); rgb(224pt)=(0.235042,0.324671,0.04227); rgb(225pt)=(0.230056,0.321962,0.0441205); rgb(226pt)=(0.22509,0.319254,0.0458704); rgb(227pt)=(0.220142,0.316548,0.0475043); rgb(228pt)=(0.215212,0.313846,0.0490067); rgb(229pt)=(0.210296,0.311149,0.0503624); rgb(230pt)=(0.205395,0.308459,0.0515759); rgb(231pt)=(0.200514,0.305763,0.052757); rgb(232pt)=(0.195655,0.303061,0.0539242); rgb(233pt)=(0.190817,0.300353,0.0550763); rgb(234pt)=(0.186001,0.297639,0.0562123); rgb(235pt)=(0.181207,0.294918,0.0573313); rgb(236pt)=(0.176434,0.292191,0.0584321); rgb(237pt)=(0.171685,0.289458,0.0595136); rgb(238pt)=(0.166957,0.286719,0.060575); rgb(239pt)=(0.162252,0.283973,0.0616151); rgb(240pt)=(0.15757,0.281221,0.0626328); rgb(241pt)=(0.152911,0.278463,0.0636271); rgb(242pt)=(0.148275,0.275699,0.0645971); rgb(243pt)=(0.143663,0.272929,0.0655416); rgb(244pt)=(0.139074,0.270152,0.0664596); rgb(245pt)=(0.134508,0.26737,0.06735); rgb(246pt)=(0.129967,0.264581,0.0682118); rgb(247pt)=(0.125449,0.261787,0.0690441); rgb(248pt)=(0.120956,0.258986,0.0698456); rgb(249pt)=(0.116487,0.25618,0.0706154); rgb(250pt)=(0.112043,0.253367,0.0713525); rgb(251pt)=(0.107623,0.250549,0.0720557); rgb(252pt)=(0.103229,0.247724,0.0727241); rgb(253pt)=(0.0988592,0.244894,0.0733566); rgb(254pt)=(0.0945149,0.242058,0.0739522); rgb(255pt)=(0.0901961,0.239216,0.0745098)}, mesh/rows=49]
table[row sep=crcr, point meta=\thisrow{c}] {%
%
x	y	z	c\\
56	0.093	0.180553068454304	0.180553068454304\\
56	0.09666	0.22260912760537	0.22260912760537\\
56	0.10032	0.257739427223574	0.257739427223574\\
56	0.10398	0.285943967308919	0.285943967308919\\
56	0.10764	0.307222747861397	0.307222747861397\\
56	0.1113	0.321575768881014	0.321575768881014\\
56	0.11496	0.329003030367767	0.329003030367767\\
56	0.11862	0.329504532321661	0.329504532321661\\
56	0.12228	0.323080274742689	0.323080274742689\\
56	0.12594	0.309730257630856	0.309730257630856\\
56	0.1296	0.289454480986161	0.289454480986161\\
56	0.13326	0.262252944808604	0.262252944808604\\
56	0.13692	0.228125649098184	0.228125649098184\\
56	0.14058	0.187072593854899	0.187072593854899\\
56	0.14424	0.139093779078755	0.139093779078755\\
56	0.1479	0.0841892047697463	0.0841892047697463\\
56	0.15156	0.0223588709278761	0.0223588709278761\\
56	0.15522	-0.0463972224468563	-0.0463972224468563\\
56	0.15888	-0.122079075354454	-0.122079075354454\\
56	0.16254	-0.204686687794908	-0.204686687794908\\
56	0.1662	-0.294220059768228	-0.294220059768228\\
56	0.16986	-0.390679191274413	-0.390679191274413\\
56	0.17352	-0.494064082313459	-0.494064082313459\\
56	0.17718	-0.604374732885367	-0.604374732885367\\
56	0.18084	-0.721611142990136	-0.721611142990136\\
56	0.1845	-0.84577331262777	-0.84577331262777\\
56	0.18816	-0.976861241798263	-0.976861241798263\\
56	0.19182	-1.11487493050162	-1.11487493050162\\
56	0.19548	-1.25981437873785	-1.25981437873785\\
56	0.19914	-1.41167958650693	-1.41167958650693\\
56	0.2028	-1.57047055380887	-1.57047055380887\\
56	0.20646	-1.73618728064368	-1.73618728064368\\
56	0.21012	-1.90882976701135	-1.90882976701135\\
56	0.21378	-2.08839801291188	-2.08839801291188\\
56	0.21744	-2.27489201834528	-2.27489201834528\\
56	0.2211	-2.46831178331154	-2.46831178331154\\
56	0.22476	-2.66865730781066	-2.66865730781066\\
56	0.22842	-2.87592859184264	-2.87592859184264\\
56	0.23208	-3.09012563540749	-3.09012563540749\\
56	0.23574	-3.3112484385052	-3.3112484385052\\
56	0.2394	-3.53929700113576	-3.53929700113576\\
56	0.24306	-3.7742713232992	-3.7742713232992\\
56	0.24672	-4.01617140499549	-4.01617140499549\\
56	0.25038	-4.26499724622465	-4.26499724622465\\
56	0.25404	-4.52074884698668	-4.52074884698668\\
56	0.2577	-4.78342620728156	-4.78342620728156\\
56	0.26136	-5.0530293271093	-5.0530293271093\\
56	0.26502	-5.32955820646991	-5.32955820646991\\
56	0.26868	-5.61301284536338	-5.61301284536338\\
56	0.27234	-5.90339324378971	-5.90339324378971\\
56	0.276	-6.20069940174891	-6.20069940174891\\
56.375	0.093	0.153540956621682	0.153540956621682\\
56.375	0.09666	0.198203740293651	0.198203740293651\\
56.375	0.10032	0.235940764432757	0.235940764432757\\
56.375	0.10398	0.266752029039	0.266752029039\\
56.375	0.10764	0.290637534112382	0.290637534112382\\
56.375	0.1113	0.307597279652901	0.307597279652901\\
56.375	0.11496	0.317631265660558	0.317631265660558\\
56.375	0.11862	0.32073949213535	0.32073949213535\\
56.375	0.12228	0.31692195907728	0.31692195907728\\
56.375	0.12594	0.30617866648635	0.30617866648635\\
56.375	0.1296	0.288509614362557	0.288509614362557\\
56.375	0.13326	0.2639148027059	0.2639148027059\\
56.375	0.13692	0.23239423151638	0.23239423151638\\
56.375	0.14058	0.193947900793997	0.193947900793997\\
56.375	0.14424	0.148575810538754	0.148575810538754\\
56.375	0.1479	0.0962779607506485	0.0962779607506485\\
56.375	0.15156	0.0370543514296786	0.0370543514296786\\
56.375	0.15522	-0.0290950174241535	-0.0290950174241535\\
56.375	0.15888	-0.102170145810845	-0.102170145810845\\
56.375	0.16254	-0.182171033730403	-0.182171033730403\\
56.375	0.1662	-0.269097681182823	-0.269097681182823\\
56.375	0.16986	-0.362950088168104	-0.362950088168104\\
56.375	0.17352	-0.463728254686246	-0.463728254686246\\
56.375	0.17718	-0.571432180737255	-0.571432180737255\\
56.375	0.18084	-0.686061866321123	-0.686061866321123\\
56.375	0.1845	-0.807617311437855	-0.807617311437855\\
56.375	0.18816	-0.936098516087448	-0.936098516087448\\
56.375	0.19182	-1.0715054802699	-1.0715054802699\\
56.375	0.19548	-1.21383820398522	-1.21383820398522\\
56.375	0.19914	-1.3630966872334	-1.3630966872334\\
56.375	0.2028	-1.51928093001445	-1.51928093001445\\
56.375	0.20646	-1.68239093232835	-1.68239093232835\\
56.375	0.21012	-1.85242669417513	-1.85242669417513\\
56.375	0.21378	-2.02938821555476	-2.02938821555476\\
56.375	0.21744	-2.21327549646725	-2.21327549646725\\
56.375	0.2211	-2.4040885369126	-2.4040885369126\\
56.375	0.22476	-2.60182733689083	-2.60182733689083\\
56.375	0.22842	-2.80649189640191	-2.80649189640191\\
56.375	0.23208	-3.01808221544585	-3.01808221544585\\
56.375	0.23574	-3.23659829402266	-3.23659829402266\\
56.375	0.2394	-3.46204013213233	-3.46204013213233\\
56.375	0.24306	-3.69440772977486	-3.69440772977486\\
56.375	0.24672	-3.93370108695025	-3.93370108695025\\
56.375	0.25038	-4.17992020365851	-4.17992020365851\\
56.375	0.25404	-4.43306507989963	-4.43306507989963\\
56.375	0.2577	-4.69313571567361	-4.69313571567361\\
56.375	0.26136	-4.96013211098046	-4.96013211098046\\
56.375	0.26502	-5.23405426582016	-5.23405426582016\\
56.375	0.26868	-5.51490218019273	-5.51490218019273\\
56.375	0.27234	-5.80267585409816	-5.80267585409816\\
56.375	0.276	-6.09737528753645	-6.09737528753645\\
56.75	0.093	0.126052789258237	0.126052789258237\\
56.75	0.09666	0.173322297451108	0.173322297451108\\
56.75	0.10032	0.213666046111114	0.213666046111114\\
56.75	0.10398	0.24708403523826	0.24708403523826\\
56.75	0.10764	0.273576264832543	0.273576264832543\\
56.75	0.1113	0.293142734893963	0.293142734893963\\
56.75	0.11496	0.30578344542252	0.30578344542252\\
56.75	0.11862	0.311498396418216	0.311498396418216\\
56.75	0.12228	0.310287587881046	0.310287587881046\\
56.75	0.12594	0.302151019811018	0.302151019811018\\
56.75	0.1296	0.287088692208125	0.287088692208125\\
56.75	0.13326	0.26510060507237	0.26510060507237\\
56.75	0.13692	0.236186758403752	0.236186758403752\\
56.75	0.14058	0.200347152202272	0.200347152202272\\
56.75	0.14424	0.15758178646793	0.15758178646793\\
56.75	0.1479	0.107890661200726	0.107890661200726\\
56.75	0.15156	0.0512737764006594	0.0512737764006594\\
56.75	0.15522	-0.0122688679322707	-0.0122688679322707\\
56.75	0.15888	-0.0827372717980657	-0.0827372717980657\\
56.75	0.16254	-0.160131435196719	-0.160131435196719\\
56.75	0.1662	-0.244451358128237	-0.244451358128237\\
56.75	0.16986	-0.335697040592616	-0.335697040592616\\
56.75	0.17352	-0.433868482589858	-0.433868482589858\\
56.75	0.17718	-0.538965684119967	-0.538965684119967\\
56.75	0.18084	-0.650988645182933	-0.650988645182933\\
56.75	0.1845	-0.769937365778764	-0.769937365778764\\
56.75	0.18816	-0.895811845907454	-0.895811845907454\\
56.75	0.19182	-1.02861208556901	-1.02861208556901\\
56.75	0.19548	-1.16833808476343	-1.16833808476343\\
56.75	0.19914	-1.31498984349071	-1.31498984349071\\
56.75	0.2028	-1.46856736175085	-1.46856736175085\\
56.75	0.20646	-1.62907063954386	-1.62907063954386\\
56.75	0.21012	-1.79649967686972	-1.79649967686972\\
56.75	0.21378	-1.97085447372845	-1.97085447372845\\
56.75	0.21744	-2.15213503012005	-2.15213503012005\\
56.75	0.2211	-2.3403413460445	-2.3403413460445\\
56.75	0.22476	-2.53547342150182	-2.53547342150182\\
56.75	0.22842	-2.737531256492	-2.737531256492\\
56.75	0.23208	-2.94651485101504	-2.94651485101504\\
56.75	0.23574	-3.16242420507095	-3.16242420507095\\
56.75	0.2394	-3.38525931865971	-3.38525931865971\\
56.75	0.24306	-3.61502019178135	-3.61502019178135\\
56.75	0.24672	-3.85170682443584	-3.85170682443584\\
56.75	0.25038	-4.09531921662319	-4.09531921662319\\
56.75	0.25404	-4.34585736834341	-4.34585736834341\\
56.75	0.2577	-4.60332127959649	-4.60332127959649\\
56.75	0.26136	-4.86771095038243	-4.86771095038243\\
56.75	0.26502	-5.13902638070124	-5.13902638070124\\
56.75	0.26868	-5.41726757055291	-5.41726757055291\\
56.75	0.27234	-5.70243451993743	-5.70243451993743\\
56.75	0.276	-5.99452722885483	-5.99452722885483\\
57.125	0.093	0.0980885663639728	0.0980885663639728\\
57.125	0.09666	0.147964799077744	0.147964799077744\\
57.125	0.10032	0.190915272258655	0.190915272258655\\
57.125	0.10398	0.2269399859067	0.2269399859067\\
57.125	0.10764	0.256038940021884	0.256038940021884\\
57.125	0.1113	0.278212134604207	0.278212134604207\\
57.125	0.11496	0.293459569653665	0.293459569653665\\
57.125	0.11862	0.301781245170262	0.301781245170262\\
57.125	0.12228	0.303177161153995	0.303177161153995\\
57.125	0.12594	0.297647317604867	0.297647317604867\\
57.125	0.1296	0.285191714522875	0.285191714522875\\
57.125	0.13326	0.265810351908024	0.265810351908024\\
57.125	0.13692	0.239503229760306	0.239503229760306\\
57.125	0.14058	0.206270348079728	0.206270348079728\\
57.125	0.14424	0.166111706866288	0.166111706866288\\
57.125	0.1479	0.119027306119984	0.119027306119984\\
57.125	0.15156	0.0650171458408195	0.0650171458408195\\
57.125	0.15522	0.00408122602878791	0.00408122602878791\\
57.125	0.15888	-0.0637804533161015	-0.0637804533161015\\
57.125	0.16254	-0.138567892193853	-0.138567892193853\\
57.125	0.1662	-0.220281090604472	-0.220281090604472\\
57.125	0.16986	-0.308920048547949	-0.308920048547949\\
57.125	0.17352	-0.404484766024289	-0.404484766024289\\
57.125	0.17718	-0.506975243033496	-0.506975243033496\\
57.125	0.18084	-0.61639147957556	-0.61639147957556\\
57.125	0.1845	-0.732733475650489	-0.732733475650489\\
57.125	0.18816	-0.856001231258281	-0.856001231258281\\
57.125	0.19182	-0.986194746398933	-0.986194746398933\\
57.125	0.19548	-1.12331402107245	-1.12331402107245\\
57.125	0.19914	-1.26735905527883	-1.26735905527883\\
57.125	0.2028	-1.41832984901807	-1.41832984901807\\
57.125	0.20646	-1.57622640229017	-1.57622640229017\\
57.125	0.21012	-1.74104871509514	-1.74104871509514\\
57.125	0.21378	-1.91279678743297	-1.91279678743297\\
57.125	0.21744	-2.09147061930366	-2.09147061930366\\
57.125	0.2211	-2.27707021070721	-2.27707021070721\\
57.125	0.22476	-2.46959556164363	-2.46959556164363\\
57.125	0.22842	-2.66904667211291	-2.66904667211291\\
57.125	0.23208	-2.87542354211505	-2.87542354211505\\
57.125	0.23574	-3.08872617165005	-3.08872617165005\\
57.125	0.2394	-3.30895456071792	-3.30895456071792\\
57.125	0.24306	-3.53610870931865	-3.53610870931865\\
57.125	0.24672	-3.77018861745224	-3.77018861745224\\
57.125	0.25038	-4.0111942851187	-4.0111942851187\\
57.125	0.25404	-4.25912571231801	-4.25912571231801\\
57.125	0.2577	-4.51398289905019	-4.51398289905019\\
57.125	0.26136	-4.77576584531523	-4.77576584531523\\
57.125	0.26502	-5.04447455111313	-5.04447455111313\\
57.125	0.26868	-5.3201090164439	-5.3201090164439\\
57.125	0.27234	-5.60266924130753	-5.60266924130753\\
57.125	0.276	-5.89215522570402	-5.89215522570402\\
57.5	0.093	0.0696482879388882	0.0696482879388882\\
57.5	0.09666	0.12213124517356	0.12213124517356\\
57.5	0.10032	0.167688442875372	0.167688442875372\\
57.5	0.10398	0.20631988104432	0.20631988104432\\
57.5	0.10764	0.238025559680404	0.238025559680404\\
57.5	0.1113	0.262805478783629	0.262805478783629\\
57.5	0.11496	0.280659638353989	0.280659638353989\\
57.5	0.11862	0.291588038391486	0.291588038391486\\
57.5	0.12228	0.295590678896122	0.295590678896122\\
57.5	0.12594	0.292667559867896	0.292667559867896\\
57.5	0.1296	0.282818681306805	0.282818681306805\\
57.5	0.13326	0.266044043212853	0.266044043212853\\
57.5	0.13692	0.242343645586041	0.242343645586041\\
57.5	0.14058	0.211717488426363	0.211717488426363\\
57.5	0.14424	0.174165571733823	0.174165571733823\\
57.5	0.1479	0.129687895508421	0.129687895508421\\
57.5	0.15156	0.0782844597501589	0.0782844597501589\\
57.5	0.15522	0.0199552644590293	0.0199552644590293\\
57.5	0.15888	-0.0452996903649581	-0.0452996903649581\\
57.5	0.16254	-0.117480404721811	-0.117480404721811\\
57.5	0.1662	-0.196586878611525	-0.196586878611525\\
57.5	0.16986	-0.282619112034105	-0.282619112034105\\
57.5	0.17352	-0.375577104989541	-0.375577104989541\\
57.5	0.17718	-0.475460857477846	-0.475460857477846\\
57.5	0.18084	-0.582270369499008	-0.582270369499008\\
57.5	0.1845	-0.69600564105304	-0.69600564105304\\
57.5	0.18816	-0.816666672139924	-0.816666672139924\\
57.5	0.19182	-0.944253462759677	-0.944253462759677\\
57.5	0.19548	-1.07876601291229	-1.07876601291229\\
57.5	0.19914	-1.22020432259777	-1.22020432259777\\
57.5	0.2028	-1.36856839181611	-1.36856839181611\\
57.5	0.20646	-1.52385822056731	-1.52385822056731\\
57.5	0.21012	-1.68607380885138	-1.68607380885138\\
57.5	0.21378	-1.8552151566683	-1.8552151566683\\
57.5	0.21744	-2.03128226401809	-2.03128226401809\\
57.5	0.2211	-2.21427513090075	-2.21427513090075\\
57.5	0.22476	-2.40419375731626	-2.40419375731626\\
57.5	0.22842	-2.60103814326464	-2.60103814326464\\
57.5	0.23208	-2.80480828874588	-2.80480828874588\\
57.5	0.23574	-3.01550419375998	-3.01550419375998\\
57.5	0.2394	-3.23312585830694	-3.23312585830694\\
57.5	0.24306	-3.45767328238677	-3.45767328238677\\
57.5	0.24672	-3.68914646599946	-3.68914646599946\\
57.5	0.25038	-3.92754540914502	-3.92754540914502\\
57.5	0.25404	-4.17287011182343	-4.17287011182343\\
57.5	0.2577	-4.42512057403471	-4.42512057403471\\
57.5	0.26136	-4.68429679577885	-4.68429679577885\\
57.5	0.26502	-4.95039877705585	-4.95039877705585\\
57.5	0.26868	-5.22342651786571	-5.22342651786571\\
57.5	0.27234	-5.50338001820844	-5.50338001820844\\
57.5	0.276	-5.79025927808403	-5.79025927808403\\
57.875	0.093	0.0407319539829802	0.0407319539829802\\
57.875	0.09666	0.095821635738556	0.095821635738556\\
57.875	0.10032	0.143985557961267	0.143985557961267\\
57.875	0.10398	0.185223720651116	0.185223720651116\\
57.875	0.10764	0.219536123808103	0.219536123808103\\
57.875	0.1113	0.24692276743223	0.24692276743223\\
57.875	0.11496	0.26738365152349	0.26738365152349\\
57.875	0.11862	0.280918776081891	0.280918776081891\\
57.875	0.12228	0.287528141107427	0.287528141107427\\
57.875	0.12594	0.287211746600101	0.287211746600101\\
57.875	0.1296	0.279969592559913	0.279969592559913\\
57.875	0.13326	0.265801678986862	0.265801678986862\\
57.875	0.13692	0.24470800588095	0.24470800588095\\
57.875	0.14058	0.216688573242174	0.216688573242174\\
57.875	0.14424	0.181743381070535	0.181743381070535\\
57.875	0.1479	0.139872429366036	0.139872429366036\\
57.875	0.15156	0.0910757181286712	0.0910757181286712\\
57.875	0.15522	0.0353532473584455	0.0353532473584455\\
57.875	0.15888	-0.0272949829446398	-0.0272949829446398\\
57.875	0.16254	-0.0968689727805927	-0.0968689727805927\\
57.875	0.1662	-0.173368722149406	-0.173368722149406\\
57.875	0.16986	-0.256794231051079	-0.256794231051079\\
57.875	0.17352	-0.347145499485618	-0.347145499485618\\
57.875	0.17718	-0.444422527453021	-0.444422527453021\\
57.875	0.18084	-0.548625314953281	-0.548625314953281\\
57.875	0.1845	-0.65975386198641	-0.65975386198641\\
57.875	0.18816	-0.777808168552397	-0.777808168552397\\
57.875	0.19182	-0.902788234651245	-0.902788234651245\\
57.875	0.19548	-1.03469406028296	-1.03469406028296\\
57.875	0.19914	-1.17352564544754	-1.17352564544754\\
57.875	0.2028	-1.31928299014497	-1.31928299014497\\
57.875	0.20646	-1.47196609437528	-1.47196609437528\\
57.875	0.21012	-1.63157495813844	-1.63157495813844\\
57.875	0.21378	-1.79810958143446	-1.79810958143446\\
57.875	0.21744	-1.97156996426335	-1.97156996426335\\
57.875	0.2211	-2.15195610662511	-2.15195610662511\\
57.875	0.22476	-2.33926800851972	-2.33926800851972\\
57.875	0.22842	-2.53350566994719	-2.53350566994719\\
57.875	0.23208	-2.73466909090753	-2.73466909090753\\
57.875	0.23574	-2.94275827140073	-2.94275827140073\\
57.875	0.2394	-3.15777321142679	-3.15777321142679\\
57.875	0.24306	-3.37971391098572	-3.37971391098572\\
57.875	0.24672	-3.6085803700775	-3.6085803700775\\
57.875	0.25038	-3.84437258870216	-3.84437258870216\\
57.875	0.25404	-4.08709056685967	-4.08709056685967\\
57.875	0.2577	-4.33673430455005	-4.33673430455005\\
57.875	0.26136	-4.59330380177329	-4.59330380177329\\
57.875	0.26502	-4.85679905852939	-4.85679905852939\\
57.875	0.26868	-5.12722007481835	-5.12722007481835\\
57.875	0.27234	-5.40456685064017	-5.40456685064017\\
57.875	0.276	-5.68883938599486	-5.68883938599486\\
58.25	0.093	0.0113395644962559	0.0113395644962559\\
58.25	0.09666	0.0690359707727302	0.0690359707727302\\
58.25	0.10032	0.119806617516343	0.119806617516343\\
58.25	0.10398	0.163651504727096	0.163651504727096\\
58.25	0.10764	0.200570632404983	0.200570632404983\\
58.25	0.1113	0.230564000550009	0.230564000550009\\
58.25	0.11496	0.253631609162174	0.253631609162174\\
58.25	0.11862	0.269773458241473	0.269773458241473\\
58.25	0.12228	0.278989547787912	0.278989547787912\\
58.25	0.12594	0.281279877801488	0.281279877801488\\
58.25	0.1296	0.276644448282202	0.276644448282202\\
58.25	0.13326	0.265083259230053	0.265083259230053\\
58.25	0.13692	0.246596310645042	0.246596310645042\\
58.25	0.14058	0.221183602527168	0.221183602527168\\
58.25	0.14424	0.188845134876431	0.188845134876431\\
58.25	0.1479	0.149580907692832	0.149580907692832\\
58.25	0.15156	0.103390920976369	0.103390920976369\\
58.25	0.15522	0.0502751747270453	0.0502751747270453\\
58.25	0.15888	-0.00976633105514146	-0.00976633105514146\\
58.25	0.16254	-0.0767335963701887	-0.0767335963701887\\
58.25	0.1662	-0.1506266212181	-0.1506266212181\\
58.25	0.16986	-0.231445405598876	-0.231445405598876\\
58.25	0.17352	-0.319189949512512	-0.319189949512512\\
58.25	0.17718	-0.413860252959013	-0.413860252959013\\
58.25	0.18084	-0.51545631593837	-0.51545631593837\\
58.25	0.1845	-0.623978138450598	-0.623978138450598\\
58.25	0.18816	-0.739425720495682	-0.739425720495682\\
58.25	0.19182	-0.861799062073631	-0.861799062073631\\
58.25	0.19548	-0.991098163184446	-0.991098163184446\\
58.25	0.19914	-1.12732302382812	-1.12732302382812\\
58.25	0.2028	-1.27047364400465	-1.27047364400465\\
58.25	0.20646	-1.42055002371406	-1.42055002371406\\
58.25	0.21012	-1.57755216295632	-1.57755216295632\\
58.25	0.21378	-1.74148006173144	-1.74148006173144\\
58.25	0.21744	-1.91233372003943	-1.91233372003943\\
58.25	0.2211	-2.09011313788027	-2.09011313788027\\
58.25	0.22476	-2.27481831525399	-2.27481831525399\\
58.25	0.22842	-2.46644925216056	-2.46644925216056\\
58.25	0.23208	-2.6650059486	-2.6650059486\\
58.25	0.23574	-2.8704884045723	-2.8704884045723\\
58.25	0.2394	-3.08289662007746	-3.08289662007746\\
58.25	0.24306	-3.30223059511548	-3.30223059511548\\
58.25	0.24672	-3.52849032968637	-3.52849032968637\\
58.25	0.25038	-3.76167582379012	-3.76167582379012\\
58.25	0.25404	-4.00178707742673	-4.00178707742673\\
58.25	0.2577	-4.24882409059621	-4.24882409059621\\
58.25	0.26136	-4.50278686329855	-4.50278686329855\\
58.25	0.26502	-4.76367539553374	-4.76367539553374\\
58.25	0.26868	-5.0314896873018	-5.0314896873018\\
58.25	0.27234	-5.30622973860273	-5.30622973860273\\
58.25	0.276	-5.58789554943651	-5.58789554943651\\
58.625	0.093	-0.0185288805212926	-0.0185288805212926\\
58.625	0.09666	0.0417742502760836	0.0417742502760836\\
58.625	0.10032	0.0951516215405985	0.0951516215405985\\
58.625	0.10398	0.141603233272252	0.141603233272252\\
58.625	0.10764	0.181129085471041	0.181129085471041\\
58.625	0.1113	0.213729178136969	0.213729178136969\\
58.625	0.11496	0.239403511270034	0.239403511270034\\
58.625	0.11862	0.258152084870236	0.258152084870236\\
58.625	0.12228	0.269974898937576	0.269974898937576\\
58.625	0.12594	0.274871953472053	0.274871953472053\\
58.625	0.1296	0.272843248473668	0.272843248473668\\
58.625	0.13326	0.26388878394242	0.26388878394242\\
58.625	0.13692	0.24800855987831	0.24800855987831\\
58.625	0.14058	0.225202576281339	0.225202576281339\\
58.625	0.14424	0.195470833151502	0.195470833151502\\
58.625	0.1479	0.158813330488806	0.158813330488806\\
58.625	0.15156	0.115230068293244	0.115230068293244\\
58.625	0.15522	0.0647210465648227	0.0647210465648227\\
58.625	0.15888	0.00728626530353793	0.00728626530353793\\
58.625	0.16254	-0.0570742754906108	-0.0570742754906108\\
58.625	0.1662	-0.12836057581762	-0.12836057581762\\
58.625	0.16986	-0.206572635677494	-0.206572635677494\\
58.625	0.17352	-0.291710455070228	-0.291710455070228\\
58.625	0.17718	-0.383774033995826	-0.383774033995826\\
58.625	0.18084	-0.482763372454285	-0.482763372454285\\
58.625	0.1845	-0.588678470445609	-0.588678470445609\\
58.625	0.18816	-0.701519327969795	-0.701519327969795\\
58.625	0.19182	-0.821285945026842	-0.821285945026842\\
58.625	0.19548	-0.947978321616752	-0.947978321616752\\
58.625	0.19914	-1.08159645773952	-1.08159645773952\\
58.625	0.2028	-1.22214035339516	-1.22214035339516\\
58.625	0.20646	-1.36961000858366	-1.36961000858366\\
58.625	0.21012	-1.52400542330502	-1.52400542330502\\
58.625	0.21378	-1.68532659755924	-1.68532659755924\\
58.625	0.21744	-1.85357353134632	-1.85357353134632\\
58.625	0.2211	-2.02874622466627	-2.02874622466627\\
58.625	0.22476	-2.21084467751908	-2.21084467751908\\
58.625	0.22842	-2.39986888990475	-2.39986888990475\\
58.625	0.23208	-2.59581886182329	-2.59581886182329\\
58.625	0.23574	-2.79869459327469	-2.79869459327469\\
58.625	0.2394	-3.00849608425895	-3.00849608425895\\
58.625	0.24306	-3.22522333477607	-3.22522333477607\\
58.625	0.24672	-3.44887634482606	-3.44887634482606\\
58.625	0.25038	-3.67945511440891	-3.67945511440891\\
58.625	0.25404	-3.91695964352462	-3.91695964352462\\
58.625	0.2577	-4.16138993217319	-4.16138993217319\\
58.625	0.26136	-4.41274598035463	-4.41274598035463\\
58.625	0.26502	-4.67102778806892	-4.67102778806892\\
58.625	0.26868	-4.93623535531608	-4.93623535531608\\
58.625	0.27234	-5.2083686820961	-5.2083686820961\\
58.625	0.276	-5.48742776840899	-5.48742776840899\\
59	0.093	-0.0488733810696647	-0.0488733810696647\\
59	0.09666	0.0140364742486154	0.0140364742486154\\
59	0.10032	0.0700205700340306	0.0700205700340306\\
59	0.10398	0.119078906286584	0.119078906286584\\
59	0.10764	0.161211483006277	0.161211483006277\\
59	0.1113	0.196418300193105	0.196418300193105\\
59	0.11496	0.224699357847071	0.224699357847071\\
59	0.11862	0.246054655968176	0.246054655968176\\
59	0.12228	0.260484194556418	0.260484194556418\\
59	0.12594	0.267987973611795	0.267987973611795\\
59	0.1296	0.268565993134313	0.268565993134313\\
59	0.13326	0.262218253123967	0.262218253123967\\
59	0.13692	0.248944753580758	0.248944753580758\\
59	0.14058	0.228745494504686	0.228745494504686\\
59	0.14424	0.201620475895753	0.201620475895753\\
59	0.1479	0.167569697753959	0.167569697753959\\
59	0.15156	0.1265931600793	0.1265931600793\\
59	0.15522	0.0786908628717748	0.0786908628717748\\
59	0.15888	0.0238628061313939	0.0238628061313939\\
59	0.16254	-0.0378910101418528	-0.0378910101418528\\
59	0.1662	-0.106570585947962	-0.106570585947962\\
59	0.16986	-0.182175921286932	-0.182175921286932\\
59	0.17352	-0.264707016158764	-0.264707016158764\\
59	0.17718	-0.354163870563462	-0.354163870563462\\
59	0.18084	-0.450546484501022	-0.450546484501022\\
59	0.1845	-0.553854857971443	-0.553854857971443\\
59	0.18816	-0.664088990974726	-0.664088990974726\\
59	0.19182	-0.781248883510871	-0.781248883510871\\
59	0.19548	-0.905334535579879	-0.905334535579879\\
59	0.19914	-1.03634594718175	-1.03634594718175\\
59	0.2028	-1.17428311831649	-1.17428311831649\\
59	0.20646	-1.31914604898408	-1.31914604898408\\
59	0.21012	-1.47093473918454	-1.47093473918454\\
59	0.21378	-1.62964918891786	-1.62964918891786\\
59	0.21744	-1.79528939818404	-1.79528939818404\\
59	0.2211	-1.96785536698309	-1.96785536698309\\
59	0.22476	-2.147347095315	-2.147347095315\\
59	0.22842	-2.33376458317977	-2.33376458317977\\
59	0.23208	-2.52710783057741	-2.52710783057741\\
59	0.23574	-2.7273768375079	-2.7273768375079\\
59	0.2394	-2.93457160397126	-2.93457160397126\\
59	0.24306	-3.14869212996748	-3.14869212996748\\
59	0.24672	-3.36973841549656	-3.36973841549656\\
59	0.25038	-3.59771046055852	-3.59771046055852\\
59	0.25404	-3.83260826515332	-3.83260826515332\\
59	0.2577	-4.07443182928099	-4.07443182928099\\
59	0.26136	-4.32318115294153	-4.32318115294153\\
59	0.26502	-4.57885623613492	-4.57885623613492\\
59	0.26868	-4.84145707886118	-4.84145707886118\\
59	0.27234	-5.1109836811203	-5.1109836811203\\
59	0.276	-5.38743604291228	-5.38743604291228\\
59.375	0.093	-0.0796939371488548	-0.0796939371488548\\
59.375	0.09666	-0.0141773573096744	-0.0141773573096744\\
59.375	0.10032	0.0444134629966446	0.0444134629966446\\
59.375	0.10398	0.0960785237701005	0.0960785237701005\\
59.375	0.10764	0.140817825010693	0.140817825010693\\
59.375	0.1113	0.178631366718423	0.178631366718423\\
59.375	0.11496	0.209519148893289	0.209519148893289\\
59.375	0.11862	0.233481171535297	0.233481171535297\\
59.375	0.12228	0.250517434644438	0.250517434644438\\
59.375	0.12594	0.26062793822072	0.26062793822072\\
59.375	0.1296	0.263812682264136	0.263812682264136\\
59.375	0.13326	0.260071666774692	0.260071666774692\\
59.375	0.13692	0.249404891752385	0.249404891752385\\
59.375	0.14058	0.231812357197215	0.231812357197215\\
59.375	0.14424	0.207294063109183	0.207294063109183\\
59.375	0.1479	0.175850009488289	0.175850009488289\\
59.375	0.15156	0.137480196334534	0.137480196334534\\
59.375	0.15522	0.0921846236479125	0.0921846236479125\\
59.375	0.15888	0.03996329142843	0.03996329142843\\
59.375	0.16254	-0.0191838003239129	-0.0191838003239129\\
59.375	0.1662	-0.0852566516091215	-0.0852566516091215\\
59.375	0.16986	-0.158255262427192	-0.158255262427192\\
59.375	0.17352	-0.238179632778123	-0.238179632778123\\
59.375	0.17718	-0.325029762661919	-0.325029762661919\\
59.375	0.18084	-0.418805652078575	-0.418805652078575\\
59.375	0.1845	-0.519507301028098	-0.519507301028098\\
59.375	0.18816	-0.627134709510479	-0.627134709510479\\
59.375	0.19182	-0.741687877525722	-0.741687877525722\\
59.375	0.19548	-0.863166805073831	-0.863166805073831\\
59.375	0.19914	-0.991571492154799	-0.991571492154799\\
59.375	0.2028	-1.12690193876863	-1.12690193876863\\
59.375	0.20646	-1.26915814491533	-1.26915814491533\\
59.375	0.21012	-1.41834011059488	-1.41834011059488\\
59.375	0.21378	-1.5744478358073	-1.5744478358073\\
59.375	0.21744	-1.73748132055259	-1.73748132055259\\
59.375	0.2211	-1.90744056483073	-1.90744056483073\\
59.375	0.22476	-2.08432556864174	-2.08432556864174\\
59.375	0.22842	-2.26813633198561	-2.26813633198561\\
59.375	0.23208	-2.45887285486234	-2.45887285486234\\
59.375	0.23574	-2.65653513727193	-2.65653513727193\\
59.375	0.2394	-2.86112317921439	-2.86112317921439\\
59.375	0.24306	-3.07263698068971	-3.07263698068971\\
59.375	0.24672	-3.29107654169789	-3.29107654169789\\
59.375	0.25038	-3.51644186223894	-3.51644186223894\\
59.375	0.25404	-3.74873294231285	-3.74873294231285\\
59.375	0.2577	-3.98794978191962	-3.98794978191962\\
59.375	0.26136	-4.23409238105925	-4.23409238105925\\
59.375	0.26502	-4.48716073973174	-4.48716073973174\\
59.375	0.26868	-4.7471548579371	-4.7471548579371\\
59.375	0.27234	-5.01407473567532	-5.01407473567532\\
59.375	0.276	-5.2879203729464	-5.2879203729464\\
59.75	0.093	-0.110990548758866	-0.110990548758866\\
59.75	0.09666	-0.0428672443987832	-0.0428672443987832\\
59.75	0.10032	0.0183303004284361	0.0183303004284361\\
59.75	0.10398	0.0726020857227923	0.0726020857227923\\
59.75	0.10764	0.119948111484287	0.119948111484287\\
59.75	0.1113	0.160368377712919	0.160368377712919\\
59.75	0.11496	0.193862884408687	0.193862884408687\\
59.75	0.11862	0.220431631571597	0.220431631571597\\
59.75	0.12228	0.24007461920164	0.24007461920164\\
59.75	0.12594	0.252791847298822	0.252791847298822\\
59.75	0.1296	0.258583315863143	0.258583315863143\\
59.75	0.13326	0.257449024894599	0.257449024894599\\
59.75	0.13692	0.249388974393192	0.249388974393192\\
59.75	0.14058	0.234403164358924	0.234403164358924\\
59.75	0.14424	0.212491594791795	0.212491594791795\\
59.75	0.1479	0.183654265691802	0.183654265691802\\
59.75	0.15156	0.147891177058947	0.147891177058947\\
59.75	0.15522	0.105202328893226	0.105202328893226\\
59.75	0.15888	0.0555877211946489	0.0555877211946489\\
59.75	0.16254	-0.000952646036797233	-0.000952646036797233\\
59.75	0.1662	-0.0644187728011021	-0.0644187728011021\\
59.75	0.16986	-0.134810659098272	-0.134810659098272\\
59.75	0.17352	-0.212128304928299	-0.212128304928299\\
59.75	0.17718	-0.296371710291197	-0.296371710291197\\
59.75	0.18084	-0.38754087518695	-0.38754087518695\\
59.75	0.1845	-0.485635799615567	-0.485635799615567\\
59.75	0.18816	-0.59065648357705	-0.59065648357705\\
59.75	0.19182	-0.702602927071391	-0.702602927071391\\
59.75	0.19548	-0.821475130098598	-0.821475130098598\\
59.75	0.19914	-0.947273092658664	-0.947273092658664\\
59.75	0.2028	-1.0799968147516	-1.0799968147516\\
59.75	0.20646	-1.21964629637739	-1.21964629637739\\
59.75	0.21012	-1.36622153753605	-1.36622153753605\\
59.75	0.21378	-1.51972253822756	-1.51972253822756\\
59.75	0.21744	-1.68014929845195	-1.68014929845195\\
59.75	0.2211	-1.84750181820919	-1.84750181820919\\
59.75	0.22476	-2.0217800974993	-2.0217800974993\\
59.75	0.22842	-2.20298413632226	-2.20298413632226\\
59.75	0.23208	-2.39111393467809	-2.39111393467809\\
59.75	0.23574	-2.58616949256679	-2.58616949256679\\
59.75	0.2394	-2.78815080998834	-2.78815080998834\\
59.75	0.24306	-2.99705788694276	-2.99705788694276\\
59.75	0.24672	-3.21289072343004	-3.21289072343004\\
59.75	0.25038	-3.43564931945019	-3.43564931945019\\
59.75	0.25404	-3.66533367500319	-3.66533367500319\\
59.75	0.2577	-3.90194379008906	-3.90194379008906\\
59.75	0.26136	-4.14547966470779	-4.14547966470779\\
59.75	0.26502	-4.39594129885938	-4.39594129885938\\
59.75	0.26868	-4.65332869254384	-4.65332869254384\\
59.75	0.27234	-4.91764184576116	-4.91764184576116\\
59.75	0.276	-5.18888075851134	-5.18888075851134\\
60.125	0.093	-0.142763215899697	-0.142763215899697\\
60.125	0.09666	-0.0720331870187128	-0.0720331870187128\\
60.125	0.10032	-0.00822891767059142	-0.00822891767059142\\
60.125	0.10398	0.048649592144665	0.048649592144665\\
60.125	0.10764	0.0986023424270619	0.0986023424270619\\
60.125	0.1113	0.141629333176596	0.141629333176596\\
60.125	0.11496	0.177730564393266	0.177730564393266\\
60.125	0.11862	0.206906036077074	0.206906036077074\\
60.125	0.12228	0.229155748228019	0.229155748228019\\
60.125	0.12594	0.244479700846104	0.244479700846104\\
60.125	0.1296	0.252877893931324	0.252877893931324\\
60.125	0.13326	0.254350327483684	0.254350327483684\\
60.125	0.13692	0.248897001503178	0.248897001503178\\
60.125	0.14058	0.23651791598981	0.23651791598981\\
60.125	0.14424	0.217213070943584	0.217213070943584\\
60.125	0.1479	0.19098246636449	0.19098246636449\\
60.125	0.15156	0.157826102252539	0.157826102252539\\
60.125	0.15522	0.11774397860772	0.11774397860772\\
60.125	0.15888	0.07073609543004	0.07073609543004\\
60.125	0.16254	0.0168024527195012	0.0168024527195012\\
60.125	0.1662	-0.0440569495239052	-0.0440569495239052\\
60.125	0.16986	-0.111842111300169	-0.111842111300169\\
60.125	0.17352	-0.1865530326093	-0.1865530326093\\
60.125	0.17718	-0.268189713451294	-0.268189713451294\\
60.125	0.18084	-0.356752153826145	-0.356752153826145\\
60.125	0.1845	-0.452240353733865	-0.452240353733865\\
60.125	0.18816	-0.554654313174444	-0.554654313174444\\
60.125	0.19182	-0.663994032147883	-0.663994032147883\\
60.125	0.19548	-0.78025951065419	-0.78025951065419\\
60.125	0.19914	-0.903450748693354	-0.903450748693354\\
60.125	0.2028	-1.03356774626539	-1.03356774626539\\
60.125	0.20646	-1.17061050337028	-1.17061050337028\\
60.125	0.21012	-1.31457902000803	-1.31457902000803\\
60.125	0.21378	-1.46547329617865	-1.46547329617865\\
60.125	0.21744	-1.62329333188213	-1.62329333188213\\
60.125	0.2211	-1.78803912711847	-1.78803912711847\\
60.125	0.22476	-1.95971068188768	-1.95971068188768\\
60.125	0.22842	-2.13830799618974	-2.13830799618974\\
60.125	0.23208	-2.32383107002467	-2.32383107002467\\
60.125	0.23574	-2.51627990339246	-2.51627990339246\\
60.125	0.2394	-2.71565449629312	-2.71565449629312\\
60.125	0.24306	-2.92195484872663	-2.92195484872663\\
60.125	0.24672	-3.13518096069301	-3.13518096069301\\
60.125	0.25038	-3.35533283219225	-3.35533283219225\\
60.125	0.25404	-3.58241046322436	-3.58241046322436\\
60.125	0.2577	-3.81641385378932	-3.81641385378932\\
60.125	0.26136	-4.05734300388715	-4.05734300388715\\
60.125	0.26502	-4.30519791351784	-4.30519791351784\\
60.125	0.26868	-4.5599785826814	-4.5599785826814\\
60.125	0.27234	-4.82168501137782	-4.82168501137782\\
60.125	0.276	-5.0903171996071	-5.0903171996071\\
60.5	0.093	-0.175011938571352	-0.175011938571352\\
60.5	0.09666	-0.101675185169466	-0.101675185169466\\
60.5	0.10032	-0.0352641913004441	-0.0352641913004441\\
60.5	0.10398	0.0242210430357161	0.0242210430357161\\
60.5	0.10764	0.0767805178390133	0.0767805178390133\\
60.5	0.1113	0.122414233109448	0.122414233109448\\
60.5	0.11496	0.16112218884702	0.16112218884702\\
60.5	0.11862	0.19290438505173	0.19290438505173\\
60.5	0.12228	0.217760821723577	0.217760821723577\\
60.5	0.12594	0.235691498862562	0.235691498862562\\
60.5	0.1296	0.246696416468684	0.246696416468684\\
60.5	0.13326	0.250775574541945	0.250775574541945\\
60.5	0.13692	0.247928973082342	0.247928973082342\\
60.5	0.14058	0.238156612089875	0.238156612089875\\
60.5	0.14424	0.22145849156455	0.22145849156455\\
60.5	0.1479	0.197834611506359	0.197834611506359\\
60.5	0.15156	0.167284971915306	0.167284971915306\\
60.5	0.15522	0.129809572791389	0.129809572791389\\
60.5	0.15888	0.0854084141346112	0.0854084141346112\\
60.5	0.16254	0.0340814959449744	0.0340814959449744\\
60.5	0.1662	-0.0241711817775299	-0.0241711817775299\\
60.5	0.16986	-0.0893496190328955	-0.0893496190328955\\
60.5	0.17352	-0.161453815821124	-0.161453815821124\\
60.5	0.17718	-0.240483772142213	-0.240483772142213\\
60.5	0.18084	-0.326439487996165	-0.326439487996165\\
60.5	0.1845	-0.419320963382983	-0.419320963382983\\
60.5	0.18816	-0.51912819830266	-0.51912819830266\\
60.5	0.19182	-0.625861192755201	-0.625861192755201\\
60.5	0.19548	-0.739519946740602	-0.739519946740602\\
60.5	0.19914	-0.860104460258867	-0.860104460258867\\
60.5	0.2028	-0.987614733309996	-0.987614733309996\\
60.5	0.20646	-1.12205076589399	-1.12205076589399\\
60.5	0.21012	-1.26341255801084	-1.26341255801084\\
60.5	0.21378	-1.41170010966055	-1.41170010966055\\
60.5	0.21744	-1.56691342084313	-1.56691342084313\\
60.5	0.2211	-1.72905249155857	-1.72905249155857\\
60.5	0.22476	-1.89811732180688	-1.89811732180688\\
60.5	0.22842	-2.07410791158804	-2.07410791158804\\
60.5	0.23208	-2.25702426090207	-2.25702426090207\\
60.5	0.23574	-2.44686636974896	-2.44686636974896\\
60.5	0.2394	-2.64363423812871	-2.64363423812871\\
60.5	0.24306	-2.84732786604133	-2.84732786604133\\
60.5	0.24672	-3.05794725348681	-3.05794725348681\\
60.5	0.25038	-3.27549240046515	-3.27549240046515\\
60.5	0.25404	-3.49996330697635	-3.49996330697635\\
60.5	0.2577	-3.73135997302042	-3.73135997302042\\
60.5	0.26136	-3.96968239859735	-3.96968239859735\\
60.5	0.26502	-4.21493058370713	-4.21493058370713\\
60.5	0.26868	-4.46710452834979	-4.46710452834979\\
60.5	0.27234	-4.7262042325253	-4.7262042325253\\
60.5	0.276	-4.99222969623368	-4.99222969623368\\
60.875	0.093	-0.207736716773826	-0.207736716773826\\
60.875	0.09666	-0.131793238851038	-0.131793238851038\\
60.875	0.10032	-0.062775520461114	-0.062775520461114\\
60.875	0.10398	-0.000683561604053473	-0.000683561604053473\\
60.875	0.10764	0.0544826377201439	0.0544826377201439\\
60.875	0.1113	0.10272307751148	0.10272307751148\\
60.875	0.11496	0.144037757769955	0.144037757769955\\
60.875	0.11862	0.178426678495567	0.178426678495567\\
60.875	0.12228	0.205889839688314	0.205889839688314\\
60.875	0.12594	0.226427241348199	0.226427241348199\\
60.875	0.1296	0.240038883475226	0.240038883475226\\
60.875	0.13326	0.246724766069386	0.246724766069386\\
60.875	0.13692	0.246484889130685	0.246484889130685\\
60.875	0.14058	0.239319252659122	0.239319252659122\\
60.875	0.14424	0.225227856654694	0.225227856654694\\
60.875	0.1479	0.204210701117407	0.204210701117407\\
60.875	0.15156	0.176267786047256	0.176267786047256\\
60.875	0.15522	0.141399111444241	0.141399111444241\\
60.875	0.15888	0.0996046773083652	0.0996046773083652\\
60.875	0.16254	0.0508844836396269	0.0508844836396269\\
60.875	0.1662	-0.00476146956197532	-0.00476146956197532\\
60.875	0.16986	-0.0673331822964407	-0.0673331822964407\\
60.875	0.17352	-0.136830654563766	-0.136830654563766\\
60.875	0.17718	-0.213253886363955	-0.213253886363955\\
60.875	0.18084	-0.296602877697001	-0.296602877697001\\
60.875	0.1845	-0.386877628562919	-0.386877628562919\\
60.875	0.18816	-0.484078138961692	-0.484078138961692\\
60.875	0.19182	-0.588204408893331	-0.588204408893331\\
60.875	0.19548	-0.699256438357835	-0.699256438357835\\
60.875	0.19914	-0.817234227355199	-0.817234227355199\\
60.875	0.2028	-0.942137775885426	-0.942137775885426\\
60.875	0.20646	-1.07396708394852	-1.07396708394852\\
60.875	0.21012	-1.21272215154447	-1.21272215154447\\
60.875	0.21378	-1.35840297867328	-1.35840297867328\\
60.875	0.21744	-1.51100956533496	-1.51100956533496\\
60.875	0.2211	-1.6705419115295	-1.6705419115295\\
60.875	0.22476	-1.8370000172569	-1.8370000172569\\
60.875	0.22842	-2.01038388251716	-2.01038388251716\\
60.875	0.23208	-2.19069350731029	-2.19069350731029\\
60.875	0.23574	-2.37792889163628	-2.37792889163628\\
60.875	0.2394	-2.57209003549513	-2.57209003549513\\
60.875	0.24306	-2.77317693888684	-2.77317693888684\\
60.875	0.24672	-2.98118960181142	-2.98118960181142\\
60.875	0.25038	-3.19612802426886	-3.19612802426886\\
60.875	0.25404	-3.41799220625916	-3.41799220625916\\
60.875	0.2577	-3.64678214778232	-3.64678214778232\\
60.875	0.26136	-3.88249784883835	-3.88249784883835\\
60.875	0.26502	-4.12513930942724	-4.12513930942724\\
60.875	0.26868	-4.37470652954899	-4.37470652954899\\
60.875	0.27234	-4.6311995092036	-4.6311995092036\\
60.875	0.276	-4.89461824839108	-4.89461824839108\\
61.25	0.093	-0.240937550507121	-0.240937550507121\\
61.25	0.09666	-0.162387348063434	-0.162387348063434\\
61.25	0.10032	-0.0907629051526082	-0.0907629051526082\\
61.25	0.10398	-0.0260642217746438	-0.0260642217746438\\
61.25	0.10764	0.0317087020704574	0.0317087020704574\\
61.25	0.1113	0.0825558663826924	0.0825558663826924\\
61.25	0.11496	0.126477271162069	0.126477271162069\\
61.25	0.11862	0.163472916408583	0.163472916408583\\
61.25	0.12228	0.193542802122232	0.193542802122232\\
61.25	0.12594	0.216686928303019	0.216686928303019\\
61.25	0.1296	0.232905294950946	0.232905294950946\\
61.25	0.13326	0.242197902066007	0.242197902066007\\
61.25	0.13692	0.244564749648208	0.244564749648208\\
61.25	0.14058	0.240005837697545	0.240005837697545\\
61.25	0.14424	0.228521166214021	0.228521166214021\\
61.25	0.1479	0.210110735197634	0.210110735197634\\
61.25	0.15156	0.184774544648383	0.184774544648383\\
61.25	0.15522	0.152512594566272	0.152512594566272\\
61.25	0.15888	0.113324884951298	0.113324884951298\\
61.25	0.16254	0.0672114158034605	0.0672114158034605\\
61.25	0.1662	0.014172187122762	0.014172187122762\\
61.25	0.16986	-0.0457928010907995	-0.0457928010907995\\
61.25	0.17352	-0.112683548837228	-0.112683548837228\\
61.25	0.17718	-0.186500056116516	-0.186500056116516\\
61.25	0.18084	-0.267242322928663	-0.267242322928663\\
61.25	0.1845	-0.354910349273679	-0.354910349273679\\
61.25	0.18816	-0.44950413515155	-0.44950413515155\\
61.25	0.19182	-0.551023680562286	-0.551023680562286\\
61.25	0.19548	-0.659468985505889	-0.659468985505889\\
61.25	0.19914	-0.77484004998235	-0.77484004998235\\
61.25	0.2028	-0.897136873991675	-0.897136873991675\\
61.25	0.20646	-1.02635945753386	-1.02635945753386\\
61.25	0.21012	-1.16250780060891	-1.16250780060891\\
61.25	0.21378	-1.30558190321682	-1.30558190321682\\
61.25	0.21744	-1.4555817653576	-1.4555817653576\\
61.25	0.2211	-1.61250738703124	-1.61250738703124\\
61.25	0.22476	-1.77635876823774	-1.77635876823774\\
61.25	0.22842	-1.9471359089771	-1.9471359089771\\
61.25	0.23208	-2.12483880924932	-2.12483880924932\\
61.25	0.23574	-2.30946746905441	-2.30946746905441\\
61.25	0.2394	-2.50102188839236	-2.50102188839236\\
61.25	0.24306	-2.69950206726317	-2.69950206726317\\
61.25	0.24672	-2.90490800566685	-2.90490800566685\\
61.25	0.25038	-3.11723970360339	-3.11723970360339\\
61.25	0.25404	-3.33649716107279	-3.33649716107279\\
61.25	0.2577	-3.56268037807505	-3.56268037807505\\
61.25	0.26136	-3.79578935461018	-3.79578935461018\\
61.25	0.26502	-4.03582409067816	-4.03582409067816\\
61.25	0.26868	-4.28278458627901	-4.28278458627901\\
61.25	0.27234	-4.53667084141272	-4.53667084141272\\
61.25	0.276	-4.7974828560793	-4.7974828560793\\
61.625	0.093	-0.274614439771237	-0.274614439771237\\
61.625	0.09666	-0.19345751280665	-0.19345751280665\\
61.625	0.10032	-0.119226345374922	-0.119226345374922\\
61.625	0.10398	-0.0519209374760576	-0.0519209374760576\\
61.625	0.10764	0.00845871088994388	0.00845871088994388\\
61.625	0.1113	0.0619125997230827	0.0619125997230827\\
61.625	0.11496	0.108440729023359	0.108440729023359\\
61.625	0.11862	0.148043098790774	0.148043098790774\\
61.625	0.12228	0.180719709025325	0.180719709025325\\
61.625	0.12594	0.206470559727014	0.206470559727014\\
61.625	0.1296	0.225295650895841	0.225295650895841\\
61.625	0.13326	0.237194982531806	0.237194982531806\\
61.625	0.13692	0.242168554634909	0.242168554634909\\
61.625	0.14058	0.240216367205146	0.240216367205146\\
61.625	0.14424	0.231338420242523	0.231338420242523\\
61.625	0.1479	0.215534713747039	0.215534713747039\\
61.625	0.15156	0.192805247718691	0.192805247718691\\
61.625	0.15522	0.163150022157478	0.163150022157478\\
61.625	0.15888	0.126569037063406	0.126569037063406\\
61.625	0.16254	0.0830622924364706	0.0830622924364706\\
61.625	0.1662	0.0326297882766706	0.0326297882766706\\
61.625	0.16986	-0.0247284754159907	-0.0247284754159907\\
61.625	0.17352	-0.0890124986415115	-0.0890124986415115\\
61.625	0.17718	-0.160222281399897	-0.160222281399897\\
61.625	0.18084	-0.238357823691146	-0.238357823691146\\
61.625	0.1845	-0.32341912551526	-0.32341912551526\\
61.625	0.18816	-0.415406186872229	-0.415406186872229\\
61.625	0.19182	-0.514319007762067	-0.514319007762067\\
61.625	0.19548	-0.620157588184767	-0.620157588184767\\
61.625	0.19914	-0.732921928140327	-0.732921928140327\\
61.625	0.2028	-0.85261202762875	-0.85261202762875\\
61.625	0.20646	-0.979227886650039	-0.979227886650039\\
61.625	0.21012	-1.11276950520419	-1.11276950520419\\
61.625	0.21378	-1.2532368832912	-1.2532368832912\\
61.625	0.21744	-1.40063002091107	-1.40063002091107\\
61.625	0.2211	-1.55494891806381	-1.55494891806381\\
61.625	0.22476	-1.71619357474941	-1.71619357474941\\
61.625	0.22842	-1.88436399096786	-1.88436399096786\\
61.625	0.23208	-2.05946016671919	-2.05946016671919\\
61.625	0.23574	-2.24148210200337	-2.24148210200337\\
61.625	0.2394	-2.43042979682042	-2.43042979682042\\
61.625	0.24306	-2.62630325117033	-2.62630325117033\\
61.625	0.24672	-2.82910246505311	-2.82910246505311\\
61.625	0.25038	-3.03882743846875	-3.03882743846875\\
61.625	0.25404	-3.25547817141724	-3.25547817141724\\
61.625	0.2577	-3.4790546638986	-3.4790546638986\\
61.625	0.26136	-3.70955691591283	-3.70955691591283\\
61.625	0.26502	-3.94698492745991	-3.94698492745991\\
61.625	0.26868	-4.19133869853986	-4.19133869853986\\
61.625	0.27234	-4.44261822915267	-4.44261822915267\\
61.625	0.276	-4.70082351929835	-4.70082351929835\\
62	0.093	-0.308767384566174	-0.308767384566174\\
62	0.09666	-0.225003733080685	-0.225003733080685\\
62	0.10032	-0.148165841128057	-0.148165841128057\\
62	0.10398	-0.0782537087082904	-0.0782537087082904\\
62	0.10764	-0.0152673358213886	-0.0152673358213886\\
62	0.1113	0.0407932775326523	0.0407932775326523\\
62	0.11496	0.0899281313538327	0.0899281313538327\\
62	0.11862	0.132137225642147	0.132137225642147\\
62	0.12228	0.167420560397601	0.167420560397601\\
62	0.12594	0.19577813562019	0.19577813562019\\
62	0.1296	0.217209951309919	0.217209951309919\\
62	0.13326	0.231716007466784	0.231716007466784\\
62	0.13692	0.239296304090788	0.239296304090788\\
62	0.14058	0.239950841181931	0.239950841181931\\
62	0.14424	0.233679618740205	0.233679618740205\\
62	0.1479	0.220482636765624	0.220482636765624\\
62	0.15156	0.200359895258178	0.200359895258178\\
62	0.15522	0.173311394217867	0.173311394217867\\
62	0.15888	0.139337133644697	0.139337133644697\\
62	0.16254	0.0984371135386599	0.0984371135386599\\
62	0.1662	0.0506113338997656	0.0506113338997656\\
62	0.16986	-0.00414020527199721	-0.00414020527199721\\
62	0.17352	-0.065817503976616	-0.065817503976616\\
62	0.17718	-0.1344205622141	-0.1344205622141\\
62	0.18084	-0.209949379984447	-0.209949379984447\\
62	0.1845	-0.292403957287657	-0.292403957287657\\
62	0.18816	-0.381784294123729	-0.381784294123729\\
62	0.19182	-0.478090390492662	-0.478090390492662\\
62	0.19548	-0.581322246394462	-0.581322246394462\\
62	0.19914	-0.691479861829121	-0.691479861829121\\
62	0.2028	-0.808563236796642	-0.808563236796642\\
62	0.20646	-0.932572371297026	-0.932572371297026\\
62	0.21012	-1.06350726533027	-1.06350726533027\\
62	0.21378	-1.20136791889638	-1.20136791889638\\
62	0.21744	-1.34615433199535	-1.34615433199535\\
62	0.2211	-1.49786650462719	-1.49786650462719\\
62	0.22476	-1.65650443679189	-1.65650443679189\\
62	0.22842	-1.82206812848944	-1.82206812848944\\
62	0.23208	-1.99455757971987	-1.99455757971987\\
62	0.23574	-2.17397279048315	-2.17397279048315\\
62	0.2394	-2.3603137607793	-2.3603137607793\\
62	0.24306	-2.55358049060831	-2.55358049060831\\
62	0.24672	-2.75377297997018	-2.75377297997018\\
62	0.25038	-2.96089122886492	-2.96089122886492\\
62	0.25404	-3.17493523729251	-3.17493523729251\\
62	0.2577	-3.39590500525297	-3.39590500525297\\
62	0.26136	-3.62380053274629	-3.62380053274629\\
62	0.26502	-3.85862181977248	-3.85862181977248\\
62	0.26868	-4.10036886633152	-4.10036886633152\\
62	0.27234	-4.34904167242343	-4.34904167242343\\
62	0.276	-4.6046402380482	-4.6046402380482\\
62.375	0.093	-0.343396384891932	-0.343396384891932\\
62.375	0.09666	-0.257026008885541	-0.257026008885541\\
62.375	0.10032	-0.177581392412011	-0.177581392412011\\
62.375	0.10398	-0.105062535471344	-0.105062535471344\\
62.375	0.10764	-0.0394694380635401	-0.0394694380635401\\
62.375	0.1113	0.0191978998114029	0.0191978998114029\\
62.375	0.11496	0.0709394781534836	0.0709394781534836\\
62.375	0.11862	0.1157552969627	0.1157552969627\\
62.375	0.12228	0.153645356239056	0.153645356239056\\
62.375	0.12594	0.184609655982546	0.184609655982546\\
62.375	0.1296	0.208648196193177	0.208648196193177\\
62.375	0.13326	0.225760976870943	0.225760976870943\\
62.375	0.13692	0.235947998015849	0.235947998015849\\
62.375	0.14058	0.239209259627891	0.239209259627891\\
62.375	0.14424	0.235544761707071	0.235544761707071\\
62.375	0.1479	0.224954504253388	0.224954504253388\\
62.375	0.15156	0.207438487266844	0.207438487266844\\
62.375	0.15522	0.182996710747435	0.182996710747435\\
62.375	0.15888	0.151629174695164	0.151629174695164\\
62.375	0.16254	0.113335879110032	0.113335879110032\\
62.375	0.1662	0.0681168239920362	0.0681168239920362\\
62.375	0.16986	0.015972009341179	0.015972009341179\\
62.375	0.17352	-0.0430985648425413	-0.0430985648425413\\
62.375	0.17718	-0.109094898559126	-0.109094898559126\\
62.375	0.18084	-0.182016991808569	-0.182016991808569\\
62.375	0.1845	-0.26186484459088	-0.26186484459088\\
62.375	0.18816	-0.348638456906047	-0.348638456906047\\
62.375	0.19182	-0.442337828754081	-0.442337828754081\\
62.375	0.19548	-0.542962960134975	-0.542962960134975\\
62.375	0.19914	-0.650513851048734	-0.650513851048734\\
62.375	0.2028	-0.764990501495356	-0.764990501495356\\
62.375	0.20646	-0.886392911474838	-0.886392911474838\\
62.375	0.21012	-1.01472108098719	-1.01472108098719\\
62.375	0.21378	-1.14997501003239	-1.14997501003239\\
62.375	0.21744	-1.29215469861046	-1.29215469861046\\
62.375	0.2211	-1.4412601467214	-1.4412601467214\\
62.375	0.22476	-1.59729135436519	-1.59729135436519\\
62.375	0.22842	-1.76024832154185	-1.76024832154185\\
62.375	0.23208	-1.93013104825137	-1.93013104825137\\
62.375	0.23574	-2.10693953449375	-2.10693953449375\\
62.375	0.2394	-2.290673780269	-2.290673780269\\
62.375	0.24306	-2.4813337855771	-2.4813337855771\\
62.375	0.24672	-2.67891955041808	-2.67891955041808\\
62.375	0.25038	-2.88343107479191	-2.88343107479191\\
62.375	0.25404	-3.0948683586986	-3.0948683586986\\
62.375	0.2577	-3.31323140213816	-3.31323140213816\\
62.375	0.26136	-3.53852020511059	-3.53852020511059\\
62.375	0.26502	-3.77073476761586	-3.77073476761586\\
62.375	0.26868	-4.00987508965401	-4.00987508965401\\
62.375	0.27234	-4.25594117122501	-4.25594117122501\\
62.375	0.276	-4.50893301232889	-4.50893301232889\\
62.75	0.093	-0.378501440748513	-0.378501440748513\\
62.75	0.09666	-0.289524340221219	-0.289524340221219\\
62.75	0.10032	-0.207472999226789	-0.207472999226789\\
62.75	0.10398	-0.13234741776522	-0.13234741776522\\
62.75	0.10764	-0.0641475958365141	-0.0641475958365141\\
62.75	0.1113	-0.0028735334406691	-0.0028735334406691\\
62.75	0.11496	0.0514747694223101	0.0514747694223101\\
62.75	0.11862	0.0988973127524306	0.0988973127524306\\
62.75	0.12228	0.139394096549684	0.139394096549684\\
62.75	0.12594	0.17296512081408	0.17296512081408\\
62.75	0.1296	0.19961038554561	0.19961038554561\\
62.75	0.13326	0.21932989074428	0.21932989074428\\
62.75	0.13692	0.232123636410084	0.232123636410084\\
62.75	0.14058	0.23799162254303	0.23799162254303\\
62.75	0.14424	0.236933849143108	0.236933849143108\\
62.75	0.1479	0.228950316210329	0.228950316210329\\
62.75	0.15156	0.214041023744685	0.214041023744685\\
62.75	0.15522	0.192205971746179	0.192205971746179\\
62.75	0.15888	0.163445160214808	0.163445160214808\\
62.75	0.16254	0.12775858915058	0.12775858915058\\
62.75	0.1662	0.0851462585534861	0.0851462585534861\\
62.75	0.16986	0.0356081684235274	0.0356081684235274\\
62.75	0.17352	-0.0208556812392926	-0.0208556812392926\\
62.75	0.17718	-0.0842452904349713	-0.0842452904349713\\
62.75	0.18084	-0.154560659163515	-0.154560659163515\\
62.75	0.1845	-0.231801787424926	-0.231801787424926\\
62.75	0.18816	-0.315968675219192	-0.315968675219192\\
62.75	0.19182	-0.407061322546321	-0.407061322546321\\
62.75	0.19548	-0.505079729406319	-0.505079729406319\\
62.75	0.19914	-0.610023895799175	-0.610023895799175\\
62.75	0.2028	-0.721893821724896	-0.721893821724896\\
62.75	0.20646	-0.840689507183475	-0.840689507183475\\
62.75	0.21012	-0.966410952174918	-0.966410952174918\\
62.75	0.21378	-1.09905815669922	-1.09905815669922\\
62.75	0.21744	-1.2386311207564	-1.2386311207564\\
62.75	0.2211	-1.38512984434642	-1.38512984434642\\
62.75	0.22476	-1.53855432746932	-1.53855432746932\\
62.75	0.22842	-1.69890457012507	-1.69890457012507\\
62.75	0.23208	-1.8661805723137	-1.8661805723137\\
62.75	0.23574	-2.04038233403518	-2.04038233403518\\
62.75	0.2394	-2.22150985528952	-2.22150985528952\\
62.75	0.24306	-2.40956313607673	-2.40956313607673\\
62.75	0.24672	-2.6045421763968	-2.6045421763968\\
62.75	0.25038	-2.80644697624973	-2.80644697624973\\
62.75	0.25404	-3.01527753563552	-3.01527753563552\\
62.75	0.2577	-3.23103385455418	-3.23103385455418\\
62.75	0.26136	-3.4537159330057	-3.4537159330057\\
62.75	0.26502	-3.68332377099008	-3.68332377099008\\
62.75	0.26868	-3.91985736850732	-3.91985736850732\\
62.75	0.27234	-4.16331672555743	-4.16331672555743\\
62.75	0.276	-4.4137018421404	-4.4137018421404\\
63.125	0.093	-0.414082552135913	-0.414082552135913\\
63.125	0.09666	-0.322498727087719	-0.322498727087719\\
63.125	0.10032	-0.237840661572385	-0.237840661572385\\
63.125	0.10398	-0.160108355589914	-0.160108355589914\\
63.125	0.10764	-0.0893018091403079	-0.0893018091403079\\
63.125	0.1113	-0.0254210222235627	-0.0254210222235627\\
63.125	0.11496	0.0315340051603186	0.0315340051603186\\
63.125	0.11862	0.0815632730113394	0.0815632730113394\\
63.125	0.12228	0.124666781329497	0.124666781329497\\
63.125	0.12594	0.160844530114793	0.160844530114793\\
63.125	0.1296	0.190096519367226	0.190096519367226\\
63.125	0.13326	0.212422749086795	0.212422749086795\\
63.125	0.13692	0.227823219273505	0.227823219273505\\
63.125	0.14058	0.236297929927349	0.236297929927349\\
63.125	0.14424	0.237846881048331	0.237846881048331\\
63.125	0.1479	0.232470072636453	0.232470072636453\\
63.125	0.15156	0.220167504691709	0.220167504691709\\
63.125	0.15522	0.200939177214103	0.200939177214103\\
63.125	0.15888	0.174785090203637	0.174785090203637\\
63.125	0.16254	0.141705243660306	0.141705243660306\\
63.125	0.1662	0.101699637584113	0.101699637584113\\
63.125	0.16986	0.0547682719750577	0.0547682719750577\\
63.125	0.17352	0.000911146833145082	0.000911146833145082\\
63.125	0.17718	-0.0598717378416378	-0.0598717378416378\\
63.125	0.18084	-0.127580382049279	-0.127580382049279\\
63.125	0.1845	-0.202214785789785	-0.202214785789785\\
63.125	0.18816	-0.283774949063153	-0.283774949063153\\
63.125	0.19182	-0.372260871869383	-0.372260871869383\\
63.125	0.19548	-0.467672554208475	-0.467672554208475\\
63.125	0.19914	-0.570009996080429	-0.570009996080429\\
63.125	0.2028	-0.679273197485248	-0.679273197485248\\
63.125	0.20646	-0.795462158422929	-0.795462158422929\\
63.125	0.21012	-0.918576878893473	-0.918576878893473\\
63.125	0.21378	-1.04861735889688	-1.04861735889688\\
63.125	0.21744	-1.18558359843314	-1.18558359843314\\
63.125	0.2211	-1.32947559750227	-1.32947559750227\\
63.125	0.22476	-1.48029335610427	-1.48029335610427\\
63.125	0.22842	-1.63803687423912	-1.63803687423912\\
63.125	0.23208	-1.80270615190684	-1.80270615190684\\
63.125	0.23574	-1.97430118910742	-1.97430118910742\\
63.125	0.2394	-2.15282198584086	-2.15282198584086\\
63.125	0.24306	-2.33826854210717	-2.33826854210717\\
63.125	0.24672	-2.53064085790633	-2.53064085790633\\
63.125	0.25038	-2.72993893323837	-2.72993893323837\\
63.125	0.25404	-2.93616276810326	-2.93616276810326\\
63.125	0.2577	-3.14931236250101	-3.14931236250101\\
63.125	0.26136	-3.36938771643163	-3.36938771643163\\
63.125	0.26502	-3.59638882989511	-3.59638882989511\\
63.125	0.26868	-3.83031570289145	-3.83031570289145\\
63.125	0.27234	-4.07116833542065	-4.07116833542065\\
63.125	0.276	-4.31894672748272	-4.31894672748272\\
63.5	0.093	-0.450139719054136	-0.450139719054136\\
63.5	0.09666	-0.355949169485038	-0.355949169485038\\
63.5	0.10032	-0.268684379448804	-0.268684379448804\\
63.5	0.10398	-0.188345348945433	-0.188345348945433\\
63.5	0.10764	-0.114932077974923	-0.114932077974923\\
63.5	0.1113	-0.0484445665372752	-0.0484445665372752\\
63.5	0.11496	0.0111171853675063	0.0111171853675063\\
63.5	0.11862	0.0637531777394291	0.0637531777394291\\
63.5	0.12228	0.109463410578487	0.109463410578487\\
63.5	0.12594	0.148247883884685	0.148247883884685\\
63.5	0.1296	0.180106597658019	0.180106597658019\\
63.5	0.13326	0.20503955189849	0.20503955189849\\
63.5	0.13692	0.223046746606098	0.223046746606098\\
63.5	0.14058	0.234128181780846	0.234128181780846\\
63.5	0.14424	0.23828385742273	0.23828385742273\\
63.5	0.1479	0.235513773531752	0.235513773531752\\
63.5	0.15156	0.225817930107912	0.225817930107912\\
63.5	0.15522	0.209196327151208	0.209196327151208\\
63.5	0.15888	0.185648964661643	0.185648964661643\\
63.5	0.16254	0.155175842639215	0.155175842639215\\
63.5	0.1662	0.117776961083924	0.117776961083924\\
63.5	0.16986	0.0734523199957708	0.0734523199957708\\
63.5	0.17352	0.0222019193747549	0.0222019193747549\\
63.5	0.17718	-0.0359742407791241	-0.0359742407791241\\
63.5	0.18084	-0.101076160465867	-0.101076160465867\\
63.5	0.1845	-0.173103839685472	-0.173103839685472\\
63.5	0.18816	-0.252057278437936	-0.252057278437936\\
63.5	0.19182	-0.337936476723264	-0.337936476723264\\
63.5	0.19548	-0.430741434541456	-0.430741434541456\\
63.5	0.19914	-0.530472151892509	-0.530472151892509\\
63.5	0.2028	-0.637128628776425	-0.637128628776425\\
63.5	0.20646	-0.750710865193204	-0.750710865193204\\
63.5	0.21012	-0.871218861142843	-0.871218861142843\\
63.5	0.21378	-0.998652616625348	-0.998652616625348\\
63.5	0.21744	-1.13301213164071	-1.13301213164071\\
63.5	0.2211	-1.27429740618894	-1.27429740618894\\
63.5	0.22476	-1.42250844027003	-1.42250844027003\\
63.5	0.22842	-1.57764523388398	-1.57764523388398\\
63.5	0.23208	-1.7397077870308	-1.7397077870308\\
63.5	0.23574	-1.90869609971049	-1.90869609971049\\
63.5	0.2394	-2.08461017192302	-2.08461017192302\\
63.5	0.24306	-2.26745000366843	-2.26745000366843\\
63.5	0.24672	-2.45721559494669	-2.45721559494669\\
63.5	0.25038	-2.65390694575783	-2.65390694575783\\
63.5	0.25404	-2.85752405610181	-2.85752405610181\\
63.5	0.2577	-3.06806692597867	-3.06806692597867\\
63.5	0.26136	-3.28553555538839	-3.28553555538839\\
63.5	0.26502	-3.50992994433096	-3.50992994433096\\
63.5	0.26868	-3.7412500928064	-3.7412500928064\\
63.5	0.27234	-3.9794960008147	-3.9794960008147\\
63.5	0.276	-4.22466766835587	-4.22466766835587\\
63.875	0.093	-0.486672941503178	-0.486672941503178\\
63.875	0.09666	-0.38987566741318	-0.38987566741318\\
63.875	0.10032	-0.300004152856043	-0.300004152856043\\
63.875	0.10398	-0.217058397831772	-0.217058397831772\\
63.875	0.10764	-0.14103840234036	-0.14103840234036\\
63.875	0.1113	-0.0719441663818121	-0.0719441663818121\\
63.875	0.11496	-0.00977568995612677	-0.00977568995612677\\
63.875	0.11862	0.0454670269366964	0.0454670269366964\\
63.875	0.12228	0.0937839842966564	0.0937839842966564\\
63.875	0.12594	0.135175182123755	0.135175182123755\\
63.875	0.1296	0.16964062041799	0.16964062041799\\
63.875	0.13326	0.197180299179365	0.197180299179365\\
63.875	0.13692	0.217794218407872	0.217794218407872\\
63.875	0.14058	0.231482378103522	0.231482378103522\\
63.875	0.14424	0.238244778266308	0.238244778266308\\
63.875	0.1479	0.238081418896232	0.238081418896232\\
63.875	0.15156	0.23099229999329	0.23099229999329\\
63.875	0.15522	0.216977421557488	0.216977421557488\\
63.875	0.15888	0.196036783588825	0.196036783588825\\
63.875	0.16254	0.1681703860873	0.1681703860873\\
63.875	0.1662	0.133378229052907	0.133378229052907\\
63.875	0.16986	0.0916603124856561	0.0916603124856561\\
63.875	0.17352	0.0430166363855422	0.0430166363855422\\
63.875	0.17718	-0.0125527992474339	-0.0125527992474339\\
63.875	0.18084	-0.0750479944132714	-0.0750479944132714\\
63.875	0.1845	-0.14446894911198	-0.14446894911198\\
63.875	0.18816	-0.22081566334354	-0.22081566334354\\
63.875	0.19182	-0.304088137107966	-0.304088137107966\\
63.875	0.19548	-0.394286370405258	-0.394286370405258\\
63.875	0.19914	-0.491410363235412	-0.491410363235412\\
63.875	0.2028	-0.595460115598422	-0.595460115598422\\
63.875	0.20646	-0.706435627494303	-0.706435627494303\\
63.875	0.21012	-0.82433689892304	-0.82433689892304\\
63.875	0.21378	-0.949163929884643	-0.949163929884643\\
63.875	0.21744	-1.08091672037911	-1.08091672037911\\
63.875	0.2211	-1.21959527040644	-1.21959527040644\\
63.875	0.22476	-1.36519957996662	-1.36519957996662\\
63.875	0.22842	-1.51772964905968	-1.51772964905968\\
63.875	0.23208	-1.67718547768559	-1.67718547768559\\
63.875	0.23574	-1.84356706584437	-1.84356706584437\\
63.875	0.2394	-2.01687441353601	-2.01687441353601\\
63.875	0.24306	-2.19710752076051	-2.19710752076051\\
63.875	0.24672	-2.38426638751788	-2.38426638751788\\
63.875	0.25038	-2.5783510138081	-2.5783510138081\\
63.875	0.25404	-2.77936139963119	-2.77936139963119\\
63.875	0.2577	-2.98729754498715	-2.98729754498715\\
63.875	0.26136	-3.20215944987596	-3.20215944987596\\
63.875	0.26502	-3.42394711429764	-3.42394711429764\\
63.875	0.26868	-3.65266053825217	-3.65266053825217\\
63.875	0.27234	-3.88829972173958	-3.88829972173958\\
63.875	0.276	-4.13086466475984	-4.13086466475984\\
64.25	0.093	-0.523682219483041	-0.523682219483041\\
64.25	0.09666	-0.424278220872141	-0.424278220872141\\
64.25	0.10032	-0.331799981794106	-0.331799981794106\\
64.25	0.10398	-0.246247502248931	-0.246247502248931\\
64.25	0.10764	-0.167620782236618	-0.167620782236618\\
64.25	0.1113	-0.0959198217571671	-0.0959198217571671\\
64.25	0.11496	-0.0311446208105814	-0.0311446208105814\\
64.25	0.11862	0.0267048206031419	0.0267048206031419\\
64.25	0.12228	0.077628502484004	0.077628502484004\\
64.25	0.12594	0.121626424832004	0.121626424832004\\
64.25	0.1296	0.158698587647142	0.158698587647142\\
64.25	0.13326	0.188844990929414	0.188844990929414\\
64.25	0.13692	0.212065634678828	0.212065634678828\\
64.25	0.14058	0.228360518895379	0.228360518895379\\
64.25	0.14424	0.237729643579063	0.237729643579063\\
64.25	0.1479	0.24017300872989	0.24017300872989\\
64.25	0.15156	0.235690614347851	0.235690614347851\\
64.25	0.15522	0.224282460432951	0.224282460432951\\
64.25	0.15888	0.20594854698519	0.20594854698519\\
64.25	0.16254	0.180688874004563	0.180688874004563\\
64.25	0.1662	0.148503441491076	0.148503441491076\\
64.25	0.16986	0.109392249444721	0.109392249444721\\
64.25	0.17352	0.0633552978655114	0.0633552978655114\\
64.25	0.17718	0.0103925867534365	0.0103925867534365\\
64.25	0.18084	-0.0494958838915025	-0.0494958838915025\\
64.25	0.1845	-0.116310114069304	-0.116310114069304\\
64.25	0.18816	-0.190050103779967	-0.190050103779967\\
64.25	0.19182	-0.270715853023493	-0.270715853023493\\
64.25	0.19548	-0.358307361799879	-0.358307361799879\\
64.25	0.19914	-0.452824630109131	-0.452824630109131\\
64.25	0.2028	-0.554267657951243	-0.554267657951243\\
64.25	0.20646	-0.662636445326221	-0.662636445326221\\
64.25	0.21012	-0.77793099223406	-0.77793099223406\\
64.25	0.21378	-0.900151298674754	-0.900151298674754\\
64.25	0.21744	-1.02929736464832	-1.02929736464832\\
64.25	0.2211	-1.16536919015475	-1.16536919015475\\
64.25	0.22476	-1.30836677519403	-1.30836677519403\\
64.25	0.22842	-1.45829011976619	-1.45829011976619\\
64.25	0.23208	-1.6151392238712	-1.6151392238712\\
64.25	0.23574	-1.77891408750907	-1.77891408750907\\
64.25	0.2394	-1.94961471067981	-1.94961471067981\\
64.25	0.24306	-2.12724109338341	-2.12724109338341\\
64.25	0.24672	-2.31179323561988	-2.31179323561988\\
64.25	0.25038	-2.5032711373892	-2.5032711373892\\
64.25	0.25404	-2.70167479869139	-2.70167479869139\\
64.25	0.2577	-2.90700421952644	-2.90700421952644\\
64.25	0.26136	-3.11925939989436	-3.11925939989436\\
64.25	0.26502	-3.33844033979513	-3.33844033979513\\
64.25	0.26868	-3.56454703922877	-3.56454703922877\\
64.25	0.27234	-3.79757949819527	-3.79757949819527\\
64.25	0.276	-4.03753771669463	-4.03753771669463\\
64.625	0.093	-0.561167552993725	-0.561167552993725\\
64.625	0.09666	-0.459156829861923	-0.459156829861923\\
64.625	0.10032	-0.364071866262986	-0.364071866262986\\
64.625	0.10398	-0.275912662196911	-0.275912662196911\\
64.625	0.10764	-0.194679217663698	-0.194679217663698\\
64.625	0.1113	-0.120371532663345	-0.120371532663345\\
64.625	0.11496	-0.0529896071958569	-0.0529896071958569\\
64.625	0.11862	0.00746655873876856	0.00746655873876856\\
64.625	0.12228	0.0609969651405327	0.0609969651405327\\
64.625	0.12594	0.107601612009435	0.107601612009435\\
64.625	0.1296	0.147280499345473	0.147280499345473\\
64.625	0.13326	0.180033627148648	0.180033627148648\\
64.625	0.13692	0.205860995418963	0.205860995418963\\
64.625	0.14058	0.224762604156413	0.224762604156413\\
64.625	0.14424	0.236738453361004	0.236738453361004\\
64.625	0.1479	0.241788543032728	0.241788543032728\\
64.625	0.15156	0.239912873171589	0.239912873171589\\
64.625	0.15522	0.231111443777593	0.231111443777593\\
64.625	0.15888	0.21538425485073	0.21538425485073\\
64.625	0.16254	0.192731306391007	0.192731306391007\\
64.625	0.1662	0.16315259839842	0.16315259839842\\
64.625	0.16986	0.126648130872973	0.126648130872973\\
64.625	0.17352	0.0832179038146599	0.0832179038146599\\
64.625	0.17718	0.032861917223487	0.032861917223487\\
64.625	0.18084	-0.0244198289005499	-0.0244198289005499\\
64.625	0.1845	-0.0886273345574491	-0.0886273345574491\\
64.625	0.18816	-0.15976059974721	-0.15976059974721\\
64.625	0.19182	-0.237819624469836	-0.237819624469836\\
64.625	0.19548	-0.32280440872532	-0.32280440872532\\
64.625	0.19914	-0.41471495251367	-0.41471495251367\\
64.625	0.2028	-0.513551255834884	-0.513551255834884\\
64.625	0.20646	-0.61931331868896	-0.61931331868896\\
64.625	0.21012	-0.732001141075893	-0.732001141075893\\
64.625	0.21378	-0.851614722995691	-0.851614722995691\\
64.625	0.21744	-0.978154064448358	-0.978154064448358\\
64.625	0.2211	-1.11161916543388	-1.11161916543388\\
64.625	0.22476	-1.25201002595227	-1.25201002595227\\
64.625	0.22842	-1.39932664600351	-1.39932664600351\\
64.625	0.23208	-1.55356902558762	-1.55356902558762\\
64.625	0.23574	-1.7147371647046	-1.7147371647046\\
64.625	0.2394	-1.88283106335444	-1.88283106335444\\
64.625	0.24306	-2.05785072153713	-2.05785072153713\\
64.625	0.24672	-2.23979613925269	-2.23979613925269\\
64.625	0.25038	-2.42866731650112	-2.42866731650112\\
64.625	0.25404	-2.62446425328241	-2.62446425328241\\
64.625	0.2577	-2.82718694959656	-2.82718694959656\\
64.625	0.26136	-3.03683540544357	-3.03683540544357\\
64.625	0.26502	-3.25340962082344	-3.25340962082344\\
64.625	0.26868	-3.47690959573618	-3.47690959573618\\
64.625	0.27234	-3.70733533018178	-3.70733533018178\\
64.625	0.276	-3.94468682416024	-3.94468682416024\\
65	0.093	-0.59912894203523	-0.59912894203523\\
65	0.09666	-0.49451149438253	-0.49451149438253\\
65	0.10032	-0.396819806262687	-0.396819806262687\\
65	0.10398	-0.306053877675709	-0.306053877675709\\
65	0.10764	-0.222213708621595	-0.222213708621595\\
65	0.1113	-0.145299299100343	-0.145299299100343\\
65	0.11496	-0.0753106491119531	-0.0753106491119531\\
65	0.11862	-0.0122477586564256	-0.0122477586564256\\
65	0.12228	0.0438893722662388	0.0438893722662388\\
65	0.12594	0.0931007436560414	0.0931007436560414\\
65	0.1296	0.135386355512983	0.135386355512983\\
65	0.13326	0.170746207837059	0.170746207837059\\
65	0.13692	0.199180300628274	0.199180300628274\\
65	0.14058	0.220688633886629	0.220688633886629\\
65	0.14424	0.235271207612117	0.235271207612117\\
65	0.1479	0.242928021804745	0.242928021804745\\
65	0.15156	0.243659076464513	0.243659076464513\\
65	0.15522	0.237464371591414	0.237464371591414\\
65	0.15888	0.224343907185453	0.224343907185453\\
65	0.16254	0.20429768324663	0.20429768324663\\
65	0.1662	0.177325699774947	0.177325699774947\\
65	0.16986	0.143427956770397	0.143427956770397\\
65	0.17352	0.102604454232988	0.102604454232988\\
65	0.17718	0.0548551921627158	0.0548551921627158\\
65	0.18084	0.000180170559580972	0.000180170559580972\\
65	0.1845	-0.0614206105764179	-0.0614206105764179\\
65	0.18816	-0.129947151245275	-0.129947151245275\\
65	0.19182	-0.205399451446999	-0.205399451446999\\
65	0.19548	-0.287777511181584	-0.287777511181584\\
65	0.19914	-0.377081330449032	-0.377081330449032\\
65	0.2028	-0.47331090924934	-0.47331090924934\\
65	0.20646	-0.576466247582514	-0.576466247582514\\
65	0.21012	-0.686547345448552	-0.686547345448552\\
65	0.21378	-0.803554202847449	-0.803554202847449\\
65	0.21744	-0.927486819779206	-0.927486819779206\\
65	0.2211	-1.05834519624383	-1.05834519624383\\
65	0.22476	-1.19612933224132	-1.19612933224132\\
65	0.22842	-1.34083922777166	-1.34083922777166\\
65	0.23208	-1.49247488283487	-1.49247488283487\\
65	0.23574	-1.65103629743095	-1.65103629743095\\
65	0.2394	-1.81652347155988	-1.81652347155988\\
65	0.24306	-1.98893640522168	-1.98893640522168\\
65	0.24672	-2.16827509841634	-2.16827509841634\\
65	0.25038	-2.35453955114387	-2.35453955114387\\
65	0.25404	-2.54772976340425	-2.54772976340425\\
65	0.2577	-2.74784573519749	-2.74784573519749\\
65	0.26136	-2.95488746652361	-2.95488746652361\\
65	0.26502	-3.16885495738257	-3.16885495738257\\
65	0.26868	-3.38974820777441	-3.38974820777441\\
65	0.27234	-3.61756721769911	-3.61756721769911\\
65	0.276	-3.85231198715666	-3.85231198715666\\
65.375	0.093	-0.637566386607558	-0.637566386607558\\
65.375	0.09666	-0.530342214433956	-0.530342214433956\\
65.375	0.10032	-0.430043801793213	-0.430043801793213\\
65.375	0.10398	-0.336671148685333	-0.336671148685333\\
65.375	0.10764	-0.250224255110318	-0.250224255110318\\
65.375	0.1113	-0.170703121068164	-0.170703121068164\\
65.375	0.11496	-0.0981077465588709	-0.0981077465588709\\
65.375	0.11862	-0.0324381315824431	-0.0324381315824431\\
65.375	0.12228	0.0263057238611233	0.0263057238611233\\
65.375	0.12594	0.0781238197718261	0.0781238197718261\\
65.375	0.1296	0.123016156149668	0.123016156149668\\
65.375	0.13326	0.160982732994646	0.160982732994646\\
65.375	0.13692	0.192023550306763	0.192023550306763\\
65.375	0.14058	0.216138608086021	0.216138608086021\\
65.375	0.14424	0.23332790633241	0.23332790633241\\
65.375	0.1479	0.24359144504594	0.24359144504594\\
65.375	0.15156	0.246929224226605	0.246929224226605\\
65.375	0.15522	0.24334124387441	0.24334124387441\\
65.375	0.15888	0.232827503989351	0.232827503989351\\
65.375	0.16254	0.215388004571428	0.215388004571428\\
65.375	0.1662	0.191022745620645	0.191022745620645\\
65.375	0.16986	0.159731727137001	0.159731727137001\\
65.375	0.17352	0.121514949120494	0.121514949120494\\
65.375	0.17718	0.0763724115711213	0.0763724115711213\\
65.375	0.18084	0.0243041144888885	0.0243041144888885\\
65.375	0.1845	-0.0346899421262101	-0.0346899421262101\\
65.375	0.18816	-0.100609758274167	-0.100609758274167\\
65.375	0.19182	-0.173455333954989	-0.173455333954989\\
65.375	0.19548	-0.253226669168672	-0.253226669168672\\
65.375	0.19914	-0.339923763915218	-0.339923763915218\\
65.375	0.2028	-0.433546618194628	-0.433546618194628\\
65.375	0.20646	-0.5340952320069	-0.5340952320069\\
65.375	0.21012	-0.641569605352032	-0.641569605352032\\
65.375	0.21378	-0.755969738230027	-0.755969738230027\\
65.375	0.21744	-0.877295630640885	-0.877295630640885\\
65.375	0.2211	-1.00554728258461	-1.00554728258461\\
65.375	0.22476	-1.14072469406119	-1.14072469406119\\
65.375	0.22842	-1.28282786507064	-1.28282786507064\\
65.375	0.23208	-1.43185679561294	-1.43185679561294\\
65.375	0.23574	-1.58781148568812	-1.58781148568812\\
65.375	0.2394	-1.75069193529615	-1.75069193529615\\
65.375	0.24306	-1.92049814443705	-1.92049814443705\\
65.375	0.24672	-2.0972301131108	-2.0972301131108\\
65.375	0.25038	-2.28088784131743	-2.28088784131743\\
65.375	0.25404	-2.47147132905691	-2.47147132905691\\
65.375	0.2577	-2.66898057632926	-2.66898057632926\\
65.375	0.26136	-2.87341558313447	-2.87341558313447\\
65.375	0.26502	-3.08477634947254	-3.08477634947254\\
65.375	0.26868	-3.30306287534347	-3.30306287534347\\
65.375	0.27234	-3.52827516074727	-3.52827516074727\\
65.375	0.276	-3.76041320568392	-3.76041320568392\\
65.75	0.093	-0.676479886710705	-0.676479886710705\\
65.75	0.09666	-0.5666489900162	-0.5666489900162\\
65.75	0.10032	-0.463743852854557	-0.463743852854557\\
65.75	0.10398	-0.367764475225775	-0.367764475225775\\
65.75	0.10764	-0.278710857129858	-0.278710857129858\\
65.75	0.1113	-0.196582998566802	-0.196582998566802\\
65.75	0.11496	-0.121380899536609	-0.121380899536609\\
65.75	0.11862	-0.0531045600392788	-0.0531045600392788\\
65.75	0.12228	0.0082460199251897	0.0082460199251897\\
65.75	0.12594	0.0626708403567928	0.0626708403567928\\
65.75	0.1296	0.110169901255537	0.110169901255537\\
65.75	0.13326	0.150743202621417	0.150743202621417\\
65.75	0.13692	0.184390744454435	0.184390744454435\\
65.75	0.14058	0.211112526754592	0.211112526754592\\
65.75	0.14424	0.230908549521883	0.230908549521883\\
65.75	0.1479	0.243778812756315	0.243778812756315\\
65.75	0.15156	0.249723316457882	0.249723316457882\\
65.75	0.15522	0.248742060626589	0.248742060626589\\
65.75	0.15888	0.240835045262433	0.240835045262433\\
65.75	0.16254	0.226002270365412	0.226002270365412\\
65.75	0.1662	0.204243735935531	0.204243735935531\\
65.75	0.16986	0.175559441972785	0.175559441972785\\
65.75	0.17352	0.13994938847718	0.13994938847718\\
65.75	0.17718	0.0974135754487087	0.0974135754487087\\
65.75	0.18084	0.0479520028873779	0.0479520028873779\\
65.75	0.1845	-0.00843532920681689	-0.00843532920681689\\
65.75	0.18816	-0.0717484208338703	-0.0717484208338703\\
65.75	0.19182	-0.14198727199379	-0.14198727199379\\
65.75	0.19548	-0.219151882686578	-0.219151882686578\\
65.75	0.19914	-0.303242252912222	-0.303242252912222\\
65.75	0.2028	-0.39425838267073	-0.39425838267073\\
65.75	0.20646	-0.492200271962099	-0.492200271962099\\
65.75	0.21012	-0.59706792078633	-0.59706792078633\\
65.75	0.21378	-0.708861329143422	-0.708861329143422\\
65.75	0.21744	-0.827580497033383	-0.827580497033383\\
65.75	0.2211	-0.953225424456202	-0.953225424456202\\
65.75	0.22476	-1.08579611141189	-1.08579611141189\\
65.75	0.22842	-1.22529255790043	-1.22529255790043\\
65.75	0.23208	-1.37171476392184	-1.37171476392184\\
65.75	0.23574	-1.52506272947611	-1.52506272947611\\
65.75	0.2394	-1.68533645456324	-1.68533645456324\\
65.75	0.24306	-1.85253593918323	-1.85253593918323\\
65.75	0.24672	-2.02666118333609	-2.02666118333609\\
65.75	0.25038	-2.20771218702181	-2.20771218702181\\
65.75	0.25404	-2.39568895024039	-2.39568895024039\\
65.75	0.2577	-2.59059147299184	-2.59059147299184\\
65.75	0.26136	-2.79241975527614	-2.79241975527614\\
65.75	0.26502	-3.00117379709332	-3.00117379709332\\
65.75	0.26868	-3.21685359844334	-3.21685359844334\\
65.75	0.27234	-3.43945915932624	-3.43945915932624\\
65.75	0.276	-3.668990479742	-3.668990479742\\
66.125	0.093	-0.715869442344674	-0.715869442344674\\
66.125	0.09666	-0.603431821129267	-0.603431821129267\\
66.125	0.10032	-0.497919959446724	-0.497919959446724\\
66.125	0.10398	-0.39933385729704	-0.39933385729704\\
66.125	0.10764	-0.307673514680221	-0.307673514680221\\
66.125	0.1113	-0.222938931596265	-0.222938931596265\\
66.125	0.11496	-0.145130108045169	-0.145130108045169\\
66.125	0.11862	-0.0742470440269387	-0.0742470440269387\\
66.125	0.12228	-0.0102897395415682	-0.0102897395415682\\
66.125	0.12594	0.0467418054109388	0.0467418054109388\\
66.125	0.1296	0.096847590830583	0.096847590830583\\
66.125	0.13326	0.140027616717365	0.140027616717365\\
66.125	0.13692	0.176281883071284	0.176281883071284\\
66.125	0.14058	0.205610389892344	0.205610389892344\\
66.125	0.14424	0.228013137180534	0.228013137180534\\
66.125	0.1479	0.243490124935867	0.243490124935867\\
66.125	0.15156	0.252041353158337	0.252041353158337\\
66.125	0.15522	0.253666821847946	0.253666821847946\\
66.125	0.15888	0.24836653100469	0.24836653100469\\
66.125	0.16254	0.23614048062857	0.23614048062857\\
66.125	0.1662	0.216988670719591	0.216988670719591\\
66.125	0.16986	0.190911101277747	0.190911101277747\\
66.125	0.17352	0.157907772303042	0.157907772303042\\
66.125	0.17718	0.117978683795474	0.117978683795474\\
66.125	0.18084	0.0711238357550421	0.0711238357550421\\
66.125	0.1845	0.0173432281817458	0.0173432281817458\\
66.125	0.18816	-0.0433631389244056	-0.0433631389244056\\
66.125	0.19182	-0.110995265563423	-0.110995265563423\\
66.125	0.19548	-0.185553151735306	-0.185553151735306\\
66.125	0.19914	-0.267036797440051	-0.267036797440051\\
66.125	0.2028	-0.355446202677657	-0.355446202677657\\
66.125	0.20646	-0.450781367448124	-0.450781367448124\\
66.125	0.21012	-0.553042291751453	-0.553042291751453\\
66.125	0.21378	-0.662228975587643	-0.662228975587643\\
66.125	0.21744	-0.778341418956705	-0.778341418956705\\
66.125	0.2211	-0.901379621858622	-0.901379621858622\\
66.125	0.22476	-1.0313435842934	-1.0313435842934\\
66.125	0.22842	-1.16823330626104	-1.16823330626104\\
66.125	0.23208	-1.31204878776155	-1.31204878776155\\
66.125	0.23574	-1.46279002879492	-1.46279002879492\\
66.125	0.2394	-1.62045702936115	-1.62045702936115\\
66.125	0.24306	-1.78504978946024	-1.78504978946024\\
66.125	0.24672	-1.9565683090922	-1.9565683090922\\
66.125	0.25038	-2.13501258825702	-2.13501258825702\\
66.125	0.25404	-2.3203826269547	-2.3203826269547\\
66.125	0.2577	-2.51267842518524	-2.51267842518524\\
66.125	0.26136	-2.71189998294865	-2.71189998294865\\
66.125	0.26502	-2.91804730024491	-2.91804730024491\\
66.125	0.26868	-3.13112037707404	-3.13112037707404\\
66.125	0.27234	-3.35111921343604	-3.35111921343604\\
66.125	0.276	-3.57804380933089	-3.57804380933089\\
66.5	0.093	-0.755735053509463	-0.755735053509463\\
66.5	0.09666	-0.640690707773156	-0.640690707773156\\
66.5	0.10032	-0.53257212156971	-0.53257212156971\\
66.5	0.10398	-0.431379294899126	-0.431379294899126\\
66.5	0.10764	-0.337112227761405	-0.337112227761405\\
66.5	0.1113	-0.249770920156547	-0.249770920156547\\
66.5	0.11496	-0.169355372084552	-0.169355372084552\\
66.5	0.11862	-0.0958655835454185	-0.0958655835454185\\
66.5	0.12228	-0.0293015545391477	-0.0293015545391477\\
66.5	0.12594	0.0303367149342613	0.0303367149342613\\
66.5	0.1296	0.0830492248748076	0.0830492248748076\\
66.5	0.13326	0.12883597528249	0.12883597528249\\
66.5	0.13692	0.167696966157313	0.167696966157313\\
66.5	0.14058	0.199632197499268	0.199632197499268\\
66.5	0.14424	0.224641669308365	0.224641669308365\\
66.5	0.1479	0.242725381584601	0.242725381584601\\
66.5	0.15156	0.253883334327969	0.253883334327969\\
66.5	0.15522	0.258115527538478	0.258115527538478\\
66.5	0.15888	0.255421961216127	0.255421961216127\\
66.5	0.16254	0.245802635360907	0.245802635360907\\
66.5	0.1662	0.229257549972829	0.229257549972829\\
66.5	0.16986	0.205786705051889	0.205786705051889\\
66.5	0.17352	0.175390100598084	0.175390100598084\\
66.5	0.17718	0.138067736611415	0.138067736611415\\
66.5	0.18084	0.0938196130918847	0.0938196130918847\\
66.5	0.1845	0.042645730039494	0.042645730039494\\
66.5	0.18816	-0.0154539125457589	-0.0154539125457589\\
66.5	0.19182	-0.0804793146638776	-0.0804793146638776\\
66.5	0.19548	-0.152430476314855	-0.152430476314855\\
66.5	0.19914	-0.231307397498695	-0.231307397498695\\
66.5	0.2028	-0.317110078215402	-0.317110078215402\\
66.5	0.20646	-0.409838518464968	-0.409838518464968\\
66.5	0.21012	-0.509492718247397	-0.509492718247397\\
66.5	0.21378	-0.616072677562689	-0.616072677562689\\
66.5	0.21744	-0.729578396410842	-0.729578396410842\\
66.5	0.2211	-0.85000987479186	-0.85000987479186\\
66.5	0.22476	-0.977367112705741	-0.977367112705741\\
66.5	0.22842	-1.11165011015248	-1.11165011015248\\
66.5	0.23208	-1.25285886713209	-1.25285886713209\\
66.5	0.23574	-1.40099338364455	-1.40099338364455\\
66.5	0.2394	-1.55605365968988	-1.55605365968988\\
66.5	0.24306	-1.71803969526807	-1.71803969526807\\
66.5	0.24672	-1.88695149037913	-1.88695149037913\\
66.5	0.25038	-2.06278904502305	-2.06278904502305\\
66.5	0.25404	-2.24555235919982	-2.24555235919982\\
66.5	0.2577	-2.43524143290947	-2.43524143290947\\
66.5	0.26136	-2.63185626615197	-2.63185626615197\\
66.5	0.26502	-2.83539685892734	-2.83539685892734\\
66.5	0.26868	-3.04586321123557	-3.04586321123557\\
66.5	0.27234	-3.26325532307665	-3.26325532307665\\
66.5	0.276	-3.48757319445061	-3.48757319445061\\
66.875	0.093	-0.796076720205075	-0.796076720205075\\
66.875	0.09666	-0.678425649947864	-0.678425649947864\\
66.875	0.10032	-0.567700339223517	-0.567700339223517\\
66.875	0.10398	-0.463900788032034	-0.463900788032034\\
66.875	0.10764	-0.367026996373411	-0.367026996373411\\
66.875	0.1113	-0.277078964247651	-0.277078964247651\\
66.875	0.11496	-0.194056691654754	-0.194056691654754\\
66.875	0.11862	-0.117960178594718	-0.117960178594718\\
66.875	0.12228	-0.0487894250675471	-0.0487894250675471\\
66.875	0.12594	0.0134555689267639	0.0134555689267639\\
66.875	0.1296	0.0687748033882123	0.0687748033882123\\
66.875	0.13326	0.117168278316797	0.117168278316797\\
66.875	0.13692	0.15863599371252	0.15863599371252\\
66.875	0.14058	0.193177949575379	0.193177949575379\\
66.875	0.14424	0.220794145905375	0.220794145905375\\
66.875	0.1479	0.241484582702511	0.241484582702511\\
66.875	0.15156	0.255249259966782	0.255249259966782\\
66.875	0.15522	0.262088177698193	0.262088177698193\\
66.875	0.15888	0.262001335896739	0.262001335896739\\
66.875	0.16254	0.254988734562427	0.254988734562427\\
66.875	0.1662	0.241050373695247	0.241050373695247\\
66.875	0.16986	0.220186253295209	0.220186253295209\\
66.875	0.17352	0.192396373362304	0.192396373362304\\
66.875	0.17718	0.157680733896541	0.157680733896541\\
66.875	0.18084	0.116039334897913	0.116039334897913\\
66.875	0.1845	0.0674721763664206	0.0674721763664206\\
66.875	0.18816	0.0119792583020661	0.0119792583020661\\
66.875	0.19182	-0.0504394192951469	-0.0504394192951469\\
66.875	0.19548	-0.119783856425226	-0.119783856425226\\
66.875	0.19914	-0.196054053088163	-0.196054053088163\\
66.875	0.2028	-0.279250009283968	-0.279250009283968\\
66.875	0.20646	-0.369371725012632	-0.369371725012632\\
66.875	0.21012	-0.466419200274164	-0.466419200274164\\
66.875	0.21378	-0.57039243506855	-0.57039243506855\\
66.875	0.21744	-0.681291429395804	-0.681291429395804\\
66.875	0.2211	-0.79911618325592	-0.79911618325592\\
66.875	0.22476	-0.923866696648895	-0.923866696648895\\
66.875	0.22842	-1.05554296957474	-1.05554296957474\\
66.875	0.23208	-1.19414500203344	-1.19414500203344\\
66.875	0.23574	-1.33967279402501	-1.33967279402501\\
66.875	0.2394	-1.49212634554943	-1.49212634554943\\
66.875	0.24306	-1.65150565660672	-1.65150565660672\\
66.875	0.24672	-1.81781072719687	-1.81781072719687\\
66.875	0.25038	-1.99104155731989	-1.99104155731989\\
66.875	0.25404	-2.17119814697577	-2.17119814697577\\
66.875	0.2577	-2.35828049616451	-2.35828049616451\\
66.875	0.26136	-2.55228860488611	-2.55228860488611\\
66.875	0.26502	-2.75322247314058	-2.75322247314058\\
66.875	0.26868	-2.96108210092791	-2.96108210092791\\
66.875	0.27234	-3.17586748824809	-3.17586748824809\\
66.875	0.276	-3.39757863510115	-3.39757863510115\\
67.25	0.093	-0.836894442431506	-0.836894442431506\\
67.25	0.09666	-0.716636647653395	-0.716636647653395\\
67.25	0.10032	-0.603304612408146	-0.603304612408146\\
67.25	0.10398	-0.496898336695759	-0.496898336695759\\
67.25	0.10764	-0.397417820516236	-0.397417820516236\\
67.25	0.1113	-0.304863063869573	-0.304863063869573\\
67.25	0.11496	-0.219234066755777	-0.219234066755777\\
67.25	0.11862	-0.140530829174839	-0.140530829174839\\
67.25	0.12228	-0.0687533511267655	-0.0687533511267655\\
67.25	0.12594	-0.00390163261155418	-0.00390163261155418\\
67.25	0.1296	0.0540243263707962	0.0540243263707962\\
67.25	0.13326	0.105024525820283	0.105024525820283\\
67.25	0.13692	0.149098965736904	0.149098965736904\\
67.25	0.14058	0.186247646120665	0.186247646120665\\
67.25	0.14424	0.216470566971567	0.216470566971567\\
67.25	0.1479	0.239767728289602	0.239767728289602\\
67.25	0.15156	0.256139130074775	0.256139130074775\\
67.25	0.15522	0.265584772327088	0.265584772327088\\
67.25	0.15888	0.268104655046536	0.268104655046536\\
67.25	0.16254	0.263698778233122	0.263698778233122\\
67.25	0.1662	0.252367141886848	0.252367141886848\\
67.25	0.16986	0.234109746007708	0.234109746007708\\
67.25	0.17352	0.208926590595706	0.208926590595706\\
67.25	0.17718	0.176817675650843	0.176817675650843\\
67.25	0.18084	0.137783001173117	0.137783001173117\\
67.25	0.1845	0.0918225671625272	0.0918225671625272\\
67.25	0.18816	0.0389363736190784	0.0389363736190784\\
67.25	0.19182	-0.0208755794572362	-0.0208755794572362\\
67.25	0.19548	-0.087613292066413	-0.087613292066413\\
67.25	0.19914	-0.161276764208448	-0.161276764208448\\
67.25	0.2028	-0.241865995883355	-0.241865995883355\\
67.25	0.20646	-0.329380987091117	-0.329380987091117\\
67.25	0.21012	-0.423821737831743	-0.423821737831743\\
67.25	0.21378	-0.525188248105231	-0.525188248105231\\
67.25	0.21744	-0.633480517911586	-0.633480517911586\\
67.25	0.2211	-0.748698547250797	-0.748698547250797\\
67.25	0.22476	-0.870842336122873	-0.870842336122873\\
67.25	0.22842	-0.999911884527812	-0.999911884527812\\
67.25	0.23208	-1.13590719246562	-1.13590719246562\\
67.25	0.23574	-1.27882825993628	-1.27882825993628\\
67.25	0.2394	-1.4286750869398	-1.4286750869398\\
67.25	0.24306	-1.5854476734762	-1.5854476734762\\
67.25	0.24672	-1.74914601954544	-1.74914601954544\\
67.25	0.25038	-1.91977012514756	-1.91977012514756\\
67.25	0.25404	-2.09731999028253	-2.09731999028253\\
67.25	0.2577	-2.28179561495037	-2.28179561495037\\
67.25	0.26136	-2.47319699915108	-2.47319699915108\\
67.25	0.26502	-2.67152414288464	-2.67152414288464\\
67.25	0.26868	-2.87677704615107	-2.87677704615107\\
67.25	0.27234	-3.08895570895035	-3.08895570895035\\
67.25	0.276	-3.3080601312825	-3.3080601312825\\
67.625	0.093	-0.878188220188758	-0.878188220188758\\
67.625	0.09666	-0.755323700889747	-0.755323700889747\\
67.625	0.10032	-0.639384941123597	-0.639384941123597\\
67.625	0.10398	-0.530371940890307	-0.530371940890307\\
67.625	0.10764	-0.428284700189883	-0.428284700189883\\
67.625	0.1113	-0.33312321902232	-0.33312321902232\\
67.625	0.11496	-0.244887497387618	-0.244887497387618\\
67.625	0.11862	-0.163577535285782	-0.163577535285782\\
67.625	0.12228	-0.0891933327168064	-0.0891933327168064\\
67.625	0.12594	-0.021734889680693	-0.021734889680693\\
67.625	0.1296	0.0387977938225559	0.0387977938225559\\
67.625	0.13326	0.0924047177929461	0.0924047177929461\\
67.625	0.13692	0.139085882230471	0.139085882230471\\
67.625	0.14058	0.178841287135135	0.178841287135135\\
67.625	0.14424	0.211670932506933	0.211670932506933\\
67.625	0.1479	0.237574818345871	0.237574818345871\\
67.625	0.15156	0.256552944651947	0.256552944651947\\
67.625	0.15522	0.268605311425162	0.268605311425162\\
67.625	0.15888	0.273731918665512	0.273731918665512\\
67.625	0.16254	0.271932766373	0.271932766373\\
67.625	0.1662	0.263207854547624	0.263207854547624\\
67.625	0.16986	0.247557183189385	0.247557183189385\\
67.625	0.17352	0.224980752298286	0.224980752298286\\
67.625	0.17718	0.195478561874328	0.195478561874328\\
67.625	0.18084	0.1590506119175	0.1590506119175\\
67.625	0.1845	0.115696902427812	0.115696902427812\\
67.625	0.18816	0.065417433405262	0.065417433405262\\
67.625	0.19182	0.00821220484985297	0.00821220484985297\\
67.625	0.19548	-0.0559187832384254	-0.0559187832384254\\
67.625	0.19914	-0.126975530859559	-0.126975530859559\\
67.625	0.2028	-0.20495803801356	-0.20495803801356\\
67.625	0.20646	-0.289866304700423	-0.289866304700423\\
67.625	0.21012	-0.38170033092015	-0.38170033092015\\
67.625	0.21378	-0.480460116672733	-0.480460116672733\\
67.625	0.21744	-0.586145661958186	-0.586145661958186\\
67.625	0.2211	-0.698756966776495	-0.698756966776495\\
67.625	0.22476	-0.818294031127673	-0.818294031127673\\
67.625	0.22842	-0.944756855011706	-0.944756855011706\\
67.625	0.23208	-1.07814543842861	-1.07814543842861\\
67.625	0.23574	-1.21845978137837	-1.21845978137837\\
67.625	0.2394	-1.365699883861	-1.365699883861\\
67.625	0.24306	-1.51986574587648	-1.51986574587648\\
67.625	0.24672	-1.68095736742483	-1.68095736742483\\
67.625	0.25038	-1.84897474850605	-1.84897474850605\\
67.625	0.25404	-2.02391788912012	-2.02391788912012\\
67.625	0.2577	-2.20578678926706	-2.20578678926706\\
67.625	0.26136	-2.39458144894686	-2.39458144894686\\
67.625	0.26502	-2.59030186815952	-2.59030186815952\\
67.625	0.26868	-2.79294804690505	-2.79294804690505\\
67.625	0.27234	-3.00251998518343	-3.00251998518343\\
67.625	0.276	-3.21901768299468	-3.21901768299468\\
68	0.093	-0.919958053476832	-0.919958053476832\\
68	0.09666	-0.794486809656918	-0.794486809656918\\
68	0.10032	-0.675941325369867	-0.675941325369867\\
68	0.10398	-0.564321600615676	-0.564321600615676\\
68	0.10764	-0.459627635394348	-0.459627635394348\\
68	0.1113	-0.361859429705884	-0.361859429705884\\
68	0.11496	-0.271016983550283	-0.271016983550283\\
68	0.11862	-0.187100296927544	-0.187100296927544\\
68	0.12228	-0.110109369837667	-0.110109369837667\\
68	0.12594	-0.0400442022806553	-0.0400442022806553\\
68	0.1296	0.0230952057434974	0.0230952057434974\\
68	0.13326	0.079308854234788	0.079308854234788\\
68	0.13692	0.128596743193214	0.128596743193214\\
68	0.14058	0.170958872618779	0.170958872618779\\
68	0.14424	0.206395242511481	0.206395242511481\\
68	0.1479	0.23490585287132	0.23490585287132\\
68	0.15156	0.256490703698296	0.256490703698296\\
68	0.15522	0.271149794992411	0.271149794992411\\
68	0.15888	0.278883126753667	0.278883126753667\\
68	0.16254	0.279690698982055	0.279690698982055\\
68	0.1662	0.273572511677583	0.273572511677583\\
68	0.16986	0.260528564840247	0.260528564840247\\
68	0.17352	0.240558858470049	0.240558858470049\\
68	0.17718	0.213663392566987	0.213663392566987\\
68	0.18084	0.179842167131065	0.179842167131065\\
68	0.1845	0.139095182162276	0.139095182162276\\
68	0.18816	0.091422437660631	0.091422437660631\\
68	0.19182	0.0368239336261205	0.0368239336261205\\
68	0.19548	-0.0247003299412523	-0.0247003299412523\\
68	0.19914	-0.0931503530414908	-0.0931503530414908\\
68	0.2028	-0.168526135674586	-0.168526135674586\\
68	0.20646	-0.250827677840551	-0.250827677840551\\
68	0.21012	-0.340054979539373	-0.340054979539373\\
68	0.21378	-0.436208040771056	-0.436208040771056\\
68	0.21744	-0.539286861535608	-0.539286861535608\\
68	0.2211	-0.649291441833014	-0.649291441833014\\
68	0.22476	-0.76622178166329	-0.76622178166329\\
68	0.22842	-0.890077881026425	-0.890077881026425\\
68	0.23208	-1.02085973992242	-1.02085973992242\\
68	0.23574	-1.15856735835129	-1.15856735835129\\
68	0.2394	-1.30320073631301	-1.30320073631301\\
68	0.24306	-1.4547598738076	-1.4547598738076\\
68	0.24672	-1.61324477083504	-1.61324477083504\\
68	0.25038	-1.77865542739536	-1.77865542739536\\
68	0.25404	-1.95099184348853	-1.95099184348853\\
68	0.2577	-2.13025401911456	-2.13025401911456\\
68	0.26136	-2.31644195427347	-2.31644195427347\\
68	0.26502	-2.50955564896522	-2.50955564896522\\
68	0.26868	-2.70959510318985	-2.70959510318985\\
68	0.27234	-2.91656031694733	-2.91656031694733\\
68	0.276	-3.13045129023768	-3.13045129023768\\
68.375	0.093	-0.962203942295729	-0.962203942295729\\
68.375	0.09666	-0.834125973954913	-0.834125973954913\\
68.375	0.10032	-0.712973765146959	-0.712973765146959\\
68.375	0.10398	-0.598747315871869	-0.598747315871869\\
68.375	0.10764	-0.491446626129641	-0.491446626129641\\
68.375	0.1113	-0.391071695920274	-0.391071695920274\\
68.375	0.11496	-0.297622525243771	-0.297622525243771\\
68.375	0.11862	-0.211099114100131	-0.211099114100131\\
68.375	0.12228	-0.131501462489353	-0.131501462489353\\
68.375	0.12594	-0.0588295704114339	-0.0588295704114339\\
68.375	0.1296	0.00691656213361558	0.00691656213361558\\
68.375	0.13326	0.0657369351458064	0.0657369351458064\\
68.375	0.13692	0.117631548625136	0.117631548625136\\
68.375	0.14058	0.162600402571603	0.162600402571603\\
68.375	0.14424	0.200643496985205	0.200643496985205\\
68.375	0.1479	0.231760831865945	0.231760831865945\\
68.375	0.15156	0.255952407213824	0.255952407213824\\
68.375	0.15522	0.27321822302884	0.27321822302884\\
68.375	0.15888	0.283558279310994	0.283558279310994\\
68.375	0.16254	0.286972576060286	0.286972576060286\\
68.375	0.1662	0.283461113276714	0.283461113276714\\
68.375	0.16986	0.273023890960276	0.273023890960276\\
68.375	0.17352	0.255660909110981	0.255660909110981\\
68.375	0.17718	0.23137216772882	0.23137216772882\\
68.375	0.18084	0.200157666813803	0.200157666813803\\
68.375	0.1845	0.162017406365916	0.162017406365916\\
68.375	0.18816	0.116951386385173	0.116951386385173\\
68.375	0.19182	0.064959606871561	0.064959606871561\\
68.375	0.19548	0.00604206782508676	0.00604206782508676\\
68.375	0.19914	-0.0598012307542461	-0.0598012307542461\\
68.375	0.2028	-0.132570288866443	-0.132570288866443\\
68.375	0.20646	-0.212265106511502	-0.212265106511502\\
68.375	0.21012	-0.298885683689429	-0.298885683689429\\
68.375	0.21378	-0.392432020400207	-0.392432020400207\\
68.375	0.21744	-0.492904116643857	-0.492904116643857\\
68.375	0.2211	-0.600301972420365	-0.600301972420365\\
68.375	0.22476	-0.714625587729735	-0.714625587729735\\
68.375	0.22842	-0.835874962571971	-0.835874962571971\\
68.375	0.23208	-0.964050096947069	-0.964050096947069\\
68.375	0.23574	-1.09915099085503	-1.09915099085503\\
68.375	0.2394	-1.24117764429585	-1.24117764429585\\
68.375	0.24306	-1.39013005726953	-1.39013005726953\\
68.375	0.24672	-1.54600822977608	-1.54600822977608\\
68.375	0.25038	-1.70881216181549	-1.70881216181549\\
68.375	0.25404	-1.87854185338776	-1.87854185338776\\
68.375	0.2577	-2.05519730449289	-2.05519730449289\\
68.375	0.26136	-2.2387785151309	-2.2387785151309\\
68.375	0.26502	-2.42928548530176	-2.42928548530176\\
68.375	0.26868	-2.62671821500548	-2.62671821500548\\
68.375	0.27234	-2.83107670424206	-2.83107670424206\\
68.375	0.276	-3.04236095301151	-3.04236095301151\\
68.75	0.093	-1.00492588664544	-1.00492588664544\\
68.75	0.09666	-0.874241193783726	-0.874241193783726\\
68.75	0.10032	-0.75048226045487	-0.75048226045487\\
68.75	0.10398	-0.633649086658879	-0.633649086658879\\
68.75	0.10764	-0.523741672395748	-0.523741672395748\\
68.75	0.1113	-0.420760017665482	-0.420760017665482\\
68.75	0.11496	-0.324704122468078	-0.324704122468078\\
68.75	0.11862	-0.235573986803535	-0.235573986803535\\
68.75	0.12228	-0.153369610671855	-0.153369610671855\\
68.75	0.12594	-0.0780909940730377	-0.0780909940730377\\
68.75	0.1296	-0.00973813700708437	-0.00973813700708437\\
68.75	0.13326	0.0516889605260102	0.0516889605260102\\
68.75	0.13692	0.10619029852624	0.10619029852624\\
68.75	0.14058	0.153765876993606	0.153765876993606\\
68.75	0.14424	0.194415695928112	0.194415695928112\\
68.75	0.1479	0.228139755329755	0.228139755329755\\
68.75	0.15156	0.254938055198531	0.254938055198531\\
68.75	0.15522	0.274810595534451	0.274810595534451\\
68.75	0.15888	0.287757376337507	0.287757376337507\\
68.75	0.16254	0.293778397607696	0.293778397607696\\
68.75	0.1662	0.292873659345028	0.292873659345028\\
68.75	0.16986	0.285043161549496	0.285043161549496\\
68.75	0.17352	0.2702869042211	0.2702869042211\\
68.75	0.17718	0.248604887359841	0.248604887359841\\
68.75	0.18084	0.219997110965723	0.219997110965723\\
68.75	0.1845	0.184463575038734	0.184463575038734\\
68.75	0.18816	0.142004279578893	0.142004279578893\\
68.75	0.19182	0.092619224586187	0.092619224586187\\
68.75	0.19548	0.0363084100606113	0.0363084100606113\\
68.75	0.19914	-0.0269281639978161	-0.0269281639978161\\
68.75	0.2028	-0.0970904975891145	-0.0970904975891145\\
68.75	0.20646	-0.174178590713275	-0.174178590713275\\
68.75	0.21012	-0.258192443370293	-0.258192443370293\\
68.75	0.21378	-0.349132055560172	-0.349132055560172\\
68.75	0.21744	-0.44699742728292	-0.44699742728292\\
68.75	0.2211	-0.551788558538529	-0.551788558538529\\
68.75	0.22476	-0.663505449326998	-0.663505449326998\\
68.75	0.22842	-0.782148099648332	-0.782148099648332\\
68.75	0.23208	-0.907716509502524	-0.907716509502524\\
68.75	0.23574	-1.04021067888958	-1.04021067888958\\
68.75	0.2394	-1.17963060780951	-1.17963060780951\\
68.75	0.24306	-1.32597629626229	-1.32597629626229\\
68.75	0.24672	-1.47924774424793	-1.47924774424793\\
68.75	0.25038	-1.63944495176644	-1.63944495176644\\
68.75	0.25404	-1.80656791881781	-1.80656791881781\\
68.75	0.2577	-1.98061664540205	-1.98061664540205\\
68.75	0.26136	-2.16159113151914	-2.16159113151914\\
68.75	0.26502	-2.3494913771691	-2.3494913771691\\
68.75	0.26868	-2.54431738235192	-2.54431738235192\\
68.75	0.27234	-2.7460691470676	-2.7460691470676\\
68.75	0.276	-2.95474667131615	-2.95474667131615\\
69.125	0.093	-1.04812388652598	-1.04812388652598\\
69.125	0.09666	-0.914832469143363	-0.914832469143363\\
69.125	0.10032	-0.788466811293607	-0.788466811293607\\
69.125	0.10398	-0.669026912976713	-0.669026912976713\\
69.125	0.10764	-0.55651277419268	-0.55651277419268\\
69.125	0.1113	-0.450924394941511	-0.450924394941511\\
69.125	0.11496	-0.352261775223204	-0.352261775223204\\
69.125	0.11862	-0.260524915037761	-0.260524915037761\\
69.125	0.12228	-0.175713814385181	-0.175713814385181\\
69.125	0.12594	-0.0978284732654613	-0.0978284732654613\\
69.125	0.1296	-0.0268688916786077	-0.0268688916786077\\
69.125	0.13326	0.0371649303753889	0.0371649303753889\\
69.125	0.13692	0.0942729928965189	0.0942729928965189\\
69.125	0.14058	0.144455295884788	0.144455295884788\\
69.125	0.14424	0.187711839340193	0.187711839340193\\
69.125	0.1479	0.22404262326274	0.22404262326274\\
69.125	0.15156	0.253447647652418	0.253447647652418\\
69.125	0.15522	0.275926912509238	0.275926912509238\\
69.125	0.15888	0.291480417833195	0.291480417833195\\
69.125	0.16254	0.300108163624289	0.300108163624289\\
69.125	0.1662	0.301810149882518	0.301810149882518\\
69.125	0.16986	0.296586376607888	0.296586376607888\\
69.125	0.17352	0.284436843800394	0.284436843800394\\
69.125	0.17718	0.265361551460037	0.265361551460037\\
69.125	0.18084	0.239360499586818	0.239360499586818\\
69.125	0.1845	0.206433688180738	0.206433688180738\\
69.125	0.18816	0.166581117241792	0.166581117241792\\
69.125	0.19182	0.119802786769988	0.119802786769988\\
69.125	0.19548	0.0660986967653177	0.0660986967653177\\
69.125	0.19914	0.00546884722778884	0.00546884722778884\\
69.125	0.2028	-0.0620867618426075	-0.0620867618426075\\
69.125	0.20646	-0.136568130445863	-0.136568130445863\\
69.125	0.21012	-0.217975258581985	-0.217975258581985\\
69.125	0.21378	-0.306308146250963	-0.306308146250963\\
69.125	0.21744	-0.401566793452808	-0.401566793452808\\
69.125	0.2211	-0.503751200187512	-0.503751200187512\\
69.125	0.22476	-0.612861366455085	-0.612861366455085\\
69.125	0.22842	-0.728897292255514	-0.728897292255514\\
69.125	0.23208	-0.851858977588808	-0.851858977588808\\
69.125	0.23574	-0.981746422454966	-0.981746422454966\\
69.125	0.2394	-1.11855962685398	-1.11855962685398\\
69.125	0.24306	-1.26229859078586	-1.26229859078586\\
69.125	0.24672	-1.41296331425061	-1.41296331425061\\
69.125	0.25038	-1.57055379724821	-1.57055379724821\\
69.125	0.25404	-1.73507003977869	-1.73507003977869\\
69.125	0.2577	-1.90651204184202	-1.90651204184202\\
69.125	0.26136	-2.08487980343821	-2.08487980343821\\
69.125	0.26502	-2.27017332456726	-2.27017332456726\\
69.125	0.26868	-2.46239260522919	-2.46239260522919\\
69.125	0.27234	-2.66153764542396	-2.66153764542396\\
69.125	0.276	-2.86760844515161	-2.86760844515161\\
69.5	0.093	-1.09179794193734	-1.09179794193734\\
69.5	0.09666	-0.955899800033817	-0.955899800033817\\
69.5	0.10032	-0.826927417663159	-0.826927417663159\\
69.5	0.10398	-0.704880794825364	-0.704880794825364\\
69.5	0.10764	-0.589759931520432	-0.589759931520432\\
69.5	0.1113	-0.481564827748361	-0.481564827748361\\
69.5	0.11496	-0.380295483509152	-0.380295483509152\\
69.5	0.11862	-0.285951898802808	-0.285951898802808\\
69.5	0.12228	-0.198534073629325	-0.198534073629325\\
69.5	0.12594	-0.118042007988705	-0.118042007988705\\
69.5	0.1296	-0.0444757018809474	-0.0444757018809474\\
69.5	0.13326	0.0221648446939495	0.0221648446939495\\
69.5	0.13692	0.0818796317359798	0.0818796317359798\\
69.5	0.14058	0.134668659245151	0.134668659245151\\
69.5	0.14424	0.18053192722146	0.18053192722146\\
69.5	0.1479	0.219469435664903	0.219469435664903\\
69.5	0.15156	0.251481184575486	0.251481184575486\\
69.5	0.15522	0.276567173953209	0.276567173953209\\
69.5	0.15888	0.294727403798064	0.294727403798064\\
69.5	0.16254	0.305961874110059	0.305961874110059\\
69.5	0.1662	0.310270584889195	0.310270584889195\\
69.5	0.16986	0.307653536135462	0.307653536135462\\
69.5	0.17352	0.298110727848869	0.298110727848869\\
69.5	0.17718	0.281642160029417	0.281642160029417\\
69.5	0.18084	0.2582478326771	0.2582478326771\\
69.5	0.1845	0.227927745791918	0.227927745791918\\
69.5	0.18816	0.190681899373878	0.190681899373878\\
69.5	0.19182	0.146510293422972	0.146510293422972\\
69.5	0.19548	0.0954129279392042	0.0954129279392042\\
69.5	0.19914	0.0373898029225739	0.0373898029225739\\
69.5	0.2028	-0.0275590816269204	-0.0275590816269204\\
69.5	0.20646	-0.0994337257092734	-0.0994337257092734\\
69.5	0.21012	-0.17823412932449	-0.17823412932449\\
69.5	0.21378	-0.26396029247257	-0.26396029247257\\
69.5	0.21744	-0.356612215153513	-0.356612215153513\\
69.5	0.2211	-0.456189897367318	-0.456189897367318\\
69.5	0.22476	-0.562693339113986	-0.562693339113986\\
69.5	0.22842	-0.676122540393512	-0.676122540393512\\
69.5	0.23208	-0.796477501205908	-0.796477501205908\\
69.5	0.23574	-0.923758221551164	-0.923758221551164\\
69.5	0.2394	-1.05796470142928	-1.05796470142928\\
69.5	0.24306	-1.19909694084026	-1.19909694084026\\
69.5	0.24672	-1.3471549397841	-1.3471549397841\\
69.5	0.25038	-1.50213869826081	-1.50213869826081\\
69.5	0.25404	-1.66404821627037	-1.66404821627037\\
69.5	0.2577	-1.83288349381281	-1.83288349381281\\
69.5	0.26136	-2.0086445308881	-2.0086445308881\\
69.5	0.26502	-2.19133132749625	-2.19133132749625\\
69.5	0.26868	-2.38094388363727	-2.38094388363727\\
69.5	0.27234	-2.57748219931115	-2.57748219931115\\
69.5	0.276	-2.78094627451789	-2.78094627451789\\
69.875	0.093	-1.13594805287952	-1.13594805287952\\
69.875	0.09666	-0.997443186455095	-0.997443186455095\\
69.875	0.10032	-0.865864079563537	-0.865864079563537\\
69.875	0.10398	-0.741210732204838	-0.741210732204838\\
69.875	0.10764	-0.623483144379005	-0.623483144379005\\
69.875	0.1113	-0.512681316086032	-0.512681316086032\\
69.875	0.11496	-0.408805247325923	-0.408805247325923\\
69.875	0.11862	-0.311854938098678	-0.311854938098678\\
69.875	0.12228	-0.221830388404292	-0.221830388404292\\
69.875	0.12594	-0.138731598242772	-0.138731598242772\\
69.875	0.1296	-0.062558567614114	-0.062558567614114\\
69.875	0.13326	0.00668870348168316	0.00668870348168316\\
69.875	0.13692	0.0690102150446208	0.0690102150446208\\
69.875	0.14058	0.124405967074691	0.124405967074691\\
69.875	0.14424	0.1728759595719	0.1728759595719\\
69.875	0.1479	0.214420192536247	0.214420192536247\\
69.875	0.15156	0.24903866596773	0.24903866596773\\
69.875	0.15522	0.276731379866353	0.276731379866353\\
69.875	0.15888	0.297498334232114	0.297498334232114\\
69.875	0.16254	0.311339529065009	0.311339529065009\\
69.875	0.1662	0.318254964365042	0.318254964365042\\
69.875	0.16986	0.318244640132213	0.318244640132213\\
69.875	0.17352	0.311308556366527	0.311308556366527\\
69.875	0.17718	0.29744671306797	0.29744671306797\\
69.875	0.18084	0.276659110236555	0.276659110236555\\
69.875	0.1845	0.248945747872275	0.248945747872275\\
69.875	0.18816	0.214306625975134	0.214306625975134\\
69.875	0.19182	0.172741744545133	0.172741744545133\\
69.875	0.19548	0.124251103582267	0.124251103582267\\
69.875	0.19914	0.0688347030865391	0.0688347030865391\\
69.875	0.2028	0.00649254305794678	0.00649254305794678\\
69.875	0.20646	-0.0627753765035077	-0.0627753765035077\\
69.875	0.21012	-0.138969055597823	-0.138969055597823\\
69.875	0.21378	-0.222088494225	-0.222088494225\\
69.875	0.21744	-0.312133692385041	-0.312133692385041\\
69.875	0.2211	-0.409104650077944	-0.409104650077944\\
69.875	0.22476	-0.513001367303714	-0.513001367303714\\
69.875	0.22842	-0.623823844062342	-0.623823844062342\\
69.875	0.23208	-0.741572080353828	-0.741572080353828\\
69.875	0.23574	-0.866246076178186	-0.866246076178186\\
69.875	0.2394	-0.9978458315354	-0.9978458315354\\
69.875	0.24306	-1.13637134642548	-1.13637134642548\\
69.875	0.24672	-1.28182262084842	-1.28182262084842\\
69.875	0.25038	-1.43419965480423	-1.43419965480423\\
69.875	0.25404	-1.59350244829289	-1.59350244829289\\
69.875	0.2577	-1.75973100131441	-1.75973100131441\\
69.875	0.26136	-1.93288531386881	-1.93288531386881\\
69.875	0.26502	-2.11296538595606	-2.11296538595606\\
69.875	0.26868	-2.29997121757618	-2.29997121757618\\
69.875	0.27234	-2.49390280872915	-2.49390280872915\\
69.875	0.276	-2.694760159415	-2.694760159415\\
70.25	0.093	-1.18057421935251	-1.18057421935251\\
70.25	0.09666	-1.03946262840719	-1.03946262840719\\
70.25	0.10032	-0.905276796994731	-0.905276796994731\\
70.25	0.10398	-0.778016725115133	-0.778016725115133\\
70.25	0.10764	-0.657682412768397	-0.657682412768397\\
70.25	0.1113	-0.544273859954524	-0.544273859954524\\
70.25	0.11496	-0.437791066673512	-0.437791066673512\\
70.25	0.11862	-0.338234032925365	-0.338234032925365\\
70.25	0.12228	-0.245602758710081	-0.245602758710081\\
70.25	0.12594	-0.159897244027657	-0.159897244027657\\
70.25	0.1296	-0.0811174888780952	-0.0811174888780952\\
70.25	0.13326	-0.00926349326139775	-0.00926349326139775\\
70.25	0.13692	0.0556647428224366	0.0556647428224366\\
70.25	0.14058	0.113667219373412	0.113667219373412\\
70.25	0.14424	0.164743936391521	0.164743936391521\\
70.25	0.1479	0.208894893876769	0.208894893876769\\
70.25	0.15156	0.246120091829155	0.246120091829155\\
70.25	0.15522	0.276419530248679	0.276419530248679\\
70.25	0.15888	0.299793209135342	0.299793209135342\\
70.25	0.16254	0.316241128489141	0.316241128489141\\
70.25	0.1662	0.325763288310077	0.325763288310077\\
70.25	0.16986	0.328359688598147	0.328359688598147\\
70.25	0.17352	0.324030329353359	0.324030329353359\\
70.25	0.17718	0.312775210575708	0.312775210575708\\
70.25	0.18084	0.294594332265195	0.294594332265195\\
70.25	0.1845	0.269487694421814	0.269487694421814\\
70.25	0.18816	0.237455297045578	0.237455297045578\\
70.25	0.19182	0.198497140136476	0.198497140136476\\
70.25	0.19548	0.152613223694509	0.152613223694509\\
70.25	0.19914	0.0998035477196826	0.0998035477196826\\
70.25	0.2028	0.0400681122119924	0.0400681122119924\\
70.25	0.20646	-0.0265930828285601	-0.0265930828285601\\
70.25	0.21012	-0.100180037401973	-0.100180037401973\\
70.25	0.21378	-0.180692751508248	-0.180692751508248\\
70.25	0.21744	-0.268131225147391	-0.268131225147391\\
70.25	0.2211	-0.362495458319392	-0.362495458319392\\
70.25	0.22476	-0.463785451024256	-0.463785451024256\\
70.25	0.22842	-0.572001203261982	-0.572001203261982\\
70.25	0.23208	-0.687142715032573	-0.687142715032573\\
70.25	0.23574	-0.809209986336025	-0.809209986336025\\
70.25	0.2394	-0.938203017172338	-0.938203017172338\\
70.25	0.24306	-1.07412180754152	-1.07412180754152\\
70.25	0.24672	-1.21696635744355	-1.21696635744355\\
70.25	0.25038	-1.36673666687846	-1.36673666687846\\
70.25	0.25404	-1.52343273584622	-1.52343273584622\\
70.25	0.2577	-1.68705456434685	-1.68705456434685\\
70.25	0.26136	-1.85760215238034	-1.85760215238034\\
70.25	0.26502	-2.03507549994669	-2.03507549994669\\
70.25	0.26868	-2.21947460704591	-2.21947460704591\\
70.25	0.27234	-2.41079947367798	-2.41079947367798\\
70.25	0.276	-2.60905009984292	-2.60905009984292\\
70.625	0.093	-1.22567644135634	-1.22567644135634\\
70.625	0.09666	-1.08195812589012	-1.08195812589012\\
70.625	0.10032	-0.945165569956754	-0.945165569956754\\
70.625	0.10398	-0.815298773556254	-0.815298773556254\\
70.625	0.10764	-0.692357736688618	-0.692357736688618\\
70.625	0.1113	-0.576342459353842	-0.576342459353842\\
70.625	0.11496	-0.46725294155193	-0.46725294155193\\
70.625	0.11862	-0.365089183282881	-0.365089183282881\\
70.625	0.12228	-0.269851184546695	-0.269851184546695\\
70.625	0.12594	-0.18153894534337	-0.18153894534337\\
70.625	0.1296	-0.100152465672907	-0.100152465672907\\
70.625	0.13326	-0.0256917455353056	-0.0256917455353056\\
70.625	0.13692	0.0418432150694308	0.0418432150694308\\
70.625	0.14058	0.102452416141307	0.102452416141307\\
70.625	0.14424	0.15613585768032	0.15613585768032\\
70.625	0.1479	0.202893539686466	0.202893539686466\\
70.625	0.15156	0.242725462159752	0.242725462159752\\
70.625	0.15522	0.275631625100178	0.275631625100178\\
70.625	0.15888	0.301612028507741	0.301612028507741\\
70.625	0.16254	0.320666672382441	0.320666672382441\\
70.625	0.1662	0.332795556724279	0.332795556724279\\
70.625	0.16986	0.337998681533251	0.337998681533251\\
70.625	0.17352	0.336276046809365	0.336276046809365\\
70.625	0.17718	0.327627652552613	0.327627652552613\\
70.625	0.18084	0.312053498763001	0.312053498763001\\
70.625	0.1845	0.289553585440522	0.289553585440522\\
70.625	0.18816	0.260127912585189	0.260127912585189\\
70.625	0.19182	0.223776480196989	0.223776480196989\\
70.625	0.19548	0.180499288275923	0.180499288275923\\
70.625	0.19914	0.130296336821999	0.130296336821999\\
70.625	0.2028	0.073167625835211	0.073167625835211\\
70.625	0.20646	0.00911315531556056	0.00911315531556056\\
70.625	0.21012	-0.0618670747369539	-0.0618670747369539\\
70.625	0.21378	-0.139773064322327	-0.139773064322327\\
70.625	0.21744	-0.224604813440568	-0.224604813440568\\
70.625	0.2211	-0.316362322091667	-0.316362322091667\\
70.625	0.22476	-0.415045590275632	-0.415045590275632\\
70.625	0.22842	-0.520654617992456	-0.520654617992456\\
70.625	0.23208	-0.633189405242145	-0.633189405242145\\
70.625	0.23574	-0.752649952024695	-0.752649952024695\\
70.625	0.2394	-0.879036258340109	-0.879036258340109\\
70.625	0.24306	-1.01234832418838	-1.01234832418838\\
70.625	0.24672	-1.15258614956952	-1.15258614956952\\
70.625	0.25038	-1.29974973448352	-1.29974973448352\\
70.625	0.25404	-1.45383907893038	-1.45383907893038\\
70.625	0.2577	-1.61485418291011	-1.61485418291011\\
70.625	0.26136	-1.7827950464227	-1.7827950464227\\
70.625	0.26502	-1.95766166946814	-1.95766166946814\\
70.625	0.26868	-2.13945405204646	-2.13945405204646\\
70.625	0.27234	-2.32817219415764	-2.32817219415764\\
70.625	0.276	-2.52381609580167	-2.52381609580167\\
71	0.093	-1.27125471889098	-1.27125471889098\\
71	0.09666	-1.12492967890386	-1.12492967890386\\
71	0.10032	-0.985530398449591	-0.985530398449591\\
71	0.10398	-0.853056877528191	-0.853056877528191\\
71	0.10764	-0.727509116139651	-0.727509116139651\\
71	0.1113	-0.608887114283977	-0.608887114283977\\
71	0.11496	-0.497190871961161	-0.497190871961161\\
71	0.11862	-0.392420389171212	-0.392420389171212\\
71	0.12228	-0.294575665914122	-0.294575665914122\\
71	0.12594	-0.203656702189897	-0.203656702189897\\
71	0.1296	-0.119663497998533	-0.119663497998533\\
71	0.13326	-0.0425960533400316	-0.0425960533400316\\
71	0.13692	0.0275456317856051	0.0275456317856051\\
71	0.14058	0.0907615573783831	0.0907615573783831\\
71	0.14424	0.147051723438296	0.147051723438296\\
71	0.1479	0.19641612996535	0.19641612996535\\
71	0.15156	0.238854776959537	0.238854776959537\\
71	0.15522	0.274367664420863	0.274367664420863\\
71	0.15888	0.302954792349328	0.302954792349328\\
71	0.16254	0.324616160744927	0.324616160744927\\
71	0.1662	0.339351769607664	0.339351769607664\\
71	0.16986	0.347161618937542	0.347161618937542\\
71	0.17352	0.348045708734555	0.348045708734555\\
71	0.17718	0.342004038998704	0.342004038998704\\
71	0.18084	0.329036609729995	0.329036609729995\\
71	0.1845	0.309143420928418	0.309143420928418\\
71	0.18816	0.282324472593986	0.282324472593986\\
71	0.19182	0.248579764726689	0.248579764726689\\
71	0.19548	0.207909297326525	0.207909297326525\\
71	0.19914	0.160313070393499	0.160313070393499\\
71	0.2028	0.105791083927613	0.105791083927613\\
71	0.20646	0.0443433379288649	0.0443433379288649\\
71	0.21012	-0.024030167602751	-0.024030167602751\\
71	0.21378	-0.0993294326672221	-0.0993294326672221\\
71	0.21744	-0.181554457264557	-0.181554457264557\\
71	0.2211	-0.270705241394758	-0.270705241394758\\
71	0.22476	-0.366781785057821	-0.366781785057821\\
71	0.22842	-0.469784088253743	-0.469784088253743\\
71	0.23208	-0.579712150982527	-0.579712150982527\\
71	0.23574	-0.696565973244178	-0.696565973244178\\
71	0.2394	-0.820345555038687	-0.820345555038687\\
71	0.24306	-0.951050896366066	-0.951050896366066\\
71	0.24672	-1.0886819972263	-1.0886819972263\\
71	0.25038	-1.2332388576194	-1.2332388576194\\
71	0.25404	-1.38472147754536	-1.38472147754536\\
71	0.2577	-1.54312985700419	-1.54312985700419\\
71	0.26136	-1.70846399599587	-1.70846399599587\\
71	0.26502	-1.88072389452042	-1.88072389452042\\
71	0.26868	-2.05990955257783	-2.05990955257783\\
71	0.27234	-2.2460209701681	-2.2460209701681\\
71	0.276	-2.43905814729125	-2.43905814729125\\
71.375	0.093	-1.31730905195644	-1.31730905195644\\
71.375	0.09666	-1.16837728744841	-1.16837728744841\\
71.375	0.10032	-1.02637128247325	-1.02637128247325\\
71.375	0.10398	-0.891291037030946	-0.891291037030946\\
71.375	0.10764	-0.763136551121507	-0.763136551121507\\
71.375	0.1113	-0.641907824744928	-0.641907824744928\\
71.375	0.11496	-0.527604857901214	-0.527604857901214\\
71.375	0.11862	-0.420227650590362	-0.420227650590362\\
71.375	0.12228	-0.319776202812372	-0.319776202812372\\
71.375	0.12594	-0.226250514567247	-0.226250514567247\\
71.375	0.1296	-0.139650585854979	-0.139650585854979\\
71.375	0.13326	-0.0599764166755739	-0.0599764166755739\\
71.375	0.13692	0.0127719929709631	0.0127719929709631\\
71.375	0.14058	0.0785946430846431	0.0785946430846431\\
71.375	0.14424	0.137491533665457	0.137491533665457\\
71.375	0.1479	0.18946266471341	0.18946266471341\\
71.375	0.15156	0.234508036228497	0.234508036228497\\
71.375	0.15522	0.272627648210728	0.272627648210728\\
71.375	0.15888	0.303821500660091	0.303821500660091\\
71.375	0.16254	0.328089593576594	0.328089593576594\\
71.375	0.1662	0.345431926960234	0.345431926960234\\
71.375	0.16986	0.355848500811013	0.355848500811013\\
71.375	0.17352	0.359339315128924	0.359339315128924\\
71.375	0.17718	0.355904369913979	0.355904369913979\\
71.375	0.18084	0.345543665166172	0.345543665166172\\
71.375	0.1845	0.328257200885494	0.328257200885494\\
71.375	0.18816	0.304044977071964	0.304044977071964\\
71.375	0.19182	0.272906993725565	0.272906993725565\\
71.375	0.19548	0.234843250846307	0.234843250846307\\
71.375	0.19914	0.18985374843418	0.18985374843418\\
71.375	0.2028	0.137938486489196	0.137938486489196\\
71.375	0.20646	0.0790974650113494	0.0790974650113494\\
71.375	0.21012	0.0133306840006391	0.0133306840006391\\
71.375	0.21378	-0.0593618565429335	-0.0593618565429335\\
71.375	0.21744	-0.138980156619366	-0.138980156619366\\
71.375	0.2211	-0.225524216228662	-0.225524216228662\\
71.375	0.22476	-0.318994035370826	-0.318994035370826\\
71.375	0.22842	-0.419389614045846	-0.419389614045846\\
71.375	0.23208	-0.526710952253735	-0.526710952253735\\
71.375	0.23574	-0.640958049994481	-0.640958049994481\\
71.375	0.2394	-0.762130907268087	-0.762130907268087\\
71.375	0.24306	-0.890229524074561	-0.890229524074561\\
71.375	0.24672	-1.0252539004139	-1.0252539004139\\
71.375	0.25038	-1.1672040362861	-1.1672040362861\\
71.375	0.25404	-1.31607993169116	-1.31607993169116\\
71.375	0.2577	-1.47188158662908	-1.47188158662908\\
71.375	0.26136	-1.63460900109986	-1.63460900109986\\
71.375	0.26502	-1.80426217510351	-1.80426217510351\\
71.375	0.26868	-1.98084110864002	-1.98084110864002\\
71.375	0.27234	-2.16434580170939	-2.16434580170939\\
71.375	0.276	-2.35477625431163	-2.35477625431163\\
71.75	0.093	-1.36383944055273	-1.36383944055273\\
71.75	0.09666	-1.2123009515238	-1.2123009515238\\
71.75	0.10032	-1.06768822202773	-1.06768822202773\\
71.75	0.10398	-0.930001252064525	-0.930001252064525\\
71.75	0.10764	-0.799240041634186	-0.799240041634186\\
71.75	0.1113	-0.675404590736704	-0.675404590736704\\
71.75	0.11496	-0.558494899372088	-0.558494899372088\\
71.75	0.11862	-0.448510967540334	-0.448510967540334\\
71.75	0.12228	-0.345452795241444	-0.345452795241444\\
71.75	0.12594	-0.249320382475411	-0.249320382475411\\
71.75	0.1296	-0.160113729242247	-0.160113729242247\\
71.75	0.13326	-0.0778328355419449	-0.0778328355419449\\
71.75	0.13692	-0.00247770137450054	-0.00247770137450054\\
71.75	0.14058	0.065951673260078	0.065951673260078\\
71.75	0.14424	0.127455288361795	0.127455288361795\\
71.75	0.1479	0.182033143930649	0.182033143930649\\
71.75	0.15156	0.22968523996664	0.22968523996664\\
71.75	0.15522	0.270411576469771	0.270411576469771\\
71.75	0.15888	0.304212153440036	0.304212153440036\\
71.75	0.16254	0.33108697087744	0.33108697087744\\
71.75	0.1662	0.351036028781981	0.351036028781981\\
71.75	0.16986	0.364059327153659	0.364059327153659\\
71.75	0.17352	0.370156865992479	0.370156865992479\\
71.75	0.17718	0.369328645298433	0.369328645298433\\
71.75	0.18084	0.361574665071521	0.361574665071521\\
71.75	0.1845	0.346894925311751	0.346894925311751\\
71.75	0.18816	0.32528942601912	0.32528942601912\\
71.75	0.19182	0.296758167193627	0.296758167193627\\
71.75	0.19548	0.261301148835264	0.261301148835264\\
71.75	0.19914	0.218918370944042	0.218918370944042\\
71.75	0.2028	0.16960983351996	0.16960983351996\\
71.75	0.20646	0.113375536563012	0.113375536563012\\
71.75	0.21012	0.0502154800732004	0.0502154800732004\\
71.75	0.21378	-0.0198703359494665	-0.0198703359494665\\
71.75	0.21744	-0.096881911505001	-0.096881911505001\\
71.75	0.2211	-0.180819246593394	-0.180819246593394\\
71.75	0.22476	-0.271682341214653	-0.271682341214653\\
71.75	0.22842	-0.369471195368778	-0.369471195368778\\
71.75	0.23208	-0.474185809055758	-0.474185809055758\\
71.75	0.23574	-0.585826182275605	-0.585826182275605\\
71.75	0.2394	-0.704392315028317	-0.704392315028317\\
71.75	0.24306	-0.829884207313889	-0.829884207313889\\
71.75	0.24672	-0.962301859132316	-0.962301859132316\\
71.75	0.25038	-1.10164527048362	-1.10164527048362\\
71.75	0.25404	-1.24791444136777	-1.24791444136777\\
71.75	0.2577	-1.40110937178479	-1.40110937178479\\
71.75	0.26136	-1.56123006173468	-1.56123006173468\\
71.75	0.26502	-1.72827651121742	-1.72827651121742\\
71.75	0.26868	-1.90224872023304	-1.90224872023304\\
71.75	0.27234	-2.0831466887815	-2.0831466887815\\
71.75	0.276	-2.27097041686283	-2.27097041686283\\
72.125	0.093	-1.41084588467983	-1.41084588467983\\
72.125	0.09666	-1.25670067113	-1.25670067113\\
72.125	0.10032	-1.10948121711303	-1.10948121711303\\
72.125	0.10398	-0.969187522628925	-0.969187522628925\\
72.125	0.10764	-0.835819587677682	-0.835819587677682\\
72.125	0.1113	-0.709377412259302	-0.709377412259302\\
72.125	0.11496	-0.589860996373784	-0.589860996373784\\
72.125	0.11862	-0.477270340021128	-0.477270340021128\\
72.125	0.12228	-0.371605443201337	-0.371605443201337\\
72.125	0.12594	-0.272866305914405	-0.272866305914405\\
72.125	0.1296	-0.181052928160336	-0.181052928160336\\
72.125	0.13326	-0.0961653099391304	-0.0961653099391304\\
72.125	0.13692	-0.0182034512507894	-0.0182034512507894\\
72.125	0.14058	0.0528326479046948	0.0528326479046948\\
72.125	0.14424	0.116942987527312	0.116942987527312\\
72.125	0.1479	0.174127567617067	0.174127567617067\\
72.125	0.15156	0.224386388173958	0.224386388173958\\
72.125	0.15522	0.267719449197992	0.267719449197992\\
72.125	0.15888	0.30412675068916	0.30412675068916\\
72.125	0.16254	0.333608292647464	0.333608292647464\\
72.125	0.1662	0.356164075072907	0.356164075072907\\
72.125	0.16986	0.371794097965487	0.371794097965487\\
72.125	0.17352	0.380498361325206	0.380498361325206\\
72.125	0.17718	0.382276865152061	0.382276865152061\\
72.125	0.18084	0.377129609446055	0.377129609446055\\
72.125	0.1845	0.36505659420718	0.36505659420718\\
72.125	0.18816	0.346057819435455	0.346057819435455\\
72.125	0.19182	0.32013328513086	0.32013328513086\\
72.125	0.19548	0.287282991293399	0.287282991293399\\
72.125	0.19914	0.247506937923083	0.247506937923083\\
72.125	0.2028	0.200805125019903	0.200805125019903\\
72.125	0.20646	0.147177552583853	0.147177552583853\\
72.125	0.21012	0.0866242206149472	0.0866242206149472\\
72.125	0.21378	0.0191451291131788	0.0191451291131788\\
72.125	0.21744	-0.0552597219214572	-0.0552597219214572\\
72.125	0.2211	-0.136590332488948	-0.136590332488948\\
72.125	0.22476	-0.224846702589305	-0.224846702589305\\
72.125	0.22842	-0.320028832222524	-0.320028832222524\\
72.125	0.23208	-0.422136721388609	-0.422136721388609\\
72.125	0.23574	-0.531170370087551	-0.531170370087551\\
72.125	0.2394	-0.647129778319357	-0.647129778319357\\
72.125	0.24306	-0.77001494608403	-0.77001494608403\\
72.125	0.24672	-0.899825873381559	-0.899825873381559\\
72.125	0.25038	-1.03656256021196	-1.03656256021196\\
72.125	0.25404	-1.18022500657521	-1.18022500657521\\
72.125	0.2577	-1.33081321247133	-1.33081321247133\\
72.125	0.26136	-1.48832717790032	-1.48832717790032\\
72.125	0.26502	-1.65276690286216	-1.65276690286216\\
72.125	0.26868	-1.82413238735687	-1.82413238735687\\
72.125	0.27234	-2.00242363138444	-2.00242363138444\\
72.125	0.276	-2.18764063494487	-2.18764063494487\\
72.5	0.093	-1.45832838433776	-1.45832838433776\\
72.5	0.09666	-1.30157644626702	-1.30157644626702\\
72.5	0.10032	-1.15175026772915	-1.15175026772915\\
72.5	0.10398	-1.00884984872415	-1.00884984872415\\
72.5	0.10764	-0.872875189252003	-0.872875189252003\\
72.5	0.1113	-0.743826289312721	-0.743826289312721\\
72.5	0.11496	-0.621703148906301	-0.621703148906301\\
72.5	0.11862	-0.506505768032743	-0.506505768032743\\
72.5	0.12228	-0.398234146692049	-0.398234146692049\\
72.5	0.12594	-0.296888284884218	-0.296888284884218\\
72.5	0.1296	-0.202468182609249	-0.202468182609249\\
72.5	0.13326	-0.114973839867141	-0.114973839867141\\
72.5	0.13692	-0.0344052566578963	-0.0344052566578963\\
72.5	0.14058	0.0392375670184846	0.0392375670184846\\
72.5	0.14424	0.105954631162004	0.105954631162004\\
72.5	0.1479	0.165745935772662	0.165745935772662\\
72.5	0.15156	0.218611480850456	0.218611480850456\\
72.5	0.15522	0.264551266395387	0.264551266395387\\
72.5	0.15888	0.303565292407462	0.303565292407462\\
72.5	0.16254	0.335653558886669	0.335653558886669\\
72.5	0.1662	0.360816065833011	0.360816065833011\\
72.5	0.16986	0.379052813246493	0.379052813246493\\
72.5	0.17352	0.390363801127114	0.390363801127114\\
72.5	0.17718	0.394749029474868	0.394749029474868\\
72.5	0.18084	0.392208498289763	0.392208498289763\\
72.5	0.1845	0.382742207571795	0.382742207571795\\
72.5	0.18816	0.366350157320964	0.366350157320964\\
72.5	0.19182	0.343032347537271	0.343032347537271\\
72.5	0.19548	0.312788778220716	0.312788778220716\\
72.5	0.19914	0.275619449371298	0.275619449371298\\
72.5	0.2028	0.231524360989017	0.231524360989017\\
72.5	0.20646	0.180503513073873	0.180503513073873\\
72.5	0.21012	0.122556905625865	0.122556905625865\\
72.5	0.21378	0.0576845386450024	0.0576845386450024\\
72.5	0.21744	-0.014113587868728	-0.014113587868728\\
72.5	0.2211	-0.0928374739153206	-0.0928374739153206\\
72.5	0.22476	-0.178487119494779	-0.178487119494779\\
72.5	0.22842	-0.271062524607096	-0.271062524607096\\
72.5	0.23208	-0.370563689252275	-0.370563689252275\\
72.5	0.23574	-0.476990613430326	-0.476990613430326\\
72.5	0.2394	-0.590343297141223	-0.590343297141223\\
72.5	0.24306	-0.71062174038499	-0.71062174038499\\
72.5	0.24672	-0.83782594316162	-0.83782594316162\\
72.5	0.25038	-0.971955905471122	-0.971955905471122\\
72.5	0.25404	-1.11301162731347	-1.11301162731347\\
72.5	0.2577	-1.26099310868869	-1.26099310868869\\
72.5	0.26136	-1.41590034959678	-1.41590034959678\\
72.5	0.26502	-1.57773335003772	-1.57773335003772\\
72.5	0.26868	-1.74649211001152	-1.74649211001152\\
72.5	0.27234	-1.92217662951819	-1.92217662951819\\
72.5	0.276	-2.10478690855772	-2.10478690855772\\
72.875	0.093	-1.5062869395265	-1.5062869395265\\
72.875	0.09666	-1.34692827693487	-1.34692827693487\\
72.875	0.10032	-1.1944953738761	-1.1944953738761\\
72.875	0.10398	-1.04898823035019	-1.04898823035019\\
72.875	0.10764	-0.910406846357143	-0.910406846357143\\
72.875	0.1113	-0.778751221896959	-0.778751221896959\\
72.875	0.11496	-0.654021356969638	-0.654021356969638\\
72.875	0.11862	-0.53621725157518	-0.53621725157518\\
72.875	0.12228	-0.425338905713585	-0.425338905713585\\
72.875	0.12594	-0.321386319384851	-0.321386319384851\\
72.875	0.1296	-0.224359492588978	-0.224359492588978\\
72.875	0.13326	-0.13425842532597	-0.13425842532597\\
72.875	0.13692	-0.0510831175958248	-0.0510831175958248\\
72.875	0.14058	0.0251664306014616	0.0251664306014616\\
72.875	0.14424	0.0944902192658814	0.0944902192658814\\
72.875	0.1479	0.156888248397436	0.156888248397436\\
72.875	0.15156	0.212360517996133	0.212360517996133\\
72.875	0.15522	0.260907028061968	0.260907028061968\\
72.875	0.15888	0.302527778594938	0.302527778594938\\
72.875	0.16254	0.337222769595048	0.337222769595048\\
72.875	0.1662	0.364992001062295	0.364992001062295\\
72.875	0.16986	0.385835472996676	0.385835472996676\\
72.875	0.17352	0.399753185398199	0.399753185398199\\
72.875	0.17718	0.406745138266858	0.406745138266858\\
72.875	0.18084	0.406811331602656	0.406811331602656\\
72.875	0.1845	0.399951765405586	0.399951765405586\\
72.875	0.18816	0.386166439675661	0.386166439675661\\
72.875	0.19182	0.365455354412866	0.365455354412866\\
72.875	0.19548	0.337818509617209	0.337818509617209\\
72.875	0.19914	0.303255905288694	0.303255905288694\\
72.875	0.2028	0.261767541427314	0.261767541427314\\
72.875	0.20646	0.213353418033073	0.213353418033073\\
72.875	0.21012	0.158013535105971	0.158013535105971\\
72.875	0.21378	0.0957478926460062	0.0957478926460062\\
72.875	0.21744	0.0265564906531743	0.0265564906531743\\
72.875	0.2211	-0.0495606708725163	-0.0495606708725163\\
72.875	0.22476	-0.132603591931069	-0.132603591931069\\
72.875	0.22842	-0.222572272522484	-0.222572272522484\\
72.875	0.23208	-0.319466712646765	-0.319466712646765\\
72.875	0.23574	-0.42328691230391	-0.42328691230391\\
72.875	0.2394	-0.534032871493912	-0.534032871493912\\
72.875	0.24306	-0.651704590216781	-0.651704590216781\\
72.875	0.24672	-0.776302068472509	-0.776302068472509\\
72.875	0.25038	-0.907825306261097	-0.907825306261097\\
72.875	0.25404	-1.04627430358256	-1.04627430358256\\
72.875	0.2577	-1.19164906043688	-1.19164906043688\\
72.875	0.26136	-1.34394957682405	-1.34394957682405\\
72.875	0.26502	-1.50317585274409	-1.50317585274409\\
72.875	0.26868	-1.669327888197	-1.669327888197\\
72.875	0.27234	-1.84240568318276	-1.84240568318276\\
72.875	0.276	-2.02240923770139	-2.02240923770139\\
73.25	0.093	-1.55472155024607	-1.55472155024607\\
73.25	0.09666	-1.39275616313354	-1.39275616313354\\
73.25	0.10032	-1.23771653555387	-1.23771653555387\\
73.25	0.10398	-1.08960266750706	-1.08960266750706\\
73.25	0.10764	-0.948414558993108	-0.948414558993108\\
73.25	0.1113	-0.814152210012023	-0.814152210012023\\
73.25	0.11496	-0.686815620563801	-0.686815620563801\\
73.25	0.11862	-0.566404790648439	-0.566404790648439\\
73.25	0.12228	-0.452919720265944	-0.452919720265944\\
73.25	0.12594	-0.346360409416309	-0.346360409416309\\
73.25	0.1296	-0.246726858099536	-0.246726858099536\\
73.25	0.13326	-0.154019066315628	-0.154019066315628\\
73.25	0.13692	-0.0682370340645786	-0.0682370340645786\\
73.25	0.14058	0.0106192386536064	0.0106192386536064\\
73.25	0.14424	0.08254975183893	0.08254975183893\\
73.25	0.1479	0.147554505491389	0.147554505491389\\
73.25	0.15156	0.205633499610986	0.205633499610986\\
73.25	0.15522	0.256786734197721	0.256786734197721\\
73.25	0.15888	0.301014209251597	0.301014209251597\\
73.25	0.16254	0.338315924772605	0.338315924772605\\
73.25	0.1662	0.368691880760751	0.368691880760751\\
73.25	0.16986	0.392142077216037	0.392142077216037\\
73.25	0.17352	0.408666514138462	0.408666514138462\\
73.25	0.17718	0.418265191528016	0.418265191528016\\
73.25	0.18084	0.420938109384716	0.420938109384716\\
73.25	0.1845	0.416685267708552	0.416685267708552\\
73.25	0.18816	0.405506666499521	0.405506666499521\\
73.25	0.19182	0.387402305757632	0.387402305757632\\
73.25	0.19548	0.362372185482881	0.362372185482881\\
73.25	0.19914	0.330416305675268	0.330416305675268\\
73.25	0.2028	0.291534666334787	0.291534666334787\\
73.25	0.20646	0.245727267461447	0.245727267461447\\
73.25	0.21012	0.192994109055244	0.192994109055244\\
73.25	0.21378	0.133335191116181	0.133335191116181\\
73.25	0.21744	0.0667505136442514	0.0667505136442514\\
73.25	0.2211	-0.0067599233605371	-0.0067599233605371\\
73.25	0.22476	-0.0871961198981914	-0.0871961198981914\\
73.25	0.22842	-0.174558075968708	-0.174558075968708\\
73.25	0.23208	-0.26884579157208	-0.26884579157208\\
73.25	0.23574	-0.370059266708326	-0.370059266708326\\
73.25	0.2394	-0.478198501377426	-0.478198501377426\\
73.25	0.24306	-0.593263495579389	-0.593263495579389\\
73.25	0.24672	-0.715254249314215	-0.715254249314215\\
73.25	0.25038	-0.844170762581912	-0.844170762581912\\
73.25	0.25404	-0.980013035382461	-0.980013035382461\\
73.25	0.2577	-1.12278106771587	-1.12278106771587\\
73.25	0.26136	-1.27247485958215	-1.27247485958215\\
73.25	0.26502	-1.4290944109813	-1.4290944109813\\
73.25	0.26868	-1.5926397219133	-1.5926397219133\\
73.25	0.27234	-1.76311079237816	-1.76311079237816\\
73.25	0.276	-1.94050762237589	-1.94050762237589\\
73.625	0.093	-1.60363221649646	-1.60363221649646\\
73.625	0.09666	-1.43906010486303	-1.43906010486303\\
73.625	0.10032	-1.28141375276245	-1.28141375276245\\
73.625	0.10398	-1.13069316019474	-1.13069316019474\\
73.625	0.10764	-0.986898327159891	-0.986898327159891\\
73.625	0.1113	-0.850029253657904	-0.850029253657904\\
73.625	0.11496	-0.720085939688778	-0.720085939688778\\
73.625	0.11862	-0.597068385252519	-0.597068385252519\\
73.625	0.12228	-0.48097659034912	-0.48097659034912\\
73.625	0.12594	-0.371810554978585	-0.371810554978585\\
73.625	0.1296	-0.269570279140908	-0.269570279140908\\
73.625	0.13326	-0.1742557628361	-0.1742557628361\\
73.625	0.13692	-0.0858670060641504	-0.0858670060641504\\
73.625	0.14058	-0.00440400882506342	-0.00440400882506342\\
73.625	0.14424	0.0701332288811605	0.0701332288811605\\
73.625	0.1479	0.137744707054523	0.137744707054523\\
73.625	0.15156	0.198430425695021	0.198430425695021\\
73.625	0.15522	0.25219038480266	0.25219038480266\\
73.625	0.15888	0.29902458437743	0.29902458437743\\
73.625	0.16254	0.338933024419344	0.338933024419344\\
73.625	0.1662	0.371915704928392	0.371915704928392\\
73.625	0.16986	0.397972625904576	0.397972625904576\\
73.625	0.17352	0.417103787347903	0.417103787347903\\
73.625	0.17718	0.429309189258364	0.429309189258364\\
73.625	0.18084	0.434588831635965	0.434588831635965\\
73.625	0.1845	0.432942714480696	0.432942714480696\\
73.625	0.18816	0.424370837792575	0.424370837792575\\
73.625	0.19182	0.408873201571584	0.408873201571584\\
73.625	0.19548	0.386449805817731	0.386449805817731\\
73.625	0.19914	0.35710065053102	0.35710065053102\\
73.625	0.2028	0.320825735711441	0.320825735711441\\
73.625	0.20646	0.277625061359004	0.277625061359004\\
73.625	0.21012	0.227498627473702	0.227498627473702\\
73.625	0.21378	0.170446434055542	0.170446434055542\\
73.625	0.21744	0.10646848110451	0.10646848110451\\
73.625	0.2211	0.035564768620624	0.035564768620624\\
73.625	0.22476	-0.0422647033961248	-0.0422647033961248\\
73.625	0.22842	-0.127019934945739	-0.127019934945739\\
73.625	0.23208	-0.21870092602822	-0.21870092602822\\
73.625	0.23574	-0.317307676643553	-0.317307676643553\\
73.625	0.2394	-0.422840186791758	-0.422840186791758\\
73.625	0.24306	-0.535298456472823	-0.535298456472823\\
73.625	0.24672	-0.654682485686747	-0.654682485686747\\
73.625	0.25038	-0.780992274433538	-0.780992274433538\\
73.625	0.25404	-0.914227822713189	-0.914227822713189\\
73.625	0.2577	-1.0543891305257	-1.0543891305257\\
73.625	0.26136	-1.20147619787107	-1.20147619787107\\
73.625	0.26502	-1.35548902474932	-1.35548902474932\\
73.625	0.26868	-1.51642761116042	-1.51642761116042\\
73.625	0.27234	-1.68429195710438	-1.68429195710438\\
73.625	0.276	-1.85908206258121	-1.85908206258121\\
74	0.093	-1.65301893827767	-1.65301893827767\\
74	0.09666	-1.48584010212333	-1.48584010212333\\
74	0.10032	-1.32558702550186	-1.32558702550186\\
74	0.10398	-1.17225970841324	-1.17225970841324\\
74	0.10764	-1.0258581508575	-1.0258581508575\\
74	0.1113	-0.886382352834607	-0.886382352834607\\
74	0.11496	-0.753832314344582	-0.753832314344582\\
74	0.11862	-0.62820803538742	-0.62820803538742\\
74	0.12228	-0.509509515963117	-0.509509515963117\\
74	0.12594	-0.397736756071681	-0.397736756071681\\
74	0.1296	-0.292889755713104	-0.292889755713104\\
74	0.13326	-0.194968514887392	-0.194968514887392\\
74	0.13692	-0.103973033594542	-0.103973033594542\\
74	0.14058	-0.0199033118345531	-0.0199033118345531\\
74	0.14424	0.0572406503925711	0.0572406503925711\\
74	0.1479	0.127458853086837	0.127458853086837\\
74	0.15156	0.190751296248235	0.190751296248235\\
74	0.15522	0.247117979876775	0.247117979876775\\
74	0.15888	0.296558903972451	0.296558903972451\\
74	0.16254	0.339074068535263	0.339074068535263\\
74	0.1662	0.374663473565213	0.374663473565213\\
74	0.16986	0.4033271190623	0.4033271190623\\
74	0.17352	0.425065005026529	0.425065005026529\\
74	0.17718	0.439877131457887	0.439877131457887\\
74	0.18084	0.447763498356387	0.447763498356387\\
74	0.1845	0.448724105722027	0.448724105722027\\
74	0.18816	0.442758953554801	0.442758953554801\\
74	0.19182	0.429868041854716	0.429868041854716\\
74	0.19548	0.410051370621765	0.410051370621765\\
74	0.19914	0.383308939855956	0.383308939855956\\
74	0.2028	0.349640749557279	0.349640749557279\\
74	0.20646	0.30904679972574	0.30904679972574\\
74	0.21012	0.261527090361341	0.261527090361341\\
74	0.21378	0.207081621464079	0.207081621464079\\
74	0.21744	0.145710393033957	0.145710393033957\\
74	0.2211	0.0774134050709687	0.0774134050709687\\
74	0.22476	0.00219065757511494	0.00219065757511494\\
74	0.22842	-0.0799578494535975	-0.0799578494535975\\
74	0.23208	-0.169032116015169	-0.169032116015169\\
74	0.23574	-0.265032142109611	-0.265032142109611\\
74	0.2394	-0.367957927736906	-0.367957927736906\\
74	0.24306	-0.477809472897066	-0.477809472897066\\
74	0.24672	-0.594586777590095	-0.594586777590095\\
74	0.25038	-0.718289841815981	-0.718289841815981\\
74	0.25404	-0.848918665574733	-0.848918665574733\\
74	0.2577	-0.986473248866343	-0.986473248866343\\
74	0.26136	-1.13095359169082	-1.13095359169082\\
74	0.26502	-1.28235969404815	-1.28235969404815\\
74	0.26868	-1.44069155593835	-1.44069155593835\\
74	0.27234	-1.60594917736141	-1.60594917736141\\
74	0.276	-1.77813255831734	-1.77813255831734\\
};
\end{axis}

\begin{axis}[%
width=2.616cm,
height=2.517cm,
at={(6.484cm,13.986cm)},
scale only axis,
xmin=56,
xmax=74,
tick align=outside,
xlabel style={font=\color{white!15!black}},
xlabel={$L_{cut}$},
ymin=0.093,
ymax=0.276,
ylabel style={font=\color{white!15!black}},
ylabel={$D_{rlx}$},
zmin=-519.510499343224,
zmax=54.6628885544483,
zlabel style={font=\color{white!15!black}},
zlabel={$x_4$},
view={-140}{50},
axis background/.style={fill=white},
xmajorgrids,
ymajorgrids,
zmajorgrids,
legend style={at={(1.03,1)}, anchor=north west, legend cell align=left, align=left, draw=white!15!black}
]
\addplot3[only marks, mark=*, mark options={}, mark size=1.5000pt, color=mycolor1, fill=mycolor1] table[row sep=crcr]{%
x	y	z\\
74	0.123	-39.0721401660649\\
72	0.113	-47.7749008694261\\
61	0.095	0.193368675424212\\
56	0.093	0.282324301664\\
};
\addlegendentry{data1}

\addplot3[only marks, mark=*, mark options={}, mark size=1.5000pt, color=mycolor2, fill=mycolor2] table[row sep=crcr]{%
x	y	z\\
67	0.276	-284.827879534065\\
66	0.255	-161.402461014926\\
62	0.209	-65.2263092513417\\
57	0.193	-62.6946226828375\\
};
\addlegendentry{data2}

\addplot3[only marks, mark=*, mark options={}, mark size=1.5000pt, color=black, fill=black] table[row sep=crcr]{%
x	y	z\\
69	0.104	-38.9467438256712\\
};
\addlegendentry{data3}

\addplot3[only marks, mark=*, mark options={}, mark size=1.5000pt, color=black, fill=black] table[row sep=crcr]{%
x	y	z\\
64	0.23	-100.740842510344\\
};
\addlegendentry{data4}


\addplot3[%
surf,
fill opacity=0.7, shader=interp, colormap={mymap}{[1pt] rgb(0pt)=(1,0.905882,0); rgb(1pt)=(1,0.901964,0); rgb(2pt)=(1,0.898051,0); rgb(3pt)=(1,0.894144,0); rgb(4pt)=(1,0.890243,0); rgb(5pt)=(1,0.886349,0); rgb(6pt)=(1,0.88246,0); rgb(7pt)=(1,0.878577,0); rgb(8pt)=(1,0.8747,0); rgb(9pt)=(1,0.870829,0); rgb(10pt)=(1,0.866964,0); rgb(11pt)=(1,0.863106,0); rgb(12pt)=(1,0.859253,0); rgb(13pt)=(1,0.855406,0); rgb(14pt)=(1,0.851566,0); rgb(15pt)=(1,0.847732,0); rgb(16pt)=(1,0.843903,0); rgb(17pt)=(1,0.840081,0); rgb(18pt)=(1,0.836265,0); rgb(19pt)=(1,0.832455,0); rgb(20pt)=(1,0.828652,0); rgb(21pt)=(1,0.824854,0); rgb(22pt)=(1,0.821063,0); rgb(23pt)=(1,0.817278,0); rgb(24pt)=(1,0.8135,0); rgb(25pt)=(1,0.809727,0); rgb(26pt)=(1,0.805961,0); rgb(27pt)=(1,0.8022,0); rgb(28pt)=(1,0.798445,0); rgb(29pt)=(1,0.794696,0); rgb(30pt)=(1,0.790953,0); rgb(31pt)=(1,0.787215,0); rgb(32pt)=(1,0.783484,0); rgb(33pt)=(1,0.779758,0); rgb(34pt)=(1,0.776038,0); rgb(35pt)=(1,0.772324,0); rgb(36pt)=(1,0.768615,0); rgb(37pt)=(1,0.764913,0); rgb(38pt)=(1,0.761217,0); rgb(39pt)=(1,0.757527,0); rgb(40pt)=(1,0.753843,0); rgb(41pt)=(1,0.750165,0); rgb(42pt)=(1,0.746493,0); rgb(43pt)=(1,0.742827,0); rgb(44pt)=(1,0.739167,0); rgb(45pt)=(1,0.735514,0); rgb(46pt)=(1,0.731867,0); rgb(47pt)=(1,0.728226,0); rgb(48pt)=(1,0.724591,0); rgb(49pt)=(1,0.720963,0); rgb(50pt)=(1,0.717341,0); rgb(51pt)=(1,0.713725,0); rgb(52pt)=(0.999994,0.710077,0); rgb(53pt)=(0.999974,0.706363,0); rgb(54pt)=(0.999942,0.702592,0); rgb(55pt)=(0.999898,0.698775,0); rgb(56pt)=(0.999841,0.694921,0); rgb(57pt)=(0.999771,0.691039,0); rgb(58pt)=(0.99969,0.687139,0); rgb(59pt)=(0.999596,0.68323,0); rgb(60pt)=(0.99949,0.679323,0); rgb(61pt)=(0.999372,0.675427,0); rgb(62pt)=(0.999242,0.67155,0); rgb(63pt)=(0.9991,0.667704,0); rgb(64pt)=(0.998946,0.663897,0); rgb(65pt)=(0.998781,0.660138,0); rgb(66pt)=(0.998605,0.656439,0); rgb(67pt)=(0.998416,0.652807,0); rgb(68pt)=(0.998217,0.649253,0); rgb(69pt)=(0.998006,0.645786,0); rgb(70pt)=(0.997785,0.642416,0); rgb(71pt)=(0.997552,0.639152,0); rgb(72pt)=(0.997308,0.636004,0); rgb(73pt)=(0.997053,0.632982,0); rgb(74pt)=(0.996788,0.630095,0); rgb(75pt)=(0.996512,0.627352,0); rgb(76pt)=(0.996226,0.624763,0); rgb(77pt)=(0.995851,0.622329,0); rgb(78pt)=(0.99494,0.619997,0); rgb(79pt)=(0.99345,0.617753,0); rgb(80pt)=(0.991419,0.61559,0); rgb(81pt)=(0.988885,0.613503,0); rgb(82pt)=(0.985886,0.611486,0); rgb(83pt)=(0.98246,0.609532,0); rgb(84pt)=(0.978643,0.607636,0); rgb(85pt)=(0.974475,0.605791,0); rgb(86pt)=(0.969992,0.603992,0); rgb(87pt)=(0.965232,0.602233,0); rgb(88pt)=(0.960233,0.600507,0); rgb(89pt)=(0.955033,0.598808,0); rgb(90pt)=(0.949669,0.59713,0); rgb(91pt)=(0.94418,0.595468,0); rgb(92pt)=(0.938602,0.593815,0); rgb(93pt)=(0.932974,0.592166,0); rgb(94pt)=(0.927333,0.590513,0); rgb(95pt)=(0.921717,0.588852,0); rgb(96pt)=(0.916164,0.587176,0); rgb(97pt)=(0.910711,0.585479,0); rgb(98pt)=(0.905397,0.583755,0); rgb(99pt)=(0.900258,0.581999,0); rgb(100pt)=(0.895333,0.580203,0); rgb(101pt)=(0.890659,0.578362,0); rgb(102pt)=(0.886275,0.576471,0); rgb(103pt)=(0.882047,0.574545,0); rgb(104pt)=(0.877819,0.572608,0); rgb(105pt)=(0.873592,0.57066,0); rgb(106pt)=(0.869366,0.568701,0); rgb(107pt)=(0.865143,0.566733,0); rgb(108pt)=(0.860924,0.564756,0); rgb(109pt)=(0.856708,0.562771,0); rgb(110pt)=(0.852497,0.560778,0); rgb(111pt)=(0.848292,0.558779,0); rgb(112pt)=(0.844092,0.556774,0); rgb(113pt)=(0.8399,0.554763,0); rgb(114pt)=(0.835716,0.552749,0); rgb(115pt)=(0.831541,0.55073,0); rgb(116pt)=(0.827374,0.548709,0); rgb(117pt)=(0.823219,0.546686,0); rgb(118pt)=(0.819074,0.54466,0); rgb(119pt)=(0.81494,0.542635,0); rgb(120pt)=(0.81082,0.540609,0); rgb(121pt)=(0.806712,0.538584,0); rgb(122pt)=(0.802619,0.53656,0); rgb(123pt)=(0.798541,0.534539,0); rgb(124pt)=(0.794478,0.532521,0); rgb(125pt)=(0.790431,0.530506,0); rgb(126pt)=(0.786402,0.528496,0); rgb(127pt)=(0.782391,0.526491,0); rgb(128pt)=(0.77841,0.524489,0); rgb(129pt)=(0.774523,0.522478,0); rgb(130pt)=(0.770731,0.520455,0); rgb(131pt)=(0.767022,0.518424,0); rgb(132pt)=(0.763384,0.516385,0); rgb(133pt)=(0.759804,0.514339,0); rgb(134pt)=(0.756272,0.51229,0); rgb(135pt)=(0.752775,0.510237,0); rgb(136pt)=(0.749302,0.508182,0); rgb(137pt)=(0.74584,0.506128,0); rgb(138pt)=(0.742378,0.504075,0); rgb(139pt)=(0.738904,0.502025,0); rgb(140pt)=(0.735406,0.499979,0); rgb(141pt)=(0.731872,0.49794,0); rgb(142pt)=(0.72829,0.495909,0); rgb(143pt)=(0.724649,0.493887,0); rgb(144pt)=(0.720936,0.491875,0); rgb(145pt)=(0.71714,0.489876,0); rgb(146pt)=(0.713249,0.487891,0); rgb(147pt)=(0.709251,0.485921,0); rgb(148pt)=(0.705134,0.483968,0); rgb(149pt)=(0.700887,0.482033,0); rgb(150pt)=(0.696497,0.480118,0); rgb(151pt)=(0.691952,0.478225,0); rgb(152pt)=(0.687242,0.476355,0); rgb(153pt)=(0.682353,0.47451,0); rgb(154pt)=(0.677195,0.472696,0); rgb(155pt)=(0.6717,0.470916,0); rgb(156pt)=(0.665891,0.469169,0); rgb(157pt)=(0.659791,0.46745,0); rgb(158pt)=(0.653423,0.465756,0); rgb(159pt)=(0.64681,0.464084,0); rgb(160pt)=(0.639976,0.462432,0); rgb(161pt)=(0.632943,0.460795,0); rgb(162pt)=(0.625734,0.459171,0); rgb(163pt)=(0.618373,0.457556,0); rgb(164pt)=(0.610882,0.455948,0); rgb(165pt)=(0.603284,0.454343,0); rgb(166pt)=(0.595604,0.452737,0); rgb(167pt)=(0.587863,0.451129,0); rgb(168pt)=(0.580084,0.449514,0); rgb(169pt)=(0.572292,0.447889,0); rgb(170pt)=(0.564508,0.446252,0); rgb(171pt)=(0.556756,0.444599,0); rgb(172pt)=(0.549059,0.442927,0); rgb(173pt)=(0.54144,0.441232,0); rgb(174pt)=(0.533922,0.439512,0); rgb(175pt)=(0.526529,0.437764,0); rgb(176pt)=(0.519282,0.435983,0); rgb(177pt)=(0.512206,0.434168,0); rgb(178pt)=(0.505323,0.432315,0); rgb(179pt)=(0.498628,0.430422,3.92506e-06); rgb(180pt)=(0.491973,0.428504,3.49981e-05); rgb(181pt)=(0.485331,0.426562,9.63073e-05); rgb(182pt)=(0.478704,0.424596,0.000186979); rgb(183pt)=(0.472096,0.422609,0.000306141); rgb(184pt)=(0.465508,0.420599,0.00045292); rgb(185pt)=(0.458942,0.418567,0.000626441); rgb(186pt)=(0.452401,0.416515,0.000825833); rgb(187pt)=(0.445885,0.414441,0.00105022); rgb(188pt)=(0.439399,0.412348,0.00129873); rgb(189pt)=(0.432942,0.410234,0.00157049); rgb(190pt)=(0.426518,0.408102,0.00186463); rgb(191pt)=(0.420129,0.40595,0.00218028); rgb(192pt)=(0.413777,0.40378,0.00251655); rgb(193pt)=(0.407464,0.401592,0.00287258); rgb(194pt)=(0.401191,0.399386,0.00324749); rgb(195pt)=(0.394962,0.397164,0.00364042); rgb(196pt)=(0.388777,0.394925,0.00405048); rgb(197pt)=(0.38264,0.39267,0.00447681); rgb(198pt)=(0.376552,0.390399,0.00491852); rgb(199pt)=(0.370516,0.388113,0.00537476); rgb(200pt)=(0.364532,0.385812,0.00584464); rgb(201pt)=(0.358605,0.383497,0.00632729); rgb(202pt)=(0.352735,0.381168,0.00682184); rgb(203pt)=(0.346925,0.378826,0.00732741); rgb(204pt)=(0.341176,0.376471,0.00784314); rgb(205pt)=(0.335485,0.374093,0.00847245); rgb(206pt)=(0.329843,0.371682,0.00930909); rgb(207pt)=(0.324249,0.369242,0.0103377); rgb(208pt)=(0.318701,0.366772,0.0115428); rgb(209pt)=(0.313198,0.364275,0.0129091); rgb(210pt)=(0.307739,0.361753,0.0144211); rgb(211pt)=(0.302322,0.359206,0.0160634); rgb(212pt)=(0.296945,0.356637,0.0178207); rgb(213pt)=(0.291607,0.354048,0.0196776); rgb(214pt)=(0.286307,0.35144,0.0216186); rgb(215pt)=(0.281043,0.348814,0.0236284); rgb(216pt)=(0.275813,0.346172,0.0256916); rgb(217pt)=(0.270616,0.343517,0.0277927); rgb(218pt)=(0.265451,0.340849,0.0299163); rgb(219pt)=(0.260317,0.33817,0.0320472); rgb(220pt)=(0.25521,0.335482,0.0341698); rgb(221pt)=(0.250131,0.332786,0.0362688); rgb(222pt)=(0.245078,0.330085,0.0383287); rgb(223pt)=(0.240048,0.327379,0.0403343); rgb(224pt)=(0.235042,0.324671,0.04227); rgb(225pt)=(0.230056,0.321962,0.0441205); rgb(226pt)=(0.22509,0.319254,0.0458704); rgb(227pt)=(0.220142,0.316548,0.0475043); rgb(228pt)=(0.215212,0.313846,0.0490067); rgb(229pt)=(0.210296,0.311149,0.0503624); rgb(230pt)=(0.205395,0.308459,0.0515759); rgb(231pt)=(0.200514,0.305763,0.052757); rgb(232pt)=(0.195655,0.303061,0.0539242); rgb(233pt)=(0.190817,0.300353,0.0550763); rgb(234pt)=(0.186001,0.297639,0.0562123); rgb(235pt)=(0.181207,0.294918,0.0573313); rgb(236pt)=(0.176434,0.292191,0.0584321); rgb(237pt)=(0.171685,0.289458,0.0595136); rgb(238pt)=(0.166957,0.286719,0.060575); rgb(239pt)=(0.162252,0.283973,0.0616151); rgb(240pt)=(0.15757,0.281221,0.0626328); rgb(241pt)=(0.152911,0.278463,0.0636271); rgb(242pt)=(0.148275,0.275699,0.0645971); rgb(243pt)=(0.143663,0.272929,0.0655416); rgb(244pt)=(0.139074,0.270152,0.0664596); rgb(245pt)=(0.134508,0.26737,0.06735); rgb(246pt)=(0.129967,0.264581,0.0682118); rgb(247pt)=(0.125449,0.261787,0.0690441); rgb(248pt)=(0.120956,0.258986,0.0698456); rgb(249pt)=(0.116487,0.25618,0.0706154); rgb(250pt)=(0.112043,0.253367,0.0713525); rgb(251pt)=(0.107623,0.250549,0.0720557); rgb(252pt)=(0.103229,0.247724,0.0727241); rgb(253pt)=(0.0988592,0.244894,0.0733566); rgb(254pt)=(0.0945149,0.242058,0.0739522); rgb(255pt)=(0.0901961,0.239216,0.0745098)}, mesh/rows=49]
table[row sep=crcr, point meta=\thisrow{c}] {%
%
x	y	z	c\\
56	0.093	2.63618490932626	2.63618490932626\\
56	0.09666	8.08469384065845	8.08469384065845\\
56	0.10032	12.8845724611178	12.8845724611178\\
56	0.10398	17.0358207707043	17.0358207707043\\
56	0.10764	20.5384387694181	20.5384387694181\\
56	0.1113	23.3924264572591	23.3924264572591\\
56	0.11496	25.5977838342274	25.5977838342274\\
56	0.11862	27.1545109003229	27.1545109003229\\
56	0.12228	28.0626076555454	28.0626076555454\\
56	0.12594	28.3220740998954	28.3220740998954\\
56	0.1296	27.9329102333726	27.9329102333726\\
56	0.13326	26.8951160559768	26.8951160559768\\
56	0.13692	25.2086915677084	25.2086915677084\\
56	0.14058	22.8736367685671	22.8736367685671\\
56	0.14424	19.8899516585528	19.8899516585528\\
56	0.1479	16.2576362376664	16.2576362376664\\
56	0.15156	11.9766905059066	11.9766905059066\\
56	0.15522	7.04711446327451	7.04711446327451\\
56	0.15888	1.46890810976902	1.46890810976902\\
56	0.16254	-4.75792855460895	-4.75792855460895\\
56	0.1662	-11.6333955298594	-11.6333955298594\\
56	0.16986	-19.157492815983	-19.157492815983\\
56	0.17352	-27.3302204129794	-27.3302204129794\\
56	0.17718	-36.1515783208482	-36.1515783208482\\
56	0.18084	-45.6215665395905	-45.6215665395905\\
56	0.1845	-55.7401850692049	-55.7401850692049\\
56	0.18816	-66.5074339096922	-66.5074339096922\\
56	0.19182	-77.923313061052	-77.923313061052\\
56	0.19548	-89.9878225232856	-89.9878225232856\\
56	0.19914	-102.700962296391	-102.700962296391\\
56	0.2028	-116.062732380369	-116.062732380369\\
56	0.20646	-130.073132775221	-130.073132775221\\
56	0.21012	-144.732163480945	-144.732163480945\\
56	0.21378	-160.039824497542	-160.039824497542\\
56	0.21744	-175.996115825012	-175.996115825012\\
56	0.2211	-192.601037463354	-192.601037463354\\
56	0.22476	-209.854589412569	-209.854589412569\\
56	0.22842	-227.756771672657	-227.756771672657\\
56	0.23208	-246.307584243618	-246.307584243618\\
56	0.23574	-265.507027125451	-265.507027125451\\
56	0.2394	-285.355100318158	-285.355100318158\\
56	0.24306	-305.851803821737	-305.851803821737\\
56	0.24672	-326.997137636189	-326.997137636189\\
56	0.25038	-348.791101761514	-348.791101761514\\
56	0.25404	-371.233696197711	-371.233696197711\\
56	0.2577	-394.324920944781	-394.324920944781\\
56	0.26136	-418.064776002724	-418.064776002724\\
56	0.26502	-442.45326137154	-442.45326137154\\
56	0.26868	-467.490377051229	-467.490377051229\\
56	0.27234	-493.17612304179	-493.17612304179\\
56	0.276	-519.510499343224	-519.510499343224\\
56.375	0.093	2.61837991449133	2.61837991449133\\
56.375	0.09666	8.2720821567211	8.2720821567211\\
56.375	0.10032	13.277154088078	13.277154088078\\
56.375	0.10398	17.6335957085619	17.6335957085619\\
56.375	0.10764	21.3414070181733	21.3414070181733\\
56.375	0.1113	24.4005880169119	24.4005880169119\\
56.375	0.11496	26.8111387047775	26.8111387047775\\
56.375	0.11862	28.5730590817704	28.5730590817704\\
56.375	0.12228	29.6863491478907	29.6863491478907\\
56.375	0.12594	30.151008903138	30.151008903138\\
56.375	0.1296	29.9670383475126	29.9670383475126\\
56.375	0.13326	29.1344374810146	29.1344374810146\\
56.375	0.13692	27.6532063036436	27.6532063036436\\
56.375	0.14058	25.5233448153998	25.5233448153998\\
56.375	0.14424	22.7448530162832	22.7448530162832\\
56.375	0.1479	19.3177309062941	19.3177309062941\\
56.375	0.15156	15.2419784854317	15.2419784854317\\
56.375	0.15522	10.5175957536969	10.5175957536969\\
56.375	0.15888	5.14458271108924	5.14458271108924\\
56.375	0.16254	-0.877060642390916	-0.877060642390916\\
56.375	0.1662	-7.54733430674446	-7.54733430674446\\
56.375	0.16986	-14.8662382819703	-14.8662382819703\\
56.375	0.17352	-22.8337725680689	-22.8337725680689\\
56.375	0.17718	-31.4499371650408	-31.4499371650408\\
56.375	0.18084	-40.7147320728848	-40.7147320728848\\
56.375	0.1845	-50.6281572916018	-50.6281572916018\\
56.375	0.18816	-61.1902128211917	-61.1902128211917\\
56.375	0.19182	-72.4008986616542	-72.4008986616542\\
56.375	0.19548	-84.2602148129899	-84.2602148129899\\
56.375	0.19914	-96.7681612751985	-96.7681612751985\\
56.375	0.2028	-109.924738048279	-109.924738048279\\
56.375	0.20646	-123.729945132233	-123.729945132233\\
56.375	0.21012	-138.18378252706	-138.18378252706\\
56.375	0.21378	-153.286250232759	-153.286250232759\\
56.375	0.21744	-169.037348249331	-169.037348249331\\
56.375	0.2211	-185.437076576776	-185.437076576776\\
56.375	0.22476	-202.485435215094	-202.485435215094\\
56.375	0.22842	-220.182424164284	-220.182424164284\\
56.375	0.23208	-238.528043424347	-238.528043424347\\
56.375	0.23574	-257.522292995284	-257.522292995284\\
56.375	0.2394	-277.165172877093	-277.165172877093\\
56.375	0.24306	-297.456683069774	-297.456683069774\\
56.375	0.24672	-318.396823573328	-318.396823573328\\
56.375	0.25038	-339.985594387756	-339.985594387756\\
56.375	0.25404	-362.222995513055	-362.222995513055\\
56.375	0.2577	-385.109026949228	-385.109026949228\\
56.375	0.26136	-408.643688696274	-408.643688696274\\
56.375	0.26502	-432.826980754192	-432.826980754192\\
56.375	0.26868	-457.658903122983	-457.658903122983\\
56.375	0.27234	-483.139455802647	-483.139455802647\\
56.375	0.276	-509.268638793184	-509.268638793184\\
56.75	0.093	2.46902754295738	2.46902754295738\\
56.75	0.09666	8.32792309608428	8.32792309608428\\
56.75	0.10032	13.5381883383388	13.5381883383388\\
56.75	0.10398	18.0998232697204	18.0998232697204\\
56.75	0.10764	22.0128278902292	22.0128278902292\\
56.75	0.1113	25.2772021998652	25.2772021998652\\
56.75	0.11496	27.8929461986284	27.8929461986284\\
56.75	0.11862	29.8600598865189	29.8600598865189\\
56.75	0.12228	31.1785432635365	31.1785432635365\\
56.75	0.12594	31.8483963296814	31.8483963296814\\
56.75	0.1296	31.8696190849536	31.8696190849536\\
56.75	0.13326	31.242211529353	31.242211529353\\
56.75	0.13692	29.9661736628795	29.9661736628795\\
56.75	0.14058	28.0415054855333	28.0415054855333\\
56.75	0.14424	25.4682069973143	25.4682069973143\\
56.75	0.1479	22.2462781982227	22.2462781982227\\
56.75	0.15156	18.3757190882577	18.3757190882577\\
56.75	0.15522	13.8565296674208	13.8565296674208\\
56.75	0.15888	8.68870993571045	8.68870993571045\\
56.75	0.16254	2.87225989312765	2.87225989312765\\
56.75	0.1662	-3.59282046032808	-3.59282046032808\\
56.75	0.16986	-10.7065311246565	-10.7065311246565\\
56.75	0.17352	-18.4688720998578	-18.4688720998578\\
56.75	0.17718	-26.8798433859318	-26.8798433859318\\
56.75	0.18084	-35.9394449828785	-35.9394449828785\\
56.75	0.1845	-45.6476768906982	-45.6476768906982\\
56.75	0.18816	-56.0045391093907	-56.0045391093907\\
56.75	0.19182	-67.0100316389559	-67.0100316389559\\
56.75	0.19548	-78.6641544793938	-78.6641544793938\\
56.75	0.19914	-90.9669076307046	-90.9669076307046\\
56.75	0.2028	-103.918291092888	-103.918291092888\\
56.75	0.20646	-117.518304865945	-117.518304865945\\
56.75	0.21012	-131.766948949873	-131.766948949873\\
56.75	0.21378	-146.664223344675	-146.664223344675\\
56.75	0.21744	-162.21012805035	-162.21012805035\\
56.75	0.2211	-178.404663066897	-178.404663066897\\
56.75	0.22476	-195.247828394317	-195.247828394317\\
56.75	0.22842	-212.73962403261	-212.73962403261\\
56.75	0.23208	-230.880049981776	-230.880049981776\\
56.75	0.23574	-249.669106241814	-249.669106241814\\
56.75	0.2394	-269.106792812726	-269.106792812726\\
56.75	0.24306	-289.19310969451	-289.19310969451\\
56.75	0.24672	-309.928056887167	-309.928056887167\\
56.75	0.25038	-331.311634390697	-331.311634390697\\
56.75	0.25404	-353.343842205099	-353.343842205099\\
56.75	0.2577	-376.024680330374	-376.024680330374\\
56.75	0.26136	-399.354148766522	-399.354148766522\\
56.75	0.26502	-423.332247513543	-423.332247513543\\
56.75	0.26868	-447.958976571436	-447.958976571436\\
56.75	0.27234	-473.234335940202	-473.234335940202\\
56.75	0.276	-499.158325619841	-499.158325619841\\
57.125	0.093	2.18812779472395	2.18812779472395\\
57.125	0.09666	8.25221665874844	8.25221665874844\\
57.125	0.10032	13.6676752119006	13.6676752119006\\
57.125	0.10398	18.4345034541795	18.4345034541795\\
57.125	0.10764	22.5527013855859	22.5527013855859\\
57.125	0.1113	26.0222690061195	26.0222690061195\\
57.125	0.11496	28.8432063157803	28.8432063157803\\
57.125	0.11862	31.0155133145683	31.0155133145683\\
57.125	0.12228	32.5391900024836	32.5391900024836\\
57.125	0.12594	33.4142363795258	33.4142363795258\\
57.125	0.1296	33.6406524456955	33.6406524456955\\
57.125	0.13326	33.2184382009925	33.2184382009925\\
57.125	0.13692	32.1475936454166	32.1475936454166\\
57.125	0.14058	30.4281187789678	30.4281187789678\\
57.125	0.14424	28.0600136016463	28.0600136016463\\
57.125	0.1479	25.0432781134519	25.0432781134519\\
57.125	0.15156	21.3779123143852	21.3779123143852\\
57.125	0.15522	17.0639162044451	17.0639162044451\\
57.125	0.15888	12.1012897836326	12.1012897836326\\
57.125	0.16254	6.49003305194719	6.49003305194719\\
57.125	0.1662	0.230146009389273	0.230146009389273\\
57.125	0.16986	-6.67837134404181	-6.67837134404181\\
57.125	0.17352	-14.2355190083457	-14.2355190083457\\
57.125	0.17718	-22.441296983522	-22.441296983522\\
57.125	0.18084	-31.2957052695713	-31.2957052695713\\
57.125	0.1845	-40.7987438664936	-40.7987438664936\\
57.125	0.18816	-50.9504127742883	-50.9504127742883\\
57.125	0.19182	-61.7507119929556	-61.7507119929556\\
57.125	0.19548	-73.1996415224962	-73.1996415224962\\
57.125	0.19914	-85.2972013629096	-85.2972013629096\\
57.125	0.2028	-98.0433915141955	-98.0433915141955\\
57.125	0.20646	-111.438211976355	-111.438211976355\\
57.125	0.21012	-125.481662749386	-125.481662749386\\
57.125	0.21378	-140.17374383329	-140.17374383329\\
57.125	0.21744	-155.514455228067	-155.514455228067\\
57.125	0.2211	-171.503796933717	-171.503796933717\\
57.125	0.22476	-188.14176895024	-188.14176895024\\
57.125	0.22842	-205.428371277636	-205.428371277636\\
57.125	0.23208	-223.363603915903	-223.363603915903\\
57.125	0.23574	-241.947466865045	-241.947466865045\\
57.125	0.2394	-261.179960125058	-261.179960125058\\
57.125	0.24306	-281.061083695945	-281.061083695945\\
57.125	0.24672	-301.590837577704	-301.590837577704\\
57.125	0.25038	-322.769221770337	-322.769221770337\\
57.125	0.25404	-344.596236273841	-344.596236273841\\
57.125	0.2577	-367.071881088219	-367.071881088219\\
57.125	0.26136	-390.196156213469	-390.196156213469\\
57.125	0.26502	-413.969061649593	-413.969061649593\\
57.125	0.26868	-438.390597396589	-438.390597396589\\
57.125	0.27234	-463.460763454457	-463.460763454457\\
57.125	0.276	-489.179559823199	-489.179559823199\\
57.5	0.093	1.77568066979174	1.77568066979174\\
57.5	0.09666	8.04496284471381	8.04496284471381\\
57.5	0.10032	13.6656147087633	13.6656147087633\\
57.5	0.10398	18.6376362619401	18.6376362619401\\
57.5	0.10764	22.9610275042436	22.9610275042436\\
57.5	0.1113	26.6357884356747	26.6357884356747\\
57.5	0.11496	29.6619190562331	29.6619190562331\\
57.5	0.11862	32.0394193659187	32.0394193659187\\
57.5	0.12228	33.7682893647313	33.7682893647313\\
57.5	0.12594	34.8485290526712	34.8485290526712\\
57.5	0.1296	35.2801384297385	35.2801384297385\\
57.5	0.13326	35.0631174959331	35.0631174959331\\
57.5	0.13692	34.1974662512545	34.1974662512545\\
57.5	0.14058	32.6831846957035	32.6831846957035\\
57.5	0.14424	30.5202728292792	30.5202728292792\\
57.5	0.1479	27.7087306519826	27.7087306519826\\
57.5	0.15156	24.2485581638132	24.2485581638132\\
57.5	0.15522	20.1397553647705	20.1397553647705\\
57.5	0.15888	15.3823222548558	15.3823222548558\\
57.5	0.16254	9.97625883406772	9.97625883406772\\
57.5	0.1662	3.92156510240716	3.92156510240716\\
57.5	0.16986	-2.78175894012611	-2.78175894012611\\
57.5	0.17352	-10.1337132935327	-10.1337132935327\\
57.5	0.17718	-18.1342979578116	-18.1342979578116\\
57.5	0.18084	-26.783512932963	-26.783512932963\\
57.5	0.1845	-36.0813582189876	-36.0813582189876\\
57.5	0.18816	-46.0278338158849	-46.0278338158849\\
57.5	0.19182	-56.6229397236553	-56.6229397236553\\
57.5	0.19548	-67.8666759422981	-67.8666759422981\\
57.5	0.19914	-79.7590424718137	-79.7590424718137\\
57.5	0.2028	-92.3000393122022	-92.3000393122022\\
57.5	0.20646	-105.489666463463	-105.489666463463\\
57.5	0.21012	-119.327923925598	-119.327923925598\\
57.5	0.21378	-133.814811698604	-133.814811698604\\
57.5	0.21744	-148.950329782484	-148.950329782484\\
57.5	0.2211	-164.734478177236	-164.734478177236\\
57.5	0.22476	-181.167256882861	-181.167256882861\\
57.5	0.22842	-198.248665899359	-198.248665899359\\
57.5	0.23208	-215.97870522673	-215.97870522673\\
57.5	0.23574	-234.357374864974	-234.357374864974\\
57.5	0.2394	-253.38467481409	-253.38467481409\\
57.5	0.24306	-273.060605074079	-273.060605074079\\
57.5	0.24672	-293.385165644941	-293.385165644941\\
57.5	0.25038	-314.358356526675	-314.358356526675\\
57.5	0.25404	-335.980177719282	-335.980177719282\\
57.5	0.2577	-358.250629222763	-358.250629222763\\
57.5	0.26136	-381.169711037116	-381.169711037116\\
57.5	0.26502	-404.737423162341	-404.737423162341\\
57.5	0.26868	-428.95376559844	-428.95376559844\\
57.5	0.27234	-453.818738345411	-453.818738345411\\
57.5	0.276	-479.332341403255	-479.332341403255\\
57.875	0.093	1.23168616815983	1.23168616815983\\
57.875	0.09666	7.70616165397925	7.70616165397925\\
57.875	0.10032	13.5320068289265	13.5320068289265\\
57.875	0.10398	18.7092216930007	18.7092216930007\\
57.875	0.10764	23.237806246202	23.237806246202\\
57.875	0.1113	27.1177604885305	27.1177604885305\\
57.875	0.11496	30.3490844199862	30.3490844199862\\
57.875	0.11862	32.9317780405694	32.9317780405694\\
57.875	0.12228	34.8658413502797	34.8658413502797\\
57.875	0.12594	36.1512743491168	36.1512743491168\\
57.875	0.1296	36.7880770370815	36.7880770370815\\
57.875	0.13326	36.7762494141737	36.7762494141737\\
57.875	0.13692	36.1157914803925	36.1157914803925\\
57.875	0.14058	34.8067032357391	34.8067032357391\\
57.875	0.14424	32.8489846802125	32.8489846802125\\
57.875	0.1479	30.2426358138133	30.2426358138133\\
57.875	0.15156	26.9876566365413	26.9876566365413\\
57.875	0.15522	23.0840471483968	23.0840471483968\\
57.875	0.15888	18.5318073493791	18.5318073493791\\
57.875	0.16254	13.3309372394888	13.3309372394888\\
57.875	0.1662	7.48143681872557	7.48143681872557\\
57.875	0.16986	0.983306087089659	0.983306087089659\\
57.875	0.17352	-6.16345495541907	-6.16345495541907\\
57.875	0.17718	-13.9588463088006	-13.9588463088006\\
57.875	0.18084	-22.4028679730548	-22.4028679730548\\
57.875	0.1845	-31.4955199481819	-31.4955199481819\\
57.875	0.18816	-41.2368022341815	-41.2368022341815\\
57.875	0.19182	-51.6267148310541	-51.6267148310541\\
57.875	0.19548	-62.6652577387995	-62.6652577387995\\
57.875	0.19914	-74.3524309574177	-74.3524309574177\\
57.875	0.2028	-86.6882344869089	-86.6882344869089\\
57.875	0.20646	-99.6726683272727	-99.6726683272727\\
57.875	0.21012	-113.305732478509	-113.305732478509\\
57.875	0.21378	-127.587426940618	-127.587426940618\\
57.875	0.21744	-142.5177517136	-142.5177517136\\
57.875	0.2211	-158.096706797455	-158.096706797455\\
57.875	0.22476	-174.324292192183	-174.324292192183\\
57.875	0.22842	-191.200507897783	-191.200507897783\\
57.875	0.23208	-208.725353914256	-208.725353914256\\
57.875	0.23574	-226.898830241603	-226.898830241603\\
57.875	0.2394	-245.720936879821	-245.720936879821\\
57.875	0.24306	-265.191673828913	-265.191673828913\\
57.875	0.24672	-285.311041088877	-285.311041088877\\
57.875	0.25038	-306.079038659714	-306.079038659714\\
57.875	0.25404	-327.495666541423	-327.495666541423\\
57.875	0.2577	-349.560924734006	-349.560924734006\\
57.875	0.26136	-372.274813237462	-372.274813237462\\
57.875	0.26502	-395.63733205179	-395.63733205179\\
57.875	0.26868	-419.648481176991	-419.648481176991\\
57.875	0.27234	-444.308260613065	-444.308260613065\\
57.875	0.276	-469.616670360012	-469.616670360012\\
58.25	0.093	0.55614428982912	0.55614428982912\\
58.25	0.09666	7.23581308654613	7.23581308654613\\
58.25	0.10032	13.2668515723908	13.2668515723908\\
58.25	0.10398	18.6492597473623	18.6492597473623\\
58.25	0.10764	23.3830376114612	23.3830376114612\\
58.25	0.1113	27.4681851646873	27.4681851646873\\
58.25	0.11496	30.9047024070408	30.9047024070408\\
58.25	0.11862	33.6925893385211	33.6925893385211\\
58.25	0.12228	35.8318459591289	35.8318459591289\\
58.25	0.12594	37.3224722688639	37.3224722688639\\
58.25	0.1296	38.164468267726	38.164468267726\\
58.25	0.13326	38.3578339557157	38.3578339557157\\
58.25	0.13692	37.9025693328321	37.9025693328321\\
58.25	0.14058	36.798674399076	36.798674399076\\
58.25	0.14424	35.0461491544469	35.0461491544469\\
58.25	0.1479	32.6449935989455	32.6449935989455\\
58.25	0.15156	29.5952077325708	29.5952077325708\\
58.25	0.15522	25.8967915553237	25.8967915553237\\
58.25	0.15888	21.5497450672033	21.5497450672033\\
58.25	0.16254	16.5540682682108	16.5540682682108\\
58.25	0.1662	10.909761158345	10.909761158345\\
58.25	0.16986	4.61682373760686	4.61682373760686\\
58.25	0.17352	-2.32474399400451	-2.32474399400451\\
58.25	0.17718	-9.91494203648824	-9.91494203648824\\
58.25	0.18084	-18.153770389845	-18.153770389845\\
58.25	0.1845	-27.0412290540744	-27.0412290540744\\
58.25	0.18816	-36.577318029177	-36.577318029177\\
58.25	0.19182	-46.7620373151518	-46.7620373151518\\
58.25	0.19548	-57.5953869119999	-57.5953869119999\\
58.25	0.19914	-69.0773668197207	-69.0773668197207\\
58.25	0.2028	-81.2079770383141	-81.2079770383141\\
58.25	0.20646	-93.9872175677801	-93.9872175677801\\
58.25	0.21012	-107.415088408119	-107.415088408119\\
58.25	0.21378	-121.491589559331	-121.491589559331\\
58.25	0.21744	-136.216721021415	-136.216721021415\\
58.25	0.2211	-151.590482794373	-151.590482794373\\
58.25	0.22476	-167.612874878203	-167.612874878203\\
58.25	0.22842	-184.283897272906	-184.283897272906\\
58.25	0.23208	-201.603549978481	-201.603549978481\\
58.25	0.23574	-219.57183299493	-219.57183299493\\
58.25	0.2394	-238.188746322251	-238.188746322251\\
58.25	0.24306	-257.454289960445	-257.454289960445\\
58.25	0.24672	-277.368463909512	-277.368463909512\\
58.25	0.25038	-297.931268169452	-297.931268169452\\
58.25	0.25404	-319.142702740263	-319.142702740263\\
58.25	0.2577	-341.002767621949	-341.002767621949\\
58.25	0.26136	-363.511462814507	-363.511462814507\\
58.25	0.26502	-386.668788317937	-386.668788317937\\
58.25	0.26868	-410.474744132241	-410.474744132241\\
58.25	0.27234	-434.929330257417	-434.929330257417\\
58.25	0.276	-460.032546693466	-460.032546693466\\
58.625	0.093	-0.250944965201057	-0.250944965201057\\
58.625	0.09666	6.63391714241376	6.63391714241376\\
58.625	0.10032	12.8701489391558	12.8701489391558\\
58.625	0.10398	18.4577504250251	18.4577504250251\\
58.625	0.10764	23.3967216000211	23.3967216000211\\
58.625	0.1113	27.6870624641448	27.6870624641448\\
58.625	0.11496	31.3287730173957	31.3287730173957\\
58.625	0.11862	34.3218532597738	34.3218532597738\\
58.625	0.12228	36.6663031912792	36.6663031912792\\
58.625	0.12594	38.3621228119115	38.3621228119115\\
58.625	0.1296	39.4093121216712	39.4093121216712\\
58.625	0.13326	39.8078711205583	39.8078711205583\\
58.625	0.13692	39.5577998085725	39.5577998085725\\
58.625	0.14058	38.6590981857138	38.6590981857138\\
58.625	0.14424	37.1117662519824	37.1117662519824\\
58.625	0.1479	34.9158040073784	34.9158040073784\\
58.625	0.15156	32.071211451901	32.071211451901\\
58.625	0.15522	28.5779885855513	28.5779885855513\\
58.625	0.15888	24.4361354083292	24.4361354083292\\
58.625	0.16254	19.6456519202336	19.6456519202336\\
58.625	0.1662	14.2065381212656	14.2065381212656\\
58.625	0.16986	8.11879401142482	8.11879401142482\\
58.625	0.17352	1.38241959071081	1.38241959071081\\
58.625	0.17718	-6.00258514087557	-6.00258514087557\\
58.625	0.18084	-14.0362201833345	-14.0362201833345\\
58.625	0.1845	-22.7184855366665	-22.7184855366665\\
58.625	0.18816	-32.0493812008714	-32.0493812008714\\
58.625	0.19182	-42.0289071759488	-42.0289071759488\\
58.625	0.19548	-52.657063461899	-52.657063461899\\
58.625	0.19914	-63.9338500587226	-63.9338500587226\\
58.625	0.2028	-75.8592669664181	-75.8592669664181\\
58.625	0.20646	-88.4333141849872	-88.4333141849872\\
58.625	0.21012	-101.655991714429	-101.655991714429\\
58.625	0.21378	-115.527299554743	-115.527299554743\\
58.625	0.21744	-130.04723770593	-130.04723770593\\
58.625	0.2211	-145.21580616799	-145.21580616799\\
58.625	0.22476	-161.033004940922	-161.033004940922\\
58.625	0.22842	-177.498834024728	-177.498834024728\\
58.625	0.23208	-194.613293419405	-194.613293419405\\
58.625	0.23574	-212.376383124957	-212.376383124957\\
58.625	0.2394	-230.78810314138	-230.78810314138\\
58.625	0.24306	-249.848453468677	-249.848453468677\\
58.625	0.24672	-269.557434106846	-269.557434106846\\
58.625	0.25038	-289.915045055888	-289.915045055888\\
58.625	0.25404	-310.921286315802	-310.921286315802\\
58.625	0.2577	-332.576157886591	-332.576157886591\\
58.625	0.26136	-354.879659768251	-354.879659768251\\
58.625	0.26502	-377.831791960784	-377.831791960784\\
58.625	0.26868	-401.43255446419	-401.43255446419\\
58.625	0.27234	-425.681947278469	-425.681947278469\\
58.625	0.276	-450.57997040362	-450.57997040362\\
59	0.093	-1.18958159693025	-1.18958159693025\\
59	0.09666	5.90047382158193	5.90047382158193\\
59	0.10032	12.3418989292217	12.3418989292217\\
59	0.10398	18.1346937259884	18.1346937259884\\
59	0.10764	23.2788582118823	23.2788582118823\\
59	0.1113	27.7743923869033	27.7743923869033\\
59	0.11496	31.6212962510515	31.6212962510515\\
59	0.11862	34.8195698043273	34.8195698043273\\
59	0.12228	37.36921304673	37.36921304673\\
59	0.12594	39.2702259782602	39.2702259782602\\
59	0.1296	40.5226085989174	40.5226085989174\\
59	0.13326	41.1263609087018	41.1263609087018\\
59	0.13692	41.0814829076134	41.0814829076134\\
59	0.14058	40.3879745956521	40.3879745956521\\
59	0.14424	39.0458359728181	39.0458359728181\\
59	0.1479	37.0550670391118	37.0550670391118\\
59	0.15156	34.4156677945323	34.4156677945323\\
59	0.15522	31.1276382390799	31.1276382390799\\
59	0.15888	27.1909783727551	27.1909783727551\\
59	0.16254	22.6056881955574	22.6056881955574\\
59	0.1662	17.3717677074867	17.3717677074867\\
59	0.16986	11.4892169085433	11.4892169085433\\
59	0.17352	4.95803579872711	4.95803579872711\\
59	0.17718	-2.22177562196191	-2.22177562196191\\
59	0.18084	-10.0502173535235	-10.0502173535235\\
59	0.1845	-18.5272893959582	-18.5272893959582\\
59	0.18816	-27.6529917492652	-27.6529917492652\\
59	0.19182	-37.4273244134453	-37.4273244134453\\
59	0.19548	-47.8502873884981	-47.8502873884981\\
59	0.19914	-58.9218806744238	-58.9218806744238\\
59	0.2028	-70.642104271222	-70.642104271222\\
59	0.20646	-83.0109581788934	-83.0109581788934\\
59	0.21012	-96.0284423974374	-96.0284423974374\\
59	0.21378	-109.694556926854	-109.694556926854\\
59	0.21744	-124.009301767144	-124.009301767144\\
59	0.2211	-138.972676918306	-138.972676918306\\
59	0.22476	-154.584682380341	-154.584682380341\\
59	0.22842	-170.845318153249	-170.845318153249\\
59	0.23208	-187.75458423703	-187.75458423703\\
59	0.23574	-205.312480631683	-205.312480631683\\
59	0.2394	-223.519007337209	-223.519007337209\\
59	0.24306	-242.374164353608	-242.374164353608\\
59	0.24672	-261.87795168088	-261.87795168088\\
59	0.25038	-282.030369319024	-282.030369319024\\
59	0.25404	-302.831417268042	-302.831417268042\\
59	0.2577	-324.281095527932	-324.281095527932\\
59	0.26136	-346.379404098695	-346.379404098695\\
59	0.26502	-369.12634298033	-369.12634298033\\
59	0.26868	-392.521912172838	-392.521912172838\\
59	0.27234	-416.56611167622	-416.56611167622\\
59	0.276	-441.258941490474	-441.258941490474\\
59.375	0.093	-2.25976560535892	-2.25976560535892\\
59.375	0.09666	5.03548312405107	5.03548312405107\\
59.375	0.10032	11.6821015425882	11.6821015425882\\
59.375	0.10398	17.6800896502523	17.6800896502523\\
59.375	0.10764	23.0294474470437	23.0294474470437\\
59.375	0.1113	27.7301749329623	27.7301749329623\\
59.375	0.11496	31.7822721080084	31.7822721080084\\
59.375	0.11862	35.1857389721815	35.1857389721815\\
59.375	0.12228	37.9405755254818	37.9405755254818\\
59.375	0.12594	40.0467817679091	40.0467817679091\\
59.375	0.1296	41.5043576994641	41.5043576994641\\
59.375	0.13326	42.3133033201464	42.3133033201464\\
59.375	0.13692	42.4736186299548	42.4736186299548\\
59.375	0.14058	41.9853036288913	41.9853036288913\\
59.375	0.14424	40.8483583169551	40.8483583169551\\
59.375	0.1479	39.0627826941462	39.0627826941462\\
59.375	0.15156	36.6285767604641	36.6285767604641\\
59.375	0.15522	33.5457405159091	33.5457405159091\\
59.375	0.15888	29.8142739604816	29.8142739604816\\
59.375	0.16254	25.4341770941817	25.4341770941817\\
59.375	0.1662	20.4054499170084	20.4054499170084\\
59.375	0.16986	14.7280924289628	14.7280924289628\\
59.375	0.17352	8.40210463004394	8.40210463004394\\
59.375	0.17718	1.42748652025273	1.42748652025273\\
59.375	0.18084	-6.19576190041153	-6.19576190041153\\
59.375	0.1845	-14.4676406319484	-14.4676406319484\\
59.375	0.18816	-23.388149674358	-23.388149674358\\
59.375	0.19182	-32.9572890276407	-32.9572890276407\\
59.375	0.19548	-43.1750586917958	-43.1750586917958\\
59.375	0.19914	-54.0414586668242	-54.0414586668242\\
59.375	0.2028	-65.556488952725	-65.556488952725\\
59.375	0.20646	-77.720149549499	-77.720149549499\\
59.375	0.21012	-90.5324404571452	-90.5324404571452\\
59.375	0.21378	-103.993361675664	-103.993361675664\\
59.375	0.21744	-118.102913205057	-118.102913205057\\
59.375	0.2211	-132.861095045321	-132.861095045321\\
59.375	0.22476	-148.267907196459	-148.267907196459\\
59.375	0.22842	-164.323349658469	-164.323349658469\\
59.375	0.23208	-181.027422431352	-181.027422431352\\
59.375	0.23574	-198.380125515108	-198.380125515108\\
59.375	0.2394	-216.381458909737	-216.381458909737\\
59.375	0.24306	-235.031422615238	-235.031422615238\\
59.375	0.24672	-254.330016631612	-254.330016631612\\
59.375	0.25038	-274.277240958859	-274.277240958859\\
59.375	0.25404	-294.873095596979	-294.873095596979\\
59.375	0.2577	-316.117580545972	-316.117580545972\\
59.375	0.26136	-338.010695805837	-338.010695805837\\
59.375	0.26502	-360.552441376575	-360.552441376575\\
59.375	0.26868	-383.742817258186	-383.742817258186\\
59.375	0.27234	-407.58182345067	-407.58182345067\\
59.375	0.276	-432.069459954026	-432.069459954026\\
59.75	0.093	-3.46149699048615	-3.46149699048615\\
59.75	0.09666	4.03894504982119	4.03894504982119\\
59.75	0.10032	10.8907567792557	10.8907567792557\\
59.75	0.10398	17.0939381978176	17.0939381978176\\
59.75	0.10764	22.6484893055061	22.6484893055061\\
59.75	0.1113	27.5544101023223	27.5544101023223\\
59.75	0.11496	31.8117005882658	31.8117005882658\\
59.75	0.11862	35.4203607633366	35.4203607633366\\
59.75	0.12228	38.3803906275343	38.3803906275343\\
59.75	0.12594	40.691790180859	40.691790180859\\
59.75	0.1296	42.3545594233118	42.3545594233118\\
59.75	0.13326	43.368698354891	43.368698354891\\
59.75	0.13692	43.7342069755982	43.7342069755982\\
59.75	0.14058	43.4510852854315	43.4510852854315\\
59.75	0.14424	42.5193332843932	42.5193332843932\\
59.75	0.1479	40.9389509724812	40.9389509724812\\
59.75	0.15156	38.7099383496968	38.7099383496968\\
59.75	0.15522	35.8322954160396	35.8322954160396\\
59.75	0.15888	32.3060221715095	32.3060221715095\\
59.75	0.16254	28.131118616107	28.131118616107\\
59.75	0.1662	23.307584749831	23.307584749831\\
59.75	0.16986	17.8354205726832	17.8354205726832\\
59.75	0.17352	11.7146260846617	11.7146260846617\\
59.75	0.17718	4.94520128576835	4.94520128576835\\
59.75	0.18084	-2.47285382399855	-2.47285382399855\\
59.75	0.1845	-10.539539244638	-10.539539244638\\
59.75	0.18816	-19.2548549761503	-19.2548549761503\\
59.75	0.19182	-28.6188010185352	-28.6188010185352\\
59.75	0.19548	-38.631377371793	-38.631377371793\\
59.75	0.19914	-49.2925840359239	-49.2925840359239\\
59.75	0.2028	-60.6024210109269	-60.6024210109269\\
59.75	0.20646	-72.5608882968036	-72.5608882968036\\
59.75	0.21012	-85.167985893552	-85.167985893552\\
59.75	0.21378	-98.4237138011738	-98.4237138011738\\
59.75	0.21744	-112.328072019668	-112.328072019668\\
59.75	0.2211	-126.881060549036	-126.881060549036\\
59.75	0.22476	-142.082679389275	-142.082679389275\\
59.75	0.22842	-157.932928540388	-157.932928540388\\
59.75	0.23208	-174.431808002374	-174.431808002374\\
59.75	0.23574	-191.579317775232	-191.579317775232\\
59.75	0.2394	-209.375457858963	-209.375457858963\\
59.75	0.24306	-227.820228253568	-227.820228253568\\
59.75	0.24672	-246.913628959044	-246.913628959044\\
59.75	0.25038	-266.655659975394	-266.655659975394\\
59.75	0.25404	-287.046321302616	-287.046321302616\\
59.75	0.2577	-308.085612940711	-308.085612940711\\
59.75	0.26136	-329.773534889679	-329.773534889679\\
59.75	0.26502	-352.110087149519	-352.110087149519\\
59.75	0.26868	-375.095269720233	-375.095269720233\\
59.75	0.27234	-398.729082601819	-398.729082601819\\
59.75	0.276	-423.011525794278	-423.011525794278\\
60.125	0.093	-4.79477575231331	-4.79477575231331\\
60.125	0.09666	2.91085959889139	2.91085959889139\\
60.125	0.10032	9.96786463922373	9.96786463922373\\
60.125	0.10398	16.3762393686829	16.3762393686829\\
60.125	0.10764	22.1359837872693	22.1359837872693\\
60.125	0.1113	27.2470978949829	27.2470978949829\\
60.125	0.11496	31.7095816918236	31.7095816918236\\
60.125	0.11862	35.5234351777921	35.5234351777921\\
60.125	0.12228	38.6886583528874	38.6886583528874\\
60.125	0.12594	41.2052512171098	41.2052512171098\\
60.125	0.1296	43.0732137704596	43.0732137704596\\
60.125	0.13326	44.2925460129366	44.2925460129366\\
60.125	0.13692	44.8632479445407	44.8632479445407\\
60.125	0.14058	44.7853195652723	44.7853195652723\\
60.125	0.14424	44.0587608751308	44.0587608751308\\
60.125	0.1479	42.6835718741166	42.6835718741166\\
60.125	0.15156	40.6597525622296	40.6597525622296\\
60.125	0.15522	37.9873029394703	37.9873029394703\\
60.125	0.15888	34.6662230058375	34.6662230058375\\
60.125	0.16254	30.6965127613323	30.6965127613323\\
60.125	0.1662	26.0781722059542	26.0781722059542\\
60.125	0.16986	20.8112013397038	20.8112013397038\\
60.125	0.17352	14.8956001625801	14.8956001625801\\
60.125	0.17718	8.33136867458359	8.33136867458359\\
60.125	0.18084	1.1185068757145	1.1185068757145\\
60.125	0.1845	-6.74298523402717	-6.74298523402717\\
60.125	0.18816	-15.2531076546421	-15.2531076546421\\
60.125	0.19182	-24.4118603861297	-24.4118603861297\\
60.125	0.19548	-34.21924342849	-34.21924342849\\
60.125	0.19914	-44.6752567817232	-44.6752567817232\\
60.125	0.2028	-55.7799004458288	-55.7799004458288\\
60.125	0.20646	-67.5331744208077	-67.5331744208077\\
60.125	0.21012	-79.9350787066587	-79.9350787066587\\
60.125	0.21378	-92.9856133033827	-92.9856133033827\\
60.125	0.21744	-106.68477821098	-106.68477821098\\
60.125	0.2211	-121.03257342945	-121.03257342945\\
60.125	0.22476	-136.028998958792	-136.028998958792\\
60.125	0.22842	-151.674054799007	-151.674054799007\\
60.125	0.23208	-167.967740950095	-167.967740950095\\
60.125	0.23574	-184.910057412057	-184.910057412057\\
60.125	0.2394	-202.50100418489	-202.50100418489\\
60.125	0.24306	-220.740581268597	-220.740581268597\\
60.125	0.24672	-239.628788663176	-239.628788663176\\
60.125	0.25038	-259.165626368628	-259.165626368628\\
60.125	0.25404	-279.351094384953	-279.351094384953\\
60.125	0.2577	-300.18519271215	-300.18519271215\\
60.125	0.26136	-321.66792135022	-321.66792135022\\
60.125	0.26502	-343.799280299163	-343.799280299163\\
60.125	0.26868	-366.579269558979	-366.579269558979\\
60.125	0.27234	-390.007889129668	-390.007889129668\\
60.125	0.276	-414.08513901123	-414.08513901123\\
60.5	0.093	-6.25960189083949	-6.25960189083949\\
60.5	0.09666	1.65122677126303	1.65122677126303\\
60.5	0.10032	8.91342512249273	8.91342512249273\\
60.5	0.10398	15.5269931628493	15.5269931628493\\
60.5	0.10764	21.4919308923332	21.4919308923332\\
60.5	0.1113	26.8082383109444	26.8082383109444\\
60.5	0.11496	31.475915418683	31.475915418683\\
60.5	0.11862	35.4949622155488	35.4949622155488\\
60.5	0.12228	38.8653787015414	38.8653787015414\\
60.5	0.12594	41.5871648766612	41.5871648766612\\
60.5	0.1296	43.6603207409088	43.6603207409088\\
60.5	0.13326	45.0848462942836	45.0848462942836\\
60.5	0.13692	45.860741536785	45.860741536785\\
60.5	0.14058	45.988006468414	45.988006468414\\
60.5	0.14424	45.4666410891699	45.4666410891699\\
60.5	0.1479	44.2966453990535	44.2966453990535\\
60.5	0.15156	42.4780193980644	42.4780193980644\\
60.5	0.15522	40.0107630862019	40.0107630862019\\
60.5	0.15888	36.894876463467	36.894876463467\\
60.5	0.16254	33.1303595298591	33.1303595298591\\
60.5	0.1662	28.7172122853788	28.7172122853788\\
60.5	0.16986	23.6554347300257	23.6554347300257\\
60.5	0.17352	17.9450268637999	17.9450268637999\\
60.5	0.17718	11.5859886867007	11.5859886867007\\
60.5	0.18084	4.57832019872899	4.57832019872899\\
60.5	0.1845	-3.07797860011533	-3.07797860011533\\
60.5	0.18816	-11.3829077098325	-11.3829077098325\\
60.5	0.19182	-20.3364671304222	-20.3364671304222\\
60.5	0.19548	-29.9386568618852	-29.9386568618852\\
60.5	0.19914	-40.189476904221	-40.189476904221\\
60.5	0.2028	-51.0889272574293	-51.0889272574293\\
60.5	0.20646	-62.6370079215103	-62.6370079215103\\
60.5	0.21012	-74.8337188964645	-74.8337188964645\\
60.5	0.21378	-87.6790601822906	-87.6790601822906\\
60.5	0.21744	-101.17303177899	-101.17303177899\\
60.5	0.2211	-115.315633686562	-115.315633686562\\
60.5	0.22476	-130.106865905008	-130.106865905008\\
60.5	0.22842	-145.546728434325	-145.546728434325\\
60.5	0.23208	-161.635221274516	-161.635221274516\\
60.5	0.23574	-178.372344425579	-178.372344425579\\
60.5	0.2394	-195.758097887515	-195.758097887515\\
60.5	0.24306	-213.792481660324	-213.792481660324\\
60.5	0.24672	-232.475495744006	-232.475495744006\\
60.5	0.25038	-251.80714013856	-251.80714013856\\
60.5	0.25404	-271.787414843988	-271.787414843988\\
60.5	0.2577	-292.416319860288	-292.416319860288\\
60.5	0.26136	-313.69385518746	-313.69385518746\\
60.5	0.26502	-335.620020825506	-335.620020825506\\
60.5	0.26868	-358.194816774424	-358.194816774424\\
60.5	0.27234	-381.418243034215	-381.418243034215\\
60.5	0.276	-405.290299604879	-405.290299604879\\
60.875	0.093	-7.85597540606446	-7.85597540606446\\
60.875	0.09666	0.260046566935642	0.260046566935642\\
60.875	0.10032	7.7274382290627	7.7274382290627\\
60.875	0.10398	14.5461995803169	14.5461995803169\\
60.875	0.10764	20.7163306206984	20.7163306206984\\
60.875	0.1113	26.2378313502073	26.2378313502073\\
60.875	0.11496	31.1107017688428	31.1107017688428\\
60.875	0.11862	35.3349418766065	35.3349418766065\\
60.875	0.12228	38.9105516734965	38.9105516734965\\
60.875	0.12594	41.8375311595141	41.8375311595141\\
60.875	0.1296	44.115880334659	44.115880334659\\
60.875	0.13326	45.7455991989311	45.7455991989311\\
60.875	0.13692	46.7266877523304	46.7266877523304\\
60.875	0.14058	47.0591459948567	47.0591459948567\\
60.875	0.14424	46.7429739265104	46.7429739265104\\
60.875	0.1479	45.7781715472914	45.7781715472914\\
60.875	0.15156	44.1647388571996	44.1647388571996\\
60.875	0.15522	41.9026758562345	41.9026758562345\\
60.875	0.15888	38.9919825443974	38.9919825443974\\
60.875	0.16254	35.4326589216873	35.4326589216873\\
60.875	0.1662	31.2247049881043	31.2247049881043\\
60.875	0.16986	26.3681207436487	26.3681207436487\\
60.875	0.17352	20.8629061883202	20.8629061883202\\
60.875	0.17718	14.7090613221184	14.7090613221184\\
60.875	0.18084	7.90658614504446	7.90658614504446\\
60.875	0.1845	0.455480657097496	0.455480657097496\\
60.875	0.18816	-7.64425514172228	-7.64425514172228\\
60.875	0.19182	-16.3926212514147	-16.3926212514147\\
60.875	0.19548	-25.7896176719798	-25.7896176719798\\
60.875	0.19914	-35.8352444034178	-35.8352444034178\\
60.875	0.2028	-46.5295014457283	-46.5295014457283\\
60.875	0.20646	-57.8723887989124	-57.8723887989124\\
60.875	0.21012	-69.8639064629688	-69.8639064629688\\
60.875	0.21378	-82.5040544378976	-82.5040544378976\\
60.875	0.21744	-95.7928327236996	-95.7928327236996\\
60.875	0.2211	-109.730241320374	-109.730241320374\\
60.875	0.22476	-124.316280227922	-124.316280227922\\
60.875	0.22842	-139.550949446342	-139.550949446342\\
60.875	0.23208	-155.434248975635	-155.434248975635\\
60.875	0.23574	-171.966178815801	-171.966178815801\\
60.875	0.2394	-189.146738966839	-189.146738966839\\
60.875	0.24306	-206.975929428751	-206.975929428751\\
60.875	0.24672	-225.453750201535	-225.453750201535\\
60.875	0.25038	-244.580201285193	-244.580201285193\\
60.875	0.25404	-264.355282679722	-264.355282679722\\
60.875	0.2577	-284.778994385124	-284.778994385124\\
60.875	0.26136	-305.8513364014	-305.8513364014\\
60.875	0.26502	-327.572308728548	-327.572308728548\\
60.875	0.26868	-349.941911366569	-349.941911366569\\
60.875	0.27234	-372.960144315462	-372.960144315462\\
60.875	0.276	-396.627007575228	-396.627007575228\\
61.25	0.093	-9.58389629798867	-9.58389629798867\\
61.25	0.09666	-1.26268101409121	-1.26268101409121\\
61.25	0.10032	6.40990395893343	6.40990395893343\\
61.25	0.10398	13.4338586210852	13.4338586210852\\
61.25	0.10764	19.8091829723641	19.8091829723641\\
61.25	0.1113	25.5358770127701	25.5358770127701\\
61.25	0.11496	30.6139407423039	30.6139407423039\\
61.25	0.11862	35.0433741609649	35.0433741609649\\
61.25	0.12228	38.8241772687527	38.8241772687527\\
61.25	0.12594	41.9563500656677	41.9563500656677\\
61.25	0.1296	44.4398925517099	44.4398925517099\\
61.25	0.13326	46.2748047268794	46.2748047268794\\
61.25	0.13692	47.4610865911761	47.4610865911761\\
61.25	0.14058	47.9987381446002	47.9987381446002\\
61.25	0.14424	47.8877593871517	47.8877593871517\\
61.25	0.1479	47.1281503188301	47.1281503188301\\
61.25	0.15156	45.7199109396356	45.7199109396356\\
61.25	0.15522	43.6630412495683	43.6630412495683\\
61.25	0.15888	40.9575412486286	40.9575412486286\\
61.25	0.16254	37.6034109368159	37.6034109368159\\
61.25	0.1662	33.6006503141302	33.6006503141302\\
61.25	0.16986	28.9492593805719	28.9492593805719\\
61.25	0.17352	23.6492381361412	23.6492381361412\\
61.25	0.17718	17.7005865808372	17.7005865808372\\
61.25	0.18084	11.1033047146607	11.1033047146607\\
61.25	0.1845	3.85739253761108	3.85739253761108\\
61.25	0.18816	-4.03714995031089	-4.03714995031089\\
61.25	0.19182	-12.5803227491054	-12.5803227491054\\
61.25	0.19548	-21.7721258587733	-21.7721258587733\\
61.25	0.19914	-31.6125592793139	-31.6125592793139\\
61.25	0.2028	-42.1016230107275	-42.1016230107275\\
61.25	0.20646	-53.2393170530138	-53.2393170530138\\
61.25	0.21012	-65.0256414061723	-65.0256414061723\\
61.25	0.21378	-77.4605960702033	-77.4605960702033\\
61.25	0.21744	-90.544181045108	-90.544181045108\\
61.25	0.2211	-104.276396330885	-104.276396330885\\
61.25	0.22476	-118.657241927535	-118.657241927535\\
61.25	0.22842	-133.686717835058	-133.686717835058\\
61.25	0.23208	-149.364824053454	-149.364824053454\\
61.25	0.23574	-165.691560582722	-165.691560582722\\
61.25	0.2394	-182.666927422863	-182.666927422863\\
61.25	0.24306	-200.290924573877	-200.290924573877\\
61.25	0.24672	-218.563552035764	-218.563552035764\\
61.25	0.25038	-237.484809808523	-237.484809808523\\
61.25	0.25404	-257.054697892155	-257.054697892155\\
61.25	0.2577	-277.273216286661	-277.273216286661\\
61.25	0.26136	-298.140364992038	-298.140364992038\\
61.25	0.26502	-319.656144008288	-319.656144008288\\
61.25	0.26868	-341.820553335412	-341.820553335412\\
61.25	0.27234	-364.633592973408	-364.633592973408\\
61.25	0.276	-388.095262922277	-388.095262922277\\
61.625	0.093	-11.4433645666121	-11.4433645666121\\
61.625	0.09666	-2.91695597181709	-2.91695597181709\\
61.625	0.10032	4.96082231210514	4.96082231210514\\
61.625	0.10398	12.1899702851542	12.1899702851542\\
61.625	0.10764	18.7704879473309	18.7704879473309\\
61.625	0.1113	24.7023752986348	24.7023752986348\\
61.625	0.11496	29.9856323390659	29.9856323390659\\
61.625	0.11862	34.6202590686239	34.6202590686239\\
61.625	0.12228	38.606255487309	38.606255487309\\
61.625	0.12594	41.9436215951218	41.9436215951218\\
61.625	0.1296	44.6323573920614	44.6323573920614\\
61.625	0.13326	46.6724628781287	46.6724628781287\\
61.625	0.13692	48.0639380533227	48.0639380533227\\
61.625	0.14058	48.8067829176442	48.8067829176442\\
61.625	0.14424	48.9009974710931	48.9009974710931\\
61.625	0.1479	48.3465817136693	48.3465817136693\\
61.625	0.15156	47.1435356453721	47.1435356453721\\
61.625	0.15522	45.2918592662027	45.2918592662027\\
61.625	0.15888	42.7915525761603	42.7915525761603\\
61.625	0.16254	39.6426155752449	39.6426155752449\\
61.625	0.1662	35.8450482634571	35.8450482634571\\
61.625	0.16986	31.3988506407966	31.3988506407966\\
61.625	0.17352	26.3040227072632	26.3040227072632\\
61.625	0.17718	20.5605644628566	20.5605644628566\\
61.625	0.18084	14.1684759075774	14.1684759075774\\
61.625	0.1845	7.12775704142564	7.12775704142564\\
61.625	0.18816	-0.561592135598971	-0.561592135598971\\
61.625	0.19182	-8.89957162349617	-8.89957162349617\\
61.625	0.19548	-17.8861814222666	-17.8861814222666\\
61.625	0.19914	-27.5214215319099	-27.5214215319099\\
61.625	0.2028	-37.8052919524257	-37.8052919524257\\
61.625	0.20646	-48.7377926838142	-48.7377926838142\\
61.625	0.21012	-60.3189237260754	-60.3189237260754\\
61.625	0.21378	-72.548685079209	-72.548685079209\\
61.625	0.21744	-85.4270767432163	-85.4270767432163\\
61.625	0.2211	-98.9540987180962	-98.9540987180962\\
61.625	0.22476	-113.129751003848	-113.129751003848\\
61.625	0.22842	-127.954033600474	-127.954033600474\\
61.625	0.23208	-143.426946507972	-143.426946507972\\
61.625	0.23574	-159.548489726343	-159.548489726343\\
61.625	0.2394	-176.318663255586	-176.318663255586\\
61.625	0.24306	-193.737467095703	-193.737467095703\\
61.625	0.24672	-211.804901246692	-211.804901246692\\
61.625	0.25038	-230.520965708554	-230.520965708554\\
61.625	0.25404	-249.885660481288	-249.885660481288\\
61.625	0.2577	-269.898985564895	-269.898985564895\\
61.625	0.26136	-290.560940959376	-290.560940959376\\
61.625	0.26502	-311.871526664729	-311.871526664729\\
61.625	0.26868	-333.830742680955	-333.830742680955\\
61.625	0.27234	-356.438589008053	-356.438589008053\\
61.625	0.276	-379.695065646025	-379.695065646025\\
62	0.093	-13.4343802119351	-13.4343802119351\\
62	0.09666	-4.70277830624221	-4.70277830624221\\
62	0.10032	3.38019328857737	3.38019328857737\\
62	0.10398	10.8145345725238	10.8145345725238\\
62	0.10764	17.6002455455983	17.6002455455983\\
62	0.1113	23.7373262077991	23.7373262077991\\
62	0.11496	29.2257765591281	29.2257765591281\\
62	0.11862	34.0655965995833	34.0655965995833\\
62	0.12228	38.2567863291667	38.2567863291667\\
62	0.12594	41.7993457478764	41.7993457478764\\
62	0.1296	44.6932748557139	44.6932748557139\\
62	0.13326	46.9385736526785	46.9385736526785\\
62	0.13692	48.5352421387703	48.5352421387703\\
62	0.14058	49.4832803139892	49.4832803139892\\
62	0.14424	49.7826881783359	49.7826881783359\\
62	0.1479	49.4334657318094	49.4334657318094\\
62	0.15156	48.4356129744097	48.4356129744097\\
62	0.15522	46.7891299061375	46.7891299061375\\
62	0.15888	44.4940165269929	44.4940165269929\\
62	0.16254	41.550272836975	41.550272836975\\
62	0.1662	37.957898836085	37.957898836085\\
62	0.16986	33.7168945243218	33.7168945243218\\
62	0.17352	28.8272599016858	28.8272599016858\\
62	0.17718	23.2889949681766	23.2889949681766\\
62	0.18084	17.1020997237952	17.1020997237952\\
62	0.1845	10.2665741685407	10.2665741685407\\
62	0.18816	2.78241830241393	2.78241830241393\\
62	0.19182	-5.35036787458591	-5.35036787458591\\
62	0.19548	-14.1317843624586	-14.1317843624586\\
62	0.19914	-23.5618311612045	-23.5618311612045\\
62	0.2028	-33.6405082708225	-33.6405082708225\\
62	0.20646	-44.367815691314	-44.367815691314\\
62	0.21012	-55.7437534226774	-55.7437534226774\\
62	0.21378	-67.7683214649137	-67.7683214649137\\
62	0.21744	-80.4415198180232	-80.4415198180232\\
62	0.2211	-93.7633484820058	-93.7633484820058\\
62	0.22476	-107.73380745686	-107.73380745686\\
62	0.22842	-122.352896742588	-122.352896742588\\
62	0.23208	-137.620616339189	-137.620616339189\\
62	0.23574	-153.536966246662	-153.536966246662\\
62	0.2394	-170.101946465008	-170.101946465008\\
62	0.24306	-187.315556994227	-187.315556994227\\
62	0.24672	-205.177797834319	-205.177797834319\\
62	0.25038	-223.688668985283	-223.688668985283\\
62	0.25404	-242.84817044712	-242.84817044712\\
62	0.2577	-262.65630221983	-262.65630221983\\
62	0.26136	-283.113064303413	-283.113064303413\\
62	0.26502	-304.218456697868	-304.218456697868\\
62	0.26868	-325.972479403196	-325.972479403196\\
62	0.27234	-348.375132419397	-348.375132419397\\
62	0.276	-371.426415746472	-371.426415746472\\
62.375	0.093	-15.5569432339566	-15.5569432339566\\
62.375	0.09666	-6.62014801736657	-6.62014801736657\\
62.375	0.10032	1.66801688835059	1.66801688835059\\
62.375	0.10398	9.30755148319486	9.30755148319486\\
62.375	0.10764	16.2984557671663	16.2984557671663\\
62.375	0.1113	22.6407297402653	22.6407297402653\\
62.375	0.11496	28.3343734024912	28.3343734024912\\
62.375	0.11862	33.3793867538442	33.3793867538442\\
62.375	0.12228	37.7757697943246	37.7757697943246\\
62.375	0.12594	41.5235225239325	41.5235225239325\\
62.375	0.1296	44.6226449426673	44.6226449426673\\
62.375	0.13326	47.0731370505298	47.0731370505298\\
62.375	0.13692	48.874998847519	48.874998847519\\
62.375	0.14058	50.0282303336352	50.0282303336352\\
62.375	0.14424	50.5328315088792	50.5328315088792\\
62.375	0.1479	50.3888023732501	50.3888023732501\\
62.375	0.15156	49.5961429267481	49.5961429267481\\
62.375	0.15522	48.1548531693738	48.1548531693738\\
62.375	0.15888	46.0649331011261	46.0649331011261\\
62.375	0.16254	43.3263827220064	43.3263827220064\\
62.375	0.1662	39.9392020320133	39.9392020320133\\
62.375	0.16986	35.9033910311475	35.9033910311475\\
62.375	0.17352	31.2189497194094	31.2189497194094\\
62.375	0.17718	25.8858780967979	25.8858780967979\\
62.375	0.18084	19.9041761633139	19.9041761633139\\
62.375	0.1845	13.2738439189573	13.2738439189573\\
62.375	0.18816	5.99488136372736	5.99488136372736\\
62.375	0.19182	-1.93271150237467	-1.93271150237467\\
62.375	0.19548	-10.50893467935	-10.50893467935\\
62.375	0.19914	-19.7337881671981	-19.7337881671981\\
62.375	0.2028	-29.6072719659187	-29.6072719659187\\
62.375	0.20646	-40.1293860755125	-40.1293860755125\\
62.375	0.21012	-51.3001304959789	-51.3001304959789\\
62.375	0.21378	-63.1195052273174	-63.1195052273174\\
62.375	0.21744	-75.5875102695295	-75.5875102695295\\
62.375	0.2211	-88.7041456226143	-88.7041456226143\\
62.375	0.22476	-102.469411286571	-102.469411286571\\
62.375	0.22842	-116.883307261402	-116.883307261402\\
62.375	0.23208	-131.945833547105	-131.945833547105\\
62.375	0.23574	-147.656990143681	-147.656990143681\\
62.375	0.2394	-164.016777051129	-164.016777051129\\
62.375	0.24306	-181.025194269451	-181.025194269451\\
62.375	0.24672	-198.682241798644	-198.682241798644\\
62.375	0.25038	-216.987919638712	-216.987919638712\\
62.375	0.25404	-235.942227789651	-235.942227789651\\
62.375	0.2577	-255.545166251464	-255.545166251464\\
62.375	0.26136	-275.796735024149	-275.796735024149\\
62.375	0.26502	-296.696934107707	-296.696934107707\\
62.375	0.26868	-318.245763502137	-318.245763502137\\
62.375	0.27234	-340.443223207441	-340.443223207441\\
62.375	0.276	-363.289313223617	-363.289313223617\\
62.75	0.093	-17.8110536326775	-17.8110536326775\\
62.75	0.09666	-8.66906510518996	-8.66906510518996\\
62.75	0.10032	-0.175706888575206	-0.175706888575206\\
62.75	0.10398	7.66902101716641	7.66902101716641\\
62.75	0.10764	14.8651186120352	14.8651186120352\\
62.75	0.1113	21.4125858960316	21.4125858960316\\
62.75	0.11496	27.3114228691553	27.3114228691553\\
62.75	0.11862	32.5616295314061	32.5616295314061\\
62.75	0.12228	37.1632058827843	37.1632058827843\\
62.75	0.12594	41.1161519232891	41.1161519232891\\
62.75	0.1296	44.4204676529213	44.4204676529213\\
62.75	0.13326	47.0761530716811	47.0761530716811\\
62.75	0.13692	49.0832081795681	49.0832081795681\\
62.75	0.14058	50.4416329765821	50.4416329765821\\
62.75	0.14424	51.1514274627235	51.1514274627235\\
62.75	0.1479	51.2125916379922	51.2125916379922\\
62.75	0.15156	50.6251255023876	50.6251255023876\\
62.75	0.15522	49.3890290559102	49.3890290559102\\
62.75	0.15888	47.5043022985608	47.5043022985608\\
62.75	0.16254	44.970945230338	44.970945230338\\
62.75	0.1662	41.7889578512427	41.7889578512427\\
62.75	0.16986	37.9583401612747	37.9583401612747\\
62.75	0.17352	33.4790921604339	33.4790921604339\\
62.75	0.17718	28.3512138487198	28.3512138487198\\
62.75	0.18084	22.5747052261331	22.5747052261331\\
62.75	0.1845	16.1495662926739	16.1495662926739\\
62.75	0.18816	9.07579704834177	9.07579704834177\\
62.75	0.19182	1.35339749313709	1.35339749313709\\
62.75	0.19548	-7.0176323729404	-7.0176323729404\\
62.75	0.19914	-16.0372925498912	-16.0372925498912\\
62.75	0.2028	-25.7055830377144	-25.7055830377144\\
62.75	0.20646	-36.0225038364108	-36.0225038364108\\
62.75	0.21012	-46.9880549459795	-46.9880549459795\\
62.75	0.21378	-58.6022363664206	-58.6022363664206\\
62.75	0.21744	-70.8650480977349	-70.8650480977349\\
62.75	0.2211	-83.7764901399219	-83.7764901399219\\
62.75	0.22476	-97.336562492982	-97.336562492982\\
62.75	0.22842	-111.545265156915	-111.545265156915\\
62.75	0.23208	-126.40259813172	-126.40259813172\\
62.75	0.23574	-141.908561417399	-141.908561417399\\
62.75	0.2394	-158.063155013949	-158.063155013949\\
62.75	0.24306	-174.866378921374	-174.866378921374\\
62.75	0.24672	-192.31823313967	-192.31823313967\\
62.75	0.25038	-210.418717668839	-210.418717668839\\
62.75	0.25404	-229.167832508881	-229.167832508881\\
62.75	0.2577	-248.565577659797	-248.565577659797\\
62.75	0.26136	-268.611953121584	-268.611953121584\\
62.75	0.26502	-289.306958894244	-289.306958894244\\
62.75	0.26868	-310.650594977777	-310.650594977777\\
62.75	0.27234	-332.642861372184	-332.642861372184\\
62.75	0.276	-355.283758077462	-355.283758077462\\
63.125	0.093	-20.196711408098	-20.196711408098\\
63.125	0.09666	-10.8495295697126	-10.8495295697126\\
63.125	0.10032	-2.15097804220048	-2.15097804220048\\
63.125	0.10398	5.89894317443895	5.89894317443895\\
63.125	0.10764	13.3002340802051	13.3002340802051\\
63.125	0.1113	20.0528946750993	20.0528946750993\\
63.125	0.11496	26.1569249591203	26.1569249591203\\
63.125	0.11862	31.6123249322686	31.6123249322686\\
63.125	0.12228	36.4190945945436	36.4190945945436\\
63.125	0.12594	40.5772339459467	40.5772339459467\\
63.125	0.1296	44.0867429864762	44.0867429864762\\
63.125	0.13326	46.9476217161339	46.9476217161339\\
63.125	0.13692	49.1598701349178	49.1598701349178\\
63.125	0.14058	50.7234882428296	50.7234882428296\\
63.125	0.14424	51.6384760398684	51.6384760398684\\
63.125	0.1479	51.9048335260344	51.9048335260344\\
63.125	0.15156	51.5225607013281	51.5225607013281\\
63.125	0.15522	50.491657565748	50.491657565748\\
63.125	0.15888	48.812124119296	48.812124119296\\
63.125	0.16254	46.4839603619705	46.4839603619705\\
63.125	0.1662	43.507166293773	43.507166293773\\
63.125	0.16986	39.8817419147019	39.8817419147019\\
63.125	0.17352	35.6076872247589	35.6076872247589\\
63.125	0.17718	30.6850022239422	30.6850022239422\\
63.125	0.18084	25.1136869122538	25.1136869122538\\
63.125	0.1845	18.8937412896914	18.8937412896914\\
63.125	0.18816	12.0251653562572	12.0251653562572\\
63.125	0.19182	4.50795911194984	4.50795911194984\\
63.125	0.19548	-3.65787744323029	-3.65787744323029\\
63.125	0.19914	-12.4723443092832	-12.4723443092832\\
63.125	0.2028	-21.9354414862091	-21.9354414862091\\
63.125	0.20646	-32.0471689740077	-32.0471689740077\\
63.125	0.21012	-42.807526772679	-42.807526772679\\
63.125	0.21378	-54.2165148822223	-54.2165148822223\\
63.125	0.21744	-66.2741333026397	-66.2741333026397\\
63.125	0.2211	-78.9803820339289	-78.9803820339289\\
63.125	0.22476	-92.3352610760912	-92.3352610760912\\
63.125	0.22842	-106.338770429127	-106.338770429127\\
63.125	0.23208	-120.990910093035	-120.990910093035\\
63.125	0.23574	-136.291680067815	-136.291680067815\\
63.125	0.2394	-152.241080353469	-152.241080353469\\
63.125	0.24306	-168.839110949996	-168.839110949996\\
63.125	0.24672	-186.085771857395	-186.085771857395\\
63.125	0.25038	-203.981063075666	-203.981063075666\\
63.125	0.25404	-222.52498460481	-222.52498460481\\
63.125	0.2577	-241.717536444828	-241.717536444828\\
63.125	0.26136	-261.558718595719	-261.558718595719\\
63.125	0.26502	-282.048531057481	-282.048531057481\\
63.125	0.26868	-303.186973830117	-303.186973830117\\
63.125	0.27234	-324.974046913625	-324.974046913625\\
63.125	0.276	-347.409750308007	-347.409750308007\\
63.5	0.093	-22.7139165602174	-22.7139165602174\\
63.5	0.09666	-13.1615414109347	-13.1615414109347\\
63.5	0.10032	-4.25779657252477	-4.25779657252477\\
63.5	0.10398	3.99731795501202	3.99731795501202\\
63.5	0.10764	11.603802171676	11.603802171676\\
63.5	0.1113	18.5616560774671	18.5616560774671\\
63.5	0.11496	24.8708796723864	24.8708796723864\\
63.5	0.11862	30.531472956432	30.531472956432\\
63.5	0.12228	35.5434359296048	35.5434359296048\\
63.5	0.12594	39.9067685919048	39.9067685919048\\
63.5	0.1296	43.6214709433322	43.6214709433322\\
63.5	0.13326	46.6875429838872	46.6875429838872\\
63.5	0.13692	49.1049847135693	49.1049847135693\\
63.5	0.14058	50.8737961323781	50.8737961323781\\
63.5	0.14424	51.9939772403142	51.9939772403142\\
63.5	0.1479	52.465528037378	52.465528037378\\
63.5	0.15156	52.2884485235691	52.2884485235691\\
63.5	0.15522	51.4627386988868	51.4627386988868\\
63.5	0.15888	49.9883985633317	49.9883985633317\\
63.5	0.16254	47.865428116904	47.865428116904\\
63.5	0.1662	45.0938273596039	45.0938273596039\\
63.5	0.16986	41.6735962914306	41.6735962914306\\
63.5	0.17352	37.604734912385	37.604734912385\\
63.5	0.17718	32.8872432224661	32.8872432224661\\
63.5	0.18084	27.5211212216745	27.5211212216745\\
63.5	0.1845	21.5063689100105	21.5063689100105\\
63.5	0.18816	14.8429862874735	14.8429862874735\\
63.5	0.19182	7.53097335406358	7.53097335406358\\
63.5	0.19548	-0.429669890219202	-0.429669890219202\\
63.5	0.19914	-9.0389434453748	-9.0389434453748\\
63.5	0.2028	-18.2968473114024	-18.2968473114024\\
63.5	0.20646	-28.2033814883036	-28.2033814883036\\
63.5	0.21012	-38.7585459760776	-38.7585459760776\\
63.5	0.21378	-49.962340774724	-49.962340774724\\
63.5	0.21744	-61.8147658842436	-61.8147658842436\\
63.5	0.2211	-74.3158213046358	-74.3158213046358\\
63.5	0.22476	-87.4655070358999	-87.4655070358999\\
63.5	0.22842	-101.263823078038	-101.263823078038\\
63.5	0.23208	-115.710769431048	-115.710769431048\\
63.5	0.23574	-130.806346094932	-130.806346094932\\
63.5	0.2394	-146.550553069688	-146.550553069688\\
63.5	0.24306	-162.943390355317	-162.943390355317\\
63.5	0.24672	-179.984857951818	-179.984857951818\\
63.5	0.25038	-197.674955859192	-197.674955859192\\
63.5	0.25404	-216.013684077439	-216.013684077439\\
63.5	0.2577	-235.001042606559	-235.001042606559\\
63.5	0.26136	-254.637031446552	-254.637031446552\\
63.5	0.26502	-274.921650597417	-274.921650597417\\
63.5	0.26868	-295.854900059155	-295.854900059155\\
63.5	0.27234	-317.436779831766	-317.436779831766\\
63.5	0.276	-339.66728991525	-339.66728991525\\
63.875	0.093	-25.3626690890361	-25.3626690890361\\
63.875	0.09666	-15.605100628856	-15.605100628856\\
63.875	0.10032	-6.49616247954876	-6.49616247954876\\
63.875	0.10398	1.96414535888584	1.96414535888584\\
63.875	0.10764	9.7758228864476	9.7758228864476\\
63.875	0.1113	16.9388701031365	16.9388701031365\\
63.875	0.11496	23.4532870089527	23.4532870089527\\
63.875	0.11862	29.3190736038957	29.3190736038957\\
63.875	0.12228	34.5362298879659	34.5362298879659\\
63.875	0.12594	39.1047558611637	39.1047558611637\\
63.875	0.1296	43.0246515234888	43.0246515234888\\
63.875	0.13326	46.2959168749412	46.2959168749412\\
63.875	0.13692	48.9185519155207	48.9185519155207\\
63.875	0.14058	50.8925566452273	50.8925566452273\\
63.875	0.14424	52.2179310640607	52.2179310640607\\
63.875	0.1479	52.8946751720224	52.8946751720224\\
63.875	0.15156	52.9227889691103	52.9227889691103\\
63.875	0.15522	52.3022724553255	52.3022724553255\\
63.875	0.15888	51.0331256306681	51.0331256306681\\
63.875	0.16254	49.1153484951383	49.1153484951383\\
63.875	0.1662	46.5489410487355	46.5489410487355\\
63.875	0.16986	43.33390329146	43.33390329146\\
63.875	0.17352	39.4702352233118	39.4702352233118\\
63.875	0.17718	34.9579368442902	34.9579368442902\\
63.875	0.18084	29.7970081543961	29.7970081543961\\
63.875	0.1845	23.9874491536293	23.9874491536293\\
63.875	0.18816	17.5292598419898	17.5292598419898\\
63.875	0.19182	10.4224402194776	10.4224402194776\\
63.875	0.19548	2.66699028609264	2.66699028609264\\
63.875	0.19914	-5.73708995816514	-5.73708995816514\\
63.875	0.2028	-14.7898005132959	-14.7898005132959\\
63.875	0.20646	-24.4911413792997	-24.4911413792997\\
63.875	0.21012	-34.8411125561759	-34.8411125561759\\
63.875	0.21378	-45.8397140439245	-45.8397140439245\\
63.875	0.21744	-57.4869458425467	-57.4869458425467\\
63.875	0.2211	-69.7828079520411	-69.7828079520411\\
63.875	0.22476	-82.7273003724083	-82.7273003724083\\
63.875	0.22842	-96.3204231036484	-96.3204231036484\\
63.875	0.23208	-110.562176145761	-110.562176145761\\
63.875	0.23574	-125.452559498747	-125.452559498747\\
63.875	0.2394	-140.991573162606	-140.991573162606\\
63.875	0.24306	-157.179217137337	-157.179217137337\\
63.875	0.24672	-174.015491422941	-174.015491422941\\
63.875	0.25038	-191.500396019418	-191.500396019418\\
63.875	0.25404	-209.633930926767	-209.633930926767\\
63.875	0.2577	-228.41609614499	-228.41609614499\\
63.875	0.26136	-247.846891674085	-247.846891674085\\
63.875	0.26502	-267.926317514052	-267.926317514052\\
63.875	0.26868	-288.654373664893	-288.654373664893\\
63.875	0.27234	-310.031060126607	-310.031060126607\\
63.875	0.276	-332.056376899193	-332.056376899193\\
64.25	0.093	-28.1429689945534	-28.1429689945534\\
64.25	0.09666	-18.1802072234759	-18.1802072234759\\
64.25	0.10032	-8.86607576327086	-8.86607576327086\\
64.25	0.10398	-0.200574613939352	-0.200574613939352\\
64.25	0.10764	7.81629622452022	7.81629622452022\\
64.25	0.1113	15.1845367521061	15.1845367521061\\
64.25	0.11496	21.90414696882	21.90414696882\\
64.25	0.11862	27.9751268746608	27.9751268746608\\
64.25	0.12228	33.3974764696288	33.3974764696288\\
64.25	0.12594	38.171195753724	38.171195753724\\
64.25	0.1296	42.2962847269465	42.2962847269465\\
64.25	0.13326	45.7727433892962	45.7727433892962\\
64.25	0.13692	48.6005717407731	48.6005717407731\\
64.25	0.14058	50.7797697813775	50.7797697813775\\
64.25	0.14424	52.3103375111083	52.3103375111083\\
64.25	0.1479	53.1922749299673	53.1922749299673\\
64.25	0.15156	53.4255820379531	53.4255820379531\\
64.25	0.15522	53.010258835066	53.010258835066\\
64.25	0.15888	51.9463053213065	51.9463053213065\\
64.25	0.16254	50.2337214966735	50.2337214966735\\
64.25	0.1662	47.8725073611681	47.8725073611681\\
64.25	0.16986	44.86266291479	44.86266291479\\
64.25	0.17352	41.2041881575395	41.2041881575395\\
64.25	0.17718	36.8970830894153	36.8970830894153\\
64.25	0.18084	31.9413477104194	31.9413477104194\\
64.25	0.1845	26.3369820205501	26.3369820205501\\
64.25	0.18816	20.0839860198079	20.0839860198079\\
64.25	0.19182	13.1823597081935	13.1823597081935\\
64.25	0.19548	5.63210308570547	5.63210308570547\\
64.25	0.19914	-2.56678384765496	-2.56678384765496\\
64.25	0.2028	-11.4143010918883	-11.4143010918883\\
64.25	0.20646	-20.9104486469939	-20.9104486469939\\
64.25	0.21012	-31.0552265129727	-31.0552265129727\\
64.25	0.21378	-41.8486346898239	-41.8486346898239\\
64.25	0.21744	-53.2906731775488	-53.2906731775488\\
64.25	0.2211	-65.3813419761454	-65.3813419761454\\
64.25	0.22476	-78.1206410856153	-78.1206410856153\\
64.25	0.22842	-91.508570505958	-91.508570505958\\
64.25	0.23208	-105.545130237173	-105.545130237173\\
64.25	0.23574	-120.230320279261	-120.230320279261\\
64.25	0.2394	-135.564140632222	-135.564140632222\\
64.25	0.24306	-151.546591296056	-151.546591296056\\
64.25	0.24672	-168.177672270763	-168.177672270763\\
64.25	0.25038	-185.457383556342	-185.457383556342\\
64.25	0.25404	-203.385725152794	-203.385725152794\\
64.25	0.2577	-221.962697060119	-221.962697060119\\
64.25	0.26136	-241.188299278316	-241.188299278316\\
64.25	0.26502	-261.062531807387	-261.062531807387\\
64.25	0.26868	-281.58539464733	-281.58539464733\\
64.25	0.27234	-302.756887798146	-302.756887798146\\
64.25	0.276	-324.577011259835	-324.577011259835\\
64.625	0.093	-31.0548162767704	-31.0548162767704\\
64.625	0.09666	-20.8868611947951	-20.8868611947951\\
64.625	0.10032	-11.3675364236927	-11.3675364236927\\
64.625	0.10398	-2.49684196346334	-2.49684196346334\\
64.625	0.10764	5.72522218589313	5.72522218589313\\
64.625	0.1113	13.2986560243772	13.2986560243772\\
64.625	0.11496	20.2234595519881	20.2234595519881\\
64.625	0.11862	26.4996327687267	26.4996327687267\\
64.625	0.12228	32.1271756745921	32.1271756745921\\
64.625	0.12594	37.1060882695851	37.1060882695851\\
64.625	0.1296	41.4363705537049	41.4363705537049\\
64.625	0.13326	45.118022526952	45.118022526952\\
64.625	0.13692	48.1510441893267	48.1510441893267\\
64.625	0.14058	50.5354355408284	50.5354355408284\\
64.625	0.14424	52.271196581457	52.271196581457\\
64.625	0.1479	53.3583273112134	53.3583273112134\\
64.625	0.15156	53.7968277300965	53.7968277300965\\
64.625	0.15522	53.5866978381073	53.5866978381073\\
64.625	0.15888	52.7279376352446	52.7279376352446\\
64.625	0.16254	51.2205471215095	51.2205471215095\\
64.625	0.1662	49.0645262969019	49.0645262969019\\
64.625	0.16986	46.2598751614212	46.2598751614212\\
64.625	0.17352	42.8065937150681	42.8065937150681\\
64.625	0.17718	38.7046819578417	38.7046819578417\\
64.625	0.18084	33.9541398897427	33.9541398897427\\
64.625	0.1845	28.5549675107711	28.5549675107711\\
64.625	0.18816	22.5071648209267	22.5071648209267\\
64.625	0.19182	15.8107318202093	15.8107318202093\\
64.625	0.19548	8.46566850861905	8.46566850861905\\
64.625	0.19914	0.471974886156431	0.471974886156431\\
64.625	0.2028	-8.17034904717912	-8.17034904717912\\
64.625	0.20646	-17.4613032913878	-17.4613032913878\\
64.625	0.21012	-27.4008878464688	-27.4008878464688\\
64.625	0.21378	-37.9891027124222	-37.9891027124222\\
64.625	0.21744	-49.2259478892493	-49.2259478892493\\
64.625	0.2211	-61.111423376949	-61.111423376949\\
64.625	0.22476	-73.645529175521	-73.645529175521\\
64.625	0.22842	-86.8282652849664	-86.8282652849664\\
64.625	0.23208	-100.659631705284	-100.659631705284\\
64.625	0.23574	-115.139628436475	-115.139628436475\\
64.625	0.2394	-130.268255478539	-130.268255478539\\
64.625	0.24306	-146.045512831475	-146.045512831475\\
64.625	0.24672	-162.471400495284	-162.471400495284\\
64.625	0.25038	-179.545918469965	-179.545918469965\\
64.625	0.25404	-197.26906675552	-197.26906675552\\
64.625	0.2577	-215.640845351948	-215.640845351948\\
64.625	0.26136	-234.661254259247	-234.661254259247\\
64.625	0.26502	-254.33029347742	-254.33029347742\\
64.625	0.26868	-274.647963006466	-274.647963006466\\
64.625	0.27234	-295.614262846384	-295.614262846384\\
64.625	0.276	-317.229192997175	-317.229192997175\\
65	0.093	-34.0982109356866	-34.0982109356866\\
65	0.09666	-23.725062542814	-23.725062542814\\
65	0.10032	-14.0005444608142	-14.0005444608142\\
65	0.10398	-4.92465668968748	-4.92465668968748\\
65	0.10764	3.5026007705668	3.5026007705668\\
65	0.1113	11.2812279199483	11.2812279199483\\
65	0.11496	18.411224758457	18.411224758457\\
65	0.11862	24.8925912860929	24.8925912860929\\
65	0.12228	30.7253275028561	30.7253275028561\\
65	0.12594	35.909433408746	35.909433408746\\
65	0.1296	40.4449090037632	40.4449090037632\\
65	0.13326	44.3317542879086	44.3317542879086\\
65	0.13692	47.5699692611801	47.5699692611801\\
65	0.14058	50.1595539235792	50.1595539235792\\
65	0.14424	52.1005082751057	52.1005082751057\\
65	0.1479	53.3928323157594	53.3928323157594\\
65	0.15156	54.0365260455403	54.0365260455403\\
65	0.15522	54.0315894644484	54.0315894644484\\
65	0.15888	53.3780225724836	53.3780225724836\\
65	0.16254	52.0758253696463	52.0758253696463\\
65	0.1662	50.124997855936	50.124997855936\\
65	0.16986	47.5255400313526	47.5255400313526\\
65	0.17352	44.2774518958969	44.2774518958969\\
65	0.17718	40.3807334495679	40.3807334495679\\
65	0.18084	35.8353846923667	35.8353846923667\\
65	0.1845	30.6414056242925	30.6414056242925\\
65	0.18816	24.7987962453454	24.7987962453454\\
65	0.19182	18.3075565555258	18.3075565555258\\
65	0.19548	11.1676865548334	11.1676865548334\\
65	0.19914	3.37918624326812	3.37918624326812\\
65	0.2028	-5.05794437917007	-5.05794437917007\\
65	0.20646	-14.143705312481	-14.143705312481\\
65	0.21012	-23.878096556665	-23.878096556665\\
65	0.21378	-34.2611181117206	-34.2611181117206\\
65	0.21744	-45.2927699776508	-45.2927699776508\\
65	0.2211	-56.9730521544527	-56.9730521544527\\
65	0.22476	-69.3019646421269	-69.3019646421269\\
65	0.22842	-82.2795074406749	-82.2795074406749\\
65	0.23208	-95.905680550095	-95.905680550095\\
65	0.23574	-110.180483970388	-110.180483970388\\
65	0.2394	-125.103917701554	-125.103917701554\\
65	0.24306	-140.675981743593	-140.675981743593\\
65	0.24672	-156.896676096505	-156.896676096505\\
65	0.25038	-173.766000760289	-173.766000760289\\
65	0.25404	-191.283955734945	-191.283955734945\\
65	0.2577	-209.450541020475	-209.450541020475\\
65	0.26136	-228.265756616878	-228.265756616878\\
65	0.26502	-247.729602524153	-247.729602524153\\
65	0.26868	-267.842078742301	-267.842078742301\\
65	0.27234	-288.603185271322	-288.603185271322\\
65	0.276	-310.012922111216	-310.012922111216\\
65.375	0.093	-37.273152971302	-37.273152971302\\
65.375	0.09666	-26.6948112675321	-26.6948112675321\\
65.375	0.10032	-16.7650998746344	-16.7650998746344\\
65.375	0.10398	-7.48401879261041	-7.48401879261041\\
65.375	0.10764	1.14843197854168	1.14843197854168\\
65.375	0.1113	9.13225243882005	9.13225243882005\\
65.375	0.11496	16.4674425882266	16.4674425882266\\
65.375	0.11862	23.1540024267599	23.1540024267599\\
65.375	0.12228	29.1919319544204	29.1919319544204\\
65.375	0.12594	34.5812311712081	34.5812311712081\\
65.375	0.1296	39.3219000771231	39.3219000771231\\
65.375	0.13326	43.4139386721658	43.4139386721658\\
65.375	0.13692	46.8573469563348	46.8573469563348\\
65.375	0.14058	49.6521249296317	49.6521249296317\\
65.375	0.14424	51.7982725920555	51.7982725920555\\
65.375	0.1479	53.2957899436066	53.2957899436066\\
65.375	0.15156	54.1446769842853	54.1446769842853\\
65.375	0.15522	54.3449337140903	54.3449337140903\\
65.375	0.15888	53.8965601330233	53.8965601330233\\
65.375	0.16254	52.7995562410833	52.7995562410833\\
65.375	0.1662	51.0539220382705	51.0539220382705\\
65.375	0.16986	48.6596575245849	48.6596575245849\\
65.375	0.17352	45.6167627000265	45.6167627000265\\
65.375	0.17718	41.9252375645957	41.9252375645957\\
65.375	0.18084	37.5850821182914	37.5850821182914\\
65.375	0.1845	32.596296361115	32.596296361115\\
65.375	0.18816	26.9588802930658	26.9588802930658\\
65.375	0.19182	20.6728339141431	20.6728339141431\\
65.375	0.19548	13.7381572243485	13.7381572243485\\
65.375	0.19914	6.15485022368011	6.15485022368011\\
65.375	0.2028	-2.07708708786026	-2.07708708786026\\
65.375	0.20646	-10.9576547102738	-10.9576547102738\\
65.375	0.21012	-20.48685264356	-20.48685264356\\
65.375	0.21378	-30.6646808877183	-30.6646808877183\\
65.375	0.21744	-41.4911394427506	-41.4911394427506\\
65.375	0.2211	-52.9662283086548	-52.9662283086548\\
65.375	0.22476	-65.089947485432	-65.089947485432\\
65.375	0.22842	-77.8622969730823	-77.8622969730823\\
65.375	0.23208	-91.283276771605	-91.283276771605\\
65.375	0.23574	-105.352886881001	-105.352886881001\\
65.375	0.2394	-120.071127301269	-120.071127301269\\
65.375	0.24306	-135.43799803241	-135.43799803241\\
65.375	0.24672	-151.453499074424	-151.453499074424\\
65.375	0.25038	-168.117630427311	-168.117630427311\\
65.375	0.25404	-185.430392091071	-185.430392091071\\
65.375	0.2577	-203.391784065703	-203.391784065703\\
65.375	0.26136	-222.001806351208	-222.001806351208\\
65.375	0.26502	-241.260458947586	-241.260458947586\\
65.375	0.26868	-261.167741854836	-261.167741854836\\
65.375	0.27234	-281.72365507296	-281.72365507296\\
65.375	0.276	-302.928198601956	-302.928198601956\\
65.75	0.093	-40.5796423836165	-40.5796423836165\\
65.75	0.09666	-29.7961073689487	-29.7961073689487\\
65.75	0.10032	-19.6612026651542	-19.6612026651542\\
65.75	0.10398	-10.1749282722319	-10.1749282722319\\
65.75	0.10764	-1.33728419018291	-1.33728419018291\\
65.75	0.1113	6.85172958099372	6.85172958099372\\
65.75	0.11496	14.3921130412971	14.3921130412971\\
65.75	0.11862	21.2838661907282	21.2838661907282\\
65.75	0.12228	27.5269890292866	27.5269890292866\\
65.75	0.12594	33.1214815569717	33.1214815569717\\
65.75	0.1296	38.067343773784	38.067343773784\\
65.75	0.13326	42.3645756797241	42.3645756797241\\
65.75	0.13692	46.0131772747913	46.0131772747913\\
65.75	0.14058	49.0131485589851	49.0131485589851\\
65.75	0.14424	51.3644895323063	51.3644895323063\\
65.75	0.1479	53.0672001947556	53.0672001947556\\
65.75	0.15156	54.1212805463313	54.1212805463313\\
65.75	0.15522	54.5267305870341	54.5267305870341\\
65.75	0.15888	54.2835503168644	54.2835503168644\\
65.75	0.16254	53.3917397358223	53.3917397358223\\
65.75	0.1662	51.8512988439068	51.8512988439068\\
65.75	0.16986	49.6622276411185	49.6622276411185\\
65.75	0.17352	46.8245261274579	46.8245261274579\\
65.75	0.17718	43.3381943029241	43.3381943029241\\
65.75	0.18084	39.2032321675181	39.2032321675181\\
65.75	0.1845	34.4196397212386	34.4196397212386\\
65.75	0.18816	28.9874169640872	28.9874169640872\\
65.75	0.19182	22.9065638960623	22.9065638960623\\
65.75	0.19548	16.1770805171645	16.1770805171645\\
65.75	0.19914	8.798966827394	8.798966827394\\
65.75	0.2028	0.772222826751431	0.772222826751431\\
65.75	0.20646	-7.90315148476475	-7.90315148476475\\
65.75	0.21012	-17.2271561071536	-17.2271561071536\\
65.75	0.21378	-27.199791040415	-27.199791040415\\
65.75	0.21744	-37.8210562845486	-37.8210562845486\\
65.75	0.2211	-49.0909518395558	-49.0909518395558\\
65.75	0.22476	-61.0094777054358	-61.0094777054358\\
65.75	0.22842	-73.5766338821882	-73.5766338821882\\
65.75	0.23208	-86.7924203698135	-86.7924203698135\\
65.75	0.23574	-100.656837168312	-100.656837168312\\
65.75	0.2394	-115.169884277683	-115.169884277683\\
65.75	0.24306	-130.331561697926	-130.331561697926\\
65.75	0.24672	-146.141869429043	-146.141869429043\\
65.75	0.25038	-162.600807471032	-162.600807471032\\
65.75	0.25404	-179.708375823894	-179.708375823894\\
65.75	0.2577	-197.464574487628	-197.464574487628\\
65.75	0.26136	-215.869403462236	-215.869403462236\\
65.75	0.26502	-234.922862747717	-234.922862747717\\
65.75	0.26868	-254.624952344069	-254.624952344069\\
65.75	0.27234	-274.975672251295	-274.975672251295\\
65.75	0.276	-295.975022469394	-295.975022469394\\
66.125	0.093	-44.0176791726298	-44.0176791726298\\
66.125	0.09666	-33.0289508470646	-33.0289508470646\\
66.125	0.10032	-22.6888528323723	-22.6888528323723\\
66.125	0.10398	-12.9973851285527	-12.9973851285527\\
66.125	0.10764	-3.9545477356063	-3.9545477356063\\
66.125	0.1113	4.43965934646769	4.43965934646769\\
66.125	0.11496	12.1852361176694	12.1852361176694\\
66.125	0.11862	19.2821825779974	19.2821825779974\\
66.125	0.12228	25.7304987274526	25.7304987274526\\
66.125	0.12594	31.5301845660355	31.5301845660355\\
66.125	0.1296	36.6812400937457	36.6812400937457\\
66.125	0.13326	41.1836653105831	41.1836653105831\\
66.125	0.13692	45.0374602165477	45.0374602165477\\
66.125	0.14058	48.2426248116393	48.2426248116393\\
66.125	0.14424	50.7991590958583	50.7991590958583\\
66.125	0.1479	52.7070630692045	52.7070630692045\\
66.125	0.15156	53.966336731678	53.966336731678\\
66.125	0.15522	54.5769800832786	54.5769800832786\\
66.125	0.15888	54.5389931240063	54.5389931240063\\
66.125	0.16254	53.8523758538615	53.8523758538615\\
66.125	0.1662	52.5171282728434	52.5171282728434\\
66.125	0.16986	50.5332503809529	50.5332503809529\\
66.125	0.17352	47.9007421781897	47.9007421781897\\
66.125	0.17718	44.6196036645537	44.6196036645537\\
66.125	0.18084	40.6898348400446	40.6898348400446\\
66.125	0.1845	36.1114357046629	36.1114357046629\\
66.125	0.18816	30.8844062584088	30.8844062584088\\
66.125	0.19182	25.0087465012813	25.0087465012813\\
66.125	0.19548	18.4844564332814	18.4844564332814\\
66.125	0.19914	11.3115360544086	11.3115360544086\\
66.125	0.2028	3.48998536466297	3.48998536466297\\
66.125	0.20646	-4.9801956359554	-4.9801956359554\\
66.125	0.21012	-14.0990069474465	-14.0990069474465\\
66.125	0.21378	-23.86644856981	-23.86644856981\\
66.125	0.21744	-34.2825205030467	-34.2825205030467\\
66.125	0.2211	-45.3472227471561	-45.3472227471561\\
66.125	0.22476	-57.0605553021383	-57.0605553021383\\
66.125	0.22842	-69.4225181679938	-69.4225181679938\\
66.125	0.23208	-82.4331113447213	-82.4331113447213\\
66.125	0.23574	-96.0923348323222	-96.0923348323222\\
66.125	0.2394	-110.400188630795	-110.400188630795\\
66.125	0.24306	-125.356672740141	-125.356672740141\\
66.125	0.24672	-140.961787160361	-140.961787160361\\
66.125	0.25038	-157.215531891452	-157.215531891452\\
66.125	0.25404	-174.117906933417	-174.117906933417\\
66.125	0.2577	-191.668912286254	-191.668912286254\\
66.125	0.26136	-209.868547949964	-209.868547949964\\
66.125	0.26502	-228.716813924547	-228.716813924547\\
66.125	0.26868	-248.213710210002	-248.213710210002\\
66.125	0.27234	-268.359236806331	-268.359236806331\\
66.125	0.276	-289.153393713532	-289.153393713532\\
66.5	0.093	-47.587263338343	-47.587263338343\\
66.5	0.09666	-36.3933417018805	-36.3933417018805\\
66.5	0.10032	-25.8480503762908	-25.8480503762908\\
66.5	0.10398	-15.9513893615733	-15.9513893615733\\
66.5	0.10764	-6.70335865772961	-6.70335865772961\\
66.5	0.1113	1.89604173524219	1.89604173524219\\
66.5	0.11496	9.84681181734123	9.84681181734123\\
66.5	0.11862	17.1489515885666	17.1489515885666\\
66.5	0.12228	23.8024610489201	23.8024610489201\\
66.5	0.12594	29.8073401983999	29.8073401983999\\
66.5	0.1296	35.1635890370079	35.1635890370079\\
66.5	0.13326	39.8712075647427	39.8712075647427\\
66.5	0.13692	43.9301957816046	43.9301957816046\\
66.5	0.14058	47.340553687594	47.340553687594\\
66.5	0.14424	50.1022812827099	50.1022812827099\\
66.5	0.1479	52.215378566954	52.215378566954\\
66.5	0.15156	53.6798455403252	53.6798455403252\\
66.5	0.15522	54.4956822028228	54.4956822028228\\
66.5	0.15888	54.6628885544483	54.6628885544483\\
66.5	0.16254	54.1814645952009	54.1814645952009\\
66.5	0.1662	53.0514103250805	53.0514103250805\\
66.5	0.16986	51.2727257440874	51.2727257440874\\
66.5	0.17352	48.8454108522216	48.8454108522216\\
66.5	0.17718	45.7694656494833	45.7694656494833\\
66.5	0.18084	42.0448901358716	42.0448901358716\\
66.5	0.1845	37.6716843113877	37.6716843113877\\
66.5	0.18816	32.649848176031	32.649848176031\\
66.5	0.19182	26.9793817298013	26.9793817298013\\
66.5	0.19548	20.6602849726987	20.6602849726987\\
66.5	0.19914	13.6925579047233	13.6925579047233\\
66.5	0.2028	6.07620052587504	6.07620052587504\\
66.5	0.20646	-2.18878716384552	-2.18878716384552\\
66.5	0.21012	-11.1024051644392	-11.1024051644392\\
66.5	0.21378	-20.6646534759054	-20.6646534759054\\
66.5	0.21744	-30.8755320982443	-30.8755320982443\\
66.5	0.2211	-41.7350410314568	-41.7350410314568\\
66.5	0.22476	-53.2431802755411	-53.2431802755411\\
66.5	0.22842	-65.3999498304988	-65.3999498304988\\
66.5	0.23208	-78.205349696329	-78.205349696329\\
66.5	0.23574	-91.6593798730321	-91.6593798730321\\
66.5	0.2394	-105.762040360608	-105.762040360608\\
66.5	0.24306	-120.513331159057	-120.513331159057\\
66.5	0.24672	-135.913252268378	-135.913252268378\\
66.5	0.25038	-151.961803688573	-151.961803688573\\
66.5	0.25404	-168.658985419639	-168.658985419639\\
66.5	0.2577	-186.00479746158	-186.00479746158\\
66.5	0.26136	-203.999239814392	-203.999239814392\\
66.5	0.26502	-222.642312478077	-222.642312478077\\
66.5	0.26868	-241.934015452635	-241.934015452635\\
66.5	0.27234	-261.874348738065	-261.874348738065\\
66.5	0.276	-282.463312334369	-282.463312334369\\
66.875	0.093	-51.288394880755	-51.288394880755\\
66.875	0.09666	-39.8892799333946	-39.8892799333946\\
66.875	0.10032	-29.1387952969071	-29.1387952969071\\
66.875	0.10398	-19.0369409712928	-19.0369409712928\\
66.875	0.10764	-9.58371695655126	-9.58371695655126\\
66.875	0.1113	-0.779123252682098	-0.779123252682098\\
66.875	0.11496	7.3768401403143	7.3768401403143\\
66.875	0.11862	14.8841732224375	14.8841732224375\\
66.875	0.12228	21.7428759936879	21.7428759936879\\
66.875	0.12594	27.952948454066	27.952948454066\\
66.875	0.1296	33.5143906035713	33.5143906035713\\
66.875	0.13326	38.4272024422035	38.4272024422035\\
66.875	0.13692	42.6913839699627	42.6913839699627\\
66.875	0.14058	46.30693518685	46.30693518685\\
66.875	0.14424	49.2738560928636	49.2738560928636\\
66.875	0.1479	51.5921466880051	51.5921466880051\\
66.875	0.15156	53.2618069722737	53.2618069722737\\
66.875	0.15522	54.282836945669	54.282836945669\\
66.875	0.15888	54.6552366081919	54.6552366081919\\
66.875	0.16254	54.3790059598418	54.3790059598418\\
66.875	0.1662	53.4541450006193	53.4541450006193\\
66.875	0.16986	51.8806537305236	51.8806537305236\\
66.875	0.17352	49.6585321495555	49.6585321495555\\
66.875	0.17718	46.7877802577142	46.7877802577142\\
66.875	0.18084	43.2683980550007	43.2683980550007\\
66.875	0.1845	39.1003855414137	39.1003855414137\\
66.875	0.18816	34.2837427169549	34.2837427169549\\
66.875	0.19182	28.8184695816225	28.8184695816225\\
66.875	0.19548	22.7045661354173	22.7045661354173\\
66.875	0.19914	15.9420323783393	15.9420323783393\\
66.875	0.2028	8.53086831038922	8.53086831038922\\
66.875	0.20646	0.47107393156557	0.47107393156557\\
66.875	0.21012	-8.23735075813079	-8.23735075813079\\
66.875	0.21378	-17.5944057586987	-17.5944057586987\\
66.875	0.21744	-27.6000910701407	-27.6000910701407\\
66.875	0.2211	-38.2544066924554	-38.2544066924554\\
66.875	0.22476	-49.5573526256424	-49.5573526256424\\
66.875	0.22842	-61.5089288697027	-61.5089288697027\\
66.875	0.23208	-74.109135424635	-74.109135424635\\
66.875	0.23574	-87.3579722904408	-87.3579722904408\\
66.875	0.2394	-101.255439467119	-101.255439467119\\
66.875	0.24306	-115.80153695467	-115.80153695467\\
66.875	0.24672	-130.996264753094	-130.996264753094\\
66.875	0.25038	-146.839622862391	-146.839622862391\\
66.875	0.25404	-163.33161128256	-163.33161128256\\
66.875	0.2577	-180.472230013603	-180.472230013603\\
66.875	0.26136	-198.261479055518	-198.261479055518\\
66.875	0.26502	-216.699358408305	-216.699358408305\\
66.875	0.26868	-235.785868071966	-235.785868071966\\
66.875	0.27234	-255.521008046499	-255.521008046499\\
66.875	0.276	-275.904778331905	-275.904778331905\\
67.25	0.093	-55.121073799866	-55.121073799866\\
67.25	0.09666	-43.5167655416083	-43.5167655416083\\
67.25	0.10032	-32.5610875942234	-32.5610875942234\\
67.25	0.10398	-22.2540399577113	-22.2540399577113\\
67.25	0.10764	-12.5956226320719	-12.5956226320719\\
67.25	0.1113	-3.58583561730541	-3.58583561730541\\
67.25	0.11496	4.77532108658835	4.77532108658835\\
67.25	0.11862	12.4878474796093	12.4878474796093\\
67.25	0.12228	19.5517435617576	19.5517435617576\\
67.25	0.12594	25.967009333033	25.967009333033\\
67.25	0.1296	31.7336447934357	31.7336447934357\\
67.25	0.13326	36.8516499429657	36.8516499429657\\
67.25	0.13692	41.3210247816227	41.3210247816227\\
67.25	0.14058	45.1417693094069	45.1417693094069\\
67.25	0.14424	48.3138835263184	48.3138835263184\\
67.25	0.1479	50.8373674323572	50.8373674323572\\
67.25	0.15156	52.7122210275231	52.7122210275231\\
67.25	0.15522	53.9384443118158	53.9384443118158\\
67.25	0.15888	54.5160372852365	54.5160372852365\\
67.25	0.16254	54.4449999477843	54.4449999477843\\
67.25	0.1662	53.7253322994586	53.7253322994586\\
67.25	0.16986	52.3570343402607	52.3570343402607\\
67.25	0.17352	50.34010607019	50.34010607019\\
67.25	0.17718	47.6745474892465	47.6745474892465\\
67.25	0.18084	44.3603585974299	44.3603585974299\\
67.25	0.1845	40.3975393947412	40.3975393947412\\
67.25	0.18816	35.7860898811792	35.7860898811792\\
67.25	0.19182	30.5260100567447	30.5260100567447\\
67.25	0.19548	24.6172999214368	24.6172999214368\\
67.25	0.19914	18.0599594752566	18.0599594752566\\
67.25	0.2028	10.8539887182039	10.8539887182039\\
67.25	0.20646	2.99938765027809	2.99938765027809\\
67.25	0.21012	-5.50384372852091	-5.50384372852091\\
67.25	0.21378	-14.6557054181919	-14.6557054181919\\
67.25	0.21744	-24.4561974187361	-24.4561974187361\\
67.25	0.2211	-34.905319730153	-34.905319730153\\
67.25	0.22476	-46.0030723524426	-46.0030723524426\\
67.25	0.22842	-57.7494552856051	-57.7494552856051\\
67.25	0.23208	-70.1444685296401	-70.1444685296401\\
67.25	0.23574	-83.1881120845485	-83.1881120845485\\
67.25	0.2394	-96.8803859503291	-96.8803859503291\\
67.25	0.24306	-111.221290126983	-111.221290126983\\
67.25	0.24672	-126.210824614509	-126.210824614509\\
67.25	0.25038	-141.848989412908	-141.848989412908\\
67.25	0.25404	-158.13578452218	-158.13578452218\\
67.25	0.2577	-175.071209942325	-175.071209942325\\
67.25	0.26136	-192.655265673342	-192.655265673342\\
67.25	0.26502	-210.887951715232	-210.887951715232\\
67.25	0.26868	-229.769268067995	-229.769268067995\\
67.25	0.27234	-249.299214731632	-249.299214731632\\
67.25	0.276	-269.47779170614	-269.47779170614\\
67.625	0.093	-59.0853000956764	-59.0853000956764\\
67.625	0.09666	-47.275798526521	-47.275798526521\\
67.625	0.10032	-36.1149272682387	-36.1149272682387\\
67.625	0.10398	-25.6026863208292	-25.6026863208292\\
67.625	0.10764	-15.7390756842921	-15.7390756842921\\
67.625	0.1113	-6.52409535862819	-6.52409535862819\\
67.625	0.11496	2.04225465616292	2.04225465616292\\
67.625	0.11862	9.95997436008173	9.95997436008173\\
67.625	0.12228	17.2290637531269	17.2290637531269\\
67.625	0.12594	23.8495228353001	23.8495228353001\\
67.625	0.1296	29.8213516066002	29.8213516066002\\
67.625	0.13326	35.1445500670275	35.1445500670275\\
67.625	0.13692	39.8191182165824	39.8191182165824\\
67.625	0.14058	43.8450560552639	43.8450560552639\\
67.625	0.14424	47.2223635830727	47.2223635830727\\
67.625	0.1479	49.9510408000093	49.9510408000093\\
67.625	0.15156	52.0310877060726	52.0310877060726\\
67.625	0.15522	53.4625043012632	53.4625043012632\\
67.625	0.15888	54.2452905855812	54.2452905855812\\
67.625	0.16254	54.3794465590263	54.3794465590263\\
67.625	0.1662	53.8649722215985	53.8649722215985\\
67.625	0.16986	52.7018675732979	52.7018675732979\\
67.625	0.17352	50.8901326141246	50.8901326141246\\
67.625	0.17718	48.4297673440789	48.4297673440789\\
67.625	0.18084	45.3207717631601	45.3207717631601\\
67.625	0.1845	41.5631458713683	41.5631458713683\\
67.625	0.18816	37.1568896687041	37.1568896687041\\
67.625	0.19182	32.1020031551669	32.1020031551669\\
67.625	0.19548	26.3984863307569	26.3984863307569\\
67.625	0.19914	20.0463391954745	20.0463391954745\\
67.625	0.2028	13.0455617493187	13.0455617493187\\
67.625	0.20646	5.39615399229024	5.39615399229024\\
67.625	0.21012	-2.90188407561095	-2.90188407561095\\
67.625	0.21378	-11.8485524543846	-11.8485524543846\\
67.625	0.21744	-21.443851144031	-21.443851144031\\
67.625	0.2211	-31.6877801445501	-31.6877801445501\\
67.625	0.22476	-42.5803394559423	-42.5803394559423\\
67.625	0.22842	-54.1215290782075	-54.1215290782075\\
67.625	0.23208	-66.3113490113451	-66.3113490113451\\
67.625	0.23574	-79.1497992553557	-79.1497992553557\\
67.625	0.2394	-92.636879810239	-92.636879810239\\
67.625	0.24306	-106.772590675995	-106.772590675995\\
67.625	0.24672	-121.556931852624	-121.556931852624\\
67.625	0.25038	-136.989903340126	-136.989903340126\\
67.625	0.25404	-153.0715051385	-153.0715051385\\
67.625	0.2577	-169.801737247747	-169.801737247747\\
67.625	0.26136	-187.180599667867	-187.180599667867\\
67.625	0.26502	-205.20809239886	-205.20809239886\\
67.625	0.26868	-223.884215440725	-223.884215440725\\
67.625	0.27234	-243.208968793464	-243.208968793464\\
67.625	0.276	-263.182352457075	-263.182352457075\\
68	0.093	-63.1810737681859	-63.1810737681859\\
68	0.09666	-51.1663788881331	-51.1663788881331\\
68	0.10032	-39.8003143189535	-39.8003143189535\\
68	0.10398	-29.0828800606462	-29.0828800606462\\
68	0.10764	-19.0140761132117	-19.0140761132117\\
68	0.1113	-9.59390247665044	-9.59390247665044\\
68	0.11496	-0.822359150961518	-0.822359150961518\\
68	0.11862	7.30055386385465	7.30055386385465\\
68	0.12228	14.7748365677976	14.7748365677976\\
68	0.12594	21.6004889608677	21.6004889608677\\
68	0.1296	27.7775110430656	27.7775110430656\\
68	0.13326	33.3059028143907	33.3059028143907\\
68	0.13692	38.1856642748425	38.1856642748425\\
68	0.14058	42.4167954244218	42.4167954244218\\
68	0.14424	45.9992962631285	45.9992962631285\\
68	0.1479	48.933166790962	48.933166790962\\
68	0.15156	51.2184070079231	51.2184070079231\\
68	0.15522	52.8550169140115	52.8550169140115\\
68	0.15888	53.8429965092264	53.8429965092264\\
68	0.16254	54.1823457935693	54.1823457935693\\
68	0.1662	53.8730647670393	53.8730647670393\\
68	0.16986	52.9151534296361	52.9151534296361\\
68	0.17352	51.3086117813606	51.3086117813606\\
68	0.17718	49.0534398222122	49.0534398222122\\
68	0.18084	46.1496375521908	46.1496375521908\\
68	0.1845	42.5972049712964	42.5972049712964\\
68	0.18816	38.39614207953	38.39614207953\\
68	0.19182	33.5464488768902	33.5464488768902\\
68	0.19548	28.0481253633775	28.0481253633775\\
68	0.19914	21.9011715389925	21.9011715389925\\
68	0.2028	15.1055874037345	15.1055874037345\\
68	0.20646	7.66137295760382	7.66137295760382\\
68	0.21012	-0.431471799400015	-0.431471799400015\\
68	0.21378	-9.17294686727587	-9.17294686727587\\
68	0.21744	-18.5630522460253	-18.5630522460253\\
68	0.2211	-28.6017879356475	-28.6017879356475\\
68	0.22476	-39.2891539361415	-39.2891539361415\\
68	0.22842	-50.6251502475093	-50.6251502475093\\
68	0.23208	-62.6097768697491	-62.6097768697491\\
68	0.23574	-75.2430338028623	-75.2430338028623\\
68	0.2394	-88.5249210468482	-88.5249210468482\\
68	0.24306	-102.455438601707	-102.455438601707\\
68	0.24672	-117.034586467438	-117.034586467438\\
68	0.25038	-132.262364644042	-132.262364644042\\
68	0.25404	-148.138773131519	-148.138773131519\\
68	0.2577	-164.663811929869	-164.663811929869\\
68	0.26136	-181.837481039091	-181.837481039091\\
68	0.26502	-199.659780459186	-199.659780459186\\
68	0.26868	-218.130710190155	-218.130710190155\\
68	0.27234	-237.250270231995	-237.250270231995\\
68	0.276	-257.018460584709	-257.018460584709\\
68.375	0.093	-67.4083948173945	-67.4083948173945\\
68.375	0.09666	-55.1885066264443	-55.1885066264443\\
68.375	0.10032	-43.6172487463664	-43.6172487463664\\
68.375	0.10398	-32.6946211771617	-32.6946211771617\\
68.375	0.10764	-22.4206239188303	-22.4206239188303\\
68.375	0.1113	-12.7952569713713	-12.7952569713713\\
68.375	0.11496	-3.81852033478452	-3.81852033478452\\
68.375	0.11862	4.509585990929	4.509585990929\\
68.375	0.12228	12.1890620057693	12.1890620057693\\
68.375	0.12594	19.2199077097372	19.2199077097372\\
68.375	0.1296	25.602123102832	25.602123102832\\
68.375	0.13326	31.335708185055	31.335708185055\\
68.375	0.13692	36.4206629564046	36.4206629564046\\
68.375	0.14058	40.8569874168812	40.8569874168812\\
68.375	0.14424	44.6446815664852	44.6446815664852\\
68.375	0.1479	47.783745405217	47.783745405217\\
68.375	0.15156	50.2741789330751	50.2741789330751\\
68.375	0.15522	52.1159821500607	52.1159821500607\\
68.375	0.15888	53.3091550561739	53.3091550561739\\
68.375	0.16254	53.8536976514138	53.8536976514138\\
68.375	0.1662	53.7496099357811	53.7496099357811\\
68.375	0.16986	52.9968919092757	52.9968919092757\\
68.375	0.17352	51.5955435718976	51.5955435718976\\
68.375	0.17718	49.5455649236466	49.5455649236466\\
68.375	0.18084	46.8469559645225	46.8469559645225\\
68.375	0.1845	43.4997166945263	43.4997166945263\\
68.375	0.18816	39.5038471136573	39.5038471136573\\
68.375	0.19182	34.8593472219148	34.8593472219148\\
68.375	0.19548	29.5662170193	29.5662170193\\
68.375	0.19914	23.6244565058123	23.6244565058123\\
68.375	0.2028	17.0340656814517	17.0340656814517\\
68.375	0.20646	9.79504454621838	9.79504454621838\\
68.375	0.21012	1.90739310011236	1.90739310011236\\
68.375	0.21378	-6.62888865686568	-6.62888865686568\\
68.375	0.21744	-15.8138007247182	-15.8138007247182\\
68.375	0.2211	-25.6473431034426	-25.6473431034426\\
68.375	0.22476	-36.1295157930392	-36.1295157930392\\
68.375	0.22842	-47.2603187935092	-47.2603187935092\\
68.375	0.23208	-59.0397521048517	-59.0397521048517\\
68.375	0.23574	-71.4678157270675	-71.4678157270675\\
68.375	0.2394	-84.5445096601557	-84.5445096601557\\
68.375	0.24306	-98.2698339041167	-98.2698339041167\\
68.375	0.24672	-112.64378845895	-112.64378845895\\
68.375	0.25038	-127.666373324657	-127.666373324657\\
68.375	0.25404	-143.337588501236	-143.337588501236\\
68.375	0.2577	-159.657433988688	-159.657433988688\\
68.375	0.26136	-176.625909787014	-176.625909787014\\
68.375	0.26502	-194.243015896211	-194.243015896211\\
68.375	0.26868	-212.508752316282	-212.508752316282\\
68.375	0.27234	-231.423119047225	-231.423119047225\\
68.375	0.276	-250.986116089041	-250.986116089041\\
68.75	0.093	-71.767263243302	-71.767263243302\\
68.75	0.09666	-59.3421817414544	-59.3421817414544\\
68.75	0.10032	-47.5657305504797	-47.5657305504797\\
68.75	0.10398	-36.4379096703772	-36.4379096703772\\
68.75	0.10764	-25.9587191011479	-25.9587191011479\\
68.75	0.1113	-16.1281588427915	-16.1281588427915\\
68.75	0.11496	-6.94622889530746	-6.94622889530746\\
68.75	0.11862	1.58707074130342	1.58707074130342\\
68.75	0.12228	9.47174006704154	9.47174006704154\\
68.75	0.12594	16.7077790819073	16.7077790819073\\
68.75	0.1296	23.2951877858994	23.2951877858994\\
68.75	0.13326	29.2339661790197	29.2339661790197\\
68.75	0.13692	34.5241142612671	34.5241142612671\\
68.75	0.14058	39.1656320326416	39.1656320326416\\
68.75	0.14424	43.1585194931425	43.1585194931425\\
68.75	0.1479	46.5027766427717	46.5027766427717\\
68.75	0.15156	49.1984034815275	49.1984034815275\\
68.75	0.15522	51.2454000094106	51.2454000094106\\
68.75	0.15888	52.6437662264211	52.6437662264211\\
68.75	0.16254	53.3935021325592	53.3935021325592\\
68.75	0.1662	53.4946077278234	53.4946077278234\\
68.75	0.16986	52.9470830122159	52.9470830122159\\
68.75	0.17352	51.7509279857355	51.7509279857355\\
68.75	0.17718	49.9061426483814	49.9061426483814\\
68.75	0.18084	47.4127270001557	47.4127270001557\\
68.75	0.1845	44.2706810410568	44.2706810410568\\
68.75	0.18816	40.4800047710847	40.4800047710847\\
68.75	0.19182	36.0406981902401	36.0406981902401\\
68.75	0.19548	30.952761298523	30.952761298523\\
68.75	0.19914	25.2161940959322	25.2161940959322\\
68.75	0.2028	18.8309965824694	18.8309965824694\\
68.75	0.20646	11.7971687581339	11.7971687581339\\
68.75	0.21012	4.11471062292526	4.11471062292526\\
68.75	0.21378	-4.21637782315588	-4.21637782315588\\
68.75	0.21744	-13.1960965801097	-13.1960965801097\\
68.75	0.2211	-22.8244456479372	-22.8244456479372\\
68.75	0.22476	-33.1014250266364	-33.1014250266364\\
68.75	0.22842	-44.0270347162091	-44.0270347162091\\
68.75	0.23208	-55.6012747166537	-55.6012747166537\\
68.75	0.23574	-67.8241450279722	-67.8241450279722\\
68.75	0.2394	-80.6956456501625	-80.6956456501625\\
68.75	0.24306	-94.2157765832262	-94.2157765832262\\
68.75	0.24672	-108.384537827163	-108.384537827163\\
68.75	0.25038	-123.201929381972	-123.201929381972\\
68.75	0.25404	-138.667951247653	-138.667951247653\\
68.75	0.2577	-154.782603424208	-154.782603424208\\
68.75	0.26136	-171.545885911636	-171.545885911636\\
68.75	0.26502	-188.957798709936	-188.957798709936\\
68.75	0.26868	-207.018341819109	-207.018341819109\\
68.75	0.27234	-225.727515239154	-225.727515239154\\
68.75	0.276	-245.085318970073	-245.085318970073\\
69.125	0.093	-76.2576790459094	-76.2576790459094\\
69.125	0.09666	-63.6274042331645	-63.6274042331645\\
69.125	0.10032	-51.6457597312915	-51.6457597312915\\
69.125	0.10398	-40.3127455402921	-40.3127455402921\\
69.125	0.10764	-29.6283616601655	-29.6283616601655\\
69.125	0.1113	-19.5926080909113	-19.5926080909113\\
69.125	0.11496	-10.2054848325299	-10.2054848325299\\
69.125	0.11862	-1.46699188502163	-1.46699188502163\\
69.125	0.12228	6.6228707516143	6.6228707516143\\
69.125	0.12594	14.0641030773774	14.0641030773774\\
69.125	0.1296	20.8567050922674	20.8567050922674\\
69.125	0.13326	27.000676796285	27.000676796285\\
69.125	0.13692	32.4960181894298	32.4960181894298\\
69.125	0.14058	37.3427292717012	37.3427292717012\\
69.125	0.14424	41.5408100431004	41.5408100431004\\
69.125	0.1479	45.0902605036268	45.0902605036268\\
69.125	0.15156	47.9910806532801	47.9910806532801\\
69.125	0.15522	50.2432704920609	50.2432704920609\\
69.125	0.15888	51.8468300199688	51.8468300199688\\
69.125	0.16254	52.8017592370043	52.8017592370043\\
69.125	0.1662	53.1080581431668	53.1080581431668\\
69.125	0.16986	52.7657267384561	52.7657267384561\\
69.125	0.17352	51.7747650228731	51.7747650228731\\
69.125	0.17718	50.1351729964173	50.1351729964173\\
69.125	0.18084	47.8469506590884	47.8469506590884\\
69.125	0.1845	44.9100980108869	44.9100980108869\\
69.125	0.18816	41.3246150518127	41.3246150518127\\
69.125	0.19182	37.0905017818653	37.0905017818653\\
69.125	0.19548	32.2077582010456	32.2077582010456\\
69.125	0.19914	26.6763843093527	26.6763843093527\\
69.125	0.2028	20.4963801067877	20.4963801067877\\
69.125	0.20646	13.6677455933491	13.6677455933491\\
69.125	0.21012	6.19048076903823	6.19048076903823\\
69.125	0.21378	-1.93541436614555	-1.93541436614555\\
69.125	0.21744	-10.7099398122016	-10.7099398122016\\
69.125	0.2211	-20.1330955691312	-20.1330955691312\\
69.125	0.22476	-30.2048816369331	-30.2048816369331\\
69.125	0.22842	-40.9252980156084	-40.9252980156084\\
69.125	0.23208	-52.2943447051557	-52.2943447051557\\
69.125	0.23574	-64.3120217055764	-64.3120217055764\\
69.125	0.2394	-76.9783290168698	-76.9783290168698\\
69.125	0.24306	-90.2932666390357	-90.2932666390357\\
69.125	0.24672	-104.256834572075	-104.256834572075\\
69.125	0.25038	-118.869032815986	-118.869032815986\\
69.125	0.25404	-134.12986137077	-134.12986137077\\
69.125	0.2577	-150.039320236428	-150.039320236428\\
69.125	0.26136	-166.597409412957	-166.597409412957\\
69.125	0.26502	-183.80412890036	-183.80412890036\\
69.125	0.26868	-201.659478698635	-201.659478698635\\
69.125	0.27234	-220.163458807784	-220.163458807784\\
69.125	0.276	-239.316069227805	-239.316069227805\\
69.5	0.093	-80.8796422252159	-80.8796422252159\\
69.5	0.09666	-68.0441741015732	-68.0441741015732\\
69.5	0.10032	-55.8573362888033	-55.8573362888033\\
69.5	0.10398	-44.319128786906	-44.319128786906\\
69.5	0.10764	-33.4295515958817	-33.4295515958817\\
69.5	0.1113	-23.1886047157301	-23.1886047157301\\
69.5	0.11496	-13.5962881464513	-13.5962881464513\\
69.5	0.11862	-4.65260188804524	-4.65260188804524\\
69.5	0.12228	3.64245405948805	3.64245405948805\\
69.5	0.12594	11.2888796961485	11.2888796961485\\
69.5	0.1296	18.2866750219363	18.2866750219363\\
69.5	0.13326	24.6358400368513	24.6358400368513\\
69.5	0.13692	30.3363747408934	30.3363747408934\\
69.5	0.14058	35.3882791340631	35.3882791340631\\
69.5	0.14424	39.7915532163591	39.7915532163591\\
69.5	0.1479	43.5461969877834	43.5461969877834\\
69.5	0.15156	46.6522104483345	46.6522104483345\\
69.5	0.15522	49.1095935980122	49.1095935980122\\
69.5	0.15888	50.918346436818	50.918346436818\\
69.5	0.16254	52.0784689647508	52.0784689647508\\
69.5	0.1662	52.5899611818106	52.5899611818106\\
69.5	0.16986	52.4528230879973	52.4528230879973\\
69.5	0.17352	51.6670546833121	51.6670546833121\\
69.5	0.17718	50.2326559677537	50.2326559677537\\
69.5	0.18084	48.1496269413226	48.1496269413226\\
69.5	0.1845	45.4179676040185	45.4179676040185\\
69.5	0.18816	42.037677955842	42.037677955842\\
69.5	0.19182	38.0087579967925	38.0087579967925\\
69.5	0.19548	33.3312077268702	33.3312077268702\\
69.5	0.19914	28.0050271460746	28.0050271460746\\
69.5	0.2028	22.0302162544069	22.0302162544069\\
69.5	0.20646	15.4067750518661	15.4067750518661\\
69.5	0.21012	8.13470353845219	8.13470353845219\\
69.5	0.21378	0.21400171416667	0.21400171416667\\
69.5	0.21744	-8.35533042099246	-8.35533042099246\\
69.5	0.2211	-17.5732928670243	-17.5732928670243\\
69.5	0.22476	-27.4398856239288	-27.4398856239288\\
69.5	0.22842	-37.9551086917063	-37.9551086917063\\
69.5	0.23208	-49.1189620703562	-49.1189620703562\\
69.5	0.23574	-60.9314457598796	-60.9314457598796\\
69.5	0.2394	-73.3925597602752	-73.3925597602752\\
69.5	0.24306	-86.5023040715437	-86.5023040715437\\
69.5	0.24672	-100.260678693685	-100.260678693685\\
69.5	0.25038	-114.667683626699	-114.667683626699\\
69.5	0.25404	-129.723318870586	-129.723318870586\\
69.5	0.2577	-145.427584425346	-145.427584425346\\
69.5	0.26136	-161.780480290978	-161.780480290978\\
69.5	0.26502	-178.782006467483	-178.782006467483\\
69.5	0.26868	-196.432162954861	-196.432162954861\\
69.5	0.27234	-214.730949753112	-214.730949753112\\
69.5	0.276	-233.678366862235	-233.678366862235\\
69.875	0.093	-85.6331527812214	-85.6331527812214\\
69.875	0.09666	-72.5924913466809	-72.5924913466809\\
69.875	0.10032	-60.2004602230132	-60.2004602230132\\
69.875	0.10398	-48.457059410219	-48.457059410219\\
69.875	0.10764	-37.3622889082968	-37.3622889082968\\
69.875	0.1113	-26.9161487172474	-26.9161487172474\\
69.875	0.11496	-17.1186388370713	-17.1186388370713\\
69.875	0.11862	-7.96975926776787	-7.96975926776787\\
69.875	0.12228	0.530489990663227	0.530489990663227\\
69.875	0.12594	8.38210893822105	8.38210893822105\\
69.875	0.1296	15.5850975749062	15.5850975749062\\
69.875	0.13326	22.139455900719	22.139455900719\\
69.875	0.13692	28.0451839156585	28.0451839156585\\
69.875	0.14058	33.3022816197255	33.3022816197255\\
69.875	0.14424	37.9107490129194	37.9107490129194\\
69.875	0.1479	41.870586095241	41.870586095241\\
69.875	0.15156	45.1817928666894	45.1817928666894\\
69.875	0.15522	47.844369327265	47.844369327265\\
69.875	0.15888	49.8583154769681	49.8583154769681\\
69.875	0.16254	51.2236313157982	51.2236313157982\\
69.875	0.1662	51.9403168437555	51.9403168437555\\
69.875	0.16986	52.0083720608404	52.0083720608404\\
69.875	0.17352	51.4277969670526	51.4277969670526\\
69.875	0.17718	50.198591562391	50.198591562391\\
69.875	0.18084	48.3207558468578	48.3207558468578\\
69.875	0.1845	45.7942898204515	45.7942898204515\\
69.875	0.18816	42.6191934831724	42.6191934831724\\
69.875	0.19182	38.7954668350197	38.7954668350197\\
69.875	0.19548	34.3231098759952	34.3231098759952\\
69.875	0.19914	29.2021226060974	29.2021226060974\\
69.875	0.2028	23.4325050253271	23.4325050253271\\
69.875	0.20646	17.0142571336837	17.0142571336837\\
69.875	0.21012	9.94737893116758	9.94737893116758\\
69.875	0.21378	2.23187041777896	2.23187041777896\\
69.875	0.21744	-6.13226840648235	-6.13226840648235\\
69.875	0.2211	-15.1450375416164	-15.1450375416164\\
69.875	0.22476	-24.8064369876236	-24.8064369876236\\
69.875	0.22842	-35.1164667445037	-35.1164667445037\\
69.875	0.23208	-46.0751268122563	-46.0751268122563\\
69.875	0.23574	-57.6824171908818	-57.6824171908818\\
69.875	0.2394	-69.9383378803795	-69.9383378803795\\
69.875	0.24306	-82.8428888807507	-82.8428888807507\\
69.875	0.24672	-96.3960701919946	-96.3960701919946\\
69.875	0.25038	-110.597881814111	-110.597881814111\\
69.875	0.25404	-125.4483237471	-125.4483237471\\
69.875	0.2577	-140.947395990962	-140.947395990962\\
69.875	0.26136	-157.095098545698	-157.095098545698\\
69.875	0.26502	-173.891431411305	-173.891431411305\\
69.875	0.26868	-191.336394587785	-191.336394587785\\
69.875	0.27234	-209.429988075138	-209.429988075138\\
69.875	0.276	-228.172211873365	-228.172211873365\\
70.25	0.093	-90.5182107139264	-90.5182107139264\\
70.25	0.09666	-77.2723559684885	-77.2723559684885\\
70.25	0.10032	-64.6751315339234	-64.6751315339234\\
70.25	0.10398	-52.726537410231	-52.726537410231\\
70.25	0.10764	-41.4265735974119	-41.4265735974119\\
70.25	0.1113	-30.7752400954652	-30.7752400954652\\
70.25	0.11496	-20.7725369043912	-20.7725369043912\\
70.25	0.11862	-11.41846402419	-11.41846402419\\
70.25	0.12228	-2.71302145486152	-2.71302145486152\\
70.25	0.12594	5.34379080359366	5.34379080359366\\
70.25	0.1296	12.7519727511761	12.7519727511761\\
70.25	0.13326	19.5115243878868	19.5115243878868\\
70.25	0.13692	25.6224457137236	25.6224457137236\\
70.25	0.14058	31.084736728688	31.084736728688\\
70.25	0.14424	35.8983974327797	35.8983974327797\\
70.25	0.1479	40.0634278259987	40.0634278259987\\
70.25	0.15156	43.5798279083449	43.5798279083449\\
70.25	0.15522	46.4475976798178	46.4475976798178\\
70.25	0.15888	48.6667371404183	48.6667371404183\\
70.25	0.16254	50.2372462901462	50.2372462901462\\
70.25	0.1662	51.1591251290008	51.1591251290008\\
70.25	0.16986	51.4323736569831	51.4323736569831\\
70.25	0.17352	51.0569918740927	51.0569918740927\\
70.25	0.17718	50.0329797803289	50.0329797803289\\
70.25	0.18084	48.360337375693	48.360337375693\\
70.25	0.1845	46.0390646601841	46.0390646601841\\
70.25	0.18816	43.0691616338023	43.0691616338023\\
70.25	0.19182	39.450628296548	39.450628296548\\
70.25	0.19548	35.1834646484203	35.1834646484203\\
70.25	0.19914	30.2676706894204	30.2676706894204\\
70.25	0.2028	24.7032464195474	24.7032464195474\\
70.25	0.20646	18.4901918388018	18.4901918388018\\
70.25	0.21012	11.628506947183	11.628506947183\\
70.25	0.21378	4.11819174469224	4.11819174469224\\
70.25	0.21744	-4.04075376867218	-4.04075376867218\\
70.25	0.2211	-12.8483295929084	-12.8483295929084\\
70.25	0.22476	-22.3045357280182	-22.3045357280182\\
70.25	0.22842	-32.4093721740005	-32.4093721740005\\
70.25	0.23208	-43.1628389308553	-43.1628389308553\\
70.25	0.23574	-54.5649359985835	-54.5649359985835\\
70.25	0.2394	-66.6156633771843	-66.6156633771843\\
70.25	0.24306	-79.3150210666577	-79.3150210666577\\
70.25	0.24672	-92.6630090670042	-92.6630090670042\\
70.25	0.25038	-106.659627378223	-106.659627378223\\
70.25	0.25404	-121.304876000315	-121.304876000315\\
70.25	0.2577	-136.598754933279	-136.598754933279\\
70.25	0.26136	-152.541264177117	-152.541264177117\\
70.25	0.26502	-169.132403731827	-169.132403731827\\
70.25	0.26868	-186.37217359741	-186.37217359741\\
70.25	0.27234	-204.260573773865	-204.260573773865\\
70.25	0.276	-222.797604261194	-222.797604261194\\
70.625	0.093	-95.5348160233299	-95.5348160233299\\
70.625	0.09666	-82.0837679669947	-82.0837679669947\\
70.625	0.10032	-69.2813502215322	-69.2813502215322\\
70.625	0.10398	-57.1275627869425	-57.1275627869425\\
70.625	0.10764	-45.6224056632255	-45.6224056632255\\
70.625	0.1113	-34.7658788503815	-34.7658788503815\\
70.625	0.11496	-24.5579823484101	-24.5579823484101\\
70.625	0.11862	-14.9987161573115	-14.9987161573115\\
70.625	0.12228	-6.08808027708574	-6.08808027708574\\
70.625	0.12594	2.17392529226726	2.17392529226726\\
70.625	0.1296	9.78730055074755	9.78730055074755\\
70.625	0.13326	16.7520454983555	16.7520454983555\\
70.625	0.13692	23.0681601350897	23.0681601350897\\
70.625	0.14058	28.7356444609524	28.7356444609524\\
70.625	0.14424	33.754498475941	33.754498475941\\
70.625	0.1479	38.1247221800578	38.1247221800578\\
70.625	0.15156	41.8463155733014	41.8463155733014\\
70.625	0.15522	44.9192786556721	44.9192786556721\\
70.625	0.15888	47.3436114271699	47.3436114271699\\
70.625	0.16254	49.1193138877952	49.1193138877952\\
70.625	0.1662	50.2463860375476	50.2463860375476\\
70.625	0.16986	50.7248278764273	50.7248278764273\\
70.625	0.17352	50.5546394044342	50.5546394044342\\
70.625	0.17718	49.7358206215682	49.7358206215682\\
70.625	0.18084	48.2683715278297	48.2683715278297\\
70.625	0.1845	46.1522921232181	46.1522921232181\\
70.625	0.18816	43.3875824077342	43.3875824077342\\
70.625	0.19182	39.9742423813772	39.9742423813772\\
70.625	0.19548	35.9122720441474	35.9122720441474\\
70.625	0.19914	31.2016713960447	31.2016713960447\\
70.625	0.2028	25.8424404370692	25.8424404370692\\
70.625	0.20646	19.8345791672209	19.8345791672209\\
70.625	0.21012	13.1780875864999	13.1780875864999\\
70.625	0.21378	5.87296569490695	5.87296569490695\\
70.625	0.21744	-2.08078650755965	-2.08078650755965\\
70.625	0.2211	-10.683169020899	-10.683169020899\\
70.625	0.22476	-19.934181845111	-19.934181845111\\
70.625	0.22842	-29.8338249801959	-29.8338249801959\\
70.625	0.23208	-40.3820984261538	-40.3820984261538\\
70.625	0.23574	-51.5790021829841	-51.5790021829841\\
70.625	0.2394	-63.4245362506872	-63.4245362506872\\
70.625	0.24306	-75.9187006292632	-75.9187006292632\\
70.625	0.24672	-89.0614953187119	-89.0614953187119\\
70.625	0.25038	-102.852920319034	-102.852920319034\\
70.625	0.25404	-117.292975630228	-117.292975630228\\
70.625	0.2577	-132.381661252295	-132.381661252295\\
70.625	0.26136	-148.118977185235	-148.118977185235\\
70.625	0.26502	-164.504923429047	-164.504923429047\\
70.625	0.26868	-181.539499983732	-181.539499983732\\
70.625	0.27234	-199.22270684929	-199.22270684929\\
70.625	0.276	-217.554544025721	-217.554544025721\\
71	0.093	-100.682968709433	-100.682968709433\\
71	0.09666	-87.0267273421998	-87.0267273421998\\
71	0.10032	-74.0191162858396	-74.0191162858396\\
71	0.10398	-61.6601355403529	-61.6601355403529\\
71	0.10764	-49.9497851057382	-49.9497851057382\\
71	0.1113	-38.8880649819963	-38.8880649819963\\
71	0.11496	-28.4749751691272	-28.4749751691272\\
71	0.11862	-18.7105156671317	-18.7105156671317\\
71	0.12228	-9.59468647600806	-9.59468647600806\\
71	0.12594	-1.12748759575771	-1.12748759575771\\
71	0.1296	6.69108097361993	6.69108097361993\\
71	0.13326	13.8610192321253	13.8610192321253\\
71	0.13692	20.3823271797577	20.3823271797577\\
71	0.14058	26.2550048165173	26.2550048165173\\
71	0.14424	31.4790521424032	31.4790521424032\\
71	0.1479	36.0544691574174	36.0544691574174\\
71	0.15156	39.9812558615588	39.9812558615588\\
71	0.15522	43.2594122548269	43.2594122548269\\
71	0.15888	45.8889383372225	45.8889383372225\\
71	0.16254	47.8698341087452	47.8698341087452\\
71	0.1662	49.2020995693954	49.2020995693954\\
71	0.16986	49.885734719172	49.885734719172\\
71	0.17352	49.9207395580767	49.9207395580767\\
71	0.17718	49.3071140861085	49.3071140861085\\
71	0.18084	48.0448583032673	48.0448583032673\\
71	0.1845	46.1339722095531	46.1339722095531\\
71	0.18816	43.5744558049665	43.5744558049665\\
71	0.19182	40.3663090895074	40.3663090895074\\
71	0.19548	36.5095320631749	36.5095320631749\\
71	0.19914	32.0041247259696	32.0041247259696\\
71	0.2028	26.8500870778919	26.8500870778919\\
71	0.20646	21.0474191189414	21.0474191189414\\
71	0.21012	14.5961208491178	14.5961208491178\\
71	0.21378	7.49619226842174	7.49619226842174\\
71	0.21744	-0.252366623147054	-0.252366623147054\\
71	0.2211	-8.64955582558855	-8.64955582558855\\
71	0.22476	-17.6953753389032	-17.6953753389032\\
71	0.22842	-27.3898251630908	-27.3898251630908\\
71	0.23208	-37.7329052981509	-37.7329052981509\\
71	0.23574	-48.7246157440838	-48.7246157440838\\
71	0.2394	-60.3649565008891	-60.3649565008891\\
71	0.24306	-72.6539275685677	-72.6539275685677\\
71	0.24672	-85.5915289471191	-85.5915289471191\\
71	0.25038	-99.1777606365431	-99.1777606365431\\
71	0.25404	-113.41262263684	-113.41262263684\\
71	0.2577	-128.296114948009	-128.296114948009\\
71	0.26136	-143.828237570051	-143.828237570051\\
71	0.26502	-160.008990502967	-160.008990502967\\
71	0.26868	-176.838373746755	-176.838373746755\\
71	0.27234	-194.316387301415	-194.316387301415\\
71	0.276	-212.443031166948	-212.443031166948\\
71.375	0.093	-105.962668772235	-105.962668772235\\
71.375	0.09666	-92.101234094105	-92.101234094105\\
71.375	0.10032	-78.8884297268474	-78.8884297268474\\
71.375	0.10398	-66.3242556704629	-66.3242556704629\\
71.375	0.10764	-54.4087119249504	-54.4087119249504\\
71.375	0.1113	-43.1417984903115	-43.1417984903115\\
71.375	0.11496	-32.5235153665446	-32.5235153665446\\
71.375	0.11862	-22.5538625536513	-22.5538625536513\\
71.375	0.12228	-13.2328400516308	-13.2328400516308\\
71.375	0.12594	-4.56044786048261	-4.56044786048261\\
71.375	0.1296	3.46331401979285	3.46331401979285\\
71.375	0.13326	10.8384455891955	10.8384455891955\\
71.375	0.13692	17.5649468477254	17.5649468477254\\
71.375	0.14058	23.6428177953823	23.6428177953823\\
71.375	0.14424	29.0720584321665	29.0720584321665\\
71.375	0.1479	33.8526687580776	33.8526687580776\\
71.375	0.15156	37.9846487731163	37.9846487731163\\
71.375	0.15522	41.4679984772822	41.4679984772822\\
71.375	0.15888	44.3027178705752	44.3027178705752\\
71.375	0.16254	46.4888069529957	46.4888069529957\\
71.375	0.1662	48.0262657245428	48.0262657245428\\
71.375	0.16986	48.9150941852176	48.9150941852176\\
71.375	0.17352	49.1552923350197	49.1552923350197\\
71.375	0.17718	48.7468601739484	48.7468601739484\\
71.375	0.18084	47.6897977020051	47.6897977020051\\
71.375	0.1845	45.9841049191887	45.9841049191887\\
71.375	0.18816	43.6297818254999	43.6297818254999\\
71.375	0.19182	40.6268284209376	40.6268284209376\\
71.375	0.19548	36.975244705503	36.975244705503\\
71.375	0.19914	32.6750306791951	32.6750306791951\\
71.375	0.2028	27.7261863420151	27.7261863420151\\
71.375	0.20646	22.1287116939616	22.1287116939616\\
71.375	0.21012	15.8826067350358	15.8826067350358\\
71.375	0.21378	8.98787146523705	8.98787146523705\\
71.375	0.21744	1.44450588456607	1.44450588456607\\
71.375	0.2211	-6.74749000697852	-6.74749000697852\\
71.375	0.22476	-15.5881162093954	-15.5881162093954\\
71.375	0.22842	-25.0773727226851	-25.0773727226851\\
71.375	0.23208	-35.2152595468478	-35.2152595468478\\
71.375	0.23574	-46.001776681883	-46.001776681883\\
71.375	0.2394	-57.4369241277914	-57.4369241277914\\
71.375	0.24306	-69.5207018845722	-69.5207018845722\\
71.375	0.24672	-82.2531099522262	-82.2531099522262\\
71.375	0.25038	-95.6341483307524	-95.6341483307524\\
71.375	0.25404	-109.663817020151	-109.663817020151\\
71.375	0.2577	-124.342116020424	-124.342116020424\\
71.375	0.26136	-139.669045331568	-139.669045331568\\
71.375	0.26502	-155.644604953586	-155.644604953586\\
71.375	0.26868	-172.268794886476	-172.268794886476\\
71.375	0.27234	-189.54161513024	-189.54161513024\\
71.375	0.276	-207.463065684875	-207.463065684875\\
71.75	0.093	-111.373916211736	-111.373916211736\\
71.75	0.09666	-97.3072882227082	-97.3072882227082\\
71.75	0.10032	-83.8892905445532	-83.8892905445532\\
71.75	0.10398	-71.1199231772709	-71.1199231772709\\
71.75	0.10764	-58.9991861208615	-58.9991861208615\\
71.75	0.1113	-47.5270793753244	-47.5270793753244\\
71.75	0.11496	-36.7036029406606	-36.7036029406606\\
71.75	0.11862	-26.5287568168695	-26.5287568168695\\
71.75	0.12228	-17.0025410039511	-17.0025410039511\\
71.75	0.12594	-8.12495550190562	-8.12495550190562\\
71.75	0.1296	0.103999689267198	0.103999689267198\\
71.75	0.13326	7.68432456956725	7.68432456956725\\
71.75	0.13692	14.6160191389944	14.6160191389944\\
71.75	0.14058	20.8990833975491	20.8990833975491\\
71.75	0.14424	26.5335173452303	26.5335173452303\\
71.75	0.1479	31.5193209820396	31.5193209820396\\
71.75	0.15156	35.8564943079757	35.8564943079757\\
71.75	0.15522	39.545037323039	39.545037323039\\
71.75	0.15888	42.5849500272298	42.5849500272298\\
71.75	0.16254	44.9762324205476	44.9762324205476\\
71.75	0.1662	46.7188845029925	46.7188845029925\\
71.75	0.16986	47.8129062745647	47.8129062745647\\
71.75	0.17352	48.2582977352641	48.2582977352641\\
71.75	0.17718	48.0550588850907	48.0550588850907\\
71.75	0.18084	47.2031897240447	47.2031897240447\\
71.75	0.1845	45.7026902521256	45.7026902521256\\
71.75	0.18816	43.5535604693342	43.5535604693342\\
71.75	0.19182	40.7558003756698	40.7558003756698\\
71.75	0.19548	37.3094099711325	37.3094099711325\\
71.75	0.19914	33.2143892557224	33.2143892557224\\
71.75	0.2028	28.4707382294398	28.4707382294398\\
71.75	0.20646	23.0784568922841	23.0784568922841\\
71.75	0.21012	17.0375452442556	17.0375452442556\\
71.75	0.21378	10.3480032853547	10.3480032853547\\
71.75	0.21744	3.00983101558063	3.00983101558063\\
71.75	0.2211	-4.97697156506615	-4.97697156506615\\
71.75	0.22476	-13.6124044565856	-13.6124044565856\\
71.75	0.22842	-22.8964676589781	-22.8964676589781\\
71.75	0.23208	-32.8291611722429	-32.8291611722429\\
71.75	0.23574	-43.4104849963808	-43.4104849963808\\
71.75	0.2394	-54.6404391313913	-54.6404391313913\\
71.75	0.24306	-66.5190235772748	-66.5190235772748\\
71.75	0.24672	-79.0462383340309	-79.0462383340309\\
71.75	0.25038	-92.2220834016603	-92.2220834016603\\
71.75	0.25404	-106.046558780161	-106.046558780161\\
71.75	0.2577	-120.519664469536	-120.519664469536\\
71.75	0.26136	-135.641400469784	-135.641400469784\\
71.75	0.26502	-151.411766780903	-151.411766780903\\
71.75	0.26868	-167.830763402896	-167.830763402896\\
71.75	0.27234	-184.898390335762	-184.898390335762\\
71.75	0.276	-202.6146475795	-202.6146475795\\
72.125	0.093	-116.916711027937	-116.916711027937\\
72.125	0.09666	-102.644889728012	-102.644889728012\\
72.125	0.10032	-89.021698738959	-89.021698738959\\
72.125	0.10398	-76.0471380607789	-76.0471380607789\\
72.125	0.10764	-63.7212076934721	-63.7212076934721\\
72.125	0.1113	-52.0439076370382	-52.0439076370382\\
72.125	0.11496	-41.0152378914765	-41.0152378914765\\
72.125	0.11862	-30.6351984567876	-30.6351984567876\\
72.125	0.12228	-20.9037893329719	-20.9037893329719\\
72.125	0.12594	-11.821010520029	-11.821010520029\\
72.125	0.1296	-3.38686201795838	-3.38686201795838\\
72.125	0.13326	4.39865617323949	4.39865617323949\\
72.125	0.13692	11.535544053564	11.535544053564\\
72.125	0.14058	18.0238016230161	18.0238016230161\\
72.125	0.14424	23.863428881595	23.863428881595\\
72.125	0.1479	29.0544258293017	29.0544258293017\\
72.125	0.15156	33.5967924661356	33.5967924661356\\
72.125	0.15522	37.4905287920958	37.4905287920958\\
72.125	0.15888	40.7356348071839	40.7356348071839\\
72.125	0.16254	43.3321105113996	43.3321105113996\\
72.125	0.1662	45.2799559047419	45.2799559047419\\
72.125	0.16986	46.5791709872115	46.5791709872115\\
72.125	0.17352	47.2297557588087	47.2297557588087\\
72.125	0.17718	47.2317102195331	47.2317102195331\\
72.125	0.18084	46.5850343693844	46.5850343693844\\
72.125	0.1845	45.2897282083632	45.2897282083632\\
72.125	0.18816	43.3457917364691	43.3457917364691\\
72.125	0.19182	40.753224953702	40.753224953702\\
72.125	0.19548	37.5120278600621	37.5120278600621\\
72.125	0.19914	33.6222004555498	33.6222004555498\\
72.125	0.2028	29.0837427401646	29.0837427401646\\
72.125	0.20646	23.8966547139062	23.8966547139062\\
72.125	0.21012	18.0609363767751	18.0609363767751\\
72.125	0.21378	11.5765877287724	11.5765877287724\\
72.125	0.21744	4.44360876989526	4.44360876989526\\
72.125	0.2211	-3.33800049985371	-3.33800049985371\\
72.125	0.22476	-11.7682400804754	-11.7682400804754\\
72.125	0.22842	-20.8471099719704	-20.8471099719704\\
72.125	0.23208	-30.574610174338	-30.574610174338\\
72.125	0.23574	-40.9507406875784	-40.9507406875784\\
72.125	0.2394	-51.9755015116912	-51.9755015116912\\
72.125	0.24306	-63.6488926466773	-63.6488926466773\\
72.125	0.24672	-75.9709140925361	-75.9709140925361\\
72.125	0.25038	-88.9415658492676	-88.9415658492676\\
72.125	0.25404	-102.560847916871	-102.560847916871\\
72.125	0.2577	-116.828760295348	-116.828760295348\\
72.125	0.26136	-131.745302984698	-131.745302984698\\
72.125	0.26502	-147.310475984921	-147.310475984921\\
72.125	0.26868	-163.524279296016	-163.524279296016\\
72.125	0.27234	-180.386712917984	-180.386712917984\\
72.125	0.276	-197.897776850825	-197.897776850825\\
72.5	0.093	-122.591053220836	-122.591053220836\\
72.5	0.09666	-108.114038610014	-108.114038610014\\
72.5	0.10032	-94.2856543100629	-94.2856543100629\\
72.5	0.10398	-81.1059003209859	-81.1059003209859\\
72.5	0.10764	-68.5747766427818	-68.5747766427818\\
72.5	0.1113	-56.6922832754495	-56.6922832754495\\
72.5	0.11496	-45.4584202189905	-45.4584202189905\\
72.5	0.11862	-34.8731874734042	-34.8731874734042\\
72.5	0.12228	-24.9365850386912	-24.9365850386912\\
72.5	0.12594	-15.6486129148505	-15.6486129148505\\
72.5	0.1296	-7.00927110188252	-7.00927110188252\\
72.5	0.13326	0.981440400212705	0.981440400212705\\
72.5	0.13692	8.32352159143505	8.32352159143505\\
72.5	0.14058	15.0169724717845	15.0169724717845\\
72.5	0.14424	21.0617930412612	21.0617930412612\\
72.5	0.1479	26.4579832998648	26.4579832998648\\
72.5	0.15156	31.2055432475966	31.2055432475966\\
72.5	0.15522	35.3044728844545	35.3044728844545\\
72.5	0.15888	38.7547722104405	38.7547722104405\\
72.5	0.16254	41.556441225553	41.556441225553\\
72.5	0.1662	43.7094799297927	43.7094799297927\\
72.5	0.16986	45.2138883231605	45.2138883231605\\
72.5	0.17352	46.0696664056546	46.0696664056546\\
72.5	0.17718	46.2768141772764	46.2768141772764\\
72.5	0.18084	45.8353316380255	45.8353316380255\\
72.5	0.1845	44.7452187879016	44.7452187879016\\
72.5	0.18816	43.0064756269049	43.0064756269049\\
72.5	0.19182	40.6191021550356	40.6191021550356\\
72.5	0.19548	37.5830983722935	37.5830983722935\\
72.5	0.19914	33.8984642786781	33.8984642786781\\
72.5	0.2028	29.5651998741907	29.5651998741907\\
72.5	0.20646	24.5833051588297	24.5833051588297\\
72.5	0.21012	18.9527801325964	18.9527801325964\\
72.5	0.21378	12.6736247954907	12.6736247954907\\
72.5	0.21744	5.74583914751224	5.74583914751224\\
72.5	0.2211	-1.83057681133982	-1.83057681133982\\
72.5	0.22476	-10.0556230810641	-10.0556230810641\\
72.5	0.22842	-18.9292996616614	-18.9292996616614\\
72.5	0.23208	-28.4516065531316	-28.4516065531316\\
72.5	0.23574	-38.6225437554742	-38.6225437554742\\
72.5	0.2394	-49.44211126869	-49.44211126869\\
72.5	0.24306	-60.9103090927783	-60.9103090927783\\
72.5	0.24672	-73.0271372277393	-73.0271372277393\\
72.5	0.25038	-85.7925956735735	-85.7925956735735\\
72.5	0.25404	-99.2066844302799	-99.2066844302799\\
72.5	0.2577	-113.26940349786	-113.26940349786\\
72.5	0.26136	-127.980752876312	-127.980752876312\\
72.5	0.26502	-143.340732565636	-143.340732565636\\
72.5	0.26868	-159.349342565835	-159.349342565835\\
72.5	0.27234	-176.006582876905	-176.006582876905\\
72.5	0.276	-193.312453498848	-193.312453498848\\
72.875	0.093	-128.396942790435	-128.396942790435\\
72.875	0.09666	-113.714734868715	-113.714734868715\\
72.875	0.10032	-99.6811572578672	-99.6811572578672\\
72.875	0.10398	-86.2962099578924	-86.2962099578924\\
72.875	0.10764	-73.5598929687905	-73.5598929687905\\
72.875	0.1113	-61.4722062905613	-61.4722062905613\\
72.875	0.11496	-50.0331499232045	-50.0331499232045\\
72.875	0.11862	-39.2427238667208	-39.2427238667208\\
72.875	0.12228	-29.10092812111	-29.10092812111\\
72.875	0.12594	-19.6077626863719	-19.6077626863719\\
72.875	0.1296	-10.7632275625066	-10.7632275625066\\
72.875	0.13326	-2.56732274951355	-2.56732274951355\\
72.875	0.13692	4.97995175260615	4.97995175260615\\
72.875	0.14058	11.8785959438529	11.8785959438529\\
72.875	0.14424	18.1286098242271	18.1286098242271\\
72.875	0.1479	23.7299933937289	23.7299933937289\\
72.875	0.15156	28.6827466523575	28.6827466523575\\
72.875	0.15522	32.9868696001133	32.9868696001133\\
72.875	0.15888	36.6423622369966	36.6423622369966\\
72.875	0.16254	39.649224563007	39.649224563007\\
72.875	0.1662	42.0074565781445	42.0074565781445\\
72.875	0.16986	43.7170582824092	43.7170582824092\\
72.875	0.17352	44.7780296758011	44.7780296758011\\
72.875	0.17718	45.1903707583202	45.1903707583202\\
72.875	0.18084	44.9540815299667	44.9540815299667\\
72.875	0.1845	44.0691619907402	44.0691619907402\\
72.875	0.18816	42.5356121406413	42.5356121406413\\
72.875	0.19182	40.3534319796694	40.3534319796694\\
72.875	0.19548	37.5226215078246	37.5226215078246\\
72.875	0.19914	34.043180725107	34.043180725107\\
72.875	0.2028	29.915109631517	29.915109631517\\
72.875	0.20646	25.1384082270538	25.1384082270538\\
72.875	0.21012	19.7130765117179	19.7130765117179\\
72.875	0.21378	13.6391144855095	13.6391144855095\\
72.875	0.21744	6.91652214842793	6.91652214842793\\
72.875	0.2211	-0.454700499526325	-0.454700499526325\\
72.875	0.22476	-8.47455345835283	-8.47455345835283\\
72.875	0.22842	-17.1430367280527	-17.1430367280527\\
72.875	0.23208	-26.4601503086251	-26.4601503086251\\
72.875	0.23574	-36.4258942000704	-36.4258942000704\\
72.875	0.2394	-47.0402684023884	-47.0402684023884\\
72.875	0.24306	-58.3032729155793	-58.3032729155793\\
72.875	0.24672	-70.214907739643	-70.214907739643\\
72.875	0.25038	-82.7751728745793	-82.7751728745793\\
72.875	0.25404	-95.9840683203879	-95.9840683203879\\
72.875	0.2577	-109.84159407707	-109.84159407707\\
72.875	0.26136	-124.347750144625	-124.347750144625\\
72.875	0.26502	-139.502536523052	-139.502536523052\\
72.875	0.26868	-155.305953212353	-155.305953212353\\
72.875	0.27234	-171.758000212526	-171.758000212526\\
72.875	0.276	-188.858677523572	-188.858677523572\\
73.25	0.093	-134.334379736733	-134.334379736733\\
73.25	0.09666	-119.446978504115	-119.446978504115\\
73.25	0.10032	-105.208207582371	-105.208207582371\\
73.25	0.10398	-91.6180669714984	-91.6180669714984\\
73.25	0.10764	-78.6765566714986	-78.6765566714986\\
73.25	0.1113	-66.3836766823716	-66.3836766823716\\
73.25	0.11496	-54.7394270041174	-54.7394270041174\\
73.25	0.11862	-43.7438076367364	-43.7438076367364\\
73.25	0.12228	-33.3968185802282	-33.3968185802282\\
73.25	0.12594	-23.6984598345924	-23.6984598345924\\
73.25	0.1296	-14.6487313998297	-14.6487313998297\\
73.25	0.13326	-6.24763327593928	-6.24763327593928\\
73.25	0.13692	1.50483453707778	1.50483453707778\\
73.25	0.14058	8.60867203922237	8.60867203922237\\
73.25	0.14424	15.0638792304943	15.0638792304943\\
73.25	0.1479	20.8704561108935	20.8704561108935\\
73.25	0.15156	26.0284026804195	26.0284026804195\\
73.25	0.15522	30.5377189390726	30.5377189390726\\
73.25	0.15888	34.3984048868533	34.3984048868533\\
73.25	0.16254	37.6104605237615	37.6104605237615\\
73.25	0.1662	40.1738858497963	40.1738858497963\\
73.25	0.16986	42.0886808649584	42.0886808649584\\
73.25	0.17352	43.3548455692481	43.3548455692481\\
73.25	0.17718	43.9723799626651	43.9723799626651\\
73.25	0.18084	43.9412840452089	43.9412840452089\\
73.25	0.1845	43.2615578168802	43.2615578168802\\
73.25	0.18816	41.9332012776787	41.9332012776787\\
73.25	0.19182	39.9562144276041	39.9562144276041\\
73.25	0.19548	37.3305972666572	37.3305972666572\\
73.25	0.19914	34.0563497948369	34.0563497948369\\
73.25	0.2028	30.1334720121442	30.1334720121442\\
73.25	0.20646	25.5619639185788	25.5619639185788\\
73.25	0.21012	20.3418255141403	20.3418255141403\\
73.25	0.21378	14.4730567988297	14.4730567988297\\
73.25	0.21744	7.95565777264596	7.95565777264596\\
73.25	0.2211	0.789628435588611	0.789628435588611\\
73.25	0.22476	-7.02503121234054	-7.02503121234054\\
73.25	0.22842	-15.4883211711426	-15.4883211711426\\
73.25	0.23208	-24.6002414408176	-24.6002414408176\\
73.25	0.23574	-34.3607920213656	-34.3607920213656\\
73.25	0.2394	-44.7699729127858	-44.7699729127858\\
73.25	0.24306	-55.8277841150793	-55.8277841150793\\
73.25	0.24672	-67.5342256282456	-67.5342256282456\\
73.25	0.25038	-79.8892974522846	-79.8892974522846\\
73.25	0.25404	-92.8929995871954	-92.8929995871954\\
73.25	0.2577	-106.545332032981	-106.545332032981\\
73.25	0.26136	-120.846294789638	-120.846294789638\\
73.25	0.26502	-135.795887857167	-135.795887857167\\
73.25	0.26868	-151.39411123557	-151.39411123557\\
73.25	0.27234	-167.640964924845	-167.640964924845\\
73.25	0.276	-184.536448924994	-184.536448924994\\
73.625	0.093	-140.403364059731	-140.403364059731\\
73.625	0.09666	-125.310769516215	-125.310769516215\\
73.625	0.10032	-110.866805283572	-110.866805283572\\
73.625	0.10398	-97.0714713618024	-97.0714713618024\\
73.625	0.10764	-83.9247677509053	-83.9247677509053\\
73.625	0.1113	-71.426694450881	-71.426694450881\\
73.625	0.11496	-59.5772514617294	-59.5772514617294\\
73.625	0.11862	-48.3764387834506	-48.3764387834506\\
73.625	0.12228	-37.8242564160446	-37.8242564160446\\
73.625	0.12594	-27.9207043595118	-27.9207043595118\\
73.625	0.1296	-18.6657826138513	-18.6657826138513\\
73.625	0.13326	-10.0594911790636	-10.0594911790636\\
73.625	0.13692	-2.1018300551487	-2.1018300551487\\
73.625	0.14058	5.20720075789325	5.20720075789325\\
73.625	0.14424	11.8676012600625	11.8676012600625\\
73.625	0.1479	17.8793714513591	17.8793714513591\\
73.625	0.15156	23.2425113317829	23.2425113317829\\
73.625	0.15522	27.9570209013339	27.9570209013339\\
73.625	0.15888	32.0229001600119	32.0229001600119\\
73.625	0.16254	35.4401491078174	35.4401491078174\\
73.625	0.1662	38.2087677447496	38.2087677447496\\
73.625	0.16986	40.32875607081	40.32875607081\\
73.625	0.17352	41.8001140859966	41.8001140859966\\
73.625	0.17718	42.6228417903113	42.6228417903113\\
73.625	0.18084	42.7969391837526	42.7969391837526\\
73.625	0.1845	42.3224062663217	42.3224062663217\\
73.625	0.18816	41.1992430380175	41.1992430380175\\
73.625	0.19182	39.4274494988402	39.4274494988402\\
73.625	0.19548	37.0070256487907	37.0070256487907\\
73.625	0.19914	33.9379714878683	33.9379714878683\\
73.625	0.2028	30.2202870160729	30.2202870160729\\
73.625	0.20646	25.8539722334049	25.8539722334049\\
73.625	0.21012	20.8390271398641	20.8390271398641\\
73.625	0.21378	15.1754517354505	15.1754517354505\\
73.625	0.21744	8.86324602016452	8.86324602016452\\
73.625	0.2211	1.90240999400498	1.90240999400498\\
73.625	0.22476	-5.70705634302635	-5.70705634302635\\
73.625	0.22842	-13.9651529909315	-13.9651529909315\\
73.625	0.23208	-22.8718799497087	-22.8718799497087\\
73.625	0.23574	-32.4272372193593	-32.4272372193593\\
73.625	0.2394	-42.6312247998821	-42.6312247998821\\
73.625	0.24306	-53.4838426912779	-53.4838426912779\\
73.625	0.24672	-64.9850908935464	-64.9850908935464\\
73.625	0.25038	-77.134969406688	-77.134969406688\\
73.625	0.25404	-89.9334782307014	-89.9334782307014\\
73.625	0.2577	-103.380617365589	-103.380617365589\\
73.625	0.26136	-117.476386811349	-117.476386811349\\
73.625	0.26502	-132.220786567981	-132.220786567981\\
73.625	0.26868	-147.613816635486	-147.613816635486\\
73.625	0.27234	-163.655477013865	-163.655477013865\\
73.625	0.276	-180.345767703115	-180.345767703115\\
74	0.093	-146.603895759427	-146.603895759427\\
74	0.09666	-131.306107905014	-131.306107905014\\
74	0.10032	-116.656950361474	-116.656950361474\\
74	0.10398	-102.656423128806	-102.656423128806\\
74	0.10764	-89.3045262070119	-89.3045262070119\\
74	0.1113	-76.6012595960898	-76.6012595960898\\
74	0.11496	-64.5466232960409	-64.5466232960409\\
74	0.11862	-53.1406173068647	-53.1406173068647\\
74	0.12228	-42.3832416285613	-42.3832416285613\\
74	0.12594	-32.2744962611308	-32.2744962611308\\
74	0.1296	-22.8143812045729	-22.8143812045729\\
74	0.13326	-14.0028964588873	-14.0028964588873\\
74	0.13692	-5.8400420240751	-5.8400420240751\\
74	0.14058	1.67418209986465	1.67418209986465\\
74	0.14424	8.5397759129313	8.5397759129313\\
74	0.1479	14.7567394151253	14.7567394151253\\
74	0.15156	20.3250726064464	20.3250726064464\\
74	0.15522	25.2447754868947	25.2447754868947\\
74	0.15888	29.515848056471	29.515848056471\\
74	0.16254	33.1382903151739	33.1382903151739\\
74	0.1662	36.1121022630039	36.1121022630039\\
74	0.16986	38.4372838999611	38.4372838999611\\
74	0.17352	40.1138352260456	40.1138352260456\\
74	0.17718	41.1417562412572	41.1417562412572\\
74	0.18084	41.5210469455963	41.5210469455963\\
74	0.1845	41.2517073390627	41.2517073390627\\
74	0.18816	40.3337374216563	40.3337374216563\\
74	0.19182	38.7671371933769	38.7671371933769\\
74	0.19548	36.5519066542247	36.5519066542247\\
74	0.19914	33.6880458041996	33.6880458041996\\
74	0.2028	30.1755546433021	30.1755546433021\\
74	0.20646	26.0144331715314	26.0144331715314\\
74	0.21012	21.204681388888	21.204681388888\\
74	0.21378	15.7462992953726	15.7462992953726\\
74	0.21744	9.63928689098361	9.63928689098361\\
74	0.2211	2.88364417572188	2.88364417572188\\
74	0.22476	-4.52062885041255	-4.52062885041255\\
74	0.22842	-12.5735321874199	-12.5735321874199\\
74	0.23208	-21.2750658352998	-21.2750658352998\\
74	0.23574	-30.6252297940525	-30.6252297940525\\
74	0.2394	-40.624024063678	-40.624024063678\\
74	0.24306	-51.2714486441764	-51.2714486441764\\
74	0.24672	-62.5675035355475	-62.5675035355475\\
74	0.25038	-74.5121887377913	-74.5121887377913\\
74	0.25404	-87.1055042509079	-87.1055042509079\\
74	0.2577	-100.347450074898	-100.347450074898\\
74	0.26136	-114.23802620976	-114.23802620976\\
74	0.26502	-128.777232655494	-128.777232655494\\
74	0.26868	-143.965069412103	-143.965069412103\\
74	0.27234	-159.801536479582	-159.801536479582\\
74	0.276	-176.286633857937	-176.286633857937\\
};
\end{axis}

\begin{axis}[%
width=2.616cm,
height=2.517cm,
at={(0cm,10.49cm)},
scale only axis,
xmin=56,
xmax=74,
tick align=outside,
xlabel style={font=\color{white!15!black}},
xlabel={$L_{cut}$},
ymin=0.093,
ymax=0.276,
ylabel style={font=\color{white!15!black}},
ylabel={$D_{rlx}$},
zmin=-5.1976538076126,
zmax=37.7080580786817,
zlabel style={font=\color{white!15!black}},
zlabel={$x_3,x_4$},
view={-140}{50},
axis background/.style={fill=white},
xmajorgrids,
ymajorgrids,
zmajorgrids,
legend style={at={(1.03,1)}, anchor=north west, legend cell align=left, align=left, draw=white!15!black}
]
\addplot3[only marks, mark=*, mark options={}, mark size=1.5000pt, color=mycolor1, fill=mycolor1] table[row sep=crcr]{%
x	y	z\\
74	0.123	2.27881970207594\\
72	0.113	2.51722855506521\\
61	0.095	0.101197735723776\\
56	0.093	-0.10896595373643\\
};
\addlegendentry{data1}

\addplot3[only marks, mark=*, mark options={}, mark size=1.5000pt, color=mycolor2, fill=mycolor2] table[row sep=crcr]{%
x	y	z\\
67	0.276	20.093048450268\\
66	0.255	10.8938397383643\\
62	0.209	3.44322705304763\\
57	0.193	3.64195653380341\\
};
\addlegendentry{data2}

\addplot3[only marks, mark=*, mark options={}, mark size=1.5000pt, color=black, fill=black] table[row sep=crcr]{%
x	y	z\\
69	0.104	2.39904989364183\\
};
\addlegendentry{data3}

\addplot3[only marks, mark=*, mark options={}, mark size=1.5000pt, color=black, fill=black] table[row sep=crcr]{%
x	y	z\\
64	0.23	6.21650928315715\\
};
\addlegendentry{data4}


\addplot3[%
surf,
fill opacity=0.7, shader=interp, colormap={mymap}{[1pt] rgb(0pt)=(1,0.905882,0); rgb(1pt)=(1,0.901964,0); rgb(2pt)=(1,0.898051,0); rgb(3pt)=(1,0.894144,0); rgb(4pt)=(1,0.890243,0); rgb(5pt)=(1,0.886349,0); rgb(6pt)=(1,0.88246,0); rgb(7pt)=(1,0.878577,0); rgb(8pt)=(1,0.8747,0); rgb(9pt)=(1,0.870829,0); rgb(10pt)=(1,0.866964,0); rgb(11pt)=(1,0.863106,0); rgb(12pt)=(1,0.859253,0); rgb(13pt)=(1,0.855406,0); rgb(14pt)=(1,0.851566,0); rgb(15pt)=(1,0.847732,0); rgb(16pt)=(1,0.843903,0); rgb(17pt)=(1,0.840081,0); rgb(18pt)=(1,0.836265,0); rgb(19pt)=(1,0.832455,0); rgb(20pt)=(1,0.828652,0); rgb(21pt)=(1,0.824854,0); rgb(22pt)=(1,0.821063,0); rgb(23pt)=(1,0.817278,0); rgb(24pt)=(1,0.8135,0); rgb(25pt)=(1,0.809727,0); rgb(26pt)=(1,0.805961,0); rgb(27pt)=(1,0.8022,0); rgb(28pt)=(1,0.798445,0); rgb(29pt)=(1,0.794696,0); rgb(30pt)=(1,0.790953,0); rgb(31pt)=(1,0.787215,0); rgb(32pt)=(1,0.783484,0); rgb(33pt)=(1,0.779758,0); rgb(34pt)=(1,0.776038,0); rgb(35pt)=(1,0.772324,0); rgb(36pt)=(1,0.768615,0); rgb(37pt)=(1,0.764913,0); rgb(38pt)=(1,0.761217,0); rgb(39pt)=(1,0.757527,0); rgb(40pt)=(1,0.753843,0); rgb(41pt)=(1,0.750165,0); rgb(42pt)=(1,0.746493,0); rgb(43pt)=(1,0.742827,0); rgb(44pt)=(1,0.739167,0); rgb(45pt)=(1,0.735514,0); rgb(46pt)=(1,0.731867,0); rgb(47pt)=(1,0.728226,0); rgb(48pt)=(1,0.724591,0); rgb(49pt)=(1,0.720963,0); rgb(50pt)=(1,0.717341,0); rgb(51pt)=(1,0.713725,0); rgb(52pt)=(0.999994,0.710077,0); rgb(53pt)=(0.999974,0.706363,0); rgb(54pt)=(0.999942,0.702592,0); rgb(55pt)=(0.999898,0.698775,0); rgb(56pt)=(0.999841,0.694921,0); rgb(57pt)=(0.999771,0.691039,0); rgb(58pt)=(0.99969,0.687139,0); rgb(59pt)=(0.999596,0.68323,0); rgb(60pt)=(0.99949,0.679323,0); rgb(61pt)=(0.999372,0.675427,0); rgb(62pt)=(0.999242,0.67155,0); rgb(63pt)=(0.9991,0.667704,0); rgb(64pt)=(0.998946,0.663897,0); rgb(65pt)=(0.998781,0.660138,0); rgb(66pt)=(0.998605,0.656439,0); rgb(67pt)=(0.998416,0.652807,0); rgb(68pt)=(0.998217,0.649253,0); rgb(69pt)=(0.998006,0.645786,0); rgb(70pt)=(0.997785,0.642416,0); rgb(71pt)=(0.997552,0.639152,0); rgb(72pt)=(0.997308,0.636004,0); rgb(73pt)=(0.997053,0.632982,0); rgb(74pt)=(0.996788,0.630095,0); rgb(75pt)=(0.996512,0.627352,0); rgb(76pt)=(0.996226,0.624763,0); rgb(77pt)=(0.995851,0.622329,0); rgb(78pt)=(0.99494,0.619997,0); rgb(79pt)=(0.99345,0.617753,0); rgb(80pt)=(0.991419,0.61559,0); rgb(81pt)=(0.988885,0.613503,0); rgb(82pt)=(0.985886,0.611486,0); rgb(83pt)=(0.98246,0.609532,0); rgb(84pt)=(0.978643,0.607636,0); rgb(85pt)=(0.974475,0.605791,0); rgb(86pt)=(0.969992,0.603992,0); rgb(87pt)=(0.965232,0.602233,0); rgb(88pt)=(0.960233,0.600507,0); rgb(89pt)=(0.955033,0.598808,0); rgb(90pt)=(0.949669,0.59713,0); rgb(91pt)=(0.94418,0.595468,0); rgb(92pt)=(0.938602,0.593815,0); rgb(93pt)=(0.932974,0.592166,0); rgb(94pt)=(0.927333,0.590513,0); rgb(95pt)=(0.921717,0.588852,0); rgb(96pt)=(0.916164,0.587176,0); rgb(97pt)=(0.910711,0.585479,0); rgb(98pt)=(0.905397,0.583755,0); rgb(99pt)=(0.900258,0.581999,0); rgb(100pt)=(0.895333,0.580203,0); rgb(101pt)=(0.890659,0.578362,0); rgb(102pt)=(0.886275,0.576471,0); rgb(103pt)=(0.882047,0.574545,0); rgb(104pt)=(0.877819,0.572608,0); rgb(105pt)=(0.873592,0.57066,0); rgb(106pt)=(0.869366,0.568701,0); rgb(107pt)=(0.865143,0.566733,0); rgb(108pt)=(0.860924,0.564756,0); rgb(109pt)=(0.856708,0.562771,0); rgb(110pt)=(0.852497,0.560778,0); rgb(111pt)=(0.848292,0.558779,0); rgb(112pt)=(0.844092,0.556774,0); rgb(113pt)=(0.8399,0.554763,0); rgb(114pt)=(0.835716,0.552749,0); rgb(115pt)=(0.831541,0.55073,0); rgb(116pt)=(0.827374,0.548709,0); rgb(117pt)=(0.823219,0.546686,0); rgb(118pt)=(0.819074,0.54466,0); rgb(119pt)=(0.81494,0.542635,0); rgb(120pt)=(0.81082,0.540609,0); rgb(121pt)=(0.806712,0.538584,0); rgb(122pt)=(0.802619,0.53656,0); rgb(123pt)=(0.798541,0.534539,0); rgb(124pt)=(0.794478,0.532521,0); rgb(125pt)=(0.790431,0.530506,0); rgb(126pt)=(0.786402,0.528496,0); rgb(127pt)=(0.782391,0.526491,0); rgb(128pt)=(0.77841,0.524489,0); rgb(129pt)=(0.774523,0.522478,0); rgb(130pt)=(0.770731,0.520455,0); rgb(131pt)=(0.767022,0.518424,0); rgb(132pt)=(0.763384,0.516385,0); rgb(133pt)=(0.759804,0.514339,0); rgb(134pt)=(0.756272,0.51229,0); rgb(135pt)=(0.752775,0.510237,0); rgb(136pt)=(0.749302,0.508182,0); rgb(137pt)=(0.74584,0.506128,0); rgb(138pt)=(0.742378,0.504075,0); rgb(139pt)=(0.738904,0.502025,0); rgb(140pt)=(0.735406,0.499979,0); rgb(141pt)=(0.731872,0.49794,0); rgb(142pt)=(0.72829,0.495909,0); rgb(143pt)=(0.724649,0.493887,0); rgb(144pt)=(0.720936,0.491875,0); rgb(145pt)=(0.71714,0.489876,0); rgb(146pt)=(0.713249,0.487891,0); rgb(147pt)=(0.709251,0.485921,0); rgb(148pt)=(0.705134,0.483968,0); rgb(149pt)=(0.700887,0.482033,0); rgb(150pt)=(0.696497,0.480118,0); rgb(151pt)=(0.691952,0.478225,0); rgb(152pt)=(0.687242,0.476355,0); rgb(153pt)=(0.682353,0.47451,0); rgb(154pt)=(0.677195,0.472696,0); rgb(155pt)=(0.6717,0.470916,0); rgb(156pt)=(0.665891,0.469169,0); rgb(157pt)=(0.659791,0.46745,0); rgb(158pt)=(0.653423,0.465756,0); rgb(159pt)=(0.64681,0.464084,0); rgb(160pt)=(0.639976,0.462432,0); rgb(161pt)=(0.632943,0.460795,0); rgb(162pt)=(0.625734,0.459171,0); rgb(163pt)=(0.618373,0.457556,0); rgb(164pt)=(0.610882,0.455948,0); rgb(165pt)=(0.603284,0.454343,0); rgb(166pt)=(0.595604,0.452737,0); rgb(167pt)=(0.587863,0.451129,0); rgb(168pt)=(0.580084,0.449514,0); rgb(169pt)=(0.572292,0.447889,0); rgb(170pt)=(0.564508,0.446252,0); rgb(171pt)=(0.556756,0.444599,0); rgb(172pt)=(0.549059,0.442927,0); rgb(173pt)=(0.54144,0.441232,0); rgb(174pt)=(0.533922,0.439512,0); rgb(175pt)=(0.526529,0.437764,0); rgb(176pt)=(0.519282,0.435983,0); rgb(177pt)=(0.512206,0.434168,0); rgb(178pt)=(0.505323,0.432315,0); rgb(179pt)=(0.498628,0.430422,3.92506e-06); rgb(180pt)=(0.491973,0.428504,3.49981e-05); rgb(181pt)=(0.485331,0.426562,9.63073e-05); rgb(182pt)=(0.478704,0.424596,0.000186979); rgb(183pt)=(0.472096,0.422609,0.000306141); rgb(184pt)=(0.465508,0.420599,0.00045292); rgb(185pt)=(0.458942,0.418567,0.000626441); rgb(186pt)=(0.452401,0.416515,0.000825833); rgb(187pt)=(0.445885,0.414441,0.00105022); rgb(188pt)=(0.439399,0.412348,0.00129873); rgb(189pt)=(0.432942,0.410234,0.00157049); rgb(190pt)=(0.426518,0.408102,0.00186463); rgb(191pt)=(0.420129,0.40595,0.00218028); rgb(192pt)=(0.413777,0.40378,0.00251655); rgb(193pt)=(0.407464,0.401592,0.00287258); rgb(194pt)=(0.401191,0.399386,0.00324749); rgb(195pt)=(0.394962,0.397164,0.00364042); rgb(196pt)=(0.388777,0.394925,0.00405048); rgb(197pt)=(0.38264,0.39267,0.00447681); rgb(198pt)=(0.376552,0.390399,0.00491852); rgb(199pt)=(0.370516,0.388113,0.00537476); rgb(200pt)=(0.364532,0.385812,0.00584464); rgb(201pt)=(0.358605,0.383497,0.00632729); rgb(202pt)=(0.352735,0.381168,0.00682184); rgb(203pt)=(0.346925,0.378826,0.00732741); rgb(204pt)=(0.341176,0.376471,0.00784314); rgb(205pt)=(0.335485,0.374093,0.00847245); rgb(206pt)=(0.329843,0.371682,0.00930909); rgb(207pt)=(0.324249,0.369242,0.0103377); rgb(208pt)=(0.318701,0.366772,0.0115428); rgb(209pt)=(0.313198,0.364275,0.0129091); rgb(210pt)=(0.307739,0.361753,0.0144211); rgb(211pt)=(0.302322,0.359206,0.0160634); rgb(212pt)=(0.296945,0.356637,0.0178207); rgb(213pt)=(0.291607,0.354048,0.0196776); rgb(214pt)=(0.286307,0.35144,0.0216186); rgb(215pt)=(0.281043,0.348814,0.0236284); rgb(216pt)=(0.275813,0.346172,0.0256916); rgb(217pt)=(0.270616,0.343517,0.0277927); rgb(218pt)=(0.265451,0.340849,0.0299163); rgb(219pt)=(0.260317,0.33817,0.0320472); rgb(220pt)=(0.25521,0.335482,0.0341698); rgb(221pt)=(0.250131,0.332786,0.0362688); rgb(222pt)=(0.245078,0.330085,0.0383287); rgb(223pt)=(0.240048,0.327379,0.0403343); rgb(224pt)=(0.235042,0.324671,0.04227); rgb(225pt)=(0.230056,0.321962,0.0441205); rgb(226pt)=(0.22509,0.319254,0.0458704); rgb(227pt)=(0.220142,0.316548,0.0475043); rgb(228pt)=(0.215212,0.313846,0.0490067); rgb(229pt)=(0.210296,0.311149,0.0503624); rgb(230pt)=(0.205395,0.308459,0.0515759); rgb(231pt)=(0.200514,0.305763,0.052757); rgb(232pt)=(0.195655,0.303061,0.0539242); rgb(233pt)=(0.190817,0.300353,0.0550763); rgb(234pt)=(0.186001,0.297639,0.0562123); rgb(235pt)=(0.181207,0.294918,0.0573313); rgb(236pt)=(0.176434,0.292191,0.0584321); rgb(237pt)=(0.171685,0.289458,0.0595136); rgb(238pt)=(0.166957,0.286719,0.060575); rgb(239pt)=(0.162252,0.283973,0.0616151); rgb(240pt)=(0.15757,0.281221,0.0626328); rgb(241pt)=(0.152911,0.278463,0.0636271); rgb(242pt)=(0.148275,0.275699,0.0645971); rgb(243pt)=(0.143663,0.272929,0.0655416); rgb(244pt)=(0.139074,0.270152,0.0664596); rgb(245pt)=(0.134508,0.26737,0.06735); rgb(246pt)=(0.129967,0.264581,0.0682118); rgb(247pt)=(0.125449,0.261787,0.0690441); rgb(248pt)=(0.120956,0.258986,0.0698456); rgb(249pt)=(0.116487,0.25618,0.0706154); rgb(250pt)=(0.112043,0.253367,0.0713525); rgb(251pt)=(0.107623,0.250549,0.0720557); rgb(252pt)=(0.103229,0.247724,0.0727241); rgb(253pt)=(0.0988592,0.244894,0.0733566); rgb(254pt)=(0.0945149,0.242058,0.0739522); rgb(255pt)=(0.0901961,0.239216,0.0745098)}, mesh/rows=49]
table[row sep=crcr, point meta=\thisrow{c}] {%
%
x	y	z	c\\
56	0.093	-0.13503761957163	-0.13503761957163\\
56	0.09666	-0.608243050596752	-0.608243050596752\\
56	0.10032	-1.03124165121413	-1.03124165121413\\
56	0.10398	-1.40403342142373	-1.40403342142373\\
56	0.10764	-1.72661836122556	-1.72661836122556\\
56	0.1113	-1.99899647061962	-1.99899647061962\\
56	0.11496	-2.22116774960593	-2.22116774960593\\
56	0.11862	-2.39313219818447	-2.39313219818447\\
56	0.12228	-2.51488981635525	-2.51488981635525\\
56	0.12594	-2.58644060411829	-2.58644060411829\\
56	0.1296	-2.60778456147355	-2.60778456147355\\
56	0.13326	-2.57892168842105	-2.57892168842105\\
56	0.13692	-2.4998519849608	-2.4998519849608\\
56	0.14058	-2.37057545109275	-2.37057545109275\\
56	0.14424	-2.19109208681695	-2.19109208681695\\
56	0.1479	-1.9614018921334	-1.9614018921334\\
56	0.15156	-1.68150486704207	-1.68150486704207\\
56	0.15522	-1.35140101154298	-1.35140101154298\\
56	0.15888	-0.971090325636176	-0.971090325636176\\
56	0.16254	-0.540572809321567	-0.540572809321567\\
56	0.1662	-0.0598484625991915	-0.0598484625991915\\
56	0.16986	0.471082714530951	0.471082714530951\\
56	0.17352	1.05222072206885	1.05222072206885\\
56	0.17718	1.68356556001451	1.68356556001451\\
56	0.18084	2.36511722836793	2.36511722836793\\
56	0.1845	3.09687572712913	3.09687572712913\\
56	0.18816	3.87884105629806	3.87884105629806\\
56	0.19182	4.71101321587477	4.71101321587477\\
56	0.19548	5.59339220585923	5.59339220585923\\
56	0.19914	6.52597802625147	6.52597802625147\\
56	0.2028	7.50877067705149	7.50877067705149\\
56	0.20646	8.54177015825927	8.54177015825927\\
56	0.21012	9.6249764698748	9.6249764698748\\
56	0.21378	10.7583896118981	10.7583896118981\\
56	0.21744	11.9420095843291	11.9420095843291\\
56	0.2211	13.1758363871679	13.1758363871679\\
56	0.22476	14.4598700204145	14.4598700204145\\
56	0.22842	15.7941104840688	15.7941104840688\\
56	0.23208	17.1785577781309	17.1785577781309\\
56	0.23574	18.6132119026008	18.6132119026008\\
56	0.2394	20.0980728574785	20.0980728574785\\
56	0.24306	21.6331406427638	21.6331406427638\\
56	0.24672	23.218415258457	23.218415258457\\
56	0.25038	24.8538967045579	24.8538967045579\\
56	0.25404	26.5395849810666	26.5395849810666\\
56	0.2577	28.275480087983	28.275480087983\\
56	0.26136	30.0615820253072	30.0615820253072\\
56	0.26502	31.8978907930392	31.8978907930392\\
56	0.26868	33.7844063911789	33.7844063911789\\
56	0.27234	35.7211288197264	35.7211288197264\\
56	0.276	37.7080580786817	37.7080580786817\\
56.375	0.093	-0.145922382547607	-0.145922382547607\\
56.375	0.09666	-0.634251583028526	-0.634251583028526\\
56.375	0.10032	-1.07237395310169	-1.07237395310169\\
56.375	0.10398	-1.4602894927671	-1.4602894927671\\
56.375	0.10764	-1.79799820202474	-1.79799820202474\\
56.375	0.1113	-2.08550008087459	-2.08550008087459\\
56.375	0.11496	-2.32279512931673	-2.32279512931673\\
56.375	0.11862	-2.50988334735106	-2.50988334735106\\
56.375	0.12228	-2.64676473497762	-2.64676473497762\\
56.375	0.12594	-2.73343929219645	-2.73343929219645\\
56.375	0.1296	-2.76990701900752	-2.76990701900752\\
56.375	0.13326	-2.7561679154108	-2.7561679154108\\
56.375	0.13692	-2.69222198140636	-2.69222198140636\\
56.375	0.14058	-2.5780692169941	-2.5780692169941\\
56.375	0.14424	-2.41370962217414	-2.41370962217414\\
56.375	0.1479	-2.19914319694637	-2.19914319694637\\
56.375	0.15156	-1.93436994131086	-1.93436994131086\\
56.375	0.15522	-1.61938985526758	-1.61938985526758\\
56.375	0.15888	-1.25420293881653	-1.25420293881653\\
56.375	0.16254	-0.838809191957729	-0.838809191957729\\
56.375	0.1662	-0.373208614691137	-0.373208614691137\\
56.375	0.16986	0.142598792983193	0.142598792983193\\
56.375	0.17352	0.708613031065276	0.708613031065276\\
56.375	0.17718	1.32483409955515	1.32483409955515\\
56.375	0.18084	1.99126199845274	1.99126199845274\\
56.375	0.1845	2.70789672775815	2.70789672775815\\
56.375	0.18816	3.4747382874713	3.4747382874713\\
56.375	0.19182	4.2917866775922	4.2917866775922\\
56.375	0.19548	5.15904189812088	5.15904189812088\\
56.375	0.19914	6.07650394905733	6.07650394905733\\
56.375	0.2028	7.04417283040154	7.04417283040154\\
56.375	0.20646	8.06204854215348	8.06204854215348\\
56.375	0.21012	9.13013108431322	9.13013108431322\\
56.375	0.21378	10.2484204568807	10.2484204568807\\
56.375	0.21744	11.4169166598559	11.4169166598559\\
56.375	0.2211	12.635619693239	12.635619693239\\
56.375	0.22476	13.9045295570298	13.9045295570298\\
56.375	0.22842	15.2236462512283	15.2236462512283\\
56.375	0.23208	16.5929697758346	16.5929697758346\\
56.375	0.23574	18.0125001308487	18.0125001308487\\
56.375	0.2394	19.4822373162705	19.4822373162705\\
56.375	0.24306	21.0021813321001	21.0021813321001\\
56.375	0.24672	22.5723321783374	22.5723321783374\\
56.375	0.25038	24.1926898549826	24.1926898549826\\
56.375	0.25404	25.8632543620354	25.8632543620354\\
56.375	0.2577	27.5840256994961	27.5840256994961\\
56.375	0.26136	29.3550038673645	29.3550038673645\\
56.375	0.26502	31.1761888656407	31.1761888656407\\
56.375	0.26868	33.0475806943246	33.0475806943246\\
56.375	0.27234	34.9691793534163	34.9691793534163\\
56.375	0.276	36.9409848429157	36.9409848429157\\
56.75	0.093	-0.146939947967834	-0.146939947967834\\
56.75	0.09666	-0.650392917904551	-0.650392917904551\\
56.75	0.10032	-1.10363905743353	-1.10363905743353\\
56.75	0.10398	-1.50667836655472	-1.50667836655472\\
56.75	0.10764	-1.85951084526814	-1.85951084526814\\
56.75	0.1113	-2.16213649357383	-2.16213649357383\\
56.75	0.11496	-2.41455531147173	-2.41455531147173\\
56.75	0.11862	-2.61676729896187	-2.61676729896187\\
56.75	0.12228	-2.76877245604427	-2.76877245604427\\
56.75	0.12594	-2.87057078271886	-2.87057078271886\\
56.75	0.1296	-2.92216227898574	-2.92216227898574\\
56.75	0.13326	-2.92354694484483	-2.92354694484483\\
56.75	0.13692	-2.87472478029615	-2.87472478029615\\
56.75	0.14058	-2.77569578533973	-2.77569578533973\\
56.75	0.14424	-2.62645995997555	-2.62645995997555\\
56.75	0.1479	-2.42701730420359	-2.42701730420359\\
56.75	0.15156	-2.17736781802386	-2.17736781802386\\
56.75	0.15522	-1.87751150143639	-1.87751150143639\\
56.75	0.15888	-1.52744835444116	-1.52744835444116\\
56.75	0.16254	-1.12717837703811	-1.12717837703811\\
56.75	0.1662	-0.676701569227362	-0.676701569227362\\
56.75	0.16986	-0.176017931008843	-0.176017931008843\\
56.75	0.17352	0.374872537617485	0.374872537617485\\
56.75	0.17718	0.975969836651522	0.975969836651522\\
56.75	0.18084	1.62727396609335	1.62727396609335\\
56.75	0.1845	2.32878492594295	2.32878492594295\\
56.75	0.18816	3.08050271620029	3.08050271620029\\
56.75	0.19182	3.8824273368654	3.8824273368654\\
56.75	0.19548	4.73455878793827	4.73455878793827\\
56.75	0.19914	5.63689706941891	5.63689706941891\\
56.75	0.2028	6.58944218130731	6.58944218130731\\
56.75	0.20646	7.59219412360352	7.59219412360352\\
56.75	0.21012	8.64515289630742	8.64515289630742\\
56.75	0.21378	9.74831849941909	9.74831849941909\\
56.75	0.21744	10.9016909329386	10.9016909329386\\
56.75	0.2211	12.1052701968658	12.1052701968658\\
56.75	0.22476	13.3590562912007	13.3590562912007\\
56.75	0.22842	14.6630492159435	14.6630492159435\\
56.75	0.23208	16.017248971094	16.017248971094\\
56.75	0.23574	17.4216555566523	17.4216555566523\\
56.75	0.2394	18.8762689726183	18.8762689726183\\
56.75	0.24306	20.3810892189921	20.3810892189921\\
56.75	0.24672	21.9361162957736	21.9361162957736\\
56.75	0.25038	23.541350202963	23.541350202963\\
56.75	0.25404	25.19679094056	25.19679094056\\
56.75	0.2577	26.9024385085649	26.9024385085649\\
56.75	0.26136	28.6582929069775	28.6582929069775\\
56.75	0.26502	30.4643541357979	30.4643541357979\\
56.75	0.26868	32.320622195026	32.320622195026\\
56.75	0.27234	34.2270970846619	34.2270970846619\\
56.75	0.276	36.1837788047055	36.1837788047055\\
57.125	0.093	-0.138090315832283	-0.138090315832283\\
57.125	0.09666	-0.656667055224812	-0.656667055224812\\
57.125	0.10032	-1.12503696420957	-1.12503696420957\\
57.125	0.10398	-1.54320004278658	-1.54320004278658\\
57.125	0.10764	-1.91115629095581	-1.91115629095581\\
57.125	0.1113	-2.22890570871728	-2.22890570871728\\
57.125	0.11496	-2.49644829607099	-2.49644829607099\\
57.125	0.11862	-2.71378405301692	-2.71378405301692\\
57.125	0.12228	-2.88091297955511	-2.88091297955511\\
57.125	0.12594	-2.99783507568553	-2.99783507568553\\
57.125	0.1296	-3.06455034140819	-3.06455034140819\\
57.125	0.13326	-3.0810587767231	-3.0810587767231\\
57.125	0.13692	-3.04736038163023	-3.04736038163023\\
57.125	0.14058	-2.96345515612959	-2.96345515612959\\
57.125	0.14424	-2.82934310022119	-2.82934310022119\\
57.125	0.1479	-2.64502421390505	-2.64502421390505\\
57.125	0.15156	-2.41049849718113	-2.41049849718113\\
57.125	0.15522	-2.12576595004945	-2.12576595004945\\
57.125	0.15888	-1.79082657250999	-1.79082657250999\\
57.125	0.16254	-1.40568036456279	-1.40568036456279\\
57.125	0.1662	-0.970327326207823	-0.970327326207823\\
57.125	0.16986	-0.484767457445088	-0.484767457445088\\
57.125	0.17352	0.0509992417253997	0.0509992417253997\\
57.125	0.17718	0.636972771303682	0.636972771303682\\
57.125	0.18084	1.2731531312897	1.2731531312897\\
57.125	0.1845	1.95954032168349	1.95954032168349\\
57.125	0.18816	2.69613434248504	2.69613434248504\\
57.125	0.19182	3.48293519369435	3.48293519369435\\
57.125	0.19548	4.31994287531143	4.31994287531143\\
57.125	0.19914	5.20715738733625	5.20715738733625\\
57.125	0.2028	6.14457872976887	6.14457872976887\\
57.125	0.20646	7.13220690260927	7.13220690260927\\
57.125	0.21012	8.17004190585739	8.17004190585739\\
57.125	0.21378	9.25808373951327	9.25808373951327\\
57.125	0.21744	10.3963324035769	10.3963324035769\\
57.125	0.2211	11.5847878980483	11.5847878980483\\
57.125	0.22476	12.8234502229275	12.8234502229275\\
57.125	0.22842	14.1123193782144	14.1123193782144\\
57.125	0.23208	15.4513953639091	15.4513953639091\\
57.125	0.23574	16.8406781800116	16.8406781800116\\
57.125	0.2394	18.2801678265218	18.2801678265218\\
57.125	0.24306	19.7698643034398	19.7698643034398\\
57.125	0.24672	21.3097676107656	21.3097676107656\\
57.125	0.25038	22.8998777484991	22.8998777484991\\
57.125	0.25404	24.5401947166404	24.5401947166404\\
57.125	0.2577	26.2307185151895	26.2307185151895\\
57.125	0.26136	27.9714491441463	27.9714491441463\\
57.125	0.26502	29.7623866035108	29.7623866035108\\
57.125	0.26868	31.6035308932831	31.6035308932831\\
57.125	0.27234	33.4948820134632	33.4948820134632\\
57.125	0.276	35.4364399640511	35.4364399640511\\
57.5	0.093	-0.11937348614094	-0.11937348614094\\
57.5	0.09666	-0.653073994989281	-0.653073994989281\\
57.5	0.10032	-1.13656767342985	-1.13656767342985\\
57.5	0.10398	-1.56985452146264	-1.56985452146264\\
57.5	0.10764	-1.95293453908766	-1.95293453908766\\
57.5	0.1113	-2.28580772630495	-2.28580772630495\\
57.5	0.11496	-2.56847408311447	-2.56847408311447\\
57.5	0.11862	-2.8009336095162	-2.8009336095162\\
57.5	0.12228	-2.98318630551017	-2.98318630551017\\
57.5	0.12594	-3.11523217109638	-3.11523217109638\\
57.5	0.1296	-3.19707120627486	-3.19707120627486\\
57.5	0.13326	-3.22870341104554	-3.22870341104554\\
57.5	0.13692	-3.21012878540849	-3.21012878540849\\
57.5	0.14058	-3.14134732936363	-3.14134732936363\\
57.5	0.14424	-3.02235904291105	-3.02235904291105\\
57.5	0.1479	-2.85316392605071	-2.85316392605071\\
57.5	0.15156	-2.63376197878258	-2.63376197878258\\
57.5	0.15522	-2.36415320110668	-2.36415320110668\\
57.5	0.15888	-2.04433759302307	-2.04433759302307\\
57.5	0.16254	-1.67431515453165	-1.67431515453165\\
57.5	0.1662	-1.25408588563246	-1.25408588563246\\
57.5	0.16986	-0.783649786325512	-0.783649786325512\\
57.5	0.17352	-0.263006856610836	-0.263006856610836\\
57.5	0.17718	0.307842903511606	0.307842903511606\\
57.5	0.18084	0.928899494041836	0.928899494041836\\
57.5	0.1845	1.60016291497985	1.60016291497985\\
57.5	0.18816	2.32163316632558	2.32163316632558\\
57.5	0.19182	3.09331024807911	3.09331024807911\\
57.5	0.19548	3.91519416024038	3.91519416024038\\
57.5	0.19914	4.78728490280942	4.78728490280942\\
57.5	0.2028	5.70958247578619	5.70958247578619\\
57.5	0.20646	6.68208687917078	6.68208687917078\\
57.5	0.21012	7.70479811296312	7.70479811296312\\
57.5	0.21378	8.77771617716321	8.77771617716321\\
57.5	0.21744	9.90084107177107	9.90084107177107\\
57.5	0.2211	11.0741727967866	11.0741727967866\\
57.5	0.22476	12.29771135221	12.29771135221\\
57.5	0.22842	13.5714567380412	13.5714567380412\\
57.5	0.23208	14.8954089542801	14.8954089542801\\
57.5	0.23574	16.2695680009268	16.2695680009268\\
57.5	0.2394	17.6939338779812	17.6939338779812\\
57.5	0.24306	19.1685065854434	19.1685065854434\\
57.5	0.24672	20.6932861233134	20.6932861233134\\
57.5	0.25038	22.2682724915911	22.2682724915911\\
57.5	0.25404	23.8934656902765	23.8934656902765\\
57.5	0.2577	25.5688657193698	25.5688657193698\\
57.5	0.26136	27.2944725788708	27.2944725788708\\
57.5	0.26502	29.0702862687796	29.0702862687796\\
57.5	0.26868	30.8963067890961	30.8963067890961\\
57.5	0.27234	32.7725341398204	32.7725341398204\\
57.5	0.276	34.6989683209525	34.6989683209525\\
57.875	0.093	-0.0907894588938767	-0.0907894588938767\\
57.875	0.09666	-0.639613737198029	-0.639613737198029\\
57.875	0.10032	-1.13823118509438	-1.13823118509438\\
57.875	0.10398	-1.58664180258299	-1.58664180258299\\
57.875	0.10764	-1.98484558966379	-1.98484558966379\\
57.875	0.1113	-2.33284254633688	-2.33284254633688\\
57.875	0.11496	-2.63063267260219	-2.63063267260219\\
57.875	0.11862	-2.87821596845974	-2.87821596845974\\
57.875	0.12228	-3.07559243390949	-3.07559243390949\\
57.875	0.12594	-3.22276206895151	-3.22276206895151\\
57.875	0.1296	-3.31972487358577	-3.31972487358577\\
57.875	0.13326	-3.36648084781227	-3.36648084781227\\
57.875	0.13692	-3.363029991631	-3.363029991631\\
57.875	0.14058	-3.30937230504198	-3.30937230504198\\
57.875	0.14424	-3.20550778804515	-3.20550778804515\\
57.875	0.1479	-3.05143644064063	-3.05143644064063\\
57.875	0.15156	-2.84715826282828	-2.84715826282828\\
57.875	0.15522	-2.59267325460822	-2.59267325460822\\
57.875	0.15888	-2.28798141598036	-2.28798141598036\\
57.875	0.16254	-1.93308274694476	-1.93308274694476\\
57.875	0.1662	-1.52797724750138	-1.52797724750138\\
57.875	0.16986	-1.07266491765024	-1.07266491765024\\
57.875	0.17352	-0.567145757391351	-0.567145757391351\\
57.875	0.17718	-0.0114197667246927	-0.0114197667246927\\
57.875	0.18084	0.594513054349726	0.594513054349726\\
57.875	0.1845	1.25065270583193	1.25065270583193\\
57.875	0.18816	1.95699918772188	1.95699918772188\\
57.875	0.19182	2.71355250001959	2.71355250001959\\
57.875	0.19548	3.52031264272507	3.52031264272507\\
57.875	0.19914	4.3772796158383	4.3772796158383\\
57.875	0.2028	5.28445341935929	5.28445341935929\\
57.875	0.20646	6.2418340532881	6.2418340532881\\
57.875	0.21012	7.2494215176246	7.2494215176246\\
57.875	0.21378	8.30721581236891	8.30721581236891\\
57.875	0.21744	9.41521693752092	9.41521693752092\\
57.875	0.2211	10.5734248930808	10.5734248930808\\
57.875	0.22476	11.7818396790483	11.7818396790483\\
57.875	0.22842	13.0404612954237	13.0404612954237\\
57.875	0.23208	14.3492897422068	14.3492897422068\\
57.875	0.23574	15.7083250193977	15.7083250193977\\
57.875	0.2394	17.1175671269963	17.1175671269963\\
57.875	0.24306	18.5770160650027	18.5770160650027\\
57.875	0.24672	20.0866718334168	20.0866718334168\\
57.875	0.25038	21.6465344322388	21.6465344322388\\
57.875	0.25404	23.2566038614684	23.2566038614684\\
57.875	0.2577	24.9168801211059	24.9168801211059\\
57.875	0.26136	26.6273632111511	26.6273632111511\\
57.875	0.26502	28.3880531316041	28.3880531316041\\
57.875	0.26868	30.1989498824648	30.1989498824648\\
57.875	0.27234	32.0600534637333	32.0600534637333\\
57.875	0.276	33.9713638754096	33.9713638754096\\
58.25	0.093	-0.0523382340910636	-0.0523382340910636\\
58.25	0.09666	-0.616286281850972	-0.616286281850972\\
58.25	0.10032	-1.13002749920314	-1.13002749920314\\
58.25	0.10398	-1.59356188614755	-1.59356188614755\\
58.25	0.10764	-2.00688944268417	-2.00688944268417\\
58.25	0.1113	-2.37001016881305	-2.37001016881305\\
58.25	0.11496	-2.68292406453413	-2.68292406453413\\
58.25	0.11862	-2.9456311298475	-2.9456311298475\\
58.25	0.12228	-3.15813136475306	-3.15813136475306\\
58.25	0.12594	-3.32042476925086	-3.32042476925086\\
58.25	0.1296	-3.43251134334093	-3.43251134334093\\
58.25	0.13326	-3.49439108702325	-3.49439108702325\\
58.25	0.13692	-3.50606400029776	-3.50606400029776\\
58.25	0.14058	-3.46753008316455	-3.46753008316455\\
58.25	0.14424	-3.37878933562354	-3.37878933562354\\
58.25	0.1479	-3.23984175767477	-3.23984175767477\\
58.25	0.15156	-3.05068734931826	-3.05068734931826\\
58.25	0.15522	-2.81132611055398	-2.81132611055398\\
58.25	0.15888	-2.52175804138194	-2.52175804138194\\
58.25	0.16254	-2.18198314180211	-2.18198314180211\\
58.25	0.1662	-1.79200141181452	-1.79200141181452\\
58.25	0.16986	-1.3518128514192	-1.3518128514192\\
58.25	0.17352	-0.861417460616117	-0.861417460616117\\
58.25	0.17718	-0.320815239405242	-0.320815239405242\\
58.25	0.18084	0.269993812213393	0.269993812213393\\
58.25	0.1845	0.91100969423978	0.91100969423978\\
58.25	0.18816	1.60223240667392	1.60223240667392\\
58.25	0.19182	2.34366194951582	2.34366194951582\\
58.25	0.19548	3.13529832276552	3.13529832276552\\
58.25	0.19914	3.97714152642294	3.97714152642294\\
58.25	0.2028	4.86919156048818	4.86919156048818\\
58.25	0.20646	5.81144842496114	5.81144842496114\\
58.25	0.21012	6.80391211984185	6.80391211984185\\
58.25	0.21378	7.84658264513033	7.84658264513033\\
58.25	0.21744	8.93946000082661	8.93946000082661\\
58.25	0.2211	10.0825441869306	10.0825441869306\\
58.25	0.22476	11.2758352034424	11.2758352034424\\
58.25	0.22842	12.5193330503619	12.5193330503619\\
58.25	0.23208	13.8130377276893	13.8130377276893\\
58.25	0.23574	15.1569492354243	15.1569492354243\\
58.25	0.2394	16.5510675735672	16.5510675735672\\
58.25	0.24306	17.9953927421177	17.9953927421177\\
58.25	0.24672	19.4899247410761	19.4899247410761\\
58.25	0.25038	21.0346635704422	21.0346635704422\\
58.25	0.25404	22.6296092302161	22.6296092302161\\
58.25	0.2577	24.2747617203978	24.2747617203978\\
58.25	0.26136	25.9701210409872	25.9701210409872\\
58.25	0.26502	27.7156871919844	27.7156871919844\\
58.25	0.26868	29.5114601733893	29.5114601733893\\
58.25	0.27234	31.357439985202	31.357439985202\\
58.25	0.276	33.2536266274224	33.2536266274224\\
58.625	0.093	-0.00401981173245858	-0.00401981173245858\\
58.625	0.09666	-0.583091628948178	-0.583091628948178\\
58.625	0.10032	-1.11195661575616	-1.11195661575616\\
58.625	0.10398	-1.59061477215635	-1.59061477215635\\
58.625	0.10764	-2.01906609814878	-2.01906609814878\\
58.625	0.1113	-2.39731059373344	-2.39731059373344\\
58.625	0.11496	-2.72534825891034	-2.72534825891034\\
58.625	0.11862	-3.00317909367949	-3.00317909367949\\
58.625	0.12228	-3.23080309804087	-3.23080309804087\\
58.625	0.12594	-3.40822027199448	-3.40822027199448\\
58.625	0.1296	-3.53543061554033	-3.53543061554033\\
58.625	0.13326	-3.61243412867843	-3.61243412867843\\
58.625	0.13692	-3.63923081140875	-3.63923081140875\\
58.625	0.14058	-3.61582066373133	-3.61582066373133\\
58.625	0.14424	-3.54220368564613	-3.54220368564613\\
58.625	0.1479	-3.41837987715317	-3.41837987715317\\
58.625	0.15156	-3.24434923825245	-3.24434923825245\\
58.625	0.15522	-3.02011176894398	-3.02011176894398\\
58.625	0.15888	-2.74566746922772	-2.74566746922772\\
58.625	0.16254	-2.42101633910371	-2.42101633910371\\
58.625	0.1662	-2.04615837857193	-2.04615837857193\\
58.625	0.16986	-1.62109358763239	-1.62109358763239\\
58.625	0.17352	-1.14582196628509	-1.14582196628509\\
58.625	0.17718	-0.620343514530028	-0.620343514530028\\
58.625	0.18084	-0.0446582323672047	-0.0446582323672047\\
58.625	0.1845	0.581233880203399	0.581233880203399\\
58.625	0.18816	1.2573328231817	1.2573328231817\\
58.625	0.19182	1.98363859656784	1.98363859656784\\
58.625	0.19548	2.76015120036173	2.76015120036173\\
58.625	0.19914	3.58687063456334	3.58687063456334\\
58.625	0.2028	4.46379689917276	4.46379689917276\\
58.625	0.20646	5.39092999418995	5.39092999418995\\
58.625	0.21012	6.36826991961487	6.36826991961487\\
58.625	0.21378	7.39581667544756	7.39581667544756\\
58.625	0.21744	8.47357026168801	8.47357026168801\\
58.625	0.2211	9.60153067833623	9.60153067833623\\
58.625	0.22476	10.7796979253922	10.7796979253922\\
58.625	0.22842	12.008072002856	12.008072002856\\
58.625	0.23208	13.2866529107275	13.2866529107275\\
58.625	0.23574	14.6154406490067	14.6154406490067\\
58.625	0.2394	15.9944352176937	15.9944352176937\\
58.625	0.24306	17.4236366167885	17.4236366167885\\
58.625	0.24672	18.9030448462911	18.9030448462911\\
58.625	0.25038	20.4326599062015	20.4326599062015\\
58.625	0.25404	22.0124817965195	22.0124817965195\\
58.625	0.2577	23.6425105172454	23.6425105172454\\
58.625	0.26136	25.322746068379	25.322746068379\\
58.625	0.26502	27.0531884499204	27.0531884499204\\
58.625	0.26868	28.8338376618695	28.8338376618695\\
58.625	0.27234	30.6646937042264	30.6646937042264\\
58.625	0.276	32.5457565769911	32.5457565769911\\
59	0.093	0.0541658081819385	0.0541658081819385\\
59	0.09666	-0.540029778489593	-0.540029778489593\\
59	0.10032	-1.08401853475335	-1.08401853475335\\
59	0.10398	-1.57780046060934	-1.57780046060934\\
59	0.10764	-2.0213755560576	-2.0213755560576\\
59	0.1113	-2.41474382109805	-2.41474382109805\\
59	0.11496	-2.75790525573076	-2.75790525573076\\
59	0.11862	-3.05085985995569	-3.05085985995569\\
59	0.12228	-3.29360763377288	-3.29360763377288\\
59	0.12594	-3.48614857718228	-3.48614857718228\\
59	0.1296	-3.62848269018394	-3.62848269018394\\
59	0.13326	-3.72060997277785	-3.72060997277785\\
59	0.13692	-3.76253042496396	-3.76253042496396\\
59	0.14058	-3.75424404674235	-3.75424404674235\\
59	0.14424	-3.69575083811293	-3.69575083811293\\
59	0.1479	-3.58705079907578	-3.58705079907578\\
59	0.15156	-3.42814392963087	-3.42814392963087\\
59	0.15522	-3.21903022977816	-3.21903022977816\\
59	0.15888	-2.95970969951774	-2.95970969951774\\
59	0.16254	-2.65018233884951	-2.65018233884951\\
59	0.1662	-2.29044814777355	-2.29044814777355\\
59	0.16986	-1.88050712628981	-1.88050712628981\\
59	0.17352	-1.4203592743983	-1.4203592743983\\
59	0.17718	-0.910004592099021	-0.910004592099021\\
59	0.18084	-0.349443079392039	-0.349443079392039\\
59	0.1845	0.261325263722782	0.261325263722782\\
59	0.18816	0.922300437245326	0.922300437245326\\
59	0.19182	1.63348244117563	1.63348244117563\\
59	0.19548	2.39487127551374	2.39487127551374\\
59	0.19914	3.20646694025956	3.20646694025956\\
59	0.2028	4.06826943541317	4.06826943541317\\
59	0.20646	4.98027876097457	4.98027876097457\\
59	0.21012	5.94249491694366	5.94249491694366\\
59	0.21378	6.95491790332056	6.95491790332056\\
59	0.21744	8.01754772010523	8.01754772010523\\
59	0.2211	9.13038436729761	9.13038436729761\\
59	0.22476	10.2934278448978	10.2934278448978\\
59	0.22842	11.5066781529058	11.5066781529058\\
59	0.23208	12.7701352913214	12.7701352913214\\
59	0.23574	14.083799260145	14.083799260145\\
59	0.2394	15.4476700593762	15.4476700593762\\
59	0.24306	16.8617476890152	16.8617476890152\\
59	0.24672	18.3260321490619	18.3260321490619\\
59	0.25038	19.8405234395165	19.8405234395165\\
59	0.25404	21.4052215603787	21.4052215603787\\
59	0.2577	23.0201265116488	23.0201265116488\\
59	0.26136	24.6852382933266	24.6852382933266\\
59	0.26502	26.4005569054122	26.4005569054122\\
59	0.26868	28.1660823479055	28.1660823479055\\
59	0.27234	29.9818146208066	29.9818146208066\\
59	0.276	31.8477537241154	31.8477537241154\\
59.375	0.093	0.122218625652042	0.122218625652042\\
59.375	0.09666	-0.487100730475273	-0.487100730475273\\
59.375	0.10032	-1.04621325619485	-1.04621325619485\\
59.375	0.10398	-1.55511895150664	-1.55511895150664\\
59.375	0.10764	-2.01381781641066	-2.01381781641066\\
59.375	0.1113	-2.42230985090695	-2.42230985090695\\
59.375	0.11496	-2.78059505499542	-2.78059505499542\\
59.375	0.11862	-3.08867342867618	-3.08867342867618\\
59.375	0.12228	-3.34654497194913	-3.34654497194913\\
59.375	0.12594	-3.55420968481437	-3.55420968481437\\
59.375	0.1296	-3.71166756727182	-3.71166756727182\\
59.375	0.13326	-3.81891861932151	-3.81891861932151\\
59.375	0.13692	-3.87596284096345	-3.87596284096345\\
59.375	0.14058	-3.88280023219763	-3.88280023219763\\
59.375	0.14424	-3.83943079302399	-3.83943079302399\\
59.375	0.1479	-3.74585452344266	-3.74585452344266\\
59.375	0.15156	-3.60207142345353	-3.60207142345353\\
59.375	0.15522	-3.40808149305663	-3.40808149305663\\
59.375	0.15888	-3.16388473225199	-3.16388473225199\\
59.375	0.16254	-2.86948114103958	-2.86948114103958\\
59.375	0.1662	-2.5248707194194	-2.5248707194194\\
59.375	0.16986	-2.13005346739148	-2.13005346739148\\
59.375	0.17352	-1.68502938495575	-1.68502938495575\\
59.375	0.17718	-1.18979847211231	-1.18979847211231\\
59.375	0.18084	-0.644360728861081	-0.644360728861081\\
59.375	0.1845	-0.0487161552020723	-0.0487161552020723\\
59.375	0.18816	0.59713524886466	0.59713524886466\\
59.375	0.19182	1.29319348333915	1.29319348333915\\
59.375	0.19548	2.03945854822148	2.03945854822148\\
59.375	0.19914	2.83593044351149	2.83593044351149\\
59.375	0.2028	3.68260916920929	3.68260916920929\\
59.375	0.20646	4.5794947253149	4.5794947253149\\
59.375	0.21012	5.52658711182821	5.52658711182821\\
59.375	0.21378	6.52388632874933	6.52388632874933\\
59.375	0.21744	7.57139237607815	7.57139237607815\\
59.375	0.2211	8.66910525381475	8.66910525381475\\
59.375	0.22476	9.81702496195916	9.81702496195916\\
59.375	0.22842	11.0151515005113	11.0151515005113\\
59.375	0.23208	12.2634848694712	12.2634848694712\\
59.375	0.23574	13.5620250688389	13.5620250688389\\
59.375	0.2394	14.9107720986143	14.9107720986143\\
59.375	0.24306	16.3097259587975	16.3097259587975\\
59.375	0.24672	17.7588866493885	17.7588866493885\\
59.375	0.25038	19.2582541703872	19.2582541703872\\
59.375	0.25404	20.8078285217937	20.8078285217937\\
59.375	0.2577	22.4076097036079	22.4076097036079\\
59.375	0.26136	24.05759771583	24.05759771583\\
59.375	0.26502	25.7577925584597	25.7577925584597\\
59.375	0.26868	27.5081942314972	27.5081942314972\\
59.375	0.27234	29.3088027349425	29.3088027349425\\
59.375	0.276	31.1596180687956	31.1596180687956\\
59.75	0.093	0.200138640677938	0.200138640677938\\
59.75	0.09666	-0.424304484905189	-0.424304484905189\\
59.75	0.10032	-0.998540780080546	-0.998540780080546\\
59.75	0.10398	-1.52257024484815	-1.52257024484815\\
59.75	0.10764	-1.99639287920796	-1.99639287920796\\
59.75	0.1113	-2.42000868316003	-2.42000868316003\\
59.75	0.11496	-2.79341765670434	-2.79341765670434\\
59.75	0.11862	-3.11661979984089	-3.11661979984089\\
59.75	0.12228	-3.38961511256967	-3.38961511256967\\
59.75	0.12594	-3.61240359489067	-3.61240359489067\\
59.75	0.1296	-3.78498524680393	-3.78498524680393\\
59.75	0.13326	-3.9073600683094	-3.9073600683094\\
59.75	0.13692	-3.97952805940713	-3.97952805940713\\
59.75	0.14058	-4.00148922009712	-4.00148922009712\\
59.75	0.14424	-3.97324355037929	-3.97324355037929\\
59.75	0.1479	-3.89479105025374	-3.89479105025374\\
59.75	0.15156	-3.76613171972043	-3.76613171972043\\
59.75	0.15522	-3.58726555877934	-3.58726555877934\\
59.75	0.15888	-3.35819256743049	-3.35819256743049\\
59.75	0.16254	-3.07891274567388	-3.07891274567388\\
59.75	0.1662	-2.74942609350951	-2.74942609350951\\
59.75	0.16986	-2.36973261093738	-2.36973261093738\\
59.75	0.17352	-1.93983229795749	-1.93983229795749\\
59.75	0.17718	-1.4597251545698	-1.4597251545698\\
59.75	0.18084	-0.929411180774387	-0.929411180774387\\
59.75	0.1845	-0.348890376571163	-0.348890376571163\\
59.75	0.18816	0.281837258039758	0.281837258039758\\
59.75	0.19182	0.962771723058466	0.962771723058466\\
59.75	0.19548	1.69391301848495	1.69391301848495\\
59.75	0.19914	2.47526114431921	2.47526114431921\\
59.75	0.2028	3.30681610056122	3.30681610056122\\
59.75	0.20646	4.188577887211	4.188577887211\\
59.75	0.21012	5.12054650426852	5.12054650426852\\
59.75	0.21378	6.1027219517338	6.1027219517338\\
59.75	0.21744	7.13510422960684	7.13510422960684\\
59.75	0.2211	8.21769333788765	8.21769333788765\\
59.75	0.22476	9.35048927657628	9.35048927657628\\
59.75	0.22842	10.5334920456726	10.5334920456726\\
59.75	0.23208	11.7667016451767	11.7667016451767\\
59.75	0.23574	13.0501180750886	13.0501180750886\\
59.75	0.2394	14.3837413354082	14.3837413354082\\
59.75	0.24306	15.7675714261356	15.7675714261356\\
59.75	0.24672	17.2016083472708	17.2016083472708\\
59.75	0.25038	18.6858520988137	18.6858520988137\\
59.75	0.25404	20.2203026807644	20.2203026807644\\
59.75	0.2577	21.8049600931228	21.8049600931228\\
59.75	0.26136	23.4398243358891	23.4398243358891\\
59.75	0.26502	25.124895409063	25.124895409063\\
59.75	0.26868	26.8601733126447	26.8601733126447\\
59.75	0.27234	28.6456580466343	28.6456580466343\\
59.75	0.276	30.4813496110315	30.4813496110315\\
60.125	0.093	0.287925853259626	0.287925853259626\\
60.125	0.09666	-0.351641041779313	-0.351641041779313\\
60.125	0.10032	-0.941001106410482	-0.941001106410482\\
60.125	0.10398	-1.48015434063387	-1.48015434063387\\
60.125	0.10764	-1.96910074444949	-1.96910074444949\\
60.125	0.1113	-2.40784031785737	-2.40784031785737\\
60.125	0.11496	-2.79637306085746	-2.79637306085746\\
60.125	0.11862	-3.1346989734498	-3.1346989734498\\
60.125	0.12228	-3.42281805563439	-3.42281805563439\\
60.125	0.12594	-3.6607303074112	-3.6607303074112\\
60.125	0.1296	-3.84843572878025	-3.84843572878025\\
60.125	0.13326	-3.98593431974153	-3.98593431974153\\
60.125	0.13692	-4.07322608029504	-4.07322608029504\\
60.125	0.14058	-4.11031101044085	-4.11031101044085\\
60.125	0.14424	-4.09718911017883	-4.09718911017883\\
60.125	0.1479	-4.03386037950906	-4.03386037950906\\
60.125	0.15156	-3.92032481843153	-3.92032481843153\\
60.125	0.15522	-3.75658242694626	-3.75658242694626\\
60.125	0.15888	-3.54263320505321	-3.54263320505321\\
60.125	0.16254	-3.27847715275239	-3.27847715275239\\
60.125	0.1662	-2.96411427004381	-2.96411427004381\\
60.125	0.16986	-2.59954455692748	-2.59954455692748\\
60.125	0.17352	-2.18476801340338	-2.18476801340338\\
60.125	0.17718	-1.71978463947153	-1.71978463947153\\
60.125	0.18084	-1.2045944351319	-1.2045944351319\\
60.125	0.1845	-0.639197400384489	-0.639197400384489\\
60.125	0.18816	-0.0235935352293524	-0.0235935352293524\\
60.125	0.19182	0.642217160333544	0.642217160333544\\
60.125	0.19548	1.35823468630424	1.35823468630424\\
60.125	0.19914	2.12445904268269	2.12445904268269\\
60.125	0.2028	2.94089022946892	2.94089022946892\\
60.125	0.20646	3.80752824666291	3.80752824666291\\
60.125	0.21012	4.72437309426459	4.72437309426459\\
60.125	0.21378	5.69142477227409	5.69142477227409\\
60.125	0.21744	6.70868328069135	6.70868328069135\\
60.125	0.2211	7.77614861951638	7.77614861951638\\
60.125	0.22476	8.89382078874917	8.89382078874917\\
60.125	0.22842	10.0616997883897	10.0616997883897\\
60.125	0.23208	11.279785618438	11.279785618438\\
60.125	0.23574	12.5480782788941	12.5480782788941\\
60.125	0.2394	13.8665777697579	13.8665777697579\\
60.125	0.24306	15.2352840910295	15.2352840910295\\
60.125	0.24672	16.6541972427088	16.6541972427088\\
60.125	0.25038	18.123317224796	18.123317224796\\
60.125	0.25404	19.6426440372909	19.6426440372909\\
60.125	0.2577	21.2121776801936	21.2121776801936\\
60.125	0.26136	22.831918153504	22.831918153504\\
60.125	0.26502	24.5018654572221	24.5018654572221\\
60.125	0.26868	26.2220195913481	26.2220195913481\\
60.125	0.27234	27.9923805558818	27.9923805558818\\
60.125	0.276	29.8129483508232	29.8129483508232\\
60.5	0.093	0.385580263397021	0.385580263397021\\
60.5	0.09666	-0.269110401097702	-0.269110401097702\\
60.5	0.10032	-0.873594235184655	-0.873594235184655\\
60.5	0.10398	-1.42787123886385	-1.42787123886385\\
60.5	0.10764	-1.93194141213526	-1.93194141213526\\
60.5	0.1113	-2.38580475499895	-2.38580475499895\\
60.5	0.11496	-2.78946126745485	-2.78946126745485\\
60.5	0.11862	-3.14291094950297	-3.14291094950297\\
60.5	0.12228	-3.44615380114335	-3.44615380114335\\
60.5	0.12594	-3.69918982237597	-3.69918982237597\\
60.5	0.1296	-3.90201901320083	-3.90201901320083\\
60.5	0.13326	-4.05464137361793	-4.05464137361793\\
60.5	0.13692	-4.15705690362725	-4.15705690362725\\
60.5	0.14058	-4.20926560322881	-4.20926560322881\\
60.5	0.14424	-4.21126747242261	-4.21126747242261\\
60.5	0.1479	-4.16306251120865	-4.16306251120865\\
60.5	0.15156	-4.06465071958693	-4.06465071958693\\
60.5	0.15522	-3.91603209755741	-3.91603209755741\\
60.5	0.15888	-3.71720664512018	-3.71720664512018\\
60.5	0.16254	-3.46817436227517	-3.46817436227517\\
60.5	0.1662	-3.1689352490224	-3.1689352490224\\
60.5	0.16986	-2.81948930536186	-2.81948930536186\\
60.5	0.17352	-2.41983653129356	-2.41983653129356\\
60.5	0.17718	-1.96997692681747	-1.96997692681747\\
60.5	0.18084	-1.46991049193368	-1.46991049193368\\
60.5	0.1845	-0.919637226642052	-0.919637226642052\\
60.5	0.18816	-0.319157130942727	-0.319157130942727\\
60.5	0.19182	0.331529795164386	0.331529795164386\\
60.5	0.19548	1.03242355167927	1.03242355167927\\
60.5	0.19914	1.7835241386019	1.7835241386019\\
60.5	0.2028	2.58483155593233	2.58483155593233\\
60.5	0.20646	3.43634580367053	3.43634580367053\\
60.5	0.21012	4.33806688181643	4.33806688181643\\
60.5	0.21378	5.28999479037014	5.28999479037014\\
60.5	0.21744	6.29212952933156	6.29212952933156\\
60.5	0.2211	7.34447109870081	7.34447109870081\\
60.5	0.22476	8.44701949847776	8.44701949847776\\
60.5	0.22842	9.59977472866254	9.59977472866254\\
60.5	0.23208	10.8027367892551	10.8027367892551\\
60.5	0.23574	12.0559056802553	12.0559056802553\\
60.5	0.2394	13.3592814016633	13.3592814016633\\
60.5	0.24306	14.7128639534792	14.7128639534792\\
60.5	0.24672	16.1166533357027	16.1166533357027\\
60.5	0.25038	17.5706495483341	17.5706495483341\\
60.5	0.25404	19.0748525913731	19.0748525913731\\
60.5	0.2577	20.62926246482	20.62926246482\\
60.5	0.26136	22.2338791686746	22.2338791686746\\
60.5	0.26502	23.888702702937	23.888702702937\\
60.5	0.26868	25.5937330676071	25.5937330676071\\
60.5	0.27234	27.348970262685	27.348970262685\\
60.5	0.276	29.1544142881707	29.1544142881707\\
60.875	0.093	0.493101871090193	0.493101871090193\\
60.875	0.09666	-0.176712562860313	-0.176712562860313\\
60.875	0.10032	-0.796320166403078	-0.796320166403078\\
60.875	0.10398	-1.36572093953809	-1.36572093953809\\
60.875	0.10764	-1.88491488226533	-1.88491488226533\\
60.875	0.1113	-2.35390199458478	-2.35390199458478\\
60.875	0.11496	-2.77268227649647	-2.77268227649647\\
60.875	0.11862	-3.14125572800043	-3.14125572800043\\
60.875	0.12228	-3.45962234909659	-3.45962234909659\\
60.875	0.12594	-3.72778213978499	-3.72778213978499\\
60.875	0.1296	-3.94573510006566	-3.94573510006566\\
60.875	0.13326	-4.11348122993857	-4.11348122993857\\
60.875	0.13692	-4.23102052940365	-4.23102052940365\\
60.875	0.14058	-4.29835299846105	-4.29835299846105\\
60.875	0.14424	-4.31547863711063	-4.31547863711063\\
60.875	0.1479	-4.28239744535249	-4.28239744535249\\
60.875	0.15156	-4.19910942318655	-4.19910942318655\\
60.875	0.15522	-4.06561457061284	-4.06561457061284\\
60.875	0.15888	-3.88191288763139	-3.88191288763139\\
60.875	0.16254	-3.6480043742422	-3.6480043742422\\
60.875	0.1662	-3.36388903044524	-3.36388903044524\\
60.875	0.16986	-3.02956685624048	-3.02956685624048\\
60.875	0.17352	-2.64503785162797	-2.64503785162797\\
60.875	0.17718	-2.21030201660772	-2.21030201660772\\
60.875	0.18084	-1.72535935117971	-1.72535935117971\\
60.875	0.1845	-1.19020985534389	-1.19020985534389\\
60.875	0.18816	-0.604853529100353	-0.604853529100353\\
60.875	0.19182	0.0307096275509764	0.0307096275509764\\
60.875	0.19548	0.716479614610051	0.716479614610051\\
60.875	0.19914	1.4524564320769	1.4524564320769\\
60.875	0.2028	2.23864007995154	2.23864007995154\\
60.875	0.20646	3.07503055823391	3.07503055823391\\
60.875	0.21012	3.96162786692402	3.96162786692402\\
60.875	0.21378	4.89843200602189	4.89843200602189\\
60.875	0.21744	5.88544297552758	5.88544297552758\\
60.875	0.2211	6.92266077544099	6.92266077544099\\
60.875	0.22476	8.01008540576215	8.01008540576215\\
60.875	0.22842	9.14771686649109	9.14771686649109\\
60.875	0.23208	10.3355551576278	10.3355551576278\\
60.875	0.23574	11.5736002791723	11.5736002791723\\
60.875	0.2394	12.8618522311245	12.8618522311245\\
60.875	0.24306	14.2003110134845	14.2003110134845\\
60.875	0.24672	15.5889766262523	15.5889766262523\\
60.875	0.25038	17.0278490694278	17.0278490694278\\
60.875	0.25404	18.5169283430111	18.5169283430111\\
60.875	0.2577	20.0562144470022	20.0562144470022\\
60.875	0.26136	21.645707381401	21.645707381401\\
60.875	0.26502	23.2854071462076	23.2854071462076\\
60.875	0.26868	24.9753137414219	24.9753137414219\\
60.875	0.27234	26.715427167044	26.715427167044\\
60.875	0.276	28.5057474230738	28.5057474230738\\
61.25	0.093	0.610490676339186	0.610490676339186\\
61.25	0.09666	-0.0744475270671323	-0.0744475270671323\\
61.25	0.10032	-0.70917890006568	-0.70917890006568\\
61.25	0.10398	-1.29370344265648	-1.29370344265648\\
61.25	0.10764	-1.8280211548395	-1.8280211548395\\
61.25	0.1113	-2.31213203661479	-2.31213203661479\\
61.25	0.11496	-2.74603608798229	-2.74603608798229\\
61.25	0.11862	-3.129733308942	-3.129733308942\\
61.25	0.12228	-3.46322369949398	-3.46322369949398\\
61.25	0.12594	-3.74650725963819	-3.74650725963819\\
61.25	0.1296	-3.97958398937465	-3.97958398937465\\
61.25	0.13326	-4.16245388870334	-4.16245388870334\\
61.25	0.13692	-4.29511695762426	-4.29511695762426\\
61.25	0.14058	-4.37757319613744	-4.37757319613744\\
61.25	0.14424	-4.40982260424283	-4.40982260424283\\
61.25	0.1479	-4.39186518194047	-4.39186518194047\\
61.25	0.15156	-4.32370092923035	-4.32370092923035\\
61.25	0.15522	-4.20532984611246	-4.20532984611246\\
61.25	0.15888	-4.03675193258682	-4.03675193258682\\
61.25	0.16254	-3.81796718865341	-3.81796718865341\\
61.25	0.1662	-3.54897561431223	-3.54897561431223\\
61.25	0.16986	-3.22977720956328	-3.22977720956328\\
61.25	0.17352	-2.86037197440658	-2.86037197440658\\
61.25	0.17718	-2.44075990884212	-2.44075990884212\\
61.25	0.18084	-1.97094101286989	-1.97094101286989\\
61.25	0.1845	-1.45091528648989	-1.45091528648989\\
61.25	0.18816	-0.88068272970213	-0.88068272970213\\
61.25	0.19182	-0.260243342506612	-0.260243342506612\\
61.25	0.19548	0.410402875096679	0.410402875096679\\
61.25	0.19914	1.13125592310772	1.13125592310772\\
61.25	0.2028	1.90231580152654	1.90231580152654\\
61.25	0.20646	2.72358251035313	2.72358251035313\\
61.25	0.21012	3.59505604958746	3.59505604958746\\
61.25	0.21378	4.51673641922955	4.51673641922955\\
61.25	0.21744	5.48862361927939	5.48862361927939\\
61.25	0.2211	6.51071764973702	6.51071764973702\\
61.25	0.22476	7.5830185106024	7.5830185106024\\
61.25	0.22842	8.70552620187556	8.70552620187556\\
61.25	0.23208	9.87824072355646	9.87824072355646\\
61.25	0.23574	11.1011620756452	11.1011620756452\\
61.25	0.2394	12.3742902581416	12.3742902581416\\
61.25	0.24306	13.6976252710458	13.6976252710458\\
61.25	0.24672	15.0711671143577	15.0711671143577\\
61.25	0.25038	16.4949157880775	16.4949157880775\\
61.25	0.25404	17.968871292205	17.968871292205\\
61.25	0.2577	19.4930336267402	19.4930336267402\\
61.25	0.26136	21.0674027916832	21.0674027916832\\
61.25	0.26502	22.691978787034	22.691978787034\\
61.25	0.26868	24.3667616127925	24.3667616127925\\
61.25	0.27234	26.0917512689588	26.0917512689588\\
61.25	0.276	27.8669477555329	27.8669477555329\\
61.625	0.093	0.737746679143886	0.737746679143886\\
61.625	0.09666	0.0376847062817554	0.0376847062817554\\
61.625	0.10032	-0.612170436172605	-0.612170436172605\\
61.625	0.10398	-1.21181874821918	-1.21181874821918\\
61.625	0.10764	-1.76126022985802	-1.76126022985802\\
61.625	0.1113	-2.26049488108909	-2.26049488108909\\
61.625	0.11496	-2.70952270191238	-2.70952270191238\\
61.625	0.11862	-3.1083436923279	-3.1083436923279\\
61.625	0.12228	-3.45695785233569	-3.45695785233569\\
61.625	0.12594	-3.75536518193572	-3.75536518193572\\
61.625	0.1296	-4.00356568112795	-4.00356568112795\\
61.625	0.13326	-4.20155934991246	-4.20155934991246\\
61.625	0.13692	-4.34934618828919	-4.34934618828919\\
61.625	0.14058	-4.44692619625813	-4.44692619625813\\
61.625	0.14424	-4.49429937381933	-4.49429937381933\\
61.625	0.1479	-4.49146572097278	-4.49146572097278\\
61.625	0.15156	-4.43842523771844	-4.43842523771844\\
61.625	0.15522	-4.33517792405636	-4.33517792405636\\
61.625	0.15888	-4.18172377998651	-4.18172377998651\\
61.625	0.16254	-3.97806280550888	-3.97806280550888\\
61.625	0.1662	-3.72419500062351	-3.72419500062351\\
61.625	0.16986	-3.42012036533038	-3.42012036533038\\
61.625	0.17352	-3.06583889962946	-3.06583889962946\\
61.625	0.17718	-2.66135060352081	-2.66135060352081\\
61.625	0.18084	-2.2066554770044	-2.2066554770044\\
61.625	0.1845	-1.70175352008017	-1.70175352008017\\
61.625	0.18816	-1.14664473274823	-1.14664473274823\\
61.625	0.19182	-0.541329115008523	-0.541329115008523\\
61.625	0.19548	0.114193333138985	0.114193333138985\\
61.625	0.19914	0.81992261169421	0.81992261169421\\
61.625	0.2028	1.57585872065722	1.57585872065722\\
61.625	0.20646	2.38200166002802	2.38200166002802\\
61.625	0.21012	3.23835142980657	3.23835142980657\\
61.625	0.21378	4.14490802999282	4.14490802999282\\
61.625	0.21744	5.10167146058689	5.10167146058689\\
61.625	0.2211	6.10864172158873	6.10864172158873\\
61.625	0.22476	7.16581881299832	7.16581881299832\\
61.625	0.22842	8.27320273481564	8.27320273481564\\
61.625	0.23208	9.43079348704076	9.43079348704076\\
61.625	0.23574	10.6385910696736	10.6385910696736\\
61.625	0.2394	11.8965954827143	11.8965954827143\\
61.625	0.24306	13.2048067261627	13.2048067261627\\
61.625	0.24672	14.5632248000189	14.5632248000189\\
61.625	0.25038	15.9718497042828	15.9718497042828\\
61.625	0.25404	17.4306814389545	17.4306814389545\\
61.625	0.2577	18.9397200040339	18.9397200040339\\
61.625	0.26136	20.4989653995212	20.4989653995212\\
61.625	0.26502	22.1084176254161	22.1084176254161\\
61.625	0.26868	23.7680766817188	23.7680766817188\\
61.625	0.27234	25.4779425684293	25.4779425684293\\
61.625	0.276	27.2380152855476	27.2380152855476\\
62	0.093	0.874869879504349	0.874869879504349\\
62	0.09666	0.159684137186435	0.159684137186435\\
62	0.10032	-0.505294774723737	-0.505294774723737\\
62	0.10398	-1.1200668562261	-1.1200668562261\\
62	0.10764	-1.68463210732075	-1.68463210732075\\
62	0.1113	-2.19899052800761	-2.19899052800761\\
62	0.11496	-2.6631421182867	-2.6631421182867\\
62	0.11862	-3.07708687815804	-3.07708687815804\\
62	0.12228	-3.44082480762161	-3.44082480762161\\
62	0.12594	-3.75435590667742	-3.75435590667742\\
62	0.1296	-4.01768017532547	-4.01768017532547\\
62	0.13326	-4.23079761356578	-4.23079761356578\\
62	0.13692	-4.39370822139827	-4.39370822139827\\
62	0.14058	-4.50641199882305	-4.50641199882305\\
62	0.14424	-4.56890894584006	-4.56890894584006\\
62	0.1479	-4.58119906244927	-4.58119906244927\\
62	0.15156	-4.54328234865077	-4.54328234865077\\
62	0.15522	-4.45515880444444	-4.45515880444444\\
62	0.15888	-4.3168284298304	-4.3168284298304\\
62	0.16254	-4.12829122480861	-4.12829122480861\\
62	0.1662	-3.889547189379	-3.889547189379\\
62	0.16986	-3.60059632354168	-3.60059632354168\\
62	0.17352	-3.26143862729658	-3.26143862729658\\
62	0.17718	-2.87207410064371	-2.87207410064371\\
62	0.18084	-2.43250274358308	-2.43250274358308\\
62	0.1845	-1.94272455611467	-1.94272455611467\\
62	0.18816	-1.40273953823854	-1.40273953823854\\
62	0.19182	-0.812547689954613	-0.812547689954613\\
62	0.19548	-0.172149011262917	-0.172149011262917\\
62	0.19914	0.518456497836524	0.518456497836524\\
62	0.2028	1.25926883734375	1.25926883734375\\
62	0.20646	2.05028800725871	2.05028800725871\\
62	0.21012	2.89151400758148	2.89151400758148\\
62	0.21378	3.78294683831194	3.78294683831194\\
62	0.21744	4.72458649945017	4.72458649945017\\
62	0.2211	5.71643299099622	5.71643299099622\\
62	0.22476	6.75848631294998	6.75848631294998\\
62	0.22842	7.85074646531157	7.85074646531157\\
62	0.23208	8.99321344808085	8.99321344808085\\
62	0.23574	10.1858872612579	10.1858872612579\\
62	0.2394	11.4287679048428	11.4287679048428\\
62	0.24306	12.7218553788354	12.7218553788354\\
62	0.24672	14.0651496832358	14.0651496832358\\
62	0.25038	15.4586508180439	15.4586508180439\\
62	0.25404	16.9023587832598	16.9023587832598\\
62	0.2577	18.3962735788834	18.3962735788834\\
62	0.26136	19.9403952049149	19.9403952049149\\
62	0.26502	21.534723661354	21.534723661354\\
62	0.26868	23.1792589482009	23.1792589482009\\
62	0.27234	24.8740010654556	24.8740010654556\\
62	0.276	26.6189500131181	26.6189500131181\\
62.375	0.093	1.0218602774206	1.0218602774206\\
62.375	0.09666	0.291550765646878	0.291550765646878\\
62.375	0.10032	-0.388551915719077	-0.388551915719077\\
62.375	0.10398	-1.01844776667728	-1.01844776667728\\
62.375	0.10764	-1.59813678722771	-1.59813678722771\\
62.375	0.1113	-2.12761897737038	-2.12761897737038\\
62.375	0.11496	-2.60689433710526	-2.60689433710526\\
62.375	0.11862	-3.03596286643238	-3.03596286643238\\
62.375	0.12228	-3.41482456535176	-3.41482456535176\\
62.375	0.12594	-3.74347943386338	-3.74347943386338\\
62.375	0.1296	-4.02192747196722	-4.02192747196722\\
62.375	0.13326	-4.25016867966332	-4.25016867966332\\
62.375	0.13692	-4.42820305695162	-4.42820305695162\\
62.375	0.14058	-4.55603060383221	-4.55603060383221\\
62.375	0.14424	-4.63365132030498	-4.63365132030498\\
62.375	0.1479	-4.66106520637003	-4.66106520637003\\
62.375	0.15156	-4.63827226202731	-4.63827226202731\\
62.375	0.15522	-4.56527248727682	-4.56527248727682\\
62.375	0.15888	-4.44206588211856	-4.44206588211856\\
62.375	0.16254	-4.26865244655256	-4.26865244655256\\
62.375	0.1662	-4.04503218057876	-4.04503218057876\\
62.375	0.16986	-3.77120508419722	-3.77120508419722\\
62.375	0.17352	-3.4471711574079	-3.4471711574079\\
62.375	0.17718	-3.07293040021084	-3.07293040021084\\
62.375	0.18084	-2.64848281260603	-2.64848281260603\\
62.375	0.1845	-2.17382839459343	-2.17382839459343\\
62.375	0.18816	-1.64896714617308	-1.64896714617308\\
62.375	0.19182	-1.07389906734497	-1.07389906734497\\
62.375	0.19548	-0.448624158109055	-0.448624158109055\\
62.375	0.19914	0.226857581534574	0.226857581534574\\
62.375	0.2028	0.95254615158602	0.95254615158602\\
62.375	0.20646	1.7284415520452	1.7284415520452\\
62.375	0.21012	2.55454378291212	2.55454378291212\\
62.375	0.21378	3.4308528441868	3.4308528441868\\
62.375	0.21744	4.35736873586924	4.35736873586924\\
62.375	0.2211	5.33409145795946	5.33409145795946\\
62.375	0.22476	6.36102101045749	6.36102101045749\\
62.375	0.22842	7.43815739336324	7.43815739336324\\
62.375	0.23208	8.56550060667674	8.56550060667674\\
62.375	0.23574	9.74305065039803	9.74305065039803\\
62.375	0.2394	10.9708075245271	10.9708075245271\\
62.375	0.24306	12.2487712290638	12.2487712290638\\
62.375	0.24672	13.5769417640084	13.5769417640084\\
62.375	0.25038	14.9553191293608	14.9553191293608\\
62.375	0.25404	16.3839033251208	16.3839033251208\\
62.375	0.2577	17.8626943512887	17.8626943512887\\
62.375	0.26136	19.3916922078643	19.3916922078643\\
62.375	0.26502	20.9708968948477	20.9708968948477\\
62.375	0.26868	22.6003084122388	22.6003084122388\\
62.375	0.27234	24.2799267600377	24.2799267600377\\
62.375	0.276	26.0097519382444	26.0097519382444\\
62.75	0.093	1.17871787289262	1.17871787289262\\
62.75	0.09666	0.433284591663085	0.433284591663085\\
62.75	0.10032	-0.261941859158682	-0.261941859158682\\
62.75	0.10398	-0.906961479572669	-0.906961479572669\\
62.75	0.10764	-1.50177426957888	-1.50177426957888\\
62.75	0.1113	-2.04638022917734	-2.04638022917734\\
62.75	0.11496	-2.54077935836803	-2.54077935836803\\
62.75	0.11862	-2.98497165715096	-2.98497165715096\\
62.75	0.12228	-3.37895712552613	-3.37895712552613\\
62.75	0.12594	-3.72273576349356	-3.72273576349356\\
62.75	0.1296	-4.0163075710532	-4.0163075710532\\
62.75	0.13326	-4.25967254820512	-4.25967254820512\\
62.75	0.13692	-4.45283069494923	-4.45283069494923\\
62.75	0.14058	-4.59578201128557	-4.59578201128557\\
62.75	0.14424	-4.68852649721416	-4.68852649721416\\
62.75	0.1479	-4.73106415273502	-4.73106415273502\\
62.75	0.15156	-4.72339497784808	-4.72339497784808\\
62.75	0.15522	-4.66551897255341	-4.66551897255341\\
62.75	0.15888	-4.55743613685093	-4.55743613685093\\
62.75	0.16254	-4.39914647074071	-4.39914647074071\\
62.75	0.1662	-4.19064997422275	-4.19064997422275\\
62.75	0.16986	-3.931946647297	-3.931946647297\\
62.75	0.17352	-3.62303648996352	-3.62303648996352\\
62.75	0.17718	-3.26391950222224	-3.26391950222224\\
62.75	0.18084	-2.85459568407321	-2.85459568407321\\
62.75	0.1845	-2.3950650355164	-2.3950650355164\\
62.75	0.18816	-1.88532755655186	-1.88532755655186\\
62.75	0.19182	-1.32538324717953	-1.32538324717953\\
62.75	0.19548	-0.71523210739943	-0.71523210739943\\
62.75	0.19914	-0.0548741372116126	-0.0548741372116126\\
62.75	0.2028	0.655690663384021	0.655690663384021\\
62.75	0.20646	1.41646229438741	1.41646229438741\\
62.75	0.21012	2.22744075579855	2.22744075579855\\
62.75	0.21378	3.08862604761745	3.08862604761745\\
62.75	0.21744	4.00001816984411	4.00001816984411\\
62.75	0.2211	4.96161712247849	4.96161712247849\\
62.75	0.22476	5.97342290552068	5.97342290552068\\
62.75	0.22842	7.03543551897064	7.03543551897064\\
62.75	0.23208	8.14765496282835	8.14765496282835\\
62.75	0.23574	9.31008123709387	9.31008123709387\\
62.75	0.2394	10.5227143417671	10.5227143417671\\
62.75	0.24306	11.7855542768481	11.7855542768481\\
62.75	0.24672	13.0986010423368	13.0986010423368\\
62.75	0.25038	14.4618546382334	14.4618546382334\\
62.75	0.25404	15.8753150645377	15.8753150645377\\
62.75	0.2577	17.3389823212497	17.3389823212497\\
62.75	0.26136	18.8528564083695	18.8528564083695\\
62.75	0.26502	20.4169373258971	20.4169373258971\\
62.75	0.26868	22.0312250738324	22.0312250738324\\
62.75	0.27234	23.6957196521756	23.6957196521756\\
62.75	0.276	25.4104210609264	25.4104210609264\\
63.125	0.093	1.34544266592038	1.34544266592038\\
63.125	0.09666	0.584885615235056	0.584885615235056\\
63.125	0.10032	-0.125464605042495	-0.125464605042495\\
63.125	0.10398	-0.785607994912294	-0.785607994912294\\
63.125	0.10764	-1.39554455437429	-1.39554455437429\\
63.125	0.1113	-1.95527428342859	-1.95527428342859\\
63.125	0.11496	-2.46479718207506	-2.46479718207506\\
63.125	0.11862	-2.92411325031378	-2.92411325031378\\
63.125	0.12228	-3.33322248814478	-3.33322248814478\\
63.125	0.12594	-3.69212489556797	-3.69212489556797\\
63.125	0.1296	-4.00082047258343	-4.00082047258343\\
63.125	0.13326	-4.25930921919112	-4.25930921919112\\
63.125	0.13692	-4.46759113539102	-4.46759113539102\\
63.125	0.14058	-4.6256662211832	-4.6256662211832\\
63.125	0.14424	-4.7335344765676	-4.7335344765676\\
63.125	0.1479	-4.79119590154424	-4.79119590154424\\
63.125	0.15156	-4.79865049611309	-4.79865049611309\\
63.125	0.15522	-4.75589826027423	-4.75589826027423\\
63.125	0.15888	-4.66293919402757	-4.66293919402757\\
63.125	0.16254	-4.51977329737316	-4.51977329737316\\
63.125	0.1662	-4.32640057031095	-4.32640057031095\\
63.125	0.16986	-4.08282101284101	-4.08282101284101\\
63.125	0.17352	-3.78903462496331	-3.78903462496331\\
63.125	0.17718	-3.44504140667785	-3.44504140667785\\
63.125	0.18084	-3.05084135798463	-3.05084135798463\\
63.125	0.1845	-2.6064344788836	-2.6064344788836\\
63.125	0.18816	-2.11182076937487	-2.11182076937487\\
63.125	0.19182	-1.56700022945836	-1.56700022945836\\
63.125	0.19548	-0.971972859134041	-0.971972859134041\\
63.125	0.19914	-0.326738658402007	-0.326738658402007\\
63.125	0.2028	0.368702372737786	0.368702372737786\\
63.125	0.20646	1.1143502342854	1.1143502342854\\
63.125	0.21012	1.91020492624075	1.91020492624075\\
63.125	0.21378	2.75626644860381	2.75626644860381\\
63.125	0.21744	3.65253480137469	3.65253480137469\\
63.125	0.2211	4.59900998455328	4.59900998455328\\
63.125	0.22476	5.59569199813969	5.59569199813969\\
63.125	0.22842	6.64258084213387	6.64258084213387\\
63.125	0.23208	7.73967651653574	7.73967651653574\\
63.125	0.23574	8.88697902134541	8.88697902134541\\
63.125	0.2394	10.0844883565629	10.0844883565629\\
63.125	0.24306	11.3322045221881	11.3322045221881\\
63.125	0.24672	12.630127518221	12.630127518221\\
63.125	0.25038	13.9782573446618	13.9782573446618\\
63.125	0.25404	15.3765940015103	15.3765940015103\\
63.125	0.2577	16.8251374887665	16.8251374887665\\
63.125	0.26136	18.3238878064305	18.3238878064305\\
63.125	0.26502	19.8728449545023	19.8728449545023\\
63.125	0.26868	21.4720089329819	21.4720089329819\\
63.125	0.27234	23.1213797418692	23.1213797418692\\
63.125	0.276	24.8209573811642	24.8209573811642\\
63.5	0.093	1.52203465650392	1.52203465650392\\
63.5	0.09666	0.74635383636279	0.74635383636279\\
63.5	0.10032	0.0208798466294269	0.0208798466294269\\
63.5	0.10398	-0.654387312696127	-0.654387312696127\\
63.5	0.10764	-1.27944764161397	-1.27944764161397\\
63.5	0.1113	-1.85430114012404	-1.85430114012404\\
63.5	0.11496	-2.37894780822633	-2.37894780822633\\
63.5	0.11862	-2.85338764592086	-2.85338764592086\\
63.5	0.12228	-3.27762065320765	-3.27762065320765\\
63.5	0.12594	-3.65164683008662	-3.65164683008662\\
63.5	0.1296	-3.97546617655789	-3.97546617655789\\
63.5	0.13326	-4.2490786926214	-4.2490786926214\\
63.5	0.13692	-4.47248437827707	-4.47248437827707\\
63.5	0.14058	-4.64568323352507	-4.64568323352507\\
63.5	0.14424	-4.76867525836525	-4.76867525836525\\
63.5	0.1479	-4.8414604527977	-4.8414604527977\\
63.5	0.15156	-4.86403881682237	-4.86403881682237\\
63.5	0.15522	-4.83641035043926	-4.83641035043926\\
63.5	0.15888	-4.75857505364841	-4.75857505364841\\
63.5	0.16254	-4.63053292644981	-4.63053292644981\\
63.5	0.1662	-4.45228396884342	-4.45228396884342\\
63.5	0.16986	-4.22382818082926	-4.22382818082926\\
63.5	0.17352	-3.94516556240738	-3.94516556240738\\
63.5	0.17718	-3.6162961135777	-3.6162961135777\\
63.5	0.18084	-3.23721983434026	-3.23721983434026\\
63.5	0.1845	-2.80793672469507	-2.80793672469507\\
63.5	0.18816	-2.32844678464213	-2.32844678464213\\
63.5	0.19182	-1.79875001418139	-1.79875001418139\\
63.5	0.19548	-1.21884641331289	-1.21884641331289\\
63.5	0.19914	-0.588735982036667	-0.588735982036667\\
63.5	0.2028	0.0915812796473432	0.0915812796473432\\
63.5	0.20646	0.82210537173917	0.82210537173917\\
63.5	0.21012	1.60283629423868	1.60283629423868\\
63.5	0.21378	2.43377404714596	2.43377404714596\\
63.5	0.21744	3.31491863046105	3.31491863046105\\
63.5	0.2211	4.24627004418386	4.24627004418386\\
63.5	0.22476	5.22782828831443	5.22782828831443\\
63.5	0.22842	6.25959336285283	6.25959336285283\\
63.5	0.23208	7.34156526779891	7.34156526779891\\
63.5	0.23574	8.4737440031528	8.4737440031528\\
63.5	0.2394	9.65612956891442	9.65612956891442\\
63.5	0.24306	10.8887219650838	10.8887219650838\\
63.5	0.24672	12.171521191661	12.171521191661\\
63.5	0.25038	13.5045272486459	13.5045272486459\\
63.5	0.25404	14.8877401360386	14.8877401360386\\
63.5	0.2577	16.3211598538391	16.3211598538391\\
63.5	0.26136	17.8047864020473	17.8047864020473\\
63.5	0.26502	19.3386197806633	19.3386197806633\\
63.5	0.26868	20.922659989687	20.922659989687\\
63.5	0.27234	22.5569070291185	22.5569070291185\\
63.5	0.276	24.2413608989578	24.2413608989578\\
63.875	0.093	1.70849384464323	1.70849384464323\\
63.875	0.09666	0.917689255046316	0.917689255046316\\
63.875	0.10032	0.177091495857141	0.177091495857141\\
63.875	0.10398	-0.513299432924224	-0.513299432924224\\
63.875	0.10764	-1.15348353129785	-1.15348353129785\\
63.875	0.1113	-1.74346079926374	-1.74346079926374\\
63.875	0.11496	-2.28323123682183	-2.28323123682183\\
63.875	0.11862	-2.77279484397215	-2.77279484397215\\
63.875	0.12228	-3.21215162071472	-3.21215162071472\\
63.875	0.12594	-3.60130156704953	-3.60130156704953\\
63.875	0.1296	-3.94024468297658	-3.94024468297658\\
63.875	0.13326	-4.22898096849588	-4.22898096849588\\
63.875	0.13692	-4.46751042360739	-4.46751042360739\\
63.875	0.14058	-4.65583304831114	-4.65583304831114\\
63.875	0.14424	-4.79394884260714	-4.79394884260714\\
63.875	0.1479	-4.88185780649538	-4.88185780649538\\
63.875	0.15156	-4.91955993997585	-4.91955993997585\\
63.875	0.15522	-4.90705524304855	-4.90705524304855\\
63.875	0.15888	-4.84434371571349	-4.84434371571349\\
63.875	0.16254	-4.73142535797067	-4.73142535797067\\
63.875	0.1662	-4.56830016982009	-4.56830016982009\\
63.875	0.16986	-4.35496815126174	-4.35496815126174\\
63.875	0.17352	-4.09142930229564	-4.09142930229564\\
63.875	0.17718	-3.77768362292178	-3.77768362292178\\
63.875	0.18084	-3.41373111314015	-3.41373111314015\\
63.875	0.1845	-2.99957177295074	-2.99957177295074\\
63.875	0.18816	-2.53520560235361	-2.53520560235361\\
63.875	0.19182	-2.02063260134869	-2.02063260134869\\
63.875	0.19548	-1.455852769936	-1.455852769936\\
63.875	0.19914	-0.840866108115534	-0.840866108115534\\
63.875	0.2028	-0.175672615887308	-0.175672615887308\\
63.875	0.20646	0.539727706748678	0.539727706748678\\
63.875	0.21012	1.30533485979241	1.30533485979241\\
63.875	0.21378	2.1211488432439	2.1211488432439\\
63.875	0.21744	2.98716965710315	2.98716965710315\\
63.875	0.2211	3.90339730137018	3.90339730137018\\
63.875	0.22476	4.86983177604496	4.86983177604496\\
63.875	0.22842	5.88647308112752	5.88647308112752\\
63.875	0.23208	6.95332121661782	6.95332121661782\\
63.875	0.23574	8.07037618251593	8.07037618251593\\
63.875	0.2394	9.23763797882177	9.23763797882177\\
63.875	0.24306	10.4551066055354	10.4551066055354\\
63.875	0.24672	11.7227820626567	11.7227820626567\\
63.875	0.25038	13.0406643501859	13.0406643501859\\
63.875	0.25404	14.4087534681228	14.4087534681228\\
63.875	0.2577	15.8270494164674	15.8270494164674\\
63.875	0.26136	17.2955521952198	17.2955521952198\\
63.875	0.26502	18.81426180438	18.81426180438\\
63.875	0.26868	20.3831782439479	20.3831782439479\\
63.875	0.27234	22.0023015139236	22.0023015139236\\
63.875	0.276	23.6716316143071	23.6716316143071\\
64.25	0.093	1.90482023033825	1.90482023033825\\
64.25	0.09666	1.09889187128552	1.09889187128552\\
64.25	0.10032	0.34317034264059	0.34317034264059\\
64.25	0.10398	-0.362344355596587	-0.362344355596587\\
64.25	0.10764	-1.01765222342602	-1.01765222342602\\
64.25	0.1113	-1.62275326084769	-1.62275326084769\\
64.25	0.11496	-2.17764746786158	-2.17764746786158\\
64.25	0.11862	-2.6823348444677	-2.6823348444677\\
64.25	0.12228	-3.13681539066609	-3.13681539066609\\
64.25	0.12594	-3.54108910645671	-3.54108910645671\\
64.25	0.1296	-3.89515599183954	-3.89515599183954\\
64.25	0.13326	-4.19901604681462	-4.19901604681462\\
64.25	0.13692	-4.45266927138195	-4.45266927138195\\
64.25	0.14058	-4.65611566554151	-4.65611566554151\\
64.25	0.14424	-4.80935522929329	-4.80935522929329\\
64.25	0.1479	-4.91238796263734	-4.91238796263734\\
64.25	0.15156	-4.96521386557362	-4.96521386557362\\
64.25	0.15522	-4.96783293810211	-4.96783293810211\\
64.25	0.15888	-4.92024518022286	-4.92024518022286\\
64.25	0.16254	-4.82245059193585	-4.82245059193585\\
64.25	0.1662	-4.67444917324106	-4.67444917324106\\
64.25	0.16986	-4.47624092413849	-4.47624092413849\\
64.25	0.17352	-4.22782584462821	-4.22782584462821\\
64.25	0.17718	-3.92920393471015	-3.92920393471015\\
64.25	0.18084	-3.58037519438431	-3.58037519438431\\
64.25	0.1845	-3.18133962365069	-3.18133962365069\\
64.25	0.18816	-2.73209722250937	-2.73209722250937\\
64.25	0.19182	-2.23264799096023	-2.23264799096023\\
64.25	0.19548	-1.68299192900335	-1.68299192900335\\
64.25	0.19914	-1.0831290366387	-1.0831290366387\\
64.25	0.2028	-0.433059313866281	-0.433059313866281\\
64.25	0.20646	0.267217239313922	0.267217239313922\\
64.25	0.21012	1.01770062290187	1.01770062290187\\
64.25	0.21378	1.81839083689752	1.81839083689752\\
64.25	0.21744	2.66928788130099	2.66928788130099\\
64.25	0.2211	3.57039175611223	3.57039175611223\\
64.25	0.22476	4.52170246133123	4.52170246133123\\
64.25	0.22842	5.52321999695795	5.52321999695795\\
64.25	0.23208	6.57494436299247	6.57494436299247\\
64.25	0.23574	7.67687555943479	7.67687555943479\\
64.25	0.2394	8.82901358628479	8.82901358628479\\
64.25	0.24306	10.0313584435426	10.0313584435426\\
64.25	0.24672	11.2839101312082	11.2839101312082\\
64.25	0.25038	12.5866686492815	12.5866686492815\\
64.25	0.25404	13.9396339977626	13.9396339977626\\
64.25	0.2577	15.3428061766515	15.3428061766515\\
64.25	0.26136	16.7961851859481	16.7961851859481\\
64.25	0.26502	18.2997710256525	18.2997710256525\\
64.25	0.26868	19.8535636957646	19.8535636957646\\
64.25	0.27234	21.4575631962845	21.4575631962845\\
64.25	0.276	23.1117695272122	23.1117695272122\\
64.625	0.093	2.11101381358911	2.11101381358911\\
64.625	0.09666	1.2899616850806	1.2899616850806\\
64.625	0.10032	0.51911638697986	0.51911638697986\\
64.625	0.10398	-0.201522080713158	-0.201522080713158\\
64.625	0.10764	-0.871953717998348	-0.871953717998348\\
64.625	0.1113	-1.49217852487583	-1.49217852487583\\
64.625	0.11496	-2.06219650134553	-2.06219650134553\\
64.625	0.11862	-2.58200764740746	-2.58200764740746\\
64.625	0.12228	-3.0516119630616	-3.0516119630616\\
64.625	0.12594	-3.47100944830804	-3.47100944830804\\
64.625	0.1296	-3.84020010314666	-3.84020010314666\\
64.625	0.13326	-4.15918392757757	-4.15918392757757\\
64.625	0.13692	-4.42796092160066	-4.42796092160066\\
64.625	0.14058	-4.64653108521603	-4.64653108521603\\
64.625	0.14424	-4.81489441842362	-4.81489441842362\\
64.625	0.1479	-4.93305092122345	-4.93305092122345\\
64.625	0.15156	-5.00100059361552	-5.00100059361552\\
64.625	0.15522	-5.01874343559982	-5.01874343559982\\
64.625	0.15888	-4.98627944717638	-4.98627944717638\\
64.625	0.16254	-4.90360862834516	-4.90360862834516\\
64.625	0.1662	-4.77073097910618	-4.77073097910618\\
64.625	0.16986	-4.58764649945942	-4.58764649945942\\
64.625	0.17352	-4.35435518940492	-4.35435518940492\\
64.625	0.17718	-4.07085704894265	-4.07085704894265\\
64.625	0.18084	-3.73715207807262	-3.73715207807262\\
64.625	0.1845	-3.35324027679481	-3.35324027679481\\
64.625	0.18816	-2.91912164510927	-2.91912164510927\\
64.625	0.19182	-2.43479618301595	-2.43479618301595\\
64.625	0.19548	-1.90026389051485	-1.90026389051485\\
64.625	0.19914	-1.31552476760601	-1.31552476760601\\
64.625	0.2028	-0.680578814289404	-0.680578814289404\\
64.625	0.20646	0.00457396943501465	0.00457396943501465\\
64.625	0.21012	0.739933583567122	0.739933583567122\\
64.625	0.21378	1.52550002810705	1.52550002810705\\
64.625	0.21744	2.36127330305467	2.36127330305467\\
64.625	0.2211	3.24725340841013	3.24725340841013\\
64.625	0.22476	4.18344034417329	4.18344034417329\\
64.625	0.22842	5.16983411034428	5.16983411034428\\
64.625	0.23208	6.20643470692296	6.20643470692296\\
64.625	0.23574	7.29324213390944	7.29324213390944\\
64.625	0.2394	8.43025639130371	8.43025639130371\\
64.625	0.24306	9.61747747910569	9.61747747910569\\
64.625	0.24672	10.8549053973155	10.8549053973155\\
64.625	0.25038	12.142540145933	12.142540145933\\
64.625	0.25404	13.4803817249583	13.4803817249583\\
64.625	0.2577	14.8684301343914	14.8684301343914\\
64.625	0.26136	16.3066853742322	16.3066853742322\\
64.625	0.26502	17.7951474444808	17.7951474444808\\
64.625	0.26868	19.3338163451371	19.3338163451371\\
64.625	0.27234	20.9226920762012	20.9226920762012\\
64.625	0.276	22.5617746376731	22.5617746376731\\
65	0.093	2.32707459439571	2.32707459439571\\
65	0.09666	1.49089869643139	1.49089869643139\\
65	0.10032	0.704929628874837	0.704929628874837\\
65	0.10398	-0.0308326082739647	-0.0308326082739647\\
65	0.10764	-0.716388015014967	-0.716388015014967\\
65	0.1113	-1.35173659134824	-1.35173659134824\\
65	0.11496	-1.93687833727371	-1.93687833727371\\
65	0.11862	-2.47181325279146	-2.47181325279146\\
65	0.12228	-2.95654133790141	-2.95654133790141\\
65	0.12594	-3.3910625926036	-3.3910625926036\\
65	0.1296	-3.77537701689806	-3.77537701689806\\
65	0.13326	-4.10948461078476	-4.10948461078476\\
65	0.13692	-4.39338537426369	-4.39338537426369\\
65	0.14058	-4.62707930733485	-4.62707930733485\\
65	0.14424	-4.81056640999824	-4.81056640999824\\
65	0.1479	-4.94384668225386	-4.94384668225386\\
65	0.15156	-5.02692012410171	-5.02692012410171\\
65	0.15522	-5.05978673554183	-5.05978673554183\\
65	0.15888	-5.04244651657417	-5.04244651657417\\
65	0.16254	-4.97489946719876	-4.97489946719876\\
65	0.1662	-4.85714558741559	-4.85714558741559\\
65	0.16986	-4.68918487722462	-4.68918487722462\\
65	0.17352	-4.4710173366259	-4.4710173366259\\
65	0.17718	-4.20264296561944	-4.20264296561944\\
65	0.18084	-3.88406176420522	-3.88406176420522\\
65	0.1845	-3.51527373238322	-3.51527373238322\\
65	0.18816	-3.09627887015347	-3.09627887015347\\
65	0.19182	-2.62707717751593	-2.62707717751593\\
65	0.19548	-2.10766865447064	-2.10766865447064\\
65	0.19914	-1.53805330101761	-1.53805330101761\\
65	0.2028	-0.918231117156765	-0.918231117156765\\
65	0.20646	-0.248202102888186	-0.248202102888186\\
65	0.21012	0.472033741788138	0.472033741788138\\
65	0.21378	1.24247641687222	1.24247641687222\\
65	0.21744	2.06312592236412	2.06312592236412\\
65	0.2211	2.93398225826374	2.93398225826374\\
65	0.22476	3.85504542457112	3.85504542457112\\
65	0.22842	4.82631542128627	4.82631542128627\\
65	0.23208	5.84779224840916	5.84779224840916\\
65	0.23574	6.91947590593986	6.91947590593986\\
65	0.2394	8.04136639387835	8.04136639387835\\
65	0.24306	9.21346371222454	9.21346371222454\\
65	0.24672	10.4357678609785	10.4357678609785\\
65	0.25038	11.7082788401402	11.7082788401402\\
65	0.25404	13.0309966497097	13.0309966497097\\
65	0.2577	14.403921289687	14.403921289687\\
65	0.26136	15.827052760072	15.827052760072\\
65	0.26502	17.3003910608648	17.3003910608648\\
65	0.26868	18.8239361920653	18.8239361920653\\
65	0.27234	20.3976881536736	20.3976881536736\\
65	0.276	22.0216469456897	22.0216469456897\\
65.375	0.093	2.55300257275805	2.55300257275805\\
65.375	0.09666	1.70170290533794	1.70170290533794\\
65.375	0.10032	0.900610068325577	0.900610068325577\\
65.375	0.10398	0.149724061720963	0.149724061720963\\
65.375	0.10764	-0.550955114475851	-0.550955114475851\\
65.375	0.1113	-1.2014274602649	-1.2014274602649\\
65.375	0.11496	-1.80169297564619	-1.80169297564619\\
65.375	0.11862	-2.35175166061972	-2.35175166061972\\
65.375	0.12228	-2.85160351518549	-2.85160351518549\\
65.375	0.12594	-3.30124853934349	-3.30124853934349\\
65.375	0.1296	-3.70068673309373	-3.70068673309373\\
65.375	0.13326	-4.04991809643622	-4.04991809643622\\
65.375	0.13692	-4.34894262937092	-4.34894262937092\\
65.375	0.14058	-4.59776033189792	-4.59776033189792\\
65.375	0.14424	-4.79637120401711	-4.79637120401711\\
65.375	0.1479	-4.94477524572854	-4.94477524572854\\
65.375	0.15156	-5.0429724570322	-5.0429724570322\\
65.375	0.15522	-5.0909628379281	-5.0909628379281\\
65.375	0.15888	-5.08874638841625	-5.08874638841625\\
65.375	0.16254	-5.03632310849662	-5.03632310849662\\
65.375	0.1662	-4.93369299816924	-4.93369299816924\\
65.375	0.16986	-4.78085605743411	-4.78085605743411\\
65.375	0.17352	-4.5778122862912	-4.5778122862912\\
65.375	0.17718	-4.32456168474052	-4.32456168474052\\
65.375	0.18084	-4.02110425278209	-4.02110425278209\\
65.375	0.1845	-3.66743999041587	-3.66743999041587\\
65.375	0.18816	-3.26356889764193	-3.26356889764193\\
65.375	0.19182	-2.8094909744602	-2.8094909744602\\
65.375	0.19548	-2.3052062208707	-2.3052062208707\\
65.375	0.19914	-1.75071463687345	-1.75071463687345\\
65.375	0.2028	-1.14601622246845	-1.14601622246845\\
65.375	0.20646	-0.491110977655651	-0.491110977655651\\
65.375	0.21012	0.214001097564889	0.214001097564889\\
65.375	0.21378	0.969320003193189	0.969320003193189\\
65.375	0.21744	1.77484573922925	1.77484573922925\\
65.375	0.2211	2.63057830567308	2.63057830567308\\
65.375	0.22476	3.53651770252468	3.53651770252468\\
65.375	0.22842	4.49266392978404	4.49266392978404\\
65.375	0.23208	5.49901698745116	5.49901698745116\\
65.375	0.23574	6.55557687552601	6.55557687552601\\
65.375	0.2394	7.66234359400866	7.66234359400866\\
65.375	0.24306	8.81931714289907	8.81931714289907\\
65.375	0.24672	10.0264975221972	10.0264975221972\\
65.375	0.25038	11.2838847319032	11.2838847319032\\
65.375	0.25404	12.5914787720169	12.5914787720169\\
65.375	0.2577	13.9492796425384	13.9492796425384\\
65.375	0.26136	15.3572873434676	15.3572873434676\\
65.375	0.26502	16.8155018748046	16.8155018748046\\
65.375	0.26868	18.3239232365493	18.3239232365493\\
65.375	0.27234	19.8825514287018	19.8825514287018\\
65.375	0.276	21.4913864512621	21.4913864512621\\
65.75	0.093	2.78879774867615	2.78879774867615\\
65.75	0.09666	1.92237431180023	1.92237431180023\\
65.75	0.10032	1.10615770533211	1.10615770533211\\
65.75	0.10398	0.340147929271684	0.340147929271684\\
65.75	0.10764	-0.375655016380914	-0.375655016380914\\
65.75	0.1113	-1.04125113162578	-1.04125113162578\\
65.75	0.11496	-1.65664041646288	-1.65664041646288\\
65.75	0.11862	-2.22182287089219	-2.22182287089219\\
65.75	0.12228	-2.73679849491377	-2.73679849491377\\
65.75	0.12594	-3.20156728852758	-3.20156728852758\\
65.75	0.1296	-3.61612925173361	-3.61612925173361\\
65.75	0.13326	-3.98048438453191	-3.98048438453191\\
65.75	0.13692	-4.29463268692243	-4.29463268692243\\
65.75	0.14058	-4.55857415890518	-4.55857415890518\\
65.75	0.14424	-4.77230880048018	-4.77230880048018\\
65.75	0.1479	-4.93583661164739	-4.93583661164739\\
65.75	0.15156	-5.04915759240686	-5.04915759240686\\
65.75	0.15522	-5.11227174275857	-5.11227174275857\\
65.75	0.15888	-5.12517906270254	-5.12517906270254\\
65.75	0.16254	-5.0878795522387	-5.0878795522387\\
65.75	0.1662	-5.00037321136709	-5.00037321136709\\
65.75	0.16986	-4.86266004008777	-4.86266004008777\\
65.75	0.17352	-4.67474003840065	-4.67474003840065\\
65.75	0.17718	-4.43661320630579	-4.43661320630579\\
65.75	0.18084	-4.14827954380316	-4.14827954380316\\
65.75	0.1845	-3.80973905089273	-3.80973905089273\\
65.75	0.18816	-3.4209917275746	-3.4209917275746\\
65.75	0.19182	-2.98203757384866	-2.98203757384866\\
65.75	0.19548	-2.49287658971497	-2.49287658971497\\
65.75	0.19914	-1.95350877517353	-1.95350877517353\\
65.75	0.2028	-1.36393413022428	-1.36393413022428\\
65.75	0.20646	-0.724152654867325	-0.724152654867325\\
65.75	0.21012	-0.0341643491025678	-0.0341643491025678\\
65.75	0.21378	0.706030787069949	0.706030787069949\\
65.75	0.21744	1.49643275365017	1.49643275365017\\
65.75	0.2211	2.33704155063822	2.33704155063822\\
65.75	0.22476	3.22785717803403	3.22785717803403\\
65.75	0.22842	4.16887963583756	4.16887963583756\\
65.75	0.23208	5.16010892404888	5.16010892404888\\
65.75	0.23574	6.20154504266796	6.20154504266796\\
65.75	0.2394	7.29318799169482	7.29318799169482\\
65.75	0.24306	8.43503777112939	8.43503777112939\\
65.75	0.24672	9.62709438097177	9.62709438097177\\
65.75	0.25038	10.8693578212219	10.8693578212219\\
65.75	0.25404	12.1618280918798	12.1618280918798\\
65.75	0.2577	13.5045051929455	13.5045051929455\\
65.75	0.26136	14.8973891244189	14.8973891244189\\
65.75	0.26502	16.3404798863001	16.3404798863001\\
65.75	0.26868	17.833777478589	17.833777478589\\
65.75	0.27234	19.3772819012857	19.3772819012857\\
65.75	0.276	20.9709931543902	20.9709931543902\\
66.125	0.093	3.03446012215004	3.03446012215004\\
66.125	0.09666	2.15291291581831	2.15291291581831\\
66.125	0.10032	1.32157253989438	1.32157253989438\\
66.125	0.10398	0.540438994378196	0.540438994378196\\
66.125	0.10764	-0.190487720730243	-0.190487720730243\\
66.125	0.1113	-0.871207605430918	-0.871207605430918\\
66.125	0.11496	-1.50172065972377	-1.50172065972377\\
66.125	0.11862	-2.08202688360893	-2.08202688360893\\
66.125	0.12228	-2.61212627708626	-2.61212627708626\\
66.125	0.12594	-3.09201884015589	-3.09201884015589\\
66.125	0.1296	-3.52170457281773	-3.52170457281773\\
66.125	0.13326	-3.9011834750718	-3.9011834750718\\
66.125	0.13692	-4.23045554691814	-4.23045554691814\\
66.125	0.14058	-4.5095207883567	-4.5095207883567\\
66.125	0.14424	-4.73837919938748	-4.73837919938748\\
66.125	0.1479	-4.91703078001051	-4.91703078001051\\
66.125	0.15156	-5.04547553022579	-5.04547553022579\\
66.125	0.15522	-5.12371345003329	-5.12371345003329\\
66.125	0.15888	-5.15174453943303	-5.15174453943303\\
66.125	0.16254	-5.12956879842501	-5.12956879842501\\
66.125	0.1662	-5.05718622700924	-5.05718622700924\\
66.125	0.16986	-4.93459682518565	-4.93459682518565\\
66.125	0.17352	-4.76180059295437	-4.76180059295437\\
66.125	0.17718	-4.53879753031529	-4.53879753031529\\
66.125	0.18084	-4.26558763726845	-4.26558763726845\\
66.125	0.1845	-3.94217091381385	-3.94217091381385\\
66.125	0.18816	-3.56854735995151	-3.56854735995151\\
66.125	0.19182	-3.14471697568138	-3.14471697568138\\
66.125	0.19548	-2.67067976100347	-2.67067976100347\\
66.125	0.19914	-2.14643571591782	-2.14643571591782\\
66.125	0.2028	-1.57198484042438	-1.57198484042438\\
66.125	0.20646	-0.947327134523206	-0.947327134523206\\
66.125	0.21012	-0.27246259821429	-0.27246259821429\\
66.125	0.21378	0.452608768502444	0.452608768502444\\
66.125	0.21744	1.22788696562688	1.22788696562688\\
66.125	0.2211	2.05337199315909	2.05337199315909\\
66.125	0.22476	2.92906385109912	2.92906385109912\\
66.125	0.22842	3.85496253944686	3.85496253944686\\
66.125	0.23208	4.83106805820235	4.83106805820235\\
66.125	0.23574	5.85738040736564	5.85738040736564\\
66.125	0.2394	6.93389958693672	6.93389958693672\\
66.125	0.24306	8.0606255969155	8.0606255969155\\
66.125	0.24672	9.2375584373021	9.2375584373021\\
66.125	0.25038	10.4646981080964	10.4646981080964\\
66.125	0.25404	11.7420446092985	11.7420446092985\\
66.125	0.2577	13.0695979409084	13.0695979409084\\
66.125	0.26136	14.447358102926	14.447358102926\\
66.125	0.26502	15.8753250953514	15.8753250953514\\
66.125	0.26868	17.3534989181845	17.3534989181845\\
66.125	0.27234	18.8818795714254	18.8818795714254\\
66.125	0.276	20.4604670550741	20.4604670550741\\
66.5	0.093	3.2899896931797	3.2899896931797\\
66.5	0.09666	2.39331871739218	2.39331871739218\\
66.5	0.10032	1.54685457201244	1.54685457201244\\
66.5	0.10398	0.750597257040443	0.750597257040443\\
66.5	0.10764	0.00454677247622115	0.00454677247622115\\
66.5	0.1113	-0.691296881680238	-0.691296881680238\\
66.5	0.11496	-1.33693370542893	-1.33693370542893\\
66.5	0.11862	-1.93236369876987	-1.93236369876987\\
66.5	0.12228	-2.47758686170305	-2.47758686170305\\
66.5	0.12594	-2.97260319422845	-2.97260319422845\\
66.5	0.1296	-3.41741269634608	-3.41741269634608\\
66.5	0.13326	-3.81201536805597	-3.81201536805597\\
66.5	0.13692	-4.15641120935808	-4.15641120935808\\
66.5	0.14058	-4.45060022025246	-4.45060022025246\\
66.5	0.14424	-4.69458240073902	-4.69458240073902\\
66.5	0.1479	-4.88835775081786	-4.88835775081786\\
66.5	0.15156	-5.03192627048893	-5.03192627048893\\
66.5	0.15522	-5.12528795975221	-5.12528795975221\\
66.5	0.15888	-5.16844281860777	-5.16844281860777\\
66.5	0.16254	-5.16139084705555	-5.16139084705555\\
66.5	0.1662	-5.10413204509557	-5.10413204509557\\
66.5	0.16986	-4.99666641272779	-4.99666641272779\\
66.5	0.17352	-4.83899394995229	-4.83899394995229\\
66.5	0.17718	-4.63111465676902	-4.63111465676902\\
66.5	0.18084	-4.37302853317799	-4.37302853317799\\
66.5	0.1845	-4.06473557917919	-4.06473557917919\\
66.5	0.18816	-3.70623579477265	-3.70623579477265\\
66.5	0.19182	-3.2975291799583	-3.2975291799583\\
66.5	0.19548	-2.83861573473621	-2.83861573473621\\
66.5	0.19914	-2.32949545910637	-2.32949545910637\\
66.5	0.2028	-1.77016835306874	-1.77016835306874\\
66.5	0.20646	-1.16063441662335	-1.16063441662335\\
66.5	0.21012	-0.500893649770219	-0.500893649770219\\
66.5	0.21378	0.209053947490673	0.209053947490673\\
66.5	0.21744	0.969208375159326	0.969208375159326\\
66.5	0.2211	1.77956963323575	1.77956963323575\\
66.5	0.22476	2.64013772171994	2.64013772171994\\
66.5	0.22842	3.55091264061195	3.55091264061195\\
66.5	0.23208	4.51189438991166	4.51189438991166\\
66.5	0.23574	5.52308296961911	5.52308296961911\\
66.5	0.2394	6.58447837973435	6.58447837973435\\
66.5	0.24306	7.6960806202574	7.6960806202574\\
66.5	0.24672	8.85788969118816	8.85788969118816\\
66.5	0.25038	10.0699055925267	10.0699055925267\\
66.5	0.25404	11.3321283242729	11.3321283242729\\
66.5	0.2577	12.6445578864271	12.6445578864271\\
66.5	0.26136	14.0071942789889	14.0071942789889\\
66.5	0.26502	15.4200375019585	15.4200375019585\\
66.5	0.26868	16.8830875553358	16.8830875553358\\
66.5	0.27234	18.3963444391209	18.3963444391209\\
66.5	0.276	19.9598081533138	19.9598081533138\\
66.875	0.093	3.55538646176509	3.55538646176509\\
66.875	0.09666	2.64359171652176	2.64359171652176\\
66.875	0.10032	1.78200380168623	1.78200380168623\\
66.875	0.10398	0.970622717258454	0.970622717258454\\
66.875	0.10764	0.20944846323842	0.20944846323842\\
66.875	0.1113	-0.501518960373851	-0.501518960373851\\
66.875	0.11496	-1.16227955357833	-1.16227955357833\\
66.875	0.11862	-1.77283331637505	-1.77283331637505\\
66.875	0.12228	-2.33318024876401	-2.33318024876401\\
66.875	0.12594	-2.8433203507452	-2.8433203507452\\
66.875	0.1296	-3.30325362231866	-3.30325362231866\\
66.875	0.13326	-3.71298006348437	-3.71298006348437\\
66.875	0.13692	-4.07249967424227	-4.07249967424227\\
66.875	0.14058	-4.38181245459243	-4.38181245459243\\
66.875	0.14424	-4.64091840453483	-4.64091840453483\\
66.875	0.1479	-4.84981752406945	-4.84981752406945\\
66.875	0.15156	-5.00850981319631	-5.00850981319631\\
66.875	0.15522	-5.11699527191539	-5.11699527191539\\
66.875	0.15888	-5.17527390022676	-5.17527390022676\\
66.875	0.16254	-5.18334569813033	-5.18334569813033\\
66.875	0.1662	-5.14121066562613	-5.14121066562613\\
66.875	0.16986	-5.0488688027142	-5.0488688027142\\
66.875	0.17352	-4.90632010939448	-4.90632010939448\\
66.875	0.17718	-4.71356458566702	-4.71356458566702\\
66.875	0.18084	-4.47060223153178	-4.47060223153178\\
66.875	0.1845	-4.17743304698875	-4.17743304698875\\
66.875	0.18816	-3.834057032038	-3.834057032038\\
66.875	0.19182	-3.4404741866795	-3.4404741866795\\
66.875	0.19548	-2.99668451091318	-2.99668451091318\\
66.875	0.19914	-2.50268800473916	-2.50268800473916\\
66.875	0.2028	-1.95848466815734	-1.95848466815734\\
66.875	0.20646	-1.36407450116774	-1.36407450116774\\
66.875	0.21012	-0.719457503770386	-0.719457503770386\\
66.875	0.21378	-0.0246336759652763	-0.0246336759652763\\
66.875	0.21744	0.720396982247593	0.720396982247593\\
66.875	0.2211	1.51563447086818	1.51563447086818\\
66.875	0.22476	2.36107878989658	2.36107878989658\\
66.875	0.22842	3.25672993933276	3.25672993933276\\
66.875	0.23208	4.20258791917668	4.20258791917668\\
66.875	0.23574	5.19865272942835	5.19865272942835\\
66.875	0.2394	6.2449243700878	6.2449243700878\\
66.875	0.24306	7.34140284115502	7.34140284115502\\
66.875	0.24672	8.48808814262999	8.48808814262999\\
66.875	0.25038	9.68498027451277	9.68498027451277\\
66.875	0.25404	10.9320792368032	10.9320792368032\\
66.875	0.2577	12.2293850295015	12.2293850295015\\
66.875	0.26136	13.5768976526075	13.5768976526075\\
66.875	0.26502	14.9746171061213	14.9746171061213\\
66.875	0.26868	16.4225433900428	16.4225433900428\\
66.875	0.27234	17.9206765043721	17.9206765043721\\
66.875	0.276	19.4690164491092	19.4690164491092\\
67.25	0.093	3.83065042790627	3.83065042790627\\
67.25	0.09666	2.90373191320713	2.90373191320713\\
67.25	0.10032	2.02702022891582	2.02702022891582\\
67.25	0.10398	1.2005153750322	1.2005153750322\\
67.25	0.10764	0.424217351556411	0.424217351556411\\
67.25	0.1113	-0.301873841511672	-0.301873841511672\\
67.25	0.11496	-0.977758204171934	-0.977758204171934\\
67.25	0.11862	-1.60343573642447	-1.60343573642447\\
67.25	0.12228	-2.17890643826924	-2.17890643826924\\
67.25	0.12594	-2.70417030970624	-2.70417030970624\\
67.25	0.1296	-3.17922735073549	-3.17922735073549\\
67.25	0.13326	-3.60407756135697	-3.60407756135697\\
67.25	0.13692	-3.97872094157069	-3.97872094157069\\
67.25	0.14058	-4.30315749137666	-4.30315749137666\\
67.25	0.14424	-4.57738721077482	-4.57738721077482\\
67.25	0.1479	-4.80141009976528	-4.80141009976528\\
67.25	0.15156	-4.97522615834792	-4.97522615834792\\
67.25	0.15522	-5.09883538652284	-5.09883538652284\\
67.25	0.15888	-5.17223778428997	-5.17223778428997\\
67.25	0.16254	-5.19543335164935	-5.19543335164935\\
67.25	0.1662	-5.16842208860096	-5.16842208860096\\
67.25	0.16986	-5.09120399514481	-5.09120399514481\\
67.25	0.17352	-4.9637790712809	-4.9637790712809\\
67.25	0.17718	-4.78614731700923	-4.78614731700923\\
67.25	0.18084	-4.5583087323298	-4.5583087323298\\
67.25	0.1845	-4.28026331724259	-4.28026331724259\\
67.25	0.18816	-3.95201107174762	-3.95201107174762\\
67.25	0.19182	-3.57355199584489	-3.57355199584489\\
67.25	0.19548	-3.1448860895344	-3.1448860895344\\
67.25	0.19914	-2.66601335281615	-2.66601335281615\\
67.25	0.2028	-2.13693378569015	-2.13693378569015\\
67.25	0.20646	-1.55764738815633	-1.55764738815633\\
67.25	0.21012	-0.928154160214817	-0.928154160214817\\
67.25	0.21378	-0.248454101865491	-0.248454101865491\\
67.25	0.21744	0.481452786891538	0.481452786891538\\
67.25	0.2211	1.2615665060564	1.2615665060564\\
67.25	0.22476	2.09188705562896	2.09188705562896\\
67.25	0.22842	2.97241443560935	2.97241443560935\\
67.25	0.23208	3.90314864599743	3.90314864599743\\
67.25	0.23574	4.88408968679337	4.88408968679337\\
67.25	0.2394	5.91523755799699	5.91523755799699\\
67.25	0.24306	6.99659225960842	6.99659225960842\\
67.25	0.24672	8.12815379162755	8.12815379162755\\
67.25	0.25038	9.30992215405449	9.30992215405449\\
67.25	0.25404	10.5418973468892	10.5418973468892\\
67.25	0.2577	11.8240793701317	11.8240793701317\\
67.25	0.26136	13.1564682237819	13.1564682237819\\
67.25	0.26502	14.5390639078398	14.5390639078398\\
67.25	0.26868	15.9718664223056	15.9718664223056\\
67.25	0.27234	17.4548757671791	17.4548757671791\\
67.25	0.276	18.9880919424604	18.9880919424604\\
67.625	0.093	4.11578159160321	4.11578159160321\\
67.625	0.09666	3.17373930744829	3.17373930744829\\
67.625	0.10032	2.28190385370114	2.28190385370114\\
67.625	0.10398	1.44027523036174	1.44027523036174\\
67.625	0.10764	0.648853437430137	0.648853437430137\\
67.625	0.1113	-0.0923615250937289	-0.0923615250937289\\
67.625	0.11496	-0.783369657209803	-0.783369657209803\\
67.625	0.11862	-1.42417095891815	-1.42417095891815\\
67.625	0.12228	-2.0147654302187	-2.0147654302187\\
67.625	0.12594	-2.55515307111149	-2.55515307111149\\
67.625	0.1296	-3.04533388159655	-3.04533388159655\\
67.625	0.13326	-3.48530786167385	-3.48530786167385\\
67.625	0.13692	-3.87507501134334	-3.87507501134334\\
67.625	0.14058	-4.21463533060513	-4.21463533060513\\
67.625	0.14424	-4.5039888194591	-4.5039888194591\\
67.625	0.1479	-4.74313547790531	-4.74313547790531\\
67.625	0.15156	-4.93207530594379	-4.93207530594379\\
67.625	0.15522	-5.0708083035745	-5.0708083035745\\
67.625	0.15888	-5.15933447079744	-5.15933447079744\\
67.625	0.16254	-5.1976538076126	-5.1976538076126\\
67.625	0.1662	-5.18576631402	-5.18576631402\\
67.625	0.16986	-5.12367199001966	-5.12367199001966\\
67.625	0.17352	-5.01137083561157	-5.01137083561157\\
67.625	0.17718	-4.84886285079568	-4.84886285079568\\
67.625	0.18084	-4.63614803557206	-4.63614803557206\\
67.625	0.1845	-4.37322638994063	-4.37322638994063\\
67.625	0.18816	-4.06009791390147	-4.06009791390147\\
67.625	0.19182	-3.69676260745456	-3.69676260745456\\
67.625	0.19548	-3.28322047059984	-3.28322047059984\\
67.625	0.19914	-2.81947150333741	-2.81947150333741\\
67.625	0.2028	-2.30551570566716	-2.30551570566716\\
67.625	0.20646	-1.74135307758918	-1.74135307758918\\
67.625	0.21012	-1.12698361910346	-1.12698361910346\\
67.625	0.21378	-0.46240733020997	-0.46240733020997\\
67.625	0.21744	0.252375789091332	0.252375789091332\\
67.625	0.2211	1.01736573880035	1.01736573880035\\
67.625	0.22476	1.83256251891713	1.83256251891713\\
67.625	0.22842	2.69796612944168	2.69796612944168\\
67.625	0.23208	3.61357657037404	3.61357657037404\\
67.625	0.23574	4.57939384171414	4.57939384171414\\
67.625	0.2394	5.59541794346197	5.59541794346197\\
67.625	0.24306	6.66164887561756	6.66164887561756\\
67.625	0.24672	7.77808663818091	7.77808663818091\\
67.625	0.25038	8.94473123115212	8.94473123115212\\
67.625	0.25404	10.161582654531	10.161582654531\\
67.625	0.2577	11.4286409083176	11.4286409083176\\
67.625	0.26136	12.7459059925121	12.7459059925121\\
67.625	0.26502	14.1133779071142	14.1133779071142\\
67.625	0.26868	15.5310566521241	15.5310566521241\\
67.625	0.27234	16.9989422275419	16.9989422275419\\
67.625	0.276	18.5170346333674	18.5170346333674\\
68	0.093	4.41077995285589	4.41077995285589\\
68	0.09666	3.45361389924519	3.45361389924519\\
68	0.10032	2.54665467604222	2.54665467604222\\
68	0.10398	1.68990228324704	1.68990228324704\\
68	0.10764	0.883356720859627	0.883356720859627\\
68	0.1113	0.127017988879977	0.127017988879977\\
68	0.11496	-0.579113912691909	-0.579113912691909\\
68	0.11862	-1.23503898385604	-1.23503898385604\\
68	0.12228	-1.8407572246124	-1.8407572246124\\
68	0.12594	-2.396268634961	-2.396268634961\\
68	0.1296	-2.90157321490184	-2.90157321490184\\
68	0.13326	-3.35667096443493	-3.35667096443493\\
68	0.13692	-3.76156188356023	-3.76156188356023\\
68	0.14058	-4.1162459722778	-4.1162459722778\\
68	0.14424	-4.42072323058758	-4.42072323058758\\
68	0.1479	-4.67499365848961	-4.67499365848961\\
68	0.15156	-4.87905725598387	-4.87905725598387\\
68	0.15522	-5.03291402307037	-5.03291402307037\\
68	0.15888	-5.13656395974912	-5.13656395974912\\
68	0.16254	-5.19000706602009	-5.19000706602009\\
68	0.1662	-5.1932433418833	-5.1932433418833\\
68	0.16986	-5.14627278733875	-5.14627278733875\\
68	0.17352	-5.04909540238643	-5.04909540238643\\
68	0.17718	-4.90171118702636	-4.90171118702636\\
68	0.18084	-4.70412014125852	-4.70412014125852\\
68	0.1845	-4.45632226508293	-4.45632226508293\\
68	0.18816	-4.15831755849956	-4.15831755849956\\
68	0.19182	-3.81010602150843	-3.81010602150843\\
68	0.19548	-3.41168765410953	-3.41168765410953\\
68	0.19914	-2.96306245630291	-2.96306245630291\\
68	0.2028	-2.46423042808847	-2.46423042808847\\
68	0.20646	-1.91519156946627	-1.91519156946627\\
68	0.21012	-1.31594588043633	-1.31594588043633\\
68	0.21378	-0.666493360998629	-0.666493360998629\\
68	0.21744	0.0331659888468323	0.0331659888468323\\
68	0.2211	0.783032169100068	0.783032169100068\\
68	0.22476	1.58310517976106	1.58310517976106\\
68	0.22842	2.43338502082983	2.43338502082983\\
68	0.23208	3.33387169230635	3.33387169230635\\
68	0.23574	4.28456519419061	4.28456519419061\\
68	0.2394	5.28546552648265	5.28546552648265\\
68	0.24306	6.33657268918246	6.33657268918246\\
68	0.24672	7.43788668229003	7.43788668229003\\
68	0.25038	8.5894075058054	8.5894075058054\\
68	0.25404	9.79113515972848	9.79113515972848\\
68	0.2577	11.0430696440594	11.0430696440594\\
68	0.26136	12.345210958798	12.345210958798\\
68	0.26502	13.6975591039444	13.6975591039444\\
68	0.26868	15.1001140794985	15.1001140794985\\
68	0.27234	16.5528758854604	16.5528758854604\\
68	0.276	18.0558445218301	18.0558445218301\\
68.375	0.093	4.71564551166439	4.71564551166439\\
68.375	0.09666	3.74335568859791	3.74335568859791\\
68.375	0.10032	2.82127269593913	2.82127269593913\\
68.375	0.10398	1.94939653368816	1.94939653368816\\
68.375	0.10764	1.12772720184494	1.12772720184494\\
68.375	0.1113	0.356264700409476	0.356264700409476\\
68.375	0.11496	-0.364990970618194	-0.364990970618194\\
68.375	0.11862	-1.03603981123811	-1.03603981123811\\
68.375	0.12228	-1.65688182145028	-1.65688182145028\\
68.375	0.12594	-2.2275170012547	-2.2275170012547\\
68.375	0.1296	-2.74794535065132	-2.74794535065132\\
68.375	0.13326	-3.21816686964021	-3.21816686964021\\
68.375	0.13692	-3.63818155822133	-3.63818155822133\\
68.375	0.14058	-4.00798941639469	-4.00798941639469\\
68.375	0.14424	-4.32759044416028	-4.32759044416028\\
68.375	0.1479	-4.59698464151809	-4.59698464151809\\
68.375	0.15156	-4.81617200846816	-4.81617200846816\\
68.375	0.15522	-4.98515254501047	-4.98515254501047\\
68.375	0.15888	-5.103926251145	-5.103926251145\\
68.375	0.16254	-5.17249312687179	-5.17249312687179\\
68.375	0.1662	-5.19085317219078	-5.19085317219078\\
68.375	0.16986	-5.15900638710201	-5.15900638710201\\
68.375	0.17352	-5.07695277160551	-5.07695277160551\\
68.375	0.17718	-4.94469232570125	-4.94469232570125\\
68.375	0.18084	-4.76222504938922	-4.76222504938922\\
68.375	0.1845	-4.52955094266939	-4.52955094266939\\
68.375	0.18816	-4.24667000554186	-4.24667000554186\\
68.375	0.19182	-3.91358223800651	-3.91358223800651\\
68.375	0.19548	-3.53028764006342	-3.53028764006342\\
68.375	0.19914	-3.09678621171255	-3.09678621171255\\
68.375	0.2028	-2.61307795295396	-2.61307795295396\\
68.375	0.20646	-2.07916286378754	-2.07916286378754\\
68.375	0.21012	-1.49504094421339	-1.49504094421339\\
68.375	0.21378	-0.860712194231525	-0.860712194231525\\
68.375	0.21744	-0.176176613841847	-0.176176613841847\\
68.375	0.2211	0.558565796955605	0.558565796955605\\
68.375	0.22476	1.34351503816082	1.34351503816082\\
68.375	0.22842	2.17867110977375	2.17867110977375\\
68.375	0.23208	3.06403401179448	3.06403401179448\\
68.375	0.23574	3.99960374422295	3.99960374422295\\
68.375	0.2394	4.98538030705922	4.98538030705922\\
68.375	0.24306	6.02136370030324	6.02136370030324\\
68.375	0.24672	7.10755392395497	7.10755392395497\\
68.375	0.25038	8.24395097801455	8.24395097801455\\
68.375	0.25404	9.43055486248186	9.43055486248186\\
68.375	0.2577	10.6673655773569	10.6673655773569\\
68.375	0.26136	11.9543831226397	11.9543831226397\\
68.375	0.26502	13.2916074983303	13.2916074983303\\
68.375	0.26868	14.6790387044286	14.6790387044286\\
68.375	0.27234	16.1166767409348	16.1166767409348\\
68.375	0.276	17.6045216078487	17.6045216078487\\
68.75	0.093	5.0303782680286	5.0303782680286\\
68.75	0.09666	4.04296467550627	4.04296467550627\\
68.75	0.10032	3.10575791339174	3.10575791339174\\
68.75	0.10398	2.21875798168496	2.21875798168496\\
68.75	0.10764	1.38196488038593	1.38196488038593\\
68.75	0.1113	0.595378609494681	0.595378609494681\\
68.75	0.11496	-0.141000830988801	-0.141000830988801\\
68.75	0.11862	-0.827173441064527	-0.827173441064527\\
68.75	0.12228	-1.46313922073249	-1.46313922073249\\
68.75	0.12594	-2.04889816999268	-2.04889816999268\\
68.75	0.1296	-2.58445028884512	-2.58445028884512\\
68.75	0.13326	-3.06979557728982	-3.06979557728982\\
68.75	0.13692	-3.5049340353267	-3.5049340353267\\
68.75	0.14058	-3.88986566295586	-3.88986566295586\\
68.75	0.14424	-4.22459046017727	-4.22459046017727\\
68.75	0.1479	-4.50910842699089	-4.50910842699089\\
68.75	0.15156	-4.74341956339675	-4.74341956339675\\
68.75	0.15522	-4.92752386939487	-4.92752386939487\\
68.75	0.15888	-5.06142134498521	-5.06142134498521\\
68.75	0.16254	-5.14511199016776	-5.14511199016776\\
68.75	0.1662	-5.17859580494259	-5.17859580494259\\
68.75	0.16986	-5.16187278930963	-5.16187278930963\\
68.75	0.17352	-5.09494294326891	-5.09494294326891\\
68.75	0.17718	-4.97780626682043	-4.97780626682043\\
68.75	0.18084	-4.81046275996419	-4.81046275996419\\
68.75	0.1845	-4.59291242270019	-4.59291242270019\\
68.75	0.18816	-4.32515525502842	-4.32515525502842\\
68.75	0.19182	-4.00719125694891	-4.00719125694891\\
68.75	0.19548	-3.63902042846161	-3.63902042846161\\
68.75	0.19914	-3.22064276956655	-3.22064276956655\\
68.75	0.2028	-2.75205828026371	-2.75205828026371\\
68.75	0.20646	-2.23326696055314	-2.23326696055314\\
68.75	0.21012	-1.66426881043476	-1.66426881043476\\
68.75	0.21378	-1.04506382990868	-1.04506382990868\\
68.75	0.21744	-0.375652018974847	-0.375652018974847\\
68.75	0.2211	0.343966622366821	0.343966622366821\\
68.75	0.22476	1.11379209411619	1.11379209411619\\
68.75	0.22842	1.93382439627339	1.93382439627339\\
68.75	0.23208	2.80406352883828	2.80406352883828\\
68.75	0.23574	3.72450949181098	3.72450949181098\\
68.75	0.2394	4.69516228519146	4.69516228519146\\
68.75	0.24306	5.71602190897964	5.71602190897964\\
68.75	0.24672	6.78708836317558	6.78708836317558\\
68.75	0.25038	7.90836164777933	7.90836164777933\\
68.75	0.25404	9.07984176279085	9.07984176279085\\
68.75	0.2577	10.3015287082102	10.3015287082102\\
68.75	0.26136	11.5734224840372	11.5734224840372\\
68.75	0.26502	12.8955230902719	12.8955230902719\\
68.75	0.26868	14.2678305269144	14.2678305269144\\
68.75	0.27234	15.6903447939648	15.6903447939648\\
68.75	0.276	17.1630658914229	17.1630658914229\\
69.125	0.093	5.3549782219486	5.3549782219486\\
69.125	0.09666	4.35244085997049	4.35244085997049\\
69.125	0.10032	3.40011032840015	3.40011032840015\\
69.125	0.10398	2.49798662723756	2.49798662723756\\
69.125	0.10764	1.64606975648273	1.64606975648273\\
69.125	0.1113	0.844359716135678	0.844359716135678\\
69.125	0.11496	0.0928565061963837	0.0928565061963837\\
69.125	0.11862	-0.608439873335126	-0.608439873335126\\
69.125	0.12228	-1.2595294224589	-1.2595294224589\\
69.125	0.12594	-1.86041214117488	-1.86041214117488\\
69.125	0.1296	-2.41108802948312	-2.41108802948312\\
69.125	0.13326	-2.91155708738361	-2.91155708738361\\
69.125	0.13692	-3.36181931487633	-3.36181931487633\\
69.125	0.14058	-3.76187471196128	-3.76187471196128\\
69.125	0.14424	-4.11172327863846	-4.11172327863846\\
69.125	0.1479	-4.4113650149079	-4.4113650149079\\
69.125	0.15156	-4.66079992076957	-4.66079992076957\\
69.125	0.15522	-4.86002799622344	-4.86002799622344\\
69.125	0.15888	-5.0090492412696	-5.0090492412696\\
69.125	0.16254	-5.10786365590798	-5.10786365590798\\
69.125	0.1662	-5.15647124013857	-5.15647124013857\\
69.125	0.16986	-5.15487199396142	-5.15487199396142\\
69.125	0.17352	-5.10306591737652	-5.10306591737652\\
69.125	0.17718	-5.00105301038382	-5.00105301038382\\
69.125	0.18084	-4.84883327298339	-4.84883327298339\\
69.125	0.1845	-4.64640670517518	-4.64640670517518\\
69.125	0.18816	-4.39377330695922	-4.39377330695922\\
69.125	0.19182	-4.0909330783355	-4.0909330783355\\
69.125	0.19548	-3.737886019304	-3.737886019304\\
69.125	0.19914	-3.33463212986476	-3.33463212986476\\
69.125	0.2028	-2.88117141001773	-2.88117141001773\\
69.125	0.20646	-2.37750385976294	-2.37750385976294\\
69.125	0.21012	-1.8236294791004	-1.8236294791004\\
69.125	0.21378	-1.21954826803005	-1.21954826803005\\
69.125	0.21744	-0.565260226551999	-0.565260226551999\\
69.125	0.2211	0.139234645333829	0.139234645333829\\
69.125	0.22476	0.893936347627417	0.893936347627417\\
69.125	0.22842	1.69884488032878	1.69884488032878\\
69.125	0.23208	2.55396024343788	2.55396024343788\\
69.125	0.23574	3.45928243695479	3.45928243695479\\
69.125	0.2394	4.41481146087943	4.41481146087943\\
69.125	0.24306	5.42054731521183	5.42054731521183\\
69.125	0.24672	6.47648999995199	6.47648999995199\\
69.125	0.25038	7.58263951510001	7.58263951510001\\
69.125	0.25404	8.73899586065563	8.73899586065563\\
69.125	0.2577	9.94555903661916	9.94555903661916\\
69.125	0.26136	11.2023290429904	11.2023290429904\\
69.125	0.26502	12.5093058797694	12.5093058797694\\
69.125	0.26868	13.8664895469561	13.8664895469561\\
69.125	0.27234	15.2738800445506	15.2738800445506\\
69.125	0.276	16.7314773725529	16.7314773725529\\
69.5	0.093	5.68944537342437	5.68944537342437\\
69.5	0.09666	4.67178424199047	4.67178424199047\\
69.5	0.10032	3.70432994096431	3.70432994096431\\
69.5	0.10398	2.78708247034591	2.78708247034591\\
69.5	0.10764	1.92004183013531	1.92004183013531\\
69.5	0.1113	1.10320802033247	1.10320802033247\\
69.5	0.11496	0.336581040937389	0.336581040937389\\
69.5	0.11862	-0.379839108049932	-0.379839108049932\\
69.5	0.12228	-1.04605242662949	-1.04605242662949\\
69.5	0.12594	-1.66205891480131	-1.66205891480131\\
69.5	0.1296	-2.22785857256537	-2.22785857256537\\
69.5	0.13326	-2.74345139992164	-2.74345139992164\\
69.5	0.13692	-3.20883739687014	-3.20883739687014\\
69.5	0.14058	-3.6240165634109	-3.6240165634109\\
69.5	0.14424	-3.98898889954387	-3.98898889954387\\
69.5	0.1479	-4.30375440526912	-4.30375440526912\\
69.5	0.15156	-4.56831308058657	-4.56831308058657\\
69.5	0.15522	-4.78266492549626	-4.78266492549626\\
69.5	0.15888	-4.9468099399982	-4.9468099399982\\
69.5	0.16254	-5.06074812409239	-5.06074812409239\\
69.5	0.1662	-5.12447947777879	-5.12447947777879\\
69.5	0.16986	-5.13800400105742	-5.13800400105742\\
69.5	0.17352	-5.10132169392831	-5.10132169392831\\
69.5	0.17718	-5.01443255639145	-5.01443255639145\\
69.5	0.18084	-4.8773365884468	-4.8773365884468\\
69.5	0.1845	-4.69003379009438	-4.69003379009438\\
69.5	0.18816	-4.45252416133422	-4.45252416133422\\
69.5	0.19182	-4.16480770216631	-4.16480770216631\\
69.5	0.19548	-3.8268844125906	-3.8268844125906\\
69.5	0.19914	-3.43875429260714	-3.43875429260714\\
69.5	0.2028	-3.00041734221593	-3.00041734221593\\
69.5	0.20646	-2.51187356141692	-2.51187356141692\\
69.5	0.21012	-1.97312295021017	-1.97312295021017\\
69.5	0.21378	-1.38416550859566	-1.38416550859566\\
69.5	0.21744	-0.745001236573387	-0.745001236573387\\
69.5	0.2211	-0.0556301341433993	-0.0556301341433993\\
69.5	0.22476	0.683947798694405	0.683947798694405\\
69.5	0.22842	1.47373256193998	1.47373256193998\\
69.5	0.23208	2.31372415559331	2.31372415559331\\
69.5	0.23574	3.20392257965437	3.20392257965437\\
69.5	0.2394	4.14432783412323	4.14432783412323\\
69.5	0.24306	5.13493991899985	5.13493991899985\\
69.5	0.24672	6.17575883428422	6.17575883428422\\
69.5	0.25038	7.26678457997635	7.26678457997635\\
69.5	0.25404	8.40801715607624	8.40801715607624\\
69.5	0.2577	9.59945656258392	9.59945656258392\\
69.5	0.26136	10.8411027994994	10.8411027994994\\
69.5	0.26502	12.1329558668226	12.1329558668226\\
69.5	0.26868	13.4750157645535	13.4750157645535\\
69.5	0.27234	14.8672824926922	14.8672824926922\\
69.5	0.276	16.3097560512387	16.3097560512387\\
69.875	0.093	6.03377972245589	6.03377972245589\\
69.875	0.09666	5.00099482156618	5.00099482156618\\
69.875	0.10032	4.01841675108425	4.01841675108425\\
69.875	0.10398	3.08604551101003	3.08604551101003\\
69.875	0.10764	2.20388110134364	2.20388110134364\\
69.875	0.1113	1.37192352208496	1.37192352208496\\
69.875	0.11496	0.590172773234102	0.590172773234102\\
69.875	0.11862	-0.141371145209003	-0.141371145209003\\
69.875	0.12228	-0.822708233244398	-0.822708233244398\\
69.875	0.12594	-1.45383849087197	-1.45383849087197\\
69.875	0.1296	-2.03476191809182	-2.03476191809182\\
69.875	0.13326	-2.5654785149039	-2.5654785149039\\
69.875	0.13692	-3.04598828130821	-3.04598828130821\\
69.875	0.14058	-3.47629121730476	-3.47629121730476\\
69.875	0.14424	-3.85638732289357	-3.85638732289357\\
69.875	0.1479	-4.18627659807457	-4.18627659807457\\
69.875	0.15156	-4.46595904284784	-4.46595904284784\\
69.875	0.15522	-4.69543465721333	-4.69543465721333\\
69.875	0.15888	-4.87470344117109	-4.87470344117109\\
69.875	0.16254	-5.00376539472104	-5.00376539472104\\
69.875	0.1662	-5.08262051786328	-5.08262051786328\\
69.875	0.16986	-5.11126881059769	-5.11126881059769\\
69.875	0.17352	-5.08971027292439	-5.08971027292439\\
69.875	0.17718	-5.01794490484331	-5.01794490484331\\
69.875	0.18084	-4.89597270635448	-4.89597270635448\\
69.875	0.1845	-4.72379367745786	-4.72379367745786\\
69.875	0.18816	-4.5014078181535	-4.5014078181535\\
69.875	0.19182	-4.22881512844137	-4.22881512844137\\
69.875	0.19548	-3.90601560832147	-3.90601560832147\\
69.875	0.19914	-3.53300925779382	-3.53300925779382\\
69.875	0.2028	-3.10979607685836	-3.10979607685836\\
69.875	0.20646	-2.63637606551519	-2.63637606551519\\
69.875	0.21012	-2.11274922376423	-2.11274922376423\\
69.875	0.21378	-1.53891555160556	-1.53891555160556\\
69.875	0.21744	-0.914875049039068	-0.914875049039068\\
69.875	0.2211	-0.240627716064864	-0.240627716064864\\
69.875	0.22476	0.483826447317156	0.483826447317156\\
69.875	0.22842	1.25848744110695	1.25848744110695\\
69.875	0.23208	2.08335526530443	2.08335526530443\\
69.875	0.23574	2.95842991990972	2.95842991990972\\
69.875	0.2394	3.88371140492279	3.88371140492279\\
69.875	0.24306	4.85919972034357	4.85919972034357\\
69.875	0.24672	5.88489486617216	5.88489486617216\\
69.875	0.25038	6.96079684240856	6.96079684240856\\
69.875	0.25404	8.08690564905261	8.08690564905261\\
69.875	0.2577	9.26322128610445	9.26322128610445\\
69.875	0.26136	10.4897437535641	10.4897437535641\\
69.875	0.26502	11.7664730514315	11.7664730514315\\
69.875	0.26868	13.0934091797066	13.0934091797066\\
69.875	0.27234	14.4705521383896	14.4705521383896\\
69.875	0.276	15.8979019274803	15.8979019274803\\
70.25	0.093	6.38798126904316	6.38798126904316\\
70.25	0.09666	5.34007259869763	5.34007259869763\\
70.25	0.10032	4.34237075875988	4.34237075875988\\
70.25	0.10398	3.39487574922989	3.39487574922989\\
70.25	0.10764	2.49758757010769	2.49758757010769\\
70.25	0.1113	1.65050622139325	1.65050622139325\\
70.25	0.11496	0.853631703086549	0.853631703086549\\
70.25	0.11862	0.106964015187632	0.106964015187632\\
70.25	0.12228	-0.589496842303518	-0.589496842303518\\
70.25	0.12594	-1.23575086938693	-1.23575086938693\\
70.25	0.1296	-1.83179806606256	-1.83179806606256\\
70.25	0.13326	-2.37763843233043	-2.37763843233043\\
70.25	0.13692	-2.87327196819055	-2.87327196819055\\
70.25	0.14058	-3.31869867364291	-3.31869867364291\\
70.25	0.14424	-3.7139185486875	-3.7139185486875\\
70.25	0.1479	-4.05893159332432	-4.05893159332432\\
70.25	0.15156	-4.35373780755337	-4.35373780755337\\
70.25	0.15522	-4.59833719137468	-4.59833719137468\\
70.25	0.15888	-4.79272974478821	-4.79272974478821\\
70.25	0.16254	-4.936915467794	-4.936915467794\\
70.25	0.1662	-5.030894360392	-5.030894360392\\
70.25	0.16986	-5.07466642258223	-5.07466642258223\\
70.25	0.17352	-5.06823165436473	-5.06823165436473\\
70.25	0.17718	-5.01159005573944	-5.01159005573944\\
70.25	0.18084	-4.90474162670642	-4.90474162670642\\
70.25	0.1845	-4.74768636726559	-4.74768636726559\\
70.25	0.18816	-4.54042427741703	-4.54042427741703\\
70.25	0.19182	-4.28295535716072	-4.28295535716072\\
70.25	0.19548	-3.9752796064966	-3.9752796064966\\
70.25	0.19914	-3.61739702542474	-3.61739702542474\\
70.25	0.2028	-3.20930761394511	-3.20930761394511\\
70.25	0.20646	-2.75101137205773	-2.75101137205773\\
70.25	0.21012	-2.24250829976255	-2.24250829976255\\
70.25	0.21378	-1.68379839705966	-1.68379839705966\\
70.25	0.21744	-1.07488166394901	-1.07488166394901\\
70.25	0.2211	-0.415758100430594	-0.415758100430594\\
70.25	0.22476	0.293572293495643	0.293572293495643\\
70.25	0.22842	1.0531095178296	1.0531095178296\\
70.25	0.23208	1.8628535725713	1.8628535725713\\
70.25	0.23574	2.7228044577208	2.7228044577208\\
70.25	0.2394	3.63296217327803	3.63296217327803\\
70.25	0.24306	4.59332671924308	4.59332671924308\\
70.25	0.24672	5.60389809561583	5.60389809561583\\
70.25	0.25038	6.66467630239639	6.66467630239639\\
70.25	0.25404	7.77566133958472	7.77566133958472\\
70.25	0.2577	8.93685320718072	8.93685320718072\\
70.25	0.26136	10.1482519051846	10.1482519051846\\
70.25	0.26502	11.4098574335962	11.4098574335962\\
70.25	0.26868	12.7216697924155	12.7216697924155\\
70.25	0.27234	14.0836889816426	14.0836889816426\\
70.25	0.276	15.4959150012775	15.4959150012775\\
70.625	0.093	6.75205001318621	6.75205001318621\\
70.625	0.09666	5.68901757338491	5.68901757338491\\
70.625	0.10032	4.67619196399137	4.67619196399137\\
70.625	0.10398	3.71357318500559	3.71357318500559\\
70.625	0.10764	2.80116123642758	2.80116123642758\\
70.625	0.1113	1.93895611825733	1.93895611825733\\
70.625	0.11496	1.12695783049485	1.12695783049485\\
70.625	0.11862	0.365166373140116	0.365166373140116\\
70.625	0.12228	-0.346418253806846	-0.346418253806846\\
70.625	0.12594	-1.00779605034604	-1.00779605034604\\
70.625	0.1296	-1.61896701647748	-1.61896701647748\\
70.625	0.13326	-2.17993115220116	-2.17993115220116\\
70.625	0.13692	-2.69068845751707	-2.69068845751707\\
70.625	0.14058	-3.15123893242524	-3.15123893242524\\
70.625	0.14424	-3.56158257692562	-3.56158257692562\\
70.625	0.1479	-3.92171939101824	-3.92171939101824\\
70.625	0.15156	-4.2316493747031	-4.2316493747031\\
70.625	0.15522	-4.4913725279802	-4.4913725279802\\
70.625	0.15888	-4.70088885084954	-4.70088885084954\\
70.625	0.16254	-4.86019834331112	-4.86019834331112\\
70.625	0.1662	-4.9693010053649	-4.9693010053649\\
70.625	0.16986	-5.02819683701094	-5.02819683701094\\
70.625	0.17352	-5.03688583824923	-5.03688583824923\\
70.625	0.17718	-4.99536800907975	-4.99536800907975\\
70.625	0.18084	-4.90364334950251	-4.90364334950251\\
70.625	0.1845	-4.76171185951749	-4.76171185951749\\
70.625	0.18816	-4.56957353912475	-4.56957353912475\\
70.625	0.19182	-4.32722838832422	-4.32722838832422\\
70.625	0.19548	-4.03467640711591	-4.03467640711591\\
70.625	0.19914	-3.69191759549986	-3.69191759549986\\
70.625	0.2028	-3.29895195347602	-3.29895195347602\\
70.625	0.20646	-2.85577948104442	-2.85577948104442\\
70.625	0.21012	-2.36240017820508	-2.36240017820508\\
70.625	0.21378	-1.81881404495797	-1.81881404495797\\
70.625	0.21744	-1.22502108130311	-1.22502108130311\\
70.625	0.2211	-0.581021287240475	-0.581021287240475\\
70.625	0.22476	0.113185337229922	0.113185337229922\\
70.625	0.22842	0.857598792108092	0.857598792108092\\
70.625	0.23208	1.65221907739401	1.65221907739401\\
70.625	0.23574	2.49704619308773	2.49704619308773\\
70.625	0.2394	3.39208013918918	3.39208013918918\\
70.625	0.24306	4.33732091569838	4.33732091569838\\
70.625	0.24672	5.33276852261535	5.33276852261535\\
70.625	0.25038	6.37842295994012	6.37842295994012\\
70.625	0.25404	7.47428422767261	7.47428422767261\\
70.625	0.2577	8.62035232581289	8.62035232581289\\
70.625	0.26136	9.81662725436092	9.81662725436092\\
70.625	0.26502	11.0631090133167	11.0631090133167\\
70.625	0.26868	12.3597976026802	12.3597976026802\\
70.625	0.27234	13.7066930224516	13.7066930224516\\
70.625	0.276	15.1037952726306	15.1037952726306\\
71	0.093	7.12598595488503	7.12598595488503\\
71	0.09666	6.04782974562791	6.04782974562791\\
71	0.10032	5.01988036677857	5.01988036677857\\
71	0.10398	4.042137818337	4.042137818337\\
71	0.10764	3.11460210030318	3.11460210030318\\
71	0.1113	2.23727321267712	2.23727321267712\\
71	0.11496	1.41015115545885	1.41015115545885\\
71	0.11862	0.633235928648336	0.633235928648336\\
71	0.12228	-0.0934724677544381	-0.0934724677544381\\
71	0.12594	-0.769974033749449	-0.769974033749449\\
71	0.1296	-1.39626876933667	-1.39626876933667\\
71	0.13326	-1.97235667451613	-1.97235667451613\\
71	0.13692	-2.49823774928785	-2.49823774928785\\
71	0.14058	-2.9739119936518	-2.9739119936518\\
71	0.14424	-3.399379407608	-3.399379407608\\
71	0.1479	-3.77463999115643	-3.77463999115643\\
71	0.15156	-4.09969374429708	-4.09969374429708\\
71	0.15522	-4.37454066702995	-4.37454066702995\\
71	0.15888	-4.59918075935511	-4.59918075935511\\
71	0.16254	-4.77361402127247	-4.77361402127247\\
71	0.1662	-4.89784045278209	-4.89784045278209\\
71	0.16986	-4.97186005388394	-4.97186005388394\\
71	0.17352	-4.99567282457802	-4.99567282457802\\
71	0.17718	-4.96927876486435	-4.96927876486435\\
71	0.18084	-4.89267787474289	-4.89267787474289\\
71	0.1845	-4.76587015421369	-4.76587015421369\\
71	0.18816	-4.58885560327673	-4.58885560327673\\
71	0.19182	-4.36163422193201	-4.36163422193201\\
71	0.19548	-4.08420601017949	-4.08420601017949\\
71	0.19914	-3.75657096801925	-3.75657096801925\\
71	0.2028	-3.37872909545122	-3.37872909545122\\
71	0.20646	-2.95068039247541	-2.95068039247541\\
71	0.21012	-2.47242485909184	-2.47242485909184\\
71	0.21378	-1.94396249530058	-1.94396249530058\\
71	0.21744	-1.3652933011015	-1.3652933011015\\
71	0.2211	-0.736417276494649	-0.736417276494649\\
71	0.22476	-0.0573344214800358	-0.0573344214800358\\
71	0.22842	0.671955263942294	0.671955263942294\\
71	0.23208	1.45145177977243	1.45145177977243\\
71	0.23574	2.2811551260103	2.2811551260103\\
71	0.2394	3.16106530265597	3.16106530265597\\
71	0.24306	4.0911823097094	4.0911823097094\\
71	0.24672	5.07150614717058	5.07150614717058\\
71	0.25038	6.10203681503951	6.10203681503951\\
71	0.25404	7.18277431331622	7.18277431331622\\
71	0.2577	8.31371864200071	8.31371864200071\\
71	0.26136	9.4948698010929	9.4948698010929\\
71	0.26502	10.7262277905929	10.7262277905929\\
71	0.26868	12.0077926105006	12.0077926105006\\
71	0.27234	13.3395642608161	13.3395642608161\\
71	0.276	14.7215427415395	14.7215427415395\\
71.375	0.093	7.50978909413961	7.50978909413961\\
71.375	0.09666	6.41650911542668	6.41650911542668\\
71.375	0.10032	5.37343596712152	5.37343596712152\\
71.375	0.10398	4.38056964922417	4.38056964922417\\
71.375	0.10764	3.43791016173454	3.43791016173454\\
71.375	0.1113	2.5454575046527	2.5454575046527\\
71.375	0.11496	1.70321167797862	1.70321167797862\\
71.375	0.11862	0.911172681712319	0.911172681712319\\
71.375	0.12228	0.169340515853733	0.169340515853733\\
71.375	0.12594	-0.522284819597061	-0.522284819597061\\
71.375	0.1296	-1.1637033246401	-1.1637033246401\\
71.375	0.13326	-1.75491499927537	-1.75491499927537\\
71.375	0.13692	-2.29591984350287	-2.29591984350287\\
71.375	0.14058	-2.78671785732264	-2.78671785732264\\
71.375	0.14424	-3.22730904073461	-3.22730904073461\\
71.375	0.1479	-3.61769339373883	-3.61769339373883\\
71.375	0.15156	-3.95787091633532	-3.95787091633532\\
71.375	0.15522	-4.24784160852398	-4.24784160852398\\
71.375	0.15888	-4.48760547030492	-4.48760547030492\\
71.375	0.16254	-4.67716250167812	-4.67716250167812\\
71.375	0.1662	-4.81651270264352	-4.81651270264352\\
71.375	0.16986	-4.90565607320116	-4.90565607320116\\
71.375	0.17352	-4.94459261335104	-4.94459261335104\\
71.375	0.17718	-4.93332232309316	-4.93332232309316\\
71.375	0.18084	-4.87184520242752	-4.87184520242752\\
71.375	0.1845	-4.76016125135409	-4.76016125135409\\
71.375	0.18816	-4.59827046987294	-4.59827046987294\\
71.375	0.19182	-4.38617285798401	-4.38617285798401\\
71.375	0.19548	-4.1238684156873	-4.1238684156873\\
71.375	0.19914	-3.81135714298281	-3.81135714298281\\
71.375	0.2028	-3.4486390398706	-3.4486390398706\\
71.375	0.20646	-3.03571410635062	-3.03571410635062\\
71.375	0.21012	-2.57258234242285	-2.57258234242285\\
71.375	0.21378	-2.05924374808737	-2.05924374808737\\
71.375	0.21744	-1.49569832334413	-1.49569832334413\\
71.375	0.2211	-0.881946068193059	-0.881946068193059\\
71.375	0.22476	-0.217986982634287	-0.217986982634287\\
71.375	0.22842	0.496178933332317	0.496178933332317\\
71.375	0.23208	1.26055167970661	1.26055167970661\\
71.375	0.23574	2.0751312564887	2.0751312564887\\
71.375	0.2394	2.93991766367859	2.93991766367859\\
71.375	0.24306	3.85491090127617	3.85491090127617\\
71.375	0.24672	4.82011096928157	4.82011096928157\\
71.375	0.25038	5.83551786769472	5.83551786769472\\
71.375	0.25404	6.90113159651558	6.90113159651558\\
71.375	0.2577	8.01695215574429	8.01695215574429\\
71.375	0.26136	9.18297954538076	9.18297954538076\\
71.375	0.26502	10.3992137654249	10.3992137654249\\
71.375	0.26868	11.6656548158768	11.6656548158768\\
71.375	0.27234	12.9823026967366	12.9823026967366\\
71.375	0.276	14.349157408004	14.349157408004\\
71.75	0.093	7.90345943094995	7.90345943094995\\
71.75	0.09666	6.79505568278124	6.79505568278124\\
71.75	0.10032	5.7368587650203	5.7368587650203\\
71.75	0.10398	4.72886867766714	4.72886867766714\\
71.75	0.10764	3.77108542072175	3.77108542072175\\
71.75	0.1113	2.86350899418409	2.86350899418409\\
71.75	0.11496	2.0061393980542	2.0061393980542\\
71.75	0.11862	1.19897663233207	1.19897663233207\\
71.75	0.12228	0.442020697017725	0.442020697017725\\
71.75	0.12594	-0.264728407888882	-0.264728407888882\\
71.75	0.1296	-0.9212706823877	-0.9212706823877\\
71.75	0.13326	-1.52760612647879	-1.52760612647879\\
71.75	0.13692	-2.0837347401621	-2.0837347401621\\
71.75	0.14058	-2.58965652343765	-2.58965652343765\\
71.75	0.14424	-3.04537147630543	-3.04537147630543\\
71.75	0.1479	-3.45087959876547	-3.45087959876547\\
71.75	0.15156	-3.80618089081771	-3.80618089081771\\
71.75	0.15522	-4.11127535246221	-4.11127535246221\\
71.75	0.15888	-4.36616298369893	-4.36616298369893\\
71.75	0.16254	-4.57084378452792	-4.57084378452792\\
71.75	0.1662	-4.72531775494913	-4.72531775494913\\
71.75	0.16986	-4.82958489496255	-4.82958489496255\\
71.75	0.17352	-4.88364520456825	-4.88364520456825\\
71.75	0.17718	-4.88749868376615	-4.88749868376615\\
71.75	0.18084	-4.84114533255632	-4.84114533255632\\
71.75	0.1845	-4.7445851509387	-4.7445851509387\\
71.75	0.18816	-4.59781813891334	-4.59781813891334\\
71.75	0.19182	-4.40084429648022	-4.40084429648022\\
71.75	0.19548	-4.15366362363935	-4.15366362363935\\
71.75	0.19914	-3.85627612039065	-3.85627612039065\\
71.75	0.2028	-3.50868178673421	-3.50868178673421\\
71.75	0.20646	-3.11088062267002	-3.11088062267002\\
71.75	0.21012	-2.66287262819809	-2.66287262819809\\
71.75	0.21378	-2.16465780331839	-2.16465780331839\\
71.75	0.21744	-1.61623614803088	-1.61623614803088\\
71.75	0.2211	-1.01760766233565	-1.01760766233565\\
71.75	0.22476	-0.36877234623266	-0.36877234623266\\
71.75	0.22842	0.330269800278103	0.330269800278103\\
71.75	0.23208	1.07951877719661	1.07951877719661\\
71.75	0.23574	1.87897458452292	1.87897458452292\\
71.75	0.2394	2.72863722225696	2.72863722225696\\
71.75	0.24306	3.62850669039877	3.62850669039877\\
71.75	0.24672	4.57858298894838	4.57858298894838\\
71.75	0.25038	5.57886611790569	5.57886611790569\\
71.75	0.25404	6.62935607727083	6.62935607727083\\
71.75	0.2577	7.7300528670437	7.7300528670437\\
71.75	0.26136	8.88095648722432	8.88095648722432\\
71.75	0.26502	10.0820669378128	10.0820669378128\\
71.75	0.26868	11.3333842188089	11.3333842188089\\
71.75	0.27234	12.6349083302128	12.6349083302128\\
71.75	0.276	13.9866392720245	13.9866392720245\\
72.125	0.093	8.30699696531603	8.30699696531603\\
72.125	0.09666	7.18346944769151	7.18346944769151\\
72.125	0.10032	6.11014876047476	6.11014876047476\\
72.125	0.10398	5.08703490366578	5.08703490366578\\
72.125	0.10764	4.11412787726458	4.11412787726458\\
72.125	0.1113	3.19142768127114	3.19142768127114\\
72.125	0.11496	2.31893431568547	2.31893431568547\\
72.125	0.11862	1.49664778050752	1.49664778050752\\
72.125	0.12228	0.724568075737366	0.724568075737366\\
72.125	0.12594	0.00269520137497636	0.00269520137497636\\
72.125	0.1296	-0.668970842579682	-0.668970842579682\\
72.125	0.13326	-1.29043005612655	-1.29043005612655\\
72.125	0.13692	-1.86168243926565	-1.86168243926565\\
72.125	0.14058	-2.38272799199698	-2.38272799199698\\
72.125	0.14424	-2.85356671432055	-2.85356671432055\\
72.125	0.1479	-3.2741986062364	-3.2741986062364\\
72.125	0.15156	-3.64462366774445	-3.64462366774445\\
72.125	0.15522	-3.96484189884473	-3.96484189884473\\
72.125	0.15888	-4.23485329953727	-4.23485329953727\\
72.125	0.16254	-4.45465786982204	-4.45465786982204\\
72.125	0.1662	-4.62425560969903	-4.62425560969903\\
72.125	0.16986	-4.74364651916827	-4.74364651916827\\
72.125	0.17352	-4.8128305982298	-4.8128305982298\\
72.125	0.17718	-4.83180784688351	-4.83180784688351\\
72.125	0.18084	-4.80057826512947	-4.80057826512947\\
72.125	0.1845	-4.71914185296764	-4.71914185296764\\
72.125	0.18816	-4.58749861039806	-4.58749861039806\\
72.125	0.19182	-4.40564853742072	-4.40564853742072\\
72.125	0.19548	-4.17359163403566	-4.17359163403566\\
72.125	0.19914	-3.89132790024274	-3.89132790024274\\
72.125	0.2028	-3.55885733604215	-3.55885733604215\\
72.125	0.20646	-3.17617994143374	-3.17617994143374\\
72.125	0.21012	-2.74329571641759	-2.74329571641759\\
72.125	0.21378	-2.26020466099368	-2.26020466099368\\
72.125	0.21744	-1.72690677516201	-1.72690677516201\\
72.125	0.2211	-1.14340205892262	-1.14340205892262\\
72.125	0.22476	-0.509690512275355	-0.509690512275355\\
72.125	0.22842	0.174227864779567	0.174227864779567\\
72.125	0.23208	0.908353072242349	0.908353072242349\\
72.125	0.23574	1.69268511011282	1.69268511011282\\
72.125	0.2394	2.52722397839102	2.52722397839102\\
72.125	0.24306	3.4119696770771	3.4119696770771\\
72.125	0.24672	4.34692220617082	4.34692220617082\\
72.125	0.25038	5.3320815656724	5.3320815656724\\
72.125	0.25404	6.36744775558169	6.36744775558169\\
72.125	0.2577	7.45302077589872	7.45302077589872\\
72.125	0.26136	8.58880062662362	8.58880062662362\\
72.125	0.26502	9.77478730775616	9.77478730775616\\
72.125	0.26868	11.0109808192965	11.0109808192965\\
72.125	0.27234	12.2973811612446	12.2973811612446\\
72.125	0.276	13.6339883336005	13.6339883336005\\
72.5	0.093	8.72040169723791	8.72040169723791\\
72.5	0.09666	7.58175041015763	7.58175041015763\\
72.5	0.10032	6.49330595348506	6.49330595348506\\
72.5	0.10398	5.45506832722031	5.45506832722031\\
72.5	0.10764	4.46703753136326	4.46703753136326\\
72.5	0.1113	3.52921356591404	3.52921356591404\\
72.5	0.11496	2.64159643087253	2.64159643087253\\
72.5	0.11862	1.80418612623882	1.80418612623882\\
72.5	0.12228	1.01698265201286	1.01698265201286\\
72.5	0.12594	0.279986008194655	0.279986008194655\\
72.5	0.1296	-0.406803805215759	-0.406803805215759\\
72.5	0.13326	-1.04338678821844	-1.04338678821844\\
72.5	0.13692	-1.62976294081335	-1.62976294081335\\
72.5	0.14058	-2.16593226300049	-2.16593226300049\\
72.5	0.14424	-2.65189475477987	-2.65189475477987\\
72.5	0.1479	-3.08765041615148	-3.08765041615148\\
72.5	0.15156	-3.47319924711537	-3.47319924711537\\
72.5	0.15522	-3.80854124767144	-3.80854124767144\\
72.5	0.15888	-4.09367641781979	-4.09367641781979\\
72.5	0.16254	-4.32860475756033	-4.32860475756033\\
72.5	0.1662	-4.51332626689317	-4.51332626689317\\
72.5	0.16986	-4.64784094581819	-4.64784094581819\\
72.5	0.17352	-4.73214879433545	-4.73214879433545\\
72.5	0.17718	-4.766249812445	-4.766249812445\\
72.5	0.18084	-4.75014400014674	-4.75014400014674\\
72.5	0.1845	-4.68383135744075	-4.68383135744075\\
72.5	0.18816	-4.56731188432695	-4.56731188432695\\
72.5	0.19182	-4.4005855808054	-4.4005855808054\\
72.5	0.19548	-4.18365244687612	-4.18365244687612\\
72.5	0.19914	-3.91651248253905	-3.91651248253905\\
72.5	0.2028	-3.59916568779424	-3.59916568779424\\
72.5	0.20646	-3.23161206264162	-3.23161206264162\\
72.5	0.21012	-2.8138516070813	-2.8138516070813\\
72.5	0.21378	-2.34588432111318	-2.34588432111318\\
72.5	0.21744	-1.82771020473734	-1.82771020473734\\
72.5	0.2211	-1.25932925795368	-1.25932925795368\\
72.5	0.22476	-0.640741480762259	-0.640741480762259\\
72.5	0.22842	0.0280531268369373	0.0280531268369373\\
72.5	0.23208	0.747054564843822	0.747054564843822\\
72.5	0.23574	1.51626283325851	1.51626283325851\\
72.5	0.2394	2.33567793208098	2.33567793208098\\
72.5	0.24306	3.20529986131122	3.20529986131122\\
72.5	0.24672	4.12512862094916	4.12512862094916\\
72.5	0.25038	5.09516421099489	5.09516421099489\\
72.5	0.25404	6.11540663144841	6.11540663144841\\
72.5	0.2577	7.18585588230971	7.18585588230971\\
72.5	0.26136	8.30651196357871	8.30651196357871\\
72.5	0.26502	9.47737487525546	9.47737487525546\\
72.5	0.26868	10.69844461734	10.69844461734\\
72.5	0.27234	11.9697211898324	11.9697211898324\\
72.5	0.276	13.2912045927325	13.2912045927325\\
72.875	0.093	9.14367362671551	9.14367362671551\\
72.875	0.09666	7.98989857017942	7.98989857017942\\
72.875	0.10032	6.88633034405107	6.88633034405107\\
72.875	0.10398	5.83296894833048	5.83296894833048\\
72.875	0.10764	4.82981438301768	4.82981438301768\\
72.875	0.1113	3.87686664811265	3.87686664811265\\
72.875	0.11496	2.97412574361535	2.97412574361535\\
72.875	0.11862	2.1215916695258	2.1215916695258\\
72.875	0.12228	1.31926442584408	1.31926442584408\\
72.875	0.12594	0.567144012570068	0.567144012570068\\
72.875	0.1296	-0.134769570296186	-0.134769570296186\\
72.875	0.13326	-0.786476322754623	-0.786476322754623\\
72.875	0.13692	-1.38797624480534	-1.38797624480534\\
72.875	0.14058	-1.9392693364483	-1.9392693364483\\
72.875	0.14424	-2.44035559768346	-2.44035559768346\\
72.875	0.1479	-2.89123502851088	-2.89123502851088\\
72.875	0.15156	-3.29190762893055	-3.29190762893055\\
72.875	0.15522	-3.64237339894243	-3.64237339894243\\
72.875	0.15888	-3.94263233854657	-3.94263233854657\\
72.875	0.16254	-4.19268444774293	-4.19268444774293\\
72.875	0.1662	-4.39252972653155	-4.39252972653155\\
72.875	0.16986	-4.5421681749124	-4.5421681749124\\
72.875	0.17352	-4.64159979288545	-4.64159979288545\\
72.875	0.17718	-4.69082458045079	-4.69082458045079\\
72.875	0.18084	-4.68984253760836	-4.68984253760836\\
72.875	0.1845	-4.6386536643581	-4.6386536643581\\
72.875	0.18816	-4.53725796070015	-4.53725796070015\\
72.875	0.19182	-4.38565542663443	-4.38565542663443\\
72.875	0.19548	-4.18384606216094	-4.18384606216094\\
72.875	0.19914	-3.93182986727965	-3.93182986727965\\
72.875	0.2028	-3.62960684199062	-3.62960684199062\\
72.875	0.20646	-3.27717698629378	-3.27717698629378\\
72.875	0.21012	-2.87454030018925	-2.87454030018925\\
72.875	0.21378	-2.42169678367696	-2.42169678367696\\
72.875	0.21744	-1.91864643675692	-1.91864643675692\\
72.875	0.2211	-1.36538925942909	-1.36538925942909\\
72.875	0.22476	-0.761925251693455	-0.761925251693455\\
72.875	0.22842	-0.108254413550043	-0.108254413550043\\
72.875	0.23208	0.595623255001058	0.595623255001058\\
72.875	0.23574	1.34970775395996	1.34970775395996\\
72.875	0.2394	2.1539990833266	2.1539990833266\\
72.875	0.24306	3.00849724310099	3.00849724310099\\
72.875	0.24672	3.91320223328314	3.91320223328314\\
72.875	0.25038	4.8681140538731	4.8681140538731\\
72.875	0.25404	5.87323270487083	5.87323270487083\\
72.875	0.2577	6.92855818627629	6.92855818627629\\
72.875	0.26136	8.03409049808957	8.03409049808957\\
72.875	0.26502	9.18982964031053	9.18982964031053\\
72.875	0.26868	10.3957756129392	10.3957756129392\\
72.875	0.27234	11.6519284159758	11.6519284159758\\
72.875	0.276	12.9582880494201	12.9582880494201\\
73.25	0.093	9.57681275374894	9.57681275374894\\
73.25	0.09666	8.40791392775704	8.40791392775704\\
73.25	0.10032	7.28922193217288	7.28922193217288\\
73.25	0.10398	6.22073676699653	6.22073676699653\\
73.25	0.10764	5.20245843222792	5.20245843222792\\
73.25	0.1113	4.23438692786704	4.23438692786704\\
73.25	0.11496	3.31652225391399	3.31652225391399\\
73.25	0.11862	2.44886441036866	2.44886441036866\\
73.25	0.12228	1.6314133972311	1.6314133972311\\
73.25	0.12594	0.864169214501302	0.864169214501302\\
73.25	0.1296	0.147131862179265	0.147131862179265\\
73.25	0.13326	-0.519698659735013	-0.519698659735013\\
73.25	0.13692	-1.13632235124152	-1.13632235124152\\
73.25	0.14058	-1.70273921234025	-1.70273921234025\\
73.25	0.14424	-2.21894924303123	-2.21894924303123\\
73.25	0.1479	-2.68495244331446	-2.68495244331446\\
73.25	0.15156	-3.10074881318992	-3.10074881318992\\
73.25	0.15522	-3.46633835265761	-3.46633835265761\\
73.25	0.15888	-3.78172106171753	-3.78172106171753\\
73.25	0.16254	-4.0468969403697	-4.0468969403697\\
73.25	0.1662	-4.2618659886141	-4.2618659886141\\
73.25	0.16986	-4.42662820645074	-4.42662820645074\\
73.25	0.17352	-4.54118359387963	-4.54118359387963\\
73.25	0.17718	-4.60553215090075	-4.60553215090075\\
73.25	0.18084	-4.61967387751411	-4.61967387751411\\
73.25	0.1845	-4.58360877371969	-4.58360877371969\\
73.25	0.18816	-4.49733683951752	-4.49733683951752\\
73.25	0.19182	-4.36085807490758	-4.36085807490758\\
73.25	0.19548	-4.17417247988988	-4.17417247988988\\
73.25	0.19914	-3.93728005446442	-3.93728005446442\\
73.25	0.2028	-3.65018079863118	-3.65018079863118\\
73.25	0.20646	-3.31287471239018	-3.31287471239018\\
73.25	0.21012	-2.92536179574144	-2.92536179574144\\
73.25	0.21378	-2.48764204868493	-2.48764204868493\\
73.25	0.21744	-1.99971547122067	-1.99971547122067\\
73.25	0.2211	-1.46158206334863	-1.46158206334863\\
73.25	0.22476	-0.873241825068831	-0.873241825068831\\
73.25	0.22842	-0.234694756381259	-0.234694756381259\\
73.25	0.23208	0.454059142714115	0.454059142714115\\
73.25	0.23574	1.19301987221718	1.19301987221718\\
73.25	0.2394	1.98218743212803	1.98218743212803\\
73.25	0.24306	2.82156182244664	2.82156182244664\\
73.25	0.24672	3.71114304317301	3.71114304317301\\
73.25	0.25038	4.65093109430718	4.65093109430718\\
73.25	0.25404	5.64092597584907	5.64092597584907\\
73.25	0.2577	6.68112768779875	6.68112768779875\\
73.25	0.26136	7.77153623015619	7.77153623015619\\
73.25	0.26502	8.91215160292137	8.91215160292137\\
73.25	0.26868	10.1029738060943	10.1029738060943\\
73.25	0.27234	11.344002839675	11.344002839675\\
73.25	0.276	12.6352387036635	12.6352387036635\\
73.625	0.093	10.0198190783381	10.0198190783381\\
73.625	0.09666	8.83579648289042	8.83579648289042\\
73.625	0.10032	7.70198071785047	7.70198071785047\\
73.625	0.10398	6.61837178321828	6.61837178321828\\
73.625	0.10764	5.58496967899389	5.58496967899389\\
73.625	0.1113	4.60177440517723	4.60177440517723\\
73.625	0.11496	3.66878596176834	3.66878596176834\\
73.625	0.11862	2.78600434876723	2.78600434876723\\
73.625	0.12228	1.95342956617388	1.95342956617388\\
73.625	0.12594	1.17106161398827	1.17106161398827\\
73.625	0.1296	0.438900492210422	0.438900492210422\\
73.625	0.13326	-0.24305379915964	-0.24305379915964\\
73.625	0.13692	-0.874801260121927	-0.874801260121927\\
73.625	0.14058	-1.45634189067648	-1.45634189067648\\
73.625	0.14424	-1.98767569082327	-1.98767569082327\\
73.625	0.1479	-2.46880266056228	-2.46880266056228\\
73.625	0.15156	-2.89972279989352	-2.89972279989352\\
73.625	0.15522	-3.28043610881699	-3.28043610881699\\
73.625	0.15888	-3.61094258733275	-3.61094258733275\\
73.625	0.16254	-3.89124223544071	-3.89124223544071\\
73.625	0.1662	-4.1213350531409	-4.1213350531409\\
73.625	0.16986	-4.30122104043338	-4.30122104043338\\
73.625	0.17352	-4.43090019731805	-4.43090019731805\\
73.625	0.17718	-4.51037252379495	-4.51037252379495\\
73.625	0.18084	-4.53963801986409	-4.53963801986409\\
73.625	0.1845	-4.51869668552551	-4.51869668552551\\
73.625	0.18816	-4.44754852077912	-4.44754852077912\\
73.625	0.19182	-4.32619352562497	-4.32619352562497\\
73.625	0.19548	-4.15463170006305	-4.15463170006305\\
73.625	0.19914	-3.93286304409338	-3.93286304409338\\
73.625	0.2028	-3.66088755771604	-3.66088755771604\\
73.625	0.20646	-3.33870524093082	-3.33870524093082\\
73.625	0.21012	-2.96631609373786	-2.96631609373786\\
73.625	0.21378	-2.54372011613714	-2.54372011613714\\
73.625	0.21744	-2.07091730812866	-2.07091730812866\\
73.625	0.2211	-1.5479076697124	-1.5479076697124\\
73.625	0.22476	-0.974691200888387	-0.974691200888387\\
73.625	0.22842	-0.351267901656655	-0.351267901656655\\
73.625	0.23208	0.322362227982879	0.322362227982879\\
73.625	0.23574	1.04619918803016	1.04619918803016\\
73.625	0.2394	1.82024297848523	1.82024297848523\\
73.625	0.24306	2.64449359934805	2.64449359934805\\
73.625	0.24672	3.51895105061864	3.51895105061864\\
73.625	0.25038	4.44361533229697	4.44361533229697\\
73.625	0.25404	5.41848644438308	5.41848644438308\\
73.625	0.2577	6.44356438687691	6.44356438687691\\
73.625	0.26136	7.51884915977857	7.51884915977857\\
73.625	0.26502	8.64434076308797	8.64434076308797\\
73.625	0.26868	9.82003919680511	9.82003919680511\\
73.625	0.27234	11.0459444609301	11.0459444609301\\
73.625	0.276	12.3220565554628	12.3220565554628\\
74	0.093	10.472692600483	10.472692600483\\
74	0.09666	9.27354623557951	9.27354623557951\\
74	0.10032	8.12460670108375	8.12460670108375\\
74	0.10398	7.0258739969958	7.0258739969958\\
74	0.10764	5.97734812331557	5.97734812331557\\
74	0.1113	4.97902908004313	4.97902908004313\\
74	0.11496	4.03091686717845	4.03091686717845\\
74	0.11862	3.13301148472152	3.13301148472152\\
74	0.12228	2.28531293267237	2.28531293267237\\
74	0.12594	1.48782121103098	1.48782121103098\\
74	0.1296	0.740536319797343	0.740536319797343\\
74	0.13326	0.0434582589714694	0.0434582589714694\\
74	0.13692	-0.60341297144663	-0.60341297144663\\
74	0.14058	-1.20007737145699	-1.20007737145699\\
74	0.14424	-1.74653494105954	-1.74653494105954\\
74	0.1479	-2.24278568025439	-2.24278568025439\\
74	0.15156	-2.68882958904141	-2.68882958904141\\
74	0.15522	-3.08466666742073	-3.08466666742073\\
74	0.15888	-3.43029691539227	-3.43029691539227\\
74	0.16254	-3.72572033295601	-3.72572033295601\\
74	0.1662	-3.97093692011204	-3.97093692011204\\
74	0.16986	-4.16594667686024	-4.16594667686024\\
74	0.17352	-4.31074960320075	-4.31074960320075\\
74	0.17718	-4.40534569913344	-4.40534569913344\\
74	0.18084	-4.44973496465843	-4.44973496465843\\
74	0.1845	-4.44391739977557	-4.44391739977557\\
74	0.18816	-4.38789300448502	-4.38789300448502\\
74	0.19182	-4.28166177878671	-4.28166177878671\\
74	0.19548	-4.12522372268057	-4.12522372268057\\
74	0.19914	-3.91857883616669	-3.91857883616669\\
74	0.2028	-3.66172711924507	-3.66172711924507\\
74	0.20646	-3.3546685719157	-3.3546685719157\\
74	0.21012	-2.99740319417857	-2.99740319417857\\
74	0.21378	-2.58993098603364	-2.58993098603364\\
74	0.21744	-2.13225194748094	-2.13225194748094\\
74	0.2211	-1.62436607852052	-1.62436607852052\\
74	0.22476	-1.06627337915235	-1.06627337915235\\
74	0.22842	-0.457973849376344	-0.457973849376344\\
74	0.23208	0.200532510807406	0.200532510807406\\
74	0.23574	0.909245701398902	0.909245701398902\\
74	0.2394	1.66816572239813	1.66816572239813\\
74	0.24306	2.47729257380512	2.47729257380512\\
74	0.24672	3.33662625561992	3.33662625561992\\
74	0.25038	4.24616676784247	4.24616676784247\\
74	0.25404	5.20591411047279	5.20591411047279\\
74	0.2577	6.21586828351084	6.21586828351084\\
74	0.26136	7.27602928695666	7.27602928695666\\
74	0.26502	8.38639712081027	8.38639712081027\\
74	0.26868	9.54697178507163	9.54697178507163\\
74	0.27234	10.7577532797407	10.7577532797407\\
74	0.276	12.0187416048176	12.0187416048176\\
};
\end{axis}

\begin{axis}[%
width=2.616cm,
height=2.517cm,
at={(6.484cm,10.49cm)},
scale only axis,
xmin=56,
xmax=74,
tick align=outside,
xlabel style={font=\color{white!15!black}},
xlabel={$L_{cut}$},
ymin=0.093,
ymax=0.276,
ylabel style={font=\color{white!15!black}},
ylabel={$D_{rlx}$},
zmin=-622.950676493194,
zmax=59.9792562086666,
zlabel style={font=\color{white!15!black}},
zlabel={$c$},
view={-140}{50},
axis background/.style={fill=white},
xmajorgrids,
ymajorgrids,
zmajorgrids,
legend style={at={(1.03,1)}, anchor=north west, legend cell align=left, align=left, draw=white!15!black}
]
\addplot3[only marks, mark=*, mark options={}, mark size=1.5000pt, color=mycolor1, fill=mycolor1] table[row sep=crcr]{%
x	y	z\\
74	0.123	-43.5304434229015\\
72	0.113	-56.7271164371914\\
61	0.095	1.52122667321791\\
56	0.093	2.3511157265787\\
};
\addlegendentry{data1}

\addplot3[only marks, mark=*, mark options={}, mark size=1.5000pt, color=mycolor2, fill=mycolor2] table[row sep=crcr]{%
x	y	z\\
67	0.276	-331.679044356687\\
66	0.255	-201.012579496977\\
62	0.209	-81.0349638953616\\
57	0.193	-85.4906913253865\\
};
\addlegendentry{data2}

\addplot3[only marks, mark=*, mark options={}, mark size=1.5000pt, color=black, fill=black] table[row sep=crcr]{%
x	y	z\\
69	0.104	-45.0189554417853\\
};
\addlegendentry{data3}

\addplot3[only marks, mark=*, mark options={}, mark size=1.5000pt, color=black, fill=black] table[row sep=crcr]{%
x	y	z\\
64	0.23	-125.459643885363\\
};
\addlegendentry{data4}


\addplot3[%
surf,
fill opacity=0.7, shader=interp, colormap={mymap}{[1pt] rgb(0pt)=(1,0.905882,0); rgb(1pt)=(1,0.901964,0); rgb(2pt)=(1,0.898051,0); rgb(3pt)=(1,0.894144,0); rgb(4pt)=(1,0.890243,0); rgb(5pt)=(1,0.886349,0); rgb(6pt)=(1,0.88246,0); rgb(7pt)=(1,0.878577,0); rgb(8pt)=(1,0.8747,0); rgb(9pt)=(1,0.870829,0); rgb(10pt)=(1,0.866964,0); rgb(11pt)=(1,0.863106,0); rgb(12pt)=(1,0.859253,0); rgb(13pt)=(1,0.855406,0); rgb(14pt)=(1,0.851566,0); rgb(15pt)=(1,0.847732,0); rgb(16pt)=(1,0.843903,0); rgb(17pt)=(1,0.840081,0); rgb(18pt)=(1,0.836265,0); rgb(19pt)=(1,0.832455,0); rgb(20pt)=(1,0.828652,0); rgb(21pt)=(1,0.824854,0); rgb(22pt)=(1,0.821063,0); rgb(23pt)=(1,0.817278,0); rgb(24pt)=(1,0.8135,0); rgb(25pt)=(1,0.809727,0); rgb(26pt)=(1,0.805961,0); rgb(27pt)=(1,0.8022,0); rgb(28pt)=(1,0.798445,0); rgb(29pt)=(1,0.794696,0); rgb(30pt)=(1,0.790953,0); rgb(31pt)=(1,0.787215,0); rgb(32pt)=(1,0.783484,0); rgb(33pt)=(1,0.779758,0); rgb(34pt)=(1,0.776038,0); rgb(35pt)=(1,0.772324,0); rgb(36pt)=(1,0.768615,0); rgb(37pt)=(1,0.764913,0); rgb(38pt)=(1,0.761217,0); rgb(39pt)=(1,0.757527,0); rgb(40pt)=(1,0.753843,0); rgb(41pt)=(1,0.750165,0); rgb(42pt)=(1,0.746493,0); rgb(43pt)=(1,0.742827,0); rgb(44pt)=(1,0.739167,0); rgb(45pt)=(1,0.735514,0); rgb(46pt)=(1,0.731867,0); rgb(47pt)=(1,0.728226,0); rgb(48pt)=(1,0.724591,0); rgb(49pt)=(1,0.720963,0); rgb(50pt)=(1,0.717341,0); rgb(51pt)=(1,0.713725,0); rgb(52pt)=(0.999994,0.710077,0); rgb(53pt)=(0.999974,0.706363,0); rgb(54pt)=(0.999942,0.702592,0); rgb(55pt)=(0.999898,0.698775,0); rgb(56pt)=(0.999841,0.694921,0); rgb(57pt)=(0.999771,0.691039,0); rgb(58pt)=(0.99969,0.687139,0); rgb(59pt)=(0.999596,0.68323,0); rgb(60pt)=(0.99949,0.679323,0); rgb(61pt)=(0.999372,0.675427,0); rgb(62pt)=(0.999242,0.67155,0); rgb(63pt)=(0.9991,0.667704,0); rgb(64pt)=(0.998946,0.663897,0); rgb(65pt)=(0.998781,0.660138,0); rgb(66pt)=(0.998605,0.656439,0); rgb(67pt)=(0.998416,0.652807,0); rgb(68pt)=(0.998217,0.649253,0); rgb(69pt)=(0.998006,0.645786,0); rgb(70pt)=(0.997785,0.642416,0); rgb(71pt)=(0.997552,0.639152,0); rgb(72pt)=(0.997308,0.636004,0); rgb(73pt)=(0.997053,0.632982,0); rgb(74pt)=(0.996788,0.630095,0); rgb(75pt)=(0.996512,0.627352,0); rgb(76pt)=(0.996226,0.624763,0); rgb(77pt)=(0.995851,0.622329,0); rgb(78pt)=(0.99494,0.619997,0); rgb(79pt)=(0.99345,0.617753,0); rgb(80pt)=(0.991419,0.61559,0); rgb(81pt)=(0.988885,0.613503,0); rgb(82pt)=(0.985886,0.611486,0); rgb(83pt)=(0.98246,0.609532,0); rgb(84pt)=(0.978643,0.607636,0); rgb(85pt)=(0.974475,0.605791,0); rgb(86pt)=(0.969992,0.603992,0); rgb(87pt)=(0.965232,0.602233,0); rgb(88pt)=(0.960233,0.600507,0); rgb(89pt)=(0.955033,0.598808,0); rgb(90pt)=(0.949669,0.59713,0); rgb(91pt)=(0.94418,0.595468,0); rgb(92pt)=(0.938602,0.593815,0); rgb(93pt)=(0.932974,0.592166,0); rgb(94pt)=(0.927333,0.590513,0); rgb(95pt)=(0.921717,0.588852,0); rgb(96pt)=(0.916164,0.587176,0); rgb(97pt)=(0.910711,0.585479,0); rgb(98pt)=(0.905397,0.583755,0); rgb(99pt)=(0.900258,0.581999,0); rgb(100pt)=(0.895333,0.580203,0); rgb(101pt)=(0.890659,0.578362,0); rgb(102pt)=(0.886275,0.576471,0); rgb(103pt)=(0.882047,0.574545,0); rgb(104pt)=(0.877819,0.572608,0); rgb(105pt)=(0.873592,0.57066,0); rgb(106pt)=(0.869366,0.568701,0); rgb(107pt)=(0.865143,0.566733,0); rgb(108pt)=(0.860924,0.564756,0); rgb(109pt)=(0.856708,0.562771,0); rgb(110pt)=(0.852497,0.560778,0); rgb(111pt)=(0.848292,0.558779,0); rgb(112pt)=(0.844092,0.556774,0); rgb(113pt)=(0.8399,0.554763,0); rgb(114pt)=(0.835716,0.552749,0); rgb(115pt)=(0.831541,0.55073,0); rgb(116pt)=(0.827374,0.548709,0); rgb(117pt)=(0.823219,0.546686,0); rgb(118pt)=(0.819074,0.54466,0); rgb(119pt)=(0.81494,0.542635,0); rgb(120pt)=(0.81082,0.540609,0); rgb(121pt)=(0.806712,0.538584,0); rgb(122pt)=(0.802619,0.53656,0); rgb(123pt)=(0.798541,0.534539,0); rgb(124pt)=(0.794478,0.532521,0); rgb(125pt)=(0.790431,0.530506,0); rgb(126pt)=(0.786402,0.528496,0); rgb(127pt)=(0.782391,0.526491,0); rgb(128pt)=(0.77841,0.524489,0); rgb(129pt)=(0.774523,0.522478,0); rgb(130pt)=(0.770731,0.520455,0); rgb(131pt)=(0.767022,0.518424,0); rgb(132pt)=(0.763384,0.516385,0); rgb(133pt)=(0.759804,0.514339,0); rgb(134pt)=(0.756272,0.51229,0); rgb(135pt)=(0.752775,0.510237,0); rgb(136pt)=(0.749302,0.508182,0); rgb(137pt)=(0.74584,0.506128,0); rgb(138pt)=(0.742378,0.504075,0); rgb(139pt)=(0.738904,0.502025,0); rgb(140pt)=(0.735406,0.499979,0); rgb(141pt)=(0.731872,0.49794,0); rgb(142pt)=(0.72829,0.495909,0); rgb(143pt)=(0.724649,0.493887,0); rgb(144pt)=(0.720936,0.491875,0); rgb(145pt)=(0.71714,0.489876,0); rgb(146pt)=(0.713249,0.487891,0); rgb(147pt)=(0.709251,0.485921,0); rgb(148pt)=(0.705134,0.483968,0); rgb(149pt)=(0.700887,0.482033,0); rgb(150pt)=(0.696497,0.480118,0); rgb(151pt)=(0.691952,0.478225,0); rgb(152pt)=(0.687242,0.476355,0); rgb(153pt)=(0.682353,0.47451,0); rgb(154pt)=(0.677195,0.472696,0); rgb(155pt)=(0.6717,0.470916,0); rgb(156pt)=(0.665891,0.469169,0); rgb(157pt)=(0.659791,0.46745,0); rgb(158pt)=(0.653423,0.465756,0); rgb(159pt)=(0.64681,0.464084,0); rgb(160pt)=(0.639976,0.462432,0); rgb(161pt)=(0.632943,0.460795,0); rgb(162pt)=(0.625734,0.459171,0); rgb(163pt)=(0.618373,0.457556,0); rgb(164pt)=(0.610882,0.455948,0); rgb(165pt)=(0.603284,0.454343,0); rgb(166pt)=(0.595604,0.452737,0); rgb(167pt)=(0.587863,0.451129,0); rgb(168pt)=(0.580084,0.449514,0); rgb(169pt)=(0.572292,0.447889,0); rgb(170pt)=(0.564508,0.446252,0); rgb(171pt)=(0.556756,0.444599,0); rgb(172pt)=(0.549059,0.442927,0); rgb(173pt)=(0.54144,0.441232,0); rgb(174pt)=(0.533922,0.439512,0); rgb(175pt)=(0.526529,0.437764,0); rgb(176pt)=(0.519282,0.435983,0); rgb(177pt)=(0.512206,0.434168,0); rgb(178pt)=(0.505323,0.432315,0); rgb(179pt)=(0.498628,0.430422,3.92506e-06); rgb(180pt)=(0.491973,0.428504,3.49981e-05); rgb(181pt)=(0.485331,0.426562,9.63073e-05); rgb(182pt)=(0.478704,0.424596,0.000186979); rgb(183pt)=(0.472096,0.422609,0.000306141); rgb(184pt)=(0.465508,0.420599,0.00045292); rgb(185pt)=(0.458942,0.418567,0.000626441); rgb(186pt)=(0.452401,0.416515,0.000825833); rgb(187pt)=(0.445885,0.414441,0.00105022); rgb(188pt)=(0.439399,0.412348,0.00129873); rgb(189pt)=(0.432942,0.410234,0.00157049); rgb(190pt)=(0.426518,0.408102,0.00186463); rgb(191pt)=(0.420129,0.40595,0.00218028); rgb(192pt)=(0.413777,0.40378,0.00251655); rgb(193pt)=(0.407464,0.401592,0.00287258); rgb(194pt)=(0.401191,0.399386,0.00324749); rgb(195pt)=(0.394962,0.397164,0.00364042); rgb(196pt)=(0.388777,0.394925,0.00405048); rgb(197pt)=(0.38264,0.39267,0.00447681); rgb(198pt)=(0.376552,0.390399,0.00491852); rgb(199pt)=(0.370516,0.388113,0.00537476); rgb(200pt)=(0.364532,0.385812,0.00584464); rgb(201pt)=(0.358605,0.383497,0.00632729); rgb(202pt)=(0.352735,0.381168,0.00682184); rgb(203pt)=(0.346925,0.378826,0.00732741); rgb(204pt)=(0.341176,0.376471,0.00784314); rgb(205pt)=(0.335485,0.374093,0.00847245); rgb(206pt)=(0.329843,0.371682,0.00930909); rgb(207pt)=(0.324249,0.369242,0.0103377); rgb(208pt)=(0.318701,0.366772,0.0115428); rgb(209pt)=(0.313198,0.364275,0.0129091); rgb(210pt)=(0.307739,0.361753,0.0144211); rgb(211pt)=(0.302322,0.359206,0.0160634); rgb(212pt)=(0.296945,0.356637,0.0178207); rgb(213pt)=(0.291607,0.354048,0.0196776); rgb(214pt)=(0.286307,0.35144,0.0216186); rgb(215pt)=(0.281043,0.348814,0.0236284); rgb(216pt)=(0.275813,0.346172,0.0256916); rgb(217pt)=(0.270616,0.343517,0.0277927); rgb(218pt)=(0.265451,0.340849,0.0299163); rgb(219pt)=(0.260317,0.33817,0.0320472); rgb(220pt)=(0.25521,0.335482,0.0341698); rgb(221pt)=(0.250131,0.332786,0.0362688); rgb(222pt)=(0.245078,0.330085,0.0383287); rgb(223pt)=(0.240048,0.327379,0.0403343); rgb(224pt)=(0.235042,0.324671,0.04227); rgb(225pt)=(0.230056,0.321962,0.0441205); rgb(226pt)=(0.22509,0.319254,0.0458704); rgb(227pt)=(0.220142,0.316548,0.0475043); rgb(228pt)=(0.215212,0.313846,0.0490067); rgb(229pt)=(0.210296,0.311149,0.0503624); rgb(230pt)=(0.205395,0.308459,0.0515759); rgb(231pt)=(0.200514,0.305763,0.052757); rgb(232pt)=(0.195655,0.303061,0.0539242); rgb(233pt)=(0.190817,0.300353,0.0550763); rgb(234pt)=(0.186001,0.297639,0.0562123); rgb(235pt)=(0.181207,0.294918,0.0573313); rgb(236pt)=(0.176434,0.292191,0.0584321); rgb(237pt)=(0.171685,0.289458,0.0595136); rgb(238pt)=(0.166957,0.286719,0.060575); rgb(239pt)=(0.162252,0.283973,0.0616151); rgb(240pt)=(0.15757,0.281221,0.0626328); rgb(241pt)=(0.152911,0.278463,0.0636271); rgb(242pt)=(0.148275,0.275699,0.0645971); rgb(243pt)=(0.143663,0.272929,0.0655416); rgb(244pt)=(0.139074,0.270152,0.0664596); rgb(245pt)=(0.134508,0.26737,0.06735); rgb(246pt)=(0.129967,0.264581,0.0682118); rgb(247pt)=(0.125449,0.261787,0.0690441); rgb(248pt)=(0.120956,0.258986,0.0698456); rgb(249pt)=(0.116487,0.25618,0.0706154); rgb(250pt)=(0.112043,0.253367,0.0713525); rgb(251pt)=(0.107623,0.250549,0.0720557); rgb(252pt)=(0.103229,0.247724,0.0727241); rgb(253pt)=(0.0988592,0.244894,0.0733566); rgb(254pt)=(0.0945149,0.242058,0.0739522); rgb(255pt)=(0.0901961,0.239216,0.0745098)}, mesh/rows=49]
table[row sep=crcr, point meta=\thisrow{c}] {%
%
x	y	z	c\\
56	0.093	5.23341873629511	5.23341873629511\\
56	0.09666	10.9330290965688	10.9330290965688\\
56	0.10032	15.8871989562354	15.8871989562354\\
56	0.10398	20.0959283152955	20.0959283152955\\
56	0.10764	23.5592171737487	23.5592171737487\\
56	0.1113	26.2770655315956	26.2770655315956\\
56	0.11496	28.2494733888357	28.2494733888357\\
56	0.11862	29.4764407454691	29.4764407454691\\
56	0.12228	29.9579676014959	29.9579676014959\\
56	0.12594	29.694053956916	29.694053956916\\
56	0.1296	28.6846998117295	28.6846998117295\\
56	0.13326	26.9299051659363	26.9299051659363\\
56	0.13692	24.4296700195364	24.4296700195364\\
56	0.14058	21.1839943725302	21.1839943725302\\
56	0.14424	17.1928782249169	17.1928782249169\\
56	0.1479	12.4563215766971	12.4563215766971\\
56	0.15156	6.97432442787056	6.97432442787056\\
56	0.15522	0.746886778437329	0.746886778437329\\
56	0.15888	-6.22599137160228	-6.22599137160228\\
56	0.16254	-13.9443100222487	-13.9443100222487\\
56	0.1662	-22.4080691735019	-22.4080691735019\\
56	0.16986	-31.6172688253617	-31.6172688253617\\
56	0.17352	-41.5719089778282	-41.5719089778282\\
56	0.17718	-52.2719896309013	-52.2719896309013\\
56	0.18084	-63.7175107845811	-63.7175107845811\\
56	0.1845	-75.9084724388676	-75.9084724388676\\
56	0.18816	-88.844874593761	-88.844874593761\\
56	0.19182	-102.526717249261	-102.526717249261\\
56	0.19548	-116.954000405367	-116.954000405367\\
56	0.19914	-132.12672406208	-132.12672406208\\
56	0.2028	-148.0448882194	-148.0448882194\\
56	0.20646	-164.708492877327	-164.708492877327\\
56	0.21012	-182.11753803586	-182.11753803586\\
56	0.21378	-200.272023695	-200.272023695\\
56	0.21744	-219.171949854746	-219.171949854746\\
56	0.2211	-238.817316515099	-238.817316515099\\
56	0.22476	-259.208123676058	-259.208123676058\\
56	0.22842	-280.344371337625	-280.344371337625\\
56	0.23208	-302.226059499798	-302.226059499798\\
56	0.23574	-324.853188162578	-324.853188162578\\
56	0.2394	-348.225757325964	-348.225757325964\\
56	0.24306	-372.343766989957	-372.343766989957\\
56	0.24672	-397.207217154557	-397.207217154557\\
56	0.25038	-422.816107819763	-422.816107819763\\
56	0.25404	-449.170438985576	-449.170438985576\\
56	0.2577	-476.270210651996	-476.270210651996\\
56	0.26136	-504.115422819022	-504.115422819022\\
56	0.26502	-532.706075486655	-532.706075486655\\
56	0.26868	-562.042168654895	-562.042168654895\\
56	0.27234	-592.123702323741	-592.123702323741\\
56	0.276	-622.950676493194	-622.950676493194\\
56.375	0.093	5.13922530544937	5.13922530544937\\
56.375	0.09666	11.0879005393737	11.0879005393737\\
56.375	0.10032	16.2911352726913	16.2911352726913\\
56.375	0.10398	20.7489295054023	20.7489295054023\\
56.375	0.10764	24.4612832375064	24.4612832375064\\
56.375	0.1113	27.4281964690043	27.4281964690043\\
56.375	0.11496	29.649669199895	29.649669199895\\
56.375	0.11862	31.1257014301796	31.1257014301796\\
56.375	0.12228	31.8562931598571	31.8562931598571\\
56.375	0.12594	31.8414443889279	31.8414443889279\\
56.375	0.1296	31.0811551173925	31.0811551173925\\
56.375	0.13326	29.5754253452505	29.5754253452505\\
56.375	0.13692	27.3242550725013	27.3242550725013\\
56.375	0.14058	24.3276442991457	24.3276442991457\\
56.375	0.14424	20.5855930251831	20.5855930251831\\
56.375	0.1479	16.0981012506145	16.0981012506145\\
56.375	0.15156	10.8651689754387	10.8651689754387\\
56.375	0.15522	4.88679619965615	4.88679619965615\\
56.375	0.15888	-1.83701707673276	-1.83701707673276\\
56.375	0.16254	-9.30627085372805	-9.30627085372805\\
56.375	0.1662	-17.5209651313305	-17.5209651313305\\
56.375	0.16986	-26.4810999095397	-26.4810999095397\\
56.375	0.17352	-36.186675188355	-36.186675188355\\
56.375	0.17718	-46.6376909677774	-46.6376909677774\\
56.375	0.18084	-57.8341472478065	-57.8341472478065\\
56.375	0.1845	-69.7760440284419	-69.7760440284419\\
56.375	0.18816	-82.4633813096841	-82.4633813096841\\
56.375	0.19182	-95.8961590915331	-95.8961590915331\\
56.375	0.19548	-110.074377373988	-110.074377373988\\
56.375	0.19914	-124.998036157051	-124.998036157051\\
56.375	0.2028	-140.66713544072	-140.66713544072\\
56.375	0.20646	-157.081675224996	-157.081675224996\\
56.375	0.21012	-174.241655509878	-174.241655509878\\
56.375	0.21378	-192.147076295367	-192.147076295367\\
56.375	0.21744	-210.797937581462	-210.797937581462\\
56.375	0.2211	-230.194239368164	-230.194239368164\\
56.375	0.22476	-250.335981655473	-250.335981655473\\
56.375	0.22842	-271.223164443388	-271.223164443388\\
56.375	0.23208	-292.855787731911	-292.855787731911\\
56.375	0.23574	-315.23385152104	-315.23385152104\\
56.375	0.2394	-338.357355810776	-338.357355810776\\
56.375	0.24306	-362.226300601117	-362.226300601117\\
56.375	0.24672	-386.840685892066	-386.840685892066\\
56.375	0.25038	-412.200511683622	-412.200511683622\\
56.375	0.25404	-438.305777975784	-438.305777975784\\
56.375	0.2577	-465.156484768553	-465.156484768553\\
56.375	0.26136	-492.752632061929	-492.752632061929\\
56.375	0.26502	-521.09421985591	-521.09421985591\\
56.375	0.26868	-550.181248150499	-550.181248150499\\
56.375	0.27234	-580.013716945694	-580.013716945694\\
56.375	0.276	-610.591626241497	-610.591626241497\\
56.75	0.093	4.89453996364705	4.89453996364705\\
56.75	0.09666	11.0922800712223	11.0922800712223\\
56.75	0.10032	16.5445796781906	16.5445796781906\\
56.75	0.10398	21.2514387845525	21.2514387845525\\
56.75	0.10764	25.2128573903078	25.2128573903078\\
56.75	0.1113	28.4288354954563	28.4288354954563\\
56.75	0.11496	30.8993730999982	30.8993730999982\\
56.75	0.11862	32.6244702039331	32.6244702039331\\
56.75	0.12228	33.6041268072617	33.6041268072617\\
56.75	0.12594	33.8383429099836	33.8383429099836\\
56.75	0.1296	33.327118512099	33.327118512099\\
56.75	0.13326	32.0704536136076	32.0704536136076\\
56.75	0.13692	30.0683482145091	30.0683482145091\\
56.75	0.14058	27.3208023148047	27.3208023148047\\
56.75	0.14424	23.8278159144928	23.8278159144928\\
56.75	0.1479	19.5893890135749	19.5893890135749\\
56.75	0.15156	14.6055216120502	14.6055216120502\\
56.75	0.15522	8.87621370991837	8.87621370991837\\
56.75	0.15888	2.40146530718062	2.40146530718062\\
56.75	0.16254	-4.81872359616443	-4.81872359616443\\
56.75	0.1662	-12.7843530001157	-12.7843530001157\\
56.75	0.16986	-21.4954229046738	-21.4954229046738\\
56.75	0.17352	-30.9519333098384	-30.9519333098384\\
56.75	0.17718	-41.1538842156097	-41.1538842156097\\
56.75	0.18084	-52.1012756219881	-52.1012756219881\\
56.75	0.1845	-63.7941075289727	-63.7941075289727\\
56.75	0.18816	-76.2323799365643	-76.2323799365643\\
56.75	0.19182	-89.4160928447621	-89.4160928447621\\
56.75	0.19548	-103.345246253567	-103.345246253567\\
56.75	0.19914	-118.019840162978	-118.019840162978\\
56.75	0.2028	-133.439874572996	-133.439874572996\\
56.75	0.20646	-149.605349483621	-149.605349483621\\
56.75	0.21012	-166.516264894853	-166.516264894853\\
56.75	0.21378	-184.17262080669	-184.17262080669\\
56.75	0.21744	-202.574417219135	-202.574417219135\\
56.75	0.2211	-221.721654132186	-221.721654132186\\
56.75	0.22476	-241.614331545845	-241.614331545845\\
56.75	0.22842	-262.252449460109	-262.252449460109\\
56.75	0.23208	-283.63600787498	-283.63600787498\\
56.75	0.23574	-305.765006790459	-305.765006790459\\
56.75	0.2394	-328.639446206543	-328.639446206543\\
56.75	0.24306	-352.259326123234	-352.259326123234\\
56.75	0.24672	-376.624646540532	-376.624646540532\\
56.75	0.25038	-401.735407458437	-401.735407458437\\
56.75	0.25404	-427.591608876949	-427.591608876949\\
56.75	0.2577	-454.193250796067	-454.193250796067\\
56.75	0.26136	-481.540333215791	-481.540333215791\\
56.75	0.26502	-509.632856136122	-509.632856136122\\
56.75	0.26868	-538.47081955706	-538.47081955706\\
56.75	0.27234	-568.054223478604	-568.054223478604\\
56.75	0.276	-598.383067900756	-598.383067900756\\
57.125	0.093	4.49936271088927	4.49936271088927\\
57.125	0.09666	10.9461676921152	10.9461676921152\\
57.125	0.10032	16.6475321727347	16.6475321727347\\
57.125	0.10398	21.6034561527471	21.6034561527471\\
57.125	0.10764	25.813939632153	25.813939632153\\
57.125	0.1113	29.2789826109532	29.2789826109532\\
57.125	0.11496	31.9985850891453	31.9985850891453\\
57.125	0.11862	33.9727470667318	33.9727470667318\\
57.125	0.12228	35.2014685437111	35.2014685437111\\
57.125	0.12594	35.6847495200837	35.6847495200837\\
57.125	0.1296	35.4225899958498	35.4225899958498\\
57.125	0.13326	34.4149899710091	34.4149899710091\\
57.125	0.13692	32.6619494455617	32.6619494455617\\
57.125	0.14058	30.163468419508	30.163468419508\\
57.125	0.14424	26.9195468928473	26.9195468928473\\
57.125	0.1479	22.93018486558	22.93018486558\\
57.125	0.15156	18.1953823377061	18.1953823377061\\
57.125	0.15522	12.7151393092258	12.7151393092258\\
57.125	0.15888	6.48945578013831	6.48945578013831\\
57.125	0.16254	-0.481668249555582	-0.481668249555582\\
57.125	0.1662	-8.1982327798562	-8.1982327798562\\
57.125	0.16986	-16.6602378107635	-16.6602378107635\\
57.125	0.17352	-25.867683342277	-25.867683342277\\
57.125	0.17718	-35.8205693743976	-35.8205693743976\\
57.125	0.18084	-46.5188959071248	-46.5188959071248\\
57.125	0.1845	-57.9626629404588	-57.9626629404588\\
57.125	0.18816	-70.1518704743992	-70.1518704743992\\
57.125	0.19182	-83.0865185089464	-83.0865185089464\\
57.125	0.19548	-96.7666070441003	-96.7666070441003\\
57.125	0.19914	-111.19213607986	-111.19213607986\\
57.125	0.2028	-126.363105616228	-126.363105616228\\
57.125	0.20646	-142.279515653202	-142.279515653202\\
57.125	0.21012	-158.941366190782	-158.941366190782\\
57.125	0.21378	-176.348657228969	-176.348657228969\\
57.125	0.21744	-194.501388767763	-194.501388767763\\
57.125	0.2211	-213.399560807163	-213.399560807163\\
57.125	0.22476	-233.043173347171	-233.043173347171\\
57.125	0.22842	-253.432226387785	-253.432226387785\\
57.125	0.23208	-274.566719929005	-274.566719929005\\
57.125	0.23574	-296.446653970832	-296.446653970832\\
57.125	0.2394	-319.072028513266	-319.072028513266\\
57.125	0.24306	-342.442843556307	-342.442843556307\\
57.125	0.24672	-366.559099099954	-366.559099099954\\
57.125	0.25038	-391.420795144207	-391.420795144207\\
57.125	0.25404	-417.027931689068	-417.027931689068\\
57.125	0.2577	-443.380508734535	-443.380508734535\\
57.125	0.26136	-470.478526280609	-470.478526280609\\
57.125	0.26502	-498.321984327289	-498.321984327289\\
57.125	0.26868	-526.910882874576	-526.910882874576\\
57.125	0.27234	-556.24522192247	-556.24522192247\\
57.125	0.276	-586.32500147097	-586.32500147097\\
57.5	0.093	3.95369354717513	3.95369354717513\\
57.5	0.09666	10.649563402052	10.649563402052\\
57.5	0.10032	16.5999927563221	16.5999927563221\\
57.5	0.10398	21.8049816099859	21.8049816099859\\
57.5	0.10764	26.2645299630425	26.2645299630425\\
57.5	0.1113	29.978637815493	29.978637815493\\
57.5	0.11496	32.9473051673363	32.9473051673363\\
57.5	0.11862	35.1705320185734	35.1705320185734\\
57.5	0.12228	36.6483183692039	36.6483183692039\\
57.5	0.12594	37.3806642192272	37.3806642192272\\
57.5	0.1296	37.3675695686444	37.3675695686444\\
57.5	0.13326	36.6090344174544	36.6090344174544\\
57.5	0.13692	35.1050587656582	35.1050587656582\\
57.5	0.14058	32.8556426132552	32.8556426132552\\
57.5	0.14424	29.8607859602452	29.8607859602452\\
57.5	0.1479	26.1204888066286	26.1204888066286\\
57.5	0.15156	21.6347511524058	21.6347511524058\\
57.5	0.15522	16.4035729975762	16.4035729975762\\
57.5	0.15888	10.4269543421399	10.4269543421399\\
57.5	0.16254	3.70489518609713	3.70489518609713\\
57.5	0.1662	-3.76260447055279	-3.76260447055279\\
57.5	0.16986	-11.9755446278094	-11.9755446278094\\
57.5	0.17352	-20.9339252856722	-20.9339252856722\\
57.5	0.17718	-30.6377464441421	-30.6377464441421\\
57.5	0.18084	-41.0870081032182	-41.0870081032182\\
57.5	0.1845	-52.2817102629015	-52.2817102629015\\
57.5	0.18816	-64.2218529231911	-64.2218529231911\\
57.5	0.19182	-76.9074360840872	-76.9074360840872\\
57.5	0.19548	-90.3384597455904	-90.3384597455904\\
57.5	0.19914	-104.5149239077	-104.5149239077\\
57.5	0.2028	-119.436828570416	-119.436828570416\\
57.5	0.20646	-135.10417373374	-135.10417373374\\
57.5	0.21012	-151.516959397669	-151.516959397669\\
57.5	0.21378	-168.675185562205	-168.675185562205\\
57.5	0.21744	-186.578852227348	-186.578852227348\\
57.5	0.2211	-205.227959393098	-205.227959393098\\
57.5	0.22476	-224.622507059454	-224.622507059454\\
57.5	0.22842	-244.762495226417	-244.762495226417\\
57.5	0.23208	-265.647923893986	-265.647923893986\\
57.5	0.23574	-287.278793062163	-287.278793062163\\
57.5	0.2394	-309.655102730946	-309.655102730946\\
57.5	0.24306	-332.776852900335	-332.776852900335\\
57.5	0.24672	-356.644043570331	-356.644043570331\\
57.5	0.25038	-381.256674740934	-381.256674740934\\
57.5	0.25404	-406.614746412144	-406.614746412144\\
57.5	0.2577	-432.718258583961	-432.718258583961\\
57.5	0.26136	-459.567211256383	-459.567211256383\\
57.5	0.26502	-487.161604429413	-487.161604429413\\
57.5	0.26868	-515.501438103048	-515.501438103048\\
57.5	0.27234	-544.586712277292	-544.586712277292\\
57.5	0.276	-574.417426952141	-574.417426952141\\
57.875	0.093	3.25753247250577	3.25753247250577\\
57.875	0.09666	10.2024672010336	10.2024672010336\\
57.875	0.10032	16.4019614289546	16.4019614289546\\
57.875	0.10398	21.8560151562691	21.8560151562691\\
57.875	0.10764	26.5646283829764	26.5646283829764\\
57.875	0.1113	30.527801109078	30.527801109078\\
57.875	0.11496	33.745533334572	33.745533334572\\
57.875	0.11862	36.2178250594603	36.2178250594603\\
57.875	0.12228	37.9446762837414	37.9446762837414\\
57.875	0.12594	38.9260870074159	38.9260870074159\\
57.875	0.1296	39.1620572304834	39.1620572304834\\
57.875	0.13326	38.6525869529445	38.6525869529445\\
57.875	0.13692	37.397676174799	37.397676174799\\
57.875	0.14058	35.3973248960472	35.3973248960472\\
57.875	0.14424	32.6515331166878	32.6515331166878\\
57.875	0.1479	29.1603008367224	29.1603008367224\\
57.875	0.15156	24.9236280561503	24.9236280561503\\
57.875	0.15522	19.9415147749714	19.9415147749714\\
57.875	0.15888	14.2139609931858	14.2139609931858\\
57.875	0.16254	7.74096671079417	7.74096671079417\\
57.875	0.1662	0.522531927794944	0.522531927794944\\
57.875	0.16986	-7.44134335581055	-7.44134335581055\\
57.875	0.17352	-16.1506591400226	-16.1506591400226\\
57.875	0.17718	-25.6054154248418	-25.6054154248418\\
57.875	0.18084	-35.8056122102668	-35.8056122102668\\
57.875	0.1845	-46.7512494962989	-46.7512494962989\\
57.875	0.18816	-58.4423272829379	-58.4423272829379\\
57.875	0.19182	-70.8788455701832	-70.8788455701832\\
57.875	0.19548	-84.0608043580357	-84.0608043580357\\
57.875	0.19914	-97.988203646494	-97.988203646494\\
57.875	0.2028	-112.66104343556	-112.66104343556\\
57.875	0.20646	-128.079323725232	-128.079323725232\\
57.875	0.21012	-144.243044515511	-144.243044515511\\
57.875	0.21378	-161.152205806396	-161.152205806396\\
57.875	0.21744	-178.806807597888	-178.806807597888\\
57.875	0.2211	-197.206849889987	-197.206849889987\\
57.875	0.22476	-216.352332682692	-216.352332682692\\
57.875	0.22842	-236.243255976004	-236.243255976004\\
57.875	0.23208	-256.879619769923	-256.879619769923\\
57.875	0.23574	-278.261424064449	-278.261424064449\\
57.875	0.2394	-300.388668859581	-300.388668859581\\
57.875	0.24306	-323.261354155319	-323.261354155319\\
57.875	0.24672	-346.879479951665	-346.879479951665\\
57.875	0.25038	-371.243046248617	-371.243046248617\\
57.875	0.25404	-396.352053046176	-396.352053046176\\
57.875	0.2577	-422.206500344341	-422.206500344341\\
57.875	0.26136	-448.806388143113	-448.806388143113\\
57.875	0.26502	-476.151716442492	-476.151716442492\\
57.875	0.26868	-504.242485242476	-504.242485242476\\
57.875	0.27234	-533.078694543069	-533.078694543069\\
57.875	0.276	-562.660344344267	-562.660344344267\\
58.25	0.093	2.41087948688005	2.41087948688005\\
58.25	0.09666	9.60487908905856	9.60487908905856\\
58.25	0.10032	16.0534381906303	16.0534381906303\\
58.25	0.10398	21.7565567915959	21.7565567915959\\
58.25	0.10764	26.7142348919544	26.7142348919544\\
58.25	0.1113	30.9264724917066	30.9264724917066\\
58.25	0.11496	34.3932695908518	34.3932695908518\\
58.25	0.11862	37.1146261893903	37.1146261893903\\
58.25	0.12228	39.0905422873222	39.0905422873222\\
58.25	0.12594	40.3210178846473	40.3210178846473\\
58.25	0.1296	40.8060529813664	40.8060529813664\\
58.25	0.13326	40.5456475774783	40.5456475774783\\
58.25	0.13692	39.5398016729839	39.5398016729839\\
58.25	0.14058	37.7885152678823	37.7885152678823\\
58.25	0.14424	35.2917883621741	35.2917883621741\\
58.25	0.1479	32.0496209558594	32.0496209558594\\
58.25	0.15156	28.0620130489384	28.0620130489384\\
58.25	0.15522	23.3289646414103	23.3289646414103\\
58.25	0.15888	17.8504757332757	17.8504757332757\\
58.25	0.16254	11.6265463245344	11.6265463245344\\
58.25	0.1662	4.65717641518631	4.65717641518631\\
58.25	0.16986	-3.05763399476848	-3.05763399476848\\
58.25	0.17352	-11.5178849053294	-11.5178849053294\\
58.25	0.17718	-20.7235763164974	-20.7235763164974\\
58.25	0.18084	-30.6747082282717	-30.6747082282717\\
58.25	0.1845	-41.3712806406531	-41.3712806406531\\
58.25	0.18816	-52.8132935536414	-52.8132935536414\\
58.25	0.19182	-65.0007469672361	-65.0007469672361\\
58.25	0.19548	-77.933640881437	-77.933640881437\\
58.25	0.19914	-91.6119752962445	-91.6119752962445\\
58.25	0.2028	-106.03575021166	-106.03575021166\\
58.25	0.20646	-121.204965627681	-121.204965627681\\
58.25	0.21012	-137.119621544309	-137.119621544309\\
58.25	0.21378	-153.779717961544	-153.779717961544\\
58.25	0.21744	-171.185254879385	-171.185254879385\\
58.25	0.2211	-189.336232297832	-189.336232297832\\
58.25	0.22476	-208.232650216887	-208.232650216887\\
58.25	0.22842	-227.874508636548	-227.874508636548\\
58.25	0.23208	-248.261807556816	-248.261807556816\\
58.25	0.23574	-269.394546977691	-269.394546977691\\
58.25	0.2394	-291.272726899172	-291.272726899172\\
58.25	0.24306	-313.896347321259	-313.896347321259\\
58.25	0.24672	-337.265408243954	-337.265408243954\\
58.25	0.25038	-361.379909667256	-361.379909667256\\
58.25	0.25404	-386.239851591163	-386.239851591163\\
58.25	0.2577	-411.845234015678	-411.845234015678\\
58.25	0.26136	-438.196056940799	-438.196056940799\\
58.25	0.26502	-465.292320366527	-465.292320366527\\
58.25	0.26868	-493.134024292861	-493.134024292861\\
58.25	0.27234	-521.721168719802	-521.721168719802\\
58.25	0.276	-551.05375364735	-551.05375364735\\
58.625	0.093	1.41373459029819	1.41373459029819\\
58.625	0.09666	8.85679906612739	8.85679906612739\\
58.625	0.10032	15.5544230413503	15.5544230413503\\
58.625	0.10398	21.5066065159666	21.5066065159666\\
58.625	0.10764	26.7133494899762	26.7133494899762\\
58.625	0.1113	31.1746519633792	31.1746519633792\\
58.625	0.11496	34.890513936175	34.890513936175\\
58.625	0.11862	37.8609354083647	37.8609354083647\\
58.625	0.12228	40.0859163799473	40.0859163799473\\
58.625	0.12594	41.5654568509236	41.5654568509236\\
58.625	0.1296	42.2995568212933	42.2995568212933\\
58.625	0.13326	42.2882162910559	42.2882162910559\\
58.625	0.13692	41.5314352602122	41.5314352602122\\
58.625	0.14058	40.0292137287618	40.0292137287618\\
58.625	0.14424	37.7815516967047	37.7815516967047\\
58.625	0.1479	34.7884491640407	34.7884491640407\\
58.625	0.15156	31.0499061307704	31.0499061307704\\
58.625	0.15522	26.5659225968934	26.5659225968934\\
58.625	0.15888	21.3364985624096	21.3364985624096\\
58.625	0.16254	15.3616340273189	15.3616340273189\\
58.625	0.1662	8.641328991622	8.641328991622\\
58.625	0.16986	1.17558345531791	1.17558345531791\\
58.625	0.17352	-7.03560258159234	-7.03560258159234\\
58.625	0.17718	-15.9922291191092	-15.9922291191092\\
58.625	0.18084	-25.6942961572328	-25.6942961572328\\
58.625	0.1845	-36.1418036959635	-36.1418036959635\\
58.625	0.18816	-47.3347517353006	-47.3347517353006\\
58.625	0.19182	-59.2731402752446	-59.2731402752446\\
58.625	0.19548	-71.9569693157948	-71.9569693157948\\
58.625	0.19914	-85.3862388569516	-85.3862388569516\\
58.625	0.2028	-99.5609488987157	-99.5609488987157\\
58.625	0.20646	-114.481099441086	-114.481099441086\\
58.625	0.21012	-130.146690484063	-130.146690484063\\
58.625	0.21378	-146.557722027647	-146.557722027647\\
58.625	0.21744	-163.714194071837	-163.714194071837\\
58.625	0.2211	-181.616106616634	-181.616106616634\\
58.625	0.22476	-200.263459662038	-200.263459662038\\
58.625	0.22842	-219.656253208048	-219.656253208048\\
58.625	0.23208	-239.794487254665	-239.794487254665\\
58.625	0.23574	-260.678161801889	-260.678161801889\\
58.625	0.2394	-282.30727684972	-282.30727684972\\
58.625	0.24306	-304.681832398156	-304.681832398156\\
58.625	0.24672	-327.8018284472	-327.8018284472\\
58.625	0.25038	-351.66726499685	-351.66726499685\\
58.625	0.25404	-376.278142047107	-376.278142047107\\
58.625	0.2577	-401.634459597971	-401.634459597971\\
58.625	0.26136	-427.736217649441	-427.736217649441\\
58.625	0.26502	-454.583416201518	-454.583416201518\\
58.625	0.26868	-482.176055254202	-482.176055254202\\
58.625	0.27234	-510.514134807492	-510.514134807492\\
58.625	0.276	-539.597654861388	-539.597654861388\\
59	0.093	0.266097782761108	0.266097782761108\\
59	0.09666	7.95822713224146	7.95822713224146\\
59	0.10032	14.904915981115	14.904915981115\\
59	0.10398	21.1061643293821	21.1061643293821\\
59	0.10764	26.5619721770424	26.5619721770424\\
59	0.1113	31.272339524096	31.272339524096\\
59	0.11496	35.237266370543	35.237266370543\\
59	0.11862	38.4567527163834	38.4567527163834\\
59	0.12228	40.9307985616171	40.9307985616171\\
59	0.12594	42.6594039062441	42.6594039062441\\
59	0.1296	43.6425687502646	43.6425687502646\\
59	0.13326	43.8802930936783	43.8802930936783\\
59	0.13692	43.3725769364858	43.3725769364858\\
59	0.14058	42.119420278686	42.119420278686\\
59	0.14424	40.1208231202796	40.1208231202796\\
59	0.1479	37.3767854612663	37.3767854612663\\
59	0.15156	33.8873073016472	33.8873073016472\\
59	0.15522	29.6523886414204	29.6523886414204\\
59	0.15888	24.6720294805878	24.6720294805878\\
59	0.16254	18.9462298191482	18.9462298191482\\
59	0.1662	12.474989657102	12.474989657102\\
59	0.16986	5.25830899444907	5.25830899444907\\
59	0.17352	-2.70381216881049	-2.70381216881049\\
59	0.17718	-11.4113738326766	-11.4113738326766\\
59	0.18084	-20.8643759971495	-20.8643759971495\\
59	0.1845	-31.0628186622291	-31.0628186622291\\
59	0.18816	-42.0067018279151	-42.0067018279151\\
59	0.19182	-53.6960254942079	-53.6960254942079\\
59	0.19548	-66.1307896611079	-66.1307896611079\\
59	0.19914	-79.3109943286136	-79.3109943286136\\
59	0.2028	-93.2366394967269	-93.2366394967269\\
59	0.20646	-107.907725165446	-107.907725165446\\
59	0.21012	-123.324251334772	-123.324251334772\\
59	0.21378	-139.486218004705	-139.486218004705\\
59	0.21744	-156.393625175245	-156.393625175245\\
59	0.2211	-174.046472846391	-174.046472846391\\
59	0.22476	-192.444761018143	-192.444761018143\\
59	0.22842	-211.588489690503	-211.588489690503\\
59	0.23208	-231.477658863469	-231.477658863469\\
59	0.23574	-252.112268537043	-252.112268537043\\
59	0.2394	-273.492318711222	-273.492318711222\\
59	0.24306	-295.617809386008	-295.617809386008\\
59	0.24672	-318.4887405614	-318.4887405614\\
59	0.25038	-342.1051122374	-342.1051122374\\
59	0.25404	-366.466924414006	-366.466924414006\\
59	0.2577	-391.57417709122	-391.57417709122\\
59	0.26136	-417.426870269039	-417.426870269039\\
59	0.26502	-444.025003947464	-444.025003947464\\
59	0.26868	-471.368578126497	-471.368578126497\\
59	0.27234	-499.457592806137	-499.457592806137\\
59	0.276	-528.292047986383	-528.292047986383\\
59.375	0.093	-1.03203093573234	-1.03203093573234\\
59.375	0.09666	6.90916328739871	6.90916328739871\\
59.375	0.10032	14.104917009923	14.104917009923\\
59.375	0.10398	20.5552302318416	20.5552302318416\\
59.375	0.10764	26.2601029531526	26.2601029531526\\
59.375	0.1113	31.219535173857	31.219535173857\\
59.375	0.11496	35.4335268939546	35.4335268939546\\
59.375	0.11862	38.9020781134462	38.9020781134462\\
59.375	0.12228	41.6251888323306	41.6251888323306\\
59.375	0.12594	43.6028590506087	43.6028590506087\\
59.375	0.1296	44.8350887682799	44.8350887682799\\
59.375	0.13326	45.3218779853443	45.3218779853443\\
59.375	0.13692	45.0632267018025	45.0632267018025\\
59.375	0.14058	44.0591349176539	44.0591349176539\\
59.375	0.14424	42.3096026328982	42.3096026328982\\
59.375	0.1479	39.814629847536	39.814629847536\\
59.375	0.15156	36.5742165615676	36.5742165615676\\
59.375	0.15522	32.588362774992	32.588362774992\\
59.375	0.15888	27.85706848781	27.85706848781\\
59.375	0.16254	22.3803337000212	22.3803337000212\\
59.375	0.1662	16.1581584116261	16.1581584116261\\
59.375	0.16986	9.19054262262387	9.19054262262387\\
59.375	0.17352	1.47748633301501	1.47748633301501\\
59.375	0.17718	-6.9810104572	-6.9810104572\\
59.375	0.18084	-16.1849477480222	-16.1849477480222\\
59.375	0.1845	-26.1343255394511	-26.1343255394511\\
59.375	0.18816	-36.8291438314864	-36.8291438314864\\
59.375	0.19182	-48.269402624128	-48.269402624128\\
59.375	0.19548	-60.4551019173769	-60.4551019173769\\
59.375	0.19914	-73.3862417112318	-73.3862417112318\\
59.375	0.2028	-87.0628220056944	-87.0628220056944\\
59.375	0.20646	-101.484842800763	-101.484842800763\\
59.375	0.21012	-116.652304096438	-116.652304096438\\
59.375	0.21378	-132.56520589272	-132.56520589272\\
59.375	0.21744	-149.223548189609	-149.223548189609\\
59.375	0.2211	-166.627330987104	-166.627330987104\\
59.375	0.22476	-184.776554285206	-184.776554285206\\
59.375	0.22842	-203.671218083914	-203.671218083914\\
59.375	0.23208	-223.31132238323	-223.31132238323\\
59.375	0.23574	-243.696867183152	-243.696867183152\\
59.375	0.2394	-264.827852483681	-264.827852483681\\
59.375	0.24306	-286.704278284816	-286.704278284816\\
59.375	0.24672	-309.326144586558	-309.326144586558\\
59.375	0.25038	-332.693451388906	-332.693451388906\\
59.375	0.25404	-356.806198691862	-356.806198691862\\
59.375	0.2577	-381.664386495424	-381.664386495424\\
59.375	0.26136	-407.268014799592	-407.268014799592\\
59.375	0.26502	-433.617083604368	-433.617083604368\\
59.375	0.26868	-460.711592909749	-460.711592909749\\
59.375	0.27234	-488.551542715738	-488.551542715738\\
59.375	0.276	-517.136933022333	-517.136933022333\\
59.75	0.093	-2.48065156518169	-2.48065156518169\\
59.75	0.09666	5.70960753160006	5.70960753160006\\
59.75	0.10032	13.1544261277755	13.1544261277755\\
59.75	0.10398	19.8538042233444	19.8538042233444\\
59.75	0.10764	25.8077418183065	25.8077418183065\\
59.75	0.1113	31.016238912662	31.016238912662\\
59.75	0.11496	35.4792955064108	35.4792955064108\\
59.75	0.11862	39.1969115995531	39.1969115995531\\
59.75	0.12228	42.1690871920882	42.1690871920882\\
59.75	0.12594	44.395822284017	44.395822284017\\
59.75	0.1296	45.8771168753393	45.8771168753393\\
59.75	0.13326	46.6129709660544	46.6129709660544\\
59.75	0.13692	46.6033845561633	46.6033845561633\\
59.75	0.14058	45.8483576456654	45.8483576456654\\
59.75	0.14424	44.3478902345609	44.3478902345609\\
59.75	0.1479	42.1019823228498	42.1019823228498\\
59.75	0.15156	39.1106339105316	39.1106339105316\\
59.75	0.15522	35.3738449976072	35.3738449976072\\
59.75	0.15888	30.8916155840764	30.8916155840764\\
59.75	0.16254	25.6639456699382	25.6639456699382\\
59.75	0.1662	19.6908352551939	19.6908352551939\\
59.75	0.16986	12.9722843398423	12.9722843398423\\
59.75	0.17352	5.5082929238846	5.5082929238846\\
59.75	0.17718	-2.70113899267972	-2.70113899267972\\
59.75	0.18084	-11.6560114098507	-11.6560114098507\\
59.75	0.1845	-21.3563243276285	-21.3563243276285\\
59.75	0.18816	-31.8020777460131	-31.8020777460131\\
59.75	0.19182	-42.9932716650045	-42.9932716650045\\
59.75	0.19548	-54.9299060846022	-54.9299060846022\\
59.75	0.19914	-67.6119810048065	-67.6119810048065\\
59.75	0.2028	-81.0394964256179	-81.0394964256179\\
59.75	0.20646	-95.2124523470357	-95.2124523470357\\
59.75	0.21012	-110.13084876906	-110.13084876906\\
59.75	0.21378	-125.794685691691	-125.794685691691\\
59.75	0.21744	-142.203963114929	-142.203963114929\\
59.75	0.2211	-159.358681038773	-159.358681038773\\
59.75	0.22476	-177.258839463224	-177.258839463224\\
59.75	0.22842	-195.904438388282	-195.904438388282\\
59.75	0.23208	-215.295477813946	-215.295477813946\\
59.75	0.23574	-235.431957740218	-235.431957740218\\
59.75	0.2394	-256.313878167095	-256.313878167095\\
59.75	0.24306	-277.94123909458	-277.94123909458\\
59.75	0.24672	-300.314040522671	-300.314040522671\\
59.75	0.25038	-323.432282451369	-323.432282451369\\
59.75	0.25404	-347.295964880673	-347.295964880673\\
59.75	0.2577	-371.905087810584	-371.905087810584\\
59.75	0.26136	-397.259651241102	-397.259651241102\\
59.75	0.26502	-423.359655172226	-423.359655172226\\
59.75	0.26868	-450.205099603957	-450.205099603957\\
59.75	0.27234	-477.795984536295	-477.795984536295\\
59.75	0.276	-506.132309969239	-506.132309969239\\
60.125	0.093	-4.07976410558695	-4.07976410558695\\
60.125	0.09666	4.35955986484595	4.35955986484595\\
60.125	0.10032	12.0534433346721	12.0534433346721\\
60.125	0.10398	19.0018863038921	19.0018863038921\\
60.125	0.10764	25.2048887725045	25.2048887725045\\
60.125	0.1113	30.6624507405111	30.6624507405111\\
60.125	0.11496	35.3745722079102	35.3745722079102\\
60.125	0.11862	39.3412531747036	39.3412531747036\\
60.125	0.12228	42.5624936408898	42.5624936408898\\
60.125	0.12594	45.0382936064694	45.0382936064694\\
60.125	0.1296	46.7686530714428	46.7686530714428\\
60.125	0.13326	47.7535720358086	47.7535720358086\\
60.125	0.13692	47.9930504995687	47.9930504995687\\
60.125	0.14058	47.4870884627214	47.4870884627214\\
60.125	0.14424	46.2356859252676	46.2356859252676\\
60.125	0.1479	44.2388428872077	44.2388428872077\\
60.125	0.15156	41.4965593485402	41.4965593485402\\
60.125	0.15522	38.0088353092669	38.0088353092669\\
60.125	0.15888	33.7756707693864	33.7756707693864\\
60.125	0.16254	28.7970657288994	28.7970657288994\\
60.125	0.1662	23.0730201878057	23.0730201878057\\
60.125	0.16986	16.6035341461053	16.6035341461053\\
60.125	0.17352	9.38860760379828	9.38860760379828\\
60.125	0.17718	1.42824056088466	1.42824056088466\\
60.125	0.18084	-7.27756698263522	-7.27756698263522\\
60.125	0.1845	-16.7288150267623	-16.7288150267623\\
60.125	0.18816	-26.9255035714957	-26.9255035714957\\
60.125	0.19182	-37.8676326168364	-37.8676326168364\\
60.125	0.19548	-49.5552021627834	-49.5552021627834\\
60.125	0.19914	-61.9882122093366	-61.9882122093366\\
60.125	0.2028	-75.1666627564969	-75.1666627564969\\
60.125	0.20646	-89.0905538042643	-89.0905538042643\\
60.125	0.21012	-103.759885352638	-103.759885352638\\
60.125	0.21378	-119.174657401618	-119.174657401618\\
60.125	0.21744	-135.334869951205	-135.334869951205\\
60.125	0.2211	-152.240523001398	-152.240523001398\\
60.125	0.22476	-169.891616552198	-169.891616552198\\
60.125	0.22842	-188.288150603605	-188.288150603605\\
60.125	0.23208	-207.430125155619	-207.430125155619\\
60.125	0.23574	-227.317540208239	-227.317540208239\\
60.125	0.2394	-247.950395761467	-247.950395761467\\
60.125	0.24306	-269.3286918153	-269.3286918153\\
60.125	0.24672	-291.45242836974	-291.45242836974\\
60.125	0.25038	-314.321605424787	-314.321605424787\\
60.125	0.25404	-337.936222980441	-337.936222980441\\
60.125	0.2577	-362.296281036701	-362.296281036701\\
60.125	0.26136	-387.401779593567	-387.401779593567\\
60.125	0.26502	-413.252718651041	-413.252718651041\\
60.125	0.26868	-439.849098209121	-439.849098209121\\
60.125	0.27234	-467.190918267808	-467.190918267808\\
60.125	0.276	-495.278178827101	-495.278178827101\\
60.5	0.093	-5.82936855694743	-5.82936855694743\\
60.5	0.09666	2.85902028713616	2.85902028713616\\
60.5	0.10032	10.8019686306134	10.8019686306134\\
60.5	0.10398	17.9994764734842	17.9994764734842\\
60.5	0.10764	24.4515438157477	24.4515438157477\\
60.5	0.1113	30.1581706574046	30.1581706574046\\
60.5	0.11496	35.1193569984553	35.1193569984553\\
60.5	0.11862	39.3351028388993	39.3351028388993\\
60.5	0.12228	42.8054081787363	42.8054081787363\\
60.5	0.12594	45.5302730179665	45.5302730179665\\
60.5	0.1296	47.5096973565907	47.5096973565907\\
60.5	0.13326	48.7436811946081	48.7436811946081\\
60.5	0.13692	49.2322245320183	49.2322245320183\\
60.5	0.14058	48.9753273688223	48.9753273688223\\
60.5	0.14424	47.9729897050191	47.9729897050191\\
60.5	0.1479	46.22521154061	46.22521154061\\
60.5	0.15156	43.7319928755936	43.7319928755936\\
60.5	0.15522	40.4933337099706	40.4933337099706\\
60.5	0.15888	36.5092340437416	36.5092340437416\\
60.5	0.16254	31.7796938769053	31.7796938769053\\
60.5	0.1662	26.3047132094628	26.3047132094628\\
60.5	0.16986	20.0842920414131	20.0842920414131\\
60.5	0.17352	13.1184303727572	13.1184303727572\\
60.5	0.17718	5.40712820349427	5.40712820349427\\
60.5	0.18084	-3.04961446637537	-3.04961446637537\\
60.5	0.1845	-12.2517976368513	-12.2517976368513\\
60.5	0.18816	-22.199421307934	-22.199421307934\\
60.5	0.19182	-32.8924854796236	-32.8924854796236\\
60.5	0.19548	-44.3309901519194	-44.3309901519194\\
60.5	0.19914	-56.5149353248219	-56.5149353248219\\
60.5	0.2028	-69.4443209983315	-69.4443209983315\\
60.5	0.20646	-83.1191471724478	-83.1191471724478\\
60.5	0.21012	-97.5394138471706	-97.5394138471706\\
60.5	0.21378	-112.7051210225	-112.7051210225\\
60.5	0.21744	-128.616268698436	-128.616268698436\\
60.5	0.2211	-145.272856874978	-145.272856874978\\
60.5	0.22476	-162.674885552128	-162.674885552128\\
60.5	0.22842	-180.822354729884	-180.822354729884\\
60.5	0.23208	-199.715264408247	-199.715264408247\\
60.5	0.23574	-219.353614587216	-219.353614587216\\
60.5	0.2394	-239.737405266792	-239.737405266792\\
60.5	0.24306	-260.866636446975	-260.866636446975\\
60.5	0.24672	-282.741308127764	-282.741308127764\\
60.5	0.25038	-305.36142030916	-305.36142030916\\
60.5	0.25404	-328.726972991163	-328.726972991163\\
60.5	0.2577	-352.837966173772	-352.837966173772\\
60.5	0.26136	-377.694399856988	-377.694399856988\\
60.5	0.26502	-403.296274040811	-403.296274040811\\
60.5	0.26868	-429.64358872524	-429.64358872524\\
60.5	0.27234	-456.736343910276	-456.736343910276\\
60.5	0.276	-484.574539595918	-484.574539595918\\
60.875	0.093	-7.72946491926496	-7.72946491926496\\
60.875	0.09666	1.20798879846978	1.20798879846978\\
60.875	0.10032	9.40000201559775	9.40000201559775\\
60.875	0.10398	16.8465747321192	16.8465747321192\\
60.875	0.10764	23.5477069480339	23.5477069480339\\
60.875	0.1113	29.5033986633419	29.5033986633419\\
60.875	0.11496	34.7136498780433	34.7136498780433\\
60.875	0.11862	39.178460592138	39.178460592138\\
60.875	0.12228	42.8978308056261	42.8978308056261\\
60.875	0.12594	45.8717605185075	45.8717605185075\\
60.875	0.1296	48.1002497307824	48.1002497307824\\
60.875	0.13326	49.5832984424505	49.5832984424505\\
60.875	0.13692	50.3209066535119	50.3209066535119\\
60.875	0.14058	50.3130743639665	50.3130743639665\\
60.875	0.14424	49.5598015738141	49.5598015738141\\
60.875	0.1479	48.0610882830561	48.0610882830561\\
60.875	0.15156	45.8169344916904	45.8169344916904\\
60.875	0.15522	42.8273401997185	42.8273401997185\\
60.875	0.15888	39.0923054071397	39.0923054071397\\
60.875	0.16254	34.6118301139546	34.6118301139546\\
60.875	0.1662	29.3859143201628	29.3859143201628\\
60.875	0.16986	23.4145580257642	23.4145580257642\\
60.875	0.17352	16.6977612307591	16.6977612307591\\
60.875	0.17718	9.23552393514683	9.23552393514683\\
60.875	0.18084	1.02784613892834	1.02784613892834\\
60.875	0.1845	-7.92527215789687	-7.92527215789687\\
60.875	0.18816	-17.6238309553285	-17.6238309553285\\
60.875	0.19182	-28.0678302533673	-28.0678302533673\\
60.875	0.19548	-39.2572700520125	-39.2572700520125\\
60.875	0.19914	-51.1921503512642	-51.1921503512642\\
60.875	0.2028	-63.8724711511227	-63.8724711511227\\
60.875	0.20646	-77.2982324515883	-77.2982324515883\\
60.875	0.21012	-91.46943425266	-91.46943425266\\
60.875	0.21378	-106.386076554338	-106.386076554338\\
60.875	0.21744	-122.048159356624	-122.048159356624\\
60.875	0.2211	-138.455682659516	-138.455682659516\\
60.875	0.22476	-155.608646463014	-155.608646463014\\
60.875	0.22842	-173.507050767119	-173.507050767119\\
60.875	0.23208	-192.150895571831	-192.150895571831\\
60.875	0.23574	-211.54018087715	-211.54018087715\\
60.875	0.2394	-231.674906683075	-231.674906683075\\
60.875	0.24306	-252.555072989607	-252.555072989607\\
60.875	0.24672	-274.180679796745	-274.180679796745\\
60.875	0.25038	-296.55172710449	-296.55172710449\\
60.875	0.25404	-319.668214912842	-319.668214912842\\
60.875	0.2577	-343.530143221801	-343.530143221801\\
60.875	0.26136	-368.137512031366	-368.137512031366\\
60.875	0.26502	-393.490321341538	-393.490321341538\\
60.875	0.26868	-419.588571152316	-419.588571152316\\
60.875	0.27234	-446.4322614637	-446.4322614637\\
60.875	0.276	-474.021392275693	-474.021392275693\\
61.25	0.093	-9.78005319253748	-9.78005319253748\\
61.25	0.09666	-0.593534601152044	-0.593534601152044\\
61.25	0.10032	7.84754348962662	7.84754348962662\\
61.25	0.10398	15.5431810797992	15.5431810797992\\
61.25	0.10764	22.4933781693641	22.4933781693641\\
61.25	0.1113	28.6981347583233	28.6981347583233\\
61.25	0.11496	34.1574508466758	34.1574508466758\\
61.25	0.11862	38.8713264344213	38.8713264344213\\
61.25	0.12228	42.8397615215601	42.8397615215601\\
61.25	0.12594	46.0627561080922	46.0627561080922\\
61.25	0.1296	48.5403101940177	48.5403101940177\\
61.25	0.13326	50.272423779337	50.272423779337\\
61.25	0.13692	51.2590968640491	51.2590968640491\\
61.25	0.14058	51.5003294481544	51.5003294481544\\
61.25	0.14424	50.9961215316536	50.9961215316536\\
61.25	0.1479	49.7464731145458	49.7464731145458\\
61.25	0.15156	47.7513841968313	47.7513841968313\\
61.25	0.15522	45.0108547785101	45.0108547785101\\
61.25	0.15888	41.5248848595825	41.5248848595825\\
61.25	0.16254	37.293474440048	37.293474440048\\
61.25	0.1662	32.3166235199069	32.3166235199069\\
61.25	0.16986	26.594332099159	26.594332099159\\
61.25	0.17352	20.126600177805	20.126600177805\\
61.25	0.17718	12.9134277558439	12.9134277558439\\
61.25	0.18084	4.95481483327615	4.95481483327615\\
61.25	0.1845	-3.74923858989791	-3.74923858989791\\
61.25	0.18816	-13.1987325136793	-13.1987325136793\\
61.25	0.19182	-23.393666938067	-23.393666938067\\
61.25	0.19548	-34.3340418630614	-34.3340418630614\\
61.25	0.19914	-46.019857288662	-46.019857288662\\
61.25	0.2028	-58.4511132148698	-58.4511132148698\\
61.25	0.20646	-71.6278096416843	-71.6278096416843\\
61.25	0.21012	-85.5499465691053	-85.5499465691053\\
61.25	0.21378	-100.217523997133	-100.217523997133\\
61.25	0.21744	-115.630541925767	-115.630541925767\\
61.25	0.2211	-131.789000355008	-131.789000355008\\
61.25	0.22476	-148.692899284856	-148.692899284856\\
61.25	0.22842	-166.34223871531	-166.34223871531\\
61.25	0.23208	-184.737018646371	-184.737018646371\\
61.25	0.23574	-203.877239078039	-203.877239078039\\
61.25	0.2394	-223.762900010313	-223.762900010313\\
61.25	0.24306	-244.394001443194	-244.394001443194\\
61.25	0.24672	-265.770543376682	-265.770543376682\\
61.25	0.25038	-287.892525810776	-287.892525810776\\
61.25	0.25404	-310.759948745477	-310.759948745477\\
61.25	0.2577	-334.372812180784	-334.372812180784\\
61.25	0.26136	-358.731116116699	-358.731116116699\\
61.25	0.26502	-383.83486055322	-383.83486055322\\
61.25	0.26868	-409.684045490347	-409.684045490347\\
61.25	0.27234	-436.278670928081	-436.278670928081\\
61.25	0.276	-463.618736866422	-463.618736866422\\
61.625	0.093	-11.9811333767657	-11.9811333767657\\
61.625	0.09666	-2.54554991172955	-2.54554991172955\\
61.625	0.10032	6.14459305270026	6.14459305270026\\
61.625	0.10398	14.0892955165235	14.0892955165235\\
61.625	0.10764	21.2885574797396	21.2885574797396\\
61.625	0.1113	27.7423789423495	27.7423789423495\\
61.625	0.11496	33.4507599043527	33.4507599043527\\
61.625	0.11862	38.4137003657489	38.4137003657489\\
61.625	0.12228	42.6312003265384	42.6312003265384\\
61.625	0.12594	46.103259786722	46.103259786722\\
61.625	0.1296	48.8298787462983	48.8298787462983\\
61.625	0.13326	50.8110572052678	50.8110572052678\\
61.625	0.13692	52.0467951636315	52.0467951636315\\
61.625	0.14058	52.5370926213875	52.5370926213875\\
61.625	0.14424	52.2819495785374	52.2819495785374\\
61.625	0.1479	51.2813660350807	51.2813660350807\\
61.625	0.15156	49.5353419910169	49.5353419910169\\
61.625	0.15522	47.0438774463464	47.0438774463464\\
61.625	0.15888	43.80697240107	43.80697240107\\
61.625	0.16254	39.8246268551862	39.8246268551862\\
61.625	0.1662	35.0968408086962	35.0968408086962\\
61.625	0.16986	29.6236142615995	29.6236142615995\\
61.625	0.17352	23.4049472138958	23.4049472138958\\
61.625	0.17718	16.4408396655854	16.4408396655854\\
61.625	0.18084	8.73129161666873	8.73129161666873\\
61.625	0.1845	0.276303067145363	0.276303067145363\\
61.625	0.18816	-8.92412598298483	-8.92412598298483\\
61.625	0.19182	-18.8699955337219	-18.8699955337219\\
61.625	0.19548	-29.5613055850656	-29.5613055850656\\
61.625	0.19914	-40.9980561370151	-40.9980561370151\\
61.625	0.2028	-53.1802471895721	-53.1802471895721\\
61.625	0.20646	-66.1078787427359	-66.1078787427359\\
61.625	0.21012	-79.7809507965062	-79.7809507965062\\
61.625	0.21378	-94.1994633508823	-94.1994633508823\\
61.625	0.21744	-109.363416405866	-109.363416405866\\
61.625	0.2211	-125.272809961456	-125.272809961456\\
61.625	0.22476	-141.927644017653	-141.927644017653\\
61.625	0.22842	-159.327918574456	-159.327918574456\\
61.625	0.23208	-177.473633631867	-177.473633631867\\
61.625	0.23574	-196.364789189884	-196.364789189884\\
61.625	0.2394	-216.001385248507	-216.001385248507\\
61.625	0.24306	-236.383421807737	-236.383421807737\\
61.625	0.24672	-257.510898867574	-257.510898867574\\
61.625	0.25038	-279.383816428017	-279.383816428017\\
61.625	0.25404	-302.002174489067	-302.002174489067\\
61.625	0.2577	-325.365973050725	-325.365973050725\\
61.625	0.26136	-349.475212112987	-349.475212112987\\
61.625	0.26502	-374.329891675857	-374.329891675857\\
61.625	0.26868	-399.930011739334	-399.930011739334\\
61.625	0.27234	-426.275572303417	-426.275572303417\\
61.625	0.276	-453.366573368107	-453.366573368107\\
62	0.093	-14.3327054719507	-14.3327054719507\\
62	0.09666	-4.64805713326297	-4.64805713326297\\
62	0.10032	4.29115070481754	4.29115070481754\\
62	0.10398	12.4849180422915	12.4849180422915\\
62	0.10764	19.9332448791583	19.9332448791583\\
62	0.1113	26.6361312154193	26.6361312154193\\
62	0.11496	32.5935770510728	32.5935770510728\\
62	0.11862	37.8055823861205	37.8055823861205\\
62	0.12228	42.2721472205607	42.2721472205607\\
62	0.12594	45.9932715543947	45.9932715543947\\
62	0.1296	48.9689553876225	48.9689553876225\\
62	0.13326	51.1991987202427	51.1991987202427\\
62	0.13692	52.6840015522566	52.6840015522566\\
62	0.14058	53.4233638836643	53.4233638836643\\
62	0.14424	53.4172857144648	53.4172857144648\\
62	0.1479	52.6657670446589	52.6657670446589\\
62	0.15156	51.1688078742458	51.1688078742458\\
62	0.15522	48.9264082032264	48.9264082032264\\
62	0.15888	45.9385680316007	45.9385680316007\\
62	0.16254	42.2052873593681	42.2052873593681\\
62	0.1662	37.7265661865288	37.7265661865288\\
62	0.16986	32.5024045130828	32.5024045130828\\
62	0.17352	26.5328023390301	26.5328023390301\\
62	0.17718	19.8177596643709	19.8177596643709\\
62	0.18084	12.3572764891049	12.3572764891049\\
62	0.1845	4.15135281323228	4.15135281323228\\
62	0.18816	-4.80001136324722	-4.80001136324722\\
62	0.19182	-14.4968160403331	-14.4968160403331\\
62	0.19548	-24.9390612180257	-24.9390612180257\\
62	0.19914	-36.1267468963249	-36.1267468963249\\
62	0.2028	-48.0598730752308	-48.0598730752308\\
62	0.20646	-60.7384397547439	-60.7384397547439\\
62	0.21012	-74.1624469348631	-74.1624469348631\\
62	0.21378	-88.3318946155885	-88.3318946155885\\
62	0.21744	-103.246782796921	-103.246782796921\\
62	0.2211	-118.907111478861	-118.907111478861\\
62	0.22476	-135.312880661406	-135.312880661406\\
62	0.22842	-152.464090344559	-152.464090344559\\
62	0.23208	-170.360740528318	-170.360740528318\\
62	0.23574	-189.002831212685	-189.002831212685\\
62	0.2394	-208.390362397657	-208.390362397657\\
62	0.24306	-228.523334083236	-228.523334083236\\
62	0.24672	-249.401746269422	-249.401746269422\\
62	0.25038	-271.025598956214	-271.025598956214\\
62	0.25404	-293.394892143614	-293.394892143614\\
62	0.2577	-316.50962583162	-316.50962583162\\
62	0.26136	-340.369800020232	-340.369800020232\\
62	0.26502	-364.975414709452	-364.975414709452\\
62	0.26868	-390.326469899277	-390.326469899277\\
62	0.27234	-416.422965589709	-416.422965589709\\
62	0.276	-443.264901780749	-443.264901780749\\
62.375	0.093	-16.8347694780909	-16.8347694780909\\
62.375	0.09666	-6.90105626575297	-6.90105626575297\\
62.375	0.10032	2.28721644597823	2.28721644597823\\
62.375	0.10398	10.7300486571033	10.7300486571033\\
62.375	0.10764	18.4274403676213	18.4274403676213\\
62.375	0.1113	25.379391577533	25.379391577533\\
62.375	0.11496	31.5859022868376	31.5859022868376\\
62.375	0.11862	37.0469724955361	37.0469724955361\\
62.375	0.12228	41.762602203627	41.762602203627\\
62.375	0.12594	45.732791411112	45.732791411112\\
62.375	0.1296	48.9575401179906	48.9575401179906\\
62.375	0.13326	51.4368483242615	51.4368483242615\\
62.375	0.13692	53.1707160299266	53.1707160299266\\
62.375	0.14058	54.1591432349849	54.1591432349849\\
62.375	0.14424	54.4021299394366	54.4021299394366\\
62.375	0.1479	53.8996761432809	53.8996761432809\\
62.375	0.15156	52.6517818465194	52.6517818465194\\
62.375	0.15522	50.6584470491507	50.6584470491507\\
62.375	0.15888	47.9196717511757	47.9196717511757\\
62.375	0.16254	44.4354559525938	44.4354559525938\\
62.375	0.1662	40.2057996534052	40.2057996534052\\
62.375	0.16986	35.2307028536098	35.2307028536098\\
62.375	0.17352	29.5101655532084	29.5101655532084\\
62.375	0.17718	23.0441877522003	23.0441877522003\\
62.375	0.18084	15.832769450585	15.832769450585\\
62.375	0.1845	7.87591064836306	7.87591064836306\\
62.375	0.18816	-0.826388654465291	-0.826388654465291\\
62.375	0.19182	-10.2741284579005	-10.2741284579005\\
62.375	0.19548	-20.4673087619424	-20.4673087619424\\
62.375	0.19914	-31.4059295665904	-31.4059295665904\\
62.375	0.2028	-43.0899908718457	-43.0899908718457\\
62.375	0.20646	-55.5194926777076	-55.5194926777076\\
62.375	0.21012	-68.694434984176	-68.694434984176\\
62.375	0.21378	-82.6148177912507	-82.6148177912507\\
62.375	0.21744	-97.2806410989326	-97.2806410989326\\
62.375	0.2211	-112.691904907221	-112.691904907221\\
62.375	0.22476	-128.848609216116	-128.848609216116\\
62.375	0.22842	-145.750754025618	-145.750754025618\\
62.375	0.23208	-163.398339335726	-163.398339335726\\
62.375	0.23574	-181.791365146441	-181.791365146441\\
62.375	0.2394	-200.929831457763	-200.929831457763\\
62.375	0.24306	-220.813738269691	-220.813738269691\\
62.375	0.24672	-241.443085582226	-241.443085582226\\
62.375	0.25038	-262.817873395368	-262.817873395368\\
62.375	0.25404	-284.938101709116	-284.938101709116\\
62.375	0.2577	-307.803770523472	-307.803770523472\\
62.375	0.26136	-331.414879838433	-331.414879838433\\
62.375	0.26502	-355.771429654001	-355.771429654001\\
62.375	0.26868	-380.873419970176	-380.873419970176\\
62.375	0.27234	-406.720850786958	-406.720850786958\\
62.375	0.276	-433.313722104346	-433.313722104346\\
62.75	0.093	-19.4873253951878	-19.4873253951878\\
62.75	0.09666	-9.3045473091982	-9.3045473091982\\
62.75	0.10032	0.132790276183698	0.132790276183698\\
62.75	0.10398	8.8246873609595	8.8246873609595\\
62.75	0.10764	16.7711439451286	16.7711439451286\\
62.75	0.1113	23.9721600286906	23.9721600286906\\
62.75	0.11496	30.4277356116463	30.4277356116463\\
62.75	0.11862	36.1378706939955	36.1378706939955\\
62.75	0.12228	41.1025652757375	41.1025652757375\\
62.75	0.12594	45.3218193568737	45.3218193568737\\
62.75	0.1296	48.7956329374025	48.7956329374025\\
62.75	0.13326	51.5240060173246	51.5240060173246\\
62.75	0.13692	53.5069385966408	53.5069385966408\\
62.75	0.14058	54.7444306753493	54.7444306753493\\
62.75	0.14424	55.2364822534522	55.2364822534522\\
62.75	0.1479	54.9830933309477	54.9830933309477\\
62.75	0.15156	53.9842639078364	53.9842639078364\\
62.75	0.15522	52.2399939841189	52.2399939841189\\
62.75	0.15888	49.750283559795	49.750283559795\\
62.75	0.16254	46.5151326348638	46.5151326348638\\
62.75	0.1662	42.5345412093263	42.5345412093263\\
62.75	0.16986	37.8085092831817	37.8085092831817\\
62.75	0.17352	32.3370368564309	32.3370368564309\\
62.75	0.17718	26.1201239290735	26.1201239290735\\
62.75	0.18084	19.157770501109	19.157770501109\\
62.75	0.1845	11.4499765725386	11.4499765725386\\
62.75	0.18816	2.99674214336096	2.99674214336096\\
62.75	0.19182	-6.20193278642353	-6.20193278642353\\
62.75	0.19548	-16.1460482168147	-16.1460482168147\\
62.75	0.19914	-26.8356041478116	-26.8356041478116\\
62.75	0.2028	-38.2706005794162	-38.2706005794162\\
62.75	0.20646	-50.4510375116274	-50.4510375116274\\
62.75	0.21012	-63.3769149444447	-63.3769149444447\\
62.75	0.21378	-77.0482328778687	-77.0482328778687\\
62.75	0.21744	-91.4649913118999	-91.4649913118999\\
62.75	0.2211	-106.627190246537	-106.627190246537\\
62.75	0.22476	-122.534829681781	-122.534829681781\\
62.75	0.22842	-139.187909617632	-139.187909617632\\
62.75	0.23208	-156.58643005409	-156.58643005409\\
62.75	0.23574	-174.730390991154	-174.730390991154\\
62.75	0.2394	-193.619792428825	-193.619792428825\\
62.75	0.24306	-213.254634367102	-213.254634367102\\
62.75	0.24672	-233.634916805987	-233.634916805987\\
62.75	0.25038	-254.760639745478	-254.760639745478\\
62.75	0.25404	-276.631803185575	-276.631803185575\\
62.75	0.2577	-299.24840712628	-299.24840712628\\
62.75	0.26136	-322.61045156759	-322.61045156759\\
62.75	0.26502	-346.717936509507	-346.717936509507\\
62.75	0.26868	-371.570861952032	-371.570861952032\\
62.75	0.27234	-397.169227895162	-397.169227895162\\
62.75	0.276	-423.5130343389	-423.5130343389\\
63.125	0.093	-22.2903732232394	-22.2903732232394\\
63.125	0.09666	-11.8585302635996	-11.8585302635996\\
63.125	0.10032	-2.17212780456606	-2.17212780456606\\
63.125	0.10398	6.76883415386044	6.76883415386044\\
63.125	0.10764	14.9643556116802	14.9643556116802\\
63.125	0.1113	22.4144365688933	22.4144365688933\\
63.125	0.11496	29.1190770254998	29.1190770254998\\
63.125	0.11862	35.0782769814996	35.0782769814996\\
63.125	0.12228	40.2920364368928	40.2920364368928\\
63.125	0.12594	44.7603553916793	44.7603553916793\\
63.125	0.1296	48.4832338458592	48.4832338458592\\
63.125	0.13326	51.4606717994324	51.4606717994324\\
63.125	0.13692	53.6926692523989	53.6926692523989\\
63.125	0.14058	55.1792262047591	55.1792262047591\\
63.125	0.14424	55.9203426565122	55.9203426565122\\
63.125	0.1479	55.9160186076588	55.9160186076588\\
63.125	0.15156	55.1662540581987	55.1662540581987\\
63.125	0.15522	53.6710490081318	53.6710490081318\\
63.125	0.15888	51.4304034574586	51.4304034574586\\
63.125	0.16254	48.4443174061786	48.4443174061786\\
63.125	0.1662	44.7127908542918	44.7127908542918\\
63.125	0.16986	40.2358238017983	40.2358238017983\\
63.125	0.17352	35.0134162486983	35.0134162486983\\
63.125	0.17718	29.0455681949916	29.0455681949916\\
63.125	0.18084	22.3322796406782	22.3322796406782\\
63.125	0.1845	14.873550585758	14.873550585758\\
63.125	0.18816	6.66938103023153	6.66938103023153\\
63.125	0.19182	-2.2802290259018	-2.2802290259018\\
63.125	0.19548	-11.9752795826419	-11.9752795826419\\
63.125	0.19914	-22.4157706399885	-22.4157706399885\\
63.125	0.2028	-33.6017021979419	-33.6017021979419\\
63.125	0.20646	-45.533074256502	-45.533074256502\\
63.125	0.21012	-58.2098868156691	-58.2098868156691\\
63.125	0.21378	-71.6321398754419	-71.6321398754419\\
63.125	0.21744	-85.7998334358219	-85.7998334358219\\
63.125	0.2211	-100.712967496808	-100.712967496808\\
63.125	0.22476	-116.371542058402	-116.371542058402\\
63.125	0.22842	-132.775557120602	-132.775557120602\\
63.125	0.23208	-149.925012683409	-149.925012683409\\
63.125	0.23574	-167.819908746822	-167.819908746822\\
63.125	0.2394	-186.460245310842	-186.460245310842\\
63.125	0.24306	-205.846022375468	-205.846022375468\\
63.125	0.24672	-225.977239940702	-225.977239940702\\
63.125	0.25038	-246.853898006542	-246.853898006542\\
63.125	0.25404	-268.475996572989	-268.475996572989\\
63.125	0.2577	-290.843535640042	-290.843535640042\\
63.125	0.26136	-313.956515207702	-313.956515207702\\
63.125	0.26502	-337.814935275969	-337.814935275969\\
63.125	0.26868	-362.418795844841	-362.418795844841\\
63.125	0.27234	-387.768096914321	-387.768096914321\\
63.125	0.276	-413.862838484408	-413.862838484408\\
63.5	0.093	-25.2439129622476	-25.2439129622476\\
63.5	0.09666	-14.5630051289566	-14.5630051289566\\
63.5	0.10032	-4.62753779627286	-4.62753779627286\\
63.5	0.10398	4.56248903580433	4.56248903580433\\
63.5	0.10764	13.0070753672753	13.0070753672753\\
63.5	0.1113	20.7062211981395	20.7062211981395\\
63.5	0.11496	27.6599265283967	27.6599265283967\\
63.5	0.11862	33.8681913580472	33.8681913580472\\
63.5	0.12228	39.3310156870916	39.3310156870916\\
63.5	0.12594	44.0483995155292	44.0483995155292\\
63.5	0.1296	48.0203428433598	48.0203428433598\\
63.5	0.13326	51.2468456705837	51.2468456705837\\
63.5	0.13692	53.7279079972013	53.7279079972013\\
63.5	0.14058	55.4635298232122	55.4635298232122\\
63.5	0.14424	56.4537111486164	56.4537111486164\\
63.5	0.1479	56.6984519734133	56.6984519734133\\
63.5	0.15156	56.1977522976043	56.1977522976043\\
63.5	0.15522	54.9516121211886	54.9516121211886\\
63.5	0.15888	52.9600314441661	52.9600314441661\\
63.5	0.16254	50.2230102665368	50.2230102665368\\
63.5	0.1662	46.7405485883007	46.7405485883007\\
63.5	0.16986	42.5126464094584	42.5126464094584\\
63.5	0.17352	37.539303730009	37.539303730009\\
63.5	0.17718	31.820520549953	31.820520549953\\
63.5	0.18084	25.3562968692908	25.3562968692908\\
63.5	0.1845	18.1466326880218	18.1466326880218\\
63.5	0.18816	10.191528006146	10.191528006146\\
63.5	0.19182	1.49098282366333	1.49098282366333\\
63.5	0.19548	-7.95500285942603	-7.95500285942603\\
63.5	0.19914	-18.1464290431215	-18.1464290431215\\
63.5	0.2028	-29.0832957274242	-29.0832957274242\\
63.5	0.20646	-40.7656029123336	-40.7656029123336\\
63.5	0.21012	-53.1933505978495	-53.1933505978495\\
63.5	0.21378	-66.3665387839717	-66.3665387839717\\
63.5	0.21744	-80.285167470701	-80.285167470701\\
63.5	0.2211	-94.9492366580364	-94.9492366580364\\
63.5	0.22476	-110.358746345979	-110.358746345979\\
63.5	0.22842	-126.513696534528	-126.513696534528\\
63.5	0.23208	-143.414087223684	-143.414087223684\\
63.5	0.23574	-161.059918413447	-161.059918413447\\
63.5	0.2394	-179.451190103816	-179.451190103816\\
63.5	0.24306	-198.587902294791	-198.587902294791\\
63.5	0.24672	-218.470054986374	-218.470054986374\\
63.5	0.25038	-239.097648178563	-239.097648178563\\
63.5	0.25404	-260.470681871359	-260.470681871359\\
63.5	0.2577	-282.589156064761	-282.589156064761\\
63.5	0.26136	-305.45307075877	-305.45307075877\\
63.5	0.26502	-329.062425953386	-329.062425953386\\
63.5	0.26868	-353.417221648608	-353.417221648608\\
63.5	0.27234	-378.517457844437	-378.517457844437\\
63.5	0.276	-404.363134540873	-404.363134540873\\
63.875	0.093	-28.3479446122114	-28.3479446122114\\
63.875	0.09666	-17.4179719052698	-17.4179719052698\\
63.875	0.10032	-7.23343969893489	-7.23343969893489\\
63.875	0.10398	2.20565200679346	2.20565200679346\\
63.875	0.10764	10.8993032119146	10.8993032119146\\
63.875	0.1113	18.84751391643	18.84751391643\\
63.875	0.11496	26.0502841203379	26.0502841203379\\
63.875	0.11862	32.5076138236396	32.5076138236396\\
63.875	0.12228	38.2195030263346	38.2195030263346\\
63.875	0.12594	43.1859517284225	43.1859517284225\\
63.875	0.1296	47.4069599299047	47.4069599299047\\
63.875	0.13326	50.8825276307793	50.8825276307793\\
63.875	0.13692	53.6126548310481	53.6126548310481\\
63.875	0.14058	55.5973415307092	55.5973415307092\\
63.875	0.14424	56.8365877297646	56.8365877297646\\
63.875	0.1479	57.3303934282126	57.3303934282126\\
63.875	0.15156	57.0787586260539	57.0787586260539\\
63.875	0.15522	56.0816833232893	56.0816833232893\\
63.875	0.15888	54.339167519918	54.339167519918\\
63.875	0.16254	51.8512112159393	51.8512112159393\\
63.875	0.1662	48.6178144113539	48.6178144113539\\
63.875	0.16986	44.6389771061628	44.6389771061628\\
63.875	0.17352	39.9146993003641	39.9146993003641\\
63.875	0.17718	34.4449809939592	34.4449809939592\\
63.875	0.18084	28.2298221869477	28.2298221869477\\
63.875	0.1845	21.2692228793294	21.2692228793294\\
63.875	0.18816	13.5631830711043	13.5631830711043\\
63.875	0.19182	5.11170276227278	5.11170276227278\\
63.875	0.19548	-4.08521804716588	-4.08521804716588\\
63.875	0.19914	-14.0275793572102	-14.0275793572102\\
63.875	0.2028	-24.7153811678622	-24.7153811678622\\
63.875	0.20646	-36.1486234791209	-36.1486234791209\\
63.875	0.21012	-48.3273062909857	-48.3273062909857\\
63.875	0.21378	-61.2514296034572	-61.2514296034572\\
63.875	0.21744	-74.9209934165353	-74.9209934165353\\
63.875	0.2211	-89.3359977302205	-89.3359977302205\\
63.875	0.22476	-104.496442544512	-104.496442544512\\
63.875	0.22842	-120.40232785941	-120.40232785941\\
63.875	0.23208	-137.053653674915	-137.053653674915\\
63.875	0.23574	-154.450419991027	-154.450419991027\\
63.875	0.2394	-172.592626807745	-172.592626807745\\
63.875	0.24306	-191.48027412507	-191.48027412507\\
63.875	0.24672	-211.113361943001	-211.113361943001\\
63.875	0.25038	-231.49189026154	-231.49189026154\\
63.875	0.25404	-252.615859080685	-252.615859080685\\
63.875	0.2577	-274.485268400437	-274.485268400437\\
63.875	0.26136	-297.100118220794	-297.100118220794\\
63.875	0.26502	-320.460408541759	-320.460408541759\\
63.875	0.26868	-344.56613936333	-344.56613936333\\
63.875	0.27234	-369.417310685509	-369.417310685509\\
63.875	0.276	-395.013922508293	-395.013922508293\\
64.25	0.093	-31.602468173131	-31.602468173131\\
64.25	0.09666	-20.4234305925382	-20.4234305925382\\
64.25	0.10032	-9.98983351255259	-9.98983351255259\\
64.25	0.10398	-0.301676933173553	-0.301676933173553\\
64.25	0.10764	8.64103914559877	8.64103914559877\\
64.25	0.1113	16.8383147237644	16.8383147237644\\
64.25	0.11496	24.2901498013234	24.2901498013234\\
64.25	0.11862	30.9965443782763	30.9965443782763\\
64.25	0.12228	36.9574984546215	36.9574984546215\\
64.25	0.12594	42.173012030361	42.173012030361\\
64.25	0.1296	46.6430851054939	46.6430851054939\\
64.25	0.13326	50.3677176800197	50.3677176800197\\
64.25	0.13692	53.3469097539387	53.3469097539387\\
64.25	0.14058	55.5806613272509	55.5806613272509\\
64.25	0.14424	57.0689723999566	57.0689723999566\\
64.25	0.1479	57.8118429720557	57.8118429720557\\
64.25	0.15156	57.8092730435482	57.8092730435482\\
64.25	0.15522	57.0612626144343	57.0612626144343\\
64.25	0.15888	55.5678116847137	55.5678116847137\\
64.25	0.16254	53.3289202543862	53.3289202543862\\
64.25	0.1662	50.3445883234519	50.3445883234519\\
64.25	0.16986	46.614815891911	46.614815891911\\
64.25	0.17352	42.1396029597635	42.1396029597635\\
64.25	0.17718	36.9189495270093	36.9189495270093\\
64.25	0.18084	30.9528555936485	30.9528555936485\\
64.25	0.1845	24.2413211596813	24.2413211596813\\
64.25	0.18816	16.7843462251074	16.7843462251074\\
64.25	0.19182	8.58193078992656	8.58193078992656\\
64.25	0.19548	-0.365925145861411	-0.365925145861411\\
64.25	0.19914	-10.0592215822551	-10.0592215822551\\
64.25	0.2028	-20.4979585192559	-20.4979585192559\\
64.25	0.20646	-31.6821359568635	-31.6821359568635\\
64.25	0.21012	-43.6117538950775	-43.6117538950775\\
64.25	0.21378	-56.2868123338978	-56.2868123338978\\
64.25	0.21744	-69.7073112733258	-69.7073112733258\\
64.25	0.2211	-83.8732507133598	-83.8732507133598\\
64.25	0.22476	-98.7846306540002	-98.7846306540002\\
64.25	0.22842	-114.441451095248	-114.441451095248\\
64.25	0.23208	-130.843712037102	-130.843712037102\\
64.25	0.23574	-147.991413479563	-147.991413479563\\
64.25	0.2394	-165.88455542263	-165.88455542263\\
64.25	0.24306	-184.523137866304	-184.523137866304\\
64.25	0.24672	-203.907160810585	-203.907160810585\\
64.25	0.25038	-224.036624255472	-224.036624255472\\
64.25	0.25404	-244.911528200966	-244.911528200966\\
64.25	0.2577	-266.531872647067	-266.531872647067\\
64.25	0.26136	-288.897657593775	-288.897657593775\\
64.25	0.26502	-312.008883041088	-312.008883041088\\
64.25	0.26868	-335.865548989009	-335.865548989009\\
64.25	0.27234	-360.467655437536	-360.467655437536\\
64.25	0.276	-385.81520238667	-385.81520238667\\
64.625	0.093	-35.0074836450071	-35.0074836450071\\
64.625	0.09666	-23.5793811907636	-23.5793811907636\\
64.625	0.10032	-12.8967192371273	-12.8967192371273\\
64.625	0.10398	-2.95949778409715	-2.95949778409715\\
64.625	0.10764	6.23228316832586	6.23228316832586\\
64.625	0.1113	14.6786236201427	14.6786236201427\\
64.625	0.11496	22.3795235713528	22.3795235713528\\
64.625	0.11862	29.3349830219564	29.3349830219564\\
64.625	0.12228	35.5450019719523	35.5450019719523\\
64.625	0.12594	41.0095804213425	41.0095804213425\\
64.625	0.1296	45.7287183701261	45.7287183701261\\
64.625	0.13326	49.702415818303	49.702415818303\\
64.625	0.13692	52.9306727658727	52.9306727658727\\
64.625	0.14058	55.4134892128361	55.4134892128361\\
64.625	0.14424	57.1508651591929	57.1508651591929\\
64.625	0.1479	58.1428006049427	58.1428006049427\\
64.625	0.15156	58.3892955500863	58.3892955500863\\
64.625	0.15522	57.8903499946227	57.8903499946227\\
64.625	0.15888	56.6459639385532	56.6459639385532\\
64.625	0.16254	54.6561373818764	54.6561373818764\\
64.625	0.1662	51.9208703245929	51.9208703245929\\
64.625	0.16986	48.4401627667031	48.4401627667031\\
64.625	0.17352	44.2140147082067	44.2140147082067\\
64.625	0.17718	39.2424261491033	39.2424261491033\\
64.625	0.18084	33.5253970893931	33.5253970893931\\
64.625	0.1845	27.0629275290767	27.0629275290767\\
64.625	0.18816	19.8550174681534	19.8550174681534\\
64.625	0.19182	11.9016669066233	11.9016669066233\\
64.625	0.19548	3.20287584448693	3.20287584448693\\
64.625	0.19914	-6.24135571825605	-6.24135571825605\\
64.625	0.2028	-16.4310277816062	-16.4310277816062\\
64.625	0.20646	-27.3661403455631	-27.3661403455631\\
64.625	0.21012	-39.0466934101264	-39.0466934101264\\
64.625	0.21378	-51.4726869752956	-51.4726869752956\\
64.625	0.21744	-64.6441210410724	-64.6441210410724\\
64.625	0.2211	-78.5609956074557	-78.5609956074557\\
64.625	0.22476	-93.2233106744454	-93.2233106744454\\
64.625	0.22842	-108.631066242042	-108.631066242042\\
64.625	0.23208	-124.784262310245	-124.784262310245\\
64.625	0.23574	-141.682898879055	-141.682898879055\\
64.625	0.2394	-159.326975948472	-159.326975948472\\
64.625	0.24306	-177.716493518495	-177.716493518495\\
64.625	0.24672	-196.851451589125	-196.851451589125\\
64.625	0.25038	-216.731850160361	-216.731850160361\\
64.625	0.25404	-237.357689232204	-237.357689232204\\
64.625	0.2577	-258.728968804655	-258.728968804655\\
64.625	0.26136	-280.845688877711	-280.845688877711\\
64.625	0.26502	-303.707849451374	-303.707849451374\\
64.625	0.26868	-327.315450525643	-327.315450525643\\
64.625	0.27234	-351.66849210052	-351.66849210052\\
64.625	0.276	-376.766974176003	-376.766974176003\\
65	0.093	-38.562991027839	-38.562991027839\\
65	0.09666	-26.8858236999448	-26.8858236999448\\
65	0.10032	-15.9540968726573	-15.9540968726573\\
65	0.10398	-5.76781054597643	-5.76781054597643\\
65	0.10764	3.67303528009728	3.67303528009728\\
65	0.1113	12.3684406055652	12.3684406055652\\
65	0.11496	20.3184054304261	20.3184054304261\\
65	0.11862	27.5229297546803	27.5229297546803\\
65	0.12228	33.9820135783274	33.9820135783274\\
65	0.12594	39.6956569013687	39.6956569013687\\
65	0.1296	44.6638597238031	44.6638597238031\\
65	0.13326	48.8866220456306	48.8866220456306\\
65	0.13692	52.3639438668511	52.3639438668511\\
65	0.14058	55.0958251874656	55.0958251874656\\
65	0.14424	57.0822660074731	57.0822660074731\\
65	0.1479	58.3232663268741	58.3232663268741\\
65	0.15156	58.8188261456679	58.8188261456679\\
65	0.15522	58.5689454638559	58.5689454638559\\
65	0.15888	57.5736242814367	57.5736242814367\\
65	0.16254	55.832862598411	55.832862598411\\
65	0.1662	53.3466604147782	53.3466604147782\\
65	0.16986	50.1150177305395	50.1150177305395\\
65	0.17352	46.1379345456934	46.1379345456934\\
65	0.17718	41.4154108602411	41.4154108602411\\
65	0.18084	35.9474466741821	35.9474466741821\\
65	0.1845	29.7340419875163	29.7340419875163\\
65	0.18816	22.7751968002442	22.7751968002442\\
65	0.19182	15.0709111123648	15.0709111123648\\
65	0.19548	6.62118492387913	6.62118492387913\\
65	0.19914	-2.57398176521269	-2.57398176521269\\
65	0.2028	-12.5145889549121	-12.5145889549121\\
65	0.20646	-23.2006366452179	-23.2006366452179\\
65	0.21012	-34.6321248361305	-34.6321248361305\\
65	0.21378	-46.809053527649	-46.809053527649\\
65	0.21744	-59.7314227197751	-59.7314227197751\\
65	0.2211	-73.3992324125072	-73.3992324125072\\
65	0.22476	-87.8124826058463	-87.8124826058463\\
65	0.22842	-102.971173299792	-102.971173299792\\
65	0.23208	-118.875304494344	-118.875304494344\\
65	0.23574	-135.524876189504	-135.524876189504\\
65	0.2394	-152.919888385269	-152.919888385269\\
65	0.24306	-171.060341081641	-171.060341081641\\
65	0.24672	-189.94623427862	-189.94623427862\\
65	0.25038	-209.577567976206	-209.577567976206\\
65	0.25404	-229.954342174398	-229.954342174398\\
65	0.2577	-251.076556873197	-251.076556873197\\
65	0.26136	-272.944212072603	-272.944212072603\\
65	0.26502	-295.557307772615	-295.557307772615\\
65	0.26868	-318.915843973234	-318.915843973234\\
65	0.27234	-343.019820674459	-343.019820674459\\
65	0.276	-367.869237876292	-367.869237876292\\
65.375	0.093	-42.2689903216265	-42.2689903216265\\
65.375	0.09666	-30.3427581200816	-30.3427581200816\\
65.375	0.10032	-19.1619664191434	-19.1619664191434\\
65.375	0.10398	-8.72661521881139	-8.72661521881139\\
65.375	0.10764	0.963295480913473	0.963295480913473\\
65.375	0.1113	9.90776568003213	9.90776568003213\\
65.375	0.11496	18.1067953785437	18.1067953785437\\
65.375	0.11862	25.5603845764486	25.5603845764486\\
65.375	0.12228	32.2685332737469	32.2685332737469\\
65.375	0.12594	38.2312414704389	38.2312414704389\\
65.375	0.1296	43.4485091665243	43.4485091665243\\
65.375	0.13326	47.9203363620026	47.9203363620026\\
65.375	0.13692	51.6467230568742	51.6467230568742\\
65.375	0.14058	54.627669251139	54.627669251139\\
65.375	0.14424	56.8631749447976	56.8631749447976\\
65.375	0.1479	58.3532401378493	58.3532401378493\\
65.375	0.15156	59.0978648302943	59.0978648302943\\
65.375	0.15522	59.097049022133	59.097049022133\\
65.375	0.15888	58.3507927133649	58.3507927133649\\
65.375	0.16254	56.8590959039899	56.8590959039899\\
65.375	0.1662	54.6219585940082	54.6219585940082\\
65.375	0.16986	51.6393807834198	51.6393807834198\\
65.375	0.17352	47.9113624722253	47.9113624722253\\
65.375	0.17718	43.4379036604237	43.4379036604237\\
65.375	0.18084	38.2190043480153	38.2190043480153\\
65.375	0.1845	32.2546645350008	32.2546645350008\\
65.375	0.18816	25.5448842213789	25.5448842213789\\
65.375	0.19182	18.0896634071506	18.0896634071506\\
65.375	0.19548	9.88900209231565	9.88900209231565\\
65.375	0.19914	0.942900276874525	0.942900276874525\\
65.375	0.2028	-8.74864203917377	-8.74864203917377\\
65.375	0.20646	-19.1856248558288	-19.1856248558288\\
65.375	0.21012	-30.3680481730903	-30.3680481730903\\
65.375	0.21378	-42.2959119909581	-42.2959119909581\\
65.375	0.21744	-54.969216309433	-54.969216309433\\
65.375	0.2211	-68.3879611285145	-68.3879611285145\\
65.375	0.22476	-82.5521464482028	-82.5521464482028\\
65.375	0.22842	-97.4617722684973	-97.4617722684973\\
65.375	0.23208	-113.116838589399	-113.116838589399\\
65.375	0.23574	-129.517345410907	-129.517345410907\\
65.375	0.2394	-146.663292733022	-146.663292733022\\
65.375	0.24306	-164.554680555743	-164.554680555743\\
65.375	0.24672	-183.191508879071	-183.191508879071\\
65.375	0.25038	-202.573777703006	-202.573777703006\\
65.375	0.25404	-222.701487027548	-222.701487027548\\
65.375	0.2577	-243.574636852696	-243.574636852696\\
65.375	0.26136	-265.193227178451	-265.193227178451\\
65.375	0.26502	-287.557258004812	-287.557258004812\\
65.375	0.26868	-310.66672933178	-310.66672933178\\
65.375	0.27234	-334.521641159355	-334.521641159355\\
65.375	0.276	-359.121993487536	-359.121993487536\\
65.75	0.093	-46.1254815263696	-46.1254815263696\\
65.75	0.09666	-33.9501844511736	-33.9501844511736\\
65.75	0.10032	-22.5203278765843	-22.5203278765843\\
65.75	0.10398	-11.8359118026016	-11.8359118026016\\
65.75	0.10764	-1.89693622922647	-1.89693622922647\\
65.75	0.1113	7.29659884354334	7.29659884354334\\
65.75	0.11496	15.7446934157056	15.7446934157056\\
65.75	0.11862	23.4473474872617	23.4473474872617\\
65.75	0.12228	30.4045610582111	30.4045610582111\\
65.75	0.12594	36.6163341285538	36.6163341285538\\
65.75	0.1296	42.0826666982895	42.0826666982895\\
65.75	0.13326	46.8035587674185	46.8035587674185\\
65.75	0.13692	50.7790103359412	50.7790103359412\\
65.75	0.14058	54.0090214038571	54.0090214038571\\
65.75	0.14424	56.4935919711664	56.4935919711664\\
65.75	0.1479	58.2327220378693	58.2327220378693\\
65.75	0.15156	59.2264116039649	59.2264116039649\\
65.75	0.15522	59.4746606694544	59.4746606694544\\
65.75	0.15888	58.9774692343374	58.9774692343374\\
65.75	0.16254	57.7348372986131	57.7348372986131\\
65.75	0.1662	55.7467648622821	55.7467648622821\\
65.75	0.16986	53.0132519253449	53.0132519253449\\
65.75	0.17352	49.534298487801	49.534298487801\\
65.75	0.17718	45.3099045496501	45.3099045496501\\
65.75	0.18084	40.3400701108934	40.3400701108934\\
65.75	0.1845	34.6247951715291	34.6247951715291\\
65.75	0.18816	28.1640797315583	28.1640797315583\\
65.75	0.19182	20.9579237909808	20.9579237909808\\
65.75	0.19548	13.0063273497969	13.0063273497969\\
65.75	0.19914	4.30929040800652	4.30929040800652\\
65.75	0.2028	-5.13318703439109	-5.13318703439109\\
65.75	0.20646	-15.3211049773954	-15.3211049773954\\
65.75	0.21012	-26.2544634210058	-26.2544634210058\\
65.75	0.21378	-37.9332623652233	-37.9332623652233\\
65.75	0.21744	-50.3575018100466	-50.3575018100466\\
65.75	0.2211	-63.5271817554774	-63.5271817554774\\
65.75	0.22476	-77.4423022015146	-77.4423022015146\\
65.75	0.22842	-92.1028631481583	-92.1028631481583\\
65.75	0.23208	-107.508864595409	-107.508864595409\\
65.75	0.23574	-123.660306543267	-123.660306543267\\
65.75	0.2394	-140.557188991731	-140.557188991731\\
65.75	0.24306	-158.199511940801	-158.199511940801\\
65.75	0.24672	-176.587275390478	-176.587275390478\\
65.75	0.25038	-195.720479340762	-195.720479340762\\
65.75	0.25404	-215.599123791653	-215.599123791653\\
65.75	0.2577	-236.22320874315	-236.22320874315\\
65.75	0.26136	-257.592734195254	-257.592734195254\\
65.75	0.26502	-279.707700147964	-279.707700147964\\
65.75	0.26868	-302.568106601282	-302.568106601282\\
65.75	0.27234	-326.173953555206	-326.173953555206\\
65.75	0.276	-350.525241009736	-350.525241009736\\
66.125	0.093	-50.132464642069	-50.132464642069\\
66.125	0.09666	-37.7081026932222	-37.7081026932222\\
66.125	0.10032	-26.0291812449822	-26.0291812449822\\
66.125	0.10398	-15.0957002973488	-15.0957002973488\\
66.125	0.10764	-4.90765985032209	-4.90765985032209\\
66.125	0.1113	4.53494009609796	4.53494009609796\\
66.125	0.11496	13.2320995419118	13.2320995419118\\
66.125	0.11862	21.1838184871181	21.1838184871181\\
66.125	0.12228	28.3900969317187	28.3900969317187\\
66.125	0.12594	34.8509348757116	34.8509348757116\\
66.125	0.1296	40.5663323190989	40.5663323190989\\
66.125	0.13326	45.5362892618786	45.5362892618786\\
66.125	0.13692	49.7608057040525	49.7608057040525\\
66.125	0.14058	53.2398816456191	53.2398816456191\\
66.125	0.14424	55.9735170865791	55.9735170865791\\
66.125	0.1479	57.9617120269327	57.9617120269327\\
66.125	0.15156	59.2044664666795	59.2044664666795\\
66.125	0.15522	59.7017804058196	59.7017804058196\\
66.125	0.15888	59.4536538443533	59.4536538443533\\
66.125	0.16254	58.4600867822802	58.4600867822802\\
66.125	0.1662	56.7210792195999	56.7210792195999\\
66.125	0.16986	54.2366311563138	54.2366311563138\\
66.125	0.17352	51.0067425924207	51.0067425924207\\
66.125	0.17718	47.0314135279209	47.0314135279209\\
66.125	0.18084	42.310643962814	42.310643962814\\
66.125	0.1845	36.8444338971012	36.8444338971012\\
66.125	0.18816	30.6327833307816	30.6327833307816\\
66.125	0.19182	23.6756922638548	23.6756922638548\\
66.125	0.19548	15.9731606963217	15.9731606963217\\
66.125	0.19914	7.52518862818238	7.52518862818238\\
66.125	0.2028	-1.66822394056453	-1.66822394056453\\
66.125	0.20646	-11.6070770099177	-11.6070770099177\\
66.125	0.21012	-22.2913705798774	-22.2913705798774\\
66.125	0.21378	-33.7211046504442	-33.7211046504442\\
66.125	0.21744	-45.8962792216169	-45.8962792216169\\
66.125	0.2211	-58.8168942933964	-58.8168942933964\\
66.125	0.22476	-72.482949865783	-72.482949865783\\
66.125	0.22842	-86.894445938776	-86.894445938776\\
66.125	0.23208	-102.051382512376	-102.051382512376\\
66.125	0.23574	-117.953759586583	-117.953759586583\\
66.125	0.2394	-134.601577161396	-134.601577161396\\
66.125	0.24306	-151.994835236815	-151.994835236815\\
66.125	0.24672	-170.133533812841	-170.133533812841\\
66.125	0.25038	-189.017672889475	-189.017672889475\\
66.125	0.25404	-208.647252466714	-208.647252466714\\
66.125	0.2577	-229.022272544561	-229.022272544561\\
66.125	0.26136	-250.142733123014	-250.142733123014\\
66.125	0.26502	-272.008634202073	-272.008634202073\\
66.125	0.26868	-294.619975781739	-294.619975781739\\
66.125	0.27234	-317.976757862013	-317.976757862013\\
66.125	0.276	-342.078980442892	-342.078980442892\\
66.5	0.093	-54.2899396687244	-54.2899396687244\\
66.5	0.09666	-41.616512846227	-41.616512846227\\
66.5	0.10032	-29.6885265243358	-29.6885265243358\\
66.5	0.10398	-18.5059807030513	-18.5059807030513\\
66.5	0.10764	-8.06887538237385	-8.06887538237385\\
66.5	0.1113	1.62278943769689	1.62278943769689\\
66.5	0.11496	10.569013757161	10.569013757161\\
66.5	0.11862	18.7697975760189	18.7697975760189\\
66.5	0.12228	26.2251408942702	26.2251408942702\\
66.5	0.12594	32.9350437119143	32.9350437119143\\
66.5	0.1296	38.8995060289518	38.8995060289518\\
66.5	0.13326	44.1185278453826	44.1185278453826\\
66.5	0.13692	48.5921091612072	48.5921091612072\\
66.5	0.14058	52.320249976425	52.320249976425\\
66.5	0.14424	55.3029502910357	55.3029502910357\\
66.5	0.1479	57.5402101050399	57.5402101050399\\
66.5	0.15156	59.0320294184374	59.0320294184374\\
66.5	0.15522	59.7784082312292	59.7784082312292\\
66.5	0.15888	59.7793465434131	59.7793465434131\\
66.5	0.16254	59.0348443549907	59.0348443549907\\
66.5	0.1662	57.544901665962	57.544901665962\\
66.5	0.16986	55.3095184763266	55.3095184763266\\
66.5	0.17352	52.3286947860842	52.3286947860842\\
66.5	0.17718	48.6024305952351	48.6024305952351\\
66.5	0.18084	44.1307259037793	44.1307259037793\\
66.5	0.1845	38.9135807117173	38.9135807117173\\
66.5	0.18816	32.9509950190484	32.9509950190484\\
66.5	0.19182	26.2429688257727	26.2429688257727\\
66.5	0.19548	18.7895021318902	18.7895021318902\\
66.5	0.19914	10.5905949374016	10.5905949374016\\
66.5	0.2028	1.64624724230589	1.64624724230589\\
66.5	0.20646	-8.04354095339659	-8.04354095339659\\
66.5	0.21012	-18.4787696497056	-18.4787696497056\\
66.5	0.21378	-29.6594388466212	-29.6594388466212\\
66.5	0.21744	-41.5855485441427	-41.5855485441427\\
66.5	0.2211	-54.2570987422716	-54.2570987422716\\
66.5	0.22476	-67.6740894410075	-67.6740894410075\\
66.5	0.22842	-81.8365206403494	-81.8365206403494\\
66.5	0.23208	-96.7443923402984	-96.7443923402984\\
66.5	0.23574	-112.397704540854	-112.397704540854\\
66.5	0.2394	-128.796457242016	-128.796457242016\\
66.5	0.24306	-145.940650443785	-145.940650443785\\
66.5	0.24672	-163.830284146161	-163.830284146161\\
66.5	0.25038	-182.465358349143	-182.465358349143\\
66.5	0.25404	-201.845873052732	-201.845873052732\\
66.5	0.2577	-221.971828256928	-221.971828256928\\
66.5	0.26136	-242.84322396173	-242.84322396173\\
66.5	0.26502	-264.460060167138	-264.460060167138\\
66.5	0.26868	-286.822336873154	-286.822336873154\\
66.5	0.27234	-309.930054079776	-309.930054079776\\
66.5	0.276	-333.783211787005	-333.783211787005\\
66.875	0.093	-58.5979066063351	-58.5979066063351\\
66.875	0.09666	-45.6754149101865	-45.6754149101865\\
66.875	0.10032	-33.4983637146447	-33.4983637146447\\
66.875	0.10398	-22.0667530197094	-22.0667530197094\\
66.875	0.10764	-11.3805828253813	-11.3805828253813\\
66.875	0.1113	-1.43985313165894	-1.43985313165894\\
66.875	0.11496	7.75543606145584	7.75543606145584\\
66.875	0.11862	16.2052847539645	16.2052847539645\\
66.875	0.12228	23.909692945866	23.909692945866\\
66.875	0.12594	30.8686606371612	30.8686606371612\\
66.875	0.1296	37.0821878278499	37.0821878278499\\
66.875	0.13326	42.5502745179319	42.5502745179319\\
66.875	0.13692	47.2729207074071	47.2729207074071\\
66.875	0.14058	51.2501263962756	51.2501263962756\\
66.875	0.14424	54.4818915845375	54.4818915845375\\
66.875	0.1479	56.9682162721924	56.9682162721924\\
66.875	0.15156	58.7091004592406	58.7091004592406\\
66.875	0.15522	59.7045441456826	59.7045441456826\\
66.875	0.15888	59.9545473315186	59.9545473315186\\
66.875	0.16254	59.4591100167464	59.4591100167464\\
66.875	0.1662	58.2182322013684	58.2182322013684\\
66.875	0.16986	56.2319138853837	56.2319138853837\\
66.875	0.17352	53.5001550687924	53.5001550687924\\
66.875	0.17718	50.0229557515945	50.0229557515945\\
66.875	0.18084	45.8003159337899	45.8003159337899\\
66.875	0.1845	40.8322356153781	40.8322356153781\\
66.875	0.18816	35.1187147963603	35.1187147963603\\
66.875	0.19182	28.6597534767353	28.6597534767353\\
66.875	0.19548	21.455351656504	21.455351656504\\
66.875	0.19914	13.5055093356661	13.5055093356661\\
66.875	0.2028	4.81022651422109	4.81022651422109\\
66.875	0.20646	-4.63049680783024	-4.63049680783024\\
66.875	0.21012	-14.8166606304885	-14.8166606304885\\
66.875	0.21378	-25.7482649537531	-25.7482649537531\\
66.875	0.21744	-37.4253097776243	-37.4253097776243\\
66.875	0.2211	-49.8477951021021	-49.8477951021021\\
66.875	0.22476	-63.0157209271867	-63.0157209271867\\
66.875	0.22842	-76.9290872528784	-76.9290872528784\\
66.875	0.23208	-91.5878940791763	-91.5878940791763\\
66.875	0.23574	-106.992141406081	-106.992141406081\\
66.875	0.2394	-123.141829233592	-123.141829233592\\
66.875	0.24306	-140.03695756171	-140.03695756171\\
66.875	0.24672	-157.677526390435	-157.677526390435\\
66.875	0.25038	-176.063535719766	-176.063535719766\\
66.875	0.25404	-195.194985549705	-195.194985549705\\
66.875	0.2577	-215.071875880249	-215.071875880249\\
66.875	0.26136	-235.694206711401	-235.694206711401\\
66.875	0.26502	-257.061978043158	-257.061978043158\\
66.875	0.26868	-279.175189875523	-279.175189875523\\
66.875	0.27234	-302.033842208495	-302.033842208495\\
66.875	0.276	-325.637935042072	-325.637935042072\\
67.25	0.093	-63.0563654549023	-63.0563654549023\\
67.25	0.09666	-49.884808885103	-49.884808885103\\
67.25	0.10032	-37.4586928159101	-37.4586928159101\\
67.25	0.10398	-25.7780172473246	-25.7780172473246\\
67.25	0.10764	-14.8427821793449	-14.8427821793449\\
67.25	0.1113	-4.65298761197226	-4.65298761197226\\
67.25	0.11496	4.79136645479366	4.79136645479366\\
67.25	0.11862	13.4902800209534	13.4902800209534\\
67.25	0.12228	21.4437530865056	21.4437530865056\\
67.25	0.12594	28.651785651452	28.651785651452\\
67.25	0.1296	35.1143777157914	35.1143777157914\\
67.25	0.13326	40.8315292795241	40.8315292795241\\
67.25	0.13692	45.80324034265	45.80324034265\\
67.25	0.14058	50.0295109051692	50.0295109051692\\
67.25	0.14424	53.5103409670822	53.5103409670822\\
67.25	0.1479	56.2457305283879	56.2457305283879\\
67.25	0.15156	58.2356795890877	58.2356795890877\\
67.25	0.15522	59.4801881491803	59.4801881491803\\
67.25	0.15888	59.9792562086666	59.9792562086666\\
67.25	0.16254	59.732883767546	59.732883767546\\
67.25	0.1662	58.7410708258183	58.7410708258183\\
67.25	0.16986	57.0038173834847	57.0038173834847\\
67.25	0.17352	54.5211234405441	54.5211234405441\\
67.25	0.17718	51.2929889969969	51.2929889969969\\
67.25	0.18084	47.3194140528434	47.3194140528434\\
67.25	0.1845	42.6003986080827	42.6003986080827\\
67.25	0.18816	37.1359426627153	37.1359426627153\\
67.25	0.19182	30.9260462167414	30.9260462167414\\
67.25	0.19548	23.9707092701608	23.9707092701608\\
67.25	0.19914	16.2699318229741	16.2699318229741\\
67.25	0.2028	7.82371387517969	7.82371387517969\\
67.25	0.20646	-1.36794457322094	-1.36794457322094\\
67.25	0.21012	-11.3050435222281	-11.3050435222281\\
67.25	0.21378	-21.9875829718419	-21.9875829718419\\
67.25	0.21744	-33.415562922062	-33.415562922062\\
67.25	0.2211	-45.5889833728895	-45.5889833728895\\
67.25	0.22476	-58.5078443243231	-58.5078443243231\\
67.25	0.22842	-72.1721457763635	-72.1721457763635\\
67.25	0.23208	-86.5818877290108	-86.5818877290108\\
67.25	0.23574	-101.737070182265	-101.737070182265\\
67.25	0.2394	-117.637693136126	-117.637693136126\\
67.25	0.24306	-134.283756590592	-134.283756590592\\
67.25	0.24672	-151.675260545666	-151.675260545666\\
67.25	0.25038	-169.812205001346	-169.812205001346\\
67.25	0.25404	-188.694589957634	-188.694589957634\\
67.25	0.2577	-208.322415414528	-208.322415414528\\
67.25	0.26136	-228.695681372028	-228.695681372028\\
67.25	0.26502	-249.814387830135	-249.814387830135\\
67.25	0.26868	-271.678534788849	-271.678534788849\\
67.25	0.27234	-294.288122248169	-294.288122248169\\
67.25	0.276	-317.643150208096	-317.643150208096\\
67.625	0.093	-67.6653162144253	-67.6653162144253\\
67.625	0.09666	-54.2446947709753	-54.2446947709753\\
67.625	0.10032	-41.5695138281316	-41.5695138281316\\
67.625	0.10398	-29.6397733858945	-29.6397733858945\\
67.625	0.10764	-18.4554734442646	-18.4554734442646\\
67.625	0.1113	-8.01661400324082	-8.01661400324082\\
67.625	0.11496	1.6768049371758	1.6768049371758\\
67.625	0.11862	10.6247833769863	10.6247833769863\\
67.625	0.12228	18.8273213161896	18.8273213161896\\
67.625	0.12594	26.2844187547867	26.2844187547867\\
67.625	0.1296	32.9960756927773	32.9960756927773\\
67.625	0.13326	38.9622921301606	38.9622921301606\\
67.625	0.13692	44.1830680669373	44.1830680669373\\
67.625	0.14058	48.658403503108	48.658403503108\\
67.625	0.14424	52.3882984386713	52.3882984386713\\
67.625	0.1479	55.3727528736281	55.3727528736281\\
67.625	0.15156	57.6117668079781	57.6117668079781\\
67.625	0.15522	59.1053402417219	59.1053402417219\\
67.625	0.15888	59.8534731748589	59.8534731748589\\
67.625	0.16254	59.856165607389	59.856165607389\\
67.625	0.1662	59.1134175393129	59.1134175393129\\
67.625	0.16986	57.62522897063	57.62522897063\\
67.625	0.17352	55.3915999013401	55.3915999013401\\
67.625	0.17718	52.412530331444	52.412530331444\\
67.625	0.18084	48.6880202609412	48.6880202609412\\
67.625	0.1845	44.2180696898313	44.2180696898313\\
67.625	0.18816	39.0026786181149	39.0026786181149\\
67.625	0.19182	33.0418470457918	33.0418470457918\\
67.625	0.19548	26.3355749728619	26.3355749728619\\
67.625	0.19914	18.8838623993258	18.8838623993258\\
67.625	0.2028	10.6867093251826	10.6867093251826\\
67.625	0.20646	1.74411575043314	1.74411575043314\\
67.625	0.21012	-7.94391832492329	-7.94391832492329\\
67.625	0.21378	-18.3773929008864	-18.3773929008864\\
67.625	0.21744	-29.5563079774558	-29.5563079774558\\
67.625	0.2211	-41.4806635546322	-41.4806635546322\\
67.625	0.22476	-54.150459632415	-54.150459632415\\
67.625	0.22842	-67.5656962108044	-67.5656962108044\\
67.625	0.23208	-81.7263732898009	-81.7263732898009\\
67.625	0.23574	-96.6324908694041	-96.6324908694041\\
67.625	0.2394	-112.284048949614	-112.284048949614\\
67.625	0.24306	-128.68104753043	-128.68104753043\\
67.625	0.24672	-145.823486611853	-145.823486611853\\
67.625	0.25038	-163.711366193883	-163.711366193883\\
67.625	0.25404	-182.344686276519	-182.344686276519\\
67.625	0.2577	-201.723446859761	-201.723446859761\\
67.625	0.26136	-221.847647943611	-221.847647943611\\
67.625	0.26502	-242.717289528068	-242.717289528068\\
67.625	0.26868	-264.332371613131	-264.332371613131\\
67.625	0.27234	-286.692894198799	-286.692894198799\\
67.625	0.276	-309.798857285076	-309.798857285076\\
68	0.093	-72.4247588849043	-72.4247588849043\\
68	0.09666	-58.7550725678032	-58.7550725678032\\
68	0.10032	-45.8308267513088	-45.8308267513088\\
68	0.10398	-33.652021435421	-33.652021435421\\
68	0.10764	-22.2186566201404	-22.2186566201404\\
68	0.1113	-11.5307323054655	-11.5307323054655\\
68	0.11496	-1.58824849139819	-1.58824849139819\\
68	0.11862	7.60879482206298	7.60879482206298\\
68	0.12228	16.0603976349175	16.0603976349175\\
68	0.12594	23.7665599471653	23.7665599471653\\
68	0.1296	30.7272817588065	30.7272817588065\\
68	0.13326	36.942563069841	36.942563069841\\
68	0.13692	42.4124038802688	42.4124038802688\\
68	0.14058	47.1368041900898	47.1368041900898\\
68	0.14424	51.1157639993042	51.1157639993042\\
68	0.1479	54.3492833079122	54.3492833079122\\
68	0.15156	56.8373621159134	56.8373621159134\\
68	0.15522	58.5800004233079	58.5800004233079\\
68	0.15888	59.5771982300955	59.5771982300955\\
68	0.16254	59.8289555362768	59.8289555362768\\
68	0.1662	59.3352723418509	59.3352723418509\\
68	0.16986	58.0961486468192	58.0961486468192\\
68	0.17352	56.11158445118	56.11158445118\\
68	0.17718	53.3815797549346	53.3815797549346\\
68	0.18084	49.906134558083	49.906134558083\\
68	0.1845	45.6852488606237	45.6852488606237\\
68	0.18816	40.718922662558	40.718922662558\\
68	0.19182	35.007155963886	35.007155963886\\
68	0.19548	28.5499487646073	28.5499487646073\\
68	0.19914	21.3473010647219	21.3473010647219\\
68	0.2028	13.3992128642294	13.3992128642294\\
68	0.20646	4.70568416313063	4.70568416313063\\
68	0.21012	-4.73328503857465	-4.73328503857465\\
68	0.21378	-14.9176947408871	-14.9176947408871\\
68	0.21744	-25.8475449438054	-25.8475449438054\\
68	0.2211	-37.522835647331	-37.522835647331\\
68	0.22476	-49.9435668514632	-49.9435668514632\\
68	0.22842	-63.1097385562018	-63.1097385562018\\
68	0.23208	-77.0213507615472	-77.0213507615472\\
68	0.23574	-91.6784034674997	-91.6784034674997\\
68	0.2394	-107.080896674058	-107.080896674058\\
68	0.24306	-123.228830381224	-123.228830381224\\
68	0.24672	-140.122204588996	-140.122204588996\\
68	0.25038	-157.761019297375	-157.761019297375\\
68	0.25404	-176.14527450636	-176.14527450636\\
68	0.2577	-195.274970215952	-195.274970215952\\
68	0.26136	-215.150106426151	-215.150106426151\\
68	0.26502	-235.770683136956	-235.770683136956\\
68	0.26868	-257.136700348368	-257.136700348368\\
68	0.27234	-279.248158060386	-279.248158060386\\
68	0.276	-302.105056273012	-302.105056273012\\
68.375	0.093	-77.3346934663386	-77.3346934663386\\
68.375	0.09666	-63.4159422755868	-63.4159422755868\\
68.375	0.10032	-50.2426315854413	-50.2426315854413\\
68.375	0.10398	-37.8147613959028	-37.8147613959028\\
68.375	0.10764	-26.132331706971	-26.132331706971\\
68.375	0.1113	-15.1953425186454	-15.1953425186454\\
68.375	0.11496	-5.00379383092741	-5.00379383092741\\
68.375	0.11862	4.44231435618491	4.44231435618491\\
68.375	0.12228	13.1429820426901	13.1429820426901\\
68.375	0.12594	21.0982092285886	21.0982092285886\\
68.375	0.1296	28.307995913881	28.307995913881\\
68.375	0.13326	34.7723420985662	34.7723420985662\\
68.375	0.13692	40.4912477826447	40.4912477826447\\
68.375	0.14058	45.4647129661168	45.4647129661168\\
68.375	0.14424	49.6927376489824	49.6927376489824\\
68.375	0.1479	53.1753218312406	53.1753218312406\\
68.375	0.15156	55.9124655128925	55.9124655128925\\
68.375	0.15522	57.9041686939377	57.9041686939377\\
68.375	0.15888	59.1504313743769	59.1504313743769\\
68.375	0.16254	59.6512535542089	59.6512535542089\\
68.375	0.1662	59.4066352334341	59.4066352334341\\
68.375	0.16986	58.4165764120527	58.4165764120527\\
68.375	0.17352	56.6810770900651	56.6810770900651\\
68.375	0.17718	54.2001372674699	54.2001372674699\\
68.375	0.18084	50.973756944269	50.973756944269\\
68.375	0.1845	47.0019361204609	47.0019361204609\\
68.375	0.18816	42.2846747960464	42.2846747960464\\
68.375	0.19182	36.821972971025	36.821972971025\\
68.375	0.19548	30.613830645397	30.613830645397\\
68.375	0.19914	23.6602478191628	23.6602478191628\\
68.375	0.2028	15.961224492321	15.961224492321\\
68.375	0.20646	7.51676066487289	7.51676066487289\\
68.375	0.21012	-1.67314366318169	-1.67314366318169\\
68.375	0.21378	-11.6084884918425	-11.6084884918425\\
68.375	0.21744	-22.2892738211106	-22.2892738211106\\
68.375	0.2211	-33.7154996509851	-33.7154996509851\\
68.375	0.22476	-45.8871659814661	-45.8871659814661\\
68.375	0.22842	-58.804272812554	-58.804272812554\\
68.375	0.23208	-72.4668201442487	-72.4668201442487\\
68.375	0.23574	-86.8748079765501	-86.8748079765501\\
68.375	0.2394	-102.028236309458	-102.028236309458\\
68.375	0.24306	-117.927105142973	-117.927105142973\\
68.375	0.24672	-134.571414477094	-134.571414477094\\
68.375	0.25038	-151.961164311822	-151.961164311822\\
68.375	0.25404	-170.096354647156	-170.096354647156\\
68.375	0.2577	-188.976985483097	-188.976985483097\\
68.375	0.26136	-208.603056819646	-208.603056819646\\
68.375	0.26502	-228.9745686568	-228.9745686568\\
68.375	0.26868	-250.09152099456	-250.09152099456\\
68.375	0.27234	-271.953913832929	-271.953913832929\\
68.375	0.276	-294.561747171903	-294.561747171903\\
68.75	0.093	-82.3951199587291	-82.3951199587291\\
68.75	0.09666	-68.2273038943266	-68.2273038943266\\
68.75	0.10032	-54.8049283305303	-54.8049283305303\\
68.75	0.10398	-42.1279932673407	-42.1279932673407\\
68.75	0.10764	-30.1964987047582	-30.1964987047582\\
68.75	0.1113	-19.0104446427815	-19.0104446427815\\
68.75	0.11496	-8.5698310814123	-8.5698310814123\\
68.75	0.11862	1.12534197935025	1.12534197935025\\
68.75	0.12228	10.0750745395066	10.0750745395066\\
68.75	0.12594	18.2793665990562	18.2793665990562\\
68.75	0.1296	25.7382181579989	25.7382181579989\\
68.75	0.13326	32.4516292163352	32.4516292163352\\
68.75	0.13692	38.4195997740649	38.4195997740649\\
68.75	0.14058	43.6421298311873	43.6421298311873\\
68.75	0.14424	48.119219387704	48.119219387704\\
68.75	0.1479	51.8508684436133	51.8508684436133\\
68.75	0.15156	54.8370769989159	54.8370769989159\\
68.75	0.15522	57.0778450536122	57.0778450536122\\
68.75	0.15888	58.5731726077017	58.5731726077017\\
68.75	0.16254	59.3230596611844	59.3230596611844\\
68.75	0.1662	59.3275062140608	59.3275062140608\\
68.75	0.16986	58.5865122663305	58.5865122663305\\
68.75	0.17352	57.1000778179936	57.1000778179936\\
68.75	0.17718	54.8682028690496	54.8682028690496\\
68.75	0.18084	51.8908874194993	51.8908874194993\\
68.75	0.1845	48.1681314693419	48.1681314693419\\
68.75	0.18816	43.6999350185786	43.6999350185786\\
68.75	0.19182	38.4862980672079	38.4862980672079\\
68.75	0.19548	32.5272206152306	32.5272206152306\\
68.75	0.19914	25.8227026626471	25.8227026626471\\
68.75	0.2028	18.3727442094569	18.3727442094569\\
68.75	0.20646	10.1773452556595	10.1773452556595\\
68.75	0.21012	1.23650580125559	1.23650580125559\\
68.75	0.21378	-8.44977415375456	-8.44977415375456\\
68.75	0.21744	-18.8814946093714	-18.8814946093714\\
68.75	0.2211	-30.0586555655952	-30.0586555655952\\
68.75	0.22476	-41.9812570224256	-41.9812570224256\\
68.75	0.22842	-54.6492989798624	-54.6492989798624\\
68.75	0.23208	-68.0627814379063	-68.0627814379063\\
68.75	0.23574	-82.221704396557	-82.221704396557\\
68.75	0.2394	-97.1260678558142	-97.1260678558142\\
68.75	0.24306	-112.775871815677	-112.775871815677\\
68.75	0.24672	-129.171116276148	-129.171116276148\\
68.75	0.25038	-146.311801237224	-146.311801237224\\
68.75	0.25404	-164.197926698908	-164.197926698908\\
68.75	0.2577	-182.829492661199	-182.829492661199\\
68.75	0.26136	-202.206499124096	-202.206499124096\\
68.75	0.26502	-222.328946087599	-222.328946087599\\
68.75	0.26868	-243.19683355171	-243.19683355171\\
68.75	0.27234	-264.810161516427	-264.810161516427\\
68.75	0.276	-287.16892998175	-287.16892998175\\
69.125	0.093	-87.6060383620761	-87.6060383620761\\
69.125	0.09666	-73.189157424022	-73.189157424022\\
69.125	0.10032	-59.5177169865751	-59.5177169865751\\
69.125	0.10398	-46.5917170497347	-46.5917170497347\\
69.125	0.10764	-34.4111576135011	-34.4111576135011\\
69.125	0.1113	-22.9760386778736	-22.9760386778736\\
69.125	0.11496	-12.2863602428538	-12.2863602428538\\
69.125	0.11862	-2.34212230844008	-2.34212230844008\\
69.125	0.12228	6.85667512536696	6.85667512536696\\
69.125	0.12594	15.3100320585673	15.3100320585673\\
69.125	0.1296	23.0179484911611	23.0179484911611\\
69.125	0.13326	29.9804244231481	29.9804244231481\\
69.125	0.13692	36.1974598545285	36.1974598545285\\
69.125	0.14058	41.6690547853025	41.6690547853025\\
69.125	0.14424	46.3952092154694	46.3952092154694\\
69.125	0.1479	50.3759231450299	50.3759231450299\\
69.125	0.15156	53.6111965739832	53.6111965739832\\
69.125	0.15522	56.1010295023302	56.1010295023302\\
69.125	0.15888	57.8454219300709	57.8454219300709\\
69.125	0.16254	58.8443738572047	58.8443738572047\\
69.125	0.1662	59.0978852837313	59.0978852837313\\
69.125	0.16986	58.6059562096522	58.6059562096522\\
69.125	0.17352	57.3685866349659	57.3685866349659\\
69.125	0.17718	55.3857765596726	55.3857765596726\\
69.125	0.18084	52.6575259837736	52.6575259837736\\
69.125	0.1845	49.1838349072673	49.1838349072673\\
69.125	0.18816	44.9647033301542	44.9647033301542\\
69.125	0.19182	40.0001312524347	40.0001312524347\\
69.125	0.19548	34.290118674108	34.290118674108\\
69.125	0.19914	27.8346655951757	27.8346655951757\\
69.125	0.2028	20.6337720156362	20.6337720156362\\
69.125	0.20646	12.6874379354895	12.6874379354895\\
69.125	0.21012	3.99566335473673	3.99566335473673\\
69.125	0.21378	-5.44155172662272	-5.44155172662272\\
69.125	0.21744	-15.6242073085889	-15.6242073085889\\
69.125	0.2211	-26.5523033911616	-26.5523033911616\\
69.125	0.22476	-38.2258399743412	-38.2258399743412\\
69.125	0.22842	-50.6448170581273	-50.6448170581273\\
69.125	0.23208	-63.8092346425201	-63.8092346425201\\
69.125	0.23574	-77.7190927275201	-77.7190927275201\\
69.125	0.2394	-92.3743913131261	-92.3743913131261\\
69.125	0.24306	-107.775130399338	-107.775130399338\\
69.125	0.24672	-123.921309986158	-123.921309986158\\
69.125	0.25038	-140.812930073585	-140.812930073585\\
69.125	0.25404	-158.449990661617	-158.449990661617\\
69.125	0.2577	-176.832491750257	-176.832491750257\\
69.125	0.26136	-195.960433339503	-195.960433339503\\
69.125	0.26502	-215.833815429356	-215.833815429356\\
69.125	0.26868	-236.452638019815	-236.452638019815\\
69.125	0.27234	-257.816901110881	-257.816901110881\\
69.125	0.276	-279.926604702554	-279.926604702554\\
69.5	0.093	-92.9674486763779	-92.9674486763779\\
69.5	0.09666	-78.3015028646735	-78.3015028646735\\
69.5	0.10032	-64.3809975535755	-64.3809975535755\\
69.5	0.10398	-51.205932743084	-51.205932743084\\
69.5	0.10764	-38.7763084332001	-38.7763084332001\\
69.5	0.1113	-27.092124623922	-27.092124623922\\
69.5	0.11496	-16.1533813152505	-16.1533813152505\\
69.5	0.11862	-5.9600785071861	-5.9600785071861\\
69.5	0.12228	3.48778380027164	3.48778380027164\\
69.5	0.12594	12.1902056071231	12.1902056071231\\
69.5	0.1296	20.147186913368	20.147186913368\\
69.5	0.13326	27.3587277190053	27.3587277190053\\
69.5	0.13692	33.8248280240368	33.8248280240368\\
69.5	0.14058	39.5454878284615	39.5454878284615\\
69.5	0.14424	44.5207071322797	44.5207071322797\\
69.5	0.1479	48.7504859354904	48.7504859354904\\
69.5	0.15156	52.2348242380953	52.2348242380953\\
69.5	0.15522	54.973722040093	54.973722040093\\
69.5	0.15888	56.9671793414848	56.9671793414848\\
69.5	0.16254	58.2151961422688	58.2151961422688\\
69.5	0.1662	58.7177724424466	58.7177724424466\\
69.5	0.16986	58.4749082420177	58.4749082420177\\
69.5	0.17352	57.4866035409826	57.4866035409826\\
69.5	0.17718	55.7528583393409	55.7528583393409\\
69.5	0.18084	53.2736726370925	53.2736726370925\\
69.5	0.1845	50.049046434237	50.049046434237\\
69.5	0.18816	46.0789797307746	46.0789797307746\\
69.5	0.19182	41.3634725267062	41.3634725267062\\
69.5	0.19548	35.9025248220307	35.9025248220307\\
69.5	0.19914	29.6961366167491	29.6961366167491\\
69.5	0.2028	22.7443079108598	22.7443079108598\\
69.5	0.20646	15.0470387043642	15.0470387043642\\
69.5	0.21012	6.60432899726266	6.60432899726266\\
69.5	0.21378	-2.5838212104461	-2.5838212104461\\
69.5	0.21744	-12.5174119187611	-12.5174119187611\\
69.5	0.2211	-23.1964431276831	-23.1964431276831\\
69.5	0.22476	-34.620914837212	-34.620914837212\\
69.5	0.22842	-46.790827047347	-46.790827047347\\
69.5	0.23208	-59.7061797580891	-59.7061797580891\\
69.5	0.23574	-73.3669729694379	-73.3669729694379\\
69.5	0.2394	-87.7732066813933	-87.7732066813933\\
69.5	0.24306	-102.924880893955	-102.924880893955\\
69.5	0.24672	-118.821995607124	-118.821995607124\\
69.5	0.25038	-135.464550820899	-135.464550820899\\
69.5	0.25404	-152.852546535281	-152.852546535281\\
69.5	0.2577	-170.985982750269	-170.985982750269\\
69.5	0.26136	-189.864859465864	-189.864859465864\\
69.5	0.26502	-209.489176682067	-209.489176682067\\
69.5	0.26868	-229.858934398875	-229.858934398875\\
69.5	0.27234	-250.974132616291	-250.974132616291\\
69.5	0.276	-272.834771334312	-272.834771334312\\
69.875	0.093	-98.4793509016363	-98.4793509016363\\
69.875	0.09666	-83.5643402162808	-83.5643402162808\\
69.875	0.10032	-69.394770031532	-69.394770031532\\
69.875	0.10398	-55.9706403473903	-55.9706403473903\\
69.875	0.10764	-43.2919511638548	-43.2919511638548\\
69.875	0.1113	-31.358702480926	-31.358702480926\\
69.875	0.11496	-20.1708942986038	-20.1708942986038\\
69.875	0.11862	-9.72852661688825	-9.72852661688825\\
69.875	0.12228	-0.0315994357798104	-0.0315994357798104\\
69.875	0.12594	8.91988724472236	8.91988724472236\\
69.875	0.1296	17.125933424618	17.125933424618\\
69.875	0.13326	24.5865391039064	24.5865391039064\\
69.875	0.13692	31.3017042825891	31.3017042825891\\
69.875	0.14058	37.271428960664	37.271428960664\\
69.875	0.14424	42.4957131381333	42.4957131381333\\
69.875	0.1479	46.9745568149951	46.9745568149951\\
69.875	0.15156	50.7079599912503	50.7079599912503\\
69.875	0.15522	53.6959226668992	53.6959226668992\\
69.875	0.15888	55.9384448419416	55.9384448419416\\
69.875	0.16254	57.4355265163769	57.4355265163769\\
69.875	0.1662	58.1871676902058	58.1871676902058\\
69.875	0.16986	58.1933683634276	58.1933683634276\\
69.875	0.17352	57.4541285360436	57.4541285360436\\
69.875	0.17718	55.9694482080522	55.9694482080522\\
69.875	0.18084	53.739327379454	53.739327379454\\
69.875	0.1845	50.7637660502501	50.7637660502501\\
69.875	0.18816	47.0427642204388	47.0427642204388\\
69.875	0.19182	42.5763218900207	42.5763218900207\\
69.875	0.19548	37.3644390589964	37.3644390589964\\
69.875	0.19914	31.4071157273654	31.4071157273654\\
69.875	0.2028	24.7043518951273	24.7043518951273\\
69.875	0.20646	17.2561475622829	17.2561475622829\\
69.875	0.21012	9.06250272883153	9.06250272883153\\
69.875	0.21378	0.123417394773924	0.123417394773924\\
69.875	0.21744	-9.5611084398904	-9.5611084398904\\
69.875	0.2211	-19.9910747751612	-19.9910747751612\\
69.875	0.22476	-31.1664816110394	-31.1664816110394\\
69.875	0.22842	-43.0873289475237	-43.0873289475237\\
69.875	0.23208	-55.7536167846147	-55.7536167846147\\
69.875	0.23574	-69.1653451223128	-69.1653451223128\\
69.875	0.2394	-83.3225139606179	-83.3225139606179\\
69.875	0.24306	-98.2251232995284	-98.2251232995284\\
69.875	0.24672	-113.873173139046	-113.873173139046\\
69.875	0.25038	-130.266663479171	-130.266663479171\\
69.875	0.25404	-147.405594319902	-147.405594319902\\
69.875	0.2577	-165.289965661239	-165.289965661239\\
69.875	0.26136	-183.919777503183	-183.919777503183\\
69.875	0.26502	-203.295029845734	-203.295029845734\\
69.875	0.26868	-223.415722688892	-223.415722688892\\
69.875	0.27234	-244.281856032657	-244.281856032657\\
69.875	0.276	-265.893429877028	-265.893429877028\\
70.25	0.093	-104.14174503785	-104.14174503785\\
70.25	0.09666	-88.9776694788437	-88.9776694788437\\
70.25	0.10032	-74.5590344204442	-74.5590344204442\\
70.25	0.10398	-60.8858398626513	-60.8858398626513\\
70.25	0.10764	-47.9580858054651	-47.9580858054651\\
70.25	0.1113	-35.7757722488856	-35.7757722488856\\
70.25	0.11496	-24.3388991929128	-24.3388991929128\\
70.25	0.11862	-13.6474666375465	-13.6474666375465\\
70.25	0.12228	-3.70147458278694	-3.70147458278694\\
70.25	0.12594	5.49907697136592	5.49907697136592\\
70.25	0.1296	13.9541880249122	13.9541880249122\\
70.25	0.13326	21.6638585778518	21.6638585778518\\
70.25	0.13692	28.6280886301852	28.6280886301852\\
70.25	0.14058	34.8468781819117	34.8468781819117\\
70.25	0.14424	40.3202272330312	40.3202272330312\\
70.25	0.1479	45.0481357835442	45.0481357835442\\
70.25	0.15156	49.0306038334501	49.0306038334501\\
70.25	0.15522	52.2676313827501	52.2676313827501\\
70.25	0.15888	54.7592184314428	54.7592184314428\\
70.25	0.16254	56.5053649795292	56.5053649795292\\
70.25	0.1662	57.5060710270088	57.5060710270088\\
70.25	0.16986	57.7613365738817	57.7613365738817\\
70.25	0.17352	57.2711616201485	57.2711616201485\\
70.25	0.17718	56.0355461658078	56.0355461658078\\
70.25	0.18084	54.0544902108608	54.0544902108608\\
70.25	0.1845	51.3279937553075	51.3279937553075\\
70.25	0.18816	47.8560567991474	47.8560567991474\\
70.25	0.19182	43.63867934238	43.63867934238\\
70.25	0.19548	38.6758613850063	38.6758613850063\\
70.25	0.19914	32.9676029270265	32.9676029270265\\
70.25	0.2028	26.5139039684395	26.5139039684395\\
70.25	0.20646	19.3147645092454	19.3147645092454\\
70.25	0.21012	11.3701845494452	11.3701845494452\\
70.25	0.21378	2.68016408903827	2.68016408903827\\
70.25	0.21744	-6.75529687197491	-6.75529687197491\\
70.25	0.2211	-16.9361983335955	-16.9361983335955\\
70.25	0.22476	-27.8625402958226	-27.8625402958226\\
70.25	0.22842	-39.5343227586557	-39.5343227586557\\
70.25	0.23208	-51.9515457220959	-51.9515457220959\\
70.25	0.23574	-65.1142091861429	-65.1142091861429\\
70.25	0.2394	-79.0223131507973	-79.0223131507973\\
70.25	0.24306	-93.6758576160566	-93.6758576160566\\
70.25	0.24672	-109.074842581924	-109.074842581924\\
70.25	0.25038	-125.219268048397	-125.219268048397\\
70.25	0.25404	-142.109134015478	-142.109134015478\\
70.25	0.2577	-159.744440483165	-159.744440483165\\
70.25	0.26136	-178.125187451457	-178.125187451457\\
70.25	0.26502	-197.251374920358	-197.251374920358\\
70.25	0.26868	-217.123002889865	-217.123002889865\\
70.25	0.27234	-237.740071359979	-237.740071359979\\
70.25	0.276	-259.102580330698	-259.102580330698\\
70.625	0.093	-109.954631085021	-109.954631085021\\
70.625	0.09666	-94.5414906523632	-94.5414906523632\\
70.625	0.10032	-79.8737907203126	-79.8737907203126\\
70.625	0.10398	-65.951531288869	-65.951531288869\\
70.625	0.10764	-52.7747123580321	-52.7747123580321\\
70.625	0.1113	-40.3433339278014	-40.3433339278014\\
70.625	0.11496	-28.6573959981774	-28.6573959981774\\
70.625	0.11862	-17.7168985691605	-17.7168985691605\\
70.625	0.12228	-7.52184164075021	-7.52184164075021\\
70.625	0.12594	1.92777478705381	1.92777478705381\\
70.625	0.1296	10.6319507142513	10.6319507142513\\
70.625	0.13326	18.5906861408411	18.5906861408411\\
70.625	0.13692	25.8039810668251	25.8039810668251\\
70.625	0.14058	32.2718354922024	32.2718354922024\\
70.625	0.14424	37.994249416973	37.994249416973\\
70.625	0.1479	42.9712228411368	42.9712228411368\\
70.625	0.15156	47.2027557646937	47.2027557646937\\
70.625	0.15522	50.6888481876445	50.6888481876445\\
70.625	0.15888	53.4295001099883	53.4295001099883\\
70.625	0.16254	55.4247115317254	55.4247115317254\\
70.625	0.1662	56.6744824528557	56.6744824528557\\
70.625	0.16986	57.1788128733798	57.1788128733798\\
70.625	0.17352	56.9377027932973	56.9377027932973\\
70.625	0.17718	55.9511522126077	55.9511522126077\\
70.625	0.18084	54.2191611313114	54.2191611313114\\
70.625	0.1845	51.7417295494088	51.7417295494088\\
70.625	0.18816	48.5188574668994	48.5188574668994\\
70.625	0.19182	44.5505448837831	44.5505448837831\\
70.625	0.19548	39.8367918000602	39.8367918000602\\
70.625	0.19914	34.3775982157315	34.3775982157315\\
70.625	0.2028	28.1729641307952	28.1729641307952\\
70.625	0.20646	21.2228895452522	21.2228895452522\\
70.625	0.21012	13.5273744591027	13.5273744591027\\
70.625	0.21378	5.08641887234648	5.08641887234648\\
70.625	0.21744	-4.099977215016	-4.099977215016\\
70.625	0.2211	-14.0318138029854	-14.0318138029854\\
70.625	0.22476	-24.7090908915618	-24.7090908915618\\
70.625	0.22842	-36.1318084807438	-36.1318084807438\\
70.625	0.23208	-48.2999665705333	-48.2999665705333\\
70.625	0.23574	-61.2135651609296	-61.2135651609296\\
70.625	0.2394	-74.8726042519329	-74.8726042519329\\
70.625	0.24306	-89.277083843541	-89.277083843541\\
70.625	0.24672	-104.427003935758	-104.427003935758\\
70.625	0.25038	-120.32236452858	-120.32236452858\\
70.625	0.25404	-136.963165622009	-136.963165622009\\
70.625	0.2577	-154.349407216046	-154.349407216046\\
70.625	0.26136	-172.481089310688	-172.481089310688\\
70.625	0.26502	-191.358211905938	-191.358211905938\\
70.625	0.26868	-210.980775001793	-210.980775001793\\
70.625	0.27234	-231.348778598257	-231.348778598257\\
70.625	0.276	-252.462222695326	-252.462222695326\\
71	0.093	-115.918009043146	-115.918009043146\\
71	0.09666	-100.255803736838	-100.255803736838\\
71	0.10032	-85.3390389311366	-85.3390389311366\\
71	0.10398	-71.1677146260419	-71.1677146260419\\
71	0.10764	-57.7418308215538	-57.7418308215538\\
71	0.1113	-45.0613875176725	-45.0613875176725\\
71	0.11496	-33.1263847143978	-33.1263847143978\\
71	0.11862	-21.9368224117301	-21.9368224117301\\
71	0.12228	-11.4927006096687	-11.4927006096687\\
71	0.12594	-1.79401930821399	-1.79401930821399\\
71	0.1296	7.15922149263417	7.15922149263417\\
71	0.13326	15.3670217928756	15.3670217928756\\
71	0.13692	22.8293815925099	22.8293815925099\\
71	0.14058	29.5463008915383	29.5463008915383\\
71	0.14424	35.5177796899596	35.5177796899596\\
71	0.1479	40.7438179877745	40.7438179877745\\
71	0.15156	45.2244157849822	45.2244157849822\\
71	0.15522	48.9595730815836	48.9595730815836\\
71	0.15888	51.9492898775786	51.9492898775786\\
71	0.16254	54.1935661729664	54.1935661729664\\
71	0.1662	55.6924019677479	55.6924019677479\\
71	0.16986	56.4457972619222	56.4457972619222\\
71	0.17352	56.4537520554908	56.4537520554908\\
71	0.17718	55.7162663484519	55.7162663484519\\
71	0.18084	54.2333401408072	54.2333401408072\\
71	0.1845	52.0049734325548	52.0049734325548\\
71	0.18816	49.0311662236966	49.0311662236966\\
71	0.19182	45.311918514231	45.311918514231\\
71	0.19548	40.8472303041592	40.8472303041592\\
71	0.19914	35.6371015934808	35.6371015934808\\
71	0.2028	29.6815323821957	29.6815323821957\\
71	0.20646	22.9805226703033	22.9805226703033\\
71	0.21012	15.534072457805	15.534072457805\\
71	0.21378	7.34218174469993	7.34218174469993\\
71	0.21744	-1.59514946901186	-1.59514946901186\\
71	0.2211	-11.2779211833306	-11.2779211833306\\
71	0.22476	-21.7061333982558	-21.7061333982558\\
71	0.22842	-32.8797861137871	-32.8797861137871\\
71	0.23208	-44.798879329926	-44.798879329926\\
71	0.23574	-57.463413046672	-57.463413046672\\
71	0.2394	-70.8733872640232	-70.8733872640232\\
71	0.24306	-85.0288019819811	-85.0288019819811\\
71	0.24672	-99.9296572005467	-99.9296572005467\\
71	0.25038	-115.575952919719	-115.575952919719\\
71	0.25404	-131.967689139497	-131.967689139497\\
71	0.2577	-149.104865859883	-149.104865859883\\
71	0.26136	-166.987483080874	-166.987483080874\\
71	0.26502	-185.615540802473	-185.615540802473\\
71	0.26868	-204.989039024677	-204.989039024677\\
71	0.27234	-225.107977747489	-225.107977747489\\
71	0.276	-245.972356970907	-245.972356970907\\
71.375	0.093	-122.031878912228	-122.031878912228\\
71.375	0.09666	-106.120608732269	-106.120608732269\\
71.375	0.10032	-90.9547790529168	-90.9547790529168\\
71.375	0.10398	-76.5343898741713	-76.5343898741713\\
71.375	0.10764	-62.8594411960326	-62.8594411960326\\
71.375	0.1113	-49.9299330185001	-49.9299330185001\\
71.375	0.11496	-37.7458653415752	-37.7458653415752\\
71.375	0.11862	-26.3072381652559	-26.3072381652559\\
71.375	0.12228	-15.6140514895438	-15.6140514895438\\
71.375	0.12594	-5.66630531443792	-5.66630531443792\\
71.375	0.1296	3.53600036006094	3.53600036006094\\
71.375	0.13326	11.9928655339531	11.9928655339531\\
71.375	0.13692	19.7042902072385	19.7042902072385\\
71.375	0.14058	26.6702743799171	26.6702743799171\\
71.375	0.14424	32.8908180519896	32.8908180519896\\
71.375	0.1479	38.3659212234552	38.3659212234552\\
71.375	0.15156	43.095583894314	43.095583894314\\
71.375	0.15522	47.0798060645661	47.0798060645661\\
71.375	0.15888	50.3185877342119	50.3185877342119\\
71.375	0.16254	52.8119289032508	52.8119289032508\\
71.375	0.1662	54.5598295716829	54.5598295716829\\
71.375	0.16986	55.5622897395084	55.5622897395084\\
71.375	0.17352	55.8193094067277	55.8193094067277\\
71.375	0.17718	55.33088857334	55.33088857334\\
71.375	0.18084	54.0970272393459	54.0970272393459\\
71.375	0.1845	52.1177254047448	52.1177254047448\\
71.375	0.18816	49.3929830695367	49.3929830695367\\
71.375	0.19182	45.9228002337223	45.9228002337223\\
71.375	0.19548	41.7071768973012	41.7071768973012\\
71.375	0.19914	36.746113060274	36.746113060274\\
71.375	0.2028	31.0396087226395	31.0396087226395\\
71.375	0.20646	24.5876638843984	24.5876638843984\\
71.375	0.21012	17.3902785455502	17.3902785455502\\
71.375	0.21378	9.44745270609633	9.44745270609633\\
71.375	0.21744	0.759186366035237	0.759186366035237\\
71.375	0.2211	-8.67452047463235	-8.67452047463235\\
71.375	0.22476	-18.8536678159069	-18.8536678159069\\
71.375	0.22842	-29.7782556577879	-29.7782556577879\\
71.375	0.23208	-41.4482840002756	-41.4482840002756\\
71.375	0.23574	-53.8637528433705	-53.8637528433705\\
71.375	0.2394	-67.0246621870715	-67.0246621870715\\
71.375	0.24306	-80.9310120313783	-80.9310120313783\\
71.375	0.24672	-95.5828023762926	-95.5828023762926\\
71.375	0.25038	-110.980033221813	-110.980033221813\\
71.375	0.25404	-127.122704567941	-127.122704567941\\
71.375	0.2577	-144.010816414675	-144.010816414675\\
71.375	0.26136	-161.644368762016	-161.644368762016\\
71.375	0.26502	-180.023361609964	-180.023361609964\\
71.375	0.26868	-199.147794958518	-199.147794958518\\
71.375	0.27234	-219.017668807679	-219.017668807679\\
71.375	0.276	-239.632983157447	-239.632983157447\\
71.75	0.093	-128.296240692266	-128.296240692266\\
71.75	0.09666	-112.135905638656	-112.135905638656\\
71.75	0.10032	-96.7210110856526	-96.7210110856526\\
71.75	0.10398	-82.0515570332565	-82.0515570332565\\
71.75	0.10764	-68.1275434814671	-68.1275434814671\\
71.75	0.1113	-54.9489704302839	-54.9489704302839\\
71.75	0.11496	-42.5158378797073	-42.5158378797073\\
71.75	0.11862	-30.8281458297374	-30.8281458297374\\
71.75	0.12228	-19.8858942803745	-19.8858942803745\\
71.75	0.12594	-9.68908323161844	-9.68908323161844\\
71.75	0.1296	-0.23771268346843	-0.23771268346843\\
71.75	0.13326	8.46821736407486	8.46821736407486\\
71.75	0.13692	16.4287069110114	16.4287069110114\\
71.75	0.14058	23.6437559573412	23.6437559573412\\
71.75	0.14424	30.113364503064	30.113364503064\\
71.75	0.1479	35.8375325481802	35.8375325481802\\
71.75	0.15156	40.8162600926902	40.8162600926902\\
71.75	0.15522	45.0495471365934	45.0495471365934\\
71.75	0.15888	48.5373936798899	48.5373936798899\\
71.75	0.16254	51.2797997225795	51.2797997225795\\
71.75	0.1662	53.2767652646628	53.2767652646628\\
71.75	0.16986	54.528290306139	54.528290306139\\
71.75	0.17352	55.034374847009	55.034374847009\\
71.75	0.17718	54.7950188872719	54.7950188872719\\
71.75	0.18084	53.810222426929	53.810222426929\\
71.75	0.1845	52.0799854659786	52.0799854659786\\
71.75	0.18816	49.6043080044217	49.6043080044217\\
71.75	0.19182	46.383190042258	46.383190042258\\
71.75	0.19548	42.4166315794876	42.4166315794876\\
71.75	0.19914	37.7046326161114	37.7046326161114\\
71.75	0.2028	32.2471931521277	32.2471931521277\\
71.75	0.20646	26.0443131875372	26.0443131875372\\
71.75	0.21012	19.0959927223403	19.0959927223403\\
71.75	0.21378	11.402231756537	11.402231756537\\
71.75	0.21744	2.96303029012711	2.96303029012711\\
71.75	0.2211	-6.22161167688932	-6.22161167688932\\
71.75	0.22476	-16.1516941445136	-16.1516941445136\\
71.75	0.22842	-26.8272171127426	-26.8272171127426\\
71.75	0.23208	-38.2481805815801	-38.2481805815801\\
71.75	0.23574	-50.4145845510238	-50.4145845510238\\
71.75	0.2394	-63.3264290210745	-63.3264290210745\\
71.75	0.24306	-76.9837139917311	-76.9837139917311\\
71.75	0.24672	-91.3864394629943	-91.3864394629943\\
71.75	0.25038	-106.534605434864	-106.534605434864\\
71.75	0.25404	-122.428211907341	-122.428211907341\\
71.75	0.2577	-139.067258880424	-139.067258880424\\
71.75	0.26136	-156.451746354114	-156.451746354114\\
71.75	0.26502	-174.581674328411	-174.581674328411\\
71.75	0.26868	-193.457042803315	-193.457042803315\\
71.75	0.27234	-213.077851778824	-213.077851778824\\
71.75	0.276	-233.444101254942	-233.444101254942\\
72.125	0.093	-134.711094383259	-134.711094383259\\
72.125	0.09666	-118.301694455998	-118.301694455998\\
72.125	0.10032	-102.637735029345	-102.637735029345\\
72.125	0.10398	-87.7192161032968	-87.7192161032968\\
72.125	0.10764	-73.5461376778567	-73.5461376778567\\
72.125	0.1113	-60.1184997530224	-60.1184997530224\\
72.125	0.11496	-47.4363023287956	-47.4363023287956\\
72.125	0.11862	-35.499545405175	-35.499545405175\\
72.125	0.12228	-24.308228982161	-24.308228982161\\
72.125	0.12594	-13.8623530597537	-13.8623530597537\\
72.125	0.1296	-4.16191763795302	-4.16191763795302\\
72.125	0.13326	4.79307728324051	4.79307728324051\\
72.125	0.13692	13.0026317038282	13.0026317038282\\
72.125	0.14058	20.4667456238087	20.4667456238087\\
72.125	0.14424	27.1854190431826	27.1854190431826\\
72.125	0.1479	33.15865196195	33.15865196195\\
72.125	0.15156	38.3864443801107	38.3864443801107\\
72.125	0.15522	42.8687962976642	42.8687962976642\\
72.125	0.15888	46.6057077146122	46.6057077146122\\
72.125	0.16254	49.5971786309525	49.5971786309525\\
72.125	0.1662	51.8432090466865	51.8432090466865\\
72.125	0.16986	53.3437989618143	53.3437989618143\\
72.125	0.17352	54.0989483763345	54.0989483763345\\
72.125	0.17718	54.1086572902486	54.1086572902486\\
72.125	0.18084	53.372925703556	53.372925703556\\
72.125	0.1845	51.8917536162567	51.8917536162567\\
72.125	0.18816	49.6651410283505	49.6651410283505\\
72.125	0.19182	46.6930879398379	46.6930879398379\\
72.125	0.19548	42.9755943507187	42.9755943507187\\
72.125	0.19914	38.5126602609928	38.5126602609928\\
72.125	0.2028	33.3042856706602	33.3042856706602\\
72.125	0.20646	27.3504705797204	27.3504705797204\\
72.125	0.21012	20.6512149881746	20.6512149881746\\
72.125	0.21378	13.2065188960225	13.2065188960225\\
72.125	0.21744	5.01638230326284	5.01638230326284\\
72.125	0.2211	-3.91919479010244	-3.91919479010244\\
72.125	0.22476	-13.6002123840756	-13.6002123840756\\
72.125	0.22842	-24.0266704786552	-24.0266704786552\\
72.125	0.23208	-35.1985690738406	-35.1985690738406\\
72.125	0.23574	-47.1159081696333	-47.1159081696333\\
72.125	0.2394	-59.7786877660328	-59.7786877660328\\
72.125	0.24306	-73.1869078630391	-73.1869078630391\\
72.125	0.24672	-87.3405684606512	-87.3405684606512\\
72.125	0.25038	-102.23966955887	-102.23966955887\\
72.125	0.25404	-117.884211157696	-117.884211157696\\
72.125	0.2577	-134.274193257129	-134.274193257129\\
72.125	0.26136	-151.409615857168	-151.409615857168\\
72.125	0.26502	-169.290478957814	-169.290478957814\\
72.125	0.26868	-187.916782559066	-187.916782559066\\
72.125	0.27234	-207.288526660926	-207.288526660926\\
72.125	0.276	-227.405711263391	-227.405711263391\\
72.5	0.093	-141.276439985209	-141.276439985209\\
72.5	0.09666	-124.617975184297	-124.617975184297\\
72.5	0.10032	-108.704950883992	-108.704950883992\\
72.5	0.10398	-93.5373670842943	-93.5373670842943\\
72.5	0.10764	-79.115223785203	-79.115223785203\\
72.5	0.1113	-65.4385209867175	-65.4385209867175\\
72.5	0.11496	-52.5072586888396	-52.5072586888396\\
72.5	0.11862	-40.3214368915682	-40.3214368915682\\
72.5	0.12228	-28.8810555949036	-28.8810555949036\\
72.5	0.12594	-18.1861147988456	-18.1861147988456\\
72.5	0.1296	-8.23661450339421	-8.23661450339421\\
72.5	0.13326	0.967445291450474	0.967445291450474\\
72.5	0.13692	9.42606458568889	9.42606458568889\\
72.5	0.14058	17.1392433793205	17.1392433793205\\
72.5	0.14424	24.1069816723451	24.1069816723451\\
72.5	0.1479	30.3292794647632	30.3292794647632\\
72.5	0.15156	35.8061367565746	35.8061367565746\\
72.5	0.15522	40.5375535477797	40.5375535477797\\
72.5	0.15888	44.523529838378	44.523529838378\\
72.5	0.16254	47.7640656283694	47.7640656283694\\
72.5	0.1662	50.2591609177541	50.2591609177541\\
72.5	0.16986	52.0088157065321	52.0088157065321\\
72.5	0.17352	53.013029994704	53.013029994704\\
72.5	0.17718	53.2718037822688	53.2718037822688\\
72.5	0.18084	52.7851370692273	52.7851370692273\\
72.5	0.1845	51.5530298555786	51.5530298555786\\
72.5	0.18816	49.5754821413236	49.5754821413236\\
72.5	0.19182	46.8524939264618	46.8524939264618\\
72.5	0.19548	43.3840652109932	43.3840652109932\\
72.5	0.19914	39.1701959949185	39.1701959949185\\
72.5	0.2028	34.2108862782366	34.2108862782366\\
72.5	0.20646	28.5061360609479	28.5061360609479\\
72.5	0.21012	22.0559453430528	22.0559453430528\\
72.5	0.21378	14.8603141245515	14.8603141245515\\
72.5	0.21744	6.9192424054429	6.9192424054429\\
72.5	0.2211	-1.76726981427214	-1.76726981427214\\
72.5	0.22476	-11.1992225345941	-11.1992225345941\\
72.5	0.22842	-21.3766157555217	-21.3766157555217\\
72.5	0.23208	-32.2994494770578	-32.2994494770578\\
72.5	0.23574	-43.9677236991993	-43.9677236991993\\
72.5	0.2394	-56.3814384219477	-56.3814384219477\\
72.5	0.24306	-69.5405936453028	-69.5405936453028\\
72.5	0.24672	-83.4451893692637	-83.4451893692637\\
72.5	0.25038	-98.0952255938334	-98.0952255938334\\
72.5	0.25404	-113.490702319008	-113.490702319008\\
72.5	0.2577	-129.63161954479	-129.63161954479\\
72.5	0.26136	-146.517977271178	-146.517977271178\\
72.5	0.26502	-164.149775498173	-164.149775498173\\
72.5	0.26868	-182.527014225775	-182.527014225775\\
72.5	0.27234	-201.649693453982	-201.649693453982\\
72.5	0.276	-221.517813182798	-221.517813182798\\
72.875	0.093	-147.992277498114	-147.992277498114\\
72.875	0.09666	-131.084747823552	-131.084747823552\\
72.875	0.10032	-114.922658649596	-114.922658649596\\
72.875	0.10398	-99.5060099762469	-99.5060099762469\\
72.875	0.10764	-84.8348018035049	-84.8348018035049\\
72.875	0.1113	-70.9090341313692	-70.9090341313692\\
72.875	0.11496	-57.7287069598406	-57.7287069598406\\
72.875	0.11862	-45.2938202889181	-45.2938202889181\\
72.875	0.12228	-33.6043741186023	-33.6043741186023\\
72.875	0.12594	-22.6603684488932	-22.6603684488932\\
72.875	0.1296	-12.4618032797906	-12.4618032797906\\
72.875	0.13326	-3.00867861129524	-3.00867861129524\\
72.875	0.13692	5.69900555659342	5.69900555659342\\
72.875	0.14058	13.6612492238762	13.6612492238762\\
72.875	0.14424	20.8780523905515	20.8780523905515\\
72.875	0.1479	27.3494150566207	27.3494150566207\\
72.875	0.15156	33.0753372220832	33.0753372220832\\
72.875	0.15522	38.0558188869391	38.0558188869391\\
72.875	0.15888	42.290860051188	42.290860051188\\
72.875	0.16254	45.7804607148302	45.7804607148302\\
72.875	0.1662	48.524620877866	48.524620877866\\
72.875	0.16986	50.5233405402947	50.5233405402947\\
72.875	0.17352	51.7766197021173	51.7766197021173\\
72.875	0.17718	52.2844583633332	52.2844583633332\\
72.875	0.18084	52.0468565239424	52.0468565239424\\
72.875	0.1845	51.0638141839445	51.0638141839445\\
72.875	0.18816	49.3353313433406	49.3353313433406\\
72.875	0.19182	46.8614080021295	46.8614080021295\\
72.875	0.19548	43.6420441603116	43.6420441603116\\
72.875	0.19914	39.677239817888	39.677239817888\\
72.875	0.2028	34.9669949748568	34.9669949748568\\
72.875	0.20646	29.5113096312189	29.5113096312189\\
72.875	0.21012	23.3101837869749	23.3101837869749\\
72.875	0.21378	16.3636174421242	16.3636174421242\\
72.875	0.21744	8.67161059666682	8.67161059666682\\
72.875	0.2211	0.234163250602023	0.234163250602023\\
72.875	0.22476	-8.94872459606881	-8.94872459606881\\
72.875	0.22842	-18.8770529433461	-18.8770529433461\\
72.875	0.23208	-29.5508217912302	-29.5508217912302\\
72.875	0.23574	-40.9700311397214	-40.9700311397214\\
72.875	0.2394	-53.1346809888196	-53.1346809888196\\
72.875	0.24306	-66.0447713385227	-66.0447713385227\\
72.875	0.24672	-79.7003021888343	-79.7003021888343\\
72.875	0.25038	-94.101273539751	-94.101273539751\\
72.875	0.25404	-109.247685391276	-109.247685391276\\
72.875	0.2577	-125.139537743407	-125.139537743407\\
72.875	0.26136	-141.776830596144	-141.776830596144\\
72.875	0.26502	-159.159563949488	-159.159563949488\\
72.875	0.26868	-177.287737803438	-177.287737803438\\
72.875	0.27234	-196.161352157997	-196.161352157997\\
72.875	0.276	-215.78040701316	-215.78040701316\\
73.25	0.093	-154.858606921975	-154.858606921975\\
73.25	0.09666	-137.702012373762	-137.702012373762\\
73.25	0.10032	-121.290858326156	-121.290858326156\\
73.25	0.10398	-105.625144779156	-105.625144779156\\
73.25	0.10764	-90.7048717327626	-90.7048717327626\\
73.25	0.1113	-76.5300391869761	-76.5300391869761\\
73.25	0.11496	-63.1006471417963	-63.1006471417963\\
73.25	0.11862	-50.4166955972232	-50.4166955972232\\
73.25	0.12228	-38.4781845532567	-38.4781845532567\\
73.25	0.12594	-27.2851140098969	-27.2851140098969\\
73.25	0.1296	-16.8374839671436	-16.8374839671436\\
73.25	0.13326	-7.13529442499708	-7.13529442499708\\
73.25	0.13692	1.82145461654272	1.82145461654272\\
73.25	0.14058	10.0327631574762	10.0327631574762\\
73.25	0.14424	17.4986311978026	17.4986311978026\\
73.25	0.1479	24.2190587375226	24.2190587375226\\
73.25	0.15156	30.1940457766353	30.1940457766353\\
73.25	0.15522	35.4235923151423	35.4235923151423\\
73.25	0.15888	39.9076983530424	39.9076983530424\\
73.25	0.16254	43.6463638903352	43.6463638903352\\
73.25	0.1662	46.6395889270218	46.6395889270218\\
73.25	0.16986	48.8873734631017	48.8873734631017\\
73.25	0.17352	50.3897174985749	50.3897174985749\\
73.25	0.17718	51.1466210334415	51.1466210334415\\
73.25	0.18084	51.1580840677019	51.1580840677019\\
73.25	0.1845	50.4241066013551	50.4241066013551\\
73.25	0.18816	48.9446886344015	48.9446886344015\\
73.25	0.19182	46.7198301668415	46.7198301668415\\
73.25	0.19548	43.7495311986747	43.7495311986747\\
73.25	0.19914	40.0337917299014	40.0337917299014\\
73.25	0.2028	35.5726117605213	35.5726117605213\\
73.25	0.20646	30.3659912905346	30.3659912905346\\
73.25	0.21012	24.4139303199408	24.4139303199408\\
73.25	0.21378	17.7164288487413	17.7164288487413\\
73.25	0.21744	10.2734868769351	10.2734868769351\\
73.25	0.2211	2.08510440452142	2.08510440452142\\
73.25	0.22476	-6.84871856849918	-6.84871856849918\\
73.25	0.22842	-16.5279820421254	-16.5279820421254\\
73.25	0.23208	-26.9526860163583	-26.9526860163583\\
73.25	0.23574	-38.1228304911992	-38.1228304911992\\
73.25	0.2394	-50.0384154666463	-50.0384154666463\\
73.25	0.24306	-62.6994409426982	-62.6994409426982\\
73.25	0.24672	-76.1059069193586	-76.1059069193586\\
73.25	0.25038	-90.257813396626	-90.257813396626\\
73.25	0.25404	-105.155160374499	-105.155160374499\\
73.25	0.2577	-120.797947852979	-120.797947852979\\
73.25	0.26136	-137.186175832066	-137.186175832066\\
73.25	0.26502	-154.319844311759	-154.319844311759\\
73.25	0.26868	-172.198953292059	-172.198953292059\\
73.25	0.27234	-190.823502772965	-190.823502772965\\
73.25	0.276	-210.193492754478	-210.193492754478\\
73.625	0.093	-161.875428256792	-161.875428256792\\
73.625	0.09666	-144.469768834928	-144.469768834928\\
73.625	0.10032	-127.809549913671	-127.809549913671\\
73.625	0.10398	-111.89477149302	-111.89477149302\\
73.625	0.10764	-96.7254335729759	-96.7254335729759\\
73.625	0.1113	-82.3015361535383	-82.3015361535383\\
73.625	0.11496	-68.6230792347078	-68.6230792347078\\
73.625	0.11862	-55.6900628164839	-55.6900628164839\\
73.625	0.12228	-43.5024868988667	-43.5024868988667\\
73.625	0.12594	-32.0603514818558	-32.0603514818558\\
73.625	0.1296	-21.3636565654514	-21.3636565654514\\
73.625	0.13326	-11.4124021496542	-11.4124021496542\\
73.625	0.13692	-2.20658823446365	-2.20658823446365\\
73.625	0.14058	6.25378518012099	6.25378518012099\\
73.625	0.14424	13.9687180940981	13.9687180940981\\
73.625	0.1479	20.9382105074687	20.9382105074687\\
73.625	0.15156	27.1622624202327	27.1622624202327\\
73.625	0.15522	32.6408738323898	32.6408738323898\\
73.625	0.15888	37.3740447439411	37.3740447439411\\
73.625	0.16254	41.3617751548851	41.3617751548851\\
73.625	0.1662	44.6040650652224	44.6040650652224\\
73.625	0.16986	47.1009144749529	47.1009144749529\\
73.625	0.17352	48.8523233840777	48.8523233840777\\
73.625	0.17718	49.8582917925951	49.8582917925951\\
73.625	0.18084	50.1188197005057	50.1188197005057\\
73.625	0.1845	49.63390710781	49.63390710781\\
73.625	0.18816	48.4035540145076	48.4035540145076\\
73.625	0.19182	46.4277604205982	46.4277604205982\\
73.625	0.19548	43.7065263260822	43.7065263260822\\
73.625	0.19914	40.23985173096	40.23985173096\\
73.625	0.2028	36.0277366352307	36.0277366352307\\
73.625	0.20646	31.0701810388946	31.0701810388946\\
73.625	0.21012	25.367184941952	25.367184941952\\
73.625	0.21378	18.9187483444027	18.9187483444027\\
73.625	0.21744	11.7248712462476	11.7248712462476\\
73.625	0.2211	3.78555364748513	3.78555364748513\\
73.625	0.22476	-4.89920445188432	-4.89920445188432\\
73.625	0.22842	-14.3294030518603	-14.3294030518603\\
73.625	0.23208	-24.5050421524429	-24.5050421524429\\
73.625	0.23574	-35.4261217536318	-35.4261217536318\\
73.625	0.2394	-47.0926418554277	-47.0926418554277\\
73.625	0.24306	-59.5046024578294	-59.5046024578294\\
73.625	0.24672	-72.6620035608387	-72.6620035608387\\
73.625	0.25038	-86.5648451644549	-86.5648451644549\\
73.625	0.25404	-101.213127268677	-101.213127268677\\
73.625	0.2577	-116.606849873507	-116.606849873507\\
73.625	0.26136	-132.746012978943	-132.746012978943\\
73.625	0.26502	-149.630616584985	-149.630616584985\\
73.625	0.26868	-167.260660691634	-167.260660691634\\
73.625	0.27234	-185.636145298889	-185.636145298889\\
73.625	0.276	-204.757070406751	-204.757070406751\\
74	0.093	-169.042741502565	-169.042741502565\\
74	0.09666	-151.38801720705	-151.38801720705\\
74	0.10032	-134.478733412142	-134.478733412142\\
74	0.10398	-118.31489011784	-118.31489011784\\
74	0.10764	-102.896487324145	-102.896487324145\\
74	0.1113	-88.223525031057	-88.223525031057\\
74	0.11496	-74.2960032385754	-74.2960032385754\\
74	0.11862	-61.1139219467004	-61.1139219467004\\
74	0.12228	-48.677281155432	-48.677281155432\\
74	0.12594	-36.9860808647708	-36.9860808647708\\
74	0.1296	-26.0403210747162	-26.0403210747162\\
74	0.13326	-15.8400017852678	-15.8400017852678\\
74	0.13692	-6.38512299642616	-6.38512299642616\\
74	0.14058	2.32431529180872	2.32431529180872\\
74	0.14424	10.288313079437	10.288313079437\\
74	0.1479	17.5068703664588	17.5068703664588\\
74	0.15156	23.9799871528738	23.9799871528738\\
74	0.15522	29.7076634386822	29.7076634386822\\
74	0.15888	34.6898992238837	34.6898992238837\\
74	0.16254	38.9266945084788	38.9266945084788\\
74	0.1662	42.4180492924672	42.4180492924672\\
74	0.16986	45.1639635758485	45.1639635758485\\
74	0.17352	47.1644373586236	47.1644373586236\\
74	0.17718	48.4194706407916	48.4194706407916\\
74	0.18084	48.9290634223538	48.9290634223538\\
74	0.1845	48.6932157033084	48.6932157033084\\
74	0.18816	47.7119274836566	47.7119274836566\\
74	0.19182	45.9851987633984	45.9851987633984\\
74	0.19548	43.5130295425336	43.5130295425336\\
74	0.19914	40.2954198210621	40.2954198210621\\
74	0.2028	36.3323695989834	36.3323695989834\\
74	0.20646	31.6238788762985	31.6238788762985\\
74	0.21012	26.1699476530071	26.1699476530071\\
74	0.21378	19.9705759291089	19.9705759291089\\
74	0.21744	13.0257637046041	13.0257637046041\\
74	0.2211	5.33551097949271	5.33551097949271\\
74	0.22476	-3.10018224622559	-3.10018224622559\\
74	0.22842	-12.2813159725513	-12.2813159725513\\
74	0.23208	-22.2078901994828	-22.2078901994828\\
74	0.23574	-32.8799049270215	-32.8799049270215\\
74	0.2394	-44.2973601551653	-44.2973601551653\\
74	0.24306	-56.4602558839176	-56.4602558839176\\
74	0.24672	-69.3685921132758	-69.3685921132758\\
74	0.25038	-83.0223688432409	-83.0223688432409\\
74	0.25404	-97.4215860738123	-97.4215860738123\\
74	0.2577	-112.566243804991	-112.566243804991\\
74	0.26136	-128.456342036775	-128.456342036775\\
74	0.26502	-145.091880769167	-145.091880769167\\
74	0.26868	-162.472860002166	-162.472860002166\\
74	0.27234	-180.59927973577	-180.59927973577\\
74	0.276	-199.471139969982	-199.471139969982\\
};
\end{axis}

\begin{axis}[%
width=2.616cm,
height=2.517cm,
at={(0cm,6.993cm)},
scale only axis,
xmin=56,
xmax=74,
tick align=outside,
xlabel style={font=\color{white!15!black}},
xlabel={$L_{cut}$},
ymin=0.093,
ymax=0.276,
ylabel style={font=\color{white!15!black}},
ylabel={$D_{rlx}$},
zmin=-4.07676020842683,
zmax=14.7369266352855,
zlabel style={font=\color{white!15!black}},
zlabel={$x_1$},
view={-140}{50},
axis background/.style={fill=white},
xmajorgrids,
ymajorgrids,
zmajorgrids,
legend style={at={(1.03,1)}, anchor=north west, legend cell align=left, align=left, draw=white!15!black}
]
\addplot3[only marks, mark=*, mark options={}, mark size=1.5000pt, color=mycolor1, fill=mycolor1] table[row sep=crcr]{%
x	y	z\\
74	0.123	0.893705603091675\\
72	0.113	-0.12107671482601\\
61	0.095	0.301368196488441\\
56	0.093	0.615358936761545\\
};
\addlegendentry{data1}

\addplot3[only marks, mark=*, mark options={}, mark size=1.5000pt, color=mycolor2, fill=mycolor2] table[row sep=crcr]{%
x	y	z\\
67	0.276	8.48146407715713\\
66	0.255	0.675435465503392\\
62	0.209	0.49651711681375\\
57	0.193	-0.52441764196167\\
};
\addlegendentry{data2}

\addplot3[only marks, mark=*, mark options={}, mark size=1.5000pt, color=black, fill=black] table[row sep=crcr]{%
x	y	z\\
69	0.104	0.558829069437584\\
};
\addlegendentry{data3}

\addplot3[only marks, mark=*, mark options={}, mark size=1.5000pt, color=black, fill=black] table[row sep=crcr]{%
x	y	z\\
64	0.23	0.509863748878793\\
};
\addlegendentry{data4}


\addplot3[%
surf,
fill opacity=0.7, shader=interp, colormap={mymap}{[1pt] rgb(0pt)=(1,0.905882,0); rgb(1pt)=(1,0.901964,0); rgb(2pt)=(1,0.898051,0); rgb(3pt)=(1,0.894144,0); rgb(4pt)=(1,0.890243,0); rgb(5pt)=(1,0.886349,0); rgb(6pt)=(1,0.88246,0); rgb(7pt)=(1,0.878577,0); rgb(8pt)=(1,0.8747,0); rgb(9pt)=(1,0.870829,0); rgb(10pt)=(1,0.866964,0); rgb(11pt)=(1,0.863106,0); rgb(12pt)=(1,0.859253,0); rgb(13pt)=(1,0.855406,0); rgb(14pt)=(1,0.851566,0); rgb(15pt)=(1,0.847732,0); rgb(16pt)=(1,0.843903,0); rgb(17pt)=(1,0.840081,0); rgb(18pt)=(1,0.836265,0); rgb(19pt)=(1,0.832455,0); rgb(20pt)=(1,0.828652,0); rgb(21pt)=(1,0.824854,0); rgb(22pt)=(1,0.821063,0); rgb(23pt)=(1,0.817278,0); rgb(24pt)=(1,0.8135,0); rgb(25pt)=(1,0.809727,0); rgb(26pt)=(1,0.805961,0); rgb(27pt)=(1,0.8022,0); rgb(28pt)=(1,0.798445,0); rgb(29pt)=(1,0.794696,0); rgb(30pt)=(1,0.790953,0); rgb(31pt)=(1,0.787215,0); rgb(32pt)=(1,0.783484,0); rgb(33pt)=(1,0.779758,0); rgb(34pt)=(1,0.776038,0); rgb(35pt)=(1,0.772324,0); rgb(36pt)=(1,0.768615,0); rgb(37pt)=(1,0.764913,0); rgb(38pt)=(1,0.761217,0); rgb(39pt)=(1,0.757527,0); rgb(40pt)=(1,0.753843,0); rgb(41pt)=(1,0.750165,0); rgb(42pt)=(1,0.746493,0); rgb(43pt)=(1,0.742827,0); rgb(44pt)=(1,0.739167,0); rgb(45pt)=(1,0.735514,0); rgb(46pt)=(1,0.731867,0); rgb(47pt)=(1,0.728226,0); rgb(48pt)=(1,0.724591,0); rgb(49pt)=(1,0.720963,0); rgb(50pt)=(1,0.717341,0); rgb(51pt)=(1,0.713725,0); rgb(52pt)=(0.999994,0.710077,0); rgb(53pt)=(0.999974,0.706363,0); rgb(54pt)=(0.999942,0.702592,0); rgb(55pt)=(0.999898,0.698775,0); rgb(56pt)=(0.999841,0.694921,0); rgb(57pt)=(0.999771,0.691039,0); rgb(58pt)=(0.99969,0.687139,0); rgb(59pt)=(0.999596,0.68323,0); rgb(60pt)=(0.99949,0.679323,0); rgb(61pt)=(0.999372,0.675427,0); rgb(62pt)=(0.999242,0.67155,0); rgb(63pt)=(0.9991,0.667704,0); rgb(64pt)=(0.998946,0.663897,0); rgb(65pt)=(0.998781,0.660138,0); rgb(66pt)=(0.998605,0.656439,0); rgb(67pt)=(0.998416,0.652807,0); rgb(68pt)=(0.998217,0.649253,0); rgb(69pt)=(0.998006,0.645786,0); rgb(70pt)=(0.997785,0.642416,0); rgb(71pt)=(0.997552,0.639152,0); rgb(72pt)=(0.997308,0.636004,0); rgb(73pt)=(0.997053,0.632982,0); rgb(74pt)=(0.996788,0.630095,0); rgb(75pt)=(0.996512,0.627352,0); rgb(76pt)=(0.996226,0.624763,0); rgb(77pt)=(0.995851,0.622329,0); rgb(78pt)=(0.99494,0.619997,0); rgb(79pt)=(0.99345,0.617753,0); rgb(80pt)=(0.991419,0.61559,0); rgb(81pt)=(0.988885,0.613503,0); rgb(82pt)=(0.985886,0.611486,0); rgb(83pt)=(0.98246,0.609532,0); rgb(84pt)=(0.978643,0.607636,0); rgb(85pt)=(0.974475,0.605791,0); rgb(86pt)=(0.969992,0.603992,0); rgb(87pt)=(0.965232,0.602233,0); rgb(88pt)=(0.960233,0.600507,0); rgb(89pt)=(0.955033,0.598808,0); rgb(90pt)=(0.949669,0.59713,0); rgb(91pt)=(0.94418,0.595468,0); rgb(92pt)=(0.938602,0.593815,0); rgb(93pt)=(0.932974,0.592166,0); rgb(94pt)=(0.927333,0.590513,0); rgb(95pt)=(0.921717,0.588852,0); rgb(96pt)=(0.916164,0.587176,0); rgb(97pt)=(0.910711,0.585479,0); rgb(98pt)=(0.905397,0.583755,0); rgb(99pt)=(0.900258,0.581999,0); rgb(100pt)=(0.895333,0.580203,0); rgb(101pt)=(0.890659,0.578362,0); rgb(102pt)=(0.886275,0.576471,0); rgb(103pt)=(0.882047,0.574545,0); rgb(104pt)=(0.877819,0.572608,0); rgb(105pt)=(0.873592,0.57066,0); rgb(106pt)=(0.869366,0.568701,0); rgb(107pt)=(0.865143,0.566733,0); rgb(108pt)=(0.860924,0.564756,0); rgb(109pt)=(0.856708,0.562771,0); rgb(110pt)=(0.852497,0.560778,0); rgb(111pt)=(0.848292,0.558779,0); rgb(112pt)=(0.844092,0.556774,0); rgb(113pt)=(0.8399,0.554763,0); rgb(114pt)=(0.835716,0.552749,0); rgb(115pt)=(0.831541,0.55073,0); rgb(116pt)=(0.827374,0.548709,0); rgb(117pt)=(0.823219,0.546686,0); rgb(118pt)=(0.819074,0.54466,0); rgb(119pt)=(0.81494,0.542635,0); rgb(120pt)=(0.81082,0.540609,0); rgb(121pt)=(0.806712,0.538584,0); rgb(122pt)=(0.802619,0.53656,0); rgb(123pt)=(0.798541,0.534539,0); rgb(124pt)=(0.794478,0.532521,0); rgb(125pt)=(0.790431,0.530506,0); rgb(126pt)=(0.786402,0.528496,0); rgb(127pt)=(0.782391,0.526491,0); rgb(128pt)=(0.77841,0.524489,0); rgb(129pt)=(0.774523,0.522478,0); rgb(130pt)=(0.770731,0.520455,0); rgb(131pt)=(0.767022,0.518424,0); rgb(132pt)=(0.763384,0.516385,0); rgb(133pt)=(0.759804,0.514339,0); rgb(134pt)=(0.756272,0.51229,0); rgb(135pt)=(0.752775,0.510237,0); rgb(136pt)=(0.749302,0.508182,0); rgb(137pt)=(0.74584,0.506128,0); rgb(138pt)=(0.742378,0.504075,0); rgb(139pt)=(0.738904,0.502025,0); rgb(140pt)=(0.735406,0.499979,0); rgb(141pt)=(0.731872,0.49794,0); rgb(142pt)=(0.72829,0.495909,0); rgb(143pt)=(0.724649,0.493887,0); rgb(144pt)=(0.720936,0.491875,0); rgb(145pt)=(0.71714,0.489876,0); rgb(146pt)=(0.713249,0.487891,0); rgb(147pt)=(0.709251,0.485921,0); rgb(148pt)=(0.705134,0.483968,0); rgb(149pt)=(0.700887,0.482033,0); rgb(150pt)=(0.696497,0.480118,0); rgb(151pt)=(0.691952,0.478225,0); rgb(152pt)=(0.687242,0.476355,0); rgb(153pt)=(0.682353,0.47451,0); rgb(154pt)=(0.677195,0.472696,0); rgb(155pt)=(0.6717,0.470916,0); rgb(156pt)=(0.665891,0.469169,0); rgb(157pt)=(0.659791,0.46745,0); rgb(158pt)=(0.653423,0.465756,0); rgb(159pt)=(0.64681,0.464084,0); rgb(160pt)=(0.639976,0.462432,0); rgb(161pt)=(0.632943,0.460795,0); rgb(162pt)=(0.625734,0.459171,0); rgb(163pt)=(0.618373,0.457556,0); rgb(164pt)=(0.610882,0.455948,0); rgb(165pt)=(0.603284,0.454343,0); rgb(166pt)=(0.595604,0.452737,0); rgb(167pt)=(0.587863,0.451129,0); rgb(168pt)=(0.580084,0.449514,0); rgb(169pt)=(0.572292,0.447889,0); rgb(170pt)=(0.564508,0.446252,0); rgb(171pt)=(0.556756,0.444599,0); rgb(172pt)=(0.549059,0.442927,0); rgb(173pt)=(0.54144,0.441232,0); rgb(174pt)=(0.533922,0.439512,0); rgb(175pt)=(0.526529,0.437764,0); rgb(176pt)=(0.519282,0.435983,0); rgb(177pt)=(0.512206,0.434168,0); rgb(178pt)=(0.505323,0.432315,0); rgb(179pt)=(0.498628,0.430422,3.92506e-06); rgb(180pt)=(0.491973,0.428504,3.49981e-05); rgb(181pt)=(0.485331,0.426562,9.63073e-05); rgb(182pt)=(0.478704,0.424596,0.000186979); rgb(183pt)=(0.472096,0.422609,0.000306141); rgb(184pt)=(0.465508,0.420599,0.00045292); rgb(185pt)=(0.458942,0.418567,0.000626441); rgb(186pt)=(0.452401,0.416515,0.000825833); rgb(187pt)=(0.445885,0.414441,0.00105022); rgb(188pt)=(0.439399,0.412348,0.00129873); rgb(189pt)=(0.432942,0.410234,0.00157049); rgb(190pt)=(0.426518,0.408102,0.00186463); rgb(191pt)=(0.420129,0.40595,0.00218028); rgb(192pt)=(0.413777,0.40378,0.00251655); rgb(193pt)=(0.407464,0.401592,0.00287258); rgb(194pt)=(0.401191,0.399386,0.00324749); rgb(195pt)=(0.394962,0.397164,0.00364042); rgb(196pt)=(0.388777,0.394925,0.00405048); rgb(197pt)=(0.38264,0.39267,0.00447681); rgb(198pt)=(0.376552,0.390399,0.00491852); rgb(199pt)=(0.370516,0.388113,0.00537476); rgb(200pt)=(0.364532,0.385812,0.00584464); rgb(201pt)=(0.358605,0.383497,0.00632729); rgb(202pt)=(0.352735,0.381168,0.00682184); rgb(203pt)=(0.346925,0.378826,0.00732741); rgb(204pt)=(0.341176,0.376471,0.00784314); rgb(205pt)=(0.335485,0.374093,0.00847245); rgb(206pt)=(0.329843,0.371682,0.00930909); rgb(207pt)=(0.324249,0.369242,0.0103377); rgb(208pt)=(0.318701,0.366772,0.0115428); rgb(209pt)=(0.313198,0.364275,0.0129091); rgb(210pt)=(0.307739,0.361753,0.0144211); rgb(211pt)=(0.302322,0.359206,0.0160634); rgb(212pt)=(0.296945,0.356637,0.0178207); rgb(213pt)=(0.291607,0.354048,0.0196776); rgb(214pt)=(0.286307,0.35144,0.0216186); rgb(215pt)=(0.281043,0.348814,0.0236284); rgb(216pt)=(0.275813,0.346172,0.0256916); rgb(217pt)=(0.270616,0.343517,0.0277927); rgb(218pt)=(0.265451,0.340849,0.0299163); rgb(219pt)=(0.260317,0.33817,0.0320472); rgb(220pt)=(0.25521,0.335482,0.0341698); rgb(221pt)=(0.250131,0.332786,0.0362688); rgb(222pt)=(0.245078,0.330085,0.0383287); rgb(223pt)=(0.240048,0.327379,0.0403343); rgb(224pt)=(0.235042,0.324671,0.04227); rgb(225pt)=(0.230056,0.321962,0.0441205); rgb(226pt)=(0.22509,0.319254,0.0458704); rgb(227pt)=(0.220142,0.316548,0.0475043); rgb(228pt)=(0.215212,0.313846,0.0490067); rgb(229pt)=(0.210296,0.311149,0.0503624); rgb(230pt)=(0.205395,0.308459,0.0515759); rgb(231pt)=(0.200514,0.305763,0.052757); rgb(232pt)=(0.195655,0.303061,0.0539242); rgb(233pt)=(0.190817,0.300353,0.0550763); rgb(234pt)=(0.186001,0.297639,0.0562123); rgb(235pt)=(0.181207,0.294918,0.0573313); rgb(236pt)=(0.176434,0.292191,0.0584321); rgb(237pt)=(0.171685,0.289458,0.0595136); rgb(238pt)=(0.166957,0.286719,0.060575); rgb(239pt)=(0.162252,0.283973,0.0616151); rgb(240pt)=(0.15757,0.281221,0.0626328); rgb(241pt)=(0.152911,0.278463,0.0636271); rgb(242pt)=(0.148275,0.275699,0.0645971); rgb(243pt)=(0.143663,0.272929,0.0655416); rgb(244pt)=(0.139074,0.270152,0.0664596); rgb(245pt)=(0.134508,0.26737,0.06735); rgb(246pt)=(0.129967,0.264581,0.0682118); rgb(247pt)=(0.125449,0.261787,0.0690441); rgb(248pt)=(0.120956,0.258986,0.0698456); rgb(249pt)=(0.116487,0.25618,0.0706154); rgb(250pt)=(0.112043,0.253367,0.0713525); rgb(251pt)=(0.107623,0.250549,0.0720557); rgb(252pt)=(0.103229,0.247724,0.0727241); rgb(253pt)=(0.0988592,0.244894,0.0733566); rgb(254pt)=(0.0945149,0.242058,0.0739522); rgb(255pt)=(0.0901961,0.239216,0.0745098)}, mesh/rows=49]
table[row sep=crcr, point meta=\thisrow{c}] {%
%
x	y	z	c\\
56	0.093	0.726239006785844	0.726239006785844\\
56	0.09666	0.380356276863404	0.380356276863404\\
56	0.10032	0.0600285054100453	0.0600285054100453\\
56	0.10398	-0.234744307574239	-0.234744307574239\\
56	0.10764	-0.503962162089456	-0.503962162089456\\
56	0.1113	-0.747625058135577	-0.747625058135577\\
56	0.11496	-0.965732995712623	-0.965732995712623\\
56	0.11862	-1.15828597482061	-1.15828597482061\\
56	0.12228	-1.32528399545949	-1.32528399545949\\
56	0.12594	-1.4667270576293	-1.4667270576293\\
56	0.1296	-1.58261516133003	-1.58261516133003\\
56	0.13326	-1.67294830656168	-1.67294830656168\\
56	0.13692	-1.73772649332426	-1.73772649332426\\
56	0.14058	-1.77694972161774	-1.77694972161774\\
56	0.14424	-1.79061799144217	-1.79061799144217\\
56	0.1479	-1.7787313027975	-1.7787313027975\\
56	0.15156	-1.74128965568377	-1.74128965568377\\
56	0.15522	-1.67829305010095	-1.67829305010095\\
56	0.15888	-1.58974148604904	-1.58974148604904\\
56	0.16254	-1.47563496352806	-1.47563496352806\\
56	0.1662	-1.335973482538	-1.335973482538\\
56	0.16986	-1.17075704307886	-1.17075704307886\\
56	0.17352	-0.979985645150656	-0.979985645150656\\
56	0.17718	-0.76365928875336	-0.76365928875336\\
56	0.18084	-0.521777973886984	-0.521777973886984\\
56	0.1845	-0.254341700551528	-0.254341700551528\\
56	0.18816	0.0386495312530002	0.0386495312530002\\
56	0.19182	0.357195721526615	0.357195721526615\\
56	0.19548	0.701296870269303	0.701296870269303\\
56	0.19914	1.07095297748108	1.07095297748108\\
56	0.2028	1.46616404316192	1.46616404316192\\
56	0.20646	1.88693006731184	1.88693006731184\\
56	0.21012	2.33325104993086	2.33325104993086\\
56	0.21378	2.80512699101893	2.80512699101893\\
56	0.21744	3.3025578905761	3.3025578905761\\
56	0.2211	3.82554374860234	3.82554374860234\\
56	0.22476	4.37408456509767	4.37408456509767\\
56	0.22842	4.94818034006207	4.94818034006207\\
56	0.23208	5.54783107349555	5.54783107349555\\
56	0.23574	6.17303676539812	6.17303676539812\\
56	0.2394	6.82379741576975	6.82379741576975\\
56	0.24306	7.50011302461047	7.50011302461047\\
56	0.24672	8.20198359192025	8.20198359192025\\
56	0.25038	8.92940911769914	8.92940911769914\\
56	0.25404	9.68238960194709	9.68238960194709\\
56	0.2577	10.4609250446641	10.4609250446641\\
56	0.26136	11.2650154458503	11.2650154458503\\
56	0.26502	12.0946608055054	12.0946608055054\\
56	0.26868	12.9498611236297	12.9498611236297\\
56	0.27234	13.830616400223	13.830616400223\\
56	0.276	14.7369266352855	14.7369266352855\\
56.375	0.093	0.661522114442375	0.661522114442375\\
56.375	0.09666	0.309792224571021	0.309792224571021\\
56.375	0.10032	-0.0163827068312514	-0.0163827068312514\\
56.375	0.10398	-0.317002679764435	-0.317002679764435\\
56.375	0.10764	-0.592067694228566	-0.592067694228566\\
56.375	0.1113	-0.841577750223587	-0.841577750223587\\
56.375	0.11496	-1.06553284774955	-1.06553284774955\\
56.375	0.11862	-1.26393298680643	-1.26393298680643\\
56.375	0.12228	-1.43677816739423	-1.43677816739423\\
56.375	0.12594	-1.58406838951294	-1.58406838951294\\
56.375	0.1296	-1.70580365316258	-1.70580365316258\\
56.375	0.13326	-1.80198395834315	-1.80198395834315\\
56.375	0.13692	-1.87260930505463	-1.87260930505463\\
56.375	0.14058	-1.91767969329702	-1.91767969329702\\
56.375	0.14424	-1.93719512307036	-1.93719512307036\\
56.375	0.1479	-1.93115559437459	-1.93115559437459\\
56.375	0.15156	-1.89956110720976	-1.89956110720976\\
56.375	0.15522	-1.84241166157585	-1.84241166157585\\
56.375	0.15888	-1.75970725747286	-1.75970725747286\\
56.375	0.16254	-1.65144789490079	-1.65144789490079\\
56.375	0.1662	-1.51763357385963	-1.51763357385963\\
56.375	0.16986	-1.3582642943494	-1.3582642943494\\
56.375	0.17352	-1.1733400563701	-1.1733400563701\\
56.375	0.17718	-0.962860859921719	-0.962860859921719\\
56.375	0.18084	-0.726826705004243	-0.726826705004243\\
56.375	0.1845	-0.465237591617701	-0.465237591617701\\
56.375	0.18816	-0.178093519762086	-0.178093519762086\\
56.375	0.19182	0.134605510562629	0.134605510562629\\
56.375	0.19548	0.472859499356403	0.472859499356403\\
56.375	0.19914	0.836668446619264	0.836668446619264\\
56.375	0.2028	1.22603235235121	1.22603235235121\\
56.375	0.20646	1.64095121655223	1.64095121655223\\
56.375	0.21012	2.08142503922233	2.08142503922233\\
56.375	0.21378	2.5474538203615	2.5474538203615\\
56.375	0.21744	3.03903755996976	3.03903755996976\\
56.375	0.2211	3.55617625804709	3.55617625804709\\
56.375	0.22476	4.0988699145935	4.0988699145935\\
56.375	0.22842	4.66711852960901	4.66711852960901\\
56.375	0.23208	5.26092210309357	5.26092210309357\\
56.375	0.23574	5.88028063504724	5.88028063504724\\
56.375	0.2394	6.52519412546997	6.52519412546997\\
56.375	0.24306	7.19566257436176	7.19566257436176\\
56.375	0.24672	7.89168598172266	7.89168598172266\\
56.375	0.25038	8.61326434755262	8.61326434755262\\
56.375	0.25404	9.36039767185167	9.36039767185167\\
56.375	0.2577	10.1330859546198	10.1330859546198\\
56.375	0.26136	10.931329195857	10.931329195857\\
56.375	0.26502	11.7551273955633	11.7551273955633\\
56.375	0.26868	12.6044805537386	12.6044805537386\\
56.375	0.27234	13.4793886703831	13.4793886703831\\
56.375	0.276	14.3798517454966	14.3798517454966\\
56.75	0.093	0.603056289622542	0.603056289622542\\
56.75	0.09666	0.245479239802274	0.245479239802274\\
56.75	0.10032	-0.0865428515488986	-0.0865428515488986\\
56.75	0.10398	-0.393009984430996	-0.393009984430996\\
56.75	0.10764	-0.673922158844027	-0.673922158844027\\
56.75	0.1113	-0.929279374787962	-0.929279374787962\\
56.75	0.11496	-1.15908163226282	-1.15908163226282\\
56.75	0.11862	-1.36332893126862	-1.36332893126862\\
56.75	0.12228	-1.54202127180533	-1.54202127180533\\
56.75	0.12594	-1.69515865387294	-1.69515865387294\\
56.75	0.1296	-1.8227410774715	-1.8227410774715\\
56.75	0.13326	-1.92476854260098	-1.92476854260098\\
56.75	0.13692	-2.00124104926135	-2.00124104926135\\
56.75	0.14058	-2.05215859745266	-2.05215859745266\\
56.75	0.14424	-2.0775211871749	-2.0775211871749\\
56.75	0.1479	-2.07732881842805	-2.07732881842805\\
56.75	0.15156	-2.05158149121212	-2.05158149121212\\
56.75	0.15522	-2.00027920552712	-2.00027920552712\\
56.75	0.15888	-1.92342196137304	-1.92342196137304\\
56.75	0.16254	-1.82100975874987	-1.82100975874987\\
56.75	0.1662	-1.69304259765763	-1.69304259765763\\
56.75	0.16986	-1.53952047809631	-1.53952047809631\\
56.75	0.17352	-1.36044340006591	-1.36044340006591\\
56.75	0.17718	-1.15581136356643	-1.15581136356643\\
56.75	0.18084	-0.925624368597866	-0.925624368597866\\
56.75	0.1845	-0.669882415160224	-0.669882415160224\\
56.75	0.18816	-0.388585503253509	-0.388585503253509\\
56.75	0.19182	-0.0817336328777216	-0.0817336328777216\\
56.75	0.19548	0.250673195967138	0.250673195967138\\
56.75	0.19914	0.608634983281114	0.608634983281114\\
56.75	0.2028	0.992151729064148	0.992151729064148\\
56.75	0.20646	1.40122343331625	1.40122343331625\\
56.75	0.21012	1.83585009603744	1.83585009603744\\
56.75	0.21378	2.29603171722771	2.29603171722771\\
56.75	0.21744	2.78176829688705	2.78176829688705\\
56.75	0.2211	3.29305983501547	3.29305983501547\\
56.75	0.22476	3.829906331613	3.829906331613\\
56.75	0.22842	4.39230778667957	4.39230778667957\\
56.75	0.23208	4.98026420021524	4.98026420021524\\
56.75	0.23574	5.59377557221999	5.59377557221999\\
56.75	0.2394	6.23284190269381	6.23284190269381\\
56.75	0.24306	6.8974631916367	6.8974631916367\\
56.75	0.24672	7.58763943904869	7.58763943904869\\
56.75	0.25038	8.30337064492975	8.30337064492975\\
56.75	0.25404	9.04465680927989	9.04465680927989\\
56.75	0.2577	9.8114979320991	9.8114979320991\\
56.75	0.26136	10.6038940133874	10.6038940133874\\
56.75	0.26502	11.4218450531448	11.4218450531448\\
56.75	0.26868	12.2653510513712	12.2653510513712\\
56.75	0.27234	13.1344120080668	13.1344120080668\\
56.75	0.276	14.0290279232314	14.0290279232314\\
57.125	0.093	0.550841532326272	0.550841532326272\\
57.125	0.09666	0.187417322557105	0.187417322557105\\
57.125	0.10032	-0.150451928742982	-0.150451928742982\\
57.125	0.10398	-0.462766221573993	-0.462766221573993\\
57.125	0.10764	-0.749525555935923	-0.749525555935923\\
57.125	0.1113	-1.01072993182877	-1.01072993182877\\
57.125	0.11496	-1.24637934925254	-1.24637934925254\\
57.125	0.11862	-1.45647380820724	-1.45647380820724\\
57.125	0.12228	-1.64101330869285	-1.64101330869285\\
57.125	0.12594	-1.79999785070939	-1.79999785070939\\
57.125	0.1296	-1.93342743425685	-1.93342743425685\\
57.125	0.13326	-2.04130205933523	-2.04130205933523\\
57.125	0.13692	-2.12362172594453	-2.12362172594453\\
57.125	0.14058	-2.18038643408474	-2.18038643408474\\
57.125	0.14424	-2.21159618375588	-2.21159618375588\\
57.125	0.1479	-2.21725097495794	-2.21725097495794\\
57.125	0.15156	-2.19735080769092	-2.19735080769092\\
57.125	0.15522	-2.15189568195483	-2.15189568195483\\
57.125	0.15888	-2.08088559774965	-2.08088559774965\\
57.125	0.16254	-1.98432055507539	-1.98432055507539\\
57.125	0.1662	-1.86220055393206	-1.86220055393206\\
57.125	0.16986	-1.71452559431965	-1.71452559431965\\
57.125	0.17352	-1.54129567623816	-1.54129567623816\\
57.125	0.17718	-1.34251079968759	-1.34251079968759\\
57.125	0.18084	-1.11817096466794	-1.11817096466794\\
57.125	0.1845	-0.868276171179211	-0.868276171179211\\
57.125	0.18816	-0.592826419221396	-0.592826419221396\\
57.125	0.19182	-0.291821708794508	-0.291821708794508\\
57.125	0.19548	0.0347379601014524	0.0347379601014524\\
57.125	0.19914	0.3868525874665	0.3868525874665\\
57.125	0.2028	0.764522173300634	0.764522173300634\\
57.125	0.20646	1.16774671760384	1.16774671760384\\
57.125	0.21012	1.59652622037611	1.59652622037611\\
57.125	0.21378	2.05086068161747	2.05086068161747\\
57.125	0.21744	2.53075010132791	2.53075010132791\\
57.125	0.2211	3.03619447950743	3.03619447950743\\
57.125	0.22476	3.56719381615603	3.56719381615603\\
57.125	0.22842	4.12374811127371	4.12374811127371\\
57.125	0.23208	4.70585736486046	4.70585736486046\\
57.125	0.23574	5.31352157691631	5.31352157691631\\
57.125	0.2394	5.94674074744122	5.94674074744122\\
57.125	0.24306	6.60551487643521	6.60551487643521\\
57.125	0.24672	7.28984396389828	7.28984396389828\\
57.125	0.25038	7.99972800983044	7.99972800983044\\
57.125	0.25404	8.73516701423166	8.73516701423166\\
57.125	0.2577	9.49616097710198	9.49616097710198\\
57.125	0.26136	10.2827098984414	10.2827098984414\\
57.125	0.26502	11.0948137782498	11.0948137782498\\
57.125	0.26868	11.9324726165274	11.9324726165274\\
57.125	0.27234	12.795686413274	12.795686413274\\
57.125	0.276	13.6844551684897	13.6844551684897\\
57.5	0.093	0.50487784255361	0.50487784255361\\
57.5	0.09666	0.135606472835528	0.135606472835528\\
57.5	0.10032	-0.208109938413472	-0.208109938413472\\
57.5	0.10398	-0.526271391193383	-0.526271391193383\\
57.5	0.10764	-0.818877885504227	-0.818877885504227\\
57.5	0.1113	-1.08592942134598	-1.08592942134598\\
57.5	0.11496	-1.32742599871866	-1.32742599871866\\
57.5	0.11862	-1.54336761762226	-1.54336761762226\\
57.5	0.12228	-1.73375427805678	-1.73375427805678\\
57.5	0.12594	-1.89858598002222	-1.89858598002222\\
57.5	0.1296	-2.03786272351859	-2.03786272351859\\
57.5	0.13326	-2.15158450854587	-2.15158450854587\\
57.5	0.13692	-2.23975133510408	-2.23975133510408\\
57.5	0.14058	-2.30236320319321	-2.30236320319321\\
57.5	0.14424	-2.33942011281325	-2.33942011281325\\
57.5	0.1479	-2.35092206396423	-2.35092206396423\\
57.5	0.15156	-2.33686905664612	-2.33686905664612\\
57.5	0.15522	-2.29726109085892	-2.29726109085892\\
57.5	0.15888	-2.23209816660266	-2.23209816660266\\
57.5	0.16254	-2.14138028387732	-2.14138028387732\\
57.5	0.1662	-2.02510744268288	-2.02510744268288\\
57.5	0.16986	-1.88327964301937	-1.88327964301937\\
57.5	0.17352	-1.7158968848868	-1.7158968848868\\
57.5	0.17718	-1.52295916828514	-1.52295916828514\\
57.5	0.18084	-1.30446649321439	-1.30446649321439\\
57.5	0.1845	-1.06041885967456	-1.06041885967456\\
57.5	0.18816	-0.790816267665676	-0.790816267665676\\
57.5	0.19182	-0.495658717187688	-0.495658717187688\\
57.5	0.19548	-0.174946208240627	-0.174946208240627\\
57.5	0.19914	0.171321259175507	0.171321259175507\\
57.5	0.2028	0.543143685060727	0.543143685060727\\
57.5	0.20646	0.940521069415034	0.940521069415034\\
57.5	0.21012	1.36345341223841	1.36345341223841\\
57.5	0.21378	1.81194071353084	1.81194071353084\\
57.5	0.21744	2.28598297329238	2.28598297329238\\
57.5	0.2211	2.785580191523	2.785580191523\\
57.5	0.22476	3.31073236822269	3.31073236822269\\
57.5	0.22842	3.86143950339144	3.86143950339144\\
57.5	0.23208	4.4377015970293	4.4377015970293\\
57.5	0.23574	5.03951864913623	5.03951864913623\\
57.5	0.2394	5.66689065971224	5.66689065971224\\
57.5	0.24306	6.31981762875732	6.31981762875732\\
57.5	0.24672	6.99829955627149	6.99829955627149\\
57.5	0.25038	7.70233644225474	7.70233644225474\\
57.5	0.25404	8.43192828670705	8.43192828670705\\
57.5	0.2577	9.18707508962846	9.18707508962846\\
57.5	0.26136	9.96777685101895	9.96777685101895\\
57.5	0.26502	10.7740335708785	10.7740335708785\\
57.5	0.26868	11.6058452492071	11.6058452492071\\
57.5	0.27234	12.4632118860048	12.4632118860048\\
57.5	0.276	13.3461334812716	13.3461334812716\\
57.875	0.093	0.465165220304582	0.465165220304582\\
57.875	0.09666	0.0900466906376014	0.0900466906376014\\
57.875	0.10032	-0.259516880560298	-0.259516880560298\\
57.875	0.10398	-0.583525493289123	-0.583525493289123\\
57.875	0.10764	-0.881979147548867	-0.881979147548867\\
57.875	0.1113	-1.15487784333953	-1.15487784333953\\
57.875	0.11496	-1.40222158066113	-1.40222158066113\\
57.875	0.11862	-1.62401035951364	-1.62401035951364\\
57.875	0.12228	-1.82024417989706	-1.82024417989706\\
57.875	0.12594	-1.99092304181141	-1.99092304181141\\
57.875	0.1296	-2.13604694525669	-2.13604694525669\\
57.875	0.13326	-2.25561589023288	-2.25561589023288\\
57.875	0.13692	-2.34962987673999	-2.34962987673999\\
57.875	0.14058	-2.41808890477803	-2.41808890477803\\
57.875	0.14424	-2.46099297434699	-2.46099297434699\\
57.875	0.1479	-2.47834208544686	-2.47834208544686\\
57.875	0.15156	-2.47013623807766	-2.47013623807766\\
57.875	0.15522	-2.43637543223937	-2.43637543223937\\
57.875	0.15888	-2.37705966793202	-2.37705966793202\\
57.875	0.16254	-2.29218894515558	-2.29218894515558\\
57.875	0.1662	-2.18176326391006	-2.18176326391006\\
57.875	0.16986	-2.04578262419546	-2.04578262419546\\
57.875	0.17352	-1.88424702601178	-1.88424702601178\\
57.875	0.17718	-1.69715646935904	-1.69715646935904\\
57.875	0.18084	-1.48451095423719	-1.48451095423719\\
57.875	0.1845	-1.24631048064628	-1.24631048064628\\
57.875	0.18816	-0.982555048586292	-0.982555048586292\\
57.875	0.19182	-0.693244658057218	-0.693244658057218\\
57.875	0.19548	-0.378379309059085	-0.378379309059085\\
57.875	0.19914	-0.0379590015918367	-0.0379590015918367\\
57.875	0.2028	0.32801626434447	0.32801626434447\\
57.875	0.20646	0.719546488749863	0.719546488749863\\
57.875	0.21012	1.13663167162432	1.13663167162432\\
57.875	0.21378	1.57927181296787	1.57927181296787\\
57.875	0.21744	2.04746691278049	2.04746691278049\\
57.875	0.2211	2.5412169710622	2.5412169710622\\
57.875	0.22476	3.06052198781298	3.06052198781298\\
57.875	0.22842	3.60538196303283	3.60538196303283\\
57.875	0.23208	4.17579689672178	4.17579689672178\\
57.875	0.23574	4.77176678887981	4.77176678887981\\
57.875	0.2394	5.39329163950691	5.39329163950691\\
57.875	0.24306	6.04037144860308	6.04037144860308\\
57.875	0.24672	6.71300621616832	6.71300621616832\\
57.875	0.25038	7.41119594220269	7.41119594220269\\
57.875	0.25404	8.13494062670608	8.13494062670608\\
57.875	0.2577	8.88424026967857	8.88424026967857\\
57.875	0.26136	9.65909487112016	9.65909487112016\\
57.875	0.26502	10.4595044310308	10.4595044310308\\
57.875	0.26868	11.2854689494105	11.2854689494105\\
57.875	0.27234	12.1369884262593	12.1369884262593\\
57.875	0.276	13.0140628615772	13.0140628615772\\
58.25	0.093	0.43170366557912	0.43170366557912\\
58.25	0.09666	0.0507379759632247	0.0507379759632247\\
58.25	0.10032	-0.304672755183574	-0.304672755183574\\
58.25	0.10398	-0.634528527861313	-0.634528527861313\\
58.25	0.10764	-0.938829342069971	-0.938829342069971\\
58.25	0.1113	-1.21757519780955	-1.21757519780955\\
58.25	0.11496	-1.47076609508003	-1.47076609508003\\
58.25	0.11862	-1.69840203388146	-1.69840203388146\\
58.25	0.12228	-1.9004830142138	-1.9004830142138\\
58.25	0.12594	-2.07700903607704	-2.07700903607704\\
58.25	0.1296	-2.22798009947123	-2.22798009947123\\
58.25	0.13326	-2.35339620439634	-2.35339620439634\\
58.25	0.13692	-2.45325735085236	-2.45325735085236\\
58.25	0.14058	-2.5275635388393	-2.5275635388393\\
58.25	0.14424	-2.57631476835717	-2.57631476835717\\
58.25	0.1479	-2.59951103940595	-2.59951103940595\\
58.25	0.15156	-2.59715235198566	-2.59715235198566\\
58.25	0.15522	-2.56923870609627	-2.56923870609627\\
58.25	0.15888	-2.51577010173782	-2.51577010173782\\
58.25	0.16254	-2.43674653891029	-2.43674653891029\\
58.25	0.1662	-2.33216801761369	-2.33216801761369\\
58.25	0.16986	-2.202034537848	-2.202034537848\\
58.25	0.17352	-2.04634609961322	-2.04634609961322\\
58.25	0.17718	-1.86510270290938	-1.86510270290938\\
58.25	0.18084	-1.65830434773645	-1.65830434773645\\
58.25	0.1845	-1.42595103409445	-1.42595103409445\\
58.25	0.18816	-1.16804276198336	-1.16804276198336\\
58.25	0.19182	-0.884579531403197	-0.884579531403197\\
58.25	0.19548	-0.575561342353964	-0.575561342353964\\
58.25	0.19914	-0.240988194835644	-0.240988194835644\\
58.25	0.2028	0.119139911151763	0.119139911151763\\
58.25	0.20646	0.504822975608242	0.504822975608242\\
58.25	0.21012	0.916060998533801	0.916060998533801\\
58.25	0.21378	1.35285397992843	1.35285397992843\\
58.25	0.21744	1.81520191979216	1.81520191979216\\
58.25	0.2211	2.30310481812495	2.30310481812495\\
58.25	0.22476	2.81656267492682	2.81656267492682\\
58.25	0.22842	3.35557549019777	3.35557549019777\\
58.25	0.23208	3.92014326393782	3.92014326393782\\
58.25	0.23574	4.51026599614692	4.51026599614692\\
58.25	0.2394	5.12594368682512	5.12594368682512\\
58.25	0.24306	5.76717633597239	5.76717633597239\\
58.25	0.24672	6.43396394358874	6.43396394358874\\
58.25	0.25038	7.12630650967417	7.12630650967417\\
58.25	0.25404	7.84420403422867	7.84420403422867\\
58.25	0.2577	8.58765651725226	8.58765651725226\\
58.25	0.26136	9.35666395874495	9.35666395874495\\
58.25	0.26502	10.1512263587067	10.1512263587067\\
58.25	0.26868	10.9713437171375	10.9713437171375\\
58.25	0.27234	11.8170160340374	11.8170160340374\\
58.25	0.276	12.6882433094064	12.6882433094064\\
58.625	0.093	0.404493178377278	0.404493178377278\\
58.625	0.09666	0.0176803288124692	0.0176803288124692\\
58.625	0.10032	-0.343577562283244	-0.343577562283244\\
58.625	0.10398	-0.679280494909882	-0.679280494909882\\
58.625	0.10764	-0.989428469067439	-0.989428469067439\\
58.625	0.1113	-1.27402148475591	-1.27402148475591\\
58.625	0.11496	-1.53305954197533	-1.53305954197533\\
58.625	0.11862	-1.76654264072565	-1.76654264072565\\
58.625	0.12228	-1.97447078100691	-1.97447078100691\\
58.625	0.12594	-2.15684396281907	-2.15684396281907\\
58.625	0.1296	-2.31366218616215	-2.31366218616215\\
58.625	0.13326	-2.44492545103616	-2.44492545103616\\
58.625	0.13692	-2.55063375744109	-2.55063375744109\\
58.625	0.14058	-2.63078710537694	-2.63078710537694\\
58.625	0.14424	-2.68538549484372	-2.68538549484372\\
58.625	0.1479	-2.71442892584141	-2.71442892584141\\
58.625	0.15156	-2.71791739837002	-2.71791739837002\\
58.625	0.15522	-2.69585091242955	-2.69585091242955\\
58.625	0.15888	-2.64822946802001	-2.64822946802001\\
58.625	0.16254	-2.5750530651414	-2.5750530651414\\
58.625	0.1662	-2.47632170379369	-2.47632170379369\\
58.625	0.16986	-2.3520353839769	-2.3520353839769\\
58.625	0.17352	-2.20219410569104	-2.20219410569104\\
58.625	0.17718	-2.0267978689361	-2.0267978689361\\
58.625	0.18084	-1.82584667371208	-1.82584667371208\\
58.625	0.1845	-1.59934052001901	-1.59934052001901\\
58.625	0.18816	-1.34727940785682	-1.34727940785682\\
58.625	0.19182	-1.06966333722556	-1.06966333722556\\
58.625	0.19548	-0.766492308125223	-0.766492308125223\\
58.625	0.19914	-0.437766320555802	-0.437766320555802\\
58.625	0.2028	-0.0834853745173234	-0.0834853745173234\\
58.625	0.20646	0.296350529990256	0.296350529990256\\
58.625	0.21012	0.701741392966902	0.701741392966902\\
58.625	0.21378	1.13268721441263	1.13268721441263\\
58.625	0.21744	1.58918799432745	1.58918799432745\\
58.625	0.2211	2.07124373271134	2.07124373271134\\
58.625	0.22476	2.5788544295643	2.5788544295643\\
58.625	0.22842	3.11202008488635	3.11202008488635\\
58.625	0.23208	3.67074069867747	3.67074069867747\\
58.625	0.23574	4.25501627093768	4.25501627093768\\
58.625	0.2394	4.86484680166696	4.86484680166696\\
58.625	0.24306	5.50023229086533	5.50023229086533\\
58.625	0.24672	6.16117273853274	6.16117273853274\\
58.625	0.25038	6.84766814466928	6.84766814466928\\
58.625	0.25404	7.55971850927487	7.55971850927487\\
58.625	0.2577	8.29732383234956	8.29732383234956\\
58.625	0.26136	9.06048411389332	9.06048411389332\\
58.625	0.26502	9.84919935390616	9.84919935390616\\
58.625	0.26868	10.6634695523881	10.6634695523881\\
58.625	0.27234	11.503294709339	11.503294709339\\
58.625	0.276	12.3686748247591	12.3686748247591\\
59	0.093	0.383533758699071	0.383533758699071\\
59	0.09666	-0.0091262508146368	-0.0091262508146368\\
59	0.10032	-0.376231301859264	-0.376231301859264\\
59	0.10398	-0.717781394434816	-0.717781394434816\\
59	0.10764	-1.03377652854129	-1.03377652854129\\
59	0.1113	-1.32421670417868	-1.32421670417868\\
59	0.11496	-1.58910192134699	-1.58910192134699\\
59	0.11862	-1.82843218004623	-1.82843218004623\\
59	0.12228	-2.04220748027637	-2.04220748027637\\
59	0.12594	-2.23042782203744	-2.23042782203744\\
59	0.1296	-2.39309320532944	-2.39309320532944\\
59	0.13326	-2.53020363015236	-2.53020363015236\\
59	0.13692	-2.6417590965062	-2.6417590965062\\
59	0.14058	-2.72775960439095	-2.72775960439095\\
59	0.14424	-2.78820515380664	-2.78820515380664\\
59	0.1479	-2.82309574475324	-2.82309574475324\\
59	0.15156	-2.83243137723076	-2.83243137723076\\
59	0.15522	-2.81621205123921	-2.81621205123921\\
59	0.15888	-2.77443776677856	-2.77443776677856\\
59	0.16254	-2.70710852384884	-2.70710852384884\\
59	0.1662	-2.61422432245006	-2.61422432245006\\
59	0.16986	-2.49578516258217	-2.49578516258217\\
59	0.17352	-2.35179104424522	-2.35179104424522\\
59	0.17718	-2.1822419674392	-2.1822419674392\\
59	0.18084	-1.98713793216407	-1.98713793216407\\
59	0.1845	-1.7664789384199	-1.7664789384199\\
59	0.18816	-1.52026498620663	-1.52026498620663\\
59	0.19182	-1.24849607552428	-1.24849607552428\\
59	0.19548	-0.95117220637286	-0.95117220637286\\
59	0.19914	-0.628293378752353	-0.628293378752353\\
59	0.2028	-0.279859592662774	-0.279859592662774\\
59	0.20646	0.0941291518959062	0.0941291518959062\\
59	0.21012	0.493672854923652	0.493672854923652\\
59	0.21378	0.918771516420456	0.918771516420456\\
59	0.21744	1.36942513638637	1.36942513638637\\
59	0.2211	1.84563371482135	1.84563371482135\\
59	0.22476	2.3473972517254	2.3473972517254\\
59	0.22842	2.87471574709854	2.87471574709854\\
59	0.23208	3.42758920094075	3.42758920094075\\
59	0.23574	4.00601761325205	4.00601761325205\\
59	0.2394	4.61000098403243	4.61000098403243\\
59	0.24306	5.2395393132819	5.2395393132819\\
59	0.24672	5.89463260100041	5.89463260100041\\
59	0.25038	6.57528084718805	6.57528084718805\\
59	0.25404	7.28148405184471	7.28148405184471\\
59	0.2577	8.01324221497048	8.01324221497048\\
59	0.26136	8.77055533656537	8.77055533656537\\
59	0.26502	9.55342341662927	9.55342341662927\\
59	0.26868	10.3618464551623	10.3618464551623\\
59	0.27234	11.1958244521643	11.1958244521643\\
59	0.276	12.0553574076355	12.0553574076355\\
59.375	0.093	0.368825406544415	0.368825406544415\\
59.375	0.09666	-0.0296817629181927	-0.0296817629181927\\
59.375	0.10032	-0.402633973911719	-0.402633973911719\\
59.375	0.10398	-0.750031226436185	-0.750031226436185\\
59.375	0.10764	-1.07187352049157	-1.07187352049157\\
59.375	0.1113	-1.36816085607786	-1.36816085607786\\
59.375	0.11496	-1.63889323319507	-1.63889323319507\\
59.375	0.11862	-1.88407065184322	-1.88407065184322\\
59.375	0.12228	-2.10369311202229	-2.10369311202229\\
59.375	0.12594	-2.29776061373225	-2.29776061373225\\
59.375	0.1296	-2.46627315697317	-2.46627315697317\\
59.375	0.13326	-2.60923074174498	-2.60923074174498\\
59.375	0.13692	-2.72663336804775	-2.72663336804775\\
59.375	0.14058	-2.8184810358814	-2.8184810358814\\
59.375	0.14424	-2.884773745246	-2.884773745246\\
59.375	0.1479	-2.9255114961415	-2.9255114961415\\
59.375	0.15156	-2.94069428856793	-2.94069428856793\\
59.375	0.15522	-2.93032212252528	-2.93032212252528\\
59.375	0.15888	-2.89439499801354	-2.89439499801354\\
59.375	0.16254	-2.83291291503274	-2.83291291503274\\
59.375	0.1662	-2.74587587358286	-2.74587587358286\\
59.375	0.16986	-2.63328387366389	-2.63328387366389\\
59.375	0.17352	-2.49513691527584	-2.49513691527584\\
59.375	0.17718	-2.33143499841873	-2.33143499841873\\
59.375	0.18084	-2.14217812309252	-2.14217812309252\\
59.375	0.1845	-1.92736628929725	-1.92736628929725\\
59.375	0.18816	-1.68699949703287	-1.68699949703287\\
59.375	0.19182	-1.42107774629944	-1.42107774629944\\
59.375	0.19548	-1.12960103709693	-1.12960103709693\\
59.375	0.19914	-0.812569369425326	-0.812569369425326\\
59.375	0.2028	-0.46998274328466	-0.46998274328466\\
59.375	0.20646	-0.10184115867488	-0.10184115867488\\
59.375	0.21012	0.291855384403938	0.291855384403938\\
59.375	0.21378	0.711106885951843	0.711106885951843\\
59.375	0.21744	1.15591334596884	1.15591334596884\\
59.375	0.2211	1.62627476445492	1.62627476445492\\
59.375	0.22476	2.12219114141006	2.12219114141006\\
59.375	0.22842	2.6436624768343	2.6436624768343\\
59.375	0.23208	3.19068877072761	3.19068877072761\\
59.375	0.23574	3.76327002309001	3.76327002309001\\
59.375	0.2394	4.36140623392146	4.36140623392146\\
59.375	0.24306	4.98509740322203	4.98509740322203\\
59.375	0.24672	5.63434353099161	5.63434353099161\\
59.375	0.25038	6.30914461723035	6.30914461723035\\
59.375	0.25404	7.00950066193812	7.00950066193812\\
59.375	0.2577	7.73541166511498	7.73541166511498\\
59.375	0.26136	8.48687762676094	8.48687762676094\\
59.375	0.26502	9.26389854687592	9.26389854687592\\
59.375	0.26868	10.06647442546	10.06647442546\\
59.375	0.27234	10.8946052625132	10.8946052625132\\
59.375	0.276	11.7482910580354	11.7482910580354\\
59.75	0.093	0.360368121913394	0.360368121913394\\
59.75	0.09666	-0.0439862074981274	-0.0439862074981274\\
59.75	0.10032	-0.422785578440568	-0.422785578440568\\
59.75	0.10398	-0.776029990913933	-0.776029990913933\\
59.75	0.10764	-1.10371944491822	-1.10371944491822\\
59.75	0.1113	-1.40585394045342	-1.40585394045342\\
59.75	0.11496	-1.68243347751955	-1.68243347751955\\
59.75	0.11862	-1.9334580561166	-1.9334580561166\\
59.75	0.12228	-2.15892767624456	-2.15892767624456\\
59.75	0.12594	-2.35884233790345	-2.35884233790345\\
59.75	0.1296	-2.53320204109327	-2.53320204109327\\
59.75	0.13326	-2.682006785814	-2.682006785814\\
59.75	0.13692	-2.80525657206566	-2.80525657206566\\
59.75	0.14058	-2.90295139984822	-2.90295139984822\\
59.75	0.14424	-2.97509126916173	-2.97509126916173\\
59.75	0.1479	-3.02167618000615	-3.02167618000615\\
59.75	0.15156	-3.04270613238147	-3.04270613238147\\
59.75	0.15522	-3.03818112628774	-3.03818112628774\\
59.75	0.15888	-3.00810116172491	-3.00810116172491\\
59.75	0.16254	-2.95246623869301	-2.95246623869301\\
59.75	0.1662	-2.87127635719205	-2.87127635719205\\
59.75	0.16986	-2.76453151722198	-2.76453151722198\\
59.75	0.17352	-2.63223171878284	-2.63223171878284\\
59.75	0.17718	-2.47437696187464	-2.47437696187464\\
59.75	0.18084	-2.29096724649733	-2.29096724649733\\
59.75	0.1845	-2.08200257265097	-2.08200257265097\\
59.75	0.18816	-1.84748294033551	-1.84748294033551\\
59.75	0.19182	-1.58740834955098	-1.58740834955098\\
59.75	0.19548	-1.30177880029737	-1.30177880029737\\
59.75	0.19914	-0.990594292574677	-0.990594292574677\\
59.75	0.2028	-0.653854826382911	-0.653854826382911\\
59.75	0.20646	-0.291560401722045	-0.291560401722045\\
59.75	0.21012	0.0962889814078736	0.0962889814078736\\
59.75	0.21378	0.509693323006864	0.509693323006864\\
59.75	0.21744	0.948652623074963	0.948652623074963\\
59.75	0.2211	1.41316688161213	1.41316688161213\\
59.75	0.22476	1.90323609861836	1.90323609861836\\
59.75	0.22842	2.4188602740937	2.4188602740937\\
59.75	0.23208	2.96003940803809	2.96003940803809\\
59.75	0.23574	3.52677350045156	3.52677350045156\\
59.75	0.2394	4.11906255133412	4.11906255133412\\
59.75	0.24306	4.73690656068578	4.73690656068578\\
59.75	0.24672	5.38030552850647	5.38030552850647\\
59.75	0.25038	6.04925945479628	6.04925945479628\\
59.75	0.25404	6.74376833955515	6.74376833955515\\
59.75	0.2577	7.46383218278311	7.46383218278311\\
59.75	0.26136	8.20945098448014	8.20945098448014\\
59.75	0.26502	8.98062474464625	8.98062474464625\\
59.75	0.26868	9.77735346328143	9.77735346328143\\
59.75	0.27234	10.5996371403857	10.5996371403857\\
59.75	0.276	11.447475775959	11.447475775959\\
60.125	0.093	0.358161904806009	0.358161904806009\\
60.125	0.09666	-0.0520395845544268	-0.0520395845544268\\
60.125	0.10032	-0.436686115445781	-0.436686115445781\\
60.125	0.10398	-0.79577768786806	-0.79577768786806\\
60.125	0.10764	-1.12931430182124	-1.12931430182124\\
60.125	0.1113	-1.43729595730536	-1.43729595730536\\
60.125	0.11496	-1.7197226543204	-1.7197226543204\\
60.125	0.11862	-1.97659439286635	-1.97659439286635\\
60.125	0.12228	-2.20791117294323	-2.20791117294323\\
60.125	0.12594	-2.41367299455104	-2.41367299455104\\
60.125	0.1296	-2.59387985768975	-2.59387985768975\\
60.125	0.13326	-2.74853176235938	-2.74853176235938\\
60.125	0.13692	-2.87762870855996	-2.87762870855996\\
60.125	0.14058	-2.98117069629143	-2.98117069629143\\
60.125	0.14424	-3.05915772555385	-3.05915772555385\\
60.125	0.1479	-3.11158979634717	-3.11158979634717\\
60.125	0.15156	-3.1384669086714	-3.1384669086714\\
60.125	0.15522	-3.13978906252659	-3.13978906252659\\
60.125	0.15888	-3.11555625791266	-3.11555625791266\\
60.125	0.16254	-3.06576849482966	-3.06576849482966\\
60.125	0.1662	-2.99042577327761	-2.99042577327761\\
60.125	0.16986	-2.88952809325645	-2.88952809325645\\
60.125	0.17352	-2.76307545476622	-2.76307545476622\\
60.125	0.17718	-2.61106785780693	-2.61106785780693\\
60.125	0.18084	-2.43350530237853	-2.43350530237853\\
60.125	0.1845	-2.23038778848106	-2.23038778848106\\
60.125	0.18816	-2.00171531611453	-2.00171531611453\\
60.125	0.19182	-1.74748788527889	-1.74748788527889\\
60.125	0.19548	-1.4677054959742	-1.4677054959742\\
60.125	0.19914	-1.16236814820041	-1.16236814820041\\
60.125	0.2028	-0.831475841957555	-0.831475841957555\\
60.125	0.20646	-0.475028577245602	-0.475028577245602\\
60.125	0.21012	-0.0930263540645981	-0.0930263540645981\\
60.125	0.21378	0.314530827585507	0.314530827585507\\
60.125	0.21744	0.747642967704678	0.747642967704678\\
60.125	0.2211	1.20631006629294	1.20631006629294\\
60.125	0.22476	1.69053212335027	1.69053212335027\\
60.125	0.22842	2.2003091388767	2.2003091388767\\
60.125	0.23208	2.73564111287217	2.73564111287217\\
60.125	0.23574	3.29652804533676	3.29652804533676\\
60.125	0.2394	3.88296993627042	3.88296993627042\\
60.125	0.24306	4.49496678567313	4.49496678567313\\
60.125	0.24672	5.13251859354494	5.13251859354494\\
60.125	0.25038	5.79562535988585	5.79562535988585\\
60.125	0.25404	6.48428708469579	6.48428708469579\\
60.125	0.2577	7.19850376797486	7.19850376797486\\
60.125	0.26136	7.93827540972299	7.93827540972299\\
60.125	0.26502	8.70360200994017	8.70360200994017\\
60.125	0.26868	9.49448356862645	9.49448356862645\\
60.125	0.27234	10.3109200857818	10.3109200857818\\
60.125	0.276	11.1529115614062	11.1529115614062\\
60.5	0.093	0.362206755222173	0.362206755222173\\
60.5	0.09666	-0.0538418940871619	-0.0538418940871619\\
60.5	0.10032	-0.444335584927416	-0.444335584927416\\
60.5	0.10398	-0.809274317298609	-0.809274317298609\\
60.5	0.10764	-1.14865809120071	-1.14865809120071\\
60.5	0.1113	-1.46248690663372	-1.46248690663372\\
60.5	0.11496	-1.75076076359768	-1.75076076359768\\
60.5	0.11862	-2.01347966209254	-2.01347966209254\\
60.5	0.12228	-2.25064360211832	-2.25064360211832\\
60.5	0.12594	-2.46225258367504	-2.46225258367504\\
60.5	0.1296	-2.64830660676265	-2.64830660676265\\
60.5	0.13326	-2.8088056713812	-2.8088056713812\\
60.5	0.13692	-2.94374977753068	-2.94374977753068\\
60.5	0.14058	-3.05313892521106	-3.05313892521106\\
60.5	0.14424	-3.13697311442238	-3.13697311442238\\
60.5	0.1479	-3.19525234516461	-3.19525234516461\\
60.5	0.15156	-3.22797661743776	-3.22797661743776\\
60.5	0.15522	-3.23514593124185	-3.23514593124185\\
60.5	0.15888	-3.21676028657683	-3.21676028657683\\
60.5	0.16254	-3.17281968344275	-3.17281968344275\\
60.5	0.1662	-3.1033241218396	-3.1033241218396\\
60.5	0.16986	-3.00827360176734	-3.00827360176734\\
60.5	0.17352	-2.88766812322601	-2.88766812322601\\
60.5	0.17718	-2.74150768621563	-2.74150768621563\\
60.5	0.18084	-2.56979229073615	-2.56979229073615\\
60.5	0.1845	-2.3725219367876	-2.3725219367876\\
60.5	0.18816	-2.14969662436995	-2.14969662436995\\
60.5	0.19182	-1.90131635348323	-1.90131635348323\\
60.5	0.19548	-1.62738112412745	-1.62738112412745\\
60.5	0.19914	-1.32789093630256	-1.32789093630256\\
60.5	0.2028	-1.00284579000862	-1.00284579000862\\
60.5	0.20646	-0.652245685245582	-0.652245685245582\\
60.5	0.21012	-0.276090622013463	-0.276090622013463\\
60.5	0.21378	0.125619399687714	0.125619399687714\\
60.5	0.21744	0.552884379857986	0.552884379857986\\
60.5	0.2211	1.00570431849734	1.00570431849734\\
60.5	0.22476	1.48407921560575	1.48407921560575\\
60.5	0.22842	1.98800907118328	1.98800907118328\\
60.5	0.23208	2.51749388522985	2.51749388522985\\
60.5	0.23574	3.07253365774552	3.07253365774552\\
60.5	0.2394	3.65312838873027	3.65312838873027\\
60.5	0.24306	4.25927807818408	4.25927807818408\\
60.5	0.24672	4.89098272610697	4.89098272610697\\
60.5	0.25038	5.54824233249898	5.54824233249898\\
60.5	0.25404	6.23105689736002	6.23105689736002\\
60.5	0.2577	6.93942642069015	6.93942642069015\\
60.5	0.26136	7.67335090248939	7.67335090248939\\
60.5	0.26502	8.43283034275767	8.43283034275767\\
60.5	0.26868	9.21786474149502	9.21786474149502\\
60.5	0.27234	10.0284540987015	10.0284540987015\\
60.5	0.276	10.864598414377	10.864598414377\\
60.875	0.093	0.372502673161945	0.372502673161945\\
60.875	0.09666	-0.0493931360963042	-0.0493931360963042\\
60.875	0.10032	-0.445733986885472	-0.445733986885472\\
60.875	0.10398	-0.816519879205565	-0.816519879205565\\
60.875	0.10764	-1.16175081305658	-1.16175081305658\\
60.875	0.1113	-1.48142678843849	-1.48142678843849\\
60.875	0.11496	-1.77554780535136	-1.77554780535136\\
60.875	0.11862	-2.04411386379512	-2.04411386379512\\
60.875	0.12228	-2.2871249637698	-2.2871249637698\\
60.875	0.12594	-2.50458110527543	-2.50458110527543\\
60.875	0.1296	-2.69648228831196	-2.69648228831196\\
60.875	0.13326	-2.86282851287942	-2.86282851287942\\
60.875	0.13692	-3.0036197789778	-3.0036197789778\\
60.875	0.14058	-3.1188560866071	-3.1188560866071\\
60.875	0.14424	-3.20853743576732	-3.20853743576732\\
60.875	0.1479	-3.27266382645846	-3.27266382645846\\
60.875	0.15156	-3.31123525868053	-3.31123525868053\\
60.875	0.15522	-3.32425173243351	-3.32425173243351\\
60.875	0.15888	-3.31171324771741	-3.31171324771741\\
60.875	0.16254	-3.27361980453222	-3.27361980453222\\
60.875	0.1662	-3.20997140287799	-3.20997140287799\\
60.875	0.16986	-3.12076804275464	-3.12076804275464\\
60.875	0.17352	-3.00600972416222	-3.00600972416222\\
60.875	0.17718	-2.86569644710075	-2.86569644710075\\
60.875	0.18084	-2.69982821157016	-2.69982821157016\\
60.875	0.1845	-2.50840501757052	-2.50840501757052\\
60.875	0.18816	-2.29142686510178	-2.29142686510178\\
60.875	0.19182	-2.04889375416398	-2.04889375416398\\
60.875	0.19548	-1.7808056847571	-1.7808056847571\\
60.875	0.19914	-1.48716265688113	-1.48716265688113\\
60.875	0.2028	-1.16796467053608	-1.16796467053608\\
60.875	0.20646	-0.823211725721954	-0.823211725721954\\
60.875	0.21012	-0.452903822438763	-0.452903822438763\\
60.875	0.21378	-0.0570409606864857	-0.0570409606864857\\
60.875	0.21744	0.364376859534872	0.364376859534872\\
60.875	0.2211	0.811349638225337	0.811349638225337\\
60.875	0.22476	1.28387737538485	1.28387737538485\\
60.875	0.22842	1.78196007101345	1.78196007101345\\
60.875	0.23208	2.30559772511112	2.30559772511112\\
60.875	0.23574	2.85479033767789	2.85479033767789\\
60.875	0.2394	3.42953790871372	3.42953790871372\\
60.875	0.24306	4.02984043821863	4.02984043821863\\
60.875	0.24672	4.65569792619262	4.65569792619262\\
60.875	0.25038	5.3071103726357	5.3071103726357\\
60.875	0.25404	5.98407777754784	5.98407777754784\\
60.875	0.2577	6.68660014092907	6.68660014092907\\
60.875	0.26136	7.41467746277938	7.41467746277938\\
60.875	0.26502	8.16830974309876	8.16830974309876\\
60.875	0.26868	8.94749698188721	8.94749698188721\\
60.875	0.27234	9.75223917914477	9.75223917914477\\
60.875	0.276	10.5825363348714	10.5825363348714\\
61.25	0.093	0.389049658625366	0.389049658625366\\
61.25	0.09666	-0.038693310581797	-0.038693310581797\\
61.25	0.10032	-0.440881321319864	-0.440881321319864\\
61.25	0.10398	-0.817514373588871	-0.817514373588871\\
61.25	0.10764	-1.16859246738878	-1.16859246738878\\
61.25	0.1113	-1.49411560271961	-1.49411560271961\\
61.25	0.11496	-1.79408377958138	-1.79408377958138\\
61.25	0.11862	-2.06849699797406	-2.06849699797406\\
61.25	0.12228	-2.31735525789765	-2.31735525789765\\
61.25	0.12594	-2.54065855935218	-2.54065855935218\\
61.25	0.1296	-2.73840690233762	-2.73840690233762\\
61.25	0.13326	-2.91060028685398	-2.91060028685398\\
61.25	0.13692	-3.05723871290128	-3.05723871290128\\
61.25	0.14058	-3.17832218047948	-3.17832218047948\\
61.25	0.14424	-3.27385068958861	-3.27385068958861\\
61.25	0.1479	-3.34382424022865	-3.34382424022865\\
61.25	0.15156	-3.38824283239961	-3.38824283239961\\
61.25	0.15522	-3.40710646610152	-3.40710646610152\\
61.25	0.15888	-3.40041514133432	-3.40041514133432\\
61.25	0.16254	-3.36816885809805	-3.36816885809805\\
61.25	0.1662	-3.31036761639271	-3.31036761639271\\
61.25	0.16986	-3.22701141621828	-3.22701141621828\\
61.25	0.17352	-3.11810025757477	-3.11810025757477\\
61.25	0.17718	-2.9836341404622	-2.9836341404622\\
61.25	0.18084	-2.82361306488053	-2.82361306488053\\
61.25	0.1845	-2.6380370308298	-2.6380370308298\\
61.25	0.18816	-2.42690603830998	-2.42690603830998\\
61.25	0.19182	-2.19022008732107	-2.19022008732107\\
61.25	0.19548	-1.92797917786309	-1.92797917786309\\
61.25	0.19914	-1.64018330993603	-1.64018330993603\\
61.25	0.2028	-1.3268324835399	-1.3268324835399\\
61.25	0.20646	-0.987926698674677	-0.987926698674677\\
61.25	0.21012	-0.623465955340386	-0.623465955340386\\
61.25	0.21378	-0.233450253537022	-0.233450253537022\\
61.25	0.21744	0.182120406735422	0.182120406735422\\
61.25	0.2211	0.623246025476988	0.623246025476988\\
61.25	0.22476	1.08992660268758	1.08992660268758\\
61.25	0.22842	1.58216213836727	1.58216213836727\\
61.25	0.23208	2.09995263251604	2.09995263251604\\
61.25	0.23574	2.64329808513389	2.64329808513389\\
61.25	0.2394	3.21219849622084	3.21219849622084\\
61.25	0.24306	3.80665386577682	3.80665386577682\\
61.25	0.24672	4.42666419380191	4.42666419380191\\
61.25	0.25038	5.07222948029609	5.07222948029609\\
61.25	0.25404	5.74334972525931	5.74334972525931\\
61.25	0.2577	6.44002492869164	6.44002492869164\\
61.25	0.26136	7.16225509059305	7.16225509059305\\
61.25	0.26502	7.9100402109635	7.9100402109635\\
61.25	0.26868	8.68338028980308	8.68338028980308\\
61.25	0.27234	9.48227532711171	9.48227532711171\\
61.25	0.276	10.3067253228894	10.3067253228894\\
61.625	0.093	0.411847711612365	0.411847711612365\\
61.625	0.09666	-0.0217424175437113	-0.0217424175437113\\
61.625	0.10032	-0.429777588230678	-0.429777588230678\\
61.625	0.10398	-0.812257800448599	-0.812257800448599\\
61.625	0.10764	-1.16918305419741	-1.16918305419741\\
61.625	0.1113	-1.50055334947715	-1.50055334947715\\
61.625	0.11496	-1.80636868628783	-1.80636868628783\\
61.625	0.11862	-2.08662906462943	-2.08662906462943\\
61.625	0.12228	-2.34133448450192	-2.34133448450192\\
61.625	0.12594	-2.57048494590536	-2.57048494590536\\
61.625	0.1296	-2.7740804488397	-2.7740804488397\\
61.625	0.13326	-2.95212099330496	-2.95212099330496\\
61.625	0.13692	-3.10460657930117	-3.10460657930117\\
61.625	0.14058	-3.23153720682829	-3.23153720682829\\
61.625	0.14424	-3.33291287588633	-3.33291287588633\\
61.625	0.1479	-3.40873358647529	-3.40873358647529\\
61.625	0.15156	-3.45899933859515	-3.45899933859515\\
61.625	0.15522	-3.48371013224596	-3.48371013224596\\
61.625	0.15888	-3.48286596742767	-3.48286596742767\\
61.625	0.16254	-3.45646684414032	-3.45646684414032\\
61.625	0.1662	-3.40451276238388	-3.40451276238388\\
61.625	0.16986	-3.32700372215836	-3.32700372215836\\
61.625	0.17352	-3.22393972346375	-3.22393972346375\\
61.625	0.17718	-3.09532076630009	-3.09532076630009\\
61.625	0.18084	-2.94114685066732	-2.94114685066732\\
61.625	0.1845	-2.7614179765655	-2.7614179765655\\
61.625	0.18816	-2.55613414399458	-2.55613414399458\\
61.625	0.19182	-2.32529535295459	-2.32529535295459\\
61.625	0.19548	-2.06890160344552	-2.06890160344552\\
61.625	0.19914	-1.78695289546737	-1.78695289546737\\
61.625	0.2028	-1.47944922902015	-1.47944922902015\\
61.625	0.20646	-1.14639060410382	-1.14639060410382\\
61.625	0.21012	-0.787777020718444	-0.787777020718444\\
61.625	0.21378	-0.40360847886398	-0.40360847886398\\
61.625	0.21744	0.00611502145955001	0.00611502145955001\\
61.625	0.2211	0.441393480252188	0.441393480252188\\
61.625	0.22476	0.902226897513906	0.902226897513906\\
61.625	0.22842	1.38861527324467	1.38861527324467\\
61.625	0.23208	1.90055860744454	1.90055860744454\\
61.625	0.23574	2.43805690011349	2.43805690011349\\
61.625	0.2394	3.00111015125152	3.00111015125152\\
61.625	0.24306	3.5897183608586	3.5897183608586\\
61.625	0.24672	4.20388152893476	4.20388152893476\\
61.625	0.25038	4.84359965548004	4.84359965548004\\
61.625	0.25404	5.50887274049435	5.50887274049435\\
61.625	0.2577	6.19970078397779	6.19970078397779\\
61.625	0.26136	6.91608378593027	6.91608378593027\\
61.625	0.26502	7.65802174635185	7.65802174635185\\
61.625	0.26868	8.4255146652425	8.4255146652425\\
61.625	0.27234	9.2185625426022	9.2185625426022\\
61.625	0.276	10.037165378431	10.037165378431\\
62	0.093	0.440896832122958	0.440896832122958\\
62	0.09666	0.00145954301798135	0.00145954301798135\\
62	0.10032	-0.412422787617913	-0.412422787617913\\
62	0.10398	-0.800750159784734	-0.800750159784734\\
62	0.10764	-1.16352257348246	-1.16352257348246\\
62	0.1113	-1.5007400287111	-1.5007400287111\\
62	0.11496	-1.8124025254707	-1.8124025254707\\
62	0.11862	-2.09851006376119	-2.09851006376119\\
62	0.12228	-2.35906264358259	-2.35906264358259\\
62	0.12594	-2.59406026493494	-2.59406026493494\\
62	0.1296	-2.80350292781819	-2.80350292781819\\
62	0.13326	-2.98739063223238	-2.98739063223238\\
62	0.13692	-3.14572337817749	-3.14572337817749\\
62	0.14058	-3.27850116565351	-3.27850116565351\\
62	0.14424	-3.38572399466046	-3.38572399466046\\
62	0.1479	-3.4673918651983	-3.4673918651983\\
62	0.15156	-3.52350477726709	-3.52350477726709\\
62	0.15522	-3.5540627308668	-3.5540627308668\\
62	0.15888	-3.55906572599743	-3.55906572599743\\
62	0.16254	-3.53851376265897	-3.53851376265897\\
62	0.1662	-3.49240684085145	-3.49240684085145\\
62	0.16986	-3.42074496057483	-3.42074496057483\\
62	0.17352	-3.32352812182915	-3.32352812182915\\
62	0.17718	-3.20075632461439	-3.20075632461439\\
62	0.18084	-3.05242956893052	-3.05242956893052\\
62	0.1845	-2.87854785477762	-2.87854785477762\\
62	0.18816	-2.67911118215559	-2.67911118215559\\
62	0.19182	-2.45411955106452	-2.45411955106452\\
62	0.19548	-2.20357296150436	-2.20357296150436\\
62	0.19914	-1.92747141347511	-1.92747141347511\\
62	0.2028	-1.6258149069768	-1.6258149069768\\
62	0.20646	-1.29860344200939	-1.29860344200939\\
62	0.21012	-0.945837018572909	-0.945837018572909\\
62	0.21378	-0.567515636667359	-0.567515636667359\\
62	0.21744	-0.163639296292729	-0.163639296292729\\
62	0.2211	0.265792002551009	0.265792002551009\\
62	0.22476	0.720778259863799	0.720778259863799\\
62	0.22842	1.20131947564567	1.20131947564567\\
62	0.23208	1.70741564989664	1.70741564989664\\
62	0.23574	2.23906678261665	2.23906678261665\\
62	0.2394	2.79627287380578	2.79627287380578\\
62	0.24306	3.37903392346396	3.37903392346396\\
62	0.24672	3.98734993159123	3.98734993159123\\
62	0.25038	4.62122089818758	4.62122089818758\\
62	0.25404	5.28064682325299	5.28064682325299\\
62	0.2577	5.9656277067875	5.9656277067875\\
62	0.26136	6.67616354879111	6.67616354879111\\
62	0.26502	7.41225434926376	7.41225434926376\\
62	0.26868	8.17390010820549	8.17390010820549\\
62	0.27234	8.96110082561631	8.96110082561631\\
62	0.276	9.77385650149623	9.77385650149623\\
62.375	0.093	0.476197020157185	0.476197020157185\\
62.375	0.09666	0.0309125711032952	0.0309125711032952\\
62.375	0.10032	-0.388816919481499	-0.388816919481499\\
62.375	0.10398	-0.782991451597233	-0.782991451597233\\
62.375	0.10764	-1.15161102524387	-1.15161102524387\\
62.375	0.1113	-1.49467564042143	-1.49467564042143\\
62.375	0.11496	-1.81218529712991	-1.81218529712991\\
62.375	0.11862	-2.10413999536931	-2.10413999536931\\
62.375	0.12228	-2.37053973513964	-2.37053973513964\\
62.375	0.12594	-2.61138451644088	-2.61138451644088\\
62.375	0.1296	-2.82667433927305	-2.82667433927305\\
62.375	0.13326	-3.01640920363613	-3.01640920363613\\
62.375	0.13692	-3.18058910953016	-3.18058910953016\\
62.375	0.14058	-3.31921405695508	-3.31921405695508\\
62.375	0.14424	-3.43228404591094	-3.43228404591094\\
62.375	0.1479	-3.51979907639771	-3.51979907639771\\
62.375	0.15156	-3.5817591484154	-3.5817591484154\\
62.375	0.15522	-3.61816426196401	-3.61816426196401\\
62.375	0.15888	-3.62901441704354	-3.62901441704354\\
62.375	0.16254	-3.61430961365399	-3.61430961365399\\
62.375	0.1662	-3.57404985179538	-3.57404985179538\\
62.375	0.16986	-3.50823513146766	-3.50823513146766\\
62.375	0.17352	-3.41686545267088	-3.41686545267088\\
62.375	0.17718	-3.29994081540504	-3.29994081540504\\
62.375	0.18084	-3.1574612196701	-3.1574612196701\\
62.375	0.1845	-2.98942666546609	-2.98942666546609\\
62.375	0.18816	-2.79583715279298	-2.79583715279298\\
62.375	0.19182	-2.5766926816508	-2.5766926816508\\
62.375	0.19548	-2.33199325203955	-2.33199325203955\\
62.375	0.19914	-2.06173886395921	-2.06173886395921\\
62.375	0.2028	-1.76592951740982	-1.76592951740982\\
62.375	0.20646	-1.44456521239132	-1.44456521239132\\
62.375	0.21012	-1.09764594890373	-1.09764594890373\\
62.375	0.21378	-0.725171726947103	-0.725171726947103\\
62.375	0.21744	-0.327142546521372	-0.327142546521372\\
62.375	0.2211	0.0964415923734379	0.0964415923734379\\
62.375	0.22476	0.545580689737328	0.545580689737328\\
62.375	0.22842	1.0202747455703	1.0202747455703\\
62.375	0.23208	1.52052375987234	1.52052375987234\\
62.375	0.23574	2.04632773264348	2.04632773264348\\
62.375	0.2394	2.59768666388369	2.59768666388369\\
62.375	0.24306	3.17460055359297	3.17460055359297\\
62.375	0.24672	3.7770694017713	3.7770694017713\\
62.375	0.25038	4.40509320841878	4.40509320841878\\
62.375	0.25404	5.05867197353527	5.05867197353527\\
62.375	0.2577	5.73780569712088	5.73780569712088\\
62.375	0.26136	6.44249437917556	6.44249437917556\\
62.375	0.26502	7.17273801969931	7.17273801969931\\
62.375	0.26868	7.92853661869214	7.92853661869214\\
62.375	0.27234	8.70989017615403	8.70989017615403\\
62.375	0.276	9.51679869208505	9.51679869208505\\
62.75	0.093	0.517748275714991	0.517748275714991\\
62.75	0.09666	0.0666166667121875	0.0666166667121875\\
62.75	0.10032	-0.358959983821507	-0.358959983821507\\
62.75	0.10398	-0.758981675886155	-0.758981675886155\\
62.75	0.10764	-1.13344840948169	-1.13344840948169\\
62.75	0.1113	-1.48236018460816	-1.48236018460816\\
62.75	0.11496	-1.80571700126556	-1.80571700126556\\
62.75	0.11862	-2.10351885945386	-2.10351885945386\\
62.75	0.12228	-2.3757657591731	-2.3757657591731\\
62.75	0.12594	-2.62245770042325	-2.62245770042325\\
62.75	0.1296	-2.84359468320432	-2.84359468320432\\
62.75	0.13326	-3.03917670751632	-3.03917670751632\\
62.75	0.13692	-3.20920377335925	-3.20920377335925\\
62.75	0.14058	-3.35367588073308	-3.35367588073308\\
62.75	0.14424	-3.47259302963784	-3.47259302963784\\
62.75	0.1479	-3.56595522007353	-3.56595522007353\\
62.75	0.15156	-3.63376245204012	-3.63376245204012\\
62.75	0.15522	-3.67601472553764	-3.67601472553764\\
62.75	0.15888	-3.6927120405661	-3.6927120405661\\
62.75	0.16254	-3.68385439712545	-3.68385439712545\\
62.75	0.1662	-3.64944179521574	-3.64944179521574\\
62.75	0.16986	-3.58947423483695	-3.58947423483695\\
62.75	0.17352	-3.50395171598907	-3.50395171598907\\
62.75	0.17718	-3.39287423867212	-3.39287423867212\\
62.75	0.18084	-3.25624180288609	-3.25624180288609\\
62.75	0.1845	-3.09405440863097	-3.09405440863097\\
62.75	0.18816	-2.9063120559068	-2.9063120559068\\
62.75	0.19182	-2.69301474471352	-2.69301474471352\\
62.75	0.19548	-2.45416247505119	-2.45416247505119\\
62.75	0.19914	-2.18975524691975	-2.18975524691975\\
62.75	0.2028	-1.89979306031925	-1.89979306031925\\
62.75	0.20646	-1.58427591524966	-1.58427591524966\\
62.75	0.21012	-1.24320381171099	-1.24320381171099\\
62.75	0.21378	-0.876576749703268	-0.876576749703268\\
62.75	0.21744	-0.484394729226437	-0.484394729226437\\
62.75	0.2211	-0.0666577502805268	-0.0666577502805268\\
62.75	0.22476	0.376634187134435	0.376634187134435\\
62.75	0.22842	0.845481083018505	0.845481083018505\\
62.75	0.23208	1.33988293737165	1.33988293737165\\
62.75	0.23574	1.85983975019386	1.85983975019386\\
62.75	0.2394	2.40535152148517	2.40535152148517\\
62.75	0.24306	2.97641825124552	2.97641825124552\\
62.75	0.24672	3.57303993947498	3.57303993947498\\
62.75	0.25038	4.19521658617354	4.19521658617354\\
62.75	0.25404	4.84294819134112	4.84294819134112\\
62.75	0.2577	5.5162347549778	5.5162347549778\\
62.75	0.26136	6.21507627708358	6.21507627708358\\
62.75	0.26502	6.93947275765844	6.93947275765844\\
62.75	0.26868	7.68942419670236	7.68942419670236\\
62.75	0.27234	8.46493059421536	8.46493059421536\\
62.75	0.276	9.26599195019745	9.26599195019745\\
63.125	0.093	0.565550598796404	0.565550598796404\\
63.125	0.09666	0.108571829844701	0.108571829844701\\
63.125	0.10032	-0.322851980637921	-0.322851980637921\\
63.125	0.10398	-0.728720832651469	-0.728720832651469\\
63.125	0.10764	-1.10903472619591	-1.10903472619591\\
63.125	0.1113	-1.46379366127129	-1.46379366127129\\
63.125	0.11496	-1.7929976378776	-1.7929976378776\\
63.125	0.11862	-2.0966466560148	-2.0966466560148\\
63.125	0.12228	-2.37474071568295	-2.37474071568295\\
63.125	0.12594	-2.62727981688201	-2.62727981688201\\
63.125	0.1296	-2.85426395961199	-2.85426395961199\\
63.125	0.13326	-3.05569314387289	-3.05569314387289\\
63.125	0.13692	-3.23156736966473	-3.23156736966473\\
63.125	0.14058	-3.38188663698746	-3.38188663698746\\
63.125	0.14424	-3.50665094584114	-3.50665094584114\\
63.125	0.1479	-3.60586029622574	-3.60586029622574\\
63.125	0.15156	-3.67951468814123	-3.67951468814123\\
63.125	0.15522	-3.72761412158766	-3.72761412158766\\
63.125	0.15888	-3.75015859656502	-3.75015859656502\\
63.125	0.16254	-3.74714811307327	-3.74714811307327\\
63.125	0.1662	-3.71858267111248	-3.71858267111248\\
63.125	0.16986	-3.6644622706826	-3.6644622706826\\
63.125	0.17352	-3.58478691178362	-3.58478691178362\\
63.125	0.17718	-3.47955659441558	-3.47955659441558\\
63.125	0.18084	-3.34877131857846	-3.34877131857846\\
63.125	0.1845	-3.19243108427226	-3.19243108427226\\
63.125	0.18816	-3.01053589149697	-3.01053589149697\\
63.125	0.19182	-2.80308574025262	-2.80308574025262\\
63.125	0.19548	-2.57008063053918	-2.57008063053918\\
63.125	0.19914	-2.31152056235666	-2.31152056235666\\
63.125	0.2028	-2.02740553570506	-2.02740553570506\\
63.125	0.20646	-1.71773555058439	-1.71773555058439\\
63.125	0.21012	-1.38251060699462	-1.38251060699462\\
63.125	0.21378	-1.0217307049358	-1.0217307049358\\
63.125	0.21744	-0.635395844407867	-0.635395844407867\\
63.125	0.2211	-0.223506025410884	-0.223506025410884\\
63.125	0.22476	0.213938752055178	0.213938752055178\\
63.125	0.22842	0.67693848799032	0.67693848799032\\
63.125	0.23208	1.16549318239456	1.16549318239456\\
63.125	0.23574	1.67960283526788	1.67960283526788\\
63.125	0.2394	2.21926744661028	2.21926744661028\\
63.125	0.24306	2.78448701642174	2.78448701642174\\
63.125	0.24672	3.37526154470227	3.37526154470227\\
63.125	0.25038	3.9915910314519	3.9915910314519\\
63.125	0.25404	4.63347547667058	4.63347547667058\\
63.125	0.2577	5.30091488035839	5.30091488035839\\
63.125	0.26136	5.99390924251527	5.99390924251527\\
63.125	0.26502	6.7124585631412	6.7124585631412\\
63.125	0.26868	7.4565628422362	7.4565628422362\\
63.125	0.27234	8.22622207980029	8.22622207980029\\
63.125	0.276	9.02143627583345	9.02143627583345\\
63.5	0.093	0.619603989401424	0.619603989401424\\
63.5	0.09666	0.156778060500821	0.156778060500821\\
63.5	0.10032	-0.2804929099307	-0.2804929099307\\
63.5	0.10398	-0.692208921893162	-0.692208921893162\\
63.5	0.10764	-1.07836997538651	-1.07836997538651\\
63.5	0.1113	-1.4389760704108	-1.4389760704108\\
63.5	0.11496	-1.77402720696601	-1.77402720696601\\
63.5	0.11862	-2.08352338505214	-2.08352338505214\\
63.5	0.12228	-2.36746460466917	-2.36746460466917\\
63.5	0.12594	-2.62585086581716	-2.62585086581716\\
63.5	0.1296	-2.85868216849604	-2.85868216849604\\
63.5	0.13326	-3.06595851270585	-3.06595851270585\\
63.5	0.13692	-3.24767989844659	-3.24767989844659\\
63.5	0.14058	-3.40384632571823	-3.40384632571823\\
63.5	0.14424	-3.53445779452083	-3.53445779452083\\
63.5	0.1479	-3.63951430485431	-3.63951430485431\\
63.5	0.15156	-3.71901585671873	-3.71901585671873\\
63.5	0.15522	-3.77296245011406	-3.77296245011406\\
63.5	0.15888	-3.80135408504032	-3.80135408504032\\
63.5	0.16254	-3.8041907614975	-3.8041907614975\\
63.5	0.1662	-3.7814724794856	-3.7814724794856\\
63.5	0.16986	-3.73319923900463	-3.73319923900463\\
63.5	0.17352	-3.65937104005457	-3.65937104005457\\
63.5	0.17718	-3.55998788263544	-3.55998788263544\\
63.5	0.18084	-3.43504976674723	-3.43504976674723\\
63.5	0.1845	-3.28455669238993	-3.28455669238993\\
63.5	0.18816	-3.10850865956355	-3.10850865956355\\
63.5	0.19182	-2.9069056682681	-2.9069056682681\\
63.5	0.19548	-2.67974771850358	-2.67974771850358\\
63.5	0.19914	-2.42703481026997	-2.42703481026997\\
63.5	0.2028	-2.14876694356727	-2.14876694356727\\
63.5	0.20646	-1.8449441183955	-1.8449441183955\\
63.5	0.21012	-1.51556633475466	-1.51556633475466\\
63.5	0.21378	-1.16063359264474	-1.16063359264474\\
63.5	0.21744	-0.780145892065732	-0.780145892065732\\
63.5	0.2211	-0.374103233017621	-0.374103233017621\\
63.5	0.22476	0.0574943844995133	0.0574943844995133\\
63.5	0.22842	0.514646960485784	0.514646960485784\\
63.5	0.23208	0.997354494941099	0.997354494941099\\
63.5	0.23574	1.50561698786549	1.50561698786549\\
63.5	0.2394	2.03943443925896	2.03943443925896\\
63.5	0.24306	2.59880684912154	2.59880684912154\\
63.5	0.24672	3.18373421745315	3.18373421745315\\
63.5	0.25038	3.79421654425391	3.79421654425391\\
63.5	0.25404	4.43025382952366	4.43025382952366\\
63.5	0.2577	5.09184607326257	5.09184607326257\\
63.5	0.26136	5.7789932754705	5.7789932754705\\
63.5	0.26502	6.49169543614755	6.49169543614755\\
63.5	0.26868	7.22995255529362	7.22995255529362\\
63.5	0.27234	7.99376463290885	7.99376463290885\\
63.5	0.276	8.78313166899308	8.78313166899308\\
63.875	0.093	0.679908447530051	0.679908447530051\\
63.875	0.09666	0.211235358680534	0.211235358680534\\
63.875	0.10032	-0.231882771699887	-0.231882771699887\\
63.875	0.10398	-0.649445943611262	-0.649445943611262\\
63.875	0.10764	-1.04145415705353	-1.04145415705353\\
63.875	0.1113	-1.40790741202671	-1.40790741202671\\
63.875	0.11496	-1.74880570853083	-1.74880570853083\\
63.875	0.11862	-2.06414904656586	-2.06414904656586\\
63.875	0.12228	-2.35393742613181	-2.35393742613181\\
63.875	0.12594	-2.6181708472287	-2.6181708472287\\
63.875	0.1296	-2.85684930985649	-2.85684930985649\\
63.875	0.13326	-3.06997281401521	-3.06997281401521\\
63.875	0.13692	-3.25754135970487	-3.25754135970487\\
63.875	0.14058	-3.41955494692542	-3.41955494692542\\
63.875	0.14424	-3.55601357567692	-3.55601357567692\\
63.875	0.1479	-3.66691724595932	-3.66691724595932\\
63.875	0.15156	-3.75226595777264	-3.75226595777264\\
63.875	0.15522	-3.81205971111687	-3.81205971111687\\
63.875	0.15888	-3.84629850599204	-3.84629850599204\\
63.875	0.16254	-3.85498234239814	-3.85498234239814\\
63.875	0.1662	-3.83811122033514	-3.83811122033514\\
63.875	0.16986	-3.79568513980306	-3.79568513980306\\
63.875	0.17352	-3.72770410080191	-3.72770410080191\\
63.875	0.17718	-3.6341681033317	-3.6341681033317\\
63.875	0.18084	-3.51507714739239	-3.51507714739239\\
63.875	0.1845	-3.37043123298401	-3.37043123298401\\
63.875	0.18816	-3.20023036010654	-3.20023036010654\\
63.875	0.19182	-3.00447452875999	-3.00447452875999\\
63.875	0.19548	-2.78316373894437	-2.78316373894437\\
63.875	0.19914	-2.53629799065965	-2.53629799065965\\
63.875	0.2028	-2.26387728390588	-2.26387728390588\\
63.875	0.20646	-1.96590161868301	-1.96590161868301\\
63.875	0.21012	-1.64237099499108	-1.64237099499108\\
63.875	0.21378	-1.29328541283008	-1.29328541283008\\
63.875	0.21744	-0.918644872199977	-0.918644872199977\\
63.875	0.2211	-0.518449373100793	-0.518449373100793\\
63.875	0.22476	-0.0926989155325302	-0.0926989155325302\\
63.875	0.22842	0.358606500504813	0.358606500504813\\
63.875	0.23208	0.835466875011228	0.835466875011228\\
63.875	0.23574	1.33788220798669	1.33788220798669\\
63.875	0.2394	1.86585249943126	1.86585249943126\\
63.875	0.24306	2.41937774934492	2.41937774934492\\
63.875	0.24672	2.99845795772765	2.99845795772765\\
63.875	0.25038	3.60309312457948	3.60309312457948\\
63.875	0.25404	4.23328324990034	4.23328324990034\\
63.875	0.2577	4.88902833369032	4.88902833369032\\
63.875	0.26136	5.57032837594937	5.57032837594937\\
63.875	0.26502	6.27718337667747	6.27718337667747\\
63.875	0.26868	7.00959333587467	7.00959333587467\\
63.875	0.27234	7.76755825354097	7.76755825354097\\
63.875	0.276	8.55107812967633	8.55107812967633\\
64.25	0.093	0.746463973182271	0.746463973182271\\
64.25	0.09666	0.27194372438384	0.27194372438384\\
64.25	0.10032	-0.177021565945495	-0.177021565945495\\
64.25	0.10398	-0.60043189780577	-0.60043189780577\\
64.25	0.10764	-0.998287271196935	-0.998287271196935\\
64.25	0.1113	-1.37058768611903	-1.37058768611903\\
64.25	0.11496	-1.71733314257207	-1.71733314257207\\
64.25	0.11862	-2.03852364055601	-2.03852364055601\\
64.25	0.12228	-2.33415918007086	-2.33415918007086\\
64.25	0.12594	-2.60423976111666	-2.60423976111666\\
64.25	0.1296	-2.84876538369335	-2.84876538369335\\
64.25	0.13326	-3.06773604780098	-3.06773604780098\\
64.25	0.13692	-3.26115175343955	-3.26115175343955\\
64.25	0.14058	-3.42901250060901	-3.42901250060901\\
64.25	0.14424	-3.57131828930941	-3.57131828930941\\
64.25	0.1479	-3.68806911954071	-3.68806911954071\\
64.25	0.15156	-3.77926499130294	-3.77926499130294\\
64.25	0.15522	-3.8449059045961	-3.8449059045961\\
64.25	0.15888	-3.88499185942017	-3.88499185942017\\
64.25	0.16254	-3.89952285577515	-3.89952285577515\\
64.25	0.1662	-3.88849889366108	-3.88849889366108\\
64.25	0.16986	-3.85191997307792	-3.85191997307792\\
64.25	0.17352	-3.78978609402568	-3.78978609402568\\
64.25	0.17718	-3.70209725650437	-3.70209725650437\\
64.25	0.18084	-3.58885346051396	-3.58885346051396\\
64.25	0.1845	-3.45005470605449	-3.45005470605449\\
64.25	0.18816	-3.28570099312594	-3.28570099312594\\
64.25	0.19182	-3.09579232172829	-3.09579232172829\\
64.25	0.19548	-2.88032869186159	-2.88032869186159\\
64.25	0.19914	-2.63931010352578	-2.63931010352578\\
64.25	0.2028	-2.37273655672091	-2.37273655672091\\
64.25	0.20646	-2.08060805144697	-2.08060805144697\\
64.25	0.21012	-1.76292458770393	-1.76292458770393\\
64.25	0.21378	-1.41968616549183	-1.41968616549183\\
64.25	0.21744	-1.05089278481063	-1.05089278481063\\
64.25	0.2211	-0.656544445660344	-0.656544445660344\\
64.25	0.22476	-0.236641148041009	-0.236641148041009\\
64.25	0.22842	0.208817108047434	0.208817108047434\\
64.25	0.23208	0.679830322604921	0.679830322604921\\
64.25	0.23574	1.17639849563151	1.17639849563151\\
64.25	0.2394	1.69852162712716	1.69852162712716\\
64.25	0.24306	2.24619971709194	2.24619971709194\\
64.25	0.24672	2.81943276552575	2.81943276552575\\
64.25	0.25038	3.41822077242865	3.41822077242865\\
64.25	0.25404	4.04256373780061	4.04256373780061\\
64.25	0.2577	4.69246166164166	4.69246166164166\\
64.25	0.26136	5.36791454395184	5.36791454395184\\
64.25	0.26502	6.06892238473104	6.06892238473104\\
64.25	0.26868	6.79548518397931	6.79548518397931\\
64.25	0.27234	7.54760294169668	7.54760294169668\\
64.25	0.276	8.32527565788314	8.32527565788314\\
64.625	0.093	0.819270566358112	0.819270566358112\\
64.625	0.09666	0.338903157610781	0.338903157610781\\
64.625	0.10032	-0.115909292667453	-0.115909292667453\\
64.625	0.10398	-0.545166784476642	-0.545166784476642\\
64.625	0.10764	-0.948869317816721	-0.948869317816721\\
64.625	0.1113	-1.32701689268772	-1.32701689268772\\
64.625	0.11496	-1.67960950908965	-1.67960950908965\\
64.625	0.11862	-2.00664716702251	-2.00664716702251\\
64.625	0.12228	-2.30812986648627	-2.30812986648627\\
64.625	0.12594	-2.58405760748099	-2.58405760748099\\
64.625	0.1296	-2.83443039000658	-2.83443039000658\\
64.625	0.13326	-3.05924821406312	-3.05924821406312\\
64.625	0.13692	-3.25851107965059	-3.25851107965059\\
64.625	0.14058	-3.43221898676896	-3.43221898676896\\
64.625	0.14424	-3.58037193541826	-3.58037193541826\\
64.625	0.1479	-3.70296992559849	-3.70296992559849\\
64.625	0.15156	-3.80001295730962	-3.80001295730962\\
64.625	0.15522	-3.87150103055168	-3.87150103055168\\
64.625	0.15888	-3.91743414532467	-3.91743414532467\\
64.625	0.16254	-3.93781230162856	-3.93781230162856\\
64.625	0.1662	-3.93263549946339	-3.93263549946339\\
64.625	0.16986	-3.90190373882914	-3.90190373882914\\
64.625	0.17352	-3.8456170197258	-3.8456170197258\\
64.625	0.17718	-3.76377534215339	-3.76377534215339\\
64.625	0.18084	-3.65637870611189	-3.65637870611189\\
64.625	0.1845	-3.52342711160136	-3.52342711160136\\
64.625	0.18816	-3.36492055862168	-3.36492055862168\\
64.625	0.19182	-3.18085904717295	-3.18085904717295\\
64.625	0.19548	-2.97124257725515	-2.97124257725515\\
64.625	0.19914	-2.73607114886827	-2.73607114886827\\
64.625	0.2028	-2.4753447620123	-2.4753447620123\\
64.625	0.20646	-2.18906341668725	-2.18906341668725\\
64.625	0.21012	-1.87722711289312	-1.87722711289312\\
64.625	0.21378	-1.53983585062992	-1.53983585062992\\
64.625	0.21744	-1.17688962989764	-1.17688962989764\\
64.625	0.2211	-0.788388450696289	-0.788388450696289\\
64.625	0.22476	-0.374332313025853	-0.374332313025853\\
64.625	0.22842	0.0652787831136621	0.0652787831136621\\
64.625	0.23208	0.530444837722278	0.530444837722278\\
64.625	0.23574	1.02116585079994	1.02116585079994\\
64.625	0.2394	1.53744182234671	1.53744182234671\\
64.625	0.24306	2.07927275236254	2.07927275236254\\
64.625	0.24672	2.64665864084745	2.64665864084745\\
64.625	0.25038	3.23959948780145	3.23959948780145\\
64.625	0.25404	3.85809529322451	3.85809529322451\\
64.625	0.2577	4.50214605711669	4.50214605711669\\
64.625	0.26136	5.17175177947789	5.17175177947789\\
64.625	0.26502	5.86691246030821	5.86691246030821\\
64.625	0.26868	6.58762809960758	6.58762809960758\\
64.625	0.27234	7.33389869737606	7.33389869737606\\
64.625	0.276	8.10572425361359	8.10572425361359\\
65	0.093	0.898328227057545	0.898328227057545\\
65	0.09666	0.412113658361315	0.412113658361315\\
65	0.10032	-0.0485459518658473	-0.0485459518658473\\
65	0.10398	-0.483650603623936	-0.483650603623936\\
65	0.10764	-0.893200296912928	-0.893200296912928\\
65	0.1113	-1.27719503173284	-1.27719503173284\\
65	0.11496	-1.63563480808367	-1.63563480808367\\
65	0.11862	-1.96851962596543	-1.96851962596543\\
65	0.12228	-2.27584948537811	-2.27584948537811\\
65	0.12594	-2.55762438632172	-2.55762438632172\\
65	0.1296	-2.81384432879624	-2.81384432879624\\
65	0.13326	-3.04450931280167	-3.04450931280167\\
65	0.13692	-3.24961933833806	-3.24961933833806\\
65	0.14058	-3.42917440540533	-3.42917440540533\\
65	0.14424	-3.58317451400355	-3.58317451400355\\
65	0.1479	-3.71161966413266	-3.71161966413266\\
65	0.15156	-3.81450985579271	-3.81450985579271\\
65	0.15522	-3.8918450889837	-3.8918450889837\\
65	0.15888	-3.94362536370558	-3.94362536370558\\
65	0.16254	-3.96985067995839	-3.96985067995839\\
65	0.1662	-3.97052103774212	-3.97052103774212\\
65	0.16986	-3.94563643705677	-3.94563643705677\\
65	0.17352	-3.89519687790234	-3.89519687790234\\
65	0.17718	-3.81920236027885	-3.81920236027885\\
65	0.18084	-3.71765288418628	-3.71765288418628\\
65	0.1845	-3.59054844962461	-3.59054844962461\\
65	0.18816	-3.43788905659386	-3.43788905659386\\
65	0.19182	-3.25967470509404	-3.25967470509404\\
65	0.19548	-3.05590539512514	-3.05590539512514\\
65	0.19914	-2.82658112668715	-2.82658112668715\\
65	0.2028	-2.57170189978011	-2.57170189978011\\
65	0.20646	-2.29126771440396	-2.29126771440396\\
65	0.21012	-1.98527857055876	-1.98527857055876\\
65	0.21378	-1.65373446824446	-1.65373446824446\\
65	0.21744	-1.29663540746108	-1.29663540746108\\
65	0.2211	-0.913981388208626	-0.913981388208626\\
65	0.22476	-0.505772410487118	-0.505772410487118\\
65	0.22842	-0.0720084742965028	-0.0720084742965028\\
65	0.23208	0.387310420363185	0.387310420363185\\
65	0.23574	0.872184273491975	0.872184273491975\\
65	0.2394	1.38261308508982	1.38261308508982\\
65	0.24306	1.91859685515675	1.91859685515675\\
65	0.24672	2.48013558369276	2.48013558369276\\
65	0.25038	3.06722927069786	3.06722927069786\\
65	0.25404	3.67987791617199	3.67987791617199\\
65	0.2577	4.31808152011524	4.31808152011524\\
65	0.26136	4.98184008252757	4.98184008252757\\
65	0.26502	5.67115360340894	5.67115360340894\\
65	0.26868	6.38602208275941	6.38602208275941\\
65	0.27234	7.12644552057898	7.12644552057898\\
65	0.276	7.89242391686761	7.89242391686761\\
65.375	0.093	0.983636955280586	0.983636955280586\\
65.375	0.09666	0.491575226635442	0.491575226635442\\
65.375	0.10032	0.0250684564593797	0.0250684564593797\\
65.375	0.10398	-0.415883355247608	-0.415883355247608\\
65.375	0.10764	-0.831280208485515	-0.831280208485515\\
65.375	0.1113	-1.22112210325433	-1.22112210325433\\
65.375	0.11496	-1.58540903955409	-1.58540903955409\\
65.375	0.11862	-1.92414101738475	-1.92414101738475\\
65.375	0.12228	-2.23731803674632	-2.23731803674632\\
65.375	0.12594	-2.52494009763885	-2.52494009763885\\
65.375	0.1296	-2.78700720006227	-2.78700720006227\\
65.375	0.13326	-3.02351934401662	-3.02351934401662\\
65.375	0.13692	-3.2344765295019	-3.2344765295019\\
65.375	0.14058	-3.41987875651809	-3.41987875651809\\
65.375	0.14424	-3.57972602506521	-3.57972602506521\\
65.375	0.1479	-3.71401833514324	-3.71401833514324\\
65.375	0.15156	-3.8227556867522	-3.8227556867522\\
65.375	0.15522	-3.90593807989208	-3.90593807989208\\
65.375	0.15888	-3.96356551456288	-3.96356551456288\\
65.375	0.16254	-3.99563799076459	-3.99563799076459\\
65.375	0.1662	-4.00215550849724	-4.00215550849724\\
65.375	0.16986	-3.98311806776079	-3.98311806776079\\
65.375	0.17352	-3.93852566855526	-3.93852566855526\\
65.375	0.17718	-3.86837831088069	-3.86837831088069\\
65.375	0.18084	-3.772675994737	-3.772675994737\\
65.375	0.1845	-3.65141872012426	-3.65141872012426\\
65.375	0.18816	-3.50460648704244	-3.50460648704244\\
65.375	0.19182	-3.33223929549151	-3.33223929549151\\
65.375	0.19548	-3.13431714547151	-3.13431714547151\\
65.375	0.19914	-2.91084003698246	-2.91084003698246\\
65.375	0.2028	-2.66180797002431	-2.66180797002431\\
65.375	0.20646	-2.38722094459707	-2.38722094459707\\
65.375	0.21012	-2.08707896070076	-2.08707896070076\\
65.375	0.21378	-1.76138201833539	-1.76138201833539\\
65.375	0.21744	-1.41013011750091	-1.41013011750091\\
65.375	0.2211	-1.03332325819736	-1.03332325819736\\
65.375	0.22476	-0.630961440424748	-0.630961440424748\\
65.375	0.22842	-0.203044664183032	-0.203044664183032\\
65.375	0.23208	0.250427070527756	0.250427070527756\\
65.375	0.23574	0.729453763707617	0.729453763707617\\
65.375	0.2394	1.23403541535654	1.23403541535654\\
65.375	0.24306	1.76417202547459	1.76417202547459\\
65.375	0.24672	2.31986359406167	2.31986359406167\\
65.375	0.25038	2.90111012111785	2.90111012111785\\
65.375	0.25404	3.50791160664308	3.50791160664308\\
65.375	0.2577	4.14026805063743	4.14026805063743\\
65.375	0.26136	4.79817945310086	4.79817945310086\\
65.375	0.26502	5.48164581403333	5.48164581403333\\
65.375	0.26868	6.1906671334349	6.1906671334349\\
65.375	0.27234	6.92524341130557	6.92524341130557\\
65.375	0.276	7.68537464764528	7.68537464764528\\
65.75	0.093	1.07519675102726	1.07519675102726\\
65.75	0.09666	0.577287862433204	0.577287862433204\\
65.75	0.10032	0.104933932308242	0.104933932308242\\
65.75	0.10398	-0.34186503934766	-0.34186503934766\\
65.75	0.10764	-0.763109052534466	-0.763109052534466\\
65.75	0.1113	-1.1587981072522	-1.1587981072522\\
65.75	0.11496	-1.52893220350087	-1.52893220350087\\
65.75	0.11862	-1.87351134128042	-1.87351134128042\\
65.75	0.12228	-2.19253552059093	-2.19253552059093\\
65.75	0.12594	-2.48600474143235	-2.48600474143235\\
65.75	0.1296	-2.75391900380468	-2.75391900380468\\
65.75	0.13326	-2.99627830770794	-2.99627830770794\\
65.75	0.13692	-3.21308265314213	-3.21308265314213\\
65.75	0.14058	-3.40433204010722	-3.40433204010722\\
65.75	0.14424	-3.57002646860325	-3.57002646860325\\
65.75	0.1479	-3.71016593863019	-3.71016593863019\\
65.75	0.15156	-3.82475045018805	-3.82475045018805\\
65.75	0.15522	-3.91378000327684	-3.91378000327684\\
65.75	0.15888	-3.97725459789655	-3.97725459789655\\
65.75	0.16254	-4.01517423404717	-4.01517423404717\\
65.75	0.1662	-4.02753891172873	-4.02753891172873\\
65.75	0.16986	-4.01434863094119	-4.01434863094119\\
65.75	0.17352	-3.97560339168458	-3.97560339168458\\
65.75	0.17718	-3.9113031939589	-3.9113031939589\\
65.75	0.18084	-3.82144803776411	-3.82144803776411\\
65.75	0.1845	-3.7060379231003	-3.7060379231003\\
65.75	0.18816	-3.56507284996735	-3.56507284996735\\
65.75	0.19182	-3.39855281836535	-3.39855281836535\\
65.75	0.19548	-3.20647782829428	-3.20647782829428\\
65.75	0.19914	-2.9888478797541	-2.9888478797541\\
65.75	0.2028	-2.74566297274488	-2.74566297274488\\
65.75	0.20646	-2.47692310726654	-2.47692310726654\\
65.75	0.21012	-2.18262828331913	-2.18262828331913\\
65.75	0.21378	-1.86277850090269	-1.86277850090269\\
65.75	0.21744	-1.51737376001711	-1.51737376001711\\
65.75	0.2211	-1.14641406066245	-1.14641406066245\\
65.75	0.22476	-0.749899402838771	-0.749899402838771\\
65.75	0.22842	-0.327829786545955	-0.327829786545955\\
65.75	0.23208	0.119794788215934	0.119794788215934\\
65.75	0.23574	0.592974321446867	0.592974321446867\\
65.75	0.2394	1.09170881314692	1.09170881314692\\
65.75	0.24306	1.61599826331602	1.61599826331602\\
65.75	0.24672	2.16584267195419	2.16584267195419\\
65.75	0.25038	2.7412420390615	2.7412420390615\\
65.75	0.25404	3.3421963646378	3.3421963646378\\
65.75	0.2577	3.96870564868325	3.96870564868325\\
65.75	0.26136	4.62076989119775	4.62076989119775\\
65.75	0.26502	5.29838909218132	5.29838909218132\\
65.75	0.26868	6.001563251634	6.001563251634\\
65.75	0.27234	6.73029236955571	6.73029236955571\\
65.75	0.276	7.48457644594654	7.48457644594654\\
66.125	0.093	1.1730076142975	1.1730076142975\\
66.125	0.09666	0.66925156575453	0.66925156575453\\
66.125	0.10032	0.191050475680669	0.191050475680669\\
66.125	0.10398	-0.261595655924147	-0.261595655924147\\
66.125	0.10764	-0.688686829059867	-0.688686829059867\\
66.125	0.1113	-1.09022304372649	-1.09022304372649\\
66.125	0.11496	-1.46620429992407	-1.46620429992407\\
66.125	0.11862	-1.81663059765255	-1.81663059765255\\
66.125	0.12228	-2.14150193691194	-2.14150193691194\\
66.125	0.12594	-2.44081831770228	-2.44081831770228\\
66.125	0.1296	-2.71457974002352	-2.71457974002352\\
66.125	0.13326	-2.96278620387568	-2.96278620387568\\
66.125	0.13692	-3.18543770925879	-3.18543770925879\\
66.125	0.14058	-3.38253425617279	-3.38253425617279\\
66.125	0.14424	-3.55407584461773	-3.55407584461773\\
66.125	0.1479	-3.70006247459357	-3.70006247459357\\
66.125	0.15156	-3.82049414610034	-3.82049414610034\\
66.125	0.15522	-3.91537085913805	-3.91537085913805\\
66.125	0.15888	-3.98469261370666	-3.98469261370666\\
66.125	0.16254	-4.02845940980618	-4.02845940980618\\
66.125	0.1662	-4.04667124743665	-4.04667124743665\\
66.125	0.16986	-4.03932812659803	-4.03932812659803\\
66.125	0.17352	-4.00643004729033	-4.00643004729033\\
66.125	0.17718	-3.94797700951355	-3.94797700951355\\
66.125	0.18084	-3.86396901326769	-3.86396901326769\\
66.125	0.1845	-3.75440605855275	-3.75440605855275\\
66.125	0.18816	-3.61928814536873	-3.61928814536873\\
66.125	0.19182	-3.45861527371563	-3.45861527371563\\
66.125	0.19548	-3.27238744359346	-3.27238744359346\\
66.125	0.19914	-3.0606046550022	-3.0606046550022\\
66.125	0.2028	-2.82326690794186	-2.82326690794186\\
66.125	0.20646	-2.56037420241244	-2.56037420241244\\
66.125	0.21012	-2.27192653841396	-2.27192653841396\\
66.125	0.21378	-1.95792391594639	-1.95792391594639\\
66.125	0.21744	-1.61836633500974	-1.61836633500974\\
66.125	0.2211	-1.25325379560401	-1.25325379560401\\
66.125	0.22476	-0.862586297729202	-0.862586297729202\\
66.125	0.22842	-0.446363841385313	-0.446363841385313\\
66.125	0.23208	-0.00458642657235231	-0.00458642657235231\\
66.125	0.23574	0.46274594670971	0.46274594670971\\
66.125	0.2394	0.95563327846083	0.95563327846083\\
66.125	0.24306	1.47407556868103	1.47407556868103\\
66.125	0.24672	2.01807281737031	2.01807281737031\\
66.125	0.25038	2.58762502452868	2.58762502452868\\
66.125	0.25404	3.18273219015609	3.18273219015609\\
66.125	0.2577	3.80339431425264	3.80339431425264\\
66.125	0.26136	4.44961139681824	4.44961139681824\\
66.125	0.26502	5.12138343785291	5.12138343785291\\
66.125	0.26868	5.81871043735666	5.81871043735666\\
66.125	0.27234	6.54159239532947	6.54159239532947\\
66.125	0.276	7.29002931177141	7.29002931177141\\
66.5	0.093	1.27706954509135	1.27706954509135\\
66.5	0.09666	0.767466336599464	0.767466336599464\\
66.5	0.10032	0.283418086576688	0.283418086576688\\
66.5	0.10398	-0.175075204977027	-0.175075204977027\\
66.5	0.10764	-0.608013538061646	-0.608013538061646\\
66.5	0.1113	-1.0153969126772	-1.0153969126772\\
66.5	0.11496	-1.39722532882367	-1.39722532882367\\
66.5	0.11862	-1.75349878650106	-1.75349878650106\\
66.5	0.12228	-2.08421728570938	-2.08421728570938\\
66.5	0.12594	-2.3893808264486	-2.3893808264486\\
66.5	0.1296	-2.66898940871877	-2.66898940871877\\
66.5	0.13326	-2.92304303251983	-2.92304303251983\\
66.5	0.13692	-3.15154169785184	-3.15154169785184\\
66.5	0.14058	-3.35448540471475	-3.35448540471475\\
66.5	0.14424	-3.5318741531086	-3.5318741531086\\
66.5	0.1479	-3.68370794303335	-3.68370794303335\\
66.5	0.15156	-3.80998677448902	-3.80998677448902\\
66.5	0.15522	-3.91071064747564	-3.91071064747564\\
66.5	0.15888	-3.98587956199315	-3.98587956199315\\
66.5	0.16254	-4.03549351804158	-4.03549351804158\\
66.5	0.1662	-4.05955251562097	-4.05955251562097\\
66.5	0.16986	-4.05805655473124	-4.05805655473124\\
66.5	0.17352	-4.03100563537244	-4.03100563537244\\
66.5	0.17718	-3.97839975754459	-3.97839975754459\\
66.5	0.18084	-3.90023892124763	-3.90023892124763\\
66.5	0.1845	-3.7965231264816	-3.7965231264816\\
66.5	0.18816	-3.6672523732465	-3.6672523732465\\
66.5	0.19182	-3.5124266615423	-3.5124266615423\\
66.5	0.19548	-3.33204599136903	-3.33204599136903\\
66.5	0.19914	-3.1261103627267	-3.1261103627267\\
66.5	0.2028	-2.89461977561528	-2.89461977561528\\
66.5	0.20646	-2.63757423003474	-2.63757423003474\\
66.5	0.21012	-2.35497372598518	-2.35497372598518\\
66.5	0.21378	-2.04681826346651	-2.04681826346651\\
66.5	0.21744	-1.71310784247876	-1.71310784247876\\
66.5	0.2211	-1.35384246302196	-1.35384246302196\\
66.5	0.22476	-0.969022125096053	-0.969022125096053\\
66.5	0.22842	-0.558646828701065	-0.558646828701065\\
66.5	0.23208	-0.122716573837003	-0.122716573837003\\
66.5	0.23574	0.338768639496131	0.338768639496131\\
66.5	0.2394	0.825808811298351	0.825808811298351\\
66.5	0.24306	1.33840394156962	1.33840394156962\\
66.5	0.24672	1.87655403031	1.87655403031\\
66.5	0.25038	2.44025907751948	2.44025907751948\\
66.5	0.25404	3.02951908319798	3.02951908319798\\
66.5	0.2577	3.64433404734561	3.64433404734561\\
66.5	0.26136	4.28470396996231	4.28470396996231\\
66.5	0.26502	4.95062885104805	4.95062885104805\\
66.5	0.26868	5.6421086906029	5.6421086906029\\
66.5	0.27234	6.35914348862684	6.35914348862684\\
66.5	0.276	7.10173324511982	7.10173324511982\\
66.875	0.093	1.38738254340883	1.38738254340883\\
66.875	0.09666	0.871932174968046	0.871932174968046\\
66.875	0.10032	0.382036764996357	0.382036764996357\\
66.875	0.10398	-0.0823036865062718	-0.0823036865062718\\
66.875	0.10764	-0.521089179539805	-0.521089179539805\\
66.875	0.1113	-0.934319714104257	-0.934319714104257\\
66.875	0.11496	-1.32199529019965	-1.32199529019965\\
66.875	0.11862	-1.68411590782595	-1.68411590782595\\
66.875	0.12228	-2.02068156698315	-2.02068156698315\\
66.875	0.12594	-2.3316922676713	-2.3316922676713\\
66.875	0.1296	-2.61714800989036	-2.61714800989036\\
66.875	0.13326	-2.87704879364034	-2.87704879364034\\
66.875	0.13692	-3.11139461892126	-3.11139461892126\\
66.875	0.14058	-3.32018548573308	-3.32018548573308\\
66.875	0.14424	-3.50342139407582	-3.50342139407582\\
66.875	0.1479	-3.6611023439495	-3.6611023439495\\
66.875	0.15156	-3.79322833535407	-3.79322833535407\\
66.875	0.15522	-3.89979936828959	-3.89979936828959\\
66.875	0.15888	-3.98081544275603	-3.98081544275603\\
66.875	0.16254	-4.03627655875336	-4.03627655875336\\
66.875	0.1662	-4.06618271628165	-4.06618271628165\\
66.875	0.16986	-4.07053391534082	-4.07053391534082\\
66.875	0.17352	-4.04933015593095	-4.04933015593095\\
66.875	0.17718	-4.00257143805197	-4.00257143805197\\
66.875	0.18084	-3.93025776170394	-3.93025776170394\\
66.875	0.1845	-3.83238912688683	-3.83238912688683\\
66.875	0.18816	-3.7089655336006	-3.7089655336006\\
66.875	0.19182	-3.55998698184533	-3.55998698184533\\
66.875	0.19548	-3.38545347162099	-3.38545347162099\\
66.875	0.19914	-3.18536500292753	-3.18536500292753\\
66.875	0.2028	-2.95972157576504	-2.95972157576504\\
66.875	0.20646	-2.70852319013343	-2.70852319013343\\
66.875	0.21012	-2.43176984603274	-2.43176984603274\\
66.875	0.21378	-2.129461543463	-2.129461543463\\
66.875	0.21744	-1.80159828242418	-1.80159828242418\\
66.875	0.2211	-1.44818006291625	-1.44818006291625\\
66.875	0.22476	-1.06920688493927	-1.06920688493927\\
66.875	0.22842	-0.664678748493181	-0.664678748493181\\
66.875	0.23208	-0.234595653578047	-0.234595653578047\\
66.875	0.23574	0.221042399806215	0.221042399806215\\
66.875	0.2394	0.702235411659508	0.702235411659508\\
66.875	0.24306	1.20898338198188	1.20898338198188\\
66.875	0.24672	1.74128631077336	1.74128631077336\\
66.875	0.25038	2.29914419803391	2.29914419803391\\
66.875	0.25404	2.88255704376351	2.88255704376351\\
66.875	0.2577	3.49152484796224	3.49152484796224\\
66.875	0.26136	4.12604761063004	4.12604761063004\\
66.875	0.26502	4.78612533176685	4.78612533176685\\
66.875	0.26868	5.4717580113728	5.4717580113728\\
66.875	0.27234	6.18294564944782	6.18294564944782\\
66.875	0.276	6.91968824599189	6.91968824599189\\
67.25	0.093	1.50394660924989	1.50394660924989\\
67.25	0.09666	0.982649080860194	0.982649080860194\\
67.25	0.10032	0.486906510939605	0.486906510939605\\
67.25	0.10398	0.0167188994880618	0.0167188994880618\\
67.25	0.10764	-0.427913753494371	-0.427913753494371\\
67.25	0.1113	-0.846991448007737	-0.846991448007737\\
67.25	0.11496	-1.24051418405203	-1.24051418405203\\
67.25	0.11862	-1.60848196162724	-1.60848196162724\\
67.25	0.12228	-1.95089478073336	-1.95089478073336\\
67.25	0.12594	-2.26775264137041	-2.26775264137041\\
67.25	0.1296	-2.55905554353837	-2.55905554353837\\
67.25	0.13326	-2.82480348723726	-2.82480348723726\\
67.25	0.13692	-3.06499647246708	-3.06499647246708\\
67.25	0.14058	-3.27963449922781	-3.27963449922781\\
67.25	0.14424	-3.46871756751947	-3.46871756751947\\
67.25	0.1479	-3.63224567734205	-3.63224567734205\\
67.25	0.15156	-3.77021882869553	-3.77021882869553\\
67.25	0.15522	-3.88263702157997	-3.88263702157997\\
67.25	0.15888	-3.9695002559953	-3.9695002559953\\
67.25	0.16254	-4.03080853194154	-4.03080853194154\\
67.25	0.1662	-4.06656184941872	-4.06656184941872\\
67.25	0.16986	-4.07676020842683	-4.07676020842683\\
67.25	0.17352	-4.06140360896585	-4.06140360896585\\
67.25	0.17718	-4.0204920510358	-4.0204920510358\\
67.25	0.18084	-3.95402553463667	-3.95402553463667\\
67.25	0.1845	-3.86200405976846	-3.86200405976846\\
67.25	0.18816	-3.74442762643116	-3.74442762643116\\
67.25	0.19182	-3.60129623462479	-3.60129623462479\\
67.25	0.19548	-3.43260988434935	-3.43260988434935\\
67.25	0.19914	-3.23836857560482	-3.23836857560482\\
67.25	0.2028	-3.0185723083912	-3.0185723083912\\
67.25	0.20646	-2.77322108270851	-2.77322108270851\\
67.25	0.21012	-2.50231489855673	-2.50231489855673\\
67.25	0.21378	-2.20585375593591	-2.20585375593591\\
67.25	0.21744	-1.88383765484596	-1.88383765484596\\
67.25	0.2211	-1.53626659528696	-1.53626659528696\\
67.25	0.22476	-1.16314057725888	-1.16314057725888\\
67.25	0.22842	-0.76445960076169	-0.76445960076169\\
67.25	0.23208	-0.340223665795456	-0.340223665795456\\
67.25	0.23574	0.10956722763985	0.10956722763985\\
67.25	0.2394	0.584913079544272	0.584913079544272\\
67.25	0.24306	1.08581388991774	1.08581388991774\\
67.25	0.24672	1.6122696587603	1.6122696587603\\
67.25	0.25038	2.16428038607194	2.16428038607194\\
67.25	0.25404	2.74184607185265	2.74184607185265\\
67.25	0.2577	3.34496671610245	3.34496671610245\\
67.25	0.26136	3.97364231882132	3.97364231882132\\
67.25	0.26502	4.62787288000926	4.62787288000926\\
67.25	0.26868	5.30765839966628	5.30765839966628\\
67.25	0.27234	6.0129988777924	6.0129988777924\\
67.25	0.276	6.74389431438757	6.74389431438757\\
67.625	0.093	1.62676174261456	1.62676174261456\\
67.625	0.09666	1.09961705427595	1.09961705427595\\
67.625	0.10032	0.598027324406445	0.598027324406445\\
67.625	0.10398	0.121992553006002	0.121992553006002\\
67.625	0.10764	-0.328487259925344	-0.328487259925344\\
67.625	0.1113	-0.75341211438761	-0.75341211438761\\
67.625	0.11496	-1.15278201038081	-1.15278201038081\\
67.625	0.11862	-1.52659694790493	-1.52659694790493\\
67.625	0.12228	-1.87485692695996	-1.87485692695996\\
67.625	0.12594	-2.19756194754591	-2.19756194754591\\
67.625	0.1296	-2.49471200966278	-2.49471200966278\\
67.625	0.13326	-2.76630711331058	-2.76630711331058\\
67.625	0.13692	-3.01234725848932	-3.01234725848932\\
67.625	0.14058	-3.23283244519895	-3.23283244519895\\
67.625	0.14424	-3.42776267343952	-3.42776267343952\\
67.625	0.1479	-3.597137943211	-3.597137943211\\
67.625	0.15156	-3.7409582545134	-3.7409582545134\\
67.625	0.15522	-3.85922360734673	-3.85922360734673\\
67.625	0.15888	-3.95193400171097	-3.95193400171097\\
67.625	0.16254	-4.01908943760613	-4.01908943760613\\
67.625	0.1662	-4.06068991503222	-4.06068991503222\\
67.625	0.16986	-4.07673543398925	-4.07673543398925\\
67.625	0.17352	-4.06722599447715	-4.06722599447715\\
67.625	0.17718	-4.03216159649602	-4.03216159649602\\
67.625	0.18084	-3.97154224004579	-3.97154224004579\\
67.625	0.1845	-3.88536792512648	-3.88536792512648\\
67.625	0.18816	-3.77363865173811	-3.77363865173811\\
67.625	0.19182	-3.63635441988064	-3.63635441988064\\
67.625	0.19548	-3.4735152295541	-3.4735152295541\\
67.625	0.19914	-3.28512108075846	-3.28512108075846\\
67.625	0.2028	-3.07117197349378	-3.07117197349378\\
67.625	0.20646	-2.83166790775999	-2.83166790775999\\
67.625	0.21012	-2.56660888355713	-2.56660888355713\\
67.625	0.21378	-2.27599490088519	-2.27599490088519\\
67.625	0.21744	-1.95982595974417	-1.95982595974417\\
67.625	0.2211	-1.61810206013406	-1.61810206013406\\
67.625	0.22476	-1.25082320205491	-1.25082320205491\\
67.625	0.22842	-0.85798938550662	-0.85798938550662\\
67.625	0.23208	-0.439600610489315	-0.439600610489315\\
67.625	0.23574	0.00434312299709205	0.00434312299709205\\
67.625	0.2394	0.473841814952614	0.473841814952614\\
67.625	0.24306	0.968895465377159	0.968895465377159\\
67.625	0.24672	1.48950407427081	1.48950407427081\\
67.625	0.25038	2.03566764163356	2.03566764163356\\
67.625	0.25404	2.60738616746534	2.60738616746534\\
67.625	0.2577	3.20465965176626	3.20465965176626\\
67.625	0.26136	3.82748809453621	3.82748809453621\\
67.625	0.26502	4.47587149577525	4.47587149577525\\
67.625	0.26868	5.14980985548337	5.14980985548337\\
67.625	0.27234	5.84930317366056	5.84930317366056\\
67.625	0.276	6.57435145030686	6.57435145030686\\
68	0.093	1.75582794350285	1.75582794350285\\
68	0.09666	1.22283609521535	1.22283609521535\\
68	0.10032	0.715399205396935	0.715399205396935\\
68	0.10398	0.233517274047578	0.233517274047578\\
68	0.10764	-0.222809698832668	-0.222809698832668\\
68	0.1113	-0.653581713243847	-0.653581713243847\\
68	0.11496	-1.05879876918596	-1.05879876918596\\
68	0.11862	-1.43846086665898	-1.43846086665898\\
68	0.12228	-1.79256800566291	-1.79256800566291\\
68	0.12594	-2.12112018619779	-2.12112018619779\\
68	0.1296	-2.42411740826358	-2.42411740826358\\
68	0.13326	-2.70155967186028	-2.70155967186028\\
68	0.13692	-2.95344697698792	-2.95344697698792\\
68	0.14058	-3.17977932364646	-3.17977932364646\\
68	0.14424	-3.38055671183594	-3.38055671183594\\
68	0.1479	-3.55577914155632	-3.55577914155632\\
68	0.15156	-3.70544661280762	-3.70544661280762\\
68	0.15522	-3.82955912558987	-3.82955912558987\\
68	0.15888	-3.928116679903	-3.928116679903\\
68	0.16254	-4.00111927574709	-4.00111927574709\\
68	0.1662	-4.04856691312211	-4.04856691312211\\
68	0.16986	-4.07045959202801	-4.07045959202801\\
68	0.17352	-4.06679731246484	-4.06679731246484\\
68	0.17718	-4.03758007443261	-4.03758007443261\\
68	0.18084	-3.98280787793128	-3.98280787793128\\
68	0.1845	-3.9024807229609	-3.9024807229609\\
68	0.18816	-3.7965986095214	-3.7965986095214\\
68	0.19182	-3.66516153761285	-3.66516153761285\\
68	0.19548	-3.50816950723524	-3.50816950723524\\
68	0.19914	-3.32562251838851	-3.32562251838851\\
68	0.2028	-3.11752057107272	-3.11752057107272\\
68	0.20646	-2.88386366528783	-2.88386366528783\\
68	0.21012	-2.6246518010339	-2.6246518010339\\
68	0.21378	-2.33988497831086	-2.33988497831086\\
68	0.21744	-2.02956319711873	-2.02956319711873\\
68	0.2211	-1.69368645745753	-1.69368645745753\\
68	0.22476	-1.33225475932731	-1.33225475932731\\
68	0.22842	-0.945268102727944	-0.945268102727944\\
68	0.23208	-0.532726487659509	-0.532726487659509\\
68	0.23574	-0.0946299141220024	-0.0946299141220024\\
68	0.2394	0.369021617884563	0.369021617884563\\
68	0.24306	0.858228108360237	0.858228108360237\\
68	0.24672	1.37298955730499	1.37298955730499\\
68	0.25038	1.91330596471881	1.91330596471881\\
68	0.25404	2.47917733060169	2.47917733060169\\
68	0.2577	3.07060365495369	3.07060365495369\\
68	0.26136	3.68758493777476	3.68758493777476\\
68	0.26502	4.33012117906488	4.33012117906488\\
68	0.26868	4.99821237882406	4.99821237882406\\
68	0.27234	5.69185853705238	5.69185853705238\\
68	0.276	6.41105965374976	6.41105965374976\\
68.375	0.093	1.89114521191472	1.89114521191472\\
68.375	0.09666	1.3523062036783	1.3523062036783\\
68.375	0.10032	0.839022153910975	0.839022153910975\\
68.375	0.10398	0.351293062612719	0.351293062612719\\
68.375	0.10764	-0.110881070216442	-0.110881070216442\\
68.375	0.1113	-0.547500244576534	-0.547500244576534\\
68.375	0.11496	-0.958564460467551	-0.958564460467551\\
68.375	0.11862	-1.34407371788946	-1.34407371788946\\
68.375	0.12228	-1.70402801684232	-1.70402801684232\\
68.375	0.12594	-2.0384273573261	-2.0384273573261\\
68.375	0.1296	-2.34727173934079	-2.34727173934079\\
68.375	0.13326	-2.6305611628864	-2.6305611628864\\
68.375	0.13692	-2.88829562796294	-2.88829562796294\\
68.375	0.14058	-3.1204751345704	-3.1204751345704\\
68.375	0.14424	-3.32709968270878	-3.32709968270878\\
68.375	0.1479	-3.50816927237806	-3.50816927237806\\
68.375	0.15156	-3.66368390357829	-3.66368390357829\\
68.375	0.15522	-3.79364357630944	-3.79364357630944\\
68.375	0.15888	-3.8980482905715	-3.8980482905715\\
68.375	0.16254	-3.97689804636449	-3.97689804636449\\
68.375	0.1662	-4.0301928436884	-4.0301928436884\\
68.375	0.16986	-4.0579326825432	-4.0579326825432\\
68.375	0.17352	-4.06011756292896	-4.06011756292896\\
68.375	0.17718	-4.03674748484563	-4.03674748484563\\
68.375	0.18084	-3.9878224482932	-3.9878224482932\\
68.375	0.1845	-3.91334245327172	-3.91334245327172\\
68.375	0.18816	-3.81330749978115	-3.81330749978115\\
68.375	0.19182	-3.6877175878215	-3.6877175878215\\
68.375	0.19548	-3.53657271739279	-3.53657271739279\\
68.375	0.19914	-3.35987288849498	-3.35987288849498\\
68.375	0.2028	-3.15761810112809	-3.15761810112809\\
68.375	0.20646	-2.92980835529213	-2.92980835529213\\
68.375	0.21012	-2.67644365098707	-2.67644365098707\\
68.375	0.21378	-2.39752398821296	-2.39752398821296\\
68.375	0.21744	-2.09304936696974	-2.09304936696974\\
68.375	0.2211	-1.76301978725746	-1.76301978725746\\
68.375	0.22476	-1.40743524907611	-1.40743524907611\\
68.375	0.22842	-1.02629575242567	-1.02629575242567\\
68.375	0.23208	-0.619601297306168	-0.619601297306168\\
68.375	0.23574	-0.187351883717561	-0.187351883717561\\
68.375	0.2394	0.270452488340133	0.270452488340133\\
68.375	0.24306	0.753811818866879	0.753811818866879\\
68.375	0.24672	1.2627261078627	1.2627261078627\\
68.375	0.25038	1.79719535532765	1.79719535532765\\
68.375	0.25404	2.3572195612616	2.3572195612616\\
68.375	0.2577	2.9427987256647	2.9427987256647\\
68.375	0.26136	3.55393284853685	3.55393284853685\\
68.375	0.26502	4.19062192987806	4.19062192987806\\
68.375	0.26868	4.85286596968835	4.85286596968835\\
68.375	0.27234	5.54066496796774	5.54066496796774\\
68.375	0.276	6.25401892471622	6.25401892471622\\
68.75	0.093	2.03271354785021	2.03271354785021\\
68.75	0.09666	1.48802737966488	1.48802737966488\\
68.75	0.10032	0.96889616994865	0.96889616994865\\
68.75	0.10398	0.47531991870148	0.47531991870148\\
68.75	0.10764	0.00729862592342023	0.00729862592342023\\
68.75	0.1113	-0.435167708385572	-0.435167708385572\\
68.75	0.11496	-0.852079084225503	-0.852079084225503\\
68.75	0.11862	-1.24343550159634	-1.24343550159634\\
68.75	0.12228	-1.60923696049809	-1.60923696049809\\
68.75	0.12594	-1.94948346093079	-1.94948346093079\\
68.75	0.1296	-2.26417500289438	-2.26417500289438\\
68.75	0.13326	-2.55331158638889	-2.55331158638889\\
68.75	0.13692	-2.81689321141436	-2.81689321141436\\
68.75	0.14058	-3.05491987797072	-3.05491987797072\\
68.75	0.14424	-3.267391586058	-3.267391586058\\
68.75	0.1479	-3.45430833567621	-3.45430833567621\\
68.75	0.15156	-3.61567012682534	-3.61567012682534\\
68.75	0.15522	-3.75147695950538	-3.75147695950538\\
68.75	0.15888	-3.86172883371637	-3.86172883371637\\
68.75	0.16254	-3.94642574945824	-3.94642574945824\\
68.75	0.1662	-4.00556770673107	-4.00556770673107\\
68.75	0.16986	-4.03915470553478	-4.03915470553478\\
68.75	0.17352	-4.04718674586943	-4.04718674586943\\
68.75	0.17718	-4.02966382773504	-4.02966382773504\\
68.75	0.18084	-3.98658595113153	-3.98658595113153\\
68.75	0.1845	-3.91795311605895	-3.91795311605895\\
68.75	0.18816	-3.82376532251727	-3.82376532251727\\
68.75	0.19182	-3.70402257050653	-3.70402257050653\\
68.75	0.19548	-3.55872486002671	-3.55872486002671\\
68.75	0.19914	-3.38787219107781	-3.38787219107781\\
68.75	0.2028	-3.19146456365985	-3.19146456365985\\
68.75	0.20646	-2.96950197777279	-2.96950197777279\\
68.75	0.21012	-2.72198443341663	-2.72198443341663\\
68.75	0.21378	-2.44891193059144	-2.44891193059144\\
68.75	0.21744	-2.15028446929715	-2.15028446929715\\
68.75	0.2211	-1.82610204953377	-1.82610204953377\\
68.75	0.22476	-1.47636467130132	-1.47636467130132\\
68.75	0.22842	-1.10107233459978	-1.10107233459978\\
68.75	0.23208	-0.700225039429149	-0.700225039429149\\
68.75	0.23574	-0.27382278578947	-0.27382278578947\\
68.75	0.2394	0.178134426319296	0.178134426319296\\
68.75	0.24306	0.655646596897142	0.655646596897142\\
68.75	0.24672	1.15871372594404	1.15871372594404\\
68.75	0.25038	1.68733581346009	1.68733581346009\\
68.75	0.25404	2.24151285944514	2.24151285944514\\
68.75	0.2577	2.82124486389934	2.82124486389934\\
68.75	0.26136	3.42653182682255	3.42653182682255\\
68.75	0.26502	4.05737374821487	4.05737374821487\\
68.75	0.26868	4.71377062807626	4.71377062807626\\
68.75	0.27234	5.39572246640675	5.39572246640675\\
68.75	0.276	6.1032292632063	6.1032292632063\\
69.125	0.093	2.18053295130932	2.18053295130932\\
69.125	0.09666	1.62999962317509	1.62999962317509\\
69.125	0.10032	1.10502125350995	1.10502125350995\\
69.125	0.10398	0.605597842313877	0.605597842313877\\
69.125	0.10764	0.131729389586889	0.131729389586889\\
69.125	0.1113	-0.316584104671003	-0.316584104671003\\
69.125	0.11496	-0.739342640459833	-0.739342640459833\\
69.125	0.11862	-1.13654621777959	-1.13654621777959\\
69.125	0.12228	-1.50819483663024	-1.50819483663024\\
69.125	0.12594	-1.85428849701184	-1.85428849701184\\
69.125	0.1296	-2.17482719892434	-2.17482719892434\\
69.125	0.13326	-2.46981094236778	-2.46981094236778\\
69.125	0.13692	-2.73923972734213	-2.73923972734213\\
69.125	0.14058	-2.98311355384742	-2.98311355384742\\
69.125	0.14424	-3.2014324218836	-3.2014324218836\\
69.125	0.1479	-3.39419633145071	-3.39419633145071\\
69.125	0.15156	-3.56140528254873	-3.56140528254873\\
69.125	0.15522	-3.70305927517771	-3.70305927517771\\
69.125	0.15888	-3.8191583093376	-3.8191583093376\\
69.125	0.16254	-3.90970238502839	-3.90970238502839\\
69.125	0.1662	-3.9746915022501	-3.9746915022501\\
69.125	0.16986	-4.01412566100273	-4.01412566100273\\
69.125	0.17352	-4.02800486128628	-4.02800486128628\\
69.125	0.17718	-4.01632910310079	-4.01632910310079\\
69.125	0.18084	-3.97909838644621	-3.97909838644621\\
69.125	0.1845	-3.91631271132253	-3.91631271132253\\
69.125	0.18816	-3.82797207772975	-3.82797207772975\\
69.125	0.19182	-3.71407648566791	-3.71407648566791\\
69.125	0.19548	-3.57462593513702	-3.57462593513702\\
69.125	0.19914	-3.40962042613705	-3.40962042613705\\
69.125	0.2028	-3.21905995866796	-3.21905995866796\\
69.125	0.20646	-3.00294453272979	-3.00294453272979\\
69.125	0.21012	-2.76127414832256	-2.76127414832256\\
69.125	0.21378	-2.49404880544628	-2.49404880544628\\
69.125	0.21744	-2.20126850410091	-2.20126850410091\\
69.125	0.2211	-1.88293324428643	-1.88293324428643\\
69.125	0.22476	-1.53904302600288	-1.53904302600288\\
69.125	0.22842	-1.16959784925024	-1.16959784925024\\
69.125	0.23208	-0.774597714028538	-0.774597714028538\\
69.125	0.23574	-0.354042620337786	-0.354042620337786\\
69.125	0.2394	0.0920674318220804	0.0920674318220804\\
69.125	0.24306	0.563732442451027	0.563732442451027\\
69.125	0.24672	1.06095241154905	1.06095241154905\\
69.125	0.25038	1.58372733911614	1.58372733911614\\
69.125	0.25404	2.13205722515229	2.13205722515229\\
69.125	0.2577	2.70594206965757	2.70594206965757\\
69.125	0.26136	3.30538187263191	3.30538187263191\\
69.125	0.26502	3.93037663407533	3.93037663407533\\
69.125	0.26868	4.58092635398782	4.58092635398782\\
69.125	0.27234	5.25703103236938	5.25703103236938\\
69.125	0.276	5.95869066922	5.95869066922\\
69.5	0.093	2.33460342229199	2.33460342229199\\
69.5	0.09666	1.77822293420886	1.77822293420886\\
69.5	0.10032	1.24739740459482	1.24739740459482\\
69.5	0.10398	0.742126833449838	0.742126833449838\\
69.5	0.10764	0.26241122077395	0.26241122077395\\
69.5	0.1113	-0.191749433432856	-0.191749433432856\\
69.5	0.11496	-0.6203551291706	-0.6203551291706\\
69.5	0.11862	-1.02340586643925	-1.02340586643925\\
69.5	0.12228	-1.40090164523882	-1.40090164523882\\
69.5	0.12594	-1.75284246556933	-1.75284246556933\\
69.5	0.1296	-2.07922832743075	-2.07922832743075\\
69.5	0.13326	-2.38005923082308	-2.38005923082308\\
69.5	0.13692	-2.65533517574636	-2.65533517574636\\
69.5	0.14058	-2.90505616220051	-2.90505616220051\\
69.5	0.14424	-3.12922219018562	-3.12922219018562\\
69.5	0.1479	-3.32783325970166	-3.32783325970166\\
69.5	0.15156	-3.50088937074858	-3.50088937074858\\
69.5	0.15522	-3.64839052332646	-3.64839052332646\\
69.5	0.15888	-3.77033671743525	-3.77033671743525\\
69.5	0.16254	-3.86672795307494	-3.86672795307494\\
69.5	0.1662	-3.93756423024557	-3.93756423024557\\
69.5	0.16986	-3.98284554894713	-3.98284554894713\\
69.5	0.17352	-4.00257190917959	-4.00257190917959\\
69.5	0.17718	-3.99674331094299	-3.99674331094299\\
69.5	0.18084	-3.96535975423728	-3.96535975423728\\
69.5	0.1845	-3.90842123906253	-3.90842123906253\\
69.5	0.18816	-3.82592776541868	-3.82592776541868\\
69.5	0.19182	-3.71787933330577	-3.71787933330577\\
69.5	0.19548	-3.58427594272378	-3.58427594272378\\
69.5	0.19914	-3.42511759367267	-3.42511759367267\\
69.5	0.2028	-3.24040428615251	-3.24040428615251\\
69.5	0.20646	-3.03013602016328	-3.03013602016328\\
69.5	0.21012	-2.79431279570495	-2.79431279570495\\
69.5	0.21378	-2.53293461277756	-2.53293461277756\\
69.5	0.21744	-2.24600147138109	-2.24600147138109\\
69.5	0.2211	-1.93351337151552	-1.93351337151552\\
69.5	0.22476	-1.59547031318089	-1.59547031318089\\
69.5	0.22842	-1.23187229637715	-1.23187229637715\\
69.5	0.23208	-0.842719321104376	-0.842719321104376\\
69.5	0.23574	-0.428011387362496	-0.428011387362496\\
69.5	0.2394	0.0122515048484431	0.0122515048484431\\
69.5	0.24306	0.47806935552849	0.47806935552849\\
69.5	0.24672	0.969442164677588	0.969442164677588\\
69.5	0.25038	1.48636993229581	1.48636993229581\\
69.5	0.25404	2.02885265838306	2.02885265838306\\
69.5	0.2577	2.5968903429394	2.5968903429394\\
69.5	0.26136	3.19048298596482	3.19048298596482\\
69.5	0.26502	3.80963058745931	3.80963058745931\\
69.5	0.26868	4.45433314742293	4.45433314742293\\
69.5	0.27234	5.12459066585559	5.12459066585559\\
69.5	0.276	5.82040314275731	5.82040314275731\\
69.875	0.093	2.49492496079827	2.49492496079827\\
69.875	0.09666	1.93269731276623	1.93269731276623\\
69.875	0.10032	1.39602462320327	1.39602462320327\\
69.875	0.10398	0.884906892109392	0.884906892109392\\
69.875	0.10764	0.39934411948459	0.39934411948459\\
69.875	0.1113	-0.0606636946711294	-0.0606636946711294\\
69.875	0.11496	-0.495116550357773	-0.495116550357773\\
69.875	0.11862	-0.90401444757534	-0.90401444757534\\
69.875	0.12228	-1.28735738632381	-1.28735738632381\\
69.875	0.12594	-1.64514536660323	-1.64514536660323\\
69.875	0.1296	-1.97737838841355	-1.97737838841355\\
69.875	0.13326	-2.28405645175478	-2.28405645175478\\
69.875	0.13692	-2.56517955662695	-2.56517955662695\\
69.875	0.14058	-2.82074770303007	-2.82074770303007\\
69.875	0.14424	-3.05076089096408	-3.05076089096408\\
69.875	0.1479	-3.25521912042901	-3.25521912042901\\
69.875	0.15156	-3.43412239142484	-3.43412239142484\\
69.875	0.15522	-3.58747070395161	-3.58747070395161\\
69.875	0.15888	-3.7152640580093	-3.7152640580093\\
69.875	0.16254	-3.81750245359792	-3.81750245359792\\
69.875	0.1662	-3.89418589071746	-3.89418589071746\\
69.875	0.16986	-3.94531436936791	-3.94531436936791\\
69.875	0.17352	-3.9708878895493	-3.9708878895493\\
69.875	0.17718	-3.9709064512616	-3.9709064512616\\
69.875	0.18084	-3.94537005450482	-3.94537005450482\\
69.875	0.1845	-3.89427869927896	-3.89427869927896\\
69.875	0.18816	-3.81763238558402	-3.81763238558402\\
69.875	0.19182	-3.71543111342	-3.71543111342\\
69.875	0.19548	-3.58767488278691	-3.58767488278691\\
69.875	0.19914	-3.43436369368474	-3.43436369368474\\
69.875	0.2028	-3.25549754611347	-3.25549754611347\\
69.875	0.20646	-3.05107644007314	-3.05107644007314\\
69.875	0.21012	-2.82110037556374	-2.82110037556374\\
69.875	0.21378	-2.56556935258525	-2.56556935258525\\
69.875	0.21744	-2.28448337113768	-2.28448337113768\\
69.875	0.2211	-1.97784243122101	-1.97784243122101\\
69.875	0.22476	-1.64564653283531	-1.64564653283531\\
69.875	0.22842	-1.28789567598047	-1.28789567598047\\
69.875	0.23208	-0.904589860656593	-0.904589860656593\\
69.875	0.23574	-0.495729086863612	-0.495729086863612\\
69.875	0.2394	-0.0613133546016016	-0.0613133546016016\\
69.875	0.24306	0.398657336129546	0.398657336129546\\
69.875	0.24672	0.884182985329716	0.884182985329716\\
69.875	0.25038	1.39526359299904	1.39526359299904\\
69.875	0.25404	1.93189915913736	1.93189915913736\\
69.875	0.2577	2.49408968374483	2.49408968374483\\
69.875	0.26136	3.08183516682135	3.08183516682135\\
69.875	0.26502	3.69513560836694	3.69513560836694\\
69.875	0.26868	4.3339910083816	4.3339910083816\\
69.875	0.27234	4.99840136686537	4.99840136686537\\
69.875	0.276	5.68836668381819	5.68836668381819\\
70.25	0.093	2.6614975668282	2.6614975668282\\
70.25	0.09666	2.09342275884725	2.09342275884725\\
70.25	0.10032	1.55090290933539	1.55090290933539\\
70.25	0.10398	1.0339380182926	1.0339380182926\\
70.25	0.10764	0.542528085718894	0.542528085718894\\
70.25	0.1113	0.0766731116142605	0.0766731116142605\\
70.25	0.11496	-0.363626904021297	-0.363626904021297\\
70.25	0.11862	-0.778371961187764	-0.778371961187764\\
70.25	0.12228	-1.16756205988515	-1.16756205988515\\
70.25	0.12594	-1.53119720011347	-1.53119720011347\\
70.25	0.1296	-1.8692773818727	-1.8692773818727\\
70.25	0.13326	-2.18180260516283	-2.18180260516283\\
70.25	0.13692	-2.46877286998393	-2.46877286998393\\
70.25	0.14058	-2.73018817633594	-2.73018817633594\\
70.25	0.14424	-2.96604852421886	-2.96604852421886\\
70.25	0.1479	-3.17635391363269	-3.17635391363269\\
70.25	0.15156	-3.36110434457744	-3.36110434457744\\
70.25	0.15522	-3.52029981705311	-3.52029981705311\\
70.25	0.15888	-3.65394033105973	-3.65394033105973\\
70.25	0.16254	-3.76202588659725	-3.76202588659725\\
70.25	0.1662	-3.84455648366569	-3.84455648366569\\
70.25	0.16986	-3.90153212226505	-3.90153212226505\\
70.25	0.17352	-3.93295280239536	-3.93295280239536\\
70.25	0.17718	-3.93881852405656	-3.93881852405656\\
70.25	0.18084	-3.91912928724868	-3.91912928724868\\
70.25	0.1845	-3.87388509197172	-3.87388509197172\\
70.25	0.18816	-3.80308593822571	-3.80308593822571\\
70.25	0.19182	-3.70673182601059	-3.70673182601059\\
70.25	0.19548	-3.5848227553264	-3.5848227553264\\
70.25	0.19914	-3.43735872617312	-3.43735872617312\\
70.25	0.2028	-3.26433973855079	-3.26433973855079\\
70.25	0.20646	-3.06576579245935	-3.06576579245935\\
70.25	0.21012	-2.84163688789885	-2.84163688789885\\
70.25	0.21378	-2.59195302486929	-2.59195302486929\\
70.25	0.21744	-2.31671420337062	-2.31671420337062\\
70.25	0.2211	-2.01592042340287	-2.01592042340287\\
70.25	0.22476	-1.68957168496608	-1.68957168496608\\
70.25	0.22842	-1.33766798806017	-1.33766798806017\\
70.25	0.23208	-0.960209332685189	-0.960209332685189\\
70.25	0.23574	-0.557195718841108	-0.557195718841108\\
70.25	0.2394	-0.128627146527968	-0.128627146527968\\
70.25	0.24306	0.325496384254251	0.325496384254251\\
70.25	0.24672	0.805174873505521	0.805174873505521\\
70.25	0.25038	1.31040832122591	1.31040832122591\\
70.25	0.25404	1.84119672741534	1.84119672741534\\
70.25	0.2577	2.39754009207388	2.39754009207388\\
70.25	0.26136	2.9794384152015	2.9794384152015\\
70.25	0.26502	3.58689169679819	3.58689169679819\\
70.25	0.26868	4.21989993686395	4.21989993686395\\
70.25	0.27234	4.87846313539879	4.87846313539879\\
70.25	0.276	5.56258129240271	5.56258129240271\\
70.625	0.093	2.83432124038171	2.83432124038171\\
70.625	0.09666	2.26039927245184	2.26039927245184\\
70.625	0.10032	1.71203226299107	1.71203226299107\\
70.625	0.10398	1.18922021199938	1.18922021199938\\
70.625	0.10764	0.691963119476762	0.691963119476762\\
70.625	0.1113	0.220260985423229	0.220260985423229\\
70.625	0.11496	-0.225886190161242	-0.225886190161242\\
70.625	0.11862	-0.646478407276609	-0.646478407276609\\
70.625	0.12228	-1.04151566592291	-1.04151566592291\\
70.625	0.12594	-1.41099796610013	-1.41099796610013\\
70.625	0.1296	-1.75492530780829	-1.75492530780829\\
70.625	0.13326	-2.07329769104733	-2.07329769104733\\
70.625	0.13692	-2.36611511581733	-2.36611511581733\\
70.625	0.14058	-2.63337758211821	-2.63337758211821\\
70.625	0.14424	-2.87508508995005	-2.87508508995005\\
70.625	0.1479	-3.09123763931282	-3.09123763931282\\
70.625	0.15156	-3.28183523020647	-3.28183523020647\\
70.625	0.15522	-3.44687786263104	-3.44687786263104\\
70.625	0.15888	-3.58636553658656	-3.58636553658656\\
70.625	0.16254	-3.700298252073	-3.700298252073\\
70.625	0.1662	-3.78867600909034	-3.78867600909034\\
70.625	0.16986	-3.85149880763863	-3.85149880763863\\
70.625	0.17352	-3.88876664771781	-3.88876664771781\\
70.625	0.17718	-3.90047952932794	-3.90047952932794\\
70.625	0.18084	-3.88663745246896	-3.88663745246896\\
70.625	0.1845	-3.84724041714093	-3.84724041714093\\
70.625	0.18816	-3.78228842334379	-3.78228842334379\\
70.625	0.19182	-3.6917814710776	-3.6917814710776\\
70.625	0.19548	-3.57571956034234	-3.57571956034234\\
70.625	0.19914	-3.43410269113796	-3.43410269113796\\
70.625	0.2028	-3.26693086346452	-3.26693086346452\\
70.625	0.20646	-3.07420407732199	-3.07420407732199\\
70.625	0.21012	-2.85592233271041	-2.85592233271041\\
70.625	0.21378	-2.61208562962975	-2.61208562962975\\
70.625	0.21744	-2.34269396807998	-2.34269396807998\\
70.625	0.2211	-2.04774734806114	-2.04774734806114\\
70.625	0.22476	-1.72724576957324	-1.72724576957324\\
70.625	0.22842	-1.38118923261626	-1.38118923261626\\
70.625	0.23208	-1.00957773719015	-1.00957773719015\\
70.625	0.23574	-0.612411283295025	-0.612411283295025\\
70.625	0.2394	-0.189689870930785	-0.189689870930785\\
70.625	0.24306	0.258586499902506	0.258586499902506\\
70.625	0.24672	0.732417829204877	0.732417829204877\\
70.625	0.25038	1.23180411697637	1.23180411697637\\
70.625	0.25404	1.75674536321692	1.75674536321692\\
70.625	0.2577	2.30724156792654	2.30724156792654\\
70.625	0.26136	2.88329273110523	2.88329273110523\\
70.625	0.26502	3.48489885275302	3.48489885275302\\
70.625	0.26868	4.11205993286988	4.11205993286988\\
70.625	0.27234	4.76477597145582	4.76477597145582\\
70.625	0.276	5.44304696851081	5.44304696851081\\
71	0.093	3.01339598145881	3.01339598145881\\
71	0.09666	2.43362685358005	2.43362685358005\\
71	0.10032	1.87941268417036	1.87941268417036\\
71	0.10398	1.35075347322975	1.35075347322975\\
71	0.10764	0.847649220758237	0.847649220758237\\
71	0.1113	0.370099926755804	0.370099926755804\\
71	0.11496	-0.0818944087775808	-0.0818944087775808\\
71	0.11862	-0.508333785841861	-0.508333785841861\\
71	0.12228	-0.909218204437073	-0.909218204437073\\
71	0.12594	-1.28454766456321	-1.28454766456321\\
71	0.1296	-1.63432216622024	-1.63432216622024\\
71	0.13326	-1.95854170940821	-1.95854170940821\\
71	0.13692	-2.25720629412711	-2.25720629412711\\
71	0.14058	-2.53031592037692	-2.53031592037692\\
71	0.14424	-2.77787058815766	-2.77787058815766\\
71	0.1479	-2.99987029746932	-2.99987029746932\\
71	0.15156	-3.19631504831187	-3.19631504831187\\
71	0.15522	-3.36720484068537	-3.36720484068537\\
71	0.15888	-3.51253967458979	-3.51253967458979\\
71	0.16254	-3.63231955002514	-3.63231955002514\\
71	0.1662	-3.7265444669914	-3.7265444669914\\
71	0.16986	-3.79521442548859	-3.79521442548859\\
71	0.17352	-3.83832942551667	-3.83832942551667\\
71	0.17718	-3.8558894670757	-3.8558894670757\\
71	0.18084	-3.84789455016565	-3.84789455016565\\
71	0.1845	-3.81434467478652	-3.81434467478652\\
71	0.18816	-3.7552398409383	-3.7552398409383\\
71	0.19182	-3.67058004862098	-3.67058004862098\\
71	0.19548	-3.56036529783465	-3.56036529783465\\
71	0.19914	-3.42459558857917	-3.42459558857917\\
71	0.2028	-3.26327092085467	-3.26327092085467\\
71	0.20646	-3.07639129466103	-3.07639129466103\\
71	0.21012	-2.86395670999836	-2.86395670999836\\
71	0.21378	-2.62596716686659	-2.62596716686659\\
71	0.21744	-2.36242266526573	-2.36242266526573\\
71	0.2211	-2.07332320519581	-2.07332320519581\\
71	0.22476	-1.75866878665681	-1.75866878665681\\
71	0.22842	-1.41845940964873	-1.41845940964873\\
71	0.23208	-1.05269507417157	-1.05269507417157\\
71	0.23574	-0.66137578022532	-0.66137578022532\\
71	0.2394	-0.244501527810009	-0.244501527810009\\
71	0.24306	0.197927683074411	0.197927683074411\\
71	0.24672	0.665911852427854	0.665911852427854\\
71	0.25038	1.15945098025045	1.15945098025045\\
71	0.25404	1.67854506654204	1.67854506654204\\
71	0.2577	2.22319411130279	2.22319411130279\\
71	0.26136	2.79339811453261	2.79339811453261\\
71	0.26502	3.38915707623144	3.38915707623144\\
71	0.26868	4.01047099639941	4.01047099639941\\
71	0.27234	4.65733987503641	4.65733987503641\\
71	0.276	5.32976371214254	5.32976371214254\\
71.375	0.093	3.19872179005955	3.19872179005955\\
71.375	0.09666	2.61310550223187	2.61310550223187\\
71.375	0.10032	2.05304417287329	2.05304417287329\\
71.375	0.10398	1.51853780198376	1.51853780198376\\
71.375	0.10764	1.00958638956333	1.00958638956333\\
71.375	0.1113	0.526189935611987	0.526189935611987\\
71.375	0.11496	0.0683484401297161	0.0683484401297161\\
71.375	0.11862	-0.363938096883478	-0.363938096883478\\
71.375	0.12228	-0.770669675427589	-0.770669675427589\\
71.375	0.12594	-1.15184629550262	-1.15184629550262\\
71.375	0.1296	-1.50746795710858	-1.50746795710858\\
71.375	0.13326	-1.83753466024546	-1.83753466024546\\
71.375	0.13692	-2.14204640491328	-2.14204640491328\\
71.375	0.14058	-2.42100319111199	-2.42100319111199\\
71.375	0.14424	-2.67440501884163	-2.67440501884163\\
71.375	0.1479	-2.90225188810219	-2.90225188810219\\
71.375	0.15156	-3.10454379889367	-3.10454379889367\\
71.375	0.15522	-3.2812807512161	-3.2812807512161\\
71.375	0.15888	-3.43246274506942	-3.43246274506942\\
71.375	0.16254	-3.55808978045366	-3.55808978045366\\
71.375	0.1662	-3.65816185736883	-3.65816185736883\\
71.375	0.16986	-3.73267897581491	-3.73267897581491\\
71.375	0.17352	-3.78164113579192	-3.78164113579192\\
71.375	0.17718	-3.80504833729985	-3.80504833729985\\
71.375	0.18084	-3.8029005803387	-3.8029005803387\\
71.375	0.1845	-3.77519786490847	-3.77519786490847\\
71.375	0.18816	-3.72194019100918	-3.72194019100918\\
71.375	0.19182	-3.64312755864079	-3.64312755864079\\
71.375	0.19548	-3.53875996780333	-3.53875996780333\\
71.375	0.19914	-3.40883741849678	-3.40883741849678\\
71.375	0.2028	-3.25335991072117	-3.25335991072117\\
71.375	0.20646	-3.07232744447646	-3.07232744447646\\
71.375	0.21012	-2.86574001976269	-2.86574001976269\\
71.375	0.21378	-2.63359763657983	-2.63359763657983\\
71.375	0.21744	-2.37590029492789	-2.37590029492789\\
71.375	0.2211	-2.09264799480687	-2.09264799480687\\
71.375	0.22476	-1.78384073621677	-1.78384073621677\\
71.375	0.22842	-1.44947851915759	-1.44947851915759\\
71.375	0.23208	-1.08956134362933	-1.08956134362933\\
71.375	0.23574	-0.704089209631981	-0.704089209631981\\
71.375	0.2394	-0.293062117165569	-0.293062117165569\\
71.375	0.24306	0.143519933769923	0.143519933769923\\
71.375	0.24672	0.605656943174466	0.605656943174466\\
71.375	0.25038	1.09334891104813	1.09334891104813\\
71.375	0.25404	1.60659583739083	1.60659583739083\\
71.375	0.2577	2.14539772220265	2.14539772220265\\
71.375	0.26136	2.70975456548356	2.70975456548356\\
71.375	0.26502	3.2996663672335	3.2996663672335\\
71.375	0.26868	3.91513312745253	3.91513312745253\\
71.375	0.27234	4.55615484614067	4.55615484614067\\
71.375	0.276	5.22273152329787	5.22273152329787\\
71.75	0.093	3.39029866618386	3.39029866618386\\
71.75	0.09666	2.79883521840727	2.79883521840727\\
71.75	0.10032	2.23292672909979	2.23292672909979\\
71.75	0.10398	1.69257319826135	1.69257319826135\\
71.75	0.10764	1.17777462589202	1.17777462589202\\
71.75	0.1113	0.688531011991762	0.688531011991762\\
71.75	0.11496	0.224842356560577	0.224842356560577\\
71.75	0.11862	-0.213291340401545	-0.213291340401545\\
71.75	0.12228	-0.625870078894527	-0.625870078894527\\
71.75	0.12594	-1.01289385891849	-1.01289385891849\\
71.75	0.1296	-1.37436268047338	-1.37436268047338\\
71.75	0.13326	-1.71027654355915	-1.71027654355915\\
71.75	0.13692	-2.02063544817585	-2.02063544817585\\
71.75	0.14058	-2.30543939432349	-2.30543939432349\\
71.75	0.14424	-2.56468838200205	-2.56468838200205\\
71.75	0.1479	-2.79838241121151	-2.79838241121151\\
71.75	0.15156	-3.00652148195189	-3.00652148195189\\
71.75	0.15522	-3.18910559422322	-3.18910559422322\\
71.75	0.15888	-3.34613474802544	-3.34613474802544\\
71.75	0.16254	-3.47760894335858	-3.47760894335858\\
71.75	0.1662	-3.58352818022268	-3.58352818022268\\
71.75	0.16986	-3.66389245861766	-3.66389245861766\\
71.75	0.17352	-3.71870177854357	-3.71870177854357\\
71.75	0.17718	-3.74795614000043	-3.74795614000043\\
71.75	0.18084	-3.75165554298817	-3.75165554298817\\
71.75	0.1845	-3.72979998750687	-3.72979998750687\\
71.75	0.18816	-3.68238947355646	-3.68238947355646\\
71.75	0.19182	-3.60942400113699	-3.60942400113699\\
71.75	0.19548	-3.51090357024846	-3.51090357024846\\
71.75	0.19914	-3.38682818089078	-3.38682818089078\\
71.75	0.2028	-3.23719783306407	-3.23719783306407\\
71.75	0.20646	-3.06201252676829	-3.06201252676829\\
71.75	0.21012	-2.86127226200342	-2.86127226200342\\
71.75	0.21378	-2.63497703876946	-2.63497703876946\\
71.75	0.21744	-2.38312685706644	-2.38312685706644\\
71.75	0.2211	-2.10572171689432	-2.10572171689432\\
71.75	0.22476	-1.80276161825315	-1.80276161825315\\
71.75	0.22842	-1.47424656114287	-1.47424656114287\\
71.75	0.23208	-1.12017654556352	-1.12017654556352\\
71.75	0.23574	-0.740551571515091	-0.740551571515091\\
71.75	0.2394	-0.335371638997579	-0.335371638997579\\
71.75	0.24306	0.0953632519889851	0.0953632519889851\\
71.75	0.24672	0.551653101444657	0.551653101444657\\
71.75	0.25038	1.03349790936942	1.03349790936942\\
71.75	0.25404	1.54089767576322	1.54089767576322\\
71.75	0.2577	2.07385240062611	2.07385240062611\\
71.75	0.26136	2.6323620839581	2.6323620839581\\
71.75	0.26502	3.21642672575916	3.21642672575916\\
71.75	0.26868	3.82604632602927	3.82604632602927\\
71.75	0.27234	4.46122088476848	4.46122088476848\\
71.75	0.276	5.1219504019768	5.1219504019768\\
72.125	0.093	3.58812660983177	3.58812660983177\\
72.125	0.09666	2.99081600210628	2.99081600210628\\
72.125	0.10032	2.41906035284988	2.41906035284988\\
72.125	0.10398	1.87285966206255	1.87285966206255\\
72.125	0.10764	1.3522139297443	1.3522139297443\\
72.125	0.1113	0.857123155895144	0.857123155895144\\
72.125	0.11496	0.387587340515045	0.387587340515045\\
72.125	0.11862	-0.0563935163959481	-0.0563935163959481\\
72.125	0.12228	-0.474819414837887	-0.474819414837887\\
72.125	0.12594	-0.867690354810748	-0.867690354810748\\
72.125	0.1296	-1.23500633631451	-1.23500633631451\\
72.125	0.13326	-1.57676735934921	-1.57676735934921\\
72.125	0.13692	-1.89297342391484	-1.89297342391484\\
72.125	0.14058	-2.18362453001137	-2.18362453001137\\
72.125	0.14424	-2.44872067763884	-2.44872067763884\\
72.125	0.1479	-2.6882618667972	-2.6882618667972\\
72.125	0.15156	-2.90224809748651	-2.90224809748651\\
72.125	0.15522	-3.09067936970673	-3.09067936970673\\
72.125	0.15888	-3.25355568345785	-3.25355568345785\\
72.125	0.16254	-3.39087703873992	-3.39087703873992\\
72.125	0.1662	-3.50264343555292	-3.50264343555292\\
72.125	0.16986	-3.5888548738968	-3.5888548738968\\
72.125	0.17352	-3.64951135377164	-3.64951135377164\\
72.125	0.17718	-3.68461287517739	-3.68461287517739\\
72.125	0.18084	-3.69415943811404	-3.69415943811404\\
72.125	0.1845	-3.67815104258167	-3.67815104258167\\
72.125	0.18816	-3.63658768858015	-3.63658768858015\\
72.125	0.19182	-3.56946937610959	-3.56946937610959\\
72.125	0.19548	-3.47679610516995	-3.47679610516995\\
72.125	0.19914	-3.3585678757612	-3.3585678757612\\
72.125	0.2028	-3.21478468788342	-3.21478468788342\\
72.125	0.20646	-3.04544654153651	-3.04544654153651\\
72.125	0.21012	-2.85055343672054	-2.85055343672054\\
72.125	0.21378	-2.63010537343553	-2.63010537343553\\
72.125	0.21744	-2.38410235168139	-2.38410235168139\\
72.125	0.2211	-2.11254437145817	-2.11254437145817\\
72.125	0.22476	-1.81543143276593	-1.81543143276593\\
72.125	0.22842	-1.49276353560455	-1.49276353560455\\
72.125	0.23208	-1.14454067997409	-1.14454067997409\\
72.125	0.23574	-0.770762865874595	-0.770762865874595\\
72.125	0.2394	-0.371430093305982	-0.371430093305982\\
72.125	0.24306	0.0534576377317109	0.0534576377317109\\
72.125	0.24672	0.503900327238426	0.503900327238426\\
72.125	0.25038	0.979897975214293	0.979897975214293\\
72.125	0.25404	1.48145058165916	1.48145058165916\\
72.125	0.2577	2.00855814657318	2.00855814657318\\
72.125	0.26136	2.56122066995627	2.56122066995627\\
72.125	0.26502	3.13943815180838	3.13943815180838\\
72.125	0.26868	3.74321059212961	3.74321059212961\\
72.125	0.27234	4.37253799091992	4.37253799091992\\
72.125	0.276	5.02742034817929	5.02742034817929\\
72.5	0.093	3.79220562100332	3.79220562100332\\
72.5	0.09666	3.18904785332891	3.18904785332891\\
72.5	0.10032	2.6114450441236	2.6114450441236\\
72.5	0.10398	2.05939719338736	2.05939719338736\\
72.5	0.10764	1.53290430112021	1.53290430112021\\
72.5	0.1113	1.03196636732213	1.03196636732213\\
72.5	0.11496	0.556583391993135	0.556583391993135\\
72.5	0.11862	0.106755375133213	0.106755375133213\\
72.5	0.12228	-0.317517683257597	-0.317517683257597\\
72.5	0.12594	-0.716235783179386	-0.716235783179386\\
72.5	0.1296	-1.08939892463207	-1.08939892463207\\
72.5	0.13326	-1.43700710761565	-1.43700710761565\\
72.5	0.13692	-1.7590603321302	-1.7590603321302\\
72.5	0.14058	-2.05555859817564	-2.05555859817564\\
72.5	0.14424	-2.326501905752	-2.326501905752\\
72.5	0.1479	-2.57189025485929	-2.57189025485929\\
72.5	0.15156	-2.7917236454975	-2.7917236454975\\
72.5	0.15522	-2.98600207766663	-2.98600207766663\\
72.5	0.15888	-3.15472555136667	-3.15472555136667\\
72.5	0.16254	-3.29789406659764	-3.29789406659764\\
72.5	0.1662	-3.41550762335953	-3.41550762335953\\
72.5	0.16986	-3.50756622165235	-3.50756622165235\\
72.5	0.17352	-3.57406986147608	-3.57406986147608\\
72.5	0.17718	-3.61501854283074	-3.61501854283074\\
72.5	0.18084	-3.63041226571632	-3.63041226571632\\
72.5	0.1845	-3.62025103013281	-3.62025103013281\\
72.5	0.18816	-3.58453483608022	-3.58453483608022\\
72.5	0.19182	-3.52326368355856	-3.52326368355856\\
72.5	0.19548	-3.43643757256783	-3.43643757256783\\
72.5	0.19914	-3.324056503108	-3.324056503108\\
72.5	0.2028	-3.18612047517912	-3.18612047517912\\
72.5	0.20646	-3.02262948878111	-3.02262948878111\\
72.5	0.21012	-2.83358354391407	-2.83358354391407\\
72.5	0.21378	-2.61898264057796	-2.61898264057796\\
72.5	0.21744	-2.37882677877272	-2.37882677877272\\
72.5	0.2211	-2.11311595849843	-2.11311595849843\\
72.5	0.22476	-1.82185017975505	-1.82185017975505\\
72.5	0.22842	-1.5050294425426	-1.5050294425426\\
72.5	0.23208	-1.16265374686107	-1.16265374686107\\
72.5	0.23574	-0.794723092710477	-0.794723092710477\\
72.5	0.2394	-0.401237480090764	-0.401237480090764\\
72.5	0.24306	0.0178030909980009	0.0178030909980009\\
72.5	0.24672	0.462398620555845	0.462398620555845\\
72.5	0.25038	0.932549108582755	0.932549108582755\\
72.5	0.25404	1.42825455507875	1.42825455507875\\
72.5	0.2577	1.94951496004384	1.94951496004384\\
72.5	0.26136	2.49633032347803	2.49633032347803\\
72.5	0.26502	3.06870064538124	3.06870064538124\\
72.5	0.26868	3.66662592575358	3.66662592575358\\
72.5	0.27234	4.29010616459496	4.29010616459496\\
72.5	0.276	4.93914136190543	4.93914136190543\\
72.875	0.093	4.00253569969845	4.00253569969845\\
72.875	0.09666	3.39353077207515	3.39353077207515\\
72.875	0.10032	2.81008080292092	2.81008080292092\\
72.875	0.10398	2.25218579223577	2.25218579223577\\
72.875	0.10764	1.71984574001972	1.71984574001972\\
72.875	0.1113	1.21306064627274	1.21306064627274\\
72.875	0.11496	0.731830510994817	0.731830510994817\\
72.875	0.11862	0.276155334185995	0.276155334185995\\
72.875	0.12228	-0.153964884153742	-0.153964884153742\\
72.875	0.12594	-0.558530144024402	-0.558530144024402\\
72.875	0.1296	-0.937540445426016	-0.937540445426016\\
72.875	0.13326	-1.29099578835852	-1.29099578835852\\
72.875	0.13692	-1.61889617282194	-1.61889617282194\\
72.875	0.14058	-1.92124159881631	-1.92124159881631\\
72.875	0.14424	-2.19803206634158	-2.19803206634158\\
72.875	0.1479	-2.44926757539776	-2.44926757539776\\
72.875	0.15156	-2.6749481259849	-2.6749481259849\\
72.875	0.15522	-2.87507371810293	-2.87507371810293\\
72.875	0.15888	-3.04964435175187	-3.04964435175187\\
72.875	0.16254	-3.19866002693177	-3.19866002693177\\
72.875	0.1662	-3.32212074364256	-3.32212074364256\\
72.875	0.16986	-3.42002650188427	-3.42002650188427\\
72.875	0.17352	-3.49237730165691	-3.49237730165691\\
72.875	0.17718	-3.53917314296049	-3.53917314296049\\
72.875	0.18084	-3.56041402579497	-3.56041402579497\\
72.875	0.1845	-3.55609995016039	-3.55609995016039\\
72.875	0.18816	-3.5262309160567	-3.5262309160567\\
72.875	0.19182	-3.47080692348394	-3.47080692348394\\
72.875	0.19548	-3.38982797244213	-3.38982797244213\\
72.875	0.19914	-3.28329406293118	-3.28329406293118\\
72.875	0.2028	-3.1512051949512	-3.1512051949512\\
72.875	0.20646	-2.99356136850215	-2.99356136850215\\
72.875	0.21012	-2.81036258358397	-2.81036258358397\\
72.875	0.21378	-2.60160884019677	-2.60160884019677\\
72.875	0.21744	-2.36730013834048	-2.36730013834048\\
72.875	0.2211	-2.10743647801506	-2.10743647801506\\
72.875	0.22476	-1.82201785922062	-1.82201785922062\\
72.875	0.22842	-1.51104428195706	-1.51104428195706\\
72.875	0.23208	-1.17451574622444	-1.17451574622444\\
72.875	0.23574	-0.812432252022738	-0.812432252022738\\
72.875	0.2394	-0.424793799351953	-0.424793799351953\\
72.875	0.24306	-0.0116003882120879	-0.0116003882120879\\
72.875	0.24672	0.427147981396828	0.427147981396828\\
72.875	0.25038	0.891451309474867	0.891451309474867\\
72.875	0.25404	1.38130959602196	1.38130959602196\\
72.875	0.2577	1.89672284103813	1.89672284103813\\
72.875	0.26136	2.43769104452339	2.43769104452339\\
72.875	0.26502	3.00421420647773	3.00421420647773\\
72.875	0.26868	3.59629232690111	3.59629232690111\\
72.875	0.27234	4.21392540579362	4.21392540579362\\
72.875	0.276	4.85711344315519	4.85711344315519\\
73.25	0.093	4.21911684591718	4.21911684591718\\
73.25	0.09666	3.60426475834496	3.60426475834496\\
73.25	0.10032	3.01496762924184	3.01496762924184\\
73.25	0.10398	2.45122545860777	2.45122545860777\\
73.25	0.10764	1.91303824644282	1.91303824644282\\
73.25	0.1113	1.40040599274692	1.40040599274692\\
73.25	0.11496	0.91332869752012	0.91332869752012\\
73.25	0.11862	0.451806360762371	0.451806360762371\\
73.25	0.12228	0.0158389824737331	0.0158389824737331\\
73.25	0.12594	-0.394573437345855	-0.394573437345855\\
73.25	0.1296	-0.77943089869634	-0.77943089869634\\
73.25	0.13326	-1.13873340157777	-1.13873340157777\\
73.25	0.13692	-1.47248094599012	-1.47248094599012\\
73.25	0.14058	-1.78067353193336	-1.78067353193336\\
73.25	0.14424	-2.06331115940755	-2.06331115940755\\
73.25	0.1479	-2.32039382841267	-2.32039382841267\\
73.25	0.15156	-2.55192153894868	-2.55192153894868\\
73.25	0.15522	-2.75789429101563	-2.75789429101563\\
73.25	0.15888	-2.9383120846135	-2.9383120846135\\
73.25	0.16254	-3.09317491974227	-3.09317491974227\\
73.25	0.1662	-3.22248279640199	-3.22248279640199\\
73.25	0.16986	-3.32623571459261	-3.32623571459261\\
73.25	0.17352	-3.40443367431417	-3.40443367431417\\
73.25	0.17718	-3.45707667556663	-3.45707667556663\\
73.25	0.18084	-3.48416471835003	-3.48416471835003\\
73.25	0.1845	-3.48569780266435	-3.48569780266435\\
73.25	0.18816	-3.46167592850956	-3.46167592850956\\
73.25	0.19182	-3.4120990958857	-3.4120990958857\\
73.25	0.19548	-3.33696730479282	-3.33696730479282\\
73.25	0.19914	-3.2362805552308	-3.2362805552308\\
73.25	0.2028	-3.11003884719972	-3.11003884719972\\
73.25	0.20646	-2.95824218069956	-2.95824218069956\\
73.25	0.21012	-2.78089055573032	-2.78089055573032\\
73.25	0.21378	-2.57798397229198	-2.57798397229198\\
73.25	0.21744	-2.3495224303846	-2.3495224303846\\
73.25	0.2211	-2.09550593000813	-2.09550593000813\\
73.25	0.22476	-1.81593447116256	-1.81593447116256\\
73.25	0.22842	-1.5108080538479	-1.5108080538479\\
73.25	0.23208	-1.18012667806421	-1.18012667806421\\
73.25	0.23574	-0.823890343811406	-0.823890343811406\\
73.25	0.2394	-0.442099051089521	-0.442099051089521\\
73.25	0.24306	-0.0347527998985839	-0.0347527998985839\\
73.25	0.24672	0.398148409761433	0.398148409761433\\
73.25	0.25038	0.856604577890572	0.856604577890572\\
73.25	0.25404	1.34061570448874	1.34061570448874\\
73.25	0.2577	1.850181789556	1.850181789556\\
73.25	0.26136	2.38530283309237	2.38530283309237\\
73.25	0.26502	2.94597883509778	2.94597883509778\\
73.25	0.26868	3.53220979557229	3.53220979557229\\
73.25	0.27234	4.14399571451587	4.14399571451587\\
73.25	0.276	4.78133659192854	4.78133659192854\\
73.625	0.093	4.44194905965955	4.44194905965955\\
73.625	0.09666	3.82124981213841	3.82124981213841\\
73.625	0.10032	3.22610552308639	3.22610552308639\\
73.625	0.10398	2.65651619250341	2.65651619250341\\
73.625	0.10764	2.11248182038953	2.11248182038953\\
73.625	0.1113	1.59400240674475	1.59400240674475\\
73.625	0.11496	1.101077951569	1.101077951569\\
73.625	0.11862	0.633708454862381	0.633708454862381\\
73.625	0.12228	0.191893916624815	0.191893916624815\\
73.625	0.12594	-0.224365663143672	-0.224365663143672\\
73.625	0.1296	-0.615070284443057	-0.615070284443057\\
73.625	0.13326	-0.980219947273387	-0.980219947273387\\
73.625	0.13692	-1.31981465163464	-1.31981465163464\\
73.625	0.14058	-1.6338543975268	-1.6338543975268\\
73.625	0.14424	-1.9223391849499	-1.9223391849499\\
73.625	0.1479	-2.18526901390391	-2.18526901390391\\
73.625	0.15156	-2.42264388438885	-2.42264388438885\\
73.625	0.15522	-2.6344637964047	-2.6344637964047\\
73.625	0.15888	-2.82072874995147	-2.82072874995147\\
73.625	0.16254	-2.98143874502917	-2.98143874502917\\
73.625	0.1662	-3.11659378163779	-3.11659378163779\\
73.625	0.16986	-3.22619385977733	-3.22619385977733\\
73.625	0.17352	-3.31023897944777	-3.31023897944777\\
73.625	0.17718	-3.36872914064915	-3.36872914064915\\
73.625	0.18084	-3.40166434338146	-3.40166434338146\\
73.625	0.1845	-3.40904458764468	-3.40904458764468\\
73.625	0.18816	-3.39086987343882	-3.39086987343882\\
73.625	0.19182	-3.34714020076388	-3.34714020076388\\
73.625	0.19548	-3.27785556961987	-3.27785556961987\\
73.625	0.19914	-3.18301598000675	-3.18301598000675\\
73.625	0.2028	-3.0626214319246	-3.0626214319246\\
73.625	0.20646	-2.91667192537331	-2.91667192537331\\
73.625	0.21012	-2.745167460353	-2.745167460353\\
73.625	0.21378	-2.54810803686359	-2.54810803686359\\
73.625	0.21744	-2.3254936549051	-2.3254936549051\\
73.625	0.2211	-2.07732431447754	-2.07732431447754\\
73.625	0.22476	-1.80360001558089	-1.80360001558089\\
73.625	0.22842	-1.50432075821517	-1.50432075821517\\
73.625	0.23208	-1.17948654238034	-1.17948654238034\\
73.625	0.23574	-0.829097368076468	-0.829097368076468\\
73.625	0.2394	-0.45315323530351	-0.45315323530351\\
73.625	0.24306	-0.0516541440614446	-0.0516541440614446\\
73.625	0.24672	0.375399905649672	0.375399905649672\\
73.625	0.25038	0.828008913829883	0.828008913829883\\
73.625	0.25404	1.30617288047912	1.30617288047912\\
73.625	0.2577	1.80989180559752	1.80989180559752\\
73.625	0.26136	2.33916568918498	2.33916568918498\\
73.625	0.26502	2.89399453124146	2.89399453124146\\
73.625	0.26868	3.47437833176707	3.47437833176707\\
73.625	0.27234	4.08031709076175	4.08031709076175\\
73.625	0.276	4.71181080822549	4.71181080822549\\
74	0.093	4.67103234092548	4.67103234092548\\
74	0.09666	4.04448593345546	4.04448593345546\\
74	0.10032	3.4434944844545	3.4434944844545\\
74	0.10398	2.86805799392263	2.86805799392263\\
74	0.10764	2.31817646185984	2.31817646185984\\
74	0.1113	1.79384988826614	1.79384988826614\\
74	0.11496	1.29507827314152	1.29507827314152\\
74	0.11862	0.82186161648597	0.82186161648597\\
74	0.12228	0.374199918299505	0.374199918299505\\
74	0.12594	-0.0479068214179108	-0.0479068214179108\\
74	0.1296	-0.444458602666224	-0.444458602666224\\
74	0.13326	-0.815455425445425	-0.815455425445425\\
74	0.13692	-1.16089728975561	-1.16089728975561\\
74	0.14058	-1.48078419559667	-1.48078419559667\\
74	0.14424	-1.77511614296869	-1.77511614296869\\
74	0.1479	-2.04389313187161	-2.04389313187161\\
74	0.15156	-2.28711516230544	-2.28711516230544\\
74	0.15522	-2.50478223427019	-2.50478223427019\\
74	0.15888	-2.69689434776587	-2.69689434776587\\
74	0.16254	-2.86345150279246	-2.86345150279246\\
74	0.1662	-3.00445369935001	-3.00445369935001\\
74	0.16986	-3.11990093743845	-3.11990093743845\\
74	0.17352	-3.20979321705781	-3.20979321705781\\
74	0.17718	-3.2741305382081	-3.2741305382081\\
74	0.18084	-3.3129129008893	-3.3129129008893\\
74	0.1845	-3.32614030510145	-3.32614030510145\\
74	0.18816	-3.31381275084449	-3.31381275084449\\
74	0.19182	-3.27593023811846	-3.27593023811846\\
74	0.19548	-3.21249276692335	-3.21249276692335\\
74	0.19914	-3.12350033725915	-3.12350033725915\\
74	0.2028	-3.0089529491259	-3.0089529491259\\
74	0.20646	-2.86885060252354	-2.86885060252354\\
74	0.21012	-2.70319329745212	-2.70319329745212\\
74	0.21378	-2.51198103391165	-2.51198103391165\\
74	0.21744	-2.29521381190203	-2.29521381190203\\
74	0.2211	-2.05289163142336	-2.05289163142336\\
74	0.22476	-1.78501449247562	-1.78501449247562\\
74	0.22842	-1.49158239505882	-1.49158239505882\\
74	0.23208	-1.17259533917289	-1.17259533917289\\
74	0.23574	-0.828053324817922	-0.828053324817922\\
74	0.2394	-0.457956351993893	-0.457956351993893\\
74	0.24306	-0.0623044207007268	-0.0623044207007268\\
74	0.24672	0.358902469061462	0.358902469061462\\
74	0.25038	0.805664317292774	0.805664317292774\\
74	0.25404	1.27798112399314	1.27798112399314\\
74	0.2577	1.77585288916258	1.77585288916258\\
74	0.26136	2.29927961280114	2.29927961280114\\
74	0.26502	2.84826129490872	2.84826129490872\\
74	0.26868	3.42279793548543	3.42279793548543\\
74	0.27234	4.02288953453119	4.02288953453119\\
74	0.276	4.64853609204603	4.64853609204603\\
};
\end{axis}

\begin{axis}[%
width=2.616cm,
height=2.517cm,
at={(6.484cm,6.993cm)},
scale only axis,
xmin=56,
xmax=74,
tick align=outside,
xlabel style={font=\color{white!15!black}},
xlabel={$L_{cut}$},
ymin=0.093,
ymax=0.276,
ylabel style={font=\color{white!15!black}},
ylabel={$D_{rlx}$},
zmin=-0.529880612550263,
zmax=29.3610203033778,
zlabel style={font=\color{white!15!black}},
zlabel={$x_2$},
view={-140}{50},
axis background/.style={fill=white},
xmajorgrids,
ymajorgrids,
zmajorgrids,
legend style={at={(1.03,1)}, anchor=north west, legend cell align=left, align=left, draw=white!15!black}
]
\addplot3[only marks, mark=*, mark options={}, mark size=1.5000pt, color=mycolor1, fill=mycolor1] table[row sep=crcr]{%
x	y	z\\
74	0.123	4.12707712021966\\
72	0.113	6.67911940066308\\
61	0.095	-0.368517449146112\\
56	0.093	0.154113924565954\\
};
\addlegendentry{data1}

\addplot3[only marks, mark=*, mark options={}, mark size=1.5000pt, color=mycolor2, fill=mycolor2] table[row sep=crcr]{%
x	y	z\\
67	0.276	19.4374970155956\\
66	0.255	17.2396062347327\\
62	0.209	7.55839717566143\\
57	0.193	7.28151958477348\\
};
\addlegendentry{data2}

\addplot3[only marks, mark=*, mark options={}, mark size=1.5000pt, color=black, fill=black] table[row sep=crcr]{%
x	y	z\\
69	0.104	3.79233318389896\\
};
\addlegendentry{data3}

\addplot3[only marks, mark=*, mark options={}, mark size=1.5000pt, color=black, fill=black] table[row sep=crcr]{%
x	y	z\\
64	0.23	11.1762561946187\\
};
\addlegendentry{data4}


\addplot3[%
surf,
fill opacity=0.7, shader=interp, colormap={mymap}{[1pt] rgb(0pt)=(1,0.905882,0); rgb(1pt)=(1,0.901964,0); rgb(2pt)=(1,0.898051,0); rgb(3pt)=(1,0.894144,0); rgb(4pt)=(1,0.890243,0); rgb(5pt)=(1,0.886349,0); rgb(6pt)=(1,0.88246,0); rgb(7pt)=(1,0.878577,0); rgb(8pt)=(1,0.8747,0); rgb(9pt)=(1,0.870829,0); rgb(10pt)=(1,0.866964,0); rgb(11pt)=(1,0.863106,0); rgb(12pt)=(1,0.859253,0); rgb(13pt)=(1,0.855406,0); rgb(14pt)=(1,0.851566,0); rgb(15pt)=(1,0.847732,0); rgb(16pt)=(1,0.843903,0); rgb(17pt)=(1,0.840081,0); rgb(18pt)=(1,0.836265,0); rgb(19pt)=(1,0.832455,0); rgb(20pt)=(1,0.828652,0); rgb(21pt)=(1,0.824854,0); rgb(22pt)=(1,0.821063,0); rgb(23pt)=(1,0.817278,0); rgb(24pt)=(1,0.8135,0); rgb(25pt)=(1,0.809727,0); rgb(26pt)=(1,0.805961,0); rgb(27pt)=(1,0.8022,0); rgb(28pt)=(1,0.798445,0); rgb(29pt)=(1,0.794696,0); rgb(30pt)=(1,0.790953,0); rgb(31pt)=(1,0.787215,0); rgb(32pt)=(1,0.783484,0); rgb(33pt)=(1,0.779758,0); rgb(34pt)=(1,0.776038,0); rgb(35pt)=(1,0.772324,0); rgb(36pt)=(1,0.768615,0); rgb(37pt)=(1,0.764913,0); rgb(38pt)=(1,0.761217,0); rgb(39pt)=(1,0.757527,0); rgb(40pt)=(1,0.753843,0); rgb(41pt)=(1,0.750165,0); rgb(42pt)=(1,0.746493,0); rgb(43pt)=(1,0.742827,0); rgb(44pt)=(1,0.739167,0); rgb(45pt)=(1,0.735514,0); rgb(46pt)=(1,0.731867,0); rgb(47pt)=(1,0.728226,0); rgb(48pt)=(1,0.724591,0); rgb(49pt)=(1,0.720963,0); rgb(50pt)=(1,0.717341,0); rgb(51pt)=(1,0.713725,0); rgb(52pt)=(0.999994,0.710077,0); rgb(53pt)=(0.999974,0.706363,0); rgb(54pt)=(0.999942,0.702592,0); rgb(55pt)=(0.999898,0.698775,0); rgb(56pt)=(0.999841,0.694921,0); rgb(57pt)=(0.999771,0.691039,0); rgb(58pt)=(0.99969,0.687139,0); rgb(59pt)=(0.999596,0.68323,0); rgb(60pt)=(0.99949,0.679323,0); rgb(61pt)=(0.999372,0.675427,0); rgb(62pt)=(0.999242,0.67155,0); rgb(63pt)=(0.9991,0.667704,0); rgb(64pt)=(0.998946,0.663897,0); rgb(65pt)=(0.998781,0.660138,0); rgb(66pt)=(0.998605,0.656439,0); rgb(67pt)=(0.998416,0.652807,0); rgb(68pt)=(0.998217,0.649253,0); rgb(69pt)=(0.998006,0.645786,0); rgb(70pt)=(0.997785,0.642416,0); rgb(71pt)=(0.997552,0.639152,0); rgb(72pt)=(0.997308,0.636004,0); rgb(73pt)=(0.997053,0.632982,0); rgb(74pt)=(0.996788,0.630095,0); rgb(75pt)=(0.996512,0.627352,0); rgb(76pt)=(0.996226,0.624763,0); rgb(77pt)=(0.995851,0.622329,0); rgb(78pt)=(0.99494,0.619997,0); rgb(79pt)=(0.99345,0.617753,0); rgb(80pt)=(0.991419,0.61559,0); rgb(81pt)=(0.988885,0.613503,0); rgb(82pt)=(0.985886,0.611486,0); rgb(83pt)=(0.98246,0.609532,0); rgb(84pt)=(0.978643,0.607636,0); rgb(85pt)=(0.974475,0.605791,0); rgb(86pt)=(0.969992,0.603992,0); rgb(87pt)=(0.965232,0.602233,0); rgb(88pt)=(0.960233,0.600507,0); rgb(89pt)=(0.955033,0.598808,0); rgb(90pt)=(0.949669,0.59713,0); rgb(91pt)=(0.94418,0.595468,0); rgb(92pt)=(0.938602,0.593815,0); rgb(93pt)=(0.932974,0.592166,0); rgb(94pt)=(0.927333,0.590513,0); rgb(95pt)=(0.921717,0.588852,0); rgb(96pt)=(0.916164,0.587176,0); rgb(97pt)=(0.910711,0.585479,0); rgb(98pt)=(0.905397,0.583755,0); rgb(99pt)=(0.900258,0.581999,0); rgb(100pt)=(0.895333,0.580203,0); rgb(101pt)=(0.890659,0.578362,0); rgb(102pt)=(0.886275,0.576471,0); rgb(103pt)=(0.882047,0.574545,0); rgb(104pt)=(0.877819,0.572608,0); rgb(105pt)=(0.873592,0.57066,0); rgb(106pt)=(0.869366,0.568701,0); rgb(107pt)=(0.865143,0.566733,0); rgb(108pt)=(0.860924,0.564756,0); rgb(109pt)=(0.856708,0.562771,0); rgb(110pt)=(0.852497,0.560778,0); rgb(111pt)=(0.848292,0.558779,0); rgb(112pt)=(0.844092,0.556774,0); rgb(113pt)=(0.8399,0.554763,0); rgb(114pt)=(0.835716,0.552749,0); rgb(115pt)=(0.831541,0.55073,0); rgb(116pt)=(0.827374,0.548709,0); rgb(117pt)=(0.823219,0.546686,0); rgb(118pt)=(0.819074,0.54466,0); rgb(119pt)=(0.81494,0.542635,0); rgb(120pt)=(0.81082,0.540609,0); rgb(121pt)=(0.806712,0.538584,0); rgb(122pt)=(0.802619,0.53656,0); rgb(123pt)=(0.798541,0.534539,0); rgb(124pt)=(0.794478,0.532521,0); rgb(125pt)=(0.790431,0.530506,0); rgb(126pt)=(0.786402,0.528496,0); rgb(127pt)=(0.782391,0.526491,0); rgb(128pt)=(0.77841,0.524489,0); rgb(129pt)=(0.774523,0.522478,0); rgb(130pt)=(0.770731,0.520455,0); rgb(131pt)=(0.767022,0.518424,0); rgb(132pt)=(0.763384,0.516385,0); rgb(133pt)=(0.759804,0.514339,0); rgb(134pt)=(0.756272,0.51229,0); rgb(135pt)=(0.752775,0.510237,0); rgb(136pt)=(0.749302,0.508182,0); rgb(137pt)=(0.74584,0.506128,0); rgb(138pt)=(0.742378,0.504075,0); rgb(139pt)=(0.738904,0.502025,0); rgb(140pt)=(0.735406,0.499979,0); rgb(141pt)=(0.731872,0.49794,0); rgb(142pt)=(0.72829,0.495909,0); rgb(143pt)=(0.724649,0.493887,0); rgb(144pt)=(0.720936,0.491875,0); rgb(145pt)=(0.71714,0.489876,0); rgb(146pt)=(0.713249,0.487891,0); rgb(147pt)=(0.709251,0.485921,0); rgb(148pt)=(0.705134,0.483968,0); rgb(149pt)=(0.700887,0.482033,0); rgb(150pt)=(0.696497,0.480118,0); rgb(151pt)=(0.691952,0.478225,0); rgb(152pt)=(0.687242,0.476355,0); rgb(153pt)=(0.682353,0.47451,0); rgb(154pt)=(0.677195,0.472696,0); rgb(155pt)=(0.6717,0.470916,0); rgb(156pt)=(0.665891,0.469169,0); rgb(157pt)=(0.659791,0.46745,0); rgb(158pt)=(0.653423,0.465756,0); rgb(159pt)=(0.64681,0.464084,0); rgb(160pt)=(0.639976,0.462432,0); rgb(161pt)=(0.632943,0.460795,0); rgb(162pt)=(0.625734,0.459171,0); rgb(163pt)=(0.618373,0.457556,0); rgb(164pt)=(0.610882,0.455948,0); rgb(165pt)=(0.603284,0.454343,0); rgb(166pt)=(0.595604,0.452737,0); rgb(167pt)=(0.587863,0.451129,0); rgb(168pt)=(0.580084,0.449514,0); rgb(169pt)=(0.572292,0.447889,0); rgb(170pt)=(0.564508,0.446252,0); rgb(171pt)=(0.556756,0.444599,0); rgb(172pt)=(0.549059,0.442927,0); rgb(173pt)=(0.54144,0.441232,0); rgb(174pt)=(0.533922,0.439512,0); rgb(175pt)=(0.526529,0.437764,0); rgb(176pt)=(0.519282,0.435983,0); rgb(177pt)=(0.512206,0.434168,0); rgb(178pt)=(0.505323,0.432315,0); rgb(179pt)=(0.498628,0.430422,3.92506e-06); rgb(180pt)=(0.491973,0.428504,3.49981e-05); rgb(181pt)=(0.485331,0.426562,9.63073e-05); rgb(182pt)=(0.478704,0.424596,0.000186979); rgb(183pt)=(0.472096,0.422609,0.000306141); rgb(184pt)=(0.465508,0.420599,0.00045292); rgb(185pt)=(0.458942,0.418567,0.000626441); rgb(186pt)=(0.452401,0.416515,0.000825833); rgb(187pt)=(0.445885,0.414441,0.00105022); rgb(188pt)=(0.439399,0.412348,0.00129873); rgb(189pt)=(0.432942,0.410234,0.00157049); rgb(190pt)=(0.426518,0.408102,0.00186463); rgb(191pt)=(0.420129,0.40595,0.00218028); rgb(192pt)=(0.413777,0.40378,0.00251655); rgb(193pt)=(0.407464,0.401592,0.00287258); rgb(194pt)=(0.401191,0.399386,0.00324749); rgb(195pt)=(0.394962,0.397164,0.00364042); rgb(196pt)=(0.388777,0.394925,0.00405048); rgb(197pt)=(0.38264,0.39267,0.00447681); rgb(198pt)=(0.376552,0.390399,0.00491852); rgb(199pt)=(0.370516,0.388113,0.00537476); rgb(200pt)=(0.364532,0.385812,0.00584464); rgb(201pt)=(0.358605,0.383497,0.00632729); rgb(202pt)=(0.352735,0.381168,0.00682184); rgb(203pt)=(0.346925,0.378826,0.00732741); rgb(204pt)=(0.341176,0.376471,0.00784314); rgb(205pt)=(0.335485,0.374093,0.00847245); rgb(206pt)=(0.329843,0.371682,0.00930909); rgb(207pt)=(0.324249,0.369242,0.0103377); rgb(208pt)=(0.318701,0.366772,0.0115428); rgb(209pt)=(0.313198,0.364275,0.0129091); rgb(210pt)=(0.307739,0.361753,0.0144211); rgb(211pt)=(0.302322,0.359206,0.0160634); rgb(212pt)=(0.296945,0.356637,0.0178207); rgb(213pt)=(0.291607,0.354048,0.0196776); rgb(214pt)=(0.286307,0.35144,0.0216186); rgb(215pt)=(0.281043,0.348814,0.0236284); rgb(216pt)=(0.275813,0.346172,0.0256916); rgb(217pt)=(0.270616,0.343517,0.0277927); rgb(218pt)=(0.265451,0.340849,0.0299163); rgb(219pt)=(0.260317,0.33817,0.0320472); rgb(220pt)=(0.25521,0.335482,0.0341698); rgb(221pt)=(0.250131,0.332786,0.0362688); rgb(222pt)=(0.245078,0.330085,0.0383287); rgb(223pt)=(0.240048,0.327379,0.0403343); rgb(224pt)=(0.235042,0.324671,0.04227); rgb(225pt)=(0.230056,0.321962,0.0441205); rgb(226pt)=(0.22509,0.319254,0.0458704); rgb(227pt)=(0.220142,0.316548,0.0475043); rgb(228pt)=(0.215212,0.313846,0.0490067); rgb(229pt)=(0.210296,0.311149,0.0503624); rgb(230pt)=(0.205395,0.308459,0.0515759); rgb(231pt)=(0.200514,0.305763,0.052757); rgb(232pt)=(0.195655,0.303061,0.0539242); rgb(233pt)=(0.190817,0.300353,0.0550763); rgb(234pt)=(0.186001,0.297639,0.0562123); rgb(235pt)=(0.181207,0.294918,0.0573313); rgb(236pt)=(0.176434,0.292191,0.0584321); rgb(237pt)=(0.171685,0.289458,0.0595136); rgb(238pt)=(0.166957,0.286719,0.060575); rgb(239pt)=(0.162252,0.283973,0.0616151); rgb(240pt)=(0.15757,0.281221,0.0626328); rgb(241pt)=(0.152911,0.278463,0.0636271); rgb(242pt)=(0.148275,0.275699,0.0645971); rgb(243pt)=(0.143663,0.272929,0.0655416); rgb(244pt)=(0.139074,0.270152,0.0664596); rgb(245pt)=(0.134508,0.26737,0.06735); rgb(246pt)=(0.129967,0.264581,0.0682118); rgb(247pt)=(0.125449,0.261787,0.0690441); rgb(248pt)=(0.120956,0.258986,0.0698456); rgb(249pt)=(0.116487,0.25618,0.0706154); rgb(250pt)=(0.112043,0.253367,0.0713525); rgb(251pt)=(0.107623,0.250549,0.0720557); rgb(252pt)=(0.103229,0.247724,0.0727241); rgb(253pt)=(0.0988592,0.244894,0.0733566); rgb(254pt)=(0.0945149,0.242058,0.0739522); rgb(255pt)=(0.0901961,0.239216,0.0745098)}, mesh/rows=49]
table[row sep=crcr, point meta=\thisrow{c}] {%
%
x	y	z	c\\
56	0.093	-0.482085322428722	-0.482085322428722\\
56	0.09666	-0.518915462439013	-0.518915462439013\\
56	0.10032	-0.529880612550263	-0.529880612550263\\
56	0.10398	-0.514980772762495	-0.514980772762495\\
56	0.10764	-0.474215943075659	-0.474215943075659\\
56	0.1113	-0.407586123489812	-0.407586123489812\\
56	0.11496	-0.315091314004928	-0.315091314004928\\
56	0.11862	-0.19673151462098	-0.19673151462098\\
56	0.12228	-0.05250672533802	-0.05250672533802\\
56	0.12594	0.11758305384399	0.11758305384399\\
56	0.1296	0.313537822925035	0.313537822925035\\
56	0.13326	0.535357581905107	0.535357581905107\\
56	0.13692	0.78304233078423	0.78304233078423\\
56	0.14058	1.0565920695624	1.0565920695624\\
56	0.14424	1.3560067982396	1.3560067982396\\
56	0.1479	1.68128651681584	1.68128651681584\\
56	0.15156	2.0324312252911	2.0324312252911\\
56	0.15522	2.40944092366541	2.40944092366541\\
56	0.15888	2.81231561193877	2.81231561193877\\
56	0.16254	3.24105529011115	3.24105529011115\\
56	0.1662	3.69565995818257	3.69565995818257\\
56	0.16986	4.17612961615303	4.17612961615303\\
56	0.17352	4.68246426402252	4.68246426402252\\
56	0.17718	5.21466390179106	5.21466390179106\\
56	0.18084	5.77272852945864	5.77272852945864\\
56	0.1845	6.35665814702524	6.35665814702524\\
56	0.18816	6.96645275449091	6.96645275449091\\
56	0.19182	7.60211235185559	7.60211235185559\\
56	0.19548	8.26363693911933	8.26363693911933\\
56	0.19914	8.95102651628208	8.95102651628208\\
56	0.2028	9.66428108334389	9.66428108334389\\
56	0.20646	10.4034006403047	10.4034006403047\\
56	0.21012	11.1683851871646	11.1683851871646\\
56	0.21378	11.9592347239235	11.9592347239235\\
56	0.21744	12.7759492505815	12.7759492505815\\
56	0.2211	13.6185287671385	13.6185287671385\\
56	0.22476	14.4869732735945	14.4869732735945\\
56	0.22842	15.3812827699496	15.3812827699496\\
56	0.23208	16.3014572562036	16.3014572562036\\
56	0.23574	17.2474967323568	17.2474967323568\\
56	0.2394	18.219401198409	18.219401198409\\
56	0.24306	19.2171706543602	19.2171706543602\\
56	0.24672	20.2408051002104	20.2408051002104\\
56	0.25038	21.2903045359598	21.2903045359598\\
56	0.25404	22.3656689616081	22.3656689616081\\
56	0.2577	23.4668983771554	23.4668983771554\\
56	0.26136	24.5939927826019	24.5939927826019\\
56	0.26502	25.7469521779473	25.7469521779473\\
56	0.26868	26.9257765631918	26.9257765631918\\
56	0.27234	28.1304659383353	28.1304659383353\\
56	0.276	29.3610203033778	29.3610203033778\\
56.375	0.093	-0.422721158020499	-0.422721158020499\\
56.375	0.09666	-0.468371136194337	-0.468371136194337\\
56.375	0.10032	-0.488156124469134	-0.488156124469134\\
56.375	0.10398	-0.482076122844898	-0.482076122844898\\
56.375	0.10764	-0.450131131321623	-0.450131131321623\\
56.375	0.1113	-0.392321149899308	-0.392321149899308\\
56.375	0.11496	-0.308646178577956	-0.308646178577956\\
56.375	0.11862	-0.199106217357569	-0.199106217357569\\
56.375	0.12228	-0.0637012662381551	-0.0637012662381551\\
56.375	0.12594	0.0975686747803088	0.0975686747803088\\
56.375	0.1296	0.284703605697807	0.284703605697807\\
56.375	0.13326	0.497703526514346	0.497703526514346\\
56.375	0.13692	0.736568437229923	0.736568437229923\\
56.375	0.14058	1.00129833784453	1.00129833784453\\
56.375	0.14424	1.29189322835819	1.29189322835819\\
56.375	0.1479	1.60835310877088	1.60835310877088\\
56.375	0.15156	1.95067797908261	1.95067797908261\\
56.375	0.15522	2.31886783929336	2.31886783929336\\
56.375	0.15888	2.71292268940317	2.71292268940317\\
56.375	0.16254	3.13284252941201	3.13284252941201\\
56.375	0.1662	3.57862735931989	3.57862735931989\\
56.375	0.16986	4.05027717912679	4.05027717912679\\
56.375	0.17352	4.54779198883275	4.54779198883275\\
56.375	0.17718	5.07117178843774	5.07117178843774\\
56.375	0.18084	5.62041657794177	5.62041657794177\\
56.375	0.1845	6.19552635734485	6.19552635734485\\
56.375	0.18816	6.79650112664695	6.79650112664695\\
56.375	0.19182	7.4233408858481	7.4233408858481\\
56.375	0.19548	8.07604563494828	8.07604563494828\\
56.375	0.19914	8.75461537394749	8.75461537394749\\
56.375	0.2028	9.45905010284574	9.45905010284574\\
56.375	0.20646	10.189349821643	10.189349821643\\
56.375	0.21012	10.9455145303394	10.9455145303394\\
56.375	0.21378	11.7275442289347	11.7275442289347\\
56.375	0.21744	12.5354389174291	12.5354389174291\\
56.375	0.2211	13.3691985958226	13.3691985958226\\
56.375	0.22476	14.2288232641151	14.2288232641151\\
56.375	0.22842	15.1143129223066	15.1143129223066\\
56.375	0.23208	16.0256675703971	16.0256675703971\\
56.375	0.23574	16.9628872083867	16.9628872083867\\
56.375	0.2394	17.9259718362754	17.9259718362754\\
56.375	0.24306	18.914921454063	18.914921454063\\
56.375	0.24672	19.9297360617498	19.9297360617498\\
56.375	0.25038	20.9704156593355	20.9704156593355\\
56.375	0.25404	22.0369602468203	22.0369602468203\\
56.375	0.2577	23.1293698242041	23.1293698242041\\
56.375	0.26136	24.247644391487	24.247644391487\\
56.375	0.26502	25.3917839486689	25.3917839486689\\
56.375	0.26868	26.5617884957498	26.5617884957498\\
56.375	0.27234	27.7576580327298	27.7576580327298\\
56.375	0.276	28.9793925596088	28.9793925596088\\
56.75	0.093	-0.357927968544116	-0.357927968544116\\
56.75	0.09666	-0.412397784881501	-0.412397784881501\\
56.75	0.10032	-0.441002611319844	-0.441002611319844\\
56.75	0.10398	-0.443742447859155	-0.443742447859155\\
56.75	0.10764	-0.420617294499426	-0.420617294499426\\
56.75	0.1113	-0.371627151240657	-0.371627151240657\\
56.75	0.11496	-0.296772018082866	-0.296772018082866\\
56.75	0.11862	-0.196051895026025	-0.196051895026025\\
56.75	0.12228	-0.069466782070144	-0.069466782070144\\
56.75	0.12594	0.0829833207847734	0.0829833207847734\\
56.75	0.1296	0.261298413538725	0.261298413538725\\
56.75	0.13326	0.465478496191718	0.465478496191718\\
56.75	0.13692	0.695523568743749	0.695523568743749\\
56.75	0.14058	0.95143363119481	0.95143363119481\\
56.75	0.14424	1.23320868354492	1.23320868354492\\
56.75	0.1479	1.54084872579406	1.54084872579406\\
56.75	0.15156	1.87435375794225	1.87435375794225\\
56.75	0.15522	2.23372377998946	2.23372377998946\\
56.75	0.15888	2.61895879193572	2.61895879193572\\
56.75	0.16254	3.03005879378101	3.03005879378101\\
56.75	0.1662	3.46702378552535	3.46702378552535\\
56.75	0.16986	3.92985376716872	3.92985376716872\\
56.75	0.17352	4.41854873871112	4.41854873871112\\
56.75	0.17718	4.93310870015257	4.93310870015257\\
56.75	0.18084	5.47353365149305	5.47353365149305\\
56.75	0.1845	6.03982359273257	6.03982359273257\\
56.75	0.18816	6.63197852387114	6.63197852387114\\
56.75	0.19182	7.24999844490873	7.24999844490873\\
56.75	0.19548	7.89388335584538	7.89388335584538\\
56.75	0.19914	8.56363325668104	8.56363325668104\\
56.75	0.2028	9.25924814741575	9.25924814741575\\
56.75	0.20646	9.9807280280495	9.9807280280495\\
56.75	0.21012	10.7280728985823	10.7280728985823\\
56.75	0.21378	11.5012827590141	11.5012827590141\\
56.75	0.21744	12.300357609345	12.300357609345\\
56.75	0.2211	13.1252974495749	13.1252974495749\\
56.75	0.22476	13.9761022797038	13.9761022797038\\
56.75	0.22842	14.8527720997318	14.8527720997318\\
56.75	0.23208	15.7553069096588	15.7553069096588\\
56.75	0.23574	16.6837067094848	16.6837067094848\\
56.75	0.2394	17.6379714992099	17.6379714992099\\
56.75	0.24306	18.6181012788341	18.6181012788341\\
56.75	0.24672	19.6240960483572	19.6240960483572\\
56.75	0.25038	20.6559558077794	20.6559558077794\\
56.75	0.25404	21.7136805571007	21.7136805571007\\
56.75	0.2577	22.797270296321	22.797270296321\\
56.75	0.26136	23.9067250254403	23.9067250254403\\
56.75	0.26502	25.0420447444586	25.0420447444586\\
56.75	0.26868	26.203229453376	26.203229453376\\
56.75	0.27234	27.3902791521924	27.3902791521924\\
56.75	0.276	28.6031938409079	28.6031938409079\\
57.125	0.093	-0.28770575399958	-0.28770575399958\\
57.125	0.09666	-0.350995408500511	-0.350995408500511\\
57.125	0.10032	-0.388420073102401	-0.388420073102401\\
57.125	0.10398	-0.399979747805258	-0.399979747805258\\
57.125	0.10764	-0.385674432609076	-0.385674432609076\\
57.125	0.1113	-0.345504127513854	-0.345504127513854\\
57.125	0.11496	-0.279468832519594	-0.279468832519594\\
57.125	0.11862	-0.1875685476263	-0.1875685476263\\
57.125	0.12228	-0.0698032728339797	-0.0698032728339797\\
57.125	0.12594	0.0738269918574055	0.0738269918574055\\
57.125	0.1296	0.243322246447811	0.243322246447811\\
57.125	0.13326	0.438682490937243	0.438682490937243\\
57.125	0.13692	0.659907725325741	0.659907725325741\\
57.125	0.14058	0.906997949613256	0.906997949613256\\
57.125	0.14424	1.17995316379982	1.17995316379982\\
57.125	0.1479	1.47877336788542	1.47877336788542\\
57.125	0.15156	1.80345856187005	1.80345856187005\\
57.125	0.15522	2.15400874575372	2.15400874575372\\
57.125	0.15888	2.53042391953643	2.53042391953643\\
57.125	0.16254	2.93270408321819	2.93270408321819\\
57.125	0.1662	3.36084923679898	3.36084923679898\\
57.125	0.16986	3.81485938027879	3.81485938027879\\
57.125	0.17352	4.29473451365765	4.29473451365765\\
57.125	0.17718	4.80047463693557	4.80047463693557\\
57.125	0.18084	5.33207975011249	5.33207975011249\\
57.125	0.1845	5.88954985318847	5.88954985318847\\
57.125	0.18816	6.47288494616348	6.47288494616348\\
57.125	0.19182	7.08208502903754	7.08208502903754\\
57.125	0.19548	7.71715010181062	7.71715010181062\\
57.125	0.19914	8.37808016448275	8.37808016448275\\
57.125	0.2028	9.06487521705393	9.06487521705393\\
57.125	0.20646	9.77753525952414	9.77753525952414\\
57.125	0.21012	10.5160602918934	10.5160602918934\\
57.125	0.21378	11.2804503141616	11.2804503141616\\
57.125	0.21744	12.070705326329	12.070705326329\\
57.125	0.2211	12.8868253283953	12.8868253283953\\
57.125	0.22476	13.7288103203607	13.7288103203607\\
57.125	0.22842	14.5966603022251	14.5966603022251\\
57.125	0.23208	15.4903752739886	15.4903752739886\\
57.125	0.23574	16.4099552356511	16.4099552356511\\
57.125	0.2394	17.3554001872127	17.3554001872127\\
57.125	0.24306	18.3267101286732	18.3267101286732\\
57.125	0.24672	19.3238850600328	19.3238850600328\\
57.125	0.25038	20.3469249812915	20.3469249812915\\
57.125	0.25404	21.3958298924492	21.3958298924492\\
57.125	0.2577	22.4705997935059	22.4705997935059\\
57.125	0.26136	23.5712346844617	23.5712346844617\\
57.125	0.26502	24.6977345653165	24.6977345653165\\
57.125	0.26868	25.8500994360704	25.8500994360704\\
57.125	0.27234	27.0283292967232	27.0283292967232\\
57.125	0.276	28.2324241472752	28.2324241472752\\
57.5	0.093	-0.212054514386933	-0.212054514386933\\
57.5	0.09666	-0.284164007051411	-0.284164007051411\\
57.5	0.10032	-0.330408509816833	-0.330408509816833\\
57.5	0.10398	-0.35078802268325	-0.35078802268325\\
57.5	0.10764	-0.3453025456506	-0.3453025456506\\
57.5	0.1113	-0.313952078718939	-0.313952078718939\\
57.5	0.11496	-0.256736621888226	-0.256736621888226\\
57.5	0.11862	-0.173656175158479	-0.173656175158479\\
57.5	0.12228	-0.0647107385296906	-0.0647107385296906\\
57.5	0.12594	0.0700996879981339	0.0700996879981339\\
57.5	0.1296	0.230775104425007	0.230775104425007\\
57.5	0.13326	0.417315510750907	0.417315510750907\\
57.5	0.13692	0.629720906975844	0.629720906975844\\
57.5	0.14058	0.867991293099813	0.867991293099813\\
57.5	0.14424	1.13212666912283	1.13212666912283\\
57.5	0.1479	1.42212703504488	1.42212703504488\\
57.5	0.15156	1.73799239086598	1.73799239086598\\
57.5	0.15522	2.07972273658611	2.07972273658611\\
57.5	0.15888	2.44731807220527	2.44731807220527\\
57.5	0.16254	2.84077839772347	2.84077839772347\\
57.5	0.1662	3.26010371314071	3.26010371314071\\
57.5	0.16986	3.70529401845699	3.70529401845699\\
57.5	0.17352	4.17634931367232	4.17634931367232\\
57.5	0.17718	4.67326959878667	4.67326959878667\\
57.5	0.18084	5.19605487380006	5.19605487380006\\
57.5	0.1845	5.74470513871248	5.74470513871248\\
57.5	0.18816	6.31922039352396	6.31922039352396\\
57.5	0.19182	6.91960063823446	6.91960063823446\\
57.5	0.19548	7.54584587284401	7.54584587284401\\
57.5	0.19914	8.19795609735258	8.19795609735258\\
57.5	0.2028	8.87593131176023	8.87593131176023\\
57.5	0.20646	9.57977151606687	9.57977151606687\\
57.5	0.21012	10.3094767102726	10.3094767102726\\
57.5	0.21378	11.0650468943773	11.0650468943773\\
57.5	0.21744	11.8464820683811	11.8464820683811\\
57.5	0.2211	12.6537822322839	12.6537822322839\\
57.5	0.22476	13.4869473860857	13.4869473860857\\
57.5	0.22842	14.3459775297866	14.3459775297866\\
57.5	0.23208	15.2308726633865	15.2308726633865\\
57.5	0.23574	16.1416327868855	16.1416327868855\\
57.5	0.2394	17.0782579002835	17.0782579002835\\
57.5	0.24306	18.0407480035805	18.0407480035805\\
57.5	0.24672	19.0291030967766	19.0291030967766\\
57.5	0.25038	20.0433231798717	20.0433231798717\\
57.5	0.25404	21.0834082528658	21.0834082528658\\
57.5	0.2577	22.149358315759	22.149358315759\\
57.5	0.26136	23.2411733685513	23.2411733685513\\
57.5	0.26502	24.3588534112425	24.3588534112425\\
57.5	0.26868	25.5023984438328	25.5023984438328\\
57.5	0.27234	26.6718084663222	26.6718084663222\\
57.5	0.276	27.8670834787105	27.8670834787105\\
57.875	0.093	-0.130974249706105	-0.130974249706105\\
57.875	0.09666	-0.211903580534129	-0.211903580534129\\
57.875	0.10032	-0.266967921463111	-0.266967921463111\\
57.875	0.10398	-0.296167272493062	-0.296167272493062\\
57.875	0.10764	-0.299501633623972	-0.299501633623972\\
57.875	0.1113	-0.276971004855843	-0.276971004855843\\
57.875	0.11496	-0.228575386188677	-0.228575386188677\\
57.875	0.11862	-0.154314777622476	-0.154314777622476\\
57.875	0.12228	-0.054189179157234	-0.054189179157234\\
57.875	0.12594	0.071801409207044	0.071801409207044\\
57.875	0.1296	0.223656987470356	0.223656987470356\\
57.875	0.13326	0.401377555632724	0.401377555632724\\
57.875	0.13692	0.604963113694115	0.604963113694115\\
57.875	0.14058	0.834413661654537	0.834413661654537\\
57.875	0.14424	1.08972919951402	1.08972919951402\\
57.875	0.1479	1.37090972727251	1.37090972727251\\
57.875	0.15156	1.67795524493007	1.67795524493007\\
57.875	0.15522	2.01086575248665	2.01086575248665\\
57.875	0.15888	2.36964124994228	2.36964124994228\\
57.875	0.16254	2.75428173729693	2.75428173729693\\
57.875	0.1662	3.16478721455062	3.16478721455062\\
57.875	0.16986	3.60115768170336	3.60115768170336\\
57.875	0.17352	4.06339313875512	4.06339313875512\\
57.875	0.17718	4.55149358570594	4.55149358570594\\
57.875	0.18084	5.06545902255579	5.06545902255579\\
57.875	0.1845	5.60528944930468	5.60528944930468\\
57.875	0.18816	6.1709848659526	6.1709848659526\\
57.875	0.19182	6.76254527249956	6.76254527249956\\
57.875	0.19548	7.37997066894555	7.37997066894555\\
57.875	0.19914	8.02326105529059	8.02326105529059\\
57.875	0.2028	8.69241643153467	8.69241643153467\\
57.875	0.20646	9.38743679767777	9.38743679767777\\
57.875	0.21012	10.1083221537199	10.1083221537199\\
57.875	0.21378	10.8550724996611	10.8550724996611\\
57.875	0.21744	11.6276878355013	11.6276878355013\\
57.875	0.2211	12.4261681612406	12.4261681612406\\
57.875	0.22476	13.2505134768789	13.2505134768789\\
57.875	0.22842	14.1007237824162	14.1007237824162\\
57.875	0.23208	14.9767990778526	14.9767990778526\\
57.875	0.23574	15.878739363188	15.878739363188\\
57.875	0.2394	16.8065446384225	16.8065446384225\\
57.875	0.24306	17.760214903556	17.760214903556\\
57.875	0.24672	18.7397501585885	18.7397501585885\\
57.875	0.25038	19.7451504035201	19.7451504035201\\
57.875	0.25404	20.7764156383507	20.7764156383507\\
57.875	0.2577	21.8335458630803	21.8335458630803\\
57.875	0.26136	22.916541077709	22.916541077709\\
57.875	0.26502	24.0254012822367	24.0254012822367\\
57.875	0.26868	25.1601264766634	25.1601264766634\\
57.875	0.27234	26.3207166609892	26.3207166609892\\
57.875	0.276	27.5071718352141	27.5071718352141\\
58.25	0.093	-0.0444649599571587	-0.0444649599571587\\
58.25	0.09666	-0.134214128948715	-0.134214128948715\\
58.25	0.10032	-0.198098308041244	-0.198098308041244\\
58.25	0.10398	-0.236117497234741	-0.236117497234741\\
58.25	0.10764	-0.248271696529184	-0.248271696529184\\
58.25	0.1113	-0.234560905924615	-0.234560905924615\\
58.25	0.11496	-0.194985125420981	-0.194985125420981\\
58.25	0.11862	-0.129544355018341	-0.129544355018341\\
58.25	0.12228	-0.0382385947166313	-0.0382385947166313\\
58.25	0.12594	0.0789321554841003	0.0789321554841003\\
58.25	0.1296	0.221967895583866	0.221967895583866\\
58.25	0.13326	0.390868625582687	0.390868625582687\\
58.25	0.13692	0.585634345480532	0.585634345480532\\
58.25	0.14058	0.806265055277422	0.806265055277422\\
58.25	0.14424	1.05276075497335	1.05276075497335\\
58.25	0.1479	1.3251214445683	1.3251214445683\\
58.25	0.15156	1.6233471240623	1.6233471240623\\
58.25	0.15522	1.94743779345533	1.94743779345533\\
58.25	0.15888	2.29739345274741	2.29739345274741\\
58.25	0.16254	2.67321410193852	2.67321410193852\\
58.25	0.1662	3.07489974102867	3.07489974102867\\
58.25	0.16986	3.50245037001785	3.50245037001785\\
58.25	0.17352	3.95586598890609	3.95586598890609\\
58.25	0.17718	4.43514659769335	4.43514659769335\\
58.25	0.18084	4.94029219637965	4.94029219637965\\
58.25	0.1845	5.471302784965	5.471302784965\\
58.25	0.18816	6.02817836344936	6.02817836344936\\
58.25	0.19182	6.61091893183279	6.61091893183279\\
58.25	0.19548	7.21952449011523	7.21952449011523\\
58.25	0.19914	7.85399503829673	7.85399503829673\\
58.25	0.2028	8.51433057637725	8.51433057637725\\
58.25	0.20646	9.20053110435682	9.20053110435682\\
58.25	0.21012	9.91259662223543	9.91259662223543\\
58.25	0.21378	10.6505271300131	10.6505271300131\\
58.25	0.21744	11.4143226276898	11.4143226276898\\
58.25	0.2211	12.2039831152655	12.2039831152655\\
58.25	0.22476	13.0195085927402	13.0195085927402\\
58.25	0.22842	13.860899060114	13.860899060114\\
58.25	0.23208	14.7281545173868	14.7281545173868\\
58.25	0.23574	15.6212749645587	15.6212749645587\\
58.25	0.2394	16.5402604016296	16.5402604016296\\
58.25	0.24306	17.4851108285996	17.4851108285996\\
58.25	0.24672	18.4558262454686	18.4558262454686\\
58.25	0.25038	19.4524066522366	19.4524066522366\\
58.25	0.25404	20.4748520489036	20.4748520489036\\
58.25	0.2577	21.5231624354697	21.5231624354697\\
58.25	0.26136	22.5973378119348	22.5973378119348\\
58.25	0.26502	23.697378178299	23.697378178299\\
58.25	0.26868	24.8232835345622	24.8232835345622\\
58.25	0.27234	25.9750538807245	25.9750538807245\\
58.25	0.276	27.1526892167858	27.1526892167858\\
58.625	0.093	0.0474733548599691	0.0474733548599691\\
58.625	0.09666	-0.0510956522951336	-0.0510956522951336\\
58.625	0.10032	-0.123799669551209	-0.123799669551209\\
58.625	0.10398	-0.170638696908252	-0.170638696908252\\
58.625	0.10764	-0.191612734366256	-0.191612734366256\\
58.625	0.1113	-0.18672178192522	-0.18672178192522\\
58.625	0.11496	-0.155965839585146	-0.155965839585146\\
58.625	0.11862	-0.0993449073460386	-0.0993449073460386\\
58.625	0.12228	-0.0168589852078895	-0.0168589852078895\\
58.625	0.12594	0.0914919268293097	0.0914919268293097\\
58.625	0.1296	0.225707828765529	0.225707828765529\\
58.625	0.13326	0.385788720600804	0.385788720600804\\
58.625	0.13692	0.571734602335102	0.571734602335102\\
58.625	0.14058	0.783545473968445	0.783545473968445\\
58.625	0.14424	1.02122133550082	1.02122133550082\\
58.625	0.1479	1.28476218693223	1.28476218693223\\
58.625	0.15156	1.57416802826268	1.57416802826268\\
58.625	0.15522	1.88943885949218	1.88943885949218\\
58.625	0.15888	2.23057468062071	2.23057468062071\\
58.625	0.16254	2.59757549164828	2.59757549164828\\
58.625	0.1662	2.99044129257488	2.99044129257488\\
58.625	0.16986	3.40917208340053	3.40917208340053\\
58.625	0.17352	3.85376786412521	3.85376786412521\\
58.625	0.17718	4.32422863474892	4.32422863474892\\
58.625	0.18084	4.82055439527167	4.82055439527167\\
58.625	0.1845	5.34274514569347	5.34274514569347\\
58.625	0.18816	5.8908008860143	5.8908008860143\\
58.625	0.19182	6.46472161623416	6.46472161623416\\
58.625	0.19548	7.06450733635307	7.06450733635307\\
58.625	0.19914	7.69015804637102	7.69015804637102\\
58.625	0.2028	8.341673746288	8.341673746288\\
58.625	0.20646	9.01905443610404	9.01905443610404\\
58.625	0.21012	9.72230011581908	9.72230011581908\\
58.625	0.21378	10.4514107854332	10.4514107854332\\
58.625	0.21744	11.2063864449463	11.2063864449463\\
58.625	0.2211	11.9872270943585	11.9872270943585\\
58.625	0.22476	12.7939327336697	12.7939327336697\\
58.625	0.22842	13.6265033628799	13.6265033628799\\
58.625	0.23208	14.4849389819892	14.4849389819892\\
58.625	0.23574	15.3692395909976	15.3692395909976\\
58.625	0.2394	16.2794051899049	16.2794051899049\\
58.625	0.24306	17.2154357787113	17.2154357787113\\
58.625	0.24672	18.1773313574168	18.1773313574168\\
58.625	0.25038	19.1650919260212	19.1650919260212\\
58.625	0.25404	20.1787174845247	20.1787174845247\\
58.625	0.2577	21.2182080329273	21.2182080329273\\
58.625	0.26136	22.2835635712289	22.2835635712289\\
58.625	0.26502	23.3747840994295	23.3747840994295\\
58.625	0.26868	24.4918696175291	24.4918696175291\\
58.625	0.27234	25.6348201255279	25.6348201255279\\
58.625	0.276	26.8036356234256	26.8036356234256\\
59	0.093	0.144840694745222	0.144840694745222\\
59	0.09666	0.0374518494265583	0.0374518494265583\\
59	0.10032	-0.0440720059930637	-0.0440720059930637\\
59	0.10398	-0.0997308715136391	-0.0997308715136391\\
59	0.10764	-0.129524747135175	-0.129524747135175\\
59	0.1113	-0.1334536328577	-0.1334536328577\\
59	0.11496	-0.111517528681158	-0.111517528681158\\
59	0.11862	-0.0637164346055972	-0.0637164346055972\\
59	0.12228	0.00994964936900544	0.00994964936900544\\
59	0.12594	0.109480723242658	0.109480723242658\\
59	0.1296	0.234876787015331	0.234876787015331\\
59	0.13326	0.386137840687045	0.386137840687045\\
59	0.13692	0.563263884257811	0.563263884257811\\
59	0.14058	0.766254917727593	0.766254917727593\\
59	0.14424	0.995110941096438	0.995110941096438\\
59	0.1479	1.2498319543643	1.2498319543643\\
59	0.15156	1.53041795753121	1.53041795753121\\
59	0.15522	1.83686895059716	1.83686895059716\\
59	0.15888	2.16918493356214	2.16918493356214\\
59	0.16254	2.52736590642617	2.52736590642617\\
59	0.1662	2.91141186918922	2.91141186918922\\
59	0.16986	3.32132282185133	3.32132282185133\\
59	0.17352	3.75709876441245	3.75709876441245\\
59	0.17718	4.21873969687264	4.21873969687264\\
59	0.18084	4.70624561923183	4.70624561923183\\
59	0.1845	5.21961653149009	5.21961653149009\\
59	0.18816	5.75885243364736	5.75885243364736\\
59	0.19182	6.32395332570369	6.32395332570369\\
59	0.19548	6.91491920765906	6.91491920765906\\
59	0.19914	7.53175007951346	7.53175007951346\\
59	0.2028	8.1744459412669	8.1744459412669\\
59	0.20646	8.84300679291938	8.84300679291938\\
59	0.21012	9.53743263447089	9.53743263447089\\
59	0.21378	10.2577234659214	10.2577234659214\\
59	0.21744	11.003879287271	11.003879287271\\
59	0.2211	11.7759000985196	11.7759000985196\\
59	0.22476	12.5737858996673	12.5737858996673\\
59	0.22842	13.397536690714	13.397536690714\\
59	0.23208	14.2471524716597	14.2471524716597\\
59	0.23574	15.1226332425045	15.1226332425045\\
59	0.2394	16.0239790032484	16.0239790032484\\
59	0.24306	16.9511897538912	16.9511897538912\\
59	0.24672	17.9042654944331	17.9042654944331\\
59	0.25038	18.883206224874	18.883206224874\\
59	0.25404	19.888011945214	19.888011945214\\
59	0.2577	20.918682655453	20.918682655453\\
59	0.26136	21.975218355591	21.975218355591\\
59	0.26502	23.0576190456281	23.0576190456281\\
59	0.26868	24.1658847255642	24.1658847255642\\
59	0.27234	25.3000153953994	25.3000153953994\\
59	0.276	26.4600110551336	26.4600110551336\\
59.375	0.093	0.247637059698627	0.247637059698627\\
59.375	0.09666	0.131428376216418	0.131428376216418\\
59.375	0.10032	0.0410846826332634	0.0410846826332634\\
59.375	0.10398	-0.0233940210508585	-0.0233940210508585\\
59.375	0.10764	-0.0620077348359551	-0.0620077348359551\\
59.375	0.1113	-0.0747564587220122	-0.0747564587220122\\
59.375	0.11496	-0.0616401927090315	-0.0616401927090315\\
59.375	0.11862	-0.0226589367970167	-0.0226589367970167\\
59.375	0.12228	0.0421873090140537	0.0421873090140537\\
59.375	0.12594	0.13289854472416	0.13289854472416\\
59.375	0.1296	0.249474770333286	0.249474770333286\\
59.375	0.13326	0.391915985841454	0.391915985841454\\
59.375	0.13692	0.560222191248673	0.560222191248673\\
59.375	0.14058	0.754393386554923	0.754393386554923\\
59.375	0.14424	0.974429571760222	0.974429571760222\\
59.375	0.1479	1.22033074686454	1.22033074686454\\
59.375	0.15156	1.4920969118679	1.4920969118679\\
59.375	0.15522	1.78972806677029	1.78972806677029\\
59.375	0.15888	2.11322421157173	2.11322421157173\\
59.375	0.16254	2.46258534627222	2.46258534627222\\
59.375	0.1662	2.83781147087172	2.83781147087172\\
59.375	0.16986	3.23890258537027	3.23890258537027\\
59.375	0.17352	3.66585868976787	3.66585868976787\\
59.375	0.17718	4.1186797840645	4.1186797840645\\
59.375	0.18084	4.59736586826014	4.59736586826014\\
59.375	0.1845	5.10191694235488	5.10191694235488\\
59.375	0.18816	5.63233300634858	5.63233300634858\\
59.375	0.19182	6.18861406024138	6.18861406024138\\
59.375	0.19548	6.77076010403319	6.77076010403319\\
59.375	0.19914	7.37877113772404	7.37877113772404\\
59.375	0.2028	8.01264716131394	8.01264716131394\\
59.375	0.20646	8.67238817480287	8.67238817480287\\
59.375	0.21012	9.35799417819084	9.35799417819084\\
59.375	0.21378	10.0694651714779	10.0694651714779\\
59.375	0.21744	10.8068011546639	10.8068011546639\\
59.375	0.2211	11.570002127749	11.570002127749\\
59.375	0.22476	12.3590680907331	12.3590680907331\\
59.375	0.22842	13.1739990436163	13.1739990436163\\
59.375	0.23208	14.0147949863984	14.0147949863984\\
59.375	0.23574	14.8814559190797	14.8814559190797\\
59.375	0.2394	15.7739818416599	15.7739818416599\\
59.375	0.24306	16.6923727541392	16.6923727541392\\
59.375	0.24672	17.6366286565176	17.6366286565176\\
59.375	0.25038	18.606749548795	18.606749548795\\
59.375	0.25404	19.6027354309714	19.6027354309714\\
59.375	0.2577	20.6245863030469	20.6245863030469\\
59.375	0.26136	21.6723021650214	21.6723021650214\\
59.375	0.26502	22.7458830168949	22.7458830168949\\
59.375	0.26868	23.8453288586674	23.8453288586674\\
59.375	0.27234	24.9706396903391	24.9706396903391\\
59.375	0.276	26.1218155119097	26.1218155119097\\
59.75	0.093	0.355862449720158	0.355862449720158\\
59.75	0.09666	0.230833928074416	0.230833928074416\\
59.75	0.10032	0.131670396327715	0.131670396327715\\
59.75	0.10398	0.0583718544800327	0.0583718544800327\\
59.75	0.10764	0.0109383025313896	0.0109383025313896\\
59.75	0.1113	-0.0106302595181997	-0.0106302595181997\\
59.75	0.11496	-0.00633383166876555	-0.00633383166876555\\
59.75	0.11862	0.0238275860797028	0.0238275860797028\\
59.75	0.12228	0.0798539937272267	0.0798539937272267\\
59.75	0.12594	0.161745391273787	0.161745391273787\\
59.75	0.1296	0.269501778719366	0.269501778719366\\
59.75	0.13326	0.403123156063987	0.403123156063987\\
59.75	0.13692	0.56260952330766	0.56260952330766\\
59.75	0.14058	0.747960880450364	0.747960880450364\\
59.75	0.14424	0.959177227492102	0.959177227492102\\
59.75	0.1479	1.19625856443289	1.19625856443289\\
59.75	0.15156	1.45920489127271	1.45920489127271\\
59.75	0.15522	1.74801620801156	1.74801620801156\\
59.75	0.15888	2.06269251464946	2.06269251464946\\
59.75	0.16254	2.40323381118638	2.40323381118638\\
59.75	0.1662	2.76964009762235	2.76964009762235\\
59.75	0.16986	3.16191137395735	3.16191137395735\\
59.75	0.17352	3.5800476401914	3.5800476401914\\
59.75	0.17718	4.02404889632448	4.02404889632448\\
59.75	0.18084	4.49391514235659	4.49391514235659\\
59.75	0.1845	4.98964637828776	4.98964637828776\\
59.75	0.18816	5.51124260411793	5.51124260411793\\
59.75	0.19182	6.05870381984717	6.05870381984717\\
59.75	0.19548	6.63203002547545	6.63203002547545\\
59.75	0.19914	7.23122122100276	7.23122122100276\\
59.75	0.2028	7.85627740642911	7.85627740642911\\
59.75	0.20646	8.5071985817545	8.5071985817545\\
59.75	0.21012	9.18398474697891	9.18398474697891\\
59.75	0.21378	9.88663590210238	9.88663590210238\\
59.75	0.21744	10.6151520471249	10.6151520471249\\
59.75	0.2211	11.3695331820464	11.3695331820464\\
59.75	0.22476	12.149779306867	12.149779306867\\
59.75	0.22842	12.9558904215866	12.9558904215866\\
59.75	0.23208	13.7878665262052	13.7878665262052\\
59.75	0.23574	14.6457076207229	14.6457076207229\\
59.75	0.2394	15.5294137051397	15.5294137051397\\
59.75	0.24306	16.4389847794554	16.4389847794554\\
59.75	0.24672	17.3744208436702	17.3744208436702\\
59.75	0.25038	18.3357218977841	18.3357218977841\\
59.75	0.25404	19.3228879417969	19.3228879417969\\
59.75	0.2577	20.3359189757089	20.3359189757089\\
59.75	0.26136	21.3748149995198	21.3748149995198\\
59.75	0.26502	22.4395760132298	22.4395760132298\\
59.75	0.26868	23.5302020168388	23.5302020168388\\
59.75	0.27234	24.6466930103469	24.6466930103469\\
59.75	0.276	25.789048993754	25.789048993754\\
60.125	0.093	0.469516864809885	0.469516864809885\\
60.125	0.09666	0.335668505000582	0.335668505000582\\
60.125	0.10032	0.22768513509032	0.22768513509032\\
60.125	0.10398	0.145566755079106	0.145566755079106\\
60.125	0.10764	0.0893133649669302	0.0893133649669302\\
60.125	0.1113	0.0589249647537802	0.0589249647537802\\
60.125	0.11496	0.0544015544396679	0.0544015544396679\\
60.125	0.11862	0.075743134024604	0.075743134024604\\
60.125	0.12228	0.122949703508567	0.122949703508567\\
60.125	0.12594	0.196021262891581	0.196021262891581\\
60.125	0.1296	0.294957812173614	0.294957812173614\\
60.125	0.13326	0.419759351354703	0.419759351354703\\
60.125	0.13692	0.570425880434829	0.570425880434829\\
60.125	0.14058	0.746957399413986	0.746957399413986\\
60.125	0.14424	0.949353908292178	0.949353908292178\\
60.125	0.1479	1.1776154070694	1.1776154070694\\
60.125	0.15156	1.4317418957457	1.4317418957457\\
60.125	0.15522	1.711733374321	1.711733374321\\
60.125	0.15888	2.01758984279535	2.01758984279535\\
60.125	0.16254	2.34931130116872	2.34931130116872\\
60.125	0.1662	2.70689774944115	2.70689774944115\\
60.125	0.16986	3.09034918761262	3.09034918761262\\
60.125	0.17352	3.49966561568311	3.49966561568311\\
60.125	0.17718	3.93484703365265	3.93484703365265\\
60.125	0.18084	4.39589344152123	4.39589344152123\\
60.125	0.1845	4.88280483928884	4.88280483928884\\
60.125	0.18816	5.39558122695548	5.39558122695548\\
60.125	0.19182	5.93422260452116	5.93422260452116\\
60.125	0.19548	6.4987289719859	6.4987289719859\\
60.125	0.19914	7.08910032934965	7.08910032934965\\
60.125	0.2028	7.70533667661247	7.70533667661247\\
60.125	0.20646	8.3474380137743	8.3474380137743\\
60.125	0.21012	9.01540434083518	9.01540434083518\\
60.125	0.21378	9.70923565779508	9.70923565779508\\
60.125	0.21744	10.428931964654	10.428931964654\\
60.125	0.2211	11.174493261412	11.174493261412\\
60.125	0.22476	11.945919548069	11.945919548069\\
60.125	0.22842	12.7432108246251	12.7432108246251\\
60.125	0.23208	13.5663670910802	13.5663670910802\\
60.125	0.23574	14.4153883474344	14.4153883474344\\
60.125	0.2394	15.2902745936875	15.2902745936875\\
60.125	0.24306	16.1910258298398	16.1910258298398\\
60.125	0.24672	17.117642055891	17.117642055891\\
60.125	0.25038	18.0701232718413	18.0701232718413\\
60.125	0.25404	19.0484694776906	19.0484694776906\\
60.125	0.2577	20.052680673439	20.052680673439\\
60.125	0.26136	21.0827568590864	21.0827568590864\\
60.125	0.26502	22.1386980346329	22.1386980346329\\
60.125	0.26868	23.2205042000783	23.2205042000783\\
60.125	0.27234	24.3281753554229	24.3281753554229\\
60.125	0.276	25.4617115006664	25.4617115006664\\
60.5	0.093	0.588600304967708	0.588600304967708\\
60.5	0.09666	0.445932106994873	0.445932106994873\\
60.5	0.10032	0.329128898921065	0.329128898921065\\
60.5	0.10398	0.238190680746289	0.238190680746289\\
60.5	0.10764	0.173117452470567	0.173117452470567\\
60.5	0.1113	0.133909214093885	0.133909214093885\\
60.5	0.11496	0.120565965616226	0.120565965616226\\
60.5	0.11862	0.133087707037616	0.133087707037616\\
60.5	0.12228	0.171474438358032	0.171474438358032\\
60.5	0.12594	0.235726159577499	0.235726159577499\\
60.5	0.1296	0.325842870696	0.325842870696\\
60.5	0.13326	0.441824571713543	0.441824571713543\\
60.5	0.13692	0.583671262630109	0.583671262630109\\
60.5	0.14058	0.751382943445719	0.751382943445719\\
60.5	0.14424	0.944959614160378	0.944959614160378\\
60.5	0.1479	1.16440127477406	1.16440127477406\\
60.5	0.15156	1.40970792528679	1.40970792528679\\
60.5	0.15522	1.68087956569854	1.68087956569854\\
60.5	0.15888	1.97791619600936	1.97791619600936\\
60.5	0.16254	2.30081781621919	2.30081781621919\\
60.5	0.1662	2.64958442632807	2.64958442632807\\
60.5	0.16986	3.02421602633598	3.02421602633598\\
60.5	0.17352	3.42471261624294	3.42471261624294\\
60.5	0.17718	3.85107419604893	3.85107419604893\\
60.5	0.18084	4.30330076575396	4.30330076575396\\
60.5	0.1845	4.78139232535801	4.78139232535801\\
60.5	0.18816	5.28534887486112	5.28534887486112\\
60.5	0.19182	5.81517041426327	5.81517041426327\\
60.5	0.19548	6.37085694356445	6.37085694356445\\
60.5	0.19914	6.95240846276467	6.95240846276467\\
60.5	0.2028	7.55982497186393	7.55982497186393\\
60.5	0.20646	8.19310647086221	8.19310647086221\\
60.5	0.21012	8.85225295975954	8.85225295975954\\
60.5	0.21378	9.53726443855592	9.53726443855592\\
60.5	0.21744	10.2481409072513	10.2481409072513\\
60.5	0.2211	10.9848823658458	10.9848823658458\\
60.5	0.22476	11.7474888143392	11.7474888143392\\
60.5	0.22842	12.5359602527318	12.5359602527318\\
60.5	0.23208	13.3502966810233	13.3502966810233\\
60.5	0.23574	14.1904980992139	14.1904980992139\\
60.5	0.2394	15.0565645073035	15.0565645073035\\
60.5	0.24306	15.9484959052922	15.9484959052922\\
60.5	0.24672	16.8662922931799	16.8662922931799\\
60.5	0.25038	17.8099536709667	17.8099536709667\\
60.5	0.25404	18.7794800386525	18.7794800386525\\
60.5	0.2577	19.7748713962373	19.7748713962373\\
60.5	0.26136	20.7961277437211	20.7961277437211\\
60.5	0.26502	21.843249081104	21.843249081104\\
60.5	0.26868	22.916235408386	22.916235408386\\
60.5	0.27234	24.0150867255669	24.0150867255669\\
60.5	0.276	25.139803032647	25.139803032647\\
60.875	0.093	0.713112770193698	0.713112770193698\\
60.875	0.09666	0.561624734057331	0.561624734057331\\
60.875	0.10032	0.436001687819976	0.436001687819976\\
60.875	0.10398	0.336243631481654	0.336243631481654\\
60.875	0.10764	0.2623505650424	0.2623505650424\\
60.875	0.1113	0.214322488502157	0.214322488502157\\
60.875	0.11496	0.192159401860952	0.192159401860952\\
60.875	0.11862	0.195861305118795	0.195861305118795\\
60.875	0.12228	0.225428198275679	0.225428198275679\\
60.875	0.12594	0.280860081331586	0.280860081331586\\
60.875	0.1296	0.36215695428654	0.36215695428654\\
60.875	0.13326	0.469318817140536	0.469318817140536\\
60.875	0.13692	0.60234566989357	0.60234566989357\\
60.875	0.14058	0.761237512545634	0.761237512545634\\
60.875	0.14424	0.945994345096747	0.945994345096747\\
60.875	0.1479	1.15661616754688	1.15661616754688\\
60.875	0.15156	1.39310297989606	1.39310297989606\\
60.875	0.15522	1.65545478214427	1.65545478214427\\
60.875	0.15888	1.94367157429154	1.94367157429154\\
60.875	0.16254	2.25775335633783	2.25775335633783\\
60.875	0.1662	2.59770012828316	2.59770012828316\\
60.875	0.16986	2.96351189012753	2.96351189012753\\
60.875	0.17352	3.35518864187095	3.35518864187095\\
60.875	0.17718	3.77273038351338	3.77273038351338\\
60.875	0.18084	4.21613711505487	4.21613711505487\\
60.875	0.1845	4.68540883649538	4.68540883649538\\
60.875	0.18816	5.18054554783493	5.18054554783493\\
60.875	0.19182	5.70154724907355	5.70154724907355\\
60.875	0.19548	6.24841394021117	6.24841394021117\\
60.875	0.19914	6.82114562124785	6.82114562124785\\
60.875	0.2028	7.41974229218358	7.41974229218358\\
60.875	0.20646	8.04420395301832	8.04420395301832\\
60.875	0.21012	8.69453060375209	8.69453060375209\\
60.875	0.21378	9.37072224438491	9.37072224438491\\
60.875	0.21744	10.0727788749168	10.0727788749168\\
60.875	0.2211	10.8007004953477	10.8007004953477\\
60.875	0.22476	11.5544871056776	11.5544871056776\\
60.875	0.22842	12.3341387059066	12.3341387059066\\
60.875	0.23208	13.1396552960346	13.1396552960346\\
60.875	0.23574	13.9710368760616	13.9710368760616\\
60.875	0.2394	14.8282834459877	14.8282834459877\\
60.875	0.24306	15.7113950058129	15.7113950058129\\
60.875	0.24672	16.620371555537	16.620371555537\\
60.875	0.25038	17.5552130951602	17.5552130951602\\
60.875	0.25404	18.5159196246825	18.5159196246825\\
60.875	0.2577	19.5024911441038	19.5024911441038\\
60.875	0.26136	20.5149276534241	20.5149276534241\\
60.875	0.26502	21.5532291526434	21.5532291526434\\
60.875	0.26868	22.6173956417618	22.6173956417618\\
60.875	0.27234	23.7074271207792	23.7074271207792\\
60.875	0.276	24.8233235896957	24.8233235896957\\
61.25	0.093	0.843054260487813	0.843054260487813\\
61.25	0.09666	0.682746386187885	0.682746386187885\\
61.25	0.10032	0.548303501786998	0.548303501786998\\
61.25	0.10398	0.439725607285144	0.439725607285144\\
61.25	0.10764	0.357012702682329	0.357012702682329\\
61.25	0.1113	0.30016478797854	0.30016478797854\\
61.25	0.11496	0.269181863173788	0.269181863173788\\
61.25	0.11862	0.264063928268085	0.264063928268085\\
61.25	0.12228	0.284810983261423	0.284810983261423\\
61.25	0.12594	0.331423028153797	0.331423028153797\\
61.25	0.1296	0.403900062945191	0.403900062945191\\
61.25	0.13326	0.50224208763564	0.50224208763564\\
61.25	0.13692	0.626449102225127	0.626449102225127\\
61.25	0.14058	0.776521106713645	0.776521106713645\\
61.25	0.14424	0.952458101101211	0.952458101101211\\
61.25	0.1479	1.15426008538781	1.15426008538781\\
61.25	0.15156	1.38192705957345	1.38192705957345\\
61.25	0.15522	1.6354590236581	1.6354590236581\\
61.25	0.15888	1.91485597764182	1.91485597764182\\
61.25	0.16254	2.22011792152457	2.22011792152457\\
61.25	0.1662	2.55124485530636	2.55124485530636\\
61.25	0.16986	2.90823677898717	2.90823677898717\\
61.25	0.17352	3.29109369256704	3.29109369256704\\
61.25	0.17718	3.69981559604594	3.69981559604594\\
61.25	0.18084	4.13440248942387	4.13440248942387\\
61.25	0.1845	4.59485437270085	4.59485437270085\\
61.25	0.18816	5.08117124587687	5.08117124587687\\
61.25	0.19182	5.59335310895192	5.59335310895192\\
61.25	0.19548	6.13139996192601	6.13139996192601\\
61.25	0.19914	6.69531180479913	6.69531180479913\\
61.25	0.2028	7.2850886375713	7.2850886375713\\
61.25	0.20646	7.90073046024251	7.90073046024251\\
61.25	0.21012	8.54223727281274	8.54223727281274\\
61.25	0.21378	9.20960907528202	9.20960907528202\\
61.25	0.21744	9.90284586765034	9.90284586765034\\
61.25	0.2211	10.6219476499177	10.6219476499177\\
61.25	0.22476	11.3669144220841	11.3669144220841\\
61.25	0.22842	12.1377461841495	12.1377461841495\\
61.25	0.23208	12.934442936114	12.934442936114\\
61.25	0.23574	13.7570046779775	13.7570046779775\\
61.25	0.2394	14.60543140974	14.60543140974\\
61.25	0.24306	15.4797231314016	15.4797231314016\\
61.25	0.24672	16.3798798429622	16.3798798429622\\
61.25	0.25038	17.3059015444219	17.3059015444219\\
61.25	0.25404	18.2577882357806	18.2577882357806\\
61.25	0.2577	19.2355399170383	19.2355399170383\\
61.25	0.26136	20.2391565881951	20.2391565881951\\
61.25	0.26502	21.2686382492509	21.2686382492509\\
61.25	0.26868	22.3239849002057	22.3239849002057\\
61.25	0.27234	23.4051965410596	23.4051965410596\\
61.25	0.276	24.5122731718125	24.5122731718125\\
61.625	0.093	0.978424775850096	0.978424775850096\\
61.625	0.09666	0.809297063386635	0.809297063386635\\
61.625	0.10032	0.666034340822188	0.666034340822188\\
61.625	0.10398	0.548636608156787	0.548636608156787\\
61.625	0.10764	0.457103865390426	0.457103865390426\\
61.625	0.1113	0.39143611252309	0.39143611252309\\
61.625	0.11496	0.351633349554806	0.351633349554806\\
61.625	0.11862	0.337695576485556	0.337695576485556\\
61.625	0.12228	0.349622793315334	0.349622793315334\\
61.625	0.12594	0.387415000044175	0.387415000044175\\
61.625	0.1296	0.451072196672023	0.451072196672023\\
61.625	0.13326	0.540594383198925	0.540594383198925\\
61.625	0.13692	0.655981559624866	0.655981559624866\\
61.625	0.14058	0.797233725949837	0.797233725949837\\
61.625	0.14424	0.964350882173871	0.964350882173871\\
61.625	0.1479	1.15733302829691	1.15733302829691\\
61.625	0.15156	1.37618016431899	1.37618016431899\\
61.625	0.15522	1.62089229024012	1.62089229024012\\
61.625	0.15888	1.8914694060603	1.8914694060603\\
61.625	0.16254	2.18791151177949	2.18791151177949\\
61.625	0.1662	2.51021860739773	2.51021860739773\\
61.625	0.16986	2.85839069291501	2.85839069291501\\
61.625	0.17352	3.23242776833133	3.23242776833133\\
61.625	0.17718	3.63232983364668	3.63232983364668\\
61.625	0.18084	4.05809688886106	4.05809688886106\\
61.625	0.1845	4.50972893397451	4.50972893397451\\
61.625	0.18816	4.98722596898697	4.98722596898697\\
61.625	0.19182	5.49058799389847	5.49058799389847\\
61.625	0.19548	6.01981500870902	6.01981500870902\\
61.625	0.19914	6.5749070134186	6.5749070134186\\
61.625	0.2028	7.15586400802722	7.15586400802722\\
61.625	0.20646	7.76268599253488	7.76268599253488\\
61.625	0.21012	8.39537296694157	8.39537296694157\\
61.625	0.21378	9.0539249312473	9.0539249312473\\
61.625	0.21744	9.73834188545207	9.73834188545207\\
61.625	0.2211	10.4486238295559	10.4486238295559\\
61.625	0.22476	11.1847707635587	11.1847707635587\\
61.625	0.22842	11.9467826874606	11.9467826874606\\
61.625	0.23208	12.7346596012615	12.7346596012615\\
61.625	0.23574	13.5484015049615	13.5484015049615\\
61.625	0.2394	14.3880083985605	14.3880083985605\\
61.625	0.24306	15.2534802820585	15.2534802820585\\
61.625	0.24672	16.1448171554556	16.1448171554556\\
61.625	0.25038	17.0620190187517	17.0620190187517\\
61.625	0.25404	18.0050858719468	18.0050858719468\\
61.625	0.2577	18.974017715041	18.974017715041\\
61.625	0.26136	19.9688145480343	19.9688145480343\\
61.625	0.26502	20.9894763709265	20.9894763709265\\
61.625	0.26868	22.0360031837178	22.0360031837178\\
61.625	0.27234	23.1083949864082	23.1083949864082\\
61.625	0.276	24.2066517789975	24.2066517789975\\
62	0.093	1.11922431628055	1.11922431628055\\
62	0.09666	0.941276765653525	0.941276765653525\\
62	0.10032	0.789194204925531	0.789194204925531\\
62	0.10398	0.662976634096584	0.662976634096584\\
62	0.10764	0.562624053166676	0.562624053166676\\
62	0.1113	0.488136462135794	0.488136462135794\\
62	0.11496	0.439513861003963	0.439513861003963\\
62	0.11862	0.416756249771167	0.416756249771167\\
62	0.12228	0.419863628437412	0.419863628437412\\
62	0.12594	0.448835997002693	0.448835997002693\\
62	0.1296	0.503673355467008	0.503673355467008\\
62	0.13326	0.584375703830364	0.584375703830364\\
62	0.13692	0.690943042092758	0.690943042092758\\
62	0.14058	0.823375370254183	0.823375370254183\\
62	0.14424	0.981672688314656	0.981672688314656\\
62	0.1479	1.16583499627415	1.16583499627415\\
62	0.15156	1.37586229413271	1.37586229413271\\
62	0.15522	1.61175458189028	1.61175458189028\\
62	0.15888	1.87351185954691	1.87351185954691\\
62	0.16254	2.16113412710257	2.16113412710257\\
62	0.1662	2.47462138455726	2.47462138455726\\
62	0.16986	2.813973631911	2.813973631911\\
62	0.17352	3.17919086916377	3.17919086916377\\
62	0.17718	3.57027309631558	3.57027309631558\\
62	0.18084	3.98722031336643	3.98722031336643\\
62	0.1845	4.43003252031632	4.43003252031632\\
62	0.18816	4.89870971716521	4.89870971716521\\
62	0.19182	5.39325190391317	5.39325190391317\\
62	0.19548	5.91365908056017	5.91365908056017\\
62	0.19914	6.45993124710621	6.45993124710621\\
62	0.2028	7.0320684035513	7.0320684035513\\
62	0.20646	7.6300705498954	7.6300705498954\\
62	0.21012	8.25393768613854	8.25393768613854\\
62	0.21378	8.90366981228073	8.90366981228073\\
62	0.21744	9.57926692832197	9.57926692832197\\
62	0.2211	10.2807290342622	10.2807290342622\\
62	0.22476	11.0080561301015	11.0080561301015\\
62	0.22842	11.7612482158399	11.7612482158399\\
62	0.23208	12.5403052914772	12.5403052914772\\
62	0.23574	13.3452273570137	13.3452273570137\\
62	0.2394	14.1760144124491	14.1760144124491\\
62	0.24306	15.0326664577836	15.0326664577836\\
62	0.24672	15.9151834930171	15.9151834930171\\
62	0.25038	16.8235655181497	16.8235655181497\\
62	0.25404	17.7578125331813	17.7578125331813\\
62	0.2577	18.7179245381119	18.7179245381119\\
62	0.26136	19.7039015329416	19.7039015329416\\
62	0.26502	20.7157435176703	20.7157435176703\\
62	0.26868	21.7534504922981	21.7534504922981\\
62	0.27234	22.8170224568249	22.8170224568249\\
62	0.276	23.9064594112507	23.9064594112507\\
62.375	0.093	1.26545288177911	1.26545288177911\\
62.375	0.09666	1.07868549298854	1.07868549298854\\
62.375	0.10032	0.917783094097013	0.917783094097013\\
62.375	0.10398	0.782745685104505	0.782745685104505\\
62.375	0.10764	0.673573266011065	0.673573266011065\\
62.375	0.1113	0.590265836816636	0.590265836816636\\
62.375	0.11496	0.532823397521259	0.532823397521259\\
62.375	0.11862	0.501245948124916	0.501245948124916\\
62.375	0.12228	0.495533488627601	0.495533488627601\\
62.375	0.12594	0.515686019029349	0.515686019029349\\
62.375	0.1296	0.561703539330118	0.561703539330118\\
62.375	0.13326	0.633586049529928	0.633586049529928\\
62.375	0.13692	0.731333549628776	0.731333549628776\\
62.375	0.14058	0.854946039626668	0.854946039626668\\
62.375	0.14424	1.00442351952358	1.00442351952358\\
62.375	0.1479	1.17976598931954	1.17976598931954\\
62.375	0.15156	1.38097344901454	1.38097344901454\\
62.375	0.15522	1.60804589860858	1.60804589860858\\
62.375	0.15888	1.86098333810166	1.86098333810166\\
62.375	0.16254	2.13978576749376	2.13978576749376\\
62.375	0.1662	2.44445318678491	2.44445318678491\\
62.375	0.16986	2.7749855959751	2.7749855959751\\
62.375	0.17352	3.13138299506434	3.13138299506434\\
62.375	0.17718	3.5136453840526	3.5136453840526\\
62.375	0.18084	3.9217727629399	3.9217727629399\\
62.375	0.1845	4.35576513172623	4.35576513172623\\
62.375	0.18816	4.81562249041159	4.81562249041159\\
62.375	0.19182	5.30134483899602	5.30134483899602\\
62.375	0.19548	5.81293217747946	5.81293217747946\\
62.375	0.19914	6.35038450586195	6.35038450586195\\
62.375	0.2028	6.91370182414349	6.91370182414349\\
62.375	0.20646	7.50288413232406	7.50288413232406\\
62.375	0.21012	8.11793143040366	8.11793143040366\\
62.375	0.21378	8.7588437183823	8.7588437183823\\
62.375	0.21744	9.42562099625998	9.42562099625998\\
62.375	0.2211	10.1182632640367	10.1182632640367\\
62.375	0.22476	10.8367705217125	10.8367705217125\\
62.375	0.22842	11.5811427692872	11.5811427692872\\
62.375	0.23208	12.3513800067611	12.3513800067611\\
62.375	0.23574	13.1474822341339	13.1474822341339\\
62.375	0.2394	13.9694494514059	13.9694494514059\\
62.375	0.24306	14.8172816585768	14.8172816585768\\
62.375	0.24672	15.6909788556468	15.6909788556468\\
62.375	0.25038	16.5905410426158	16.5905410426158\\
62.375	0.25404	17.5159682194838	17.5159682194838\\
62.375	0.2577	18.467260386251	18.467260386251\\
62.375	0.26136	19.4444175429171	19.4444175429171\\
62.375	0.26502	20.4474396894823	20.4474396894823\\
62.375	0.26868	21.4763268259464	21.4763268259464\\
62.375	0.27234	22.5310789523097	22.5310789523097\\
62.375	0.276	23.611696068572	23.611696068572\\
62.75	0.093	1.41711047234582	1.41711047234582\\
62.75	0.09666	1.22152324539171	1.22152324539171\\
62.75	0.10032	1.05180100833663	1.05180100833663\\
62.75	0.10398	0.907943761180594	0.907943761180594\\
62.75	0.10764	0.789951503923607	0.789951503923607\\
62.75	0.1113	0.697824236565632	0.697824236565632\\
62.75	0.11496	0.631561959106708	0.631561959106708\\
62.75	0.11862	0.591164671546819	0.591164671546819\\
62.75	0.12228	0.576632373885971	0.576632373885971\\
62.75	0.12594	0.587965066124159	0.587965066124159\\
62.75	0.1296	0.625162748261381	0.625162748261381\\
62.75	0.13326	0.688225420297645	0.688225420297645\\
62.75	0.13692	0.77715308223296	0.77715308223296\\
62.75	0.14058	0.891945734067292	0.891945734067292\\
62.75	0.14424	1.03260337580067	1.03260337580067\\
62.75	0.1479	1.19912600743309	1.19912600743309\\
62.75	0.15156	1.39151362896454	1.39151362896454\\
62.75	0.15522	1.60976624039503	1.60976624039503\\
62.75	0.15888	1.85388384172457	1.85388384172457\\
62.75	0.16254	2.12386643295312	2.12386643295312\\
62.75	0.1662	2.41971401408074	2.41971401408074\\
62.75	0.16986	2.74142658510738	2.74142658510738\\
62.75	0.17352	3.08900414603306	3.08900414603306\\
62.75	0.17718	3.46244669685777	3.46244669685777\\
62.75	0.18084	3.86175423758153	3.86175423758153\\
62.75	0.1845	4.28692676820432	4.28692676820432\\
62.75	0.18816	4.73796428872615	4.73796428872615\\
62.75	0.19182	5.21486679914702	5.21486679914702\\
62.75	0.19548	5.71763429946693	5.71763429946693\\
62.75	0.19914	6.24626678968586	6.24626678968586\\
62.75	0.2028	6.80076426980387	6.80076426980387\\
62.75	0.20646	7.38112673982088	7.38112673982088\\
62.75	0.21012	7.98735419973693	7.98735419973693\\
62.75	0.21378	8.61944664955202	8.61944664955202\\
62.75	0.21744	9.27740408926616	9.27740408926616\\
62.75	0.2211	9.96122651887934	9.96122651887934\\
62.75	0.22476	10.6709139383915	10.6709139383915\\
62.75	0.22842	11.4064663478028	11.4064663478028\\
62.75	0.23208	12.1678837471131	12.1678837471131\\
62.75	0.23574	12.9551661363224	12.9551661363224\\
62.75	0.2394	13.7683135154308	13.7683135154308\\
62.75	0.24306	14.6073258844382	14.6073258844382\\
62.75	0.24672	15.4722032433446	15.4722032433446\\
62.75	0.25038	16.3629455921501	16.3629455921501\\
62.75	0.25404	17.2795529308546	17.2795529308546\\
62.75	0.2577	18.2220252594581	18.2220252594581\\
62.75	0.26136	19.1903625779607	19.1903625779607\\
62.75	0.26502	20.1845648863623	20.1845648863623\\
62.75	0.26868	21.204632184663	21.204632184663\\
62.75	0.27234	22.2505644728627	22.2505644728627\\
62.75	0.276	23.3223617509614	23.3223617509614\\
63.125	0.093	1.57419708798069	1.57419708798069\\
63.125	0.09666	1.36979002286303	1.36979002286303\\
63.125	0.10032	1.19124794764441	1.19124794764441\\
63.125	0.10398	1.03857086232482	1.03857086232482\\
63.125	0.10764	0.911758766904288	0.911758766904288\\
63.125	0.1113	0.810811661382767	0.810811661382767\\
63.125	0.11496	0.735729545760297	0.735729545760297\\
63.125	0.11862	0.686512420036861	0.686512420036861\\
63.125	0.12228	0.663160284212466	0.663160284212466\\
63.125	0.12594	0.665673138287108	0.665673138287108\\
63.125	0.1296	0.694050982260798	0.694050982260798\\
63.125	0.13326	0.748293816133515	0.748293816133515\\
63.125	0.13692	0.82840163990527	0.82840163990527\\
63.125	0.14058	0.934374453576055	0.934374453576055\\
63.125	0.14424	1.0662122571459	1.0662122571459\\
63.125	0.1479	1.22391505061476	1.22391505061476\\
63.125	0.15156	1.40748283398268	1.40748283398268\\
63.125	0.15522	1.61691560724961	1.61691560724961\\
63.125	0.15888	1.85221337041561	1.85221337041561\\
63.125	0.16254	2.11337612348062	2.11337612348062\\
63.125	0.1662	2.40040386644467	2.40040386644467\\
63.125	0.16986	2.71329659930778	2.71329659930778\\
63.125	0.17352	3.05205432206991	3.05205432206991\\
63.125	0.17718	3.41667703473108	3.41667703473108\\
63.125	0.18084	3.80716473729129	3.80716473729129\\
63.125	0.1845	4.22351742975056	4.22351742975056\\
63.125	0.18816	4.66573511210883	4.66573511210883\\
63.125	0.19182	5.13381778436614	5.13381778436614\\
63.125	0.19548	5.62776544652251	5.62776544652251\\
63.125	0.19914	6.14757809857791	6.14757809857791\\
63.125	0.2028	6.69325574053236	6.69325574053236\\
63.125	0.20646	7.26479837238582	7.26479837238582\\
63.125	0.21012	7.86220599413834	7.86220599413834\\
63.125	0.21378	8.48547860578989	8.48547860578989\\
63.125	0.21744	9.13461620734047	9.13461620734047\\
63.125	0.2211	9.80961879879008	9.80961879879008\\
63.125	0.22476	10.5104863801388	10.5104863801388\\
63.125	0.22842	11.2372189513865	11.2372189513865\\
63.125	0.23208	11.9898165125332	11.9898165125332\\
63.125	0.23574	12.768279063579	12.768279063579\\
63.125	0.2394	13.5726066045238	13.5726066045238\\
63.125	0.24306	14.4027991353676	14.4027991353676\\
63.125	0.24672	15.2588566561105	15.2588566561105\\
63.125	0.25038	16.1407791667525	16.1407791667525\\
63.125	0.25404	17.0485666672934	17.0485666672934\\
63.125	0.2577	17.9822191577334	17.9822191577334\\
63.125	0.26136	18.9417366380725	18.9417366380725\\
63.125	0.26502	19.9271191083106	19.9271191083106\\
63.125	0.26868	20.9383665684477	20.9383665684477\\
63.125	0.27234	21.9754790184838	21.9754790184838\\
63.125	0.276	23.038456458419	23.038456458419\\
63.5	0.093	1.73671272868371	1.73671272868371\\
63.5	0.09666	1.5234858254025	1.5234858254025\\
63.5	0.10032	1.33612391202034	1.33612391202034\\
63.5	0.10398	1.1746269885372	1.1746269885372\\
63.5	0.10764	1.03899505495311	1.03899505495311\\
63.5	0.1113	0.929228111268054	0.929228111268054\\
63.5	0.11496	0.845326157482038	0.845326157482038\\
63.5	0.11862	0.787289193595056	0.787289193595056\\
63.5	0.12228	0.755117219607115	0.755117219607115\\
63.5	0.12594	0.74881023551821	0.74881023551821\\
63.5	0.1296	0.768368241328339	0.768368241328339\\
63.5	0.13326	0.813791237037524	0.813791237037524\\
63.5	0.13692	0.885079222645732	0.885079222645732\\
63.5	0.14058	0.982232198152971	0.982232198152971\\
63.5	0.14424	1.10525016355927	1.10525016355927\\
63.5	0.1479	1.25413311886459	1.25413311886459\\
63.5	0.15156	1.42888106406895	1.42888106406895\\
63.5	0.15522	1.62949399917235	1.62949399917235\\
63.5	0.15888	1.85597192417478	1.85597192417478\\
63.5	0.16254	2.10831483907627	2.10831483907627\\
63.5	0.1662	2.38652274387678	2.38652274387678\\
63.5	0.16986	2.69059563857633	2.69059563857633\\
63.5	0.17352	3.02053352317493	3.02053352317493\\
63.5	0.17718	3.37633639767255	3.37633639767255\\
63.5	0.18084	3.75800426206921	3.75800426206921\\
63.5	0.1845	4.16553711636492	4.16553711636492\\
63.5	0.18816	4.59893496055965	4.59893496055965\\
63.5	0.19182	5.05819779465343	5.05819779465343\\
63.5	0.19548	5.54332561864624	5.54332561864624\\
63.5	0.19914	6.0543184325381	6.0543184325381\\
63.5	0.2028	6.591176236329	6.591176236329\\
63.5	0.20646	7.15389903001893	7.15389903001893\\
63.5	0.21012	7.74248681360788	7.74248681360788\\
63.5	0.21378	8.35693958709589	8.35693958709589\\
63.5	0.21744	8.99725735048294	8.99725735048294\\
63.5	0.2211	9.66344010376901	9.66344010376901\\
63.5	0.22476	10.3554878469541	10.3554878469541\\
63.5	0.22842	11.0734005800383	11.0734005800383\\
63.5	0.23208	11.8171783030215	11.8171783030215\\
63.5	0.23574	12.5868210159037	12.5868210159037\\
63.5	0.2394	13.382328718685	13.382328718685\\
63.5	0.24306	14.2037014113653	14.2037014113653\\
63.5	0.24672	15.0509390939446	15.0509390939446\\
63.5	0.25038	15.924041766423	15.924041766423\\
63.5	0.25404	16.8230094288004	16.8230094288004\\
63.5	0.2577	17.7478420810769	17.7478420810769\\
63.5	0.26136	18.6985397232524	18.6985397232524\\
63.5	0.26502	19.6751023553269	19.6751023553269\\
63.5	0.26868	20.6775299773005	20.6775299773005\\
63.5	0.27234	21.7058225891731	21.7058225891731\\
63.5	0.276	22.7599801909447	22.7599801909447\\
63.875	0.093	1.90465739445485	1.90465739445485\\
63.875	0.09666	1.68261065301011	1.68261065301011\\
63.875	0.10032	1.48642890146439	1.48642890146439\\
63.875	0.10398	1.31611213981772	1.31611213981772\\
63.875	0.10764	1.1716603680701	1.1716603680701\\
63.875	0.1113	1.05307358622148	1.05307358622148\\
63.875	0.11496	0.960351794271919	0.960351794271919\\
63.875	0.11862	0.89349499222139	0.89349499222139\\
63.875	0.12228	0.852503180069917	0.852503180069917\\
63.875	0.12594	0.837376357817465	0.837376357817465\\
63.875	0.1296	0.848114525464048	0.848114525464048\\
63.875	0.13326	0.884717683009686	0.884717683009686\\
63.875	0.13692	0.947185830454348	0.947185830454348\\
63.875	0.14058	1.03551896779805	1.03551896779805\\
63.875	0.14424	1.1497170950408	1.1497170950408\\
63.875	0.1479	1.28978021218257	1.28978021218257\\
63.875	0.15156	1.45570831922338	1.45570831922338\\
63.875	0.15522	1.64750141616323	1.64750141616323\\
63.875	0.15888	1.86515950300215	1.86515950300215\\
63.875	0.16254	2.10868257974006	2.10868257974006\\
63.875	0.1662	2.37807064637702	2.37807064637702\\
63.875	0.16986	2.67332370291304	2.67332370291304\\
63.875	0.17352	2.99444174934808	2.99444174934808\\
63.875	0.17718	3.34142478568215	3.34142478568215\\
63.875	0.18084	3.71427281191527	3.71427281191527\\
63.875	0.1845	4.11298582804744	4.11298582804744\\
63.875	0.18816	4.53756383407862	4.53756383407862\\
63.875	0.19182	4.98800683000886	4.98800683000886\\
63.875	0.19548	5.46431481583814	5.46431481583814\\
63.875	0.19914	5.96648779156644	5.96648779156644\\
63.875	0.2028	6.49452575719381	6.49452575719381\\
63.875	0.20646	7.04842871272017	7.04842871272017\\
63.875	0.21012	7.62819665814558	7.62819665814558\\
63.875	0.21378	8.23382959347006	8.23382959347006\\
63.875	0.21744	8.86532751869355	8.86532751869355\\
63.875	0.2211	9.52269043381607	9.52269043381607\\
63.875	0.22476	10.2059183388376	10.2059183388376\\
63.875	0.22842	10.9150112337583	10.9150112337583\\
63.875	0.23208	11.6499691185779	11.6499691185779\\
63.875	0.23574	12.4107919932966	12.4107919932966\\
63.875	0.2394	13.1974798579143	13.1974798579143\\
63.875	0.24306	14.0100327124311	14.0100327124311\\
63.875	0.24672	14.8484505568469	14.8484505568469\\
63.875	0.25038	15.7127333911617	15.7127333911617\\
63.875	0.25404	16.6028812153756	16.6028812153756\\
63.875	0.2577	17.5188940294885	17.5188940294885\\
63.875	0.26136	18.4607718335005	18.4607718335005\\
63.875	0.26502	19.4285146274114	19.4285146274114\\
63.875	0.26868	20.4221224112215	20.4221224112215\\
63.875	0.27234	21.4415951849305	21.4415951849305\\
63.875	0.276	22.4869329485386	22.4869329485386\\
64.25	0.093	2.07803108529415	2.07803108529415\\
64.25	0.09666	1.84716450568587	1.84716450568587\\
64.25	0.10032	1.64216291597661	1.64216291597661\\
64.25	0.10398	1.4630263161664	1.4630263161664\\
64.25	0.10764	1.30975470625521	1.30975470625521\\
64.25	0.1113	1.18234808624306	1.18234808624306\\
64.25	0.11496	1.08080645612995	1.08080645612995\\
64.25	0.11862	1.00512981591588	1.00512981591588\\
64.25	0.12228	0.955318165600843	0.955318165600843\\
64.25	0.12594	0.93137150518486	0.93137150518486\\
64.25	0.1296	0.93328983466791	0.93328983466791\\
64.25	0.13326	0.961073154049988	0.961073154049988\\
64.25	0.13692	1.0147214633311	1.0147214633311\\
64.25	0.14058	1.09423476251126	1.09423476251126\\
64.25	0.14424	1.19961305159046	1.19961305159046\\
64.25	0.1479	1.33085633056869	1.33085633056869\\
64.25	0.15156	1.48796459944595	1.48796459944595\\
64.25	0.15522	1.67093785822227	1.67093785822227\\
64.25	0.15888	1.87977610689761	1.87977610689761\\
64.25	0.16254	2.114479345472	2.114479345472\\
64.25	0.1662	2.37504757394542	2.37504757394542\\
64.25	0.16986	2.66148079231787	2.66148079231787\\
64.25	0.17352	2.97377900058938	2.97377900058938\\
64.25	0.17718	3.31194219875991	3.31194219875991\\
64.25	0.18084	3.67597038682948	3.67597038682948\\
64.25	0.1845	4.06586356479809	4.06586356479809\\
64.25	0.18816	4.48162173266574	4.48162173266574\\
64.25	0.19182	4.92324489043242	4.92324489043242\\
64.25	0.19548	5.39073303809817	5.39073303809817\\
64.25	0.19914	5.88408617566292	5.88408617566292\\
64.25	0.2028	6.40330430312674	6.40330430312674\\
64.25	0.20646	6.94838742048956	6.94838742048956\\
64.25	0.21012	7.51933552775144	7.51933552775144\\
64.25	0.21378	8.11614862491235	8.11614862491235\\
64.25	0.21744	8.73882671197231	8.73882671197231\\
64.25	0.2211	9.38736978893128	9.38736978893128\\
64.25	0.22476	10.0617778557893	10.0617778557893\\
64.25	0.22842	10.7620509125464	10.7620509125464\\
64.25	0.23208	11.4881889592025	11.4881889592025\\
64.25	0.23574	12.2401919957576	12.2401919957576\\
64.25	0.2394	13.0180600222118	13.0180600222118\\
64.25	0.24306	13.821793038565	13.821793038565\\
64.25	0.24672	14.6513910448173	14.6513910448173\\
64.25	0.25038	15.5068540409686	15.5068540409686\\
64.25	0.25404	16.3881820270189	16.3881820270189\\
64.25	0.2577	17.2953750029683	17.2953750029683\\
64.25	0.26136	18.2284329688167	18.2284329688167\\
64.25	0.26502	19.1873559245641	19.1873559245641\\
64.25	0.26868	20.1721438702106	20.1721438702106\\
64.25	0.27234	21.1827968057561	21.1827968057561\\
64.25	0.276	22.2193147312007	22.2193147312007\\
64.625	0.093	2.25683380120159	2.25683380120159\\
64.625	0.09666	2.01714738342976	2.01714738342976\\
64.625	0.10032	1.80332595555695	1.80332595555695\\
64.625	0.10398	1.61536951758318	1.61536951758318\\
64.625	0.10764	1.45327806950846	1.45327806950846\\
64.625	0.1113	1.31705161133278	1.31705161133278\\
64.625	0.11496	1.20669014305611	1.20669014305611\\
64.625	0.11862	1.1221936646785	1.1221936646785\\
64.625	0.12228	1.06356217619992	1.06356217619992\\
64.625	0.12594	1.03079567762038	1.03079567762038\\
64.625	0.1296	1.02389416893988	1.02389416893988\\
64.625	0.13326	1.04285765015843	1.04285765015843\\
64.625	0.13692	1.087686121276	1.087686121276\\
64.625	0.14058	1.15837958229261	1.15837958229261\\
64.625	0.14424	1.25493803320826	1.25493803320826\\
64.625	0.1479	1.37736147402294	1.37736147402294\\
64.625	0.15156	1.52564990473667	1.52564990473667\\
64.625	0.15522	1.69980332534943	1.69980332534943\\
64.625	0.15888	1.89982173586124	1.89982173586124\\
64.625	0.16254	2.12570513627207	2.12570513627207\\
64.625	0.1662	2.37745352658194	2.37745352658194\\
64.625	0.16986	2.65506690679086	2.65506690679086\\
64.625	0.17352	2.95854527689881	2.95854527689881\\
64.625	0.17718	3.28788863690581	3.28788863690581\\
64.625	0.18084	3.64309698681183	3.64309698681183\\
64.625	0.1845	4.02417032661691	4.02417032661691\\
64.625	0.18816	4.43110865632099	4.43110865632099\\
64.625	0.19182	4.86391197592415	4.86391197592415\\
64.625	0.19548	5.32258028542633	5.32258028542633\\
64.625	0.19914	5.80711358482753	5.80711358482753\\
64.625	0.2028	6.31751187412781	6.31751187412781\\
64.625	0.20646	6.85377515332708	6.85377515332708\\
64.625	0.21012	7.41590342242542	7.41590342242542\\
64.625	0.21378	8.00389668142278	8.00389668142278\\
64.625	0.21744	8.61775493031919	8.61775493031919\\
64.625	0.2211	9.25747816911463	9.25747816911463\\
64.625	0.22476	9.92306639780912	9.92306639780912\\
64.625	0.22842	10.6145196164027	10.6145196164027\\
64.625	0.23208	11.3318378248952	11.3318378248952\\
64.625	0.23574	12.0750210232868	12.0750210232868\\
64.625	0.2394	12.8440692115774	12.8440692115774\\
64.625	0.24306	13.6389823897671	13.6389823897671\\
64.625	0.24672	14.4597605578558	14.4597605578558\\
64.625	0.25038	15.3064037158436	15.3064037158436\\
64.625	0.25404	16.1789118637303	16.1789118637303\\
64.625	0.2577	17.0772850015162	17.0772850015162\\
64.625	0.26136	18.001523129201	18.001523129201\\
64.625	0.26502	18.9516262467849	18.9516262467849\\
64.625	0.26868	19.9275943542679	19.9275943542679\\
64.625	0.27234	20.9294274516498	20.9294274516498\\
64.625	0.276	21.9571255389308	21.9571255389308\\
65	0.093	2.44106554217719	2.44106554217719\\
65	0.09666	2.19255928624182	2.19255928624182\\
65	0.10032	1.96991802020546	1.96991802020546\\
65	0.10398	1.77314174406814	1.77314174406814\\
65	0.10764	1.60223045782988	1.60223045782988\\
65	0.1113	1.45718416149064	1.45718416149064\\
65	0.11496	1.33800285505044	1.33800285505044\\
65	0.11862	1.24468653850928	1.24468653850928\\
65	0.12228	1.17723521186716	1.17723521186716\\
65	0.12594	1.13564887512407	1.13564887512407\\
65	0.1296	1.11992752828002	1.11992752828002\\
65	0.13326	1.13007117133502	1.13007117133502\\
65	0.13692	1.16607980428904	1.16607980428904\\
65	0.14058	1.22795342714211	1.22795342714211\\
65	0.14424	1.31569203989421	1.31569203989421\\
65	0.1479	1.42929564254535	1.42929564254535\\
65	0.15156	1.56876423509554	1.56876423509554\\
65	0.15522	1.73409781754475	1.73409781754475\\
65	0.15888	1.92529638989302	1.92529638989302\\
65	0.16254	2.14235995214029	2.14235995214029\\
65	0.1662	2.38528850428663	2.38528850428663\\
65	0.16986	2.65408204633201	2.65408204633201\\
65	0.17352	2.94874057827641	2.94874057827641\\
65	0.17718	3.26926410011986	3.26926410011986\\
65	0.18084	3.61565261186233	3.61565261186233\\
65	0.1845	3.98790611350385	3.98790611350385\\
65	0.18816	4.3860246050444	4.3860246050444\\
65	0.19182	4.810008086484	4.810008086484\\
65	0.19548	5.25985655782265	5.25985655782265\\
65	0.19914	5.7355700190603	5.7355700190603\\
65	0.2028	6.23714847019703	6.23714847019703\\
65	0.20646	6.76459191123278	6.76459191123278\\
65	0.21012	7.31790034216755	7.31790034216755\\
65	0.21378	7.89707376300138	7.89707376300138\\
65	0.21744	8.50211217373425	8.50211217373425\\
65	0.2211	9.13301557436614	9.13301557436614\\
65	0.22476	9.78978396489707	9.78978396489707\\
65	0.22842	10.472417345327	10.472417345327\\
65	0.23208	11.1809157156561	11.1809157156561\\
65	0.23574	11.9152790758841	11.9152790758841\\
65	0.2394	12.6755074260112	12.6755074260112\\
65	0.24306	13.4616007660373	13.4616007660373\\
65	0.24672	14.2735590959625	14.2735590959625\\
65	0.25038	15.1113824157867	15.1113824157867\\
65	0.25404	15.9750707255099	15.9750707255099\\
65	0.2577	16.8646240251322	16.8646240251322\\
65	0.26136	17.7800423146535	17.7800423146535\\
65	0.26502	18.7213255940739	18.7213255940739\\
65	0.26868	19.6884738633932	19.6884738633932\\
65	0.27234	20.6814871226117	20.6814871226117\\
65	0.276	21.7003653717291	21.7003653717291\\
65.375	0.093	2.63072630822095	2.63072630822095\\
65.375	0.09666	2.37340021412201	2.37340021412201\\
65.375	0.10032	2.14193910992211	2.14193910992211\\
65.375	0.10398	1.93634299562126	1.93634299562126\\
65.375	0.10764	1.75661187121945	1.75661187121945\\
65.375	0.1113	1.60274573671667	1.60274573671667\\
65.375	0.11496	1.47474459211292	1.47474459211292\\
65.375	0.11862	1.37260843740822	1.37260843740822\\
65.375	0.12228	1.29633727260253	1.29633727260253\\
65.375	0.12594	1.24593109769591	1.24593109769591\\
65.375	0.1296	1.22138991268832	1.22138991268832\\
65.375	0.13326	1.22271371757977	1.22271371757977\\
65.375	0.13692	1.24990251237024	1.24990251237024\\
65.375	0.14058	1.30295629705977	1.30295629705977\\
65.375	0.14424	1.38187507164833	1.38187507164833\\
65.375	0.1479	1.48665883613592	1.48665883613592\\
65.375	0.15156	1.61730759052255	1.61730759052255\\
65.375	0.15522	1.77382133480823	1.77382133480823\\
65.375	0.15888	1.95620006899293	1.95620006899293\\
65.375	0.16254	2.16444379307668	2.16444379307668\\
65.375	0.1662	2.39855250705947	2.39855250705947\\
65.375	0.16986	2.65852621094129	2.65852621094129\\
65.375	0.17352	2.94436490472216	2.94436490472216\\
65.375	0.17718	3.25606858840206	3.25606858840206\\
65.375	0.18084	3.59363726198099	3.59363726198099\\
65.375	0.1845	3.95707092545898	3.95707092545898\\
65.375	0.18816	4.34636957883597	4.34636957883597\\
65.375	0.19182	4.76153322211203	4.76153322211203\\
65.375	0.19548	5.20256185528712	5.20256185528712\\
65.375	0.19914	5.66945547836124	5.66945547836124\\
65.375	0.2028	6.16221409133441	6.16221409133441\\
65.375	0.20646	6.68083769420661	6.68083769420661\\
65.375	0.21012	7.22532628697785	7.22532628697785\\
65.375	0.21378	7.79567986964812	7.79567986964812\\
65.375	0.21744	8.39189844221744	8.39189844221744\\
65.375	0.2211	9.0139820046858	9.0139820046858\\
65.375	0.22476	9.66193055705317	9.66193055705317\\
65.375	0.22842	10.3357440993196	10.3357440993196\\
65.375	0.23208	11.0354226314851	11.0354226314851\\
65.375	0.23574	11.7609661535496	11.7609661535496\\
65.375	0.2394	12.5123746655131	12.5123746655131\\
65.375	0.24306	13.2896481673757	13.2896481673757\\
65.375	0.24672	14.0927866591373	14.0927866591373\\
65.375	0.25038	14.921790140798	14.921790140798\\
65.375	0.25404	15.7766586123577	15.7766586123577\\
65.375	0.2577	16.6573920738164	16.6573920738164\\
65.375	0.26136	17.5639905251742	17.5639905251742\\
65.375	0.26502	18.496453966431	18.496453966431\\
65.375	0.26868	19.4547823975868	19.4547823975868\\
65.375	0.27234	20.4389758186417	20.4389758186417\\
65.375	0.276	21.4490342295956	21.4490342295956\\
65.75	0.093	2.82581609933282	2.82581609933282\\
65.75	0.09666	2.55967016707035	2.55967016707035\\
65.75	0.10032	2.3193892247069	2.3193892247069\\
65.75	0.10398	2.10497327224251	2.10497327224251\\
65.75	0.10764	1.91642230967713	1.91642230967713\\
65.75	0.1113	1.75373633701082	1.75373633701082\\
65.75	0.11496	1.61691535424352	1.61691535424352\\
65.75	0.11862	1.50595936137527	1.50595936137527\\
65.75	0.12228	1.42086835840605	1.42086835840605\\
65.75	0.12594	1.36164234533588	1.36164234533588\\
65.75	0.1296	1.32828132216475	1.32828132216475\\
65.75	0.13326	1.32078528889264	1.32078528889264\\
65.75	0.13692	1.33915424551958	1.33915424551958\\
65.75	0.14058	1.38338819204556	1.38338819204556\\
65.75	0.14424	1.45348712847058	1.45348712847058\\
65.75	0.1479	1.54945105479462	1.54945105479462\\
65.75	0.15156	1.6712799710177	1.6712799710177\\
65.75	0.15522	1.81897387713984	1.81897387713984\\
65.75	0.15888	1.99253277316102	1.99253277316102\\
65.75	0.16254	2.1919566590812	2.1919566590812\\
65.75	0.1662	2.41724553490044	2.41724553490044\\
65.75	0.16986	2.66839940061872	2.66839940061872\\
65.75	0.17352	2.94541825623605	2.94541825623605\\
65.75	0.17718	3.24830210175239	3.24830210175239\\
65.75	0.18084	3.57705093716777	3.57705093716777\\
65.75	0.1845	3.93166476248223	3.93166476248223\\
65.75	0.18816	4.31214357769569	4.31214357769569\\
65.75	0.19182	4.71848738280818	4.71848738280818\\
65.75	0.19548	5.15069617781972	5.15069617781972\\
65.75	0.19914	5.60876996273031	5.60876996273031\\
65.75	0.2028	6.09270873753994	6.09270873753994\\
65.75	0.20646	6.60251250224858	6.60251250224858\\
65.75	0.21012	7.13818125685627	7.13818125685627\\
65.75	0.21378	7.69971500136301	7.69971500136301\\
65.75	0.21744	8.28711373576878	8.28711373576878\\
65.75	0.2211	8.90037746007359	8.90037746007359\\
65.75	0.22476	9.53950617427742	9.53950617427742\\
65.75	0.22842	10.2044998783803	10.2044998783803\\
65.75	0.23208	10.8953585723822	10.8953585723822\\
65.75	0.23574	11.6120822562832	11.6120822562832\\
65.75	0.2394	12.3546709300832	12.3546709300832\\
65.75	0.24306	13.1231245937822	13.1231245937822\\
65.75	0.24672	13.9174432473803	13.9174432473803\\
65.75	0.25038	14.7376268908774	14.7376268908774\\
65.75	0.25404	15.5836755242735	15.5836755242735\\
65.75	0.2577	16.4555891475688	16.4555891475688\\
65.75	0.26136	17.3533677607629	17.3533677607629\\
65.75	0.26502	18.2770113638562	18.2770113638562\\
65.75	0.26868	19.2265199568485	19.2265199568485\\
65.75	0.27234	20.2018935397398	20.2018935397398\\
65.75	0.276	21.2031321125302	21.2031321125302\\
66.125	0.093	3.02633491551286	3.02633491551286\\
66.125	0.09666	2.75136914508684	2.75136914508684\\
66.125	0.10032	2.50226836455986	2.50226836455986\\
66.125	0.10398	2.27903257393192	2.27903257393192\\
66.125	0.10764	2.08166177320301	2.08166177320301\\
66.125	0.1113	1.91015596237312	1.91015596237312\\
66.125	0.11496	1.76451514144229	1.76451514144229\\
66.125	0.11862	1.6447393104105	1.6447393104105\\
66.125	0.12228	1.55082846927773	1.55082846927773\\
66.125	0.12594	1.48278261804402	1.48278261804402\\
66.125	0.1296	1.44060175670932	1.44060175670932\\
66.125	0.13326	1.42428588527368	1.42428588527368\\
66.125	0.13692	1.43383500373708	1.43383500373708\\
66.125	0.14058	1.4692491120995	1.4692491120995\\
66.125	0.14424	1.53052821036098	1.53052821036098\\
66.125	0.1479	1.61767229852148	1.61767229852148\\
66.125	0.15156	1.73068137658102	1.73068137658102\\
66.125	0.15522	1.8695554445396	1.8695554445396\\
66.125	0.15888	2.03429450239722	2.03429450239722\\
66.125	0.16254	2.22489855015388	2.22489855015388\\
66.125	0.1662	2.44136758780958	2.44136758780958\\
66.125	0.16986	2.68370161536431	2.68370161536431\\
66.125	0.17352	2.95190063281807	2.95190063281807\\
66.125	0.17718	3.24596464017089	3.24596464017089\\
66.125	0.18084	3.56589363742272	3.56589363742272\\
66.125	0.1845	3.91168762457362	3.91168762457362\\
66.125	0.18816	4.28334660162354	4.28334660162354\\
66.125	0.19182	4.68087056857248	4.68087056857248\\
66.125	0.19548	5.10425952542048	5.10425952542048\\
66.125	0.19914	5.55351347216752	5.55351347216752\\
66.125	0.2028	6.02863240881361	6.02863240881361\\
66.125	0.20646	6.52961633535872	6.52961633535872\\
66.125	0.21012	7.05646525180285	7.05646525180285\\
66.125	0.21378	7.60917915814604	7.60917915814604\\
66.125	0.21744	8.18775805438827	8.18775805438827\\
66.125	0.2211	8.79220194052954	8.79220194052954\\
66.125	0.22476	9.42251081656984	9.42251081656984\\
66.125	0.22842	10.0786846825092	10.0786846825092\\
66.125	0.23208	10.7607235383475	10.7607235383475\\
66.125	0.23574	11.468627384085	11.468627384085\\
66.125	0.2394	12.2023962197214	12.2023962197214\\
66.125	0.24306	12.9620300452569	12.9620300452569\\
66.125	0.24672	13.7475288606914	13.7475288606914\\
66.125	0.25038	14.558892666025	14.558892666025\\
66.125	0.25404	15.3961214612576	15.3961214612576\\
66.125	0.2577	16.2592152463892	16.2592152463892\\
66.125	0.26136	17.1481740214199	17.1481740214199\\
66.125	0.26502	18.0629977863496	18.0629977863496\\
66.125	0.26868	19.0036865411784	19.0036865411784\\
66.125	0.27234	19.9702402859062	19.9702402859062\\
66.125	0.276	20.962659020533	20.962659020533\\
66.5	0.093	3.23228275676106	3.23228275676106\\
66.5	0.09666	2.94849714817148	2.94849714817148\\
66.5	0.10032	2.69057652948094	2.69057652948094\\
66.5	0.10398	2.45852090068947	2.45852090068947\\
66.5	0.10764	2.252330261797	2.252330261797\\
66.5	0.1113	2.07200461280359	2.07200461280359\\
66.5	0.11496	1.9175439537092	1.9175439537092\\
66.5	0.11862	1.78894828451386	1.78894828451386\\
66.5	0.12228	1.68621760521755	1.68621760521755\\
66.5	0.12594	1.6093519158203	1.6093519158203\\
66.5	0.1296	1.55835121632206	1.55835121632206\\
66.5	0.13326	1.53321550672287	1.53321550672287\\
66.5	0.13692	1.53394478702272	1.53394478702272\\
66.5	0.14058	1.5605390572216	1.5605390572216\\
66.5	0.14424	1.61299831731953	1.61299831731953\\
66.5	0.1479	1.69132256731648	1.69132256731648\\
66.5	0.15156	1.79551180721248	1.79551180721248\\
66.5	0.15522	1.92556603700751	1.92556603700751\\
66.5	0.15888	2.08148525670158	2.08148525670158\\
66.5	0.16254	2.2632694662947	2.2632694662947\\
66.5	0.1662	2.47091866578685	2.47091866578685\\
66.5	0.16986	2.70443285517804	2.70443285517804\\
66.5	0.17352	2.96381203446827	2.96381203446827\\
66.5	0.17718	3.24905620365753	3.24905620365753\\
66.5	0.18084	3.56016536274584	3.56016536274584\\
66.5	0.1845	3.89713951173317	3.89713951173317\\
66.5	0.18816	4.25997865061954	4.25997865061954\\
66.5	0.19182	4.64868277940494	4.64868277940494\\
66.5	0.19548	5.06325189808941	5.06325189808941\\
66.5	0.19914	5.50368600667289	5.50368600667289\\
66.5	0.2028	5.96998510515542	5.96998510515542\\
66.5	0.20646	6.46214919353699	6.46214919353699\\
66.5	0.21012	6.98017827181759	6.98017827181759\\
66.5	0.21378	7.52407233999724	7.52407233999724\\
66.5	0.21744	8.09383139807591	8.09383139807591\\
66.5	0.2211	8.68945544605363	8.68945544605363\\
66.5	0.22476	9.31094448393037	9.31094448393037\\
66.5	0.22842	9.95829851170618	9.95829851170618\\
66.5	0.23208	10.631517529381	10.631517529381\\
66.5	0.23574	11.3306015369549	11.3306015369549\\
66.5	0.2394	12.0555505344278	12.0555505344278\\
66.5	0.24306	12.8063645217997	12.8063645217997\\
66.5	0.24672	13.5830434990707	13.5830434990707\\
66.5	0.25038	14.3855874662407	14.3855874662407\\
66.5	0.25404	15.2139964233098	15.2139964233098\\
66.5	0.2577	16.0682703702779	16.0682703702779\\
66.5	0.26136	16.948409307145	16.948409307145\\
66.5	0.26502	17.8544132339112	17.8544132339112\\
66.5	0.26868	18.7862821505764	18.7862821505764\\
66.5	0.27234	19.7440160571406	19.7440160571406\\
66.5	0.276	20.7276149536039	20.7276149536039\\
66.875	0.093	3.44365962307737	3.44365962307737\\
66.875	0.09666	3.15105417632426	3.15105417632426\\
66.875	0.10032	2.88431371947019	2.88431371947019\\
66.875	0.10398	2.64343825251514	2.64343825251514\\
66.875	0.10764	2.42842777545915	2.42842777545915\\
66.875	0.1113	2.23928228830217	2.23928228830217\\
66.875	0.11496	2.07600179104425	2.07600179104425\\
66.875	0.11862	1.93858628368537	1.93858628368537\\
66.875	0.12228	1.82703576622552	1.82703576622552\\
66.875	0.12594	1.74135023866471	1.74135023866471\\
66.875	0.1296	1.68152970100292	1.68152970100292\\
66.875	0.13326	1.64757415324019	1.64757415324019\\
66.875	0.13692	1.63948359537649	1.63948359537649\\
66.875	0.14058	1.65725802741183	1.65725802741183\\
66.875	0.14424	1.70089744934621	1.70089744934621\\
66.875	0.1479	1.77040186117963	1.77040186117963\\
66.875	0.15156	1.86577126291208	1.86577126291208\\
66.875	0.15522	1.98700565454356	1.98700565454356\\
66.875	0.15888	2.13410503607409	2.13410503607409\\
66.875	0.16254	2.30706940750366	2.30706940750366\\
66.875	0.1662	2.50589876883226	2.50589876883226\\
66.875	0.16986	2.73059312005991	2.73059312005991\\
66.875	0.17352	2.98115246118659	2.98115246118659\\
66.875	0.17718	3.25757679221231	3.25757679221231\\
66.875	0.18084	3.55986611313704	3.55986611313704\\
66.875	0.1845	3.88802042396084	3.88802042396084\\
66.875	0.18816	4.24203972468367	4.24203972468367\\
66.875	0.19182	4.62192401530555	4.62192401530555\\
66.875	0.19548	5.02767329582645	5.02767329582645\\
66.875	0.19914	5.45928756624639	5.45928756624639\\
66.875	0.2028	5.91676682656537	5.91676682656537\\
66.875	0.20646	6.4001110767834	6.4001110767834\\
66.875	0.21012	6.90932031690045	6.90932031690045\\
66.875	0.21378	7.44439454691655	7.44439454691655\\
66.875	0.21744	8.00533376683168	8.00533376683168\\
66.875	0.2211	8.59213797664586	8.59213797664586\\
66.875	0.22476	9.20480717635907	9.20480717635907\\
66.875	0.22842	9.84334136597132	9.84334136597132\\
66.875	0.23208	10.5077405454826	10.5077405454826\\
66.875	0.23574	11.1980047148929	11.1980047148929\\
66.875	0.2394	11.9141338742023	11.9141338742023\\
66.875	0.24306	12.6561280234107	12.6561280234107\\
66.875	0.24672	13.4239871625181	13.4239871625181\\
66.875	0.25038	14.2177112915246	14.2177112915246\\
66.875	0.25404	15.0373004104301	15.0373004104301\\
66.875	0.2577	15.8827545192347	15.8827545192347\\
66.875	0.26136	16.7540736179382	16.7540736179382\\
66.875	0.26502	17.6512577065408	17.6512577065408\\
66.875	0.26868	18.5743067850425	18.5743067850425\\
66.875	0.27234	19.5232208534432	19.5232208534432\\
66.875	0.276	20.4979999117429	20.4979999117429\\
67.25	0.093	3.66046551446184	3.66046551446184\\
67.25	0.09666	3.3590402295452	3.3590402295452\\
67.25	0.10032	3.08347993452757	3.08347993452757\\
67.25	0.10398	2.83378462940899	2.83378462940899\\
67.25	0.10764	2.60995431418944	2.60995431418944\\
67.25	0.1113	2.41198898886892	2.41198898886892\\
67.25	0.11496	2.23988865344745	2.23988865344745\\
67.25	0.11862	2.09365330792502	2.09365330792502\\
67.25	0.12228	1.97328295230162	1.97328295230162\\
67.25	0.12594	1.87877758657728	1.87877758657728\\
67.25	0.1296	1.81013721075194	1.81013721075194\\
67.25	0.13326	1.76736182482567	1.76736182482567\\
67.25	0.13692	1.75045142879842	1.75045142879842\\
67.25	0.14058	1.75940602267021	1.75940602267021\\
67.25	0.14424	1.79422560644106	1.79422560644106\\
67.25	0.1479	1.85491018011091	1.85491018011091\\
67.25	0.15156	1.94145974367981	1.94145974367981\\
67.25	0.15522	2.05387429714775	2.05387429714775\\
67.25	0.15888	2.19215384051475	2.19215384051475\\
67.25	0.16254	2.35629837378076	2.35629837378076\\
67.25	0.1662	2.54630789694582	2.54630789694582\\
67.25	0.16986	2.76218241000991	2.76218241000991\\
67.25	0.17352	3.00392191297305	3.00392191297305\\
67.25	0.17718	3.27152640583522	3.27152640583522\\
67.25	0.18084	3.56499588859642	3.56499588859642\\
67.25	0.1845	3.88433036125669	3.88433036125669\\
67.25	0.18816	4.22952982381596	4.22952982381596\\
67.25	0.19182	4.6005942762743	4.6005942762743\\
67.25	0.19548	4.99752371863165	4.99752371863165\\
67.25	0.19914	5.42031815088805	5.42031815088805\\
67.25	0.2028	5.8689775730435	5.8689775730435\\
67.25	0.20646	6.34350198509797	6.34350198509797\\
67.25	0.21012	6.84389138705147	6.84389138705147\\
67.25	0.21378	7.37014577890402	7.37014577890402\\
67.25	0.21744	7.92226516065561	7.92226516065561\\
67.25	0.2211	8.50024953230624	8.50024953230624\\
67.25	0.22476	9.10409889385589	9.10409889385589\\
67.25	0.22842	9.73381324530461	9.73381324530461\\
67.25	0.23208	10.3893925866523	10.3893925866523\\
67.25	0.23574	11.0708369178991	11.0708369178991\\
67.25	0.2394	11.7781462390449	11.7781462390449\\
67.25	0.24306	12.5113205500898	12.5113205500898\\
67.25	0.24672	13.2703598510337	13.2703598510337\\
67.25	0.25038	14.0552641418766	14.0552641418766\\
67.25	0.25404	14.8660334226186	14.8660334226186\\
67.25	0.2577	15.7026676932596	15.7026676932596\\
67.25	0.26136	16.5651669537996	16.5651669537996\\
67.25	0.26502	17.4535312042387	17.4535312042387\\
67.25	0.26868	18.3677604445768	18.3677604445768\\
67.25	0.27234	19.307854674814	19.307854674814\\
67.25	0.276	20.2738138949501	20.2738138949501\\
67.625	0.093	3.88270043091446	3.88270043091446\\
67.625	0.09666	3.57245530783426	3.57245530783426\\
67.625	0.10032	3.2880751746531	3.2880751746531\\
67.625	0.10398	3.02956003137097	3.02956003137097\\
67.625	0.10764	2.79690987798788	2.79690987798788\\
67.625	0.1113	2.59012471450381	2.59012471450381\\
67.625	0.11496	2.4092045409188	2.4092045409188\\
67.625	0.11862	2.25414935723282	2.25414935723282\\
67.625	0.12228	2.12495916344588	2.12495916344588\\
67.625	0.12594	2.02163395955798	2.02163395955798\\
67.625	0.1296	1.94417374556911	1.94417374556911\\
67.625	0.13326	1.89257852147928	1.89257852147928\\
67.625	0.13692	1.86684828728849	1.86684828728849\\
67.625	0.14058	1.86698304299674	1.86698304299674\\
67.625	0.14424	1.89298278860404	1.89298278860404\\
67.625	0.1479	1.94484752411034	1.94484752411034\\
67.625	0.15156	2.0225772495157	2.0225772495157\\
67.625	0.15522	2.1261719648201	2.1261719648201\\
67.625	0.15888	2.25563167002355	2.25563167002355\\
67.625	0.16254	2.41095636512602	2.41095636512602\\
67.625	0.1662	2.59214605012754	2.59214605012754\\
67.625	0.16986	2.79920072502809	2.79920072502809\\
67.625	0.17352	3.03212038982767	3.03212038982767\\
67.625	0.17718	3.2909050445263	3.2909050445263\\
67.625	0.18084	3.57555468912397	3.57555468912397\\
67.625	0.1845	3.88606932362067	3.88606932362067\\
67.625	0.18816	4.22244894801639	4.22244894801639\\
67.625	0.19182	4.58469356231117	4.58469356231117\\
67.625	0.19548	4.97280316650498	4.97280316650498\\
67.625	0.19914	5.38677776059785	5.38677776059785\\
67.625	0.2028	5.82661734458974	5.82661734458974\\
67.625	0.20646	6.29232191848067	6.29232191848067\\
67.625	0.21012	6.78389148227063	6.78389148227063\\
67.625	0.21378	7.30132603595964	7.30132603595964\\
67.625	0.21744	7.84462557954768	7.84462557954768\\
67.625	0.2211	8.41379011303476	8.41379011303476\\
67.625	0.22476	9.00881963642088	9.00881963642088\\
67.625	0.22842	9.62971414970604	9.62971414970604\\
67.625	0.23208	10.2764736528902	10.2764736528902\\
67.625	0.23574	10.9490981459735	10.9490981459735\\
67.625	0.2394	11.6475876289557	11.6475876289557\\
67.625	0.24306	12.371942101837	12.371942101837\\
67.625	0.24672	13.1221615646174	13.1221615646174\\
67.625	0.25038	13.8982460172968	13.8982460172968\\
67.625	0.25404	14.7001954598752	14.7001954598752\\
67.625	0.2577	15.5280098923526	15.5280098923526\\
67.625	0.26136	16.3816893147291	16.3816893147291\\
67.625	0.26502	17.2612337270047	17.2612337270047\\
67.625	0.26868	18.1666431291792	18.1666431291792\\
67.625	0.27234	19.0979175212529	19.0979175212529\\
67.625	0.276	20.0550569032255	20.0550569032255\\
68	0.093	4.11036437243524	4.11036437243524\\
68	0.09666	3.79129941119147	3.79129941119147\\
68	0.10032	3.49809943984678	3.49809943984678\\
68	0.10398	3.23076445840109	3.23076445840109\\
68	0.10764	2.98929446685446	2.98929446685446\\
68	0.1113	2.77368946520685	2.77368946520685\\
68	0.11496	2.58394945345829	2.58394945345829\\
68	0.11862	2.42007443160877	2.42007443160877\\
68	0.12228	2.28206439965827	2.28206439965827\\
68	0.12594	2.16991935760683	2.16991935760683\\
68	0.1296	2.08363930545442	2.08363930545442\\
68	0.13326	2.02322424320105	2.02322424320105\\
68	0.13692	1.98867417084671	1.98867417084671\\
68	0.14058	1.97998908839142	1.97998908839142\\
68	0.14424	1.99716899583515	1.99716899583515\\
68	0.1479	2.04021389317794	2.04021389317794\\
68	0.15156	2.10912378041974	2.10912378041974\\
68	0.15522	2.2038986575606	2.2038986575606\\
68	0.15888	2.3245385246005	2.3245385246005\\
68	0.16254	2.47104338153941	2.47104338153941\\
68	0.1662	2.64341322837738	2.64341322837738\\
68	0.16986	2.8416480651144	2.8416480651144\\
68	0.17352	3.06574789175044	3.06574789175044\\
68	0.17718	3.31571270828552	3.31571270828552\\
68	0.18084	3.59154251471964	3.59154251471964\\
68	0.1845	3.89323731105279	3.89323731105279\\
68	0.18816	4.22079709728497	4.22079709728497\\
68	0.19182	4.57422187341621	4.57422187341621\\
68	0.19548	4.9535116394465	4.9535116394465\\
68	0.19914	5.35866639537579	5.35866639537579\\
68	0.2028	5.78968614120414	5.78968614120414\\
68	0.20646	6.24657087693152	6.24657087693152\\
68	0.21012	6.72932060255793	6.72932060255793\\
68	0.21378	7.23793531808339	7.23793531808339\\
68	0.21744	7.7724150235079	7.7724150235079\\
68	0.2211	8.33275971883144	8.33275971883144\\
68	0.22476	8.918969404054	8.918969404054\\
68	0.22842	9.53104407917562	9.53104407917562\\
68	0.23208	10.1689837441963	10.1689837441963\\
68	0.23574	10.8327883991159	10.8327883991159\\
68	0.2394	11.5224580439347	11.5224580439347\\
68	0.24306	12.2379926786524	12.2379926786524\\
68	0.24672	12.9793923032692	12.9793923032692\\
68	0.25038	13.7466569177851	13.7466569177851\\
68	0.25404	14.5397865221999	14.5397865221999\\
68	0.2577	15.3587811165139	15.3587811165139\\
68	0.26136	16.2036407007268	16.2036407007268\\
68	0.26502	17.0743652748388	17.0743652748388\\
68	0.26868	17.9709548388498	17.9709548388498\\
68	0.27234	18.8934093927599	18.8934093927599\\
68	0.276	19.841728936569	19.841728936569\\
68.375	0.093	4.34345733902415	4.34345733902415\\
68.375	0.09666	4.01557253961685	4.01557253961685\\
68.375	0.10032	3.7135527301086	3.7135527301086\\
68.375	0.10398	3.43739791049937	3.43739791049937\\
68.375	0.10764	3.18710808078919	3.18710808078919\\
68.375	0.1113	2.96268324097805	2.96268324097805\\
68.375	0.11496	2.76412339106593	2.76412339106593\\
68.375	0.11862	2.59142853105287	2.59142853105287\\
68.375	0.12228	2.44459866093882	2.44459866093882\\
68.375	0.12594	2.32363378072384	2.32363378072384\\
68.375	0.1296	2.22853389040788	2.22853389040788\\
68.375	0.13326	2.15929898999096	2.15929898999096\\
68.375	0.13692	2.11592907947308	2.11592907947308\\
68.375	0.14058	2.09842415885424	2.09842415885424\\
68.375	0.14424	2.10678422813444	2.10678422813444\\
68.375	0.1479	2.14100928731367	2.14100928731367\\
68.375	0.15156	2.20109933639194	2.20109933639194\\
68.375	0.15522	2.28705437536923	2.28705437536923\\
68.375	0.15888	2.39887440424557	2.39887440424557\\
68.375	0.16254	2.53655942302097	2.53655942302097\\
68.375	0.1662	2.70010943169539	2.70010943169539\\
68.375	0.16986	2.88952443026885	2.88952443026885\\
68.375	0.17352	3.10480441874135	3.10480441874135\\
68.375	0.17718	3.34594939711288	3.34594939711288\\
68.375	0.18084	3.61295936538346	3.61295936538346\\
68.375	0.1845	3.90583432355308	3.90583432355308\\
68.375	0.18816	4.2245742716217	4.2245742716217\\
68.375	0.19182	4.56917920958939	4.56917920958939\\
68.375	0.19548	4.93964913745614	4.93964913745614\\
68.375	0.19914	5.33598405522189	5.33598405522189\\
68.375	0.2028	5.75818396288668	5.75818396288668\\
68.375	0.20646	6.20624886045052	6.20624886045052\\
68.375	0.21012	6.6801787479134	6.6801787479134\\
68.375	0.21378	7.17997362527532	7.17997362527532\\
68.375	0.21744	7.70563349253625	7.70563349253625\\
68.375	0.2211	8.25715834969624	8.25715834969624\\
68.375	0.22476	8.83454819675526	8.83454819675526\\
68.375	0.22842	9.43780303371336	9.43780303371336\\
68.375	0.23208	10.0669228605704	10.0669228605704\\
68.375	0.23574	10.7219076773266	10.7219076773266\\
68.375	0.2394	11.4027574839818	11.4027574839818\\
68.375	0.24306	12.109472280536	12.109472280536\\
68.375	0.24672	12.8420520669892	12.8420520669892\\
68.375	0.25038	13.6004968433415	13.6004968433415\\
68.375	0.25404	14.3848066095928	14.3848066095928\\
68.375	0.2577	15.1949813657432	15.1949813657432\\
68.375	0.26136	16.0310211117926	16.0310211117926\\
68.375	0.26502	16.8929258477411	16.8929258477411\\
68.375	0.26868	17.7806955735885	17.7806955735885\\
68.375	0.27234	18.6943302893351	18.6943302893351\\
68.375	0.276	19.6338299949806	19.6338299949806\\
68.75	0.093	4.5819793306812	4.5819793306812\\
68.75	0.09666	4.24527469311036	4.24527469311036\\
68.75	0.10032	3.93443504543856	3.93443504543856\\
68.75	0.10398	3.64946038766579	3.64946038766579\\
68.75	0.10764	3.39035071979206	3.39035071979206\\
68.75	0.1113	3.15710604181737	3.15710604181737\\
68.75	0.11496	2.94972635374171	2.94972635374171\\
68.75	0.11862	2.76821165556511	2.76821165556511\\
68.75	0.12228	2.61256194728752	2.61256194728752\\
68.75	0.12594	2.48277722890899	2.48277722890899\\
68.75	0.1296	2.37885750042947	2.37885750042947\\
68.75	0.13326	2.30080276184902	2.30080276184902\\
68.75	0.13692	2.24861301316759	2.24861301316759\\
68.75	0.14058	2.22228825438521	2.22228825438521\\
68.75	0.14424	2.22182848550186	2.22182848550186\\
68.75	0.1479	2.24723370651754	2.24723370651754\\
68.75	0.15156	2.29850391743225	2.29850391743225\\
68.75	0.15522	2.37563911824602	2.37563911824602\\
68.75	0.15888	2.47863930895882	2.47863930895882\\
68.75	0.16254	2.60750448957065	2.60750448957065\\
68.75	0.1662	2.76223466008154	2.76223466008154\\
68.75	0.16986	2.94282982049145	2.94282982049145\\
68.75	0.17352	3.1492899708004	3.1492899708004\\
68.75	0.17718	3.38161511100839	3.38161511100839\\
68.75	0.18084	3.63980524111543	3.63980524111543\\
68.75	0.1845	3.92386036112148	3.92386036112148\\
68.75	0.18816	4.23378047102657	4.23378047102657\\
68.75	0.19182	4.56956557083073	4.56956557083073\\
68.75	0.19548	4.93121566053392	4.93121566053392\\
68.75	0.19914	5.31873074013612	5.31873074013612\\
68.75	0.2028	5.73211080963737	5.73211080963737\\
68.75	0.20646	6.17135586903768	6.17135586903768\\
68.75	0.21012	6.636465918337	6.636465918337\\
68.75	0.21378	7.12744095753536	7.12744095753536\\
68.75	0.21744	7.64428098663278	7.64428098663278\\
68.75	0.2211	8.18698600562921	8.18698600562921\\
68.75	0.22476	8.7555560145247	8.7555560145247\\
68.75	0.22842	9.3499910133192	9.3499910133192\\
68.75	0.23208	9.97029100201277	9.97029100201277\\
68.75	0.23574	10.6164559806054	10.6164559806054\\
68.75	0.2394	11.288485949097	11.288485949097\\
68.75	0.24306	11.9863809074877	11.9863809074877\\
68.75	0.24672	12.7101408557774	12.7101408557774\\
68.75	0.25038	13.4597657939661	13.4597657939661\\
68.75	0.25404	14.2352557220539	14.2352557220539\\
68.75	0.2577	15.0366106400407	15.0366106400407\\
68.75	0.26136	15.8638305479266	15.8638305479266\\
68.75	0.26502	16.7169154457115	16.7169154457115\\
68.75	0.26868	17.5958653333954	17.5958653333954\\
68.75	0.27234	18.5006802109784	18.5006802109784\\
68.75	0.276	19.4313600784604	19.4313600784604\\
69.125	0.093	4.82593034740639	4.82593034740639\\
69.125	0.09666	4.480405871672	4.480405871672\\
69.125	0.10032	4.16074638583665	4.16074638583665\\
69.125	0.10398	3.86695188990035	3.86695188990035\\
69.125	0.10764	3.59902238386308	3.59902238386308\\
69.125	0.1113	3.35695786772484	3.35695786772484\\
69.125	0.11496	3.14075834148564	3.14075834148564\\
69.125	0.11862	2.95042380514547	2.95042380514547\\
69.125	0.12228	2.78595425870435	2.78595425870435\\
69.125	0.12594	2.64734970216228	2.64734970216228\\
69.125	0.1296	2.53461013551922	2.53461013551922\\
69.125	0.13326	2.44773555877521	2.44773555877521\\
69.125	0.13692	2.38672597193024	2.38672597193024\\
69.125	0.14058	2.35158137498431	2.35158137498431\\
69.125	0.14424	2.34230176793741	2.34230176793741\\
69.125	0.1479	2.35888715078955	2.35888715078955\\
69.125	0.15156	2.40133752354073	2.40133752354073\\
69.125	0.15522	2.46965288619094	2.46965288619094\\
69.125	0.15888	2.5638332387402	2.5638332387402\\
69.125	0.16254	2.68387858118849	2.68387858118849\\
69.125	0.1662	2.82978891353582	2.82978891353582\\
69.125	0.16986	3.0015642357822	3.0015642357822\\
69.125	0.17352	3.1992045479276	3.1992045479276\\
69.125	0.17718	3.42270984997204	3.42270984997204\\
69.125	0.18084	3.67208014191553	3.67208014191553\\
69.125	0.1845	3.94731542375806	3.94731542375806\\
69.125	0.18816	4.24841569549958	4.24841569549958\\
69.125	0.19182	4.5753809571402	4.5753809571402\\
69.125	0.19548	4.92821120867983	4.92821120867983\\
69.125	0.19914	5.30690645011849	5.30690645011849\\
69.125	0.2028	5.71146668145622	5.71146668145622\\
69.125	0.20646	6.14189190269295	6.14189190269295\\
69.125	0.21012	6.59818211382871	6.59818211382871\\
69.125	0.21378	7.08033731486356	7.08033731486356\\
69.125	0.21744	7.58835750579742	7.58835750579742\\
69.125	0.2211	8.12224268663032	8.12224268663032\\
69.125	0.22476	8.68199285736225	8.68199285736225\\
69.125	0.22842	9.26760801799322	9.26760801799322\\
69.125	0.23208	9.87908816852323	9.87908816852323\\
69.125	0.23574	10.5164333089523	10.5164333089523\\
69.125	0.2394	11.1796434392804	11.1796434392804\\
69.125	0.24306	11.8687185595075	11.8687185595075\\
69.125	0.24672	12.5836586696337	12.5836586696337\\
69.125	0.25038	13.3244637696589	13.3244637696589\\
69.125	0.25404	14.0911338595831	14.0911338595831\\
69.125	0.2577	14.8836689394064	14.8836689394064\\
69.125	0.26136	15.7020690091287	15.7020690091287\\
69.125	0.26502	16.54633406875	16.54633406875\\
69.125	0.26868	17.4164641182704	17.4164641182704\\
69.125	0.27234	18.3124591576898	18.3124591576898\\
69.125	0.276	19.2343191870083	19.2343191870083\\
69.5	0.093	5.07531038919975	5.07531038919975\\
69.5	0.09666	4.72096607530182	4.72096607530182\\
69.5	0.10032	4.39248675130293	4.39248675130293\\
69.5	0.10398	4.08987241720307	4.08987241720307\\
69.5	0.10764	3.81312307300225	3.81312307300225\\
69.5	0.1113	3.56223871870047	3.56223871870047\\
69.5	0.11496	3.33721935429772	3.33721935429772\\
69.5	0.11862	3.13806497979402	3.13806497979402\\
69.5	0.12228	2.96477559518934	2.96477559518934\\
69.5	0.12594	2.81735120048372	2.81735120048372\\
69.5	0.1296	2.69579179567712	2.69579179567712\\
69.5	0.13326	2.60009738076956	2.60009738076956\\
69.5	0.13692	2.53026795576105	2.53026795576105\\
69.5	0.14058	2.48630352065156	2.48630352065156\\
69.5	0.14424	2.46820407544113	2.46820407544113\\
69.5	0.1479	2.47596962012971	2.47596962012971\\
69.5	0.15156	2.50960015471735	2.50960015471735\\
69.5	0.15522	2.56909567920402	2.56909567920402\\
69.5	0.15888	2.65445619358974	2.65445619358974\\
69.5	0.16254	2.76568169787448	2.76568169787448\\
69.5	0.1662	2.90277219205828	2.90277219205828\\
69.5	0.16986	3.06572767614109	3.06572767614109\\
69.5	0.17352	3.25454815012295	3.25454815012295\\
69.5	0.17718	3.46923361400386	3.46923361400386\\
69.5	0.18084	3.70978406778379	3.70978406778379\\
69.5	0.1845	3.97619951146278	3.97619951146278\\
69.5	0.18816	4.26847994504077	4.26847994504077\\
69.5	0.19182	4.58662536851784	4.58662536851784\\
69.5	0.19548	4.9306357818939	4.9306357818939\\
69.5	0.19914	5.30051118516903	5.30051118516903\\
69.5	0.2028	5.6962515783432	5.6962515783432\\
69.5	0.20646	6.11785696141641	6.11785696141641\\
69.5	0.21012	6.56532733438866	6.56532733438866\\
69.5	0.21378	7.03866269725992	7.03866269725992\\
69.5	0.21744	7.53786305003024	7.53786305003024\\
69.5	0.2211	8.06292839269958	8.06292839269958\\
69.5	0.22476	8.61385872526797	8.61385872526797\\
69.5	0.22842	9.19065404773541	9.19065404773541\\
69.5	0.23208	9.79331436010186	9.79331436010186\\
69.5	0.23574	10.4218396623674	10.4218396623674\\
69.5	0.2394	11.0762299545319	11.0762299545319\\
69.5	0.24306	11.7564852365955	11.7564852365955\\
69.5	0.24672	12.4626055085581	12.4626055085581\\
69.5	0.25038	13.1945907704197	13.1945907704197\\
69.5	0.25404	13.9524410221804	13.9524410221804\\
69.5	0.2577	14.7361562638402	14.7361562638402\\
69.5	0.26136	15.545736495399	15.545736495399\\
69.5	0.26502	16.3811817168567	16.3811817168567\\
69.5	0.26868	17.2424919282136	17.2424919282136\\
69.5	0.27234	18.1296671294695	18.1296671294695\\
69.5	0.276	19.0427073206244	19.0427073206244\\
69.875	0.093	5.33011945606126	5.33011945606126\\
69.875	0.09666	4.96695530399978	4.96695530399978\\
69.875	0.10032	4.62965614183734	4.62965614183734\\
69.875	0.10398	4.31822196957393	4.31822196957393\\
69.875	0.10764	4.03265278720957	4.03265278720957\\
69.875	0.1113	3.77294859474424	3.77294859474424\\
69.875	0.11496	3.53910939217795	3.53910939217795\\
69.875	0.11862	3.33113517951071	3.33113517951071\\
69.875	0.12228	3.14902595674249	3.14902595674249\\
69.875	0.12594	2.99278172387331	2.99278172387331\\
69.875	0.1296	2.86240248090316	2.86240248090316\\
69.875	0.13326	2.75788822783207	2.75788822783207\\
69.875	0.13692	2.67923896466	2.67923896466\\
69.875	0.14058	2.62645469138698	2.62645469138698\\
69.875	0.14424	2.59953540801299	2.59953540801299\\
69.875	0.1479	2.59848111453805	2.59848111453805\\
69.875	0.15156	2.62329181096214	2.62329181096214\\
69.875	0.15522	2.67396749728526	2.67396749728526\\
69.875	0.15888	2.75050817350743	2.75050817350743\\
69.875	0.16254	2.85291383962862	2.85291383962862\\
69.875	0.1662	2.98118449564886	2.98118449564886\\
69.875	0.16986	3.13532014156814	3.13532014156814\\
69.875	0.17352	3.31532077738645	3.31532077738645\\
69.875	0.17718	3.52118640310382	3.52118640310382\\
69.875	0.18084	3.7529170187202	3.7529170187202\\
69.875	0.1845	4.01051262423563	4.01051262423563\\
69.875	0.18816	4.29397321965009	4.29397321965009\\
69.875	0.19182	4.60329880496359	4.60329880496359\\
69.875	0.19548	4.93848938017613	4.93848938017613\\
69.875	0.19914	5.29954494528772	5.29954494528772\\
69.875	0.2028	5.68646550029833	5.68646550029833\\
69.875	0.20646	6.09925104520801	6.09925104520801\\
69.875	0.21012	6.53790158001669	6.53790158001669\\
69.875	0.21378	7.00241710472442	7.00241710472442\\
69.875	0.21744	7.49279761933118	7.49279761933118\\
69.875	0.2211	8.00904312383699	8.00904312383699\\
69.875	0.22476	8.55115361824183	8.55115361824183\\
69.875	0.22842	9.11912910254571	9.11912910254571\\
69.875	0.23208	9.71296957674862	9.71296957674862\\
69.875	0.23574	10.3326750408506	10.3326750408506\\
69.875	0.2394	10.9782454948516	10.9782454948516\\
69.875	0.24306	11.6496809387516	11.6496809387516\\
69.875	0.24672	12.3469813725507	12.3469813725507\\
69.875	0.25038	13.0701467962488	13.0701467962488\\
69.875	0.25404	13.8191772098459	13.8191772098459\\
69.875	0.2577	14.5940726133421	14.5940726133421\\
69.875	0.26136	15.3948330067374	15.3948330067374\\
69.875	0.26502	16.2214583900316	16.2214583900316\\
69.875	0.26868	17.0739487632249	17.0739487632249\\
69.875	0.27234	17.9523041263173	17.9523041263173\\
69.875	0.276	18.8565244793086	18.8565244793086\\
70.25	0.093	5.5903575479909	5.5903575479909\\
70.25	0.09666	5.21837355776588	5.21837355776588\\
70.25	0.10032	4.8722545574399	4.8722545574399\\
70.25	0.10398	4.55200054701294	4.55200054701294\\
70.25	0.10764	4.25761152648504	4.25761152648504\\
70.25	0.1113	3.98908749585616	3.98908749585616\\
70.25	0.11496	3.74642845512632	3.74642845512632\\
70.25	0.11862	3.52963440429553	3.52963440429553\\
70.25	0.12228	3.33870534336377	3.33870534336377\\
70.25	0.12594	3.17364127233104	3.17364127233104\\
70.25	0.1296	3.03444219119736	3.03444219119736\\
70.25	0.13326	2.92110809996273	2.92110809996273\\
70.25	0.13692	2.83363899862711	2.83363899862711\\
70.25	0.14058	2.77203488719054	2.77203488719054\\
70.25	0.14424	2.73629576565301	2.73629576565301\\
70.25	0.1479	2.72642163401451	2.72642163401451\\
70.25	0.15156	2.74241249227506	2.74241249227506\\
70.25	0.15522	2.78426834043463	2.78426834043463\\
70.25	0.15888	2.85198917849326	2.85198917849326\\
70.25	0.16254	2.94557500645091	2.94557500645091\\
70.25	0.1662	3.06502582430761	3.06502582430761\\
70.25	0.16986	3.21034163206333	3.21034163206333\\
70.25	0.17352	3.38152242971811	3.38152242971811\\
70.25	0.17718	3.57856821727191	3.57856821727191\\
70.25	0.18084	3.80147899472475	3.80147899472475\\
70.25	0.1845	4.05025476207665	4.05025476207665\\
70.25	0.18816	4.32489551932755	4.32489551932755\\
70.25	0.19182	4.62540126647752	4.62540126647752\\
70.25	0.19548	4.95177200352652	4.95177200352652\\
70.25	0.19914	5.30400773047454	5.30400773047454\\
70.25	0.2028	5.68210844732163	5.68210844732163\\
70.25	0.20646	6.08607415406775	6.08607415406775\\
70.25	0.21012	6.51590485071287	6.51590485071287\\
70.25	0.21378	6.97160053725707	6.97160053725707\\
70.25	0.21744	7.4531612137003	7.4531612137003\\
70.25	0.2211	7.96058688004254	7.96058688004254\\
70.25	0.22476	8.49387753628385	8.49387753628385\\
70.25	0.22842	9.05303318242419	9.05303318242419\\
70.25	0.23208	9.63805381846355	9.63805381846355\\
70.25	0.23574	10.248939444402	10.248939444402\\
70.25	0.2394	10.8856900602394	10.8856900602394\\
70.25	0.24306	11.5483056659759	11.5483056659759\\
70.25	0.24672	12.2367862616114	12.2367862616114\\
70.25	0.25038	12.951131847146	12.951131847146\\
70.25	0.25404	13.6913424225796	13.6913424225796\\
70.25	0.2577	14.4574179879123	14.4574179879123\\
70.25	0.26136	15.2493585431439	15.2493585431439\\
70.25	0.26502	16.0671640882746	16.0671640882746\\
70.25	0.26868	16.9108346233044	16.9108346233044\\
70.25	0.27234	17.7803701482332	17.7803701482332\\
70.25	0.276	18.675770663061	18.675770663061\\
70.625	0.093	5.85602466498868	5.85602466498868\\
70.625	0.09666	5.47522083660013	5.47522083660013\\
70.625	0.10032	5.12028199811059	5.12028199811059\\
70.625	0.10398	4.79120814952009	4.79120814952009\\
70.625	0.10764	4.48799929082863	4.48799929082863\\
70.625	0.1113	4.21065542203622	4.21065542203622\\
70.625	0.11496	3.95917654314283	3.95917654314283\\
70.625	0.11862	3.7335626541485	3.7335626541485\\
70.625	0.12228	3.53381375505319	3.53381375505319\\
70.625	0.12594	3.35992984585691	3.35992984585691\\
70.625	0.1296	3.21191092655969	3.21191092655969\\
70.625	0.13326	3.0897569971615	3.0897569971615\\
70.625	0.13692	2.99346805766234	2.99346805766234\\
70.625	0.14058	2.92304410806223	2.92304410806223\\
70.625	0.14424	2.87848514836115	2.87848514836115\\
70.625	0.1479	2.8597911785591	2.8597911785591\\
70.625	0.15156	2.86696219865612	2.86696219865612\\
70.625	0.15522	2.89999820865213	2.89999820865213\\
70.625	0.15888	2.95889920854722	2.95889920854722\\
70.625	0.16254	3.04366519834133	3.04366519834133\\
70.625	0.1662	3.15429617803447	3.15429617803447\\
70.625	0.16986	3.29079214762666	3.29079214762666\\
70.625	0.17352	3.45315310711788	3.45315310711788\\
70.625	0.17718	3.64137905650815	3.64137905650815\\
70.625	0.18084	3.85546999579744	3.85546999579744\\
70.625	0.1845	4.09542592498578	4.09542592498578\\
70.625	0.18816	4.36124684407314	4.36124684407314\\
70.625	0.19182	4.65293275305958	4.65293275305958\\
70.625	0.19548	4.97048365194502	4.97048365194502\\
70.625	0.19914	5.31389954072951	5.31389954072951\\
70.625	0.2028	5.68318041941304	5.68318041941304\\
70.625	0.20646	6.0783262879956	6.0783262879956\\
70.625	0.21012	6.49933714647719	6.49933714647719\\
70.625	0.21378	6.94621299485785	6.94621299485785\\
70.625	0.21744	7.41895383313752	7.41895383313752\\
70.625	0.2211	7.91755966131623	7.91755966131623\\
70.625	0.22476	8.44203047939398	8.44203047939398\\
70.625	0.22842	8.99236628737079	8.99236628737079\\
70.625	0.23208	9.56856708524661	9.56856708524661\\
70.625	0.23574	10.1706328730215	10.1706328730215\\
70.625	0.2394	10.7985636506954	10.7985636506954\\
70.625	0.24306	11.4523594182683	11.4523594182683\\
70.625	0.24672	12.1320201757403	12.1320201757403\\
70.625	0.25038	12.8375459231113	12.8375459231113\\
70.625	0.25404	13.5689366603814	13.5689366603814\\
70.625	0.2577	14.3261923875505	14.3261923875505\\
70.625	0.26136	15.1093131046186	15.1093131046186\\
70.625	0.26502	15.9182988115858	15.9182988115858\\
70.625	0.26868	16.753149508452	16.753149508452\\
70.625	0.27234	17.6138651952172	17.6138651952172\\
70.625	0.276	18.5004458718815	18.5004458718815\\
71	0.093	6.12712080705462	6.12712080705462\\
71	0.09666	5.73749714050251	5.73749714050251\\
71	0.10032	5.37373846384943	5.37373846384943\\
71	0.10398	5.0358447770954	5.0358447770954\\
71	0.10764	4.7238160802404	4.7238160802404\\
71	0.1113	4.43765237328442	4.43765237328442\\
71	0.11496	4.1773536562275	4.1773536562275\\
71	0.11862	3.94291992906962	3.94291992906962\\
71	0.12228	3.73435119181075	3.73435119181075\\
71	0.12594	3.55164744445094	3.55164744445094\\
71	0.1296	3.39480868699017	3.39480868699017\\
71	0.13326	3.26383491942844	3.26383491942844\\
71	0.13692	3.15872614176574	3.15872614176574\\
71	0.14058	3.07948235400207	3.07948235400207\\
71	0.14424	3.02610355613745	3.02610355613745\\
71	0.1479	2.99858974817185	2.99858974817185\\
71	0.15156	2.9969409301053	2.9969409301053\\
71	0.15522	3.0211571019378	3.0211571019378\\
71	0.15888	3.07123826366934	3.07123826366934\\
71	0.16254	3.14718441529989	3.14718441529989\\
71	0.1662	3.2489955568295	3.2489955568295\\
71	0.16986	3.37667168825814	3.37667168825814\\
71	0.17352	3.53021280958581	3.53021280958581\\
71	0.17718	3.70961892081253	3.70961892081253\\
71	0.18084	3.91489002193828	3.91489002193828\\
71	0.1845	4.14602611296309	4.14602611296309\\
71	0.18816	4.40302719388689	4.40302719388689\\
71	0.19182	4.68589326470977	4.68589326470977\\
71	0.19548	4.99462432543168	4.99462432543168\\
71	0.19914	5.32922037605263	5.32922037605263\\
71	0.2028	5.6896814165726	5.6896814165726\\
71	0.20646	6.07600744699163	6.07600744699163\\
71	0.21012	6.48819846730966	6.48819846730966\\
71	0.21378	6.92625447752676	6.92625447752676\\
71	0.21744	7.3901754776429	7.3901754776429\\
71	0.2211	7.87996146765808	7.87996146765808\\
71	0.22476	8.39561244757229	8.39561244757229\\
71	0.22842	8.93712841738554	8.93712841738554\\
71	0.23208	9.5045093770978	9.5045093770978\\
71	0.23574	10.0977553267091	10.0977553267091\\
71	0.2394	10.7168662662195	10.7168662662195\\
71	0.24306	11.3618421956289	11.3618421956289\\
71	0.24672	12.0326831149373	12.0326831149373\\
71	0.25038	12.7293890241448	12.7293890241448\\
71	0.25404	13.4519599232513	13.4519599232513\\
71	0.2577	14.2003958122569	14.2003958122569\\
71	0.26136	14.9746966911615	14.9746966911615\\
71	0.26502	15.7748625599651	15.7748625599651\\
71	0.26868	16.6008934186677	16.6008934186677\\
71	0.27234	17.4527892672694	17.4527892672694\\
71	0.276	18.3305501057702	18.3305501057702\\
71.375	0.093	6.40364597418871	6.40364597418871\\
71.375	0.09666	6.00520246947305	6.00520246947305\\
71.375	0.10032	5.63262395465643	5.63262395465643\\
71.375	0.10398	5.28591042973885	5.28591042973885\\
71.375	0.10764	4.9650618947203	4.9650618947203\\
71.375	0.1113	4.67007834960079	4.67007834960079\\
71.375	0.11496	4.40095979438032	4.40095979438032\\
71.375	0.11862	4.15770622905889	4.15770622905889\\
71.375	0.12228	3.94031765363648	3.94031765363648\\
71.375	0.12594	3.74879406811313	3.74879406811313\\
71.375	0.1296	3.58313547248881	3.58313547248881\\
71.375	0.13326	3.44334186676352	3.44334186676352\\
71.375	0.13692	3.32941325093728	3.32941325093728\\
71.375	0.14058	3.24134962501007	3.24134962501007\\
71.375	0.14424	3.17915098898191	3.17915098898191\\
71.375	0.1479	3.14281734285277	3.14281734285277\\
71.375	0.15156	3.13234868662267	3.13234868662267\\
71.375	0.15522	3.14774502029162	3.14774502029162\\
71.375	0.15888	3.18900634385962	3.18900634385962\\
71.375	0.16254	3.25613265732663	3.25613265732663\\
71.375	0.1662	3.34912396069268	3.34912396069268\\
71.375	0.16986	3.46798025395778	3.46798025395778\\
71.375	0.17352	3.61270153712191	3.61270153712191\\
71.375	0.17718	3.78328781018509	3.78328781018509\\
71.375	0.18084	3.97973907314728	3.97973907314728\\
71.375	0.1845	4.20205532600855	4.20205532600855\\
71.375	0.18816	4.45023656876882	4.45023656876882\\
71.375	0.19182	4.72428280142814	4.72428280142814\\
71.375	0.19548	5.02419402398649	5.02419402398649\\
71.375	0.19914	5.34997023644388	5.34997023644388\\
71.375	0.2028	5.70161143880035	5.70161143880035\\
71.375	0.20646	6.07911763105582	6.07911763105582\\
71.375	0.21012	6.48248881321031	6.48248881321031\\
71.375	0.21378	6.91172498526388	6.91172498526388\\
71.375	0.21744	7.36682614721646	7.36682614721646\\
71.375	0.2211	7.84779229906808	7.84779229906808\\
71.375	0.22476	8.35462344081876	8.35462344081876\\
71.375	0.22842	8.88731957246845	8.88731957246845\\
71.375	0.23208	9.44588069401718	9.44588069401718\\
71.375	0.23574	10.030306805465	10.030306805465\\
71.375	0.2394	10.6405979068118	10.6405979068118\\
71.375	0.24306	11.2767539980576	11.2767539980576\\
71.375	0.24672	11.9387750792025	11.9387750792025\\
71.375	0.25038	12.6266611502464	12.6266611502464\\
71.375	0.25404	13.3404122111894	13.3404122111894\\
71.375	0.2577	14.0800282620314	14.0800282620314\\
71.375	0.26136	14.8455093027725	14.8455093027725\\
71.375	0.26502	15.6368553334125	15.6368553334125\\
71.375	0.26868	16.4540663539516	16.4540663539516\\
71.375	0.27234	17.2971423643898	17.2971423643898\\
71.375	0.276	18.166083364727	18.166083364727\\
71.75	0.093	6.68560016639094	6.68560016639094\\
71.75	0.09666	6.27833682351173	6.27833682351173\\
71.75	0.10032	5.89693847053158	5.89693847053158\\
71.75	0.10398	5.54140510745045	5.54140510745045\\
71.75	0.10764	5.21173673426835	5.21173673426835\\
71.75	0.1113	4.90793335098528	4.90793335098528\\
71.75	0.11496	4.62999495760127	4.62999495760127\\
71.75	0.11862	4.3779215541163	4.3779215541163\\
71.75	0.12228	4.15171314053035	4.15171314053035\\
71.75	0.12594	3.95136971684345	3.95136971684345\\
71.75	0.1296	3.77689128305557	3.77689128305557\\
71.75	0.13326	3.62827783916675	3.62827783916675\\
71.75	0.13692	3.50552938517696	3.50552938517696\\
71.75	0.14058	3.4086459210862	3.4086459210862\\
71.75	0.14424	3.3376274468945	3.3376274468945\\
71.75	0.1479	3.29247396260181	3.29247396260181\\
71.75	0.15156	3.27318546820818	3.27318546820818\\
71.75	0.15522	3.27976196371356	3.27976196371356\\
71.75	0.15888	3.312203449118	3.312203449118\\
71.75	0.16254	3.37050992442148	3.37050992442148\\
71.75	0.1662	3.454681389624	3.454681389624\\
71.75	0.16986	3.56471784472555	3.56471784472555\\
71.75	0.17352	3.70061928972614	3.70061928972614\\
71.75	0.17718	3.86238572462576	3.86238572462576\\
71.75	0.18084	4.05001714942441	4.05001714942441\\
71.75	0.1845	4.26351356412212	4.26351356412212\\
71.75	0.18816	4.50287496871886	4.50287496871886\\
71.75	0.19182	4.76810136321465	4.76810136321465\\
71.75	0.19548	5.05919274760947	5.05919274760947\\
71.75	0.19914	5.3761491219033	5.3761491219033\\
71.75	0.2028	5.7189704860962	5.7189704860962\\
71.75	0.20646	6.08765684018811	6.08765684018811\\
71.75	0.21012	6.4822081841791	6.4822081841791\\
71.75	0.21378	6.90262451806908	6.90262451806908\\
71.75	0.21744	7.34890584185815	7.34890584185815\\
71.75	0.2211	7.82105215554621	7.82105215554621\\
71.75	0.22476	8.31906345913333	8.31906345913333\\
71.75	0.22842	8.84293975261949	8.84293975261949\\
71.75	0.23208	9.39268103600466	9.39268103600466\\
71.75	0.23574	9.96828730928891	9.96828730928891\\
71.75	0.2394	10.5697585724722	10.5697585724722\\
71.75	0.24306	11.1970948255545	11.1970948255545\\
71.75	0.24672	11.8502960685358	11.8502960685358\\
71.75	0.25038	12.5293623014162	12.5293623014162\\
71.75	0.25404	13.2342935241956	13.2342935241956\\
71.75	0.2577	13.9650897368741	13.9650897368741\\
71.75	0.26136	14.7217509394516	14.7217509394516\\
71.75	0.26502	15.5042771319281	15.5042771319281\\
71.75	0.26868	16.3126683143037	16.3126683143037\\
71.75	0.27234	17.1469244865783	17.1469244865783\\
71.75	0.276	18.0070456487519	18.0070456487519\\
72.125	0.093	6.9729833836613	6.9729833836613\\
72.125	0.09666	6.55690020261855	6.55690020261855\\
72.125	0.10032	6.16668201147484	6.16668201147484\\
72.125	0.10398	5.80232881023015	5.80232881023015\\
72.125	0.10764	5.46384059888453	5.46384059888453\\
72.125	0.1113	5.15121737743793	5.15121737743793\\
72.125	0.11496	4.86445914589035	4.86445914589035\\
72.125	0.11862	4.60356590424183	4.60356590424183\\
72.125	0.12228	4.36853765249235	4.36853765249235\\
72.125	0.12594	4.15937439064189	4.15937439064189\\
72.125	0.1296	3.97607611869048	3.97607611869048\\
72.125	0.13326	3.81864283663811	3.81864283663811\\
72.125	0.13692	3.68707454448478	3.68707454448478\\
72.125	0.14058	3.58137124223047	3.58137124223047\\
72.125	0.14424	3.50153292987522	3.50153292987522\\
72.125	0.1479	3.44755960741899	3.44755960741899\\
72.125	0.15156	3.41945127486182	3.41945127486182\\
72.125	0.15522	3.41720793220366	3.41720793220366\\
72.125	0.15888	3.44082957944455	3.44082957944455\\
72.125	0.16254	3.49031621658447	3.49031621658447\\
72.125	0.1662	3.56566784362344	3.56566784362344\\
72.125	0.16986	3.66688446056145	3.66688446056145\\
72.125	0.17352	3.79396606739849	3.79396606739849\\
72.125	0.17718	3.94691266413458	3.94691266413458\\
72.125	0.18084	4.12572425076968	4.12572425076968\\
72.125	0.1845	4.33040082730386	4.33040082730386\\
72.125	0.18816	4.56094239373704	4.56094239373704\\
72.125	0.19182	4.81734895006927	4.81734895006927\\
72.125	0.19548	5.09962049630055	5.09962049630055\\
72.125	0.19914	5.40775703243085	5.40775703243085\\
72.125	0.2028	5.7417585584602	5.7417585584602\\
72.125	0.20646	6.10162507438857	6.10162507438857\\
72.125	0.21012	6.487356580216	6.487356580216\\
72.125	0.21378	6.89895307594245	6.89895307594245\\
72.125	0.21744	7.33641456156793	7.33641456156793\\
72.125	0.2211	7.79974103709249	7.79974103709249\\
72.125	0.22476	8.28893250251605	8.28893250251605\\
72.125	0.22842	8.80398895783868	8.80398895783868\\
72.125	0.23208	9.34491040306031	9.34491040306031\\
72.125	0.23574	9.911696838181	9.911696838181\\
72.125	0.2394	10.5043482632007	10.5043482632007\\
72.125	0.24306	11.1228646781195	11.1228646781195\\
72.125	0.24672	11.7672460829373	11.7672460829373\\
72.125	0.25038	12.4374924776541	12.4374924776541\\
72.125	0.25404	13.13360386227	13.13360386227\\
72.125	0.2577	13.8555802367849	13.8555802367849\\
72.125	0.26136	14.6034216011988	14.6034216011988\\
72.125	0.26502	15.3771279555118	15.3771279555118\\
72.125	0.26868	16.1766992997239	16.1766992997239\\
72.125	0.27234	17.0021356338349	17.0021356338349\\
72.125	0.276	17.853436957845	17.853436957845\\
72.5	0.093	7.26579562599981	7.26579562599981\\
72.5	0.09666	6.84089260679352	6.84089260679352\\
72.5	0.10032	6.44185457748626	6.44185457748626\\
72.5	0.10398	6.06868153807804	6.06868153807804\\
72.5	0.10764	5.72137348856886	5.72137348856886\\
72.5	0.1113	5.39993042895872	5.39993042895872\\
72.5	0.11496	5.1043523592476	5.1043523592476\\
72.5	0.11862	4.83463927943553	4.83463927943553\\
72.5	0.12228	4.59079118952249	4.59079118952249\\
72.5	0.12594	4.3728080895085	4.3728080895085\\
72.5	0.1296	4.18068997939354	4.18068997939354\\
72.5	0.13326	4.01443685917762	4.01443685917762\\
72.5	0.13692	3.87404872886075	3.87404872886075\\
72.5	0.14058	3.7595255884429	3.7595255884429\\
72.5	0.14424	3.6708674379241	3.6708674379241\\
72.5	0.1479	3.60807427730432	3.60807427730432\\
72.5	0.15156	3.57114610658359	3.57114610658359\\
72.5	0.15522	3.5600829257619	3.5600829257619\\
72.5	0.15888	3.57488473483926	3.57488473483926\\
72.5	0.16254	3.61555153381563	3.61555153381563\\
72.5	0.1662	3.68208332269105	3.68208332269105\\
72.5	0.16986	3.77448010146551	3.77448010146551\\
72.5	0.17352	3.89274187013901	3.89274187013901\\
72.5	0.17718	4.03686862871155	4.03686862871155\\
72.5	0.18084	4.20686037718311	4.20686037718311\\
72.5	0.1845	4.40271711555373	4.40271711555373\\
72.5	0.18816	4.62443884382338	4.62443884382338\\
72.5	0.19182	4.87202556199207	4.87202556199207\\
72.5	0.19548	5.14547727005979	5.14547727005979\\
72.5	0.19914	5.44479396802653	5.44479396802653\\
72.5	0.2028	5.76997565589235	5.76997565589235\\
72.5	0.20646	6.12102233365719	6.12102233365719\\
72.5	0.21012	6.49793400132106	6.49793400132106\\
72.5	0.21378	6.90071065888397	6.90071065888397\\
72.5	0.21744	7.32935230634592	7.32935230634592\\
72.5	0.2211	7.78385894370689	7.78385894370689\\
72.5	0.22476	8.26423057096692	8.26423057096692\\
72.5	0.22842	8.77046718812598	8.77046718812598\\
72.5	0.23208	9.30256879518409	9.30256879518409\\
72.5	0.23574	9.86053539214124	9.86053539214124\\
72.5	0.2394	10.4443669789974	10.4443669789974\\
72.5	0.24306	11.0540635557526	11.0540635557526\\
72.5	0.24672	11.6896251224069	11.6896251224069\\
72.5	0.25038	12.3510516789602	12.3510516789602\\
72.5	0.25404	13.0383432254125	13.0383432254125\\
72.5	0.2577	13.7514997617639	13.7514997617639\\
72.5	0.26136	14.4905212880143	14.4905212880143\\
72.5	0.26502	15.2554078041637	15.2554078041637\\
72.5	0.26868	16.0461593102122	16.0461593102122\\
72.5	0.27234	16.8627758061597	16.8627758061597\\
72.5	0.276	17.7052572920062	17.7052572920062\\
72.875	0.093	7.56403689340649	7.56403689340649\\
72.875	0.09666	7.13031403603665	7.13031403603665\\
72.875	0.10032	6.72245616856584	6.72245616856584\\
72.875	0.10398	6.34046329099408	6.34046329099408\\
72.875	0.10764	5.98433540332134	5.98433540332134\\
72.875	0.1113	5.65407250554765	5.65407250554765\\
72.875	0.11496	5.349674597673	5.349674597673\\
72.875	0.11862	5.07114167969739	5.07114167969739\\
72.875	0.12228	4.81847375162081	4.81847375162081\\
72.875	0.12594	4.59167081344326	4.59167081344326\\
72.875	0.1296	4.39073286516477	4.39073286516477\\
72.875	0.13326	4.21565990678529	4.21565990678529\\
72.875	0.13692	4.06645193830487	4.06645193830487\\
72.875	0.14058	3.94310895972347	3.94310895972347\\
72.875	0.14424	3.84563097104113	3.84563097104113\\
72.875	0.1479	3.7740179722578	3.7740179722578\\
72.875	0.15156	3.72826996337354	3.72826996337354\\
72.875	0.15522	3.70838694438831	3.70838694438831\\
72.875	0.15888	3.71436891530212	3.71436891530212\\
72.875	0.16254	3.74621587611494	3.74621587611494\\
72.875	0.1662	3.80392782682682	3.80392782682682\\
72.875	0.16986	3.88750476743773	3.88750476743773\\
72.875	0.17352	3.99694669794768	3.99694669794768\\
72.875	0.17718	4.13225361835667	4.13225361835667\\
72.875	0.18084	4.29342552866469	4.29342552866469\\
72.875	0.1845	4.48046242887177	4.48046242887177\\
72.875	0.18816	4.69336431897786	4.69336431897786\\
72.875	0.19182	4.93213119898299	4.93213119898299\\
72.875	0.19548	5.19676306888719	5.19676306888719\\
72.875	0.19914	5.48725992869039	5.48725992869039\\
72.875	0.2028	5.80362177839265	5.80362177839265\\
72.875	0.20646	6.14584861799396	6.14584861799396\\
72.875	0.21012	6.51394044749426	6.51394044749426\\
72.875	0.21378	6.90789726689362	6.90789726689362\\
72.875	0.21744	7.32771907619204	7.32771907619204\\
72.875	0.2211	7.77340587538947	7.77340587538947\\
72.875	0.22476	8.24495766448597	8.24495766448597\\
72.875	0.22842	8.74237444348147	8.74237444348147\\
72.875	0.23208	9.26565621237602	9.26565621237602\\
72.875	0.23574	9.81480297116964	9.81480297116964\\
72.875	0.2394	10.3898147198623	10.3898147198623\\
72.875	0.24306	10.9906914584539	10.9906914584539\\
72.875	0.24672	11.6174331869446	11.6174331869446\\
72.875	0.25038	12.2700399053344	12.2700399053344\\
72.875	0.25404	12.9485116136231	12.9485116136231\\
72.875	0.2577	13.652848311811	13.652848311811\\
72.875	0.26136	14.3830499998978	14.3830499998978\\
72.875	0.26502	15.1391166778837	15.1391166778837\\
72.875	0.26868	15.9210483457687	15.9210483457687\\
72.875	0.27234	16.7288450035526	16.7288450035526\\
72.875	0.276	17.5625066512356	17.5625066512356\\
73.25	0.093	7.86770718588131	7.86770718588131\\
73.25	0.09666	7.42516449034792	7.42516449034792\\
73.25	0.10032	7.00848678471358	7.00848678471358\\
73.25	0.10398	6.61767406897826	6.61767406897826\\
73.25	0.10764	6.25272634314199	6.25272634314199\\
73.25	0.1113	5.91364360720476	5.91364360720476\\
73.25	0.11496	5.60042586116655	5.60042586116655\\
73.25	0.11862	5.31307310502738	5.31307310502738\\
73.25	0.12228	5.05158533878725	5.05158533878725\\
73.25	0.12594	4.81596256244618	4.81596256244618\\
73.25	0.1296	4.60620477600413	4.60620477600413\\
73.25	0.13326	4.42231197946112	4.42231197946112\\
73.25	0.13692	4.26428417281715	4.26428417281715\\
73.25	0.14058	4.1321213560722	4.1321213560722\\
73.25	0.14424	4.02582352922631	4.02582352922631\\
73.25	0.1479	3.94539069227945	3.94539069227945\\
73.25	0.15156	3.89082284523163	3.89082284523163\\
73.25	0.15522	3.86211998808285	3.86211998808285\\
73.25	0.15888	3.85928212083311	3.85928212083311\\
73.25	0.16254	3.88230924348239	3.88230924348239\\
73.25	0.1662	3.93120135603072	3.93120135603072\\
73.25	0.16986	4.00595845847809	4.00595845847809\\
73.25	0.17352	4.10658055082451	4.10658055082451\\
73.25	0.17718	4.23306763306993	4.23306763306993\\
73.25	0.18084	4.38541970521441	4.38541970521441\\
73.25	0.1845	4.56363676725793	4.56363676725793\\
73.25	0.18816	4.76771881920049	4.76771881920049\\
73.25	0.19182	4.99766586104209	4.99766586104209\\
73.25	0.19548	5.25347789278272	5.25347789278272\\
73.25	0.19914	5.53515491442239	5.53515491442239\\
73.25	0.2028	5.84269692596111	5.84269692596111\\
73.25	0.20646	6.17610392739886	6.17610392739886\\
73.25	0.21012	6.53537591873561	6.53537591873561\\
73.25	0.21378	6.92051289997146	6.92051289997146\\
73.25	0.21744	7.33151487110629	7.33151487110629\\
73.25	0.2211	7.76838183214022	7.76838183214022\\
73.25	0.22476	8.23111378307316	8.23111378307316\\
73.25	0.22842	8.71971072390513	8.71971072390513\\
73.25	0.23208	9.23417265463611	9.23417265463611\\
73.25	0.23574	9.77449957526618	9.77449957526618\\
73.25	0.2394	10.3406914857953	10.3406914857953\\
73.25	0.24306	10.9327483862234	10.9327483862234\\
73.25	0.24672	11.5506702765505	11.5506702765505\\
73.25	0.25038	12.1944571567768	12.1944571567768\\
73.25	0.25404	12.864109026902	12.864109026902\\
73.25	0.2577	13.5596258869263	13.5596258869263\\
73.25	0.26136	14.2810077368496	14.2810077368496\\
73.25	0.26502	15.0282545766719	15.0282545766719\\
73.25	0.26868	15.8013664063933	15.8013664063933\\
73.25	0.27234	16.6003432260137	16.6003432260137\\
73.25	0.276	17.4251850355332	17.4251850355332\\
73.625	0.093	8.17680650342425	8.17680650342425\\
73.625	0.09666	7.72544396972734	7.72544396972734\\
73.625	0.10032	7.29994642592943	7.29994642592943\\
73.625	0.10398	6.90031387203058	6.90031387203058\\
73.625	0.10764	6.52654630803075	6.52654630803075\\
73.625	0.1113	6.17864373392997	6.17864373392997\\
73.625	0.11496	5.85660614972823	5.85660614972823\\
73.625	0.11862	5.56043355542552	5.56043355542552\\
73.625	0.12228	5.29012595102185	5.29012595102185\\
73.625	0.12594	5.04568333651722	5.04568333651722\\
73.625	0.1296	4.82710571191162	4.82710571191162\\
73.625	0.13326	4.63439307720506	4.63439307720506\\
73.625	0.13692	4.46754543239755	4.46754543239755\\
73.625	0.14058	4.32656277748907	4.32656277748907\\
73.625	0.14424	4.21144511247963	4.21144511247963\\
73.625	0.1479	4.12219243736921	4.12219243736921\\
73.625	0.15156	4.05880475215786	4.05880475215786\\
73.625	0.15522	4.02128205684552	4.02128205684552\\
73.625	0.15888	4.00962435143222	4.00962435143222\\
73.625	0.16254	4.02383163591798	4.02383163591798\\
73.625	0.1662	4.06390391030277	4.06390391030277\\
73.625	0.16986	4.12984117458659	4.12984117458659\\
73.625	0.17352	4.22164342876943	4.22164342876943\\
73.625	0.17718	4.33931067285134	4.33931067285134\\
73.625	0.18084	4.48284290683226	4.48284290683226\\
73.625	0.1845	4.65224013071226	4.65224013071226\\
73.625	0.18816	4.84750234449125	4.84750234449125\\
73.625	0.19182	5.06862954816932	5.06862954816932\\
73.625	0.19548	5.31562174174639	5.31562174174639\\
73.625	0.19914	5.5884789252225	5.5884789252225\\
73.625	0.2028	5.88720109859769	5.88720109859769\\
73.625	0.20646	6.21178826187191	6.21178826187191\\
73.625	0.21012	6.56224041504512	6.56224041504512\\
73.625	0.21378	6.93855755811739	6.93855755811739\\
73.625	0.21744	7.34073969108871	7.34073969108871\\
73.625	0.2211	7.76878681395905	7.76878681395905\\
73.625	0.22476	8.22269892672846	8.22269892672846\\
73.625	0.22842	8.7024760293969	8.7024760293969\\
73.625	0.23208	9.20811812196435	9.20811812196435\\
73.625	0.23574	9.73962520443085	9.73962520443085\\
73.625	0.2394	10.2969972767964	10.2969972767964\\
73.625	0.24306	10.880234339061	10.880234339061\\
73.625	0.24672	11.4893363912246	11.4893363912246\\
73.625	0.25038	12.1243034332872	12.1243034332872\\
73.625	0.25404	12.7851354652489	12.7851354652489\\
73.625	0.2577	13.4718324871097	13.4718324871097\\
73.625	0.26136	14.1843944988694	14.1843944988694\\
73.625	0.26502	14.9228215005282	14.9228215005282\\
73.625	0.26868	15.6871134920861	15.6871134920861\\
73.625	0.27234	16.477270473543	16.477270473543\\
73.625	0.276	17.2932924448989	17.2932924448989\\
74	0.093	8.49133484603537	8.49133484603537\\
74	0.09666	8.0311524741749	8.0311524741749\\
74	0.10032	7.59683509221345	7.59683509221345\\
74	0.10398	7.18838270015106	7.18838270015106\\
74	0.10764	6.80579529798769	6.80579529798769\\
74	0.1113	6.44907288572336	6.44907288572336\\
74	0.11496	6.11821546335806	6.11821546335806\\
74	0.11862	5.81322303089182	5.81322303089182\\
74	0.12228	5.53409558832459	5.53409558832459\\
74	0.12594	5.28083313565643	5.28083313565643\\
74	0.1296	5.05343567288728	5.05343567288728\\
74	0.13326	4.85190320001718	4.85190320001718\\
74	0.13692	4.67623571704612	4.67623571704612\\
74	0.14058	4.52643322397408	4.52643322397408\\
74	0.14424	4.40249572080111	4.40249572080111\\
74	0.1479	4.30442320752716	4.30442320752716\\
74	0.15156	4.23221568415224	4.23221568415224\\
74	0.15522	4.18587315067635	4.18587315067635\\
74	0.15888	4.16539560709954	4.16539560709954\\
74	0.16254	4.17078305342173	4.17078305342173\\
74	0.1662	4.20203548964298	4.20203548964298\\
74	0.16986	4.25915291576325	4.25915291576325\\
74	0.17352	4.34213533178256	4.34213533178256\\
74	0.17718	4.45098273770092	4.45098273770092\\
74	0.18084	4.5856951335183	4.5856951335183\\
74	0.1845	4.74627251923474	4.74627251923474\\
74	0.18816	4.9327148948502	4.9327148948502\\
74	0.19182	5.1450222603647	5.1450222603647\\
74	0.19548	5.38319461577824	5.38319461577824\\
74	0.19914	5.64723196109082	5.64723196109082\\
74	0.2028	5.93713429630245	5.93713429630245\\
74	0.20646	6.25290162141311	6.25290162141311\\
74	0.21012	6.59453393642279	6.59453393642279\\
74	0.21378	6.96203124133152	6.96203124133152\\
74	0.21744	7.35539353613929	7.35539353613929\\
74	0.2211	7.7746208208461	7.7746208208461\\
74	0.22476	8.21971309545194	8.21971309545194\\
74	0.22842	8.69067035995682	8.69067035995682\\
74	0.23208	9.18749261436074	9.18749261436074\\
74	0.23574	9.71017985866371	9.71017985866371\\
74	0.2394	10.2587320928657	10.2587320928657\\
74	0.24306	10.8331493169667	10.8331493169667\\
74	0.24672	11.4334315309668	11.4334315309668\\
74	0.25038	12.0595787348659	12.0595787348659\\
74	0.25404	12.7115909286641	12.7115909286641\\
74	0.2577	13.3894681123612	13.3894681123612\\
74	0.26136	14.0932102859574	14.0932102859574\\
74	0.26502	14.8228174494527	14.8228174494527\\
74	0.26868	15.578289602847	15.578289602847\\
74	0.27234	16.3596267461403	16.3596267461403\\
74	0.276	17.1668288793327	17.1668288793327\\
};
\end{axis}

\begin{axis}[%
width=2.616cm,
height=2.517cm,
at={(0cm,3.497cm)},
scale only axis,
xmin=56,
xmax=74,
tick align=outside,
xlabel style={font=\color{white!15!black}},
xlabel={$L_{cut}$},
ymin=0.093,
ymax=0.276,
ylabel style={font=\color{white!15!black}},
ylabel={$D_{rlx}$},
zmin=-1.04441093327234,
zmax=10.9608681880445,
zlabel style={font=\color{white!15!black}},
zlabel={$x_1,x_4$},
view={-140}{50},
axis background/.style={fill=white},
xmajorgrids,
ymajorgrids,
zmajorgrids,
legend style={at={(1.03,1)}, anchor=north west, legend cell align=left, align=left, draw=white!15!black}
]
\addplot3[only marks, mark=*, mark options={}, mark size=1.5000pt, color=mycolor1, fill=mycolor1] table[row sep=crcr]{%
x	y	z\\
74	0.123	0.654041924845694\\
72	0.113	1.20961382721373\\
61	0.095	0.00587406381307785\\
56	0.093	0.0536824135013003\\
};
\addlegendentry{data1}

\addplot3[only marks, mark=*, mark options={}, mark size=1.5000pt, color=mycolor2, fill=mycolor2] table[row sep=crcr]{%
x	y	z\\
67	0.276	6.10127297653778\\
66	0.255	3.43164090297767\\
62	0.209	1.57688045288337\\
57	0.193	1.29469161204866\\
};
\addlegendentry{data2}

\addplot3[only marks, mark=*, mark options={}, mark size=1.5000pt, color=black, fill=black] table[row sep=crcr]{%
x	y	z\\
69	0.104	0.945594356016375\\
};
\addlegendentry{data3}

\addplot3[only marks, mark=*, mark options={}, mark size=1.5000pt, color=black, fill=black] table[row sep=crcr]{%
x	y	z\\
64	0.23	2.23809167770276\\
};
\addlegendentry{data4}


\addplot3[%
surf,
fill opacity=0.7, shader=interp, colormap={mymap}{[1pt] rgb(0pt)=(1,0.905882,0); rgb(1pt)=(1,0.901964,0); rgb(2pt)=(1,0.898051,0); rgb(3pt)=(1,0.894144,0); rgb(4pt)=(1,0.890243,0); rgb(5pt)=(1,0.886349,0); rgb(6pt)=(1,0.88246,0); rgb(7pt)=(1,0.878577,0); rgb(8pt)=(1,0.8747,0); rgb(9pt)=(1,0.870829,0); rgb(10pt)=(1,0.866964,0); rgb(11pt)=(1,0.863106,0); rgb(12pt)=(1,0.859253,0); rgb(13pt)=(1,0.855406,0); rgb(14pt)=(1,0.851566,0); rgb(15pt)=(1,0.847732,0); rgb(16pt)=(1,0.843903,0); rgb(17pt)=(1,0.840081,0); rgb(18pt)=(1,0.836265,0); rgb(19pt)=(1,0.832455,0); rgb(20pt)=(1,0.828652,0); rgb(21pt)=(1,0.824854,0); rgb(22pt)=(1,0.821063,0); rgb(23pt)=(1,0.817278,0); rgb(24pt)=(1,0.8135,0); rgb(25pt)=(1,0.809727,0); rgb(26pt)=(1,0.805961,0); rgb(27pt)=(1,0.8022,0); rgb(28pt)=(1,0.798445,0); rgb(29pt)=(1,0.794696,0); rgb(30pt)=(1,0.790953,0); rgb(31pt)=(1,0.787215,0); rgb(32pt)=(1,0.783484,0); rgb(33pt)=(1,0.779758,0); rgb(34pt)=(1,0.776038,0); rgb(35pt)=(1,0.772324,0); rgb(36pt)=(1,0.768615,0); rgb(37pt)=(1,0.764913,0); rgb(38pt)=(1,0.761217,0); rgb(39pt)=(1,0.757527,0); rgb(40pt)=(1,0.753843,0); rgb(41pt)=(1,0.750165,0); rgb(42pt)=(1,0.746493,0); rgb(43pt)=(1,0.742827,0); rgb(44pt)=(1,0.739167,0); rgb(45pt)=(1,0.735514,0); rgb(46pt)=(1,0.731867,0); rgb(47pt)=(1,0.728226,0); rgb(48pt)=(1,0.724591,0); rgb(49pt)=(1,0.720963,0); rgb(50pt)=(1,0.717341,0); rgb(51pt)=(1,0.713725,0); rgb(52pt)=(0.999994,0.710077,0); rgb(53pt)=(0.999974,0.706363,0); rgb(54pt)=(0.999942,0.702592,0); rgb(55pt)=(0.999898,0.698775,0); rgb(56pt)=(0.999841,0.694921,0); rgb(57pt)=(0.999771,0.691039,0); rgb(58pt)=(0.99969,0.687139,0); rgb(59pt)=(0.999596,0.68323,0); rgb(60pt)=(0.99949,0.679323,0); rgb(61pt)=(0.999372,0.675427,0); rgb(62pt)=(0.999242,0.67155,0); rgb(63pt)=(0.9991,0.667704,0); rgb(64pt)=(0.998946,0.663897,0); rgb(65pt)=(0.998781,0.660138,0); rgb(66pt)=(0.998605,0.656439,0); rgb(67pt)=(0.998416,0.652807,0); rgb(68pt)=(0.998217,0.649253,0); rgb(69pt)=(0.998006,0.645786,0); rgb(70pt)=(0.997785,0.642416,0); rgb(71pt)=(0.997552,0.639152,0); rgb(72pt)=(0.997308,0.636004,0); rgb(73pt)=(0.997053,0.632982,0); rgb(74pt)=(0.996788,0.630095,0); rgb(75pt)=(0.996512,0.627352,0); rgb(76pt)=(0.996226,0.624763,0); rgb(77pt)=(0.995851,0.622329,0); rgb(78pt)=(0.99494,0.619997,0); rgb(79pt)=(0.99345,0.617753,0); rgb(80pt)=(0.991419,0.61559,0); rgb(81pt)=(0.988885,0.613503,0); rgb(82pt)=(0.985886,0.611486,0); rgb(83pt)=(0.98246,0.609532,0); rgb(84pt)=(0.978643,0.607636,0); rgb(85pt)=(0.974475,0.605791,0); rgb(86pt)=(0.969992,0.603992,0); rgb(87pt)=(0.965232,0.602233,0); rgb(88pt)=(0.960233,0.600507,0); rgb(89pt)=(0.955033,0.598808,0); rgb(90pt)=(0.949669,0.59713,0); rgb(91pt)=(0.94418,0.595468,0); rgb(92pt)=(0.938602,0.593815,0); rgb(93pt)=(0.932974,0.592166,0); rgb(94pt)=(0.927333,0.590513,0); rgb(95pt)=(0.921717,0.588852,0); rgb(96pt)=(0.916164,0.587176,0); rgb(97pt)=(0.910711,0.585479,0); rgb(98pt)=(0.905397,0.583755,0); rgb(99pt)=(0.900258,0.581999,0); rgb(100pt)=(0.895333,0.580203,0); rgb(101pt)=(0.890659,0.578362,0); rgb(102pt)=(0.886275,0.576471,0); rgb(103pt)=(0.882047,0.574545,0); rgb(104pt)=(0.877819,0.572608,0); rgb(105pt)=(0.873592,0.57066,0); rgb(106pt)=(0.869366,0.568701,0); rgb(107pt)=(0.865143,0.566733,0); rgb(108pt)=(0.860924,0.564756,0); rgb(109pt)=(0.856708,0.562771,0); rgb(110pt)=(0.852497,0.560778,0); rgb(111pt)=(0.848292,0.558779,0); rgb(112pt)=(0.844092,0.556774,0); rgb(113pt)=(0.8399,0.554763,0); rgb(114pt)=(0.835716,0.552749,0); rgb(115pt)=(0.831541,0.55073,0); rgb(116pt)=(0.827374,0.548709,0); rgb(117pt)=(0.823219,0.546686,0); rgb(118pt)=(0.819074,0.54466,0); rgb(119pt)=(0.81494,0.542635,0); rgb(120pt)=(0.81082,0.540609,0); rgb(121pt)=(0.806712,0.538584,0); rgb(122pt)=(0.802619,0.53656,0); rgb(123pt)=(0.798541,0.534539,0); rgb(124pt)=(0.794478,0.532521,0); rgb(125pt)=(0.790431,0.530506,0); rgb(126pt)=(0.786402,0.528496,0); rgb(127pt)=(0.782391,0.526491,0); rgb(128pt)=(0.77841,0.524489,0); rgb(129pt)=(0.774523,0.522478,0); rgb(130pt)=(0.770731,0.520455,0); rgb(131pt)=(0.767022,0.518424,0); rgb(132pt)=(0.763384,0.516385,0); rgb(133pt)=(0.759804,0.514339,0); rgb(134pt)=(0.756272,0.51229,0); rgb(135pt)=(0.752775,0.510237,0); rgb(136pt)=(0.749302,0.508182,0); rgb(137pt)=(0.74584,0.506128,0); rgb(138pt)=(0.742378,0.504075,0); rgb(139pt)=(0.738904,0.502025,0); rgb(140pt)=(0.735406,0.499979,0); rgb(141pt)=(0.731872,0.49794,0); rgb(142pt)=(0.72829,0.495909,0); rgb(143pt)=(0.724649,0.493887,0); rgb(144pt)=(0.720936,0.491875,0); rgb(145pt)=(0.71714,0.489876,0); rgb(146pt)=(0.713249,0.487891,0); rgb(147pt)=(0.709251,0.485921,0); rgb(148pt)=(0.705134,0.483968,0); rgb(149pt)=(0.700887,0.482033,0); rgb(150pt)=(0.696497,0.480118,0); rgb(151pt)=(0.691952,0.478225,0); rgb(152pt)=(0.687242,0.476355,0); rgb(153pt)=(0.682353,0.47451,0); rgb(154pt)=(0.677195,0.472696,0); rgb(155pt)=(0.6717,0.470916,0); rgb(156pt)=(0.665891,0.469169,0); rgb(157pt)=(0.659791,0.46745,0); rgb(158pt)=(0.653423,0.465756,0); rgb(159pt)=(0.64681,0.464084,0); rgb(160pt)=(0.639976,0.462432,0); rgb(161pt)=(0.632943,0.460795,0); rgb(162pt)=(0.625734,0.459171,0); rgb(163pt)=(0.618373,0.457556,0); rgb(164pt)=(0.610882,0.455948,0); rgb(165pt)=(0.603284,0.454343,0); rgb(166pt)=(0.595604,0.452737,0); rgb(167pt)=(0.587863,0.451129,0); rgb(168pt)=(0.580084,0.449514,0); rgb(169pt)=(0.572292,0.447889,0); rgb(170pt)=(0.564508,0.446252,0); rgb(171pt)=(0.556756,0.444599,0); rgb(172pt)=(0.549059,0.442927,0); rgb(173pt)=(0.54144,0.441232,0); rgb(174pt)=(0.533922,0.439512,0); rgb(175pt)=(0.526529,0.437764,0); rgb(176pt)=(0.519282,0.435983,0); rgb(177pt)=(0.512206,0.434168,0); rgb(178pt)=(0.505323,0.432315,0); rgb(179pt)=(0.498628,0.430422,3.92506e-06); rgb(180pt)=(0.491973,0.428504,3.49981e-05); rgb(181pt)=(0.485331,0.426562,9.63073e-05); rgb(182pt)=(0.478704,0.424596,0.000186979); rgb(183pt)=(0.472096,0.422609,0.000306141); rgb(184pt)=(0.465508,0.420599,0.00045292); rgb(185pt)=(0.458942,0.418567,0.000626441); rgb(186pt)=(0.452401,0.416515,0.000825833); rgb(187pt)=(0.445885,0.414441,0.00105022); rgb(188pt)=(0.439399,0.412348,0.00129873); rgb(189pt)=(0.432942,0.410234,0.00157049); rgb(190pt)=(0.426518,0.408102,0.00186463); rgb(191pt)=(0.420129,0.40595,0.00218028); rgb(192pt)=(0.413777,0.40378,0.00251655); rgb(193pt)=(0.407464,0.401592,0.00287258); rgb(194pt)=(0.401191,0.399386,0.00324749); rgb(195pt)=(0.394962,0.397164,0.00364042); rgb(196pt)=(0.388777,0.394925,0.00405048); rgb(197pt)=(0.38264,0.39267,0.00447681); rgb(198pt)=(0.376552,0.390399,0.00491852); rgb(199pt)=(0.370516,0.388113,0.00537476); rgb(200pt)=(0.364532,0.385812,0.00584464); rgb(201pt)=(0.358605,0.383497,0.00632729); rgb(202pt)=(0.352735,0.381168,0.00682184); rgb(203pt)=(0.346925,0.378826,0.00732741); rgb(204pt)=(0.341176,0.376471,0.00784314); rgb(205pt)=(0.335485,0.374093,0.00847245); rgb(206pt)=(0.329843,0.371682,0.00930909); rgb(207pt)=(0.324249,0.369242,0.0103377); rgb(208pt)=(0.318701,0.366772,0.0115428); rgb(209pt)=(0.313198,0.364275,0.0129091); rgb(210pt)=(0.307739,0.361753,0.0144211); rgb(211pt)=(0.302322,0.359206,0.0160634); rgb(212pt)=(0.296945,0.356637,0.0178207); rgb(213pt)=(0.291607,0.354048,0.0196776); rgb(214pt)=(0.286307,0.35144,0.0216186); rgb(215pt)=(0.281043,0.348814,0.0236284); rgb(216pt)=(0.275813,0.346172,0.0256916); rgb(217pt)=(0.270616,0.343517,0.0277927); rgb(218pt)=(0.265451,0.340849,0.0299163); rgb(219pt)=(0.260317,0.33817,0.0320472); rgb(220pt)=(0.25521,0.335482,0.0341698); rgb(221pt)=(0.250131,0.332786,0.0362688); rgb(222pt)=(0.245078,0.330085,0.0383287); rgb(223pt)=(0.240048,0.327379,0.0403343); rgb(224pt)=(0.235042,0.324671,0.04227); rgb(225pt)=(0.230056,0.321962,0.0441205); rgb(226pt)=(0.22509,0.319254,0.0458704); rgb(227pt)=(0.220142,0.316548,0.0475043); rgb(228pt)=(0.215212,0.313846,0.0490067); rgb(229pt)=(0.210296,0.311149,0.0503624); rgb(230pt)=(0.205395,0.308459,0.0515759); rgb(231pt)=(0.200514,0.305763,0.052757); rgb(232pt)=(0.195655,0.303061,0.0539242); rgb(233pt)=(0.190817,0.300353,0.0550763); rgb(234pt)=(0.186001,0.297639,0.0562123); rgb(235pt)=(0.181207,0.294918,0.0573313); rgb(236pt)=(0.176434,0.292191,0.0584321); rgb(237pt)=(0.171685,0.289458,0.0595136); rgb(238pt)=(0.166957,0.286719,0.060575); rgb(239pt)=(0.162252,0.283973,0.0616151); rgb(240pt)=(0.15757,0.281221,0.0626328); rgb(241pt)=(0.152911,0.278463,0.0636271); rgb(242pt)=(0.148275,0.275699,0.0645971); rgb(243pt)=(0.143663,0.272929,0.0655416); rgb(244pt)=(0.139074,0.270152,0.0664596); rgb(245pt)=(0.134508,0.26737,0.06735); rgb(246pt)=(0.129967,0.264581,0.0682118); rgb(247pt)=(0.125449,0.261787,0.0690441); rgb(248pt)=(0.120956,0.258986,0.0698456); rgb(249pt)=(0.116487,0.25618,0.0706154); rgb(250pt)=(0.112043,0.253367,0.0713525); rgb(251pt)=(0.107623,0.250549,0.0720557); rgb(252pt)=(0.103229,0.247724,0.0727241); rgb(253pt)=(0.0988592,0.244894,0.0733566); rgb(254pt)=(0.0945149,0.242058,0.0739522); rgb(255pt)=(0.0901961,0.239216,0.0745098)}, mesh/rows=49]
table[row sep=crcr, point meta=\thisrow{c}] {%
%
x	y	z	c\\
56	0.093	-0.0701949061170666	-0.0701949061170666\\
56	0.09666	-0.185388768975631	-0.185388768975631\\
56	0.10032	-0.286875892048822	-0.286875892048822\\
56	0.10398	-0.374656275336632	-0.374656275336632\\
56	0.10764	-0.448729918839061	-0.448729918839061\\
56	0.1113	-0.509096822556106	-0.509096822556106\\
56	0.11496	-0.555756986487776	-0.555756986487776\\
56	0.11862	-0.588710410634068	-0.588710410634068\\
56	0.12228	-0.607957094994979	-0.607957094994979\\
56	0.12594	-0.613497039570511	-0.613497039570511\\
56	0.1296	-0.605330244360657	-0.605330244360657\\
56	0.13326	-0.58345670936543	-0.58345670936543\\
56	0.13692	-0.547876434584824	-0.547876434584824\\
56	0.14058	-0.498589420018842	-0.498589420018842\\
56	0.14424	-0.435595665667467	-0.435595665667467\\
56	0.1479	-0.358895171530722	-0.358895171530722\\
56	0.15156	-0.268487937608599	-0.268487937608599\\
56	0.15522	-0.164373963901093	-0.164373963901093\\
56	0.15888	-0.0465532504082127	-0.0465532504082127\\
56	0.16254	0.0849742028700504	0.0849742028700504\\
56	0.1662	0.230208395933699	0.230208395933699\\
56	0.16986	0.389149328782723	0.389149328782723\\
56	0.17352	0.561797001417125	0.561797001417125\\
56	0.17718	0.748151413836904	0.748151413836904\\
56	0.18084	0.948212566042063	0.948212566042063\\
56	0.1845	1.16198045803261	1.16198045803261\\
56	0.18816	1.38945508980853	1.38945508980853\\
56	0.19182	1.63063646136983	1.63063646136983\\
56	0.19548	1.88552457271651	1.88552457271651\\
56	0.19914	2.15411942384857	2.15411942384857\\
56	0.2028	2.43642101476601	2.43642101476601\\
56	0.20646	2.73242934546883	2.73242934546883\\
56	0.21012	3.04214441595703	3.04214441595703\\
56	0.21378	3.36556622623061	3.36556622623061\\
56	0.21744	3.70269477628955	3.70269477628955\\
56	0.2211	4.0535300661339	4.0535300661339\\
56	0.22476	4.41807209576361	4.41807209576361\\
56	0.22842	4.79632086517871	4.79632086517871\\
56	0.23208	5.18827637437918	5.18827637437918\\
56	0.23574	5.59393862336504	5.59393862336504\\
56	0.2394	6.01330761213627	6.01330761213627\\
56	0.24306	6.44638334069289	6.44638334069289\\
56	0.24672	6.89316580903489	6.89316580903489\\
56	0.25038	7.35365501716225	7.35365501716225\\
56	0.25404	7.827850965075	7.827850965075\\
56	0.2577	8.31575365277314	8.31575365277314\\
56	0.26136	8.81736308025665	8.81736308025665\\
56	0.26502	9.33267924752554	9.33267924752554\\
56	0.26868	9.86170215457981	9.86170215457981\\
56	0.27234	10.4044318014195	10.4044318014195\\
56	0.276	10.9608681880445	10.9608681880445\\
56.375	0.093	-0.0547158919139008	-0.0547158919139008\\
56.375	0.09666	-0.174315266948668	-0.174315266948668\\
56.375	0.10032	-0.280207902198053	-0.280207902198053\\
56.375	0.10398	-0.372393797662061	-0.372393797662061\\
56.375	0.10764	-0.450872953340693	-0.450872953340693\\
56.375	0.1113	-0.51564536923394	-0.51564536923394\\
56.375	0.11496	-0.566711045341809	-0.566711045341809\\
56.375	0.11862	-0.604069981664299	-0.604069981664299\\
56.375	0.12228	-0.627722178201412	-0.627722178201412\\
56.375	0.12594	-0.637667634953143	-0.637667634953143\\
56.375	0.1296	-0.633906351919494	-0.633906351919494\\
56.375	0.13326	-0.616438329100466	-0.616438329100466\\
56.375	0.13692	-0.585263566496058	-0.585263566496058\\
56.375	0.14058	-0.540382064106275	-0.540382064106275\\
56.375	0.14424	-0.481793821931099	-0.481793821931099\\
56.375	0.1479	-0.40949883997056	-0.40949883997056\\
56.375	0.15156	-0.323497118224628	-0.323497118224628\\
56.375	0.15522	-0.223788656693328	-0.223788656693328\\
56.375	0.15888	-0.110373455376639	-0.110373455376639\\
56.375	0.16254	0.0167484857254259	0.0167484857254259\\
56.375	0.1662	0.157577166612869	0.157577166612869\\
56.375	0.16986	0.312112587285695	0.312112587285695\\
56.375	0.17352	0.480354747743897	0.480354747743897\\
56.375	0.17718	0.662303647987478	0.662303647987478\\
56.375	0.18084	0.857959288016438	0.857959288016438\\
56.375	0.1845	1.06732166783078	1.06732166783078\\
56.375	0.18816	1.2903907874305	1.2903907874305\\
56.375	0.19182	1.5271666468156	1.5271666468156\\
56.375	0.19548	1.77764924598608	1.77764924598608\\
56.375	0.19914	2.04183858494194	2.04183858494194\\
56.375	0.2028	2.31973466368319	2.31973466368319\\
56.375	0.20646	2.6113374822098	2.6113374822098\\
56.375	0.21012	2.9166470405218	2.9166470405218\\
56.375	0.21378	3.23566333861918	3.23566333861918\\
56.375	0.21744	3.56838637650193	3.56838637650193\\
56.375	0.2211	3.91481615417007	3.91481615417007\\
56.375	0.22476	4.27495267162359	4.27495267162359\\
56.375	0.22842	4.64879592886248	4.64879592886248\\
56.375	0.23208	5.03634592588676	5.03634592588676\\
56.375	0.23574	5.43760266269642	5.43760266269642\\
56.375	0.2394	5.85256613929144	5.85256613929144\\
56.375	0.24306	6.28123635567187	6.28123635567187\\
56.375	0.24672	6.72361331183765	6.72361331183765\\
56.375	0.25038	7.17969700778883	7.17969700778883\\
56.375	0.25404	7.64948744352538	7.64948744352538\\
56.375	0.2577	8.13298461904731	8.13298461904731\\
56.375	0.26136	8.63018853435463	8.63018853435463\\
56.375	0.26502	9.14109918944732	9.14109918944732\\
56.375	0.26868	9.66571658432538	9.66571658432538\\
56.375	0.27234	10.2040407189888	10.2040407189888\\
56.375	0.276	10.7560715934377	10.7560715934377\\
56.75	0.093	-0.0371066698165734	-0.0371066698165734\\
56.75	0.09666	-0.161111557027539	-0.161111557027539\\
56.75	0.10032	-0.271409704453126	-0.271409704453126\\
56.75	0.10398	-0.368001112093333	-0.368001112093333\\
56.75	0.10764	-0.450885779948163	-0.450885779948163\\
56.75	0.1113	-0.520063708017609	-0.520063708017609\\
56.75	0.11496	-0.57553489630168	-0.57553489630168\\
56.75	0.11862	-0.617299344800372	-0.617299344800372\\
56.75	0.12228	-0.64535705351368	-0.64535705351368\\
56.75	0.12594	-0.65970802244161	-0.65970802244161\\
56.75	0.1296	-0.660352251584159	-0.660352251584159\\
56.75	0.13326	-0.64728974094133	-0.64728974094133\\
56.75	0.13692	-0.620520490513128	-0.620520490513128\\
56.75	0.14058	-0.580044500299543	-0.580044500299543\\
56.75	0.14424	-0.525861770300573	-0.525861770300573\\
56.75	0.1479	-0.457972300516225	-0.457972300516225\\
56.75	0.15156	-0.376376090946499	-0.376376090946499\\
56.75	0.15522	-0.28107314159139	-0.28107314159139\\
56.75	0.15888	-0.172063452450907	-0.172063452450907\\
56.75	0.16254	-0.0493470235250406	-0.0493470235250406\\
56.75	0.1662	0.0870761451862041	0.0870761451862041\\
56.75	0.16986	0.237206053682824	0.237206053682824\\
56.75	0.17352	0.401042701964835	0.401042701964835\\
56.75	0.17718	0.578586090032211	0.578586090032211\\
56.75	0.18084	0.769836217884972	0.769836217884972\\
56.75	0.1845	0.974793085523117	0.974793085523117\\
56.75	0.18816	1.19345669294664	1.19345669294664\\
56.75	0.19182	1.42582704015554	1.42582704015554\\
56.75	0.19548	1.67190412714982	1.67190412714982\\
56.75	0.19914	1.93168795392948	1.93168795392948\\
56.75	0.2028	2.20517852049452	2.20517852049452\\
56.75	0.20646	2.49237582684495	2.49237582684495\\
56.75	0.21012	2.79327987298074	2.79327987298074\\
56.75	0.21378	3.10789065890192	3.10789065890192\\
56.75	0.21744	3.43620818460847	3.43620818460847\\
56.75	0.2211	3.77823245010041	3.77823245010041\\
56.75	0.22476	4.13396345537772	4.13396345537772\\
56.75	0.22842	4.50340120044042	4.50340120044042\\
56.75	0.23208	4.8865456852885	4.8865456852885\\
56.75	0.23574	5.28339690992196	5.28339690992196\\
56.75	0.2394	5.69395487434079	5.69395487434079\\
56.75	0.24306	6.118219578545	6.118219578545\\
56.75	0.24672	6.55619102253461	6.55619102253461\\
56.75	0.25038	7.00786920630957	7.00786920630957\\
56.75	0.25404	7.47325412986992	7.47325412986992\\
56.75	0.2577	7.95234579321566	7.95234579321566\\
56.75	0.26136	8.44514419634677	8.44514419634677\\
56.75	0.26502	8.95164933926326	8.95164933926326\\
56.75	0.26868	9.47186122196513	9.47186122196513\\
56.75	0.27234	10.0057798444524	10.0057798444524\\
56.75	0.276	10.553405206725	10.553405206725\\
57.125	0.093	-0.0173672398250844	-0.0173672398250844\\
57.125	0.09666	-0.145777639212248	-0.145777639212248\\
57.125	0.10032	-0.260481298814038	-0.260481298814038\\
57.125	0.10398	-0.361478218630447	-0.361478218630447\\
57.125	0.10764	-0.448768398661476	-0.448768398661476\\
57.125	0.1113	-0.522351838907124	-0.522351838907124\\
57.125	0.11496	-0.582228539367393	-0.582228539367393\\
57.125	0.11862	-0.62839850004228	-0.62839850004228\\
57.125	0.12228	-0.660861720931787	-0.660861720931787\\
57.125	0.12594	-0.679618202035922	-0.679618202035922\\
57.125	0.1296	-0.68466794335467	-0.68466794335467\\
57.125	0.13326	-0.676010944888047	-0.676010944888047\\
57.125	0.13692	-0.653647206636036	-0.653647206636036\\
57.125	0.14058	-0.61757672859865	-0.61757672859865\\
57.125	0.14424	-0.567799510775878	-0.567799510775878\\
57.125	0.1479	-0.504315553167736	-0.504315553167736\\
57.125	0.15156	-0.427124855774208	-0.427124855774208\\
57.125	0.15522	-0.336227418595298	-0.336227418595298\\
57.125	0.15888	-0.231623241631013	-0.231623241631013\\
57.125	0.16254	-0.113312324881353	-0.113312324881353\\
57.125	0.1662	0.0187053316536936	0.0187053316536936\\
57.125	0.16986	0.164429727974122	0.164429727974122\\
57.125	0.17352	0.32386086407992	0.32386086407992\\
57.125	0.17718	0.496998739971104	0.496998739971104\\
57.125	0.18084	0.683843355647667	0.683843355647667\\
57.125	0.1845	0.884394711109607	0.884394711109607\\
57.125	0.18816	1.09865280635693	1.09865280635693\\
57.125	0.19182	1.32661764138963	1.32661764138963\\
57.125	0.19548	1.56828921620772	1.56828921620772\\
57.125	0.19914	1.82366753081117	1.82366753081117\\
57.125	0.2028	2.09275258520001	2.09275258520001\\
57.125	0.20646	2.37554437937424	2.37554437937424\\
57.125	0.21012	2.67204291333384	2.67204291333384\\
57.125	0.21378	2.98224818707881	2.98224818707881\\
57.125	0.21744	3.30616020060917	3.30616020060917\\
57.125	0.2211	3.64377895392491	3.64377895392491\\
57.125	0.22476	3.99510444702602	3.99510444702602\\
57.125	0.22842	4.36013667991252	4.36013667991252\\
57.125	0.23208	4.7388756525844	4.7388756525844\\
57.125	0.23574	5.13132136504166	5.13132136504166\\
57.125	0.2394	5.53747381728428	5.53747381728428\\
57.125	0.24306	5.9573330093123	5.9573330093123\\
57.125	0.24672	6.3908989411257	6.3908989411257\\
57.125	0.25038	6.83817161272447	6.83817161272447\\
57.125	0.25404	7.29915102410862	7.29915102410862\\
57.125	0.2577	7.77383717527815	7.77383717527815\\
57.125	0.26136	8.26223006623307	8.26223006623307\\
57.125	0.26502	8.76432969697336	8.76432969697336\\
57.125	0.26868	9.28013606749903	9.28013606749903\\
57.125	0.27234	9.80964917781008	9.80964917781008\\
57.125	0.276	10.3528690279065	10.3528690279065\\
57.5	0.093	0.00450239806055208	0.00450239806055208\\
57.5	0.09666	-0.128313513502814	-0.128313513502814\\
57.5	0.10032	-0.247422685280799	-0.247422685280799\\
57.5	0.10398	-0.35282511727341	-0.35282511727341\\
57.5	0.10764	-0.444520809480637	-0.444520809480637\\
57.5	0.1113	-0.522509761902484	-0.522509761902484\\
57.5	0.11496	-0.586791974538952	-0.586791974538952\\
57.5	0.11862	-0.637367447390044	-0.637367447390044\\
57.5	0.12228	-0.67423618045575	-0.67423618045575\\
57.5	0.12594	-0.697398173736083	-0.697398173736083\\
57.5	0.1296	-0.70685342723103	-0.70685342723103\\
57.5	0.13326	-0.702601940940605	-0.702601940940605\\
57.5	0.13692	-0.684643714864793	-0.684643714864793\\
57.5	0.14058	-0.652978749003612	-0.652978749003612\\
57.5	0.14424	-0.607607043357039	-0.607607043357039\\
57.5	0.1479	-0.548528597925088	-0.548528597925088\\
57.5	0.15156	-0.475743412707766	-0.475743412707766\\
57.5	0.15522	-0.389251487705062	-0.389251487705062\\
57.5	0.15888	-0.289052822916975	-0.289052822916975\\
57.5	0.16254	-0.175147418343506	-0.175147418343506\\
57.5	0.1662	-0.0475352739846659	-0.0475352739846659\\
57.5	0.16986	0.0937836101595639	0.0937836101595639\\
57.5	0.17352	0.248809234089171	0.248809234089171\\
57.5	0.17718	0.417541597804149	0.417541597804149\\
57.5	0.18084	0.599980701304506	0.599980701304506\\
57.5	0.1845	0.796126544590255	0.796126544590255\\
57.5	0.18816	1.00597912766137	1.00597912766137\\
57.5	0.19182	1.22953845051788	1.22953845051788\\
57.5	0.19548	1.46680451315976	1.46680451315976\\
57.5	0.19914	1.71777731558702	1.71777731558702\\
57.5	0.2028	1.98245685779966	1.98245685779966\\
57.5	0.20646	2.26084313979769	2.26084313979769\\
57.5	0.21012	2.55293616158108	2.55293616158108\\
57.5	0.21378	2.85873592314986	2.85873592314986\\
57.5	0.21744	3.17824242450401	3.17824242450401\\
57.5	0.2211	3.51145566564355	3.51145566564355\\
57.5	0.22476	3.85837564656846	3.85837564656846\\
57.5	0.22842	4.21900236727876	4.21900236727876\\
57.5	0.23208	4.59333582777444	4.59333582777444\\
57.5	0.23574	4.9813760280555	4.9813760280555\\
57.5	0.2394	5.38312296812193	5.38312296812193\\
57.5	0.24306	5.79857664797375	5.79857664797375\\
57.5	0.24672	6.22773706761095	6.22773706761095\\
57.5	0.25038	6.67060422703352	6.67060422703352\\
57.5	0.25404	7.12717812624147	7.12717812624147\\
57.5	0.2577	7.5974587652348	7.5974587652348\\
57.5	0.26136	8.08144614401352	8.08144614401352\\
57.5	0.26502	8.57914026257761	8.57914026257761\\
57.5	0.26868	9.09054112092707	9.09054112092707\\
57.5	0.27234	9.61564871906193	9.61564871906193\\
57.5	0.276	10.1544630569822	10.1544630569822\\
57.875	0.093	0.0285022438403431	0.0285022438403431\\
57.875	0.09666	-0.108719179899218	-0.108719179899218\\
57.875	0.10032	-0.232233863853408	-0.232233863853408\\
57.875	0.10398	-0.342041808022215	-0.342041808022215\\
57.875	0.10764	-0.438143012405644	-0.438143012405644\\
57.875	0.1113	-0.520537477003693	-0.520537477003693\\
57.875	0.11496	-0.589225201816363	-0.589225201816363\\
57.875	0.11862	-0.644206186843654	-0.644206186843654\\
57.875	0.12228	-0.685480432085558	-0.685480432085558\\
57.875	0.12594	-0.71304793754209	-0.71304793754209\\
57.875	0.1296	-0.726908703213235	-0.726908703213235\\
57.875	0.13326	-0.727062729099009	-0.727062729099009\\
57.875	0.13692	-0.713510015199402	-0.713510015199402\\
57.875	0.14058	-0.686250561514413	-0.686250561514413\\
57.875	0.14424	-0.645284368044045	-0.645284368044045\\
57.875	0.1479	-0.5906114347883	-0.5906114347883\\
57.875	0.15156	-0.52223176174717	-0.52223176174717\\
57.875	0.15522	-0.440145348920657	-0.440145348920657\\
57.875	0.15888	-0.344352196308776	-0.344352196308776\\
57.875	0.16254	-0.234852303911513	-0.234852303911513\\
57.875	0.1662	-0.111645671728864	-0.111645671728864\\
57.875	0.16986	0.0252677002391604	0.0252677002391604\\
57.875	0.17352	0.175887811992569	0.175887811992569\\
57.875	0.17718	0.340214663531341	0.340214663531341\\
57.875	0.18084	0.518248254855507	0.518248254855507\\
57.875	0.1845	0.709988585965057	0.709988585965057\\
57.875	0.18816	0.915435656859973	0.915435656859973\\
57.875	0.19182	1.13458946754027	1.13458946754027\\
57.875	0.19548	1.36745001800596	1.36745001800596\\
57.875	0.19914	1.61401730825702	1.61401730825702\\
57.875	0.2028	1.87429133829346	1.87429133829346\\
57.875	0.20646	2.14827210811528	2.14827210811528\\
57.875	0.21012	2.43595961772248	2.43595961772248\\
57.875	0.21378	2.73735386711506	2.73735386711506\\
57.875	0.21744	3.05245485629301	3.05245485629301\\
57.875	0.2211	3.38126258525635	3.38126258525635\\
57.875	0.22476	3.72377705400507	3.72377705400507\\
57.875	0.22842	4.07999826253916	4.07999826253916\\
57.875	0.23208	4.44992621085864	4.44992621085864\\
57.875	0.23574	4.8335608989635	4.8335608989635\\
57.875	0.2394	5.23090232685373	5.23090232685373\\
57.875	0.24306	5.64195049452935	5.64195049452935\\
57.875	0.24672	6.06670540199034	6.06670540199034\\
57.875	0.25038	6.50516704923672	6.50516704923672\\
57.875	0.25404	6.95733543626847	6.95733543626847\\
57.875	0.2577	7.42321056308562	7.42321056308562\\
57.875	0.26136	7.90279242968812	7.90279242968812\\
57.875	0.26502	8.39608103607601	8.39608103607601\\
57.875	0.26868	8.90307638224928	8.90307638224928\\
57.875	0.27234	9.42377846820793	9.42377846820793\\
57.875	0.276	9.95818729395196	9.95818729395196\\
58.25	0.093	0.0546322975142957	0.0546322975142957\\
58.25	0.09666	-0.086994638401471	-0.086994638401471\\
58.25	0.10032	-0.214914834531856	-0.214914834531856\\
58.25	0.10398	-0.329128290876865	-0.329128290876865\\
58.25	0.10764	-0.429635007436493	-0.429635007436493\\
58.25	0.1113	-0.51643498421074	-0.51643498421074\\
58.25	0.11496	-0.589528221199608	-0.589528221199608\\
58.25	0.11862	-0.648914718403098	-0.648914718403098\\
58.25	0.12228	-0.694594475821208	-0.694594475821208\\
58.25	0.12594	-0.726567493453938	-0.726567493453938\\
58.25	0.1296	-0.744833771301289	-0.744833771301289\\
58.25	0.13326	-0.749393309363255	-0.749393309363255\\
58.25	0.13692	-0.740246107639846	-0.740246107639846\\
58.25	0.14058	-0.717392166131056	-0.717392166131056\\
58.25	0.14424	-0.680831484836894	-0.680831484836894\\
58.25	0.1479	-0.63056406375734	-0.63056406375734\\
58.25	0.15156	-0.566589902892416	-0.566589902892416\\
58.25	0.15522	-0.488909002242108	-0.488909002242108\\
58.25	0.15888	-0.397521361806419	-0.397521361806419\\
58.25	0.16254	-0.292426981585354	-0.292426981585354\\
58.25	0.1662	-0.173625861578911	-0.173625861578911\\
58.25	0.16986	-0.0411180017870851	-0.0411180017870851\\
58.25	0.17352	0.105096597790125	0.105096597790125\\
58.25	0.17718	0.265017937152699	0.265017937152699\\
58.25	0.18084	0.438646016300666	0.438646016300666\\
58.25	0.1845	0.625980835234003	0.625980835234003\\
58.25	0.18816	0.827022393952728	0.827022393952728\\
58.25	0.19182	1.04177069245683	1.04177069245683\\
58.25	0.19548	1.27022573074632	1.27022573074632\\
58.25	0.19914	1.51238750882117	1.51238750882117\\
58.25	0.2028	1.76825602668141	1.76825602668141\\
58.25	0.20646	2.03783128432704	2.03783128432704\\
58.25	0.21012	2.32111328175803	2.32111328175803\\
58.25	0.21378	2.61810201897442	2.61810201897442\\
58.25	0.21744	2.92879749597617	2.92879749597617\\
58.25	0.2211	3.25319971276331	3.25319971276331\\
58.25	0.22476	3.59130866933583	3.59130866933583\\
58.25	0.22842	3.94312436569372	3.94312436569372\\
58.25	0.23208	4.308646801837	4.308646801837\\
58.25	0.23574	4.68787597776566	4.68787597776566\\
58.25	0.2394	5.08081189347969	5.08081189347969\\
58.25	0.24306	5.48745454897911	5.48745454897911\\
58.25	0.24672	5.9078039442639	5.9078039442639\\
58.25	0.25038	6.34186007933408	6.34186007933408\\
58.25	0.25404	6.78962295418963	6.78962295418963\\
58.25	0.2577	7.25109256883058	7.25109256883058\\
58.25	0.26136	7.72626892325688	7.72626892325688\\
58.25	0.26502	8.21515201746858	8.21515201746858\\
58.25	0.26868	8.71774185146564	8.71774185146564\\
58.25	0.27234	9.23403842524809	9.23403842524809\\
58.25	0.276	9.76404173881592	9.76404173881592\\
58.625	0.093	0.0828925590824028	0.0828925590824028\\
58.625	0.09666	-0.0631398890095625	-0.0631398890095625\\
58.625	0.10032	-0.19546559731615	-0.19546559731615\\
58.625	0.10398	-0.314084565837357	-0.314084565837357\\
58.625	0.10764	-0.418996794573187	-0.418996794573187\\
58.625	0.1113	-0.510202283523633	-0.510202283523633\\
58.625	0.11496	-0.5877010326887	-0.5877010326887\\
58.625	0.11862	-0.651493042068388	-0.651493042068388\\
58.625	0.12228	-0.701578311662696	-0.701578311662696\\
58.625	0.12594	-0.737956841471625	-0.737956841471625\\
58.625	0.1296	-0.760628631495175	-0.760628631495175\\
58.625	0.13326	-0.769593681733346	-0.769593681733346\\
58.625	0.13692	-0.764851992186136	-0.764851992186136\\
58.625	0.14058	-0.746403562853544	-0.746403562853544\\
58.625	0.14424	-0.714248393735573	-0.714248393735573\\
58.625	0.1479	-0.668386484832226	-0.668386484832226\\
58.625	0.15156	-0.608817836143507	-0.608817836143507\\
58.625	0.15522	-0.535542447669391	-0.535542447669391\\
58.625	0.15888	-0.448560319409907	-0.448560319409907\\
58.625	0.16254	-0.347871451365041	-0.347871451365041\\
58.625	0.1662	-0.233475843534796	-0.233475843534796\\
58.625	0.16986	-0.105373495919169	-0.105373495919169\\
58.625	0.17352	0.0364355914818351	0.0364355914818351\\
58.625	0.17718	0.191951418668211	0.191951418668211\\
58.625	0.18084	0.361173985639979	0.361173985639979\\
58.625	0.1845	0.544103292397118	0.544103292397118\\
58.625	0.18816	0.740739338939644	0.740739338939644\\
58.625	0.19182	0.951082125267547	0.951082125267547\\
58.625	0.19548	1.17513165138083	1.17513165138083\\
58.625	0.19914	1.41288791727949	1.41288791727949\\
58.625	0.2028	1.66435092296354	1.66435092296354\\
58.625	0.20646	1.92952066843296	1.92952066843296\\
58.625	0.21012	2.20839715368775	2.20839715368775\\
58.625	0.21378	2.50098037872793	2.50098037872793\\
58.625	0.21744	2.80727034355349	2.80727034355349\\
58.625	0.2211	3.12726704816443	3.12726704816443\\
58.625	0.22476	3.46097049256075	3.46097049256075\\
58.625	0.22842	3.80838067674244	3.80838067674244\\
58.625	0.23208	4.16949760070952	4.16949760070952\\
58.625	0.23574	4.54432126446198	4.54432126446198\\
58.625	0.2394	4.93285166799981	4.93285166799981\\
58.625	0.24306	5.33508881132303	5.33508881132303\\
58.625	0.24672	5.75103269443162	5.75103269443162\\
58.625	0.25038	6.1806833173256	6.1806833173256\\
58.625	0.25404	6.62404068000495	6.62404068000495\\
58.625	0.2577	7.08110478246969	7.08110478246969\\
58.625	0.26136	7.55187562471981	7.55187562471981\\
58.625	0.26502	8.03635320675529	8.03635320675529\\
58.625	0.26868	8.53453752857617	8.53453752857617\\
58.625	0.27234	9.04642859018242	9.04642859018242\\
58.625	0.276	9.57202639157405	9.57202639157405\\
59	0.093	0.113283028544668	0.113283028544668\\
59	0.09666	-0.0371549317234994	-0.0371549317234994\\
59	0.10032	-0.173886152206282	-0.173886152206282\\
59	0.10398	-0.296910632903694	-0.296910632903694\\
59	0.10764	-0.406228373815712	-0.406228373815712\\
59	0.1113	-0.501839374942364	-0.501839374942364\\
59	0.11496	-0.583743636283629	-0.583743636283629\\
59	0.11862	-0.651941157839516	-0.651941157839516\\
59	0.12228	-0.706431939610023	-0.706431939610023\\
59	0.12594	-0.747215981595158	-0.747215981595158\\
59	0.1296	-0.774293283794906	-0.774293283794906\\
59	0.13326	-0.787663846209275	-0.787663846209275\\
59	0.13692	-0.787327668838264	-0.787327668838264\\
59	0.14058	-0.773284751681878	-0.773284751681878\\
59	0.14424	-0.745535094740113	-0.745535094740113\\
59	0.1479	-0.704078698012957	-0.704078698012957\\
59	0.15156	-0.648915561500436	-0.648915561500436\\
59	0.15522	-0.580045685202519	-0.580045685202519\\
59	0.15888	-0.497469069119234	-0.497469069119234\\
59	0.16254	-0.401185713250573	-0.401185713250573\\
59	0.1662	-0.291195617596527	-0.291195617596527\\
59	0.16986	-0.167498782157098	-0.167498782157098\\
59	0.17352	-0.0300952069322928	-0.0300952069322928\\
59	0.17718	0.121015108077884	0.121015108077884\\
59	0.18084	0.285832162873447	0.285832162873447\\
59	0.1845	0.464355957454401	0.464355957454401\\
59	0.18816	0.656586491820715	0.656586491820715\\
59	0.19182	0.862523765972419	0.862523765972419\\
59	0.19548	1.0821677799095	1.0821677799095\\
59	0.19914	1.31551853363197	1.31551853363197\\
59	0.2028	1.56257602713981	1.56257602713981\\
59	0.20646	1.82334026043302	1.82334026043302\\
59	0.21012	2.09781123351163	2.09781123351163\\
59	0.21378	2.38598894637561	2.38598894637561\\
59	0.21744	2.68787339902496	2.68787339902496\\
59	0.2211	3.0034645914597	3.0034645914597\\
59	0.22476	3.33276252367981	3.33276252367981\\
59	0.22842	3.67576719568532	3.67576719568532\\
59	0.23208	4.0324786074762	4.0324786074762\\
59	0.23574	4.40289675905245	4.40289675905245\\
59	0.2394	4.78702165041409	4.78702165041409\\
59	0.24306	5.18485328156111	5.18485328156111\\
59	0.24672	5.59639165249351	5.59639165249351\\
59	0.25038	6.02163676321128	6.02163676321128\\
59	0.25404	6.46058861371443	6.46058861371443\\
59	0.2577	6.91324720400298	6.91324720400298\\
59	0.26136	7.37961253407688	7.37961253407688\\
59	0.26502	7.85968460393617	7.85968460393617\\
59	0.26868	8.35346341358084	8.35346341358084\\
59	0.27234	8.86094896301089	8.86094896301089\\
59	0.276	9.38214125222633	9.38214125222633\\
59.375	0.093	0.145803705901084	0.145803705901084\\
59.375	0.09666	-0.0090397665432782	-0.0090397665432782\\
59.375	0.10032	-0.150176499202266	-0.150176499202266\\
59.375	0.10398	-0.27760649207587	-0.27760649207587\\
59.375	0.10764	-0.391329745164094	-0.391329745164094\\
59.375	0.1113	-0.491346258466944	-0.491346258466944\\
59.375	0.11496	-0.577656031984415	-0.577656031984415\\
59.375	0.11862	-0.650259065716501	-0.650259065716501\\
59.375	0.12228	-0.709155359663206	-0.709155359663206\\
59.375	0.12594	-0.754344913824539	-0.754344913824539\\
59.375	0.1296	-0.785827728200486	-0.785827728200486\\
59.375	0.13326	-0.803603802791061	-0.803603802791061\\
59.375	0.13692	-0.807673137596248	-0.807673137596248\\
59.375	0.14058	-0.798035732616054	-0.798035732616054\\
59.375	0.14424	-0.774691587850487	-0.774691587850487\\
59.375	0.1479	-0.737640703299544	-0.737640703299544\\
59.375	0.15156	-0.686883078963215	-0.686883078963215\\
59.375	0.15522	-0.622418714841503	-0.622418714841503\\
59.375	0.15888	-0.544247610934416	-0.544247610934416\\
59.375	0.16254	-0.452369767241947	-0.452369767241947\\
59.375	0.1662	-0.346785183764107	-0.346785183764107\\
59.375	0.16986	-0.227493860500877	-0.227493860500877\\
59.375	0.17352	-0.0944957974522698	-0.0944957974522698\\
59.375	0.17718	0.0522090053817088	0.0522090053817088\\
59.375	0.18084	0.212620548001073	0.212620548001073\\
59.375	0.1845	0.386738830405815	0.386738830405815\\
59.375	0.18816	0.574563852595936	0.574563852595936\\
59.375	0.19182	0.776095614571442	0.776095614571442\\
59.375	0.19548	0.991334116332322	0.991334116332322\\
59.375	0.19914	1.22027935787859	1.22027935787859\\
59.375	0.2028	1.46293133921023	1.46293133921023\\
59.375	0.20646	1.71929006032726	1.71929006032726\\
59.375	0.21012	1.98935552122965	1.98935552122965\\
59.375	0.21378	2.27312772191743	2.27312772191743\\
59.375	0.21744	2.57060666239059	2.57060666239059\\
59.375	0.2211	2.88179234264913	2.88179234264913\\
59.375	0.22476	3.20668476269305	3.20668476269305\\
59.375	0.22842	3.54528392252234	3.54528392252234\\
59.375	0.23208	3.89758982213703	3.89758982213703\\
59.375	0.23574	4.26360246153709	4.26360246153709\\
59.375	0.2394	4.64332184072251	4.64332184072251\\
59.375	0.24306	5.03674795969333	5.03674795969333\\
59.375	0.24672	5.44388081844953	5.44388081844953\\
59.375	0.25038	5.86472041699111	5.86472041699111\\
59.375	0.25404	6.29926675531806	6.29926675531806\\
59.375	0.2577	6.74751983343041	6.74751983343041\\
59.375	0.26136	7.20947965132811	7.20947965132811\\
59.375	0.26502	7.6851462090112	7.6851462090112\\
59.375	0.26868	8.17451950647967	8.17451950647967\\
59.375	0.27234	8.67759954373352	8.67759954373352\\
59.375	0.276	9.19438632077276	9.19438632077276\\
59.75	0.093	0.180454591151658	0.180454591151658\\
59.75	0.09666	0.0212056065310904	0.0212056065310904\\
59.75	0.10032	-0.124336638304096	-0.124336638304096\\
59.75	0.10398	-0.256172143353899	-0.256172143353899\\
59.75	0.10764	-0.374300908618321	-0.374300908618321\\
59.75	0.1113	-0.47872293409737	-0.47872293409737\\
59.75	0.11496	-0.569438219791039	-0.569438219791039\\
59.75	0.11862	-0.646446765699331	-0.646446765699331\\
59.75	0.12228	-0.709748571822234	-0.709748571822234\\
59.75	0.12594	-0.759343638159759	-0.759343638159759\\
59.75	0.1296	-0.795231964711911	-0.795231964711911\\
59.75	0.13326	-0.817413551478685	-0.817413551478685\\
59.75	0.13692	-0.825888398460071	-0.825888398460071\\
59.75	0.14058	-0.820656505656082	-0.820656505656082\\
59.75	0.14424	-0.801717873066707	-0.801717873066707\\
59.75	0.1479	-0.769072500691962	-0.769072500691962\\
59.75	0.15156	-0.722720388531839	-0.722720388531839\\
59.75	0.15522	-0.662661536586326	-0.662661536586326\\
59.75	0.15888	-0.588895944855437	-0.588895944855437\\
59.75	0.16254	-0.501423613339174	-0.501423613339174\\
59.75	0.1662	-0.400244542037525	-0.400244542037525\\
59.75	0.16986	-0.285358730950501	-0.285358730950501\\
59.75	0.17352	-0.156766180078092	-0.156766180078092\\
59.75	0.17718	-0.0144668894203122	-0.0144668894203122\\
59.75	0.18084	0.141539141022854	0.141539141022854\\
59.75	0.1845	0.311251911251397	0.311251911251397\\
59.75	0.18816	0.49467142126532	0.49467142126532\\
59.75	0.19182	0.69179767106462	0.69179767106462\\
59.75	0.19548	0.902630660649301	0.902630660649301\\
59.75	0.19914	1.12717039001937	1.12717039001937\\
59.75	0.2028	1.36541685917481	1.36541685917481\\
59.75	0.20646	1.61737006811564	1.61737006811564\\
59.75	0.21012	1.88303001684183	1.88303001684183\\
59.75	0.21378	2.16239670535341	2.16239670535341\\
59.75	0.21744	2.45547013365037	2.45547013365037\\
59.75	0.2211	2.76225030173271	2.76225030173271\\
59.75	0.22476	3.08273720960043	3.08273720960043\\
59.75	0.22842	3.41693085725352	3.41693085725352\\
59.75	0.23208	3.764831244692	3.764831244692\\
59.75	0.23574	4.12643837191586	4.12643837191586\\
59.75	0.2394	4.5017522389251	4.5017522389251\\
59.75	0.24306	4.89077284571972	4.89077284571972\\
59.75	0.24672	5.29350019229971	5.29350019229971\\
59.75	0.25038	5.70993427866509	5.70993427866509\\
59.75	0.25404	6.14007510481584	6.14007510481584\\
59.75	0.2577	6.58392267075199	6.58392267075199\\
59.75	0.26136	7.04147697647349	7.04147697647349\\
59.75	0.26502	7.51273802198038	7.51273802198038\\
59.75	0.26868	7.99770580727265	7.99770580727265\\
59.75	0.27234	8.49638033235031	8.49638033235031\\
59.75	0.276	9.00876159721334	9.00876159721334\\
60.125	0.093	0.217235684296384	0.217235684296384\\
60.125	0.09666	0.0535811874996206	0.0535811874996206\\
60.125	0.10032	-0.0963665695117646	-0.0963665695117646\\
60.125	0.10398	-0.232607586737773	-0.232607586737773\\
60.125	0.10764	-0.355141864178393	-0.355141864178393\\
60.125	0.1113	-0.463969401833648	-0.463969401833648\\
60.125	0.11496	-0.559090199703509	-0.559090199703509\\
60.125	0.11862	-0.640504257787999	-0.640504257787999\\
60.125	0.12228	-0.708211576087108	-0.708211576087108\\
60.125	0.12594	-0.762212154600832	-0.762212154600832\\
60.125	0.1296	-0.802505993329182	-0.802505993329182\\
60.125	0.13326	-0.829093092272155	-0.829093092272155\\
60.125	0.13692	-0.841973451429739	-0.841973451429739\\
60.125	0.14058	-0.841147070801956	-0.841147070801956\\
60.125	0.14424	-0.826613950388779	-0.826613950388779\\
60.125	0.1479	-0.798374090190233	-0.798374090190233\\
60.125	0.15156	-0.756427490206308	-0.756427490206308\\
60.125	0.15522	-0.700774150436994	-0.700774150436994\\
60.125	0.15888	-0.631414070882311	-0.631414070882311\\
60.125	0.16254	-0.548347251542239	-0.548347251542239\\
60.125	0.1662	-0.451573692416803	-0.451573692416803\\
60.125	0.16986	-0.34109339350597	-0.34109339350597\\
60.125	0.17352	-0.21690635480976	-0.21690635480976\\
60.125	0.17718	-0.0790125763281786	-0.0790125763281786\\
60.125	0.18084	0.0725879419387816	0.0725879419387816\\
60.125	0.1845	0.237895199991133	0.237895199991133\\
60.125	0.18816	0.416909197828851	0.416909197828851\\
60.125	0.19182	0.609629935451959	0.609629935451959\\
60.125	0.19548	0.816057412860435	0.816057412860435\\
60.125	0.19914	1.0361916300543	1.0361916300543\\
60.125	0.2028	1.27003258703354	1.27003258703354\\
60.125	0.20646	1.51758028379817	1.51758028379817\\
60.125	0.21012	1.77883472034816	1.77883472034816\\
60.125	0.21378	2.05379589668355	2.05379589668355\\
60.125	0.21744	2.3424638128043	2.3424638128043\\
60.125	0.2211	2.64483846871044	2.64483846871044\\
60.125	0.22476	2.96091986440196	2.96091986440196\\
60.125	0.22842	3.29070799987886	3.29070799987886\\
60.125	0.23208	3.63420287514114	3.63420287514114\\
60.125	0.23574	3.9914044901888	3.9914044901888\\
60.125	0.2394	4.36231284502183	4.36231284502183\\
60.125	0.24306	4.74692793964025	4.74692793964025\\
60.125	0.24672	5.14524977404405	5.14524977404405\\
60.125	0.25038	5.55727834823322	5.55727834823322\\
60.125	0.25404	5.98301366220778	5.98301366220778\\
60.125	0.2577	6.42245571596773	6.42245571596773\\
60.125	0.26136	6.87560450951303	6.87560450951303\\
60.125	0.26502	7.34246004284372	7.34246004284372\\
60.125	0.26868	7.82302231595979	7.82302231595979\\
60.125	0.27234	8.31729132886124	8.31729132886124\\
60.125	0.276	8.82526708154808	8.82526708154808\\
60.5	0.093	0.25614698533527	0.25614698533527\\
60.5	0.09666	0.0880869763623089	0.0880869763623089\\
60.5	0.10032	-0.0662662928252749	-0.0662662928252749\\
60.5	0.10398	-0.206912822227482	-0.206912822227482\\
60.5	0.10764	-0.333852611844308	-0.333852611844308\\
60.5	0.1113	-0.447085661675754	-0.447085661675754\\
60.5	0.11496	-0.546611971721821	-0.546611971721821\\
60.5	0.11862	-0.632431541982509	-0.632431541982509\\
60.5	0.12228	-0.704544372457817	-0.704544372457817\\
60.5	0.12594	-0.762950463147746	-0.762950463147746\\
60.5	0.1296	-0.807649814052295	-0.807649814052295\\
60.5	0.13326	-0.838642425171466	-0.838642425171466\\
60.5	0.13692	-0.855928296505256	-0.855928296505256\\
60.5	0.14058	-0.859507428053664	-0.859507428053664\\
60.5	0.14424	-0.849379819816694	-0.849379819816694\\
60.5	0.1479	-0.825545471794346	-0.825545471794346\\
60.5	0.15156	-0.788004383986612	-0.788004383986612\\
60.5	0.15522	-0.736756556393503	-0.736756556393503\\
60.5	0.15888	-0.671801989015012	-0.671801989015012\\
60.5	0.16254	-0.593140681851146	-0.593140681851146\\
60.5	0.1662	-0.500772634901908	-0.500772634901908\\
60.5	0.16986	-0.394697848167274	-0.394697848167274\\
60.5	0.17352	-0.27491632164727	-0.27491632164727\\
60.5	0.17718	-0.141428055341887	-0.141428055341887\\
60.5	0.18084	0.00576695074888178	0.00576695074888178\\
60.5	0.1845	0.166668696625027	0.166668696625027\\
60.5	0.18816	0.341277182286547	0.341277182286547\\
60.5	0.19182	0.52959240773345	0.52959240773345\\
60.5	0.19548	0.731614372965741	0.731614372965741\\
60.5	0.19914	0.947343077983394	0.947343077983394\\
60.5	0.2028	1.17677852278644	1.17677852278644\\
60.5	0.20646	1.41992070737487	1.41992070737487\\
60.5	0.21012	1.67676963174866	1.67676963174866\\
60.5	0.21378	1.94732529590785	1.94732529590785\\
60.5	0.21744	2.2315876998524	2.2315876998524\\
60.5	0.2211	2.52955684358234	2.52955684358234\\
60.5	0.22476	2.84123272709766	2.84123272709766\\
60.5	0.22842	3.16661535039836	3.16661535039836\\
60.5	0.23208	3.50570471348444	3.50570471348444\\
60.5	0.23574	3.8585008163559	3.8585008163559\\
60.5	0.2394	4.22500365901274	4.22500365901274\\
60.5	0.24306	4.60521324145495	4.60521324145495\\
60.5	0.24672	4.99912956368255	4.99912956368255\\
60.5	0.25038	5.40675262569553	5.40675262569553\\
60.5	0.25404	5.82808242749388	5.82808242749388\\
60.5	0.2577	6.26311896907762	6.26311896907762\\
60.5	0.26136	6.71186225044673	6.71186225044673\\
60.5	0.26502	7.17431227160123	7.17431227160123\\
60.5	0.26868	7.65046903254109	7.65046903254109\\
60.5	0.27234	8.14033253326635	8.14033253326635\\
60.5	0.276	8.64390277377698	8.64390277377698\\
60.875	0.093	0.297188494268319	0.297188494268319\\
60.875	0.09666	0.124722973119155	0.124722973119155\\
60.875	0.10032	-0.034035808244627	-0.034035808244627\\
60.875	0.10398	-0.179087849823032	-0.179087849823032\\
60.875	0.10764	-0.310433151616057	-0.310433151616057\\
60.875	0.1113	-0.428071713623702	-0.428071713623702\\
60.875	0.11496	-0.532003535845974	-0.532003535845974\\
60.875	0.11862	-0.622228618282854	-0.622228618282854\\
60.875	0.12228	-0.69874696093436	-0.69874696093436\\
60.875	0.12594	-0.761558563800495	-0.761558563800495\\
60.875	0.1296	-0.810663426881243	-0.810663426881243\\
60.875	0.13326	-0.846061550176605	-0.846061550176605\\
60.875	0.13692	-0.867752933686601	-0.867752933686601\\
60.875	0.14058	-0.875737577411208	-0.875737577411208\\
60.875	0.14424	-0.870015481350435	-0.870015481350435\\
60.875	0.1479	-0.850586645504293	-0.850586645504293\\
60.875	0.15156	-0.817451069872758	-0.817451069872758\\
60.875	0.15522	-0.770608754455848	-0.770608754455848\\
60.875	0.15888	-0.710059699253563	-0.710059699253563\\
60.875	0.16254	-0.635803904265895	-0.635803904265895\\
60.875	0.1662	-0.547841369492842	-0.547841369492842\\
60.875	0.16986	-0.44617209493442	-0.44617209493442\\
60.875	0.17352	-0.330796080590614	-0.330796080590614\\
60.875	0.17718	-0.20171332646143	-0.20171332646143\\
60.875	0.18084	-0.0589238325468671	-0.0589238325468671\\
60.875	0.1845	0.0975724011530801	0.0975724011530801\\
60.875	0.18816	0.267775374638401	0.267775374638401\\
60.875	0.19182	0.451685087909105	0.451685087909105\\
60.875	0.19548	0.64930154096519	0.64930154096519\\
60.875	0.19914	0.860624733806652	0.860624733806652\\
60.875	0.2028	1.0856546664335	1.0856546664335\\
60.875	0.20646	1.32439133884572	1.32439133884572\\
60.875	0.21012	1.57683475104333	1.57683475104333\\
60.875	0.21378	1.8429849030263	1.8429849030263\\
60.875	0.21744	2.12284179479466	2.12284179479466\\
60.875	0.2211	2.4164054263484	2.4164054263484\\
60.875	0.22476	2.72367579768752	2.72367579768752\\
60.875	0.22842	3.04465290881202	3.04465290881202\\
60.875	0.23208	3.3793367597219	3.3793367597219\\
60.875	0.23574	3.72772735041716	3.72772735041716\\
60.875	0.2394	4.0898246808978	4.0898246808978\\
60.875	0.24306	4.46562875116381	4.46562875116381\\
60.875	0.24672	4.8551395612152	4.8551395612152\\
60.875	0.25038	5.25835711105199	5.25835711105199\\
60.875	0.25404	5.67528140067414	5.67528140067414\\
60.875	0.2577	6.10591243008169	6.10591243008169\\
60.875	0.26136	6.55025019927459	6.55025019927459\\
60.875	0.26502	7.00829470825288	7.00829470825288\\
60.875	0.26868	7.48004595701656	7.48004595701656\\
60.875	0.27234	7.96550394556561	7.96550394556561\\
60.875	0.276	8.46466867390004	8.46466867390004\\
61.25	0.093	0.340360211095518	0.340360211095518\\
61.25	0.09666	0.163489177770156	0.163489177770156\\
61.25	0.10032	0.000324884230175293	0.000324884230175293\\
61.25	0.10398	-0.149132669524436	-0.149132669524436\\
61.25	0.10764	-0.284883483493666	-0.284883483493666\\
61.25	0.1113	-0.406927557677509	-0.406927557677509\\
61.25	0.11496	-0.515264892075973	-0.515264892075973\\
61.25	0.11862	-0.609895486689059	-0.609895486689059\\
61.25	0.12228	-0.690819341516764	-0.690819341516764\\
61.25	0.12594	-0.75803645655909	-0.75803645655909\\
61.25	0.1296	-0.811546831816036	-0.811546831816036\\
61.25	0.13326	-0.851350467287611	-0.851350467287611\\
61.25	0.13692	-0.877447362973799	-0.877447362973799\\
61.25	0.14058	-0.889837518874611	-0.889837518874611\\
61.25	0.14424	-0.888520934990037	-0.888520934990037\\
61.25	0.1479	-0.873497611320087	-0.873497611320087\\
61.25	0.15156	-0.844767547864757	-0.844767547864757\\
61.25	0.15522	-0.802330744624046	-0.802330744624046\\
61.25	0.15888	-0.746187201597959	-0.746187201597959\\
61.25	0.16254	-0.676336918786497	-0.676336918786497\\
61.25	0.1662	-0.592779896189649	-0.592779896189649\\
61.25	0.16986	-0.495516133807419	-0.495516133807419\\
61.25	0.17352	-0.384545631639812	-0.384545631639812\\
61.25	0.17718	-0.259868389686826	-0.259868389686826\\
61.25	0.18084	-0.121484407948461	-0.121484407948461\\
61.25	0.1845	0.0306063135752872	0.0306063135752872\\
61.25	0.18816	0.196403774884402	0.196403774884402\\
61.25	0.19182	0.375907975978915	0.375907975978915\\
61.25	0.19548	0.569118916858795	0.569118916858795\\
61.25	0.19914	0.776036597524058	0.776036597524058\\
61.25	0.2028	0.996661017974699	0.996661017974699\\
61.25	0.20646	1.23099217821073	1.23099217821073\\
61.25	0.21012	1.47903007823213	1.47903007823213\\
61.25	0.21378	1.74077471803891	1.74077471803891\\
61.25	0.21744	2.01622609763106	2.01622609763106\\
61.25	0.2211	2.3053842170086	2.3053842170086\\
61.25	0.22476	2.60824907617152	2.60824907617152\\
61.25	0.22842	2.92482067511983	2.92482067511983\\
61.25	0.23208	3.25509901385351	3.25509901385351\\
61.25	0.23574	3.59908409237256	3.59908409237256\\
61.25	0.2394	3.956775910677	3.956775910677\\
61.25	0.24306	4.32817446876682	4.32817446876682\\
61.25	0.24672	4.71327976664201	4.71327976664201\\
61.25	0.25038	5.11209180430259	5.11209180430259\\
61.25	0.25404	5.52461058174855	5.52461058174855\\
61.25	0.2577	5.9508360989799	5.9508360989799\\
61.25	0.26136	6.3907683559966	6.3907683559966\\
61.25	0.26502	6.84440735279869	6.84440735279869\\
61.25	0.26868	7.31175308938616	7.31175308938616\\
61.25	0.27234	7.79280556575902	7.79280556575902\\
61.25	0.276	8.28756478191725	8.28756478191725\\
61.625	0.093	0.385662135816865	0.385662135816865\\
61.625	0.09666	0.204385590315304	0.204385590315304\\
61.625	0.10032	0.036815784599125	0.036815784599125\\
61.625	0.10398	-0.117047281331685	-0.117047281331685\\
61.625	0.10764	-0.257203607477114	-0.257203607477114\\
61.625	0.1113	-0.383653193837148	-0.383653193837148\\
61.625	0.11496	-0.496396040411818	-0.496396040411818\\
61.625	0.11862	-0.595432147201109	-0.595432147201109\\
61.625	0.12228	-0.680761514205012	-0.680761514205012\\
61.625	0.12594	-0.752384141423537	-0.752384141423537\\
61.625	0.1296	-0.810300028856689	-0.810300028856689\\
61.625	0.13326	-0.854509176504456	-0.854509176504456\\
61.625	0.13692	-0.885011584366842	-0.885011584366842\\
61.625	0.14058	-0.901807252443852	-0.901807252443852\\
61.625	0.14424	-0.904896180735484	-0.904896180735484\\
61.625	0.1479	-0.894278369241732	-0.894278369241732\\
61.625	0.15156	-0.869953817962601	-0.869953817962601\\
61.625	0.15522	-0.831922526898095	-0.831922526898095\\
61.625	0.15888	-0.780184496048207	-0.780184496048207\\
61.625	0.16254	-0.714739725412937	-0.714739725412937\\
61.625	0.1662	-0.635588214992287	-0.635588214992287\\
61.625	0.16986	-0.542729964786263	-0.542729964786263\\
61.625	0.17352	-0.436164974794854	-0.436164974794854\\
61.625	0.17718	-0.315893245018074	-0.315893245018074\\
61.625	0.18084	-0.181914775455908	-0.181914775455908\\
61.625	0.1845	-0.0342295661083654	-0.0342295661083654\\
61.625	0.18816	0.127162383024565	0.127162383024565\\
61.625	0.19182	0.302261071942873	0.302261071942873\\
61.625	0.19548	0.491066500646554	0.491066500646554\\
61.625	0.19914	0.693578669135618	0.693578669135618\\
61.625	0.2028	0.90979757741006	0.90979757741006\\
61.625	0.20646	1.13972322546989	1.13972322546989\\
61.625	0.21012	1.38335561331509	1.38335561331509\\
61.625	0.21378	1.64069474094567	1.64069474094567\\
61.625	0.21744	1.91174060836162	1.91174060836162\\
61.625	0.2211	2.19649321556297	2.19649321556297\\
61.625	0.22476	2.49495256254968	2.49495256254968\\
61.625	0.22842	2.80711864932178	2.80711864932178\\
61.625	0.23208	3.13299147587927	3.13299147587927\\
61.625	0.23574	3.47257104222213	3.47257104222213\\
61.625	0.2394	3.82585734835036	3.82585734835036\\
61.625	0.24306	4.19285039426398	4.19285039426398\\
61.625	0.24672	4.57355017996298	4.57355017996298\\
61.625	0.25038	4.96795670544736	4.96795670544736\\
61.625	0.25404	5.37606997071711	5.37606997071711\\
61.625	0.2577	5.79788997577226	5.79788997577226\\
61.625	0.26136	6.23341672061277	6.23341672061277\\
61.625	0.26502	6.68265020523866	6.68265020523866\\
61.625	0.26868	7.14559042964993	7.14559042964993\\
61.625	0.27234	7.62223739384658	7.62223739384658\\
61.625	0.276	8.11259109782861	8.11259109782861\\
62	0.093	0.433094268432381	0.433094268432381\\
62	0.09666	0.247412210754614	0.247412210754614\\
62	0.10032	0.0754368928622293	0.0754368928622293\\
62	0.10398	-0.0828316852447717	-0.0828316852447717\\
62	0.10764	-0.227393523566399	-0.227393523566399\\
62	0.1113	-0.358248622102646	-0.358248622102646\\
62	0.11496	-0.475396980853508	-0.475396980853508\\
62	0.11862	-0.57883859981899	-0.57883859981899\\
62	0.12228	-0.668573478999099	-0.668573478999099\\
62	0.12594	-0.74460161839383	-0.74460161839383\\
62	0.1296	-0.806923018003173	-0.806923018003173\\
62	0.13326	-0.855537677827138	-0.855537677827138\\
62	0.13692	-0.890445597865737	-0.890445597865737\\
62	0.14058	-0.911646778118939	-0.911646778118939\\
62	0.14424	-0.91914121858677	-0.91914121858677\\
62	0.1479	-0.912928919269223	-0.912928919269223\\
62	0.15156	-0.893009880166291	-0.893009880166291\\
62	0.15522	-0.859384101277977	-0.859384101277977\\
62	0.15888	-0.812051582604287	-0.812051582604287\\
62	0.16254	-0.751012324145222	-0.751012324145222\\
62	0.1662	-0.676266325900771	-0.676266325900771\\
62	0.16986	-0.587813587870945	-0.587813587870945\\
62	0.17352	-0.485654110055735	-0.485654110055735\\
62	0.17718	-0.369787892455154	-0.369787892455154\\
62	0.18084	-0.240214935069186	-0.240214935069186\\
62	0.1845	-0.096935237897835	-0.096935237897835\\
62	0.18816	0.0600511990588828	0.0600511990588828\\
62	0.19182	0.230744375800985	0.230744375800985\\
62	0.19548	0.415144292328467	0.415144292328467\\
62	0.19914	0.613250948641333	0.613250948641333\\
62	0.2028	0.825064344739577	0.825064344739577\\
62	0.20646	1.0505844806232	1.0505844806232\\
62	0.21012	1.2898113562922	1.2898113562922\\
62	0.21378	1.54274497174659	1.54274497174659\\
62	0.21744	1.80938532698634	1.80938532698634\\
62	0.2211	2.08973242201149	2.08973242201149\\
62	0.22476	2.38378625682201	2.38378625682201\\
62	0.22842	2.6915468314179	2.6915468314179\\
62	0.23208	3.01301414579919	3.01301414579919\\
62	0.23574	3.34818819996585	3.34818819996585\\
62	0.2394	3.69706899391788	3.69706899391788\\
62	0.24306	4.0596565276553	4.0596565276553\\
62	0.24672	4.4359508011781	4.4359508011781\\
62	0.25038	4.82595181448628	4.82595181448628\\
62	0.25404	5.22965956757983	5.22965956757983\\
62	0.2577	5.64707406045877	5.64707406045877\\
62	0.26136	6.07819529312309	6.07819529312309\\
62	0.26502	6.52302326557279	6.52302326557279\\
62	0.26868	6.98155797780785	6.98155797780785\\
62	0.27234	7.4537994298283	7.4537994298283\\
62	0.276	7.93974762163414	7.93974762163414\\
62.375	0.093	0.48265660894204	0.48265660894204\\
62.375	0.09666	0.292569039088082	0.292569039088082\\
62.375	0.10032	0.116188209019492	0.116188209019492\\
62.375	0.10398	-0.0464858812637079	-0.0464858812637079\\
62.375	0.10764	-0.195453231761534	-0.195453231761534\\
62.375	0.1113	-0.330713842473973	-0.330713842473973\\
62.375	0.11496	-0.45226771340104	-0.45226771340104\\
62.375	0.11862	-0.560114844542728	-0.560114844542728\\
62.375	0.12228	-0.654255235899035	-0.654255235899035\\
62.375	0.12594	-0.734688887469957	-0.734688887469957\\
62.375	0.1296	-0.801415799255507	-0.801415799255507\\
62.375	0.13326	-0.854435971255677	-0.854435971255677\\
62.375	0.13692	-0.89374940347046	-0.89374940347046\\
62.375	0.14058	-0.919356095899868	-0.919356095899868\\
62.375	0.14424	-0.931256048543897	-0.931256048543897\\
62.375	0.1479	-0.929449261402549	-0.929449261402549\\
62.375	0.15156	-0.913935734475823	-0.913935734475823\\
62.375	0.15522	-0.884715467763714	-0.884715467763714\\
62.375	0.15888	-0.841788461266223	-0.841788461266223\\
62.375	0.16254	-0.785154714983349	-0.785154714983349\\
62.375	0.1662	-0.714814228915111	-0.714814228915111\\
62.375	0.16986	-0.630767003061477	-0.630767003061477\\
62.375	0.17352	-0.533013037422466	-0.533013037422466\\
62.375	0.17718	-0.421552331998083	-0.421552331998083\\
62.375	0.18084	-0.296384886788321	-0.296384886788321\\
62.375	0.1845	-0.157510701793175	-0.157510701793175\\
62.375	0.18816	-0.00492977701264863	-0.00492977701264863\\
62.375	0.19182	0.161357887553262	0.161357887553262\\
62.375	0.19548	0.341352291904538	0.341352291904538\\
62.375	0.19914	0.535053436041206	0.535053436041206\\
62.375	0.2028	0.742461319963251	0.742461319963251\\
62.375	0.20646	0.963575943670683	0.963575943670683\\
62.375	0.21012	1.19839730716348	1.19839730716348\\
62.375	0.21378	1.44692541044166	1.44692541044166\\
62.375	0.21744	1.70916025350522	1.70916025350522\\
62.375	0.2211	1.98510183635416	1.98510183635416\\
62.375	0.22476	2.27475015898847	2.27475015898847\\
62.375	0.22842	2.57810522140818	2.57810522140818\\
62.375	0.23208	2.89516702361326	2.89516702361326\\
62.375	0.23574	3.22593556560372	3.22593556560372\\
62.375	0.2394	3.57041084737956	3.57041084737956\\
62.375	0.24306	3.92859286894078	3.92859286894078\\
62.375	0.24672	4.30048163028738	4.30048163028738\\
62.375	0.25038	4.68607713141935	4.68607713141935\\
62.375	0.25404	5.08537937233671	5.08537937233671\\
62.375	0.2577	5.49838835303945	5.49838835303945\\
62.375	0.26136	5.92510407352757	5.92510407352757\\
62.375	0.26502	6.36552653380106	6.36552653380106\\
62.375	0.26868	6.81965573385993	6.81965573385993\\
62.375	0.27234	7.28749167370419	7.28749167370419\\
62.375	0.276	7.76903435333382	7.76903435333382\\
62.75	0.093	0.534349157345865	0.534349157345865\\
62.75	0.09666	0.339856075315701	0.339856075315701\\
62.75	0.10032	0.159069733070919	0.159069733070919\\
62.75	0.10398	-0.00800986938848602	-0.00800986938848602\\
62.75	0.10764	-0.161382732062511	-0.161382732062511\\
62.75	0.1113	-0.301048854951155	-0.301048854951155\\
62.75	0.11496	-0.427008238054413	-0.427008238054413\\
62.75	0.11862	-0.539260881372307	-0.539260881372307\\
62.75	0.12228	-0.637806784904806	-0.637806784904806\\
62.75	0.12594	-0.722645948651934	-0.722645948651934\\
62.75	0.1296	-0.793778372613682	-0.793778372613682\\
62.75	0.13326	-0.851204056790051	-0.851204056790051\\
62.75	0.13692	-0.89492300118104	-0.89492300118104\\
62.75	0.14058	-0.924935205786653	-0.924935205786653\\
62.75	0.14424	-0.941240670606874	-0.941240670606874\\
62.75	0.1479	-0.943839395641724	-0.943839395641724\\
62.75	0.15156	-0.932731380891196	-0.932731380891196\\
62.75	0.15522	-0.907916626355279	-0.907916626355279\\
62.75	0.15888	-0.869395132033993	-0.869395132033993\\
62.75	0.16254	-0.817166897927326	-0.817166897927326\\
62.75	0.1662	-0.751231924035279	-0.751231924035279\\
62.75	0.16986	-0.67159021035785	-0.67159021035785\\
62.75	0.17352	-0.578241756895038	-0.578241756895038\\
62.75	0.17718	-0.47118656364686	-0.47118656364686\\
62.75	0.18084	-0.35042463061329	-0.35042463061329\\
62.75	0.1845	-0.215955957794343	-0.215955957794343\\
62.75	0.18816	-0.067780545190022	-0.067780545190022\\
62.75	0.19182	0.0941016071996899	0.0941016071996899\\
62.75	0.19548	0.269690499374768	0.269690499374768\\
62.75	0.19914	0.458986131335237	0.458986131335237\\
62.75	0.2028	0.661988503081076	0.661988503081076\\
62.75	0.20646	0.87869761461231	0.87869761461231\\
62.75	0.21012	1.1091134659289	1.1091134659289\\
62.75	0.21378	1.35323605703089	1.35323605703089\\
62.75	0.21744	1.61106538791825	1.61106538791825\\
62.75	0.2211	1.88260145859099	1.88260145859099\\
62.75	0.22476	2.16784426904912	2.16784426904912\\
62.75	0.22842	2.46679381929261	2.46679381929261\\
62.75	0.23208	2.77945010932149	2.77945010932149\\
62.75	0.23574	3.10581313913575	3.10581313913575\\
62.75	0.2394	3.44588290873539	3.44588290873539\\
62.75	0.24306	3.79965941812041	3.79965941812041\\
62.75	0.24672	4.16714266729081	4.16714266729081\\
62.75	0.25038	4.54833265624659	4.54833265624659\\
62.75	0.25404	4.94322938498774	4.94322938498774\\
62.75	0.2577	5.35183285351428	5.35183285351428\\
62.75	0.26136	5.77414306182619	5.77414306182619\\
62.75	0.26502	6.21016000992349	6.21016000992349\\
62.75	0.26868	6.65988369780616	6.65988369780616\\
62.75	0.27234	7.12331412547422	7.12331412547422\\
62.75	0.276	7.60045129292765	7.60045129292765\\
63.125	0.093	0.588171913643844	0.588171913643844\\
63.125	0.09666	0.389273319437489	0.389273319437489\\
63.125	0.10032	0.204081465016501	0.204081465016501\\
63.125	0.10398	0.0325963503808975	0.0325963503808975\\
63.125	0.10764	-0.125182024469326	-0.125182024469326\\
63.125	0.1113	-0.269253659534169	-0.269253659534169\\
63.125	0.11496	-0.399618554813633	-0.399618554813633\\
63.125	0.11862	-0.516276710307718	-0.516276710307718\\
63.125	0.12228	-0.619228126016423	-0.619228126016423\\
63.125	0.12594	-0.708472801939749	-0.708472801939749\\
63.125	0.1296	-0.784010738077695	-0.784010738077695\\
63.125	0.13326	-0.845841934430263	-0.845841934430263\\
63.125	0.13692	-0.89396639099745	-0.89396639099745\\
63.125	0.14058	-0.928384107779262	-0.928384107779262\\
63.125	0.14424	-0.949095084775688	-0.949095084775688\\
63.125	0.1479	-0.956099321986738	-0.956099321986738\\
63.125	0.15156	-0.949396819412408	-0.949396819412408\\
63.125	0.15522	-0.928987577052697	-0.928987577052697\\
63.125	0.15888	-0.89487159490761	-0.89487159490761\\
63.125	0.16254	-0.847048872977133	-0.847048872977133\\
63.125	0.1662	-0.785519411261292	-0.785519411261292\\
63.125	0.16986	-0.710283209760062	-0.710283209760062\\
63.125	0.17352	-0.621340268473448	-0.621340268473448\\
63.125	0.17718	-0.518690587401469	-0.518690587401469\\
63.125	0.18084	-0.402334166544104	-0.402334166544104\\
63.125	0.1845	-0.272271005901356	-0.272271005901356\\
63.125	0.18816	-0.128501105473234	-0.128501105473234\\
63.125	0.19182	0.0289755347402796	0.0289755347402796\\
63.125	0.19548	0.200158914739166	0.200158914739166\\
63.125	0.19914	0.385049034523423	0.385049034523423\\
63.125	0.2028	0.58364589409307	0.58364589409307\\
63.125	0.20646	0.795949493448092	0.795949493448092\\
63.125	0.21012	1.0219598325885	1.0219598325885\\
63.125	0.21378	1.26167691151428	1.26167691151428\\
63.125	0.21744	1.51510073022543	1.51510073022543\\
63.125	0.2211	1.78223128872198	1.78223128872198\\
63.125	0.22476	2.0630685870039	2.0630685870039\\
63.125	0.22842	2.35761262507121	2.35761262507121\\
63.125	0.23208	2.66586340292389	2.66586340292389\\
63.125	0.23574	2.98782092056195	2.98782092056195\\
63.125	0.2394	3.32348517798538	3.32348517798538\\
63.125	0.24306	3.6728561751942	3.6728561751942\\
63.125	0.24672	4.0359339121884	4.0359339121884\\
63.125	0.25038	4.41271838896798	4.41271838896798\\
63.125	0.25404	4.80320960553293	4.80320960553293\\
63.125	0.2577	5.20740756188328	5.20740756188328\\
63.125	0.26136	5.62531225801899	5.62531225801899\\
63.125	0.26502	6.05692369394009	6.05692369394009\\
63.125	0.26868	6.50224186964656	6.50224186964656\\
63.125	0.27234	6.96126678513841	6.96126678513841\\
63.125	0.276	7.43399844041565	7.43399844041565\\
63.5	0.093	0.644124877835978	0.644124877835978\\
63.5	0.09666	0.440820771453417	0.440820771453417\\
63.5	0.10032	0.251223404856231	0.251223404856231\\
63.5	0.10398	0.0753327780444355	0.0753327780444355\\
63.5	0.10764	-0.0868511089819934	-0.0868511089819934\\
63.5	0.1113	-0.235328256223035	-0.235328256223035\\
63.5	0.11496	-0.370098663678697	-0.370098663678697\\
63.5	0.11862	-0.491162331348988	-0.491162331348988\\
63.5	0.12228	-0.598519259233892	-0.598519259233892\\
63.5	0.12594	-0.692169447333416	-0.692169447333416\\
63.5	0.1296	-0.772112895647561	-0.772112895647561\\
63.5	0.13326	-0.838349604176328	-0.838349604176328\\
63.5	0.13692	-0.890879572919721	-0.890879572919721\\
63.5	0.14058	-0.929702801877731	-0.929702801877731\\
63.5	0.14424	-0.954819291050349	-0.954819291050349\\
63.5	0.1479	-0.966229040437604	-0.966229040437604\\
63.5	0.15156	-0.963932050039473	-0.963932050039473\\
63.5	0.15522	-0.94792831985596	-0.94792831985596\\
63.5	0.15888	-0.918217849887071	-0.918217849887071\\
63.5	0.16254	-0.874800640132801	-0.874800640132801\\
63.5	0.1662	-0.817676690593151	-0.817676690593151\\
63.5	0.16986	-0.746846001268127	-0.746846001268127\\
63.5	0.17352	-0.662308572157711	-0.662308572157711\\
63.5	0.17718	-0.564064403261931	-0.564064403261931\\
63.5	0.18084	-0.452113494580765	-0.452113494580765\\
63.5	0.1845	-0.326455846114214	-0.326455846114214\\
63.5	0.18816	-0.187091457862291	-0.187091457862291\\
63.5	0.19182	-0.0340203298249833	-0.0340203298249833\\
63.5	0.19548	0.132757537997705	0.132757537997705\\
63.5	0.19914	0.313242145605763	0.313242145605763\\
63.5	0.2028	0.507433492999212	0.507433492999212\\
63.5	0.20646	0.715331580178042	0.715331580178042\\
63.5	0.21012	0.936936407142237	0.936936407142237\\
63.5	0.21378	1.17224797389182	1.17224797389182\\
63.5	0.21744	1.42126628042677	1.42126628042677\\
63.5	0.2211	1.68399132674713	1.68399132674713\\
63.5	0.22476	1.96042311285285	1.96042311285285\\
63.5	0.22842	2.25056163874394	2.25056163874394\\
63.5	0.23208	2.55440690442042	2.55440690442042\\
63.5	0.23574	2.8719589098823	2.8719589098823\\
63.5	0.2394	3.20321765512952	3.20321765512952\\
63.5	0.24306	3.54818314016214	3.54818314016214\\
63.5	0.24672	3.90685536498015	3.90685536498015\\
63.5	0.25038	4.27923432958352	4.27923432958352\\
63.5	0.25404	4.66532003397228	4.66532003397228\\
63.5	0.2577	5.06511247814642	5.06511247814642\\
63.5	0.26136	5.47861166210593	5.47861166210593\\
63.5	0.26502	5.90581758585083	5.90581758585083\\
63.5	0.26868	6.3467302493811	6.3467302493811\\
63.5	0.27234	6.80134965269676	6.80134965269676\\
63.5	0.276	7.2696757957978	7.2696757957978\\
63.875	0.093	0.70220804992227	0.70220804992227\\
63.875	0.09666	0.49449843136351	0.49449843136351\\
63.875	0.10032	0.300495552590132	0.300495552590132\\
63.875	0.10398	0.120199413602124	0.120199413602124\\
63.875	0.10764	-0.046389985600503	-0.046389985600503\\
63.875	0.1113	-0.199272645017743	-0.199272645017743\\
63.875	0.11496	-0.338448564649604	-0.338448564649604\\
63.875	0.11862	-0.463917744496094	-0.463917744496094\\
63.875	0.12228	-0.575680184557195	-0.575680184557195\\
63.875	0.12594	-0.673735884832919	-0.673735884832919\\
63.875	0.1296	-0.758084845323269	-0.758084845323269\\
63.875	0.13326	-0.828727066028234	-0.828727066028234\\
63.875	0.13692	-0.885662546947819	-0.885662546947819\\
63.875	0.14058	-0.928891288082035	-0.928891288082035\\
63.875	0.14424	-0.958413289430858	-0.958413289430858\\
63.875	0.1479	-0.974228550994304	-0.974228550994304\\
63.875	0.15156	-0.976337072772372	-0.976337072772372\\
63.875	0.15522	-0.964738854765057	-0.964738854765057\\
63.875	0.15888	-0.939433896972375	-0.939433896972375\\
63.875	0.16254	-0.900422199394303	-0.900422199394303\\
63.875	0.1662	-0.847703762030852	-0.847703762030852\\
63.875	0.16986	-0.781278584882026	-0.781278584882026\\
63.875	0.17352	-0.701146667947816	-0.701146667947816\\
63.875	0.17718	-0.607308011228234	-0.607308011228234\\
63.875	0.18084	-0.499762614723267	-0.499762614723267\\
63.875	0.1845	-0.378510478432915	-0.378510478432915\\
63.875	0.18816	-0.24355160235719	-0.24355160235719\\
63.875	0.19182	-0.0948859864960809	-0.0948859864960809\\
63.875	0.19548	0.0674863691504015	0.0674863691504015\\
63.875	0.19914	0.243565464582268	0.243565464582268\\
63.875	0.2028	0.433351299799511	0.433351299799511\\
63.875	0.20646	0.636843874802135	0.636843874802135\\
63.875	0.21012	0.85404318959014	0.85404318959014\\
63.875	0.21378	1.08494924416352	1.08494924416352\\
63.875	0.21744	1.32956203852228	1.32956203852228\\
63.875	0.2211	1.58788157266643	1.58788157266643\\
63.875	0.22476	1.85990784659594	1.85990784659594\\
63.875	0.22842	2.14564086031084	2.14564086031084\\
63.875	0.23208	2.44508061381113	2.44508061381113\\
63.875	0.23574	2.75822710709679	2.75822710709679\\
63.875	0.2394	3.08508034016783	3.08508034016783\\
63.875	0.24306	3.42564031302425	3.42564031302425\\
63.875	0.24672	3.77990702566605	3.77990702566605\\
63.875	0.25038	4.14788047809323	4.14788047809323\\
63.875	0.25404	4.52956067030578	4.52956067030578\\
63.875	0.2577	4.92494760230372	4.92494760230372\\
63.875	0.26136	5.33404127408704	5.33404127408704\\
63.875	0.26502	5.75684168565574	5.75684168565574\\
63.875	0.26868	6.19334883700981	6.19334883700981\\
63.875	0.27234	6.64356272814926	6.64356272814926\\
63.875	0.276	7.1074833590741	7.1074833590741\\
64.25	0.093	0.762421429902716	0.762421429902716\\
64.25	0.09666	0.550306299167758	0.550306299167758\\
64.25	0.10032	0.351897908218175	0.351897908218175\\
64.25	0.10398	0.167196257053968	0.167196257053968\\
64.25	0.10764	-0.00379865432485804	-0.00379865432485804\\
64.25	0.1113	-0.161086825918296	-0.161086825918296\\
64.25	0.11496	-0.304668257726356	-0.304668257726356\\
64.25	0.11862	-0.434542949749044	-0.434542949749044\\
64.25	0.12228	-0.550710901986352	-0.550710901986352\\
64.25	0.12594	-0.653172114438274	-0.653172114438274\\
64.25	0.1296	-0.741926587104816	-0.741926587104816\\
64.25	0.13326	-0.816974319985986	-0.816974319985986\\
64.25	0.13692	-0.878315313081776	-0.878315313081776\\
64.25	0.14058	-0.925949566392184	-0.925949566392184\\
64.25	0.14424	-0.959877079917213	-0.959877079917213\\
64.25	0.1479	-0.980097853656858	-0.980097853656858\\
64.25	0.15156	-0.986611887611124	-0.986611887611124\\
64.25	0.15522	-0.979419181780015	-0.979419181780015\\
64.25	0.15888	-0.958519736163524	-0.958519736163524\\
64.25	0.16254	-0.92391355076165	-0.92391355076165\\
64.25	0.1662	-0.875600625574412	-0.875600625574412\\
64.25	0.16986	-0.813580960601778	-0.813580960601778\\
64.25	0.17352	-0.737854555843766	-0.737854555843766\\
64.25	0.17718	-0.648421411300383	-0.648421411300383\\
64.25	0.18084	-0.545281526971614	-0.545281526971614\\
64.25	0.1845	-0.428434902857461	-0.428434902857461\\
64.25	0.18816	-0.297881538957942	-0.297881538957942\\
64.25	0.19182	-0.153621435273031	-0.153621435273031\\
64.25	0.19548	0.00434540819725271	0.00434540819725271\\
64.25	0.19914	0.176018991452914	0.176018991452914\\
64.25	0.2028	0.361399314493966	0.361399314493966\\
64.25	0.20646	0.560486377320384	0.560486377320384\\
64.25	0.21012	0.77328017993219	0.77328017993219\\
64.25	0.21378	0.999780722329373	0.999780722329373\\
64.25	0.21744	1.23998800451193	1.23998800451193\\
64.25	0.2211	1.49390202647988	1.49390202647988\\
64.25	0.22476	1.7615227882332	1.7615227882332\\
64.25	0.22842	2.0428502897719	2.0428502897719\\
64.25	0.23208	2.33788453109598	2.33788453109598\\
64.25	0.23574	2.64662551220545	2.64662551220545\\
64.25	0.2394	2.96907323310028	2.96907323310028\\
64.25	0.24306	3.3052276937805	3.3052276937805\\
64.25	0.24672	3.65508889424611	3.65508889424611\\
64.25	0.25038	4.01865683449708	4.01865683449708\\
64.25	0.25404	4.39593151453344	4.39593151453344\\
64.25	0.2577	4.78691293435518	4.78691293435518\\
64.25	0.26136	5.1916010939623	5.1916010939623\\
64.25	0.26502	5.60999599335479	5.60999599335479\\
64.25	0.26868	6.04209763253267	6.04209763253267\\
64.25	0.27234	6.48790601149592	6.48790601149592\\
64.25	0.276	6.94742113024455	6.94742113024455\\
64.625	0.093	0.824765017777317	0.824765017777317\\
64.625	0.09666	0.608244374866153	0.608244374866153\\
64.625	0.10032	0.405430471740371	0.405430471740371\\
64.625	0.10398	0.216323308399973	0.216323308399973\\
64.625	0.10764	0.0409228848449414	0.0409228848449414\\
64.625	0.1113	-0.120770798924696	-0.120770798924696\\
64.625	0.11496	-0.268757742908961	-0.268757742908961\\
64.625	0.11862	-0.403037947107848	-0.403037947107848\\
64.625	0.12228	-0.523611411521347	-0.523611411521347\\
64.625	0.12594	-0.630478136149474	-0.630478136149474\\
64.625	0.1296	-0.723638120992222	-0.723638120992222\\
64.625	0.13326	-0.803091366049584	-0.803091366049584\\
64.625	0.13692	-0.868837871321579	-0.868837871321579\\
64.625	0.14058	-0.920877636808179	-0.920877636808179\\
64.625	0.14424	-0.959210662509406	-0.959210662509406\\
64.625	0.1479	-0.983836948425257	-0.983836948425257\\
64.625	0.15156	-0.994756494555729	-0.994756494555729\\
64.625	0.15522	-0.991969300900818	-0.991969300900818\\
64.625	0.15888	-0.975475367460525	-0.975475367460525\\
64.625	0.16254	-0.94527469423485	-0.94527469423485\\
64.625	0.1662	-0.901367281223804	-0.901367281223804\\
64.625	0.16986	-0.843753128427375	-0.843753128427375\\
64.625	0.17352	-0.772432235845562	-0.772432235845562\\
64.625	0.17718	-0.687404603478377	-0.687404603478377\\
64.625	0.18084	-0.588670231325814	-0.588670231325814\\
64.625	0.1845	-0.476229119387867	-0.476229119387867\\
64.625	0.18816	-0.350081267664539	-0.350081267664539\\
64.625	0.19182	-0.210226676155827	-0.210226676155827\\
64.625	0.19548	-0.0566653448617416	-0.0566653448617416\\
64.625	0.19914	0.110602726217721	0.110602726217721\\
64.625	0.2028	0.291577537082567	0.291577537082567\\
64.625	0.20646	0.486259087732801	0.486259087732801\\
64.625	0.21012	0.694647378168394	0.694647378168394\\
64.625	0.21378	0.916742408389378	0.916742408389378\\
64.625	0.21744	1.15254417839574	1.15254417839574\\
64.625	0.2211	1.40205268818749	1.40205268818749\\
64.625	0.22476	1.66526793776461	1.66526793776461\\
64.625	0.22842	1.94218992712711	1.94218992712711\\
64.625	0.23208	2.23281865627499	2.23281865627499\\
64.625	0.23574	2.53715412520826	2.53715412520826\\
64.625	0.2394	2.85519633392689	2.85519633392689\\
64.625	0.24306	3.18694528243091	3.18694528243091\\
64.625	0.24672	3.53240097072032	3.53240097072032\\
64.625	0.25038	3.89156339879509	3.89156339879509\\
64.625	0.25404	4.26443256665524	4.26443256665524\\
64.625	0.2577	4.65100847430079	4.65100847430079\\
64.625	0.26136	5.05129112173171	5.05129112173171\\
64.625	0.26502	5.465280508948	5.465280508948\\
64.625	0.26868	5.89297663594967	5.89297663594967\\
64.625	0.27234	6.33437950273673	6.33437950273673\\
64.625	0.276	6.78948910930916	6.78948910930916\\
65	0.093	0.889238813546086	0.889238813546086\\
65	0.09666	0.668312658458724	0.668312658458724\\
65	0.10032	0.461093243156744	0.461093243156744\\
65	0.10398	0.26758056764014	0.26758056764014\\
65	0.10764	0.0877746319089168	0.0877746319089168\\
65	0.1113	-0.0783245640369259	-0.0783245640369259\\
65	0.11496	-0.23071702019739	-0.23071702019739\\
65	0.11862	-0.369402736572475	-0.369402736572475\\
65	0.12228	-0.49438171316218	-0.49438171316218\\
65	0.12594	-0.605653949966506	-0.605653949966506\\
65	0.1296	-0.703219446985452	-0.703219446985452\\
65	0.13326	-0.78707820421902	-0.78707820421902\\
65	0.13692	-0.857230221667207	-0.857230221667207\\
65	0.14058	-0.913675499330012	-0.913675499330012\\
65	0.14424	-0.956414037207438	-0.956414037207438\\
65	0.1479	-0.985445835299487	-0.985445835299487\\
65	0.15156	-1.00077089360616	-1.00077089360616\\
65	0.15522	-1.00238921212745	-1.00238921212745\\
65	0.15888	-0.990300790863351	-0.990300790863351\\
65	0.16254	-0.964505629813882	-0.964505629813882\\
65	0.1662	-0.925003728979034	-0.925003728979034\\
65	0.16986	-0.871795088358803	-0.871795088358803\\
65	0.17352	-0.804879707953189	-0.804879707953189\\
65	0.17718	-0.724257587762203	-0.724257587762203\\
65	0.18084	-0.629928727785838	-0.629928727785838\\
65	0.1845	-0.521893128024082	-0.521893128024082\\
65	0.18816	-0.400150788476967	-0.400150788476967\\
65	0.19182	-0.264701709144454	-0.264701709144454\\
65	0.19548	-0.115545890026574	-0.115545890026574\\
65	0.19914	0.0473166688766966	0.0473166688766966\\
65	0.2028	0.223885967565344	0.223885967565344\\
65	0.20646	0.414162006039366	0.414162006039366\\
65	0.21012	0.618144784298774	0.618144784298774\\
65	0.21378	0.835834302343553	0.835834302343553\\
65	0.21744	1.06723056017372	1.06723056017372\\
65	0.2211	1.31233355778926	1.31233355778926\\
65	0.22476	1.57114329519019	1.57114329519019\\
65	0.22842	1.84365977237648	1.84365977237648\\
65	0.23208	2.12988298934817	2.12988298934817\\
65	0.23574	2.42981294610524	2.42981294610524\\
65	0.2394	2.74344964264766	2.74344964264766\\
65	0.24306	3.07079307897549	3.07079307897549\\
65	0.24672	3.41184325508869	3.41184325508869\\
65	0.25038	3.76660017098727	3.76660017098727\\
65	0.25404	4.13506382667122	4.13506382667122\\
65	0.2577	4.51723422214057	4.51723422214057\\
65	0.26136	4.91311135739529	4.91311135739529\\
65	0.26502	5.32269523243538	5.32269523243538\\
65	0.26868	5.74598584726085	5.74598584726085\\
65	0.27234	6.18298320187171	6.18298320187171\\
65	0.276	6.63368729626794	6.63368729626794\\
65.375	0.093	0.955842817208989	0.955842817208989\\
65.375	0.09666	0.730511149945428	0.730511149945428\\
65.375	0.10032	0.518886222467249	0.518886222467249\\
65.375	0.10398	0.32096803477444	0.32096803477444\\
65.375	0.10764	0.136756586867025	0.136756586867025\\
65.375	0.1113	-0.033748121255023	-0.033748121255023\\
65.375	0.11496	-0.190546089591678	-0.190546089591678\\
65.375	0.11862	-0.333637318142969	-0.333637318142969\\
65.375	0.12228	-0.463021806908873	-0.463021806908873\\
65.375	0.12594	-0.578699555889397	-0.578699555889397\\
65.375	0.1296	-0.680670565084542	-0.680670565084542\\
65.375	0.13326	-0.768934834494308	-0.768934834494308\\
65.375	0.13692	-0.843492364118694	-0.843492364118694\\
65.375	0.14058	-0.904343153957697	-0.904343153957697\\
65.375	0.14424	-0.951487204011329	-0.951487204011329\\
65.375	0.1479	-0.984924514279577	-0.984924514279577\\
65.375	0.15156	-1.00465508476245	-1.00465508476245\\
65.375	0.15522	-1.01067891545993	-1.01067891545993\\
65.375	0.15888	-1.00299600637204	-1.00299600637204\\
65.375	0.16254	-0.981606357498766	-0.981606357498766\\
65.375	0.1662	-0.946509968840116	-0.946509968840116\\
65.375	0.16986	-0.897706840396085	-0.897706840396085\\
65.375	0.17352	-0.835196972166676	-0.835196972166676\\
65.375	0.17718	-0.758980364151896	-0.758980364151896\\
65.375	0.18084	-0.669057016351722	-0.669057016351722\\
65.375	0.1845	-0.565426928766179	-0.565426928766179\\
65.375	0.18816	-0.448090101395248	-0.448090101395248\\
65.375	0.19182	-0.317046534238941	-0.317046534238941\\
65.375	0.19548	-0.172296227297259	-0.172296227297259\\
65.375	0.19914	-0.0138391805701872	-0.0138391805701872\\
65.375	0.2028	0.158324605942255	0.158324605942255\\
65.375	0.20646	0.344195132240092	0.344195132240092\\
65.375	0.21012	0.543772398323288	0.543772398323288\\
65.375	0.21378	0.757056404191868	0.757056404191868\\
65.375	0.21744	0.984047149845832	0.984047149845832\\
65.375	0.2211	1.22474463528518	1.22474463528518\\
65.375	0.22476	1.4791488605099	1.4791488605099\\
65.375	0.22842	1.74725982552	1.74725982552\\
65.375	0.23208	2.02907753031548	2.02907753031548\\
65.375	0.23574	2.32460197489635	2.32460197489635\\
65.375	0.2394	2.63383315926258	2.63383315926258\\
65.375	0.24306	2.9567710834142	2.9567710834142\\
65.375	0.24672	3.29341574735121	3.29341574735121\\
65.375	0.25038	3.64376715107358	3.64376715107358\\
65.375	0.25404	4.00782529458134	4.00782529458134\\
65.375	0.2577	4.38559017787449	4.38559017787449\\
65.375	0.26136	4.777061800953	4.777061800953\\
65.375	0.26502	5.1822401638169	5.1822401638169\\
65.375	0.26868	5.60112526646617	5.60112526646617\\
65.375	0.27234	6.03371710890082	6.03371710890082\\
65.375	0.276	6.48001569112086	6.48001569112086\\
65.75	0.093	1.02457702876606	1.02457702876606\\
65.75	0.09666	0.794839849326294	0.794839849326294\\
65.75	0.10032	0.578809409671917	0.578809409671917\\
65.75	0.10398	0.376485709802916	0.376485709802916\\
65.75	0.10764	0.187868749719295	0.187868749719295\\
65.75	0.1113	0.0129585294210557	0.0129585294210557\\
65.75	0.11496	-0.148244951091805	-0.148244951091805\\
65.75	0.11862	-0.295741691819295	-0.295741691819295\\
65.75	0.12228	-0.429531692761397	-0.429531692761397\\
65.75	0.12594	-0.54961495391812	-0.54961495391812\\
65.75	0.1296	-0.65599147528947	-0.65599147528947\\
65.75	0.13326	-0.748661256875435	-0.748661256875435\\
65.75	0.13692	-0.827624298676019	-0.827624298676019\\
65.75	0.14058	-0.892880600691221	-0.892880600691221\\
65.75	0.14424	-0.944430162921051	-0.944430162921051\\
65.75	0.1479	-0.982272985365498	-0.982272985365498\\
65.75	0.15156	-1.00640906802457	-1.00640906802457\\
65.75	0.15522	-1.01683841089826	-1.01683841089826\\
65.75	0.15888	-1.01356101398656	-1.01356101398656\\
65.75	0.16254	-0.996576877289488	-0.996576877289488\\
65.75	0.1662	-0.965886000807037	-0.965886000807037\\
65.75	0.16986	-0.921488384539211	-0.921488384539211\\
65.75	0.17352	-0.863384028485994	-0.863384028485994\\
65.75	0.17718	-0.791572932647412	-0.791572932647412\\
65.75	0.18084	-0.706055097023444	-0.706055097023444\\
65.75	0.1845	-0.6068305216141	-0.6068305216141\\
65.75	0.18816	-0.493899206419368	-0.493899206419368\\
65.75	0.19182	-0.367261151439259	-0.367261151439259\\
65.75	0.19548	-0.226916356673769	-0.226916356673769\\
65.75	0.19914	-0.0728648221229093	-0.0728648221229093\\
65.75	0.2028	0.0948934522133413	0.0948934522133413\\
65.75	0.20646	0.276358466334965	0.276358466334965\\
65.75	0.21012	0.47153022024197	0.47153022024197\\
65.75	0.21378	0.680408713934352	0.680408713934352\\
65.75	0.21744	0.90299394741211	0.90299394741211\\
65.75	0.2211	1.13928592067526	1.13928592067526\\
65.75	0.22476	1.38928463372379	1.38928463372379\\
65.75	0.22842	1.65299008655768	1.65299008655768\\
65.75	0.23208	1.93040227917697	1.93040227917697\\
65.75	0.23574	2.22152121158163	2.22152121158163\\
65.75	0.2394	2.52634688377167	2.52634688377167\\
65.75	0.24306	2.84487929574709	2.84487929574709\\
65.75	0.24672	3.17711844750789	3.17711844750789\\
65.75	0.25038	3.52306433905407	3.52306433905407\\
65.75	0.25404	3.88271697038563	3.88271697038563\\
65.75	0.2577	4.25607634150257	4.25607634150257\\
65.75	0.26136	4.64314245240489	4.64314245240489\\
65.75	0.26502	5.04391530309258	5.04391530309258\\
65.75	0.26868	5.45839489356566	5.45839489356566\\
65.75	0.27234	5.88658122382412	5.88658122382412\\
65.75	0.276	6.32847429386796	6.32847429386796\\
66.125	0.093	1.09544144821729	1.09544144821729\\
66.125	0.09666	0.861298756601322	0.861298756601322\\
66.125	0.10032	0.640862804770745	0.640862804770745\\
66.125	0.10398	0.434133592725546	0.434133592725546\\
66.125	0.10764	0.24111112046572	0.24111112046572\\
66.125	0.1113	0.0617953879912747	0.0617953879912747\\
66.125	0.11496	-0.103813604697785	-0.103813604697785\\
66.125	0.11862	-0.255715857601466	-0.255715857601466\\
66.125	0.12228	-0.393911370719773	-0.393911370719773\\
66.125	0.12594	-0.518400144052695	-0.518400144052695\\
66.125	0.1296	-0.629182177600237	-0.629182177600237\\
66.125	0.13326	-0.726257471362407	-0.726257471362407\\
66.125	0.13692	-0.80962602533919	-0.80962602533919\\
66.125	0.14058	-0.879287839530598	-0.879287839530598\\
66.125	0.14424	-0.935242913936619	-0.935242913936619\\
66.125	0.1479	-0.977491248557271	-0.977491248557271\\
66.125	0.15156	-1.00603284339254	-1.00603284339254\\
66.125	0.15522	-1.02086769844242	-1.02086769844242\\
66.125	0.15888	-1.02199581370694	-1.02199581370694\\
66.125	0.16254	-1.00941718918606	-1.00941718918606\\
66.125	0.1662	-0.983131824879818	-0.983131824879818\\
66.125	0.16986	-0.943139720788183	-0.943139720788183\\
66.125	0.17352	-0.889440876911165	-0.889440876911165\\
66.125	0.17718	-0.822035293248781	-0.822035293248781\\
66.125	0.18084	-0.740922969801012	-0.740922969801012\\
66.125	0.1845	-0.646103906567866	-0.646103906567866\\
66.125	0.18816	-0.53757810354934	-0.53757810354934\\
66.125	0.19182	-0.415345560745429	-0.415345560745429\\
66.125	0.19548	-0.279406278156138	-0.279406278156138\\
66.125	0.19914	-0.129760255781477	-0.129760255781477\\
66.125	0.2028	0.033592506378568	0.033592506378568\\
66.125	0.20646	0.210652008323994	0.210652008323994\\
66.125	0.21012	0.4014182500548	0.4014182500548\\
66.125	0.21378	0.605891231570983	0.605891231570983\\
66.125	0.21744	0.824070952872543	0.824070952872543\\
66.125	0.2211	1.05595741395949	1.05595741395949\\
66.125	0.22476	1.30155061483181	1.30155061483181\\
66.125	0.22842	1.56085055548951	1.56085055548951\\
66.125	0.23208	1.8338572359326	1.8338572359326\\
66.125	0.23574	2.12057065616107	2.12057065616107\\
66.125	0.2394	2.4209908161749	2.4209908161749\\
66.125	0.24306	2.73511771597413	2.73511771597413\\
66.125	0.24672	3.06295135555873	3.06295135555873\\
66.125	0.25038	3.4044917349287	3.4044917349287\\
66.125	0.25404	3.75973885408406	3.75973885408406\\
66.125	0.2577	4.1286927130248	4.1286927130248\\
66.125	0.26136	4.51135331175092	4.51135331175092\\
66.125	0.26502	4.90772065026242	4.90772065026242\\
66.125	0.26868	5.31779472855929	5.31779472855929\\
66.125	0.27234	5.74157554664154	5.74157554664154\\
66.125	0.276	6.17906310450919	6.17906310450919\\
66.5	0.093	1.16843607556266	1.16843607556266\\
66.5	0.09666	0.929887871770504	0.929887871770504\\
66.5	0.10032	0.705046407763722	0.705046407763722\\
66.5	0.10398	0.493911683542317	0.493911683542317\\
66.5	0.10764	0.296483699106299	0.296483699106299\\
66.5	0.1113	0.112762454455662	0.112762454455662\\
66.5	0.11496	-0.0572520504096099	-0.0572520504096099\\
66.5	0.11862	-0.213559815489489	-0.213559815489489\\
66.5	0.12228	-0.356160840783988	-0.356160840783988\\
66.5	0.12594	-0.485055126293116	-0.485055126293116\\
66.5	0.1296	-0.600242672016863	-0.600242672016863\\
66.5	0.13326	-0.701723477955225	-0.701723477955225\\
66.5	0.13692	-0.789497544108206	-0.789497544108206\\
66.5	0.14058	-0.86356487047582	-0.86356487047582\\
66.5	0.14424	-0.92392545705804	-0.92392545705804\\
66.5	0.1479	-0.97057930385489	-0.97057930385489\\
66.5	0.15156	-1.00352641086636	-1.00352641086636\\
66.5	0.15522	-1.02276677809244	-1.02276677809244\\
66.5	0.15888	-1.02830040553315	-1.02830040553315\\
66.5	0.16254	-1.02012729318848	-1.02012729318848\\
66.5	0.1662	-0.99824744105843	-0.99824744105843\\
66.5	0.16986	-0.962660849142994	-0.962660849142994\\
66.5	0.17352	-0.913367517442188	-0.913367517442188\\
66.5	0.17718	-0.850367445956003	-0.850367445956003\\
66.5	0.18084	-0.773660634684433	-0.773660634684433\\
66.5	0.1845	-0.683247083627478	-0.683247083627478\\
66.5	0.18816	-0.579126792785157	-0.579126792785157\\
66.5	0.19182	-0.461299762157445	-0.461299762157445\\
66.5	0.19548	-0.329765991744353	-0.329765991744353\\
66.5	0.19914	-0.18452548154589	-0.18452548154589\\
66.5	0.2028	-0.0255782315620436	-0.0255782315620436\\
66.5	0.20646	0.14707575820719	0.14707575820719\\
66.5	0.21012	0.333436487761784	0.333436487761784\\
66.5	0.21378	0.533503957101768	0.533503957101768\\
66.5	0.21744	0.74727816622713	0.74727816622713\\
66.5	0.2211	0.974759115137878	0.974759115137878\\
66.5	0.22476	1.215946803834	1.215946803834\\
66.5	0.22842	1.4708412323155	1.4708412323155\\
66.5	0.23208	1.73944240058239	1.73944240058239\\
66.5	0.23574	2.02175030863465	2.02175030863465\\
66.5	0.2394	2.31776495647229	2.31776495647229\\
66.5	0.24306	2.62748634409531	2.62748634409531\\
66.5	0.24672	2.95091447150371	2.95091447150371\\
66.5	0.25038	3.28804933869749	3.28804933869749\\
66.5	0.25404	3.63889094567665	3.63889094567665\\
66.5	0.2577	4.00343929244119	4.00343929244119\\
66.5	0.26136	4.38169437899111	4.38169437899111\\
66.5	0.26502	4.77365620532641	4.77365620532641\\
66.5	0.26868	5.17932477144709	5.17932477144709\\
66.5	0.27234	5.59870007735315	5.59870007735315\\
66.5	0.276	6.03178212304458	6.03178212304458\\
66.875	0.093	1.24356091080221	1.24356091080221\\
66.875	0.09666	1.00060719483385	1.00060719483385\\
66.875	0.10032	0.771360218650867	0.771360218650867\\
66.875	0.10398	0.555819982253263	0.555819982253263\\
66.875	0.10764	0.353986485641047	0.353986485641047\\
66.875	0.1113	0.165859728814205	0.165859728814205\\
66.875	0.11496	-0.00856028822725197	-0.00856028822725197\\
66.875	0.11862	-0.169273565483337	-0.169273565483337\\
66.875	0.12228	-0.316280102954042	-0.316280102954042\\
66.875	0.12594	-0.449579900639367	-0.449579900639367\\
66.875	0.1296	-0.569172958539314	-0.569172958539314\\
66.875	0.13326	-0.675059276653874	-0.675059276653874\\
66.875	0.13692	-0.767238854983061	-0.767238854983061\\
66.875	0.14058	-0.845711693526866	-0.845711693526866\\
66.875	0.14424	-0.910477792285292	-0.910477792285292\\
66.875	0.1479	-0.961537151258334	-0.961537151258334\\
66.875	0.15156	-0.998889770446004	-0.998889770446004\\
66.875	0.15522	-1.02253564984829	-1.02253564984829\\
66.875	0.15888	-1.0324747894652	-1.0324747894652\\
66.875	0.16254	-1.02870718929673	-1.02870718929673\\
66.875	0.1662	-1.01123284934288	-1.01123284934288\\
66.875	0.16986	-0.980051769603643	-0.980051769603643\\
66.875	0.17352	-0.935163950079035	-0.935163950079035\\
66.875	0.17718	-0.876569390769049	-0.876569390769049\\
66.875	0.18084	-0.804268091673677	-0.804268091673677\\
66.875	0.1845	-0.718260052792928	-0.718260052792928\\
66.875	0.18816	-0.618545274126806	-0.618545274126806\\
66.875	0.19182	-0.505123755675292	-0.505123755675292\\
66.875	0.19548	-0.377995497438405	-0.377995497438405\\
66.875	0.19914	-0.237160499416142	-0.237160499416142\\
66.875	0.2028	-0.0826187616084937	-0.0826187616084937\\
66.875	0.20646	0.0856297159845347	0.0856297159845347\\
66.875	0.21012	0.267584933362944	0.267584933362944\\
66.875	0.21378	0.463246890526722	0.463246890526722\\
66.875	0.21744	0.672615587475885	0.672615587475885\\
66.875	0.2211	0.895691024210436	0.895691024210436\\
66.875	0.22476	1.13247320073036	1.13247320073036\\
66.875	0.22842	1.38296211703566	1.38296211703566\\
66.875	0.23208	1.64715777312635	1.64715777312635\\
66.875	0.23574	1.92506016900241	1.92506016900241\\
66.875	0.2394	2.21666930466384	2.21666930466384\\
66.875	0.24306	2.52198518011067	2.52198518011067\\
66.875	0.24672	2.84100779534288	2.84100779534288\\
66.875	0.25038	3.17373715036045	3.17373715036045\\
66.875	0.25404	3.52017324516341	3.52017324516341\\
66.875	0.2577	3.88031607975176	3.88031607975176\\
66.875	0.26136	4.25416565412547	4.25416565412547\\
66.875	0.26502	4.64172196828456	4.64172196828456\\
66.875	0.26868	5.04298502222904	5.04298502222904\\
66.875	0.27234	5.45795481595889	5.45795481595889\\
66.875	0.276	5.88663134947414	5.88663134947414\\
67.25	0.093	1.32081595393591	1.32081595393591\\
67.25	0.09666	1.07345672579135	1.07345672579135\\
67.25	0.10032	0.839804237432166	0.839804237432166\\
67.25	0.10398	0.619858488858364	0.619858488858364\\
67.25	0.10764	0.41361948006995	0.41361948006995\\
67.25	0.1113	0.221087211066902	0.221087211066902\\
67.25	0.11496	0.0422616818492463	0.0422616818492463\\
67.25	0.11862	-0.122857107583044	-0.122857107583044\\
67.25	0.12228	-0.27426915722994	-0.27426915722994\\
67.25	0.12594	-0.411974467091465	-0.411974467091465\\
67.25	0.1296	-0.53597303716761	-0.53597303716761\\
67.25	0.13326	-0.646264867458376	-0.646264867458376\\
67.25	0.13692	-0.742849957963761	-0.742849957963761\\
67.25	0.14058	-0.825728308683765	-0.825728308683765\\
67.25	0.14424	-0.894899919618389	-0.894899919618389\\
67.25	0.1479	-0.950364790767637	-0.950364790767637\\
67.25	0.15156	-0.992122922131498	-0.992122922131498\\
67.25	0.15522	-1.02017431370999	-1.02017431370999\\
67.25	0.15888	-1.0345189655031	-1.0345189655031\\
67.25	0.16254	-1.03515687751083	-1.03515687751083\\
67.25	0.1662	-1.02208804973318	-1.02208804973318\\
67.25	0.16986	-0.995312482170144	-0.995312482170144\\
67.25	0.17352	-0.954830174821728	-0.954830174821728\\
67.25	0.17718	-0.90064112768794	-0.90064112768794\\
67.25	0.18084	-0.832745340768774	-0.832745340768774\\
67.25	0.1845	-0.751142814064224	-0.751142814064224\\
67.25	0.18816	-0.6558335475743	-0.6558335475743\\
67.25	0.19182	-0.546817541298985	-0.546817541298985\\
67.25	0.19548	-0.424094795238304	-0.424094795238304\\
67.25	0.19914	-0.287665309392231	-0.287665309392231\\
67.25	0.2028	-0.137529083760782	-0.137529083760782\\
67.25	0.20646	0.0263138816560478	0.0263138816560478\\
67.25	0.21012	0.203863586858251	0.203863586858251\\
67.25	0.21378	0.395120031845831	0.395120031845831\\
67.25	0.21744	0.600083216618795	0.600083216618795\\
67.25	0.2211	0.81875314117714	0.81875314117714\\
67.25	0.22476	1.05112980552087	1.05112980552087\\
67.25	0.22842	1.29721320964996	1.29721320964996\\
67.25	0.23208	1.55700335356445	1.55700335356445\\
67.25	0.23574	1.83050023726432	1.83050023726432\\
67.25	0.2394	2.11770386074955	2.11770386074955\\
67.25	0.24306	2.41861422402018	2.41861422402018\\
67.25	0.24672	2.73323132707618	2.73323132707618\\
67.25	0.25038	3.06155516991756	3.06155516991756\\
67.25	0.25404	3.40358575254432	3.40358575254432\\
67.25	0.2577	3.75932307495646	3.75932307495646\\
67.25	0.26136	4.12876713715399	4.12876713715399\\
67.25	0.26502	4.51191793913687	4.51191793913687\\
67.25	0.26868	4.90877548090516	4.90877548090516\\
67.25	0.27234	5.31933976245882	5.31933976245882\\
67.25	0.276	5.74361078379785	5.74361078379785\\
67.625	0.093	1.40020120496375	1.40020120496375\\
67.625	0.09666	1.14843646464299	1.14843646464299\\
67.625	0.10032	0.910378464107613	0.910378464107613\\
67.625	0.10398	0.686027203357613	0.686027203357613\\
67.625	0.10764	0.475382682392993	0.475382682392993\\
67.625	0.1113	0.278444901213746	0.278444901213746\\
67.625	0.11496	0.0952138598198848	0.0952138598198848\\
67.625	0.11862	-0.0743104417885974	-0.0743104417885974\\
67.625	0.12228	-0.230128003611699	-0.230128003611699\\
67.625	0.12594	-0.372238825649422	-0.372238825649422\\
67.625	0.1296	-0.500642907901765	-0.500642907901765\\
67.625	0.13326	-0.61534025036873	-0.61534025036873\\
67.625	0.13692	-0.716330853050314	-0.716330853050314\\
67.625	0.14058	-0.803614715946523	-0.803614715946523\\
67.625	0.14424	-0.877191839057346	-0.877191839057346\\
67.625	0.1479	-0.937062222382792	-0.937062222382792\\
67.625	0.15156	-0.983225865922853	-0.983225865922853\\
67.625	0.15522	-1.01568276967754	-1.01568276967754\\
67.625	0.15888	-1.03443293364685	-1.03443293364685\\
67.625	0.16254	-1.03947635783078	-1.03947635783078\\
67.625	0.1662	-1.03081304222932	-1.03081304222932\\
67.625	0.16986	-1.00844298684249	-1.00844298684249\\
67.625	0.17352	-0.972366191670281	-0.972366191670281\\
67.625	0.17718	-0.922582656712699	-0.922582656712699\\
67.625	0.18084	-0.859092381969724	-0.859092381969724\\
67.625	0.1845	-0.781895367441372	-0.781895367441372\\
67.625	0.18816	-0.690991613127647	-0.690991613127647\\
67.625	0.19182	-0.586381119028538	-0.586381119028538\\
67.625	0.19548	-0.468063885144055	-0.468063885144055\\
67.625	0.19914	-0.336039911474181	-0.336039911474181\\
67.625	0.2028	-0.190309198018937	-0.190309198018937\\
67.625	0.20646	-0.030871744778306	-0.030871744778306\\
67.625	0.21012	0.142272448247699	0.142272448247699\\
67.625	0.21378	0.32912338105908	0.32912338105908\\
67.625	0.21744	0.529681053655846	0.529681053655846\\
67.625	0.2211	0.743945466037992	0.743945466037992\\
67.625	0.22476	0.971916618205515	0.971916618205515\\
67.625	0.22842	1.21359451015842	1.21359451015842\\
67.625	0.23208	1.4689791418967	1.4689791418967\\
67.625	0.23574	1.73807051342038	1.73807051342038\\
67.625	0.2394	2.02086862472941	2.02086862472941\\
67.625	0.24306	2.31737347582384	2.31737347582384\\
67.625	0.24672	2.62758506670364	2.62758506670364\\
67.625	0.25038	2.95150339736881	2.95150339736881\\
67.625	0.25404	3.28912846781937	3.28912846781937\\
67.625	0.2577	3.64046027805531	3.64046027805531\\
67.625	0.26136	4.00549882807665	4.00549882807665\\
67.625	0.26502	4.38424411788333	4.38424411788333\\
67.625	0.26868	4.77669614747541	4.77669614747541\\
67.625	0.27234	5.18285491685286	5.18285491685286\\
67.625	0.276	5.6027204260157	5.6027204260157\\
68	0.093	1.48171666388576	1.48171666388576\\
68	0.09666	1.2255464113888	1.2255464113888\\
68	0.10032	0.983082898677222	0.983082898677222\\
68	0.10398	0.754326125751016	0.754326125751016\\
68	0.10764	0.539276092610204	0.539276092610204\\
68	0.1113	0.337932799254759	0.337932799254759\\
68	0.11496	0.150296245684699	0.150296245684699\\
68	0.11862	-0.0236335680999815	-0.0236335680999815\\
68	0.12228	-0.183856642099289	-0.183856642099289\\
68	0.12594	-0.33037297631321	-0.33037297631321\\
68	0.1296	-0.463182570741752	-0.463182570741752\\
68	0.13326	-0.582285425384915	-0.582285425384915\\
68	0.13692	-0.687681540242698	-0.687681540242698\\
68	0.14058	-0.779370915315106	-0.779370915315106\\
68	0.14424	-0.857353550602134	-0.857353550602134\\
68	0.1479	-0.921629446103779	-0.921629446103779\\
68	0.15156	-0.972198601820045	-0.972198601820045\\
68	0.15522	-1.00906101775092	-1.00906101775092\\
68	0.15888	-1.03221669389644	-1.03221669389644\\
68	0.16254	-1.04166563025656	-1.04166563025656\\
68	0.1662	-1.03740782683132	-1.03740782683132\\
68	0.16986	-1.01944328362068	-1.01944328362068\\
68	0.17352	-0.987772000624664	-0.987772000624664\\
68	0.17718	-0.942393977843281	-0.942393977843281\\
68	0.18084	-0.883309215276512	-0.883309215276512\\
68	0.1845	-0.810517712924366	-0.810517712924366\\
68	0.18816	-0.724019470786832	-0.724019470786832\\
68	0.19182	-0.623814488863921	-0.623814488863921\\
68	0.19548	-0.509902767155637	-0.509902767155637\\
68	0.19914	-0.382284305661969	-0.382284305661969\\
68	0.2028	-0.240959104382917	-0.240959104382917\\
68	0.20646	-0.085927163318491	-0.085927163318491\\
68	0.21012	0.0828115175313151	0.0828115175313151\\
68	0.21378	0.265256938166498	0.265256938166498\\
68	0.21744	0.461409098587065	0.461409098587065\\
68	0.2211	0.671267998793013	0.671267998793013\\
68	0.22476	0.894833638784331	0.894833638784331\\
68	0.22842	1.13210601856104	1.13210601856104\\
68	0.23208	1.38308513812312	1.38308513812312\\
68	0.23574	1.64777099747059	1.64777099747059\\
68	0.2394	1.92616359660343	1.92616359660343\\
68	0.24306	2.21826293552165	2.21826293552165\\
68	0.24672	2.52406901422525	2.52406901422525\\
68	0.25038	2.84358183271424	2.84358183271424\\
68	0.25404	3.17680139098859	3.17680139098859\\
68	0.2577	3.52372768904834	3.52372768904834\\
68	0.26136	3.88436072689346	3.88436072689346\\
68	0.26502	4.25870050452395	4.25870050452395\\
68	0.26868	4.64674702193984	4.64674702193984\\
68	0.27234	5.04850027914109	5.04850027914109\\
68	0.276	5.46396027612772	5.46396027612772\\
68.375	0.093	1.56536233070192	1.56536233070192\\
68.375	0.09666	1.30478656602875	1.30478656602875\\
68.375	0.10032	1.05791754114098	1.05791754114098\\
68.375	0.10398	0.82475525603858	0.82475525603858\\
68.375	0.10764	0.605299710721563	0.605299710721563\\
68.375	0.1113	0.399550905189919	0.399550905189919\\
68.375	0.11496	0.207508839443654	0.207508839443654\\
68.375	0.11862	0.0291735134827746	0.0291735134827746\\
68.375	0.12228	-0.135455072692724	-0.135455072692724\\
68.375	0.12594	-0.286376919082851	-0.286376919082851\\
68.375	0.1296	-0.423592025687592	-0.423592025687592\\
68.375	0.13326	-0.547100392506961	-0.547100392506961\\
68.375	0.13692	-0.656902019540942	-0.656902019540942\\
68.375	0.14058	-0.752996906789541	-0.752996906789541\\
68.375	0.14424	-0.835385054252768	-0.835385054252768\\
68.375	0.1479	-0.904066461930618	-0.904066461930618\\
68.375	0.15156	-0.959041129823083	-0.959041129823083\\
68.375	0.15522	-1.00030905793017	-1.00030905793017\\
68.375	0.15888	-1.02787024625187	-1.02787024625187\\
68.375	0.16254	-1.0417246947882	-1.0417246947882\\
68.375	0.1662	-1.04187240353915	-1.04187240353915\\
68.375	0.16986	-1.02831337250471	-1.02831337250471\\
68.375	0.17352	-1.0010476016849	-1.0010476016849\\
68.375	0.17718	-0.960075091079716	-0.960075091079716\\
68.375	0.18084	-0.905395840689145	-0.905395840689145\\
68.375	0.1845	-0.837009850513198	-0.837009850513198\\
68.375	0.18816	-0.75491712055187	-0.75491712055187\\
68.375	0.19182	-0.659117650805157	-0.659117650805157\\
68.375	0.19548	-0.549611441273072	-0.549611441273072\\
68.375	0.19914	-0.426398491955602	-0.426398491955602\\
68.375	0.2028	-0.289478802852756	-0.289478802852756\\
68.375	0.20646	-0.138852373964529	-0.138852373964529\\
68.375	0.21012	0.0254807947090789	0.0254807947090789\\
68.375	0.21378	0.203520703168063	0.203520703168063\\
68.375	0.21744	0.395267351412432	0.395267351412432\\
68.375	0.2211	0.600720739442181	0.600720739442181\\
68.375	0.22476	0.819880867257307	0.819880867257307\\
68.375	0.22842	1.05274773485781	1.05274773485781\\
68.375	0.23208	1.29932134224369	1.29932134224369\\
68.375	0.23574	1.55960168941496	1.55960168941496\\
68.375	0.2394	1.8335887763716	1.8335887763716\\
68.375	0.24306	2.12128260311362	2.12128260311362\\
68.375	0.24672	2.42268316964102	2.42268316964102\\
68.375	0.25038	2.73779047595381	2.73779047595381\\
68.375	0.25404	3.06660452205195	3.06660452205195\\
68.375	0.2577	3.4091253079355	3.4091253079355\\
68.375	0.26136	3.76535283360442	3.76535283360442\\
68.375	0.26502	4.13528709905872	4.13528709905872\\
68.375	0.26868	4.5189281042984	4.5189281042984\\
68.375	0.27234	4.91627584932346	4.91627584932346\\
68.375	0.276	5.32733033413389	5.32733033413389\\
68.75	0.093	1.65113820541223	1.65113820541223\\
68.75	0.09666	1.38615692856288	1.38615692856288\\
68.75	0.10032	1.1348823914989	1.1348823914989\\
68.75	0.10398	0.897314594220299	0.897314594220299\\
68.75	0.10764	0.673453536727076	0.673453536727076\\
68.75	0.1113	0.463299219019234	0.463299219019234\\
68.75	0.11496	0.266851641096777	0.266851641096777\\
68.75	0.11862	0.0841108029596995	0.0841108029596995\\
68.75	0.12228	-0.0849232953920049	-0.0849232953920049\\
68.75	0.12594	-0.240250653958331	-0.240250653958331\\
68.75	0.1296	-0.38187127273927	-0.38187127273927\\
68.75	0.13326	-0.50978515173483	-0.50978515173483\\
68.75	0.13692	-0.623992290945017	-0.623992290945017\\
68.75	0.14058	-0.724492690369821	-0.724492690369821\\
68.75	0.14424	-0.811286350009247	-0.811286350009247\\
68.75	0.1479	-0.884373269863289	-0.884373269863289\\
68.75	0.15156	-0.943753449931959	-0.943753449931959\\
68.75	0.15522	-0.98942689021524	-0.98942689021524\\
68.75	0.15888	-1.02139359071315	-1.02139359071315\\
68.75	0.16254	-1.03965355142568	-1.03965355142568\\
68.75	0.1662	-1.04420677235283	-1.04420677235283\\
68.75	0.16986	-1.03505325349459	-1.03505325349459\\
68.75	0.17352	-1.01219299485098	-1.01219299485098\\
68.75	0.17718	-0.975625996421989	-0.975625996421989\\
68.75	0.18084	-0.925352258207624	-0.925352258207624\\
68.75	0.1845	-0.861371780207875	-0.861371780207875\\
68.75	0.18816	-0.783684562422746	-0.783684562422746\\
68.75	0.19182	-0.692290604852232	-0.692290604852232\\
68.75	0.19548	-0.587189907496338	-0.587189907496338\\
68.75	0.19914	-0.468382470355074	-0.468382470355074\\
68.75	0.2028	-0.335868293428426	-0.335868293428426\\
68.75	0.20646	-0.189647376716398	-0.189647376716398\\
68.75	0.21012	-0.0297197202189885	-0.0297197202189885\\
68.75	0.21378	0.143914676063797	0.143914676063797\\
68.75	0.21744	0.331255812131953	0.331255812131953\\
68.75	0.2211	0.532303687985504	0.532303687985504\\
68.75	0.22476	0.747058303624424	0.747058303624424\\
68.75	0.22842	0.975519659048732	0.975519659048732\\
68.75	0.23208	1.21768775425842	1.21768775425842\\
68.75	0.23574	1.47356258925349	1.47356258925349\\
68.75	0.2394	1.74314416403393	1.74314416403393\\
68.75	0.24306	2.02643247859975	2.02643247859975\\
68.75	0.24672	2.32342753295094	2.32342753295094\\
68.75	0.25038	2.63412932708753	2.63412932708753\\
68.75	0.25404	2.95853786100948	2.95853786100948\\
68.75	0.2577	3.29665313471684	3.29665313471684\\
68.75	0.26136	3.64847514820956	3.64847514820956\\
68.75	0.26502	4.01400390148765	4.01400390148765\\
68.75	0.26868	4.39323939455114	4.39323939455114\\
68.75	0.27234	4.78618162739999	4.78618162739999\\
68.75	0.276	5.19283060003423	5.19283060003423\\
69.125	0.093	1.73904428801671	1.73904428801671\\
69.125	0.09666	1.46965749899115	1.46965749899115\\
69.125	0.10032	1.21397744975097	1.21397744975097\\
69.125	0.10398	0.97200414029618	0.97200414029618\\
69.125	0.10764	0.743737570626751	0.743737570626751\\
69.125	0.1113	0.529177740742718	0.529177740742718\\
69.125	0.11496	0.328324650644055	0.328324650644055\\
69.125	0.11862	0.141178300330772	0.141178300330772\\
69.125	0.12228	-0.0322613101971241	-0.0322613101971241\\
69.125	0.12594	-0.191994180939648	-0.191994180939648\\
69.125	0.1296	-0.338020311896793	-0.338020311896793\\
69.125	0.13326	-0.470339703068552	-0.470339703068552\\
69.125	0.13692	-0.588952354454937	-0.588952354454937\\
69.125	0.14058	-0.69385826605594	-0.69385826605594\\
69.125	0.14424	-0.785057437871572	-0.785057437871572\\
69.125	0.1479	-0.862549869901812	-0.862549869901812\\
69.125	0.15156	-0.926335562146674	-0.926335562146674\\
69.125	0.15522	-0.976414514606153	-0.976414514606153\\
69.125	0.15888	-1.01278672728027	-1.01278672728027\\
69.125	0.16254	-1.03545220016899	-1.03545220016899\\
69.125	0.1662	-1.04441093327234	-1.04441093327234\\
69.125	0.16986	-1.0396629265903	-1.0396629265903\\
69.125	0.17352	-1.0212081801229	-1.0212081801229\\
69.125	0.17718	-0.989046693870108	-0.989046693870108\\
69.125	0.18084	-0.943178467831935	-0.943178467831935\\
69.125	0.1845	-0.883603502008391	-0.883603502008391\\
69.125	0.18816	-0.81032179639946	-0.81032179639946\\
69.125	0.19182	-0.723333351005145	-0.723333351005145\\
69.125	0.19548	-0.622638165825457	-0.622638165825457\\
69.125	0.19914	-0.508236240860391	-0.508236240860391\\
69.125	0.2028	-0.380127576109942	-0.380127576109942\\
69.125	0.20646	-0.238312171574105	-0.238312171574105\\
69.125	0.21012	-0.0827900272529014	-0.0827900272529014\\
69.125	0.21378	0.0864388568536789	0.0864388568536789\\
69.125	0.21744	0.269374480745643	0.269374480745643\\
69.125	0.2211	0.466016844422995	0.466016844422995\\
69.125	0.22476	0.676365947885717	0.676365947885717\\
69.125	0.22842	0.900421791133819	0.900421791133819\\
69.125	0.23208	1.13818437416731	1.13818437416731\\
69.125	0.23574	1.38965369698618	1.38965369698618\\
69.125	0.2394	1.65482975959042	1.65482975959042\\
69.125	0.24306	1.93371256198004	1.93371256198004\\
69.125	0.24672	2.22630210415505	2.22630210415505\\
69.125	0.25038	2.53259838611542	2.53259838611542\\
69.125	0.25404	2.85260140786119	2.85260140786119\\
69.125	0.2577	3.18631116939233	3.18631116939233\\
69.125	0.26136	3.53372767070885	3.53372767070885\\
69.125	0.26502	3.89485091181075	3.89485091181075\\
69.125	0.26868	4.26968089269802	4.26968089269802\\
69.125	0.27234	4.65821761337068	4.65821761337068\\
69.125	0.276	5.06046107382873	5.06046107382873\\
69.5	0.093	1.82908057851534	1.82908057851534\\
69.5	0.09666	1.55528827731358	1.55528827731358\\
69.5	0.10032	1.2952027158972	1.2952027158972\\
69.5	0.10398	1.0488238942662	1.0488238942662\\
69.5	0.10764	0.816151812420581	0.816151812420581\\
69.5	0.1113	0.597186470360349	0.597186470360349\\
69.5	0.11496	0.391927868085488	0.391927868085488\\
69.5	0.11862	0.200376005596006	0.200376005596006\\
69.5	0.12228	0.0225308828919113	0.0225308828919113\\
69.5	0.12594	-0.141607500026819	-0.141607500026819\\
69.5	0.1296	-0.292039143160155	-0.292039143160155\\
69.5	0.13326	-0.428764046508119	-0.428764046508119\\
69.5	0.13692	-0.551782210070703	-0.551782210070703\\
69.5	0.14058	-0.661093633847912	-0.661093633847912\\
69.5	0.14424	-0.756698317839728	-0.756698317839728\\
69.5	0.1479	-0.838596262046181	-0.838596262046181\\
69.5	0.15156	-0.906787466467241	-0.906787466467241\\
69.5	0.15522	-0.961271931102926	-0.961271931102926\\
69.5	0.15888	-1.00204965595324	-1.00204965595324\\
69.5	0.16254	-1.02912064101816	-1.02912064101816\\
69.5	0.1662	-1.04248488629771	-1.04248488629771\\
69.5	0.16986	-1.04214239179187	-1.04214239179187\\
69.5	0.17352	-1.02809315750066	-1.02809315750066\\
69.5	0.17718	-1.00033718342407	-1.00033718342407\\
69.5	0.18084	-0.958874469562105	-0.958874469562105\\
69.5	0.1845	-0.903705015914753	-0.903705015914753\\
69.5	0.18816	-0.83482882248202	-0.83482882248202\\
69.5	0.19182	-0.752245889263911	-0.752245889263911\\
69.5	0.19548	-0.655956216260421	-0.655956216260421\\
69.5	0.19914	-0.545959803471554	-0.545959803471554\\
69.5	0.2028	-0.422256650897303	-0.422256650897303\\
69.5	0.20646	-0.284846758537672	-0.284846758537672\\
69.5	0.21012	-0.133730126392667	-0.133730126392667\\
69.5	0.21378	0.031093245537722	0.031093245537722\\
69.5	0.21744	0.209623357253481	0.209623357253481\\
69.5	0.2211	0.401860208754634	0.401860208754634\\
69.5	0.22476	0.60780380004115	0.60780380004115\\
69.5	0.22842	0.827454131113061	0.827454131113061\\
69.5	0.23208	1.06081120197035	1.06081120197035\\
69.5	0.23574	1.30787501261301	1.30787501261301\\
69.5	0.2394	1.56864556304105	1.56864556304105\\
69.5	0.24306	1.84312285325447	1.84312285325447\\
69.5	0.24672	2.13130688325329	2.13130688325329\\
69.5	0.25038	2.43319765303746	2.43319765303746\\
69.5	0.25404	2.74879516260702	2.74879516260702\\
69.5	0.2577	3.07809941196196	3.07809941196196\\
69.5	0.26136	3.42111040110228	3.42111040110228\\
69.5	0.26502	3.77782813002798	3.77782813002798\\
69.5	0.26868	4.14825259873907	4.14825259873907\\
69.5	0.27234	4.53238380723551	4.53238380723551\\
69.5	0.276	4.93022175551737	4.93022175551737\\
69.875	0.093	1.92124707690812	1.92124707690812\\
69.875	0.09666	1.64304926353017	1.64304926353017\\
69.875	0.10032	1.37855818993759	1.37855818993759\\
69.875	0.10398	1.12777385613039	1.12777385613039\\
69.875	0.10764	0.890696262108565	0.890696262108565\\
69.875	0.1113	0.667325407872134	0.667325407872134\\
69.875	0.11496	0.457661293421068	0.457661293421068\\
69.875	0.11862	0.261703918755394	0.261703918755394\\
69.875	0.12228	0.079453283875087	0.079453283875087\\
69.875	0.12594	-0.0890906112198344	-0.0890906112198344\\
69.875	0.1296	-0.243927766529376	-0.243927766529376\\
69.875	0.13326	-0.385058182053532	-0.385058182053532\\
69.875	0.13692	-0.512481857792322	-0.512481857792322\\
69.875	0.14058	-0.626198793745729	-0.626198793745729\\
69.875	0.14424	-0.726208989913744	-0.726208989913744\\
69.875	0.1479	-0.812512446296395	-0.812512446296395\\
69.875	0.15156	-0.885109162893654	-0.885109162893654\\
69.875	0.15522	-0.943999139705545	-0.943999139705545\\
69.875	0.15888	-0.989182376732046	-0.989182376732046\\
69.875	0.16254	-1.02065887397317	-1.02065887397317\\
69.875	0.1662	-1.03842863142893	-1.03842863142893\\
69.875	0.16986	-1.04249164909928	-1.04249164909928\\
69.875	0.17352	-1.03284792698427	-1.03284792698427\\
69.875	0.17718	-1.00949746508388	-1.00949746508388\\
69.875	0.18084	-0.972440263398113	-0.972440263398113\\
69.875	0.1845	-0.921676321926959	-0.921676321926959\\
69.875	0.18816	-0.857205640670433	-0.857205640670433\\
69.875	0.19182	-0.779028219628522	-0.779028219628522\\
69.875	0.19548	-0.687144058801231	-0.687144058801231\\
69.875	0.19914	-0.581553158188562	-0.581553158188562\\
69.875	0.2028	-0.462255517790517	-0.462255517790517\\
69.875	0.20646	-0.329251137607084	-0.329251137607084\\
69.875	0.21012	-0.182540017638278	-0.182540017638278\\
69.875	0.21378	-0.0221221578840876	-0.0221221578840876\\
69.875	0.21744	0.152002441655473	0.152002441655473\\
69.875	0.2211	0.33983378098042	0.33983378098042\\
69.875	0.22476	0.541371860090752	0.541371860090752\\
69.875	0.22842	0.75661667898645	0.75661667898645\\
69.875	0.23208	0.985568237667536	0.985568237667536\\
69.875	0.23574	1.22822653613401	1.22822653613401\\
69.875	0.2394	1.48459157438584	1.48459157438584\\
69.875	0.24306	1.75466335242307	1.75466335242307\\
69.875	0.24672	2.03844187024567	2.03844187024567\\
69.875	0.25038	2.33592712785367	2.33592712785367\\
69.875	0.25404	2.64711912524702	2.64711912524702\\
69.875	0.2577	2.97201786242576	2.97201786242576\\
69.875	0.26136	3.31062333938988	3.31062333938988\\
69.875	0.26502	3.66293555613937	3.66293555613937\\
69.875	0.26868	4.02895451267426	4.02895451267426\\
69.875	0.27234	4.40868020899453	4.40868020899453\\
69.875	0.276	4.80211264510016	4.80211264510016\\
70.25	0.093	2.01554378319506	2.01554378319506\\
70.25	0.09666	1.73294045764091	1.73294045764091\\
70.25	0.10032	1.46404387187213	1.46404387187213\\
70.25	0.10398	1.20885402588873	1.20885402588873\\
70.25	0.10764	0.967370919690704	0.967370919690704\\
70.25	0.1113	0.739594553278067	0.739594553278067\\
70.25	0.11496	0.525524926650809	0.525524926650809\\
70.25	0.11862	0.32516203980893	0.32516203980893\\
70.25	0.12228	0.138505892752431	0.138505892752431\\
70.25	0.12594	-0.0344435145186957	-0.0344435145186957\\
70.25	0.1296	-0.193686182004429	-0.193686182004429\\
70.25	0.13326	-0.339222109704798	-0.339222109704798\\
70.25	0.13692	-0.471051297619779	-0.471051297619779\\
70.25	0.14058	-0.589173745749385	-0.589173745749385\\
70.25	0.14424	-0.693589454093605	-0.693589454093605\\
70.25	0.1479	-0.784298422652448	-0.784298422652448\\
70.25	0.15156	-0.861300651425912	-0.861300651425912\\
70.25	0.15522	-0.924596140413994	-0.924596140413994\\
70.25	0.15888	-0.974184889616708	-0.974184889616708\\
70.25	0.16254	-1.01006689903403	-1.01006689903403\\
70.25	0.1662	-1.03224216866598	-1.03224216866598\\
70.25	0.16986	-1.04071069851254	-1.04071069851254\\
70.25	0.17352	-1.03547248857373	-1.03547248857373\\
70.25	0.17718	-1.01652753884954	-1.01652753884954\\
70.25	0.18084	-0.983875849339967	-0.983875849339967\\
70.25	0.1845	-0.937517420045012	-0.937517420045012\\
70.25	0.18816	-0.877452250964691	-0.877452250964691\\
70.25	0.19182	-0.803680342098978	-0.803680342098978\\
70.25	0.19548	-0.716201693447886	-0.716201693447886\\
70.25	0.19914	-0.615016305011416	-0.615016305011416\\
70.25	0.2028	-0.500124176789569	-0.500124176789569\\
70.25	0.20646	-0.371525308782342	-0.371525308782342\\
70.25	0.21012	-0.229219700989734	-0.229219700989734\\
70.25	0.21378	-0.0732073534117426	-0.0732073534117426\\
70.25	0.21744	0.096511733951619	0.096511733951619\\
70.25	0.2211	0.279937561100368	0.279937561100368\\
70.25	0.22476	0.477070128034494	0.477070128034494\\
70.25	0.22842	0.687909434753994	0.687909434753994\\
70.25	0.23208	0.912455481258888	0.912455481258888\\
70.25	0.23574	1.15070826754915	1.15070826754915\\
70.25	0.2394	1.40266779362479	1.40266779362479\\
70.25	0.24306	1.66833405948582	1.66833405948582\\
70.25	0.24672	1.94770706513222	1.94770706513222\\
70.25	0.25038	2.24078681056401	2.24078681056401\\
70.25	0.25404	2.54757329578116	2.54757329578116\\
70.25	0.2577	2.86806652078371	2.86806652078371\\
70.25	0.26136	3.20226648557164	3.20226648557164\\
70.25	0.26502	3.55017319014492	3.55017319014492\\
70.25	0.26868	3.91178663450361	3.91178663450361\\
70.25	0.27234	4.28710681864766	4.28710681864766\\
70.25	0.276	4.67613374257711	4.67613374257711\\
70.625	0.093	2.11197069737617	2.11197069737617\\
70.625	0.09666	1.82496185964581	1.82496185964581\\
70.625	0.10032	1.55165976170083	1.55165976170083\\
70.625	0.10398	1.29206440354123	1.29206440354123\\
70.625	0.10764	1.04617578516702	1.04617578516702\\
70.625	0.1113	0.813993906578176	0.813993906578176\\
70.625	0.11496	0.595518767774712	0.595518767774712\\
70.625	0.11862	0.390750368756642	0.390750368756642\\
70.625	0.12228	0.199688709523937	0.199688709523937\\
70.625	0.12594	0.0223337900766118	0.0223337900766118\\
70.625	0.1296	-0.14131438958532	-0.14131438958532\\
70.625	0.13326	-0.291255829461887	-0.291255829461887\\
70.625	0.13692	-0.427490529553074	-0.427490529553074\\
70.625	0.14058	-0.550018489858878	-0.550018489858878\\
70.625	0.14424	-0.65883971037929	-0.65883971037929\\
70.625	0.1479	-0.753954191114339	-0.753954191114339\\
70.625	0.15156	-0.835361932064002	-0.835361932064002\\
70.625	0.15522	-0.903062933228282	-0.903062933228282\\
70.625	0.15888	-0.957057194607195	-0.957057194607195\\
70.625	0.16254	-0.997344716200718	-0.997344716200718\\
70.625	0.1662	-1.02392549800887	-1.02392549800887\\
70.625	0.16986	-1.03679954003163	-1.03679954003163\\
70.625	0.17352	-1.03596684226902	-1.03596684226902\\
70.625	0.17718	-1.02142740472103	-1.02142740472103\\
70.625	0.18084	-0.993181227387659	-0.993181227387659\\
70.625	0.1845	-0.951228310268903	-0.951228310268903\\
70.625	0.18816	-0.89556865336478	-0.89556865336478\\
70.625	0.19182	-0.826202256675266	-0.826202256675266\\
70.625	0.19548	-0.743129120200379	-0.743129120200379\\
70.625	0.19914	-0.646349243940108	-0.646349243940108\\
70.625	0.2028	-0.535862627894453	-0.535862627894453\\
70.625	0.20646	-0.411669272063424	-0.411669272063424\\
70.625	0.21012	-0.273769176447015	-0.273769176447015\\
70.625	0.21378	-0.122162341045229	-0.122162341045229\\
70.625	0.21744	0.0431512341419342	0.0431512341419342\\
70.625	0.2211	0.222171549114485	0.222171549114485\\
70.625	0.22476	0.414898603872413	0.414898603872413\\
70.625	0.22842	0.621332398415714	0.621332398415714\\
70.625	0.23208	0.841472932744395	0.841472932744395\\
70.625	0.23574	1.07532020685846	1.07532020685846\\
70.625	0.2394	1.32287422075791	1.32287422075791\\
70.625	0.24306	1.58413497444273	1.58413497444273\\
70.625	0.24672	1.85910246791293	1.85910246791293\\
70.625	0.25038	2.14777670116852	2.14777670116852\\
70.625	0.25404	2.45015767420947	2.45015767420947\\
70.625	0.2577	2.76624538703583	2.76624538703583\\
70.625	0.26136	3.09603983964755	3.09603983964755\\
70.625	0.26502	3.43954103204464	3.43954103204464\\
70.625	0.26868	3.79674896422713	3.79674896422713\\
70.625	0.27234	4.16766363619498	4.16766363619498\\
70.625	0.276	4.55228504794823	4.55228504794823\\
71	0.093	2.21052781945142	2.21052781945142\\
71	0.09666	1.91911346954487	1.91911346954487\\
71	0.10032	1.6414058594237	1.6414058594237\\
71	0.10398	1.37740498908789	1.37740498908789\\
71	0.10764	1.12711085853747	1.12711085853747\\
71	0.1113	0.890523467772439	0.890523467772439\\
71	0.11496	0.667642816792777	0.667642816792777\\
71	0.11862	0.458468905598494	0.458468905598494\\
71	0.12228	0.263001734189598	0.263001734189598\\
71	0.12594	0.0812413025660739	0.0812413025660739\\
71	0.1296	-0.0868123892720636	-0.0868123892720636\\
71	0.13326	-0.241159341324829	-0.241159341324829\\
71	0.13692	-0.381799553592208	-0.381799553592208\\
71	0.14058	-0.508733026074211	-0.508733026074211\\
71	0.14424	-0.621959758770835	-0.621959758770835\\
71	0.1479	-0.721479751682082	-0.721479751682082\\
71	0.15156	-0.807293004807944	-0.807293004807944\\
71	0.15522	-0.87939951814843	-0.87939951814843\\
71	0.15888	-0.937799291703534	-0.937799291703534\\
71	0.16254	-0.982492325473256	-0.982492325473256\\
71	0.1662	-1.01347861945761	-1.01347861945761\\
71	0.16986	-1.03075817365657	-1.03075817365657\\
71	0.17352	-1.03433098807015	-1.03433098807015\\
71	0.17718	-1.02419706269836	-1.02419706269836\\
71	0.18084	-1.0003563975412	-1.0003563975412\\
71	0.1845	-0.962808992598639	-0.962808992598639\\
71	0.18816	-0.911554847870715	-0.911554847870715\\
71	0.19182	-0.8465939633574	-0.8465939633574\\
71	0.19548	-0.767926339058704	-0.767926339058704\\
71	0.19914	-0.675551974974638	-0.675551974974638\\
71	0.2028	-0.569470871105189	-0.569470871105189\\
71	0.20646	-0.449683027450352	-0.449683027450352\\
71	0.21012	-0.316188444010148	-0.316188444010148\\
71	0.21378	-0.168987120784561	-0.168987120784561\\
71	0.21744	-0.00807905777360318	-0.00807905777360318\\
71	0.2211	0.166535745022749	0.166535745022749\\
71	0.22476	0.354857287604471	0.354857287604471\\
71	0.22842	0.55688556997158	0.55688556997158\\
71	0.23208	0.772620592124071	0.772620592124071\\
71	0.23574	1.00206235406193	1.00206235406193\\
71	0.2394	1.24521085578518	1.24521085578518\\
71	0.24306	1.50206609729381	1.50206609729381\\
71	0.24672	1.77262807858781	1.77262807858781\\
71	0.25038	2.05689679966719	2.05689679966719\\
71	0.25404	2.35487226053196	2.35487226053196\\
71	0.2577	2.66655446118209	2.66655446118209\\
71	0.26136	2.99194340161762	2.99194340161762\\
71	0.26502	3.33103908183852	3.33103908183852\\
71	0.26868	3.6838415018448	3.6838415018448\\
71	0.27234	4.05035066163645	4.05035066163645\\
71	0.276	4.4305665612135	4.4305665612135\\
71.375	0.093	2.31121514942083	2.31121514942083\\
71.375	0.09666	2.01539528733808	2.01539528733808\\
71.375	0.10032	1.7332821650407	1.7332821650407\\
71.375	0.10398	1.4648757825287	1.4648757825287\\
71.375	0.10764	1.21017613980208	1.21017613980208\\
71.375	0.1113	0.969183236860843	0.969183236860843\\
71.375	0.11496	0.741897073704982	0.741897073704982\\
71.375	0.11862	0.5283176503345	0.5283176503345\\
71.375	0.12228	0.328444966749413	0.328444966749413\\
71.375	0.12594	0.142279022949683	0.142279022949683\\
71.375	0.1296	-0.0301801810646527	-0.0301801810646527\\
71.375	0.13326	-0.188932645293617	-0.188932645293617\\
71.375	0.13692	-0.333978369737201	-0.333978369737201\\
71.375	0.14058	-0.465317354395403	-0.465317354395403\\
71.375	0.14424	-0.582949599268225	-0.582949599268225\\
71.375	0.1479	-0.686875104355664	-0.686875104355664\\
71.375	0.15156	-0.777093869657731	-0.777093869657731\\
71.375	0.15522	-0.853605895174409	-0.853605895174409\\
71.375	0.15888	-0.916411180905719	-0.916411180905719\\
71.375	0.16254	-0.965509726851646	-0.965509726851646\\
71.375	0.1662	-1.00090153301219	-1.00090153301219\\
71.375	0.16986	-1.02258659938735	-1.02258659938735\\
71.375	0.17352	-1.03056492597714	-1.03056492597714\\
71.375	0.17718	-1.02483651278155	-1.02483651278155\\
71.375	0.18084	-1.00540135980058	-1.00540135980058\\
71.375	0.1845	-0.97225946703422	-0.97225946703422\\
71.375	0.18816	-0.925410834482495	-0.925410834482495\\
71.375	0.19182	-0.864855462145385	-0.864855462145385\\
71.375	0.19548	-0.790593350022888	-0.790593350022888\\
71.375	0.19914	-0.702624498115021	-0.702624498115021\\
71.375	0.2028	-0.60094890642177	-0.60094890642177\\
71.375	0.20646	-0.485566574943139	-0.485566574943139\\
71.375	0.21012	-0.356477503679134	-0.356477503679134\\
71.375	0.21378	-0.213681692629745	-0.213681692629745\\
71.375	0.21744	-0.0571791417949861	-0.0571791417949861\\
71.375	0.2211	0.113030148825167	0.113030148825167\\
71.375	0.22476	0.296946179230698	0.296946179230698\\
71.375	0.22842	0.494568949421602	0.494568949421602\\
71.375	0.23208	0.705898459397886	0.705898459397886\\
71.375	0.23574	0.930934709159558	0.930934709159558\\
71.375	0.2394	1.1696776987066	1.1696776987066\\
71.375	0.24306	1.42212742803902	1.42212742803902\\
71.375	0.24672	1.68828389715682	1.68828389715682\\
71.375	0.25038	1.96814710606001	1.96814710606001\\
71.375	0.25404	2.26171705474857	2.26171705474857\\
71.375	0.2577	2.56899374322251	2.56899374322251\\
71.375	0.26136	2.88997717148185	2.88997717148185\\
71.375	0.26502	3.22466733952654	3.22466733952654\\
71.375	0.26868	3.57306424735662	3.57306424735662\\
71.375	0.27234	3.93516789497207	3.93516789497207\\
71.375	0.276	4.31097828237291	4.31097828237291\\
71.75	0.093	2.4140326872844	2.4140326872844\\
71.75	0.09666	2.11380731302544	2.11380731302544\\
71.75	0.10032	1.82728867855186	1.82728867855186\\
71.75	0.10398	1.55447678386366	1.55447678386366\\
71.75	0.10764	1.29537162896085	1.29537162896085\\
71.75	0.1113	1.04997321384341	1.04997321384341\\
71.75	0.11496	0.818281538511349	0.818281538511349\\
71.75	0.11862	0.600296602964669	0.600296602964669\\
71.75	0.12228	0.396018407203369	0.396018407203369\\
71.75	0.12594	0.205446951227454	0.205446951227454\\
71.75	0.1296	0.0285822350369127	0.0285822350369127\\
71.75	0.13326	-0.13457574136825	-0.13457574136825\\
71.75	0.13692	-0.284026977988033	-0.284026977988033\\
71.75	0.14058	-0.419771474822433	-0.419771474822433\\
71.75	0.14424	-0.541809231871461	-0.541809231871461\\
71.75	0.1479	-0.650140249135099	-0.650140249135099\\
71.75	0.15156	-0.744764526613364	-0.744764526613364\\
71.75	0.15522	-0.825682064306248	-0.825682064306248\\
71.75	0.15888	-0.892892862213756	-0.892892862213756\\
71.75	0.16254	-0.946396920335875	-0.946396920335875\\
71.75	0.1662	-0.986194238672629	-0.986194238672629\\
71.75	0.16986	-1.01228481722399	-1.01228481722399\\
71.75	0.17352	-1.02466865598997	-1.02466865598997\\
71.75	0.17718	-1.02334575497058	-1.02334575497058\\
71.75	0.18084	-1.00831611416581	-1.00831611416581\\
71.75	0.1845	-0.979579733575662	-0.979579733575662\\
71.75	0.18816	-0.937136613200128	-0.937136613200128\\
71.75	0.19182	-0.880986753039217	-0.880986753039217\\
71.75	0.19548	-0.811130153092918	-0.811130153092918\\
71.75	0.19914	-0.727566813361257	-0.727566813361257\\
71.75	0.2028	-0.630296733844204	-0.630296733844204\\
71.75	0.20646	-0.519319914541772	-0.519319914541772\\
71.75	0.21012	-0.394636355453965	-0.394636355453965\\
71.75	0.21378	-0.256246056580775	-0.256246056580775\\
71.75	0.21744	-0.104149017922214	-0.104149017922214\\
71.75	0.2211	0.0616547605217406	0.0616547605217406\\
71.75	0.22476	0.241165278751065	0.241165278751065\\
71.75	0.22842	0.434382536765778	0.434382536765778\\
71.75	0.23208	0.641306534565857	0.641306534565857\\
71.75	0.23574	0.861937272151323	0.861937272151323\\
71.75	0.2394	1.09627474952217	1.09627474952217\\
71.75	0.24306	1.34431896667839	1.34431896667839\\
71.75	0.24672	1.60606992362	1.60606992362\\
71.75	0.25038	1.88152762034698	1.88152762034698\\
71.75	0.25404	2.17069205685934	2.17069205685934\\
71.75	0.2577	2.47356323315709	2.47356323315709\\
71.75	0.26136	2.79014114924021	2.79014114924021\\
71.75	0.26502	3.12042580510872	3.12042580510872\\
71.75	0.26868	3.46441720076259	3.46441720076259\\
71.75	0.27234	3.82211533620185	3.82211533620185\\
71.75	0.276	4.19352021142649	4.19352021142649\\
72.125	0.093	2.51898043304211	2.51898043304211\\
72.125	0.09666	2.21434954660695	2.21434954660695\\
72.125	0.10032	1.92342539995718	1.92342539995718\\
72.125	0.10398	1.64620799309278	1.64620799309278\\
72.125	0.10764	1.38269732601376	1.38269732601376\\
72.125	0.1113	1.13289339872013	1.13289339872013\\
72.125	0.11496	0.89679621121187	0.89679621121187\\
72.125	0.11862	0.674405763488991	0.674405763488991\\
72.125	0.12228	0.465722055551493	0.465722055551493\\
72.125	0.12594	0.270745087399373	0.270745087399373\\
72.125	0.1296	0.0894748590326326	0.0894748590326326\\
72.125	0.13326	-0.0780886295487289	-0.0780886295487289\\
72.125	0.13692	-0.23194537834471	-0.23194537834471\\
72.125	0.14058	-0.372095387355316	-0.372095387355316\\
72.125	0.14424	-0.498538656580543	-0.498538656580543\\
72.125	0.1479	-0.611275186020379	-0.611275186020379\\
72.125	0.15156	-0.710304975674843	-0.710304975674843\\
72.125	0.15522	-0.795628025543918	-0.795628025543918\\
72.125	0.15888	-0.867244335627632	-0.867244335627632\\
72.125	0.16254	-0.925153905925956	-0.925153905925956\\
72.125	0.1662	-0.969356736438902	-0.969356736438902\\
72.125	0.16986	-0.999852827166466	-0.999852827166466\\
72.125	0.17352	-1.01664217810865	-1.01664217810865\\
72.125	0.17718	-1.01972478926546	-1.01972478926546\\
72.125	0.18084	-1.00910066063689	-1.00910066063689\\
72.125	0.1845	-0.984769792222934	-0.984769792222934\\
72.125	0.18816	-0.946732184023606	-0.946732184023606\\
72.125	0.19182	-0.894987836038894	-0.894987836038894\\
72.125	0.19548	-0.829536748268794	-0.829536748268794\\
72.125	0.19914	-0.750378920713331	-0.750378920713331\\
72.125	0.2028	-0.657514353372477	-0.657514353372477\\
72.125	0.20646	-0.55094304624625	-0.55094304624625\\
72.125	0.21012	-0.430664999334642	-0.430664999334642\\
72.125	0.21378	-0.29668021263765	-0.29668021263765\\
72.125	0.21744	-0.148988686155288	-0.148988686155288\\
72.125	0.2211	0.0124095801124753	0.0124095801124753\\
72.125	0.22476	0.187514586165587	0.187514586165587\\
72.125	0.22842	0.376326332004094	0.376326332004094\\
72.125	0.23208	0.578844817627981	0.578844817627981\\
72.125	0.23574	0.795070043037256	0.795070043037256\\
72.125	0.2394	1.0250020082319	1.0250020082319\\
72.125	0.24306	1.26864071321192	1.26864071321192\\
72.125	0.24672	1.52598615797733	1.52598615797733\\
72.125	0.25038	1.79703834252812	1.79703834252812\\
72.125	0.25404	2.08179726686426	2.08179726686426\\
72.125	0.2577	2.38026293098581	2.38026293098581\\
72.125	0.26136	2.69243533489274	2.69243533489274\\
72.125	0.26502	3.01831447858503	3.01831447858503\\
72.125	0.26868	3.35790036206273	3.35790036206273\\
72.125	0.27234	3.71119298532577	3.71119298532577\\
72.125	0.276	4.07819234837422	4.07819234837422\\
72.5	0.093	2.626058386694	2.626058386694\\
72.5	0.09666	2.31702198808264	2.31702198808264\\
72.5	0.10032	2.02169232925667	2.02169232925667\\
72.5	0.10398	1.74006941021607	1.74006941021607\\
72.5	0.10764	1.47215323096085	1.47215323096085\\
72.5	0.1113	1.21794379149102	1.21794379149102\\
72.5	0.11496	0.977441091806553	0.977441091806553\\
72.5	0.11862	0.750645131907476	0.750645131907476\\
72.5	0.12228	0.537555911793778	0.537555911793778\\
72.5	0.12594	0.338173431465453	0.338173431465453\\
72.5	0.1296	0.152497690922521	0.152497690922521\\
72.5	0.13326	-0.0194713098350459	-0.0194713098350459\\
72.5	0.13692	-0.177733570807225	-0.177733570807225\\
72.5	0.14058	-0.32228909199403	-0.32228909199403\\
72.5	0.14424	-0.453137873395448	-0.453137873395448\\
72.5	0.1479	-0.57027991501149	-0.57027991501149\\
72.5	0.15156	-0.673715216842153	-0.673715216842153\\
72.5	0.15522	-0.76344377888744	-0.76344377888744\\
72.5	0.15888	-0.839465601147339	-0.839465601147339\\
72.5	0.16254	-0.901780683621862	-0.901780683621862\\
72.5	0.1662	-0.950389026311013	-0.950389026311013\\
72.5	0.16986	-0.985290629214775	-0.985290629214775\\
72.5	0.17352	-1.00648549233316	-1.00648549233316\\
72.5	0.17718	-1.01397361566617	-1.01397361566617\\
72.5	0.18084	-1.00775499921379	-1.00775499921379\\
72.5	0.1845	-0.987829642976045	-0.987829642976045\\
72.5	0.18816	-0.954197546952916	-0.954197546952916\\
72.5	0.19182	-0.906858711144395	-0.906858711144395\\
72.5	0.19548	-0.845813135550507	-0.845813135550507\\
72.5	0.19914	-0.771060820171236	-0.771060820171236\\
72.5	0.2028	-0.682601765006588	-0.682601765006588\\
72.5	0.20646	-0.580435970056552	-0.580435970056552\\
72.5	0.21012	-0.464563435321143	-0.464563435321143\\
72.5	0.21378	-0.334984160800357	-0.334984160800357\\
72.5	0.21744	-0.191698146494193	-0.191698146494193\\
72.5	0.2211	-0.0347053924026497	-0.0347053924026497\\
72.5	0.22476	0.135994101474285	0.135994101474285\\
72.5	0.22842	0.3204003351366	0.3204003351366\\
72.5	0.23208	0.518513308584282	0.518513308584282\\
72.5	0.23574	0.730333021817351	0.730333021817351\\
72.5	0.2394	0.955859474835787	0.955859474835787\\
72.5	0.24306	1.19509266763962	1.19509266763962\\
72.5	0.24672	1.44803260022882	1.44803260022882\\
72.5	0.25038	1.71467927260341	1.71467927260341\\
72.5	0.25404	1.99503268476336	1.99503268476336\\
72.5	0.2577	2.28909283670872	2.28909283670872\\
72.5	0.26136	2.59685972843945	2.59685972843945\\
72.5	0.26502	2.91833335995554	2.91833335995554\\
72.5	0.26868	3.25351373125702	3.25351373125702\\
72.5	0.27234	3.60240084234388	3.60240084234388\\
72.5	0.276	3.96499469321612	3.96499469321612\\
72.875	0.093	2.73526654824003	2.73526654824003\\
72.875	0.09666	2.42182463745247	2.42182463745247\\
72.875	0.10032	2.1220894664503	2.1220894664503\\
72.875	0.10398	1.8360610352335	1.8360610352335\\
72.875	0.10764	1.56373934380209	1.56373934380209\\
72.875	0.1113	1.30512439215605	1.30512439215605\\
72.875	0.11496	1.06021618029538	1.06021618029538\\
72.875	0.11862	0.829014708220107	0.829014708220107\\
72.875	0.12228	0.611519975930212	0.611519975930212\\
72.875	0.12594	0.407731983425688	0.407731983425688\\
72.875	0.1296	0.217650730706557	0.217650730706557\\
72.875	0.13326	0.0412762177727917	0.0412762177727917\\
72.875	0.13692	-0.121391555375594	-0.121391555375594\\
72.875	0.14058	-0.270352588738596	-0.270352588738596\\
72.875	0.14424	-0.405606882316206	-0.405606882316206\\
72.875	0.1479	-0.527154436108454	-0.527154436108454\\
72.875	0.15156	-0.634995250115315	-0.634995250115315\\
72.875	0.15522	-0.729129324336801	-0.729129324336801\\
72.875	0.15888	-0.809556658772905	-0.809556658772905\\
72.875	0.16254	-0.876277253423627	-0.876277253423627\\
72.875	0.1662	-0.929291108288977	-0.929291108288977\\
72.875	0.16986	-0.968598223368938	-0.968598223368938\\
72.875	0.17352	-0.994198598663521	-0.994198598663521\\
72.875	0.17718	-1.00609223417273	-1.00609223417273\\
72.875	0.18084	-1.00427912989656	-1.00427912989656\\
72.875	0.1845	-0.988759285835009	-0.988759285835009\\
72.875	0.18816	-0.959532701988078	-0.959532701988078\\
72.875	0.19182	-0.916599378355762	-0.916599378355762\\
72.875	0.19548	-0.859959314938074	-0.859959314938074\\
72.875	0.19914	-0.789612511735001	-0.789612511735001\\
72.875	0.2028	-0.705558968746544	-0.705558968746544\\
72.875	0.20646	-0.607798685972707	-0.607798685972707\\
72.875	0.21012	-0.496331663413503	-0.496331663413503\\
72.875	0.21378	-0.371157901068916	-0.371157901068916\\
72.875	0.21744	-0.232277398938951	-0.232277398938951\\
72.875	0.2211	-0.0796901570235988	-0.0796901570235988\\
72.875	0.22476	0.0866038246771303	0.0866038246771303\\
72.875	0.22842	0.26660454616324	0.26660454616324\\
72.875	0.23208	0.460312007434716	0.460312007434716\\
72.875	0.23574	0.667726208491594	0.667726208491594\\
72.875	0.2394	0.888847149333838	0.888847149333838\\
72.875	0.24306	1.12367482996145	1.12367482996145\\
72.875	0.24672	1.37220925037446	1.37220925037446\\
72.875	0.25038	1.63445041057285	1.63445041057285\\
72.875	0.25404	1.9103983105566	1.9103983105566\\
72.875	0.2577	2.20005295032577	2.20005295032577\\
72.875	0.26136	2.50341432988028	2.50341432988028\\
72.875	0.26502	2.82048244922018	2.82048244922018\\
72.875	0.26868	3.15125730834546	3.15125730834546\\
72.875	0.27234	3.49573890725613	3.49573890725613\\
72.875	0.276	3.85392724595216	3.85392724595216\\
73.25	0.093	2.84660491768022	2.84660491768022\\
73.25	0.09666	2.52875749471646	2.52875749471646\\
73.25	0.10032	2.22461681153808	2.22461681153808\\
73.25	0.10398	1.93418286814509	1.93418286814509\\
73.25	0.10764	1.65745566453748	1.65745566453748\\
73.25	0.1113	1.39443520071524	1.39443520071524\\
73.25	0.11496	1.14512147667838	1.14512147667838\\
73.25	0.11862	0.909514492426894	0.909514492426894\\
73.25	0.12228	0.687614247960807	0.687614247960807\\
73.25	0.12594	0.479420743280084	0.479420743280084\\
73.25	0.1296	0.284933978384741	0.284933978384741\\
73.25	0.13326	0.104153953274777	0.104153953274777\\
73.25	0.13692	-0.0629193320498	-0.0629193320498\\
73.25	0.14058	-0.216285877589002	-0.216285877589002\\
73.25	0.14424	-0.355945683342817	-0.355945683342817\\
73.25	0.1479	-0.481898749311263	-0.481898749311263\\
73.25	0.15156	-0.594145075494323	-0.594145075494323\\
73.25	0.15522	-0.692684661892015	-0.692684661892015\\
73.25	0.15888	-0.777517508504317	-0.777517508504317\\
73.25	0.16254	-0.848643615331238	-0.848643615331238\\
73.25	0.1662	-0.906062982372779	-0.906062982372779\\
73.25	0.16986	-0.949775609628945	-0.949775609628945\\
73.25	0.17352	-0.979781497099728	-0.979781497099728\\
73.25	0.17718	-0.996080644785138	-0.996080644785138\\
73.25	0.18084	-0.998673052685163	-0.998673052685163\\
73.25	0.1845	-0.987558720799811	-0.987558720799811\\
73.25	0.18816	-0.962737649129078	-0.962737649129078\\
73.25	0.19182	-0.924209837672969	-0.924209837672969\\
73.25	0.19548	-0.871975286431478	-0.871975286431478\\
73.25	0.19914	-0.806033995404604	-0.806033995404604\\
73.25	0.2028	-0.726385964592353	-0.726385964592353\\
73.25	0.20646	-0.633031193994714	-0.633031193994714\\
73.25	0.21012	-0.525969683611709	-0.525969683611709\\
73.25	0.21378	-0.405201433443327	-0.405201433443327\\
73.25	0.21744	-0.270726443489547	-0.270726443489547\\
73.25	0.2211	-0.122544713750401	-0.122544713750401\\
73.25	0.22476	0.039343755774123	0.039343755774123\\
73.25	0.22842	0.214938965084027	0.214938965084027\\
73.25	0.23208	0.404240914179312	0.404240914179312\\
73.25	0.23574	0.607249603059984	0.607249603059984\\
73.25	0.2394	0.823965031726036	0.823965031726036\\
73.25	0.24306	1.05438720017746	1.05438720017746\\
73.25	0.24672	1.29851610841426	1.29851610841426\\
73.25	0.25038	1.55635175643645	1.55635175643645\\
73.25	0.25404	1.82789414424401	1.82789414424401\\
73.25	0.2577	2.11314327183697	2.11314327183697\\
73.25	0.26136	2.41209913921527	2.41209913921527\\
73.25	0.26502	2.72476174637898	2.72476174637898\\
73.25	0.26868	3.05113109332806	3.05113109332806\\
73.25	0.27234	3.39120718006253	3.39120718006253\\
73.25	0.276	3.74499000658236	3.74499000658236\\
73.625	0.093	2.96007349501456	2.96007349501456\\
73.625	0.09666	2.63782055987461	2.63782055987461\\
73.625	0.10032	2.32927436452004	2.32927436452004\\
73.625	0.10398	2.03443490895084	2.03443490895084\\
73.625	0.10764	1.75330219316703	1.75330219316703\\
73.625	0.1113	1.48587621716858	1.48587621716858\\
73.625	0.11496	1.23215698095553	1.23215698095553\\
73.625	0.11862	0.992144484527849	0.992144484527849\\
73.625	0.12228	0.765838727885556	0.765838727885556\\
73.625	0.12594	0.553239711028635	0.553239711028635\\
73.625	0.1296	0.354347433957093	0.354347433957093\\
73.625	0.13326	0.16916189667093	0.16916189667093\\
73.625	0.13692	-0.00231690082984493	-0.00231690082984493\\
73.625	0.14058	-0.160088958545252	-0.160088958545252\\
73.625	0.14424	-0.304154276475266	-0.304154276475266\\
73.625	0.1479	-0.434512854619911	-0.434512854619911\\
73.625	0.15156	-0.551164692979169	-0.551164692979169\\
73.625	0.15522	-0.654109791553052	-0.654109791553052\\
73.625	0.15888	-0.743348150341561	-0.743348150341561\\
73.625	0.16254	-0.818879769344679	-0.818879769344679\\
73.625	0.1662	-0.880704648562434	-0.880704648562434\\
73.625	0.16986	-0.928822787994791	-0.928822787994791\\
73.625	0.17352	-0.963234187641772	-0.963234187641772\\
73.625	0.17718	-0.983938847503381	-0.983938847503381\\
73.625	0.18084	-0.990936767579612	-0.990936767579612\\
73.625	0.1845	-0.984227947870451	-0.984227947870451\\
73.625	0.18816	-0.963812388375924	-0.963812388375924\\
73.625	0.19182	-0.929690089096006	-0.929690089096006\\
73.625	0.19548	-0.881861050030722	-0.881861050030722\\
73.625	0.19914	-0.820325271180046	-0.820325271180046\\
73.625	0.2028	-0.745082752543993	-0.745082752543993\\
73.625	0.20646	-0.65613349412256	-0.65613349412256\\
73.625	0.21012	-0.553477495915747	-0.553477495915747\\
73.625	0.21378	-0.43711475792357	-0.43711475792357\\
73.625	0.21744	-0.307045280145996	-0.307045280145996\\
73.625	0.2211	-0.163269062583041	-0.163269062583041\\
73.625	0.22476	-0.00578610523472278	-0.00578610523472278\\
73.625	0.22842	0.16540359189899	0.16540359189899\\
73.625	0.23208	0.350300028818069	0.350300028818069\\
73.625	0.23574	0.548903205522549	0.548903205522549\\
73.625	0.2394	0.761213122012396	0.761213122012396\\
73.625	0.24306	0.987229778287613	0.987229778287613\\
73.625	0.24672	1.22695317434822	1.22695317434822\\
73.625	0.25038	1.48038331019421	1.48038331019421\\
73.625	0.25404	1.74752018582557	1.74752018582557\\
73.625	0.2577	2.02836380124233	2.02836380124233\\
73.625	0.26136	2.32291415644444	2.32291415644444\\
73.625	0.26502	2.63117125143194	2.63117125143194\\
73.625	0.26868	2.95313508620483	2.95313508620483\\
73.625	0.27234	3.28880566076308	3.28880566076308\\
73.625	0.276	3.63818297510673	3.63818297510673\\
74	0.093	3.07567228024306	3.07567228024306\\
74	0.09666	2.74901383292691	2.74901383292691\\
74	0.10032	2.43606212539614	2.43606212539614\\
74	0.10398	2.13681715765074	2.13681715765074\\
74	0.10764	1.85127892969072	1.85127892969072\\
74	0.1113	1.57944744151609	1.57944744151609\\
74	0.11496	1.32132269312683	1.32132269312683\\
74	0.11862	1.07690468452295	1.07690468452295\\
74	0.12228	0.846193415704453	0.846193415704453\\
74	0.12594	0.629188886671333	0.629188886671333\\
74	0.1296	0.425891097423593	0.425891097423593\\
74	0.13326	0.236300047961231	0.236300047961231\\
74	0.13692	0.0604157382842505	0.0604157382842505\\
74	0.14058	-0.101761831607348	-0.101761831607348\\
74	0.14424	-0.250232661713568	-0.250232661713568\\
74	0.1479	-0.384996752034411	-0.384996752034411\\
74	0.15156	-0.506054102569875	-0.506054102569875\\
74	0.15522	-0.613404713319957	-0.613404713319957\\
74	0.15888	-0.707048584284657	-0.707048584284657\\
74	0.16254	-0.786985715463981	-0.786985715463981\\
74	0.1662	-0.853216106857927	-0.853216106857927\\
74	0.16986	-0.90573975846649	-0.90573975846649\\
74	0.17352	-0.944556670289669	-0.944556670289669\\
74	0.17718	-0.969666842327484	-0.969666842327484\\
74	0.18084	-0.981070274579906	-0.981070274579906\\
74	0.1845	-0.978766967046958	-0.978766967046958\\
74	0.18816	-0.962756919728623	-0.962756919728623\\
74	0.19182	-0.933040132624903	-0.933040132624903\\
74	0.19548	-0.889616605735817	-0.889616605735817\\
74	0.19914	-0.83248633906134	-0.83248633906134\\
74	0.2028	-0.761649332601493	-0.761649332601493\\
74	0.20646	-0.677105586356259	-0.677105586356259\\
74	0.21012	-0.578855100325651	-0.578855100325651\\
74	0.21378	-0.466897874509666	-0.466897874509666\\
74	0.21744	-0.341233908908297	-0.341233908908297\\
74	0.2211	-0.201863203521548	-0.201863203521548\\
74	0.22476	-0.0487857583494069	-0.0487857583494069\\
74	0.22842	0.1179984266081	0.1179984266081\\
74	0.23208	0.298489351350987	0.298489351350987\\
74	0.23574	0.492687015879248	0.492687015879248\\
74	0.2394	0.700591420192904	0.700591420192904\\
74	0.24306	0.922202564291915	0.922202564291915\\
74	0.24672	1.15752044817634	1.15752044817634\\
74	0.25038	1.40654507184611	1.40654507184611\\
74	0.25404	1.66927643530127	1.66927643530127\\
74	0.2577	1.94571453854184	1.94571453854184\\
74	0.26136	2.23585938156776	2.23585938156776\\
74	0.26502	2.53971096437905	2.53971096437905\\
74	0.26868	2.85726928697574	2.85726928697574\\
74	0.27234	3.18853434935778	3.18853434935778\\
74	0.276	3.53350615152524	3.53350615152524\\
};
\end{axis}

\begin{axis}[%
width=2.616cm,
height=2.517cm,
at={(6.484cm,3.497cm)},
scale only axis,
xmin=56,
xmax=74,
tick align=outside,
xlabel style={font=\color{white!15!black}},
xlabel={$L_{cut}$},
ymin=0.093,
ymax=0.276,
ylabel style={font=\color{white!15!black}},
ylabel={$D_{rlx}$},
zmin=-0.181133490981492,
zmax=6.27902675959869,
zlabel style={font=\color{white!15!black}},
zlabel={$x_3,x_3$},
view={-140}{50},
axis background/.style={fill=white},
xmajorgrids,
ymajorgrids,
zmajorgrids,
legend style={at={(1.03,1)}, anchor=north west, legend cell align=left, align=left, draw=white!15!black}
]
\addplot3[only marks, mark=*, mark options={}, mark size=1.5000pt, color=mycolor1, fill=mycolor1] table[row sep=crcr]{%
x	y	z\\
74	0.123	0.677389376112652\\
72	0.113	1.09189057212795\\
61	0.095	-0.050115452689465\\
56	0.093	-0.000890857505210356\\
};
\addlegendentry{data1}

\addplot3[only marks, mark=*, mark options={}, mark size=1.5000pt, color=mycolor2, fill=mycolor2] table[row sep=crcr]{%
x	y	z\\
67	0.276	3.89482573495266\\
66	0.255	2.44089515356944\\
62	0.209	1.5982493404421\\
57	0.193	1.21362679453753\\
};
\addlegendentry{data2}

\addplot3[only marks, mark=*, mark options={}, mark size=1.5000pt, color=black, fill=black] table[row sep=crcr]{%
x	y	z\\
69	0.104	0.733100432351303\\
};
\addlegendentry{data3}

\addplot3[only marks, mark=*, mark options={}, mark size=1.5000pt, color=black, fill=black] table[row sep=crcr]{%
x	y	z\\
64	0.23	1.89924145253115\\
};
\addlegendentry{data4}


\addplot3[%
surf,
fill opacity=0.7, shader=interp, colormap={mymap}{[1pt] rgb(0pt)=(1,0.905882,0); rgb(1pt)=(1,0.901964,0); rgb(2pt)=(1,0.898051,0); rgb(3pt)=(1,0.894144,0); rgb(4pt)=(1,0.890243,0); rgb(5pt)=(1,0.886349,0); rgb(6pt)=(1,0.88246,0); rgb(7pt)=(1,0.878577,0); rgb(8pt)=(1,0.8747,0); rgb(9pt)=(1,0.870829,0); rgb(10pt)=(1,0.866964,0); rgb(11pt)=(1,0.863106,0); rgb(12pt)=(1,0.859253,0); rgb(13pt)=(1,0.855406,0); rgb(14pt)=(1,0.851566,0); rgb(15pt)=(1,0.847732,0); rgb(16pt)=(1,0.843903,0); rgb(17pt)=(1,0.840081,0); rgb(18pt)=(1,0.836265,0); rgb(19pt)=(1,0.832455,0); rgb(20pt)=(1,0.828652,0); rgb(21pt)=(1,0.824854,0); rgb(22pt)=(1,0.821063,0); rgb(23pt)=(1,0.817278,0); rgb(24pt)=(1,0.8135,0); rgb(25pt)=(1,0.809727,0); rgb(26pt)=(1,0.805961,0); rgb(27pt)=(1,0.8022,0); rgb(28pt)=(1,0.798445,0); rgb(29pt)=(1,0.794696,0); rgb(30pt)=(1,0.790953,0); rgb(31pt)=(1,0.787215,0); rgb(32pt)=(1,0.783484,0); rgb(33pt)=(1,0.779758,0); rgb(34pt)=(1,0.776038,0); rgb(35pt)=(1,0.772324,0); rgb(36pt)=(1,0.768615,0); rgb(37pt)=(1,0.764913,0); rgb(38pt)=(1,0.761217,0); rgb(39pt)=(1,0.757527,0); rgb(40pt)=(1,0.753843,0); rgb(41pt)=(1,0.750165,0); rgb(42pt)=(1,0.746493,0); rgb(43pt)=(1,0.742827,0); rgb(44pt)=(1,0.739167,0); rgb(45pt)=(1,0.735514,0); rgb(46pt)=(1,0.731867,0); rgb(47pt)=(1,0.728226,0); rgb(48pt)=(1,0.724591,0); rgb(49pt)=(1,0.720963,0); rgb(50pt)=(1,0.717341,0); rgb(51pt)=(1,0.713725,0); rgb(52pt)=(0.999994,0.710077,0); rgb(53pt)=(0.999974,0.706363,0); rgb(54pt)=(0.999942,0.702592,0); rgb(55pt)=(0.999898,0.698775,0); rgb(56pt)=(0.999841,0.694921,0); rgb(57pt)=(0.999771,0.691039,0); rgb(58pt)=(0.99969,0.687139,0); rgb(59pt)=(0.999596,0.68323,0); rgb(60pt)=(0.99949,0.679323,0); rgb(61pt)=(0.999372,0.675427,0); rgb(62pt)=(0.999242,0.67155,0); rgb(63pt)=(0.9991,0.667704,0); rgb(64pt)=(0.998946,0.663897,0); rgb(65pt)=(0.998781,0.660138,0); rgb(66pt)=(0.998605,0.656439,0); rgb(67pt)=(0.998416,0.652807,0); rgb(68pt)=(0.998217,0.649253,0); rgb(69pt)=(0.998006,0.645786,0); rgb(70pt)=(0.997785,0.642416,0); rgb(71pt)=(0.997552,0.639152,0); rgb(72pt)=(0.997308,0.636004,0); rgb(73pt)=(0.997053,0.632982,0); rgb(74pt)=(0.996788,0.630095,0); rgb(75pt)=(0.996512,0.627352,0); rgb(76pt)=(0.996226,0.624763,0); rgb(77pt)=(0.995851,0.622329,0); rgb(78pt)=(0.99494,0.619997,0); rgb(79pt)=(0.99345,0.617753,0); rgb(80pt)=(0.991419,0.61559,0); rgb(81pt)=(0.988885,0.613503,0); rgb(82pt)=(0.985886,0.611486,0); rgb(83pt)=(0.98246,0.609532,0); rgb(84pt)=(0.978643,0.607636,0); rgb(85pt)=(0.974475,0.605791,0); rgb(86pt)=(0.969992,0.603992,0); rgb(87pt)=(0.965232,0.602233,0); rgb(88pt)=(0.960233,0.600507,0); rgb(89pt)=(0.955033,0.598808,0); rgb(90pt)=(0.949669,0.59713,0); rgb(91pt)=(0.94418,0.595468,0); rgb(92pt)=(0.938602,0.593815,0); rgb(93pt)=(0.932974,0.592166,0); rgb(94pt)=(0.927333,0.590513,0); rgb(95pt)=(0.921717,0.588852,0); rgb(96pt)=(0.916164,0.587176,0); rgb(97pt)=(0.910711,0.585479,0); rgb(98pt)=(0.905397,0.583755,0); rgb(99pt)=(0.900258,0.581999,0); rgb(100pt)=(0.895333,0.580203,0); rgb(101pt)=(0.890659,0.578362,0); rgb(102pt)=(0.886275,0.576471,0); rgb(103pt)=(0.882047,0.574545,0); rgb(104pt)=(0.877819,0.572608,0); rgb(105pt)=(0.873592,0.57066,0); rgb(106pt)=(0.869366,0.568701,0); rgb(107pt)=(0.865143,0.566733,0); rgb(108pt)=(0.860924,0.564756,0); rgb(109pt)=(0.856708,0.562771,0); rgb(110pt)=(0.852497,0.560778,0); rgb(111pt)=(0.848292,0.558779,0); rgb(112pt)=(0.844092,0.556774,0); rgb(113pt)=(0.8399,0.554763,0); rgb(114pt)=(0.835716,0.552749,0); rgb(115pt)=(0.831541,0.55073,0); rgb(116pt)=(0.827374,0.548709,0); rgb(117pt)=(0.823219,0.546686,0); rgb(118pt)=(0.819074,0.54466,0); rgb(119pt)=(0.81494,0.542635,0); rgb(120pt)=(0.81082,0.540609,0); rgb(121pt)=(0.806712,0.538584,0); rgb(122pt)=(0.802619,0.53656,0); rgb(123pt)=(0.798541,0.534539,0); rgb(124pt)=(0.794478,0.532521,0); rgb(125pt)=(0.790431,0.530506,0); rgb(126pt)=(0.786402,0.528496,0); rgb(127pt)=(0.782391,0.526491,0); rgb(128pt)=(0.77841,0.524489,0); rgb(129pt)=(0.774523,0.522478,0); rgb(130pt)=(0.770731,0.520455,0); rgb(131pt)=(0.767022,0.518424,0); rgb(132pt)=(0.763384,0.516385,0); rgb(133pt)=(0.759804,0.514339,0); rgb(134pt)=(0.756272,0.51229,0); rgb(135pt)=(0.752775,0.510237,0); rgb(136pt)=(0.749302,0.508182,0); rgb(137pt)=(0.74584,0.506128,0); rgb(138pt)=(0.742378,0.504075,0); rgb(139pt)=(0.738904,0.502025,0); rgb(140pt)=(0.735406,0.499979,0); rgb(141pt)=(0.731872,0.49794,0); rgb(142pt)=(0.72829,0.495909,0); rgb(143pt)=(0.724649,0.493887,0); rgb(144pt)=(0.720936,0.491875,0); rgb(145pt)=(0.71714,0.489876,0); rgb(146pt)=(0.713249,0.487891,0); rgb(147pt)=(0.709251,0.485921,0); rgb(148pt)=(0.705134,0.483968,0); rgb(149pt)=(0.700887,0.482033,0); rgb(150pt)=(0.696497,0.480118,0); rgb(151pt)=(0.691952,0.478225,0); rgb(152pt)=(0.687242,0.476355,0); rgb(153pt)=(0.682353,0.47451,0); rgb(154pt)=(0.677195,0.472696,0); rgb(155pt)=(0.6717,0.470916,0); rgb(156pt)=(0.665891,0.469169,0); rgb(157pt)=(0.659791,0.46745,0); rgb(158pt)=(0.653423,0.465756,0); rgb(159pt)=(0.64681,0.464084,0); rgb(160pt)=(0.639976,0.462432,0); rgb(161pt)=(0.632943,0.460795,0); rgb(162pt)=(0.625734,0.459171,0); rgb(163pt)=(0.618373,0.457556,0); rgb(164pt)=(0.610882,0.455948,0); rgb(165pt)=(0.603284,0.454343,0); rgb(166pt)=(0.595604,0.452737,0); rgb(167pt)=(0.587863,0.451129,0); rgb(168pt)=(0.580084,0.449514,0); rgb(169pt)=(0.572292,0.447889,0); rgb(170pt)=(0.564508,0.446252,0); rgb(171pt)=(0.556756,0.444599,0); rgb(172pt)=(0.549059,0.442927,0); rgb(173pt)=(0.54144,0.441232,0); rgb(174pt)=(0.533922,0.439512,0); rgb(175pt)=(0.526529,0.437764,0); rgb(176pt)=(0.519282,0.435983,0); rgb(177pt)=(0.512206,0.434168,0); rgb(178pt)=(0.505323,0.432315,0); rgb(179pt)=(0.498628,0.430422,3.92506e-06); rgb(180pt)=(0.491973,0.428504,3.49981e-05); rgb(181pt)=(0.485331,0.426562,9.63073e-05); rgb(182pt)=(0.478704,0.424596,0.000186979); rgb(183pt)=(0.472096,0.422609,0.000306141); rgb(184pt)=(0.465508,0.420599,0.00045292); rgb(185pt)=(0.458942,0.418567,0.000626441); rgb(186pt)=(0.452401,0.416515,0.000825833); rgb(187pt)=(0.445885,0.414441,0.00105022); rgb(188pt)=(0.439399,0.412348,0.00129873); rgb(189pt)=(0.432942,0.410234,0.00157049); rgb(190pt)=(0.426518,0.408102,0.00186463); rgb(191pt)=(0.420129,0.40595,0.00218028); rgb(192pt)=(0.413777,0.40378,0.00251655); rgb(193pt)=(0.407464,0.401592,0.00287258); rgb(194pt)=(0.401191,0.399386,0.00324749); rgb(195pt)=(0.394962,0.397164,0.00364042); rgb(196pt)=(0.388777,0.394925,0.00405048); rgb(197pt)=(0.38264,0.39267,0.00447681); rgb(198pt)=(0.376552,0.390399,0.00491852); rgb(199pt)=(0.370516,0.388113,0.00537476); rgb(200pt)=(0.364532,0.385812,0.00584464); rgb(201pt)=(0.358605,0.383497,0.00632729); rgb(202pt)=(0.352735,0.381168,0.00682184); rgb(203pt)=(0.346925,0.378826,0.00732741); rgb(204pt)=(0.341176,0.376471,0.00784314); rgb(205pt)=(0.335485,0.374093,0.00847245); rgb(206pt)=(0.329843,0.371682,0.00930909); rgb(207pt)=(0.324249,0.369242,0.0103377); rgb(208pt)=(0.318701,0.366772,0.0115428); rgb(209pt)=(0.313198,0.364275,0.0129091); rgb(210pt)=(0.307739,0.361753,0.0144211); rgb(211pt)=(0.302322,0.359206,0.0160634); rgb(212pt)=(0.296945,0.356637,0.0178207); rgb(213pt)=(0.291607,0.354048,0.0196776); rgb(214pt)=(0.286307,0.35144,0.0216186); rgb(215pt)=(0.281043,0.348814,0.0236284); rgb(216pt)=(0.275813,0.346172,0.0256916); rgb(217pt)=(0.270616,0.343517,0.0277927); rgb(218pt)=(0.265451,0.340849,0.0299163); rgb(219pt)=(0.260317,0.33817,0.0320472); rgb(220pt)=(0.25521,0.335482,0.0341698); rgb(221pt)=(0.250131,0.332786,0.0362688); rgb(222pt)=(0.245078,0.330085,0.0383287); rgb(223pt)=(0.240048,0.327379,0.0403343); rgb(224pt)=(0.235042,0.324671,0.04227); rgb(225pt)=(0.230056,0.321962,0.0441205); rgb(226pt)=(0.22509,0.319254,0.0458704); rgb(227pt)=(0.220142,0.316548,0.0475043); rgb(228pt)=(0.215212,0.313846,0.0490067); rgb(229pt)=(0.210296,0.311149,0.0503624); rgb(230pt)=(0.205395,0.308459,0.0515759); rgb(231pt)=(0.200514,0.305763,0.052757); rgb(232pt)=(0.195655,0.303061,0.0539242); rgb(233pt)=(0.190817,0.300353,0.0550763); rgb(234pt)=(0.186001,0.297639,0.0562123); rgb(235pt)=(0.181207,0.294918,0.0573313); rgb(236pt)=(0.176434,0.292191,0.0584321); rgb(237pt)=(0.171685,0.289458,0.0595136); rgb(238pt)=(0.166957,0.286719,0.060575); rgb(239pt)=(0.162252,0.283973,0.0616151); rgb(240pt)=(0.15757,0.281221,0.0626328); rgb(241pt)=(0.152911,0.278463,0.0636271); rgb(242pt)=(0.148275,0.275699,0.0645971); rgb(243pt)=(0.143663,0.272929,0.0655416); rgb(244pt)=(0.139074,0.270152,0.0664596); rgb(245pt)=(0.134508,0.26737,0.06735); rgb(246pt)=(0.129967,0.264581,0.0682118); rgb(247pt)=(0.125449,0.261787,0.0690441); rgb(248pt)=(0.120956,0.258986,0.0698456); rgb(249pt)=(0.116487,0.25618,0.0706154); rgb(250pt)=(0.112043,0.253367,0.0713525); rgb(251pt)=(0.107623,0.250549,0.0720557); rgb(252pt)=(0.103229,0.247724,0.0727241); rgb(253pt)=(0.0988592,0.244894,0.0733566); rgb(254pt)=(0.0945149,0.242058,0.0739522); rgb(255pt)=(0.0901961,0.239216,0.0745098)}, mesh/rows=49]
table[row sep=crcr, point meta=\thisrow{c}] {%
%
x	y	z	c\\
56	0.093	-0.122309648166462	-0.122309648166462\\
56	0.09666	-0.146324262491464	-0.146324262491464\\
56	0.10032	-0.164133107735636	-0.164133107735636\\
56	0.10398	-0.175736183898981	-0.175736183898981\\
56	0.10764	-0.181133490981492	-0.181133490981492\\
56	0.1113	-0.180325028983178	-0.180325028983178\\
56	0.11496	-0.173310797904038	-0.173310797904038\\
56	0.11862	-0.160090797744066	-0.160090797744066\\
56	0.12228	-0.140665028503266	-0.140665028503266\\
56	0.12594	-0.115033490181637	-0.115033490181637\\
56	0.1296	-0.0831961827791794	-0.0831961827791794\\
56	0.13326	-0.0451531062958921	-0.0451531062958921\\
56	0.13692	-0.000904260731779516	-0.000904260731779516\\
56	0.14058	0.0495503539131636	0.0495503539131636\\
56	0.14424	0.106210737638938	0.106210737638938\\
56	0.1479	0.16907689044554	0.16907689044554\\
56	0.15156	0.23814881233297	0.23814881233297\\
56	0.15522	0.313426503301228	0.313426503301228\\
56	0.15888	0.394909963350317	0.394909963350317\\
56	0.16254	0.482599192480232	0.482599192480232\\
56	0.1662	0.57649419069098	0.57649419069098\\
56	0.16986	0.676594957982553	0.676594957982553\\
56	0.17352	0.782901494354955	0.782901494354955\\
56	0.17718	0.895413799808185	0.895413799808185\\
56	0.18084	1.01413187434224	1.01413187434224\\
56	0.1845	1.13905571795713	1.13905571795713\\
56	0.18816	1.27018533065285	1.27018533065285\\
56	0.19182	1.4075207124294	1.4075207124294\\
56	0.19548	1.55106186328677	1.55106186328677\\
56	0.19914	1.70080878322498	1.70080878322498\\
56	0.2028	1.85676147224401	1.85676147224401\\
56	0.20646	2.01891993034387	2.01891993034387\\
56	0.21012	2.18728415752456	2.18728415752456\\
56	0.21378	2.36185415378607	2.36185415378607\\
56	0.21744	2.54262991912843	2.54262991912843\\
56	0.2211	2.7296114535516	2.7296114535516\\
56	0.22476	2.92279875705561	2.92279875705561\\
56	0.22842	3.12219182964044	3.12219182964044\\
56	0.23208	3.3277906713061	3.3277906713061\\
56	0.23574	3.53959528205259	3.53959528205259\\
56	0.2394	3.75760566187992	3.75760566187992\\
56	0.24306	3.98182181078806	3.98182181078806\\
56	0.24672	4.21224372877704	4.21224372877704\\
56	0.25038	4.44887141584685	4.44887141584685\\
56	0.25404	4.69170487199748	4.69170487199748\\
56	0.2577	4.94074409722895	4.94074409722895\\
56	0.26136	5.19598909154124	5.19598909154124\\
56	0.26502	5.45743985493436	5.45743985493436\\
56	0.26868	5.72509638740831	5.72509638740831\\
56	0.27234	5.99895868896308	5.99895868896308\\
56	0.276	6.27902675959869	6.27902675959869\\
56.375	0.093	-0.10527582723688	-0.10527582723688\\
56.375	0.09666	-0.131649771672584	-0.131649771672584\\
56.375	0.10032	-0.151817947027459	-0.151817947027459\\
56.375	0.10398	-0.165780353301506	-0.165780353301506\\
56.375	0.10764	-0.17353699049472	-0.17353699049472\\
56.375	0.1113	-0.17508785860711	-0.17508785860711\\
56.375	0.11496	-0.170432957638668	-0.170432957638668\\
56.375	0.11862	-0.159572287589397	-0.159572287589397\\
56.375	0.12228	-0.142505848459301	-0.142505848459301\\
56.375	0.12594	-0.119233640248372	-0.119233640248372\\
56.375	0.1296	-0.0897556629566161	-0.0897556629566161\\
56.375	0.13326	-0.054071916584034	-0.054071916584034\\
56.375	0.13692	-0.0121824011306195	-0.0121824011306195\\
56.375	0.14058	0.035912883403622	0.035912883403622\\
56.375	0.14424	0.0902139370186932	0.0902139370186932\\
56.375	0.1479	0.150720759714593	0.150720759714593\\
56.375	0.15156	0.217433351491321	0.217433351491321\\
56.375	0.15522	0.29035171234888	0.29035171234888\\
56.375	0.15888	0.369475842287265	0.369475842287265\\
56.375	0.16254	0.454805741306481	0.454805741306481\\
56.375	0.1662	0.546341409406521	0.546341409406521\\
56.375	0.16986	0.644082846587395	0.644082846587395\\
56.375	0.17352	0.748030052849097	0.748030052849097\\
56.375	0.17718	0.858183028191627	0.858183028191627\\
56.375	0.18084	0.974541772614986	0.974541772614986\\
56.375	0.1845	1.09710628611917	1.09710628611917\\
56.375	0.18816	1.22587656870419	1.22587656870419\\
56.375	0.19182	1.36085262037003	1.36085262037003\\
56.375	0.19548	1.5020344411167	1.5020344411167\\
56.375	0.19914	1.64942203094421	1.64942203094421\\
56.375	0.2028	1.80301538985254	1.80301538985254\\
56.375	0.20646	1.9628145178417	1.9628145178417\\
56.375	0.21012	2.12881941491168	2.12881941491168\\
56.375	0.21378	2.3010300810625	2.3010300810625\\
56.375	0.21744	2.47944651629414	2.47944651629414\\
56.375	0.2211	2.66406872060662	2.66406872060662\\
56.375	0.22476	2.85489669399992	2.85489669399992\\
56.375	0.22842	3.05193043647406	3.05193043647406\\
56.375	0.23208	3.25516994802901	3.25516994802901\\
56.375	0.23574	3.4646152286648	3.4646152286648\\
56.375	0.2394	3.68026627838142	3.68026627838142\\
56.375	0.24306	3.90212309717887	3.90212309717887\\
56.375	0.24672	4.13018568505715	4.13018568505715\\
56.375	0.25038	4.36445404201625	4.36445404201625\\
56.375	0.25404	4.60492816805618	4.60492816805618\\
56.375	0.2577	4.85160806317695	4.85160806317695\\
56.375	0.26136	5.10449372737853	5.10449372737853\\
56.375	0.26502	5.36358516066096	5.36358516066096\\
56.375	0.26868	5.6288823630242	5.6288823630242\\
56.375	0.27234	5.90038533446828	5.90038533446828\\
56.375	0.276	6.17809407499318	6.17809407499318\\
56.75	0.093	-0.08727518413049	-0.08727518413049\\
56.75	0.09666	-0.116008458676893	-0.116008458676893\\
56.75	0.10032	-0.138535964142469	-0.138535964142469\\
56.75	0.10398	-0.154857700527219	-0.154857700527219\\
56.75	0.10764	-0.164973667831133	-0.164973667831133\\
56.75	0.1113	-0.168883866054227	-0.168883866054227\\
56.75	0.11496	-0.166588295196486	-0.166588295196486\\
56.75	0.11862	-0.158086955257917	-0.158086955257917\\
56.75	0.12228	-0.143379846238521	-0.143379846238521\\
56.75	0.12594	-0.122466968138295	-0.122466968138295\\
56.75	0.1296	-0.0953483209572425	-0.0953483209572425\\
56.75	0.13326	-0.0620239046953603	-0.0620239046953603\\
56.75	0.13692	-0.0224937193526493	-0.0224937193526493\\
56.75	0.14058	0.0232422350708923	0.0232422350708923\\
56.75	0.14424	0.0751839585752618	0.0751839585752618\\
56.75	0.1479	0.133331451160458	0.133331451160458\\
56.75	0.15156	0.197684712826485	0.197684712826485\\
56.75	0.15522	0.26824374357334	0.26824374357334\\
56.75	0.15888	0.345008543401025	0.345008543401025\\
56.75	0.16254	0.427979112309537	0.427979112309537\\
56.75	0.1662	0.517155450298878	0.517155450298878\\
56.75	0.16986	0.612537557369052	0.612537557369052\\
56.75	0.17352	0.71412543352005	0.71412543352005\\
56.75	0.17718	0.821919078751874	0.821919078751874\\
56.75	0.18084	0.935918493064532	0.935918493064532\\
56.75	0.1845	1.05612367645802	1.05612367645802\\
56.75	0.18816	1.18253462893233	1.18253462893233\\
56.75	0.19182	1.31515135048748	1.31515135048748\\
56.75	0.19548	1.45397384112345	1.45397384112345\\
56.75	0.19914	1.59900210084025	1.59900210084025\\
56.75	0.2028	1.75023612963788	1.75023612963788\\
56.75	0.20646	1.90767592751634	1.90767592751634\\
56.75	0.21012	2.07132149447562	2.07132149447562\\
56.75	0.21378	2.24117283051573	2.24117283051573\\
56.75	0.21744	2.41722993563668	2.41722993563668\\
56.75	0.2211	2.59949280983845	2.59949280983845\\
56.75	0.22476	2.78796145312105	2.78796145312105\\
56.75	0.22842	2.98263586548448	2.98263586548448\\
56.75	0.23208	3.18351604692874	3.18351604692874\\
56.75	0.23574	3.39060199745383	3.39060199745383\\
56.75	0.2394	3.60389371705975	3.60389371705975\\
56.75	0.24306	3.82339120574649	3.82339120574649\\
56.75	0.24672	4.04909446351406	4.04909446351406\\
56.75	0.25038	4.28100349036247	4.28100349036247\\
56.75	0.25404	4.51911828629169	4.51911828629169\\
56.75	0.2577	4.76343885130176	4.76343885130176\\
56.75	0.26136	5.01396518539264	5.01396518539264\\
56.75	0.26502	5.27069728856437	5.27069728856437\\
56.75	0.26868	5.53363516081691	5.53363516081691\\
56.75	0.27234	5.80277880215028	5.80277880215028\\
56.75	0.276	6.07812821256449	6.07812821256449\\
57.125	0.093	-0.0683077188472843	-0.0683077188472843\\
57.125	0.09666	-0.0994003235043928	-0.0994003235043928\\
57.125	0.10032	-0.12428715908067	-0.12428715908067\\
57.125	0.10398	-0.14296822557612	-0.14296822557612\\
57.125	0.10764	-0.155443522990738	-0.155443522990738\\
57.125	0.1113	-0.161713051324531	-0.161713051324531\\
57.125	0.11496	-0.161776810577494	-0.161776810577494\\
57.125	0.11862	-0.155634800749628	-0.155634800749628\\
57.125	0.12228	-0.143287021840932	-0.143287021840932\\
57.125	0.12594	-0.12473347385141	-0.12473347385141\\
57.125	0.1296	-0.099974156781057	-0.099974156781057\\
57.125	0.13326	-0.0690090706298783	-0.0690090706298783\\
57.125	0.13692	-0.0318382153978671	-0.0318382153978671\\
57.125	0.14058	0.011538408914971	0.011538408914971\\
57.125	0.14424	0.0611208023086389	0.0611208023086389\\
57.125	0.1479	0.116908964783137	0.116908964783137\\
57.125	0.15156	0.17890289633846	0.17890289633846\\
57.125	0.15522	0.247102596974612	0.247102596974612\\
57.125	0.15888	0.321508066691594	0.321508066691594\\
57.125	0.16254	0.402119305489406	0.402119305489406\\
57.125	0.1662	0.488936313368047	0.488936313368047\\
57.125	0.16986	0.581959090327514	0.581959090327514\\
57.125	0.17352	0.681187636367812	0.681187636367812\\
57.125	0.17718	0.786621951488939	0.786621951488939\\
57.125	0.18084	0.898262035690891	0.898262035690891\\
57.125	0.1845	1.01610788897367	1.01610788897367\\
57.125	0.18816	1.14015951133728	1.14015951133728\\
57.125	0.19182	1.27041690278173	1.27041690278173\\
57.125	0.19548	1.406880063307	1.406880063307\\
57.125	0.19914	1.5495489929131	1.5495489929131\\
57.125	0.2028	1.69842369160002	1.69842369160002\\
57.125	0.20646	1.85350415936778	1.85350415936778\\
57.125	0.21012	2.01479039621636	2.01479039621636\\
57.125	0.21378	2.18228240214578	2.18228240214578\\
57.125	0.21744	2.35598017715601	2.35598017715601\\
57.125	0.2211	2.53588372124709	2.53588372124709\\
57.125	0.22476	2.72199303441899	2.72199303441899\\
57.125	0.22842	2.91430811667172	2.91430811667172\\
57.125	0.23208	3.11282896800527	3.11282896800527\\
57.125	0.23574	3.31755558841966	3.31755558841966\\
57.125	0.2394	3.52848797791488	3.52848797791488\\
57.125	0.24306	3.74562613649092	3.74562613649092\\
57.125	0.24672	3.96897006414779	3.96897006414779\\
57.125	0.25038	4.19851976088549	4.19851976088549\\
57.125	0.25404	4.43427522670402	4.43427522670402\\
57.125	0.2577	4.67623646160338	4.67623646160338\\
57.125	0.26136	4.92440346558357	4.92440346558357\\
57.125	0.26502	5.17877623864458	5.17877623864458\\
57.125	0.26868	5.43935478078642	5.43935478078642\\
57.125	0.27234	5.7061390920091	5.7061390920091\\
57.125	0.276	5.9791291723126	5.9791291723126\\
57.5	0.093	-0.0483734313872701	-0.0483734313872701\\
57.5	0.09666	-0.0818253661550785	-0.0818253661550785\\
57.5	0.10032	-0.109071531842058	-0.109071531842058\\
57.5	0.10398	-0.130111928448209	-0.130111928448209\\
57.5	0.10764	-0.14494655597353	-0.14494655597353\\
57.5	0.1113	-0.153575414418023	-0.153575414418023\\
57.5	0.11496	-0.155998503781688	-0.155998503781688\\
57.5	0.11862	-0.152215824064524	-0.152215824064524\\
57.5	0.12228	-0.142227375266531	-0.142227375266531\\
57.5	0.12594	-0.126033157387709	-0.126033157387709\\
57.5	0.1296	-0.103633170428058	-0.103633170428058\\
57.5	0.13326	-0.0750274143875806	-0.0750274143875806\\
57.5	0.13692	-0.040215889266273	-0.040215889266273\\
57.5	0.14058	0.000801404935865335	0.000801404935865335\\
57.5	0.14424	0.0480244682188316	0.0480244682188316\\
57.5	0.1479	0.101453300582625	0.101453300582625\\
57.5	0.15156	0.161087902027248	0.161087902027248\\
57.5	0.15522	0.2269282725527	0.2269282725527\\
57.5	0.15888	0.298974412158982	0.298974412158982\\
57.5	0.16254	0.37722632084609	0.37722632084609\\
57.5	0.1662	0.461683998614028	0.461683998614028\\
57.5	0.16986	0.552347445462795	0.552347445462795\\
57.5	0.17352	0.64921666139239	0.64921666139239\\
57.5	0.17718	0.75229164640281	0.75229164640281\\
57.5	0.18084	0.861572400494069	0.861572400494069\\
57.5	0.1845	0.977058923666148	0.977058923666148\\
57.5	0.18816	1.09875121591906	1.09875121591906\\
57.5	0.19182	1.2266492772528	1.2266492772528\\
57.5	0.19548	1.36075310766737	1.36075310766737\\
57.5	0.19914	1.50106270716276	1.50106270716276\\
57.5	0.2028	1.64757807573899	1.64757807573899\\
57.5	0.20646	1.80029921339604	1.80029921339604\\
57.5	0.21012	1.95922612013392	1.95922612013392\\
57.5	0.21378	2.12435879595263	2.12435879595263\\
57.5	0.21744	2.29569724085217	2.29569724085217\\
57.5	0.2211	2.47324145483254	2.47324145483254\\
57.5	0.22476	2.65699143789374	2.65699143789374\\
57.5	0.22842	2.84694719003577	2.84694719003577\\
57.5	0.23208	3.04310871125862	3.04310871125862\\
57.5	0.23574	3.24547600156231	3.24547600156231\\
57.5	0.2394	3.45404906094682	3.45404906094682\\
57.5	0.24306	3.66882788941216	3.66882788941216\\
57.5	0.24672	3.88981248695833	3.88981248695833\\
57.5	0.25038	4.11700285358533	4.11700285358533\\
57.5	0.25404	4.35039898929316	4.35039898929316\\
57.5	0.2577	4.59000089408182	4.59000089408182\\
57.5	0.26136	4.8358085679513	4.8358085679513\\
57.5	0.26502	5.08782201090161	5.08782201090161\\
57.5	0.26868	5.34604122293275	5.34604122293275\\
57.5	0.27234	5.61046620404473	5.61046620404473\\
57.5	0.276	5.88109695423753	5.88109695423753\\
57.875	0.093	-0.0274723217504405	-0.0274723217504405\\
57.875	0.09666	-0.0632835866289505	-0.0632835866289505\\
57.875	0.10032	-0.0928890824266331	-0.0928890824266331\\
57.875	0.10398	-0.116288809143486	-0.116288809143486\\
57.875	0.10764	-0.133482766779507	-0.133482766779507\\
57.875	0.1113	-0.144470955334704	-0.144470955334704\\
57.875	0.11496	-0.14925337480907	-0.14925337480907\\
57.875	0.11862	-0.147830025202608	-0.147830025202608\\
57.875	0.12228	-0.140200906515314	-0.140200906515314\\
57.875	0.12594	-0.126366018747196	-0.126366018747196\\
57.875	0.1296	-0.106325361898248	-0.106325361898248\\
57.875	0.13326	-0.080078935968471	-0.080078935968471\\
57.875	0.13692	-0.047626740957865	-0.047626740957865\\
57.875	0.14058	-0.00896877686642839	-0.00896877686642839\\
57.875	0.14424	0.0358949563058344	0.0358949563058344\\
57.875	0.1479	0.0869644585589242	0.0869644585589242\\
57.875	0.15156	0.144239729892847	0.144239729892847\\
57.875	0.15522	0.207720770307599	0.207720770307599\\
57.875	0.15888	0.277407579803178	0.277407579803178\\
57.875	0.16254	0.353300158379583	0.353300158379583\\
57.875	0.1662	0.435398506036821	0.435398506036821\\
57.875	0.16986	0.523702622774884	0.523702622774884\\
57.875	0.17352	0.618212508593779	0.618212508593779\\
57.875	0.17718	0.7189281634935	0.7189281634935\\
57.875	0.18084	0.825849587474051	0.825849587474051\\
57.875	0.1845	0.93897678053543	0.93897678053543\\
57.875	0.18816	1.05830974267764	1.05830974267764\\
57.875	0.19182	1.18384847390068	1.18384847390068\\
57.875	0.19548	1.31559297420454	1.31559297420454\\
57.875	0.19914	1.45354324358924	1.45354324358924\\
57.875	0.2028	1.59769928205476	1.59769928205476\\
57.875	0.20646	1.74806108960112	1.74806108960112\\
57.875	0.21012	1.90462866622829	1.90462866622829\\
57.875	0.21378	2.0674020119363	2.0674020119363\\
57.875	0.21744	2.23638112672514	2.23638112672514\\
57.875	0.2211	2.41156601059481	2.41156601059481\\
57.875	0.22476	2.59295666354531	2.59295666354531\\
57.875	0.22842	2.78055308557663	2.78055308557663\\
57.875	0.23208	2.97435527668878	2.97435527668878\\
57.875	0.23574	3.17436323688177	3.17436323688177\\
57.875	0.2394	3.38057696615558	3.38057696615558\\
57.875	0.24306	3.59299646451022	3.59299646451022\\
57.875	0.24672	3.81162173194569	3.81162173194569\\
57.875	0.25038	4.03645276846198	4.03645276846198\\
57.875	0.25404	4.26748957405911	4.26748957405911\\
57.875	0.2577	4.50473214873707	4.50473214873707\\
57.875	0.26136	4.74818049249585	4.74818049249585\\
57.875	0.26502	4.99783460533546	4.99783460533546\\
57.875	0.26868	5.2536944872559	5.2536944872559\\
57.875	0.27234	5.51576013825717	5.51576013825717\\
57.875	0.276	5.78403155833927	5.78403155833927\\
58.25	0.093	-0.00560438993680057	-0.00560438993680057\\
58.25	0.09666	-0.0437749849260123	-0.0437749849260123\\
58.25	0.10032	-0.0757398108343947	-0.0757398108343947\\
58.25	0.10398	-0.101498867661949	-0.101498867661949\\
58.25	0.10764	-0.121052155408674	-0.121052155408674\\
58.25	0.1113	-0.134399674074571	-0.134399674074571\\
58.25	0.11496	-0.14154142365964	-0.14154142365964\\
58.25	0.11862	-0.14247740416388	-0.14247740416388\\
58.25	0.12228	-0.137207615587291	-0.137207615587291\\
58.25	0.12594	-0.125732057929871	-0.125732057929871\\
58.25	0.1296	-0.108050731191625	-0.108050731191625\\
58.25	0.13326	-0.0841636353725512	-0.0841636353725512\\
58.25	0.13692	-0.0540707704726469	-0.0540707704726469\\
58.25	0.14058	-0.0177721364919119	-0.0177721364919119\\
58.25	0.14424	0.0247322665696474	0.0247322665696474\\
58.25	0.1479	0.0734424387120409	0.0734424387120409\\
58.25	0.15156	0.128358379935261	0.128358379935261\\
58.25	0.15522	0.189480090239309	0.189480090239309\\
58.25	0.15888	0.256807569624184	0.256807569624184\\
58.25	0.16254	0.33034081808989	0.33034081808989\\
58.25	0.1662	0.410079835636424	0.410079835636424\\
58.25	0.16986	0.496024622263787	0.496024622263787\\
58.25	0.17352	0.588175177971979	0.588175177971979\\
58.25	0.17718	0.686531502760996	0.686531502760996\\
58.25	0.18084	0.791093596630848	0.791093596630848\\
58.25	0.1845	0.901861459581527	0.901861459581527\\
58.25	0.18816	1.01883509161303	1.01883509161303\\
58.25	0.19182	1.14201449272537	1.14201449272537\\
58.25	0.19548	1.27139966291853	1.27139966291853\\
58.25	0.19914	1.40699060219253	1.40699060219253\\
58.25	0.2028	1.54878731054735	1.54878731054735\\
58.25	0.20646	1.696789787983	1.696789787983\\
58.25	0.21012	1.85099803449947	1.85099803449947\\
58.25	0.21378	2.01141205009679	2.01141205009679\\
58.25	0.21744	2.17803183477492	2.17803183477492\\
58.25	0.2211	2.35085738853388	2.35085738853388\\
58.25	0.22476	2.52988871137368	2.52988871137368\\
58.25	0.22842	2.71512580329431	2.71512580329431\\
58.25	0.23208	2.90656866429575	2.90656866429575\\
58.25	0.23574	3.10421729437804	3.10421729437804\\
58.25	0.2394	3.30807169354115	3.30807169354115\\
58.25	0.24306	3.51813186178508	3.51813186178508\\
58.25	0.24672	3.73439779910985	3.73439779910985\\
58.25	0.25038	3.95686950551545	3.95686950551545\\
58.25	0.25404	4.18554698100187	4.18554698100187\\
58.25	0.2577	4.42043022556912	4.42043022556912\\
58.25	0.26136	4.6615192392172	4.6615192392172\\
58.25	0.26502	4.90881402194611	4.90881402194611\\
58.25	0.26868	5.16231457375585	5.16231457375585\\
58.25	0.27234	5.42202089464642	5.42202089464642\\
58.25	0.276	5.68793298461782	5.68793298461782\\
58.625	0.093	0.0172303640536549	0.0172303640536549\\
58.625	0.09666	-0.0232995610462603	-0.0232995610462603\\
58.625	0.10032	-0.0576237170653444	-0.0576237170653444\\
58.625	0.10398	-0.0857421040036006	-0.0857421040036006\\
58.625	0.10764	-0.107654721861029	-0.107654721861029\\
58.625	0.1113	-0.123361570637625	-0.123361570637625\\
58.625	0.11496	-0.132862650333396	-0.132862650333396\\
58.625	0.11862	-0.136157960948338	-0.136157960948338\\
58.625	0.12228	-0.133247502482449	-0.133247502482449\\
58.625	0.12594	-0.124131274935734	-0.124131274935734\\
58.625	0.1296	-0.108809278308188	-0.108809278308188\\
58.625	0.13326	-0.0872815125998159	-0.0872815125998159\\
58.625	0.13692	-0.0595479778106149	-0.0595479778106149\\
58.625	0.14058	-0.0256086739405834	-0.0256086739405834\\
58.625	0.14424	0.0145363990102796	0.0145363990102796\\
58.625	0.1479	0.0608872410419696	0.0608872410419696\\
58.625	0.15156	0.113443852154486	0.113443852154486\\
58.625	0.15522	0.172206232347831	0.172206232347831\\
58.625	0.15888	0.237174381622006	0.237174381622006\\
58.625	0.16254	0.308348299977008	0.308348299977008\\
58.625	0.1662	0.385727987412842	0.385727987412842\\
58.625	0.16986	0.469313443929506	0.469313443929506\\
58.625	0.17352	0.559104669526994	0.559104669526994\\
58.625	0.17718	0.655101664205311	0.655101664205311\\
58.625	0.18084	0.757304427964456	0.757304427964456\\
58.625	0.1845	0.865712960804435	0.865712960804435\\
58.625	0.18816	0.980327262725243	0.980327262725243\\
58.625	0.19182	1.10114733372687	1.10114733372687\\
58.625	0.19548	1.22817317380934	1.22817317380934\\
58.625	0.19914	1.36140478297263	1.36140478297263\\
58.625	0.2028	1.50084216121675	1.50084216121675\\
58.625	0.20646	1.6464853085417	1.6464853085417\\
58.625	0.21012	1.79833422494748	1.79833422494748\\
58.625	0.21378	1.95638891043408	1.95638891043408\\
58.625	0.21744	2.12064936500151	2.12064936500151\\
58.625	0.2211	2.29111558864978	2.29111558864978\\
58.625	0.22476	2.46778758137887	2.46778758137887\\
58.625	0.22842	2.65066534318879	2.65066534318879\\
58.625	0.23208	2.83974887407954	2.83974887407954\\
58.625	0.23574	3.03503817405112	3.03503817405112\\
58.625	0.2394	3.23653324310353	3.23653324310353\\
58.625	0.24306	3.44423408123676	3.44423408123676\\
58.625	0.24672	3.65814068845083	3.65814068845083\\
58.625	0.25038	3.87825306474573	3.87825306474573\\
58.625	0.25404	4.10457121012145	4.10457121012145\\
58.625	0.2577	4.337095124578	4.337095124578\\
58.625	0.26136	4.57582480811537	4.57582480811537\\
58.625	0.26502	4.82076026073359	4.82076026073359\\
58.625	0.26868	5.07190148243262	5.07190148243262\\
58.625	0.27234	5.32924847321249	5.32924847321249\\
58.625	0.276	5.59280123307318	5.59280123307318\\
59	0.093	0.0410319402209205	0.0410319402209205\\
59	0.09666	-0.00185731498969632	-0.00185731498969632\\
59	0.10032	-0.0385408011194839	-0.0385408011194839\\
59	0.10398	-0.0690185181684417	-0.0690185181684417\\
59	0.10764	-0.0932904661365699	-0.0932904661365699\\
59	0.1113	-0.11135664502387	-0.11135664502387\\
59	0.11496	-0.123217054830342	-0.123217054830342\\
59	0.11862	-0.128871695555985	-0.128871695555985\\
59	0.12228	-0.128320567200801	-0.128320567200801\\
59	0.12594	-0.121563669764785	-0.121563669764785\\
59	0.1296	-0.108601003247943	-0.108601003247943\\
59	0.13326	-0.0894325676502703	-0.0894325676502703\\
59	0.13692	-0.0640583629717693	-0.0640583629717693\\
59	0.14058	-0.0324783892124412	-0.0324783892124412\\
59	0.14424	0.00530735362771839	0.00530735362771839\\
59	0.1479	0.049298865548705	0.049298865548705\\
59	0.15156	0.0994961465505213	0.0994961465505213\\
59	0.15522	0.155899196633163	0.155899196633163\\
59	0.15888	0.218508015796639	0.218508015796639\\
59	0.16254	0.28732260404094	0.28732260404094\\
59	0.1662	0.362342961366071	0.362342961366071\\
59	0.16986	0.443569087772032	0.443569087772032\\
59	0.17352	0.53100098325882	0.53100098325882\\
59	0.17718	0.624638647826437	0.624638647826437\\
59	0.18084	0.724482081474882	0.724482081474882\\
59	0.1845	0.830531284204154	0.830531284204154\\
59	0.18816	0.942786256014262	0.942786256014262\\
59	0.19182	1.06124699690519	1.06124699690519\\
59	0.19548	1.18591350687695	1.18591350687695\\
59	0.19914	1.31678578592954	1.31678578592954\\
59	0.2028	1.45386383406296	1.45386383406296\\
59	0.20646	1.59714765127721	1.59714765127721\\
59	0.21012	1.74663723757228	1.74663723757228\\
59	0.21378	1.90233259294819	1.90233259294819\\
59	0.21744	2.06423371740492	2.06423371740492\\
59	0.2211	2.23234061094248	2.23234061094248\\
59	0.22476	2.40665327356087	2.40665327356087\\
59	0.22842	2.58717170526009	2.58717170526009\\
59	0.23208	2.77389590604014	2.77389590604014\\
59	0.23574	2.96682587590102	2.96682587590102\\
59	0.2394	3.16596161484272	3.16596161484272\\
59	0.24306	3.37130312286525	3.37130312286525\\
59	0.24672	3.58285039996862	3.58285039996862\\
59	0.25038	3.80060344615281	3.80060344615281\\
59	0.25404	4.02456226141783	4.02456226141783\\
59	0.2577	4.25472684576368	4.25472684576368\\
59	0.26136	4.49109719919036	4.49109719919036\\
59	0.26502	4.73367332169787	4.73367332169787\\
59	0.26868	4.9824552132862	4.9824552132862\\
59	0.27234	5.23744287395537	5.23744287395537\\
59	0.276	5.49863630370536	5.49863630370536\\
59.375	0.093	0.0658003385649963	0.0658003385649963\\
59.375	0.09666	0.0205517532436779	0.0205517532436779\\
59.375	0.10032	-0.0184910629968114	-0.0184910629968114\\
59.375	0.10398	-0.0513281101564727	-0.0513281101564727\\
59.375	0.10764	-0.0779593882353007	-0.0779593882353007\\
59.375	0.1113	-0.0983848972333039	-0.0983848972333039\\
59.375	0.11496	-0.112604637150477	-0.112604637150477\\
59.375	0.11862	-0.120618607986821	-0.120618607986821\\
59.375	0.12228	-0.122426809742338	-0.122426809742338\\
59.375	0.12594	-0.118029242417025	-0.118029242417025\\
59.375	0.1296	-0.107425906010883	-0.107425906010883\\
59.375	0.13326	-0.0906168005239145	-0.0906168005239145\\
59.375	0.13692	-0.0676019259561169	-0.0676019259561169\\
59.375	0.14058	-0.0383812823074887	-0.0383812823074887\\
59.375	0.14424	-0.00295486957803259	-0.00295486957803259\\
59.375	0.1479	0.0386773122322541	0.0386773122322541\\
59.375	0.15156	0.086515263123367	0.086515263123367\\
59.375	0.15522	0.140558983095309	0.140558983095309\\
59.375	0.15888	0.200808472148081	0.200808472148081\\
59.375	0.16254	0.267263730281683	0.267263730281683\\
59.375	0.1662	0.33992475749611	0.33992475749611\\
59.375	0.16986	0.418791553791371	0.418791553791371\\
59.375	0.17352	0.503864119167456	0.503864119167456\\
59.375	0.17718	0.595142453624369	0.595142453624369\\
59.375	0.18084	0.692626557162114	0.692626557162114\\
59.375	0.1845	0.796316429780687	0.796316429780687\\
59.375	0.18816	0.906212071480088	0.906212071480088\\
59.375	0.19182	1.02231348226032	1.02231348226032\\
59.375	0.19548	1.14462066212138	1.14462066212138\\
59.375	0.19914	1.27313361106327	1.27313361106327\\
59.375	0.2028	1.40785232908598	1.40785232908598\\
59.375	0.20646	1.54877681618953	1.54877681618953\\
59.375	0.21012	1.6959070723739	1.6959070723739\\
59.375	0.21378	1.84924309763911	1.84924309763911\\
59.375	0.21744	2.00878489198513	2.00878489198513\\
59.375	0.2211	2.17453245541199	2.17453245541199\\
59.375	0.22476	2.34648578791968	2.34648578791968\\
59.375	0.22842	2.5246448895082	2.5246448895082\\
59.375	0.23208	2.70900976017755	2.70900976017755\\
59.375	0.23574	2.89958039992772	2.89958039992772\\
59.375	0.2394	3.09635680875873	3.09635680875873\\
59.375	0.24306	3.29933898667056	3.29933898667056\\
59.375	0.24672	3.50852693366322	3.50852693366322\\
59.375	0.25038	3.72392064973671	3.72392064973671\\
59.375	0.25404	3.94552013489103	3.94552013489103\\
59.375	0.2577	4.17332538912618	4.17332538912618\\
59.375	0.26136	4.40733641244215	4.40733641244215\\
59.375	0.26502	4.64755320483896	4.64755320483896\\
59.375	0.26868	4.89397576631659	4.89397576631659\\
59.375	0.27234	5.14660409687505	5.14660409687505\\
59.375	0.276	5.40543819651435	5.40543819651435\\
59.75	0.093	0.091535559085886	0.091535559085886\\
59.75	0.09666	0.0439276436538676	0.0439276436538676\\
59.75	0.10032	0.00252549730267848	0.00252549730267848\\
59.75	0.10398	-0.0326708799676827	-0.0326708799676827\\
59.75	0.10764	-0.0616614881572177	-0.0616614881572177\\
59.75	0.1113	-0.0844463272659208	-0.0844463272659208\\
59.75	0.11496	-0.101025397293795	-0.101025397293795\\
59.75	0.11862	-0.111398698240841	-0.111398698240841\\
59.75	0.12228	-0.115566230107058	-0.115566230107058\\
59.75	0.12594	-0.113527992892446	-0.113527992892446\\
59.75	0.1296	-0.10528398659701	-0.10528398659701\\
59.75	0.13326	-0.0908342112207414	-0.0908342112207414\\
59.75	0.13692	-0.0701786667636437	-0.0701786667636437\\
59.75	0.14058	-0.0433173532257189	-0.0433173532257189\\
59.75	0.14424	-0.0102502706069627	-0.0102502706069627\\
59.75	0.1479	0.0290225810926206	0.0290225810926206\\
59.75	0.15156	0.0745012018730336	0.0745012018730336\\
59.75	0.15522	0.126185591734275	0.126185591734275\\
59.75	0.15888	0.184075750676341	0.184075750676341\\
59.75	0.16254	0.248171678699239	0.248171678699239\\
59.75	0.1662	0.318473375802967	0.318473375802967\\
59.75	0.16986	0.394980841987524	0.394980841987524\\
59.75	0.17352	0.477694077252909	0.477694077252909\\
59.75	0.17718	0.566613081599122	0.566613081599122\\
59.75	0.18084	0.661737855026161	0.661737855026161\\
59.75	0.1845	0.763068397534033	0.763068397534033\\
59.75	0.18816	0.870604709122734	0.870604709122734\\
59.75	0.19182	0.984346789792264	0.984346789792264\\
59.75	0.19548	1.10429463954262	1.10429463954262\\
59.75	0.19914	1.23044825837381	1.23044825837381\\
59.75	0.2028	1.36280764628582	1.36280764628582\\
59.75	0.20646	1.50137280327867	1.50137280327867\\
59.75	0.21012	1.64614372935233	1.64614372935233\\
59.75	0.21378	1.79712042450684	1.79712042450684\\
59.75	0.21744	1.95430288874216	1.95430288874216\\
59.75	0.2211	2.11769112205832	2.11769112205832\\
59.75	0.22476	2.28728512445531	2.28728512445531\\
59.75	0.22842	2.46308489593313	2.46308489593313\\
59.75	0.23208	2.64509043649177	2.64509043649177\\
59.75	0.23574	2.83330174613125	2.83330174613125\\
59.75	0.2394	3.02771882485155	3.02771882485155\\
59.75	0.24306	3.22834167265268	3.22834167265268\\
59.75	0.24672	3.43517028953464	3.43517028953464\\
59.75	0.25038	3.64820467549742	3.64820467549742\\
59.75	0.25404	3.86744483054104	3.86744483054104\\
59.75	0.2577	4.09289075466549	4.09289075466549\\
59.75	0.26136	4.32454244787076	4.32454244787076\\
59.75	0.26502	4.56239991015687	4.56239991015687\\
59.75	0.26868	4.8064631415238	4.8064631415238\\
59.75	0.27234	5.05673214197156	5.05673214197156\\
59.75	0.276	5.31320691150015	5.31320691150015\\
60.125	0.093	0.118237601783589	0.118237601783589\\
60.125	0.09666	0.0682703562408693	0.0682703562408693\\
60.125	0.10032	0.0245088797789768	0.0245088797789768\\
60.125	0.10398	-0.0130468276020861	-0.0130468276020861\\
60.125	0.10764	-0.044396765902321	-0.044396765902321\\
60.125	0.1113	-0.0695409351217275	-0.0695409351217275\\
60.125	0.11496	-0.0884793352603035	-0.0884793352603035\\
60.125	0.11862	-0.101211966318053	-0.101211966318053\\
60.125	0.12228	-0.107738828294972	-0.107738828294972\\
60.125	0.12594	-0.108059921191063	-0.108059921191063\\
60.125	0.1296	-0.102175245006327	-0.102175245006327\\
60.125	0.13326	-0.0900847997407581	-0.0900847997407581\\
60.125	0.13692	-0.0717885853943638	-0.0717885853943638\\
60.125	0.14058	-0.047286601967139	-0.047286601967139\\
60.125	0.14424	-0.0165788494590862	-0.0165788494590862\\
60.125	0.1479	0.0203346721297972	0.0203346721297972\\
60.125	0.15156	0.0634539627995068	0.0634539627995068\\
60.125	0.15522	0.112779022550045	0.112779022550045\\
60.125	0.15888	0.168309851381414	0.168309851381414\\
60.125	0.16254	0.230046449293609	0.230046449293609\\
60.125	0.1662	0.297988816286633	0.297988816286633\\
60.125	0.16986	0.37213695236049	0.37213695236049\\
60.125	0.17352	0.452490857515172	0.452490857515172\\
60.125	0.17718	0.539050531750686	0.539050531750686\\
60.125	0.18084	0.631815975067024	0.631815975067024\\
60.125	0.1845	0.730787187464193	0.730787187464193\\
60.125	0.18816	0.835964168942191	0.835964168942191\\
60.125	0.19182	0.947346919501017	0.947346919501017\\
60.125	0.19548	1.06493543914067	1.06493543914067\\
60.125	0.19914	1.18872972786116	1.18872972786116\\
60.125	0.2028	1.31872978566247	1.31872978566247\\
60.125	0.20646	1.45493561254461	1.45493561254461\\
60.125	0.21012	1.59734720850758	1.59734720850758\\
60.125	0.21378	1.74596457355138	1.74596457355138\\
60.125	0.21744	1.90078770767601	1.90078770767601\\
60.125	0.2211	2.06181661088146	2.06181661088146\\
60.125	0.22476	2.22905128316775	2.22905128316775\\
60.125	0.22842	2.40249172453487	2.40249172453487\\
60.125	0.23208	2.5821379349828	2.5821379349828\\
60.125	0.23574	2.76798991451158	2.76798991451158\\
60.125	0.2394	2.96004766312118	2.96004766312118\\
60.125	0.24306	3.15831118081161	3.15831118081161\\
60.125	0.24672	3.36278046758287	3.36278046758287\\
60.125	0.25038	3.57345552343495	3.57345552343495\\
60.125	0.25404	3.79033634836787	3.79033634836787\\
60.125	0.2577	4.01342294238161	4.01342294238161\\
60.125	0.26136	4.24271530547618	4.24271530547618\\
60.125	0.26502	4.47821343765158	4.47821343765158\\
60.125	0.26868	4.71991733890781	4.71991733890781\\
60.125	0.27234	4.96782700924487	4.96782700924487\\
60.125	0.276	5.22194244866276	5.22194244866276\\
60.5	0.093	0.145906466658105	0.145906466658105\\
60.5	0.09666	0.0935798910046812	0.0935798910046812\\
60.5	0.10032	0.047459084432087	0.047459084432087\\
60.5	0.10398	0.00754404694032251	0.00754404694032251\\
60.5	0.10764	-0.0261652214706123	-0.0261652214706123\\
60.5	0.1113	-0.0536687208007223	-0.0536687208007223\\
60.5	0.11496	-0.0749664510500017	-0.0749664510500017\\
60.5	0.11862	-0.0900584122184513	-0.0900584122184513\\
60.5	0.12228	-0.0989446043060731	-0.0989446043060731\\
60.5	0.12594	-0.101625027312865	-0.101625027312865\\
60.5	0.1296	-0.0980996812388284	-0.0980996812388284\\
60.5	0.13326	-0.0883685660839628	-0.0883685660839628\\
60.5	0.13692	-0.072431681848272	-0.072431681848272\\
60.5	0.14058	-0.050289028531747	-0.050289028531747\\
60.5	0.14424	-0.0219406061343941	-0.0219406061343941\\
60.5	0.1479	0.0126135853437823	0.0126135853437823\\
60.5	0.15156	0.053373545902792	0.053373545902792\\
60.5	0.15522	0.10033927554263	0.10033927554263\\
60.5	0.15888	0.153510774263296	0.153510774263296\\
60.5	0.16254	0.212888042064791	0.212888042064791\\
60.5	0.1662	0.278471078947112	0.278471078947112\\
60.5	0.16986	0.350259884910265	0.350259884910265\\
60.5	0.17352	0.428254459954247	0.428254459954247\\
60.5	0.17718	0.512454804079054	0.512454804079054\\
60.5	0.18084	0.602860917284692	0.602860917284692\\
60.5	0.1845	0.699472799571161	0.699472799571161\\
60.5	0.18816	0.802290450938459	0.802290450938459\\
60.5	0.19182	0.911313871386582	0.911313871386582\\
60.5	0.19548	1.02654306091554	1.02654306091554\\
60.5	0.19914	1.14797801952532	1.14797801952532\\
60.5	0.2028	1.27561874721593	1.27561874721593\\
60.5	0.20646	1.40946524398737	1.40946524398737\\
60.5	0.21012	1.54951750983964	1.54951750983964\\
60.5	0.21378	1.69577554477274	1.69577554477274\\
60.5	0.21744	1.84823934878666	1.84823934878666\\
60.5	0.2211	2.00690892188141	2.00690892188141\\
60.5	0.22476	2.171784264057	2.171784264057\\
60.5	0.22842	2.34286537531341	2.34286537531341\\
60.5	0.23208	2.52015225565065	2.52015225565065\\
60.5	0.23574	2.70364490506872	2.70364490506872\\
60.5	0.2394	2.89334332356762	2.89334332356762\\
60.5	0.24306	3.08924751114734	3.08924751114734\\
60.5	0.24672	3.2913574678079	3.2913574678079\\
60.5	0.25038	3.49967319354929	3.49967319354929\\
60.5	0.25404	3.7141946883715	3.7141946883715\\
60.5	0.2577	3.93492195227455	3.93492195227455\\
60.5	0.26136	4.16185498525841	4.16185498525841\\
60.5	0.26502	4.39499378732311	4.39499378732311\\
60.5	0.26868	4.63433835846864	4.63433835846864\\
60.5	0.27234	4.879888698695	4.879888698695\\
60.5	0.276	5.13164480800218	5.13164480800218\\
60.875	0.093	0.17454215370943	0.17454215370943\\
60.875	0.09666	0.119856247945305	0.119856247945305\\
60.875	0.10032	0.0713761112620093	0.0713761112620093\\
60.875	0.10398	0.0291017436595413	0.0291017436595413\\
60.875	0.10764	-0.00696685486209336	-0.00696685486209336\\
60.875	0.1113	-0.0368296843029032	-0.0368296843029032\\
60.875	0.11496	-0.0604867446628861	-0.0604867446628861\\
60.875	0.11862	-0.0779380359420374	-0.0779380359420374\\
60.875	0.12228	-0.0891835581403608	-0.0891835581403608\\
60.875	0.12594	-0.0942233112578523	-0.0942233112578523\\
60.875	0.1296	-0.093057295294523	-0.093057295294523\\
60.875	0.13326	-0.0856855102503573	-0.0856855102503573\\
60.875	0.13692	-0.0721079561253664	-0.0721079561253664\\
60.875	0.14058	-0.0523246329195448	-0.0523246329195448\\
60.875	0.14424	-0.0263355406328953	-0.0263355406328953\\
60.875	0.1479	0.00585932073458117	0.00585932073458117\\
60.875	0.15156	0.0442599511828874	0.0442599511828874\\
60.875	0.15522	0.088866350712026	0.088866350712026\\
60.875	0.15888	0.139678519321988	0.139678519321988\\
60.875	0.16254	0.19669645701278	0.19669645701278\\
60.875	0.1662	0.259920163784404	0.259920163784404\\
60.875	0.16986	0.329349639636854	0.329349639636854\\
60.875	0.17352	0.404984884570136	0.404984884570136\\
60.875	0.17718	0.486825898584243	0.486825898584243\\
60.875	0.18084	0.574872681679174	0.574872681679174\\
60.875	0.1845	0.669125233854944	0.669125233854944\\
60.875	0.18816	0.769583555111534	0.769583555111534\\
60.875	0.19182	0.876247645448961	0.876247645448961\\
60.875	0.19548	0.989117504867211	0.989117504867211\\
60.875	0.19914	1.10819313336629	1.10819313336629\\
60.875	0.2028	1.2334745309462	1.2334745309462\\
60.875	0.20646	1.36496169760694	1.36496169760694\\
60.875	0.21012	1.50265463334851	1.50265463334851\\
60.875	0.21378	1.6465533381709	1.6465533381709\\
60.875	0.21744	1.79665781207413	1.79665781207413\\
60.875	0.2211	1.95296805505818	1.95296805505818\\
60.875	0.22476	2.11548406712306	2.11548406712306\\
60.875	0.22842	2.28420584826877	2.28420584826877\\
60.875	0.23208	2.45913339849531	2.45913339849531\\
60.875	0.23574	2.64026671780268	2.64026671780268\\
60.875	0.2394	2.82760580619088	2.82760580619088\\
60.875	0.24306	3.0211506636599	3.0211506636599\\
60.875	0.24672	3.22090129020976	3.22090129020976\\
60.875	0.25038	3.42685768584044	3.42685768584044\\
60.875	0.25404	3.63901985055195	3.63901985055195\\
60.875	0.2577	3.85738778434429	3.85738778434429\\
60.875	0.26136	4.08196148721746	4.08196148721746\\
60.875	0.26502	4.31274095917146	4.31274095917146\\
60.875	0.26868	4.54972620020628	4.54972620020628\\
60.875	0.27234	4.79291721032193	4.79291721032193\\
60.875	0.276	5.04231398951842	5.04231398951842\\
61.25	0.093	0.204144662937571	0.204144662937571\\
61.25	0.09666	0.147099427062745	0.147099427062745\\
61.25	0.10032	0.0962599602687453	0.0962599602687453\\
61.25	0.10398	0.0516262625555757	0.0516262625555757\\
61.25	0.10764	0.0131983339232375	0.0131983339232375\\
61.25	0.1113	-0.0190238256282722	-0.0190238256282722\\
61.25	0.11496	-0.0450402160989567	-0.0450402160989567\\
61.25	0.11862	-0.0648508374888097	-0.0648508374888097\\
61.25	0.12228	-0.078455689797833	-0.078455689797833\\
61.25	0.12594	-0.0858547730260315	-0.0858547730260315\\
61.25	0.1296	-0.0870480871733985	-0.0870480871733985\\
61.25	0.13326	-0.0820356322399363	-0.0820356322399363\\
61.25	0.13692	-0.0708174082256487	-0.0708174082256487\\
61.25	0.14058	-0.0533934151305271	-0.0533934151305271\\
61.25	0.14424	-0.0297636529545811	-0.0297636529545811\\
61.25	0.1479	7.18783021955716e-05	7.18783021955716e-05\\
61.25	0.15156	0.0361131786398019	0.0361131786398019\\
61.25	0.15522	0.0783602480582335	0.0783602480582335\\
61.25	0.15888	0.126813086557496	0.126813086557496\\
61.25	0.16254	0.181471694137588	0.181471694137588\\
61.25	0.1662	0.242336070798508	0.242336070798508\\
61.25	0.16986	0.309406216540259	0.309406216540259\\
61.25	0.17352	0.382682131362837	0.382682131362837\\
61.25	0.17718	0.462163815266237	0.462163815266237\\
61.25	0.18084	0.547851268250476	0.547851268250476\\
61.25	0.1845	0.639744490315538	0.639744490315538\\
61.25	0.18816	0.737843481461429	0.737843481461429\\
61.25	0.19182	0.842148241688152	0.842148241688152\\
61.25	0.19548	0.952658770995702	0.952658770995702\\
61.25	0.19914	1.06937506938408	1.06937506938408\\
61.25	0.2028	1.19229713685329	1.19229713685329\\
61.25	0.20646	1.32142497340332	1.32142497340332\\
61.25	0.21012	1.45675857903419	1.45675857903419\\
61.25	0.21378	1.59829795374588	1.59829795374588\\
61.25	0.21744	1.74604309753841	1.74604309753841\\
61.25	0.2211	1.89999401041175	1.89999401041175\\
61.25	0.22476	2.06015069236594	2.06015069236594\\
61.25	0.22842	2.22651314340095	2.22651314340095\\
61.25	0.23208	2.39908136351678	2.39908136351678\\
61.25	0.23574	2.57785535271345	2.57785535271345\\
61.25	0.2394	2.76283511099094	2.76283511099094\\
61.25	0.24306	2.95402063834927	2.95402063834927\\
61.25	0.24672	3.15141193478842	3.15141193478842\\
61.25	0.25038	3.3550090003084	3.3550090003084\\
61.25	0.25404	3.56481183490921	3.56481183490921\\
61.25	0.2577	3.78082043859085	3.78082043859085\\
61.25	0.26136	4.00303481135331	4.00303481135331\\
61.25	0.26502	4.23145495319661	4.23145495319661\\
61.25	0.26868	4.46608086412073	4.46608086412073\\
61.25	0.27234	4.70691254412569	4.70691254412569\\
61.25	0.276	4.95394999321147	4.95394999321147\\
61.625	0.093	0.234713994342521	0.234713994342521\\
61.625	0.09666	0.175309428356994	0.175309428356994\\
61.625	0.10032	0.122110631452295	0.122110631452295\\
61.625	0.10398	0.075117603628422	0.075117603628422\\
61.625	0.10764	0.034330344885384	0.034330344885384\\
61.625	0.1113	-0.000251144776829193	-0.000251144776829193\\
61.625	0.11496	-0.0286268653582171	-0.0286268653582171\\
61.625	0.11862	-0.05079681685877	-0.05079681685877\\
61.625	0.12228	-0.0667609992784968	-0.0667609992784968\\
61.625	0.12594	-0.0765194126173951	-0.0765194126173951\\
61.625	0.1296	-0.080072056875462	-0.080072056875462\\
61.625	0.13326	-0.0774189320527068	-0.0774189320527068\\
61.625	0.13692	-0.0685600381491156	-0.0685600381491156\\
61.625	0.14058	-0.0534953751647009	-0.0534953751647009\\
61.625	0.14424	-0.0322249430994512	-0.0322249430994512\\
61.625	0.1479	-0.00474874195337804	-0.00474874195337804\\
61.625	0.15156	0.0289332282735213	0.0289332282735213\\
61.625	0.15522	0.0688209675812566	0.0688209675812566\\
61.625	0.15888	0.114914475969815	0.114914475969815\\
61.625	0.16254	0.167213753439207	0.167213753439207\\
61.625	0.1662	0.225718799989425	0.225718799989425\\
61.625	0.16986	0.290429615620472	0.290429615620472\\
61.625	0.17352	0.361346200332346	0.361346200332346\\
61.625	0.17718	0.43846855412505	0.43846855412505\\
61.625	0.18084	0.521796676998582	0.521796676998582\\
61.625	0.1845	0.611330568952944	0.611330568952944\\
61.625	0.18816	0.707070229988135	0.707070229988135\\
61.625	0.19182	0.809015660104155	0.809015660104155\\
61.625	0.19548	0.917166859301005	0.917166859301005\\
61.625	0.19914	1.03152382757868	1.03152382757868\\
61.625	0.2028	1.15208656493719	1.15208656493719\\
61.625	0.20646	1.27885507137652	1.27885507137652\\
61.625	0.21012	1.41182934689668	1.41182934689668\\
61.625	0.21378	1.55100939149767	1.55100939149767\\
61.625	0.21744	1.6963952051795	1.6963952051795\\
61.625	0.2211	1.84798678794214	1.84798678794214\\
61.625	0.22476	2.00578413978562	2.00578413978562\\
61.625	0.22842	2.16978726070993	2.16978726070993\\
61.625	0.23208	2.33999615071507	2.33999615071507\\
61.625	0.23574	2.51641080980103	2.51641080980103\\
61.625	0.2394	2.69903123796783	2.69903123796783\\
61.625	0.24306	2.88785743521545	2.88785743521545\\
61.625	0.24672	3.0828894015439	3.0828894015439\\
61.625	0.25038	3.28412713695318	3.28412713695318\\
61.625	0.25404	3.49157064144328	3.49157064144328\\
61.625	0.2577	3.70521991501422	3.70521991501422\\
61.625	0.26136	3.92507495766599	3.92507495766599\\
61.625	0.26502	4.15113576939858	4.15113576939858\\
61.625	0.26868	4.383402350212	4.383402350212\\
61.625	0.27234	4.62187470010625	4.62187470010625\\
61.625	0.276	4.86655281908133	4.86655281908133\\
62	0.093	0.266250147924286	0.266250147924286\\
62	0.09666	0.204486251828054	0.204486251828054\\
62	0.10032	0.148928124812653	0.148928124812653\\
62	0.10398	0.0995757668780803	0.0995757668780803\\
62	0.10764	0.0564291780243389	0.0564291780243389\\
62	0.1113	0.0194883582514223	0.0194883582514223\\
62	0.11496	-0.0112466924406656	-0.0112466924406656\\
62	0.11862	-0.0357759740519219	-0.0357759740519219\\
62	0.12228	-0.0540994865823485	-0.0540994865823485\\
62	0.12594	-0.0662172300319468	-0.0662172300319468\\
62	0.1296	-0.0721292044007207	-0.0721292044007207\\
62	0.13326	-0.0718354096886618	-0.0718354096886618\\
62	0.13692	-0.0653358458957776	-0.0653358458957776\\
62	0.14058	-0.0526305130220628	-0.0526305130220628\\
62	0.14424	-0.0337194110675165	-0.0337194110675165\\
62	0.1479	-0.00860254003214678	-0.00860254003214678\\
62	0.15156	0.0227201000840562	0.0227201000840562\\
62	0.15522	0.060248509281088	0.060248509281088\\
62	0.15888	0.103982687558947	0.103982687558947\\
62	0.16254	0.153922634917632	0.153922634917632\\
62	0.1662	0.210068351357149	0.210068351357149\\
62	0.16986	0.272419836877496	0.272419836877496\\
62	0.17352	0.340977091478671	0.340977091478671\\
62	0.17718	0.415740115160675	0.415740115160675\\
62	0.18084	0.4967089079235	0.4967089079235\\
62	0.1845	0.583883469767162	0.583883469767162\\
62	0.18816	0.677263800691653	0.677263800691653\\
62	0.19182	0.77684990069697	0.77684990069697\\
62	0.19548	0.882641769783117	0.882641769783117\\
62	0.19914	0.994639407950091	0.994639407950091\\
62	0.2028	1.11284281519789	1.11284281519789\\
62	0.20646	1.23725199152653	1.23725199152653\\
62	0.21012	1.36786693693599	1.36786693693599\\
62	0.21378	1.50468765142628	1.50468765142628\\
62	0.21744	1.6477141349974	1.6477141349974\\
62	0.2211	1.79694638764934	1.79694638764934\\
62	0.22476	1.95238440938212	1.95238440938212\\
62	0.22842	2.11402820019573	2.11402820019573\\
62	0.23208	2.28187776009016	2.28187776009016\\
62	0.23574	2.45593308906542	2.45593308906542\\
62	0.2394	2.63619418712151	2.63619418712151\\
62	0.24306	2.82266105425843	2.82266105425843\\
62	0.24672	3.01533369047618	3.01533369047618\\
62	0.25038	3.21421209577476	3.21421209577476\\
62	0.25404	3.41929627015416	3.41929627015416\\
62	0.2577	3.6305862136144	3.6305862136144\\
62	0.26136	3.84808192615546	3.84808192615546\\
62	0.26502	4.07178340777736	4.07178340777736\\
62	0.26868	4.30169065848007	4.30169065848007\\
62	0.27234	4.53780367826362	4.53780367826362\\
62	0.276	4.780122467128	4.780122467128\\
62.375	0.093	0.29875312368286	0.29875312368286\\
62.375	0.09666	0.23462989747593	0.23462989747593\\
62.375	0.10032	0.176712440349825	0.176712440349825\\
62.375	0.10398	0.125000752304551	0.125000752304551\\
62.375	0.10764	0.0794948333401058	0.0794948333401058\\
62.375	0.1113	0.0401946834564892	0.0401946834564892\\
62.375	0.11496	0.00710030265370154	0.00710030265370154\\
62.375	0.11862	-0.0197883090682582	-0.0197883090682582\\
62.375	0.12228	-0.0404711517093883	-0.0404711517093883\\
62.375	0.12594	-0.05494822526969	-0.05494822526969\\
62.375	0.1296	-0.0632195297491602	-0.0632195297491602\\
62.375	0.13326	-0.0652850651478047	-0.0652850651478047\\
62.375	0.13692	-0.0611448314656204	-0.0611448314656204\\
62.375	0.14058	-0.0507988287026091	-0.0507988287026091\\
62.375	0.14424	-0.0342470568587663	-0.0342470568587663\\
62.375	0.1479	-0.0114895159340964	-0.0114895159340964\\
62.375	0.15156	0.0174737940714031	0.0174737940714031\\
62.375	0.15522	0.0526428731577315	0.0526428731577315\\
62.375	0.15888	0.0940177213248905	0.0940177213248905\\
62.375	0.16254	0.141598338572876	0.141598338572876\\
62.375	0.1662	0.19538472490169	0.19538472490169\\
62.375	0.16986	0.255376880311333	0.255376880311333\\
62.375	0.17352	0.321574804801808	0.321574804801808\\
62.375	0.17718	0.393978498373105	0.393978498373105\\
62.375	0.18084	0.472587961025237	0.472587961025237\\
62.375	0.1845	0.557403192758192	0.557403192758192\\
62.375	0.18816	0.648424193571984	0.648424193571984\\
62.375	0.19182	0.745650963466597	0.745650963466597\\
62.375	0.19548	0.84908350244204	0.84908350244204\\
62.375	0.19914	0.958721810498318	0.958721810498318\\
62.375	0.2028	1.07456588763542	1.07456588763542\\
62.375	0.20646	1.19661573385335	1.19661573385335\\
62.375	0.21012	1.3248713491521	1.3248713491521\\
62.375	0.21378	1.45933273353169	1.45933273353169\\
62.375	0.21744	1.59999988699211	1.59999988699211\\
62.375	0.2211	1.74687280953336	1.74687280953336\\
62.375	0.22476	1.89995150115543	1.89995150115543\\
62.375	0.22842	2.05923596185834	2.05923596185834\\
62.375	0.23208	2.22472619164207	2.22472619164207\\
62.375	0.23574	2.39642219050663	2.39642219050663\\
62.375	0.2394	2.57432395845202	2.57432395845202\\
62.375	0.24306	2.75843149547823	2.75843149547823\\
62.375	0.24672	2.94874480158528	2.94874480158528\\
62.375	0.25038	3.14526387677316	3.14526387677316\\
62.375	0.25404	3.34798872104187	3.34798872104187\\
62.375	0.2577	3.5569193343914	3.5569193343914\\
62.375	0.26136	3.77205571682176	3.77205571682176\\
62.375	0.26502	3.99339786833295	3.99339786833295\\
62.375	0.26868	4.22094578892496	4.22094578892496\\
62.375	0.27234	4.45469947859781	4.45469947859781\\
62.375	0.276	4.69465893735149	4.69465893735149\\
62.75	0.093	0.332222921618249	0.332222921618249\\
62.75	0.09666	0.265740365300615	0.265740365300615\\
62.75	0.10032	0.205463578063809	0.205463578063809\\
62.75	0.10398	0.151392559907833	0.151392559907833\\
62.75	0.10764	0.103527310832688	0.103527310832688\\
62.75	0.1113	0.0618678308383682	0.0618678308383682\\
62.75	0.11496	0.0264141199248806	0.0264141199248806\\
62.75	0.11862	-0.00283382190778259	-0.00283382190778259\\
62.75	0.12228	-0.0258759946596125	-0.0258759946596125\\
62.75	0.12594	-0.0427123983306177	-0.0427123983306177\\
62.75	0.1296	-0.0533430329207913	-0.0533430329207913\\
62.75	0.13326	-0.0577678984301393	-0.0577678984301393\\
62.75	0.13692	-0.0559869948586549	-0.0559869948586549\\
62.75	0.14058	-0.0480003222063434	-0.0480003222063434\\
62.75	0.14424	-0.0338078804732005	-0.0338078804732005\\
62.75	0.1479	-0.0134096696592341	-0.0134096696592341\\
62.75	0.15156	0.0131943102355621	0.0131943102355621\\
62.75	0.15522	0.0460040592111906	0.0460040592111906\\
62.75	0.15888	0.0850195772676461	0.0850195772676461\\
62.75	0.16254	0.130240864404931	0.130240864404931\\
62.75	0.1662	0.181667920623042	0.181667920623042\\
62.75	0.16986	0.239300745921986	0.239300745921986\\
62.75	0.17352	0.303139340301757	0.303139340301757\\
62.75	0.17718	0.373183703762354	0.373183703762354\\
62.75	0.18084	0.449433836303779	0.449433836303779\\
62.75	0.1845	0.531889737926038	0.531889737926038\\
62.75	0.18816	0.620551408629126	0.620551408629126\\
62.75	0.19182	0.715418848413035	0.715418848413035\\
62.75	0.19548	0.816492057277779	0.816492057277779\\
62.75	0.19914	0.923771035223353	0.923771035223353\\
62.75	0.2028	1.03725578224975	1.03725578224975\\
62.75	0.20646	1.15694629835698	1.15694629835698\\
62.75	0.21012	1.28284258354504	1.28284258354504\\
62.75	0.21378	1.41494463781392	1.41494463781392\\
62.75	0.21744	1.55325246116364	1.55325246116364\\
62.75	0.2211	1.69776605359418	1.69776605359418\\
62.75	0.22476	1.84848541510556	1.84848541510556\\
62.75	0.22842	2.00541054569776	2.00541054569776\\
62.75	0.23208	2.16854144537078	2.16854144537078\\
62.75	0.23574	2.33787811412465	2.33787811412465\\
62.75	0.2394	2.51342055195934	2.51342055195934\\
62.75	0.24306	2.69516875887485	2.69516875887485\\
62.75	0.24672	2.88312273487119	2.88312273487119\\
62.75	0.25038	3.07728247994837	3.07728247994837\\
62.75	0.25404	3.27764799410637	3.27764799410637\\
62.75	0.2577	3.48421927734521	3.48421927734521\\
62.75	0.26136	3.69699632966486	3.69699632966486\\
62.75	0.26502	3.91597915106535	3.91597915106535\\
62.75	0.26868	4.14116774154667	4.14116774154667\\
62.75	0.27234	4.37256210110882	4.37256210110882\\
62.75	0.276	4.61016222975179	4.61016222975179\\
63.125	0.093	0.366659541730448	0.366659541730448\\
63.125	0.09666	0.297817655302113	0.297817655302113\\
63.125	0.10032	0.235181537954607	0.235181537954607\\
63.125	0.10398	0.178751189687929	0.178751189687929\\
63.125	0.10764	0.128526610502081	0.128526610502081\\
63.125	0.1113	0.084507800397061	0.084507800397061\\
63.125	0.11496	0.0466947593728699	0.0466947593728699\\
63.125	0.11862	0.0150874874295068	0.0150874874295068\\
63.125	0.12228	-0.0103140154330266	-0.0103140154330266\\
63.125	0.12594	-0.0295097492147316	-0.0295097492147316\\
63.125	0.1296	-0.0424997139156087	-0.0424997139156087\\
63.125	0.13326	-0.0492839095356565	-0.0492839095356565\\
63.125	0.13692	-0.0498623360748756	-0.0498623360748756\\
63.125	0.14058	-0.0442349935332675	-0.0442349935332675\\
63.125	0.14424	-0.032401881910828	-0.032401881910828\\
63.125	0.1479	-0.014363001207558	-0.014363001207558\\
63.125	0.15156	0.00988164857653473	0.00988164857653473\\
63.125	0.15522	0.0403320674414598	0.0403320674414598\\
63.125	0.15888	0.0769882553872119	0.0769882553872119\\
63.125	0.16254	0.119850212413794	0.119850212413794\\
63.125	0.1662	0.168917938521208	0.168917938521208\\
63.125	0.16986	0.224191433709448	0.224191433709448\\
63.125	0.17352	0.285670697978516	0.285670697978516\\
63.125	0.17718	0.35335573132841	0.35335573132841\\
63.125	0.18084	0.427246533759135	0.427246533759135\\
63.125	0.1845	0.50734310527069	0.50734310527069\\
63.125	0.18816	0.593645445863075	0.593645445863075\\
63.125	0.19182	0.686153555536288	0.686153555536288\\
63.125	0.19548	0.784867434290328	0.784867434290328\\
63.125	0.19914	0.889787082125199	0.889787082125199\\
63.125	0.2028	1.0009124990409	1.0009124990409\\
63.125	0.20646	1.11824368503743	1.11824368503743\\
63.125	0.21012	1.24178064011478	1.24178064011478\\
63.125	0.21378	1.37152336427297	1.37152336427297\\
63.125	0.21744	1.50747185751198	1.50747185751198\\
63.125	0.2211	1.64962611983182	1.64962611983182\\
63.125	0.22476	1.79798615123249	1.79798615123249\\
63.125	0.22842	1.95255195171399	1.95255195171399\\
63.125	0.23208	2.11332352127632	2.11332352127632\\
63.125	0.23574	2.28030085991948	2.28030085991948\\
63.125	0.2394	2.45348396764346	2.45348396764346\\
63.125	0.24306	2.63287284444827	2.63287284444827\\
63.125	0.24672	2.81846749033392	2.81846749033392\\
63.125	0.25038	3.01026790530039	3.01026790530039\\
63.125	0.25404	3.20827408934769	3.20827408934769\\
63.125	0.2577	3.41248604247582	3.41248604247582\\
63.125	0.26136	3.62290376468478	3.62290376468478\\
63.125	0.26502	3.83952725597457	3.83952725597457\\
63.125	0.26868	4.06235651634518	4.06235651634518\\
63.125	0.27234	4.29139154579662	4.29139154579662\\
63.125	0.276	4.5266323443289	4.5266323443289\\
63.5	0.093	0.402062984019461	0.402062984019461\\
63.5	0.09666	0.330861767480424	0.330861767480424\\
63.5	0.10032	0.265866320022215	0.265866320022215\\
63.5	0.10398	0.207076641644835	0.207076641644835\\
63.5	0.10764	0.154492732348284	0.154492732348284\\
63.5	0.1113	0.108114592132564	0.108114592132564\\
63.5	0.11496	0.0679422209976694	0.0679422209976694\\
63.5	0.11862	0.0339756189436065	0.0339756189436065\\
63.5	0.12228	0.00621478597036962	0.00621478597036962\\
63.5	0.12594	-0.0153402779220388	-0.0153402779220388\\
63.5	0.1296	-0.0306895727336158	-0.0306895727336158\\
63.5	0.13326	-0.0398330984643671	-0.0398330984643671\\
63.5	0.13692	-0.042770855114286	-0.042770855114286\\
63.5	0.14058	-0.0395028426833779	-0.0395028426833779\\
63.5	0.14424	-0.0300290611716418	-0.0300290611716418\\
63.5	0.1479	-0.0143495105790787	-0.0143495105790787\\
63.5	0.15156	0.00753580909431761	0.00753580909431761\\
63.5	0.15522	0.0356268978485428	0.0356268978485428\\
63.5	0.15888	0.0699237556835914	0.0699237556835914\\
63.5	0.16254	0.110426382599473	0.110426382599473\\
63.5	0.1662	0.157134778596184	0.157134778596184\\
63.5	0.16986	0.210048943673721	0.210048943673721\\
63.5	0.17352	0.269168877832089	0.269168877832089\\
63.5	0.17718	0.334494581071283	0.334494581071283\\
63.5	0.18084	0.406026053391308	0.406026053391308\\
63.5	0.1845	0.48376329479216	0.48376329479216\\
63.5	0.18816	0.567706305273841	0.567706305273841\\
63.5	0.19182	0.657855084836351	0.657855084836351\\
63.5	0.19548	0.754209633479691	0.754209633479691\\
63.5	0.19914	0.856769951203859	0.856769951203859\\
63.5	0.2028	0.965536038008855	0.965536038008855\\
63.5	0.20646	1.08050789389468	1.08050789389468\\
63.5	0.21012	1.20168551886133	1.20168551886133\\
63.5	0.21378	1.32906891290882	1.32906891290882\\
63.5	0.21744	1.46265807603713	1.46265807603713\\
63.5	0.2211	1.60245300824627	1.60245300824627\\
63.5	0.22476	1.74845370953624	1.74845370953624\\
63.5	0.22842	1.90066017990704	1.90066017990704\\
63.5	0.23208	2.05907241935866	2.05907241935866\\
63.5	0.23574	2.22369042789112	2.22369042789112\\
63.5	0.2394	2.39451420550441	2.39451420550441\\
63.5	0.24306	2.57154375219851	2.57154375219851\\
63.5	0.24672	2.75477906797346	2.75477906797346\\
63.5	0.25038	2.94422015282923	2.94422015282923\\
63.5	0.25404	3.13986700676582	3.13986700676582\\
63.5	0.2577	3.34171962978325	3.34171962978325\\
63.5	0.26136	3.54977802188151	3.54977802188151\\
63.5	0.26502	3.7640421830606	3.7640421830606\\
63.5	0.26868	3.98451211332051	3.98451211332051\\
63.5	0.27234	4.21118781266125	4.21118781266125\\
63.5	0.276	4.44406928108283	4.44406928108283\\
63.875	0.093	0.438433248485288	0.438433248485288\\
63.875	0.09666	0.364872701835548	0.364872701835548\\
63.875	0.10032	0.297517924266637	0.297517924266637\\
63.875	0.10398	0.236368915778555	0.236368915778555\\
63.875	0.10764	0.181425676371304	0.181425676371304\\
63.875	0.1113	0.132688206044881	0.132688206044881\\
63.875	0.11496	0.0901565047992827	0.0901565047992827\\
63.875	0.11862	0.0538305726345198	0.0538305726345198\\
63.875	0.12228	0.0237104095505796	0.0237104095505796\\
63.875	0.12594	-0.00020398445252523	-0.00020398445252523\\
63.875	0.1296	-0.0179126093748092	-0.0179126093748092\\
63.875	0.13326	-0.0294154652162604	-0.0294154652162604\\
63.875	0.13692	-0.0347125519768827	-0.0347125519768827\\
63.875	0.14058	-0.0338038696566745	-0.0338038696566745\\
63.875	0.14424	-0.0266894182556383	-0.0266894182556383\\
63.875	0.1479	-0.0133691977737787	-0.0133691977737787\\
63.875	0.15156	0.00615679178891426	0.00615679178891426\\
63.875	0.15522	0.031888550432436	0.031888550432436\\
63.875	0.15888	0.0638260781567848	0.0638260781567848\\
63.875	0.16254	0.101969374961967	0.101969374961967\\
63.875	0.1662	0.146318440847974	0.146318440847974\\
63.875	0.16986	0.196873275814811	0.196873275814811\\
63.875	0.17352	0.253633879862476	0.253633879862476\\
63.875	0.17718	0.31660025299097	0.31660025299097\\
63.875	0.18084	0.385772395200288	0.385772395200288\\
63.875	0.1845	0.46115030649044	0.46115030649044\\
63.875	0.18816	0.542733986861421	0.542733986861421\\
63.875	0.19182	0.630523436313227	0.630523436313227\\
63.875	0.19548	0.724518654845868	0.724518654845868\\
63.875	0.19914	0.824719642459335	0.824719642459335\\
63.875	0.2028	0.931126399153628	0.931126399153628\\
63.875	0.20646	1.04373892492875	1.04373892492875\\
63.875	0.21012	1.1625572197847	1.1625572197847\\
63.875	0.21378	1.28758128372148	1.28758128372148\\
63.875	0.21744	1.41881111673909	1.41881111673909\\
63.875	0.2211	1.55624671883753	1.55624671883753\\
63.875	0.22476	1.6998880900168	1.6998880900168\\
63.875	0.22842	1.8497352302769	1.8497352302769\\
63.875	0.23208	2.00578813961782	2.00578813961782\\
63.875	0.23574	2.16804681803958	2.16804681803958\\
63.875	0.2394	2.33651126554216	2.33651126554216\\
63.875	0.24306	2.51118148212556	2.51118148212556\\
63.875	0.24672	2.69205746778981	2.69205746778981\\
63.875	0.25038	2.87913922253488	2.87913922253488\\
63.875	0.25404	3.07242674636078	3.07242674636078\\
63.875	0.2577	3.2719200392675	3.2719200392675\\
63.875	0.26136	3.47761910125505	3.47761910125505\\
63.875	0.26502	3.68952393232344	3.68952393232344\\
63.875	0.26868	3.90763453247265	3.90763453247265\\
63.875	0.27234	4.13195090170269	4.13195090170269\\
63.875	0.276	4.36247304001356	4.36247304001356\\
64.25	0.093	0.475770335127923	0.475770335127923\\
64.25	0.09666	0.399850458367481	0.399850458367481\\
64.25	0.10032	0.330136350687869	0.330136350687869\\
64.25	0.10398	0.266628012089087	0.266628012089087\\
64.25	0.10764	0.209325442571132	0.209325442571132\\
64.25	0.1113	0.158228642134006	0.158228642134006\\
64.25	0.11496	0.113337610777708	0.113337610777708\\
64.25	0.11862	0.0746523485022417	0.0746523485022417\\
64.25	0.12228	0.0421728553076015	0.0421728553076015\\
64.25	0.12594	0.0158991311937933	0.0158991311937933\\
64.25	0.1296	-0.00416882383919059	-0.00416882383919059\\
64.25	0.13326	-0.0180310097913452	-0.0180310097913452\\
64.25	0.13692	-0.0256874266626674	-0.0256874266626674\\
64.25	0.14058	-0.0271380744531626	-0.0271380744531626\\
64.25	0.14424	-0.0223829531628299	-0.0223829531628299\\
64.25	0.1479	-0.0114220627916666	-0.0114220627916666\\
64.25	0.15156	0.0057445966603229	0.0057445966603229\\
64.25	0.15522	0.0291170251931412	0.0291170251931412\\
64.25	0.15888	0.0586952228067901	0.0586952228067901\\
64.25	0.16254	0.0944791895012651	0.0944791895012651\\
64.25	0.1662	0.136468925276573	0.136468925276573\\
64.25	0.16986	0.18466443013271	0.18466443013271\\
64.25	0.17352	0.239065704069671	0.239065704069671\\
64.25	0.17718	0.299672747087461	0.299672747087461\\
64.25	0.18084	0.36648555918608	0.36648555918608\\
64.25	0.1845	0.439504140365532	0.439504140365532\\
64.25	0.18816	0.518728490625806	0.518728490625806\\
64.25	0.19182	0.604158609966916	0.604158609966916\\
64.25	0.19548	0.695794498388853	0.695794498388853\\
64.25	0.19914	0.793636155891617	0.793636155891617\\
64.25	0.2028	0.89768358247521	0.89768358247521\\
64.25	0.20646	1.00793677813963	1.00793677813963\\
64.25	0.21012	1.12439574288488	1.12439574288488\\
64.25	0.21378	1.24706047671096	1.24706047671096\\
64.25	0.21744	1.37593097961787	1.37593097961787\\
64.25	0.2211	1.51100725160561	1.51100725160561\\
64.25	0.22476	1.65228929267417	1.65228929267417\\
64.25	0.22842	1.79977710282357	1.79977710282357\\
64.25	0.23208	1.95347068205378	1.95347068205378\\
64.25	0.23574	2.11337003036484	2.11337003036484\\
64.25	0.2394	2.27947514775672	2.27947514775672\\
64.25	0.24306	2.45178603422943	2.45178603422943\\
64.25	0.24672	2.63030268978297	2.63030268978297\\
64.25	0.25038	2.81502511441734	2.81502511441734\\
64.25	0.25404	3.00595330813253	3.00595330813253\\
64.25	0.2577	3.20308727092855	3.20308727092855\\
64.25	0.26136	3.40642700280541	3.40642700280541\\
64.25	0.26502	3.61597250376309	3.61597250376309\\
64.25	0.26868	3.8317237738016	3.8317237738016\\
64.25	0.27234	4.05368081292094	4.05368081292094\\
64.25	0.276	4.2818436211211	4.2818436211211\\
64.625	0.093	0.51407424394737	0.51407424394737\\
64.625	0.09666	0.435795037076228	0.435795037076228\\
64.625	0.10032	0.363721599285916	0.363721599285916\\
64.625	0.10398	0.297853930576431	0.297853930576431\\
64.625	0.10764	0.238192030947777	0.238192030947777\\
64.625	0.1113	0.184735900399946	0.184735900399946\\
64.625	0.11496	0.137485538932949	0.137485538932949\\
64.625	0.11862	0.096440946546779	0.096440946546779\\
64.625	0.12228	0.061602123241439	0.061602123241439\\
64.625	0.12594	0.0329690690169238	0.0329690690169238\\
64.625	0.1296	0.01054178387324	0.01054178387324\\
64.625	0.13326	-0.00567973218961448	-0.00567973218961448\\
64.625	0.13692	-0.0156954791716402	-0.0156954791716402\\
64.625	0.14058	-0.0195054570728388	-0.0195054570728388\\
64.625	0.14424	-0.0171096658932024	-0.0171096658932024\\
64.625	0.1479	-0.00850810563274251	-0.00850810563274251\\
64.625	0.15156	0.00629922370854352	0.00629922370854352\\
64.625	0.15522	0.0273123221306619	0.0273123221306619\\
64.625	0.15888	0.0545311896336074	0.0545311896336074\\
64.625	0.16254	0.0879558262173825	0.0879558262173825\\
64.625	0.1662	0.127586231881986	0.127586231881986\\
64.625	0.16986	0.17342240662742	0.17342240662742\\
64.625	0.17352	0.225464350453682	0.225464350453682\\
64.625	0.17718	0.283712063360772	0.283712063360772\\
64.625	0.18084	0.34816554534869	0.34816554534869\\
64.625	0.1845	0.418824796417436	0.418824796417436\\
64.625	0.18816	0.49568981656701	0.49568981656701\\
64.625	0.19182	0.578760605797417	0.578760605797417\\
64.625	0.19548	0.66803716410865	0.66803716410865\\
64.625	0.19914	0.763519491500714	0.763519491500714\\
64.625	0.2028	0.865207587973604	0.865207587973604\\
64.625	0.20646	0.973101453527326	0.973101453527326\\
64.625	0.21012	1.08720108816187	1.08720108816187\\
64.625	0.21378	1.20750649187725	1.20750649187725\\
64.625	0.21744	1.33401766467346	1.33401766467346\\
64.625	0.2211	1.46673460655049	1.46673460655049\\
64.625	0.22476	1.60565731750836	1.60565731750836\\
64.625	0.22842	1.75078579754705	1.75078579754705\\
64.625	0.23208	1.90212004666657	1.90212004666657\\
64.625	0.23574	2.05966006486692	2.05966006486692\\
64.625	0.2394	2.2234058521481	2.2234058521481\\
64.625	0.24306	2.39335740851011	2.39335740851011\\
64.625	0.24672	2.56951473395294	2.56951473395294\\
64.625	0.25038	2.75187782847661	2.75187782847661\\
64.625	0.25404	2.9404466920811	2.9404466920811\\
64.625	0.2577	3.13522132476642	3.13522132476642\\
64.625	0.26136	3.33620172653257	3.33620172653257\\
64.625	0.26502	3.54338789737955	3.54338789737955\\
64.625	0.26868	3.75677983730736	3.75677983730736\\
64.625	0.27234	3.976377546316	3.976377546316\\
64.625	0.276	4.20218102440547	4.20218102440547\\
65	0.093	0.553344974943635	0.553344974943635\\
65	0.09666	0.472706437961793	0.472706437961793\\
65	0.10032	0.398273670060774	0.398273670060774\\
65	0.10398	0.330046671240589	0.330046671240589\\
65	0.10764	0.268025441501231	0.268025441501231\\
65	0.1113	0.212209980842701	0.212209980842701\\
65	0.11496	0.162600289265003	0.162600289265003\\
65	0.11862	0.11919636676813	0.11919636676813\\
65	0.12228	0.0819982133520867	0.0819982133520867\\
65	0.12594	0.0510058290168751	0.0510058290168751\\
65	0.1296	0.0262192137624879	0.0262192137624879\\
65	0.13326	0.00763836758893	0.00763836758893\\
65	0.13692	-0.00473670950379557	-0.00473670950379557\\
65	0.14058	-0.0109060175156976	-0.0109060175156976\\
65	0.14424	-0.0108695564467647	-0.0108695564467647\\
65	0.1479	-0.00462732629701179	-0.00462732629701179\\
65	0.15156	0.00782067293357791	0.00782067293357791\\
65	0.15522	0.0264744412449929	0.0264744412449929\\
65	0.15888	0.0513339786372384	0.0513339786372384\\
65	0.16254	0.0823992851103137	0.0823992851103137\\
65	0.1662	0.119670360664214	0.119670360664214\\
65	0.16986	0.163147205298944	0.163147205298944\\
65	0.17352	0.212829819014506	0.212829819014506\\
65	0.17718	0.26871820181089	0.26871820181089\\
65	0.18084	0.330812353688108	0.330812353688108\\
65	0.1845	0.399112274646154	0.399112274646154\\
65	0.18816	0.473617964685028	0.473617964685028\\
65	0.19182	0.554329423804734	0.554329423804734\\
65	0.19548	0.641246652005265	0.641246652005265\\
65	0.19914	0.734369649286629	0.734369649286629\\
65	0.2028	0.833698415648815	0.833698415648815\\
65	0.20646	0.939232951091837	0.939232951091837\\
65	0.21012	1.05097325561568	1.05097325561568\\
65	0.21378	1.16891932922036	1.16891932922036\\
65	0.21744	1.29307117190586	1.29307117190586\\
65	0.2211	1.42342878367219	1.42342878367219\\
65	0.22476	1.55999216451936	1.55999216451936\\
65	0.22842	1.70276131444735	1.70276131444735\\
65	0.23208	1.85173623345617	1.85173623345617\\
65	0.23574	2.00691692154582	2.00691692154582\\
65	0.2394	2.16830337871629	2.16830337871629\\
65	0.24306	2.3358956049676	2.3358956049676\\
65	0.24672	2.50969360029973	2.50969360029973\\
65	0.25038	2.6896973647127	2.6896973647127\\
65	0.25404	2.87590689820648	2.87590689820648\\
65	0.2577	3.06832220078111	3.06832220078111\\
65	0.26136	3.26694327243655	3.26694327243655\\
65	0.26502	3.47177011317284	3.47177011317284\\
65	0.26868	3.68280272298994	3.68280272298994\\
65	0.27234	3.90004110188788	3.90004110188788\\
65	0.276	4.12348524986664	4.12348524986664\\
65.375	0.093	0.59358252811671	0.59358252811671\\
65.375	0.09666	0.510584661024161	0.510584661024161\\
65.375	0.10032	0.433792563012445	0.433792563012445\\
65.375	0.10398	0.363206234081557	0.363206234081557\\
65.375	0.10764	0.298825674231499	0.298825674231499\\
65.375	0.1113	0.240650883462266	0.240650883462266\\
65.375	0.11496	0.188681861773865	0.188681861773865\\
65.375	0.11862	0.142918609166292	0.142918609166292\\
65.375	0.12228	0.103361125639548	0.103361125639548\\
65.375	0.12594	0.0700094111936296	0.0700094111936296\\
65.375	0.1296	0.0428634658285425	0.0428634658285425\\
65.375	0.13326	0.0219232895442847	0.0219232895442847\\
65.375	0.13692	0.00718888234085213	0.00718888234085213\\
65.375	0.14058	-0.00133975578174628	-0.00133975578174628\\
65.375	0.14424	-0.00366262482351676	-0.00366262482351676\\
65.375	0.1479	0.000220275215536248	0.000220275215536248\\
65.375	0.15156	0.0103089443354225	0.0103089443354225\\
65.375	0.15522	0.0266033825361376	0.0266033825361376\\
65.375	0.15888	0.0491035898176797	0.0491035898176797\\
65.375	0.16254	0.0778095661800515	0.0778095661800515\\
65.375	0.1662	0.112721311623252	0.112721311623252\\
65.375	0.16986	0.153838826147283	0.153838826147283\\
65.375	0.17352	0.201162109752141	0.201162109752141\\
65.375	0.17718	0.254691162437824	0.254691162437824\\
65.375	0.18084	0.314425984204343	0.314425984204343\\
65.375	0.1845	0.380366575051685	0.380366575051685\\
65.375	0.18816	0.452512934979856	0.452512934979856\\
65.375	0.19182	0.530865063988859	0.530865063988859\\
65.375	0.19548	0.615422962078689	0.615422962078689\\
65.375	0.19914	0.70618662924935	0.70618662924935\\
65.375	0.2028	0.803156065500836	0.803156065500836\\
65.375	0.20646	0.906331270833155	0.906331270833155\\
65.375	0.21012	1.01571224524629	1.01571224524629\\
65.375	0.21378	1.13129898874027	1.13129898874027\\
65.375	0.21744	1.25309150131507	1.25309150131507\\
65.375	0.2211	1.3810897829707	1.3810897829707\\
65.375	0.22476	1.51529383370717	1.51529383370717\\
65.375	0.22842	1.65570365352446	1.65570365352446\\
65.375	0.23208	1.80231924242257	1.80231924242257\\
65.375	0.23574	1.95514060040152	1.95514060040152\\
65.375	0.2394	2.1141677274613	2.1141677274613\\
65.375	0.24306	2.27940062360189	2.27940062360189\\
65.375	0.24672	2.45083928882333	2.45083928882333\\
65.375	0.25038	2.62848372312559	2.62848372312559\\
65.375	0.25404	2.81233392650868	2.81233392650868\\
65.375	0.2577	3.0023898989726	3.0023898989726\\
65.375	0.26136	3.19865164051735	3.19865164051735\\
65.375	0.26502	3.40111915114292	3.40111915114292\\
65.375	0.26868	3.60979243084933	3.60979243084933\\
65.375	0.27234	3.82467147963656	3.82467147963656\\
65.375	0.276	4.04575629750463	4.04575629750463\\
65.75	0.093	0.634786903466594	0.634786903466594\\
65.75	0.09666	0.549429706263346	0.549429706263346\\
65.75	0.10032	0.470278278140927	0.470278278140927\\
65.75	0.10398	0.397332619099335	0.397332619099335\\
65.75	0.10764	0.330592729138577	0.330592729138577\\
65.75	0.1113	0.270058608258644	0.270058608258644\\
65.75	0.11496	0.21573025645954	0.21573025645954\\
65.75	0.11862	0.167607673741263	0.167607673741263\\
65.75	0.12228	0.125690860103816	0.125690860103816\\
65.75	0.12594	0.0899798155471978	0.0899798155471978\\
65.75	0.1296	0.0604745400714108	0.0604745400714108\\
65.75	0.13326	0.0371750336764496	0.0371750336764496\\
65.75	0.13692	0.0200812963623207	0.0200812963623207\\
65.75	0.14058	0.0091933281290153	0.0091933281290153\\
65.75	0.14424	0.00451112897654493	0.00451112897654493\\
65.75	0.1479	0.00603469890489805	0.00603469890489805\\
65.75	0.15156	0.0137640379140809	0.0137640379140809\\
65.75	0.15522	0.0276991460040925	0.0276991460040925\\
65.75	0.15888	0.0478400231749347	0.0478400231749347\\
65.75	0.16254	0.0741866694266031	0.0741866694266031\\
65.75	0.1662	0.1067390847591	0.1067390847591\\
65.75	0.16986	0.145497269172431	0.145497269172431\\
65.75	0.17352	0.190461222666586	0.190461222666586\\
65.75	0.17718	0.241630945241573	0.241630945241573\\
65.75	0.18084	0.299006436897384	0.299006436897384\\
65.75	0.1845	0.362587697634023	0.362587697634023\\
65.75	0.18816	0.432374727451498	0.432374727451498\\
65.75	0.19182	0.508367526349794	0.508367526349794\\
65.75	0.19548	0.590566094328924	0.590566094328924\\
65.75	0.19914	0.678970431388882	0.678970431388882\\
65.75	0.2028	0.773580537529668	0.773580537529668\\
65.75	0.20646	0.874396412751283	0.874396412751283\\
65.75	0.21012	0.981418057053723	0.981418057053723\\
65.75	0.21378	1.09464547043699	1.09464547043699\\
65.75	0.21744	1.21407865290109	1.21407865290109\\
65.75	0.2211	1.33971760444603	1.33971760444603\\
65.75	0.22476	1.47156232507179	1.47156232507179\\
65.75	0.22842	1.60961281477837	1.60961281477837\\
65.75	0.23208	1.75386907356579	1.75386907356579\\
65.75	0.23574	1.90433110143404	1.90433110143404\\
65.75	0.2394	2.06099889838311	2.06099889838311\\
65.75	0.24306	2.22387246441301	2.22387246441301\\
65.75	0.24672	2.39295179952374	2.39295179952374\\
65.75	0.25038	2.5682369037153	2.5682369037153\\
65.75	0.25404	2.74972777698768	2.74972777698768\\
65.75	0.2577	2.9374244193409	2.9374244193409\\
65.75	0.26136	3.13132683077495	3.13132683077495\\
65.75	0.26502	3.33143501128982	3.33143501128982\\
65.75	0.26868	3.53774896088552	3.53774896088552\\
65.75	0.27234	3.75026867956206	3.75026867956206\\
65.75	0.276	3.96899416731942	3.96899416731942\\
66.125	0.093	0.676958100993293	0.676958100993293\\
66.125	0.09666	0.589241573679341	0.589241573679341\\
66.125	0.10032	0.507730815446222	0.507730815446222\\
66.125	0.10398	0.432425826293927	0.432425826293927\\
66.125	0.10764	0.363326606222466	0.363326606222466\\
66.125	0.1113	0.300433155231833	0.300433155231833\\
66.125	0.11496	0.243745473322025	0.243745473322025\\
66.125	0.11862	0.193263560493048	0.193263560493048\\
66.125	0.12228	0.148987416744902	0.148987416744902\\
66.125	0.12594	0.110917042077583	0.110917042077583\\
66.125	0.1296	0.0790524364910929	0.0790524364910929\\
66.125	0.13326	0.0533935999854283	0.0533935999854283\\
66.125	0.13692	0.0339405325605924	0.0339405325605924\\
66.125	0.14058	0.0206932342165906	0.0206932342165906\\
66.125	0.14424	0.0136517049534168	0.0136517049534168\\
66.125	0.1479	0.0128159447710701	0.0128159447710701\\
66.125	0.15156	0.0181859536695494	0.0181859536695494\\
66.125	0.15522	0.0297617316488612	0.0297617316488612\\
66.125	0.15888	0.047543278709	0.047543278709\\
66.125	0.16254	0.0715305948499685	0.0715305948499685\\
66.125	0.1662	0.101723680071762	0.101723680071762\\
66.125	0.16986	0.138122534374389	0.138122534374389\\
66.125	0.17352	0.180727157757844	0.180727157757844\\
66.125	0.17718	0.229537550222124	0.229537550222124\\
66.125	0.18084	0.284553711767236	0.284553711767236\\
66.125	0.1845	0.345775642393178	0.345775642393178\\
66.125	0.18816	0.413203342099949	0.413203342099949\\
66.125	0.19182	0.486836810887546	0.486836810887546\\
66.125	0.19548	0.566676048755973	0.566676048755973\\
66.125	0.19914	0.652721055705227	0.652721055705227\\
66.125	0.2028	0.74497183173531	0.74497183173531\\
66.125	0.20646	0.843428376846225	0.843428376846225\\
66.125	0.21012	0.948090691037962	0.948090691037962\\
66.125	0.21378	1.05895877431053	1.05895877431053\\
66.125	0.21744	1.17603262666393	1.17603262666393\\
66.125	0.2211	1.29931224809816	1.29931224809816\\
66.125	0.22476	1.42879763861322	1.42879763861322\\
66.125	0.22842	1.56448879820911	1.56448879820911\\
66.125	0.23208	1.70638572688581	1.70638572688581\\
66.125	0.23574	1.85448842464336	1.85448842464336\\
66.125	0.2394	2.00879689148173	2.00879689148173\\
66.125	0.24306	2.16931112740093	2.16931112740093\\
66.125	0.24672	2.33603113240096	2.33603113240096\\
66.125	0.25038	2.50895690648182	2.50895690648182\\
66.125	0.25404	2.6880884496435	2.6880884496435\\
66.125	0.2577	2.87342576188602	2.87342576188602\\
66.125	0.26136	3.06496884320937	3.06496884320937\\
66.125	0.26502	3.26271769361354	3.26271769361354\\
66.125	0.26868	3.46667231309854	3.46667231309854\\
66.125	0.27234	3.67683270166436	3.67683270166436\\
66.125	0.276	3.89319885931103	3.89319885931103\\
66.5	0.093	0.720096120696802	0.720096120696802\\
66.5	0.09666	0.630020263272154	0.630020263272154\\
66.5	0.10032	0.546150174928328	0.546150174928328\\
66.5	0.10398	0.468485855665333	0.468485855665333\\
66.5	0.10764	0.397027305483172	0.397027305483172\\
66.5	0.1113	0.331774524381835	0.331774524381835\\
66.5	0.11496	0.272727512361327	0.272727512361327\\
66.5	0.11862	0.219886269421647	0.219886269421647\\
66.5	0.12228	0.173250795562797	0.173250795562797\\
66.5	0.12594	0.132821090784776	0.132821090784776\\
66.5	0.1296	0.0985971550875853	0.0985971550875853\\
66.5	0.13326	0.0705789884712207	0.0705789884712207\\
66.5	0.13692	0.0487665909356849	0.0487665909356849\\
66.5	0.14058	0.0331599624809797	0.0331599624809797\\
66.5	0.14424	0.0237591031071025	0.0237591031071025\\
66.5	0.1479	0.0205640128140558	0.0205640128140558\\
66.5	0.15156	0.0235746916018318	0.0235746916018318\\
66.5	0.15522	0.0327911394704401	0.0327911394704401\\
66.5	0.15888	0.048213356419879	0.048213356419879\\
66.5	0.16254	0.0698413424501441	0.0698413424501441\\
66.5	0.1662	0.0976750975612379	0.0976750975612379\\
66.5	0.16986	0.131714621753165	0.131714621753165\\
66.5	0.17352	0.171959915025917	0.171959915025917\\
66.5	0.17718	0.218410977379497	0.218410977379497\\
66.5	0.18084	0.271067808813909	0.271067808813909\\
66.5	0.1845	0.329930409329144	0.329930409329144\\
66.5	0.18816	0.394998778925215	0.394998778925215\\
66.5	0.19182	0.466272917602108	0.466272917602108\\
66.5	0.19548	0.543752825359832	0.543752825359832\\
66.5	0.19914	0.627438502198389	0.627438502198389\\
66.5	0.2028	0.717329948117769	0.717329948117769\\
66.5	0.20646	0.81342716311798	0.81342716311798\\
66.5	0.21012	0.915730147199021	0.915730147199021\\
66.5	0.21378	1.02423890036088	1.02423890036088\\
66.5	0.21744	1.13895342260358	1.13895342260358\\
66.5	0.2211	1.25987371392711	1.25987371392711\\
66.5	0.22476	1.38699977433147	1.38699977433147\\
66.5	0.22842	1.52033160381665	1.52033160381665\\
66.5	0.23208	1.65986920238266	1.65986920238266\\
66.5	0.23574	1.8056125700295	1.8056125700295\\
66.5	0.2394	1.95756170675718	1.95756170675718\\
66.5	0.24306	2.11571661256567	2.11571661256567\\
66.5	0.24672	2.280077287455	2.280077287455\\
66.5	0.25038	2.45064373142516	2.45064373142516\\
66.5	0.25404	2.62741594447614	2.62741594447614\\
66.5	0.2577	2.81039392660795	2.81039392660795\\
66.5	0.26136	2.99957767782059	2.99957767782059\\
66.5	0.26502	3.19496719811407	3.19496719811407\\
66.5	0.26868	3.39656248748837	3.39656248748837\\
66.5	0.27234	3.60436354594349	3.60436354594349\\
66.5	0.276	3.81837037347945	3.81837037347945\\
66.875	0.093	0.764200962577127	0.764200962577127\\
66.875	0.09666	0.671765775041771	0.671765775041771\\
66.875	0.10032	0.585536356587245	0.585536356587245\\
66.875	0.10398	0.505512707213551	0.505512707213551\\
66.875	0.10764	0.431694826920686	0.431694826920686\\
66.875	0.1113	0.364082715708646	0.364082715708646\\
66.875	0.11496	0.302676373577438	0.302676373577438\\
66.875	0.11862	0.247475800527059	0.247475800527059\\
66.875	0.12228	0.198480996557505	0.198480996557505\\
66.875	0.12594	0.15569196166878	0.15569196166878\\
66.875	0.1296	0.11910869586089	0.11910869586089\\
66.875	0.13326	0.0887311991338215	0.0887311991338215\\
66.875	0.13692	0.0645594714875823	0.0645594714875823\\
66.875	0.14058	0.0465935129221773	0.0465935129221773\\
66.875	0.14424	0.0348333234376001	0.0348333234376001\\
66.875	0.1479	0.0292789030338465	0.0292789030338465\\
66.875	0.15156	0.0299302517109226	0.0299302517109226\\
66.875	0.15522	0.036787369468831	0.036787369468831\\
66.875	0.15888	0.0498502563075665	0.0498502563075665\\
66.875	0.16254	0.0691189122271316	0.0691189122271316\\
66.875	0.1662	0.0945933372275256	0.0945933372275256\\
66.875	0.16986	0.126273531308749	0.126273531308749\\
66.875	0.17352	0.164159494470797	0.164159494470797\\
66.875	0.17718	0.208251226713674	0.208251226713674\\
66.875	0.18084	0.258548728037386	0.258548728037386\\
66.875	0.1845	0.315051998441922	0.315051998441922\\
66.875	0.18816	0.377761037927286	0.377761037927286\\
66.875	0.19182	0.446675846493479	0.446675846493479\\
66.875	0.19548	0.521796424140502	0.521796424140502\\
66.875	0.19914	0.603122770868357	0.603122770868357\\
66.875	0.2028	0.690654886677036	0.690654886677036\\
66.875	0.20646	0.784392771566548	0.784392771566548\\
66.875	0.21012	0.884336425536885	0.884336425536885\\
66.875	0.21378	0.990485848588049	0.990485848588049\\
66.875	0.21744	1.10284104072004	1.10284104072004\\
66.875	0.2211	1.22140200193287	1.22140200193287\\
66.875	0.22476	1.34616873222652	1.34616873222652\\
66.875	0.22842	1.47714123160101	1.47714123160101\\
66.875	0.23208	1.61431950005631	1.61431950005631\\
66.875	0.23574	1.75770353759245	1.75770353759245\\
66.875	0.2394	1.90729334420942	1.90729334420942\\
66.875	0.24306	2.06308891990722	2.06308891990722\\
66.875	0.24672	2.22509026468585	2.22509026468585\\
66.875	0.25038	2.3932973785453	2.3932973785453\\
66.875	0.25404	2.56771026148558	2.56771026148558\\
66.875	0.2577	2.74832891350669	2.74832891350669\\
66.875	0.26136	2.93515333460863	2.93515333460863\\
66.875	0.26502	3.1281835247914	3.1281835247914\\
66.875	0.26868	3.327419484055	3.327419484055\\
66.875	0.27234	3.53286121239942	3.53286121239942\\
66.875	0.276	3.74450870982468	3.74450870982468\\
67.25	0.093	0.80927262663426	0.80927262663426\\
67.25	0.09666	0.714478108988208	0.714478108988208\\
67.25	0.10032	0.625889360422978	0.625889360422978\\
67.25	0.10398	0.543506380938581	0.543506380938581\\
67.25	0.10764	0.467329170535012	0.467329170535012\\
67.25	0.1113	0.397357729212272	0.397357729212272\\
67.25	0.11496	0.333592056970365	0.333592056970365\\
67.25	0.11862	0.276032153809282	0.276032153809282\\
67.25	0.12228	0.224678019729028	0.224678019729028\\
67.25	0.12594	0.179529654729603	0.179529654729603\\
67.25	0.1296	0.140587058811006	0.140587058811006\\
67.25	0.13326	0.107850231973238	0.107850231973238\\
67.25	0.13692	0.0813191742162989	0.0813191742162989\\
67.25	0.14058	0.0609938855401904	0.0609938855401904\\
67.25	0.14424	0.0468743659449098	0.0468743659449098\\
67.25	0.1479	0.0389606154304563	0.0389606154304563\\
67.25	0.15156	0.0372526339968324	0.0372526339968324\\
67.25	0.15522	0.0417504216440374	0.0417504216440374\\
67.25	0.15888	0.0524539783720694	0.0524539783720694\\
67.25	0.16254	0.0693633041809312	0.0693633041809312\\
67.25	0.1662	0.0924783990706253	0.0924783990706253\\
67.25	0.16986	0.121799263041146	0.121799263041146\\
67.25	0.17352	0.157325896092494	0.157325896092494\\
67.25	0.17718	0.199058298224671	0.199058298224671\\
67.25	0.18084	0.246996469437676	0.246996469437676\\
67.25	0.1845	0.301140409731511	0.301140409731511\\
67.25	0.18816	0.361490119106175	0.361490119106175\\
67.25	0.19182	0.428045597561669	0.428045597561669\\
67.25	0.19548	0.500806845097989	0.500806845097989\\
67.25	0.19914	0.57977386171514	0.57977386171514\\
67.25	0.2028	0.664946647413119	0.664946647413119\\
67.25	0.20646	0.756325202191928	0.756325202191928\\
67.25	0.21012	0.853909526051561	0.853909526051561\\
67.25	0.21378	0.957699618992025	0.957699618992025\\
67.25	0.21744	1.06769548101332	1.06769548101332\\
67.25	0.2211	1.18389711211544	1.18389711211544\\
67.25	0.22476	1.30630451229839	1.30630451229839\\
67.25	0.22842	1.43491768156218	1.43491768156218\\
67.25	0.23208	1.56973661990678	1.56973661990678\\
67.25	0.23574	1.71076132733222	1.71076132733222\\
67.25	0.2394	1.85799180383849	1.85799180383849\\
67.25	0.24306	2.01142804942558	2.01142804942558\\
67.25	0.24672	2.1710700640935	2.1710700640935\\
67.25	0.25038	2.33691784784226	2.33691784784226\\
67.25	0.25404	2.50897140067184	2.50897140067184\\
67.25	0.2577	2.68723072258225	2.68723072258225\\
67.25	0.26136	2.87169581357348	2.87169581357348\\
67.25	0.26502	3.06236667364555	3.06236667364555\\
67.25	0.26868	3.25924330279845	3.25924330279845\\
67.25	0.27234	3.46232570103217	3.46232570103217\\
67.25	0.276	3.67161386834672	3.67161386834672\\
67.625	0.093	0.85531111286821	0.85531111286821\\
67.625	0.09666	0.758157265111451	0.758157265111451\\
67.625	0.10032	0.667209186435525	0.667209186435525\\
67.625	0.10398	0.582466876840424	0.582466876840424\\
67.625	0.10764	0.503930336326156	0.503930336326156\\
67.625	0.1113	0.431599564892712	0.431599564892712\\
67.625	0.11496	0.365474562540101	0.365474562540101\\
67.625	0.11862	0.305555329268318	0.305555329268318\\
67.625	0.12228	0.251841865077362	0.251841865077362\\
67.625	0.12594	0.204334169967233	0.204334169967233\\
67.625	0.1296	0.163032243937936	0.163032243937936\\
67.625	0.13326	0.127936086989468	0.127936086989468\\
67.625	0.13692	0.0990456991218256	0.0990456991218256\\
67.625	0.14058	0.0763610803350137	0.0763610803350137\\
67.625	0.14424	0.0598822306290332	0.0598822306290332\\
67.625	0.1479	0.0496091500038798	0.0496091500038798\\
67.625	0.15156	0.0455418384595525	0.0455418384595525\\
67.625	0.15522	0.0476802959960541	0.0476802959960541\\
67.625	0.15888	0.0560245226133862	0.0560245226133862\\
67.625	0.16254	0.070574518311548	0.070574518311548\\
67.625	0.1662	0.0913302830905387	0.0913302830905387\\
67.625	0.16986	0.118291816950359	0.118291816950359\\
67.625	0.17352	0.151459119891004	0.151459119891004\\
67.625	0.17718	0.190832191912481	0.190832191912481\\
67.625	0.18084	0.236411033014786	0.236411033014786\\
67.625	0.1845	0.288195643197914	0.288195643197914\\
67.625	0.18816	0.346186022461879	0.346186022461879\\
67.625	0.19182	0.410382170806669	0.410382170806669\\
67.625	0.19548	0.480784088232289	0.480784088232289\\
67.625	0.19914	0.55739177473874	0.55739177473874\\
67.625	0.2028	0.640205230326012	0.640205230326012\\
67.625	0.20646	0.729224454994121	0.729224454994121\\
67.625	0.21012	0.824449448743051	0.824449448743051\\
67.625	0.21378	0.925880211572816	0.925880211572816\\
67.625	0.21744	1.03351674348341	1.03351674348341\\
67.625	0.2211	1.14735904447483	1.14735904447483\\
67.625	0.22476	1.26740711454708	1.26740711454708\\
67.625	0.22842	1.39366095370016	1.39366095370016\\
67.625	0.23208	1.52612056193406	1.52612056193406\\
67.625	0.23574	1.6647859392488	1.6647859392488\\
67.625	0.2394	1.80965708564437	1.80965708564437\\
67.625	0.24306	1.96073400112076	1.96073400112076\\
67.625	0.24672	2.11801668567798	2.11801668567798\\
67.625	0.25038	2.28150513931603	2.28150513931603\\
67.625	0.25404	2.45119936203491	2.45119936203491\\
67.625	0.2577	2.62709935383462	2.62709935383462\\
67.625	0.26136	2.80920511471515	2.80920511471515\\
67.625	0.26502	2.99751664467652	2.99751664467652\\
67.625	0.26868	3.19203394371871	3.19203394371871\\
67.625	0.27234	3.39275701184173	3.39275701184173\\
67.625	0.276	3.59968584904559	3.59968584904559\\
68	0.093	0.902316421278967	0.902316421278967\\
68	0.09666	0.802803243411505	0.802803243411505\\
68	0.10032	0.709495834624879	0.709495834624879\\
68	0.10398	0.622394194919074	0.622394194919074\\
68	0.10764	0.541498324294106	0.541498324294106\\
68	0.1113	0.466808222749959	0.466808222749959\\
68	0.11496	0.398323890286645	0.398323890286645\\
68	0.11862	0.336045326904158	0.336045326904158\\
68	0.12228	0.279972532602502	0.279972532602502\\
68	0.12594	0.230105507381673	0.230105507381673\\
68	0.1296	0.186444251241676	0.186444251241676\\
68	0.13326	0.148988764182501	0.148988764182501\\
68	0.13692	0.117739046204159	0.117739046204159\\
68	0.14058	0.0926950973066472	0.0926950973066472\\
68	0.14424	0.0738569174899633	0.0738569174899633\\
68	0.1479	0.0612245067541064	0.0612245067541064\\
68	0.15156	0.0547978650990792	0.0547978650990792\\
68	0.15522	0.0545769925248809	0.0545769925248809\\
68	0.15888	0.0605618890315132	0.0605618890315132\\
68	0.16254	0.0727525546189716	0.0727525546189716\\
68	0.1662	0.0911489892872552	0.0911489892872552\\
68	0.16986	0.115751193036372	0.115751193036372\\
68	0.17352	0.146559165866321	0.146559165866321\\
68	0.17718	0.183572907777094	0.183572907777094\\
68	0.18084	0.226792418768696	0.226792418768696\\
68	0.1845	0.276217698841128	0.276217698841128\\
68	0.18816	0.331848747994385	0.331848747994385\\
68	0.19182	0.393685566228479	0.393685566228479\\
68	0.19548	0.461728153543392	0.461728153543392\\
68	0.19914	0.535976509939143	0.535976509939143\\
68	0.2028	0.616430635415716	0.616430635415716\\
68	0.20646	0.703090529973117	0.703090529973117\\
68	0.21012	0.795956193611351	0.795956193611351\\
68	0.21378	0.895027626330412	0.895027626330412\\
68	0.21744	1.0003048281303	1.0003048281303\\
68	0.2211	1.11178779901102	1.11178779901102\\
68	0.22476	1.22947653897257	1.22947653897257\\
68	0.22842	1.35337104801495	1.35337104801495\\
68	0.23208	1.48347132613815	1.48347132613815\\
68	0.23574	1.61977737334218	1.61977737334218\\
68	0.2394	1.76228918962705	1.76228918962705\\
68	0.24306	1.91100677499274	1.91100677499274\\
68	0.24672	2.06593012943926	2.06593012943926\\
68	0.25038	2.22705925296661	2.22705925296661\\
68	0.25404	2.39439414557478	2.39439414557478\\
68	0.2577	2.56793480726379	2.56793480726379\\
68	0.26136	2.74768123803363	2.74768123803363\\
68	0.26502	2.93363343788429	2.93363343788429\\
68	0.26868	3.12579140681578	3.12579140681578\\
68	0.27234	3.3241551448281	3.3241551448281\\
68	0.276	3.52872465192126	3.52872465192126\\
68.375	0.093	0.950288551866537	0.950288551866537\\
68.375	0.09666	0.848416043888379	0.848416043888379\\
68.375	0.10032	0.752749304991046	0.752749304991046\\
68.375	0.10398	0.663288335174542	0.663288335174542\\
68.375	0.10764	0.58003313443887	0.58003313443887\\
68.375	0.1113	0.502983702784023	0.502983702784023\\
68.375	0.11496	0.432140040210009	0.432140040210009\\
68.375	0.11862	0.367502146716819	0.367502146716819\\
68.375	0.12228	0.309070022304462	0.309070022304462\\
68.375	0.12594	0.256843666972931	0.256843666972931\\
68.375	0.1296	0.21082308072223	0.21082308072223\\
68.375	0.13326	0.171008263552356	0.171008263552356\\
68.375	0.13692	0.137399215463313	0.137399215463313\\
68.375	0.14058	0.109995936455094	0.109995936455094\\
68.375	0.14424	0.0887984265277106	0.0887984265277106\\
68.375	0.1479	0.0738066856811539	0.0738066856811539\\
68.375	0.15156	0.0650207139154233	0.0650207139154233\\
68.375	0.15522	0.0624405112305215	0.0624405112305215\\
68.375	0.15888	0.0660660776264539	0.0660660776264539\\
68.375	0.16254	0.0758974131032089	0.0758974131032089\\
68.375	0.1662	0.0919345176607962	0.0919345176607962\\
68.375	0.16986	0.11417739129921	0.11417739129921\\
68.375	0.17352	0.142626034018451	0.142626034018451\\
68.375	0.17718	0.177280445818525	0.177280445818525\\
68.375	0.18084	0.218140626699427	0.218140626699427\\
68.375	0.1845	0.265206576661155	0.265206576661155\\
68.375	0.18816	0.318478295703713	0.318478295703713\\
68.375	0.19182	0.377955783827099	0.377955783827099\\
68.375	0.19548	0.44363904103132	0.44363904103132\\
68.375	0.19914	0.515528067316364	0.515528067316364\\
68.375	0.2028	0.593622862682237	0.593622862682237\\
68.375	0.20646	0.677923427128938	0.677923427128938\\
68.375	0.21012	0.768429760656465	0.768429760656465\\
68.375	0.21378	0.865141863264826	0.865141863264826\\
68.375	0.21744	0.968059734954016	0.968059734954016\\
68.375	0.2211	1.07718337572403	1.07718337572403\\
68.375	0.22476	1.19251278557488	1.19251278557488\\
68.375	0.22842	1.31404796450656	1.31404796450656\\
68.375	0.23208	1.44178891251905	1.44178891251905\\
68.375	0.23574	1.57573562961239	1.57573562961239\\
68.375	0.2394	1.71588811578655	1.71588811578655\\
68.375	0.24306	1.86224637104154	1.86224637104154\\
68.375	0.24672	2.01481039537736	2.01481039537736\\
68.375	0.25038	2.17358018879401	2.17358018879401\\
68.375	0.25404	2.33855575129148	2.33855575129148\\
68.375	0.2577	2.50973708286979	2.50973708286979\\
68.375	0.26136	2.68712418352892	2.68712418352892\\
68.375	0.26502	2.87071705326888	2.87071705326888\\
68.375	0.26868	3.06051569208967	3.06051569208967\\
68.375	0.27234	3.25652009999129	3.25652009999129\\
68.375	0.276	3.45873027697374	3.45873027697374\\
68.75	0.093	0.999227504630922	0.999227504630922\\
68.75	0.09666	0.89499566654206	0.89499566654206\\
68.75	0.10032	0.796969597534024	0.796969597534024\\
68.75	0.10398	0.705149297606819	0.705149297606819\\
68.75	0.10764	0.619534766760448	0.619534766760448\\
68.75	0.1113	0.540126004994898	0.540126004994898\\
68.75	0.11496	0.46692301231018	0.46692301231018\\
68.75	0.11862	0.39992578870629	0.39992578870629\\
68.75	0.12228	0.33913433418323	0.33913433418323\\
68.75	0.12594	0.284548648740998	0.284548648740998\\
68.75	0.1296	0.236168732379594	0.236168732379594\\
68.75	0.13326	0.19399458509902	0.19399458509902\\
68.75	0.13692	0.158026206899274	0.158026206899274\\
68.75	0.14058	0.128263597780355	0.128263597780355\\
68.75	0.14424	0.104706757742268	0.104706757742268\\
68.75	0.1479	0.087355686785008	0.087355686785008\\
68.75	0.15156	0.0762103849085811	0.0762103849085811\\
68.75	0.15522	0.0712708521129795	0.0712708521129795\\
68.75	0.15888	0.0725370883982048	0.0725370883982048\\
68.75	0.16254	0.0800090937642599	0.0800090937642599\\
68.75	0.1662	0.0936868682111403	0.0936868682111403\\
68.75	0.16986	0.113570411738854	0.113570411738854\\
68.75	0.17352	0.139659724347395	0.139659724347395\\
68.75	0.17718	0.171954806036769	0.171954806036769\\
68.75	0.18084	0.210455656806968	0.210455656806968\\
68.75	0.1845	0.255162276657993	0.255162276657993\\
68.75	0.18816	0.306074665589851	0.306074665589851\\
68.75	0.19182	0.363192823602537	0.363192823602537\\
68.75	0.19548	0.426516750696051	0.426516750696051\\
68.75	0.19914	0.496046446870395	0.496046446870395\\
68.75	0.2028	0.571781912125568	0.571781912125568\\
68.75	0.20646	0.653723146461569	0.653723146461569\\
68.75	0.21012	0.741870149878393	0.741870149878393\\
68.75	0.21378	0.83622292237605	0.83622292237605\\
68.75	0.21744	0.936781463954537	0.936781463954537\\
68.75	0.2211	1.04354577461385	1.04354577461385\\
68.75	0.22476	1.15651585435399	1.15651585435399\\
68.75	0.22842	1.27569170317497	1.27569170317497\\
68.75	0.23208	1.40107332107677	1.40107332107677\\
68.75	0.23574	1.5326607080594	1.5326607080594\\
68.75	0.2394	1.67045386412286	1.67045386412286\\
68.75	0.24306	1.81445278926715	1.81445278926715\\
68.75	0.24672	1.96465748349226	1.96465748349226\\
68.75	0.25038	2.12106794679821	2.12106794679821\\
68.75	0.25404	2.28368417918498	2.28368417918498\\
68.75	0.2577	2.45250618065259	2.45250618065259\\
68.75	0.26136	2.62753395120102	2.62753395120102\\
68.75	0.26502	2.80876749083028	2.80876749083028\\
68.75	0.26868	2.99620679954037	2.99620679954037\\
68.75	0.27234	3.18985187733128	3.18985187733128\\
68.75	0.276	3.38970272420303	3.38970272420303\\
69.125	0.093	1.04913327957212	1.04913327957212\\
69.125	0.09666	0.942542111372555	0.942542111372555\\
69.125	0.10032	0.842156712253819	0.842156712253819\\
69.125	0.10398	0.747977082215911	0.747977082215911\\
69.125	0.10764	0.660003221258836	0.660003221258836\\
69.125	0.1113	0.578235129382586	0.578235129382586\\
69.125	0.11496	0.502672806587165	0.502672806587165\\
69.125	0.11862	0.433316252872575	0.433316252872575\\
69.125	0.12228	0.370165468238811	0.370165468238811\\
69.125	0.12594	0.313220452685876	0.313220452685876\\
69.125	0.1296	0.262481206213772	0.262481206213772\\
69.125	0.13326	0.217947728822494	0.217947728822494\\
69.125	0.13692	0.179620020512049	0.179620020512049\\
69.125	0.14058	0.147498081282427	0.147498081282427\\
69.125	0.14424	0.12158191113364	0.12158191113364\\
69.125	0.1479	0.101871510065676	0.101871510065676\\
69.125	0.15156	0.0883668780785456	0.0883668780785456\\
69.125	0.15522	0.081068015172244	0.081068015172244\\
69.125	0.15888	0.079974921346766	0.079974921346766\\
69.125	0.16254	0.0850875966021212	0.0850875966021212\\
69.125	0.1662	0.0964060409383052	0.0964060409383052\\
69.125	0.16986	0.113930254355312	0.113930254355312\\
69.125	0.17352	0.137660236853153	0.137660236853153\\
69.125	0.17718	0.16759598843182	0.16759598843182\\
69.125	0.18084	0.203737509091319	0.203737509091319\\
69.125	0.1845	0.246084798831644	0.246084798831644\\
69.125	0.18816	0.294637857652798	0.294637857652798\\
69.125	0.19182	0.349396685554781	0.349396685554781\\
69.125	0.19548	0.410361282537595	0.410361282537595\\
69.125	0.19914	0.477531648601239	0.477531648601239\\
69.125	0.2028	0.550907783745705	0.550907783745705\\
69.125	0.20646	0.630489687971007	0.630489687971007\\
69.125	0.21012	0.716277361277131	0.716277361277131\\
69.125	0.21378	0.808270803664085	0.808270803664085\\
69.125	0.21744	0.906470015131871	0.906470015131871\\
69.125	0.2211	1.01087499568048	1.01087499568048\\
69.125	0.22476	1.12148574530993	1.12148574530993\\
69.125	0.22842	1.2383022640202	1.2383022640202\\
69.125	0.23208	1.3613245518113	1.3613245518113\\
69.125	0.23574	1.49055260868323	1.49055260868323\\
69.125	0.2394	1.62598643463599	1.62598643463599\\
69.125	0.24306	1.76762602966957	1.76762602966957\\
69.125	0.24672	1.91547139378399	1.91547139378399\\
69.125	0.25038	2.06952252697923	2.06952252697923\\
69.125	0.25404	2.2297794292553	2.2297794292553\\
69.125	0.2577	2.3962421006122	2.3962421006122\\
69.125	0.26136	2.56891054104993	2.56891054104993\\
69.125	0.26502	2.74778475056849	2.74778475056849\\
69.125	0.26868	2.93286472916787	2.93286472916787\\
69.125	0.27234	3.12415047684809	3.12415047684809\\
69.125	0.276	3.32164199360913	3.32164199360913\\
69.5	0.093	1.10000587669013	1.10000587669013\\
69.5	0.09666	0.991055378379863	0.991055378379863\\
69.5	0.10032	0.888310649150427	0.888310649150427\\
69.5	0.10398	0.791771689001816	0.791771689001816\\
69.5	0.10764	0.701438497934038	0.701438497934038\\
69.5	0.1113	0.617311075947088	0.617311075947088\\
69.5	0.11496	0.539389423040963	0.539389423040963\\
69.5	0.11862	0.467673539215673	0.467673539215673\\
69.5	0.12228	0.402163424471206	0.402163424471206\\
69.5	0.12594	0.342859078807575	0.342859078807575\\
69.5	0.1296	0.289760502224764	0.289760502224764\\
69.5	0.13326	0.242867694722786	0.242867694722786\\
69.5	0.13692	0.202180656301637	0.202180656301637\\
69.5	0.14058	0.167699386961315	0.167699386961315\\
69.5	0.14424	0.139423886701825	0.139423886701825\\
69.5	0.1479	0.117354155523161	0.117354155523161\\
69.5	0.15156	0.101490193425327	0.101490193425327\\
69.5	0.15522	0.0918320004083224	0.0918320004083224\\
69.5	0.15888	0.0883795764721444	0.0883795764721444\\
69.5	0.16254	0.0911329216167962	0.0911329216167962\\
69.5	0.1662	0.100092035842277	0.100092035842277\\
69.5	0.16986	0.115256919148587	0.115256919148587\\
69.5	0.17352	0.136627571535725	0.136627571535725\\
69.5	0.17718	0.164203993003692	0.164203993003692\\
69.5	0.18084	0.197986183552487	0.197986183552487\\
69.5	0.1845	0.237974143182109	0.237974143182109\\
69.5	0.18816	0.284167871892564	0.284167871892564\\
69.5	0.19182	0.336567369683847	0.336567369683847\\
69.5	0.19548	0.395172636555953	0.395172636555953\\
69.5	0.19914	0.459983672508898	0.459983672508898\\
69.5	0.2028	0.531000477542664	0.531000477542664\\
69.5	0.20646	0.608223051657259	0.608223051657259\\
69.5	0.21012	0.691651394852682	0.691651394852682\\
69.5	0.21378	0.781285507128937	0.781285507128937\\
69.5	0.21744	0.87712538848602	0.87712538848602\\
69.5	0.2211	0.979171038923932	0.979171038923932\\
69.5	0.22476	1.08742245844268	1.08742245844268\\
69.5	0.22842	1.20187964704224	1.20187964704224\\
69.5	0.23208	1.32254260472264	1.32254260472264\\
69.5	0.23574	1.44941133148387	1.44941133148387\\
69.5	0.2394	1.58248582732592	1.58248582732592\\
69.5	0.24306	1.72176609224881	1.72176609224881\\
69.5	0.24672	1.86725212625252	1.86725212625252\\
69.5	0.25038	2.01894392933706	2.01894392933706\\
69.5	0.25404	2.17684150150243	2.17684150150243\\
69.5	0.2577	2.34094484274863	2.34094484274863\\
69.5	0.26136	2.51125395307566	2.51125395307566\\
69.5	0.26502	2.68776883248352	2.68776883248352\\
69.5	0.26868	2.8704894809722	2.8704894809722\\
69.5	0.27234	3.05941589854171	3.05941589854171\\
69.5	0.276	3.25454808519206	3.25454808519206\\
69.875	0.093	1.15184529598495	1.15184529598495\\
69.875	0.09666	1.04053546756398	1.04053546756398\\
69.875	0.10032	0.935431408223843	0.935431408223843\\
69.875	0.10398	0.836533117964531	0.836533117964531\\
69.875	0.10764	0.743840596786053	0.743840596786053\\
69.875	0.1113	0.6573538446884	0.6573538446884\\
69.875	0.11496	0.577072861671575	0.577072861671575\\
69.875	0.11862	0.502997647735582	0.502997647735582\\
69.875	0.12228	0.435128202880415	0.435128202880415\\
69.875	0.12594	0.373464527106077	0.373464527106077\\
69.875	0.1296	0.318006620412566	0.318006620412566\\
69.875	0.13326	0.268754482799888	0.268754482799888\\
69.875	0.13692	0.225708114268036	0.225708114268036\\
69.875	0.14058	0.188867514817014	0.188867514817014\\
69.875	0.14424	0.15823268444682	0.15823268444682\\
69.875	0.1479	0.133803623157457	0.133803623157457\\
69.875	0.15156	0.115580330948919	0.115580330948919\\
69.875	0.15522	0.103562807821211	0.103562807821211\\
69.875	0.15888	0.0977510537743331	0.0977510537743331\\
69.875	0.16254	0.098145068808285	0.098145068808285\\
69.875	0.1662	0.104744852923066	0.104744852923066\\
69.875	0.16986	0.117550406118669	0.117550406118669\\
69.875	0.17352	0.136561728395107	0.136561728395107\\
69.875	0.17718	0.161778819752371	0.161778819752371\\
69.875	0.18084	0.193201680190466	0.193201680190466\\
69.875	0.1845	0.230830309709384	0.230830309709384\\
69.875	0.18816	0.274664708309139	0.274664708309139\\
69.875	0.19182	0.324704875989715	0.324704875989715\\
69.875	0.19548	0.380950812751129	0.380950812751129\\
69.875	0.19914	0.443402518593366	0.443402518593366\\
69.875	0.2028	0.512059993516432	0.512059993516432\\
69.875	0.20646	0.586923237520327	0.586923237520327\\
69.875	0.21012	0.667992250605048	0.667992250605048\\
69.875	0.21378	0.755267032770599	0.755267032770599\\
69.875	0.21744	0.848747584016982	0.848747584016982\\
69.875	0.2211	0.948433904344194	0.948433904344194\\
69.875	0.22476	1.05432599375223	1.05432599375223\\
69.875	0.22842	1.1664238522411	1.1664238522411\\
69.875	0.23208	1.28472747981079	1.28472747981079\\
69.875	0.23574	1.40923687646132	1.40923687646132\\
69.875	0.2394	1.53995204219267	1.53995204219267\\
69.875	0.24306	1.67687297700486	1.67687297700486\\
69.875	0.24672	1.81999968089787	1.81999968089787\\
69.875	0.25038	1.96933215387171	1.96933215387171\\
69.875	0.25404	2.12487039592637	2.12487039592637\\
69.875	0.2577	2.28661440706187	2.28661440706187\\
69.875	0.26136	2.4545641872782	2.4545641872782\\
69.875	0.26502	2.62871973657535	2.62871973657535\\
69.875	0.26868	2.80908105495334	2.80908105495334\\
69.875	0.27234	2.99564814241214	2.99564814241214\\
69.875	0.276	3.18842099895179	3.18842099895179\\
70.25	0.093	1.20465153745658	1.20465153745658\\
70.25	0.09666	1.09098237892491	1.09098237892491\\
70.25	0.10032	0.983518989474072	0.983518989474072\\
70.25	0.10398	0.88226136910406	0.88226136910406\\
70.25	0.10764	0.787209517814879	0.787209517814879\\
70.25	0.1113	0.698363435606526	0.698363435606526\\
70.25	0.11496	0.615723122479001	0.615723122479001\\
70.25	0.11862	0.539288578432305	0.539288578432305\\
70.25	0.12228	0.469059803466434	0.469059803466434\\
70.25	0.12594	0.405036797581396	0.405036797581396\\
70.25	0.1296	0.347219560777186	0.347219560777186\\
70.25	0.13326	0.295608093053801	0.295608093053801\\
70.25	0.13692	0.250202394411248	0.250202394411248\\
70.25	0.14058	0.211002464849527	0.211002464849527\\
70.25	0.14424	0.178008304368633	0.178008304368633\\
70.25	0.1479	0.151219912968566	0.151219912968566\\
70.25	0.15156	0.130637290649329	0.130637290649329\\
70.25	0.15522	0.11626043741092	0.11626043741092\\
70.25	0.15888	0.108089353253339	0.108089353253339\\
70.25	0.16254	0.106124038176588	0.106124038176588\\
70.25	0.1662	0.110364492180661	0.110364492180661\\
70.25	0.16986	0.120810715265568	0.120810715265568\\
70.25	0.17352	0.137462707431303	0.137462707431303\\
70.25	0.17718	0.160320468677867	0.160320468677867\\
70.25	0.18084	0.189383999005258	0.189383999005258\\
70.25	0.1845	0.224653298413481	0.224653298413481\\
70.25	0.18816	0.266128366902528	0.266128366902528\\
70.25	0.19182	0.313809204472404	0.313809204472404\\
70.25	0.19548	0.367695811123111	0.367695811123111\\
70.25	0.19914	0.427788186854649	0.427788186854649\\
70.25	0.2028	0.494086331667015	0.494086331667015\\
70.25	0.20646	0.56659024556021	0.56659024556021\\
70.25	0.21012	0.645299928534227	0.645299928534227\\
70.25	0.21378	0.730215380589078	0.730215380589078\\
70.25	0.21744	0.821336601724758	0.821336601724758\\
70.25	0.2211	0.918663591941263	0.918663591941263\\
70.25	0.22476	1.0221963512386	1.0221963512386\\
70.25	0.22842	1.13193487961677	1.13193487961677\\
70.25	0.23208	1.24787917707576	1.24787917707576\\
70.25	0.23574	1.37002924361559	1.37002924361559\\
70.25	0.2394	1.49838507923624	1.49838507923624\\
70.25	0.24306	1.63294668393772	1.63294668393772\\
70.25	0.24672	1.77371405772003	1.77371405772003\\
70.25	0.25038	1.92068720058317	1.92068720058317\\
70.25	0.25404	2.07386611252713	2.07386611252713\\
70.25	0.2577	2.23325079355193	2.23325079355193\\
70.25	0.26136	2.39884124365755	2.39884124365755\\
70.25	0.26502	2.57063746284401	2.57063746284401\\
70.25	0.26868	2.74863945111128	2.74863945111128\\
70.25	0.27234	2.9328472084594	2.9328472084594\\
70.25	0.276	3.12326073488833	3.12326073488833\\
70.625	0.093	1.25842460110503	1.25842460110503\\
70.625	0.09666	1.14239611246266	1.14239611246266\\
70.625	0.10032	1.03257339290111	1.03257339290111\\
70.625	0.10398	0.9289564424204	0.9289564424204\\
70.625	0.10764	0.831545261020515	0.831545261020515\\
70.625	0.1113	0.740339848701458	0.740339848701458\\
70.625	0.11496	0.655340205463234	0.655340205463234\\
70.625	0.11862	0.576546331305834	0.576546331305834\\
70.625	0.12228	0.503958226229264	0.503958226229264\\
70.625	0.12594	0.437575890233525	0.437575890233525\\
70.625	0.1296	0.377399323318611	0.377399323318611\\
70.625	0.13326	0.323428525484527	0.323428525484527\\
70.625	0.13692	0.275663496731271	0.275663496731271\\
70.625	0.14058	0.234104237058846	0.234104237058846\\
70.625	0.14424	0.198750746467248	0.198750746467248\\
70.625	0.1479	0.169603024956482	0.169603024956482\\
70.625	0.15156	0.146661072526541	0.146661072526541\\
70.625	0.15522	0.129924889177426	0.129924889177426\\
70.625	0.15888	0.119394474909148	0.119394474909148\\
70.625	0.16254	0.115069829721693	0.115069829721693\\
70.625	0.1662	0.116950953615067	0.116950953615067\\
70.625	0.16986	0.125037846589271	0.125037846589271\\
70.625	0.17352	0.139330508644306	0.139330508644306\\
70.625	0.17718	0.159828939780166	0.159828939780166\\
70.625	0.18084	0.186533139996857	0.186533139996857\\
70.625	0.1845	0.219443109294373	0.219443109294373\\
70.625	0.18816	0.258558847672724	0.258558847672724\\
70.625	0.19182	0.303880355131897	0.303880355131897\\
70.625	0.19548	0.355407631671904	0.355407631671904\\
70.625	0.19914	0.413140677292741	0.413140677292741\\
70.625	0.2028	0.477079491994401	0.477079491994401\\
70.625	0.20646	0.547224075776896	0.547224075776896\\
70.625	0.21012	0.623574428640209	0.623574428640209\\
70.625	0.21378	0.70613055058436	0.70613055058436\\
70.625	0.21744	0.794892441609337	0.794892441609337\\
70.625	0.2211	0.889860101715142	0.889860101715142\\
70.625	0.22476	0.991033530901781	0.991033530901781\\
70.625	0.22842	1.09841272916925	1.09841272916925\\
70.625	0.23208	1.21199769651753	1.21199769651753\\
70.625	0.23574	1.33178843294666	1.33178843294666\\
70.625	0.2394	1.45778493845661	1.45778493845661\\
70.625	0.24306	1.58998721304738	1.58998721304738\\
70.625	0.24672	1.72839525671899	1.72839525671899\\
70.625	0.25038	1.87300906947143	1.87300906947143\\
70.625	0.25404	2.02382865130469	2.02382865130469\\
70.625	0.2577	2.18085400221879	2.18085400221879\\
70.625	0.26136	2.34408512221371	2.34408512221371\\
70.625	0.26502	2.51352201128946	2.51352201128946\\
70.625	0.26868	2.68916466944604	2.68916466944604\\
70.625	0.27234	2.87101309668345	2.87101309668345\\
70.625	0.276	3.05906729300169	3.05906729300169\\
71	0.093	1.31316448693029	1.31316448693029\\
71	0.09666	1.19477666817721	1.19477666817721\\
71	0.10032	1.08259461850497	1.08259461850497\\
71	0.10398	0.976618337913553	0.976618337913553\\
71	0.10764	0.876847826402968	0.876847826402968\\
71	0.1113	0.783283083973208	0.783283083973208\\
71	0.11496	0.69592411062428	0.69592411062428\\
71	0.11862	0.61477090635618	0.61477090635618\\
71	0.12228	0.53982347116891	0.53982347116891\\
71	0.12594	0.471081805062465	0.471081805062465\\
71	0.1296	0.408545908036851	0.408545908036851\\
71	0.13326	0.352215780092067	0.352215780092067\\
71	0.13692	0.302091421228107	0.302091421228107\\
71	0.14058	0.258172831444979	0.258172831444979\\
71	0.14424	0.220460010742682	0.220460010742682\\
71	0.1479	0.188952959121211	0.188952959121211\\
71	0.15156	0.163651676580571	0.163651676580571\\
71	0.15522	0.144556163120759	0.144556163120759\\
71	0.15888	0.131666418741775	0.131666418741775\\
71	0.16254	0.12498244344362	0.12498244344362\\
71	0.1662	0.124504237226294	0.124504237226294\\
71	0.16986	0.130231800089794	0.130231800089794\\
71	0.17352	0.142165132034125	0.142165132034125\\
71	0.17718	0.160304233059286	0.160304233059286\\
71	0.18084	0.184649103165274	0.184649103165274\\
71	0.1845	0.215199742352086	0.215199742352086\\
71	0.18816	0.251956150619733	0.251956150619733\\
71	0.19182	0.29491832796821	0.29491832796821\\
71	0.19548	0.34408627439751	0.34408627439751\\
71	0.19914	0.399459989907648	0.399459989907648\\
71	0.2028	0.461039474498607	0.461039474498607\\
71	0.20646	0.528824728170395	0.528824728170395\\
71	0.21012	0.602815750923012	0.602815750923012\\
71	0.21378	0.68301254275646	0.68301254275646\\
71	0.21744	0.769415103670736	0.769415103670736\\
71	0.2211	0.862023433665842	0.862023433665842\\
71	0.22476	0.960837532741774	0.960837532741774\\
71	0.22842	1.06585740089854	1.06585740089854\\
71	0.23208	1.17708303813612	1.17708303813612\\
71	0.23574	1.29451444445455	1.29451444445455\\
71	0.2394	1.41815161985379	1.41815161985379\\
71	0.24306	1.54799456433387	1.54799456433387\\
71	0.24672	1.68404327789478	1.68404327789478\\
71	0.25038	1.82629776053651	1.82629776053651\\
71	0.25404	1.97475801225907	1.97475801225907\\
71	0.2577	2.12942403306247	2.12942403306247\\
71	0.26136	2.29029582294669	2.29029582294669\\
71	0.26502	2.45737338191174	2.45737338191174\\
71	0.26868	2.63065670995761	2.63065670995761\\
71	0.27234	2.81014580708432	2.81014580708432\\
71	0.276	2.99584067329186	2.99584067329186\\
71.375	0.093	1.36887119493236	1.36887119493236\\
71.375	0.09666	1.24812404606858	1.24812404606858\\
71.375	0.10032	1.13358266628563	1.13358266628563\\
71.375	0.10398	1.02524705558352	1.02524705558352\\
71.375	0.10764	0.923117213962228	0.923117213962228\\
71.375	0.1113	0.827193141421768	0.827193141421768\\
71.375	0.11496	0.737474837962137	0.737474837962137\\
71.375	0.11862	0.653962303583337	0.653962303583337\\
71.375	0.12228	0.576655538285364	0.576655538285364\\
71.375	0.12594	0.505554542068218	0.505554542068218\\
71.375	0.1296	0.440659314931905	0.440659314931905\\
71.375	0.13326	0.381969856876413	0.381969856876413\\
71.375	0.13692	0.329486167901758	0.329486167901758\\
71.375	0.14058	0.283208248007925	0.283208248007925\\
71.375	0.14424	0.243136097194928	0.243136097194928\\
71.375	0.1479	0.209269715462755	0.209269715462755\\
71.375	0.15156	0.181609102811411	0.181609102811411\\
71.375	0.15522	0.160154259240896	0.160154259240896\\
71.375	0.15888	0.144905184751211	0.144905184751211\\
71.375	0.16254	0.135861879342356	0.135861879342356\\
71.375	0.1662	0.133024343014323	0.133024343014323\\
71.375	0.16986	0.136392575767124	0.136392575767124\\
71.375	0.17352	0.145966577600755	0.145966577600755\\
71.375	0.17718	0.161746348515212	0.161746348515212\\
71.375	0.18084	0.183731888510497	0.183731888510497\\
71.375	0.1845	0.211923197586616	0.211923197586616\\
71.375	0.18816	0.246320275743557	0.246320275743557\\
71.375	0.19182	0.286923122981333	0.286923122981333\\
71.375	0.19548	0.333731739299933	0.333731739299933\\
71.375	0.19914	0.386746124699364	0.386746124699364\\
71.375	0.2028	0.445966279179624	0.445966279179624\\
71.375	0.20646	0.511392202740712	0.511392202740712\\
71.375	0.21012	0.583023895382626	0.583023895382626\\
71.375	0.21378	0.66086135710537	0.66086135710537\\
71.375	0.21744	0.744904587908943	0.744904587908943\\
71.375	0.2211	0.835153587793345	0.835153587793345\\
71.375	0.22476	0.931608356758577	0.931608356758577\\
71.375	0.22842	1.03426889480464	1.03426889480464\\
71.375	0.23208	1.14313520193153	1.14313520193153\\
71.375	0.23574	1.25820727813924	1.25820727813924\\
71.375	0.2394	1.3794851234278	1.3794851234278\\
71.375	0.24306	1.50696873779717	1.50696873779717\\
71.375	0.24672	1.64065812124737	1.64065812124737\\
71.375	0.25038	1.78055327377841	1.78055327377841\\
71.375	0.25404	1.92665419539026	1.92665419539026\\
71.375	0.2577	2.07896088608296	2.07896088608296\\
71.375	0.26136	2.23747334585648	2.23747334585648\\
71.375	0.26502	2.40219157471082	2.40219157471082\\
71.375	0.26868	2.573115572646	2.573115572646\\
71.375	0.27234	2.750245339662	2.750245339662\\
71.375	0.276	2.93358087575884	2.93358087575884\\
71.75	0.093	1.42554472511124	1.42554472511124\\
71.75	0.09666	1.30243824613676	1.30243824613676\\
71.75	0.10032	1.18553753624311	1.18553753624311\\
71.75	0.10398	1.07484259543029	1.07484259543029\\
71.75	0.10764	0.970353423698302	0.970353423698302\\
71.75	0.1113	0.872070021047142	0.872070021047142\\
71.75	0.11496	0.779992387476807	0.779992387476807\\
71.75	0.11862	0.694120522987304	0.694120522987304\\
71.75	0.12228	0.614454427578631	0.614454427578631\\
71.75	0.12594	0.540994101250786	0.540994101250786\\
71.75	0.1296	0.473739544003768	0.473739544003768\\
71.75	0.13326	0.412690755837577	0.412690755837577\\
71.75	0.13692	0.357847736752214	0.357847736752214\\
71.75	0.14058	0.309210486747682	0.309210486747682\\
71.75	0.14424	0.266779005823982	0.266779005823982\\
71.75	0.1479	0.230553293981108	0.230553293981108\\
71.75	0.15156	0.200533351219065	0.200533351219065\\
71.75	0.15522	0.176719177537846	0.176719177537846\\
71.75	0.15888	0.159110772937462	0.159110772937462\\
71.75	0.16254	0.1477081374179	0.1477081374179\\
71.75	0.1662	0.142511270979167	0.142511270979167\\
71.75	0.16986	0.143520173621267	0.143520173621267\\
71.75	0.17352	0.150734845344195	0.150734845344195\\
71.75	0.17718	0.164155286147952	0.164155286147952\\
71.75	0.18084	0.183781496032537	0.183781496032537\\
71.75	0.1845	0.209613474997946	0.209613474997946\\
71.75	0.18816	0.24165122304419	0.24165122304419\\
71.75	0.19182	0.279894740171263	0.279894740171263\\
71.75	0.19548	0.324344026379164	0.324344026379164\\
71.75	0.19914	0.374999081667895	0.374999081667895\\
71.75	0.2028	0.431859906037447	0.431859906037447\\
71.75	0.20646	0.494926499487836	0.494926499487836\\
71.75	0.21012	0.564198862019046	0.564198862019046\\
71.75	0.21378	0.63967699363109	0.63967699363109\\
71.75	0.21744	0.721360894323963	0.721360894323963\\
71.75	0.2211	0.809250564097661	0.809250564097661\\
71.75	0.22476	0.903346002952194	0.903346002952194\\
71.75	0.22842	1.00364721088755	1.00364721088755\\
71.75	0.23208	1.11015418790374	1.11015418790374\\
71.75	0.23574	1.22286693400075	1.22286693400075\\
71.75	0.2394	1.3417854491786	1.3417854491786\\
71.75	0.24306	1.46690973343727	1.46690973343727\\
71.75	0.24672	1.59823978677678	1.59823978677678\\
71.75	0.25038	1.73577560919711	1.73577560919711\\
71.75	0.25404	1.87951720069827	1.87951720069827\\
71.75	0.2577	2.02946456128026	2.02946456128026\\
71.75	0.26136	2.18561769094307	2.18561769094307\\
71.75	0.26502	2.34797658968672	2.34797658968672\\
71.75	0.26868	2.51654125751119	2.51654125751119\\
71.75	0.27234	2.69131169441649	2.69131169441649\\
71.75	0.276	2.87228790040263	2.87228790040263\\
72.125	0.093	1.48318507746693	1.48318507746693\\
72.125	0.09666	1.35771926838175	1.35771926838175\\
72.125	0.10032	1.2384592283774	1.2384592283774\\
72.125	0.10398	1.12540495745388	1.12540495745388\\
72.125	0.10764	1.01855645561119	1.01855645561119\\
72.125	0.1113	0.917913722849329	0.917913722849329\\
72.125	0.11496	0.823476759168295	0.823476759168295\\
72.125	0.11862	0.735245564568088	0.735245564568088\\
72.125	0.12228	0.653220139048711	0.653220139048711\\
72.125	0.12594	0.577400482610163	0.577400482610163\\
72.125	0.1296	0.507786595252446	0.507786595252446\\
72.125	0.13326	0.444378476975551	0.444378476975551\\
72.125	0.13692	0.387176127779492	0.387176127779492\\
72.125	0.14058	0.336179547664257	0.336179547664257\\
72.125	0.14424	0.291388736629853	0.291388736629853\\
72.125	0.1479	0.252803694676279	0.252803694676279\\
72.125	0.15156	0.220424421803532	0.220424421803532\\
72.125	0.15522	0.19425091801161	0.19425091801161\\
72.125	0.15888	0.174283183300526	0.174283183300526\\
72.125	0.16254	0.160521217670264	0.160521217670264\\
72.125	0.1662	0.152965021120831	0.152965021120831\\
72.125	0.16986	0.151614593652228	0.151614593652228\\
72.125	0.17352	0.156469935264453	0.156469935264453\\
72.125	0.17718	0.167531045957507	0.167531045957507\\
72.125	0.18084	0.184797925731392	0.184797925731392\\
72.125	0.1845	0.208270574586104	0.208270574586104\\
72.125	0.18816	0.237948992521641	0.237948992521641\\
72.125	0.19182	0.273833179538007	0.273833179538007\\
72.125	0.19548	0.315923135635208	0.315923135635208\\
72.125	0.19914	0.364218860813239	0.364218860813239\\
72.125	0.2028	0.418720355072091	0.418720355072091\\
72.125	0.20646	0.47942761841178	0.47942761841178\\
72.125	0.21012	0.546340650832287	0.546340650832287\\
72.125	0.21378	0.619459452333627	0.619459452333627\\
72.125	0.21744	0.698784022915801	0.698784022915801\\
72.125	0.2211	0.784314362578799	0.784314362578799\\
72.125	0.22476	0.876050471322625	0.876050471322625\\
72.125	0.22842	0.973992349147284	0.973992349147284\\
72.125	0.23208	1.07813999605277	1.07813999605277\\
72.125	0.23574	1.18849341203908	1.18849341203908\\
72.125	0.2394	1.30505259710623	1.30505259710623\\
72.125	0.24306	1.4278175512542	1.4278175512542\\
72.125	0.24672	1.556788274483	1.556788274483\\
72.125	0.25038	1.69196476679263	1.69196476679263\\
72.125	0.25404	1.83334702818308	1.83334702818308\\
72.125	0.2577	1.98093505865437	1.98093505865437\\
72.125	0.26136	2.13472885820649	2.13472885820649\\
72.125	0.26502	2.29472842683943	2.29472842683943\\
72.125	0.26868	2.4609337645532	2.4609337645532\\
72.125	0.27234	2.6333448713478	2.6333448713478\\
72.125	0.276	2.81196174722324	2.81196174722324\\
72.5	0.093	1.54179225199944	1.54179225199944\\
72.5	0.09666	1.41396711280356	1.41396711280356\\
72.5	0.10032	1.29234774268851	1.29234774268851\\
72.5	0.10398	1.17693414165429	1.17693414165429\\
72.5	0.10764	1.06772630970089	1.06772630970089\\
72.5	0.1113	0.964724246828327	0.964724246828327\\
72.5	0.11496	0.867927953036593	0.867927953036593\\
72.5	0.11862	0.777337428325683	0.777337428325683\\
72.5	0.12228	0.692952672695606	0.692952672695606\\
72.5	0.12594	0.614773686146354	0.614773686146354\\
72.5	0.1296	0.542800468677934	0.542800468677934\\
72.5	0.13326	0.477033020290339	0.477033020290339\\
72.5	0.13692	0.417471340983576	0.417471340983576\\
72.5	0.14058	0.364115430757641	0.364115430757641\\
72.5	0.14424	0.316965289612537	0.316965289612537\\
72.5	0.1479	0.276020917548257	0.276020917548257\\
72.5	0.15156	0.24128231456481	0.24128231456481\\
72.5	0.15522	0.212749480662191	0.212749480662191\\
72.5	0.15888	0.1904224158404	0.1904224158404\\
72.5	0.16254	0.174301120099439	0.174301120099439\\
72.5	0.1662	0.164385593439299	0.164385593439299\\
72.5	0.16986	0.160675835859996	0.160675835859996\\
72.5	0.17352	0.163171847361521	0.163171847361521\\
72.5	0.17718	0.171873627943874	0.171873627943874\\
72.5	0.18084	0.186781177607052	0.186781177607052\\
72.5	0.1845	0.207894496351065	0.207894496351065\\
72.5	0.18816	0.235213584175902	0.235213584175902\\
72.5	0.19182	0.268738441081569	0.268738441081569\\
72.5	0.19548	0.308469067068069	0.308469067068069\\
72.5	0.19914	0.354405462135393	0.354405462135393\\
72.5	0.2028	0.406547626283546	0.406547626283546\\
72.5	0.20646	0.464895559512527	0.464895559512527\\
72.5	0.21012	0.529449261822338	0.529449261822338\\
72.5	0.21378	0.600208733212975	0.600208733212975\\
72.5	0.21744	0.677173973684445	0.677173973684445\\
72.5	0.2211	0.76034498323674	0.76034498323674\\
72.5	0.22476	0.849721761869873	0.849721761869873\\
72.5	0.22842	0.945304309583825	0.945304309583825\\
72.5	0.23208	1.04709262637861	1.04709262637861\\
72.5	0.23574	1.15508671225422	1.15508671225422\\
72.5	0.2394	1.26928656721067	1.26928656721067\\
72.5	0.24306	1.38969219124793	1.38969219124793\\
72.5	0.24672	1.51630358436603	1.51630358436603\\
72.5	0.25038	1.64912074656496	1.64912074656496\\
72.5	0.25404	1.78814367784471	1.78814367784471\\
72.5	0.2577	1.9333723782053	1.9333723782053\\
72.5	0.26136	2.08480684764671	2.08480684764671\\
72.5	0.26502	2.24244708616896	2.24244708616896\\
72.5	0.26868	2.40629309377202	2.40629309377202\\
72.5	0.27234	2.57634487045592	2.57634487045592\\
72.5	0.276	2.75260241622065	2.75260241622065\\
72.875	0.093	1.60136624870876	1.60136624870876\\
72.875	0.09666	1.47118177940218	1.47118177940218\\
72.875	0.10032	1.34720307917642	1.34720307917642\\
72.875	0.10398	1.2294301480315	1.2294301480315\\
72.875	0.10764	1.11786298596741	1.11786298596741\\
72.875	0.1113	1.01250159298414	1.01250159298414\\
72.875	0.11496	0.913345969081701	0.913345969081701\\
72.875	0.11862	0.820396114260091	0.820396114260091\\
72.875	0.12228	0.733652028519311	0.733652028519311\\
72.875	0.12594	0.653113711859359	0.653113711859359\\
72.875	0.1296	0.578781164280235	0.578781164280235\\
72.875	0.13326	0.51065438578194	0.51065438578194\\
72.875	0.13692	0.448733376364475	0.448733376364475\\
72.875	0.14058	0.393018136027836	0.393018136027836\\
72.875	0.14424	0.343508664772028	0.343508664772028\\
72.875	0.1479	0.300204962597048	0.300204962597048\\
72.875	0.15156	0.263107029502901	0.263107029502901\\
72.875	0.15522	0.232214865489579	0.232214865489579\\
72.875	0.15888	0.207528470557085	0.207528470557085\\
72.875	0.16254	0.18904784470542	0.18904784470542\\
72.875	0.1662	0.176772987934587	0.176772987934587\\
72.875	0.16986	0.170703900244577	0.170703900244577\\
72.875	0.17352	0.170840581635399	0.170840581635399\\
72.875	0.17718	0.177183032107049	0.177183032107049\\
72.875	0.18084	0.189731251659531	0.189731251659531\\
72.875	0.1845	0.208485240292839	0.208485240292839\\
72.875	0.18816	0.233444998006973	0.233444998006973\\
72.875	0.19182	0.264610524801936	0.264610524801936\\
72.875	0.19548	0.301981820677737	0.301981820677737\\
72.875	0.19914	0.345558885634361	0.345558885634361\\
72.875	0.2028	0.395341719671807	0.395341719671807\\
72.875	0.20646	0.451330322790088	0.451330322790088\\
72.875	0.21012	0.513524694989195	0.513524694989195\\
72.875	0.21378	0.581924836269136	0.581924836269136\\
72.875	0.21744	0.656530746629903	0.656530746629903\\
72.875	0.2211	0.737342426071494	0.737342426071494\\
72.875	0.22476	0.82435987459392	0.82435987459392\\
72.875	0.22842	0.91758309219718	0.91758309219718\\
72.875	0.23208	1.01701207888126	1.01701207888126\\
72.875	0.23574	1.12264683464617	1.12264683464617\\
72.875	0.2394	1.23448735949191	1.23448735949191\\
72.875	0.24306	1.35253365341848	1.35253365341848\\
72.875	0.24672	1.47678571642587	1.47678571642587\\
72.875	0.25038	1.6072435485141	1.6072435485141\\
72.875	0.25404	1.74390714968315	1.74390714968315\\
72.875	0.2577	1.88677651993304	1.88677651993304\\
72.875	0.26136	2.03585165926375	2.03585165926375\\
72.875	0.26502	2.19113256767529	2.19113256767529\\
72.875	0.26868	2.35261924516766	2.35261924516766\\
72.875	0.27234	2.52031169174085	2.52031169174085\\
72.875	0.276	2.69420990739488	2.69420990739488\\
73.25	0.093	1.66190706759489	1.66190706759489\\
73.25	0.09666	1.52936326817761	1.52936326817761\\
73.25	0.10032	1.40302523784115	1.40302523784115\\
73.25	0.10398	1.28289297658552	1.28289297658552\\
73.25	0.10764	1.16896648441073	1.16896648441073\\
73.25	0.1113	1.06124576131676	1.06124576131676\\
73.25	0.11496	0.959730807303619	0.959730807303619\\
73.25	0.11862	0.864421622371309	0.864421622371309\\
73.25	0.12228	0.775318206519826	0.775318206519826\\
73.25	0.12594	0.69242055974917	0.69242055974917\\
73.25	0.1296	0.615728682059347	0.615728682059347\\
73.25	0.13326	0.545242573450349	0.545242573450349\\
73.25	0.13692	0.480962233922183	0.480962233922183\\
73.25	0.14058	0.422887663474844	0.422887663474844\\
73.25	0.14424	0.371018862108333	0.371018862108333\\
73.25	0.1479	0.325355829822653	0.325355829822653\\
73.25	0.15156	0.285898566617803	0.285898566617803\\
73.25	0.15522	0.252647072493778	0.252647072493778\\
73.25	0.15888	0.225601347450583	0.225601347450583\\
73.25	0.16254	0.204761391488218	0.204761391488218\\
73.25	0.1662	0.190127204606679	0.190127204606679\\
73.25	0.16986	0.181698786805969	0.181698786805969\\
73.25	0.17352	0.17947613808609	0.17947613808609\\
73.25	0.17718	0.183459258447041	0.183459258447041\\
73.25	0.18084	0.193648147888815	0.193648147888815\\
73.25	0.1845	0.210042806411424	0.210042806411424\\
73.25	0.18816	0.232643234014859	0.232643234014859\\
73.25	0.19182	0.261449430699122	0.261449430699122\\
73.25	0.19548	0.296461396464215	0.296461396464215\\
73.25	0.19914	0.337679131310139	0.337679131310139\\
73.25	0.2028	0.385102635236885	0.385102635236885\\
73.25	0.20646	0.438731908244467	0.438731908244467\\
73.25	0.21012	0.49856695033287	0.49856695033287\\
73.25	0.21378	0.564607761502108	0.564607761502108\\
73.25	0.21744	0.636854341752175	0.636854341752175\\
73.25	0.2211	0.715306691083066	0.715306691083066\\
73.25	0.22476	0.799964809494792	0.799964809494792\\
73.25	0.22842	0.890828696987345	0.890828696987345\\
73.25	0.23208	0.987898353560718	0.987898353560718\\
73.25	0.23574	1.09117377921493	1.09117377921493\\
73.25	0.2394	1.20065497394997	1.20065497394997\\
73.25	0.24306	1.31634193776583	1.31634193776583\\
73.25	0.24672	1.43823467066253	1.43823467066253\\
73.25	0.25038	1.56633317264005	1.56633317264005\\
73.25	0.25404	1.7006374436984	1.7006374436984\\
73.25	0.2577	1.84114748383759	1.84114748383759\\
73.25	0.26136	1.98786329305759	1.98786329305759\\
73.25	0.26502	2.14078487135844	2.14078487135844\\
73.25	0.26868	2.2999122187401	2.2999122187401\\
73.25	0.27234	2.4652453352026	2.4652453352026\\
73.25	0.276	2.63678422074592	2.63678422074592\\
73.625	0.093	1.72341470865783	1.72341470865783\\
73.625	0.09666	1.58851157912985	1.58851157912985\\
73.625	0.10032	1.45981421868269	1.45981421868269\\
73.625	0.10398	1.33732262731636	1.33732262731636\\
73.625	0.10764	1.22103680503087	1.22103680503087\\
73.625	0.1113	1.11095675182619	1.11095675182619\\
73.625	0.11496	1.00708246770235	1.00708246770235\\
73.625	0.11862	0.909413952659341	0.909413952659341\\
73.625	0.12228	0.817951206697154	0.817951206697154\\
73.625	0.12594	0.732694229815803	0.732694229815803\\
73.625	0.1296	0.653643022015272	0.653643022015272\\
73.625	0.13326	0.580797583295574	0.580797583295574\\
73.625	0.13692	0.514157913656705	0.514157913656705\\
73.625	0.14058	0.453724013098663	0.453724013098663\\
73.625	0.14424	0.399495881621456	0.399495881621456\\
73.625	0.1479	0.351473519225072	0.351473519225072\\
73.625	0.15156	0.309656925909518	0.309656925909518\\
73.625	0.15522	0.27404610167479	0.27404610167479\\
73.625	0.15888	0.244641046520895	0.244641046520895\\
73.625	0.16254	0.22144176044783	0.22144176044783\\
73.625	0.1662	0.204448243455591	0.204448243455591\\
73.625	0.16986	0.193660495544178	0.193660495544178\\
73.625	0.17352	0.189078516713596	0.189078516713596\\
73.625	0.17718	0.190702306963843	0.190702306963843\\
73.625	0.18084	0.198531866294918	0.198531866294918\\
73.625	0.1845	0.212567194706823	0.212567194706823\\
73.625	0.18816	0.232808292199557	0.232808292199557\\
73.625	0.19182	0.25925515877312	0.25925515877312\\
73.625	0.19548	0.291907794427507	0.291907794427507\\
73.625	0.19914	0.330766199162731	0.330766199162731\\
73.625	0.2028	0.375830372978777	0.375830372978777\\
73.625	0.20646	0.427100315875659	0.427100315875659\\
73.625	0.21012	0.484576027853359	0.484576027853359\\
73.625	0.21378	0.548257508911894	0.548257508911894\\
73.625	0.21744	0.618144759051257	0.618144759051257\\
73.625	0.2211	0.694237778271445	0.694237778271445\\
73.625	0.22476	0.776536566572471	0.776536566572471\\
73.625	0.22842	0.865041123954324	0.865041123954324\\
73.625	0.23208	0.959751450416997	0.959751450416997\\
73.625	0.23574	1.0606675459605	1.0606675459605\\
73.625	0.2394	1.16778941058484	1.16778941058484\\
73.625	0.24306	1.28111704429	1.28111704429\\
73.625	0.24672	1.400650447076	1.400650447076\\
73.625	0.25038	1.52638961894282	1.52638961894282\\
73.625	0.25404	1.65833455989047	1.65833455989047\\
73.625	0.2577	1.79648526991894	1.79648526991894\\
73.625	0.26136	1.94084174902826	1.94084174902826\\
73.625	0.26502	2.0914039972184	2.0914039972184\\
73.625	0.26868	2.24817201448936	2.24817201448936\\
73.625	0.27234	2.41114580084115	2.41114580084115\\
73.625	0.276	2.58032535627378	2.58032535627378\\
74	0.093	1.78588917189759	1.78588917189759\\
74	0.09666	1.6486267122589	1.6486267122589\\
74	0.10032	1.51757002170104	1.51757002170104\\
74	0.10398	1.39271910022401	1.39271910022401\\
74	0.10764	1.27407394782781	1.27407394782781\\
74	0.1113	1.16163456451244	1.16163456451244\\
74	0.11496	1.05540095027789	1.05540095027789\\
74	0.11862	0.95537310512418	0.95537310512418\\
74	0.12228	0.861551029051293	0.861551029051293\\
74	0.12594	0.773934722059238	0.773934722059238\\
74	0.1296	0.692524184148008	0.692524184148008\\
74	0.13326	0.617319415317606	0.617319415317606\\
74	0.13692	0.548320415568037	0.548320415568037\\
74	0.14058	0.485527184899295	0.485527184899295\\
74	0.14424	0.428939723311381	0.428939723311381\\
74	0.1479	0.378558030804298	0.378558030804298\\
74	0.15156	0.334382107378044	0.334382107378044\\
74	0.15522	0.296411953032612	0.296411953032612\\
74	0.15888	0.264647567768018	0.264647567768018\\
74	0.16254	0.239088951584246	0.239088951584246\\
74	0.1662	0.219736104481306	0.219736104481306\\
74	0.16986	0.206589026459193	0.206589026459193\\
74	0.17352	0.199647717517908	0.199647717517908\\
74	0.17718	0.198912177657455	0.198912177657455\\
74	0.18084	0.20438240687783	0.20438240687783\\
74	0.1845	0.216058405179032	0.216058405179032\\
74	0.18816	0.233940172561063	0.233940172561063\\
74	0.19182	0.258027709023922	0.258027709023922\\
74	0.19548	0.288321014567609	0.288321014567609\\
74	0.19914	0.324820089192134	0.324820089192134\\
74	0.2028	0.367524932897473	0.367524932897473\\
74	0.20646	0.416435545683655	0.416435545683655\\
74	0.21012	0.471551927550655	0.471551927550655\\
74	0.21378	0.532874078498486	0.532874078498486\\
74	0.21744	0.600401998527149	0.600401998527149\\
74	0.2211	0.674135687636634	0.674135687636634\\
74	0.22476	0.75407514582696	0.75407514582696\\
74	0.22842	0.840220373098106	0.840220373098106\\
74	0.23208	0.932571369450079	0.932571369450079\\
74	0.23574	1.03112813488289	1.03112813488289\\
74	0.2394	1.13589066939653	1.13589066939653\\
74	0.24306	1.24685897299098	1.24685897299098\\
74	0.24672	1.36403304566628	1.36403304566628\\
74	0.25038	1.48741288742239	1.48741288742239\\
74	0.25404	1.61699849825934	1.61699849825934\\
74	0.2577	1.75278987817712	1.75278987817712\\
74	0.26136	1.89478702717572	1.89478702717572\\
74	0.26502	2.04298994525517	2.04298994525517\\
74	0.26868	2.19739863241542	2.19739863241542\\
74	0.27234	2.35801308865652	2.35801308865652\\
74	0.276	2.52483331397844	2.52483331397844\\
};
\end{axis}

\begin{axis}[%
width=2.616cm,
height=2.517cm,
at={(0cm,0cm)},
scale only axis,
xmin=56,
xmax=74,
tick align=outside,
xlabel style={font=\color{white!15!black}},
xlabel={$L_{cut}$},
ymin=0.093,
ymax=0.276,
ylabel style={font=\color{white!15!black}},
ylabel={$D_{rlx}$},
zmin=-0.0529761706320979,
zmax=2.07241988601554,
zlabel style={font=\color{white!15!black}},
zlabel={$x_2,x_2$},
view={-140}{50},
axis background/.style={fill=white},
xmajorgrids,
ymajorgrids,
zmajorgrids,
legend style={at={(1.03,1)}, anchor=north west, legend cell align=left, align=left, draw=white!15!black}
]
\addplot3[only marks, mark=*, mark options={}, mark size=1.5000pt, color=mycolor1, fill=mycolor1] table[row sep=crcr]{%
x	y	z\\
74	0.123	0.210484614807782\\
72	0.113	0.426720639897933\\
61	0.095	-0.0316981769537125\\
56	0.093	0.0157114443831866\\
};
\addlegendentry{data1}

\addplot3[only marks, mark=*, mark options={}, mark size=1.5000pt, color=mycolor2, fill=mycolor2] table[row sep=crcr]{%
x	y	z\\
67	0.276	1.46071942907469\\
66	0.255	1.31337535124637\\
62	0.209	0.489410819850672\\
57	0.193	0.45517023967435\\
};
\addlegendentry{data2}

\addplot3[only marks, mark=*, mark options={}, mark size=1.5000pt, color=black, fill=black] table[row sep=crcr]{%
x	y	z\\
69	0.104	0.231391128866393\\
};
\addlegendentry{data3}

\addplot3[only marks, mark=*, mark options={}, mark size=1.5000pt, color=black, fill=black] table[row sep=crcr]{%
x	y	z\\
64	0.23	0.797774633113731\\
};
\addlegendentry{data4}


\addplot3[%
surf,
fill opacity=0.7, shader=interp, colormap={mymap}{[1pt] rgb(0pt)=(1,0.905882,0); rgb(1pt)=(1,0.901964,0); rgb(2pt)=(1,0.898051,0); rgb(3pt)=(1,0.894144,0); rgb(4pt)=(1,0.890243,0); rgb(5pt)=(1,0.886349,0); rgb(6pt)=(1,0.88246,0); rgb(7pt)=(1,0.878577,0); rgb(8pt)=(1,0.8747,0); rgb(9pt)=(1,0.870829,0); rgb(10pt)=(1,0.866964,0); rgb(11pt)=(1,0.863106,0); rgb(12pt)=(1,0.859253,0); rgb(13pt)=(1,0.855406,0); rgb(14pt)=(1,0.851566,0); rgb(15pt)=(1,0.847732,0); rgb(16pt)=(1,0.843903,0); rgb(17pt)=(1,0.840081,0); rgb(18pt)=(1,0.836265,0); rgb(19pt)=(1,0.832455,0); rgb(20pt)=(1,0.828652,0); rgb(21pt)=(1,0.824854,0); rgb(22pt)=(1,0.821063,0); rgb(23pt)=(1,0.817278,0); rgb(24pt)=(1,0.8135,0); rgb(25pt)=(1,0.809727,0); rgb(26pt)=(1,0.805961,0); rgb(27pt)=(1,0.8022,0); rgb(28pt)=(1,0.798445,0); rgb(29pt)=(1,0.794696,0); rgb(30pt)=(1,0.790953,0); rgb(31pt)=(1,0.787215,0); rgb(32pt)=(1,0.783484,0); rgb(33pt)=(1,0.779758,0); rgb(34pt)=(1,0.776038,0); rgb(35pt)=(1,0.772324,0); rgb(36pt)=(1,0.768615,0); rgb(37pt)=(1,0.764913,0); rgb(38pt)=(1,0.761217,0); rgb(39pt)=(1,0.757527,0); rgb(40pt)=(1,0.753843,0); rgb(41pt)=(1,0.750165,0); rgb(42pt)=(1,0.746493,0); rgb(43pt)=(1,0.742827,0); rgb(44pt)=(1,0.739167,0); rgb(45pt)=(1,0.735514,0); rgb(46pt)=(1,0.731867,0); rgb(47pt)=(1,0.728226,0); rgb(48pt)=(1,0.724591,0); rgb(49pt)=(1,0.720963,0); rgb(50pt)=(1,0.717341,0); rgb(51pt)=(1,0.713725,0); rgb(52pt)=(0.999994,0.710077,0); rgb(53pt)=(0.999974,0.706363,0); rgb(54pt)=(0.999942,0.702592,0); rgb(55pt)=(0.999898,0.698775,0); rgb(56pt)=(0.999841,0.694921,0); rgb(57pt)=(0.999771,0.691039,0); rgb(58pt)=(0.99969,0.687139,0); rgb(59pt)=(0.999596,0.68323,0); rgb(60pt)=(0.99949,0.679323,0); rgb(61pt)=(0.999372,0.675427,0); rgb(62pt)=(0.999242,0.67155,0); rgb(63pt)=(0.9991,0.667704,0); rgb(64pt)=(0.998946,0.663897,0); rgb(65pt)=(0.998781,0.660138,0); rgb(66pt)=(0.998605,0.656439,0); rgb(67pt)=(0.998416,0.652807,0); rgb(68pt)=(0.998217,0.649253,0); rgb(69pt)=(0.998006,0.645786,0); rgb(70pt)=(0.997785,0.642416,0); rgb(71pt)=(0.997552,0.639152,0); rgb(72pt)=(0.997308,0.636004,0); rgb(73pt)=(0.997053,0.632982,0); rgb(74pt)=(0.996788,0.630095,0); rgb(75pt)=(0.996512,0.627352,0); rgb(76pt)=(0.996226,0.624763,0); rgb(77pt)=(0.995851,0.622329,0); rgb(78pt)=(0.99494,0.619997,0); rgb(79pt)=(0.99345,0.617753,0); rgb(80pt)=(0.991419,0.61559,0); rgb(81pt)=(0.988885,0.613503,0); rgb(82pt)=(0.985886,0.611486,0); rgb(83pt)=(0.98246,0.609532,0); rgb(84pt)=(0.978643,0.607636,0); rgb(85pt)=(0.974475,0.605791,0); rgb(86pt)=(0.969992,0.603992,0); rgb(87pt)=(0.965232,0.602233,0); rgb(88pt)=(0.960233,0.600507,0); rgb(89pt)=(0.955033,0.598808,0); rgb(90pt)=(0.949669,0.59713,0); rgb(91pt)=(0.94418,0.595468,0); rgb(92pt)=(0.938602,0.593815,0); rgb(93pt)=(0.932974,0.592166,0); rgb(94pt)=(0.927333,0.590513,0); rgb(95pt)=(0.921717,0.588852,0); rgb(96pt)=(0.916164,0.587176,0); rgb(97pt)=(0.910711,0.585479,0); rgb(98pt)=(0.905397,0.583755,0); rgb(99pt)=(0.900258,0.581999,0); rgb(100pt)=(0.895333,0.580203,0); rgb(101pt)=(0.890659,0.578362,0); rgb(102pt)=(0.886275,0.576471,0); rgb(103pt)=(0.882047,0.574545,0); rgb(104pt)=(0.877819,0.572608,0); rgb(105pt)=(0.873592,0.57066,0); rgb(106pt)=(0.869366,0.568701,0); rgb(107pt)=(0.865143,0.566733,0); rgb(108pt)=(0.860924,0.564756,0); rgb(109pt)=(0.856708,0.562771,0); rgb(110pt)=(0.852497,0.560778,0); rgb(111pt)=(0.848292,0.558779,0); rgb(112pt)=(0.844092,0.556774,0); rgb(113pt)=(0.8399,0.554763,0); rgb(114pt)=(0.835716,0.552749,0); rgb(115pt)=(0.831541,0.55073,0); rgb(116pt)=(0.827374,0.548709,0); rgb(117pt)=(0.823219,0.546686,0); rgb(118pt)=(0.819074,0.54466,0); rgb(119pt)=(0.81494,0.542635,0); rgb(120pt)=(0.81082,0.540609,0); rgb(121pt)=(0.806712,0.538584,0); rgb(122pt)=(0.802619,0.53656,0); rgb(123pt)=(0.798541,0.534539,0); rgb(124pt)=(0.794478,0.532521,0); rgb(125pt)=(0.790431,0.530506,0); rgb(126pt)=(0.786402,0.528496,0); rgb(127pt)=(0.782391,0.526491,0); rgb(128pt)=(0.77841,0.524489,0); rgb(129pt)=(0.774523,0.522478,0); rgb(130pt)=(0.770731,0.520455,0); rgb(131pt)=(0.767022,0.518424,0); rgb(132pt)=(0.763384,0.516385,0); rgb(133pt)=(0.759804,0.514339,0); rgb(134pt)=(0.756272,0.51229,0); rgb(135pt)=(0.752775,0.510237,0); rgb(136pt)=(0.749302,0.508182,0); rgb(137pt)=(0.74584,0.506128,0); rgb(138pt)=(0.742378,0.504075,0); rgb(139pt)=(0.738904,0.502025,0); rgb(140pt)=(0.735406,0.499979,0); rgb(141pt)=(0.731872,0.49794,0); rgb(142pt)=(0.72829,0.495909,0); rgb(143pt)=(0.724649,0.493887,0); rgb(144pt)=(0.720936,0.491875,0); rgb(145pt)=(0.71714,0.489876,0); rgb(146pt)=(0.713249,0.487891,0); rgb(147pt)=(0.709251,0.485921,0); rgb(148pt)=(0.705134,0.483968,0); rgb(149pt)=(0.700887,0.482033,0); rgb(150pt)=(0.696497,0.480118,0); rgb(151pt)=(0.691952,0.478225,0); rgb(152pt)=(0.687242,0.476355,0); rgb(153pt)=(0.682353,0.47451,0); rgb(154pt)=(0.677195,0.472696,0); rgb(155pt)=(0.6717,0.470916,0); rgb(156pt)=(0.665891,0.469169,0); rgb(157pt)=(0.659791,0.46745,0); rgb(158pt)=(0.653423,0.465756,0); rgb(159pt)=(0.64681,0.464084,0); rgb(160pt)=(0.639976,0.462432,0); rgb(161pt)=(0.632943,0.460795,0); rgb(162pt)=(0.625734,0.459171,0); rgb(163pt)=(0.618373,0.457556,0); rgb(164pt)=(0.610882,0.455948,0); rgb(165pt)=(0.603284,0.454343,0); rgb(166pt)=(0.595604,0.452737,0); rgb(167pt)=(0.587863,0.451129,0); rgb(168pt)=(0.580084,0.449514,0); rgb(169pt)=(0.572292,0.447889,0); rgb(170pt)=(0.564508,0.446252,0); rgb(171pt)=(0.556756,0.444599,0); rgb(172pt)=(0.549059,0.442927,0); rgb(173pt)=(0.54144,0.441232,0); rgb(174pt)=(0.533922,0.439512,0); rgb(175pt)=(0.526529,0.437764,0); rgb(176pt)=(0.519282,0.435983,0); rgb(177pt)=(0.512206,0.434168,0); rgb(178pt)=(0.505323,0.432315,0); rgb(179pt)=(0.498628,0.430422,3.92506e-06); rgb(180pt)=(0.491973,0.428504,3.49981e-05); rgb(181pt)=(0.485331,0.426562,9.63073e-05); rgb(182pt)=(0.478704,0.424596,0.000186979); rgb(183pt)=(0.472096,0.422609,0.000306141); rgb(184pt)=(0.465508,0.420599,0.00045292); rgb(185pt)=(0.458942,0.418567,0.000626441); rgb(186pt)=(0.452401,0.416515,0.000825833); rgb(187pt)=(0.445885,0.414441,0.00105022); rgb(188pt)=(0.439399,0.412348,0.00129873); rgb(189pt)=(0.432942,0.410234,0.00157049); rgb(190pt)=(0.426518,0.408102,0.00186463); rgb(191pt)=(0.420129,0.40595,0.00218028); rgb(192pt)=(0.413777,0.40378,0.00251655); rgb(193pt)=(0.407464,0.401592,0.00287258); rgb(194pt)=(0.401191,0.399386,0.00324749); rgb(195pt)=(0.394962,0.397164,0.00364042); rgb(196pt)=(0.388777,0.394925,0.00405048); rgb(197pt)=(0.38264,0.39267,0.00447681); rgb(198pt)=(0.376552,0.390399,0.00491852); rgb(199pt)=(0.370516,0.388113,0.00537476); rgb(200pt)=(0.364532,0.385812,0.00584464); rgb(201pt)=(0.358605,0.383497,0.00632729); rgb(202pt)=(0.352735,0.381168,0.00682184); rgb(203pt)=(0.346925,0.378826,0.00732741); rgb(204pt)=(0.341176,0.376471,0.00784314); rgb(205pt)=(0.335485,0.374093,0.00847245); rgb(206pt)=(0.329843,0.371682,0.00930909); rgb(207pt)=(0.324249,0.369242,0.0103377); rgb(208pt)=(0.318701,0.366772,0.0115428); rgb(209pt)=(0.313198,0.364275,0.0129091); rgb(210pt)=(0.307739,0.361753,0.0144211); rgb(211pt)=(0.302322,0.359206,0.0160634); rgb(212pt)=(0.296945,0.356637,0.0178207); rgb(213pt)=(0.291607,0.354048,0.0196776); rgb(214pt)=(0.286307,0.35144,0.0216186); rgb(215pt)=(0.281043,0.348814,0.0236284); rgb(216pt)=(0.275813,0.346172,0.0256916); rgb(217pt)=(0.270616,0.343517,0.0277927); rgb(218pt)=(0.265451,0.340849,0.0299163); rgb(219pt)=(0.260317,0.33817,0.0320472); rgb(220pt)=(0.25521,0.335482,0.0341698); rgb(221pt)=(0.250131,0.332786,0.0362688); rgb(222pt)=(0.245078,0.330085,0.0383287); rgb(223pt)=(0.240048,0.327379,0.0403343); rgb(224pt)=(0.235042,0.324671,0.04227); rgb(225pt)=(0.230056,0.321962,0.0441205); rgb(226pt)=(0.22509,0.319254,0.0458704); rgb(227pt)=(0.220142,0.316548,0.0475043); rgb(228pt)=(0.215212,0.313846,0.0490067); rgb(229pt)=(0.210296,0.311149,0.0503624); rgb(230pt)=(0.205395,0.308459,0.0515759); rgb(231pt)=(0.200514,0.305763,0.052757); rgb(232pt)=(0.195655,0.303061,0.0539242); rgb(233pt)=(0.190817,0.300353,0.0550763); rgb(234pt)=(0.186001,0.297639,0.0562123); rgb(235pt)=(0.181207,0.294918,0.0573313); rgb(236pt)=(0.176434,0.292191,0.0584321); rgb(237pt)=(0.171685,0.289458,0.0595136); rgb(238pt)=(0.166957,0.286719,0.060575); rgb(239pt)=(0.162252,0.283973,0.0616151); rgb(240pt)=(0.15757,0.281221,0.0626328); rgb(241pt)=(0.152911,0.278463,0.0636271); rgb(242pt)=(0.148275,0.275699,0.0645971); rgb(243pt)=(0.143663,0.272929,0.0655416); rgb(244pt)=(0.139074,0.270152,0.0664596); rgb(245pt)=(0.134508,0.26737,0.06735); rgb(246pt)=(0.129967,0.264581,0.0682118); rgb(247pt)=(0.125449,0.261787,0.0690441); rgb(248pt)=(0.120956,0.258986,0.0698456); rgb(249pt)=(0.116487,0.25618,0.0706154); rgb(250pt)=(0.112043,0.253367,0.0713525); rgb(251pt)=(0.107623,0.250549,0.0720557); rgb(252pt)=(0.103229,0.247724,0.0727241); rgb(253pt)=(0.0988592,0.244894,0.0733566); rgb(254pt)=(0.0945149,0.242058,0.0739522); rgb(255pt)=(0.0901961,0.239216,0.0745098)}, mesh/rows=49]
table[row sep=crcr, point meta=\thisrow{c}] {%
%
x	y	z	c\\
56	0.093	-0.0319235315233359	-0.0319235315233359\\
56	0.09666	-0.0402493834493129	-0.0402493834493129\\
56	0.10032	-0.0465175733231773	-0.0465175733231773\\
56	0.10398	-0.0507281011449294	-0.0507281011449294\\
56	0.10764	-0.0528809669145699	-0.0528809669145699\\
56	0.1113	-0.0529761706320979	-0.0529761706320979\\
56	0.11496	-0.0510137122975125	-0.0510137122975125\\
56	0.11862	-0.0469935919108151	-0.0469935919108151\\
56	0.12228	-0.0409158094720048	-0.0409158094720048\\
56	0.12594	-0.0327803649810823	-0.0327803649810823\\
56	0.1296	-0.0225872584380475	-0.0225872584380475\\
56	0.13326	-0.0103364898429013	-0.0103364898429013\\
56	0.13692	0.00397194080435859	0.00397194080435859\\
56	0.14058	0.0203380335037306	0.0203380335037306\\
56	0.14424	0.0387617882552154	0.0387617882552154\\
56	0.1479	0.0592432050588125	0.0592432050588125\\
56	0.15156	0.0817822839145219	0.0817822839145219\\
56	0.15522	0.106379024822344	0.106379024822344\\
56	0.15888	0.133033427782278	0.133033427782278\\
56	0.16254	0.161745492794325	0.161745492794325\\
56	0.1662	0.192515219858484	0.192515219858484\\
56	0.16986	0.225342608974755	0.225342608974755\\
56	0.17352	0.260227660143139	0.260227660143139\\
56	0.17718	0.297170373363636	0.297170373363636\\
56	0.18084	0.336170748636245	0.336170748636245\\
56	0.1845	0.377228785960966	0.377228785960966\\
56	0.18816	0.4203444853378	0.4203444853378\\
56	0.19182	0.465517846766745	0.465517846766745\\
56	0.19548	0.512748870247804	0.512748870247804\\
56	0.19914	0.562037555780976	0.562037555780976\\
56	0.2028	0.613383903366259	0.613383903366259\\
56	0.20646	0.666787913003656	0.666787913003656\\
56	0.21012	0.722249584693163	0.722249584693163\\
56	0.21378	0.779768918434784	0.779768918434784\\
56	0.21744	0.839345914228517	0.839345914228517\\
56	0.2211	0.900980572074363	0.900980572074363\\
56	0.22476	0.964672891972321	0.964672891972321\\
56	0.22842	1.03042287392239	1.03042287392239\\
56	0.23208	1.09823051792457	1.09823051792457\\
56	0.23574	1.16809582397887	1.16809582397887\\
56	0.2394	1.24001879208528	1.24001879208528\\
56	0.24306	1.3139994222438	1.3139994222438\\
56	0.24672	1.39003771445443	1.39003771445443\\
56	0.25038	1.46813366871718	1.46813366871718\\
56	0.25404	1.54828728503203	1.54828728503203\\
56	0.2577	1.63049856339901	1.63049856339901\\
56	0.26136	1.71476750381809	1.71476750381809\\
56	0.26502	1.80109410628928	1.80109410628928\\
56	0.26868	1.88947837081259	1.88947837081259\\
56	0.27234	1.97992029738801	1.97992029738801\\
56	0.276	2.07241988601554	2.07241988601554\\
56.375	0.093	-0.0273435325599678	-0.0273435325599678\\
56.375	0.09666	-0.0362027020849496	-0.0362027020849496\\
56.375	0.10032	-0.0430042095578196	-0.0430042095578196\\
56.375	0.10398	-0.0477480549785774	-0.0477480549785774\\
56.375	0.10764	-0.0504342383472226	-0.0504342383472226\\
56.375	0.1113	-0.0510627596637545	-0.0510627596637545\\
56.375	0.11496	-0.0496336189281739	-0.0496336189281739\\
56.375	0.11862	-0.0461468161404821	-0.0461468161404821\\
56.375	0.12228	-0.0406023513006775	-0.0406023513006775\\
56.375	0.12594	-0.0330002244087597	-0.0330002244087597\\
56.375	0.1296	-0.0233404354647297	-0.0233404354647297\\
56.375	0.13326	-0.0116229844685882	-0.0116229844685882\\
56.375	0.13692	0.00215212857966685	0.00215212857966685\\
56.375	0.14058	0.0179849036800341	0.0179849036800341\\
56.375	0.14424	0.0358753408325132	0.0358753408325132\\
56.375	0.1479	0.0558234400371056	0.0558234400371056\\
56.375	0.15156	0.0778292012938093	0.0778292012938093\\
56.375	0.15522	0.101892624602627	0.101892624602627\\
56.375	0.15888	0.128013709963556	0.128013709963556\\
56.375	0.16254	0.156192457376598	0.156192457376598\\
56.375	0.1662	0.186428866841752	0.186428866841752\\
56.375	0.16986	0.218722938359019	0.218722938359019\\
56.375	0.17352	0.253074671928397	0.253074671928397\\
56.375	0.17718	0.289484067549889	0.289484067549889\\
56.375	0.18084	0.327951125223492	0.327951125223492\\
56.375	0.1845	0.368475844949209	0.368475844949209\\
56.375	0.18816	0.411058226727038	0.411058226727038\\
56.375	0.19182	0.455698270556979	0.455698270556979\\
56.375	0.19548	0.502395976439032	0.502395976439032\\
56.375	0.19914	0.551151344373198	0.551151344373198\\
56.375	0.2028	0.601964374359477	0.601964374359477\\
56.375	0.20646	0.654835066397869	0.654835066397869\\
56.375	0.21012	0.709763420488372	0.709763420488372\\
56.375	0.21378	0.766749436630987	0.766749436630987\\
56.375	0.21744	0.825793114825715	0.825793114825715\\
56.375	0.2211	0.886894455072556	0.886894455072556\\
56.375	0.22476	0.95005345737151	0.95005345737151\\
56.375	0.22842	1.01527012172257	1.01527012172257\\
56.375	0.23208	1.08254444812575	1.08254444812575\\
56.375	0.23574	1.15187643658104	1.15187643658104\\
56.375	0.2394	1.22326608708845	1.22326608708845\\
56.375	0.24306	1.29671339964796	1.29671339964796\\
56.375	0.24672	1.37221837425959	1.37221837425959\\
56.375	0.25038	1.44978101092333	1.44978101092333\\
56.375	0.25404	1.52940130963918	1.52940130963918\\
56.375	0.2577	1.61107927040715	1.61107927040715\\
56.375	0.26136	1.69481489322723	1.69481489322723\\
56.375	0.26502	1.78060817809941	1.78060817809941\\
56.375	0.26868	1.86845912502372	1.86845912502372\\
56.375	0.27234	1.95836773400013	1.95836773400013\\
56.375	0.276	2.05033400502866	2.05033400502866\\
56.75	0.093	-0.0224569143792763	-0.0224569143792763\\
56.75	0.09666	-0.0318494015032629	-0.0318494015032629\\
56.75	0.10032	-0.0391842265751368	-0.0391842265751368\\
56.75	0.10398	-0.0444613895949002	-0.0444613895949002\\
56.75	0.10764	-0.0476808905625493	-0.0476808905625493\\
56.75	0.1113	-0.0488427294780878	-0.0488427294780878\\
56.75	0.11496	-0.0479469063415119	-0.0479469063415119\\
56.75	0.11862	-0.0449934211528249	-0.0449934211528249\\
56.75	0.12228	-0.039982273912025	-0.039982273912025\\
56.75	0.12594	-0.032913464619112	-0.032913464619112\\
56.75	0.1296	-0.0237869932740868	-0.0237869932740868\\
56.75	0.13326	-0.0126028598769501	-0.0126028598769501\\
56.75	0.13692	0.000638935572299326	0.000638935572299326\\
56.75	0.14058	0.015938393073661	0.015938393073661\\
56.75	0.14424	0.0332955126271361	0.0332955126271361\\
56.75	0.1479	0.0527102942327238	0.0527102942327238\\
56.75	0.15156	0.0741827378904227	0.0741827378904227\\
56.75	0.15522	0.0977128436002346	0.0977128436002346\\
56.75	0.15888	0.123300611362158	0.123300611362158\\
56.75	0.16254	0.150946041176196	0.150946041176196\\
56.75	0.1662	0.180649133042345	0.180649133042345\\
56.75	0.16986	0.212409886960607	0.212409886960607\\
56.75	0.17352	0.24622830293098	0.24622830293098\\
56.75	0.17718	0.282104380953466	0.282104380953466\\
56.75	0.18084	0.320038121028065	0.320038121028065\\
56.75	0.1845	0.360029523154777	0.360029523154777\\
56.75	0.18816	0.402078587333601	0.402078587333601\\
56.75	0.19182	0.446185313564537	0.446185313564537\\
56.75	0.19548	0.492349701847585	0.492349701847585\\
56.75	0.19914	0.540571752182746	0.540571752182746\\
56.75	0.2028	0.59085146457002	0.59085146457002\\
56.75	0.20646	0.643188839009407	0.643188839009407\\
56.75	0.21012	0.697583875500905	0.697583875500905\\
56.75	0.21378	0.754036574044515	0.754036574044515\\
56.75	0.21744	0.812546934640238	0.812546934640238\\
56.75	0.2211	0.873114957288074	0.873114957288074\\
56.75	0.22476	0.935740641988021	0.935740641988021\\
56.75	0.22842	1.00042398874008	1.00042398874008\\
56.75	0.23208	1.06716499754426	1.06716499754426\\
56.75	0.23574	1.13596366840054	1.13596366840054\\
56.75	0.2394	1.20682000130894	1.20682000130894\\
56.75	0.24306	1.27973399626945	1.27973399626945\\
56.75	0.24672	1.35470565328207	1.35470565328207\\
56.75	0.25038	1.43173497234681	1.43173497234681\\
56.75	0.25404	1.51082195346366	1.51082195346366\\
56.75	0.2577	1.59196659663262	1.59196659663262\\
56.75	0.26136	1.67516890185369	1.67516890185369\\
56.75	0.26502	1.76042886912687	1.76042886912687\\
56.75	0.26868	1.84774649845217	1.84774649845217\\
56.75	0.27234	1.93712178982958	1.93712178982958\\
56.75	0.276	2.0285547432591	2.0285547432591\\
57.125	0.093	-0.0172636769812584	-0.0172636769812584\\
57.125	0.09666	-0.0271894817042497	-0.0271894817042497\\
57.125	0.10032	-0.0350576243751293	-0.0350576243751293\\
57.125	0.10398	-0.0408681049938975	-0.0408681049938975\\
57.125	0.10764	-0.0446209235605523	-0.0446209235605523\\
57.125	0.1113	-0.0463160800750946	-0.0463160800750946\\
57.125	0.11496	-0.0459535745375244	-0.0459535745375244\\
57.125	0.11862	-0.0435334069478421	-0.0435334069478421\\
57.125	0.12228	-0.0390555773060461	-0.0390555773060461\\
57.125	0.12594	-0.0325200856121397	-0.0325200856121397\\
57.125	0.1296	-0.0239269318661193	-0.0239269318661193\\
57.125	0.13326	-0.0132761160679882	-0.0132761160679882\\
57.125	0.13692	-0.000567638217742639	-0.000567638217742639\\
57.125	0.14058	0.0141985016846142	0.0141985016846142\\
57.125	0.14424	0.0310223036390838	0.0310223036390838\\
57.125	0.1479	0.0499037676456657	0.0499037676456657\\
57.125	0.15156	0.0708428937043599	0.0708428937043599\\
57.125	0.15522	0.093839681815167	0.093839681815167\\
57.125	0.15888	0.118894131978087	0.118894131978087\\
57.125	0.16254	0.146006244193118	0.146006244193118\\
57.125	0.1662	0.175176018460263	0.175176018460263\\
57.125	0.16986	0.206403454779519	0.206403454779519\\
57.125	0.17352	0.239688553150888	0.239688553150888\\
57.125	0.17718	0.275031313574369	0.275031313574369\\
57.125	0.18084	0.312431736049963	0.312431736049963\\
57.125	0.1845	0.35188982057767	0.35188982057767\\
57.125	0.18816	0.393405567157489	0.393405567157489\\
57.125	0.19182	0.436978975789419	0.436978975789419\\
57.125	0.19548	0.482610046473463	0.482610046473463\\
57.125	0.19914	0.53029877920962	0.53029877920962\\
57.125	0.2028	0.580045173997888	0.580045173997888\\
57.125	0.20646	0.63184923083827	0.63184923083827\\
57.125	0.21012	0.685710949730762	0.685710949730762\\
57.125	0.21378	0.741630330675368	0.741630330675368\\
57.125	0.21744	0.799607373672086	0.799607373672086\\
57.125	0.2211	0.859642078720917	0.859642078720917\\
57.125	0.22476	0.92173444582186	0.92173444582186\\
57.125	0.22842	0.985884474974915	0.985884474974915\\
57.125	0.23208	1.05209216618008	1.05209216618008\\
57.125	0.23574	1.12035751943736	1.12035751943736\\
57.125	0.2394	1.19068053474676	1.19068053474676\\
57.125	0.24306	1.26306121210826	1.26306121210826\\
57.125	0.24672	1.33749955152188	1.33749955152188\\
57.125	0.25038	1.41399555298761	1.41399555298761\\
57.125	0.25404	1.49254921650545	1.49254921650545\\
57.125	0.2577	1.57316054207541	1.57316054207541\\
57.125	0.26136	1.65582952969748	1.65582952969748\\
57.125	0.26502	1.74055617937166	1.74055617937166\\
57.125	0.26868	1.82734049109795	1.82734049109795\\
57.125	0.27234	1.91618246487635	1.91618246487635\\
57.125	0.276	2.00708210070687	2.00708210070687\\
57.5	0.093	-0.0117638203659172	-0.0117638203659172\\
57.5	0.09666	-0.0222229426879132	-0.0222229426879132\\
57.5	0.10032	-0.0306244029577984	-0.0306244029577984\\
57.5	0.10398	-0.0369682011755705	-0.0369682011755705\\
57.5	0.10764	-0.0412543373412301	-0.0412543373412301\\
57.5	0.1113	-0.0434828114547781	-0.0434828114547781\\
57.5	0.11496	-0.0436536235162117	-0.0436536235162117\\
57.5	0.11862	-0.0417667735255352	-0.0417667735255352\\
57.5	0.12228	-0.0378222614827448	-0.0378222614827448\\
57.5	0.12594	-0.0318200873878423	-0.0318200873878423\\
57.5	0.1296	-0.0237602512408284	-0.0237602512408284\\
57.5	0.13326	-0.0136427530417003	-0.0136427530417003\\
57.5	0.13692	-0.00146759279046127	-0.00146759279046127\\
57.5	0.14058	0.0127652295128908	0.0127652295128908\\
57.5	0.14424	0.0290557138683556	0.0290557138683556\\
57.5	0.1479	0.0474038602759328	0.0474038602759328\\
57.5	0.15156	0.0678096687356222	0.0678096687356222\\
57.5	0.15522	0.0902731392474245	0.0902731392474245\\
57.5	0.15888	0.114794271811339	0.114794271811339\\
57.5	0.16254	0.141373066427366	0.141373066427366\\
57.5	0.1662	0.170009523095505	0.170009523095505\\
57.5	0.16986	0.200703641815756	0.200703641815756\\
57.5	0.17352	0.23345542258812	0.23345542258812\\
57.5	0.17718	0.268264865412597	0.268264865412597\\
57.5	0.18084	0.305131970289185	0.305131970289185\\
57.5	0.1845	0.344056737217887	0.344056737217887\\
57.5	0.18816	0.385039166198701	0.385039166198701\\
57.5	0.19182	0.428079257231626	0.428079257231626\\
57.5	0.19548	0.473177010316665	0.473177010316665\\
57.5	0.19914	0.520332425453816	0.520332425453816\\
57.5	0.2028	0.56954550264308	0.56954550264308\\
57.5	0.20646	0.620816241884456	0.620816241884456\\
57.5	0.21012	0.674144643177945	0.674144643177945\\
57.5	0.21378	0.729530706523545	0.729530706523545\\
57.5	0.21744	0.786974431921259	0.786974431921259\\
57.5	0.2211	0.846475819371085	0.846475819371085\\
57.5	0.22476	0.908034868873023	0.908034868873023\\
57.5	0.22842	0.971651580427073	0.971651580427073\\
57.5	0.23208	1.03732595403324	1.03732595403324\\
57.5	0.23574	1.10505798969151	1.10505798969151\\
57.5	0.2394	1.1748476874019	1.1748476874019\\
57.5	0.24306	1.2466950471644	1.2466950471644\\
57.5	0.24672	1.32060006897901	1.32060006897901\\
57.5	0.25038	1.39656275284574	1.39656275284574\\
57.5	0.25404	1.47458309876458	1.47458309876458\\
57.5	0.2577	1.55466110673553	1.55466110673553\\
57.5	0.26136	1.63679677675859	1.63679677675859\\
57.5	0.26502	1.72099010883376	1.72099010883376\\
57.5	0.26868	1.80724110296105	1.80724110296105\\
57.5	0.27234	1.89554975914045	1.89554975914045\\
57.5	0.276	1.98591607737196	1.98591607737196\\
57.875	0.093	-0.00595734453324948	-0.00595734453324948\\
57.875	0.09666	-0.0169497844542512	-0.0169497844542512\\
57.875	0.10032	-0.0258845623231412	-0.0258845623231412\\
57.875	0.10398	-0.032761678139918	-0.032761678139918\\
57.875	0.10764	-0.0375811319045832	-0.0375811319045832\\
57.875	0.1113	-0.040342923617136	-0.040342923617136\\
57.875	0.11496	-0.0410470532775753	-0.0410470532775753\\
57.875	0.11862	-0.0396935208859026	-0.0396935208859026\\
57.875	0.12228	-0.036282326442117	-0.036282326442117\\
57.875	0.12594	-0.030813469946221	-0.030813469946221\\
57.875	0.1296	-0.023286951398211	-0.023286951398211\\
57.875	0.13326	-0.0137027707980886	-0.0137027707980886\\
57.875	0.13692	-0.00206092814585346	-0.00206092814585346\\
57.875	0.14058	0.011638576558493	0.011638576558493\\
57.875	0.14424	0.027395743314953	0.027395743314953\\
57.875	0.1479	0.0452105721235254	0.0452105721235254\\
57.875	0.15156	0.0650830629842092	0.0650830629842092\\
57.875	0.15522	0.0870132158970067	0.0870132158970067\\
57.875	0.15888	0.111001030861916	0.111001030861916\\
57.875	0.16254	0.137046507878937	0.137046507878937\\
57.875	0.1662	0.165149646948072	0.165149646948072\\
57.875	0.16986	0.195310448069319	0.195310448069319\\
57.875	0.17352	0.227528911242677	0.227528911242677\\
57.875	0.17718	0.261805036468149	0.261805036468149\\
57.875	0.18084	0.298138823745732	0.298138823745732\\
57.875	0.1845	0.33653027307543	0.33653027307543\\
57.875	0.18816	0.376979384457238	0.376979384457238\\
57.875	0.19182	0.41948615789116	0.41948615789116\\
57.875	0.19548	0.464050593377193	0.464050593377193\\
57.875	0.19914	0.510672690915339	0.510672690915339\\
57.875	0.2028	0.559352450505598	0.559352450505598\\
57.875	0.20646	0.610089872147969	0.610089872147969\\
57.875	0.21012	0.662884955842452	0.662884955842452\\
57.875	0.21378	0.717737701589048	0.717737701589048\\
57.875	0.21744	0.774648109387757	0.774648109387757\\
57.875	0.2211	0.833616179238577	0.833616179238577\\
57.875	0.22476	0.894641911141509	0.894641911141509\\
57.875	0.22842	0.957725305096556	0.957725305096556\\
57.875	0.23208	1.02286636110371	1.02286636110371\\
57.875	0.23574	1.09006507916299	1.09006507916299\\
57.875	0.2394	1.15932145927437	1.15932145927437\\
57.875	0.24306	1.23063550143786	1.23063550143786\\
57.875	0.24672	1.30400720565347	1.30400720565347\\
57.875	0.25038	1.37943657192119	1.37943657192119\\
57.875	0.25404	1.45692360024102	1.45692360024102\\
57.875	0.2577	1.53646829061297	1.53646829061297\\
57.875	0.26136	1.61807064303703	1.61807064303703\\
57.875	0.26502	1.7017306575132	1.7017306575132\\
57.875	0.26868	1.78744833404148	1.78744833404148\\
57.875	0.27234	1.87522367262187	1.87522367262187\\
57.875	0.276	1.96505667325438	1.96505667325438\\
58.25	0.093	0.000155750516743325	0.000155750516743325\\
58.25	0.09666	-0.0113700070032641	-0.0113700070032641\\
58.25	0.10032	-0.0208381024711588	-0.0208381024711588\\
58.25	0.10398	-0.0282485358869413	-0.0282485358869413\\
58.25	0.10764	-0.0336013072506113	-0.0336013072506113\\
58.25	0.1113	-0.0368964165621679	-0.0368964165621679\\
58.25	0.11496	-0.038133863821612	-0.038133863821612\\
58.25	0.11862	-0.037313649028945	-0.037313649028945\\
58.25	0.12228	-0.034435772184165	-0.034435772184165\\
58.25	0.12594	-0.0295002332872729	-0.0295002332872729\\
58.25	0.1296	-0.0225070323382686	-0.0225070323382686\\
58.25	0.13326	-0.0134561693371509	-0.0134561693371509\\
58.25	0.13692	-0.00234764428392142	-0.00234764428392142\\
58.25	0.14058	0.0108185428214211	0.0108185428214211\\
58.25	0.14424	0.0260423919788755	0.0260423919788755\\
58.25	0.1479	0.0433239031884431	0.0433239031884431\\
58.25	0.15156	0.0626630764501221	0.0626630764501221\\
58.25	0.15522	0.084059911763914	0.084059911763914\\
58.25	0.15888	0.107514409129819	0.107514409129819\\
58.25	0.16254	0.133026568547835	0.133026568547835\\
58.25	0.1662	0.160596390017964	0.160596390017964\\
58.25	0.16986	0.190223873540206	0.190223873540206\\
58.25	0.17352	0.22190901911456	0.22190901911456\\
58.25	0.17718	0.255651826741026	0.255651826741026\\
58.25	0.18084	0.291452296419605	0.291452296419605\\
58.25	0.1845	0.329310428150297	0.329310428150297\\
58.25	0.18816	0.369226221933101	0.369226221933101\\
58.25	0.19182	0.411199677768017	0.411199677768017\\
58.25	0.19548	0.455230795655045	0.455230795655045\\
58.25	0.19914	0.501319575594187	0.501319575594187\\
58.25	0.2028	0.549466017585441	0.549466017585441\\
58.25	0.20646	0.599670121628807	0.599670121628807\\
58.25	0.21012	0.651931887724286	0.651931887724286\\
58.25	0.21378	0.706251315871875	0.706251315871875\\
58.25	0.21744	0.762628406071579	0.762628406071579\\
58.25	0.2211	0.821063158323394	0.821063158323394\\
58.25	0.22476	0.881555572627323	0.881555572627323\\
58.25	0.22842	0.944105648983363	0.944105648983363\\
58.25	0.23208	1.00871338739152	1.00871338739152\\
58.25	0.23574	1.07537878785178	1.07537878785178\\
58.25	0.2394	1.14410185036416	1.14410185036416\\
58.25	0.24306	1.21488257492865	1.21488257492865\\
58.25	0.24672	1.28772096154525	1.28772096154525\\
58.25	0.25038	1.36261701021397	1.36261701021397\\
58.25	0.25404	1.4395707209348	1.4395707209348\\
58.25	0.2577	1.51858209370774	1.51858209370774\\
58.25	0.26136	1.59965112853279	1.59965112853279\\
58.25	0.26502	1.68277782540995	1.68277782540995\\
58.25	0.26868	1.76796218433923	1.76796218433923\\
58.25	0.27234	1.85520420532062	1.85520420532062\\
58.25	0.276	1.94450388835412	1.94450388835412\\
58.625	0.093	0.0065754647840599	0.0065754647840599\\
58.625	0.09666	-0.00548361033495226	-0.00548361033495226\\
58.625	0.10032	-0.0154850234018518	-0.0154850234018518\\
58.625	0.10398	-0.0234287744166399	-0.0234287744166399\\
58.625	0.10764	-0.0293148633793138	-0.0293148633793138\\
58.625	0.1113	-0.0331432902898769	-0.0331432902898769\\
58.625	0.11496	-0.0349140551483258	-0.0349140551483258\\
58.625	0.11862	-0.0346271579546644	-0.0346271579546644\\
58.625	0.12228	-0.0322825987088884	-0.0322825987088884\\
58.625	0.12594	-0.027880377411001	-0.027880377411001\\
58.625	0.1296	-0.0214204940610014	-0.0214204940610014\\
58.625	0.13326	-0.0129029486588885	-0.0129029486588885\\
58.625	0.13692	-0.00232774120466472	-0.00232774120466472\\
58.625	0.14058	0.0103051283016731	0.0103051283016731\\
58.625	0.14424	0.0249956598601226	0.0249956598601226\\
58.625	0.1479	0.0417438534706847	0.0417438534706847\\
58.625	0.15156	0.0605497091333589	0.0605497091333589\\
58.625	0.15522	0.081413226848146	0.081413226848146\\
58.625	0.15888	0.104334406615046	0.104334406615046\\
58.625	0.16254	0.129313248434058	0.129313248434058\\
58.625	0.1662	0.156349752305182	0.156349752305182\\
58.625	0.16986	0.185443918228418	0.185443918228418\\
58.625	0.17352	0.216595746203767	0.216595746203767\\
58.625	0.17718	0.249805236231229	0.249805236231229\\
58.625	0.18084	0.285072388310802	0.285072388310802\\
58.625	0.1845	0.322397202442489	0.322397202442489\\
58.625	0.18816	0.361779678626288	0.361779678626288\\
58.625	0.19182	0.403219816862198	0.403219816862198\\
58.625	0.19548	0.446717617150223	0.446717617150223\\
58.625	0.19914	0.492273079490359	0.492273079490359\\
58.625	0.2028	0.539886203882608	0.539886203882608\\
58.625	0.20646	0.58955699032697	0.58955699032697\\
58.625	0.21012	0.641285438823443	0.641285438823443\\
58.625	0.21378	0.695071549372027	0.695071549372027\\
58.625	0.21744	0.750915321972727	0.750915321972727\\
58.625	0.2211	0.808816756625537	0.808816756625537\\
58.625	0.22476	0.86877585333046	0.86877585333046\\
58.625	0.22842	0.930792612087496	0.930792612087496\\
58.625	0.23208	0.994867032896644	0.994867032896644\\
58.625	0.23574	1.06099911575791	1.06099911575791\\
58.625	0.2394	1.12918886067128	1.12918886067128\\
58.625	0.24306	1.19943626763676	1.19943626763676\\
58.625	0.24672	1.27174133665436	1.27174133665436\\
58.625	0.25038	1.34610406772407	1.34610406772407\\
58.625	0.25404	1.42252446084589	1.42252446084589\\
58.625	0.2577	1.50100251601983	1.50100251601983\\
58.625	0.26136	1.58153823324588	1.58153823324588\\
58.625	0.26502	1.66413161252404	1.66413161252404\\
58.625	0.26868	1.74878265385431	1.74878265385431\\
58.625	0.27234	1.83549135723669	1.83549135723669\\
58.625	0.276	1.92425772267119	1.92425772267119\\
59	0.093	0.0133017982687011	0.0133017982687011\\
59	0.09666	0.000709405550684217	0.000709405550684217\\
59	0.10032	-0.00982532511522005	-0.00982532511522005\\
59	0.10398	-0.0183023937290121	-0.0183023937290121\\
59	0.10764	-0.0247218002906925	-0.0247218002906925\\
59	0.1113	-0.0290835448002595	-0.0290835448002595\\
59	0.11496	-0.031387627257714	-0.031387627257714\\
59	0.11862	-0.0316340476630574	-0.0316340476630574\\
59	0.12228	-0.0298228060162871	-0.0298228060162871\\
59	0.12594	-0.0259539023174045	-0.0259539023174045\\
59	0.1296	-0.0200273365664096	-0.0200273365664096\\
59	0.13326	-0.0120431087633024	-0.0120431087633024\\
59	0.13692	-0.00200121890808247	-0.00200121890808247\\
59	0.14058	0.0100983329992497	0.0100983329992497\\
59	0.14424	0.0242555469586936	0.0242555469586936\\
59	0.1479	0.0404704229702517	0.0404704229702517\\
59	0.15156	0.0587429610339212	0.0587429610339212\\
59	0.15522	0.0790731611497026	0.0790731611497026\\
59	0.15888	0.101461023317598	0.101461023317598\\
59	0.16254	0.125906547537605	0.125906547537605\\
59	0.1662	0.152409733809724	0.152409733809724\\
59	0.16986	0.180970582133955	0.180970582133955\\
59	0.17352	0.211589092510299	0.211589092510299\\
59	0.17718	0.244265264938756	0.244265264938756\\
59	0.18084	0.278999099419325	0.278999099419325\\
59	0.1845	0.315790595952007	0.315790595952007\\
59	0.18816	0.3546397545368	0.3546397545368\\
59	0.19182	0.395546575173706	0.395546575173706\\
59	0.19548	0.438511057862724	0.438511057862724\\
59	0.19914	0.483533202603856	0.483533202603856\\
59	0.2028	0.5306130093971	0.5306130093971\\
59	0.20646	0.579750478242457	0.579750478242457\\
59	0.21012	0.630945609139924	0.630945609139924\\
59	0.21378	0.684198402089505	0.684198402089505\\
59	0.21744	0.739508857091199	0.739508857091199\\
59	0.2211	0.796876974145004	0.796876974145004\\
59	0.22476	0.856302753250923	0.856302753250923\\
59	0.22842	0.917786194408953	0.917786194408953\\
59	0.23208	0.981327297619096	0.981327297619096\\
59	0.23574	1.04692606288135	1.04692606288135\\
59	0.2394	1.11458249019572	1.11458249019572\\
59	0.24306	1.1842965795622	1.1842965795622\\
59	0.24672	1.25606833098079	1.25606833098079\\
59	0.25038	1.3298977444515	1.3298977444515\\
59	0.25404	1.40578481997432	1.40578481997432\\
59	0.2577	1.48372955754925	1.48372955754925\\
59	0.26136	1.56373195717629	1.56373195717629\\
59	0.26502	1.64579201885544	1.64579201885544\\
59	0.26868	1.72990974258671	1.72990974258671\\
59	0.27234	1.81608512837009	1.81608512837009\\
59	0.276	1.90431817620558	1.90431817620558\\
59.375	0.093	0.0203347509706675	0.0203347509706675\\
59.375	0.09666	0.00720904065364492	0.00720904065364492\\
59.375	0.10032	-0.00385900761126412	-0.00385900761126412\\
59.375	0.10398	-0.0128693938240609	-0.0128693938240609\\
59.375	0.10764	-0.0198221179847461	-0.0198221179847461\\
59.375	0.1113	-0.0247171800933188	-0.0247171800933188\\
59.375	0.11496	-0.0275545801497781	-0.0275545801497781\\
59.375	0.11862	-0.0283343181541262	-0.0283343181541262\\
59.375	0.12228	-0.0270563941063606	-0.0270563941063606\\
59.375	0.12594	-0.0237208080064837	-0.0237208080064837\\
59.375	0.1296	-0.0183275598544936	-0.0183275598544936\\
59.375	0.13326	-0.0108766496503911	-0.0108766496503911\\
59.375	0.13692	-0.00136807739417688	-0.00136807739417688\\
59.375	0.14058	0.0101981569141505	0.0101981569141505\\
59.375	0.14424	0.0238220532745905	0.0238220532745905\\
59.375	0.1479	0.039503611687143	0.039503611687143\\
59.375	0.15156	0.0572428321518068	0.0572428321518068\\
59.375	0.15522	0.0770397146685835	0.0770397146685835\\
59.375	0.15888	0.0988942592374737	0.0988942592374737\\
59.375	0.16254	0.122806465858475	0.122806465858475\\
59.375	0.1662	0.14877633453159	0.14877633453159\\
59.375	0.16986	0.176803865256817	0.176803865256817\\
59.375	0.17352	0.206889058034156	0.206889058034156\\
59.375	0.17718	0.239031912863606	0.239031912863606\\
59.375	0.18084	0.27323242974517	0.27323242974517\\
59.375	0.1845	0.309490608678848	0.309490608678848\\
59.375	0.18816	0.347806449664636	0.347806449664636\\
59.375	0.19182	0.388179952702538	0.388179952702538\\
59.375	0.19548	0.430611117792551	0.430611117792551\\
59.375	0.19914	0.475099944934678	0.475099944934678\\
59.375	0.2028	0.521646434128916	0.521646434128916\\
59.375	0.20646	0.570250585375268	0.570250585375268\\
59.375	0.21012	0.620912398673731	0.620912398673731\\
59.375	0.21378	0.673631874024306	0.673631874024306\\
59.375	0.21744	0.728409011426995	0.728409011426995\\
59.375	0.2211	0.785243810881796	0.785243810881796\\
59.375	0.22476	0.844136272388709	0.844136272388709\\
59.375	0.22842	0.905086395947735	0.905086395947735\\
59.375	0.23208	0.968094181558873	0.968094181558873\\
59.375	0.23574	1.03315962922212	1.03315962922212\\
59.375	0.2394	1.10028273893749	1.10028273893749\\
59.375	0.24306	1.16946351070496	1.16946351070496\\
59.375	0.24672	1.24070194452455	1.24070194452455\\
59.375	0.25038	1.31399804039625	1.31399804039625\\
59.375	0.25404	1.38935179832006	1.38935179832006\\
59.375	0.2577	1.46676321829599	1.46676321829599\\
59.375	0.26136	1.54623230032403	1.54623230032403\\
59.375	0.26502	1.62775904440418	1.62775904440418\\
59.375	0.26868	1.71134345053644	1.71134345053644\\
59.375	0.27234	1.79698551872081	1.79698551872081\\
59.375	0.276	1.8846852489573	1.8846852489573\\
59.75	0.093	0.0276743228899585	0.0276743228899585\\
59.75	0.09666	0.0140152949739321	0.0140152949739321\\
59.75	0.10032	0.00241392911001737	0.00241392911001737\\
59.75	0.10398	-0.00712977470178511	-0.00712977470178511\\
59.75	0.10764	-0.014615816461475	-0.014615816461475\\
59.75	0.1113	-0.0200441961690525	-0.0200441961690525\\
59.75	0.11496	-0.0234149138245165	-0.0234149138245165\\
59.75	0.11862	-0.0247279694278695	-0.0247279694278695\\
59.75	0.12228	-0.0239833629791095	-0.0239833629791095\\
59.75	0.12594	-0.0211810944782365	-0.0211810944782365\\
59.75	0.1296	-0.0163211639252512	-0.0163211639252512\\
59.75	0.13326	-0.00940357132015435	-0.00940357132015435\\
59.75	0.13692	-0.00042831666294485	-0.00042831666294485\\
59.75	0.14058	0.0106046000463778	0.0106046000463778\\
59.75	0.14424	0.0236951788078121	0.0236951788078121\\
59.75	0.1479	0.0388434196213598	0.0388434196213598\\
59.75	0.15156	0.0560493224870189	0.0560493224870189\\
59.75	0.15522	0.0753128874047908	0.0753128874047908\\
59.75	0.15888	0.0966341143746754	0.0966341143746754\\
59.75	0.16254	0.120013003396672	0.120013003396672\\
59.75	0.1662	0.145449554470782	0.145449554470782\\
59.75	0.16986	0.172943767597003	0.172943767597003\\
59.75	0.17352	0.202495642775337	0.202495642775337\\
59.75	0.17718	0.234105180005783	0.234105180005783\\
59.75	0.18084	0.267772379288342	0.267772379288342\\
59.75	0.1845	0.303497240623014	0.303497240623014\\
59.75	0.18816	0.341279764009798	0.341279764009798\\
59.75	0.19182	0.381119949448694	0.381119949448694\\
59.75	0.19548	0.423017796939702	0.423017796939702\\
59.75	0.19914	0.466973306482824	0.466973306482824\\
59.75	0.2028	0.512986478078058	0.512986478078058\\
59.75	0.20646	0.561057311725405	0.561057311725405\\
59.75	0.21012	0.611185807424862	0.611185807424862\\
59.75	0.21378	0.663371965176434	0.663371965176434\\
59.75	0.21744	0.717615784980117	0.717615784980117\\
59.75	0.2211	0.773917266835913	0.773917266835913\\
59.75	0.22476	0.832276410743821	0.832276410743821\\
59.75	0.22842	0.892693216703841	0.892693216703841\\
59.75	0.23208	0.955167684715974	0.955167684715974\\
59.75	0.23574	1.01969981478022	1.01969981478022\\
59.75	0.2394	1.08628960689658	1.08628960689658\\
59.75	0.24306	1.15493706106505	1.15493706106505\\
59.75	0.24672	1.22564217728563	1.22564217728563\\
59.75	0.25038	1.29840495555833	1.29840495555833\\
59.75	0.25404	1.37322539588313	1.37322539588313\\
59.75	0.2577	1.45010349826006	1.45010349826006\\
59.75	0.26136	1.52903926268909	1.52903926268909\\
59.75	0.26502	1.61003268917023	1.61003268917023\\
59.75	0.26868	1.69308377770349	1.69308377770349\\
59.75	0.27234	1.77819252828886	1.77819252828886\\
59.75	0.276	1.86535894092634	1.86535894092634\\
60.125	0.093	0.0353205140265742	0.0353205140265742\\
60.125	0.09666	0.0211281685115421	0.0211281685115421\\
60.125	0.10032	0.00899348504862263	0.00899348504862263\\
60.125	0.10398	-0.00108353636218461	-0.00108353636218461\\
60.125	0.10764	-0.0091028957208793	-0.0091028957208793\\
60.125	0.1113	-0.0150645930274615	-0.0150645930274615\\
60.125	0.11496	-0.0189686282819312	-0.0189686282819312\\
60.125	0.11862	-0.0208150014842889	-0.0208150014842889\\
60.125	0.12228	-0.0206037126345338	-0.0206037126345338\\
60.125	0.12594	-0.0183347617326663	-0.0183347617326663\\
60.125	0.1296	-0.0140081487786858	-0.0140081487786858\\
60.125	0.13326	-0.00762387377259377	-0.00762387377259377\\
60.125	0.13692	0.000818063285610959	0.000818063285610959\\
60.125	0.14058	0.0113176623959279	0.0113176623959279\\
60.125	0.14424	0.0238749235583575	0.0238749235583575\\
60.125	0.1479	0.0384898467729005	0.0384898467729005\\
60.125	0.15156	0.0551624320395547	0.0551624320395547\\
60.125	0.15522	0.0738926793583219	0.0738926793583219\\
60.125	0.15888	0.0946805887292008	0.0946805887292008\\
60.125	0.16254	0.117526160152193	0.117526160152193\\
60.125	0.1662	0.142429393627298	0.142429393627298\\
60.125	0.16986	0.169390289154514	0.169390289154514\\
60.125	0.17352	0.198408846733842	0.198408846733842\\
60.125	0.17718	0.229485066365285	0.229485066365285\\
60.125	0.18084	0.262618948048838	0.262618948048838\\
60.125	0.1845	0.297810491784506	0.297810491784506\\
60.125	0.18816	0.335059697572285	0.335059697572285\\
60.125	0.19182	0.374366565412175	0.374366565412175\\
60.125	0.19548	0.415731095304179	0.415731095304179\\
60.125	0.19914	0.459153287248296	0.459153287248296\\
60.125	0.2028	0.504633141244525	0.504633141244525\\
60.125	0.20646	0.552170657292866	0.552170657292866\\
60.125	0.21012	0.601765835393318	0.601765835393318\\
60.125	0.21378	0.653418675545885	0.653418675545885\\
60.125	0.21744	0.707129177750563	0.707129177750563\\
60.125	0.2211	0.762897342007354	0.762897342007354\\
60.125	0.22476	0.820723168316258	0.820723168316258\\
60.125	0.22842	0.880606656677273	0.880606656677273\\
60.125	0.23208	0.942547807090402	0.942547807090402\\
60.125	0.23574	1.00654661955564	1.00654661955564\\
60.125	0.2394	1.07260309407299	1.07260309407299\\
60.125	0.24306	1.14071723064246	1.14071723064246\\
60.125	0.24672	1.21088902926404	1.21088902926404\\
60.125	0.25038	1.28311848993773	1.28311848993773\\
60.125	0.25404	1.35740561266353	1.35740561266353\\
60.125	0.2577	1.43375039744145	1.43375039744145\\
60.125	0.26136	1.51215284427148	1.51215284427148\\
60.125	0.26502	1.59261295315361	1.59261295315361\\
60.125	0.26868	1.67513072408787	1.67513072408787\\
60.125	0.27234	1.75970615707423	1.75970615707423\\
60.125	0.276	1.84633925211271	1.84633925211271\\
60.5	0.093	0.043273324380515	0.043273324380515\\
60.5	0.09666	0.028547661266479	0.028547661266479\\
60.5	0.10032	0.0158796602045539	0.0158796602045539\\
60.5	0.10398	0.00526932119474188	0.00526932119474188\\
60.5	0.10764	-0.00328335576295757	-0.00328335576295757\\
60.5	0.1113	-0.00977837066854548	-0.00977837066854548\\
60.5	0.11496	-0.0142157235220191	-0.0142157235220191\\
60.5	0.11862	-0.0165954143233824	-0.0165954143233824\\
60.5	0.12228	-0.016917443072632	-0.016917443072632\\
60.5	0.12594	-0.0151818097697693	-0.0151818097697693\\
60.5	0.1296	-0.0113885144147945	-0.0113885144147945\\
60.5	0.13326	-0.00553755700770719	-0.00553755700770719\\
60.5	0.13692	0.00237106245149188	0.00237106245149188\\
60.5	0.14058	0.0123373439628049	0.0123373439628049\\
60.5	0.14424	0.0243612875262298	0.0243612875262298\\
60.5	0.1479	0.0384428931417662	0.0384428931417662\\
60.5	0.15156	0.0545821608094166	0.0545821608094166\\
60.5	0.15522	0.0727790905291781	0.0727790905291781\\
60.5	0.15888	0.0930336823010522	0.0930336823010522\\
60.5	0.16254	0.11534593612504	0.11534593612504\\
60.5	0.1662	0.139715852001139	0.139715852001139\\
60.5	0.16986	0.166143429929352	0.166143429929352\\
60.5	0.17352	0.194628669909675	0.194628669909675\\
60.5	0.17718	0.225171571942111	0.225171571942111\\
60.5	0.18084	0.25777213602666	0.25777213602666\\
60.5	0.1845	0.292430362163322	0.292430362163322\\
60.5	0.18816	0.329146250352096	0.329146250352096\\
60.5	0.19182	0.367919800592982	0.367919800592982\\
60.5	0.19548	0.40875101288598	0.40875101288598\\
60.5	0.19914	0.451639887231093	0.451639887231093\\
60.5	0.2028	0.496586423628316	0.496586423628316\\
60.5	0.20646	0.543590622077653	0.543590622077653\\
60.5	0.21012	0.592652482579101	0.592652482579101\\
60.5	0.21378	0.643772005132661	0.643772005132661\\
60.5	0.21744	0.696949189738335	0.696949189738335\\
60.5	0.2211	0.752184036396121	0.752184036396121\\
60.5	0.22476	0.809476545106018	0.809476545106018\\
60.5	0.22842	0.86882671586803	0.86882671586803\\
60.5	0.23208	0.930234548682153	0.930234548682153\\
60.5	0.23574	0.993700043548389	0.993700043548389\\
60.5	0.2394	1.05922320046674	1.05922320046674\\
60.5	0.24306	1.1268040194372	1.1268040194372\\
60.5	0.24672	1.19644250045977	1.19644250045977\\
60.5	0.25038	1.26813864353446	1.26813864353446\\
60.5	0.25404	1.34189244866125	1.34189244866125\\
60.5	0.2577	1.41770391584016	1.41770391584016\\
60.5	0.26136	1.49557304507119	1.49557304507119\\
60.5	0.26502	1.57549983635432	1.57549983635432\\
60.5	0.26868	1.65748428968957	1.65748428968957\\
60.5	0.27234	1.74152640507693	1.74152640507693\\
60.5	0.276	1.8276261825164	1.8276261825164\\
60.875	0.093	0.0515327539517805	0.0515327539517805\\
60.875	0.09666	0.0362737732387388	0.0362737732387388\\
60.875	0.10032	0.0230724545778098	0.0230724545778098\\
60.875	0.10398	0.0119287979689922	0.0119287979689922\\
60.875	0.10764	0.00284280341228704	0.00284280341228704\\
60.875	0.1113	-0.00418552909230474	-0.00418552909230474\\
60.875	0.11496	-0.00915619954478397	-0.00915619954478397\\
60.875	0.11862	-0.0120692079451521	-0.0120692079451521\\
60.875	0.12228	-0.0129245542934073	-0.0129245542934073\\
60.875	0.12594	-0.0117222385895495	-0.0117222385895495\\
60.875	0.1296	-0.00846226083357937	-0.00846226083357937\\
60.875	0.13326	-0.00314462102549684	-0.00314462102549684\\
60.875	0.13692	0.00423068083469746	0.00423068083469746\\
60.875	0.14058	0.0136636447470049	0.0136636447470049\\
60.875	0.14424	0.025154270711425	0.025154270711425\\
60.875	0.1479	0.0387025587279575	0.0387025587279575\\
60.875	0.15156	0.0543085087966022	0.0543085087966022\\
60.875	0.15522	0.0719721209173581	0.0719721209173581\\
60.875	0.15888	0.0916933950902283	0.0916933950902283\\
60.875	0.16254	0.11347233131521	0.11347233131521\\
60.875	0.1662	0.137308929592304	0.137308929592304\\
60.875	0.16986	0.163203189921512	0.163203189921512\\
60.875	0.17352	0.19115511230283	0.19115511230283\\
60.875	0.17718	0.221164696736262	0.221164696736262\\
60.875	0.18084	0.253231943221805	0.253231943221805\\
60.875	0.1845	0.287356851759462	0.287356851759462\\
60.875	0.18816	0.323539422349231	0.323539422349231\\
60.875	0.19182	0.361779654991113	0.361779654991113\\
60.875	0.19548	0.402077549685106	0.402077549685106\\
60.875	0.19914	0.444433106431213	0.444433106431213\\
60.875	0.2028	0.488846325229431	0.488846325229431\\
60.875	0.20646	0.535317206079764	0.535317206079764\\
60.875	0.21012	0.583845748982207	0.583845748982207\\
60.875	0.21378	0.634431953936762	0.634431953936762\\
60.875	0.21744	0.68707582094343	0.68707582094343\\
60.875	0.2211	0.741777350002211	0.741777350002211\\
60.875	0.22476	0.798536541113105	0.798536541113105\\
60.875	0.22842	0.85735339427611	0.85735339427611\\
60.875	0.23208	0.918227909491229	0.918227909491229\\
60.875	0.23574	0.98116008675846	0.98116008675846\\
60.875	0.2394	1.0461499260778	1.0461499260778\\
60.875	0.24306	1.11319742744926	1.11319742744926\\
60.875	0.24672	1.18230259087283	1.18230259087283\\
60.875	0.25038	1.25346541634851	1.25346541634851\\
60.875	0.25404	1.3266859038763	1.3266859038763\\
60.875	0.2577	1.40196405345621	1.40196405345621\\
60.875	0.26136	1.47929986508822	1.47929986508822\\
60.875	0.26502	1.55869333877235	1.55869333877235\\
60.875	0.26868	1.64014447450859	1.64014447450859\\
60.875	0.27234	1.72365327229695	1.72365327229695\\
60.875	0.276	1.80921973213742	1.80921973213742\\
61.25	0.093	0.0600988027403715	0.0600988027403715\\
61.25	0.09666	0.044306504428326	0.044306504428326\\
61.25	0.10032	0.0305718681683913	0.0305718681683913\\
61.25	0.10398	0.0188948939605689	0.0188948939605689\\
61.25	0.10764	0.00927558180485899	0.00927558180485899\\
61.25	0.1113	0.00171393170126155	0.00171393170126155\\
61.25	0.11496	-0.00379005635022334	-0.00379005635022334\\
61.25	0.11862	-0.00723638234959623	-0.00723638234959623\\
61.25	0.12228	-0.00862504629685446	-0.00862504629685446\\
61.25	0.12594	-0.00795604819200224	-0.00795604819200224\\
61.25	0.1296	-0.00522938803503781	-0.00522938803503781\\
61.25	0.13326	-0.000445065825960933	-0.000445065825960933\\
61.25	0.13692	0.0063969184352286	0.0063969184352286\\
61.25	0.14058	0.0152965647485312	0.0152965647485312\\
61.25	0.14424	0.0262538731139457	0.0262538731139457\\
61.25	0.1479	0.0392688435314743	0.0392688435314743\\
61.25	0.15156	0.0543414760011134	0.0543414760011134\\
61.25	0.15522	0.0714717705228654	0.0714717705228654\\
61.25	0.15888	0.09065972709673	0.09065972709673\\
61.25	0.16254	0.111905345722707	0.111905345722707\\
61.25	0.1662	0.135208626400796	0.135208626400796\\
61.25	0.16986	0.160569569130998	0.160569569130998\\
61.25	0.17352	0.187988173913312	0.187988173913312\\
61.25	0.17718	0.217464440747738	0.217464440747738\\
61.25	0.18084	0.248998369634277	0.248998369634277\\
61.25	0.1845	0.28258996057293	0.28258996057293\\
61.25	0.18816	0.318239213563693	0.318239213563693\\
61.25	0.19182	0.355946128606569	0.355946128606569\\
61.25	0.19548	0.395710705701558	0.395710705701558\\
61.25	0.19914	0.437532944848659	0.437532944848659\\
61.25	0.2028	0.481412846047873	0.481412846047873\\
61.25	0.20646	0.527350409299201	0.527350409299201\\
61.25	0.21012	0.575345634602638	0.575345634602638\\
61.25	0.21378	0.625398521958189	0.625398521958189\\
61.25	0.21744	0.677509071365853	0.677509071365853\\
61.25	0.2211	0.731677282825628	0.731677282825628\\
61.25	0.22476	0.787903156337517	0.787903156337517\\
61.25	0.22842	0.846186691901517	0.846186691901517\\
61.25	0.23208	0.906527889517631	0.906527889517631\\
61.25	0.23574	0.968926749185857	0.968926749185857\\
61.25	0.2394	1.03338327090619	1.03338327090619\\
61.25	0.24306	1.09989745467864	1.09989745467864\\
61.25	0.24672	1.16846930050321	1.16846930050321\\
61.25	0.25038	1.23909880837988	1.23909880837988\\
61.25	0.25404	1.31178597830867	1.31178597830867\\
61.25	0.2577	1.38653081028957	1.38653081028957\\
61.25	0.26136	1.46333330432258	1.46333330432258\\
61.25	0.26502	1.54219346040771	1.54219346040771\\
61.25	0.26868	1.62311127854495	1.62311127854495\\
61.25	0.27234	1.7060867587343	1.7060867587343\\
61.25	0.276	1.79111990097576	1.79111990097576\\
61.625	0.093	0.0689714707462881	0.0689714707462881\\
61.625	0.09666	0.052645854835236	0.052645854835236\\
61.625	0.10032	0.0383779009762966	0.0383779009762966\\
61.625	0.10398	0.0261676091694694	0.0261676091694694\\
61.625	0.10764	0.0160149794147547	0.0160149794147547\\
61.625	0.1113	0.00792001171215162	0.00792001171215162\\
61.625	0.11496	0.00188270606166285	0.00188270606166285\\
61.625	0.11862	-0.0020969375367148	-0.0020969375367148\\
61.625	0.12228	-0.00401891908297958	-0.00401891908297958\\
61.625	0.12594	-0.00388323857713213	-0.00388323857713213\\
61.625	0.1296	-0.00168989601917247	-0.00168989601917247\\
61.625	0.13326	0.00256110859090053	0.00256110859090053\\
61.625	0.13692	0.0088697752530853	0.0088697752530853\\
61.625	0.14058	0.0172361039673832	0.0172361039673832\\
61.625	0.14424	0.0276600947337919	0.0276600947337919\\
61.625	0.1479	0.0401417475523149	0.0401417475523149\\
61.625	0.15156	0.0546810624229492	0.0546810624229492\\
61.625	0.15522	0.0712780393456964	0.0712780393456964\\
61.625	0.15888	0.0899326783205554	0.0899326783205554\\
61.625	0.16254	0.110644979347528	0.110644979347528\\
61.625	0.1662	0.133414942426612	0.133414942426612\\
61.625	0.16986	0.158242567557809	0.158242567557809\\
61.625	0.17352	0.185127854741119	0.185127854741119\\
61.625	0.17718	0.21407080397654	0.21407080397654\\
61.625	0.18084	0.245071415264073	0.245071415264073\\
61.625	0.1845	0.278129688603721	0.278129688603721\\
61.625	0.18816	0.31324562399548	0.31324562399548\\
61.625	0.19182	0.35041922143935	0.35041922143935\\
61.625	0.19548	0.389650480935334	0.389650480935334\\
61.625	0.19914	0.430939402483431	0.430939402483431\\
61.625	0.2028	0.47428598608364	0.47428598608364\\
61.625	0.20646	0.519690231735961	0.519690231735961\\
61.625	0.21012	0.567152139440395	0.567152139440395\\
61.625	0.21378	0.61667170919694	0.61667170919694\\
61.625	0.21744	0.668248941005599	0.668248941005599\\
61.625	0.2211	0.72188383486637	0.72188383486637\\
61.625	0.22476	0.777576390779253	0.777576390779253\\
61.625	0.22842	0.835326608744249	0.835326608744249\\
61.625	0.23208	0.895134488761357	0.895134488761357\\
61.625	0.23574	0.957000030830579	0.957000030830579\\
61.625	0.2394	1.02092323495191	1.02092323495191\\
61.625	0.24306	1.08690410112536	1.08690410112536\\
61.625	0.24672	1.15494262935091	1.15494262935091\\
61.625	0.25038	1.22503881962859	1.22503881962859\\
61.625	0.25404	1.29719267195837	1.29719267195837\\
61.625	0.2577	1.37140418634026	1.37140418634026\\
61.625	0.26136	1.44767336277427	1.44767336277427\\
61.625	0.26502	1.52600020126039	1.52600020126039\\
61.625	0.26868	1.60638470179862	1.60638470179862\\
61.625	0.27234	1.68882686438897	1.68882686438897\\
61.625	0.276	1.77332668903143	1.77332668903143\\
62	0.093	0.0781507579695275	0.0781507579695275\\
62	0.09666	0.0612918244594707	0.0612918244594707\\
62	0.10032	0.0464905530015265	0.0464905530015265\\
62	0.10398	0.0337469435956936	0.0337469435956936\\
62	0.10764	0.0230609962419742	0.0230609962419742\\
62	0.1113	0.0144327109403672	0.0144327109403672\\
62	0.11496	0.00786208769087282	0.00786208769087282\\
62	0.11862	0.0033491264934904	0.0033491264934904\\
62	0.12228	0.000893827348220855	0.000893827348220855\\
62	0.12594	0.000496190255063533	0.000496190255063533\\
62	0.1296	0.00215621521401843	0.00215621521401843\\
62	0.13326	0.00587390222508577	0.00587390222508577\\
62	0.13692	0.0116492512882658	0.0116492512882658\\
62	0.14058	0.0194822624035571	0.0194822624035571\\
62	0.14424	0.029372935570962	0.029372935570962\\
62	0.1479	0.0413212707904793	0.0413212707904793\\
62	0.15156	0.0553272680621089	0.0553272680621089\\
62	0.15522	0.0713909273858513	0.0713909273858513\\
62	0.15888	0.0895122487617064	0.0895122487617064\\
62	0.16254	0.109691232189673	0.109691232189673\\
62	0.1662	0.131927877669752	0.131927877669752\\
62	0.16986	0.156222185201944	0.156222185201944\\
62	0.17352	0.182574154786249	0.182574154786249\\
62	0.17718	0.210983786422665	0.210983786422665\\
62	0.18084	0.241451080111194	0.241451080111194\\
62	0.1845	0.273976035851836	0.273976035851836\\
62	0.18816	0.30855865364459	0.30855865364459\\
62	0.19182	0.345198933489456	0.345198933489456\\
62	0.19548	0.383896875386435	0.383896875386435\\
62	0.19914	0.424652479335526	0.424652479335526\\
62	0.2028	0.46746574533673	0.46746574533673\\
62	0.20646	0.512336673390047	0.512336673390047\\
62	0.21012	0.559265263495474	0.559265263495474\\
62	0.21378	0.608251515653015	0.608251515653015\\
62	0.21744	0.65929542986267	0.65929542986267\\
62	0.2211	0.712397006124435	0.712397006124435\\
62	0.22476	0.767556244438313	0.767556244438313\\
62	0.22842	0.824773144804304	0.824773144804304\\
62	0.23208	0.884047707222408	0.884047707222408\\
62	0.23574	0.945379931692623	0.945379931692623\\
62	0.2394	1.00876981821495	1.00876981821495\\
62	0.24306	1.07421736678939	1.07421736678939\\
62	0.24672	1.14172257741594	1.14172257741594\\
62	0.25038	1.21128545009461	1.21128545009461\\
62	0.25404	1.28290598482539	1.28290598482539\\
62	0.2577	1.35658418160828	1.35658418160828\\
62	0.26136	1.43232004044328	1.43232004044328\\
62	0.26502	1.5101135613304	1.5101135613304\\
62	0.26868	1.58996474426962	1.58996474426962\\
62	0.27234	1.67187358926096	1.67187358926096\\
62	0.276	1.75584009630442	1.75584009630442\\
62.375	0.093	0.0876366644100934	0.0876366644100934\\
62.375	0.09666	0.0702444133010309	0.0702444133010309\\
62.375	0.10032	0.054909824244082	0.054909824244082\\
62.375	0.10398	0.0416328972392452	0.0416328972392452\\
62.375	0.10764	0.0304136322865202	0.0304136322865202\\
62.375	0.1113	0.0212520293859075	0.0212520293859075\\
62.375	0.11496	0.0141480885374083	0.0141480885374083\\
62.375	0.11862	0.00910180974102115	0.00910180974102115\\
62.375	0.12228	0.00611319299674595	0.00611319299674595\\
62.375	0.12594	0.00518223830458386	0.00518223830458386\\
62.375	0.1296	0.00630894566453399	0.00630894566453399\\
62.375	0.13326	0.00949331507659568	0.00949331507659568\\
62.375	0.13692	0.0147353465407709	0.0147353465407709\\
62.375	0.14058	0.0220350400570584	0.0220350400570584\\
62.375	0.14424	0.0313923956254585	0.0313923956254585\\
62.375	0.1479	0.0428074132459701	0.0428074132459701\\
62.375	0.15156	0.0562800929185949	0.0562800929185949\\
62.375	0.15522	0.0718104346433326	0.0718104346433326\\
62.375	0.15888	0.089398438420182	0.089398438420182\\
62.375	0.16254	0.109044104249143	0.109044104249143\\
62.375	0.1662	0.130747432130218	0.130747432130218\\
62.375	0.16986	0.154508422063405	0.154508422063405\\
62.375	0.17352	0.180327074048704	0.180327074048704\\
62.375	0.17718	0.208203388086116	0.208203388086116\\
62.375	0.18084	0.238137364175639	0.238137364175639\\
62.375	0.1845	0.270129002317276	0.270129002317276\\
62.375	0.18816	0.304178302511025	0.304178302511025\\
62.375	0.19182	0.340285264756886	0.340285264756886\\
62.375	0.19548	0.37844988905486	0.37844988905486\\
62.375	0.19914	0.418672175404947	0.418672175404947\\
62.375	0.2028	0.460952123807146	0.460952123807146\\
62.375	0.20646	0.505289734261458	0.505289734261458\\
62.375	0.21012	0.551685006767881	0.551685006767881\\
62.375	0.21378	0.600137941326417	0.600137941326417\\
62.375	0.21744	0.650648537937065	0.650648537937065\\
62.375	0.2211	0.703216796599826	0.703216796599826\\
62.375	0.22476	0.757842717314699	0.757842717314699\\
62.375	0.22842	0.814526300081685	0.814526300081685\\
62.375	0.23208	0.873267544900783	0.873267544900783\\
62.375	0.23574	0.934066451771995	0.934066451771995\\
62.375	0.2394	0.996923020695317	0.996923020695317\\
62.375	0.24306	1.06183725167075	1.06183725167075\\
62.375	0.24672	1.1288091446983	1.1288091446983\\
62.375	0.25038	1.19783869977796	1.19783869977796\\
62.375	0.25404	1.26892591690973	1.26892591690973\\
62.375	0.2577	1.34207079609362	1.34207079609362\\
62.375	0.26136	1.41727333732962	1.41727333732962\\
62.375	0.26502	1.49453354061773	1.49453354061773\\
62.375	0.26868	1.57385140595795	1.57385140595795\\
62.375	0.27234	1.65522693335028	1.65522693335028\\
62.375	0.276	1.73866012279473	1.73866012279473\\
62.75	0.093	0.0974291900679822	0.0974291900679822\\
62.75	0.09666	0.0795036213599158	0.0795036213599158\\
62.75	0.10032	0.0636357147039612	0.0636357147039612\\
62.75	0.10398	0.0498254701001188	0.0498254701001188\\
62.75	0.10764	0.038072887548389	0.038072887548389\\
62.75	0.1113	0.0283779670487725	0.0283779670487725\\
62.75	0.11496	0.0207407086012676	0.0207407086012676\\
62.75	0.11862	0.0151611122058748	0.0151611122058748\\
62.75	0.12228	0.0116391778625957	0.0116391778625957\\
62.75	0.12594	0.010174905571428	0.010174905571428\\
62.75	0.1296	0.0107682953323724	0.0107682953323724\\
62.75	0.13326	0.0134193471454302	0.0134193471454302\\
62.75	0.13692	0.0181280610106007	0.0181280610106007\\
62.75	0.14058	0.0248944369278825	0.0248944369278825\\
62.75	0.14424	0.0337184748972779	0.0337184748972779\\
62.75	0.1479	0.0446001749187848	0.0446001749187848\\
62.75	0.15156	0.0575395369924048	0.0575395369924048\\
62.75	0.15522	0.0725365611181368	0.0725365611181368\\
62.75	0.15888	0.0895912472959814	0.0895912472959814\\
62.75	0.16254	0.108703595525938	0.108703595525938\\
62.75	0.1662	0.129873605808008	0.129873605808008\\
62.75	0.16986	0.15310127814219	0.15310127814219\\
62.75	0.17352	0.178386612528483	0.178386612528483\\
62.75	0.17718	0.20572960896689	0.20572960896689\\
62.75	0.18084	0.235130267457409	0.235130267457409\\
62.75	0.1845	0.266588588000042	0.266588588000042\\
62.75	0.18816	0.300104570594786	0.300104570594786\\
62.75	0.19182	0.335678215241641	0.335678215241641\\
62.75	0.19548	0.37330952194061	0.37330952194061\\
62.75	0.19914	0.412998490691691	0.412998490691691\\
62.75	0.2028	0.454745121494885	0.454745121494885\\
62.75	0.20646	0.498549414350193	0.498549414350193\\
62.75	0.21012	0.544411369257611	0.544411369257611\\
62.75	0.21378	0.592330986217141	0.592330986217141\\
62.75	0.21744	0.642308265228785	0.642308265228785\\
62.75	0.2211	0.69434320629254	0.69434320629254\\
62.75	0.22476	0.748435809408408	0.748435809408408\\
62.75	0.22842	0.80458607457639	0.80458607457639\\
62.75	0.23208	0.862794001796483	0.862794001796483\\
62.75	0.23574	0.923059591068689	0.923059591068689\\
62.75	0.2394	0.985382842393007	0.985382842393007\\
62.75	0.24306	1.04976375576944	1.04976375576944\\
62.75	0.24672	1.11620233119798	1.11620233119798\\
62.75	0.25038	1.18469856867864	1.18469856867864\\
62.75	0.25404	1.2552524682114	1.2552524682114\\
62.75	0.2577	1.32786402979629	1.32786402979629\\
62.75	0.26136	1.40253325343328	1.40253325343328\\
62.75	0.26502	1.47926013912238	1.47926013912238\\
62.75	0.26868	1.5580446868636	1.5580446868636\\
62.75	0.27234	1.63888689665693	1.63888689665693\\
62.75	0.276	1.72178676850237	1.72178676850237\\
63.125	0.093	0.107528334943197	0.107528334943197\\
63.125	0.09666	0.0890694486361254	0.0890694486361254\\
63.125	0.10032	0.072668224381166	0.072668224381166\\
63.125	0.10398	0.0583246621783189	0.0583246621783189\\
63.125	0.10764	0.0460387620275843	0.0460387620275843\\
63.125	0.1113	0.0358105239289621	0.0358105239289621\\
63.125	0.11496	0.0276399478824525	0.0276399478824525\\
63.125	0.11862	0.0215270338880549	0.0215270338880549\\
63.125	0.12228	0.0174717819457701	0.0174717819457701\\
63.125	0.12594	0.0154741920555976	0.0154741920555976\\
63.125	0.1296	0.0155342642175382	0.0155342642175382\\
63.125	0.13326	0.0176519984315904	0.0176519984315904\\
63.125	0.13692	0.0218273946977552	0.0218273946977552\\
63.125	0.14058	0.0280604530160322	0.0280604530160322\\
63.125	0.14424	0.0363511733864219	0.0363511733864219\\
63.125	0.1479	0.0466995558089249	0.0466995558089249\\
63.125	0.15156	0.0591056002835402	0.0591056002835402\\
63.125	0.15522	0.0735693068102665	0.0735693068102665\\
63.125	0.15888	0.0900906753891064	0.0900906753891064\\
63.125	0.16254	0.108669706020057	0.108669706020057\\
63.125	0.1662	0.129306398703123	0.129306398703123\\
63.125	0.16986	0.152000753438299	0.152000753438299\\
63.125	0.17352	0.176752770225589	0.176752770225589\\
63.125	0.17718	0.20356244906499	0.20356244906499\\
63.125	0.18084	0.232429789956504	0.232429789956504\\
63.125	0.1845	0.263354792900131	0.263354792900131\\
63.125	0.18816	0.296337457895871	0.296337457895871\\
63.125	0.19182	0.331377784943722	0.331377784943722\\
63.125	0.19548	0.368475774043685	0.368475774043685\\
63.125	0.19914	0.407631425195762	0.407631425195762\\
63.125	0.2028	0.44884473839995	0.44884473839995\\
63.125	0.20646	0.492115713656253	0.492115713656253\\
63.125	0.21012	0.537444350964666	0.537444350964666\\
63.125	0.21378	0.584830650325192	0.584830650325192\\
63.125	0.21744	0.63427461173783	0.63427461173783\\
63.125	0.2211	0.685776235202581	0.685776235202581\\
63.125	0.22476	0.739335520719444	0.739335520719444\\
63.125	0.22842	0.79495246828842	0.79495246828842\\
63.125	0.23208	0.852627077909509	0.852627077909509\\
63.125	0.23574	0.91235934958271	0.91235934958271\\
63.125	0.2394	0.974149283308023	0.974149283308023\\
63.125	0.24306	1.03799687908545	1.03799687908545\\
63.125	0.24672	1.10390213691499	1.10390213691499\\
63.125	0.25038	1.17186505679664	1.17186505679664\\
63.125	0.25404	1.2418856387304	1.2418856387304\\
63.125	0.2577	1.31396388271628	1.31396388271628\\
63.125	0.26136	1.38809978875426	1.38809978875426\\
63.125	0.26502	1.46429335684436	1.46429335684436\\
63.125	0.26868	1.54254458698658	1.54254458698658\\
63.125	0.27234	1.6228534791809	1.6228534791809\\
63.125	0.276	1.70522003342734	1.70522003342734\\
63.5	0.093	0.117934099035737	0.117934099035737\\
63.5	0.09666	0.0989418951296597	0.0989418951296597\\
63.5	0.10032	0.0820073532756964	0.0820073532756964\\
63.5	0.10398	0.0671304734738436	0.0671304734738436\\
63.5	0.10764	0.0543112557241042	0.0543112557241042\\
63.5	0.1113	0.0435497000264773	0.0435497000264773\\
63.5	0.11496	0.034845806380962	0.034845806380962\\
63.5	0.11862	0.0281995747875605	0.0281995747875605\\
63.5	0.12228	0.0236110052462701	0.0236110052462701\\
63.5	0.12594	0.0210800977570929	0.0210800977570929\\
63.5	0.1296	0.0206068523200278	0.0206068523200278\\
63.5	0.13326	0.0221912689350761	0.0221912689350761\\
63.5	0.13692	0.0258333476022352	0.0258333476022352\\
63.5	0.14058	0.0315330883215075	0.0315330883215075\\
63.5	0.14424	0.0392904910928924	0.0392904910928924\\
63.5	0.1479	0.0491055559163898	0.0491055559163898\\
63.5	0.15156	0.0609782827920002	0.0609782827920002\\
63.5	0.15522	0.0749086717197218	0.0749086717197218\\
63.5	0.15888	0.0908967226995561	0.0908967226995561\\
63.5	0.16254	0.108942435731503	0.108942435731503\\
63.5	0.1662	0.129045810815563	0.129045810815563\\
63.5	0.16986	0.151206847951735	0.151206847951735\\
63.5	0.17352	0.175425547140019	0.175425547140019\\
63.5	0.17718	0.201701908380416	0.201701908380416\\
63.5	0.18084	0.230035931672925	0.230035931672925\\
63.5	0.1845	0.260427617017547	0.260427617017547\\
63.5	0.18816	0.292876964414281	0.292876964414281\\
63.5	0.19182	0.327383973863127	0.327383973863127\\
63.5	0.19548	0.363948645364086	0.363948645364086\\
63.5	0.19914	0.402570978917157	0.402570978917157\\
63.5	0.2028	0.443250974522341	0.443250974522341\\
63.5	0.20646	0.485988632179638	0.485988632179638\\
63.5	0.21012	0.530783951889046	0.530783951889046\\
63.5	0.21378	0.577636933650567	0.577636933650567\\
63.5	0.21744	0.626547577464201	0.626547577464201\\
63.5	0.2211	0.677515883329947	0.677515883329947\\
63.5	0.22476	0.730541851247805	0.730541851247805\\
63.5	0.22842	0.785625481217775	0.785625481217775\\
63.5	0.23208	0.84276677323986	0.84276677323986\\
63.5	0.23574	0.901965727314056	0.901965727314056\\
63.5	0.2394	0.963222343440363	0.963222343440363\\
63.5	0.24306	1.02653662161878	1.02653662161878\\
63.5	0.24672	1.09190856184932	1.09190856184932\\
63.5	0.25038	1.15933816413196	1.15933816413196\\
63.5	0.25404	1.22882542846672	1.22882542846672\\
63.5	0.2577	1.30037035485359	1.30037035485359\\
63.5	0.26136	1.37397294329257	1.37397294329257\\
63.5	0.26502	1.44963319378367	1.44963319378367\\
63.5	0.26868	1.52735110632688	1.52735110632688\\
63.5	0.27234	1.6071266809222	1.6071266809222\\
63.5	0.276	1.68895991756963	1.68895991756963\\
63.875	0.093	0.128646482345601	0.128646482345601\\
63.875	0.09666	0.109120960840519	0.109120960840519\\
63.875	0.10032	0.0916531013875506	0.0916531013875506\\
63.875	0.10398	0.076242903986693	0.076242903986693\\
63.875	0.10764	0.0628903686379479	0.0628903686379479\\
63.875	0.1113	0.0515954953413162	0.0515954953413162\\
63.875	0.11496	0.0423582840967971	0.0423582840967971\\
63.875	0.11862	0.0351787349043891	0.0351787349043891\\
63.875	0.12228	0.0300568477640948	0.0300568477640948\\
63.875	0.12594	0.0269926226759118	0.0269926226759118\\
63.875	0.1296	0.025986059639842	0.025986059639842\\
63.875	0.13326	0.0270371586558846	0.0270371586558846\\
63.875	0.13692	0.0301459197240399	0.0301459197240399\\
63.875	0.14058	0.0353123428443074	0.0353123428443074\\
63.875	0.14424	0.0425364280166867	0.0425364280166867\\
63.875	0.1479	0.0518181752411793	0.0518181752411793\\
63.875	0.15156	0.0631575845177841	0.0631575845177841\\
63.875	0.15522	0.0765546558465018	0.0765546558465018\\
63.875	0.15888	0.0920093892273304	0.0920093892273304\\
63.875	0.16254	0.109521784660273	0.109521784660273\\
63.875	0.1662	0.129091842145328	0.129091842145328\\
63.875	0.16986	0.150719561682495	0.150719561682495\\
63.875	0.17352	0.174404943271774	0.174404943271774\\
63.875	0.17718	0.200147986913165	0.200147986913165\\
63.875	0.18084	0.227948692606669	0.227948692606669\\
63.875	0.1845	0.257807060352286	0.257807060352286\\
63.875	0.18816	0.289723090150016	0.289723090150016\\
63.875	0.19182	0.323696781999856	0.323696781999856\\
63.875	0.19548	0.35972813590181	0.35972813590181\\
63.875	0.19914	0.397817151855877	0.397817151855877\\
63.875	0.2028	0.437963829862055	0.437963829862055\\
63.875	0.20646	0.480168169920347	0.480168169920347\\
63.875	0.21012	0.52443017203075	0.52443017203075\\
63.875	0.21378	0.570749836193266	0.570749836193266\\
63.875	0.21744	0.619127162407895	0.619127162407895\\
63.875	0.2211	0.669562150674636	0.669562150674636\\
63.875	0.22476	0.72205480099349	0.72205480099349\\
63.875	0.22842	0.776605113364456	0.776605113364456\\
63.875	0.23208	0.833213087787534	0.833213087787534\\
63.875	0.23574	0.891878724262726	0.891878724262726\\
63.875	0.2394	0.952602022790028	0.952602022790028\\
63.875	0.24306	1.01538298336944	1.01538298336944\\
63.875	0.24672	1.08022160600097	1.08022160600097\\
63.875	0.25038	1.14711789068461	1.14711789068461\\
63.875	0.25404	1.21607183742037	1.21607183742037\\
63.875	0.2577	1.28708344620823	1.28708344620823\\
63.875	0.26136	1.36015271704821	1.36015271704821\\
63.875	0.26502	1.4352796499403	1.4352796499403\\
63.875	0.26868	1.5124642448845	1.5124642448845\\
63.875	0.27234	1.59170650188082	1.59170650188082\\
63.875	0.276	1.67300642092924	1.67300642092924\\
64.25	0.093	0.13966548487279	0.13966548487279\\
64.25	0.09666	0.119606645768703	0.119606645768703\\
64.25	0.10032	0.101605468716728	0.101605468716728\\
64.25	0.10398	0.0856619537168661	0.0856619537168661\\
64.25	0.10764	0.0717761007691172	0.0717761007691172\\
64.25	0.1113	0.0599479098734799	0.0599479098734799\\
64.25	0.11496	0.0501773810299559	0.0501773810299559\\
64.25	0.11862	0.0424645142385431	0.0424645142385431\\
64.25	0.12228	0.0368093094992432	0.0368093094992432\\
64.25	0.12594	0.0332117668120564	0.0332117668120564\\
64.25	0.1296	0.0316718861769818	0.0316718861769818\\
64.25	0.13326	0.0321896675940188	0.0321896675940188\\
64.25	0.13692	0.0347651110631684	0.0347651110631684\\
64.25	0.14058	0.0393982165844311	0.0393982165844311\\
64.25	0.14424	0.0460889841578056	0.0460889841578056\\
64.25	0.1479	0.0548374137832934	0.0548374137832934\\
64.25	0.15156	0.0656435054608935	0.0656435054608935\\
64.25	0.15522	0.0785072591906055	0.0785072591906055\\
64.25	0.15888	0.0934286749724302	0.0934286749724302\\
64.25	0.16254	0.110407752806367	0.110407752806367\\
64.25	0.1662	0.129444492692417	0.129444492692417\\
64.25	0.16986	0.150538894630579	0.150538894630579\\
64.25	0.17352	0.173690958620853	0.173690958620853\\
64.25	0.17718	0.198900684663239	0.198900684663239\\
64.25	0.18084	0.226168072757738	0.226168072757738\\
64.25	0.1845	0.255493122904351	0.255493122904351\\
64.25	0.18816	0.286875835103075	0.286875835103075\\
64.25	0.19182	0.32031620935391	0.32031620935391\\
64.25	0.19548	0.355814245656859	0.355814245656859\\
64.25	0.19914	0.393369944011921	0.393369944011921\\
64.25	0.2028	0.432983304419095	0.432983304419095\\
64.25	0.20646	0.474654326878382	0.474654326878382\\
64.25	0.21012	0.51838301138978	0.51838301138978\\
64.25	0.21378	0.564169357953291	0.564169357953291\\
64.25	0.21744	0.612013366568914	0.612013366568914\\
64.25	0.2211	0.661915037236651	0.661915037236651\\
64.25	0.22476	0.713874369956499	0.713874369956499\\
64.25	0.22842	0.76789136472846	0.76789136472846\\
64.25	0.23208	0.823966021552534	0.823966021552534\\
64.25	0.23574	0.88209834042872	0.88209834042872\\
64.25	0.2394	0.942288321357017	0.942288321357017\\
64.25	0.24306	1.00453596433743	1.00453596433743\\
64.25	0.24672	1.06884126936995	1.06884126936995\\
64.25	0.25038	1.13520423645459	1.13520423645459\\
64.25	0.25404	1.20362486559133	1.20362486559133\\
64.25	0.2577	1.2741031567802	1.2741031567802\\
64.25	0.26136	1.34663911002117	1.34663911002117\\
64.25	0.26502	1.42123272531425	1.42123272531425\\
64.25	0.26868	1.49788400265945	1.49788400265945\\
64.25	0.27234	1.57659294205676	1.57659294205676\\
64.25	0.276	1.65735954350618	1.65735954350618\\
64.625	0.093	0.150991106617304	0.150991106617304\\
64.625	0.09666	0.130398949914213	0.130398949914213\\
64.625	0.10032	0.111864455263233	0.111864455263233\\
64.625	0.10398	0.0953876226643666	0.0953876226643666\\
64.625	0.10764	0.0809684521176112	0.0809684521176112\\
64.625	0.1113	0.0686069436229699	0.0686069436229699\\
64.625	0.11496	0.0583030971804404	0.0583030971804404\\
64.625	0.11862	0.0500569127900228	0.0500569127900228\\
64.625	0.12228	0.0438683904517181	0.0438683904517181\\
64.625	0.12594	0.0397375301655256	0.0397375301655256\\
64.625	0.1296	0.0376643319314462	0.0376643319314462\\
64.625	0.13326	0.0376487957494784	0.0376487957494784\\
64.625	0.13692	0.0396909216196233	0.0396909216196233\\
64.625	0.14058	0.0437907095418804	0.0437907095418804\\
64.625	0.14424	0.049948159516251	0.049948159516251\\
64.625	0.1479	0.0581632715427332	0.0581632715427332\\
64.625	0.15156	0.0684360456213275	0.0684360456213275\\
64.625	0.15522	0.0807664817520348	0.0807664817520348\\
64.625	0.15888	0.0951545799348548	0.0951545799348548\\
64.625	0.16254	0.111600340169787	0.111600340169787\\
64.625	0.1662	0.130103762456831	0.130103762456831\\
64.625	0.16986	0.150664846795988	0.150664846795988\\
64.625	0.17352	0.173283593187257	0.173283593187257\\
64.625	0.17718	0.197960001630638	0.197960001630638\\
64.625	0.18084	0.224694072126133	0.224694072126133\\
64.625	0.1845	0.25348580467374	0.25348580467374\\
64.625	0.18816	0.28433519927346	0.28433519927346\\
64.625	0.19182	0.317242255925291	0.317242255925291\\
64.625	0.19548	0.352206974629234	0.352206974629234\\
64.625	0.19914	0.389229355385291	0.389229355385291\\
64.625	0.2028	0.42830939819346	0.42830939819346\\
64.625	0.20646	0.469447103053742	0.469447103053742\\
64.625	0.21012	0.512642469966134	0.512642469966134\\
64.625	0.21378	0.557895498930641	0.557895498930641\\
64.625	0.21744	0.605206189947259	0.605206189947259\\
64.625	0.2211	0.654574543015991	0.654574543015991\\
64.625	0.22476	0.706000558136834	0.706000558136834\\
64.625	0.22842	0.75948423530979	0.75948423530979\\
64.625	0.23208	0.815025574534858	0.815025574534858\\
64.625	0.23574	0.87262457581204	0.87262457581204\\
64.625	0.2394	0.932281239141332	0.932281239141332\\
64.625	0.24306	0.993995564522737	0.993995564522737\\
64.625	0.24672	1.05776755195626	1.05776755195626\\
64.625	0.25038	1.12359720144189	1.12359720144189\\
64.625	0.25404	1.19148451297963	1.19148451297963\\
64.625	0.2577	1.26142948656948	1.26142948656948\\
64.625	0.26136	1.33343212221145	1.33343212221145\\
64.625	0.26502	1.40749241990553	1.40749241990553\\
64.625	0.26868	1.48361037965173	1.48361037965173\\
64.625	0.27234	1.56178600145003	1.56178600145003\\
64.625	0.276	1.64201928530045	1.64201928530045\\
65	0.093	0.162623347579144	0.162623347579144\\
65	0.09666	0.141497873277047	0.141497873277047\\
65	0.10032	0.122430061027063	0.122430061027063\\
65	0.10398	0.105419910829191	0.105419910829191\\
65	0.10764	0.0904674226834316	0.0904674226834316\\
65	0.1113	0.0775725965897838	0.0775725965897838\\
65	0.11496	0.0667354325482503	0.0667354325482503\\
65	0.11862	0.0579559305588271	0.0579559305588271\\
65	0.12228	0.0512340906215176	0.0512340906215176\\
65	0.12594	0.0465699127363204	0.0465699127363204\\
65	0.1296	0.0439633969032354	0.0439633969032354\\
65	0.13326	0.0434145431222637	0.0434145431222637\\
65	0.13692	0.0449233513934029	0.0449233513934029\\
65	0.14058	0.0484898217166552	0.0484898217166552\\
65	0.14424	0.0541139540920201	0.0541139540920201\\
65	0.1479	0.0617957485194975	0.0617957485194975\\
65	0.15156	0.0715352049990872	0.0715352049990872\\
65	0.15522	0.0833323235307897	0.0833323235307897\\
65	0.15888	0.0971871041146048	0.0971871041146048\\
65	0.16254	0.113099546750531	0.113099546750531\\
65	0.1662	0.131069651438571	0.131069651438571\\
65	0.16986	0.151097418178724	0.151097418178724\\
65	0.17352	0.173182846970988	0.173182846970988\\
65	0.17718	0.197325937815364	0.197325937815364\\
65	0.18084	0.223526690711853	0.223526690711853\\
65	0.1845	0.251785105660455	0.251785105660455\\
65	0.18816	0.28210118266117	0.28210118266117\\
65	0.19182	0.314474921713995	0.314474921713995\\
65	0.19548	0.348906322818934	0.348906322818934\\
65	0.19914	0.385395385975986	0.385395385975986\\
65	0.2028	0.42394211118515	0.42394211118515\\
65	0.20646	0.464546498446427	0.464546498446427\\
65	0.21012	0.507208547759816	0.507208547759816\\
65	0.21378	0.551928259125316	0.551928259125316\\
65	0.21744	0.598705632542929	0.598705632542929\\
65	0.2211	0.647540668012655	0.647540668012655\\
65	0.22476	0.698433365534495	0.698433365534495\\
65	0.22842	0.751383725108445	0.751383725108445\\
65	0.23208	0.806391746734509	0.806391746734509\\
65	0.23574	0.863457430412685	0.863457430412685\\
65	0.2394	0.922580776142972	0.922580776142972\\
65	0.24306	0.983761783925373	0.983761783925373\\
65	0.24672	1.04700045375989	1.04700045375989\\
65	0.25038	1.11229678564651	1.11229678564651\\
65	0.25404	1.17965077958525	1.17965077958525\\
65	0.2577	1.2490624355761	1.2490624355761\\
65	0.26136	1.32053175361906	1.32053175361906\\
65	0.26502	1.39405873371414	1.39405873371414\\
65	0.26868	1.46964337586133	1.46964337586133\\
65	0.27234	1.54728568006063	1.54728568006063\\
65	0.276	1.62698564631204	1.62698564631204\\
65.375	0.093	0.174562207758307	0.174562207758307\\
65.375	0.09666	0.152903415857206	0.152903415857206\\
65.375	0.10032	0.133302286008216	0.133302286008216\\
65.375	0.10398	0.115758818211339	0.115758818211339\\
65.375	0.10764	0.100273012466575	0.100273012466575\\
65.375	0.1113	0.0868448687739232	0.0868448687739232\\
65.375	0.11496	0.0754743871333832	0.0754743871333832\\
65.375	0.11862	0.0661615675449561	0.0661615675449561\\
65.375	0.12228	0.0589064100086418	0.0589064100086418\\
65.375	0.12594	0.0537089145244389	0.0537089145244389\\
65.375	0.1296	0.0505690810923491	0.0505690810923491\\
65.375	0.13326	0.0494869097123718	0.0494869097123718\\
65.375	0.13692	0.0504624003845062	0.0504624003845062\\
65.375	0.14058	0.0534955531087538	0.0534955531087538\\
65.375	0.14424	0.058586367885114	0.058586367885114\\
65.375	0.1479	0.0657348447135866	0.0657348447135866\\
65.375	0.15156	0.0749409835941715	0.0749409835941715\\
65.375	0.15522	0.0862047845268692	0.0862047845268692\\
65.375	0.15888	0.0995262475116787	0.0995262475116787\\
65.375	0.16254	0.1149053725486	0.1149053725486\\
65.375	0.1662	0.132342159637635	0.132342159637635\\
65.375	0.16986	0.151836608778782	0.151836608778782\\
65.375	0.17352	0.173388719972041	0.173388719972041\\
65.375	0.17718	0.196998493217413	0.196998493217413\\
65.375	0.18084	0.222665928514897	0.222665928514897\\
65.375	0.1845	0.250391025864495	0.250391025864495\\
65.375	0.18816	0.280173785266204	0.280173785266204\\
65.375	0.19182	0.312014206720025	0.312014206720025\\
65.375	0.19548	0.345912290225959	0.345912290225959\\
65.375	0.19914	0.381868035784005	0.381868035784005\\
65.375	0.2028	0.419881443394164	0.419881443394164\\
65.375	0.20646	0.459952513056436	0.459952513056436\\
65.375	0.21012	0.502081244770819	0.502081244770819\\
65.375	0.21378	0.546267638537314	0.546267638537314\\
65.375	0.21744	0.592511694355923	0.592511694355923\\
65.375	0.2211	0.640813412226644	0.640813412226644\\
65.375	0.22476	0.691172792149477	0.691172792149477\\
65.375	0.22842	0.743589834124424	0.743589834124424\\
65.375	0.23208	0.798064538151483	0.798064538151483\\
65.375	0.23574	0.854596904230654	0.854596904230654\\
65.375	0.2394	0.913186932361937	0.913186932361937\\
65.375	0.24306	0.973834622545333	0.973834622545333\\
65.375	0.24672	1.03653997478084	1.03653997478084\\
65.375	0.25038	1.10130298906846	1.10130298906846\\
65.375	0.25404	1.16812366540819	1.16812366540819\\
65.375	0.2577	1.23700200380004	1.23700200380004\\
65.375	0.26136	1.307938004244	1.307938004244\\
65.375	0.26502	1.38093166674007	1.38093166674007\\
65.375	0.26868	1.45598299128825	1.45598299128825\\
65.375	0.27234	1.53309197788855	1.53309197788855\\
65.375	0.276	1.61225862654095	1.61225862654095\\
65.75	0.093	0.186807687154797	0.186807687154797\\
65.75	0.09666	0.16461557765469	0.16461557765469\\
65.75	0.10032	0.144481130206696	0.144481130206696\\
65.75	0.10398	0.126404344810814	0.126404344810814\\
65.75	0.10764	0.110385221467044	0.110385221467044\\
65.75	0.1113	0.0964237601753872	0.0964237601753872\\
65.75	0.11496	0.0845199609358434	0.0845199609358434\\
65.75	0.11862	0.0746738237484097	0.0746738237484097\\
65.75	0.12228	0.0668853486130907	0.0668853486130907\\
65.75	0.12594	0.0611545355298839	0.0611545355298839\\
65.75	0.1296	0.0574813844987885	0.0574813844987885\\
65.75	0.13326	0.0558658955198064	0.0558658955198064\\
65.75	0.13692	0.056308068592936	0.056308068592936\\
65.75	0.14058	0.0588079037181788	0.0588079037181788\\
65.75	0.14424	0.0633654008955333	0.0633654008955333\\
65.75	0.1479	0.0699805601250012	0.0699805601250012\\
65.75	0.15156	0.0786533814065813	0.0786533814065813\\
65.75	0.15522	0.0893838647402725	0.0893838647402725\\
65.75	0.15888	0.102172010126078	0.102172010126078\\
65.75	0.16254	0.117017817563995	0.117017817563995\\
65.75	0.1662	0.133921287054024	0.133921287054024\\
65.75	0.16986	0.152882418596167	0.152882418596167\\
65.75	0.17352	0.173901212190421	0.173901212190421\\
65.75	0.17718	0.196977667836788	0.196977667836788\\
65.75	0.18084	0.222111785535266	0.222111785535266\\
65.75	0.1845	0.249303565285859	0.249303565285859\\
65.75	0.18816	0.278553007088563	0.278553007088563\\
65.75	0.19182	0.309860110943379	0.309860110943379\\
65.75	0.19548	0.343224876850308	0.343224876850308\\
65.75	0.19914	0.37864730480935	0.37864730480935\\
65.75	0.2028	0.416127394820504	0.416127394820504\\
65.75	0.20646	0.45566514688377	0.45566514688377\\
65.75	0.21012	0.497260560999148	0.497260560999148\\
65.75	0.21378	0.54091363716664	0.54091363716664\\
65.75	0.21744	0.586624375386244	0.586624375386244\\
65.75	0.2211	0.634392775657959	0.634392775657959\\
65.75	0.22476	0.684218837981788	0.684218837981788\\
65.75	0.22842	0.736102562357729	0.736102562357729\\
65.75	0.23208	0.790043948785783	0.790043948785783\\
65.75	0.23574	0.846042997265949	0.846042997265949\\
65.75	0.2394	0.904099707798227	0.904099707798227\\
65.75	0.24306	0.964214080382618	0.964214080382618\\
65.75	0.24672	1.02638611501912	1.02638611501912\\
65.75	0.25038	1.09061581170774	1.09061581170774\\
65.75	0.25404	1.15690317044846	1.15690317044846\\
65.75	0.2577	1.22524819124131	1.22524819124131\\
65.75	0.26136	1.29565087408626	1.29565087408626\\
65.75	0.26502	1.36811121898332	1.36811121898332\\
65.75	0.26868	1.4426292259325	1.4426292259325\\
65.75	0.27234	1.51920489493379	1.51920489493379\\
65.75	0.276	1.59783822598719	1.59783822598719\\
66.125	0.093	0.199359785768608	0.199359785768608\\
66.125	0.09666	0.176634358669497	0.176634358669497\\
66.125	0.10032	0.155966593622498	0.155966593622498\\
66.125	0.10398	0.137356490627611	0.137356490627611\\
66.125	0.10764	0.120804049684836	0.120804049684836\\
66.125	0.1113	0.106309270794174	0.106309270794174\\
66.125	0.11496	0.0938721539556255	0.0938721539556255\\
66.125	0.11862	0.083492699169188	0.083492699169188\\
66.125	0.12228	0.0751709064348633	0.0751709064348633\\
66.125	0.12594	0.0689067757526509	0.0689067757526509\\
66.125	0.1296	0.0647003071225507	0.0647003071225507\\
66.125	0.13326	0.0625515005445629	0.0625515005445629\\
66.125	0.13692	0.0624603560186878	0.0624603560186878\\
66.125	0.14058	0.0644268735449258	0.0644268735449258\\
66.125	0.14424	0.0684510531232756	0.0684510531232756\\
66.125	0.1479	0.0745328947537378	0.0745328947537378\\
66.125	0.15156	0.0826723984363131	0.0826723984363131\\
66.125	0.15522	0.0928695641710005	0.0928695641710005\\
66.125	0.15888	0.1051243919578	0.1051243919578\\
66.125	0.16254	0.119436881796712	0.119436881796712\\
66.125	0.1662	0.135807033687737	0.135807033687737\\
66.125	0.16986	0.154234847630875	0.154234847630875\\
66.125	0.17352	0.174720323626123	0.174720323626123\\
66.125	0.17718	0.197263461673484	0.197263461673484\\
66.125	0.18084	0.221864261772959	0.221864261772959\\
66.125	0.1845	0.248522723924546	0.248522723924546\\
66.125	0.18816	0.277238848128246	0.277238848128246\\
66.125	0.19182	0.308012634384057	0.308012634384057\\
66.125	0.19548	0.340844082691981	0.340844082691981\\
66.125	0.19914	0.375733193052017	0.375733193052017\\
66.125	0.2028	0.412679965464166	0.412679965464166\\
66.125	0.20646	0.451684399928428	0.451684399928428\\
66.125	0.21012	0.492746496444801	0.492746496444801\\
66.125	0.21378	0.535866255013287	0.535866255013287\\
66.125	0.21744	0.581043675633886	0.581043675633886\\
66.125	0.2211	0.628278758306597	0.628278758306597\\
66.125	0.22476	0.677571503031421	0.677571503031421\\
66.125	0.22842	0.728921909808357	0.728921909808357\\
66.125	0.23208	0.782329978637406	0.782329978637406\\
66.125	0.23574	0.837795709518567	0.837795709518567\\
66.125	0.2394	0.89531910245184	0.89531910245184\\
66.125	0.24306	0.954900157437225	0.954900157437225\\
66.125	0.24672	1.01653887447472	1.01653887447472\\
66.125	0.25038	1.08023525356433	1.08023525356433\\
66.125	0.25404	1.14598929470606	1.14598929470606\\
66.125	0.2577	1.21380099789989	1.21380099789989\\
66.125	0.26136	1.28367036314584	1.28367036314584\\
66.125	0.26502	1.3555973904439	1.3555973904439\\
66.125	0.26868	1.42958207979407	1.42958207979407\\
66.125	0.27234	1.50562443119636	1.50562443119636\\
66.125	0.276	1.58372444465076	1.58372444465076\\
66.5	0.093	0.212218503599749	0.212218503599749\\
66.5	0.09666	0.188959758901631	0.188959758901631\\
66.5	0.10032	0.167758676255628	0.167758676255628\\
66.5	0.10398	0.148615255661736	0.148615255661736\\
66.5	0.10764	0.131529497119956	0.131529497119956\\
66.5	0.1113	0.116501400630289	0.116501400630289\\
66.5	0.11496	0.103530966192735	0.103530966192735\\
66.5	0.11862	0.0926181938072927	0.0926181938072927\\
66.5	0.12228	0.0837630834739633	0.0837630834739633\\
66.5	0.12594	0.0769656351927461	0.0769656351927461\\
66.5	0.1296	0.0722258489636411	0.0722258489636411\\
66.5	0.13326	0.0695437247866486	0.0695437247866486\\
66.5	0.13692	0.0689192626617678	0.0689192626617678\\
66.5	0.14058	0.0703524625890011	0.0703524625890011\\
66.5	0.14424	0.0738433245683461	0.0738433245683461\\
66.5	0.1479	0.0793918485998035	0.0793918485998035\\
66.5	0.15156	0.0869980346833732	0.0869980346833732\\
66.5	0.15522	0.0966618828190549	0.0966618828190549\\
66.5	0.15888	0.108383393006851	0.108383393006851\\
66.5	0.16254	0.122162565246757	0.122162565246757\\
66.5	0.1662	0.137999399538777	0.137999399538777\\
66.5	0.16986	0.15589389588291	0.15589389588291\\
66.5	0.17352	0.175846054279154	0.175846054279154\\
66.5	0.17718	0.19785587472751	0.19785587472751\\
66.5	0.18084	0.221923357227979	0.221923357227979\\
66.5	0.1845	0.248048501780561	0.248048501780561\\
66.5	0.18816	0.276231308385256	0.276231308385256\\
66.5	0.19182	0.306471777042062	0.306471777042062\\
66.5	0.19548	0.338769907750981	0.338769907750981\\
66.5	0.19914	0.373125700512012	0.373125700512012\\
66.5	0.2028	0.409539155325157	0.409539155325157\\
66.5	0.20646	0.448010272190414	0.448010272190414\\
66.5	0.21012	0.488539051107782	0.488539051107782\\
66.5	0.21378	0.531125492077262	0.531125492077262\\
66.5	0.21744	0.575769595098857	0.575769595098857\\
66.5	0.2211	0.622471360172563	0.622471360172563\\
66.5	0.22476	0.671230787298382	0.671230787298382\\
66.5	0.22842	0.722047876476312	0.722047876476312\\
66.5	0.23208	0.774922627706355	0.774922627706355\\
66.5	0.23574	0.829855040988512	0.829855040988512\\
66.5	0.2394	0.886845116322781	0.886845116322781\\
66.5	0.24306	0.945892853709161	0.945892853709161\\
66.5	0.24672	1.00699825314765	1.00699825314765\\
66.5	0.25038	1.07016131463826	1.07016131463826\\
66.5	0.25404	1.13538203818098	1.13538203818098\\
66.5	0.2577	1.20266042377581	1.20266042377581\\
66.5	0.26136	1.27199647142275	1.27199647142275\\
66.5	0.26502	1.34339018112181	1.34339018112181\\
66.5	0.26868	1.41684155287297	1.41684155287297\\
66.5	0.27234	1.49235058667625	1.49235058667625\\
66.5	0.276	1.56991728253165	1.56991728253165\\
66.875	0.093	0.225383840648211	0.225383840648211\\
66.875	0.09666	0.201591778351089	0.201591778351089\\
66.875	0.10032	0.17985737810608	0.17985737810608\\
66.875	0.10398	0.160180639913183	0.160180639913183\\
66.875	0.10764	0.142561563772399	0.142561563772399\\
66.875	0.1113	0.127000149683727	0.127000149683727\\
66.875	0.11496	0.113496397647167	0.113496397647167\\
66.875	0.11862	0.102050307662719	0.102050307662719\\
66.875	0.12228	0.0926618797303853	0.0926618797303853\\
66.875	0.12594	0.0853311138501633	0.0853311138501633\\
66.875	0.1296	0.0800580100220536	0.0800580100220536\\
66.875	0.13326	0.0768425682460563	0.0768425682460563\\
66.875	0.13692	0.0756847885221716	0.0756847885221716\\
66.875	0.14058	0.0765846708503983	0.0765846708503983\\
66.875	0.14424	0.0795422152307386	0.0795422152307386\\
66.875	0.1479	0.0845574216631912	0.0845574216631912\\
66.875	0.15156	0.0916302901477561	0.0916302901477561\\
66.875	0.15522	0.100760820684434	0.100760820684434\\
66.875	0.15888	0.111949013273223	0.111949013273223\\
66.875	0.16254	0.125194867914125	0.125194867914125\\
66.875	0.1662	0.14049838460714	0.14049838460714\\
66.875	0.16986	0.157859563352267	0.157859563352267\\
66.875	0.17352	0.177278404149507	0.177278404149507\\
66.875	0.17718	0.198754906998858	0.198754906998858\\
66.875	0.18084	0.222289071900322	0.222289071900322\\
66.875	0.1845	0.2478808988539	0.2478808988539\\
66.875	0.18816	0.275530387859589	0.275530387859589\\
66.875	0.19182	0.30523753891739	0.30523753891739\\
66.875	0.19548	0.337002352027304	0.337002352027304\\
66.875	0.19914	0.370824827189331	0.370824827189331\\
66.875	0.2028	0.406704964403469	0.406704964403469\\
66.875	0.20646	0.444642763669722	0.444642763669722\\
66.875	0.21012	0.484638224988085	0.484638224988085\\
66.875	0.21378	0.526691348358561	0.526691348358561\\
66.875	0.21744	0.57080213378115	0.57080213378115\\
66.875	0.2211	0.61697058125585	0.61697058125585\\
66.875	0.22476	0.665196690782664	0.665196690782664\\
66.875	0.22842	0.71548046236159	0.71548046236159\\
66.875	0.23208	0.767821895992629	0.767821895992629\\
66.875	0.23574	0.822220991675781	0.822220991675781\\
66.875	0.2394	0.878677749411042	0.878677749411042\\
66.875	0.24306	0.937192169198418	0.937192169198418\\
66.875	0.24672	0.997764251037907	0.997764251037907\\
66.875	0.25038	1.06039399492951	1.06039399492951\\
66.875	0.25404	1.12508140087322	1.12508140087322\\
66.875	0.2577	1.19182646886905	1.19182646886905\\
66.875	0.26136	1.26062919891698	1.26062919891698\\
66.875	0.26502	1.33148959101703	1.33148959101703\\
66.875	0.26868	1.4044076451692	1.4044076451692\\
66.875	0.27234	1.47938336137347	1.47938336137347\\
66.875	0.276	1.55641673962986	1.55641673962986\\
67.25	0.093	0.238855796913999	0.238855796913999\\
67.25	0.09666	0.214530417017872	0.214530417017872\\
67.25	0.10032	0.192262699173858	0.192262699173858\\
67.25	0.10398	0.172052643381956	0.172052643381956\\
67.25	0.10764	0.153900249642167	0.153900249642167\\
67.25	0.1113	0.13780551795449	0.13780551795449\\
67.25	0.11496	0.123768448318925	0.123768448318925\\
67.25	0.11862	0.111789040735473	0.111789040735473\\
67.25	0.12228	0.101867295204134	0.101867295204134\\
67.25	0.12594	0.0940032117249061	0.0940032117249061\\
67.25	0.1296	0.0881967902977916	0.0881967902977916\\
67.25	0.13326	0.0844480309227886	0.0844480309227886\\
67.25	0.13692	0.0827569335998992	0.0827569335998992\\
67.25	0.14058	0.083123498329122	0.083123498329122\\
67.25	0.14424	0.0855477251104566	0.0855477251104566\\
67.25	0.1479	0.0900296139439045	0.0900296139439045\\
67.25	0.15156	0.0965691648294638	0.0965691648294638\\
67.25	0.15522	0.105166377767136	0.105166377767136\\
67.25	0.15888	0.115821252756921	0.115821252756921\\
67.25	0.16254	0.128533789798818	0.128533789798818\\
67.25	0.1662	0.143303988892828	0.143303988892828\\
67.25	0.16986	0.16013185003895	0.16013185003895\\
67.25	0.17352	0.179017373237184	0.179017373237184\\
67.25	0.17718	0.199960558487531	0.199960558487531\\
67.25	0.18084	0.22296140578999	0.22296140578999\\
67.25	0.1845	0.248019915144563	0.248019915144563\\
67.25	0.18816	0.275136086551247	0.275136086551247\\
67.25	0.19182	0.304309920010043	0.304309920010043\\
67.25	0.19548	0.335541415520952	0.335541415520952\\
67.25	0.19914	0.368830573083974	0.368830573083974\\
67.25	0.2028	0.404177392699108	0.404177392699108\\
67.25	0.20646	0.441581874366355	0.441581874366355\\
67.25	0.21012	0.481044018085713	0.481044018085713\\
67.25	0.21378	0.522563823857184	0.522563823857184\\
67.25	0.21744	0.566141291680768	0.566141291680768\\
67.25	0.2211	0.611776421556464	0.611776421556464\\
67.25	0.22476	0.659469213484273	0.659469213484273\\
67.25	0.22842	0.709219667464193	0.709219667464193\\
67.25	0.23208	0.761027783496228	0.761027783496228\\
67.25	0.23574	0.814893561580375	0.814893561580375\\
67.25	0.2394	0.870817001716632	0.870817001716632\\
67.25	0.24306	0.928798103905002	0.928798103905002\\
67.25	0.24672	0.988836868145485	0.988836868145485\\
67.25	0.25038	1.05093329443808	1.05093329443808\\
67.25	0.25404	1.11508738278279	1.11508738278279\\
67.25	0.2577	1.18129913317961	1.18129913317961\\
67.25	0.26136	1.24956854562854	1.24956854562854\\
67.25	0.26502	1.31989562012959	1.31989562012959\\
67.25	0.26868	1.39228035668275	1.39228035668275\\
67.25	0.27234	1.46672275528802	1.46672275528802\\
67.25	0.276	1.5432228159454	1.5432228159454\\
67.625	0.093	0.252634372397111	0.252634372397111\\
67.625	0.09666	0.227775674901979	0.227775674901979\\
67.625	0.10032	0.204974639458961	0.204974639458961\\
67.625	0.10398	0.184231266068053	0.184231266068053\\
67.625	0.10764	0.16554555472926	0.16554555472926\\
67.625	0.1113	0.148917505442577	0.148917505442577\\
67.625	0.11496	0.134347118208008	0.134347118208008\\
67.625	0.11862	0.121834393025551	0.121834393025551\\
67.625	0.12228	0.111379329895206	0.111379329895206\\
67.625	0.12594	0.102981928816974	0.102981928816974\\
67.625	0.1296	0.0966421897908543	0.0966421897908543\\
67.625	0.13326	0.0923601128168465	0.0923601128168465\\
67.625	0.13692	0.0901356978949523	0.0901356978949523\\
67.625	0.14058	0.0899689450251695	0.0899689450251695\\
67.625	0.14424	0.0918598542074993	0.0918598542074993\\
67.625	0.1479	0.0958084254419416	0.0958084254419416\\
67.625	0.15156	0.101814658728497	0.101814658728497\\
67.625	0.15522	0.109878554067163	0.109878554067163\\
67.625	0.15888	0.120000111457944	0.120000111457944\\
67.625	0.16254	0.132179330900836	0.132179330900836\\
67.625	0.1662	0.146416212395841	0.146416212395841\\
67.625	0.16986	0.162710755942959	0.162710755942959\\
67.625	0.17352	0.181062961542187	0.181062961542187\\
67.625	0.17718	0.201472829193528	0.201472829193528\\
67.625	0.18084	0.223940358896983	0.223940358896983\\
67.625	0.1845	0.24846555065255	0.24846555065255\\
67.625	0.18816	0.27504840446023	0.27504840446023\\
67.625	0.19182	0.303688920320021	0.303688920320021\\
67.625	0.19548	0.334387098231925	0.334387098231925\\
67.625	0.19914	0.367142938195941	0.367142938195941\\
67.625	0.2028	0.401956440212071	0.401956440212071\\
67.625	0.20646	0.438827604280313	0.438827604280313\\
67.625	0.21012	0.477756430400667	0.477756430400667\\
67.625	0.21378	0.518742918573132	0.518742918573132\\
67.625	0.21744	0.561787068797711	0.561787068797711\\
67.625	0.2211	0.606888881074401	0.606888881074401\\
67.625	0.22476	0.654048355403205	0.654048355403205\\
67.625	0.22842	0.703265491784122	0.703265491784122\\
67.625	0.23208	0.754540290217151	0.754540290217151\\
67.625	0.23574	0.807872750702292	0.807872750702292\\
67.625	0.2394	0.863262873239544	0.863262873239544\\
67.625	0.24306	0.92071065782891	0.92071065782891\\
67.625	0.24672	0.980216104470389	0.980216104470389\\
67.625	0.25038	1.04177921316398	1.04177921316398\\
67.625	0.25404	1.10539998390968	1.10539998390968\\
67.625	0.2577	1.1710784167075	1.1710784167075\\
67.625	0.26136	1.23881451155743	1.23881451155743\\
67.625	0.26502	1.30860826845947	1.30860826845947\\
67.625	0.26868	1.38045968741362	1.38045968741362\\
67.625	0.27234	1.45436876841988	1.45436876841988\\
67.625	0.276	1.53033551147826	1.53033551147826\\
68	0.093	0.26671956709755	0.26671956709755\\
68	0.09666	0.241327552003413	0.241327552003413\\
68	0.10032	0.217993198961389	0.217993198961389\\
68	0.10398	0.196716507971477	0.196716507971477\\
68	0.10764	0.177497479033677	0.177497479033677\\
68	0.1113	0.16033611214799	0.16033611214799\\
68	0.11496	0.145232407314416	0.145232407314416\\
68	0.11862	0.132186364532954	0.132186364532954\\
68	0.12228	0.121197983803605	0.121197983803605\\
68	0.12594	0.112267265126367	0.112267265126367\\
68	0.1296	0.105394208501242	0.105394208501242\\
68	0.13326	0.100578813928231	0.100578813928231\\
68	0.13692	0.0978210814073301	0.0978210814073301\\
68	0.14058	0.0971210109385425	0.0971210109385425\\
68	0.14424	0.0984786025218676	0.0984786025218676\\
68	0.1479	0.101893856157305	0.101893856157305\\
68	0.15156	0.107366771844856	0.107366771844856\\
68	0.15522	0.114897349584517	0.114897349584517\\
68	0.15888	0.124485589376293	0.124485589376293\\
68	0.16254	0.136131491220179	0.136131491220179\\
68	0.1662	0.14983505511618	0.14983505511618\\
68	0.16986	0.165596281064292	0.165596281064292\\
68	0.17352	0.183415169064516	0.183415169064516\\
68	0.17718	0.203291719116852	0.203291719116852\\
68	0.18084	0.225225931221302	0.225225931221302\\
68	0.1845	0.249217805377864	0.249217805377864\\
68	0.18816	0.275267341586538	0.275267341586538\\
68	0.19182	0.303374539847324	0.303374539847324\\
68	0.19548	0.333539400160224	0.333539400160224\\
68	0.19914	0.365761922525235	0.365761922525235\\
68	0.2028	0.400042106942359	0.400042106942359\\
68	0.20646	0.436379953411596	0.436379953411596\\
68	0.21012	0.474775461932945	0.474775461932945\\
68	0.21378	0.515228632506405	0.515228632506405\\
68	0.21744	0.557739465131979	0.557739465131979\\
68	0.2211	0.602307959809665	0.602307959809665\\
68	0.22476	0.648934116539464	0.648934116539464\\
68	0.22842	0.697617935321375	0.697617935321375\\
68	0.23208	0.748359416155399	0.748359416155399\\
68	0.23574	0.801158559041534	0.801158559041534\\
68	0.2394	0.856015363979783	0.856015363979783\\
68	0.24306	0.912929830970144	0.912929830970144\\
68	0.24672	0.971901960012618	0.971901960012618\\
68	0.25038	1.0329317511072	1.0329317511072\\
68	0.25404	1.0960192042539	1.0960192042539\\
68	0.2577	1.16116431945271	1.16116431945271\\
68	0.26136	1.22836709670363	1.22836709670363\\
68	0.26502	1.29762753600667	1.29762753600667\\
68	0.26868	1.36894563736182	1.36894563736182\\
68	0.27234	1.44232140076908	1.44232140076908\\
68	0.276	1.51775482622845	1.51775482622845\\
68.375	0.093	0.281111381015312	0.281111381015312\\
68.375	0.09666	0.25518604832217	0.25518604832217\\
68.375	0.10032	0.231318377681141	0.231318377681141\\
68.375	0.10398	0.209508369092225	0.209508369092225\\
68.375	0.10764	0.18975602255542	0.18975602255542\\
68.375	0.1113	0.172061338070727	0.172061338070727\\
68.375	0.11496	0.156424315638149	0.156424315638149\\
68.375	0.11862	0.142844955257681	0.142844955257681\\
68.375	0.12228	0.131323256929327	0.131323256929327\\
68.375	0.12594	0.121859220653085	0.121859220653085\\
68.375	0.1296	0.114452846428954	0.114452846428954\\
68.375	0.13326	0.109104134256938	0.109104134256938\\
68.375	0.13692	0.105813084137033	0.105813084137033\\
68.375	0.14058	0.10457969606924	0.10457969606924\\
68.375	0.14424	0.10540397005356	0.10540397005356\\
68.375	0.1479	0.108285906089993	0.108285906089993\\
68.375	0.15156	0.113225504178538	0.113225504178538\\
68.375	0.15522	0.120222764319196	0.120222764319196\\
68.375	0.15888	0.129277686511966	0.129277686511966\\
68.375	0.16254	0.140390270756847	0.140390270756847\\
68.375	0.1662	0.153560517053842	0.153560517053842\\
68.375	0.16986	0.168788425402949	0.168788425402949\\
68.375	0.17352	0.186073995804168	0.186073995804168\\
68.375	0.17718	0.2054172282575	0.2054172282575\\
68.375	0.18084	0.226818122762944	0.226818122762944\\
68.375	0.1845	0.250276679320502	0.250276679320502\\
68.375	0.18816	0.275792897930171	0.275792897930171\\
68.375	0.19182	0.303366778591953	0.303366778591953\\
68.375	0.19548	0.332998321305846	0.332998321305846\\
68.375	0.19914	0.364687526071853	0.364687526071853\\
68.375	0.2028	0.398434392889972	0.398434392889972\\
68.375	0.20646	0.434238921760204	0.434238921760204\\
68.375	0.21012	0.472101112682547	0.472101112682547\\
68.375	0.21378	0.512020965657003	0.512020965657003\\
68.375	0.21744	0.553998480683572	0.553998480683572\\
68.375	0.2211	0.598033657762254	0.598033657762254\\
68.375	0.22476	0.644126496893048	0.644126496893048\\
68.375	0.22842	0.692276998075953	0.692276998075953\\
68.375	0.23208	0.742485161310971	0.742485161310971\\
68.375	0.23574	0.794750986598102	0.794750986598102\\
68.375	0.2394	0.849074473937346	0.849074473937346\\
68.375	0.24306	0.905455623328701	0.905455623328701\\
68.375	0.24672	0.963894434772171	0.963894434772171\\
68.375	0.25038	1.02439090826775	1.02439090826775\\
68.375	0.25404	1.08694504381544	1.08694504381544\\
68.375	0.2577	1.15155684141525	1.15155684141525\\
68.375	0.26136	1.21822630106717	1.21822630106717\\
68.375	0.26502	1.2869534227712	1.2869534227712\\
68.375	0.26868	1.35773820652734	1.35773820652734\\
68.375	0.27234	1.4305806523356	1.4305806523356\\
68.375	0.276	1.50548076019596	1.50548076019596\\
68.75	0.093	0.295809814150399	0.295809814150399\\
68.75	0.09666	0.269351163858253	0.269351163858253\\
68.75	0.10032	0.244950175618219	0.244950175618219\\
68.75	0.10398	0.222606849430296	0.222606849430296\\
68.75	0.10764	0.202321185294488	0.202321185294488\\
68.75	0.1113	0.18409318321079	0.18409318321079\\
68.75	0.11496	0.167922843179206	0.167922843179206\\
68.75	0.11862	0.153810165199734	0.153810165199734\\
68.75	0.12228	0.141755149272375	0.141755149272375\\
68.75	0.12594	0.131757795397127	0.131757795397127\\
68.75	0.1296	0.123818103573992	0.123818103573992\\
68.75	0.13326	0.11793607380297	0.11793607380297\\
68.75	0.13692	0.114111706084061	0.114111706084061\\
68.75	0.14058	0.112345000417263	0.112345000417263\\
68.75	0.14424	0.112635956802578	0.112635956802578\\
68.75	0.1479	0.114984575240006	0.114984575240006\\
68.75	0.15156	0.119390855729545	0.119390855729545\\
68.75	0.15522	0.125854798271197	0.125854798271197\\
68.75	0.15888	0.134376402864963	0.134376402864963\\
68.75	0.16254	0.14495566951084	0.14495566951084\\
68.75	0.1662	0.157592598208829	0.157592598208829\\
68.75	0.16986	0.172287188958932	0.172287188958932\\
68.75	0.17352	0.189039441761147	0.189039441761147\\
68.75	0.17718	0.207849356615473	0.207849356615473\\
68.75	0.18084	0.228716933521912	0.228716933521912\\
68.75	0.1845	0.251642172480464	0.251642172480464\\
68.75	0.18816	0.276625073491129	0.276625073491129\\
68.75	0.19182	0.303665636553905	0.303665636553905\\
68.75	0.19548	0.332763861668794	0.332763861668794\\
68.75	0.19914	0.363919748835796	0.363919748835796\\
68.75	0.2028	0.39713329805491	0.39713329805491\\
68.75	0.20646	0.432404509326137	0.432404509326137\\
68.75	0.21012	0.469733382649476	0.469733382649476\\
68.75	0.21378	0.509119918024926	0.509119918024926\\
68.75	0.21744	0.55056411545249	0.55056411545249\\
68.75	0.2211	0.594065974932167	0.594065974932167\\
68.75	0.22476	0.639625496463956	0.639625496463956\\
68.75	0.22842	0.687242680047856	0.687242680047856\\
68.75	0.23208	0.73691752568387	0.73691752568387\\
68.75	0.23574	0.788650033371995	0.788650033371995\\
68.75	0.2394	0.842440203112234	0.842440203112234\\
68.75	0.24306	0.898288034904585	0.898288034904585\\
68.75	0.24672	0.956193528749049	0.956193528749049\\
68.75	0.25038	1.01615668464562	1.01615668464562\\
68.75	0.25404	1.07817750259431	1.07817750259431\\
68.75	0.2577	1.14225598259511	1.14225598259511\\
68.75	0.26136	1.20839212464803	1.20839212464803\\
68.75	0.26502	1.27658592875305	1.27658592875305\\
68.75	0.26868	1.34683739491019	1.34683739491019\\
68.75	0.27234	1.41914652311944	1.41914652311944\\
68.75	0.276	1.4935133133808	1.4935133133808\\
69.125	0.093	0.310814866502811	0.310814866502811\\
69.125	0.09666	0.28382289861166	0.28382289861166\\
69.125	0.10032	0.258888592772621	0.258888592772621\\
69.125	0.10398	0.236011948985694	0.236011948985694\\
69.125	0.10764	0.215192967250879	0.215192967250879\\
69.125	0.1113	0.196431647568177	0.196431647568177\\
69.125	0.11496	0.179727989937589	0.179727989937589\\
69.125	0.11862	0.165081994359111	0.165081994359111\\
69.125	0.12228	0.152493660832747	0.152493660832747\\
69.125	0.12594	0.141962989358494	0.141962989358494\\
69.125	0.1296	0.133489979936354	0.133489979936354\\
69.125	0.13326	0.127074632566327	0.127074632566327\\
69.125	0.13692	0.122716947248412	0.122716947248412\\
69.125	0.14058	0.120416923982611	0.120416923982611\\
69.125	0.14424	0.12017456276892	0.12017456276892\\
69.125	0.1479	0.121989863607344	0.121989863607344\\
69.125	0.15156	0.125862826497878	0.125862826497878\\
69.125	0.15522	0.131793451440525	0.131793451440525\\
69.125	0.15888	0.139781738435286	0.139781738435286\\
69.125	0.16254	0.149827687482158	0.149827687482158\\
69.125	0.1662	0.161931298581142	0.161931298581142\\
69.125	0.16986	0.17609257173224	0.17609257173224\\
69.125	0.17352	0.192311506935448	0.192311506935448\\
69.125	0.17718	0.210588104190771	0.210588104190771\\
69.125	0.18084	0.230922363498204	0.230922363498204\\
69.125	0.1845	0.253314284857753	0.253314284857753\\
69.125	0.18816	0.277763868269411	0.277763868269411\\
69.125	0.19182	0.304271113733183	0.304271113733183\\
69.125	0.19548	0.332836021249066	0.332836021249066\\
69.125	0.19914	0.363458590817064	0.363458590817064\\
69.125	0.2028	0.396138822437172	0.396138822437172\\
69.125	0.20646	0.430876716109395	0.430876716109395\\
69.125	0.21012	0.467672271833728	0.467672271833728\\
69.125	0.21378	0.506525489610174	0.506525489610174\\
69.125	0.21744	0.547436369438733	0.547436369438733\\
69.125	0.2211	0.590404911319403	0.590404911319403\\
69.125	0.22476	0.635431115252187	0.635431115252187\\
69.125	0.22842	0.682514981237083	0.682514981237083\\
69.125	0.23208	0.731656509274093	0.731656509274093\\
69.125	0.23574	0.782855699363213	0.782855699363213\\
69.125	0.2394	0.836112551504447	0.836112551504447\\
69.125	0.24306	0.891427065697793	0.891427065697793\\
69.125	0.24672	0.948799241943251	0.948799241943251\\
69.125	0.25038	1.00822908024082	1.00822908024082\\
69.125	0.25404	1.06971658059051	1.06971658059051\\
69.125	0.2577	1.1332617429923	1.1332617429923\\
69.125	0.26136	1.19886456744621	1.19886456744621\\
69.125	0.26502	1.26652505395223	1.26652505395223\\
69.125	0.26868	1.33624320251036	1.33624320251036\\
69.125	0.27234	1.40801901312061	1.40801901312061\\
69.125	0.276	1.48185248578297	1.48185248578297\\
69.5	0.093	0.326126538072548	0.326126538072548\\
69.5	0.09666	0.298601252582392	0.298601252582392\\
69.5	0.10032	0.273133629144348	0.273133629144348\\
69.5	0.10398	0.249723667758416	0.249723667758416\\
69.5	0.10764	0.228371368424597	0.228371368424597\\
69.5	0.1113	0.209076731142889	0.209076731142889\\
69.5	0.11496	0.191839755913296	0.191839755913296\\
69.5	0.11862	0.176660442735812	0.176660442735812\\
69.5	0.12228	0.163538791610444	0.163538791610444\\
69.5	0.12594	0.152474802537186	0.152474802537186\\
69.5	0.1296	0.143468475516042	0.143468475516042\\
69.5	0.13326	0.13651981054701	0.13651981054701\\
69.5	0.13692	0.13162880763009	0.13162880763009\\
69.5	0.14058	0.128795466765282	0.128795466765282\\
69.5	0.14424	0.128019787952587	0.128019787952587\\
69.5	0.1479	0.129301771192005	0.129301771192005\\
69.5	0.15156	0.132641416483535	0.132641416483535\\
69.5	0.15522	0.138038723827177	0.138038723827177\\
69.5	0.15888	0.145493693222932	0.145493693222932\\
69.5	0.16254	0.155006324670799	0.155006324670799\\
69.5	0.1662	0.16657661817078	0.16657661817078\\
69.5	0.16986	0.180204573722872	0.180204573722872\\
69.5	0.17352	0.195890191327075	0.195890191327075\\
69.5	0.17718	0.213633470983392	0.213633470983392\\
69.5	0.18084	0.233434412691822	0.233434412691822\\
69.5	0.1845	0.255293016452364	0.255293016452364\\
69.5	0.18816	0.279209282265018	0.279209282265018\\
69.5	0.19182	0.305183210129786	0.305183210129786\\
69.5	0.19548	0.333214800046664	0.333214800046664\\
69.5	0.19914	0.363304052015656	0.363304052015656\\
69.5	0.2028	0.39545096603676	0.39545096603676\\
69.5	0.20646	0.429655542109976	0.429655542109976\\
69.5	0.21012	0.465917780235306	0.465917780235306\\
69.5	0.21378	0.504237680412746	0.504237680412746\\
69.5	0.21744	0.544615242642301	0.544615242642301\\
69.5	0.2211	0.587050466923966	0.587050466923966\\
69.5	0.22476	0.631543353257745	0.631543353257745\\
69.5	0.22842	0.678093901643636	0.678093901643636\\
69.5	0.23208	0.72670211208164	0.72670211208164\\
69.5	0.23574	0.777367984571756	0.777367984571756\\
69.5	0.2394	0.830091519113983	0.830091519113983\\
69.5	0.24306	0.884872715708325	0.884872715708325\\
69.5	0.24672	0.941711574354779	0.941711574354779\\
69.5	0.25038	1.00060809505335	1.00060809505335\\
69.5	0.25404	1.06156227780402	1.06156227780402\\
69.5	0.2577	1.12457412260681	1.12457412260681\\
69.5	0.26136	1.18964362946172	1.18964362946172\\
69.5	0.26502	1.25677079836873	1.25677079836873\\
69.5	0.26868	1.32595562932786	1.32595562932786\\
69.5	0.27234	1.3971981223391	1.3971981223391\\
69.5	0.276	1.47049827740245	1.47049827740245\\
69.875	0.093	0.341744828859611	0.341744828859611\\
69.875	0.09666	0.31368622577045	0.31368622577045\\
69.875	0.10032	0.2876852847334	0.2876852847334\\
69.875	0.10398	0.263742005748463	0.263742005748463\\
69.875	0.10764	0.241856388815639	0.241856388815639\\
69.875	0.1113	0.222028433934927	0.222028433934927\\
69.875	0.11496	0.204258141106329	0.204258141106329\\
69.875	0.11862	0.188545510329841	0.188545510329841\\
69.875	0.12228	0.174890541605467	0.174890541605467\\
69.875	0.12594	0.163293234933204	0.163293234933204\\
69.875	0.1296	0.153753590313054	0.153753590313054\\
69.875	0.13326	0.146271607745017	0.146271607745017\\
69.875	0.13692	0.140847287229093	0.140847287229093\\
69.875	0.14058	0.13748062876528	0.13748062876528\\
69.875	0.14424	0.13617163235358	0.13617163235358\\
69.875	0.1479	0.136920297993993	0.136920297993993\\
69.875	0.15156	0.139726625686518	0.139726625686518\\
69.875	0.15522	0.144590615431155	0.144590615431155\\
69.875	0.15888	0.151512267227906	0.151512267227906\\
69.875	0.16254	0.160491581076768	0.160491581076768\\
69.875	0.1662	0.171528556977742	0.171528556977742\\
69.875	0.16986	0.18462319493083	0.18462319493083\\
69.875	0.17352	0.199775494936028	0.199775494936028\\
69.875	0.17718	0.216985456993341	0.216985456993341\\
69.875	0.18084	0.236253081102765	0.236253081102765\\
69.875	0.1845	0.257578367264303	0.257578367264303\\
69.875	0.18816	0.280961315477952	0.280961315477952\\
69.875	0.19182	0.306401925743712	0.306401925743712\\
69.875	0.19548	0.333900198061587	0.333900198061587\\
69.875	0.19914	0.363456132431574	0.363456132431574\\
69.875	0.2028	0.395069728853674	0.395069728853674\\
69.875	0.20646	0.428740987327886	0.428740987327886\\
69.875	0.21012	0.464469907854208	0.464469907854208\\
69.875	0.21378	0.502256490432644	0.502256490432644\\
69.875	0.21744	0.542100735063193	0.542100735063193\\
69.875	0.2211	0.584002641745855	0.584002641745855\\
69.875	0.22476	0.627962210480629	0.627962210480629\\
69.875	0.22842	0.673979441267515	0.673979441267515\\
69.875	0.23208	0.722054334106514	0.722054334106514\\
69.875	0.23574	0.772186888997624	0.772186888997624\\
69.875	0.2394	0.824377105940847	0.824377105940847\\
69.875	0.24306	0.878624984936184	0.878624984936184\\
69.875	0.24672	0.934930525983632	0.934930525983632\\
69.875	0.25038	0.993293729083194	0.993293729083194\\
69.875	0.25404	1.05371459423487	1.05371459423487\\
69.875	0.2577	1.11619312143865	1.11619312143865\\
69.875	0.26136	1.18072931069455	1.18072931069455\\
69.875	0.26502	1.24732316200256	1.24732316200256\\
69.875	0.26868	1.31597467536268	1.31597467536268\\
69.875	0.27234	1.38668385077492	1.38668385077492\\
69.875	0.276	1.45945068823927	1.45945068823927\\
70.25	0.093	0.357669738863997	0.357669738863997\\
70.25	0.09666	0.329077818175831	0.329077818175831\\
70.25	0.10032	0.302543559539777	0.302543559539777\\
70.25	0.10398	0.278066962955835	0.278066962955835\\
70.25	0.10764	0.255648028424005	0.255648028424005\\
70.25	0.1113	0.235286755944288	0.235286755944288\\
70.25	0.11496	0.216983145516685	0.216983145516685\\
70.25	0.11862	0.200737197141192	0.200737197141192\\
70.25	0.12228	0.186548910817813	0.186548910817813\\
70.25	0.12594	0.174418286546545	0.174418286546545\\
70.25	0.1296	0.164345324327392	0.164345324327392\\
70.25	0.13326	0.156330024160349	0.156330024160349\\
70.25	0.13692	0.150372386045418	0.150372386045418\\
70.25	0.14058	0.146472409982601	0.146472409982601\\
70.25	0.14424	0.144630095971896	0.144630095971896\\
70.25	0.1479	0.144845444013304	0.144845444013304\\
70.25	0.15156	0.147118454106824	0.147118454106824\\
70.25	0.15522	0.151449126252456	0.151449126252456\\
70.25	0.15888	0.157837460450201	0.157837460450201\\
70.25	0.16254	0.166283456700059	0.166283456700059\\
70.25	0.1662	0.176787115002028	0.176787115002028\\
70.25	0.16986	0.189348435356111	0.189348435356111\\
70.25	0.17352	0.203967417762305	0.203967417762305\\
70.25	0.17718	0.220644062220612	0.220644062220612\\
70.25	0.18084	0.239378368731032	0.239378368731032\\
70.25	0.1845	0.260170337293564	0.260170337293564\\
70.25	0.18816	0.283019967908208	0.283019967908208\\
70.25	0.19182	0.307927260574964	0.307927260574964\\
70.25	0.19548	0.334892215293834	0.334892215293834\\
70.25	0.19914	0.363914832064816	0.363914832064816\\
70.25	0.2028	0.394995110887908	0.394995110887908\\
70.25	0.20646	0.428133051763117	0.428133051763117\\
70.25	0.21012	0.463328654690435	0.463328654690435\\
70.25	0.21378	0.500581919669866	0.500581919669866\\
70.25	0.21744	0.539892846701411	0.539892846701411\\
70.25	0.2211	0.581261435785066	0.581261435785066\\
70.25	0.22476	0.624687686920835	0.624687686920835\\
70.25	0.22842	0.670171600108717	0.670171600108717\\
70.25	0.23208	0.71771317534871	0.71771317534871\\
70.25	0.23574	0.767312412640817	0.767312412640817\\
70.25	0.2394	0.818969311985034	0.818969311985034\\
70.25	0.24306	0.872683873381365	0.872683873381365\\
70.25	0.24672	0.92845609682981	0.92845609682981\\
70.25	0.25038	0.986285982330364	0.986285982330364\\
70.25	0.25404	1.04617352988303	1.04617352988303\\
70.25	0.2577	1.10811873948781	1.10811873948781\\
70.25	0.26136	1.17212161114471	1.17212161114471\\
70.25	0.26502	1.23818214485371	1.23818214485371\\
70.25	0.26868	1.30630034061483	1.30630034061483\\
70.25	0.27234	1.37647619842806	1.37647619842806\\
70.25	0.276	1.4487097182934	1.4487097182934\\
70.625	0.093	0.373901268085709	0.373901268085709\\
70.625	0.09666	0.344776029798537	0.344776029798537\\
70.625	0.10032	0.317708453563479	0.317708453563479\\
70.625	0.10398	0.292698539380531	0.292698539380531\\
70.625	0.10764	0.269746287249697	0.269746287249697\\
70.625	0.1113	0.248851697170975	0.248851697170975\\
70.625	0.11496	0.230014769144366	0.230014769144366\\
70.625	0.11862	0.213235503169869	0.213235503169869\\
70.625	0.12228	0.198513899247485	0.198513899247485\\
70.625	0.12594	0.185849957377213	0.185849957377213\\
70.625	0.1296	0.175243677559053	0.175243677559053\\
70.625	0.13326	0.166695059793006	0.166695059793006\\
70.625	0.13692	0.160204104079071	0.160204104079071\\
70.625	0.14058	0.155770810417248	0.155770810417248\\
70.625	0.14424	0.153395178807539	0.153395178807539\\
70.625	0.1479	0.153077209249941	0.153077209249941\\
70.625	0.15156	0.154816901744457	0.154816901744457\\
70.625	0.15522	0.158614256291084	0.158614256291084\\
70.625	0.15888	0.164469272889823	0.164469272889823\\
70.625	0.16254	0.172381951540675	0.172381951540675\\
70.625	0.1662	0.182352292243641	0.182352292243641\\
70.625	0.16986	0.194380294998719	0.194380294998719\\
70.625	0.17352	0.208465959805908	0.208465959805908\\
70.625	0.17718	0.224609286665209	0.224609286665209\\
70.625	0.18084	0.242810275576623	0.242810275576623\\
70.625	0.1845	0.263068926540151	0.263068926540151\\
70.625	0.18816	0.28538523955579	0.28538523955579\\
70.625	0.19182	0.309759214623542	0.309759214623542\\
70.625	0.19548	0.336190851743405	0.336190851743405\\
70.625	0.19914	0.364680150915383	0.364680150915383\\
70.625	0.2028	0.395227112139471	0.395227112139471\\
70.625	0.20646	0.427831735415674	0.427831735415674\\
70.625	0.21012	0.462494020743988	0.462494020743988\\
70.625	0.21378	0.499213968124413	0.499213968124413\\
70.625	0.21744	0.537991577556953	0.537991577556953\\
70.625	0.2211	0.578826849041604	0.578826849041604\\
70.625	0.22476	0.621719782578367	0.621719782578367\\
70.625	0.22842	0.666670378167242	0.666670378167242\\
70.625	0.23208	0.713678635808233	0.713678635808233\\
70.625	0.23574	0.762744555501334	0.762744555501334\\
70.625	0.2394	0.813868137246546	0.813868137246546\\
70.625	0.24306	0.867049381043873	0.867049381043873\\
70.625	0.24672	0.922288286893312	0.922288286893312\\
70.625	0.25038	0.979584854794862	0.979584854794862\\
70.625	0.25404	1.03893908474852	1.03893908474852\\
70.625	0.2577	1.1003509767543	1.1003509767543\\
70.625	0.26136	1.16382053081219	1.16382053081219\\
70.625	0.26502	1.22934774692219	1.22934774692219\\
70.625	0.26868	1.2969326250843	1.2969326250843\\
70.625	0.27234	1.36657516529853	1.36657516529853\\
70.625	0.276	1.43827536756486	1.43827536756486\\
71	0.093	0.390439416524744	0.390439416524744\\
71	0.09666	0.360780860638568	0.360780860638568\\
71	0.10032	0.333179966804503	0.333179966804503\\
71	0.10398	0.307636735022552	0.307636735022552\\
71	0.10764	0.284151165292712	0.284151165292712\\
71	0.1113	0.262723257614986	0.262723257614986\\
71	0.11496	0.243353011989371	0.243353011989371\\
71	0.11862	0.226040428415869	0.226040428415869\\
71	0.12228	0.21078550689448	0.21078550689448\\
71	0.12594	0.197588247425203	0.197588247425203\\
71	0.1296	0.186448650008038	0.186448650008038\\
71	0.13326	0.177366714642986	0.177366714642986\\
71	0.13692	0.170342441330046	0.170342441330046\\
71	0.14058	0.165375830069219	0.165375830069219\\
71	0.14424	0.162466880860504	0.162466880860504\\
71	0.1479	0.161615593703902	0.161615593703902\\
71	0.15156	0.162821968599412	0.162821968599412\\
71	0.15522	0.166086005547034	0.166086005547034\\
71	0.15888	0.17140770454677	0.17140770454677\\
71	0.16254	0.178787065598616	0.178787065598616\\
71	0.1662	0.188224088702577	0.188224088702577\\
71	0.16986	0.199718773858649	0.199718773858649\\
71	0.17352	0.213271121066833	0.213271121066833\\
71	0.17718	0.228881130327129	0.228881130327129\\
71	0.18084	0.246548801639539	0.246548801639539\\
71	0.1845	0.266274135004062	0.266274135004062\\
71	0.18816	0.288057130420696	0.288057130420696\\
71	0.19182	0.311897787889443	0.311897787889443\\
71	0.19548	0.337796107410302	0.337796107410302\\
71	0.19914	0.365752088983274	0.365752088983274\\
71	0.2028	0.395765732608359	0.395765732608359\\
71	0.20646	0.427837038285556	0.427837038285556\\
71	0.21012	0.461966006014863	0.461966006014863\\
71	0.21378	0.498152635796286	0.498152635796286\\
71	0.21744	0.536396927629819	0.536396927629819\\
71	0.2211	0.576698881515465	0.576698881515465\\
71	0.22476	0.619058497453223	0.619058497453223\\
71	0.22842	0.663475775443096	0.663475775443096\\
71	0.23208	0.709950715485079	0.709950715485079\\
71	0.23574	0.758483317579175	0.758483317579175\\
71	0.2394	0.809073581725384	0.809073581725384\\
71	0.24306	0.861721507923703	0.861721507923703\\
71	0.24672	0.916427096174137	0.916427096174137\\
71	0.25038	0.973190346476684	0.973190346476684\\
71	0.25404	1.03201125883134	1.03201125883134\\
71	0.2577	1.09288983323811	1.09288983323811\\
71	0.26136	1.155826069697	1.155826069697\\
71	0.26502	1.22081996820799	1.22081996820799\\
71	0.26868	1.2878715287711	1.2878715287711\\
71	0.27234	1.35698075138632	1.35698075138632\\
71	0.276	1.42814763605365	1.42814763605365\\
71.375	0.093	0.407284184181106	0.407284184181106\\
71.375	0.09666	0.377092310695925	0.377092310695925\\
71.375	0.10032	0.348958099262855	0.348958099262855\\
71.375	0.10398	0.3228815498819	0.3228815498819\\
71.375	0.10764	0.298862662553055	0.298862662553055\\
71.375	0.1113	0.276901437276322	0.276901437276322\\
71.375	0.11496	0.256997874051704	0.256997874051704\\
71.375	0.11862	0.239151972879196	0.239151972879196\\
71.375	0.12228	0.223363733758802	0.223363733758802\\
71.375	0.12594	0.209633156690521	0.209633156690521\\
71.375	0.1296	0.19796024167435	0.19796024167435\\
71.375	0.13326	0.188344988710294	0.188344988710294\\
71.375	0.13692	0.180787397798348	0.180787397798348\\
71.375	0.14058	0.175287468938516	0.175287468938516\\
71.375	0.14424	0.171845202130796	0.171845202130796\\
71.375	0.1479	0.170460597375189	0.170460597375189\\
71.375	0.15156	0.171133654671694	0.171133654671694\\
71.375	0.15522	0.173864374020312	0.173864374020312\\
71.375	0.15888	0.178652755421042	0.178652755421042\\
71.375	0.16254	0.185498798873883	0.185498798873883\\
71.375	0.1662	0.194402504378838	0.194402504378838\\
71.375	0.16986	0.205363871935906	0.205363871935906\\
71.375	0.17352	0.218382901545084	0.218382901545084\\
71.375	0.17718	0.233459593206377	0.233459593206377\\
71.375	0.18084	0.250593946919781	0.250593946919781\\
71.375	0.1845	0.269785962685299	0.269785962685299\\
71.375	0.18816	0.291035640502929	0.291035640502929\\
71.375	0.19182	0.31434298037267	0.31434298037267\\
71.375	0.19548	0.339707982294523	0.339707982294523\\
71.375	0.19914	0.367130646268491	0.367130646268491\\
71.375	0.2028	0.39661097229457	0.39661097229457\\
71.375	0.20646	0.428148960372762	0.428148960372762\\
71.375	0.21012	0.461744610503066	0.461744610503066\\
71.375	0.21378	0.497397922685482	0.497397922685482\\
71.375	0.21744	0.53510889692001	0.53510889692001\\
71.375	0.2211	0.574877533206652	0.574877533206652\\
71.375	0.22476	0.616703831545405	0.616703831545405\\
71.375	0.22842	0.660587791936271	0.660587791936271\\
71.375	0.23208	0.706529414379251	0.706529414379251\\
71.375	0.23574	0.754528698874342	0.754528698874342\\
71.375	0.2394	0.804585645421544	0.804585645421544\\
71.375	0.24306	0.856700254020861	0.856700254020861\\
71.375	0.24672	0.910872524672289	0.910872524672289\\
71.375	0.25038	0.96710245737583	0.96710245737583\\
71.375	0.25404	1.02539005213148	1.02539005213148\\
71.375	0.2577	1.08573530893925	1.08573530893925\\
71.375	0.26136	1.14813822779913	1.14813822779913\\
71.375	0.26502	1.21259880871112	1.21259880871112\\
71.375	0.26868	1.27911705167522	1.27911705167522\\
71.375	0.27234	1.34769295669144	1.34769295669144\\
71.375	0.276	1.41832652375976	1.41832652375976\\
71.75	0.093	0.424435571054792	0.424435571054792\\
71.75	0.09666	0.393710379970606	0.393710379970606\\
71.75	0.10032	0.365042850938532	0.365042850938532\\
71.75	0.10398	0.33843298395857	0.33843298395857\\
71.75	0.10764	0.313880779030721	0.313880779030721\\
71.75	0.1113	0.291386236154984	0.291386236154984\\
71.75	0.11496	0.27094935533136	0.27094935533136\\
71.75	0.11862	0.252570136559847	0.252570136559847\\
71.75	0.12228	0.236248579840449	0.236248579840449\\
71.75	0.12594	0.221984685173161	0.221984685173161\\
71.75	0.1296	0.209778452557987	0.209778452557987\\
71.75	0.13326	0.199629881994925	0.199629881994925\\
71.75	0.13692	0.191538973483975	0.191538973483975\\
71.75	0.14058	0.185505727025137	0.185505727025137\\
71.75	0.14424	0.181530142618412	0.181530142618412\\
71.75	0.1479	0.1796122202638	0.1796122202638\\
71.75	0.15156	0.1797519599613	0.1797519599613\\
71.75	0.15522	0.181949361710913	0.181949361710913\\
71.75	0.15888	0.186204425512638	0.186204425512638\\
71.75	0.16254	0.192517151366475	0.192517151366475\\
71.75	0.1662	0.200887539272425	0.200887539272425\\
71.75	0.16986	0.211315589230487	0.211315589230487\\
71.75	0.17352	0.223801301240662	0.223801301240662\\
71.75	0.17718	0.238344675302949	0.238344675302949\\
71.75	0.18084	0.254945711417348	0.254945711417348\\
71.75	0.1845	0.27360440958386	0.27360440958386\\
71.75	0.18816	0.294320769802486	0.294320769802486\\
71.75	0.19182	0.317094792073221	0.317094792073221\\
71.75	0.19548	0.341926476396071	0.341926476396071\\
71.75	0.19914	0.368815822771033	0.368815822771033\\
71.75	0.2028	0.397762831198106	0.397762831198106\\
71.75	0.20646	0.428767501677294	0.428767501677294\\
71.75	0.21012	0.461829834208593	0.461829834208593\\
71.75	0.21378	0.496949828792002	0.496949828792002\\
71.75	0.21744	0.534127485427529	0.534127485427529\\
71.75	0.2211	0.573362804115163	0.573362804115163\\
71.75	0.22476	0.614655784854913	0.614655784854913\\
71.75	0.22842	0.658006427646774	0.658006427646774\\
71.75	0.23208	0.703414732490747	0.703414732490747\\
71.75	0.23574	0.750880699386834	0.750880699386834\\
71.75	0.2394	0.800404328335032	0.800404328335032\\
71.75	0.24306	0.851985619335344	0.851985619335344\\
71.75	0.24672	0.905624572387766	0.905624572387766\\
71.75	0.25038	0.961321187492303	0.961321187492303\\
71.75	0.25404	1.01907546464895	1.01907546464895\\
71.75	0.2577	1.07888740385771	1.07888740385771\\
71.75	0.26136	1.14075700511858	1.14075700511858\\
71.75	0.26502	1.20468426843157	1.20468426843157\\
71.75	0.26868	1.27066919379667	1.27066919379667\\
71.75	0.27234	1.33871178121388	1.33871178121388\\
71.75	0.276	1.4088120306832	1.4088120306832\\
72.125	0.093	0.441893577145803	0.441893577145803\\
72.125	0.09666	0.410635068462611	0.410635068462611\\
72.125	0.10032	0.381434221831532	0.381434221831532\\
72.125	0.10398	0.354291037252565	0.354291037252565\\
72.125	0.10764	0.329205514725711	0.329205514725711\\
72.125	0.1113	0.306177654250969	0.306177654250969\\
72.125	0.11496	0.28520745582834	0.28520745582834\\
72.125	0.11862	0.266294919457823	0.266294919457823\\
72.125	0.12228	0.249440045139419	0.249440045139419\\
72.125	0.12594	0.234642832873127	0.234642832873127\\
72.125	0.1296	0.221903282658947	0.221903282658947\\
72.125	0.13326	0.211221394496879	0.211221394496879\\
72.125	0.13692	0.202597168386925	0.202597168386925\\
72.125	0.14058	0.196030604329083	0.196030604329083\\
72.125	0.14424	0.191521702323354	0.191521702323354\\
72.125	0.1479	0.189070462369736	0.189070462369736\\
72.125	0.15156	0.188676884468231	0.188676884468231\\
72.125	0.15522	0.190340968618839	0.190340968618839\\
72.125	0.15888	0.194062714821559	0.194062714821559\\
72.125	0.16254	0.19984212307639	0.19984212307639\\
72.125	0.1662	0.207679193383336	0.207679193383336\\
72.125	0.16986	0.217573925742394	0.217573925742394\\
72.125	0.17352	0.229526320153562	0.229526320153562\\
72.125	0.17718	0.243536376616845	0.243536376616845\\
72.125	0.18084	0.259604095132238	0.259604095132238\\
72.125	0.1845	0.277729475699747	0.277729475699747\\
72.125	0.18816	0.297912518319366	0.297912518319366\\
72.125	0.19182	0.320153222991097	0.320153222991097\\
72.125	0.19548	0.344451589714942	0.344451589714942\\
72.125	0.19914	0.370807618490898	0.370807618490898\\
72.125	0.2028	0.399221309318968	0.399221309318968\\
72.125	0.20646	0.429692662199149	0.429692662199149\\
72.125	0.21012	0.462221677131443	0.462221677131443\\
72.125	0.21378	0.496808354115849	0.496808354115849\\
72.125	0.21744	0.533452693152368	0.533452693152368\\
72.125	0.2211	0.572154694241	0.572154694241\\
72.125	0.22476	0.612914357381744	0.612914357381744\\
72.125	0.22842	0.6557316825746	0.6557316825746\\
72.125	0.23208	0.700606669819568	0.700606669819568\\
72.125	0.23574	0.74753931911665	0.74753931911665\\
72.125	0.2394	0.796529630465843	0.796529630465843\\
72.125	0.24306	0.847577603867149	0.847577603867149\\
72.125	0.24672	0.900683239320569	0.900683239320569\\
72.125	0.25038	0.955846536826098	0.955846536826098\\
72.125	0.25404	1.01306749638374	1.01306749638374\\
72.125	0.2577	1.0723461179935	1.0723461179935\\
72.125	0.26136	1.13368240165537	1.13368240165537\\
72.125	0.26502	1.19707634736935	1.19707634736935\\
72.125	0.26868	1.26252795513544	1.26252795513544\\
72.125	0.27234	1.33003722495364	1.33003722495364\\
72.125	0.276	1.39960415682396	1.39960415682396\\
72.5	0.093	0.459658202454138	0.459658202454138\\
72.5	0.09666	0.427866376171942	0.427866376171942\\
72.5	0.10032	0.398132211941858	0.398132211941858\\
72.5	0.10398	0.370455709763886	0.370455709763886\\
72.5	0.10764	0.344836869638027	0.344836869638027\\
72.5	0.1113	0.32127569156428	0.32127569156428\\
72.5	0.11496	0.299772175542646	0.299772175542646\\
72.5	0.11862	0.280326321573124	0.280326321573124\\
72.5	0.12228	0.262938129655715	0.262938129655715\\
72.5	0.12594	0.247607599790418	0.247607599790418\\
72.5	0.1296	0.234334731977233	0.234334731977233\\
72.5	0.13326	0.223119526216161	0.223119526216161\\
72.5	0.13692	0.213961982507202	0.213961982507202\\
72.5	0.14058	0.206862100850354	0.206862100850354\\
72.5	0.14424	0.20181988124562	0.20181988124562\\
72.5	0.1479	0.198835323692997	0.198835323692997\\
72.5	0.15156	0.197908428192487	0.197908428192487\\
72.5	0.15522	0.19903919474409	0.19903919474409\\
72.5	0.15888	0.202227623347804	0.202227623347804\\
72.5	0.16254	0.207473714003632	0.207473714003632\\
72.5	0.1662	0.214777466711572	0.214777466711572\\
72.5	0.16986	0.224138881471625	0.224138881471625\\
72.5	0.17352	0.235557958283788	0.235557958283788\\
72.5	0.17718	0.249034697148065	0.249034697148065\\
72.5	0.18084	0.264569098064455	0.264569098064455\\
72.5	0.1845	0.282161161032958	0.282161161032958\\
72.5	0.18816	0.301810886053573	0.301810886053573\\
72.5	0.19182	0.323518273126299	0.323518273126299\\
72.5	0.19548	0.347283322251137	0.347283322251137\\
72.5	0.19914	0.37310603342809	0.37310603342809\\
72.5	0.2028	0.400986406657154	0.400986406657154\\
72.5	0.20646	0.43092444193833	0.43092444193833\\
72.5	0.21012	0.46292013927162	0.46292013927162\\
72.5	0.21378	0.496973498657022	0.496973498657022\\
72.5	0.21744	0.533084520094535	0.533084520094535\\
72.5	0.2211	0.57125320358416	0.57125320358416\\
72.5	0.22476	0.6114795491259	0.6114795491259\\
72.5	0.22842	0.653763556719752	0.653763556719752\\
72.5	0.23208	0.698105226365715	0.698105226365715\\
72.5	0.23574	0.744504558063791	0.744504558063791\\
72.5	0.2394	0.792961551813979	0.792961551813979\\
72.5	0.24306	0.84347620761628	0.84347620761628\\
72.5	0.24672	0.896048525470694	0.896048525470694\\
72.5	0.25038	0.95067850537722	0.95067850537722\\
72.5	0.25404	1.00736614733586	1.00736614733586\\
72.5	0.2577	1.06611145134661	1.06611145134661\\
72.5	0.26136	1.12691441740947	1.12691441740947\\
72.5	0.26502	1.18977504552445	1.18977504552445\\
72.5	0.26868	1.25469333569154	1.25469333569154\\
72.5	0.27234	1.32166928791074	1.32166928791074\\
72.5	0.276	1.39070290218205	1.39070290218205\\
72.875	0.093	0.477729446979799	0.477729446979799\\
72.875	0.09666	0.445404303098598	0.445404303098598\\
72.875	0.10032	0.415136821269509	0.415136821269509\\
72.875	0.10398	0.386927001492532	0.386927001492532\\
72.875	0.10764	0.360774843767667	0.360774843767667\\
72.875	0.1113	0.336680348094916	0.336680348094916\\
72.875	0.11496	0.314643514474277	0.314643514474277\\
72.875	0.11862	0.29466434290575	0.29466434290575\\
72.875	0.12228	0.276742833389336	0.276742833389336\\
72.875	0.12594	0.260878985925033	0.260878985925033\\
72.875	0.1296	0.247072800512845	0.247072800512845\\
72.875	0.13326	0.235324277152767	0.235324277152767\\
72.875	0.13692	0.225633415844802	0.225633415844802\\
72.875	0.14058	0.21800021658895	0.21800021658895\\
72.875	0.14424	0.21242467938521	0.21242467938521\\
72.875	0.1479	0.208906804233583	0.208906804233583\\
72.875	0.15156	0.207446591134068	0.207446591134068\\
72.875	0.15522	0.208044040086666	0.208044040086666\\
72.875	0.15888	0.210699151091375	0.210699151091375\\
72.875	0.16254	0.215411924148198	0.215411924148198\\
72.875	0.1662	0.222182359257133	0.222182359257133\\
72.875	0.16986	0.23101045641818	0.23101045641818\\
72.875	0.17352	0.24189621563134	0.24189621563134\\
72.875	0.17718	0.254839636896611	0.254839636896611\\
72.875	0.18084	0.269840720213996	0.269840720213996\\
72.875	0.1845	0.286899465583492	0.286899465583492\\
72.875	0.18816	0.306015873005104	0.306015873005104\\
72.875	0.19182	0.327189942478824	0.327189942478824\\
72.875	0.19548	0.350421674004659	0.350421674004659\\
72.875	0.19914	0.375711067582606	0.375711067582606\\
72.875	0.2028	0.403058123212664	0.403058123212664\\
72.875	0.20646	0.432462840894838	0.432462840894838\\
72.875	0.21012	0.46392522062912	0.46392522062912\\
72.875	0.21378	0.497445262415516	0.497445262415516\\
72.875	0.21744	0.533022966254026	0.533022966254026\\
72.875	0.2211	0.570658332144647	0.570658332144647\\
72.875	0.22476	0.610351360087381	0.610351360087381\\
72.875	0.22842	0.652102050082226	0.652102050082226\\
72.875	0.23208	0.695910402129186	0.695910402129186\\
72.875	0.23574	0.741776416228257	0.741776416228257\\
72.875	0.2394	0.789700092379442	0.789700092379442\\
72.875	0.24306	0.839681430582736	0.839681430582736\\
72.875	0.24672	0.891720430838145	0.891720430838145\\
72.875	0.25038	0.945817093145667	0.945817093145667\\
72.875	0.25404	1.0019714175053	1.0019714175053\\
72.875	0.2577	1.06018340391705	1.06018340391705\\
72.875	0.26136	1.1204530523809	1.1204530523809\\
72.875	0.26502	1.18278036289687	1.18278036289687\\
72.875	0.26868	1.24716533546496	1.24716533546496\\
72.875	0.27234	1.31360797008515	1.31360797008515\\
72.875	0.276	1.38210826675746	1.38210826675746\\
73.25	0.093	0.496107310722785	0.496107310722785\\
73.25	0.09666	0.46324884924258	0.46324884924258\\
73.25	0.10032	0.432448049814486	0.432448049814486\\
73.25	0.10398	0.403704912438503	0.403704912438503\\
73.25	0.10764	0.377019437114634	0.377019437114634\\
73.25	0.1113	0.352391623842877	0.352391623842877\\
73.25	0.11496	0.329821472623234	0.329821472623234\\
73.25	0.11862	0.309308983455701	0.309308983455701\\
73.25	0.12228	0.290854156340282	0.290854156340282\\
73.25	0.12594	0.274456991276975	0.274456991276975\\
73.25	0.1296	0.260117488265781	0.260117488265781\\
73.25	0.13326	0.247835647306698	0.247835647306698\\
73.25	0.13692	0.237611468399729	0.237611468399729\\
73.25	0.14058	0.229444951544871	0.229444951544871\\
73.25	0.14424	0.223336096742127	0.223336096742127\\
73.25	0.1479	0.219284903991495	0.219284903991495\\
73.25	0.15156	0.217291373292975	0.217291373292975\\
73.25	0.15522	0.217355504646568	0.217355504646568\\
73.25	0.15888	0.219477298052273	0.219477298052273\\
73.25	0.16254	0.22365675351009	0.22365675351009\\
73.25	0.1662	0.229893871020019	0.229893871020019\\
73.25	0.16986	0.238188650582062	0.238188650582062\\
73.25	0.17352	0.248541092196217	0.248541092196217\\
73.25	0.17718	0.260951195862483	0.260951195862483\\
73.25	0.18084	0.275418961580864	0.275418961580864\\
73.25	0.1845	0.291944389351356	0.291944389351356\\
73.25	0.18816	0.310527479173961	0.310527479173961\\
73.25	0.19182	0.331168231048677	0.331168231048677\\
73.25	0.19548	0.353866644975505	0.353866644975505\\
73.25	0.19914	0.378622720954447	0.378622720954447\\
73.25	0.2028	0.405436458985501	0.405436458985501\\
73.25	0.20646	0.43430785906867	0.43430785906867\\
73.25	0.21012	0.465236921203948	0.465236921203948\\
73.25	0.21378	0.498223645391338	0.498223645391338\\
73.25	0.21744	0.533268031630844	0.533268031630844\\
73.25	0.2211	0.570370079922459	0.570370079922459\\
73.25	0.22476	0.609529790266188	0.609529790266188\\
73.25	0.22842	0.650747162662029	0.650747162662029\\
73.25	0.23208	0.694022197109984	0.694022197109984\\
73.25	0.23574	0.73935489361005	0.73935489361005\\
73.25	0.2394	0.786745252162228	0.786745252162228\\
73.25	0.24306	0.836193272766518	0.836193272766518\\
73.25	0.24672	0.887698955422923	0.887698955422923\\
73.25	0.25038	0.94126230013144	0.94126230013144\\
73.25	0.25404	0.996883306892067	0.996883306892067\\
73.25	0.2577	1.05456197570481	1.05456197570481\\
73.25	0.26136	1.11429830656966	1.11429830656966\\
73.25	0.26502	1.17609229948663	1.17609229948663\\
73.25	0.26868	1.2399439544557	1.2399439544557\\
73.25	0.27234	1.3058532714769	1.3058532714769\\
73.25	0.276	1.3738202505502	1.3738202505502\\
73.625	0.093	0.514791793683095	0.514791793683095\\
73.625	0.09666	0.481400014603884	0.481400014603884\\
73.625	0.10032	0.450065897576784	0.450065897576784\\
73.625	0.10398	0.420789442601798	0.420789442601798\\
73.625	0.10764	0.393570649678923	0.393570649678923\\
73.625	0.1113	0.368409518808162	0.368409518808162\\
73.625	0.11496	0.345306049989513	0.345306049989513\\
73.625	0.11862	0.324260243222975	0.324260243222975\\
73.625	0.12228	0.305272098508551	0.305272098508551\\
73.625	0.12594	0.28834161584624	0.28834161584624\\
73.625	0.1296	0.27346879523604	0.27346879523604\\
73.625	0.13326	0.260653636677952	0.260653636677952\\
73.625	0.13692	0.249896140171978	0.249896140171978\\
73.625	0.14058	0.241196305718116	0.241196305718116\\
73.625	0.14424	0.234554133316366	0.234554133316366\\
73.625	0.1479	0.229969622966729	0.229969622966729\\
73.625	0.15156	0.227442774669204	0.227442774669204\\
73.625	0.15522	0.226973588423792	0.226973588423792\\
73.625	0.15888	0.228562064230491	0.228562064230491\\
73.625	0.16254	0.232208202089303	0.232208202089303\\
73.625	0.1662	0.237912002000229	0.237912002000229\\
73.625	0.16986	0.245673463963267	0.245673463963267\\
73.625	0.17352	0.255492587978416	0.255492587978416\\
73.625	0.17718	0.267369374045677	0.267369374045677\\
73.625	0.18084	0.281303822165052	0.281303822165052\\
73.625	0.1845	0.29729593233654	0.29729593233654\\
73.625	0.18816	0.315345704560141	0.315345704560141\\
73.625	0.19182	0.335453138835851	0.335453138835851\\
73.625	0.19548	0.357618235163674	0.357618235163674\\
73.625	0.19914	0.381840993543612	0.381840993543612\\
73.625	0.2028	0.408121413975662	0.408121413975662\\
73.625	0.20646	0.436459496459825	0.436459496459825\\
73.625	0.21012	0.466855240996098	0.466855240996098\\
73.625	0.21378	0.499308647584483	0.499308647584483\\
73.625	0.21744	0.533819716224982	0.533819716224982\\
73.625	0.2211	0.570388446917594	0.570388446917594\\
73.625	0.22476	0.609014839662319	0.609014839662319\\
73.625	0.22842	0.649698894459154	0.649698894459154\\
73.625	0.23208	0.692440611308103	0.692440611308103\\
73.625	0.23574	0.737239990209164	0.737239990209164\\
73.625	0.2394	0.784097031162338	0.784097031162338\\
73.625	0.24306	0.833011734167624	0.833011734167624\\
73.625	0.24672	0.883984099225022	0.883984099225022\\
73.625	0.25038	0.937014126334533	0.937014126334533\\
73.625	0.25404	0.992101815496156	0.992101815496156\\
73.625	0.2577	1.04924716670989	1.04924716670989\\
73.625	0.26136	1.10845017997574	1.10845017997574\\
73.625	0.26502	1.1697108552937	1.1697108552937\\
73.625	0.26868	1.23302919266377	1.23302919266377\\
73.625	0.27234	1.29840519208596	1.29840519208596\\
73.625	0.276	1.36583885356026	1.36583885356026\\
74	0.093	0.53378289586073	0.53378289586073\\
74	0.09666	0.499857799182514	0.499857799182514\\
74	0.10032	0.467990364556409	0.467990364556409\\
74	0.10398	0.438180591982418	0.438180591982418\\
74	0.10764	0.410428481460538	0.410428481460538\\
74	0.1113	0.384734032990772	0.384734032990772\\
74	0.11496	0.361097246573118	0.361097246573118\\
74	0.11862	0.339518122207576	0.339518122207576\\
74	0.12228	0.319996659894147	0.319996659894147\\
74	0.12594	0.302532859632829	0.302532859632829\\
74	0.1296	0.287126721423625	0.287126721423625\\
74	0.13326	0.273778245266534	0.273778245266534\\
74	0.13692	0.262487431161553	0.262487431161553\\
74	0.14058	0.253254279108686	0.253254279108686\\
74	0.14424	0.246078789107931	0.246078789107931\\
74	0.1479	0.24096096115929	0.24096096115929\\
74	0.15156	0.23790079526276	0.23790079526276\\
74	0.15522	0.236898291418342	0.236898291418342\\
74	0.15888	0.237953449626037	0.237953449626037\\
74	0.16254	0.241066269885844	0.241066269885844\\
74	0.1662	0.246236752197765	0.246236752197765\\
74	0.16986	0.253464896561797	0.253464896561797\\
74	0.17352	0.262750702977941	0.262750702977941\\
74	0.17718	0.274094171446198	0.274094171446198\\
74	0.18084	0.287495301966569	0.287495301966569\\
74	0.1845	0.30295409453905	0.30295409453905\\
74	0.18816	0.320470549163646	0.320470549163646\\
74	0.19182	0.340044665840351	0.340044665840351\\
74	0.19548	0.36167644456917	0.36167644456917\\
74	0.19914	0.385365885350102	0.385365885350102\\
74	0.2028	0.411112988183147	0.411112988183147\\
74	0.20646	0.438917753068304	0.438917753068304\\
74	0.21012	0.468780180005573	0.468780180005573\\
74	0.21378	0.500700268994954	0.500700268994954\\
74	0.21744	0.534678020036448	0.534678020036448\\
74	0.2211	0.570713433130054	0.570713433130054\\
74	0.22476	0.608806508275773	0.608806508275773\\
74	0.22842	0.648957245473604	0.648957245473604\\
74	0.23208	0.69116564472355	0.69116564472355\\
74	0.23574	0.735431706025605	0.735431706025605\\
74	0.2394	0.781755429379775	0.781755429379775\\
74	0.24306	0.830136814786054	0.830136814786054\\
74	0.24672	0.880575862244448	0.880575862244448\\
74	0.25038	0.933072571754955	0.933072571754955\\
74	0.25404	0.987626943317572	0.987626943317572\\
74	0.2577	1.0442389769323	1.0442389769323\\
74	0.26136	1.10290867259915	1.10290867259915\\
74	0.26502	1.1636360303181	1.1636360303181\\
74	0.26868	1.22642105008917	1.22642105008917\\
74	0.27234	1.29126373191235	1.29126373191235\\
74	0.276	1.35816407578764	1.35816407578764\\
};
\end{axis}
\end{tikzpicture}%
%	\caption{The quantitative relationship between external parameters ($L_{cut}, D_{rlx}$) and internal parameters $\theta$ is obtained by fitting a polynomial surface to the internal parameter values estimated for the sets \dataset1-\dataset5 (\ref{tikz:thetas1}) and sets \dataset6-\dataset10 (\ref{tikz:thetas2}). The fitted surfaces (\ref{tikz:theta_surf}) are then used to compute the internal model parameters corresponding to the settings of choice (\ref{tikz:thetaidentified}).}\label{fig:surfaces_all}
%\end{figure}
%\begin{figure}[!t]
%	\centering
%	\subfloat[\dataset3]{% This file was created by matlab2tikz.
%
\definecolor{mycolor1}{rgb}{0.00000,0.44700,0.74100}%
\definecolor{mycolor2}{rgb}{0.85000,0.32500,0.09800}%
%
\begin{tikzpicture}

\begin{axis}[%
width=11.411cm,
height=6cm,
at={(0cm,0cm)},
scale only axis,
xmin=1000,
xmax=1500,
ymin=-45,
ymax=0,
axis background/.style={fill=white},
legend style={legend cell align=left, align=left, draw=white!15!black}
]
\addplot [color=mycolor1]
  table[row sep=crcr]{%
1001	-17.09\\
1002	-19.531\\
1003	-14.648\\
1004	-14.648\\
1005	-19.531\\
1006	-18.311\\
1007	-23.193\\
1008	-18.311\\
1009	-9.766\\
1010	-15.869\\
1011	-12.207\\
1012	-14.648\\
1013	-10.986\\
1014	-3.662\\
1015	-2.441\\
1016	-4.883\\
1017	-6.104\\
1018	-14.648\\
1019	-17.09\\
1020	-17.09\\
1021	-13.428\\
1022	-9.766\\
1023	-18.311\\
1024	-14.648\\
1025	-10.986\\
1026	-13.428\\
1027	-12.207\\
1028	-10.986\\
1029	-20.752\\
1030	-19.531\\
1031	-12.207\\
1032	-20.752\\
1033	-20.752\\
1034	-15.869\\
1035	-13.428\\
1036	-9.766\\
1037	-14.648\\
1038	-14.648\\
1039	-14.648\\
1040	-14.648\\
1041	-14.648\\
1042	-15.869\\
1043	-21.973\\
1044	-20.752\\
1045	-13.428\\
1046	-13.428\\
1047	-8.545\\
1048	-12.207\\
1049	-12.207\\
1050	-13.428\\
1051	-10.986\\
1052	-13.428\\
1053	-14.648\\
1054	-13.428\\
1055	-20.752\\
1056	-13.428\\
1057	-9.766\\
1058	-7.324\\
1059	-8.545\\
1060	-10.986\\
1061	-6.104\\
1062	-9.766\\
1063	-10.986\\
1064	-6.104\\
1065	-6.104\\
1066	-9.766\\
1067	-9.766\\
1068	-15.869\\
1069	-14.648\\
1070	-17.09\\
1071	-15.869\\
1072	-18.311\\
1073	-13.428\\
1074	-14.648\\
1075	-12.207\\
1076	-10.986\\
1077	-12.207\\
1078	-23.193\\
1079	-28.076\\
1080	-30.518\\
1081	-29.297\\
1082	-18.311\\
1083	-28.076\\
1084	-32.959\\
1085	-31.738\\
1086	-20.752\\
1087	-34.18\\
1088	-40.283\\
1089	-29.297\\
1090	-24.414\\
1091	-19.531\\
1092	-15.869\\
1093	-12.207\\
1094	-14.648\\
1095	-9.766\\
1096	-7.324\\
1097	-7.324\\
1098	-10.986\\
1099	-13.428\\
1100	-12.207\\
1101	-12.207\\
1102	-15.869\\
1103	-18.311\\
1104	-19.531\\
1105	-15.869\\
1106	-21.973\\
1107	-15.869\\
1108	-15.869\\
1109	-17.09\\
1110	-12.207\\
1111	-12.207\\
1112	-15.869\\
1113	-14.648\\
1114	-8.545\\
1115	-12.207\\
1116	-8.545\\
1117	-9.766\\
1118	-9.766\\
1119	-7.324\\
1120	-9.766\\
1121	-12.207\\
1122	-17.09\\
1123	-17.09\\
1124	-17.09\\
1125	-10.986\\
1126	-12.207\\
1127	-23.193\\
1128	-15.869\\
1129	-20.752\\
1130	-21.973\\
1131	-12.207\\
1132	-9.766\\
1133	-12.207\\
1134	-13.428\\
1135	-21.973\\
1136	-25.635\\
1137	-26.855\\
1138	-19.531\\
1139	-20.752\\
1140	-18.311\\
1141	-17.09\\
1142	-18.311\\
1143	-13.428\\
1144	-12.207\\
1145	-9.766\\
1146	-9.766\\
1147	-10.986\\
1148	-15.869\\
1149	-23.193\\
1150	-17.09\\
1151	-13.428\\
1152	-10.986\\
1153	-10.986\\
1154	-7.324\\
1155	-6.104\\
1156	-6.104\\
1157	-12.207\\
1158	-7.324\\
1159	-8.545\\
1160	-8.545\\
1161	-8.545\\
1162	-7.324\\
1163	-4.883\\
1164	-3.662\\
1165	-8.545\\
1166	-15.869\\
1167	-18.311\\
1168	-20.752\\
1169	-15.869\\
1170	-10.986\\
1171	-8.545\\
1172	-4.883\\
1173	-9.766\\
1174	-8.545\\
1175	-15.869\\
1176	-17.09\\
1177	-29.297\\
1178	-32.959\\
1179	-25.635\\
1180	-25.635\\
1181	-18.311\\
1182	-23.193\\
1183	-21.973\\
1184	-24.414\\
1185	-15.869\\
1186	-17.09\\
1187	-17.09\\
1188	-15.869\\
1189	-15.869\\
1190	-13.428\\
1191	-23.193\\
1192	-25.635\\
1193	-19.531\\
1194	-13.428\\
1195	-14.648\\
1196	-23.193\\
1197	-24.414\\
1198	-28.076\\
1199	-30.518\\
1200	-29.297\\
1201	-23.193\\
1202	-21.973\\
1203	-25.635\\
1204	-28.076\\
1205	-18.311\\
1206	-12.207\\
1207	-19.531\\
1208	-14.648\\
1209	-9.766\\
1210	-13.428\\
1211	-14.648\\
1212	-10.986\\
1213	-10.986\\
1214	-10.986\\
1215	-8.545\\
1216	-10.986\\
1217	-18.311\\
1218	-17.09\\
1219	-12.207\\
1220	-12.207\\
1221	-18.311\\
1222	-26.855\\
1223	-17.09\\
1224	-14.648\\
1225	-10.986\\
1226	-13.428\\
1227	-10.986\\
1228	-8.545\\
1229	-8.545\\
1230	-10.986\\
1231	-13.428\\
1232	-14.648\\
1233	-12.207\\
1234	-15.869\\
1235	-20.752\\
1236	-17.09\\
1237	-12.207\\
1238	-15.869\\
1239	-20.752\\
1240	-13.428\\
1241	-7.324\\
1242	-13.428\\
1243	-10.986\\
1244	-12.207\\
1245	-9.766\\
1246	-8.545\\
1247	-13.428\\
1248	-13.428\\
1249	-9.766\\
1250	-12.207\\
1251	-9.766\\
1252	-7.324\\
1253	-12.207\\
1254	-8.545\\
1255	-9.766\\
1256	-7.324\\
1257	-4.883\\
1258	-12.207\\
1259	-15.869\\
1260	-17.09\\
1261	-23.193\\
1262	-15.869\\
1263	-9.766\\
1264	-8.545\\
1265	-8.545\\
1266	-10.986\\
1267	-8.545\\
1268	-4.883\\
1269	-10.986\\
1270	-15.869\\
1271	-13.428\\
1272	-20.752\\
1273	-15.869\\
1274	-14.648\\
1275	-8.545\\
1276	-10.986\\
1277	-4.883\\
1278	-3.662\\
1279	-4.883\\
1280	-9.766\\
1281	-10.986\\
1282	-12.207\\
1283	-13.428\\
1284	-20.752\\
1285	-17.09\\
1286	-14.648\\
1287	-17.09\\
1288	-19.531\\
1289	-20.752\\
1290	-17.09\\
1291	-13.428\\
1292	-7.324\\
1293	-6.104\\
1294	-4.883\\
1295	-7.324\\
1296	-7.324\\
1297	-4.883\\
1298	-8.545\\
1299	-12.207\\
1300	-10.986\\
1301	-12.207\\
1302	-12.207\\
1303	-8.545\\
1304	-4.883\\
1305	-9.766\\
1306	-15.869\\
1307	-13.428\\
1308	-15.869\\
1309	-13.428\\
1310	-9.766\\
1311	-14.648\\
1312	-18.311\\
1313	-18.311\\
1314	-13.428\\
1315	-20.752\\
1316	-15.869\\
1317	-17.09\\
1318	-18.311\\
1319	-19.531\\
1320	-13.428\\
1321	-13.428\\
1322	-18.311\\
1323	-26.855\\
1324	-21.973\\
1325	-13.428\\
1326	-13.428\\
1327	-13.428\\
1328	-15.869\\
1329	-17.09\\
1330	-12.207\\
1331	-14.648\\
1332	-18.311\\
1333	-20.752\\
1334	-13.428\\
1335	-12.207\\
1336	-15.869\\
1337	-23.193\\
1338	-20.752\\
1339	-21.973\\
1340	-14.648\\
1341	-14.648\\
1342	-12.207\\
1343	-9.766\\
1344	-4.883\\
1345	-3.662\\
1346	-3.662\\
1347	-7.324\\
1348	-13.428\\
1349	-14.648\\
1350	-9.766\\
1351	-12.207\\
1352	-13.428\\
1353	-9.766\\
1354	-8.545\\
1355	-8.545\\
1356	-6.104\\
1357	-9.766\\
1358	-14.648\\
1359	-15.869\\
1360	-10.986\\
1361	-8.545\\
1362	-8.545\\
1363	-8.545\\
1364	-7.324\\
1365	-10.986\\
1366	-20.752\\
1367	-18.311\\
1368	-18.311\\
1369	-24.414\\
1370	-23.193\\
1371	-18.311\\
1372	-14.648\\
1373	-14.648\\
1374	-14.648\\
1375	-17.09\\
1376	-19.531\\
1377	-19.531\\
1378	-25.635\\
1379	-26.855\\
1380	-31.738\\
1381	-24.414\\
1382	-25.635\\
1383	-26.855\\
1384	-21.973\\
1385	-24.414\\
1386	-20.752\\
1387	-13.428\\
1388	-9.766\\
1389	-9.766\\
1390	-12.207\\
1391	-9.766\\
1392	-7.324\\
1393	-10.986\\
1394	-8.545\\
1395	-6.104\\
1396	-7.324\\
1397	-9.766\\
1398	-6.104\\
1399	-12.207\\
1400	-14.648\\
1401	-10.986\\
1402	-17.09\\
1403	-24.414\\
1404	-24.414\\
1405	-28.076\\
1406	-23.193\\
1407	-21.973\\
1408	-17.09\\
1409	-9.766\\
1410	-10.986\\
1411	-10.986\\
1412	-7.324\\
1413	-10.986\\
1414	-9.766\\
1415	-2.441\\
1416	-6.104\\
1417	-9.766\\
1418	-12.207\\
1419	-14.648\\
1420	-17.09\\
1421	-14.648\\
1422	-17.09\\
1423	-14.648\\
1424	-13.428\\
1425	-12.207\\
1426	-10.986\\
1427	-14.648\\
1428	-13.428\\
1429	-10.986\\
1430	-13.428\\
1431	-18.311\\
1432	-13.428\\
1433	-10.986\\
1434	-10.986\\
1435	-10.986\\
1436	-8.545\\
1437	-7.324\\
1438	-10.986\\
1439	-13.428\\
1440	-13.428\\
1441	-10.986\\
1442	-13.428\\
1443	-13.428\\
1444	-8.545\\
1445	-7.324\\
1446	-15.869\\
1447	-19.531\\
1448	-18.311\\
1449	-18.311\\
1450	-15.869\\
1451	-13.428\\
1452	-10.986\\
1453	-9.766\\
1454	-13.428\\
1455	-13.428\\
1456	-8.545\\
1457	-12.207\\
1458	-14.648\\
1459	-14.648\\
1460	-17.09\\
1461	-17.09\\
1462	-20.752\\
1463	-28.076\\
1464	-30.518\\
1465	-20.752\\
1466	-13.428\\
1467	-8.545\\
1468	-6.104\\
1469	-8.545\\
1470	-7.324\\
1471	-10.986\\
1472	-12.207\\
1473	-12.207\\
1474	-13.428\\
1475	-15.869\\
1476	-19.531\\
1477	-19.531\\
1478	-28.076\\
1479	-23.193\\
1480	-18.311\\
1481	-14.648\\
1482	-14.648\\
1483	-14.648\\
1484	-17.09\\
1485	-14.648\\
1486	-12.207\\
1487	-13.428\\
1488	-8.545\\
1489	-7.324\\
1490	-17.09\\
1491	-14.648\\
1492	-12.207\\
1493	-23.193\\
1494	-18.311\\
1495	-15.869\\
1496	-21.973\\
1497	-23.193\\
1498	-14.648\\
1499	-8.545\\
1500	-9.766\\
};
\addlegendentry{True output}

\addplot [color=mycolor2, dashed]
  table[row sep=crcr]{%
1001	-16.2544084324132\\
1002	-19.6784314654058\\
1003	-18.4694064379188\\
1004	-16.808444273372\\
1005	-20.6966073824176\\
1006	-20.9228368423976\\
1007	-22.1587303653248\\
1008	-18.9521036415523\\
1009	-11.8528184041622\\
1010	-12.7096794347379\\
1011	-13.8175781037501\\
1012	-15.6679853833977\\
1013	-14.3386067349356\\
1014	-7.44403282626742\\
1015	-4.84179229558087\\
1016	-4.54550097920922\\
1017	-10.2804050270476\\
1018	-18.6655311719662\\
1019	-18.1549098275812\\
1020	-17.4191972855401\\
1021	-13.4252469213605\\
1022	-9.17865984794861\\
1023	-16.2818126798251\\
1024	-13.7571919631815\\
1025	-10.5807017792587\\
1026	-14.2599782556075\\
1027	-15.154128510943\\
1028	-12.4174418048987\\
1029	-17.53235469554\\
1030	-17.7336160690974\\
1031	-13.9935471848428\\
1032	-18.0980518167264\\
1033	-18.7878295060797\\
1034	-15.4979932091656\\
1035	-13.5603169684388\\
1036	-12.1797314005566\\
1037	-13.1051813061236\\
1038	-15.7294621689508\\
1039	-15.6323923658574\\
1040	-14.7765407946033\\
1041	-15.2799981502103\\
1042	-16.1267527815814\\
1043	-20.7294894715381\\
1044	-20.1842888046437\\
1045	-14.5951753403296\\
1046	-13.2927339903717\\
1047	-9.3980852245608\\
1048	-8.85439564137756\\
1049	-11.798318625229\\
1050	-11.0472339196892\\
1051	-10.7573700569601\\
1052	-13.1990443753616\\
1053	-14.7950788980108\\
1054	-13.936346824909\\
1055	-19.082879732539\\
1056	-16.7249996261008\\
1057	-10.2278868353418\\
1058	-8.69434294053836\\
1059	-11.1020033263205\\
1060	-12.987304759357\\
1061	-9.34951353070526\\
1062	-7.73606898176154\\
1063	-10.3743385229674\\
1064	-10.0455354645127\\
1065	-8.48831955364719\\
1066	-9.63152303207739\\
1067	-10.9789221054409\\
1068	-15.0189076211229\\
1069	-15.2806827829759\\
1070	-16.1020245211333\\
1071	-15.150236735767\\
1072	-16.8885419592947\\
1073	-15.2238467414441\\
1074	-13.9143051685176\\
1075	-13.403883692648\\
1076	-13.3908046806023\\
1077	-13.2243945584786\\
1078	-19.4480049966518\\
1079	-28.0915231142917\\
1080	-28.0500060563517\\
1081	-26.3403873290015\\
1082	-17.8012091741569\\
1083	-23.9064089149332\\
1084	-29.4195298563511\\
1085	-29.4307810776151\\
1086	-24.5333862763381\\
1087	-29.104831772407\\
1088	-36.7576295772235\\
1089	-31.0427706540446\\
1090	-25.9573241886477\\
1091	-19.3996066196343\\
1092	-16.3207098869828\\
1093	-14.9975119964015\\
1094	-16.82868549986\\
1095	-15.2466510647744\\
1096	-10.2244615534988\\
1097	-8.87393978912629\\
1098	-10.9634891238692\\
1099	-15.1539564038274\\
1100	-14.9939122571507\\
1101	-14.3165649458959\\
1102	-16.4749682652069\\
1103	-17.2238466260922\\
1104	-22.8153223502335\\
1105	-20.4911041569864\\
1106	-21.1853897594344\\
1107	-18.2177827556279\\
1108	-15.5837736244175\\
1109	-18.3555436470466\\
1110	-15.5015971466965\\
1111	-13.3585428517132\\
1112	-15.8307333071819\\
1113	-16.2282068026087\\
1114	-13.6860003512247\\
1115	-12.9044223960587\\
1116	-11.3625920426877\\
1117	-10.9782608629634\\
1118	-12.7442756626388\\
1119	-9.68135471814524\\
1120	-10.305789783288\\
1121	-12.9001495353767\\
1122	-17.4203389924972\\
1123	-17.6192604625073\\
1124	-17.4693869219177\\
1125	-12.9322076282119\\
1126	-15.7640748279044\\
1127	-23.3623018086218\\
1128	-18.9655203825629\\
1129	-20.0128629086368\\
1130	-20.0479847138475\\
1131	-14.4372798903749\\
1132	-10.8742898576152\\
1133	-13.0820665019645\\
1134	-15.473437897281\\
1135	-22.1782547076418\\
1136	-26.5874891593911\\
1137	-24.8767165615015\\
1138	-20.5082794697614\\
1139	-19.9681113803986\\
1140	-19.9426151363302\\
1141	-19.1193818741666\\
1142	-19.0101868104049\\
1143	-14.9405152990754\\
1144	-12.8990628847109\\
1145	-12.8555382082797\\
1146	-12.3782126615457\\
1147	-13.0024343845593\\
1148	-17.0020755277756\\
1149	-23.167721937726\\
1150	-20.3951667518943\\
1151	-14.3352183417148\\
1152	-12.3101052450357\\
1153	-12.7580852849988\\
1154	-9.21958081888343\\
1155	-7.03579615222564\\
1156	-7.6371671344161\\
1157	-12.0324260455699\\
1158	-11.0239458390177\\
1159	-10.3258758133424\\
1160	-10.9525550166718\\
1161	-10.4952761063185\\
1162	-8.89794490956174\\
1163	-6.6632141263712\\
1164	-5.6983153353273\\
1165	-7.31948257145027\\
1166	-14.1186087211607\\
1167	-19.9865808717484\\
1168	-20.8177938049727\\
1169	-14.3108181747597\\
1170	-10.4342152254251\\
1171	-8.16990825687893\\
1172	-6.24645664147086\\
1173	-11.1073003375735\\
1174	-12.2887173362077\\
1175	-14.9267611035226\\
1176	-18.3690661996537\\
1177	-27.851208592189\\
1178	-32.4577482565085\\
1179	-26.715107596588\\
1180	-25.5303583039839\\
1181	-20.2593205408655\\
1182	-20.1582673670785\\
1183	-22.3870367173567\\
1184	-23.5832162259803\\
1185	-19.9504331819161\\
1186	-18.4518060196064\\
1187	-18.8048014634398\\
1188	-17.5500662664769\\
1189	-19.9452655751639\\
1190	-16.7209575859891\\
1191	-21.6427658538413\\
1192	-26.2199839461113\\
1193	-20.2355473933409\\
1194	-13.7332665969689\\
1195	-14.3873856981996\\
1196	-21.7049257574783\\
1197	-25.740191565238\\
1198	-28.9325589713508\\
1199	-30.0812668087843\\
1200	-29.7380183221512\\
1201	-24.4602239498588\\
1202	-22.9667126141095\\
1203	-25.5048049626084\\
1204	-28.1970904569503\\
1205	-21.050222443931\\
1206	-14.0730410737496\\
1207	-18.0998939022126\\
1208	-17.4538044696922\\
1209	-12.512827650467\\
1210	-14.6040993517467\\
1211	-16.9679844229643\\
1212	-14.5589696264217\\
1213	-11.788998462881\\
1214	-13.6317279292864\\
1215	-13.3119111302729\\
1216	-15.3280439648325\\
1217	-20.0029332067836\\
1218	-18.1717852907307\\
1219	-14.127607127733\\
1220	-13.5270208026619\\
1221	-18.8821127395247\\
1222	-28.2616004918596\\
1223	-23.5926493917803\\
1224	-15.299735945443\\
1225	-12.6923634812545\\
1226	-13.611928310293\\
1227	-14.5223060600453\\
1228	-11.7058663704877\\
1229	-12.15813558423\\
1230	-13.4090539366346\\
1231	-16.1409010758558\\
1232	-17.2230805998956\\
1233	-12.6125598720241\\
1234	-16.2247741527587\\
1235	-19.8499830994172\\
1236	-18.9719662328848\\
1237	-16.1614079558337\\
1238	-16.6114728584001\\
1239	-21.3276078098923\\
1240	-16.067122200011\\
1241	-8.27962822813456\\
1242	-9.36526521977697\\
1243	-11.7600191415394\\
1244	-15.3978005869961\\
1245	-14.3125331812177\\
1246	-11.4078191687475\\
1247	-14.5271207894377\\
1248	-14.939523591165\\
1249	-11.9385304087829\\
1250	-13.2742764067251\\
1251	-12.8286527137758\\
1252	-12.1372705637239\\
1253	-12.8001513611471\\
1254	-9.75447689528525\\
1255	-9.78241061901902\\
1256	-8.65806327789189\\
1257	-8.92744811724361\\
1258	-13.6387172403008\\
1259	-15.5372739991444\\
1260	-18.5671567199769\\
1261	-23.5022281343083\\
1262	-17.3024119744391\\
1263	-10.9770957785649\\
1264	-8.85152871675196\\
1265	-9.10440676125351\\
1266	-12.3757726449821\\
1267	-9.62054100429461\\
1268	-9.36214608997011\\
1269	-11.4389495280509\\
1270	-16.7925015105597\\
1271	-16.8553266476345\\
1272	-19.2791333455932\\
1273	-15.1182679519265\\
1274	-13.1445213827191\\
1275	-8.76766490206218\\
1276	-7.95338704948604\\
1277	-6.24834775975781\\
1278	-6.30499782546326\\
1279	-7.65712929622795\\
1280	-12.5552957639892\\
1281	-12.7645759240793\\
1282	-13.670698569621\\
1283	-13.927479082271\\
1284	-20.2786394692024\\
1285	-20.0040635874071\\
1286	-14.4990829573281\\
1287	-16.2773254554442\\
1288	-17.2474851042186\\
1289	-21.5793879232505\\
1290	-18.786828441596\\
1291	-13.5856863944853\\
1292	-8.38770436620366\\
1293	-6.13782613442004\\
1294	-6.72676772790699\\
1295	-8.00938793121579\\
1296	-8.37702931587724\\
1297	-8.15570575546226\\
1298	-9.27470105752691\\
1299	-12.8270871833449\\
1300	-12.8268374921413\\
1301	-11.987140001754\\
1302	-13.2601625091307\\
1303	-9.79883361904022\\
1304	-6.18522325105869\\
1305	-9.67557035632813\\
1306	-14.3217399822542\\
1307	-14.6439571289018\\
1308	-14.3703140966494\\
1309	-13.2573471672347\\
1310	-10.8234098835799\\
1311	-13.0787973312762\\
1312	-17.6015914495073\\
1313	-18.1384248338261\\
1314	-13.6085957585025\\
1315	-19.985191546109\\
1316	-17.778857730442\\
1317	-16.0426999646584\\
1318	-18.4953746223584\\
1319	-17.9579517739445\\
1320	-15.2502432342235\\
1321	-13.5019488667962\\
1322	-18.5731633823892\\
1323	-26.7349226524533\\
1324	-21.4407267360076\\
1325	-13.5665942108172\\
1326	-12.69513667355\\
1327	-14.6814641167286\\
1328	-17.0682269776479\\
1329	-19.20580786714\\
1330	-15.5597189946954\\
1331	-16.0452394223114\\
1332	-19.4233336411496\\
1333	-20.7796392780605\\
1334	-15.9370042261635\\
1335	-13.3222733353638\\
1336	-16.012432815511\\
1337	-24.114687104208\\
1338	-21.9829975672411\\
1339	-21.1317463041282\\
1340	-14.6801544999161\\
1341	-13.0008380235077\\
1342	-13.9711091317937\\
1343	-9.6745934707723\\
1344	-7.41613521814182\\
1345	-5.92998498597442\\
1346	-5.4557390416145\\
1347	-9.28709108308916\\
1348	-13.5514940190652\\
1349	-15.252452498203\\
1350	-11.9154004664842\\
1351	-11.7334289302019\\
1352	-12.9446728208797\\
1353	-10.0643371194037\\
1354	-9.53248838095564\\
1355	-10.0153912816262\\
1356	-8.71139397513706\\
1357	-9.39292478797364\\
1358	-13.16271953603\\
1359	-16.5859715639316\\
1360	-11.8953696188008\\
1361	-9.14208527922344\\
1362	-8.31739782923184\\
1363	-9.76705470184315\\
1364	-7.95344848619723\\
1365	-13.6080707885404\\
1366	-21.55946293057\\
1367	-17.546362828429\\
1368	-20.4378641225136\\
1369	-24.3410254518377\\
1370	-23.4698488226127\\
1371	-19.4498556615119\\
1372	-15.114133909924\\
1373	-15.2264464976553\\
1374	-16.1807900591869\\
1375	-18.0949162361165\\
1376	-19.8948422890144\\
1377	-20.2227686266751\\
1378	-23.3211310571006\\
1379	-26.5667932597847\\
1380	-30.4314040539168\\
1381	-23.7752909828169\\
1382	-23.3013796323162\\
1383	-27.3417227277126\\
1384	-23.5782563957375\\
1385	-24.3338802314773\\
1386	-23.0671927312753\\
1387	-14.1843093249258\\
1388	-10.3089042626479\\
1389	-11.1353974452765\\
1390	-15.7143123302583\\
1391	-14.2590332980253\\
1392	-10.5565989134302\\
1393	-12.26196666661\\
1394	-10.5434706799983\\
1395	-8.42721063823467\\
1396	-9.50816622630345\\
1397	-10.5180892689169\\
1398	-8.90307418225506\\
1399	-12.2867508780186\\
1400	-16.3022269388728\\
1401	-13.3158148699\\
1402	-17.3049590601717\\
1403	-25.0986964263964\\
1404	-23.5151325279167\\
1405	-26.6859407619676\\
1406	-21.3246604032314\\
1407	-20.7889834364449\\
1408	-17.1261106797617\\
1409	-11.5254621660102\\
1410	-13.82542584849\\
1411	-12.3154481825973\\
1412	-10.0009240553235\\
1413	-12.2035097353274\\
1414	-8.78833860648254\\
1415	-6.14870805508551\\
1416	-6.77109431158825\\
1417	-11.0259215446822\\
1418	-15.4908797439366\\
1419	-17.2158961934068\\
1420	-16.7824358279673\\
1421	-15.7354424598073\\
1422	-16.9710631838357\\
1423	-15.4252942177324\\
1424	-13.2518011421633\\
1425	-13.9449177862891\\
1426	-12.410575002843\\
1427	-16.3077865681473\\
1428	-13.163282792679\\
1429	-10.7082180478666\\
1430	-14.0984480994347\\
1431	-19.2113713606015\\
1432	-16.3237127369873\\
1433	-12.3392530119952\\
1434	-11.1156506213328\\
1435	-12.483654076653\\
1436	-10.0527220323849\\
1437	-9.03548443780648\\
1438	-11.4990832644455\\
1439	-13.3356157240271\\
1440	-14.852025086969\\
1441	-12.6346760541515\\
1442	-13.9801377399325\\
1443	-14.072281322048\\
1444	-9.91737290857994\\
1445	-9.37117379251508\\
1446	-14.9938357753822\\
1447	-21.0315101132253\\
1448	-19.9530055362183\\
1449	-16.9506895911218\\
1450	-16.4827996557728\\
1451	-14.2274284101201\\
1452	-13.6311051278151\\
1453	-12.0477448165558\\
1454	-14.7131123797151\\
1455	-14.4415217343763\\
1456	-10.2630105315706\\
1457	-12.8512040564805\\
1458	-14.5652552764133\\
1459	-16.9707049073063\\
1460	-17.3673739817332\\
1461	-17.0742696912879\\
1462	-21.7843945956104\\
1463	-26.3722958486861\\
1464	-29.3747449768872\\
1465	-21.4247272691214\\
1466	-13.4368340879029\\
1467	-9.79237995649919\\
1468	-7.23434880201334\\
1469	-9.58493109759048\\
1470	-10.6835599600807\\
1471	-14.8813258575221\\
1472	-14.5219802048633\\
1473	-12.1270814828398\\
1474	-14.5018409022483\\
1475	-16.3746793658774\\
1476	-18.0653930473907\\
1477	-19.1205443432551\\
1478	-26.8696429245056\\
1479	-24.8688689769717\\
1480	-19.9412052684943\\
1481	-15.0994740289957\\
1482	-14.4370419945822\\
1483	-16.2572305697912\\
1484	-18.663241615435\\
1485	-15.4378494013665\\
1486	-14.5315200962571\\
1487	-14.6253751242782\\
1488	-9.92882645584426\\
1489	-12.9614069952347\\
1490	-18.3166463572213\\
1491	-14.7782304392814\\
1492	-14.330210264084\\
1493	-22.6429107333282\\
1494	-19.4061482491018\\
1495	-17.3110631700649\\
1496	-22.3020660687066\\
1497	-22.0855669350404\\
1498	-16.2601519400603\\
1499	-10.4112199568007\\
1500	-11.3790128514524\\
};
\addlegendentry{Generated output}

\end{axis}
\end{tikzpicture}%\label{fig:c3all}}\\
%	\subfloat[\dataset8]{% This file was created by matlab2tikz.
%
\definecolor{mycolor1}{rgb}{0.00000,0.44700,0.74100}%
\definecolor{mycolor2}{rgb}{0.85000,0.32500,0.09800}%
%
\begin{tikzpicture}

\begin{axis}[%
width=11.411cm,
height=4cm,
at={(0cm,0cm)},
scale only axis,
xmin=1000,
xmax=1500,
ymin=-300,
ymax=0,
axis background/.style={fill=white},
legend style={legend cell align=left, align=left, draw=white!15!black}
]
\addplot [color=mycolor1]
  table[row sep=crcr]{%
1001	-96.436\\
1002	-122.07\\
1003	-100.098\\
1004	-93.994\\
1005	-130.615\\
1006	-125.732\\
1007	-153.809\\
1008	-115.967\\
1009	-61.035\\
1010	-74.463\\
1011	-73.242\\
1012	-86.67\\
1013	-73.242\\
1014	-34.18\\
1015	-20.752\\
1016	-23.193\\
1017	-58.594\\
1018	-100.098\\
1019	-109.863\\
1020	-114.746\\
1021	-83.008\\
1022	-52.49\\
1023	-104.98\\
1024	-81.787\\
1025	-59.814\\
1026	-97.656\\
1027	-83.008\\
1028	-72.021\\
1029	-118.408\\
1030	-107.422\\
1031	-83.008\\
1032	-119.629\\
1033	-123.291\\
1034	-95.215\\
1035	-80.566\\
1036	-62.256\\
1037	-75.684\\
1038	-95.215\\
1039	-87.891\\
1040	-91.553\\
1041	-96.436\\
1042	-102.539\\
1043	-141.602\\
1044	-128.174\\
1045	-85.449\\
1046	-79.346\\
1047	-47.607\\
1048	-48.828\\
1049	-68.359\\
1050	-52.49\\
1051	-57.373\\
1052	-75.684\\
1053	-89.111\\
1054	-81.787\\
1055	-128.174\\
1056	-98.877\\
1057	-57.373\\
1058	-45.166\\
1059	-62.256\\
1060	-72.021\\
1061	-40.283\\
1062	-42.725\\
1063	-56.152\\
1064	-46.387\\
1065	-40.283\\
1066	-52.49\\
1067	-62.256\\
1068	-92.773\\
1069	-90.332\\
1070	-107.422\\
1071	-98.877\\
1072	-115.967\\
1073	-91.553\\
1074	-91.553\\
1075	-79.346\\
1076	-75.684\\
1077	-76.904\\
1078	-133.057\\
1079	-189.209\\
1080	-194.092\\
1081	-189.209\\
1082	-119.629\\
1083	-170.898\\
1084	-213.623\\
1085	-219.727\\
1086	-169.678\\
1087	-219.727\\
1088	-279.541\\
1089	-197.754\\
1090	-161.133\\
1091	-109.863\\
1092	-85.449\\
1093	-73.242\\
1094	-91.553\\
1095	-62.256\\
1096	-46.387\\
1097	-42.725\\
1098	-54.932\\
1099	-79.346\\
1100	-74.463\\
1101	-75.684\\
1102	-100.098\\
1103	-103.76\\
1104	-141.602\\
1105	-114.746\\
1106	-142.822\\
1107	-104.98\\
1108	-90.332\\
1109	-103.76\\
1110	-76.904\\
1111	-69.58\\
1112	-84.229\\
1113	-86.67\\
1114	-59.814\\
1115	-64.697\\
1116	-48.828\\
1117	-53.711\\
1118	-57.373\\
1119	-41.504\\
1120	-52.49\\
1121	-65.918\\
1122	-101.318\\
1123	-104.98\\
1124	-114.746\\
1125	-68.359\\
1126	-93.994\\
1127	-150.146\\
1128	-114.746\\
1129	-125.732\\
1130	-128.174\\
1131	-80.566\\
1132	-53.711\\
1133	-75.684\\
1134	-85.449\\
1135	-137.939\\
1136	-172.119\\
1137	-170.898\\
1138	-123.291\\
1139	-130.615\\
1140	-115.967\\
1141	-113.525\\
1142	-111.084\\
1143	-76.904\\
1144	-68.359\\
1145	-61.035\\
1146	-57.373\\
1147	-62.256\\
1148	-91.553\\
1149	-140.381\\
1150	-111.084\\
1151	-79.346\\
1152	-64.697\\
1153	-62.256\\
1154	-40.283\\
1155	-29.297\\
1156	-36.621\\
1157	-63.477\\
1158	-51.27\\
1159	-52.49\\
1160	-58.594\\
1161	-54.932\\
1162	-37.842\\
1163	-28.076\\
1164	-23.193\\
1165	-39.063\\
1166	-83.008\\
1167	-117.188\\
1168	-130.615\\
1169	-86.67\\
1170	-58.594\\
1171	-45.166\\
1172	-32.959\\
1173	-67.139\\
1174	-65.918\\
1175	-91.553\\
1176	-117.188\\
1177	-186.768\\
1178	-216.064\\
1179	-177.002\\
1180	-168.457\\
1181	-109.863\\
1182	-122.07\\
1183	-131.836\\
1184	-146.484\\
1185	-108.643\\
1186	-100.098\\
1187	-98.877\\
1188	-89.111\\
1189	-106.201\\
1190	-78.125\\
1191	-130.615\\
1192	-159.912\\
1193	-113.525\\
1194	-70.801\\
1195	-73.242\\
1196	-123.291\\
1197	-153.809\\
1198	-189.209\\
1199	-202.637\\
1200	-202.637\\
1201	-153.809\\
1202	-137.939\\
1203	-158.691\\
1204	-187.988\\
1205	-109.863\\
1206	-72.021\\
1207	-101.318\\
1208	-86.67\\
1209	-53.711\\
1210	-74.463\\
1211	-84.229\\
1212	-67.139\\
1213	-51.27\\
1214	-70.801\\
1215	-58.594\\
1216	-79.346\\
1217	-107.422\\
1218	-100.098\\
1219	-69.58\\
1220	-70.801\\
1221	-107.422\\
1222	-177.002\\
1223	-128.174\\
1224	-79.346\\
1225	-61.035\\
1226	-70.801\\
1227	-64.697\\
1228	-48.828\\
1229	-53.711\\
1230	-65.918\\
1231	-83.008\\
1232	-91.553\\
1233	-63.477\\
1234	-104.98\\
1235	-124.512\\
1236	-118.408\\
1237	-91.553\\
1238	-100.098\\
1239	-135.498\\
1240	-87.891\\
1241	-42.725\\
1242	-57.373\\
1243	-62.256\\
1244	-81.787\\
1245	-72.021\\
1246	-52.49\\
1247	-79.346\\
1248	-80.566\\
1249	-57.373\\
1250	-70.801\\
1251	-63.477\\
1252	-56.152\\
1253	-67.139\\
1254	-41.504\\
1255	-48.828\\
1256	-36.621\\
1257	-40.283\\
1258	-75.684\\
1259	-84.229\\
1260	-113.525\\
1261	-146.484\\
1262	-98.877\\
1263	-58.594\\
1264	-46.387\\
1265	-43.945\\
1266	-63.477\\
1267	-42.725\\
1268	-47.607\\
1269	-61.035\\
1270	-102.539\\
1271	-93.994\\
1272	-134.277\\
1273	-89.111\\
1274	-81.787\\
1275	-46.387\\
1276	-45.166\\
1277	-28.076\\
1278	-35.4\\
1279	-37.842\\
1280	-67.139\\
1281	-70.801\\
1282	-79.346\\
1283	-85.449\\
1284	-125.732\\
1285	-112.305\\
1286	-84.229\\
1287	-102.539\\
1288	-109.863\\
1289	-144.043\\
1290	-115.967\\
1291	-75.684\\
1292	-40.283\\
1293	-29.297\\
1294	-29.297\\
1295	-36.621\\
1296	-39.063\\
1297	-37.842\\
1298	-50.049\\
1299	-74.463\\
1300	-68.359\\
1301	-70.801\\
1302	-83.008\\
1303	-51.27\\
1304	-30.518\\
1305	-58.594\\
1306	-90.332\\
1307	-87.891\\
1308	-97.656\\
1309	-83.008\\
1310	-61.035\\
1311	-85.449\\
1312	-118.408\\
1313	-118.408\\
1314	-84.229\\
1315	-136.719\\
1316	-100.098\\
1317	-101.318\\
1318	-117.188\\
1319	-115.967\\
1320	-86.67\\
1321	-76.904\\
1322	-128.174\\
1323	-178.223\\
1324	-139.16\\
1325	-79.346\\
1326	-76.904\\
1327	-79.346\\
1328	-98.877\\
1329	-114.746\\
1330	-85.449\\
1331	-90.332\\
1332	-120.85\\
1333	-131.836\\
1334	-87.891\\
1335	-68.359\\
1336	-91.553\\
1337	-156.25\\
1338	-137.939\\
1339	-140.381\\
1340	-84.229\\
1341	-76.904\\
1342	-72.021\\
1343	-47.607\\
1344	-30.518\\
1345	-26.855\\
1346	-23.193\\
1347	-54.932\\
1348	-70.801\\
1349	-87.891\\
1350	-65.918\\
1351	-73.242\\
1352	-83.008\\
1353	-53.711\\
1354	-56.152\\
1355	-53.711\\
1356	-40.283\\
1357	-51.27\\
1358	-84.229\\
1359	-106.201\\
1360	-63.477\\
1361	-48.828\\
1362	-40.283\\
1363	-50.049\\
1364	-35.4\\
1365	-76.904\\
1366	-130.615\\
1367	-104.98\\
1368	-139.16\\
1369	-162.354\\
1370	-164.795\\
1371	-124.512\\
1372	-90.332\\
1373	-89.111\\
1374	-89.111\\
1375	-106.201\\
1376	-125.732\\
1377	-125.732\\
1378	-168.457\\
1379	-183.105\\
1380	-219.727\\
1381	-147.705\\
1382	-166.016\\
1383	-184.326\\
1384	-140.381\\
1385	-162.354\\
1386	-139.16\\
1387	-80.566\\
1388	-52.49\\
1389	-56.152\\
1390	-81.787\\
1391	-65.918\\
1392	-43.945\\
1393	-62.256\\
1394	-47.607\\
1395	-36.621\\
1396	-46.387\\
1397	-52.49\\
1398	-36.621\\
1399	-70.801\\
1400	-92.773\\
1401	-68.359\\
1402	-114.746\\
1403	-164.795\\
1404	-150.146\\
1405	-196.533\\
1406	-131.836\\
1407	-144.043\\
1408	-96.436\\
1409	-63.477\\
1410	-76.904\\
1411	-59.814\\
1412	-45.166\\
1413	-64.697\\
1414	-36.621\\
1415	-28.076\\
1416	-34.18\\
1417	-70.801\\
1418	-86.67\\
1419	-107.422\\
1420	-103.76\\
1421	-100.098\\
1422	-114.746\\
1423	-93.994\\
1424	-76.904\\
1425	-76.904\\
1426	-62.256\\
1427	-97.656\\
1428	-68.359\\
1429	-56.152\\
1430	-81.787\\
1431	-113.525\\
1432	-86.67\\
1433	-65.918\\
1434	-56.152\\
1435	-65.918\\
1436	-45.166\\
1437	-45.166\\
1438	-59.814\\
1439	-75.684\\
1440	-84.229\\
1441	-65.918\\
1442	-86.67\\
1443	-79.346\\
1444	-47.607\\
1445	-50.049\\
1446	-95.215\\
1447	-131.836\\
1448	-131.836\\
1449	-107.422\\
1450	-103.76\\
1451	-79.346\\
1452	-76.904\\
1453	-61.035\\
1454	-89.111\\
1455	-75.684\\
1456	-50.049\\
1457	-74.463\\
1458	-85.449\\
1459	-101.318\\
1460	-109.863\\
1461	-109.863\\
1462	-148.926\\
1463	-181.885\\
1464	-205.078\\
1465	-129.395\\
1466	-73.242\\
1467	-47.607\\
1468	-34.18\\
1469	-54.932\\
1470	-46.387\\
1471	-80.566\\
1472	-72.021\\
1473	-64.697\\
1474	-85.449\\
1475	-98.877\\
1476	-117.188\\
1477	-123.291\\
1478	-175.781\\
1479	-150.146\\
1480	-117.188\\
1481	-80.566\\
1482	-80.566\\
1483	-83.008\\
1484	-106.201\\
1485	-75.684\\
1486	-79.346\\
1487	-74.463\\
1488	-42.725\\
1489	-74.463\\
1490	-107.422\\
1491	-73.242\\
1492	-80.566\\
1493	-147.705\\
1494	-109.863\\
1495	-106.201\\
1496	-147.705\\
1497	-140.381\\
1498	-87.891\\
1499	-56.152\\
1500	-58.594\\
};
\addlegendentry{True output}

\addplot [color=mycolor2, dashed]
  table[row sep=crcr]{%
1001	-85.8753194309476\\
1002	-108.781639883586\\
1003	-93.9971352236054\\
1004	-89.4769620518349\\
1005	-119.591861913865\\
1006	-105.37012339383\\
1007	-132.748002827412\\
1008	-104.931418829282\\
1009	-50.7601956046762\\
1010	-64.7978571115109\\
1011	-63.091270596465\\
1012	-84.4126674072102\\
1013	-69.2119098849374\\
1014	-26.7699837195623\\
1015	-18.3306172958516\\
1016	-15.8808735346695\\
1017	-53.0748025112783\\
1018	-97.3228658698575\\
1019	-102.863633599276\\
1020	-98.0123581477109\\
1021	-69.5588405018012\\
1022	-38.1337479864725\\
1023	-92.1824625674546\\
1024	-61.232473224748\\
1025	-51.6236894989416\\
1026	-87.1612801400761\\
1027	-71.4204636179158\\
1028	-63.5636364069265\\
1029	-100.65660818097\\
1030	-87.9957273661725\\
1031	-70.965353167227\\
1032	-98.0230495746468\\
1033	-94.939543324921\\
1034	-80.7517017428449\\
1035	-68.4945391729448\\
1036	-54.3482835410582\\
1037	-66.2222647435294\\
1038	-77.4626561977041\\
1039	-78.0909293294652\\
1040	-77.8273732099553\\
1041	-79.7046134684116\\
1042	-80.8790822483895\\
1043	-114.384588506976\\
1044	-105.334768346877\\
1045	-75.0886062545121\\
1046	-67.6856116421191\\
1047	-36.4188824210649\\
1048	-39.5083382611128\\
1049	-59.4441899399693\\
1050	-46.7752821420833\\
1051	-51.5214545606774\\
1052	-68.6411163418096\\
1053	-74.7718287208595\\
1054	-71.5512199073742\\
1055	-109.226794052489\\
1056	-79.7402527181015\\
1057	-48.1114904478686\\
1058	-39.4820025183618\\
1059	-49.3076279251573\\
1060	-63.3149115379242\\
1061	-34.8493690610817\\
1062	-38.5337218745843\\
1063	-49.2206075428904\\
1064	-40.831678648907\\
1065	-38.4985419273349\\
1066	-47.8752421004022\\
1067	-48.9106540998103\\
1068	-86.5100920442133\\
1069	-73.0380923142298\\
1070	-91.127476297453\\
1071	-80.5810406288184\\
1072	-87.0679984204401\\
1073	-74.7059643790269\\
1074	-71.5420879807798\\
1075	-66.2449137376916\\
1076	-66.3628972452542\\
1077	-64.9165907405524\\
1078	-105.871660164578\\
1079	-156.124272162081\\
1080	-161.950353740005\\
1081	-152.348444137692\\
1082	-97.4789747582967\\
1083	-138.238557564284\\
1084	-167.527532494399\\
1085	-175.705589849747\\
1086	-145.926184900032\\
1087	-173.465826460385\\
1088	-230.097457873237\\
1089	-177.48649776774\\
1090	-151.995196266803\\
1091	-100.854903047692\\
1092	-78.415414035954\\
1093	-71.2109419819919\\
1094	-89.2216589227962\\
1095	-64.7728724749512\\
1096	-41.5759078744271\\
1097	-38.1973742799172\\
1098	-51.9328665712435\\
1099	-78.1821067749213\\
1100	-72.7745024199946\\
1101	-73.9523879066717\\
1102	-90.0837642160231\\
1103	-94.0695678110735\\
1104	-126.593835807569\\
1105	-111.533531497967\\
1106	-120.376468248942\\
1107	-94.8671539347381\\
1108	-79.6200677720868\\
1109	-95.1111536353321\\
1110	-72.3356015513057\\
1111	-67.8745998996506\\
1112	-83.0981241119346\\
1113	-81.829502828045\\
1114	-64.2414308287966\\
1115	-66.084776005444\\
1116	-48.3031136622154\\
1117	-50.7104719607017\\
1118	-58.4493853410394\\
1119	-44.3349950407641\\
1120	-53.177836206463\\
1121	-63.3657708982104\\
1122	-98.3247066776971\\
1123	-91.423270661829\\
1124	-104.602744900986\\
1125	-57.7551947094704\\
1126	-84.0869072410205\\
1127	-127.746098160631\\
1128	-103.125254792831\\
1129	-114.127737094335\\
1130	-108.02354264251\\
1131	-68.5553535397495\\
1132	-44.7647335257141\\
1133	-67.8567974171455\\
1134	-75.5623348330448\\
1135	-128.193725504506\\
1136	-153.480712274598\\
1137	-147.863990039711\\
1138	-106.517638807322\\
1139	-116.335061328273\\
1140	-102.681497220486\\
1141	-102.061976430711\\
1142	-105.155399409294\\
1143	-70.2745849162117\\
1144	-66.5955112641441\\
1145	-58.9899653439101\\
1146	-55.1198393486046\\
1147	-65.1693196128136\\
1148	-90.976775812978\\
1149	-127.914389033552\\
1150	-111.906014153481\\
1151	-72.9943182006562\\
1152	-63.8772451718367\\
1153	-58.6779885501048\\
1154	-35.863700018439\\
1155	-26.6155707721642\\
1156	-33.229264268989\\
1157	-56.3498730338843\\
1158	-47.330356796638\\
1159	-53.8163057196835\\
1160	-53.1877171264908\\
1161	-51.2093513441862\\
1162	-36.81308124997\\
1163	-29.1910499350058\\
1164	-21.5781470303059\\
1165	-32.4733686066015\\
1166	-75.1278842966346\\
1167	-110.076965854749\\
1168	-116.119716498216\\
1169	-78.3146589429499\\
1170	-48.7260495340249\\
1171	-35.3459158980858\\
1172	-24.1731448233555\\
1173	-57.6062145819661\\
1174	-55.4796197837517\\
1175	-80.5905891156146\\
1176	-101.086816817413\\
1177	-167.280821096556\\
1178	-194.111189093913\\
1179	-160.480615184035\\
1180	-142.39605511998\\
1181	-94.7774414374659\\
1182	-106.060588158228\\
1183	-120.715253313523\\
1184	-128.484513278411\\
1185	-108.391580578881\\
1186	-90.9299478125879\\
1187	-99.4737946517123\\
1188	-90.7316911365598\\
1189	-101.890125044506\\
1190	-79.7285919434336\\
1191	-123.157676476296\\
1192	-150.711461055865\\
1193	-107.698178240291\\
1194	-72.6265482365427\\
1195	-67.8927605706514\\
1196	-113.050186088277\\
1197	-142.890964063668\\
1198	-163.723426537908\\
1199	-172.263508749547\\
1200	-173.839927489658\\
1201	-133.208033200068\\
1202	-126.158214551762\\
1203	-140.596760888325\\
1204	-163.350405058493\\
1205	-101.403196301878\\
1206	-65.2391281274397\\
1207	-100.683710252807\\
1208	-84.1816266562866\\
1209	-55.9670068833543\\
1210	-75.4742404761536\\
1211	-83.2558060034768\\
1212	-69.6608562487324\\
1213	-61.6786873182431\\
1214	-71.4458024610429\\
1215	-61.9725762181995\\
1216	-74.7901946550626\\
1217	-110.024834667972\\
1218	-95.8657607998313\\
1219	-72.9683453089676\\
1220	-69.0695310178456\\
1221	-98.9449660908285\\
1222	-162.629000519607\\
1223	-131.063730125957\\
1224	-76.2723254447896\\
1225	-61.1107152781439\\
1226	-65.7491615767186\\
1227	-67.470413667549\\
1228	-49.1841801091534\\
1229	-57.8854200706458\\
1230	-64.8104871154354\\
1231	-80.259021306057\\
1232	-91.5654948388575\\
1233	-63.9233617485858\\
1234	-97.3873289477313\\
1235	-115.40906353394\\
1236	-104.740933046337\\
1237	-85.7313463304801\\
1238	-89.7695107349774\\
1239	-115.924956583777\\
1240	-88.8136502479587\\
1241	-33.2360691400586\\
1242	-46.5748933730594\\
1243	-54.8035880488062\\
1244	-80.0850008193095\\
1245	-73.1982642876902\\
1246	-56.4007942180387\\
1247	-80.0462257120047\\
1248	-71.2052238414766\\
1249	-54.4108195799318\\
1250	-70.8091458944999\\
1251	-59.4726036565844\\
1252	-54.67852336657\\
1253	-66.854305013281\\
1254	-39.9644413841457\\
1255	-48.7842771317063\\
1256	-34.7754811374475\\
1257	-39.2533453360145\\
1258	-73.0578620754696\\
1259	-74.7943472812265\\
1260	-100.500265043451\\
1261	-130.694503811318\\
1262	-87.8095769292832\\
1263	-51.733422545951\\
1264	-37.2436113487789\\
1265	-36.4281563222941\\
1266	-61.999449899189\\
1267	-40.6447394460592\\
1268	-46.4608540029103\\
1269	-55.6027456365547\\
1270	-94.0902193805918\\
1271	-86.1143710581564\\
1272	-119.316934676935\\
1273	-81.2768284011889\\
1274	-66.8837202421193\\
1275	-34.9824538780651\\
1276	-34.4918317297114\\
1277	-23.0450256337749\\
1278	-29.7136205072769\\
1279	-34.8106857336405\\
1280	-59.4230247912385\\
1281	-64.4154725377828\\
1282	-69.0595691553988\\
1283	-70.8605323852624\\
1284	-105.817180741107\\
1285	-104.228008085444\\
1286	-73.7340174897765\\
1287	-84.8247007511963\\
1288	-91.8180332614308\\
1289	-117.469017863465\\
1290	-96.850520316557\\
1291	-65.3880606847457\\
1292	-27.4950299137061\\
1293	-19.5860339489155\\
1294	-22.9010040231359\\
1295	-31.5123507144682\\
1296	-36.3969397055861\\
1297	-36.5725145374553\\
1298	-42.3788018341545\\
1299	-68.7830363336627\\
1300	-62.3210048096361\\
1301	-60.3268333419125\\
1302	-67.4747241276074\\
1303	-41.8624133922293\\
1304	-26.1537403028127\\
1305	-47.4366150059315\\
1306	-73.4419527346313\\
1307	-73.4568729321435\\
1308	-84.7303749732699\\
1309	-66.3030977609285\\
1310	-49.7074713107729\\
1311	-67.6604312593494\\
1312	-92.9696912050613\\
1313	-96.5341896471685\\
1314	-71.087228260832\\
1315	-111.011733008237\\
1316	-85.6634069077201\\
1317	-80.2076230405853\\
1318	-100.800796401373\\
1319	-94.0503923781959\\
1320	-78.2902773634089\\
1321	-70.7628424049921\\
1322	-108.254530389483\\
1323	-155.321668725876\\
1324	-121.965678734549\\
1325	-71.3428456902913\\
1326	-63.3981376807902\\
1327	-72.7779484724131\\
1328	-86.1912519360701\\
1329	-100.44638213561\\
1330	-79.1820027513763\\
1331	-86.3852366128105\\
1332	-105.627695369782\\
1333	-108.872639896226\\
1334	-78.568164569368\\
1335	-68.3448867557254\\
1336	-83.625015225832\\
1337	-142.326041442057\\
1338	-125.262843661751\\
1339	-121.786096006837\\
1340	-75.6874763333978\\
1341	-63.9985454872379\\
1342	-66.3261659974687\\
1343	-41.4792911407153\\
1344	-26.8358561054815\\
1345	-21.6072415821625\\
1346	-19.6454966451899\\
1347	-45.6763682414572\\
1348	-66.0596084490792\\
1349	-81.9618729056616\\
1350	-61.6581559049635\\
1351	-60.1545064834246\\
1352	-67.412717184959\\
1353	-41.5368718136529\\
1354	-46.877549362553\\
1355	-44.6305403180059\\
1356	-34.5441377230994\\
1357	-46.6414019916622\\
1358	-69.8959545683771\\
1359	-90.4879389567632\\
1360	-54.783980190887\\
1361	-42.5558664366689\\
1362	-35.7675959796606\\
1363	-44.3539180288988\\
1364	-30.8605075023745\\
1365	-71.595278080099\\
1366	-117.822702922853\\
1367	-96.6473969861117\\
1368	-117.085548430926\\
1369	-131.279495354627\\
1370	-134.934688263069\\
1371	-101.027599161101\\
1372	-74.8848790898374\\
1373	-78.6807413607741\\
1374	-77.6423134331835\\
1375	-94.6265183781658\\
1376	-109.767869991751\\
1377	-106.469346463895\\
1378	-140.678746959245\\
1379	-155.272208864962\\
1380	-179.080590027487\\
1381	-130.117913706523\\
1382	-143.108848601011\\
1383	-151.229453521815\\
1384	-127.773185148012\\
1385	-143.472841281598\\
1386	-127.514207646434\\
1387	-74.2185978015859\\
1388	-44.1383035022135\\
1389	-48.6826930070896\\
1390	-75.0990218398922\\
1391	-65.077997810542\\
1392	-48.0030551351493\\
1393	-64.430220166484\\
1394	-46.1486281306527\\
1395	-38.6025161037026\\
1396	-46.6235612616906\\
1397	-45.989179944938\\
1398	-40.5269717466235\\
1399	-64.5949862813005\\
1400	-84.9270920304017\\
1401	-64.7303566033156\\
1402	-107.982971091144\\
1403	-154.572459938842\\
1404	-139.631691871692\\
1405	-169.689541213313\\
1406	-115.742171459397\\
1407	-125.923179443759\\
1408	-86.2478429453038\\
1409	-50.6559169897315\\
1410	-68.603876671552\\
1411	-57.779566212028\\
1412	-44.3087283108162\\
1413	-61.0544342328997\\
1414	-40.0091969757442\\
1415	-24.2625171046024\\
1416	-31.004984755023\\
1417	-61.1036177004554\\
1418	-84.1611643193462\\
1419	-98.3163922123665\\
1420	-94.852424476086\\
1421	-83.119802100634\\
1422	-93.3745984882527\\
1423	-78.6577586790949\\
1424	-67.1798684228548\\
1425	-69.3656532165716\\
1426	-57.0799848570848\\
1427	-88.6937020404028\\
1428	-59.358434044011\\
1429	-52.3270385600308\\
1430	-77.1925682023166\\
1431	-97.5080462220379\\
1432	-78.2427483775616\\
1433	-61.2220475911394\\
1434	-53.7292782117421\\
1435	-60.7581804872539\\
1436	-42.3133245026683\\
1437	-41.9694714682451\\
1438	-56.4871716356366\\
1439	-64.988890069028\\
1440	-75.0554974336556\\
1441	-62.9351279867257\\
1442	-74.2984306382432\\
1443	-66.704308670413\\
1444	-41.9106765896419\\
1445	-45.8926649235957\\
1446	-82.5223072421996\\
1447	-122.290462162068\\
1448	-114.611401217799\\
1449	-92.9289612490724\\
1450	-85.7552321031439\\
1451	-71.5008367714728\\
1452	-66.0034132229418\\
1453	-58.9966785520113\\
1454	-74.9224185591654\\
1455	-73.6594273449711\\
1456	-46.2219870634231\\
1457	-70.5866325762508\\
1458	-72.0293066053135\\
1459	-86.7302686876223\\
1460	-97.9145343453538\\
1461	-88.8819528796082\\
1462	-130.048466440714\\
1463	-153.802365108242\\
1464	-171.909933070513\\
1465	-119.781073435148\\
1466	-59.5547006638205\\
1467	-34.5459670687416\\
1468	-23.6582956069475\\
1469	-43.9698068112067\\
1470	-41.4964879151288\\
1471	-77.2745487949726\\
1472	-67.5499316189091\\
1473	-61.1453597395674\\
1474	-76.3720589849358\\
1475	-82.7225733286181\\
1476	-100.453992392765\\
1477	-104.926781649374\\
1478	-150.565173087192\\
1479	-137.569833697728\\
1480	-103.06213405071\\
1481	-75.9488041356325\\
1482	-69.1769556823104\\
1483	-81.0937986897942\\
1484	-96.9775612197013\\
1485	-74.4542090511825\\
1486	-74.1610999136659\\
1487	-77.8980794226734\\
1488	-36.4625732041909\\
1489	-68.8612788585364\\
1490	-96.6516698438288\\
1491	-71.7690033021533\\
1492	-71.9602050416052\\
1493	-138.915718975692\\
1494	-95.8531889743408\\
1495	-98.0002194423134\\
1496	-126.559191871058\\
1497	-118.628918920201\\
1498	-83.127924338426\\
1499	-51.5167627711542\\
1500	-51.4925279935518\\
};
\addlegendentry{Generated output}

\end{axis}
\end{tikzpicture}%\label{fig:c8all}}
%	\caption{Samples of the output obtained experimentally and the output generated by the identified model.}\label{fig:Callout}
%\end{figure}
%
%\begin{figure}[!t]
%	\centering
%	% This file was created by matlab2tikz.
%
\definecolor{mycolor1}{rgb}{0.00000,0.44700,0.74100}%
\definecolor{mycolor2}{rgb}{0.85000,0.32500,0.09800}%
%
\begin{tikzpicture}

\begin{axis}[%
width=4.927cm,
height=3cm,
at={(0cm,14.516cm)},
scale only axis,
xmin=-60,
xmax=0,
xlabel style={font=\color{white!15!black}},
xlabel={y(t-1)},
ymin=-58.594,
ymax=0,
ylabel style={font=\color{white!15!black}},
ylabel={y(t)},
axis background/.style={fill=white},
title style={font=\small},
title={C1, R = 0.7745},
axis x line*=bottom,
axis y line*=left
]
\addplot[only marks, mark=*, mark options={}, mark size=1.5000pt, color=mycolor1, fill=mycolor1] table[row sep=crcr]{%
x	y\\
-19.531	-23.193\\
-23.193	-28.076\\
-28.076	-26.855\\
-26.855	-20.752\\
-20.752	-29.297\\
-29.297	-28.076\\
-28.076	-32.959\\
-32.959	-25.635\\
-25.635	-14.648\\
-14.648	-20.752\\
-20.752	-17.09\\
-17.09	-18.311\\
-18.311	-20.752\\
-20.752	-7.324\\
-7.324	-4.883\\
-4.883	-4.883\\
-4.883	-13.428\\
-13.428	-23.193\\
-23.193	-24.414\\
-24.414	-25.635\\
-25.635	-19.531\\
-19.531	-12.207\\
-12.207	-24.414\\
-24.414	-19.531\\
-19.531	-14.648\\
-14.648	-20.752\\
-20.752	-18.311\\
-18.311	-17.09\\
-17.09	-29.297\\
-29.297	-25.635\\
-25.635	-18.311\\
-18.311	-28.076\\
-28.076	-28.076\\
-28.076	-21.973\\
-21.973	-18.311\\
-18.311	-15.869\\
-15.869	-15.869\\
-15.869	-21.973\\
-21.973	-21.973\\
-21.973	-21.973\\
-21.973	-21.973\\
-21.973	-23.193\\
-23.193	-30.518\\
-30.518	-29.297\\
-29.297	-19.531\\
-19.531	-18.311\\
-18.311	-10.986\\
-10.986	-14.648\\
-14.648	-15.869\\
-15.869	-12.207\\
-12.207	-13.428\\
-13.428	-18.311\\
-18.311	-19.531\\
-19.531	-18.311\\
-18.311	-29.297\\
-29.297	-21.973\\
-21.973	-12.207\\
-12.207	-12.207\\
-12.207	-13.428\\
-13.428	-17.09\\
-17.09	-9.766\\
-9.766	-10.986\\
-10.986	-14.648\\
-14.648	-9.766\\
-9.766	-7.324\\
-7.324	-13.428\\
-13.428	-13.428\\
-13.428	-21.973\\
-21.973	-20.752\\
-20.752	-25.635\\
-25.635	-21.973\\
-21.973	-25.635\\
-25.635	-20.752\\
-20.752	-20.752\\
-20.752	-18.311\\
-18.311	-18.311\\
-18.311	-17.09\\
-17.09	-30.518\\
-30.518	-41.504\\
-41.504	-41.504\\
-41.504	-42.725\\
-42.725	-28.076\\
-28.076	-37.842\\
-37.842	-46.387\\
-46.387	-46.387\\
-46.387	-35.4\\
-35.4	-47.607\\
-47.607	-58.594\\
-58.594	-43.945\\
-43.945	-34.18\\
-34.18	-24.414\\
-24.414	-20.752\\
-20.752	-17.09\\
-17.09	-21.973\\
-21.973	-17.09\\
-17.09	-12.207\\
-12.207	-9.766\\
-9.766	-12.207\\
-12.207	-18.311\\
-18.311	-18.311\\
-18.311	-18.311\\
-18.311	-21.973\\
-21.973	-23.193\\
-23.193	-30.518\\
-30.518	-28.076\\
-28.076	-30.518\\
-30.518	-25.635\\
-25.635	-20.752\\
-20.752	-24.414\\
-24.414	-18.311\\
-18.311	-13.428\\
-13.428	-19.531\\
-19.531	-20.752\\
-20.752	-12.207\\
-12.207	-15.869\\
-15.869	-12.207\\
-12.207	-13.428\\
-13.428	-13.428\\
-13.428	-9.766\\
-9.766	-10.986\\
-10.986	-17.09\\
-17.09	-23.193\\
-23.193	-25.635\\
-25.635	-26.855\\
-26.855	-15.869\\
-15.869	-20.752\\
-20.752	-32.959\\
-32.959	-24.414\\
-24.414	-28.076\\
-28.076	-29.297\\
-29.297	-18.311\\
-18.311	-12.207\\
-12.207	-17.09\\
-17.09	-18.311\\
-18.311	-30.518\\
-30.518	-36.621\\
-36.621	-36.621\\
-36.621	-28.076\\
-28.076	-28.076\\
-28.076	-25.635\\
-25.635	-25.635\\
-25.635	-25.635\\
-25.635	-20.752\\
-20.752	-15.869\\
-15.869	-14.648\\
-14.648	-13.428\\
-13.428	-13.428\\
-13.428	-20.752\\
-20.752	-31.738\\
-31.738	-26.855\\
-26.855	-17.09\\
-17.09	-15.869\\
-15.869	-15.869\\
-15.869	-9.766\\
-9.766	-7.324\\
-7.324	-7.324\\
-7.324	-15.869\\
-15.869	-10.986\\
-10.986	-14.648\\
-14.648	-14.648\\
-14.648	-13.428\\
-13.428	-9.766\\
-9.766	-6.104\\
-6.104	-4.883\\
-4.883	-7.324\\
-7.324	-18.311\\
-18.311	-26.855\\
-26.855	-28.076\\
-28.076	-20.752\\
-20.752	-13.428\\
-13.428	-10.986\\
-10.986	-6.104\\
-6.104	-12.207\\
-12.207	-14.648\\
-14.648	-18.311\\
-18.311	-26.855\\
-26.855	-39.063\\
-39.063	-47.607\\
-47.607	-42.725\\
-42.725	-37.842\\
-37.842	-28.076\\
-28.076	-30.518\\
-30.518	-31.738\\
-31.738	-34.18\\
-34.18	-24.414\\
-24.414	-23.193\\
-23.193	-23.193\\
-23.193	-21.973\\
-21.973	-23.193\\
-23.193	-18.311\\
-18.311	-30.518\\
-30.518	-37.842\\
-37.842	-28.076\\
-28.076	-18.311\\
-18.311	-18.311\\
-18.311	-30.518\\
-30.518	-35.4\\
-35.4	-40.283\\
-40.283	-45.166\\
-45.166	-45.166\\
-45.166	-34.18\\
-34.18	-32.959\\
-32.959	-36.621\\
-36.621	-41.504\\
-41.504	-29.297\\
-29.297	-17.09\\
-17.09	-23.193\\
-23.193	-20.752\\
-20.752	-13.428\\
-13.428	-18.311\\
-18.311	-19.531\\
-19.531	-14.648\\
-14.648	-13.428\\
-13.428	-15.869\\
-15.869	-13.428\\
-13.428	-17.09\\
-17.09	-25.635\\
-25.635	-25.635\\
-25.635	-15.869\\
-15.869	-14.648\\
-14.648	-24.414\\
-24.414	-40.283\\
-40.283	-29.297\\
-29.297	-20.752\\
-20.752	-14.648\\
-14.648	-18.311\\
-18.311	-15.869\\
-15.869	-12.207\\
-12.207	-13.428\\
-13.428	-15.869\\
-15.869	-18.311\\
-18.311	-20.752\\
-20.752	-15.869\\
-15.869	-23.193\\
-23.193	-29.297\\
-29.297	-25.635\\
-25.635	-20.752\\
-20.752	-23.193\\
-23.193	-30.518\\
-30.518	-21.973\\
-21.973	-8.545\\
-8.545	-13.428\\
-13.428	-13.428\\
-13.428	-17.09\\
-17.09	-17.09\\
-17.09	-13.428\\
-13.428	-17.09\\
-17.09	-19.531\\
-19.531	-13.428\\
-13.428	-15.869\\
-15.869	-15.869\\
-15.869	-12.207\\
-12.207	-15.869\\
-15.869	-9.766\\
-9.766	-10.986\\
-10.986	-8.545\\
-8.545	-8.545\\
-8.545	-14.648\\
-14.648	-20.752\\
-20.752	-24.414\\
-24.414	-35.4\\
-35.4	-24.414\\
-24.414	-13.428\\
-13.428	-12.207\\
-12.207	-10.986\\
-10.986	-13.428\\
-13.428	-10.986\\
-10.986	-12.207\\
-12.207	-14.648\\
-14.648	-24.414\\
-24.414	-21.973\\
-21.973	-26.855\\
-26.855	-23.193\\
-23.193	-17.09\\
-17.09	-13.428\\
-13.428	-10.986\\
-10.986	-8.545\\
-8.545	-6.104\\
-6.104	-9.766\\
-9.766	-14.648\\
-14.648	-18.311\\
-18.311	-18.311\\
-18.311	-19.531\\
-19.531	-29.297\\
-29.297	-25.635\\
-25.635	-18.311\\
-18.311	-24.414\\
-24.414	-24.414\\
-24.414	-31.738\\
-31.738	-26.855\\
-26.855	-17.09\\
-17.09	-9.766\\
-9.766	-6.104\\
-6.104	-7.324\\
-7.324	-9.766\\
-9.766	-8.545\\
-8.545	-8.545\\
-8.545	-12.207\\
-12.207	-17.09\\
-17.09	-15.869\\
-15.869	-15.869\\
-15.869	-19.531\\
-19.531	-13.428\\
-13.428	-4.883\\
-4.883	-9.766\\
-9.766	-21.973\\
-21.973	-19.531\\
-19.531	-21.973\\
-21.973	-18.311\\
-18.311	-14.648\\
-14.648	-19.531\\
-19.531	-25.635\\
-25.635	-25.635\\
-25.635	-19.531\\
-19.531	-26.855\\
-26.855	-26.855\\
-26.855	-23.193\\
-23.193	-28.076\\
-28.076	-26.855\\
-26.855	-20.752\\
-20.752	-19.531\\
-19.531	-29.297\\
-29.297	-40.283\\
-40.283	-30.518\\
-30.518	-18.311\\
-18.311	-17.09\\
-17.09	-18.311\\
-18.311	-21.973\\
-21.973	-26.855\\
-26.855	-19.531\\
-19.531	-19.531\\
-19.531	-26.855\\
-26.855	-28.076\\
-28.076	-20.752\\
-20.752	-14.648\\
-14.648	-20.752\\
-20.752	-34.18\\
-34.18	-30.518\\
-30.518	-31.738\\
-31.738	-21.973\\
-21.973	-17.09\\
-17.09	-17.09\\
-17.09	-10.986\\
-10.986	-6.104\\
-6.104	-6.104\\
-6.104	-4.883\\
-4.883	-10.986\\
-10.986	-19.531\\
-19.531	-18.311\\
-18.311	-15.869\\
-15.869	-14.648\\
-14.648	-20.752\\
-20.752	-14.648\\
-14.648	-9.766\\
-9.766	-13.428\\
-13.428	-8.545\\
-8.545	-10.986\\
-10.986	-18.311\\
-18.311	-23.193\\
-23.193	-17.09\\
-17.09	-9.766\\
-9.766	-10.986\\
-10.986	-12.207\\
-12.207	-9.766\\
-9.766	-13.428\\
-13.428	-31.738\\
-31.738	-25.635\\
-25.635	-31.738\\
-31.738	-37.842\\
-37.842	-36.621\\
-36.621	-26.855\\
-26.855	-20.752\\
-20.752	-18.311\\
-18.311	-21.973\\
-21.973	-23.193\\
-23.193	-29.297\\
-29.297	-29.297\\
-29.297	-36.621\\
-36.621	-42.725\\
-42.725	-46.387\\
-46.387	-37.842\\
-37.842	-35.4\\
-35.4	-40.283\\
-40.283	-31.738\\
-31.738	-34.18\\
-34.18	-31.738\\
-31.738	-18.311\\
-18.311	-12.207\\
-12.207	-13.428\\
-13.428	-18.311\\
-18.311	-17.09\\
-17.09	-8.545\\
-8.545	-14.648\\
-14.648	-13.428\\
-13.428	-6.104\\
-6.104	-10.986\\
-10.986	-12.207\\
-12.207	-7.324\\
-7.324	-18.311\\
-18.311	-20.752\\
-20.752	-14.648\\
-14.648	-23.193\\
-23.193	-37.842\\
-37.842	-32.959\\
-32.959	-37.842\\
-37.842	-32.959\\
-32.959	-29.297\\
-29.297	-20.752\\
-20.752	-14.648\\
-14.648	-17.09\\
-17.09	-13.428\\
-13.428	-10.986\\
-10.986	-14.648\\
-14.648	-13.428\\
-13.428	-3.662\\
-3.662	-7.324\\
-7.324	-15.869\\
-15.869	-20.752\\
-20.752	-21.973\\
-21.973	-23.193\\
-23.193	-21.973\\
-21.973	-26.855\\
-26.855	-21.973\\
-21.973	-17.09\\
-17.09	-17.09\\
-17.09	-14.648\\
-14.648	-20.752\\
-20.752	-18.311\\
-18.311	-13.428\\
-13.428	-18.311\\
-18.311	-26.855\\
-26.855	-20.752\\
-20.752	-15.869\\
-15.869	-13.428\\
-13.428	-14.648\\
-14.648	-12.207\\
-12.207	-8.545\\
-8.545	-14.648\\
-14.648	-19.531\\
-19.531	-18.311\\
-18.311	-15.869\\
-15.869	-19.531\\
-19.531	-18.311\\
-18.311	-10.986\\
-10.986	-10.986\\
-10.986	-21.973\\
-21.973	-30.518\\
-30.518	-29.297\\
-29.297	-23.193\\
-23.193	-21.973\\
-21.973	-19.531\\
-19.531	-18.311\\
-18.311	-15.869\\
-15.869	-19.531\\
-19.531	-17.09\\
-17.09	-13.428\\
-13.428	-17.09\\
-17.09	-19.531\\
-19.531	-21.973\\
-21.973	-25.635\\
-25.635	-25.635\\
-25.635	-32.959\\
-32.959	-41.504\\
-41.504	-45.166\\
-45.166	-31.738\\
-31.738	-17.09\\
-17.09	-13.428\\
-13.428	-8.545\\
-8.545	-10.986\\
-10.986	-13.428\\
-13.428	-17.09\\
-17.09	-18.311\\
-18.311	-13.428\\
-13.428	-19.531\\
-19.531	-24.414\\
-24.414	-25.635\\
-25.635	-26.855\\
-26.855	-39.063\\
-39.063	-35.4\\
-35.4	-25.635\\
-25.635	-20.752\\
-20.752	-20.752\\
-20.752	-19.531\\
-19.531	-23.193\\
-23.193	-19.531\\
-19.531	-15.869\\
-15.869	-17.09\\
-17.09	-10.986\\
-10.986	-14.648\\
-14.648	-28.076\\
-28.076	-19.531\\
-19.531	-15.869\\
-15.869	-34.18\\
-34.18	-28.076\\
-28.076	-21.973\\
-21.973	-34.18\\
-34.18	-31.738\\
-31.738	-20.752\\
-20.752	-15.869\\
-15.869	-13.428\\
-13.428	-23.193\\
-23.193	-35.4\\
-35.4	-37.842\\
-37.842	-36.621\\
-36.621	-29.297\\
-29.297	-20.752\\
-20.752	-15.869\\
-15.869	-14.648\\
-14.648	-10.986\\
-10.986	-10.986\\
-10.986	-13.428\\
-13.428	-15.869\\
-15.869	-10.986\\
-10.986	-13.428\\
-13.428	-21.973\\
-21.973	-21.973\\
-21.973	-26.855\\
-26.855	-35.4\\
-35.4	-43.945\\
-43.945	-28.076\\
-28.076	-28.076\\
-28.076	-28.076\\
-28.076	-24.414\\
-24.414	-18.311\\
-18.311	-20.752\\
-20.752	-34.18\\
-34.18	-37.842\\
-37.842	-24.414\\
-24.414	-12.207\\
-12.207	-10.986\\
-10.986	-3.662\\
-3.662	-8.545\\
-8.545	-9.766\\
-9.766	-12.207\\
-12.207	-13.428\\
-13.428	-13.428\\
-13.428	-7.324\\
-7.324	-8.545\\
-8.545	-10.986\\
-10.986	-8.545\\
-8.545	-12.207\\
-12.207	-14.648\\
-14.648	-24.414\\
-24.414	-28.076\\
-28.076	-24.414\\
-24.414	-28.076\\
-28.076	-35.4\\
-35.4	-35.4\\
-35.4	-40.283\\
-40.283	-40.283\\
-40.283	-29.297\\
-29.297	-19.531\\
-19.531	-18.311\\
-18.311	-30.518\\
-30.518	-37.842\\
-37.842	-26.855\\
-26.855	-15.869\\
-15.869	-9.766\\
-9.766	-4.883\\
-4.883	-6.104\\
-6.104	-6.104\\
-6.104	-6.104\\
-6.104	-12.207\\
-12.207	-14.648\\
-14.648	-12.207\\
-12.207	-9.766\\
-9.766	-4.883\\
-4.883	-8.545\\
-8.545	-8.545\\
-8.545	-8.545\\
-8.545	-7.324\\
-7.324	-4.883\\
-4.883	-12.207\\
-12.207	-25.635\\
-25.635	-24.414\\
-24.414	-31.738\\
-31.738	-25.635\\
-25.635	-32.959\\
-32.959	-46.387\\
-46.387	-42.725\\
-42.725	-39.063\\
-39.063	-34.18\\
-34.18	-31.738\\
-31.738	-32.959\\
-32.959	-20.752\\
-20.752	-20.752\\
-20.752	-13.428\\
-13.428	-14.648\\
-14.648	-23.193\\
-23.193	-20.752\\
-20.752	-15.869\\
-15.869	-19.531\\
-19.531	-17.09\\
-17.09	-17.09\\
-17.09	-20.752\\
-20.752	-19.531\\
-19.531	-24.414\\
-24.414	-20.752\\
-20.752	-15.869\\
-15.869	-18.311\\
-18.311	-14.648\\
-14.648	-2.441\\
-2.441	-9.766\\
-9.766	-13.428\\
-13.428	-8.545\\
-8.545	-4.883\\
-4.883	-8.545\\
-8.545	-12.207\\
-12.207	-12.207\\
-12.207	-14.648\\
-14.648	-13.428\\
-13.428	-19.531\\
-19.531	-19.531\\
-19.531	-12.207\\
-12.207	-19.531\\
-19.531	-17.09\\
-17.09	-7.324\\
-7.324	-7.324\\
-7.324	-3.662\\
-3.662	-10.986\\
-10.986	-8.545\\
-8.545	-7.324\\
-7.324	-19.531\\
-19.531	-20.752\\
-20.752	-10.986\\
-10.986	-14.648\\
-14.648	-12.207\\
-12.207	-13.428\\
-13.428	-14.648\\
-14.648	-8.545\\
-8.545	-10.986\\
-10.986	-7.324\\
-7.324	-7.324\\
-7.324	-9.766\\
-9.766	-14.648\\
-14.648	-14.648\\
-14.648	-9.766\\
-9.766	-9.766\\
-9.766	-9.766\\
-9.766	-9.766\\
-9.766	-18.311\\
-18.311	-26.855\\
-26.855	-18.311\\
-18.311	-17.09\\
-17.09	-15.869\\
-15.869	-21.973\\
-21.973	-21.973\\
-21.973	-13.428\\
-13.428	-14.648\\
-14.648	-13.428\\
-13.428	-17.09\\
-17.09	-12.207\\
-12.207	-8.545\\
-8.545	-10.986\\
-10.986	-15.869\\
-15.869	-12.207\\
-12.207	-14.648\\
-14.648	-14.648\\
-14.648	-21.973\\
-21.973	-18.311\\
-18.311	-25.635\\
-25.635	-35.4\\
-35.4	-28.076\\
-28.076	-18.311\\
-18.311	-15.869\\
-15.869	-23.193\\
-23.193	-28.076\\
-28.076	-20.752\\
-20.752	-23.193\\
-23.193	-18.311\\
-18.311	-7.324\\
-7.324	-8.545\\
-8.545	-20.752\\
-20.752	-18.311\\
-18.311	-18.311\\
-18.311	-23.193\\
-23.193	-23.193\\
-23.193	-19.531\\
-19.531	-15.869\\
-15.869	-18.311\\
-18.311	-23.193\\
-23.193	-19.531\\
-19.531	-18.311\\
-18.311	-19.531\\
-19.531	-13.428\\
-13.428	-14.648\\
-14.648	-13.428\\
-13.428	-13.428\\
-13.428	-20.752\\
-20.752	-25.635\\
-25.635	-23.193\\
-23.193	-23.193\\
-23.193	-17.09\\
-17.09	-12.207\\
-12.207	-15.869\\
-15.869	-15.869\\
-15.869	-21.973\\
-21.973	-24.414\\
-24.414	-28.076\\
-28.076	-23.193\\
-23.193	-13.428\\
-13.428	-24.414\\
-24.414	-36.621\\
-36.621	-24.414\\
-24.414	-23.193\\
-23.193	-29.297\\
-29.297	-21.973\\
-21.973	-20.752\\
-20.752	-20.752\\
-20.752	-18.311\\
-18.311	-20.752\\
-20.752	-25.635\\
-25.635	-19.531\\
-19.531	-21.973\\
-21.973	-21.973\\
-21.973	-20.752\\
-20.752	-19.531\\
-19.531	-21.973\\
-21.973	-17.09\\
-17.09	-18.311\\
-18.311	-17.09\\
-17.09	-17.09\\
-17.09	-8.545\\
-8.545	-2.441\\
-2.441	-2.441\\
-2.441	-8.545\\
-8.545	-19.531\\
-19.531	-25.635\\
-25.635	-23.193\\
-23.193	-17.09\\
-17.09	-20.752\\
-20.752	-19.531\\
-19.531	-15.869\\
-15.869	-12.207\\
-12.207	-14.648\\
-14.648	-14.648\\
-14.648	-12.207\\
-12.207	-15.869\\
-15.869	-13.428\\
-13.428	-10.986\\
-10.986	-8.545\\
-8.545	-14.648\\
-14.648	-21.973\\
-21.973	-17.09\\
-17.09	-17.09\\
-17.09	-17.09\\
-17.09	-23.193\\
-23.193	-20.752\\
-20.752	-28.076\\
-28.076	-21.973\\
-21.973	-23.193\\
-23.193	-23.193\\
-23.193	-15.869\\
-15.869	-20.752\\
-20.752	-19.531\\
-19.531	-20.752\\
-20.752	-21.973\\
-21.973	-25.635\\
-25.635	-19.531\\
-19.531	-24.414\\
-24.414	-30.518\\
-30.518	-25.635\\
-25.635	-23.193\\
-23.193	-18.311\\
-18.311	-21.973\\
-21.973	-19.531\\
-19.531	-14.648\\
-14.648	-13.428\\
-13.428	-15.869\\
-15.869	-13.428\\
-13.428	-26.855\\
-26.855	-35.4\\
-35.4	-29.297\\
-29.297	-29.297\\
-29.297	-29.297\\
-29.297	-18.311\\
-18.311	-14.648\\
-14.648	-12.207\\
-12.207	-10.986\\
-10.986	-10.986\\
-10.986	-18.311\\
-18.311	-23.193\\
-23.193	-25.635\\
-25.635	-21.973\\
-21.973	-15.869\\
-15.869	-25.635\\
-25.635	-30.518\\
-30.518	-31.738\\
-31.738	-25.635\\
-25.635	-42.725\\
-42.725	-32.959\\
-32.959	-18.311\\
-18.311	-13.428\\
-13.428	-13.428\\
-13.428	-19.531\\
-19.531	-26.855\\
-26.855	-31.738\\
-31.738	-26.855\\
-26.855	-19.531\\
-19.531	-13.428\\
-13.428	-13.428\\
-13.428	-13.428\\
-13.428	-15.869\\
-15.869	-9.766\\
-9.766	-9.766\\
-9.766	-8.545\\
-8.545	-4.883\\
-4.883	-10.986\\
-10.986	-14.648\\
-14.648	-15.869\\
-15.869	-15.869\\
-15.869	-17.09\\
-17.09	-18.311\\
-18.311	-23.193\\
-23.193	-14.648\\
-14.648	-21.973\\
-21.973	-26.855\\
-26.855	-21.973\\
-21.973	-23.193\\
-23.193	-21.973\\
-21.973	-20.752\\
-20.752	-29.297\\
-29.297	-23.193\\
-23.193	-17.09\\
-17.09	-20.752\\
-20.752	-20.752\\
-20.752	-13.428\\
-13.428	-20.752\\
-20.752	-26.855\\
-26.855	-28.076\\
-28.076	-29.297\\
-29.297	-45.166\\
-45.166	-30.518\\
-30.518	-26.855\\
-26.855	-18.311\\
-18.311	-26.855\\
-26.855	-35.4\\
-35.4	-24.414\\
-24.414	-19.531\\
-19.531	-26.855\\
-26.855	-20.752\\
-20.752	-10.986\\
-10.986	-15.869\\
-15.869	-13.428\\
-13.428	-8.545\\
-8.545	-8.545\\
-8.545	-8.545\\
-8.545	-7.324\\
-7.324	-10.986\\
-10.986	-14.648\\
-14.648	-17.09\\
-17.09	-23.193\\
-23.193	-21.973\\
-21.973	-25.635\\
-25.635	-28.076\\
-28.076	-28.076\\
-28.076	-19.531\\
-19.531	-21.973\\
-21.973	-19.531\\
-19.531	-20.752\\
-20.752	-29.297\\
-29.297	-23.193\\
-23.193	-26.855\\
-26.855	-23.193\\
-23.193	-19.531\\
-19.531	-29.297\\
-29.297	-25.635\\
-25.635	-26.855\\
-26.855	-24.414\\
-24.414	-15.869\\
-15.869	-9.766\\
-9.766	-8.545\\
-8.545	-10.986\\
-10.986	-12.207\\
-12.207	-18.311\\
-18.311	-13.428\\
-13.428	-21.973\\
-21.973	-31.738\\
-31.738	-36.621\\
-36.621	-26.855\\
-26.855	-29.297\\
-29.297	-26.855\\
-26.855	-18.311\\
-18.311	-21.973\\
-21.973	-23.193\\
-23.193	-19.531\\
-19.531	-23.193\\
-23.193	-26.855\\
-26.855	-39.063\\
-39.063	-34.18\\
-34.18	-31.738\\
-31.738	-37.842\\
-37.842	-28.076\\
-28.076	-30.518\\
-30.518	-24.414\\
-24.414	-23.193\\
-23.193	-30.518\\
-30.518	-34.18\\
-34.18	-9.766\\
-9.766	-20.752\\
-20.752	-14.648\\
-14.648	-23.193\\
-23.193	-20.752\\
-20.752	-13.428\\
-13.428	-17.09\\
-17.09	-26.855\\
-26.855	-43.945\\
-43.945	-42.725\\
-42.725	-37.842\\
-37.842	-32.959\\
-32.959	-39.063\\
-39.063	-45.166\\
-45.166	-32.959\\
-32.959	-34.18\\
-34.18	-36.621\\
-36.621	-25.635\\
-25.635	-21.973\\
-21.973	-26.855\\
-26.855	-19.531\\
-19.531	-13.428\\
-13.428	-19.531\\
-19.531	-13.428\\
-13.428	-19.531\\
-19.531	-15.869\\
-15.869	-19.531\\
-19.531	-24.414\\
-24.414	-21.973\\
-21.973	-15.869\\
-15.869	-19.531\\
-19.531	-21.973\\
-21.973	-23.193\\
-23.193	-19.531\\
-19.531	-15.869\\
-15.869	-26.855\\
-26.855	-26.855\\
-26.855	-18.311\\
-18.311	-24.414\\
-24.414	-21.973\\
-21.973	-18.311\\
-18.311	-13.428\\
-13.428	-8.545\\
-8.545	-4.883\\
-4.883	-18.311\\
-18.311	-13.428\\
-13.428	-12.207\\
-12.207	-7.324\\
-7.324	-12.207\\
-12.207	-17.09\\
-17.09	-20.752\\
-20.752	-23.193\\
-23.193	-28.076\\
-28.076	-26.855\\
-26.855	-26.855\\
-26.855	-18.311\\
-18.311	-7.324\\
-7.324	-15.869\\
-15.869	-10.986\\
-10.986	-14.648\\
-14.648	-20.752\\
-20.752	-20.752\\
-20.752	-36.621\\
-36.621	-28.076\\
-28.076	-25.635\\
-25.635	-34.18\\
-34.18	-23.193\\
-23.193	-13.428\\
-13.428	-13.428\\
-13.428	-17.09\\
-17.09	-20.752\\
-20.752	-30.518\\
-30.518	-21.973\\
-21.973	-19.531\\
-19.531	-17.09\\
-17.09	-20.752\\
-20.752	-25.635\\
-25.635	-40.283\\
-40.283	-32.959\\
-32.959	-32.959\\
-32.959	-21.973\\
-21.973	-14.648\\
-14.648	-14.648\\
-14.648	-7.324\\
-7.324	-9.766\\
-9.766	-8.545\\
-8.545	-13.428\\
-13.428	-23.193\\
-23.193	-31.738\\
-31.738	-31.738\\
-31.738	-43.945\\
-43.945	-35.4\\
-35.4	-20.752\\
-20.752	-21.973\\
-21.973	-19.531\\
-19.531	-21.973\\
-21.973	-29.297\\
-29.297	-36.621\\
-36.621	-26.855\\
-26.855	-19.531\\
-19.531	-21.973\\
-21.973	-29.297\\
-29.297	-19.531\\
-19.531	-14.648\\
-14.648	-13.428\\
-13.428	-9.766\\
-9.766	-7.324\\
-7.324	-6.104\\
-6.104	-12.207\\
-12.207	-17.09\\
-17.09	-18.311\\
-18.311	-10.986\\
-10.986	-12.207\\
-12.207	-6.104\\
-6.104	-2.441\\
-2.441	-6.104\\
-6.104	-10.986\\
-10.986	-13.428\\
-13.428	-17.09\\
-17.09	-19.531\\
-19.531	-23.193\\
-23.193	-29.297\\
-29.297	-40.283\\
-40.283	-40.283\\
-40.283	-42.725\\
-42.725	-37.842\\
-37.842	-21.973\\
-21.973	-20.752\\
-20.752	-21.973\\
-21.973	-28.076\\
-28.076	-21.973\\
-21.973	-17.09\\
-17.09	-12.207\\
-12.207	-12.207\\
-12.207	-19.531\\
-19.531	-17.09\\
-17.09	-8.545\\
-8.545	-6.104\\
-6.104	-10.986\\
-10.986	-14.648\\
-14.648	-9.766\\
-9.766	-18.311\\
-18.311	-12.207\\
-12.207	-21.973\\
-21.973	-29.297\\
-29.297	-21.973\\
-21.973	-29.297\\
-29.297	-40.283\\
-40.283	-42.725\\
-42.725	-31.738\\
-31.738	-24.414\\
-24.414	-17.09\\
-17.09	-10.986\\
-10.986	-18.311\\
-18.311	-9.766\\
-9.766	-20.752\\
-20.752	-14.648\\
-14.648	-14.648\\
-14.648	-19.531\\
-19.531	-12.207\\
-12.207	-20.752\\
-20.752	-13.428\\
-13.428	-9.766\\
-9.766	-10.986\\
-10.986	-7.324\\
-7.324	-8.545\\
-8.545	-6.104\\
-6.104	-7.324\\
-7.324	-6.104\\
-6.104	-9.766\\
-9.766	-9.766\\
-9.766	-9.766\\
-9.766	-14.648\\
-14.648	-20.752\\
-20.752	-15.869\\
-15.869	-15.869\\
-15.869	-25.635\\
-25.635	-30.518\\
-30.518	-21.973\\
-21.973	-25.635\\
-25.635	-9.766\\
-9.766	-7.324\\
-7.324	-12.207\\
-12.207	-20.752\\
-20.752	-20.752\\
-20.752	-30.518\\
-30.518	-29.297\\
-29.297	-19.531\\
-19.531	-14.648\\
-14.648	-18.311\\
-18.311	-17.09\\
-17.09	-14.648\\
-14.648	-7.324\\
-7.324	-13.428\\
-13.428	-10.986\\
-10.986	-7.324\\
-7.324	-7.324\\
-7.324	-12.207\\
-12.207	-14.648\\
-14.648	-15.869\\
-15.869	-8.545\\
-8.545	-19.531\\
-19.531	-25.635\\
-25.635	-17.09\\
-17.09	-25.635\\
-25.635	-25.635\\
-25.635	-19.531\\
-19.531	-12.207\\
-12.207	-19.531\\
-19.531	-17.09\\
-17.09	-9.766\\
-9.766	-14.648\\
-14.648	-7.324\\
-7.324	-4.883\\
-4.883	-4.883\\
-4.883	-8.545\\
-8.545	-13.428\\
-13.428	-12.207\\
-12.207	-13.428\\
-13.428	-15.869\\
-15.869	-9.766\\
-9.766	-7.324\\
-7.324	-6.104\\
-6.104	-7.324\\
-7.324	-9.766\\
-9.766	-9.766\\
-9.766	-15.869\\
-15.869	-18.311\\
-18.311	-15.869\\
-15.869	-17.09\\
-17.09	-21.973\\
-21.973	-28.076\\
-28.076	-30.518\\
-30.518	-30.518\\
-30.518	-21.973\\
-21.973	-24.414\\
-24.414	-24.414\\
-24.414	-25.635\\
-25.635	-29.297\\
-29.297	-37.842\\
-37.842	-32.959\\
-32.959	-29.297\\
-29.297	-24.414\\
-24.414	-24.414\\
-24.414	-23.193\\
-23.193	-29.297\\
-29.297	-17.09\\
-17.09	-21.973\\
-21.973	-23.193\\
-23.193	-26.855\\
-26.855	-20.752\\
-20.752	-25.635\\
-25.635	-18.311\\
-18.311	-18.311\\
-18.311	-12.207\\
-12.207	-20.752\\
-20.752	-28.076\\
-28.076	-20.752\\
-20.752	-13.428\\
-13.428	-13.428\\
-13.428	-9.766\\
-9.766	-7.324\\
-7.324	-9.766\\
-9.766	-4.883\\
-4.883	-7.324\\
-7.324	-8.545\\
-8.545	-4.883\\
-4.883	-9.766\\
-9.766	-7.324\\
-7.324	-4.883\\
-4.883	-7.324\\
-7.324	-12.207\\
-12.207	-10.986\\
-10.986	-13.428\\
-13.428	-13.428\\
-13.428	-15.869\\
-15.869	-13.428\\
-13.428	-17.09\\
-17.09	-29.297\\
-29.297	-20.752\\
-20.752	-13.428\\
-13.428	-19.531\\
-19.531	-13.428\\
-13.428	-8.545\\
-8.545	-15.869\\
-15.869	-8.545\\
-8.545	-6.104\\
-6.104	-12.207\\
-12.207	-15.869\\
-15.869	-15.869\\
-15.869	-9.766\\
-9.766	-6.104\\
-6.104	-6.104\\
-6.104	-6.104\\
-6.104	-12.207\\
-12.207	-12.207\\
-12.207	-12.207\\
-12.207	-18.311\\
-18.311	-24.414\\
-24.414	-15.869\\
-15.869	-23.193\\
-23.193	-28.076\\
-28.076	-25.635\\
-25.635	-21.973\\
-21.973	-36.621\\
-36.621	-30.518\\
-30.518	-32.959\\
-32.959	-28.076\\
-28.076	-35.4\\
-35.4	-37.842\\
-37.842	-25.635\\
-25.635	-13.428\\
-13.428	-13.428\\
-13.428	-18.311\\
-18.311	-21.973\\
-21.973	-23.193\\
-23.193	-14.648\\
-14.648	-8.545\\
-8.545	-13.428\\
-13.428	-21.973\\
-21.973	-28.076\\
-28.076	-25.635\\
-25.635	-15.869\\
-15.869	-13.428\\
-13.428	-17.09\\
-17.09	-26.855\\
-26.855	-30.518\\
-30.518	-30.518\\
-30.518	-23.193\\
-23.193	-31.738\\
-31.738	-34.18\\
-34.18	-30.518\\
-30.518	-42.725\\
-42.725	-36.621\\
-36.621	-30.518\\
-30.518	-37.842\\
-37.842	-36.621\\
-36.621	-18.311\\
-18.311	-17.09\\
-17.09	-20.752\\
-20.752	-25.635\\
-25.635	-26.855\\
-26.855	-18.311\\
-18.311	-24.414\\
-24.414	-21.973\\
-21.973	-14.648\\
-14.648	-19.531\\
-19.531	-18.311\\
-18.311	-26.855\\
-26.855	-30.518\\
-30.518	-23.193\\
-23.193	-17.09\\
-17.09	-10.986\\
-10.986	-15.869\\
-15.869	-12.207\\
-12.207	-12.207\\
-12.207	-7.324\\
-7.324	-15.869\\
-15.869	-19.531\\
-19.531	-23.193\\
-23.193	-21.973\\
-21.973	-13.428\\
-13.428	-9.766\\
-9.766	-7.324\\
-7.324	-9.766\\
-9.766	-14.648\\
-14.648	-17.09\\
-17.09	-14.648\\
-14.648	-24.414\\
-24.414	-26.855\\
-26.855	-30.518\\
-30.518	-30.518\\
-30.518	-35.4\\
-35.4	-23.193\\
-23.193	-19.531\\
-19.531	-14.648\\
-14.648	-17.09\\
-17.09	-20.752\\
-20.752	-15.869\\
-15.869	-10.986\\
-10.986	-10.986\\
-10.986	-18.311\\
-18.311	-23.193\\
-23.193	-20.752\\
-20.752	-31.738\\
-31.738	-36.621\\
-36.621	-25.635\\
-25.635	-18.311\\
-18.311	-13.428\\
-13.428	-10.986\\
-10.986	-20.752\\
-20.752	-14.648\\
-14.648	-17.09\\
-17.09	-23.193\\
-23.193	-25.635\\
-25.635	-24.414\\
-24.414	-28.076\\
-28.076	-18.311\\
-18.311	-21.973\\
-21.973	-15.869\\
-15.869	-14.648\\
-14.648	-20.752\\
-20.752	-28.076\\
-28.076	-30.518\\
-30.518	-18.311\\
-18.311	-29.297\\
-29.297	-30.518\\
-30.518	-21.973\\
-21.973	-14.648\\
-14.648	-21.973\\
-21.973	-17.09\\
-17.09	-17.09\\
-17.09	-19.531\\
-19.531	-13.428\\
-13.428	-18.311\\
-18.311	-34.18\\
-34.18	-36.621\\
-36.621	-25.635\\
-25.635	-28.076\\
-28.076	-28.076\\
-28.076	-28.076\\
-28.076	-20.752\\
-20.752	-20.752\\
-20.752	-14.648\\
-14.648	-20.752\\
-20.752	-20.752\\
-20.752	-32.959\\
-32.959	-37.842\\
-37.842	-47.607\\
-47.607	-37.842\\
-37.842	-30.518\\
-30.518	-23.193\\
-23.193	-25.635\\
-25.635	-19.531\\
-19.531	-14.648\\
-14.648	-13.428\\
-13.428	-15.869\\
-15.869	-10.986\\
-10.986	-7.324\\
-7.324	-12.207\\
-12.207	-17.09\\
-17.09	-12.207\\
-12.207	-7.324\\
-7.324	-7.324\\
-7.324	-18.311\\
-18.311	-25.635\\
-25.635	-12.207\\
-12.207	-12.207\\
-12.207	-15.869\\
-15.869	-14.648\\
-14.648	-18.311\\
-18.311	-17.09\\
-17.09	-10.986\\
-10.986	-8.545\\
-8.545	-7.324\\
-7.324	-9.766\\
-9.766	-18.311\\
-18.311	-18.311\\
-18.311	-15.869\\
-15.869	-10.986\\
-10.986	-7.324\\
-7.324	-8.545\\
-8.545	-17.09\\
-17.09	-19.531\\
-19.531	-19.531\\
-19.531	-15.869\\
-15.869	-13.428\\
-13.428	-14.648\\
-14.648	-20.752\\
-20.752	-28.076\\
-28.076	-32.959\\
-32.959	-24.414\\
-24.414	-17.09\\
-17.09	-19.531\\
-19.531	-18.311\\
-18.311	-17.09\\
-17.09	-12.207\\
-12.207	-13.428\\
-13.428	-15.869\\
-15.869	-14.648\\
-14.648	-10.986\\
-10.986	-6.104\\
-6.104	-8.545\\
-8.545	-13.428\\
-13.428	-20.752\\
-20.752	-24.414\\
-24.414	-28.076\\
-28.076	-25.635\\
-25.635	-29.297\\
-29.297	-36.621\\
-36.621	-47.607\\
-47.607	-37.842\\
-37.842	-28.076\\
-28.076	-31.738\\
-31.738	-36.621\\
-36.621	-43.945\\
-43.945	-30.518\\
-30.518	-13.428\\
-13.428	-9.766\\
-9.766	-10.986\\
-10.986	-4.883\\
-4.883	-4.883\\
-4.883	-6.104\\
-6.104	-9.766\\
-9.766	-13.428\\
-13.428	-14.648\\
-14.648	-10.986\\
-10.986	-8.545\\
-8.545	-13.428\\
-13.428	-15.869\\
-15.869	-17.09\\
-17.09	-15.869\\
-15.869	-13.428\\
-13.428	-8.545\\
-8.545	-8.545\\
-8.545	-10.986\\
-10.986	-14.648\\
-14.648	-10.986\\
-10.986	-7.324\\
-7.324	-10.986\\
-10.986	-10.986\\
-10.986	-8.545\\
-8.545	-13.428\\
-13.428	-18.311\\
-18.311	-18.311\\
-18.311	-13.428\\
-13.428	-19.531\\
-19.531	-20.752\\
-20.752	-17.09\\
-17.09	-14.648\\
-14.648	-21.973\\
-21.973	-26.855\\
-26.855	-29.297\\
-29.297	-29.297\\
-29.297	-24.414\\
-24.414	-21.973\\
-21.973	-20.752\\
-20.752	-28.076\\
-28.076	-21.973\\
-21.973	-26.855\\
-26.855	-28.076\\
-28.076	-26.855\\
-26.855	-47.607\\
-47.607	-37.842\\
-37.842	-20.752\\
-20.752	-14.648\\
-14.648	-14.648\\
-14.648	-18.311\\
-18.311	-24.414\\
-24.414	-26.855\\
-26.855	-29.297\\
-29.297	-31.738\\
-31.738	-20.752\\
-20.752	-19.531\\
-19.531	-20.752\\
-20.752	-20.752\\
-20.752	-14.648\\
-14.648	-12.207\\
-12.207	-17.09\\
-17.09	-17.09\\
-17.09	-21.973\\
-21.973	-28.076\\
-28.076	-25.635\\
-25.635	-18.311\\
-18.311	-14.648\\
-14.648	-21.973\\
-21.973	-28.076\\
-28.076	-34.18\\
-34.18	-37.842\\
-37.842	-42.725\\
-42.725	-35.4\\
-35.4	-32.959\\
-32.959	-20.752\\
-20.752	-12.207\\
-12.207	-24.414\\
-24.414	-32.959\\
-32.959	-19.531\\
-19.531	-25.635\\
-25.635	-36.621\\
-36.621	-45.166\\
-45.166	-31.738\\
-31.738	-18.311\\
-18.311	-12.207\\
-12.207	-15.869\\
-15.869	-14.648\\
-14.648	-13.428\\
-13.428	-10.986\\
-10.986	-12.207\\
-12.207	-9.766\\
-9.766	-12.207\\
-12.207	-10.986\\
-10.986	-13.428\\
-13.428	-20.752\\
-20.752	-20.752\\
-20.752	-24.414\\
-24.414	-30.518\\
-30.518	-31.738\\
-31.738	-19.531\\
-19.531	-17.09\\
-17.09	-21.973\\
-21.973	-14.648\\
-14.648	-13.428\\
-13.428	-10.986\\
-10.986	-14.648\\
-14.648	-14.648\\
-14.648	-8.545\\
-8.545	-4.883\\
-4.883	-6.104\\
-6.104	-8.545\\
-8.545	-14.648\\
-14.648	-14.648\\
-14.648	-9.766\\
-9.766	-13.428\\
-13.428	-17.09\\
-17.09	-20.752\\
-20.752	-17.09\\
-17.09	-14.648\\
-14.648	-17.09\\
-17.09	-20.752\\
-20.752	-18.311\\
-18.311	-20.752\\
-20.752	-20.752\\
-20.752	-13.428\\
-13.428	-12.207\\
-12.207	-15.869\\
-15.869	-13.428\\
-13.428	-14.648\\
-14.648	-20.752\\
-20.752	-30.518\\
-30.518	-24.414\\
-24.414	-20.752\\
-20.752	-31.738\\
-31.738	-23.193\\
-23.193	-15.869\\
-15.869	-14.648\\
-14.648	-9.766\\
-9.766	-9.766\\
-9.766	-7.324\\
-7.324	-7.324\\
-7.324	-17.09\\
-17.09	-13.428\\
-13.428	-7.324\\
-7.324	-13.428\\
-13.428	-14.648\\
-14.648	-10.986\\
-10.986	-9.766\\
-9.766	-4.883\\
-4.883	-6.104\\
-6.104	-17.09\\
-17.09	-20.752\\
-20.752	-15.869\\
-15.869	-12.207\\
-12.207	-17.09\\
-17.09	-18.311\\
-18.311	-20.752\\
-20.752	-21.973\\
-21.973	-26.855\\
-26.855	-26.855\\
-26.855	-21.973\\
-21.973	-15.869\\
-15.869	-12.207\\
-12.207	-13.428\\
-13.428	-15.869\\
-15.869	-13.428\\
-13.428	-8.545\\
-8.545	-18.311\\
-18.311	-23.193\\
-23.193	-20.752\\
-20.752	-24.414\\
-24.414	-30.518\\
-30.518	-20.752\\
-20.752	-23.193\\
-23.193	-30.518\\
-30.518	-24.414\\
-24.414	-29.297\\
-29.297	-29.297\\
-29.297	-45.166\\
-45.166	-42.725\\
-42.725	-30.518\\
-30.518	-32.959\\
-32.959	-42.725\\
-42.725	-41.504\\
-41.504	-46.387\\
-46.387	-43.945\\
-43.945	-53.711\\
-53.711	-46.387\\
-46.387	-29.297\\
-29.297	-19.531\\
-19.531	-20.752\\
-20.752	-14.648\\
-14.648	-14.648\\
-14.648	-19.531\\
-19.531	-29.297\\
-29.297	-20.752\\
-20.752	-14.648\\
-14.648	-17.09\\
-17.09	-12.207\\
-12.207	-12.207\\
-12.207	-3.662\\
-3.662	-9.766\\
-9.766	-9.766\\
-9.766	-9.766\\
-9.766	-10.986\\
-10.986	-6.104\\
-6.104	-4.883\\
-4.883	-3.662\\
-3.662	-7.324\\
-7.324	-7.324\\
-7.324	-6.104\\
-6.104	-13.428\\
-13.428	-18.311\\
-18.311	-17.09\\
-17.09	-13.428\\
-13.428	-15.869\\
-15.869	-26.855\\
-26.855	-18.311\\
-18.311	-4.883\\
-4.883	-7.324\\
-7.324	-4.883\\
-4.883	-4.883\\
-4.883	-15.869\\
-15.869	-18.311\\
-18.311	-12.207\\
-12.207	-18.311\\
-18.311	-29.297\\
-29.297	-25.635\\
-25.635	-18.311\\
-18.311	-13.428\\
-13.428	-14.648\\
-14.648	-23.193\\
-23.193	-26.855\\
-26.855	-18.311\\
-18.311	-10.986\\
-10.986	-12.207\\
-12.207	-12.207\\
-12.207	-4.883\\
-4.883	-3.662\\
-3.662	-7.324\\
-7.324	-20.752\\
-20.752	-20.752\\
-20.752	-13.428\\
-13.428	-10.986\\
-10.986	-10.986\\
-10.986	-10.986\\
-10.986	-4.883\\
-4.883	-2.441\\
-2.441	-7.324\\
-7.324	-15.869\\
-15.869	-21.973\\
-21.973	-23.193\\
-23.193	-21.973\\
-21.973	-23.193\\
-23.193	-23.193\\
-23.193	-24.414\\
-24.414	-19.531\\
-19.531	-25.635\\
-25.635	-35.4\\
-35.4	-34.18\\
-34.18	-18.311\\
-18.311	-8.545\\
-8.545	-14.648\\
-14.648	-28.076\\
-28.076	-21.973\\
-21.973	-15.869\\
-15.869	-13.428\\
-13.428	-24.414\\
-24.414	-34.18\\
-34.18	-37.842\\
-37.842	-37.842\\
-37.842	-39.063\\
-39.063	-29.297\\
-29.297	-28.076\\
-28.076	-18.311\\
-18.311	-12.207\\
-12.207	-18.311\\
-18.311	-19.531\\
-19.531	-20.752\\
-20.752	-23.193\\
-23.193	-15.869\\
-15.869	-17.09\\
-17.09	-20.752\\
-20.752	-9.766\\
-9.766	-6.104\\
-6.104	-6.104\\
-6.104	-8.545\\
-8.545	-6.104\\
-6.104	-1.221\\
-1.221	-4.883\\
-4.883	-17.09\\
-17.09	-13.428\\
-13.428	-10.986\\
-10.986	-15.869\\
-15.869	-24.414\\
-24.414	-20.752\\
-20.752	-7.324\\
-7.324	-6.104\\
-6.104	-8.545\\
-8.545	-26.855\\
-26.855	-31.738\\
-31.738	-41.504\\
-41.504	-50.049\\
-50.049	-47.607\\
-47.607	-34.18\\
-34.18	-34.18\\
-34.18	-43.945\\
-43.945	-36.621\\
-36.621	-32.959\\
-32.959	-37.842\\
-37.842	-41.504\\
-41.504	-41.504\\
-41.504	-56.152\\
-56.152	-41.504\\
-41.504	-21.973\\
-21.973	-13.428\\
-13.428	-9.766\\
-9.766	-12.207\\
-12.207	-12.207\\
-12.207	-10.986\\
-10.986	-20.752\\
-20.752	-18.311\\
-18.311	-19.531\\
-19.531	-20.752\\
-20.752	-14.648\\
-14.648	-13.428\\
-13.428	-21.973\\
-21.973	-23.193\\
-23.193	-13.428\\
-13.428	-7.324\\
-7.324	-9.766\\
-9.766	-12.207\\
-12.207	-29.297\\
-29.297	-35.4\\
-35.4	-32.959\\
-32.959	-35.4\\
-35.4	-43.945\\
-43.945	-52.49\\
-52.49	-46.387\\
-46.387	-30.518\\
-30.518	-19.531\\
-19.531	-13.428\\
-13.428	-13.428\\
-13.428	-14.648\\
-14.648	-17.09\\
-17.09	-9.766\\
-9.766	-12.207\\
-12.207	-13.428\\
-13.428	-15.869\\
-15.869	-19.531\\
-19.531	-20.752\\
-20.752	-18.311\\
-18.311	-8.545\\
-8.545	-4.883\\
-4.883	-4.883\\
-4.883	-3.662\\
-3.662	-6.104\\
-6.104	-10.986\\
-10.986	-14.648\\
-14.648	-13.428\\
-13.428	-23.193\\
-23.193	-18.311\\
-18.311	-23.193\\
-23.193	-40.283\\
-40.283	-30.518\\
-30.518	-25.635\\
-25.635	-32.959\\
-32.959	-36.621\\
-36.621	-26.855\\
-26.855	-21.973\\
-21.973	-15.869\\
-15.869	-15.869\\
-15.869	-30.518\\
-30.518	-48.828\\
-48.828	-57.373\\
-57.373	-46.387\\
-46.387	-39.063\\
-39.063	-30.518\\
-30.518	-30.518\\
-30.518	-37.842\\
-37.842	-28.076\\
-28.076	-19.531\\
-19.531	-19.531\\
-19.531	-25.635\\
-25.635	-29.297\\
-29.297	-15.869\\
-15.869	-14.648\\
-14.648	-14.648\\
-14.648	-24.414\\
-24.414	-28.076\\
-28.076	-29.297\\
-29.297	-20.752\\
-20.752	-17.09\\
-17.09	-17.09\\
-17.09	-15.869\\
-15.869	-10.986\\
-10.986	-9.766\\
-9.766	-8.545\\
-8.545	-6.104\\
-6.104	-7.324\\
-7.324	-12.207\\
-12.207	-18.311\\
-18.311	-18.311\\
-18.311	-13.428\\
-13.428	-9.766\\
-9.766	-8.545\\
-8.545	-10.986\\
-10.986	-15.869\\
-15.869	-18.311\\
-18.311	-15.869\\
-15.869	-10.986\\
-10.986	-24.414\\
-24.414	-26.855\\
-26.855	-20.752\\
-20.752	-8.545\\
-8.545	-13.428\\
-13.428	-17.09\\
-17.09	-17.09\\
-17.09	-12.207\\
-12.207	-8.545\\
-8.545	-12.207\\
-12.207	-26.855\\
-26.855	-13.428\\
-13.428	-7.324\\
-7.324	-9.766\\
-9.766	-18.311\\
-18.311	-21.973\\
-21.973	-12.207\\
-12.207	-10.986\\
-10.986	-20.752\\
-20.752	-29.297\\
-29.297	-25.635\\
-25.635	-37.842\\
-37.842	-40.283\\
-40.283	-30.518\\
-30.518	-23.193\\
-23.193	-30.518\\
-30.518	-25.635\\
-25.635	-26.855\\
-26.855	-36.621\\
-36.621	-28.076\\
-28.076	-14.648\\
-14.648	-25.635\\
-25.635	-36.621\\
-36.621	-41.504\\
-41.504	-31.738\\
-31.738	-28.076\\
-28.076	-34.18\\
-34.18	-30.518\\
-30.518	-17.09\\
-17.09	-17.09\\
-17.09	-21.973\\
-21.973	-23.193\\
-23.193	-20.752\\
-20.752	-24.414\\
-24.414	-15.869\\
-15.869	-19.531\\
-19.531	-28.076\\
-28.076	-19.531\\
-19.531	-13.428\\
-13.428	-10.986\\
-10.986	-8.545\\
-8.545	-8.545\\
-8.545	-17.09\\
-17.09	-18.311\\
-18.311	-19.531\\
-19.531	-26.855\\
-26.855	-31.738\\
-31.738	-28.076\\
-28.076	-21.973\\
-21.973	-13.428\\
-13.428	-15.869\\
-15.869	-29.297\\
-29.297	-20.752\\
-20.752	-7.324\\
-7.324	-17.09\\
-17.09	-25.635\\
-25.635	-28.076\\
-28.076	-35.4\\
-35.4	-46.387\\
-46.387	-36.621\\
-36.621	-29.297\\
-29.297	-34.18\\
-34.18	-25.635\\
-25.635	-18.311\\
-18.311	-21.973\\
-21.973	-29.297\\
-29.297	-28.076\\
-28.076	-28.076\\
-28.076	-18.311\\
-18.311	-14.648\\
-14.648	-12.207\\
-12.207	-17.09\\
-17.09	-18.311\\
-18.311	-13.428\\
-13.428	-24.414\\
-24.414	-30.518\\
-30.518	-18.311\\
-18.311	-13.428\\
-13.428	-23.193\\
-23.193	-14.648\\
-14.648	-6.104\\
-6.104	-12.207\\
-12.207	-13.428\\
-13.428	-10.986\\
-10.986	-7.324\\
-7.324	-6.104\\
-6.104	-7.324\\
-7.324	-10.986\\
-10.986	-20.752\\
-20.752	-25.635\\
-25.635	-29.297\\
-29.297	-32.959\\
-32.959	-18.311\\
-18.311	-12.207\\
-12.207	-28.076\\
-28.076	-41.504\\
-41.504	-31.738\\
-31.738	-29.297\\
-29.297	-45.166\\
-45.166	-34.18\\
-34.18	-30.518\\
-30.518	-25.635\\
-25.635	-23.193\\
-23.193	-31.738\\
-31.738	-24.414\\
-24.414	-20.752\\
-20.752	-30.518\\
-30.518	-40.283\\
-40.283	-40.283\\
-40.283	-15.869\\
-15.869	-9.766\\
-9.766	-28.076\\
-28.076	-20.752\\
-20.752	-8.545\\
-8.545	-8.545\\
-8.545	-14.648\\
-14.648	-19.531\\
-19.531	-28.076\\
-28.076	-18.311\\
-18.311	-13.428\\
-13.428	-10.986\\
-10.986	-6.104\\
-6.104	-8.545\\
-8.545	-6.104\\
-6.104	-7.324\\
-7.324	-17.09\\
-17.09	-14.648\\
-14.648	-6.104\\
-6.104	-6.104\\
-6.104	-8.545\\
-8.545	-10.986\\
-10.986	-8.545\\
-8.545	-9.766\\
-9.766	-9.766\\
-9.766	-2.441\\
-2.441	-12.207\\
-12.207	-18.311\\
-18.311	-24.414\\
-24.414	-34.18\\
-34.18	-30.518\\
-30.518	-14.648\\
-14.648	-12.207\\
-12.207	-12.207\\
-12.207	-3.662\\
-3.662	-4.883\\
-4.883	-13.428\\
-13.428	-18.311\\
-18.311	-14.648\\
-14.648	-14.648\\
-14.648	-18.311\\
-18.311	-19.531\\
-19.531	-19.531\\
-19.531	-25.635\\
-25.635	-20.752\\
-20.752	-34.18\\
-34.18	-18.311\\
-18.311	-12.207\\
-12.207	-8.545\\
-8.545	-15.869\\
-15.869	-13.428\\
-13.428	-9.766\\
-9.766	-3.662\\
-3.662	-10.986\\
-10.986	-8.545\\
-8.545	-2.441\\
-2.441	-7.324\\
-7.324	-9.766\\
-9.766	-14.648\\
-14.648	-19.531\\
-19.531	-26.855\\
-26.855	-21.973\\
-21.973	-7.324\\
-7.324	-13.428\\
-13.428	-21.973\\
-21.973	-10.986\\
-10.986	-8.545\\
-8.545	-21.973\\
-21.973	-18.311\\
-18.311	-28.076\\
-28.076	-36.621\\
-36.621	-23.193\\
-23.193	-18.311\\
-18.311	-15.869\\
};
\addplot [color=mycolor2, line width=2.0pt, forget plot]
  table[row sep=crcr]{%
-19.531	-18.7469676958216\\
-23.193	-22.2619641477237\\
-28.076	-26.9489460359371\\
-26.855	-25.7769605996257\\
-20.752	-19.9189531321331\\
-29.297	-28.1209314722486\\
-28.076	-26.9489460359371\\
-32.959	-31.6359279241506\\
-25.635	-24.6059350203465\\
-14.648	-14.0599858076082\\
-20.752	-19.9189531321331\\
-17.09	-16.403956680231\\
-18.311	-17.5759421165424\\
-20.752	-19.9189531321331\\
-7.324	-7.02999290380409\\
-4.883	-4.68698188821346\\
-13.428	-12.888960228329\\
-23.193	-22.2619641477237\\
-24.414	-23.4339495840351\\
-25.635	-24.6059350203465\\
-19.531	-18.7469676958216\\
-12.207	-11.7169747920175\\
-24.414	-23.4339495840351\\
-19.531	-18.7469676958216\\
-14.648	-14.0599858076082\\
-20.752	-19.9189531321331\\
-18.311	-17.5759421165424\\
-17.09	-16.403956680231\\
-29.297	-28.1209314722486\\
-25.635	-24.6059350203465\\
-18.311	-17.5759421165424\\
-28.076	-26.9489460359371\\
-21.973	-21.0909385684445\\
-18.311	-17.5759421165424\\
-15.869	-15.2319712439196\\
-21.973	-21.0909385684445\\
-23.193	-22.2619641477237\\
-30.518	-29.29291690856\\
-29.297	-28.1209314722486\\
-19.531	-18.7469676958216\\
-18.311	-17.5759421165424\\
-10.986	-10.5449893557061\\
-14.648	-14.0599858076082\\
-15.869	-15.2319712439196\\
-12.207	-11.7169747920175\\
-13.428	-12.888960228329\\
-18.311	-17.5759421165424\\
-19.531	-18.7469676958216\\
-18.311	-17.5759421165424\\
-29.297	-28.1209314722486\\
-21.973	-21.0909385684445\\
-12.207	-11.7169747920175\\
-13.428	-12.888960228329\\
-17.09	-16.403956680231\\
-9.766	-9.37396377642692\\
-10.986	-10.5449893557061\\
-14.648	-14.0599858076082\\
-9.766	-9.37396377642692\\
-7.324	-7.02999290380409\\
-13.428	-12.888960228329\\
-21.973	-21.0909385684445\\
-20.752	-19.9189531321331\\
-25.635	-24.6059350203465\\
-21.973	-21.0909385684445\\
-25.635	-24.6059350203465\\
-20.752	-19.9189531321331\\
-18.311	-17.5759421165424\\
-17.09	-16.403956680231\\
-30.518	-29.29291690856\\
-41.504	-39.8379062642661\\
-42.725	-41.0098917005775\\
-28.076	-26.9489460359371\\
-37.842	-36.3229098123641\\
-46.387	-44.5248881524796\\
-35.4	-33.9789389397412\\
-47.607	-45.6959137317588\\
-58.594	-56.2418629444971\\
-43.945	-42.1809172798567\\
-34.18	-32.807913360462\\
-24.414	-23.4339495840351\\
-20.752	-19.9189531321331\\
-17.09	-16.403956680231\\
-21.973	-21.0909385684445\\
-17.09	-16.403956680231\\
-12.207	-11.7169747920175\\
-9.766	-9.37396377642692\\
-12.207	-11.7169747920175\\
-18.311	-17.5759421165424\\
-21.973	-21.0909385684445\\
-23.193	-22.2619641477237\\
-30.518	-29.29291690856\\
-28.076	-26.9489460359371\\
-30.518	-29.29291690856\\
-25.635	-24.6059350203465\\
-20.752	-19.9189531321331\\
-24.414	-23.4339495840351\\
-18.311	-17.5759421165424\\
-13.428	-12.888960228329\\
-19.531	-18.7469676958216\\
-20.752	-19.9189531321331\\
-12.207	-11.7169747920175\\
-15.869	-15.2319712439196\\
-12.207	-11.7169747920175\\
-13.428	-12.888960228329\\
-9.766	-9.37396377642692\\
-10.986	-10.5449893557061\\
-17.09	-16.403956680231\\
-23.193	-22.2619641477237\\
-25.635	-24.6059350203465\\
-26.855	-25.7769605996257\\
-15.869	-15.2319712439196\\
-20.752	-19.9189531321331\\
-32.959	-31.6359279241506\\
-24.414	-23.4339495840351\\
-28.076	-26.9489460359371\\
-29.297	-28.1209314722486\\
-18.311	-17.5759421165424\\
-12.207	-11.7169747920175\\
-17.09	-16.403956680231\\
-18.311	-17.5759421165424\\
-30.518	-29.29291690856\\
-36.621	-35.1509243760527\\
-28.076	-26.9489460359371\\
-25.635	-24.6059350203465\\
-20.752	-19.9189531321331\\
-15.869	-15.2319712439196\\
-14.648	-14.0599858076082\\
-13.428	-12.888960228329\\
-20.752	-19.9189531321331\\
-31.738	-30.4639424878392\\
-26.855	-25.7769605996257\\
-17.09	-16.403956680231\\
-15.869	-15.2319712439196\\
-9.766	-9.37396377642692\\
-7.324	-7.02999290380409\\
-15.869	-15.2319712439196\\
-10.986	-10.5449893557061\\
-14.648	-14.0599858076082\\
-13.428	-12.888960228329\\
-9.766	-9.37396377642692\\
-6.104	-5.85896732452487\\
-4.883	-4.68698188821346\\
-7.324	-7.02999290380409\\
-18.311	-17.5759421165424\\
-26.855	-25.7769605996257\\
-28.076	-26.9489460359371\\
-20.752	-19.9189531321331\\
-13.428	-12.888960228329\\
-10.986	-10.5449893557061\\
-6.104	-5.85896732452487\\
-12.207	-11.7169747920175\\
-14.648	-14.0599858076082\\
-18.311	-17.5759421165424\\
-26.855	-25.7769605996257\\
-39.063	-37.4948952486755\\
-47.607	-45.6959137317588\\
-42.725	-41.0098917005775\\
-37.842	-36.3229098123641\\
-28.076	-26.9489460359371\\
-30.518	-29.29291690856\\
-31.738	-30.4639424878392\\
-34.18	-32.807913360462\\
-24.414	-23.4339495840351\\
-23.193	-22.2619641477237\\
-21.973	-21.0909385684445\\
-23.193	-22.2619641477237\\
-18.311	-17.5759421165424\\
-30.518	-29.29291690856\\
-37.842	-36.3229098123641\\
-28.076	-26.9489460359371\\
-18.311	-17.5759421165424\\
-30.518	-29.29291690856\\
-35.4	-33.9789389397412\\
-40.283	-38.6659208279547\\
-45.166	-43.3529027161681\\
-34.18	-32.807913360462\\
-32.959	-31.6359279241506\\
-36.621	-35.1509243760527\\
-41.504	-39.8379062642661\\
-29.297	-28.1209314722486\\
-17.09	-16.403956680231\\
-23.193	-22.2619641477237\\
-20.752	-19.9189531321331\\
-13.428	-12.888960228329\\
-18.311	-17.5759421165424\\
-19.531	-18.7469676958216\\
-14.648	-14.0599858076082\\
-13.428	-12.888960228329\\
-15.869	-15.2319712439196\\
-13.428	-12.888960228329\\
-17.09	-16.403956680231\\
-25.635	-24.6059350203465\\
-15.869	-15.2319712439196\\
-14.648	-14.0599858076082\\
-24.414	-23.4339495840351\\
-40.283	-38.6659208279547\\
-29.297	-28.1209314722486\\
-20.752	-19.9189531321331\\
-14.648	-14.0599858076082\\
-18.311	-17.5759421165424\\
-15.869	-15.2319712439196\\
-12.207	-11.7169747920175\\
-13.428	-12.888960228329\\
-15.869	-15.2319712439196\\
-18.311	-17.5759421165424\\
-20.752	-19.9189531321331\\
-15.869	-15.2319712439196\\
-23.193	-22.2619641477237\\
-29.297	-28.1209314722486\\
-25.635	-24.6059350203465\\
-20.752	-19.9189531321331\\
-23.193	-22.2619641477237\\
-30.518	-29.29291690856\\
-21.973	-21.0909385684445\\
-8.545	-8.2019783401155\\
-13.428	-12.888960228329\\
-17.09	-16.403956680231\\
-13.428	-12.888960228329\\
-17.09	-16.403956680231\\
-19.531	-18.7469676958216\\
-13.428	-12.888960228329\\
-15.869	-15.2319712439196\\
-12.207	-11.7169747920175\\
-15.869	-15.2319712439196\\
-9.766	-9.37396377642692\\
-10.986	-10.5449893557061\\
-8.545	-8.2019783401155\\
-14.648	-14.0599858076082\\
-20.752	-19.9189531321331\\
-24.414	-23.4339495840351\\
-35.4	-33.9789389397412\\
-24.414	-23.4339495840351\\
-13.428	-12.888960228329\\
-12.207	-11.7169747920175\\
-10.986	-10.5449893557061\\
-13.428	-12.888960228329\\
-10.986	-10.5449893557061\\
-12.207	-11.7169747920175\\
-14.648	-14.0599858076082\\
-24.414	-23.4339495840351\\
-21.973	-21.0909385684445\\
-26.855	-25.7769605996257\\
-23.193	-22.2619641477237\\
-17.09	-16.403956680231\\
-13.428	-12.888960228329\\
-10.986	-10.5449893557061\\
-8.545	-8.2019783401155\\
-6.104	-5.85896732452487\\
-9.766	-9.37396377642692\\
-14.648	-14.0599858076082\\
-18.311	-17.5759421165424\\
-19.531	-18.7469676958216\\
-29.297	-28.1209314722486\\
-25.635	-24.6059350203465\\
-18.311	-17.5759421165424\\
-24.414	-23.4339495840351\\
-31.738	-30.4639424878392\\
-26.855	-25.7769605996257\\
-17.09	-16.403956680231\\
-9.766	-9.37396377642692\\
-6.104	-5.85896732452487\\
-7.324	-7.02999290380409\\
-9.766	-9.37396377642692\\
-8.545	-8.2019783401155\\
-12.207	-11.7169747920175\\
-17.09	-16.403956680231\\
-15.869	-15.2319712439196\\
-19.531	-18.7469676958216\\
-13.428	-12.888960228329\\
-4.883	-4.68698188821346\\
-9.766	-9.37396377642692\\
-21.973	-21.0909385684445\\
-19.531	-18.7469676958216\\
-21.973	-21.0909385684445\\
-18.311	-17.5759421165424\\
-14.648	-14.0599858076082\\
-19.531	-18.7469676958216\\
-25.635	-24.6059350203465\\
-19.531	-18.7469676958216\\
-26.855	-25.7769605996257\\
-23.193	-22.2619641477237\\
-28.076	-26.9489460359371\\
-26.855	-25.7769605996257\\
-20.752	-19.9189531321331\\
-19.531	-18.7469676958216\\
-29.297	-28.1209314722486\\
-40.283	-38.6659208279547\\
-30.518	-29.29291690856\\
-18.311	-17.5759421165424\\
-17.09	-16.403956680231\\
-18.311	-17.5759421165424\\
-21.973	-21.0909385684445\\
-26.855	-25.7769605996257\\
-19.531	-18.7469676958216\\
-26.855	-25.7769605996257\\
-28.076	-26.9489460359371\\
-20.752	-19.9189531321331\\
-14.648	-14.0599858076082\\
-20.752	-19.9189531321331\\
-34.18	-32.807913360462\\
-30.518	-29.29291690856\\
-31.738	-30.4639424878392\\
-21.973	-21.0909385684445\\
-17.09	-16.403956680231\\
-10.986	-10.5449893557061\\
-6.104	-5.85896732452487\\
-4.883	-4.68698188821346\\
-10.986	-10.5449893557061\\
-19.531	-18.7469676958216\\
-18.311	-17.5759421165424\\
-15.869	-15.2319712439196\\
-14.648	-14.0599858076082\\
-20.752	-19.9189531321331\\
-14.648	-14.0599858076082\\
-9.766	-9.37396377642692\\
-13.428	-12.888960228329\\
-8.545	-8.2019783401155\\
-10.986	-10.5449893557061\\
-18.311	-17.5759421165424\\
-23.193	-22.2619641477237\\
-17.09	-16.403956680231\\
-9.766	-9.37396377642692\\
-10.986	-10.5449893557061\\
-12.207	-11.7169747920175\\
-9.766	-9.37396377642692\\
-13.428	-12.888960228329\\
-31.738	-30.4639424878392\\
-25.635	-24.6059350203465\\
-31.738	-30.4639424878392\\
-37.842	-36.3229098123641\\
-36.621	-35.1509243760527\\
-26.855	-25.7769605996257\\
-20.752	-19.9189531321331\\
-18.311	-17.5759421165424\\
-21.973	-21.0909385684445\\
-23.193	-22.2619641477237\\
-29.297	-28.1209314722486\\
-36.621	-35.1509243760527\\
-42.725	-41.0098917005775\\
-46.387	-44.5248881524796\\
-37.842	-36.3229098123641\\
-35.4	-33.9789389397412\\
-40.283	-38.6659208279547\\
-31.738	-30.4639424878392\\
-34.18	-32.807913360462\\
-31.738	-30.4639424878392\\
-18.311	-17.5759421165424\\
-12.207	-11.7169747920175\\
-13.428	-12.888960228329\\
-18.311	-17.5759421165424\\
-17.09	-16.403956680231\\
-8.545	-8.2019783401155\\
-14.648	-14.0599858076082\\
-13.428	-12.888960228329\\
-6.104	-5.85896732452487\\
-10.986	-10.5449893557061\\
-12.207	-11.7169747920175\\
-7.324	-7.02999290380409\\
-18.311	-17.5759421165424\\
-20.752	-19.9189531321331\\
-14.648	-14.0599858076082\\
-23.193	-22.2619641477237\\
-37.842	-36.3229098123641\\
-32.959	-31.6359279241506\\
-37.842	-36.3229098123641\\
-32.959	-31.6359279241506\\
-29.297	-28.1209314722486\\
-20.752	-19.9189531321331\\
-14.648	-14.0599858076082\\
-17.09	-16.403956680231\\
-13.428	-12.888960228329\\
-10.986	-10.5449893557061\\
-14.648	-14.0599858076082\\
-13.428	-12.888960228329\\
-3.662	-3.51499645190204\\
-7.324	-7.02999290380409\\
-15.869	-15.2319712439196\\
-20.752	-19.9189531321331\\
-21.973	-21.0909385684445\\
-23.193	-22.2619641477237\\
-21.973	-21.0909385684445\\
-26.855	-25.7769605996257\\
-21.973	-21.0909385684445\\
-17.09	-16.403956680231\\
-14.648	-14.0599858076082\\
-20.752	-19.9189531321331\\
-18.311	-17.5759421165424\\
-13.428	-12.888960228329\\
-18.311	-17.5759421165424\\
-26.855	-25.7769605996257\\
-20.752	-19.9189531321331\\
-15.869	-15.2319712439196\\
-13.428	-12.888960228329\\
-14.648	-14.0599858076082\\
-12.207	-11.7169747920175\\
-8.545	-8.2019783401155\\
-14.648	-14.0599858076082\\
-19.531	-18.7469676958216\\
-18.311	-17.5759421165424\\
-15.869	-15.2319712439196\\
-19.531	-18.7469676958216\\
-18.311	-17.5759421165424\\
-10.986	-10.5449893557061\\
-21.973	-21.0909385684445\\
-30.518	-29.29291690856\\
-29.297	-28.1209314722486\\
-23.193	-22.2619641477237\\
-21.973	-21.0909385684445\\
-19.531	-18.7469676958216\\
-18.311	-17.5759421165424\\
-15.869	-15.2319712439196\\
-19.531	-18.7469676958216\\
-17.09	-16.403956680231\\
-13.428	-12.888960228329\\
-17.09	-16.403956680231\\
-19.531	-18.7469676958216\\
-21.973	-21.0909385684445\\
-25.635	-24.6059350203465\\
-32.959	-31.6359279241506\\
-41.504	-39.8379062642661\\
-45.166	-43.3529027161681\\
-31.738	-30.4639424878392\\
-17.09	-16.403956680231\\
-13.428	-12.888960228329\\
-8.545	-8.2019783401155\\
-10.986	-10.5449893557061\\
-13.428	-12.888960228329\\
-17.09	-16.403956680231\\
-18.311	-17.5759421165424\\
-13.428	-12.888960228329\\
-19.531	-18.7469676958216\\
-24.414	-23.4339495840351\\
-25.635	-24.6059350203465\\
-26.855	-25.7769605996257\\
-39.063	-37.4948952486755\\
-35.4	-33.9789389397412\\
-25.635	-24.6059350203465\\
-20.752	-19.9189531321331\\
-19.531	-18.7469676958216\\
-23.193	-22.2619641477237\\
-19.531	-18.7469676958216\\
-15.869	-15.2319712439196\\
-17.09	-16.403956680231\\
-10.986	-10.5449893557061\\
-14.648	-14.0599858076082\\
-28.076	-26.9489460359371\\
-19.531	-18.7469676958216\\
-15.869	-15.2319712439196\\
-34.18	-32.807913360462\\
-28.076	-26.9489460359371\\
-21.973	-21.0909385684445\\
-34.18	-32.807913360462\\
-31.738	-30.4639424878392\\
-20.752	-19.9189531321331\\
-15.869	-15.2319712439196\\
-13.428	-12.888960228329\\
-23.193	-22.2619641477237\\
-35.4	-33.9789389397412\\
-37.842	-36.3229098123641\\
-36.621	-35.1509243760527\\
-29.297	-28.1209314722486\\
-20.752	-19.9189531321331\\
-15.869	-15.2319712439196\\
-14.648	-14.0599858076082\\
-10.986	-10.5449893557061\\
-13.428	-12.888960228329\\
-15.869	-15.2319712439196\\
-10.986	-10.5449893557061\\
-13.428	-12.888960228329\\
-21.973	-21.0909385684445\\
-26.855	-25.7769605996257\\
-35.4	-33.9789389397412\\
-43.945	-42.1809172798567\\
-28.076	-26.9489460359371\\
-24.414	-23.4339495840351\\
-18.311	-17.5759421165424\\
-20.752	-19.9189531321331\\
-34.18	-32.807913360462\\
-37.842	-36.3229098123641\\
-24.414	-23.4339495840351\\
-12.207	-11.7169747920175\\
-10.986	-10.5449893557061\\
-3.662	-3.51499645190204\\
-8.545	-8.2019783401155\\
-9.766	-9.37396377642692\\
-12.207	-11.7169747920175\\
-13.428	-12.888960228329\\
-7.324	-7.02999290380409\\
-8.545	-8.2019783401155\\
-10.986	-10.5449893557061\\
-8.545	-8.2019783401155\\
-12.207	-11.7169747920175\\
-14.648	-14.0599858076082\\
-24.414	-23.4339495840351\\
-28.076	-26.9489460359371\\
-24.414	-23.4339495840351\\
-28.076	-26.9489460359371\\
-35.4	-33.9789389397412\\
-40.283	-38.6659208279547\\
-29.297	-28.1209314722486\\
-19.531	-18.7469676958216\\
-18.311	-17.5759421165424\\
-30.518	-29.29291690856\\
-37.842	-36.3229098123641\\
-26.855	-25.7769605996257\\
-15.869	-15.2319712439196\\
-9.766	-9.37396377642692\\
-4.883	-4.68698188821346\\
-6.104	-5.85896732452487\\
-12.207	-11.7169747920175\\
-14.648	-14.0599858076082\\
-12.207	-11.7169747920175\\
-9.766	-9.37396377642692\\
-4.883	-4.68698188821346\\
-8.545	-8.2019783401155\\
-7.324	-7.02999290380409\\
-4.883	-4.68698188821346\\
-12.207	-11.7169747920175\\
-25.635	-24.6059350203465\\
-24.414	-23.4339495840351\\
-31.738	-30.4639424878392\\
-25.635	-24.6059350203465\\
-32.959	-31.6359279241506\\
-46.387	-44.5248881524796\\
-42.725	-41.0098917005775\\
-39.063	-37.4948952486755\\
-34.18	-32.807913360462\\
-31.738	-30.4639424878392\\
-32.959	-31.6359279241506\\
-20.752	-19.9189531321331\\
-13.428	-12.888960228329\\
-14.648	-14.0599858076082\\
-23.193	-22.2619641477237\\
-20.752	-19.9189531321331\\
-15.869	-15.2319712439196\\
-19.531	-18.7469676958216\\
-17.09	-16.403956680231\\
-20.752	-19.9189531321331\\
-19.531	-18.7469676958216\\
-24.414	-23.4339495840351\\
-20.752	-19.9189531321331\\
-15.869	-15.2319712439196\\
-18.311	-17.5759421165424\\
-14.648	-14.0599858076082\\
-2.441	-2.34301101559063\\
-9.766	-9.37396377642692\\
-13.428	-12.888960228329\\
-8.545	-8.2019783401155\\
-4.883	-4.68698188821346\\
-8.545	-8.2019783401155\\
-12.207	-11.7169747920175\\
-14.648	-14.0599858076082\\
-13.428	-12.888960228329\\
-19.531	-18.7469676958216\\
-12.207	-11.7169747920175\\
-19.531	-18.7469676958216\\
-17.09	-16.403956680231\\
-7.324	-7.02999290380409\\
-3.662	-3.51499645190204\\
-10.986	-10.5449893557061\\
-8.545	-8.2019783401155\\
-7.324	-7.02999290380409\\
-19.531	-18.7469676958216\\
-20.752	-19.9189531321331\\
-10.986	-10.5449893557061\\
-14.648	-14.0599858076082\\
-12.207	-11.7169747920175\\
-13.428	-12.888960228329\\
-14.648	-14.0599858076082\\
-8.545	-8.2019783401155\\
-10.986	-10.5449893557061\\
-7.324	-7.02999290380409\\
-9.766	-9.37396377642692\\
-14.648	-14.0599858076082\\
-9.766	-9.37396377642692\\
-18.311	-17.5759421165424\\
-26.855	-25.7769605996257\\
-18.311	-17.5759421165424\\
-17.09	-16.403956680231\\
-15.869	-15.2319712439196\\
-21.973	-21.0909385684445\\
-13.428	-12.888960228329\\
-14.648	-14.0599858076082\\
-13.428	-12.888960228329\\
-17.09	-16.403956680231\\
-12.207	-11.7169747920175\\
-8.545	-8.2019783401155\\
-10.986	-10.5449893557061\\
-15.869	-15.2319712439196\\
-12.207	-11.7169747920175\\
-14.648	-14.0599858076082\\
-21.973	-21.0909385684445\\
-18.311	-17.5759421165424\\
-25.635	-24.6059350203465\\
-35.4	-33.9789389397412\\
-28.076	-26.9489460359371\\
-18.311	-17.5759421165424\\
-15.869	-15.2319712439196\\
-23.193	-22.2619641477237\\
-28.076	-26.9489460359371\\
-20.752	-19.9189531321331\\
-23.193	-22.2619641477237\\
-18.311	-17.5759421165424\\
-7.324	-7.02999290380409\\
-8.545	-8.2019783401155\\
-20.752	-19.9189531321331\\
-18.311	-17.5759421165424\\
-23.193	-22.2619641477237\\
-19.531	-18.7469676958216\\
-15.869	-15.2319712439196\\
-18.311	-17.5759421165424\\
-23.193	-22.2619641477237\\
-19.531	-18.7469676958216\\
-18.311	-17.5759421165424\\
-19.531	-18.7469676958216\\
-13.428	-12.888960228329\\
-14.648	-14.0599858076082\\
-13.428	-12.888960228329\\
-20.752	-19.9189531321331\\
-25.635	-24.6059350203465\\
-23.193	-22.2619641477237\\
-17.09	-16.403956680231\\
-12.207	-11.7169747920175\\
-15.869	-15.2319712439196\\
-21.973	-21.0909385684445\\
-24.414	-23.4339495840351\\
-28.076	-26.9489460359371\\
-23.193	-22.2619641477237\\
-13.428	-12.888960228329\\
-24.414	-23.4339495840351\\
-36.621	-35.1509243760527\\
-24.414	-23.4339495840351\\
-23.193	-22.2619641477237\\
-29.297	-28.1209314722486\\
-21.973	-21.0909385684445\\
-20.752	-19.9189531321331\\
-18.311	-17.5759421165424\\
-20.752	-19.9189531321331\\
-25.635	-24.6059350203465\\
-19.531	-18.7469676958216\\
-21.973	-21.0909385684445\\
-20.752	-19.9189531321331\\
-19.531	-18.7469676958216\\
-21.973	-21.0909385684445\\
-17.09	-16.403956680231\\
-18.311	-17.5759421165424\\
-17.09	-16.403956680231\\
-8.545	-8.2019783401155\\
-2.441	-2.34301101559063\\
-8.545	-8.2019783401155\\
-19.531	-18.7469676958216\\
-25.635	-24.6059350203465\\
-23.193	-22.2619641477237\\
-17.09	-16.403956680231\\
-20.752	-19.9189531321331\\
-19.531	-18.7469676958216\\
-15.869	-15.2319712439196\\
-12.207	-11.7169747920175\\
-14.648	-14.0599858076082\\
-12.207	-11.7169747920175\\
-15.869	-15.2319712439196\\
-13.428	-12.888960228329\\
-10.986	-10.5449893557061\\
-8.545	-8.2019783401155\\
-14.648	-14.0599858076082\\
-21.973	-21.0909385684445\\
-17.09	-16.403956680231\\
-23.193	-22.2619641477237\\
-20.752	-19.9189531321331\\
-28.076	-26.9489460359371\\
-21.973	-21.0909385684445\\
-23.193	-22.2619641477237\\
-15.869	-15.2319712439196\\
-20.752	-19.9189531321331\\
-19.531	-18.7469676958216\\
-20.752	-19.9189531321331\\
-21.973	-21.0909385684445\\
-25.635	-24.6059350203465\\
-19.531	-18.7469676958216\\
-24.414	-23.4339495840351\\
-30.518	-29.29291690856\\
-25.635	-24.6059350203465\\
-23.193	-22.2619641477237\\
-18.311	-17.5759421165424\\
-21.973	-21.0909385684445\\
-19.531	-18.7469676958216\\
-14.648	-14.0599858076082\\
-13.428	-12.888960228329\\
-15.869	-15.2319712439196\\
-13.428	-12.888960228329\\
-26.855	-25.7769605996257\\
-35.4	-33.9789389397412\\
-29.297	-28.1209314722486\\
-18.311	-17.5759421165424\\
-14.648	-14.0599858076082\\
-12.207	-11.7169747920175\\
-10.986	-10.5449893557061\\
-18.311	-17.5759421165424\\
-23.193	-22.2619641477237\\
-25.635	-24.6059350203465\\
-21.973	-21.0909385684445\\
-15.869	-15.2319712439196\\
-25.635	-24.6059350203465\\
-30.518	-29.29291690856\\
-31.738	-30.4639424878392\\
-25.635	-24.6059350203465\\
-42.725	-41.0098917005775\\
-32.959	-31.6359279241506\\
-18.311	-17.5759421165424\\
-13.428	-12.888960228329\\
-19.531	-18.7469676958216\\
-26.855	-25.7769605996257\\
-31.738	-30.4639424878392\\
-26.855	-25.7769605996257\\
-19.531	-18.7469676958216\\
-13.428	-12.888960228329\\
-15.869	-15.2319712439196\\
-9.766	-9.37396377642692\\
-8.545	-8.2019783401155\\
-4.883	-4.68698188821346\\
-10.986	-10.5449893557061\\
-14.648	-14.0599858076082\\
-15.869	-15.2319712439196\\
-17.09	-16.403956680231\\
-18.311	-17.5759421165424\\
-23.193	-22.2619641477237\\
-14.648	-14.0599858076082\\
-21.973	-21.0909385684445\\
-26.855	-25.7769605996257\\
-21.973	-21.0909385684445\\
-23.193	-22.2619641477237\\
-21.973	-21.0909385684445\\
-20.752	-19.9189531321331\\
-29.297	-28.1209314722486\\
-23.193	-22.2619641477237\\
-17.09	-16.403956680231\\
-20.752	-19.9189531321331\\
-13.428	-12.888960228329\\
-20.752	-19.9189531321331\\
-26.855	-25.7769605996257\\
-28.076	-26.9489460359371\\
-29.297	-28.1209314722486\\
-45.166	-43.3529027161681\\
-30.518	-29.29291690856\\
-26.855	-25.7769605996257\\
-18.311	-17.5759421165424\\
-26.855	-25.7769605996257\\
-35.4	-33.9789389397412\\
-24.414	-23.4339495840351\\
-19.531	-18.7469676958216\\
-26.855	-25.7769605996257\\
-20.752	-19.9189531321331\\
-10.986	-10.5449893557061\\
-15.869	-15.2319712439196\\
-13.428	-12.888960228329\\
-8.545	-8.2019783401155\\
-7.324	-7.02999290380409\\
-10.986	-10.5449893557061\\
-14.648	-14.0599858076082\\
-17.09	-16.403956680231\\
-23.193	-22.2619641477237\\
-21.973	-21.0909385684445\\
-25.635	-24.6059350203465\\
-28.076	-26.9489460359371\\
-19.531	-18.7469676958216\\
-21.973	-21.0909385684445\\
-19.531	-18.7469676958216\\
-20.752	-19.9189531321331\\
-29.297	-28.1209314722486\\
-23.193	-22.2619641477237\\
-26.855	-25.7769605996257\\
-23.193	-22.2619641477237\\
-19.531	-18.7469676958216\\
-29.297	-28.1209314722486\\
-25.635	-24.6059350203465\\
-26.855	-25.7769605996257\\
-24.414	-23.4339495840351\\
-15.869	-15.2319712439196\\
-9.766	-9.37396377642692\\
-8.545	-8.2019783401155\\
-10.986	-10.5449893557061\\
-12.207	-11.7169747920175\\
-18.311	-17.5759421165424\\
-13.428	-12.888960228329\\
-21.973	-21.0909385684445\\
-31.738	-30.4639424878392\\
-36.621	-35.1509243760527\\
-26.855	-25.7769605996257\\
-29.297	-28.1209314722486\\
-26.855	-25.7769605996257\\
-18.311	-17.5759421165424\\
-21.973	-21.0909385684445\\
-23.193	-22.2619641477237\\
-19.531	-18.7469676958216\\
-23.193	-22.2619641477237\\
-26.855	-25.7769605996257\\
-39.063	-37.4948952486755\\
-34.18	-32.807913360462\\
-31.738	-30.4639424878392\\
-37.842	-36.3229098123641\\
-28.076	-26.9489460359371\\
-30.518	-29.29291690856\\
-24.414	-23.4339495840351\\
-23.193	-22.2619641477237\\
-30.518	-29.29291690856\\
-34.18	-32.807913360462\\
-9.766	-9.37396377642692\\
-20.752	-19.9189531321331\\
-14.648	-14.0599858076082\\
-23.193	-22.2619641477237\\
-20.752	-19.9189531321331\\
-13.428	-12.888960228329\\
-17.09	-16.403956680231\\
-26.855	-25.7769605996257\\
-43.945	-42.1809172798567\\
-42.725	-41.0098917005775\\
-37.842	-36.3229098123641\\
-32.959	-31.6359279241506\\
-39.063	-37.4948952486755\\
-45.166	-43.3529027161681\\
-32.959	-31.6359279241506\\
-34.18	-32.807913360462\\
-36.621	-35.1509243760527\\
-25.635	-24.6059350203465\\
-21.973	-21.0909385684445\\
-26.855	-25.7769605996257\\
-19.531	-18.7469676958216\\
-13.428	-12.888960228329\\
-19.531	-18.7469676958216\\
-13.428	-12.888960228329\\
-19.531	-18.7469676958216\\
-15.869	-15.2319712439196\\
-19.531	-18.7469676958216\\
-24.414	-23.4339495840351\\
-21.973	-21.0909385684445\\
-15.869	-15.2319712439196\\
-19.531	-18.7469676958216\\
-21.973	-21.0909385684445\\
-23.193	-22.2619641477237\\
-19.531	-18.7469676958216\\
-15.869	-15.2319712439196\\
-26.855	-25.7769605996257\\
-18.311	-17.5759421165424\\
-24.414	-23.4339495840351\\
-21.973	-21.0909385684445\\
-18.311	-17.5759421165424\\
-13.428	-12.888960228329\\
-8.545	-8.2019783401155\\
-4.883	-4.68698188821346\\
-18.311	-17.5759421165424\\
-13.428	-12.888960228329\\
-12.207	-11.7169747920175\\
-7.324	-7.02999290380409\\
-12.207	-11.7169747920175\\
-17.09	-16.403956680231\\
-20.752	-19.9189531321331\\
-23.193	-22.2619641477237\\
-28.076	-26.9489460359371\\
-26.855	-25.7769605996257\\
-18.311	-17.5759421165424\\
-7.324	-7.02999290380409\\
-15.869	-15.2319712439196\\
-10.986	-10.5449893557061\\
-14.648	-14.0599858076082\\
-20.752	-19.9189531321331\\
-36.621	-35.1509243760527\\
-28.076	-26.9489460359371\\
-25.635	-24.6059350203465\\
-34.18	-32.807913360462\\
-23.193	-22.2619641477237\\
-13.428	-12.888960228329\\
-17.09	-16.403956680231\\
-20.752	-19.9189531321331\\
-30.518	-29.29291690856\\
-21.973	-21.0909385684445\\
-19.531	-18.7469676958216\\
-17.09	-16.403956680231\\
-20.752	-19.9189531321331\\
-25.635	-24.6059350203465\\
-40.283	-38.6659208279547\\
-32.959	-31.6359279241506\\
-21.973	-21.0909385684445\\
-14.648	-14.0599858076082\\
-7.324	-7.02999290380409\\
-9.766	-9.37396377642692\\
-8.545	-8.2019783401155\\
-13.428	-12.888960228329\\
-23.193	-22.2619641477237\\
-31.738	-30.4639424878392\\
-43.945	-42.1809172798567\\
-35.4	-33.9789389397412\\
-20.752	-19.9189531321331\\
-21.973	-21.0909385684445\\
-19.531	-18.7469676958216\\
-21.973	-21.0909385684445\\
-29.297	-28.1209314722486\\
-36.621	-35.1509243760527\\
-26.855	-25.7769605996257\\
-19.531	-18.7469676958216\\
-21.973	-21.0909385684445\\
-29.297	-28.1209314722486\\
-19.531	-18.7469676958216\\
-14.648	-14.0599858076082\\
-13.428	-12.888960228329\\
-9.766	-9.37396377642692\\
-7.324	-7.02999290380409\\
-6.104	-5.85896732452487\\
-12.207	-11.7169747920175\\
-17.09	-16.403956680231\\
-18.311	-17.5759421165424\\
-10.986	-10.5449893557061\\
-12.207	-11.7169747920175\\
-6.104	-5.85896732452487\\
-2.441	-2.34301101559063\\
-6.104	-5.85896732452487\\
-10.986	-10.5449893557061\\
-13.428	-12.888960228329\\
-17.09	-16.403956680231\\
-19.531	-18.7469676958216\\
-23.193	-22.2619641477237\\
-29.297	-28.1209314722486\\
-40.283	-38.6659208279547\\
-42.725	-41.0098917005775\\
-37.842	-36.3229098123641\\
-21.973	-21.0909385684445\\
-20.752	-19.9189531321331\\
-21.973	-21.0909385684445\\
-28.076	-26.9489460359371\\
-21.973	-21.0909385684445\\
-17.09	-16.403956680231\\
-12.207	-11.7169747920175\\
-19.531	-18.7469676958216\\
-17.09	-16.403956680231\\
-8.545	-8.2019783401155\\
-6.104	-5.85896732452487\\
-10.986	-10.5449893557061\\
-14.648	-14.0599858076082\\
-9.766	-9.37396377642692\\
-18.311	-17.5759421165424\\
-12.207	-11.7169747920175\\
-21.973	-21.0909385684445\\
-29.297	-28.1209314722486\\
-21.973	-21.0909385684445\\
-29.297	-28.1209314722486\\
-40.283	-38.6659208279547\\
-42.725	-41.0098917005775\\
-31.738	-30.4639424878392\\
-24.414	-23.4339495840351\\
-17.09	-16.403956680231\\
-10.986	-10.5449893557061\\
-18.311	-17.5759421165424\\
-9.766	-9.37396377642692\\
-20.752	-19.9189531321331\\
-14.648	-14.0599858076082\\
-19.531	-18.7469676958216\\
-12.207	-11.7169747920175\\
-20.752	-19.9189531321331\\
-13.428	-12.888960228329\\
-9.766	-9.37396377642692\\
-10.986	-10.5449893557061\\
-7.324	-7.02999290380409\\
-8.545	-8.2019783401155\\
-6.104	-5.85896732452487\\
-7.324	-7.02999290380409\\
-6.104	-5.85896732452487\\
-9.766	-9.37396377642692\\
-14.648	-14.0599858076082\\
-20.752	-19.9189531321331\\
-15.869	-15.2319712439196\\
-25.635	-24.6059350203465\\
-30.518	-29.29291690856\\
-21.973	-21.0909385684445\\
-25.635	-24.6059350203465\\
-9.766	-9.37396377642692\\
-7.324	-7.02999290380409\\
-12.207	-11.7169747920175\\
-20.752	-19.9189531321331\\
-30.518	-29.29291690856\\
-29.297	-28.1209314722486\\
-19.531	-18.7469676958216\\
-14.648	-14.0599858076082\\
-18.311	-17.5759421165424\\
-17.09	-16.403956680231\\
-14.648	-14.0599858076082\\
-7.324	-7.02999290380409\\
-13.428	-12.888960228329\\
-10.986	-10.5449893557061\\
-7.324	-7.02999290380409\\
-12.207	-11.7169747920175\\
-14.648	-14.0599858076082\\
-15.869	-15.2319712439196\\
-8.545	-8.2019783401155\\
-19.531	-18.7469676958216\\
-25.635	-24.6059350203465\\
-17.09	-16.403956680231\\
-25.635	-24.6059350203465\\
-19.531	-18.7469676958216\\
-12.207	-11.7169747920175\\
-19.531	-18.7469676958216\\
-17.09	-16.403956680231\\
-9.766	-9.37396377642692\\
-14.648	-14.0599858076082\\
-7.324	-7.02999290380409\\
-4.883	-4.68698188821346\\
-8.545	-8.2019783401155\\
-13.428	-12.888960228329\\
-12.207	-11.7169747920175\\
-13.428	-12.888960228329\\
-15.869	-15.2319712439196\\
-9.766	-9.37396377642692\\
-7.324	-7.02999290380409\\
-6.104	-5.85896732452487\\
-7.324	-7.02999290380409\\
-9.766	-9.37396377642692\\
-15.869	-15.2319712439196\\
-18.311	-17.5759421165424\\
-15.869	-15.2319712439196\\
-17.09	-16.403956680231\\
-21.973	-21.0909385684445\\
-28.076	-26.9489460359371\\
-30.518	-29.29291690856\\
-21.973	-21.0909385684445\\
-24.414	-23.4339495840351\\
-25.635	-24.6059350203465\\
-29.297	-28.1209314722486\\
-37.842	-36.3229098123641\\
-32.959	-31.6359279241506\\
-29.297	-28.1209314722486\\
-24.414	-23.4339495840351\\
-23.193	-22.2619641477237\\
-29.297	-28.1209314722486\\
-17.09	-16.403956680231\\
-21.973	-21.0909385684445\\
-23.193	-22.2619641477237\\
-26.855	-25.7769605996257\\
-20.752	-19.9189531321331\\
-25.635	-24.6059350203465\\
-18.311	-17.5759421165424\\
-12.207	-11.7169747920175\\
-20.752	-19.9189531321331\\
-28.076	-26.9489460359371\\
-20.752	-19.9189531321331\\
-13.428	-12.888960228329\\
-9.766	-9.37396377642692\\
-7.324	-7.02999290380409\\
-9.766	-9.37396377642692\\
-4.883	-4.68698188821346\\
-7.324	-7.02999290380409\\
-8.545	-8.2019783401155\\
-4.883	-4.68698188821346\\
-9.766	-9.37396377642692\\
-7.324	-7.02999290380409\\
-4.883	-4.68698188821346\\
-7.324	-7.02999290380409\\
-12.207	-11.7169747920175\\
-10.986	-10.5449893557061\\
-13.428	-12.888960228329\\
-15.869	-15.2319712439196\\
-13.428	-12.888960228329\\
-17.09	-16.403956680231\\
-29.297	-28.1209314722486\\
-20.752	-19.9189531321331\\
-13.428	-12.888960228329\\
-19.531	-18.7469676958216\\
-13.428	-12.888960228329\\
-8.545	-8.2019783401155\\
-15.869	-15.2319712439196\\
-8.545	-8.2019783401155\\
-6.104	-5.85896732452487\\
-12.207	-11.7169747920175\\
-15.869	-15.2319712439196\\
-9.766	-9.37396377642692\\
-6.104	-5.85896732452487\\
-12.207	-11.7169747920175\\
-18.311	-17.5759421165424\\
-24.414	-23.4339495840351\\
-15.869	-15.2319712439196\\
-23.193	-22.2619641477237\\
-28.076	-26.9489460359371\\
-25.635	-24.6059350203465\\
-21.973	-21.0909385684445\\
-36.621	-35.1509243760527\\
-30.518	-29.29291690856\\
-32.959	-31.6359279241506\\
-28.076	-26.9489460359371\\
-35.4	-33.9789389397412\\
-37.842	-36.3229098123641\\
-25.635	-24.6059350203465\\
-13.428	-12.888960228329\\
-18.311	-17.5759421165424\\
-21.973	-21.0909385684445\\
-23.193	-22.2619641477237\\
-14.648	-14.0599858076082\\
-8.545	-8.2019783401155\\
-13.428	-12.888960228329\\
-21.973	-21.0909385684445\\
-28.076	-26.9489460359371\\
-25.635	-24.6059350203465\\
-15.869	-15.2319712439196\\
-13.428	-12.888960228329\\
-17.09	-16.403956680231\\
-26.855	-25.7769605996257\\
-30.518	-29.29291690856\\
-23.193	-22.2619641477237\\
-31.738	-30.4639424878392\\
-34.18	-32.807913360462\\
-30.518	-29.29291690856\\
-42.725	-41.0098917005775\\
-36.621	-35.1509243760527\\
-30.518	-29.29291690856\\
-37.842	-36.3229098123641\\
-36.621	-35.1509243760527\\
-18.311	-17.5759421165424\\
-17.09	-16.403956680231\\
-20.752	-19.9189531321331\\
-25.635	-24.6059350203465\\
-26.855	-25.7769605996257\\
-18.311	-17.5759421165424\\
-24.414	-23.4339495840351\\
-21.973	-21.0909385684445\\
-14.648	-14.0599858076082\\
-19.531	-18.7469676958216\\
-18.311	-17.5759421165424\\
-26.855	-25.7769605996257\\
-30.518	-29.29291690856\\
-23.193	-22.2619641477237\\
-17.09	-16.403956680231\\
-10.986	-10.5449893557061\\
-15.869	-15.2319712439196\\
-12.207	-11.7169747920175\\
-7.324	-7.02999290380409\\
-15.869	-15.2319712439196\\
-19.531	-18.7469676958216\\
-23.193	-22.2619641477237\\
-21.973	-21.0909385684445\\
-13.428	-12.888960228329\\
-9.766	-9.37396377642692\\
-7.324	-7.02999290380409\\
-9.766	-9.37396377642692\\
-14.648	-14.0599858076082\\
-17.09	-16.403956680231\\
-14.648	-14.0599858076082\\
-24.414	-23.4339495840351\\
-26.855	-25.7769605996257\\
-30.518	-29.29291690856\\
-35.4	-33.9789389397412\\
-23.193	-22.2619641477237\\
-19.531	-18.7469676958216\\
-14.648	-14.0599858076082\\
-17.09	-16.403956680231\\
-20.752	-19.9189531321331\\
-15.869	-15.2319712439196\\
-10.986	-10.5449893557061\\
-18.311	-17.5759421165424\\
-23.193	-22.2619641477237\\
-20.752	-19.9189531321331\\
-31.738	-30.4639424878392\\
-36.621	-35.1509243760527\\
-25.635	-24.6059350203465\\
-18.311	-17.5759421165424\\
-13.428	-12.888960228329\\
-10.986	-10.5449893557061\\
-20.752	-19.9189531321331\\
-14.648	-14.0599858076082\\
-17.09	-16.403956680231\\
-23.193	-22.2619641477237\\
-25.635	-24.6059350203465\\
-24.414	-23.4339495840351\\
-28.076	-26.9489460359371\\
-18.311	-17.5759421165424\\
-21.973	-21.0909385684445\\
-15.869	-15.2319712439196\\
-14.648	-14.0599858076082\\
-20.752	-19.9189531321331\\
-28.076	-26.9489460359371\\
-30.518	-29.29291690856\\
-18.311	-17.5759421165424\\
-29.297	-28.1209314722486\\
-30.518	-29.29291690856\\
-21.973	-21.0909385684445\\
-14.648	-14.0599858076082\\
-21.973	-21.0909385684445\\
-17.09	-16.403956680231\\
-19.531	-18.7469676958216\\
-13.428	-12.888960228329\\
-18.311	-17.5759421165424\\
-34.18	-32.807913360462\\
-36.621	-35.1509243760527\\
-25.635	-24.6059350203465\\
-28.076	-26.9489460359371\\
-20.752	-19.9189531321331\\
-14.648	-14.0599858076082\\
-20.752	-19.9189531321331\\
-32.959	-31.6359279241506\\
-37.842	-36.3229098123641\\
-47.607	-45.6959137317588\\
-37.842	-36.3229098123641\\
-30.518	-29.29291690856\\
-23.193	-22.2619641477237\\
-25.635	-24.6059350203465\\
-19.531	-18.7469676958216\\
-14.648	-14.0599858076082\\
-13.428	-12.888960228329\\
-15.869	-15.2319712439196\\
-10.986	-10.5449893557061\\
-7.324	-7.02999290380409\\
-12.207	-11.7169747920175\\
-17.09	-16.403956680231\\
-12.207	-11.7169747920175\\
-7.324	-7.02999290380409\\
-18.311	-17.5759421165424\\
-25.635	-24.6059350203465\\
-12.207	-11.7169747920175\\
-15.869	-15.2319712439196\\
-14.648	-14.0599858076082\\
-18.311	-17.5759421165424\\
-17.09	-16.403956680231\\
-10.986	-10.5449893557061\\
-8.545	-8.2019783401155\\
-7.324	-7.02999290380409\\
-9.766	-9.37396377642692\\
-18.311	-17.5759421165424\\
-15.869	-15.2319712439196\\
-10.986	-10.5449893557061\\
-7.324	-7.02999290380409\\
-8.545	-8.2019783401155\\
-17.09	-16.403956680231\\
-19.531	-18.7469676958216\\
-15.869	-15.2319712439196\\
-13.428	-12.888960228329\\
-14.648	-14.0599858076082\\
-20.752	-19.9189531321331\\
-28.076	-26.9489460359371\\
-32.959	-31.6359279241506\\
-24.414	-23.4339495840351\\
-17.09	-16.403956680231\\
-19.531	-18.7469676958216\\
-18.311	-17.5759421165424\\
-17.09	-16.403956680231\\
-12.207	-11.7169747920175\\
-13.428	-12.888960228329\\
-15.869	-15.2319712439196\\
-14.648	-14.0599858076082\\
-10.986	-10.5449893557061\\
-6.104	-5.85896732452487\\
-8.545	-8.2019783401155\\
-13.428	-12.888960228329\\
-20.752	-19.9189531321331\\
-24.414	-23.4339495840351\\
-28.076	-26.9489460359371\\
-25.635	-24.6059350203465\\
-29.297	-28.1209314722486\\
-36.621	-35.1509243760527\\
-47.607	-45.6959137317588\\
-37.842	-36.3229098123641\\
-28.076	-26.9489460359371\\
-31.738	-30.4639424878392\\
-36.621	-35.1509243760527\\
-43.945	-42.1809172798567\\
-30.518	-29.29291690856\\
-13.428	-12.888960228329\\
-9.766	-9.37396377642692\\
-10.986	-10.5449893557061\\
-4.883	-4.68698188821346\\
-6.104	-5.85896732452487\\
-9.766	-9.37396377642692\\
-13.428	-12.888960228329\\
-14.648	-14.0599858076082\\
-10.986	-10.5449893557061\\
-8.545	-8.2019783401155\\
-13.428	-12.888960228329\\
-15.869	-15.2319712439196\\
-17.09	-16.403956680231\\
-15.869	-15.2319712439196\\
-13.428	-12.888960228329\\
-8.545	-8.2019783401155\\
-10.986	-10.5449893557061\\
-14.648	-14.0599858076082\\
-10.986	-10.5449893557061\\
-7.324	-7.02999290380409\\
-10.986	-10.5449893557061\\
-8.545	-8.2019783401155\\
-13.428	-12.888960228329\\
-18.311	-17.5759421165424\\
-13.428	-12.888960228329\\
-19.531	-18.7469676958216\\
-20.752	-19.9189531321331\\
-17.09	-16.403956680231\\
-14.648	-14.0599858076082\\
-21.973	-21.0909385684445\\
-26.855	-25.7769605996257\\
-29.297	-28.1209314722486\\
-24.414	-23.4339495840351\\
-21.973	-21.0909385684445\\
-20.752	-19.9189531321331\\
-28.076	-26.9489460359371\\
-21.973	-21.0909385684445\\
-26.855	-25.7769605996257\\
-28.076	-26.9489460359371\\
-26.855	-25.7769605996257\\
-47.607	-45.6959137317588\\
-37.842	-36.3229098123641\\
-20.752	-19.9189531321331\\
-14.648	-14.0599858076082\\
-18.311	-17.5759421165424\\
-24.414	-23.4339495840351\\
-26.855	-25.7769605996257\\
-29.297	-28.1209314722486\\
-31.738	-30.4639424878392\\
-20.752	-19.9189531321331\\
-19.531	-18.7469676958216\\
-20.752	-19.9189531321331\\
-14.648	-14.0599858076082\\
-12.207	-11.7169747920175\\
-17.09	-16.403956680231\\
-21.973	-21.0909385684445\\
-28.076	-26.9489460359371\\
-25.635	-24.6059350203465\\
-18.311	-17.5759421165424\\
-14.648	-14.0599858076082\\
-21.973	-21.0909385684445\\
-28.076	-26.9489460359371\\
-34.18	-32.807913360462\\
-37.842	-36.3229098123641\\
-42.725	-41.0098917005775\\
-35.4	-33.9789389397412\\
-32.959	-31.6359279241506\\
-20.752	-19.9189531321331\\
-12.207	-11.7169747920175\\
-24.414	-23.4339495840351\\
-32.959	-31.6359279241506\\
-19.531	-18.7469676958216\\
-25.635	-24.6059350203465\\
-36.621	-35.1509243760527\\
-45.166	-43.3529027161681\\
-31.738	-30.4639424878392\\
-18.311	-17.5759421165424\\
-12.207	-11.7169747920175\\
-15.869	-15.2319712439196\\
-14.648	-14.0599858076082\\
-13.428	-12.888960228329\\
-10.986	-10.5449893557061\\
-12.207	-11.7169747920175\\
-9.766	-9.37396377642692\\
-12.207	-11.7169747920175\\
-10.986	-10.5449893557061\\
-13.428	-12.888960228329\\
-20.752	-19.9189531321331\\
-24.414	-23.4339495840351\\
-30.518	-29.29291690856\\
-31.738	-30.4639424878392\\
-19.531	-18.7469676958216\\
-17.09	-16.403956680231\\
-21.973	-21.0909385684445\\
-14.648	-14.0599858076082\\
-13.428	-12.888960228329\\
-10.986	-10.5449893557061\\
-14.648	-14.0599858076082\\
-8.545	-8.2019783401155\\
-4.883	-4.68698188821346\\
-6.104	-5.85896732452487\\
-8.545	-8.2019783401155\\
-14.648	-14.0599858076082\\
-9.766	-9.37396377642692\\
-13.428	-12.888960228329\\
-17.09	-16.403956680231\\
-20.752	-19.9189531321331\\
-17.09	-16.403956680231\\
-14.648	-14.0599858076082\\
-17.09	-16.403956680231\\
-20.752	-19.9189531321331\\
-18.311	-17.5759421165424\\
-20.752	-19.9189531321331\\
-13.428	-12.888960228329\\
-12.207	-11.7169747920175\\
-15.869	-15.2319712439196\\
-13.428	-12.888960228329\\
-14.648	-14.0599858076082\\
-20.752	-19.9189531321331\\
-30.518	-29.29291690856\\
-24.414	-23.4339495840351\\
-20.752	-19.9189531321331\\
-31.738	-30.4639424878392\\
-23.193	-22.2619641477237\\
-15.869	-15.2319712439196\\
-14.648	-14.0599858076082\\
-9.766	-9.37396377642692\\
-7.324	-7.02999290380409\\
-17.09	-16.403956680231\\
-13.428	-12.888960228329\\
-7.324	-7.02999290380409\\
-13.428	-12.888960228329\\
-14.648	-14.0599858076082\\
-10.986	-10.5449893557061\\
-9.766	-9.37396377642692\\
-4.883	-4.68698188821346\\
-6.104	-5.85896732452487\\
-17.09	-16.403956680231\\
-20.752	-19.9189531321331\\
-15.869	-15.2319712439196\\
-12.207	-11.7169747920175\\
-17.09	-16.403956680231\\
-18.311	-17.5759421165424\\
-20.752	-19.9189531321331\\
-21.973	-21.0909385684445\\
-26.855	-25.7769605996257\\
-21.973	-21.0909385684445\\
-15.869	-15.2319712439196\\
-12.207	-11.7169747920175\\
-13.428	-12.888960228329\\
-15.869	-15.2319712439196\\
-13.428	-12.888960228329\\
-8.545	-8.2019783401155\\
-18.311	-17.5759421165424\\
-23.193	-22.2619641477237\\
-20.752	-19.9189531321331\\
-24.414	-23.4339495840351\\
-30.518	-29.29291690856\\
-20.752	-19.9189531321331\\
-23.193	-22.2619641477237\\
-30.518	-29.29291690856\\
-24.414	-23.4339495840351\\
-29.297	-28.1209314722486\\
-45.166	-43.3529027161681\\
-42.725	-41.0098917005775\\
-30.518	-29.29291690856\\
-32.959	-31.6359279241506\\
-42.725	-41.0098917005775\\
-41.504	-39.8379062642661\\
-46.387	-44.5248881524796\\
-43.945	-42.1809172798567\\
-53.711	-51.5548810562836\\
-46.387	-44.5248881524796\\
-29.297	-28.1209314722486\\
-19.531	-18.7469676958216\\
-20.752	-19.9189531321331\\
-14.648	-14.0599858076082\\
-19.531	-18.7469676958216\\
-29.297	-28.1209314722486\\
-20.752	-19.9189531321331\\
-14.648	-14.0599858076082\\
-17.09	-16.403956680231\\
-12.207	-11.7169747920175\\
-3.662	-3.51499645190204\\
-9.766	-9.37396377642692\\
-10.986	-10.5449893557061\\
-6.104	-5.85896732452487\\
-4.883	-4.68698188821346\\
-3.662	-3.51499645190204\\
-7.324	-7.02999290380409\\
-6.104	-5.85896732452487\\
-13.428	-12.888960228329\\
-18.311	-17.5759421165424\\
-17.09	-16.403956680231\\
-13.428	-12.888960228329\\
-15.869	-15.2319712439196\\
-26.855	-25.7769605996257\\
-18.311	-17.5759421165424\\
-4.883	-4.68698188821346\\
-7.324	-7.02999290380409\\
-4.883	-4.68698188821346\\
-15.869	-15.2319712439196\\
-18.311	-17.5759421165424\\
-12.207	-11.7169747920175\\
-18.311	-17.5759421165424\\
-29.297	-28.1209314722486\\
-25.635	-24.6059350203465\\
-18.311	-17.5759421165424\\
-13.428	-12.888960228329\\
-14.648	-14.0599858076082\\
-23.193	-22.2619641477237\\
-26.855	-25.7769605996257\\
-18.311	-17.5759421165424\\
-10.986	-10.5449893557061\\
-12.207	-11.7169747920175\\
-4.883	-4.68698188821346\\
-3.662	-3.51499645190204\\
-7.324	-7.02999290380409\\
-20.752	-19.9189531321331\\
-13.428	-12.888960228329\\
-10.986	-10.5449893557061\\
-4.883	-4.68698188821346\\
-2.441	-2.34301101559063\\
-7.324	-7.02999290380409\\
-15.869	-15.2319712439196\\
-21.973	-21.0909385684445\\
-23.193	-22.2619641477237\\
-21.973	-21.0909385684445\\
-23.193	-22.2619641477237\\
-24.414	-23.4339495840351\\
-19.531	-18.7469676958216\\
-25.635	-24.6059350203465\\
-35.4	-33.9789389397412\\
-34.18	-32.807913360462\\
-18.311	-17.5759421165424\\
-8.545	-8.2019783401155\\
-14.648	-14.0599858076082\\
-28.076	-26.9489460359371\\
-21.973	-21.0909385684445\\
-15.869	-15.2319712439196\\
-13.428	-12.888960228329\\
-24.414	-23.4339495840351\\
-34.18	-32.807913360462\\
-37.842	-36.3229098123641\\
-39.063	-37.4948952486755\\
-29.297	-28.1209314722486\\
-28.076	-26.9489460359371\\
-18.311	-17.5759421165424\\
-12.207	-11.7169747920175\\
-18.311	-17.5759421165424\\
-19.531	-18.7469676958216\\
-20.752	-19.9189531321331\\
-23.193	-22.2619641477237\\
-15.869	-15.2319712439196\\
-17.09	-16.403956680231\\
-20.752	-19.9189531321331\\
-9.766	-9.37396377642692\\
-6.104	-5.85896732452487\\
-8.545	-8.2019783401155\\
-6.104	-5.85896732452487\\
-1.221	-1.17198543631141\\
-4.883	-4.68698188821346\\
-17.09	-16.403956680231\\
-13.428	-12.888960228329\\
-10.986	-10.5449893557061\\
-15.869	-15.2319712439196\\
-24.414	-23.4339495840351\\
-20.752	-19.9189531321331\\
-7.324	-7.02999290380409\\
-6.104	-5.85896732452487\\
-8.545	-8.2019783401155\\
-26.855	-25.7769605996257\\
-31.738	-30.4639424878392\\
-41.504	-39.8379062642661\\
-50.049	-48.0398846043816\\
-47.607	-45.6959137317588\\
-34.18	-32.807913360462\\
-43.945	-42.1809172798567\\
-36.621	-35.1509243760527\\
-32.959	-31.6359279241506\\
-37.842	-36.3229098123641\\
-41.504	-39.8379062642661\\
-56.152	-53.8978920718743\\
-41.504	-39.8379062642661\\
-21.973	-21.0909385684445\\
-13.428	-12.888960228329\\
-9.766	-9.37396377642692\\
-12.207	-11.7169747920175\\
-10.986	-10.5449893557061\\
-20.752	-19.9189531321331\\
-18.311	-17.5759421165424\\
-19.531	-18.7469676958216\\
-20.752	-19.9189531321331\\
-14.648	-14.0599858076082\\
-13.428	-12.888960228329\\
-21.973	-21.0909385684445\\
-23.193	-22.2619641477237\\
-13.428	-12.888960228329\\
-7.324	-7.02999290380409\\
-9.766	-9.37396377642692\\
-12.207	-11.7169747920175\\
-29.297	-28.1209314722486\\
-35.4	-33.9789389397412\\
-32.959	-31.6359279241506\\
-35.4	-33.9789389397412\\
-43.945	-42.1809172798567\\
-52.49	-50.3828956199722\\
-46.387	-44.5248881524796\\
-30.518	-29.29291690856\\
-19.531	-18.7469676958216\\
-13.428	-12.888960228329\\
-14.648	-14.0599858076082\\
-17.09	-16.403956680231\\
-9.766	-9.37396377642692\\
-12.207	-11.7169747920175\\
-13.428	-12.888960228329\\
-15.869	-15.2319712439196\\
-19.531	-18.7469676958216\\
-20.752	-19.9189531321331\\
-18.311	-17.5759421165424\\
-8.545	-8.2019783401155\\
-4.883	-4.68698188821346\\
-3.662	-3.51499645190204\\
-6.104	-5.85896732452487\\
-10.986	-10.5449893557061\\
-14.648	-14.0599858076082\\
-13.428	-12.888960228329\\
-23.193	-22.2619641477237\\
-18.311	-17.5759421165424\\
-23.193	-22.2619641477237\\
-40.283	-38.6659208279547\\
-30.518	-29.29291690856\\
-25.635	-24.6059350203465\\
-32.959	-31.6359279241506\\
-36.621	-35.1509243760527\\
-26.855	-25.7769605996257\\
-21.973	-21.0909385684445\\
-15.869	-15.2319712439196\\
-30.518	-29.29291690856\\
-48.828	-46.8678991680702\\
-57.373	-55.0698775081857\\
-46.387	-44.5248881524796\\
-39.063	-37.4948952486755\\
-30.518	-29.29291690856\\
-37.842	-36.3229098123641\\
-28.076	-26.9489460359371\\
-19.531	-18.7469676958216\\
-25.635	-24.6059350203465\\
-29.297	-28.1209314722486\\
-15.869	-15.2319712439196\\
-14.648	-14.0599858076082\\
-24.414	-23.4339495840351\\
-28.076	-26.9489460359371\\
-29.297	-28.1209314722486\\
-20.752	-19.9189531321331\\
-17.09	-16.403956680231\\
-15.869	-15.2319712439196\\
-10.986	-10.5449893557061\\
-9.766	-9.37396377642692\\
-8.545	-8.2019783401155\\
-6.104	-5.85896732452487\\
-7.324	-7.02999290380409\\
-12.207	-11.7169747920175\\
-18.311	-17.5759421165424\\
-13.428	-12.888960228329\\
-9.766	-9.37396377642692\\
-8.545	-8.2019783401155\\
-10.986	-10.5449893557061\\
-15.869	-15.2319712439196\\
-18.311	-17.5759421165424\\
-15.869	-15.2319712439196\\
-10.986	-10.5449893557061\\
-24.414	-23.4339495840351\\
-26.855	-25.7769605996257\\
-20.752	-19.9189531321331\\
-8.545	-8.2019783401155\\
-13.428	-12.888960228329\\
-17.09	-16.403956680231\\
-12.207	-11.7169747920175\\
-8.545	-8.2019783401155\\
-12.207	-11.7169747920175\\
-26.855	-25.7769605996257\\
-13.428	-12.888960228329\\
-7.324	-7.02999290380409\\
-9.766	-9.37396377642692\\
-18.311	-17.5759421165424\\
-21.973	-21.0909385684445\\
-12.207	-11.7169747920175\\
-10.986	-10.5449893557061\\
-20.752	-19.9189531321331\\
-29.297	-28.1209314722486\\
-25.635	-24.6059350203465\\
-37.842	-36.3229098123641\\
-40.283	-38.6659208279547\\
-30.518	-29.29291690856\\
-23.193	-22.2619641477237\\
-30.518	-29.29291690856\\
-25.635	-24.6059350203465\\
-26.855	-25.7769605996257\\
-36.621	-35.1509243760527\\
-28.076	-26.9489460359371\\
-14.648	-14.0599858076082\\
-25.635	-24.6059350203465\\
-36.621	-35.1509243760527\\
-41.504	-39.8379062642661\\
-31.738	-30.4639424878392\\
-28.076	-26.9489460359371\\
-34.18	-32.807913360462\\
-30.518	-29.29291690856\\
-17.09	-16.403956680231\\
-21.973	-21.0909385684445\\
-23.193	-22.2619641477237\\
-20.752	-19.9189531321331\\
-24.414	-23.4339495840351\\
-15.869	-15.2319712439196\\
-19.531	-18.7469676958216\\
-28.076	-26.9489460359371\\
-19.531	-18.7469676958216\\
-13.428	-12.888960228329\\
-10.986	-10.5449893557061\\
-8.545	-8.2019783401155\\
-17.09	-16.403956680231\\
-18.311	-17.5759421165424\\
-19.531	-18.7469676958216\\
-26.855	-25.7769605996257\\
-31.738	-30.4639424878392\\
-28.076	-26.9489460359371\\
-21.973	-21.0909385684445\\
-13.428	-12.888960228329\\
-15.869	-15.2319712439196\\
-29.297	-28.1209314722486\\
-20.752	-19.9189531321331\\
-7.324	-7.02999290380409\\
-17.09	-16.403956680231\\
-25.635	-24.6059350203465\\
-28.076	-26.9489460359371\\
-35.4	-33.9789389397412\\
-46.387	-44.5248881524796\\
-36.621	-35.1509243760527\\
-29.297	-28.1209314722486\\
-34.18	-32.807913360462\\
-25.635	-24.6059350203465\\
-18.311	-17.5759421165424\\
-21.973	-21.0909385684445\\
-29.297	-28.1209314722486\\
-28.076	-26.9489460359371\\
-18.311	-17.5759421165424\\
-14.648	-14.0599858076082\\
-12.207	-11.7169747920175\\
-17.09	-16.403956680231\\
-18.311	-17.5759421165424\\
-13.428	-12.888960228329\\
-24.414	-23.4339495840351\\
-30.518	-29.29291690856\\
-18.311	-17.5759421165424\\
-13.428	-12.888960228329\\
-23.193	-22.2619641477237\\
-14.648	-14.0599858076082\\
-6.104	-5.85896732452487\\
-12.207	-11.7169747920175\\
-13.428	-12.888960228329\\
-10.986	-10.5449893557061\\
-7.324	-7.02999290380409\\
-6.104	-5.85896732452487\\
-7.324	-7.02999290380409\\
-10.986	-10.5449893557061\\
-20.752	-19.9189531321331\\
-25.635	-24.6059350203465\\
-29.297	-28.1209314722486\\
-32.959	-31.6359279241506\\
-18.311	-17.5759421165424\\
-12.207	-11.7169747920175\\
-28.076	-26.9489460359371\\
-41.504	-39.8379062642661\\
-31.738	-30.4639424878392\\
-29.297	-28.1209314722486\\
-45.166	-43.3529027161681\\
-34.18	-32.807913360462\\
-30.518	-29.29291690856\\
-25.635	-24.6059350203465\\
-23.193	-22.2619641477237\\
-31.738	-30.4639424878392\\
-24.414	-23.4339495840351\\
-20.752	-19.9189531321331\\
-30.518	-29.29291690856\\
-40.283	-38.6659208279547\\
-15.869	-15.2319712439196\\
-9.766	-9.37396377642692\\
-28.076	-26.9489460359371\\
-20.752	-19.9189531321331\\
-8.545	-8.2019783401155\\
-14.648	-14.0599858076082\\
-19.531	-18.7469676958216\\
-28.076	-26.9489460359371\\
-18.311	-17.5759421165424\\
-13.428	-12.888960228329\\
-10.986	-10.5449893557061\\
-6.104	-5.85896732452487\\
-8.545	-8.2019783401155\\
-6.104	-5.85896732452487\\
-7.324	-7.02999290380409\\
-17.09	-16.403956680231\\
-14.648	-14.0599858076082\\
-6.104	-5.85896732452487\\
-8.545	-8.2019783401155\\
-10.986	-10.5449893557061\\
-8.545	-8.2019783401155\\
-9.766	-9.37396377642692\\
-2.441	-2.34301101559063\\
-12.207	-11.7169747920175\\
-18.311	-17.5759421165424\\
-24.414	-23.4339495840351\\
-34.18	-32.807913360462\\
-30.518	-29.29291690856\\
-14.648	-14.0599858076082\\
-12.207	-11.7169747920175\\
-3.662	-3.51499645190204\\
-4.883	-4.68698188821346\\
-13.428	-12.888960228329\\
-18.311	-17.5759421165424\\
-14.648	-14.0599858076082\\
-18.311	-17.5759421165424\\
-19.531	-18.7469676958216\\
-25.635	-24.6059350203465\\
-20.752	-19.9189531321331\\
-34.18	-32.807913360462\\
-18.311	-17.5759421165424\\
-12.207	-11.7169747920175\\
-8.545	-8.2019783401155\\
-15.869	-15.2319712439196\\
-13.428	-12.888960228329\\
-9.766	-9.37396377642692\\
-3.662	-3.51499645190204\\
-10.986	-10.5449893557061\\
-8.545	-8.2019783401155\\
-2.441	-2.34301101559063\\
-7.324	-7.02999290380409\\
-9.766	-9.37396377642692\\
-14.648	-14.0599858076082\\
-19.531	-18.7469676958216\\
-26.855	-25.7769605996257\\
-21.973	-21.0909385684445\\
-7.324	-7.02999290380409\\
-13.428	-12.888960228329\\
-21.973	-21.0909385684445\\
-10.986	-10.5449893557061\\
-8.545	-8.2019783401155\\
-21.973	-21.0909385684445\\
-18.311	-17.5759421165424\\
-28.076	-26.9489460359371\\
-36.621	-35.1509243760527\\
-23.193	-22.2619641477237\\
-18.311	-17.5759421165424\\
};
\end{axis}

\begin{axis}[%
width=4.927cm,
height=3cm,
at={(7cm,14.516cm)},
scale only axis,
xmin=-60,
xmax=0,
xlabel style={font=\color{white!15!black}},
xlabel={y(t-1)},
ymin=-50,
ymax=0,
ylabel style={font=\color{white!15!black}},
ylabel={y(t)},
axis background/.style={fill=white},
title style={font=\small},
title={C2, R = 0.7625},
axis x line*=bottom,
axis y line*=left
]
\addplot[only marks, mark=*, mark options={}, mark size=1.5000pt, color=mycolor1, fill=mycolor1] table[row sep=crcr]{%
x	y\\
-18.311	-19.531\\
-19.531	-24.414\\
-24.414	-19.531\\
-19.531	-20.752\\
-20.752	-25.635\\
-25.635	-24.414\\
-24.414	-28.076\\
-28.076	-23.193\\
-23.193	-13.428\\
-13.428	-17.09\\
-17.09	-17.09\\
-17.09	-18.311\\
-18.311	-15.869\\
-15.869	-8.545\\
-8.545	-6.104\\
-6.104	-7.324\\
-7.324	-9.766\\
-9.766	-21.973\\
-21.973	-23.193\\
-23.193	-23.193\\
-23.193	-19.531\\
-19.531	-10.986\\
-10.986	-17.09\\
-17.09	-19.531\\
-19.531	-13.428\\
-13.428	-18.311\\
-18.311	-19.531\\
-19.531	-14.648\\
-14.648	-24.414\\
-24.414	-21.973\\
-21.973	-15.869\\
-15.869	-23.193\\
-23.193	-25.635\\
-25.635	-19.531\\
-19.531	-17.09\\
-17.09	-14.648\\
-14.648	-17.09\\
-17.09	-20.752\\
-20.752	-18.311\\
-18.311	-18.311\\
-18.311	-19.531\\
-19.531	-20.752\\
-20.752	-28.076\\
-28.076	-25.635\\
-25.635	-17.09\\
-17.09	-15.869\\
-15.869	-12.207\\
-12.207	-10.986\\
-10.986	-15.869\\
-15.869	-12.207\\
-12.207	-12.207\\
-12.207	-15.869\\
-15.869	-18.311\\
-18.311	-15.869\\
-15.869	-25.635\\
-25.635	-21.973\\
-21.973	-12.207\\
-12.207	-9.766\\
-9.766	-13.428\\
-13.428	-15.869\\
-15.869	-10.986\\
-10.986	-10.986\\
-10.986	-12.207\\
-12.207	-9.766\\
-9.766	-8.545\\
-8.545	-12.207\\
-12.207	-14.648\\
-14.648	-17.09\\
-17.09	-18.311\\
-18.311	-23.193\\
-23.193	-21.973\\
-21.973	-21.973\\
-21.973	-17.09\\
-17.09	-17.09\\
-17.09	-17.09\\
-17.09	-15.869\\
-15.869	-17.09\\
-17.09	-24.414\\
-24.414	-35.4\\
-35.4	-36.621\\
-36.621	-34.18\\
-34.18	-24.414\\
-24.414	-29.297\\
-29.297	-36.621\\
-36.621	-40.283\\
-40.283	-31.738\\
-31.738	-37.842\\
-37.842	-48.828\\
-48.828	-39.063\\
-39.063	-30.518\\
-30.518	-25.635\\
-25.635	-19.531\\
-19.531	-14.648\\
-14.648	-19.531\\
-19.531	-15.869\\
-15.869	-9.766\\
-9.766	-9.766\\
-9.766	-12.207\\
-12.207	-17.09\\
-17.09	-15.869\\
-15.869	-15.869\\
-15.869	-19.531\\
-19.531	-19.531\\
-19.531	-28.076\\
-28.076	-23.193\\
-23.193	-25.635\\
-25.635	-23.193\\
-23.193	-18.311\\
-18.311	-20.752\\
-20.752	-17.09\\
-17.09	-14.648\\
-14.648	-18.311\\
-18.311	-17.09\\
-17.09	-15.869\\
-15.869	-13.428\\
-13.428	-10.986\\
-10.986	-12.207\\
-12.207	-13.428\\
-13.428	-10.986\\
-10.986	-9.766\\
-9.766	-14.648\\
-14.648	-19.531\\
-19.531	-23.193\\
-23.193	-23.193\\
-23.193	-17.09\\
-17.09	-18.311\\
-18.311	-30.518\\
-30.518	-23.193\\
-23.193	-24.414\\
-24.414	-26.855\\
-26.855	-17.09\\
-17.09	-10.986\\
-10.986	-17.09\\
-17.09	-18.311\\
-18.311	-26.855\\
-26.855	-34.18\\
-34.18	-31.738\\
-31.738	-25.635\\
-25.635	-24.414\\
-24.414	-23.193\\
-23.193	-23.193\\
-23.193	-21.973\\
-21.973	-18.311\\
-18.311	-14.648\\
-14.648	-13.428\\
-13.428	-12.207\\
-12.207	-13.428\\
-13.428	-19.531\\
-19.531	-25.635\\
-25.635	-21.973\\
-21.973	-14.648\\
-14.648	-13.428\\
-13.428	-14.648\\
-14.648	-12.207\\
-12.207	-8.545\\
-8.545	-8.545\\
-8.545	-13.428\\
-13.428	-12.207\\
-12.207	-12.207\\
-12.207	-12.207\\
-12.207	-12.207\\
-12.207	-8.545\\
-8.545	-7.324\\
-7.324	-4.883\\
-4.883	-7.324\\
-7.324	-17.09\\
-17.09	-24.414\\
-24.414	-24.414\\
-24.414	-19.531\\
-19.531	-12.207\\
-12.207	-12.207\\
-12.207	-8.545\\
-8.545	-10.986\\
-10.986	-14.648\\
-14.648	-14.648\\
-14.648	-24.414\\
-24.414	-34.18\\
-34.18	-40.283\\
-40.283	-32.959\\
-32.959	-31.738\\
-31.738	-23.193\\
-23.193	-25.635\\
-25.635	-26.855\\
-26.855	-28.076\\
-28.076	-21.973\\
-21.973	-20.752\\
-20.752	-19.531\\
-19.531	-19.531\\
-19.531	-21.973\\
-21.973	-19.531\\
-19.531	-23.193\\
-23.193	-29.297\\
-29.297	-24.414\\
-24.414	-14.648\\
-14.648	-15.869\\
-15.869	-25.635\\
-25.635	-32.959\\
-32.959	-35.4\\
-35.4	-39.063\\
-39.063	-37.842\\
-37.842	-30.518\\
-30.518	-26.855\\
-26.855	-30.518\\
-30.518	-35.4\\
-35.4	-25.635\\
-25.635	-14.648\\
-14.648	-19.531\\
-19.531	-20.752\\
-20.752	-12.207\\
-12.207	-13.428\\
-13.428	-18.311\\
-18.311	-14.648\\
-14.648	-12.207\\
-12.207	-18.311\\
-18.311	-10.986\\
-10.986	-14.648\\
-14.648	-24.414\\
-24.414	-21.973\\
-21.973	-14.648\\
-14.648	-14.648\\
-14.648	-20.752\\
-20.752	-34.18\\
-34.18	-28.076\\
-28.076	-15.869\\
-15.869	-14.648\\
-14.648	-14.648\\
-14.648	-12.207\\
-12.207	-10.986\\
-10.986	-10.986\\
-10.986	-13.428\\
-13.428	-17.09\\
-17.09	-19.531\\
-19.531	-14.648\\
-14.648	-19.531\\
-19.531	-26.855\\
-26.855	-23.193\\
-23.193	-17.09\\
-17.09	-21.973\\
-21.973	-25.635\\
-25.635	-21.973\\
-21.973	-8.545\\
-8.545	-12.207\\
-12.207	-13.428\\
-13.428	-15.869\\
-15.869	-15.869\\
-15.869	-10.986\\
-10.986	-17.09\\
-17.09	-15.869\\
-15.869	-10.986\\
-10.986	-14.648\\
-14.648	-14.648\\
-14.648	-12.207\\
-12.207	-14.648\\
-14.648	-10.986\\
-10.986	-9.766\\
-9.766	-12.207\\
-12.207	-8.545\\
-8.545	-17.09\\
-17.09	-18.311\\
-18.311	-20.752\\
-20.752	-29.297\\
-29.297	-20.752\\
-20.752	-10.986\\
-10.986	-10.986\\
-10.986	-8.545\\
-8.545	-13.428\\
-13.428	-12.207\\
-12.207	-9.766\\
-9.766	-15.869\\
-15.869	-19.531\\
-19.531	-20.752\\
-20.752	-23.193\\
-23.193	-23.193\\
-23.193	-17.09\\
-17.09	-15.869\\
-15.869	-10.986\\
-10.986	-10.986\\
-10.986	-7.324\\
-7.324	-8.545\\
-8.545	-10.986\\
-10.986	-17.09\\
-17.09	-18.311\\
-18.311	-17.09\\
-17.09	-24.414\\
-24.414	-25.635\\
-25.635	-15.869\\
-15.869	-19.531\\
-19.531	-23.193\\
-23.193	-26.855\\
-26.855	-25.635\\
-25.635	-17.09\\
-17.09	-10.986\\
-10.986	-7.324\\
-7.324	-6.104\\
-6.104	-8.545\\
-8.545	-9.766\\
-9.766	-8.545\\
-8.545	-10.986\\
-10.986	-17.09\\
-17.09	-15.869\\
-15.869	-13.428\\
-13.428	-15.869\\
-15.869	-12.207\\
-12.207	-6.104\\
-6.104	-9.766\\
-9.766	-20.752\\
-20.752	-17.09\\
-17.09	-18.311\\
-18.311	-15.869\\
-15.869	-10.986\\
-10.986	-17.09\\
-17.09	-23.193\\
-23.193	-23.193\\
-23.193	-17.09\\
-17.09	-26.855\\
-26.855	-25.635\\
-25.635	-18.311\\
-18.311	-23.193\\
-23.193	-25.635\\
-25.635	-19.531\\
-19.531	-15.869\\
-15.869	-24.414\\
-24.414	-35.4\\
-35.4	-29.297\\
-29.297	-17.09\\
-17.09	-17.09\\
-17.09	-18.311\\
-18.311	-20.752\\
-20.752	-23.193\\
-23.193	-19.531\\
-19.531	-17.09\\
-17.09	-24.414\\
-24.414	-25.635\\
-25.635	-18.311\\
-18.311	-15.869\\
-15.869	-18.311\\
-18.311	-28.076\\
-28.076	-26.855\\
-26.855	-24.414\\
-24.414	-19.531\\
-19.531	-17.09\\
-17.09	-18.311\\
-18.311	-13.428\\
-13.428	-6.104\\
-6.104	-6.104\\
-6.104	-6.104\\
-6.104	-9.766\\
-9.766	-19.531\\
-19.531	-15.869\\
-15.869	-14.648\\
-14.648	-13.428\\
-13.428	-17.09\\
-17.09	-13.428\\
-13.428	-9.766\\
-9.766	-12.207\\
-12.207	-9.766\\
-9.766	-8.545\\
-8.545	-17.09\\
-17.09	-20.752\\
-20.752	-17.09\\
-17.09	-10.986\\
-10.986	-10.986\\
-10.986	-13.428\\
-13.428	-9.766\\
-9.766	-10.986\\
-10.986	-28.076\\
-28.076	-20.752\\
-20.752	-25.635\\
-25.635	-35.4\\
-35.4	-30.518\\
-30.518	-24.414\\
-24.414	-18.311\\
-18.311	-18.311\\
-18.311	-19.531\\
-19.531	-21.973\\
-21.973	-25.635\\
-25.635	-25.635\\
-25.635	-31.738\\
-31.738	-34.18\\
-34.18	-41.504\\
-41.504	-32.959\\
-32.959	-28.076\\
-28.076	-35.4\\
-35.4	-26.855\\
-26.855	-29.297\\
-29.297	-28.076\\
-28.076	-17.09\\
-17.09	-10.986\\
-10.986	-12.207\\
-12.207	-17.09\\
-17.09	-14.648\\
-14.648	-9.766\\
-9.766	-13.428\\
-13.428	-13.428\\
-13.428	-7.324\\
-7.324	-8.545\\
-8.545	-12.207\\
-12.207	-7.324\\
-7.324	-14.648\\
-14.648	-20.752\\
-20.752	-13.428\\
-13.428	-21.973\\
-21.973	-30.518\\
-30.518	-29.297\\
-29.297	-35.4\\
-35.4	-29.297\\
-29.297	-28.076\\
-28.076	-23.193\\
-23.193	-12.207\\
-12.207	-14.648\\
-14.648	-15.869\\
-15.869	-10.986\\
-10.986	-12.207\\
-12.207	-13.428\\
-13.428	-3.662\\
-3.662	-8.545\\
-8.545	-17.09\\
-17.09	-19.531\\
-19.531	-20.752\\
-20.752	-23.193\\
-23.193	-20.752\\
-20.752	-23.193\\
-23.193	-18.311\\
-18.311	-15.869\\
-15.869	-15.869\\
-15.869	-12.207\\
-12.207	-18.311\\
-18.311	-17.09\\
-17.09	-12.207\\
-12.207	-17.09\\
-17.09	-23.193\\
-23.193	-17.09\\
-17.09	-13.428\\
-13.428	-12.207\\
-12.207	-14.648\\
-14.648	-7.324\\
-7.324	-8.545\\
-8.545	-12.207\\
-12.207	-17.09\\
-17.09	-17.09\\
-17.09	-14.648\\
-14.648	-15.869\\
-15.869	-18.311\\
-18.311	-12.207\\
-12.207	-9.766\\
-9.766	-18.311\\
-18.311	-26.855\\
-26.855	-26.855\\
-26.855	-21.973\\
-21.973	-19.531\\
-19.531	-17.09\\
-17.09	-14.648\\
-14.648	-13.428\\
-13.428	-17.09\\
-17.09	-17.09\\
-17.09	-10.986\\
-10.986	-14.648\\
-14.648	-18.311\\
-18.311	-19.531\\
-19.531	-21.973\\
-21.973	-21.973\\
-21.973	-29.297\\
-29.297	-35.4\\
-35.4	-36.621\\
-36.621	-26.855\\
-26.855	-14.648\\
-14.648	-10.986\\
-10.986	-7.324\\
-7.324	-10.986\\
-10.986	-10.986\\
-10.986	-14.648\\
-14.648	-13.428\\
-13.428	-12.207\\
-12.207	-15.869\\
-15.869	-20.752\\
-20.752	-20.752\\
-20.752	-24.414\\
-24.414	-31.738\\
-31.738	-29.297\\
-29.297	-26.855\\
-26.855	-18.311\\
-18.311	-17.09\\
-17.09	-20.752\\
-20.752	-20.752\\
-20.752	-17.09\\
-17.09	-14.648\\
-14.648	-18.311\\
-18.311	-12.207\\
-12.207	-13.428\\
-13.428	-20.752\\
-20.752	-17.09\\
-17.09	-18.311\\
-18.311	-32.959\\
-32.959	-25.635\\
-25.635	-20.752\\
-20.752	-28.076\\
-28.076	-28.076\\
-28.076	-19.531\\
-19.531	-13.428\\
-13.428	-13.428\\
-13.428	-19.531\\
-19.531	-30.518\\
-30.518	-34.18\\
-34.18	-31.738\\
-31.738	-25.635\\
-25.635	-17.09\\
-17.09	-14.648\\
-14.648	-13.428\\
-13.428	-10.986\\
-10.986	-8.545\\
-8.545	-12.207\\
-12.207	-14.648\\
-14.648	-10.986\\
-10.986	-13.428\\
-13.428	-19.531\\
-19.531	-20.752\\
-20.752	-21.973\\
-21.973	-29.297\\
-29.297	-34.18\\
-34.18	-25.635\\
-25.635	-23.193\\
-23.193	-24.414\\
-24.414	-20.752\\
-20.752	-14.648\\
-14.648	-18.311\\
-18.311	-29.297\\
-29.297	-31.738\\
-31.738	-19.531\\
-19.531	-12.207\\
-12.207	-8.545\\
-8.545	-7.324\\
-7.324	-9.766\\
-9.766	-8.545\\
-8.545	-12.207\\
-12.207	-14.648\\
-14.648	-13.428\\
-13.428	-9.766\\
-9.766	-10.986\\
-10.986	-10.986\\
-10.986	-9.766\\
-9.766	-10.986\\
-10.986	-14.648\\
-14.648	-21.973\\
-21.973	-18.311\\
-18.311	-20.752\\
-20.752	-21.973\\
-21.973	-30.518\\
-30.518	-31.738\\
-31.738	-34.18\\
-34.18	-34.18\\
-34.18	-24.414\\
-24.414	-17.09\\
-17.09	-15.869\\
-15.869	-26.855\\
-26.855	-31.738\\
-31.738	-23.193\\
-23.193	-18.311\\
-18.311	-9.766\\
-9.766	-4.883\\
-4.883	-6.104\\
-6.104	-6.104\\
-6.104	-7.324\\
-7.324	-14.648\\
-14.648	-10.986\\
-10.986	-12.207\\
-12.207	-8.545\\
-8.545	-4.883\\
-4.883	-8.545\\
-8.545	-8.545\\
-8.545	-8.545\\
-8.545	-9.766\\
-9.766	-7.324\\
-7.324	-10.986\\
-10.986	-24.414\\
-24.414	-26.855\\
-26.855	-25.635\\
-25.635	-21.973\\
-21.973	-30.518\\
-30.518	-40.283\\
-40.283	-34.18\\
-34.18	-31.738\\
-31.738	-26.855\\
-26.855	-28.076\\
-28.076	-29.297\\
-29.297	-20.752\\
-20.752	-17.09\\
-17.09	-14.648\\
-14.648	-12.207\\
-12.207	-20.752\\
-20.752	-18.311\\
-18.311	-15.869\\
-15.869	-18.311\\
-18.311	-18.311\\
-18.311	-14.648\\
-14.648	-17.09\\
-17.09	-17.09\\
-17.09	-20.752\\
-20.752	-19.531\\
-19.531	-13.428\\
-13.428	-17.09\\
-17.09	-9.766\\
-9.766	-4.883\\
-4.883	-10.986\\
-10.986	-13.428\\
-13.428	-8.545\\
-8.545	-2.441\\
-2.441	-8.545\\
-8.545	-12.207\\
-12.207	-12.207\\
-12.207	-15.869\\
-15.869	-13.428\\
-13.428	-18.311\\
-18.311	-17.09\\
-17.09	-10.986\\
-10.986	-17.09\\
-17.09	-14.648\\
-14.648	-9.766\\
-9.766	-7.324\\
-7.324	-6.104\\
-6.104	-7.324\\
-7.324	-8.545\\
-8.545	-6.104\\
-6.104	-13.428\\
-13.428	-17.09\\
-17.09	-10.986\\
-10.986	-14.648\\
-14.648	-10.986\\
-10.986	-14.648\\
-14.648	-10.986\\
-10.986	-9.766\\
-9.766	-9.766\\
-9.766	-7.324\\
-7.324	-8.545\\
-8.545	-10.986\\
-10.986	-14.648\\
-14.648	-12.207\\
-12.207	-8.545\\
-8.545	-8.545\\
-8.545	-8.545\\
-8.545	-10.986\\
-10.986	-21.973\\
-21.973	-24.414\\
-24.414	-18.311\\
-18.311	-17.09\\
-17.09	-13.428\\
-13.428	-19.531\\
-19.531	-17.09\\
-17.09	-9.766\\
-9.766	-14.648\\
-14.648	-13.428\\
-13.428	-15.869\\
-15.869	-12.207\\
-12.207	-14.648\\
-14.648	-12.207\\
-12.207	-15.869\\
-15.869	-12.207\\
-12.207	-13.428\\
-13.428	-14.648\\
-14.648	-19.531\\
-19.531	-18.311\\
-18.311	-21.973\\
-21.973	-30.518\\
-30.518	-24.414\\
-24.414	-14.648\\
-14.648	-14.648\\
-14.648	-15.869\\
-15.869	-23.193\\
-23.193	-18.311\\
-18.311	-23.193\\
-23.193	-17.09\\
-17.09	-8.545\\
-8.545	-9.766\\
-9.766	-19.531\\
-19.531	-14.648\\
-14.648	-13.428\\
-13.428	-20.752\\
-20.752	-21.973\\
-21.973	-18.311\\
-18.311	-14.648\\
-14.648	-18.311\\
-18.311	-20.752\\
-20.752	-18.311\\
-18.311	-15.869\\
-15.869	-15.869\\
-15.869	-12.207\\
-12.207	-14.648\\
-14.648	-12.207\\
-12.207	-13.428\\
-13.428	-19.531\\
-19.531	-20.752\\
-20.752	-19.531\\
-19.531	-19.531\\
-19.531	-14.648\\
-14.648	-10.986\\
-10.986	-13.428\\
-13.428	-14.648\\
-14.648	-18.311\\
-18.311	-20.752\\
-20.752	-24.414\\
-24.414	-18.311\\
-18.311	-12.207\\
-12.207	-21.973\\
-21.973	-30.518\\
-30.518	-20.752\\
-20.752	-20.752\\
-20.752	-25.635\\
-25.635	-18.311\\
-18.311	-17.09\\
-17.09	-18.311\\
-18.311	-15.869\\
-15.869	-18.311\\
-18.311	-21.973\\
-21.973	-17.09\\
-17.09	-20.752\\
-20.752	-20.752\\
-20.752	-18.311\\
-18.311	-14.648\\
-14.648	-21.973\\
-21.973	-14.648\\
-14.648	-17.09\\
-17.09	-17.09\\
-17.09	-17.09\\
-17.09	-10.986\\
-10.986	-6.104\\
-6.104	-4.883\\
-4.883	-10.986\\
-10.986	-20.752\\
-20.752	-21.973\\
-21.973	-19.531\\
-19.531	-13.428\\
-13.428	-17.09\\
-17.09	-14.648\\
-14.648	-14.648\\
-14.648	-9.766\\
-9.766	-15.869\\
-15.869	-14.648\\
-14.648	-12.207\\
-12.207	-13.428\\
-13.428	-12.207\\
-12.207	-9.766\\
-9.766	-7.324\\
-7.324	-8.545\\
-8.545	-18.311\\
-18.311	-13.428\\
-13.428	-18.311\\
-18.311	-15.869\\
-15.869	-21.973\\
-21.973	-19.531\\
-19.531	-25.635\\
-25.635	-15.869\\
-15.869	-23.193\\
-23.193	-21.973\\
-21.973	-13.428\\
-13.428	-17.09\\
-17.09	-14.648\\
-14.648	-19.531\\
-19.531	-20.752\\
-20.752	-24.414\\
-24.414	-17.09\\
-17.09	-24.414\\
-24.414	-25.635\\
-25.635	-23.193\\
-23.193	-18.311\\
-18.311	-14.648\\
-14.648	-19.531\\
-19.531	-17.09\\
-17.09	-14.648\\
-14.648	-12.207\\
-12.207	-13.428\\
-13.428	-12.207\\
-12.207	-25.635\\
-25.635	-30.518\\
-30.518	-25.635\\
-25.635	-23.193\\
-23.193	-25.635\\
-25.635	-14.648\\
-14.648	-13.428\\
-13.428	-9.766\\
-9.766	-9.766\\
-9.766	-12.207\\
-12.207	-18.311\\
-18.311	-19.531\\
-19.531	-23.193\\
-23.193	-18.311\\
-18.311	-15.869\\
-15.869	-19.531\\
-19.531	-28.076\\
-28.076	-28.076\\
-28.076	-23.193\\
-23.193	-37.842\\
-37.842	-25.635\\
-25.635	-15.869\\
-15.869	-13.428\\
-13.428	-12.207\\
-12.207	-19.531\\
-19.531	-23.193\\
-23.193	-28.076\\
-28.076	-24.414\\
-24.414	-18.311\\
-18.311	-10.986\\
-10.986	-14.648\\
-14.648	-10.986\\
-10.986	-13.428\\
-13.428	-8.545\\
-8.545	-10.986\\
-10.986	-8.545\\
-8.545	-7.324\\
-7.324	-9.766\\
-9.766	-13.428\\
-13.428	-15.869\\
-15.869	-17.09\\
-17.09	-17.09\\
-17.09	-15.869\\
-15.869	-19.531\\
-19.531	-12.207\\
-12.207	-21.973\\
-21.973	-23.193\\
-23.193	-20.752\\
-20.752	-18.311\\
-18.311	-18.311\\
-18.311	-19.531\\
-19.531	-24.414\\
-24.414	-19.531\\
-19.531	-15.869\\
-15.869	-18.311\\
-18.311	-17.09\\
-17.09	-13.428\\
-13.428	-20.752\\
-20.752	-25.635\\
-25.635	-23.193\\
-23.193	-25.635\\
-25.635	-32.959\\
-32.959	-24.414\\
-24.414	-23.193\\
-23.193	-19.531\\
-19.531	-23.193\\
-23.193	-29.297\\
-29.297	-20.752\\
-20.752	-19.531\\
-19.531	-24.414\\
-24.414	-17.09\\
-17.09	-10.986\\
-10.986	-15.869\\
-15.869	-8.545\\
-8.545	-8.545\\
-8.545	-8.545\\
-8.545	-8.545\\
-8.545	-9.766\\
-9.766	-10.986\\
-10.986	-14.648\\
-14.648	-17.09\\
-17.09	-19.531\\
-19.531	-20.752\\
-20.752	-21.973\\
-21.973	-25.635\\
-25.635	-24.414\\
-24.414	-17.09\\
-17.09	-20.752\\
-20.752	-17.09\\
-17.09	-19.531\\
-19.531	-24.414\\
-24.414	-21.973\\
-21.973	-23.193\\
-23.193	-20.752\\
-20.752	-19.531\\
-19.531	-26.855\\
-26.855	-21.973\\
-21.973	-23.193\\
-23.193	-21.973\\
-21.973	-14.648\\
-14.648	-9.766\\
-9.766	-8.545\\
-8.545	-13.428\\
-13.428	-12.207\\
-12.207	-17.09\\
-17.09	-14.648\\
-14.648	-18.311\\
-18.311	-25.635\\
-25.635	-30.518\\
-30.518	-21.973\\
-21.973	-26.855\\
-26.855	-23.193\\
-23.193	-18.311\\
-18.311	-19.531\\
-19.531	-19.531\\
-19.531	-18.311\\
-18.311	-20.752\\
-20.752	-25.635\\
-25.635	-32.959\\
-32.959	-29.297\\
-29.297	-29.297\\
-29.297	-30.518\\
-30.518	-23.193\\
-23.193	-25.635\\
-25.635	-19.531\\
-19.531	-18.311\\
-18.311	-24.414\\
-24.414	-29.297\\
-29.297	-9.766\\
-9.766	-19.531\\
-19.531	-12.207\\
-12.207	-18.311\\
-18.311	-18.311\\
-18.311	-10.986\\
-10.986	-12.207\\
-12.207	-23.193\\
-23.193	-37.842\\
-37.842	-35.4\\
-35.4	-34.18\\
-34.18	-26.855\\
-26.855	-31.738\\
-31.738	-36.621\\
-36.621	-28.076\\
-28.076	-30.518\\
-30.518	-31.738\\
-31.738	-23.193\\
-23.193	-18.311\\
-18.311	-24.414\\
-24.414	-17.09\\
-17.09	-13.428\\
-13.428	-18.311\\
-18.311	-12.207\\
-12.207	-17.09\\
-17.09	-14.648\\
-14.648	-18.311\\
-18.311	-20.752\\
-20.752	-15.869\\
-15.869	-12.207\\
-12.207	-15.869\\
-15.869	-18.311\\
-18.311	-20.752\\
-20.752	-18.311\\
-18.311	-18.311\\
-18.311	-24.414\\
-24.414	-23.193\\
-23.193	-15.869\\
-15.869	-25.635\\
-25.635	-20.752\\
-20.752	-17.09\\
-17.09	-14.648\\
-14.648	-9.766\\
-9.766	-7.324\\
-7.324	-15.869\\
-15.869	-13.428\\
-13.428	-10.986\\
-10.986	-7.324\\
-7.324	-12.207\\
-12.207	-14.648\\
-14.648	-15.869\\
-15.869	-19.531\\
-19.531	-23.193\\
-23.193	-21.973\\
-21.973	-24.414\\
-24.414	-17.09\\
-17.09	-8.545\\
-8.545	-14.648\\
-14.648	-9.766\\
-9.766	-13.428\\
-13.428	-17.09\\
-17.09	-18.311\\
-18.311	-28.076\\
-28.076	-18.311\\
-18.311	-21.973\\
-21.973	-23.193\\
-23.193	-18.311\\
-18.311	-14.648\\
-14.648	-13.428\\
-13.428	-17.09\\
-17.09	-18.311\\
-18.311	-26.855\\
-26.855	-15.869\\
-15.869	-15.869\\
-15.869	-15.869\\
-15.869	-17.09\\
-17.09	-20.752\\
-20.752	-32.959\\
-32.959	-28.076\\
-28.076	-28.076\\
-28.076	-19.531\\
-19.531	-13.428\\
-13.428	-15.869\\
-15.869	-9.766\\
-9.766	-6.104\\
-6.104	-17.09\\
-17.09	-17.09\\
-17.09	-19.531\\
-19.531	-25.635\\
-25.635	-28.076\\
-28.076	-35.4\\
-35.4	-31.738\\
-31.738	-18.311\\
-18.311	-18.311\\
-18.311	-13.428\\
-13.428	-18.311\\
-18.311	-24.414\\
-24.414	-29.297\\
-29.297	-23.193\\
-23.193	-18.311\\
-18.311	-20.752\\
-20.752	-25.635\\
-25.635	-18.311\\
-18.311	-13.428\\
-13.428	-12.207\\
-12.207	-7.324\\
-7.324	-8.545\\
-8.545	-4.883\\
-4.883	-12.207\\
-12.207	-15.869\\
-15.869	-15.869\\
-15.869	-12.207\\
-12.207	-13.428\\
-13.428	-7.324\\
-7.324	-3.662\\
-3.662	-7.324\\
-7.324	-10.986\\
-10.986	-12.207\\
-12.207	-14.648\\
-14.648	-18.311\\
-18.311	-20.752\\
-20.752	-24.414\\
-24.414	-34.18\\
-34.18	-32.959\\
-32.959	-36.621\\
-36.621	-31.738\\
-31.738	-20.752\\
-20.752	-18.311\\
-18.311	-18.311\\
-18.311	-21.973\\
-21.973	-17.09\\
-17.09	-15.869\\
-15.869	-12.207\\
-12.207	-12.207\\
-12.207	-17.09\\
-17.09	-14.648\\
-14.648	-8.545\\
-8.545	-7.324\\
-7.324	-10.986\\
-10.986	-14.648\\
-14.648	-13.428\\
-13.428	-14.648\\
-14.648	-12.207\\
-12.207	-15.869\\
-15.869	-23.193\\
-23.193	-17.09\\
-17.09	-24.414\\
-24.414	-32.959\\
-32.959	-35.4\\
-35.4	-26.855\\
-26.855	-19.531\\
-19.531	-14.648\\
-14.648	-9.766\\
-9.766	-15.869\\
-15.869	-10.986\\
-10.986	-17.09\\
-17.09	-14.648\\
-14.648	-12.207\\
-12.207	-17.09\\
-17.09	-12.207\\
-12.207	-12.207\\
-12.207	-13.428\\
-13.428	-9.766\\
-9.766	-9.766\\
-9.766	-8.545\\
-8.545	-8.545\\
-8.545	-7.324\\
-7.324	-6.104\\
-6.104	-6.104\\
-6.104	-9.766\\
-9.766	-10.986\\
-10.986	-8.545\\
-8.545	-15.869\\
-15.869	-17.09\\
-17.09	-14.648\\
-14.648	-14.648\\
-14.648	-20.752\\
-20.752	-25.635\\
-25.635	-18.311\\
-18.311	-21.973\\
-21.973	-9.766\\
-9.766	-6.104\\
-6.104	-14.648\\
-14.648	-18.311\\
-18.311	-19.531\\
-19.531	-26.855\\
-26.855	-25.635\\
-25.635	-17.09\\
-17.09	-13.428\\
-13.428	-15.869\\
-15.869	-14.648\\
-14.648	-10.986\\
-10.986	-7.324\\
-7.324	-12.207\\
-12.207	-7.324\\
-7.324	-6.104\\
-6.104	-7.324\\
-7.324	-9.766\\
-9.766	-13.428\\
-13.428	-13.428\\
-13.428	-9.766\\
-9.766	-17.09\\
-17.09	-23.193\\
-23.193	-14.648\\
-14.648	-20.752\\
-20.752	-23.193\\
-23.193	-15.869\\
-15.869	-10.986\\
-10.986	-14.648\\
-14.648	-14.648\\
-14.648	-9.766\\
-9.766	-12.207\\
-12.207	-7.324\\
-7.324	-6.104\\
-6.104	-6.104\\
-6.104	-8.545\\
-8.545	-12.207\\
-12.207	-10.986\\
-10.986	-12.207\\
-12.207	-14.648\\
-14.648	-9.766\\
-9.766	-7.324\\
-7.324	-6.104\\
-6.104	-8.545\\
-8.545	-8.545\\
-8.545	-9.766\\
-9.766	-18.311\\
-18.311	-18.311\\
-18.311	-14.648\\
-14.648	-14.648\\
-14.648	-19.531\\
-19.531	-24.414\\
-24.414	-25.635\\
-25.635	-26.855\\
-26.855	-21.973\\
-21.973	-21.973\\
-21.973	-21.973\\
-21.973	-23.193\\
-23.193	-25.635\\
-25.635	-30.518\\
-30.518	-26.855\\
-26.855	-24.414\\
-24.414	-20.752\\
-20.752	-19.531\\
-19.531	-19.531\\
-19.531	-24.414\\
-24.414	-15.869\\
-15.869	-19.531\\
-19.531	-21.973\\
-21.973	-24.414\\
-24.414	-17.09\\
-17.09	-20.752\\
-20.752	-18.311\\
-18.311	-17.09\\
-17.09	-12.207\\
-12.207	-17.09\\
-17.09	-24.414\\
-24.414	-19.531\\
-19.531	-10.986\\
-10.986	-14.648\\
-14.648	-12.207\\
-12.207	-9.766\\
-9.766	-10.986\\
-10.986	-7.324\\
-7.324	-9.766\\
-9.766	-8.545\\
-8.545	-6.104\\
-6.104	-7.324\\
-7.324	-9.766\\
-9.766	-7.324\\
-7.324	-8.545\\
-8.545	-9.766\\
-9.766	-10.986\\
-10.986	-13.428\\
-13.428	-12.207\\
-12.207	-15.869\\
-15.869	-13.428\\
-13.428	-14.648\\
-14.648	-24.414\\
-24.414	-17.09\\
-17.09	-15.869\\
-15.869	-18.311\\
-18.311	-13.428\\
-13.428	-8.545\\
-8.545	-19.531\\
-19.531	-10.986\\
-10.986	-8.545\\
-8.545	-10.986\\
-10.986	-13.428\\
-13.428	-12.207\\
-12.207	-10.986\\
-10.986	-6.104\\
-6.104	-6.104\\
-6.104	-6.104\\
-6.104	-12.207\\
-12.207	-10.986\\
-10.986	-12.207\\
-12.207	-15.869\\
-15.869	-19.531\\
-19.531	-15.869\\
-15.869	-20.752\\
-20.752	-24.414\\
-24.414	-20.752\\
-20.752	-20.752\\
-20.752	-31.738\\
-31.738	-23.193\\
-23.193	-30.518\\
-30.518	-24.414\\
-24.414	-26.855\\
-26.855	-30.518\\
-30.518	-20.752\\
-20.752	-10.986\\
-10.986	-12.207\\
-12.207	-15.869\\
-15.869	-18.311\\
-18.311	-18.311\\
-18.311	-15.869\\
-15.869	-9.766\\
-9.766	-10.986\\
-10.986	-18.311\\
-18.311	-24.414\\
-24.414	-23.193\\
-23.193	-17.09\\
-17.09	-12.207\\
-12.207	-14.648\\
-14.648	-24.414\\
-24.414	-25.635\\
-25.635	-26.855\\
-26.855	-20.752\\
-20.752	-26.855\\
-26.855	-29.297\\
-29.297	-26.855\\
-26.855	-35.4\\
-35.4	-31.738\\
-31.738	-26.855\\
-26.855	-31.738\\
-31.738	-26.855\\
-26.855	-17.09\\
-17.09	-14.648\\
-14.648	-18.311\\
-18.311	-20.752\\
-20.752	-21.973\\
-21.973	-18.311\\
-18.311	-20.752\\
-20.752	-18.311\\
-18.311	-14.648\\
-14.648	-17.09\\
-17.09	-17.09\\
-17.09	-25.635\\
-25.635	-24.414\\
-24.414	-20.752\\
-20.752	-14.648\\
-14.648	-12.207\\
-12.207	-13.428\\
-13.428	-14.648\\
-14.648	-9.766\\
-9.766	-8.545\\
-8.545	-14.648\\
-14.648	-17.09\\
-17.09	-20.752\\
-20.752	-19.531\\
-19.531	-12.207\\
-12.207	-8.545\\
-8.545	-9.766\\
-9.766	-8.545\\
-8.545	-13.428\\
-13.428	-14.648\\
-14.648	-13.428\\
-13.428	-19.531\\
-19.531	-25.635\\
-25.635	-24.414\\
-24.414	-26.855\\
-26.855	-30.518\\
-30.518	-17.09\\
-17.09	-17.09\\
-17.09	-14.648\\
-14.648	-15.869\\
-15.869	-18.311\\
-18.311	-15.869\\
-15.869	-10.986\\
-10.986	-9.766\\
-9.766	-17.09\\
-17.09	-20.752\\
-20.752	-18.311\\
-18.311	-26.855\\
-26.855	-31.738\\
-31.738	-21.973\\
-21.973	-14.648\\
-14.648	-13.428\\
-13.428	-10.986\\
-10.986	-18.311\\
-18.311	-17.09\\
-17.09	-15.869\\
-15.869	-18.311\\
-18.311	-23.193\\
-23.193	-20.752\\
-20.752	-24.414\\
-24.414	-17.09\\
-17.09	-18.311\\
-18.311	-14.648\\
-14.648	-12.207\\
-12.207	-17.09\\
-17.09	-24.414\\
-24.414	-25.635\\
-25.635	-19.531\\
-19.531	-35.4\\
-35.4	-26.855\\
-26.855	-20.752\\
-20.752	-13.428\\
-13.428	-17.09\\
-17.09	-17.09\\
-17.09	-15.869\\
-15.869	-17.09\\
-17.09	-14.648\\
-14.648	-13.428\\
-13.428	-28.076\\
-28.076	-30.518\\
-30.518	-18.311\\
-18.311	-23.193\\
-23.193	-24.414\\
-24.414	-24.414\\
-24.414	-18.311\\
-18.311	-18.311\\
-18.311	-15.869\\
-15.869	-18.311\\
-18.311	-21.973\\
-21.973	-28.076\\
-28.076	-32.959\\
-32.959	-42.725\\
-42.725	-35.4\\
-35.4	-24.414\\
-24.414	-21.973\\
-21.973	-21.973\\
-21.973	-18.311\\
-18.311	-13.428\\
-13.428	-12.207\\
-12.207	-13.428\\
-13.428	-8.545\\
-8.545	-8.545\\
-8.545	-9.766\\
-9.766	-14.648\\
-14.648	-12.207\\
-12.207	-8.545\\
-8.545	-7.324\\
-7.324	-14.648\\
-14.648	-23.193\\
-23.193	-12.207\\
-12.207	-12.207\\
-12.207	-17.09\\
-17.09	-14.648\\
-14.648	-17.09\\
-17.09	-15.869\\
-15.869	-10.986\\
-10.986	-9.766\\
-9.766	-6.104\\
-6.104	-8.545\\
-8.545	-15.869\\
-15.869	-18.311\\
-18.311	-13.428\\
-13.428	-9.766\\
-9.766	-7.324\\
-7.324	-12.207\\
-12.207	-13.428\\
-13.428	-18.311\\
-18.311	-18.311\\
-18.311	-14.648\\
-14.648	-12.207\\
-12.207	-13.428\\
-13.428	-18.311\\
-18.311	-25.635\\
-25.635	-28.076\\
-28.076	-23.193\\
-23.193	-14.648\\
-14.648	-14.648\\
-14.648	-15.869\\
-15.869	-14.648\\
-14.648	-12.207\\
-12.207	-12.207\\
-12.207	-14.648\\
-14.648	-14.648\\
-14.648	-10.986\\
-10.986	-8.545\\
-8.545	-8.545\\
-8.545	-13.428\\
-13.428	-17.09\\
-17.09	-21.973\\
-21.973	-23.193\\
-23.193	-23.193\\
-23.193	-23.193\\
-23.193	-30.518\\
-30.518	-40.283\\
-40.283	-34.18\\
-34.18	-23.193\\
-23.193	-26.855\\
-26.855	-29.297\\
-29.297	-36.621\\
-36.621	-28.076\\
-28.076	-14.648\\
-14.648	-9.766\\
-9.766	-12.207\\
-12.207	-9.766\\
-9.766	-6.104\\
-6.104	-7.324\\
-7.324	-9.766\\
-9.766	-12.207\\
-12.207	-12.207\\
-12.207	-10.986\\
-10.986	-9.766\\
-9.766	-10.986\\
-10.986	-14.648\\
-14.648	-14.648\\
-14.648	-13.428\\
-13.428	-10.986\\
-10.986	-7.324\\
-7.324	-7.324\\
-7.324	-10.986\\
-10.986	-12.207\\
-12.207	-9.766\\
-9.766	-7.324\\
-7.324	-10.986\\
-10.986	-13.428\\
-13.428	-9.766\\
-9.766	-13.428\\
-13.428	-14.648\\
-14.648	-19.531\\
-19.531	-13.428\\
-13.428	-15.869\\
-15.869	-17.09\\
-17.09	-15.869\\
-15.869	-13.428\\
-13.428	-18.311\\
-18.311	-25.635\\
-25.635	-23.193\\
-23.193	-25.635\\
-25.635	-21.973\\
-21.973	-19.531\\
-19.531	-18.311\\
-18.311	-21.973\\
-21.973	-19.531\\
-19.531	-20.752\\
-20.752	-23.193\\
-23.193	-23.193\\
-23.193	-40.283\\
-40.283	-34.18\\
-34.18	-17.09\\
-17.09	-12.207\\
-12.207	-14.648\\
-14.648	-17.09\\
-17.09	-20.752\\
-20.752	-23.193\\
-23.193	-24.414\\
-24.414	-26.855\\
-26.855	-19.531\\
-19.531	-15.869\\
-15.869	-19.531\\
-19.531	-20.752\\
-20.752	-12.207\\
-12.207	-10.986\\
-10.986	-13.428\\
-13.428	-13.428\\
-13.428	-17.09\\
-17.09	-23.193\\
-23.193	-21.973\\
-21.973	-14.648\\
-14.648	-12.207\\
-12.207	-18.311\\
-18.311	-23.193\\
-23.193	-26.855\\
-26.855	-31.738\\
-31.738	-36.621\\
-36.621	-30.518\\
-30.518	-29.297\\
-29.297	-19.531\\
-19.531	-13.428\\
-13.428	-21.973\\
-21.973	-28.076\\
-28.076	-18.311\\
-18.311	-20.752\\
-20.752	-35.4\\
-35.4	-37.842\\
-37.842	-25.635\\
-25.635	-17.09\\
-17.09	-9.766\\
-9.766	-12.207\\
-12.207	-14.648\\
-14.648	-12.207\\
-12.207	-10.986\\
-10.986	-10.986\\
-10.986	-12.207\\
-12.207	-8.545\\
-8.545	-12.207\\
-12.207	-15.869\\
-15.869	-18.311\\
-18.311	-17.09\\
-17.09	-19.531\\
-19.531	-28.076\\
-28.076	-24.414\\
-24.414	-17.09\\
-17.09	-15.869\\
-15.869	-19.531\\
-19.531	-14.648\\
-14.648	-10.986\\
-10.986	-10.986\\
-10.986	-13.428\\
-13.428	-12.207\\
-12.207	-9.766\\
-9.766	-6.104\\
-6.104	-6.104\\
-6.104	-8.545\\
-8.545	-14.648\\
-14.648	-13.428\\
-13.428	-10.986\\
-10.986	-8.545\\
-8.545	-15.869\\
-15.869	-18.311\\
-18.311	-10.986\\
-10.986	-13.428\\
-13.428	-15.869\\
-15.869	-17.09\\
-17.09	-17.09\\
-17.09	-19.531\\
-19.531	-19.531\\
-19.531	-13.428\\
-13.428	-9.766\\
-9.766	-14.648\\
-14.648	-10.986\\
-10.986	-13.428\\
-13.428	-19.531\\
-19.531	-25.635\\
-25.635	-21.973\\
-21.973	-18.311\\
-18.311	-26.855\\
-26.855	-24.414\\
-24.414	-14.648\\
-14.648	-10.986\\
-10.986	-10.986\\
-10.986	-9.766\\
-9.766	-8.545\\
-8.545	-7.324\\
-7.324	-14.648\\
-14.648	-13.428\\
-13.428	-8.545\\
-8.545	-12.207\\
-12.207	-13.428\\
-13.428	-9.766\\
-9.766	-8.545\\
-8.545	-4.883\\
-4.883	-7.324\\
-7.324	-14.648\\
-14.648	-19.531\\
-19.531	-17.09\\
-17.09	-12.207\\
-12.207	-14.648\\
-14.648	-14.648\\
-14.648	-15.869\\
-15.869	-20.752\\
-20.752	-23.193\\
-23.193	-23.193\\
-23.193	-19.531\\
-19.531	-10.986\\
-10.986	-10.986\\
-10.986	-12.207\\
-12.207	-15.869\\
-15.869	-14.648\\
-14.648	-8.545\\
-8.545	-17.09\\
-17.09	-20.752\\
-20.752	-18.311\\
-18.311	-20.752\\
-20.752	-25.635\\
-25.635	-19.531\\
-19.531	-18.311\\
-18.311	-26.855\\
-26.855	-21.973\\
-21.973	-24.414\\
-24.414	-25.635\\
-25.635	-37.842\\
-37.842	-39.063\\
-39.063	-25.635\\
-25.635	-32.959\\
-32.959	-36.621\\
-36.621	-35.4\\
-35.4	-37.842\\
-37.842	-36.621\\
-36.621	-43.945\\
-43.945	-41.504\\
-41.504	-24.414\\
-24.414	-19.531\\
-19.531	-18.311\\
-18.311	-15.869\\
-15.869	-12.207\\
-12.207	-17.09\\
-17.09	-23.193\\
-23.193	-18.311\\
-18.311	-13.428\\
-13.428	-15.869\\
-15.869	-12.207\\
-12.207	-10.986\\
-10.986	-10.986\\
-10.986	-7.324\\
-7.324	-8.545\\
-8.545	-15.869\\
-15.869	-8.545\\
-8.545	-8.545\\
-8.545	-6.104\\
-6.104	-3.662\\
-3.662	-7.324\\
-7.324	-8.545\\
-8.545	-6.104\\
-6.104	-12.207\\
-12.207	-14.648\\
-14.648	-14.648\\
-14.648	-12.207\\
-12.207	-15.869\\
-15.869	-21.973\\
-21.973	-17.09\\
-17.09	-6.104\\
-6.104	-10.986\\
-10.986	-6.104\\
-6.104	-7.324\\
-7.324	-13.428\\
-13.428	-15.869\\
-15.869	-15.869\\
-15.869	-12.207\\
-12.207	-28.076\\
-28.076	-23.193\\
-23.193	-17.09\\
-17.09	-13.428\\
-13.428	-14.648\\
-14.648	-20.752\\
-20.752	-23.193\\
-23.193	-18.311\\
-18.311	-6.104\\
-6.104	-12.207\\
-12.207	-9.766\\
-9.766	-6.104\\
-6.104	-3.662\\
-3.662	-7.324\\
-7.324	-19.531\\
-19.531	-19.531\\
-19.531	-10.986\\
-10.986	-8.545\\
-8.545	-8.545\\
-8.545	-9.766\\
-9.766	-6.104\\
-6.104	-3.662\\
-3.662	-9.766\\
-9.766	-14.648\\
-14.648	-18.311\\
-18.311	-20.752\\
-20.752	-19.531\\
-19.531	-20.752\\
-20.752	-20.752\\
-20.752	-19.531\\
-19.531	-18.311\\
-18.311	-21.973\\
-21.973	-30.518\\
-30.518	-26.855\\
-26.855	-17.09\\
-17.09	-8.545\\
-8.545	-12.207\\
-12.207	-26.855\\
-26.855	-21.973\\
-21.973	-13.428\\
-13.428	-15.869\\
-15.869	-20.752\\
-20.752	-29.297\\
-29.297	-29.297\\
-29.297	-31.738\\
-31.738	-29.297\\
-29.297	-26.855\\
-26.855	-24.414\\
-24.414	-17.09\\
-17.09	-13.428\\
-13.428	-15.869\\
-15.869	-19.531\\
-19.531	-17.09\\
-17.09	-19.531\\
-19.531	-13.428\\
-13.428	-18.311\\
-18.311	-17.09\\
-17.09	-7.324\\
-7.324	-4.883\\
-4.883	-4.883\\
-4.883	-7.324\\
-7.324	-8.545\\
-8.545	-3.662\\
-3.662	-7.324\\
-7.324	-17.09\\
-17.09	-3.662\\
-3.662	-10.986\\
-10.986	-14.648\\
-14.648	-20.752\\
-20.752	-18.311\\
-18.311	-7.324\\
-7.324	-7.324\\
-7.324	-7.324\\
-7.324	-24.414\\
-24.414	-26.855\\
-26.855	-34.18\\
-34.18	-42.725\\
-42.725	-40.283\\
-40.283	-29.297\\
-29.297	-29.297\\
-29.297	-36.621\\
-36.621	-32.959\\
-32.959	-29.297\\
-29.297	-30.518\\
-30.518	-35.4\\
-35.4	-35.4\\
-35.4	-46.387\\
-46.387	-36.621\\
-36.621	-18.311\\
-18.311	-10.986\\
-10.986	-9.766\\
-9.766	-10.986\\
-10.986	-12.207\\
-12.207	-10.986\\
-10.986	-17.09\\
-17.09	-18.311\\
-18.311	-14.648\\
-14.648	-19.531\\
-19.531	-12.207\\
-12.207	-17.09\\
-17.09	-18.311\\
-18.311	-19.531\\
-19.531	-10.986\\
-10.986	-6.104\\
-6.104	-10.986\\
-10.986	-13.428\\
-13.428	-17.09\\
-17.09	-30.518\\
-30.518	-30.518\\
-30.518	-32.959\\
-32.959	-35.4\\
-35.4	-45.166\\
-45.166	-40.283\\
-40.283	-26.855\\
-26.855	-17.09\\
-17.09	-13.428\\
-13.428	-13.428\\
-13.428	-14.648\\
-14.648	-15.869\\
-15.869	-8.545\\
-8.545	-12.207\\
-12.207	-12.207\\
-12.207	-14.648\\
-14.648	-18.311\\
-18.311	-18.311\\
-18.311	-15.869\\
-15.869	-7.324\\
-7.324	-4.883\\
-4.883	-7.324\\
-7.324	-4.883\\
-4.883	-6.104\\
-6.104	-10.986\\
-10.986	-14.648\\
-14.648	-15.869\\
-15.869	-19.531\\
-19.531	-17.09\\
-17.09	-19.531\\
-19.531	-34.18\\
-34.18	-26.855\\
-26.855	-21.973\\
-21.973	-28.076\\
-28.076	-31.738\\
-31.738	-20.752\\
-20.752	-18.311\\
-18.311	-13.428\\
-13.428	-15.869\\
-15.869	-28.076\\
-28.076	-41.504\\
-41.504	-46.387\\
-46.387	-36.621\\
-36.621	-31.738\\
-31.738	-26.855\\
-26.855	-28.076\\
-28.076	-34.18\\
-34.18	-23.193\\
-23.193	-18.311\\
-18.311	-18.311\\
-18.311	-24.414\\
-24.414	-25.635\\
-25.635	-14.648\\
-14.648	-14.648\\
-14.648	-13.428\\
-13.428	-17.09\\
-17.09	-26.855\\
-26.855	-25.635\\
-25.635	-20.752\\
-20.752	-18.311\\
-18.311	-17.09\\
-17.09	-14.648\\
-14.648	-8.545\\
-8.545	-8.545\\
-8.545	-7.324\\
-7.324	-7.324\\
-7.324	-6.104\\
-6.104	-10.986\\
-10.986	-17.09\\
-17.09	-18.311\\
-18.311	-13.428\\
-13.428	-9.766\\
-9.766	-8.545\\
-8.545	-10.986\\
-10.986	-14.648\\
-14.648	-17.09\\
-17.09	-14.648\\
-14.648	-9.766\\
-9.766	-20.752\\
-20.752	-21.973\\
-21.973	-15.869\\
-15.869	-4.883\\
-4.883	-10.986\\
-10.986	-13.428\\
-13.428	-14.648\\
-14.648	-9.766\\
-9.766	-7.324\\
-7.324	-13.428\\
-13.428	-21.973\\
-21.973	-13.428\\
-13.428	-3.662\\
-3.662	-9.766\\
-9.766	-14.648\\
-14.648	-20.752\\
-20.752	-14.648\\
-14.648	-12.207\\
-12.207	-20.752\\
-20.752	-26.855\\
-26.855	-23.193\\
-23.193	-29.297\\
-29.297	-35.4\\
-35.4	-24.414\\
-24.414	-19.531\\
-19.531	-26.855\\
-26.855	-19.531\\
-19.531	-23.193\\
-23.193	-29.297\\
-29.297	-21.973\\
-21.973	-10.986\\
-10.986	-20.752\\
-20.752	-34.18\\
-34.18	-37.842\\
-37.842	-29.297\\
-29.297	-24.414\\
-24.414	-31.738\\
-31.738	-26.855\\
-26.855	-17.09\\
-17.09	-17.09\\
-17.09	-19.531\\
-19.531	-21.973\\
-21.973	-19.531\\
-19.531	-21.973\\
-21.973	-14.648\\
-14.648	-19.531\\
-19.531	-24.414\\
-24.414	-18.311\\
-18.311	-14.648\\
-14.648	-12.207\\
-12.207	-8.545\\
-8.545	-9.766\\
-9.766	-17.09\\
-17.09	-15.869\\
-15.869	-19.531\\
-19.531	-25.635\\
-25.635	-26.855\\
-26.855	-20.752\\
-20.752	-18.311\\
-18.311	-10.986\\
-10.986	-14.648\\
-14.648	-24.414\\
-24.414	-18.311\\
-18.311	-8.545\\
-8.545	-15.869\\
-15.869	-25.635\\
-25.635	-23.193\\
-23.193	-30.518\\
-30.518	-39.063\\
-39.063	-30.518\\
-30.518	-26.855\\
-26.855	-29.297\\
-29.297	-20.752\\
-20.752	-14.648\\
-14.648	-17.09\\
-17.09	-24.414\\
-24.414	-24.414\\
-24.414	-25.635\\
-25.635	-13.428\\
-13.428	-14.648\\
-14.648	-12.207\\
-12.207	-17.09\\
-17.09	-14.648\\
-14.648	-12.207\\
-12.207	-20.752\\
-20.752	-24.414\\
-24.414	-15.869\\
-15.869	-12.207\\
-12.207	-20.752\\
-20.752	-13.428\\
-13.428	-9.766\\
-9.766	-10.986\\
-10.986	-12.207\\
-12.207	-10.986\\
-10.986	-6.104\\
-6.104	-6.104\\
-6.104	-10.986\\
-10.986	-12.207\\
-12.207	-20.752\\
-20.752	-19.531\\
-19.531	-21.973\\
-21.973	-28.076\\
-28.076	-14.648\\
-14.648	-14.648\\
-14.648	-25.635\\
-25.635	-36.621\\
-36.621	-26.855\\
-26.855	-25.635\\
-25.635	-39.063\\
-39.063	-31.738\\
-31.738	-25.635\\
-25.635	-20.752\\
-20.752	-20.752\\
-20.752	-25.635\\
-25.635	-18.311\\
-18.311	-17.09\\
-17.09	-28.076\\
-28.076	-31.738\\
-31.738	-31.738\\
-31.738	-18.311\\
-18.311	-8.545\\
-8.545	-15.869\\
-15.869	-15.869\\
-15.869	-8.545\\
-8.545	-7.324\\
-7.324	-10.986\\
-10.986	-13.428\\
-13.428	-23.193\\
-23.193	-17.09\\
-17.09	-10.986\\
-10.986	-9.766\\
-9.766	-7.324\\
-7.324	-8.545\\
-8.545	-7.324\\
-7.324	-7.324\\
-7.324	-14.648\\
-14.648	-12.207\\
-12.207	-7.324\\
-7.324	-8.545\\
-8.545	-9.766\\
-9.766	-10.986\\
-10.986	-8.545\\
-8.545	-9.766\\
-9.766	-8.545\\
-8.545	-4.883\\
-4.883	-13.428\\
-13.428	-17.09\\
-17.09	-23.193\\
-23.193	-30.518\\
-30.518	-26.855\\
-26.855	-13.428\\
-13.428	-10.986\\
-10.986	-12.207\\
-12.207	-7.324\\
-7.324	-9.766\\
-9.766	-10.986\\
-10.986	-14.648\\
-14.648	-13.428\\
-13.428	-13.428\\
-13.428	-15.869\\
-15.869	-18.311\\
-18.311	-17.09\\
-17.09	-21.973\\
-21.973	-21.973\\
-21.973	-30.518\\
-30.518	-19.531\\
-19.531	-9.766\\
-9.766	-8.545\\
-8.545	-15.869\\
-15.869	-10.986\\
-10.986	-9.766\\
-9.766	-4.883\\
-4.883	-10.986\\
-10.986	-7.324\\
-7.324	-3.662\\
-3.662	-6.104\\
-6.104	-10.986\\
-10.986	-13.428\\
-13.428	-15.869\\
-15.869	-24.414\\
-24.414	-17.09\\
-17.09	-7.324\\
-7.324	-13.428\\
-13.428	-18.311\\
-18.311	-10.986\\
-10.986	-13.428\\
-13.428	-20.752\\
-20.752	-17.09\\
-17.09	-25.635\\
-25.635	-30.518\\
-30.518	-20.752\\
-20.752	-17.09\\
-17.09	-15.869\\
};
\addplot [color=mycolor2, line width=2.0pt, forget plot]
  table[row sep=crcr]{%
-18.311	-17.6579837260103\\
-19.531	-18.8344754602538\\
-24.414	-23.5433354096891\\
-19.531	-18.8344754602538\\
-20.752	-20.0119315319844\\
-25.635	-24.7207914814196\\
-24.414	-23.5433354096891\\
-28.076	-27.0747392873937\\
-23.193	-22.3658793379585\\
-13.428	-12.9491237765751\\
-17.09	-16.4805276542798\\
-18.311	-17.6579837260103\\
-15.869	-15.3030715825492\\
-8.545	-8.24026382713988\\
-6.104	-5.88631602116581\\
-7.324	-7.0628077554093\\
-9.766	-9.41771989887046\\
-21.973	-21.189387603715\\
-23.193	-22.3658793379585\\
-19.531	-18.8344754602538\\
-10.986	-10.594211633114\\
-17.09	-16.4805276542798\\
-19.531	-18.8344754602538\\
-13.428	-12.9491237765751\\
-18.311	-17.6579837260103\\
-19.531	-18.8344754602538\\
-14.648	-14.1256155108186\\
-24.414	-23.5433354096891\\
-21.973	-21.189387603715\\
-15.869	-15.3030715825492\\
-23.193	-22.3658793379585\\
-25.635	-24.7207914814196\\
-19.531	-18.8344754602538\\
-17.09	-16.4805276542798\\
-14.648	-14.1256155108186\\
-17.09	-16.4805276542798\\
-20.752	-20.0119315319844\\
-18.311	-17.6579837260103\\
-19.531	-18.8344754602538\\
-20.752	-20.0119315319844\\
-28.076	-27.0747392873937\\
-25.635	-24.7207914814196\\
-17.09	-16.4805276542798\\
-15.869	-15.3030715825492\\
-12.207	-11.7716677048445\\
-10.986	-10.594211633114\\
-15.869	-15.3030715825492\\
-12.207	-11.7716677048445\\
-15.869	-15.3030715825492\\
-18.311	-17.6579837260103\\
-15.869	-15.3030715825492\\
-25.635	-24.7207914814196\\
-21.973	-21.189387603715\\
-12.207	-11.7716677048445\\
-9.766	-9.41771989887046\\
-13.428	-12.9491237765751\\
-15.869	-15.3030715825492\\
-10.986	-10.594211633114\\
-12.207	-11.7716677048445\\
-9.766	-9.41771989887046\\
-8.545	-8.24026382713988\\
-12.207	-11.7716677048445\\
-14.648	-14.1256155108186\\
-17.09	-16.4805276542798\\
-18.311	-17.6579837260103\\
-23.193	-22.3658793379585\\
-21.973	-21.189387603715\\
-17.09	-16.4805276542798\\
-15.869	-15.3030715825492\\
-17.09	-16.4805276542798\\
-24.414	-23.5433354096891\\
-35.4	-34.137547042803\\
-36.621	-35.3150031145336\\
-34.18	-32.9610553085595\\
-24.414	-23.5433354096891\\
-29.297	-28.2521953591243\\
-36.621	-35.3150031145336\\
-40.283	-38.8464069922382\\
-31.738	-30.6061431650984\\
-37.842	-36.4924591862642\\
-48.828	-47.0866708193781\\
-39.063	-37.6699152579947\\
-30.518	-29.4296514308549\\
-25.635	-24.7207914814196\\
-19.531	-18.8344754602538\\
-14.648	-14.1256155108186\\
-19.531	-18.8344754602538\\
-15.869	-15.3030715825492\\
-9.766	-9.41771989887046\\
-12.207	-11.7716677048445\\
-17.09	-16.4805276542798\\
-15.869	-15.3030715825492\\
-19.531	-18.8344754602538\\
-28.076	-27.0747392873937\\
-23.193	-22.3658793379585\\
-25.635	-24.7207914814196\\
-23.193	-22.3658793379585\\
-18.311	-17.6579837260103\\
-20.752	-20.0119315319844\\
-17.09	-16.4805276542798\\
-14.648	-14.1256155108186\\
-18.311	-17.6579837260103\\
-17.09	-16.4805276542798\\
-15.869	-15.3030715825492\\
-13.428	-12.9491237765751\\
-10.986	-10.594211633114\\
-12.207	-11.7716677048445\\
-13.428	-12.9491237765751\\
-10.986	-10.594211633114\\
-9.766	-9.41771989887046\\
-14.648	-14.1256155108186\\
-19.531	-18.8344754602538\\
-23.193	-22.3658793379585\\
-17.09	-16.4805276542798\\
-18.311	-17.6579837260103\\
-30.518	-29.4296514308549\\
-23.193	-22.3658793379585\\
-24.414	-23.5433354096891\\
-26.855	-25.8972832156631\\
-17.09	-16.4805276542798\\
-10.986	-10.594211633114\\
-17.09	-16.4805276542798\\
-18.311	-17.6579837260103\\
-26.855	-25.8972832156631\\
-34.18	-32.9610553085595\\
-31.738	-30.6061431650984\\
-25.635	-24.7207914814196\\
-24.414	-23.5433354096891\\
-23.193	-22.3658793379585\\
-21.973	-21.189387603715\\
-18.311	-17.6579837260103\\
-14.648	-14.1256155108186\\
-13.428	-12.9491237765751\\
-12.207	-11.7716677048445\\
-13.428	-12.9491237765751\\
-19.531	-18.8344754602538\\
-25.635	-24.7207914814196\\
-21.973	-21.189387603715\\
-14.648	-14.1256155108186\\
-13.428	-12.9491237765751\\
-14.648	-14.1256155108186\\
-12.207	-11.7716677048445\\
-8.545	-8.24026382713988\\
-13.428	-12.9491237765751\\
-12.207	-11.7716677048445\\
-8.545	-8.24026382713988\\
-7.324	-7.0628077554093\\
-4.883	-4.70885994943523\\
-7.324	-7.0628077554093\\
-17.09	-16.4805276542798\\
-24.414	-23.5433354096891\\
-19.531	-18.8344754602538\\
-12.207	-11.7716677048445\\
-8.545	-8.24026382713988\\
-10.986	-10.594211633114\\
-14.648	-14.1256155108186\\
-24.414	-23.5433354096891\\
-34.18	-32.9610553085595\\
-40.283	-38.8464069922382\\
-32.959	-31.7835992368289\\
-31.738	-30.6061431650984\\
-23.193	-22.3658793379585\\
-25.635	-24.7207914814196\\
-26.855	-25.8972832156631\\
-28.076	-27.0747392873937\\
-21.973	-21.189387603715\\
-20.752	-20.0119315319844\\
-19.531	-18.8344754602538\\
-21.973	-21.189387603715\\
-19.531	-18.8344754602538\\
-23.193	-22.3658793379585\\
-29.297	-28.2521953591243\\
-24.414	-23.5433354096891\\
-14.648	-14.1256155108186\\
-15.869	-15.3030715825492\\
-25.635	-24.7207914814196\\
-32.959	-31.7835992368289\\
-35.4	-34.137547042803\\
-39.063	-37.6699152579947\\
-37.842	-36.4924591862642\\
-30.518	-29.4296514308549\\
-26.855	-25.8972832156631\\
-30.518	-29.4296514308549\\
-35.4	-34.137547042803\\
-25.635	-24.7207914814196\\
-14.648	-14.1256155108186\\
-19.531	-18.8344754602538\\
-20.752	-20.0119315319844\\
-12.207	-11.7716677048445\\
-13.428	-12.9491237765751\\
-18.311	-17.6579837260103\\
-14.648	-14.1256155108186\\
-12.207	-11.7716677048445\\
-18.311	-17.6579837260103\\
-10.986	-10.594211633114\\
-14.648	-14.1256155108186\\
-24.414	-23.5433354096891\\
-21.973	-21.189387603715\\
-14.648	-14.1256155108186\\
-20.752	-20.0119315319844\\
-34.18	-32.9610553085595\\
-28.076	-27.0747392873937\\
-15.869	-15.3030715825492\\
-14.648	-14.1256155108186\\
-12.207	-11.7716677048445\\
-10.986	-10.594211633114\\
-13.428	-12.9491237765751\\
-17.09	-16.4805276542798\\
-19.531	-18.8344754602538\\
-14.648	-14.1256155108186\\
-19.531	-18.8344754602538\\
-26.855	-25.8972832156631\\
-23.193	-22.3658793379585\\
-17.09	-16.4805276542798\\
-21.973	-21.189387603715\\
-25.635	-24.7207914814196\\
-21.973	-21.189387603715\\
-8.545	-8.24026382713988\\
-12.207	-11.7716677048445\\
-13.428	-12.9491237765751\\
-15.869	-15.3030715825492\\
-10.986	-10.594211633114\\
-17.09	-16.4805276542798\\
-15.869	-15.3030715825492\\
-10.986	-10.594211633114\\
-14.648	-14.1256155108186\\
-12.207	-11.7716677048445\\
-14.648	-14.1256155108186\\
-10.986	-10.594211633114\\
-9.766	-9.41771989887046\\
-12.207	-11.7716677048445\\
-8.545	-8.24026382713988\\
-17.09	-16.4805276542798\\
-18.311	-17.6579837260103\\
-20.752	-20.0119315319844\\
-29.297	-28.2521953591243\\
-20.752	-20.0119315319844\\
-10.986	-10.594211633114\\
-8.545	-8.24026382713988\\
-13.428	-12.9491237765751\\
-12.207	-11.7716677048445\\
-9.766	-9.41771989887046\\
-15.869	-15.3030715825492\\
-19.531	-18.8344754602538\\
-20.752	-20.0119315319844\\
-23.193	-22.3658793379585\\
-17.09	-16.4805276542798\\
-15.869	-15.3030715825492\\
-10.986	-10.594211633114\\
-7.324	-7.0628077554093\\
-8.545	-8.24026382713988\\
-10.986	-10.594211633114\\
-17.09	-16.4805276542798\\
-18.311	-17.6579837260103\\
-17.09	-16.4805276542798\\
-24.414	-23.5433354096891\\
-25.635	-24.7207914814196\\
-15.869	-15.3030715825492\\
-19.531	-18.8344754602538\\
-23.193	-22.3658793379585\\
-26.855	-25.8972832156631\\
-25.635	-24.7207914814196\\
-17.09	-16.4805276542798\\
-10.986	-10.594211633114\\
-7.324	-7.0628077554093\\
-6.104	-5.88631602116581\\
-8.545	-8.24026382713988\\
-9.766	-9.41771989887046\\
-8.545	-8.24026382713988\\
-10.986	-10.594211633114\\
-17.09	-16.4805276542798\\
-15.869	-15.3030715825492\\
-13.428	-12.9491237765751\\
-15.869	-15.3030715825492\\
-12.207	-11.7716677048445\\
-6.104	-5.88631602116581\\
-9.766	-9.41771989887046\\
-20.752	-20.0119315319844\\
-17.09	-16.4805276542798\\
-18.311	-17.6579837260103\\
-15.869	-15.3030715825492\\
-10.986	-10.594211633114\\
-17.09	-16.4805276542798\\
-23.193	-22.3658793379585\\
-17.09	-16.4805276542798\\
-26.855	-25.8972832156631\\
-25.635	-24.7207914814196\\
-18.311	-17.6579837260103\\
-23.193	-22.3658793379585\\
-25.635	-24.7207914814196\\
-19.531	-18.8344754602538\\
-15.869	-15.3030715825492\\
-24.414	-23.5433354096891\\
-35.4	-34.137547042803\\
-29.297	-28.2521953591243\\
-17.09	-16.4805276542798\\
-18.311	-17.6579837260103\\
-20.752	-20.0119315319844\\
-23.193	-22.3658793379585\\
-19.531	-18.8344754602538\\
-17.09	-16.4805276542798\\
-24.414	-23.5433354096891\\
-25.635	-24.7207914814196\\
-18.311	-17.6579837260103\\
-15.869	-15.3030715825492\\
-18.311	-17.6579837260103\\
-28.076	-27.0747392873937\\
-26.855	-25.8972832156631\\
-24.414	-23.5433354096891\\
-19.531	-18.8344754602538\\
-17.09	-16.4805276542798\\
-18.311	-17.6579837260103\\
-13.428	-12.9491237765751\\
-6.104	-5.88631602116581\\
-9.766	-9.41771989887046\\
-19.531	-18.8344754602538\\
-15.869	-15.3030715825492\\
-14.648	-14.1256155108186\\
-13.428	-12.9491237765751\\
-17.09	-16.4805276542798\\
-13.428	-12.9491237765751\\
-9.766	-9.41771989887046\\
-12.207	-11.7716677048445\\
-9.766	-9.41771989887046\\
-8.545	-8.24026382713988\\
-17.09	-16.4805276542798\\
-20.752	-20.0119315319844\\
-17.09	-16.4805276542798\\
-10.986	-10.594211633114\\
-13.428	-12.9491237765751\\
-9.766	-9.41771989887046\\
-10.986	-10.594211633114\\
-28.076	-27.0747392873937\\
-20.752	-20.0119315319844\\
-25.635	-24.7207914814196\\
-35.4	-34.137547042803\\
-30.518	-29.4296514308549\\
-24.414	-23.5433354096891\\
-18.311	-17.6579837260103\\
-19.531	-18.8344754602538\\
-21.973	-21.189387603715\\
-25.635	-24.7207914814196\\
-31.738	-30.6061431650984\\
-34.18	-32.9610553085595\\
-41.504	-40.0238630639688\\
-32.959	-31.7835992368289\\
-28.076	-27.0747392873937\\
-35.4	-34.137547042803\\
-26.855	-25.8972832156631\\
-29.297	-28.2521953591243\\
-28.076	-27.0747392873937\\
-17.09	-16.4805276542798\\
-10.986	-10.594211633114\\
-12.207	-11.7716677048445\\
-17.09	-16.4805276542798\\
-14.648	-14.1256155108186\\
-9.766	-9.41771989887046\\
-13.428	-12.9491237765751\\
-7.324	-7.0628077554093\\
-8.545	-8.24026382713988\\
-12.207	-11.7716677048445\\
-7.324	-7.0628077554093\\
-14.648	-14.1256155108186\\
-20.752	-20.0119315319844\\
-13.428	-12.9491237765751\\
-21.973	-21.189387603715\\
-30.518	-29.4296514308549\\
-29.297	-28.2521953591243\\
-35.4	-34.137547042803\\
-29.297	-28.2521953591243\\
-28.076	-27.0747392873937\\
-23.193	-22.3658793379585\\
-12.207	-11.7716677048445\\
-14.648	-14.1256155108186\\
-15.869	-15.3030715825492\\
-10.986	-10.594211633114\\
-12.207	-11.7716677048445\\
-13.428	-12.9491237765751\\
-3.662	-3.53140387770465\\
-8.545	-8.24026382713988\\
-17.09	-16.4805276542798\\
-19.531	-18.8344754602538\\
-20.752	-20.0119315319844\\
-23.193	-22.3658793379585\\
-20.752	-20.0119315319844\\
-23.193	-22.3658793379585\\
-18.311	-17.6579837260103\\
-15.869	-15.3030715825492\\
-12.207	-11.7716677048445\\
-18.311	-17.6579837260103\\
-17.09	-16.4805276542798\\
-12.207	-11.7716677048445\\
-17.09	-16.4805276542798\\
-23.193	-22.3658793379585\\
-17.09	-16.4805276542798\\
-13.428	-12.9491237765751\\
-12.207	-11.7716677048445\\
-14.648	-14.1256155108186\\
-7.324	-7.0628077554093\\
-8.545	-8.24026382713988\\
-12.207	-11.7716677048445\\
-17.09	-16.4805276542798\\
-14.648	-14.1256155108186\\
-15.869	-15.3030715825492\\
-18.311	-17.6579837260103\\
-12.207	-11.7716677048445\\
-9.766	-9.41771989887046\\
-18.311	-17.6579837260103\\
-26.855	-25.8972832156631\\
-21.973	-21.189387603715\\
-19.531	-18.8344754602538\\
-17.09	-16.4805276542798\\
-14.648	-14.1256155108186\\
-13.428	-12.9491237765751\\
-17.09	-16.4805276542798\\
-10.986	-10.594211633114\\
-14.648	-14.1256155108186\\
-18.311	-17.6579837260103\\
-19.531	-18.8344754602538\\
-21.973	-21.189387603715\\
-29.297	-28.2521953591243\\
-35.4	-34.137547042803\\
-36.621	-35.3150031145336\\
-26.855	-25.8972832156631\\
-14.648	-14.1256155108186\\
-10.986	-10.594211633114\\
-7.324	-7.0628077554093\\
-10.986	-10.594211633114\\
-14.648	-14.1256155108186\\
-13.428	-12.9491237765751\\
-12.207	-11.7716677048445\\
-15.869	-15.3030715825492\\
-20.752	-20.0119315319844\\
-24.414	-23.5433354096891\\
-31.738	-30.6061431650984\\
-29.297	-28.2521953591243\\
-26.855	-25.8972832156631\\
-18.311	-17.6579837260103\\
-17.09	-16.4805276542798\\
-20.752	-20.0119315319844\\
-17.09	-16.4805276542798\\
-14.648	-14.1256155108186\\
-18.311	-17.6579837260103\\
-12.207	-11.7716677048445\\
-13.428	-12.9491237765751\\
-20.752	-20.0119315319844\\
-17.09	-16.4805276542798\\
-18.311	-17.6579837260103\\
-32.959	-31.7835992368289\\
-25.635	-24.7207914814196\\
-20.752	-20.0119315319844\\
-28.076	-27.0747392873937\\
-19.531	-18.8344754602538\\
-13.428	-12.9491237765751\\
-19.531	-18.8344754602538\\
-30.518	-29.4296514308549\\
-34.18	-32.9610553085595\\
-31.738	-30.6061431650984\\
-25.635	-24.7207914814196\\
-17.09	-16.4805276542798\\
-14.648	-14.1256155108186\\
-13.428	-12.9491237765751\\
-10.986	-10.594211633114\\
-8.545	-8.24026382713988\\
-12.207	-11.7716677048445\\
-14.648	-14.1256155108186\\
-10.986	-10.594211633114\\
-13.428	-12.9491237765751\\
-19.531	-18.8344754602538\\
-20.752	-20.0119315319844\\
-21.973	-21.189387603715\\
-29.297	-28.2521953591243\\
-34.18	-32.9610553085595\\
-25.635	-24.7207914814196\\
-23.193	-22.3658793379585\\
-24.414	-23.5433354096891\\
-20.752	-20.0119315319844\\
-14.648	-14.1256155108186\\
-18.311	-17.6579837260103\\
-29.297	-28.2521953591243\\
-31.738	-30.6061431650984\\
-19.531	-18.8344754602538\\
-12.207	-11.7716677048445\\
-8.545	-8.24026382713988\\
-7.324	-7.0628077554093\\
-9.766	-9.41771989887046\\
-8.545	-8.24026382713988\\
-12.207	-11.7716677048445\\
-14.648	-14.1256155108186\\
-13.428	-12.9491237765751\\
-9.766	-9.41771989887046\\
-10.986	-10.594211633114\\
-9.766	-9.41771989887046\\
-10.986	-10.594211633114\\
-14.648	-14.1256155108186\\
-21.973	-21.189387603715\\
-18.311	-17.6579837260103\\
-20.752	-20.0119315319844\\
-21.973	-21.189387603715\\
-30.518	-29.4296514308549\\
-31.738	-30.6061431650984\\
-34.18	-32.9610553085595\\
-24.414	-23.5433354096891\\
-17.09	-16.4805276542798\\
-15.869	-15.3030715825492\\
-26.855	-25.8972832156631\\
-31.738	-30.6061431650984\\
-23.193	-22.3658793379585\\
-18.311	-17.6579837260103\\
-9.766	-9.41771989887046\\
-4.883	-4.70885994943523\\
-6.104	-5.88631602116581\\
-7.324	-7.0628077554093\\
-14.648	-14.1256155108186\\
-10.986	-10.594211633114\\
-12.207	-11.7716677048445\\
-8.545	-8.24026382713988\\
-4.883	-4.70885994943523\\
-8.545	-8.24026382713988\\
-9.766	-9.41771989887046\\
-7.324	-7.0628077554093\\
-10.986	-10.594211633114\\
-24.414	-23.5433354096891\\
-26.855	-25.8972832156631\\
-25.635	-24.7207914814196\\
-21.973	-21.189387603715\\
-30.518	-29.4296514308549\\
-40.283	-38.8464069922382\\
-34.18	-32.9610553085595\\
-31.738	-30.6061431650984\\
-26.855	-25.8972832156631\\
-28.076	-27.0747392873937\\
-29.297	-28.2521953591243\\
-20.752	-20.0119315319844\\
-17.09	-16.4805276542798\\
-14.648	-14.1256155108186\\
-12.207	-11.7716677048445\\
-20.752	-20.0119315319844\\
-18.311	-17.6579837260103\\
-15.869	-15.3030715825492\\
-18.311	-17.6579837260103\\
-14.648	-14.1256155108186\\
-17.09	-16.4805276542798\\
-20.752	-20.0119315319844\\
-19.531	-18.8344754602538\\
-13.428	-12.9491237765751\\
-17.09	-16.4805276542798\\
-9.766	-9.41771989887046\\
-4.883	-4.70885994943523\\
-10.986	-10.594211633114\\
-13.428	-12.9491237765751\\
-8.545	-8.24026382713988\\
-2.441	-2.35394780597407\\
-8.545	-8.24026382713988\\
-12.207	-11.7716677048445\\
-15.869	-15.3030715825492\\
-13.428	-12.9491237765751\\
-18.311	-17.6579837260103\\
-17.09	-16.4805276542798\\
-10.986	-10.594211633114\\
-17.09	-16.4805276542798\\
-14.648	-14.1256155108186\\
-9.766	-9.41771989887046\\
-7.324	-7.0628077554093\\
-6.104	-5.88631602116581\\
-7.324	-7.0628077554093\\
-8.545	-8.24026382713988\\
-6.104	-5.88631602116581\\
-13.428	-12.9491237765751\\
-17.09	-16.4805276542798\\
-10.986	-10.594211633114\\
-14.648	-14.1256155108186\\
-10.986	-10.594211633114\\
-14.648	-14.1256155108186\\
-10.986	-10.594211633114\\
-9.766	-9.41771989887046\\
-7.324	-7.0628077554093\\
-8.545	-8.24026382713988\\
-10.986	-10.594211633114\\
-14.648	-14.1256155108186\\
-12.207	-11.7716677048445\\
-8.545	-8.24026382713988\\
-10.986	-10.594211633114\\
-21.973	-21.189387603715\\
-24.414	-23.5433354096891\\
-18.311	-17.6579837260103\\
-17.09	-16.4805276542798\\
-13.428	-12.9491237765751\\
-19.531	-18.8344754602538\\
-17.09	-16.4805276542798\\
-9.766	-9.41771989887046\\
-14.648	-14.1256155108186\\
-13.428	-12.9491237765751\\
-15.869	-15.3030715825492\\
-12.207	-11.7716677048445\\
-14.648	-14.1256155108186\\
-12.207	-11.7716677048445\\
-15.869	-15.3030715825492\\
-12.207	-11.7716677048445\\
-13.428	-12.9491237765751\\
-14.648	-14.1256155108186\\
-19.531	-18.8344754602538\\
-18.311	-17.6579837260103\\
-21.973	-21.189387603715\\
-30.518	-29.4296514308549\\
-24.414	-23.5433354096891\\
-14.648	-14.1256155108186\\
-15.869	-15.3030715825492\\
-23.193	-22.3658793379585\\
-18.311	-17.6579837260103\\
-23.193	-22.3658793379585\\
-17.09	-16.4805276542798\\
-8.545	-8.24026382713988\\
-9.766	-9.41771989887046\\
-19.531	-18.8344754602538\\
-14.648	-14.1256155108186\\
-13.428	-12.9491237765751\\
-20.752	-20.0119315319844\\
-21.973	-21.189387603715\\
-18.311	-17.6579837260103\\
-14.648	-14.1256155108186\\
-18.311	-17.6579837260103\\
-20.752	-20.0119315319844\\
-18.311	-17.6579837260103\\
-15.869	-15.3030715825492\\
-12.207	-11.7716677048445\\
-14.648	-14.1256155108186\\
-12.207	-11.7716677048445\\
-13.428	-12.9491237765751\\
-19.531	-18.8344754602538\\
-20.752	-20.0119315319844\\
-19.531	-18.8344754602538\\
-14.648	-14.1256155108186\\
-10.986	-10.594211633114\\
-13.428	-12.9491237765751\\
-14.648	-14.1256155108186\\
-18.311	-17.6579837260103\\
-20.752	-20.0119315319844\\
-24.414	-23.5433354096891\\
-18.311	-17.6579837260103\\
-12.207	-11.7716677048445\\
-21.973	-21.189387603715\\
-30.518	-29.4296514308549\\
-20.752	-20.0119315319844\\
-25.635	-24.7207914814196\\
-18.311	-17.6579837260103\\
-17.09	-16.4805276542798\\
-18.311	-17.6579837260103\\
-15.869	-15.3030715825492\\
-18.311	-17.6579837260103\\
-21.973	-21.189387603715\\
-17.09	-16.4805276542798\\
-20.752	-20.0119315319844\\
-18.311	-17.6579837260103\\
-14.648	-14.1256155108186\\
-21.973	-21.189387603715\\
-14.648	-14.1256155108186\\
-17.09	-16.4805276542798\\
-10.986	-10.594211633114\\
-6.104	-5.88631602116581\\
-4.883	-4.70885994943523\\
-10.986	-10.594211633114\\
-20.752	-20.0119315319844\\
-21.973	-21.189387603715\\
-19.531	-18.8344754602538\\
-13.428	-12.9491237765751\\
-17.09	-16.4805276542798\\
-14.648	-14.1256155108186\\
-9.766	-9.41771989887046\\
-15.869	-15.3030715825492\\
-14.648	-14.1256155108186\\
-12.207	-11.7716677048445\\
-13.428	-12.9491237765751\\
-12.207	-11.7716677048445\\
-9.766	-9.41771989887046\\
-7.324	-7.0628077554093\\
-8.545	-8.24026382713988\\
-18.311	-17.6579837260103\\
-13.428	-12.9491237765751\\
-18.311	-17.6579837260103\\
-15.869	-15.3030715825492\\
-21.973	-21.189387603715\\
-19.531	-18.8344754602538\\
-25.635	-24.7207914814196\\
-15.869	-15.3030715825492\\
-23.193	-22.3658793379585\\
-21.973	-21.189387603715\\
-13.428	-12.9491237765751\\
-17.09	-16.4805276542798\\
-14.648	-14.1256155108186\\
-19.531	-18.8344754602538\\
-20.752	-20.0119315319844\\
-24.414	-23.5433354096891\\
-17.09	-16.4805276542798\\
-24.414	-23.5433354096891\\
-25.635	-24.7207914814196\\
-23.193	-22.3658793379585\\
-18.311	-17.6579837260103\\
-14.648	-14.1256155108186\\
-19.531	-18.8344754602538\\
-17.09	-16.4805276542798\\
-14.648	-14.1256155108186\\
-12.207	-11.7716677048445\\
-13.428	-12.9491237765751\\
-12.207	-11.7716677048445\\
-25.635	-24.7207914814196\\
-30.518	-29.4296514308549\\
-25.635	-24.7207914814196\\
-23.193	-22.3658793379585\\
-25.635	-24.7207914814196\\
-14.648	-14.1256155108186\\
-13.428	-12.9491237765751\\
-9.766	-9.41771989887046\\
-12.207	-11.7716677048445\\
-18.311	-17.6579837260103\\
-19.531	-18.8344754602538\\
-23.193	-22.3658793379585\\
-18.311	-17.6579837260103\\
-15.869	-15.3030715825492\\
-19.531	-18.8344754602538\\
-28.076	-27.0747392873937\\
-23.193	-22.3658793379585\\
-37.842	-36.4924591862642\\
-25.635	-24.7207914814196\\
-15.869	-15.3030715825492\\
-13.428	-12.9491237765751\\
-12.207	-11.7716677048445\\
-19.531	-18.8344754602538\\
-23.193	-22.3658793379585\\
-28.076	-27.0747392873937\\
-24.414	-23.5433354096891\\
-18.311	-17.6579837260103\\
-10.986	-10.594211633114\\
-14.648	-14.1256155108186\\
-10.986	-10.594211633114\\
-13.428	-12.9491237765751\\
-8.545	-8.24026382713988\\
-10.986	-10.594211633114\\
-8.545	-8.24026382713988\\
-7.324	-7.0628077554093\\
-9.766	-9.41771989887046\\
-13.428	-12.9491237765751\\
-15.869	-15.3030715825492\\
-17.09	-16.4805276542798\\
-15.869	-15.3030715825492\\
-19.531	-18.8344754602538\\
-12.207	-11.7716677048445\\
-21.973	-21.189387603715\\
-23.193	-22.3658793379585\\
-20.752	-20.0119315319844\\
-18.311	-17.6579837260103\\
-19.531	-18.8344754602538\\
-24.414	-23.5433354096891\\
-19.531	-18.8344754602538\\
-15.869	-15.3030715825492\\
-18.311	-17.6579837260103\\
-17.09	-16.4805276542798\\
-13.428	-12.9491237765751\\
-20.752	-20.0119315319844\\
-25.635	-24.7207914814196\\
-23.193	-22.3658793379585\\
-25.635	-24.7207914814196\\
-32.959	-31.7835992368289\\
-24.414	-23.5433354096891\\
-23.193	-22.3658793379585\\
-19.531	-18.8344754602538\\
-23.193	-22.3658793379585\\
-29.297	-28.2521953591243\\
-20.752	-20.0119315319844\\
-19.531	-18.8344754602538\\
-24.414	-23.5433354096891\\
-17.09	-16.4805276542798\\
-10.986	-10.594211633114\\
-15.869	-15.3030715825492\\
-8.545	-8.24026382713988\\
-9.766	-9.41771989887046\\
-10.986	-10.594211633114\\
-14.648	-14.1256155108186\\
-17.09	-16.4805276542798\\
-19.531	-18.8344754602538\\
-20.752	-20.0119315319844\\
-21.973	-21.189387603715\\
-25.635	-24.7207914814196\\
-24.414	-23.5433354096891\\
-17.09	-16.4805276542798\\
-20.752	-20.0119315319844\\
-17.09	-16.4805276542798\\
-19.531	-18.8344754602538\\
-24.414	-23.5433354096891\\
-21.973	-21.189387603715\\
-23.193	-22.3658793379585\\
-20.752	-20.0119315319844\\
-19.531	-18.8344754602538\\
-26.855	-25.8972832156631\\
-21.973	-21.189387603715\\
-23.193	-22.3658793379585\\
-21.973	-21.189387603715\\
-14.648	-14.1256155108186\\
-9.766	-9.41771989887046\\
-8.545	-8.24026382713988\\
-13.428	-12.9491237765751\\
-12.207	-11.7716677048445\\
-17.09	-16.4805276542798\\
-14.648	-14.1256155108186\\
-18.311	-17.6579837260103\\
-25.635	-24.7207914814196\\
-30.518	-29.4296514308549\\
-21.973	-21.189387603715\\
-26.855	-25.8972832156631\\
-23.193	-22.3658793379585\\
-18.311	-17.6579837260103\\
-19.531	-18.8344754602538\\
-18.311	-17.6579837260103\\
-20.752	-20.0119315319844\\
-25.635	-24.7207914814196\\
-32.959	-31.7835992368289\\
-29.297	-28.2521953591243\\
-30.518	-29.4296514308549\\
-23.193	-22.3658793379585\\
-25.635	-24.7207914814196\\
-19.531	-18.8344754602538\\
-18.311	-17.6579837260103\\
-24.414	-23.5433354096891\\
-29.297	-28.2521953591243\\
-9.766	-9.41771989887046\\
-19.531	-18.8344754602538\\
-12.207	-11.7716677048445\\
-18.311	-17.6579837260103\\
-10.986	-10.594211633114\\
-12.207	-11.7716677048445\\
-23.193	-22.3658793379585\\
-37.842	-36.4924591862642\\
-35.4	-34.137547042803\\
-34.18	-32.9610553085595\\
-26.855	-25.8972832156631\\
-31.738	-30.6061431650984\\
-36.621	-35.3150031145336\\
-28.076	-27.0747392873937\\
-30.518	-29.4296514308549\\
-31.738	-30.6061431650984\\
-23.193	-22.3658793379585\\
-18.311	-17.6579837260103\\
-24.414	-23.5433354096891\\
-17.09	-16.4805276542798\\
-13.428	-12.9491237765751\\
-18.311	-17.6579837260103\\
-12.207	-11.7716677048445\\
-17.09	-16.4805276542798\\
-14.648	-14.1256155108186\\
-18.311	-17.6579837260103\\
-20.752	-20.0119315319844\\
-15.869	-15.3030715825492\\
-12.207	-11.7716677048445\\
-15.869	-15.3030715825492\\
-18.311	-17.6579837260103\\
-20.752	-20.0119315319844\\
-18.311	-17.6579837260103\\
-24.414	-23.5433354096891\\
-23.193	-22.3658793379585\\
-15.869	-15.3030715825492\\
-25.635	-24.7207914814196\\
-20.752	-20.0119315319844\\
-17.09	-16.4805276542798\\
-14.648	-14.1256155108186\\
-9.766	-9.41771989887046\\
-7.324	-7.0628077554093\\
-15.869	-15.3030715825492\\
-13.428	-12.9491237765751\\
-10.986	-10.594211633114\\
-7.324	-7.0628077554093\\
-12.207	-11.7716677048445\\
-14.648	-14.1256155108186\\
-15.869	-15.3030715825492\\
-19.531	-18.8344754602538\\
-23.193	-22.3658793379585\\
-21.973	-21.189387603715\\
-24.414	-23.5433354096891\\
-17.09	-16.4805276542798\\
-8.545	-8.24026382713988\\
-14.648	-14.1256155108186\\
-9.766	-9.41771989887046\\
-13.428	-12.9491237765751\\
-17.09	-16.4805276542798\\
-18.311	-17.6579837260103\\
-28.076	-27.0747392873937\\
-18.311	-17.6579837260103\\
-21.973	-21.189387603715\\
-23.193	-22.3658793379585\\
-18.311	-17.6579837260103\\
-14.648	-14.1256155108186\\
-13.428	-12.9491237765751\\
-17.09	-16.4805276542798\\
-18.311	-17.6579837260103\\
-26.855	-25.8972832156631\\
-15.869	-15.3030715825492\\
-17.09	-16.4805276542798\\
-20.752	-20.0119315319844\\
-32.959	-31.7835992368289\\
-28.076	-27.0747392873937\\
-19.531	-18.8344754602538\\
-13.428	-12.9491237765751\\
-15.869	-15.3030715825492\\
-9.766	-9.41771989887046\\
-6.104	-5.88631602116581\\
-17.09	-16.4805276542798\\
-19.531	-18.8344754602538\\
-25.635	-24.7207914814196\\
-28.076	-27.0747392873937\\
-35.4	-34.137547042803\\
-31.738	-30.6061431650984\\
-18.311	-17.6579837260103\\
-13.428	-12.9491237765751\\
-18.311	-17.6579837260103\\
-24.414	-23.5433354096891\\
-29.297	-28.2521953591243\\
-23.193	-22.3658793379585\\
-18.311	-17.6579837260103\\
-20.752	-20.0119315319844\\
-25.635	-24.7207914814196\\
-18.311	-17.6579837260103\\
-13.428	-12.9491237765751\\
-12.207	-11.7716677048445\\
-7.324	-7.0628077554093\\
-8.545	-8.24026382713988\\
-4.883	-4.70885994943523\\
-12.207	-11.7716677048445\\
-15.869	-15.3030715825492\\
-12.207	-11.7716677048445\\
-13.428	-12.9491237765751\\
-7.324	-7.0628077554093\\
-3.662	-3.53140387770465\\
-7.324	-7.0628077554093\\
-10.986	-10.594211633114\\
-12.207	-11.7716677048445\\
-14.648	-14.1256155108186\\
-18.311	-17.6579837260103\\
-20.752	-20.0119315319844\\
-24.414	-23.5433354096891\\
-34.18	-32.9610553085595\\
-32.959	-31.7835992368289\\
-36.621	-35.3150031145336\\
-31.738	-30.6061431650984\\
-20.752	-20.0119315319844\\
-18.311	-17.6579837260103\\
-21.973	-21.189387603715\\
-17.09	-16.4805276542798\\
-15.869	-15.3030715825492\\
-12.207	-11.7716677048445\\
-17.09	-16.4805276542798\\
-14.648	-14.1256155108186\\
-8.545	-8.24026382713988\\
-7.324	-7.0628077554093\\
-10.986	-10.594211633114\\
-14.648	-14.1256155108186\\
-13.428	-12.9491237765751\\
-14.648	-14.1256155108186\\
-12.207	-11.7716677048445\\
-15.869	-15.3030715825492\\
-23.193	-22.3658793379585\\
-17.09	-16.4805276542798\\
-24.414	-23.5433354096891\\
-32.959	-31.7835992368289\\
-35.4	-34.137547042803\\
-26.855	-25.8972832156631\\
-19.531	-18.8344754602538\\
-14.648	-14.1256155108186\\
-9.766	-9.41771989887046\\
-15.869	-15.3030715825492\\
-10.986	-10.594211633114\\
-17.09	-16.4805276542798\\
-14.648	-14.1256155108186\\
-12.207	-11.7716677048445\\
-17.09	-16.4805276542798\\
-12.207	-11.7716677048445\\
-13.428	-12.9491237765751\\
-9.766	-9.41771989887046\\
-8.545	-8.24026382713988\\
-7.324	-7.0628077554093\\
-6.104	-5.88631602116581\\
-9.766	-9.41771989887046\\
-10.986	-10.594211633114\\
-8.545	-8.24026382713988\\
-15.869	-15.3030715825492\\
-17.09	-16.4805276542798\\
-14.648	-14.1256155108186\\
-20.752	-20.0119315319844\\
-25.635	-24.7207914814196\\
-18.311	-17.6579837260103\\
-21.973	-21.189387603715\\
-9.766	-9.41771989887046\\
-6.104	-5.88631602116581\\
-14.648	-14.1256155108186\\
-18.311	-17.6579837260103\\
-19.531	-18.8344754602538\\
-26.855	-25.8972832156631\\
-25.635	-24.7207914814196\\
-17.09	-16.4805276542798\\
-13.428	-12.9491237765751\\
-15.869	-15.3030715825492\\
-14.648	-14.1256155108186\\
-10.986	-10.594211633114\\
-7.324	-7.0628077554093\\
-12.207	-11.7716677048445\\
-7.324	-7.0628077554093\\
-6.104	-5.88631602116581\\
-7.324	-7.0628077554093\\
-9.766	-9.41771989887046\\
-13.428	-12.9491237765751\\
-9.766	-9.41771989887046\\
-17.09	-16.4805276542798\\
-23.193	-22.3658793379585\\
-14.648	-14.1256155108186\\
-20.752	-20.0119315319844\\
-23.193	-22.3658793379585\\
-15.869	-15.3030715825492\\
-10.986	-10.594211633114\\
-14.648	-14.1256155108186\\
-9.766	-9.41771989887046\\
-12.207	-11.7716677048445\\
-7.324	-7.0628077554093\\
-6.104	-5.88631602116581\\
-8.545	-8.24026382713988\\
-12.207	-11.7716677048445\\
-10.986	-10.594211633114\\
-12.207	-11.7716677048445\\
-14.648	-14.1256155108186\\
-9.766	-9.41771989887046\\
-7.324	-7.0628077554093\\
-6.104	-5.88631602116581\\
-8.545	-8.24026382713988\\
-9.766	-9.41771989887046\\
-18.311	-17.6579837260103\\
-14.648	-14.1256155108186\\
-19.531	-18.8344754602538\\
-24.414	-23.5433354096891\\
-25.635	-24.7207914814196\\
-26.855	-25.8972832156631\\
-21.973	-21.189387603715\\
-23.193	-22.3658793379585\\
-25.635	-24.7207914814196\\
-30.518	-29.4296514308549\\
-26.855	-25.8972832156631\\
-24.414	-23.5433354096891\\
-20.752	-20.0119315319844\\
-19.531	-18.8344754602538\\
-24.414	-23.5433354096891\\
-15.869	-15.3030715825492\\
-19.531	-18.8344754602538\\
-21.973	-21.189387603715\\
-24.414	-23.5433354096891\\
-17.09	-16.4805276542798\\
-20.752	-20.0119315319844\\
-18.311	-17.6579837260103\\
-17.09	-16.4805276542798\\
-12.207	-11.7716677048445\\
-17.09	-16.4805276542798\\
-24.414	-23.5433354096891\\
-19.531	-18.8344754602538\\
-10.986	-10.594211633114\\
-14.648	-14.1256155108186\\
-12.207	-11.7716677048445\\
-9.766	-9.41771989887046\\
-10.986	-10.594211633114\\
-7.324	-7.0628077554093\\
-9.766	-9.41771989887046\\
-8.545	-8.24026382713988\\
-6.104	-5.88631602116581\\
-7.324	-7.0628077554093\\
-9.766	-9.41771989887046\\
-7.324	-7.0628077554093\\
-8.545	-8.24026382713988\\
-9.766	-9.41771989887046\\
-10.986	-10.594211633114\\
-13.428	-12.9491237765751\\
-12.207	-11.7716677048445\\
-15.869	-15.3030715825492\\
-13.428	-12.9491237765751\\
-14.648	-14.1256155108186\\
-24.414	-23.5433354096891\\
-17.09	-16.4805276542798\\
-15.869	-15.3030715825492\\
-18.311	-17.6579837260103\\
-13.428	-12.9491237765751\\
-8.545	-8.24026382713988\\
-19.531	-18.8344754602538\\
-10.986	-10.594211633114\\
-8.545	-8.24026382713988\\
-10.986	-10.594211633114\\
-13.428	-12.9491237765751\\
-12.207	-11.7716677048445\\
-10.986	-10.594211633114\\
-6.104	-5.88631602116581\\
-12.207	-11.7716677048445\\
-10.986	-10.594211633114\\
-12.207	-11.7716677048445\\
-15.869	-15.3030715825492\\
-19.531	-18.8344754602538\\
-15.869	-15.3030715825492\\
-20.752	-20.0119315319844\\
-24.414	-23.5433354096891\\
-20.752	-20.0119315319844\\
-31.738	-30.6061431650984\\
-23.193	-22.3658793379585\\
-30.518	-29.4296514308549\\
-24.414	-23.5433354096891\\
-26.855	-25.8972832156631\\
-30.518	-29.4296514308549\\
-20.752	-20.0119315319844\\
-10.986	-10.594211633114\\
-12.207	-11.7716677048445\\
-15.869	-15.3030715825492\\
-18.311	-17.6579837260103\\
-15.869	-15.3030715825492\\
-9.766	-9.41771989887046\\
-10.986	-10.594211633114\\
-18.311	-17.6579837260103\\
-24.414	-23.5433354096891\\
-23.193	-22.3658793379585\\
-17.09	-16.4805276542798\\
-12.207	-11.7716677048445\\
-14.648	-14.1256155108186\\
-24.414	-23.5433354096891\\
-25.635	-24.7207914814196\\
-26.855	-25.8972832156631\\
-20.752	-20.0119315319844\\
-26.855	-25.8972832156631\\
-29.297	-28.2521953591243\\
-26.855	-25.8972832156631\\
-35.4	-34.137547042803\\
-31.738	-30.6061431650984\\
-26.855	-25.8972832156631\\
-31.738	-30.6061431650984\\
-26.855	-25.8972832156631\\
-17.09	-16.4805276542798\\
-14.648	-14.1256155108186\\
-18.311	-17.6579837260103\\
-20.752	-20.0119315319844\\
-21.973	-21.189387603715\\
-18.311	-17.6579837260103\\
-20.752	-20.0119315319844\\
-18.311	-17.6579837260103\\
-14.648	-14.1256155108186\\
-17.09	-16.4805276542798\\
-25.635	-24.7207914814196\\
-24.414	-23.5433354096891\\
-20.752	-20.0119315319844\\
-14.648	-14.1256155108186\\
-12.207	-11.7716677048445\\
-13.428	-12.9491237765751\\
-14.648	-14.1256155108186\\
-9.766	-9.41771989887046\\
-8.545	-8.24026382713988\\
-14.648	-14.1256155108186\\
-17.09	-16.4805276542798\\
-20.752	-20.0119315319844\\
-19.531	-18.8344754602538\\
-12.207	-11.7716677048445\\
-8.545	-8.24026382713988\\
-9.766	-9.41771989887046\\
-8.545	-8.24026382713988\\
-13.428	-12.9491237765751\\
-14.648	-14.1256155108186\\
-13.428	-12.9491237765751\\
-19.531	-18.8344754602538\\
-25.635	-24.7207914814196\\
-24.414	-23.5433354096891\\
-26.855	-25.8972832156631\\
-30.518	-29.4296514308549\\
-17.09	-16.4805276542798\\
-14.648	-14.1256155108186\\
-15.869	-15.3030715825492\\
-18.311	-17.6579837260103\\
-15.869	-15.3030715825492\\
-10.986	-10.594211633114\\
-9.766	-9.41771989887046\\
-17.09	-16.4805276542798\\
-20.752	-20.0119315319844\\
-18.311	-17.6579837260103\\
-26.855	-25.8972832156631\\
-31.738	-30.6061431650984\\
-21.973	-21.189387603715\\
-14.648	-14.1256155108186\\
-13.428	-12.9491237765751\\
-10.986	-10.594211633114\\
-18.311	-17.6579837260103\\
-17.09	-16.4805276542798\\
-15.869	-15.3030715825492\\
-18.311	-17.6579837260103\\
-23.193	-22.3658793379585\\
-20.752	-20.0119315319844\\
-24.414	-23.5433354096891\\
-17.09	-16.4805276542798\\
-18.311	-17.6579837260103\\
-14.648	-14.1256155108186\\
-12.207	-11.7716677048445\\
-17.09	-16.4805276542798\\
-24.414	-23.5433354096891\\
-25.635	-24.7207914814196\\
-19.531	-18.8344754602538\\
-35.4	-34.137547042803\\
-26.855	-25.8972832156631\\
-20.752	-20.0119315319844\\
-13.428	-12.9491237765751\\
-17.09	-16.4805276542798\\
-15.869	-15.3030715825492\\
-17.09	-16.4805276542798\\
-14.648	-14.1256155108186\\
-13.428	-12.9491237765751\\
-28.076	-27.0747392873937\\
-30.518	-29.4296514308549\\
-18.311	-17.6579837260103\\
-23.193	-22.3658793379585\\
-24.414	-23.5433354096891\\
-18.311	-17.6579837260103\\
-15.869	-15.3030715825492\\
-18.311	-17.6579837260103\\
-21.973	-21.189387603715\\
-28.076	-27.0747392873937\\
-32.959	-31.7835992368289\\
-42.725	-41.2013191356994\\
-35.4	-34.137547042803\\
-24.414	-23.5433354096891\\
-21.973	-21.189387603715\\
-18.311	-17.6579837260103\\
-13.428	-12.9491237765751\\
-12.207	-11.7716677048445\\
-13.428	-12.9491237765751\\
-8.545	-8.24026382713988\\
-9.766	-9.41771989887046\\
-14.648	-14.1256155108186\\
-12.207	-11.7716677048445\\
-8.545	-8.24026382713988\\
-7.324	-7.0628077554093\\
-14.648	-14.1256155108186\\
-23.193	-22.3658793379585\\
-12.207	-11.7716677048445\\
-17.09	-16.4805276542798\\
-14.648	-14.1256155108186\\
-17.09	-16.4805276542798\\
-15.869	-15.3030715825492\\
-10.986	-10.594211633114\\
-9.766	-9.41771989887046\\
-6.104	-5.88631602116581\\
-8.545	-8.24026382713988\\
-15.869	-15.3030715825492\\
-18.311	-17.6579837260103\\
-13.428	-12.9491237765751\\
-9.766	-9.41771989887046\\
-7.324	-7.0628077554093\\
-12.207	-11.7716677048445\\
-13.428	-12.9491237765751\\
-18.311	-17.6579837260103\\
-14.648	-14.1256155108186\\
-12.207	-11.7716677048445\\
-13.428	-12.9491237765751\\
-18.311	-17.6579837260103\\
-25.635	-24.7207914814196\\
-28.076	-27.0747392873937\\
-23.193	-22.3658793379585\\
-14.648	-14.1256155108186\\
-15.869	-15.3030715825492\\
-14.648	-14.1256155108186\\
-12.207	-11.7716677048445\\
-14.648	-14.1256155108186\\
-10.986	-10.594211633114\\
-8.545	-8.24026382713988\\
-13.428	-12.9491237765751\\
-17.09	-16.4805276542798\\
-21.973	-21.189387603715\\
-23.193	-22.3658793379585\\
-30.518	-29.4296514308549\\
-40.283	-38.8464069922382\\
-34.18	-32.9610553085595\\
-23.193	-22.3658793379585\\
-26.855	-25.8972832156631\\
-29.297	-28.2521953591243\\
-36.621	-35.3150031145336\\
-28.076	-27.0747392873937\\
-14.648	-14.1256155108186\\
-9.766	-9.41771989887046\\
-12.207	-11.7716677048445\\
-9.766	-9.41771989887046\\
-6.104	-5.88631602116581\\
-7.324	-7.0628077554093\\
-9.766	-9.41771989887046\\
-12.207	-11.7716677048445\\
-10.986	-10.594211633114\\
-9.766	-9.41771989887046\\
-10.986	-10.594211633114\\
-14.648	-14.1256155108186\\
-13.428	-12.9491237765751\\
-10.986	-10.594211633114\\
-7.324	-7.0628077554093\\
-10.986	-10.594211633114\\
-12.207	-11.7716677048445\\
-9.766	-9.41771989887046\\
-7.324	-7.0628077554093\\
-10.986	-10.594211633114\\
-13.428	-12.9491237765751\\
-9.766	-9.41771989887046\\
-13.428	-12.9491237765751\\
-14.648	-14.1256155108186\\
-19.531	-18.8344754602538\\
-13.428	-12.9491237765751\\
-15.869	-15.3030715825492\\
-17.09	-16.4805276542798\\
-15.869	-15.3030715825492\\
-13.428	-12.9491237765751\\
-18.311	-17.6579837260103\\
-25.635	-24.7207914814196\\
-23.193	-22.3658793379585\\
-25.635	-24.7207914814196\\
-21.973	-21.189387603715\\
-19.531	-18.8344754602538\\
-18.311	-17.6579837260103\\
-21.973	-21.189387603715\\
-19.531	-18.8344754602538\\
-20.752	-20.0119315319844\\
-23.193	-22.3658793379585\\
-40.283	-38.8464069922382\\
-34.18	-32.9610553085595\\
-17.09	-16.4805276542798\\
-12.207	-11.7716677048445\\
-14.648	-14.1256155108186\\
-17.09	-16.4805276542798\\
-20.752	-20.0119315319844\\
-23.193	-22.3658793379585\\
-24.414	-23.5433354096891\\
-26.855	-25.8972832156631\\
-19.531	-18.8344754602538\\
-15.869	-15.3030715825492\\
-19.531	-18.8344754602538\\
-20.752	-20.0119315319844\\
-12.207	-11.7716677048445\\
-10.986	-10.594211633114\\
-13.428	-12.9491237765751\\
-17.09	-16.4805276542798\\
-23.193	-22.3658793379585\\
-21.973	-21.189387603715\\
-14.648	-14.1256155108186\\
-12.207	-11.7716677048445\\
-18.311	-17.6579837260103\\
-23.193	-22.3658793379585\\
-26.855	-25.8972832156631\\
-31.738	-30.6061431650984\\
-36.621	-35.3150031145336\\
-30.518	-29.4296514308549\\
-29.297	-28.2521953591243\\
-19.531	-18.8344754602538\\
-13.428	-12.9491237765751\\
-21.973	-21.189387603715\\
-28.076	-27.0747392873937\\
-18.311	-17.6579837260103\\
-20.752	-20.0119315319844\\
-35.4	-34.137547042803\\
-37.842	-36.4924591862642\\
-25.635	-24.7207914814196\\
-17.09	-16.4805276542798\\
-9.766	-9.41771989887046\\
-12.207	-11.7716677048445\\
-14.648	-14.1256155108186\\
-12.207	-11.7716677048445\\
-10.986	-10.594211633114\\
-12.207	-11.7716677048445\\
-8.545	-8.24026382713988\\
-12.207	-11.7716677048445\\
-15.869	-15.3030715825492\\
-18.311	-17.6579837260103\\
-17.09	-16.4805276542798\\
-19.531	-18.8344754602538\\
-28.076	-27.0747392873937\\
-24.414	-23.5433354096891\\
-17.09	-16.4805276542798\\
-15.869	-15.3030715825492\\
-19.531	-18.8344754602538\\
-14.648	-14.1256155108186\\
-10.986	-10.594211633114\\
-13.428	-12.9491237765751\\
-12.207	-11.7716677048445\\
-9.766	-9.41771989887046\\
-6.104	-5.88631602116581\\
-8.545	-8.24026382713988\\
-14.648	-14.1256155108186\\
-13.428	-12.9491237765751\\
-10.986	-10.594211633114\\
-8.545	-8.24026382713988\\
-15.869	-15.3030715825492\\
-18.311	-17.6579837260103\\
-10.986	-10.594211633114\\
-13.428	-12.9491237765751\\
-15.869	-15.3030715825492\\
-17.09	-16.4805276542798\\
-19.531	-18.8344754602538\\
-13.428	-12.9491237765751\\
-9.766	-9.41771989887046\\
-14.648	-14.1256155108186\\
-10.986	-10.594211633114\\
-13.428	-12.9491237765751\\
-19.531	-18.8344754602538\\
-25.635	-24.7207914814196\\
-21.973	-21.189387603715\\
-18.311	-17.6579837260103\\
-26.855	-25.8972832156631\\
-24.414	-23.5433354096891\\
-14.648	-14.1256155108186\\
-10.986	-10.594211633114\\
-9.766	-9.41771989887046\\
-8.545	-8.24026382713988\\
-7.324	-7.0628077554093\\
-14.648	-14.1256155108186\\
-13.428	-12.9491237765751\\
-8.545	-8.24026382713988\\
-12.207	-11.7716677048445\\
-13.428	-12.9491237765751\\
-9.766	-9.41771989887046\\
-8.545	-8.24026382713988\\
-4.883	-4.70885994943523\\
-7.324	-7.0628077554093\\
-14.648	-14.1256155108186\\
-19.531	-18.8344754602538\\
-17.09	-16.4805276542798\\
-12.207	-11.7716677048445\\
-14.648	-14.1256155108186\\
-15.869	-15.3030715825492\\
-20.752	-20.0119315319844\\
-23.193	-22.3658793379585\\
-19.531	-18.8344754602538\\
-10.986	-10.594211633114\\
-12.207	-11.7716677048445\\
-15.869	-15.3030715825492\\
-14.648	-14.1256155108186\\
-8.545	-8.24026382713988\\
-17.09	-16.4805276542798\\
-20.752	-20.0119315319844\\
-18.311	-17.6579837260103\\
-20.752	-20.0119315319844\\
-25.635	-24.7207914814196\\
-19.531	-18.8344754602538\\
-18.311	-17.6579837260103\\
-26.855	-25.8972832156631\\
-21.973	-21.189387603715\\
-24.414	-23.5433354096891\\
-25.635	-24.7207914814196\\
-37.842	-36.4924591862642\\
-39.063	-37.6699152579947\\
-25.635	-24.7207914814196\\
-32.959	-31.7835992368289\\
-36.621	-35.3150031145336\\
-35.4	-34.137547042803\\
-37.842	-36.4924591862642\\
-36.621	-35.3150031145336\\
-43.945	-42.3778108699429\\
-41.504	-40.0238630639688\\
-24.414	-23.5433354096891\\
-19.531	-18.8344754602538\\
-18.311	-17.6579837260103\\
-15.869	-15.3030715825492\\
-12.207	-11.7716677048445\\
-17.09	-16.4805276542798\\
-23.193	-22.3658793379585\\
-18.311	-17.6579837260103\\
-13.428	-12.9491237765751\\
-15.869	-15.3030715825492\\
-12.207	-11.7716677048445\\
-10.986	-10.594211633114\\
-7.324	-7.0628077554093\\
-8.545	-8.24026382713988\\
-15.869	-15.3030715825492\\
-8.545	-8.24026382713988\\
-6.104	-5.88631602116581\\
-3.662	-3.53140387770465\\
-7.324	-7.0628077554093\\
-8.545	-8.24026382713988\\
-6.104	-5.88631602116581\\
-12.207	-11.7716677048445\\
-14.648	-14.1256155108186\\
-12.207	-11.7716677048445\\
-15.869	-15.3030715825492\\
-21.973	-21.189387603715\\
-17.09	-16.4805276542798\\
-6.104	-5.88631602116581\\
-10.986	-10.594211633114\\
-6.104	-5.88631602116581\\
-7.324	-7.0628077554093\\
-13.428	-12.9491237765751\\
-15.869	-15.3030715825492\\
-12.207	-11.7716677048445\\
-28.076	-27.0747392873937\\
-23.193	-22.3658793379585\\
-17.09	-16.4805276542798\\
-13.428	-12.9491237765751\\
-14.648	-14.1256155108186\\
-20.752	-20.0119315319844\\
-23.193	-22.3658793379585\\
-18.311	-17.6579837260103\\
-6.104	-5.88631602116581\\
-12.207	-11.7716677048445\\
-9.766	-9.41771989887046\\
-6.104	-5.88631602116581\\
-3.662	-3.53140387770465\\
-7.324	-7.0628077554093\\
-19.531	-18.8344754602538\\
-10.986	-10.594211633114\\
-8.545	-8.24026382713988\\
-9.766	-9.41771989887046\\
-6.104	-5.88631602116581\\
-3.662	-3.53140387770465\\
-9.766	-9.41771989887046\\
-14.648	-14.1256155108186\\
-18.311	-17.6579837260103\\
-20.752	-20.0119315319844\\
-19.531	-18.8344754602538\\
-20.752	-20.0119315319844\\
-19.531	-18.8344754602538\\
-18.311	-17.6579837260103\\
-21.973	-21.189387603715\\
-30.518	-29.4296514308549\\
-26.855	-25.8972832156631\\
-17.09	-16.4805276542798\\
-8.545	-8.24026382713988\\
-12.207	-11.7716677048445\\
-26.855	-25.8972832156631\\
-21.973	-21.189387603715\\
-13.428	-12.9491237765751\\
-15.869	-15.3030715825492\\
-20.752	-20.0119315319844\\
-29.297	-28.2521953591243\\
-31.738	-30.6061431650984\\
-29.297	-28.2521953591243\\
-26.855	-25.8972832156631\\
-24.414	-23.5433354096891\\
-17.09	-16.4805276542798\\
-13.428	-12.9491237765751\\
-15.869	-15.3030715825492\\
-19.531	-18.8344754602538\\
-17.09	-16.4805276542798\\
-19.531	-18.8344754602538\\
-13.428	-12.9491237765751\\
-18.311	-17.6579837260103\\
-17.09	-16.4805276542798\\
-7.324	-7.0628077554093\\
-4.883	-4.70885994943523\\
-7.324	-7.0628077554093\\
-8.545	-8.24026382713988\\
-3.662	-3.53140387770465\\
-7.324	-7.0628077554093\\
-17.09	-16.4805276542798\\
-3.662	-3.53140387770465\\
-10.986	-10.594211633114\\
-14.648	-14.1256155108186\\
-20.752	-20.0119315319844\\
-18.311	-17.6579837260103\\
-7.324	-7.0628077554093\\
-24.414	-23.5433354096891\\
-26.855	-25.8972832156631\\
-34.18	-32.9610553085595\\
-42.725	-41.2013191356994\\
-40.283	-38.8464069922382\\
-29.297	-28.2521953591243\\
-36.621	-35.3150031145336\\
-32.959	-31.7835992368289\\
-29.297	-28.2521953591243\\
-30.518	-29.4296514308549\\
-35.4	-34.137547042803\\
-46.387	-44.732723013404\\
-36.621	-35.3150031145336\\
-18.311	-17.6579837260103\\
-10.986	-10.594211633114\\
-9.766	-9.41771989887046\\
-10.986	-10.594211633114\\
-12.207	-11.7716677048445\\
-10.986	-10.594211633114\\
-17.09	-16.4805276542798\\
-18.311	-17.6579837260103\\
-14.648	-14.1256155108186\\
-19.531	-18.8344754602538\\
-12.207	-11.7716677048445\\
-17.09	-16.4805276542798\\
-18.311	-17.6579837260103\\
-19.531	-18.8344754602538\\
-10.986	-10.594211633114\\
-6.104	-5.88631602116581\\
-10.986	-10.594211633114\\
-13.428	-12.9491237765751\\
-17.09	-16.4805276542798\\
-30.518	-29.4296514308549\\
-32.959	-31.7835992368289\\
-35.4	-34.137547042803\\
-45.166	-43.5552669416735\\
-40.283	-38.8464069922382\\
-26.855	-25.8972832156631\\
-17.09	-16.4805276542798\\
-13.428	-12.9491237765751\\
-14.648	-14.1256155108186\\
-15.869	-15.3030715825492\\
-8.545	-8.24026382713988\\
-12.207	-11.7716677048445\\
-14.648	-14.1256155108186\\
-18.311	-17.6579837260103\\
-15.869	-15.3030715825492\\
-7.324	-7.0628077554093\\
-4.883	-4.70885994943523\\
-7.324	-7.0628077554093\\
-4.883	-4.70885994943523\\
-6.104	-5.88631602116581\\
-10.986	-10.594211633114\\
-14.648	-14.1256155108186\\
-15.869	-15.3030715825492\\
-19.531	-18.8344754602538\\
-17.09	-16.4805276542798\\
-19.531	-18.8344754602538\\
-34.18	-32.9610553085595\\
-26.855	-25.8972832156631\\
-21.973	-21.189387603715\\
-28.076	-27.0747392873937\\
-31.738	-30.6061431650984\\
-20.752	-20.0119315319844\\
-18.311	-17.6579837260103\\
-13.428	-12.9491237765751\\
-15.869	-15.3030715825492\\
-28.076	-27.0747392873937\\
-41.504	-40.0238630639688\\
-46.387	-44.732723013404\\
-36.621	-35.3150031145336\\
-31.738	-30.6061431650984\\
-26.855	-25.8972832156631\\
-28.076	-27.0747392873937\\
-34.18	-32.9610553085595\\
-23.193	-22.3658793379585\\
-18.311	-17.6579837260103\\
-24.414	-23.5433354096891\\
-25.635	-24.7207914814196\\
-14.648	-14.1256155108186\\
-13.428	-12.9491237765751\\
-17.09	-16.4805276542798\\
-26.855	-25.8972832156631\\
-25.635	-24.7207914814196\\
-20.752	-20.0119315319844\\
-18.311	-17.6579837260103\\
-17.09	-16.4805276542798\\
-14.648	-14.1256155108186\\
-8.545	-8.24026382713988\\
-7.324	-7.0628077554093\\
-6.104	-5.88631602116581\\
-10.986	-10.594211633114\\
-17.09	-16.4805276542798\\
-18.311	-17.6579837260103\\
-13.428	-12.9491237765751\\
-9.766	-9.41771989887046\\
-8.545	-8.24026382713988\\
-10.986	-10.594211633114\\
-14.648	-14.1256155108186\\
-17.09	-16.4805276542798\\
-14.648	-14.1256155108186\\
-9.766	-9.41771989887046\\
-20.752	-20.0119315319844\\
-21.973	-21.189387603715\\
-15.869	-15.3030715825492\\
-4.883	-4.70885994943523\\
-10.986	-10.594211633114\\
-13.428	-12.9491237765751\\
-14.648	-14.1256155108186\\
-9.766	-9.41771989887046\\
-7.324	-7.0628077554093\\
-13.428	-12.9491237765751\\
-21.973	-21.189387603715\\
-13.428	-12.9491237765751\\
-3.662	-3.53140387770465\\
-9.766	-9.41771989887046\\
-14.648	-14.1256155108186\\
-20.752	-20.0119315319844\\
-14.648	-14.1256155108186\\
-12.207	-11.7716677048445\\
-20.752	-20.0119315319844\\
-26.855	-25.8972832156631\\
-23.193	-22.3658793379585\\
-29.297	-28.2521953591243\\
-35.4	-34.137547042803\\
-24.414	-23.5433354096891\\
-19.531	-18.8344754602538\\
-26.855	-25.8972832156631\\
-19.531	-18.8344754602538\\
-23.193	-22.3658793379585\\
-29.297	-28.2521953591243\\
-21.973	-21.189387603715\\
-10.986	-10.594211633114\\
-20.752	-20.0119315319844\\
-34.18	-32.9610553085595\\
-37.842	-36.4924591862642\\
-29.297	-28.2521953591243\\
-24.414	-23.5433354096891\\
-31.738	-30.6061431650984\\
-26.855	-25.8972832156631\\
-17.09	-16.4805276542798\\
-19.531	-18.8344754602538\\
-21.973	-21.189387603715\\
-19.531	-18.8344754602538\\
-21.973	-21.189387603715\\
-14.648	-14.1256155108186\\
-19.531	-18.8344754602538\\
-24.414	-23.5433354096891\\
-18.311	-17.6579837260103\\
-14.648	-14.1256155108186\\
-12.207	-11.7716677048445\\
-8.545	-8.24026382713988\\
-9.766	-9.41771989887046\\
-17.09	-16.4805276542798\\
-15.869	-15.3030715825492\\
-19.531	-18.8344754602538\\
-25.635	-24.7207914814196\\
-26.855	-25.8972832156631\\
-20.752	-20.0119315319844\\
-18.311	-17.6579837260103\\
-10.986	-10.594211633114\\
-14.648	-14.1256155108186\\
-24.414	-23.5433354096891\\
-18.311	-17.6579837260103\\
-8.545	-8.24026382713988\\
-15.869	-15.3030715825492\\
-25.635	-24.7207914814196\\
-23.193	-22.3658793379585\\
-30.518	-29.4296514308549\\
-39.063	-37.6699152579947\\
-30.518	-29.4296514308549\\
-26.855	-25.8972832156631\\
-29.297	-28.2521953591243\\
-20.752	-20.0119315319844\\
-14.648	-14.1256155108186\\
-17.09	-16.4805276542798\\
-24.414	-23.5433354096891\\
-25.635	-24.7207914814196\\
-13.428	-12.9491237765751\\
-14.648	-14.1256155108186\\
-12.207	-11.7716677048445\\
-17.09	-16.4805276542798\\
-14.648	-14.1256155108186\\
-12.207	-11.7716677048445\\
-20.752	-20.0119315319844\\
-24.414	-23.5433354096891\\
-15.869	-15.3030715825492\\
-12.207	-11.7716677048445\\
-20.752	-20.0119315319844\\
-13.428	-12.9491237765751\\
-9.766	-9.41771989887046\\
-10.986	-10.594211633114\\
-12.207	-11.7716677048445\\
-10.986	-10.594211633114\\
-6.104	-5.88631602116581\\
-10.986	-10.594211633114\\
-12.207	-11.7716677048445\\
-20.752	-20.0119315319844\\
-19.531	-18.8344754602538\\
-21.973	-21.189387603715\\
-28.076	-27.0747392873937\\
-14.648	-14.1256155108186\\
-25.635	-24.7207914814196\\
-36.621	-35.3150031145336\\
-26.855	-25.8972832156631\\
-25.635	-24.7207914814196\\
-39.063	-37.6699152579947\\
-31.738	-30.6061431650984\\
-25.635	-24.7207914814196\\
-20.752	-20.0119315319844\\
-25.635	-24.7207914814196\\
-18.311	-17.6579837260103\\
-17.09	-16.4805276542798\\
-28.076	-27.0747392873937\\
-31.738	-30.6061431650984\\
-18.311	-17.6579837260103\\
-8.545	-8.24026382713988\\
-15.869	-15.3030715825492\\
-8.545	-8.24026382713988\\
-7.324	-7.0628077554093\\
-10.986	-10.594211633114\\
-13.428	-12.9491237765751\\
-23.193	-22.3658793379585\\
-17.09	-16.4805276542798\\
-10.986	-10.594211633114\\
-9.766	-9.41771989887046\\
-7.324	-7.0628077554093\\
-8.545	-8.24026382713988\\
-7.324	-7.0628077554093\\
-14.648	-14.1256155108186\\
-12.207	-11.7716677048445\\
-7.324	-7.0628077554093\\
-8.545	-8.24026382713988\\
-9.766	-9.41771989887046\\
-10.986	-10.594211633114\\
-8.545	-8.24026382713988\\
-9.766	-9.41771989887046\\
-8.545	-8.24026382713988\\
-4.883	-4.70885994943523\\
-13.428	-12.9491237765751\\
-17.09	-16.4805276542798\\
-23.193	-22.3658793379585\\
-30.518	-29.4296514308549\\
-26.855	-25.8972832156631\\
-13.428	-12.9491237765751\\
-10.986	-10.594211633114\\
-12.207	-11.7716677048445\\
-7.324	-7.0628077554093\\
-9.766	-9.41771989887046\\
-10.986	-10.594211633114\\
-14.648	-14.1256155108186\\
-13.428	-12.9491237765751\\
-15.869	-15.3030715825492\\
-18.311	-17.6579837260103\\
-17.09	-16.4805276542798\\
-21.973	-21.189387603715\\
-30.518	-29.4296514308549\\
-19.531	-18.8344754602538\\
-9.766	-9.41771989887046\\
-8.545	-8.24026382713988\\
-15.869	-15.3030715825492\\
-10.986	-10.594211633114\\
-9.766	-9.41771989887046\\
-4.883	-4.70885994943523\\
-10.986	-10.594211633114\\
-7.324	-7.0628077554093\\
-3.662	-3.53140387770465\\
-6.104	-5.88631602116581\\
-10.986	-10.594211633114\\
-13.428	-12.9491237765751\\
-15.869	-15.3030715825492\\
-24.414	-23.5433354096891\\
-17.09	-16.4805276542798\\
-7.324	-7.0628077554093\\
-13.428	-12.9491237765751\\
-18.311	-17.6579837260103\\
-10.986	-10.594211633114\\
-13.428	-12.9491237765751\\
-20.752	-20.0119315319844\\
-17.09	-16.4805276542798\\
-25.635	-24.7207914814196\\
-30.518	-29.4296514308549\\
-20.752	-20.0119315319844\\
-17.09	-16.4805276542798\\
};
\end{axis}

\begin{axis}[%
width=4.927cm,
height=3cm,
at={(0cm,9.677cm)},
scale only axis,
xmin=-30,
xmax=1.221,
xlabel style={font=\color{white!15!black}},
xlabel={y(t-1)},
ymin=-26.855,
ymax=1.221,
ylabel style={font=\color{white!15!black}},
ylabel={y(t)},
axis background/.style={fill=white},
title style={font=\small},
title={C4, R = 0.6952},
axis x line*=bottom,
axis y line*=left
]
\addplot[only marks, mark=*, mark options={}, mark size=1.5000pt, color=mycolor1, fill=mycolor1] table[row sep=crcr]{%
x	y\\
-10.986	-10.986\\
-10.986	-15.869\\
-15.869	-10.986\\
-10.986	-10.986\\
-10.986	-14.648\\
-14.648	-14.648\\
-14.648	-17.09\\
-17.09	-10.986\\
-10.986	-7.324\\
-7.324	-14.648\\
-14.648	-8.545\\
-8.545	-9.766\\
-9.766	-6.104\\
-6.104	-3.662\\
-3.662	-3.662\\
-3.662	-3.662\\
-3.662	-2.441\\
-2.441	-13.428\\
-13.428	-13.428\\
-13.428	-13.428\\
-13.428	-9.766\\
-9.766	-7.324\\
-7.324	-10.986\\
-10.986	-12.207\\
-12.207	-6.104\\
-6.104	-9.766\\
-9.766	-13.428\\
-13.428	-7.324\\
-7.324	-15.869\\
-15.869	-13.428\\
-13.428	-10.986\\
-10.986	-17.09\\
-17.09	-13.428\\
-13.428	-10.986\\
-10.986	-8.545\\
-8.545	-8.545\\
-8.545	-10.986\\
-10.986	-12.207\\
-12.207	-10.986\\
-10.986	-9.766\\
-9.766	-10.986\\
-10.986	-10.986\\
-10.986	-15.869\\
-15.869	-14.648\\
-14.648	-10.986\\
-10.986	-10.986\\
-10.986	-6.104\\
-6.104	-4.883\\
-4.883	-8.545\\
-8.545	-7.324\\
-7.324	-7.324\\
-7.324	-10.986\\
-10.986	-9.766\\
-9.766	-9.766\\
-9.766	-15.869\\
-15.869	-8.545\\
-8.545	-6.104\\
-6.104	-7.324\\
-7.324	-8.545\\
-8.545	-9.766\\
-9.766	-4.883\\
-4.883	-7.324\\
-7.324	-7.324\\
-7.324	-7.324\\
-7.324	-6.104\\
-6.104	-7.324\\
-7.324	-8.545\\
-8.545	-9.766\\
-9.766	-10.986\\
-10.986	-10.986\\
-10.986	-12.207\\
-12.207	-13.428\\
-13.428	-10.986\\
-10.986	-12.207\\
-12.207	-9.766\\
-9.766	-9.766\\
-9.766	-9.766\\
-9.766	-17.09\\
-17.09	-20.752\\
-20.752	-19.531\\
-19.531	-19.531\\
-19.531	-15.869\\
-15.869	-20.752\\
-20.752	-23.193\\
-23.193	-23.193\\
-23.193	-14.648\\
-14.648	-23.193\\
-23.193	-26.855\\
-26.855	-21.973\\
-21.973	-15.869\\
-15.869	-14.648\\
-14.648	-10.986\\
-10.986	-8.545\\
-8.545	-10.986\\
-10.986	-8.545\\
-8.545	-6.104\\
-6.104	-6.104\\
-6.104	-8.545\\
-8.545	-9.766\\
-9.766	-10.986\\
-10.986	-9.766\\
-9.766	-12.207\\
-12.207	-10.986\\
-10.986	-15.869\\
-15.869	-13.428\\
-13.428	-18.311\\
-18.311	-12.207\\
-12.207	-10.986\\
-10.986	-12.207\\
-12.207	-9.766\\
-9.766	-8.545\\
-8.545	-10.986\\
-10.986	-10.986\\
-10.986	-8.545\\
-8.545	-8.545\\
-8.545	-7.324\\
-7.324	-7.324\\
-7.324	-7.324\\
-7.324	-4.883\\
-4.883	-6.104\\
-6.104	-8.545\\
-8.545	-13.428\\
-13.428	-12.207\\
-12.207	-13.428\\
-13.428	-8.545\\
-8.545	-15.869\\
-15.869	-17.09\\
-17.09	-13.428\\
-13.428	-14.648\\
-14.648	-14.648\\
-14.648	-10.986\\
-10.986	-7.324\\
-7.324	-8.545\\
-8.545	-8.545\\
-8.545	-17.09\\
-17.09	-20.752\\
-20.752	-19.531\\
-19.531	-14.648\\
-14.648	-15.869\\
-15.869	-13.428\\
-13.428	-12.207\\
-12.207	-12.207\\
-12.207	-9.766\\
-9.766	-8.545\\
-8.545	-8.545\\
-8.545	-9.766\\
-9.766	-8.545\\
-8.545	-13.428\\
-13.428	-15.869\\
-15.869	-10.986\\
-10.986	-8.545\\
-8.545	-9.766\\
-9.766	-8.545\\
-8.545	-6.104\\
-6.104	-3.662\\
-3.662	-4.883\\
-4.883	-8.545\\
-8.545	-9.766\\
-9.766	-4.883\\
-4.883	-7.324\\
-7.324	-7.324\\
-7.324	-3.662\\
-3.662	-3.662\\
-3.662	-2.441\\
-2.441	-4.883\\
-4.883	-10.986\\
-10.986	-12.207\\
-12.207	-14.648\\
-14.648	-13.428\\
-13.428	-8.545\\
-8.545	-7.324\\
-7.324	-4.883\\
-4.883	-7.324\\
-7.324	-6.104\\
-6.104	-10.986\\
-10.986	-12.207\\
-12.207	-21.973\\
-21.973	-23.193\\
-23.193	-21.973\\
-21.973	-18.311\\
-18.311	-13.428\\
-13.428	-17.09\\
-17.09	-15.869\\
-15.869	-19.531\\
-19.531	-12.207\\
-12.207	-13.428\\
-13.428	-12.207\\
-12.207	-13.428\\
-13.428	-13.428\\
-13.428	-10.986\\
-10.986	-18.311\\
-18.311	-17.09\\
-17.09	-13.428\\
-13.428	-7.324\\
-7.324	-12.207\\
-12.207	-15.869\\
-15.869	-17.09\\
-17.09	-19.531\\
-19.531	-21.973\\
-21.973	-20.752\\
-20.752	-15.869\\
-15.869	-14.648\\
-14.648	-18.311\\
-18.311	-19.531\\
-19.531	-13.428\\
-13.428	-9.766\\
-9.766	-13.428\\
-13.428	-9.766\\
-9.766	-7.324\\
-7.324	-9.766\\
-9.766	-10.986\\
-10.986	-7.324\\
-7.324	-7.324\\
-7.324	-9.766\\
-9.766	-6.104\\
-6.104	-10.986\\
-10.986	-13.428\\
-13.428	-13.428\\
-13.428	-8.545\\
-8.545	-9.766\\
-9.766	-12.207\\
-12.207	-18.311\\
-18.311	-15.869\\
-15.869	-9.766\\
-9.766	-8.545\\
-8.545	-9.766\\
-9.766	-7.324\\
-7.324	-8.545\\
-8.545	-7.324\\
-7.324	-8.545\\
-8.545	-10.986\\
-10.986	-10.986\\
-10.986	-6.104\\
-6.104	-10.986\\
-10.986	-13.428\\
-13.428	-13.428\\
-13.428	-9.766\\
-9.766	-12.207\\
-12.207	-13.428\\
-13.428	-10.986\\
-10.986	-4.883\\
-4.883	-10.986\\
-10.986	-8.545\\
-8.545	-8.545\\
-8.545	-6.104\\
-6.104	-6.104\\
-6.104	-10.986\\
-10.986	-10.986\\
-10.986	-6.104\\
-6.104	-8.545\\
-8.545	-8.545\\
-8.545	-4.883\\
-4.883	-8.545\\
-8.545	-4.883\\
-4.883	-7.324\\
-7.324	-4.883\\
-4.883	-4.883\\
-4.883	-10.986\\
-10.986	-12.207\\
-12.207	-12.207\\
-12.207	-15.869\\
-15.869	-10.986\\
-10.986	-6.104\\
-6.104	-7.324\\
-7.324	-4.883\\
-4.883	-7.324\\
-7.324	-6.104\\
-6.104	-3.662\\
-3.662	-8.545\\
-8.545	-9.766\\
-9.766	-9.766\\
-9.766	-17.09\\
-17.09	-10.986\\
-10.986	-13.428\\
-13.428	-7.324\\
-7.324	-8.545\\
-8.545	-3.662\\
-3.662	-1.221\\
-1.221	-6.104\\
-6.104	-9.766\\
-9.766	-8.545\\
-8.545	-9.766\\
-9.766	-9.766\\
-9.766	-17.09\\
-17.09	-10.986\\
-10.986	-10.986\\
-10.986	-10.986\\
-10.986	-13.428\\
-13.428	-15.869\\
-15.869	-15.869\\
-15.869	-10.986\\
-10.986	-6.104\\
-6.104	-3.662\\
-3.662	-4.883\\
-4.883	-6.104\\
-6.104	-4.883\\
-4.883	-6.104\\
-6.104	-6.104\\
-6.104	-9.766\\
-9.766	-9.766\\
-9.766	-8.545\\
-8.545	-9.766\\
-9.766	-6.104\\
-6.104	-3.662\\
-3.662	-7.324\\
-7.324	-10.986\\
-10.986	-8.545\\
-8.545	-12.207\\
-12.207	-8.545\\
-8.545	-7.324\\
-7.324	-8.545\\
-8.545	-12.207\\
-12.207	-14.648\\
-14.648	-10.986\\
-10.986	-17.09\\
-17.09	-9.766\\
-9.766	-12.207\\
-12.207	-13.428\\
-13.428	-13.428\\
-13.428	-9.766\\
-9.766	-9.766\\
-9.766	-12.207\\
-12.207	-18.311\\
-18.311	-15.869\\
-15.869	-8.545\\
-8.545	-12.207\\
-12.207	-9.766\\
-9.766	-12.207\\
-12.207	-13.428\\
-13.428	-9.766\\
-9.766	-9.766\\
-9.766	-14.648\\
-14.648	-13.428\\
-13.428	-12.207\\
-12.207	-8.545\\
-8.545	-9.766\\
-9.766	-15.869\\
-15.869	-14.648\\
-14.648	-15.869\\
-15.869	-12.207\\
-12.207	-10.986\\
-10.986	-9.766\\
-9.766	-8.545\\
-8.545	-4.883\\
-4.883	-3.662\\
-3.662	-3.662\\
-3.662	-7.324\\
-7.324	-10.986\\
-10.986	-9.766\\
-9.766	-8.545\\
-8.545	-10.986\\
-10.986	-9.766\\
-9.766	-8.545\\
-8.545	-3.662\\
-3.662	-6.104\\
-6.104	-6.104\\
-6.104	-6.104\\
-6.104	-8.545\\
-8.545	-10.986\\
-10.986	-7.324\\
-7.324	-6.104\\
-6.104	-7.324\\
-7.324	-7.324\\
-7.324	-6.104\\
-6.104	-8.545\\
-8.545	-15.869\\
-15.869	-14.648\\
-14.648	-17.09\\
-17.09	-17.09\\
-17.09	-17.09\\
-17.09	-13.428\\
-13.428	-10.986\\
-10.986	-9.766\\
-9.766	-10.986\\
-10.986	-12.207\\
-12.207	-14.648\\
-14.648	-14.648\\
-14.648	-17.09\\
-17.09	-19.531\\
-19.531	-21.973\\
-21.973	-18.311\\
-18.311	-21.973\\
-21.973	-19.531\\
-19.531	-14.648\\
-14.648	-17.09\\
-17.09	-14.648\\
-14.648	-10.986\\
-10.986	-8.545\\
-8.545	-8.545\\
-8.545	-10.986\\
-10.986	-7.324\\
-7.324	-7.324\\
-7.324	-7.324\\
-7.324	-7.324\\
-7.324	-3.662\\
-3.662	-7.324\\
-7.324	-7.324\\
-7.324	-6.104\\
-6.104	-8.545\\
-8.545	-12.207\\
-12.207	-8.545\\
-8.545	-12.207\\
-12.207	-18.311\\
-18.311	-17.09\\
-17.09	-19.531\\
-19.531	-17.09\\
-17.09	-14.648\\
-14.648	-14.648\\
-14.648	-7.324\\
-7.324	-9.766\\
-9.766	-8.545\\
-8.545	-6.104\\
-6.104	-7.324\\
-7.324	-8.545\\
-8.545	-2.441\\
-2.441	-4.883\\
-4.883	-8.545\\
-8.545	-9.766\\
-9.766	-12.207\\
-12.207	-12.207\\
-12.207	-10.986\\
-10.986	-12.207\\
-12.207	-14.648\\
-14.648	-10.986\\
-10.986	-9.766\\
-9.766	-7.324\\
-7.324	-10.986\\
-10.986	-10.986\\
-10.986	-7.324\\
-7.324	-10.986\\
-10.986	-13.428\\
-13.428	-10.986\\
-10.986	-8.545\\
-8.545	-8.545\\
-8.545	-8.545\\
-8.545	-6.104\\
-6.104	-7.324\\
-7.324	-6.104\\
-6.104	-9.766\\
-9.766	-8.545\\
-8.545	-10.986\\
-10.986	-10.986\\
-10.986	-8.545\\
-8.545	-6.104\\
-6.104	-6.104\\
-6.104	-10.986\\
-10.986	-14.648\\
-14.648	-14.648\\
-14.648	-12.207\\
-12.207	-10.986\\
-10.986	-9.766\\
-9.766	-8.545\\
-8.545	-8.545\\
-8.545	-9.766\\
-9.766	-12.207\\
-12.207	-7.324\\
-7.324	-7.324\\
-7.324	-10.986\\
-10.986	-10.986\\
-10.986	-13.428\\
-13.428	-13.428\\
-13.428	-14.648\\
-14.648	-20.752\\
-20.752	-21.973\\
-21.973	-14.648\\
-14.648	-9.766\\
-9.766	-7.324\\
-7.324	-6.104\\
-6.104	-7.324\\
-7.324	-4.883\\
-4.883	-7.324\\
-7.324	-7.324\\
-7.324	-7.324\\
-7.324	-10.986\\
-10.986	-12.207\\
-12.207	-13.428\\
-13.428	-13.428\\
-13.428	-18.311\\
-18.311	-17.09\\
-17.09	-12.207\\
-12.207	-12.207\\
-12.207	-9.766\\
-9.766	-9.766\\
-9.766	-13.428\\
-13.428	-9.766\\
-9.766	-8.545\\
-8.545	-10.986\\
-10.986	-6.104\\
-6.104	-10.986\\
-10.986	-14.648\\
-14.648	-8.545\\
-8.545	-8.545\\
-8.545	-15.869\\
-15.869	-13.428\\
-13.428	-10.986\\
-10.986	-17.09\\
-17.09	-17.09\\
-17.09	-10.986\\
-10.986	-8.545\\
-8.545	-9.766\\
-9.766	-12.207\\
-12.207	-18.311\\
-18.311	-19.531\\
-19.531	-18.311\\
-18.311	-15.869\\
-15.869	-9.766\\
-9.766	-9.766\\
-9.766	-7.324\\
-7.324	-6.104\\
-6.104	-6.104\\
-6.104	-7.324\\
-7.324	-9.766\\
-9.766	-8.545\\
-8.545	-6.104\\
-6.104	-12.207\\
-12.207	-10.986\\
-10.986	-13.428\\
-13.428	-17.09\\
-17.09	-19.531\\
-19.531	-15.869\\
-15.869	-12.207\\
-12.207	-15.869\\
-15.869	-12.207\\
-12.207	-9.766\\
-9.766	-10.986\\
-10.986	-15.869\\
-15.869	-18.311\\
-18.311	-12.207\\
-12.207	-8.545\\
-8.545	-4.883\\
-4.883	-2.441\\
-2.441	0\\
0	-6.104\\
-6.104	-8.545\\
-8.545	-8.545\\
-8.545	-7.324\\
-7.324	-6.104\\
-6.104	-6.104\\
-6.104	-7.324\\
-7.324	-4.883\\
-4.883	-6.104\\
-6.104	-8.545\\
-8.545	-10.986\\
-10.986	-13.428\\
-13.428	-10.986\\
-10.986	-12.207\\
-12.207	-17.09\\
-17.09	-17.09\\
-17.09	-19.531\\
-19.531	-20.752\\
-20.752	-13.428\\
-13.428	-10.986\\
-10.986	-10.986\\
-10.986	-14.648\\
-14.648	-17.09\\
-17.09	-13.428\\
-13.428	-9.766\\
-9.766	-7.324\\
-7.324	-3.662\\
-3.662	-4.883\\
-4.883	-6.104\\
-6.104	-1.221\\
-1.221	-9.766\\
-9.766	-7.324\\
-7.324	-6.104\\
-6.104	-4.883\\
-4.883	-3.662\\
-3.662	-6.104\\
-6.104	-6.104\\
-6.104	-4.883\\
-4.883	-3.662\\
-3.662	-3.662\\
-3.662	-4.883\\
-4.883	-12.207\\
-12.207	-13.428\\
-13.428	-15.869\\
-15.869	-14.648\\
-14.648	-14.648\\
-14.648	-23.193\\
-23.193	-20.752\\
-20.752	-18.311\\
-18.311	-18.311\\
-18.311	-15.869\\
-15.869	-15.869\\
-15.869	-14.648\\
-14.648	-10.986\\
-10.986	-8.545\\
-8.545	-8.545\\
-8.545	-12.207\\
-12.207	-14.648\\
-14.648	-8.545\\
-8.545	-9.766\\
-9.766	-9.766\\
-9.766	-9.766\\
-9.766	-10.986\\
-10.986	-10.986\\
-10.986	-12.207\\
-12.207	-10.986\\
-10.986	-9.766\\
-9.766	-9.766\\
-9.766	-10.986\\
-10.986	-2.441\\
-2.441	-4.883\\
-4.883	-8.545\\
-8.545	-7.324\\
-7.324	1.221\\
1.221	-2.441\\
-2.441	-7.324\\
-7.324	-7.324\\
-7.324	-8.545\\
-8.545	-8.545\\
-8.545	-8.545\\
-8.545	-10.986\\
-10.986	-7.324\\
-7.324	-8.545\\
-8.545	-10.986\\
-10.986	-4.883\\
-4.883	-4.883\\
-4.883	-3.662\\
-3.662	-3.662\\
-3.662	-2.441\\
-2.441	-3.662\\
-3.662	-10.986\\
-10.986	-12.207\\
-12.207	-8.545\\
-8.545	-8.545\\
-8.545	-7.324\\
-7.324	-6.104\\
-6.104	-8.545\\
-8.545	-6.104\\
-6.104	-6.104\\
-6.104	-6.104\\
-6.104	-3.662\\
-3.662	-4.883\\
-4.883	-7.324\\
-7.324	-7.324\\
-7.324	-3.662\\
-3.662	-4.883\\
-4.883	-6.104\\
-6.104	-6.104\\
-6.104	-9.766\\
-9.766	-13.428\\
-13.428	-6.104\\
-6.104	-7.324\\
-7.324	-9.766\\
-9.766	-8.545\\
-8.545	-12.207\\
-12.207	-7.324\\
-7.324	-7.324\\
-7.324	-10.986\\
-10.986	-7.324\\
-7.324	-9.766\\
-9.766	-6.104\\
-6.104	-7.324\\
-7.324	-8.545\\
-8.545	-8.545\\
-8.545	-7.324\\
-7.324	-8.545\\
-8.545	-8.545\\
-8.545	-10.986\\
-10.986	-10.986\\
-10.986	-18.311\\
-18.311	-13.428\\
-13.428	-10.986\\
-10.986	-9.766\\
-9.766	-9.766\\
-9.766	-13.428\\
-13.428	-9.766\\
-9.766	-10.986\\
-10.986	-14.648\\
-14.648	-4.883\\
-4.883	-3.662\\
-3.662	-10.986\\
-10.986	-7.324\\
-7.324	-9.766\\
-9.766	-10.986\\
-10.986	-12.207\\
-12.207	-10.986\\
-10.986	-9.766\\
-9.766	-8.545\\
-8.545	-12.207\\
-12.207	-8.545\\
-8.545	-10.986\\
-10.986	-10.986\\
-10.986	-7.324\\
-7.324	-8.545\\
-8.545	-4.883\\
-4.883	-6.104\\
-6.104	-10.986\\
-10.986	-13.428\\
-13.428	-10.986\\
-10.986	-13.428\\
-13.428	-9.766\\
-9.766	-7.324\\
-7.324	-7.324\\
-7.324	-7.324\\
-7.324	-9.766\\
-9.766	-12.207\\
-12.207	-13.428\\
-13.428	-12.207\\
-12.207	-7.324\\
-7.324	-8.545\\
-8.545	-19.531\\
-19.531	-12.207\\
-12.207	-13.428\\
-13.428	-15.869\\
-15.869	-10.986\\
-10.986	-9.766\\
-9.766	-9.766\\
-9.766	-9.766\\
-9.766	-8.545\\
-8.545	-13.428\\
-13.428	-10.986\\
-10.986	-9.766\\
-9.766	-10.986\\
-10.986	-10.986\\
-10.986	-8.545\\
-8.545	-9.766\\
-9.766	-4.883\\
-4.883	-9.766\\
-9.766	-12.207\\
-12.207	-9.766\\
-9.766	-7.324\\
-7.324	-3.662\\
-3.662	-2.441\\
-2.441	-3.662\\
-3.662	-9.766\\
-9.766	-13.428\\
-13.428	-10.986\\
-10.986	-10.986\\
-10.986	-13.428\\
-13.428	-12.207\\
-12.207	-7.324\\
-7.324	-6.104\\
-6.104	-3.662\\
-3.662	-8.545\\
-8.545	-4.883\\
-4.883	-8.545\\
-8.545	-7.324\\
-7.324	-7.324\\
-7.324	-6.104\\
-6.104	-4.883\\
-4.883	-9.766\\
-9.766	-10.986\\
-10.986	-6.104\\
-6.104	-10.986\\
-10.986	-8.545\\
-8.545	-9.766\\
-9.766	-14.648\\
-14.648	-8.545\\
-8.545	-12.207\\
-12.207	-12.207\\
-12.207	-3.662\\
-3.662	-12.207\\
-12.207	-13.428\\
-13.428	-9.766\\
-9.766	-15.869\\
-15.869	-13.428\\
-13.428	-13.428\\
-13.428	-10.986\\
-10.986	-15.869\\
-15.869	-13.428\\
-13.428	-12.207\\
-12.207	-10.986\\
-10.986	-8.545\\
-8.545	-9.766\\
-9.766	-8.545\\
-8.545	-6.104\\
-6.104	-8.545\\
-8.545	-7.324\\
-7.324	-10.986\\
-10.986	-15.869\\
-15.869	-13.428\\
-13.428	-14.648\\
-14.648	-14.648\\
-14.648	-10.986\\
-10.986	-7.324\\
-7.324	-7.324\\
-7.324	-4.883\\
-4.883	-6.104\\
-6.104	-9.766\\
-9.766	-12.207\\
-12.207	-12.207\\
-12.207	-12.207\\
-12.207	-8.545\\
-8.545	-9.766\\
-9.766	-17.09\\
-17.09	-14.648\\
-14.648	-13.428\\
-13.428	-17.09\\
-17.09	-23.193\\
-23.193	-7.324\\
-7.324	-7.324\\
-7.324	-8.545\\
-8.545	-8.545\\
-8.545	-12.207\\
-12.207	-17.09\\
-17.09	-14.648\\
-14.648	-10.986\\
-10.986	-9.766\\
-9.766	-6.104\\
-6.104	-6.104\\
-6.104	-8.545\\
-8.545	-6.104\\
-6.104	-3.662\\
-3.662	-6.104\\
-6.104	-3.662\\
-3.662	-4.883\\
-4.883	-12.207\\
-12.207	-7.324\\
-7.324	-9.766\\
-9.766	-7.324\\
-7.324	-9.766\\
-9.766	-9.766\\
-9.766	-10.986\\
-10.986	-13.428\\
-13.428	-10.986\\
-10.986	-13.428\\
-13.428	-12.207\\
-12.207	-13.428\\
-13.428	-9.766\\
-9.766	-14.648\\
-14.648	-13.428\\
-13.428	-7.324\\
-7.324	-9.766\\
-9.766	-10.986\\
-10.986	-8.545\\
-8.545	-13.428\\
-13.428	-13.428\\
-13.428	-14.648\\
-14.648	-13.428\\
-13.428	-20.752\\
-20.752	-20.752\\
-20.752	-10.986\\
-10.986	-12.207\\
-12.207	-10.986\\
-10.986	-15.869\\
-15.869	-7.324\\
-7.324	-10.986\\
-10.986	-15.869\\
-15.869	-4.883\\
-4.883	-7.324\\
-7.324	-9.766\\
-9.766	-10.986\\
-10.986	-6.104\\
-6.104	-6.104\\
-6.104	-4.883\\
-4.883	-4.883\\
-4.883	-6.104\\
-6.104	-7.324\\
-7.324	-7.324\\
-7.324	-9.766\\
-9.766	-13.428\\
-13.428	-10.986\\
-10.986	-14.648\\
-14.648	-13.428\\
-13.428	-12.207\\
-12.207	-8.545\\
-8.545	-10.986\\
-10.986	-9.766\\
-9.766	-15.869\\
-15.869	-13.428\\
-13.428	-13.428\\
-13.428	-12.207\\
-12.207	-7.324\\
-7.324	-12.207\\
-12.207	-14.648\\
-14.648	-10.986\\
-10.986	-13.428\\
-13.428	-9.766\\
-9.766	-4.883\\
-4.883	-6.104\\
-6.104	-6.104\\
-6.104	-9.766\\
-9.766	-6.104\\
-6.104	-8.545\\
-8.545	-8.545\\
-8.545	-14.648\\
-14.648	-20.752\\
-20.752	-12.207\\
-12.207	-14.648\\
-14.648	-14.648\\
-14.648	-10.986\\
-10.986	-9.766\\
-9.766	-10.986\\
-10.986	-10.986\\
-10.986	-12.207\\
-12.207	-14.648\\
-14.648	-18.311\\
-18.311	-19.531\\
-19.531	-14.648\\
-14.648	-19.531\\
-19.531	-15.869\\
-15.869	-14.648\\
-14.648	-12.207\\
-12.207	-10.986\\
-10.986	-13.428\\
-13.428	-18.311\\
-18.311	-7.324\\
-7.324	-8.545\\
-8.545	-12.207\\
-12.207	-9.766\\
-9.766	-8.545\\
-8.545	-3.662\\
-3.662	-7.324\\
-7.324	-13.428\\
-13.428	-20.752\\
-20.752	-21.973\\
-21.973	-19.531\\
-19.531	-14.648\\
-14.648	-17.09\\
-17.09	-21.973\\
-21.973	-17.09\\
-17.09	-14.648\\
-14.648	-18.311\\
-18.311	-13.428\\
-13.428	-13.428\\
-13.428	-15.869\\
-15.869	-12.207\\
-12.207	-7.324\\
-7.324	-10.986\\
-10.986	-10.986\\
-10.986	-6.104\\
-6.104	-9.766\\
-9.766	-7.324\\
-7.324	-13.428\\
-13.428	-9.766\\
-9.766	-8.545\\
-8.545	-9.766\\
-9.766	-12.207\\
-12.207	-12.207\\
-12.207	-10.986\\
-10.986	-7.324\\
-7.324	-13.428\\
-13.428	-14.648\\
-14.648	-9.766\\
-9.766	-13.428\\
-13.428	-12.207\\
-12.207	-9.766\\
-9.766	-9.766\\
-9.766	-6.104\\
-6.104	-4.883\\
-4.883	-6.104\\
-6.104	-9.766\\
-9.766	-4.883\\
-4.883	-6.104\\
-6.104	-4.883\\
-4.883	-12.207\\
-12.207	-10.986\\
-10.986	-12.207\\
-12.207	-14.648\\
-14.648	-13.428\\
-13.428	-14.648\\
-14.648	-10.986\\
-10.986	-4.883\\
-4.883	-8.545\\
-8.545	-7.324\\
-7.324	-7.324\\
-7.324	-10.986\\
-10.986	-10.986\\
-10.986	-15.869\\
-15.869	-15.869\\
-15.869	-9.766\\
-9.766	-15.869\\
-15.869	-17.09\\
-17.09	-6.104\\
-6.104	-7.324\\
-7.324	-9.766\\
-9.766	-10.986\\
-10.986	-14.648\\
-14.648	-4.883\\
-4.883	-10.986\\
-10.986	-9.766\\
-9.766	-10.986\\
-10.986	-13.428\\
-13.428	-18.311\\
-18.311	-15.869\\
-15.869	-14.648\\
-14.648	-13.428\\
-13.428	-8.545\\
-8.545	-9.766\\
-9.766	0\\
0	-7.324\\
-7.324	-10.986\\
-10.986	-7.324\\
-7.324	-8.545\\
-8.545	-17.09\\
-17.09	-15.869\\
-15.869	-18.311\\
-18.311	-18.311\\
-18.311	-10.986\\
-10.986	-10.986\\
-10.986	-8.545\\
-8.545	-8.545\\
-8.545	-15.869\\
-15.869	-17.09\\
-17.09	-14.648\\
-14.648	-9.766\\
-9.766	-10.986\\
-10.986	-15.869\\
-15.869	-13.428\\
-13.428	-7.324\\
-7.324	-7.324\\
-7.324	-7.324\\
-7.324	-7.324\\
-7.324	-4.883\\
-4.883	-6.104\\
-6.104	-10.986\\
-10.986	-9.766\\
-9.766	-6.104\\
-6.104	-7.324\\
-7.324	-6.104\\
-6.104	-1.221\\
-1.221	-1.221\\
-1.221	-7.324\\
-7.324	-7.324\\
-7.324	-7.324\\
-7.324	-10.986\\
-10.986	-10.986\\
-10.986	-13.428\\
-13.428	-18.311\\
-18.311	-20.752\\
-20.752	-18.311\\
-18.311	-17.09\\
-17.09	-12.207\\
-12.207	-10.986\\
-10.986	-12.207\\
-12.207	-12.207\\
-12.207	-12.207\\
-12.207	-8.545\\
-8.545	-7.324\\
-7.324	-6.104\\
-6.104	-9.766\\
-9.766	-10.986\\
-10.986	-7.324\\
-7.324	-3.662\\
-3.662	-6.104\\
-6.104	-7.324\\
-7.324	-4.883\\
-4.883	-8.545\\
-8.545	-8.545\\
-8.545	-8.545\\
-8.545	-14.648\\
-14.648	-10.986\\
-10.986	-14.648\\
-14.648	-19.531\\
-19.531	-21.973\\
-21.973	-14.648\\
-14.648	-10.986\\
-10.986	-8.545\\
-8.545	-7.324\\
-7.324	-9.766\\
-9.766	-8.545\\
-8.545	-8.545\\
-8.545	-9.766\\
-9.766	-7.324\\
-7.324	-9.766\\
-9.766	-8.545\\
-8.545	-7.324\\
-7.324	-9.766\\
-9.766	-4.883\\
-4.883	-2.441\\
-2.441	-1.221\\
-1.221	-3.662\\
-3.662	-6.104\\
-6.104	-2.441\\
-2.441	-2.441\\
-2.441	-7.324\\
-7.324	-6.104\\
-6.104	-6.104\\
-6.104	-9.766\\
-9.766	-10.986\\
-10.986	-8.545\\
-8.545	-7.324\\
-7.324	-13.428\\
-13.428	-19.531\\
-19.531	-13.428\\
-13.428	-10.986\\
-10.986	-9.766\\
-9.766	-3.662\\
-3.662	-7.324\\
-7.324	-8.545\\
-8.545	-9.766\\
-9.766	-13.428\\
-13.428	-13.428\\
-13.428	-10.986\\
-10.986	-9.766\\
-9.766	-9.766\\
-9.766	-8.545\\
-8.545	-7.324\\
-7.324	-4.883\\
-4.883	-4.883\\
-4.883	-4.883\\
-4.883	-3.662\\
-3.662	-6.104\\
-6.104	-7.324\\
-7.324	-10.986\\
-10.986	-9.766\\
-9.766	-4.883\\
-4.883	-9.766\\
-9.766	-12.207\\
-12.207	-6.104\\
-6.104	-13.428\\
-13.428	-12.207\\
-12.207	-7.324\\
-7.324	-8.545\\
-8.545	-12.207\\
-12.207	-9.766\\
-9.766	-6.104\\
-6.104	-9.766\\
-9.766	-4.883\\
-4.883	-3.662\\
-3.662	-2.441\\
-2.441	-4.883\\
-4.883	-7.324\\
-7.324	-7.324\\
-7.324	-8.545\\
-8.545	-8.545\\
-8.545	-4.883\\
-4.883	-3.662\\
-3.662	-2.441\\
-2.441	-4.883\\
-4.883	-4.883\\
-4.883	-4.883\\
-4.883	-8.545\\
-8.545	-10.986\\
-10.986	-6.104\\
-6.104	-8.545\\
-8.545	-10.986\\
-10.986	-13.428\\
-13.428	-13.428\\
-13.428	-13.428\\
-13.428	-10.986\\
-10.986	-12.207\\
-12.207	-13.428\\
-13.428	-14.648\\
-14.648	-13.428\\
-13.428	-18.311\\
-18.311	-15.869\\
-15.869	-15.869\\
-15.869	-12.207\\
-12.207	-13.428\\
-13.428	-13.428\\
-13.428	-13.428\\
-13.428	-9.766\\
-9.766	-7.324\\
-7.324	-10.986\\
-10.986	-13.428\\
-13.428	-12.207\\
-12.207	-10.986\\
-10.986	-8.545\\
-8.545	-12.207\\
-12.207	-8.545\\
-8.545	-14.648\\
-14.648	-15.869\\
-15.869	-10.986\\
-10.986	-6.104\\
-6.104	-9.766\\
-9.766	-2.441\\
-2.441	-3.662\\
-3.662	-7.324\\
-7.324	-4.883\\
-4.883	-3.662\\
-3.662	-8.545\\
-8.545	-4.883\\
-4.883	-7.324\\
-7.324	-6.104\\
-6.104	-4.883\\
-4.883	-4.883\\
-4.883	-6.104\\
-6.104	-6.104\\
-6.104	-7.324\\
-7.324	-6.104\\
-6.104	-7.324\\
-7.324	-7.324\\
-7.324	-7.324\\
-7.324	-14.648\\
-14.648	-9.766\\
-9.766	-8.545\\
-8.545	-10.986\\
-10.986	-7.324\\
-7.324	-6.104\\
-6.104	-10.986\\
-10.986	-8.545\\
-8.545	-4.883\\
-4.883	-10.986\\
-10.986	-8.545\\
-8.545	-8.545\\
-8.545	-7.324\\
-7.324	-2.441\\
-2.441	-3.662\\
-3.662	-4.883\\
-4.883	-4.883\\
-4.883	-7.324\\
-7.324	-8.545\\
-8.545	-8.545\\
-8.545	-9.766\\
-9.766	-7.324\\
-7.324	-10.986\\
-10.986	-14.648\\
-14.648	-12.207\\
-12.207	-10.986\\
-10.986	-14.648\\
-14.648	-12.207\\
-12.207	-17.09\\
-17.09	-15.869\\
-15.869	-19.531\\
-19.531	-20.752\\
-20.752	-13.428\\
-13.428	-6.104\\
-6.104	-6.104\\
-6.104	-8.545\\
-8.545	-12.207\\
-12.207	-12.207\\
-12.207	-8.545\\
-8.545	-7.324\\
-7.324	-10.986\\
-10.986	-13.428\\
-13.428	-14.648\\
-14.648	-13.428\\
-13.428	-9.766\\
-9.766	-6.104\\
-6.104	-10.986\\
-10.986	-12.207\\
-12.207	-14.648\\
-14.648	-14.648\\
-14.648	-12.207\\
-12.207	-19.531\\
-19.531	-19.531\\
-19.531	-17.09\\
-17.09	-20.752\\
-20.752	-18.311\\
-18.311	-15.869\\
-15.869	-17.09\\
-17.09	-15.869\\
-15.869	-10.986\\
-10.986	-8.545\\
-8.545	-9.766\\
-9.766	-12.207\\
-12.207	-12.207\\
-12.207	-8.545\\
-8.545	-13.428\\
-13.428	-12.207\\
-12.207	-8.545\\
-8.545	-9.766\\
-9.766	-12.207\\
-12.207	-13.428\\
-13.428	-14.648\\
-14.648	-10.986\\
-10.986	-9.766\\
-9.766	-8.545\\
-8.545	-12.207\\
-12.207	-8.545\\
-8.545	-7.324\\
-7.324	-2.441\\
-2.441	-7.324\\
-7.324	-9.766\\
-9.766	-12.207\\
-12.207	-12.207\\
-12.207	-8.545\\
-8.545	-6.104\\
-6.104	-6.104\\
-6.104	-7.324\\
-7.324	-8.545\\
-8.545	-9.766\\
-9.766	-6.104\\
-6.104	-13.428\\
-13.428	-13.428\\
-13.428	-14.648\\
-14.648	-15.869\\
-15.869	-17.09\\
-17.09	-12.207\\
-12.207	-12.207\\
-12.207	-7.324\\
-7.324	-8.545\\
-8.545	-12.207\\
-12.207	-10.986\\
-10.986	-6.104\\
-6.104	-6.104\\
-6.104	-8.545\\
-8.545	-12.207\\
-12.207	-10.986\\
-10.986	-18.311\\
-18.311	-18.311\\
-18.311	-12.207\\
-12.207	-8.545\\
-8.545	-7.324\\
-7.324	-8.545\\
-8.545	-13.428\\
-13.428	-12.207\\
-12.207	-13.428\\
-13.428	-14.648\\
-14.648	-12.207\\
-12.207	-10.986\\
-10.986	-13.428\\
-13.428	-9.766\\
-9.766	-12.207\\
-12.207	-7.324\\
-7.324	-7.324\\
-7.324	-10.986\\
-10.986	-13.428\\
-13.428	-14.648\\
-14.648	-8.545\\
-8.545	-14.648\\
-14.648	-13.428\\
-13.428	-12.207\\
-12.207	-7.324\\
-7.324	-17.09\\
-17.09	-8.545\\
-8.545	-10.986\\
-10.986	-12.207\\
-12.207	-8.545\\
-8.545	-6.104\\
-6.104	-17.09\\
-17.09	-20.752\\
-20.752	-12.207\\
-12.207	-15.869\\
-15.869	-15.869\\
-15.869	-13.428\\
-13.428	-12.207\\
-12.207	-10.986\\
-10.986	-8.545\\
-8.545	-7.324\\
-7.324	-10.986\\
-10.986	-17.09\\
-17.09	-17.09\\
-17.09	-20.752\\
-20.752	-15.869\\
-15.869	-14.648\\
-14.648	-10.986\\
-10.986	-13.428\\
-13.428	-9.766\\
-9.766	-8.545\\
-8.545	-8.545\\
-8.545	-7.324\\
-7.324	-7.324\\
-7.324	-3.662\\
-3.662	-8.545\\
-8.545	-10.986\\
-10.986	-6.104\\
-6.104	-6.104\\
-6.104	-6.104\\
-6.104	-6.104\\
-6.104	-14.648\\
-14.648	-6.104\\
-6.104	-3.662\\
-3.662	-7.324\\
-7.324	-8.545\\
-8.545	-9.766\\
-9.766	-8.545\\
-8.545	-7.324\\
-7.324	-6.104\\
-6.104	-4.883\\
-4.883	-4.883\\
-4.883	-10.986\\
-10.986	-10.986\\
-10.986	-8.545\\
-8.545	-6.104\\
-6.104	-4.883\\
-4.883	-8.545\\
-8.545	-10.986\\
-10.986	-10.986\\
-10.986	-9.766\\
-9.766	-6.104\\
-6.104	-7.324\\
-7.324	-7.324\\
-7.324	-10.986\\
-10.986	-14.648\\
-14.648	-15.869\\
-15.869	-13.428\\
-13.428	-7.324\\
-7.324	-12.207\\
-12.207	-9.766\\
-9.766	-9.766\\
-9.766	-6.104\\
-6.104	-9.766\\
-9.766	-8.545\\
-8.545	-8.545\\
-8.545	-6.104\\
-6.104	-4.883\\
-4.883	-7.324\\
-7.324	-9.766\\
-9.766	-13.428\\
-13.428	-13.428\\
-13.428	-12.207\\
-12.207	-13.428\\
-13.428	-13.428\\
-13.428	-17.09\\
-17.09	-21.973\\
-21.973	-23.193\\
-23.193	-13.428\\
-13.428	-17.09\\
-17.09	-15.869\\
-15.869	-20.752\\
-20.752	-14.648\\
-14.648	-7.324\\
-7.324	-6.104\\
-6.104	-7.324\\
-7.324	-6.104\\
-6.104	-3.662\\
-3.662	-2.441\\
-2.441	-4.883\\
-4.883	-8.545\\
-8.545	-9.766\\
-9.766	-4.883\\
-4.883	-7.324\\
-7.324	-6.104\\
-6.104	-8.545\\
-8.545	-8.545\\
-8.545	-7.324\\
-7.324	-4.883\\
-4.883	-3.662\\
-3.662	-3.662\\
-3.662	-6.104\\
-6.104	-7.324\\
-7.324	-3.662\\
-3.662	-3.662\\
-3.662	-9.766\\
-9.766	-7.324\\
-7.324	-9.766\\
-9.766	-7.324\\
-7.324	-8.545\\
-8.545	-12.207\\
-12.207	-9.766\\
-9.766	-12.207\\
-12.207	-10.986\\
-10.986	-7.324\\
-7.324	-6.104\\
-6.104	-13.428\\
-13.428	-13.428\\
-13.428	-13.428\\
-13.428	-14.648\\
-14.648	-10.986\\
-10.986	-10.986\\
-10.986	-10.986\\
-10.986	-12.207\\
-12.207	-8.545\\
-8.545	-17.09\\
-17.09	-13.428\\
-13.428	-14.648\\
-14.648	-23.193\\
-23.193	-15.869\\
-15.869	-8.545\\
-8.545	-7.324\\
-7.324	-6.104\\
-6.104	-9.766\\
-9.766	-13.428\\
-13.428	-15.869\\
-15.869	-14.648\\
-14.648	-15.869\\
-15.869	-9.766\\
-9.766	-9.766\\
-9.766	-10.986\\
-10.986	-12.207\\
-12.207	-6.104\\
-6.104	-4.883\\
-4.883	-7.324\\
-7.324	-8.545\\
-8.545	-10.986\\
-10.986	-13.428\\
-13.428	-10.986\\
-10.986	-8.545\\
-8.545	-7.324\\
-7.324	-10.986\\
-10.986	-14.648\\
-14.648	-15.869\\
-15.869	-17.09\\
-17.09	-20.752\\
-20.752	-14.648\\
-14.648	-18.311\\
-18.311	-12.207\\
-12.207	-8.545\\
-8.545	-14.648\\
-14.648	-15.869\\
-15.869	-10.986\\
-10.986	-10.986\\
-10.986	-15.869\\
-15.869	-19.531\\
-19.531	-17.09\\
-17.09	-7.324\\
-7.324	-6.104\\
-6.104	-12.207\\
-12.207	-6.104\\
-6.104	-8.545\\
-8.545	-6.104\\
-6.104	-9.766\\
-9.766	-3.662\\
-3.662	-4.883\\
-4.883	-7.324\\
-7.324	-9.766\\
-9.766	-10.986\\
-10.986	-12.207\\
-12.207	-13.428\\
-13.428	-15.869\\
-15.869	-13.428\\
-13.428	-8.545\\
-8.545	-7.324\\
-7.324	-10.986\\
-10.986	-8.545\\
-8.545	-7.324\\
-7.324	-4.883\\
-4.883	-4.883\\
-4.883	-6.104\\
-6.104	-4.883\\
-4.883	-3.662\\
-3.662	-4.883\\
-4.883	-4.883\\
-4.883	-8.545\\
-8.545	-7.324\\
-7.324	-6.104\\
-6.104	-4.883\\
-4.883	-9.766\\
-9.766	-9.766\\
-9.766	-7.324\\
-7.324	-9.766\\
-9.766	-8.545\\
-8.545	-10.986\\
-10.986	-9.766\\
-9.766	-10.986\\
-10.986	-9.766\\
-9.766	-7.324\\
-7.324	-7.324\\
-7.324	-7.324\\
-7.324	-8.545\\
-8.545	-8.545\\
-8.545	-9.766\\
-9.766	-14.648\\
-14.648	-9.766\\
-9.766	-10.986\\
-10.986	-15.869\\
-15.869	-10.986\\
-10.986	-9.766\\
-9.766	-8.545\\
-8.545	-7.324\\
-7.324	-4.883\\
-4.883	-3.662\\
-3.662	-6.104\\
-6.104	-10.986\\
-10.986	-7.324\\
-7.324	-9.766\\
-9.766	-7.324\\
-7.324	-8.545\\
-8.545	-2.441\\
-2.441	-4.883\\
-4.883	-2.441\\
-2.441	-2.441\\
-2.441	-8.545\\
-8.545	-10.986\\
-10.986	-7.324\\
-7.324	-7.324\\
-7.324	-9.766\\
-9.766	-9.766\\
-9.766	-9.766\\
-9.766	-12.207\\
-12.207	-12.207\\
-12.207	-13.428\\
-13.428	-12.207\\
-12.207	-8.545\\
-8.545	-7.324\\
-7.324	-7.324\\
-7.324	-9.766\\
-9.766	-8.545\\
-8.545	-8.545\\
-8.545	-8.545\\
-8.545	-12.207\\
-12.207	-10.986\\
-10.986	-12.207\\
-12.207	-15.869\\
-15.869	-8.545\\
-8.545	-8.545\\
-8.545	-13.428\\
-13.428	-13.428\\
-13.428	-15.869\\
-15.869	-14.648\\
-14.648	-21.973\\
-21.973	-25.635\\
-25.635	-15.869\\
-15.869	-18.311\\
-18.311	-21.973\\
-21.973	-20.752\\
-20.752	-24.414\\
-24.414	-23.193\\
-23.193	-25.635\\
-25.635	-20.752\\
-20.752	-14.648\\
-14.648	-10.986\\
-10.986	-13.428\\
-13.428	-9.766\\
-9.766	-8.545\\
-8.545	-9.766\\
-9.766	-13.428\\
-13.428	-10.986\\
-10.986	-8.545\\
-8.545	-8.545\\
-8.545	-8.545\\
-8.545	-8.545\\
-8.545	-6.104\\
-6.104	-10.986\\
-10.986	-7.324\\
-7.324	-7.324\\
-7.324	-6.104\\
-6.104	-7.324\\
-7.324	-3.662\\
-3.662	-3.662\\
-3.662	-3.662\\
-3.662	-3.662\\
-3.662	-6.104\\
-6.104	-6.104\\
-6.104	-9.766\\
-9.766	-10.986\\
-10.986	-8.545\\
-8.545	-8.545\\
-8.545	-12.207\\
-12.207	-9.766\\
-9.766	-4.883\\
-4.883	-7.324\\
-7.324	-4.883\\
-4.883	-6.104\\
-6.104	-8.545\\
-8.545	-8.545\\
-8.545	-4.883\\
-4.883	-9.766\\
-9.766	-14.648\\
-14.648	-15.869\\
-15.869	-10.986\\
-10.986	-7.324\\
-7.324	-8.545\\
-8.545	-12.207\\
-12.207	-13.428\\
-13.428	-7.324\\
-7.324	-6.104\\
-6.104	-6.104\\
-6.104	-8.545\\
-8.545	-2.441\\
-2.441	-2.441\\
-2.441	-7.324\\
-7.324	-10.986\\
-10.986	-10.986\\
-10.986	-7.324\\
-7.324	-6.104\\
-6.104	-6.104\\
-6.104	-3.662\\
-3.662	-7.324\\
-7.324	-1.221\\
-1.221	-3.662\\
-3.662	-7.324\\
-7.324	-10.986\\
-10.986	-10.986\\
-10.986	-10.986\\
-10.986	-12.207\\
-12.207	-10.986\\
-10.986	-10.986\\
-10.986	-8.545\\
-8.545	-10.986\\
-10.986	-17.09\\
-17.09	-18.311\\
-18.311	-8.545\\
-8.545	-4.883\\
-4.883	-4.883\\
-4.883	-14.648\\
-14.648	-10.986\\
-10.986	-8.545\\
-8.545	-9.766\\
-9.766	-10.986\\
-10.986	-17.09\\
-17.09	-18.311\\
-18.311	-18.311\\
-18.311	-19.531\\
-19.531	-14.648\\
-14.648	-13.428\\
-13.428	-9.766\\
-9.766	-9.766\\
-9.766	-8.545\\
-8.545	-10.986\\
-10.986	-9.766\\
-9.766	-12.207\\
-12.207	-8.545\\
-8.545	-9.766\\
-9.766	-8.545\\
-8.545	-6.104\\
-6.104	-4.883\\
-4.883	-4.883\\
-4.883	-4.883\\
-4.883	-4.883\\
-4.883	-2.441\\
-2.441	-4.883\\
-4.883	-7.324\\
-7.324	-6.104\\
-6.104	-6.104\\
-6.104	-8.545\\
-8.545	-12.207\\
-12.207	-8.545\\
-8.545	-4.883\\
-4.883	-1.221\\
-1.221	-7.324\\
-7.324	-13.428\\
-13.428	-14.648\\
-14.648	-20.752\\
-20.752	-24.414\\
-24.414	-24.414\\
-24.414	-15.869\\
-15.869	-17.09\\
-17.09	-21.973\\
-21.973	-18.311\\
-18.311	-15.869\\
-15.869	-18.311\\
-18.311	-20.752\\
-20.752	-19.531\\
-19.531	-26.855\\
-26.855	-20.752\\
-20.752	-12.207\\
-12.207	-7.324\\
-7.324	-6.104\\
-6.104	-6.104\\
-6.104	-6.104\\
-6.104	-6.104\\
-6.104	-10.986\\
-10.986	-12.207\\
-12.207	-8.545\\
-8.545	-10.986\\
-10.986	-6.104\\
-6.104	-3.662\\
-3.662	-9.766\\
-9.766	-12.207\\
-12.207	-9.766\\
-9.766	-6.104\\
-6.104	-6.104\\
-6.104	-7.324\\
-7.324	-10.986\\
-10.986	-17.09\\
-17.09	-15.869\\
-15.869	-17.09\\
-17.09	-20.752\\
-20.752	-25.635\\
-25.635	-21.973\\
-21.973	-17.09\\
-17.09	-10.986\\
-10.986	-8.545\\
-8.545	-4.883\\
-4.883	-7.324\\
-7.324	-9.766\\
-9.766	-8.545\\
-8.545	-6.104\\
-6.104	-7.324\\
-7.324	-8.545\\
-8.545	-8.545\\
-8.545	-10.986\\
-10.986	-9.766\\
-9.766	-4.883\\
-4.883	-4.883\\
-4.883	-2.441\\
-2.441	-2.441\\
-2.441	-3.662\\
-3.662	-7.324\\
-7.324	-8.545\\
-8.545	-7.324\\
-7.324	-10.986\\
-10.986	-8.545\\
-8.545	-8.545\\
-8.545	-19.531\\
-19.531	-14.648\\
-14.648	-12.207\\
-12.207	-17.09\\
-17.09	-18.311\\
-18.311	-13.428\\
-13.428	-12.207\\
-12.207	-10.986\\
-10.986	-9.766\\
-9.766	-14.648\\
-14.648	-24.414\\
-24.414	-25.635\\
-25.635	-20.752\\
-20.752	-19.531\\
-19.531	-14.648\\
-14.648	-17.09\\
-17.09	-19.531\\
-19.531	-14.648\\
-14.648	-10.986\\
-10.986	-9.766\\
-9.766	-12.207\\
-12.207	-13.428\\
-13.428	-8.545\\
-8.545	-7.324\\
-7.324	-7.324\\
-7.324	-8.545\\
-8.545	-14.648\\
-14.648	-17.09\\
-17.09	-12.207\\
-12.207	-10.986\\
-10.986	-9.766\\
-9.766	-9.766\\
-9.766	-6.104\\
-6.104	-4.883\\
-4.883	-6.104\\
-6.104	-6.104\\
-6.104	-4.883\\
-4.883	-7.324\\
-7.324	-9.766\\
-9.766	-10.986\\
-10.986	-8.545\\
-8.545	-3.662\\
-3.662	-3.662\\
-3.662	-6.104\\
-6.104	-8.545\\
-8.545	-9.766\\
-9.766	-8.545\\
-8.545	-4.883\\
-4.883	-10.986\\
-10.986	-14.648\\
-14.648	-10.986\\
-10.986	-7.324\\
-7.324	-9.766\\
-9.766	-8.545\\
-8.545	-8.545\\
-8.545	-6.104\\
-6.104	-6.104\\
-6.104	-4.883\\
-4.883	-10.986\\
-10.986	-9.766\\
-9.766	-3.662\\
-3.662	-6.104\\
-6.104	-9.766\\
-9.766	-12.207\\
-12.207	-7.324\\
-7.324	-6.104\\
-6.104	-10.986\\
-10.986	-13.428\\
-13.428	-12.207\\
-12.207	-17.09\\
-17.09	-19.531\\
-19.531	-12.207\\
-12.207	-12.207\\
-12.207	-17.09\\
-17.09	-13.428\\
-13.428	-10.986\\
-10.986	-18.311\\
-18.311	-14.648\\
-14.648	-7.324\\
-7.324	-8.545\\
-8.545	-19.531\\
-19.531	-21.973\\
-21.973	-17.09\\
-17.09	-12.207\\
-12.207	-17.09\\
-17.09	-14.648\\
-14.648	-9.766\\
-9.766	-8.545\\
-8.545	-12.207\\
-12.207	-12.207\\
-12.207	-12.207\\
-12.207	-12.207\\
-12.207	-9.766\\
-9.766	-9.766\\
-9.766	-12.207\\
-12.207	-10.986\\
-10.986	-6.104\\
-6.104	-6.104\\
-6.104	-4.883\\
-4.883	-6.104\\
-6.104	-10.986\\
-10.986	-8.545\\
-8.545	-13.428\\
-13.428	-13.428\\
-13.428	-15.869\\
-15.869	-13.428\\
-13.428	-9.766\\
-9.766	-9.766\\
-9.766	-6.104\\
-6.104	-14.648\\
-14.648	-10.986\\
-10.986	-7.324\\
-7.324	-13.428\\
-13.428	-10.986\\
-10.986	-13.428\\
-13.428	-15.869\\
-15.869	-20.752\\
-20.752	-17.09\\
-17.09	-14.648\\
-14.648	-15.869\\
-15.869	-14.648\\
-14.648	-7.324\\
-7.324	-12.207\\
-12.207	-13.428\\
-13.428	-13.428\\
-13.428	-15.869\\
-15.869	-12.207\\
-12.207	-8.545\\
-8.545	-8.545\\
-8.545	-8.545\\
-8.545	-8.545\\
-8.545	-6.104\\
-6.104	-10.986\\
-10.986	-13.428\\
-13.428	-7.324\\
-7.324	-7.324\\
-7.324	-10.986\\
-10.986	-9.766\\
-9.766	-3.662\\
-3.662	-6.104\\
-6.104	-9.766\\
-9.766	-6.104\\
-6.104	-4.883\\
-4.883	-4.883\\
-4.883	-2.441\\
-2.441	-8.545\\
-8.545	-8.545\\
-8.545	-9.766\\
-9.766	-13.428\\
-13.428	-17.09\\
-17.09	-13.428\\
-13.428	-6.104\\
-6.104	-12.207\\
-12.207	-20.752\\
-20.752	-15.869\\
-15.869	-14.648\\
-14.648	-19.531\\
-19.531	-17.09\\
-17.09	-13.428\\
-13.428	-13.428\\
-13.428	-12.207\\
-12.207	-15.869\\
-15.869	-14.648\\
-14.648	-10.986\\
-10.986	-14.648\\
-14.648	-20.752\\
-20.752	-18.311\\
-18.311	-9.766\\
-9.766	-7.324\\
-7.324	-6.104\\
-6.104	-14.648\\
-14.648	-2.441\\
-2.441	-3.662\\
-3.662	-7.324\\
-7.324	-10.986\\
-10.986	-12.207\\
-12.207	-12.207\\
-12.207	-9.766\\
-9.766	-3.662\\
-3.662	-4.883\\
-4.883	-3.662\\
-3.662	-7.324\\
-7.324	-3.662\\
-3.662	-8.545\\
-8.545	-8.545\\
-8.545	-7.324\\
-7.324	-6.104\\
-6.104	-6.104\\
-6.104	-6.104\\
-6.104	-7.324\\
-7.324	-3.662\\
-3.662	0\\
0	-2.441\\
-2.441	-6.104\\
-6.104	-10.986\\
-10.986	-10.986\\
-10.986	-18.311\\
-18.311	-14.648\\
-14.648	-9.766\\
-9.766	-6.104\\
-6.104	-8.545\\
-8.545	-1.221\\
-1.221	-3.662\\
-3.662	-6.104\\
-6.104	-12.207\\
-12.207	-7.324\\
-7.324	-7.324\\
-7.324	-9.766\\
-9.766	-10.986\\
-10.986	-9.766\\
-9.766	-10.986\\
-10.986	-13.428\\
-13.428	-17.09\\
-17.09	-14.648\\
-14.648	-2.441\\
-2.441	-3.662\\
-3.662	-8.545\\
-8.545	-8.545\\
-8.545	-4.883\\
-4.883	-6.104\\
-6.104	-3.662\\
-3.662	-1.221\\
-1.221	-1.221\\
-1.221	-2.441\\
-2.441	-4.883\\
-4.883	-4.883\\
-4.883	-8.545\\
-8.545	-13.428\\
-13.428	-13.428\\
-13.428	-6.104\\
-6.104	-6.104\\
-6.104	-13.428\\
-13.428	-7.324\\
-7.324	-3.662\\
-3.662	-12.207\\
-12.207	-8.545\\
-8.545	-13.428\\
-13.428	-21.973\\
-21.973	-10.986\\
-10.986	-8.545\\
-8.545	-10.986\\
};
\addplot [color=mycolor2, line width=2.0pt, forget plot]
  table[row sep=crcr]{%
-10.986	-10.4803982980153\\
-15.869	-15.1386710896782\\
-10.986	-10.4803982980153\\
-14.648	-13.9738643973537\\
-17.09	-16.3034777820027\\
-10.986	-10.4803982980153\\
-7.324	-6.98693219867687\\
-14.648	-13.9738643973537\\
-8.545	-8.15173889100135\\
-9.766	-9.31654558332583\\
-6.104	-5.82307948398739\\
-3.662	-3.49346609933844\\
-2.441	-2.32865940701396\\
-13.428	-12.8100116826643\\
-9.766	-9.31654558332583\\
-7.324	-6.98693219867687\\
-10.986	-10.4803982980153\\
-12.207	-11.6452049903398\\
-6.104	-5.82307948398739\\
-9.766	-9.31654558332583\\
-13.428	-12.8100116826643\\
-7.324	-6.98693219867687\\
-15.869	-15.1386710896782\\
-13.428	-12.8100116826643\\
-10.986	-10.4803982980153\\
-17.09	-16.3034777820027\\
-13.428	-12.8100116826643\\
-10.986	-10.4803982980153\\
-8.545	-8.15173889100135\\
-10.986	-10.4803982980153\\
-12.207	-11.6452049903398\\
-10.986	-10.4803982980153\\
-9.766	-9.31654558332583\\
-10.986	-10.4803982980153\\
-15.869	-15.1386710896782\\
-14.648	-13.9738643973537\\
-10.986	-10.4803982980153\\
-6.104	-5.82307948398739\\
-4.883	-4.65827279166291\\
-8.545	-8.15173889100135\\
-7.324	-6.98693219867687\\
-10.986	-10.4803982980153\\
-9.766	-9.31654558332583\\
-15.869	-15.1386710896782\\
-8.545	-8.15173889100135\\
-6.104	-5.82307948398739\\
-7.324	-6.98693219867687\\
-8.545	-8.15173889100135\\
-9.766	-9.31654558332583\\
-4.883	-4.65827279166291\\
-7.324	-6.98693219867687\\
-6.104	-5.82307948398739\\
-7.324	-6.98693219867687\\
-8.545	-8.15173889100135\\
-9.766	-9.31654558332583\\
-10.986	-10.4803982980153\\
-12.207	-11.6452049903398\\
-13.428	-12.8100116826643\\
-10.986	-10.4803982980153\\
-12.207	-11.6452049903398\\
-9.766	-9.31654558332583\\
-17.09	-16.3034777820027\\
-20.752	-19.7969438813411\\
-19.531	-18.6321371890167\\
-15.869	-15.1386710896782\\
-20.752	-19.7969438813411\\
-23.193	-22.1256032883551\\
-14.648	-13.9738643973537\\
-23.193	-22.1256032883551\\
-26.855	-25.6190693876935\\
-21.973	-20.9617505736656\\
-15.869	-15.1386710896782\\
-14.648	-13.9738643973537\\
-10.986	-10.4803982980153\\
-8.545	-8.15173889100135\\
-10.986	-10.4803982980153\\
-8.545	-8.15173889100135\\
-6.104	-5.82307948398739\\
-8.545	-8.15173889100135\\
-9.766	-9.31654558332583\\
-10.986	-10.4803982980153\\
-9.766	-9.31654558332583\\
-12.207	-11.6452049903398\\
-10.986	-10.4803982980153\\
-15.869	-15.1386710896782\\
-13.428	-12.8100116826643\\
-18.311	-17.4682844743272\\
-12.207	-11.6452049903398\\
-10.986	-10.4803982980153\\
-12.207	-11.6452049903398\\
-9.766	-9.31654558332583\\
-8.545	-8.15173889100135\\
-10.986	-10.4803982980153\\
-8.545	-8.15173889100135\\
-7.324	-6.98693219867687\\
-4.883	-4.65827279166291\\
-6.104	-5.82307948398739\\
-8.545	-8.15173889100135\\
-13.428	-12.8100116826643\\
-12.207	-11.6452049903398\\
-13.428	-12.8100116826643\\
-8.545	-8.15173889100135\\
-15.869	-15.1386710896782\\
-17.09	-16.3034777820027\\
-13.428	-12.8100116826643\\
-14.648	-13.9738643973537\\
-10.986	-10.4803982980153\\
-7.324	-6.98693219867687\\
-8.545	-8.15173889100135\\
-17.09	-16.3034777820027\\
-20.752	-19.7969438813411\\
-19.531	-18.6321371890167\\
-14.648	-13.9738643973537\\
-15.869	-15.1386710896782\\
-13.428	-12.8100116826643\\
-12.207	-11.6452049903398\\
-9.766	-9.31654558332583\\
-8.545	-8.15173889100135\\
-9.766	-9.31654558332583\\
-8.545	-8.15173889100135\\
-13.428	-12.8100116826643\\
-15.869	-15.1386710896782\\
-10.986	-10.4803982980153\\
-8.545	-8.15173889100135\\
-9.766	-9.31654558332583\\
-8.545	-8.15173889100135\\
-6.104	-5.82307948398739\\
-3.662	-3.49346609933844\\
-4.883	-4.65827279166291\\
-8.545	-8.15173889100135\\
-9.766	-9.31654558332583\\
-4.883	-4.65827279166291\\
-7.324	-6.98693219867687\\
-3.662	-3.49346609933844\\
-2.441	-2.32865940701396\\
-4.883	-4.65827279166291\\
-10.986	-10.4803982980153\\
-12.207	-11.6452049903398\\
-14.648	-13.9738643973537\\
-13.428	-12.8100116826643\\
-8.545	-8.15173889100135\\
-7.324	-6.98693219867687\\
-4.883	-4.65827279166291\\
-7.324	-6.98693219867687\\
-6.104	-5.82307948398739\\
-10.986	-10.4803982980153\\
-12.207	-11.6452049903398\\
-21.973	-20.9617505736656\\
-23.193	-22.1256032883551\\
-21.973	-20.9617505736656\\
-18.311	-17.4682844743272\\
-13.428	-12.8100116826643\\
-17.09	-16.3034777820027\\
-15.869	-15.1386710896782\\
-19.531	-18.6321371890167\\
-12.207	-11.6452049903398\\
-13.428	-12.8100116826643\\
-12.207	-11.6452049903398\\
-13.428	-12.8100116826643\\
-10.986	-10.4803982980153\\
-18.311	-17.4682844743272\\
-17.09	-16.3034777820027\\
-13.428	-12.8100116826643\\
-7.324	-6.98693219867687\\
-12.207	-11.6452049903398\\
-15.869	-15.1386710896782\\
-17.09	-16.3034777820027\\
-19.531	-18.6321371890167\\
-21.973	-20.9617505736656\\
-20.752	-19.7969438813411\\
-15.869	-15.1386710896782\\
-14.648	-13.9738643973537\\
-18.311	-17.4682844743272\\
-19.531	-18.6321371890167\\
-13.428	-12.8100116826643\\
-9.766	-9.31654558332583\\
-13.428	-12.8100116826643\\
-9.766	-9.31654558332583\\
-7.324	-6.98693219867687\\
-9.766	-9.31654558332583\\
-10.986	-10.4803982980153\\
-7.324	-6.98693219867687\\
-9.766	-9.31654558332583\\
-6.104	-5.82307948398739\\
-10.986	-10.4803982980153\\
-13.428	-12.8100116826643\\
-8.545	-8.15173889100135\\
-9.766	-9.31654558332583\\
-12.207	-11.6452049903398\\
-18.311	-17.4682844743272\\
-15.869	-15.1386710896782\\
-9.766	-9.31654558332583\\
-8.545	-8.15173889100135\\
-9.766	-9.31654558332583\\
-7.324	-6.98693219867687\\
-8.545	-8.15173889100135\\
-7.324	-6.98693219867687\\
-8.545	-8.15173889100135\\
-10.986	-10.4803982980153\\
-6.104	-5.82307948398739\\
-10.986	-10.4803982980153\\
-13.428	-12.8100116826643\\
-9.766	-9.31654558332583\\
-12.207	-11.6452049903398\\
-13.428	-12.8100116826643\\
-10.986	-10.4803982980153\\
-4.883	-4.65827279166291\\
-10.986	-10.4803982980153\\
-8.545	-8.15173889100135\\
-6.104	-5.82307948398739\\
-10.986	-10.4803982980153\\
-6.104	-5.82307948398739\\
-8.545	-8.15173889100135\\
-4.883	-4.65827279166291\\
-8.545	-8.15173889100135\\
-4.883	-4.65827279166291\\
-7.324	-6.98693219867687\\
-4.883	-4.65827279166291\\
-10.986	-10.4803982980153\\
-12.207	-11.6452049903398\\
-15.869	-15.1386710896782\\
-10.986	-10.4803982980153\\
-6.104	-5.82307948398739\\
-7.324	-6.98693219867687\\
-4.883	-4.65827279166291\\
-7.324	-6.98693219867687\\
-6.104	-5.82307948398739\\
-3.662	-3.49346609933844\\
-8.545	-8.15173889100135\\
-9.766	-9.31654558332583\\
-17.09	-16.3034777820027\\
-10.986	-10.4803982980153\\
-13.428	-12.8100116826643\\
-7.324	-6.98693219867687\\
-8.545	-8.15173889100135\\
-3.662	-3.49346609933844\\
-1.221	-1.16480669232448\\
-6.104	-5.82307948398739\\
-9.766	-9.31654558332583\\
-8.545	-8.15173889100135\\
-9.766	-9.31654558332583\\
-17.09	-16.3034777820027\\
-10.986	-10.4803982980153\\
-13.428	-12.8100116826643\\
-15.869	-15.1386710896782\\
-10.986	-10.4803982980153\\
-6.104	-5.82307948398739\\
-3.662	-3.49346609933844\\
-4.883	-4.65827279166291\\
-6.104	-5.82307948398739\\
-4.883	-4.65827279166291\\
-6.104	-5.82307948398739\\
-9.766	-9.31654558332583\\
-8.545	-8.15173889100135\\
-9.766	-9.31654558332583\\
-6.104	-5.82307948398739\\
-3.662	-3.49346609933844\\
-7.324	-6.98693219867687\\
-10.986	-10.4803982980153\\
-8.545	-8.15173889100135\\
-12.207	-11.6452049903398\\
-8.545	-8.15173889100135\\
-7.324	-6.98693219867687\\
-8.545	-8.15173889100135\\
-12.207	-11.6452049903398\\
-14.648	-13.9738643973537\\
-10.986	-10.4803982980153\\
-17.09	-16.3034777820027\\
-9.766	-9.31654558332583\\
-12.207	-11.6452049903398\\
-13.428	-12.8100116826643\\
-9.766	-9.31654558332583\\
-12.207	-11.6452049903398\\
-18.311	-17.4682844743272\\
-15.869	-15.1386710896782\\
-8.545	-8.15173889100135\\
-12.207	-11.6452049903398\\
-9.766	-9.31654558332583\\
-12.207	-11.6452049903398\\
-13.428	-12.8100116826643\\
-9.766	-9.31654558332583\\
-14.648	-13.9738643973537\\
-13.428	-12.8100116826643\\
-12.207	-11.6452049903398\\
-8.545	-8.15173889100135\\
-9.766	-9.31654558332583\\
-15.869	-15.1386710896782\\
-14.648	-13.9738643973537\\
-15.869	-15.1386710896782\\
-12.207	-11.6452049903398\\
-10.986	-10.4803982980153\\
-9.766	-9.31654558332583\\
-8.545	-8.15173889100135\\
-4.883	-4.65827279166291\\
-3.662	-3.49346609933844\\
-7.324	-6.98693219867687\\
-10.986	-10.4803982980153\\
-9.766	-9.31654558332583\\
-8.545	-8.15173889100135\\
-10.986	-10.4803982980153\\
-9.766	-9.31654558332583\\
-8.545	-8.15173889100135\\
-3.662	-3.49346609933844\\
-6.104	-5.82307948398739\\
-8.545	-8.15173889100135\\
-10.986	-10.4803982980153\\
-7.324	-6.98693219867687\\
-6.104	-5.82307948398739\\
-7.324	-6.98693219867687\\
-6.104	-5.82307948398739\\
-8.545	-8.15173889100135\\
-15.869	-15.1386710896782\\
-14.648	-13.9738643973537\\
-17.09	-16.3034777820027\\
-13.428	-12.8100116826643\\
-10.986	-10.4803982980153\\
-9.766	-9.31654558332583\\
-10.986	-10.4803982980153\\
-12.207	-11.6452049903398\\
-14.648	-13.9738643973537\\
-17.09	-16.3034777820027\\
-19.531	-18.6321371890167\\
-21.973	-20.9617505736656\\
-18.311	-17.4682844743272\\
-21.973	-20.9617505736656\\
-19.531	-18.6321371890167\\
-14.648	-13.9738643973537\\
-17.09	-16.3034777820027\\
-14.648	-13.9738643973537\\
-10.986	-10.4803982980153\\
-8.545	-8.15173889100135\\
-10.986	-10.4803982980153\\
-7.324	-6.98693219867687\\
-3.662	-3.49346609933844\\
-7.324	-6.98693219867687\\
-6.104	-5.82307948398739\\
-8.545	-8.15173889100135\\
-12.207	-11.6452049903398\\
-8.545	-8.15173889100135\\
-12.207	-11.6452049903398\\
-18.311	-17.4682844743272\\
-17.09	-16.3034777820027\\
-19.531	-18.6321371890167\\
-17.09	-16.3034777820027\\
-14.648	-13.9738643973537\\
-7.324	-6.98693219867687\\
-9.766	-9.31654558332583\\
-8.545	-8.15173889100135\\
-6.104	-5.82307948398739\\
-7.324	-6.98693219867687\\
-8.545	-8.15173889100135\\
-2.441	-2.32865940701396\\
-4.883	-4.65827279166291\\
-8.545	-8.15173889100135\\
-9.766	-9.31654558332583\\
-12.207	-11.6452049903398\\
-10.986	-10.4803982980153\\
-12.207	-11.6452049903398\\
-14.648	-13.9738643973537\\
-10.986	-10.4803982980153\\
-9.766	-9.31654558332583\\
-7.324	-6.98693219867687\\
-10.986	-10.4803982980153\\
-7.324	-6.98693219867687\\
-10.986	-10.4803982980153\\
-13.428	-12.8100116826643\\
-10.986	-10.4803982980153\\
-8.545	-8.15173889100135\\
-6.104	-5.82307948398739\\
-7.324	-6.98693219867687\\
-6.104	-5.82307948398739\\
-9.766	-9.31654558332583\\
-8.545	-8.15173889100135\\
-10.986	-10.4803982980153\\
-8.545	-8.15173889100135\\
-6.104	-5.82307948398739\\
-10.986	-10.4803982980153\\
-14.648	-13.9738643973537\\
-12.207	-11.6452049903398\\
-10.986	-10.4803982980153\\
-9.766	-9.31654558332583\\
-8.545	-8.15173889100135\\
-9.766	-9.31654558332583\\
-12.207	-11.6452049903398\\
-7.324	-6.98693219867687\\
-10.986	-10.4803982980153\\
-13.428	-12.8100116826643\\
-14.648	-13.9738643973537\\
-20.752	-19.7969438813411\\
-21.973	-20.9617505736656\\
-14.648	-13.9738643973537\\
-9.766	-9.31654558332583\\
-7.324	-6.98693219867687\\
-6.104	-5.82307948398739\\
-7.324	-6.98693219867687\\
-4.883	-4.65827279166291\\
-7.324	-6.98693219867687\\
-10.986	-10.4803982980153\\
-12.207	-11.6452049903398\\
-13.428	-12.8100116826643\\
-18.311	-17.4682844743272\\
-17.09	-16.3034777820027\\
-12.207	-11.6452049903398\\
-9.766	-9.31654558332583\\
-13.428	-12.8100116826643\\
-9.766	-9.31654558332583\\
-8.545	-8.15173889100135\\
-10.986	-10.4803982980153\\
-6.104	-5.82307948398739\\
-10.986	-10.4803982980153\\
-14.648	-13.9738643973537\\
-8.545	-8.15173889100135\\
-15.869	-15.1386710896782\\
-13.428	-12.8100116826643\\
-10.986	-10.4803982980153\\
-17.09	-16.3034777820027\\
-10.986	-10.4803982980153\\
-8.545	-8.15173889100135\\
-9.766	-9.31654558332583\\
-12.207	-11.6452049903398\\
-18.311	-17.4682844743272\\
-19.531	-18.6321371890167\\
-18.311	-17.4682844743272\\
-15.869	-15.1386710896782\\
-9.766	-9.31654558332583\\
-7.324	-6.98693219867687\\
-6.104	-5.82307948398739\\
-7.324	-6.98693219867687\\
-9.766	-9.31654558332583\\
-8.545	-8.15173889100135\\
-6.104	-5.82307948398739\\
-12.207	-11.6452049903398\\
-10.986	-10.4803982980153\\
-13.428	-12.8100116826643\\
-17.09	-16.3034777820027\\
-19.531	-18.6321371890167\\
-15.869	-15.1386710896782\\
-12.207	-11.6452049903398\\
-15.869	-15.1386710896782\\
-12.207	-11.6452049903398\\
-9.766	-9.31654558332583\\
-10.986	-10.4803982980153\\
-15.869	-15.1386710896782\\
-18.311	-17.4682844743272\\
-12.207	-11.6452049903398\\
-8.545	-8.15173889100135\\
-4.883	-4.65827279166291\\
-2.441	-2.32865940701396\\
0	0\\
-6.104	-5.82307948398739\\
-8.545	-8.15173889100135\\
-7.324	-6.98693219867687\\
-6.104	-5.82307948398739\\
-7.324	-6.98693219867687\\
-4.883	-4.65827279166291\\
-6.104	-5.82307948398739\\
-8.545	-8.15173889100135\\
-10.986	-10.4803982980153\\
-13.428	-12.8100116826643\\
-10.986	-10.4803982980153\\
-12.207	-11.6452049903398\\
-17.09	-16.3034777820027\\
-19.531	-18.6321371890167\\
-20.752	-19.7969438813411\\
-13.428	-12.8100116826643\\
-10.986	-10.4803982980153\\
-14.648	-13.9738643973537\\
-17.09	-16.3034777820027\\
-13.428	-12.8100116826643\\
-9.766	-9.31654558332583\\
-7.324	-6.98693219867687\\
-3.662	-3.49346609933844\\
-4.883	-4.65827279166291\\
-6.104	-5.82307948398739\\
-1.221	-1.16480669232448\\
-9.766	-9.31654558332583\\
-7.324	-6.98693219867687\\
-6.104	-5.82307948398739\\
-4.883	-4.65827279166291\\
-3.662	-3.49346609933844\\
-6.104	-5.82307948398739\\
-4.883	-4.65827279166291\\
-3.662	-3.49346609933844\\
-4.883	-4.65827279166291\\
-12.207	-11.6452049903398\\
-13.428	-12.8100116826643\\
-15.869	-15.1386710896782\\
-14.648	-13.9738643973537\\
-23.193	-22.1256032883551\\
-20.752	-19.7969438813411\\
-18.311	-17.4682844743272\\
-15.869	-15.1386710896782\\
-14.648	-13.9738643973537\\
-10.986	-10.4803982980153\\
-8.545	-8.15173889100135\\
-12.207	-11.6452049903398\\
-14.648	-13.9738643973537\\
-8.545	-8.15173889100135\\
-9.766	-9.31654558332583\\
-10.986	-10.4803982980153\\
-12.207	-11.6452049903398\\
-10.986	-10.4803982980153\\
-9.766	-9.31654558332583\\
-10.986	-10.4803982980153\\
-2.441	-2.32865940701396\\
-4.883	-4.65827279166291\\
-8.545	-8.15173889100135\\
-7.324	-6.98693219867687\\
1.221	1.16480669232448\\
-2.441	-2.32865940701396\\
-7.324	-6.98693219867687\\
-8.545	-8.15173889100135\\
-10.986	-10.4803982980153\\
-7.324	-6.98693219867687\\
-8.545	-8.15173889100135\\
-10.986	-10.4803982980153\\
-4.883	-4.65827279166291\\
-3.662	-3.49346609933844\\
-2.441	-2.32865940701396\\
-3.662	-3.49346609933844\\
-10.986	-10.4803982980153\\
-12.207	-11.6452049903398\\
-8.545	-8.15173889100135\\
-7.324	-6.98693219867687\\
-6.104	-5.82307948398739\\
-8.545	-8.15173889100135\\
-6.104	-5.82307948398739\\
-3.662	-3.49346609933844\\
-4.883	-4.65827279166291\\
-7.324	-6.98693219867687\\
-3.662	-3.49346609933844\\
-4.883	-4.65827279166291\\
-6.104	-5.82307948398739\\
-9.766	-9.31654558332583\\
-13.428	-12.8100116826643\\
-6.104	-5.82307948398739\\
-7.324	-6.98693219867687\\
-9.766	-9.31654558332583\\
-8.545	-8.15173889100135\\
-12.207	-11.6452049903398\\
-7.324	-6.98693219867687\\
-10.986	-10.4803982980153\\
-7.324	-6.98693219867687\\
-9.766	-9.31654558332583\\
-6.104	-5.82307948398739\\
-7.324	-6.98693219867687\\
-8.545	-8.15173889100135\\
-7.324	-6.98693219867687\\
-8.545	-8.15173889100135\\
-10.986	-10.4803982980153\\
-18.311	-17.4682844743272\\
-13.428	-12.8100116826643\\
-10.986	-10.4803982980153\\
-9.766	-9.31654558332583\\
-13.428	-12.8100116826643\\
-9.766	-9.31654558332583\\
-10.986	-10.4803982980153\\
-14.648	-13.9738643973537\\
-4.883	-4.65827279166291\\
-3.662	-3.49346609933844\\
-10.986	-10.4803982980153\\
-7.324	-6.98693219867687\\
-9.766	-9.31654558332583\\
-10.986	-10.4803982980153\\
-12.207	-11.6452049903398\\
-10.986	-10.4803982980153\\
-9.766	-9.31654558332583\\
-8.545	-8.15173889100135\\
-12.207	-11.6452049903398\\
-8.545	-8.15173889100135\\
-10.986	-10.4803982980153\\
-7.324	-6.98693219867687\\
-8.545	-8.15173889100135\\
-4.883	-4.65827279166291\\
-6.104	-5.82307948398739\\
-10.986	-10.4803982980153\\
-13.428	-12.8100116826643\\
-10.986	-10.4803982980153\\
-13.428	-12.8100116826643\\
-9.766	-9.31654558332583\\
-7.324	-6.98693219867687\\
-9.766	-9.31654558332583\\
-12.207	-11.6452049903398\\
-13.428	-12.8100116826643\\
-12.207	-11.6452049903398\\
-7.324	-6.98693219867687\\
-8.545	-8.15173889100135\\
-19.531	-18.6321371890167\\
-12.207	-11.6452049903398\\
-13.428	-12.8100116826643\\
-15.869	-15.1386710896782\\
-10.986	-10.4803982980153\\
-9.766	-9.31654558332583\\
-8.545	-8.15173889100135\\
-13.428	-12.8100116826643\\
-10.986	-10.4803982980153\\
-9.766	-9.31654558332583\\
-10.986	-10.4803982980153\\
-8.545	-8.15173889100135\\
-9.766	-9.31654558332583\\
-4.883	-4.65827279166291\\
-9.766	-9.31654558332583\\
-12.207	-11.6452049903398\\
-9.766	-9.31654558332583\\
-7.324	-6.98693219867687\\
-3.662	-3.49346609933844\\
-2.441	-2.32865940701396\\
-3.662	-3.49346609933844\\
-9.766	-9.31654558332583\\
-13.428	-12.8100116826643\\
-10.986	-10.4803982980153\\
-13.428	-12.8100116826643\\
-12.207	-11.6452049903398\\
-7.324	-6.98693219867687\\
-6.104	-5.82307948398739\\
-3.662	-3.49346609933844\\
-8.545	-8.15173889100135\\
-4.883	-4.65827279166291\\
-8.545	-8.15173889100135\\
-7.324	-6.98693219867687\\
-6.104	-5.82307948398739\\
-4.883	-4.65827279166291\\
-9.766	-9.31654558332583\\
-10.986	-10.4803982980153\\
-6.104	-5.82307948398739\\
-10.986	-10.4803982980153\\
-8.545	-8.15173889100135\\
-9.766	-9.31654558332583\\
-14.648	-13.9738643973537\\
-8.545	-8.15173889100135\\
-12.207	-11.6452049903398\\
-3.662	-3.49346609933844\\
-12.207	-11.6452049903398\\
-13.428	-12.8100116826643\\
-9.766	-9.31654558332583\\
-15.869	-15.1386710896782\\
-13.428	-12.8100116826643\\
-10.986	-10.4803982980153\\
-15.869	-15.1386710896782\\
-13.428	-12.8100116826643\\
-12.207	-11.6452049903398\\
-10.986	-10.4803982980153\\
-8.545	-8.15173889100135\\
-9.766	-9.31654558332583\\
-8.545	-8.15173889100135\\
-6.104	-5.82307948398739\\
-8.545	-8.15173889100135\\
-7.324	-6.98693219867687\\
-10.986	-10.4803982980153\\
-15.869	-15.1386710896782\\
-13.428	-12.8100116826643\\
-14.648	-13.9738643973537\\
-10.986	-10.4803982980153\\
-7.324	-6.98693219867687\\
-4.883	-4.65827279166291\\
-6.104	-5.82307948398739\\
-9.766	-9.31654558332583\\
-12.207	-11.6452049903398\\
-8.545	-8.15173889100135\\
-9.766	-9.31654558332583\\
-17.09	-16.3034777820027\\
-14.648	-13.9738643973537\\
-13.428	-12.8100116826643\\
-17.09	-16.3034777820027\\
-23.193	-22.1256032883551\\
-7.324	-6.98693219867687\\
-8.545	-8.15173889100135\\
-12.207	-11.6452049903398\\
-17.09	-16.3034777820027\\
-14.648	-13.9738643973537\\
-10.986	-10.4803982980153\\
-9.766	-9.31654558332583\\
-6.104	-5.82307948398739\\
-8.545	-8.15173889100135\\
-6.104	-5.82307948398739\\
-3.662	-3.49346609933844\\
-6.104	-5.82307948398739\\
-3.662	-3.49346609933844\\
-4.883	-4.65827279166291\\
-12.207	-11.6452049903398\\
-7.324	-6.98693219867687\\
-9.766	-9.31654558332583\\
-7.324	-6.98693219867687\\
-9.766	-9.31654558332583\\
-10.986	-10.4803982980153\\
-13.428	-12.8100116826643\\
-10.986	-10.4803982980153\\
-13.428	-12.8100116826643\\
-12.207	-11.6452049903398\\
-13.428	-12.8100116826643\\
-9.766	-9.31654558332583\\
-14.648	-13.9738643973537\\
-13.428	-12.8100116826643\\
-7.324	-6.98693219867687\\
-9.766	-9.31654558332583\\
-10.986	-10.4803982980153\\
-8.545	-8.15173889100135\\
-13.428	-12.8100116826643\\
-14.648	-13.9738643973537\\
-13.428	-12.8100116826643\\
-20.752	-19.7969438813411\\
-10.986	-10.4803982980153\\
-12.207	-11.6452049903398\\
-10.986	-10.4803982980153\\
-15.869	-15.1386710896782\\
-7.324	-6.98693219867687\\
-10.986	-10.4803982980153\\
-15.869	-15.1386710896782\\
-4.883	-4.65827279166291\\
-7.324	-6.98693219867687\\
-9.766	-9.31654558332583\\
-10.986	-10.4803982980153\\
-6.104	-5.82307948398739\\
-4.883	-4.65827279166291\\
-6.104	-5.82307948398739\\
-7.324	-6.98693219867687\\
-9.766	-9.31654558332583\\
-13.428	-12.8100116826643\\
-10.986	-10.4803982980153\\
-14.648	-13.9738643973537\\
-13.428	-12.8100116826643\\
-12.207	-11.6452049903398\\
-8.545	-8.15173889100135\\
-10.986	-10.4803982980153\\
-9.766	-9.31654558332583\\
-15.869	-15.1386710896782\\
-13.428	-12.8100116826643\\
-12.207	-11.6452049903398\\
-7.324	-6.98693219867687\\
-12.207	-11.6452049903398\\
-14.648	-13.9738643973537\\
-10.986	-10.4803982980153\\
-13.428	-12.8100116826643\\
-9.766	-9.31654558332583\\
-4.883	-4.65827279166291\\
-6.104	-5.82307948398739\\
-9.766	-9.31654558332583\\
-6.104	-5.82307948398739\\
-8.545	-8.15173889100135\\
-14.648	-13.9738643973537\\
-20.752	-19.7969438813411\\
-12.207	-11.6452049903398\\
-14.648	-13.9738643973537\\
-10.986	-10.4803982980153\\
-9.766	-9.31654558332583\\
-10.986	-10.4803982980153\\
-12.207	-11.6452049903398\\
-14.648	-13.9738643973537\\
-18.311	-17.4682844743272\\
-19.531	-18.6321371890167\\
-14.648	-13.9738643973537\\
-19.531	-18.6321371890167\\
-15.869	-15.1386710896782\\
-14.648	-13.9738643973537\\
-12.207	-11.6452049903398\\
-10.986	-10.4803982980153\\
-13.428	-12.8100116826643\\
-18.311	-17.4682844743272\\
-7.324	-6.98693219867687\\
-8.545	-8.15173889100135\\
-12.207	-11.6452049903398\\
-9.766	-9.31654558332583\\
-8.545	-8.15173889100135\\
-3.662	-3.49346609933844\\
-7.324	-6.98693219867687\\
-13.428	-12.8100116826643\\
-20.752	-19.7969438813411\\
-21.973	-20.9617505736656\\
-19.531	-18.6321371890167\\
-14.648	-13.9738643973537\\
-17.09	-16.3034777820027\\
-21.973	-20.9617505736656\\
-17.09	-16.3034777820027\\
-14.648	-13.9738643973537\\
-18.311	-17.4682844743272\\
-13.428	-12.8100116826643\\
-15.869	-15.1386710896782\\
-12.207	-11.6452049903398\\
-7.324	-6.98693219867687\\
-10.986	-10.4803982980153\\
-6.104	-5.82307948398739\\
-9.766	-9.31654558332583\\
-7.324	-6.98693219867687\\
-13.428	-12.8100116826643\\
-9.766	-9.31654558332583\\
-8.545	-8.15173889100135\\
-9.766	-9.31654558332583\\
-12.207	-11.6452049903398\\
-10.986	-10.4803982980153\\
-7.324	-6.98693219867687\\
-13.428	-12.8100116826643\\
-14.648	-13.9738643973537\\
-9.766	-9.31654558332583\\
-13.428	-12.8100116826643\\
-12.207	-11.6452049903398\\
-9.766	-9.31654558332583\\
-6.104	-5.82307948398739\\
-4.883	-4.65827279166291\\
-6.104	-5.82307948398739\\
-9.766	-9.31654558332583\\
-4.883	-4.65827279166291\\
-6.104	-5.82307948398739\\
-4.883	-4.65827279166291\\
-12.207	-11.6452049903398\\
-10.986	-10.4803982980153\\
-12.207	-11.6452049903398\\
-14.648	-13.9738643973537\\
-13.428	-12.8100116826643\\
-14.648	-13.9738643973537\\
-10.986	-10.4803982980153\\
-4.883	-4.65827279166291\\
-8.545	-8.15173889100135\\
-7.324	-6.98693219867687\\
-10.986	-10.4803982980153\\
-15.869	-15.1386710896782\\
-9.766	-9.31654558332583\\
-15.869	-15.1386710896782\\
-17.09	-16.3034777820027\\
-6.104	-5.82307948398739\\
-7.324	-6.98693219867687\\
-9.766	-9.31654558332583\\
-10.986	-10.4803982980153\\
-14.648	-13.9738643973537\\
-4.883	-4.65827279166291\\
-10.986	-10.4803982980153\\
-9.766	-9.31654558332583\\
-10.986	-10.4803982980153\\
-13.428	-12.8100116826643\\
-18.311	-17.4682844743272\\
-15.869	-15.1386710896782\\
-14.648	-13.9738643973537\\
-13.428	-12.8100116826643\\
-8.545	-8.15173889100135\\
-9.766	-9.31654558332583\\
0	0\\
-7.324	-6.98693219867687\\
-10.986	-10.4803982980153\\
-7.324	-6.98693219867687\\
-8.545	-8.15173889100135\\
-17.09	-16.3034777820027\\
-15.869	-15.1386710896782\\
-18.311	-17.4682844743272\\
-10.986	-10.4803982980153\\
-8.545	-8.15173889100135\\
-15.869	-15.1386710896782\\
-17.09	-16.3034777820027\\
-14.648	-13.9738643973537\\
-9.766	-9.31654558332583\\
-10.986	-10.4803982980153\\
-15.869	-15.1386710896782\\
-13.428	-12.8100116826643\\
-7.324	-6.98693219867687\\
-4.883	-4.65827279166291\\
-6.104	-5.82307948398739\\
-10.986	-10.4803982980153\\
-9.766	-9.31654558332583\\
-6.104	-5.82307948398739\\
-7.324	-6.98693219867687\\
-6.104	-5.82307948398739\\
-1.221	-1.16480669232448\\
-7.324	-6.98693219867687\\
-10.986	-10.4803982980153\\
-13.428	-12.8100116826643\\
-18.311	-17.4682844743272\\
-20.752	-19.7969438813411\\
-18.311	-17.4682844743272\\
-17.09	-16.3034777820027\\
-12.207	-11.6452049903398\\
-10.986	-10.4803982980153\\
-12.207	-11.6452049903398\\
-8.545	-8.15173889100135\\
-7.324	-6.98693219867687\\
-6.104	-5.82307948398739\\
-9.766	-9.31654558332583\\
-10.986	-10.4803982980153\\
-7.324	-6.98693219867687\\
-3.662	-3.49346609933844\\
-6.104	-5.82307948398739\\
-7.324	-6.98693219867687\\
-4.883	-4.65827279166291\\
-8.545	-8.15173889100135\\
-14.648	-13.9738643973537\\
-10.986	-10.4803982980153\\
-14.648	-13.9738643973537\\
-19.531	-18.6321371890167\\
-21.973	-20.9617505736656\\
-14.648	-13.9738643973537\\
-10.986	-10.4803982980153\\
-8.545	-8.15173889100135\\
-7.324	-6.98693219867687\\
-9.766	-9.31654558332583\\
-8.545	-8.15173889100135\\
-9.766	-9.31654558332583\\
-7.324	-6.98693219867687\\
-9.766	-9.31654558332583\\
-8.545	-8.15173889100135\\
-7.324	-6.98693219867687\\
-9.766	-9.31654558332583\\
-4.883	-4.65827279166291\\
-2.441	-2.32865940701396\\
-1.221	-1.16480669232448\\
-3.662	-3.49346609933844\\
-6.104	-5.82307948398739\\
-2.441	-2.32865940701396\\
-7.324	-6.98693219867687\\
-6.104	-5.82307948398739\\
-9.766	-9.31654558332583\\
-10.986	-10.4803982980153\\
-8.545	-8.15173889100135\\
-7.324	-6.98693219867687\\
-13.428	-12.8100116826643\\
-19.531	-18.6321371890167\\
-13.428	-12.8100116826643\\
-10.986	-10.4803982980153\\
-9.766	-9.31654558332583\\
-3.662	-3.49346609933844\\
-7.324	-6.98693219867687\\
-8.545	-8.15173889100135\\
-9.766	-9.31654558332583\\
-13.428	-12.8100116826643\\
-10.986	-10.4803982980153\\
-9.766	-9.31654558332583\\
-8.545	-8.15173889100135\\
-7.324	-6.98693219867687\\
-4.883	-4.65827279166291\\
-3.662	-3.49346609933844\\
-6.104	-5.82307948398739\\
-7.324	-6.98693219867687\\
-10.986	-10.4803982980153\\
-9.766	-9.31654558332583\\
-4.883	-4.65827279166291\\
-9.766	-9.31654558332583\\
-12.207	-11.6452049903398\\
-6.104	-5.82307948398739\\
-13.428	-12.8100116826643\\
-12.207	-11.6452049903398\\
-7.324	-6.98693219867687\\
-8.545	-8.15173889100135\\
-12.207	-11.6452049903398\\
-9.766	-9.31654558332583\\
-6.104	-5.82307948398739\\
-9.766	-9.31654558332583\\
-4.883	-4.65827279166291\\
-3.662	-3.49346609933844\\
-2.441	-2.32865940701396\\
-4.883	-4.65827279166291\\
-7.324	-6.98693219867687\\
-8.545	-8.15173889100135\\
-4.883	-4.65827279166291\\
-3.662	-3.49346609933844\\
-2.441	-2.32865940701396\\
-4.883	-4.65827279166291\\
-8.545	-8.15173889100135\\
-10.986	-10.4803982980153\\
-6.104	-5.82307948398739\\
-8.545	-8.15173889100135\\
-10.986	-10.4803982980153\\
-13.428	-12.8100116826643\\
-10.986	-10.4803982980153\\
-12.207	-11.6452049903398\\
-13.428	-12.8100116826643\\
-14.648	-13.9738643973537\\
-13.428	-12.8100116826643\\
-18.311	-17.4682844743272\\
-15.869	-15.1386710896782\\
-12.207	-11.6452049903398\\
-13.428	-12.8100116826643\\
-9.766	-9.31654558332583\\
-7.324	-6.98693219867687\\
-10.986	-10.4803982980153\\
-13.428	-12.8100116826643\\
-12.207	-11.6452049903398\\
-10.986	-10.4803982980153\\
-8.545	-8.15173889100135\\
-12.207	-11.6452049903398\\
-8.545	-8.15173889100135\\
-14.648	-13.9738643973537\\
-15.869	-15.1386710896782\\
-10.986	-10.4803982980153\\
-6.104	-5.82307948398739\\
-9.766	-9.31654558332583\\
-2.441	-2.32865940701396\\
-3.662	-3.49346609933844\\
-7.324	-6.98693219867687\\
-4.883	-4.65827279166291\\
-3.662	-3.49346609933844\\
-8.545	-8.15173889100135\\
-4.883	-4.65827279166291\\
-7.324	-6.98693219867687\\
-6.104	-5.82307948398739\\
-4.883	-4.65827279166291\\
-6.104	-5.82307948398739\\
-7.324	-6.98693219867687\\
-6.104	-5.82307948398739\\
-7.324	-6.98693219867687\\
-14.648	-13.9738643973537\\
-9.766	-9.31654558332583\\
-8.545	-8.15173889100135\\
-10.986	-10.4803982980153\\
-7.324	-6.98693219867687\\
-6.104	-5.82307948398739\\
-10.986	-10.4803982980153\\
-8.545	-8.15173889100135\\
-4.883	-4.65827279166291\\
-10.986	-10.4803982980153\\
-8.545	-8.15173889100135\\
-7.324	-6.98693219867687\\
-2.441	-2.32865940701396\\
-3.662	-3.49346609933844\\
-4.883	-4.65827279166291\\
-7.324	-6.98693219867687\\
-8.545	-8.15173889100135\\
-9.766	-9.31654558332583\\
-7.324	-6.98693219867687\\
-10.986	-10.4803982980153\\
-14.648	-13.9738643973537\\
-12.207	-11.6452049903398\\
-10.986	-10.4803982980153\\
-14.648	-13.9738643973537\\
-12.207	-11.6452049903398\\
-17.09	-16.3034777820027\\
-15.869	-15.1386710896782\\
-19.531	-18.6321371890167\\
-20.752	-19.7969438813411\\
-13.428	-12.8100116826643\\
-6.104	-5.82307948398739\\
-8.545	-8.15173889100135\\
-12.207	-11.6452049903398\\
-8.545	-8.15173889100135\\
-7.324	-6.98693219867687\\
-10.986	-10.4803982980153\\
-13.428	-12.8100116826643\\
-14.648	-13.9738643973537\\
-13.428	-12.8100116826643\\
-9.766	-9.31654558332583\\
-6.104	-5.82307948398739\\
-10.986	-10.4803982980153\\
-12.207	-11.6452049903398\\
-14.648	-13.9738643973537\\
-12.207	-11.6452049903398\\
-19.531	-18.6321371890167\\
-17.09	-16.3034777820027\\
-20.752	-19.7969438813411\\
-18.311	-17.4682844743272\\
-15.869	-15.1386710896782\\
-17.09	-16.3034777820027\\
-15.869	-15.1386710896782\\
-10.986	-10.4803982980153\\
-8.545	-8.15173889100135\\
-9.766	-9.31654558332583\\
-12.207	-11.6452049903398\\
-8.545	-8.15173889100135\\
-13.428	-12.8100116826643\\
-12.207	-11.6452049903398\\
-8.545	-8.15173889100135\\
-9.766	-9.31654558332583\\
-12.207	-11.6452049903398\\
-13.428	-12.8100116826643\\
-14.648	-13.9738643973537\\
-10.986	-10.4803982980153\\
-9.766	-9.31654558332583\\
-8.545	-8.15173889100135\\
-12.207	-11.6452049903398\\
-8.545	-8.15173889100135\\
-7.324	-6.98693219867687\\
-2.441	-2.32865940701396\\
-7.324	-6.98693219867687\\
-9.766	-9.31654558332583\\
-12.207	-11.6452049903398\\
-8.545	-8.15173889100135\\
-6.104	-5.82307948398739\\
-7.324	-6.98693219867687\\
-8.545	-8.15173889100135\\
-9.766	-9.31654558332583\\
-6.104	-5.82307948398739\\
-13.428	-12.8100116826643\\
-14.648	-13.9738643973537\\
-15.869	-15.1386710896782\\
-17.09	-16.3034777820027\\
-12.207	-11.6452049903398\\
-7.324	-6.98693219867687\\
-8.545	-8.15173889100135\\
-12.207	-11.6452049903398\\
-10.986	-10.4803982980153\\
-6.104	-5.82307948398739\\
-8.545	-8.15173889100135\\
-12.207	-11.6452049903398\\
-10.986	-10.4803982980153\\
-18.311	-17.4682844743272\\
-12.207	-11.6452049903398\\
-8.545	-8.15173889100135\\
-7.324	-6.98693219867687\\
-8.545	-8.15173889100135\\
-13.428	-12.8100116826643\\
-12.207	-11.6452049903398\\
-13.428	-12.8100116826643\\
-14.648	-13.9738643973537\\
-12.207	-11.6452049903398\\
-10.986	-10.4803982980153\\
-13.428	-12.8100116826643\\
-9.766	-9.31654558332583\\
-12.207	-11.6452049903398\\
-7.324	-6.98693219867687\\
-10.986	-10.4803982980153\\
-13.428	-12.8100116826643\\
-14.648	-13.9738643973537\\
-8.545	-8.15173889100135\\
-14.648	-13.9738643973537\\
-13.428	-12.8100116826643\\
-12.207	-11.6452049903398\\
-7.324	-6.98693219867687\\
-17.09	-16.3034777820027\\
-8.545	-8.15173889100135\\
-10.986	-10.4803982980153\\
-12.207	-11.6452049903398\\
-8.545	-8.15173889100135\\
-6.104	-5.82307948398739\\
-17.09	-16.3034777820027\\
-20.752	-19.7969438813411\\
-12.207	-11.6452049903398\\
-15.869	-15.1386710896782\\
-13.428	-12.8100116826643\\
-12.207	-11.6452049903398\\
-10.986	-10.4803982980153\\
-8.545	-8.15173889100135\\
-7.324	-6.98693219867687\\
-10.986	-10.4803982980153\\
-17.09	-16.3034777820027\\
-20.752	-19.7969438813411\\
-15.869	-15.1386710896782\\
-14.648	-13.9738643973537\\
-10.986	-10.4803982980153\\
-13.428	-12.8100116826643\\
-9.766	-9.31654558332583\\
-8.545	-8.15173889100135\\
-7.324	-6.98693219867687\\
-3.662	-3.49346609933844\\
-8.545	-8.15173889100135\\
-10.986	-10.4803982980153\\
-6.104	-5.82307948398739\\
-14.648	-13.9738643973537\\
-6.104	-5.82307948398739\\
-3.662	-3.49346609933844\\
-7.324	-6.98693219867687\\
-8.545	-8.15173889100135\\
-9.766	-9.31654558332583\\
-8.545	-8.15173889100135\\
-7.324	-6.98693219867687\\
-6.104	-5.82307948398739\\
-4.883	-4.65827279166291\\
-10.986	-10.4803982980153\\
-8.545	-8.15173889100135\\
-6.104	-5.82307948398739\\
-4.883	-4.65827279166291\\
-8.545	-8.15173889100135\\
-10.986	-10.4803982980153\\
-9.766	-9.31654558332583\\
-6.104	-5.82307948398739\\
-7.324	-6.98693219867687\\
-10.986	-10.4803982980153\\
-14.648	-13.9738643973537\\
-15.869	-15.1386710896782\\
-13.428	-12.8100116826643\\
-7.324	-6.98693219867687\\
-12.207	-11.6452049903398\\
-9.766	-9.31654558332583\\
-6.104	-5.82307948398739\\
-9.766	-9.31654558332583\\
-8.545	-8.15173889100135\\
-6.104	-5.82307948398739\\
-4.883	-4.65827279166291\\
-7.324	-6.98693219867687\\
-9.766	-9.31654558332583\\
-13.428	-12.8100116826643\\
-12.207	-11.6452049903398\\
-13.428	-12.8100116826643\\
-17.09	-16.3034777820027\\
-21.973	-20.9617505736656\\
-23.193	-22.1256032883551\\
-13.428	-12.8100116826643\\
-17.09	-16.3034777820027\\
-15.869	-15.1386710896782\\
-20.752	-19.7969438813411\\
-14.648	-13.9738643973537\\
-7.324	-6.98693219867687\\
-6.104	-5.82307948398739\\
-7.324	-6.98693219867687\\
-6.104	-5.82307948398739\\
-3.662	-3.49346609933844\\
-2.441	-2.32865940701396\\
-4.883	-4.65827279166291\\
-8.545	-8.15173889100135\\
-9.766	-9.31654558332583\\
-4.883	-4.65827279166291\\
-7.324	-6.98693219867687\\
-6.104	-5.82307948398739\\
-8.545	-8.15173889100135\\
-7.324	-6.98693219867687\\
-4.883	-4.65827279166291\\
-3.662	-3.49346609933844\\
-6.104	-5.82307948398739\\
-7.324	-6.98693219867687\\
-3.662	-3.49346609933844\\
-9.766	-9.31654558332583\\
-7.324	-6.98693219867687\\
-9.766	-9.31654558332583\\
-7.324	-6.98693219867687\\
-8.545	-8.15173889100135\\
-12.207	-11.6452049903398\\
-9.766	-9.31654558332583\\
-12.207	-11.6452049903398\\
-10.986	-10.4803982980153\\
-7.324	-6.98693219867687\\
-6.104	-5.82307948398739\\
-13.428	-12.8100116826643\\
-14.648	-13.9738643973537\\
-10.986	-10.4803982980153\\
-12.207	-11.6452049903398\\
-8.545	-8.15173889100135\\
-17.09	-16.3034777820027\\
-13.428	-12.8100116826643\\
-14.648	-13.9738643973537\\
-23.193	-22.1256032883551\\
-15.869	-15.1386710896782\\
-8.545	-8.15173889100135\\
-7.324	-6.98693219867687\\
-6.104	-5.82307948398739\\
-9.766	-9.31654558332583\\
-13.428	-12.8100116826643\\
-15.869	-15.1386710896782\\
-14.648	-13.9738643973537\\
-15.869	-15.1386710896782\\
-9.766	-9.31654558332583\\
-10.986	-10.4803982980153\\
-12.207	-11.6452049903398\\
-6.104	-5.82307948398739\\
-4.883	-4.65827279166291\\
-7.324	-6.98693219867687\\
-8.545	-8.15173889100135\\
-10.986	-10.4803982980153\\
-13.428	-12.8100116826643\\
-10.986	-10.4803982980153\\
-8.545	-8.15173889100135\\
-7.324	-6.98693219867687\\
-10.986	-10.4803982980153\\
-14.648	-13.9738643973537\\
-15.869	-15.1386710896782\\
-17.09	-16.3034777820027\\
-20.752	-19.7969438813411\\
-14.648	-13.9738643973537\\
-18.311	-17.4682844743272\\
-12.207	-11.6452049903398\\
-8.545	-8.15173889100135\\
-14.648	-13.9738643973537\\
-15.869	-15.1386710896782\\
-10.986	-10.4803982980153\\
-15.869	-15.1386710896782\\
-19.531	-18.6321371890167\\
-17.09	-16.3034777820027\\
-7.324	-6.98693219867687\\
-6.104	-5.82307948398739\\
-12.207	-11.6452049903398\\
-6.104	-5.82307948398739\\
-8.545	-8.15173889100135\\
-6.104	-5.82307948398739\\
-9.766	-9.31654558332583\\
-3.662	-3.49346609933844\\
-4.883	-4.65827279166291\\
-7.324	-6.98693219867687\\
-9.766	-9.31654558332583\\
-10.986	-10.4803982980153\\
-12.207	-11.6452049903398\\
-13.428	-12.8100116826643\\
-15.869	-15.1386710896782\\
-13.428	-12.8100116826643\\
-8.545	-8.15173889100135\\
-7.324	-6.98693219867687\\
-10.986	-10.4803982980153\\
-8.545	-8.15173889100135\\
-7.324	-6.98693219867687\\
-4.883	-4.65827279166291\\
-6.104	-5.82307948398739\\
-4.883	-4.65827279166291\\
-3.662	-3.49346609933844\\
-4.883	-4.65827279166291\\
-8.545	-8.15173889100135\\
-7.324	-6.98693219867687\\
-6.104	-5.82307948398739\\
-4.883	-4.65827279166291\\
-9.766	-9.31654558332583\\
-7.324	-6.98693219867687\\
-9.766	-9.31654558332583\\
-8.545	-8.15173889100135\\
-10.986	-10.4803982980153\\
-9.766	-9.31654558332583\\
-10.986	-10.4803982980153\\
-9.766	-9.31654558332583\\
-7.324	-6.98693219867687\\
-8.545	-8.15173889100135\\
-9.766	-9.31654558332583\\
-14.648	-13.9738643973537\\
-9.766	-9.31654558332583\\
-10.986	-10.4803982980153\\
-15.869	-15.1386710896782\\
-10.986	-10.4803982980153\\
-9.766	-9.31654558332583\\
-8.545	-8.15173889100135\\
-7.324	-6.98693219867687\\
-4.883	-4.65827279166291\\
-3.662	-3.49346609933844\\
-6.104	-5.82307948398739\\
-10.986	-10.4803982980153\\
-7.324	-6.98693219867687\\
-9.766	-9.31654558332583\\
-7.324	-6.98693219867687\\
-8.545	-8.15173889100135\\
-2.441	-2.32865940701396\\
-4.883	-4.65827279166291\\
-2.441	-2.32865940701396\\
-8.545	-8.15173889100135\\
-10.986	-10.4803982980153\\
-7.324	-6.98693219867687\\
-9.766	-9.31654558332583\\
-12.207	-11.6452049903398\\
-13.428	-12.8100116826643\\
-12.207	-11.6452049903398\\
-8.545	-8.15173889100135\\
-7.324	-6.98693219867687\\
-9.766	-9.31654558332583\\
-8.545	-8.15173889100135\\
-12.207	-11.6452049903398\\
-10.986	-10.4803982980153\\
-12.207	-11.6452049903398\\
-15.869	-15.1386710896782\\
-8.545	-8.15173889100135\\
-13.428	-12.8100116826643\\
-15.869	-15.1386710896782\\
-14.648	-13.9738643973537\\
-21.973	-20.9617505736656\\
-25.635	-24.455216673004\\
-15.869	-15.1386710896782\\
-18.311	-17.4682844743272\\
-21.973	-20.9617505736656\\
-20.752	-19.7969438813411\\
-24.414	-23.2904099806796\\
-23.193	-22.1256032883551\\
-25.635	-24.455216673004\\
-20.752	-19.7969438813411\\
-14.648	-13.9738643973537\\
-10.986	-10.4803982980153\\
-13.428	-12.8100116826643\\
-9.766	-9.31654558332583\\
-8.545	-8.15173889100135\\
-9.766	-9.31654558332583\\
-13.428	-12.8100116826643\\
-10.986	-10.4803982980153\\
-8.545	-8.15173889100135\\
-6.104	-5.82307948398739\\
-10.986	-10.4803982980153\\
-7.324	-6.98693219867687\\
-6.104	-5.82307948398739\\
-7.324	-6.98693219867687\\
-3.662	-3.49346609933844\\
-6.104	-5.82307948398739\\
-9.766	-9.31654558332583\\
-10.986	-10.4803982980153\\
-8.545	-8.15173889100135\\
-12.207	-11.6452049903398\\
-9.766	-9.31654558332583\\
-4.883	-4.65827279166291\\
-7.324	-6.98693219867687\\
-4.883	-4.65827279166291\\
-6.104	-5.82307948398739\\
-8.545	-8.15173889100135\\
-4.883	-4.65827279166291\\
-9.766	-9.31654558332583\\
-14.648	-13.9738643973537\\
-15.869	-15.1386710896782\\
-10.986	-10.4803982980153\\
-7.324	-6.98693219867687\\
-8.545	-8.15173889100135\\
-12.207	-11.6452049903398\\
-13.428	-12.8100116826643\\
-7.324	-6.98693219867687\\
-6.104	-5.82307948398739\\
-8.545	-8.15173889100135\\
-2.441	-2.32865940701396\\
-7.324	-6.98693219867687\\
-10.986	-10.4803982980153\\
-7.324	-6.98693219867687\\
-6.104	-5.82307948398739\\
-3.662	-3.49346609933844\\
-7.324	-6.98693219867687\\
-1.221	-1.16480669232448\\
-3.662	-3.49346609933844\\
-7.324	-6.98693219867687\\
-10.986	-10.4803982980153\\
-12.207	-11.6452049903398\\
-10.986	-10.4803982980153\\
-8.545	-8.15173889100135\\
-10.986	-10.4803982980153\\
-17.09	-16.3034777820027\\
-18.311	-17.4682844743272\\
-8.545	-8.15173889100135\\
-4.883	-4.65827279166291\\
-14.648	-13.9738643973537\\
-10.986	-10.4803982980153\\
-8.545	-8.15173889100135\\
-9.766	-9.31654558332583\\
-10.986	-10.4803982980153\\
-17.09	-16.3034777820027\\
-18.311	-17.4682844743272\\
-19.531	-18.6321371890167\\
-14.648	-13.9738643973537\\
-13.428	-12.8100116826643\\
-9.766	-9.31654558332583\\
-8.545	-8.15173889100135\\
-10.986	-10.4803982980153\\
-9.766	-9.31654558332583\\
-12.207	-11.6452049903398\\
-8.545	-8.15173889100135\\
-9.766	-9.31654558332583\\
-8.545	-8.15173889100135\\
-6.104	-5.82307948398739\\
-4.883	-4.65827279166291\\
-2.441	-2.32865940701396\\
-4.883	-4.65827279166291\\
-7.324	-6.98693219867687\\
-6.104	-5.82307948398739\\
-8.545	-8.15173889100135\\
-12.207	-11.6452049903398\\
-8.545	-8.15173889100135\\
-4.883	-4.65827279166291\\
-1.221	-1.16480669232448\\
-7.324	-6.98693219867687\\
-13.428	-12.8100116826643\\
-14.648	-13.9738643973537\\
-20.752	-19.7969438813411\\
-24.414	-23.2904099806796\\
-15.869	-15.1386710896782\\
-17.09	-16.3034777820027\\
-21.973	-20.9617505736656\\
-18.311	-17.4682844743272\\
-15.869	-15.1386710896782\\
-18.311	-17.4682844743272\\
-20.752	-19.7969438813411\\
-19.531	-18.6321371890167\\
-26.855	-25.6190693876935\\
-20.752	-19.7969438813411\\
-12.207	-11.6452049903398\\
-7.324	-6.98693219867687\\
-6.104	-5.82307948398739\\
-10.986	-10.4803982980153\\
-12.207	-11.6452049903398\\
-8.545	-8.15173889100135\\
-10.986	-10.4803982980153\\
-6.104	-5.82307948398739\\
-3.662	-3.49346609933844\\
-9.766	-9.31654558332583\\
-12.207	-11.6452049903398\\
-9.766	-9.31654558332583\\
-6.104	-5.82307948398739\\
-7.324	-6.98693219867687\\
-10.986	-10.4803982980153\\
-17.09	-16.3034777820027\\
-15.869	-15.1386710896782\\
-17.09	-16.3034777820027\\
-20.752	-19.7969438813411\\
-25.635	-24.455216673004\\
-21.973	-20.9617505736656\\
-17.09	-16.3034777820027\\
-10.986	-10.4803982980153\\
-8.545	-8.15173889100135\\
-4.883	-4.65827279166291\\
-7.324	-6.98693219867687\\
-9.766	-9.31654558332583\\
-8.545	-8.15173889100135\\
-6.104	-5.82307948398739\\
-7.324	-6.98693219867687\\
-8.545	-8.15173889100135\\
-10.986	-10.4803982980153\\
-9.766	-9.31654558332583\\
-4.883	-4.65827279166291\\
-2.441	-2.32865940701396\\
-3.662	-3.49346609933844\\
-7.324	-6.98693219867687\\
-8.545	-8.15173889100135\\
-7.324	-6.98693219867687\\
-10.986	-10.4803982980153\\
-8.545	-8.15173889100135\\
-19.531	-18.6321371890167\\
-14.648	-13.9738643973537\\
-12.207	-11.6452049903398\\
-17.09	-16.3034777820027\\
-18.311	-17.4682844743272\\
-13.428	-12.8100116826643\\
-12.207	-11.6452049903398\\
-10.986	-10.4803982980153\\
-9.766	-9.31654558332583\\
-14.648	-13.9738643973537\\
-24.414	-23.2904099806796\\
-25.635	-24.455216673004\\
-20.752	-19.7969438813411\\
-19.531	-18.6321371890167\\
-14.648	-13.9738643973537\\
-17.09	-16.3034777820027\\
-19.531	-18.6321371890167\\
-14.648	-13.9738643973537\\
-10.986	-10.4803982980153\\
-9.766	-9.31654558332583\\
-12.207	-11.6452049903398\\
-13.428	-12.8100116826643\\
-8.545	-8.15173889100135\\
-7.324	-6.98693219867687\\
-8.545	-8.15173889100135\\
-14.648	-13.9738643973537\\
-17.09	-16.3034777820027\\
-12.207	-11.6452049903398\\
-10.986	-10.4803982980153\\
-9.766	-9.31654558332583\\
-6.104	-5.82307948398739\\
-4.883	-4.65827279166291\\
-6.104	-5.82307948398739\\
-4.883	-4.65827279166291\\
-7.324	-6.98693219867687\\
-9.766	-9.31654558332583\\
-10.986	-10.4803982980153\\
-8.545	-8.15173889100135\\
-3.662	-3.49346609933844\\
-6.104	-5.82307948398739\\
-8.545	-8.15173889100135\\
-9.766	-9.31654558332583\\
-8.545	-8.15173889100135\\
-4.883	-4.65827279166291\\
-10.986	-10.4803982980153\\
-14.648	-13.9738643973537\\
-10.986	-10.4803982980153\\
-7.324	-6.98693219867687\\
-9.766	-9.31654558332583\\
-8.545	-8.15173889100135\\
-6.104	-5.82307948398739\\
-4.883	-4.65827279166291\\
-10.986	-10.4803982980153\\
-9.766	-9.31654558332583\\
-3.662	-3.49346609933844\\
-6.104	-5.82307948398739\\
-9.766	-9.31654558332583\\
-12.207	-11.6452049903398\\
-7.324	-6.98693219867687\\
-6.104	-5.82307948398739\\
-10.986	-10.4803982980153\\
-13.428	-12.8100116826643\\
-12.207	-11.6452049903398\\
-17.09	-16.3034777820027\\
-19.531	-18.6321371890167\\
-12.207	-11.6452049903398\\
-17.09	-16.3034777820027\\
-13.428	-12.8100116826643\\
-10.986	-10.4803982980153\\
-18.311	-17.4682844743272\\
-14.648	-13.9738643973537\\
-7.324	-6.98693219867687\\
-8.545	-8.15173889100135\\
-19.531	-18.6321371890167\\
-21.973	-20.9617505736656\\
-17.09	-16.3034777820027\\
-12.207	-11.6452049903398\\
-17.09	-16.3034777820027\\
-14.648	-13.9738643973537\\
-9.766	-9.31654558332583\\
-8.545	-8.15173889100135\\
-12.207	-11.6452049903398\\
-9.766	-9.31654558332583\\
-12.207	-11.6452049903398\\
-10.986	-10.4803982980153\\
-6.104	-5.82307948398739\\
-4.883	-4.65827279166291\\
-6.104	-5.82307948398739\\
-10.986	-10.4803982980153\\
-8.545	-8.15173889100135\\
-13.428	-12.8100116826643\\
-15.869	-15.1386710896782\\
-13.428	-12.8100116826643\\
-9.766	-9.31654558332583\\
-6.104	-5.82307948398739\\
-14.648	-13.9738643973537\\
-10.986	-10.4803982980153\\
-7.324	-6.98693219867687\\
-13.428	-12.8100116826643\\
-10.986	-10.4803982980153\\
-13.428	-12.8100116826643\\
-15.869	-15.1386710896782\\
-20.752	-19.7969438813411\\
-17.09	-16.3034777820027\\
-14.648	-13.9738643973537\\
-15.869	-15.1386710896782\\
-14.648	-13.9738643973537\\
-7.324	-6.98693219867687\\
-12.207	-11.6452049903398\\
-13.428	-12.8100116826643\\
-15.869	-15.1386710896782\\
-12.207	-11.6452049903398\\
-8.545	-8.15173889100135\\
-6.104	-5.82307948398739\\
-10.986	-10.4803982980153\\
-13.428	-12.8100116826643\\
-7.324	-6.98693219867687\\
-10.986	-10.4803982980153\\
-9.766	-9.31654558332583\\
-3.662	-3.49346609933844\\
-6.104	-5.82307948398739\\
-9.766	-9.31654558332583\\
-6.104	-5.82307948398739\\
-4.883	-4.65827279166291\\
-2.441	-2.32865940701396\\
-8.545	-8.15173889100135\\
-9.766	-9.31654558332583\\
-13.428	-12.8100116826643\\
-17.09	-16.3034777820027\\
-13.428	-12.8100116826643\\
-6.104	-5.82307948398739\\
-12.207	-11.6452049903398\\
-20.752	-19.7969438813411\\
-15.869	-15.1386710896782\\
-14.648	-13.9738643973537\\
-19.531	-18.6321371890167\\
-17.09	-16.3034777820027\\
-13.428	-12.8100116826643\\
-12.207	-11.6452049903398\\
-15.869	-15.1386710896782\\
-14.648	-13.9738643973537\\
-10.986	-10.4803982980153\\
-14.648	-13.9738643973537\\
-20.752	-19.7969438813411\\
-18.311	-17.4682844743272\\
-9.766	-9.31654558332583\\
-7.324	-6.98693219867687\\
-6.104	-5.82307948398739\\
-14.648	-13.9738643973537\\
-2.441	-2.32865940701396\\
-3.662	-3.49346609933844\\
-7.324	-6.98693219867687\\
-10.986	-10.4803982980153\\
-12.207	-11.6452049903398\\
-9.766	-9.31654558332583\\
-3.662	-3.49346609933844\\
-4.883	-4.65827279166291\\
-3.662	-3.49346609933844\\
-7.324	-6.98693219867687\\
-3.662	-3.49346609933844\\
-8.545	-8.15173889100135\\
-7.324	-6.98693219867687\\
-6.104	-5.82307948398739\\
-7.324	-6.98693219867687\\
-3.662	-3.49346609933844\\
0	0\\
-2.441	-2.32865940701396\\
-6.104	-5.82307948398739\\
-10.986	-10.4803982980153\\
-18.311	-17.4682844743272\\
-14.648	-13.9738643973537\\
-9.766	-9.31654558332583\\
-6.104	-5.82307948398739\\
-8.545	-8.15173889100135\\
-1.221	-1.16480669232448\\
-3.662	-3.49346609933844\\
-6.104	-5.82307948398739\\
-12.207	-11.6452049903398\\
-7.324	-6.98693219867687\\
-9.766	-9.31654558332583\\
-10.986	-10.4803982980153\\
-9.766	-9.31654558332583\\
-10.986	-10.4803982980153\\
-13.428	-12.8100116826643\\
-17.09	-16.3034777820027\\
-14.648	-13.9738643973537\\
-2.441	-2.32865940701396\\
-3.662	-3.49346609933844\\
-8.545	-8.15173889100135\\
-4.883	-4.65827279166291\\
-6.104	-5.82307948398739\\
-3.662	-3.49346609933844\\
-1.221	-1.16480669232448\\
-2.441	-2.32865940701396\\
-4.883	-4.65827279166291\\
-8.545	-8.15173889100135\\
-13.428	-12.8100116826643\\
-6.104	-5.82307948398739\\
-13.428	-12.8100116826643\\
-7.324	-6.98693219867687\\
-3.662	-3.49346609933844\\
-12.207	-11.6452049903398\\
-8.545	-8.15173889100135\\
-13.428	-12.8100116826643\\
-21.973	-20.9617505736656\\
-10.986	-10.4803982980153\\
-8.545	-8.15173889100135\\
};
\end{axis}

\begin{axis}[%
width=4.927cm,
height=3cm,
at={(7cm,9.677cm)},
scale only axis,
xmin=-30,
xmax=10,
xlabel style={font=\color{white!15!black}},
xlabel={y(t-1)},
ymin=-29.297,
ymax=2.441,
ylabel style={font=\color{white!15!black}},
ylabel={y(t)},
axis background/.style={fill=white},
title style={font=\small},
title={C5, R = 0.7172},
axis x line*=bottom,
axis y line*=left
]
\addplot[only marks, mark=*, mark options={}, mark size=1.5000pt, color=mycolor1, fill=mycolor1] table[row sep=crcr]{%
x	y\\
-12.207	-10.986\\
-10.986	-15.869\\
-15.869	-10.986\\
-10.986	-12.207\\
-12.207	-13.428\\
-13.428	-15.869\\
-15.869	-14.648\\
-14.648	-14.648\\
-14.648	-7.324\\
-7.324	-8.545\\
-8.545	-10.986\\
-10.986	-9.766\\
-9.766	-9.766\\
-9.766	-6.104\\
-6.104	-2.441\\
-2.441	-3.662\\
-3.662	-2.441\\
-2.441	-12.207\\
-12.207	-13.428\\
-13.428	-12.207\\
-12.207	-10.986\\
-10.986	-6.104\\
-6.104	-6.104\\
-6.104	-1.221\\
-1.221	-6.104\\
-6.104	-10.986\\
-10.986	-12.207\\
-12.207	-8.545\\
-8.545	-14.648\\
-14.648	-14.648\\
-14.648	-9.766\\
-9.766	-13.428\\
-13.428	-12.207\\
-12.207	-9.766\\
-9.766	-9.766\\
-9.766	-8.545\\
-8.545	-8.545\\
-8.545	-12.207\\
-12.207	-10.986\\
-10.986	-10.986\\
-10.986	-10.986\\
-10.986	-10.986\\
-10.986	-14.648\\
-14.648	-15.869\\
-15.869	-9.766\\
-9.766	-8.545\\
-8.545	-8.545\\
-8.545	-6.104\\
-6.104	-7.324\\
-7.324	-7.324\\
-7.324	-4.883\\
-4.883	-9.766\\
-9.766	-10.986\\
-10.986	-8.545\\
-8.545	-13.428\\
-13.428	-15.869\\
-15.869	-8.545\\
-8.545	-6.104\\
-6.104	-6.104\\
-6.104	-9.766\\
-9.766	-6.104\\
-6.104	-4.883\\
-4.883	-7.324\\
-7.324	-7.324\\
-7.324	-3.662\\
-3.662	-6.104\\
-6.104	-8.545\\
-8.545	-10.986\\
-10.986	-10.986\\
-10.986	-9.766\\
-9.766	-13.428\\
-13.428	-10.986\\
-10.986	-12.207\\
-12.207	-10.986\\
-10.986	-10.986\\
-10.986	-7.324\\
-7.324	-8.545\\
-8.545	-14.648\\
-14.648	-20.752\\
-20.752	-20.752\\
-20.752	-19.531\\
-19.531	-13.428\\
-13.428	-15.869\\
-15.869	-24.414\\
-24.414	-23.193\\
-23.193	-15.869\\
-15.869	-21.973\\
-21.973	-28.076\\
-28.076	-23.193\\
-23.193	-17.09\\
-17.09	-14.648\\
-14.648	-13.428\\
-13.428	-9.766\\
-9.766	-10.986\\
-10.986	-12.207\\
-12.207	-4.883\\
-4.883	-4.883\\
-4.883	-7.324\\
-7.324	-10.986\\
-10.986	-9.766\\
-9.766	-9.766\\
-9.766	-10.986\\
-10.986	-13.428\\
-13.428	-14.648\\
-14.648	-15.869\\
-15.869	-14.648\\
-14.648	-17.09\\
-17.09	-12.207\\
-12.207	-12.207\\
-12.207	-10.986\\
-10.986	-8.545\\
-8.545	-9.766\\
-9.766	-12.207\\
-12.207	-8.545\\
-8.545	-9.766\\
-9.766	-7.324\\
-7.324	-6.104\\
-6.104	-8.545\\
-8.545	-6.104\\
-6.104	-4.883\\
-4.883	-8.545\\
-8.545	-13.428\\
-13.428	-12.207\\
-12.207	-10.986\\
-10.986	-7.324\\
-7.324	-8.545\\
-8.545	-19.531\\
-19.531	-14.648\\
-14.648	-10.986\\
-10.986	-17.09\\
-17.09	-12.207\\
-12.207	-6.104\\
-6.104	-9.766\\
-9.766	-8.545\\
-8.545	-14.648\\
-14.648	-15.869\\
-15.869	-19.531\\
-19.531	-14.648\\
-14.648	-14.648\\
-14.648	-13.428\\
-13.428	-13.428\\
-13.428	-14.648\\
-14.648	-10.986\\
-10.986	-7.324\\
-7.324	-7.324\\
-7.324	-6.104\\
-6.104	-8.545\\
-8.545	-10.986\\
-10.986	-15.869\\
-15.869	-14.648\\
-14.648	-8.545\\
-8.545	-8.545\\
-8.545	-9.766\\
-9.766	-6.104\\
-6.104	-2.441\\
-2.441	-4.883\\
-4.883	-7.324\\
-7.324	-8.545\\
-8.545	-6.104\\
-6.104	-7.324\\
-7.324	-6.104\\
-6.104	-4.883\\
-4.883	-4.883\\
-4.883	-1.221\\
-1.221	-3.662\\
-3.662	-12.207\\
-12.207	-14.648\\
-14.648	-15.869\\
-15.869	-12.207\\
-12.207	-7.324\\
-7.324	-6.104\\
-6.104	-3.662\\
-3.662	-7.324\\
-7.324	-8.545\\
-8.545	-9.766\\
-9.766	-12.207\\
-12.207	-18.311\\
-18.311	-18.311\\
-18.311	-19.531\\
-19.531	-18.311\\
-18.311	-15.869\\
-15.869	-15.869\\
-15.869	-17.09\\
-17.09	-18.311\\
-18.311	-12.207\\
-12.207	-10.986\\
-10.986	-12.207\\
-12.207	-10.986\\
-10.986	-12.207\\
-12.207	-9.766\\
-9.766	-15.869\\
-15.869	-18.311\\
-18.311	-12.207\\
-12.207	-7.324\\
-7.324	-9.766\\
-9.766	-17.09\\
-17.09	-18.311\\
-18.311	-18.311\\
-18.311	-21.973\\
-21.973	-20.752\\
-20.752	-17.09\\
-17.09	-15.869\\
-15.869	-18.311\\
-18.311	-20.752\\
-20.752	-17.09\\
-17.09	-8.545\\
-8.545	-14.648\\
-14.648	-12.207\\
-12.207	-7.324\\
-7.324	-10.986\\
-10.986	-9.766\\
-9.766	-8.545\\
-8.545	-7.324\\
-7.324	-8.545\\
-8.545	-8.545\\
-8.545	-9.766\\
-9.766	-13.428\\
-13.428	-14.648\\
-14.648	-7.324\\
-7.324	-8.545\\
-8.545	-12.207\\
-12.207	-17.09\\
-17.09	-15.869\\
-15.869	-8.545\\
-8.545	-8.545\\
-8.545	-9.766\\
-9.766	-8.545\\
-8.545	-7.324\\
-7.324	-6.104\\
-6.104	-8.545\\
-8.545	-9.766\\
-9.766	-9.766\\
-9.766	-9.766\\
-9.766	-12.207\\
-12.207	-14.648\\
-14.648	-15.869\\
-15.869	-10.986\\
-10.986	-13.428\\
-13.428	-17.09\\
-17.09	-12.207\\
-12.207	-3.662\\
-3.662	-9.766\\
-9.766	-8.545\\
-8.545	-9.766\\
-9.766	-8.545\\
-8.545	-8.545\\
-8.545	-8.545\\
-8.545	-10.986\\
-10.986	-7.324\\
-7.324	-7.324\\
-7.324	-9.766\\
-9.766	-6.104\\
-6.104	-8.545\\
-8.545	-6.104\\
-6.104	-3.662\\
-3.662	-6.104\\
-6.104	-4.883\\
-4.883	-10.986\\
-10.986	-12.207\\
-12.207	-12.207\\
-12.207	-18.311\\
-18.311	-10.986\\
-10.986	-4.883\\
-4.883	-6.104\\
-6.104	-6.104\\
-6.104	-8.545\\
-8.545	-6.104\\
-6.104	-8.545\\
-8.545	-7.324\\
-7.324	-13.428\\
-13.428	-9.766\\
-9.766	-14.648\\
-14.648	-10.986\\
-10.986	-9.766\\
-9.766	-6.104\\
-6.104	-3.662\\
-3.662	-3.662\\
-3.662	-1.221\\
-1.221	-2.441\\
-2.441	-8.545\\
-8.545	-8.545\\
-8.545	-7.324\\
-7.324	-9.766\\
-9.766	-15.869\\
-15.869	-10.986\\
-10.986	-9.766\\
-9.766	-9.766\\
-9.766	-13.428\\
-13.428	-14.648\\
-14.648	-12.207\\
-12.207	-12.207\\
-12.207	-6.104\\
-6.104	-2.441\\
-2.441	-4.883\\
-4.883	-4.883\\
-4.883	-6.104\\
-6.104	-3.662\\
-3.662	-4.883\\
-4.883	-9.766\\
-9.766	-7.324\\
-7.324	-7.324\\
-7.324	-10.986\\
-10.986	-7.324\\
-7.324	-4.883\\
-4.883	-9.766\\
-9.766	-12.207\\
-12.207	-9.766\\
-9.766	-10.986\\
-10.986	-7.324\\
-7.324	-7.324\\
-7.324	-10.986\\
-10.986	-14.648\\
-14.648	-13.428\\
-13.428	-8.545\\
-8.545	-14.648\\
-14.648	-13.428\\
-13.428	-13.428\\
-13.428	-13.428\\
-13.428	-13.428\\
-13.428	-10.986\\
-10.986	-10.986\\
-10.986	-12.207\\
-12.207	-18.311\\
-18.311	-17.09\\
-17.09	-10.986\\
-10.986	-10.986\\
-10.986	-10.986\\
-10.986	-13.428\\
-13.428	-13.428\\
-13.428	-9.766\\
-9.766	-12.207\\
-12.207	-14.648\\
-14.648	-14.648\\
-14.648	-9.766\\
-9.766	-8.545\\
-8.545	-10.986\\
-10.986	-18.311\\
-18.311	-14.648\\
-14.648	-13.428\\
-13.428	-9.766\\
-9.766	-10.986\\
-10.986	-9.766\\
-9.766	-6.104\\
-6.104	-4.883\\
-4.883	-3.662\\
-3.662	-3.662\\
-3.662	-7.324\\
-7.324	-10.986\\
-10.986	-9.766\\
-9.766	-6.104\\
-6.104	-7.324\\
-7.324	-9.766\\
-9.766	-6.104\\
-6.104	-6.104\\
-6.104	-9.766\\
-9.766	-6.104\\
-6.104	-7.324\\
-7.324	-12.207\\
-12.207	-12.207\\
-12.207	-8.545\\
-8.545	-4.883\\
-4.883	-6.104\\
-6.104	-7.324\\
-7.324	-6.104\\
-6.104	-7.324\\
-7.324	-14.648\\
-14.648	-9.766\\
-9.766	-17.09\\
-17.09	-20.752\\
-20.752	-18.311\\
-18.311	-13.428\\
-13.428	-10.986\\
-10.986	-10.986\\
-10.986	-12.207\\
-12.207	-13.428\\
-13.428	-14.648\\
-14.648	-15.869\\
-15.869	-20.752\\
-20.752	-20.752\\
-20.752	-24.414\\
-24.414	-15.869\\
-15.869	-20.752\\
-20.752	-20.752\\
-20.752	-14.648\\
-14.648	-17.09\\
-17.09	-18.311\\
-18.311	-9.766\\
-9.766	-7.324\\
-7.324	-8.545\\
-8.545	-9.766\\
-9.766	-7.324\\
-7.324	-6.104\\
-6.104	-7.324\\
-7.324	-6.104\\
-6.104	-3.662\\
-3.662	-4.883\\
-4.883	-7.324\\
-7.324	-2.441\\
-2.441	-10.986\\
-10.986	-8.545\\
-8.545	-7.324\\
-7.324	-15.869\\
-15.869	-19.531\\
-19.531	-19.531\\
-19.531	-19.531\\
-19.531	-14.648\\
-14.648	-17.09\\
-17.09	-12.207\\
-12.207	-7.324\\
-7.324	-10.986\\
-10.986	-7.324\\
-7.324	-3.662\\
-3.662	-6.104\\
-6.104	-4.883\\
-4.883	-2.441\\
-2.441	-4.883\\
-4.883	-10.986\\
-10.986	-9.766\\
-9.766	-10.986\\
-10.986	-10.986\\
-10.986	-12.207\\
-12.207	-14.648\\
-14.648	-10.986\\
-10.986	-10.986\\
-10.986	-8.545\\
-8.545	-7.324\\
-7.324	-13.428\\
-13.428	-9.766\\
-9.766	-6.104\\
-6.104	-10.986\\
-10.986	-13.428\\
-13.428	-10.986\\
-10.986	-7.324\\
-7.324	-7.324\\
-7.324	-7.324\\
-7.324	-4.883\\
-4.883	-6.104\\
-6.104	-7.324\\
-7.324	-10.986\\
-10.986	-8.545\\
-8.545	-7.324\\
-7.324	-10.986\\
-10.986	-8.545\\
-8.545	-6.104\\
-6.104	-7.324\\
-7.324	-13.428\\
-13.428	-14.648\\
-14.648	-13.428\\
-13.428	-12.207\\
-12.207	-12.207\\
-12.207	-8.545\\
-8.545	-9.766\\
-9.766	-8.545\\
-8.545	-12.207\\
-12.207	-7.324\\
-7.324	-4.883\\
-4.883	-9.766\\
-9.766	-9.766\\
-9.766	-10.986\\
-10.986	-12.207\\
-12.207	-10.986\\
-10.986	-17.09\\
-17.09	-20.752\\
-20.752	-23.193\\
-23.193	-15.869\\
-15.869	-9.766\\
-9.766	-6.104\\
-6.104	-6.104\\
-6.104	-9.766\\
-9.766	-6.104\\
-6.104	-9.766\\
-9.766	-6.104\\
-6.104	-7.324\\
-7.324	-8.545\\
-8.545	-10.986\\
-10.986	-13.428\\
-13.428	-14.648\\
-14.648	-19.531\\
-19.531	-14.648\\
-14.648	-13.428\\
-13.428	-10.986\\
-10.986	-10.986\\
-10.986	-10.986\\
-10.986	-12.207\\
-12.207	-9.766\\
-9.766	-10.986\\
-10.986	-9.766\\
-9.766	-7.324\\
-7.324	-12.207\\
-12.207	-13.428\\
-13.428	-8.545\\
-8.545	-9.766\\
-9.766	-18.311\\
-18.311	-12.207\\
-12.207	-12.207\\
-12.207	-17.09\\
-17.09	-14.648\\
-14.648	-10.986\\
-10.986	-7.324\\
-7.324	-7.324\\
-7.324	-12.207\\
-12.207	-17.09\\
-17.09	-20.752\\
-20.752	-18.311\\
-18.311	-13.428\\
-13.428	-9.766\\
-9.766	-8.545\\
-8.545	-6.104\\
-6.104	-7.324\\
-7.324	-4.883\\
-4.883	-8.545\\
-8.545	-8.545\\
-8.545	-6.104\\
-6.104	-9.766\\
-9.766	-10.986\\
-10.986	-12.207\\
-12.207	-14.648\\
-14.648	-17.09\\
-17.09	-19.531\\
-19.531	-18.311\\
-18.311	-13.428\\
-13.428	-14.648\\
-14.648	-12.207\\
-12.207	-7.324\\
-7.324	-13.428\\
-13.428	-19.531\\
-19.531	-18.311\\
-18.311	-12.207\\
-12.207	-6.104\\
-6.104	-4.883\\
-4.883	-1.221\\
-1.221	2.441\\
2.441	-4.883\\
-4.883	-6.104\\
-6.104	-8.545\\
-8.545	-6.104\\
-6.104	-4.883\\
-4.883	-4.883\\
-4.883	-7.324\\
-7.324	-4.883\\
-4.883	-4.883\\
-4.883	-8.545\\
-8.545	-12.207\\
-12.207	-13.428\\
-13.428	-10.986\\
-10.986	-14.648\\
-14.648	-17.09\\
-17.09	-17.09\\
-17.09	-20.752\\
-20.752	-20.752\\
-20.752	-14.648\\
-14.648	-10.986\\
-10.986	-10.986\\
-10.986	-18.311\\
-18.311	-18.311\\
-18.311	-12.207\\
-12.207	-8.545\\
-8.545	-4.883\\
-4.883	-4.883\\
-4.883	-4.883\\
-4.883	-2.441\\
-2.441	-6.104\\
-6.104	-8.545\\
-8.545	-7.324\\
-7.324	-7.324\\
-7.324	-6.104\\
-6.104	-3.662\\
-3.662	-4.883\\
-4.883	-4.883\\
-4.883	-6.104\\
-6.104	-2.441\\
-2.441	-2.441\\
-2.441	-7.324\\
-7.324	-9.766\\
-9.766	-13.428\\
-13.428	-15.869\\
-15.869	-10.986\\
-10.986	-18.311\\
-18.311	-23.193\\
-23.193	-21.973\\
-21.973	-19.531\\
-19.531	-14.648\\
-14.648	-14.648\\
-14.648	-17.09\\
-17.09	-13.428\\
-13.428	-9.766\\
-9.766	-7.324\\
-7.324	-7.324\\
-7.324	-13.428\\
-13.428	-8.545\\
-8.545	-12.207\\
-12.207	-10.986\\
-10.986	-8.545\\
-8.545	-8.545\\
-8.545	-10.986\\
-10.986	-9.766\\
-9.766	-12.207\\
-12.207	-10.986\\
-10.986	-8.545\\
-8.545	-12.207\\
-12.207	-6.104\\
-6.104	-2.441\\
-2.441	-7.324\\
-7.324	-7.324\\
-7.324	-2.441\\
-2.441	-1.221\\
-1.221	-2.441\\
-2.441	-6.104\\
-6.104	-6.104\\
-6.104	-8.545\\
-8.545	-7.324\\
-7.324	-13.428\\
-13.428	-8.545\\
-8.545	-6.104\\
-6.104	-10.986\\
-10.986	-7.324\\
-7.324	-4.883\\
-4.883	-4.883\\
-4.883	-2.441\\
-2.441	-3.662\\
-3.662	-3.662\\
-3.662	-6.104\\
-6.104	-9.766\\
-9.766	-8.545\\
-8.545	-4.883\\
-4.883	-7.324\\
-7.324	-7.324\\
-7.324	-7.324\\
-7.324	-6.104\\
-6.104	-2.441\\
-2.441	-7.324\\
-7.324	-3.662\\
-3.662	-4.883\\
-4.883	-4.883\\
-4.883	-7.324\\
-7.324	-6.104\\
-6.104	-4.883\\
-4.883	-3.662\\
-3.662	-4.883\\
-4.883	-4.883\\
-4.883	-13.428\\
-13.428	-13.428\\
-13.428	-7.324\\
-7.324	-12.207\\
-12.207	-7.324\\
-7.324	-12.207\\
-12.207	-8.545\\
-8.545	-7.324\\
-7.324	-8.545\\
-8.545	-7.324\\
-7.324	-4.883\\
-4.883	-6.104\\
-6.104	-9.766\\
-9.766	-7.324\\
-7.324	-12.207\\
-12.207	-6.104\\
-6.104	-8.545\\
-8.545	-6.104\\
-6.104	-12.207\\
-12.207	-7.324\\
-7.324	-14.648\\
-14.648	-17.09\\
-17.09	-13.428\\
-13.428	-9.766\\
-9.766	-9.766\\
-9.766	-12.207\\
-12.207	-13.428\\
-13.428	-8.545\\
-8.545	-13.428\\
-13.428	-10.986\\
-10.986	-4.883\\
-4.883	-3.662\\
-3.662	-12.207\\
-12.207	-10.986\\
-10.986	-8.545\\
-8.545	-10.986\\
-10.986	-12.207\\
-12.207	-9.766\\
-9.766	-7.324\\
-7.324	-9.766\\
-9.766	-12.207\\
-12.207	-8.545\\
-8.545	-10.986\\
-10.986	-9.766\\
-9.766	-6.104\\
-6.104	-8.545\\
-8.545	-4.883\\
-4.883	-8.545\\
-8.545	-10.986\\
-10.986	-14.648\\
-14.648	-13.428\\
-13.428	-13.428\\
-13.428	-8.545\\
-8.545	-6.104\\
-6.104	-7.324\\
-7.324	-8.545\\
-8.545	-12.207\\
-12.207	-12.207\\
-12.207	-14.648\\
-14.648	-10.986\\
-10.986	-8.545\\
-8.545	-12.207\\
-12.207	-18.311\\
-18.311	-12.207\\
-12.207	-14.648\\
-14.648	-14.648\\
-14.648	-9.766\\
-9.766	-9.766\\
-9.766	-12.207\\
-12.207	-9.766\\
-9.766	-10.986\\
-10.986	-14.648\\
-14.648	-9.766\\
-9.766	-12.207\\
-12.207	-10.986\\
-10.986	-10.986\\
-10.986	-6.104\\
-6.104	-9.766\\
-9.766	-8.545\\
-8.545	-9.766\\
-9.766	-9.766\\
-9.766	-9.766\\
-9.766	-6.104\\
-6.104	-3.662\\
-3.662	-2.441\\
-2.441	-6.104\\
-6.104	-10.986\\
-10.986	-12.207\\
-12.207	-10.986\\
-10.986	-8.545\\
-8.545	-12.207\\
-12.207	-8.545\\
-8.545	-9.766\\
-9.766	-6.104\\
-6.104	-8.545\\
-8.545	-7.324\\
-7.324	-7.324\\
-7.324	-7.324\\
-7.324	-7.324\\
-7.324	-6.104\\
-6.104	-6.104\\
-6.104	-4.883\\
-4.883	-10.986\\
-10.986	-7.324\\
-7.324	-9.766\\
-9.766	-7.324\\
-7.324	-12.207\\
-12.207	-9.766\\
-9.766	-14.648\\
-14.648	-8.545\\
-8.545	-13.428\\
-13.428	-12.207\\
-12.207	-7.324\\
-7.324	-10.986\\
-10.986	-9.766\\
-9.766	-10.986\\
-10.986	-12.207\\
-12.207	-13.428\\
-13.428	-7.324\\
-7.324	-13.428\\
-13.428	-14.648\\
-14.648	-14.648\\
-14.648	-9.766\\
-9.766	-9.766\\
-9.766	-10.986\\
-10.986	-7.324\\
-7.324	-9.766\\
-9.766	-7.324\\
-7.324	-8.545\\
-8.545	-7.324\\
-7.324	-13.428\\
-13.428	-15.869\\
-15.869	-15.869\\
-15.869	-14.648\\
-14.648	-15.869\\
-15.869	-8.545\\
-8.545	-7.324\\
-7.324	-6.104\\
-6.104	-4.883\\
-4.883	-4.883\\
-4.883	-8.545\\
-8.545	-10.986\\
-10.986	-13.428\\
-13.428	-12.207\\
-12.207	-7.324\\
-7.324	-12.207\\
-12.207	-15.869\\
-15.869	-15.869\\
-15.869	-13.428\\
-13.428	-20.752\\
-20.752	-17.09\\
-17.09	-10.986\\
-10.986	-6.104\\
-6.104	-7.324\\
-7.324	-10.986\\
-10.986	-13.428\\
-13.428	-17.09\\
-17.09	-12.207\\
-12.207	-10.986\\
-10.986	-6.104\\
-6.104	-9.766\\
-9.766	-7.324\\
-7.324	-7.324\\
-7.324	-4.883\\
-4.883	-4.883\\
-4.883	-3.662\\
-3.662	-3.662\\
-3.662	-6.104\\
-6.104	-10.986\\
-10.986	-8.545\\
-8.545	-9.766\\
-9.766	-8.545\\
-8.545	-8.545\\
-8.545	-12.207\\
-12.207	-7.324\\
-7.324	-12.207\\
-12.207	-13.428\\
-13.428	-8.545\\
-8.545	-10.986\\
-10.986	-8.545\\
-8.545	-10.986\\
-10.986	-15.869\\
-15.869	-10.986\\
-10.986	-8.545\\
-8.545	-9.766\\
-9.766	-10.986\\
-10.986	-7.324\\
-7.324	-8.545\\
-8.545	-13.428\\
-13.428	-14.648\\
-14.648	-15.869\\
-15.869	-20.752\\
-20.752	-15.869\\
-15.869	-14.648\\
-14.648	-10.986\\
-10.986	-10.986\\
-10.986	-17.09\\
-17.09	-13.428\\
-13.428	-10.986\\
-10.986	-14.648\\
-14.648	-9.766\\
-9.766	-7.324\\
-7.324	-9.766\\
-9.766	-6.104\\
-6.104	-3.662\\
-3.662	-6.104\\
-6.104	-4.883\\
-4.883	-4.883\\
-4.883	-4.883\\
-4.883	-8.545\\
-8.545	-9.766\\
-9.766	-12.207\\
-12.207	-12.207\\
-12.207	-12.207\\
-12.207	-14.648\\
-14.648	-14.648\\
-14.648	-10.986\\
-10.986	-10.986\\
-10.986	-9.766\\
-9.766	-10.986\\
-10.986	-14.648\\
-14.648	-10.986\\
-10.986	-10.986\\
-10.986	-9.766\\
-9.766	-10.986\\
-10.986	-14.648\\
-14.648	-12.207\\
-12.207	-12.207\\
-12.207	-13.428\\
-13.428	-9.766\\
-9.766	-6.104\\
-6.104	-4.883\\
-4.883	-7.324\\
-7.324	-7.324\\
-7.324	-7.324\\
-7.324	-6.104\\
-6.104	-8.545\\
-8.545	-14.648\\
-14.648	-18.311\\
-18.311	-12.207\\
-12.207	-14.648\\
-14.648	-13.428\\
-13.428	-10.986\\
-10.986	-10.986\\
-10.986	-10.986\\
-10.986	-10.986\\
-10.986	-13.428\\
-13.428	-14.648\\
-14.648	-19.531\\
-19.531	-17.09\\
-17.09	-15.869\\
-15.869	-19.531\\
-19.531	-13.428\\
-13.428	-15.869\\
-15.869	-13.428\\
-13.428	-10.986\\
-10.986	-13.428\\
-13.428	-17.09\\
-17.09	-7.324\\
-7.324	-10.986\\
-10.986	-7.324\\
-7.324	-8.545\\
-8.545	-10.986\\
-10.986	-7.324\\
-7.324	-8.545\\
-8.545	-15.869\\
-15.869	-20.752\\
-20.752	-21.973\\
-21.973	-19.531\\
-19.531	-15.869\\
-15.869	-19.531\\
-19.531	-21.973\\
-21.973	-17.09\\
-17.09	-18.311\\
-18.311	-17.09\\
-17.09	-12.207\\
-12.207	-9.766\\
-9.766	-13.428\\
-13.428	-10.986\\
-10.986	-6.104\\
-6.104	-9.766\\
-9.766	-8.545\\
-8.545	-8.545\\
-8.545	-8.545\\
-8.545	-9.766\\
-9.766	-12.207\\
-12.207	-8.545\\
-8.545	-7.324\\
-7.324	-8.545\\
-8.545	-10.986\\
-10.986	-10.986\\
-10.986	-9.766\\
-9.766	-9.766\\
-9.766	-14.648\\
-14.648	-13.428\\
-13.428	-9.766\\
-9.766	-13.428\\
-13.428	-13.428\\
-13.428	-10.986\\
-10.986	-8.545\\
-8.545	-7.324\\
-7.324	-4.883\\
-4.883	-6.104\\
-6.104	-8.545\\
-8.545	-6.104\\
-6.104	-4.883\\
-4.883	-4.883\\
-4.883	-7.324\\
-7.324	-7.324\\
-7.324	-12.207\\
-12.207	-14.648\\
-14.648	-13.428\\
-13.428	-13.428\\
-13.428	-10.986\\
-10.986	-4.883\\
-4.883	-6.104\\
-6.104	-7.324\\
-7.324	-8.545\\
-8.545	-8.545\\
-8.545	-10.986\\
-10.986	-14.648\\
-14.648	-14.648\\
-14.648	-9.766\\
-9.766	-14.648\\
-14.648	-13.428\\
-13.428	-6.104\\
-6.104	-8.545\\
-8.545	-7.324\\
-7.324	-10.986\\
-10.986	-17.09\\
-17.09	-12.207\\
-12.207	-9.766\\
-9.766	-8.545\\
-8.545	-10.986\\
-10.986	-12.207\\
-12.207	-19.531\\
-19.531	-15.869\\
-15.869	-17.09\\
-17.09	-12.207\\
-12.207	-7.324\\
-7.324	-8.545\\
-8.545	-4.883\\
-4.883	-6.104\\
-6.104	-8.545\\
-8.545	-3.662\\
-3.662	-8.545\\
-8.545	-13.428\\
-13.428	-15.869\\
-15.869	-20.752\\
-20.752	-18.311\\
-18.311	-9.766\\
-9.766	-10.986\\
-10.986	-8.545\\
-8.545	-13.428\\
-13.428	-15.869\\
-15.869	-15.869\\
-15.869	-15.869\\
-15.869	-9.766\\
-9.766	-10.986\\
-10.986	-14.648\\
-14.648	-13.428\\
-13.428	-7.324\\
-7.324	-6.104\\
-6.104	-4.883\\
-4.883	-3.662\\
-3.662	-4.883\\
-4.883	-3.662\\
-3.662	-8.545\\
-8.545	-8.545\\
-8.545	-6.104\\
-6.104	-8.545\\
-8.545	-6.104\\
-6.104	-3.662\\
-3.662	-2.441\\
-2.441	-6.104\\
-6.104	-8.545\\
-8.545	-8.545\\
-8.545	-7.324\\
-7.324	-10.986\\
-10.986	-14.648\\
-14.648	-19.531\\
-19.531	-21.973\\
-21.973	-20.752\\
-20.752	-19.531\\
-19.531	-13.428\\
-13.428	-10.986\\
-10.986	-10.986\\
-10.986	-12.207\\
-12.207	-10.986\\
-10.986	-7.324\\
-7.324	-7.324\\
-7.324	-7.324\\
-7.324	-8.545\\
-8.545	-10.986\\
-10.986	-7.324\\
-7.324	-2.441\\
-2.441	-4.883\\
-4.883	-6.104\\
-6.104	-7.324\\
-7.324	-6.104\\
-6.104	-8.545\\
-8.545	-6.104\\
-6.104	-14.648\\
-14.648	-10.986\\
-10.986	-12.207\\
-12.207	-20.752\\
-20.752	-20.752\\
-20.752	-15.869\\
-15.869	-13.428\\
-13.428	-8.545\\
-8.545	-7.324\\
-7.324	-6.104\\
-6.104	-10.986\\
-10.986	-7.324\\
-7.324	-10.986\\
-10.986	-8.545\\
-8.545	-10.986\\
-10.986	-4.883\\
-4.883	-7.324\\
-7.324	-9.766\\
-9.766	-3.662\\
-3.662	-3.662\\
-3.662	-6.104\\
-6.104	-2.441\\
-2.441	-3.662\\
-3.662	-3.662\\
-3.662	-2.441\\
-2.441	-6.104\\
-6.104	-6.104\\
-6.104	-6.104\\
-6.104	-7.324\\
-7.324	-10.986\\
-10.986	-7.324\\
-7.324	-7.324\\
-7.324	-12.207\\
-12.207	-14.648\\
-14.648	-9.766\\
-9.766	-10.986\\
-10.986	-9.766\\
-9.766	-2.441\\
-2.441	-9.766\\
-9.766	-8.545\\
-8.545	-12.207\\
-12.207	-13.428\\
-13.428	-15.869\\
-15.869	-10.986\\
-10.986	-8.545\\
-8.545	-7.324\\
-7.324	-8.545\\
-8.545	-6.104\\
-6.104	-4.883\\
-4.883	-4.883\\
-4.883	-4.883\\
-4.883	-3.662\\
-3.662	-4.883\\
-4.883	-3.662\\
-3.662	-8.545\\
-8.545	-8.545\\
-8.545	-4.883\\
-4.883	-7.324\\
-7.324	-14.648\\
-14.648	-6.104\\
-6.104	-9.766\\
-9.766	-17.09\\
-17.09	-9.766\\
-9.766	-6.104\\
-6.104	-9.766\\
-9.766	-10.986\\
-10.986	-8.545\\
-8.545	-3.662\\
-3.662	-6.104\\
-6.104	-1.221\\
-1.221	-2.441\\
-2.441	-6.104\\
-6.104	-6.104\\
-6.104	-7.324\\
-7.324	-6.104\\
-6.104	-8.545\\
-8.545	-7.324\\
-7.324	-3.662\\
-3.662	-1.221\\
-1.221	-6.104\\
-6.104	-4.883\\
-4.883	-7.324\\
-7.324	-10.986\\
-10.986	-7.324\\
-7.324	-8.545\\
-8.545	-9.766\\
-9.766	-12.207\\
-12.207	-14.648\\
-14.648	-17.09\\
-17.09	-17.09\\
-17.09	-12.207\\
-12.207	-10.986\\
-10.986	-12.207\\
-12.207	-12.207\\
-12.207	-13.428\\
-13.428	-18.311\\
-18.311	-17.09\\
-17.09	-13.428\\
-13.428	-12.207\\
-12.207	-12.207\\
-12.207	-12.207\\
-12.207	-13.428\\
-13.428	-10.986\\
-10.986	-10.986\\
-10.986	-12.207\\
-12.207	-13.428\\
-13.428	-12.207\\
-12.207	-10.986\\
-10.986	-10.986\\
-10.986	-9.766\\
-9.766	-9.766\\
-9.766	-7.324\\
-7.324	-12.207\\
-12.207	-8.545\\
-8.545	-6.104\\
-6.104	-8.545\\
-8.545	-8.545\\
-8.545	-3.662\\
-3.662	-7.324\\
-7.324	-7.324\\
-7.324	-3.662\\
-3.662	-6.104\\
-6.104	-4.883\\
-4.883	-2.441\\
-2.441	-6.104\\
-6.104	-3.662\\
-3.662	-3.662\\
-3.662	-8.545\\
-8.545	-3.662\\
-3.662	-8.545\\
-8.545	-4.883\\
-4.883	-7.324\\
-7.324	-8.545\\
-8.545	-7.324\\
-7.324	-13.428\\
-13.428	-4.883\\
-4.883	-8.545\\
-8.545	-10.986\\
-10.986	-10.986\\
-10.986	-6.104\\
-6.104	-7.324\\
-7.324	-12.207\\
-12.207	-3.662\\
-3.662	-6.104\\
-6.104	-8.545\\
-8.545	-6.104\\
-6.104	-9.766\\
-9.766	-1.221\\
-1.221	-3.662\\
-3.662	-4.883\\
-4.883	-4.883\\
-4.883	-7.324\\
-7.324	-3.662\\
-3.662	-8.545\\
-8.545	-10.986\\
-10.986	-9.766\\
-9.766	-8.545\\
-8.545	-14.648\\
-14.648	-10.986\\
-10.986	-10.986\\
-10.986	-18.311\\
-18.311	-18.311\\
-18.311	-13.428\\
-13.428	-17.09\\
-17.09	-13.428\\
-13.428	-17.09\\
-17.09	-13.428\\
-13.428	-4.883\\
-4.883	-7.324\\
-7.324	-12.207\\
-12.207	-10.986\\
-10.986	-10.986\\
-10.986	-8.545\\
-8.545	-4.883\\
-4.883	-6.104\\
-6.104	-9.766\\
-9.766	-15.869\\
-15.869	-13.428\\
-13.428	-9.766\\
-9.766	-9.766\\
-9.766	-7.324\\
-7.324	-12.207\\
-12.207	-15.869\\
-15.869	-14.648\\
-14.648	-13.428\\
-13.428	-13.428\\
-13.428	-19.531\\
-19.531	-17.09\\
-17.09	-19.531\\
-19.531	-20.752\\
-20.752	-15.869\\
-15.869	-18.311\\
-18.311	-15.869\\
-15.869	-8.545\\
-8.545	-8.545\\
-8.545	-10.986\\
-10.986	-13.428\\
-13.428	-13.428\\
-13.428	-10.986\\
-10.986	-9.766\\
-9.766	-13.428\\
-13.428	-9.766\\
-9.766	-12.207\\
-12.207	-9.766\\
-9.766	-14.648\\
-14.648	-18.311\\
-18.311	-10.986\\
-10.986	-8.545\\
-8.545	-7.324\\
-7.324	-7.324\\
-7.324	-12.207\\
-12.207	-3.662\\
-3.662	-4.883\\
-4.883	-13.428\\
-13.428	-10.986\\
-10.986	-10.986\\
-10.986	-9.766\\
-9.766	-8.545\\
-8.545	-3.662\\
-3.662	-6.104\\
-6.104	-4.883\\
-4.883	-8.545\\
-8.545	-7.324\\
-7.324	-7.324\\
-7.324	-13.428\\
-13.428	-13.428\\
-13.428	-15.869\\
-15.869	-14.648\\
-14.648	-17.09\\
-17.09	-14.648\\
-14.648	-9.766\\
-9.766	-8.545\\
-8.545	-8.545\\
-8.545	-9.766\\
-9.766	-9.766\\
-9.766	-6.104\\
-6.104	-4.883\\
-4.883	-10.986\\
-10.986	-12.207\\
-12.207	-9.766\\
-9.766	-14.648\\
-14.648	-17.09\\
-17.09	-12.207\\
-12.207	-8.545\\
-8.545	-7.324\\
-7.324	-7.324\\
-7.324	-9.766\\
-9.766	-8.545\\
-8.545	-9.766\\
-9.766	-15.869\\
-15.869	-12.207\\
-12.207	-12.207\\
-12.207	-12.207\\
-12.207	-10.986\\
-10.986	-10.986\\
-10.986	-10.986\\
-10.986	-7.324\\
-7.324	-12.207\\
-12.207	-14.648\\
-14.648	-14.648\\
-14.648	-12.207\\
-12.207	-10.986\\
-10.986	-17.09\\
-17.09	-10.986\\
-10.986	-8.545\\
-8.545	-10.986\\
-10.986	-7.324\\
-7.324	-9.766\\
-9.766	-10.986\\
-10.986	-8.545\\
-8.545	-9.766\\
-9.766	-15.869\\
-15.869	-18.311\\
-18.311	-10.986\\
-10.986	-12.207\\
-12.207	-13.428\\
-13.428	-14.648\\
-14.648	-9.766\\
-9.766	-9.766\\
-9.766	-8.545\\
-8.545	-10.986\\
-10.986	-8.545\\
-8.545	-15.869\\
-15.869	-20.752\\
-20.752	-23.193\\
-23.193	-21.973\\
-21.973	-14.648\\
-14.648	-13.428\\
-13.428	-12.207\\
-12.207	-12.207\\
-12.207	-8.545\\
-8.545	-6.104\\
-6.104	-7.324\\
-7.324	-7.324\\
-7.324	-3.662\\
-3.662	-4.883\\
-4.883	-10.986\\
-10.986	-7.324\\
-7.324	-3.662\\
-3.662	-3.662\\
-3.662	-8.545\\
-8.545	-10.986\\
-10.986	-9.766\\
-9.766	-6.104\\
-6.104	-12.207\\
-12.207	-6.104\\
-6.104	-7.324\\
-7.324	-8.545\\
-8.545	-6.104\\
-6.104	-6.104\\
-6.104	-4.883\\
-4.883	-6.104\\
-6.104	-8.545\\
-8.545	-9.766\\
-9.766	-8.545\\
-8.545	-6.104\\
-6.104	-4.883\\
-4.883	-4.883\\
-4.883	-8.545\\
-8.545	-9.766\\
-9.766	-8.545\\
-8.545	-8.545\\
-8.545	-6.104\\
-6.104	-7.324\\
-7.324	-8.545\\
-8.545	-14.648\\
-14.648	-17.09\\
-17.09	-13.428\\
-13.428	-7.324\\
-7.324	-9.766\\
-9.766	-9.766\\
-9.766	-8.545\\
-8.545	-7.324\\
-7.324	-8.545\\
-8.545	-8.545\\
-8.545	-7.324\\
-7.324	-3.662\\
-3.662	-3.662\\
-3.662	-4.883\\
-4.883	-7.324\\
-7.324	-9.766\\
-9.766	-12.207\\
-12.207	-10.986\\
-10.986	-14.648\\
-14.648	-13.428\\
-13.428	-17.09\\
-17.09	-23.193\\
-23.193	-17.09\\
-17.09	-14.648\\
-14.648	-17.09\\
-17.09	-18.311\\
-18.311	-19.531\\
-19.531	-18.311\\
-18.311	-8.545\\
-8.545	-4.883\\
-4.883	-6.104\\
-6.104	-3.662\\
-3.662	-3.662\\
-3.662	-3.662\\
-3.662	-6.104\\
-6.104	-6.104\\
-6.104	-8.545\\
-8.545	-4.883\\
-4.883	-6.104\\
-6.104	-6.104\\
-6.104	-8.545\\
-8.545	-7.324\\
-7.324	-7.324\\
-7.324	-6.104\\
-6.104	-6.104\\
-6.104	-6.104\\
-6.104	-7.324\\
-7.324	-9.766\\
-9.766	-4.883\\
-4.883	-3.662\\
-3.662	-6.104\\
-6.104	-4.883\\
-4.883	-8.545\\
-8.545	-9.766\\
-9.766	-7.324\\
-7.324	-9.766\\
-9.766	-7.324\\
-7.324	-9.766\\
-9.766	-14.648\\
-14.648	-7.324\\
-7.324	-7.324\\
-7.324	-10.986\\
-10.986	-14.648\\
-14.648	-14.648\\
-14.648	-14.648\\
-14.648	-10.986\\
-10.986	-12.207\\
-12.207	-12.207\\
-12.207	-13.428\\
-13.428	-12.207\\
-12.207	-13.428\\
-13.428	-15.869\\
-15.869	-13.428\\
-13.428	-21.973\\
-21.973	-20.752\\
-20.752	-12.207\\
-12.207	-8.545\\
-8.545	-8.545\\
-8.545	-8.545\\
-8.545	-12.207\\
-12.207	-12.207\\
-12.207	-13.428\\
-13.428	-15.869\\
-15.869	-10.986\\
-10.986	-9.766\\
-9.766	-12.207\\
-12.207	-9.766\\
-9.766	-9.766\\
-9.766	-3.662\\
-3.662	-9.766\\
-9.766	-12.207\\
-12.207	-10.986\\
-10.986	-14.648\\
-14.648	-13.428\\
-13.428	-8.545\\
-8.545	-8.545\\
-8.545	-12.207\\
-12.207	-12.207\\
-12.207	-15.869\\
-15.869	-19.531\\
-19.531	-20.752\\
-20.752	-18.311\\
-18.311	-17.09\\
-17.09	-10.986\\
-10.986	-8.545\\
-8.545	-13.428\\
-13.428	-19.531\\
-19.531	-10.986\\
-10.986	-12.207\\
-12.207	-21.973\\
-21.973	-21.973\\
-21.973	-19.531\\
-19.531	-9.766\\
-9.766	-6.104\\
-6.104	-9.766\\
-9.766	-10.986\\
-10.986	-7.324\\
-7.324	-4.883\\
-4.883	-6.104\\
-6.104	-6.104\\
-6.104	-8.545\\
-8.545	-7.324\\
-7.324	-6.104\\
-6.104	-8.545\\
-8.545	-9.766\\
-9.766	-12.207\\
-12.207	-14.648\\
-14.648	-14.648\\
-14.648	-9.766\\
-9.766	-12.207\\
-12.207	-9.766\\
-9.766	-8.545\\
-8.545	-7.324\\
-7.324	-7.324\\
-7.324	-8.545\\
-8.545	-9.766\\
-9.766	-6.104\\
-6.104	-3.662\\
-3.662	-3.662\\
-3.662	-4.883\\
-4.883	-7.324\\
-7.324	-7.324\\
-7.324	-6.104\\
-6.104	-10.986\\
-10.986	-8.545\\
-8.545	-9.766\\
-9.766	-6.104\\
-6.104	-4.883\\
-4.883	-7.324\\
-7.324	-10.986\\
-10.986	-9.766\\
-9.766	-9.766\\
-9.766	-10.986\\
-10.986	-7.324\\
-7.324	-7.324\\
-7.324	-8.545\\
-8.545	-7.324\\
-7.324	-7.324\\
-7.324	-12.207\\
-12.207	-15.869\\
-15.869	-12.207\\
-12.207	-12.207\\
-12.207	-15.869\\
-15.869	-13.428\\
-13.428	-8.545\\
-8.545	-7.324\\
-7.324	-4.883\\
-4.883	-4.883\\
-4.883	-3.662\\
-3.662	-3.662\\
-3.662	-10.986\\
-10.986	-6.104\\
-6.104	-4.883\\
-4.883	-6.104\\
-6.104	-8.545\\
-8.545	-7.324\\
-7.324	-6.104\\
-6.104	-4.883\\
-4.883	-6.104\\
-6.104	-10.986\\
-10.986	-10.986\\
-10.986	-7.324\\
-7.324	-4.883\\
-4.883	-9.766\\
-9.766	-8.545\\
-8.545	-9.766\\
-9.766	-10.986\\
-10.986	-13.428\\
-13.428	-13.428\\
-13.428	-12.207\\
-12.207	-8.545\\
-8.545	-7.324\\
-7.324	-6.104\\
-6.104	-9.766\\
-9.766	-6.104\\
-6.104	-6.104\\
-6.104	-8.545\\
-8.545	-12.207\\
-12.207	-9.766\\
-9.766	-10.986\\
-10.986	-15.869\\
-15.869	-10.986\\
-10.986	-10.986\\
-10.986	-14.648\\
-14.648	-12.207\\
-12.207	-15.869\\
-15.869	-13.428\\
-13.428	-21.973\\
-21.973	-21.973\\
-21.973	-15.869\\
-15.869	-19.531\\
-19.531	-21.973\\
-21.973	-20.752\\
-20.752	-23.193\\
-23.193	-21.973\\
-21.973	-25.635\\
-25.635	-20.752\\
-20.752	-13.428\\
-13.428	-9.766\\
-9.766	-12.207\\
-12.207	-8.545\\
-8.545	-7.324\\
-7.324	-12.207\\
-12.207	-14.648\\
-14.648	-9.766\\
-9.766	-7.324\\
-7.324	-9.766\\
-9.766	-7.324\\
-7.324	-8.545\\
-8.545	-6.104\\
-6.104	-9.766\\
-9.766	-4.883\\
-4.883	-4.883\\
-4.883	-1.221\\
-1.221	-3.662\\
-3.662	-2.441\\
-2.441	-3.662\\
-3.662	-4.883\\
-4.883	-4.883\\
-4.883	-4.883\\
-4.883	-3.662\\
-3.662	-8.545\\
-8.545	-9.766\\
-9.766	-6.104\\
-6.104	-12.207\\
-12.207	-14.648\\
-14.648	-8.545\\
-8.545	-2.441\\
-2.441	-3.662\\
-3.662	-2.441\\
-2.441	-6.104\\
-6.104	-7.324\\
-7.324	-8.545\\
-8.545	-3.662\\
-3.662	-7.324\\
-7.324	-14.648\\
-14.648	-14.648\\
-14.648	-9.766\\
-9.766	-7.324\\
-7.324	-12.207\\
-12.207	-13.428\\
-13.428	-14.648\\
-14.648	-7.324\\
-7.324	-4.883\\
-4.883	-6.104\\
-6.104	-6.104\\
-6.104	-3.662\\
-3.662	-3.662\\
-3.662	-8.545\\
-8.545	-14.648\\
-14.648	-10.986\\
-10.986	-6.104\\
-6.104	-3.662\\
-3.662	-4.883\\
-4.883	-7.324\\
-7.324	-4.883\\
-4.883	-3.662\\
-3.662	-6.104\\
-6.104	-6.104\\
-6.104	-12.207\\
-12.207	-9.766\\
-9.766	-10.986\\
-10.986	-12.207\\
-12.207	-10.986\\
-10.986	-12.207\\
-12.207	-10.986\\
-10.986	-13.428\\
-13.428	-18.311\\
-18.311	-15.869\\
-15.869	-7.324\\
-7.324	-4.883\\
-4.883	-8.545\\
-8.545	-14.648\\
-14.648	-10.986\\
-10.986	-4.883\\
-4.883	-10.986\\
-10.986	-9.766\\
-9.766	-17.09\\
-17.09	-20.752\\
-20.752	-19.531\\
-19.531	-19.531\\
-19.531	-15.869\\
-15.869	-14.648\\
-14.648	-7.324\\
-7.324	-9.766\\
-9.766	-8.545\\
-8.545	-10.986\\
-10.986	-10.986\\
-10.986	-12.207\\
-12.207	-7.324\\
-7.324	-10.986\\
-10.986	-8.545\\
-8.545	-6.104\\
-6.104	-3.662\\
-3.662	-4.883\\
-4.883	-4.883\\
-4.883	-4.883\\
-4.883	-3.662\\
-3.662	-6.104\\
-6.104	-9.766\\
-9.766	-6.104\\
-6.104	-6.104\\
-6.104	-6.104\\
-6.104	-13.428\\
-13.428	-6.104\\
-6.104	-3.662\\
-3.662	-1.221\\
-1.221	-6.104\\
-6.104	-10.986\\
-10.986	-14.648\\
-14.648	-21.973\\
-21.973	-23.193\\
-23.193	-24.414\\
-24.414	-17.09\\
-17.09	-18.311\\
-18.311	-20.752\\
-20.752	-18.311\\
-18.311	-17.09\\
-17.09	-18.311\\
-18.311	-20.752\\
-20.752	-20.752\\
-20.752	-28.076\\
-28.076	-19.531\\
-19.531	-12.207\\
-12.207	-7.324\\
-7.324	-6.104\\
-6.104	-6.104\\
-6.104	-6.104\\
-6.104	-4.883\\
-4.883	-12.207\\
-12.207	-7.324\\
-7.324	-10.986\\
-10.986	-10.986\\
-10.986	-7.324\\
-7.324	-12.207\\
-12.207	-12.207\\
-12.207	-12.207\\
-12.207	-6.104\\
-6.104	-4.883\\
-4.883	-6.104\\
-6.104	-8.545\\
-8.545	-13.428\\
-13.428	-19.531\\
-19.531	-18.311\\
-18.311	-18.311\\
-18.311	-23.193\\
-23.193	-29.297\\
-29.297	-21.973\\
-21.973	-14.648\\
-14.648	-9.766\\
-9.766	-6.104\\
-6.104	-10.986\\
-10.986	-8.545\\
-8.545	-9.766\\
-9.766	-7.324\\
-7.324	-8.545\\
-8.545	-7.324\\
-7.324	-9.766\\
-9.766	-9.766\\
-9.766	-10.986\\
-10.986	-7.324\\
-7.324	-6.104\\
-6.104	-3.662\\
-3.662	-3.662\\
-3.662	-1.221\\
-1.221	-3.662\\
-3.662	-6.104\\
-6.104	-7.324\\
-7.324	-7.324\\
-7.324	-12.207\\
-12.207	-7.324\\
-7.324	-12.207\\
-12.207	-18.311\\
-18.311	-12.207\\
-12.207	-12.207\\
-12.207	-17.09\\
-17.09	-17.09\\
-17.09	-12.207\\
-12.207	-14.648\\
-14.648	-8.545\\
-8.545	-12.207\\
-12.207	-17.09\\
-17.09	-24.414\\
-24.414	-29.297\\
-29.297	-23.193\\
-23.193	-18.311\\
-18.311	-15.869\\
-15.869	-18.311\\
-18.311	-19.531\\
-19.531	-13.428\\
-13.428	-12.207\\
-12.207	-8.545\\
-8.545	-14.648\\
-14.648	-12.207\\
-12.207	-8.545\\
-8.545	-8.545\\
-8.545	-7.324\\
-7.324	-12.207\\
-12.207	-14.648\\
-14.648	-15.869\\
-15.869	-9.766\\
-9.766	-9.766\\
-9.766	-12.207\\
-12.207	-7.324\\
-7.324	-7.324\\
-7.324	-4.883\\
-4.883	-4.883\\
-4.883	-3.662\\
-3.662	-4.883\\
-4.883	-10.986\\
-10.986	-8.545\\
-8.545	-10.986\\
-10.986	-7.324\\
-7.324	-6.104\\
-6.104	-3.662\\
-3.662	-8.545\\
-8.545	-8.545\\
-8.545	-9.766\\
-9.766	-7.324\\
-7.324	-4.883\\
-4.883	-8.545\\
-8.545	-14.648\\
-14.648	-8.545\\
-8.545	-2.441\\
-2.441	-10.986\\
-10.986	-9.766\\
-9.766	-9.766\\
-9.766	-7.324\\
-7.324	-3.662\\
-3.662	-10.986\\
-10.986	-10.986\\
-10.986	-6.104\\
-6.104	-3.662\\
-3.662	-10.986\\
-10.986	-13.428\\
-13.428	-12.207\\
-12.207	-6.104\\
-6.104	-7.324\\
-7.324	-12.207\\
-12.207	-14.648\\
-14.648	-12.207\\
-12.207	-18.311\\
-18.311	-21.973\\
-21.973	-13.428\\
-13.428	-13.428\\
-13.428	-17.09\\
-17.09	-10.986\\
-10.986	-17.09\\
-17.09	-17.09\\
-17.09	-13.428\\
-13.428	-6.104\\
-6.104	-9.766\\
-9.766	-19.531\\
-19.531	-21.973\\
-21.973	-15.869\\
-15.869	-15.869\\
-15.869	-19.531\\
-19.531	-14.648\\
-14.648	-10.986\\
-10.986	-9.766\\
-9.766	-12.207\\
-12.207	-12.207\\
-12.207	-12.207\\
-12.207	-13.428\\
-13.428	-8.545\\
-8.545	-10.986\\
-10.986	-13.428\\
-13.428	-8.545\\
-8.545	-7.324\\
-7.324	-6.104\\
-6.104	-6.104\\
-6.104	-3.662\\
-3.662	-7.324\\
-7.324	-7.324\\
-7.324	-8.545\\
-8.545	-12.207\\
-12.207	-15.869\\
-15.869	-10.986\\
-10.986	-12.207\\
-12.207	-8.545\\
-8.545	-13.428\\
-13.428	-14.648\\
-14.648	-8.545\\
-8.545	-3.662\\
-3.662	-10.986\\
-10.986	-13.428\\
-13.428	-13.428\\
-13.428	-18.311\\
-18.311	-24.414\\
-24.414	-15.869\\
-15.869	-15.869\\
-15.869	-17.09\\
-17.09	-12.207\\
-12.207	-9.766\\
-9.766	-12.207\\
-12.207	-14.648\\
-14.648	-14.648\\
-14.648	-13.428\\
-13.428	-9.766\\
-9.766	-8.545\\
-8.545	-6.104\\
-6.104	-10.986\\
-10.986	-7.324\\
-7.324	-7.324\\
-7.324	-14.648\\
-14.648	-15.869\\
-15.869	-8.545\\
-8.545	-7.324\\
-7.324	-12.207\\
-12.207	-7.324\\
-7.324	-6.104\\
-6.104	-7.324\\
-7.324	-9.766\\
-9.766	-6.104\\
-6.104	-3.662\\
-3.662	-3.662\\
-3.662	-2.441\\
-2.441	-6.104\\
-6.104	-9.766\\
-9.766	-9.766\\
-9.766	-13.428\\
-13.428	-15.869\\
-15.869	-10.986\\
-10.986	-8.545\\
-8.545	-15.869\\
-15.869	-20.752\\
-20.752	-15.869\\
-15.869	-14.648\\
-14.648	-23.193\\
-23.193	-17.09\\
-17.09	-15.869\\
-15.869	-13.428\\
-13.428	-13.428\\
-13.428	-15.869\\
-15.869	-10.986\\
-10.986	-9.766\\
-9.766	-17.09\\
-17.09	-18.311\\
-18.311	-19.531\\
-19.531	-12.207\\
-12.207	-4.883\\
-4.883	-7.324\\
-7.324	-10.986\\
-10.986	-2.441\\
-2.441	-3.662\\
-3.662	-8.545\\
-8.545	-8.545\\
-8.545	-13.428\\
-13.428	-8.545\\
-8.545	-6.104\\
-6.104	-6.104\\
-6.104	-3.662\\
-3.662	-6.104\\
-6.104	-3.662\\
-3.662	-4.883\\
-4.883	-8.545\\
-8.545	-6.104\\
-6.104	-1.221\\
-1.221	-4.883\\
-4.883	-6.104\\
-6.104	-6.104\\
-6.104	-3.662\\
-3.662	-3.662\\
-3.662	-3.662\\
-3.662	-3.662\\
-3.662	-8.545\\
-8.545	-10.986\\
-10.986	-14.648\\
-14.648	-19.531\\
-19.531	-15.869\\
-15.869	-7.324\\
-7.324	-7.324\\
-7.324	-6.104\\
-6.104	-4.883\\
-4.883	-3.662\\
-3.662	-7.324\\
-7.324	-8.545\\
-8.545	-6.104\\
-6.104	-8.545\\
-8.545	-8.545\\
-8.545	-9.766\\
-9.766	-9.766\\
-9.766	-10.986\\
-10.986	-12.207\\
-12.207	-14.648\\
-14.648	-9.766\\
-9.766	-6.104\\
-6.104	-6.104\\
-6.104	-9.766\\
-9.766	-4.883\\
-4.883	-7.324\\
-7.324	-2.441\\
-2.441	-6.104\\
-6.104	-6.104\\
-6.104	-3.662\\
-3.662	-3.662\\
-3.662	-6.104\\
-6.104	-7.324\\
-7.324	-10.986\\
-10.986	-13.428\\
-13.428	-8.545\\
-8.545	-2.441\\
-2.441	-6.104\\
-6.104	-9.766\\
-9.766	-4.883\\
-4.883	-1.221\\
-1.221	-12.207\\
-12.207	-7.324\\
-7.324	-14.648\\
-14.648	-18.311\\
-18.311	-13.428\\
-13.428	-8.545\\
-8.545	-8.545\\
};
\addplot [color=mycolor2, line width=2.0pt, forget plot]
  table[row sep=crcr]{%
-12.207	-11.6408886335849\\
-10.986	-10.476513683015\\
-15.869	-15.1330598612566\\
-10.986	-10.476513683015\\
-12.207	-11.6408886335849\\
-13.428	-12.8052635841548\\
-15.869	-15.1330598612566\\
-14.648	-13.9686849106866\\
-7.324	-6.98434245534331\\
-8.545	-8.14871740591324\\
-10.986	-10.476513683015\\
-9.766	-9.31309235648318\\
-6.104	-5.82092112881152\\
-2.441	-2.32779627710172\\
-3.662	-3.49217122767166\\
-2.441	-2.32779627710172\\
-12.207	-11.6408886335849\\
-13.428	-12.8052635841548\\
-12.207	-11.6408886335849\\
-10.986	-10.476513683015\\
-6.104	-5.82092112881152\\
-1.221	-1.16437495056993\\
-6.104	-5.82092112881152\\
-10.986	-10.476513683015\\
-12.207	-11.6408886335849\\
-8.545	-8.14871740591324\\
-14.648	-13.9686849106866\\
-9.766	-9.31309235648318\\
-13.428	-12.8052635841548\\
-12.207	-11.6408886335849\\
-9.766	-9.31309235648318\\
-8.545	-8.14871740591324\\
-12.207	-11.6408886335849\\
-10.986	-10.476513683015\\
-14.648	-13.9686849106866\\
-15.869	-15.1330598612566\\
-9.766	-9.31309235648318\\
-8.545	-8.14871740591324\\
-6.104	-5.82092112881152\\
-7.324	-6.98434245534331\\
-4.883	-4.65654617824159\\
-9.766	-9.31309235648318\\
-10.986	-10.476513683015\\
-8.545	-8.14871740591324\\
-13.428	-12.8052635841548\\
-15.869	-15.1330598612566\\
-8.545	-8.14871740591324\\
-6.104	-5.82092112881152\\
-9.766	-9.31309235648318\\
-6.104	-5.82092112881152\\
-4.883	-4.65654617824159\\
-7.324	-6.98434245534331\\
-3.662	-3.49217122767166\\
-6.104	-5.82092112881152\\
-8.545	-8.14871740591324\\
-10.986	-10.476513683015\\
-9.766	-9.31309235648318\\
-13.428	-12.8052635841548\\
-10.986	-10.476513683015\\
-12.207	-11.6408886335849\\
-10.986	-10.476513683015\\
-7.324	-6.98434245534331\\
-8.545	-8.14871740591324\\
-14.648	-13.9686849106866\\
-20.752	-19.7896060394981\\
-19.531	-18.6252310889282\\
-13.428	-12.8052635841548\\
-15.869	-15.1330598612566\\
-24.414	-23.2817772671698\\
-23.193	-22.1174023165999\\
-15.869	-15.1330598612566\\
-21.973	-20.9539809900681\\
-28.076	-26.7739484948415\\
-23.193	-22.1174023165999\\
-17.09	-16.2974348118265\\
-14.648	-13.9686849106866\\
-13.428	-12.8052635841548\\
-9.766	-9.31309235648318\\
-10.986	-10.476513683015\\
-12.207	-11.6408886335849\\
-4.883	-4.65654617824159\\
-7.324	-6.98434245534331\\
-10.986	-10.476513683015\\
-9.766	-9.31309235648318\\
-10.986	-10.476513683015\\
-13.428	-12.8052635841548\\
-14.648	-13.9686849106866\\
-15.869	-15.1330598612566\\
-14.648	-13.9686849106866\\
-17.09	-16.2974348118265\\
-12.207	-11.6408886335849\\
-10.986	-10.476513683015\\
-8.545	-8.14871740591324\\
-9.766	-9.31309235648318\\
-12.207	-11.6408886335849\\
-8.545	-8.14871740591324\\
-9.766	-9.31309235648318\\
-7.324	-6.98434245534331\\
-6.104	-5.82092112881152\\
-8.545	-8.14871740591324\\
-6.104	-5.82092112881152\\
-4.883	-4.65654617824159\\
-8.545	-8.14871740591324\\
-13.428	-12.8052635841548\\
-12.207	-11.6408886335849\\
-10.986	-10.476513683015\\
-7.324	-6.98434245534331\\
-8.545	-8.14871740591324\\
-19.531	-18.6252310889282\\
-14.648	-13.9686849106866\\
-10.986	-10.476513683015\\
-17.09	-16.2974348118265\\
-12.207	-11.6408886335849\\
-6.104	-5.82092112881152\\
-9.766	-9.31309235648318\\
-8.545	-8.14871740591324\\
-14.648	-13.9686849106866\\
-15.869	-15.1330598612566\\
-19.531	-18.6252310889282\\
-14.648	-13.9686849106866\\
-13.428	-12.8052635841548\\
-14.648	-13.9686849106866\\
-10.986	-10.476513683015\\
-7.324	-6.98434245534331\\
-6.104	-5.82092112881152\\
-8.545	-8.14871740591324\\
-10.986	-10.476513683015\\
-15.869	-15.1330598612566\\
-14.648	-13.9686849106866\\
-8.545	-8.14871740591324\\
-9.766	-9.31309235648318\\
-6.104	-5.82092112881152\\
-2.441	-2.32779627710172\\
-4.883	-4.65654617824159\\
-7.324	-6.98434245534331\\
-8.545	-8.14871740591324\\
-6.104	-5.82092112881152\\
-7.324	-6.98434245534331\\
-6.104	-5.82092112881152\\
-4.883	-4.65654617824159\\
-1.221	-1.16437495056993\\
-3.662	-3.49217122767166\\
-12.207	-11.6408886335849\\
-14.648	-13.9686849106866\\
-15.869	-15.1330598612566\\
-12.207	-11.6408886335849\\
-7.324	-6.98434245534331\\
-6.104	-5.82092112881152\\
-3.662	-3.49217122767166\\
-7.324	-6.98434245534331\\
-8.545	-8.14871740591324\\
-9.766	-9.31309235648318\\
-12.207	-11.6408886335849\\
-18.311	-17.4618097623964\\
-19.531	-18.6252310889282\\
-18.311	-17.4618097623964\\
-15.869	-15.1330598612566\\
-17.09	-16.2974348118265\\
-18.311	-17.4618097623964\\
-12.207	-11.6408886335849\\
-10.986	-10.476513683015\\
-12.207	-11.6408886335849\\
-10.986	-10.476513683015\\
-12.207	-11.6408886335849\\
-9.766	-9.31309235648318\\
-15.869	-15.1330598612566\\
-18.311	-17.4618097623964\\
-12.207	-11.6408886335849\\
-7.324	-6.98434245534331\\
-9.766	-9.31309235648318\\
-17.09	-16.2974348118265\\
-18.311	-17.4618097623964\\
-21.973	-20.9539809900681\\
-20.752	-19.7896060394981\\
-17.09	-16.2974348118265\\
-15.869	-15.1330598612566\\
-18.311	-17.4618097623964\\
-20.752	-19.7896060394981\\
-17.09	-16.2974348118265\\
-8.545	-8.14871740591324\\
-14.648	-13.9686849106866\\
-12.207	-11.6408886335849\\
-7.324	-6.98434245534331\\
-10.986	-10.476513683015\\
-9.766	-9.31309235648318\\
-8.545	-8.14871740591324\\
-7.324	-6.98434245534331\\
-8.545	-8.14871740591324\\
-9.766	-9.31309235648318\\
-13.428	-12.8052635841548\\
-14.648	-13.9686849106866\\
-7.324	-6.98434245534331\\
-8.545	-8.14871740591324\\
-12.207	-11.6408886335849\\
-17.09	-16.2974348118265\\
-15.869	-15.1330598612566\\
-8.545	-8.14871740591324\\
-9.766	-9.31309235648318\\
-8.545	-8.14871740591324\\
-7.324	-6.98434245534331\\
-6.104	-5.82092112881152\\
-8.545	-8.14871740591324\\
-9.766	-9.31309235648318\\
-12.207	-11.6408886335849\\
-14.648	-13.9686849106866\\
-15.869	-15.1330598612566\\
-10.986	-10.476513683015\\
-13.428	-12.8052635841548\\
-17.09	-16.2974348118265\\
-12.207	-11.6408886335849\\
-3.662	-3.49217122767166\\
-9.766	-9.31309235648318\\
-8.545	-8.14871740591324\\
-9.766	-9.31309235648318\\
-8.545	-8.14871740591324\\
-10.986	-10.476513683015\\
-7.324	-6.98434245534331\\
-9.766	-9.31309235648318\\
-6.104	-5.82092112881152\\
-8.545	-8.14871740591324\\
-6.104	-5.82092112881152\\
-3.662	-3.49217122767166\\
-6.104	-5.82092112881152\\
-4.883	-4.65654617824159\\
-10.986	-10.476513683015\\
-12.207	-11.6408886335849\\
-18.311	-17.4618097623964\\
-10.986	-10.476513683015\\
-4.883	-4.65654617824159\\
-6.104	-5.82092112881152\\
-8.545	-8.14871740591324\\
-6.104	-5.82092112881152\\
-8.545	-8.14871740591324\\
-7.324	-6.98434245534331\\
-13.428	-12.8052635841548\\
-9.766	-9.31309235648318\\
-14.648	-13.9686849106866\\
-10.986	-10.476513683015\\
-9.766	-9.31309235648318\\
-6.104	-5.82092112881152\\
-3.662	-3.49217122767166\\
-1.221	-1.16437495056993\\
-2.441	-2.32779627710172\\
-8.545	-8.14871740591324\\
-7.324	-6.98434245534331\\
-9.766	-9.31309235648318\\
-15.869	-15.1330598612566\\
-10.986	-10.476513683015\\
-9.766	-9.31309235648318\\
-13.428	-12.8052635841548\\
-14.648	-13.9686849106866\\
-12.207	-11.6408886335849\\
-6.104	-5.82092112881152\\
-2.441	-2.32779627710172\\
-4.883	-4.65654617824159\\
-6.104	-5.82092112881152\\
-3.662	-3.49217122767166\\
-4.883	-4.65654617824159\\
-9.766	-9.31309235648318\\
-7.324	-6.98434245534331\\
-10.986	-10.476513683015\\
-7.324	-6.98434245534331\\
-4.883	-4.65654617824159\\
-9.766	-9.31309235648318\\
-12.207	-11.6408886335849\\
-9.766	-9.31309235648318\\
-10.986	-10.476513683015\\
-7.324	-6.98434245534331\\
-10.986	-10.476513683015\\
-14.648	-13.9686849106866\\
-13.428	-12.8052635841548\\
-8.545	-8.14871740591324\\
-14.648	-13.9686849106866\\
-13.428	-12.8052635841548\\
-10.986	-10.476513683015\\
-12.207	-11.6408886335849\\
-18.311	-17.4618097623964\\
-17.09	-16.2974348118265\\
-10.986	-10.476513683015\\
-13.428	-12.8052635841548\\
-9.766	-9.31309235648318\\
-12.207	-11.6408886335849\\
-14.648	-13.9686849106866\\
-9.766	-9.31309235648318\\
-8.545	-8.14871740591324\\
-10.986	-10.476513683015\\
-18.311	-17.4618097623964\\
-14.648	-13.9686849106866\\
-13.428	-12.8052635841548\\
-9.766	-9.31309235648318\\
-10.986	-10.476513683015\\
-9.766	-9.31309235648318\\
-6.104	-5.82092112881152\\
-4.883	-4.65654617824159\\
-3.662	-3.49217122767166\\
-7.324	-6.98434245534331\\
-10.986	-10.476513683015\\
-9.766	-9.31309235648318\\
-6.104	-5.82092112881152\\
-7.324	-6.98434245534331\\
-9.766	-9.31309235648318\\
-6.104	-5.82092112881152\\
-9.766	-9.31309235648318\\
-6.104	-5.82092112881152\\
-7.324	-6.98434245534331\\
-12.207	-11.6408886335849\\
-8.545	-8.14871740591324\\
-4.883	-4.65654617824159\\
-6.104	-5.82092112881152\\
-7.324	-6.98434245534331\\
-6.104	-5.82092112881152\\
-7.324	-6.98434245534331\\
-14.648	-13.9686849106866\\
-9.766	-9.31309235648318\\
-17.09	-16.2974348118265\\
-20.752	-19.7896060394981\\
-18.311	-17.4618097623964\\
-13.428	-12.8052635841548\\
-10.986	-10.476513683015\\
-12.207	-11.6408886335849\\
-13.428	-12.8052635841548\\
-14.648	-13.9686849106866\\
-15.869	-15.1330598612566\\
-20.752	-19.7896060394981\\
-24.414	-23.2817772671698\\
-15.869	-15.1330598612566\\
-20.752	-19.7896060394981\\
-14.648	-13.9686849106866\\
-17.09	-16.2974348118265\\
-18.311	-17.4618097623964\\
-9.766	-9.31309235648318\\
-7.324	-6.98434245534331\\
-8.545	-8.14871740591324\\
-9.766	-9.31309235648318\\
-7.324	-6.98434245534331\\
-6.104	-5.82092112881152\\
-7.324	-6.98434245534331\\
-6.104	-5.82092112881152\\
-3.662	-3.49217122767166\\
-4.883	-4.65654617824159\\
-7.324	-6.98434245534331\\
-2.441	-2.32779627710172\\
-10.986	-10.476513683015\\
-8.545	-8.14871740591324\\
-7.324	-6.98434245534331\\
-15.869	-15.1330598612566\\
-19.531	-18.6252310889282\\
-14.648	-13.9686849106866\\
-17.09	-16.2974348118265\\
-12.207	-11.6408886335849\\
-7.324	-6.98434245534331\\
-10.986	-10.476513683015\\
-7.324	-6.98434245534331\\
-3.662	-3.49217122767166\\
-6.104	-5.82092112881152\\
-4.883	-4.65654617824159\\
-2.441	-2.32779627710172\\
-4.883	-4.65654617824159\\
-10.986	-10.476513683015\\
-9.766	-9.31309235648318\\
-10.986	-10.476513683015\\
-12.207	-11.6408886335849\\
-14.648	-13.9686849106866\\
-10.986	-10.476513683015\\
-8.545	-8.14871740591324\\
-7.324	-6.98434245534331\\
-13.428	-12.8052635841548\\
-9.766	-9.31309235648318\\
-6.104	-5.82092112881152\\
-10.986	-10.476513683015\\
-13.428	-12.8052635841548\\
-10.986	-10.476513683015\\
-7.324	-6.98434245534331\\
-4.883	-4.65654617824159\\
-6.104	-5.82092112881152\\
-7.324	-6.98434245534331\\
-10.986	-10.476513683015\\
-8.545	-8.14871740591324\\
-7.324	-6.98434245534331\\
-10.986	-10.476513683015\\
-8.545	-8.14871740591324\\
-6.104	-5.82092112881152\\
-7.324	-6.98434245534331\\
-13.428	-12.8052635841548\\
-14.648	-13.9686849106866\\
-13.428	-12.8052635841548\\
-12.207	-11.6408886335849\\
-8.545	-8.14871740591324\\
-9.766	-9.31309235648318\\
-8.545	-8.14871740591324\\
-12.207	-11.6408886335849\\
-7.324	-6.98434245534331\\
-4.883	-4.65654617824159\\
-9.766	-9.31309235648318\\
-10.986	-10.476513683015\\
-12.207	-11.6408886335849\\
-10.986	-10.476513683015\\
-17.09	-16.2974348118265\\
-20.752	-19.7896060394981\\
-23.193	-22.1174023165999\\
-15.869	-15.1330598612566\\
-9.766	-9.31309235648318\\
-6.104	-5.82092112881152\\
-9.766	-9.31309235648318\\
-6.104	-5.82092112881152\\
-9.766	-9.31309235648318\\
-6.104	-5.82092112881152\\
-7.324	-6.98434245534331\\
-8.545	-8.14871740591324\\
-10.986	-10.476513683015\\
-13.428	-12.8052635841548\\
-14.648	-13.9686849106866\\
-19.531	-18.6252310889282\\
-14.648	-13.9686849106866\\
-13.428	-12.8052635841548\\
-10.986	-10.476513683015\\
-12.207	-11.6408886335849\\
-9.766	-9.31309235648318\\
-10.986	-10.476513683015\\
-9.766	-9.31309235648318\\
-7.324	-6.98434245534331\\
-12.207	-11.6408886335849\\
-13.428	-12.8052635841548\\
-8.545	-8.14871740591324\\
-9.766	-9.31309235648318\\
-18.311	-17.4618097623964\\
-12.207	-11.6408886335849\\
-17.09	-16.2974348118265\\
-14.648	-13.9686849106866\\
-10.986	-10.476513683015\\
-7.324	-6.98434245534331\\
-12.207	-11.6408886335849\\
-17.09	-16.2974348118265\\
-20.752	-19.7896060394981\\
-18.311	-17.4618097623964\\
-13.428	-12.8052635841548\\
-9.766	-9.31309235648318\\
-8.545	-8.14871740591324\\
-6.104	-5.82092112881152\\
-7.324	-6.98434245534331\\
-4.883	-4.65654617824159\\
-8.545	-8.14871740591324\\
-6.104	-5.82092112881152\\
-9.766	-9.31309235648318\\
-10.986	-10.476513683015\\
-12.207	-11.6408886335849\\
-14.648	-13.9686849106866\\
-17.09	-16.2974348118265\\
-19.531	-18.6252310889282\\
-18.311	-17.4618097623964\\
-13.428	-12.8052635841548\\
-14.648	-13.9686849106866\\
-12.207	-11.6408886335849\\
-7.324	-6.98434245534331\\
-13.428	-12.8052635841548\\
-19.531	-18.6252310889282\\
-18.311	-17.4618097623964\\
-12.207	-11.6408886335849\\
-6.104	-5.82092112881152\\
-4.883	-4.65654617824159\\
-1.221	-1.16437495056993\\
2.441	2.32779627710172\\
-4.883	-4.65654617824159\\
-6.104	-5.82092112881152\\
-8.545	-8.14871740591324\\
-6.104	-5.82092112881152\\
-4.883	-4.65654617824159\\
-7.324	-6.98434245534331\\
-4.883	-4.65654617824159\\
-8.545	-8.14871740591324\\
-12.207	-11.6408886335849\\
-13.428	-12.8052635841548\\
-10.986	-10.476513683015\\
-14.648	-13.9686849106866\\
-17.09	-16.2974348118265\\
-20.752	-19.7896060394981\\
-14.648	-13.9686849106866\\
-10.986	-10.476513683015\\
-18.311	-17.4618097623964\\
-12.207	-11.6408886335849\\
-8.545	-8.14871740591324\\
-4.883	-4.65654617824159\\
-2.441	-2.32779627710172\\
-6.104	-5.82092112881152\\
-8.545	-8.14871740591324\\
-7.324	-6.98434245534331\\
-6.104	-5.82092112881152\\
-3.662	-3.49217122767166\\
-4.883	-4.65654617824159\\
-6.104	-5.82092112881152\\
-2.441	-2.32779627710172\\
-7.324	-6.98434245534331\\
-9.766	-9.31309235648318\\
-13.428	-12.8052635841548\\
-15.869	-15.1330598612566\\
-10.986	-10.476513683015\\
-18.311	-17.4618097623964\\
-23.193	-22.1174023165999\\
-21.973	-20.9539809900681\\
-19.531	-18.6252310889282\\
-14.648	-13.9686849106866\\
-17.09	-16.2974348118265\\
-13.428	-12.8052635841548\\
-9.766	-9.31309235648318\\
-7.324	-6.98434245534331\\
-13.428	-12.8052635841548\\
-8.545	-8.14871740591324\\
-12.207	-11.6408886335849\\
-10.986	-10.476513683015\\
-8.545	-8.14871740591324\\
-10.986	-10.476513683015\\
-9.766	-9.31309235648318\\
-12.207	-11.6408886335849\\
-10.986	-10.476513683015\\
-8.545	-8.14871740591324\\
-12.207	-11.6408886335849\\
-6.104	-5.82092112881152\\
-2.441	-2.32779627710172\\
-7.324	-6.98434245534331\\
-2.441	-2.32779627710172\\
-1.221	-1.16437495056993\\
-2.441	-2.32779627710172\\
-6.104	-5.82092112881152\\
-8.545	-8.14871740591324\\
-7.324	-6.98434245534331\\
-13.428	-12.8052635841548\\
-8.545	-8.14871740591324\\
-6.104	-5.82092112881152\\
-10.986	-10.476513683015\\
-7.324	-6.98434245534331\\
-4.883	-4.65654617824159\\
-2.441	-2.32779627710172\\
-3.662	-3.49217122767166\\
-6.104	-5.82092112881152\\
-9.766	-9.31309235648318\\
-8.545	-8.14871740591324\\
-4.883	-4.65654617824159\\
-7.324	-6.98434245534331\\
-6.104	-5.82092112881152\\
-2.441	-2.32779627710172\\
-7.324	-6.98434245534331\\
-3.662	-3.49217122767166\\
-4.883	-4.65654617824159\\
-7.324	-6.98434245534331\\
-6.104	-5.82092112881152\\
-4.883	-4.65654617824159\\
-3.662	-3.49217122767166\\
-4.883	-4.65654617824159\\
-13.428	-12.8052635841548\\
-7.324	-6.98434245534331\\
-12.207	-11.6408886335849\\
-7.324	-6.98434245534331\\
-12.207	-11.6408886335849\\
-8.545	-8.14871740591324\\
-7.324	-6.98434245534331\\
-8.545	-8.14871740591324\\
-7.324	-6.98434245534331\\
-4.883	-4.65654617824159\\
-6.104	-5.82092112881152\\
-9.766	-9.31309235648318\\
-7.324	-6.98434245534331\\
-12.207	-11.6408886335849\\
-6.104	-5.82092112881152\\
-8.545	-8.14871740591324\\
-6.104	-5.82092112881152\\
-12.207	-11.6408886335849\\
-7.324	-6.98434245534331\\
-14.648	-13.9686849106866\\
-17.09	-16.2974348118265\\
-13.428	-12.8052635841548\\
-9.766	-9.31309235648318\\
-12.207	-11.6408886335849\\
-13.428	-12.8052635841548\\
-8.545	-8.14871740591324\\
-13.428	-12.8052635841548\\
-10.986	-10.476513683015\\
-4.883	-4.65654617824159\\
-3.662	-3.49217122767166\\
-12.207	-11.6408886335849\\
-10.986	-10.476513683015\\
-8.545	-8.14871740591324\\
-10.986	-10.476513683015\\
-12.207	-11.6408886335849\\
-9.766	-9.31309235648318\\
-7.324	-6.98434245534331\\
-9.766	-9.31309235648318\\
-12.207	-11.6408886335849\\
-8.545	-8.14871740591324\\
-10.986	-10.476513683015\\
-9.766	-9.31309235648318\\
-6.104	-5.82092112881152\\
-8.545	-8.14871740591324\\
-4.883	-4.65654617824159\\
-8.545	-8.14871740591324\\
-10.986	-10.476513683015\\
-14.648	-13.9686849106866\\
-13.428	-12.8052635841548\\
-8.545	-8.14871740591324\\
-6.104	-5.82092112881152\\
-7.324	-6.98434245534331\\
-8.545	-8.14871740591324\\
-12.207	-11.6408886335849\\
-14.648	-13.9686849106866\\
-10.986	-10.476513683015\\
-8.545	-8.14871740591324\\
-12.207	-11.6408886335849\\
-18.311	-17.4618097623964\\
-12.207	-11.6408886335849\\
-14.648	-13.9686849106866\\
-9.766	-9.31309235648318\\
-12.207	-11.6408886335849\\
-9.766	-9.31309235648318\\
-10.986	-10.476513683015\\
-14.648	-13.9686849106866\\
-9.766	-9.31309235648318\\
-12.207	-11.6408886335849\\
-10.986	-10.476513683015\\
-6.104	-5.82092112881152\\
-9.766	-9.31309235648318\\
-8.545	-8.14871740591324\\
-9.766	-9.31309235648318\\
-6.104	-5.82092112881152\\
-3.662	-3.49217122767166\\
-2.441	-2.32779627710172\\
-6.104	-5.82092112881152\\
-10.986	-10.476513683015\\
-12.207	-11.6408886335849\\
-10.986	-10.476513683015\\
-8.545	-8.14871740591324\\
-12.207	-11.6408886335849\\
-8.545	-8.14871740591324\\
-9.766	-9.31309235648318\\
-6.104	-5.82092112881152\\
-8.545	-8.14871740591324\\
-7.324	-6.98434245534331\\
-6.104	-5.82092112881152\\
-4.883	-4.65654617824159\\
-10.986	-10.476513683015\\
-7.324	-6.98434245534331\\
-9.766	-9.31309235648318\\
-7.324	-6.98434245534331\\
-12.207	-11.6408886335849\\
-9.766	-9.31309235648318\\
-14.648	-13.9686849106866\\
-8.545	-8.14871740591324\\
-13.428	-12.8052635841548\\
-12.207	-11.6408886335849\\
-7.324	-6.98434245534331\\
-10.986	-10.476513683015\\
-9.766	-9.31309235648318\\
-10.986	-10.476513683015\\
-12.207	-11.6408886335849\\
-13.428	-12.8052635841548\\
-7.324	-6.98434245534331\\
-13.428	-12.8052635841548\\
-14.648	-13.9686849106866\\
-9.766	-9.31309235648318\\
-10.986	-10.476513683015\\
-7.324	-6.98434245534331\\
-9.766	-9.31309235648318\\
-7.324	-6.98434245534331\\
-8.545	-8.14871740591324\\
-7.324	-6.98434245534331\\
-13.428	-12.8052635841548\\
-15.869	-15.1330598612566\\
-14.648	-13.9686849106866\\
-15.869	-15.1330598612566\\
-8.545	-8.14871740591324\\
-7.324	-6.98434245534331\\
-6.104	-5.82092112881152\\
-4.883	-4.65654617824159\\
-8.545	-8.14871740591324\\
-10.986	-10.476513683015\\
-13.428	-12.8052635841548\\
-12.207	-11.6408886335849\\
-7.324	-6.98434245534331\\
-12.207	-11.6408886335849\\
-15.869	-15.1330598612566\\
-13.428	-12.8052635841548\\
-20.752	-19.7896060394981\\
-17.09	-16.2974348118265\\
-10.986	-10.476513683015\\
-6.104	-5.82092112881152\\
-7.324	-6.98434245534331\\
-10.986	-10.476513683015\\
-13.428	-12.8052635841548\\
-17.09	-16.2974348118265\\
-12.207	-11.6408886335849\\
-10.986	-10.476513683015\\
-6.104	-5.82092112881152\\
-9.766	-9.31309235648318\\
-7.324	-6.98434245534331\\
-4.883	-4.65654617824159\\
-3.662	-3.49217122767166\\
-6.104	-5.82092112881152\\
-10.986	-10.476513683015\\
-8.545	-8.14871740591324\\
-9.766	-9.31309235648318\\
-8.545	-8.14871740591324\\
-12.207	-11.6408886335849\\
-7.324	-6.98434245534331\\
-12.207	-11.6408886335849\\
-13.428	-12.8052635841548\\
-8.545	-8.14871740591324\\
-10.986	-10.476513683015\\
-8.545	-8.14871740591324\\
-10.986	-10.476513683015\\
-15.869	-15.1330598612566\\
-10.986	-10.476513683015\\
-8.545	-8.14871740591324\\
-9.766	-9.31309235648318\\
-10.986	-10.476513683015\\
-7.324	-6.98434245534331\\
-8.545	-8.14871740591324\\
-13.428	-12.8052635841548\\
-14.648	-13.9686849106866\\
-15.869	-15.1330598612566\\
-20.752	-19.7896060394981\\
-15.869	-15.1330598612566\\
-14.648	-13.9686849106866\\
-10.986	-10.476513683015\\
-17.09	-16.2974348118265\\
-13.428	-12.8052635841548\\
-10.986	-10.476513683015\\
-14.648	-13.9686849106866\\
-9.766	-9.31309235648318\\
-7.324	-6.98434245534331\\
-9.766	-9.31309235648318\\
-6.104	-5.82092112881152\\
-3.662	-3.49217122767166\\
-6.104	-5.82092112881152\\
-4.883	-4.65654617824159\\
-8.545	-8.14871740591324\\
-9.766	-9.31309235648318\\
-12.207	-11.6408886335849\\
-14.648	-13.9686849106866\\
-10.986	-10.476513683015\\
-9.766	-9.31309235648318\\
-10.986	-10.476513683015\\
-14.648	-13.9686849106866\\
-10.986	-10.476513683015\\
-9.766	-9.31309235648318\\
-10.986	-10.476513683015\\
-14.648	-13.9686849106866\\
-12.207	-11.6408886335849\\
-13.428	-12.8052635841548\\
-9.766	-9.31309235648318\\
-6.104	-5.82092112881152\\
-4.883	-4.65654617824159\\
-7.324	-6.98434245534331\\
-6.104	-5.82092112881152\\
-8.545	-8.14871740591324\\
-14.648	-13.9686849106866\\
-18.311	-17.4618097623964\\
-12.207	-11.6408886335849\\
-14.648	-13.9686849106866\\
-13.428	-12.8052635841548\\
-10.986	-10.476513683015\\
-13.428	-12.8052635841548\\
-14.648	-13.9686849106866\\
-19.531	-18.6252310889282\\
-17.09	-16.2974348118265\\
-15.869	-15.1330598612566\\
-19.531	-18.6252310889282\\
-13.428	-12.8052635841548\\
-15.869	-15.1330598612566\\
-13.428	-12.8052635841548\\
-10.986	-10.476513683015\\
-13.428	-12.8052635841548\\
-17.09	-16.2974348118265\\
-7.324	-6.98434245534331\\
-10.986	-10.476513683015\\
-7.324	-6.98434245534331\\
-8.545	-8.14871740591324\\
-10.986	-10.476513683015\\
-7.324	-6.98434245534331\\
-8.545	-8.14871740591324\\
-15.869	-15.1330598612566\\
-20.752	-19.7896060394981\\
-21.973	-20.9539809900681\\
-19.531	-18.6252310889282\\
-15.869	-15.1330598612566\\
-19.531	-18.6252310889282\\
-21.973	-20.9539809900681\\
-17.09	-16.2974348118265\\
-18.311	-17.4618097623964\\
-17.09	-16.2974348118265\\
-12.207	-11.6408886335849\\
-9.766	-9.31309235648318\\
-13.428	-12.8052635841548\\
-10.986	-10.476513683015\\
-6.104	-5.82092112881152\\
-9.766	-9.31309235648318\\
-8.545	-8.14871740591324\\
-9.766	-9.31309235648318\\
-12.207	-11.6408886335849\\
-8.545	-8.14871740591324\\
-7.324	-6.98434245534331\\
-8.545	-8.14871740591324\\
-10.986	-10.476513683015\\
-9.766	-9.31309235648318\\
-14.648	-13.9686849106866\\
-13.428	-12.8052635841548\\
-9.766	-9.31309235648318\\
-13.428	-12.8052635841548\\
-10.986	-10.476513683015\\
-8.545	-8.14871740591324\\
-7.324	-6.98434245534331\\
-4.883	-4.65654617824159\\
-6.104	-5.82092112881152\\
-8.545	-8.14871740591324\\
-6.104	-5.82092112881152\\
-4.883	-4.65654617824159\\
-7.324	-6.98434245534331\\
-12.207	-11.6408886335849\\
-14.648	-13.9686849106866\\
-13.428	-12.8052635841548\\
-10.986	-10.476513683015\\
-4.883	-4.65654617824159\\
-6.104	-5.82092112881152\\
-7.324	-6.98434245534331\\
-8.545	-8.14871740591324\\
-10.986	-10.476513683015\\
-14.648	-13.9686849106866\\
-9.766	-9.31309235648318\\
-14.648	-13.9686849106866\\
-13.428	-12.8052635841548\\
-6.104	-5.82092112881152\\
-8.545	-8.14871740591324\\
-7.324	-6.98434245534331\\
-10.986	-10.476513683015\\
-17.09	-16.2974348118265\\
-12.207	-11.6408886335849\\
-9.766	-9.31309235648318\\
-8.545	-8.14871740591324\\
-10.986	-10.476513683015\\
-12.207	-11.6408886335849\\
-19.531	-18.6252310889282\\
-15.869	-15.1330598612566\\
-17.09	-16.2974348118265\\
-12.207	-11.6408886335849\\
-7.324	-6.98434245534331\\
-8.545	-8.14871740591324\\
-4.883	-4.65654617824159\\
-6.104	-5.82092112881152\\
-8.545	-8.14871740591324\\
-3.662	-3.49217122767166\\
-8.545	-8.14871740591324\\
-13.428	-12.8052635841548\\
-15.869	-15.1330598612566\\
-20.752	-19.7896060394981\\
-18.311	-17.4618097623964\\
-9.766	-9.31309235648318\\
-10.986	-10.476513683015\\
-8.545	-8.14871740591324\\
-13.428	-12.8052635841548\\
-15.869	-15.1330598612566\\
-9.766	-9.31309235648318\\
-10.986	-10.476513683015\\
-14.648	-13.9686849106866\\
-13.428	-12.8052635841548\\
-7.324	-6.98434245534331\\
-6.104	-5.82092112881152\\
-4.883	-4.65654617824159\\
-3.662	-3.49217122767166\\
-4.883	-4.65654617824159\\
-3.662	-3.49217122767166\\
-8.545	-8.14871740591324\\
-6.104	-5.82092112881152\\
-8.545	-8.14871740591324\\
-6.104	-5.82092112881152\\
-3.662	-3.49217122767166\\
-2.441	-2.32779627710172\\
-6.104	-5.82092112881152\\
-8.545	-8.14871740591324\\
-7.324	-6.98434245534331\\
-10.986	-10.476513683015\\
-14.648	-13.9686849106866\\
-19.531	-18.6252310889282\\
-21.973	-20.9539809900681\\
-20.752	-19.7896060394981\\
-19.531	-18.6252310889282\\
-13.428	-12.8052635841548\\
-10.986	-10.476513683015\\
-12.207	-11.6408886335849\\
-10.986	-10.476513683015\\
-7.324	-6.98434245534331\\
-8.545	-8.14871740591324\\
-10.986	-10.476513683015\\
-7.324	-6.98434245534331\\
-2.441	-2.32779627710172\\
-4.883	-4.65654617824159\\
-6.104	-5.82092112881152\\
-7.324	-6.98434245534331\\
-6.104	-5.82092112881152\\
-8.545	-8.14871740591324\\
-6.104	-5.82092112881152\\
-14.648	-13.9686849106866\\
-10.986	-10.476513683015\\
-12.207	-11.6408886335849\\
-20.752	-19.7896060394981\\
-15.869	-15.1330598612566\\
-13.428	-12.8052635841548\\
-8.545	-8.14871740591324\\
-7.324	-6.98434245534331\\
-6.104	-5.82092112881152\\
-10.986	-10.476513683015\\
-7.324	-6.98434245534331\\
-10.986	-10.476513683015\\
-8.545	-8.14871740591324\\
-10.986	-10.476513683015\\
-4.883	-4.65654617824159\\
-7.324	-6.98434245534331\\
-9.766	-9.31309235648318\\
-3.662	-3.49217122767166\\
-6.104	-5.82092112881152\\
-2.441	-2.32779627710172\\
-3.662	-3.49217122767166\\
-2.441	-2.32779627710172\\
-6.104	-5.82092112881152\\
-7.324	-6.98434245534331\\
-10.986	-10.476513683015\\
-7.324	-6.98434245534331\\
-12.207	-11.6408886335849\\
-14.648	-13.9686849106866\\
-9.766	-9.31309235648318\\
-10.986	-10.476513683015\\
-9.766	-9.31309235648318\\
-2.441	-2.32779627710172\\
-9.766	-9.31309235648318\\
-8.545	-8.14871740591324\\
-12.207	-11.6408886335849\\
-13.428	-12.8052635841548\\
-15.869	-15.1330598612566\\
-10.986	-10.476513683015\\
-8.545	-8.14871740591324\\
-7.324	-6.98434245534331\\
-8.545	-8.14871740591324\\
-6.104	-5.82092112881152\\
-4.883	-4.65654617824159\\
-3.662	-3.49217122767166\\
-4.883	-4.65654617824159\\
-3.662	-3.49217122767166\\
-8.545	-8.14871740591324\\
-4.883	-4.65654617824159\\
-7.324	-6.98434245534331\\
-14.648	-13.9686849106866\\
-6.104	-5.82092112881152\\
-9.766	-9.31309235648318\\
-17.09	-16.2974348118265\\
-9.766	-9.31309235648318\\
-6.104	-5.82092112881152\\
-9.766	-9.31309235648318\\
-10.986	-10.476513683015\\
-8.545	-8.14871740591324\\
-3.662	-3.49217122767166\\
-6.104	-5.82092112881152\\
-1.221	-1.16437495056993\\
-2.441	-2.32779627710172\\
-6.104	-5.82092112881152\\
-7.324	-6.98434245534331\\
-6.104	-5.82092112881152\\
-8.545	-8.14871740591324\\
-7.324	-6.98434245534331\\
-3.662	-3.49217122767166\\
-1.221	-1.16437495056993\\
-6.104	-5.82092112881152\\
-4.883	-4.65654617824159\\
-7.324	-6.98434245534331\\
-10.986	-10.476513683015\\
-7.324	-6.98434245534331\\
-8.545	-8.14871740591324\\
-9.766	-9.31309235648318\\
-12.207	-11.6408886335849\\
-14.648	-13.9686849106866\\
-17.09	-16.2974348118265\\
-12.207	-11.6408886335849\\
-10.986	-10.476513683015\\
-12.207	-11.6408886335849\\
-13.428	-12.8052635841548\\
-18.311	-17.4618097623964\\
-17.09	-16.2974348118265\\
-13.428	-12.8052635841548\\
-12.207	-11.6408886335849\\
-13.428	-12.8052635841548\\
-10.986	-10.476513683015\\
-12.207	-11.6408886335849\\
-13.428	-12.8052635841548\\
-12.207	-11.6408886335849\\
-10.986	-10.476513683015\\
-9.766	-9.31309235648318\\
-7.324	-6.98434245534331\\
-12.207	-11.6408886335849\\
-8.545	-8.14871740591324\\
-6.104	-5.82092112881152\\
-8.545	-8.14871740591324\\
-3.662	-3.49217122767166\\
-7.324	-6.98434245534331\\
-3.662	-3.49217122767166\\
-6.104	-5.82092112881152\\
-4.883	-4.65654617824159\\
-2.441	-2.32779627710172\\
-6.104	-5.82092112881152\\
-3.662	-3.49217122767166\\
-8.545	-8.14871740591324\\
-3.662	-3.49217122767166\\
-8.545	-8.14871740591324\\
-4.883	-4.65654617824159\\
-7.324	-6.98434245534331\\
-8.545	-8.14871740591324\\
-7.324	-6.98434245534331\\
-13.428	-12.8052635841548\\
-4.883	-4.65654617824159\\
-8.545	-8.14871740591324\\
-10.986	-10.476513683015\\
-6.104	-5.82092112881152\\
-7.324	-6.98434245534331\\
-12.207	-11.6408886335849\\
-3.662	-3.49217122767166\\
-6.104	-5.82092112881152\\
-8.545	-8.14871740591324\\
-6.104	-5.82092112881152\\
-9.766	-9.31309235648318\\
-1.221	-1.16437495056993\\
-3.662	-3.49217122767166\\
-4.883	-4.65654617824159\\
-7.324	-6.98434245534331\\
-3.662	-3.49217122767166\\
-8.545	-8.14871740591324\\
-10.986	-10.476513683015\\
-9.766	-9.31309235648318\\
-8.545	-8.14871740591324\\
-14.648	-13.9686849106866\\
-10.986	-10.476513683015\\
-18.311	-17.4618097623964\\
-13.428	-12.8052635841548\\
-17.09	-16.2974348118265\\
-13.428	-12.8052635841548\\
-17.09	-16.2974348118265\\
-13.428	-12.8052635841548\\
-4.883	-4.65654617824159\\
-7.324	-6.98434245534331\\
-12.207	-11.6408886335849\\
-10.986	-10.476513683015\\
-8.545	-8.14871740591324\\
-4.883	-4.65654617824159\\
-6.104	-5.82092112881152\\
-9.766	-9.31309235648318\\
-15.869	-15.1330598612566\\
-13.428	-12.8052635841548\\
-9.766	-9.31309235648318\\
-7.324	-6.98434245534331\\
-12.207	-11.6408886335849\\
-15.869	-15.1330598612566\\
-14.648	-13.9686849106866\\
-13.428	-12.8052635841548\\
-19.531	-18.6252310889282\\
-17.09	-16.2974348118265\\
-19.531	-18.6252310889282\\
-20.752	-19.7896060394981\\
-15.869	-15.1330598612566\\
-18.311	-17.4618097623964\\
-15.869	-15.1330598612566\\
-8.545	-8.14871740591324\\
-10.986	-10.476513683015\\
-13.428	-12.8052635841548\\
-10.986	-10.476513683015\\
-9.766	-9.31309235648318\\
-13.428	-12.8052635841548\\
-9.766	-9.31309235648318\\
-12.207	-11.6408886335849\\
-9.766	-9.31309235648318\\
-14.648	-13.9686849106866\\
-18.311	-17.4618097623964\\
-10.986	-10.476513683015\\
-8.545	-8.14871740591324\\
-7.324	-6.98434245534331\\
-12.207	-11.6408886335849\\
-3.662	-3.49217122767166\\
-4.883	-4.65654617824159\\
-13.428	-12.8052635841548\\
-10.986	-10.476513683015\\
-9.766	-9.31309235648318\\
-8.545	-8.14871740591324\\
-3.662	-3.49217122767166\\
-6.104	-5.82092112881152\\
-4.883	-4.65654617824159\\
-8.545	-8.14871740591324\\
-7.324	-6.98434245534331\\
-13.428	-12.8052635841548\\
-15.869	-15.1330598612566\\
-14.648	-13.9686849106866\\
-17.09	-16.2974348118265\\
-14.648	-13.9686849106866\\
-9.766	-9.31309235648318\\
-8.545	-8.14871740591324\\
-9.766	-9.31309235648318\\
-6.104	-5.82092112881152\\
-4.883	-4.65654617824159\\
-10.986	-10.476513683015\\
-12.207	-11.6408886335849\\
-9.766	-9.31309235648318\\
-14.648	-13.9686849106866\\
-17.09	-16.2974348118265\\
-12.207	-11.6408886335849\\
-8.545	-8.14871740591324\\
-7.324	-6.98434245534331\\
-9.766	-9.31309235648318\\
-8.545	-8.14871740591324\\
-9.766	-9.31309235648318\\
-15.869	-15.1330598612566\\
-12.207	-11.6408886335849\\
-10.986	-10.476513683015\\
-7.324	-6.98434245534331\\
-12.207	-11.6408886335849\\
-14.648	-13.9686849106866\\
-12.207	-11.6408886335849\\
-10.986	-10.476513683015\\
-17.09	-16.2974348118265\\
-10.986	-10.476513683015\\
-8.545	-8.14871740591324\\
-10.986	-10.476513683015\\
-7.324	-6.98434245534331\\
-9.766	-9.31309235648318\\
-10.986	-10.476513683015\\
-8.545	-8.14871740591324\\
-9.766	-9.31309235648318\\
-15.869	-15.1330598612566\\
-18.311	-17.4618097623964\\
-10.986	-10.476513683015\\
-12.207	-11.6408886335849\\
-13.428	-12.8052635841548\\
-14.648	-13.9686849106866\\
-9.766	-9.31309235648318\\
-8.545	-8.14871740591324\\
-10.986	-10.476513683015\\
-8.545	-8.14871740591324\\
-15.869	-15.1330598612566\\
-20.752	-19.7896060394981\\
-23.193	-22.1174023165999\\
-21.973	-20.9539809900681\\
-14.648	-13.9686849106866\\
-13.428	-12.8052635841548\\
-12.207	-11.6408886335849\\
-8.545	-8.14871740591324\\
-6.104	-5.82092112881152\\
-7.324	-6.98434245534331\\
-3.662	-3.49217122767166\\
-4.883	-4.65654617824159\\
-10.986	-10.476513683015\\
-7.324	-6.98434245534331\\
-3.662	-3.49217122767166\\
-8.545	-8.14871740591324\\
-10.986	-10.476513683015\\
-9.766	-9.31309235648318\\
-6.104	-5.82092112881152\\
-12.207	-11.6408886335849\\
-6.104	-5.82092112881152\\
-7.324	-6.98434245534331\\
-8.545	-8.14871740591324\\
-6.104	-5.82092112881152\\
-4.883	-4.65654617824159\\
-6.104	-5.82092112881152\\
-8.545	-8.14871740591324\\
-9.766	-9.31309235648318\\
-8.545	-8.14871740591324\\
-6.104	-5.82092112881152\\
-4.883	-4.65654617824159\\
-8.545	-8.14871740591324\\
-9.766	-9.31309235648318\\
-8.545	-8.14871740591324\\
-6.104	-5.82092112881152\\
-7.324	-6.98434245534331\\
-8.545	-8.14871740591324\\
-14.648	-13.9686849106866\\
-17.09	-16.2974348118265\\
-13.428	-12.8052635841548\\
-7.324	-6.98434245534331\\
-9.766	-9.31309235648318\\
-8.545	-8.14871740591324\\
-7.324	-6.98434245534331\\
-8.545	-8.14871740591324\\
-7.324	-6.98434245534331\\
-3.662	-3.49217122767166\\
-4.883	-4.65654617824159\\
-7.324	-6.98434245534331\\
-9.766	-9.31309235648318\\
-12.207	-11.6408886335849\\
-10.986	-10.476513683015\\
-14.648	-13.9686849106866\\
-13.428	-12.8052635841548\\
-17.09	-16.2974348118265\\
-23.193	-22.1174023165999\\
-17.09	-16.2974348118265\\
-14.648	-13.9686849106866\\
-17.09	-16.2974348118265\\
-18.311	-17.4618097623964\\
-19.531	-18.6252310889282\\
-18.311	-17.4618097623964\\
-8.545	-8.14871740591324\\
-4.883	-4.65654617824159\\
-6.104	-5.82092112881152\\
-3.662	-3.49217122767166\\
-6.104	-5.82092112881152\\
-8.545	-8.14871740591324\\
-4.883	-4.65654617824159\\
-6.104	-5.82092112881152\\
-8.545	-8.14871740591324\\
-7.324	-6.98434245534331\\
-6.104	-5.82092112881152\\
-7.324	-6.98434245534331\\
-9.766	-9.31309235648318\\
-4.883	-4.65654617824159\\
-3.662	-3.49217122767166\\
-6.104	-5.82092112881152\\
-4.883	-4.65654617824159\\
-8.545	-8.14871740591324\\
-9.766	-9.31309235648318\\
-7.324	-6.98434245534331\\
-9.766	-9.31309235648318\\
-7.324	-6.98434245534331\\
-9.766	-9.31309235648318\\
-14.648	-13.9686849106866\\
-7.324	-6.98434245534331\\
-10.986	-10.476513683015\\
-14.648	-13.9686849106866\\
-10.986	-10.476513683015\\
-12.207	-11.6408886335849\\
-13.428	-12.8052635841548\\
-12.207	-11.6408886335849\\
-13.428	-12.8052635841548\\
-15.869	-15.1330598612566\\
-13.428	-12.8052635841548\\
-21.973	-20.9539809900681\\
-20.752	-19.7896060394981\\
-12.207	-11.6408886335849\\
-8.545	-8.14871740591324\\
-12.207	-11.6408886335849\\
-13.428	-12.8052635841548\\
-15.869	-15.1330598612566\\
-10.986	-10.476513683015\\
-9.766	-9.31309235648318\\
-12.207	-11.6408886335849\\
-9.766	-9.31309235648318\\
-3.662	-3.49217122767166\\
-9.766	-9.31309235648318\\
-12.207	-11.6408886335849\\
-10.986	-10.476513683015\\
-14.648	-13.9686849106866\\
-13.428	-12.8052635841548\\
-8.545	-8.14871740591324\\
-12.207	-11.6408886335849\\
-15.869	-15.1330598612566\\
-19.531	-18.6252310889282\\
-20.752	-19.7896060394981\\
-18.311	-17.4618097623964\\
-17.09	-16.2974348118265\\
-10.986	-10.476513683015\\
-8.545	-8.14871740591324\\
-13.428	-12.8052635841548\\
-19.531	-18.6252310889282\\
-10.986	-10.476513683015\\
-12.207	-11.6408886335849\\
-21.973	-20.9539809900681\\
-19.531	-18.6252310889282\\
-9.766	-9.31309235648318\\
-6.104	-5.82092112881152\\
-9.766	-9.31309235648318\\
-10.986	-10.476513683015\\
-7.324	-6.98434245534331\\
-4.883	-4.65654617824159\\
-6.104	-5.82092112881152\\
-8.545	-8.14871740591324\\
-7.324	-6.98434245534331\\
-6.104	-5.82092112881152\\
-8.545	-8.14871740591324\\
-9.766	-9.31309235648318\\
-12.207	-11.6408886335849\\
-14.648	-13.9686849106866\\
-9.766	-9.31309235648318\\
-12.207	-11.6408886335849\\
-9.766	-9.31309235648318\\
-8.545	-8.14871740591324\\
-7.324	-6.98434245534331\\
-8.545	-8.14871740591324\\
-9.766	-9.31309235648318\\
-6.104	-5.82092112881152\\
-3.662	-3.49217122767166\\
-4.883	-4.65654617824159\\
-7.324	-6.98434245534331\\
-6.104	-5.82092112881152\\
-10.986	-10.476513683015\\
-8.545	-8.14871740591324\\
-9.766	-9.31309235648318\\
-6.104	-5.82092112881152\\
-4.883	-4.65654617824159\\
-7.324	-6.98434245534331\\
-10.986	-10.476513683015\\
-9.766	-9.31309235648318\\
-10.986	-10.476513683015\\
-7.324	-6.98434245534331\\
-8.545	-8.14871740591324\\
-7.324	-6.98434245534331\\
-12.207	-11.6408886335849\\
-15.869	-15.1330598612566\\
-12.207	-11.6408886335849\\
-15.869	-15.1330598612566\\
-13.428	-12.8052635841548\\
-8.545	-8.14871740591324\\
-7.324	-6.98434245534331\\
-4.883	-4.65654617824159\\
-3.662	-3.49217122767166\\
-10.986	-10.476513683015\\
-6.104	-5.82092112881152\\
-4.883	-4.65654617824159\\
-6.104	-5.82092112881152\\
-8.545	-8.14871740591324\\
-7.324	-6.98434245534331\\
-6.104	-5.82092112881152\\
-4.883	-4.65654617824159\\
-6.104	-5.82092112881152\\
-10.986	-10.476513683015\\
-7.324	-6.98434245534331\\
-4.883	-4.65654617824159\\
-9.766	-9.31309235648318\\
-8.545	-8.14871740591324\\
-9.766	-9.31309235648318\\
-10.986	-10.476513683015\\
-13.428	-12.8052635841548\\
-12.207	-11.6408886335849\\
-8.545	-8.14871740591324\\
-7.324	-6.98434245534331\\
-6.104	-5.82092112881152\\
-9.766	-9.31309235648318\\
-6.104	-5.82092112881152\\
-8.545	-8.14871740591324\\
-12.207	-11.6408886335849\\
-9.766	-9.31309235648318\\
-10.986	-10.476513683015\\
-15.869	-15.1330598612566\\
-10.986	-10.476513683015\\
-14.648	-13.9686849106866\\
-12.207	-11.6408886335849\\
-15.869	-15.1330598612566\\
-13.428	-12.8052635841548\\
-21.973	-20.9539809900681\\
-15.869	-15.1330598612566\\
-19.531	-18.6252310889282\\
-21.973	-20.9539809900681\\
-20.752	-19.7896060394981\\
-23.193	-22.1174023165999\\
-21.973	-20.9539809900681\\
-25.635	-24.4461522177397\\
-20.752	-19.7896060394981\\
-13.428	-12.8052635841548\\
-9.766	-9.31309235648318\\
-12.207	-11.6408886335849\\
-8.545	-8.14871740591324\\
-7.324	-6.98434245534331\\
-12.207	-11.6408886335849\\
-14.648	-13.9686849106866\\
-9.766	-9.31309235648318\\
-7.324	-6.98434245534331\\
-9.766	-9.31309235648318\\
-7.324	-6.98434245534331\\
-8.545	-8.14871740591324\\
-6.104	-5.82092112881152\\
-9.766	-9.31309235648318\\
-4.883	-4.65654617824159\\
-1.221	-1.16437495056993\\
-3.662	-3.49217122767166\\
-2.441	-2.32779627710172\\
-3.662	-3.49217122767166\\
-4.883	-4.65654617824159\\
-3.662	-3.49217122767166\\
-8.545	-8.14871740591324\\
-9.766	-9.31309235648318\\
-6.104	-5.82092112881152\\
-12.207	-11.6408886335849\\
-14.648	-13.9686849106866\\
-8.545	-8.14871740591324\\
-2.441	-2.32779627710172\\
-3.662	-3.49217122767166\\
-2.441	-2.32779627710172\\
-6.104	-5.82092112881152\\
-7.324	-6.98434245534331\\
-8.545	-8.14871740591324\\
-3.662	-3.49217122767166\\
-7.324	-6.98434245534331\\
-14.648	-13.9686849106866\\
-9.766	-9.31309235648318\\
-7.324	-6.98434245534331\\
-12.207	-11.6408886335849\\
-13.428	-12.8052635841548\\
-14.648	-13.9686849106866\\
-7.324	-6.98434245534331\\
-4.883	-4.65654617824159\\
-6.104	-5.82092112881152\\
-3.662	-3.49217122767166\\
-8.545	-8.14871740591324\\
-14.648	-13.9686849106866\\
-10.986	-10.476513683015\\
-6.104	-5.82092112881152\\
-3.662	-3.49217122767166\\
-4.883	-4.65654617824159\\
-7.324	-6.98434245534331\\
-4.883	-4.65654617824159\\
-3.662	-3.49217122767166\\
-6.104	-5.82092112881152\\
-12.207	-11.6408886335849\\
-9.766	-9.31309235648318\\
-10.986	-10.476513683015\\
-12.207	-11.6408886335849\\
-10.986	-10.476513683015\\
-12.207	-11.6408886335849\\
-10.986	-10.476513683015\\
-13.428	-12.8052635841548\\
-18.311	-17.4618097623964\\
-15.869	-15.1330598612566\\
-7.324	-6.98434245534331\\
-4.883	-4.65654617824159\\
-8.545	-8.14871740591324\\
-14.648	-13.9686849106866\\
-10.986	-10.476513683015\\
-4.883	-4.65654617824159\\
-10.986	-10.476513683015\\
-9.766	-9.31309235648318\\
-17.09	-16.2974348118265\\
-20.752	-19.7896060394981\\
-19.531	-18.6252310889282\\
-15.869	-15.1330598612566\\
-14.648	-13.9686849106866\\
-7.324	-6.98434245534331\\
-9.766	-9.31309235648318\\
-8.545	-8.14871740591324\\
-10.986	-10.476513683015\\
-12.207	-11.6408886335849\\
-7.324	-6.98434245534331\\
-10.986	-10.476513683015\\
-8.545	-8.14871740591324\\
-6.104	-5.82092112881152\\
-3.662	-3.49217122767166\\
-4.883	-4.65654617824159\\
-3.662	-3.49217122767166\\
-6.104	-5.82092112881152\\
-9.766	-9.31309235648318\\
-6.104	-5.82092112881152\\
-13.428	-12.8052635841548\\
-6.104	-5.82092112881152\\
-3.662	-3.49217122767166\\
-1.221	-1.16437495056993\\
-6.104	-5.82092112881152\\
-10.986	-10.476513683015\\
-14.648	-13.9686849106866\\
-21.973	-20.9539809900681\\
-23.193	-22.1174023165999\\
-24.414	-23.2817772671698\\
-17.09	-16.2974348118265\\
-18.311	-17.4618097623964\\
-20.752	-19.7896060394981\\
-18.311	-17.4618097623964\\
-17.09	-16.2974348118265\\
-18.311	-17.4618097623964\\
-20.752	-19.7896060394981\\
-28.076	-26.7739484948415\\
-19.531	-18.6252310889282\\
-12.207	-11.6408886335849\\
-7.324	-6.98434245534331\\
-6.104	-5.82092112881152\\
-4.883	-4.65654617824159\\
-12.207	-11.6408886335849\\
-7.324	-6.98434245534331\\
-10.986	-10.476513683015\\
-7.324	-6.98434245534331\\
-12.207	-11.6408886335849\\
-6.104	-5.82092112881152\\
-4.883	-4.65654617824159\\
-6.104	-5.82092112881152\\
-8.545	-8.14871740591324\\
-13.428	-12.8052635841548\\
-19.531	-18.6252310889282\\
-18.311	-17.4618097623964\\
-23.193	-22.1174023165999\\
-29.297	-27.9383234454114\\
-21.973	-20.9539809900681\\
-14.648	-13.9686849106866\\
-9.766	-9.31309235648318\\
-6.104	-5.82092112881152\\
-10.986	-10.476513683015\\
-8.545	-8.14871740591324\\
-9.766	-9.31309235648318\\
-7.324	-6.98434245534331\\
-8.545	-8.14871740591324\\
-7.324	-6.98434245534331\\
-9.766	-9.31309235648318\\
-10.986	-10.476513683015\\
-7.324	-6.98434245534331\\
-6.104	-5.82092112881152\\
-3.662	-3.49217122767166\\
-1.221	-1.16437495056993\\
-3.662	-3.49217122767166\\
-6.104	-5.82092112881152\\
-7.324	-6.98434245534331\\
-12.207	-11.6408886335849\\
-7.324	-6.98434245534331\\
-12.207	-11.6408886335849\\
-18.311	-17.4618097623964\\
-12.207	-11.6408886335849\\
-17.09	-16.2974348118265\\
-12.207	-11.6408886335849\\
-14.648	-13.9686849106866\\
-8.545	-8.14871740591324\\
-12.207	-11.6408886335849\\
-17.09	-16.2974348118265\\
-24.414	-23.2817772671698\\
-29.297	-27.9383234454114\\
-23.193	-22.1174023165999\\
-18.311	-17.4618097623964\\
-15.869	-15.1330598612566\\
-18.311	-17.4618097623964\\
-19.531	-18.6252310889282\\
-13.428	-12.8052635841548\\
-12.207	-11.6408886335849\\
-8.545	-8.14871740591324\\
-14.648	-13.9686849106866\\
-12.207	-11.6408886335849\\
-8.545	-8.14871740591324\\
-7.324	-6.98434245534331\\
-12.207	-11.6408886335849\\
-14.648	-13.9686849106866\\
-15.869	-15.1330598612566\\
-9.766	-9.31309235648318\\
-12.207	-11.6408886335849\\
-7.324	-6.98434245534331\\
-4.883	-4.65654617824159\\
-3.662	-3.49217122767166\\
-4.883	-4.65654617824159\\
-10.986	-10.476513683015\\
-8.545	-8.14871740591324\\
-10.986	-10.476513683015\\
-7.324	-6.98434245534331\\
-6.104	-5.82092112881152\\
-3.662	-3.49217122767166\\
-8.545	-8.14871740591324\\
-9.766	-9.31309235648318\\
-7.324	-6.98434245534331\\
-4.883	-4.65654617824159\\
-8.545	-8.14871740591324\\
-14.648	-13.9686849106866\\
-8.545	-8.14871740591324\\
-2.441	-2.32779627710172\\
-10.986	-10.476513683015\\
-9.766	-9.31309235648318\\
-7.324	-6.98434245534331\\
-3.662	-3.49217122767166\\
-10.986	-10.476513683015\\
-6.104	-5.82092112881152\\
-3.662	-3.49217122767166\\
-10.986	-10.476513683015\\
-13.428	-12.8052635841548\\
-12.207	-11.6408886335849\\
-6.104	-5.82092112881152\\
-7.324	-6.98434245534331\\
-12.207	-11.6408886335849\\
-14.648	-13.9686849106866\\
-12.207	-11.6408886335849\\
-18.311	-17.4618097623964\\
-21.973	-20.9539809900681\\
-13.428	-12.8052635841548\\
-17.09	-16.2974348118265\\
-10.986	-10.476513683015\\
-17.09	-16.2974348118265\\
-13.428	-12.8052635841548\\
-6.104	-5.82092112881152\\
-9.766	-9.31309235648318\\
-19.531	-18.6252310889282\\
-21.973	-20.9539809900681\\
-15.869	-15.1330598612566\\
-19.531	-18.6252310889282\\
-14.648	-13.9686849106866\\
-10.986	-10.476513683015\\
-9.766	-9.31309235648318\\
-12.207	-11.6408886335849\\
-13.428	-12.8052635841548\\
-8.545	-8.14871740591324\\
-10.986	-10.476513683015\\
-13.428	-12.8052635841548\\
-8.545	-8.14871740591324\\
-7.324	-6.98434245534331\\
-6.104	-5.82092112881152\\
-3.662	-3.49217122767166\\
-7.324	-6.98434245534331\\
-8.545	-8.14871740591324\\
-12.207	-11.6408886335849\\
-15.869	-15.1330598612566\\
-10.986	-10.476513683015\\
-12.207	-11.6408886335849\\
-8.545	-8.14871740591324\\
-13.428	-12.8052635841548\\
-14.648	-13.9686849106866\\
-8.545	-8.14871740591324\\
-3.662	-3.49217122767166\\
-10.986	-10.476513683015\\
-13.428	-12.8052635841548\\
-18.311	-17.4618097623964\\
-24.414	-23.2817772671698\\
-15.869	-15.1330598612566\\
-17.09	-16.2974348118265\\
-12.207	-11.6408886335849\\
-9.766	-9.31309235648318\\
-12.207	-11.6408886335849\\
-14.648	-13.9686849106866\\
-13.428	-12.8052635841548\\
-9.766	-9.31309235648318\\
-8.545	-8.14871740591324\\
-6.104	-5.82092112881152\\
-10.986	-10.476513683015\\
-7.324	-6.98434245534331\\
-14.648	-13.9686849106866\\
-15.869	-15.1330598612566\\
-8.545	-8.14871740591324\\
-7.324	-6.98434245534331\\
-12.207	-11.6408886335849\\
-7.324	-6.98434245534331\\
-6.104	-5.82092112881152\\
-7.324	-6.98434245534331\\
-9.766	-9.31309235648318\\
-6.104	-5.82092112881152\\
-3.662	-3.49217122767166\\
-2.441	-2.32779627710172\\
-6.104	-5.82092112881152\\
-9.766	-9.31309235648318\\
-13.428	-12.8052635841548\\
-15.869	-15.1330598612566\\
-10.986	-10.476513683015\\
-8.545	-8.14871740591324\\
-15.869	-15.1330598612566\\
-20.752	-19.7896060394981\\
-15.869	-15.1330598612566\\
-14.648	-13.9686849106866\\
-23.193	-22.1174023165999\\
-17.09	-16.2974348118265\\
-15.869	-15.1330598612566\\
-13.428	-12.8052635841548\\
-15.869	-15.1330598612566\\
-10.986	-10.476513683015\\
-9.766	-9.31309235648318\\
-17.09	-16.2974348118265\\
-18.311	-17.4618097623964\\
-19.531	-18.6252310889282\\
-12.207	-11.6408886335849\\
-4.883	-4.65654617824159\\
-7.324	-6.98434245534331\\
-10.986	-10.476513683015\\
-2.441	-2.32779627710172\\
-3.662	-3.49217122767166\\
-8.545	-8.14871740591324\\
-13.428	-12.8052635841548\\
-8.545	-8.14871740591324\\
-6.104	-5.82092112881152\\
-3.662	-3.49217122767166\\
-6.104	-5.82092112881152\\
-3.662	-3.49217122767166\\
-4.883	-4.65654617824159\\
-8.545	-8.14871740591324\\
-6.104	-5.82092112881152\\
-1.221	-1.16437495056993\\
-4.883	-4.65654617824159\\
-6.104	-5.82092112881152\\
-3.662	-3.49217122767166\\
-8.545	-8.14871740591324\\
-10.986	-10.476513683015\\
-14.648	-13.9686849106866\\
-19.531	-18.6252310889282\\
-15.869	-15.1330598612566\\
-7.324	-6.98434245534331\\
-6.104	-5.82092112881152\\
-4.883	-4.65654617824159\\
-3.662	-3.49217122767166\\
-7.324	-6.98434245534331\\
-8.545	-8.14871740591324\\
-6.104	-5.82092112881152\\
-8.545	-8.14871740591324\\
-9.766	-9.31309235648318\\
-10.986	-10.476513683015\\
-12.207	-11.6408886335849\\
-14.648	-13.9686849106866\\
-9.766	-9.31309235648318\\
-6.104	-5.82092112881152\\
-9.766	-9.31309235648318\\
-4.883	-4.65654617824159\\
-7.324	-6.98434245534331\\
-2.441	-2.32779627710172\\
-6.104	-5.82092112881152\\
-3.662	-3.49217122767166\\
-6.104	-5.82092112881152\\
-7.324	-6.98434245534331\\
-10.986	-10.476513683015\\
-13.428	-12.8052635841548\\
-8.545	-8.14871740591324\\
-2.441	-2.32779627710172\\
-6.104	-5.82092112881152\\
-9.766	-9.31309235648318\\
-4.883	-4.65654617824159\\
-1.221	-1.16437495056993\\
-12.207	-11.6408886335849\\
-7.324	-6.98434245534331\\
-14.648	-13.9686849106866\\
-18.311	-17.4618097623964\\
-13.428	-12.8052635841548\\
-8.545	-8.14871740591324\\
};
\end{axis}

\begin{axis}[%
width=4.927cm,
height=3cm,
at={(0cm,4.839cm)},
scale only axis,
xmin=-404.053,
xmax=0,
xlabel style={font=\color{white!15!black}},
xlabel={y(t-1)},
ymin=-500,
ymax=0,
ylabel style={font=\color{white!15!black}},
ylabel={y(t)},
axis background/.style={fill=white},
title style={font=\small},
title={C6, R = 0.7809},
axis x line*=bottom,
axis y line*=left
]
\addplot[only marks, mark=*, mark options={}, mark size=1.5000pt, color=mycolor1, fill=mycolor1] table[row sep=crcr]{%
x	y\\
-122.07	-140.381\\
-140.381	-178.223\\
-178.223	-147.705\\
-147.705	-139.16\\
-139.16	-192.871\\
-192.871	-185.547\\
-185.547	-219.727\\
-219.727	-169.678\\
-169.678	-86.67\\
-86.67	-107.422\\
-107.422	-102.539\\
-102.539	-126.953\\
-126.953	-103.76\\
-103.76	-41.504\\
-41.504	-31.738\\
-31.738	-28.076\\
-28.076	-89.111\\
-89.111	-157.471\\
-157.471	-167.236\\
-167.236	-174.561\\
-174.561	-124.512\\
-124.512	-74.463\\
-74.463	-158.691\\
-158.691	-111.084\\
-111.084	-83.008\\
-83.008	-142.822\\
-142.822	-123.291\\
-123.291	-103.76\\
-103.76	-173.34\\
-173.34	-164.795\\
-164.795	-124.512\\
-124.512	-180.664\\
-180.664	-179.443\\
-179.443	-142.822\\
-142.822	-117.188\\
-117.188	-89.111\\
-89.111	-108.643\\
-108.643	-137.939\\
-137.939	-130.615\\
-130.615	-136.719\\
-136.719	-140.381\\
-140.381	-151.367\\
-151.367	-213.623\\
-213.623	-191.65\\
-191.65	-126.953\\
-126.953	-115.967\\
-115.967	-64.697\\
-64.697	-69.58\\
-69.58	-96.436\\
-96.436	-74.463\\
-74.463	-78.125\\
-78.125	-111.084\\
-111.084	-128.174\\
-128.174	-119.629\\
-119.629	-190.43\\
-190.43	-139.16\\
-139.16	-81.787\\
-81.787	-63.477\\
-63.477	-85.449\\
-85.449	-101.318\\
-101.318	-51.27\\
-51.27	-58.594\\
-58.594	-81.787\\
-81.787	-65.918\\
-65.918	-56.152\\
-56.152	-74.463\\
-74.463	-86.67\\
-86.67	-139.16\\
-139.16	-130.615\\
-130.615	-163.574\\
-163.574	-146.484\\
-146.484	-170.898\\
-170.898	-139.16\\
-139.16	-133.057\\
-133.057	-115.967\\
-115.967	-111.084\\
-111.084	-111.084\\
-111.084	-196.533\\
-196.533	-280.762\\
-280.762	-291.748\\
-291.748	-280.762\\
-280.762	-175.781\\
-175.781	-240.479\\
-240.479	-303.955\\
-303.955	-317.383\\
-317.383	-239.258\\
-239.258	-313.721\\
-313.721	-404.053\\
-404.053	-289.307\\
-289.307	-238.037\\
-238.037	-158.691\\
-158.691	-122.07\\
-122.07	-102.539\\
-102.539	-129.395\\
-129.395	-92.773\\
-92.773	-61.035\\
-61.035	-59.814\\
-59.814	-75.684\\
-75.684	-118.408\\
-118.408	-107.422\\
-107.422	-111.084\\
-111.084	-146.484\\
-146.484	-152.588\\
-152.588	-207.52\\
-207.52	-167.236\\
-167.236	-208.74\\
-208.74	-148.926\\
-148.926	-130.615\\
-130.615	-151.367\\
-151.367	-108.643\\
-108.643	-98.877\\
-98.877	-122.07\\
-122.07	-123.291\\
-123.291	-85.449\\
-85.449	-95.215\\
-95.215	-67.139\\
-67.139	-76.904\\
-76.904	-81.787\\
-81.787	-57.373\\
-57.373	-75.684\\
-75.684	-93.994\\
-93.994	-151.367\\
-151.367	-152.588\\
-152.588	-170.898\\
-170.898	-97.656\\
-97.656	-140.381\\
-140.381	-211.182\\
-211.182	-161.133\\
-161.133	-186.768\\
-186.768	-191.65\\
-191.65	-112.305\\
-112.305	-75.684\\
-75.684	-107.422\\
-107.422	-123.291\\
-123.291	-205.078\\
-205.078	-252.686\\
-252.686	-252.686\\
-252.686	-184.326\\
-184.326	-189.209\\
-189.209	-167.236\\
-167.236	-163.574\\
-163.574	-162.354\\
-162.354	-108.643\\
-108.643	-96.436\\
-96.436	-86.67\\
-86.67	-78.125\\
-78.125	-87.891\\
-87.891	-133.057\\
-133.057	-203.857\\
-203.857	-161.133\\
-161.133	-109.863\\
-109.863	-92.773\\
-92.773	-89.111\\
-89.111	-52.49\\
-52.49	-41.504\\
-41.504	-47.607\\
-47.607	-90.332\\
-90.332	-67.139\\
-67.139	-78.125\\
-78.125	-85.449\\
-85.449	-80.566\\
-80.566	-54.932\\
-54.932	-37.842\\
-37.842	-30.518\\
-30.518	-54.932\\
-54.932	-119.629\\
-119.629	-172.119\\
-172.119	-194.092\\
-194.092	-136.719\\
-136.719	-91.553\\
-91.553	-64.697\\
-64.697	-43.945\\
-43.945	-98.877\\
-98.877	-91.553\\
-91.553	-137.939\\
-137.939	-168.457\\
-168.457	-275.879\\
-275.879	-316.162\\
-316.162	-260.01\\
-260.01	-249.023\\
-249.023	-159.912\\
-159.912	-185.547\\
-185.547	-195.313\\
-195.313	-212.402\\
-212.402	-158.691\\
-158.691	-144.043\\
-144.043	-142.822\\
-142.822	-129.395\\
-129.395	-156.25\\
-156.25	-109.863\\
-109.863	-195.313\\
-195.313	-234.375\\
-234.375	-175.781\\
-175.781	-104.98\\
-104.98	-111.084\\
-111.084	-192.871\\
-192.871	-234.375\\
-234.375	-289.307\\
-289.307	-303.955\\
-303.955	-302.734\\
-302.734	-228.271\\
-228.271	-205.078\\
-205.078	-235.596\\
-235.596	-275.879\\
-275.879	-170.898\\
-170.898	-104.98\\
-104.98	-152.588\\
-152.588	-122.07\\
-122.07	-79.346\\
-79.346	-109.863\\
-109.863	-122.07\\
-122.07	-95.215\\
-95.215	-75.684\\
-75.684	-101.318\\
-101.318	-80.566\\
-80.566	-115.967\\
-115.967	-162.354\\
-162.354	-147.705\\
-147.705	-100.098\\
-100.098	-101.318\\
-101.318	-157.471\\
-157.471	-261.23\\
-261.23	-185.547\\
-185.547	-118.408\\
-118.408	-85.449\\
-85.449	-101.318\\
-101.318	-90.332\\
-90.332	-67.139\\
-67.139	-76.904\\
-76.904	-92.773\\
-92.773	-120.85\\
-120.85	-134.277\\
-134.277	-89.111\\
-89.111	-156.25\\
-156.25	-179.443\\
-179.443	-170.898\\
-170.898	-140.381\\
-140.381	-150.146\\
-150.146	-202.637\\
-202.637	-125.732\\
-125.732	-53.711\\
-53.711	-78.125\\
-78.125	-85.449\\
-85.449	-119.629\\
-119.629	-101.318\\
-101.318	-74.463\\
-74.463	-118.408\\
-118.408	-112.305\\
-112.305	-79.346\\
-79.346	-102.539\\
-102.539	-90.332\\
-90.332	-80.566\\
-80.566	-95.215\\
-95.215	-54.932\\
-54.932	-68.359\\
-68.359	-47.607\\
-47.607	-51.27\\
-51.27	-106.201\\
-106.201	-122.07\\
-122.07	-167.236\\
-167.236	-219.727\\
-219.727	-142.822\\
-142.822	-83.008\\
-83.008	-63.477\\
-63.477	-62.256\\
-62.256	-91.553\\
-91.553	-54.932\\
-54.932	-70.801\\
-70.801	-85.449\\
-85.449	-150.146\\
-150.146	-133.057\\
-133.057	-190.43\\
-190.43	-124.512\\
-124.512	-117.188\\
-117.188	-56.152\\
-56.152	-62.256\\
-62.256	-34.18\\
-34.18	-46.387\\
-46.387	-51.27\\
-51.27	-95.215\\
-95.215	-100.098\\
-100.098	-117.188\\
-117.188	-123.291\\
-123.291	-197.754\\
-197.754	-170.898\\
-170.898	-128.174\\
-128.174	-153.809\\
-153.809	-163.574\\
-163.574	-213.623\\
-213.623	-161.133\\
-161.133	-104.98\\
-104.98	-53.711\\
-53.711	-40.283\\
-40.283	-41.504\\
-41.504	-52.49\\
-52.49	-54.932\\
-54.932	-51.27\\
-51.27	-70.801\\
-70.801	-108.643\\
-108.643	-98.877\\
-98.877	-103.76\\
-103.76	-118.408\\
-118.408	-69.58\\
-69.58	-36.621\\
-36.621	-81.787\\
-81.787	-125.732\\
-125.732	-125.732\\
-125.732	-142.822\\
-142.822	-118.408\\
-118.408	-87.891\\
-87.891	-123.291\\
-123.291	-172.119\\
-172.119	-172.119\\
-172.119	-120.85\\
-120.85	-207.52\\
-207.52	-155.029\\
-155.029	-151.367\\
-151.367	-173.34\\
-173.34	-172.119\\
-172.119	-130.615\\
-130.615	-112.305\\
-112.305	-192.871\\
-192.871	-266.113\\
-266.113	-202.637\\
-202.637	-124.512\\
-124.512	-118.408\\
-118.408	-115.967\\
-115.967	-150.146\\
-150.146	-173.34\\
-173.34	-125.732\\
-125.732	-134.277\\
-134.277	-177.002\\
-177.002	-192.871\\
-192.871	-120.85\\
-120.85	-97.656\\
-97.656	-130.615\\
-130.615	-218.506\\
-218.506	-195.313\\
-195.313	-203.857\\
-203.857	-115.967\\
-115.967	-108.643\\
-108.643	-101.318\\
-101.318	-63.477\\
-63.477	-41.504\\
-41.504	-34.18\\
-34.18	-29.297\\
-29.297	-78.125\\
-78.125	-108.643\\
-108.643	-131.836\\
-131.836	-93.994\\
-93.994	-107.422\\
-107.422	-119.629\\
-119.629	-72.021\\
-72.021	-79.346\\
-79.346	-73.242\\
-73.242	-54.932\\
-54.932	-75.684\\
-75.684	-120.85\\
-120.85	-156.25\\
-156.25	-95.215\\
-95.215	-73.242\\
-73.242	-58.594\\
-58.594	-74.463\\
-74.463	-45.166\\
-45.166	-117.188\\
-117.188	-195.313\\
-195.313	-163.574\\
-163.574	-216.064\\
-216.064	-241.699\\
-241.699	-249.023\\
-249.023	-181.885\\
-181.885	-131.836\\
-131.836	-130.615\\
-130.615	-129.395\\
-129.395	-155.029\\
-155.029	-184.326\\
-184.326	-181.885\\
-181.885	-247.803\\
-247.803	-270.996\\
-270.996	-328.369\\
-328.369	-217.285\\
-217.285	-238.037\\
-238.037	-264.893\\
-264.893	-205.078\\
-205.078	-239.258\\
-239.258	-200.195\\
-200.195	-113.525\\
-113.525	-74.463\\
-74.463	-79.346\\
-79.346	-117.188\\
-117.188	-97.656\\
-97.656	-64.697\\
-64.697	-90.332\\
-90.332	-63.477\\
-63.477	-48.828\\
-48.828	-64.697\\
-64.697	-72.021\\
-72.021	-48.828\\
-48.828	-100.098\\
-100.098	-135.498\\
-135.498	-97.656\\
-97.656	-173.34\\
-173.34	-241.699\\
-241.699	-220.947\\
-220.947	-279.541\\
-279.541	-187.988\\
-187.988	-206.299\\
-206.299	-134.277\\
-134.277	-85.449\\
-85.449	-108.643\\
-108.643	-87.891\\
-87.891	-63.477\\
-63.477	-92.773\\
-92.773	-50.049\\
-50.049	-32.959\\
-32.959	-43.945\\
-43.945	-98.877\\
-98.877	-125.732\\
-125.732	-157.471\\
-157.471	-152.588\\
-152.588	-146.484\\
-146.484	-168.457\\
-168.457	-135.498\\
-135.498	-108.643\\
-108.643	-112.305\\
-112.305	-89.111\\
-89.111	-146.484\\
-146.484	-96.436\\
-96.436	-80.566\\
-80.566	-120.85\\
-120.85	-164.795\\
-164.795	-123.291\\
-123.291	-93.994\\
-93.994	-80.566\\
-80.566	-91.553\\
-91.553	-61.035\\
-61.035	-61.035\\
-61.035	-87.891\\
-87.891	-109.863\\
-109.863	-124.512\\
-124.512	-96.436\\
-96.436	-128.174\\
-128.174	-114.746\\
-114.746	-61.035\\
-61.035	-69.58\\
-69.58	-135.498\\
-135.498	-190.43\\
-190.43	-186.768\\
-186.768	-157.471\\
-157.471	-152.588\\
-152.588	-114.746\\
-114.746	-109.863\\
-109.863	-87.891\\
-87.891	-125.732\\
-125.732	-117.188\\
-117.188	-70.801\\
-70.801	-107.422\\
-107.422	-126.953\\
-126.953	-150.146\\
-150.146	-164.795\\
-164.795	-163.574\\
-163.574	-222.168\\
-222.168	-267.334\\
-267.334	-302.734\\
-302.734	-190.43\\
-190.43	-106.201\\
-106.201	-68.359\\
-68.359	-45.166\\
-45.166	-70.801\\
-70.801	-64.697\\
-64.697	-119.629\\
-119.629	-102.539\\
-102.539	-93.994\\
-93.994	-124.512\\
-124.512	-145.264\\
-145.264	-172.119\\
-172.119	-183.105\\
-183.105	-258.789\\
-258.789	-231.934\\
-231.934	-177.002\\
-177.002	-120.85\\
-120.85	-120.85\\
-120.85	-123.291\\
-123.291	-157.471\\
-157.471	-108.643\\
-108.643	-113.525\\
-113.525	-107.422\\
-107.422	-58.594\\
-58.594	-102.539\\
-102.539	-152.588\\
-152.588	-107.422\\
-107.422	-115.967\\
-115.967	-216.064\\
-216.064	-159.912\\
-159.912	-156.25\\
-156.25	-219.727\\
-219.727	-206.299\\
-206.299	-129.395\\
-129.395	-85.449\\
-85.449	-85.449\\
-85.449	-146.484\\
-146.484	-236.816\\
-236.816	-246.582\\
-246.582	-251.465\\
-251.465	-192.871\\
-192.871	-122.07\\
-122.07	-103.76\\
-103.76	-72.021\\
-72.021	-69.58\\
-69.58	-58.594\\
-58.594	-89.111\\
-89.111	-102.539\\
-102.539	-58.594\\
-58.594	-90.332\\
-90.332	-126.953\\
-126.953	-145.264\\
-145.264	-185.547\\
-185.547	-233.154\\
-233.154	-281.982\\
-281.982	-197.754\\
-197.754	-172.119\\
-172.119	-178.223\\
-178.223	-156.25\\
-156.25	-103.76\\
-103.76	-128.174\\
-128.174	-214.844\\
-214.844	-246.582\\
-246.582	-141.602\\
-141.602	-79.346\\
-79.346	-52.49\\
-52.49	-28.076\\
-28.076	-35.4\\
-35.4	-64.697\\
-64.697	-74.463\\
-74.463	-102.539\\
-102.539	-81.787\\
-81.787	-64.697\\
-64.697	-62.256\\
-62.256	-61.035\\
-61.035	-54.932\\
-54.932	-65.918\\
-65.918	-102.539\\
-102.539	-148.926\\
-148.926	-157.471\\
-157.471	-155.029\\
-155.029	-183.105\\
-183.105	-225.83\\
-225.83	-225.83\\
-225.83	-270.996\\
-270.996	-247.803\\
-247.803	-184.326\\
-184.326	-125.732\\
-125.732	-122.07\\
-122.07	-205.078\\
-205.078	-236.816\\
-236.816	-166.016\\
-166.016	-107.422\\
-107.422	-53.711\\
-53.711	-40.283\\
-40.283	-42.725\\
-42.725	-25.635\\
-25.635	-61.035\\
-61.035	-84.229\\
-84.229	-89.111\\
-89.111	-78.125\\
-78.125	-47.607\\
-47.607	-35.4\\
-35.4	-52.49\\
-52.49	-56.152\\
-56.152	-62.256\\
-62.256	-47.607\\
-47.607	-37.842\\
-37.842	-69.58\\
-69.58	-164.795\\
-164.795	-190.43\\
-190.43	-217.285\\
-217.285	-147.705\\
-147.705	-208.74\\
-208.74	-291.748\\
-291.748	-261.23\\
-261.23	-273.438\\
-273.438	-200.195\\
-200.195	-202.637\\
-202.637	-212.402\\
-212.402	-139.16\\
-139.16	-133.057\\
-133.057	-80.566\\
-80.566	-75.684\\
-75.684	-135.498\\
-135.498	-93.994\\
-93.994	-104.98\\
-104.98	-115.967\\
-115.967	-111.084\\
-111.084	-113.525\\
-113.525	-120.85\\
-120.85	-136.719\\
-136.719	-147.705\\
-147.705	-125.732\\
-125.732	-92.773\\
-92.773	-120.85\\
-120.85	-57.373\\
-57.373	-26.855\\
-26.855	-46.387\\
-46.387	-73.242\\
-73.242	-37.842\\
-37.842	-17.09\\
-17.09	-39.063\\
-39.063	-70.801\\
-70.801	-84.229\\
-84.229	-104.98\\
-104.98	-87.891\\
-87.891	-142.822\\
-142.822	-123.291\\
-123.291	-83.008\\
-83.008	-125.732\\
-125.732	-70.801\\
-70.801	-56.152\\
-56.152	-39.063\\
-39.063	-24.414\\
-24.414	-48.828\\
-48.828	-36.621\\
-36.621	-65.918\\
-65.918	-111.084\\
-111.084	-106.201\\
-106.201	-76.904\\
-76.904	-86.67\\
-86.67	-64.697\\
-64.697	-92.773\\
-92.773	-67.139\\
-67.139	-63.477\\
-63.477	-58.594\\
-58.594	-43.945\\
-43.945	-57.373\\
-57.373	-51.27\\
-51.27	-90.332\\
-90.332	-83.008\\
-83.008	-67.139\\
-67.139	-64.697\\
-64.697	-51.27\\
-51.27	-61.035\\
-61.035	-135.498\\
-135.498	-179.443\\
-179.443	-123.291\\
-123.291	-114.746\\
-114.746	-80.566\\
-80.566	-136.719\\
-136.719	-125.732\\
-125.732	-80.566\\
-80.566	-102.539\\
-102.539	-76.904\\
-76.904	-108.643\\
-108.643	-64.697\\
-64.697	-64.697\\
-64.697	-80.566\\
-80.566	-104.98\\
-104.98	-74.463\\
-74.463	-81.787\\
-81.787	-85.449\\
-85.449	-135.498\\
-135.498	-115.967\\
-115.967	-178.223\\
-178.223	-231.934\\
-231.934	-183.105\\
-183.105	-117.188\\
-117.188	-86.67\\
-86.67	-135.498\\
-135.498	-173.34\\
-173.34	-133.057\\
-133.057	-163.574\\
-163.574	-98.877\\
-98.877	-50.049\\
-50.049	-52.49\\
-52.49	-119.629\\
-119.629	-107.422\\
-107.422	-100.098\\
-100.098	-156.25\\
-156.25	-146.484\\
-146.484	-133.057\\
-133.057	-90.332\\
-90.332	-114.746\\
-114.746	-148.926\\
-148.926	-120.85\\
-120.85	-125.732\\
-125.732	-112.305\\
-112.305	-80.566\\
-80.566	-97.656\\
-97.656	-70.801\\
-70.801	-79.346\\
-79.346	-135.498\\
-135.498	-156.25\\
-156.25	-164.795\\
-164.795	-146.484\\
-146.484	-104.98\\
-104.98	-72.021\\
-72.021	-85.449\\
-85.449	-101.318\\
-101.318	-129.395\\
-129.395	-159.912\\
-159.912	-190.43\\
-190.43	-146.484\\
-146.484	-84.229\\
-84.229	-155.029\\
-155.029	-222.168\\
-222.168	-155.029\\
-155.029	-155.029\\
-155.029	-181.885\\
-181.885	-137.939\\
-137.939	-113.525\\
-113.525	-129.395\\
-129.395	-115.967\\
-115.967	-131.836\\
-131.836	-167.236\\
-167.236	-125.732\\
-125.732	-146.484\\
-146.484	-137.939\\
-137.939	-141.602\\
-141.602	-109.863\\
-109.863	-144.043\\
-144.043	-102.539\\
-102.539	-106.201\\
-106.201	-119.629\\
-119.629	-115.967\\
-115.967	-58.594\\
-58.594	-35.4\\
-35.4	-28.076\\
-28.076	-47.607\\
-47.607	-122.07\\
-122.07	-161.133\\
-161.133	-161.133\\
-161.133	-101.318\\
-101.318	-119.629\\
-119.629	-104.98\\
-104.98	-92.773\\
-92.773	-58.594\\
-58.594	-81.787\\
-81.787	-81.787\\
-81.787	-80.566\\
-80.566	-95.215\\
-95.215	-79.346\\
-79.346	-75.684\\
-75.684	-50.049\\
-50.049	-72.021\\
-72.021	-139.16\\
-139.16	-96.436\\
-96.436	-119.629\\
-119.629	-107.422\\
-107.422	-145.264\\
-145.264	-135.498\\
-135.498	-186.768\\
-186.768	-137.939\\
-137.939	-151.367\\
-151.367	-155.029\\
-155.029	-72.021\\
-72.021	-131.836\\
-131.836	-95.215\\
-95.215	-113.525\\
-113.525	-128.174\\
-128.174	-166.016\\
-166.016	-123.291\\
-123.291	-158.691\\
-158.691	-201.416\\
-201.416	-164.795\\
-164.795	-142.822\\
-142.822	-113.525\\
-113.525	-136.719\\
-136.719	-113.525\\
-113.525	-96.436\\
-96.436	-81.787\\
-81.787	-96.436\\
-96.436	-83.008\\
-83.008	-172.119\\
-172.119	-225.83\\
-225.83	-191.65\\
-191.65	-187.988\\
-187.988	-181.885\\
-181.885	-101.318\\
-101.318	-89.111\\
-89.111	-72.021\\
-72.021	-62.256\\
-62.256	-68.359\\
-68.359	-115.967\\
-115.967	-146.484\\
-146.484	-167.236\\
-167.236	-141.602\\
-141.602	-93.994\\
-93.994	-169.678\\
-169.678	-201.416\\
-201.416	-224.609\\
-224.609	-178.223\\
-178.223	-288.086\\
-288.086	-209.961\\
-209.961	-117.188\\
-117.188	-84.229\\
-84.229	-69.58\\
-69.58	-125.732\\
-125.732	-162.354\\
-162.354	-218.506\\
-218.506	-166.016\\
-166.016	-126.953\\
-126.953	-69.58\\
-69.58	-72.021\\
-72.021	-78.125\\
-78.125	-87.891\\
-87.891	-56.152\\
-56.152	-70.801\\
-70.801	-59.814\\
-59.814	-36.621\\
-36.621	-83.008\\
-83.008	-93.994\\
-93.994	-118.408\\
-118.408	-107.422\\
-107.422	-112.305\\
-112.305	-113.525\\
-113.525	-146.484\\
-146.484	-100.098\\
-100.098	-142.822\\
-142.822	-169.678\\
-169.678	-130.615\\
-130.615	-139.16\\
-139.16	-119.629\\
-119.629	-135.498\\
-135.498	-191.65\\
-191.65	-148.926\\
-148.926	-114.746\\
-114.746	-128.174\\
-128.174	-115.967\\
-115.967	-79.346\\
-79.346	-122.07\\
-122.07	-181.885\\
-181.885	-170.898\\
-170.898	-187.988\\
-187.988	-279.541\\
-279.541	-187.988\\
-187.988	-159.912\\
-159.912	-120.85\\
-120.85	-152.588\\
-152.588	-224.609\\
-224.609	-148.926\\
-148.926	-129.395\\
-129.395	-186.768\\
-186.768	-119.629\\
-119.629	-68.359\\
-68.359	-87.891\\
-87.891	-68.359\\
-68.359	-52.49\\
-52.49	-57.373\\
-57.373	-50.049\\
-50.049	-48.828\\
-48.828	-64.697\\
-64.697	-95.215\\
-95.215	-107.422\\
-107.422	-147.705\\
-147.705	-139.16\\
-139.16	-166.016\\
-166.016	-194.092\\
-194.092	-174.561\\
-174.561	-125.732\\
-125.732	-134.277\\
-134.277	-119.629\\
-119.629	-130.615\\
-130.615	-183.105\\
-183.105	-140.381\\
-140.381	-158.691\\
-158.691	-136.719\\
-136.719	-129.395\\
-129.395	-201.416\\
-201.416	-166.016\\
-166.016	-167.236\\
-167.236	-152.588\\
-152.588	-91.553\\
-91.553	-58.594\\
-58.594	-46.387\\
-46.387	-84.229\\
-84.229	-79.346\\
-79.346	-106.201\\
-106.201	-79.346\\
-79.346	-108.643\\
-108.643	-191.65\\
-191.65	-235.596\\
-235.596	-185.547\\
-185.547	-190.43\\
-190.43	-169.678\\
-169.678	-118.408\\
-118.408	-133.057\\
-133.057	-145.264\\
-145.264	-125.732\\
-125.732	-144.043\\
-144.043	-181.885\\
-181.885	-260.01\\
-260.01	-230.713\\
-230.713	-214.844\\
-214.844	-239.258\\
-239.258	-162.354\\
-162.354	-173.34\\
-173.34	-139.16\\
-139.16	-141.602\\
-141.602	-186.768\\
-186.768	-213.623\\
-213.623	-84.229\\
-84.229	-100.098\\
-100.098	-85.449\\
-85.449	-119.629\\
-119.629	-123.291\\
-123.291	-75.684\\
-75.684	-98.877\\
-98.877	-173.34\\
-173.34	-303.955\\
-303.955	-289.307\\
-289.307	-261.23\\
-261.23	-211.182\\
-211.182	-240.479\\
-240.479	-290.527\\
-290.527	-217.285\\
-217.285	-218.506\\
-218.506	-225.83\\
-225.83	-152.588\\
-152.588	-131.836\\
-131.836	-169.678\\
-169.678	-115.967\\
-115.967	-81.787\\
-81.787	-114.746\\
-114.746	-84.229\\
-84.229	-98.877\\
-98.877	-95.215\\
-95.215	-115.967\\
-115.967	-158.691\\
-158.691	-126.953\\
-126.953	-90.332\\
-90.332	-112.305\\
-112.305	-129.395\\
-129.395	-153.809\\
-153.809	-129.395\\
-129.395	-106.201\\
-106.201	-167.236\\
-167.236	-159.912\\
-159.912	-111.084\\
-111.084	-183.105\\
-183.105	-168.457\\
-168.457	-119.629\\
-119.629	-81.787\\
-81.787	-58.594\\
-58.594	-45.166\\
-45.166	-98.877\\
-98.877	-96.436\\
-96.436	-65.918\\
-65.918	-47.607\\
-47.607	-57.373\\
-57.373	-102.539\\
-102.539	-119.629\\
-119.629	-158.691\\
-158.691	-180.664\\
-180.664	-172.119\\
-172.119	-181.885\\
-181.885	-112.305\\
-112.305	-46.387\\
-46.387	-70.801\\
-70.801	-62.256\\
-62.256	-78.125\\
-78.125	-126.953\\
-126.953	-139.16\\
-139.16	-201.416\\
-201.416	-133.057\\
-133.057	-124.512\\
-124.512	-191.65\\
-191.65	-142.822\\
-142.822	-95.215\\
-95.215	-69.58\\
-69.58	-93.994\\
-93.994	-129.395\\
-129.395	-190.43\\
-190.43	-130.615\\
-130.615	-106.201\\
-106.201	-98.877\\
-98.877	-122.07\\
-122.07	-163.574\\
-163.574	-255.127\\
-255.127	-219.727\\
-219.727	-214.844\\
-214.844	-137.939\\
-137.939	-90.332\\
-90.332	-107.422\\
-107.422	-50.049\\
-50.049	-74.463\\
-74.463	-69.58\\
-69.58	-79.346\\
-79.346	-134.277\\
-134.277	-185.547\\
-185.547	-217.285\\
-217.285	-278.32\\
-278.32	-239.258\\
-239.258	-129.395\\
-129.395	-118.408\\
-118.408	-98.877\\
-98.877	-136.719\\
-136.719	-179.443\\
-179.443	-239.258\\
-239.258	-179.443\\
-179.443	-122.07\\
-122.07	-139.16\\
-139.16	-184.326\\
-184.326	-133.057\\
-133.057	-89.111\\
-89.111	-73.242\\
-73.242	-50.049\\
-50.049	-48.828\\
-48.828	-36.621\\
-36.621	-64.697\\
-64.697	-98.877\\
-98.877	-107.422\\
-107.422	-80.566\\
-80.566	-73.242\\
-73.242	-34.18\\
-34.18	-17.09\\
-17.09	-30.518\\
-30.518	-57.373\\
-57.373	-87.891\\
-87.891	-111.084\\
-111.084	-130.615\\
-130.615	-150.146\\
-150.146	-190.43\\
-190.43	-266.113\\
-266.113	-261.23\\
-261.23	-280.762\\
-280.762	-249.023\\
-249.023	-136.719\\
-136.719	-123.291\\
-123.291	-125.732\\
-125.732	-166.016\\
-166.016	-145.264\\
-145.264	-102.539\\
-102.539	-79.346\\
-79.346	-63.477\\
-63.477	-111.084\\
-111.084	-108.643\\
-108.643	-56.152\\
-56.152	-41.504\\
-41.504	-69.58\\
-69.58	-96.436\\
-96.436	-76.904\\
-76.904	-106.201\\
-106.201	-86.67\\
-86.67	-122.07\\
-122.07	-181.885\\
-181.885	-146.484\\
-146.484	-190.43\\
-190.43	-260.01\\
-260.01	-281.982\\
-281.982	-223.389\\
-223.389	-145.264\\
-145.264	-107.422\\
-107.422	-64.697\\
-64.697	-101.318\\
-101.318	-75.684\\
-75.684	-126.953\\
-126.953	-115.967\\
-115.967	-92.773\\
-92.773	-119.629\\
-119.629	-69.58\\
-69.58	-98.877\\
-98.877	-75.684\\
-75.684	-46.387\\
-46.387	-54.932\\
-54.932	-42.725\\
-42.725	-47.607\\
-47.607	-46.387\\
-46.387	-30.518\\
-30.518	-42.725\\
-42.725	-62.256\\
-62.256	-61.035\\
-61.035	-61.035\\
-61.035	-97.656\\
-97.656	-130.615\\
-130.615	-108.643\\
-108.643	-98.877\\
-98.877	-156.25\\
-156.25	-190.43\\
-190.43	-150.146\\
-150.146	-156.25\\
-156.25	-74.463\\
-74.463	-43.945\\
-43.945	-90.332\\
-90.332	-133.057\\
-133.057	-145.264\\
-145.264	-202.637\\
-202.637	-183.105\\
-183.105	-130.615\\
-130.615	-89.111\\
-89.111	-93.994\\
-93.994	-102.539\\
-102.539	-81.787\\
-81.787	-48.828\\
-48.828	-64.697\\
-64.697	-52.49\\
-52.49	-41.504\\
-41.504	-32.959\\
-32.959	-61.035\\
-61.035	-93.994\\
-93.994	-100.098\\
-100.098	-69.58\\
-69.58	-123.291\\
-123.291	-172.119\\
-172.119	-109.863\\
-109.863	-146.484\\
-146.484	-163.574\\
-163.574	-118.408\\
-118.408	-69.58\\
-69.58	-86.67\\
-86.67	-108.643\\
-108.643	-59.814\\
-59.814	-69.58\\
-69.58	-53.711\\
-53.711	-31.738\\
-31.738	-31.738\\
-31.738	-56.152\\
-56.152	-76.904\\
-76.904	-67.139\\
-67.139	-83.008\\
-83.008	-100.098\\
-100.098	-73.242\\
-73.242	-37.842\\
-37.842	-32.959\\
-32.959	-41.504\\
-41.504	-50.049\\
-50.049	-61.035\\
-61.035	-128.174\\
-128.174	-130.615\\
-130.615	-109.863\\
-109.863	-107.422\\
-107.422	-141.602\\
-141.602	-185.547\\
-185.547	-201.416\\
-201.416	-205.078\\
-205.078	-151.367\\
-151.367	-157.471\\
-157.471	-162.354\\
-162.354	-166.016\\
-166.016	-185.547\\
-185.547	-246.582\\
-246.582	-216.064\\
-216.064	-195.313\\
-195.313	-157.471\\
-157.471	-147.705\\
-147.705	-142.822\\
-142.822	-167.236\\
-167.236	-103.76\\
-103.76	-120.85\\
-120.85	-148.926\\
-148.926	-179.443\\
-179.443	-137.939\\
-137.939	-155.029\\
-155.029	-129.395\\
-129.395	-112.305\\
-112.305	-70.801\\
-70.801	-111.084\\
-111.084	-177.002\\
-177.002	-141.602\\
-141.602	-81.787\\
-81.787	-97.656\\
-97.656	-53.711\\
-53.711	-50.049\\
-50.049	-70.801\\
-70.801	-45.166\\
-45.166	-46.387\\
-46.387	-54.932\\
-54.932	-36.621\\
-36.621	-62.256\\
-62.256	-51.27\\
-51.27	-30.518\\
-30.518	-57.373\\
-57.373	-64.697\\
-64.697	-80.566\\
-80.566	-75.684\\
-75.684	-76.904\\
-76.904	-104.98\\
-104.98	-92.773\\
-92.773	-104.98\\
-104.98	-180.664\\
-180.664	-128.174\\
-128.174	-112.305\\
-112.305	-122.07\\
-122.07	-86.67\\
-86.67	-51.27\\
-51.27	-92.773\\
-92.773	-73.242\\
-73.242	-43.945\\
-43.945	-81.787\\
-81.787	-80.566\\
-80.566	-98.877\\
-98.877	-61.035\\
-61.035	-36.621\\
-36.621	-29.297\\
-29.297	-36.621\\
-36.621	-69.58\\
-69.58	-68.359\\
-68.359	-80.566\\
-80.566	-109.863\\
-109.863	-150.146\\
-150.146	-109.863\\
-109.863	-152.588\\
-152.588	-177.002\\
-177.002	-153.809\\
-153.809	-145.264\\
-145.264	-230.713\\
-230.713	-167.236\\
-167.236	-208.74\\
-208.74	-178.223\\
-178.223	-203.857\\
-203.857	-234.375\\
-234.375	-136.719\\
-136.719	-81.787\\
-81.787	-65.918\\
-65.918	-108.643\\
-108.643	-134.277\\
-134.277	-137.939\\
-137.939	-90.332\\
-90.332	-50.049\\
-50.049	-67.139\\
-67.139	-128.174\\
-128.174	-179.443\\
-179.443	-177.002\\
-177.002	-103.76\\
-103.76	-81.787\\
-81.787	-109.863\\
-109.863	-175.781\\
-175.781	-200.195\\
-200.195	-203.857\\
-203.857	-150.146\\
-150.146	-208.74\\
-208.74	-227.051\\
-227.051	-212.402\\
-212.402	-290.527\\
-290.527	-251.465\\
-251.465	-209.961\\
-209.961	-241.699\\
-241.699	-207.52\\
-207.52	-106.201\\
-106.201	-100.098\\
-100.098	-126.953\\
-126.953	-155.029\\
-155.029	-158.691\\
-158.691	-115.967\\
-115.967	-148.926\\
-148.926	-130.615\\
-130.615	-96.436\\
-96.436	-129.395\\
-129.395	-122.07\\
-122.07	-175.781\\
-175.781	-181.885\\
-181.885	-134.277\\
-134.277	-97.656\\
-97.656	-58.594\\
-58.594	-96.436\\
-96.436	-74.463\\
-74.463	-65.918\\
-65.918	-53.711\\
-53.711	-95.215\\
-95.215	-123.291\\
-123.291	-137.939\\
-137.939	-136.719\\
-136.719	-80.566\\
-80.566	-62.256\\
-62.256	-42.725\\
-42.725	-54.932\\
-54.932	-91.553\\
-91.553	-95.215\\
-95.215	-91.553\\
-91.553	-150.146\\
-150.146	-174.561\\
-174.561	-198.975\\
-198.975	-198.975\\
-198.975	-229.492\\
-229.492	-137.939\\
-137.939	-113.525\\
-113.525	-87.891\\
-87.891	-93.994\\
-93.994	-128.174\\
-128.174	-106.201\\
-106.201	-69.58\\
-69.58	-64.697\\
-64.697	-124.512\\
-124.512	-153.809\\
-153.809	-134.277\\
-134.277	-211.182\\
-211.182	-230.713\\
-230.713	-159.912\\
-159.912	-111.084\\
-111.084	-76.904\\
-76.904	-70.801\\
-70.801	-131.836\\
-131.836	-106.201\\
-106.201	-122.07\\
-122.07	-139.16\\
-139.16	-156.25\\
-156.25	-151.367\\
-151.367	-168.457\\
-168.457	-115.967\\
-115.967	-130.615\\
-130.615	-91.553\\
-91.553	-84.229\\
-84.229	-129.395\\
-129.395	-183.105\\
-183.105	-201.416\\
-201.416	-106.201\\
-106.201	-220.947\\
-220.947	-207.52\\
-207.52	-139.16\\
-139.16	-76.904\\
-76.904	-123.291\\
-123.291	-107.422\\
-107.422	-104.98\\
-104.98	-123.291\\
-123.291	-80.566\\
-80.566	-108.643\\
-108.643	-207.52\\
-207.52	-219.727\\
-219.727	-145.264\\
-145.264	-159.912\\
-159.912	-177.002\\
-177.002	-180.664\\
-180.664	-131.836\\
-131.836	-128.174\\
-128.174	-84.229\\
-84.229	-133.057\\
-133.057	-134.277\\
-134.277	-229.492\\
-229.492	-251.465\\
-251.465	-332.031\\
-332.031	-238.037\\
-238.037	-194.092\\
-194.092	-147.705\\
-147.705	-162.354\\
-162.354	-122.07\\
-122.07	-84.229\\
-84.229	-81.787\\
-81.787	-85.449\\
-85.449	-62.256\\
-62.256	-46.387\\
-46.387	-74.463\\
-74.463	-100.098\\
-100.098	-68.359\\
-68.359	-46.387\\
-46.387	-46.387\\
-46.387	-112.305\\
-112.305	-157.471\\
-157.471	-73.242\\
-73.242	-95.215\\
-95.215	-97.656\\
-97.656	-91.553\\
-91.553	-113.525\\
-113.525	-98.877\\
-98.877	-68.359\\
-68.359	-48.828\\
-48.828	-40.283\\
-40.283	-53.711\\
-53.711	-106.201\\
-106.201	-126.953\\
-126.953	-96.436\\
-96.436	-63.477\\
-63.477	-39.063\\
-39.063	-56.152\\
-56.152	-96.436\\
-96.436	-123.291\\
-123.291	-133.057\\
-133.057	-90.332\\
-90.332	-86.67\\
-86.67	-95.215\\
-95.215	-133.057\\
-133.057	-185.547\\
-185.547	-216.064\\
-216.064	-164.795\\
-164.795	-102.539\\
-102.539	-107.422\\
-107.422	-100.098\\
-100.098	-103.76\\
-103.76	-65.918\\
-65.918	-87.891\\
-87.891	-93.994\\
-93.994	-91.553\\
-91.553	-56.152\\
-56.152	-36.621\\
-36.621	-56.152\\
-56.152	-84.229\\
-84.229	-134.277\\
-134.277	-148.926\\
-148.926	-179.443\\
-179.443	-159.912\\
-159.912	-180.664\\
-180.664	-239.258\\
-239.258	-319.824\\
-319.824	-253.906\\
-253.906	-173.34\\
-173.34	-189.209\\
-189.209	-220.947\\
-220.947	-288.086\\
-288.086	-180.664\\
-180.664	-89.111\\
-89.111	-59.814\\
-59.814	-62.256\\
-62.256	-42.725\\
-42.725	-28.076\\
-28.076	-39.063\\
-39.063	-59.814\\
-59.814	-83.008\\
-83.008	-86.67\\
-86.67	-65.918\\
-65.918	-62.256\\
-62.256	-78.125\\
-78.125	-100.098\\
-100.098	-104.98\\
-104.98	-96.436\\
-96.436	-68.359\\
-68.359	-47.607\\
-47.607	-45.166\\
-45.166	-72.021\\
-72.021	-87.891\\
-87.891	-64.697\\
-64.697	-41.504\\
-41.504	-76.904\\
-76.904	-48.828\\
-48.828	-69.58\\
-69.58	-79.346\\
-79.346	-120.85\\
-120.85	-126.953\\
-126.953	-84.229\\
-84.229	-125.732\\
-125.732	-123.291\\
-123.291	-101.318\\
-101.318	-84.229\\
-84.229	-142.822\\
-142.822	-175.781\\
-175.781	-175.781\\
-175.781	-194.092\\
-194.092	-147.705\\
-147.705	-136.719\\
-136.719	-126.953\\
-126.953	-173.34\\
-173.34	-135.498\\
-135.498	-184.326\\
-184.326	-172.119\\
-172.119	-178.223\\
-178.223	-303.955\\
-303.955	-231.934\\
-231.934	-125.732\\
-125.732	-83.008\\
-83.008	-87.891\\
-87.891	-112.305\\
-112.305	-146.484\\
-146.484	-172.119\\
-172.119	-189.209\\
-189.209	-192.871\\
-192.871	-125.732\\
-125.732	-111.084\\
-111.084	-130.615\\
-130.615	-129.395\\
-129.395	-76.904\\
-76.904	-57.373\\
-57.373	-103.76\\
-103.76	-101.318\\
-101.318	-142.822\\
-142.822	-174.561\\
-174.561	-155.029\\
-155.029	-106.201\\
-106.201	-85.449\\
-85.449	-135.498\\
-135.498	-168.457\\
-168.457	-230.713\\
-230.713	-244.141\\
-244.141	-286.865\\
-286.865	-236.816\\
-236.816	-216.064\\
-216.064	-123.291\\
-123.291	-85.449\\
-85.449	-153.809\\
-153.809	-202.637\\
-202.637	-115.967\\
-115.967	-186.768\\
-186.768	-249.023\\
-249.023	-308.838\\
-308.838	-189.209\\
-189.209	-112.305\\
-112.305	-63.477\\
-63.477	-95.215\\
-95.215	-85.449\\
-85.449	-93.994\\
-93.994	-51.27\\
-51.27	-76.904\\
-76.904	-52.49\\
-52.49	-74.463\\
-74.463	-58.594\\
-58.594	-79.346\\
-79.346	-124.512\\
-124.512	-112.305\\
-112.305	-162.354\\
-162.354	-191.65\\
-191.65	-190.43\\
-190.43	-103.76\\
-103.76	-109.863\\
-109.863	-130.615\\
-130.615	-85.449\\
-85.449	-80.566\\
-80.566	-54.932\\
-54.932	-90.332\\
-90.332	-79.346\\
-79.346	-47.607\\
-47.607	-26.855\\
-26.855	-31.738\\
-31.738	-58.594\\
-58.594	-85.449\\
-85.449	-91.553\\
-91.553	-56.152\\
-56.152	-67.139\\
-67.139	-102.539\\
-102.539	-135.498\\
-135.498	-93.994\\
-93.994	-87.891\\
-87.891	-104.98\\
-104.98	-125.732\\
-125.732	-115.967\\
-115.967	-137.939\\
-137.939	-126.953\\
-126.953	-89.111\\
-89.111	-72.021\\
-72.021	-95.215\\
-95.215	-78.125\\
-78.125	-97.656\\
-97.656	-129.395\\
-129.395	-203.857\\
-203.857	-139.16\\
-139.16	-140.381\\
-140.381	-206.299\\
-206.299	-152.588\\
-152.588	-96.436\\
-96.436	-69.58\\
-69.58	-54.932\\
-54.932	-58.594\\
-58.594	-46.387\\
-46.387	-50.049\\
-50.049	-106.201\\
-106.201	-45.166\\
-45.166	-50.049\\
-50.049	-65.918\\
-65.918	-93.994\\
-93.994	-69.58\\
-69.58	-53.711\\
-53.711	-28.076\\
-28.076	-54.932\\
-54.932	-101.318\\
-101.318	-139.16\\
-139.16	-108.643\\
-108.643	-89.111\\
-89.111	-101.318\\
-101.318	-102.539\\
-102.539	-119.629\\
-119.629	-144.043\\
-144.043	-168.457\\
-168.457	-164.795\\
-164.795	-140.381\\
-140.381	-91.553\\
-91.553	-73.242\\
-73.242	-79.346\\
-79.346	-112.305\\
-112.305	-64.697\\
-64.697	-62.256\\
-62.256	-114.746\\
-114.746	-139.16\\
-139.16	-133.057\\
-133.057	-157.471\\
-157.471	-205.078\\
-205.078	-104.98\\
-104.98	-184.326\\
-184.326	-185.547\\
-185.547	-170.898\\
-170.898	-184.326\\
-184.326	-181.885\\
-181.885	-312.5\\
-312.5	-286.865\\
-286.865	-195.313\\
-195.313	-236.816\\
-236.816	-283.203\\
-283.203	-275.879\\
-275.879	-311.279\\
-311.279	-281.982\\
-281.982	-372.314\\
-372.314	-280.762\\
-280.762	-196.533\\
-196.533	-118.408\\
-118.408	-124.512\\
-124.512	-92.773\\
-92.773	-79.346\\
-79.346	-123.291\\
-123.291	-172.119\\
-172.119	-104.98\\
-104.98	-92.773\\
-92.773	-91.553\\
-91.553	-59.814\\
-59.814	-80.566\\
-80.566	-36.621\\
-36.621	-54.932\\
-54.932	-43.945\\
-43.945	-65.918\\
-65.918	-45.166\\
-45.166	-40.283\\
-40.283	-23.193\\
-23.193	-23.193\\
-23.193	-46.387\\
-46.387	-30.518\\
-30.518	-37.842\\
-37.842	-81.787\\
-81.787	-112.305\\
-112.305	-115.967\\
-115.967	-81.787\\
-81.787	-107.422\\
-107.422	-166.016\\
-166.016	-80.566\\
-80.566	-48.828\\
-48.828	-37.842\\
-37.842	-28.076\\
-28.076	-48.828\\
-48.828	-81.787\\
-81.787	-123.291\\
-123.291	-72.021\\
-72.021	-129.395\\
-129.395	-173.34\\
-173.34	-175.781\\
-175.781	-130.615\\
-130.615	-75.684\\
-75.684	-100.098\\
-100.098	-135.498\\
-135.498	-178.223\\
-178.223	-93.994\\
-93.994	-53.711\\
-53.711	-84.229\\
-84.229	-63.477\\
-63.477	-31.738\\
-31.738	-21.973\\
-21.973	-54.932\\
-54.932	-126.953\\
-126.953	-126.953\\
-126.953	-92.773\\
-92.773	-57.373\\
-57.373	-61.035\\
-61.035	-18.311\\
-18.311	-35.4\\
-35.4	-30.518\\
-30.518	-58.594\\
-58.594	-84.229\\
-84.229	-131.836\\
-131.836	-135.498\\
-135.498	-148.926\\
-148.926	-153.809\\
-153.809	-148.926\\
-148.926	-145.264\\
-145.264	-125.732\\
-125.732	-155.029\\
-155.029	-236.816\\
-236.816	-213.623\\
-213.623	-111.084\\
-111.084	-58.594\\
-58.594	-135.498\\
-135.498	-158.691\\
-158.691	-142.822\\
-142.822	-78.125\\
-78.125	-83.008\\
-83.008	-142.822\\
-142.822	-217.285\\
-217.285	-224.609\\
-224.609	-250.244\\
-250.244	-239.258\\
-239.258	-178.223\\
-178.223	-183.105\\
-183.105	-84.229\\
-84.229	-79.346\\
-79.346	-101.318\\
-101.318	-126.953\\
-126.953	-134.277\\
-134.277	-142.822\\
-142.822	-86.67\\
-86.67	-124.512\\
-124.512	-102.539\\
-102.539	-59.814\\
-59.814	-39.063\\
-39.063	-32.959\\
-32.959	-53.711\\
-53.711	-40.283\\
-40.283	-23.193\\
-23.193	-48.828\\
-48.828	-100.098\\
-100.098	-64.697\\
-64.697	-65.918\\
-65.918	-86.67\\
-86.67	-146.484\\
-146.484	-97.656\\
-97.656	-52.49\\
-52.49	-29.297\\
-29.297	-81.787\\
-81.787	-159.912\\
-159.912	-201.416\\
-201.416	-280.762\\
-280.762	-333.252\\
-333.252	-323.486\\
-323.486	-207.52\\
-207.52	-212.402\\
-212.402	-277.1\\
-277.1	-231.934\\
-231.934	-214.844\\
-214.844	-242.92\\
-242.92	-263.672\\
-263.672	-270.996\\
-270.996	-357.666\\
-357.666	-238.037\\
-238.037	-135.498\\
-135.498	-70.801\\
-70.801	-59.814\\
-59.814	-67.139\\
-67.139	-67.139\\
-67.139	-69.58\\
-69.58	-126.953\\
-126.953	-93.994\\
-93.994	-107.422\\
-107.422	-136.719\\
-136.719	-70.801\\
-70.801	-85.449\\
-85.449	-117.188\\
-117.188	-140.381\\
-140.381	-76.904\\
-76.904	-51.27\\
-51.27	-68.359\\
-68.359	-64.697\\
-64.697	-167.236\\
-167.236	-227.051\\
-227.051	-214.844\\
-214.844	-249.023\\
-249.023	-274.658\\
-274.658	-360.107\\
-360.107	-306.396\\
-306.396	-190.43\\
-190.43	-125.732\\
-125.732	-69.58\\
-69.58	-78.125\\
-78.125	-81.787\\
-81.787	-109.863\\
-109.863	-72.021\\
-72.021	-68.359\\
-68.359	-75.684\\
-75.684	-95.215\\
-95.215	-107.422\\
-107.422	-135.498\\
-135.498	-95.215\\
-95.215	-45.166\\
-45.166	-34.18\\
-34.18	-28.076\\
-28.076	-32.959\\
-32.959	-39.063\\
-39.063	-72.021\\
-72.021	-73.242\\
-73.242	-95.215\\
-95.215	-140.381\\
-140.381	-100.098\\
-100.098	-168.457\\
-168.457	-244.141\\
-244.141	-190.43\\
-190.43	-172.119\\
-172.119	-207.52\\
-207.52	-239.258\\
-239.258	-148.926\\
-148.926	-130.615\\
-130.615	-80.566\\
-80.566	-92.773\\
-92.773	-187.988\\
-187.988	-314.941\\
-314.941	-371.094\\
-371.094	-294.189\\
-294.189	-247.803\\
-247.803	-175.781\\
-175.781	-191.65\\
-191.65	-256.348\\
-256.348	-147.705\\
-147.705	-128.174\\
-128.174	-96.436\\
-96.436	-185.547\\
-185.547	-173.34\\
-173.34	-92.773\\
-92.773	-90.332\\
-90.332	-65.918\\
-65.918	-119.629\\
-119.629	-180.664\\
-180.664	-186.768\\
-186.768	-141.602\\
-141.602	-104.98\\
-104.98	-115.967\\
-115.967	-89.111\\
-89.111	-64.697\\
-64.697	-52.49\\
-52.49	-45.166\\
-45.166	-36.621\\
-36.621	-46.387\\
-46.387	-76.904\\
-76.904	-113.525\\
-113.525	-113.525\\
-113.525	-72.021\\
-72.021	-62.256\\
-62.256	-51.27\\
-51.27	-68.359\\
-68.359	-95.215\\
-95.215	-103.76\\
-103.76	-100.098\\
-100.098	-54.932\\
-54.932	-123.291\\
-123.291	-172.119\\
-172.119	-124.512\\
-124.512	-50.049\\
-50.049	-61.035\\
-61.035	-95.215\\
-95.215	-95.215\\
-95.215	-84.229\\
-84.229	-51.27\\
-51.27	-93.994\\
-93.994	-144.043\\
-144.043	-86.67\\
-86.67	-32.959\\
-32.959	-42.725\\
-42.725	-111.084\\
-111.084	-139.16\\
-139.16	-81.787\\
-81.787	-68.359\\
-68.359	-133.057\\
-133.057	-179.443\\
-179.443	-157.471\\
-157.471	-222.168\\
-222.168	-268.555\\
-268.555	-173.34\\
-173.34	-139.16\\
-139.16	-197.754\\
-197.754	-133.057\\
-133.057	-179.443\\
-179.443	-220.947\\
-220.947	-172.119\\
-172.119	-84.229\\
-84.229	-142.822\\
-142.822	-238.037\\
-238.037	-280.762\\
-280.762	-205.078\\
-205.078	-175.781\\
-175.781	-229.492\\
-229.492	-175.781\\
-175.781	-112.305\\
-112.305	-101.318\\
-101.318	-130.615\\
-130.615	-136.719\\
-136.719	-136.719\\
-136.719	-155.029\\
-155.029	-106.201\\
-106.201	-114.746\\
-114.746	-163.574\\
-163.574	-92.773\\
-92.773	-76.904\\
-76.904	-61.035\\
-61.035	-48.828\\
-48.828	-57.373\\
-57.373	-102.539\\
-102.539	-93.994\\
-93.994	-133.057\\
-133.057	-169.678\\
-169.678	-202.637\\
-202.637	-157.471\\
-157.471	-136.719\\
-136.719	-61.035\\
-61.035	-95.215\\
-95.215	-167.236\\
-167.236	-111.084\\
-111.084	-56.152\\
-56.152	-113.525\\
-113.525	-172.119\\
-172.119	-178.223\\
-178.223	-233.154\\
-233.154	-305.176\\
-305.176	-213.623\\
-213.623	-181.885\\
-181.885	-216.064\\
-216.064	-140.381\\
-140.381	-113.525\\
-113.525	-129.395\\
-129.395	-164.795\\
-164.795	-174.561\\
-174.561	-177.002\\
-177.002	-97.656\\
-97.656	-86.67\\
-86.67	-69.58\\
-69.58	-101.318\\
-101.318	-92.773\\
-92.773	-74.463\\
-74.463	-151.367\\
-151.367	-162.354\\
-162.354	-112.305\\
-112.305	-93.994\\
-93.994	-136.719\\
-136.719	-68.359\\
-68.359	-43.945\\
-43.945	-54.932\\
-54.932	-78.125\\
-78.125	-68.359\\
-68.359	-47.607\\
-47.607	-26.855\\
-26.855	-48.828\\
-48.828	-78.125\\
-78.125	-124.512\\
-124.512	-142.822\\
-142.822	-187.988\\
-187.988	-209.961\\
-209.961	-107.422\\
-107.422	-83.008\\
-83.008	-178.223\\
-178.223	-261.23\\
-261.23	-202.637\\
-202.637	-189.209\\
-189.209	-283.203\\
-283.203	-227.051\\
-227.051	-185.547\\
-185.547	-133.057\\
-133.057	-140.381\\
-140.381	-185.547\\
-185.547	-129.395\\
-129.395	-112.305\\
-112.305	-189.209\\
-189.209	-236.816\\
-236.816	-247.803\\
-247.803	-111.084\\
-111.084	-52.49\\
-52.49	-141.602\\
-141.602	-111.084\\
-111.084	-52.49\\
-52.49	-37.842\\
-37.842	-68.359\\
-68.359	-103.76\\
-103.76	-167.236\\
-167.236	-117.188\\
-117.188	-86.67\\
-86.67	-52.49\\
-52.49	-34.18\\
-34.18	-48.828\\
-48.828	-40.283\\
-40.283	-56.152\\
-56.152	-96.436\\
-96.436	-87.891\\
-87.891	-43.945\\
-43.945	-39.063\\
-39.063	-52.49\\
-52.49	-67.139\\
-67.139	-39.063\\
-39.063	-62.256\\
-62.256	-45.166\\
-45.166	-34.18\\
-34.18	-79.346\\
-79.346	-120.85\\
-120.85	-175.781\\
-175.781	-239.258\\
-239.258	-207.52\\
-207.52	-100.098\\
-100.098	-70.801\\
-70.801	-74.463\\
-74.463	-41.504\\
-41.504	-36.621\\
-36.621	-85.449\\
-85.449	-91.553\\
-91.553	-87.891\\
-87.891	-103.76\\
-103.76	-122.07\\
-122.07	-130.615\\
-130.615	-135.498\\
-135.498	-166.016\\
-166.016	-152.588\\
-152.588	-223.389\\
-223.389	-115.967\\
-115.967	-51.27\\
-51.27	-50.049\\
-50.049	-95.215\\
-95.215	-74.463\\
-74.463	-67.139\\
-67.139	-31.738\\
-31.738	-48.828\\
-48.828	-37.842\\
-37.842	-18.311\\
-18.311	-32.959\\
-32.959	-69.58\\
-69.58	-87.891\\
-87.891	-130.615\\
-130.615	-183.105\\
-183.105	-128.174\\
-128.174	-51.27\\
-51.27	-93.994\\
-93.994	-106.201\\
-106.201	-50.049\\
-50.049	-68.359\\
-68.359	-139.16\\
-139.16	-128.174\\
-128.174	-207.52\\
-207.52	-240.479\\
-240.479	-148.926\\
-148.926	-114.746\\
-114.746	-100.098\\
};
\addplot [color=mycolor2, line width=2.0pt, forget plot]
  table[row sep=crcr]{%
-122.07	-116.783530728682\\
-140.381	-134.301538684551\\
-178.223	-170.504720218383\\
-147.705	-141.308359189646\\
-139.16	-133.133416369324\\
-192.871	-184.518361228571\\
-185.547	-177.511540723477\\
-219.727	-210.211312004761\\
-169.678	-162.329777398062\\
-86.67	-82.9165938253044\\
-107.422	-102.769889718494\\
-102.539	-98.0983571507198\\
-126.953	-121.455063296456\\
-103.76	-99.2664794659465\\
-41.504	-39.7065917863786\\
-31.738	-30.3635266508308\\
-28.076	-26.8601163982837\\
-89.111	-85.2518817626249\\
-157.471	-150.651424325193\\
-167.236	-159.993532767608\\
-174.561	-167.001309965835\\
-124.512	-119.119775359136\\
-74.463	-71.2382407524361\\
-158.691	-151.818589947287\\
-111.084	-106.273299971041\\
-83.008	-79.4131835727572\\
-142.822	-136.636826621872\\
-123.291	-117.951653043909\\
-103.76	-99.2664794659465\\
-173.34	-165.833187650609\\
-164.795	-157.658244830288\\
-124.512	-119.119775359136\\
-180.664	-172.840008155703\\
-179.443	-171.671885840476\\
-142.822	-136.636826621872\\
-117.188	-112.112954854041\\
-89.111	-85.2518817626249\\
-108.643	-103.93801203372\\
-137.939	-131.965294054098\\
-130.615	-124.958473549003\\
-136.719	-130.798128432004\\
-140.381	-134.301538684551\\
-151.367	-144.811769442193\\
-213.623	-204.371657121761\\
-191.65	-183.350238913345\\
-126.953	-121.455063296456\\
-115.967	-110.944832538815\\
-64.697	-61.8951756168884\\
-69.58	-66.5667081846622\\
-96.436	-92.2596589608521\\
-74.463	-71.2382407524361\\
-78.125	-74.7416510049833\\
-111.084	-106.273299971041\\
-128.174	-122.623185611683\\
-119.629	-114.448242791362\\
-190.43	-182.183073291251\\
-139.16	-133.133416369324\\
-81.787	-78.2450612575305\\
-63.477	-60.7280099947946\\
-85.449	-81.7484715100777\\
-101.318	-96.9302348354931\\
-51.27	-49.0496569219263\\
-58.594	-56.0564774270207\\
-81.787	-78.2450612575305\\
-65.918	-63.0632979321151\\
-56.152	-53.7202327965673\\
-74.463	-71.2382407524361\\
-86.67	-82.9165938253044\\
-139.16	-133.133416369324\\
-130.615	-124.958473549003\\
-163.574	-156.490122515061\\
-146.484	-140.140236874419\\
-170.898	-163.496943020155\\
-139.16	-133.133416369324\\
-133.057	-127.294718179457\\
-115.967	-110.944832538815\\
-111.084	-106.273299971041\\
-196.533	-188.021771481118\\
-280.762	-268.603077369102\\
-291.748	-279.113308126744\\
-280.762	-268.603077369102\\
-175.781	-168.168475587929\\
-240.479	-230.06460789795\\
-303.955	-290.791661199612\\
-317.383	-303.638136587707\\
-239.258	-228.896485582724\\
-313.721	-300.13472633516\\
-404.053	-386.554730413011\\
-289.307	-276.778020189423\\
-238.037	-227.728363267497\\
-158.691	-151.818589947287\\
-122.07	-116.783530728682\\
-102.539	-98.0983571507198\\
-129.395	-123.79130792691\\
-92.773	-88.755292015172\\
-61.035	-58.3917653643412\\
-59.814	-57.2236430491145\\
-75.684	-72.4063630676628\\
-118.408	-113.280120476135\\
-107.422	-102.769889718494\\
-111.084	-106.273299971041\\
-146.484	-140.140236874419\\
-152.588	-145.979891757419\\
-207.52	-198.532958931893\\
-167.236	-159.993532767608\\
-208.74	-199.700124553987\\
-148.926	-142.476481504872\\
-130.615	-124.958473549003\\
-151.367	-144.811769442193\\
-108.643	-103.93801203372\\
-98.877	-94.5949468981726\\
-122.07	-116.783530728682\\
-123.291	-117.951653043909\\
-85.449	-81.7484715100777\\
-95.215	-91.0915366456254\\
-67.139	-64.2314202473417\\
-76.904	-73.5735286897566\\
-81.787	-78.2450612575305\\
-57.373	-54.888355111794\\
-75.684	-72.4063630676628\\
-93.994	-89.9234143303987\\
-151.367	-144.811769442193\\
-152.588	-145.979891757419\\
-170.898	-163.496943020155\\
-97.656	-93.4268245829459\\
-140.381	-134.301538684551\\
-211.182	-202.03636918444\\
-161.133	-154.154834577741\\
-186.768	-178.679663038704\\
-191.65	-183.350238913345\\
-112.305	-107.441422286268\\
-75.684	-72.4063630676628\\
-107.422	-102.769889718494\\
-123.291	-117.951653043909\\
-205.078	-196.19671430144\\
-252.686	-241.742960970819\\
-184.326	-176.34341840825\\
-189.209	-181.014950976024\\
-167.236	-159.993532767608\\
-163.574	-156.490122515061\\
-162.354	-155.322956892967\\
-108.643	-103.93801203372\\
-96.436	-92.2596589608521\\
-86.67	-82.9165938253044\\
-78.125	-74.7416510049833\\
-87.891	-84.084716140531\\
-133.057	-127.294718179457\\
-203.857	-195.028591986213\\
-161.133	-154.154834577741\\
-109.863	-105.105177655814\\
-92.773	-88.755292015172\\
-89.111	-85.2518817626249\\
-52.49	-50.2168225440201\\
-41.504	-39.7065917863786\\
-47.607	-45.5452899762463\\
-90.332	-86.4200040778515\\
-67.139	-64.2314202473417\\
-78.125	-74.7416510049833\\
-85.449	-81.7484715100777\\
-80.566	-77.0769389423038\\
-54.932	-52.5530671744735\\
-37.842	-36.2031815338314\\
-30.518	-29.196361028737\\
-54.932	-52.5530671744735\\
-119.629	-114.448242791362\\
-172.119	-164.665065335382\\
-194.092	-185.686483543798\\
-136.719	-130.798128432004\\
-91.553	-87.5881263930782\\
-64.697	-61.8951756168884\\
-43.945	-42.0418797236991\\
-98.877	-94.5949468981726\\
-91.553	-87.5881263930782\\
-137.939	-131.965294054098\\
-168.457	-161.161655082835\\
-275.879	-263.931544801329\\
-316.162	-302.47001427248\\
-260.01	-248.749781475913\\
-249.023	-238.238594025139\\
-159.912	-152.986712262514\\
-185.547	-177.511540723477\\
-195.313	-186.854605859025\\
-212.402	-203.203534806534\\
-158.691	-151.818589947287\\
-144.043	-137.804948937098\\
-142.822	-136.636826621872\\
-129.395	-123.79130792691\\
-156.25	-149.483302009967\\
-109.863	-105.105177655814\\
-195.313	-186.854605859025\\
-234.375	-224.22495301495\\
-175.781	-168.168475587929\\
-104.98	-100.43364508804\\
-111.084	-106.273299971041\\
-192.871	-184.518361228571\\
-234.375	-224.22495301495\\
-289.307	-276.778020189423\\
-303.955	-290.791661199612\\
-302.734	-289.623538884385\\
-228.271	-218.385298131949\\
-205.078	-196.19671430144\\
-235.596	-225.393075330177\\
-275.879	-263.931544801329\\
-170.898	-163.496943020155\\
-104.98	-100.43364508804\\
-152.588	-145.979891757419\\
-122.07	-116.783530728682\\
-79.346	-75.90977332021\\
-109.863	-105.105177655814\\
-122.07	-116.783530728682\\
-95.215	-91.0915366456254\\
-75.684	-72.4063630676628\\
-101.318	-96.9302348354931\\
-80.566	-77.0769389423038\\
-115.967	-110.944832538815\\
-162.354	-155.322956892967\\
-147.705	-141.308359189646\\
-100.098	-95.7630692133993\\
-101.318	-96.9302348354931\\
-157.471	-150.651424325193\\
-261.23	-249.916947098007\\
-185.547	-177.511540723477\\
-118.408	-113.280120476135\\
-85.449	-81.7484715100777\\
-101.318	-96.9302348354931\\
-90.332	-86.4200040778515\\
-67.139	-64.2314202473417\\
-76.904	-73.5735286897566\\
-92.773	-88.755292015172\\
-120.85	-115.616365106589\\
-134.277	-128.461883801551\\
-89.111	-85.2518817626249\\
-156.25	-149.483302009967\\
-179.443	-171.671885840476\\
-170.898	-163.496943020155\\
-140.381	-134.301538684551\\
-150.146	-143.643647126966\\
-202.637	-193.861426364119\\
-125.732	-120.28694098123\\
-53.711	-51.3849448592468\\
-78.125	-74.7416510049833\\
-85.449	-81.7484715100777\\
-119.629	-114.448242791362\\
-101.318	-96.9302348354931\\
-74.463	-71.2382407524361\\
-118.408	-113.280120476135\\
-112.305	-107.441422286268\\
-79.346	-75.90977332021\\
-102.539	-98.0983571507198\\
-90.332	-86.4200040778515\\
-80.566	-77.0769389423038\\
-95.215	-91.0915366456254\\
-54.932	-52.5530671744735\\
-68.359	-65.3985858694356\\
-47.607	-45.5452899762463\\
-51.27	-49.0496569219263\\
-106.201	-101.601767403267\\
-122.07	-116.783530728682\\
-167.236	-159.993532767608\\
-219.727	-210.211312004761\\
-142.822	-136.636826621872\\
-83.008	-79.4131835727572\\
-63.477	-60.7280099947946\\
-62.256	-59.5598876795679\\
-91.553	-87.5881263930782\\
-54.932	-52.5530671744735\\
-70.801	-67.7348304998889\\
-85.449	-81.7484715100777\\
-150.146	-143.643647126966\\
-133.057	-127.294718179457\\
-190.43	-182.183073291251\\
-124.512	-119.119775359136\\
-117.188	-112.112954854041\\
-56.152	-53.7202327965673\\
-62.256	-59.5598876795679\\
-34.18	-32.6997712812842\\
-46.387	-44.3781243541525\\
-51.27	-49.0496569219263\\
-95.215	-91.0915366456254\\
-100.098	-95.7630692133993\\
-117.188	-112.112954854041\\
-123.291	-117.951653043909\\
-197.754	-189.189893796345\\
-170.898	-163.496943020155\\
-128.174	-122.623185611683\\
-153.809	-147.148014072646\\
-163.574	-156.490122515061\\
-213.623	-204.371657121761\\
-161.133	-154.154834577741\\
-104.98	-100.43364508804\\
-53.711	-51.3849448592468\\
-40.283	-38.5384694711519\\
-41.504	-39.7065917863786\\
-52.49	-50.2168225440201\\
-54.932	-52.5530671744735\\
-51.27	-49.0496569219263\\
-70.801	-67.7348304998889\\
-108.643	-103.93801203372\\
-98.877	-94.5949468981726\\
-103.76	-99.2664794659465\\
-118.408	-113.280120476135\\
-69.58	-66.5667081846622\\
-36.621	-35.0350592186047\\
-81.787	-78.2450612575305\\
-125.732	-120.28694098123\\
-142.822	-136.636826621872\\
-118.408	-113.280120476135\\
-87.891	-84.084716140531\\
-123.291	-117.951653043909\\
-172.119	-164.665065335382\\
-120.85	-115.616365106589\\
-207.52	-198.532958931893\\
-155.029	-148.31517969474\\
-151.367	-144.811769442193\\
-173.34	-165.833187650609\\
-172.119	-164.665065335382\\
-130.615	-124.958473549003\\
-112.305	-107.441422286268\\
-192.871	-184.518361228571\\
-266.113	-254.588479665781\\
-202.637	-193.861426364119\\
-124.512	-119.119775359136\\
-118.408	-113.280120476135\\
-115.967	-110.944832538815\\
-150.146	-143.643647126966\\
-173.34	-165.833187650609\\
-125.732	-120.28694098123\\
-134.277	-128.461883801551\\
-177.002	-169.336597903156\\
-192.871	-184.518361228571\\
-120.85	-115.616365106589\\
-97.656	-93.4268245829459\\
-130.615	-124.958473549003\\
-218.506	-209.043189689535\\
-195.313	-186.854605859025\\
-203.857	-195.028591986213\\
-115.967	-110.944832538815\\
-108.643	-103.93801203372\\
-101.318	-96.9302348354931\\
-63.477	-60.7280099947946\\
-41.504	-39.7065917863786\\
-34.18	-32.6997712812842\\
-29.297	-28.0282387135103\\
-78.125	-74.7416510049833\\
-108.643	-103.93801203372\\
-131.836	-126.12659586423\\
-93.994	-89.9234143303987\\
-107.422	-102.769889718494\\
-119.629	-114.448242791362\\
-72.021	-68.9019961219828\\
-79.346	-75.90977332021\\
-73.242	-70.0701184372094\\
-54.932	-52.5530671744735\\
-75.684	-72.4063630676628\\
-120.85	-115.616365106589\\
-156.25	-149.483302009967\\
-95.215	-91.0915366456254\\
-73.242	-70.0701184372094\\
-58.594	-56.0564774270207\\
-74.463	-71.2382407524361\\
-45.166	-43.2100020389258\\
-117.188	-112.112954854041\\
-195.313	-186.854605859025\\
-163.574	-156.490122515061\\
-216.064	-206.706945059081\\
-241.699	-231.231773520044\\
-249.023	-238.238594025139\\
-181.885	-174.00813047093\\
-131.836	-126.12659586423\\
-130.615	-124.958473549003\\
-129.395	-123.79130792691\\
-155.029	-148.31517969474\\
-184.326	-176.34341840825\\
-181.885	-174.00813047093\\
-247.803	-237.071428403045\\
-270.996	-259.260012233555\\
-328.369	-314.148367345349\\
-217.285	-207.875067374308\\
-238.037	-227.728363267497\\
-264.893	-253.421314043687\\
-205.078	-196.19671430144\\
-239.258	-228.896485582724\\
-200.195	-191.525181733666\\
-113.525	-108.608587908361\\
-74.463	-71.2382407524361\\
-79.346	-75.90977332021\\
-117.188	-112.112954854041\\
-97.656	-93.4268245829459\\
-64.697	-61.8951756168884\\
-90.332	-86.4200040778515\\
-63.477	-60.7280099947946\\
-48.828	-46.713412291473\\
-64.697	-61.8951756168884\\
-72.021	-68.9019961219828\\
-48.828	-46.713412291473\\
-100.098	-95.7630692133993\\
-135.498	-129.630006116777\\
-97.656	-93.4268245829459\\
-173.34	-165.833187650609\\
-241.699	-231.231773520044\\
-220.947	-211.378477626855\\
-279.541	-267.434955053876\\
-187.988	-179.846828660797\\
-206.299	-197.364836616666\\
-134.277	-128.461883801551\\
-85.449	-81.7484715100777\\
-108.643	-103.93801203372\\
-87.891	-84.084716140531\\
-63.477	-60.7280099947946\\
-92.773	-88.755292015172\\
-50.049	-47.8815346066996\\
-32.959	-31.5316489660575\\
-43.945	-42.0418797236991\\
-98.877	-94.5949468981726\\
-125.732	-120.28694098123\\
-157.471	-150.651424325193\\
-152.588	-145.979891757419\\
-146.484	-140.140236874419\\
-168.457	-161.161655082835\\
-135.498	-129.630006116777\\
-108.643	-103.93801203372\\
-112.305	-107.441422286268\\
-89.111	-85.2518817626249\\
-146.484	-140.140236874419\\
-96.436	-92.2596589608521\\
-80.566	-77.0769389423038\\
-120.85	-115.616365106589\\
-164.795	-157.658244830288\\
-123.291	-117.951653043909\\
-93.994	-89.9234143303987\\
-80.566	-77.0769389423038\\
-91.553	-87.5881263930782\\
-61.035	-58.3917653643412\\
-87.891	-84.084716140531\\
-109.863	-105.105177655814\\
-124.512	-119.119775359136\\
-96.436	-92.2596589608521\\
-128.174	-122.623185611683\\
-114.746	-109.776710223588\\
-61.035	-58.3917653643412\\
-69.58	-66.5667081846622\\
-135.498	-129.630006116777\\
-190.43	-182.183073291251\\
-186.768	-178.679663038704\\
-157.471	-150.651424325193\\
-152.588	-145.979891757419\\
-114.746	-109.776710223588\\
-109.863	-105.105177655814\\
-87.891	-84.084716140531\\
-125.732	-120.28694098123\\
-117.188	-112.112954854041\\
-70.801	-67.7348304998889\\
-107.422	-102.769889718494\\
-126.953	-121.455063296456\\
-150.146	-143.643647126966\\
-164.795	-157.658244830288\\
-163.574	-156.490122515061\\
-222.168	-212.546599942082\\
-267.334	-255.756601981007\\
-302.734	-289.623538884385\\
-190.43	-182.183073291251\\
-106.201	-101.601767403267\\
-68.359	-65.3985858694356\\
-45.166	-43.2100020389258\\
-70.801	-67.7348304998889\\
-64.697	-61.8951756168884\\
-119.629	-114.448242791362\\
-102.539	-98.0983571507198\\
-93.994	-89.9234143303987\\
-124.512	-119.119775359136\\
-145.264	-138.973071252325\\
-172.119	-164.665065335382\\
-183.105	-175.175296093024\\
-258.789	-247.581659160686\\
-231.934	-221.889665077629\\
-177.002	-169.336597903156\\
-120.85	-115.616365106589\\
-123.291	-117.951653043909\\
-157.471	-150.651424325193\\
-108.643	-103.93801203372\\
-113.525	-108.608587908361\\
-107.422	-102.769889718494\\
-58.594	-56.0564774270207\\
-102.539	-98.0983571507198\\
-152.588	-145.979891757419\\
-107.422	-102.769889718494\\
-115.967	-110.944832538815\\
-216.064	-206.706945059081\\
-159.912	-152.986712262514\\
-156.25	-149.483302009967\\
-219.727	-210.211312004761\\
-206.299	-197.364836616666\\
-129.395	-123.79130792691\\
-85.449	-81.7484715100777\\
-146.484	-140.140236874419\\
-236.816	-226.56024095227\\
-246.582	-235.903306087818\\
-251.465	-240.574838655592\\
-192.871	-184.518361228571\\
-122.07	-116.783530728682\\
-103.76	-99.2664794659465\\
-72.021	-68.9019961219828\\
-69.58	-66.5667081846622\\
-58.594	-56.0564774270207\\
-89.111	-85.2518817626249\\
-102.539	-98.0983571507198\\
-58.594	-56.0564774270207\\
-90.332	-86.4200040778515\\
-126.953	-121.455063296456\\
-145.264	-138.973071252325\\
-185.547	-177.511540723477\\
-233.154	-223.056830699723\\
-281.982	-269.770242991196\\
-197.754	-189.189893796345\\
-172.119	-164.665065335382\\
-178.223	-170.504720218383\\
-156.25	-149.483302009967\\
-103.76	-99.2664794659465\\
-128.174	-122.623185611683\\
-214.844	-205.539779436987\\
-246.582	-235.903306087818\\
-141.602	-135.469660999778\\
-79.346	-75.90977332021\\
-52.49	-50.2168225440201\\
-28.076	-26.8601163982837\\
-35.4	-33.866936903378\\
-64.697	-61.8951756168884\\
-74.463	-71.2382407524361\\
-102.539	-98.0983571507198\\
-81.787	-78.2450612575305\\
-64.697	-61.8951756168884\\
-62.256	-59.5598876795679\\
-61.035	-58.3917653643412\\
-54.932	-52.5530671744735\\
-65.918	-63.0632979321151\\
-102.539	-98.0983571507198\\
-148.926	-142.476481504872\\
-157.471	-150.651424325193\\
-155.029	-148.31517969474\\
-183.105	-175.175296093024\\
-225.83	-216.050010194629\\
-270.996	-259.260012233555\\
-247.803	-237.071428403045\\
-184.326	-176.34341840825\\
-125.732	-120.28694098123\\
-122.07	-116.783530728682\\
-205.078	-196.19671430144\\
-236.816	-226.56024095227\\
-166.016	-158.826367145514\\
-107.422	-102.769889718494\\
-53.711	-51.3849448592468\\
-40.283	-38.5384694711519\\
-42.725	-40.8747141016053\\
-25.635	-24.5248284609632\\
-61.035	-58.3917653643412\\
-84.229	-80.5813058879839\\
-89.111	-85.2518817626249\\
-78.125	-74.7416510049833\\
-47.607	-45.5452899762463\\
-35.4	-33.866936903378\\
-52.49	-50.2168225440201\\
-56.152	-53.7202327965673\\
-62.256	-59.5598876795679\\
-47.607	-45.5452899762463\\
-37.842	-36.2031815338314\\
-69.58	-66.5667081846622\\
-164.795	-157.658244830288\\
-190.43	-182.183073291251\\
-217.285	-207.875067374308\\
-147.705	-141.308359189646\\
-208.74	-199.700124553987\\
-291.748	-279.113308126744\\
-261.23	-249.916947098007\\
-273.438	-261.596256864008\\
-200.195	-191.525181733666\\
-202.637	-193.861426364119\\
-212.402	-203.203534806534\\
-139.16	-133.133416369324\\
-133.057	-127.294718179457\\
-80.566	-77.0769389423038\\
-75.684	-72.4063630676628\\
-135.498	-129.630006116777\\
-93.994	-89.9234143303987\\
-104.98	-100.43364508804\\
-115.967	-110.944832538815\\
-111.084	-106.273299971041\\
-113.525	-108.608587908361\\
-120.85	-115.616365106589\\
-136.719	-130.798128432004\\
-147.705	-141.308359189646\\
-125.732	-120.28694098123\\
-92.773	-88.755292015172\\
-120.85	-115.616365106589\\
-57.373	-54.888355111794\\
-26.855	-25.691994083057\\
-46.387	-44.3781243541525\\
-73.242	-70.0701184372094\\
-37.842	-36.2031815338314\\
-17.09	-16.3498856406421\\
-39.063	-37.3713038490581\\
-70.801	-67.7348304998889\\
-84.229	-80.5813058879839\\
-104.98	-100.43364508804\\
-87.891	-84.084716140531\\
-142.822	-136.636826621872\\
-123.291	-117.951653043909\\
-83.008	-79.4131835727572\\
-125.732	-120.28694098123\\
-70.801	-67.7348304998889\\
-56.152	-53.7202327965673\\
-39.063	-37.3713038490581\\
-24.414	-23.3567061457365\\
-48.828	-46.713412291473\\
-36.621	-35.0350592186047\\
-65.918	-63.0632979321151\\
-111.084	-106.273299971041\\
-106.201	-101.601767403267\\
-76.904	-73.5735286897566\\
-86.67	-82.9165938253044\\
-64.697	-61.8951756168884\\
-92.773	-88.755292015172\\
-67.139	-64.2314202473417\\
-63.477	-60.7280099947946\\
-58.594	-56.0564774270207\\
-43.945	-42.0418797236991\\
-57.373	-54.888355111794\\
-51.27	-49.0496569219263\\
-90.332	-86.4200040778515\\
-83.008	-79.4131835727572\\
-67.139	-64.2314202473417\\
-64.697	-61.8951756168884\\
-51.27	-49.0496569219263\\
-61.035	-58.3917653643412\\
-135.498	-129.630006116777\\
-179.443	-171.671885840476\\
-123.291	-117.951653043909\\
-114.746	-109.776710223588\\
-80.566	-77.0769389423038\\
-136.719	-130.798128432004\\
-125.732	-120.28694098123\\
-80.566	-77.0769389423038\\
-102.539	-98.0983571507198\\
-76.904	-73.5735286897566\\
-108.643	-103.93801203372\\
-64.697	-61.8951756168884\\
-80.566	-77.0769389423038\\
-104.98	-100.43364508804\\
-74.463	-71.2382407524361\\
-81.787	-78.2450612575305\\
-85.449	-81.7484715100777\\
-135.498	-129.630006116777\\
-115.967	-110.944832538815\\
-178.223	-170.504720218383\\
-231.934	-221.889665077629\\
-183.105	-175.175296093024\\
-117.188	-112.112954854041\\
-86.67	-82.9165938253044\\
-135.498	-129.630006116777\\
-173.34	-165.833187650609\\
-133.057	-127.294718179457\\
-163.574	-156.490122515061\\
-98.877	-94.5949468981726\\
-50.049	-47.8815346066996\\
-52.49	-50.2168225440201\\
-119.629	-114.448242791362\\
-107.422	-102.769889718494\\
-100.098	-95.7630692133993\\
-156.25	-149.483302009967\\
-146.484	-140.140236874419\\
-133.057	-127.294718179457\\
-90.332	-86.4200040778515\\
-114.746	-109.776710223588\\
-148.926	-142.476481504872\\
-120.85	-115.616365106589\\
-125.732	-120.28694098123\\
-112.305	-107.441422286268\\
-80.566	-77.0769389423038\\
-97.656	-93.4268245829459\\
-70.801	-67.7348304998889\\
-79.346	-75.90977332021\\
-135.498	-129.630006116777\\
-156.25	-149.483302009967\\
-164.795	-157.658244830288\\
-146.484	-140.140236874419\\
-104.98	-100.43364508804\\
-72.021	-68.9019961219828\\
-85.449	-81.7484715100777\\
-101.318	-96.9302348354931\\
-129.395	-123.79130792691\\
-159.912	-152.986712262514\\
-190.43	-182.183073291251\\
-146.484	-140.140236874419\\
-84.229	-80.5813058879839\\
-155.029	-148.31517969474\\
-222.168	-212.546599942082\\
-155.029	-148.31517969474\\
-181.885	-174.00813047093\\
-137.939	-131.965294054098\\
-113.525	-108.608587908361\\
-129.395	-123.79130792691\\
-115.967	-110.944832538815\\
-131.836	-126.12659586423\\
-167.236	-159.993532767608\\
-125.732	-120.28694098123\\
-146.484	-140.140236874419\\
-137.939	-131.965294054098\\
-141.602	-135.469660999778\\
-109.863	-105.105177655814\\
-144.043	-137.804948937098\\
-102.539	-98.0983571507198\\
-106.201	-101.601767403267\\
-119.629	-114.448242791362\\
-115.967	-110.944832538815\\
-58.594	-56.0564774270207\\
-35.4	-33.866936903378\\
-28.076	-26.8601163982837\\
-47.607	-45.5452899762463\\
-122.07	-116.783530728682\\
-161.133	-154.154834577741\\
-101.318	-96.9302348354931\\
-119.629	-114.448242791362\\
-104.98	-100.43364508804\\
-92.773	-88.755292015172\\
-58.594	-56.0564774270207\\
-81.787	-78.2450612575305\\
-80.566	-77.0769389423038\\
-95.215	-91.0915366456254\\
-79.346	-75.90977332021\\
-75.684	-72.4063630676628\\
-50.049	-47.8815346066996\\
-72.021	-68.9019961219828\\
-139.16	-133.133416369324\\
-96.436	-92.2596589608521\\
-119.629	-114.448242791362\\
-107.422	-102.769889718494\\
-145.264	-138.973071252325\\
-135.498	-129.630006116777\\
-186.768	-178.679663038704\\
-137.939	-131.965294054098\\
-151.367	-144.811769442193\\
-155.029	-148.31517969474\\
-72.021	-68.9019961219828\\
-131.836	-126.12659586423\\
-95.215	-91.0915366456254\\
-113.525	-108.608587908361\\
-128.174	-122.623185611683\\
-166.016	-158.826367145514\\
-123.291	-117.951653043909\\
-158.691	-151.818589947287\\
-201.416	-192.693304048892\\
-164.795	-157.658244830288\\
-142.822	-136.636826621872\\
-113.525	-108.608587908361\\
-136.719	-130.798128432004\\
-113.525	-108.608587908361\\
-96.436	-92.2596589608521\\
-81.787	-78.2450612575305\\
-96.436	-92.2596589608521\\
-83.008	-79.4131835727572\\
-172.119	-164.665065335382\\
-225.83	-216.050010194629\\
-191.65	-183.350238913345\\
-187.988	-179.846828660797\\
-181.885	-174.00813047093\\
-101.318	-96.9302348354931\\
-89.111	-85.2518817626249\\
-72.021	-68.9019961219828\\
-62.256	-59.5598876795679\\
-68.359	-65.3985858694356\\
-115.967	-110.944832538815\\
-146.484	-140.140236874419\\
-167.236	-159.993532767608\\
-141.602	-135.469660999778\\
-93.994	-89.9234143303987\\
-169.678	-162.329777398062\\
-201.416	-192.693304048892\\
-224.609	-214.881887879402\\
-178.223	-170.504720218383\\
-288.086	-275.609897874197\\
-209.961	-200.868246869213\\
-117.188	-112.112954854041\\
-84.229	-80.5813058879839\\
-69.58	-66.5667081846622\\
-125.732	-120.28694098123\\
-162.354	-155.322956892967\\
-218.506	-209.043189689535\\
-166.016	-158.826367145514\\
-126.953	-121.455063296456\\
-69.58	-66.5667081846622\\
-72.021	-68.9019961219828\\
-78.125	-74.7416510049833\\
-87.891	-84.084716140531\\
-56.152	-53.7202327965673\\
-70.801	-67.7348304998889\\
-59.814	-57.2236430491145\\
-36.621	-35.0350592186047\\
-83.008	-79.4131835727572\\
-93.994	-89.9234143303987\\
-118.408	-113.280120476135\\
-107.422	-102.769889718494\\
-112.305	-107.441422286268\\
-113.525	-108.608587908361\\
-146.484	-140.140236874419\\
-100.098	-95.7630692133993\\
-142.822	-136.636826621872\\
-169.678	-162.329777398062\\
-130.615	-124.958473549003\\
-139.16	-133.133416369324\\
-119.629	-114.448242791362\\
-135.498	-129.630006116777\\
-191.65	-183.350238913345\\
-148.926	-142.476481504872\\
-114.746	-109.776710223588\\
-128.174	-122.623185611683\\
-115.967	-110.944832538815\\
-79.346	-75.90977332021\\
-122.07	-116.783530728682\\
-181.885	-174.00813047093\\
-170.898	-163.496943020155\\
-187.988	-179.846828660797\\
-279.541	-267.434955053876\\
-187.988	-179.846828660797\\
-159.912	-152.986712262514\\
-120.85	-115.616365106589\\
-152.588	-145.979891757419\\
-224.609	-214.881887879402\\
-148.926	-142.476481504872\\
-129.395	-123.79130792691\\
-186.768	-178.679663038704\\
-119.629	-114.448242791362\\
-68.359	-65.3985858694356\\
-87.891	-84.084716140531\\
-68.359	-65.3985858694356\\
-52.49	-50.2168225440201\\
-57.373	-54.888355111794\\
-50.049	-47.8815346066996\\
-48.828	-46.713412291473\\
-64.697	-61.8951756168884\\
-95.215	-91.0915366456254\\
-107.422	-102.769889718494\\
-147.705	-141.308359189646\\
-139.16	-133.133416369324\\
-166.016	-158.826367145514\\
-194.092	-185.686483543798\\
-174.561	-167.001309965835\\
-125.732	-120.28694098123\\
-134.277	-128.461883801551\\
-119.629	-114.448242791362\\
-130.615	-124.958473549003\\
-183.105	-175.175296093024\\
-140.381	-134.301538684551\\
-158.691	-151.818589947287\\
-136.719	-130.798128432004\\
-129.395	-123.79130792691\\
-201.416	-192.693304048892\\
-166.016	-158.826367145514\\
-167.236	-159.993532767608\\
-152.588	-145.979891757419\\
-91.553	-87.5881263930782\\
-58.594	-56.0564774270207\\
-46.387	-44.3781243541525\\
-84.229	-80.5813058879839\\
-79.346	-75.90977332021\\
-106.201	-101.601767403267\\
-79.346	-75.90977332021\\
-108.643	-103.93801203372\\
-191.65	-183.350238913345\\
-235.596	-225.393075330177\\
-185.547	-177.511540723477\\
-190.43	-182.183073291251\\
-169.678	-162.329777398062\\
-118.408	-113.280120476135\\
-133.057	-127.294718179457\\
-145.264	-138.973071252325\\
-125.732	-120.28694098123\\
-144.043	-137.804948937098\\
-181.885	-174.00813047093\\
-260.01	-248.749781475913\\
-230.713	-220.721542762403\\
-214.844	-205.539779436987\\
-239.258	-228.896485582724\\
-162.354	-155.322956892967\\
-173.34	-165.833187650609\\
-139.16	-133.133416369324\\
-141.602	-135.469660999778\\
-186.768	-178.679663038704\\
-213.623	-204.371657121761\\
-84.229	-80.5813058879839\\
-100.098	-95.7630692133993\\
-85.449	-81.7484715100777\\
-119.629	-114.448242791362\\
-123.291	-117.951653043909\\
-75.684	-72.4063630676628\\
-98.877	-94.5949468981726\\
-173.34	-165.833187650609\\
-303.955	-290.791661199612\\
-289.307	-276.778020189423\\
-261.23	-249.916947098007\\
-211.182	-202.03636918444\\
-240.479	-230.06460789795\\
-290.527	-277.945185811517\\
-217.285	-207.875067374308\\
-218.506	-209.043189689535\\
-225.83	-216.050010194629\\
-152.588	-145.979891757419\\
-131.836	-126.12659586423\\
-169.678	-162.329777398062\\
-115.967	-110.944832538815\\
-81.787	-78.2450612575305\\
-114.746	-109.776710223588\\
-84.229	-80.5813058879839\\
-98.877	-94.5949468981726\\
-95.215	-91.0915366456254\\
-115.967	-110.944832538815\\
-158.691	-151.818589947287\\
-126.953	-121.455063296456\\
-90.332	-86.4200040778515\\
-112.305	-107.441422286268\\
-129.395	-123.79130792691\\
-153.809	-147.148014072646\\
-129.395	-123.79130792691\\
-106.201	-101.601767403267\\
-167.236	-159.993532767608\\
-159.912	-152.986712262514\\
-111.084	-106.273299971041\\
-183.105	-175.175296093024\\
-168.457	-161.161655082835\\
-119.629	-114.448242791362\\
-81.787	-78.2450612575305\\
-58.594	-56.0564774270207\\
-45.166	-43.2100020389258\\
-98.877	-94.5949468981726\\
-96.436	-92.2596589608521\\
-65.918	-63.0632979321151\\
-47.607	-45.5452899762463\\
-57.373	-54.888355111794\\
-102.539	-98.0983571507198\\
-119.629	-114.448242791362\\
-158.691	-151.818589947287\\
-180.664	-172.840008155703\\
-172.119	-164.665065335382\\
-181.885	-174.00813047093\\
-112.305	-107.441422286268\\
-46.387	-44.3781243541525\\
-70.801	-67.7348304998889\\
-62.256	-59.5598876795679\\
-78.125	-74.7416510049833\\
-126.953	-121.455063296456\\
-139.16	-133.133416369324\\
-201.416	-192.693304048892\\
-133.057	-127.294718179457\\
-124.512	-119.119775359136\\
-191.65	-183.350238913345\\
-142.822	-136.636826621872\\
-95.215	-91.0915366456254\\
-69.58	-66.5667081846622\\
-93.994	-89.9234143303987\\
-129.395	-123.79130792691\\
-190.43	-182.183073291251\\
-130.615	-124.958473549003\\
-106.201	-101.601767403267\\
-98.877	-94.5949468981726\\
-122.07	-116.783530728682\\
-163.574	-156.490122515061\\
-255.127	-244.078248908139\\
-219.727	-210.211312004761\\
-214.844	-205.539779436987\\
-137.939	-131.965294054098\\
-90.332	-86.4200040778515\\
-107.422	-102.769889718494\\
-50.049	-47.8815346066996\\
-74.463	-71.2382407524361\\
-69.58	-66.5667081846622\\
-79.346	-75.90977332021\\
-134.277	-128.461883801551\\
-185.547	-177.511540723477\\
-217.285	-207.875067374308\\
-278.32	-266.266832738649\\
-239.258	-228.896485582724\\
-129.395	-123.79130792691\\
-118.408	-113.280120476135\\
-98.877	-94.5949468981726\\
-136.719	-130.798128432004\\
-179.443	-171.671885840476\\
-239.258	-228.896485582724\\
-179.443	-171.671885840476\\
-122.07	-116.783530728682\\
-139.16	-133.133416369324\\
-184.326	-176.34341840825\\
-133.057	-127.294718179457\\
-89.111	-85.2518817626249\\
-73.242	-70.0701184372094\\
-50.049	-47.8815346066996\\
-48.828	-46.713412291473\\
-36.621	-35.0350592186047\\
-64.697	-61.8951756168884\\
-98.877	-94.5949468981726\\
-107.422	-102.769889718494\\
-80.566	-77.0769389423038\\
-73.242	-70.0701184372094\\
-34.18	-32.6997712812842\\
-17.09	-16.3498856406421\\
-30.518	-29.196361028737\\
-57.373	-54.888355111794\\
-87.891	-84.084716140531\\
-111.084	-106.273299971041\\
-130.615	-124.958473549003\\
-150.146	-143.643647126966\\
-190.43	-182.183073291251\\
-266.113	-254.588479665781\\
-261.23	-249.916947098007\\
-280.762	-268.603077369102\\
-249.023	-238.238594025139\\
-136.719	-130.798128432004\\
-123.291	-117.951653043909\\
-125.732	-120.28694098123\\
-166.016	-158.826367145514\\
-145.264	-138.973071252325\\
-102.539	-98.0983571507198\\
-79.346	-75.90977332021\\
-63.477	-60.7280099947946\\
-111.084	-106.273299971041\\
-108.643	-103.93801203372\\
-56.152	-53.7202327965673\\
-41.504	-39.7065917863786\\
-69.58	-66.5667081846622\\
-96.436	-92.2596589608521\\
-76.904	-73.5735286897566\\
-106.201	-101.601767403267\\
-86.67	-82.9165938253044\\
-122.07	-116.783530728682\\
-181.885	-174.00813047093\\
-146.484	-140.140236874419\\
-190.43	-182.183073291251\\
-260.01	-248.749781475913\\
-281.982	-269.770242991196\\
-223.389	-213.714722257308\\
-145.264	-138.973071252325\\
-107.422	-102.769889718494\\
-64.697	-61.8951756168884\\
-101.318	-96.9302348354931\\
-75.684	-72.4063630676628\\
-126.953	-121.455063296456\\
-115.967	-110.944832538815\\
-92.773	-88.755292015172\\
-119.629	-114.448242791362\\
-69.58	-66.5667081846622\\
-98.877	-94.5949468981726\\
-75.684	-72.4063630676628\\
-46.387	-44.3781243541525\\
-54.932	-52.5530671744735\\
-42.725	-40.8747141016053\\
-47.607	-45.5452899762463\\
-46.387	-44.3781243541525\\
-30.518	-29.196361028737\\
-42.725	-40.8747141016053\\
-62.256	-59.5598876795679\\
-61.035	-58.3917653643412\\
-97.656	-93.4268245829459\\
-130.615	-124.958473549003\\
-108.643	-103.93801203372\\
-98.877	-94.5949468981726\\
-156.25	-149.483302009967\\
-190.43	-182.183073291251\\
-150.146	-143.643647126966\\
-156.25	-149.483302009967\\
-74.463	-71.2382407524361\\
-43.945	-42.0418797236991\\
-90.332	-86.4200040778515\\
-133.057	-127.294718179457\\
-145.264	-138.973071252325\\
-202.637	-193.861426364119\\
-183.105	-175.175296093024\\
-130.615	-124.958473549003\\
-89.111	-85.2518817626249\\
-93.994	-89.9234143303987\\
-102.539	-98.0983571507198\\
-81.787	-78.2450612575305\\
-48.828	-46.713412291473\\
-64.697	-61.8951756168884\\
-52.49	-50.2168225440201\\
-41.504	-39.7065917863786\\
-32.959	-31.5316489660575\\
-61.035	-58.3917653643412\\
-93.994	-89.9234143303987\\
-100.098	-95.7630692133993\\
-69.58	-66.5667081846622\\
-123.291	-117.951653043909\\
-172.119	-164.665065335382\\
-109.863	-105.105177655814\\
-146.484	-140.140236874419\\
-163.574	-156.490122515061\\
-118.408	-113.280120476135\\
-69.58	-66.5667081846622\\
-86.67	-82.9165938253044\\
-108.643	-103.93801203372\\
-59.814	-57.2236430491145\\
-69.58	-66.5667081846622\\
-53.711	-51.3849448592468\\
-31.738	-30.3635266508308\\
-56.152	-53.7202327965673\\
-76.904	-73.5735286897566\\
-67.139	-64.2314202473417\\
-83.008	-79.4131835727572\\
-100.098	-95.7630692133993\\
-73.242	-70.0701184372094\\
-37.842	-36.2031815338314\\
-32.959	-31.5316489660575\\
-41.504	-39.7065917863786\\
-50.049	-47.8815346066996\\
-61.035	-58.3917653643412\\
-128.174	-122.623185611683\\
-130.615	-124.958473549003\\
-109.863	-105.105177655814\\
-107.422	-102.769889718494\\
-141.602	-135.469660999778\\
-185.547	-177.511540723477\\
-201.416	-192.693304048892\\
-205.078	-196.19671430144\\
-151.367	-144.811769442193\\
-157.471	-150.651424325193\\
-162.354	-155.322956892967\\
-166.016	-158.826367145514\\
-185.547	-177.511540723477\\
-246.582	-235.903306087818\\
-216.064	-206.706945059081\\
-195.313	-186.854605859025\\
-157.471	-150.651424325193\\
-147.705	-141.308359189646\\
-142.822	-136.636826621872\\
-167.236	-159.993532767608\\
-103.76	-99.2664794659465\\
-120.85	-115.616365106589\\
-148.926	-142.476481504872\\
-179.443	-171.671885840476\\
-137.939	-131.965294054098\\
-155.029	-148.31517969474\\
-129.395	-123.79130792691\\
-112.305	-107.441422286268\\
-70.801	-67.7348304998889\\
-111.084	-106.273299971041\\
-177.002	-169.336597903156\\
-141.602	-135.469660999778\\
-81.787	-78.2450612575305\\
-97.656	-93.4268245829459\\
-53.711	-51.3849448592468\\
-50.049	-47.8815346066996\\
-70.801	-67.7348304998889\\
-45.166	-43.2100020389258\\
-46.387	-44.3781243541525\\
-54.932	-52.5530671744735\\
-36.621	-35.0350592186047\\
-62.256	-59.5598876795679\\
-51.27	-49.0496569219263\\
-30.518	-29.196361028737\\
-57.373	-54.888355111794\\
-64.697	-61.8951756168884\\
-80.566	-77.0769389423038\\
-75.684	-72.4063630676628\\
-76.904	-73.5735286897566\\
-104.98	-100.43364508804\\
-92.773	-88.755292015172\\
-104.98	-100.43364508804\\
-180.664	-172.840008155703\\
-128.174	-122.623185611683\\
-112.305	-107.441422286268\\
-122.07	-116.783530728682\\
-86.67	-82.9165938253044\\
-51.27	-49.0496569219263\\
-92.773	-88.755292015172\\
-73.242	-70.0701184372094\\
-43.945	-42.0418797236991\\
-81.787	-78.2450612575305\\
-80.566	-77.0769389423038\\
-98.877	-94.5949468981726\\
-61.035	-58.3917653643412\\
-36.621	-35.0350592186047\\
-29.297	-28.0282387135103\\
-36.621	-35.0350592186047\\
-69.58	-66.5667081846622\\
-68.359	-65.3985858694356\\
-80.566	-77.0769389423038\\
-109.863	-105.105177655814\\
-150.146	-143.643647126966\\
-109.863	-105.105177655814\\
-152.588	-145.979891757419\\
-177.002	-169.336597903156\\
-153.809	-147.148014072646\\
-145.264	-138.973071252325\\
-230.713	-220.721542762403\\
-167.236	-159.993532767608\\
-208.74	-199.700124553987\\
-178.223	-170.504720218383\\
-203.857	-195.028591986213\\
-234.375	-224.22495301495\\
-136.719	-130.798128432004\\
-81.787	-78.2450612575305\\
-65.918	-63.0632979321151\\
-108.643	-103.93801203372\\
-134.277	-128.461883801551\\
-137.939	-131.965294054098\\
-90.332	-86.4200040778515\\
-50.049	-47.8815346066996\\
-67.139	-64.2314202473417\\
-128.174	-122.623185611683\\
-179.443	-171.671885840476\\
-177.002	-169.336597903156\\
-103.76	-99.2664794659465\\
-81.787	-78.2450612575305\\
-109.863	-105.105177655814\\
-175.781	-168.168475587929\\
-200.195	-191.525181733666\\
-203.857	-195.028591986213\\
-150.146	-143.643647126966\\
-208.74	-199.700124553987\\
-227.051	-217.218132509856\\
-212.402	-203.203534806534\\
-290.527	-277.945185811517\\
-251.465	-240.574838655592\\
-209.961	-200.868246869213\\
-241.699	-231.231773520044\\
-207.52	-198.532958931893\\
-106.201	-101.601767403267\\
-100.098	-95.7630692133993\\
-126.953	-121.455063296456\\
-155.029	-148.31517969474\\
-158.691	-151.818589947287\\
-115.967	-110.944832538815\\
-148.926	-142.476481504872\\
-130.615	-124.958473549003\\
-96.436	-92.2596589608521\\
-129.395	-123.79130792691\\
-122.07	-116.783530728682\\
-175.781	-168.168475587929\\
-181.885	-174.00813047093\\
-134.277	-128.461883801551\\
-97.656	-93.4268245829459\\
-58.594	-56.0564774270207\\
-96.436	-92.2596589608521\\
-74.463	-71.2382407524361\\
-65.918	-63.0632979321151\\
-53.711	-51.3849448592468\\
-95.215	-91.0915366456254\\
-123.291	-117.951653043909\\
-137.939	-131.965294054098\\
-136.719	-130.798128432004\\
-80.566	-77.0769389423038\\
-62.256	-59.5598876795679\\
-42.725	-40.8747141016053\\
-54.932	-52.5530671744735\\
-91.553	-87.5881263930782\\
-95.215	-91.0915366456254\\
-91.553	-87.5881263930782\\
-150.146	-143.643647126966\\
-174.561	-167.001309965835\\
-198.975	-190.358016111572\\
-229.492	-219.553420447176\\
-137.939	-131.965294054098\\
-113.525	-108.608587908361\\
-87.891	-84.084716140531\\
-93.994	-89.9234143303987\\
-128.174	-122.623185611683\\
-106.201	-101.601767403267\\
-69.58	-66.5667081846622\\
-64.697	-61.8951756168884\\
-124.512	-119.119775359136\\
-153.809	-147.148014072646\\
-134.277	-128.461883801551\\
-211.182	-202.03636918444\\
-230.713	-220.721542762403\\
-159.912	-152.986712262514\\
-111.084	-106.273299971041\\
-76.904	-73.5735286897566\\
-70.801	-67.7348304998889\\
-131.836	-126.12659586423\\
-106.201	-101.601767403267\\
-122.07	-116.783530728682\\
-139.16	-133.133416369324\\
-156.25	-149.483302009967\\
-151.367	-144.811769442193\\
-168.457	-161.161655082835\\
-115.967	-110.944832538815\\
-130.615	-124.958473549003\\
-91.553	-87.5881263930782\\
-84.229	-80.5813058879839\\
-129.395	-123.79130792691\\
-183.105	-175.175296093024\\
-201.416	-192.693304048892\\
-106.201	-101.601767403267\\
-220.947	-211.378477626855\\
-207.52	-198.532958931893\\
-139.16	-133.133416369324\\
-76.904	-73.5735286897566\\
-123.291	-117.951653043909\\
-107.422	-102.769889718494\\
-104.98	-100.43364508804\\
-123.291	-117.951653043909\\
-80.566	-77.0769389423038\\
-108.643	-103.93801203372\\
-207.52	-198.532958931893\\
-219.727	-210.211312004761\\
-145.264	-138.973071252325\\
-159.912	-152.986712262514\\
-177.002	-169.336597903156\\
-180.664	-172.840008155703\\
-131.836	-126.12659586423\\
-128.174	-122.623185611683\\
-84.229	-80.5813058879839\\
-133.057	-127.294718179457\\
-134.277	-128.461883801551\\
-229.492	-219.553420447176\\
-251.465	-240.574838655592\\
-332.031	-317.651777597896\\
-238.037	-227.728363267497\\
-194.092	-185.686483543798\\
-147.705	-141.308359189646\\
-162.354	-155.322956892967\\
-122.07	-116.783530728682\\
-84.229	-80.5813058879839\\
-81.787	-78.2450612575305\\
-85.449	-81.7484715100777\\
-62.256	-59.5598876795679\\
-46.387	-44.3781243541525\\
-74.463	-71.2382407524361\\
-100.098	-95.7630692133993\\
-68.359	-65.3985858694356\\
-46.387	-44.3781243541525\\
-112.305	-107.441422286268\\
-157.471	-150.651424325193\\
-73.242	-70.0701184372094\\
-95.215	-91.0915366456254\\
-97.656	-93.4268245829459\\
-91.553	-87.5881263930782\\
-113.525	-108.608587908361\\
-98.877	-94.5949468981726\\
-68.359	-65.3985858694356\\
-48.828	-46.713412291473\\
-40.283	-38.5384694711519\\
-53.711	-51.3849448592468\\
-106.201	-101.601767403267\\
-126.953	-121.455063296456\\
-96.436	-92.2596589608521\\
-63.477	-60.7280099947946\\
-39.063	-37.3713038490581\\
-56.152	-53.7202327965673\\
-96.436	-92.2596589608521\\
-123.291	-117.951653043909\\
-133.057	-127.294718179457\\
-90.332	-86.4200040778515\\
-86.67	-82.9165938253044\\
-95.215	-91.0915366456254\\
-133.057	-127.294718179457\\
-185.547	-177.511540723477\\
-216.064	-206.706945059081\\
-164.795	-157.658244830288\\
-102.539	-98.0983571507198\\
-107.422	-102.769889718494\\
-100.098	-95.7630692133993\\
-103.76	-99.2664794659465\\
-65.918	-63.0632979321151\\
-87.891	-84.084716140531\\
-93.994	-89.9234143303987\\
-91.553	-87.5881263930782\\
-56.152	-53.7202327965673\\
-36.621	-35.0350592186047\\
-56.152	-53.7202327965673\\
-84.229	-80.5813058879839\\
-134.277	-128.461883801551\\
-148.926	-142.476481504872\\
-179.443	-171.671885840476\\
-159.912	-152.986712262514\\
-180.664	-172.840008155703\\
-239.258	-228.896485582724\\
-319.824	-305.973424525028\\
-253.906	-242.910126592913\\
-173.34	-165.833187650609\\
-189.209	-181.014950976024\\
-220.947	-211.378477626855\\
-288.086	-275.609897874197\\
-180.664	-172.840008155703\\
-89.111	-85.2518817626249\\
-59.814	-57.2236430491145\\
-62.256	-59.5598876795679\\
-42.725	-40.8747141016053\\
-28.076	-26.8601163982837\\
-39.063	-37.3713038490581\\
-59.814	-57.2236430491145\\
-83.008	-79.4131835727572\\
-86.67	-82.9165938253044\\
-65.918	-63.0632979321151\\
-62.256	-59.5598876795679\\
-78.125	-74.7416510049833\\
-100.098	-95.7630692133993\\
-104.98	-100.43364508804\\
-96.436	-92.2596589608521\\
-68.359	-65.3985858694356\\
-47.607	-45.5452899762463\\
-45.166	-43.2100020389258\\
-72.021	-68.9019961219828\\
-87.891	-84.084716140531\\
-64.697	-61.8951756168884\\
-41.504	-39.7065917863786\\
-76.904	-73.5735286897566\\
-48.828	-46.713412291473\\
-69.58	-66.5667081846622\\
-79.346	-75.90977332021\\
-120.85	-115.616365106589\\
-126.953	-121.455063296456\\
-84.229	-80.5813058879839\\
-125.732	-120.28694098123\\
-123.291	-117.951653043909\\
-101.318	-96.9302348354931\\
-84.229	-80.5813058879839\\
-142.822	-136.636826621872\\
-175.781	-168.168475587929\\
-194.092	-185.686483543798\\
-147.705	-141.308359189646\\
-136.719	-130.798128432004\\
-126.953	-121.455063296456\\
-173.34	-165.833187650609\\
-135.498	-129.630006116777\\
-184.326	-176.34341840825\\
-172.119	-164.665065335382\\
-178.223	-170.504720218383\\
-303.955	-290.791661199612\\
-231.934	-221.889665077629\\
-125.732	-120.28694098123\\
-83.008	-79.4131835727572\\
-87.891	-84.084716140531\\
-112.305	-107.441422286268\\
-146.484	-140.140236874419\\
-172.119	-164.665065335382\\
-189.209	-181.014950976024\\
-192.871	-184.518361228571\\
-125.732	-120.28694098123\\
-111.084	-106.273299971041\\
-130.615	-124.958473549003\\
-129.395	-123.79130792691\\
-76.904	-73.5735286897566\\
-57.373	-54.888355111794\\
-103.76	-99.2664794659465\\
-101.318	-96.9302348354931\\
-142.822	-136.636826621872\\
-174.561	-167.001309965835\\
-155.029	-148.31517969474\\
-106.201	-101.601767403267\\
-85.449	-81.7484715100777\\
-135.498	-129.630006116777\\
-168.457	-161.161655082835\\
-230.713	-220.721542762403\\
-244.141	-233.568018150498\\
-286.865	-274.44177555897\\
-236.816	-226.56024095227\\
-216.064	-206.706945059081\\
-123.291	-117.951653043909\\
-85.449	-81.7484715100777\\
-153.809	-147.148014072646\\
-202.637	-193.861426364119\\
-115.967	-110.944832538815\\
-186.768	-178.679663038704\\
-249.023	-238.238594025139\\
-308.838	-295.463193767386\\
-189.209	-181.014950976024\\
-112.305	-107.441422286268\\
-63.477	-60.7280099947946\\
-95.215	-91.0915366456254\\
-85.449	-81.7484715100777\\
-93.994	-89.9234143303987\\
-51.27	-49.0496569219263\\
-76.904	-73.5735286897566\\
-52.49	-50.2168225440201\\
-74.463	-71.2382407524361\\
-58.594	-56.0564774270207\\
-79.346	-75.90977332021\\
-124.512	-119.119775359136\\
-112.305	-107.441422286268\\
-162.354	-155.322956892967\\
-191.65	-183.350238913345\\
-190.43	-182.183073291251\\
-103.76	-99.2664794659465\\
-109.863	-105.105177655814\\
-130.615	-124.958473549003\\
-85.449	-81.7484715100777\\
-80.566	-77.0769389423038\\
-54.932	-52.5530671744735\\
-90.332	-86.4200040778515\\
-79.346	-75.90977332021\\
-47.607	-45.5452899762463\\
-26.855	-25.691994083057\\
-31.738	-30.3635266508308\\
-58.594	-56.0564774270207\\
-85.449	-81.7484715100777\\
-91.553	-87.5881263930782\\
-56.152	-53.7202327965673\\
-67.139	-64.2314202473417\\
-102.539	-98.0983571507198\\
-135.498	-129.630006116777\\
-93.994	-89.9234143303987\\
-87.891	-84.084716140531\\
-104.98	-100.43364508804\\
-125.732	-120.28694098123\\
-115.967	-110.944832538815\\
-137.939	-131.965294054098\\
-126.953	-121.455063296456\\
-89.111	-85.2518817626249\\
-72.021	-68.9019961219828\\
-95.215	-91.0915366456254\\
-78.125	-74.7416510049833\\
-97.656	-93.4268245829459\\
-129.395	-123.79130792691\\
-203.857	-195.028591986213\\
-139.16	-133.133416369324\\
-140.381	-134.301538684551\\
-206.299	-197.364836616666\\
-152.588	-145.979891757419\\
-96.436	-92.2596589608521\\
-69.58	-66.5667081846622\\
-54.932	-52.5530671744735\\
-58.594	-56.0564774270207\\
-46.387	-44.3781243541525\\
-50.049	-47.8815346066996\\
-106.201	-101.601767403267\\
-45.166	-43.2100020389258\\
-50.049	-47.8815346066996\\
-65.918	-63.0632979321151\\
-93.994	-89.9234143303987\\
-69.58	-66.5667081846622\\
-53.711	-51.3849448592468\\
-28.076	-26.8601163982837\\
-54.932	-52.5530671744735\\
-101.318	-96.9302348354931\\
-139.16	-133.133416369324\\
-108.643	-103.93801203372\\
-89.111	-85.2518817626249\\
-101.318	-96.9302348354931\\
-102.539	-98.0983571507198\\
-119.629	-114.448242791362\\
-144.043	-137.804948937098\\
-168.457	-161.161655082835\\
-164.795	-157.658244830288\\
-140.381	-134.301538684551\\
-91.553	-87.5881263930782\\
-73.242	-70.0701184372094\\
-79.346	-75.90977332021\\
-112.305	-107.441422286268\\
-64.697	-61.8951756168884\\
-62.256	-59.5598876795679\\
-114.746	-109.776710223588\\
-139.16	-133.133416369324\\
-133.057	-127.294718179457\\
-157.471	-150.651424325193\\
-205.078	-196.19671430144\\
-104.98	-100.43364508804\\
-184.326	-176.34341840825\\
-185.547	-177.511540723477\\
-170.898	-163.496943020155\\
-184.326	-176.34341840825\\
-181.885	-174.00813047093\\
-312.5	-298.966604019933\\
-286.865	-274.44177555897\\
-195.313	-186.854605859025\\
-236.816	-226.56024095227\\
-283.203	-270.938365306423\\
-275.879	-263.931544801329\\
-311.279	-297.798481704707\\
-281.982	-269.770242991196\\
-372.314	-356.190247069048\\
-280.762	-268.603077369102\\
-196.533	-188.021771481118\\
-118.408	-113.280120476135\\
-124.512	-119.119775359136\\
-92.773	-88.755292015172\\
-79.346	-75.90977332021\\
-123.291	-117.951653043909\\
-172.119	-164.665065335382\\
-104.98	-100.43364508804\\
-92.773	-88.755292015172\\
-91.553	-87.5881263930782\\
-59.814	-57.2236430491145\\
-80.566	-77.0769389423038\\
-36.621	-35.0350592186047\\
-54.932	-52.5530671744735\\
-43.945	-42.0418797236991\\
-65.918	-63.0632979321151\\
-45.166	-43.2100020389258\\
-40.283	-38.5384694711519\\
-23.193	-22.1885838305098\\
-46.387	-44.3781243541525\\
-30.518	-29.196361028737\\
-37.842	-36.2031815338314\\
-81.787	-78.2450612575305\\
-112.305	-107.441422286268\\
-115.967	-110.944832538815\\
-81.787	-78.2450612575305\\
-107.422	-102.769889718494\\
-166.016	-158.826367145514\\
-80.566	-77.0769389423038\\
-48.828	-46.713412291473\\
-37.842	-36.2031815338314\\
-28.076	-26.8601163982837\\
-48.828	-46.713412291473\\
-81.787	-78.2450612575305\\
-123.291	-117.951653043909\\
-72.021	-68.9019961219828\\
-129.395	-123.79130792691\\
-173.34	-165.833187650609\\
-175.781	-168.168475587929\\
-130.615	-124.958473549003\\
-75.684	-72.4063630676628\\
-100.098	-95.7630692133993\\
-135.498	-129.630006116777\\
-178.223	-170.504720218383\\
-93.994	-89.9234143303987\\
-53.711	-51.3849448592468\\
-84.229	-80.5813058879839\\
-63.477	-60.7280099947946\\
-31.738	-30.3635266508308\\
-21.973	-21.021418208416\\
-54.932	-52.5530671744735\\
-126.953	-121.455063296456\\
-92.773	-88.755292015172\\
-57.373	-54.888355111794\\
-61.035	-58.3917653643412\\
-18.311	-17.5180079558688\\
-35.4	-33.866936903378\\
-30.518	-29.196361028737\\
-58.594	-56.0564774270207\\
-84.229	-80.5813058879839\\
-131.836	-126.12659586423\\
-135.498	-129.630006116777\\
-148.926	-142.476481504872\\
-153.809	-147.148014072646\\
-148.926	-142.476481504872\\
-145.264	-138.973071252325\\
-125.732	-120.28694098123\\
-155.029	-148.31517969474\\
-236.816	-226.56024095227\\
-213.623	-204.371657121761\\
-111.084	-106.273299971041\\
-58.594	-56.0564774270207\\
-135.498	-129.630006116777\\
-158.691	-151.818589947287\\
-142.822	-136.636826621872\\
-78.125	-74.7416510049833\\
-83.008	-79.4131835727572\\
-142.822	-136.636826621872\\
-217.285	-207.875067374308\\
-224.609	-214.881887879402\\
-250.244	-239.406716340365\\
-239.258	-228.896485582724\\
-178.223	-170.504720218383\\
-183.105	-175.175296093024\\
-84.229	-80.5813058879839\\
-79.346	-75.90977332021\\
-101.318	-96.9302348354931\\
-126.953	-121.455063296456\\
-134.277	-128.461883801551\\
-142.822	-136.636826621872\\
-86.67	-82.9165938253044\\
-124.512	-119.119775359136\\
-102.539	-98.0983571507198\\
-59.814	-57.2236430491145\\
-39.063	-37.3713038490581\\
-32.959	-31.5316489660575\\
-53.711	-51.3849448592468\\
-40.283	-38.5384694711519\\
-23.193	-22.1885838305098\\
-48.828	-46.713412291473\\
-100.098	-95.7630692133993\\
-64.697	-61.8951756168884\\
-65.918	-63.0632979321151\\
-86.67	-82.9165938253044\\
-146.484	-140.140236874419\\
-97.656	-93.4268245829459\\
-52.49	-50.2168225440201\\
-29.297	-28.0282387135103\\
-81.787	-78.2450612575305\\
-159.912	-152.986712262514\\
-201.416	-192.693304048892\\
-280.762	-268.603077369102\\
-333.252	-318.819899913123\\
-323.486	-309.476834777575\\
-207.52	-198.532958931893\\
-212.402	-203.203534806534\\
-277.1	-265.099667116555\\
-231.934	-221.889665077629\\
-214.844	-205.539779436987\\
-242.92	-232.399895835271\\
-263.672	-252.25319172846\\
-270.996	-259.260012233555\\
-357.666	-342.176606058859\\
-238.037	-227.728363267497\\
-135.498	-129.630006116777\\
-70.801	-67.7348304998889\\
-59.814	-57.2236430491145\\
-67.139	-64.2314202473417\\
-69.58	-66.5667081846622\\
-126.953	-121.455063296456\\
-93.994	-89.9234143303987\\
-107.422	-102.769889718494\\
-136.719	-130.798128432004\\
-70.801	-67.7348304998889\\
-85.449	-81.7484715100777\\
-117.188	-112.112954854041\\
-140.381	-134.301538684551\\
-76.904	-73.5735286897566\\
-51.27	-49.0496569219263\\
-68.359	-65.3985858694356\\
-64.697	-61.8951756168884\\
-167.236	-159.993532767608\\
-227.051	-217.218132509856\\
-214.844	-205.539779436987\\
-249.023	-238.238594025139\\
-274.658	-262.763422486102\\
-360.107	-344.51189399618\\
-306.396	-293.126949136933\\
-190.43	-182.183073291251\\
-125.732	-120.28694098123\\
-69.58	-66.5667081846622\\
-78.125	-74.7416510049833\\
-81.787	-78.2450612575305\\
-109.863	-105.105177655814\\
-72.021	-68.9019961219828\\
-68.359	-65.3985858694356\\
-75.684	-72.4063630676628\\
-95.215	-91.0915366456254\\
-107.422	-102.769889718494\\
-135.498	-129.630006116777\\
-95.215	-91.0915366456254\\
-45.166	-43.2100020389258\\
-34.18	-32.6997712812842\\
-28.076	-26.8601163982837\\
-32.959	-31.5316489660575\\
-39.063	-37.3713038490581\\
-72.021	-68.9019961219828\\
-73.242	-70.0701184372094\\
-95.215	-91.0915366456254\\
-140.381	-134.301538684551\\
-100.098	-95.7630692133993\\
-168.457	-161.161655082835\\
-244.141	-233.568018150498\\
-190.43	-182.183073291251\\
-172.119	-164.665065335382\\
-207.52	-198.532958931893\\
-239.258	-228.896485582724\\
-148.926	-142.476481504872\\
-130.615	-124.958473549003\\
-80.566	-77.0769389423038\\
-92.773	-88.755292015172\\
-187.988	-179.846828660797\\
-314.941	-301.301891957254\\
-371.094	-355.023081446954\\
-294.189	-281.448596064064\\
-247.803	-237.071428403045\\
-175.781	-168.168475587929\\
-191.65	-183.350238913345\\
-256.348	-245.246371223366\\
-147.705	-141.308359189646\\
-128.174	-122.623185611683\\
-96.436	-92.2596589608521\\
-185.547	-177.511540723477\\
-173.34	-165.833187650609\\
-92.773	-88.755292015172\\
-90.332	-86.4200040778515\\
-65.918	-63.0632979321151\\
-119.629	-114.448242791362\\
-180.664	-172.840008155703\\
-186.768	-178.679663038704\\
-141.602	-135.469660999778\\
-104.98	-100.43364508804\\
-115.967	-110.944832538815\\
-89.111	-85.2518817626249\\
-64.697	-61.8951756168884\\
-52.49	-50.2168225440201\\
-45.166	-43.2100020389258\\
-36.621	-35.0350592186047\\
-46.387	-44.3781243541525\\
-76.904	-73.5735286897566\\
-113.525	-108.608587908361\\
-72.021	-68.9019961219828\\
-62.256	-59.5598876795679\\
-51.27	-49.0496569219263\\
-68.359	-65.3985858694356\\
-95.215	-91.0915366456254\\
-103.76	-99.2664794659465\\
-100.098	-95.7630692133993\\
-54.932	-52.5530671744735\\
-123.291	-117.951653043909\\
-172.119	-164.665065335382\\
-124.512	-119.119775359136\\
-50.049	-47.8815346066996\\
-61.035	-58.3917653643412\\
-95.215	-91.0915366456254\\
-84.229	-80.5813058879839\\
-51.27	-49.0496569219263\\
-93.994	-89.9234143303987\\
-144.043	-137.804948937098\\
-86.67	-82.9165938253044\\
-32.959	-31.5316489660575\\
-42.725	-40.8747141016053\\
-111.084	-106.273299971041\\
-139.16	-133.133416369324\\
-81.787	-78.2450612575305\\
-68.359	-65.3985858694356\\
-133.057	-127.294718179457\\
-179.443	-171.671885840476\\
-157.471	-150.651424325193\\
-222.168	-212.546599942082\\
-268.555	-256.924724296234\\
-173.34	-165.833187650609\\
-139.16	-133.133416369324\\
-197.754	-189.189893796345\\
-133.057	-127.294718179457\\
-179.443	-171.671885840476\\
-220.947	-211.378477626855\\
-172.119	-164.665065335382\\
-84.229	-80.5813058879839\\
-142.822	-136.636826621872\\
-238.037	-227.728363267497\\
-280.762	-268.603077369102\\
-205.078	-196.19671430144\\
-175.781	-168.168475587929\\
-229.492	-219.553420447176\\
-175.781	-168.168475587929\\
-112.305	-107.441422286268\\
-101.318	-96.9302348354931\\
-130.615	-124.958473549003\\
-136.719	-130.798128432004\\
-155.029	-148.31517969474\\
-106.201	-101.601767403267\\
-114.746	-109.776710223588\\
-163.574	-156.490122515061\\
-92.773	-88.755292015172\\
-76.904	-73.5735286897566\\
-61.035	-58.3917653643412\\
-48.828	-46.713412291473\\
-57.373	-54.888355111794\\
-102.539	-98.0983571507198\\
-93.994	-89.9234143303987\\
-133.057	-127.294718179457\\
-169.678	-162.329777398062\\
-202.637	-193.861426364119\\
-157.471	-150.651424325193\\
-136.719	-130.798128432004\\
-61.035	-58.3917653643412\\
-95.215	-91.0915366456254\\
-167.236	-159.993532767608\\
-111.084	-106.273299971041\\
-56.152	-53.7202327965673\\
-113.525	-108.608587908361\\
-172.119	-164.665065335382\\
-178.223	-170.504720218383\\
-233.154	-223.056830699723\\
-305.176	-291.959783514839\\
-213.623	-204.371657121761\\
-181.885	-174.00813047093\\
-216.064	-206.706945059081\\
-140.381	-134.301538684551\\
-113.525	-108.608587908361\\
-129.395	-123.79130792691\\
-164.795	-157.658244830288\\
-174.561	-167.001309965835\\
-177.002	-169.336597903156\\
-97.656	-93.4268245829459\\
-86.67	-82.9165938253044\\
-69.58	-66.5667081846622\\
-101.318	-96.9302348354931\\
-92.773	-88.755292015172\\
-74.463	-71.2382407524361\\
-151.367	-144.811769442193\\
-162.354	-155.322956892967\\
-112.305	-107.441422286268\\
-93.994	-89.9234143303987\\
-136.719	-130.798128432004\\
-68.359	-65.3985858694356\\
-43.945	-42.0418797236991\\
-54.932	-52.5530671744735\\
-78.125	-74.7416510049833\\
-68.359	-65.3985858694356\\
-47.607	-45.5452899762463\\
-26.855	-25.691994083057\\
-48.828	-46.713412291473\\
-78.125	-74.7416510049833\\
-124.512	-119.119775359136\\
-142.822	-136.636826621872\\
-187.988	-179.846828660797\\
-209.961	-200.868246869213\\
-107.422	-102.769889718494\\
-83.008	-79.4131835727572\\
-178.223	-170.504720218383\\
-261.23	-249.916947098007\\
-202.637	-193.861426364119\\
-189.209	-181.014950976024\\
-283.203	-270.938365306423\\
-227.051	-217.218132509856\\
-185.547	-177.511540723477\\
-133.057	-127.294718179457\\
-140.381	-134.301538684551\\
-185.547	-177.511540723477\\
-129.395	-123.79130792691\\
-112.305	-107.441422286268\\
-189.209	-181.014950976024\\
-236.816	-226.56024095227\\
-247.803	-237.071428403045\\
-111.084	-106.273299971041\\
-52.49	-50.2168225440201\\
-141.602	-135.469660999778\\
-111.084	-106.273299971041\\
-52.49	-50.2168225440201\\
-37.842	-36.2031815338314\\
-68.359	-65.3985858694356\\
-103.76	-99.2664794659465\\
-167.236	-159.993532767608\\
-117.188	-112.112954854041\\
-86.67	-82.9165938253044\\
-52.49	-50.2168225440201\\
-34.18	-32.6997712812842\\
-48.828	-46.713412291473\\
-40.283	-38.5384694711519\\
-56.152	-53.7202327965673\\
-96.436	-92.2596589608521\\
-87.891	-84.084716140531\\
-43.945	-42.0418797236991\\
-39.063	-37.3713038490581\\
-52.49	-50.2168225440201\\
-67.139	-64.2314202473417\\
-39.063	-37.3713038490581\\
-62.256	-59.5598876795679\\
-45.166	-43.2100020389258\\
-34.18	-32.6997712812842\\
-79.346	-75.90977332021\\
-120.85	-115.616365106589\\
-175.781	-168.168475587929\\
-239.258	-228.896485582724\\
-207.52	-198.532958931893\\
-100.098	-95.7630692133993\\
-70.801	-67.7348304998889\\
-74.463	-71.2382407524361\\
-41.504	-39.7065917863786\\
-36.621	-35.0350592186047\\
-85.449	-81.7484715100777\\
-91.553	-87.5881263930782\\
-87.891	-84.084716140531\\
-103.76	-99.2664794659465\\
-122.07	-116.783530728682\\
-130.615	-124.958473549003\\
-135.498	-129.630006116777\\
-166.016	-158.826367145514\\
-152.588	-145.979891757419\\
-223.389	-213.714722257308\\
-115.967	-110.944832538815\\
-51.27	-49.0496569219263\\
-50.049	-47.8815346066996\\
-95.215	-91.0915366456254\\
-74.463	-71.2382407524361\\
-67.139	-64.2314202473417\\
-31.738	-30.3635266508308\\
-48.828	-46.713412291473\\
-37.842	-36.2031815338314\\
-18.311	-17.5180079558688\\
-32.959	-31.5316489660575\\
-69.58	-66.5667081846622\\
-87.891	-84.084716140531\\
-130.615	-124.958473549003\\
-183.105	-175.175296093024\\
-128.174	-122.623185611683\\
-51.27	-49.0496569219263\\
-93.994	-89.9234143303987\\
-106.201	-101.601767403267\\
-50.049	-47.8815346066996\\
-68.359	-65.3985858694356\\
-139.16	-133.133416369324\\
-128.174	-122.623185611683\\
-207.52	-198.532958931893\\
-240.479	-230.06460789795\\
-148.926	-142.476481504872\\
-114.746	-109.776710223588\\
};
\end{axis}

\begin{axis}[%
width=4.927cm,
height=3cm,
at={(7cm,4.839cm)},
scale only axis,
xmin=-400,
xmax=0,
xlabel style={font=\color{white!15!black}},
xlabel={y(t-1)},
ymin=-400,
ymax=0,
ylabel style={font=\color{white!15!black}},
ylabel={y(t)},
axis background/.style={fill=white},
title style={font=\small},
title={C7, R = 0.7818},
axis x line*=bottom,
axis y line*=left
]
\addplot[only marks, mark=*, mark options={}, mark size=1.5000pt, color=mycolor1, fill=mycolor1] table[row sep=crcr]{%
x	y\\
-97.656	-112.305\\
-112.305	-144.043\\
-144.043	-117.188\\
-117.188	-111.084\\
-111.084	-156.25\\
-156.25	-147.705\\
-147.705	-175.781\\
-175.781	-134.277\\
-134.277	-68.359\\
-68.359	-80.566\\
-80.566	-80.566\\
-80.566	-100.098\\
-100.098	-81.787\\
-81.787	-34.18\\
-34.18	-24.414\\
-24.414	-23.193\\
-23.193	-73.242\\
-73.242	-128.174\\
-128.174	-133.057\\
-133.057	-137.939\\
-137.939	-92.773\\
-92.773	-56.152\\
-56.152	-125.732\\
-125.732	-86.67\\
-86.67	-68.359\\
-68.359	-114.746\\
-114.746	-100.098\\
-100.098	-85.449\\
-85.449	-142.822\\
-142.822	-133.057\\
-133.057	-102.539\\
-102.539	-150.146\\
-150.146	-145.264\\
-145.264	-113.525\\
-113.525	-92.773\\
-92.773	-72.021\\
-72.021	-86.67\\
-86.67	-112.305\\
-112.305	-103.76\\
-103.76	-109.863\\
-109.863	-112.305\\
-112.305	-122.07\\
-122.07	-175.781\\
-175.781	-155.029\\
-155.029	-102.539\\
-102.539	-92.773\\
-92.773	-53.711\\
-53.711	-58.594\\
-58.594	-79.346\\
-79.346	-59.814\\
-59.814	-62.256\\
-62.256	-89.111\\
-89.111	-102.539\\
-102.539	-95.215\\
-95.215	-151.367\\
-151.367	-111.084\\
-111.084	-64.697\\
-64.697	-51.27\\
-51.27	-69.58\\
-69.58	-81.787\\
-81.787	-43.945\\
-43.945	-45.166\\
-45.166	-67.139\\
-67.139	-51.27\\
-51.27	-42.725\\
-42.725	-61.035\\
-61.035	-70.801\\
-70.801	-109.863\\
-109.863	-104.98\\
-104.98	-129.395\\
-129.395	-120.85\\
-120.85	-137.939\\
-137.939	-109.863\\
-109.863	-107.422\\
-107.422	-92.773\\
-92.773	-91.553\\
-91.553	-87.891\\
-87.891	-157.471\\
-157.471	-224.609\\
-224.609	-231.934\\
-231.934	-225.83\\
-225.83	-140.381\\
-140.381	-205.078\\
-205.078	-252.686\\
-252.686	-261.23\\
-261.23	-200.195\\
-200.195	-270.996\\
-270.996	-335.693\\
-335.693	-235.596\\
-235.596	-191.65\\
-191.65	-128.174\\
-128.174	-98.877\\
-98.877	-84.229\\
-84.229	-104.98\\
-104.98	-74.463\\
-74.463	-50.049\\
-50.049	-48.828\\
-48.828	-63.477\\
-63.477	-95.215\\
-95.215	-86.67\\
-86.67	-89.111\\
-89.111	-119.629\\
-119.629	-123.291\\
-123.291	-167.236\\
-167.236	-134.277\\
-134.277	-169.678\\
-169.678	-122.07\\
-122.07	-108.643\\
-108.643	-125.732\\
-125.732	-89.111\\
-89.111	-80.566\\
-80.566	-97.656\\
-97.656	-98.877\\
-98.877	-68.359\\
-68.359	-73.242\\
-73.242	-54.932\\
-54.932	-61.035\\
-61.035	-64.697\\
-64.697	-45.166\\
-45.166	-58.594\\
-58.594	-72.021\\
-72.021	-117.188\\
-117.188	-122.07\\
-122.07	-136.719\\
-136.719	-76.904\\
-76.904	-109.863\\
-109.863	-173.34\\
-173.34	-131.836\\
-131.836	-157.471\\
-157.471	-157.471\\
-157.471	-85.449\\
-85.449	-59.814\\
-59.814	-85.449\\
-85.449	-98.877\\
-98.877	-161.133\\
-161.133	-202.637\\
-202.637	-198.975\\
-198.975	-145.264\\
-145.264	-150.146\\
-150.146	-135.498\\
-135.498	-131.836\\
-131.836	-126.953\\
-126.953	-85.449\\
-85.449	-76.904\\
-76.904	-69.58\\
-69.58	-62.256\\
-62.256	-70.801\\
-70.801	-107.422\\
-107.422	-164.795\\
-164.795	-125.732\\
-125.732	-86.67\\
-86.67	-73.242\\
-73.242	-70.801\\
-70.801	-42.725\\
-42.725	-31.738\\
-31.738	-39.063\\
-39.063	-72.021\\
-72.021	-57.373\\
-57.373	-59.814\\
-59.814	-67.139\\
-67.139	-63.477\\
-63.477	-43.945\\
-43.945	-29.297\\
-29.297	-25.635\\
-25.635	-43.945\\
-43.945	-96.436\\
-96.436	-140.381\\
-140.381	-155.029\\
-155.029	-101.318\\
-101.318	-69.58\\
-69.58	-50.049\\
-50.049	-34.18\\
-34.18	-83.008\\
-83.008	-81.787\\
-81.787	-119.629\\
-119.629	-145.264\\
-145.264	-230.713\\
-230.713	-262.451\\
-262.451	-211.182\\
-211.182	-202.637\\
-202.637	-136.719\\
-136.719	-151.367\\
-151.367	-159.912\\
-159.912	-172.119\\
-172.119	-129.395\\
-129.395	-118.408\\
-118.408	-115.967\\
-115.967	-103.76\\
-103.76	-123.291\\
-123.291	-87.891\\
-87.891	-153.809\\
-153.809	-189.209\\
-189.209	-133.057\\
-133.057	-78.125\\
-78.125	-85.449\\
-85.449	-146.484\\
-146.484	-181.885\\
-181.885	-229.492\\
-229.492	-241.699\\
-241.699	-240.479\\
-240.479	-180.664\\
-180.664	-163.574\\
-163.574	-187.988\\
-187.988	-222.168\\
-222.168	-139.16\\
-139.16	-81.787\\
-81.787	-122.07\\
-122.07	-102.539\\
-102.539	-65.918\\
-65.918	-87.891\\
-87.891	-98.877\\
-98.877	-76.904\\
-76.904	-58.594\\
-58.594	-80.566\\
-80.566	-65.918\\
-65.918	-91.553\\
-91.553	-133.057\\
-133.057	-122.07\\
-122.07	-79.346\\
-79.346	-83.008\\
-83.008	-126.953\\
-126.953	-209.961\\
-209.961	-150.146\\
-150.146	-92.773\\
-92.773	-69.58\\
-69.58	-83.008\\
-83.008	-74.463\\
-74.463	-56.152\\
-56.152	-62.256\\
-62.256	-74.463\\
-74.463	-96.436\\
-96.436	-107.422\\
-107.422	-72.021\\
-72.021	-122.07\\
-122.07	-144.043\\
-144.043	-137.939\\
-137.939	-106.201\\
-106.201	-118.408\\
-118.408	-158.691\\
-158.691	-101.318\\
-101.318	-41.504\\
-41.504	-58.594\\
-58.594	-65.918\\
-65.918	-91.553\\
-91.553	-81.787\\
-81.787	-59.814\\
-59.814	-96.436\\
-96.436	-91.553\\
-91.553	-65.918\\
-65.918	-83.008\\
-83.008	-76.904\\
-76.904	-67.139\\
-67.139	-79.346\\
-79.346	-46.387\\
-46.387	-53.711\\
-53.711	-40.283\\
-40.283	-43.945\\
-43.945	-86.67\\
-86.67	-100.098\\
-100.098	-137.939\\
-137.939	-177.002\\
-177.002	-115.967\\
-115.967	-68.359\\
-68.359	-48.828\\
-48.828	-48.828\\
-48.828	-73.242\\
-73.242	-46.387\\
-46.387	-52.49\\
-52.49	-70.801\\
-70.801	-119.629\\
-119.629	-107.422\\
-107.422	-150.146\\
-150.146	-102.539\\
-102.539	-91.553\\
-91.553	-47.607\\
-47.607	-47.607\\
-47.607	-31.738\\
-31.738	-34.18\\
-34.18	-41.504\\
-41.504	-75.684\\
-75.684	-83.008\\
-83.008	-92.773\\
-92.773	-97.656\\
-97.656	-152.588\\
-152.588	-130.615\\
-130.615	-97.656\\
-97.656	-119.629\\
-119.629	-129.395\\
-129.395	-167.236\\
-167.236	-133.057\\
-133.057	-84.229\\
-84.229	-43.945\\
-43.945	-31.738\\
-31.738	-34.18\\
-34.18	-41.504\\
-41.504	-43.945\\
-43.945	-40.283\\
-40.283	-56.152\\
-56.152	-87.891\\
-87.891	-79.346\\
-79.346	-81.787\\
-81.787	-96.436\\
-96.436	-58.594\\
-58.594	-29.297\\
-29.297	-64.697\\
-64.697	-97.656\\
-97.656	-96.436\\
-96.436	-109.863\\
-109.863	-93.994\\
-93.994	-69.58\\
-69.58	-97.656\\
-97.656	-137.939\\
-137.939	-139.16\\
-139.16	-97.656\\
-97.656	-162.354\\
-162.354	-128.174\\
-128.174	-118.408\\
-118.408	-137.939\\
-137.939	-137.939\\
-137.939	-103.76\\
-103.76	-90.332\\
-90.332	-153.809\\
-153.809	-214.844\\
-214.844	-164.795\\
-164.795	-92.773\\
-92.773	-92.773\\
-92.773	-91.553\\
-91.553	-113.525\\
-113.525	-136.719\\
-136.719	-100.098\\
-100.098	-104.98\\
-104.98	-140.381\\
-140.381	-153.809\\
-153.809	-103.76\\
-103.76	-79.346\\
-79.346	-104.98\\
-104.98	-175.781\\
-175.781	-159.912\\
-159.912	-162.354\\
-162.354	-95.215\\
-95.215	-84.229\\
-84.229	-83.008\\
-83.008	-52.49\\
-52.49	-34.18\\
-34.18	-28.076\\
-28.076	-23.193\\
-23.193	-59.814\\
-59.814	-86.67\\
-86.67	-104.98\\
-104.98	-76.904\\
-76.904	-87.891\\
-87.891	-97.656\\
-97.656	-59.814\\
-59.814	-63.477\\
-63.477	-61.035\\
-61.035	-45.166\\
-45.166	-57.373\\
-57.373	-97.656\\
-97.656	-124.512\\
-124.512	-78.125\\
-78.125	-56.152\\
-56.152	-46.387\\
-46.387	-59.814\\
-59.814	-41.504\\
-41.504	-91.553\\
-91.553	-158.691\\
-158.691	-122.07\\
-122.07	-167.236\\
-167.236	-194.092\\
-194.092	-197.754\\
-197.754	-140.381\\
-140.381	-106.201\\
-106.201	-106.201\\
-106.201	-104.98\\
-104.98	-120.85\\
-120.85	-150.146\\
-150.146	-147.705\\
-147.705	-200.195\\
-200.195	-220.947\\
-220.947	-263.672\\
-263.672	-178.223\\
-178.223	-190.43\\
-190.43	-212.402\\
-212.402	-163.574\\
-163.574	-189.209\\
-189.209	-163.574\\
-163.574	-91.553\\
-91.553	-58.594\\
-58.594	-62.256\\
-62.256	-95.215\\
-95.215	-75.684\\
-75.684	-51.27\\
-51.27	-72.021\\
-72.021	-52.49\\
-52.49	-40.283\\
-40.283	-48.828\\
-48.828	-56.152\\
-56.152	-40.283\\
-40.283	-76.904\\
-76.904	-109.863\\
-109.863	-78.125\\
-78.125	-134.277\\
-134.277	-195.313\\
-195.313	-178.223\\
-178.223	-222.168\\
-222.168	-150.146\\
-150.146	-161.133\\
-161.133	-111.084\\
-111.084	-67.139\\
-67.139	-85.449\\
-85.449	-67.139\\
-67.139	-48.828\\
-48.828	-72.021\\
-72.021	-41.504\\
-41.504	-25.635\\
-25.635	-36.621\\
-36.621	-79.346\\
-79.346	-104.98\\
-104.98	-128.174\\
-128.174	-124.512\\
-124.512	-117.188\\
-117.188	-136.719\\
-136.719	-111.084\\
-111.084	-90.332\\
-90.332	-90.332\\
-90.332	-73.242\\
-73.242	-115.967\\
-115.967	-78.125\\
-78.125	-67.139\\
-67.139	-100.098\\
-100.098	-137.939\\
-137.939	-103.76\\
-103.76	-76.904\\
-76.904	-64.697\\
-64.697	-74.463\\
-74.463	-51.27\\
-51.27	-50.049\\
-50.049	-72.021\\
-72.021	-86.67\\
-86.67	-98.877\\
-98.877	-79.346\\
-79.346	-101.318\\
-101.318	-92.773\\
-92.773	-52.49\\
-52.49	-54.932\\
-54.932	-112.305\\
-112.305	-158.691\\
-158.691	-153.809\\
-153.809	-129.395\\
-129.395	-122.07\\
-122.07	-92.773\\
-92.773	-89.111\\
-89.111	-70.801\\
-70.801	-102.539\\
-102.539	-90.332\\
-90.332	-54.932\\
-54.932	-84.229\\
-84.229	-101.318\\
-101.318	-118.408\\
-118.408	-129.395\\
-129.395	-129.395\\
-129.395	-175.781\\
-175.781	-216.064\\
-216.064	-244.141\\
-244.141	-153.809\\
-153.809	-85.449\\
-85.449	-54.932\\
-54.932	-37.842\\
-37.842	-57.373\\
-57.373	-53.711\\
-53.711	-95.215\\
-95.215	-85.449\\
-85.449	-75.684\\
-75.684	-102.539\\
-102.539	-118.408\\
-118.408	-141.602\\
-141.602	-148.926\\
-148.926	-208.74\\
-208.74	-181.885\\
-181.885	-136.719\\
-136.719	-90.332\\
-90.332	-92.773\\
-92.773	-95.215\\
-95.215	-123.291\\
-123.291	-87.891\\
-87.891	-89.111\\
-89.111	-87.891\\
-87.891	-47.607\\
-47.607	-81.787\\
-81.787	-128.174\\
-128.174	-90.332\\
-90.332	-96.436\\
-96.436	-170.898\\
-170.898	-129.395\\
-129.395	-122.07\\
-122.07	-177.002\\
-177.002	-166.016\\
-166.016	-102.539\\
-102.539	-64.697\\
-64.697	-65.918\\
-65.918	-114.746\\
-114.746	-190.43\\
-190.43	-197.754\\
-197.754	-206.299\\
-206.299	-150.146\\
-150.146	-97.656\\
-97.656	-83.008\\
-83.008	-58.594\\
-58.594	-56.152\\
-56.152	-47.607\\
-47.607	-68.359\\
-68.359	-81.787\\
-81.787	-46.387\\
-46.387	-70.801\\
-70.801	-103.76\\
-103.76	-114.746\\
-114.746	-145.264\\
-145.264	-185.547\\
-185.547	-223.389\\
-223.389	-161.133\\
-161.133	-137.939\\
-137.939	-144.043\\
-144.043	-120.85\\
-120.85	-84.229\\
-84.229	-103.76\\
-103.76	-167.236\\
-167.236	-194.092\\
-194.092	-115.967\\
-115.967	-62.256\\
-62.256	-41.504\\
-41.504	-23.193\\
-23.193	-31.738\\
-31.738	-51.27\\
-51.27	-61.035\\
-61.035	-80.566\\
-80.566	-67.139\\
-67.139	-51.27\\
-51.27	-50.049\\
-50.049	-48.828\\
-48.828	-45.166\\
-45.166	-52.49\\
-52.49	-81.787\\
-81.787	-123.291\\
-123.291	-133.057\\
-133.057	-125.732\\
-125.732	-152.588\\
-152.588	-181.885\\
-181.885	-183.105\\
-183.105	-223.389\\
-223.389	-201.416\\
-201.416	-146.484\\
-146.484	-101.318\\
-101.318	-96.436\\
-96.436	-163.574\\
-163.574	-191.65\\
-191.65	-131.836\\
-131.836	-85.449\\
-85.449	-45.166\\
-45.166	-31.738\\
-31.738	-35.4\\
-35.4	-21.973\\
-21.973	-46.387\\
-46.387	-68.359\\
-68.359	-72.021\\
-72.021	-62.256\\
-62.256	-36.621\\
-36.621	-28.076\\
-28.076	-41.504\\
-41.504	-43.945\\
-43.945	-48.828\\
-48.828	-36.621\\
-36.621	-30.518\\
-30.518	-54.932\\
-54.932	-128.174\\
-128.174	-153.809\\
-153.809	-170.898\\
-170.898	-124.512\\
-124.512	-172.119\\
-172.119	-241.699\\
-241.699	-222.168\\
-222.168	-224.609\\
-224.609	-164.795\\
-164.795	-163.574\\
-163.574	-172.119\\
-172.119	-113.525\\
-113.525	-106.201\\
-106.201	-65.918\\
-65.918	-59.814\\
-59.814	-117.188\\
-117.188	-79.346\\
-79.346	-81.787\\
-81.787	-96.436\\
-96.436	-86.67\\
-86.67	-87.891\\
-87.891	-93.994\\
-93.994	-107.422\\
-107.422	-119.629\\
-119.629	-103.76\\
-103.76	-74.463\\
-74.463	-96.436\\
-96.436	-48.828\\
-48.828	-20.752\\
-20.752	-34.18\\
-34.18	-58.594\\
-58.594	-30.518\\
-30.518	-13.428\\
-13.428	-32.959\\
-32.959	-58.594\\
-58.594	-69.58\\
-69.58	-83.008\\
-83.008	-72.021\\
-72.021	-117.188\\
-117.188	-104.98\\
-104.98	-70.801\\
-70.801	-107.422\\
-107.422	-59.814\\
-59.814	-46.387\\
-46.387	-32.959\\
-32.959	-21.973\\
-21.973	-41.504\\
-41.504	-34.18\\
-34.18	-53.711\\
-53.711	-91.553\\
-91.553	-87.891\\
-87.891	-59.814\\
-59.814	-68.359\\
-68.359	-52.49\\
-52.49	-74.463\\
-74.463	-53.711\\
-53.711	-50.049\\
-50.049	-46.387\\
-46.387	-36.621\\
-36.621	-45.166\\
-45.166	-41.504\\
-41.504	-75.684\\
-75.684	-73.242\\
-73.242	-54.932\\
-54.932	-51.27\\
-51.27	-40.283\\
-40.283	-48.828\\
-48.828	-108.643\\
-108.643	-144.043\\
-144.043	-96.436\\
-96.436	-90.332\\
-90.332	-63.477\\
-63.477	-111.084\\
-111.084	-101.318\\
-101.318	-67.139\\
-67.139	-83.008\\
-83.008	-64.697\\
-64.697	-89.111\\
-89.111	-52.49\\
-52.49	-54.932\\
-54.932	-65.918\\
-65.918	-84.229\\
-84.229	-62.256\\
-62.256	-65.918\\
-65.918	-68.359\\
-68.359	-112.305\\
-112.305	-95.215\\
-95.215	-146.484\\
-146.484	-190.43\\
-190.43	-151.367\\
-151.367	-95.215\\
-95.215	-72.021\\
-72.021	-113.525\\
-113.525	-141.602\\
-141.602	-108.643\\
-108.643	-129.395\\
-129.395	-81.787\\
-81.787	-42.725\\
-42.725	-42.725\\
-42.725	-98.877\\
-98.877	-89.111\\
-89.111	-79.346\\
-79.346	-124.512\\
-124.512	-118.408\\
-118.408	-103.76\\
-103.76	-67.139\\
-67.139	-84.229\\
-84.229	-115.967\\
-115.967	-95.215\\
-95.215	-97.656\\
-97.656	-89.111\\
-89.111	-62.256\\
-62.256	-75.684\\
-75.684	-54.932\\
-54.932	-62.256\\
-62.256	-106.201\\
-106.201	-123.291\\
-123.291	-131.836\\
-131.836	-117.188\\
-117.188	-83.008\\
-83.008	-58.594\\
-58.594	-67.139\\
-67.139	-80.566\\
-80.566	-104.98\\
-104.98	-130.615\\
-130.615	-155.029\\
-155.029	-114.746\\
-114.746	-69.58\\
-69.58	-126.953\\
-126.953	-178.223\\
-178.223	-128.174\\
-128.174	-125.732\\
-125.732	-147.705\\
-147.705	-111.084\\
-111.084	-90.332\\
-90.332	-100.098\\
-100.098	-92.773\\
-92.773	-103.76\\
-103.76	-131.836\\
-131.836	-101.318\\
-101.318	-112.305\\
-112.305	-106.201\\
-106.201	-108.643\\
-108.643	-86.67\\
-86.67	-114.746\\
-114.746	-81.787\\
-81.787	-86.67\\
-86.67	-96.436\\
-96.436	-93.994\\
-93.994	-42.725\\
-42.725	-26.855\\
-26.855	-18.311\\
-18.311	-39.063\\
-39.063	-96.436\\
-96.436	-125.732\\
-125.732	-125.732\\
-125.732	-83.008\\
-83.008	-97.656\\
-97.656	-85.449\\
-85.449	-75.684\\
-75.684	-46.387\\
-46.387	-65.918\\
-65.918	-65.918\\
-65.918	-64.697\\
-64.697	-76.904\\
-76.904	-65.918\\
-65.918	-59.814\\
-59.814	-40.283\\
-40.283	-58.594\\
-58.594	-109.863\\
-109.863	-78.125\\
-78.125	-92.773\\
-92.773	-83.008\\
-83.008	-118.408\\
-118.408	-109.863\\
-109.863	-152.588\\
-152.588	-111.084\\
-111.084	-124.512\\
-124.512	-124.512\\
-124.512	-63.477\\
-63.477	-113.525\\
-113.525	-76.904\\
-76.904	-96.436\\
-96.436	-102.539\\
-102.539	-129.395\\
-129.395	-97.656\\
-97.656	-125.732\\
-125.732	-158.691\\
-158.691	-131.836\\
-131.836	-114.746\\
-114.746	-90.332\\
-90.332	-107.422\\
-107.422	-89.111\\
-89.111	-76.904\\
-76.904	-68.359\\
-68.359	-81.787\\
-81.787	-69.58\\
-69.58	-141.602\\
-141.602	-181.885\\
-181.885	-157.471\\
-157.471	-150.146\\
-150.146	-147.705\\
-147.705	-81.787\\
-81.787	-72.021\\
-72.021	-54.932\\
-54.932	-50.049\\
-50.049	-54.932\\
-54.932	-93.994\\
-93.994	-118.408\\
-118.408	-134.277\\
-134.277	-114.746\\
-114.746	-76.904\\
-76.904	-140.381\\
-140.381	-163.574\\
-163.574	-183.105\\
-183.105	-150.146\\
-150.146	-236.816\\
-236.816	-158.691\\
-158.691	-93.994\\
-93.994	-68.359\\
-68.359	-56.152\\
-56.152	-101.318\\
-101.318	-130.615\\
-130.615	-172.119\\
-172.119	-129.395\\
-129.395	-102.539\\
-102.539	-53.711\\
-53.711	-53.711\\
-53.711	-64.697\\
-64.697	-70.801\\
-70.801	-43.945\\
-43.945	-61.035\\
-61.035	-45.166\\
-45.166	-29.297\\
-29.297	-65.918\\
-65.918	-75.684\\
-75.684	-96.436\\
-96.436	-89.111\\
-89.111	-93.994\\
-93.994	-93.994\\
-93.994	-120.85\\
-120.85	-79.346\\
-79.346	-119.629\\
-119.629	-136.719\\
-136.719	-103.76\\
-103.76	-108.643\\
-108.643	-93.994\\
-93.994	-108.643\\
-108.643	-151.367\\
-151.367	-117.188\\
-117.188	-93.994\\
-93.994	-102.539\\
-102.539	-93.994\\
-93.994	-62.256\\
-62.256	-100.098\\
-100.098	-145.264\\
-145.264	-139.16\\
-139.16	-151.367\\
-151.367	-227.051\\
-227.051	-139.16\\
-139.16	-122.07\\
-122.07	-91.553\\
-91.553	-119.629\\
-119.629	-172.119\\
-172.119	-114.746\\
-114.746	-102.539\\
-102.539	-142.822\\
-142.822	-91.553\\
-91.553	-53.711\\
-53.711	-70.801\\
-70.801	-52.49\\
-52.49	-42.725\\
-42.725	-46.387\\
-46.387	-40.283\\
-40.283	-39.063\\
-39.063	-53.711\\
-53.711	-76.904\\
-76.904	-90.332\\
-90.332	-119.629\\
-119.629	-112.305\\
-112.305	-135.498\\
-135.498	-157.471\\
-157.471	-141.602\\
-141.602	-96.436\\
-96.436	-106.201\\
-106.201	-95.215\\
-95.215	-103.76\\
-103.76	-147.705\\
-147.705	-109.863\\
-109.863	-122.07\\
-122.07	-106.201\\
-106.201	-102.539\\
-102.539	-158.691\\
-158.691	-130.615\\
-130.615	-129.395\\
-129.395	-122.07\\
-122.07	-73.242\\
-73.242	-47.607\\
-47.607	-37.842\\
-37.842	-69.58\\
-69.58	-62.256\\
-62.256	-86.67\\
-86.67	-65.918\\
-65.918	-93.994\\
-93.994	-152.588\\
-152.588	-189.209\\
-189.209	-150.146\\
-150.146	-153.809\\
-153.809	-137.939\\
-137.939	-96.436\\
-96.436	-103.76\\
-103.76	-117.188\\
-117.188	-100.098\\
-100.098	-114.746\\
-114.746	-145.264\\
-145.264	-202.637\\
-202.637	-181.885\\
-181.885	-170.898\\
-170.898	-191.65\\
-191.65	-131.836\\
-131.836	-144.043\\
-144.043	-108.643\\
-108.643	-112.305\\
-112.305	-147.705\\
-147.705	-172.119\\
-172.119	-79.346\\
-79.346	-89.111\\
-89.111	-67.139\\
-67.139	-98.877\\
-98.877	-100.098\\
-100.098	-61.035\\
-61.035	-81.787\\
-81.787	-139.16\\
-139.16	-234.375\\
-234.375	-227.051\\
-227.051	-206.299\\
-206.299	-166.016\\
-166.016	-194.092\\
-194.092	-233.154\\
-233.154	-170.898\\
-170.898	-177.002\\
-177.002	-184.326\\
-184.326	-125.732\\
-125.732	-106.201\\
-106.201	-137.939\\
-137.939	-91.553\\
-91.553	-67.139\\
-67.139	-97.656\\
-97.656	-73.242\\
-73.242	-86.67\\
-86.67	-81.787\\
-81.787	-100.098\\
-100.098	-133.057\\
-133.057	-102.539\\
-102.539	-75.684\\
-75.684	-89.111\\
-89.111	-104.98\\
-104.98	-124.512\\
-124.512	-107.422\\
-107.422	-86.67\\
-86.67	-136.719\\
-136.719	-129.395\\
-129.395	-90.332\\
-90.332	-148.926\\
-148.926	-134.277\\
-134.277	-97.656\\
-97.656	-64.697\\
-64.697	-46.387\\
-46.387	-32.959\\
-32.959	-75.684\\
-75.684	-72.021\\
-72.021	-52.49\\
-52.49	-36.621\\
-36.621	-47.607\\
-47.607	-81.787\\
-81.787	-93.994\\
-93.994	-124.512\\
-124.512	-142.822\\
-142.822	-137.939\\
-137.939	-145.264\\
-145.264	-87.891\\
-87.891	-36.621\\
-36.621	-47.607\\
-47.607	-46.387\\
-46.387	-64.697\\
-64.697	-100.098\\
-100.098	-111.084\\
-111.084	-168.457\\
-168.457	-106.201\\
-106.201	-104.98\\
-104.98	-155.029\\
-155.029	-106.201\\
-106.201	-74.463\\
-74.463	-53.711\\
-53.711	-76.904\\
-76.904	-100.098\\
-100.098	-151.367\\
-151.367	-98.877\\
-98.877	-85.449\\
-85.449	-80.566\\
-80.566	-101.318\\
-101.318	-133.057\\
-133.057	-207.52\\
-207.52	-184.326\\
-184.326	-180.664\\
-180.664	-112.305\\
-112.305	-75.684\\
-75.684	-87.891\\
-87.891	-36.621\\
-36.621	-67.139\\
-67.139	-52.49\\
-52.49	-65.918\\
-65.918	-117.188\\
-117.188	-153.809\\
-153.809	-177.002\\
-177.002	-234.375\\
-234.375	-196.533\\
-196.533	-106.201\\
-106.201	-98.877\\
-98.877	-79.346\\
-79.346	-108.643\\
-108.643	-141.602\\
-141.602	-191.65\\
-191.65	-140.381\\
-140.381	-97.656\\
-97.656	-108.643\\
-108.643	-144.043\\
-144.043	-102.539\\
-102.539	-73.242\\
-73.242	-57.373\\
-57.373	-40.283\\
-40.283	-36.621\\
-36.621	-29.297\\
-29.297	-54.932\\
-54.932	-80.566\\
-80.566	-86.67\\
-86.67	-64.697\\
-64.697	-58.594\\
-58.594	-26.855\\
-26.855	-15.869\\
-15.869	-24.414\\
-24.414	-46.387\\
-46.387	-70.801\\
-70.801	-87.891\\
-87.891	-102.539\\
-102.539	-119.629\\
-119.629	-153.809\\
-153.809	-212.402\\
-212.402	-213.623\\
-213.623	-230.713\\
-230.713	-201.416\\
-201.416	-111.084\\
-111.084	-102.539\\
-102.539	-101.318\\
-101.318	-135.498\\
-135.498	-113.525\\
-113.525	-80.566\\
-80.566	-61.035\\
-61.035	-50.049\\
-50.049	-93.994\\
-93.994	-90.332\\
-90.332	-47.607\\
-47.607	-37.842\\
-37.842	-58.594\\
-58.594	-78.125\\
-78.125	-59.814\\
-59.814	-86.67\\
-86.67	-67.139\\
-67.139	-98.877\\
-98.877	-141.602\\
-141.602	-112.305\\
-112.305	-153.809\\
-153.809	-206.299\\
-206.299	-224.609\\
-224.609	-175.781\\
-175.781	-111.084\\
-111.084	-83.008\\
-83.008	-48.828\\
-48.828	-81.787\\
-81.787	-57.373\\
-57.373	-109.863\\
-109.863	-92.773\\
-92.773	-74.463\\
-74.463	-96.436\\
-96.436	-52.49\\
-52.49	-84.229\\
-84.229	-58.594\\
-58.594	-37.842\\
-37.842	-45.166\\
-45.166	-32.959\\
-32.959	-40.283\\
-40.283	-35.4\\
-35.4	-26.855\\
-26.855	-36.621\\
-36.621	-48.828\\
-48.828	-50.049\\
-50.049	-48.828\\
-48.828	-79.346\\
-79.346	-103.76\\
-103.76	-84.229\\
-84.229	-78.125\\
-78.125	-128.174\\
-128.174	-153.809\\
-153.809	-119.629\\
-119.629	-125.732\\
-125.732	-56.152\\
-56.152	-36.621\\
-36.621	-73.242\\
-73.242	-106.201\\
-106.201	-115.967\\
-115.967	-162.354\\
-162.354	-146.484\\
-146.484	-102.539\\
-102.539	-73.242\\
-73.242	-79.346\\
-79.346	-84.229\\
-84.229	-64.697\\
-64.697	-39.063\\
-39.063	-52.49\\
-52.49	-39.063\\
-39.063	-32.959\\
-32.959	-24.414\\
-24.414	-53.711\\
-53.711	-74.463\\
-74.463	-79.346\\
-79.346	-56.152\\
-56.152	-100.098\\
-100.098	-136.719\\
-136.719	-86.67\\
-86.67	-120.85\\
-120.85	-130.615\\
-130.615	-98.877\\
-98.877	-56.152\\
-56.152	-72.021\\
-72.021	-87.891\\
-87.891	-47.607\\
-47.607	-50.049\\
-50.049	-41.504\\
-41.504	-24.414\\
-24.414	-26.855\\
-26.855	-46.387\\
-46.387	-64.697\\
-64.697	-54.932\\
-54.932	-72.021\\
-72.021	-79.346\\
-79.346	-57.373\\
-57.373	-31.738\\
-31.738	-25.635\\
-25.635	-34.18\\
-34.18	-39.063\\
-39.063	-48.828\\
-48.828	-104.98\\
-104.98	-101.318\\
-101.318	-89.111\\
-89.111	-85.449\\
-85.449	-117.188\\
-117.188	-150.146\\
-150.146	-159.912\\
-159.912	-166.016\\
-166.016	-120.85\\
-120.85	-126.953\\
-126.953	-131.836\\
-131.836	-133.057\\
-133.057	-151.367\\
-151.367	-200.195\\
-200.195	-169.678\\
-169.678	-156.25\\
-156.25	-128.174\\
-128.174	-119.629\\
-119.629	-114.746\\
-114.746	-136.719\\
-136.719	-83.008\\
-83.008	-98.877\\
-98.877	-119.629\\
-119.629	-140.381\\
-140.381	-107.422\\
-107.422	-124.512\\
-124.512	-98.877\\
-98.877	-90.332\\
-90.332	-54.932\\
-54.932	-85.449\\
-85.449	-137.939\\
-137.939	-109.863\\
-109.863	-64.697\\
-64.697	-80.566\\
-80.566	-41.504\\
-41.504	-40.283\\
-40.283	-56.152\\
-56.152	-32.959\\
-32.959	-36.621\\
-36.621	-43.945\\
-43.945	-26.855\\
-26.855	-50.049\\
-50.049	-40.283\\
-40.283	-23.193\\
-23.193	-43.945\\
-43.945	-51.27\\
-51.27	-63.477\\
-63.477	-62.256\\
-62.256	-63.477\\
-63.477	-84.229\\
-84.229	-76.904\\
-76.904	-86.67\\
-86.67	-146.484\\
-146.484	-101.318\\
-101.318	-92.773\\
-92.773	-100.098\\
-100.098	-69.58\\
-69.58	-40.283\\
-40.283	-79.346\\
-79.346	-61.035\\
-61.035	-37.842\\
-37.842	-67.139\\
-67.139	-62.256\\
-62.256	-80.566\\
-80.566	-48.828\\
-48.828	-29.297\\
-29.297	-23.193\\
-23.193	-29.297\\
-29.297	-56.152\\
-56.152	-54.932\\
-54.932	-65.918\\
-65.918	-87.891\\
-87.891	-117.188\\
-117.188	-84.229\\
-84.229	-124.512\\
-124.512	-141.602\\
-141.602	-124.512\\
-124.512	-117.188\\
-117.188	-195.313\\
-195.313	-135.498\\
-135.498	-173.34\\
-173.34	-141.602\\
-141.602	-164.795\\
-164.795	-184.326\\
-184.326	-101.318\\
-101.318	-63.477\\
-63.477	-51.27\\
-51.27	-84.229\\
-84.229	-104.98\\
-104.98	-108.643\\
-108.643	-74.463\\
-74.463	-40.283\\
-40.283	-52.49\\
-52.49	-104.98\\
-104.98	-145.264\\
-145.264	-142.822\\
-142.822	-84.229\\
-84.229	-64.697\\
-64.697	-89.111\\
-89.111	-141.602\\
-141.602	-161.133\\
-161.133	-164.795\\
-164.795	-115.967\\
-115.967	-166.016\\
-166.016	-178.223\\
-178.223	-170.898\\
-170.898	-240.479\\
-240.479	-203.857\\
-203.857	-169.678\\
-169.678	-196.533\\
-196.533	-164.795\\
-164.795	-91.553\\
-91.553	-84.229\\
-84.229	-104.98\\
-104.98	-125.732\\
-125.732	-128.174\\
-128.174	-96.436\\
-96.436	-124.512\\
-124.512	-106.201\\
-106.201	-78.125\\
-78.125	-106.201\\
-106.201	-95.215\\
-95.215	-150.146\\
-150.146	-151.367\\
-151.367	-108.643\\
-108.643	-79.346\\
-79.346	-45.166\\
-45.166	-78.125\\
-78.125	-53.711\\
-53.711	-52.49\\
-52.49	-42.725\\
-42.725	-76.904\\
-76.904	-98.877\\
-98.877	-112.305\\
-112.305	-109.863\\
-109.863	-64.697\\
-64.697	-48.828\\
-48.828	-35.4\\
-35.4	-43.945\\
-43.945	-70.801\\
-70.801	-75.684\\
-75.684	-74.463\\
-74.463	-120.85\\
-120.85	-140.381\\
-140.381	-162.354\\
-162.354	-159.912\\
-159.912	-187.988\\
-187.988	-109.863\\
-109.863	-91.553\\
-91.553	-72.021\\
-72.021	-78.125\\
-78.125	-104.98\\
-104.98	-86.67\\
-86.67	-57.373\\
-57.373	-50.049\\
-50.049	-97.656\\
-97.656	-117.188\\
-117.188	-107.422\\
-107.422	-168.457\\
-168.457	-184.326\\
-184.326	-130.615\\
-130.615	-91.553\\
-91.553	-62.256\\
-62.256	-57.373\\
-57.373	-107.422\\
-107.422	-85.449\\
-85.449	-103.76\\
-103.76	-113.525\\
-113.525	-129.395\\
-129.395	-122.07\\
-122.07	-135.498\\
-135.498	-91.553\\
-91.553	-109.863\\
-109.863	-75.684\\
-75.684	-68.359\\
-68.359	-107.422\\
-107.422	-147.705\\
-147.705	-162.354\\
-162.354	-91.553\\
-91.553	-190.43\\
-190.43	-170.898\\
-170.898	-113.525\\
-113.525	-65.918\\
-65.918	-96.436\\
-96.436	-83.008\\
-83.008	-80.566\\
-80.566	-98.877\\
-98.877	-64.697\\
-64.697	-93.994\\
-93.994	-170.898\\
-170.898	-179.443\\
-179.443	-117.188\\
-117.188	-134.277\\
-134.277	-146.484\\
-146.484	-148.926\\
-148.926	-106.201\\
-106.201	-106.201\\
-106.201	-69.58\\
-69.58	-106.201\\
-106.201	-109.863\\
-109.863	-175.781\\
-175.781	-195.313\\
-195.313	-263.672\\
-263.672	-190.43\\
-190.43	-155.029\\
-155.029	-118.408\\
-118.408	-130.615\\
-130.615	-97.656\\
-97.656	-67.139\\
-67.139	-68.359\\
-68.359	-68.359\\
-68.359	-48.828\\
-48.828	-35.4\\
-35.4	-59.814\\
-59.814	-81.787\\
-81.787	-54.932\\
-54.932	-39.063\\
-39.063	-37.842\\
-37.842	-90.332\\
-90.332	-128.174\\
-128.174	-59.814\\
-59.814	-74.463\\
-74.463	-79.346\\
-79.346	-73.242\\
-73.242	-91.553\\
-91.553	-79.346\\
-79.346	-54.932\\
-54.932	-41.504\\
-41.504	-34.18\\
-34.18	-43.945\\
-43.945	-86.67\\
-86.67	-103.76\\
-103.76	-72.021\\
-72.021	-48.828\\
-48.828	-29.297\\
-29.297	-42.725\\
-42.725	-80.566\\
-80.566	-101.318\\
-101.318	-107.422\\
-107.422	-76.904\\
-76.904	-70.801\\
-70.801	-78.125\\
-78.125	-108.643\\
-108.643	-151.367\\
-151.367	-175.781\\
-175.781	-131.836\\
-131.836	-83.008\\
-83.008	-86.67\\
-86.67	-81.787\\
-81.787	-84.229\\
-84.229	-54.932\\
-54.932	-69.58\\
-69.58	-76.904\\
-76.904	-76.904\\
-76.904	-45.166\\
-45.166	-29.297\\
-29.297	-47.607\\
-47.607	-70.801\\
-70.801	-109.863\\
-109.863	-118.408\\
-118.408	-148.926\\
-148.926	-130.615\\
-130.615	-146.484\\
-146.484	-195.313\\
-195.313	-260.01\\
-260.01	-212.402\\
-212.402	-142.822\\
-142.822	-155.029\\
-155.029	-180.664\\
-180.664	-236.816\\
-236.816	-155.029\\
-155.029	-76.904\\
-76.904	-50.049\\
-50.049	-52.49\\
-52.49	-37.842\\
-37.842	-23.193\\
-23.193	-30.518\\
-30.518	-48.828\\
-48.828	-68.359\\
-68.359	-70.801\\
-70.801	-54.932\\
-54.932	-48.828\\
-48.828	-67.139\\
-67.139	-81.787\\
-81.787	-85.449\\
-85.449	-79.346\\
-79.346	-57.373\\
-57.373	-37.842\\
-37.842	-37.842\\
-37.842	-57.373\\
-57.373	-73.242\\
-73.242	-53.711\\
-53.711	-35.4\\
-35.4	-58.594\\
-58.594	-39.063\\
-39.063	-50.049\\
-50.049	-63.477\\
-63.477	-92.773\\
-92.773	-100.098\\
-100.098	-68.359\\
-68.359	-106.201\\
-106.201	-102.539\\
-102.539	-83.008\\
-83.008	-70.801\\
-70.801	-119.629\\
-119.629	-146.484\\
-146.484	-144.043\\
-144.043	-158.691\\
-158.691	-120.85\\
-120.85	-111.084\\
-111.084	-102.539\\
-102.539	-139.16\\
-139.16	-108.643\\
-108.643	-147.705\\
-147.705	-141.602\\
-141.602	-146.484\\
-146.484	-249.023\\
-249.023	-190.43\\
-190.43	-102.539\\
-102.539	-69.58\\
-69.58	-73.242\\
-73.242	-91.553\\
-91.553	-119.629\\
-119.629	-139.16\\
-139.16	-153.809\\
-153.809	-157.471\\
-157.471	-96.436\\
-96.436	-86.67\\
-86.67	-101.318\\
-101.318	-102.539\\
-102.539	-61.035\\
-61.035	-43.945\\
-43.945	-79.346\\
-79.346	-80.566\\
-80.566	-109.863\\
-109.863	-139.16\\
-139.16	-123.291\\
-123.291	-87.891\\
-87.891	-70.801\\
-70.801	-107.422\\
-107.422	-139.16\\
-139.16	-185.547\\
-185.547	-198.975\\
-198.975	-231.934\\
-231.934	-184.326\\
-184.326	-168.457\\
-168.457	-89.111\\
-89.111	-67.139\\
-67.139	-126.953\\
-126.953	-163.574\\
-163.574	-93.994\\
-93.994	-151.367\\
-151.367	-203.857\\
-203.857	-246.582\\
-246.582	-153.809\\
-153.809	-86.67\\
-86.67	-50.049\\
-50.049	-75.684\\
-75.684	-69.58\\
-69.58	-75.684\\
-75.684	-42.725\\
-42.725	-61.035\\
-61.035	-45.166\\
-45.166	-58.594\\
-58.594	-48.828\\
-48.828	-62.256\\
-62.256	-101.318\\
-101.318	-91.553\\
-91.553	-129.395\\
-129.395	-156.25\\
-156.25	-155.029\\
-155.029	-85.449\\
-85.449	-86.67\\
-86.67	-106.201\\
-106.201	-70.801\\
-70.801	-63.477\\
-63.477	-45.166\\
-45.166	-73.242\\
-73.242	-69.58\\
-69.58	-39.063\\
-39.063	-21.973\\
-21.973	-25.635\\
-25.635	-46.387\\
-46.387	-72.021\\
-72.021	-75.684\\
-75.684	-45.166\\
-45.166	-57.373\\
-57.373	-84.229\\
-84.229	-109.863\\
-109.863	-76.904\\
-76.904	-72.021\\
-72.021	-84.229\\
-84.229	-101.318\\
-101.318	-92.773\\
-92.773	-109.863\\
-109.863	-103.76\\
-103.76	-70.801\\
-70.801	-56.152\\
-56.152	-72.021\\
-72.021	-59.814\\
-59.814	-75.684\\
-75.684	-102.539\\
-102.539	-157.471\\
-157.471	-109.863\\
-109.863	-107.422\\
-107.422	-166.016\\
-166.016	-125.732\\
-125.732	-75.684\\
-75.684	-54.932\\
-54.932	-43.945\\
-43.945	-46.387\\
-46.387	-39.063\\
-39.063	-40.283\\
-40.283	-84.229\\
-84.229	-26.855\\
-26.855	-37.842\\
-37.842	-53.711\\
-53.711	-72.021\\
-72.021	-56.152\\
-56.152	-40.283\\
-40.283	-24.414\\
-24.414	-42.725\\
-42.725	-84.229\\
-84.229	-111.084\\
-111.084	-87.891\\
-87.891	-73.242\\
-73.242	-83.008\\
-83.008	-84.229\\
-84.229	-96.436\\
-96.436	-117.188\\
-117.188	-139.16\\
-139.16	-133.057\\
-133.057	-114.746\\
-114.746	-74.463\\
-74.463	-58.594\\
-58.594	-61.035\\
-61.035	-91.553\\
-91.553	-57.373\\
-57.373	-50.049\\
-50.049	-92.773\\
-92.773	-113.525\\
-113.525	-108.643\\
-108.643	-124.512\\
-124.512	-164.795\\
-164.795	-85.449\\
-85.449	-145.264\\
-145.264	-152.588\\
-152.588	-140.381\\
-140.381	-152.588\\
-152.588	-148.926\\
-148.926	-253.906\\
-253.906	-235.596\\
-235.596	-158.691\\
-158.691	-189.209\\
-189.209	-233.154\\
-233.154	-223.389\\
-223.389	-253.906\\
-253.906	-227.051\\
-227.051	-311.279\\
-311.279	-234.375\\
-234.375	-159.912\\
-159.912	-98.877\\
-98.877	-100.098\\
-100.098	-79.346\\
-79.346	-64.697\\
-64.697	-100.098\\
-100.098	-142.822\\
-142.822	-87.891\\
-87.891	-76.904\\
-76.904	-73.242\\
-73.242	-48.828\\
-48.828	-63.477\\
-63.477	-34.18\\
-34.18	-40.283\\
-40.283	-35.4\\
-35.4	-48.828\\
-48.828	-37.842\\
-37.842	-31.738\\
-31.738	-17.09\\
-17.09	-18.311\\
-18.311	-40.283\\
-40.283	-28.076\\
-28.076	-29.297\\
-29.297	-65.918\\
-65.918	-93.994\\
-93.994	-93.994\\
-93.994	-68.359\\
-68.359	-86.67\\
-86.67	-135.498\\
-135.498	-70.801\\
-70.801	-37.842\\
-37.842	-29.297\\
-29.297	-24.414\\
-24.414	-40.283\\
-40.283	-67.139\\
-67.139	-100.098\\
-100.098	-62.256\\
-62.256	-100.098\\
-100.098	-144.043\\
-144.043	-145.264\\
-145.264	-103.76\\
-103.76	-62.256\\
-62.256	-83.008\\
-83.008	-113.525\\
-113.525	-148.926\\
-148.926	-83.008\\
-83.008	-42.725\\
-42.725	-67.139\\
-67.139	-53.711\\
-53.711	-24.414\\
-24.414	-15.869\\
-15.869	-42.725\\
-42.725	-92.773\\
-92.773	-98.877\\
-98.877	-70.801\\
-70.801	-43.945\\
-43.945	-47.607\\
-47.607	-20.752\\
-20.752	-28.076\\
-28.076	-29.297\\
-29.297	-46.387\\
-46.387	-69.58\\
-69.58	-103.76\\
-103.76	-107.422\\
-107.422	-120.85\\
-120.85	-124.512\\
-124.512	-120.85\\
-120.85	-115.967\\
-115.967	-101.318\\
-101.318	-123.291\\
-123.291	-187.988\\
-187.988	-169.678\\
-169.678	-87.891\\
-87.891	-45.166\\
-45.166	-103.76\\
-103.76	-122.07\\
-122.07	-114.746\\
-114.746	-63.477\\
-63.477	-67.139\\
-67.139	-113.525\\
-113.525	-175.781\\
-175.781	-180.664\\
-180.664	-203.857\\
-203.857	-195.313\\
-195.313	-139.16\\
-139.16	-144.043\\
-144.043	-59.814\\
-59.814	-61.035\\
-61.035	-76.904\\
-76.904	-98.877\\
-98.877	-104.98\\
-104.98	-111.084\\
-111.084	-70.801\\
-70.801	-98.877\\
-98.877	-86.67\\
-86.67	-47.607\\
-47.607	-32.959\\
-32.959	-26.855\\
-26.855	-42.725\\
-42.725	-35.4\\
-35.4	-19.531\\
-19.531	-40.283\\
-40.283	-85.449\\
-85.449	-56.152\\
-56.152	-51.27\\
-51.27	-72.021\\
-72.021	-119.629\\
-119.629	-81.787\\
-81.787	-42.725\\
-42.725	-21.973\\
-21.973	-57.373\\
-57.373	-126.953\\
-126.953	-161.133\\
-161.133	-223.389\\
-223.389	-266.113\\
-266.113	-261.23\\
-261.23	-168.457\\
-168.457	-170.898\\
-170.898	-222.168\\
-222.168	-195.313\\
-195.313	-179.443\\
-179.443	-202.637\\
-202.637	-220.947\\
-220.947	-223.389\\
-223.389	-301.514\\
-301.514	-203.857\\
-203.857	-109.863\\
-109.863	-59.814\\
-59.814	-47.607\\
-47.607	-54.932\\
-54.932	-53.711\\
-53.711	-56.152\\
-56.152	-101.318\\
-101.318	-79.346\\
-79.346	-84.229\\
-84.229	-111.084\\
-111.084	-63.477\\
-63.477	-70.801\\
-70.801	-101.318\\
-101.318	-115.967\\
-115.967	-64.697\\
-64.697	-46.387\\
-46.387	-58.594\\
-58.594	-54.932\\
-54.932	-129.395\\
-129.395	-186.768\\
-186.768	-173.34\\
-173.34	-198.975\\
-198.975	-225.83\\
-225.83	-297.852\\
-297.852	-257.568\\
-257.568	-157.471\\
-157.471	-100.098\\
-100.098	-54.932\\
-54.932	-61.035\\
-61.035	-67.139\\
-67.139	-86.67\\
-86.67	-62.256\\
-62.256	-56.152\\
-56.152	-63.477\\
-63.477	-78.125\\
-78.125	-89.111\\
-89.111	-111.084\\
-111.084	-79.346\\
-79.346	-40.283\\
-40.283	-28.076\\
-28.076	-23.193\\
-23.193	-24.414\\
-24.414	-34.18\\
-34.18	-57.373\\
-57.373	-59.814\\
-59.814	-75.684\\
-75.684	-114.746\\
-114.746	-81.787\\
-81.787	-150.146\\
-150.146	-208.74\\
-208.74	-162.354\\
-162.354	-144.043\\
-144.043	-170.898\\
-170.898	-198.975\\
-198.975	-131.836\\
-131.836	-111.084\\
-111.084	-72.021\\
-72.021	-79.346\\
-79.346	-142.822\\
-142.822	-247.803\\
-247.803	-300.293\\
-300.293	-239.258\\
-239.258	-200.195\\
-200.195	-145.264\\
-145.264	-153.809\\
-153.809	-208.74\\
-208.74	-115.967\\
-115.967	-98.877\\
-98.877	-75.684\\
-75.684	-148.926\\
-148.926	-136.719\\
-136.719	-76.904\\
-76.904	-72.021\\
-72.021	-53.711\\
-53.711	-95.215\\
-95.215	-145.264\\
-145.264	-148.926\\
-148.926	-113.525\\
-113.525	-83.008\\
-83.008	-93.994\\
-93.994	-74.463\\
-74.463	-53.711\\
-53.711	-43.945\\
-43.945	-39.063\\
-39.063	-29.297\\
-29.297	-35.4\\
-35.4	-65.918\\
-65.918	-95.215\\
-95.215	-96.436\\
-96.436	-58.594\\
-58.594	-53.711\\
-53.711	-41.504\\
-41.504	-54.932\\
-54.932	-78.125\\
-78.125	-86.67\\
-86.67	-80.566\\
-80.566	-46.387\\
-46.387	-96.436\\
-96.436	-142.822\\
-142.822	-101.318\\
-101.318	-40.283\\
-40.283	-50.049\\
-50.049	-76.904\\
-76.904	-73.242\\
-73.242	-67.139\\
-67.139	-41.504\\
-41.504	-75.684\\
-75.684	-117.188\\
-117.188	-70.801\\
-70.801	-25.635\\
-25.635	-39.063\\
-39.063	-91.553\\
-91.553	-111.084\\
-111.084	-65.918\\
-65.918	-59.814\\
-59.814	-113.525\\
-113.525	-151.367\\
-151.367	-131.836\\
-131.836	-179.443\\
-179.443	-219.727\\
-219.727	-140.381\\
-140.381	-111.084\\
-111.084	-159.912\\
-159.912	-109.863\\
-109.863	-142.822\\
-142.822	-174.561\\
-174.561	-137.939\\
-137.939	-69.58\\
-69.58	-111.084\\
-111.084	-202.637\\
-202.637	-233.154\\
-233.154	-169.678\\
-169.678	-142.822\\
-142.822	-191.65\\
-191.65	-153.809\\
-153.809	-96.436\\
-96.436	-85.449\\
-85.449	-107.422\\
-107.422	-113.525\\
-113.525	-109.863\\
-109.863	-125.732\\
-125.732	-87.891\\
-87.891	-90.332\\
-90.332	-130.615\\
-130.615	-78.125\\
-78.125	-62.256\\
-62.256	-50.049\\
-50.049	-40.283\\
-40.283	-46.387\\
-46.387	-85.449\\
-85.449	-79.346\\
-79.346	-107.422\\
-107.422	-139.16\\
-139.16	-164.795\\
-164.795	-123.291\\
-123.291	-107.422\\
-107.422	-41.504\\
-41.504	-67.139\\
-67.139	-129.395\\
-129.395	-85.449\\
-85.449	-41.504\\
-41.504	-87.891\\
-87.891	-137.939\\
-137.939	-142.822\\
-142.822	-185.547\\
-185.547	-245.361\\
-245.361	-173.34\\
-173.34	-147.705\\
-147.705	-170.898\\
-170.898	-113.525\\
-113.525	-90.332\\
-90.332	-102.539\\
-102.539	-131.836\\
-131.836	-141.602\\
-141.602	-142.822\\
-142.822	-79.346\\
-79.346	-70.801\\
-70.801	-54.932\\
-54.932	-79.346\\
-79.346	-75.684\\
-75.684	-59.814\\
-59.814	-119.629\\
-119.629	-131.836\\
-131.836	-89.111\\
-89.111	-74.463\\
-74.463	-108.643\\
-108.643	-63.477\\
-63.477	-35.4\\
-35.4	-47.607\\
-47.607	-64.697\\
-64.697	-56.152\\
-56.152	-41.504\\
-41.504	-23.193\\
-23.193	-42.725\\
-42.725	-64.697\\
-64.697	-102.539\\
-102.539	-118.408\\
-118.408	-152.588\\
-152.588	-173.34\\
-173.34	-91.553\\
-91.553	-69.58\\
-69.58	-136.719\\
-136.719	-211.182\\
-211.182	-164.795\\
-164.795	-151.367\\
-151.367	-230.713\\
-230.713	-185.547\\
-185.547	-148.926\\
-148.926	-112.305\\
-112.305	-114.746\\
-114.746	-155.029\\
-155.029	-107.422\\
-107.422	-90.332\\
-90.332	-156.25\\
-156.25	-196.533\\
-196.533	-205.078\\
-205.078	-92.773\\
-92.773	-43.945\\
-43.945	-111.084\\
-111.084	-87.891\\
-87.891	-37.842\\
-37.842	-29.297\\
-29.297	-53.711\\
-53.711	-81.787\\
-81.787	-135.498\\
-135.498	-96.436\\
-96.436	-70.801\\
-70.801	-42.725\\
-42.725	-28.076\\
-28.076	-39.063\\
-39.063	-35.4\\
-35.4	-45.166\\
-45.166	-79.346\\
-79.346	-72.021\\
-72.021	-32.959\\
-32.959	-30.518\\
-30.518	-41.504\\
-41.504	-53.711\\
-53.711	-32.959\\
-32.959	-51.27\\
-51.27	-37.842\\
-37.842	-25.635\\
-25.635	-63.477\\
-63.477	-96.436\\
-96.436	-145.264\\
-145.264	-190.43\\
-190.43	-161.133\\
-161.133	-80.566\\
-80.566	-57.373\\
-57.373	-58.594\\
-58.594	-32.959\\
-32.959	-30.518\\
-30.518	-64.697\\
-64.697	-73.242\\
-73.242	-67.139\\
-67.139	-73.242\\
-73.242	-89.111\\
-89.111	-93.994\\
-93.994	-97.656\\
-97.656	-122.07\\
-122.07	-118.408\\
-118.408	-170.898\\
-170.898	-89.111\\
-89.111	-41.504\\
-41.504	-40.283\\
-40.283	-76.904\\
-76.904	-63.477\\
-63.477	-57.373\\
-57.373	-26.855\\
-26.855	-45.166\\
-45.166	-28.076\\
-28.076	-13.428\\
-13.428	-25.635\\
-25.635	-52.49\\
-52.49	-64.697\\
-64.697	-96.436\\
-96.436	-136.719\\
-136.719	-98.877\\
-98.877	-43.945\\
-43.945	-80.566\\
-80.566	-90.332\\
-90.332	-42.725\\
-42.725	-57.373\\
-57.373	-112.305\\
-112.305	-97.656\\
-97.656	-162.354\\
-162.354	-189.209\\
-189.209	-117.188\\
-117.188	-92.773\\
-92.773	-80.566\\
};
\addplot [color=mycolor2, line width=2.0pt, forget plot]
  table[row sep=crcr]{%
-97.656	-93.4078599971613\\
-112.305	-107.419612896096\\
-144.043	-137.776976095387\\
-117.188	-112.090197195742\\
-111.084	-106.251727696451\\
-156.25	-149.452958595032\\
-147.705	-141.279675195387\\
-175.781	-168.134339294677\\
-134.277	-128.435807496097\\
-68.359	-65.3853106982259\\
-80.566	-77.0612931978711\\
-100.098	-95.7436303964513\\
-81.787	-78.2291783975161\\
-34.18	-32.6931335985804\\
-24.414	-23.3519649992903\\
-23.193	-22.1840797996453\\
-73.242	-70.055894997871\\
-128.174	-122.598294495742\\
-133.057	-127.268878795387\\
-137.939	-131.938506596097\\
-92.773	-88.7372756975162\\
-56.152	-53.7093281985807\\
-125.732	-120.262524096452\\
-86.67	-82.8997626971611\\
-68.359	-65.3853106982259\\
-114.746	-109.754426796451\\
-100.098	-95.7436303964513\\
-85.449	-81.7318774975161\\
-142.822	-136.609090895742\\
-133.057	-127.268878795387\\
-102.539	-98.0784442968063\\
-150.146	-143.614489095742\\
-145.264	-138.944861295032\\
-113.525	-108.586541596806\\
-92.773	-88.7372756975162\\
-72.021	-68.8880097982259\\
-86.67	-82.8997626971611\\
-112.305	-107.419612896096\\
-103.76	-99.2463294964514\\
-109.863	-105.083842496806\\
-112.305	-107.419612896096\\
-122.07	-116.759824996452\\
-175.781	-168.134339294677\\
-155.029	-148.285073395387\\
-102.539	-98.0784442968063\\
-92.773	-88.7372756975162\\
-53.711	-51.3745142982257\\
-58.594	-56.0450985978708\\
-79.346	-75.894364497161\\
-59.814	-57.2120272985808\\
-62.256	-59.5477976978708\\
-89.111	-85.2345765975162\\
-102.539	-98.0784442968063\\
-95.215	-91.0730460968063\\
-151.367	-144.782374295387\\
-111.084	-106.251727696451\\
-64.697	-61.8826115982258\\
-51.27	-49.0397003978707\\
-69.58	-66.5531958978709\\
-81.787	-78.2291783975161\\
-43.945	-42.0333456989356\\
-45.166	-43.2012308985806\\
-67.139	-64.2183819975159\\
-51.27	-49.0397003978707\\
-42.725	-40.8664169982256\\
-61.035	-58.3799124982258\\
-70.801	-67.7210810975159\\
-109.863	-105.083842496806\\
-104.98	-100.413258197161\\
-129.395	-123.766179695387\\
-120.85	-115.592896295742\\
-137.939	-131.938506596097\\
-109.863	-105.083842496806\\
-107.422	-102.749028596451\\
-92.773	-88.7372756975162\\
-91.553	-87.5703469968062\\
-87.891	-84.0676478968062\\
-157.471	-150.620843794677\\
-224.609	-214.838269293258\\
-231.934	-221.844623992193\\
-225.83	-216.006154492903\\
-140.381	-134.274276995387\\
-205.078	-196.156888593613\\
-252.686	-241.693889891483\\
-261.23	-249.866216792193\\
-200.195	-191.486304293968\\
-270.996	-259.207385391484\\
-335.693	-321.089996989709\\
-235.596	-225.347323092193\\
-191.65	-183.313020894323\\
-128.174	-122.598294495742\\
-98.877	-94.5757451968063\\
-84.229	-80.5649487968061\\
-104.98	-100.413258197161\\
-74.463	-71.223780197516\\
-50.049	-47.8718151982257\\
-48.828	-46.7039299985806\\
-63.477	-60.7156828975158\\
-95.215	-91.0730460968063\\
-86.67	-82.8997626971611\\
-89.111	-85.2345765975162\\
-119.629	-114.425011096097\\
-123.291	-117.927710196097\\
-167.236	-159.961055895032\\
-134.277	-128.435807496097\\
-169.678	-162.296826294322\\
-122.07	-116.759824996452\\
-108.643	-103.916913796096\\
-125.732	-120.262524096452\\
-89.111	-85.2345765975162\\
-80.566	-77.0612931978711\\
-97.656	-93.4078599971613\\
-98.877	-94.5757451968063\\
-68.359	-65.3853106982259\\
-73.242	-70.055894997871\\
-54.932	-52.5423994978707\\
-61.035	-58.3799124982258\\
-64.697	-61.8826115982258\\
-45.166	-43.2012308985806\\
-58.594	-56.0450985978708\\
-72.021	-68.8880097982259\\
-117.188	-112.090197195742\\
-122.07	-116.759824996452\\
-136.719	-130.771577895387\\
-76.904	-73.558594097871\\
-109.863	-105.083842496806\\
-173.34	-165.799525394322\\
-131.836	-126.100993595742\\
-157.471	-150.620843794677\\
-85.449	-81.7318774975161\\
-59.814	-57.2120272985808\\
-85.449	-81.7318774975161\\
-98.877	-94.5757451968063\\
-161.133	-154.123542894677\\
-202.637	-193.822074693258\\
-198.975	-190.319375593258\\
-145.264	-138.944861295032\\
-150.146	-143.614489095742\\
-135.498	-129.603692695742\\
-131.836	-126.100993595742\\
-126.953	-121.430409296097\\
-85.449	-81.7318774975161\\
-76.904	-73.558594097871\\
-69.58	-66.5531958978709\\
-62.256	-59.5477976978708\\
-70.801	-67.7210810975159\\
-107.422	-102.749028596451\\
-164.795	-157.626241994677\\
-125.732	-120.262524096452\\
-86.67	-82.8997626971611\\
-73.242	-70.055894997871\\
-70.801	-67.7210810975159\\
-42.725	-40.8664169982256\\
-31.738	-30.3573631992904\\
-39.063	-37.3637178982255\\
-72.021	-68.8880097982259\\
-57.373	-54.8772133982257\\
-59.814	-57.2120272985808\\
-67.139	-64.2183819975159\\
-63.477	-60.7156828975158\\
-43.945	-42.0333456989356\\
-29.297	-28.0225492989354\\
-25.635	-24.5198501989353\\
-43.945	-42.0333456989356\\
-96.436	-92.2409312964513\\
-140.381	-134.274276995387\\
-155.029	-148.285073395387\\
-101.318	-96.9105590971613\\
-69.58	-66.5531958978709\\
-50.049	-47.8718151982257\\
-34.18	-32.6931335985804\\
-83.008	-79.3970635971611\\
-81.787	-78.2291783975161\\
-119.629	-114.425011096097\\
-145.264	-138.944861295032\\
-230.713	-220.676738792548\\
-262.451	-251.034101991838\\
-211.182	-201.995358092903\\
-202.637	-193.822074693258\\
-136.719	-130.771577895387\\
-151.367	-144.782374295387\\
-159.912	-152.955657695032\\
-172.119	-164.631640194677\\
-129.395	-123.766179695387\\
-118.408	-113.257125896452\\
-115.967	-110.922311996097\\
-103.76	-99.2463294964514\\
-123.291	-117.927710196097\\
-87.891	-84.0676478968062\\
-153.809	-147.118144694677\\
-189.209	-180.978206993967\\
-133.057	-127.268878795387\\
-78.125	-74.726479297516\\
-85.449	-81.7318774975161\\
-146.484	-140.111789995742\\
-181.885	-173.972808793967\\
-229.492	-219.508853592903\\
-241.699	-231.184836092548\\
-240.479	-230.017907391838\\
-180.664	-172.804923594322\\
-163.574	-156.458356795032\\
-187.988	-179.810321794322\\
-222.168	-212.503455392903\\
-139.16	-133.106391795742\\
-81.787	-78.2291783975161\\
-122.07	-116.759824996452\\
-102.539	-98.0784442968063\\
-65.918	-63.0504967978709\\
-87.891	-84.0676478968062\\
-98.877	-94.5757451968063\\
-76.904	-73.558594097871\\
-58.594	-56.0450985978708\\
-80.566	-77.0612931978711\\
-65.918	-63.0504967978709\\
-91.553	-87.5703469968062\\
-133.057	-127.268878795387\\
-122.07	-116.759824996452\\
-79.346	-75.894364497161\\
-83.008	-79.3970635971611\\
-126.953	-121.430409296097\\
-209.961	-200.827472893258\\
-150.146	-143.614489095742\\
-92.773	-88.7372756975162\\
-69.58	-66.5531958978709\\
-83.008	-79.3970635971611\\
-74.463	-71.223780197516\\
-56.152	-53.7093281985807\\
-62.256	-59.5477976978708\\
-74.463	-71.223780197516\\
-96.436	-92.2409312964513\\
-107.422	-102.749028596451\\
-72.021	-68.8880097982259\\
-122.07	-116.759824996452\\
-144.043	-137.776976095387\\
-137.939	-131.938506596097\\
-106.201	-101.581143396806\\
-118.408	-113.257125896452\\
-158.691	-151.787772495387\\
-101.318	-96.9105590971613\\
-41.504	-39.6985317985805\\
-58.594	-56.0450985978708\\
-65.918	-63.0504967978709\\
-91.553	-87.5703469968062\\
-81.787	-78.2291783975161\\
-59.814	-57.2120272985808\\
-96.436	-92.2409312964513\\
-91.553	-87.5703469968062\\
-65.918	-63.0504967978709\\
-83.008	-79.3970635971611\\
-76.904	-73.558594097871\\
-67.139	-64.2183819975159\\
-79.346	-75.894364497161\\
-46.387	-44.3691160982256\\
-53.711	-51.3745142982257\\
-40.283	-38.5306465989355\\
-43.945	-42.0333456989356\\
-86.67	-82.8997626971611\\
-100.098	-95.7436303964513\\
-137.939	-131.938506596097\\
-177.002	-169.302224494322\\
-115.967	-110.922311996097\\
-68.359	-65.3853106982259\\
-48.828	-46.7039299985806\\
-73.242	-70.055894997871\\
-46.387	-44.3691160982256\\
-52.49	-50.2066290985807\\
-70.801	-67.7210810975159\\
-119.629	-114.425011096097\\
-107.422	-102.749028596451\\
-150.146	-143.614489095742\\
-102.539	-98.0784442968063\\
-91.553	-87.5703469968062\\
-47.607	-45.5360447989356\\
-31.738	-30.3573631992904\\
-34.18	-32.6931335985804\\
-41.504	-39.6985317985805\\
-75.684	-72.391665397161\\
-83.008	-79.3970635971611\\
-92.773	-88.7372756975162\\
-97.656	-93.4078599971613\\
-152.588	-145.950259495032\\
-130.615	-124.933108396097\\
-97.656	-93.4078599971613\\
-119.629	-114.425011096097\\
-129.395	-123.766179695387\\
-167.236	-159.961055895032\\
-133.057	-127.268878795387\\
-84.229	-80.5649487968061\\
-43.945	-42.0333456989356\\
-31.738	-30.3573631992904\\
-34.18	-32.6931335985804\\
-41.504	-39.6985317985805\\
-43.945	-42.0333456989356\\
-40.283	-38.5306465989355\\
-56.152	-53.7093281985807\\
-87.891	-84.0676478968062\\
-79.346	-75.894364497161\\
-81.787	-78.2291783975161\\
-96.436	-92.2409312964513\\
-58.594	-56.0450985978708\\
-29.297	-28.0225492989354\\
-64.697	-61.8826115982258\\
-97.656	-93.4078599971613\\
-96.436	-92.2409312964513\\
-109.863	-105.083842496806\\
-93.994	-89.9051608971612\\
-69.58	-66.5531958978709\\
-97.656	-93.4078599971613\\
-137.939	-131.938506596097\\
-139.16	-133.106391795742\\
-97.656	-93.4078599971613\\
-162.354	-155.291428094322\\
-128.174	-122.598294495742\\
-118.408	-113.257125896452\\
-137.939	-131.938506596097\\
-103.76	-99.2463294964514\\
-90.332	-86.4024617971612\\
-153.809	-147.118144694677\\
-214.844	-205.498057192903\\
-164.795	-157.626241994677\\
-92.773	-88.7372756975162\\
-91.553	-87.5703469968062\\
-113.525	-108.586541596806\\
-136.719	-130.771577895387\\
-100.098	-95.7436303964513\\
-104.98	-100.413258197161\\
-140.381	-134.274276995387\\
-153.809	-147.118144694677\\
-103.76	-99.2463294964514\\
-79.346	-75.894364497161\\
-104.98	-100.413258197161\\
-175.781	-168.134339294677\\
-159.912	-152.955657695032\\
-162.354	-155.291428094322\\
-95.215	-91.0730460968063\\
-84.229	-80.5649487968061\\
-83.008	-79.3970635971611\\
-52.49	-50.2066290985807\\
-34.18	-32.6931335985804\\
-28.076	-26.8546640992904\\
-23.193	-22.1840797996453\\
-59.814	-57.2120272985808\\
-86.67	-82.8997626971611\\
-104.98	-100.413258197161\\
-76.904	-73.558594097871\\
-87.891	-84.0676478968062\\
-97.656	-93.4078599971613\\
-59.814	-57.2120272985808\\
-63.477	-60.7156828975158\\
-61.035	-58.3799124982258\\
-45.166	-43.2012308985806\\
-57.373	-54.8772133982257\\
-97.656	-93.4078599971613\\
-124.512	-119.095595395742\\
-78.125	-74.726479297516\\
-56.152	-53.7093281985807\\
-46.387	-44.3691160982256\\
-59.814	-57.2120272985808\\
-41.504	-39.6985317985805\\
-91.553	-87.5703469968062\\
-158.691	-151.787772495387\\
-122.07	-116.759824996452\\
-167.236	-159.961055895032\\
-194.092	-185.648791293613\\
-197.754	-189.151490393613\\
-140.381	-134.274276995387\\
-106.201	-101.581143396806\\
-104.98	-100.413258197161\\
-120.85	-115.592896295742\\
-150.146	-143.614489095742\\
-147.705	-141.279675195387\\
-200.195	-191.486304293968\\
-220.947	-211.335570193258\\
-263.672	-252.201987191483\\
-178.223	-170.470109693967\\
-190.43	-182.146092193613\\
-212.402	-203.162286793613\\
-163.574	-156.458356795032\\
-189.209	-180.978206993967\\
-163.574	-156.458356795032\\
-91.553	-87.5703469968062\\
-58.594	-56.0450985978708\\
-62.256	-59.5477976978708\\
-95.215	-91.0730460968063\\
-75.684	-72.391665397161\\
-51.27	-49.0397003978707\\
-72.021	-68.8880097982259\\
-52.49	-50.2066290985807\\
-40.283	-38.5306465989355\\
-48.828	-46.7039299985806\\
-56.152	-53.7093281985807\\
-40.283	-38.5306465989355\\
-76.904	-73.558594097871\\
-109.863	-105.083842496806\\
-78.125	-74.726479297516\\
-134.277	-128.435807496097\\
-195.313	-186.816676493258\\
-178.223	-170.470109693967\\
-222.168	-212.503455392903\\
-150.146	-143.614489095742\\
-161.133	-154.123542894677\\
-111.084	-106.251727696451\\
-67.139	-64.2183819975159\\
-85.449	-81.7318774975161\\
-67.139	-64.2183819975159\\
-48.828	-46.7039299985806\\
-72.021	-68.8880097982259\\
-41.504	-39.6985317985805\\
-25.635	-24.5198501989353\\
-36.621	-35.0279474989355\\
-79.346	-75.894364497161\\
-104.98	-100.413258197161\\
-128.174	-122.598294495742\\
-124.512	-119.095595395742\\
-117.188	-112.090197195742\\
-136.719	-130.771577895387\\
-111.084	-106.251727696451\\
-90.332	-86.4024617971612\\
-73.242	-70.055894997871\\
-115.967	-110.922311996097\\
-78.125	-74.726479297516\\
-67.139	-64.2183819975159\\
-100.098	-95.7436303964513\\
-137.939	-131.938506596097\\
-103.76	-99.2463294964514\\
-76.904	-73.558594097871\\
-64.697	-61.8826115982258\\
-74.463	-71.223780197516\\
-51.27	-49.0397003978707\\
-50.049	-47.8718151982257\\
-72.021	-68.8880097982259\\
-86.67	-82.8997626971611\\
-98.877	-94.5757451968063\\
-79.346	-75.894364497161\\
-101.318	-96.9105590971613\\
-92.773	-88.7372756975162\\
-52.49	-50.2066290985807\\
-54.932	-52.5423994978707\\
-112.305	-107.419612896096\\
-158.691	-151.787772495387\\
-153.809	-147.118144694677\\
-129.395	-123.766179695387\\
-122.07	-116.759824996452\\
-92.773	-88.7372756975162\\
-89.111	-85.2345765975162\\
-70.801	-67.7210810975159\\
-102.539	-98.0784442968063\\
-90.332	-86.4024617971612\\
-54.932	-52.5423994978707\\
-84.229	-80.5649487968061\\
-101.318	-96.9105590971613\\
-118.408	-113.257125896452\\
-129.395	-123.766179695387\\
-175.781	-168.134339294677\\
-216.064	-206.664985893613\\
-244.141	-233.520606491838\\
-153.809	-147.118144694677\\
-85.449	-81.7318774975161\\
-54.932	-52.5423994978707\\
-37.842	-36.1958326985805\\
-57.373	-54.8772133982257\\
-53.711	-51.3745142982257\\
-95.215	-91.0730460968063\\
-85.449	-81.7318774975161\\
-75.684	-72.391665397161\\
-102.539	-98.0784442968063\\
-118.408	-113.257125896452\\
-141.602	-135.442162195032\\
-148.926	-142.447560395032\\
-208.74	-199.659587693613\\
-181.885	-173.972808793967\\
-136.719	-130.771577895387\\
-90.332	-86.4024617971612\\
-92.773	-88.7372756975162\\
-95.215	-91.0730460968063\\
-123.291	-117.927710196097\\
-87.891	-84.0676478968062\\
-89.111	-85.2345765975162\\
-87.891	-84.0676478968062\\
-47.607	-45.5360447989356\\
-81.787	-78.2291783975161\\
-128.174	-122.598294495742\\
-90.332	-86.4024617971612\\
-96.436	-92.2409312964513\\
-170.898	-163.463754995032\\
-129.395	-123.766179695387\\
-122.07	-116.759824996452\\
-177.002	-169.302224494322\\
-166.016	-158.794127194322\\
-102.539	-98.0784442968063\\
-64.697	-61.8826115982258\\
-65.918	-63.0504967978709\\
-114.746	-109.754426796451\\
-190.43	-182.146092193613\\
-197.754	-189.151490393613\\
-206.299	-197.324773793258\\
-150.146	-143.614489095742\\
-97.656	-93.4078599971613\\
-83.008	-79.3970635971611\\
-58.594	-56.0450985978708\\
-56.152	-53.7093281985807\\
-47.607	-45.5360447989356\\
-68.359	-65.3853106982259\\
-81.787	-78.2291783975161\\
-46.387	-44.3691160982256\\
-70.801	-67.7210810975159\\
-103.76	-99.2463294964514\\
-114.746	-109.754426796451\\
-145.264	-138.944861295032\\
-185.547	-177.475507893967\\
-223.389	-213.671340592548\\
-161.133	-154.123542894677\\
-137.939	-131.938506596097\\
-144.043	-137.776976095387\\
-120.85	-115.592896295742\\
-84.229	-80.5649487968061\\
-103.76	-99.2463294964514\\
-167.236	-159.961055895032\\
-194.092	-185.648791293613\\
-115.967	-110.922311996097\\
-62.256	-59.5477976978708\\
-41.504	-39.6985317985805\\
-23.193	-22.1840797996453\\
-31.738	-30.3573631992904\\
-51.27	-49.0397003978707\\
-61.035	-58.3799124982258\\
-80.566	-77.0612931978711\\
-67.139	-64.2183819975159\\
-51.27	-49.0397003978707\\
-50.049	-47.8718151982257\\
-48.828	-46.7039299985806\\
-45.166	-43.2012308985806\\
-52.49	-50.2066290985807\\
-81.787	-78.2291783975161\\
-123.291	-117.927710196097\\
-133.057	-127.268878795387\\
-125.732	-120.262524096452\\
-152.588	-145.950259495032\\
-181.885	-173.972808793967\\
-183.105	-175.139737494677\\
-223.389	-213.671340592548\\
-201.416	-192.654189493613\\
-146.484	-140.111789995742\\
-101.318	-96.9105590971613\\
-96.436	-92.2409312964513\\
-163.574	-156.458356795032\\
-191.65	-183.313020894323\\
-131.836	-126.100993595742\\
-85.449	-81.7318774975161\\
-45.166	-43.2012308985806\\
-31.738	-30.3573631992904\\
-35.4	-33.8600622992905\\
-21.973	-21.0171510989353\\
-46.387	-44.3691160982256\\
-68.359	-65.3853106982259\\
-72.021	-68.8880097982259\\
-62.256	-59.5477976978708\\
-36.621	-35.0279474989355\\
-28.076	-26.8546640992904\\
-41.504	-39.6985317985805\\
-43.945	-42.0333456989356\\
-48.828	-46.7039299985806\\
-36.621	-35.0279474989355\\
-30.518	-29.1904344985804\\
-54.932	-52.5423994978707\\
-128.174	-122.598294495742\\
-153.809	-147.118144694677\\
-170.898	-163.463754995032\\
-124.512	-119.095595395742\\
-172.119	-164.631640194677\\
-241.699	-231.184836092548\\
-222.168	-212.503455392903\\
-224.609	-214.838269293258\\
-164.795	-157.626241994677\\
-163.574	-156.458356795032\\
-172.119	-164.631640194677\\
-113.525	-108.586541596806\\
-106.201	-101.581143396806\\
-65.918	-63.0504967978709\\
-59.814	-57.2120272985808\\
-117.188	-112.090197195742\\
-79.346	-75.894364497161\\
-81.787	-78.2291783975161\\
-96.436	-92.2409312964513\\
-86.67	-82.8997626971611\\
-87.891	-84.0676478968062\\
-93.994	-89.9051608971612\\
-107.422	-102.749028596451\\
-119.629	-114.425011096097\\
-103.76	-99.2463294964514\\
-74.463	-71.223780197516\\
-96.436	-92.2409312964513\\
-48.828	-46.7039299985806\\
-20.752	-19.8492658992903\\
-34.18	-32.6931335985804\\
-58.594	-56.0450985978708\\
-30.518	-29.1904344985804\\
-13.428	-12.8438676992902\\
-32.959	-31.5252483989354\\
-58.594	-56.0450985978708\\
-69.58	-66.5531958978709\\
-83.008	-79.3970635971611\\
-72.021	-68.8880097982259\\
-117.188	-112.090197195742\\
-104.98	-100.413258197161\\
-70.801	-67.7210810975159\\
-107.422	-102.749028596451\\
-59.814	-57.2120272985808\\
-46.387	-44.3691160982256\\
-32.959	-31.5252483989354\\
-21.973	-21.0171510989353\\
-41.504	-39.6985317985805\\
-34.18	-32.6931335985804\\
-53.711	-51.3745142982257\\
-91.553	-87.5703469968062\\
-87.891	-84.0676478968062\\
-59.814	-57.2120272985808\\
-68.359	-65.3853106982259\\
-52.49	-50.2066290985807\\
-74.463	-71.223780197516\\
-53.711	-51.3745142982257\\
-50.049	-47.8718151982257\\
-46.387	-44.3691160982256\\
-36.621	-35.0279474989355\\
-45.166	-43.2012308985806\\
-41.504	-39.6985317985805\\
-75.684	-72.391665397161\\
-73.242	-70.055894997871\\
-54.932	-52.5423994978707\\
-51.27	-49.0397003978707\\
-40.283	-38.5306465989355\\
-48.828	-46.7039299985806\\
-108.643	-103.916913796096\\
-144.043	-137.776976095387\\
-96.436	-92.2409312964513\\
-90.332	-86.4024617971612\\
-63.477	-60.7156828975158\\
-111.084	-106.251727696451\\
-101.318	-96.9105590971613\\
-67.139	-64.2183819975159\\
-83.008	-79.3970635971611\\
-64.697	-61.8826115982258\\
-89.111	-85.2345765975162\\
-52.49	-50.2066290985807\\
-54.932	-52.5423994978707\\
-65.918	-63.0504967978709\\
-84.229	-80.5649487968061\\
-62.256	-59.5477976978708\\
-65.918	-63.0504967978709\\
-68.359	-65.3853106982259\\
-112.305	-107.419612896096\\
-95.215	-91.0730460968063\\
-146.484	-140.111789995742\\
-190.43	-182.146092193613\\
-151.367	-144.782374295387\\
-95.215	-91.0730460968063\\
-72.021	-68.8880097982259\\
-113.525	-108.586541596806\\
-141.602	-135.442162195032\\
-108.643	-103.916913796096\\
-129.395	-123.766179695387\\
-81.787	-78.2291783975161\\
-42.725	-40.8664169982256\\
-98.877	-94.5757451968063\\
-89.111	-85.2345765975162\\
-79.346	-75.894364497161\\
-124.512	-119.095595395742\\
-118.408	-113.257125896452\\
-103.76	-99.2463294964514\\
-67.139	-64.2183819975159\\
-84.229	-80.5649487968061\\
-115.967	-110.922311996097\\
-95.215	-91.0730460968063\\
-97.656	-93.4078599971613\\
-89.111	-85.2345765975162\\
-62.256	-59.5477976978708\\
-75.684	-72.391665397161\\
-54.932	-52.5423994978707\\
-62.256	-59.5477976978708\\
-106.201	-101.581143396806\\
-123.291	-117.927710196097\\
-131.836	-126.100993595742\\
-117.188	-112.090197195742\\
-83.008	-79.3970635971611\\
-58.594	-56.0450985978708\\
-67.139	-64.2183819975159\\
-80.566	-77.0612931978711\\
-104.98	-100.413258197161\\
-130.615	-124.933108396097\\
-155.029	-148.285073395387\\
-114.746	-109.754426796451\\
-69.58	-66.5531958978709\\
-126.953	-121.430409296097\\
-178.223	-170.470109693967\\
-128.174	-122.598294495742\\
-125.732	-120.262524096452\\
-147.705	-141.279675195387\\
-111.084	-106.251727696451\\
-90.332	-86.4024617971612\\
-100.098	-95.7436303964513\\
-92.773	-88.7372756975162\\
-103.76	-99.2463294964514\\
-131.836	-126.100993595742\\
-101.318	-96.9105590971613\\
-112.305	-107.419612896096\\
-106.201	-101.581143396806\\
-108.643	-103.916913796096\\
-86.67	-82.8997626971611\\
-114.746	-109.754426796451\\
-81.787	-78.2291783975161\\
-86.67	-82.8997626971611\\
-96.436	-92.2409312964513\\
-93.994	-89.9051608971612\\
-42.725	-40.8664169982256\\
-26.855	-25.6867788996453\\
-18.311	-17.5144519989352\\
-39.063	-37.3637178982255\\
-96.436	-92.2409312964513\\
-125.732	-120.262524096452\\
-83.008	-79.3970635971611\\
-97.656	-93.4078599971613\\
-85.449	-81.7318774975161\\
-75.684	-72.391665397161\\
-46.387	-44.3691160982256\\
-65.918	-63.0504967978709\\
-64.697	-61.8826115982258\\
-76.904	-73.558594097871\\
-65.918	-63.0504967978709\\
-59.814	-57.2120272985808\\
-40.283	-38.5306465989355\\
-58.594	-56.0450985978708\\
-109.863	-105.083842496806\\
-78.125	-74.726479297516\\
-92.773	-88.7372756975162\\
-83.008	-79.3970635971611\\
-118.408	-113.257125896452\\
-109.863	-105.083842496806\\
-152.588	-145.950259495032\\
-111.084	-106.251727696451\\
-124.512	-119.095595395742\\
-63.477	-60.7156828975158\\
-113.525	-108.586541596806\\
-76.904	-73.558594097871\\
-96.436	-92.2409312964513\\
-102.539	-98.0784442968063\\
-129.395	-123.766179695387\\
-97.656	-93.4078599971613\\
-125.732	-120.262524096452\\
-158.691	-151.787772495387\\
-131.836	-126.100993595742\\
-114.746	-109.754426796451\\
-90.332	-86.4024617971612\\
-107.422	-102.749028596451\\
-89.111	-85.2345765975162\\
-76.904	-73.558594097871\\
-68.359	-65.3853106982259\\
-81.787	-78.2291783975161\\
-69.58	-66.5531958978709\\
-141.602	-135.442162195032\\
-181.885	-173.972808793967\\
-157.471	-150.620843794677\\
-150.146	-143.614489095742\\
-147.705	-141.279675195387\\
-81.787	-78.2291783975161\\
-72.021	-68.8880097982259\\
-54.932	-52.5423994978707\\
-50.049	-47.8718151982257\\
-54.932	-52.5423994978707\\
-93.994	-89.9051608971612\\
-118.408	-113.257125896452\\
-134.277	-128.435807496097\\
-114.746	-109.754426796451\\
-76.904	-73.558594097871\\
-140.381	-134.274276995387\\
-163.574	-156.458356795032\\
-183.105	-175.139737494677\\
-150.146	-143.614489095742\\
-236.816	-226.514251792903\\
-158.691	-151.787772495387\\
-93.994	-89.9051608971612\\
-68.359	-65.3853106982259\\
-56.152	-53.7093281985807\\
-101.318	-96.9105590971613\\
-130.615	-124.933108396097\\
-172.119	-164.631640194677\\
-129.395	-123.766179695387\\
-102.539	-98.0784442968063\\
-53.711	-51.3745142982257\\
-64.697	-61.8826115982258\\
-70.801	-67.7210810975159\\
-43.945	-42.0333456989356\\
-61.035	-58.3799124982258\\
-45.166	-43.2012308985806\\
-29.297	-28.0225492989354\\
-65.918	-63.0504967978709\\
-75.684	-72.391665397161\\
-96.436	-92.2409312964513\\
-89.111	-85.2345765975162\\
-93.994	-89.9051608971612\\
-120.85	-115.592896295742\\
-79.346	-75.894364497161\\
-119.629	-114.425011096097\\
-136.719	-130.771577895387\\
-103.76	-99.2463294964514\\
-108.643	-103.916913796096\\
-93.994	-89.9051608971612\\
-108.643	-103.916913796096\\
-151.367	-144.782374295387\\
-117.188	-112.090197195742\\
-93.994	-89.9051608971612\\
-102.539	-98.0784442968063\\
-93.994	-89.9051608971612\\
-62.256	-59.5477976978708\\
-100.098	-95.7436303964513\\
-145.264	-138.944861295032\\
-139.16	-133.106391795742\\
-151.367	-144.782374295387\\
-227.051	-217.174039692548\\
-139.16	-133.106391795742\\
-122.07	-116.759824996452\\
-91.553	-87.5703469968062\\
-119.629	-114.425011096097\\
-172.119	-164.631640194677\\
-114.746	-109.754426796451\\
-102.539	-98.0784442968063\\
-142.822	-136.609090895742\\
-91.553	-87.5703469968062\\
-53.711	-51.3745142982257\\
-70.801	-67.7210810975159\\
-52.49	-50.2066290985807\\
-42.725	-40.8664169982256\\
-46.387	-44.3691160982256\\
-40.283	-38.5306465989355\\
-39.063	-37.3637178982255\\
-53.711	-51.3745142982257\\
-76.904	-73.558594097871\\
-90.332	-86.4024617971612\\
-119.629	-114.425011096097\\
-112.305	-107.419612896096\\
-135.498	-129.603692695742\\
-157.471	-150.620843794677\\
-141.602	-135.442162195032\\
-96.436	-92.2409312964513\\
-106.201	-101.581143396806\\
-95.215	-91.0730460968063\\
-103.76	-99.2463294964514\\
-147.705	-141.279675195387\\
-109.863	-105.083842496806\\
-122.07	-116.759824996452\\
-106.201	-101.581143396806\\
-102.539	-98.0784442968063\\
-158.691	-151.787772495387\\
-130.615	-124.933108396097\\
-129.395	-123.766179695387\\
-122.07	-116.759824996452\\
-73.242	-70.055894997871\\
-47.607	-45.5360447989356\\
-37.842	-36.1958326985805\\
-69.58	-66.5531958978709\\
-62.256	-59.5477976978708\\
-86.67	-82.8997626971611\\
-65.918	-63.0504967978709\\
-93.994	-89.9051608971612\\
-152.588	-145.950259495032\\
-189.209	-180.978206993967\\
-150.146	-143.614489095742\\
-153.809	-147.118144694677\\
-137.939	-131.938506596097\\
-96.436	-92.2409312964513\\
-103.76	-99.2463294964514\\
-117.188	-112.090197195742\\
-100.098	-95.7436303964513\\
-114.746	-109.754426796451\\
-145.264	-138.944861295032\\
-202.637	-193.822074693258\\
-181.885	-173.972808793967\\
-170.898	-163.463754995032\\
-191.65	-183.313020894323\\
-131.836	-126.100993595742\\
-144.043	-137.776976095387\\
-108.643	-103.916913796096\\
-112.305	-107.419612896096\\
-147.705	-141.279675195387\\
-172.119	-164.631640194677\\
-79.346	-75.894364497161\\
-89.111	-85.2345765975162\\
-67.139	-64.2183819975159\\
-98.877	-94.5757451968063\\
-100.098	-95.7436303964513\\
-61.035	-58.3799124982258\\
-81.787	-78.2291783975161\\
-139.16	-133.106391795742\\
-234.375	-224.179437892548\\
-227.051	-217.174039692548\\
-206.299	-197.324773793258\\
-166.016	-158.794127194322\\
-194.092	-185.648791293613\\
-233.154	-223.011552692903\\
-170.898	-163.463754995032\\
-177.002	-169.302224494322\\
-184.326	-176.307622694322\\
-125.732	-120.262524096452\\
-106.201	-101.581143396806\\
-137.939	-131.938506596097\\
-91.553	-87.5703469968062\\
-67.139	-64.2183819975159\\
-97.656	-93.4078599971613\\
-73.242	-70.055894997871\\
-86.67	-82.8997626971611\\
-81.787	-78.2291783975161\\
-100.098	-95.7436303964513\\
-133.057	-127.268878795387\\
-102.539	-98.0784442968063\\
-75.684	-72.391665397161\\
-89.111	-85.2345765975162\\
-104.98	-100.413258197161\\
-124.512	-119.095595395742\\
-107.422	-102.749028596451\\
-86.67	-82.8997626971611\\
-136.719	-130.771577895387\\
-129.395	-123.766179695387\\
-90.332	-86.4024617971612\\
-148.926	-142.447560395032\\
-134.277	-128.435807496097\\
-97.656	-93.4078599971613\\
-64.697	-61.8826115982258\\
-46.387	-44.3691160982256\\
-32.959	-31.5252483989354\\
-75.684	-72.391665397161\\
-72.021	-68.8880097982259\\
-52.49	-50.2066290985807\\
-36.621	-35.0279474989355\\
-47.607	-45.5360447989356\\
-81.787	-78.2291783975161\\
-93.994	-89.9051608971612\\
-124.512	-119.095595395742\\
-142.822	-136.609090895742\\
-137.939	-131.938506596097\\
-145.264	-138.944861295032\\
-87.891	-84.0676478968062\\
-36.621	-35.0279474989355\\
-47.607	-45.5360447989356\\
-46.387	-44.3691160982256\\
-64.697	-61.8826115982258\\
-100.098	-95.7436303964513\\
-111.084	-106.251727696451\\
-168.457	-161.128941094677\\
-106.201	-101.581143396806\\
-104.98	-100.413258197161\\
-155.029	-148.285073395387\\
-106.201	-101.581143396806\\
-74.463	-71.223780197516\\
-53.711	-51.3745142982257\\
-76.904	-73.558594097871\\
-100.098	-95.7436303964513\\
-151.367	-144.782374295387\\
-98.877	-94.5757451968063\\
-85.449	-81.7318774975161\\
-80.566	-77.0612931978711\\
-101.318	-96.9105590971613\\
-133.057	-127.268878795387\\
-207.52	-198.492658992903\\
-184.326	-176.307622694322\\
-180.664	-172.804923594322\\
-112.305	-107.419612896096\\
-75.684	-72.391665397161\\
-87.891	-84.0676478968062\\
-36.621	-35.0279474989355\\
-67.139	-64.2183819975159\\
-52.49	-50.2066290985807\\
-65.918	-63.0504967978709\\
-117.188	-112.090197195742\\
-153.809	-147.118144694677\\
-177.002	-169.302224494322\\
-234.375	-224.179437892548\\
-196.533	-187.983605193968\\
-106.201	-101.581143396806\\
-98.877	-94.5757451968063\\
-79.346	-75.894364497161\\
-108.643	-103.916913796096\\
-141.602	-135.442162195032\\
-191.65	-183.313020894323\\
-140.381	-134.274276995387\\
-97.656	-93.4078599971613\\
-108.643	-103.916913796096\\
-144.043	-137.776976095387\\
-102.539	-98.0784442968063\\
-73.242	-70.055894997871\\
-57.373	-54.8772133982257\\
-40.283	-38.5306465989355\\
-36.621	-35.0279474989355\\
-29.297	-28.0225492989354\\
-54.932	-52.5423994978707\\
-80.566	-77.0612931978711\\
-86.67	-82.8997626971611\\
-64.697	-61.8826115982258\\
-58.594	-56.0450985978708\\
-26.855	-25.6867788996453\\
-15.869	-15.1786815996452\\
-24.414	-23.3519649992903\\
-46.387	-44.3691160982256\\
-70.801	-67.7210810975159\\
-87.891	-84.0676478968062\\
-102.539	-98.0784442968063\\
-119.629	-114.425011096097\\
-153.809	-147.118144694677\\
-212.402	-203.162286793613\\
-213.623	-204.330171993258\\
-230.713	-220.676738792548\\
-201.416	-192.654189493613\\
-111.084	-106.251727696451\\
-102.539	-98.0784442968063\\
-101.318	-96.9105590971613\\
-135.498	-129.603692695742\\
-113.525	-108.586541596806\\
-80.566	-77.0612931978711\\
-61.035	-58.3799124982258\\
-50.049	-47.8718151982257\\
-93.994	-89.9051608971612\\
-90.332	-86.4024617971612\\
-47.607	-45.5360447989356\\
-37.842	-36.1958326985805\\
-58.594	-56.0450985978708\\
-78.125	-74.726479297516\\
-59.814	-57.2120272985808\\
-86.67	-82.8997626971611\\
-67.139	-64.2183819975159\\
-98.877	-94.5757451968063\\
-141.602	-135.442162195032\\
-112.305	-107.419612896096\\
-153.809	-147.118144694677\\
-206.299	-197.324773793258\\
-224.609	-214.838269293258\\
-175.781	-168.134339294677\\
-111.084	-106.251727696451\\
-83.008	-79.3970635971611\\
-48.828	-46.7039299985806\\
-81.787	-78.2291783975161\\
-57.373	-54.8772133982257\\
-109.863	-105.083842496806\\
-92.773	-88.7372756975162\\
-74.463	-71.223780197516\\
-96.436	-92.2409312964513\\
-52.49	-50.2066290985807\\
-84.229	-80.5649487968061\\
-58.594	-56.0450985978708\\
-37.842	-36.1958326985805\\
-45.166	-43.2012308985806\\
-32.959	-31.5252483989354\\
-40.283	-38.5306465989355\\
-35.4	-33.8600622992905\\
-26.855	-25.6867788996453\\
-36.621	-35.0279474989355\\
-48.828	-46.7039299985806\\
-50.049	-47.8718151982257\\
-48.828	-46.7039299985806\\
-79.346	-75.894364497161\\
-103.76	-99.2463294964514\\
-84.229	-80.5649487968061\\
-78.125	-74.726479297516\\
-128.174	-122.598294495742\\
-153.809	-147.118144694677\\
-119.629	-114.425011096097\\
-125.732	-120.262524096452\\
-56.152	-53.7093281985807\\
-36.621	-35.0279474989355\\
-73.242	-70.055894997871\\
-106.201	-101.581143396806\\
-115.967	-110.922311996097\\
-162.354	-155.291428094322\\
-146.484	-140.111789995742\\
-102.539	-98.0784442968063\\
-73.242	-70.055894997871\\
-79.346	-75.894364497161\\
-84.229	-80.5649487968061\\
-64.697	-61.8826115982258\\
-39.063	-37.3637178982255\\
-52.49	-50.2066290985807\\
-39.063	-37.3637178982255\\
-32.959	-31.5252483989354\\
-24.414	-23.3519649992903\\
-53.711	-51.3745142982257\\
-74.463	-71.223780197516\\
-79.346	-75.894364497161\\
-56.152	-53.7093281985807\\
-100.098	-95.7436303964513\\
-136.719	-130.771577895387\\
-86.67	-82.8997626971611\\
-120.85	-115.592896295742\\
-130.615	-124.933108396097\\
-98.877	-94.5757451968063\\
-56.152	-53.7093281985807\\
-72.021	-68.8880097982259\\
-87.891	-84.0676478968062\\
-47.607	-45.5360447989356\\
-50.049	-47.8718151982257\\
-41.504	-39.6985317985805\\
-24.414	-23.3519649992903\\
-26.855	-25.6867788996453\\
-46.387	-44.3691160982256\\
-64.697	-61.8826115982258\\
-54.932	-52.5423994978707\\
-72.021	-68.8880097982259\\
-79.346	-75.894364497161\\
-57.373	-54.8772133982257\\
-31.738	-30.3573631992904\\
-25.635	-24.5198501989353\\
-34.18	-32.6931335985804\\
-39.063	-37.3637178982255\\
-48.828	-46.7039299985806\\
-104.98	-100.413258197161\\
-101.318	-96.9105590971613\\
-89.111	-85.2345765975162\\
-85.449	-81.7318774975161\\
-117.188	-112.090197195742\\
-150.146	-143.614489095742\\
-159.912	-152.955657695032\\
-166.016	-158.794127194322\\
-120.85	-115.592896295742\\
-126.953	-121.430409296097\\
-131.836	-126.100993595742\\
-133.057	-127.268878795387\\
-151.367	-144.782374295387\\
-200.195	-191.486304293968\\
-169.678	-162.296826294322\\
-156.25	-149.452958595032\\
-128.174	-122.598294495742\\
-119.629	-114.425011096097\\
-114.746	-109.754426796451\\
-136.719	-130.771577895387\\
-83.008	-79.3970635971611\\
-98.877	-94.5757451968063\\
-119.629	-114.425011096097\\
-140.381	-134.274276995387\\
-107.422	-102.749028596451\\
-124.512	-119.095595395742\\
-98.877	-94.5757451968063\\
-90.332	-86.4024617971612\\
-54.932	-52.5423994978707\\
-85.449	-81.7318774975161\\
-137.939	-131.938506596097\\
-109.863	-105.083842496806\\
-64.697	-61.8826115982258\\
-80.566	-77.0612931978711\\
-41.504	-39.6985317985805\\
-40.283	-38.5306465989355\\
-56.152	-53.7093281985807\\
-32.959	-31.5252483989354\\
-36.621	-35.0279474989355\\
-43.945	-42.0333456989356\\
-26.855	-25.6867788996453\\
-50.049	-47.8718151982257\\
-40.283	-38.5306465989355\\
-23.193	-22.1840797996453\\
-43.945	-42.0333456989356\\
-51.27	-49.0397003978707\\
-63.477	-60.7156828975158\\
-62.256	-59.5477976978708\\
-63.477	-60.7156828975158\\
-84.229	-80.5649487968061\\
-76.904	-73.558594097871\\
-86.67	-82.8997626971611\\
-146.484	-140.111789995742\\
-101.318	-96.9105590971613\\
-92.773	-88.7372756975162\\
-100.098	-95.7436303964513\\
-69.58	-66.5531958978709\\
-40.283	-38.5306465989355\\
-79.346	-75.894364497161\\
-61.035	-58.3799124982258\\
-37.842	-36.1958326985805\\
-67.139	-64.2183819975159\\
-62.256	-59.5477976978708\\
-80.566	-77.0612931978711\\
-48.828	-46.7039299985806\\
-29.297	-28.0225492989354\\
-23.193	-22.1840797996453\\
-29.297	-28.0225492989354\\
-56.152	-53.7093281985807\\
-54.932	-52.5423994978707\\
-65.918	-63.0504967978709\\
-87.891	-84.0676478968062\\
-117.188	-112.090197195742\\
-84.229	-80.5649487968061\\
-124.512	-119.095595395742\\
-141.602	-135.442162195032\\
-124.512	-119.095595395742\\
-117.188	-112.090197195742\\
-195.313	-186.816676493258\\
-135.498	-129.603692695742\\
-173.34	-165.799525394322\\
-141.602	-135.442162195032\\
-164.795	-157.626241994677\\
-184.326	-176.307622694322\\
-101.318	-96.9105590971613\\
-63.477	-60.7156828975158\\
-51.27	-49.0397003978707\\
-84.229	-80.5649487968061\\
-104.98	-100.413258197161\\
-108.643	-103.916913796096\\
-74.463	-71.223780197516\\
-40.283	-38.5306465989355\\
-52.49	-50.2066290985807\\
-104.98	-100.413258197161\\
-145.264	-138.944861295032\\
-142.822	-136.609090895742\\
-84.229	-80.5649487968061\\
-64.697	-61.8826115982258\\
-89.111	-85.2345765975162\\
-141.602	-135.442162195032\\
-161.133	-154.123542894677\\
-164.795	-157.626241994677\\
-115.967	-110.922311996097\\
-166.016	-158.794127194322\\
-178.223	-170.470109693967\\
-170.898	-163.463754995032\\
-240.479	-230.017907391838\\
-203.857	-194.989003393968\\
-169.678	-162.296826294322\\
-196.533	-187.983605193968\\
-164.795	-157.626241994677\\
-91.553	-87.5703469968062\\
-84.229	-80.5649487968061\\
-104.98	-100.413258197161\\
-125.732	-120.262524096452\\
-128.174	-122.598294495742\\
-96.436	-92.2409312964513\\
-124.512	-119.095595395742\\
-106.201	-101.581143396806\\
-78.125	-74.726479297516\\
-106.201	-101.581143396806\\
-95.215	-91.0730460968063\\
-150.146	-143.614489095742\\
-151.367	-144.782374295387\\
-108.643	-103.916913796096\\
-79.346	-75.894364497161\\
-45.166	-43.2012308985806\\
-78.125	-74.726479297516\\
-53.711	-51.3745142982257\\
-52.49	-50.2066290985807\\
-42.725	-40.8664169982256\\
-76.904	-73.558594097871\\
-98.877	-94.5757451968063\\
-112.305	-107.419612896096\\
-109.863	-105.083842496806\\
-64.697	-61.8826115982258\\
-48.828	-46.7039299985806\\
-35.4	-33.8600622992905\\
-43.945	-42.0333456989356\\
-70.801	-67.7210810975159\\
-75.684	-72.391665397161\\
-74.463	-71.223780197516\\
-120.85	-115.592896295742\\
-140.381	-134.274276995387\\
-162.354	-155.291428094322\\
-159.912	-152.955657695032\\
-187.988	-179.810321794322\\
-109.863	-105.083842496806\\
-91.553	-87.5703469968062\\
-72.021	-68.8880097982259\\
-78.125	-74.726479297516\\
-104.98	-100.413258197161\\
-86.67	-82.8997626971611\\
-57.373	-54.8772133982257\\
-50.049	-47.8718151982257\\
-97.656	-93.4078599971613\\
-117.188	-112.090197195742\\
-107.422	-102.749028596451\\
-168.457	-161.128941094677\\
-184.326	-176.307622694322\\
-130.615	-124.933108396097\\
-91.553	-87.5703469968062\\
-62.256	-59.5477976978708\\
-57.373	-54.8772133982257\\
-107.422	-102.749028596451\\
-85.449	-81.7318774975161\\
-103.76	-99.2463294964514\\
-113.525	-108.586541596806\\
-129.395	-123.766179695387\\
-122.07	-116.759824996452\\
-135.498	-129.603692695742\\
-91.553	-87.5703469968062\\
-109.863	-105.083842496806\\
-75.684	-72.391665397161\\
-68.359	-65.3853106982259\\
-107.422	-102.749028596451\\
-147.705	-141.279675195387\\
-162.354	-155.291428094322\\
-91.553	-87.5703469968062\\
-190.43	-182.146092193613\\
-170.898	-163.463754995032\\
-113.525	-108.586541596806\\
-65.918	-63.0504967978709\\
-96.436	-92.2409312964513\\
-83.008	-79.3970635971611\\
-80.566	-77.0612931978711\\
-98.877	-94.5757451968063\\
-64.697	-61.8826115982258\\
-93.994	-89.9051608971612\\
-170.898	-163.463754995032\\
-179.443	-171.637038394677\\
-117.188	-112.090197195742\\
-134.277	-128.435807496097\\
-146.484	-140.111789995742\\
-148.926	-142.447560395032\\
-106.201	-101.581143396806\\
-69.58	-66.5531958978709\\
-106.201	-101.581143396806\\
-109.863	-105.083842496806\\
-175.781	-168.134339294677\\
-195.313	-186.816676493258\\
-263.672	-252.201987191483\\
-190.43	-182.146092193613\\
-155.029	-148.285073395387\\
-118.408	-113.257125896452\\
-130.615	-124.933108396097\\
-97.656	-93.4078599971613\\
-67.139	-64.2183819975159\\
-68.359	-65.3853106982259\\
-48.828	-46.7039299985806\\
-35.4	-33.8600622992905\\
-59.814	-57.2120272985808\\
-81.787	-78.2291783975161\\
-54.932	-52.5423994978707\\
-39.063	-37.3637178982255\\
-37.842	-36.1958326985805\\
-90.332	-86.4024617971612\\
-128.174	-122.598294495742\\
-59.814	-57.2120272985808\\
-74.463	-71.223780197516\\
-79.346	-75.894364497161\\
-73.242	-70.055894997871\\
-91.553	-87.5703469968062\\
-79.346	-75.894364497161\\
-54.932	-52.5423994978707\\
-41.504	-39.6985317985805\\
-34.18	-32.6931335985804\\
-43.945	-42.0333456989356\\
-86.67	-82.8997626971611\\
-103.76	-99.2463294964514\\
-72.021	-68.8880097982259\\
-48.828	-46.7039299985806\\
-29.297	-28.0225492989354\\
-42.725	-40.8664169982256\\
-80.566	-77.0612931978711\\
-101.318	-96.9105590971613\\
-107.422	-102.749028596451\\
-76.904	-73.558594097871\\
-70.801	-67.7210810975159\\
-78.125	-74.726479297516\\
-108.643	-103.916913796096\\
-151.367	-144.782374295387\\
-175.781	-168.134339294677\\
-131.836	-126.100993595742\\
-83.008	-79.3970635971611\\
-86.67	-82.8997626971611\\
-81.787	-78.2291783975161\\
-84.229	-80.5649487968061\\
-54.932	-52.5423994978707\\
-69.58	-66.5531958978709\\
-76.904	-73.558594097871\\
-45.166	-43.2012308985806\\
-29.297	-28.0225492989354\\
-47.607	-45.5360447989356\\
-70.801	-67.7210810975159\\
-109.863	-105.083842496806\\
-118.408	-113.257125896452\\
-148.926	-142.447560395032\\
-130.615	-124.933108396097\\
-146.484	-140.111789995742\\
-195.313	-186.816676493258\\
-260.01	-248.699288091483\\
-212.402	-203.162286793613\\
-142.822	-136.609090895742\\
-155.029	-148.285073395387\\
-180.664	-172.804923594322\\
-236.816	-226.514251792903\\
-155.029	-148.285073395387\\
-76.904	-73.558594097871\\
-50.049	-47.8718151982257\\
-52.49	-50.2066290985807\\
-37.842	-36.1958326985805\\
-23.193	-22.1840797996453\\
-30.518	-29.1904344985804\\
-48.828	-46.7039299985806\\
-68.359	-65.3853106982259\\
-70.801	-67.7210810975159\\
-54.932	-52.5423994978707\\
-48.828	-46.7039299985806\\
-67.139	-64.2183819975159\\
-81.787	-78.2291783975161\\
-85.449	-81.7318774975161\\
-79.346	-75.894364497161\\
-57.373	-54.8772133982257\\
-37.842	-36.1958326985805\\
-57.373	-54.8772133982257\\
-73.242	-70.055894997871\\
-53.711	-51.3745142982257\\
-35.4	-33.8600622992905\\
-58.594	-56.0450985978708\\
-39.063	-37.3637178982255\\
-50.049	-47.8718151982257\\
-63.477	-60.7156828975158\\
-92.773	-88.7372756975162\\
-100.098	-95.7436303964513\\
-68.359	-65.3853106982259\\
-106.201	-101.581143396806\\
-102.539	-98.0784442968063\\
-83.008	-79.3970635971611\\
-70.801	-67.7210810975159\\
-119.629	-114.425011096097\\
-146.484	-140.111789995742\\
-144.043	-137.776976095387\\
-158.691	-151.787772495387\\
-120.85	-115.592896295742\\
-111.084	-106.251727696451\\
-102.539	-98.0784442968063\\
-139.16	-133.106391795742\\
-108.643	-103.916913796096\\
-147.705	-141.279675195387\\
-141.602	-135.442162195032\\
-146.484	-140.111789995742\\
-249.023	-238.190234292548\\
-190.43	-182.146092193613\\
-102.539	-98.0784442968063\\
-69.58	-66.5531958978709\\
-73.242	-70.055894997871\\
-91.553	-87.5703469968062\\
-119.629	-114.425011096097\\
-139.16	-133.106391795742\\
-153.809	-147.118144694677\\
-157.471	-150.620843794677\\
-96.436	-92.2409312964513\\
-86.67	-82.8997626971611\\
-101.318	-96.9105590971613\\
-102.539	-98.0784442968063\\
-61.035	-58.3799124982258\\
-43.945	-42.0333456989356\\
-79.346	-75.894364497161\\
-80.566	-77.0612931978711\\
-109.863	-105.083842496806\\
-139.16	-133.106391795742\\
-123.291	-117.927710196097\\
-87.891	-84.0676478968062\\
-70.801	-67.7210810975159\\
-107.422	-102.749028596451\\
-139.16	-133.106391795742\\
-185.547	-177.475507893967\\
-198.975	-190.319375593258\\
-231.934	-221.844623992193\\
-184.326	-176.307622694322\\
-168.457	-161.128941094677\\
-89.111	-85.2345765975162\\
-67.139	-64.2183819975159\\
-126.953	-121.430409296097\\
-163.574	-156.458356795032\\
-93.994	-89.9051608971612\\
-151.367	-144.782374295387\\
-203.857	-194.989003393968\\
-246.582	-235.855420392193\\
-153.809	-147.118144694677\\
-86.67	-82.8997626971611\\
-50.049	-47.8718151982257\\
-75.684	-72.391665397161\\
-69.58	-66.5531958978709\\
-75.684	-72.391665397161\\
-42.725	-40.8664169982256\\
-61.035	-58.3799124982258\\
-45.166	-43.2012308985806\\
-58.594	-56.0450985978708\\
-48.828	-46.7039299985806\\
-62.256	-59.5477976978708\\
-101.318	-96.9105590971613\\
-91.553	-87.5703469968062\\
-129.395	-123.766179695387\\
-156.25	-149.452958595032\\
-155.029	-148.285073395387\\
-85.449	-81.7318774975161\\
-86.67	-82.8997626971611\\
-106.201	-101.581143396806\\
-70.801	-67.7210810975159\\
-63.477	-60.7156828975158\\
-45.166	-43.2012308985806\\
-73.242	-70.055894997871\\
-69.58	-66.5531958978709\\
-39.063	-37.3637178982255\\
-21.973	-21.0171510989353\\
-25.635	-24.5198501989353\\
-46.387	-44.3691160982256\\
-72.021	-68.8880097982259\\
-75.684	-72.391665397161\\
-45.166	-43.2012308985806\\
-57.373	-54.8772133982257\\
-84.229	-80.5649487968061\\
-109.863	-105.083842496806\\
-76.904	-73.558594097871\\
-72.021	-68.8880097982259\\
-84.229	-80.5649487968061\\
-101.318	-96.9105590971613\\
-92.773	-88.7372756975162\\
-109.863	-105.083842496806\\
-103.76	-99.2463294964514\\
-70.801	-67.7210810975159\\
-56.152	-53.7093281985807\\
-72.021	-68.8880097982259\\
-59.814	-57.2120272985808\\
-75.684	-72.391665397161\\
-102.539	-98.0784442968063\\
-157.471	-150.620843794677\\
-109.863	-105.083842496806\\
-107.422	-102.749028596451\\
-166.016	-158.794127194322\\
-125.732	-120.262524096452\\
-75.684	-72.391665397161\\
-54.932	-52.5423994978707\\
-43.945	-42.0333456989356\\
-46.387	-44.3691160982256\\
-39.063	-37.3637178982255\\
-40.283	-38.5306465989355\\
-84.229	-80.5649487968061\\
-26.855	-25.6867788996453\\
-37.842	-36.1958326985805\\
-53.711	-51.3745142982257\\
-72.021	-68.8880097982259\\
-56.152	-53.7093281985807\\
-40.283	-38.5306465989355\\
-24.414	-23.3519649992903\\
-42.725	-40.8664169982256\\
-84.229	-80.5649487968061\\
-111.084	-106.251727696451\\
-87.891	-84.0676478968062\\
-73.242	-70.055894997871\\
-83.008	-79.3970635971611\\
-84.229	-80.5649487968061\\
-96.436	-92.2409312964513\\
-117.188	-112.090197195742\\
-139.16	-133.106391795742\\
-133.057	-127.268878795387\\
-114.746	-109.754426796451\\
-74.463	-71.223780197516\\
-58.594	-56.0450985978708\\
-61.035	-58.3799124982258\\
-91.553	-87.5703469968062\\
-57.373	-54.8772133982257\\
-50.049	-47.8718151982257\\
-92.773	-88.7372756975162\\
-113.525	-108.586541596806\\
-108.643	-103.916913796096\\
-124.512	-119.095595395742\\
-164.795	-157.626241994677\\
-85.449	-81.7318774975161\\
-145.264	-138.944861295032\\
-152.588	-145.950259495032\\
-140.381	-134.274276995387\\
-152.588	-145.950259495032\\
-148.926	-142.447560395032\\
-253.906	-242.860818592193\\
-235.596	-225.347323092193\\
-158.691	-151.787772495387\\
-189.209	-180.978206993967\\
-233.154	-223.011552692903\\
-223.389	-213.671340592548\\
-253.906	-242.860818592193\\
-227.051	-217.174039692548\\
-311.279	-297.738031990419\\
-234.375	-224.179437892548\\
-159.912	-152.955657695032\\
-98.877	-94.5757451968063\\
-100.098	-95.7436303964513\\
-79.346	-75.894364497161\\
-64.697	-61.8826115982258\\
-100.098	-95.7436303964513\\
-142.822	-136.609090895742\\
-87.891	-84.0676478968062\\
-76.904	-73.558594097871\\
-73.242	-70.055894997871\\
-48.828	-46.7039299985806\\
-63.477	-60.7156828975158\\
-34.18	-32.6931335985804\\
-40.283	-38.5306465989355\\
-35.4	-33.8600622992905\\
-48.828	-46.7039299985806\\
-37.842	-36.1958326985805\\
-31.738	-30.3573631992904\\
-17.09	-16.3465667992902\\
-18.311	-17.5144519989352\\
-40.283	-38.5306465989355\\
-28.076	-26.8546640992904\\
-29.297	-28.0225492989354\\
-65.918	-63.0504967978709\\
-93.994	-89.9051608971612\\
-68.359	-65.3853106982259\\
-86.67	-82.8997626971611\\
-135.498	-129.603692695742\\
-70.801	-67.7210810975159\\
-37.842	-36.1958326985805\\
-29.297	-28.0225492989354\\
-24.414	-23.3519649992903\\
-40.283	-38.5306465989355\\
-67.139	-64.2183819975159\\
-100.098	-95.7436303964513\\
-62.256	-59.5477976978708\\
-100.098	-95.7436303964513\\
-144.043	-137.776976095387\\
-145.264	-138.944861295032\\
-103.76	-99.2463294964514\\
-62.256	-59.5477976978708\\
-83.008	-79.3970635971611\\
-113.525	-108.586541596806\\
-148.926	-142.447560395032\\
-83.008	-79.3970635971611\\
-42.725	-40.8664169982256\\
-67.139	-64.2183819975159\\
-53.711	-51.3745142982257\\
-24.414	-23.3519649992903\\
-15.869	-15.1786815996452\\
-42.725	-40.8664169982256\\
-92.773	-88.7372756975162\\
-98.877	-94.5757451968063\\
-70.801	-67.7210810975159\\
-43.945	-42.0333456989356\\
-47.607	-45.5360447989356\\
-20.752	-19.8492658992903\\
-28.076	-26.8546640992904\\
-29.297	-28.0225492989354\\
-46.387	-44.3691160982256\\
-69.58	-66.5531958978709\\
-103.76	-99.2463294964514\\
-107.422	-102.749028596451\\
-120.85	-115.592896295742\\
-124.512	-119.095595395742\\
-120.85	-115.592896295742\\
-115.967	-110.922311996097\\
-101.318	-96.9105590971613\\
-123.291	-117.927710196097\\
-187.988	-179.810321794322\\
-169.678	-162.296826294322\\
-87.891	-84.0676478968062\\
-45.166	-43.2012308985806\\
-103.76	-99.2463294964514\\
-122.07	-116.759824996452\\
-114.746	-109.754426796451\\
-63.477	-60.7156828975158\\
-67.139	-64.2183819975159\\
-113.525	-108.586541596806\\
-175.781	-168.134339294677\\
-180.664	-172.804923594322\\
-203.857	-194.989003393968\\
-195.313	-186.816676493258\\
-139.16	-133.106391795742\\
-144.043	-137.776976095387\\
-59.814	-57.2120272985808\\
-61.035	-58.3799124982258\\
-76.904	-73.558594097871\\
-98.877	-94.5757451968063\\
-104.98	-100.413258197161\\
-111.084	-106.251727696451\\
-70.801	-67.7210810975159\\
-98.877	-94.5757451968063\\
-86.67	-82.8997626971611\\
-47.607	-45.5360447989356\\
-32.959	-31.5252483989354\\
-26.855	-25.6867788996453\\
-42.725	-40.8664169982256\\
-35.4	-33.8600622992905\\
-19.531	-18.6813806996453\\
-40.283	-38.5306465989355\\
-85.449	-81.7318774975161\\
-56.152	-53.7093281985807\\
-51.27	-49.0397003978707\\
-72.021	-68.8880097982259\\
-119.629	-114.425011096097\\
-81.787	-78.2291783975161\\
-42.725	-40.8664169982256\\
-21.973	-21.0171510989353\\
-57.373	-54.8772133982257\\
-126.953	-121.430409296097\\
-161.133	-154.123542894677\\
-223.389	-213.671340592548\\
-266.113	-254.536801091838\\
-261.23	-249.866216792193\\
-168.457	-161.128941094677\\
-170.898	-163.463754995032\\
-222.168	-212.503455392903\\
-195.313	-186.816676493258\\
-179.443	-171.637038394677\\
-202.637	-193.822074693258\\
-220.947	-211.335570193258\\
-223.389	-213.671340592548\\
-301.514	-288.397819890064\\
-203.857	-194.989003393968\\
-109.863	-105.083842496806\\
-59.814	-57.2120272985808\\
-47.607	-45.5360447989356\\
-54.932	-52.5423994978707\\
-53.711	-51.3745142982257\\
-56.152	-53.7093281985807\\
-101.318	-96.9105590971613\\
-79.346	-75.894364497161\\
-84.229	-80.5649487968061\\
-111.084	-106.251727696451\\
-63.477	-60.7156828975158\\
-70.801	-67.7210810975159\\
-101.318	-96.9105590971613\\
-115.967	-110.922311996097\\
-64.697	-61.8826115982258\\
-46.387	-44.3691160982256\\
-58.594	-56.0450985978708\\
-54.932	-52.5423994978707\\
-129.395	-123.766179695387\\
-186.768	-178.643393093612\\
-173.34	-165.799525394322\\
-198.975	-190.319375593258\\
-225.83	-216.006154492903\\
-297.852	-284.895120790064\\
-257.568	-246.363517692193\\
-157.471	-150.620843794677\\
-100.098	-95.7436303964513\\
-54.932	-52.5423994978707\\
-61.035	-58.3799124982258\\
-67.139	-64.2183819975159\\
-86.67	-82.8997626971611\\
-62.256	-59.5477976978708\\
-56.152	-53.7093281985807\\
-63.477	-60.7156828975158\\
-78.125	-74.726479297516\\
-89.111	-85.2345765975162\\
-111.084	-106.251727696451\\
-79.346	-75.894364497161\\
-40.283	-38.5306465989355\\
-28.076	-26.8546640992904\\
-23.193	-22.1840797996453\\
-24.414	-23.3519649992903\\
-34.18	-32.6931335985804\\
-57.373	-54.8772133982257\\
-59.814	-57.2120272985808\\
-75.684	-72.391665397161\\
-114.746	-109.754426796451\\
-81.787	-78.2291783975161\\
-150.146	-143.614489095742\\
-208.74	-199.659587693613\\
-162.354	-155.291428094322\\
-144.043	-137.776976095387\\
-170.898	-163.463754995032\\
-198.975	-190.319375593258\\
-131.836	-126.100993595742\\
-111.084	-106.251727696451\\
-72.021	-68.8880097982259\\
-79.346	-75.894364497161\\
-142.822	-136.609090895742\\
-247.803	-237.023305591838\\
-300.293	-287.229934690419\\
-239.258	-228.850022192193\\
-200.195	-191.486304293968\\
-145.264	-138.944861295032\\
-153.809	-147.118144694677\\
-208.74	-199.659587693613\\
-115.967	-110.922311996097\\
-98.877	-94.5757451968063\\
-75.684	-72.391665397161\\
-148.926	-142.447560395032\\
-136.719	-130.771577895387\\
-76.904	-73.558594097871\\
-72.021	-68.8880097982259\\
-53.711	-51.3745142982257\\
-95.215	-91.0730460968063\\
-145.264	-138.944861295032\\
-148.926	-142.447560395032\\
-113.525	-108.586541596806\\
-83.008	-79.3970635971611\\
-93.994	-89.9051608971612\\
-74.463	-71.223780197516\\
-53.711	-51.3745142982257\\
-43.945	-42.0333456989356\\
-39.063	-37.3637178982255\\
-29.297	-28.0225492989354\\
-35.4	-33.8600622992905\\
-65.918	-63.0504967978709\\
-95.215	-91.0730460968063\\
-96.436	-92.2409312964513\\
-58.594	-56.0450985978708\\
-53.711	-51.3745142982257\\
-41.504	-39.6985317985805\\
-54.932	-52.5423994978707\\
-78.125	-74.726479297516\\
-86.67	-82.8997626971611\\
-80.566	-77.0612931978711\\
-46.387	-44.3691160982256\\
-96.436	-92.2409312964513\\
-142.822	-136.609090895742\\
-101.318	-96.9105590971613\\
-40.283	-38.5306465989355\\
-50.049	-47.8718151982257\\
-76.904	-73.558594097871\\
-73.242	-70.055894997871\\
-67.139	-64.2183819975159\\
-41.504	-39.6985317985805\\
-75.684	-72.391665397161\\
-117.188	-112.090197195742\\
-70.801	-67.7210810975159\\
-25.635	-24.5198501989353\\
-39.063	-37.3637178982255\\
-91.553	-87.5703469968062\\
-111.084	-106.251727696451\\
-65.918	-63.0504967978709\\
-59.814	-57.2120272985808\\
-113.525	-108.586541596806\\
-151.367	-144.782374295387\\
-131.836	-126.100993595742\\
-179.443	-171.637038394677\\
-219.727	-210.168641492548\\
-140.381	-134.274276995387\\
-111.084	-106.251727696451\\
-159.912	-152.955657695032\\
-109.863	-105.083842496806\\
-142.822	-136.609090895742\\
-174.561	-166.967410593967\\
-137.939	-131.938506596097\\
-69.58	-66.5531958978709\\
-111.084	-106.251727696451\\
-202.637	-193.822074693258\\
-233.154	-223.011552692903\\
-169.678	-162.296826294322\\
-142.822	-136.609090895742\\
-191.65	-183.313020894323\\
-153.809	-147.118144694677\\
-96.436	-92.2409312964513\\
-85.449	-81.7318774975161\\
-107.422	-102.749028596451\\
-113.525	-108.586541596806\\
-109.863	-105.083842496806\\
-125.732	-120.262524096452\\
-87.891	-84.0676478968062\\
-90.332	-86.4024617971612\\
-130.615	-124.933108396097\\
-78.125	-74.726479297516\\
-62.256	-59.5477976978708\\
-50.049	-47.8718151982257\\
-40.283	-38.5306465989355\\
-46.387	-44.3691160982256\\
-85.449	-81.7318774975161\\
-79.346	-75.894364497161\\
-107.422	-102.749028596451\\
-139.16	-133.106391795742\\
-164.795	-157.626241994677\\
-123.291	-117.927710196097\\
-107.422	-102.749028596451\\
-41.504	-39.6985317985805\\
-67.139	-64.2183819975159\\
-129.395	-123.766179695387\\
-85.449	-81.7318774975161\\
-41.504	-39.6985317985805\\
-87.891	-84.0676478968062\\
-137.939	-131.938506596097\\
-142.822	-136.609090895742\\
-185.547	-177.475507893967\\
-245.361	-234.687535192548\\
-173.34	-165.799525394322\\
-147.705	-141.279675195387\\
-170.898	-163.463754995032\\
-113.525	-108.586541596806\\
-90.332	-86.4024617971612\\
-102.539	-98.0784442968063\\
-131.836	-126.100993595742\\
-141.602	-135.442162195032\\
-142.822	-136.609090895742\\
-79.346	-75.894364497161\\
-70.801	-67.7210810975159\\
-54.932	-52.5423994978707\\
-79.346	-75.894364497161\\
-75.684	-72.391665397161\\
-59.814	-57.2120272985808\\
-119.629	-114.425011096097\\
-131.836	-126.100993595742\\
-89.111	-85.2345765975162\\
-74.463	-71.223780197516\\
-108.643	-103.916913796096\\
-63.477	-60.7156828975158\\
-35.4	-33.8600622992905\\
-47.607	-45.5360447989356\\
-64.697	-61.8826115982258\\
-56.152	-53.7093281985807\\
-41.504	-39.6985317985805\\
-23.193	-22.1840797996453\\
-42.725	-40.8664169982256\\
-64.697	-61.8826115982258\\
-102.539	-98.0784442968063\\
-118.408	-113.257125896452\\
-152.588	-145.950259495032\\
-173.34	-165.799525394322\\
-91.553	-87.5703469968062\\
-69.58	-66.5531958978709\\
-136.719	-130.771577895387\\
-211.182	-201.995358092903\\
-164.795	-157.626241994677\\
-151.367	-144.782374295387\\
-230.713	-220.676738792548\\
-185.547	-177.475507893967\\
-148.926	-142.447560395032\\
-112.305	-107.419612896096\\
-114.746	-109.754426796451\\
-155.029	-148.285073395387\\
-107.422	-102.749028596451\\
-90.332	-86.4024617971612\\
-156.25	-149.452958595032\\
-196.533	-187.983605193968\\
-205.078	-196.156888593613\\
-92.773	-88.7372756975162\\
-43.945	-42.0333456989356\\
-111.084	-106.251727696451\\
-87.891	-84.0676478968062\\
-37.842	-36.1958326985805\\
-29.297	-28.0225492989354\\
-53.711	-51.3745142982257\\
-81.787	-78.2291783975161\\
-135.498	-129.603692695742\\
-96.436	-92.2409312964513\\
-70.801	-67.7210810975159\\
-42.725	-40.8664169982256\\
-28.076	-26.8546640992904\\
-39.063	-37.3637178982255\\
-35.4	-33.8600622992905\\
-45.166	-43.2012308985806\\
-79.346	-75.894364497161\\
-72.021	-68.8880097982259\\
-32.959	-31.5252483989354\\
-30.518	-29.1904344985804\\
-41.504	-39.6985317985805\\
-53.711	-51.3745142982257\\
-32.959	-31.5252483989354\\
-51.27	-49.0397003978707\\
-37.842	-36.1958326985805\\
-25.635	-24.5198501989353\\
-63.477	-60.7156828975158\\
-96.436	-92.2409312964513\\
-145.264	-138.944861295032\\
-190.43	-182.146092193613\\
-161.133	-154.123542894677\\
-80.566	-77.0612931978711\\
-57.373	-54.8772133982257\\
-58.594	-56.0450985978708\\
-32.959	-31.5252483989354\\
-30.518	-29.1904344985804\\
-64.697	-61.8826115982258\\
-73.242	-70.055894997871\\
-67.139	-64.2183819975159\\
-73.242	-70.055894997871\\
-89.111	-85.2345765975162\\
-93.994	-89.9051608971612\\
-97.656	-93.4078599971613\\
-122.07	-116.759824996452\\
-118.408	-113.257125896452\\
-170.898	-163.463754995032\\
-89.111	-85.2345765975162\\
-41.504	-39.6985317985805\\
-40.283	-38.5306465989355\\
-76.904	-73.558594097871\\
-63.477	-60.7156828975158\\
-57.373	-54.8772133982257\\
-26.855	-25.6867788996453\\
-45.166	-43.2012308985806\\
-28.076	-26.8546640992904\\
-13.428	-12.8438676992902\\
-25.635	-24.5198501989353\\
-52.49	-50.2066290985807\\
-64.697	-61.8826115982258\\
-96.436	-92.2409312964513\\
-136.719	-130.771577895387\\
-98.877	-94.5757451968063\\
-43.945	-42.0333456989356\\
-80.566	-77.0612931978711\\
-90.332	-86.4024617971612\\
-42.725	-40.8664169982256\\
-57.373	-54.8772133982257\\
-112.305	-107.419612896096\\
-97.656	-93.4078599971613\\
-162.354	-155.291428094322\\
-189.209	-180.978206993967\\
-117.188	-112.090197195742\\
-92.773	-88.7372756975162\\
};
\end{axis}

\begin{axis}[%
width=4.927cm,
height=3cm,
at={(0cm,0cm)},
scale only axis,
xmin=-300,
xmax=0,
xlabel style={font=\color{white!15!black}},
xlabel={y(t-1)},
ymin=-279.541,
ymax=0,
ylabel style={font=\color{white!15!black}},
ylabel={y(t)},
axis background/.style={fill=white},
title style={font=\small},
title={C8, R = 0.7799},
axis x line*=bottom,
axis y line*=left
]
\addplot[only marks, mark=*, mark options={}, mark size=1.5000pt, color=mycolor1, fill=mycolor1] table[row sep=crcr]{%
x	y\\
-84.229	-96.436\\
-96.436	-122.07\\
-122.07	-100.098\\
-100.098	-93.994\\
-93.994	-130.615\\
-130.615	-125.732\\
-125.732	-153.809\\
-153.809	-115.967\\
-115.967	-61.035\\
-61.035	-74.463\\
-74.463	-73.242\\
-73.242	-86.67\\
-86.67	-73.242\\
-73.242	-34.18\\
-34.18	-20.752\\
-20.752	-23.193\\
-23.193	-58.594\\
-58.594	-100.098\\
-100.098	-109.863\\
-109.863	-114.746\\
-114.746	-83.008\\
-83.008	-52.49\\
-52.49	-104.98\\
-104.98	-81.787\\
-81.787	-59.814\\
-59.814	-97.656\\
-97.656	-83.008\\
-83.008	-72.021\\
-72.021	-118.408\\
-118.408	-107.422\\
-107.422	-83.008\\
-83.008	-119.629\\
-119.629	-123.291\\
-123.291	-95.215\\
-95.215	-80.566\\
-80.566	-62.256\\
-62.256	-75.684\\
-75.684	-95.215\\
-95.215	-87.891\\
-87.891	-91.553\\
-91.553	-96.436\\
-96.436	-102.539\\
-102.539	-141.602\\
-141.602	-128.174\\
-128.174	-85.449\\
-85.449	-79.346\\
-79.346	-47.607\\
-47.607	-48.828\\
-48.828	-68.359\\
-68.359	-52.49\\
-52.49	-57.373\\
-57.373	-75.684\\
-75.684	-89.111\\
-89.111	-81.787\\
-81.787	-128.174\\
-128.174	-98.877\\
-98.877	-57.373\\
-57.373	-45.166\\
-45.166	-62.256\\
-62.256	-72.021\\
-72.021	-40.283\\
-40.283	-42.725\\
-42.725	-56.152\\
-56.152	-46.387\\
-46.387	-40.283\\
-40.283	-52.49\\
-52.49	-62.256\\
-62.256	-92.773\\
-92.773	-90.332\\
-90.332	-107.422\\
-107.422	-98.877\\
-98.877	-115.967\\
-115.967	-91.553\\
-91.553	-91.553\\
-91.553	-79.346\\
-79.346	-75.684\\
-75.684	-76.904\\
-76.904	-133.057\\
-133.057	-189.209\\
-189.209	-194.092\\
-194.092	-189.209\\
-189.209	-119.629\\
-119.629	-170.898\\
-170.898	-213.623\\
-213.623	-219.727\\
-219.727	-169.678\\
-169.678	-219.727\\
-219.727	-279.541\\
-279.541	-197.754\\
-197.754	-161.133\\
-161.133	-109.863\\
-109.863	-85.449\\
-85.449	-73.242\\
-73.242	-91.553\\
-91.553	-62.256\\
-62.256	-46.387\\
-46.387	-42.725\\
-42.725	-54.932\\
-54.932	-79.346\\
-79.346	-74.463\\
-74.463	-75.684\\
-75.684	-100.098\\
-100.098	-103.76\\
-103.76	-141.602\\
-141.602	-114.746\\
-114.746	-142.822\\
-142.822	-104.98\\
-104.98	-90.332\\
-90.332	-103.76\\
-103.76	-76.904\\
-76.904	-69.58\\
-69.58	-84.229\\
-84.229	-86.67\\
-86.67	-59.814\\
-59.814	-64.697\\
-64.697	-48.828\\
-48.828	-53.711\\
-53.711	-57.373\\
-57.373	-41.504\\
-41.504	-52.49\\
-52.49	-65.918\\
-65.918	-101.318\\
-101.318	-104.98\\
-104.98	-114.746\\
-114.746	-68.359\\
-68.359	-93.994\\
-93.994	-150.146\\
-150.146	-114.746\\
-114.746	-125.732\\
-125.732	-128.174\\
-128.174	-80.566\\
-80.566	-53.711\\
-53.711	-75.684\\
-75.684	-85.449\\
-85.449	-137.939\\
-137.939	-172.119\\
-172.119	-170.898\\
-170.898	-123.291\\
-123.291	-130.615\\
-130.615	-115.967\\
-115.967	-113.525\\
-113.525	-111.084\\
-111.084	-76.904\\
-76.904	-68.359\\
-68.359	-61.035\\
-61.035	-57.373\\
-57.373	-62.256\\
-62.256	-91.553\\
-91.553	-140.381\\
-140.381	-111.084\\
-111.084	-79.346\\
-79.346	-64.697\\
-64.697	-62.256\\
-62.256	-40.283\\
-40.283	-29.297\\
-29.297	-36.621\\
-36.621	-63.477\\
-63.477	-51.27\\
-51.27	-52.49\\
-52.49	-58.594\\
-58.594	-54.932\\
-54.932	-37.842\\
-37.842	-28.076\\
-28.076	-23.193\\
-23.193	-39.063\\
-39.063	-83.008\\
-83.008	-117.188\\
-117.188	-130.615\\
-130.615	-86.67\\
-86.67	-58.594\\
-58.594	-45.166\\
-45.166	-32.959\\
-32.959	-67.139\\
-67.139	-65.918\\
-65.918	-91.553\\
-91.553	-117.188\\
-117.188	-186.768\\
-186.768	-216.064\\
-216.064	-177.002\\
-177.002	-168.457\\
-168.457	-109.863\\
-109.863	-122.07\\
-122.07	-131.836\\
-131.836	-146.484\\
-146.484	-108.643\\
-108.643	-100.098\\
-100.098	-98.877\\
-98.877	-89.111\\
-89.111	-106.201\\
-106.201	-78.125\\
-78.125	-130.615\\
-130.615	-159.912\\
-159.912	-113.525\\
-113.525	-70.801\\
-70.801	-73.242\\
-73.242	-123.291\\
-123.291	-153.809\\
-153.809	-189.209\\
-189.209	-202.637\\
-202.637	-202.637\\
-202.637	-153.809\\
-153.809	-137.939\\
-137.939	-158.691\\
-158.691	-187.988\\
-187.988	-109.863\\
-109.863	-72.021\\
-72.021	-101.318\\
-101.318	-86.67\\
-86.67	-53.711\\
-53.711	-74.463\\
-74.463	-84.229\\
-84.229	-67.139\\
-67.139	-51.27\\
-51.27	-70.801\\
-70.801	-58.594\\
-58.594	-79.346\\
-79.346	-107.422\\
-107.422	-100.098\\
-100.098	-69.58\\
-69.58	-70.801\\
-70.801	-107.422\\
-107.422	-177.002\\
-177.002	-128.174\\
-128.174	-79.346\\
-79.346	-61.035\\
-61.035	-70.801\\
-70.801	-64.697\\
-64.697	-48.828\\
-48.828	-53.711\\
-53.711	-65.918\\
-65.918	-83.008\\
-83.008	-91.553\\
-91.553	-63.477\\
-63.477	-104.98\\
-104.98	-124.512\\
-124.512	-118.408\\
-118.408	-91.553\\
-91.553	-100.098\\
-100.098	-135.498\\
-135.498	-87.891\\
-87.891	-42.725\\
-42.725	-57.373\\
-57.373	-62.256\\
-62.256	-81.787\\
-81.787	-72.021\\
-72.021	-52.49\\
-52.49	-79.346\\
-79.346	-80.566\\
-80.566	-57.373\\
-57.373	-70.801\\
-70.801	-63.477\\
-63.477	-56.152\\
-56.152	-67.139\\
-67.139	-41.504\\
-41.504	-48.828\\
-48.828	-36.621\\
-36.621	-40.283\\
-40.283	-75.684\\
-75.684	-84.229\\
-84.229	-113.525\\
-113.525	-146.484\\
-146.484	-98.877\\
-98.877	-58.594\\
-58.594	-46.387\\
-46.387	-43.945\\
-43.945	-63.477\\
-63.477	-42.725\\
-42.725	-47.607\\
-47.607	-61.035\\
-61.035	-102.539\\
-102.539	-93.994\\
-93.994	-134.277\\
-134.277	-89.111\\
-89.111	-81.787\\
-81.787	-46.387\\
-46.387	-45.166\\
-45.166	-28.076\\
-28.076	-35.4\\
-35.4	-37.842\\
-37.842	-67.139\\
-67.139	-70.801\\
-70.801	-79.346\\
-79.346	-85.449\\
-85.449	-125.732\\
-125.732	-112.305\\
-112.305	-84.229\\
-84.229	-102.539\\
-102.539	-109.863\\
-109.863	-144.043\\
-144.043	-115.967\\
-115.967	-75.684\\
-75.684	-40.283\\
-40.283	-29.297\\
-29.297	-29.297\\
-29.297	-36.621\\
-36.621	-39.063\\
-39.063	-37.842\\
-37.842	-50.049\\
-50.049	-74.463\\
-74.463	-68.359\\
-68.359	-70.801\\
-70.801	-83.008\\
-83.008	-51.27\\
-51.27	-30.518\\
-30.518	-58.594\\
-58.594	-90.332\\
-90.332	-87.891\\
-87.891	-97.656\\
-97.656	-83.008\\
-83.008	-61.035\\
-61.035	-85.449\\
-85.449	-118.408\\
-118.408	-118.408\\
-118.408	-84.229\\
-84.229	-136.719\\
-136.719	-100.098\\
-100.098	-101.318\\
-101.318	-117.188\\
-117.188	-115.967\\
-115.967	-86.67\\
-86.67	-76.904\\
-76.904	-128.174\\
-128.174	-178.223\\
-178.223	-139.16\\
-139.16	-79.346\\
-79.346	-76.904\\
-76.904	-79.346\\
-79.346	-98.877\\
-98.877	-114.746\\
-114.746	-85.449\\
-85.449	-90.332\\
-90.332	-120.85\\
-120.85	-131.836\\
-131.836	-87.891\\
-87.891	-68.359\\
-68.359	-91.553\\
-91.553	-156.25\\
-156.25	-137.939\\
-137.939	-140.381\\
-140.381	-84.229\\
-84.229	-76.904\\
-76.904	-72.021\\
-72.021	-47.607\\
-47.607	-30.518\\
-30.518	-26.855\\
-26.855	-23.193\\
-23.193	-54.932\\
-54.932	-70.801\\
-70.801	-87.891\\
-87.891	-65.918\\
-65.918	-73.242\\
-73.242	-83.008\\
-83.008	-53.711\\
-53.711	-56.152\\
-56.152	-53.711\\
-53.711	-40.283\\
-40.283	-51.27\\
-51.27	-84.229\\
-84.229	-106.201\\
-106.201	-63.477\\
-63.477	-48.828\\
-48.828	-40.283\\
-40.283	-50.049\\
-50.049	-35.4\\
-35.4	-76.904\\
-76.904	-130.615\\
-130.615	-104.98\\
-104.98	-139.16\\
-139.16	-162.354\\
-162.354	-164.795\\
-164.795	-124.512\\
-124.512	-90.332\\
-90.332	-89.111\\
-89.111	-89.111\\
-89.111	-106.201\\
-106.201	-125.732\\
-125.732	-125.732\\
-125.732	-168.457\\
-168.457	-183.105\\
-183.105	-219.727\\
-219.727	-147.705\\
-147.705	-166.016\\
-166.016	-184.326\\
-184.326	-140.381\\
-140.381	-162.354\\
-162.354	-139.16\\
-139.16	-80.566\\
-80.566	-52.49\\
-52.49	-56.152\\
-56.152	-81.787\\
-81.787	-65.918\\
-65.918	-43.945\\
-43.945	-62.256\\
-62.256	-47.607\\
-47.607	-36.621\\
-36.621	-46.387\\
-46.387	-52.49\\
-52.49	-36.621\\
-36.621	-70.801\\
-70.801	-92.773\\
-92.773	-68.359\\
-68.359	-114.746\\
-114.746	-164.795\\
-164.795	-150.146\\
-150.146	-196.533\\
-196.533	-131.836\\
-131.836	-144.043\\
-144.043	-96.436\\
-96.436	-63.477\\
-63.477	-76.904\\
-76.904	-59.814\\
-59.814	-45.166\\
-45.166	-64.697\\
-64.697	-36.621\\
-36.621	-28.076\\
-28.076	-34.18\\
-34.18	-70.801\\
-70.801	-86.67\\
-86.67	-107.422\\
-107.422	-103.76\\
-103.76	-100.098\\
-100.098	-114.746\\
-114.746	-93.994\\
-93.994	-76.904\\
-76.904	-76.904\\
-76.904	-62.256\\
-62.256	-97.656\\
-97.656	-68.359\\
-68.359	-56.152\\
-56.152	-81.787\\
-81.787	-113.525\\
-113.525	-86.67\\
-86.67	-65.918\\
-65.918	-56.152\\
-56.152	-65.918\\
-65.918	-45.166\\
-45.166	-45.166\\
-45.166	-59.814\\
-59.814	-75.684\\
-75.684	-84.229\\
-84.229	-65.918\\
-65.918	-86.67\\
-86.67	-79.346\\
-79.346	-47.607\\
-47.607	-50.049\\
-50.049	-95.215\\
-95.215	-131.836\\
-131.836	-131.836\\
-131.836	-107.422\\
-107.422	-103.76\\
-103.76	-79.346\\
-79.346	-76.904\\
-76.904	-61.035\\
-61.035	-89.111\\
-89.111	-75.684\\
-75.684	-50.049\\
-50.049	-74.463\\
-74.463	-85.449\\
-85.449	-101.318\\
-101.318	-109.863\\
-109.863	-109.863\\
-109.863	-148.926\\
-148.926	-181.885\\
-181.885	-205.078\\
-205.078	-129.395\\
-129.395	-73.242\\
-73.242	-47.607\\
-47.607	-34.18\\
-34.18	-54.932\\
-54.932	-46.387\\
-46.387	-80.566\\
-80.566	-72.021\\
-72.021	-64.697\\
-64.697	-85.449\\
-85.449	-98.877\\
-98.877	-117.188\\
-117.188	-123.291\\
-123.291	-175.781\\
-175.781	-150.146\\
-150.146	-117.188\\
-117.188	-80.566\\
-80.566	-80.566\\
-80.566	-83.008\\
-83.008	-106.201\\
-106.201	-75.684\\
-75.684	-79.346\\
-79.346	-74.463\\
-74.463	-42.725\\
-42.725	-74.463\\
-74.463	-107.422\\
-107.422	-73.242\\
-73.242	-80.566\\
-80.566	-147.705\\
-147.705	-109.863\\
-109.863	-106.201\\
-106.201	-147.705\\
-147.705	-140.381\\
-140.381	-87.891\\
-87.891	-56.152\\
-56.152	-58.594\\
-58.594	-97.656\\
-97.656	-151.367\\
-151.367	-162.354\\
-162.354	-167.236\\
-167.236	-129.395\\
-129.395	-83.008\\
-83.008	-72.021\\
-72.021	-53.711\\
-53.711	-50.049\\
-50.049	-42.725\\
-42.725	-61.035\\
-61.035	-72.021\\
-72.021	-41.504\\
-41.504	-65.918\\
-65.918	-89.111\\
-89.111	-101.318\\
-101.318	-128.174\\
-128.174	-161.133\\
-161.133	-192.871\\
-192.871	-136.719\\
-136.719	-118.408\\
-118.408	-124.512\\
-124.512	-102.539\\
-102.539	-73.242\\
-73.242	-89.111\\
-89.111	-136.719\\
-136.719	-162.354\\
-162.354	-93.994\\
-93.994	-54.932\\
-54.932	-37.842\\
-37.842	-23.193\\
-23.193	-32.959\\
-32.959	-45.166\\
-45.166	-52.49\\
-52.49	-73.242\\
-73.242	-58.594\\
-58.594	-45.166\\
-45.166	-45.166\\
-45.166	-43.945\\
-43.945	-40.283\\
-40.283	-47.607\\
-47.607	-70.801\\
-70.801	-104.98\\
-104.98	-111.084\\
-111.084	-107.422\\
-107.422	-123.291\\
-123.291	-152.588\\
-152.588	-155.029\\
-155.029	-183.105\\
-183.105	-167.236\\
-167.236	-124.512\\
-124.512	-85.449\\
-85.449	-83.008\\
-83.008	-141.602\\
-141.602	-158.691\\
-158.691	-115.967\\
-115.967	-76.904\\
-76.904	-39.063\\
-39.063	-30.518\\
-30.518	-29.297\\
-29.297	-19.531\\
-19.531	-45.166\\
-45.166	-57.373\\
-57.373	-62.256\\
-62.256	-53.711\\
-53.711	-34.18\\
-34.18	-26.855\\
-26.855	-36.621\\
-36.621	-41.504\\
-41.504	-43.945\\
-43.945	-35.4\\
-35.4	-28.076\\
-28.076	-50.049\\
-50.049	-112.305\\
-112.305	-128.174\\
-128.174	-145.264\\
-145.264	-101.318\\
-101.318	-133.057\\
-133.057	-195.313\\
-195.313	-174.561\\
-174.561	-184.326\\
-184.326	-135.498\\
-135.498	-137.939\\
-137.939	-145.264\\
-145.264	-95.215\\
-95.215	-90.332\\
-90.332	-56.152\\
-56.152	-54.932\\
-54.932	-102.539\\
-102.539	-68.359\\
-68.359	-79.346\\
-79.346	-84.229\\
-84.229	-79.346\\
-79.346	-79.346\\
-79.346	-84.229\\
-84.229	-93.994\\
-93.994	-102.539\\
-102.539	-89.111\\
-89.111	-67.139\\
-67.139	-84.229\\
-84.229	-39.063\\
-39.063	-20.752\\
-20.752	-36.621\\
-36.621	-51.27\\
-51.27	-26.855\\
-26.855	-13.428\\
-13.428	-31.738\\
-31.738	-50.049\\
-50.049	-58.594\\
-58.594	-72.021\\
-72.021	-62.256\\
-62.256	-91.553\\
-91.553	-80.566\\
-80.566	-57.373\\
-57.373	-86.67\\
-86.67	-48.828\\
-48.828	-41.504\\
-41.504	-28.076\\
-28.076	-21.973\\
-21.973	-37.842\\
-37.842	-28.076\\
-28.076	-50.049\\
-50.049	-75.684\\
-75.684	-73.242\\
-73.242	-54.932\\
-54.932	-61.035\\
-61.035	-45.166\\
-45.166	-65.918\\
-65.918	-47.607\\
-47.607	-46.387\\
-46.387	-42.725\\
-42.725	-32.959\\
-32.959	-42.725\\
-42.725	-37.842\\
-37.842	-64.697\\
-64.697	-63.477\\
-63.477	-50.049\\
-50.049	-46.387\\
-46.387	-36.621\\
-36.621	-43.945\\
-43.945	-93.994\\
-93.994	-120.85\\
-120.85	-81.787\\
-81.787	-80.566\\
-80.566	-54.932\\
-54.932	-93.994\\
-93.994	-81.787\\
-81.787	-54.932\\
-54.932	-70.801\\
-70.801	-52.49\\
-52.49	-75.684\\
-75.684	-42.725\\
-42.725	-50.049\\
-50.049	-57.373\\
-57.373	-74.463\\
-74.463	-58.594\\
-58.594	-61.035\\
-61.035	-61.035\\
-61.035	-95.215\\
-95.215	-79.346\\
-79.346	-125.732\\
-125.732	-157.471\\
-157.471	-125.732\\
-125.732	-81.787\\
-81.787	-62.256\\
-62.256	-89.111\\
-89.111	-112.305\\
-112.305	-87.891\\
-87.891	-107.422\\
-107.422	-64.697\\
-64.697	-34.18\\
-34.18	-37.842\\
-37.842	-84.229\\
-84.229	-70.801\\
-70.801	-69.58\\
-69.58	-104.98\\
-104.98	-97.656\\
-97.656	-87.891\\
-87.891	-62.256\\
-62.256	-73.242\\
-73.242	-100.098\\
-100.098	-81.787\\
-81.787	-87.891\\
-87.891	-78.125\\
-78.125	-58.594\\
-58.594	-67.139\\
-67.139	-48.828\\
-48.828	-56.152\\
-56.152	-95.215\\
-95.215	-109.863\\
-109.863	-115.967\\
-115.967	-102.539\\
-102.539	-73.242\\
-73.242	-51.27\\
-51.27	-62.256\\
-62.256	-70.801\\
-70.801	-91.553\\
-91.553	-111.084\\
-111.084	-131.836\\
-131.836	-107.422\\
-107.422	-62.256\\
-62.256	-109.863\\
-109.863	-151.367\\
-151.367	-107.422\\
-107.422	-108.643\\
-108.643	-124.512\\
-124.512	-93.994\\
-93.994	-79.346\\
-79.346	-89.111\\
-89.111	-78.125\\
-78.125	-91.553\\
-91.553	-114.746\\
-114.746	-86.67\\
-86.67	-102.539\\
-102.539	-92.773\\
-92.773	-96.436\\
-96.436	-76.904\\
-76.904	-100.098\\
-100.098	-69.58\\
-69.58	-76.904\\
-76.904	-84.229\\
-84.229	-80.566\\
-80.566	-41.504\\
-41.504	-28.076\\
-28.076	-21.973\\
-21.973	-37.842\\
-37.842	-84.229\\
-84.229	-108.643\\
-108.643	-111.084\\
-111.084	-74.463\\
-74.463	-86.67\\
-86.67	-72.021\\
-72.021	-65.918\\
-65.918	-42.725\\
-42.725	-61.035\\
-61.035	-58.594\\
-58.594	-57.373\\
-57.373	-67.139\\
-67.139	-57.373\\
-57.373	-53.711\\
-53.711	-37.842\\
-37.842	-50.049\\
-50.049	-91.553\\
-91.553	-70.801\\
-70.801	-86.67\\
-86.67	-75.684\\
-75.684	-101.318\\
-101.318	-92.773\\
-92.773	-129.395\\
-129.395	-91.553\\
-91.553	-106.201\\
-106.201	-103.76\\
-103.76	-54.932\\
-54.932	-96.436\\
-96.436	-67.139\\
-67.139	-84.229\\
-84.229	-90.332\\
-90.332	-115.967\\
-115.967	-85.449\\
-85.449	-111.084\\
-111.084	-136.719\\
-136.719	-111.084\\
-111.084	-95.215\\
-95.215	-75.684\\
-75.684	-91.553\\
-91.553	-80.566\\
-80.566	-69.58\\
-69.58	-56.152\\
-56.152	-70.801\\
-70.801	-58.594\\
-58.594	-119.629\\
-119.629	-151.367\\
-151.367	-130.615\\
-130.615	-126.953\\
-126.953	-124.512\\
-124.512	-69.58\\
-69.58	-63.477\\
-63.477	-50.049\\
-50.049	-42.725\\
-42.725	-48.828\\
-48.828	-80.566\\
-80.566	-100.098\\
-100.098	-114.746\\
-114.746	-96.436\\
-96.436	-67.139\\
-67.139	-117.188\\
-117.188	-137.939\\
-137.939	-152.588\\
-152.588	-118.408\\
-118.408	-187.988\\
-187.988	-133.057\\
-133.057	-78.125\\
-78.125	-57.373\\
-57.373	-48.828\\
-48.828	-87.891\\
-87.891	-111.084\\
-111.084	-147.705\\
-147.705	-109.863\\
-109.863	-87.891\\
-87.891	-47.607\\
-47.607	-53.711\\
-53.711	-56.152\\
-56.152	-63.477\\
-63.477	-40.283\\
-40.283	-52.49\\
-52.49	-40.283\\
-40.283	-26.855\\
-26.855	-58.594\\
-58.594	-63.477\\
-63.477	-80.566\\
-80.566	-72.021\\
-72.021	-76.904\\
-76.904	-76.904\\
-76.904	-98.877\\
-98.877	-67.139\\
-67.139	-101.318\\
-101.318	-117.188\\
-117.188	-89.111\\
-89.111	-93.994\\
-93.994	-80.566\\
-80.566	-93.994\\
-93.994	-133.057\\
-133.057	-100.098\\
-100.098	-79.346\\
-79.346	-90.332\\
-90.332	-81.787\\
-81.787	-56.152\\
-56.152	-85.449\\
-85.449	-123.291\\
-123.291	-117.188\\
-117.188	-130.615\\
-130.615	-192.871\\
-192.871	-125.732\\
-125.732	-107.422\\
-107.422	-83.008\\
-83.008	-106.201\\
-106.201	-152.588\\
-152.588	-101.318\\
-101.318	-90.332\\
-90.332	-124.512\\
-124.512	-79.346\\
-79.346	-47.607\\
-47.607	-62.256\\
-62.256	-46.387\\
-46.387	-39.063\\
-39.063	-42.725\\
-42.725	-37.842\\
-37.842	-37.842\\
-37.842	-47.607\\
-47.607	-65.918\\
-65.918	-78.125\\
-78.125	-102.539\\
-102.539	-96.436\\
-96.436	-112.305\\
-112.305	-130.615\\
-130.615	-119.629\\
-119.629	-86.67\\
-86.67	-92.773\\
-92.773	-83.008\\
-83.008	-91.553\\
-91.553	-126.953\\
-126.953	-93.994\\
-93.994	-106.201\\
-106.201	-91.553\\
-91.553	-87.891\\
-87.891	-140.381\\
-140.381	-113.525\\
-113.525	-115.967\\
-115.967	-103.76\\
-103.76	-64.697\\
-64.697	-42.725\\
-42.725	-34.18\\
-34.18	-61.035\\
-61.035	-54.932\\
-54.932	-74.463\\
-74.463	-56.152\\
-56.152	-79.346\\
-79.346	-124.512\\
-124.512	-155.029\\
-155.029	-122.07\\
-122.07	-128.174\\
-128.174	-113.525\\
-113.525	-83.008\\
-83.008	-87.891\\
-87.891	-98.877\\
-98.877	-84.229\\
-84.229	-101.318\\
-101.318	-126.953\\
-126.953	-178.223\\
-178.223	-156.25\\
-156.25	-147.705\\
-147.705	-164.795\\
-164.795	-109.863\\
-109.863	-120.85\\
-120.85	-93.994\\
-93.994	-95.215\\
-95.215	-126.953\\
-126.953	-144.043\\
-144.043	-56.152\\
-56.152	-74.463\\
-74.463	-57.373\\
-57.373	-83.008\\
-83.008	-84.229\\
-84.229	-51.27\\
-51.27	-68.359\\
-68.359	-119.629\\
-119.629	-196.533\\
-196.533	-190.43\\
-190.43	-174.561\\
-174.561	-137.939\\
-137.939	-161.133\\
-161.133	-196.533\\
-196.533	-144.043\\
-144.043	-148.926\\
-148.926	-153.809\\
-153.809	-104.98\\
-104.98	-91.553\\
-91.553	-118.408\\
-118.408	-79.346\\
-79.346	-59.814\\
-59.814	-80.566\\
-80.566	-59.814\\
-59.814	-73.242\\
-73.242	-67.139\\
-67.139	-84.229\\
-84.229	-111.084\\
-111.084	-85.449\\
-85.449	-64.697\\
-64.697	-76.904\\
-76.904	-89.111\\
-89.111	-107.422\\
-107.422	-89.111\\
-89.111	-74.463\\
-74.463	-115.967\\
-115.967	-108.643\\
-108.643	-75.684\\
-75.684	-126.953\\
-126.953	-113.525\\
-113.525	-83.008\\
-83.008	-57.373\\
-57.373	-42.725\\
-42.725	-31.738\\
-31.738	-69.58\\
-69.58	-63.477\\
-63.477	-47.607\\
-47.607	-34.18\\
-34.18	-42.725\\
-42.725	-70.801\\
-70.801	-81.787\\
-81.787	-104.98\\
-104.98	-122.07\\
-122.07	-117.188\\
-117.188	-122.07\\
-122.07	-72.021\\
-72.021	-34.18\\
-34.18	-50.049\\
-50.049	-41.504\\
-41.504	-53.711\\
-53.711	-86.67\\
-86.67	-95.215\\
-95.215	-141.602\\
-141.602	-92.773\\
-92.773	-87.891\\
-87.891	-128.174\\
-128.174	-93.994\\
-93.994	-65.918\\
-65.918	-50.049\\
-50.049	-65.918\\
-65.918	-87.891\\
-87.891	-129.395\\
-129.395	-86.67\\
-86.67	-74.463\\
-74.463	-72.021\\
-72.021	-86.67\\
-86.67	-113.525\\
-113.525	-179.443\\
-179.443	-151.367\\
-151.367	-151.367\\
-151.367	-97.656\\
-97.656	-64.697\\
-64.697	-76.904\\
-76.904	-36.621\\
-36.621	-54.932\\
-54.932	-51.27\\
-51.27	-54.932\\
-54.932	-93.994\\
-93.994	-128.174\\
-128.174	-146.484\\
-146.484	-189.209\\
-189.209	-161.133\\
-161.133	-86.67\\
-86.67	-81.787\\
-81.787	-68.359\\
-68.359	-91.553\\
-91.553	-120.85\\
-120.85	-161.133\\
-161.133	-113.525\\
-113.525	-80.566\\
-80.566	-92.773\\
-92.773	-124.512\\
-124.512	-87.891\\
-87.891	-63.477\\
-63.477	-51.27\\
-51.27	-37.842\\
-37.842	-34.18\\
-34.18	-29.297\\
-29.297	-47.607\\
-47.607	-64.697\\
-64.697	-73.242\\
-73.242	-56.152\\
-56.152	-52.49\\
-52.49	-26.855\\
-26.855	-15.869\\
-15.869	-23.193\\
-23.193	-41.504\\
-41.504	-62.256\\
-62.256	-76.904\\
-76.904	-86.67\\
-86.67	-101.318\\
-101.318	-131.836\\
-131.836	-181.885\\
-181.885	-179.443\\
-179.443	-194.092\\
-194.092	-169.678\\
-169.678	-93.994\\
-93.994	-86.67\\
-86.67	-85.449\\
-85.449	-113.525\\
-113.525	-96.436\\
-96.436	-70.801\\
-70.801	-56.152\\
-56.152	-46.387\\
-46.387	-76.904\\
-76.904	-76.904\\
-76.904	-42.725\\
-42.725	-32.959\\
-32.959	-50.049\\
-50.049	-67.139\\
-67.139	-51.27\\
-51.27	-73.242\\
-73.242	-54.932\\
-54.932	-81.787\\
-81.787	-122.07\\
-122.07	-96.436\\
-96.436	-128.174\\
-128.174	-177.002\\
-177.002	-189.209\\
-189.209	-148.926\\
-148.926	-95.215\\
-95.215	-73.242\\
-73.242	-45.166\\
-45.166	-68.359\\
-68.359	-47.607\\
-47.607	-85.449\\
-85.449	-79.346\\
-79.346	-62.256\\
-62.256	-81.787\\
-81.787	-47.607\\
-47.607	-69.58\\
-69.58	-54.932\\
-54.932	-34.18\\
-34.18	-39.063\\
-39.063	-32.959\\
-32.959	-37.842\\
-37.842	-32.959\\
-32.959	-24.414\\
-24.414	-31.738\\
-31.738	-45.166\\
-45.166	-43.945\\
-43.945	-43.945\\
-43.945	-68.359\\
-68.359	-90.332\\
-90.332	-72.021\\
-72.021	-67.139\\
-67.139	-104.98\\
-104.98	-125.732\\
-125.732	-100.098\\
-100.098	-104.98\\
-104.98	-51.27\\
-51.27	-32.959\\
-32.959	-62.256\\
-62.256	-91.553\\
-91.553	-96.436\\
-96.436	-135.498\\
-135.498	-120.85\\
-120.85	-86.67\\
-86.67	-61.035\\
-61.035	-64.697\\
-64.697	-73.242\\
-73.242	-56.152\\
-56.152	-35.4\\
-35.4	-45.166\\
-45.166	-37.842\\
-37.842	-31.738\\
-31.738	-25.635\\
-25.635	-45.166\\
-45.166	-62.256\\
-62.256	-68.359\\
-68.359	-50.049\\
-50.049	-81.787\\
-81.787	-115.967\\
-115.967	-74.463\\
-74.463	-100.098\\
-100.098	-112.305\\
-112.305	-83.008\\
-83.008	-48.828\\
-48.828	-62.256\\
-62.256	-74.463\\
-74.463	-45.166\\
-45.166	-48.828\\
-48.828	-41.504\\
-41.504	-24.414\\
-24.414	-24.414\\
-24.414	-40.283\\
-40.283	-54.932\\
-54.932	-48.828\\
-48.828	-59.814\\
-59.814	-69.58\\
-69.58	-50.049\\
-50.049	-30.518\\
-30.518	-24.414\\
-24.414	-31.738\\
-31.738	-36.621\\
-36.621	-43.945\\
-43.945	-86.67\\
-86.67	-89.111\\
-89.111	-76.904\\
-76.904	-74.463\\
-74.463	-101.318\\
-101.318	-128.174\\
-128.174	-137.939\\
-137.939	-140.381\\
-140.381	-103.76\\
-103.76	-108.643\\
-108.643	-111.084\\
-111.084	-112.305\\
-112.305	-126.953\\
-126.953	-169.678\\
-169.678	-144.043\\
-144.043	-131.836\\
-131.836	-104.98\\
-104.98	-101.318\\
-101.318	-96.436\\
-96.436	-114.746\\
-114.746	-72.021\\
-72.021	-83.008\\
-83.008	-102.539\\
-102.539	-119.629\\
-119.629	-91.553\\
-91.553	-106.201\\
-106.201	-85.449\\
-85.449	-75.684\\
-75.684	-50.049\\
-50.049	-74.463\\
-74.463	-122.07\\
-122.07	-96.436\\
-96.436	-56.152\\
-56.152	-69.58\\
-69.58	-41.504\\
-41.504	-35.4\\
-35.4	-50.049\\
-50.049	-31.738\\
-31.738	-35.4\\
-35.4	-42.725\\
-42.725	-28.076\\
-28.076	-43.945\\
-43.945	-40.283\\
-40.283	-24.414\\
-24.414	-39.063\\
-39.063	-46.387\\
-46.387	-53.711\\
-53.711	-53.711\\
-53.711	-53.711\\
-53.711	-74.463\\
-74.463	-64.697\\
-64.697	-72.021\\
-72.021	-124.512\\
-124.512	-89.111\\
-89.111	-76.904\\
-76.904	-85.449\\
-85.449	-61.035\\
-61.035	-37.842\\
-37.842	-65.918\\
-65.918	-57.373\\
-57.373	-32.959\\
-32.959	-58.594\\
-58.594	-54.932\\
-54.932	-67.139\\
-67.139	-45.166\\
-45.166	-28.076\\
-28.076	-23.193\\
-23.193	-26.855\\
-26.855	-48.828\\
-48.828	-48.828\\
-48.828	-59.814\\
-59.814	-76.904\\
-76.904	-102.539\\
-102.539	-76.904\\
-76.904	-102.539\\
-102.539	-122.07\\
-122.07	-104.98\\
-104.98	-98.877\\
-98.877	-162.354\\
-162.354	-118.408\\
-118.408	-142.822\\
-142.822	-123.291\\
-123.291	-145.264\\
-145.264	-162.354\\
-162.354	-96.436\\
-96.436	-58.594\\
-58.594	-48.828\\
-48.828	-74.463\\
-74.463	-93.994\\
-93.994	-95.215\\
-95.215	-65.918\\
-65.918	-36.621\\
-36.621	-46.387\\
-46.387	-92.773\\
-92.773	-125.732\\
-125.732	-124.512\\
-124.512	-74.463\\
-74.463	-57.373\\
-57.373	-76.904\\
-76.904	-122.07\\
-122.07	-136.719\\
-136.719	-139.16\\
-139.16	-97.656\\
-97.656	-137.939\\
-137.939	-152.588\\
-152.588	-142.822\\
-142.822	-197.754\\
-197.754	-169.678\\
-169.678	-141.602\\
-141.602	-164.795\\
-164.795	-140.381\\
-140.381	-78.125\\
-78.125	-72.021\\
-72.021	-89.111\\
-89.111	-108.643\\
-108.643	-111.084\\
-111.084	-81.787\\
-81.787	-102.539\\
-102.539	-91.553\\
-91.553	-65.918\\
-65.918	-89.111\\
-89.111	-84.229\\
-84.229	-124.512\\
-124.512	-128.174\\
-128.174	-91.553\\
-91.553	-70.801\\
-70.801	-42.725\\
-42.725	-64.697\\
-64.697	-54.932\\
-54.932	-46.387\\
-46.387	-39.063\\
-39.063	-65.918\\
-65.918	-86.67\\
-86.67	-95.215\\
-95.215	-95.215\\
-95.215	-57.373\\
-57.373	-42.725\\
-42.725	-29.297\\
-29.297	-37.842\\
-37.842	-63.477\\
-63.477	-65.918\\
-65.918	-64.697\\
-64.697	-102.539\\
-102.539	-122.07\\
-122.07	-134.277\\
-134.277	-135.498\\
-135.498	-157.471\\
-157.471	-97.656\\
-97.656	-80.566\\
-80.566	-61.035\\
-61.035	-67.139\\
-67.139	-90.332\\
-90.332	-75.684\\
-75.684	-51.27\\
-51.27	-46.387\\
-46.387	-83.008\\
-83.008	-102.539\\
-102.539	-90.332\\
-90.332	-147.705\\
-147.705	-161.133\\
-161.133	-112.305\\
-112.305	-79.346\\
-79.346	-54.932\\
-54.932	-50.049\\
-50.049	-90.332\\
-90.332	-76.904\\
-76.904	-86.67\\
-86.67	-98.877\\
-98.877	-107.422\\
-107.422	-103.76\\
-103.76	-117.188\\
-117.188	-81.787\\
-81.787	-89.111\\
-89.111	-65.918\\
-65.918	-58.594\\
-58.594	-90.332\\
-90.332	-125.732\\
-125.732	-135.498\\
-135.498	-80.566\\
-80.566	-155.029\\
-155.029	-145.264\\
-145.264	-96.436\\
-96.436	-56.152\\
-56.152	-83.008\\
-83.008	-75.684\\
-75.684	-70.801\\
-70.801	-85.449\\
-85.449	-58.594\\
-58.594	-79.346\\
-79.346	-145.264\\
-145.264	-151.367\\
-151.367	-98.877\\
-98.877	-108.643\\
-108.643	-122.07\\
-122.07	-123.291\\
-123.291	-90.332\\
-90.332	-87.891\\
-87.891	-61.035\\
-61.035	-89.111\\
-89.111	-95.215\\
-95.215	-146.484\\
-146.484	-166.016\\
-166.016	-228.271\\
-228.271	-163.574\\
-163.574	-131.836\\
-131.836	-103.76\\
-103.76	-109.863\\
-109.863	-85.449\\
-85.449	-59.814\\
-59.814	-57.373\\
-57.373	-59.814\\
-59.814	-46.387\\
-46.387	-32.959\\
-32.959	-51.27\\
-51.27	-67.139\\
-67.139	-47.607\\
-47.607	-34.18\\
-34.18	-32.959\\
-32.959	-76.904\\
-76.904	-109.863\\
-109.863	-52.49\\
-52.49	-62.256\\
-62.256	-68.359\\
-68.359	-61.035\\
-61.035	-76.904\\
-76.904	-67.139\\
-67.139	-47.607\\
-47.607	-34.18\\
-34.18	-29.297\\
-29.297	-39.063\\
-39.063	-73.242\\
-73.242	-87.891\\
-87.891	-62.256\\
-62.256	-43.945\\
-43.945	-29.297\\
-29.297	-41.504\\
-41.504	-64.697\\
-64.697	-84.229\\
-84.229	-89.111\\
-89.111	-63.477\\
-63.477	-59.814\\
-59.814	-67.139\\
-67.139	-91.553\\
-91.553	-126.953\\
-126.953	-146.484\\
-146.484	-111.084\\
-111.084	-69.58\\
-69.58	-75.684\\
-75.684	-69.58\\
-69.58	-72.021\\
-72.021	-50.049\\
-50.049	-59.814\\
-59.814	-67.139\\
-67.139	-63.477\\
-63.477	-41.504\\
-41.504	-29.297\\
-29.297	-41.504\\
-41.504	-63.477\\
-63.477	-93.994\\
-93.994	-100.098\\
-100.098	-122.07\\
-122.07	-109.863\\
-109.863	-123.291\\
-123.291	-163.574\\
-163.574	-217.285\\
-217.285	-179.443\\
-179.443	-119.629\\
-119.629	-130.615\\
-130.615	-153.809\\
-153.809	-197.754\\
-197.754	-128.174\\
-128.174	-62.256\\
-62.256	-43.945\\
-43.945	-45.166\\
-45.166	-32.959\\
-32.959	-21.973\\
-21.973	-28.076\\
-28.076	-42.725\\
-42.725	-61.035\\
-61.035	-59.814\\
-59.814	-47.607\\
-47.607	-45.166\\
-45.166	-56.152\\
-56.152	-69.58\\
-69.58	-73.242\\
-73.242	-69.58\\
-69.58	-48.828\\
-48.828	-35.4\\
-35.4	-34.18\\
-34.18	-50.049\\
-50.049	-63.477\\
-63.477	-46.387\\
-46.387	-32.959\\
-32.959	-52.49\\
-52.49	-40.283\\
-40.283	-43.945\\
-43.945	-56.152\\
-56.152	-79.346\\
-79.346	-86.67\\
-86.67	-59.814\\
-59.814	-90.332\\
-90.332	-87.891\\
-87.891	-69.58\\
-69.58	-61.035\\
-61.035	-97.656\\
-97.656	-123.291\\
-123.291	-119.629\\
-119.629	-133.057\\
-133.057	-102.539\\
-102.539	-92.773\\
-92.773	-87.891\\
-87.891	-118.408\\
-118.408	-86.67\\
-86.67	-120.85\\
-120.85	-115.967\\
-115.967	-118.408\\
-118.408	-203.857\\
-203.857	-161.133\\
-161.133	-85.449\\
-85.449	-58.594\\
-58.594	-62.256\\
-62.256	-79.346\\
-79.346	-101.318\\
-101.318	-114.746\\
-114.746	-128.174\\
-128.174	-131.836\\
-131.836	-86.67\\
-86.67	-76.904\\
-76.904	-89.111\\
-89.111	-89.111\\
-89.111	-56.152\\
-56.152	-40.283\\
-40.283	-70.801\\
-70.801	-70.801\\
-70.801	-93.994\\
-93.994	-115.967\\
-115.967	-107.422\\
-107.422	-74.463\\
-74.463	-61.035\\
-61.035	-90.332\\
-90.332	-118.408\\
-118.408	-153.809\\
-153.809	-169.678\\
-169.678	-194.092\\
-194.092	-157.471\\
-157.471	-141.602\\
-141.602	-84.229\\
-84.229	-62.256\\
-62.256	-104.98\\
-104.98	-139.16\\
-139.16	-81.787\\
-81.787	-125.732\\
-125.732	-172.119\\
-172.119	-206.299\\
-206.299	-129.395\\
-129.395	-74.463\\
-74.463	-42.725\\
-42.725	-64.697\\
-64.697	-59.814\\
-59.814	-62.256\\
-62.256	-37.842\\
-37.842	-52.49\\
-52.49	-41.504\\
-41.504	-52.49\\
-52.49	-43.945\\
-43.945	-54.932\\
-54.932	-85.449\\
-85.449	-79.346\\
-79.346	-112.305\\
-112.305	-133.057\\
-133.057	-133.057\\
-133.057	-74.463\\
-74.463	-76.904\\
-76.904	-91.553\\
-91.553	-62.256\\
-62.256	-56.152\\
-56.152	-42.725\\
-42.725	-63.477\\
-63.477	-59.814\\
-59.814	-34.18\\
-34.18	-24.414\\
-24.414	-24.414\\
-24.414	-41.504\\
-41.504	-61.035\\
-61.035	-65.918\\
-65.918	-41.504\\
-41.504	-51.27\\
-51.27	-73.242\\
-73.242	-93.994\\
-93.994	-65.918\\
-65.918	-62.256\\
-62.256	-74.463\\
-74.463	-87.891\\
-87.891	-80.566\\
-80.566	-92.773\\
-92.773	-86.67\\
-86.67	-62.256\\
-62.256	-48.828\\
-48.828	-62.256\\
-62.256	-53.711\\
-53.711	-64.697\\
-64.697	-87.891\\
-87.891	-139.16\\
-139.16	-98.877\\
-98.877	-95.215\\
-95.215	-140.381\\
-140.381	-108.643\\
-108.643	-67.139\\
-67.139	-48.828\\
-48.828	-40.283\\
-40.283	-42.725\\
-42.725	-35.4\\
-35.4	-36.621\\
-36.621	-67.139\\
-67.139	-35.4\\
-35.4	-34.18\\
-34.18	-47.607\\
-47.607	-65.918\\
-65.918	-50.049\\
-50.049	-37.842\\
-37.842	-23.193\\
-23.193	-39.063\\
-39.063	-72.021\\
-72.021	-96.436\\
-96.436	-79.346\\
-79.346	-63.477\\
-63.477	-72.021\\
-72.021	-72.021\\
-72.021	-80.566\\
-80.566	-98.877\\
-98.877	-117.188\\
-117.188	-114.746\\
-114.746	-95.215\\
-95.215	-64.697\\
-64.697	-52.49\\
-52.49	-56.152\\
-56.152	-76.904\\
-76.904	-47.607\\
-47.607	-43.945\\
-43.945	-80.566\\
-80.566	-96.436\\
-96.436	-91.553\\
-91.553	-104.98\\
-104.98	-137.939\\
-137.939	-74.463\\
-74.463	-139.16\\
-139.16	-129.395\\
-129.395	-120.85\\
-120.85	-128.174\\
-128.174	-125.732\\
-125.732	-212.402\\
-212.402	-198.975\\
-198.975	-134.277\\
-134.277	-159.912\\
-159.912	-194.092\\
-194.092	-190.43\\
-190.43	-212.402\\
-212.402	-194.092\\
-194.092	-253.906\\
-253.906	-195.313\\
-195.313	-135.498\\
-135.498	-81.787\\
-81.787	-84.229\\
-84.229	-68.359\\
-68.359	-57.373\\
-57.373	-85.449\\
-85.449	-120.85\\
-120.85	-76.904\\
-76.904	-67.139\\
-67.139	-65.918\\
-65.918	-46.387\\
-46.387	-56.152\\
-56.152	-29.297\\
-29.297	-39.063\\
-39.063	-29.297\\
-29.297	-46.387\\
-46.387	-32.959\\
-32.959	-30.518\\
-30.518	-19.531\\
-19.531	-18.311\\
-18.311	-34.18\\
-34.18	-25.635\\
-25.635	-28.076\\
-28.076	-57.373\\
-57.373	-78.125\\
-78.125	-80.566\\
-80.566	-58.594\\
-58.594	-74.463\\
-74.463	-114.746\\
-114.746	-53.711\\
-53.711	-37.842\\
-37.842	-28.076\\
-28.076	-20.752\\
-20.752	-34.18\\
-34.18	-56.152\\
-56.152	-84.229\\
-84.229	-50.049\\
-50.049	-85.449\\
-85.449	-118.408\\
-118.408	-120.85\\
-120.85	-89.111\\
-89.111	-57.373\\
-57.373	-72.021\\
-72.021	-96.436\\
-96.436	-124.512\\
-124.512	-68.359\\
-68.359	-37.842\\
-37.842	-57.373\\
-57.373	-47.607\\
-47.607	-23.193\\
-23.193	-18.311\\
-18.311	-39.063\\
-39.063	-85.449\\
-85.449	-85.449\\
-85.449	-63.477\\
-63.477	-41.504\\
-41.504	-43.945\\
-43.945	-15.869\\
-15.869	-30.518\\
-30.518	-25.635\\
-25.635	-42.725\\
-42.725	-58.594\\
-58.594	-89.111\\
-89.111	-93.994\\
-93.994	-100.098\\
-100.098	-104.98\\
-104.98	-102.539\\
-102.539	-100.098\\
-100.098	-86.67\\
-86.67	-104.98\\
-104.98	-157.471\\
-157.471	-144.043\\
-144.043	-74.463\\
-74.463	-40.283\\
-40.283	-86.67\\
-86.67	-106.201\\
-106.201	-96.436\\
-96.436	-56.152\\
-56.152	-57.373\\
-57.373	-97.656\\
-97.656	-146.484\\
-146.484	-150.146\\
-150.146	-169.678\\
-169.678	-164.795\\
-164.795	-120.85\\
-120.85	-123.291\\
-123.291	-58.594\\
-58.594	-56.152\\
-56.152	-72.021\\
-72.021	-87.891\\
-87.891	-91.553\\
-91.553	-96.436\\
-96.436	-63.477\\
-63.477	-86.67\\
-86.67	-69.58\\
-69.58	-41.504\\
-41.504	-29.297\\
-29.297	-24.414\\
-24.414	-37.842\\
-37.842	-30.518\\
-30.518	-18.311\\
-18.311	-39.063\\
-39.063	-74.463\\
-74.463	-48.828\\
-48.828	-45.166\\
-45.166	-61.035\\
-61.035	-100.098\\
-100.098	-68.359\\
-68.359	-37.842\\
-37.842	-24.414\\
-24.414	-53.711\\
-53.711	-108.643\\
-108.643	-136.719\\
-136.719	-189.209\\
-189.209	-223.389\\
-223.389	-220.947\\
-220.947	-142.822\\
-142.822	-147.705\\
-147.705	-191.65\\
-191.65	-159.912\\
-159.912	-147.705\\
-147.705	-168.457\\
-168.457	-183.105\\
-183.105	-185.547\\
-185.547	-253.906\\
-253.906	-168.457\\
-168.457	-95.215\\
-95.215	-56.152\\
-56.152	-45.166\\
-45.166	-50.049\\
-50.049	-48.828\\
-48.828	-48.828\\
-48.828	-86.67\\
-86.67	-65.918\\
-65.918	-73.242\\
-73.242	-92.773\\
-92.773	-53.711\\
-53.711	-64.697\\
-64.697	-84.229\\
-84.229	-98.877\\
-98.877	-56.152\\
-56.152	-39.063\\
-39.063	-50.049\\
-50.049	-48.828\\
-48.828	-113.525\\
-113.525	-155.029\\
-155.029	-145.264\\
-145.264	-168.457\\
-168.457	-189.209\\
-189.209	-246.582\\
-246.582	-208.74\\
-208.74	-135.498\\
-135.498	-89.111\\
-89.111	-51.27\\
-51.27	-58.594\\
-58.594	-58.594\\
-58.594	-76.904\\
-76.904	-53.711\\
-53.711	-50.049\\
-50.049	-53.711\\
-53.711	-69.58\\
-69.58	-76.904\\
-76.904	-95.215\\
-95.215	-68.359\\
-68.359	-32.959\\
-32.959	-25.635\\
-25.635	-23.193\\
-23.193	-25.635\\
-25.635	-30.518\\
-30.518	-53.711\\
-53.711	-52.49\\
-52.49	-65.918\\
-65.918	-95.215\\
-95.215	-68.359\\
-68.359	-113.525\\
-113.525	-164.795\\
-164.795	-129.395\\
-129.395	-115.967\\
-115.967	-140.381\\
-140.381	-162.354\\
-162.354	-108.643\\
-108.643	-93.994\\
-93.994	-59.814\\
-59.814	-65.918\\
-65.918	-131.836\\
-131.836	-214.844\\
-214.844	-256.348\\
-256.348	-202.637\\
-202.637	-168.457\\
-168.457	-123.291\\
-123.291	-133.057\\
-133.057	-175.781\\
-175.781	-93.994\\
-93.994	-83.008\\
-83.008	-63.477\\
-63.477	-115.967\\
-115.967	-112.305\\
-112.305	-61.035\\
-61.035	-62.256\\
-62.256	-46.387\\
-46.387	-84.229\\
-84.229	-119.629\\
-119.629	-126.953\\
-126.953	-95.215\\
-95.215	-69.58\\
-69.58	-80.566\\
-80.566	-61.035\\
-61.035	-46.387\\
-46.387	-36.621\\
-36.621	-32.959\\
-32.959	-29.297\\
-29.297	-34.18\\
-34.18	-59.814\\
-59.814	-84.229\\
-84.229	-83.008\\
-83.008	-52.49\\
-52.49	-47.607\\
-47.607	-40.283\\
-40.283	-52.49\\
-52.49	-68.359\\
-68.359	-73.242\\
-73.242	-69.58\\
-69.58	-40.283\\
-40.283	-83.008\\
-83.008	-120.85\\
-120.85	-84.229\\
-84.229	-35.4\\
-35.4	-40.283\\
-40.283	-67.139\\
-67.139	-67.139\\
-67.139	-58.594\\
-58.594	-37.842\\
-37.842	-68.359\\
-68.359	-97.656\\
-97.656	-59.814\\
-59.814	-24.414\\
-24.414	-37.842\\
-37.842	-81.787\\
-81.787	-96.436\\
-96.436	-59.814\\
-59.814	-53.711\\
-53.711	-97.656\\
-97.656	-125.732\\
-125.732	-108.643\\
-108.643	-152.588\\
-152.588	-184.326\\
-184.326	-115.967\\
-115.967	-95.215\\
-95.215	-139.16\\
-139.16	-93.994\\
-93.994	-126.953\\
-126.953	-152.588\\
-152.588	-119.629\\
-119.629	-57.373\\
-57.373	-101.318\\
-101.318	-172.119\\
-172.119	-195.313\\
-195.313	-142.822\\
-142.822	-120.85\\
-120.85	-151.367\\
-151.367	-118.408\\
-118.408	-76.904\\
-76.904	-73.242\\
-73.242	-91.553\\
-91.553	-95.215\\
-95.215	-93.994\\
-93.994	-104.98\\
-104.98	-72.021\\
-72.021	-79.346\\
-79.346	-109.863\\
-109.863	-64.697\\
-64.697	-53.711\\
-53.711	-43.945\\
-43.945	-35.4\\
-35.4	-42.725\\
-42.725	-73.242\\
-73.242	-65.918\\
-65.918	-93.994\\
-93.994	-114.746\\
-114.746	-139.16\\
-139.16	-102.539\\
-102.539	-92.773\\
-92.773	-43.945\\
-43.945	-56.152\\
-56.152	-107.422\\
-107.422	-73.242\\
-73.242	-42.725\\
-42.725	-78.125\\
-78.125	-115.967\\
-115.967	-120.85\\
-120.85	-157.471\\
-157.471	-205.078\\
-205.078	-136.719\\
-136.719	-118.408\\
-118.408	-142.822\\
-142.822	-92.773\\
-92.773	-76.904\\
-76.904	-85.449\\
-85.449	-111.084\\
-111.084	-118.408\\
-118.408	-119.629\\
-119.629	-67.139\\
-67.139	-59.814\\
-59.814	-52.49\\
-52.49	-73.242\\
-73.242	-65.918\\
-65.918	-52.49\\
-52.49	-104.98\\
-104.98	-111.084\\
-111.084	-78.125\\
-78.125	-64.697\\
-64.697	-96.436\\
-96.436	-53.711\\
-53.711	-32.959\\
-32.959	-41.504\\
-41.504	-57.373\\
-57.373	-48.828\\
-48.828	-35.4\\
-35.4	-21.973\\
-21.973	-40.283\\
-40.283	-54.932\\
-54.932	-86.67\\
-86.67	-96.436\\
-96.436	-126.953\\
-126.953	-142.822\\
-142.822	-81.787\\
-81.787	-62.256\\
-62.256	-128.174\\
-128.174	-181.885\\
-181.885	-139.16\\
-139.16	-130.615\\
-130.615	-196.533\\
-196.533	-155.029\\
-155.029	-128.174\\
-128.174	-93.994\\
-93.994	-98.877\\
-98.877	-130.615\\
-130.615	-85.449\\
-85.449	-74.463\\
-74.463	-128.174\\
-128.174	-158.691\\
-158.691	-169.678\\
-169.678	-73.242\\
-73.242	-39.063\\
-39.063	-93.994\\
-93.994	-73.242\\
-73.242	-35.4\\
-35.4	-29.297\\
-29.297	-50.049\\
-50.049	-67.139\\
-67.139	-109.863\\
-109.863	-76.904\\
-76.904	-61.035\\
-61.035	-36.621\\
-36.621	-26.855\\
-26.855	-36.621\\
-36.621	-30.518\\
-30.518	-40.283\\
-40.283	-68.359\\
-68.359	-61.035\\
-61.035	-32.959\\
-32.959	-30.518\\
-30.518	-39.063\\
-39.063	-47.607\\
-47.607	-29.297\\
-29.297	-47.607\\
-47.607	-32.959\\
-32.959	-24.414\\
-24.414	-56.152\\
-56.152	-78.125\\
-78.125	-115.967\\
-115.967	-156.25\\
-156.25	-137.939\\
-137.939	-67.139\\
-67.139	-51.27\\
-51.27	-51.27\\
-51.27	-29.297\\
-29.297	-30.518\\
-30.518	-57.373\\
-57.373	-62.256\\
-62.256	-57.373\\
-57.373	-65.918\\
-65.918	-76.904\\
-76.904	-80.566\\
-80.566	-83.008\\
-83.008	-103.76\\
-103.76	-100.098\\
-100.098	-145.264\\
-145.264	-76.904\\
-76.904	-37.842\\
-37.842	-36.621\\
-36.621	-67.139\\
-67.139	-50.049\\
-50.049	-47.607\\
-47.607	-23.193\\
-23.193	-41.504\\
-41.504	-25.635\\
-25.635	-15.869\\
-15.869	-25.635\\
-25.635	-47.607\\
-47.607	-58.594\\
-58.594	-84.229\\
-84.229	-118.408\\
-118.408	-84.229\\
-84.229	-35.4\\
-35.4	-65.918\\
-65.918	-74.463\\
-74.463	-39.063\\
-39.063	-51.27\\
-51.27	-92.773\\
-92.773	-81.787\\
-81.787	-137.939\\
-137.939	-157.471\\
-157.471	-107.422\\
-107.422	-83.008\\
-83.008	-74.463\\
};
\addplot [color=mycolor2, line width=2.0pt, forget plot]
  table[row sep=crcr]{%
-84.229	-80.7789024124198\\
-96.436	-92.4858924247482\\
-122.07	-117.069900123284\\
-100.098	-95.997893524539\\
-93.994	-90.1439189988363\\
-130.615	-125.264889035821\\
-125.732	-120.581901223075\\
-153.809	-147.508841386599\\
-115.967	-111.216884636658\\
-61.035	-58.534950061642\\
-74.463	-71.4129267869263\\
-73.242	-70.2419400739704\\
-86.67	-83.1199167992547\\
-73.242	-70.2419400739704\\
-34.18	-32.7799556501503\\
-20.752	-19.901978924866\\
-23.193	-22.2429933117009\\
-58.594	-56.1939356748071\\
-100.098	-95.997893524539\\
-109.863	-105.362910110956\\
-114.746	-110.045897923702\\
-83.008	-79.6079156994638\\
-52.49	-50.3399611491044\\
-104.98	-100.679922298209\\
-81.787	-78.4369289865079\\
-59.814	-57.363963348686\\
-97.656	-93.6559200986271\\
-83.008	-79.6079156994638\\
-72.021	-69.0709533610144\\
-118.408	-113.557899023493\\
-107.422	-103.021895724121\\
-83.008	-79.6079156994638\\
-119.629	-114.728885736449\\
-123.291	-118.24088683624\\
-95.215	-91.3149057117922\\
-80.566	-77.265942273552\\
-62.256	-59.7059367745979\\
-75.684	-72.5839134998822\\
-95.215	-91.3149057117922\\
-87.891	-84.2909035122106\\
-91.553	-87.8029046120014\\
-96.436	-92.4858924247482\\
-102.539	-98.3389079113739\\
-141.602	-135.801851374271\\
-128.174	-122.923874648987\\
-85.449	-81.9489300862987\\
-79.346	-76.095914599673\\
-47.607	-45.6569733363576\\
-48.828	-46.8279600493136\\
-68.359	-65.5589522612236\\
-52.49	-50.3399611491044\\
-57.373	-55.0229489618511\\
-75.684	-72.5839134998822\\
-89.111	-85.4609311860896\\
-81.787	-78.4369289865079\\
-128.174	-122.923874648987\\
-98.877	-94.826906811583\\
-57.373	-55.0229489618511\\
-45.166	-43.3159589495227\\
-62.256	-59.7059367745979\\
-72.021	-69.0709533610144\\
-40.283	-38.632971136776\\
-42.725	-40.9749445626878\\
-56.152	-53.8519622488952\\
-46.387	-44.4869456624787\\
-40.283	-38.632971136776\\
-52.49	-50.3399611491044\\
-62.256	-59.7059367745979\\
-92.773	-88.9729322858804\\
-90.332	-86.6319178990455\\
-107.422	-103.021895724121\\
-98.877	-94.826906811583\\
-115.967	-111.216884636658\\
-91.553	-87.8029046120014\\
-79.346	-76.095914599673\\
-75.684	-72.5839134998822\\
-76.904	-73.7539411737612\\
-133.057	-127.606862461733\\
-189.209	-181.458824710629\\
-194.092	-186.141812523375\\
-189.209	-181.458824710629\\
-119.629	-114.728885736449\\
-170.898	-163.897860172597\\
-213.623	-204.872804735285\\
-219.727	-210.726779260988\\
-169.678	-162.727832498718\\
-219.727	-210.726779260988\\
-279.541	-268.090742609674\\
-197.754	-189.653813623166\\
-161.133	-154.532843586181\\
-109.863	-105.362910110956\\
-85.449	-81.9489300862987\\
-73.242	-70.2419400739704\\
-91.553	-87.8029046120014\\
-62.256	-59.7059367745979\\
-46.387	-44.4869456624787\\
-42.725	-40.9749445626878\\
-54.932	-52.6819345750162\\
-79.346	-76.095914599673\\
-74.463	-71.4129267869263\\
-75.684	-72.5839134998822\\
-100.098	-95.997893524539\\
-103.76	-99.5098946243298\\
-141.602	-135.801851374271\\
-114.746	-110.045897923702\\
-142.822	-136.97187904815\\
-104.98	-100.679922298209\\
-90.332	-86.6319178990455\\
-103.76	-99.5098946243298\\
-76.904	-73.7539411737612\\
-69.58	-66.7299389741795\\
-84.229	-80.7789024124198\\
-86.67	-83.1199167992547\\
-59.814	-57.363963348686\\
-64.697	-62.0469511614328\\
-48.828	-46.8279600493136\\
-53.711	-51.5109478620603\\
-57.373	-55.0229489618511\\
-41.504	-39.8039578497319\\
-52.49	-50.3399611491044\\
-65.918	-63.2179378743887\\
-101.318	-97.167921198418\\
-104.98	-100.679922298209\\
-114.746	-110.045897923702\\
-68.359	-65.5589522612236\\
-93.994	-90.1439189988363\\
-150.146	-143.995881247732\\
-114.746	-110.045897923702\\
-125.732	-120.581901223075\\
-128.174	-122.923874648987\\
-80.566	-77.265942273552\\
-53.711	-51.5109478620603\\
-75.684	-72.5839134998822\\
-85.449	-81.9489300862987\\
-137.939	-132.288891235403\\
-172.119	-165.068846885553\\
-170.898	-163.897860172597\\
-123.291	-118.24088683624\\
-130.615	-125.264889035821\\
-115.967	-111.216884636658\\
-113.525	-108.874911210746\\
-111.084	-106.533896823911\\
-76.904	-73.7539411737612\\
-68.359	-65.5589522612236\\
-61.035	-58.534950061642\\
-57.373	-55.0229489618511\\
-62.256	-59.7059367745979\\
-91.553	-87.8029046120014\\
-140.381	-134.630864661315\\
-111.084	-106.533896823911\\
-79.346	-76.095914599673\\
-64.697	-62.0469511614328\\
-62.256	-59.7059367745979\\
-40.283	-38.632971136776\\
-29.297	-28.0969678374035\\
-36.621	-35.1209700369852\\
-63.477	-60.8769234875538\\
-51.27	-49.1699334752254\\
-52.49	-50.3399611491044\\
-58.594	-56.1939356748071\\
-54.932	-52.6819345750162\\
-37.842	-36.2919567499411\\
-28.076	-26.9259811244476\\
-23.193	-22.2429933117009\\
-39.063	-37.462943462897\\
-83.008	-79.6079156994638\\
-117.188	-112.387871349614\\
-130.615	-125.264889035821\\
-86.67	-83.1199167992547\\
-58.594	-56.1939356748071\\
-45.166	-43.3159589495227\\
-32.959	-31.6089689371944\\
-67.139	-64.3889245873446\\
-65.918	-63.2179378743887\\
-91.553	-87.8029046120014\\
-117.188	-112.387871349614\\
-186.768	-179.117810323794\\
-216.064	-207.21381912212\\
-177.002	-169.7518346983\\
-168.457	-161.556845785763\\
-109.863	-105.362910110956\\
-122.07	-117.069900123284\\
-131.836	-126.435875748777\\
-146.484	-140.483880147941\\
-108.643	-104.192882437077\\
-100.098	-95.997893524539\\
-98.877	-94.826906811583\\
-89.111	-85.4609311860896\\
-106.201	-101.850909011165\\
-78.125	-74.9249278867171\\
-130.615	-125.264889035821\\
-159.912	-153.361856873225\\
-113.525	-108.874911210746\\
-70.801	-67.9009256871355\\
-73.242	-70.2419400739704\\
-123.291	-118.24088683624\\
-153.809	-147.508841386599\\
-189.209	-181.458824710629\\
-202.637	-194.336801435913\\
-153.809	-147.508841386599\\
-137.939	-132.288891235403\\
-158.691	-152.190870160269\\
-187.988	-180.287837997673\\
-109.863	-105.362910110956\\
-72.021	-69.0709533610144\\
-101.318	-97.167921198418\\
-86.67	-83.1199167992547\\
-53.711	-51.5109478620603\\
-74.463	-71.4129267869263\\
-84.229	-80.7789024124198\\
-67.139	-64.3889245873446\\
-51.27	-49.1699334752254\\
-70.801	-67.9009256871355\\
-58.594	-56.1939356748071\\
-79.346	-76.095914599673\\
-107.422	-103.021895724121\\
-100.098	-95.997893524539\\
-69.58	-66.7299389741795\\
-70.801	-67.9009256871355\\
-107.422	-103.021895724121\\
-177.002	-169.7518346983\\
-128.174	-122.923874648987\\
-79.346	-76.095914599673\\
-61.035	-58.534950061642\\
-70.801	-67.9009256871355\\
-64.697	-62.0469511614328\\
-48.828	-46.8279600493136\\
-53.711	-51.5109478620603\\
-65.918	-63.2179378743887\\
-83.008	-79.6079156994638\\
-91.553	-87.8029046120014\\
-63.477	-60.8769234875538\\
-104.98	-100.679922298209\\
-124.512	-119.411873549196\\
-118.408	-113.557899023493\\
-91.553	-87.8029046120014\\
-100.098	-95.997893524539\\
-135.498	-129.947876848568\\
-87.891	-84.2909035122106\\
-42.725	-40.9749445626878\\
-57.373	-55.0229489618511\\
-62.256	-59.7059367745979\\
-81.787	-78.4369289865079\\
-72.021	-69.0709533610144\\
-52.49	-50.3399611491044\\
-79.346	-76.095914599673\\
-80.566	-77.265942273552\\
-57.373	-55.0229489618511\\
-70.801	-67.9009256871355\\
-63.477	-60.8769234875538\\
-56.152	-53.8519622488952\\
-67.139	-64.3889245873446\\
-41.504	-39.8039578497319\\
-48.828	-46.8279600493136\\
-36.621	-35.1209700369852\\
-40.283	-38.632971136776\\
-75.684	-72.5839134998822\\
-84.229	-80.7789024124198\\
-113.525	-108.874911210746\\
-146.484	-140.483880147941\\
-98.877	-94.826906811583\\
-58.594	-56.1939356748071\\
-46.387	-44.4869456624787\\
-43.945	-42.1449722365668\\
-63.477	-60.8769234875538\\
-42.725	-40.9749445626878\\
-47.607	-45.6569733363576\\
-61.035	-58.534950061642\\
-102.539	-98.3389079113739\\
-93.994	-90.1439189988363\\
-134.277	-128.776890135612\\
-89.111	-85.4609311860896\\
-81.787	-78.4369289865079\\
-46.387	-44.4869456624787\\
-45.166	-43.3159589495227\\
-28.076	-26.9259811244476\\
-35.4	-33.9499833240292\\
-37.842	-36.2919567499411\\
-67.139	-64.3889245873446\\
-70.801	-67.9009256871355\\
-79.346	-76.095914599673\\
-85.449	-81.9489300862987\\
-125.732	-120.581901223075\\
-112.305	-107.704883536867\\
-84.229	-80.7789024124198\\
-102.539	-98.3389079113739\\
-109.863	-105.362910110956\\
-144.043	-138.142865761106\\
-115.967	-111.216884636658\\
-75.684	-72.5839134998822\\
-40.283	-38.632971136776\\
-29.297	-28.0969678374035\\
-36.621	-35.1209700369852\\
-39.063	-37.462943462897\\
-37.842	-36.2919567499411\\
-50.049	-47.9989467622695\\
-74.463	-71.4129267869263\\
-68.359	-65.5589522612236\\
-70.801	-67.9009256871355\\
-83.008	-79.6079156994638\\
-51.27	-49.1699334752254\\
-30.518	-29.2679545503595\\
-58.594	-56.1939356748071\\
-90.332	-86.6319178990455\\
-87.891	-84.2909035122106\\
-97.656	-93.6559200986271\\
-83.008	-79.6079156994638\\
-61.035	-58.534950061642\\
-85.449	-81.9489300862987\\
-118.408	-113.557899023493\\
-84.229	-80.7789024124198\\
-136.719	-131.118863561524\\
-100.098	-95.997893524539\\
-101.318	-97.167921198418\\
-117.188	-112.387871349614\\
-115.967	-111.216884636658\\
-86.67	-83.1199167992547\\
-76.904	-73.7539411737612\\
-128.174	-122.923874648987\\
-178.223	-170.922821411256\\
-139.16	-133.459877948359\\
-79.346	-76.095914599673\\
-76.904	-73.7539411737612\\
-79.346	-76.095914599673\\
-98.877	-94.826906811583\\
-114.746	-110.045897923702\\
-85.449	-81.9489300862987\\
-90.332	-86.6319178990455\\
-120.85	-115.899872449405\\
-131.836	-126.435875748777\\
-87.891	-84.2909035122106\\
-68.359	-65.5589522612236\\
-91.553	-87.8029046120014\\
-156.25	-149.849855773434\\
-137.939	-132.288891235403\\
-140.381	-134.630864661315\\
-84.229	-80.7789024124198\\
-76.904	-73.7539411737612\\
-72.021	-69.0709533610144\\
-47.607	-45.6569733363576\\
-30.518	-29.2679545503595\\
-26.855	-25.7549944114917\\
-23.193	-22.2429933117009\\
-54.932	-52.6819345750162\\
-70.801	-67.9009256871355\\
-87.891	-84.2909035122106\\
-65.918	-63.2179378743887\\
-73.242	-70.2419400739704\\
-83.008	-79.6079156994638\\
-53.711	-51.5109478620603\\
-56.152	-53.8519622488952\\
-53.711	-51.5109478620603\\
-40.283	-38.632971136776\\
-51.27	-49.1699334752254\\
-84.229	-80.7789024124198\\
-106.201	-101.850909011165\\
-63.477	-60.8769234875538\\
-48.828	-46.8279600493136\\
-40.283	-38.632971136776\\
-50.049	-47.9989467622695\\
-35.4	-33.9499833240292\\
-76.904	-73.7539411737612\\
-130.615	-125.264889035821\\
-104.98	-100.679922298209\\
-139.16	-133.459877948359\\
-162.354	-155.703830299137\\
-164.795	-158.044844685972\\
-124.512	-119.411873549196\\
-90.332	-86.6319178990455\\
-89.111	-85.4609311860896\\
-106.201	-101.850909011165\\
-125.732	-120.581901223075\\
-168.457	-161.556845785763\\
-183.105	-175.604850184926\\
-219.727	-210.726779260988\\
-147.705	-141.654866860897\\
-166.016	-159.215831398928\\
-184.326	-176.775836897882\\
-140.381	-134.630864661315\\
-162.354	-155.703830299137\\
-139.16	-133.459877948359\\
-80.566	-77.265942273552\\
-52.49	-50.3399611491044\\
-56.152	-53.8519622488952\\
-81.787	-78.4369289865079\\
-65.918	-63.2179378743887\\
-43.945	-42.1449722365668\\
-62.256	-59.7059367745979\\
-47.607	-45.6569733363576\\
-36.621	-35.1209700369852\\
-46.387	-44.4869456624787\\
-52.49	-50.3399611491044\\
-36.621	-35.1209700369852\\
-70.801	-67.9009256871355\\
-92.773	-88.9729322858804\\
-68.359	-65.5589522612236\\
-114.746	-110.045897923702\\
-164.795	-158.044844685972\\
-150.146	-143.995881247732\\
-196.533	-188.48282691021\\
-131.836	-126.435875748777\\
-144.043	-138.142865761106\\
-96.436	-92.4858924247482\\
-63.477	-60.8769234875538\\
-76.904	-73.7539411737612\\
-59.814	-57.363963348686\\
-45.166	-43.3159589495227\\
-64.697	-62.0469511614328\\
-36.621	-35.1209700369852\\
-28.076	-26.9259811244476\\
-34.18	-32.7799556501503\\
-70.801	-67.9009256871355\\
-86.67	-83.1199167992547\\
-107.422	-103.021895724121\\
-103.76	-99.5098946243298\\
-100.098	-95.997893524539\\
-114.746	-110.045897923702\\
-93.994	-90.1439189988363\\
-76.904	-73.7539411737612\\
-62.256	-59.7059367745979\\
-97.656	-93.6559200986271\\
-68.359	-65.5589522612236\\
-56.152	-53.8519622488952\\
-81.787	-78.4369289865079\\
-113.525	-108.874911210746\\
-86.67	-83.1199167992547\\
-65.918	-63.2179378743887\\
-56.152	-53.8519622488952\\
-65.918	-63.2179378743887\\
-45.166	-43.3159589495227\\
-59.814	-57.363963348686\\
-75.684	-72.5839134998822\\
-84.229	-80.7789024124198\\
-65.918	-63.2179378743887\\
-86.67	-83.1199167992547\\
-79.346	-76.095914599673\\
-47.607	-45.6569733363576\\
-50.049	-47.9989467622695\\
-95.215	-91.3149057117922\\
-131.836	-126.435875748777\\
-107.422	-103.021895724121\\
-103.76	-99.5098946243298\\
-79.346	-76.095914599673\\
-76.904	-73.7539411737612\\
-61.035	-58.534950061642\\
-89.111	-85.4609311860896\\
-75.684	-72.5839134998822\\
-50.049	-47.9989467622695\\
-74.463	-71.4129267869263\\
-85.449	-81.9489300862987\\
-101.318	-97.167921198418\\
-109.863	-105.362910110956\\
-148.926	-142.825853573853\\
-181.885	-174.434822511047\\
-205.078	-196.677815822748\\
-129.395	-124.094861361943\\
-73.242	-70.2419400739704\\
-47.607	-45.6569733363576\\
-34.18	-32.7799556501503\\
-54.932	-52.6819345750162\\
-46.387	-44.4869456624787\\
-80.566	-77.265942273552\\
-72.021	-69.0709533610144\\
-64.697	-62.0469511614328\\
-85.449	-81.9489300862987\\
-98.877	-94.826906811583\\
-117.188	-112.387871349614\\
-123.291	-118.24088683624\\
-175.781	-168.580847985344\\
-150.146	-143.995881247732\\
-117.188	-112.387871349614\\
-80.566	-77.265942273552\\
-83.008	-79.6079156994638\\
-106.201	-101.850909011165\\
-75.684	-72.5839134998822\\
-79.346	-76.095914599673\\
-74.463	-71.4129267869263\\
-42.725	-40.9749445626878\\
-74.463	-71.4129267869263\\
-107.422	-103.021895724121\\
-73.242	-70.2419400739704\\
-80.566	-77.265942273552\\
-147.705	-141.654866860897\\
-109.863	-105.362910110956\\
-106.201	-101.850909011165\\
-147.705	-141.654866860897\\
-140.381	-134.630864661315\\
-87.891	-84.2909035122106\\
-56.152	-53.8519622488952\\
-58.594	-56.1939356748071\\
-97.656	-93.6559200986271\\
-151.367	-145.166867960687\\
-162.354	-155.703830299137\\
-167.236	-160.385859072807\\
-129.395	-124.094861361943\\
-83.008	-79.6079156994638\\
-72.021	-69.0709533610144\\
-53.711	-51.5109478620603\\
-50.049	-47.9989467622695\\
-42.725	-40.9749445626878\\
-61.035	-58.534950061642\\
-72.021	-69.0709533610144\\
-41.504	-39.8039578497319\\
-65.918	-63.2179378743887\\
-89.111	-85.4609311860896\\
-101.318	-97.167921198418\\
-128.174	-122.923874648987\\
-161.133	-154.532843586181\\
-192.871	-184.970825810419\\
-136.719	-131.118863561524\\
-118.408	-113.557899023493\\
-124.512	-119.411873549196\\
-102.539	-98.3389079113739\\
-73.242	-70.2419400739704\\
-89.111	-85.4609311860896\\
-136.719	-131.118863561524\\
-162.354	-155.703830299137\\
-93.994	-90.1439189988363\\
-54.932	-52.6819345750162\\
-37.842	-36.2919567499411\\
-23.193	-22.2429933117009\\
-32.959	-31.6089689371944\\
-45.166	-43.3159589495227\\
-52.49	-50.3399611491044\\
-73.242	-70.2419400739704\\
-58.594	-56.1939356748071\\
-45.166	-43.3159589495227\\
-43.945	-42.1449722365668\\
-40.283	-38.632971136776\\
-47.607	-45.6569733363576\\
-70.801	-67.9009256871355\\
-104.98	-100.679922298209\\
-111.084	-106.533896823911\\
-107.422	-103.021895724121\\
-123.291	-118.24088683624\\
-152.588	-146.337854673643\\
-155.029	-148.678869060478\\
-183.105	-175.604850184926\\
-167.236	-160.385859072807\\
-124.512	-119.411873549196\\
-85.449	-81.9489300862987\\
-83.008	-79.6079156994638\\
-141.602	-135.801851374271\\
-158.691	-152.190870160269\\
-115.967	-111.216884636658\\
-76.904	-73.7539411737612\\
-39.063	-37.462943462897\\
-30.518	-29.2679545503595\\
-29.297	-28.0969678374035\\
-19.531	-18.73099221191\\
-45.166	-43.3159589495227\\
-57.373	-55.0229489618511\\
-62.256	-59.7059367745979\\
-53.711	-51.5109478620603\\
-34.18	-32.7799556501503\\
-26.855	-25.7549944114917\\
-36.621	-35.1209700369852\\
-41.504	-39.8039578497319\\
-43.945	-42.1449722365668\\
-35.4	-33.9499833240292\\
-28.076	-26.9259811244476\\
-50.049	-47.9989467622695\\
-112.305	-107.704883536867\\
-128.174	-122.923874648987\\
-145.264	-139.313852474062\\
-101.318	-97.167921198418\\
-133.057	-127.606862461733\\
-195.313	-187.312799236331\\
-174.561	-167.410820311465\\
-184.326	-176.775836897882\\
-135.498	-129.947876848568\\
-137.939	-132.288891235403\\
-145.264	-139.313852474062\\
-95.215	-91.3149057117922\\
-90.332	-86.6319178990455\\
-56.152	-53.8519622488952\\
-54.932	-52.6819345750162\\
-102.539	-98.3389079113739\\
-68.359	-65.5589522612236\\
-79.346	-76.095914599673\\
-84.229	-80.7789024124198\\
-79.346	-76.095914599673\\
-84.229	-80.7789024124198\\
-93.994	-90.1439189988363\\
-102.539	-98.3389079113739\\
-89.111	-85.4609311860896\\
-67.139	-64.3889245873446\\
-84.229	-80.7789024124198\\
-39.063	-37.462943462897\\
-20.752	-19.901978924866\\
-36.621	-35.1209700369852\\
-51.27	-49.1699334752254\\
-26.855	-25.7549944114917\\
-13.428	-12.8779767252843\\
-31.738	-30.4379822242384\\
-50.049	-47.9989467622695\\
-58.594	-56.1939356748071\\
-72.021	-69.0709533610144\\
-62.256	-59.7059367745979\\
-91.553	-87.8029046120014\\
-80.566	-77.265942273552\\
-57.373	-55.0229489618511\\
-86.67	-83.1199167992547\\
-48.828	-46.8279600493136\\
-41.504	-39.8039578497319\\
-28.076	-26.9259811244476\\
-21.973	-21.0729656378219\\
-37.842	-36.2919567499411\\
-28.076	-26.9259811244476\\
-50.049	-47.9989467622695\\
-75.684	-72.5839134998822\\
-73.242	-70.2419400739704\\
-54.932	-52.6819345750162\\
-61.035	-58.534950061642\\
-45.166	-43.3159589495227\\
-65.918	-63.2179378743887\\
-47.607	-45.6569733363576\\
-46.387	-44.4869456624787\\
-42.725	-40.9749445626878\\
-32.959	-31.6089689371944\\
-42.725	-40.9749445626878\\
-37.842	-36.2919567499411\\
-64.697	-62.0469511614328\\
-63.477	-60.8769234875538\\
-50.049	-47.9989467622695\\
-46.387	-44.4869456624787\\
-36.621	-35.1209700369852\\
-43.945	-42.1449722365668\\
-93.994	-90.1439189988363\\
-120.85	-115.899872449405\\
-81.787	-78.4369289865079\\
-80.566	-77.265942273552\\
-54.932	-52.6819345750162\\
-93.994	-90.1439189988363\\
-81.787	-78.4369289865079\\
-54.932	-52.6819345750162\\
-70.801	-67.9009256871355\\
-52.49	-50.3399611491044\\
-75.684	-72.5839134998822\\
-42.725	-40.9749445626878\\
-50.049	-47.9989467622695\\
-57.373	-55.0229489618511\\
-74.463	-71.4129267869263\\
-58.594	-56.1939356748071\\
-61.035	-58.534950061642\\
-95.215	-91.3149057117922\\
-79.346	-76.095914599673\\
-125.732	-120.581901223075\\
-157.471	-151.02084248639\\
-125.732	-120.581901223075\\
-81.787	-78.4369289865079\\
-62.256	-59.7059367745979\\
-89.111	-85.4609311860896\\
-112.305	-107.704883536867\\
-87.891	-84.2909035122106\\
-107.422	-103.021895724121\\
-64.697	-62.0469511614328\\
-34.18	-32.7799556501503\\
-37.842	-36.2919567499411\\
-84.229	-80.7789024124198\\
-70.801	-67.9009256871355\\
-69.58	-66.7299389741795\\
-104.98	-100.679922298209\\
-97.656	-93.6559200986271\\
-87.891	-84.2909035122106\\
-62.256	-59.7059367745979\\
-73.242	-70.2419400739704\\
-100.098	-95.997893524539\\
-81.787	-78.4369289865079\\
-87.891	-84.2909035122106\\
-78.125	-74.9249278867171\\
-58.594	-56.1939356748071\\
-67.139	-64.3889245873446\\
-48.828	-46.8279600493136\\
-56.152	-53.8519622488952\\
-95.215	-91.3149057117922\\
-109.863	-105.362910110956\\
-115.967	-111.216884636658\\
-102.539	-98.3389079113739\\
-73.242	-70.2419400739704\\
-51.27	-49.1699334752254\\
-62.256	-59.7059367745979\\
-70.801	-67.9009256871355\\
-91.553	-87.8029046120014\\
-111.084	-106.533896823911\\
-131.836	-126.435875748777\\
-107.422	-103.021895724121\\
-62.256	-59.7059367745979\\
-109.863	-105.362910110956\\
-151.367	-145.166867960687\\
-107.422	-103.021895724121\\
-108.643	-104.192882437077\\
-124.512	-119.411873549196\\
-93.994	-90.1439189988363\\
-79.346	-76.095914599673\\
-89.111	-85.4609311860896\\
-78.125	-74.9249278867171\\
-91.553	-87.8029046120014\\
-114.746	-110.045897923702\\
-86.67	-83.1199167992547\\
-102.539	-98.3389079113739\\
-92.773	-88.9729322858804\\
-96.436	-92.4858924247482\\
-76.904	-73.7539411737612\\
-100.098	-95.997893524539\\
-69.58	-66.7299389741795\\
-76.904	-73.7539411737612\\
-84.229	-80.7789024124198\\
-80.566	-77.265942273552\\
-41.504	-39.8039578497319\\
-28.076	-26.9259811244476\\
-21.973	-21.0729656378219\\
-37.842	-36.2919567499411\\
-84.229	-80.7789024124198\\
-108.643	-104.192882437077\\
-111.084	-106.533896823911\\
-74.463	-71.4129267869263\\
-86.67	-83.1199167992547\\
-72.021	-69.0709533610144\\
-65.918	-63.2179378743887\\
-42.725	-40.9749445626878\\
-61.035	-58.534950061642\\
-58.594	-56.1939356748071\\
-57.373	-55.0229489618511\\
-67.139	-64.3889245873446\\
-57.373	-55.0229489618511\\
-53.711	-51.5109478620603\\
-37.842	-36.2919567499411\\
-50.049	-47.9989467622695\\
-91.553	-87.8029046120014\\
-70.801	-67.9009256871355\\
-86.67	-83.1199167992547\\
-75.684	-72.5839134998822\\
-101.318	-97.167921198418\\
-92.773	-88.9729322858804\\
-129.395	-124.094861361943\\
-91.553	-87.8029046120014\\
-106.201	-101.850909011165\\
-103.76	-99.5098946243298\\
-54.932	-52.6819345750162\\
-96.436	-92.4858924247482\\
-67.139	-64.3889245873446\\
-84.229	-80.7789024124198\\
-90.332	-86.6319178990455\\
-115.967	-111.216884636658\\
-85.449	-81.9489300862987\\
-111.084	-106.533896823911\\
-136.719	-131.118863561524\\
-111.084	-106.533896823911\\
-95.215	-91.3149057117922\\
-75.684	-72.5839134998822\\
-91.553	-87.8029046120014\\
-80.566	-77.265942273552\\
-69.58	-66.7299389741795\\
-56.152	-53.8519622488952\\
-70.801	-67.9009256871355\\
-58.594	-56.1939356748071\\
-119.629	-114.728885736449\\
-151.367	-145.166867960687\\
-130.615	-125.264889035821\\
-126.953	-121.752887936031\\
-124.512	-119.411873549196\\
-69.58	-66.7299389741795\\
-63.477	-60.8769234875538\\
-50.049	-47.9989467622695\\
-42.725	-40.9749445626878\\
-48.828	-46.8279600493136\\
-80.566	-77.265942273552\\
-100.098	-95.997893524539\\
-114.746	-110.045897923702\\
-96.436	-92.4858924247482\\
-67.139	-64.3889245873446\\
-117.188	-112.387871349614\\
-137.939	-132.288891235403\\
-152.588	-146.337854673643\\
-118.408	-113.557899023493\\
-187.988	-180.287837997673\\
-133.057	-127.606862461733\\
-78.125	-74.9249278867171\\
-57.373	-55.0229489618511\\
-48.828	-46.8279600493136\\
-87.891	-84.2909035122106\\
-111.084	-106.533896823911\\
-147.705	-141.654866860897\\
-109.863	-105.362910110956\\
-87.891	-84.2909035122106\\
-47.607	-45.6569733363576\\
-53.711	-51.5109478620603\\
-56.152	-53.8519622488952\\
-63.477	-60.8769234875538\\
-40.283	-38.632971136776\\
-52.49	-50.3399611491044\\
-40.283	-38.632971136776\\
-26.855	-25.7549944114917\\
-58.594	-56.1939356748071\\
-63.477	-60.8769234875538\\
-80.566	-77.265942273552\\
-72.021	-69.0709533610144\\
-76.904	-73.7539411737612\\
-98.877	-94.826906811583\\
-67.139	-64.3889245873446\\
-101.318	-97.167921198418\\
-117.188	-112.387871349614\\
-89.111	-85.4609311860896\\
-93.994	-90.1439189988363\\
-80.566	-77.265942273552\\
-93.994	-90.1439189988363\\
-133.057	-127.606862461733\\
-100.098	-95.997893524539\\
-79.346	-76.095914599673\\
-90.332	-86.6319178990455\\
-81.787	-78.4369289865079\\
-56.152	-53.8519622488952\\
-85.449	-81.9489300862987\\
-123.291	-118.24088683624\\
-117.188	-112.387871349614\\
-130.615	-125.264889035821\\
-192.871	-184.970825810419\\
-125.732	-120.581901223075\\
-107.422	-103.021895724121\\
-83.008	-79.6079156994638\\
-106.201	-101.850909011165\\
-152.588	-146.337854673643\\
-101.318	-97.167921198418\\
-90.332	-86.6319178990455\\
-124.512	-119.411873549196\\
-79.346	-76.095914599673\\
-47.607	-45.6569733363576\\
-62.256	-59.7059367745979\\
-46.387	-44.4869456624787\\
-39.063	-37.462943462897\\
-42.725	-40.9749445626878\\
-37.842	-36.2919567499411\\
-47.607	-45.6569733363576\\
-65.918	-63.2179378743887\\
-78.125	-74.9249278867171\\
-102.539	-98.3389079113739\\
-96.436	-92.4858924247482\\
-112.305	-107.704883536867\\
-130.615	-125.264889035821\\
-119.629	-114.728885736449\\
-86.67	-83.1199167992547\\
-92.773	-88.9729322858804\\
-83.008	-79.6079156994638\\
-91.553	-87.8029046120014\\
-126.953	-121.752887936031\\
-93.994	-90.1439189988363\\
-106.201	-101.850909011165\\
-91.553	-87.8029046120014\\
-87.891	-84.2909035122106\\
-140.381	-134.630864661315\\
-113.525	-108.874911210746\\
-115.967	-111.216884636658\\
-103.76	-99.5098946243298\\
-64.697	-62.0469511614328\\
-42.725	-40.9749445626878\\
-34.18	-32.7799556501503\\
-61.035	-58.534950061642\\
-54.932	-52.6819345750162\\
-74.463	-71.4129267869263\\
-56.152	-53.8519622488952\\
-79.346	-76.095914599673\\
-124.512	-119.411873549196\\
-155.029	-148.678869060478\\
-122.07	-117.069900123284\\
-128.174	-122.923874648987\\
-113.525	-108.874911210746\\
-83.008	-79.6079156994638\\
-87.891	-84.2909035122106\\
-98.877	-94.826906811583\\
-84.229	-80.7789024124198\\
-101.318	-97.167921198418\\
-126.953	-121.752887936031\\
-178.223	-170.922821411256\\
-156.25	-149.849855773434\\
-147.705	-141.654866860897\\
-164.795	-158.044844685972\\
-109.863	-105.362910110956\\
-120.85	-115.899872449405\\
-93.994	-90.1439189988363\\
-95.215	-91.3149057117922\\
-126.953	-121.752887936031\\
-144.043	-138.142865761106\\
-56.152	-53.8519622488952\\
-74.463	-71.4129267869263\\
-57.373	-55.0229489618511\\
-83.008	-79.6079156994638\\
-84.229	-80.7789024124198\\
-51.27	-49.1699334752254\\
-68.359	-65.5589522612236\\
-119.629	-114.728885736449\\
-196.533	-188.48282691021\\
-190.43	-182.629811423584\\
-174.561	-167.410820311465\\
-137.939	-132.288891235403\\
-161.133	-154.532843586181\\
-196.533	-188.48282691021\\
-144.043	-138.142865761106\\
-148.926	-142.825853573853\\
-153.809	-147.508841386599\\
-104.98	-100.679922298209\\
-91.553	-87.8029046120014\\
-118.408	-113.557899023493\\
-79.346	-76.095914599673\\
-59.814	-57.363963348686\\
-80.566	-77.265942273552\\
-59.814	-57.363963348686\\
-73.242	-70.2419400739704\\
-67.139	-64.3889245873446\\
-84.229	-80.7789024124198\\
-111.084	-106.533896823911\\
-85.449	-81.9489300862987\\
-64.697	-62.0469511614328\\
-76.904	-73.7539411737612\\
-89.111	-85.4609311860896\\
-107.422	-103.021895724121\\
-89.111	-85.4609311860896\\
-74.463	-71.4129267869263\\
-115.967	-111.216884636658\\
-108.643	-104.192882437077\\
-75.684	-72.5839134998822\\
-126.953	-121.752887936031\\
-113.525	-108.874911210746\\
-83.008	-79.6079156994638\\
-57.373	-55.0229489618511\\
-42.725	-40.9749445626878\\
-31.738	-30.4379822242384\\
-69.58	-66.7299389741795\\
-63.477	-60.8769234875538\\
-47.607	-45.6569733363576\\
-34.18	-32.7799556501503\\
-42.725	-40.9749445626878\\
-70.801	-67.9009256871355\\
-81.787	-78.4369289865079\\
-104.98	-100.679922298209\\
-122.07	-117.069900123284\\
-117.188	-112.387871349614\\
-122.07	-117.069900123284\\
-72.021	-69.0709533610144\\
-34.18	-32.7799556501503\\
-50.049	-47.9989467622695\\
-41.504	-39.8039578497319\\
-53.711	-51.5109478620603\\
-86.67	-83.1199167992547\\
-95.215	-91.3149057117922\\
-141.602	-135.801851374271\\
-92.773	-88.9729322858804\\
-87.891	-84.2909035122106\\
-128.174	-122.923874648987\\
-93.994	-90.1439189988363\\
-65.918	-63.2179378743887\\
-50.049	-47.9989467622695\\
-65.918	-63.2179378743887\\
-87.891	-84.2909035122106\\
-129.395	-124.094861361943\\
-86.67	-83.1199167992547\\
-74.463	-71.4129267869263\\
-72.021	-69.0709533610144\\
-86.67	-83.1199167992547\\
-113.525	-108.874911210746\\
-179.443	-172.092849085135\\
-151.367	-145.166867960687\\
-97.656	-93.6559200986271\\
-64.697	-62.0469511614328\\
-76.904	-73.7539411737612\\
-36.621	-35.1209700369852\\
-54.932	-52.6819345750162\\
-51.27	-49.1699334752254\\
-54.932	-52.6819345750162\\
-93.994	-90.1439189988363\\
-128.174	-122.923874648987\\
-146.484	-140.483880147941\\
-189.209	-181.458824710629\\
-161.133	-154.532843586181\\
-86.67	-83.1199167992547\\
-81.787	-78.4369289865079\\
-68.359	-65.5589522612236\\
-91.553	-87.8029046120014\\
-120.85	-115.899872449405\\
-161.133	-154.532843586181\\
-113.525	-108.874911210746\\
-80.566	-77.265942273552\\
-92.773	-88.9729322858804\\
-124.512	-119.411873549196\\
-87.891	-84.2909035122106\\
-63.477	-60.8769234875538\\
-51.27	-49.1699334752254\\
-37.842	-36.2919567499411\\
-34.18	-32.7799556501503\\
-29.297	-28.0969678374035\\
-47.607	-45.6569733363576\\
-64.697	-62.0469511614328\\
-73.242	-70.2419400739704\\
-56.152	-53.8519622488952\\
-52.49	-50.3399611491044\\
-26.855	-25.7549944114917\\
-15.869	-15.2189911121192\\
-23.193	-22.2429933117009\\
-41.504	-39.8039578497319\\
-62.256	-59.7059367745979\\
-76.904	-73.7539411737612\\
-86.67	-83.1199167992547\\
-101.318	-97.167921198418\\
-131.836	-126.435875748777\\
-181.885	-174.434822511047\\
-179.443	-172.092849085135\\
-194.092	-186.141812523375\\
-169.678	-162.727832498718\\
-93.994	-90.1439189988363\\
-86.67	-83.1199167992547\\
-85.449	-81.9489300862987\\
-113.525	-108.874911210746\\
-96.436	-92.4858924247482\\
-70.801	-67.9009256871355\\
-56.152	-53.8519622488952\\
-46.387	-44.4869456624787\\
-76.904	-73.7539411737612\\
-42.725	-40.9749445626878\\
-32.959	-31.6089689371944\\
-50.049	-47.9989467622695\\
-67.139	-64.3889245873446\\
-51.27	-49.1699334752254\\
-73.242	-70.2419400739704\\
-54.932	-52.6819345750162\\
-81.787	-78.4369289865079\\
-122.07	-117.069900123284\\
-96.436	-92.4858924247482\\
-128.174	-122.923874648987\\
-177.002	-169.7518346983\\
-189.209	-181.458824710629\\
-148.926	-142.825853573853\\
-95.215	-91.3149057117922\\
-73.242	-70.2419400739704\\
-45.166	-43.3159589495227\\
-68.359	-65.5589522612236\\
-47.607	-45.6569733363576\\
-85.449	-81.9489300862987\\
-79.346	-76.095914599673\\
-62.256	-59.7059367745979\\
-81.787	-78.4369289865079\\
-47.607	-45.6569733363576\\
-69.58	-66.7299389741795\\
-54.932	-52.6819345750162\\
-34.18	-32.7799556501503\\
-39.063	-37.462943462897\\
-32.959	-31.6089689371944\\
-37.842	-36.2919567499411\\
-32.959	-31.6089689371944\\
-24.414	-23.4139800246568\\
-31.738	-30.4379822242384\\
-45.166	-43.3159589495227\\
-43.945	-42.1449722365668\\
-68.359	-65.5589522612236\\
-90.332	-86.6319178990455\\
-72.021	-69.0709533610144\\
-67.139	-64.3889245873446\\
-104.98	-100.679922298209\\
-125.732	-120.581901223075\\
-100.098	-95.997893524539\\
-104.98	-100.679922298209\\
-51.27	-49.1699334752254\\
-32.959	-31.6089689371944\\
-62.256	-59.7059367745979\\
-91.553	-87.8029046120014\\
-96.436	-92.4858924247482\\
-135.498	-129.947876848568\\
-120.85	-115.899872449405\\
-86.67	-83.1199167992547\\
-61.035	-58.534950061642\\
-64.697	-62.0469511614328\\
-73.242	-70.2419400739704\\
-56.152	-53.8519622488952\\
-35.4	-33.9499833240292\\
-45.166	-43.3159589495227\\
-37.842	-36.2919567499411\\
-31.738	-30.4379822242384\\
-25.635	-24.5849667376127\\
-45.166	-43.3159589495227\\
-62.256	-59.7059367745979\\
-68.359	-65.5589522612236\\
-50.049	-47.9989467622695\\
-81.787	-78.4369289865079\\
-115.967	-111.216884636658\\
-74.463	-71.4129267869263\\
-100.098	-95.997893524539\\
-112.305	-107.704883536867\\
-83.008	-79.6079156994638\\
-48.828	-46.8279600493136\\
-62.256	-59.7059367745979\\
-74.463	-71.4129267869263\\
-45.166	-43.3159589495227\\
-48.828	-46.8279600493136\\
-41.504	-39.8039578497319\\
-24.414	-23.4139800246568\\
-40.283	-38.632971136776\\
-54.932	-52.6819345750162\\
-48.828	-46.8279600493136\\
-59.814	-57.363963348686\\
-69.58	-66.7299389741795\\
-50.049	-47.9989467622695\\
-30.518	-29.2679545503595\\
-24.414	-23.4139800246568\\
-31.738	-30.4379822242384\\
-36.621	-35.1209700369852\\
-43.945	-42.1449722365668\\
-86.67	-83.1199167992547\\
-89.111	-85.4609311860896\\
-76.904	-73.7539411737612\\
-74.463	-71.4129267869263\\
-101.318	-97.167921198418\\
-128.174	-122.923874648987\\
-137.939	-132.288891235403\\
-140.381	-134.630864661315\\
-103.76	-99.5098946243298\\
-108.643	-104.192882437077\\
-111.084	-106.533896823911\\
-112.305	-107.704883536867\\
-126.953	-121.752887936031\\
-169.678	-162.727832498718\\
-144.043	-138.142865761106\\
-131.836	-126.435875748777\\
-104.98	-100.679922298209\\
-101.318	-97.167921198418\\
-96.436	-92.4858924247482\\
-114.746	-110.045897923702\\
-72.021	-69.0709533610144\\
-83.008	-79.6079156994638\\
-102.539	-98.3389079113739\\
-119.629	-114.728885736449\\
-91.553	-87.8029046120014\\
-106.201	-101.850909011165\\
-85.449	-81.9489300862987\\
-75.684	-72.5839134998822\\
-50.049	-47.9989467622695\\
-74.463	-71.4129267869263\\
-122.07	-117.069900123284\\
-96.436	-92.4858924247482\\
-56.152	-53.8519622488952\\
-69.58	-66.7299389741795\\
-41.504	-39.8039578497319\\
-35.4	-33.9499833240292\\
-50.049	-47.9989467622695\\
-31.738	-30.4379822242384\\
-35.4	-33.9499833240292\\
-42.725	-40.9749445626878\\
-28.076	-26.9259811244476\\
-43.945	-42.1449722365668\\
-40.283	-38.632971136776\\
-24.414	-23.4139800246568\\
-39.063	-37.462943462897\\
-46.387	-44.4869456624787\\
-53.711	-51.5109478620603\\
-74.463	-71.4129267869263\\
-64.697	-62.0469511614328\\
-72.021	-69.0709533610144\\
-124.512	-119.411873549196\\
-89.111	-85.4609311860896\\
-76.904	-73.7539411737612\\
-85.449	-81.9489300862987\\
-61.035	-58.534950061642\\
-37.842	-36.2919567499411\\
-65.918	-63.2179378743887\\
-57.373	-55.0229489618511\\
-32.959	-31.6089689371944\\
-58.594	-56.1939356748071\\
-54.932	-52.6819345750162\\
-67.139	-64.3889245873446\\
-45.166	-43.3159589495227\\
-28.076	-26.9259811244476\\
-23.193	-22.2429933117009\\
-26.855	-25.7549944114917\\
-48.828	-46.8279600493136\\
-59.814	-57.363963348686\\
-76.904	-73.7539411737612\\
-102.539	-98.3389079113739\\
-76.904	-73.7539411737612\\
-102.539	-98.3389079113739\\
-122.07	-117.069900123284\\
-104.98	-100.679922298209\\
-98.877	-94.826906811583\\
-162.354	-155.703830299137\\
-118.408	-113.557899023493\\
-142.822	-136.97187904815\\
-123.291	-118.24088683624\\
-145.264	-139.313852474062\\
-162.354	-155.703830299137\\
-96.436	-92.4858924247482\\
-58.594	-56.1939356748071\\
-48.828	-46.8279600493136\\
-74.463	-71.4129267869263\\
-93.994	-90.1439189988363\\
-95.215	-91.3149057117922\\
-65.918	-63.2179378743887\\
-36.621	-35.1209700369852\\
-46.387	-44.4869456624787\\
-92.773	-88.9729322858804\\
-125.732	-120.581901223075\\
-124.512	-119.411873549196\\
-74.463	-71.4129267869263\\
-57.373	-55.0229489618511\\
-76.904	-73.7539411737612\\
-122.07	-117.069900123284\\
-136.719	-131.118863561524\\
-139.16	-133.459877948359\\
-97.656	-93.6559200986271\\
-137.939	-132.288891235403\\
-152.588	-146.337854673643\\
-142.822	-136.97187904815\\
-197.754	-189.653813623166\\
-169.678	-162.727832498718\\
-141.602	-135.801851374271\\
-164.795	-158.044844685972\\
-140.381	-134.630864661315\\
-78.125	-74.9249278867171\\
-72.021	-69.0709533610144\\
-89.111	-85.4609311860896\\
-108.643	-104.192882437077\\
-111.084	-106.533896823911\\
-81.787	-78.4369289865079\\
-102.539	-98.3389079113739\\
-91.553	-87.8029046120014\\
-65.918	-63.2179378743887\\
-89.111	-85.4609311860896\\
-84.229	-80.7789024124198\\
-124.512	-119.411873549196\\
-128.174	-122.923874648987\\
-91.553	-87.8029046120014\\
-70.801	-67.9009256871355\\
-42.725	-40.9749445626878\\
-64.697	-62.0469511614328\\
-54.932	-52.6819345750162\\
-46.387	-44.4869456624787\\
-39.063	-37.462943462897\\
-65.918	-63.2179378743887\\
-86.67	-83.1199167992547\\
-95.215	-91.3149057117922\\
-57.373	-55.0229489618511\\
-42.725	-40.9749445626878\\
-29.297	-28.0969678374035\\
-37.842	-36.2919567499411\\
-63.477	-60.8769234875538\\
-65.918	-63.2179378743887\\
-64.697	-62.0469511614328\\
-102.539	-98.3389079113739\\
-122.07	-117.069900123284\\
-134.277	-128.776890135612\\
-135.498	-129.947876848568\\
-157.471	-151.02084248639\\
-97.656	-93.6559200986271\\
-80.566	-77.265942273552\\
-61.035	-58.534950061642\\
-67.139	-64.3889245873446\\
-90.332	-86.6319178990455\\
-75.684	-72.5839134998822\\
-51.27	-49.1699334752254\\
-46.387	-44.4869456624787\\
-83.008	-79.6079156994638\\
-102.539	-98.3389079113739\\
-90.332	-86.6319178990455\\
-147.705	-141.654866860897\\
-161.133	-154.532843586181\\
-112.305	-107.704883536867\\
-79.346	-76.095914599673\\
-54.932	-52.6819345750162\\
-50.049	-47.9989467622695\\
-90.332	-86.6319178990455\\
-76.904	-73.7539411737612\\
-86.67	-83.1199167992547\\
-98.877	-94.826906811583\\
-107.422	-103.021895724121\\
-103.76	-99.5098946243298\\
-117.188	-112.387871349614\\
-81.787	-78.4369289865079\\
-89.111	-85.4609311860896\\
-65.918	-63.2179378743887\\
-58.594	-56.1939356748071\\
-90.332	-86.6319178990455\\
-125.732	-120.581901223075\\
-135.498	-129.947876848568\\
-80.566	-77.265942273552\\
-155.029	-148.678869060478\\
-145.264	-139.313852474062\\
-96.436	-92.4858924247482\\
-56.152	-53.8519622488952\\
-83.008	-79.6079156994638\\
-75.684	-72.5839134998822\\
-70.801	-67.9009256871355\\
-85.449	-81.9489300862987\\
-58.594	-56.1939356748071\\
-79.346	-76.095914599673\\
-145.264	-139.313852474062\\
-151.367	-145.166867960687\\
-98.877	-94.826906811583\\
-108.643	-104.192882437077\\
-122.07	-117.069900123284\\
-123.291	-118.24088683624\\
-90.332	-86.6319178990455\\
-87.891	-84.2909035122106\\
-61.035	-58.534950061642\\
-89.111	-85.4609311860896\\
-95.215	-91.3149057117922\\
-146.484	-140.483880147941\\
-166.016	-159.215831398928\\
-228.271	-218.920809134449\\
-163.574	-156.873857973016\\
-131.836	-126.435875748777\\
-103.76	-99.5098946243298\\
-109.863	-105.362910110956\\
-85.449	-81.9489300862987\\
-59.814	-57.363963348686\\
-57.373	-55.0229489618511\\
-59.814	-57.363963348686\\
-46.387	-44.4869456624787\\
-32.959	-31.6089689371944\\
-51.27	-49.1699334752254\\
-67.139	-64.3889245873446\\
-47.607	-45.6569733363576\\
-34.18	-32.7799556501503\\
-32.959	-31.6089689371944\\
-76.904	-73.7539411737612\\
-109.863	-105.362910110956\\
-52.49	-50.3399611491044\\
-62.256	-59.7059367745979\\
-68.359	-65.5589522612236\\
-61.035	-58.534950061642\\
-76.904	-73.7539411737612\\
-67.139	-64.3889245873446\\
-47.607	-45.6569733363576\\
-34.18	-32.7799556501503\\
-29.297	-28.0969678374035\\
-39.063	-37.462943462897\\
-73.242	-70.2419400739704\\
-87.891	-84.2909035122106\\
-62.256	-59.7059367745979\\
-43.945	-42.1449722365668\\
-29.297	-28.0969678374035\\
-41.504	-39.8039578497319\\
-64.697	-62.0469511614328\\
-84.229	-80.7789024124198\\
-89.111	-85.4609311860896\\
-63.477	-60.8769234875538\\
-59.814	-57.363963348686\\
-67.139	-64.3889245873446\\
-91.553	-87.8029046120014\\
-126.953	-121.752887936031\\
-146.484	-140.483880147941\\
-111.084	-106.533896823911\\
-69.58	-66.7299389741795\\
-75.684	-72.5839134998822\\
-69.58	-66.7299389741795\\
-72.021	-69.0709533610144\\
-50.049	-47.9989467622695\\
-59.814	-57.363963348686\\
-67.139	-64.3889245873446\\
-63.477	-60.8769234875538\\
-41.504	-39.8039578497319\\
-29.297	-28.0969678374035\\
-41.504	-39.8039578497319\\
-63.477	-60.8769234875538\\
-93.994	-90.1439189988363\\
-100.098	-95.997893524539\\
-122.07	-117.069900123284\\
-109.863	-105.362910110956\\
-123.291	-118.24088683624\\
-163.574	-156.873857973016\\
-217.285	-208.384805835076\\
-179.443	-172.092849085135\\
-119.629	-114.728885736449\\
-130.615	-125.264889035821\\
-153.809	-147.508841386599\\
-197.754	-189.653813623166\\
-128.174	-122.923874648987\\
-62.256	-59.7059367745979\\
-43.945	-42.1449722365668\\
-45.166	-43.3159589495227\\
-32.959	-31.6089689371944\\
-21.973	-21.0729656378219\\
-28.076	-26.9259811244476\\
-42.725	-40.9749445626878\\
-61.035	-58.534950061642\\
-59.814	-57.363963348686\\
-47.607	-45.6569733363576\\
-45.166	-43.3159589495227\\
-56.152	-53.8519622488952\\
-69.58	-66.7299389741795\\
-73.242	-70.2419400739704\\
-69.58	-66.7299389741795\\
-48.828	-46.8279600493136\\
-35.4	-33.9499833240292\\
-34.18	-32.7799556501503\\
-50.049	-47.9989467622695\\
-63.477	-60.8769234875538\\
-46.387	-44.4869456624787\\
-32.959	-31.6089689371944\\
-52.49	-50.3399611491044\\
-40.283	-38.632971136776\\
-43.945	-42.1449722365668\\
-56.152	-53.8519622488952\\
-79.346	-76.095914599673\\
-86.67	-83.1199167992547\\
-59.814	-57.363963348686\\
-90.332	-86.6319178990455\\
-87.891	-84.2909035122106\\
-69.58	-66.7299389741795\\
-61.035	-58.534950061642\\
-97.656	-93.6559200986271\\
-123.291	-118.24088683624\\
-119.629	-114.728885736449\\
-133.057	-127.606862461733\\
-102.539	-98.3389079113739\\
-92.773	-88.9729322858804\\
-87.891	-84.2909035122106\\
-118.408	-113.557899023493\\
-86.67	-83.1199167992547\\
-120.85	-115.899872449405\\
-115.967	-111.216884636658\\
-118.408	-113.557899023493\\
-203.857	-195.506829109792\\
-161.133	-154.532843586181\\
-85.449	-81.9489300862987\\
-58.594	-56.1939356748071\\
-62.256	-59.7059367745979\\
-79.346	-76.095914599673\\
-101.318	-97.167921198418\\
-114.746	-110.045897923702\\
-128.174	-122.923874648987\\
-131.836	-126.435875748777\\
-86.67	-83.1199167992547\\
-76.904	-73.7539411737612\\
-89.111	-85.4609311860896\\
-56.152	-53.8519622488952\\
-40.283	-38.632971136776\\
-70.801	-67.9009256871355\\
-93.994	-90.1439189988363\\
-115.967	-111.216884636658\\
-107.422	-103.021895724121\\
-74.463	-71.4129267869263\\
-61.035	-58.534950061642\\
-90.332	-86.6319178990455\\
-118.408	-113.557899023493\\
-153.809	-147.508841386599\\
-169.678	-162.727832498718\\
-194.092	-186.141812523375\\
-157.471	-151.02084248639\\
-141.602	-135.801851374271\\
-84.229	-80.7789024124198\\
-62.256	-59.7059367745979\\
-104.98	-100.679922298209\\
-139.16	-133.459877948359\\
-81.787	-78.4369289865079\\
-125.732	-120.581901223075\\
-172.119	-165.068846885553\\
-206.299	-197.848802535704\\
-129.395	-124.094861361943\\
-74.463	-71.4129267869263\\
-42.725	-40.9749445626878\\
-64.697	-62.0469511614328\\
-59.814	-57.363963348686\\
-62.256	-59.7059367745979\\
-37.842	-36.2919567499411\\
-52.49	-50.3399611491044\\
-41.504	-39.8039578497319\\
-52.49	-50.3399611491044\\
-43.945	-42.1449722365668\\
-54.932	-52.6819345750162\\
-85.449	-81.9489300862987\\
-79.346	-76.095914599673\\
-112.305	-107.704883536867\\
-133.057	-127.606862461733\\
-74.463	-71.4129267869263\\
-76.904	-73.7539411737612\\
-91.553	-87.8029046120014\\
-62.256	-59.7059367745979\\
-56.152	-53.8519622488952\\
-42.725	-40.9749445626878\\
-63.477	-60.8769234875538\\
-59.814	-57.363963348686\\
-34.18	-32.7799556501503\\
-24.414	-23.4139800246568\\
-41.504	-39.8039578497319\\
-61.035	-58.534950061642\\
-65.918	-63.2179378743887\\
-41.504	-39.8039578497319\\
-51.27	-49.1699334752254\\
-73.242	-70.2419400739704\\
-93.994	-90.1439189988363\\
-65.918	-63.2179378743887\\
-62.256	-59.7059367745979\\
-74.463	-71.4129267869263\\
-87.891	-84.2909035122106\\
-80.566	-77.265942273552\\
-92.773	-88.9729322858804\\
-86.67	-83.1199167992547\\
-62.256	-59.7059367745979\\
-48.828	-46.8279600493136\\
-62.256	-59.7059367745979\\
-53.711	-51.5109478620603\\
-64.697	-62.0469511614328\\
-87.891	-84.2909035122106\\
-139.16	-133.459877948359\\
-98.877	-94.826906811583\\
-95.215	-91.3149057117922\\
-140.381	-134.630864661315\\
-108.643	-104.192882437077\\
-67.139	-64.3889245873446\\
-48.828	-46.8279600493136\\
-40.283	-38.632971136776\\
-42.725	-40.9749445626878\\
-35.4	-33.9499833240292\\
-36.621	-35.1209700369852\\
-67.139	-64.3889245873446\\
-35.4	-33.9499833240292\\
-34.18	-32.7799556501503\\
-47.607	-45.6569733363576\\
-65.918	-63.2179378743887\\
-50.049	-47.9989467622695\\
-37.842	-36.2919567499411\\
-23.193	-22.2429933117009\\
-39.063	-37.462943462897\\
-72.021	-69.0709533610144\\
-96.436	-92.4858924247482\\
-79.346	-76.095914599673\\
-63.477	-60.8769234875538\\
-72.021	-69.0709533610144\\
-80.566	-77.265942273552\\
-98.877	-94.826906811583\\
-117.188	-112.387871349614\\
-114.746	-110.045897923702\\
-95.215	-91.3149057117922\\
-64.697	-62.0469511614328\\
-52.49	-50.3399611491044\\
-56.152	-53.8519622488952\\
-76.904	-73.7539411737612\\
-47.607	-45.6569733363576\\
-43.945	-42.1449722365668\\
-80.566	-77.265942273552\\
-96.436	-92.4858924247482\\
-91.553	-87.8029046120014\\
-104.98	-100.679922298209\\
-137.939	-132.288891235403\\
-74.463	-71.4129267869263\\
-139.16	-133.459877948359\\
-129.395	-124.094861361943\\
-120.85	-115.899872449405\\
-128.174	-122.923874648987\\
-125.732	-120.581901223075\\
-212.402	-203.701818022329\\
-198.975	-190.824800336122\\
-134.277	-128.776890135612\\
-159.912	-153.361856873225\\
-194.092	-186.141812523375\\
-190.43	-182.629811423584\\
-212.402	-203.701818022329\\
-194.092	-186.141812523375\\
-253.906	-243.505775872061\\
-195.313	-187.312799236331\\
-135.498	-129.947876848568\\
-81.787	-78.4369289865079\\
-84.229	-80.7789024124198\\
-68.359	-65.5589522612236\\
-57.373	-55.0229489618511\\
-85.449	-81.9489300862987\\
-120.85	-115.899872449405\\
-76.904	-73.7539411737612\\
-67.139	-64.3889245873446\\
-65.918	-63.2179378743887\\
-46.387	-44.4869456624787\\
-56.152	-53.8519622488952\\
-29.297	-28.0969678374035\\
-39.063	-37.462943462897\\
-29.297	-28.0969678374035\\
-46.387	-44.4869456624787\\
-32.959	-31.6089689371944\\
-30.518	-29.2679545503595\\
-19.531	-18.73099221191\\
-18.311	-17.5609645380311\\
-34.18	-32.7799556501503\\
-25.635	-24.5849667376127\\
-28.076	-26.9259811244476\\
-57.373	-55.0229489618511\\
-78.125	-74.9249278867171\\
-80.566	-77.265942273552\\
-58.594	-56.1939356748071\\
-74.463	-71.4129267869263\\
-114.746	-110.045897923702\\
-53.711	-51.5109478620603\\
-37.842	-36.2919567499411\\
-28.076	-26.9259811244476\\
-20.752	-19.901978924866\\
-34.18	-32.7799556501503\\
-56.152	-53.8519622488952\\
-84.229	-80.7789024124198\\
-50.049	-47.9989467622695\\
-85.449	-81.9489300862987\\
-118.408	-113.557899023493\\
-120.85	-115.899872449405\\
-89.111	-85.4609311860896\\
-57.373	-55.0229489618511\\
-72.021	-69.0709533610144\\
-96.436	-92.4858924247482\\
-124.512	-119.411873549196\\
-68.359	-65.5589522612236\\
-37.842	-36.2919567499411\\
-57.373	-55.0229489618511\\
-47.607	-45.6569733363576\\
-23.193	-22.2429933117009\\
-18.311	-17.5609645380311\\
-39.063	-37.462943462897\\
-85.449	-81.9489300862987\\
-63.477	-60.8769234875538\\
-41.504	-39.8039578497319\\
-43.945	-42.1449722365668\\
-15.869	-15.2189911121192\\
-30.518	-29.2679545503595\\
-25.635	-24.5849667376127\\
-42.725	-40.9749445626878\\
-58.594	-56.1939356748071\\
-89.111	-85.4609311860896\\
-93.994	-90.1439189988363\\
-100.098	-95.997893524539\\
-104.98	-100.679922298209\\
-102.539	-98.3389079113739\\
-100.098	-95.997893524539\\
-86.67	-83.1199167992547\\
-104.98	-100.679922298209\\
-157.471	-151.02084248639\\
-144.043	-138.142865761106\\
-74.463	-71.4129267869263\\
-40.283	-38.632971136776\\
-86.67	-83.1199167992547\\
-106.201	-101.850909011165\\
-96.436	-92.4858924247482\\
-56.152	-53.8519622488952\\
-57.373	-55.0229489618511\\
-97.656	-93.6559200986271\\
-146.484	-140.483880147941\\
-150.146	-143.995881247732\\
-169.678	-162.727832498718\\
-164.795	-158.044844685972\\
-120.85	-115.899872449405\\
-123.291	-118.24088683624\\
-58.594	-56.1939356748071\\
-56.152	-53.8519622488952\\
-72.021	-69.0709533610144\\
-87.891	-84.2909035122106\\
-91.553	-87.8029046120014\\
-96.436	-92.4858924247482\\
-63.477	-60.8769234875538\\
-86.67	-83.1199167992547\\
-69.58	-66.7299389741795\\
-41.504	-39.8039578497319\\
-29.297	-28.0969678374035\\
-24.414	-23.4139800246568\\
-37.842	-36.2919567499411\\
-30.518	-29.2679545503595\\
-18.311	-17.5609645380311\\
-39.063	-37.462943462897\\
-74.463	-71.4129267869263\\
-48.828	-46.8279600493136\\
-45.166	-43.3159589495227\\
-61.035	-58.534950061642\\
-100.098	-95.997893524539\\
-68.359	-65.5589522612236\\
-37.842	-36.2919567499411\\
-24.414	-23.4139800246568\\
-53.711	-51.5109478620603\\
-108.643	-104.192882437077\\
-136.719	-131.118863561524\\
-189.209	-181.458824710629\\
-223.389	-214.238780360779\\
-220.947	-211.896806934867\\
-142.822	-136.97187904815\\
-147.705	-141.654866860897\\
-191.65	-183.799839097463\\
-159.912	-153.361856873225\\
-147.705	-141.654866860897\\
-168.457	-161.556845785763\\
-183.105	-175.604850184926\\
-185.547	-177.946823610838\\
-253.906	-243.505775872061\\
-168.457	-161.556845785763\\
-95.215	-91.3149057117922\\
-56.152	-53.8519622488952\\
-45.166	-43.3159589495227\\
-50.049	-47.9989467622695\\
-48.828	-46.8279600493136\\
-86.67	-83.1199167992547\\
-65.918	-63.2179378743887\\
-73.242	-70.2419400739704\\
-92.773	-88.9729322858804\\
-53.711	-51.5109478620603\\
-64.697	-62.0469511614328\\
-84.229	-80.7789024124198\\
-98.877	-94.826906811583\\
-56.152	-53.8519622488952\\
-39.063	-37.462943462897\\
-50.049	-47.9989467622695\\
-48.828	-46.8279600493136\\
-113.525	-108.874911210746\\
-155.029	-148.678869060478\\
-145.264	-139.313852474062\\
-168.457	-161.556845785763\\
-189.209	-181.458824710629\\
-246.582	-236.48177367248\\
-208.74	-200.189816922539\\
-135.498	-129.947876848568\\
-89.111	-85.4609311860896\\
-51.27	-49.1699334752254\\
-58.594	-56.1939356748071\\
-76.904	-73.7539411737612\\
-53.711	-51.5109478620603\\
-50.049	-47.9989467622695\\
-53.711	-51.5109478620603\\
-69.58	-66.7299389741795\\
-76.904	-73.7539411737612\\
-95.215	-91.3149057117922\\
-68.359	-65.5589522612236\\
-32.959	-31.6089689371944\\
-25.635	-24.5849667376127\\
-23.193	-22.2429933117009\\
-25.635	-24.5849667376127\\
-30.518	-29.2679545503595\\
-53.711	-51.5109478620603\\
-52.49	-50.3399611491044\\
-65.918	-63.2179378743887\\
-95.215	-91.3149057117922\\
-68.359	-65.5589522612236\\
-113.525	-108.874911210746\\
-164.795	-158.044844685972\\
-129.395	-124.094861361943\\
-115.967	-111.216884636658\\
-140.381	-134.630864661315\\
-162.354	-155.703830299137\\
-108.643	-104.192882437077\\
-93.994	-90.1439189988363\\
-59.814	-57.363963348686\\
-65.918	-63.2179378743887\\
-131.836	-126.435875748777\\
-214.844	-206.043791448241\\
-256.348	-245.847749297973\\
-202.637	-194.336801435913\\
-168.457	-161.556845785763\\
-123.291	-118.24088683624\\
-133.057	-127.606862461733\\
-175.781	-168.580847985344\\
-93.994	-90.1439189988363\\
-83.008	-79.6079156994638\\
-63.477	-60.8769234875538\\
-115.967	-111.216884636658\\
-112.305	-107.704883536867\\
-61.035	-58.534950061642\\
-62.256	-59.7059367745979\\
-46.387	-44.4869456624787\\
-84.229	-80.7789024124198\\
-119.629	-114.728885736449\\
-126.953	-121.752887936031\\
-95.215	-91.3149057117922\\
-69.58	-66.7299389741795\\
-80.566	-77.265942273552\\
-61.035	-58.534950061642\\
-46.387	-44.4869456624787\\
-36.621	-35.1209700369852\\
-32.959	-31.6089689371944\\
-29.297	-28.0969678374035\\
-34.18	-32.7799556501503\\
-59.814	-57.363963348686\\
-84.229	-80.7789024124198\\
-83.008	-79.6079156994638\\
-52.49	-50.3399611491044\\
-47.607	-45.6569733363576\\
-40.283	-38.632971136776\\
-52.49	-50.3399611491044\\
-68.359	-65.5589522612236\\
-73.242	-70.2419400739704\\
-69.58	-66.7299389741795\\
-40.283	-38.632971136776\\
-83.008	-79.6079156994638\\
-120.85	-115.899872449405\\
-84.229	-80.7789024124198\\
-35.4	-33.9499833240292\\
-40.283	-38.632971136776\\
-67.139	-64.3889245873446\\
-58.594	-56.1939356748071\\
-37.842	-36.2919567499411\\
-68.359	-65.5589522612236\\
-97.656	-93.6559200986271\\
-59.814	-57.363963348686\\
-24.414	-23.4139800246568\\
-37.842	-36.2919567499411\\
-81.787	-78.4369289865079\\
-96.436	-92.4858924247482\\
-59.814	-57.363963348686\\
-53.711	-51.5109478620603\\
-97.656	-93.6559200986271\\
-125.732	-120.581901223075\\
-108.643	-104.192882437077\\
-152.588	-146.337854673643\\
-184.326	-176.775836897882\\
-115.967	-111.216884636658\\
-95.215	-91.3149057117922\\
-139.16	-133.459877948359\\
-93.994	-90.1439189988363\\
-126.953	-121.752887936031\\
-152.588	-146.337854673643\\
-119.629	-114.728885736449\\
-57.373	-55.0229489618511\\
-101.318	-97.167921198418\\
-172.119	-165.068846885553\\
-195.313	-187.312799236331\\
-142.822	-136.97187904815\\
-120.85	-115.899872449405\\
-151.367	-145.166867960687\\
-118.408	-113.557899023493\\
-76.904	-73.7539411737612\\
-73.242	-70.2419400739704\\
-91.553	-87.8029046120014\\
-95.215	-91.3149057117922\\
-93.994	-90.1439189988363\\
-104.98	-100.679922298209\\
-72.021	-69.0709533610144\\
-79.346	-76.095914599673\\
-109.863	-105.362910110956\\
-64.697	-62.0469511614328\\
-53.711	-51.5109478620603\\
-43.945	-42.1449722365668\\
-35.4	-33.9499833240292\\
-42.725	-40.9749445626878\\
-73.242	-70.2419400739704\\
-65.918	-63.2179378743887\\
-93.994	-90.1439189988363\\
-114.746	-110.045897923702\\
-139.16	-133.459877948359\\
-102.539	-98.3389079113739\\
-92.773	-88.9729322858804\\
-43.945	-42.1449722365668\\
-56.152	-53.8519622488952\\
-107.422	-103.021895724121\\
-73.242	-70.2419400739704\\
-42.725	-40.9749445626878\\
-78.125	-74.9249278867171\\
-115.967	-111.216884636658\\
-120.85	-115.899872449405\\
-157.471	-151.02084248639\\
-205.078	-196.677815822748\\
-136.719	-131.118863561524\\
-118.408	-113.557899023493\\
-142.822	-136.97187904815\\
-92.773	-88.9729322858804\\
-76.904	-73.7539411737612\\
-85.449	-81.9489300862987\\
-111.084	-106.533896823911\\
-118.408	-113.557899023493\\
-119.629	-114.728885736449\\
-67.139	-64.3889245873446\\
-59.814	-57.363963348686\\
-52.49	-50.3399611491044\\
-73.242	-70.2419400739704\\
-65.918	-63.2179378743887\\
-52.49	-50.3399611491044\\
-104.98	-100.679922298209\\
-111.084	-106.533896823911\\
-78.125	-74.9249278867171\\
-64.697	-62.0469511614328\\
-96.436	-92.4858924247482\\
-53.711	-51.5109478620603\\
-32.959	-31.6089689371944\\
-41.504	-39.8039578497319\\
-57.373	-55.0229489618511\\
-48.828	-46.8279600493136\\
-35.4	-33.9499833240292\\
-21.973	-21.0729656378219\\
-40.283	-38.632971136776\\
-54.932	-52.6819345750162\\
-86.67	-83.1199167992547\\
-96.436	-92.4858924247482\\
-126.953	-121.752887936031\\
-142.822	-136.97187904815\\
-81.787	-78.4369289865079\\
-62.256	-59.7059367745979\\
-128.174	-122.923874648987\\
-181.885	-174.434822511047\\
-139.16	-133.459877948359\\
-130.615	-125.264889035821\\
-196.533	-188.48282691021\\
-155.029	-148.678869060478\\
-128.174	-122.923874648987\\
-93.994	-90.1439189988363\\
-98.877	-94.826906811583\\
-130.615	-125.264889035821\\
-85.449	-81.9489300862987\\
-74.463	-71.4129267869263\\
-128.174	-122.923874648987\\
-158.691	-152.190870160269\\
-169.678	-162.727832498718\\
-73.242	-70.2419400739704\\
-39.063	-37.462943462897\\
-93.994	-90.1439189988363\\
-73.242	-70.2419400739704\\
-35.4	-33.9499833240292\\
-29.297	-28.0969678374035\\
-50.049	-47.9989467622695\\
-67.139	-64.3889245873446\\
-109.863	-105.362910110956\\
-76.904	-73.7539411737612\\
-61.035	-58.534950061642\\
-36.621	-35.1209700369852\\
-26.855	-25.7549944114917\\
-36.621	-35.1209700369852\\
-30.518	-29.2679545503595\\
-40.283	-38.632971136776\\
-68.359	-65.5589522612236\\
-61.035	-58.534950061642\\
-32.959	-31.6089689371944\\
-30.518	-29.2679545503595\\
-39.063	-37.462943462897\\
-47.607	-45.6569733363576\\
-29.297	-28.0969678374035\\
-47.607	-45.6569733363576\\
-32.959	-31.6089689371944\\
-24.414	-23.4139800246568\\
-56.152	-53.8519622488952\\
-78.125	-74.9249278867171\\
-115.967	-111.216884636658\\
-156.25	-149.849855773434\\
-137.939	-132.288891235403\\
-67.139	-64.3889245873446\\
-51.27	-49.1699334752254\\
-29.297	-28.0969678374035\\
-30.518	-29.2679545503595\\
-57.373	-55.0229489618511\\
-62.256	-59.7059367745979\\
-57.373	-55.0229489618511\\
-65.918	-63.2179378743887\\
-76.904	-73.7539411737612\\
-80.566	-77.265942273552\\
-83.008	-79.6079156994638\\
-103.76	-99.5098946243298\\
-100.098	-95.997893524539\\
-145.264	-139.313852474062\\
-76.904	-73.7539411737612\\
-37.842	-36.2919567499411\\
-36.621	-35.1209700369852\\
-67.139	-64.3889245873446\\
-50.049	-47.9989467622695\\
-47.607	-45.6569733363576\\
-23.193	-22.2429933117009\\
-41.504	-39.8039578497319\\
-25.635	-24.5849667376127\\
-15.869	-15.2189911121192\\
-25.635	-24.5849667376127\\
-47.607	-45.6569733363576\\
-58.594	-56.1939356748071\\
-84.229	-80.7789024124198\\
-118.408	-113.557899023493\\
-84.229	-80.7789024124198\\
-35.4	-33.9499833240292\\
-65.918	-63.2179378743887\\
-74.463	-71.4129267869263\\
-39.063	-37.462943462897\\
-51.27	-49.1699334752254\\
-92.773	-88.9729322858804\\
-81.787	-78.4369289865079\\
-137.939	-132.288891235403\\
-157.471	-151.02084248639\\
-107.422	-103.021895724121\\
-83.008	-79.6079156994638\\
};
\end{axis}

\begin{axis}[%
width=4.927cm,
height=3cm,
at={(7cm,0cm)},
scale only axis,
xmin=-158.691,
xmax=0,
xlabel style={font=\color{white!15!black}},
xlabel={y(t-1)},
ymin=-158.691,
ymax=-9.43152579599362,
ylabel style={font=\color{white!15!black}},
ylabel={y(t)},
axis background/.style={fill=white},
title style={font=\small},
title={C10, R = 0.7824},
axis x line*=bottom,
axis y line*=left
]
\addplot[only marks, mark=*, mark options={}, mark size=1.5000pt, color=mycolor1, fill=mycolor1] table[row sep=crcr]{%
x	y\\
-54.932	-63.477\\
-63.477	-75.684\\
-75.684	-62.256\\
-62.256	-59.814\\
-59.814	-80.566\\
-80.566	-76.904\\
-76.904	-90.332\\
-90.332	-68.359\\
-68.359	-40.283\\
-40.283	-48.828\\
-48.828	-46.387\\
-46.387	-54.932\\
-54.932	-45.166\\
-45.166	-21.973\\
-21.973	-17.09\\
-17.09	-18.311\\
-18.311	-40.283\\
-40.283	-62.256\\
-62.256	-67.139\\
-67.139	-69.58\\
-69.58	-52.49\\
-52.49	-34.18\\
-34.18	-64.697\\
-64.697	-52.49\\
-52.49	-39.063\\
-39.063	-61.035\\
-61.035	-53.711\\
-53.711	-45.166\\
-45.166	-73.242\\
-73.242	-67.139\\
-67.139	-52.49\\
-52.49	-75.684\\
-75.684	-74.463\\
-74.463	-58.594\\
-58.594	-50.049\\
-50.049	-41.504\\
-41.504	-47.607\\
-47.607	-58.594\\
-58.594	-54.932\\
-54.932	-59.814\\
-59.814	-59.814\\
-59.814	-65.918\\
-65.918	-85.449\\
-85.449	-76.904\\
-76.904	-54.932\\
-54.932	-50.049\\
-50.049	-35.4\\
-35.4	-34.18\\
-34.18	-45.166\\
-45.166	-34.18\\
-34.18	-36.621\\
-36.621	-51.27\\
-51.27	-54.932\\
-54.932	-51.27\\
-51.27	-78.125\\
-78.125	-62.256\\
-62.256	-36.621\\
-36.621	-30.518\\
-30.518	-40.283\\
-40.283	-46.387\\
-46.387	-29.297\\
-29.297	-29.297\\
-29.297	-36.621\\
-36.621	-31.738\\
-31.738	-26.855\\
-26.855	-35.4\\
-35.4	-40.283\\
-40.283	-58.594\\
-58.594	-56.152\\
-56.152	-65.918\\
-65.918	-61.035\\
-61.035	-70.801\\
-70.801	-57.373\\
-57.373	-58.594\\
-58.594	-51.27\\
-51.27	-50.049\\
-50.049	-48.828\\
-48.828	-80.566\\
-80.566	-111.084\\
-111.084	-114.746\\
-114.746	-112.305\\
-112.305	-74.463\\
-74.463	-100.098\\
-100.098	-125.732\\
-125.732	-128.174\\
-128.174	-96.436\\
-96.436	-124.512\\
-124.512	-158.691\\
-158.691	-115.967\\
-115.967	-97.656\\
-97.656	-68.359\\
-68.359	-53.711\\
-53.711	-47.607\\
-47.607	-56.152\\
-56.152	-45.166\\
-45.166	-30.518\\
-30.518	-30.518\\
-30.518	-36.621\\
-36.621	-50.049\\
-50.049	-48.828\\
-48.828	-47.607\\
-47.607	-62.256\\
-62.256	-64.697\\
-64.697	-85.449\\
-85.449	-72.021\\
-72.021	-85.449\\
-85.449	-63.477\\
-63.477	-54.932\\
-54.932	-64.697\\
-64.697	-48.828\\
-48.828	-43.945\\
-43.945	-53.711\\
-53.711	-54.932\\
-54.932	-40.283\\
-40.283	-43.945\\
-43.945	-32.959\\
-32.959	-36.621\\
-36.621	-40.283\\
-40.283	-30.518\\
-30.518	-34.18\\
-34.18	-43.945\\
-43.945	-63.477\\
-63.477	-65.918\\
-65.918	-72.021\\
-72.021	-45.166\\
-45.166	-56.152\\
-56.152	-89.111\\
-89.111	-70.801\\
-70.801	-81.787\\
-81.787	-80.566\\
-80.566	-50.049\\
-50.049	-36.621\\
-36.621	-47.607\\
-47.607	-52.49\\
-52.49	-81.787\\
-81.787	-103.76\\
-103.76	-102.539\\
-102.539	-76.904\\
-76.904	-80.566\\
-80.566	-72.021\\
-72.021	-69.58\\
-69.58	-68.359\\
-68.359	-51.27\\
-51.27	-45.166\\
-45.166	-39.063\\
-39.063	-37.842\\
-37.842	-41.504\\
-41.504	-56.152\\
-56.152	-84.229\\
-84.229	-70.801\\
-70.801	-50.049\\
-50.049	-42.725\\
-42.725	-40.283\\
-40.283	-28.076\\
-28.076	-23.193\\
-23.193	-24.414\\
-24.414	-41.504\\
-41.504	-32.959\\
-32.959	-34.18\\
-34.18	-37.842\\
-37.842	-35.4\\
-35.4	-26.855\\
-26.855	-21.973\\
-21.973	-18.311\\
-18.311	-25.635\\
-25.635	-51.27\\
-51.27	-73.242\\
-73.242	-78.125\\
-78.125	-54.932\\
-54.932	-39.063\\
-39.063	-31.738\\
-31.738	-24.414\\
-24.414	-41.504\\
-41.504	-42.725\\
-42.725	-54.932\\
-54.932	-73.242\\
-73.242	-108.643\\
-108.643	-125.732\\
-125.732	-103.76\\
-103.76	-101.318\\
-101.318	-67.139\\
-67.139	-75.684\\
-75.684	-79.346\\
-79.346	-86.67\\
-86.67	-69.58\\
-69.58	-62.256\\
-62.256	-61.035\\
-61.035	-56.152\\
-56.152	-67.139\\
-67.139	-52.49\\
-52.49	-76.904\\
-76.904	-97.656\\
-97.656	-72.021\\
-72.021	-45.166\\
-45.166	-47.607\\
-47.607	-78.125\\
-78.125	-93.994\\
-93.994	-112.305\\
-112.305	-119.629\\
-119.629	-119.629\\
-119.629	-91.553\\
-91.553	-84.229\\
-84.229	-96.436\\
-96.436	-111.084\\
-111.084	-74.463\\
-74.463	-46.387\\
-46.387	-62.256\\
-62.256	-54.932\\
-54.932	-34.18\\
-34.18	-47.607\\
-47.607	-53.711\\
-53.711	-42.725\\
-42.725	-34.18\\
-34.18	-43.945\\
-43.945	-40.283\\
-40.283	-52.49\\
-52.49	-65.918\\
-65.918	-62.256\\
-62.256	-45.166\\
-45.166	-45.166\\
-45.166	-67.139\\
-67.139	-104.98\\
-104.98	-79.346\\
-79.346	-51.27\\
-51.27	-40.283\\
-40.283	-47.607\\
-47.607	-40.283\\
-40.283	-31.738\\
-31.738	-35.4\\
-35.4	-42.725\\
-42.725	-51.27\\
-51.27	-57.373\\
-57.373	-42.725\\
-42.725	-63.477\\
-63.477	-75.684\\
-75.684	-70.801\\
-70.801	-57.373\\
-57.373	-62.256\\
-62.256	-81.787\\
-81.787	-52.49\\
-52.49	-28.076\\
-28.076	-36.621\\
-36.621	-42.725\\
-42.725	-51.27\\
-51.27	-46.387\\
-46.387	-36.621\\
-36.621	-52.49\\
-52.49	-52.49\\
-52.49	-37.842\\
-37.842	-47.607\\
-47.607	-41.504\\
-41.504	-37.842\\
-37.842	-45.166\\
-45.166	-29.297\\
-29.297	-31.738\\
-31.738	-29.297\\
-29.297	-26.855\\
-26.855	-47.607\\
-47.607	-53.711\\
-53.711	-69.58\\
-69.58	-89.111\\
-89.111	-62.256\\
-62.256	-36.621\\
-36.621	-31.738\\
-31.738	-29.297\\
-29.297	-40.283\\
-40.283	-31.738\\
-31.738	-30.518\\
-30.518	-39.063\\
-39.063	-61.035\\
-61.035	-53.711\\
-53.711	-79.346\\
-79.346	-54.932\\
-54.932	-48.828\\
-48.828	-32.959\\
-32.959	-29.297\\
-29.297	-23.193\\
-23.193	-23.193\\
-23.193	-25.635\\
-25.635	-40.283\\
-40.283	-46.387\\
-46.387	-48.828\\
-48.828	-51.27\\
-51.27	-79.346\\
-79.346	-73.242\\
-73.242	-53.711\\
-53.711	-63.477\\
-63.477	-69.58\\
-69.58	-85.449\\
-85.449	-72.021\\
-72.021	-47.607\\
-47.607	-28.076\\
-28.076	-21.973\\
-21.973	-21.973\\
-21.973	-26.855\\
-26.855	-26.855\\
-26.855	-25.635\\
-25.635	-32.959\\
-32.959	-47.607\\
-47.607	-43.945\\
-43.945	-45.166\\
-45.166	-52.49\\
-52.49	-35.4\\
-35.4	-20.752\\
-20.752	-35.4\\
-35.4	-56.152\\
-56.152	-54.932\\
-54.932	-59.814\\
-59.814	-53.711\\
-53.711	-40.283\\
-40.283	-51.27\\
-51.27	-72.021\\
-72.021	-70.801\\
-70.801	-51.27\\
-51.27	-81.787\\
-81.787	-62.256\\
-62.256	-63.477\\
-63.477	-73.242\\
-73.242	-70.801\\
-70.801	-54.932\\
-54.932	-47.607\\
-47.607	-78.125\\
-78.125	-108.643\\
-108.643	-85.449\\
-85.449	-50.049\\
-50.049	-51.27\\
-51.27	-51.27\\
-51.27	-61.035\\
-61.035	-72.021\\
-72.021	-54.932\\
-54.932	-56.152\\
-56.152	-73.242\\
-73.242	-79.346\\
-79.346	-56.152\\
-56.152	-45.166\\
-45.166	-57.373\\
-57.373	-87.891\\
-87.891	-78.125\\
-78.125	-80.566\\
-80.566	-54.932\\
-54.932	-45.166\\
-45.166	-47.607\\
-47.607	-31.738\\
-31.738	-20.752\\
-20.752	-20.752\\
-20.752	-17.09\\
-17.09	-31.738\\
-31.738	-48.828\\
-48.828	-54.932\\
-54.932	-40.283\\
-40.283	-45.166\\
-45.166	-52.49\\
-52.49	-35.4\\
-35.4	-35.4\\
-35.4	-35.4\\
-35.4	-29.297\\
-29.297	-31.738\\
-31.738	-51.27\\
-51.27	-64.697\\
-64.697	-41.504\\
-41.504	-32.959\\
-32.959	-28.076\\
-28.076	-32.959\\
-32.959	-26.855\\
-26.855	-45.166\\
-45.166	-80.566\\
-80.566	-63.477\\
-63.477	-81.787\\
-81.787	-98.877\\
-98.877	-97.656\\
-97.656	-76.904\\
-76.904	-58.594\\
-58.594	-54.932\\
-54.932	-57.373\\
-57.373	-65.918\\
-65.918	-76.904\\
-76.904	-76.904\\
-76.904	-100.098\\
-100.098	-108.643\\
-108.643	-130.615\\
-130.615	-91.553\\
-91.553	-98.877\\
-98.877	-108.643\\
-108.643	-85.449\\
-85.449	-96.436\\
-96.436	-85.449\\
-85.449	-51.27\\
-51.27	-34.18\\
-34.18	-36.621\\
-36.621	-51.27\\
-51.27	-43.945\\
-43.945	-31.738\\
-31.738	-41.504\\
-41.504	-32.959\\
-32.959	-23.193\\
-23.193	-31.738\\
-31.738	-34.18\\
-34.18	-25.635\\
-25.635	-42.725\\
-42.725	-58.594\\
-58.594	-45.166\\
-45.166	-68.359\\
-68.359	-100.098\\
-100.098	-87.891\\
-87.891	-109.863\\
-109.863	-81.787\\
-81.787	-81.787\\
-81.787	-62.256\\
-62.256	-37.842\\
-37.842	-46.387\\
-46.387	-42.725\\
-42.725	-29.297\\
-29.297	-40.283\\
-40.283	-21.973\\
-21.973	-19.531\\
-19.531	-24.414\\
-24.414	-45.166\\
-45.166	-57.373\\
-57.373	-65.918\\
-65.918	-65.918\\
-65.918	-62.256\\
-62.256	-69.58\\
-69.58	-57.373\\
-57.373	-48.828\\
-48.828	-50.049\\
-50.049	-43.945\\
-43.945	-59.814\\
-59.814	-46.387\\
-46.387	-36.621\\
-36.621	-51.27\\
-51.27	-69.58\\
-69.58	-54.932\\
-54.932	-41.504\\
-41.504	-37.842\\
-37.842	-41.504\\
-41.504	-32.959\\
-32.959	-30.518\\
-30.518	-41.504\\
-41.504	-47.607\\
-47.607	-52.49\\
-52.49	-42.725\\
-42.725	-53.711\\
-53.711	-51.27\\
-51.27	-32.959\\
-32.959	-32.959\\
-32.959	-58.594\\
-58.594	-80.566\\
-80.566	-78.125\\
-78.125	-67.139\\
-67.139	-62.256\\
-62.256	-50.049\\
-50.049	-48.828\\
-48.828	-41.504\\
-41.504	-54.932\\
-54.932	-50.049\\
-50.049	-32.959\\
-32.959	-45.166\\
-45.166	-52.49\\
-52.49	-61.035\\
-61.035	-68.359\\
-68.359	-68.359\\
-68.359	-87.891\\
-87.891	-107.422\\
-107.422	-119.629\\
-119.629	-80.566\\
-80.566	-46.387\\
-46.387	-32.959\\
-32.959	-24.414\\
-24.414	-34.18\\
-34.18	-29.297\\
-29.297	-51.27\\
-51.27	-48.828\\
-48.828	-40.283\\
-40.283	-52.49\\
-52.49	-63.477\\
-63.477	-70.801\\
-70.801	-75.684\\
-75.684	-103.76\\
-103.76	-92.773\\
-92.773	-72.021\\
-72.021	-51.27\\
-51.27	-52.49\\
-52.49	-54.932\\
-54.932	-65.918\\
-65.918	-51.27\\
-51.27	-50.049\\
-50.049	-48.828\\
-48.828	-30.518\\
-30.518	-45.166\\
-45.166	-68.359\\
-68.359	-48.828\\
-48.828	-52.49\\
-52.49	-86.67\\
-86.67	-63.477\\
-63.477	-63.477\\
-63.477	-85.449\\
-85.449	-81.787\\
-81.787	-53.711\\
-53.711	-35.4\\
-35.4	-36.621\\
-36.621	-61.035\\
-61.035	-90.332\\
-90.332	-97.656\\
-97.656	-100.098\\
-100.098	-79.346\\
-79.346	-53.711\\
-53.711	-46.387\\
-46.387	-37.842\\
-37.842	-34.18\\
-34.18	-29.297\\
-29.297	-40.283\\
-40.283	-42.725\\
-42.725	-30.518\\
-30.518	-42.725\\
-42.725	-56.152\\
-56.152	-62.256\\
-62.256	-78.125\\
-78.125	-96.436\\
-96.436	-114.746\\
-114.746	-84.229\\
-84.229	-73.242\\
-73.242	-76.904\\
-76.904	-63.477\\
-63.477	-46.387\\
-46.387	-57.373\\
-57.373	-90.332\\
-90.332	-101.318\\
-101.318	-61.035\\
-61.035	-36.621\\
-36.621	-26.855\\
-26.855	-18.311\\
-18.311	-21.973\\
-21.973	-30.518\\
-30.518	-31.738\\
-31.738	-42.725\\
-42.725	-37.842\\
-37.842	-30.518\\
-30.518	-29.297\\
-29.297	-29.297\\
-29.297	-26.855\\
-26.855	-31.738\\
-31.738	-43.945\\
-43.945	-63.477\\
-63.477	-68.359\\
-68.359	-65.918\\
-65.918	-74.463\\
-74.463	-91.553\\
-91.553	-92.773\\
-92.773	-106.201\\
-106.201	-100.098\\
-100.098	-74.463\\
-74.463	-54.932\\
-54.932	-51.27\\
-51.27	-84.229\\
-84.229	-97.656\\
-97.656	-69.58\\
-69.58	-46.387\\
-46.387	-26.855\\
-26.855	-21.973\\
-21.973	-20.752\\
-20.752	-14.648\\
-14.648	-25.635\\
-25.635	-40.283\\
-40.283	-40.283\\
-40.283	-35.4\\
-35.4	-24.414\\
-24.414	-19.531\\
-19.531	-28.076\\
-28.076	-28.076\\
-28.076	-30.518\\
-30.518	-24.414\\
-24.414	-21.973\\
-21.973	-32.959\\
-32.959	-69.58\\
-69.58	-79.346\\
-79.346	-86.67\\
-86.67	-63.477\\
-63.477	-80.566\\
-80.566	-114.746\\
-114.746	-103.76\\
-103.76	-106.201\\
-106.201	-81.787\\
-81.787	-81.787\\
-81.787	-86.67\\
-86.67	-59.814\\
-59.814	-56.152\\
-56.152	-40.283\\
-40.283	-35.4\\
-35.4	-62.256\\
-62.256	-47.607\\
-47.607	-48.828\\
-48.828	-53.711\\
-53.711	-50.049\\
-50.049	-50.049\\
-50.049	-52.49\\
-52.49	-57.373\\
-57.373	-64.697\\
-64.697	-56.152\\
-56.152	-41.504\\
-41.504	-52.49\\
-52.49	-31.738\\
-31.738	-18.311\\
-18.311	-25.635\\
-25.635	-34.18\\
-34.18	-21.973\\
-21.973	-9.766\\
-9.766	-21.973\\
-21.973	-32.959\\
-32.959	-37.842\\
-37.842	-43.945\\
-43.945	-39.063\\
-39.063	-58.594\\
-58.594	-48.828\\
-48.828	-36.621\\
-36.621	-54.932\\
-54.932	-39.063\\
-39.063	-26.855\\
-26.855	-23.193\\
-23.193	-17.09\\
-17.09	-25.635\\
-25.635	-19.531\\
-19.531	-29.297\\
-29.297	-43.945\\
-43.945	-42.725\\
-42.725	-32.959\\
-32.959	-40.283\\
-40.283	-29.297\\
-29.297	-43.945\\
-43.945	-31.738\\
-31.738	-31.738\\
-31.738	-30.518\\
-30.518	-24.414\\
-24.414	-26.855\\
-26.855	-26.855\\
-26.855	-41.504\\
-41.504	-36.621\\
-36.621	-30.518\\
-30.518	-32.959\\
-32.959	-26.855\\
-26.855	-31.738\\
-31.738	-58.594\\
-58.594	-75.684\\
-75.684	-51.27\\
-51.27	-48.828\\
-48.828	-39.063\\
-39.063	-58.594\\
-58.594	-58.594\\
-58.594	-37.842\\
-37.842	-46.387\\
-46.387	-39.063\\
-39.063	-51.27\\
-51.27	-32.959\\
-32.959	-35.4\\
-35.4	-37.842\\
-37.842	-47.607\\
-47.607	-36.621\\
-36.621	-41.504\\
-41.504	-41.504\\
-41.504	-57.373\\
-57.373	-51.27\\
-51.27	-72.021\\
-72.021	-93.994\\
-93.994	-78.125\\
-78.125	-50.049\\
-50.049	-40.283\\
-40.283	-54.932\\
-54.932	-69.58\\
-69.58	-56.152\\
-56.152	-63.477\\
-63.477	-43.945\\
-43.945	-23.193\\
-23.193	-24.414\\
-24.414	-52.49\\
-52.49	-45.166\\
-45.166	-42.725\\
-42.725	-64.697\\
-64.697	-61.035\\
-61.035	-56.152\\
-56.152	-39.063\\
-39.063	-53.711\\
-53.711	-65.918\\
-65.918	-52.49\\
-52.49	-53.711\\
-53.711	-48.828\\
-48.828	-37.842\\
-37.842	-42.725\\
-42.725	-35.4\\
-35.4	-36.621\\
-36.621	-57.373\\
-57.373	-67.139\\
-67.139	-70.801\\
-70.801	-62.256\\
-62.256	-45.166\\
-45.166	-35.4\\
-35.4	-39.063\\
-39.063	-46.387\\
-46.387	-57.373\\
-57.373	-68.359\\
-68.359	-79.346\\
-79.346	-65.918\\
-65.918	-40.283\\
-40.283	-68.359\\
-68.359	-91.553\\
-91.553	-65.918\\
-65.918	-64.697\\
-64.697	-75.684\\
-75.684	-59.814\\
-59.814	-50.049\\
-50.049	-56.152\\
-56.152	-51.27\\
-51.27	-57.373\\
-57.373	-70.801\\
-70.801	-56.152\\
-56.152	-63.477\\
-63.477	-58.594\\
-58.594	-59.814\\
-59.814	-48.828\\
-48.828	-63.477\\
-63.477	-45.166\\
-45.166	-48.828\\
-48.828	-52.49\\
-52.49	-50.049\\
-50.049	-30.518\\
-30.518	-19.531\\
-19.531	-15.869\\
-15.869	-28.076\\
-28.076	-53.711\\
-53.711	-67.139\\
-67.139	-65.918\\
-65.918	-46.387\\
-46.387	-54.932\\
-54.932	-50.049\\
-50.049	-43.945\\
-43.945	-30.518\\
-30.518	-42.725\\
-42.725	-37.842\\
-37.842	-36.621\\
-36.621	-41.504\\
-41.504	-37.842\\
-37.842	-34.18\\
-34.18	-25.635\\
-25.635	-35.4\\
-35.4	-58.594\\
-58.594	-42.725\\
-42.725	-52.49\\
-52.49	-45.166\\
-45.166	-59.814\\
-59.814	-57.373\\
-57.373	-76.904\\
-76.904	-58.594\\
-58.594	-64.697\\
-64.697	-65.918\\
-65.918	-37.842\\
-37.842	-61.035\\
-61.035	-46.387\\
-46.387	-51.27\\
-51.27	-57.373\\
-57.373	-69.58\\
-69.58	-53.711\\
-53.711	-70.801\\
-70.801	-81.787\\
-81.787	-69.58\\
-69.58	-58.594\\
-58.594	-50.049\\
-50.049	-58.594\\
-58.594	-52.49\\
-52.49	-43.945\\
-43.945	-37.842\\
-37.842	-45.166\\
-45.166	-39.063\\
-39.063	-72.021\\
-72.021	-90.332\\
-90.332	-79.346\\
-79.346	-75.684\\
-75.684	-74.463\\
-74.463	-46.387\\
-46.387	-42.725\\
-42.725	-35.4\\
-35.4	-30.518\\
-30.518	-32.959\\
-32.959	-52.49\\
-52.49	-59.814\\
-59.814	-70.801\\
-70.801	-59.814\\
-59.814	-42.725\\
-42.725	-72.021\\
-72.021	-84.229\\
-84.229	-90.332\\
-90.332	-73.242\\
-73.242	-115.967\\
-115.967	-85.449\\
-85.449	-51.27\\
-51.27	-39.063\\
-39.063	-34.18\\
-34.18	-56.152\\
-56.152	-69.58\\
-69.58	-89.111\\
-89.111	-68.359\\
-68.359	-53.711\\
-53.711	-31.738\\
-31.738	-36.621\\
-36.621	-34.18\\
-34.18	-39.063\\
-39.063	-25.635\\
-25.635	-35.4\\
-35.4	-26.855\\
-26.855	-19.531\\
-19.531	-39.063\\
-39.063	-42.725\\
-42.725	-48.828\\
-48.828	-46.387\\
-46.387	-48.828\\
-48.828	-48.828\\
-48.828	-62.256\\
-62.256	-43.945\\
-43.945	-62.256\\
-62.256	-69.58\\
-69.58	-56.152\\
-56.152	-58.594\\
-58.594	-54.932\\
-54.932	-61.035\\
-61.035	-80.566\\
-80.566	-62.256\\
-62.256	-50.049\\
-50.049	-57.373\\
-57.373	-51.27\\
-51.27	-36.621\\
-36.621	-56.152\\
-56.152	-74.463\\
-74.463	-69.58\\
-69.58	-78.125\\
-78.125	-113.525\\
-113.525	-79.346\\
-79.346	-72.021\\
-72.021	-56.152\\
-56.152	-68.359\\
-68.359	-92.773\\
-92.773	-64.697\\
-64.697	-57.373\\
-57.373	-76.904\\
-76.904	-51.27\\
-51.27	-32.959\\
-32.959	-42.725\\
-42.725	-32.959\\
-32.959	-25.635\\
-25.635	-28.076\\
-28.076	-25.635\\
-25.635	-25.635\\
-25.635	-31.738\\
-31.738	-42.725\\
-42.725	-48.828\\
-48.828	-63.477\\
-63.477	-59.814\\
-59.814	-67.139\\
-67.139	-79.346\\
-79.346	-73.242\\
-73.242	-54.932\\
-54.932	-58.594\\
-58.594	-53.711\\
-53.711	-57.373\\
-57.373	-76.904\\
-76.904	-59.814\\
-59.814	-63.477\\
-63.477	-62.256\\
-62.256	-57.373\\
-57.373	-84.229\\
-84.229	-69.58\\
-69.58	-73.242\\
-73.242	-65.918\\
-65.918	-42.725\\
-42.725	-29.297\\
-29.297	-24.414\\
-24.414	-41.504\\
-41.504	-34.18\\
-34.18	-47.607\\
-47.607	-35.4\\
-35.4	-45.166\\
-45.166	-79.346\\
-79.346	-93.994\\
-93.994	-75.684\\
-75.684	-79.346\\
-79.346	-70.801\\
-70.801	-52.49\\
-52.49	-54.932\\
-54.932	-59.814\\
-59.814	-53.711\\
-53.711	-62.256\\
-62.256	-76.904\\
-76.904	-106.201\\
-106.201	-93.994\\
-93.994	-87.891\\
-87.891	-97.656\\
-97.656	-68.359\\
-68.359	-74.463\\
-74.463	-59.814\\
-59.814	-59.814\\
-59.814	-75.684\\
-75.684	-85.449\\
-85.449	-37.842\\
-37.842	-48.828\\
-48.828	-37.842\\
-37.842	-53.711\\
-53.711	-53.711\\
-53.711	-35.4\\
-35.4	-46.387\\
-46.387	-74.463\\
-74.463	-118.408\\
-118.408	-118.408\\
-118.408	-106.201\\
-106.201	-85.449\\
-85.449	-97.656\\
-97.656	-115.967\\
-115.967	-86.67\\
-86.67	-89.111\\
-89.111	-91.553\\
-91.553	-64.697\\
-64.697	-56.152\\
-56.152	-72.021\\
-72.021	-48.828\\
-48.828	-37.842\\
-37.842	-52.49\\
-52.49	-39.063\\
-39.063	-47.607\\
-47.607	-41.504\\
-41.504	-51.27\\
-51.27	-65.918\\
-65.918	-52.49\\
-52.49	-41.504\\
-41.504	-50.049\\
-50.049	-57.373\\
-57.373	-65.918\\
-65.918	-56.152\\
-56.152	-46.387\\
-46.387	-70.801\\
-70.801	-65.918\\
-65.918	-47.607\\
-47.607	-78.125\\
-78.125	-67.139\\
-67.139	-53.711\\
-53.711	-37.842\\
-37.842	-29.297\\
-29.297	-23.193\\
-23.193	-41.504\\
-41.504	-36.621\\
-36.621	-31.738\\
-31.738	-24.414\\
-24.414	-31.738\\
-31.738	-46.387\\
-46.387	-51.27\\
-51.27	-63.477\\
-63.477	-73.242\\
-73.242	-72.021\\
-72.021	-74.463\\
-74.463	-47.607\\
-47.607	-23.193\\
-23.193	-35.4\\
-35.4	-26.855\\
-26.855	-37.842\\
-37.842	-52.49\\
-52.49	-58.594\\
-58.594	-85.449\\
-85.449	-56.152\\
-56.152	-57.373\\
-57.373	-74.463\\
-74.463	-57.373\\
-57.373	-41.504\\
-41.504	-34.18\\
-34.18	-45.166\\
-45.166	-54.932\\
-54.932	-79.346\\
-79.346	-53.711\\
-53.711	-47.607\\
-47.607	-45.166\\
-45.166	-50.049\\
-50.049	-68.359\\
-68.359	-103.76\\
-103.76	-89.111\\
-89.111	-90.332\\
-90.332	-59.814\\
-59.814	-43.945\\
-43.945	-48.828\\
-48.828	-21.973\\
-21.973	-39.063\\
-39.063	-30.518\\
-30.518	-36.621\\
-36.621	-58.594\\
-58.594	-76.904\\
-76.904	-86.67\\
-86.67	-111.084\\
-111.084	-93.994\\
-93.994	-53.711\\
-53.711	-53.711\\
-53.711	-42.725\\
-42.725	-57.373\\
-57.373	-73.242\\
-73.242	-95.215\\
-95.215	-72.021\\
-72.021	-52.49\\
-52.49	-59.814\\
-59.814	-75.684\\
-75.684	-54.932\\
-54.932	-40.283\\
-40.283	-35.4\\
-35.4	-25.635\\
-25.635	-24.414\\
-24.414	-19.531\\
-19.531	-32.959\\
-32.959	-41.504\\
-41.504	-46.387\\
-46.387	-36.621\\
-36.621	-35.4\\
-35.4	-19.531\\
-19.531	-13.428\\
-13.428	-20.752\\
-20.752	-29.297\\
-29.297	-39.063\\
-39.063	-48.828\\
-48.828	-52.49\\
-52.49	-62.256\\
-62.256	-79.346\\
-79.346	-107.422\\
-107.422	-103.76\\
-103.76	-112.305\\
-112.305	-98.877\\
-98.877	-61.035\\
-61.035	-54.932\\
-54.932	-53.711\\
-53.711	-69.58\\
-69.58	-59.814\\
-59.814	-45.166\\
-45.166	-36.621\\
-36.621	-31.738\\
-31.738	-50.049\\
-50.049	-47.607\\
-47.607	-28.076\\
-28.076	-21.973\\
-21.973	-32.959\\
-32.959	-43.945\\
-43.945	-32.959\\
-32.959	-46.387\\
-46.387	-36.621\\
-36.621	-51.27\\
-51.27	-76.904\\
-76.904	-61.035\\
-61.035	-80.566\\
-80.566	-104.98\\
-104.98	-113.525\\
-113.525	-87.891\\
-87.891	-59.814\\
-59.814	-45.166\\
-45.166	-31.738\\
-31.738	-40.283\\
-40.283	-30.518\\
-30.518	-54.932\\
-54.932	-47.607\\
-47.607	-40.283\\
-40.283	-51.27\\
-51.27	-32.959\\
-32.959	-45.166\\
-45.166	-34.18\\
-34.18	-23.193\\
-23.193	-25.635\\
-25.635	-21.973\\
-21.973	-23.193\\
-23.193	-24.414\\
-24.414	-17.09\\
-17.09	-21.973\\
-21.973	-29.297\\
-29.297	-30.518\\
-30.518	-29.297\\
-29.297	-43.945\\
-43.945	-54.932\\
-54.932	-45.166\\
-45.166	-42.725\\
-42.725	-62.256\\
-62.256	-74.463\\
-74.463	-59.814\\
-59.814	-65.918\\
-65.918	-32.959\\
-32.959	-21.973\\
-21.973	-40.283\\
-40.283	-56.152\\
-56.152	-58.594\\
-58.594	-83.008\\
-83.008	-74.463\\
-74.463	-53.711\\
-53.711	-40.283\\
-40.283	-42.725\\
-42.725	-46.387\\
-46.387	-36.621\\
-36.621	-24.414\\
-24.414	-32.959\\
-32.959	-25.635\\
-25.635	-21.973\\
-21.973	-18.311\\
-18.311	-28.076\\
-28.076	-40.283\\
-40.283	-43.945\\
-43.945	-31.738\\
-31.738	-52.49\\
-52.49	-70.801\\
-70.801	-47.607\\
-47.607	-64.697\\
-64.697	-70.801\\
-70.801	-51.27\\
-51.27	-34.18\\
-34.18	-42.725\\
-42.725	-47.607\\
-47.607	-29.297\\
-29.297	-34.18\\
-34.18	-28.076\\
-28.076	-19.531\\
-19.531	-19.531\\
-19.531	-29.297\\
-29.297	-36.621\\
-36.621	-32.959\\
-32.959	-39.063\\
-39.063	-42.725\\
-42.725	-32.959\\
-32.959	-20.752\\
-20.752	-18.311\\
-18.311	-21.973\\
-21.973	-25.635\\
-25.635	-29.297\\
-29.297	-54.932\\
-54.932	-53.711\\
-53.711	-47.607\\
-47.607	-47.607\\
-47.607	-61.035\\
-61.035	-76.904\\
-76.904	-83.008\\
-83.008	-85.449\\
-85.449	-64.697\\
-64.697	-65.918\\
-65.918	-67.139\\
-67.139	-69.58\\
-69.58	-76.904\\
-76.904	-101.318\\
-101.318	-86.67\\
-86.67	-79.346\\
-79.346	-63.477\\
-63.477	-63.477\\
-63.477	-59.814\\
-59.814	-70.801\\
-70.801	-46.387\\
-46.387	-53.711\\
-53.711	-62.256\\
-62.256	-72.021\\
-72.021	-56.152\\
-56.152	-65.918\\
-65.918	-52.49\\
-52.49	-48.828\\
-48.828	-32.959\\
-32.959	-51.27\\
-51.27	-73.242\\
-73.242	-57.373\\
-57.373	-36.621\\
-36.621	-43.945\\
-43.945	-28.076\\
-28.076	-25.635\\
-25.635	-32.959\\
-32.959	-23.193\\
-23.193	-25.635\\
-25.635	-28.076\\
-28.076	-19.531\\
-19.531	-29.297\\
-29.297	-25.635\\
-25.635	-19.531\\
-19.531	-28.076\\
-28.076	-31.738\\
-31.738	-35.4\\
-35.4	-36.621\\
-36.621	-35.4\\
-35.4	-46.387\\
-46.387	-40.283\\
-40.283	-47.607\\
-47.607	-75.684\\
-75.684	-56.152\\
-56.152	-50.049\\
-50.049	-52.49\\
-52.49	-40.283\\
-40.283	-26.855\\
-26.855	-42.725\\
-42.725	-35.4\\
-35.4	-25.635\\
-25.635	-37.842\\
-37.842	-36.621\\
-36.621	-42.725\\
-42.725	-30.518\\
-30.518	-21.973\\
-21.973	-18.311\\
-18.311	-20.752\\
-20.752	-34.18\\
-34.18	-31.738\\
-31.738	-35.4\\
-35.4	-47.607\\
-47.607	-63.477\\
-63.477	-48.828\\
-48.828	-63.477\\
-63.477	-74.463\\
-74.463	-64.697\\
-64.697	-59.814\\
-59.814	-97.656\\
-97.656	-70.801\\
-70.801	-86.67\\
-86.67	-76.904\\
-76.904	-86.67\\
-86.67	-95.215\\
-95.215	-61.035\\
-61.035	-39.063\\
-39.063	-32.959\\
-32.959	-47.607\\
-47.607	-57.373\\
-57.373	-58.594\\
-58.594	-41.504\\
-41.504	-25.635\\
-25.635	-34.18\\
-34.18	-57.373\\
-57.373	-76.904\\
-76.904	-75.684\\
-75.684	-48.828\\
-48.828	-39.063\\
-39.063	-48.828\\
-48.828	-72.021\\
-72.021	-81.787\\
-81.787	-83.008\\
-83.008	-61.035\\
-61.035	-83.008\\
-83.008	-90.332\\
-90.332	-86.67\\
-86.67	-114.746\\
-114.746	-98.877\\
-98.877	-85.449\\
-85.449	-98.877\\
-98.877	-84.229\\
-84.229	-51.27\\
-51.27	-47.607\\
-47.607	-56.152\\
-56.152	-67.139\\
-67.139	-67.139\\
-67.139	-51.27\\
-51.27	-63.477\\
-63.477	-56.152\\
-56.152	-43.945\\
-43.945	-54.932\\
-54.932	-52.49\\
-52.49	-75.684\\
-75.684	-76.904\\
-76.904	-57.373\\
-57.373	-45.166\\
-45.166	-30.518\\
-30.518	-43.945\\
-43.945	-35.4\\
-35.4	-31.738\\
-31.738	-26.855\\
-26.855	-42.725\\
-42.725	-53.711\\
-53.711	-58.594\\
-58.594	-58.594\\
-58.594	-37.842\\
-37.842	-30.518\\
-30.518	-23.193\\
-23.193	-28.076\\
-28.076	-40.283\\
-40.283	-42.725\\
-42.725	-40.283\\
-40.283	-62.256\\
-62.256	-73.242\\
-73.242	-80.566\\
-80.566	-81.787\\
-81.787	-95.215\\
-95.215	-59.814\\
-59.814	-47.607\\
-47.607	-40.283\\
-40.283	-43.945\\
-43.945	-54.932\\
-54.932	-46.387\\
-46.387	-34.18\\
-34.18	-30.518\\
-30.518	-52.49\\
-52.49	-62.256\\
-62.256	-57.373\\
-57.373	-85.449\\
-85.449	-91.553\\
-91.553	-65.918\\
-65.918	-47.607\\
-47.607	-36.621\\
-36.621	-32.959\\
-32.959	-56.152\\
-56.152	-48.828\\
-48.828	-53.711\\
-53.711	-59.814\\
-59.814	-67.139\\
-67.139	-63.477\\
-63.477	-72.021\\
-72.021	-51.27\\
-51.27	-56.152\\
-56.152	-42.725\\
-42.725	-39.063\\
-39.063	-54.932\\
-54.932	-75.684\\
-75.684	-81.787\\
-81.787	-47.607\\
-47.607	-91.553\\
-91.553	-78.125\\
-78.125	-53.711\\
-53.711	-34.18\\
-34.18	-50.049\\
-50.049	-45.166\\
-45.166	-45.166\\
-45.166	-53.711\\
-53.711	-37.842\\
-37.842	-48.828\\
-48.828	-86.67\\
-86.67	-91.553\\
-91.553	-63.477\\
-63.477	-70.801\\
-70.801	-78.125\\
-78.125	-76.904\\
-76.904	-57.373\\
-57.373	-56.152\\
-56.152	-41.504\\
-41.504	-54.932\\
-54.932	-59.814\\
-59.814	-92.773\\
-92.773	-101.318\\
-101.318	-131.836\\
-131.836	-90.332\\
-90.332	-76.904\\
-76.904	-62.256\\
-62.256	-67.139\\
-67.139	-52.49\\
-52.49	-37.842\\
-37.842	-37.842\\
-37.842	-39.063\\
-39.063	-31.738\\
-31.738	-23.193\\
-23.193	-34.18\\
-34.18	-45.166\\
-45.166	-29.297\\
-29.297	-25.635\\
-25.635	-24.414\\
-24.414	-50.049\\
-50.049	-67.139\\
-67.139	-35.4\\
-35.4	-41.504\\
-41.504	-43.945\\
-43.945	-40.283\\
-40.283	-47.607\\
-47.607	-45.166\\
-45.166	-32.959\\
-32.959	-25.635\\
-25.635	-23.193\\
-23.193	-26.855\\
-26.855	-47.607\\
-47.607	-54.932\\
-54.932	-41.504\\
-41.504	-30.518\\
-30.518	-23.193\\
-23.193	-28.076\\
-28.076	-42.725\\
-42.725	-51.27\\
-51.27	-56.152\\
-56.152	-40.283\\
-40.283	-39.063\\
-39.063	-42.725\\
-42.725	-57.373\\
-57.373	-76.904\\
-76.904	-89.111\\
-89.111	-69.58\\
-69.58	-46.387\\
-46.387	-48.828\\
-48.828	-46.387\\
-46.387	-46.387\\
-46.387	-35.4\\
-35.4	-39.063\\
-39.063	-43.945\\
-43.945	-41.504\\
-41.504	-29.297\\
-29.297	-20.752\\
-20.752	-29.297\\
-29.297	-40.283\\
-40.283	-57.373\\
-57.373	-61.035\\
-61.035	-74.463\\
-74.463	-65.918\\
-65.918	-73.242\\
-73.242	-96.436\\
-96.436	-126.953\\
-126.953	-106.201\\
-106.201	-73.242\\
-73.242	-80.566\\
-80.566	-91.553\\
-91.553	-122.07\\
-122.07	-80.566\\
-80.566	-43.945\\
-43.945	-30.518\\
-30.518	-31.738\\
-31.738	-25.635\\
-25.635	-18.311\\
-18.311	-21.973\\
-21.973	-29.297\\
-29.297	-37.842\\
-37.842	-37.842\\
-37.842	-30.518\\
-30.518	-29.297\\
-29.297	-36.621\\
-36.621	-43.945\\
-43.945	-45.166\\
-45.166	-42.725\\
-42.725	-32.959\\
-32.959	-24.414\\
-24.414	-23.193\\
-23.193	-35.4\\
-35.4	-39.063\\
-39.063	-30.518\\
-30.518	-23.193\\
-23.193	-35.4\\
-35.4	-28.076\\
-28.076	-30.518\\
-30.518	-37.842\\
-37.842	-50.049\\
-50.049	-54.932\\
-54.932	-40.283\\
-40.283	-56.152\\
-56.152	-54.932\\
-54.932	-46.387\\
-46.387	-37.842\\
-37.842	-58.594\\
-58.594	-73.242\\
-73.242	-73.242\\
-73.242	-79.346\\
-79.346	-62.256\\
-62.256	-57.373\\
-57.373	-54.932\\
-54.932	-72.021\\
-72.021	-59.814\\
-59.814	-78.125\\
-78.125	-70.801\\
-70.801	-74.463\\
-74.463	-124.512\\
-124.512	-98.877\\
-98.877	-53.711\\
-53.711	-37.842\\
-37.842	-41.504\\
-41.504	-51.27\\
-51.27	-63.477\\
-63.477	-70.801\\
-70.801	-78.125\\
-78.125	-80.566\\
-80.566	-54.932\\
-54.932	-47.607\\
-47.607	-56.152\\
-56.152	-56.152\\
-56.152	-36.621\\
-36.621	-28.076\\
-28.076	-45.166\\
-45.166	-46.387\\
-46.387	-59.814\\
-59.814	-73.242\\
-73.242	-64.697\\
-64.697	-47.607\\
-47.607	-40.283\\
-40.283	-57.373\\
-57.373	-70.801\\
-70.801	-91.553\\
-91.553	-100.098\\
-100.098	-113.525\\
-113.525	-92.773\\
-92.773	-85.449\\
-85.449	-52.49\\
-52.49	-40.283\\
-40.283	-65.918\\
-65.918	-85.449\\
-85.449	-52.49\\
-52.49	-74.463\\
-74.463	-104.98\\
-104.98	-119.629\\
-119.629	-79.346\\
-79.346	-46.387\\
-46.387	-30.518\\
-30.518	-40.283\\
-40.283	-37.842\\
-37.842	-41.504\\
-41.504	-29.297\\
-29.297	-35.4\\
-35.4	-30.518\\
-30.518	-35.4\\
-35.4	-31.738\\
-31.738	-34.18\\
-34.18	-53.711\\
-53.711	-50.049\\
-50.049	-67.139\\
-67.139	-80.566\\
-80.566	-79.346\\
-79.346	-46.387\\
-46.387	-47.607\\
-47.607	-57.373\\
-57.373	-40.283\\
-40.283	-35.4\\
-35.4	-28.076\\
-28.076	-41.504\\
-41.504	-35.4\\
-35.4	-23.193\\
-23.193	-14.648\\
-14.648	-17.09\\
-17.09	-26.855\\
-26.855	-37.842\\
-37.842	-40.283\\
-40.283	-28.076\\
-28.076	-31.738\\
-31.738	-46.387\\
-46.387	-57.373\\
-57.373	-41.504\\
-41.504	-41.504\\
-41.504	-48.828\\
-48.828	-54.932\\
-54.932	-50.049\\
-50.049	-57.373\\
-57.373	-53.711\\
-53.711	-40.283\\
-40.283	-31.738\\
-31.738	-42.725\\
-42.725	-36.621\\
-36.621	-42.725\\
-42.725	-57.373\\
-57.373	-83.008\\
-83.008	-63.477\\
-63.477	-57.373\\
-57.373	-85.449\\
-85.449	-68.359\\
-68.359	-41.504\\
-41.504	-34.18\\
-34.18	-28.076\\
-28.076	-28.076\\
-28.076	-26.855\\
-26.855	-25.635\\
-25.635	-45.166\\
-45.166	-28.076\\
-28.076	-23.193\\
-23.193	-32.959\\
-32.959	-41.504\\
-41.504	-32.959\\
-32.959	-25.635\\
-25.635	-17.09\\
-17.09	-25.635\\
-25.635	-46.387\\
-46.387	-59.814\\
-59.814	-50.049\\
-50.049	-40.283\\
-40.283	-45.166\\
-45.166	-46.387\\
-46.387	-50.049\\
-50.049	-61.035\\
-61.035	-70.801\\
-70.801	-68.359\\
-68.359	-58.594\\
-58.594	-41.504\\
-41.504	-34.18\\
-34.18	-36.621\\
-36.621	-48.828\\
-48.828	-35.4\\
-35.4	-28.076\\
-28.076	-50.049\\
-50.049	-61.035\\
-61.035	-57.373\\
-57.373	-65.918\\
-65.918	-83.008\\
-83.008	-48.828\\
-48.828	-81.787\\
-81.787	-76.904\\
-76.904	-73.242\\
-73.242	-76.904\\
-76.904	-76.904\\
-76.904	-122.07\\
-122.07	-119.629\\
-119.629	-79.346\\
-79.346	-93.994\\
-93.994	-115.967\\
-115.967	-112.305\\
-112.305	-124.512\\
-124.512	-113.525\\
-113.525	-142.822\\
-142.822	-115.967\\
-115.967	-79.346\\
-79.346	-53.711\\
-53.711	-52.49\\
-52.49	-45.166\\
-45.166	-37.842\\
-37.842	-52.49\\
-52.49	-73.242\\
-73.242	-45.166\\
-45.166	-41.504\\
-41.504	-42.725\\
-42.725	-31.738\\
-31.738	-34.18\\
-34.18	-25.635\\
-25.635	-24.414\\
-24.414	-23.193\\
-23.193	-29.297\\
-29.297	-24.414\\
-24.414	-20.752\\
-20.752	-17.09\\
-17.09	-13.428\\
-13.428	-24.414\\
-24.414	-20.752\\
-20.752	-18.311\\
-18.311	-37.842\\
-37.842	-51.27\\
-51.27	-50.049\\
-50.049	-40.283\\
-40.283	-48.828\\
-48.828	-67.139\\
-67.139	-34.18\\
-34.18	-24.414\\
-24.414	-19.531\\
-19.531	-14.648\\
-14.648	-20.752\\
-20.752	-36.621\\
-36.621	-50.049\\
-50.049	-36.621\\
-36.621	-47.607\\
-47.607	-74.463\\
-74.463	-73.242\\
-73.242	-53.711\\
-53.711	-37.842\\
-37.842	-43.945\\
-43.945	-59.814\\
-59.814	-73.242\\
-73.242	-46.387\\
-46.387	-25.635\\
-25.635	-36.621\\
-36.621	-34.18\\
-34.18	-17.09\\
-17.09	-12.207\\
-12.207	-26.855\\
-26.855	-53.711\\
-53.711	-53.711\\
-53.711	-43.945\\
-43.945	-29.297\\
-29.297	-28.076\\
-28.076	-17.09\\
-17.09	-18.311\\
-18.311	-21.973\\
-21.973	-26.855\\
-26.855	-36.621\\
-36.621	-54.932\\
-54.932	-57.373\\
-57.373	-59.814\\
-59.814	-63.477\\
-63.477	-62.256\\
-62.256	-59.814\\
-59.814	-53.711\\
-53.711	-64.697\\
-64.697	-93.994\\
-93.994	-86.67\\
-86.67	-50.049\\
-50.049	-29.297\\
-29.297	-52.49\\
-52.49	-67.139\\
-67.139	-59.814\\
-59.814	-37.842\\
-37.842	-37.842\\
-37.842	-59.814\\
-59.814	-89.111\\
-89.111	-89.111\\
-89.111	-98.877\\
-98.877	-97.656\\
-97.656	-73.242\\
-73.242	-76.904\\
-76.904	-43.945\\
-43.945	-36.621\\
-36.621	-45.166\\
-45.166	-54.932\\
-54.932	-57.373\\
-57.373	-61.035\\
-61.035	-42.725\\
-42.725	-51.27\\
-51.27	-48.828\\
-48.828	-29.297\\
-29.297	-21.973\\
-21.973	-17.09\\
-17.09	-25.635\\
-25.635	-23.193\\
-23.193	-13.428\\
-13.428	-25.635\\
-25.635	-46.387\\
-46.387	-35.4\\
-35.4	-30.518\\
-30.518	-40.283\\
-40.283	-61.035\\
-61.035	-48.828\\
-48.828	-24.414\\
-24.414	-18.311\\
-18.311	-32.959\\
-32.959	-68.359\\
-68.359	-81.787\\
-81.787	-109.863\\
-109.863	-130.615\\
-130.615	-126.953\\
-126.953	-87.891\\
-87.891	-87.891\\
-87.891	-109.863\\
-109.863	-95.215\\
-95.215	-89.111\\
-89.111	-98.877\\
-98.877	-107.422\\
-107.422	-108.643\\
-108.643	-145.264\\
-145.264	-102.539\\
-102.539	-57.373\\
-57.373	-34.18\\
-34.18	-29.297\\
-29.297	-32.959\\
-32.959	-32.959\\
-32.959	-31.738\\
-31.738	-53.711\\
-53.711	-45.166\\
-45.166	-45.166\\
-45.166	-59.814\\
-59.814	-34.18\\
-34.18	-37.842\\
-37.842	-51.27\\
-51.27	-57.373\\
-57.373	-35.4\\
-35.4	-24.414\\
-24.414	-31.738\\
-31.738	-31.738\\
-31.738	-58.594\\
-58.594	-91.553\\
-91.553	-86.67\\
-86.67	-96.436\\
-96.436	-111.084\\
-111.084	-140.381\\
-140.381	-120.85\\
-120.85	-80.566\\
-80.566	-54.932\\
-54.932	-34.18\\
-34.18	-39.063\\
-39.063	-39.063\\
-39.063	-47.607\\
-47.607	-36.621\\
-36.621	-32.959\\
-32.959	-36.621\\
-36.621	-43.945\\
-43.945	-47.607\\
-47.607	-58.594\\
-58.594	-45.166\\
-45.166	-23.193\\
-23.193	-18.311\\
-18.311	-18.311\\
-18.311	-17.09\\
-17.09	-21.973\\
-21.973	-32.959\\
-32.959	-35.4\\
-35.4	-42.725\\
-42.725	-57.373\\
-57.373	-45.166\\
-45.166	-63.477\\
-63.477	-97.656\\
-97.656	-78.125\\
-78.125	-68.359\\
-68.359	-84.229\\
-84.229	-96.436\\
-96.436	-62.256\\
-62.256	-58.594\\
-58.594	-40.283\\
-40.283	-41.504\\
-41.504	-76.904\\
-76.904	-124.512\\
-124.512	-147.705\\
-147.705	-115.967\\
-115.967	-98.877\\
-98.877	-75.684\\
-75.684	-80.566\\
-80.566	-103.76\\
-103.76	-63.477\\
-63.477	-51.27\\
-51.27	-41.504\\
-41.504	-72.021\\
-72.021	-72.021\\
-72.021	-40.283\\
-40.283	-41.504\\
-41.504	-34.18\\
-34.18	-50.049\\
-50.049	-76.904\\
-76.904	-75.684\\
-75.684	-58.594\\
-58.594	-46.387\\
-46.387	-52.49\\
-52.49	-41.504\\
-41.504	-30.518\\
-30.518	-26.855\\
-26.855	-24.414\\
-24.414	-20.752\\
-20.752	-23.193\\
-23.193	-37.842\\
-37.842	-51.27\\
-51.27	-51.27\\
-51.27	-31.738\\
-31.738	-29.297\\
-29.297	-26.855\\
-26.855	-32.959\\
-32.959	-45.166\\
-45.166	-47.607\\
-47.607	-45.166\\
-45.166	-28.076\\
-28.076	-47.607\\
-47.607	-69.58\\
-69.58	-53.711\\
-53.711	-26.855\\
-26.855	-26.855\\
-26.855	-41.504\\
-41.504	-42.725\\
-42.725	-37.842\\
-37.842	-26.855\\
-26.855	-41.504\\
-41.504	-62.256\\
-62.256	-41.504\\
-41.504	-17.09\\
-17.09	-25.635\\
-25.635	-51.27\\
-51.27	-54.932\\
-54.932	-39.063\\
-39.063	-32.959\\
-32.959	-58.594\\
-58.594	-76.904\\
-76.904	-64.697\\
-64.697	-87.891\\
-87.891	-108.643\\
-108.643	-72.021\\
-72.021	-59.814\\
-59.814	-80.566\\
-80.566	-54.932\\
-54.932	-73.242\\
-73.242	-87.891\\
-87.891	-69.58\\
-69.58	-36.621\\
-36.621	-58.594\\
-58.594	-100.098\\
-100.098	-114.746\\
-114.746	-83.008\\
-83.008	-72.021\\
-72.021	-92.773\\
-92.773	-74.463\\
-74.463	-48.828\\
-48.828	-45.166\\
-45.166	-57.373\\
-57.373	-58.594\\
-58.594	-57.373\\
-57.373	-64.697\\
-64.697	-48.828\\
-48.828	-50.049\\
-50.049	-68.359\\
-68.359	-43.945\\
-43.945	-35.4\\
-35.4	-30.518\\
-30.518	-23.193\\
-23.193	-26.855\\
-26.855	-46.387\\
-46.387	-41.504\\
-41.504	-57.373\\
-57.373	-69.58\\
-69.58	-83.008\\
-83.008	-63.477\\
-63.477	-57.373\\
-57.373	-25.635\\
-25.635	-47.607\\
-47.607	-72.021\\
-72.021	-51.27\\
-51.27	-26.855\\
-26.855	-51.27\\
-51.27	-74.463\\
-74.463	-74.463\\
-74.463	-92.773\\
-92.773	-120.85\\
-120.85	-87.891\\
-87.891	-74.463\\
-74.463	-86.67\\
-86.67	-61.035\\
-61.035	-47.607\\
-47.607	-53.711\\
-53.711	-68.359\\
-68.359	-72.021\\
-72.021	-72.021\\
-72.021	-45.166\\
-45.166	-39.063\\
-39.063	-32.959\\
-32.959	-46.387\\
-46.387	-43.945\\
-43.945	-34.18\\
-34.18	-63.477\\
-63.477	-68.359\\
-68.359	-50.049\\
-50.049	-40.283\\
-40.283	-58.594\\
-58.594	-32.959\\
-32.959	-21.973\\
-21.973	-28.076\\
-28.076	-35.4\\
-35.4	-34.18\\
-34.18	-25.635\\
-25.635	-18.311\\
-18.311	-28.076\\
-28.076	-37.842\\
-37.842	-53.711\\
-53.711	-59.814\\
-59.814	-75.684\\
-75.684	-86.67\\
-86.67	-48.828\\
-48.828	-39.063\\
-39.063	-76.904\\
-76.904	-109.863\\
-109.863	-83.008\\
-83.008	-78.125\\
-78.125	-113.525\\
-113.525	-92.773\\
-92.773	-74.463\\
-74.463	-57.373\\
-57.373	-58.594\\
-58.594	-75.684\\
-75.684	-52.49\\
-52.49	-45.166\\
-45.166	-75.684\\
-75.684	-95.215\\
-95.215	-96.436\\
-96.436	-48.828\\
-48.828	-25.635\\
-25.635	-56.152\\
-56.152	-47.607\\
-47.607	-21.973\\
-21.973	-20.752\\
-20.752	-31.738\\
-31.738	-41.504\\
-41.504	-67.139\\
-67.139	-50.049\\
-50.049	-39.063\\
-39.063	-28.076\\
-28.076	-19.531\\
-19.531	-25.635\\
-25.635	-23.193\\
-23.193	-26.855\\
-26.855	-42.725\\
-42.725	-39.063\\
-39.063	-24.414\\
-24.414	-20.752\\
-20.752	-26.855\\
-26.855	-30.518\\
-30.518	-20.752\\
-20.752	-31.738\\
-31.738	-24.414\\
-24.414	-19.531\\
-19.531	-36.621\\
-36.621	-47.607\\
-47.607	-69.58\\
-69.58	-92.773\\
-92.773	-81.787\\
-81.787	-42.725\\
-42.725	-32.959\\
-32.959	-34.18\\
-34.18	-21.973\\
-21.973	-21.973\\
-21.973	-35.4\\
-35.4	-39.063\\
-39.063	-37.842\\
-37.842	-42.725\\
-42.725	-48.828\\
-48.828	-52.49\\
-52.49	-52.49\\
-52.49	-62.256\\
-62.256	-62.256\\
-62.256	-86.67\\
-86.67	-47.607\\
-47.607	-25.635\\
-25.635	-24.414\\
-24.414	-42.725\\
-42.725	-31.738\\
-31.738	-34.18\\
-34.18	-18.311\\
-18.311	-28.076\\
-28.076	-18.311\\
-18.311	-13.428\\
-13.428	-18.311\\
-18.311	-31.738\\
-31.738	-36.621\\
-36.621	-51.27\\
-51.27	-69.58\\
-69.58	-52.49\\
-52.49	-25.635\\
-25.635	-39.063\\
-39.063	-45.166\\
-45.166	-25.635\\
-25.635	-32.959\\
-32.959	-58.594\\
-58.594	-50.049\\
-50.049	-79.346\\
-79.346	-92.773\\
-92.773	-64.697\\
-64.697	-51.27\\
-51.27	-46.387\\
};
\addplot [color=mycolor2, line width=2.0pt, forget plot]
  table[row sep=crcr]{%
-54.932	-53.05064253794\\
-63.477	-61.3029861716452\\
-75.684	-73.0919105410589\\
-62.256	-60.1238040093569\\
-59.814	-57.7654396847801\\
-80.566	-77.8067076878991\\
-76.904	-74.2701269521906\\
-90.332	-87.2382334838927\\
-68.359	-66.0177833184854\\
-40.283	-38.9033538439495\\
-48.828	-47.1556974776548\\
-46.387	-44.7982989042347\\
-54.932	-53.05064253794\\
-45.166	-43.6191167419463\\
-21.973	-21.2204501654073\\
-17.09	-16.5046872674105\\
-18.311	-17.6838694296989\\
-40.283	-38.9033538439495\\
-62.256	-60.1238040093569\\
-67.139	-64.8395669073537\\
-69.58	-67.1969654807737\\
-52.49	-50.6922782133632\\
-34.18	-33.009374534821\\
-64.697	-62.4812025827769\\
-52.49	-50.6922782133632\\
-39.063	-37.7251374328178\\
-61.035	-58.9446218470685\\
-53.711	-51.8714603756516\\
-45.166	-43.6191167419463\\
-73.242	-70.7335462164822\\
-67.139	-64.8395669073537\\
-52.49	-50.6922782133632\\
-75.684	-73.0919105410589\\
-74.463	-71.9127283787705\\
-58.594	-56.5872232736484\\
-50.049	-48.3348796399432\\
-41.504	-40.0825360062379\\
-47.607	-45.9765153153664\\
-58.594	-56.5872232736484\\
-54.932	-53.05064253794\\
-59.814	-57.7654396847801\\
-65.918	-63.6603847450653\\
-85.449	-82.5224705858959\\
-76.904	-74.2701269521906\\
-54.932	-53.05064253794\\
-50.049	-48.3348796399432\\
-35.4	-34.1875909459527\\
-34.18	-33.009374534821\\
-45.166	-43.6191167419463\\
-34.18	-33.009374534821\\
-36.621	-35.3667731082411\\
-51.27	-49.5140618022315\\
-54.932	-53.05064253794\\
-51.27	-49.5140618022315\\
-78.125	-75.449309114479\\
-62.256	-60.1238040093569\\
-36.621	-35.3667731082411\\
-30.518	-29.4727937991126\\
-40.283	-38.9033538439495\\
-46.387	-44.7982989042347\\
-29.297	-28.2936116368242\\
-36.621	-35.3667731082411\\
-31.738	-30.6510102102443\\
-26.855	-25.9352473122475\\
-35.4	-34.1875909459527\\
-40.283	-38.9033538439495\\
-58.594	-56.5872232736484\\
-56.152	-54.2288589490717\\
-65.918	-63.6603847450653\\
-61.035	-58.9446218470685\\
-70.801	-68.3761476430621\\
-57.373	-55.40804111136\\
-58.594	-56.5872232736484\\
-51.27	-49.5140618022315\\
-50.049	-48.3348796399432\\
-48.828	-47.1556974776548\\
-80.566	-77.8067076878991\\
-111.084	-107.279501487012\\
-114.746	-110.81608222272\\
-112.305	-108.4586836493\\
-74.463	-71.9127283787705\\
-100.098	-96.6697592798863\\
-125.732	-121.425824429845\\
-128.174	-123.784188754422\\
-96.436	-93.1331785441779\\
-124.512	-120.247608018714\\
-158.691	-153.256016802378\\
-115.967	-111.995264385008\\
-97.656	-94.3113949553096\\
-68.359	-66.0177833184854\\
-53.711	-51.8714603756516\\
-47.607	-45.9765153153664\\
-56.152	-54.2288589490717\\
-45.166	-43.6191167419463\\
-30.518	-29.4727937991126\\
-36.621	-35.3667731082411\\
-50.049	-48.3348796399432\\
-48.828	-47.1556974776548\\
-47.607	-45.9765153153664\\
-62.256	-60.1238040093569\\
-64.697	-62.4812025827769\\
-85.449	-82.5224705858959\\
-72.021	-69.5543640541938\\
-85.449	-82.5224705858959\\
-63.477	-61.3029861716452\\
-54.932	-53.05064253794\\
-64.697	-62.4812025827769\\
-48.828	-47.1556974776548\\
-43.945	-42.439934579658\\
-53.711	-51.8714603756516\\
-54.932	-53.05064253794\\
-40.283	-38.9033538439495\\
-43.945	-42.439934579658\\
-32.959	-31.8301923725327\\
-36.621	-35.3667731082411\\
-40.283	-38.9033538439495\\
-30.518	-29.4727937991126\\
-34.18	-33.009374534821\\
-43.945	-42.439934579658\\
-63.477	-61.3029861716452\\
-65.918	-63.6603847450653\\
-72.021	-69.5543640541938\\
-45.166	-43.6191167419463\\
-56.152	-54.2288589490717\\
-89.111	-86.0590513216043\\
-70.801	-68.3761476430621\\
-81.787	-78.9858898501874\\
-80.566	-77.8067076878991\\
-50.049	-48.3348796399432\\
-36.621	-35.3667731082411\\
-47.607	-45.9765153153664\\
-52.49	-50.6922782133632\\
-81.787	-78.9858898501874\\
-103.76	-100.206340015595\\
-102.539	-99.0271578533064\\
-76.904	-74.2701269521906\\
-80.566	-77.8067076878991\\
-72.021	-69.5543640541938\\
-69.58	-67.1969654807737\\
-68.359	-66.0177833184854\\
-51.27	-49.5140618022315\\
-45.166	-43.6191167419463\\
-39.063	-37.7251374328178\\
-37.842	-36.5459552705295\\
-41.504	-40.0825360062379\\
-56.152	-54.2288589490717\\
-84.229	-81.3442541747642\\
-70.801	-68.3761476430621\\
-50.049	-48.3348796399432\\
-42.725	-41.2617181685263\\
-40.283	-38.9033538439495\\
-28.076	-27.1144294745358\\
-23.193	-22.398666576539\\
-24.414	-23.5778487388274\\
-41.504	-40.0825360062379\\
-32.959	-31.8301923725327\\
-34.18	-33.009374534821\\
-37.842	-36.5459552705295\\
-35.4	-34.1875909459527\\
-26.855	-25.9352473122475\\
-21.973	-21.2204501654073\\
-18.311	-17.6838694296989\\
-25.635	-24.7570309011158\\
-51.27	-49.5140618022315\\
-73.242	-70.7335462164822\\
-78.125	-75.449309114479\\
-54.932	-53.05064253794\\
-39.063	-37.7251374328178\\
-31.738	-30.6510102102443\\
-24.414	-23.5778487388274\\
-41.504	-40.0825360062379\\
-42.725	-41.2617181685263\\
-54.932	-53.05064253794\\
-73.242	-70.7335462164822\\
-108.643	-104.922102913592\\
-125.732	-121.425824429845\\
-103.76	-100.206340015595\\
-101.318	-97.847975691018\\
-67.139	-64.8395669073537\\
-75.684	-73.0919105410589\\
-79.346	-76.6284912767674\\
-86.67	-83.7016527481842\\
-69.58	-67.1969654807737\\
-62.256	-60.1238040093569\\
-61.035	-58.9446218470685\\
-56.152	-54.2288589490717\\
-67.139	-64.8395669073537\\
-52.49	-50.6922782133632\\
-76.904	-74.2701269521906\\
-97.656	-94.3113949553096\\
-72.021	-69.5543640541938\\
-45.166	-43.6191167419463\\
-47.607	-45.9765153153664\\
-78.125	-75.449309114479\\
-93.994	-90.7748142196011\\
-112.305	-108.4586836493\\
-119.629	-115.531845120717\\
-91.553	-88.4174156461811\\
-84.229	-81.3442541747642\\
-96.436	-93.1331785441779\\
-111.084	-107.279501487012\\
-74.463	-71.9127283787705\\
-46.387	-44.7982989042347\\
-62.256	-60.1238040093569\\
-54.932	-53.05064253794\\
-34.18	-33.009374534821\\
-47.607	-45.9765153153664\\
-53.711	-51.8714603756516\\
-42.725	-41.2617181685263\\
-34.18	-33.009374534821\\
-43.945	-42.439934579658\\
-40.283	-38.9033538439495\\
-52.49	-50.6922782133632\\
-65.918	-63.6603847450653\\
-62.256	-60.1238040093569\\
-45.166	-43.6191167419463\\
-67.139	-64.8395669073537\\
-104.98	-101.384556426726\\
-79.346	-76.6284912767674\\
-51.27	-49.5140618022315\\
-40.283	-38.9033538439495\\
-47.607	-45.9765153153664\\
-40.283	-38.9033538439495\\
-31.738	-30.6510102102443\\
-35.4	-34.1875909459527\\
-42.725	-41.2617181685263\\
-51.27	-49.5140618022315\\
-57.373	-55.40804111136\\
-42.725	-41.2617181685263\\
-63.477	-61.3029861716452\\
-75.684	-73.0919105410589\\
-70.801	-68.3761476430621\\
-57.373	-55.40804111136\\
-62.256	-60.1238040093569\\
-81.787	-78.9858898501874\\
-52.49	-50.6922782133632\\
-28.076	-27.1144294745358\\
-36.621	-35.3667731082411\\
-42.725	-41.2617181685263\\
-51.27	-49.5140618022315\\
-46.387	-44.7982989042347\\
-36.621	-35.3667731082411\\
-52.49	-50.6922782133632\\
-37.842	-36.5459552705295\\
-47.607	-45.9765153153664\\
-41.504	-40.0825360062379\\
-37.842	-36.5459552705295\\
-45.166	-43.6191167419463\\
-29.297	-28.2936116368242\\
-31.738	-30.6510102102443\\
-29.297	-28.2936116368242\\
-26.855	-25.9352473122475\\
-47.607	-45.9765153153664\\
-53.711	-51.8714603756516\\
-69.58	-67.1969654807737\\
-89.111	-86.0590513216043\\
-62.256	-60.1238040093569\\
-36.621	-35.3667731082411\\
-31.738	-30.6510102102443\\
-29.297	-28.2936116368242\\
-40.283	-38.9033538439495\\
-31.738	-30.6510102102443\\
-30.518	-29.4727937991126\\
-39.063	-37.7251374328178\\
-61.035	-58.9446218470685\\
-53.711	-51.8714603756516\\
-79.346	-76.6284912767674\\
-54.932	-53.05064253794\\
-48.828	-47.1556974776548\\
-32.959	-31.8301923725327\\
-29.297	-28.2936116368242\\
-23.193	-22.398666576539\\
-25.635	-24.7570309011158\\
-40.283	-38.9033538439495\\
-46.387	-44.7982989042347\\
-48.828	-47.1556974776548\\
-51.27	-49.5140618022315\\
-79.346	-76.6284912767674\\
-73.242	-70.7335462164822\\
-53.711	-51.8714603756516\\
-63.477	-61.3029861716452\\
-69.58	-67.1969654807737\\
-85.449	-82.5224705858959\\
-72.021	-69.5543640541938\\
-47.607	-45.9765153153664\\
-28.076	-27.1144294745358\\
-21.973	-21.2204501654073\\
-26.855	-25.9352473122475\\
-25.635	-24.7570309011158\\
-32.959	-31.8301923725327\\
-47.607	-45.9765153153664\\
-43.945	-42.439934579658\\
-45.166	-43.6191167419463\\
-52.49	-50.6922782133632\\
-35.4	-34.1875909459527\\
-20.752	-20.041268003119\\
-35.4	-34.1875909459527\\
-56.152	-54.2288589490717\\
-54.932	-53.05064253794\\
-59.814	-57.7654396847801\\
-53.711	-51.8714603756516\\
-40.283	-38.9033538439495\\
-51.27	-49.5140618022315\\
-72.021	-69.5543640541938\\
-70.801	-68.3761476430621\\
-51.27	-49.5140618022315\\
-81.787	-78.9858898501874\\
-62.256	-60.1238040093569\\
-63.477	-61.3029861716452\\
-73.242	-70.7335462164822\\
-70.801	-68.3761476430621\\
-54.932	-53.05064253794\\
-47.607	-45.9765153153664\\
-78.125	-75.449309114479\\
-108.643	-104.922102913592\\
-85.449	-82.5224705858959\\
-50.049	-48.3348796399432\\
-51.27	-49.5140618022315\\
-61.035	-58.9446218470685\\
-72.021	-69.5543640541938\\
-54.932	-53.05064253794\\
-56.152	-54.2288589490717\\
-73.242	-70.7335462164822\\
-79.346	-76.6284912767674\\
-56.152	-54.2288589490717\\
-45.166	-43.6191167419463\\
-57.373	-55.40804111136\\
-87.891	-84.8808349104726\\
-78.125	-75.449309114479\\
-80.566	-77.8067076878991\\
-54.932	-53.05064253794\\
-45.166	-43.6191167419463\\
-47.607	-45.9765153153664\\
-31.738	-30.6510102102443\\
-20.752	-20.041268003119\\
-17.09	-16.5046872674105\\
-31.738	-30.6510102102443\\
-48.828	-47.1556974776548\\
-54.932	-53.05064253794\\
-40.283	-38.9033538439495\\
-45.166	-43.6191167419463\\
-52.49	-50.6922782133632\\
-35.4	-34.1875909459527\\
-29.297	-28.2936116368242\\
-31.738	-30.6510102102443\\
-51.27	-49.5140618022315\\
-64.697	-62.4812025827769\\
-41.504	-40.0825360062379\\
-32.959	-31.8301923725327\\
-28.076	-27.1144294745358\\
-32.959	-31.8301923725327\\
-26.855	-25.9352473122475\\
-45.166	-43.6191167419463\\
-80.566	-77.8067076878991\\
-63.477	-61.3029861716452\\
-81.787	-78.9858898501874\\
-98.877	-95.4905771175979\\
-97.656	-94.3113949553096\\
-76.904	-74.2701269521906\\
-58.594	-56.5872232736484\\
-54.932	-53.05064253794\\
-57.373	-55.40804111136\\
-65.918	-63.6603847450653\\
-76.904	-74.2701269521906\\
-100.098	-96.6697592798863\\
-108.643	-104.922102913592\\
-130.615	-126.141587327842\\
-91.553	-88.4174156461811\\
-98.877	-95.4905771175979\\
-108.643	-104.922102913592\\
-85.449	-82.5224705858959\\
-96.436	-93.1331785441779\\
-85.449	-82.5224705858959\\
-51.27	-49.5140618022315\\
-34.18	-33.009374534821\\
-36.621	-35.3667731082411\\
-51.27	-49.5140618022315\\
-43.945	-42.439934579658\\
-31.738	-30.6510102102443\\
-41.504	-40.0825360062379\\
-32.959	-31.8301923725327\\
-23.193	-22.398666576539\\
-31.738	-30.6510102102443\\
-34.18	-33.009374534821\\
-25.635	-24.7570309011158\\
-42.725	-41.2617181685263\\
-58.594	-56.5872232736484\\
-45.166	-43.6191167419463\\
-68.359	-66.0177833184854\\
-100.098	-96.6697592798863\\
-87.891	-84.8808349104726\\
-109.863	-106.100319324723\\
-81.787	-78.9858898501874\\
-62.256	-60.1238040093569\\
-37.842	-36.5459552705295\\
-46.387	-44.7982989042347\\
-42.725	-41.2617181685263\\
-29.297	-28.2936116368242\\
-40.283	-38.9033538439495\\
-21.973	-21.2204501654073\\
-19.531	-18.8620858408306\\
-24.414	-23.5778487388274\\
-45.166	-43.6191167419463\\
-57.373	-55.40804111136\\
-65.918	-63.6603847450653\\
-62.256	-60.1238040093569\\
-69.58	-67.1969654807737\\
-57.373	-55.40804111136\\
-48.828	-47.1556974776548\\
-50.049	-48.3348796399432\\
-43.945	-42.439934579658\\
-59.814	-57.7654396847801\\
-46.387	-44.7982989042347\\
-36.621	-35.3667731082411\\
-51.27	-49.5140618022315\\
-69.58	-67.1969654807737\\
-54.932	-53.05064253794\\
-41.504	-40.0825360062379\\
-37.842	-36.5459552705295\\
-41.504	-40.0825360062379\\
-32.959	-31.8301923725327\\
-30.518	-29.4727937991126\\
-41.504	-40.0825360062379\\
-47.607	-45.9765153153664\\
-52.49	-50.6922782133632\\
-42.725	-41.2617181685263\\
-53.711	-51.8714603756516\\
-51.27	-49.5140618022315\\
-32.959	-31.8301923725327\\
-58.594	-56.5872232736484\\
-80.566	-77.8067076878991\\
-78.125	-75.449309114479\\
-67.139	-64.8395669073537\\
-62.256	-60.1238040093569\\
-50.049	-48.3348796399432\\
-48.828	-47.1556974776548\\
-41.504	-40.0825360062379\\
-54.932	-53.05064253794\\
-50.049	-48.3348796399432\\
-32.959	-31.8301923725327\\
-45.166	-43.6191167419463\\
-52.49	-50.6922782133632\\
-61.035	-58.9446218470685\\
-68.359	-66.0177833184854\\
-87.891	-84.8808349104726\\
-107.422	-103.742920751303\\
-119.629	-115.531845120717\\
-80.566	-77.8067076878991\\
-46.387	-44.7982989042347\\
-32.959	-31.8301923725327\\
-24.414	-23.5778487388274\\
-34.18	-33.009374534821\\
-29.297	-28.2936116368242\\
-51.27	-49.5140618022315\\
-48.828	-47.1556974776548\\
-40.283	-38.9033538439495\\
-52.49	-50.6922782133632\\
-63.477	-61.3029861716452\\
-70.801	-68.3761476430621\\
-75.684	-73.0919105410589\\
-103.76	-100.206340015595\\
-92.773	-89.5956320573128\\
-72.021	-69.5543640541938\\
-51.27	-49.5140618022315\\
-52.49	-50.6922782133632\\
-54.932	-53.05064253794\\
-65.918	-63.6603847450653\\
-51.27	-49.5140618022315\\
-50.049	-48.3348796399432\\
-48.828	-47.1556974776548\\
-30.518	-29.4727937991126\\
-45.166	-43.6191167419463\\
-68.359	-66.0177833184854\\
-48.828	-47.1556974776548\\
-52.49	-50.6922782133632\\
-86.67	-83.7016527481842\\
-63.477	-61.3029861716452\\
-85.449	-82.5224705858959\\
-81.787	-78.9858898501874\\
-53.711	-51.8714603756516\\
-35.4	-34.1875909459527\\
-36.621	-35.3667731082411\\
-61.035	-58.9446218470685\\
-90.332	-87.2382334838927\\
-97.656	-94.3113949553096\\
-100.098	-96.6697592798863\\
-79.346	-76.6284912767674\\
-53.711	-51.8714603756516\\
-46.387	-44.7982989042347\\
-37.842	-36.5459552705295\\
-34.18	-33.009374534821\\
-29.297	-28.2936116368242\\
-40.283	-38.9033538439495\\
-42.725	-41.2617181685263\\
-30.518	-29.4727937991126\\
-42.725	-41.2617181685263\\
-56.152	-54.2288589490717\\
-62.256	-60.1238040093569\\
-78.125	-75.449309114479\\
-96.436	-93.1331785441779\\
-114.746	-110.81608222272\\
-84.229	-81.3442541747642\\
-73.242	-70.7335462164822\\
-76.904	-74.2701269521906\\
-63.477	-61.3029861716452\\
-46.387	-44.7982989042347\\
-57.373	-55.40804111136\\
-90.332	-87.2382334838927\\
-101.318	-97.847975691018\\
-61.035	-58.9446218470685\\
-36.621	-35.3667731082411\\
-26.855	-25.9352473122475\\
-18.311	-17.6838694296989\\
-21.973	-21.2204501654073\\
-30.518	-29.4727937991126\\
-31.738	-30.6510102102443\\
-42.725	-41.2617181685263\\
-37.842	-36.5459552705295\\
-30.518	-29.4727937991126\\
-29.297	-28.2936116368242\\
-26.855	-25.9352473122475\\
-31.738	-30.6510102102443\\
-43.945	-42.439934579658\\
-63.477	-61.3029861716452\\
-68.359	-66.0177833184854\\
-65.918	-63.6603847450653\\
-74.463	-71.9127283787705\\
-91.553	-88.4174156461811\\
-92.773	-89.5956320573128\\
-106.201	-102.563738589015\\
-100.098	-96.6697592798863\\
-74.463	-71.9127283787705\\
-54.932	-53.05064253794\\
-51.27	-49.5140618022315\\
-84.229	-81.3442541747642\\
-97.656	-94.3113949553096\\
-69.58	-67.1969654807737\\
-46.387	-44.7982989042347\\
-26.855	-25.9352473122475\\
-21.973	-21.2204501654073\\
-20.752	-20.041268003119\\
-14.648	-14.1463229428338\\
-25.635	-24.7570309011158\\
-40.283	-38.9033538439495\\
-35.4	-34.1875909459527\\
-24.414	-23.5778487388274\\
-19.531	-18.8620858408306\\
-28.076	-27.1144294745358\\
-30.518	-29.4727937991126\\
-24.414	-23.5778487388274\\
-21.973	-21.2204501654073\\
-32.959	-31.8301923725327\\
-69.58	-67.1969654807737\\
-79.346	-76.6284912767674\\
-86.67	-83.7016527481842\\
-63.477	-61.3029861716452\\
-80.566	-77.8067076878991\\
-114.746	-110.81608222272\\
-103.76	-100.206340015595\\
-106.201	-102.563738589015\\
-81.787	-78.9858898501874\\
-86.67	-83.7016527481842\\
-59.814	-57.7654396847801\\
-56.152	-54.2288589490717\\
-40.283	-38.9033538439495\\
-35.4	-34.1875909459527\\
-62.256	-60.1238040093569\\
-47.607	-45.9765153153664\\
-48.828	-47.1556974776548\\
-53.711	-51.8714603756516\\
-50.049	-48.3348796399432\\
-52.49	-50.6922782133632\\
-57.373	-55.40804111136\\
-64.697	-62.4812025827769\\
-56.152	-54.2288589490717\\
-41.504	-40.0825360062379\\
-52.49	-50.6922782133632\\
-31.738	-30.6510102102443\\
-18.311	-17.6838694296989\\
-25.635	-24.7570309011158\\
-34.18	-33.009374534821\\
-21.973	-21.2204501654073\\
-9.766	-9.43152579599362\\
-21.973	-21.2204501654073\\
-32.959	-31.8301923725327\\
-37.842	-36.5459552705295\\
-43.945	-42.439934579658\\
-39.063	-37.7251374328178\\
-58.594	-56.5872232736484\\
-48.828	-47.1556974776548\\
-36.621	-35.3667731082411\\
-54.932	-53.05064253794\\
-39.063	-37.7251374328178\\
-26.855	-25.9352473122475\\
-23.193	-22.398666576539\\
-17.09	-16.5046872674105\\
-25.635	-24.7570309011158\\
-19.531	-18.8620858408306\\
-29.297	-28.2936116368242\\
-43.945	-42.439934579658\\
-42.725	-41.2617181685263\\
-32.959	-31.8301923725327\\
-40.283	-38.9033538439495\\
-29.297	-28.2936116368242\\
-43.945	-42.439934579658\\
-31.738	-30.6510102102443\\
-30.518	-29.4727937991126\\
-24.414	-23.5778487388274\\
-26.855	-25.9352473122475\\
-41.504	-40.0825360062379\\
-36.621	-35.3667731082411\\
-30.518	-29.4727937991126\\
-32.959	-31.8301923725327\\
-26.855	-25.9352473122475\\
-31.738	-30.6510102102443\\
-58.594	-56.5872232736484\\
-75.684	-73.0919105410589\\
-51.27	-49.5140618022315\\
-48.828	-47.1556974776548\\
-39.063	-37.7251374328178\\
-58.594	-56.5872232736484\\
-37.842	-36.5459552705295\\
-46.387	-44.7982989042347\\
-39.063	-37.7251374328178\\
-51.27	-49.5140618022315\\
-32.959	-31.8301923725327\\
-35.4	-34.1875909459527\\
-37.842	-36.5459552705295\\
-47.607	-45.9765153153664\\
-36.621	-35.3667731082411\\
-41.504	-40.0825360062379\\
-57.373	-55.40804111136\\
-51.27	-49.5140618022315\\
-72.021	-69.5543640541938\\
-93.994	-90.7748142196011\\
-78.125	-75.449309114479\\
-50.049	-48.3348796399432\\
-40.283	-38.9033538439495\\
-54.932	-53.05064253794\\
-69.58	-67.1969654807737\\
-56.152	-54.2288589490717\\
-63.477	-61.3029861716452\\
-43.945	-42.439934579658\\
-23.193	-22.398666576539\\
-24.414	-23.5778487388274\\
-52.49	-50.6922782133632\\
-45.166	-43.6191167419463\\
-42.725	-41.2617181685263\\
-64.697	-62.4812025827769\\
-61.035	-58.9446218470685\\
-56.152	-54.2288589490717\\
-39.063	-37.7251374328178\\
-53.711	-51.8714603756516\\
-65.918	-63.6603847450653\\
-52.49	-50.6922782133632\\
-53.711	-51.8714603756516\\
-48.828	-47.1556974776548\\
-37.842	-36.5459552705295\\
-42.725	-41.2617181685263\\
-35.4	-34.1875909459527\\
-36.621	-35.3667731082411\\
-57.373	-55.40804111136\\
-67.139	-64.8395669073537\\
-70.801	-68.3761476430621\\
-62.256	-60.1238040093569\\
-45.166	-43.6191167419463\\
-35.4	-34.1875909459527\\
-39.063	-37.7251374328178\\
-46.387	-44.7982989042347\\
-57.373	-55.40804111136\\
-68.359	-66.0177833184854\\
-79.346	-76.6284912767674\\
-65.918	-63.6603847450653\\
-40.283	-38.9033538439495\\
-68.359	-66.0177833184854\\
-91.553	-88.4174156461811\\
-65.918	-63.6603847450653\\
-64.697	-62.4812025827769\\
-75.684	-73.0919105410589\\
-59.814	-57.7654396847801\\
-50.049	-48.3348796399432\\
-56.152	-54.2288589490717\\
-51.27	-49.5140618022315\\
-57.373	-55.40804111136\\
-70.801	-68.3761476430621\\
-56.152	-54.2288589490717\\
-63.477	-61.3029861716452\\
-58.594	-56.5872232736484\\
-59.814	-57.7654396847801\\
-48.828	-47.1556974776548\\
-63.477	-61.3029861716452\\
-45.166	-43.6191167419463\\
-48.828	-47.1556974776548\\
-52.49	-50.6922782133632\\
-50.049	-48.3348796399432\\
-30.518	-29.4727937991126\\
-19.531	-18.8620858408306\\
-15.869	-15.3255051051221\\
-28.076	-27.1144294745358\\
-53.711	-51.8714603756516\\
-67.139	-64.8395669073537\\
-65.918	-63.6603847450653\\
-46.387	-44.7982989042347\\
-54.932	-53.05064253794\\
-50.049	-48.3348796399432\\
-43.945	-42.439934579658\\
-30.518	-29.4727937991126\\
-42.725	-41.2617181685263\\
-37.842	-36.5459552705295\\
-36.621	-35.3667731082411\\
-41.504	-40.0825360062379\\
-37.842	-36.5459552705295\\
-34.18	-33.009374534821\\
-25.635	-24.7570309011158\\
-35.4	-34.1875909459527\\
-58.594	-56.5872232736484\\
-42.725	-41.2617181685263\\
-52.49	-50.6922782133632\\
-45.166	-43.6191167419463\\
-59.814	-57.7654396847801\\
-57.373	-55.40804111136\\
-76.904	-74.2701269521906\\
-58.594	-56.5872232736484\\
-64.697	-62.4812025827769\\
-65.918	-63.6603847450653\\
-37.842	-36.5459552705295\\
-61.035	-58.9446218470685\\
-46.387	-44.7982989042347\\
-51.27	-49.5140618022315\\
-57.373	-55.40804111136\\
-69.58	-67.1969654807737\\
-53.711	-51.8714603756516\\
-70.801	-68.3761476430621\\
-81.787	-78.9858898501874\\
-69.58	-67.1969654807737\\
-58.594	-56.5872232736484\\
-50.049	-48.3348796399432\\
-58.594	-56.5872232736484\\
-52.49	-50.6922782133632\\
-43.945	-42.439934579658\\
-37.842	-36.5459552705295\\
-45.166	-43.6191167419463\\
-39.063	-37.7251374328178\\
-72.021	-69.5543640541938\\
-90.332	-87.2382334838927\\
-79.346	-76.6284912767674\\
-75.684	-73.0919105410589\\
-74.463	-71.9127283787705\\
-46.387	-44.7982989042347\\
-42.725	-41.2617181685263\\
-35.4	-34.1875909459527\\
-30.518	-29.4727937991126\\
-32.959	-31.8301923725327\\
-52.49	-50.6922782133632\\
-59.814	-57.7654396847801\\
-70.801	-68.3761476430621\\
-59.814	-57.7654396847801\\
-42.725	-41.2617181685263\\
-72.021	-69.5543640541938\\
-84.229	-81.3442541747642\\
-90.332	-87.2382334838927\\
-73.242	-70.7335462164822\\
-115.967	-111.995264385008\\
-85.449	-82.5224705858959\\
-51.27	-49.5140618022315\\
-39.063	-37.7251374328178\\
-34.18	-33.009374534821\\
-56.152	-54.2288589490717\\
-69.58	-67.1969654807737\\
-89.111	-86.0590513216043\\
-68.359	-66.0177833184854\\
-53.711	-51.8714603756516\\
-31.738	-30.6510102102443\\
-36.621	-35.3667731082411\\
-34.18	-33.009374534821\\
-39.063	-37.7251374328178\\
-25.635	-24.7570309011158\\
-35.4	-34.1875909459527\\
-26.855	-25.9352473122475\\
-19.531	-18.8620858408306\\
-39.063	-37.7251374328178\\
-42.725	-41.2617181685263\\
-48.828	-47.1556974776548\\
-46.387	-44.7982989042347\\
-48.828	-47.1556974776548\\
-62.256	-60.1238040093569\\
-43.945	-42.439934579658\\
-62.256	-60.1238040093569\\
-69.58	-67.1969654807737\\
-56.152	-54.2288589490717\\
-58.594	-56.5872232736484\\
-54.932	-53.05064253794\\
-61.035	-58.9446218470685\\
-80.566	-77.8067076878991\\
-62.256	-60.1238040093569\\
-50.049	-48.3348796399432\\
-57.373	-55.40804111136\\
-51.27	-49.5140618022315\\
-36.621	-35.3667731082411\\
-56.152	-54.2288589490717\\
-74.463	-71.9127283787705\\
-69.58	-67.1969654807737\\
-78.125	-75.449309114479\\
-113.525	-109.636900060432\\
-79.346	-76.6284912767674\\
-72.021	-69.5543640541938\\
-56.152	-54.2288589490717\\
-68.359	-66.0177833184854\\
-92.773	-89.5956320573128\\
-64.697	-62.4812025827769\\
-57.373	-55.40804111136\\
-76.904	-74.2701269521906\\
-51.27	-49.5140618022315\\
-32.959	-31.8301923725327\\
-42.725	-41.2617181685263\\
-32.959	-31.8301923725327\\
-25.635	-24.7570309011158\\
-28.076	-27.1144294745358\\
-25.635	-24.7570309011158\\
-31.738	-30.6510102102443\\
-42.725	-41.2617181685263\\
-48.828	-47.1556974776548\\
-63.477	-61.3029861716452\\
-59.814	-57.7654396847801\\
-67.139	-64.8395669073537\\
-79.346	-76.6284912767674\\
-73.242	-70.7335462164822\\
-54.932	-53.05064253794\\
-58.594	-56.5872232736484\\
-53.711	-51.8714603756516\\
-57.373	-55.40804111136\\
-76.904	-74.2701269521906\\
-59.814	-57.7654396847801\\
-63.477	-61.3029861716452\\
-62.256	-60.1238040093569\\
-57.373	-55.40804111136\\
-84.229	-81.3442541747642\\
-69.58	-67.1969654807737\\
-73.242	-70.7335462164822\\
-65.918	-63.6603847450653\\
-42.725	-41.2617181685263\\
-29.297	-28.2936116368242\\
-24.414	-23.5778487388274\\
-41.504	-40.0825360062379\\
-34.18	-33.009374534821\\
-47.607	-45.9765153153664\\
-35.4	-34.1875909459527\\
-45.166	-43.6191167419463\\
-79.346	-76.6284912767674\\
-93.994	-90.7748142196011\\
-75.684	-73.0919105410589\\
-79.346	-76.6284912767674\\
-70.801	-68.3761476430621\\
-52.49	-50.6922782133632\\
-54.932	-53.05064253794\\
-59.814	-57.7654396847801\\
-53.711	-51.8714603756516\\
-62.256	-60.1238040093569\\
-76.904	-74.2701269521906\\
-106.201	-102.563738589015\\
-93.994	-90.7748142196011\\
-87.891	-84.8808349104726\\
-97.656	-94.3113949553096\\
-68.359	-66.0177833184854\\
-74.463	-71.9127283787705\\
-59.814	-57.7654396847801\\
-75.684	-73.0919105410589\\
-85.449	-82.5224705858959\\
-37.842	-36.5459552705295\\
-48.828	-47.1556974776548\\
-37.842	-36.5459552705295\\
-53.711	-51.8714603756516\\
-35.4	-34.1875909459527\\
-46.387	-44.7982989042347\\
-74.463	-71.9127283787705\\
-118.408	-114.352662958429\\
-106.201	-102.563738589015\\
-85.449	-82.5224705858959\\
-97.656	-94.3113949553096\\
-115.967	-111.995264385008\\
-86.67	-83.7016527481842\\
-89.111	-86.0590513216043\\
-91.553	-88.4174156461811\\
-64.697	-62.4812025827769\\
-56.152	-54.2288589490717\\
-72.021	-69.5543640541938\\
-48.828	-47.1556974776548\\
-37.842	-36.5459552705295\\
-52.49	-50.6922782133632\\
-39.063	-37.7251374328178\\
-47.607	-45.9765153153664\\
-41.504	-40.0825360062379\\
-51.27	-49.5140618022315\\
-65.918	-63.6603847450653\\
-52.49	-50.6922782133632\\
-41.504	-40.0825360062379\\
-50.049	-48.3348796399432\\
-57.373	-55.40804111136\\
-65.918	-63.6603847450653\\
-56.152	-54.2288589490717\\
-46.387	-44.7982989042347\\
-70.801	-68.3761476430621\\
-65.918	-63.6603847450653\\
-47.607	-45.9765153153664\\
-78.125	-75.449309114479\\
-67.139	-64.8395669073537\\
-53.711	-51.8714603756516\\
-37.842	-36.5459552705295\\
-29.297	-28.2936116368242\\
-23.193	-22.398666576539\\
-41.504	-40.0825360062379\\
-36.621	-35.3667731082411\\
-31.738	-30.6510102102443\\
-24.414	-23.5778487388274\\
-31.738	-30.6510102102443\\
-46.387	-44.7982989042347\\
-51.27	-49.5140618022315\\
-63.477	-61.3029861716452\\
-73.242	-70.7335462164822\\
-72.021	-69.5543640541938\\
-74.463	-71.9127283787705\\
-47.607	-45.9765153153664\\
-23.193	-22.398666576539\\
-35.4	-34.1875909459527\\
-26.855	-25.9352473122475\\
-37.842	-36.5459552705295\\
-52.49	-50.6922782133632\\
-58.594	-56.5872232736484\\
-85.449	-82.5224705858959\\
-56.152	-54.2288589490717\\
-57.373	-55.40804111136\\
-74.463	-71.9127283787705\\
-57.373	-55.40804111136\\
-41.504	-40.0825360062379\\
-34.18	-33.009374534821\\
-45.166	-43.6191167419463\\
-54.932	-53.05064253794\\
-79.346	-76.6284912767674\\
-53.711	-51.8714603756516\\
-47.607	-45.9765153153664\\
-45.166	-43.6191167419463\\
-50.049	-48.3348796399432\\
-68.359	-66.0177833184854\\
-103.76	-100.206340015595\\
-89.111	-86.0590513216043\\
-90.332	-87.2382334838927\\
-59.814	-57.7654396847801\\
-43.945	-42.439934579658\\
-48.828	-47.1556974776548\\
-21.973	-21.2204501654073\\
-39.063	-37.7251374328178\\
-30.518	-29.4727937991126\\
-36.621	-35.3667731082411\\
-58.594	-56.5872232736484\\
-76.904	-74.2701269521906\\
-86.67	-83.7016527481842\\
-111.084	-107.279501487012\\
-93.994	-90.7748142196011\\
-53.711	-51.8714603756516\\
-42.725	-41.2617181685263\\
-57.373	-55.40804111136\\
-73.242	-70.7335462164822\\
-95.215	-91.9539963818895\\
-72.021	-69.5543640541938\\
-52.49	-50.6922782133632\\
-59.814	-57.7654396847801\\
-75.684	-73.0919105410589\\
-54.932	-53.05064253794\\
-40.283	-38.9033538439495\\
-35.4	-34.1875909459527\\
-25.635	-24.7570309011158\\
-24.414	-23.5778487388274\\
-19.531	-18.8620858408306\\
-32.959	-31.8301923725327\\
-41.504	-40.0825360062379\\
-46.387	-44.7982989042347\\
-36.621	-35.3667731082411\\
-35.4	-34.1875909459527\\
-19.531	-18.8620858408306\\
-13.428	-12.9681065317021\\
-20.752	-20.041268003119\\
-29.297	-28.2936116368242\\
-39.063	-37.7251374328178\\
-48.828	-47.1556974776548\\
-52.49	-50.6922782133632\\
-62.256	-60.1238040093569\\
-79.346	-76.6284912767674\\
-107.422	-103.742920751303\\
-103.76	-100.206340015595\\
-112.305	-108.4586836493\\
-98.877	-95.4905771175979\\
-61.035	-58.9446218470685\\
-54.932	-53.05064253794\\
-53.711	-51.8714603756516\\
-69.58	-67.1969654807737\\
-59.814	-57.7654396847801\\
-45.166	-43.6191167419463\\
-36.621	-35.3667731082411\\
-31.738	-30.6510102102443\\
-50.049	-48.3348796399432\\
-47.607	-45.9765153153664\\
-28.076	-27.1144294745358\\
-21.973	-21.2204501654073\\
-32.959	-31.8301923725327\\
-43.945	-42.439934579658\\
-32.959	-31.8301923725327\\
-46.387	-44.7982989042347\\
-36.621	-35.3667731082411\\
-51.27	-49.5140618022315\\
-76.904	-74.2701269521906\\
-61.035	-58.9446218470685\\
-80.566	-77.8067076878991\\
-104.98	-101.384556426726\\
-113.525	-109.636900060432\\
-87.891	-84.8808349104726\\
-59.814	-57.7654396847801\\
-45.166	-43.6191167419463\\
-31.738	-30.6510102102443\\
-40.283	-38.9033538439495\\
-30.518	-29.4727937991126\\
-54.932	-53.05064253794\\
-47.607	-45.9765153153664\\
-40.283	-38.9033538439495\\
-51.27	-49.5140618022315\\
-32.959	-31.8301923725327\\
-45.166	-43.6191167419463\\
-34.18	-33.009374534821\\
-23.193	-22.398666576539\\
-25.635	-24.7570309011158\\
-21.973	-21.2204501654073\\
-23.193	-22.398666576539\\
-24.414	-23.5778487388274\\
-17.09	-16.5046872674105\\
-21.973	-21.2204501654073\\
-29.297	-28.2936116368242\\
-30.518	-29.4727937991126\\
-29.297	-28.2936116368242\\
-43.945	-42.439934579658\\
-54.932	-53.05064253794\\
-45.166	-43.6191167419463\\
-42.725	-41.2617181685263\\
-62.256	-60.1238040093569\\
-74.463	-71.9127283787705\\
-59.814	-57.7654396847801\\
-65.918	-63.6603847450653\\
-32.959	-31.8301923725327\\
-21.973	-21.2204501654073\\
-40.283	-38.9033538439495\\
-56.152	-54.2288589490717\\
-58.594	-56.5872232736484\\
-83.008	-80.1650720124758\\
-74.463	-71.9127283787705\\
-53.711	-51.8714603756516\\
-40.283	-38.9033538439495\\
-42.725	-41.2617181685263\\
-46.387	-44.7982989042347\\
-36.621	-35.3667731082411\\
-24.414	-23.5778487388274\\
-32.959	-31.8301923725327\\
-25.635	-24.7570309011158\\
-21.973	-21.2204501654073\\
-18.311	-17.6838694296989\\
-28.076	-27.1144294745358\\
-40.283	-38.9033538439495\\
-43.945	-42.439934579658\\
-31.738	-30.6510102102443\\
-52.49	-50.6922782133632\\
-70.801	-68.3761476430621\\
-47.607	-45.9765153153664\\
-64.697	-62.4812025827769\\
-70.801	-68.3761476430621\\
-51.27	-49.5140618022315\\
-34.18	-33.009374534821\\
-42.725	-41.2617181685263\\
-47.607	-45.9765153153664\\
-29.297	-28.2936116368242\\
-34.18	-33.009374534821\\
-28.076	-27.1144294745358\\
-19.531	-18.8620858408306\\
-29.297	-28.2936116368242\\
-36.621	-35.3667731082411\\
-32.959	-31.8301923725327\\
-39.063	-37.7251374328178\\
-42.725	-41.2617181685263\\
-32.959	-31.8301923725327\\
-20.752	-20.041268003119\\
-18.311	-17.6838694296989\\
-21.973	-21.2204501654073\\
-25.635	-24.7570309011158\\
-29.297	-28.2936116368242\\
-54.932	-53.05064253794\\
-53.711	-51.8714603756516\\
-47.607	-45.9765153153664\\
-61.035	-58.9446218470685\\
-76.904	-74.2701269521906\\
-83.008	-80.1650720124758\\
-85.449	-82.5224705858959\\
-64.697	-62.4812025827769\\
-65.918	-63.6603847450653\\
-67.139	-64.8395669073537\\
-69.58	-67.1969654807737\\
-76.904	-74.2701269521906\\
-101.318	-97.847975691018\\
-86.67	-83.7016527481842\\
-79.346	-76.6284912767674\\
-63.477	-61.3029861716452\\
-59.814	-57.7654396847801\\
-70.801	-68.3761476430621\\
-46.387	-44.7982989042347\\
-53.711	-51.8714603756516\\
-62.256	-60.1238040093569\\
-72.021	-69.5543640541938\\
-56.152	-54.2288589490717\\
-65.918	-63.6603847450653\\
-52.49	-50.6922782133632\\
-48.828	-47.1556974776548\\
-32.959	-31.8301923725327\\
-51.27	-49.5140618022315\\
-73.242	-70.7335462164822\\
-57.373	-55.40804111136\\
-36.621	-35.3667731082411\\
-43.945	-42.439934579658\\
-28.076	-27.1144294745358\\
-25.635	-24.7570309011158\\
-32.959	-31.8301923725327\\
-23.193	-22.398666576539\\
-25.635	-24.7570309011158\\
-28.076	-27.1144294745358\\
-19.531	-18.8620858408306\\
-29.297	-28.2936116368242\\
-25.635	-24.7570309011158\\
-19.531	-18.8620858408306\\
-28.076	-27.1144294745358\\
-31.738	-30.6510102102443\\
-35.4	-34.1875909459527\\
-36.621	-35.3667731082411\\
-35.4	-34.1875909459527\\
-46.387	-44.7982989042347\\
-40.283	-38.9033538439495\\
-47.607	-45.9765153153664\\
-75.684	-73.0919105410589\\
-56.152	-54.2288589490717\\
-50.049	-48.3348796399432\\
-52.49	-50.6922782133632\\
-40.283	-38.9033538439495\\
-26.855	-25.9352473122475\\
-42.725	-41.2617181685263\\
-35.4	-34.1875909459527\\
-25.635	-24.7570309011158\\
-37.842	-36.5459552705295\\
-36.621	-35.3667731082411\\
-42.725	-41.2617181685263\\
-30.518	-29.4727937991126\\
-21.973	-21.2204501654073\\
-18.311	-17.6838694296989\\
-20.752	-20.041268003119\\
-34.18	-33.009374534821\\
-31.738	-30.6510102102443\\
-35.4	-34.1875909459527\\
-47.607	-45.9765153153664\\
-63.477	-61.3029861716452\\
-48.828	-47.1556974776548\\
-63.477	-61.3029861716452\\
-74.463	-71.9127283787705\\
-64.697	-62.4812025827769\\
-59.814	-57.7654396847801\\
-97.656	-94.3113949553096\\
-70.801	-68.3761476430621\\
-86.67	-83.7016527481842\\
-76.904	-74.2701269521906\\
-86.67	-83.7016527481842\\
-95.215	-91.9539963818895\\
-61.035	-58.9446218470685\\
-39.063	-37.7251374328178\\
-32.959	-31.8301923725327\\
-47.607	-45.9765153153664\\
-57.373	-55.40804111136\\
-58.594	-56.5872232736484\\
-41.504	-40.0825360062379\\
-25.635	-24.7570309011158\\
-34.18	-33.009374534821\\
-57.373	-55.40804111136\\
-76.904	-74.2701269521906\\
-75.684	-73.0919105410589\\
-48.828	-47.1556974776548\\
-39.063	-37.7251374328178\\
-48.828	-47.1556974776548\\
-72.021	-69.5543640541938\\
-81.787	-78.9858898501874\\
-83.008	-80.1650720124758\\
-61.035	-58.9446218470685\\
-83.008	-80.1650720124758\\
-90.332	-87.2382334838927\\
-86.67	-83.7016527481842\\
-114.746	-110.81608222272\\
-98.877	-95.4905771175979\\
-85.449	-82.5224705858959\\
-98.877	-95.4905771175979\\
-84.229	-81.3442541747642\\
-51.27	-49.5140618022315\\
-47.607	-45.9765153153664\\
-56.152	-54.2288589490717\\
-67.139	-64.8395669073537\\
-51.27	-49.5140618022315\\
-63.477	-61.3029861716452\\
-56.152	-54.2288589490717\\
-43.945	-42.439934579658\\
-54.932	-53.05064253794\\
-52.49	-50.6922782133632\\
-75.684	-73.0919105410589\\
-76.904	-74.2701269521906\\
-57.373	-55.40804111136\\
-45.166	-43.6191167419463\\
-30.518	-29.4727937991126\\
-43.945	-42.439934579658\\
-35.4	-34.1875909459527\\
-31.738	-30.6510102102443\\
-26.855	-25.9352473122475\\
-42.725	-41.2617181685263\\
-53.711	-51.8714603756516\\
-58.594	-56.5872232736484\\
-37.842	-36.5459552705295\\
-30.518	-29.4727937991126\\
-23.193	-22.398666576539\\
-28.076	-27.1144294745358\\
-40.283	-38.9033538439495\\
-42.725	-41.2617181685263\\
-40.283	-38.9033538439495\\
-62.256	-60.1238040093569\\
-73.242	-70.7335462164822\\
-80.566	-77.8067076878991\\
-81.787	-78.9858898501874\\
-95.215	-91.9539963818895\\
-59.814	-57.7654396847801\\
-47.607	-45.9765153153664\\
-40.283	-38.9033538439495\\
-43.945	-42.439934579658\\
-54.932	-53.05064253794\\
-46.387	-44.7982989042347\\
-34.18	-33.009374534821\\
-30.518	-29.4727937991126\\
-52.49	-50.6922782133632\\
-62.256	-60.1238040093569\\
-57.373	-55.40804111136\\
-85.449	-82.5224705858959\\
-91.553	-88.4174156461811\\
-65.918	-63.6603847450653\\
-47.607	-45.9765153153664\\
-36.621	-35.3667731082411\\
-32.959	-31.8301923725327\\
-56.152	-54.2288589490717\\
-48.828	-47.1556974776548\\
-53.711	-51.8714603756516\\
-59.814	-57.7654396847801\\
-67.139	-64.8395669073537\\
-63.477	-61.3029861716452\\
-72.021	-69.5543640541938\\
-51.27	-49.5140618022315\\
-56.152	-54.2288589490717\\
-42.725	-41.2617181685263\\
-39.063	-37.7251374328178\\
-54.932	-53.05064253794\\
-75.684	-73.0919105410589\\
-81.787	-78.9858898501874\\
-47.607	-45.9765153153664\\
-91.553	-88.4174156461811\\
-78.125	-75.449309114479\\
-53.711	-51.8714603756516\\
-34.18	-33.009374534821\\
-50.049	-48.3348796399432\\
-45.166	-43.6191167419463\\
-53.711	-51.8714603756516\\
-37.842	-36.5459552705295\\
-48.828	-47.1556974776548\\
-86.67	-83.7016527481842\\
-91.553	-88.4174156461811\\
-63.477	-61.3029861716452\\
-70.801	-68.3761476430621\\
-78.125	-75.449309114479\\
-76.904	-74.2701269521906\\
-57.373	-55.40804111136\\
-56.152	-54.2288589490717\\
-41.504	-40.0825360062379\\
-54.932	-53.05064253794\\
-59.814	-57.7654396847801\\
-92.773	-89.5956320573128\\
-101.318	-97.847975691018\\
-131.836	-127.320769490131\\
-90.332	-87.2382334838927\\
-76.904	-74.2701269521906\\
-62.256	-60.1238040093569\\
-67.139	-64.8395669073537\\
-52.49	-50.6922782133632\\
-37.842	-36.5459552705295\\
-39.063	-37.7251374328178\\
-31.738	-30.6510102102443\\
-23.193	-22.398666576539\\
-34.18	-33.009374534821\\
-45.166	-43.6191167419463\\
-29.297	-28.2936116368242\\
-25.635	-24.7570309011158\\
-24.414	-23.5778487388274\\
-50.049	-48.3348796399432\\
-67.139	-64.8395669073537\\
-35.4	-34.1875909459527\\
-41.504	-40.0825360062379\\
-43.945	-42.439934579658\\
-40.283	-38.9033538439495\\
-47.607	-45.9765153153664\\
-45.166	-43.6191167419463\\
-32.959	-31.8301923725327\\
-25.635	-24.7570309011158\\
-23.193	-22.398666576539\\
-26.855	-25.9352473122475\\
-47.607	-45.9765153153664\\
-54.932	-53.05064253794\\
-41.504	-40.0825360062379\\
-30.518	-29.4727937991126\\
-23.193	-22.398666576539\\
-28.076	-27.1144294745358\\
-42.725	-41.2617181685263\\
-51.27	-49.5140618022315\\
-56.152	-54.2288589490717\\
-40.283	-38.9033538439495\\
-39.063	-37.7251374328178\\
-42.725	-41.2617181685263\\
-57.373	-55.40804111136\\
-76.904	-74.2701269521906\\
-89.111	-86.0590513216043\\
-69.58	-67.1969654807737\\
-46.387	-44.7982989042347\\
-48.828	-47.1556974776548\\
-46.387	-44.7982989042347\\
-35.4	-34.1875909459527\\
-39.063	-37.7251374328178\\
-43.945	-42.439934579658\\
-41.504	-40.0825360062379\\
-29.297	-28.2936116368242\\
-20.752	-20.041268003119\\
-29.297	-28.2936116368242\\
-40.283	-38.9033538439495\\
-57.373	-55.40804111136\\
-61.035	-58.9446218470685\\
-74.463	-71.9127283787705\\
-65.918	-63.6603847450653\\
-73.242	-70.7335462164822\\
-96.436	-93.1331785441779\\
-126.953	-122.605006592134\\
-106.201	-102.563738589015\\
-73.242	-70.7335462164822\\
-80.566	-77.8067076878991\\
-91.553	-88.4174156461811\\
-122.07	-117.889243694137\\
-80.566	-77.8067076878991\\
-43.945	-42.439934579658\\
-30.518	-29.4727937991126\\
-31.738	-30.6510102102443\\
-25.635	-24.7570309011158\\
-18.311	-17.6838694296989\\
-21.973	-21.2204501654073\\
-29.297	-28.2936116368242\\
-37.842	-36.5459552705295\\
-30.518	-29.4727937991126\\
-29.297	-28.2936116368242\\
-36.621	-35.3667731082411\\
-43.945	-42.439934579658\\
-45.166	-43.6191167419463\\
-42.725	-41.2617181685263\\
-32.959	-31.8301923725327\\
-24.414	-23.5778487388274\\
-23.193	-22.398666576539\\
-35.4	-34.1875909459527\\
-39.063	-37.7251374328178\\
-30.518	-29.4727937991126\\
-23.193	-22.398666576539\\
-35.4	-34.1875909459527\\
-28.076	-27.1144294745358\\
-30.518	-29.4727937991126\\
-37.842	-36.5459552705295\\
-50.049	-48.3348796399432\\
-54.932	-53.05064253794\\
-40.283	-38.9033538439495\\
-56.152	-54.2288589490717\\
-54.932	-53.05064253794\\
-46.387	-44.7982989042347\\
-37.842	-36.5459552705295\\
-58.594	-56.5872232736484\\
-73.242	-70.7335462164822\\
-79.346	-76.6284912767674\\
-62.256	-60.1238040093569\\
-57.373	-55.40804111136\\
-54.932	-53.05064253794\\
-72.021	-69.5543640541938\\
-59.814	-57.7654396847801\\
-78.125	-75.449309114479\\
-70.801	-68.3761476430621\\
-74.463	-71.9127283787705\\
-124.512	-120.247608018714\\
-98.877	-95.4905771175979\\
-53.711	-51.8714603756516\\
-37.842	-36.5459552705295\\
-41.504	-40.0825360062379\\
-51.27	-49.5140618022315\\
-63.477	-61.3029861716452\\
-70.801	-68.3761476430621\\
-78.125	-75.449309114479\\
-80.566	-77.8067076878991\\
-54.932	-53.05064253794\\
-47.607	-45.9765153153664\\
-56.152	-54.2288589490717\\
-36.621	-35.3667731082411\\
-28.076	-27.1144294745358\\
-45.166	-43.6191167419463\\
-46.387	-44.7982989042347\\
-59.814	-57.7654396847801\\
-73.242	-70.7335462164822\\
-64.697	-62.4812025827769\\
-47.607	-45.9765153153664\\
-40.283	-38.9033538439495\\
-57.373	-55.40804111136\\
-70.801	-68.3761476430621\\
-91.553	-88.4174156461811\\
-100.098	-96.6697592798863\\
-113.525	-109.636900060432\\
-92.773	-89.5956320573128\\
-85.449	-82.5224705858959\\
-52.49	-50.6922782133632\\
-40.283	-38.9033538439495\\
-65.918	-63.6603847450653\\
-85.449	-82.5224705858959\\
-52.49	-50.6922782133632\\
-74.463	-71.9127283787705\\
-104.98	-101.384556426726\\
-119.629	-115.531845120717\\
-79.346	-76.6284912767674\\
-46.387	-44.7982989042347\\
-30.518	-29.4727937991126\\
-40.283	-38.9033538439495\\
-37.842	-36.5459552705295\\
-41.504	-40.0825360062379\\
-29.297	-28.2936116368242\\
-35.4	-34.1875909459527\\
-30.518	-29.4727937991126\\
-35.4	-34.1875909459527\\
-31.738	-30.6510102102443\\
-34.18	-33.009374534821\\
-53.711	-51.8714603756516\\
-50.049	-48.3348796399432\\
-67.139	-64.8395669073537\\
-80.566	-77.8067076878991\\
-79.346	-76.6284912767674\\
-46.387	-44.7982989042347\\
-47.607	-45.9765153153664\\
-57.373	-55.40804111136\\
-40.283	-38.9033538439495\\
-35.4	-34.1875909459527\\
-28.076	-27.1144294745358\\
-41.504	-40.0825360062379\\
-35.4	-34.1875909459527\\
-23.193	-22.398666576539\\
-14.648	-14.1463229428338\\
-17.09	-16.5046872674105\\
-26.855	-25.9352473122475\\
-37.842	-36.5459552705295\\
-40.283	-38.9033538439495\\
-28.076	-27.1144294745358\\
-31.738	-30.6510102102443\\
-46.387	-44.7982989042347\\
-57.373	-55.40804111136\\
-41.504	-40.0825360062379\\
-48.828	-47.1556974776548\\
-54.932	-53.05064253794\\
-50.049	-48.3348796399432\\
-57.373	-55.40804111136\\
-53.711	-51.8714603756516\\
-40.283	-38.9033538439495\\
-31.738	-30.6510102102443\\
-42.725	-41.2617181685263\\
-36.621	-35.3667731082411\\
-42.725	-41.2617181685263\\
-57.373	-55.40804111136\\
-83.008	-80.1650720124758\\
-63.477	-61.3029861716452\\
-57.373	-55.40804111136\\
-85.449	-82.5224705858959\\
-68.359	-66.0177833184854\\
-41.504	-40.0825360062379\\
-34.18	-33.009374534821\\
-28.076	-27.1144294745358\\
-26.855	-25.9352473122475\\
-25.635	-24.7570309011158\\
-45.166	-43.6191167419463\\
-28.076	-27.1144294745358\\
-23.193	-22.398666576539\\
-32.959	-31.8301923725327\\
-41.504	-40.0825360062379\\
-32.959	-31.8301923725327\\
-25.635	-24.7570309011158\\
-17.09	-16.5046872674105\\
-25.635	-24.7570309011158\\
-46.387	-44.7982989042347\\
-59.814	-57.7654396847801\\
-50.049	-48.3348796399432\\
-40.283	-38.9033538439495\\
-45.166	-43.6191167419463\\
-46.387	-44.7982989042347\\
-50.049	-48.3348796399432\\
-61.035	-58.9446218470685\\
-70.801	-68.3761476430621\\
-68.359	-66.0177833184854\\
-58.594	-56.5872232736484\\
-41.504	-40.0825360062379\\
-34.18	-33.009374534821\\
-36.621	-35.3667731082411\\
-48.828	-47.1556974776548\\
-35.4	-34.1875909459527\\
-28.076	-27.1144294745358\\
-50.049	-48.3348796399432\\
-61.035	-58.9446218470685\\
-57.373	-55.40804111136\\
-65.918	-63.6603847450653\\
-83.008	-80.1650720124758\\
-48.828	-47.1556974776548\\
-81.787	-78.9858898501874\\
-76.904	-74.2701269521906\\
-73.242	-70.7335462164822\\
-76.904	-74.2701269521906\\
-122.07	-117.889243694137\\
-119.629	-115.531845120717\\
-79.346	-76.6284912767674\\
-93.994	-90.7748142196011\\
-115.967	-111.995264385008\\
-112.305	-108.4586836493\\
-124.512	-120.247608018714\\
-113.525	-109.636900060432\\
-142.822	-137.930511697256\\
-115.967	-111.995264385008\\
-79.346	-76.6284912767674\\
-53.711	-51.8714603756516\\
-52.49	-50.6922782133632\\
-45.166	-43.6191167419463\\
-37.842	-36.5459552705295\\
-52.49	-50.6922782133632\\
-73.242	-70.7335462164822\\
-45.166	-43.6191167419463\\
-41.504	-40.0825360062379\\
-42.725	-41.2617181685263\\
-31.738	-30.6510102102443\\
-34.18	-33.009374534821\\
-25.635	-24.7570309011158\\
-24.414	-23.5778487388274\\
-23.193	-22.398666576539\\
-29.297	-28.2936116368242\\
-24.414	-23.5778487388274\\
-20.752	-20.041268003119\\
-17.09	-16.5046872674105\\
-13.428	-12.9681065317021\\
-24.414	-23.5778487388274\\
-20.752	-20.041268003119\\
-18.311	-17.6838694296989\\
-37.842	-36.5459552705295\\
-51.27	-49.5140618022315\\
-50.049	-48.3348796399432\\
-40.283	-38.9033538439495\\
-48.828	-47.1556974776548\\
-67.139	-64.8395669073537\\
-34.18	-33.009374534821\\
-24.414	-23.5778487388274\\
-19.531	-18.8620858408306\\
-14.648	-14.1463229428338\\
-20.752	-20.041268003119\\
-36.621	-35.3667731082411\\
-50.049	-48.3348796399432\\
-36.621	-35.3667731082411\\
-47.607	-45.9765153153664\\
-74.463	-71.9127283787705\\
-73.242	-70.7335462164822\\
-53.711	-51.8714603756516\\
-37.842	-36.5459552705295\\
-43.945	-42.439934579658\\
-59.814	-57.7654396847801\\
-73.242	-70.7335462164822\\
-46.387	-44.7982989042347\\
-25.635	-24.7570309011158\\
-36.621	-35.3667731082411\\
-34.18	-33.009374534821\\
-17.09	-16.5046872674105\\
-12.207	-11.7889243694137\\
-26.855	-25.9352473122475\\
-53.711	-51.8714603756516\\
-43.945	-42.439934579658\\
-29.297	-28.2936116368242\\
-28.076	-27.1144294745358\\
-17.09	-16.5046872674105\\
-18.311	-17.6838694296989\\
-21.973	-21.2204501654073\\
-26.855	-25.9352473122475\\
-36.621	-35.3667731082411\\
-54.932	-53.05064253794\\
-57.373	-55.40804111136\\
-59.814	-57.7654396847801\\
-63.477	-61.3029861716452\\
-62.256	-60.1238040093569\\
-59.814	-57.7654396847801\\
-53.711	-51.8714603756516\\
-64.697	-62.4812025827769\\
-93.994	-90.7748142196011\\
-86.67	-83.7016527481842\\
-50.049	-48.3348796399432\\
-29.297	-28.2936116368242\\
-52.49	-50.6922782133632\\
-67.139	-64.8395669073537\\
-59.814	-57.7654396847801\\
-37.842	-36.5459552705295\\
-59.814	-57.7654396847801\\
-89.111	-86.0590513216043\\
-98.877	-95.4905771175979\\
-97.656	-94.3113949553096\\
-73.242	-70.7335462164822\\
-76.904	-74.2701269521906\\
-43.945	-42.439934579658\\
-36.621	-35.3667731082411\\
-45.166	-43.6191167419463\\
-54.932	-53.05064253794\\
-57.373	-55.40804111136\\
-61.035	-58.9446218470685\\
-42.725	-41.2617181685263\\
-51.27	-49.5140618022315\\
-48.828	-47.1556974776548\\
-29.297	-28.2936116368242\\
-21.973	-21.2204501654073\\
-17.09	-16.5046872674105\\
-25.635	-24.7570309011158\\
-23.193	-22.398666576539\\
-13.428	-12.9681065317021\\
-25.635	-24.7570309011158\\
-46.387	-44.7982989042347\\
-35.4	-34.1875909459527\\
-30.518	-29.4727937991126\\
-40.283	-38.9033538439495\\
-61.035	-58.9446218470685\\
-48.828	-47.1556974776548\\
-24.414	-23.5778487388274\\
-18.311	-17.6838694296989\\
-32.959	-31.8301923725327\\
-68.359	-66.0177833184854\\
-81.787	-78.9858898501874\\
-109.863	-106.100319324723\\
-130.615	-126.141587327842\\
-126.953	-122.605006592134\\
-87.891	-84.8808349104726\\
-109.863	-106.100319324723\\
-95.215	-91.9539963818895\\
-89.111	-86.0590513216043\\
-98.877	-95.4905771175979\\
-107.422	-103.742920751303\\
-108.643	-104.922102913592\\
-145.264	-140.288876021833\\
-102.539	-99.0271578533064\\
-57.373	-55.40804111136\\
-34.18	-33.009374534821\\
-29.297	-28.2936116368242\\
-32.959	-31.8301923725327\\
-31.738	-30.6510102102443\\
-53.711	-51.8714603756516\\
-45.166	-43.6191167419463\\
-59.814	-57.7654396847801\\
-34.18	-33.009374534821\\
-37.842	-36.5459552705295\\
-51.27	-49.5140618022315\\
-57.373	-55.40804111136\\
-35.4	-34.1875909459527\\
-24.414	-23.5778487388274\\
-31.738	-30.6510102102443\\
-58.594	-56.5872232736484\\
-91.553	-88.4174156461811\\
-86.67	-83.7016527481842\\
-96.436	-93.1331785441779\\
-111.084	-107.279501487012\\
-140.381	-135.573113123836\\
-120.85	-116.711027283005\\
-80.566	-77.8067076878991\\
-54.932	-53.05064253794\\
-34.18	-33.009374534821\\
-39.063	-37.7251374328178\\
-47.607	-45.9765153153664\\
-36.621	-35.3667731082411\\
-32.959	-31.8301923725327\\
-36.621	-35.3667731082411\\
-43.945	-42.439934579658\\
-47.607	-45.9765153153664\\
-58.594	-56.5872232736484\\
-45.166	-43.6191167419463\\
-23.193	-22.398666576539\\
-18.311	-17.6838694296989\\
-17.09	-16.5046872674105\\
-21.973	-21.2204501654073\\
-32.959	-31.8301923725327\\
-35.4	-34.1875909459527\\
-42.725	-41.2617181685263\\
-57.373	-55.40804111136\\
-45.166	-43.6191167419463\\
-63.477	-61.3029861716452\\
-97.656	-94.3113949553096\\
-78.125	-75.449309114479\\
-68.359	-66.0177833184854\\
-84.229	-81.3442541747642\\
-96.436	-93.1331785441779\\
-62.256	-60.1238040093569\\
-58.594	-56.5872232736484\\
-40.283	-38.9033538439495\\
-41.504	-40.0825360062379\\
-76.904	-74.2701269521906\\
-124.512	-120.247608018714\\
-147.705	-142.646274595253\\
-115.967	-111.995264385008\\
-98.877	-95.4905771175979\\
-75.684	-73.0919105410589\\
-80.566	-77.8067076878991\\
-103.76	-100.206340015595\\
-63.477	-61.3029861716452\\
-51.27	-49.5140618022315\\
-41.504	-40.0825360062379\\
-72.021	-69.5543640541938\\
-40.283	-38.9033538439495\\
-41.504	-40.0825360062379\\
-34.18	-33.009374534821\\
-50.049	-48.3348796399432\\
-76.904	-74.2701269521906\\
-75.684	-73.0919105410589\\
-58.594	-56.5872232736484\\
-46.387	-44.7982989042347\\
-52.49	-50.6922782133632\\
-41.504	-40.0825360062379\\
-30.518	-29.4727937991126\\
-26.855	-25.9352473122475\\
-24.414	-23.5778487388274\\
-20.752	-20.041268003119\\
-23.193	-22.398666576539\\
-37.842	-36.5459552705295\\
-51.27	-49.5140618022315\\
-31.738	-30.6510102102443\\
-29.297	-28.2936116368242\\
-26.855	-25.9352473122475\\
-32.959	-31.8301923725327\\
-45.166	-43.6191167419463\\
-47.607	-45.9765153153664\\
-45.166	-43.6191167419463\\
-28.076	-27.1144294745358\\
-47.607	-45.9765153153664\\
-69.58	-67.1969654807737\\
-53.711	-51.8714603756516\\
-26.855	-25.9352473122475\\
-41.504	-40.0825360062379\\
-42.725	-41.2617181685263\\
-37.842	-36.5459552705295\\
-26.855	-25.9352473122475\\
-41.504	-40.0825360062379\\
-62.256	-60.1238040093569\\
-41.504	-40.0825360062379\\
-17.09	-16.5046872674105\\
-25.635	-24.7570309011158\\
-51.27	-49.5140618022315\\
-54.932	-53.05064253794\\
-39.063	-37.7251374328178\\
-32.959	-31.8301923725327\\
-58.594	-56.5872232736484\\
-76.904	-74.2701269521906\\
-64.697	-62.4812025827769\\
-87.891	-84.8808349104726\\
-108.643	-104.922102913592\\
-72.021	-69.5543640541938\\
-59.814	-57.7654396847801\\
-80.566	-77.8067076878991\\
-54.932	-53.05064253794\\
-73.242	-70.7335462164822\\
-87.891	-84.8808349104726\\
-69.58	-67.1969654807737\\
-36.621	-35.3667731082411\\
-58.594	-56.5872232736484\\
-100.098	-96.6697592798863\\
-114.746	-110.81608222272\\
-83.008	-80.1650720124758\\
-72.021	-69.5543640541938\\
-92.773	-89.5956320573128\\
-74.463	-71.9127283787705\\
-48.828	-47.1556974776548\\
-45.166	-43.6191167419463\\
-57.373	-55.40804111136\\
-58.594	-56.5872232736484\\
-57.373	-55.40804111136\\
-64.697	-62.4812025827769\\
-48.828	-47.1556974776548\\
-50.049	-48.3348796399432\\
-68.359	-66.0177833184854\\
-43.945	-42.439934579658\\
-35.4	-34.1875909459527\\
-30.518	-29.4727937991126\\
-23.193	-22.398666576539\\
-26.855	-25.9352473122475\\
-46.387	-44.7982989042347\\
-41.504	-40.0825360062379\\
-57.373	-55.40804111136\\
-69.58	-67.1969654807737\\
-83.008	-80.1650720124758\\
-63.477	-61.3029861716452\\
-57.373	-55.40804111136\\
-25.635	-24.7570309011158\\
-47.607	-45.9765153153664\\
-72.021	-69.5543640541938\\
-51.27	-49.5140618022315\\
-26.855	-25.9352473122475\\
-51.27	-49.5140618022315\\
-74.463	-71.9127283787705\\
-92.773	-89.5956320573128\\
-120.85	-116.711027283005\\
-87.891	-84.8808349104726\\
-74.463	-71.9127283787705\\
-86.67	-83.7016527481842\\
-61.035	-58.9446218470685\\
-47.607	-45.9765153153664\\
-53.711	-51.8714603756516\\
-68.359	-66.0177833184854\\
-72.021	-69.5543640541938\\
-45.166	-43.6191167419463\\
-39.063	-37.7251374328178\\
-32.959	-31.8301923725327\\
-46.387	-44.7982989042347\\
-43.945	-42.439934579658\\
-34.18	-33.009374534821\\
-63.477	-61.3029861716452\\
-68.359	-66.0177833184854\\
-50.049	-48.3348796399432\\
-40.283	-38.9033538439495\\
-58.594	-56.5872232736484\\
-32.959	-31.8301923725327\\
-21.973	-21.2204501654073\\
-28.076	-27.1144294745358\\
-35.4	-34.1875909459527\\
-34.18	-33.009374534821\\
-25.635	-24.7570309011158\\
-18.311	-17.6838694296989\\
-28.076	-27.1144294745358\\
-37.842	-36.5459552705295\\
-53.711	-51.8714603756516\\
-59.814	-57.7654396847801\\
-75.684	-73.0919105410589\\
-86.67	-83.7016527481842\\
-48.828	-47.1556974776548\\
-39.063	-37.7251374328178\\
-76.904	-74.2701269521906\\
-109.863	-106.100319324723\\
-83.008	-80.1650720124758\\
-78.125	-75.449309114479\\
-113.525	-109.636900060432\\
-92.773	-89.5956320573128\\
-74.463	-71.9127283787705\\
-57.373	-55.40804111136\\
-58.594	-56.5872232736484\\
-75.684	-73.0919105410589\\
-52.49	-50.6922782133632\\
-45.166	-43.6191167419463\\
-75.684	-73.0919105410589\\
-95.215	-91.9539963818895\\
-96.436	-93.1331785441779\\
-48.828	-47.1556974776548\\
-25.635	-24.7570309011158\\
-56.152	-54.2288589490717\\
-47.607	-45.9765153153664\\
-21.973	-21.2204501654073\\
-20.752	-20.041268003119\\
-31.738	-30.6510102102443\\
-41.504	-40.0825360062379\\
-67.139	-64.8395669073537\\
-50.049	-48.3348796399432\\
-39.063	-37.7251374328178\\
-28.076	-27.1144294745358\\
-19.531	-18.8620858408306\\
-25.635	-24.7570309011158\\
-23.193	-22.398666576539\\
-26.855	-25.9352473122475\\
-42.725	-41.2617181685263\\
-39.063	-37.7251374328178\\
-24.414	-23.5778487388274\\
-20.752	-20.041268003119\\
-26.855	-25.9352473122475\\
-30.518	-29.4727937991126\\
-20.752	-20.041268003119\\
-31.738	-30.6510102102443\\
-24.414	-23.5778487388274\\
-19.531	-18.8620858408306\\
-36.621	-35.3667731082411\\
-47.607	-45.9765153153664\\
-69.58	-67.1969654807737\\
-92.773	-89.5956320573128\\
-81.787	-78.9858898501874\\
-42.725	-41.2617181685263\\
-32.959	-31.8301923725327\\
-34.18	-33.009374534821\\
-21.973	-21.2204501654073\\
-35.4	-34.1875909459527\\
-39.063	-37.7251374328178\\
-37.842	-36.5459552705295\\
-42.725	-41.2617181685263\\
-48.828	-47.1556974776548\\
-52.49	-50.6922782133632\\
-62.256	-60.1238040093569\\
-86.67	-83.7016527481842\\
-47.607	-45.9765153153664\\
-25.635	-24.7570309011158\\
-24.414	-23.5778487388274\\
-42.725	-41.2617181685263\\
-31.738	-30.6510102102443\\
-34.18	-33.009374534821\\
-18.311	-17.6838694296989\\
-28.076	-27.1144294745358\\
-18.311	-17.6838694296989\\
-13.428	-12.9681065317021\\
-18.311	-17.6838694296989\\
-31.738	-30.6510102102443\\
-36.621	-35.3667731082411\\
-51.27	-49.5140618022315\\
-69.58	-67.1969654807737\\
-52.49	-50.6922782133632\\
-25.635	-24.7570309011158\\
-39.063	-37.7251374328178\\
-45.166	-43.6191167419463\\
-25.635	-24.7570309011158\\
-32.959	-31.8301923725327\\
-58.594	-56.5872232736484\\
-50.049	-48.3348796399432\\
-79.346	-76.6284912767674\\
-92.773	-89.5956320573128\\
-64.697	-62.4812025827769\\
-51.27	-49.5140618022315\\
};
\end{axis}
\end{tikzpicture}%
%	\caption{Scatter plots of the AR(0) term $y(t-1)$ for training datasets. Correlation coefficient with the output signal is shown for each figure.}\label{fig:regr_y}
%\end{figure}
%\begin{figure}[!t]
%	\centering
%	% This file was created by matlab2tikz.
%
\definecolor{mycolor1}{rgb}{0.00000,0.44700,0.74100}%
\definecolor{mycolor2}{rgb}{0.85000,0.32500,0.09800}%
%
\begin{tikzpicture}

\begin{axis}[%
width=4.927cm,
height=2.746cm,
at={(0cm,15.254cm)},
scale only axis,
xmin=-0.018,
xmax=-0.012,
xlabel style={font=\color{white!15!black}},
xlabel={$u(t-1)$},
ymin=-0.146485,
ymax=0,
ylabel style={font=\color{white!15!black}},
ylabel={$\delta^4 y(t)$},
axis background/.style={fill=white},
title style={font=\bfseries},
title={C1, R = 0.696},
axis x line*=bottom,
axis y line*=left
]
\addplot[only marks, mark=*, mark options={}, mark size=1.5000pt, color=mycolor1, fill=mycolor1] table[row sep=crcr]{%
x	y\\
-0.015655	-0.07019\\
-0.0157025	-0.0823975\\
-0.0158375	-0.0640875\\
-0.0157025	-0.03662\\
-0.0149225	-0.05188\\
-0.0148325	-0.042725\\
-0.0148325	-0.0457775\\
-0.01497	-0.05188\\
-0.0149225	-0.01831\\
-0.0140525	-0.0122075\\
-0.01323	-0.0122075\\
-0.0128175	-0.03357\\
-0.0137775	-0.0579825\\
-0.014785	-0.061035\\
-0.01506	-0.0640875\\
-0.0151975	-0.0488275\\
-0.0149225	-0.0305175\\
-0.0143275	-0.061035\\
-0.01497	-0.0488275\\
-0.0148325	-0.03662\\
-0.014375	-0.05188\\
-0.01474	-0.0457775\\
-0.0148325	-0.042725\\
-0.0146475	-0.0732425\\
-0.015105	-0.0640875\\
-0.0151975	-0.0457775\\
-0.01497	-0.07019\\
-0.0152425	-0.07019\\
-0.01538	-0.0549325\\
-0.0151975	-0.0457775\\
-0.01497	-0.0396725\\
-0.0146475	-0.0396725\\
-0.01474	-0.0549325\\
-0.0149225	-0.0549325\\
-0.0149225	-0.0549325\\
-0.0149225	-0.0549325\\
-0.015015	-0.0579825\\
-0.01506	-0.076295\\
-0.0155175	-0.0732425\\
-0.0155175	-0.0488275\\
-0.0151525	-0.0457775\\
-0.01497	-0.027465\\
-0.014465	-0.03662\\
-0.014145	-0.0396725\\
-0.01442	-0.0305175\\
-0.0142375	-0.03357\\
-0.01419	-0.0457775\\
-0.014465	-0.0488275\\
-0.014695	-0.0457775\\
-0.014695	-0.0732425\\
-0.0151525	-0.0549325\\
-0.01506	-0.0305175\\
-0.01451	-0.0305175\\
-0.014145	-0.03357\\
-0.0143275	-0.042725\\
-0.014465	-0.024415\\
-0.0139625	-0.027465\\
-0.0137775	-0.03662\\
-0.0140075	-0.024415\\
-0.013915	-0.01831\\
-0.0136875	-0.03357\\
-0.01387	-0.03357\\
-0.0140075	-0.0549325\\
-0.0146025	-0.05188\\
-0.01474	-0.0640875\\
-0.0149225	-0.0549325\\
-0.01497	-0.0640875\\
-0.01506	-0.05188\\
-0.01497	-0.05188\\
-0.0148775	-0.0457775\\
-0.0148325	-0.0457775\\
-0.014695	-0.042725\\
-0.01474	-0.076295\\
-0.0152425	-0.10376\\
-0.0160225	-0.10376\\
-0.0162975	-0.1068125\\
-0.0162975	-0.07019\\
-0.0157925	-0.094605\\
-0.0160675	-0.1159675\\
-0.016525	-0.1159675\\
-0.0166625	-0.0885\\
-0.0164325	-0.1190175\\
-0.0166175	-0.146485\\
-0.017165	-0.1098625\\
-0.0169825	-0.08545\\
-0.0166175	-0.061035\\
-0.016205	-0.05188\\
-0.0157025	-0.042725\\
-0.01538	-0.0549325\\
-0.0154275	-0.042725\\
-0.0151525	-0.0305175\\
-0.0145575	-0.024415\\
-0.0142375	-0.0305175\\
-0.0143275	-0.0457775\\
-0.014695	-0.0457775\\
-0.01474	-0.0457775\\
-0.01474	-0.0549325\\
-0.015015	-0.0579825\\
-0.0151525	-0.076295\\
-0.01561	-0.07019\\
-0.01561	-0.076295\\
-0.0157025	-0.0640875\\
-0.0155175	-0.05188\\
-0.0152425	-0.061035\\
-0.01538	-0.0457775\\
-0.015105	-0.03357\\
-0.0149225	-0.0488275\\
-0.015015	-0.05188\\
-0.01506	-0.0305175\\
-0.0148325	-0.0396725\\
-0.01474	-0.0305175\\
-0.014465	-0.03357\\
-0.014375	-0.03357\\
-0.014465	-0.024415\\
-0.01419	-0.027465\\
-0.01419	-0.042725\\
-0.014375	-0.0579825\\
-0.0149225	-0.0640875\\
-0.0151525	-0.0671375\\
-0.01529	-0.0396725\\
-0.01474	-0.05188\\
-0.0148775	-0.0823975\\
-0.01561	-0.061035\\
-0.0154725	-0.07019\\
-0.0155175	-0.0732425\\
-0.01561	-0.0457775\\
-0.0151975	-0.0305175\\
-0.0146025	-0.042725\\
-0.01474	-0.0457775\\
-0.0149225	-0.076295\\
-0.0154275	-0.0915525\\
-0.015975	-0.0915525\\
-0.0161125	-0.07019\\
-0.0157925	-0.07019\\
-0.0157475	-0.0640875\\
-0.015655	-0.0640875\\
-0.01561	-0.0640875\\
-0.01561	-0.05188\\
-0.0151975	-0.0396725\\
-0.0148775	-0.03662\\
-0.014785	-0.03357\\
-0.0146025	-0.03357\\
-0.0146475	-0.05188\\
-0.01497	-0.079345\\
-0.015655	-0.0671375\\
-0.015565	-0.042725\\
-0.015105	-0.0396725\\
-0.0148325	-0.0396725\\
-0.01474	-0.024415\\
-0.0142825	-0.01831\\
-0.0137325	-0.01831\\
-0.0136425	-0.0396725\\
-0.014145	-0.027465\\
-0.0141	-0.03662\\
-0.0140525	-0.03662\\
-0.014145	-0.03357\\
-0.0141	-0.024415\\
-0.01387	-0.01526\\
-0.013505	-0.0122075\\
-0.0130925	-0.01831\\
-0.013275	-0.0457775\\
-0.0142375	-0.0671375\\
-0.015015	-0.07019\\
-0.01538	-0.05188\\
-0.01497	-0.03357\\
-0.01451	-0.027465\\
-0.0141	-0.01526\\
-0.01355	-0.0305175\\
-0.0140525	-0.03662\\
-0.0142375	-0.0457775\\
-0.0146025	-0.0671375\\
-0.01506	-0.0976575\\
-0.015885	-0.1190175\\
-0.01657	-0.1068125\\
-0.01648	-0.094605\\
-0.0163875	-0.07019\\
-0.015975	-0.076295\\
-0.015885	-0.079345\\
-0.015975	-0.08545\\
-0.0160225	-0.061035\\
-0.0158375	-0.0579825\\
-0.01561	-0.0579825\\
-0.01561	-0.0549325\\
-0.0154275	-0.0579825\\
-0.0155175	-0.0457775\\
-0.01529	-0.076295\\
-0.0157925	-0.094605\\
-0.01616	-0.07019\\
-0.015885	-0.0457775\\
-0.0152425	-0.0457775\\
-0.015105	-0.076295\\
-0.01561	-0.0885\\
-0.0160225	-0.1007075\\
-0.0164325	-0.112915\\
-0.0166175	-0.112915\\
-0.0167075	-0.08545\\
-0.01648	-0.0823975\\
-0.0162975	-0.0915525\\
-0.0163875	-0.10376\\
-0.01657	-0.0732425\\
-0.01616	-0.042725\\
-0.0154275	-0.0579825\\
-0.01561	-0.05188\\
-0.0154725	-0.03357\\
-0.0149225	-0.0457775\\
-0.01506	-0.0488275\\
-0.0151975	-0.03662\\
-0.015015	-0.03357\\
-0.014695	-0.0396725\\
-0.014785	-0.03357\\
-0.014695	-0.042725\\
-0.0148325	-0.0640875\\
-0.015335	-0.0640875\\
-0.01529	-0.0396725\\
-0.01497	-0.03662\\
-0.0148775	-0.061035\\
-0.015335	-0.1007075\\
-0.0160675	-0.0732425\\
-0.015975	-0.05188\\
-0.0154725	-0.03662\\
-0.015015	-0.0457775\\
-0.0149225	-0.0396725\\
-0.0148775	-0.0305175\\
-0.0145575	-0.03357\\
-0.01451	-0.0396725\\
-0.014695	-0.0457775\\
-0.0149225	-0.05188\\
-0.01506	-0.0396725\\
-0.01474	-0.0579825\\
-0.0151525	-0.0732425\\
-0.0155175	-0.0640875\\
-0.0155175	-0.05188\\
-0.01529	-0.0579825\\
-0.015335	-0.076295\\
-0.015655	-0.0549325\\
-0.01529	-0.0213625\\
-0.014375	-0.03357\\
-0.0142825	-0.03357\\
-0.01442	-0.042725\\
-0.014695	-0.042725\\
-0.014785	-0.03357\\
-0.014465	-0.042725\\
-0.01474	-0.0488275\\
-0.0148325	-0.03357\\
-0.0145575	-0.0396725\\
-0.0146475	-0.0396725\\
-0.0145575	-0.0305175\\
-0.014465	-0.0396725\\
-0.0146025	-0.024415\\
-0.0141	-0.027465\\
-0.0139625	-0.0213625\\
-0.0137775	-0.0213625\\
-0.0137325	-0.03662\\
-0.0143275	-0.05188\\
-0.0146475	-0.061035\\
-0.015105	-0.0885\\
-0.01561	-0.061035\\
-0.015335	-0.03357\\
-0.014695	-0.0305175\\
-0.0142825	-0.027465\\
-0.0141	-0.03357\\
-0.0143275	-0.027465\\
-0.0140525	-0.0305175\\
-0.0139625	-0.03662\\
-0.01419	-0.061035\\
-0.014785	-0.0549325\\
-0.01497	-0.0671375\\
-0.015335	-0.0579825\\
-0.01506	-0.042725\\
-0.0148775	-0.03357\\
-0.0142375	-0.027465\\
-0.013915	-0.0213625\\
-0.013505	-0.01526\\
-0.0133675	-0.024415\\
-0.0134125	-0.03662\\
-0.0140075	-0.0457775\\
-0.0142825	-0.0457775\\
-0.014465	-0.0488275\\
-0.0145575	-0.0732425\\
-0.0152425	-0.0640875\\
-0.0154275	-0.0457775\\
-0.01506	-0.061035\\
-0.0151975	-0.061035\\
-0.01529	-0.079345\\
-0.015565	-0.0671375\\
-0.0154275	-0.042725\\
-0.015015	-0.024415\\
-0.01419	-0.01526\\
-0.01355	-0.01831\\
-0.0133675	-0.024415\\
-0.0134575	-0.0213625\\
-0.013505	-0.0213625\\
-0.0134575	-0.0305175\\
-0.0137325	-0.042725\\
-0.014145	-0.0396725\\
-0.0142375	-0.0396725\\
-0.0143275	-0.0488275\\
-0.014465	-0.03357\\
-0.0140525	-0.0122075\\
-0.0134575	-0.024415\\
-0.013825	-0.0549325\\
-0.014465	-0.0488275\\
-0.0145575	-0.0549325\\
-0.014695	-0.0457775\\
-0.014695	-0.03662\\
-0.014375	-0.0488275\\
-0.0145575	-0.0640875\\
-0.015015	-0.0640875\\
-0.0151975	-0.0488275\\
-0.0148775	-0.0671375\\
-0.01538	-0.0671375\\
-0.015335	-0.0579825\\
-0.0151525	-0.07019\\
-0.015335	-0.0671375\\
-0.015335	-0.05188\\
-0.0151525	-0.0488275\\
-0.0149225	-0.0732425\\
-0.01529	-0.1007075\\
-0.0160225	-0.076295\\
-0.01593	-0.0457775\\
-0.0152425	-0.042725\\
-0.015015	-0.0457775\\
-0.01497	-0.0549325\\
-0.0151975	-0.0671375\\
-0.0154275	-0.0488275\\
-0.0151525	-0.0488275\\
-0.0151525	-0.0671375\\
-0.0154275	-0.07019\\
-0.01561	-0.05188\\
-0.0152425	-0.03662\\
-0.0149225	-0.05188\\
-0.01506	-0.08545\\
-0.015655	-0.076295\\
-0.0158375	-0.079345\\
-0.0157475	-0.0549325\\
-0.01529	-0.042725\\
-0.01497	-0.042725\\
-0.0149225	-0.027465\\
-0.01451	-0.01526\\
-0.01387	-0.01526\\
-0.0133675	-0.0122075\\
-0.013	-0.027465\\
-0.013595	-0.0488275\\
-0.0142375	-0.0457775\\
-0.0145575	-0.0396725\\
-0.01442	-0.03662\\
-0.014375	-0.05188\\
-0.0145575	-0.03662\\
-0.014145	-0.024415\\
-0.0140525	-0.03357\\
-0.0140525	-0.0213625\\
-0.013825	-0.027465\\
-0.01387	-0.0457775\\
-0.01442	-0.0579825\\
-0.0148775	-0.042725\\
-0.01451	-0.024415\\
-0.014145	-0.027465\\
-0.01387	-0.0305175\\
-0.0139625	-0.024415\\
-0.0136875	-0.03357\\
-0.0142825	-0.079345\\
-0.0152425	-0.0640875\\
-0.0151525	-0.079345\\
-0.0154275	-0.094605\\
-0.0158375	-0.0915525\\
-0.01593	-0.0671375\\
-0.0157475	-0.05188\\
-0.01529	-0.0457775\\
-0.0151975	-0.0549325\\
-0.0151975	-0.0579825\\
-0.015335	-0.0732425\\
-0.0155175	-0.0732425\\
-0.015565	-0.0915525\\
-0.01593	-0.1068125\\
-0.0162975	-0.1159675\\
-0.0166175	-0.094605\\
-0.0163425	-0.0885\\
-0.01625	-0.1007075\\
-0.01648	-0.079345\\
-0.016205	-0.08545\\
-0.0162975	-0.079345\\
-0.016205	-0.0457775\\
-0.015565	-0.0305175\\
-0.0148325	-0.03357\\
-0.0146025	-0.0457775\\
-0.0149225	-0.042725\\
-0.014785	-0.0213625\\
-0.014375	-0.03662\\
-0.014465	-0.03357\\
-0.0143275	-0.01526\\
-0.013915	-0.027465\\
-0.0139625	-0.0305175\\
-0.0140525	-0.01831\\
-0.01387	-0.0457775\\
-0.01419	-0.05188\\
-0.0146475	-0.03662\\
-0.0145575	-0.0579825\\
-0.015015	-0.094605\\
-0.0157925	-0.0823975\\
-0.01593	-0.094605\\
-0.01625	-0.0823975\\
-0.015975	-0.0732425\\
-0.01593	-0.05188\\
-0.0155175	-0.03662\\
-0.0149225	-0.042725\\
-0.0148775	-0.03357\\
-0.0146475	-0.027465\\
-0.0142825	-0.03662\\
-0.014465	-0.03357\\
-0.0140075	-0.009155\\
-0.0134125	-0.01831\\
-0.01332	-0.0396725\\
-0.0140075	-0.05188\\
-0.0146025	-0.0549325\\
-0.0149225	-0.0579825\\
-0.01506	-0.0549325\\
-0.015015	-0.0671375\\
-0.0151525	-0.0549325\\
-0.015105	-0.042725\\
-0.0148325	-0.042725\\
-0.0148325	-0.03662\\
-0.0146475	-0.05188\\
-0.01497	-0.0457775\\
-0.01474	-0.03357\\
-0.01442	-0.0457775\\
-0.01474	-0.0671375\\
-0.0151975	-0.05188\\
-0.01497	-0.0396725\\
-0.014695	-0.03357\\
-0.01451	-0.03662\\
-0.0145575	-0.0305175\\
-0.0142825	-0.0213625\\
-0.0140075	-0.03662\\
-0.0142825	-0.0488275\\
-0.01451	-0.0457775\\
-0.014695	-0.0396725\\
-0.0146025	-0.0488275\\
-0.014695	-0.0457775\\
-0.014785	-0.027465\\
-0.0142825	-0.027465\\
-0.014145	-0.0549325\\
-0.01474	-0.076295\\
-0.01538	-0.0732425\\
-0.0154725	-0.0579825\\
-0.01538	-0.0549325\\
-0.0152425	-0.0488275\\
-0.01506	-0.0457775\\
-0.0148775	-0.0396725\\
-0.01474	-0.0488275\\
-0.0149225	-0.042725\\
-0.0148325	-0.03357\\
-0.01442	-0.042725\\
-0.0146025	-0.0488275\\
-0.0148325	-0.0549325\\
-0.015015	-0.0640875\\
-0.0151975	-0.0640875\\
-0.0151975	-0.0823975\\
-0.01561	-0.10376\\
-0.0160675	-0.112915\\
-0.01648	-0.079345\\
-0.0160675	-0.042725\\
-0.015335	-0.03357\\
-0.014695	-0.0213625\\
-0.0140525	-0.027465\\
-0.0141	-0.03357\\
-0.0141	-0.042725\\
-0.0146025	-0.0457775\\
-0.014695	-0.03357\\
-0.0145575	-0.0488275\\
-0.01474	-0.061035\\
-0.01497	-0.0640875\\
-0.0151975	-0.0671375\\
-0.01538	-0.0976575\\
-0.01593	-0.0885\\
-0.0161125	-0.0640875\\
-0.0157925	-0.05188\\
-0.01529	-0.05188\\
-0.0151525	-0.0488275\\
-0.0151525	-0.0579825\\
-0.01538	-0.0488275\\
-0.015105	-0.0396725\\
-0.015015	-0.042725\\
-0.01497	-0.027465\\
-0.01442	-0.03662\\
-0.0146025	-0.07019\\
-0.015105	-0.0488275\\
-0.0149225	-0.0396725\\
-0.0148775	-0.08545\\
-0.01561	-0.07019\\
-0.015565	-0.0549325\\
-0.0154275	-0.08545\\
-0.015885	-0.079345\\
-0.01593	-0.05188\\
-0.0154275	-0.0396725\\
-0.01474	-0.03357\\
-0.0146475	-0.0579825\\
-0.01506	-0.0885\\
-0.0158375	-0.094605\\
-0.0161125	-0.0915525\\
-0.016205	-0.0732425\\
-0.015975	-0.05188\\
-0.0154275	-0.0396725\\
-0.015105	-0.03662\\
-0.014695	-0.027465\\
-0.01442	-0.027465\\
-0.0143275	-0.03357\\
-0.014465	-0.0396725\\
-0.0146475	-0.027465\\
-0.0142375	-0.03357\\
-0.0143275	-0.0549325\\
-0.014785	-0.0549325\\
-0.015015	-0.0671375\\
-0.015335	-0.0885\\
-0.0157925	-0.1098625\\
-0.0162975	-0.07019\\
-0.01616	-0.07019\\
-0.015885	-0.07019\\
-0.015885	-0.061035\\
-0.0157025	-0.0457775\\
-0.01529	-0.05188\\
-0.0152425	-0.08545\\
-0.0157925	-0.094605\\
-0.01616	-0.061035\\
-0.015655	-0.0305175\\
-0.0148775	-0.027465\\
-0.0142375	-0.009155\\
-0.013505	-0.0213625\\
-0.0131375	-0.024415\\
-0.01355	-0.0305175\\
-0.013825	-0.03357\\
-0.01419	-0.03357\\
-0.01419	-0.01831\\
-0.013825	-0.0213625\\
-0.013825	-0.027465\\
-0.0136875	-0.0213625\\
-0.0136425	-0.0305175\\
-0.0137325	-0.03662\\
-0.0141	-0.061035\\
-0.014785	-0.07019\\
-0.015105	-0.061035\\
-0.015105	-0.07019\\
-0.015335	-0.0885\\
-0.0157025	-0.0885\\
-0.01593	-0.1007075\\
-0.01616	-0.1007075\\
-0.01625	-0.0732425\\
-0.015885	-0.0488275\\
-0.0154275	-0.0457775\\
-0.0151975	-0.076295\\
-0.015655	-0.094605\\
-0.015975	-0.0671375\\
-0.0157025	-0.0396725\\
-0.0151975	-0.024415\\
-0.0142825	-0.0122075\\
-0.0136425	-0.01526\\
-0.013505	-0.01526\\
-0.013	-0.01526\\
-0.0133675	-0.0305175\\
-0.01387	-0.03662\\
-0.013915	-0.0305175\\
-0.013825	-0.024415\\
-0.0134125	-0.0122075\\
-0.0130925	-0.0213625\\
-0.01323	-0.0213625\\
-0.01332	-0.0213625\\
-0.0134575	-0.01831\\
-0.01323	-0.0122075\\
-0.0130475	-0.0305175\\
-0.0134575	-0.0640875\\
-0.0146475	-0.061035\\
-0.015335	-0.079345\\
-0.01561	-0.0640875\\
-0.01538	-0.0823975\\
-0.01561	-0.1159675\\
-0.0163425	-0.1068125\\
-0.0164325	-0.0976575\\
-0.01648	-0.08545\\
-0.01616	-0.079345\\
-0.0160675	-0.0823975\\
-0.01616	-0.05188\\
-0.0157925	-0.05188\\
-0.0155175	-0.03357\\
-0.0148775	-0.03662\\
-0.014465	-0.0579825\\
-0.01506	-0.05188\\
-0.0148775	-0.0396725\\
-0.014785	-0.0488275\\
-0.01497	-0.042725\\
-0.0148775	-0.042725\\
-0.0148775	-0.05188\\
-0.01497	-0.0488275\\
-0.01506	-0.061035\\
-0.0151525	-0.05188\\
-0.015105	-0.0396725\\
-0.014785	-0.0457775\\
-0.0149225	-0.03662\\
-0.0143275	-0.0061025\\
-0.01332	-0.024415\\
-0.0131825	-0.03357\\
-0.0136425	-0.0213625\\
-0.013275	-0.0122075\\
-0.012405	-0.0213625\\
-0.01245	-0.0305175\\
-0.0131375	-0.0305175\\
-0.013595	-0.03662\\
-0.013915	-0.03357\\
-0.013915	-0.0488275\\
-0.014465	-0.0488275\\
-0.0146025	-0.0305175\\
-0.0142375	-0.0488275\\
-0.014465	-0.042725\\
-0.0140525	-0.01831\\
-0.013595	-0.01831\\
-0.01323	-0.009155\\
-0.01268	-0.027465\\
-0.0128625	-0.0213625\\
-0.0128625	-0.01831\\
-0.0131825	-0.0488275\\
-0.013915	-0.05188\\
-0.0141	-0.027465\\
-0.01387	-0.03662\\
-0.01387	-0.0305175\\
-0.0137325	-0.03357\\
-0.013915	-0.03662\\
-0.0137325	-0.0213625\\
-0.01355	-0.027465\\
-0.0134575	-0.01831\\
-0.0131825	-0.01831\\
-0.01332	-0.024415\\
-0.013275	-0.03662\\
-0.0136425	-0.03662\\
-0.0137775	-0.024415\\
-0.013595	-0.024415\\
-0.01355	-0.024415\\
-0.01332	-0.024415\\
-0.01332	-0.0457775\\
-0.0141	-0.0671375\\
-0.0148325	-0.0457775\\
-0.0146475	-0.042725\\
-0.01451	-0.0396725\\
-0.014145	-0.0549325\\
-0.01451	-0.0549325\\
-0.014695	-0.03357\\
-0.0142375	-0.03662\\
-0.0143275	-0.03357\\
-0.0140525	-0.042725\\
-0.01419	-0.0305175\\
-0.01387	-0.0213625\\
-0.0136875	-0.027465\\
-0.013915	-0.0396725\\
-0.014145	-0.0305175\\
-0.0139625	-0.03662\\
-0.013915	-0.03662\\
-0.0140075	-0.0549325\\
-0.014465	-0.0457775\\
-0.01451	-0.0640875\\
-0.01497	-0.0885\\
-0.015565	-0.07019\\
-0.0155175	-0.0457775\\
-0.015105	-0.0396725\\
-0.0146475	-0.0579825\\
-0.0148775	-0.07019\\
-0.01529	-0.05188\\
-0.015105	-0.0579825\\
-0.0151525	-0.0457775\\
-0.014785	-0.01831\\
-0.0139625	-0.0213625\\
-0.0136875	-0.05188\\
-0.014375	-0.0457775\\
-0.0146025	-0.0457775\\
-0.01442	-0.0579825\\
-0.0148325	-0.0579825\\
-0.0149225	-0.0488275\\
-0.0148325	-0.0396725\\
-0.01451	-0.0457775\\
-0.0145575	-0.0579825\\
-0.0149225	-0.0488275\\
-0.0148325	-0.0457775\\
-0.014785	-0.0488275\\
-0.014695	-0.03357\\
-0.01442	-0.03662\\
-0.01442	-0.03357\\
-0.01419	-0.03357\\
-0.014145	-0.05188\\
-0.0146475	-0.0640875\\
-0.01497	-0.0579825\\
-0.01506	-0.0579825\\
-0.015015	-0.042725\\
-0.014695	-0.0305175\\
-0.0142375	-0.0396725\\
-0.01419	-0.0396725\\
-0.01442	-0.0549325\\
-0.014695	-0.061035\\
-0.01497	-0.07019\\
-0.01529	-0.0579825\\
-0.0151975	-0.03357\\
-0.0146025	-0.061035\\
-0.01497	-0.0915525\\
-0.015565	-0.061035\\
-0.01538	-0.0579825\\
-0.0152425	-0.0732425\\
-0.0154275	-0.0549325\\
-0.0152425	-0.05188\\
-0.0149225	-0.05188\\
-0.015015	-0.0457775\\
-0.01497	-0.05188\\
-0.01497	-0.0640875\\
-0.0152425	-0.0488275\\
-0.01506	-0.0549325\\
-0.015105	-0.0549325\\
-0.015105	-0.05188\\
-0.015105	-0.0488275\\
-0.0148775	-0.0549325\\
-0.01497	-0.042725\\
-0.01474	-0.0457775\\
-0.0146475	-0.042725\\
-0.01474	-0.042725\\
-0.014785	-0.0213625\\
-0.01419	-0.0061025\\
-0.0134125	-0.0061025\\
-0.012955	-0.0213625\\
-0.0130475	-0.0488275\\
-0.0141	-0.0640875\\
-0.0148775	-0.0579825\\
-0.01497	-0.042725\\
-0.0146475	-0.05188\\
-0.0146475	-0.0488275\\
-0.0146475	-0.0396725\\
-0.014465	-0.0305175\\
-0.0140075	-0.03662\\
-0.0141	-0.03662\\
-0.014145	-0.0305175\\
-0.014145	-0.0396725\\
-0.0142825	-0.03357\\
-0.01419	-0.027465\\
-0.0141	-0.0213625\\
-0.0136875	-0.03662\\
-0.0137325	-0.0549325\\
-0.014465	-0.042725\\
-0.014375	-0.042725\\
-0.014465	-0.042725\\
-0.014465	-0.0579825\\
-0.014695	-0.05188\\
-0.0148325	-0.07019\\
-0.0151525	-0.0549325\\
-0.015015	-0.0579825\\
-0.01497	-0.0579825\\
-0.015105	-0.0396725\\
-0.01451	-0.05188\\
-0.014785	-0.0488275\\
-0.0146025	-0.05188\\
-0.0146475	-0.0549325\\
-0.014785	-0.0640875\\
-0.015015	-0.0488275\\
-0.0149225	-0.061035\\
-0.01506	-0.076295\\
-0.0154275	-0.0640875\\
-0.01529	-0.0579825\\
-0.0151525	-0.0457775\\
-0.0149225	-0.0549325\\
-0.0149225	-0.0488275\\
-0.0148325	-0.03662\\
-0.014695	-0.03357\\
-0.01451	-0.0396725\\
-0.01451	-0.03357\\
-0.014465	-0.0671375\\
-0.01506	-0.0885\\
-0.01561	-0.0732425\\
-0.01561	-0.0732425\\
-0.015565	-0.0732425\\
-0.0155175	-0.0457775\\
-0.01497	-0.03662\\
-0.0146475	-0.0305175\\
-0.01442	-0.027465\\
-0.014145	-0.027465\\
-0.014145	-0.0457775\\
-0.0145575	-0.0579825\\
-0.0149225	-0.0640875\\
-0.0151525	-0.0549325\\
-0.01506	-0.0396725\\
-0.0146475	-0.0640875\\
-0.01506	-0.076295\\
-0.0154725	-0.079345\\
-0.0157025	-0.0640875\\
-0.015565	-0.1068125\\
-0.0161125	-0.0823975\\
-0.0160675	-0.0457775\\
-0.015335	-0.03357\\
-0.014785	-0.03357\\
-0.01442	-0.0488275\\
-0.01474	-0.0671375\\
-0.0151525	-0.079345\\
-0.01561	-0.0671375\\
-0.0155175	-0.0488275\\
-0.0151975	-0.03357\\
-0.0146025	-0.03357\\
-0.0142375	-0.03357\\
-0.0143275	-0.0396725\\
-0.014375	-0.024415\\
-0.0140075	-0.024415\\
-0.0139625	-0.0213625\\
-0.013915	-0.0122075\\
-0.0133675	-0.027465\\
-0.0137775	-0.03662\\
-0.0141	-0.0396725\\
-0.01442	-0.0396725\\
-0.01442	-0.042725\\
-0.01451	-0.0457775\\
-0.0145575	-0.0579825\\
-0.014785	-0.03662\\
-0.0146025	-0.0549325\\
-0.014785	-0.0671375\\
-0.0151525	-0.0549325\\
-0.015015	-0.0579825\\
-0.01497	-0.0549325\\
-0.0148775	-0.05188\\
-0.0149225	-0.0732425\\
-0.015335	-0.0579825\\
-0.0151975	-0.042725\\
-0.0149225	-0.05188\\
-0.01497	-0.05188\\
-0.0148775	-0.03357\\
-0.0145575	-0.05188\\
-0.01474	-0.0671375\\
-0.01529	-0.07019\\
-0.01538	-0.0732425\\
-0.0154275	-0.112915\\
-0.0161125	-0.076295\\
-0.01593	-0.0671375\\
-0.015565	-0.0457775\\
-0.01529	-0.0671375\\
-0.015335	-0.0885\\
-0.0158375	-0.061035\\
-0.015565	-0.0488275\\
-0.0152425	-0.0671375\\
-0.015565	-0.05188\\
-0.0152425	-0.027465\\
-0.0146025	-0.0396725\\
-0.01451	-0.03357\\
-0.014375	-0.0213625\\
-0.0140075	-0.0213625\\
-0.0139625	-0.0213625\\
-0.013825	-0.01831\\
-0.0136875	-0.027465\\
-0.0137775	-0.03662\\
-0.01419	-0.042725\\
-0.01442	-0.0579825\\
-0.0148325	-0.0549325\\
-0.0149225	-0.0640875\\
-0.0151525	-0.07019\\
-0.0154275	-0.07019\\
-0.0154275	-0.0488275\\
-0.015105	-0.0549325\\
-0.01506	-0.0488275\\
-0.015015	-0.05188\\
-0.01506	-0.0732425\\
-0.01538	-0.0579825\\
-0.01529	-0.0671375\\
-0.0152425	-0.0579825\\
-0.0151975	-0.0488275\\
-0.015105	-0.0732425\\
-0.0154725	-0.0640875\\
-0.0154725	-0.0671375\\
-0.0154275	-0.061035\\
-0.015335	-0.0396725\\
-0.0148325	-0.024415\\
-0.0142825	-0.0213625\\
-0.01387	-0.027465\\
-0.01419	-0.0305175\\
-0.0143275	-0.0457775\\
-0.014465	-0.03357\\
-0.014375	-0.0549325\\
-0.0145575	-0.079345\\
-0.01529	-0.0915525\\
-0.0157925	-0.0671375\\
-0.0157025	-0.0732425\\
-0.015655	-0.0671375\\
-0.01561	-0.0457775\\
-0.0152425	-0.0549325\\
-0.0151975	-0.0579825\\
-0.0152425	-0.0488275\\
-0.0151525	-0.0579825\\
-0.0151975	-0.0671375\\
-0.0154725	-0.0976575\\
-0.015975	-0.08545\\
-0.0160675	-0.079345\\
-0.015975	-0.094605\\
-0.0161125	-0.07019\\
-0.0158375	-0.076295\\
-0.0157475	-0.061035\\
-0.0154725	-0.0579825\\
-0.01538	-0.076295\\
-0.01561	-0.08545\\
-0.0158375	-0.024415\\
-0.01506	-0.05188\\
-0.01474	-0.03662\\
-0.014695	-0.0579825\\
-0.0148775	-0.05188\\
-0.015015	-0.03357\\
-0.0146025	-0.042725\\
-0.0146475	-0.0671375\\
-0.0151975	-0.1098625\\
-0.0161125	-0.1068125\\
-0.01648	-0.094605\\
-0.0163875	-0.0823975\\
-0.01616	-0.0976575\\
-0.01625	-0.112915\\
-0.016525	-0.0823975\\
-0.0162975	-0.08545\\
-0.016205	-0.0915525\\
-0.0162975	-0.0640875\\
-0.01593	-0.0549325\\
-0.01561	-0.0671375\\
-0.0157925	-0.0488275\\
-0.0154275	-0.03357\\
-0.01497	-0.0488275\\
-0.01506	-0.03357\\
-0.0148775	-0.0488275\\
-0.0148325	-0.0396725\\
-0.0148325	-0.0488275\\
-0.01497	-0.061035\\
-0.01529	-0.0549325\\
-0.0152425	-0.0396725\\
-0.0149225	-0.0488275\\
-0.01497	-0.0549325\\
-0.015105	-0.0579825\\
-0.01529	-0.0488275\\
-0.0152425	-0.0396725\\
-0.015015	-0.0671375\\
-0.0154275	-0.0671375\\
-0.0154725	-0.0457775\\
-0.015105	-0.061035\\
-0.0154275	-0.0549325\\
-0.0155175	-0.0457775\\
-0.0152425	-0.03357\\
-0.014785	-0.0213625\\
-0.0143275	-0.0122075\\
-0.01387	-0.0457775\\
-0.0143275	-0.03357\\
-0.01451	-0.0305175\\
-0.0142375	-0.01831\\
-0.013915	-0.0305175\\
-0.013825	-0.042725\\
-0.0143275	-0.05188\\
-0.0146475	-0.0579825\\
-0.0149225	-0.07019\\
-0.01529	-0.0671375\\
-0.015335	-0.0671375\\
-0.0154275	-0.0457775\\
-0.01506	-0.01831\\
-0.0140525	-0.0396725\\
-0.0140075	-0.027465\\
-0.0139625	-0.03662\\
-0.0140525	-0.05188\\
-0.014695	-0.05188\\
-0.0149225	-0.0915525\\
-0.0157025	-0.07019\\
-0.015565	-0.0640875\\
-0.01529	-0.08545\\
-0.015655	-0.0579825\\
-0.0155175	-0.03357\\
-0.015015	-0.03357\\
-0.0145575	-0.042725\\
-0.0146475	-0.05188\\
-0.01497	-0.076295\\
-0.0154725	-0.0549325\\
-0.01529	-0.0488275\\
-0.01497	-0.042725\\
-0.0148775	-0.05188\\
-0.01497	-0.0640875\\
-0.01529	-0.1007075\\
-0.015975	-0.0823975\\
-0.0160675	-0.0823975\\
-0.0160225	-0.0549325\\
-0.015565	-0.03662\\
-0.0149225	-0.03662\\
-0.0148775	-0.01831\\
-0.014145	-0.024415\\
-0.01419	-0.0213625\\
-0.0142375	-0.03357\\
-0.0142375	-0.0579825\\
-0.014785	-0.079345\\
-0.0154275	-0.079345\\
-0.0157475	-0.1098625\\
-0.016205	-0.0885\\
-0.0162975	-0.05188\\
-0.015565	-0.0549325\\
-0.0151975	-0.0488275\\
-0.015105	-0.0549325\\
-0.0152425	-0.0732425\\
-0.0155175	-0.0915525\\
-0.015975	-0.0671375\\
-0.0157925	-0.0488275\\
-0.01538	-0.0549325\\
-0.01538	-0.0732425\\
-0.015655	-0.0488275\\
-0.01538	-0.03662\\
-0.01497	-0.03357\\
-0.0146025	-0.024415\\
-0.014145	-0.01831\\
-0.013825	-0.01526\\
-0.013505	-0.0305175\\
-0.0137325	-0.042725\\
-0.0142375	-0.0457775\\
-0.014465	-0.027465\\
-0.0142825	-0.0305175\\
-0.014145	-0.01526\\
-0.0134575	-0.0061025\\
-0.012635	-0.01526\\
-0.0124975	-0.027465\\
-0.0130475	-0.03357\\
-0.013595	-0.042725\\
-0.0140525	-0.0488275\\
-0.01442	-0.0579825\\
-0.014695	-0.0732425\\
-0.0151975	-0.1007075\\
-0.01593	-0.1007075\\
-0.016205	-0.1068125\\
-0.0163425	-0.094605\\
-0.0163425	-0.0549325\\
-0.015655	-0.05188\\
-0.01529	-0.0549325\\
-0.0152425	-0.07019\\
-0.0154725	-0.0549325\\
-0.0154275	-0.042725\\
-0.01506	-0.0305175\\
-0.014695	-0.0305175\\
-0.0143275	-0.0488275\\
-0.0146475	-0.042725\\
-0.014785	-0.0213625\\
-0.0142375	-0.01526\\
-0.0137325	-0.027465\\
-0.013825	-0.03662\\
-0.0142825	-0.024415\\
-0.014145	-0.0457775\\
-0.014375	-0.0305175\\
-0.0142825	-0.0549325\\
-0.0145575	-0.0732425\\
-0.0151975	-0.0549325\\
-0.0151525	-0.0732425\\
-0.0154275	-0.1007075\\
-0.0160675	-0.1068125\\
-0.0163875	-0.079345\\
-0.016205	-0.061035\\
-0.015565	-0.042725\\
-0.0151525	-0.027465\\
-0.014465	-0.0457775\\
-0.0145575	-0.024415\\
-0.014375	-0.05188\\
-0.014785	-0.03662\\
-0.01497	-0.03662\\
-0.014695	-0.0488275\\
-0.0148325	-0.0305175\\
-0.014375	-0.05188\\
-0.01451	-0.03357\\
-0.01442	-0.024415\\
-0.013915	-0.027465\\
-0.01387	-0.01831\\
-0.0136875	-0.0213625\\
-0.013595	-0.01526\\
-0.0134575	-0.01831\\
-0.0131375	-0.01526\\
-0.01323	-0.024415\\
-0.0134575	-0.024415\\
-0.013505	-0.024415\\
-0.013505	-0.03662\\
-0.0140525	-0.05188\\
-0.01451	-0.0396725\\
-0.014465	-0.0396725\\
-0.014375	-0.0640875\\
-0.0148325	-0.076295\\
-0.0152425	-0.0549325\\
-0.0151975	-0.0640875\\
-0.0151525	-0.024415\\
-0.014465	-0.01831\\
-0.0137325	-0.0305175\\
-0.0140525	-0.05188\\
-0.0146025	-0.05188\\
-0.0148325	-0.076295\\
-0.01529	-0.0732425\\
-0.0154275	-0.0488275\\
-0.0151525	-0.03662\\
-0.014695	-0.0457775\\
-0.01451	-0.042725\\
-0.0146025	-0.03662\\
-0.014465	-0.01831\\
-0.0139625	-0.03357\\
-0.01387	-0.027465\\
-0.0136875	-0.01831\\
-0.013505	-0.01831\\
-0.0130925	-0.0305175\\
-0.0134575	-0.03662\\
-0.013915	-0.0396725\\
-0.0141	-0.0213625\\
-0.013915	-0.0488275\\
-0.014375	-0.0640875\\
-0.015015	-0.042725\\
-0.01474	-0.0640875\\
-0.0148775	-0.0640875\\
-0.0151525	-0.0488275\\
-0.0148775	-0.0305175\\
-0.0143275	-0.0488275\\
-0.0142375	-0.042725\\
-0.01451	-0.024415\\
-0.0140525	-0.03662\\
-0.013915	-0.01831\\
-0.0137775	-0.0122075\\
-0.013275	-0.0122075\\
-0.0130925	-0.0213625\\
-0.01332	-0.03357\\
-0.0136875	-0.0305175\\
-0.0136425	-0.03357\\
-0.013825	-0.0396725\\
-0.0141	-0.024415\\
-0.01387	-0.01831\\
-0.01332	-0.01526\\
-0.012955	-0.01831\\
-0.0130475	-0.024415\\
-0.0131375	-0.024415\\
-0.0133675	-0.0396725\\
-0.0141	-0.0457775\\
-0.01451	-0.0396725\\
-0.014375	-0.042725\\
-0.014375	-0.0549325\\
-0.0146475	-0.07019\\
-0.015105	-0.076295\\
-0.0154275	-0.076295\\
-0.0155175	-0.0549325\\
-0.01529	-0.061035\\
-0.0152425	-0.061035\\
-0.01529	-0.0640875\\
-0.015335	-0.0732425\\
-0.0155175	-0.094605\\
-0.015885	-0.0823975\\
-0.015885	-0.0732425\\
-0.0157475	-0.061035\\
-0.01561	-0.061035\\
-0.0154275	-0.0579825\\
-0.0154275	-0.0732425\\
-0.0154725	-0.042725\\
-0.015105	-0.0549325\\
-0.01497	-0.0579825\\
-0.0151975	-0.0671375\\
-0.01538	-0.05188\\
-0.0152425	-0.0640875\\
-0.0152425	-0.0457775\\
-0.0151975	-0.0457775\\
-0.01497	-0.0305175\\
-0.01451	-0.05188\\
-0.014695	-0.07019\\
-0.01529	-0.05188\\
-0.0151975	-0.03357\\
-0.0146475	-0.03357\\
-0.0145575	-0.024415\\
-0.0140075	-0.01831\\
-0.013595	-0.024415\\
-0.013825	-0.0122075\\
-0.01355	-0.01831\\
-0.0133675	-0.0213625\\
-0.0134575	-0.0122075\\
-0.0131825	-0.024415\\
-0.0134575	-0.01831\\
-0.0134575	-0.0122075\\
-0.0130475	-0.01831\\
-0.013275	-0.0305175\\
-0.0134125	-0.027465\\
-0.0136425	-0.03357\\
-0.0136875	-0.03357\\
-0.0137325	-0.0396725\\
-0.0140525	-0.03357\\
-0.0141	-0.042725\\
-0.014145	-0.0732425\\
-0.015015	-0.05188\\
-0.014785	-0.03357\\
-0.0145575	-0.0488275\\
-0.014695	-0.03357\\
-0.014375	-0.0213625\\
-0.0137775	-0.0396725\\
-0.0141	-0.0213625\\
-0.0141	-0.01526\\
-0.013595	-0.0305175\\
-0.013915	-0.0396725\\
-0.0140075	-0.0396725\\
-0.014145	-0.024415\\
-0.01387	-0.01526\\
-0.0133675	-0.01526\\
-0.0130475	-0.01526\\
-0.013	-0.0305175\\
-0.0134575	-0.0305175\\
-0.013595	-0.0305175\\
-0.0136875	-0.0457775\\
-0.0141	-0.061035\\
-0.0146475	-0.0396725\\
-0.0145575	-0.0579825\\
-0.014785	-0.07019\\
-0.0151525	-0.0640875\\
-0.015105	-0.0549325\\
-0.01506	-0.0915525\\
-0.0157925	-0.076295\\
-0.0157475	-0.0823975\\
-0.015885	-0.07019\\
-0.0157925	-0.0885\\
-0.0158375	-0.094605\\
-0.0160675	-0.0640875\\
-0.015565	-0.03357\\
-0.0148325	-0.03357\\
-0.01442	-0.0457775\\
-0.0146475	-0.0549325\\
-0.0149225	-0.0579825\\
-0.01497	-0.03662\\
-0.01474	-0.0213625\\
-0.0140525	-0.03357\\
-0.0139625	-0.0549325\\
-0.0146475	-0.07019\\
-0.01529	-0.0640875\\
-0.01538	-0.0396725\\
-0.0148775	-0.03357\\
-0.014465	-0.042725\\
-0.0146475	-0.0671375\\
-0.0151975	-0.076295\\
-0.015565	-0.076295\\
-0.015655	-0.0579825\\
-0.01529	-0.079345\\
-0.015565	-0.08545\\
-0.0158375	-0.076295\\
-0.0157925	-0.1068125\\
-0.01625	-0.0915525\\
-0.01625	-0.076295\\
-0.0160675	-0.094605\\
-0.01625	-0.0915525\\
-0.01616	-0.0457775\\
-0.0154275	-0.042725\\
-0.01506	-0.05188\\
-0.0152425	-0.0640875\\
-0.0154275	-0.0671375\\
-0.0154725	-0.0457775\\
-0.0151975	-0.061035\\
-0.01538	-0.0549325\\
-0.01529	-0.03662\\
-0.01497	-0.0488275\\
-0.015105	-0.0457775\\
-0.01506	-0.0671375\\
-0.015565	-0.076295\\
-0.0157025	-0.0579825\\
-0.0154275	-0.042725\\
-0.01506	-0.027465\\
-0.01442	-0.0396725\\
-0.01451	-0.0305175\\
-0.01442	-0.0305175\\
-0.01419	-0.01831\\
-0.0140525	-0.0396725\\
-0.014375	-0.0488275\\
-0.014785	-0.0579825\\
-0.0149225	-0.0549325\\
-0.01497	-0.03357\\
-0.0145575	-0.024415\\
-0.014145	-0.01831\\
-0.013825	-0.024415\\
-0.0137325	-0.03662\\
-0.01419	-0.042725\\
-0.0143275	-0.03662\\
-0.014375	-0.061035\\
-0.0148325	-0.0671375\\
-0.0152425	-0.076295\\
-0.0155175	-0.076295\\
-0.01561	-0.0885\\
-0.015885	-0.0579825\\
-0.0154725	-0.0488275\\
-0.01506	-0.03662\\
-0.014785	-0.042725\\
-0.014695	-0.05188\\
-0.0149225	-0.0396725\\
-0.0148775	-0.027465\\
-0.014465	-0.027465\\
-0.0142375	-0.0457775\\
-0.014695	-0.0579825\\
-0.01506	-0.05188\\
-0.015015	-0.079345\\
-0.0155175	-0.0915525\\
-0.015885	-0.0640875\\
-0.01561	-0.0457775\\
-0.015105	-0.03357\\
-0.0146025	-0.027465\\
-0.01442	-0.05188\\
-0.01474	-0.03662\\
-0.014785	-0.042725\\
-0.0149225	-0.0579825\\
-0.01506	-0.0640875\\
-0.0151975	-0.061035\\
-0.01529	-0.07019\\
-0.01538	-0.0457775\\
-0.01506	-0.0549325\\
-0.015015	-0.0396725\\
-0.0148325	-0.03662\\
-0.0145575	-0.05188\\
-0.0149225	-0.07019\\
-0.01538	-0.076295\\
-0.01561	-0.0457775\\
-0.0151525	-0.0732425\\
-0.0157925	-0.076295\\
-0.0160225	-0.0549325\\
-0.015565	-0.03662\\
-0.0148325	-0.0549325\\
-0.015015	-0.042725\\
-0.01497	-0.042725\\
-0.0148775	-0.0488275\\
-0.015015	-0.03357\\
-0.014785	-0.0457775\\
-0.0149225	-0.08545\\
-0.0157025	-0.0915525\\
-0.015975	-0.0640875\\
-0.01561	-0.07019\\
-0.0155175	-0.07019\\
-0.015655	-0.07019\\
-0.015655	-0.05188\\
-0.01538	-0.05188\\
-0.01529	-0.03662\\
-0.0148775	-0.05188\\
-0.015015	-0.05188\\
-0.0151525	-0.0823975\\
-0.0157025	-0.094605\\
-0.0160225	-0.1190175\\
-0.01657	-0.094605\\
-0.0163875	-0.076295\\
-0.0160675	-0.0579825\\
-0.0157925	-0.0640875\\
-0.0157475	-0.0488275\\
-0.0154275	-0.03662\\
-0.01497	-0.03357\\
-0.01474	-0.0396725\\
-0.014785	-0.027465\\
-0.01442	-0.01831\\
-0.0139625	-0.0305175\\
-0.0141	-0.042725\\
-0.01451	-0.0305175\\
-0.0142375	-0.01831\\
-0.013915	-0.01831\\
-0.0137325	-0.0457775\\
-0.014465	-0.0640875\\
-0.01506	-0.0305175\\
-0.014465	-0.0305175\\
-0.014375	-0.0396725\\
-0.01442	-0.03662\\
-0.01442	-0.0457775\\
-0.0146025	-0.042725\\
-0.0145575	-0.027465\\
-0.0142825	-0.0213625\\
-0.013915	-0.01831\\
-0.01355	-0.024415\\
-0.0136425	-0.0457775\\
-0.0142825	-0.0457775\\
-0.0146475	-0.0396725\\
-0.01442	-0.027465\\
-0.0140075	-0.01831\\
-0.013505	-0.0213625\\
-0.01355	-0.042725\\
-0.0141	-0.0488275\\
-0.01451	-0.0488275\\
-0.014695	-0.0396725\\
-0.01442	-0.03357\\
-0.0142825	-0.03662\\
-0.0143275	-0.05188\\
-0.0146475	-0.07019\\
-0.0151975	-0.0823975\\
-0.01561	-0.061035\\
-0.0154275	-0.042725\\
-0.0148775	-0.0488275\\
-0.01474	-0.0457775\\
-0.01474	-0.042725\\
-0.01474	-0.0305175\\
-0.014375	-0.03357\\
-0.014465	-0.0396725\\
-0.0145575	-0.03662\\
-0.0145575	-0.027465\\
-0.0141	-0.01526\\
-0.013595	-0.0213625\\
-0.0137775	-0.03357\\
-0.0141	-0.05188\\
-0.014695	-0.061035\\
-0.015015	-0.07019\\
-0.01529	-0.0640875\\
-0.01529	-0.0732425\\
-0.0154275	-0.0915525\\
-0.0158375	-0.1190175\\
-0.0164325	-0.094605\\
-0.01648	-0.07019\\
-0.015975	-0.079345\\
-0.01593	-0.0915525\\
-0.0161125	-0.1098625\\
-0.01648	-0.076295\\
-0.0161125	-0.03357\\
-0.0151525	-0.024415\\
-0.01442	-0.027465\\
-0.01419	-0.0122075\\
-0.013915	-0.0122075\\
-0.0133675	-0.01526\\
-0.01323	-0.024415\\
-0.013505	-0.03357\\
-0.013915	-0.03662\\
-0.0140525	-0.027465\\
-0.01387	-0.0213625\\
-0.0137775	-0.03357\\
-0.0139625	-0.0396725\\
-0.014145	-0.042725\\
-0.0143275	-0.0396725\\
-0.0143275	-0.03357\\
-0.0140525	-0.0213625\\
-0.0136875	-0.0213625\\
-0.013505	-0.027465\\
-0.0137775	-0.03662\\
-0.0140075	-0.027465\\
-0.01387	-0.01831\\
-0.0134575	-0.027465\\
-0.0137775	-0.027465\\
-0.013595	-0.0213625\\
-0.0136425	-0.03357\\
-0.01387	-0.0457775\\
-0.0142375	-0.0457775\\
-0.01451	-0.03357\\
-0.01419	-0.0488275\\
-0.01451	-0.05188\\
-0.014695	-0.042725\\
-0.01451	-0.03662\\
-0.0143275	-0.0549325\\
-0.014695	-0.0671375\\
-0.0151525	-0.0732425\\
-0.0152425	-0.0732425\\
-0.0154275	-0.061035\\
-0.01529	-0.0549325\\
-0.015105	-0.05188\\
-0.01506	-0.07019\\
-0.01529	-0.0549325\\
-0.0151975	-0.0671375\\
-0.0154275	-0.07019\\
-0.0155175	-0.0671375\\
-0.0154725	-0.1190175\\
-0.0163425	-0.094605\\
-0.0162975	-0.05188\\
-0.015565	-0.03662\\
-0.0149225	-0.03662\\
-0.014785	-0.0457775\\
-0.0149225	-0.061035\\
-0.0151975	-0.0671375\\
-0.0154725	-0.0732425\\
-0.015655	-0.079345\\
-0.015655	-0.05188\\
-0.015335	-0.0488275\\
-0.015105	-0.05188\\
-0.0151975	-0.05188\\
-0.0152425	-0.03662\\
-0.014785	-0.0305175\\
-0.0142825	-0.042725\\
-0.0145575	-0.042725\\
-0.014695	-0.0549325\\
-0.01497	-0.07019\\
-0.01538	-0.0640875\\
-0.01529	-0.0457775\\
-0.01497	-0.03662\\
-0.014695	-0.0549325\\
-0.01497	-0.07019\\
-0.01538	-0.08545\\
-0.0157475	-0.094605\\
-0.0160675	-0.1068125\\
-0.0163425	-0.0885\\
-0.01625	-0.0823975\\
-0.0161125	-0.05188\\
-0.0155175	-0.0305175\\
-0.0148775	-0.061035\\
-0.01529	-0.0823975\\
-0.0157475	-0.0488275\\
-0.015335	-0.0640875\\
-0.01561	-0.0915525\\
-0.01616	-0.112915\\
-0.01648	-0.079345\\
-0.0161125	-0.0457775\\
-0.015335	-0.0305175\\
-0.0146025	-0.0396725\\
-0.0146475	-0.03662\\
-0.014695	-0.03357\\
-0.0146475	-0.027465\\
-0.0142375	-0.0305175\\
-0.0142825	-0.024415\\
-0.014145	-0.0305175\\
-0.0142375	-0.027465\\
-0.0141	-0.03357\\
-0.01419	-0.05188\\
-0.014695	-0.05188\\
-0.01474	-0.061035\\
-0.015105	-0.076295\\
-0.0155175	-0.079345\\
-0.01561	-0.0488275\\
-0.01506	-0.042725\\
-0.0148775	-0.0549325\\
-0.01506	-0.03662\\
-0.014785	-0.03357\\
-0.0145575	-0.027465\\
-0.0142375	-0.03662\\
-0.014375	-0.03662\\
-0.01442	-0.0213625\\
-0.0140075	-0.0122075\\
-0.0133675	-0.01526\\
-0.013	-0.0213625\\
-0.0134125	-0.03662\\
-0.013915	-0.03662\\
-0.014145	-0.024415\\
-0.0137775	-0.03357\\
-0.013825	-0.042725\\
-0.0142375	-0.05188\\
-0.014695	-0.042725\\
-0.014465	-0.03662\\
-0.0142825	-0.042725\\
-0.01442	-0.05188\\
-0.0146475	-0.0457775\\
-0.0146475	-0.05188\\
-0.01474	-0.05188\\
-0.014785	-0.03357\\
-0.01451	-0.0305175\\
-0.0142375	-0.0396725\\
-0.0143275	-0.03357\\
-0.0142375	-0.03662\\
-0.014375	-0.05188\\
-0.014695	-0.076295\\
-0.015335	-0.061035\\
-0.0151975	-0.05188\\
-0.015015	-0.079345\\
-0.01561	-0.0579825\\
-0.0155175	-0.0396725\\
-0.01497	-0.03662\\
-0.01442	-0.024415\\
-0.0141	-0.024415\\
-0.0140075	-0.01831\\
-0.0137775	-0.01831\\
-0.0136875	-0.042725\\
-0.0142825	-0.03357\\
-0.0137325	-0.01831\\
-0.0133675	-0.03357\\
-0.0136875	-0.03662\\
-0.0140525	-0.027465\\
-0.0140075	-0.024415\\
-0.0136875	-0.0122075\\
-0.01323	-0.01526\\
-0.0133675	-0.042725\\
-0.0140525	-0.05188\\
-0.0146025	-0.0396725\\
-0.0145575	-0.0305175\\
-0.0142825	-0.042725\\
-0.014375	-0.0457775\\
-0.014375	-0.05188\\
-0.0145575	-0.0549325\\
-0.014785	-0.0671375\\
-0.015105	-0.0671375\\
-0.0151975	-0.0549325\\
-0.01506	-0.0396725\\
-0.0146475	-0.0305175\\
-0.0143275	-0.03357\\
-0.0142375	-0.0396725\\
-0.0145575	-0.03357\\
-0.01419	-0.0213625\\
-0.013915	-0.0457775\\
-0.01442	-0.0579825\\
-0.014785	-0.05188\\
-0.014785	-0.061035\\
-0.01497	-0.076295\\
-0.01538	-0.05188\\
-0.0148775	-0.0579825\\
-0.01529	-0.076295\\
-0.01561	-0.061035\\
-0.0154725	-0.0732425\\
-0.0155175	-0.0732425\\
-0.015565	-0.112915\\
-0.016205	-0.1068125\\
-0.016525	-0.076295\\
-0.0160675	-0.0823975\\
-0.01616	-0.1068125\\
-0.01648	-0.10376\\
-0.016525	-0.1159675\\
-0.0166625	-0.1098625\\
-0.0166625	-0.1342775\\
-0.01703	-0.1159675\\
-0.0169375	-0.0732425\\
-0.01648	-0.0488275\\
-0.0157475	-0.05188\\
-0.0155175	-0.03662\\
-0.01529	-0.03662\\
-0.01497	-0.0488275\\
-0.0151525	-0.0732425\\
-0.015655	-0.05188\\
-0.01529	-0.03662\\
-0.01497	-0.042725\\
-0.0149225	-0.0305175\\
-0.01451	-0.0305175\\
-0.01451	-0.009155\\
-0.0140075	-0.024415\\
-0.0137775	-0.024415\\
-0.01355	-0.024415\\
-0.0137325	-0.027465\\
-0.0136425	-0.01526\\
-0.01332	-0.0122075\\
-0.013	-0.009155\\
-0.012635	-0.01831\\
-0.012955	-0.01831\\
-0.0128625	-0.01526\\
-0.012725	-0.03357\\
-0.013505	-0.0457775\\
-0.0141	-0.042725\\
-0.014375	-0.03357\\
-0.0141	-0.0396725\\
-0.0142825	-0.0671375\\
-0.0149225	-0.0457775\\
-0.01442	-0.0122075\\
-0.0137325	-0.01831\\
-0.01332	-0.0122075\\
-0.012955	-0.0122075\\
-0.0130925	-0.0396725\\
-0.0136425	-0.0457775\\
-0.0142825	-0.0305175\\
-0.013915	-0.0457775\\
-0.0143275	-0.0732425\\
-0.015015	-0.0640875\\
-0.0151975	-0.0457775\\
-0.015015	-0.03357\\
-0.014375	-0.03662\\
-0.014375	-0.0579825\\
-0.01474	-0.0671375\\
-0.0151525	-0.0457775\\
-0.014695	-0.027465\\
-0.0140075	-0.0305175\\
-0.0141	-0.0305175\\
-0.0140075	-0.0122075\\
-0.013275	-0.009155\\
-0.012635	-0.01831\\
-0.0130475	-0.05188\\
-0.0142375	-0.05188\\
-0.0146025	-0.03357\\
-0.0143275	-0.027465\\
-0.0137775	-0.027465\\
-0.01355	-0.027465\\
-0.0125875	-0.0122075\\
-0.012635	-0.0061025\\
-0.01268	-0.01831\\
-0.013	-0.0396725\\
-0.01355	-0.0549325\\
-0.0142825	-0.0579825\\
-0.0146475	-0.0549325\\
-0.014785	-0.0579825\\
-0.0148775	-0.0579825\\
-0.0149225	-0.061035\\
-0.0149225	-0.0488275\\
-0.014785	-0.0640875\\
-0.0149225	-0.0885\\
-0.015565	-0.08545\\
-0.0157475	-0.0457775\\
-0.01506	-0.0213625\\
-0.0141	-0.03662\\
-0.014695	-0.07019\\
-0.0151975	-0.0549325\\
-0.015105	-0.0396725\\
-0.01451	-0.03357\\
-0.0143275	-0.061035\\
-0.0148325	-0.08545\\
-0.01561	-0.094605\\
-0.0158375	-0.094605\\
-0.0160675	-0.0976575\\
-0.0161125	-0.0732425\\
-0.015885	-0.07019\\
-0.0157925	-0.0457775\\
-0.0149225	-0.0305175\\
-0.01442	-0.0457775\\
-0.014695	-0.0488275\\
-0.0149225	-0.05188\\
-0.015015	-0.0579825\\
-0.015105	-0.0396725\\
-0.01474	-0.042725\\
-0.0148775	-0.05188\\
-0.014785	-0.024415\\
-0.0142825	-0.01526\\
-0.0136425	-0.01526\\
-0.01323	-0.0213625\\
-0.0134125	-0.01526\\
-0.01323	-0.0030525\\
-0.0128175	-0.0122075\\
-0.0130925	-0.042725\\
-0.0141	-0.03357\\
-0.013915	-0.027465\\
-0.0136875	-0.0396725\\
-0.013915	-0.061035\\
-0.014695	-0.05188\\
-0.0146025	-0.01831\\
-0.01387	-0.01526\\
-0.0130925	-0.0213625\\
-0.0136875	-0.0671375\\
-0.01474	-0.079345\\
-0.0154275	-0.10376\\
-0.0160225	-0.1251225\\
-0.0166175	-0.1190175\\
-0.0168	-0.08545\\
-0.0163875	-0.08545\\
-0.016205	-0.1098625\\
-0.016525	-0.0915525\\
-0.01648	-0.0823975\\
-0.0163425	-0.094605\\
-0.01648	-0.10376\\
-0.01657	-0.10376\\
-0.0166625	-0.14038\\
-0.017165	-0.10376\\
-0.0168925	-0.0549325\\
-0.0161125	-0.03357\\
-0.01506	-0.024415\\
-0.01442	-0.0305175\\
-0.0143275	-0.0305175\\
-0.0143275	-0.027465\\
-0.0142825	-0.05188\\
-0.0148775	-0.0457775\\
-0.014785	-0.0488275\\
-0.014695	-0.05188\\
-0.015015	-0.03662\\
-0.0146025	-0.03357\\
-0.01451	-0.0549325\\
-0.0148325	-0.0579825\\
-0.01506	-0.03357\\
-0.0146475	-0.01831\\
-0.0141	-0.024415\\
-0.014145	-0.0305175\\
-0.0140525	-0.0732425\\
-0.0148775	-0.0885\\
-0.0157475	-0.0823975\\
-0.0158375	-0.0885\\
-0.0160675	-0.1098625\\
-0.0163425	-0.131225\\
-0.0169375	-0.1159675\\
-0.01703	-0.076295\\
-0.0164325	-0.0488275\\
-0.0157925	-0.03357\\
-0.0149225	-0.03357\\
-0.0145575	-0.03662\\
-0.0146025	-0.042725\\
-0.0148325	-0.024415\\
-0.014695	-0.0305175\\
-0.014465	-0.03357\\
-0.0145575	-0.0396725\\
-0.0146475	-0.0488275\\
-0.014785	-0.05188\\
-0.01506	-0.0457775\\
-0.0148775	-0.0213625\\
-0.0141	-0.0122075\\
-0.01355	-0.0122075\\
-0.0131825	-0.009155\\
-0.013	-0.01526\\
-0.0130925	-0.027465\\
-0.013595	-0.03662\\
-0.013825	-0.03357\\
-0.0140075	-0.0579825\\
-0.0146475	-0.0457775\\
-0.0145575	-0.0579825\\
-0.015015	-0.1007075\\
-0.01593	-0.076295\\
-0.015885	-0.0640875\\
-0.0157475	-0.0823975\\
-0.015885	-0.0915525\\
-0.0161125	-0.0671375\\
-0.0157925	-0.0549325\\
-0.0154725	-0.0396725\\
-0.01497	-0.0396725\\
-0.01474	-0.076295\\
-0.015565	-0.12207\\
-0.0166175	-0.1434325\\
-0.0172125	-0.1159675\\
-0.017075	-0.0976575\\
-0.0168925	-0.076295\\
-0.0163875	-0.076295\\
-0.0163425	-0.094605\\
-0.0166625	-0.07019\\
-0.0161125	-0.0488275\\
-0.015655	-0.0488275\\
-0.015335	-0.0640875\\
-0.0158375	-0.0732425\\
-0.0160675	-0.0396725\\
-0.015335	-0.03662\\
-0.01506	-0.03662\\
-0.014695	-0.061035\\
-0.01497	-0.07019\\
-0.015565	-0.0732425\\
-0.0157925	-0.05188\\
-0.01561	-0.042725\\
-0.0151975	-0.042725\\
-0.0151975	-0.0396725\\
-0.01506	-0.027465\\
-0.0146475	-0.024415\\
-0.0142375	-0.0213625\\
-0.013915	-0.01526\\
-0.0136425	-0.01831\\
-0.0136425	-0.0305175\\
-0.0141	-0.0457775\\
-0.0146475	-0.0457775\\
-0.01474	-0.03357\\
-0.014375	-0.024415\\
-0.0141	-0.0213625\\
-0.013915	-0.027465\\
-0.0140075	-0.0396725\\
-0.0143275	-0.0457775\\
-0.01451	-0.0396725\\
-0.01451	-0.027465\\
-0.0140525	-0.061035\\
-0.014465	-0.0671375\\
-0.0151975	-0.05188\\
-0.015015	-0.0213625\\
-0.0140525	-0.03357\\
-0.0137775	-0.042725\\
-0.0142375	-0.042725\\
-0.014375	-0.0305175\\
-0.0143275	-0.0213625\\
-0.013915	-0.0305175\\
-0.0142375	-0.0671375\\
-0.0148325	-0.03357\\
-0.014465	-0.01831\\
-0.013505	-0.024415\\
-0.0133675	-0.0457775\\
-0.0142825	-0.0549325\\
-0.0148775	-0.0305175\\
-0.01442	-0.027465\\
-0.014145	-0.05188\\
-0.014695	-0.0732425\\
-0.01529	-0.0640875\\
-0.015335	-0.094605\\
-0.015655	-0.1007075\\
-0.016205	-0.076295\\
-0.015885	-0.0579825\\
-0.0154725	-0.076295\\
-0.0158375	-0.0640875\\
-0.015565	-0.0671375\\
-0.0157025	-0.0915525\\
-0.0160675	-0.07019\\
-0.015885	-0.03662\\
-0.0149225	-0.0640875\\
-0.015105	-0.0915525\\
-0.015975	-0.10376\\
-0.01648	-0.079345\\
-0.01625	-0.07019\\
-0.0160225	-0.08545\\
-0.016205	-0.076295\\
-0.0161125	-0.042725\\
-0.0155175	-0.042725\\
-0.0152425	-0.0549325\\
-0.01538	-0.0579825\\
-0.0154725	-0.05188\\
-0.0154725	-0.061035\\
-0.015565	-0.0396725\\
-0.01538	-0.0488275\\
-0.01529	-0.07019\\
-0.015565	-0.0488275\\
-0.0151975	-0.03357\\
-0.014785	-0.027465\\
-0.01451	-0.0213625\\
-0.0141	-0.0213625\\
-0.0140525	-0.042725\\
-0.0146025	-0.0457775\\
-0.014695	-0.0488275\\
-0.01497	-0.0671375\\
-0.0154275	-0.079345\\
-0.0157475	-0.07019\\
-0.01561	-0.0549325\\
-0.01538	-0.03357\\
-0.0146025	-0.0396725\\
-0.01451	-0.0732425\\
-0.01529	-0.05188\\
-0.0151525	-0.01831\\
-0.01442	-0.042725\\
-0.01474	-0.0640875\\
-0.015335	-0.07019\\
-0.015565	-0.0885\\
-0.015975	-0.1159675\\
-0.01657	-0.0915525\\
-0.0163875	-0.0732425\\
-0.0161125	-0.08545\\
-0.016205	-0.0640875\\
-0.0158375	-0.0457775\\
-0.0154725	-0.0549325\\
-0.0154725	-0.0732425\\
-0.015655	-0.07019\\
-0.0157925	-0.07019\\
-0.0158375	-0.0457775\\
-0.015335	-0.03662\\
-0.01497	-0.0305175\\
-0.014695	-0.042725\\
-0.0148775	-0.0457775\\
-0.0148775	-0.03357\\
-0.014695	-0.061035\\
-0.0152425	-0.076295\\
-0.0155175	-0.0457775\\
-0.01529	-0.03357\\
-0.01497	-0.0579825\\
-0.0151975	-0.03662\\
-0.0148325	-0.01526\\
-0.014145	-0.0305175\\
-0.0140075	-0.03357\\
-0.0143275	-0.027465\\
-0.0142825	-0.01831\\
-0.0139625	-0.01526\\
-0.01332	-0.01831\\
-0.0134575	-0.027465\\
-0.0139625	-0.05188\\
-0.0145575	-0.0640875\\
-0.01497	-0.0732425\\
-0.01538	-0.0823975\\
-0.0157475	-0.0457775\\
-0.0151975	-0.0305175\\
-0.014695	-0.07019\\
-0.01538	-0.10376\\
-0.01616	-0.079345\\
-0.0161125	-0.0732425\\
-0.015975	-0.112915\\
-0.0164325	-0.08545\\
-0.01648	-0.076295\\
-0.016205	-0.0640875\\
-0.015885	-0.0579825\\
-0.0157025	-0.079345\\
-0.0160225	-0.061035\\
-0.0157025	-0.05188\\
-0.01538	-0.076295\\
-0.0158375	-0.1007075\\
-0.01625	-0.1007075\\
-0.0164325	-0.0396725\\
-0.0157925	-0.024415\\
-0.014695	-0.07019\\
-0.0154275	-0.05188\\
-0.0154275	-0.0213625\\
-0.01451	-0.0213625\\
-0.01387	-0.03662\\
-0.0141	-0.0488275\\
-0.0146025	-0.07019\\
-0.01529	-0.0457775\\
-0.0151975	-0.03357\\
-0.01474	-0.027465\\
-0.0142375	-0.01526\\
-0.013595	-0.0213625\\
-0.01355	-0.01526\\
-0.013505	-0.01831\\
-0.013595	-0.042725\\
-0.014145	-0.03662\\
-0.0143275	-0.01526\\
-0.0136875	-0.01526\\
-0.0133675	-0.0213625\\
-0.013505	-0.027465\\
-0.0137775	-0.0213625\\
-0.0133675	-0.024415\\
-0.013505	-0.024415\\
-0.01332	-0.0061025\\
-0.0130475	-0.0305175\\
-0.013595	-0.0457775\\
-0.0143275	-0.061035\\
-0.01506	-0.08545\\
-0.0157925	-0.076295\\
-0.01593	-0.03662\\
-0.015105	-0.0305175\\
-0.01451	-0.0305175\\
-0.0143275	-0.009155\\
-0.013915	-0.0122075\\
-0.0134575	-0.03357\\
-0.0139625	-0.0457775\\
-0.0142375	-0.03662\\
-0.01419	-0.03662\\
-0.0142825	-0.0457775\\
-0.014465	-0.0488275\\
-0.0146025	-0.0488275\\
-0.014695	-0.0640875\\
-0.015015	-0.05188\\
-0.0151525	-0.08545\\
-0.015655	-0.0457775\\
-0.01506	-0.0305175\\
-0.0140525	-0.0213625\\
-0.0137325	-0.0396725\\
-0.01419	-0.03357\\
-0.0142375	-0.024415\\
-0.0140525	-0.009155\\
-0.0134125	-0.027465\\
-0.01332	-0.0213625\\
-0.0131825	-0.0061025\\
-0.012635	-0.01831\\
-0.012635	-0.024415\\
-0.01323	-0.03662\\
-0.0136875	-0.0488275\\
-0.0142375	-0.0671375\\
-0.01497	-0.0549325\\
-0.0148775	-0.01831\\
-0.013915	-0.03357\\
-0.0140525	-0.0549325\\
-0.01442	-0.027465\\
-0.0137325	-0.0213625\\
-0.0136875	-0.0549325\\
-0.014465	-0.0457775\\
-0.0146475	-0.07019\\
-0.01529	-0.0915525\\
-0.0158375	-0.0579825\\
-0.0155175	-0.0457775\\
-0.015105	-0.0396725\\
-0.0149225	-0.061035\\
-0.015105	-0.079345\\
-0.015565	-0.0579825\\
-0.0154275	-0.03357\\
-0.0148325	-0.0457775\\
};
\addplot [color=mycolor2, line width=2.0pt, forget plot]
  table[row sep=crcr]{%
-0.015655	-0.015655\\
-0.0157025	-0.0157025\\
-0.0158375	-0.0158375\\
-0.0157025	-0.0157025\\
-0.0149225	-0.0149225\\
-0.0148325	-0.0148325\\
-0.01497	-0.01497\\
-0.0149225	-0.0149225\\
-0.0140525	-0.0140525\\
-0.01323	-0.01323\\
-0.0128175	-0.0128175\\
-0.0137775	-0.0137775\\
-0.014785	-0.014785\\
-0.01506	-0.01506\\
-0.0151975	-0.0151975\\
-0.0149225	-0.0149225\\
-0.0143275	-0.0143275\\
-0.01497	-0.01497\\
-0.0148325	-0.0148325\\
-0.014375	-0.014375\\
-0.01474	-0.01474\\
-0.0148325	-0.0148325\\
-0.0146475	-0.0146475\\
-0.015105	-0.015105\\
-0.0151975	-0.0151975\\
-0.01497	-0.01497\\
-0.0152425	-0.0152425\\
-0.01538	-0.01538\\
-0.0151975	-0.0151975\\
-0.01497	-0.01497\\
-0.0146475	-0.0146475\\
-0.01474	-0.01474\\
-0.0149225	-0.0149225\\
-0.015015	-0.015015\\
-0.01506	-0.01506\\
-0.0155175	-0.0155175\\
-0.0151525	-0.0151525\\
-0.01497	-0.01497\\
-0.014465	-0.014465\\
-0.014145	-0.014145\\
-0.01442	-0.01442\\
-0.0142375	-0.0142375\\
-0.01419	-0.01419\\
-0.014465	-0.014465\\
-0.014695	-0.014695\\
-0.0151525	-0.0151525\\
-0.01506	-0.01506\\
-0.01451	-0.01451\\
-0.014145	-0.014145\\
-0.0143275	-0.0143275\\
-0.014465	-0.014465\\
-0.0139625	-0.0139625\\
-0.0137775	-0.0137775\\
-0.0140075	-0.0140075\\
-0.013915	-0.013915\\
-0.0136875	-0.0136875\\
-0.01387	-0.01387\\
-0.0140075	-0.0140075\\
-0.0146025	-0.0146025\\
-0.01474	-0.01474\\
-0.0149225	-0.0149225\\
-0.01497	-0.01497\\
-0.01506	-0.01506\\
-0.01497	-0.01497\\
-0.0148775	-0.0148775\\
-0.0148325	-0.0148325\\
-0.014695	-0.014695\\
-0.01474	-0.01474\\
-0.0152425	-0.0152425\\
-0.0160225	-0.0160225\\
-0.0162975	-0.0162975\\
-0.0157925	-0.0157925\\
-0.0160675	-0.0160675\\
-0.016525	-0.016525\\
-0.0166625	-0.0166625\\
-0.0164325	-0.0164325\\
-0.0166175	-0.0166175\\
-0.017165	-0.017165\\
-0.0169825	-0.0169825\\
-0.0166175	-0.0166175\\
-0.016205	-0.016205\\
-0.0157025	-0.0157025\\
-0.01538	-0.01538\\
-0.0154275	-0.0154275\\
-0.0151525	-0.0151525\\
-0.0145575	-0.0145575\\
-0.0142375	-0.0142375\\
-0.0143275	-0.0143275\\
-0.014695	-0.014695\\
-0.01474	-0.01474\\
-0.015015	-0.015015\\
-0.0151525	-0.0151525\\
-0.01561	-0.01561\\
-0.0157025	-0.0157025\\
-0.0155175	-0.0155175\\
-0.0152425	-0.0152425\\
-0.01538	-0.01538\\
-0.015105	-0.015105\\
-0.0149225	-0.0149225\\
-0.015015	-0.015015\\
-0.01506	-0.01506\\
-0.0148325	-0.0148325\\
-0.01474	-0.01474\\
-0.014465	-0.014465\\
-0.014375	-0.014375\\
-0.014465	-0.014465\\
-0.01419	-0.01419\\
-0.014375	-0.014375\\
-0.0149225	-0.0149225\\
-0.0151525	-0.0151525\\
-0.01529	-0.01529\\
-0.01474	-0.01474\\
-0.0148775	-0.0148775\\
-0.01561	-0.01561\\
-0.0154725	-0.0154725\\
-0.0155175	-0.0155175\\
-0.01561	-0.01561\\
-0.0151975	-0.0151975\\
-0.0146025	-0.0146025\\
-0.01474	-0.01474\\
-0.0149225	-0.0149225\\
-0.0154275	-0.0154275\\
-0.015975	-0.015975\\
-0.0161125	-0.0161125\\
-0.0157925	-0.0157925\\
-0.0157475	-0.0157475\\
-0.015655	-0.015655\\
-0.01561	-0.01561\\
-0.0151975	-0.0151975\\
-0.0148775	-0.0148775\\
-0.014785	-0.014785\\
-0.0146025	-0.0146025\\
-0.0146475	-0.0146475\\
-0.01497	-0.01497\\
-0.015655	-0.015655\\
-0.015565	-0.015565\\
-0.015105	-0.015105\\
-0.0148325	-0.0148325\\
-0.01474	-0.01474\\
-0.0142825	-0.0142825\\
-0.0137325	-0.0137325\\
-0.0136425	-0.0136425\\
-0.014145	-0.014145\\
-0.0141	-0.0141\\
-0.0140525	-0.0140525\\
-0.014145	-0.014145\\
-0.0141	-0.0141\\
-0.01387	-0.01387\\
-0.013505	-0.013505\\
-0.0130925	-0.0130925\\
-0.013275	-0.013275\\
-0.0142375	-0.0142375\\
-0.015015	-0.015015\\
-0.01538	-0.01538\\
-0.01497	-0.01497\\
-0.01451	-0.01451\\
-0.0141	-0.0141\\
-0.01355	-0.01355\\
-0.0140525	-0.0140525\\
-0.0142375	-0.0142375\\
-0.0146025	-0.0146025\\
-0.01506	-0.01506\\
-0.015885	-0.015885\\
-0.01657	-0.01657\\
-0.01648	-0.01648\\
-0.0163875	-0.0163875\\
-0.015975	-0.015975\\
-0.015885	-0.015885\\
-0.015975	-0.015975\\
-0.0160225	-0.0160225\\
-0.0158375	-0.0158375\\
-0.01561	-0.01561\\
-0.0154275	-0.0154275\\
-0.0155175	-0.0155175\\
-0.01529	-0.01529\\
-0.0157925	-0.0157925\\
-0.01616	-0.01616\\
-0.015885	-0.015885\\
-0.0152425	-0.0152425\\
-0.015105	-0.015105\\
-0.01561	-0.01561\\
-0.0160225	-0.0160225\\
-0.0164325	-0.0164325\\
-0.0166175	-0.0166175\\
-0.0167075	-0.0167075\\
-0.01648	-0.01648\\
-0.0162975	-0.0162975\\
-0.0163875	-0.0163875\\
-0.01657	-0.01657\\
-0.01616	-0.01616\\
-0.0154275	-0.0154275\\
-0.01561	-0.01561\\
-0.0154725	-0.0154725\\
-0.0149225	-0.0149225\\
-0.01506	-0.01506\\
-0.0151975	-0.0151975\\
-0.015015	-0.015015\\
-0.014695	-0.014695\\
-0.014785	-0.014785\\
-0.014695	-0.014695\\
-0.0148325	-0.0148325\\
-0.015335	-0.015335\\
-0.01529	-0.01529\\
-0.01497	-0.01497\\
-0.0148775	-0.0148775\\
-0.015335	-0.015335\\
-0.0160675	-0.0160675\\
-0.015975	-0.015975\\
-0.0154725	-0.0154725\\
-0.015015	-0.015015\\
-0.0149225	-0.0149225\\
-0.0148775	-0.0148775\\
-0.0145575	-0.0145575\\
-0.01451	-0.01451\\
-0.014695	-0.014695\\
-0.0149225	-0.0149225\\
-0.01506	-0.01506\\
-0.01474	-0.01474\\
-0.0151525	-0.0151525\\
-0.0155175	-0.0155175\\
-0.01529	-0.01529\\
-0.015335	-0.015335\\
-0.015655	-0.015655\\
-0.01529	-0.01529\\
-0.014375	-0.014375\\
-0.0142825	-0.0142825\\
-0.01442	-0.01442\\
-0.014695	-0.014695\\
-0.014785	-0.014785\\
-0.014465	-0.014465\\
-0.01474	-0.01474\\
-0.0148325	-0.0148325\\
-0.0145575	-0.0145575\\
-0.0146475	-0.0146475\\
-0.0145575	-0.0145575\\
-0.014465	-0.014465\\
-0.0146025	-0.0146025\\
-0.0141	-0.0141\\
-0.0139625	-0.0139625\\
-0.0137775	-0.0137775\\
-0.0137325	-0.0137325\\
-0.0143275	-0.0143275\\
-0.0146475	-0.0146475\\
-0.015105	-0.015105\\
-0.01561	-0.01561\\
-0.015335	-0.015335\\
-0.014695	-0.014695\\
-0.0142825	-0.0142825\\
-0.0141	-0.0141\\
-0.0143275	-0.0143275\\
-0.0140525	-0.0140525\\
-0.0139625	-0.0139625\\
-0.01419	-0.01419\\
-0.014785	-0.014785\\
-0.01497	-0.01497\\
-0.015335	-0.015335\\
-0.01506	-0.01506\\
-0.0148775	-0.0148775\\
-0.0142375	-0.0142375\\
-0.013915	-0.013915\\
-0.013505	-0.013505\\
-0.0133675	-0.0133675\\
-0.0134125	-0.0134125\\
-0.0140075	-0.0140075\\
-0.0142825	-0.0142825\\
-0.014465	-0.014465\\
-0.0145575	-0.0145575\\
-0.0152425	-0.0152425\\
-0.0154275	-0.0154275\\
-0.01506	-0.01506\\
-0.0151975	-0.0151975\\
-0.01529	-0.01529\\
-0.015565	-0.015565\\
-0.0154275	-0.0154275\\
-0.015015	-0.015015\\
-0.01419	-0.01419\\
-0.01355	-0.01355\\
-0.0133675	-0.0133675\\
-0.0134575	-0.0134575\\
-0.013505	-0.013505\\
-0.0134575	-0.0134575\\
-0.0137325	-0.0137325\\
-0.014145	-0.014145\\
-0.0142375	-0.0142375\\
-0.0143275	-0.0143275\\
-0.014465	-0.014465\\
-0.0140525	-0.0140525\\
-0.0134575	-0.0134575\\
-0.013825	-0.013825\\
-0.014465	-0.014465\\
-0.0145575	-0.0145575\\
-0.014695	-0.014695\\
-0.014375	-0.014375\\
-0.0145575	-0.0145575\\
-0.015015	-0.015015\\
-0.0151975	-0.0151975\\
-0.0148775	-0.0148775\\
-0.01538	-0.01538\\
-0.015335	-0.015335\\
-0.0151525	-0.0151525\\
-0.015335	-0.015335\\
-0.0151525	-0.0151525\\
-0.0149225	-0.0149225\\
-0.01529	-0.01529\\
-0.0160225	-0.0160225\\
-0.01593	-0.01593\\
-0.0152425	-0.0152425\\
-0.015015	-0.015015\\
-0.01497	-0.01497\\
-0.0151975	-0.0151975\\
-0.0154275	-0.0154275\\
-0.0151525	-0.0151525\\
-0.0154275	-0.0154275\\
-0.01561	-0.01561\\
-0.0152425	-0.0152425\\
-0.0149225	-0.0149225\\
-0.01506	-0.01506\\
-0.015655	-0.015655\\
-0.0158375	-0.0158375\\
-0.0157475	-0.0157475\\
-0.01529	-0.01529\\
-0.01497	-0.01497\\
-0.0149225	-0.0149225\\
-0.01451	-0.01451\\
-0.01387	-0.01387\\
-0.0133675	-0.0133675\\
-0.013	-0.013\\
-0.013595	-0.013595\\
-0.0142375	-0.0142375\\
-0.0145575	-0.0145575\\
-0.01442	-0.01442\\
-0.014375	-0.014375\\
-0.0145575	-0.0145575\\
-0.014145	-0.014145\\
-0.0140525	-0.0140525\\
-0.013825	-0.013825\\
-0.01387	-0.01387\\
-0.01442	-0.01442\\
-0.0148775	-0.0148775\\
-0.01451	-0.01451\\
-0.014145	-0.014145\\
-0.01387	-0.01387\\
-0.0139625	-0.0139625\\
-0.0136875	-0.0136875\\
-0.0142825	-0.0142825\\
-0.0152425	-0.0152425\\
-0.0151525	-0.0151525\\
-0.0154275	-0.0154275\\
-0.0158375	-0.0158375\\
-0.01593	-0.01593\\
-0.0157475	-0.0157475\\
-0.01529	-0.01529\\
-0.0151975	-0.0151975\\
-0.015335	-0.015335\\
-0.0155175	-0.0155175\\
-0.015565	-0.015565\\
-0.01593	-0.01593\\
-0.0162975	-0.0162975\\
-0.0166175	-0.0166175\\
-0.0163425	-0.0163425\\
-0.01625	-0.01625\\
-0.01648	-0.01648\\
-0.016205	-0.016205\\
-0.0162975	-0.0162975\\
-0.016205	-0.016205\\
-0.015565	-0.015565\\
-0.0148325	-0.0148325\\
-0.0146025	-0.0146025\\
-0.0149225	-0.0149225\\
-0.014785	-0.014785\\
-0.014375	-0.014375\\
-0.014465	-0.014465\\
-0.0143275	-0.0143275\\
-0.013915	-0.013915\\
-0.0139625	-0.0139625\\
-0.0140525	-0.0140525\\
-0.01387	-0.01387\\
-0.01419	-0.01419\\
-0.0146475	-0.0146475\\
-0.0145575	-0.0145575\\
-0.015015	-0.015015\\
-0.0157925	-0.0157925\\
-0.01593	-0.01593\\
-0.01625	-0.01625\\
-0.015975	-0.015975\\
-0.01593	-0.01593\\
-0.0155175	-0.0155175\\
-0.0149225	-0.0149225\\
-0.0148775	-0.0148775\\
-0.0146475	-0.0146475\\
-0.0142825	-0.0142825\\
-0.014465	-0.014465\\
-0.0140075	-0.0140075\\
-0.0134125	-0.0134125\\
-0.01332	-0.01332\\
-0.0140075	-0.0140075\\
-0.0146025	-0.0146025\\
-0.0149225	-0.0149225\\
-0.01506	-0.01506\\
-0.015015	-0.015015\\
-0.0151525	-0.0151525\\
-0.015105	-0.015105\\
-0.0148325	-0.0148325\\
-0.0146475	-0.0146475\\
-0.01497	-0.01497\\
-0.01474	-0.01474\\
-0.01442	-0.01442\\
-0.01474	-0.01474\\
-0.0151975	-0.0151975\\
-0.01497	-0.01497\\
-0.014695	-0.014695\\
-0.01451	-0.01451\\
-0.0145575	-0.0145575\\
-0.0142825	-0.0142825\\
-0.0140075	-0.0140075\\
-0.0142825	-0.0142825\\
-0.01451	-0.01451\\
-0.014695	-0.014695\\
-0.0146025	-0.0146025\\
-0.014695	-0.014695\\
-0.014785	-0.014785\\
-0.0142825	-0.0142825\\
-0.014145	-0.014145\\
-0.01474	-0.01474\\
-0.01538	-0.01538\\
-0.0154725	-0.0154725\\
-0.01538	-0.01538\\
-0.0152425	-0.0152425\\
-0.01506	-0.01506\\
-0.0148775	-0.0148775\\
-0.01474	-0.01474\\
-0.0149225	-0.0149225\\
-0.0148325	-0.0148325\\
-0.01442	-0.01442\\
-0.0146025	-0.0146025\\
-0.0148325	-0.0148325\\
-0.015015	-0.015015\\
-0.0151975	-0.0151975\\
-0.01561	-0.01561\\
-0.0160675	-0.0160675\\
-0.01648	-0.01648\\
-0.0160675	-0.0160675\\
-0.015335	-0.015335\\
-0.014695	-0.014695\\
-0.0140525	-0.0140525\\
-0.0141	-0.0141\\
-0.0146025	-0.0146025\\
-0.014695	-0.014695\\
-0.0145575	-0.0145575\\
-0.01474	-0.01474\\
-0.01497	-0.01497\\
-0.0151975	-0.0151975\\
-0.01538	-0.01538\\
-0.01593	-0.01593\\
-0.0161125	-0.0161125\\
-0.0157925	-0.0157925\\
-0.01529	-0.01529\\
-0.0151525	-0.0151525\\
-0.01538	-0.01538\\
-0.015105	-0.015105\\
-0.015015	-0.015015\\
-0.01497	-0.01497\\
-0.01442	-0.01442\\
-0.0146025	-0.0146025\\
-0.015105	-0.015105\\
-0.0149225	-0.0149225\\
-0.0148775	-0.0148775\\
-0.01561	-0.01561\\
-0.015565	-0.015565\\
-0.0154275	-0.0154275\\
-0.015885	-0.015885\\
-0.01593	-0.01593\\
-0.0154275	-0.0154275\\
-0.01474	-0.01474\\
-0.0146475	-0.0146475\\
-0.01506	-0.01506\\
-0.0158375	-0.0158375\\
-0.0161125	-0.0161125\\
-0.016205	-0.016205\\
-0.015975	-0.015975\\
-0.0154275	-0.0154275\\
-0.015105	-0.015105\\
-0.014695	-0.014695\\
-0.01442	-0.01442\\
-0.0143275	-0.0143275\\
-0.014465	-0.014465\\
-0.0146475	-0.0146475\\
-0.0142375	-0.0142375\\
-0.0143275	-0.0143275\\
-0.014785	-0.014785\\
-0.015015	-0.015015\\
-0.015335	-0.015335\\
-0.0157925	-0.0157925\\
-0.0162975	-0.0162975\\
-0.01616	-0.01616\\
-0.015885	-0.015885\\
-0.0157025	-0.0157025\\
-0.01529	-0.01529\\
-0.0152425	-0.0152425\\
-0.0157925	-0.0157925\\
-0.01616	-0.01616\\
-0.015655	-0.015655\\
-0.0148775	-0.0148775\\
-0.0142375	-0.0142375\\
-0.013505	-0.013505\\
-0.0131375	-0.0131375\\
-0.01355	-0.01355\\
-0.013825	-0.013825\\
-0.01419	-0.01419\\
-0.013825	-0.013825\\
-0.0136875	-0.0136875\\
-0.0136425	-0.0136425\\
-0.0137325	-0.0137325\\
-0.0141	-0.0141\\
-0.014785	-0.014785\\
-0.015105	-0.015105\\
-0.015335	-0.015335\\
-0.0157025	-0.0157025\\
-0.01593	-0.01593\\
-0.01616	-0.01616\\
-0.01625	-0.01625\\
-0.015885	-0.015885\\
-0.0154275	-0.0154275\\
-0.0151975	-0.0151975\\
-0.015655	-0.015655\\
-0.015975	-0.015975\\
-0.0157025	-0.0157025\\
-0.0151975	-0.0151975\\
-0.0142825	-0.0142825\\
-0.0136425	-0.0136425\\
-0.013505	-0.013505\\
-0.013	-0.013\\
-0.0133675	-0.0133675\\
-0.01387	-0.01387\\
-0.013915	-0.013915\\
-0.013825	-0.013825\\
-0.0134125	-0.0134125\\
-0.0130925	-0.0130925\\
-0.01323	-0.01323\\
-0.01332	-0.01332\\
-0.0134575	-0.0134575\\
-0.01323	-0.01323\\
-0.0130475	-0.0130475\\
-0.0134575	-0.0134575\\
-0.0146475	-0.0146475\\
-0.015335	-0.015335\\
-0.01561	-0.01561\\
-0.01538	-0.01538\\
-0.01561	-0.01561\\
-0.0163425	-0.0163425\\
-0.0164325	-0.0164325\\
-0.01648	-0.01648\\
-0.01616	-0.01616\\
-0.0160675	-0.0160675\\
-0.01616	-0.01616\\
-0.0157925	-0.0157925\\
-0.0155175	-0.0155175\\
-0.0148775	-0.0148775\\
-0.014465	-0.014465\\
-0.01506	-0.01506\\
-0.0148775	-0.0148775\\
-0.014785	-0.014785\\
-0.01497	-0.01497\\
-0.0148775	-0.0148775\\
-0.01497	-0.01497\\
-0.01506	-0.01506\\
-0.0151525	-0.0151525\\
-0.015105	-0.015105\\
-0.014785	-0.014785\\
-0.0149225	-0.0149225\\
-0.0143275	-0.0143275\\
-0.01332	-0.01332\\
-0.0131825	-0.0131825\\
-0.0136425	-0.0136425\\
-0.013275	-0.013275\\
-0.012405	-0.012405\\
-0.01245	-0.01245\\
-0.0131375	-0.0131375\\
-0.013595	-0.013595\\
-0.013915	-0.013915\\
-0.014465	-0.014465\\
-0.0146025	-0.0146025\\
-0.0142375	-0.0142375\\
-0.014465	-0.014465\\
-0.0140525	-0.0140525\\
-0.013595	-0.013595\\
-0.01323	-0.01323\\
-0.01268	-0.01268\\
-0.0128625	-0.0128625\\
-0.0131825	-0.0131825\\
-0.013915	-0.013915\\
-0.0141	-0.0141\\
-0.01387	-0.01387\\
-0.0137325	-0.0137325\\
-0.013915	-0.013915\\
-0.0137325	-0.0137325\\
-0.01355	-0.01355\\
-0.0134575	-0.0134575\\
-0.0131825	-0.0131825\\
-0.01332	-0.01332\\
-0.013275	-0.013275\\
-0.0136425	-0.0136425\\
-0.0137775	-0.0137775\\
-0.013595	-0.013595\\
-0.01355	-0.01355\\
-0.01332	-0.01332\\
-0.0141	-0.0141\\
-0.0148325	-0.0148325\\
-0.0146475	-0.0146475\\
-0.01451	-0.01451\\
-0.014145	-0.014145\\
-0.01451	-0.01451\\
-0.014695	-0.014695\\
-0.0142375	-0.0142375\\
-0.0143275	-0.0143275\\
-0.0140525	-0.0140525\\
-0.01419	-0.01419\\
-0.01387	-0.01387\\
-0.0136875	-0.0136875\\
-0.013915	-0.013915\\
-0.014145	-0.014145\\
-0.0139625	-0.0139625\\
-0.013915	-0.013915\\
-0.0140075	-0.0140075\\
-0.014465	-0.014465\\
-0.01451	-0.01451\\
-0.01497	-0.01497\\
-0.015565	-0.015565\\
-0.0155175	-0.0155175\\
-0.015105	-0.015105\\
-0.0146475	-0.0146475\\
-0.0148775	-0.0148775\\
-0.01529	-0.01529\\
-0.015105	-0.015105\\
-0.0151525	-0.0151525\\
-0.014785	-0.014785\\
-0.0139625	-0.0139625\\
-0.0136875	-0.0136875\\
-0.014375	-0.014375\\
-0.0146025	-0.0146025\\
-0.01442	-0.01442\\
-0.0148325	-0.0148325\\
-0.0149225	-0.0149225\\
-0.0148325	-0.0148325\\
-0.01451	-0.01451\\
-0.0145575	-0.0145575\\
-0.0149225	-0.0149225\\
-0.0148325	-0.0148325\\
-0.014785	-0.014785\\
-0.014695	-0.014695\\
-0.01442	-0.01442\\
-0.01419	-0.01419\\
-0.014145	-0.014145\\
-0.0146475	-0.0146475\\
-0.01497	-0.01497\\
-0.01506	-0.01506\\
-0.015015	-0.015015\\
-0.014695	-0.014695\\
-0.0142375	-0.0142375\\
-0.01419	-0.01419\\
-0.01442	-0.01442\\
-0.014695	-0.014695\\
-0.01497	-0.01497\\
-0.01529	-0.01529\\
-0.0151975	-0.0151975\\
-0.0146025	-0.0146025\\
-0.01497	-0.01497\\
-0.015565	-0.015565\\
-0.01538	-0.01538\\
-0.0152425	-0.0152425\\
-0.0154275	-0.0154275\\
-0.0152425	-0.0152425\\
-0.0149225	-0.0149225\\
-0.015015	-0.015015\\
-0.01497	-0.01497\\
-0.0152425	-0.0152425\\
-0.01506	-0.01506\\
-0.015105	-0.015105\\
-0.0148775	-0.0148775\\
-0.01497	-0.01497\\
-0.01474	-0.01474\\
-0.0146475	-0.0146475\\
-0.01474	-0.01474\\
-0.014785	-0.014785\\
-0.01419	-0.01419\\
-0.0134125	-0.0134125\\
-0.012955	-0.012955\\
-0.0130475	-0.0130475\\
-0.0141	-0.0141\\
-0.0148775	-0.0148775\\
-0.01497	-0.01497\\
-0.0146475	-0.0146475\\
-0.014465	-0.014465\\
-0.0140075	-0.0140075\\
-0.0141	-0.0141\\
-0.014145	-0.014145\\
-0.0142825	-0.0142825\\
-0.01419	-0.01419\\
-0.0141	-0.0141\\
-0.0136875	-0.0136875\\
-0.0137325	-0.0137325\\
-0.014465	-0.014465\\
-0.014375	-0.014375\\
-0.014465	-0.014465\\
-0.014695	-0.014695\\
-0.0148325	-0.0148325\\
-0.0151525	-0.0151525\\
-0.015015	-0.015015\\
-0.01497	-0.01497\\
-0.015105	-0.015105\\
-0.01451	-0.01451\\
-0.014785	-0.014785\\
-0.0146025	-0.0146025\\
-0.0146475	-0.0146475\\
-0.014785	-0.014785\\
-0.015015	-0.015015\\
-0.0149225	-0.0149225\\
-0.01506	-0.01506\\
-0.0154275	-0.0154275\\
-0.01529	-0.01529\\
-0.0151525	-0.0151525\\
-0.0149225	-0.0149225\\
-0.0148325	-0.0148325\\
-0.014695	-0.014695\\
-0.01451	-0.01451\\
-0.014465	-0.014465\\
-0.01506	-0.01506\\
-0.01561	-0.01561\\
-0.015565	-0.015565\\
-0.0155175	-0.0155175\\
-0.01497	-0.01497\\
-0.0146475	-0.0146475\\
-0.01442	-0.01442\\
-0.014145	-0.014145\\
-0.0145575	-0.0145575\\
-0.0149225	-0.0149225\\
-0.0151525	-0.0151525\\
-0.01506	-0.01506\\
-0.0146475	-0.0146475\\
-0.01506	-0.01506\\
-0.0154725	-0.0154725\\
-0.0157025	-0.0157025\\
-0.015565	-0.015565\\
-0.0161125	-0.0161125\\
-0.0160675	-0.0160675\\
-0.015335	-0.015335\\
-0.014785	-0.014785\\
-0.01442	-0.01442\\
-0.01474	-0.01474\\
-0.0151525	-0.0151525\\
-0.01561	-0.01561\\
-0.0155175	-0.0155175\\
-0.0151975	-0.0151975\\
-0.0146025	-0.0146025\\
-0.0142375	-0.0142375\\
-0.0143275	-0.0143275\\
-0.014375	-0.014375\\
-0.0140075	-0.0140075\\
-0.0139625	-0.0139625\\
-0.013915	-0.013915\\
-0.0133675	-0.0133675\\
-0.0137775	-0.0137775\\
-0.0141	-0.0141\\
-0.01442	-0.01442\\
-0.01451	-0.01451\\
-0.0145575	-0.0145575\\
-0.014785	-0.014785\\
-0.0146025	-0.0146025\\
-0.014785	-0.014785\\
-0.0151525	-0.0151525\\
-0.015015	-0.015015\\
-0.01497	-0.01497\\
-0.0148775	-0.0148775\\
-0.0149225	-0.0149225\\
-0.015335	-0.015335\\
-0.0151975	-0.0151975\\
-0.0149225	-0.0149225\\
-0.01497	-0.01497\\
-0.0148775	-0.0148775\\
-0.0145575	-0.0145575\\
-0.01474	-0.01474\\
-0.01529	-0.01529\\
-0.01538	-0.01538\\
-0.0154275	-0.0154275\\
-0.0161125	-0.0161125\\
-0.01593	-0.01593\\
-0.015565	-0.015565\\
-0.01529	-0.01529\\
-0.015335	-0.015335\\
-0.0158375	-0.0158375\\
-0.015565	-0.015565\\
-0.0152425	-0.0152425\\
-0.015565	-0.015565\\
-0.0152425	-0.0152425\\
-0.0146025	-0.0146025\\
-0.01451	-0.01451\\
-0.014375	-0.014375\\
-0.0140075	-0.0140075\\
-0.0139625	-0.0139625\\
-0.013825	-0.013825\\
-0.0136875	-0.0136875\\
-0.0137775	-0.0137775\\
-0.01419	-0.01419\\
-0.01442	-0.01442\\
-0.0148325	-0.0148325\\
-0.0149225	-0.0149225\\
-0.0151525	-0.0151525\\
-0.0154275	-0.0154275\\
-0.015105	-0.015105\\
-0.01506	-0.01506\\
-0.015015	-0.015015\\
-0.01506	-0.01506\\
-0.01538	-0.01538\\
-0.01529	-0.01529\\
-0.0152425	-0.0152425\\
-0.0151975	-0.0151975\\
-0.015105	-0.015105\\
-0.0154725	-0.0154725\\
-0.0154275	-0.0154275\\
-0.015335	-0.015335\\
-0.0148325	-0.0148325\\
-0.0142825	-0.0142825\\
-0.01387	-0.01387\\
-0.01419	-0.01419\\
-0.0143275	-0.0143275\\
-0.014465	-0.014465\\
-0.014375	-0.014375\\
-0.0145575	-0.0145575\\
-0.01529	-0.01529\\
-0.0157925	-0.0157925\\
-0.0157025	-0.0157025\\
-0.015655	-0.015655\\
-0.01561	-0.01561\\
-0.0152425	-0.0152425\\
-0.0151975	-0.0151975\\
-0.0152425	-0.0152425\\
-0.0151525	-0.0151525\\
-0.0151975	-0.0151975\\
-0.0154725	-0.0154725\\
-0.015975	-0.015975\\
-0.0160675	-0.0160675\\
-0.015975	-0.015975\\
-0.0161125	-0.0161125\\
-0.0158375	-0.0158375\\
-0.0157475	-0.0157475\\
-0.0154725	-0.0154725\\
-0.01538	-0.01538\\
-0.01561	-0.01561\\
-0.0158375	-0.0158375\\
-0.01506	-0.01506\\
-0.01474	-0.01474\\
-0.014695	-0.014695\\
-0.0148775	-0.0148775\\
-0.015015	-0.015015\\
-0.0146025	-0.0146025\\
-0.0146475	-0.0146475\\
-0.0151975	-0.0151975\\
-0.0161125	-0.0161125\\
-0.01648	-0.01648\\
-0.0163875	-0.0163875\\
-0.01616	-0.01616\\
-0.01625	-0.01625\\
-0.016525	-0.016525\\
-0.0162975	-0.0162975\\
-0.016205	-0.016205\\
-0.0162975	-0.0162975\\
-0.01593	-0.01593\\
-0.01561	-0.01561\\
-0.0157925	-0.0157925\\
-0.0154275	-0.0154275\\
-0.01497	-0.01497\\
-0.01506	-0.01506\\
-0.0148775	-0.0148775\\
-0.0148325	-0.0148325\\
-0.01497	-0.01497\\
-0.01529	-0.01529\\
-0.0152425	-0.0152425\\
-0.0149225	-0.0149225\\
-0.01497	-0.01497\\
-0.015105	-0.015105\\
-0.01529	-0.01529\\
-0.0152425	-0.0152425\\
-0.015015	-0.015015\\
-0.0154275	-0.0154275\\
-0.0154725	-0.0154725\\
-0.015105	-0.015105\\
-0.0154275	-0.0154275\\
-0.0155175	-0.0155175\\
-0.0152425	-0.0152425\\
-0.014785	-0.014785\\
-0.0143275	-0.0143275\\
-0.01387	-0.01387\\
-0.0143275	-0.0143275\\
-0.01451	-0.01451\\
-0.0142375	-0.0142375\\
-0.013915	-0.013915\\
-0.013825	-0.013825\\
-0.0143275	-0.0143275\\
-0.0146475	-0.0146475\\
-0.0149225	-0.0149225\\
-0.01529	-0.01529\\
-0.015335	-0.015335\\
-0.0154275	-0.0154275\\
-0.01506	-0.01506\\
-0.0140525	-0.0140525\\
-0.0140075	-0.0140075\\
-0.0139625	-0.0139625\\
-0.0140525	-0.0140525\\
-0.014695	-0.014695\\
-0.0149225	-0.0149225\\
-0.0157025	-0.0157025\\
-0.015565	-0.015565\\
-0.01529	-0.01529\\
-0.015655	-0.015655\\
-0.0155175	-0.0155175\\
-0.015015	-0.015015\\
-0.0145575	-0.0145575\\
-0.0146475	-0.0146475\\
-0.01497	-0.01497\\
-0.0154725	-0.0154725\\
-0.01529	-0.01529\\
-0.01497	-0.01497\\
-0.0148775	-0.0148775\\
-0.01497	-0.01497\\
-0.01529	-0.01529\\
-0.015975	-0.015975\\
-0.0160675	-0.0160675\\
-0.0160225	-0.0160225\\
-0.015565	-0.015565\\
-0.0149225	-0.0149225\\
-0.0148775	-0.0148775\\
-0.014145	-0.014145\\
-0.01419	-0.01419\\
-0.0142375	-0.0142375\\
-0.014785	-0.014785\\
-0.0154275	-0.0154275\\
-0.0157475	-0.0157475\\
-0.016205	-0.016205\\
-0.0162975	-0.0162975\\
-0.015565	-0.015565\\
-0.0151975	-0.0151975\\
-0.015105	-0.015105\\
-0.0152425	-0.0152425\\
-0.0155175	-0.0155175\\
-0.015975	-0.015975\\
-0.0157925	-0.0157925\\
-0.01538	-0.01538\\
-0.015655	-0.015655\\
-0.01538	-0.01538\\
-0.01497	-0.01497\\
-0.0146025	-0.0146025\\
-0.014145	-0.014145\\
-0.013825	-0.013825\\
-0.013505	-0.013505\\
-0.0137325	-0.0137325\\
-0.0142375	-0.0142375\\
-0.014465	-0.014465\\
-0.0142825	-0.0142825\\
-0.014145	-0.014145\\
-0.0134575	-0.0134575\\
-0.012635	-0.012635\\
-0.0124975	-0.0124975\\
-0.0130475	-0.0130475\\
-0.013595	-0.013595\\
-0.0140525	-0.0140525\\
-0.01442	-0.01442\\
-0.014695	-0.014695\\
-0.0151975	-0.0151975\\
-0.01593	-0.01593\\
-0.016205	-0.016205\\
-0.0163425	-0.0163425\\
-0.015655	-0.015655\\
-0.01529	-0.01529\\
-0.0152425	-0.0152425\\
-0.0154725	-0.0154725\\
-0.0154275	-0.0154275\\
-0.01506	-0.01506\\
-0.014695	-0.014695\\
-0.0143275	-0.0143275\\
-0.0146475	-0.0146475\\
-0.014785	-0.014785\\
-0.0142375	-0.0142375\\
-0.0137325	-0.0137325\\
-0.013825	-0.013825\\
-0.0142825	-0.0142825\\
-0.014145	-0.014145\\
-0.014375	-0.014375\\
-0.0142825	-0.0142825\\
-0.0145575	-0.0145575\\
-0.0151975	-0.0151975\\
-0.0151525	-0.0151525\\
-0.0154275	-0.0154275\\
-0.0160675	-0.0160675\\
-0.0163875	-0.0163875\\
-0.016205	-0.016205\\
-0.015565	-0.015565\\
-0.0151525	-0.0151525\\
-0.014465	-0.014465\\
-0.0145575	-0.0145575\\
-0.014375	-0.014375\\
-0.014785	-0.014785\\
-0.01497	-0.01497\\
-0.014695	-0.014695\\
-0.0148325	-0.0148325\\
-0.014375	-0.014375\\
-0.01451	-0.01451\\
-0.01442	-0.01442\\
-0.013915	-0.013915\\
-0.01387	-0.01387\\
-0.0136875	-0.0136875\\
-0.013595	-0.013595\\
-0.0134575	-0.0134575\\
-0.0131375	-0.0131375\\
-0.01323	-0.01323\\
-0.0134575	-0.0134575\\
-0.013505	-0.013505\\
-0.0140525	-0.0140525\\
-0.01451	-0.01451\\
-0.014465	-0.014465\\
-0.014375	-0.014375\\
-0.0148325	-0.0148325\\
-0.0152425	-0.0152425\\
-0.0151975	-0.0151975\\
-0.0151525	-0.0151525\\
-0.014465	-0.014465\\
-0.0137325	-0.0137325\\
-0.0140525	-0.0140525\\
-0.0146025	-0.0146025\\
-0.0148325	-0.0148325\\
-0.01529	-0.01529\\
-0.0154275	-0.0154275\\
-0.0151525	-0.0151525\\
-0.014695	-0.014695\\
-0.01451	-0.01451\\
-0.0146025	-0.0146025\\
-0.014465	-0.014465\\
-0.0139625	-0.0139625\\
-0.01387	-0.01387\\
-0.0136875	-0.0136875\\
-0.013505	-0.013505\\
-0.0130925	-0.0130925\\
-0.0134575	-0.0134575\\
-0.013915	-0.013915\\
-0.0141	-0.0141\\
-0.013915	-0.013915\\
-0.014375	-0.014375\\
-0.015015	-0.015015\\
-0.01474	-0.01474\\
-0.0148775	-0.0148775\\
-0.0151525	-0.0151525\\
-0.0148775	-0.0148775\\
-0.0143275	-0.0143275\\
-0.0142375	-0.0142375\\
-0.01451	-0.01451\\
-0.0140525	-0.0140525\\
-0.013915	-0.013915\\
-0.0137775	-0.0137775\\
-0.013275	-0.013275\\
-0.0130925	-0.0130925\\
-0.01332	-0.01332\\
-0.0136875	-0.0136875\\
-0.0136425	-0.0136425\\
-0.013825	-0.013825\\
-0.0141	-0.0141\\
-0.01387	-0.01387\\
-0.01332	-0.01332\\
-0.012955	-0.012955\\
-0.0130475	-0.0130475\\
-0.0131375	-0.0131375\\
-0.0133675	-0.0133675\\
-0.0141	-0.0141\\
-0.01451	-0.01451\\
-0.014375	-0.014375\\
-0.0146475	-0.0146475\\
-0.015105	-0.015105\\
-0.0154275	-0.0154275\\
-0.0155175	-0.0155175\\
-0.01529	-0.01529\\
-0.0152425	-0.0152425\\
-0.01529	-0.01529\\
-0.015335	-0.015335\\
-0.0155175	-0.0155175\\
-0.015885	-0.015885\\
-0.0157475	-0.0157475\\
-0.01561	-0.01561\\
-0.0154275	-0.0154275\\
-0.0154725	-0.0154725\\
-0.015105	-0.015105\\
-0.01497	-0.01497\\
-0.0151975	-0.0151975\\
-0.01538	-0.01538\\
-0.0152425	-0.0152425\\
-0.0151975	-0.0151975\\
-0.01497	-0.01497\\
-0.01451	-0.01451\\
-0.014695	-0.014695\\
-0.01529	-0.01529\\
-0.0151975	-0.0151975\\
-0.0146475	-0.0146475\\
-0.0145575	-0.0145575\\
-0.0140075	-0.0140075\\
-0.013595	-0.013595\\
-0.013825	-0.013825\\
-0.01355	-0.01355\\
-0.0133675	-0.0133675\\
-0.0134575	-0.0134575\\
-0.0131825	-0.0131825\\
-0.0134575	-0.0134575\\
-0.0130475	-0.0130475\\
-0.013275	-0.013275\\
-0.0134125	-0.0134125\\
-0.0136425	-0.0136425\\
-0.0136875	-0.0136875\\
-0.0137325	-0.0137325\\
-0.0140525	-0.0140525\\
-0.0141	-0.0141\\
-0.014145	-0.014145\\
-0.015015	-0.015015\\
-0.014785	-0.014785\\
-0.0145575	-0.0145575\\
-0.014695	-0.014695\\
-0.014375	-0.014375\\
-0.0137775	-0.0137775\\
-0.0141	-0.0141\\
-0.013595	-0.013595\\
-0.013915	-0.013915\\
-0.0140075	-0.0140075\\
-0.014145	-0.014145\\
-0.01387	-0.01387\\
-0.0133675	-0.0133675\\
-0.0130475	-0.0130475\\
-0.013	-0.013\\
-0.0134575	-0.0134575\\
-0.013595	-0.013595\\
-0.0136875	-0.0136875\\
-0.0141	-0.0141\\
-0.0146475	-0.0146475\\
-0.0145575	-0.0145575\\
-0.014785	-0.014785\\
-0.0151525	-0.0151525\\
-0.015105	-0.015105\\
-0.01506	-0.01506\\
-0.0157925	-0.0157925\\
-0.0157475	-0.0157475\\
-0.015885	-0.015885\\
-0.0157925	-0.0157925\\
-0.0158375	-0.0158375\\
-0.0160675	-0.0160675\\
-0.015565	-0.015565\\
-0.0148325	-0.0148325\\
-0.01442	-0.01442\\
-0.0146475	-0.0146475\\
-0.0149225	-0.0149225\\
-0.01497	-0.01497\\
-0.01474	-0.01474\\
-0.0140525	-0.0140525\\
-0.0139625	-0.0139625\\
-0.0146475	-0.0146475\\
-0.01529	-0.01529\\
-0.01538	-0.01538\\
-0.0148775	-0.0148775\\
-0.014465	-0.014465\\
-0.0146475	-0.0146475\\
-0.0151975	-0.0151975\\
-0.015565	-0.015565\\
-0.015655	-0.015655\\
-0.01529	-0.01529\\
-0.015565	-0.015565\\
-0.0158375	-0.0158375\\
-0.0157925	-0.0157925\\
-0.01625	-0.01625\\
-0.0160675	-0.0160675\\
-0.01625	-0.01625\\
-0.01616	-0.01616\\
-0.0154275	-0.0154275\\
-0.01506	-0.01506\\
-0.0152425	-0.0152425\\
-0.0154275	-0.0154275\\
-0.0154725	-0.0154725\\
-0.0151975	-0.0151975\\
-0.01538	-0.01538\\
-0.01529	-0.01529\\
-0.01497	-0.01497\\
-0.015105	-0.015105\\
-0.01506	-0.01506\\
-0.015565	-0.015565\\
-0.0157025	-0.0157025\\
-0.0154275	-0.0154275\\
-0.01506	-0.01506\\
-0.01442	-0.01442\\
-0.01451	-0.01451\\
-0.01442	-0.01442\\
-0.01419	-0.01419\\
-0.0140525	-0.0140525\\
-0.014375	-0.014375\\
-0.014785	-0.014785\\
-0.0149225	-0.0149225\\
-0.01497	-0.01497\\
-0.0145575	-0.0145575\\
-0.014145	-0.014145\\
-0.013825	-0.013825\\
-0.0137325	-0.0137325\\
-0.01419	-0.01419\\
-0.0143275	-0.0143275\\
-0.014375	-0.014375\\
-0.0148325	-0.0148325\\
-0.0152425	-0.0152425\\
-0.0155175	-0.0155175\\
-0.01561	-0.01561\\
-0.015885	-0.015885\\
-0.0154725	-0.0154725\\
-0.01506	-0.01506\\
-0.014785	-0.014785\\
-0.014695	-0.014695\\
-0.0149225	-0.0149225\\
-0.0148775	-0.0148775\\
-0.014465	-0.014465\\
-0.0142375	-0.0142375\\
-0.014695	-0.014695\\
-0.01506	-0.01506\\
-0.015015	-0.015015\\
-0.0155175	-0.0155175\\
-0.015885	-0.015885\\
-0.01561	-0.01561\\
-0.015105	-0.015105\\
-0.0146025	-0.0146025\\
-0.01442	-0.01442\\
-0.01474	-0.01474\\
-0.014785	-0.014785\\
-0.0149225	-0.0149225\\
-0.01506	-0.01506\\
-0.0151975	-0.0151975\\
-0.01529	-0.01529\\
-0.01538	-0.01538\\
-0.01506	-0.01506\\
-0.015015	-0.015015\\
-0.0148325	-0.0148325\\
-0.0145575	-0.0145575\\
-0.0149225	-0.0149225\\
-0.01538	-0.01538\\
-0.01561	-0.01561\\
-0.0151525	-0.0151525\\
-0.0157925	-0.0157925\\
-0.0160225	-0.0160225\\
-0.015565	-0.015565\\
-0.0148325	-0.0148325\\
-0.015015	-0.015015\\
-0.01497	-0.01497\\
-0.0148775	-0.0148775\\
-0.015015	-0.015015\\
-0.014785	-0.014785\\
-0.0149225	-0.0149225\\
-0.0157025	-0.0157025\\
-0.015975	-0.015975\\
-0.01561	-0.01561\\
-0.0155175	-0.0155175\\
-0.015655	-0.015655\\
-0.01538	-0.01538\\
-0.01529	-0.01529\\
-0.0148775	-0.0148775\\
-0.015015	-0.015015\\
-0.0151525	-0.0151525\\
-0.0157025	-0.0157025\\
-0.0160225	-0.0160225\\
-0.01657	-0.01657\\
-0.0163875	-0.0163875\\
-0.0160675	-0.0160675\\
-0.0157925	-0.0157925\\
-0.0157475	-0.0157475\\
-0.0154275	-0.0154275\\
-0.01497	-0.01497\\
-0.01474	-0.01474\\
-0.014785	-0.014785\\
-0.01442	-0.01442\\
-0.0139625	-0.0139625\\
-0.0141	-0.0141\\
-0.01451	-0.01451\\
-0.0142375	-0.0142375\\
-0.013915	-0.013915\\
-0.0137325	-0.0137325\\
-0.014465	-0.014465\\
-0.01506	-0.01506\\
-0.014465	-0.014465\\
-0.014375	-0.014375\\
-0.01442	-0.01442\\
-0.0146025	-0.0146025\\
-0.0145575	-0.0145575\\
-0.0142825	-0.0142825\\
-0.013915	-0.013915\\
-0.01355	-0.01355\\
-0.0136425	-0.0136425\\
-0.0142825	-0.0142825\\
-0.0146475	-0.0146475\\
-0.01442	-0.01442\\
-0.0140075	-0.0140075\\
-0.013505	-0.013505\\
-0.01355	-0.01355\\
-0.0141	-0.0141\\
-0.01451	-0.01451\\
-0.014695	-0.014695\\
-0.01442	-0.01442\\
-0.0142825	-0.0142825\\
-0.0143275	-0.0143275\\
-0.0146475	-0.0146475\\
-0.0151975	-0.0151975\\
-0.01561	-0.01561\\
-0.0154275	-0.0154275\\
-0.0148775	-0.0148775\\
-0.01474	-0.01474\\
-0.014375	-0.014375\\
-0.014465	-0.014465\\
-0.0145575	-0.0145575\\
-0.0141	-0.0141\\
-0.013595	-0.013595\\
-0.0137775	-0.0137775\\
-0.0141	-0.0141\\
-0.014695	-0.014695\\
-0.015015	-0.015015\\
-0.01529	-0.01529\\
-0.0154275	-0.0154275\\
-0.0158375	-0.0158375\\
-0.0164325	-0.0164325\\
-0.01648	-0.01648\\
-0.015975	-0.015975\\
-0.01593	-0.01593\\
-0.0161125	-0.0161125\\
-0.01648	-0.01648\\
-0.0161125	-0.0161125\\
-0.0151525	-0.0151525\\
-0.01442	-0.01442\\
-0.01419	-0.01419\\
-0.013915	-0.013915\\
-0.0133675	-0.0133675\\
-0.01323	-0.01323\\
-0.013505	-0.013505\\
-0.013915	-0.013915\\
-0.0140525	-0.0140525\\
-0.01387	-0.01387\\
-0.0137775	-0.0137775\\
-0.0139625	-0.0139625\\
-0.014145	-0.014145\\
-0.0143275	-0.0143275\\
-0.0140525	-0.0140525\\
-0.0136875	-0.0136875\\
-0.013505	-0.013505\\
-0.0137775	-0.0137775\\
-0.0140075	-0.0140075\\
-0.01387	-0.01387\\
-0.0134575	-0.0134575\\
-0.0137775	-0.0137775\\
-0.013595	-0.013595\\
-0.0136425	-0.0136425\\
-0.01387	-0.01387\\
-0.0142375	-0.0142375\\
-0.01451	-0.01451\\
-0.01419	-0.01419\\
-0.01451	-0.01451\\
-0.014695	-0.014695\\
-0.01451	-0.01451\\
-0.0143275	-0.0143275\\
-0.014695	-0.014695\\
-0.0151525	-0.0151525\\
-0.0152425	-0.0152425\\
-0.0154275	-0.0154275\\
-0.01529	-0.01529\\
-0.015105	-0.015105\\
-0.01506	-0.01506\\
-0.01529	-0.01529\\
-0.0151975	-0.0151975\\
-0.0154275	-0.0154275\\
-0.0155175	-0.0155175\\
-0.0154725	-0.0154725\\
-0.0163425	-0.0163425\\
-0.0162975	-0.0162975\\
-0.015565	-0.015565\\
-0.0149225	-0.0149225\\
-0.014785	-0.014785\\
-0.0149225	-0.0149225\\
-0.0151975	-0.0151975\\
-0.0154725	-0.0154725\\
-0.015655	-0.015655\\
-0.015335	-0.015335\\
-0.015105	-0.015105\\
-0.0151975	-0.0151975\\
-0.0152425	-0.0152425\\
-0.014785	-0.014785\\
-0.0142825	-0.0142825\\
-0.0145575	-0.0145575\\
-0.014695	-0.014695\\
-0.01497	-0.01497\\
-0.01538	-0.01538\\
-0.01529	-0.01529\\
-0.01497	-0.01497\\
-0.014695	-0.014695\\
-0.01497	-0.01497\\
-0.01538	-0.01538\\
-0.0157475	-0.0157475\\
-0.0160675	-0.0160675\\
-0.0163425	-0.0163425\\
-0.01625	-0.01625\\
-0.0161125	-0.0161125\\
-0.0155175	-0.0155175\\
-0.0148775	-0.0148775\\
-0.01529	-0.01529\\
-0.0157475	-0.0157475\\
-0.015335	-0.015335\\
-0.01561	-0.01561\\
-0.01616	-0.01616\\
-0.01648	-0.01648\\
-0.0161125	-0.0161125\\
-0.015335	-0.015335\\
-0.0146025	-0.0146025\\
-0.0146475	-0.0146475\\
-0.014695	-0.014695\\
-0.0146475	-0.0146475\\
-0.0142375	-0.0142375\\
-0.0142825	-0.0142825\\
-0.014145	-0.014145\\
-0.0142375	-0.0142375\\
-0.0141	-0.0141\\
-0.01419	-0.01419\\
-0.014695	-0.014695\\
-0.01474	-0.01474\\
-0.015105	-0.015105\\
-0.0155175	-0.0155175\\
-0.01561	-0.01561\\
-0.01506	-0.01506\\
-0.0148775	-0.0148775\\
-0.01506	-0.01506\\
-0.014785	-0.014785\\
-0.0145575	-0.0145575\\
-0.0142375	-0.0142375\\
-0.014375	-0.014375\\
-0.01442	-0.01442\\
-0.0140075	-0.0140075\\
-0.0133675	-0.0133675\\
-0.013	-0.013\\
-0.0134125	-0.0134125\\
-0.013915	-0.013915\\
-0.014145	-0.014145\\
-0.0137775	-0.0137775\\
-0.013825	-0.013825\\
-0.0142375	-0.0142375\\
-0.014695	-0.014695\\
-0.014465	-0.014465\\
-0.0142825	-0.0142825\\
-0.01442	-0.01442\\
-0.0146475	-0.0146475\\
-0.01474	-0.01474\\
-0.014785	-0.014785\\
-0.01451	-0.01451\\
-0.0142375	-0.0142375\\
-0.0143275	-0.0143275\\
-0.0142375	-0.0142375\\
-0.014375	-0.014375\\
-0.014695	-0.014695\\
-0.015335	-0.015335\\
-0.0151975	-0.0151975\\
-0.015015	-0.015015\\
-0.01561	-0.01561\\
-0.0155175	-0.0155175\\
-0.01497	-0.01497\\
-0.01442	-0.01442\\
-0.0141	-0.0141\\
-0.0140075	-0.0140075\\
-0.0137775	-0.0137775\\
-0.0136875	-0.0136875\\
-0.0142825	-0.0142825\\
-0.0137325	-0.0137325\\
-0.0133675	-0.0133675\\
-0.0136875	-0.0136875\\
-0.0140525	-0.0140525\\
-0.0140075	-0.0140075\\
-0.0136875	-0.0136875\\
-0.01323	-0.01323\\
-0.0133675	-0.0133675\\
-0.0140525	-0.0140525\\
-0.0146025	-0.0146025\\
-0.0145575	-0.0145575\\
-0.0142825	-0.0142825\\
-0.014375	-0.014375\\
-0.0145575	-0.0145575\\
-0.014785	-0.014785\\
-0.015105	-0.015105\\
-0.0151975	-0.0151975\\
-0.01506	-0.01506\\
-0.0146475	-0.0146475\\
-0.0143275	-0.0143275\\
-0.0142375	-0.0142375\\
-0.0145575	-0.0145575\\
-0.01419	-0.01419\\
-0.013915	-0.013915\\
-0.01442	-0.01442\\
-0.014785	-0.014785\\
-0.01497	-0.01497\\
-0.01538	-0.01538\\
-0.0148775	-0.0148775\\
-0.01529	-0.01529\\
-0.01561	-0.01561\\
-0.0154725	-0.0154725\\
-0.0155175	-0.0155175\\
-0.015565	-0.015565\\
-0.016205	-0.016205\\
-0.016525	-0.016525\\
-0.0160675	-0.0160675\\
-0.01616	-0.01616\\
-0.01648	-0.01648\\
-0.016525	-0.016525\\
-0.0166625	-0.0166625\\
-0.01703	-0.01703\\
-0.0169375	-0.0169375\\
-0.01648	-0.01648\\
-0.0157475	-0.0157475\\
-0.0155175	-0.0155175\\
-0.01529	-0.01529\\
-0.01497	-0.01497\\
-0.0151525	-0.0151525\\
-0.015655	-0.015655\\
-0.01529	-0.01529\\
-0.01497	-0.01497\\
-0.0149225	-0.0149225\\
-0.01451	-0.01451\\
-0.0140075	-0.0140075\\
-0.0137775	-0.0137775\\
-0.01355	-0.01355\\
-0.0137325	-0.0137325\\
-0.0136425	-0.0136425\\
-0.01332	-0.01332\\
-0.013	-0.013\\
-0.012635	-0.012635\\
-0.012955	-0.012955\\
-0.0128625	-0.0128625\\
-0.012725	-0.012725\\
-0.013505	-0.013505\\
-0.0141	-0.0141\\
-0.014375	-0.014375\\
-0.0141	-0.0141\\
-0.0142825	-0.0142825\\
-0.0149225	-0.0149225\\
-0.01442	-0.01442\\
-0.0137325	-0.0137325\\
-0.01332	-0.01332\\
-0.012955	-0.012955\\
-0.0130925	-0.0130925\\
-0.0136425	-0.0136425\\
-0.0142825	-0.0142825\\
-0.013915	-0.013915\\
-0.0143275	-0.0143275\\
-0.015015	-0.015015\\
-0.0151975	-0.0151975\\
-0.015015	-0.015015\\
-0.014375	-0.014375\\
-0.01474	-0.01474\\
-0.0151525	-0.0151525\\
-0.014695	-0.014695\\
-0.0140075	-0.0140075\\
-0.0141	-0.0141\\
-0.0140075	-0.0140075\\
-0.013275	-0.013275\\
-0.012635	-0.012635\\
-0.0130475	-0.0130475\\
-0.0142375	-0.0142375\\
-0.0146025	-0.0146025\\
-0.0143275	-0.0143275\\
-0.0137775	-0.0137775\\
-0.01355	-0.01355\\
-0.0125875	-0.0125875\\
-0.012635	-0.012635\\
-0.01268	-0.01268\\
-0.013	-0.013\\
-0.01355	-0.01355\\
-0.0142825	-0.0142825\\
-0.0146475	-0.0146475\\
-0.014785	-0.014785\\
-0.0148775	-0.0148775\\
-0.0149225	-0.0149225\\
-0.014785	-0.014785\\
-0.0149225	-0.0149225\\
-0.015565	-0.015565\\
-0.0157475	-0.0157475\\
-0.01506	-0.01506\\
-0.0141	-0.0141\\
-0.014695	-0.014695\\
-0.0151975	-0.0151975\\
-0.015105	-0.015105\\
-0.01451	-0.01451\\
-0.0143275	-0.0143275\\
-0.0148325	-0.0148325\\
-0.01561	-0.01561\\
-0.0158375	-0.0158375\\
-0.0160675	-0.0160675\\
-0.0161125	-0.0161125\\
-0.015885	-0.015885\\
-0.0157925	-0.0157925\\
-0.0149225	-0.0149225\\
-0.01442	-0.01442\\
-0.014695	-0.014695\\
-0.0149225	-0.0149225\\
-0.015015	-0.015015\\
-0.015105	-0.015105\\
-0.01474	-0.01474\\
-0.0148775	-0.0148775\\
-0.014785	-0.014785\\
-0.0142825	-0.0142825\\
-0.0136425	-0.0136425\\
-0.01323	-0.01323\\
-0.0134125	-0.0134125\\
-0.01323	-0.01323\\
-0.0128175	-0.0128175\\
-0.0130925	-0.0130925\\
-0.0141	-0.0141\\
-0.013915	-0.013915\\
-0.0136875	-0.0136875\\
-0.013915	-0.013915\\
-0.014695	-0.014695\\
-0.0146025	-0.0146025\\
-0.01387	-0.01387\\
-0.0130925	-0.0130925\\
-0.0136875	-0.0136875\\
-0.01474	-0.01474\\
-0.0154275	-0.0154275\\
-0.0160225	-0.0160225\\
-0.0166175	-0.0166175\\
-0.0168	-0.0168\\
-0.0163875	-0.0163875\\
-0.016205	-0.016205\\
-0.016525	-0.016525\\
-0.01648	-0.01648\\
-0.0163425	-0.0163425\\
-0.01648	-0.01648\\
-0.01657	-0.01657\\
-0.0166625	-0.0166625\\
-0.017165	-0.017165\\
-0.0168925	-0.0168925\\
-0.0161125	-0.0161125\\
-0.01506	-0.01506\\
-0.01442	-0.01442\\
-0.0143275	-0.0143275\\
-0.0142825	-0.0142825\\
-0.0148775	-0.0148775\\
-0.014785	-0.014785\\
-0.014695	-0.014695\\
-0.015015	-0.015015\\
-0.0146025	-0.0146025\\
-0.01451	-0.01451\\
-0.0148325	-0.0148325\\
-0.01506	-0.01506\\
-0.0146475	-0.0146475\\
-0.0141	-0.0141\\
-0.014145	-0.014145\\
-0.0140525	-0.0140525\\
-0.0148775	-0.0148775\\
-0.0157475	-0.0157475\\
-0.0158375	-0.0158375\\
-0.0160675	-0.0160675\\
-0.0163425	-0.0163425\\
-0.0169375	-0.0169375\\
-0.01703	-0.01703\\
-0.0164325	-0.0164325\\
-0.0157925	-0.0157925\\
-0.0149225	-0.0149225\\
-0.0145575	-0.0145575\\
-0.0146025	-0.0146025\\
-0.0148325	-0.0148325\\
-0.014695	-0.014695\\
-0.014465	-0.014465\\
-0.0145575	-0.0145575\\
-0.0146475	-0.0146475\\
-0.014785	-0.014785\\
-0.01506	-0.01506\\
-0.0148775	-0.0148775\\
-0.0141	-0.0141\\
-0.01355	-0.01355\\
-0.0131825	-0.0131825\\
-0.013	-0.013\\
-0.0130925	-0.0130925\\
-0.013595	-0.013595\\
-0.013825	-0.013825\\
-0.0140075	-0.0140075\\
-0.0146475	-0.0146475\\
-0.0145575	-0.0145575\\
-0.015015	-0.015015\\
-0.01593	-0.01593\\
-0.015885	-0.015885\\
-0.0157475	-0.0157475\\
-0.015885	-0.015885\\
-0.0161125	-0.0161125\\
-0.0157925	-0.0157925\\
-0.0154725	-0.0154725\\
-0.01497	-0.01497\\
-0.01474	-0.01474\\
-0.015565	-0.015565\\
-0.0166175	-0.0166175\\
-0.0172125	-0.0172125\\
-0.017075	-0.017075\\
-0.0168925	-0.0168925\\
-0.0163875	-0.0163875\\
-0.0163425	-0.0163425\\
-0.0166625	-0.0166625\\
-0.0161125	-0.0161125\\
-0.015655	-0.015655\\
-0.015335	-0.015335\\
-0.0158375	-0.0158375\\
-0.0160675	-0.0160675\\
-0.015335	-0.015335\\
-0.01506	-0.01506\\
-0.014695	-0.014695\\
-0.01497	-0.01497\\
-0.015565	-0.015565\\
-0.0157925	-0.0157925\\
-0.01561	-0.01561\\
-0.0151975	-0.0151975\\
-0.01506	-0.01506\\
-0.0146475	-0.0146475\\
-0.0142375	-0.0142375\\
-0.013915	-0.013915\\
-0.0136425	-0.0136425\\
-0.0141	-0.0141\\
-0.0146475	-0.0146475\\
-0.01474	-0.01474\\
-0.014375	-0.014375\\
-0.0141	-0.0141\\
-0.013915	-0.013915\\
-0.0140075	-0.0140075\\
-0.0143275	-0.0143275\\
-0.01451	-0.01451\\
-0.0140525	-0.0140525\\
-0.014465	-0.014465\\
-0.0151975	-0.0151975\\
-0.015015	-0.015015\\
-0.0140525	-0.0140525\\
-0.0137775	-0.0137775\\
-0.0142375	-0.0142375\\
-0.014375	-0.014375\\
-0.0143275	-0.0143275\\
-0.013915	-0.013915\\
-0.0142375	-0.0142375\\
-0.0148325	-0.0148325\\
-0.014465	-0.014465\\
-0.013505	-0.013505\\
-0.0133675	-0.0133675\\
-0.0142825	-0.0142825\\
-0.0148775	-0.0148775\\
-0.01442	-0.01442\\
-0.014145	-0.014145\\
-0.014695	-0.014695\\
-0.01529	-0.01529\\
-0.015335	-0.015335\\
-0.015655	-0.015655\\
-0.016205	-0.016205\\
-0.015885	-0.015885\\
-0.0154725	-0.0154725\\
-0.0158375	-0.0158375\\
-0.015565	-0.015565\\
-0.0157025	-0.0157025\\
-0.0160675	-0.0160675\\
-0.015885	-0.015885\\
-0.0149225	-0.0149225\\
-0.015105	-0.015105\\
-0.015975	-0.015975\\
-0.01648	-0.01648\\
-0.01625	-0.01625\\
-0.0160225	-0.0160225\\
-0.016205	-0.016205\\
-0.0161125	-0.0161125\\
-0.0155175	-0.0155175\\
-0.0152425	-0.0152425\\
-0.01538	-0.01538\\
-0.0154725	-0.0154725\\
-0.015565	-0.015565\\
-0.01538	-0.01538\\
-0.01529	-0.01529\\
-0.015565	-0.015565\\
-0.0151975	-0.0151975\\
-0.014785	-0.014785\\
-0.01451	-0.01451\\
-0.0141	-0.0141\\
-0.0140525	-0.0140525\\
-0.0146025	-0.0146025\\
-0.014695	-0.014695\\
-0.01497	-0.01497\\
-0.0154275	-0.0154275\\
-0.0157475	-0.0157475\\
-0.01561	-0.01561\\
-0.01538	-0.01538\\
-0.0146025	-0.0146025\\
-0.01451	-0.01451\\
-0.01529	-0.01529\\
-0.0151525	-0.0151525\\
-0.01442	-0.01442\\
-0.01474	-0.01474\\
-0.015335	-0.015335\\
-0.015565	-0.015565\\
-0.015975	-0.015975\\
-0.01657	-0.01657\\
-0.0163875	-0.0163875\\
-0.0161125	-0.0161125\\
-0.016205	-0.016205\\
-0.0158375	-0.0158375\\
-0.0154725	-0.0154725\\
-0.015655	-0.015655\\
-0.0157925	-0.0157925\\
-0.0158375	-0.0158375\\
-0.015335	-0.015335\\
-0.01497	-0.01497\\
-0.014695	-0.014695\\
-0.0148775	-0.0148775\\
-0.014695	-0.014695\\
-0.0152425	-0.0152425\\
-0.0155175	-0.0155175\\
-0.01529	-0.01529\\
-0.01497	-0.01497\\
-0.0151975	-0.0151975\\
-0.0148325	-0.0148325\\
-0.014145	-0.014145\\
-0.0140075	-0.0140075\\
-0.0143275	-0.0143275\\
-0.0142825	-0.0142825\\
-0.0139625	-0.0139625\\
-0.01332	-0.01332\\
-0.0134575	-0.0134575\\
-0.0139625	-0.0139625\\
-0.0145575	-0.0145575\\
-0.01497	-0.01497\\
-0.01538	-0.01538\\
-0.0157475	-0.0157475\\
-0.0151975	-0.0151975\\
-0.014695	-0.014695\\
-0.01538	-0.01538\\
-0.01616	-0.01616\\
-0.0161125	-0.0161125\\
-0.015975	-0.015975\\
-0.0164325	-0.0164325\\
-0.01648	-0.01648\\
-0.016205	-0.016205\\
-0.015885	-0.015885\\
-0.0157025	-0.0157025\\
-0.0160225	-0.0160225\\
-0.0157025	-0.0157025\\
-0.01538	-0.01538\\
-0.0158375	-0.0158375\\
-0.01625	-0.01625\\
-0.0164325	-0.0164325\\
-0.0157925	-0.0157925\\
-0.014695	-0.014695\\
-0.0154275	-0.0154275\\
-0.01451	-0.01451\\
-0.01387	-0.01387\\
-0.0141	-0.0141\\
-0.0146025	-0.0146025\\
-0.01529	-0.01529\\
-0.0151975	-0.0151975\\
-0.01474	-0.01474\\
-0.0142375	-0.0142375\\
-0.013595	-0.013595\\
-0.01355	-0.01355\\
-0.013505	-0.013505\\
-0.013595	-0.013595\\
-0.014145	-0.014145\\
-0.0143275	-0.0143275\\
-0.0136875	-0.0136875\\
-0.0133675	-0.0133675\\
-0.013505	-0.013505\\
-0.0137775	-0.0137775\\
-0.0133675	-0.0133675\\
-0.013505	-0.013505\\
-0.01332	-0.01332\\
-0.0130475	-0.0130475\\
-0.013595	-0.013595\\
-0.0143275	-0.0143275\\
-0.01506	-0.01506\\
-0.0157925	-0.0157925\\
-0.01593	-0.01593\\
-0.015105	-0.015105\\
-0.01451	-0.01451\\
-0.0143275	-0.0143275\\
-0.013915	-0.013915\\
-0.0134575	-0.0134575\\
-0.0139625	-0.0139625\\
-0.0142375	-0.0142375\\
-0.01419	-0.01419\\
-0.0142825	-0.0142825\\
-0.014465	-0.014465\\
-0.0146025	-0.0146025\\
-0.014695	-0.014695\\
-0.015015	-0.015015\\
-0.0151525	-0.0151525\\
-0.015655	-0.015655\\
-0.01506	-0.01506\\
-0.0140525	-0.0140525\\
-0.0137325	-0.0137325\\
-0.01419	-0.01419\\
-0.0142375	-0.0142375\\
-0.0140525	-0.0140525\\
-0.0134125	-0.0134125\\
-0.01332	-0.01332\\
-0.0131825	-0.0131825\\
-0.012635	-0.012635\\
-0.01323	-0.01323\\
-0.0136875	-0.0136875\\
-0.0142375	-0.0142375\\
-0.01497	-0.01497\\
-0.0148775	-0.0148775\\
-0.013915	-0.013915\\
-0.0140525	-0.0140525\\
-0.01442	-0.01442\\
-0.0137325	-0.0137325\\
-0.0136875	-0.0136875\\
-0.014465	-0.014465\\
-0.0146475	-0.0146475\\
-0.01529	-0.01529\\
-0.0158375	-0.0158375\\
-0.0155175	-0.0155175\\
-0.015105	-0.015105\\
-0.0149225	-0.0149225\\
-0.015105	-0.015105\\
-0.015565	-0.015565\\
-0.0154275	-0.0154275\\
-0.0148325	-0.0148325\\
};
\end{axis}

\begin{axis}[%
width=4.927cm,
height=2.746cm,
at={(6.484cm,15.254cm)},
scale only axis,
xmin=-0.018,
xmax=-0.012,
xlabel style={font=\color{white!15!black}},
xlabel={$u(t-1)$},
ymin=-0.12207,
ymax=0,
ylabel style={font=\color{white!15!black}},
ylabel={$\delta^4 y(t)$},
axis background/.style={fill=white},
title style={font=\bfseries},
title={C2, R = 0.6914},
axis x line*=bottom,
axis y line*=left
]
\addplot[only marks, mark=*, mark options={}, mark size=1.5000pt, color=mycolor1, fill=mycolor1] table[row sep=crcr]{%
x	y\\
-0.01561	-0.061035\\
-0.015655	-0.07019\\
-0.0157925	-0.0579825\\
-0.015655	-0.03357\\
-0.0149225	-0.042725\\
-0.0148325	-0.042725\\
-0.0148775	-0.0457775\\
-0.0149225	-0.0396725\\
-0.0148775	-0.0213625\\
-0.0140075	-0.01526\\
-0.0131825	-0.01831\\
-0.0128625	-0.024415\\
-0.0137775	-0.0549325\\
-0.014785	-0.0579825\\
-0.01506	-0.0579825\\
-0.0151525	-0.0488275\\
-0.0148775	-0.027465\\
-0.0143275	-0.042725\\
-0.01497	-0.0488275\\
-0.0148325	-0.03357\\
-0.014375	-0.0457775\\
-0.0148325	-0.0488275\\
-0.0148775	-0.03662\\
-0.014695	-0.061035\\
-0.015105	-0.0549325\\
-0.01529	-0.0396725\\
-0.01497	-0.0579825\\
-0.01529	-0.0640875\\
-0.01538	-0.0488275\\
-0.0151525	-0.042725\\
-0.0149225	-0.03662\\
-0.0146475	-0.042725\\
-0.014695	-0.05188\\
-0.0149225	-0.0457775\\
-0.0149225	-0.0457775\\
-0.01497	-0.0488275\\
-0.01497	-0.05188\\
-0.01506	-0.07019\\
-0.0154725	-0.0640875\\
-0.0155175	-0.042725\\
-0.015105	-0.0396725\\
-0.0149225	-0.0305175\\
-0.014375	-0.027465\\
-0.0141	-0.0396725\\
-0.014375	-0.0305175\\
-0.0142375	-0.0305175\\
-0.014145	-0.0396725\\
-0.014465	-0.0457775\\
-0.014695	-0.0396725\\
-0.014695	-0.0640875\\
-0.0151525	-0.0549325\\
-0.015105	-0.0305175\\
-0.01451	-0.024415\\
-0.014145	-0.03357\\
-0.0142825	-0.0396725\\
-0.01442	-0.027465\\
-0.0139625	-0.027465\\
-0.0137325	-0.0305175\\
-0.0140075	-0.024415\\
-0.0139625	-0.0213625\\
-0.0136875	-0.0305175\\
-0.01387	-0.03662\\
-0.0140525	-0.042725\\
-0.0145575	-0.0457775\\
-0.01474	-0.0579825\\
-0.0148775	-0.0549325\\
-0.0149225	-0.0549325\\
-0.0151525	-0.042725\\
-0.01506	-0.042725\\
-0.0148775	-0.042725\\
-0.014785	-0.0396725\\
-0.01474	-0.042725\\
-0.014695	-0.061035\\
-0.0152425	-0.0885\\
-0.015975	-0.0915525\\
-0.01625	-0.08545\\
-0.0163425	-0.061035\\
-0.0158375	-0.0732425\\
-0.0160675	-0.0915525\\
-0.01648	-0.1007075\\
-0.0166175	-0.079345\\
-0.0163875	-0.094605\\
-0.0166175	-0.12207\\
-0.017165	-0.0976575\\
-0.0169375	-0.076295\\
-0.0166625	-0.0640875\\
-0.0161125	-0.0488275\\
-0.01561	-0.03662\\
-0.01529	-0.0488275\\
-0.015335	-0.0396725\\
-0.0151525	-0.024415\\
-0.01451	-0.024415\\
-0.01419	-0.0305175\\
-0.0142825	-0.042725\\
-0.014695	-0.0396725\\
-0.01474	-0.0396725\\
-0.01474	-0.0488275\\
-0.015015	-0.0488275\\
-0.0151975	-0.07019\\
-0.015565	-0.0579825\\
-0.015565	-0.0640875\\
-0.0157025	-0.0579825\\
-0.015565	-0.0457775\\
-0.01529	-0.05188\\
-0.01538	-0.042725\\
-0.0151525	-0.03662\\
-0.0149225	-0.0457775\\
-0.01506	-0.042725\\
-0.01506	-0.0396725\\
-0.014785	-0.03357\\
-0.014695	-0.027465\\
-0.01442	-0.0305175\\
-0.014375	-0.03357\\
-0.014375	-0.027465\\
-0.0141	-0.024415\\
-0.014145	-0.03662\\
-0.014375	-0.0488275\\
-0.0148775	-0.0579825\\
-0.015105	-0.0579825\\
-0.0152425	-0.042725\\
-0.0148325	-0.0457775\\
-0.015015	-0.076295\\
-0.01561	-0.0579825\\
-0.015565	-0.061035\\
-0.01561	-0.0671375\\
-0.0157475	-0.042725\\
-0.0151525	-0.027465\\
-0.0146475	-0.042725\\
-0.01474	-0.0457775\\
-0.0149225	-0.0671375\\
-0.015565	-0.08545\\
-0.0160225	-0.079345\\
-0.01616	-0.0640875\\
-0.01593	-0.061035\\
-0.0158375	-0.0579825\\
-0.0157025	-0.0579825\\
-0.015655	-0.0549325\\
-0.015655	-0.0457775\\
-0.01529	-0.03662\\
-0.01497	-0.03357\\
-0.014785	-0.0305175\\
-0.0146025	-0.03357\\
-0.0146475	-0.0488275\\
-0.01497	-0.0640875\\
-0.015565	-0.0549325\\
-0.0155175	-0.03662\\
-0.015105	-0.03357\\
-0.0148775	-0.03662\\
-0.01474	-0.0305175\\
-0.0142375	-0.0213625\\
-0.0136875	-0.0213625\\
-0.013595	-0.03357\\
-0.0141	-0.0305175\\
-0.0140525	-0.0305175\\
-0.0140525	-0.0305175\\
-0.0141	-0.0305175\\
-0.0141	-0.0213625\\
-0.013825	-0.01831\\
-0.0134575	-0.0122075\\
-0.0130475	-0.01831\\
-0.013275	-0.042725\\
-0.0142375	-0.061035\\
-0.015015	-0.061035\\
-0.015335	-0.0488275\\
-0.01506	-0.0305175\\
-0.0145575	-0.0305175\\
-0.0141	-0.0213625\\
-0.013595	-0.027465\\
-0.014145	-0.03662\\
-0.0142825	-0.03662\\
-0.0146025	-0.061035\\
-0.01506	-0.08545\\
-0.01593	-0.1007075\\
-0.016525	-0.0823975\\
-0.01648	-0.079345\\
-0.0164325	-0.0579825\\
-0.015885	-0.0640875\\
-0.015885	-0.0671375\\
-0.01593	-0.07019\\
-0.0160225	-0.0549325\\
-0.0158375	-0.05188\\
-0.01561	-0.0488275\\
-0.015565	-0.0488275\\
-0.0154275	-0.0549325\\
-0.0155175	-0.0488275\\
-0.01529	-0.0579825\\
-0.0157025	-0.0732425\\
-0.0161125	-0.061035\\
-0.0158375	-0.03662\\
-0.0152425	-0.0396725\\
-0.01506	-0.0640875\\
-0.01561	-0.0823975\\
-0.0160225	-0.0885\\
-0.0163875	-0.0976575\\
-0.0166175	-0.094605\\
-0.0167075	-0.076295\\
-0.0164325	-0.0671375\\
-0.01625	-0.076295\\
-0.0163425	-0.0885\\
-0.01657	-0.0640875\\
-0.01616	-0.03662\\
-0.0154275	-0.0488275\\
-0.015565	-0.05188\\
-0.0154725	-0.0305175\\
-0.01497	-0.03357\\
-0.01497	-0.0457775\\
-0.0151525	-0.03662\\
-0.01497	-0.0305175\\
-0.014695	-0.0457775\\
-0.01474	-0.027465\\
-0.014695	-0.03662\\
-0.0148325	-0.061035\\
-0.01529	-0.0549325\\
-0.015335	-0.03662\\
-0.015015	-0.03662\\
-0.0148775	-0.05188\\
-0.0152425	-0.08545\\
-0.0161125	-0.07019\\
-0.015975	-0.0396725\\
-0.01538	-0.03662\\
-0.0149225	-0.03662\\
-0.01497	-0.0305175\\
-0.0148775	-0.027465\\
-0.0146025	-0.027465\\
-0.01451	-0.03357\\
-0.0146475	-0.042725\\
-0.0148775	-0.0488275\\
-0.015105	-0.03662\\
-0.014785	-0.0488275\\
-0.0151525	-0.0671375\\
-0.0155175	-0.0579825\\
-0.0154725	-0.042725\\
-0.015335	-0.0549325\\
-0.01529	-0.0640875\\
-0.015655	-0.0549325\\
-0.015335	-0.0213625\\
-0.014375	-0.0305175\\
-0.0142375	-0.03357\\
-0.01442	-0.0396725\\
-0.01474	-0.0396725\\
-0.01474	-0.027465\\
-0.014465	-0.042725\\
-0.014785	-0.0396725\\
-0.0148775	-0.027465\\
-0.0146025	-0.03662\\
-0.0146475	-0.03662\\
-0.0146475	-0.0305175\\
-0.014465	-0.03662\\
-0.0145575	-0.027465\\
-0.01419	-0.024415\\
-0.0141	-0.0305175\\
-0.01387	-0.0213625\\
-0.013825	-0.042725\\
-0.0142825	-0.0457775\\
-0.0146025	-0.05188\\
-0.015015	-0.0732425\\
-0.01561	-0.05188\\
-0.01538	-0.027465\\
-0.014695	-0.027465\\
-0.0142825	-0.0213625\\
-0.0141	-0.03357\\
-0.014375	-0.0305175\\
-0.0140525	-0.024415\\
-0.0139625	-0.0396725\\
-0.01419	-0.0488275\\
-0.0149225	-0.05188\\
-0.01506	-0.0579825\\
-0.01538	-0.0579825\\
-0.0151525	-0.042725\\
-0.0149225	-0.0396725\\
-0.0142825	-0.027465\\
-0.013915	-0.027465\\
-0.01355	-0.01831\\
-0.0133675	-0.0213625\\
-0.013505	-0.027465\\
-0.0139625	-0.042725\\
-0.0143275	-0.0457775\\
-0.01442	-0.042725\\
-0.0145575	-0.061035\\
-0.0152425	-0.0640875\\
-0.01538	-0.0396725\\
-0.01506	-0.0488275\\
-0.0151525	-0.0579825\\
-0.0152425	-0.0671375\\
-0.01561	-0.0640875\\
-0.0154725	-0.042725\\
-0.015015	-0.027465\\
-0.01419	-0.01831\\
-0.01355	-0.01526\\
-0.0134125	-0.0213625\\
-0.0134575	-0.024415\\
-0.013505	-0.0213625\\
-0.0134575	-0.027465\\
-0.0136425	-0.042725\\
-0.01419	-0.0396725\\
-0.0142825	-0.03357\\
-0.0143275	-0.0396725\\
-0.014465	-0.0305175\\
-0.0141	-0.01526\\
-0.0134125	-0.024415\\
-0.0137775	-0.05188\\
-0.01442	-0.042725\\
-0.0145575	-0.0457775\\
-0.014695	-0.0396725\\
-0.0146475	-0.027465\\
-0.014375	-0.042725\\
-0.0146025	-0.0579825\\
-0.015015	-0.0579825\\
-0.0151975	-0.042725\\
-0.0148325	-0.0671375\\
-0.01529	-0.0640875\\
-0.015335	-0.0457775\\
-0.0151525	-0.0579825\\
-0.015335	-0.0640875\\
-0.01538	-0.0488275\\
-0.0151525	-0.0396725\\
-0.0149225	-0.061035\\
-0.0154275	-0.0885\\
-0.0160225	-0.0732425\\
-0.01593	-0.042725\\
-0.01529	-0.042725\\
-0.01506	-0.0457775\\
-0.015015	-0.05188\\
-0.0151975	-0.0579825\\
-0.0154275	-0.0488275\\
-0.0151975	-0.042725\\
-0.0151525	-0.061035\\
-0.0154275	-0.0640875\\
-0.01561	-0.0457775\\
-0.01529	-0.0396725\\
-0.0149225	-0.0457775\\
-0.01506	-0.07019\\
-0.0157475	-0.0671375\\
-0.0157925	-0.061035\\
-0.0157475	-0.0488275\\
-0.015335	-0.042725\\
-0.01497	-0.0457775\\
-0.0148775	-0.03357\\
-0.014465	-0.01526\\
-0.013825	-0.01526\\
-0.0133675	-0.01526\\
-0.0130925	-0.024415\\
-0.0136425	-0.0488275\\
-0.01419	-0.0396725\\
-0.0146025	-0.03662\\
-0.01442	-0.03357\\
-0.01442	-0.042725\\
-0.0146025	-0.03357\\
-0.01419	-0.024415\\
-0.0140525	-0.0305175\\
-0.0140525	-0.024415\\
-0.0137775	-0.0213625\\
-0.013915	-0.042725\\
-0.014465	-0.05188\\
-0.0148775	-0.042725\\
-0.0145575	-0.027465\\
-0.014145	-0.027465\\
-0.0139625	-0.03357\\
-0.0139625	-0.024415\\
-0.0136875	-0.027465\\
-0.0142825	-0.07019\\
-0.0152425	-0.05188\\
-0.0151975	-0.0640875\\
-0.0154725	-0.0885\\
-0.0158375	-0.076295\\
-0.015975	-0.061035\\
-0.0157475	-0.0457775\\
-0.015335	-0.0457775\\
-0.0151975	-0.0488275\\
-0.0151975	-0.0549325\\
-0.015335	-0.0640875\\
-0.0155175	-0.0640875\\
-0.015565	-0.079345\\
-0.01593	-0.08545\\
-0.0162975	-0.10376\\
-0.0166175	-0.0823975\\
-0.0163425	-0.07019\\
-0.01625	-0.0885\\
-0.0164325	-0.0671375\\
-0.01625	-0.0732425\\
-0.0162975	-0.07019\\
-0.016205	-0.042725\\
-0.0155175	-0.027465\\
-0.014785	-0.0305175\\
-0.0145575	-0.042725\\
-0.0148325	-0.03662\\
-0.01474	-0.024415\\
-0.0143275	-0.03357\\
-0.014465	-0.03357\\
-0.0142825	-0.01831\\
-0.013915	-0.0213625\\
-0.0139625	-0.0305175\\
-0.0140525	-0.01831\\
-0.01387	-0.03662\\
-0.0142375	-0.05188\\
-0.01474	-0.03357\\
-0.0145575	-0.0549325\\
-0.0151525	-0.076295\\
-0.01593	-0.0732425\\
-0.015975	-0.0885\\
-0.01625	-0.0732425\\
-0.0160675	-0.07019\\
-0.015975	-0.0579825\\
-0.015655	-0.0305175\\
-0.01497	-0.03662\\
-0.0149225	-0.0396725\\
-0.01474	-0.027465\\
-0.01442	-0.0305175\\
-0.01451	-0.03357\\
-0.0140525	-0.009155\\
-0.013505	-0.0213625\\
-0.0133675	-0.042725\\
-0.0141	-0.0488275\\
-0.0146025	-0.05188\\
-0.0149225	-0.0579825\\
-0.01506	-0.05188\\
-0.01506	-0.0579825\\
-0.0151975	-0.0457775\\
-0.015105	-0.0396725\\
-0.0148325	-0.0396725\\
-0.0148325	-0.0305175\\
-0.0146475	-0.0457775\\
-0.01497	-0.042725\\
-0.01474	-0.0305175\\
-0.014465	-0.042725\\
-0.014695	-0.0579825\\
-0.0151525	-0.042725\\
-0.015015	-0.03357\\
-0.014695	-0.0305175\\
-0.01451	-0.03662\\
-0.0145575	-0.01831\\
-0.0142825	-0.0213625\\
-0.0140525	-0.0305175\\
-0.0143275	-0.042725\\
-0.0145575	-0.042725\\
-0.014695	-0.03662\\
-0.0145575	-0.0396725\\
-0.01474	-0.0457775\\
-0.014785	-0.0305175\\
-0.0142825	-0.024415\\
-0.014145	-0.0457775\\
-0.014785	-0.0671375\\
-0.01538	-0.0671375\\
-0.0154725	-0.0549325\\
-0.015335	-0.0488275\\
-0.0152425	-0.042725\\
-0.01506	-0.03662\\
-0.0148775	-0.03357\\
-0.014695	-0.042725\\
-0.0148325	-0.042725\\
-0.0148325	-0.027465\\
-0.014465	-0.03662\\
-0.0146475	-0.0457775\\
-0.0148325	-0.0488275\\
-0.015015	-0.0549325\\
-0.0151975	-0.0549325\\
-0.0152425	-0.0732425\\
-0.01561	-0.0885\\
-0.0161125	-0.0915525\\
-0.01648	-0.0671375\\
-0.015975	-0.03662\\
-0.01529	-0.027465\\
-0.0146025	-0.01831\\
-0.013915	-0.027465\\
-0.0140075	-0.027465\\
-0.0140075	-0.03662\\
-0.0145575	-0.03357\\
-0.0146475	-0.0305175\\
-0.01451	-0.0396725\\
-0.01474	-0.05188\\
-0.0149225	-0.05188\\
-0.0151975	-0.061035\\
-0.01538	-0.079345\\
-0.01593	-0.0732425\\
-0.0160675	-0.0671375\\
-0.0157475	-0.0457775\\
-0.015335	-0.042725\\
-0.0151525	-0.05188\\
-0.0151525	-0.05188\\
-0.015335	-0.042725\\
-0.0151525	-0.03662\\
-0.01497	-0.0457775\\
-0.0149225	-0.0305175\\
-0.01442	-0.03357\\
-0.014695	-0.05188\\
-0.0151975	-0.042725\\
-0.015015	-0.0457775\\
-0.0149225	-0.0823975\\
-0.015655	-0.0640875\\
-0.01561	-0.05188\\
-0.0154275	-0.07019\\
-0.0157925	-0.07019\\
-0.015885	-0.0488275\\
-0.01538	-0.03357\\
-0.014785	-0.03357\\
-0.0146025	-0.0488275\\
-0.01506	-0.076295\\
-0.0157925	-0.08545\\
-0.0160675	-0.079345\\
-0.01616	-0.0640875\\
-0.015975	-0.042725\\
-0.0154275	-0.03662\\
-0.01506	-0.03357\\
-0.014695	-0.027465\\
-0.01442	-0.0213625\\
-0.0142825	-0.0305175\\
-0.01442	-0.03662\\
-0.0145575	-0.027465\\
-0.0142375	-0.03357\\
-0.0143275	-0.0488275\\
-0.014785	-0.05188\\
-0.01497	-0.0549325\\
-0.01538	-0.0732425\\
-0.0158375	-0.08545\\
-0.01625	-0.0640875\\
-0.0160225	-0.0579825\\
-0.0157475	-0.061035\\
-0.0157925	-0.05188\\
-0.015655	-0.03662\\
-0.01529	-0.0457775\\
-0.0152425	-0.0732425\\
-0.0158375	-0.079345\\
-0.01616	-0.0488275\\
-0.0157025	-0.0305175\\
-0.0148775	-0.0213625\\
-0.0142825	-0.01831\\
-0.0134125	-0.024415\\
-0.0130925	-0.0213625\\
-0.01355	-0.0305175\\
-0.0137775	-0.03662\\
-0.01419	-0.03357\\
-0.014145	-0.024415\\
-0.01387	-0.027465\\
-0.013825	-0.027465\\
-0.0137325	-0.024415\\
-0.0136425	-0.027465\\
-0.0136875	-0.03662\\
-0.0141	-0.0549325\\
-0.014785	-0.0457775\\
-0.01506	-0.05188\\
-0.015105	-0.0549325\\
-0.01529	-0.076295\\
-0.0157025	-0.079345\\
-0.015885	-0.08545\\
-0.01616	-0.08545\\
-0.01625	-0.061035\\
-0.015885	-0.042725\\
-0.0154725	-0.0396725\\
-0.0151975	-0.0671375\\
-0.0157475	-0.079345\\
-0.0160675	-0.0579825\\
-0.0157025	-0.0457775\\
-0.0151975	-0.024415\\
-0.0143275	-0.0122075\\
-0.0136425	-0.01526\\
-0.0134575	-0.01526\\
-0.012955	-0.01831\\
-0.01332	-0.03662\\
-0.0137775	-0.027465\\
-0.013915	-0.0305175\\
-0.01387	-0.0213625\\
-0.0134575	-0.0122075\\
-0.0130925	-0.0213625\\
-0.01323	-0.0213625\\
-0.01332	-0.0213625\\
-0.0134125	-0.024415\\
-0.01323	-0.01831\\
-0.0130925	-0.027465\\
-0.0134125	-0.061035\\
-0.0146025	-0.0671375\\
-0.015335	-0.0640875\\
-0.015565	-0.0549325\\
-0.01529	-0.076295\\
-0.015565	-0.1007075\\
-0.01625	-0.08545\\
-0.0163875	-0.079345\\
-0.0164325	-0.0671375\\
-0.016205	-0.07019\\
-0.0160225	-0.0732425\\
-0.0161125	-0.05188\\
-0.0157025	-0.042725\\
-0.0154275	-0.03662\\
-0.0149225	-0.0305175\\
-0.0145575	-0.05188\\
-0.01506	-0.0457775\\
-0.0149225	-0.0396725\\
-0.0148325	-0.0457775\\
-0.01497	-0.0457775\\
-0.0149225	-0.03662\\
-0.0149225	-0.042725\\
-0.01497	-0.042725\\
-0.015105	-0.05188\\
-0.0152425	-0.0488275\\
-0.015105	-0.03357\\
-0.0148325	-0.042725\\
-0.015015	-0.024415\\
-0.01442	-0.0122075\\
-0.0133675	-0.027465\\
-0.01323	-0.03357\\
-0.0137325	-0.0213625\\
-0.013275	-0.0061025\\
-0.01245	-0.0213625\\
-0.0125875	-0.0305175\\
-0.013275	-0.0305175\\
-0.0136425	-0.0396725\\
-0.0139625	-0.03357\\
-0.0139625	-0.0457775\\
-0.014465	-0.042725\\
-0.0146475	-0.027465\\
-0.0142375	-0.042725\\
-0.01451	-0.03662\\
-0.0140525	-0.024415\\
-0.0136425	-0.01831\\
-0.01323	-0.01526\\
-0.01268	-0.01831\\
-0.01291	-0.0213625\\
-0.0128175	-0.01526\\
-0.013275	-0.03357\\
-0.0139625	-0.042725\\
-0.014145	-0.027465\\
-0.01387	-0.03662\\
-0.01387	-0.027465\\
-0.0136875	-0.03662\\
-0.01387	-0.027465\\
-0.0137325	-0.024415\\
-0.01355	-0.024415\\
-0.013505	-0.01831\\
-0.01323	-0.0213625\\
-0.01332	-0.027465\\
-0.013275	-0.03662\\
-0.0137325	-0.0305175\\
-0.013825	-0.0213625\\
-0.0136425	-0.0213625\\
-0.013595	-0.0213625\\
-0.0133675	-0.027465\\
-0.0133675	-0.0549325\\
-0.01419	-0.061035\\
-0.0149225	-0.0457775\\
-0.0146475	-0.042725\\
-0.014465	-0.03357\\
-0.014145	-0.0488275\\
-0.01451	-0.042725\\
-0.014695	-0.024415\\
-0.0142375	-0.03662\\
-0.0142825	-0.03357\\
-0.0141	-0.0396725\\
-0.0143275	-0.0305175\\
-0.0140075	-0.03662\\
-0.0136875	-0.0305175\\
-0.01387	-0.0396725\\
-0.014145	-0.0305175\\
-0.0140075	-0.03357\\
-0.0139625	-0.03662\\
-0.0140075	-0.0488275\\
-0.01442	-0.0457775\\
-0.01451	-0.0549325\\
-0.0149225	-0.076295\\
-0.015565	-0.061035\\
-0.0155175	-0.03662\\
-0.015015	-0.03662\\
-0.0146025	-0.0396725\\
-0.0149225	-0.0579825\\
-0.015335	-0.0457775\\
-0.015105	-0.0579825\\
-0.0151975	-0.042725\\
-0.0148775	-0.0213625\\
-0.0140075	-0.024415\\
-0.0136425	-0.0488275\\
-0.014375	-0.03662\\
-0.0145575	-0.03357\\
-0.01442	-0.05188\\
-0.0148775	-0.0549325\\
-0.0149225	-0.0457775\\
-0.0148325	-0.03662\\
-0.01451	-0.0457775\\
-0.0146025	-0.05188\\
-0.0149225	-0.0457775\\
-0.014785	-0.0396725\\
-0.014785	-0.0396725\\
-0.01474	-0.0305175\\
-0.014465	-0.03662\\
-0.01442	-0.0305175\\
-0.014145	-0.03357\\
-0.0141	-0.0488275\\
-0.0146475	-0.05188\\
-0.0149225	-0.0488275\\
-0.01506	-0.0488275\\
-0.015015	-0.03662\\
-0.014695	-0.027465\\
-0.0142825	-0.03357\\
-0.0142825	-0.03662\\
-0.01442	-0.0457775\\
-0.014695	-0.05188\\
-0.01497	-0.061035\\
-0.01529	-0.0457775\\
-0.0151525	-0.0305175\\
-0.0146025	-0.0549325\\
-0.01497	-0.076295\\
-0.015565	-0.05188\\
-0.015335	-0.05188\\
-0.0152425	-0.0640875\\
-0.01538	-0.0457775\\
-0.0152425	-0.042725\\
-0.0149225	-0.0457775\\
-0.015015	-0.0396725\\
-0.0149225	-0.0457775\\
-0.01497	-0.0549325\\
-0.0152425	-0.042725\\
-0.01506	-0.05188\\
-0.015105	-0.05188\\
-0.015105	-0.0457775\\
-0.015105	-0.03662\\
-0.01497	-0.0549325\\
-0.01506	-0.03662\\
-0.0148325	-0.042725\\
-0.01474	-0.042725\\
-0.0148775	-0.042725\\
-0.0148325	-0.027465\\
-0.0142375	-0.01526\\
-0.0134575	-0.0122075\\
-0.012955	-0.027465\\
-0.0130475	-0.05188\\
-0.014145	-0.0549325\\
-0.0148775	-0.0488275\\
-0.015015	-0.03357\\
-0.014695	-0.042725\\
-0.014695	-0.03662\\
-0.014695	-0.03662\\
-0.01451	-0.024415\\
-0.0141	-0.0396725\\
-0.0141	-0.03662\\
-0.014145	-0.0305175\\
-0.014145	-0.03357\\
-0.0142825	-0.0305175\\
-0.014145	-0.024415\\
-0.0140525	-0.01831\\
-0.0137325	-0.0213625\\
-0.0137775	-0.0457775\\
-0.014465	-0.03357\\
-0.014375	-0.0457775\\
-0.014465	-0.0396725\\
-0.014465	-0.0549325\\
-0.014695	-0.0488275\\
-0.014785	-0.0640875\\
-0.015105	-0.0396725\\
-0.015015	-0.0579825\\
-0.015015	-0.0549325\\
-0.01506	-0.03357\\
-0.01451	-0.042725\\
-0.0148325	-0.03662\\
-0.014695	-0.0488275\\
-0.014695	-0.05188\\
-0.0148325	-0.061035\\
-0.01506	-0.042725\\
-0.0148775	-0.061035\\
-0.015015	-0.0640875\\
-0.0154275	-0.0579825\\
-0.015335	-0.0457775\\
-0.015105	-0.03662\\
-0.0148775	-0.0488275\\
-0.01497	-0.042725\\
-0.0149225	-0.03662\\
-0.014695	-0.0305175\\
-0.01451	-0.03357\\
-0.0145575	-0.0305175\\
-0.01451	-0.0640875\\
-0.01506	-0.076295\\
-0.01561	-0.0640875\\
-0.015655	-0.0579825\\
-0.015565	-0.0640875\\
-0.0155175	-0.03662\\
-0.01497	-0.03357\\
-0.0146025	-0.024415\\
-0.014375	-0.024415\\
-0.014145	-0.0305175\\
-0.0141	-0.0457775\\
-0.0145575	-0.0488275\\
-0.0149225	-0.0579825\\
-0.015105	-0.0457775\\
-0.01506	-0.0396725\\
-0.014695	-0.0488275\\
-0.01506	-0.07019\\
-0.0154725	-0.07019\\
-0.015655	-0.0579825\\
-0.0155175	-0.094605\\
-0.0160675	-0.0640875\\
-0.0160675	-0.0396725\\
-0.01529	-0.03357\\
-0.01474	-0.0305175\\
-0.01442	-0.0488275\\
-0.0148325	-0.0579825\\
-0.0152425	-0.07019\\
-0.0157025	-0.061035\\
-0.0154725	-0.0457775\\
-0.0151975	-0.027465\\
-0.0145575	-0.03662\\
-0.0142375	-0.027465\\
-0.0142825	-0.03357\\
-0.014375	-0.0213625\\
-0.0140525	-0.027465\\
-0.0141	-0.0213625\\
-0.0139625	-0.01831\\
-0.013505	-0.024415\\
-0.0137775	-0.03357\\
-0.0141	-0.0396725\\
-0.01442	-0.042725\\
-0.014465	-0.042725\\
-0.014465	-0.0396725\\
-0.0145575	-0.0488275\\
-0.01474	-0.0305175\\
-0.0146025	-0.0549325\\
-0.014785	-0.0579825\\
-0.0151525	-0.05188\\
-0.01497	-0.0457775\\
-0.01497	-0.0457775\\
-0.0148325	-0.0488275\\
-0.0149225	-0.061035\\
-0.01529	-0.0488275\\
-0.0151975	-0.0396725\\
-0.0149225	-0.0457775\\
-0.01497	-0.042725\\
-0.0149225	-0.03357\\
-0.0145575	-0.05188\\
-0.01474	-0.0640875\\
-0.0151975	-0.0579825\\
-0.01529	-0.0640875\\
-0.0154275	-0.0823975\\
-0.015975	-0.061035\\
-0.0157925	-0.0579825\\
-0.0155175	-0.0488275\\
-0.0152425	-0.0579825\\
-0.015335	-0.0732425\\
-0.0157925	-0.05188\\
-0.0154725	-0.0488275\\
-0.0151975	-0.061035\\
-0.0155175	-0.042725\\
-0.0151525	-0.027465\\
-0.014465	-0.0396725\\
-0.014465	-0.0213625\\
-0.0142825	-0.0213625\\
-0.0139625	-0.0213625\\
-0.013915	-0.0213625\\
-0.013825	-0.024415\\
-0.0136425	-0.027465\\
-0.013825	-0.03662\\
-0.01419	-0.042725\\
-0.01442	-0.0488275\\
-0.01474	-0.05188\\
-0.0148775	-0.0549325\\
-0.015105	-0.0640875\\
-0.01538	-0.061035\\
-0.0154275	-0.042725\\
-0.015105	-0.05188\\
-0.01506	-0.042725\\
-0.01497	-0.0488275\\
-0.015015	-0.061035\\
-0.01538	-0.0549325\\
-0.01529	-0.0579825\\
-0.01529	-0.05188\\
-0.0151975	-0.0488275\\
-0.01506	-0.0671375\\
-0.0154725	-0.0549325\\
-0.0154725	-0.0579825\\
-0.0154275	-0.0549325\\
-0.015335	-0.03662\\
-0.0148775	-0.024415\\
-0.0142825	-0.0213625\\
-0.013825	-0.03357\\
-0.01419	-0.0305175\\
-0.0142375	-0.042725\\
-0.014465	-0.03662\\
-0.01442	-0.0457775\\
-0.0145575	-0.0640875\\
-0.015335	-0.076295\\
-0.0158375	-0.0549325\\
-0.0157475	-0.0671375\\
-0.015655	-0.0579825\\
-0.015565	-0.0457775\\
-0.0151975	-0.0488275\\
-0.015105	-0.0488275\\
-0.0152425	-0.0457775\\
-0.015105	-0.05188\\
-0.0151975	-0.0640875\\
-0.0154725	-0.0823975\\
-0.015975	-0.0732425\\
-0.0161125	-0.0732425\\
-0.015975	-0.076295\\
-0.0160675	-0.0579825\\
-0.0157475	-0.0640875\\
-0.0157025	-0.0488275\\
-0.0155175	-0.0457775\\
-0.01538	-0.061035\\
-0.015655	-0.0732425\\
-0.0158375	-0.024415\\
-0.015015	-0.0488275\\
-0.01474	-0.0305175\\
-0.014695	-0.0457775\\
-0.0148775	-0.0457775\\
-0.01497	-0.027465\\
-0.0146025	-0.0305175\\
-0.01474	-0.0579825\\
-0.01529	-0.094605\\
-0.01616	-0.0885\\
-0.0164325	-0.08545\\
-0.0163875	-0.0671375\\
-0.01616	-0.079345\\
-0.0162975	-0.0915525\\
-0.01657	-0.07019\\
-0.0162975	-0.076295\\
-0.016205	-0.079345\\
-0.01625	-0.0579825\\
-0.015885	-0.0457775\\
-0.01561	-0.061035\\
-0.0157475	-0.042725\\
-0.0154275	-0.03357\\
-0.01497	-0.0457775\\
-0.015105	-0.0305175\\
-0.0148775	-0.042725\\
-0.0148325	-0.03662\\
-0.0148775	-0.0457775\\
-0.01497	-0.05188\\
-0.015335	-0.0396725\\
-0.0151975	-0.0305175\\
-0.0149225	-0.0396725\\
-0.01497	-0.0457775\\
-0.015105	-0.05188\\
-0.015335	-0.0457775\\
-0.0151975	-0.0457775\\
-0.015015	-0.061035\\
-0.0154275	-0.0579825\\
-0.0154725	-0.0396725\\
-0.0151525	-0.0640875\\
-0.015565	-0.05188\\
-0.015655	-0.042725\\
-0.01529	-0.03662\\
-0.0148775	-0.024415\\
-0.014465	-0.01831\\
-0.0139625	-0.0396725\\
-0.01442	-0.03357\\
-0.0146025	-0.027465\\
-0.0142825	-0.01831\\
-0.013915	-0.0305175\\
-0.01387	-0.03662\\
-0.014375	-0.0396725\\
-0.014695	-0.0488275\\
-0.015015	-0.0579825\\
-0.015335	-0.0549325\\
-0.01538	-0.061035\\
-0.0154725	-0.042725\\
-0.015015	-0.0213625\\
-0.0141	-0.03662\\
-0.0140075	-0.024415\\
-0.0140525	-0.03357\\
-0.0140525	-0.042725\\
-0.0146475	-0.0457775\\
-0.0148775	-0.07019\\
-0.0154725	-0.0457775\\
-0.01529	-0.0549325\\
-0.01506	-0.0579825\\
-0.0154275	-0.0457775\\
-0.01529	-0.03662\\
-0.0148325	-0.03357\\
-0.01442	-0.042725\\
-0.01451	-0.0457775\\
-0.0148325	-0.0671375\\
-0.01538	-0.0396725\\
-0.0151975	-0.0396725\\
-0.0148325	-0.0396725\\
-0.0148325	-0.042725\\
-0.01497	-0.05188\\
-0.0152425	-0.0823975\\
-0.01593	-0.07019\\
-0.0160225	-0.07019\\
-0.015975	-0.0488275\\
-0.015565	-0.03357\\
-0.01497	-0.0396725\\
-0.0149225	-0.024415\\
-0.0142825	-0.01526\\
-0.0142825	-0.042725\\
-0.0143275	-0.042725\\
-0.0143275	-0.0488275\\
-0.0149225	-0.0640875\\
-0.0155175	-0.07019\\
-0.0158375	-0.0885\\
-0.0162975	-0.079345\\
-0.0162975	-0.0457775\\
-0.01561	-0.0457775\\
-0.0152425	-0.03357\\
-0.015105	-0.0457775\\
-0.0152425	-0.061035\\
-0.0155175	-0.0732425\\
-0.0160225	-0.0579825\\
-0.015885	-0.0457775\\
-0.01538	-0.05188\\
-0.0154275	-0.0640875\\
-0.015655	-0.0457775\\
-0.0154275	-0.03357\\
-0.015015	-0.0305175\\
-0.0146475	-0.01831\\
-0.014145	-0.0213625\\
-0.01387	-0.0122075\\
-0.01355	-0.0305175\\
-0.0137775	-0.0396725\\
-0.0142825	-0.0396725\\
-0.014465	-0.0305175\\
-0.0143275	-0.03357\\
-0.014145	-0.01831\\
-0.0134575	-0.009155\\
-0.012635	-0.01831\\
-0.0125425	-0.027465\\
-0.0130475	-0.0305175\\
-0.0136875	-0.03662\\
-0.014145	-0.0457775\\
-0.01442	-0.05188\\
-0.014785	-0.061035\\
-0.0152425	-0.08545\\
-0.0160225	-0.0823975\\
-0.016205	-0.0915525\\
-0.0163425	-0.079345\\
-0.0163425	-0.05188\\
-0.015655	-0.0457775\\
-0.01529	-0.0457775\\
-0.01529	-0.0549325\\
-0.0154725	-0.042725\\
-0.0154725	-0.0396725\\
-0.015105	-0.0305175\\
-0.014785	-0.0305175\\
-0.014375	-0.042725\\
-0.014695	-0.03662\\
-0.0148325	-0.0213625\\
-0.0142375	-0.01831\\
-0.0137325	-0.027465\\
-0.013915	-0.03662\\
-0.0142825	-0.03357\\
-0.014145	-0.03662\\
-0.014375	-0.0305175\\
-0.0143275	-0.0396725\\
-0.0146025	-0.0579825\\
-0.0152425	-0.042725\\
-0.0151525	-0.061035\\
-0.0154275	-0.0823975\\
-0.0160225	-0.0885\\
-0.0162975	-0.0671375\\
-0.01616	-0.0488275\\
-0.01561	-0.03662\\
-0.0151525	-0.024415\\
-0.0145575	-0.0396725\\
-0.0146025	-0.027465\\
-0.01442	-0.042725\\
-0.014785	-0.03662\\
-0.0149225	-0.0305175\\
-0.0146475	-0.042725\\
-0.0148775	-0.0305175\\
-0.01442	-0.0305175\\
-0.0146475	-0.03357\\
-0.01451	-0.024415\\
-0.0139625	-0.024415\\
-0.013915	-0.0213625\\
-0.0136875	-0.0213625\\
-0.0136425	-0.01831\\
-0.01355	-0.01526\\
-0.0131825	-0.01526\\
-0.01323	-0.024415\\
-0.0134575	-0.027465\\
-0.013505	-0.0213625\\
-0.013505	-0.0396725\\
-0.0139625	-0.042725\\
-0.01451	-0.03662\\
-0.01442	-0.03662\\
-0.0143275	-0.05188\\
-0.014785	-0.0640875\\
-0.0152425	-0.0457775\\
-0.0151975	-0.0549325\\
-0.0151525	-0.024415\\
-0.014465	-0.01526\\
-0.0137325	-0.03662\\
-0.0140525	-0.0457775\\
-0.014695	-0.0488275\\
-0.0148775	-0.0671375\\
-0.01538	-0.0640875\\
-0.0154725	-0.042725\\
-0.0151525	-0.03357\\
-0.014695	-0.0396725\\
-0.0145575	-0.03662\\
-0.0146475	-0.027465\\
-0.014465	-0.01831\\
-0.013915	-0.0305175\\
-0.013825	-0.01831\\
-0.0137325	-0.01526\\
-0.01355	-0.01831\\
-0.0131375	-0.024415\\
-0.0134575	-0.03357\\
-0.0139625	-0.03357\\
-0.0141	-0.024415\\
-0.013915	-0.042725\\
-0.014375	-0.0579825\\
-0.015015	-0.03662\\
-0.01474	-0.05188\\
-0.0148775	-0.0579825\\
-0.0151525	-0.0396725\\
-0.0149225	-0.027465\\
-0.0142375	-0.03662\\
-0.01419	-0.03662\\
-0.014465	-0.024415\\
-0.0140075	-0.0305175\\
-0.01387	-0.01831\\
-0.0137775	-0.01526\\
-0.01323	-0.01526\\
-0.013	-0.0213625\\
-0.0133675	-0.0305175\\
-0.0136875	-0.027465\\
-0.0136875	-0.0305175\\
-0.013825	-0.03662\\
-0.0140525	-0.024415\\
-0.01387	-0.01831\\
-0.01332	-0.01526\\
-0.013	-0.0213625\\
-0.013	-0.0213625\\
-0.0131375	-0.024415\\
-0.01332	-0.0457775\\
-0.01419	-0.0457775\\
-0.0145575	-0.03662\\
-0.014465	-0.03662\\
-0.01442	-0.0488275\\
-0.014695	-0.061035\\
-0.015105	-0.0640875\\
-0.0154275	-0.0671375\\
-0.015565	-0.0549325\\
-0.01529	-0.0549325\\
-0.0152425	-0.0549325\\
-0.015335	-0.0579825\\
-0.01529	-0.0640875\\
-0.0154275	-0.076295\\
-0.0158375	-0.0671375\\
-0.0158375	-0.061035\\
-0.0157475	-0.05188\\
-0.0155175	-0.0488275\\
-0.01538	-0.0488275\\
-0.015335	-0.061035\\
-0.0154725	-0.0396725\\
-0.01506	-0.0488275\\
-0.01497	-0.0549325\\
-0.0151975	-0.061035\\
-0.01538	-0.042725\\
-0.0152425	-0.05188\\
-0.015335	-0.0457775\\
-0.0151975	-0.042725\\
-0.015015	-0.0305175\\
-0.0145575	-0.042725\\
-0.014695	-0.061035\\
-0.01529	-0.0488275\\
-0.0152425	-0.027465\\
-0.014695	-0.03662\\
-0.014695	-0.0305175\\
-0.014145	-0.024415\\
-0.0137325	-0.027465\\
-0.013915	-0.01831\\
-0.0137325	-0.024415\\
-0.013505	-0.0213625\\
-0.01355	-0.01526\\
-0.01332	-0.01831\\
-0.01355	-0.024415\\
-0.0134575	-0.01831\\
-0.0130925	-0.0213625\\
-0.01332	-0.024415\\
-0.0134575	-0.027465\\
-0.0137775	-0.03357\\
-0.0137325	-0.0305175\\
-0.013825	-0.0396725\\
-0.0140525	-0.03357\\
-0.0141	-0.03662\\
-0.01419	-0.061035\\
-0.01497	-0.042725\\
-0.0148775	-0.0396725\\
-0.014695	-0.0457775\\
-0.014785	-0.03357\\
-0.0145575	-0.0213625\\
-0.013915	-0.0488275\\
-0.014145	-0.027465\\
-0.0141	-0.0213625\\
-0.0136425	-0.027465\\
-0.013915	-0.03357\\
-0.0140525	-0.0305175\\
-0.01419	-0.027465\\
-0.013915	-0.01526\\
-0.0134125	-0.01526\\
-0.013	-0.01526\\
-0.012955	-0.0305175\\
-0.0134125	-0.027465\\
-0.01355	-0.0305175\\
-0.0136425	-0.0396725\\
-0.0140525	-0.0488275\\
-0.0146025	-0.0396725\\
-0.0145575	-0.05188\\
-0.014785	-0.061035\\
-0.0151525	-0.05188\\
-0.015105	-0.05188\\
-0.015015	-0.079345\\
-0.015655	-0.0579825\\
-0.01561	-0.076295\\
-0.0157475	-0.061035\\
-0.0157025	-0.0671375\\
-0.0157925	-0.076295\\
-0.015975	-0.05188\\
-0.0154725	-0.027465\\
-0.01474	-0.0305175\\
-0.014375	-0.0396725\\
-0.0146025	-0.0457775\\
-0.0149225	-0.0457775\\
-0.01497	-0.0396725\\
-0.014695	-0.024415\\
-0.0139625	-0.027465\\
-0.01387	-0.0457775\\
-0.0145575	-0.061035\\
-0.0151525	-0.0579825\\
-0.015335	-0.042725\\
-0.0148325	-0.0305175\\
-0.01442	-0.03662\\
-0.0146475	-0.061035\\
-0.0151975	-0.0640875\\
-0.0155175	-0.0671375\\
-0.015655	-0.05188\\
-0.015335	-0.0671375\\
-0.0155175	-0.0732425\\
-0.0157925	-0.0671375\\
-0.0157925	-0.0885\\
-0.016205	-0.079345\\
-0.01625	-0.0671375\\
-0.0160675	-0.079345\\
-0.016205	-0.0671375\\
-0.0161125	-0.042725\\
-0.01538	-0.03662\\
-0.015015	-0.0457775\\
-0.0151975	-0.05188\\
-0.0154275	-0.0549325\\
-0.0154275	-0.0457775\\
-0.0151975	-0.05188\\
-0.015335	-0.0457775\\
-0.015335	-0.03662\\
-0.015015	-0.042725\\
-0.0151525	-0.042725\\
-0.0151525	-0.0640875\\
-0.0155175	-0.061035\\
-0.0157025	-0.05188\\
-0.0154275	-0.03662\\
-0.01506	-0.0305175\\
-0.01442	-0.03357\\
-0.01451	-0.03662\\
-0.01442	-0.024415\\
-0.01419	-0.0213625\\
-0.0140525	-0.03662\\
-0.0143275	-0.042725\\
-0.01474	-0.05188\\
-0.0149225	-0.0488275\\
-0.015015	-0.0305175\\
-0.014465	-0.0213625\\
-0.014145	-0.024415\\
-0.0137775	-0.0213625\\
-0.0137325	-0.03357\\
-0.014145	-0.03662\\
-0.0143275	-0.03357\\
-0.0143275	-0.0488275\\
-0.0148775	-0.0640875\\
-0.0152425	-0.061035\\
-0.0154725	-0.0671375\\
-0.015655	-0.076295\\
-0.015885	-0.042725\\
-0.0155175	-0.042725\\
-0.015105	-0.03662\\
-0.0148325	-0.0396725\\
-0.01474	-0.0457775\\
-0.01497	-0.0396725\\
-0.0149225	-0.027465\\
-0.01451	-0.024415\\
-0.0142825	-0.042725\\
-0.014695	-0.05188\\
-0.01506	-0.0457775\\
-0.01506	-0.0671375\\
-0.015565	-0.079345\\
-0.015885	-0.0549325\\
-0.01561	-0.03662\\
-0.015105	-0.03357\\
-0.0146475	-0.027465\\
-0.014375	-0.0457775\\
-0.0148325	-0.042725\\
-0.0148325	-0.0396725\\
-0.0149225	-0.0457775\\
-0.015105	-0.0579825\\
-0.0151975	-0.05188\\
-0.0152425	-0.061035\\
-0.015335	-0.042725\\
-0.01506	-0.0457775\\
-0.01506	-0.03662\\
-0.0148325	-0.0305175\\
-0.0146025	-0.042725\\
-0.0149225	-0.061035\\
-0.01538	-0.0640875\\
-0.01561	-0.0488275\\
-0.015105	-0.0885\\
-0.01593	-0.0671375\\
-0.01616	-0.05188\\
-0.015655	-0.03357\\
-0.01497	-0.042725\\
-0.015015	-0.042725\\
-0.01506	-0.0396725\\
-0.0149225	-0.042725\\
-0.01506	-0.03662\\
-0.01474	-0.03357\\
-0.0148775	-0.07019\\
-0.015655	-0.076295\\
-0.015975	-0.0457775\\
-0.01561	-0.0579825\\
-0.015565	-0.061035\\
-0.015655	-0.061035\\
-0.0157025	-0.0457775\\
-0.0154275	-0.0457775\\
-0.0152425	-0.0396725\\
-0.0148775	-0.0457775\\
-0.01506	-0.0549325\\
-0.0151975	-0.07019\\
-0.0157925	-0.0823975\\
-0.01616	-0.1068125\\
-0.0166175	-0.0885\\
-0.016525	-0.061035\\
-0.01616	-0.0549325\\
-0.0157925	-0.0549325\\
-0.0157475	-0.0457775\\
-0.0155175	-0.03357\\
-0.01497	-0.0305175\\
-0.014785	-0.03357\\
-0.01474	-0.0213625\\
-0.01451	-0.0213625\\
-0.0140075	-0.024415\\
-0.01419	-0.03662\\
-0.01451	-0.0305175\\
-0.0142825	-0.0213625\\
-0.013915	-0.01831\\
-0.0137775	-0.03662\\
-0.014465	-0.0579825\\
-0.01506	-0.0305175\\
-0.01451	-0.0305175\\
-0.01442	-0.042725\\
-0.01451	-0.03662\\
-0.014465	-0.042725\\
-0.0146025	-0.0396725\\
-0.0146025	-0.027465\\
-0.0142825	-0.024415\\
-0.01387	-0.01526\\
-0.013595	-0.0213625\\
-0.0136875	-0.0396725\\
-0.0142825	-0.0457775\\
-0.0146475	-0.03357\\
-0.01442	-0.024415\\
-0.0140525	-0.01831\\
-0.013595	-0.0305175\\
-0.013505	-0.03357\\
-0.0141	-0.0457775\\
-0.01451	-0.0457775\\
-0.014695	-0.03662\\
-0.01442	-0.0305175\\
-0.0143275	-0.03357\\
-0.0143275	-0.0457775\\
-0.014695	-0.0640875\\
-0.0151975	-0.07019\\
-0.015655	-0.0579825\\
-0.0154725	-0.03662\\
-0.01497	-0.03662\\
-0.0148325	-0.0396725\\
-0.014785	-0.03662\\
-0.01474	-0.0305175\\
-0.014375	-0.0305175\\
-0.0143275	-0.03662\\
-0.014465	-0.03662\\
-0.014465	-0.027465\\
-0.0141	-0.0213625\\
-0.0136425	-0.0213625\\
-0.0137325	-0.03357\\
-0.0141	-0.042725\\
-0.014695	-0.0549325\\
-0.015015	-0.0579825\\
-0.01529	-0.0579825\\
-0.01529	-0.0579825\\
-0.01538	-0.076295\\
-0.0158375	-0.1007075\\
-0.01648	-0.08545\\
-0.01648	-0.0579825\\
-0.0160225	-0.0671375\\
-0.015975	-0.0732425\\
-0.016205	-0.0915525\\
-0.016525	-0.07019\\
-0.01616	-0.03662\\
-0.0151525	-0.024415\\
-0.014465	-0.0305175\\
-0.0142375	-0.024415\\
-0.01387	-0.01526\\
-0.01323	-0.01831\\
-0.0131825	-0.024415\\
-0.0134125	-0.0305175\\
-0.01387	-0.0305175\\
-0.0139625	-0.027465\\
-0.013825	-0.024415\\
-0.0137325	-0.027465\\
-0.01387	-0.03662\\
-0.0141	-0.03662\\
-0.0142825	-0.03357\\
-0.0142825	-0.027465\\
-0.0139625	-0.01831\\
-0.0136425	-0.01831\\
-0.0134575	-0.027465\\
-0.013825	-0.0305175\\
-0.0140525	-0.024415\\
-0.01387	-0.01831\\
-0.0134575	-0.027465\\
-0.0137325	-0.03357\\
-0.013595	-0.024415\\
-0.013595	-0.03357\\
-0.013825	-0.03662\\
-0.0143275	-0.0488275\\
-0.0145575	-0.03357\\
-0.0142375	-0.0396725\\
-0.0145575	-0.042725\\
-0.014785	-0.0396725\\
-0.0145575	-0.03357\\
-0.0143275	-0.0457775\\
-0.014695	-0.0640875\\
-0.0151525	-0.0579825\\
-0.01529	-0.0640875\\
-0.0154275	-0.0549325\\
-0.0152425	-0.0488275\\
-0.015105	-0.0457775\\
-0.015015	-0.0549325\\
-0.0152425	-0.0488275\\
-0.0151525	-0.05188\\
-0.01538	-0.0579825\\
-0.0154725	-0.0579825\\
-0.0154725	-0.1007075\\
-0.0162975	-0.08545\\
-0.0163425	-0.042725\\
-0.015565	-0.0305175\\
-0.0149225	-0.03662\\
-0.01474	-0.042725\\
-0.0148775	-0.05188\\
-0.0151525	-0.0579825\\
-0.01538	-0.061035\\
-0.01561	-0.0671375\\
-0.015655	-0.0488275\\
-0.01529	-0.0396725\\
-0.015105	-0.0488275\\
-0.0151525	-0.05188\\
-0.0151975	-0.0305175\\
-0.0146475	-0.027465\\
-0.01419	-0.03357\\
-0.01451	-0.03357\\
-0.0146475	-0.042725\\
-0.0148775	-0.0579825\\
-0.01529	-0.0549325\\
-0.01529	-0.03662\\
-0.01497	-0.0305175\\
-0.014695	-0.0457775\\
-0.01497	-0.0579825\\
-0.01529	-0.0671375\\
-0.0157925	-0.079345\\
-0.0161125	-0.0915525\\
-0.0163875	-0.076295\\
-0.0162975	-0.0732425\\
-0.01616	-0.0488275\\
-0.0155175	-0.03357\\
-0.0148775	-0.0549325\\
-0.01529	-0.07019\\
-0.0157925	-0.0457775\\
-0.01529	-0.05188\\
-0.01561	-0.0885\\
-0.01616	-0.094605\\
-0.016525	-0.0640875\\
-0.0160225	-0.042725\\
-0.01529	-0.024415\\
-0.0145575	-0.0305175\\
-0.0146025	-0.03662\\
-0.0146025	-0.0305175\\
-0.0146025	-0.027465\\
-0.01419	-0.027465\\
-0.01419	-0.0305175\\
-0.0140075	-0.0213625\\
-0.014145	-0.0305175\\
-0.0140075	-0.0396725\\
-0.0141	-0.0457775\\
-0.0146475	-0.042725\\
-0.01474	-0.0488275\\
-0.01506	-0.07019\\
-0.0154725	-0.061035\\
-0.01561	-0.042725\\
-0.01506	-0.0396725\\
-0.0148775	-0.0488275\\
-0.015015	-0.03662\\
-0.01474	-0.027465\\
-0.0145575	-0.027465\\
-0.01419	-0.03357\\
-0.014465	-0.0305175\\
-0.014465	-0.024415\\
-0.0140075	-0.01526\\
-0.0133675	-0.01526\\
-0.0130925	-0.0213625\\
-0.0134575	-0.03662\\
-0.013915	-0.03357\\
-0.0141	-0.027465\\
-0.0137775	-0.0213625\\
-0.013825	-0.0396725\\
-0.0142375	-0.0457775\\
-0.0146475	-0.027465\\
-0.0145575	-0.03357\\
-0.0143275	-0.0396725\\
-0.014465	-0.042725\\
-0.014695	-0.042725\\
-0.0146475	-0.0488275\\
-0.01474	-0.0488275\\
-0.014785	-0.03357\\
-0.014465	-0.024415\\
-0.01419	-0.03662\\
-0.0143275	-0.027465\\
-0.0142825	-0.03357\\
-0.0143275	-0.0488275\\
-0.0146475	-0.0640875\\
-0.015335	-0.0549325\\
-0.0152425	-0.0457775\\
-0.015105	-0.0671375\\
-0.0155175	-0.061035\\
-0.0154275	-0.03662\\
-0.0149225	-0.027465\\
-0.01442	-0.027465\\
-0.0140525	-0.024415\\
-0.0139625	-0.0213625\\
-0.0137775	-0.01831\\
-0.0136875	-0.03662\\
-0.0142825	-0.03357\\
-0.0137325	-0.0213625\\
-0.0133675	-0.0305175\\
-0.0136425	-0.03357\\
-0.0140525	-0.024415\\
-0.0140075	-0.0213625\\
-0.0136425	-0.0122075\\
-0.0131375	-0.01831\\
-0.0133675	-0.03662\\
-0.0141	-0.0488275\\
-0.0146475	-0.042725\\
-0.0146025	-0.0305175\\
-0.0143275	-0.03662\\
-0.014375	-0.03662\\
-0.01442	-0.0396725\\
-0.0145575	-0.05188\\
-0.014785	-0.0579825\\
-0.015105	-0.0579825\\
-0.0151975	-0.0488275\\
-0.01506	-0.027465\\
-0.014695	-0.027465\\
-0.014375	-0.0305175\\
-0.0142825	-0.0396725\\
-0.0145575	-0.03662\\
-0.01419	-0.0213625\\
-0.013915	-0.042725\\
-0.01442	-0.05188\\
-0.014785	-0.0457775\\
-0.014785	-0.05188\\
-0.015015	-0.0640875\\
-0.0154725	-0.0488275\\
-0.0149225	-0.0457775\\
-0.0152425	-0.0671375\\
-0.015655	-0.0549325\\
-0.0154275	-0.061035\\
-0.015565	-0.0640875\\
-0.015565	-0.094605\\
-0.0163425	-0.0976575\\
-0.0166625	-0.0640875\\
-0.01616	-0.0823975\\
-0.01625	-0.0915525\\
-0.016525	-0.0885\\
-0.01657	-0.094605\\
-0.016755	-0.0915525\\
-0.016755	-0.1098625\\
-0.01703	-0.10376\\
-0.0169825	-0.061035\\
-0.01648	-0.0488275\\
-0.0157925	-0.0457775\\
-0.0155175	-0.0396725\\
-0.01529	-0.0305175\\
-0.0149225	-0.042725\\
-0.0152425	-0.0579825\\
-0.015655	-0.0457775\\
-0.015335	-0.03357\\
-0.01506	-0.0396725\\
-0.01497	-0.0305175\\
-0.0145575	-0.027465\\
-0.014465	-0.027465\\
-0.013915	-0.01831\\
-0.0136425	-0.0213625\\
-0.013505	-0.0396725\\
-0.0136425	-0.0213625\\
-0.01355	-0.0213625\\
-0.013275	-0.01526\\
-0.01291	-0.009155\\
-0.0125875	-0.01831\\
-0.01291	-0.0213625\\
-0.0127725	-0.01526\\
-0.012725	-0.0305175\\
-0.0134575	-0.03662\\
-0.01419	-0.03662\\
-0.014375	-0.0305175\\
-0.01419	-0.0396725\\
-0.0143275	-0.0549325\\
-0.015015	-0.042725\\
-0.01451	-0.01526\\
-0.0137775	-0.027465\\
-0.01332	-0.01526\\
-0.012955	-0.01831\\
-0.0130925	-0.03357\\
-0.0136875	-0.0396725\\
-0.0142825	-0.0396725\\
-0.0140525	-0.0305175\\
-0.01442	-0.07019\\
-0.015105	-0.0579825\\
-0.0152425	-0.042725\\
-0.01506	-0.03357\\
-0.01442	-0.03662\\
-0.014375	-0.05188\\
-0.0148325	-0.0579825\\
-0.0151525	-0.0457775\\
-0.01474	-0.01526\\
-0.0139625	-0.0305175\\
-0.0140525	-0.024415\\
-0.0139625	-0.01526\\
-0.01323	-0.009155\\
-0.012635	-0.01831\\
-0.013	-0.0488275\\
-0.01419	-0.0488275\\
-0.0145575	-0.027465\\
-0.0143275	-0.0213625\\
-0.013825	-0.0213625\\
-0.0136425	-0.024415\\
-0.01268	-0.01526\\
-0.0125425	-0.009155\\
-0.01268	-0.024415\\
-0.013	-0.03662\\
-0.013595	-0.0457775\\
-0.0142375	-0.05188\\
-0.0146025	-0.0488275\\
-0.01474	-0.05188\\
-0.0148775	-0.05188\\
-0.0149225	-0.0488275\\
-0.0149225	-0.0457775\\
-0.014785	-0.0549325\\
-0.0149225	-0.076295\\
-0.015565	-0.0671375\\
-0.0157025	-0.042725\\
-0.01506	-0.0213625\\
-0.0141	-0.0305175\\
-0.014695	-0.0671375\\
-0.015105	-0.0549325\\
-0.015105	-0.03357\\
-0.0145575	-0.0396725\\
-0.0142825	-0.05188\\
-0.0148325	-0.0732425\\
-0.015565	-0.0732425\\
-0.0158375	-0.079345\\
-0.0160225	-0.0732425\\
-0.01616	-0.0671375\\
-0.015885	-0.061035\\
-0.0157925	-0.042725\\
-0.01497	-0.03357\\
-0.01442	-0.0396725\\
-0.0146475	-0.0488275\\
-0.0148775	-0.042725\\
-0.015015	-0.0488275\\
-0.01506	-0.03357\\
-0.014695	-0.0457775\\
-0.014785	-0.042725\\
-0.0148325	-0.01831\\
-0.0142825	-0.0122075\\
-0.0136425	-0.0122075\\
-0.0131825	-0.01831\\
-0.0133675	-0.0213625\\
-0.013275	-0.009155\\
-0.0128175	-0.01831\\
-0.0130925	-0.042725\\
-0.0139625	-0.009155\\
-0.013915	-0.027465\\
-0.0136875	-0.03662\\
-0.013915	-0.05188\\
-0.0146475	-0.0457775\\
-0.0146025	-0.01831\\
-0.013825	-0.01831\\
-0.0130925	-0.01831\\
-0.0136425	-0.061035\\
-0.01474	-0.0671375\\
-0.01538	-0.08545\\
-0.0161125	-0.1068125\\
-0.0167075	-0.1007075\\
-0.0168925	-0.0732425\\
-0.0164325	-0.0732425\\
-0.01625	-0.0915525\\
-0.01657	-0.0823975\\
-0.016525	-0.0732425\\
-0.0163875	-0.076295\\
-0.01648	-0.0885\\
-0.01657	-0.0885\\
-0.0166175	-0.1159675\\
-0.01712	-0.0915525\\
-0.016845	-0.0457775\\
-0.0160675	-0.027465\\
-0.0151525	-0.024415\\
-0.014465	-0.027465\\
-0.01442	-0.0305175\\
-0.014375	-0.027465\\
-0.0143275	-0.042725\\
-0.0148325	-0.0457775\\
-0.014785	-0.03662\\
-0.01474	-0.0488275\\
-0.01506	-0.0305175\\
-0.0145575	-0.042725\\
-0.01442	-0.0457775\\
-0.014785	-0.0488275\\
-0.015015	-0.027465\\
-0.0146025	-0.01526\\
-0.0140525	-0.027465\\
-0.0141	-0.03357\\
-0.0141	-0.042725\\
-0.01506	-0.076295\\
-0.015885	-0.076295\\
-0.015975	-0.0823975\\
-0.0161125	-0.0885\\
-0.0164325	-0.112915\\
-0.0169825	-0.1007075\\
-0.017075	-0.0671375\\
-0.01648	-0.042725\\
-0.0157925	-0.03357\\
-0.0149225	-0.03357\\
-0.0145575	-0.03662\\
-0.0146025	-0.0396725\\
-0.0148775	-0.0213625\\
-0.01474	-0.0305175\\
-0.014465	-0.0305175\\
-0.0145575	-0.03662\\
-0.014695	-0.0457775\\
-0.014785	-0.0457775\\
-0.01506	-0.0396725\\
-0.0149225	-0.01831\\
-0.014145	-0.0122075\\
-0.01355	-0.01831\\
-0.0131375	-0.0122075\\
-0.013	-0.01526\\
-0.0130925	-0.027465\\
-0.013595	-0.03662\\
-0.0137775	-0.0396725\\
-0.0140075	-0.0488275\\
-0.0146025	-0.042725\\
-0.01451	-0.0488275\\
-0.01497	-0.08545\\
-0.015885	-0.0671375\\
-0.015885	-0.0549325\\
-0.0157025	-0.07019\\
-0.0158375	-0.079345\\
-0.0161125	-0.05188\\
-0.0157025	-0.0457775\\
-0.0154275	-0.03357\\
-0.0148775	-0.0396725\\
-0.014695	-0.07019\\
-0.0155175	-0.10376\\
-0.01657	-0.1159675\\
-0.017165	-0.0915525\\
-0.017075	-0.079345\\
-0.016845	-0.0671375\\
-0.0164325	-0.07019\\
-0.0162975	-0.08545\\
-0.0166175	-0.0579825\\
-0.0160675	-0.0457775\\
-0.01561	-0.0457775\\
-0.015335	-0.061035\\
-0.0158375	-0.0640875\\
-0.0160675	-0.03662\\
-0.01538	-0.03662\\
-0.01506	-0.03357\\
-0.014695	-0.042725\\
-0.01506	-0.0671375\\
-0.015655	-0.0640875\\
-0.0157925	-0.05188\\
-0.015565	-0.0457775\\
-0.0151975	-0.042725\\
-0.0151975	-0.03662\\
-0.01506	-0.0213625\\
-0.0146475	-0.0213625\\
-0.0142825	-0.01831\\
-0.0140075	-0.01831\\
-0.0136425	-0.01526\\
-0.01355	-0.027465\\
-0.0140525	-0.042725\\
-0.0145575	-0.0457775\\
-0.014695	-0.03357\\
-0.0143275	-0.024415\\
-0.0141	-0.0213625\\
-0.01387	-0.027465\\
-0.0140075	-0.03662\\
-0.0143275	-0.042725\\
-0.014465	-0.03662\\
-0.014465	-0.024415\\
-0.0140075	-0.05188\\
-0.014465	-0.0549325\\
-0.0151525	-0.0396725\\
-0.0149225	-0.0122075\\
-0.0140075	-0.027465\\
-0.0136875	-0.03357\\
-0.01419	-0.03662\\
-0.014375	-0.024415\\
-0.0142825	-0.01831\\
-0.013915	-0.03357\\
-0.0142375	-0.0549325\\
-0.0148325	-0.03357\\
-0.0145575	-0.009155\\
-0.013595	-0.024415\\
-0.0134125	-0.03662\\
-0.01419	-0.05188\\
-0.014785	-0.03662\\
-0.014375	-0.0305175\\
-0.0140525	-0.05188\\
-0.0146475	-0.0671375\\
-0.0152425	-0.0579825\\
-0.01529	-0.0732425\\
-0.0157025	-0.0885\\
-0.016205	-0.061035\\
-0.0158375	-0.0488275\\
-0.0154725	-0.0671375\\
-0.0157475	-0.0488275\\
-0.0155175	-0.0579825\\
-0.015655	-0.0732425\\
-0.0160225	-0.0549325\\
-0.015885	-0.027465\\
-0.01506	-0.05188\\
-0.0152425	-0.08545\\
-0.0161125	-0.094605\\
-0.01657	-0.0732425\\
-0.0163875	-0.061035\\
-0.0161125	-0.079345\\
-0.0162975	-0.0671375\\
-0.0161125	-0.042725\\
-0.015565	-0.042725\\
-0.01529	-0.0488275\\
-0.01538	-0.0549325\\
-0.0155175	-0.0488275\\
-0.0155175	-0.0549325\\
-0.01561	-0.03662\\
-0.0154275	-0.0488275\\
-0.01529	-0.061035\\
-0.015655	-0.0457775\\
-0.01529	-0.03662\\
-0.0148325	-0.0305175\\
-0.0146025	-0.0213625\\
-0.01419	-0.024415\\
-0.0141	-0.042725\\
-0.0145575	-0.0396725\\
-0.01474	-0.0488275\\
-0.01497	-0.0640875\\
-0.01538	-0.0671375\\
-0.0157475	-0.05188\\
-0.01561	-0.0457775\\
-0.015335	-0.027465\\
-0.0145575	-0.03662\\
-0.01451	-0.061035\\
-0.01529	-0.0457775\\
-0.015105	-0.0213625\\
-0.01442	-0.0396725\\
-0.01474	-0.0640875\\
-0.01538	-0.0579825\\
-0.01561	-0.076295\\
-0.0160675	-0.0976575\\
-0.0166625	-0.076295\\
-0.01648	-0.0671375\\
-0.0161125	-0.0732425\\
-0.0162975	-0.05188\\
-0.015885	-0.03662\\
-0.0154725	-0.042725\\
-0.0154725	-0.061035\\
-0.0157025	-0.061035\\
-0.0158375	-0.0640875\\
-0.0158375	-0.03357\\
-0.0154725	-0.03662\\
-0.01506	-0.0305175\\
-0.014785	-0.042725\\
-0.0149225	-0.03662\\
-0.0149225	-0.0305175\\
-0.014695	-0.05188\\
-0.01529	-0.061035\\
-0.01561	-0.0396725\\
-0.0152425	-0.0305175\\
-0.015015	-0.05188\\
-0.0152425	-0.03357\\
-0.0148775	-0.024415\\
-0.0141	-0.027465\\
-0.0140525	-0.0305175\\
-0.0143275	-0.027465\\
-0.0142825	-0.01526\\
-0.013915	-0.01526\\
-0.01332	-0.027465\\
-0.0134125	-0.0305175\\
-0.013915	-0.05188\\
-0.0145575	-0.0488275\\
-0.01497	-0.0549325\\
-0.0154275	-0.07019\\
-0.0158375	-0.03662\\
-0.015335	-0.03662\\
-0.014785	-0.0640875\\
-0.0154725	-0.0915525\\
-0.01625	-0.0671375\\
-0.016205	-0.0640875\\
-0.0160675	-0.0976575\\
-0.01648	-0.079345\\
-0.01648	-0.0640875\\
-0.01616	-0.05188\\
-0.0157925	-0.05188\\
-0.015655	-0.0640875\\
-0.015975	-0.0457775\\
-0.0157025	-0.042725\\
-0.01538	-0.07019\\
-0.0157925	-0.079345\\
-0.01625	-0.079345\\
-0.0163875	-0.0457775\\
-0.015565	-0.0213625\\
-0.0145575	-0.0396725\\
-0.0151525	-0.0396725\\
-0.01529	-0.0213625\\
-0.014375	-0.01831\\
-0.0137775	-0.027465\\
-0.0140075	-0.03357\\
-0.0145575	-0.0579825\\
-0.0151975	-0.042725\\
-0.0151525	-0.027465\\
-0.01474	-0.024415\\
-0.01419	-0.01831\\
-0.013595	-0.0213625\\
-0.01355	-0.01831\\
-0.013505	-0.01831\\
-0.0136425	-0.03662\\
-0.0142375	-0.0305175\\
-0.0143275	-0.01831\\
-0.0136875	-0.0213625\\
-0.0134125	-0.024415\\
-0.01355	-0.027465\\
-0.0137325	-0.0213625\\
-0.0134575	-0.024415\\
-0.013595	-0.0213625\\
-0.013505	-0.0122075\\
-0.0131375	-0.03357\\
-0.0137325	-0.042725\\
-0.0142825	-0.0579825\\
-0.015105	-0.076295\\
-0.0158375	-0.0671375\\
-0.01593	-0.03357\\
-0.0151525	-0.027465\\
-0.0145575	-0.0305175\\
-0.014375	-0.01831\\
-0.01387	-0.024415\\
-0.0134125	-0.027465\\
-0.0140075	-0.03662\\
-0.0142375	-0.03357\\
-0.01419	-0.03357\\
-0.0142825	-0.0396725\\
-0.014465	-0.0457775\\
-0.0146025	-0.042725\\
-0.014695	-0.0549325\\
-0.01497	-0.0549325\\
-0.0151525	-0.076295\\
-0.01561	-0.0488275\\
-0.015105	-0.024415\\
-0.01419	-0.0213625\\
-0.01387	-0.0396725\\
-0.0142825	-0.027465\\
-0.0143275	-0.024415\\
-0.0141	-0.0122075\\
-0.0134575	-0.027465\\
-0.0134125	-0.01831\\
-0.013275	-0.009155\\
-0.012635	-0.01526\\
-0.01268	-0.027465\\
-0.01332	-0.03357\\
-0.0137325	-0.0396725\\
-0.014375	-0.061035\\
-0.015105	-0.042725\\
-0.0149225	-0.01831\\
-0.0139625	-0.03357\\
-0.0141	-0.0457775\\
-0.014465	-0.027465\\
-0.013825	-0.03357\\
-0.01387	-0.05188\\
-0.0146025	-0.042725\\
-0.014785	-0.0640875\\
-0.0152425	-0.076295\\
-0.0158375	-0.05188\\
-0.0154725	-0.042725\\
-0.015105	-0.0396725\\
-0.01497	-0.05188\\
-0.0151525	-0.07019\\
-0.015565	-0.05188\\
-0.01538	-0.0305175\\
-0.0148775	-0.0457775\\
};
\addplot [color=mycolor2, line width=2.0pt, forget plot]
  table[row sep=crcr]{%
-0.01561	-0.01561\\
-0.015655	-0.015655\\
-0.0157925	-0.0157925\\
-0.015655	-0.015655\\
-0.0149225	-0.0149225\\
-0.0148325	-0.0148325\\
-0.0148775	-0.0148775\\
-0.0149225	-0.0149225\\
-0.0148775	-0.0148775\\
-0.0140075	-0.0140075\\
-0.0131825	-0.0131825\\
-0.0128625	-0.0128625\\
-0.0137775	-0.0137775\\
-0.014785	-0.014785\\
-0.01506	-0.01506\\
-0.0151525	-0.0151525\\
-0.0148775	-0.0148775\\
-0.0143275	-0.0143275\\
-0.01497	-0.01497\\
-0.0148325	-0.0148325\\
-0.014375	-0.014375\\
-0.0148325	-0.0148325\\
-0.0148775	-0.0148775\\
-0.014695	-0.014695\\
-0.015105	-0.015105\\
-0.01529	-0.01529\\
-0.01497	-0.01497\\
-0.01529	-0.01529\\
-0.01538	-0.01538\\
-0.0151525	-0.0151525\\
-0.0149225	-0.0149225\\
-0.0146475	-0.0146475\\
-0.014695	-0.014695\\
-0.0149225	-0.0149225\\
-0.01497	-0.01497\\
-0.01506	-0.01506\\
-0.0154725	-0.0154725\\
-0.0155175	-0.0155175\\
-0.015105	-0.015105\\
-0.0149225	-0.0149225\\
-0.014375	-0.014375\\
-0.0141	-0.0141\\
-0.014375	-0.014375\\
-0.0142375	-0.0142375\\
-0.014145	-0.014145\\
-0.014465	-0.014465\\
-0.014695	-0.014695\\
-0.0151525	-0.0151525\\
-0.015105	-0.015105\\
-0.01451	-0.01451\\
-0.014145	-0.014145\\
-0.0142825	-0.0142825\\
-0.01442	-0.01442\\
-0.0139625	-0.0139625\\
-0.0137325	-0.0137325\\
-0.0140075	-0.0140075\\
-0.0139625	-0.0139625\\
-0.0136875	-0.0136875\\
-0.01387	-0.01387\\
-0.0140525	-0.0140525\\
-0.0145575	-0.0145575\\
-0.01474	-0.01474\\
-0.0148775	-0.0148775\\
-0.0149225	-0.0149225\\
-0.0151525	-0.0151525\\
-0.01506	-0.01506\\
-0.0148775	-0.0148775\\
-0.014785	-0.014785\\
-0.01474	-0.01474\\
-0.014695	-0.014695\\
-0.0152425	-0.0152425\\
-0.015975	-0.015975\\
-0.01625	-0.01625\\
-0.0163425	-0.0163425\\
-0.0158375	-0.0158375\\
-0.0160675	-0.0160675\\
-0.01648	-0.01648\\
-0.0166175	-0.0166175\\
-0.0163875	-0.0163875\\
-0.0166175	-0.0166175\\
-0.017165	-0.017165\\
-0.0169375	-0.0169375\\
-0.0166625	-0.0166625\\
-0.0161125	-0.0161125\\
-0.01561	-0.01561\\
-0.01529	-0.01529\\
-0.015335	-0.015335\\
-0.0151525	-0.0151525\\
-0.01451	-0.01451\\
-0.01419	-0.01419\\
-0.0142825	-0.0142825\\
-0.014695	-0.014695\\
-0.01474	-0.01474\\
-0.015015	-0.015015\\
-0.0151975	-0.0151975\\
-0.015565	-0.015565\\
-0.0157025	-0.0157025\\
-0.015565	-0.015565\\
-0.01529	-0.01529\\
-0.01538	-0.01538\\
-0.0151525	-0.0151525\\
-0.0149225	-0.0149225\\
-0.01506	-0.01506\\
-0.014785	-0.014785\\
-0.014695	-0.014695\\
-0.01442	-0.01442\\
-0.014375	-0.014375\\
-0.0141	-0.0141\\
-0.014145	-0.014145\\
-0.014375	-0.014375\\
-0.0148775	-0.0148775\\
-0.015105	-0.015105\\
-0.0152425	-0.0152425\\
-0.0148325	-0.0148325\\
-0.015015	-0.015015\\
-0.01561	-0.01561\\
-0.015565	-0.015565\\
-0.01561	-0.01561\\
-0.0157475	-0.0157475\\
-0.0151525	-0.0151525\\
-0.0146475	-0.0146475\\
-0.01474	-0.01474\\
-0.0149225	-0.0149225\\
-0.015565	-0.015565\\
-0.0160225	-0.0160225\\
-0.01616	-0.01616\\
-0.01593	-0.01593\\
-0.0158375	-0.0158375\\
-0.0157025	-0.0157025\\
-0.015655	-0.015655\\
-0.01529	-0.01529\\
-0.01497	-0.01497\\
-0.014785	-0.014785\\
-0.0146025	-0.0146025\\
-0.0146475	-0.0146475\\
-0.01497	-0.01497\\
-0.015565	-0.015565\\
-0.0155175	-0.0155175\\
-0.015105	-0.015105\\
-0.0148775	-0.0148775\\
-0.01474	-0.01474\\
-0.0142375	-0.0142375\\
-0.0136875	-0.0136875\\
-0.013595	-0.013595\\
-0.0141	-0.0141\\
-0.0140525	-0.0140525\\
-0.0141	-0.0141\\
-0.013825	-0.013825\\
-0.0134575	-0.0134575\\
-0.0130475	-0.0130475\\
-0.013275	-0.013275\\
-0.0142375	-0.0142375\\
-0.015015	-0.015015\\
-0.015335	-0.015335\\
-0.01506	-0.01506\\
-0.0145575	-0.0145575\\
-0.0141	-0.0141\\
-0.013595	-0.013595\\
-0.014145	-0.014145\\
-0.0142825	-0.0142825\\
-0.0146025	-0.0146025\\
-0.01506	-0.01506\\
-0.01593	-0.01593\\
-0.016525	-0.016525\\
-0.01648	-0.01648\\
-0.0164325	-0.0164325\\
-0.015885	-0.015885\\
-0.01593	-0.01593\\
-0.0160225	-0.0160225\\
-0.0158375	-0.0158375\\
-0.01561	-0.01561\\
-0.015565	-0.015565\\
-0.0154275	-0.0154275\\
-0.0155175	-0.0155175\\
-0.01529	-0.01529\\
-0.0157025	-0.0157025\\
-0.0161125	-0.0161125\\
-0.0158375	-0.0158375\\
-0.0152425	-0.0152425\\
-0.01506	-0.01506\\
-0.01561	-0.01561\\
-0.0160225	-0.0160225\\
-0.0163875	-0.0163875\\
-0.0166175	-0.0166175\\
-0.0167075	-0.0167075\\
-0.0164325	-0.0164325\\
-0.01625	-0.01625\\
-0.0163425	-0.0163425\\
-0.01657	-0.01657\\
-0.01616	-0.01616\\
-0.0154275	-0.0154275\\
-0.015565	-0.015565\\
-0.0154725	-0.0154725\\
-0.01497	-0.01497\\
-0.0151525	-0.0151525\\
-0.01497	-0.01497\\
-0.014695	-0.014695\\
-0.01474	-0.01474\\
-0.014695	-0.014695\\
-0.0148325	-0.0148325\\
-0.01529	-0.01529\\
-0.015335	-0.015335\\
-0.015015	-0.015015\\
-0.0148775	-0.0148775\\
-0.0152425	-0.0152425\\
-0.0161125	-0.0161125\\
-0.015975	-0.015975\\
-0.01538	-0.01538\\
-0.0149225	-0.0149225\\
-0.01497	-0.01497\\
-0.0148775	-0.0148775\\
-0.0146025	-0.0146025\\
-0.01451	-0.01451\\
-0.0146475	-0.0146475\\
-0.0148775	-0.0148775\\
-0.015105	-0.015105\\
-0.014785	-0.014785\\
-0.0151525	-0.0151525\\
-0.0155175	-0.0155175\\
-0.0154725	-0.0154725\\
-0.015335	-0.015335\\
-0.01529	-0.01529\\
-0.015655	-0.015655\\
-0.015335	-0.015335\\
-0.014375	-0.014375\\
-0.0142375	-0.0142375\\
-0.01442	-0.01442\\
-0.01474	-0.01474\\
-0.014465	-0.014465\\
-0.014785	-0.014785\\
-0.0148775	-0.0148775\\
-0.0146025	-0.0146025\\
-0.0146475	-0.0146475\\
-0.014465	-0.014465\\
-0.0145575	-0.0145575\\
-0.01419	-0.01419\\
-0.0141	-0.0141\\
-0.01387	-0.01387\\
-0.013825	-0.013825\\
-0.0142825	-0.0142825\\
-0.0146025	-0.0146025\\
-0.015015	-0.015015\\
-0.01561	-0.01561\\
-0.01538	-0.01538\\
-0.014695	-0.014695\\
-0.0142825	-0.0142825\\
-0.0141	-0.0141\\
-0.014375	-0.014375\\
-0.0140525	-0.0140525\\
-0.0139625	-0.0139625\\
-0.01419	-0.01419\\
-0.0149225	-0.0149225\\
-0.01506	-0.01506\\
-0.01538	-0.01538\\
-0.0151525	-0.0151525\\
-0.0149225	-0.0149225\\
-0.0142825	-0.0142825\\
-0.013915	-0.013915\\
-0.01355	-0.01355\\
-0.0133675	-0.0133675\\
-0.013505	-0.013505\\
-0.0139625	-0.0139625\\
-0.0143275	-0.0143275\\
-0.01442	-0.01442\\
-0.0145575	-0.0145575\\
-0.0152425	-0.0152425\\
-0.01538	-0.01538\\
-0.01506	-0.01506\\
-0.0151525	-0.0151525\\
-0.0152425	-0.0152425\\
-0.01561	-0.01561\\
-0.0154725	-0.0154725\\
-0.015015	-0.015015\\
-0.01419	-0.01419\\
-0.01355	-0.01355\\
-0.0134125	-0.0134125\\
-0.0134575	-0.0134575\\
-0.013505	-0.013505\\
-0.0134575	-0.0134575\\
-0.0136425	-0.0136425\\
-0.01419	-0.01419\\
-0.0142825	-0.0142825\\
-0.0143275	-0.0143275\\
-0.014465	-0.014465\\
-0.0141	-0.0141\\
-0.0134125	-0.0134125\\
-0.0137775	-0.0137775\\
-0.01442	-0.01442\\
-0.0145575	-0.0145575\\
-0.014695	-0.014695\\
-0.0146475	-0.0146475\\
-0.014375	-0.014375\\
-0.0146025	-0.0146025\\
-0.015015	-0.015015\\
-0.0151975	-0.0151975\\
-0.0148325	-0.0148325\\
-0.01529	-0.01529\\
-0.015335	-0.015335\\
-0.0151525	-0.0151525\\
-0.015335	-0.015335\\
-0.01538	-0.01538\\
-0.0151525	-0.0151525\\
-0.0149225	-0.0149225\\
-0.0154275	-0.0154275\\
-0.0160225	-0.0160225\\
-0.01593	-0.01593\\
-0.01529	-0.01529\\
-0.01506	-0.01506\\
-0.015015	-0.015015\\
-0.0151975	-0.0151975\\
-0.0154275	-0.0154275\\
-0.0151975	-0.0151975\\
-0.0151525	-0.0151525\\
-0.0154275	-0.0154275\\
-0.01561	-0.01561\\
-0.01529	-0.01529\\
-0.0149225	-0.0149225\\
-0.01506	-0.01506\\
-0.0157475	-0.0157475\\
-0.0157925	-0.0157925\\
-0.0157475	-0.0157475\\
-0.015335	-0.015335\\
-0.01497	-0.01497\\
-0.0148775	-0.0148775\\
-0.014465	-0.014465\\
-0.013825	-0.013825\\
-0.0133675	-0.0133675\\
-0.0130925	-0.0130925\\
-0.0136425	-0.0136425\\
-0.01419	-0.01419\\
-0.0146025	-0.0146025\\
-0.01442	-0.01442\\
-0.0146025	-0.0146025\\
-0.01419	-0.01419\\
-0.0140525	-0.0140525\\
-0.0137775	-0.0137775\\
-0.013915	-0.013915\\
-0.014465	-0.014465\\
-0.0148775	-0.0148775\\
-0.0145575	-0.0145575\\
-0.014145	-0.014145\\
-0.0139625	-0.0139625\\
-0.0136875	-0.0136875\\
-0.0142825	-0.0142825\\
-0.0152425	-0.0152425\\
-0.0151975	-0.0151975\\
-0.0154725	-0.0154725\\
-0.0158375	-0.0158375\\
-0.015975	-0.015975\\
-0.0157475	-0.0157475\\
-0.015335	-0.015335\\
-0.0151975	-0.0151975\\
-0.015335	-0.015335\\
-0.0155175	-0.0155175\\
-0.015565	-0.015565\\
-0.01593	-0.01593\\
-0.0162975	-0.0162975\\
-0.0166175	-0.0166175\\
-0.0163425	-0.0163425\\
-0.01625	-0.01625\\
-0.0164325	-0.0164325\\
-0.01625	-0.01625\\
-0.0162975	-0.0162975\\
-0.016205	-0.016205\\
-0.0155175	-0.0155175\\
-0.014785	-0.014785\\
-0.0145575	-0.0145575\\
-0.0148325	-0.0148325\\
-0.01474	-0.01474\\
-0.0143275	-0.0143275\\
-0.014465	-0.014465\\
-0.0142825	-0.0142825\\
-0.013915	-0.013915\\
-0.0139625	-0.0139625\\
-0.0140525	-0.0140525\\
-0.01387	-0.01387\\
-0.0142375	-0.0142375\\
-0.01474	-0.01474\\
-0.0145575	-0.0145575\\
-0.0151525	-0.0151525\\
-0.01593	-0.01593\\
-0.015975	-0.015975\\
-0.01625	-0.01625\\
-0.0160675	-0.0160675\\
-0.015975	-0.015975\\
-0.015655	-0.015655\\
-0.01497	-0.01497\\
-0.0149225	-0.0149225\\
-0.01474	-0.01474\\
-0.01442	-0.01442\\
-0.01451	-0.01451\\
-0.0140525	-0.0140525\\
-0.013505	-0.013505\\
-0.0133675	-0.0133675\\
-0.0141	-0.0141\\
-0.0146025	-0.0146025\\
-0.0149225	-0.0149225\\
-0.01506	-0.01506\\
-0.0151975	-0.0151975\\
-0.015105	-0.015105\\
-0.0148325	-0.0148325\\
-0.0146475	-0.0146475\\
-0.01497	-0.01497\\
-0.01474	-0.01474\\
-0.014465	-0.014465\\
-0.014695	-0.014695\\
-0.0151525	-0.0151525\\
-0.015015	-0.015015\\
-0.014695	-0.014695\\
-0.01451	-0.01451\\
-0.0145575	-0.0145575\\
-0.0142825	-0.0142825\\
-0.0140525	-0.0140525\\
-0.0143275	-0.0143275\\
-0.0145575	-0.0145575\\
-0.014695	-0.014695\\
-0.0145575	-0.0145575\\
-0.01474	-0.01474\\
-0.014785	-0.014785\\
-0.0142825	-0.0142825\\
-0.014145	-0.014145\\
-0.014785	-0.014785\\
-0.01538	-0.01538\\
-0.0154725	-0.0154725\\
-0.015335	-0.015335\\
-0.0152425	-0.0152425\\
-0.01506	-0.01506\\
-0.0148775	-0.0148775\\
-0.014695	-0.014695\\
-0.0148325	-0.0148325\\
-0.014465	-0.014465\\
-0.0146475	-0.0146475\\
-0.0148325	-0.0148325\\
-0.015015	-0.015015\\
-0.0151975	-0.0151975\\
-0.0152425	-0.0152425\\
-0.01561	-0.01561\\
-0.0161125	-0.0161125\\
-0.01648	-0.01648\\
-0.015975	-0.015975\\
-0.01529	-0.01529\\
-0.0146025	-0.0146025\\
-0.013915	-0.013915\\
-0.0140075	-0.0140075\\
-0.0145575	-0.0145575\\
-0.0146475	-0.0146475\\
-0.01451	-0.01451\\
-0.01474	-0.01474\\
-0.0149225	-0.0149225\\
-0.0151975	-0.0151975\\
-0.01538	-0.01538\\
-0.01593	-0.01593\\
-0.0160675	-0.0160675\\
-0.0157475	-0.0157475\\
-0.015335	-0.015335\\
-0.0151525	-0.0151525\\
-0.015335	-0.015335\\
-0.0151525	-0.0151525\\
-0.01497	-0.01497\\
-0.0149225	-0.0149225\\
-0.01442	-0.01442\\
-0.014695	-0.014695\\
-0.0151975	-0.0151975\\
-0.015015	-0.015015\\
-0.0149225	-0.0149225\\
-0.015655	-0.015655\\
-0.01561	-0.01561\\
-0.0154275	-0.0154275\\
-0.0157925	-0.0157925\\
-0.015885	-0.015885\\
-0.01538	-0.01538\\
-0.014785	-0.014785\\
-0.0146025	-0.0146025\\
-0.01506	-0.01506\\
-0.0157925	-0.0157925\\
-0.0160675	-0.0160675\\
-0.01616	-0.01616\\
-0.015975	-0.015975\\
-0.0154275	-0.0154275\\
-0.01506	-0.01506\\
-0.014695	-0.014695\\
-0.01442	-0.01442\\
-0.0142825	-0.0142825\\
-0.01442	-0.01442\\
-0.0145575	-0.0145575\\
-0.0142375	-0.0142375\\
-0.0143275	-0.0143275\\
-0.014785	-0.014785\\
-0.01497	-0.01497\\
-0.01538	-0.01538\\
-0.0158375	-0.0158375\\
-0.01625	-0.01625\\
-0.0160225	-0.0160225\\
-0.0157475	-0.0157475\\
-0.0157925	-0.0157925\\
-0.015655	-0.015655\\
-0.01529	-0.01529\\
-0.0152425	-0.0152425\\
-0.0158375	-0.0158375\\
-0.01616	-0.01616\\
-0.0157025	-0.0157025\\
-0.0148775	-0.0148775\\
-0.0142825	-0.0142825\\
-0.0134125	-0.0134125\\
-0.0130925	-0.0130925\\
-0.01355	-0.01355\\
-0.0137775	-0.0137775\\
-0.01419	-0.01419\\
-0.014145	-0.014145\\
-0.01387	-0.01387\\
-0.013825	-0.013825\\
-0.0137325	-0.0137325\\
-0.0136425	-0.0136425\\
-0.0136875	-0.0136875\\
-0.0141	-0.0141\\
-0.014785	-0.014785\\
-0.01506	-0.01506\\
-0.015105	-0.015105\\
-0.01529	-0.01529\\
-0.0157025	-0.0157025\\
-0.015885	-0.015885\\
-0.01616	-0.01616\\
-0.01625	-0.01625\\
-0.015885	-0.015885\\
-0.0154725	-0.0154725\\
-0.0151975	-0.0151975\\
-0.0157475	-0.0157475\\
-0.0160675	-0.0160675\\
-0.0157025	-0.0157025\\
-0.0151975	-0.0151975\\
-0.0143275	-0.0143275\\
-0.0136425	-0.0136425\\
-0.0134575	-0.0134575\\
-0.012955	-0.012955\\
-0.01332	-0.01332\\
-0.0137775	-0.0137775\\
-0.013915	-0.013915\\
-0.01387	-0.01387\\
-0.0134575	-0.0134575\\
-0.0130925	-0.0130925\\
-0.01323	-0.01323\\
-0.01332	-0.01332\\
-0.0134125	-0.0134125\\
-0.01323	-0.01323\\
-0.0130925	-0.0130925\\
-0.0134125	-0.0134125\\
-0.0146025	-0.0146025\\
-0.015335	-0.015335\\
-0.015565	-0.015565\\
-0.01529	-0.01529\\
-0.015565	-0.015565\\
-0.01625	-0.01625\\
-0.0163875	-0.0163875\\
-0.0164325	-0.0164325\\
-0.016205	-0.016205\\
-0.0160225	-0.0160225\\
-0.0161125	-0.0161125\\
-0.0157025	-0.0157025\\
-0.0154275	-0.0154275\\
-0.0149225	-0.0149225\\
-0.0145575	-0.0145575\\
-0.01506	-0.01506\\
-0.0149225	-0.0149225\\
-0.0148325	-0.0148325\\
-0.01497	-0.01497\\
-0.0149225	-0.0149225\\
-0.01497	-0.01497\\
-0.015105	-0.015105\\
-0.0152425	-0.0152425\\
-0.015105	-0.015105\\
-0.0148325	-0.0148325\\
-0.015015	-0.015015\\
-0.01442	-0.01442\\
-0.0133675	-0.0133675\\
-0.01323	-0.01323\\
-0.0137325	-0.0137325\\
-0.013275	-0.013275\\
-0.01245	-0.01245\\
-0.0125875	-0.0125875\\
-0.013275	-0.013275\\
-0.0136425	-0.0136425\\
-0.0139625	-0.0139625\\
-0.014465	-0.014465\\
-0.0146475	-0.0146475\\
-0.0142375	-0.0142375\\
-0.01451	-0.01451\\
-0.0140525	-0.0140525\\
-0.0136425	-0.0136425\\
-0.01323	-0.01323\\
-0.01268	-0.01268\\
-0.01291	-0.01291\\
-0.0128175	-0.0128175\\
-0.013275	-0.013275\\
-0.0139625	-0.0139625\\
-0.014145	-0.014145\\
-0.01387	-0.01387\\
-0.0136875	-0.0136875\\
-0.01387	-0.01387\\
-0.0137325	-0.0137325\\
-0.01355	-0.01355\\
-0.013505	-0.013505\\
-0.01323	-0.01323\\
-0.01332	-0.01332\\
-0.013275	-0.013275\\
-0.0137325	-0.0137325\\
-0.013825	-0.013825\\
-0.0136425	-0.0136425\\
-0.013595	-0.013595\\
-0.0133675	-0.0133675\\
-0.01419	-0.01419\\
-0.0149225	-0.0149225\\
-0.0146475	-0.0146475\\
-0.014465	-0.014465\\
-0.014145	-0.014145\\
-0.01451	-0.01451\\
-0.014695	-0.014695\\
-0.0142375	-0.0142375\\
-0.0142825	-0.0142825\\
-0.0141	-0.0141\\
-0.0143275	-0.0143275\\
-0.0140075	-0.0140075\\
-0.0136875	-0.0136875\\
-0.01387	-0.01387\\
-0.014145	-0.014145\\
-0.0140075	-0.0140075\\
-0.0139625	-0.0139625\\
-0.0140075	-0.0140075\\
-0.01442	-0.01442\\
-0.01451	-0.01451\\
-0.0149225	-0.0149225\\
-0.015565	-0.015565\\
-0.0155175	-0.0155175\\
-0.015015	-0.015015\\
-0.0146025	-0.0146025\\
-0.0149225	-0.0149225\\
-0.015335	-0.015335\\
-0.015105	-0.015105\\
-0.0151975	-0.0151975\\
-0.0148775	-0.0148775\\
-0.0140075	-0.0140075\\
-0.0136425	-0.0136425\\
-0.014375	-0.014375\\
-0.0145575	-0.0145575\\
-0.01442	-0.01442\\
-0.0148775	-0.0148775\\
-0.0149225	-0.0149225\\
-0.0148325	-0.0148325\\
-0.01451	-0.01451\\
-0.0146025	-0.0146025\\
-0.0149225	-0.0149225\\
-0.014785	-0.014785\\
-0.01474	-0.01474\\
-0.014465	-0.014465\\
-0.01442	-0.01442\\
-0.014145	-0.014145\\
-0.0141	-0.0141\\
-0.0146475	-0.0146475\\
-0.0149225	-0.0149225\\
-0.01506	-0.01506\\
-0.015015	-0.015015\\
-0.014695	-0.014695\\
-0.0142825	-0.0142825\\
-0.01442	-0.01442\\
-0.014695	-0.014695\\
-0.01497	-0.01497\\
-0.01529	-0.01529\\
-0.0151525	-0.0151525\\
-0.0146025	-0.0146025\\
-0.01497	-0.01497\\
-0.015565	-0.015565\\
-0.015335	-0.015335\\
-0.0152425	-0.0152425\\
-0.01538	-0.01538\\
-0.0152425	-0.0152425\\
-0.0149225	-0.0149225\\
-0.015015	-0.015015\\
-0.0149225	-0.0149225\\
-0.01497	-0.01497\\
-0.0152425	-0.0152425\\
-0.01506	-0.01506\\
-0.015105	-0.015105\\
-0.01497	-0.01497\\
-0.01506	-0.01506\\
-0.0148325	-0.0148325\\
-0.01474	-0.01474\\
-0.0148775	-0.0148775\\
-0.0148325	-0.0148325\\
-0.0142375	-0.0142375\\
-0.0134575	-0.0134575\\
-0.012955	-0.012955\\
-0.0130475	-0.0130475\\
-0.014145	-0.014145\\
-0.0148775	-0.0148775\\
-0.015015	-0.015015\\
-0.014695	-0.014695\\
-0.01451	-0.01451\\
-0.0141	-0.0141\\
-0.014145	-0.014145\\
-0.0142825	-0.0142825\\
-0.014145	-0.014145\\
-0.0140525	-0.0140525\\
-0.0137325	-0.0137325\\
-0.0137775	-0.0137775\\
-0.014465	-0.014465\\
-0.014375	-0.014375\\
-0.014465	-0.014465\\
-0.014695	-0.014695\\
-0.014785	-0.014785\\
-0.015105	-0.015105\\
-0.015015	-0.015015\\
-0.01506	-0.01506\\
-0.01451	-0.01451\\
-0.0148325	-0.0148325\\
-0.014695	-0.014695\\
-0.0148325	-0.0148325\\
-0.01506	-0.01506\\
-0.0148775	-0.0148775\\
-0.015015	-0.015015\\
-0.0154275	-0.0154275\\
-0.015335	-0.015335\\
-0.015105	-0.015105\\
-0.0148775	-0.0148775\\
-0.01497	-0.01497\\
-0.0149225	-0.0149225\\
-0.014695	-0.014695\\
-0.01451	-0.01451\\
-0.0145575	-0.0145575\\
-0.01451	-0.01451\\
-0.01506	-0.01506\\
-0.01561	-0.01561\\
-0.015655	-0.015655\\
-0.015565	-0.015565\\
-0.0155175	-0.0155175\\
-0.01497	-0.01497\\
-0.0146025	-0.0146025\\
-0.014375	-0.014375\\
-0.014145	-0.014145\\
-0.0141	-0.0141\\
-0.0145575	-0.0145575\\
-0.0149225	-0.0149225\\
-0.015105	-0.015105\\
-0.01506	-0.01506\\
-0.014695	-0.014695\\
-0.01506	-0.01506\\
-0.0154725	-0.0154725\\
-0.015655	-0.015655\\
-0.0155175	-0.0155175\\
-0.0160675	-0.0160675\\
-0.01529	-0.01529\\
-0.01474	-0.01474\\
-0.01442	-0.01442\\
-0.0148325	-0.0148325\\
-0.0152425	-0.0152425\\
-0.0157025	-0.0157025\\
-0.0154725	-0.0154725\\
-0.0151975	-0.0151975\\
-0.0145575	-0.0145575\\
-0.0142375	-0.0142375\\
-0.0142825	-0.0142825\\
-0.014375	-0.014375\\
-0.0140525	-0.0140525\\
-0.0141	-0.0141\\
-0.0139625	-0.0139625\\
-0.013505	-0.013505\\
-0.0137775	-0.0137775\\
-0.0141	-0.0141\\
-0.01442	-0.01442\\
-0.014465	-0.014465\\
-0.0145575	-0.0145575\\
-0.01474	-0.01474\\
-0.0146025	-0.0146025\\
-0.014785	-0.014785\\
-0.0151525	-0.0151525\\
-0.01497	-0.01497\\
-0.0148325	-0.0148325\\
-0.0149225	-0.0149225\\
-0.01529	-0.01529\\
-0.0151975	-0.0151975\\
-0.0149225	-0.0149225\\
-0.01497	-0.01497\\
-0.0149225	-0.0149225\\
-0.0145575	-0.0145575\\
-0.01474	-0.01474\\
-0.0151975	-0.0151975\\
-0.01529	-0.01529\\
-0.0154275	-0.0154275\\
-0.015975	-0.015975\\
-0.0157925	-0.0157925\\
-0.0155175	-0.0155175\\
-0.0152425	-0.0152425\\
-0.015335	-0.015335\\
-0.0157925	-0.0157925\\
-0.0154725	-0.0154725\\
-0.0151975	-0.0151975\\
-0.0155175	-0.0155175\\
-0.0151525	-0.0151525\\
-0.014465	-0.014465\\
-0.0142825	-0.0142825\\
-0.0139625	-0.0139625\\
-0.013915	-0.013915\\
-0.013825	-0.013825\\
-0.0136425	-0.0136425\\
-0.013825	-0.013825\\
-0.01419	-0.01419\\
-0.01442	-0.01442\\
-0.01474	-0.01474\\
-0.0148775	-0.0148775\\
-0.015105	-0.015105\\
-0.01538	-0.01538\\
-0.0154275	-0.0154275\\
-0.015105	-0.015105\\
-0.01506	-0.01506\\
-0.01497	-0.01497\\
-0.015015	-0.015015\\
-0.01538	-0.01538\\
-0.01529	-0.01529\\
-0.0151975	-0.0151975\\
-0.01506	-0.01506\\
-0.0154725	-0.0154725\\
-0.0154275	-0.0154275\\
-0.015335	-0.015335\\
-0.0148775	-0.0148775\\
-0.0142825	-0.0142825\\
-0.013825	-0.013825\\
-0.01419	-0.01419\\
-0.0142375	-0.0142375\\
-0.014465	-0.014465\\
-0.01442	-0.01442\\
-0.0145575	-0.0145575\\
-0.015335	-0.015335\\
-0.0158375	-0.0158375\\
-0.0157475	-0.0157475\\
-0.015655	-0.015655\\
-0.015565	-0.015565\\
-0.0151975	-0.0151975\\
-0.015105	-0.015105\\
-0.0152425	-0.0152425\\
-0.015105	-0.015105\\
-0.0151975	-0.0151975\\
-0.0154725	-0.0154725\\
-0.015975	-0.015975\\
-0.0161125	-0.0161125\\
-0.015975	-0.015975\\
-0.0160675	-0.0160675\\
-0.0157475	-0.0157475\\
-0.0157025	-0.0157025\\
-0.0155175	-0.0155175\\
-0.01538	-0.01538\\
-0.015655	-0.015655\\
-0.0158375	-0.0158375\\
-0.015015	-0.015015\\
-0.01474	-0.01474\\
-0.014695	-0.014695\\
-0.0148775	-0.0148775\\
-0.01497	-0.01497\\
-0.0146025	-0.0146025\\
-0.01474	-0.01474\\
-0.01529	-0.01529\\
-0.01616	-0.01616\\
-0.0164325	-0.0164325\\
-0.0163875	-0.0163875\\
-0.01616	-0.01616\\
-0.0162975	-0.0162975\\
-0.01657	-0.01657\\
-0.0162975	-0.0162975\\
-0.016205	-0.016205\\
-0.01625	-0.01625\\
-0.015885	-0.015885\\
-0.01561	-0.01561\\
-0.0157475	-0.0157475\\
-0.0154275	-0.0154275\\
-0.01497	-0.01497\\
-0.015105	-0.015105\\
-0.0148775	-0.0148775\\
-0.0148325	-0.0148325\\
-0.0148775	-0.0148775\\
-0.01497	-0.01497\\
-0.015335	-0.015335\\
-0.0151975	-0.0151975\\
-0.0149225	-0.0149225\\
-0.01497	-0.01497\\
-0.015105	-0.015105\\
-0.015335	-0.015335\\
-0.0151975	-0.0151975\\
-0.015015	-0.015015\\
-0.0154275	-0.0154275\\
-0.0154725	-0.0154725\\
-0.0151525	-0.0151525\\
-0.015565	-0.015565\\
-0.015655	-0.015655\\
-0.01529	-0.01529\\
-0.0148775	-0.0148775\\
-0.014465	-0.014465\\
-0.0139625	-0.0139625\\
-0.01442	-0.01442\\
-0.0146025	-0.0146025\\
-0.0142825	-0.0142825\\
-0.013915	-0.013915\\
-0.01387	-0.01387\\
-0.014375	-0.014375\\
-0.014695	-0.014695\\
-0.015015	-0.015015\\
-0.015335	-0.015335\\
-0.01538	-0.01538\\
-0.0154725	-0.0154725\\
-0.015015	-0.015015\\
-0.0141	-0.0141\\
-0.0140075	-0.0140075\\
-0.0140525	-0.0140525\\
-0.0146475	-0.0146475\\
-0.0148775	-0.0148775\\
-0.0154725	-0.0154725\\
-0.01529	-0.01529\\
-0.01506	-0.01506\\
-0.0154275	-0.0154275\\
-0.01529	-0.01529\\
-0.0148325	-0.0148325\\
-0.01442	-0.01442\\
-0.01451	-0.01451\\
-0.0148325	-0.0148325\\
-0.01538	-0.01538\\
-0.0151975	-0.0151975\\
-0.0148325	-0.0148325\\
-0.01497	-0.01497\\
-0.0152425	-0.0152425\\
-0.01593	-0.01593\\
-0.0160225	-0.0160225\\
-0.015975	-0.015975\\
-0.015565	-0.015565\\
-0.01497	-0.01497\\
-0.0149225	-0.0149225\\
-0.0142825	-0.0142825\\
-0.0143275	-0.0143275\\
-0.0149225	-0.0149225\\
-0.0155175	-0.0155175\\
-0.0158375	-0.0158375\\
-0.0162975	-0.0162975\\
-0.01561	-0.01561\\
-0.0152425	-0.0152425\\
-0.015105	-0.015105\\
-0.0152425	-0.0152425\\
-0.0155175	-0.0155175\\
-0.0160225	-0.0160225\\
-0.015885	-0.015885\\
-0.01538	-0.01538\\
-0.0154275	-0.0154275\\
-0.015655	-0.015655\\
-0.0154275	-0.0154275\\
-0.015015	-0.015015\\
-0.0146475	-0.0146475\\
-0.014145	-0.014145\\
-0.01387	-0.01387\\
-0.01355	-0.01355\\
-0.0137775	-0.0137775\\
-0.0142825	-0.0142825\\
-0.014465	-0.014465\\
-0.0143275	-0.0143275\\
-0.014145	-0.014145\\
-0.0134575	-0.0134575\\
-0.012635	-0.012635\\
-0.0125425	-0.0125425\\
-0.0130475	-0.0130475\\
-0.0136875	-0.0136875\\
-0.014145	-0.014145\\
-0.01442	-0.01442\\
-0.014785	-0.014785\\
-0.0152425	-0.0152425\\
-0.0160225	-0.0160225\\
-0.016205	-0.016205\\
-0.0163425	-0.0163425\\
-0.015655	-0.015655\\
-0.01529	-0.01529\\
-0.0154725	-0.0154725\\
-0.015105	-0.015105\\
-0.014785	-0.014785\\
-0.014375	-0.014375\\
-0.014695	-0.014695\\
-0.0148325	-0.0148325\\
-0.0142375	-0.0142375\\
-0.0137325	-0.0137325\\
-0.013915	-0.013915\\
-0.0142825	-0.0142825\\
-0.014145	-0.014145\\
-0.014375	-0.014375\\
-0.0143275	-0.0143275\\
-0.0146025	-0.0146025\\
-0.0152425	-0.0152425\\
-0.0151525	-0.0151525\\
-0.0154275	-0.0154275\\
-0.0160225	-0.0160225\\
-0.0162975	-0.0162975\\
-0.01616	-0.01616\\
-0.01561	-0.01561\\
-0.0151525	-0.0151525\\
-0.0145575	-0.0145575\\
-0.0146025	-0.0146025\\
-0.01442	-0.01442\\
-0.014785	-0.014785\\
-0.0149225	-0.0149225\\
-0.0146475	-0.0146475\\
-0.0148775	-0.0148775\\
-0.01442	-0.01442\\
-0.0146475	-0.0146475\\
-0.01451	-0.01451\\
-0.0139625	-0.0139625\\
-0.013915	-0.013915\\
-0.0136875	-0.0136875\\
-0.0136425	-0.0136425\\
-0.01355	-0.01355\\
-0.0131825	-0.0131825\\
-0.01323	-0.01323\\
-0.0134575	-0.0134575\\
-0.013505	-0.013505\\
-0.0139625	-0.0139625\\
-0.01451	-0.01451\\
-0.01442	-0.01442\\
-0.0143275	-0.0143275\\
-0.014785	-0.014785\\
-0.0152425	-0.0152425\\
-0.0151975	-0.0151975\\
-0.0151525	-0.0151525\\
-0.014465	-0.014465\\
-0.0137325	-0.0137325\\
-0.0140525	-0.0140525\\
-0.014695	-0.014695\\
-0.0148775	-0.0148775\\
-0.01538	-0.01538\\
-0.0154725	-0.0154725\\
-0.0151525	-0.0151525\\
-0.014695	-0.014695\\
-0.0145575	-0.0145575\\
-0.0146475	-0.0146475\\
-0.014465	-0.014465\\
-0.013915	-0.013915\\
-0.013825	-0.013825\\
-0.0137325	-0.0137325\\
-0.01355	-0.01355\\
-0.0131375	-0.0131375\\
-0.0134575	-0.0134575\\
-0.0139625	-0.0139625\\
-0.0141	-0.0141\\
-0.013915	-0.013915\\
-0.014375	-0.014375\\
-0.015015	-0.015015\\
-0.01474	-0.01474\\
-0.0148775	-0.0148775\\
-0.0151525	-0.0151525\\
-0.0149225	-0.0149225\\
-0.0142375	-0.0142375\\
-0.01419	-0.01419\\
-0.014465	-0.014465\\
-0.0140075	-0.0140075\\
-0.01387	-0.01387\\
-0.0137775	-0.0137775\\
-0.01323	-0.01323\\
-0.013	-0.013\\
-0.0133675	-0.0133675\\
-0.0136875	-0.0136875\\
-0.013825	-0.013825\\
-0.0140525	-0.0140525\\
-0.01387	-0.01387\\
-0.01332	-0.01332\\
-0.013	-0.013\\
-0.0131375	-0.0131375\\
-0.01332	-0.01332\\
-0.01419	-0.01419\\
-0.0145575	-0.0145575\\
-0.014465	-0.014465\\
-0.01442	-0.01442\\
-0.014695	-0.014695\\
-0.015105	-0.015105\\
-0.0154275	-0.0154275\\
-0.015565	-0.015565\\
-0.01529	-0.01529\\
-0.0152425	-0.0152425\\
-0.015335	-0.015335\\
-0.01529	-0.01529\\
-0.0154275	-0.0154275\\
-0.0158375	-0.0158375\\
-0.0157475	-0.0157475\\
-0.0155175	-0.0155175\\
-0.01538	-0.01538\\
-0.015335	-0.015335\\
-0.0154725	-0.0154725\\
-0.01506	-0.01506\\
-0.01497	-0.01497\\
-0.0151975	-0.0151975\\
-0.01538	-0.01538\\
-0.0152425	-0.0152425\\
-0.015335	-0.015335\\
-0.0151975	-0.0151975\\
-0.015015	-0.015015\\
-0.0145575	-0.0145575\\
-0.014695	-0.014695\\
-0.01529	-0.01529\\
-0.0152425	-0.0152425\\
-0.014695	-0.014695\\
-0.014145	-0.014145\\
-0.0137325	-0.0137325\\
-0.013915	-0.013915\\
-0.0137325	-0.0137325\\
-0.013505	-0.013505\\
-0.01355	-0.01355\\
-0.01332	-0.01332\\
-0.01355	-0.01355\\
-0.0134575	-0.0134575\\
-0.0130925	-0.0130925\\
-0.01332	-0.01332\\
-0.0134575	-0.0134575\\
-0.0137775	-0.0137775\\
-0.0137325	-0.0137325\\
-0.013825	-0.013825\\
-0.0140525	-0.0140525\\
-0.0141	-0.0141\\
-0.01419	-0.01419\\
-0.01497	-0.01497\\
-0.0148775	-0.0148775\\
-0.014695	-0.014695\\
-0.014785	-0.014785\\
-0.0145575	-0.0145575\\
-0.013915	-0.013915\\
-0.014145	-0.014145\\
-0.0141	-0.0141\\
-0.0136425	-0.0136425\\
-0.013915	-0.013915\\
-0.0140525	-0.0140525\\
-0.01419	-0.01419\\
-0.013915	-0.013915\\
-0.0134125	-0.0134125\\
-0.013	-0.013\\
-0.012955	-0.012955\\
-0.0134125	-0.0134125\\
-0.01355	-0.01355\\
-0.0136425	-0.0136425\\
-0.0140525	-0.0140525\\
-0.0146025	-0.0146025\\
-0.0145575	-0.0145575\\
-0.014785	-0.014785\\
-0.0151525	-0.0151525\\
-0.015105	-0.015105\\
-0.015015	-0.015015\\
-0.015655	-0.015655\\
-0.01561	-0.01561\\
-0.0157475	-0.0157475\\
-0.0157025	-0.0157025\\
-0.0157925	-0.0157925\\
-0.015975	-0.015975\\
-0.0154725	-0.0154725\\
-0.01474	-0.01474\\
-0.014375	-0.014375\\
-0.0146025	-0.0146025\\
-0.0149225	-0.0149225\\
-0.01497	-0.01497\\
-0.014695	-0.014695\\
-0.0139625	-0.0139625\\
-0.01387	-0.01387\\
-0.0145575	-0.0145575\\
-0.0151525	-0.0151525\\
-0.015335	-0.015335\\
-0.0148325	-0.0148325\\
-0.01442	-0.01442\\
-0.0146475	-0.0146475\\
-0.0151975	-0.0151975\\
-0.0155175	-0.0155175\\
-0.015655	-0.015655\\
-0.015335	-0.015335\\
-0.0155175	-0.0155175\\
-0.0157925	-0.0157925\\
-0.016205	-0.016205\\
-0.01625	-0.01625\\
-0.0160675	-0.0160675\\
-0.016205	-0.016205\\
-0.0161125	-0.0161125\\
-0.01538	-0.01538\\
-0.015015	-0.015015\\
-0.0151975	-0.0151975\\
-0.0154275	-0.0154275\\
-0.0151975	-0.0151975\\
-0.015335	-0.015335\\
-0.015015	-0.015015\\
-0.0151525	-0.0151525\\
-0.0155175	-0.0155175\\
-0.0157025	-0.0157025\\
-0.0154275	-0.0154275\\
-0.01506	-0.01506\\
-0.01442	-0.01442\\
-0.01451	-0.01451\\
-0.01442	-0.01442\\
-0.01419	-0.01419\\
-0.0140525	-0.0140525\\
-0.0143275	-0.0143275\\
-0.01474	-0.01474\\
-0.0149225	-0.0149225\\
-0.015015	-0.015015\\
-0.014465	-0.014465\\
-0.014145	-0.014145\\
-0.0137775	-0.0137775\\
-0.0137325	-0.0137325\\
-0.014145	-0.014145\\
-0.0143275	-0.0143275\\
-0.0148775	-0.0148775\\
-0.0152425	-0.0152425\\
-0.0154725	-0.0154725\\
-0.015655	-0.015655\\
-0.015885	-0.015885\\
-0.0155175	-0.0155175\\
-0.015105	-0.015105\\
-0.0148325	-0.0148325\\
-0.01474	-0.01474\\
-0.01497	-0.01497\\
-0.0149225	-0.0149225\\
-0.01451	-0.01451\\
-0.0142825	-0.0142825\\
-0.014695	-0.014695\\
-0.01506	-0.01506\\
-0.015565	-0.015565\\
-0.015885	-0.015885\\
-0.01561	-0.01561\\
-0.015105	-0.015105\\
-0.0146475	-0.0146475\\
-0.014375	-0.014375\\
-0.0148325	-0.0148325\\
-0.0149225	-0.0149225\\
-0.015105	-0.015105\\
-0.0151975	-0.0151975\\
-0.0152425	-0.0152425\\
-0.015335	-0.015335\\
-0.01506	-0.01506\\
-0.0148325	-0.0148325\\
-0.0146025	-0.0146025\\
-0.0149225	-0.0149225\\
-0.01538	-0.01538\\
-0.01561	-0.01561\\
-0.015105	-0.015105\\
-0.01593	-0.01593\\
-0.01616	-0.01616\\
-0.015655	-0.015655\\
-0.01497	-0.01497\\
-0.015015	-0.015015\\
-0.01506	-0.01506\\
-0.0149225	-0.0149225\\
-0.01506	-0.01506\\
-0.01474	-0.01474\\
-0.0148775	-0.0148775\\
-0.015655	-0.015655\\
-0.015975	-0.015975\\
-0.01561	-0.01561\\
-0.015565	-0.015565\\
-0.015655	-0.015655\\
-0.0157025	-0.0157025\\
-0.0154275	-0.0154275\\
-0.0152425	-0.0152425\\
-0.0148775	-0.0148775\\
-0.01506	-0.01506\\
-0.0151975	-0.0151975\\
-0.0157925	-0.0157925\\
-0.01616	-0.01616\\
-0.0166175	-0.0166175\\
-0.016525	-0.016525\\
-0.01616	-0.01616\\
-0.0157925	-0.0157925\\
-0.0157475	-0.0157475\\
-0.0155175	-0.0155175\\
-0.01497	-0.01497\\
-0.014785	-0.014785\\
-0.01474	-0.01474\\
-0.01451	-0.01451\\
-0.0140075	-0.0140075\\
-0.01419	-0.01419\\
-0.01451	-0.01451\\
-0.0142825	-0.0142825\\
-0.013915	-0.013915\\
-0.0137775	-0.0137775\\
-0.014465	-0.014465\\
-0.01506	-0.01506\\
-0.01451	-0.01451\\
-0.01442	-0.01442\\
-0.01451	-0.01451\\
-0.014465	-0.014465\\
-0.0146025	-0.0146025\\
-0.0142825	-0.0142825\\
-0.01387	-0.01387\\
-0.013595	-0.013595\\
-0.0136875	-0.0136875\\
-0.0142825	-0.0142825\\
-0.0146475	-0.0146475\\
-0.01442	-0.01442\\
-0.0140525	-0.0140525\\
-0.013595	-0.013595\\
-0.013505	-0.013505\\
-0.0141	-0.0141\\
-0.01451	-0.01451\\
-0.014695	-0.014695\\
-0.01442	-0.01442\\
-0.0143275	-0.0143275\\
-0.014695	-0.014695\\
-0.0151975	-0.0151975\\
-0.015655	-0.015655\\
-0.0154725	-0.0154725\\
-0.01497	-0.01497\\
-0.0148325	-0.0148325\\
-0.014785	-0.014785\\
-0.01474	-0.01474\\
-0.014375	-0.014375\\
-0.0143275	-0.0143275\\
-0.014465	-0.014465\\
-0.0141	-0.0141\\
-0.0136425	-0.0136425\\
-0.0137325	-0.0137325\\
-0.0141	-0.0141\\
-0.014695	-0.014695\\
-0.015015	-0.015015\\
-0.01529	-0.01529\\
-0.01538	-0.01538\\
-0.0158375	-0.0158375\\
-0.01648	-0.01648\\
-0.0160225	-0.0160225\\
-0.015975	-0.015975\\
-0.016205	-0.016205\\
-0.016525	-0.016525\\
-0.01616	-0.01616\\
-0.0151525	-0.0151525\\
-0.014465	-0.014465\\
-0.0142375	-0.0142375\\
-0.01387	-0.01387\\
-0.01323	-0.01323\\
-0.0131825	-0.0131825\\
-0.0134125	-0.0134125\\
-0.01387	-0.01387\\
-0.0139625	-0.0139625\\
-0.013825	-0.013825\\
-0.0137325	-0.0137325\\
-0.01387	-0.01387\\
-0.0141	-0.0141\\
-0.0142825	-0.0142825\\
-0.0139625	-0.0139625\\
-0.0136425	-0.0136425\\
-0.0134575	-0.0134575\\
-0.013825	-0.013825\\
-0.0140525	-0.0140525\\
-0.01387	-0.01387\\
-0.0134575	-0.0134575\\
-0.0137325	-0.0137325\\
-0.013595	-0.013595\\
-0.013825	-0.013825\\
-0.0143275	-0.0143275\\
-0.0145575	-0.0145575\\
-0.0142375	-0.0142375\\
-0.0145575	-0.0145575\\
-0.014785	-0.014785\\
-0.0145575	-0.0145575\\
-0.0143275	-0.0143275\\
-0.014695	-0.014695\\
-0.0151525	-0.0151525\\
-0.01529	-0.01529\\
-0.0154275	-0.0154275\\
-0.0152425	-0.0152425\\
-0.015105	-0.015105\\
-0.015015	-0.015015\\
-0.0152425	-0.0152425\\
-0.0151525	-0.0151525\\
-0.01538	-0.01538\\
-0.0154725	-0.0154725\\
-0.0162975	-0.0162975\\
-0.0163425	-0.0163425\\
-0.015565	-0.015565\\
-0.0149225	-0.0149225\\
-0.01474	-0.01474\\
-0.0148775	-0.0148775\\
-0.0151525	-0.0151525\\
-0.01538	-0.01538\\
-0.01561	-0.01561\\
-0.015655	-0.015655\\
-0.01529	-0.01529\\
-0.015105	-0.015105\\
-0.0151525	-0.0151525\\
-0.0151975	-0.0151975\\
-0.0146475	-0.0146475\\
-0.01419	-0.01419\\
-0.01451	-0.01451\\
-0.0146475	-0.0146475\\
-0.0148775	-0.0148775\\
-0.01529	-0.01529\\
-0.01497	-0.01497\\
-0.014695	-0.014695\\
-0.01497	-0.01497\\
-0.01529	-0.01529\\
-0.0157925	-0.0157925\\
-0.0161125	-0.0161125\\
-0.0163875	-0.0163875\\
-0.0162975	-0.0162975\\
-0.01616	-0.01616\\
-0.0155175	-0.0155175\\
-0.0148775	-0.0148775\\
-0.01529	-0.01529\\
-0.0157925	-0.0157925\\
-0.01529	-0.01529\\
-0.01561	-0.01561\\
-0.01616	-0.01616\\
-0.016525	-0.016525\\
-0.0160225	-0.0160225\\
-0.01529	-0.01529\\
-0.0145575	-0.0145575\\
-0.0146025	-0.0146025\\
-0.01419	-0.01419\\
-0.0140075	-0.0140075\\
-0.014145	-0.014145\\
-0.0140075	-0.0140075\\
-0.0141	-0.0141\\
-0.0146475	-0.0146475\\
-0.01474	-0.01474\\
-0.01506	-0.01506\\
-0.0154725	-0.0154725\\
-0.01561	-0.01561\\
-0.01506	-0.01506\\
-0.0148775	-0.0148775\\
-0.015015	-0.015015\\
-0.01474	-0.01474\\
-0.0145575	-0.0145575\\
-0.01419	-0.01419\\
-0.014465	-0.014465\\
-0.0140075	-0.0140075\\
-0.0133675	-0.0133675\\
-0.0130925	-0.0130925\\
-0.0134575	-0.0134575\\
-0.013915	-0.013915\\
-0.0141	-0.0141\\
-0.0137775	-0.0137775\\
-0.013825	-0.013825\\
-0.0142375	-0.0142375\\
-0.0146475	-0.0146475\\
-0.0145575	-0.0145575\\
-0.0143275	-0.0143275\\
-0.014465	-0.014465\\
-0.014695	-0.014695\\
-0.0146475	-0.0146475\\
-0.01474	-0.01474\\
-0.014785	-0.014785\\
-0.014465	-0.014465\\
-0.01419	-0.01419\\
-0.0143275	-0.0143275\\
-0.0142825	-0.0142825\\
-0.0143275	-0.0143275\\
-0.0146475	-0.0146475\\
-0.015335	-0.015335\\
-0.0152425	-0.0152425\\
-0.015105	-0.015105\\
-0.0155175	-0.0155175\\
-0.0154275	-0.0154275\\
-0.0149225	-0.0149225\\
-0.01442	-0.01442\\
-0.0140525	-0.0140525\\
-0.0139625	-0.0139625\\
-0.0137775	-0.0137775\\
-0.0136875	-0.0136875\\
-0.0142825	-0.0142825\\
-0.0137325	-0.0137325\\
-0.0133675	-0.0133675\\
-0.0136425	-0.0136425\\
-0.0140525	-0.0140525\\
-0.0140075	-0.0140075\\
-0.0136425	-0.0136425\\
-0.0131375	-0.0131375\\
-0.0133675	-0.0133675\\
-0.0141	-0.0141\\
-0.0146475	-0.0146475\\
-0.0146025	-0.0146025\\
-0.0143275	-0.0143275\\
-0.014375	-0.014375\\
-0.01442	-0.01442\\
-0.0145575	-0.0145575\\
-0.014785	-0.014785\\
-0.015105	-0.015105\\
-0.0151975	-0.0151975\\
-0.01506	-0.01506\\
-0.014695	-0.014695\\
-0.014375	-0.014375\\
-0.0142825	-0.0142825\\
-0.0145575	-0.0145575\\
-0.01419	-0.01419\\
-0.013915	-0.013915\\
-0.01442	-0.01442\\
-0.014785	-0.014785\\
-0.015015	-0.015015\\
-0.0154725	-0.0154725\\
-0.0149225	-0.0149225\\
-0.0152425	-0.0152425\\
-0.015655	-0.015655\\
-0.0154275	-0.0154275\\
-0.015565	-0.015565\\
-0.0163425	-0.0163425\\
-0.0166625	-0.0166625\\
-0.01616	-0.01616\\
-0.01625	-0.01625\\
-0.016525	-0.016525\\
-0.01657	-0.01657\\
-0.016755	-0.016755\\
-0.01703	-0.01703\\
-0.0169825	-0.0169825\\
-0.01648	-0.01648\\
-0.0157925	-0.0157925\\
-0.0155175	-0.0155175\\
-0.01529	-0.01529\\
-0.0149225	-0.0149225\\
-0.0152425	-0.0152425\\
-0.015655	-0.015655\\
-0.015335	-0.015335\\
-0.01506	-0.01506\\
-0.01497	-0.01497\\
-0.0145575	-0.0145575\\
-0.014465	-0.014465\\
-0.013915	-0.013915\\
-0.0136425	-0.0136425\\
-0.013505	-0.013505\\
-0.0136425	-0.0136425\\
-0.01355	-0.01355\\
-0.013275	-0.013275\\
-0.01291	-0.01291\\
-0.0125875	-0.0125875\\
-0.01291	-0.01291\\
-0.0127725	-0.0127725\\
-0.012725	-0.012725\\
-0.0134575	-0.0134575\\
-0.01419	-0.01419\\
-0.014375	-0.014375\\
-0.01419	-0.01419\\
-0.0143275	-0.0143275\\
-0.015015	-0.015015\\
-0.01451	-0.01451\\
-0.0137775	-0.0137775\\
-0.01332	-0.01332\\
-0.012955	-0.012955\\
-0.0130925	-0.0130925\\
-0.0136875	-0.0136875\\
-0.0142825	-0.0142825\\
-0.0140525	-0.0140525\\
-0.01442	-0.01442\\
-0.015105	-0.015105\\
-0.0152425	-0.0152425\\
-0.01506	-0.01506\\
-0.01442	-0.01442\\
-0.014375	-0.014375\\
-0.0148325	-0.0148325\\
-0.0151525	-0.0151525\\
-0.01474	-0.01474\\
-0.0139625	-0.0139625\\
-0.0140525	-0.0140525\\
-0.0139625	-0.0139625\\
-0.01323	-0.01323\\
-0.012635	-0.012635\\
-0.013	-0.013\\
-0.01419	-0.01419\\
-0.0145575	-0.0145575\\
-0.0143275	-0.0143275\\
-0.013825	-0.013825\\
-0.0136425	-0.0136425\\
-0.01268	-0.01268\\
-0.0125425	-0.0125425\\
-0.01268	-0.01268\\
-0.013	-0.013\\
-0.013595	-0.013595\\
-0.0142375	-0.0142375\\
-0.0146025	-0.0146025\\
-0.01474	-0.01474\\
-0.0148775	-0.0148775\\
-0.0149225	-0.0149225\\
-0.014785	-0.014785\\
-0.0149225	-0.0149225\\
-0.015565	-0.015565\\
-0.0157025	-0.0157025\\
-0.01506	-0.01506\\
-0.0141	-0.0141\\
-0.014695	-0.014695\\
-0.015105	-0.015105\\
-0.0145575	-0.0145575\\
-0.0142825	-0.0142825\\
-0.0148325	-0.0148325\\
-0.015565	-0.015565\\
-0.0158375	-0.0158375\\
-0.0160225	-0.0160225\\
-0.01616	-0.01616\\
-0.015885	-0.015885\\
-0.0157925	-0.0157925\\
-0.01497	-0.01497\\
-0.01442	-0.01442\\
-0.0146475	-0.0146475\\
-0.0148775	-0.0148775\\
-0.015015	-0.015015\\
-0.01506	-0.01506\\
-0.014695	-0.014695\\
-0.014785	-0.014785\\
-0.0148325	-0.0148325\\
-0.0142825	-0.0142825\\
-0.0136425	-0.0136425\\
-0.0131825	-0.0131825\\
-0.0133675	-0.0133675\\
-0.013275	-0.013275\\
-0.0128175	-0.0128175\\
-0.0130925	-0.0130925\\
-0.0139625	-0.0139625\\
-0.013915	-0.013915\\
-0.0136875	-0.0136875\\
-0.013915	-0.013915\\
-0.0146475	-0.0146475\\
-0.0146025	-0.0146025\\
-0.013825	-0.013825\\
-0.0130925	-0.0130925\\
-0.0136425	-0.0136425\\
-0.01474	-0.01474\\
-0.01538	-0.01538\\
-0.0161125	-0.0161125\\
-0.0167075	-0.0167075\\
-0.0168925	-0.0168925\\
-0.0164325	-0.0164325\\
-0.01625	-0.01625\\
-0.01657	-0.01657\\
-0.016525	-0.016525\\
-0.0163875	-0.0163875\\
-0.01648	-0.01648\\
-0.01657	-0.01657\\
-0.0166175	-0.0166175\\
-0.01712	-0.01712\\
-0.016845	-0.016845\\
-0.0160675	-0.0160675\\
-0.0151525	-0.0151525\\
-0.014465	-0.014465\\
-0.01442	-0.01442\\
-0.014375	-0.014375\\
-0.0143275	-0.0143275\\
-0.0148325	-0.0148325\\
-0.014785	-0.014785\\
-0.01474	-0.01474\\
-0.01506	-0.01506\\
-0.0145575	-0.0145575\\
-0.01442	-0.01442\\
-0.014785	-0.014785\\
-0.015015	-0.015015\\
-0.0146025	-0.0146025\\
-0.0140525	-0.0140525\\
-0.0141	-0.0141\\
-0.01506	-0.01506\\
-0.015885	-0.015885\\
-0.015975	-0.015975\\
-0.0161125	-0.0161125\\
-0.0164325	-0.0164325\\
-0.0169825	-0.0169825\\
-0.017075	-0.017075\\
-0.01648	-0.01648\\
-0.0157925	-0.0157925\\
-0.0149225	-0.0149225\\
-0.0145575	-0.0145575\\
-0.0146025	-0.0146025\\
-0.0148775	-0.0148775\\
-0.01474	-0.01474\\
-0.014465	-0.014465\\
-0.0145575	-0.0145575\\
-0.014695	-0.014695\\
-0.014785	-0.014785\\
-0.01506	-0.01506\\
-0.0149225	-0.0149225\\
-0.014145	-0.014145\\
-0.01355	-0.01355\\
-0.0131375	-0.0131375\\
-0.013	-0.013\\
-0.0130925	-0.0130925\\
-0.013595	-0.013595\\
-0.0137775	-0.0137775\\
-0.0140075	-0.0140075\\
-0.0146025	-0.0146025\\
-0.01451	-0.01451\\
-0.01497	-0.01497\\
-0.015885	-0.015885\\
-0.0157025	-0.0157025\\
-0.0158375	-0.0158375\\
-0.0161125	-0.0161125\\
-0.0157025	-0.0157025\\
-0.0154275	-0.0154275\\
-0.0148775	-0.0148775\\
-0.014695	-0.014695\\
-0.0155175	-0.0155175\\
-0.01657	-0.01657\\
-0.017165	-0.017165\\
-0.017075	-0.017075\\
-0.016845	-0.016845\\
-0.0164325	-0.0164325\\
-0.0162975	-0.0162975\\
-0.0166175	-0.0166175\\
-0.0160675	-0.0160675\\
-0.01561	-0.01561\\
-0.015335	-0.015335\\
-0.0158375	-0.0158375\\
-0.0160675	-0.0160675\\
-0.01538	-0.01538\\
-0.01506	-0.01506\\
-0.014695	-0.014695\\
-0.01506	-0.01506\\
-0.015655	-0.015655\\
-0.0157925	-0.0157925\\
-0.015565	-0.015565\\
-0.0151975	-0.0151975\\
-0.01506	-0.01506\\
-0.0146475	-0.0146475\\
-0.0142825	-0.0142825\\
-0.0140075	-0.0140075\\
-0.0136425	-0.0136425\\
-0.01355	-0.01355\\
-0.0140525	-0.0140525\\
-0.0145575	-0.0145575\\
-0.014695	-0.014695\\
-0.0143275	-0.0143275\\
-0.0141	-0.0141\\
-0.01387	-0.01387\\
-0.0140075	-0.0140075\\
-0.0143275	-0.0143275\\
-0.014465	-0.014465\\
-0.0140075	-0.0140075\\
-0.014465	-0.014465\\
-0.0151525	-0.0151525\\
-0.0149225	-0.0149225\\
-0.0140075	-0.0140075\\
-0.0136875	-0.0136875\\
-0.01419	-0.01419\\
-0.014375	-0.014375\\
-0.0142825	-0.0142825\\
-0.013915	-0.013915\\
-0.0142375	-0.0142375\\
-0.0148325	-0.0148325\\
-0.0145575	-0.0145575\\
-0.013595	-0.013595\\
-0.0134125	-0.0134125\\
-0.01419	-0.01419\\
-0.014785	-0.014785\\
-0.014375	-0.014375\\
-0.0140525	-0.0140525\\
-0.0146475	-0.0146475\\
-0.0152425	-0.0152425\\
-0.01529	-0.01529\\
-0.0157025	-0.0157025\\
-0.016205	-0.016205\\
-0.0158375	-0.0158375\\
-0.0154725	-0.0154725\\
-0.0157475	-0.0157475\\
-0.0155175	-0.0155175\\
-0.015655	-0.015655\\
-0.0160225	-0.0160225\\
-0.015885	-0.015885\\
-0.01506	-0.01506\\
-0.0152425	-0.0152425\\
-0.0161125	-0.0161125\\
-0.01657	-0.01657\\
-0.0163875	-0.0163875\\
-0.0161125	-0.0161125\\
-0.0162975	-0.0162975\\
-0.0161125	-0.0161125\\
-0.015565	-0.015565\\
-0.01529	-0.01529\\
-0.01538	-0.01538\\
-0.0155175	-0.0155175\\
-0.01561	-0.01561\\
-0.0154275	-0.0154275\\
-0.01529	-0.01529\\
-0.015655	-0.015655\\
-0.01529	-0.01529\\
-0.0148325	-0.0148325\\
-0.0146025	-0.0146025\\
-0.01419	-0.01419\\
-0.0141	-0.0141\\
-0.0145575	-0.0145575\\
-0.01474	-0.01474\\
-0.01497	-0.01497\\
-0.01538	-0.01538\\
-0.0157475	-0.0157475\\
-0.01561	-0.01561\\
-0.015335	-0.015335\\
-0.0145575	-0.0145575\\
-0.01451	-0.01451\\
-0.01529	-0.01529\\
-0.015105	-0.015105\\
-0.01442	-0.01442\\
-0.01474	-0.01474\\
-0.01538	-0.01538\\
-0.01561	-0.01561\\
-0.0160675	-0.0160675\\
-0.0166625	-0.0166625\\
-0.01648	-0.01648\\
-0.0161125	-0.0161125\\
-0.0162975	-0.0162975\\
-0.015885	-0.015885\\
-0.0154725	-0.0154725\\
-0.0157025	-0.0157025\\
-0.0158375	-0.0158375\\
-0.0154725	-0.0154725\\
-0.01506	-0.01506\\
-0.014785	-0.014785\\
-0.0149225	-0.0149225\\
-0.014695	-0.014695\\
-0.01529	-0.01529\\
-0.01561	-0.01561\\
-0.0152425	-0.0152425\\
-0.015015	-0.015015\\
-0.0152425	-0.0152425\\
-0.0148775	-0.0148775\\
-0.0141	-0.0141\\
-0.0140525	-0.0140525\\
-0.0143275	-0.0143275\\
-0.0142825	-0.0142825\\
-0.013915	-0.013915\\
-0.01332	-0.01332\\
-0.0134125	-0.0134125\\
-0.013915	-0.013915\\
-0.0145575	-0.0145575\\
-0.01497	-0.01497\\
-0.0154275	-0.0154275\\
-0.0158375	-0.0158375\\
-0.015335	-0.015335\\
-0.014785	-0.014785\\
-0.0154725	-0.0154725\\
-0.01625	-0.01625\\
-0.016205	-0.016205\\
-0.0160675	-0.0160675\\
-0.01648	-0.01648\\
-0.01616	-0.01616\\
-0.0157925	-0.0157925\\
-0.015655	-0.015655\\
-0.015975	-0.015975\\
-0.0157025	-0.0157025\\
-0.01538	-0.01538\\
-0.0157925	-0.0157925\\
-0.01625	-0.01625\\
-0.0163875	-0.0163875\\
-0.015565	-0.015565\\
-0.0145575	-0.0145575\\
-0.0151525	-0.0151525\\
-0.01529	-0.01529\\
-0.014375	-0.014375\\
-0.0137775	-0.0137775\\
-0.0140075	-0.0140075\\
-0.0145575	-0.0145575\\
-0.0151975	-0.0151975\\
-0.0151525	-0.0151525\\
-0.01474	-0.01474\\
-0.01419	-0.01419\\
-0.013595	-0.013595\\
-0.01355	-0.01355\\
-0.013505	-0.013505\\
-0.0136425	-0.0136425\\
-0.0142375	-0.0142375\\
-0.0143275	-0.0143275\\
-0.0136875	-0.0136875\\
-0.0134125	-0.0134125\\
-0.01355	-0.01355\\
-0.0137325	-0.0137325\\
-0.0134575	-0.0134575\\
-0.013595	-0.013595\\
-0.013505	-0.013505\\
-0.0131375	-0.0131375\\
-0.0137325	-0.0137325\\
-0.0142825	-0.0142825\\
-0.015105	-0.015105\\
-0.0158375	-0.0158375\\
-0.01593	-0.01593\\
-0.0151525	-0.0151525\\
-0.0145575	-0.0145575\\
-0.014375	-0.014375\\
-0.01387	-0.01387\\
-0.0134125	-0.0134125\\
-0.0140075	-0.0140075\\
-0.0142375	-0.0142375\\
-0.01419	-0.01419\\
-0.0142825	-0.0142825\\
-0.014465	-0.014465\\
-0.0146025	-0.0146025\\
-0.014695	-0.014695\\
-0.01497	-0.01497\\
-0.0151525	-0.0151525\\
-0.01561	-0.01561\\
-0.015105	-0.015105\\
-0.01419	-0.01419\\
-0.01387	-0.01387\\
-0.0142825	-0.0142825\\
-0.0143275	-0.0143275\\
-0.0141	-0.0141\\
-0.0134575	-0.0134575\\
-0.0134125	-0.0134125\\
-0.013275	-0.013275\\
-0.012635	-0.012635\\
-0.01268	-0.01268\\
-0.01332	-0.01332\\
-0.0137325	-0.0137325\\
-0.014375	-0.014375\\
-0.015105	-0.015105\\
-0.0149225	-0.0149225\\
-0.0139625	-0.0139625\\
-0.0141	-0.0141\\
-0.014465	-0.014465\\
-0.013825	-0.013825\\
-0.01387	-0.01387\\
-0.0146025	-0.0146025\\
-0.014785	-0.014785\\
-0.0152425	-0.0152425\\
-0.0158375	-0.0158375\\
-0.0154725	-0.0154725\\
-0.015105	-0.015105\\
-0.01497	-0.01497\\
-0.0151525	-0.0151525\\
-0.015565	-0.015565\\
-0.01538	-0.01538\\
-0.0148775	-0.0148775\\
};
\end{axis}

\begin{axis}[%
width=4.927cm,
height=2.746cm,
at={(0cm,11.441cm)},
scale only axis,
xmin=-0.018,
xmax=-0.012,
xlabel style={font=\color{white!15!black}},
xlabel={$u(t-1)$},
ymin=-0.1007075,
ymax=0.0030525,
ylabel style={font=\color{white!15!black}},
ylabel={$\delta^4 y(t)$},
axis background/.style={fill=white},
title style={font=\bfseries},
title={C3, R = 0.692},
axis x line*=bottom,
axis y line*=left
]
\addplot[only marks, mark=*, mark options={}, mark size=1.5000pt, color=mycolor1, fill=mycolor1] table[row sep=crcr]{%
x	y\\
-0.01561	-0.0457775\\
-0.0157025	-0.0579825\\
-0.0158375	-0.0457775\\
-0.015655	-0.024415\\
-0.01497	-0.0396725\\
-0.0148325	-0.0305175\\
-0.0148325	-0.03662\\
-0.01497	-0.027465\\
-0.0148775	-0.009155\\
-0.0140075	-0.0061025\\
-0.01323	-0.0122075\\
-0.0127725	-0.01526\\
-0.0136875	-0.03662\\
-0.014695	-0.042725\\
-0.01497	-0.042725\\
-0.015105	-0.03357\\
-0.0148325	-0.024415\\
-0.0142825	-0.0457775\\
-0.0148775	-0.03662\\
-0.01474	-0.027465\\
-0.014375	-0.03357\\
-0.014695	-0.0305175\\
-0.0148325	-0.027465\\
-0.0146025	-0.05188\\
-0.01506	-0.0488275\\
-0.0151975	-0.0305175\\
-0.0149225	-0.05188\\
-0.0152425	-0.05188\\
-0.01538	-0.0396725\\
-0.0151525	-0.03357\\
-0.0149225	-0.024415\\
-0.014695	-0.03662\\
-0.014695	-0.03662\\
-0.0149225	-0.03662\\
-0.01497	-0.03662\\
-0.0149225	-0.03662\\
-0.01497	-0.0396725\\
-0.01506	-0.0549325\\
-0.0154725	-0.05188\\
-0.015565	-0.03357\\
-0.0151525	-0.03357\\
-0.0149225	-0.0213625\\
-0.014375	-0.0305175\\
-0.0141	-0.0305175\\
-0.0143275	-0.03357\\
-0.0142375	-0.027465\\
-0.01419	-0.03357\\
-0.014465	-0.03662\\
-0.014695	-0.03357\\
-0.014695	-0.05188\\
-0.0151975	-0.03357\\
-0.0151525	-0.024415\\
-0.01451	-0.01831\\
-0.014145	-0.0213625\\
-0.0142825	-0.027465\\
-0.014465	-0.01526\\
-0.0140525	-0.024415\\
-0.0137325	-0.027465\\
-0.0140075	-0.01526\\
-0.0139625	-0.01526\\
-0.0137325	-0.024415\\
-0.01387	-0.024415\\
-0.0140525	-0.0396725\\
-0.0145575	-0.03662\\
-0.01474	-0.042725\\
-0.0149225	-0.0396725\\
-0.0149225	-0.0457775\\
-0.015105	-0.03357\\
-0.015015	-0.03662\\
-0.0148775	-0.0305175\\
-0.014785	-0.027465\\
-0.01474	-0.0305175\\
-0.014695	-0.0579825\\
-0.0152425	-0.07019\\
-0.0160225	-0.076295\\
-0.0162975	-0.0732425\\
-0.0163425	-0.0457775\\
-0.0157925	-0.07019\\
-0.0160675	-0.0823975\\
-0.01648	-0.079345\\
-0.0166175	-0.05188\\
-0.0163875	-0.08545\\
-0.0166175	-0.1007075\\
-0.01712	-0.0732425\\
-0.0169825	-0.061035\\
-0.0166625	-0.0488275\\
-0.0161125	-0.0396725\\
-0.015655	-0.0305175\\
-0.015335	-0.03662\\
-0.015335	-0.024415\\
-0.0151525	-0.01831\\
-0.0145575	-0.01831\\
-0.01419	-0.027465\\
-0.0142825	-0.03357\\
-0.014695	-0.0305175\\
-0.014785	-0.0305175\\
-0.014785	-0.0396725\\
-0.015015	-0.0457775\\
-0.0151525	-0.0488275\\
-0.015655	-0.0396725\\
-0.015655	-0.0549325\\
-0.0157475	-0.0396725\\
-0.015565	-0.0396725\\
-0.01529	-0.042725\\
-0.0154275	-0.0305175\\
-0.0151975	-0.0305175\\
-0.0149225	-0.0396725\\
-0.01506	-0.03662\\
-0.015105	-0.0213625\\
-0.0148775	-0.0305175\\
-0.01474	-0.0213625\\
-0.01451	-0.024415\\
-0.014375	-0.024415\\
-0.01451	-0.01831\\
-0.014145	-0.024415\\
-0.014145	-0.0305175\\
-0.01442	-0.042725\\
-0.0149225	-0.042725\\
-0.015105	-0.042725\\
-0.0151975	-0.027465\\
-0.0148325	-0.0305175\\
-0.015015	-0.0579825\\
-0.015655	-0.0396725\\
-0.0155175	-0.05188\\
-0.01561	-0.0549325\\
-0.015655	-0.0305175\\
-0.0151975	-0.024415\\
-0.014695	-0.0305175\\
-0.01474	-0.03357\\
-0.0149225	-0.0549325\\
-0.0155175	-0.0640875\\
-0.0160225	-0.0671375\\
-0.0161125	-0.0488275\\
-0.015885	-0.05188\\
-0.0157925	-0.0457775\\
-0.0157475	-0.042725\\
-0.015655	-0.0457775\\
-0.01561	-0.03357\\
-0.0152425	-0.0305175\\
-0.0149225	-0.024415\\
-0.014785	-0.024415\\
-0.0146475	-0.027465\\
-0.0146475	-0.0396725\\
-0.015015	-0.0579825\\
-0.01561	-0.042725\\
-0.01561	-0.03357\\
-0.0151525	-0.027465\\
-0.0148325	-0.027465\\
-0.01474	-0.01831\\
-0.0142375	-0.01526\\
-0.0137325	-0.01526\\
-0.0136425	-0.0305175\\
-0.014145	-0.01831\\
-0.0141	-0.0213625\\
-0.0140525	-0.0213625\\
-0.014145	-0.0213625\\
-0.0141	-0.01831\\
-0.01387	-0.0122075\\
-0.0134125	-0.009155\\
-0.0130475	-0.0213625\\
-0.013275	-0.0396725\\
-0.01419	-0.0457775\\
-0.01497	-0.05188\\
-0.015335	-0.0396725\\
-0.01497	-0.027465\\
-0.01451	-0.0213625\\
-0.0140525	-0.0122075\\
-0.013505	-0.024415\\
-0.0140525	-0.0213625\\
-0.0142375	-0.0396725\\
-0.0146025	-0.042725\\
-0.01506	-0.0732425\\
-0.01593	-0.0823975\\
-0.016525	-0.0640875\\
-0.0164325	-0.0640875\\
-0.0163875	-0.0457775\\
-0.0160225	-0.0579825\\
-0.015885	-0.0549325\\
-0.015975	-0.061035\\
-0.0160675	-0.0396725\\
-0.0158375	-0.042725\\
-0.015655	-0.042725\\
-0.01561	-0.0396725\\
-0.0154725	-0.0396725\\
-0.01561	-0.03357\\
-0.01538	-0.0579825\\
-0.0157025	-0.0640875\\
-0.0161125	-0.0488275\\
-0.0158375	-0.03357\\
-0.0152425	-0.03662\\
-0.015105	-0.0579825\\
-0.01561	-0.061035\\
-0.0160225	-0.07019\\
-0.0163875	-0.076295\\
-0.0166175	-0.0732425\\
-0.0167075	-0.0579825\\
-0.0164325	-0.0549325\\
-0.01625	-0.0640875\\
-0.0163425	-0.07019\\
-0.016525	-0.0457775\\
-0.0161125	-0.0305175\\
-0.0154275	-0.0488275\\
-0.0155175	-0.03662\\
-0.0154275	-0.024415\\
-0.0149225	-0.03357\\
-0.01497	-0.03662\\
-0.0151525	-0.027465\\
-0.01497	-0.027465\\
-0.0146475	-0.027465\\
-0.01474	-0.0213625\\
-0.014695	-0.027465\\
-0.0148775	-0.0457775\\
-0.015335	-0.042725\\
-0.015335	-0.0305175\\
-0.015015	-0.0305175\\
-0.0148775	-0.0457775\\
-0.01529	-0.0671375\\
-0.0160675	-0.042725\\
-0.0160225	-0.03662\\
-0.0154275	-0.027465\\
-0.015015	-0.03357\\
-0.0149225	-0.027465\\
-0.0149225	-0.0213625\\
-0.0146025	-0.0213625\\
-0.0145575	-0.027465\\
-0.0146475	-0.03357\\
-0.0149225	-0.03662\\
-0.015105	-0.0305175\\
-0.01474	-0.0396725\\
-0.015015	-0.05188\\
-0.01538	-0.042725\\
-0.0154275	-0.0305175\\
-0.0152425	-0.0396725\\
-0.0152425	-0.05188\\
-0.01561	-0.03357\\
-0.01529	-0.01831\\
-0.014375	-0.03357\\
-0.0142375	-0.027465\\
-0.014375	-0.0305175\\
-0.01474	-0.024415\\
-0.01474	-0.0213625\\
-0.014465	-0.03357\\
-0.01474	-0.03357\\
-0.0148325	-0.024415\\
-0.0145575	-0.0305175\\
-0.0146475	-0.024415\\
-0.0146025	-0.01831\\
-0.01451	-0.0305175\\
-0.0145575	-0.0213625\\
-0.01419	-0.024415\\
-0.0141	-0.01831\\
-0.01387	-0.0122075\\
-0.013825	-0.0305175\\
-0.014375	-0.0396725\\
-0.014695	-0.042725\\
-0.015105	-0.0579825\\
-0.015655	-0.0396725\\
-0.01538	-0.024415\\
-0.01474	-0.0213625\\
-0.0142825	-0.0213625\\
-0.0141	-0.027465\\
-0.014375	-0.0213625\\
-0.0140525	-0.0122075\\
-0.0139625	-0.027465\\
-0.01419	-0.0396725\\
-0.014785	-0.03357\\
-0.01497	-0.05188\\
-0.01529	-0.0396725\\
-0.01506	-0.03662\\
-0.0148325	-0.0213625\\
-0.0142375	-0.027465\\
-0.013915	-0.0122075\\
-0.0134125	-0.009155\\
-0.01323	-0.0122075\\
-0.0133675	-0.024415\\
-0.0140525	-0.027465\\
-0.0142375	-0.0305175\\
-0.014465	-0.03357\\
-0.0146025	-0.05188\\
-0.0152425	-0.042725\\
-0.0154275	-0.03662\\
-0.01506	-0.042725\\
-0.0151525	-0.0488275\\
-0.0152425	-0.05188\\
-0.01561	-0.042725\\
-0.0155175	-0.03357\\
-0.01506	-0.01831\\
-0.0143275	-0.01526\\
-0.0136875	-0.0122075\\
-0.013505	-0.01831\\
-0.01355	-0.01831\\
-0.01355	-0.0122075\\
-0.013505	-0.0213625\\
-0.0136875	-0.0305175\\
-0.01419	-0.027465\\
-0.0143275	-0.0305175\\
-0.0143275	-0.0305175\\
-0.01451	-0.0213625\\
-0.014145	-0.0122075\\
-0.0134575	-0.024415\\
-0.01387	-0.0396725\\
-0.01442	-0.03357\\
-0.0146025	-0.0396725\\
-0.014695	-0.03357\\
-0.0146475	-0.024415\\
-0.014375	-0.03662\\
-0.0145575	-0.0457775\\
-0.015015	-0.0457775\\
-0.0151975	-0.03357\\
-0.0148775	-0.05188\\
-0.01538	-0.0396725\\
-0.015335	-0.042725\\
-0.0151975	-0.0457775\\
-0.01538	-0.0488275\\
-0.01538	-0.03357\\
-0.0151525	-0.03357\\
-0.0149225	-0.0457775\\
-0.01529	-0.0671375\\
-0.015975	-0.0549325\\
-0.0158375	-0.03357\\
-0.0151975	-0.03357\\
-0.0149225	-0.03357\\
-0.01497	-0.0396725\\
-0.0151525	-0.042725\\
-0.01538	-0.0305175\\
-0.0151525	-0.03662\\
-0.0151525	-0.0457775\\
-0.0154275	-0.05188\\
-0.01561	-0.03357\\
-0.01529	-0.0305175\\
-0.01497	-0.0396725\\
-0.015105	-0.0579825\\
-0.0157475	-0.05188\\
-0.0157925	-0.0549325\\
-0.0157925	-0.03662\\
-0.01529	-0.03662\\
-0.01497	-0.0305175\\
-0.0149225	-0.024415\\
-0.014375	-0.0122075\\
-0.01387	-0.009155\\
-0.0133675	-0.009155\\
-0.013	-0.01831\\
-0.013595	-0.03357\\
-0.01419	-0.03662\\
-0.0145575	-0.024415\\
-0.014375	-0.0305175\\
-0.014375	-0.03357\\
-0.01451	-0.024415\\
-0.01419	-0.0213625\\
-0.0140525	-0.0213625\\
-0.0140525	-0.01526\\
-0.013825	-0.024415\\
-0.01387	-0.03662\\
-0.0143275	-0.0396725\\
-0.014785	-0.027465\\
-0.014465	-0.0213625\\
-0.0141	-0.0213625\\
-0.01387	-0.0213625\\
-0.0139625	-0.01831\\
-0.0136425	-0.027465\\
-0.0143275	-0.05188\\
-0.0151975	-0.0457775\\
-0.0151525	-0.0457775\\
-0.0154725	-0.061035\\
-0.015885	-0.0579825\\
-0.015975	-0.0457775\\
-0.0157475	-0.03662\\
-0.015335	-0.03662\\
-0.0151975	-0.03662\\
-0.0151975	-0.042725\\
-0.015335	-0.0488275\\
-0.0155175	-0.0488275\\
-0.01561	-0.0640875\\
-0.015885	-0.0671375\\
-0.016205	-0.079345\\
-0.01657	-0.061035\\
-0.0162975	-0.0640875\\
-0.016205	-0.0671375\\
-0.0164325	-0.0549325\\
-0.01625	-0.061035\\
-0.01625	-0.05188\\
-0.01616	-0.03357\\
-0.0154275	-0.024415\\
-0.014785	-0.024415\\
-0.0146025	-0.0305175\\
-0.0149225	-0.024415\\
-0.0148325	-0.01831\\
-0.01442	-0.027465\\
-0.01451	-0.0213625\\
-0.0142825	-0.01526\\
-0.013915	-0.01831\\
-0.0139625	-0.024415\\
-0.0140525	-0.01526\\
-0.013825	-0.0305175\\
-0.0142375	-0.03662\\
-0.01474	-0.027465\\
-0.0146025	-0.042725\\
-0.015015	-0.061035\\
-0.0157475	-0.061035\\
-0.015885	-0.07019\\
-0.016205	-0.0579825\\
-0.015975	-0.0549325\\
-0.015885	-0.042725\\
-0.015565	-0.024415\\
-0.0149225	-0.027465\\
-0.0149225	-0.027465\\
-0.014695	-0.01831\\
-0.0143275	-0.027465\\
-0.014465	-0.024415\\
-0.0140075	-0.0061025\\
-0.0134125	-0.01526\\
-0.0133675	-0.024415\\
-0.013915	-0.0305175\\
-0.01451	-0.03662\\
-0.0148775	-0.042725\\
-0.015015	-0.03662\\
-0.015015	-0.042725\\
-0.0151525	-0.03662\\
-0.01506	-0.03357\\
-0.0148325	-0.0305175\\
-0.0148325	-0.027465\\
-0.0146475	-0.03662\\
-0.01497	-0.03357\\
-0.01474	-0.027465\\
-0.01442	-0.03357\\
-0.014695	-0.0457775\\
-0.0151975	-0.03357\\
-0.015105	-0.027465\\
-0.01474	-0.027465\\
-0.01451	-0.027465\\
-0.0145575	-0.0213625\\
-0.0142375	-0.01831\\
-0.0140075	-0.027465\\
-0.0142375	-0.03357\\
-0.014465	-0.03357\\
-0.014695	-0.027465\\
-0.0145575	-0.03357\\
-0.014695	-0.03357\\
-0.01474	-0.0213625\\
-0.0142825	-0.01831\\
-0.0141	-0.0396725\\
-0.0146475	-0.0488275\\
-0.01529	-0.0457775\\
-0.0154275	-0.0457775\\
-0.01529	-0.0396725\\
-0.0152425	-0.03357\\
-0.015015	-0.027465\\
-0.0148775	-0.024415\\
-0.0146475	-0.03357\\
-0.0148325	-0.03357\\
-0.0148325	-0.0213625\\
-0.014375	-0.0305175\\
-0.0145575	-0.03662\\
-0.01474	-0.03662\\
-0.015015	-0.042725\\
-0.0151525	-0.042725\\
-0.0151975	-0.05188\\
-0.01561	-0.07019\\
-0.0160675	-0.076295\\
-0.0164325	-0.05188\\
-0.0160675	-0.03357\\
-0.015335	-0.0213625\\
-0.014695	-0.01526\\
-0.0140525	-0.0213625\\
-0.0141	-0.01831\\
-0.014145	-0.027465\\
-0.0146025	-0.0305175\\
-0.014695	-0.0305175\\
-0.01451	-0.03357\\
-0.01474	-0.0396725\\
-0.01497	-0.0488275\\
-0.0151975	-0.0488275\\
-0.01538	-0.07019\\
-0.0160225	-0.0579825\\
-0.0161125	-0.0457775\\
-0.0158375	-0.03662\\
-0.01538	-0.03662\\
-0.0151525	-0.03662\\
-0.0151975	-0.042725\\
-0.01538	-0.03662\\
-0.0151525	-0.0305175\\
-0.015015	-0.03357\\
-0.01497	-0.0213625\\
-0.01442	-0.01831\\
-0.0146025	-0.042725\\
-0.015105	-0.03662\\
-0.0149225	-0.0305175\\
-0.0148775	-0.0579825\\
-0.015565	-0.0457775\\
-0.0155175	-0.0396725\\
-0.01538	-0.0549325\\
-0.0157475	-0.0579825\\
-0.0158375	-0.03662\\
-0.0154275	-0.0213625\\
-0.01474	-0.024415\\
-0.0146025	-0.0396725\\
-0.015015	-0.0579825\\
-0.0158375	-0.0640875\\
-0.0161125	-0.0640875\\
-0.0161125	-0.05188\\
-0.015975	-0.03357\\
-0.01538	-0.027465\\
-0.015105	-0.024415\\
-0.0146475	-0.01831\\
-0.01442	-0.024415\\
-0.0142375	-0.024415\\
-0.014465	-0.027465\\
-0.0146025	-0.0213625\\
-0.0142375	-0.024415\\
-0.0143275	-0.0396725\\
-0.014785	-0.042725\\
-0.01497	-0.042725\\
-0.01538	-0.061035\\
-0.0157925	-0.0732425\\
-0.0163425	-0.0457775\\
-0.01616	-0.0457775\\
-0.015885	-0.0488275\\
-0.015885	-0.042725\\
-0.0157025	-0.03357\\
-0.0152425	-0.03357\\
-0.0152425	-0.0579825\\
-0.0158375	-0.0640875\\
-0.01616	-0.042725\\
-0.015655	-0.0213625\\
-0.0149225	-0.01526\\
-0.0142825	-0.009155\\
-0.0133675	-0.0061025\\
-0.0130475	-0.01526\\
-0.013505	-0.0213625\\
-0.0137325	-0.024415\\
-0.01419	-0.024415\\
-0.0141	-0.01831\\
-0.013825	-0.01831\\
-0.0137775	-0.0213625\\
-0.0137325	-0.01526\\
-0.0136425	-0.0213625\\
-0.0137325	-0.0305175\\
-0.0140525	-0.0396725\\
-0.01474	-0.042725\\
-0.01506	-0.042725\\
-0.01506	-0.0488275\\
-0.0152425	-0.0579825\\
-0.0157025	-0.061035\\
-0.015885	-0.0671375\\
-0.016205	-0.07019\\
-0.01625	-0.05188\\
-0.01593	-0.0396725\\
-0.0154275	-0.03357\\
-0.0152425	-0.05188\\
-0.0157475	-0.061035\\
-0.0161125	-0.042725\\
-0.0157925	-0.03357\\
-0.01529	-0.01831\\
-0.014375	-0.0122075\\
-0.0136875	-0.0122075\\
-0.013505	-0.01526\\
-0.012955	-0.0061025\\
-0.0133675	-0.0305175\\
-0.013825	-0.024415\\
-0.0139625	-0.0213625\\
-0.013915	-0.01831\\
-0.0134575	-0.0122075\\
-0.0131375	-0.01831\\
-0.01323	-0.0213625\\
-0.01332	-0.01831\\
-0.0134575	-0.0122075\\
-0.013275	-0.009155\\
-0.0130475	-0.01831\\
-0.0134575	-0.0396725\\
-0.014695	-0.0549325\\
-0.015335	-0.0579825\\
-0.015565	-0.042725\\
-0.01538	-0.0488275\\
-0.015655	-0.076295\\
-0.0163425	-0.076295\\
-0.0164325	-0.0671375\\
-0.01648	-0.061035\\
-0.016205	-0.0549325\\
-0.0160675	-0.0579825\\
-0.0161125	-0.0396725\\
-0.01561	-0.03357\\
-0.0154275	-0.027465\\
-0.0148325	-0.01831\\
-0.01451	-0.03662\\
-0.015105	-0.042725\\
-0.0148775	-0.0305175\\
-0.014785	-0.03357\\
-0.01497	-0.03662\\
-0.0148775	-0.027465\\
-0.0148775	-0.03357\\
-0.01497	-0.03662\\
-0.01506	-0.0457775\\
-0.0151525	-0.03662\\
-0.015105	-0.027465\\
-0.0148325	-0.0396725\\
-0.0148775	-0.027465\\
-0.0143275	-0.0061025\\
-0.013275	-0.01831\\
-0.0131825	-0.0213625\\
-0.0136875	-0.01831\\
-0.013275	0.0030525\\
-0.0124975	-0.0061025\\
-0.012635	-0.0213625\\
-0.013275	-0.0213625\\
-0.0136425	-0.0305175\\
-0.0140075	-0.027465\\
-0.0140075	-0.03662\\
-0.01451	-0.042725\\
-0.0146475	-0.0213625\\
-0.0142375	-0.0305175\\
-0.0145575	-0.03662\\
-0.0140525	-0.0122075\\
-0.0136425	-0.01526\\
-0.01323	-0.009155\\
-0.0127725	-0.024415\\
-0.01291	-0.01526\\
-0.0128175	-0.009155\\
-0.0131825	-0.03357\\
-0.0139625	-0.03357\\
-0.0141	-0.01831\\
-0.01387	-0.0213625\\
-0.01387	-0.01831\\
-0.0137325	-0.01831\\
-0.01387	-0.01831\\
-0.013595	-0.01831\\
-0.0134575	-0.01831\\
-0.0134575	-0.0122075\\
-0.0131825	-0.0122075\\
-0.013275	-0.01526\\
-0.01323	-0.0213625\\
-0.0136875	-0.024415\\
-0.013825	-0.01526\\
-0.013595	-0.01526\\
-0.01355	-0.01526\\
-0.01332	-0.01526\\
-0.01332	-0.03662\\
-0.01419	-0.0549325\\
-0.0149225	-0.03662\\
-0.014695	-0.03357\\
-0.01451	-0.027465\\
-0.01419	-0.027465\\
-0.0145575	-0.0396725\\
-0.014695	-0.0213625\\
-0.0142825	-0.027465\\
-0.0142825	-0.0305175\\
-0.0141	-0.027465\\
-0.0143275	-0.0305175\\
-0.0139625	-0.01526\\
-0.0137775	-0.027465\\
-0.013915	-0.0305175\\
-0.0142375	-0.024415\\
-0.0140525	-0.0213625\\
-0.0139625	-0.027465\\
-0.0140075	-0.03357\\
-0.014465	-0.03357\\
-0.01451	-0.0396725\\
-0.01497	-0.0640875\\
-0.01561	-0.0488275\\
-0.015565	-0.0305175\\
-0.015105	-0.027465\\
-0.0146025	-0.03357\\
-0.0149225	-0.0488275\\
-0.01529	-0.03357\\
-0.01506	-0.0396725\\
-0.0152425	-0.03662\\
-0.014785	-0.01526\\
-0.0139625	-0.0122075\\
-0.0136425	-0.03357\\
-0.014465	-0.0305175\\
-0.014695	-0.0305175\\
-0.01451	-0.0457775\\
-0.015015	-0.0457775\\
-0.01506	-0.03662\\
-0.0149225	-0.03357\\
-0.0145575	-0.0305175\\
-0.0146475	-0.0457775\\
-0.01497	-0.03357\\
-0.0148775	-0.0305175\\
-0.0148325	-0.03357\\
-0.01474	-0.027465\\
-0.01442	-0.027465\\
-0.01442	-0.027465\\
-0.01419	-0.024415\\
-0.0141	-0.0305175\\
-0.0146475	-0.0457775\\
-0.0149225	-0.042725\\
-0.01506	-0.0396725\\
-0.015015	-0.0305175\\
-0.014695	-0.01831\\
-0.0142825	-0.0213625\\
-0.0142375	-0.027465\\
-0.01442	-0.03357\\
-0.014695	-0.0457775\\
-0.015015	-0.05188\\
-0.015335	-0.042725\\
-0.0151525	-0.027465\\
-0.0146025	-0.0396725\\
-0.01497	-0.0640875\\
-0.01561	-0.0396725\\
-0.015335	-0.03662\\
-0.01529	-0.05188\\
-0.0154275	-0.0396725\\
-0.0152425	-0.03357\\
-0.01497	-0.0396725\\
-0.015015	-0.03662\\
-0.0149225	-0.03357\\
-0.015015	-0.042725\\
-0.0151975	-0.03662\\
-0.015105	-0.03662\\
-0.0151525	-0.0396725\\
-0.015105	-0.03357\\
-0.01506	-0.03357\\
-0.0148775	-0.042725\\
-0.01497	-0.03357\\
-0.014785	-0.0305175\\
-0.014695	-0.03662\\
-0.0148325	-0.03357\\
-0.0148325	-0.0213625\\
-0.0142375	-0.0061025\\
-0.0134575	-0.009155\\
-0.01291	-0.01526\\
-0.0130475	-0.03662\\
-0.014145	-0.0488275\\
-0.0148325	-0.0396725\\
-0.01497	-0.024415\\
-0.01474	-0.03357\\
-0.014695	-0.0305175\\
-0.0146475	-0.0305175\\
-0.014465	-0.0213625\\
-0.0140525	-0.0213625\\
-0.0140525	-0.027465\\
-0.01419	-0.0213625\\
-0.0141	-0.027465\\
-0.0142375	-0.027465\\
-0.014145	-0.024415\\
-0.0140525	-0.01831\\
-0.0137325	-0.024415\\
-0.013825	-0.03357\\
-0.01451	-0.03662\\
-0.01442	-0.027465\\
-0.01451	-0.0396725\\
-0.014465	-0.03662\\
-0.01474	-0.03662\\
-0.0148775	-0.0457775\\
-0.0151525	-0.042725\\
-0.01506	-0.03662\\
-0.01506	-0.0457775\\
-0.015105	-0.027465\\
-0.01451	-0.0488275\\
-0.014695	-0.03357\\
-0.0145575	-0.027465\\
-0.0145575	-0.0396725\\
-0.01474	-0.042725\\
-0.01497	-0.03662\\
-0.0148775	-0.03662\\
-0.015015	-0.0549325\\
-0.0154275	-0.0457775\\
-0.01529	-0.0396725\\
-0.015105	-0.03357\\
-0.0148775	-0.03662\\
-0.01497	-0.0305175\\
-0.0148325	-0.027465\\
-0.014695	-0.024415\\
-0.014465	-0.027465\\
-0.0145575	-0.027465\\
-0.01442	-0.042725\\
-0.015015	-0.0671375\\
-0.01561	-0.05188\\
-0.01561	-0.0549325\\
-0.0155175	-0.05188\\
-0.0155175	-0.03662\\
-0.0149225	-0.024415\\
-0.0146025	-0.024415\\
-0.014375	-0.01526\\
-0.014145	-0.0213625\\
-0.0141	-0.03357\\
-0.01451	-0.0396725\\
-0.0149225	-0.0457775\\
-0.0151525	-0.042725\\
-0.015105	-0.0305175\\
-0.014695	-0.0488275\\
-0.0152425	-0.0579825\\
-0.015655	-0.0579825\\
-0.0157925	-0.0549325\\
-0.015655	-0.07019\\
-0.01616	-0.07019\\
-0.01616	-0.03357\\
-0.01538	-0.024415\\
-0.0148325	-0.024415\\
-0.014465	-0.03357\\
-0.0148325	-0.05188\\
-0.0152425	-0.05188\\
-0.0157025	-0.03662\\
-0.01561	-0.03662\\
-0.0152425	-0.0213625\\
-0.0146025	-0.0213625\\
-0.0143275	-0.024415\\
-0.014375	-0.027465\\
-0.01442	-0.01831\\
-0.014145	-0.01831\\
-0.0141	-0.0213625\\
-0.0140075	-0.0122075\\
-0.0134575	-0.0213625\\
-0.01387	-0.03662\\
-0.01419	-0.024415\\
-0.014465	-0.0305175\\
-0.01451	-0.027465\\
-0.01451	-0.03357\\
-0.0146025	-0.0396725\\
-0.014785	-0.03357\\
-0.0146025	-0.03357\\
-0.0148325	-0.0457775\\
-0.0151525	-0.0396725\\
-0.01506	-0.03662\\
-0.01497	-0.03357\\
-0.0148775	-0.03357\\
-0.01497	-0.0488275\\
-0.01529	-0.0457775\\
-0.0151975	-0.027465\\
-0.01497	-0.03662\\
-0.01497	-0.03662\\
-0.0149225	-0.0213625\\
-0.014465	-0.0396725\\
-0.0146475	-0.0457775\\
-0.0151975	-0.0396725\\
-0.0152425	-0.0457775\\
-0.01538	-0.0671375\\
-0.0160225	-0.0579825\\
-0.015885	-0.0396725\\
-0.015565	-0.0396725\\
-0.01529	-0.042725\\
-0.015335	-0.0579825\\
-0.0158375	-0.0457775\\
-0.0155175	-0.0305175\\
-0.0152425	-0.0488275\\
-0.0155175	-0.0457775\\
-0.0152425	-0.024415\\
-0.0146025	-0.0305175\\
-0.01451	-0.0305175\\
-0.01442	-0.01526\\
-0.0140525	-0.01831\\
-0.0139625	-0.0213625\\
-0.013825	-0.01526\\
-0.0136875	-0.0213625\\
-0.013825	-0.024415\\
-0.01419	-0.0305175\\
-0.01451	-0.0396725\\
-0.0148775	-0.042725\\
-0.01497	-0.0396725\\
-0.0151525	-0.05188\\
-0.0154275	-0.0488275\\
-0.0154275	-0.03662\\
-0.015105	-0.03357\\
-0.015015	-0.03662\\
-0.015015	-0.03357\\
-0.015015	-0.05188\\
-0.015335	-0.0305175\\
-0.01529	-0.042725\\
-0.0152425	-0.0396725\\
-0.0151525	-0.0305175\\
-0.015105	-0.05188\\
-0.0155175	-0.0488275\\
-0.0154725	-0.0396725\\
-0.0154275	-0.042725\\
-0.01538	-0.027465\\
-0.0148775	-0.0213625\\
-0.0142375	-0.01831\\
-0.01387	-0.0213625\\
-0.014145	-0.0213625\\
-0.01419	-0.0305175\\
-0.01442	-0.0305175\\
-0.0143275	-0.0305175\\
-0.014465	-0.05188\\
-0.0152425	-0.0640875\\
-0.0157925	-0.0457775\\
-0.015655	-0.0488275\\
-0.01561	-0.0457775\\
-0.01561	-0.03357\\
-0.0151975	-0.03662\\
-0.015105	-0.0396725\\
-0.0152425	-0.03357\\
-0.0151975	-0.0396725\\
-0.0151975	-0.0488275\\
-0.0154725	-0.0640875\\
-0.015975	-0.0640875\\
-0.0161125	-0.0549325\\
-0.015975	-0.061035\\
-0.0161125	-0.0549325\\
-0.0157475	-0.0488275\\
-0.015655	-0.0396725\\
-0.0154725	-0.03662\\
-0.01538	-0.05188\\
-0.01561	-0.0579825\\
-0.0158375	-0.024415\\
-0.015015	-0.0305175\\
-0.01474	-0.03662\\
-0.0146475	-0.03357\\
-0.0148775	-0.0396725\\
-0.015015	-0.024415\\
-0.0146475	-0.0213625\\
-0.0146475	-0.0488275\\
-0.0151975	-0.0732425\\
-0.0161125	-0.076295\\
-0.01648	-0.0671375\\
-0.0163425	-0.0579825\\
-0.01616	-0.061035\\
-0.016205	-0.076295\\
-0.01648	-0.061035\\
-0.0162975	-0.0579825\\
-0.016205	-0.0640875\\
-0.01625	-0.03662\\
-0.01593	-0.03357\\
-0.015655	-0.0457775\\
-0.0158375	-0.0396725\\
-0.015565	-0.024415\\
-0.015015	-0.03357\\
-0.0151525	-0.03357\\
-0.0148775	-0.0305175\\
-0.0148775	-0.0305175\\
-0.0148325	-0.027465\\
-0.015015	-0.042725\\
-0.015335	-0.03662\\
-0.0152425	-0.0305175\\
-0.0149225	-0.0305175\\
-0.015015	-0.03662\\
-0.0151525	-0.042725\\
-0.01529	-0.03357\\
-0.0152425	-0.024415\\
-0.015015	-0.042725\\
-0.01538	-0.0488275\\
-0.0154725	-0.027465\\
-0.0151525	-0.05188\\
-0.015565	-0.0457775\\
-0.0157475	-0.03662\\
-0.015335	-0.027465\\
-0.0148775	-0.027465\\
-0.01442	-0.01831\\
-0.013915	-0.0305175\\
-0.0143275	-0.0305175\\
-0.0145575	-0.01831\\
-0.0142825	-0.0122075\\
-0.013915	-0.01831\\
-0.01387	-0.0305175\\
-0.014375	-0.0396725\\
-0.0146475	-0.03662\\
-0.01497	-0.05188\\
-0.01529	-0.0457775\\
-0.01538	-0.0457775\\
-0.0154725	-0.0396725\\
-0.01506	-0.01526\\
-0.0141	-0.0213625\\
-0.0140525	-0.027465\\
-0.0140075	-0.024415\\
-0.0141	-0.03357\\
-0.014695	-0.0396725\\
-0.0149225	-0.05188\\
-0.0155175	-0.0396725\\
-0.0152425	-0.042725\\
-0.015015	-0.05188\\
-0.0154725	-0.0457775\\
-0.01529	-0.01831\\
-0.0148775	-0.0213625\\
-0.014465	-0.0305175\\
-0.01451	-0.0396725\\
-0.0148325	-0.0488275\\
-0.01538	-0.03662\\
-0.0152425	-0.027465\\
-0.0149225	-0.0305175\\
-0.0148325	-0.03357\\
-0.01497	-0.0457775\\
-0.0152425	-0.0671375\\
-0.01593	-0.061035\\
-0.0160225	-0.05188\\
-0.015975	-0.0457775\\
-0.015565	-0.027465\\
-0.0149225	-0.027465\\
-0.01497	-0.0061025\\
-0.0143275	-0.027465\\
-0.0142375	-0.024415\\
-0.01419	-0.0305175\\
-0.01419	-0.0457775\\
-0.014785	-0.0579825\\
-0.0154275	-0.0549325\\
-0.0157475	-0.0671375\\
-0.01625	-0.061035\\
-0.0162975	-0.03357\\
-0.015565	-0.03662\\
-0.0151975	-0.0305175\\
-0.01506	-0.042725\\
-0.0151525	-0.05188\\
-0.0154725	-0.061035\\
-0.015885	-0.0488275\\
-0.0157925	-0.0305175\\
-0.01529	-0.042725\\
-0.015335	-0.05188\\
-0.015565	-0.0457775\\
-0.01538	-0.0213625\\
-0.01497	-0.0213625\\
-0.0146025	-0.01831\\
-0.0141	-0.01526\\
-0.013825	-0.0122075\\
-0.0134575	-0.024415\\
-0.0136875	-0.03357\\
-0.01419	-0.0305175\\
-0.01442	-0.01831\\
-0.0142825	-0.0213625\\
-0.0141	-0.0122075\\
-0.013505	-0.0030525\\
-0.0125875	-0.01526\\
-0.0124975	-0.0213625\\
-0.0130475	-0.024415\\
-0.013595	-0.03357\\
-0.0140525	-0.03357\\
-0.014375	-0.0396725\\
-0.014695	-0.05188\\
-0.0151975	-0.07019\\
-0.01593	-0.0732425\\
-0.01616	-0.07019\\
-0.0163425	-0.0671375\\
-0.0162975	-0.042725\\
-0.0155175	-0.0305175\\
-0.0151975	-0.03662\\
-0.0151975	-0.042725\\
-0.0154725	-0.03662\\
-0.0154275	-0.027465\\
-0.01506	-0.024415\\
-0.01474	-0.024415\\
-0.0143275	-0.03662\\
-0.0146475	-0.03357\\
-0.014785	-0.01831\\
-0.0142825	-0.009155\\
-0.0137325	-0.0213625\\
-0.013915	-0.027465\\
-0.0142375	-0.0213625\\
-0.014145	-0.0305175\\
-0.0143275	-0.027465\\
-0.0142825	-0.03357\\
-0.01451	-0.05188\\
-0.0151975	-0.0396725\\
-0.0151525	-0.05188\\
-0.01538	-0.0671375\\
-0.0160225	-0.0732425\\
-0.0162975	-0.05188\\
-0.01616	-0.0396725\\
-0.015565	-0.0305175\\
-0.015105	-0.01831\\
-0.01442	-0.0305175\\
-0.0145575	-0.024415\\
-0.01442	-0.0305175\\
-0.014785	-0.03357\\
-0.0149225	-0.027465\\
-0.0146475	-0.0305175\\
-0.0148325	-0.027465\\
-0.01442	-0.0305175\\
-0.01451	-0.01831\\
-0.01442	-0.01526\\
-0.0139625	-0.01831\\
-0.01387	-0.01831\\
-0.0136875	-0.0122075\\
-0.01355	-0.01526\\
-0.013505	-0.0122075\\
-0.0131825	-0.009155\\
-0.0131825	-0.0213625\\
-0.0134125	-0.01831\\
-0.013505	-0.01831\\
-0.013505	-0.027465\\
-0.013915	-0.03662\\
-0.01451	-0.0305175\\
-0.014465	-0.0305175\\
-0.014375	-0.0457775\\
-0.0148325	-0.0549325\\
-0.0152425	-0.03662\\
-0.0151975	-0.0457775\\
-0.0151525	-0.01831\\
-0.014465	-0.01526\\
-0.0136875	-0.027465\\
-0.0140525	-0.0396725\\
-0.014695	-0.03662\\
-0.0148775	-0.05188\\
-0.01529	-0.0488275\\
-0.0154275	-0.03357\\
-0.01506	-0.0305175\\
-0.0146475	-0.0305175\\
-0.01451	-0.0305175\\
-0.0146025	-0.0213625\\
-0.014465	-0.01526\\
-0.013915	-0.0213625\\
-0.013825	-0.024415\\
-0.0136875	-0.0122075\\
-0.0134575	-0.009155\\
-0.0131825	-0.01831\\
-0.0134575	-0.027465\\
-0.013915	-0.027465\\
-0.014145	-0.01526\\
-0.013915	-0.03662\\
-0.0143275	-0.0488275\\
-0.015015	-0.03357\\
-0.01474	-0.0457775\\
-0.0148325	-0.03662\\
-0.015105	-0.03357\\
-0.0148775	-0.024415\\
-0.0143275	-0.03357\\
-0.0142375	-0.03357\\
-0.01451	-0.01831\\
-0.0141	-0.024415\\
-0.0139625	-0.01526\\
-0.013825	-0.0122075\\
-0.013275	-0.009155\\
-0.0130475	-0.01831\\
-0.013275	-0.0213625\\
-0.0136875	-0.01831\\
-0.0136875	-0.0213625\\
-0.013825	-0.0305175\\
-0.0140525	-0.01526\\
-0.013825	-0.009155\\
-0.01332	-0.009155\\
-0.013	-0.01526\\
-0.013	-0.01526\\
-0.0131375	-0.01526\\
-0.01332	-0.0396725\\
-0.01419	-0.0305175\\
-0.0146025	-0.0305175\\
-0.01442	-0.0305175\\
-0.014465	-0.042725\\
-0.014695	-0.0488275\\
-0.015105	-0.0549325\\
-0.0154275	-0.05188\\
-0.015565	-0.0396725\\
-0.01529	-0.042725\\
-0.0152425	-0.042725\\
-0.01529	-0.042725\\
-0.015335	-0.0488275\\
-0.0154275	-0.061035\\
-0.01593	-0.05188\\
-0.015975	-0.05188\\
-0.0157925	-0.042725\\
-0.015565	-0.042725\\
-0.0154275	-0.042725\\
-0.01538	-0.05188\\
-0.0154725	-0.0305175\\
-0.015105	-0.0457775\\
-0.01497	-0.0457775\\
-0.0151975	-0.0488275\\
-0.0154725	-0.03357\\
-0.0152425	-0.042725\\
-0.0152425	-0.03357\\
-0.0151525	-0.03357\\
-0.01497	-0.027465\\
-0.01451	-0.03662\\
-0.014695	-0.0488275\\
-0.01529	-0.03662\\
-0.01529	-0.0213625\\
-0.014695	-0.03357\\
-0.0146025	-0.024415\\
-0.0140525	-0.01831\\
-0.0136875	-0.0213625\\
-0.01387	-0.01526\\
-0.01355	-0.01526\\
-0.0133675	-0.01526\\
-0.013505	-0.009155\\
-0.01323	-0.0213625\\
-0.0134575	-0.01526\\
-0.0134575	-0.0061025\\
-0.0130475	-0.01831\\
-0.013275	-0.01526\\
-0.0134575	-0.0213625\\
-0.0136425	-0.027465\\
-0.0137325	-0.01831\\
-0.0137775	-0.024415\\
-0.0140525	-0.027465\\
-0.0140525	-0.027465\\
-0.014145	-0.0549325\\
-0.01497	-0.03357\\
-0.0149225	-0.0305175\\
-0.014695	-0.03662\\
-0.01474	-0.024415\\
-0.014465	-0.01526\\
-0.013825	-0.03357\\
-0.014145	-0.01831\\
-0.0141	-0.009155\\
-0.0136425	-0.024415\\
-0.01387	-0.027465\\
-0.0140075	-0.027465\\
-0.014145	-0.01526\\
-0.013915	-0.0122075\\
-0.0133675	-0.0030525\\
-0.013	-0.01526\\
-0.01291	-0.0213625\\
-0.0134125	-0.01831\\
-0.0136425	-0.027465\\
-0.0137325	-0.03357\\
-0.0141	-0.042725\\
-0.0146025	-0.027465\\
-0.0145575	-0.042725\\
-0.014785	-0.0488275\\
-0.0151525	-0.042725\\
-0.01506	-0.03662\\
-0.015015	-0.0579825\\
-0.0157025	-0.0488275\\
-0.01561	-0.0579825\\
-0.0157025	-0.0488275\\
-0.0157025	-0.061035\\
-0.0157025	-0.061035\\
-0.015975	-0.0396725\\
-0.0154725	-0.0213625\\
-0.01474	-0.01831\\
-0.0143275	-0.03662\\
-0.0146025	-0.0396725\\
-0.0149225	-0.03662\\
-0.015015	-0.027465\\
-0.014695	-0.01526\\
-0.0140075	-0.0305175\\
-0.013915	-0.042725\\
-0.0146475	-0.05188\\
-0.0152425	-0.05188\\
-0.01538	-0.027465\\
-0.0148775	-0.0213625\\
-0.014465	-0.027465\\
-0.0146475	-0.0457775\\
-0.0151975	-0.0488275\\
-0.0154725	-0.05188\\
-0.01561	-0.0396725\\
-0.015335	-0.0579825\\
-0.015565	-0.0640875\\
-0.0157925	-0.0549325\\
-0.0157925	-0.0732425\\
-0.01625	-0.0671375\\
-0.0162975	-0.0549325\\
-0.0161125	-0.0640875\\
-0.016205	-0.05188\\
-0.0161125	-0.03357\\
-0.0154275	-0.03357\\
-0.01506	-0.03357\\
-0.0151975	-0.042725\\
-0.01538	-0.0457775\\
-0.0154725	-0.0305175\\
-0.0152425	-0.042725\\
-0.015335	-0.03662\\
-0.01529	-0.027465\\
-0.01497	-0.03662\\
-0.0151525	-0.0396725\\
-0.0151525	-0.0457775\\
-0.0155175	-0.05188\\
-0.0157025	-0.0396725\\
-0.0154275	-0.0305175\\
-0.01506	-0.01831\\
-0.014465	-0.03662\\
-0.01451	-0.0213625\\
-0.01442	-0.027465\\
-0.01419	-0.0122075\\
-0.0140525	-0.03357\\
-0.014375	-0.03357\\
-0.01474	-0.03662\\
-0.0149225	-0.03662\\
-0.01497	-0.024415\\
-0.0145575	-0.0213625\\
-0.014145	-0.01831\\
-0.0137775	-0.024415\\
-0.0137325	-0.0305175\\
-0.014145	-0.0305175\\
-0.0142825	-0.0213625\\
-0.0143275	-0.03662\\
-0.0149225	-0.05188\\
-0.01529	-0.0549325\\
-0.015565	-0.0549325\\
-0.015655	-0.0640875\\
-0.015885	-0.0396725\\
-0.0154725	-0.03662\\
-0.015105	-0.0305175\\
-0.014785	-0.0305175\\
-0.014695	-0.03662\\
-0.0149225	-0.027465\\
-0.0148775	-0.0213625\\
-0.01451	-0.01831\\
-0.0142375	-0.03662\\
-0.014695	-0.0396725\\
-0.01506	-0.03662\\
-0.015015	-0.061035\\
-0.0155175	-0.061035\\
-0.0158375	-0.042725\\
-0.015565	-0.0305175\\
-0.015105	-0.024415\\
-0.0146475	-0.0213625\\
-0.01442	-0.0396725\\
-0.0148325	-0.0305175\\
-0.0148325	-0.027465\\
-0.0149225	-0.0457775\\
-0.015105	-0.0457775\\
-0.0152425	-0.042725\\
-0.01529	-0.0488275\\
-0.015335	-0.03357\\
-0.01506	-0.0396725\\
-0.015105	-0.027465\\
-0.0148325	-0.027465\\
-0.0146025	-0.03662\\
-0.0149225	-0.0488275\\
-0.0154275	-0.05188\\
-0.01561	-0.0305175\\
-0.015105	-0.05188\\
-0.0157925	-0.05188\\
-0.0160225	-0.0396725\\
-0.0155175	-0.0213625\\
-0.0148775	-0.0457775\\
-0.01497	-0.03357\\
-0.01497	-0.03357\\
-0.0148775	-0.03662\\
-0.015015	-0.027465\\
-0.014785	-0.042725\\
-0.0148775	-0.0579825\\
-0.0157475	-0.0640875\\
-0.0160225	-0.0457775\\
-0.01561	-0.05188\\
-0.0155175	-0.0549325\\
-0.01561	-0.0488275\\
-0.0157025	-0.03357\\
-0.0154275	-0.03662\\
-0.01529	-0.027465\\
-0.0149225	-0.0457775\\
-0.01506	-0.03662\\
-0.0152425	-0.0640875\\
-0.0157475	-0.0671375\\
-0.0161125	-0.08545\\
-0.01657	-0.061035\\
-0.0164325	-0.0488275\\
-0.0160675	-0.0396725\\
-0.0157475	-0.0488275\\
-0.0157025	-0.0305175\\
-0.0154725	-0.024415\\
-0.01497	-0.0213625\\
-0.01474	-0.024415\\
-0.014695	-0.0213625\\
-0.014375	-0.01526\\
-0.013915	-0.0213625\\
-0.0141	-0.027465\\
-0.014465	-0.01526\\
-0.0142375	-0.0122075\\
-0.013915	-0.0122075\\
-0.0136875	-0.03662\\
-0.014375	-0.0457775\\
-0.01506	-0.0213625\\
-0.01451	-0.03662\\
-0.014375	-0.0305175\\
-0.01451	-0.027465\\
-0.014465	-0.03357\\
-0.0145575	-0.03357\\
-0.0145575	-0.0213625\\
-0.0142375	-0.01526\\
-0.01387	-0.0122075\\
-0.01355	-0.01526\\
-0.0136425	-0.03662\\
-0.0142375	-0.03662\\
-0.0146475	-0.027465\\
-0.01442	-0.01831\\
-0.0140525	-0.0122075\\
-0.01355	-0.01831\\
-0.01355	-0.027465\\
-0.0141	-0.03357\\
-0.01451	-0.03357\\
-0.014695	-0.01831\\
-0.014465	-0.027465\\
-0.0142825	-0.027465\\
-0.01442	-0.042725\\
-0.014695	-0.0457775\\
-0.0151525	-0.0549325\\
-0.015565	-0.0396725\\
-0.0154275	-0.0305175\\
-0.0148775	-0.03357\\
-0.014785	-0.0305175\\
-0.01474	-0.0305175\\
-0.01474	-0.01831\\
-0.01442	-0.0305175\\
-0.014375	-0.027465\\
-0.01451	-0.027465\\
-0.014465	-0.01526\\
-0.0141	-0.0122075\\
-0.013595	-0.01831\\
-0.0136875	-0.027465\\
-0.0140525	-0.0396725\\
-0.0146475	-0.0396725\\
-0.01497	-0.0457775\\
-0.0152425	-0.0457775\\
-0.01529	-0.0488275\\
-0.01538	-0.061035\\
-0.0158375	-0.0823975\\
-0.01648	-0.0640875\\
-0.016525	-0.0488275\\
-0.0160225	-0.0488275\\
-0.01593	-0.0579825\\
-0.0161125	-0.076295\\
-0.01648	-0.0488275\\
-0.0161125	-0.027465\\
-0.0151525	-0.0213625\\
-0.01442	-0.0213625\\
-0.0142375	-0.01831\\
-0.01387	-0.0122075\\
-0.013275	-0.009155\\
-0.0131375	-0.01526\\
-0.013505	-0.0213625\\
-0.013915	-0.0305175\\
-0.0140075	-0.01831\\
-0.01387	-0.01831\\
-0.0137325	-0.024415\\
-0.0139625	-0.0305175\\
-0.014145	-0.0305175\\
-0.0142825	-0.027465\\
-0.0142825	-0.024415\\
-0.0140075	-0.01526\\
-0.0137325	-0.01526\\
-0.013505	-0.024415\\
-0.0137775	-0.024415\\
-0.0140075	-0.01526\\
-0.01387	-0.0122075\\
-0.013505	-0.024415\\
-0.0137775	-0.0122075\\
-0.0136425	-0.024415\\
-0.013595	-0.024415\\
-0.013915	-0.027465\\
-0.0142375	-0.03662\\
-0.014465	-0.024415\\
-0.01419	-0.042725\\
-0.014465	-0.03662\\
-0.0146475	-0.024415\\
-0.014465	-0.01831\\
-0.0143275	-0.0396725\\
-0.014695	-0.0457775\\
-0.0151525	-0.0488275\\
-0.0152425	-0.0488275\\
-0.0154275	-0.03662\\
-0.0152425	-0.0457775\\
-0.015105	-0.03357\\
-0.01506	-0.042725\\
-0.0152425	-0.0305175\\
-0.015105	-0.0488275\\
-0.01538	-0.0457775\\
-0.0154725	-0.05188\\
-0.0154275	-0.0885\\
-0.0162975	-0.061035\\
-0.0163425	-0.03357\\
-0.015565	-0.024415\\
-0.0149225	-0.024415\\
-0.01474	-0.027465\\
-0.0149225	-0.042725\\
-0.0151975	-0.0488275\\
-0.0154275	-0.05188\\
-0.015655	-0.05188\\
-0.0157025	-0.03662\\
-0.015335	-0.03662\\
-0.01506	-0.0396725\\
-0.0151975	-0.03662\\
-0.0151975	-0.024415\\
-0.01474	-0.0213625\\
-0.0142825	-0.0305175\\
-0.0146025	-0.03357\\
-0.014695	-0.0396725\\
-0.01497	-0.0457775\\
-0.01529	-0.042725\\
-0.015335	-0.0305175\\
-0.01497	-0.024415\\
-0.014695	-0.03662\\
-0.015015	-0.0488275\\
-0.015335	-0.0579825\\
-0.0157475	-0.061035\\
-0.0160225	-0.0732425\\
-0.0163875	-0.0579825\\
-0.0162975	-0.0579825\\
-0.0161125	-0.0396725\\
-0.0155175	-0.024415\\
-0.0149225	-0.0457775\\
-0.015335	-0.0549325\\
-0.0157475	-0.0305175\\
-0.015335	-0.0488275\\
-0.01561	-0.07019\\
-0.0161125	-0.076295\\
-0.01648	-0.05188\\
-0.0160675	-0.03662\\
-0.0154275	-0.01831\\
-0.0146475	-0.03662\\
-0.0146475	-0.024415\\
-0.014695	-0.027465\\
-0.014695	-0.01831\\
-0.0142375	-0.027465\\
-0.0142375	-0.01526\\
-0.0140525	-0.027465\\
-0.014145	-0.024415\\
-0.0140075	-0.024415\\
-0.014145	-0.0396725\\
-0.0146475	-0.0305175\\
-0.01474	-0.0457775\\
-0.01506	-0.05188\\
-0.0154725	-0.05188\\
-0.01561	-0.024415\\
-0.01506	-0.03662\\
-0.0148775	-0.03662\\
-0.01506	-0.024415\\
-0.014785	-0.01831\\
-0.0145575	-0.01831\\
-0.0142375	-0.027465\\
-0.01442	-0.0213625\\
-0.014465	-0.01831\\
-0.0139625	-0.0122075\\
-0.0133675	-0.01526\\
-0.0130475	-0.01526\\
-0.0134125	-0.027465\\
-0.01387	-0.0305175\\
-0.0141	-0.0122075\\
-0.013825	-0.027465\\
-0.013825	-0.03357\\
-0.0142825	-0.0396725\\
-0.014695	-0.027465\\
-0.014465	-0.0305175\\
-0.0142825	-0.0305175\\
-0.01442	-0.0396725\\
-0.0146475	-0.0305175\\
-0.0146475	-0.03662\\
-0.01474	-0.03357\\
-0.014785	-0.027465\\
-0.014465	-0.024415\\
-0.01419	-0.027465\\
-0.0143275	-0.01831\\
-0.01419	-0.027465\\
-0.0143275	-0.03662\\
-0.014695	-0.0549325\\
-0.015335	-0.0396725\\
-0.0151975	-0.042725\\
-0.015015	-0.0579825\\
-0.0155175	-0.0488275\\
-0.015335	-0.027465\\
-0.0148325	-0.024415\\
-0.01442	-0.01831\\
-0.0140075	-0.01831\\
-0.0139625	-0.0122075\\
-0.0137775	-0.0122075\\
-0.0136875	-0.03357\\
-0.0142375	-0.0213625\\
-0.0136875	-0.024415\\
-0.013275	-0.027465\\
-0.0136425	-0.027465\\
-0.0140075	-0.01526\\
-0.0139625	-0.01526\\
-0.0136425	-0.0061025\\
-0.0131825	-0.009155\\
-0.013275	-0.027465\\
-0.0140075	-0.03662\\
-0.0145575	-0.027465\\
-0.014465	-0.024415\\
-0.0142825	-0.027465\\
-0.014375	-0.0305175\\
-0.014375	-0.03357\\
-0.01451	-0.0396725\\
-0.014785	-0.0457775\\
-0.015105	-0.042725\\
-0.0151975	-0.0396725\\
-0.01506	-0.027465\\
-0.0146475	-0.024415\\
-0.0143275	-0.0213625\\
-0.0142375	-0.0305175\\
-0.01451	-0.01831\\
-0.0142375	-0.0213625\\
-0.013915	-0.03662\\
-0.01442	-0.03662\\
-0.014785	-0.03357\\
-0.0148775	-0.042725\\
-0.015015	-0.0549325\\
-0.0154275	-0.0305175\\
-0.0148775	-0.03357\\
-0.01529	-0.0488275\\
-0.01561	-0.042725\\
-0.0154725	-0.0488275\\
-0.0155175	-0.05188\\
-0.015565	-0.076295\\
-0.01648	-0.0885\\
-0.016755	-0.0579825\\
-0.0163425	-0.0640875\\
-0.0163875	-0.079345\\
-0.0166625	-0.0732425\\
-0.0167075	-0.08545\\
-0.016845	-0.076295\\
-0.0168	-0.094605\\
-0.01712	-0.0732425\\
-0.01703	-0.05188\\
-0.016525	-0.03357\\
-0.0158375	-0.0396725\\
-0.015565	-0.027465\\
-0.015335	-0.027465\\
-0.015015	-0.03662\\
-0.0151525	-0.0488275\\
-0.01561	-0.03662\\
-0.0152425	-0.027465\\
-0.01497	-0.027465\\
-0.0149225	-0.0213625\\
-0.01451	-0.0213625\\
-0.014465	-0.01526\\
-0.01387	-0.0213625\\
-0.013595	-0.01831\\
-0.0134575	-0.0122075\\
-0.0136875	-0.01526\\
-0.013595	-0.01831\\
-0.013275	-0.009155\\
-0.012955	-0.009155\\
-0.0125875	-0.0122075\\
-0.01291	-0.0122075\\
-0.0127725	-0.0122075\\
-0.012725	-0.0213625\\
-0.013505	-0.03357\\
-0.014145	-0.0305175\\
-0.0143275	-0.024415\\
-0.0141	-0.027465\\
-0.0142825	-0.042725\\
-0.0149225	-0.0305175\\
-0.01442	-0.01526\\
-0.0137325	-0.01526\\
-0.0133675	-0.009155\\
-0.012955	-0.0122075\\
-0.0130925	-0.0213625\\
-0.0136875	-0.0305175\\
-0.0142375	-0.0213625\\
-0.0140525	-0.03662\\
-0.0145575	-0.0457775\\
-0.0151525	-0.0488275\\
-0.01529	-0.03357\\
-0.015105	-0.024415\\
-0.01442	-0.0305175\\
-0.014465	-0.0396725\\
-0.014785	-0.0488275\\
-0.0152425	-0.03357\\
-0.014695	-0.01526\\
-0.0139625	-0.024415\\
-0.0141	-0.0213625\\
-0.0139625	-0.0122075\\
-0.01332	-0.0061025\\
-0.012725	-0.01526\\
-0.0130925	-0.03662\\
-0.0142375	-0.03357\\
-0.0146475	-0.024415\\
-0.014375	-0.0213625\\
-0.013825	-0.0213625\\
-0.0136425	-0.0061025\\
-0.01268	-0.0122075\\
-0.012725	-0.0122075\\
-0.0127725	-0.01831\\
-0.0130925	-0.027465\\
-0.013595	-0.0396725\\
-0.0142825	-0.03662\\
-0.0146475	-0.0396725\\
-0.014785	-0.0396725\\
-0.0149225	-0.0396725\\
-0.01497	-0.0396725\\
-0.0149225	-0.03662\\
-0.014785	-0.042725\\
-0.0148775	-0.061035\\
-0.015565	-0.0549325\\
-0.0157925	-0.0305175\\
-0.01506	-0.01526\\
-0.0141	-0.0213625\\
-0.01474	-0.0457775\\
-0.0151525	-0.0396725\\
-0.015105	-0.024415\\
-0.0145575	-0.0213625\\
-0.0143275	-0.0396725\\
-0.014785	-0.0549325\\
-0.0155175	-0.0579825\\
-0.0158375	-0.0640875\\
-0.0160675	-0.0671375\\
-0.0161125	-0.0488275\\
-0.0158375	-0.0488275\\
-0.0157925	-0.0305175\\
-0.0149225	-0.027465\\
-0.014465	-0.0305175\\
-0.0146025	-0.03357\\
-0.0148775	-0.03357\\
-0.015015	-0.03662\\
-0.01506	-0.024415\\
-0.01474	-0.0396725\\
-0.014785	-0.027465\\
-0.0148325	-0.01831\\
-0.0142375	-0.0122075\\
-0.013595	-0.01526\\
-0.0131825	-0.01526\\
-0.0134125	-0.01526\\
-0.013275	-0.0061025\\
-0.0127725	-0.0122075\\
-0.0130925	-0.027465\\
-0.0140075	-0.03357\\
-0.01387	-0.0213625\\
-0.0136425	-0.0305175\\
-0.013915	-0.0457775\\
-0.0146025	-0.027465\\
-0.01451	-0.0122075\\
-0.0137775	-0.0061025\\
-0.0130925	-0.01831\\
-0.0136875	-0.0457775\\
-0.01474	-0.0549325\\
-0.01538	-0.07019\\
-0.0160225	-0.0823975\\
-0.0166625	-0.079345\\
-0.016845	-0.061035\\
-0.0163425	-0.0579825\\
-0.016205	-0.076295\\
-0.01657	-0.061035\\
-0.01648	-0.0579825\\
-0.0163425	-0.0671375\\
-0.0164325	-0.076295\\
-0.01657	-0.07019\\
-0.0166625	-0.0915525\\
-0.01712	-0.0671375\\
-0.016755	-0.0396725\\
-0.0160225	-0.027465\\
-0.01497	-0.01526\\
-0.014375	-0.01526\\
-0.0142825	-0.01831\\
-0.0142825	-0.0213625\\
-0.0142375	-0.03357\\
-0.0148325	-0.0305175\\
-0.014785	-0.03357\\
-0.014695	-0.03662\\
-0.015015	-0.024415\\
-0.0146025	-0.027465\\
-0.014465	-0.03662\\
-0.0148325	-0.0396725\\
-0.01506	-0.027465\\
-0.0146025	-0.01526\\
-0.0140525	-0.024415\\
-0.014145	-0.0213625\\
-0.014145	-0.0549325\\
-0.0152425	-0.0579825\\
-0.0160675	-0.061035\\
-0.0160675	-0.0640875\\
-0.01625	-0.076295\\
-0.016525	-0.0885\\
-0.01703	-0.079345\\
-0.017165	-0.05188\\
-0.016525	-0.0396725\\
-0.015885	-0.024415\\
-0.0149225	-0.027465\\
-0.0146475	-0.0305175\\
-0.0146475	-0.0305175\\
-0.0148775	-0.0213625\\
-0.0146475	-0.027465\\
-0.01442	-0.0213625\\
-0.014465	-0.027465\\
-0.0146475	-0.0305175\\
-0.014785	-0.03662\\
-0.01506	-0.03357\\
-0.0148325	-0.01526\\
-0.014145	-0.009155\\
-0.01355	-0.0122075\\
-0.0131375	-0.009155\\
-0.012955	-0.01526\\
-0.0130925	-0.01831\\
-0.013595	-0.024415\\
-0.0137775	-0.024415\\
-0.0140525	-0.0396725\\
-0.0146475	-0.0305175\\
-0.0145575	-0.03357\\
-0.01506	-0.0671375\\
-0.01593	-0.0549325\\
-0.015885	-0.042725\\
-0.0157025	-0.0579825\\
-0.01593	-0.061035\\
-0.0161125	-0.0488275\\
-0.0157475	-0.042725\\
-0.0154725	-0.027465\\
-0.01497	-0.0305175\\
-0.01474	-0.05188\\
-0.015565	-0.0823975\\
-0.01657	-0.094605\\
-0.0172125	-0.0732425\\
-0.017075	-0.0671375\\
-0.0168	-0.0488275\\
-0.0164325	-0.0579825\\
-0.0163425	-0.07019\\
-0.0167075	-0.05188\\
-0.01616	-0.03357\\
-0.015655	-0.03357\\
-0.01538	-0.042725\\
-0.015885	-0.0488275\\
-0.0160675	-0.0305175\\
-0.01538	-0.027465\\
-0.015105	-0.0305175\\
-0.014695	-0.027465\\
-0.01506	-0.05188\\
-0.015655	-0.0549325\\
-0.0158375	-0.042725\\
-0.015655	-0.03357\\
-0.01529	-0.0305175\\
-0.01529	-0.027465\\
-0.01506	-0.01831\\
-0.014695	-0.01526\\
-0.0142825	-0.01526\\
-0.0139625	-0.0122075\\
-0.0136425	-0.0122075\\
-0.01355	-0.0213625\\
-0.0141	-0.03357\\
-0.0146475	-0.03662\\
-0.014785	-0.024415\\
-0.01442	-0.01831\\
-0.01419	-0.01831\\
-0.0139625	-0.0213625\\
-0.0140075	-0.0305175\\
-0.0143275	-0.03357\\
-0.01451	-0.027465\\
-0.0145575	-0.0213625\\
-0.0140525	-0.05188\\
-0.01451	-0.042725\\
-0.0152425	-0.03357\\
-0.01506	-0.01526\\
-0.0141	-0.0213625\\
-0.013825	-0.027465\\
-0.0142825	-0.0305175\\
-0.014375	-0.0213625\\
-0.0143275	-0.0213625\\
-0.013915	-0.03357\\
-0.0143275	-0.042725\\
-0.0148775	-0.027465\\
-0.0145575	-0.01526\\
-0.013505	-0.0213625\\
-0.0133675	-0.0305175\\
-0.0142825	-0.0396725\\
-0.0148325	-0.0213625\\
-0.014375	-0.0213625\\
-0.0141	-0.0396725\\
-0.014695	-0.05188\\
-0.015335	-0.0396725\\
-0.015335	-0.0549325\\
-0.0157475	-0.0732425\\
-0.01625	-0.0488275\\
-0.01593	-0.042725\\
-0.015565	-0.05188\\
-0.015885	-0.042725\\
-0.015565	-0.0457775\\
-0.0157025	-0.0640875\\
-0.0160225	-0.0457775\\
-0.015885	-0.027465\\
-0.01506	-0.0305175\\
-0.01529	-0.0671375\\
-0.016205	-0.079345\\
-0.0166175	-0.0579825\\
-0.0163425	-0.0488275\\
-0.0160675	-0.0640875\\
-0.0162975	-0.05188\\
-0.01616	-0.03357\\
-0.015565	-0.0305175\\
-0.015335	-0.0396725\\
-0.01538	-0.042725\\
-0.0154725	-0.03662\\
-0.0154725	-0.042725\\
-0.01561	-0.024415\\
-0.01538	-0.03357\\
-0.01529	-0.0488275\\
-0.01561	-0.03662\\
-0.0151975	-0.027465\\
-0.014785	-0.024415\\
-0.01451	-0.01526\\
-0.0141	-0.01526\\
-0.0141	-0.0305175\\
-0.0145575	-0.0305175\\
-0.0146475	-0.0396725\\
-0.0148775	-0.0457775\\
-0.015335	-0.0549325\\
-0.0157025	-0.0457775\\
-0.015565	-0.03357\\
-0.01538	-0.027465\\
-0.0145575	-0.024415\\
-0.0145575	-0.05188\\
-0.015335	-0.03357\\
-0.015105	-0.0122075\\
-0.01442	-0.024415\\
-0.01474	-0.0549325\\
-0.0154275	-0.0457775\\
-0.01561	-0.0579825\\
-0.015975	-0.079345\\
-0.016525	-0.0671375\\
-0.0163425	-0.0457775\\
-0.0160675	-0.0579825\\
-0.016205	-0.0457775\\
-0.0157925	-0.0305175\\
-0.0154275	-0.03662\\
-0.0154275	-0.0457775\\
-0.0157025	-0.0488275\\
-0.0158375	-0.05188\\
-0.0158375	-0.0396725\\
-0.015335	-0.027465\\
-0.0149225	-0.0213625\\
-0.014695	-0.0213625\\
-0.0148325	-0.0305175\\
-0.0148325	-0.0213625\\
-0.014695	-0.042725\\
-0.0152425	-0.0549325\\
-0.015565	-0.03662\\
-0.0152425	-0.024415\\
-0.01497	-0.042725\\
-0.0151975	-0.0305175\\
-0.014785	-0.01526\\
-0.0141	-0.01831\\
-0.0140525	-0.0213625\\
-0.0142825	-0.01831\\
-0.0142825	-0.0122075\\
-0.013915	-0.009155\\
-0.01332	-0.0122075\\
-0.0134575	-0.0213625\\
-0.0139625	-0.03357\\
-0.0145575	-0.03662\\
-0.0149225	-0.0549325\\
-0.0154275	-0.0579825\\
-0.0157925	-0.042725\\
-0.0151975	-0.01831\\
-0.014695	-0.0488275\\
-0.0154275	-0.076295\\
-0.01625	-0.05188\\
-0.01616	-0.05188\\
-0.0160225	-0.076295\\
-0.016525	-0.0640875\\
-0.0164325	-0.0549325\\
-0.016205	-0.0457775\\
-0.0158375	-0.03662\\
-0.0157025	-0.05188\\
-0.015975	-0.0457775\\
-0.015655	-0.0305175\\
-0.015335	-0.0488275\\
-0.015885	-0.07019\\
-0.01625	-0.0640875\\
-0.0164325	-0.0305175\\
-0.0154725	-0.01526\\
-0.0145575	-0.024415\\
-0.0151975	-0.0305175\\
-0.015105	-0.01831\\
-0.0142825	-0.009155\\
-0.0136425	-0.01831\\
-0.013915	-0.024415\\
-0.014465	-0.042725\\
-0.0151975	-0.03357\\
-0.01506	-0.0213625\\
-0.01474	-0.0213625\\
-0.01419	-0.009155\\
-0.01355	-0.01526\\
-0.013595	-0.0122075\\
-0.013505	-0.0213625\\
-0.01355	-0.027465\\
-0.014145	-0.0305175\\
-0.0142375	-0.0122075\\
-0.0136425	-0.009155\\
-0.0133675	-0.01526\\
-0.01355	-0.01831\\
-0.0137325	-0.01526\\
-0.0134575	-0.0122075\\
-0.013595	-0.024415\\
-0.0134575	-0.0122075\\
-0.0131825	-0.0213625\\
-0.0137325	-0.03662\\
-0.0143275	-0.042725\\
-0.015105	-0.061035\\
-0.0158375	-0.0549325\\
-0.015975	-0.0305175\\
-0.0151525	-0.01831\\
-0.0145575	-0.01831\\
-0.01442	-0.01831\\
-0.0137775	-0.0122075\\
-0.0134125	-0.0213625\\
-0.013915	-0.03357\\
-0.01419	-0.0213625\\
-0.01419	-0.024415\\
-0.0142825	-0.0305175\\
-0.014465	-0.03357\\
-0.0146025	-0.03662\\
-0.014695	-0.0396725\\
-0.01497	-0.03662\\
-0.015105	-0.0579825\\
-0.0155175	-0.0457775\\
-0.01506	-0.0213625\\
-0.0140525	-0.0122075\\
-0.0137325	-0.024415\\
-0.01419	-0.027465\\
-0.01419	-0.0122075\\
-0.0140525	-0.01526\\
-0.0133675	-0.0122075\\
-0.01332	-0.0213625\\
-0.0131825	-0.0030525\\
-0.0125875	-0.0030525\\
-0.01268	-0.0213625\\
-0.013275	-0.024415\\
-0.0137325	-0.03662\\
-0.01419	-0.0457775\\
-0.0149225	-0.03662\\
-0.0148325	-0.01831\\
-0.013825	-0.01831\\
-0.0140525	-0.042725\\
-0.01442	-0.01526\\
-0.0137775	-0.01831\\
-0.0137775	-0.042725\\
-0.0145575	-0.03662\\
-0.014695	-0.05188\\
-0.015335	-0.0732425\\
-0.015885	-0.042725\\
-0.0154725	-0.027465\\
-0.015105	-0.0305175\\
-0.0149225	-0.03357\\
-0.0151525	-0.05188\\
-0.01561	-0.042725\\
-0.01538	-0.024415\\
-0.0148775	-0.0305175\\
};
\addplot [color=mycolor2, line width=2.0pt, forget plot]
  table[row sep=crcr]{%
-0.01561	-0.01561\\
-0.0157025	-0.0157025\\
-0.0158375	-0.0158375\\
-0.015655	-0.015655\\
-0.01497	-0.01497\\
-0.0148325	-0.0148325\\
-0.01497	-0.01497\\
-0.0148775	-0.0148775\\
-0.0140075	-0.0140075\\
-0.01323	-0.01323\\
-0.0127725	-0.0127725\\
-0.0136875	-0.0136875\\
-0.014695	-0.014695\\
-0.01497	-0.01497\\
-0.015105	-0.015105\\
-0.0148325	-0.0148325\\
-0.0142825	-0.0142825\\
-0.0148775	-0.0148775\\
-0.01474	-0.01474\\
-0.014375	-0.014375\\
-0.014695	-0.014695\\
-0.0148325	-0.0148325\\
-0.0146025	-0.0146025\\
-0.01506	-0.01506\\
-0.0151975	-0.0151975\\
-0.0149225	-0.0149225\\
-0.0152425	-0.0152425\\
-0.01538	-0.01538\\
-0.0151525	-0.0151525\\
-0.0149225	-0.0149225\\
-0.014695	-0.014695\\
-0.0149225	-0.0149225\\
-0.01497	-0.01497\\
-0.0149225	-0.0149225\\
-0.01497	-0.01497\\
-0.01506	-0.01506\\
-0.0154725	-0.0154725\\
-0.015565	-0.015565\\
-0.0151525	-0.0151525\\
-0.0149225	-0.0149225\\
-0.014375	-0.014375\\
-0.0141	-0.0141\\
-0.0143275	-0.0143275\\
-0.0142375	-0.0142375\\
-0.01419	-0.01419\\
-0.014465	-0.014465\\
-0.014695	-0.014695\\
-0.0151975	-0.0151975\\
-0.0151525	-0.0151525\\
-0.01451	-0.01451\\
-0.014145	-0.014145\\
-0.0142825	-0.0142825\\
-0.014465	-0.014465\\
-0.0140525	-0.0140525\\
-0.0137325	-0.0137325\\
-0.0140075	-0.0140075\\
-0.0139625	-0.0139625\\
-0.0137325	-0.0137325\\
-0.01387	-0.01387\\
-0.0140525	-0.0140525\\
-0.0145575	-0.0145575\\
-0.01474	-0.01474\\
-0.0149225	-0.0149225\\
-0.015105	-0.015105\\
-0.015015	-0.015015\\
-0.0148775	-0.0148775\\
-0.014785	-0.014785\\
-0.01474	-0.01474\\
-0.014695	-0.014695\\
-0.0152425	-0.0152425\\
-0.0160225	-0.0160225\\
-0.0162975	-0.0162975\\
-0.0163425	-0.0163425\\
-0.0157925	-0.0157925\\
-0.0160675	-0.0160675\\
-0.01648	-0.01648\\
-0.0166175	-0.0166175\\
-0.0163875	-0.0163875\\
-0.0166175	-0.0166175\\
-0.01712	-0.01712\\
-0.0169825	-0.0169825\\
-0.0166625	-0.0166625\\
-0.0161125	-0.0161125\\
-0.015655	-0.015655\\
-0.015335	-0.015335\\
-0.0151525	-0.0151525\\
-0.0145575	-0.0145575\\
-0.01419	-0.01419\\
-0.0142825	-0.0142825\\
-0.014695	-0.014695\\
-0.014785	-0.014785\\
-0.015015	-0.015015\\
-0.0151525	-0.0151525\\
-0.015655	-0.015655\\
-0.0157475	-0.0157475\\
-0.015565	-0.015565\\
-0.01529	-0.01529\\
-0.0154275	-0.0154275\\
-0.0151975	-0.0151975\\
-0.0149225	-0.0149225\\
-0.01506	-0.01506\\
-0.015105	-0.015105\\
-0.0148775	-0.0148775\\
-0.01474	-0.01474\\
-0.01451	-0.01451\\
-0.014375	-0.014375\\
-0.01451	-0.01451\\
-0.014145	-0.014145\\
-0.01442	-0.01442\\
-0.0149225	-0.0149225\\
-0.015105	-0.015105\\
-0.0151975	-0.0151975\\
-0.0148325	-0.0148325\\
-0.015015	-0.015015\\
-0.015655	-0.015655\\
-0.0155175	-0.0155175\\
-0.01561	-0.01561\\
-0.015655	-0.015655\\
-0.0151975	-0.0151975\\
-0.014695	-0.014695\\
-0.01474	-0.01474\\
-0.0149225	-0.0149225\\
-0.0155175	-0.0155175\\
-0.0160225	-0.0160225\\
-0.0161125	-0.0161125\\
-0.015885	-0.015885\\
-0.0157925	-0.0157925\\
-0.0157475	-0.0157475\\
-0.015655	-0.015655\\
-0.01561	-0.01561\\
-0.0152425	-0.0152425\\
-0.0149225	-0.0149225\\
-0.014785	-0.014785\\
-0.0146475	-0.0146475\\
-0.015015	-0.015015\\
-0.01561	-0.01561\\
-0.0151525	-0.0151525\\
-0.0148325	-0.0148325\\
-0.01474	-0.01474\\
-0.0142375	-0.0142375\\
-0.0137325	-0.0137325\\
-0.0136425	-0.0136425\\
-0.014145	-0.014145\\
-0.0141	-0.0141\\
-0.0140525	-0.0140525\\
-0.014145	-0.014145\\
-0.0141	-0.0141\\
-0.01387	-0.01387\\
-0.0134125	-0.0134125\\
-0.0130475	-0.0130475\\
-0.013275	-0.013275\\
-0.01419	-0.01419\\
-0.01497	-0.01497\\
-0.015335	-0.015335\\
-0.01497	-0.01497\\
-0.01451	-0.01451\\
-0.0140525	-0.0140525\\
-0.013505	-0.013505\\
-0.0140525	-0.0140525\\
-0.0142375	-0.0142375\\
-0.0146025	-0.0146025\\
-0.01506	-0.01506\\
-0.01593	-0.01593\\
-0.016525	-0.016525\\
-0.0164325	-0.0164325\\
-0.0163875	-0.0163875\\
-0.0160225	-0.0160225\\
-0.015885	-0.015885\\
-0.015975	-0.015975\\
-0.0160675	-0.0160675\\
-0.0158375	-0.0158375\\
-0.015655	-0.015655\\
-0.01561	-0.01561\\
-0.0154725	-0.0154725\\
-0.01561	-0.01561\\
-0.01538	-0.01538\\
-0.0157025	-0.0157025\\
-0.0161125	-0.0161125\\
-0.0158375	-0.0158375\\
-0.0152425	-0.0152425\\
-0.015105	-0.015105\\
-0.01561	-0.01561\\
-0.0160225	-0.0160225\\
-0.0163875	-0.0163875\\
-0.0166175	-0.0166175\\
-0.0167075	-0.0167075\\
-0.0164325	-0.0164325\\
-0.01625	-0.01625\\
-0.0163425	-0.0163425\\
-0.016525	-0.016525\\
-0.0161125	-0.0161125\\
-0.0154275	-0.0154275\\
-0.0155175	-0.0155175\\
-0.0154275	-0.0154275\\
-0.0149225	-0.0149225\\
-0.01497	-0.01497\\
-0.0151525	-0.0151525\\
-0.01497	-0.01497\\
-0.0146475	-0.0146475\\
-0.01474	-0.01474\\
-0.014695	-0.014695\\
-0.0148775	-0.0148775\\
-0.015335	-0.015335\\
-0.015015	-0.015015\\
-0.0148775	-0.0148775\\
-0.01529	-0.01529\\
-0.0160675	-0.0160675\\
-0.0160225	-0.0160225\\
-0.0154275	-0.0154275\\
-0.015015	-0.015015\\
-0.0149225	-0.0149225\\
-0.0146025	-0.0146025\\
-0.0145575	-0.0145575\\
-0.0146475	-0.0146475\\
-0.0149225	-0.0149225\\
-0.015105	-0.015105\\
-0.01474	-0.01474\\
-0.015015	-0.015015\\
-0.01538	-0.01538\\
-0.0154275	-0.0154275\\
-0.0152425	-0.0152425\\
-0.01561	-0.01561\\
-0.01529	-0.01529\\
-0.014375	-0.014375\\
-0.0142375	-0.0142375\\
-0.014375	-0.014375\\
-0.01474	-0.01474\\
-0.014465	-0.014465\\
-0.01474	-0.01474\\
-0.0148325	-0.0148325\\
-0.0145575	-0.0145575\\
-0.0146475	-0.0146475\\
-0.0146025	-0.0146025\\
-0.01451	-0.01451\\
-0.0145575	-0.0145575\\
-0.01419	-0.01419\\
-0.0141	-0.0141\\
-0.01387	-0.01387\\
-0.013825	-0.013825\\
-0.014375	-0.014375\\
-0.014695	-0.014695\\
-0.015105	-0.015105\\
-0.015655	-0.015655\\
-0.01538	-0.01538\\
-0.01474	-0.01474\\
-0.0142825	-0.0142825\\
-0.0141	-0.0141\\
-0.014375	-0.014375\\
-0.0140525	-0.0140525\\
-0.0139625	-0.0139625\\
-0.01419	-0.01419\\
-0.014785	-0.014785\\
-0.01497	-0.01497\\
-0.01529	-0.01529\\
-0.01506	-0.01506\\
-0.0148325	-0.0148325\\
-0.0142375	-0.0142375\\
-0.013915	-0.013915\\
-0.0134125	-0.0134125\\
-0.01323	-0.01323\\
-0.0133675	-0.0133675\\
-0.0140525	-0.0140525\\
-0.0142375	-0.0142375\\
-0.014465	-0.014465\\
-0.0146025	-0.0146025\\
-0.0152425	-0.0152425\\
-0.0154275	-0.0154275\\
-0.01506	-0.01506\\
-0.0151525	-0.0151525\\
-0.0152425	-0.0152425\\
-0.01561	-0.01561\\
-0.0155175	-0.0155175\\
-0.01506	-0.01506\\
-0.0143275	-0.0143275\\
-0.0136875	-0.0136875\\
-0.013505	-0.013505\\
-0.01355	-0.01355\\
-0.013505	-0.013505\\
-0.0136875	-0.0136875\\
-0.01419	-0.01419\\
-0.0143275	-0.0143275\\
-0.01451	-0.01451\\
-0.014145	-0.014145\\
-0.0134575	-0.0134575\\
-0.01387	-0.01387\\
-0.01442	-0.01442\\
-0.0146025	-0.0146025\\
-0.014695	-0.014695\\
-0.0146475	-0.0146475\\
-0.014375	-0.014375\\
-0.0145575	-0.0145575\\
-0.015015	-0.015015\\
-0.0151975	-0.0151975\\
-0.0148775	-0.0148775\\
-0.01538	-0.01538\\
-0.015335	-0.015335\\
-0.0151975	-0.0151975\\
-0.01538	-0.01538\\
-0.0151525	-0.0151525\\
-0.0149225	-0.0149225\\
-0.01529	-0.01529\\
-0.015975	-0.015975\\
-0.0158375	-0.0158375\\
-0.0151975	-0.0151975\\
-0.0149225	-0.0149225\\
-0.01497	-0.01497\\
-0.0151525	-0.0151525\\
-0.01538	-0.01538\\
-0.0151525	-0.0151525\\
-0.0154275	-0.0154275\\
-0.01561	-0.01561\\
-0.01529	-0.01529\\
-0.01497	-0.01497\\
-0.015105	-0.015105\\
-0.0157475	-0.0157475\\
-0.0157925	-0.0157925\\
-0.01529	-0.01529\\
-0.01497	-0.01497\\
-0.0149225	-0.0149225\\
-0.014375	-0.014375\\
-0.01387	-0.01387\\
-0.0133675	-0.0133675\\
-0.013	-0.013\\
-0.013595	-0.013595\\
-0.01419	-0.01419\\
-0.0145575	-0.0145575\\
-0.014375	-0.014375\\
-0.01451	-0.01451\\
-0.01419	-0.01419\\
-0.0140525	-0.0140525\\
-0.013825	-0.013825\\
-0.01387	-0.01387\\
-0.0143275	-0.0143275\\
-0.014785	-0.014785\\
-0.014465	-0.014465\\
-0.0141	-0.0141\\
-0.01387	-0.01387\\
-0.0139625	-0.0139625\\
-0.0136425	-0.0136425\\
-0.0143275	-0.0143275\\
-0.0151975	-0.0151975\\
-0.0151525	-0.0151525\\
-0.0154725	-0.0154725\\
-0.015885	-0.015885\\
-0.015975	-0.015975\\
-0.0157475	-0.0157475\\
-0.015335	-0.015335\\
-0.0151975	-0.0151975\\
-0.015335	-0.015335\\
-0.0155175	-0.0155175\\
-0.01561	-0.01561\\
-0.015885	-0.015885\\
-0.016205	-0.016205\\
-0.01657	-0.01657\\
-0.0162975	-0.0162975\\
-0.016205	-0.016205\\
-0.0164325	-0.0164325\\
-0.01625	-0.01625\\
-0.01616	-0.01616\\
-0.0154275	-0.0154275\\
-0.014785	-0.014785\\
-0.0146025	-0.0146025\\
-0.0149225	-0.0149225\\
-0.0148325	-0.0148325\\
-0.01442	-0.01442\\
-0.01451	-0.01451\\
-0.0142825	-0.0142825\\
-0.013915	-0.013915\\
-0.0139625	-0.0139625\\
-0.0140525	-0.0140525\\
-0.013825	-0.013825\\
-0.0142375	-0.0142375\\
-0.01474	-0.01474\\
-0.0146025	-0.0146025\\
-0.015015	-0.015015\\
-0.0157475	-0.0157475\\
-0.015885	-0.015885\\
-0.016205	-0.016205\\
-0.015975	-0.015975\\
-0.015885	-0.015885\\
-0.015565	-0.015565\\
-0.0149225	-0.0149225\\
-0.014695	-0.014695\\
-0.0143275	-0.0143275\\
-0.014465	-0.014465\\
-0.0140075	-0.0140075\\
-0.0134125	-0.0134125\\
-0.0133675	-0.0133675\\
-0.013915	-0.013915\\
-0.01451	-0.01451\\
-0.0148775	-0.0148775\\
-0.015015	-0.015015\\
-0.0151525	-0.0151525\\
-0.01506	-0.01506\\
-0.0148325	-0.0148325\\
-0.0146475	-0.0146475\\
-0.01497	-0.01497\\
-0.01474	-0.01474\\
-0.01442	-0.01442\\
-0.014695	-0.014695\\
-0.0151975	-0.0151975\\
-0.015105	-0.015105\\
-0.01474	-0.01474\\
-0.01451	-0.01451\\
-0.0145575	-0.0145575\\
-0.0142375	-0.0142375\\
-0.0140075	-0.0140075\\
-0.0142375	-0.0142375\\
-0.014465	-0.014465\\
-0.014695	-0.014695\\
-0.0145575	-0.0145575\\
-0.014695	-0.014695\\
-0.01474	-0.01474\\
-0.0142825	-0.0142825\\
-0.0141	-0.0141\\
-0.0146475	-0.0146475\\
-0.01529	-0.01529\\
-0.0154275	-0.0154275\\
-0.01529	-0.01529\\
-0.0152425	-0.0152425\\
-0.015015	-0.015015\\
-0.0148775	-0.0148775\\
-0.0146475	-0.0146475\\
-0.0148325	-0.0148325\\
-0.014375	-0.014375\\
-0.0145575	-0.0145575\\
-0.01474	-0.01474\\
-0.015015	-0.015015\\
-0.0151525	-0.0151525\\
-0.0151975	-0.0151975\\
-0.01561	-0.01561\\
-0.0160675	-0.0160675\\
-0.0164325	-0.0164325\\
-0.0160675	-0.0160675\\
-0.015335	-0.015335\\
-0.014695	-0.014695\\
-0.0140525	-0.0140525\\
-0.0141	-0.0141\\
-0.014145	-0.014145\\
-0.0146025	-0.0146025\\
-0.014695	-0.014695\\
-0.01451	-0.01451\\
-0.01474	-0.01474\\
-0.01497	-0.01497\\
-0.0151975	-0.0151975\\
-0.01538	-0.01538\\
-0.0160225	-0.0160225\\
-0.0161125	-0.0161125\\
-0.0158375	-0.0158375\\
-0.01538	-0.01538\\
-0.0151525	-0.0151525\\
-0.0151975	-0.0151975\\
-0.01538	-0.01538\\
-0.0151525	-0.0151525\\
-0.015015	-0.015015\\
-0.01497	-0.01497\\
-0.01442	-0.01442\\
-0.0146025	-0.0146025\\
-0.015105	-0.015105\\
-0.0149225	-0.0149225\\
-0.0148775	-0.0148775\\
-0.015565	-0.015565\\
-0.0155175	-0.0155175\\
-0.01538	-0.01538\\
-0.0157475	-0.0157475\\
-0.0158375	-0.0158375\\
-0.0154275	-0.0154275\\
-0.01474	-0.01474\\
-0.0146025	-0.0146025\\
-0.015015	-0.015015\\
-0.0158375	-0.0158375\\
-0.0161125	-0.0161125\\
-0.015975	-0.015975\\
-0.01538	-0.01538\\
-0.015105	-0.015105\\
-0.0146475	-0.0146475\\
-0.01442	-0.01442\\
-0.0142375	-0.0142375\\
-0.014465	-0.014465\\
-0.0146025	-0.0146025\\
-0.0142375	-0.0142375\\
-0.0143275	-0.0143275\\
-0.014785	-0.014785\\
-0.01497	-0.01497\\
-0.01538	-0.01538\\
-0.0157925	-0.0157925\\
-0.0163425	-0.0163425\\
-0.01616	-0.01616\\
-0.015885	-0.015885\\
-0.0157025	-0.0157025\\
-0.0152425	-0.0152425\\
-0.0158375	-0.0158375\\
-0.01616	-0.01616\\
-0.015655	-0.015655\\
-0.0149225	-0.0149225\\
-0.0142825	-0.0142825\\
-0.0133675	-0.0133675\\
-0.0130475	-0.0130475\\
-0.013505	-0.013505\\
-0.0137325	-0.0137325\\
-0.01419	-0.01419\\
-0.0141	-0.0141\\
-0.013825	-0.013825\\
-0.0137775	-0.0137775\\
-0.0137325	-0.0137325\\
-0.0136425	-0.0136425\\
-0.0137325	-0.0137325\\
-0.0140525	-0.0140525\\
-0.01474	-0.01474\\
-0.01506	-0.01506\\
-0.0152425	-0.0152425\\
-0.0157025	-0.0157025\\
-0.015885	-0.015885\\
-0.016205	-0.016205\\
-0.01625	-0.01625\\
-0.01593	-0.01593\\
-0.0154275	-0.0154275\\
-0.0152425	-0.0152425\\
-0.0157475	-0.0157475\\
-0.0161125	-0.0161125\\
-0.0157925	-0.0157925\\
-0.01529	-0.01529\\
-0.014375	-0.014375\\
-0.0136875	-0.0136875\\
-0.013505	-0.013505\\
-0.012955	-0.012955\\
-0.0133675	-0.0133675\\
-0.013825	-0.013825\\
-0.0139625	-0.0139625\\
-0.013915	-0.013915\\
-0.0134575	-0.0134575\\
-0.0131375	-0.0131375\\
-0.01323	-0.01323\\
-0.01332	-0.01332\\
-0.0134575	-0.0134575\\
-0.013275	-0.013275\\
-0.0130475	-0.0130475\\
-0.0134575	-0.0134575\\
-0.014695	-0.014695\\
-0.015335	-0.015335\\
-0.015565	-0.015565\\
-0.01538	-0.01538\\
-0.015655	-0.015655\\
-0.0163425	-0.0163425\\
-0.0164325	-0.0164325\\
-0.01648	-0.01648\\
-0.016205	-0.016205\\
-0.0160675	-0.0160675\\
-0.0161125	-0.0161125\\
-0.01561	-0.01561\\
-0.0154275	-0.0154275\\
-0.0148325	-0.0148325\\
-0.01451	-0.01451\\
-0.015105	-0.015105\\
-0.0148775	-0.0148775\\
-0.014785	-0.014785\\
-0.01497	-0.01497\\
-0.0148775	-0.0148775\\
-0.01497	-0.01497\\
-0.01506	-0.01506\\
-0.0151525	-0.0151525\\
-0.015105	-0.015105\\
-0.0148325	-0.0148325\\
-0.0148775	-0.0148775\\
-0.0143275	-0.0143275\\
-0.013275	-0.013275\\
-0.0131825	-0.0131825\\
-0.0136875	-0.0136875\\
-0.013275	-0.013275\\
-0.0124975	-0.0124975\\
-0.012635	-0.012635\\
-0.013275	-0.013275\\
-0.0136425	-0.0136425\\
-0.0140075	-0.0140075\\
-0.01451	-0.01451\\
-0.0146475	-0.0146475\\
-0.0142375	-0.0142375\\
-0.0145575	-0.0145575\\
-0.0140525	-0.0140525\\
-0.0136425	-0.0136425\\
-0.01323	-0.01323\\
-0.0127725	-0.0127725\\
-0.01291	-0.01291\\
-0.0128175	-0.0128175\\
-0.0131825	-0.0131825\\
-0.0139625	-0.0139625\\
-0.0141	-0.0141\\
-0.01387	-0.01387\\
-0.0137325	-0.0137325\\
-0.01387	-0.01387\\
-0.013595	-0.013595\\
-0.0134575	-0.0134575\\
-0.0131825	-0.0131825\\
-0.013275	-0.013275\\
-0.01323	-0.01323\\
-0.0136875	-0.0136875\\
-0.013825	-0.013825\\
-0.013595	-0.013595\\
-0.01355	-0.01355\\
-0.01332	-0.01332\\
-0.01419	-0.01419\\
-0.0149225	-0.0149225\\
-0.014695	-0.014695\\
-0.01451	-0.01451\\
-0.01419	-0.01419\\
-0.0145575	-0.0145575\\
-0.014695	-0.014695\\
-0.0142825	-0.0142825\\
-0.0141	-0.0141\\
-0.0143275	-0.0143275\\
-0.0139625	-0.0139625\\
-0.0137775	-0.0137775\\
-0.013915	-0.013915\\
-0.0142375	-0.0142375\\
-0.0140525	-0.0140525\\
-0.0139625	-0.0139625\\
-0.0140075	-0.0140075\\
-0.014465	-0.014465\\
-0.01451	-0.01451\\
-0.01497	-0.01497\\
-0.01561	-0.01561\\
-0.015565	-0.015565\\
-0.015105	-0.015105\\
-0.0146025	-0.0146025\\
-0.0149225	-0.0149225\\
-0.01529	-0.01529\\
-0.01506	-0.01506\\
-0.0152425	-0.0152425\\
-0.014785	-0.014785\\
-0.0139625	-0.0139625\\
-0.0136425	-0.0136425\\
-0.014465	-0.014465\\
-0.014695	-0.014695\\
-0.01451	-0.01451\\
-0.015015	-0.015015\\
-0.01506	-0.01506\\
-0.0149225	-0.0149225\\
-0.0145575	-0.0145575\\
-0.0146475	-0.0146475\\
-0.01497	-0.01497\\
-0.0148775	-0.0148775\\
-0.0148325	-0.0148325\\
-0.01474	-0.01474\\
-0.01442	-0.01442\\
-0.01419	-0.01419\\
-0.0141	-0.0141\\
-0.0146475	-0.0146475\\
-0.0149225	-0.0149225\\
-0.01506	-0.01506\\
-0.015015	-0.015015\\
-0.014695	-0.014695\\
-0.0142825	-0.0142825\\
-0.0142375	-0.0142375\\
-0.01442	-0.01442\\
-0.014695	-0.014695\\
-0.015015	-0.015015\\
-0.015335	-0.015335\\
-0.0151525	-0.0151525\\
-0.0146025	-0.0146025\\
-0.01497	-0.01497\\
-0.01561	-0.01561\\
-0.015335	-0.015335\\
-0.01529	-0.01529\\
-0.0154275	-0.0154275\\
-0.0152425	-0.0152425\\
-0.01497	-0.01497\\
-0.015015	-0.015015\\
-0.0149225	-0.0149225\\
-0.015015	-0.015015\\
-0.0151975	-0.0151975\\
-0.015105	-0.015105\\
-0.0151525	-0.0151525\\
-0.015105	-0.015105\\
-0.01506	-0.01506\\
-0.0148775	-0.0148775\\
-0.01497	-0.01497\\
-0.014785	-0.014785\\
-0.014695	-0.014695\\
-0.0148325	-0.0148325\\
-0.0142375	-0.0142375\\
-0.0134575	-0.0134575\\
-0.01291	-0.01291\\
-0.0130475	-0.0130475\\
-0.014145	-0.014145\\
-0.0148325	-0.0148325\\
-0.01497	-0.01497\\
-0.01474	-0.01474\\
-0.014695	-0.014695\\
-0.0146475	-0.0146475\\
-0.014465	-0.014465\\
-0.0140525	-0.0140525\\
-0.01419	-0.01419\\
-0.0141	-0.0141\\
-0.0142375	-0.0142375\\
-0.014145	-0.014145\\
-0.0140525	-0.0140525\\
-0.0137325	-0.0137325\\
-0.013825	-0.013825\\
-0.01451	-0.01451\\
-0.01442	-0.01442\\
-0.01451	-0.01451\\
-0.014465	-0.014465\\
-0.01474	-0.01474\\
-0.0148775	-0.0148775\\
-0.0151525	-0.0151525\\
-0.01506	-0.01506\\
-0.015105	-0.015105\\
-0.01451	-0.01451\\
-0.014695	-0.014695\\
-0.0145575	-0.0145575\\
-0.01474	-0.01474\\
-0.01497	-0.01497\\
-0.0148775	-0.0148775\\
-0.015015	-0.015015\\
-0.0154275	-0.0154275\\
-0.01529	-0.01529\\
-0.015105	-0.015105\\
-0.0148775	-0.0148775\\
-0.01497	-0.01497\\
-0.0148325	-0.0148325\\
-0.014695	-0.014695\\
-0.014465	-0.014465\\
-0.0145575	-0.0145575\\
-0.01442	-0.01442\\
-0.015015	-0.015015\\
-0.01561	-0.01561\\
-0.0155175	-0.0155175\\
-0.0149225	-0.0149225\\
-0.0146025	-0.0146025\\
-0.014375	-0.014375\\
-0.014145	-0.014145\\
-0.0141	-0.0141\\
-0.01451	-0.01451\\
-0.0149225	-0.0149225\\
-0.0151525	-0.0151525\\
-0.015105	-0.015105\\
-0.014695	-0.014695\\
-0.0152425	-0.0152425\\
-0.015655	-0.015655\\
-0.0157925	-0.0157925\\
-0.015655	-0.015655\\
-0.01616	-0.01616\\
-0.01538	-0.01538\\
-0.0148325	-0.0148325\\
-0.014465	-0.014465\\
-0.0148325	-0.0148325\\
-0.0152425	-0.0152425\\
-0.0157025	-0.0157025\\
-0.01561	-0.01561\\
-0.0152425	-0.0152425\\
-0.0146025	-0.0146025\\
-0.0143275	-0.0143275\\
-0.014375	-0.014375\\
-0.01442	-0.01442\\
-0.014145	-0.014145\\
-0.0141	-0.0141\\
-0.0140075	-0.0140075\\
-0.0134575	-0.0134575\\
-0.01387	-0.01387\\
-0.01419	-0.01419\\
-0.014465	-0.014465\\
-0.01451	-0.01451\\
-0.0146025	-0.0146025\\
-0.014785	-0.014785\\
-0.0146025	-0.0146025\\
-0.0148325	-0.0148325\\
-0.0151525	-0.0151525\\
-0.01506	-0.01506\\
-0.01497	-0.01497\\
-0.0148775	-0.0148775\\
-0.01497	-0.01497\\
-0.01529	-0.01529\\
-0.0151975	-0.0151975\\
-0.01497	-0.01497\\
-0.0149225	-0.0149225\\
-0.014465	-0.014465\\
-0.0146475	-0.0146475\\
-0.0151975	-0.0151975\\
-0.0152425	-0.0152425\\
-0.01538	-0.01538\\
-0.0160225	-0.0160225\\
-0.015885	-0.015885\\
-0.015565	-0.015565\\
-0.01529	-0.01529\\
-0.015335	-0.015335\\
-0.0158375	-0.0158375\\
-0.0155175	-0.0155175\\
-0.0152425	-0.0152425\\
-0.0155175	-0.0155175\\
-0.0152425	-0.0152425\\
-0.0146025	-0.0146025\\
-0.01451	-0.01451\\
-0.01442	-0.01442\\
-0.0140525	-0.0140525\\
-0.0139625	-0.0139625\\
-0.013825	-0.013825\\
-0.0136875	-0.0136875\\
-0.013825	-0.013825\\
-0.01419	-0.01419\\
-0.01451	-0.01451\\
-0.0148775	-0.0148775\\
-0.01497	-0.01497\\
-0.0151525	-0.0151525\\
-0.0154275	-0.0154275\\
-0.015105	-0.015105\\
-0.015015	-0.015015\\
-0.015335	-0.015335\\
-0.01529	-0.01529\\
-0.0152425	-0.0152425\\
-0.0151525	-0.0151525\\
-0.015105	-0.015105\\
-0.0155175	-0.0155175\\
-0.0154725	-0.0154725\\
-0.0154275	-0.0154275\\
-0.01538	-0.01538\\
-0.0148775	-0.0148775\\
-0.0142375	-0.0142375\\
-0.01387	-0.01387\\
-0.014145	-0.014145\\
-0.01419	-0.01419\\
-0.01442	-0.01442\\
-0.0143275	-0.0143275\\
-0.014465	-0.014465\\
-0.0152425	-0.0152425\\
-0.0157925	-0.0157925\\
-0.015655	-0.015655\\
-0.01561	-0.01561\\
-0.0151975	-0.0151975\\
-0.015105	-0.015105\\
-0.0152425	-0.0152425\\
-0.0151975	-0.0151975\\
-0.0154725	-0.0154725\\
-0.015975	-0.015975\\
-0.0161125	-0.0161125\\
-0.015975	-0.015975\\
-0.0161125	-0.0161125\\
-0.0157475	-0.0157475\\
-0.015655	-0.015655\\
-0.0154725	-0.0154725\\
-0.01538	-0.01538\\
-0.01561	-0.01561\\
-0.0158375	-0.0158375\\
-0.015015	-0.015015\\
-0.01474	-0.01474\\
-0.0146475	-0.0146475\\
-0.0148775	-0.0148775\\
-0.015015	-0.015015\\
-0.0146475	-0.0146475\\
-0.0151975	-0.0151975\\
-0.0161125	-0.0161125\\
-0.01648	-0.01648\\
-0.0163425	-0.0163425\\
-0.01616	-0.01616\\
-0.016205	-0.016205\\
-0.01648	-0.01648\\
-0.0162975	-0.0162975\\
-0.016205	-0.016205\\
-0.01625	-0.01625\\
-0.01593	-0.01593\\
-0.015655	-0.015655\\
-0.0158375	-0.0158375\\
-0.015565	-0.015565\\
-0.015015	-0.015015\\
-0.0151525	-0.0151525\\
-0.0148775	-0.0148775\\
-0.0148325	-0.0148325\\
-0.015015	-0.015015\\
-0.015335	-0.015335\\
-0.0152425	-0.0152425\\
-0.0149225	-0.0149225\\
-0.015015	-0.015015\\
-0.0151525	-0.0151525\\
-0.01529	-0.01529\\
-0.0152425	-0.0152425\\
-0.015015	-0.015015\\
-0.01538	-0.01538\\
-0.0154725	-0.0154725\\
-0.0151525	-0.0151525\\
-0.015565	-0.015565\\
-0.0157475	-0.0157475\\
-0.015335	-0.015335\\
-0.0148775	-0.0148775\\
-0.01442	-0.01442\\
-0.013915	-0.013915\\
-0.0143275	-0.0143275\\
-0.0145575	-0.0145575\\
-0.0142825	-0.0142825\\
-0.013915	-0.013915\\
-0.01387	-0.01387\\
-0.014375	-0.014375\\
-0.0146475	-0.0146475\\
-0.01497	-0.01497\\
-0.01529	-0.01529\\
-0.01538	-0.01538\\
-0.0154725	-0.0154725\\
-0.01506	-0.01506\\
-0.0141	-0.0141\\
-0.0140525	-0.0140525\\
-0.0140075	-0.0140075\\
-0.0141	-0.0141\\
-0.014695	-0.014695\\
-0.0149225	-0.0149225\\
-0.0155175	-0.0155175\\
-0.0152425	-0.0152425\\
-0.015015	-0.015015\\
-0.0154725	-0.0154725\\
-0.01529	-0.01529\\
-0.0148775	-0.0148775\\
-0.014465	-0.014465\\
-0.01451	-0.01451\\
-0.0148325	-0.0148325\\
-0.01538	-0.01538\\
-0.0152425	-0.0152425\\
-0.0149225	-0.0149225\\
-0.0148325	-0.0148325\\
-0.01497	-0.01497\\
-0.0152425	-0.0152425\\
-0.01593	-0.01593\\
-0.0160225	-0.0160225\\
-0.015975	-0.015975\\
-0.015565	-0.015565\\
-0.0149225	-0.0149225\\
-0.01497	-0.01497\\
-0.0143275	-0.0143275\\
-0.0142375	-0.0142375\\
-0.01419	-0.01419\\
-0.014785	-0.014785\\
-0.0154275	-0.0154275\\
-0.0157475	-0.0157475\\
-0.01625	-0.01625\\
-0.0162975	-0.0162975\\
-0.015565	-0.015565\\
-0.0151975	-0.0151975\\
-0.01506	-0.01506\\
-0.0151525	-0.0151525\\
-0.0154725	-0.0154725\\
-0.015885	-0.015885\\
-0.0157925	-0.0157925\\
-0.01529	-0.01529\\
-0.015335	-0.015335\\
-0.015565	-0.015565\\
-0.01538	-0.01538\\
-0.01497	-0.01497\\
-0.0146025	-0.0146025\\
-0.0141	-0.0141\\
-0.013825	-0.013825\\
-0.0134575	-0.0134575\\
-0.0136875	-0.0136875\\
-0.01419	-0.01419\\
-0.01442	-0.01442\\
-0.0142825	-0.0142825\\
-0.0141	-0.0141\\
-0.013505	-0.013505\\
-0.0125875	-0.0125875\\
-0.0124975	-0.0124975\\
-0.0130475	-0.0130475\\
-0.013595	-0.013595\\
-0.0140525	-0.0140525\\
-0.014375	-0.014375\\
-0.014695	-0.014695\\
-0.0151975	-0.0151975\\
-0.01593	-0.01593\\
-0.01616	-0.01616\\
-0.0163425	-0.0163425\\
-0.0162975	-0.0162975\\
-0.0155175	-0.0155175\\
-0.0151975	-0.0151975\\
-0.0154725	-0.0154725\\
-0.0154275	-0.0154275\\
-0.01506	-0.01506\\
-0.01474	-0.01474\\
-0.0143275	-0.0143275\\
-0.0146475	-0.0146475\\
-0.014785	-0.014785\\
-0.0142825	-0.0142825\\
-0.0137325	-0.0137325\\
-0.013915	-0.013915\\
-0.0142375	-0.0142375\\
-0.014145	-0.014145\\
-0.0143275	-0.0143275\\
-0.0142825	-0.0142825\\
-0.01451	-0.01451\\
-0.0151975	-0.0151975\\
-0.0151525	-0.0151525\\
-0.01538	-0.01538\\
-0.0160225	-0.0160225\\
-0.0162975	-0.0162975\\
-0.01616	-0.01616\\
-0.015565	-0.015565\\
-0.015105	-0.015105\\
-0.01442	-0.01442\\
-0.0145575	-0.0145575\\
-0.01442	-0.01442\\
-0.014785	-0.014785\\
-0.0149225	-0.0149225\\
-0.0146475	-0.0146475\\
-0.0148325	-0.0148325\\
-0.01442	-0.01442\\
-0.01451	-0.01451\\
-0.01442	-0.01442\\
-0.0139625	-0.0139625\\
-0.01387	-0.01387\\
-0.0136875	-0.0136875\\
-0.01355	-0.01355\\
-0.013505	-0.013505\\
-0.0131825	-0.0131825\\
-0.0134125	-0.0134125\\
-0.013505	-0.013505\\
-0.013915	-0.013915\\
-0.01451	-0.01451\\
-0.014465	-0.014465\\
-0.014375	-0.014375\\
-0.0148325	-0.0148325\\
-0.0152425	-0.0152425\\
-0.0151975	-0.0151975\\
-0.0151525	-0.0151525\\
-0.014465	-0.014465\\
-0.0136875	-0.0136875\\
-0.0140525	-0.0140525\\
-0.014695	-0.014695\\
-0.0148775	-0.0148775\\
-0.01529	-0.01529\\
-0.0154275	-0.0154275\\
-0.01506	-0.01506\\
-0.0146475	-0.0146475\\
-0.01451	-0.01451\\
-0.0146025	-0.0146025\\
-0.014465	-0.014465\\
-0.013915	-0.013915\\
-0.013825	-0.013825\\
-0.0136875	-0.0136875\\
-0.0134575	-0.0134575\\
-0.0131825	-0.0131825\\
-0.0134575	-0.0134575\\
-0.013915	-0.013915\\
-0.014145	-0.014145\\
-0.013915	-0.013915\\
-0.0143275	-0.0143275\\
-0.015015	-0.015015\\
-0.01474	-0.01474\\
-0.0148325	-0.0148325\\
-0.015105	-0.015105\\
-0.0148775	-0.0148775\\
-0.0143275	-0.0143275\\
-0.0142375	-0.0142375\\
-0.01451	-0.01451\\
-0.0141	-0.0141\\
-0.0139625	-0.0139625\\
-0.013825	-0.013825\\
-0.013275	-0.013275\\
-0.0130475	-0.0130475\\
-0.013275	-0.013275\\
-0.0136875	-0.0136875\\
-0.013825	-0.013825\\
-0.0140525	-0.0140525\\
-0.013825	-0.013825\\
-0.01332	-0.01332\\
-0.013	-0.013\\
-0.0131375	-0.0131375\\
-0.01332	-0.01332\\
-0.01419	-0.01419\\
-0.0146025	-0.0146025\\
-0.01442	-0.01442\\
-0.014465	-0.014465\\
-0.014695	-0.014695\\
-0.015105	-0.015105\\
-0.0154275	-0.0154275\\
-0.015565	-0.015565\\
-0.01529	-0.01529\\
-0.0152425	-0.0152425\\
-0.01529	-0.01529\\
-0.015335	-0.015335\\
-0.0154275	-0.0154275\\
-0.01593	-0.01593\\
-0.015975	-0.015975\\
-0.0157925	-0.0157925\\
-0.015565	-0.015565\\
-0.0154275	-0.0154275\\
-0.01538	-0.01538\\
-0.0154725	-0.0154725\\
-0.015105	-0.015105\\
-0.01497	-0.01497\\
-0.0151975	-0.0151975\\
-0.0154725	-0.0154725\\
-0.0152425	-0.0152425\\
-0.0151525	-0.0151525\\
-0.01497	-0.01497\\
-0.01451	-0.01451\\
-0.014695	-0.014695\\
-0.01529	-0.01529\\
-0.014695	-0.014695\\
-0.0146025	-0.0146025\\
-0.0140525	-0.0140525\\
-0.0136875	-0.0136875\\
-0.01387	-0.01387\\
-0.01355	-0.01355\\
-0.0133675	-0.0133675\\
-0.013505	-0.013505\\
-0.01323	-0.01323\\
-0.0134575	-0.0134575\\
-0.0130475	-0.0130475\\
-0.013275	-0.013275\\
-0.0134575	-0.0134575\\
-0.0136425	-0.0136425\\
-0.0137325	-0.0137325\\
-0.0137775	-0.0137775\\
-0.0140525	-0.0140525\\
-0.014145	-0.014145\\
-0.01497	-0.01497\\
-0.0149225	-0.0149225\\
-0.014695	-0.014695\\
-0.01474	-0.01474\\
-0.014465	-0.014465\\
-0.013825	-0.013825\\
-0.014145	-0.014145\\
-0.0141	-0.0141\\
-0.0136425	-0.0136425\\
-0.01387	-0.01387\\
-0.0140075	-0.0140075\\
-0.014145	-0.014145\\
-0.013915	-0.013915\\
-0.0133675	-0.0133675\\
-0.013	-0.013\\
-0.01291	-0.01291\\
-0.0134125	-0.0134125\\
-0.0136425	-0.0136425\\
-0.0137325	-0.0137325\\
-0.0141	-0.0141\\
-0.0146025	-0.0146025\\
-0.0145575	-0.0145575\\
-0.014785	-0.014785\\
-0.0151525	-0.0151525\\
-0.01506	-0.01506\\
-0.015015	-0.015015\\
-0.0157025	-0.0157025\\
-0.01561	-0.01561\\
-0.0157025	-0.0157025\\
-0.015975	-0.015975\\
-0.0154725	-0.0154725\\
-0.01474	-0.01474\\
-0.0143275	-0.0143275\\
-0.0146025	-0.0146025\\
-0.0149225	-0.0149225\\
-0.015015	-0.015015\\
-0.014695	-0.014695\\
-0.0140075	-0.0140075\\
-0.013915	-0.013915\\
-0.0146475	-0.0146475\\
-0.0152425	-0.0152425\\
-0.01538	-0.01538\\
-0.0148775	-0.0148775\\
-0.014465	-0.014465\\
-0.0146475	-0.0146475\\
-0.0151975	-0.0151975\\
-0.0154725	-0.0154725\\
-0.01561	-0.01561\\
-0.015335	-0.015335\\
-0.015565	-0.015565\\
-0.0157925	-0.0157925\\
-0.01625	-0.01625\\
-0.0162975	-0.0162975\\
-0.0161125	-0.0161125\\
-0.016205	-0.016205\\
-0.0161125	-0.0161125\\
-0.0154275	-0.0154275\\
-0.01506	-0.01506\\
-0.0151975	-0.0151975\\
-0.01538	-0.01538\\
-0.0154725	-0.0154725\\
-0.0152425	-0.0152425\\
-0.015335	-0.015335\\
-0.01529	-0.01529\\
-0.01497	-0.01497\\
-0.0151525	-0.0151525\\
-0.0155175	-0.0155175\\
-0.0157025	-0.0157025\\
-0.0154275	-0.0154275\\
-0.01506	-0.01506\\
-0.014465	-0.014465\\
-0.01451	-0.01451\\
-0.01442	-0.01442\\
-0.01419	-0.01419\\
-0.0140525	-0.0140525\\
-0.014375	-0.014375\\
-0.01474	-0.01474\\
-0.0149225	-0.0149225\\
-0.01497	-0.01497\\
-0.0145575	-0.0145575\\
-0.014145	-0.014145\\
-0.0137775	-0.0137775\\
-0.0137325	-0.0137325\\
-0.014145	-0.014145\\
-0.0142825	-0.0142825\\
-0.0143275	-0.0143275\\
-0.0149225	-0.0149225\\
-0.01529	-0.01529\\
-0.015565	-0.015565\\
-0.015655	-0.015655\\
-0.015885	-0.015885\\
-0.0154725	-0.0154725\\
-0.015105	-0.015105\\
-0.014785	-0.014785\\
-0.014695	-0.014695\\
-0.0149225	-0.0149225\\
-0.0148775	-0.0148775\\
-0.01451	-0.01451\\
-0.0142375	-0.0142375\\
-0.014695	-0.014695\\
-0.01506	-0.01506\\
-0.015015	-0.015015\\
-0.0155175	-0.0155175\\
-0.0158375	-0.0158375\\
-0.015565	-0.015565\\
-0.015105	-0.015105\\
-0.0146475	-0.0146475\\
-0.01442	-0.01442\\
-0.0148325	-0.0148325\\
-0.0149225	-0.0149225\\
-0.015105	-0.015105\\
-0.0152425	-0.0152425\\
-0.01529	-0.01529\\
-0.015335	-0.015335\\
-0.01506	-0.01506\\
-0.015105	-0.015105\\
-0.0148325	-0.0148325\\
-0.0146025	-0.0146025\\
-0.0149225	-0.0149225\\
-0.0154275	-0.0154275\\
-0.01561	-0.01561\\
-0.015105	-0.015105\\
-0.0157925	-0.0157925\\
-0.0160225	-0.0160225\\
-0.0155175	-0.0155175\\
-0.0148775	-0.0148775\\
-0.01497	-0.01497\\
-0.0148775	-0.0148775\\
-0.015015	-0.015015\\
-0.014785	-0.014785\\
-0.0148775	-0.0148775\\
-0.0157475	-0.0157475\\
-0.0160225	-0.0160225\\
-0.01561	-0.01561\\
-0.0155175	-0.0155175\\
-0.01561	-0.01561\\
-0.0157025	-0.0157025\\
-0.0154275	-0.0154275\\
-0.01529	-0.01529\\
-0.0149225	-0.0149225\\
-0.01506	-0.01506\\
-0.0152425	-0.0152425\\
-0.0157475	-0.0157475\\
-0.0161125	-0.0161125\\
-0.01657	-0.01657\\
-0.0164325	-0.0164325\\
-0.0160675	-0.0160675\\
-0.0157475	-0.0157475\\
-0.0157025	-0.0157025\\
-0.0154725	-0.0154725\\
-0.01497	-0.01497\\
-0.01474	-0.01474\\
-0.014695	-0.014695\\
-0.014375	-0.014375\\
-0.013915	-0.013915\\
-0.0141	-0.0141\\
-0.014465	-0.014465\\
-0.0142375	-0.0142375\\
-0.013915	-0.013915\\
-0.0136875	-0.0136875\\
-0.014375	-0.014375\\
-0.01506	-0.01506\\
-0.01451	-0.01451\\
-0.014375	-0.014375\\
-0.01451	-0.01451\\
-0.014465	-0.014465\\
-0.0145575	-0.0145575\\
-0.0142375	-0.0142375\\
-0.01387	-0.01387\\
-0.01355	-0.01355\\
-0.0136425	-0.0136425\\
-0.0142375	-0.0142375\\
-0.0146475	-0.0146475\\
-0.01442	-0.01442\\
-0.0140525	-0.0140525\\
-0.01355	-0.01355\\
-0.0141	-0.0141\\
-0.01451	-0.01451\\
-0.014695	-0.014695\\
-0.014465	-0.014465\\
-0.0142825	-0.0142825\\
-0.01442	-0.01442\\
-0.014695	-0.014695\\
-0.0151525	-0.0151525\\
-0.015565	-0.015565\\
-0.0154275	-0.0154275\\
-0.0148775	-0.0148775\\
-0.014785	-0.014785\\
-0.01474	-0.01474\\
-0.01442	-0.01442\\
-0.014375	-0.014375\\
-0.01451	-0.01451\\
-0.014465	-0.014465\\
-0.0141	-0.0141\\
-0.013595	-0.013595\\
-0.0136875	-0.0136875\\
-0.0140525	-0.0140525\\
-0.0146475	-0.0146475\\
-0.01497	-0.01497\\
-0.0152425	-0.0152425\\
-0.01529	-0.01529\\
-0.01538	-0.01538\\
-0.0158375	-0.0158375\\
-0.01648	-0.01648\\
-0.016525	-0.016525\\
-0.0160225	-0.0160225\\
-0.01593	-0.01593\\
-0.0161125	-0.0161125\\
-0.01648	-0.01648\\
-0.0161125	-0.0161125\\
-0.0151525	-0.0151525\\
-0.01442	-0.01442\\
-0.0142375	-0.0142375\\
-0.01387	-0.01387\\
-0.013275	-0.013275\\
-0.0131375	-0.0131375\\
-0.013505	-0.013505\\
-0.013915	-0.013915\\
-0.0140075	-0.0140075\\
-0.01387	-0.01387\\
-0.0137325	-0.0137325\\
-0.0139625	-0.0139625\\
-0.014145	-0.014145\\
-0.0142825	-0.0142825\\
-0.0140075	-0.0140075\\
-0.0137325	-0.0137325\\
-0.013505	-0.013505\\
-0.0137775	-0.0137775\\
-0.0140075	-0.0140075\\
-0.01387	-0.01387\\
-0.013505	-0.013505\\
-0.0137775	-0.0137775\\
-0.0136425	-0.0136425\\
-0.013595	-0.013595\\
-0.013915	-0.013915\\
-0.0142375	-0.0142375\\
-0.014465	-0.014465\\
-0.01419	-0.01419\\
-0.014465	-0.014465\\
-0.0146475	-0.0146475\\
-0.014465	-0.014465\\
-0.0143275	-0.0143275\\
-0.014695	-0.014695\\
-0.0151525	-0.0151525\\
-0.0152425	-0.0152425\\
-0.0154275	-0.0154275\\
-0.0152425	-0.0152425\\
-0.015105	-0.015105\\
-0.01506	-0.01506\\
-0.0152425	-0.0152425\\
-0.015105	-0.015105\\
-0.01538	-0.01538\\
-0.0154725	-0.0154725\\
-0.0154275	-0.0154275\\
-0.0162975	-0.0162975\\
-0.0163425	-0.0163425\\
-0.015565	-0.015565\\
-0.0149225	-0.0149225\\
-0.01474	-0.01474\\
-0.0149225	-0.0149225\\
-0.0151975	-0.0151975\\
-0.0154275	-0.0154275\\
-0.015655	-0.015655\\
-0.0157025	-0.0157025\\
-0.015335	-0.015335\\
-0.01506	-0.01506\\
-0.0151975	-0.0151975\\
-0.01474	-0.01474\\
-0.0142825	-0.0142825\\
-0.0146025	-0.0146025\\
-0.014695	-0.014695\\
-0.01497	-0.01497\\
-0.01529	-0.01529\\
-0.015335	-0.015335\\
-0.01497	-0.01497\\
-0.014695	-0.014695\\
-0.015015	-0.015015\\
-0.015335	-0.015335\\
-0.0157475	-0.0157475\\
-0.0160225	-0.0160225\\
-0.0163875	-0.0163875\\
-0.0162975	-0.0162975\\
-0.0161125	-0.0161125\\
-0.0155175	-0.0155175\\
-0.0149225	-0.0149225\\
-0.015335	-0.015335\\
-0.0157475	-0.0157475\\
-0.015335	-0.015335\\
-0.01561	-0.01561\\
-0.0161125	-0.0161125\\
-0.01648	-0.01648\\
-0.0160675	-0.0160675\\
-0.0154275	-0.0154275\\
-0.0146475	-0.0146475\\
-0.014695	-0.014695\\
-0.0142375	-0.0142375\\
-0.0140525	-0.0140525\\
-0.014145	-0.014145\\
-0.0140075	-0.0140075\\
-0.014145	-0.014145\\
-0.0146475	-0.0146475\\
-0.01474	-0.01474\\
-0.01506	-0.01506\\
-0.0154725	-0.0154725\\
-0.01561	-0.01561\\
-0.01506	-0.01506\\
-0.0148775	-0.0148775\\
-0.01506	-0.01506\\
-0.014785	-0.014785\\
-0.0145575	-0.0145575\\
-0.0142375	-0.0142375\\
-0.01442	-0.01442\\
-0.014465	-0.014465\\
-0.0139625	-0.0139625\\
-0.0133675	-0.0133675\\
-0.0130475	-0.0130475\\
-0.0134125	-0.0134125\\
-0.01387	-0.01387\\
-0.0141	-0.0141\\
-0.013825	-0.013825\\
-0.0142825	-0.0142825\\
-0.014695	-0.014695\\
-0.014465	-0.014465\\
-0.0142825	-0.0142825\\
-0.01442	-0.01442\\
-0.0146475	-0.0146475\\
-0.01474	-0.01474\\
-0.014785	-0.014785\\
-0.014465	-0.014465\\
-0.01419	-0.01419\\
-0.0143275	-0.0143275\\
-0.01419	-0.01419\\
-0.0143275	-0.0143275\\
-0.014695	-0.014695\\
-0.015335	-0.015335\\
-0.0151975	-0.0151975\\
-0.015015	-0.015015\\
-0.0155175	-0.0155175\\
-0.015335	-0.015335\\
-0.0148325	-0.0148325\\
-0.01442	-0.01442\\
-0.0140075	-0.0140075\\
-0.0139625	-0.0139625\\
-0.0137775	-0.0137775\\
-0.0136875	-0.0136875\\
-0.0142375	-0.0142375\\
-0.0136875	-0.0136875\\
-0.013275	-0.013275\\
-0.0136425	-0.0136425\\
-0.0140075	-0.0140075\\
-0.0139625	-0.0139625\\
-0.0136425	-0.0136425\\
-0.0131825	-0.0131825\\
-0.013275	-0.013275\\
-0.0140075	-0.0140075\\
-0.0145575	-0.0145575\\
-0.014465	-0.014465\\
-0.0142825	-0.0142825\\
-0.014375	-0.014375\\
-0.01451	-0.01451\\
-0.014785	-0.014785\\
-0.015105	-0.015105\\
-0.0151975	-0.0151975\\
-0.01506	-0.01506\\
-0.0146475	-0.0146475\\
-0.0143275	-0.0143275\\
-0.0142375	-0.0142375\\
-0.01451	-0.01451\\
-0.0142375	-0.0142375\\
-0.013915	-0.013915\\
-0.01442	-0.01442\\
-0.014785	-0.014785\\
-0.0148775	-0.0148775\\
-0.015015	-0.015015\\
-0.0154275	-0.0154275\\
-0.0148775	-0.0148775\\
-0.01529	-0.01529\\
-0.01561	-0.01561\\
-0.0154725	-0.0154725\\
-0.0155175	-0.0155175\\
-0.015565	-0.015565\\
-0.01648	-0.01648\\
-0.016755	-0.016755\\
-0.0163425	-0.0163425\\
-0.0163875	-0.0163875\\
-0.0166625	-0.0166625\\
-0.0167075	-0.0167075\\
-0.016845	-0.016845\\
-0.0168	-0.0168\\
-0.01712	-0.01712\\
-0.01703	-0.01703\\
-0.016525	-0.016525\\
-0.0158375	-0.0158375\\
-0.015565	-0.015565\\
-0.015335	-0.015335\\
-0.015015	-0.015015\\
-0.0151525	-0.0151525\\
-0.01561	-0.01561\\
-0.0152425	-0.0152425\\
-0.01497	-0.01497\\
-0.0149225	-0.0149225\\
-0.01451	-0.01451\\
-0.014465	-0.014465\\
-0.01387	-0.01387\\
-0.013595	-0.013595\\
-0.0134575	-0.0134575\\
-0.0136875	-0.0136875\\
-0.013595	-0.013595\\
-0.013275	-0.013275\\
-0.012955	-0.012955\\
-0.0125875	-0.0125875\\
-0.01291	-0.01291\\
-0.0127725	-0.0127725\\
-0.012725	-0.012725\\
-0.013505	-0.013505\\
-0.014145	-0.014145\\
-0.0143275	-0.0143275\\
-0.0141	-0.0141\\
-0.0142825	-0.0142825\\
-0.0149225	-0.0149225\\
-0.01442	-0.01442\\
-0.0137325	-0.0137325\\
-0.0133675	-0.0133675\\
-0.012955	-0.012955\\
-0.0130925	-0.0130925\\
-0.0136875	-0.0136875\\
-0.0142375	-0.0142375\\
-0.0140525	-0.0140525\\
-0.0145575	-0.0145575\\
-0.0151525	-0.0151525\\
-0.01529	-0.01529\\
-0.015105	-0.015105\\
-0.01442	-0.01442\\
-0.014465	-0.014465\\
-0.014785	-0.014785\\
-0.0152425	-0.0152425\\
-0.014695	-0.014695\\
-0.0139625	-0.0139625\\
-0.0141	-0.0141\\
-0.0139625	-0.0139625\\
-0.01332	-0.01332\\
-0.012725	-0.012725\\
-0.0130925	-0.0130925\\
-0.0142375	-0.0142375\\
-0.0146475	-0.0146475\\
-0.014375	-0.014375\\
-0.013825	-0.013825\\
-0.0136425	-0.0136425\\
-0.01268	-0.01268\\
-0.012725	-0.012725\\
-0.0127725	-0.0127725\\
-0.0130925	-0.0130925\\
-0.013595	-0.013595\\
-0.0142825	-0.0142825\\
-0.0146475	-0.0146475\\
-0.014785	-0.014785\\
-0.0149225	-0.0149225\\
-0.01497	-0.01497\\
-0.0149225	-0.0149225\\
-0.014785	-0.014785\\
-0.0148775	-0.0148775\\
-0.015565	-0.015565\\
-0.0157925	-0.0157925\\
-0.01506	-0.01506\\
-0.0141	-0.0141\\
-0.01474	-0.01474\\
-0.0151525	-0.0151525\\
-0.015105	-0.015105\\
-0.0145575	-0.0145575\\
-0.0143275	-0.0143275\\
-0.014785	-0.014785\\
-0.0155175	-0.0155175\\
-0.0158375	-0.0158375\\
-0.0160675	-0.0160675\\
-0.0161125	-0.0161125\\
-0.0158375	-0.0158375\\
-0.0157925	-0.0157925\\
-0.0149225	-0.0149225\\
-0.014465	-0.014465\\
-0.0146025	-0.0146025\\
-0.0148775	-0.0148775\\
-0.015015	-0.015015\\
-0.01506	-0.01506\\
-0.01474	-0.01474\\
-0.014785	-0.014785\\
-0.0148325	-0.0148325\\
-0.0142375	-0.0142375\\
-0.013595	-0.013595\\
-0.0131825	-0.0131825\\
-0.0134125	-0.0134125\\
-0.013275	-0.013275\\
-0.0127725	-0.0127725\\
-0.0130925	-0.0130925\\
-0.0140075	-0.0140075\\
-0.01387	-0.01387\\
-0.0136425	-0.0136425\\
-0.013915	-0.013915\\
-0.0146025	-0.0146025\\
-0.01451	-0.01451\\
-0.0137775	-0.0137775\\
-0.0130925	-0.0130925\\
-0.0136875	-0.0136875\\
-0.01474	-0.01474\\
-0.01538	-0.01538\\
-0.0160225	-0.0160225\\
-0.0166625	-0.0166625\\
-0.016845	-0.016845\\
-0.0163425	-0.0163425\\
-0.016205	-0.016205\\
-0.01657	-0.01657\\
-0.01648	-0.01648\\
-0.0163425	-0.0163425\\
-0.0164325	-0.0164325\\
-0.01657	-0.01657\\
-0.0166625	-0.0166625\\
-0.01712	-0.01712\\
-0.016755	-0.016755\\
-0.0160225	-0.0160225\\
-0.01497	-0.01497\\
-0.014375	-0.014375\\
-0.0142825	-0.0142825\\
-0.0142375	-0.0142375\\
-0.0148325	-0.0148325\\
-0.014785	-0.014785\\
-0.014695	-0.014695\\
-0.015015	-0.015015\\
-0.0146025	-0.0146025\\
-0.014465	-0.014465\\
-0.0148325	-0.0148325\\
-0.01506	-0.01506\\
-0.0146025	-0.0146025\\
-0.0140525	-0.0140525\\
-0.014145	-0.014145\\
-0.0152425	-0.0152425\\
-0.0160675	-0.0160675\\
-0.01625	-0.01625\\
-0.016525	-0.016525\\
-0.01703	-0.01703\\
-0.017165	-0.017165\\
-0.016525	-0.016525\\
-0.015885	-0.015885\\
-0.0149225	-0.0149225\\
-0.0146475	-0.0146475\\
-0.0148775	-0.0148775\\
-0.0146475	-0.0146475\\
-0.01442	-0.01442\\
-0.014465	-0.014465\\
-0.0146475	-0.0146475\\
-0.014785	-0.014785\\
-0.01506	-0.01506\\
-0.0148325	-0.0148325\\
-0.014145	-0.014145\\
-0.01355	-0.01355\\
-0.0131375	-0.0131375\\
-0.012955	-0.012955\\
-0.0130925	-0.0130925\\
-0.013595	-0.013595\\
-0.0137775	-0.0137775\\
-0.0140525	-0.0140525\\
-0.0146475	-0.0146475\\
-0.0145575	-0.0145575\\
-0.01506	-0.01506\\
-0.01593	-0.01593\\
-0.015885	-0.015885\\
-0.0157025	-0.0157025\\
-0.01593	-0.01593\\
-0.0161125	-0.0161125\\
-0.0157475	-0.0157475\\
-0.0154725	-0.0154725\\
-0.01497	-0.01497\\
-0.01474	-0.01474\\
-0.015565	-0.015565\\
-0.01657	-0.01657\\
-0.0172125	-0.0172125\\
-0.017075	-0.017075\\
-0.0168	-0.0168\\
-0.0164325	-0.0164325\\
-0.0163425	-0.0163425\\
-0.0167075	-0.0167075\\
-0.01616	-0.01616\\
-0.015655	-0.015655\\
-0.01538	-0.01538\\
-0.015885	-0.015885\\
-0.0160675	-0.0160675\\
-0.01538	-0.01538\\
-0.015105	-0.015105\\
-0.014695	-0.014695\\
-0.01506	-0.01506\\
-0.015655	-0.015655\\
-0.0158375	-0.0158375\\
-0.015655	-0.015655\\
-0.01529	-0.01529\\
-0.01506	-0.01506\\
-0.014695	-0.014695\\
-0.0142825	-0.0142825\\
-0.0139625	-0.0139625\\
-0.0136425	-0.0136425\\
-0.01355	-0.01355\\
-0.0141	-0.0141\\
-0.0146475	-0.0146475\\
-0.014785	-0.014785\\
-0.01442	-0.01442\\
-0.01419	-0.01419\\
-0.0139625	-0.0139625\\
-0.0140075	-0.0140075\\
-0.0143275	-0.0143275\\
-0.01451	-0.01451\\
-0.0145575	-0.0145575\\
-0.0140525	-0.0140525\\
-0.01451	-0.01451\\
-0.0152425	-0.0152425\\
-0.01506	-0.01506\\
-0.0141	-0.0141\\
-0.013825	-0.013825\\
-0.0142825	-0.0142825\\
-0.014375	-0.014375\\
-0.0143275	-0.0143275\\
-0.013915	-0.013915\\
-0.0143275	-0.0143275\\
-0.0148775	-0.0148775\\
-0.0145575	-0.0145575\\
-0.013505	-0.013505\\
-0.0133675	-0.0133675\\
-0.0142825	-0.0142825\\
-0.0148325	-0.0148325\\
-0.014375	-0.014375\\
-0.0141	-0.0141\\
-0.014695	-0.014695\\
-0.015335	-0.015335\\
-0.0157475	-0.0157475\\
-0.01625	-0.01625\\
-0.01593	-0.01593\\
-0.015565	-0.015565\\
-0.015885	-0.015885\\
-0.015565	-0.015565\\
-0.0157025	-0.0157025\\
-0.0160225	-0.0160225\\
-0.015885	-0.015885\\
-0.01506	-0.01506\\
-0.01529	-0.01529\\
-0.016205	-0.016205\\
-0.0166175	-0.0166175\\
-0.0163425	-0.0163425\\
-0.0160675	-0.0160675\\
-0.0162975	-0.0162975\\
-0.01616	-0.01616\\
-0.015565	-0.015565\\
-0.015335	-0.015335\\
-0.01538	-0.01538\\
-0.0154725	-0.0154725\\
-0.01561	-0.01561\\
-0.01538	-0.01538\\
-0.01529	-0.01529\\
-0.01561	-0.01561\\
-0.0151975	-0.0151975\\
-0.014785	-0.014785\\
-0.01451	-0.01451\\
-0.0141	-0.0141\\
-0.0145575	-0.0145575\\
-0.0146475	-0.0146475\\
-0.0148775	-0.0148775\\
-0.015335	-0.015335\\
-0.0157025	-0.0157025\\
-0.015565	-0.015565\\
-0.01538	-0.01538\\
-0.0145575	-0.0145575\\
-0.015335	-0.015335\\
-0.015105	-0.015105\\
-0.01442	-0.01442\\
-0.01474	-0.01474\\
-0.0154275	-0.0154275\\
-0.01561	-0.01561\\
-0.015975	-0.015975\\
-0.016525	-0.016525\\
-0.0163425	-0.0163425\\
-0.0160675	-0.0160675\\
-0.016205	-0.016205\\
-0.0157925	-0.0157925\\
-0.0154275	-0.0154275\\
-0.0157025	-0.0157025\\
-0.0158375	-0.0158375\\
-0.015335	-0.015335\\
-0.0149225	-0.0149225\\
-0.014695	-0.014695\\
-0.0148325	-0.0148325\\
-0.014695	-0.014695\\
-0.0152425	-0.0152425\\
-0.015565	-0.015565\\
-0.0152425	-0.0152425\\
-0.01497	-0.01497\\
-0.0151975	-0.0151975\\
-0.014785	-0.014785\\
-0.0141	-0.0141\\
-0.0140525	-0.0140525\\
-0.0142825	-0.0142825\\
-0.013915	-0.013915\\
-0.01332	-0.01332\\
-0.0134575	-0.0134575\\
-0.0139625	-0.0139625\\
-0.0145575	-0.0145575\\
-0.0149225	-0.0149225\\
-0.0154275	-0.0154275\\
-0.0157925	-0.0157925\\
-0.0151975	-0.0151975\\
-0.014695	-0.014695\\
-0.0154275	-0.0154275\\
-0.01625	-0.01625\\
-0.01616	-0.01616\\
-0.0160225	-0.0160225\\
-0.016525	-0.016525\\
-0.0164325	-0.0164325\\
-0.016205	-0.016205\\
-0.0158375	-0.0158375\\
-0.0157025	-0.0157025\\
-0.015975	-0.015975\\
-0.015655	-0.015655\\
-0.015335	-0.015335\\
-0.015885	-0.015885\\
-0.01625	-0.01625\\
-0.0164325	-0.0164325\\
-0.0154725	-0.0154725\\
-0.0145575	-0.0145575\\
-0.0151975	-0.0151975\\
-0.015105	-0.015105\\
-0.0142825	-0.0142825\\
-0.0136425	-0.0136425\\
-0.013915	-0.013915\\
-0.014465	-0.014465\\
-0.0151975	-0.0151975\\
-0.01506	-0.01506\\
-0.01474	-0.01474\\
-0.01419	-0.01419\\
-0.01355	-0.01355\\
-0.013595	-0.013595\\
-0.013505	-0.013505\\
-0.01355	-0.01355\\
-0.014145	-0.014145\\
-0.0142375	-0.0142375\\
-0.0136425	-0.0136425\\
-0.0133675	-0.0133675\\
-0.01355	-0.01355\\
-0.0137325	-0.0137325\\
-0.0134575	-0.0134575\\
-0.013595	-0.013595\\
-0.0134575	-0.0134575\\
-0.0131825	-0.0131825\\
-0.0137325	-0.0137325\\
-0.0143275	-0.0143275\\
-0.015105	-0.015105\\
-0.0158375	-0.0158375\\
-0.015975	-0.015975\\
-0.0151525	-0.0151525\\
-0.0145575	-0.0145575\\
-0.01442	-0.01442\\
-0.0137775	-0.0137775\\
-0.0134125	-0.0134125\\
-0.013915	-0.013915\\
-0.01419	-0.01419\\
-0.0142825	-0.0142825\\
-0.014465	-0.014465\\
-0.0146025	-0.0146025\\
-0.014695	-0.014695\\
-0.01497	-0.01497\\
-0.015105	-0.015105\\
-0.0155175	-0.0155175\\
-0.01506	-0.01506\\
-0.0140525	-0.0140525\\
-0.0137325	-0.0137325\\
-0.01419	-0.01419\\
-0.0140525	-0.0140525\\
-0.0133675	-0.0133675\\
-0.01332	-0.01332\\
-0.0131825	-0.0131825\\
-0.0125875	-0.0125875\\
-0.01268	-0.01268\\
-0.013275	-0.013275\\
-0.0137325	-0.0137325\\
-0.01419	-0.01419\\
-0.0149225	-0.0149225\\
-0.0148325	-0.0148325\\
-0.013825	-0.013825\\
-0.0140525	-0.0140525\\
-0.01442	-0.01442\\
-0.0137775	-0.0137775\\
-0.0145575	-0.0145575\\
-0.014695	-0.014695\\
-0.015335	-0.015335\\
-0.015885	-0.015885\\
-0.0154725	-0.0154725\\
-0.015105	-0.015105\\
-0.0149225	-0.0149225\\
-0.0151525	-0.0151525\\
-0.01561	-0.01561\\
-0.01538	-0.01538\\
-0.0148775	-0.0148775\\
};
\end{axis}

\begin{axis}[%
width=4.927cm,
height=2.746cm,
at={(6.484cm,11.441cm)},
scale only axis,
xmin=-0.018,
xmax=-0.012,
xlabel style={font=\color{white!15!black}},
xlabel={$u(t-1)$},
ymin=-0.0671375,
ymax=0.0030525,
ylabel style={font=\color{white!15!black}},
ylabel={$\delta^4 y(t)$},
axis background/.style={fill=white},
title style={font=\bfseries},
title={C4, R = 0.6843},
axis x line*=bottom,
axis y line*=left
]
\addplot[only marks, mark=*, mark options={}, mark size=1.5000pt, color=mycolor1, fill=mycolor1] table[row sep=crcr]{%
x	y\\
-0.015565	-0.03662\\
-0.015655	-0.042725\\
-0.0157475	-0.027465\\
-0.01561	-0.01831\\
-0.0148775	-0.03662\\
-0.014785	-0.0213625\\
-0.014785	-0.024415\\
-0.01497	-0.01526\\
-0.0148325	-0.009155\\
-0.0140075	-0.009155\\
-0.013275	-0.009155\\
-0.0128625	-0.0061025\\
-0.0137775	-0.03357\\
-0.014785	-0.03357\\
-0.01506	-0.03357\\
-0.0151525	-0.024415\\
-0.0149225	-0.01831\\
-0.014375	-0.027465\\
-0.0149225	-0.0305175\\
-0.01474	-0.01526\\
-0.0143275	-0.024415\\
-0.0148325	-0.03357\\
-0.0149225	-0.01831\\
-0.01474	-0.0396725\\
-0.0151525	-0.03357\\
-0.0152425	-0.027465\\
-0.01497	-0.042725\\
-0.01529	-0.03357\\
-0.01538	-0.027465\\
-0.0151975	-0.0213625\\
-0.0149225	-0.0213625\\
-0.014695	-0.027465\\
-0.014695	-0.0305175\\
-0.0149225	-0.027465\\
-0.0149225	-0.024415\\
-0.01497	-0.027465\\
-0.015015	-0.027465\\
-0.01506	-0.0396725\\
-0.0154725	-0.03662\\
-0.0155175	-0.027465\\
-0.015105	-0.027465\\
-0.0148775	-0.01526\\
-0.0143275	-0.0122075\\
-0.0141	-0.0213625\\
-0.0143275	-0.01831\\
-0.01419	-0.01831\\
-0.0141	-0.027465\\
-0.01442	-0.024415\\
-0.0146475	-0.024415\\
-0.0146475	-0.0396725\\
-0.0151525	-0.0213625\\
-0.015105	-0.01526\\
-0.01451	-0.01831\\
-0.014145	-0.0213625\\
-0.01419	-0.024415\\
-0.014375	-0.0122075\\
-0.0139625	-0.01831\\
-0.0136875	-0.01831\\
-0.0140075	-0.01831\\
-0.013915	-0.01526\\
-0.0136875	-0.01831\\
-0.013825	-0.0213625\\
-0.0140075	-0.024415\\
-0.0146475	-0.027465\\
-0.014785	-0.027465\\
-0.015015	-0.0305175\\
-0.01506	-0.03357\\
-0.0151975	-0.027465\\
-0.01506	-0.0305175\\
-0.01497	-0.024415\\
-0.0148775	-0.024415\\
-0.01474	-0.024415\\
-0.01474	-0.042725\\
-0.0152425	-0.05188\\
-0.015975	-0.0488275\\
-0.0162975	-0.0488275\\
-0.0163425	-0.0396725\\
-0.0158375	-0.05188\\
-0.0161125	-0.0579825\\
-0.016525	-0.0579825\\
-0.0166175	-0.03662\\
-0.01648	-0.0579825\\
-0.0166175	-0.0671375\\
-0.017165	-0.0549325\\
-0.0168925	-0.0396725\\
-0.0166625	-0.03662\\
-0.0161125	-0.027465\\
-0.015565	-0.0213625\\
-0.0152425	-0.027465\\
-0.01538	-0.0213625\\
-0.015105	-0.01526\\
-0.01451	-0.01526\\
-0.0142375	-0.0213625\\
-0.0142825	-0.024415\\
-0.014695	-0.027465\\
-0.014785	-0.024415\\
-0.01474	-0.0305175\\
-0.01506	-0.027465\\
-0.0151525	-0.0396725\\
-0.01561	-0.03357\\
-0.015565	-0.0457775\\
-0.0157475	-0.0305175\\
-0.0155175	-0.027465\\
-0.01529	-0.0305175\\
-0.01538	-0.024415\\
-0.015105	-0.0213625\\
-0.0149225	-0.027465\\
-0.01506	-0.027465\\
-0.015105	-0.0213625\\
-0.01474	-0.0213625\\
-0.014695	-0.01831\\
-0.014375	-0.01831\\
-0.0142825	-0.01831\\
-0.014375	-0.0122075\\
-0.0141	-0.01526\\
-0.014145	-0.0213625\\
-0.0143275	-0.03357\\
-0.0148775	-0.0305175\\
-0.0151525	-0.03357\\
-0.01529	-0.0213625\\
-0.014785	-0.0396725\\
-0.0149225	-0.042725\\
-0.01561	-0.03357\\
-0.0155175	-0.03662\\
-0.0155175	-0.03662\\
-0.0157025	-0.027465\\
-0.0151525	-0.01831\\
-0.0146475	-0.0213625\\
-0.014785	-0.0213625\\
-0.01497	-0.042725\\
-0.0155175	-0.05188\\
-0.0160675	-0.0488275\\
-0.016205	-0.03662\\
-0.0158375	-0.0396725\\
-0.0157925	-0.03357\\
-0.0157475	-0.0305175\\
-0.015655	-0.0305175\\
-0.01561	-0.024415\\
-0.0151975	-0.0213625\\
-0.0148775	-0.0213625\\
-0.01474	-0.024415\\
-0.0146025	-0.0213625\\
-0.0146475	-0.03357\\
-0.01497	-0.0396725\\
-0.015565	-0.027465\\
-0.0155175	-0.0213625\\
-0.015105	-0.024415\\
-0.0148325	-0.0213625\\
-0.01474	-0.01526\\
-0.0142375	-0.009155\\
-0.0136875	-0.0122075\\
-0.013595	-0.0213625\\
-0.0141	-0.024415\\
-0.0139625	-0.0122075\\
-0.013915	-0.01831\\
-0.0140525	-0.01831\\
-0.0140525	-0.009155\\
-0.013825	-0.009155\\
-0.0134125	-0.0061025\\
-0.013	-0.0122075\\
-0.01323	-0.027465\\
-0.014145	-0.0305175\\
-0.01497	-0.03662\\
-0.015335	-0.03357\\
-0.01506	-0.0213625\\
-0.0146025	-0.01831\\
-0.014145	-0.0122075\\
-0.0136425	-0.01831\\
-0.0140525	-0.01526\\
-0.0143275	-0.027465\\
-0.0146475	-0.0305175\\
-0.01506	-0.0549325\\
-0.01593	-0.0579825\\
-0.01657	-0.0549325\\
-0.016525	-0.0457775\\
-0.01648	-0.03357\\
-0.015975	-0.042725\\
-0.01593	-0.0396725\\
-0.0160675	-0.0488275\\
-0.0160675	-0.0305175\\
-0.0158375	-0.03357\\
-0.015655	-0.0305175\\
-0.015655	-0.03357\\
-0.0154725	-0.03357\\
-0.015565	-0.027465\\
-0.01538	-0.0457775\\
-0.0157475	-0.042725\\
-0.01616	-0.03357\\
-0.0158375	-0.01831\\
-0.0152425	-0.0305175\\
-0.015105	-0.0396725\\
-0.015655	-0.042725\\
-0.0160225	-0.0488275\\
-0.0164325	-0.0549325\\
-0.0166175	-0.05188\\
-0.0167075	-0.0396725\\
-0.0164325	-0.03662\\
-0.01625	-0.0457775\\
-0.0163425	-0.0488275\\
-0.016525	-0.03357\\
-0.01616	-0.024415\\
-0.0154275	-0.03357\\
-0.015565	-0.024415\\
-0.0154725	-0.01831\\
-0.01497	-0.024415\\
-0.01497	-0.027465\\
-0.0151525	-0.01831\\
-0.0149225	-0.01831\\
-0.014695	-0.024415\\
-0.014785	-0.01526\\
-0.014695	-0.027465\\
-0.0148775	-0.03357\\
-0.015335	-0.03357\\
-0.01538	-0.0213625\\
-0.015015	-0.024415\\
-0.0148775	-0.0305175\\
-0.01529	-0.0457775\\
-0.01616	-0.0396725\\
-0.0161125	-0.024415\\
-0.0154725	-0.0213625\\
-0.01506	-0.024415\\
-0.015015	-0.01831\\
-0.01497	-0.0213625\\
-0.0146475	-0.01831\\
-0.0145575	-0.0213625\\
-0.014695	-0.027465\\
-0.0149225	-0.027465\\
-0.01506	-0.01526\\
-0.0148325	-0.027465\\
-0.015105	-0.03357\\
-0.0154725	-0.03357\\
-0.0154725	-0.024415\\
-0.01529	-0.0305175\\
-0.015335	-0.03357\\
-0.01561	-0.027465\\
-0.01529	-0.0122075\\
-0.014375	-0.027465\\
-0.0142375	-0.0213625\\
-0.014375	-0.0213625\\
-0.014695	-0.01526\\
-0.014695	-0.01526\\
-0.014375	-0.027465\\
-0.014695	-0.027465\\
-0.014785	-0.01526\\
-0.01451	-0.0213625\\
-0.0146025	-0.0213625\\
-0.0145575	-0.0122075\\
-0.014465	-0.0213625\\
-0.0145575	-0.0122075\\
-0.01419	-0.01831\\
-0.0141	-0.0122075\\
-0.01387	-0.0122075\\
-0.0137325	-0.027465\\
-0.0143275	-0.0305175\\
-0.0146025	-0.0305175\\
-0.01506	-0.0396725\\
-0.01561	-0.027465\\
-0.015335	-0.01526\\
-0.014695	-0.01831\\
-0.0142825	-0.0122075\\
-0.0141	-0.01831\\
-0.01442	-0.01526\\
-0.014145	-0.009155\\
-0.0141	-0.0213625\\
-0.0142375	-0.024415\\
-0.01497	-0.024415\\
-0.015105	-0.042725\\
-0.0154275	-0.027465\\
-0.0151525	-0.03357\\
-0.0149225	-0.01831\\
-0.0142825	-0.0213625\\
-0.0139625	-0.009155\\
-0.01355	-0.0030525\\
-0.0133675	-0.01526\\
-0.0134575	-0.024415\\
-0.0140075	-0.0213625\\
-0.0142375	-0.024415\\
-0.01442	-0.024415\\
-0.0145575	-0.042725\\
-0.0152425	-0.027465\\
-0.01538	-0.027465\\
-0.01506	-0.027465\\
-0.0151975	-0.03357\\
-0.01529	-0.0396725\\
-0.015565	-0.0396725\\
-0.0154725	-0.027465\\
-0.01506	-0.01526\\
-0.014145	-0.009155\\
-0.01355	-0.0122075\\
-0.0133675	-0.01526\\
-0.0134125	-0.0122075\\
-0.013505	-0.01526\\
-0.0134575	-0.01526\\
-0.0136875	-0.024415\\
-0.01419	-0.024415\\
-0.0142825	-0.0213625\\
-0.0142825	-0.024415\\
-0.014465	-0.01526\\
-0.0140525	-0.009155\\
-0.0134125	-0.01831\\
-0.013825	-0.027465\\
-0.01442	-0.0213625\\
-0.0145575	-0.0305175\\
-0.014695	-0.0213625\\
-0.014695	-0.01831\\
-0.0143275	-0.0213625\\
-0.01451	-0.0305175\\
-0.015105	-0.03662\\
-0.0151975	-0.027465\\
-0.0149225	-0.042725\\
-0.015335	-0.024415\\
-0.015335	-0.0305175\\
-0.0151975	-0.03357\\
-0.015335	-0.03357\\
-0.01538	-0.024415\\
-0.0151525	-0.024415\\
-0.01497	-0.0305175\\
-0.015335	-0.0457775\\
-0.01593	-0.0396725\\
-0.0158375	-0.0213625\\
-0.0152425	-0.0305175\\
-0.015015	-0.024415\\
-0.01497	-0.0305175\\
-0.0151975	-0.03357\\
-0.01538	-0.024415\\
-0.0151525	-0.024415\\
-0.0151525	-0.03662\\
-0.01538	-0.03357\\
-0.015565	-0.0305175\\
-0.0151525	-0.0213625\\
-0.0148775	-0.024415\\
-0.0151525	-0.0396725\\
-0.0157025	-0.03662\\
-0.0157925	-0.0396725\\
-0.0158375	-0.0305175\\
-0.01529	-0.027465\\
-0.01497	-0.024415\\
-0.01497	-0.0213625\\
-0.014465	-0.0122075\\
-0.01387	-0.009155\\
-0.0133675	-0.009155\\
-0.0130475	-0.01831\\
-0.0136425	-0.027465\\
-0.0142375	-0.024415\\
-0.0146025	-0.0213625\\
-0.01442	-0.027465\\
-0.014375	-0.024415\\
-0.0145575	-0.0213625\\
-0.0141	-0.009155\\
-0.0140075	-0.01526\\
-0.0140075	-0.01526\\
-0.0137325	-0.01526\\
-0.013825	-0.0213625\\
-0.014375	-0.027465\\
-0.0148325	-0.01831\\
-0.01451	-0.01526\\
-0.014145	-0.01831\\
-0.01387	-0.01831\\
-0.013915	-0.01526\\
-0.0136425	-0.0213625\\
-0.0142825	-0.0396725\\
-0.0151975	-0.03662\\
-0.0152425	-0.042725\\
-0.0155175	-0.042725\\
-0.015885	-0.042725\\
-0.0160225	-0.03357\\
-0.0157475	-0.027465\\
-0.015335	-0.024415\\
-0.0151975	-0.027465\\
-0.0151975	-0.0305175\\
-0.015335	-0.03662\\
-0.0155175	-0.03662\\
-0.01561	-0.042725\\
-0.0160225	-0.0488275\\
-0.0163425	-0.0549325\\
-0.0167075	-0.0457775\\
-0.0164325	-0.0549325\\
-0.0163425	-0.0488275\\
-0.016525	-0.03662\\
-0.016205	-0.042725\\
-0.0162975	-0.03662\\
-0.016205	-0.027465\\
-0.015565	-0.0213625\\
-0.0148775	-0.0213625\\
-0.0146025	-0.027465\\
-0.0149225	-0.01831\\
-0.0148325	-0.01831\\
-0.01442	-0.01831\\
-0.01451	-0.01831\\
-0.0143275	-0.009155\\
-0.0139625	-0.01831\\
-0.0139625	-0.01831\\
-0.0140525	-0.01526\\
-0.013915	-0.0213625\\
-0.0142825	-0.0305175\\
-0.01474	-0.0213625\\
-0.0146025	-0.0305175\\
-0.0151525	-0.0457775\\
-0.0158375	-0.042725\\
-0.015975	-0.0488275\\
-0.0162975	-0.042725\\
-0.0160225	-0.03662\\
-0.015975	-0.03662\\
-0.015565	-0.01831\\
-0.0149225	-0.024415\\
-0.0149225	-0.0213625\\
-0.014785	-0.01526\\
-0.014375	-0.01831\\
-0.0145575	-0.0213625\\
-0.0141	-0.0061025\\
-0.0134575	-0.0122075\\
-0.0134125	-0.0213625\\
-0.0141	-0.024415\\
-0.0146025	-0.0305175\\
-0.01497	-0.0305175\\
-0.01506	-0.027465\\
-0.01506	-0.0305175\\
-0.0152425	-0.03662\\
-0.0151525	-0.027465\\
-0.0148775	-0.024415\\
-0.0148325	-0.01831\\
-0.014695	-0.027465\\
-0.015015	-0.027465\\
-0.01474	-0.01831\\
-0.014465	-0.027465\\
-0.014695	-0.03357\\
-0.0151975	-0.027465\\
-0.01506	-0.0213625\\
-0.01474	-0.0213625\\
-0.01451	-0.0213625\\
-0.0145575	-0.01526\\
-0.0142375	-0.01831\\
-0.0140075	-0.01526\\
-0.0142375	-0.024415\\
-0.01451	-0.0213625\\
-0.014695	-0.027465\\
-0.01451	-0.027465\\
-0.0146475	-0.0213625\\
-0.014695	-0.01526\\
-0.014145	-0.01526\\
-0.0140525	-0.027465\\
-0.0146475	-0.03662\\
-0.01529	-0.03662\\
-0.0154275	-0.0305175\\
-0.01529	-0.027465\\
-0.0151975	-0.024415\\
-0.015015	-0.0213625\\
-0.0148325	-0.0213625\\
-0.0146475	-0.024415\\
-0.0148325	-0.0305175\\
-0.0148775	-0.01831\\
-0.01442	-0.01831\\
-0.0146025	-0.027465\\
-0.0148775	-0.027465\\
-0.015015	-0.03357\\
-0.0151525	-0.03357\\
-0.0152425	-0.03662\\
-0.015655	-0.05188\\
-0.0161125	-0.0549325\\
-0.01648	-0.03662\\
-0.01616	-0.024415\\
-0.01529	-0.01831\\
-0.0146475	-0.01526\\
-0.0139625	-0.01831\\
-0.0140525	-0.0122075\\
-0.0140525	-0.01831\\
-0.0145575	-0.01831\\
-0.0146475	-0.01831\\
-0.01451	-0.027465\\
-0.01474	-0.0305175\\
-0.01497	-0.03357\\
-0.0151975	-0.03357\\
-0.01538	-0.0457775\\
-0.015975	-0.042725\\
-0.0160675	-0.0305175\\
-0.0157475	-0.0305175\\
-0.01529	-0.024415\\
-0.015105	-0.024415\\
-0.0151525	-0.03357\\
-0.015335	-0.024415\\
-0.01506	-0.0213625\\
-0.0149225	-0.027465\\
-0.0149225	-0.01526\\
-0.0142825	-0.027465\\
-0.01442	-0.03662\\
-0.015015	-0.0213625\\
-0.0148325	-0.0213625\\
-0.0148325	-0.0396725\\
-0.015565	-0.03357\\
-0.015565	-0.027465\\
-0.0154275	-0.042725\\
-0.0157475	-0.042725\\
-0.0158375	-0.027465\\
-0.0154725	-0.0213625\\
-0.0148325	-0.024415\\
-0.0146475	-0.0305175\\
-0.015105	-0.0457775\\
-0.0158375	-0.0488275\\
-0.01616	-0.0457775\\
-0.016205	-0.0396725\\
-0.015975	-0.024415\\
-0.0154275	-0.024415\\
-0.015105	-0.01831\\
-0.014695	-0.01526\\
-0.01451	-0.01526\\
-0.0142825	-0.01831\\
-0.01442	-0.024415\\
-0.0146025	-0.0213625\\
-0.01419	-0.01526\\
-0.0143275	-0.0305175\\
-0.01474	-0.027465\\
-0.015015	-0.03357\\
-0.015335	-0.042725\\
-0.0157925	-0.0488275\\
-0.01625	-0.0396725\\
-0.0160675	-0.0305175\\
-0.0157925	-0.0396725\\
-0.0157925	-0.0305175\\
-0.015655	-0.024415\\
-0.0151975	-0.027465\\
-0.0151975	-0.0396725\\
-0.0157475	-0.0457775\\
-0.0161125	-0.0305175\\
-0.015655	-0.0213625\\
-0.0148775	-0.0122075\\
-0.01419	-0.0061025\\
-0.0134575	0\\
-0.0131375	-0.01526\\
-0.01355	-0.0213625\\
-0.0137775	-0.0213625\\
-0.014145	-0.01831\\
-0.014145	-0.01526\\
-0.013825	-0.01526\\
-0.0137775	-0.01831\\
-0.0137325	-0.0122075\\
-0.0136875	-0.01526\\
-0.0136875	-0.0213625\\
-0.0141	-0.027465\\
-0.01474	-0.03357\\
-0.015015	-0.027465\\
-0.015015	-0.0305175\\
-0.01529	-0.042725\\
-0.0157025	-0.042725\\
-0.0158375	-0.0488275\\
-0.01616	-0.05188\\
-0.016205	-0.03357\\
-0.015885	-0.027465\\
-0.0154275	-0.027465\\
-0.0152425	-0.03662\\
-0.0158375	-0.042725\\
-0.0161125	-0.03357\\
-0.0157475	-0.024415\\
-0.0152425	-0.01831\\
-0.0143275	-0.009155\\
-0.0136425	-0.0122075\\
-0.0134575	-0.01526\\
-0.012955	-0.0030525\\
-0.0134125	-0.024415\\
-0.01387	-0.01831\\
-0.0139625	-0.01526\\
-0.0139625	-0.0122075\\
-0.0134575	-0.009155\\
-0.0131375	-0.01526\\
-0.01323	-0.01526\\
-0.0133675	-0.0122075\\
-0.0134125	-0.009155\\
-0.013275	-0.009155\\
-0.0130925	-0.0122075\\
-0.0134125	-0.0305175\\
-0.0148325	-0.03357\\
-0.0154275	-0.0396725\\
-0.0157025	-0.03662\\
-0.0154725	-0.03662\\
-0.0157025	-0.0579825\\
-0.0164325	-0.05188\\
-0.01657	-0.0457775\\
-0.016525	-0.0457775\\
-0.016205	-0.0396725\\
-0.0161125	-0.0396725\\
-0.01616	-0.03662\\
-0.0157475	-0.027465\\
-0.0154725	-0.0213625\\
-0.01497	-0.0213625\\
-0.01451	-0.0305175\\
-0.015105	-0.03662\\
-0.0149225	-0.0213625\\
-0.0148325	-0.024415\\
-0.01497	-0.024415\\
-0.0149225	-0.024415\\
-0.0149225	-0.027465\\
-0.0149225	-0.027465\\
-0.01506	-0.0305175\\
-0.0151975	-0.027465\\
-0.01506	-0.024415\\
-0.014785	-0.024415\\
-0.0148775	-0.027465\\
-0.0142825	-0.0061025\\
-0.01332	-0.0122075\\
-0.0131825	-0.0213625\\
-0.0136875	-0.01831\\
-0.01323	0.0030525\\
-0.01245	-0.0061025\\
-0.012635	-0.01831\\
-0.01323	-0.01831\\
-0.013595	-0.0213625\\
-0.0139625	-0.0213625\\
-0.0139625	-0.0213625\\
-0.01451	-0.027465\\
-0.0146025	-0.01831\\
-0.0142375	-0.0213625\\
-0.01451	-0.027465\\
-0.0140525	-0.0122075\\
-0.013595	-0.0122075\\
-0.0131375	-0.009155\\
-0.01268	-0.009155\\
-0.013	-0.0061025\\
-0.012955	-0.009155\\
-0.0134125	-0.027465\\
-0.0140525	-0.0305175\\
-0.01419	-0.0213625\\
-0.0139625	-0.0213625\\
-0.013915	-0.01831\\
-0.0137775	-0.01526\\
-0.013915	-0.0213625\\
-0.0137325	-0.01526\\
-0.013595	-0.01526\\
-0.013505	-0.01526\\
-0.01323	-0.009155\\
-0.0133675	-0.0122075\\
-0.013275	-0.01831\\
-0.0136875	-0.01831\\
-0.013825	-0.009155\\
-0.0136425	-0.0122075\\
-0.013595	-0.01526\\
-0.013275	-0.01526\\
-0.013275	-0.024415\\
-0.01419	-0.03357\\
-0.0148775	-0.01526\\
-0.014785	-0.01831\\
-0.0146025	-0.024415\\
-0.0142375	-0.0213625\\
-0.0146475	-0.0305175\\
-0.01474	-0.01831\\
-0.0142825	-0.01831\\
-0.0143275	-0.027465\\
-0.014145	-0.01831\\
-0.014375	-0.024415\\
-0.0139625	-0.01526\\
-0.0137325	-0.01831\\
-0.013915	-0.0213625\\
-0.01419	-0.0213625\\
-0.0140075	-0.01831\\
-0.013915	-0.0213625\\
-0.0140075	-0.0213625\\
-0.014465	-0.027465\\
-0.014465	-0.027465\\
-0.0149225	-0.0457775\\
-0.015655	-0.03357\\
-0.01561	-0.027465\\
-0.015015	-0.024415\\
-0.0146475	-0.024415\\
-0.0149225	-0.03357\\
-0.01529	-0.024415\\
-0.0151525	-0.027465\\
-0.0152425	-0.03662\\
-0.014785	-0.0122075\\
-0.0140075	-0.009155\\
-0.0136875	-0.027465\\
-0.014375	-0.01831\\
-0.0146025	-0.024415\\
-0.014465	-0.027465\\
-0.0148325	-0.0305175\\
-0.015015	-0.027465\\
-0.0148325	-0.024415\\
-0.01451	-0.0213625\\
-0.0145575	-0.0305175\\
-0.0149225	-0.0213625\\
-0.014785	-0.027465\\
-0.014785	-0.027465\\
-0.01474	-0.01831\\
-0.014465	-0.0213625\\
-0.014465	-0.0122075\\
-0.0142825	-0.01526\\
-0.0142375	-0.027465\\
-0.014695	-0.03357\\
-0.015015	-0.027465\\
-0.015105	-0.03357\\
-0.01506	-0.024415\\
-0.01474	-0.01831\\
-0.0143275	-0.01831\\
-0.0143275	-0.01831\\
-0.014465	-0.024415\\
-0.014695	-0.0305175\\
-0.015015	-0.03357\\
-0.015335	-0.0305175\\
-0.0151525	-0.01831\\
-0.0146025	-0.0213625\\
-0.015015	-0.0488275\\
-0.015565	-0.0305175\\
-0.015335	-0.03357\\
-0.0152425	-0.0396725\\
-0.0154275	-0.027465\\
-0.0151975	-0.024415\\
-0.0149225	-0.024415\\
-0.015015	-0.024415\\
-0.0149225	-0.0213625\\
-0.015015	-0.03357\\
-0.0152425	-0.027465\\
-0.01506	-0.024415\\
-0.015105	-0.027465\\
-0.0151525	-0.027465\\
-0.015105	-0.0213625\\
-0.0149225	-0.024415\\
-0.01506	-0.0122075\\
-0.0148775	-0.024415\\
-0.014785	-0.0305175\\
-0.0148775	-0.024415\\
-0.0148775	-0.01831\\
-0.0142375	-0.009155\\
-0.013505	-0.0061025\\
-0.012955	-0.009155\\
-0.0131375	-0.024415\\
-0.01419	-0.03357\\
-0.0149225	-0.027465\\
-0.015015	-0.027465\\
-0.0146475	-0.03357\\
-0.0146025	-0.0305175\\
-0.0145575	-0.01831\\
-0.01442	-0.01526\\
-0.0139625	-0.009155\\
-0.0140525	-0.0213625\\
-0.0141	-0.0122075\\
-0.0141	-0.0213625\\
-0.0142375	-0.01831\\
-0.01419	-0.01831\\
-0.0140525	-0.01526\\
-0.0136875	-0.0122075\\
-0.013825	-0.024415\\
-0.01451	-0.027465\\
-0.014375	-0.01526\\
-0.014465	-0.027465\\
-0.014465	-0.0213625\\
-0.01474	-0.024415\\
-0.014785	-0.03662\\
-0.0151975	-0.0213625\\
-0.0151525	-0.0305175\\
-0.015105	-0.0305175\\
-0.0151525	-0.009155\\
-0.0146025	-0.0305175\\
-0.01497	-0.03357\\
-0.014785	-0.024415\\
-0.014785	-0.0396725\\
-0.0148775	-0.03357\\
-0.015105	-0.03357\\
-0.01497	-0.027465\\
-0.01506	-0.0396725\\
-0.0154275	-0.03357\\
-0.01538	-0.0305175\\
-0.0151525	-0.027465\\
-0.0149225	-0.0213625\\
-0.015015	-0.024415\\
-0.0148775	-0.0213625\\
-0.014695	-0.01526\\
-0.01451	-0.0213625\\
-0.01451	-0.01831\\
-0.01442	-0.027465\\
-0.01506	-0.0396725\\
-0.015655	-0.03357\\
-0.01561	-0.03662\\
-0.015565	-0.03662\\
-0.0155175	-0.027465\\
-0.01497	-0.01831\\
-0.0146475	-0.01831\\
-0.01442	-0.0122075\\
-0.014145	-0.01526\\
-0.014145	-0.024415\\
-0.0145575	-0.0305175\\
-0.0149225	-0.0305175\\
-0.0151525	-0.0305175\\
-0.015105	-0.0213625\\
-0.01474	-0.024415\\
-0.0151525	-0.042725\\
-0.015565	-0.03662\\
-0.0157025	-0.03357\\
-0.01561	-0.042725\\
-0.0161125	-0.0579825\\
-0.0160675	-0.01831\\
-0.015335	-0.01831\\
-0.01474	-0.0213625\\
-0.01442	-0.0213625\\
-0.0148325	-0.0305175\\
-0.01529	-0.042725\\
-0.015565	-0.03662\\
-0.0154725	-0.027465\\
-0.0151525	-0.024415\\
-0.01451	-0.01526\\
-0.0142375	-0.01526\\
-0.014375	-0.0213625\\
-0.014375	-0.01526\\
-0.0141	-0.009155\\
-0.0141	-0.01526\\
-0.0139625	-0.009155\\
-0.01355	-0.0122075\\
-0.0139625	-0.0305175\\
-0.01419	-0.01831\\
-0.014465	-0.024415\\
-0.01451	-0.01831\\
-0.01451	-0.024415\\
-0.0145575	-0.024415\\
-0.0148325	-0.027465\\
-0.0146025	-0.03357\\
-0.0148775	-0.027465\\
-0.0152425	-0.03357\\
-0.01497	-0.0305175\\
-0.0149225	-0.03357\\
-0.0148775	-0.024415\\
-0.0148775	-0.03662\\
-0.01529	-0.03357\\
-0.0151525	-0.01831\\
-0.0149225	-0.024415\\
-0.01497	-0.027465\\
-0.0148775	-0.0213625\\
-0.01451	-0.03357\\
-0.014695	-0.03357\\
-0.0151975	-0.03662\\
-0.0152425	-0.03357\\
-0.0154275	-0.05188\\
-0.0160675	-0.05188\\
-0.0158375	-0.027465\\
-0.015565	-0.0305175\\
-0.0152425	-0.027465\\
-0.015335	-0.0396725\\
-0.0158375	-0.01831\\
-0.01561	-0.027465\\
-0.015335	-0.0396725\\
-0.015655	-0.0122075\\
-0.01538	-0.01831\\
-0.014695	-0.024415\\
-0.0146025	-0.027465\\
-0.01442	-0.01526\\
-0.0140525	-0.01526\\
-0.0139625	-0.0122075\\
-0.013825	-0.0122075\\
-0.0137325	-0.01526\\
-0.01387	-0.01831\\
-0.01419	-0.01831\\
-0.01442	-0.024415\\
-0.0148325	-0.03357\\
-0.01497	-0.027465\\
-0.015105	-0.03662\\
-0.0154275	-0.03357\\
-0.0154275	-0.0305175\\
-0.015105	-0.0213625\\
-0.015015	-0.027465\\
-0.01497	-0.024415\\
-0.015015	-0.0396725\\
-0.015335	-0.03357\\
-0.015105	-0.03357\\
-0.015105	-0.0305175\\
-0.015105	-0.01831\\
-0.015015	-0.0305175\\
-0.0154275	-0.03662\\
-0.0154275	-0.027465\\
-0.01538	-0.03357\\
-0.01529	-0.024415\\
-0.0148325	-0.0122075\\
-0.0142375	-0.01526\\
-0.01387	-0.01526\\
-0.01419	-0.024415\\
-0.0142375	-0.01526\\
-0.014465	-0.0213625\\
-0.014375	-0.0213625\\
-0.01451	-0.03662\\
-0.01529	-0.05188\\
-0.0158375	-0.0305175\\
-0.0157025	-0.03662\\
-0.015655	-0.03662\\
-0.01561	-0.027465\\
-0.0152425	-0.024415\\
-0.0151525	-0.027465\\
-0.01529	-0.027465\\
-0.0151525	-0.0305175\\
-0.0151525	-0.03662\\
-0.0154275	-0.0457775\\
-0.015975	-0.0488275\\
-0.0161125	-0.03662\\
-0.015975	-0.0488275\\
-0.0160675	-0.0396725\\
-0.0157475	-0.03662\\
-0.015655	-0.0305175\\
-0.0155175	-0.027465\\
-0.01538	-0.03357\\
-0.015655	-0.0457775\\
-0.0158375	-0.01831\\
-0.015015	-0.0213625\\
-0.01474	-0.0305175\\
-0.0146475	-0.024415\\
-0.0148775	-0.0213625\\
-0.01497	-0.009155\\
-0.0146475	-0.01831\\
-0.0146475	-0.03357\\
-0.015335	-0.05188\\
-0.016205	-0.0549325\\
-0.01648	-0.0488275\\
-0.0163425	-0.03662\\
-0.016205	-0.042725\\
-0.0162975	-0.0549325\\
-0.016525	-0.042725\\
-0.0162975	-0.03662\\
-0.016205	-0.0457775\\
-0.01625	-0.03357\\
-0.015885	-0.03357\\
-0.015565	-0.0396725\\
-0.0157475	-0.0305175\\
-0.0154725	-0.01831\\
-0.01497	-0.027465\\
-0.01506	-0.027465\\
-0.0148775	-0.01526\\
-0.0148325	-0.024415\\
-0.0148325	-0.01831\\
-0.01497	-0.03357\\
-0.015335	-0.024415\\
-0.01529	-0.0213625\\
-0.0149225	-0.024415\\
-0.0149225	-0.0305175\\
-0.0151525	-0.0305175\\
-0.01529	-0.027465\\
-0.0152425	-0.01831\\
-0.01506	-0.03357\\
-0.01538	-0.03662\\
-0.0154275	-0.024415\\
-0.0151525	-0.03357\\
-0.0154725	-0.0305175\\
-0.015655	-0.024415\\
-0.01529	-0.024415\\
-0.0148775	-0.01526\\
-0.014375	-0.0122075\\
-0.013915	-0.01526\\
-0.014375	-0.024415\\
-0.0146025	-0.0122075\\
-0.0142375	-0.01526\\
-0.01387	-0.0122075\\
-0.01387	-0.0305175\\
-0.0143275	-0.027465\\
-0.0146025	-0.0305175\\
-0.015015	-0.03662\\
-0.015335	-0.03357\\
-0.01538	-0.03662\\
-0.0154275	-0.027465\\
-0.01506	-0.0122075\\
-0.0141	-0.0213625\\
-0.0139625	-0.01831\\
-0.0139625	-0.01831\\
-0.0141	-0.027465\\
-0.0146475	-0.027465\\
-0.0149225	-0.0396725\\
-0.01538	-0.0396725\\
-0.0151975	-0.024415\\
-0.015015	-0.0396725\\
-0.0154725	-0.042725\\
-0.01529	-0.01526\\
-0.0148775	-0.01831\\
-0.014465	-0.024415\\
-0.01451	-0.027465\\
-0.0148775	-0.03662\\
-0.0154275	-0.0122075\\
-0.01529	-0.027465\\
-0.015015	-0.024415\\
-0.0148775	-0.027465\\
-0.015015	-0.03357\\
-0.015335	-0.0457775\\
-0.015975	-0.0396725\\
-0.0160675	-0.03662\\
-0.0160225	-0.03357\\
-0.015565	-0.0213625\\
-0.015015	-0.024415\\
-0.01497	0\\
-0.0142825	-0.01831\\
-0.0142375	-0.027465\\
-0.0142375	-0.01831\\
-0.0142375	-0.0213625\\
-0.0148775	-0.042725\\
-0.0154275	-0.0396725\\
-0.0157925	-0.0457775\\
-0.0162975	-0.0457775\\
-0.01625	-0.027465\\
-0.0154725	-0.027465\\
-0.0151525	-0.0213625\\
-0.01497	-0.0213625\\
-0.0151525	-0.0396725\\
-0.0154725	-0.042725\\
-0.015975	-0.03662\\
-0.015885	-0.024415\\
-0.01538	-0.027465\\
-0.01538	-0.0396725\\
-0.0157025	-0.03357\\
-0.0154275	-0.01831\\
-0.01497	-0.01831\\
-0.0146025	-0.01831\\
-0.014145	-0.01831\\
-0.013825	-0.0122075\\
-0.01355	-0.01526\\
-0.0137325	-0.027465\\
-0.0142825	-0.024415\\
-0.01442	-0.01526\\
-0.0142825	-0.01831\\
-0.0141	-0.01526\\
-0.013505	-0.0030525\\
-0.012635	-0.0030525\\
-0.0125425	-0.01831\\
-0.012955	-0.01831\\
-0.0137325	-0.01831\\
-0.0141	-0.027465\\
-0.01442	-0.027465\\
-0.01474	-0.03357\\
-0.0152425	-0.0457775\\
-0.01593	-0.05188\\
-0.016205	-0.0457775\\
-0.0163425	-0.042725\\
-0.0163425	-0.0305175\\
-0.015655	-0.027465\\
-0.0152425	-0.0305175\\
-0.0152425	-0.0305175\\
-0.0155175	-0.0305175\\
-0.0154275	-0.0213625\\
-0.01506	-0.01831\\
-0.014695	-0.01526\\
-0.014375	-0.024415\\
-0.0146475	-0.027465\\
-0.0148325	-0.01831\\
-0.014145	-0.009155\\
-0.0136875	-0.01526\\
-0.013915	-0.01831\\
-0.0143275	-0.0122075\\
-0.01419	-0.0213625\\
-0.014375	-0.0213625\\
-0.0143275	-0.0213625\\
-0.0146025	-0.03662\\
-0.0152425	-0.027465\\
-0.0152425	-0.03662\\
-0.0154275	-0.0488275\\
-0.0160225	-0.0549325\\
-0.0162975	-0.03662\\
-0.016205	-0.027465\\
-0.0157025	-0.0213625\\
-0.0152425	-0.01831\\
-0.01451	-0.024415\\
-0.0146025	-0.0213625\\
-0.014465	-0.0213625\\
-0.0148325	-0.024415\\
-0.01497	-0.01831\\
-0.014695	-0.024415\\
-0.0148325	-0.0213625\\
-0.014375	-0.01831\\
-0.01442	-0.024415\\
-0.01419	-0.0122075\\
-0.0137325	-0.0061025\\
-0.0137325	-0.0030525\\
-0.01355	-0.009155\\
-0.013505	-0.01526\\
-0.0134575	-0.0061025\\
-0.0131375	-0.0061025\\
-0.0131375	-0.01831\\
-0.0134125	-0.01526\\
-0.013505	-0.01526\\
-0.0134575	-0.024415\\
-0.0139625	-0.027465\\
-0.014465	-0.0213625\\
-0.01442	-0.01831\\
-0.0143275	-0.03357\\
-0.01474	-0.0488275\\
-0.0152425	-0.03357\\
-0.0151525	-0.027465\\
-0.0151525	-0.024415\\
-0.014375	-0.009155\\
-0.0136875	-0.01831\\
-0.0140075	-0.0213625\\
-0.014695	-0.024415\\
-0.0148325	-0.03357\\
-0.0154275	-0.03357\\
-0.015565	-0.027465\\
-0.0151525	-0.024415\\
-0.014695	-0.024415\\
-0.0146025	-0.0213625\\
-0.0146475	-0.01831\\
-0.01451	-0.0122075\\
-0.0139625	-0.0122075\\
-0.013825	-0.0122075\\
-0.0137325	-0.009155\\
-0.013505	-0.01526\\
-0.0131375	-0.01831\\
-0.0134125	-0.027465\\
-0.0139625	-0.024415\\
-0.0141	-0.0122075\\
-0.0139625	-0.024415\\
-0.014375	-0.0305175\\
-0.01506	-0.01526\\
-0.0148775	-0.03357\\
-0.01497	-0.0305175\\
-0.0151975	-0.01831\\
-0.01497	-0.0213625\\
-0.0143275	-0.0305175\\
-0.0142375	-0.024415\\
-0.014465	-0.01526\\
-0.0140075	-0.024415\\
-0.013915	-0.0122075\\
-0.0137775	-0.009155\\
-0.01332	-0.0061025\\
-0.0130475	-0.0122075\\
-0.01332	-0.01831\\
-0.0136425	-0.01831\\
-0.0136875	-0.0213625\\
-0.013825	-0.0213625\\
-0.0141	-0.0122075\\
-0.01387	-0.009155\\
-0.01332	-0.0061025\\
-0.013	-0.0122075\\
-0.013	-0.0122075\\
-0.0131825	-0.0122075\\
-0.01332	-0.0213625\\
-0.0140525	-0.027465\\
-0.014465	-0.01526\\
-0.014375	-0.0213625\\
-0.0143275	-0.027465\\
-0.0146025	-0.03357\\
-0.015105	-0.03357\\
-0.01538	-0.03357\\
-0.0155175	-0.027465\\
-0.01529	-0.0305175\\
-0.0152425	-0.03357\\
-0.01529	-0.03662\\
-0.01529	-0.03357\\
-0.0154275	-0.0457775\\
-0.015885	-0.0396725\\
-0.01593	-0.0396725\\
-0.0157925	-0.0305175\\
-0.015565	-0.03357\\
-0.01538	-0.03357\\
-0.015335	-0.03357\\
-0.0154725	-0.024415\\
-0.01497	-0.01831\\
-0.0148775	-0.027465\\
-0.015105	-0.03357\\
-0.01538	-0.0305175\\
-0.0152425	-0.027465\\
-0.015335	-0.0213625\\
-0.0151975	-0.0305175\\
-0.015015	-0.0213625\\
-0.0145575	-0.03662\\
-0.014695	-0.0396725\\
-0.01529	-0.027465\\
-0.01529	-0.01526\\
-0.014695	-0.024415\\
-0.0146475	-0.0061025\\
-0.0142825	-0.009155\\
-0.013825	-0.01831\\
-0.0140075	-0.0122075\\
-0.0136875	-0.009155\\
-0.0134575	-0.0213625\\
-0.01355	-0.0122075\\
-0.01332	-0.01831\\
-0.013505	-0.01526\\
-0.0134575	-0.0122075\\
-0.0130475	-0.0122075\\
-0.013275	-0.01526\\
-0.0134125	-0.01526\\
-0.01355	-0.01831\\
-0.0136425	-0.01526\\
-0.0136875	-0.01831\\
-0.0140075	-0.01831\\
-0.0140525	-0.01831\\
-0.014145	-0.03662\\
-0.0149225	-0.024415\\
-0.0149225	-0.0213625\\
-0.014695	-0.027465\\
-0.01474	-0.01831\\
-0.014465	-0.01526\\
-0.01387	-0.027465\\
-0.014145	-0.0213625\\
-0.014145	-0.0122075\\
-0.0136875	-0.027465\\
-0.013915	-0.0213625\\
-0.0140525	-0.0213625\\
-0.0142375	-0.01831\\
-0.0137775	-0.0061025\\
-0.0133675	-0.009155\\
-0.013	-0.0122075\\
-0.012955	-0.0122075\\
-0.0133675	-0.01831\\
-0.0134575	-0.0213625\\
-0.013595	-0.0213625\\
-0.0140525	-0.024415\\
-0.0146025	-0.01831\\
-0.01451	-0.027465\\
-0.014695	-0.03662\\
-0.015105	-0.0305175\\
-0.01506	-0.027465\\
-0.01497	-0.03662\\
-0.0155175	-0.0305175\\
-0.0154725	-0.042725\\
-0.01561	-0.0396725\\
-0.01561	-0.0488275\\
-0.015655	-0.05188\\
-0.015885	-0.03357\\
-0.0154275	-0.01526\\
-0.01474	-0.01526\\
-0.0143275	-0.0213625\\
-0.0146475	-0.0305175\\
-0.0148775	-0.0305175\\
-0.015015	-0.0213625\\
-0.014785	-0.01831\\
-0.0140075	-0.027465\\
-0.013915	-0.03357\\
-0.0146025	-0.03662\\
-0.0151975	-0.03357\\
-0.015335	-0.024415\\
-0.0148325	-0.01526\\
-0.014465	-0.027465\\
-0.0146025	-0.0305175\\
-0.0151525	-0.03662\\
-0.0154275	-0.03662\\
-0.015565	-0.0305175\\
-0.01529	-0.0488275\\
-0.0155175	-0.0488275\\
-0.0158375	-0.042725\\
-0.0157925	-0.05188\\
-0.016205	-0.0457775\\
-0.0162975	-0.0396725\\
-0.0160675	-0.042725\\
-0.01616	-0.0396725\\
-0.0160675	-0.027465\\
-0.01529	-0.0213625\\
-0.01497	-0.024415\\
-0.015105	-0.0305175\\
-0.015335	-0.0305175\\
-0.0154725	-0.0213625\\
-0.0151975	-0.03357\\
-0.01529	-0.0305175\\
-0.01529	-0.0213625\\
-0.01497	-0.024415\\
-0.015105	-0.0305175\\
-0.0151525	-0.03357\\
-0.0154275	-0.03662\\
-0.015655	-0.027465\\
-0.015335	-0.024415\\
-0.015015	-0.0213625\\
-0.014375	-0.0305175\\
-0.01451	-0.0213625\\
-0.014375	-0.01831\\
-0.01419	-0.0061025\\
-0.0140525	-0.01831\\
-0.01442	-0.024415\\
-0.014785	-0.0305175\\
-0.01497	-0.0305175\\
-0.01506	-0.0213625\\
-0.0145575	-0.01526\\
-0.01419	-0.01526\\
-0.013825	-0.01831\\
-0.0137325	-0.0213625\\
-0.01419	-0.024415\\
-0.0143275	-0.01526\\
-0.0143275	-0.03357\\
-0.0148325	-0.03357\\
-0.0152425	-0.03662\\
-0.0154725	-0.0396725\\
-0.01561	-0.042725\\
-0.015885	-0.0305175\\
-0.0154275	-0.0305175\\
-0.01506	-0.01831\\
-0.014785	-0.0213625\\
-0.014695	-0.0305175\\
-0.0149225	-0.027465\\
-0.0148325	-0.01526\\
-0.01451	-0.01526\\
-0.0142375	-0.0213625\\
-0.014785	-0.0305175\\
-0.015105	-0.027465\\
-0.01506	-0.0457775\\
-0.015565	-0.0457775\\
-0.01593	-0.0305175\\
-0.01561	-0.0213625\\
-0.0151525	-0.01831\\
-0.0146025	-0.0213625\\
-0.014375	-0.03357\\
-0.0148775	-0.0305175\\
-0.0148775	-0.03357\\
-0.01497	-0.03662\\
-0.015105	-0.0305175\\
-0.0152425	-0.027465\\
-0.0152425	-0.03357\\
-0.015335	-0.024415\\
-0.01506	-0.0305175\\
-0.01506	-0.01831\\
-0.0148325	-0.01831\\
-0.0146025	-0.027465\\
-0.0148775	-0.03357\\
-0.015335	-0.03662\\
-0.01561	-0.0213625\\
-0.0151525	-0.03662\\
-0.0157475	-0.03357\\
-0.0160675	-0.0305175\\
-0.015565	-0.01831\\
-0.0148325	-0.042725\\
-0.01497	-0.0213625\\
-0.015015	-0.027465\\
-0.0148325	-0.0305175\\
-0.01497	-0.0213625\\
-0.014695	-0.01526\\
-0.014785	-0.042725\\
-0.015565	-0.05188\\
-0.015885	-0.0305175\\
-0.0154725	-0.0396725\\
-0.0154725	-0.0396725\\
-0.015565	-0.03357\\
-0.015655	-0.0305175\\
-0.01538	-0.027465\\
-0.01529	-0.0213625\\
-0.014785	-0.01831\\
-0.0149225	-0.027465\\
-0.015105	-0.042725\\
-0.015655	-0.042725\\
-0.015975	-0.05188\\
-0.01648	-0.0396725\\
-0.0163875	-0.03662\\
-0.0160225	-0.027465\\
-0.0157925	-0.03357\\
-0.0157025	-0.024415\\
-0.0155175	-0.0213625\\
-0.01497	-0.0213625\\
-0.01474	-0.01831\\
-0.01474	-0.01831\\
-0.01442	-0.009155\\
-0.0139625	-0.0213625\\
-0.0141	-0.027465\\
-0.014465	-0.01526\\
-0.0142825	-0.01526\\
-0.01387	-0.01526\\
-0.0137325	-0.01526\\
-0.01451	-0.03662\\
-0.015105	-0.01526\\
-0.01451	-0.009155\\
-0.01451	-0.01831\\
-0.0145575	-0.0213625\\
-0.0145575	-0.024415\\
-0.014695	-0.0213625\\
-0.0146475	-0.01831\\
-0.0142825	-0.01526\\
-0.013915	-0.0122075\\
-0.0136425	-0.0122075\\
-0.0136875	-0.027465\\
-0.0143275	-0.027465\\
-0.014695	-0.0213625\\
-0.01451	-0.01526\\
-0.014145	-0.0122075\\
-0.0136425	-0.0213625\\
-0.0136425	-0.027465\\
-0.014145	-0.027465\\
-0.0145575	-0.024415\\
-0.014695	-0.01526\\
-0.014465	-0.01831\\
-0.0142825	-0.01831\\
-0.01442	-0.027465\\
-0.0146475	-0.03662\\
-0.0152425	-0.0396725\\
-0.01561	-0.03357\\
-0.0154725	-0.01831\\
-0.0148775	-0.0305175\\
-0.01474	-0.024415\\
-0.014695	-0.024415\\
-0.01474	-0.01526\\
-0.014375	-0.024415\\
-0.014375	-0.0213625\\
-0.014465	-0.0213625\\
-0.014465	-0.01526\\
-0.014145	-0.0122075\\
-0.013595	-0.01831\\
-0.0137325	-0.024415\\
-0.0141	-0.03357\\
-0.014695	-0.03357\\
-0.01497	-0.0305175\\
-0.01529	-0.03357\\
-0.01529	-0.03357\\
-0.0154275	-0.042725\\
-0.0158375	-0.0549325\\
-0.01648	-0.0579825\\
-0.0164325	-0.03357\\
-0.015975	-0.042725\\
-0.01593	-0.0396725\\
-0.0161125	-0.05188\\
-0.01648	-0.03662\\
-0.0161125	-0.01831\\
-0.0151525	-0.01526\\
-0.01442	-0.01831\\
-0.0142375	-0.01526\\
-0.01387	-0.009155\\
-0.013275	-0.0061025\\
-0.0130925	-0.0122075\\
-0.0134575	-0.0213625\\
-0.013825	-0.024415\\
-0.0140525	-0.0122075\\
-0.01387	-0.01831\\
-0.0137325	-0.01526\\
-0.013915	-0.0213625\\
-0.01419	-0.0213625\\
-0.0142825	-0.01831\\
-0.0142825	-0.0122075\\
-0.0140525	-0.009155\\
-0.0136425	-0.009155\\
-0.0134575	-0.01526\\
-0.0137325	-0.01831\\
-0.0140075	-0.009155\\
-0.01387	-0.009155\\
-0.0134575	-0.024415\\
-0.0137775	-0.01831\\
-0.013595	-0.024415\\
-0.013595	-0.01831\\
-0.013825	-0.0213625\\
-0.01419	-0.0305175\\
-0.014465	-0.024415\\
-0.01419	-0.0305175\\
-0.014465	-0.027465\\
-0.014695	-0.01831\\
-0.01451	-0.01526\\
-0.0143275	-0.03357\\
-0.014695	-0.03357\\
-0.0151525	-0.03357\\
-0.0152425	-0.03662\\
-0.01538	-0.027465\\
-0.0152425	-0.027465\\
-0.015105	-0.027465\\
-0.01506	-0.0305175\\
-0.01538	-0.0213625\\
-0.0151975	-0.042725\\
-0.0154275	-0.03357\\
-0.0155175	-0.03662\\
-0.0154725	-0.0579825\\
-0.0163425	-0.0396725\\
-0.0163425	-0.0213625\\
-0.015565	-0.01831\\
-0.0148775	-0.01526\\
-0.014695	-0.024415\\
-0.0148775	-0.03357\\
-0.0151525	-0.0396725\\
-0.0154275	-0.03662\\
-0.01561	-0.0396725\\
-0.0157025	-0.024415\\
-0.015335	-0.024415\\
-0.015105	-0.027465\\
-0.0151975	-0.0305175\\
-0.0151975	-0.01526\\
-0.014785	-0.0122075\\
-0.0142825	-0.01831\\
-0.01451	-0.0213625\\
-0.0146025	-0.027465\\
-0.0148775	-0.03357\\
-0.01529	-0.027465\\
-0.01529	-0.0213625\\
-0.0149225	-0.01831\\
-0.014695	-0.027465\\
-0.01497	-0.03662\\
-0.015335	-0.0396725\\
-0.0157475	-0.042725\\
-0.0160675	-0.05188\\
-0.0162975	-0.03662\\
-0.016205	-0.0457775\\
-0.0160675	-0.0305175\\
-0.0155175	-0.0213625\\
-0.0148775	-0.03662\\
-0.0152425	-0.0396725\\
-0.0157475	-0.027465\\
-0.0151975	-0.027465\\
-0.0154275	-0.0396725\\
-0.015975	-0.0488275\\
-0.01648	-0.042725\\
-0.01616	-0.01831\\
-0.015335	-0.01526\\
-0.0146025	-0.0305175\\
-0.0146025	-0.01526\\
-0.0146475	-0.0213625\\
-0.0146025	-0.01526\\
-0.01419	-0.024415\\
-0.01419	-0.009155\\
-0.0140525	-0.0122075\\
-0.0140525	-0.01831\\
-0.0140075	-0.024415\\
-0.014145	-0.027465\\
-0.0146475	-0.0305175\\
-0.01474	-0.03357\\
-0.015105	-0.0396725\\
-0.0155175	-0.03357\\
-0.01561	-0.0213625\\
-0.015105	-0.01831\\
-0.0148325	-0.027465\\
-0.015105	-0.0213625\\
-0.01474	-0.01831\\
-0.0145575	-0.0122075\\
-0.0142375	-0.0122075\\
-0.01451	-0.01526\\
-0.0145575	-0.0122075\\
-0.0140525	-0.009155\\
-0.0134125	-0.0122075\\
-0.0130925	-0.0122075\\
-0.013505	-0.0213625\\
-0.0139625	-0.01831\\
-0.014145	-0.01526\\
-0.013825	-0.0122075\\
-0.0137775	-0.024415\\
-0.01419	-0.024415\\
-0.0146025	-0.01831\\
-0.014465	-0.024415\\
-0.0142375	-0.0213625\\
-0.01442	-0.027465\\
-0.0146025	-0.024415\\
-0.0146025	-0.027465\\
-0.01474	-0.024415\\
-0.01474	-0.01831\\
-0.014465	-0.01831\\
-0.01419	-0.01831\\
-0.0142825	-0.0213625\\
-0.01419	-0.0213625\\
-0.0142825	-0.024415\\
-0.0146475	-0.03662\\
-0.01529	-0.024415\\
-0.0151525	-0.027465\\
-0.015015	-0.0396725\\
-0.0154725	-0.027465\\
-0.01538	-0.024415\\
-0.0148775	-0.0213625\\
-0.014465	-0.01831\\
-0.0141	-0.0122075\\
-0.0139625	-0.009155\\
-0.013825	-0.01526\\
-0.0136875	-0.027465\\
-0.0143275	-0.01831\\
-0.0137775	-0.024415\\
-0.0133675	-0.01831\\
-0.0136425	-0.0213625\\
-0.0140075	-0.0061025\\
-0.0139625	-0.0122075\\
-0.0136425	-0.0061025\\
-0.0131825	-0.0061025\\
-0.0134125	-0.0213625\\
-0.0141	-0.027465\\
-0.014695	-0.01831\\
-0.0146025	-0.01831\\
-0.0143275	-0.024415\\
-0.014375	-0.024415\\
-0.01442	-0.024415\\
-0.0145575	-0.0305175\\
-0.014785	-0.0305175\\
-0.015015	-0.03357\\
-0.0151525	-0.0305175\\
-0.015015	-0.0213625\\
-0.014695	-0.01831\\
-0.0142375	-0.01831\\
-0.0142375	-0.024415\\
-0.01451	-0.0213625\\
-0.01419	-0.0213625\\
-0.013915	-0.0213625\\
-0.014465	-0.0305175\\
-0.0148325	-0.027465\\
-0.0148775	-0.0305175\\
-0.01497	-0.0396725\\
-0.0154275	-0.0213625\\
-0.01497	-0.0213625\\
-0.01538	-0.03357\\
-0.01561	-0.03357\\
-0.0154725	-0.0396725\\
-0.0155175	-0.03662\\
-0.015565	-0.0549325\\
-0.0163875	-0.0640875\\
-0.0167075	-0.0396725\\
-0.01625	-0.0457775\\
-0.0163425	-0.0549325\\
-0.0166175	-0.05188\\
-0.0166625	-0.061035\\
-0.0168	-0.0579825\\
-0.016755	-0.0640875\\
-0.01712	-0.05188\\
-0.0169825	-0.03662\\
-0.01648	-0.027465\\
-0.015885	-0.03357\\
-0.015565	-0.024415\\
-0.015335	-0.0213625\\
-0.01497	-0.024415\\
-0.0152425	-0.03357\\
-0.0157025	-0.027465\\
-0.015335	-0.0213625\\
-0.015015	-0.0213625\\
-0.015015	-0.0213625\\
-0.01451	-0.0213625\\
-0.014465	-0.01526\\
-0.01387	-0.027465\\
-0.0136425	-0.01831\\
-0.01355	-0.01831\\
-0.0137325	-0.01526\\
-0.013595	-0.01831\\
-0.01332	-0.009155\\
-0.012955	-0.009155\\
-0.0125875	-0.009155\\
-0.012955	-0.009155\\
-0.0128175	-0.01526\\
-0.012725	-0.01526\\
-0.01355	-0.024415\\
-0.014145	-0.027465\\
-0.014375	-0.0213625\\
-0.01419	-0.0213625\\
-0.014375	-0.0305175\\
-0.01497	-0.024415\\
-0.01451	-0.0122075\\
-0.0137775	-0.01831\\
-0.0133675	-0.0122075\\
-0.012955	-0.01526\\
-0.0131825	-0.0213625\\
-0.013595	-0.0213625\\
-0.0142375	-0.0122075\\
-0.0139625	-0.024415\\
-0.014375	-0.03662\\
-0.01506	-0.0396725\\
-0.0151975	-0.027465\\
-0.015015	-0.01831\\
-0.014375	-0.0213625\\
-0.01442	-0.0305175\\
-0.014785	-0.03357\\
-0.0151975	-0.01831\\
-0.0148775	-0.01526\\
-0.0140525	-0.01526\\
-0.0142375	-0.0213625\\
-0.014145	-0.0061025\\
-0.0133675	-0.0061025\\
-0.01268	-0.01831\\
-0.0130925	-0.027465\\
-0.0142825	-0.027465\\
-0.0146475	-0.01831\\
-0.0143275	-0.01526\\
-0.013825	-0.01526\\
-0.0136875	-0.009155\\
-0.012635	-0.01831\\
-0.0124975	-0.0030525\\
-0.012635	-0.009155\\
-0.012955	-0.01831\\
-0.0134575	-0.027465\\
-0.0142375	-0.027465\\
-0.0145575	-0.027465\\
-0.01474	-0.0305175\\
-0.0148325	-0.027465\\
-0.0149225	-0.027465\\
-0.0148775	-0.0213625\\
-0.014785	-0.027465\\
-0.01497	-0.042725\\
-0.015655	-0.0457775\\
-0.0158375	-0.0213625\\
-0.015105	-0.0122075\\
-0.014145	-0.0122075\\
-0.01474	-0.03662\\
-0.0151975	-0.027465\\
-0.015105	-0.0213625\\
-0.01451	-0.024415\\
-0.0142825	-0.027465\\
-0.014785	-0.042725\\
-0.0155175	-0.0457775\\
-0.0157925	-0.0457775\\
-0.015975	-0.0488275\\
-0.0161125	-0.03662\\
-0.015885	-0.03357\\
-0.0157925	-0.024415\\
-0.01497	-0.024415\\
-0.01442	-0.0213625\\
-0.0146475	-0.027465\\
-0.0148775	-0.024415\\
-0.01497	-0.0305175\\
-0.01506	-0.0213625\\
-0.01474	-0.024415\\
-0.0148325	-0.0213625\\
-0.0148325	-0.01526\\
-0.0142825	-0.0122075\\
-0.0136425	-0.0122075\\
-0.01323	-0.0122075\\
-0.0134125	-0.0122075\\
-0.01332	-0.0061025\\
-0.0127725	-0.0122075\\
-0.0130475	-0.01831\\
-0.01387	-0.01526\\
-0.013825	-0.01526\\
-0.013595	-0.0213625\\
-0.013825	-0.0305175\\
-0.0145575	-0.0213625\\
-0.014465	-0.0122075\\
-0.0137775	-0.0030525\\
-0.0130925	-0.01831\\
-0.0137775	-0.03357\\
-0.014785	-0.03662\\
-0.01538	-0.05188\\
-0.016205	-0.061035\\
-0.016755	-0.061035\\
-0.0168925	-0.0396725\\
-0.0164325	-0.042725\\
-0.0163425	-0.0549325\\
-0.01657	-0.0457775\\
-0.016525	-0.0396725\\
-0.0163875	-0.0457775\\
-0.016525	-0.05188\\
-0.0166175	-0.0488275\\
-0.0166625	-0.0671375\\
-0.01703	-0.05188\\
-0.0168	-0.0305175\\
-0.015975	-0.01831\\
-0.015015	-0.01526\\
-0.014375	-0.01526\\
-0.0142825	-0.01526\\
-0.0142825	-0.01526\\
-0.0142375	-0.027465\\
-0.0148325	-0.0305175\\
-0.014695	-0.0213625\\
-0.0146475	-0.027465\\
-0.01497	-0.01526\\
-0.0146025	-0.009155\\
-0.0146025	-0.024415\\
-0.0148775	-0.0305175\\
-0.0151525	-0.024415\\
-0.0146475	-0.01526\\
-0.0141	-0.01526\\
-0.014145	-0.01831\\
-0.014145	-0.027465\\
-0.015105	-0.042725\\
-0.015885	-0.0396725\\
-0.01593	-0.042725\\
-0.01616	-0.05188\\
-0.01648	-0.0640875\\
-0.0169825	-0.0549325\\
-0.017075	-0.042725\\
-0.01648	-0.027465\\
-0.0157925	-0.0213625\\
-0.0148775	-0.0122075\\
-0.0146025	-0.01831\\
-0.0146475	-0.024415\\
-0.0148325	-0.0213625\\
-0.0145575	-0.01526\\
-0.014375	-0.01831\\
-0.014465	-0.0213625\\
-0.0146025	-0.0213625\\
-0.01474	-0.027465\\
-0.01506	-0.024415\\
-0.0148325	-0.0122075\\
-0.014145	-0.0122075\\
-0.013505	-0.0061025\\
-0.0131375	-0.0061025\\
-0.013	-0.009155\\
-0.0130475	-0.01831\\
-0.01355	-0.0213625\\
-0.0137775	-0.01831\\
-0.0140075	-0.027465\\
-0.0146475	-0.0213625\\
-0.0145575	-0.0213625\\
-0.015015	-0.0488275\\
-0.015975	-0.03662\\
-0.015885	-0.0305175\\
-0.0157475	-0.042725\\
-0.0158375	-0.0457775\\
-0.0161125	-0.03357\\
-0.0158375	-0.0305175\\
-0.0155175	-0.027465\\
-0.0149225	-0.024415\\
-0.014785	-0.03662\\
-0.015565	-0.061035\\
-0.01657	-0.0640875\\
-0.017165	-0.05188\\
-0.017075	-0.0488275\\
-0.016845	-0.03662\\
-0.0164325	-0.042725\\
-0.0163425	-0.0488275\\
-0.0166625	-0.03662\\
-0.0161125	-0.027465\\
-0.0157025	-0.024415\\
-0.01538	-0.0305175\\
-0.015885	-0.03357\\
-0.0160225	-0.0213625\\
-0.0154275	-0.01831\\
-0.01506	-0.01831\\
-0.014785	-0.0213625\\
-0.015105	-0.03662\\
-0.0157025	-0.042725\\
-0.015885	-0.0305175\\
-0.01561	-0.027465\\
-0.01529	-0.024415\\
-0.0152425	-0.024415\\
-0.015105	-0.01526\\
-0.0146475	-0.0122075\\
-0.0142825	-0.01526\\
-0.0139625	-0.01526\\
-0.0136425	-0.0122075\\
-0.013595	-0.01831\\
-0.0139625	-0.024415\\
-0.0145575	-0.027465\\
-0.014695	-0.0213625\\
-0.0143275	-0.009155\\
-0.0140525	-0.009155\\
-0.013915	-0.01526\\
-0.0139625	-0.0213625\\
-0.0143275	-0.024415\\
-0.01451	-0.0213625\\
-0.014465	-0.0122075\\
-0.0141	-0.027465\\
-0.0148325	-0.03662\\
-0.0154275	-0.027465\\
-0.0151525	-0.01831\\
-0.01419	-0.024415\\
-0.013915	-0.0213625\\
-0.0143275	-0.0213625\\
-0.01442	-0.01526\\
-0.014375	-0.01526\\
-0.013915	-0.0122075\\
-0.0142825	-0.027465\\
-0.0148325	-0.024415\\
-0.0145575	-0.009155\\
-0.0136425	-0.01526\\
-0.0134125	-0.024415\\
-0.0143275	-0.0305175\\
-0.0148325	-0.01831\\
-0.01442	-0.01526\\
-0.0141	-0.027465\\
-0.014695	-0.03357\\
-0.015335	-0.0305175\\
-0.01538	-0.042725\\
-0.0157475	-0.0488275\\
-0.0163425	-0.0305175\\
-0.0160225	-0.0305175\\
-0.015655	-0.042725\\
-0.01593	-0.03357\\
-0.015655	-0.027465\\
-0.0157475	-0.0457775\\
-0.0160675	-0.03662\\
-0.015885	-0.01831\\
-0.015105	-0.0213625\\
-0.01529	-0.0488275\\
-0.016205	-0.0549325\\
-0.01657	-0.042725\\
-0.0163425	-0.0305175\\
-0.0160675	-0.042725\\
-0.0162975	-0.03662\\
-0.0161125	-0.024415\\
-0.015565	-0.0213625\\
-0.0152425	-0.0305175\\
-0.01538	-0.0305175\\
-0.0155175	-0.0305175\\
-0.0154725	-0.0305175\\
-0.01561	-0.024415\\
-0.0152425	-0.024415\\
-0.0151975	-0.0305175\\
-0.0155175	-0.027465\\
-0.0151525	-0.01526\\
-0.01474	-0.01526\\
-0.014465	-0.0122075\\
-0.014145	-0.01526\\
-0.0140525	-0.027465\\
-0.01451	-0.0213625\\
-0.014695	-0.03357\\
-0.0148775	-0.03357\\
-0.015335	-0.0396725\\
-0.0157025	-0.03357\\
-0.015565	-0.024415\\
-0.015335	-0.024415\\
-0.01451	-0.01526\\
-0.0145575	-0.03662\\
-0.01529	-0.027465\\
-0.015105	-0.01831\\
-0.0143275	-0.03357\\
-0.0146025	-0.027465\\
-0.01529	-0.03357\\
-0.0154725	-0.0396725\\
-0.015885	-0.05188\\
-0.016525	-0.042725\\
-0.0163425	-0.03662\\
-0.0160675	-0.0396725\\
-0.016205	-0.03662\\
-0.0158375	-0.01831\\
-0.0154275	-0.0305175\\
-0.0154725	-0.03357\\
-0.0157025	-0.03357\\
-0.0158375	-0.0396725\\
-0.0158375	-0.0305175\\
-0.015335	-0.0213625\\
-0.01497	-0.0213625\\
-0.01474	-0.0213625\\
-0.0148775	-0.0213625\\
-0.0148775	-0.01526\\
-0.014695	-0.027465\\
-0.01529	-0.03357\\
-0.01561	-0.01831\\
-0.01529	-0.01831\\
-0.01506	-0.027465\\
-0.015335	-0.024415\\
-0.0149225	-0.009155\\
-0.01419	-0.01526\\
-0.0141	-0.024415\\
-0.0143275	-0.01526\\
-0.0142825	-0.0122075\\
-0.0140075	-0.0122075\\
-0.0133675	-0.0061025\\
-0.0134575	-0.0213625\\
-0.0139625	-0.0213625\\
-0.0146025	-0.024415\\
-0.01497	-0.03357\\
-0.0154275	-0.042725\\
-0.0157925	-0.03357\\
-0.0151975	-0.01526\\
-0.014695	-0.0305175\\
-0.0154725	-0.05188\\
-0.016205	-0.0396725\\
-0.0161125	-0.03662\\
-0.015975	-0.0488275\\
-0.01657	-0.042725\\
-0.016525	-0.03357\\
-0.016205	-0.03357\\
-0.015885	-0.0305175\\
-0.0157025	-0.0396725\\
-0.015975	-0.03662\\
-0.0157025	-0.027465\\
-0.01538	-0.03662\\
-0.015885	-0.05188\\
-0.0162975	-0.0457775\\
-0.0164325	-0.024415\\
-0.01561	-0.01831\\
-0.0146475	-0.01526\\
-0.01529	-0.03662\\
-0.0152425	-0.0061025\\
-0.014375	-0.009155\\
-0.0137775	-0.01831\\
-0.0140075	-0.027465\\
-0.01451	-0.0305175\\
-0.0152425	-0.0305175\\
-0.015105	-0.024415\\
-0.01474	-0.009155\\
-0.0142375	-0.0122075\\
-0.0136425	-0.009155\\
-0.0136425	-0.01831\\
-0.01355	-0.009155\\
-0.0137325	-0.0213625\\
-0.0142825	-0.0213625\\
-0.0143275	-0.01831\\
-0.0137325	-0.01526\\
-0.0134125	-0.01526\\
-0.01355	-0.01526\\
-0.0137775	-0.01831\\
-0.0134575	-0.009155\\
-0.013595	0\\
-0.013595	-0.0061025\\
-0.0131825	-0.01526\\
-0.0137325	-0.027465\\
-0.01442	-0.027465\\
-0.015105	-0.0457775\\
-0.015885	-0.03662\\
-0.015975	-0.024415\\
-0.0151525	-0.01526\\
-0.0145575	-0.0213625\\
-0.01442	-0.0030525\\
-0.013915	-0.009155\\
-0.0134575	-0.01526\\
-0.0139625	-0.0305175\\
-0.0142375	-0.01831\\
-0.01419	-0.01831\\
-0.0142825	-0.024415\\
-0.014465	-0.027465\\
-0.0146025	-0.024415\\
-0.014695	-0.027465\\
-0.015015	-0.03357\\
-0.015105	-0.042725\\
-0.0155175	-0.03662\\
-0.01506	-0.0061025\\
-0.0141	-0.009155\\
-0.0137775	-0.0213625\\
-0.0142375	-0.0213625\\
-0.01419	-0.0122075\\
-0.0140525	-0.01526\\
-0.0134125	-0.009155\\
-0.0133675	-0.0030525\\
-0.0131375	-0.0030525\\
-0.0125875	-0.0061025\\
-0.01268	-0.0122075\\
-0.01332	-0.0122075\\
-0.0137325	-0.0213625\\
-0.01442	-0.03357\\
-0.01506	-0.03357\\
-0.01497	-0.01526\\
-0.013915	-0.01526\\
-0.014145	-0.03357\\
-0.014465	-0.01831\\
-0.01387	-0.009155\\
-0.013825	-0.0305175\\
-0.0145575	-0.0213625\\
-0.014785	-0.03357\\
-0.01538	-0.0549325\\
-0.015975	-0.027465\\
-0.015565	-0.0213625\\
-0.0151525	-0.027465\\
-0.01497	-0.024415\\
-0.0151525	-0.0396725\\
-0.01561	-0.03357\\
-0.01538	-0.0213625\\
-0.0148325	-0.027465\\
};
\addplot [color=mycolor2, line width=2.0pt, forget plot]
  table[row sep=crcr]{%
-0.015565	-0.015565\\
-0.015655	-0.015655\\
-0.0157475	-0.0157475\\
-0.01561	-0.01561\\
-0.0148775	-0.0148775\\
-0.014785	-0.014785\\
-0.01497	-0.01497\\
-0.0148325	-0.0148325\\
-0.0140075	-0.0140075\\
-0.013275	-0.013275\\
-0.0128625	-0.0128625\\
-0.0137775	-0.0137775\\
-0.014785	-0.014785\\
-0.01506	-0.01506\\
-0.0151525	-0.0151525\\
-0.0149225	-0.0149225\\
-0.014375	-0.014375\\
-0.0149225	-0.0149225\\
-0.01474	-0.01474\\
-0.0143275	-0.0143275\\
-0.0148325	-0.0148325\\
-0.0149225	-0.0149225\\
-0.01474	-0.01474\\
-0.0151525	-0.0151525\\
-0.0152425	-0.0152425\\
-0.01497	-0.01497\\
-0.01529	-0.01529\\
-0.01538	-0.01538\\
-0.0151975	-0.0151975\\
-0.0149225	-0.0149225\\
-0.014695	-0.014695\\
-0.0149225	-0.0149225\\
-0.01497	-0.01497\\
-0.015015	-0.015015\\
-0.01506	-0.01506\\
-0.0154725	-0.0154725\\
-0.0155175	-0.0155175\\
-0.015105	-0.015105\\
-0.0148775	-0.0148775\\
-0.0143275	-0.0143275\\
-0.0141	-0.0141\\
-0.0143275	-0.0143275\\
-0.01419	-0.01419\\
-0.0141	-0.0141\\
-0.01442	-0.01442\\
-0.0146475	-0.0146475\\
-0.0151525	-0.0151525\\
-0.015105	-0.015105\\
-0.01451	-0.01451\\
-0.014145	-0.014145\\
-0.01419	-0.01419\\
-0.014375	-0.014375\\
-0.0139625	-0.0139625\\
-0.0136875	-0.0136875\\
-0.0140075	-0.0140075\\
-0.013915	-0.013915\\
-0.0136875	-0.0136875\\
-0.013825	-0.013825\\
-0.0140075	-0.0140075\\
-0.0146475	-0.0146475\\
-0.014785	-0.014785\\
-0.015015	-0.015015\\
-0.01506	-0.01506\\
-0.0151975	-0.0151975\\
-0.01506	-0.01506\\
-0.01497	-0.01497\\
-0.0148775	-0.0148775\\
-0.01474	-0.01474\\
-0.0152425	-0.0152425\\
-0.015975	-0.015975\\
-0.0162975	-0.0162975\\
-0.0163425	-0.0163425\\
-0.0158375	-0.0158375\\
-0.0161125	-0.0161125\\
-0.016525	-0.016525\\
-0.0166175	-0.0166175\\
-0.01648	-0.01648\\
-0.0166175	-0.0166175\\
-0.017165	-0.017165\\
-0.0168925	-0.0168925\\
-0.0166625	-0.0166625\\
-0.0161125	-0.0161125\\
-0.015565	-0.015565\\
-0.0152425	-0.0152425\\
-0.01538	-0.01538\\
-0.015105	-0.015105\\
-0.01451	-0.01451\\
-0.0142375	-0.0142375\\
-0.0142825	-0.0142825\\
-0.014695	-0.014695\\
-0.014785	-0.014785\\
-0.01474	-0.01474\\
-0.01506	-0.01506\\
-0.0151525	-0.0151525\\
-0.01561	-0.01561\\
-0.015565	-0.015565\\
-0.0157475	-0.0157475\\
-0.0155175	-0.0155175\\
-0.01529	-0.01529\\
-0.01538	-0.01538\\
-0.015105	-0.015105\\
-0.0149225	-0.0149225\\
-0.01506	-0.01506\\
-0.015105	-0.015105\\
-0.01474	-0.01474\\
-0.014695	-0.014695\\
-0.014375	-0.014375\\
-0.0142825	-0.0142825\\
-0.014375	-0.014375\\
-0.0141	-0.0141\\
-0.014145	-0.014145\\
-0.0143275	-0.0143275\\
-0.0148775	-0.0148775\\
-0.0151525	-0.0151525\\
-0.01529	-0.01529\\
-0.014785	-0.014785\\
-0.0149225	-0.0149225\\
-0.01561	-0.01561\\
-0.0155175	-0.0155175\\
-0.0157025	-0.0157025\\
-0.0151525	-0.0151525\\
-0.0146475	-0.0146475\\
-0.014785	-0.014785\\
-0.01497	-0.01497\\
-0.0155175	-0.0155175\\
-0.0160675	-0.0160675\\
-0.016205	-0.016205\\
-0.0158375	-0.0158375\\
-0.0157925	-0.0157925\\
-0.0157475	-0.0157475\\
-0.015655	-0.015655\\
-0.01561	-0.01561\\
-0.0151975	-0.0151975\\
-0.0148775	-0.0148775\\
-0.01474	-0.01474\\
-0.0146025	-0.0146025\\
-0.0146475	-0.0146475\\
-0.01497	-0.01497\\
-0.015565	-0.015565\\
-0.0155175	-0.0155175\\
-0.015105	-0.015105\\
-0.0148325	-0.0148325\\
-0.01474	-0.01474\\
-0.0142375	-0.0142375\\
-0.0136875	-0.0136875\\
-0.013595	-0.013595\\
-0.0141	-0.0141\\
-0.0139625	-0.0139625\\
-0.013915	-0.013915\\
-0.0140525	-0.0140525\\
-0.013825	-0.013825\\
-0.0134125	-0.0134125\\
-0.013	-0.013\\
-0.01323	-0.01323\\
-0.014145	-0.014145\\
-0.01497	-0.01497\\
-0.015335	-0.015335\\
-0.01506	-0.01506\\
-0.0146025	-0.0146025\\
-0.014145	-0.014145\\
-0.0136425	-0.0136425\\
-0.0140525	-0.0140525\\
-0.0143275	-0.0143275\\
-0.0146475	-0.0146475\\
-0.01506	-0.01506\\
-0.01593	-0.01593\\
-0.01657	-0.01657\\
-0.016525	-0.016525\\
-0.01648	-0.01648\\
-0.015975	-0.015975\\
-0.01593	-0.01593\\
-0.0160675	-0.0160675\\
-0.0158375	-0.0158375\\
-0.015655	-0.015655\\
-0.0154725	-0.0154725\\
-0.015565	-0.015565\\
-0.01538	-0.01538\\
-0.0157475	-0.0157475\\
-0.01616	-0.01616\\
-0.0158375	-0.0158375\\
-0.0152425	-0.0152425\\
-0.015105	-0.015105\\
-0.015655	-0.015655\\
-0.0160225	-0.0160225\\
-0.0164325	-0.0164325\\
-0.0166175	-0.0166175\\
-0.0167075	-0.0167075\\
-0.0164325	-0.0164325\\
-0.01625	-0.01625\\
-0.0163425	-0.0163425\\
-0.016525	-0.016525\\
-0.01616	-0.01616\\
-0.0154275	-0.0154275\\
-0.015565	-0.015565\\
-0.0154725	-0.0154725\\
-0.01497	-0.01497\\
-0.0151525	-0.0151525\\
-0.0149225	-0.0149225\\
-0.014695	-0.014695\\
-0.014785	-0.014785\\
-0.014695	-0.014695\\
-0.0148775	-0.0148775\\
-0.015335	-0.015335\\
-0.01538	-0.01538\\
-0.015015	-0.015015\\
-0.0148775	-0.0148775\\
-0.01529	-0.01529\\
-0.01616	-0.01616\\
-0.0161125	-0.0161125\\
-0.0154725	-0.0154725\\
-0.01506	-0.01506\\
-0.015015	-0.015015\\
-0.01497	-0.01497\\
-0.0146475	-0.0146475\\
-0.0145575	-0.0145575\\
-0.014695	-0.014695\\
-0.0149225	-0.0149225\\
-0.01506	-0.01506\\
-0.0148325	-0.0148325\\
-0.015105	-0.015105\\
-0.0154725	-0.0154725\\
-0.01529	-0.01529\\
-0.015335	-0.015335\\
-0.01561	-0.01561\\
-0.01529	-0.01529\\
-0.014375	-0.014375\\
-0.0142375	-0.0142375\\
-0.014375	-0.014375\\
-0.014695	-0.014695\\
-0.014375	-0.014375\\
-0.014695	-0.014695\\
-0.014785	-0.014785\\
-0.01451	-0.01451\\
-0.0146025	-0.0146025\\
-0.0145575	-0.0145575\\
-0.014465	-0.014465\\
-0.0145575	-0.0145575\\
-0.01419	-0.01419\\
-0.0141	-0.0141\\
-0.01387	-0.01387\\
-0.0137325	-0.0137325\\
-0.0143275	-0.0143275\\
-0.0146025	-0.0146025\\
-0.01506	-0.01506\\
-0.01561	-0.01561\\
-0.015335	-0.015335\\
-0.014695	-0.014695\\
-0.0142825	-0.0142825\\
-0.0141	-0.0141\\
-0.01442	-0.01442\\
-0.014145	-0.014145\\
-0.0141	-0.0141\\
-0.0142375	-0.0142375\\
-0.01497	-0.01497\\
-0.015105	-0.015105\\
-0.0154275	-0.0154275\\
-0.0151525	-0.0151525\\
-0.0149225	-0.0149225\\
-0.0142825	-0.0142825\\
-0.0139625	-0.0139625\\
-0.01355	-0.01355\\
-0.0133675	-0.0133675\\
-0.0134575	-0.0134575\\
-0.0140075	-0.0140075\\
-0.0142375	-0.0142375\\
-0.01442	-0.01442\\
-0.0145575	-0.0145575\\
-0.0152425	-0.0152425\\
-0.01538	-0.01538\\
-0.01506	-0.01506\\
-0.0151975	-0.0151975\\
-0.01529	-0.01529\\
-0.015565	-0.015565\\
-0.0154725	-0.0154725\\
-0.01506	-0.01506\\
-0.014145	-0.014145\\
-0.01355	-0.01355\\
-0.0133675	-0.0133675\\
-0.0134125	-0.0134125\\
-0.013505	-0.013505\\
-0.0134575	-0.0134575\\
-0.0136875	-0.0136875\\
-0.01419	-0.01419\\
-0.0142825	-0.0142825\\
-0.014465	-0.014465\\
-0.0140525	-0.0140525\\
-0.0134125	-0.0134125\\
-0.013825	-0.013825\\
-0.01442	-0.01442\\
-0.0145575	-0.0145575\\
-0.014695	-0.014695\\
-0.0143275	-0.0143275\\
-0.01451	-0.01451\\
-0.015105	-0.015105\\
-0.0151975	-0.0151975\\
-0.0149225	-0.0149225\\
-0.015335	-0.015335\\
-0.0151975	-0.0151975\\
-0.015335	-0.015335\\
-0.01538	-0.01538\\
-0.0151525	-0.0151525\\
-0.01497	-0.01497\\
-0.015335	-0.015335\\
-0.01593	-0.01593\\
-0.0158375	-0.0158375\\
-0.0152425	-0.0152425\\
-0.015015	-0.015015\\
-0.01497	-0.01497\\
-0.0151975	-0.0151975\\
-0.01538	-0.01538\\
-0.0151525	-0.0151525\\
-0.01538	-0.01538\\
-0.015565	-0.015565\\
-0.0151525	-0.0151525\\
-0.0148775	-0.0148775\\
-0.0151525	-0.0151525\\
-0.0157025	-0.0157025\\
-0.0157925	-0.0157925\\
-0.0158375	-0.0158375\\
-0.01529	-0.01529\\
-0.01497	-0.01497\\
-0.014465	-0.014465\\
-0.01387	-0.01387\\
-0.0133675	-0.0133675\\
-0.0130475	-0.0130475\\
-0.0136425	-0.0136425\\
-0.0142375	-0.0142375\\
-0.0146025	-0.0146025\\
-0.01442	-0.01442\\
-0.014375	-0.014375\\
-0.0145575	-0.0145575\\
-0.0141	-0.0141\\
-0.0140075	-0.0140075\\
-0.0137325	-0.0137325\\
-0.013825	-0.013825\\
-0.014375	-0.014375\\
-0.0148325	-0.0148325\\
-0.01451	-0.01451\\
-0.014145	-0.014145\\
-0.01387	-0.01387\\
-0.013915	-0.013915\\
-0.0136425	-0.0136425\\
-0.0142825	-0.0142825\\
-0.0151975	-0.0151975\\
-0.0152425	-0.0152425\\
-0.0155175	-0.0155175\\
-0.015885	-0.015885\\
-0.0160225	-0.0160225\\
-0.0157475	-0.0157475\\
-0.015335	-0.015335\\
-0.0151975	-0.0151975\\
-0.015335	-0.015335\\
-0.0155175	-0.0155175\\
-0.01561	-0.01561\\
-0.0160225	-0.0160225\\
-0.0163425	-0.0163425\\
-0.0167075	-0.0167075\\
-0.0164325	-0.0164325\\
-0.0163425	-0.0163425\\
-0.016525	-0.016525\\
-0.016205	-0.016205\\
-0.0162975	-0.0162975\\
-0.016205	-0.016205\\
-0.015565	-0.015565\\
-0.0148775	-0.0148775\\
-0.0146025	-0.0146025\\
-0.0149225	-0.0149225\\
-0.0148325	-0.0148325\\
-0.01442	-0.01442\\
-0.01451	-0.01451\\
-0.0143275	-0.0143275\\
-0.0139625	-0.0139625\\
-0.0140525	-0.0140525\\
-0.013915	-0.013915\\
-0.0142825	-0.0142825\\
-0.01474	-0.01474\\
-0.0146025	-0.0146025\\
-0.0151525	-0.0151525\\
-0.0158375	-0.0158375\\
-0.015975	-0.015975\\
-0.0162975	-0.0162975\\
-0.0160225	-0.0160225\\
-0.015975	-0.015975\\
-0.015565	-0.015565\\
-0.0149225	-0.0149225\\
-0.014785	-0.014785\\
-0.014375	-0.014375\\
-0.0145575	-0.0145575\\
-0.0141	-0.0141\\
-0.0134575	-0.0134575\\
-0.0134125	-0.0134125\\
-0.0141	-0.0141\\
-0.0146025	-0.0146025\\
-0.01497	-0.01497\\
-0.01506	-0.01506\\
-0.0152425	-0.0152425\\
-0.0151525	-0.0151525\\
-0.0148775	-0.0148775\\
-0.0148325	-0.0148325\\
-0.014695	-0.014695\\
-0.015015	-0.015015\\
-0.01474	-0.01474\\
-0.014465	-0.014465\\
-0.014695	-0.014695\\
-0.0151975	-0.0151975\\
-0.01506	-0.01506\\
-0.01474	-0.01474\\
-0.01451	-0.01451\\
-0.0145575	-0.0145575\\
-0.0142375	-0.0142375\\
-0.0140075	-0.0140075\\
-0.0142375	-0.0142375\\
-0.01451	-0.01451\\
-0.014695	-0.014695\\
-0.01451	-0.01451\\
-0.0146475	-0.0146475\\
-0.014695	-0.014695\\
-0.014145	-0.014145\\
-0.0140525	-0.0140525\\
-0.0146475	-0.0146475\\
-0.01529	-0.01529\\
-0.0154275	-0.0154275\\
-0.01529	-0.01529\\
-0.0151975	-0.0151975\\
-0.015015	-0.015015\\
-0.0148325	-0.0148325\\
-0.0146475	-0.0146475\\
-0.0148325	-0.0148325\\
-0.0148775	-0.0148775\\
-0.01442	-0.01442\\
-0.0146025	-0.0146025\\
-0.0148775	-0.0148775\\
-0.015015	-0.015015\\
-0.0151525	-0.0151525\\
-0.0152425	-0.0152425\\
-0.015655	-0.015655\\
-0.0161125	-0.0161125\\
-0.01648	-0.01648\\
-0.01616	-0.01616\\
-0.01529	-0.01529\\
-0.0146475	-0.0146475\\
-0.0139625	-0.0139625\\
-0.0140525	-0.0140525\\
-0.0145575	-0.0145575\\
-0.0146475	-0.0146475\\
-0.01451	-0.01451\\
-0.01474	-0.01474\\
-0.01497	-0.01497\\
-0.0151975	-0.0151975\\
-0.01538	-0.01538\\
-0.015975	-0.015975\\
-0.0160675	-0.0160675\\
-0.0157475	-0.0157475\\
-0.01529	-0.01529\\
-0.015105	-0.015105\\
-0.0151525	-0.0151525\\
-0.015335	-0.015335\\
-0.01506	-0.01506\\
-0.0149225	-0.0149225\\
-0.0142825	-0.0142825\\
-0.01442	-0.01442\\
-0.015015	-0.015015\\
-0.0148325	-0.0148325\\
-0.015565	-0.015565\\
-0.0154275	-0.0154275\\
-0.0157475	-0.0157475\\
-0.0158375	-0.0158375\\
-0.0154725	-0.0154725\\
-0.0148325	-0.0148325\\
-0.0146475	-0.0146475\\
-0.015105	-0.015105\\
-0.0158375	-0.0158375\\
-0.01616	-0.01616\\
-0.016205	-0.016205\\
-0.015975	-0.015975\\
-0.0154275	-0.0154275\\
-0.015105	-0.015105\\
-0.014695	-0.014695\\
-0.01451	-0.01451\\
-0.0142825	-0.0142825\\
-0.01442	-0.01442\\
-0.0146025	-0.0146025\\
-0.01419	-0.01419\\
-0.0143275	-0.0143275\\
-0.01474	-0.01474\\
-0.015015	-0.015015\\
-0.015335	-0.015335\\
-0.0157925	-0.0157925\\
-0.01625	-0.01625\\
-0.0160675	-0.0160675\\
-0.0157925	-0.0157925\\
-0.015655	-0.015655\\
-0.0151975	-0.0151975\\
-0.0157475	-0.0157475\\
-0.0161125	-0.0161125\\
-0.015655	-0.015655\\
-0.0148775	-0.0148775\\
-0.01419	-0.01419\\
-0.0134575	-0.0134575\\
-0.0131375	-0.0131375\\
-0.01355	-0.01355\\
-0.0137775	-0.0137775\\
-0.014145	-0.014145\\
-0.013825	-0.013825\\
-0.0137775	-0.0137775\\
-0.0137325	-0.0137325\\
-0.0136875	-0.0136875\\
-0.0141	-0.0141\\
-0.01474	-0.01474\\
-0.015015	-0.015015\\
-0.01529	-0.01529\\
-0.0157025	-0.0157025\\
-0.0158375	-0.0158375\\
-0.01616	-0.01616\\
-0.016205	-0.016205\\
-0.015885	-0.015885\\
-0.0154275	-0.0154275\\
-0.0152425	-0.0152425\\
-0.0158375	-0.0158375\\
-0.0161125	-0.0161125\\
-0.0157475	-0.0157475\\
-0.0152425	-0.0152425\\
-0.0143275	-0.0143275\\
-0.0136425	-0.0136425\\
-0.0134575	-0.0134575\\
-0.012955	-0.012955\\
-0.0134125	-0.0134125\\
-0.01387	-0.01387\\
-0.0139625	-0.0139625\\
-0.0134575	-0.0134575\\
-0.0131375	-0.0131375\\
-0.01323	-0.01323\\
-0.0133675	-0.0133675\\
-0.0134125	-0.0134125\\
-0.013275	-0.013275\\
-0.0130925	-0.0130925\\
-0.0134125	-0.0134125\\
-0.0148325	-0.0148325\\
-0.0154275	-0.0154275\\
-0.0157025	-0.0157025\\
-0.0154725	-0.0154725\\
-0.0157025	-0.0157025\\
-0.0164325	-0.0164325\\
-0.01657	-0.01657\\
-0.016525	-0.016525\\
-0.016205	-0.016205\\
-0.0161125	-0.0161125\\
-0.01616	-0.01616\\
-0.0157475	-0.0157475\\
-0.0154725	-0.0154725\\
-0.01497	-0.01497\\
-0.01451	-0.01451\\
-0.015105	-0.015105\\
-0.0149225	-0.0149225\\
-0.0148325	-0.0148325\\
-0.01497	-0.01497\\
-0.0149225	-0.0149225\\
-0.01506	-0.01506\\
-0.0151975	-0.0151975\\
-0.01506	-0.01506\\
-0.014785	-0.014785\\
-0.0148775	-0.0148775\\
-0.0142825	-0.0142825\\
-0.01332	-0.01332\\
-0.0131825	-0.0131825\\
-0.0136875	-0.0136875\\
-0.01323	-0.01323\\
-0.01245	-0.01245\\
-0.012635	-0.012635\\
-0.01323	-0.01323\\
-0.013595	-0.013595\\
-0.0139625	-0.0139625\\
-0.01451	-0.01451\\
-0.0146025	-0.0146025\\
-0.0142375	-0.0142375\\
-0.01451	-0.01451\\
-0.0140525	-0.0140525\\
-0.013595	-0.013595\\
-0.0131375	-0.0131375\\
-0.01268	-0.01268\\
-0.013	-0.013\\
-0.012955	-0.012955\\
-0.0134125	-0.0134125\\
-0.0140525	-0.0140525\\
-0.01419	-0.01419\\
-0.0139625	-0.0139625\\
-0.013915	-0.013915\\
-0.0137775	-0.0137775\\
-0.013915	-0.013915\\
-0.0137325	-0.0137325\\
-0.013595	-0.013595\\
-0.013505	-0.013505\\
-0.01323	-0.01323\\
-0.0133675	-0.0133675\\
-0.013275	-0.013275\\
-0.0136875	-0.0136875\\
-0.013825	-0.013825\\
-0.0136425	-0.0136425\\
-0.013595	-0.013595\\
-0.013275	-0.013275\\
-0.01419	-0.01419\\
-0.0148775	-0.0148775\\
-0.014785	-0.014785\\
-0.0146025	-0.0146025\\
-0.0142375	-0.0142375\\
-0.0146475	-0.0146475\\
-0.01474	-0.01474\\
-0.0142825	-0.0142825\\
-0.0143275	-0.0143275\\
-0.014145	-0.014145\\
-0.014375	-0.014375\\
-0.0139625	-0.0139625\\
-0.0137325	-0.0137325\\
-0.013915	-0.013915\\
-0.01419	-0.01419\\
-0.0140075	-0.0140075\\
-0.013915	-0.013915\\
-0.0140075	-0.0140075\\
-0.014465	-0.014465\\
-0.0149225	-0.0149225\\
-0.015655	-0.015655\\
-0.01561	-0.01561\\
-0.015015	-0.015015\\
-0.0146475	-0.0146475\\
-0.0149225	-0.0149225\\
-0.01529	-0.01529\\
-0.0151525	-0.0151525\\
-0.0152425	-0.0152425\\
-0.014785	-0.014785\\
-0.0140075	-0.0140075\\
-0.0136875	-0.0136875\\
-0.014375	-0.014375\\
-0.0146025	-0.0146025\\
-0.014465	-0.014465\\
-0.0148325	-0.0148325\\
-0.015015	-0.015015\\
-0.0148325	-0.0148325\\
-0.01451	-0.01451\\
-0.0145575	-0.0145575\\
-0.0149225	-0.0149225\\
-0.014785	-0.014785\\
-0.01474	-0.01474\\
-0.014465	-0.014465\\
-0.0142825	-0.0142825\\
-0.0142375	-0.0142375\\
-0.014695	-0.014695\\
-0.015015	-0.015015\\
-0.015105	-0.015105\\
-0.01506	-0.01506\\
-0.01474	-0.01474\\
-0.0143275	-0.0143275\\
-0.014465	-0.014465\\
-0.014695	-0.014695\\
-0.015015	-0.015015\\
-0.015335	-0.015335\\
-0.0151525	-0.0151525\\
-0.0146025	-0.0146025\\
-0.015015	-0.015015\\
-0.015565	-0.015565\\
-0.015335	-0.015335\\
-0.0152425	-0.0152425\\
-0.0154275	-0.0154275\\
-0.0151975	-0.0151975\\
-0.0149225	-0.0149225\\
-0.015015	-0.015015\\
-0.0149225	-0.0149225\\
-0.015015	-0.015015\\
-0.0152425	-0.0152425\\
-0.01506	-0.01506\\
-0.015105	-0.015105\\
-0.0151525	-0.0151525\\
-0.015105	-0.015105\\
-0.0149225	-0.0149225\\
-0.01506	-0.01506\\
-0.0148775	-0.0148775\\
-0.014785	-0.014785\\
-0.0148775	-0.0148775\\
-0.0142375	-0.0142375\\
-0.013505	-0.013505\\
-0.012955	-0.012955\\
-0.0131375	-0.0131375\\
-0.01419	-0.01419\\
-0.0149225	-0.0149225\\
-0.015015	-0.015015\\
-0.0146475	-0.0146475\\
-0.0146025	-0.0146025\\
-0.0145575	-0.0145575\\
-0.01442	-0.01442\\
-0.0139625	-0.0139625\\
-0.0140525	-0.0140525\\
-0.0141	-0.0141\\
-0.0142375	-0.0142375\\
-0.01419	-0.01419\\
-0.0140525	-0.0140525\\
-0.0136875	-0.0136875\\
-0.013825	-0.013825\\
-0.01451	-0.01451\\
-0.014375	-0.014375\\
-0.014465	-0.014465\\
-0.01474	-0.01474\\
-0.014785	-0.014785\\
-0.0151975	-0.0151975\\
-0.0151525	-0.0151525\\
-0.015105	-0.015105\\
-0.0151525	-0.0151525\\
-0.0146025	-0.0146025\\
-0.01497	-0.01497\\
-0.014785	-0.014785\\
-0.0148775	-0.0148775\\
-0.015105	-0.015105\\
-0.01497	-0.01497\\
-0.01506	-0.01506\\
-0.0154275	-0.0154275\\
-0.01538	-0.01538\\
-0.0151525	-0.0151525\\
-0.0149225	-0.0149225\\
-0.015015	-0.015015\\
-0.0148775	-0.0148775\\
-0.014695	-0.014695\\
-0.01451	-0.01451\\
-0.01442	-0.01442\\
-0.01506	-0.01506\\
-0.015655	-0.015655\\
-0.01561	-0.01561\\
-0.015565	-0.015565\\
-0.0155175	-0.0155175\\
-0.01497	-0.01497\\
-0.0146475	-0.0146475\\
-0.01442	-0.01442\\
-0.014145	-0.014145\\
-0.0145575	-0.0145575\\
-0.0149225	-0.0149225\\
-0.0151525	-0.0151525\\
-0.015105	-0.015105\\
-0.01474	-0.01474\\
-0.0151525	-0.0151525\\
-0.015565	-0.015565\\
-0.0157025	-0.0157025\\
-0.01561	-0.01561\\
-0.0161125	-0.0161125\\
-0.0160675	-0.0160675\\
-0.015335	-0.015335\\
-0.01474	-0.01474\\
-0.01442	-0.01442\\
-0.0148325	-0.0148325\\
-0.01529	-0.01529\\
-0.015565	-0.015565\\
-0.0154725	-0.0154725\\
-0.0151525	-0.0151525\\
-0.01451	-0.01451\\
-0.0142375	-0.0142375\\
-0.014375	-0.014375\\
-0.0141	-0.0141\\
-0.0139625	-0.0139625\\
-0.01355	-0.01355\\
-0.0139625	-0.0139625\\
-0.01419	-0.01419\\
-0.014465	-0.014465\\
-0.01451	-0.01451\\
-0.0145575	-0.0145575\\
-0.0148325	-0.0148325\\
-0.0146025	-0.0146025\\
-0.0148775	-0.0148775\\
-0.0152425	-0.0152425\\
-0.01497	-0.01497\\
-0.0149225	-0.0149225\\
-0.0148775	-0.0148775\\
-0.01529	-0.01529\\
-0.0151525	-0.0151525\\
-0.0149225	-0.0149225\\
-0.01497	-0.01497\\
-0.0148775	-0.0148775\\
-0.01451	-0.01451\\
-0.014695	-0.014695\\
-0.0151975	-0.0151975\\
-0.0152425	-0.0152425\\
-0.0154275	-0.0154275\\
-0.0160675	-0.0160675\\
-0.0158375	-0.0158375\\
-0.015565	-0.015565\\
-0.0152425	-0.0152425\\
-0.015335	-0.015335\\
-0.0158375	-0.0158375\\
-0.01561	-0.01561\\
-0.015335	-0.015335\\
-0.015655	-0.015655\\
-0.01538	-0.01538\\
-0.014695	-0.014695\\
-0.0146025	-0.0146025\\
-0.01442	-0.01442\\
-0.0140525	-0.0140525\\
-0.0139625	-0.0139625\\
-0.013825	-0.013825\\
-0.0137325	-0.0137325\\
-0.01387	-0.01387\\
-0.01419	-0.01419\\
-0.01442	-0.01442\\
-0.0148325	-0.0148325\\
-0.01497	-0.01497\\
-0.015105	-0.015105\\
-0.0154275	-0.0154275\\
-0.015105	-0.015105\\
-0.015015	-0.015015\\
-0.01497	-0.01497\\
-0.015015	-0.015015\\
-0.015335	-0.015335\\
-0.015105	-0.015105\\
-0.015015	-0.015015\\
-0.0154275	-0.0154275\\
-0.01538	-0.01538\\
-0.01529	-0.01529\\
-0.0148325	-0.0148325\\
-0.0142375	-0.0142375\\
-0.01387	-0.01387\\
-0.01419	-0.01419\\
-0.0142375	-0.0142375\\
-0.014465	-0.014465\\
-0.014375	-0.014375\\
-0.01451	-0.01451\\
-0.01529	-0.01529\\
-0.0158375	-0.0158375\\
-0.0157025	-0.0157025\\
-0.015655	-0.015655\\
-0.01561	-0.01561\\
-0.0152425	-0.0152425\\
-0.0151525	-0.0151525\\
-0.01529	-0.01529\\
-0.0151525	-0.0151525\\
-0.0154275	-0.0154275\\
-0.015975	-0.015975\\
-0.0161125	-0.0161125\\
-0.015975	-0.015975\\
-0.0160675	-0.0160675\\
-0.0157475	-0.0157475\\
-0.015655	-0.015655\\
-0.0155175	-0.0155175\\
-0.01538	-0.01538\\
-0.015655	-0.015655\\
-0.0158375	-0.0158375\\
-0.015015	-0.015015\\
-0.01474	-0.01474\\
-0.0146475	-0.0146475\\
-0.0148775	-0.0148775\\
-0.01497	-0.01497\\
-0.0146475	-0.0146475\\
-0.015335	-0.015335\\
-0.016205	-0.016205\\
-0.01648	-0.01648\\
-0.0163425	-0.0163425\\
-0.016205	-0.016205\\
-0.0162975	-0.0162975\\
-0.016525	-0.016525\\
-0.0162975	-0.0162975\\
-0.016205	-0.016205\\
-0.01625	-0.01625\\
-0.015885	-0.015885\\
-0.015565	-0.015565\\
-0.0157475	-0.0157475\\
-0.0154725	-0.0154725\\
-0.01497	-0.01497\\
-0.01506	-0.01506\\
-0.0148775	-0.0148775\\
-0.0148325	-0.0148325\\
-0.01497	-0.01497\\
-0.015335	-0.015335\\
-0.01529	-0.01529\\
-0.0149225	-0.0149225\\
-0.0151525	-0.0151525\\
-0.01529	-0.01529\\
-0.0152425	-0.0152425\\
-0.01506	-0.01506\\
-0.01538	-0.01538\\
-0.0154275	-0.0154275\\
-0.0151525	-0.0151525\\
-0.0154725	-0.0154725\\
-0.015655	-0.015655\\
-0.01529	-0.01529\\
-0.0148775	-0.0148775\\
-0.014375	-0.014375\\
-0.013915	-0.013915\\
-0.014375	-0.014375\\
-0.0146025	-0.0146025\\
-0.0142375	-0.0142375\\
-0.01387	-0.01387\\
-0.0143275	-0.0143275\\
-0.0146025	-0.0146025\\
-0.015015	-0.015015\\
-0.015335	-0.015335\\
-0.01538	-0.01538\\
-0.0154275	-0.0154275\\
-0.01506	-0.01506\\
-0.0141	-0.0141\\
-0.0139625	-0.0139625\\
-0.0141	-0.0141\\
-0.0146475	-0.0146475\\
-0.0149225	-0.0149225\\
-0.01538	-0.01538\\
-0.0151975	-0.0151975\\
-0.015015	-0.015015\\
-0.0154725	-0.0154725\\
-0.01529	-0.01529\\
-0.0148775	-0.0148775\\
-0.014465	-0.014465\\
-0.01451	-0.01451\\
-0.0148775	-0.0148775\\
-0.0154275	-0.0154275\\
-0.01529	-0.01529\\
-0.015015	-0.015015\\
-0.0148775	-0.0148775\\
-0.015015	-0.015015\\
-0.015335	-0.015335\\
-0.015975	-0.015975\\
-0.0160675	-0.0160675\\
-0.0160225	-0.0160225\\
-0.015565	-0.015565\\
-0.015015	-0.015015\\
-0.01497	-0.01497\\
-0.0142825	-0.0142825\\
-0.0142375	-0.0142375\\
-0.0148775	-0.0148775\\
-0.0154275	-0.0154275\\
-0.0157925	-0.0157925\\
-0.0162975	-0.0162975\\
-0.01625	-0.01625\\
-0.0154725	-0.0154725\\
-0.0151525	-0.0151525\\
-0.01497	-0.01497\\
-0.0151525	-0.0151525\\
-0.0154725	-0.0154725\\
-0.015975	-0.015975\\
-0.015885	-0.015885\\
-0.01538	-0.01538\\
-0.0157025	-0.0157025\\
-0.0154275	-0.0154275\\
-0.01497	-0.01497\\
-0.0146025	-0.0146025\\
-0.014145	-0.014145\\
-0.013825	-0.013825\\
-0.01355	-0.01355\\
-0.0137325	-0.0137325\\
-0.0142825	-0.0142825\\
-0.01442	-0.01442\\
-0.0142825	-0.0142825\\
-0.0141	-0.0141\\
-0.013505	-0.013505\\
-0.012635	-0.012635\\
-0.0125425	-0.0125425\\
-0.012955	-0.012955\\
-0.0137325	-0.0137325\\
-0.0141	-0.0141\\
-0.01442	-0.01442\\
-0.01474	-0.01474\\
-0.0152425	-0.0152425\\
-0.01593	-0.01593\\
-0.016205	-0.016205\\
-0.0163425	-0.0163425\\
-0.015655	-0.015655\\
-0.0152425	-0.0152425\\
-0.0155175	-0.0155175\\
-0.0154275	-0.0154275\\
-0.01506	-0.01506\\
-0.014695	-0.014695\\
-0.014375	-0.014375\\
-0.0146475	-0.0146475\\
-0.0148325	-0.0148325\\
-0.014145	-0.014145\\
-0.0136875	-0.0136875\\
-0.013915	-0.013915\\
-0.0143275	-0.0143275\\
-0.01419	-0.01419\\
-0.014375	-0.014375\\
-0.0143275	-0.0143275\\
-0.0146025	-0.0146025\\
-0.0152425	-0.0152425\\
-0.0154275	-0.0154275\\
-0.0160225	-0.0160225\\
-0.0162975	-0.0162975\\
-0.016205	-0.016205\\
-0.0157025	-0.0157025\\
-0.0152425	-0.0152425\\
-0.01451	-0.01451\\
-0.0146025	-0.0146025\\
-0.014465	-0.014465\\
-0.0148325	-0.0148325\\
-0.01497	-0.01497\\
-0.014695	-0.014695\\
-0.0148325	-0.0148325\\
-0.014375	-0.014375\\
-0.01442	-0.01442\\
-0.01419	-0.01419\\
-0.0137325	-0.0137325\\
-0.01355	-0.01355\\
-0.013505	-0.013505\\
-0.0134575	-0.0134575\\
-0.0131375	-0.0131375\\
-0.0134125	-0.0134125\\
-0.013505	-0.013505\\
-0.0134575	-0.0134575\\
-0.0139625	-0.0139625\\
-0.014465	-0.014465\\
-0.01442	-0.01442\\
-0.0143275	-0.0143275\\
-0.01474	-0.01474\\
-0.0152425	-0.0152425\\
-0.0151525	-0.0151525\\
-0.014375	-0.014375\\
-0.0136875	-0.0136875\\
-0.0140075	-0.0140075\\
-0.014695	-0.014695\\
-0.0148325	-0.0148325\\
-0.0154275	-0.0154275\\
-0.015565	-0.015565\\
-0.0151525	-0.0151525\\
-0.014695	-0.014695\\
-0.0146025	-0.0146025\\
-0.0146475	-0.0146475\\
-0.01451	-0.01451\\
-0.0139625	-0.0139625\\
-0.013825	-0.013825\\
-0.0137325	-0.0137325\\
-0.013505	-0.013505\\
-0.0131375	-0.0131375\\
-0.0134125	-0.0134125\\
-0.0139625	-0.0139625\\
-0.0141	-0.0141\\
-0.0139625	-0.0139625\\
-0.014375	-0.014375\\
-0.01506	-0.01506\\
-0.0148775	-0.0148775\\
-0.01497	-0.01497\\
-0.0151975	-0.0151975\\
-0.01497	-0.01497\\
-0.0143275	-0.0143275\\
-0.0142375	-0.0142375\\
-0.014465	-0.014465\\
-0.0140075	-0.0140075\\
-0.013915	-0.013915\\
-0.0137775	-0.0137775\\
-0.01332	-0.01332\\
-0.0130475	-0.0130475\\
-0.01332	-0.01332\\
-0.0136425	-0.0136425\\
-0.0136875	-0.0136875\\
-0.013825	-0.013825\\
-0.0141	-0.0141\\
-0.01387	-0.01387\\
-0.01332	-0.01332\\
-0.013	-0.013\\
-0.0131825	-0.0131825\\
-0.01332	-0.01332\\
-0.0140525	-0.0140525\\
-0.014465	-0.014465\\
-0.014375	-0.014375\\
-0.0143275	-0.0143275\\
-0.0146025	-0.0146025\\
-0.015105	-0.015105\\
-0.01538	-0.01538\\
-0.0155175	-0.0155175\\
-0.01529	-0.01529\\
-0.0152425	-0.0152425\\
-0.01529	-0.01529\\
-0.0154275	-0.0154275\\
-0.015885	-0.015885\\
-0.01593	-0.01593\\
-0.0157925	-0.0157925\\
-0.015565	-0.015565\\
-0.01538	-0.01538\\
-0.015335	-0.015335\\
-0.0154725	-0.0154725\\
-0.01497	-0.01497\\
-0.0148775	-0.0148775\\
-0.015105	-0.015105\\
-0.01538	-0.01538\\
-0.0152425	-0.0152425\\
-0.015335	-0.015335\\
-0.0151975	-0.0151975\\
-0.015015	-0.015015\\
-0.0145575	-0.0145575\\
-0.014695	-0.014695\\
-0.01529	-0.01529\\
-0.014695	-0.014695\\
-0.0146475	-0.0146475\\
-0.0142825	-0.0142825\\
-0.013825	-0.013825\\
-0.0140075	-0.0140075\\
-0.0136875	-0.0136875\\
-0.0134575	-0.0134575\\
-0.01355	-0.01355\\
-0.01332	-0.01332\\
-0.013505	-0.013505\\
-0.0134575	-0.0134575\\
-0.0130475	-0.0130475\\
-0.013275	-0.013275\\
-0.0134125	-0.0134125\\
-0.01355	-0.01355\\
-0.0136425	-0.0136425\\
-0.0136875	-0.0136875\\
-0.0140075	-0.0140075\\
-0.0140525	-0.0140525\\
-0.014145	-0.014145\\
-0.0149225	-0.0149225\\
-0.014695	-0.014695\\
-0.01474	-0.01474\\
-0.014465	-0.014465\\
-0.01387	-0.01387\\
-0.014145	-0.014145\\
-0.0136875	-0.0136875\\
-0.013915	-0.013915\\
-0.0140525	-0.0140525\\
-0.0142375	-0.0142375\\
-0.0137775	-0.0137775\\
-0.0133675	-0.0133675\\
-0.013	-0.013\\
-0.012955	-0.012955\\
-0.0133675	-0.0133675\\
-0.0134575	-0.0134575\\
-0.013595	-0.013595\\
-0.0140525	-0.0140525\\
-0.0146025	-0.0146025\\
-0.01451	-0.01451\\
-0.014695	-0.014695\\
-0.015105	-0.015105\\
-0.01506	-0.01506\\
-0.01497	-0.01497\\
-0.0155175	-0.0155175\\
-0.0154725	-0.0154725\\
-0.01561	-0.01561\\
-0.015655	-0.015655\\
-0.015885	-0.015885\\
-0.0154275	-0.0154275\\
-0.01474	-0.01474\\
-0.0143275	-0.0143275\\
-0.0146475	-0.0146475\\
-0.0148775	-0.0148775\\
-0.015015	-0.015015\\
-0.014785	-0.014785\\
-0.0140075	-0.0140075\\
-0.013915	-0.013915\\
-0.0146025	-0.0146025\\
-0.0151975	-0.0151975\\
-0.015335	-0.015335\\
-0.0148325	-0.0148325\\
-0.014465	-0.014465\\
-0.0146025	-0.0146025\\
-0.0151525	-0.0151525\\
-0.0154275	-0.0154275\\
-0.015565	-0.015565\\
-0.01529	-0.01529\\
-0.0155175	-0.0155175\\
-0.0158375	-0.0158375\\
-0.0157925	-0.0157925\\
-0.016205	-0.016205\\
-0.0162975	-0.0162975\\
-0.0160675	-0.0160675\\
-0.01616	-0.01616\\
-0.0160675	-0.0160675\\
-0.01529	-0.01529\\
-0.01497	-0.01497\\
-0.015105	-0.015105\\
-0.015335	-0.015335\\
-0.0154725	-0.0154725\\
-0.0151975	-0.0151975\\
-0.01529	-0.01529\\
-0.01497	-0.01497\\
-0.015105	-0.015105\\
-0.0151525	-0.0151525\\
-0.0154275	-0.0154275\\
-0.015655	-0.015655\\
-0.015335	-0.015335\\
-0.015015	-0.015015\\
-0.014375	-0.014375\\
-0.01451	-0.01451\\
-0.014375	-0.014375\\
-0.01419	-0.01419\\
-0.0140525	-0.0140525\\
-0.01442	-0.01442\\
-0.014785	-0.014785\\
-0.01497	-0.01497\\
-0.01506	-0.01506\\
-0.0145575	-0.0145575\\
-0.01419	-0.01419\\
-0.013825	-0.013825\\
-0.0137325	-0.0137325\\
-0.01419	-0.01419\\
-0.0143275	-0.0143275\\
-0.0148325	-0.0148325\\
-0.0152425	-0.0152425\\
-0.0154725	-0.0154725\\
-0.01561	-0.01561\\
-0.015885	-0.015885\\
-0.0154275	-0.0154275\\
-0.01506	-0.01506\\
-0.014785	-0.014785\\
-0.014695	-0.014695\\
-0.0149225	-0.0149225\\
-0.0148325	-0.0148325\\
-0.01451	-0.01451\\
-0.0142375	-0.0142375\\
-0.014785	-0.014785\\
-0.015105	-0.015105\\
-0.01506	-0.01506\\
-0.015565	-0.015565\\
-0.01593	-0.01593\\
-0.01561	-0.01561\\
-0.0151525	-0.0151525\\
-0.0146025	-0.0146025\\
-0.014375	-0.014375\\
-0.0148775	-0.0148775\\
-0.01497	-0.01497\\
-0.015105	-0.015105\\
-0.0152425	-0.0152425\\
-0.015335	-0.015335\\
-0.01506	-0.01506\\
-0.0148325	-0.0148325\\
-0.0146025	-0.0146025\\
-0.0148775	-0.0148775\\
-0.015335	-0.015335\\
-0.01561	-0.01561\\
-0.0151525	-0.0151525\\
-0.0157475	-0.0157475\\
-0.0160675	-0.0160675\\
-0.015565	-0.015565\\
-0.0148325	-0.0148325\\
-0.01497	-0.01497\\
-0.015015	-0.015015\\
-0.0148325	-0.0148325\\
-0.01497	-0.01497\\
-0.014695	-0.014695\\
-0.014785	-0.014785\\
-0.015565	-0.015565\\
-0.015885	-0.015885\\
-0.0154725	-0.0154725\\
-0.015565	-0.015565\\
-0.015655	-0.015655\\
-0.01538	-0.01538\\
-0.01529	-0.01529\\
-0.014785	-0.014785\\
-0.0149225	-0.0149225\\
-0.015105	-0.015105\\
-0.015655	-0.015655\\
-0.015975	-0.015975\\
-0.01648	-0.01648\\
-0.0163875	-0.0163875\\
-0.0160225	-0.0160225\\
-0.0157925	-0.0157925\\
-0.0157025	-0.0157025\\
-0.0155175	-0.0155175\\
-0.01497	-0.01497\\
-0.01474	-0.01474\\
-0.01442	-0.01442\\
-0.0139625	-0.0139625\\
-0.0141	-0.0141\\
-0.014465	-0.014465\\
-0.0142825	-0.0142825\\
-0.01387	-0.01387\\
-0.0137325	-0.0137325\\
-0.01451	-0.01451\\
-0.015105	-0.015105\\
-0.01451	-0.01451\\
-0.0145575	-0.0145575\\
-0.014695	-0.014695\\
-0.0146475	-0.0146475\\
-0.0142825	-0.0142825\\
-0.013915	-0.013915\\
-0.0136425	-0.0136425\\
-0.0136875	-0.0136875\\
-0.0143275	-0.0143275\\
-0.014695	-0.014695\\
-0.01451	-0.01451\\
-0.014145	-0.014145\\
-0.0136425	-0.0136425\\
-0.014145	-0.014145\\
-0.0145575	-0.0145575\\
-0.014695	-0.014695\\
-0.014465	-0.014465\\
-0.0142825	-0.0142825\\
-0.01442	-0.01442\\
-0.0146475	-0.0146475\\
-0.0152425	-0.0152425\\
-0.01561	-0.01561\\
-0.0154725	-0.0154725\\
-0.0148775	-0.0148775\\
-0.01474	-0.01474\\
-0.014695	-0.014695\\
-0.01474	-0.01474\\
-0.014375	-0.014375\\
-0.014465	-0.014465\\
-0.014145	-0.014145\\
-0.013595	-0.013595\\
-0.0137325	-0.0137325\\
-0.0141	-0.0141\\
-0.014695	-0.014695\\
-0.01497	-0.01497\\
-0.01529	-0.01529\\
-0.0154275	-0.0154275\\
-0.0158375	-0.0158375\\
-0.01648	-0.01648\\
-0.0164325	-0.0164325\\
-0.015975	-0.015975\\
-0.01593	-0.01593\\
-0.0161125	-0.0161125\\
-0.01648	-0.01648\\
-0.0161125	-0.0161125\\
-0.0151525	-0.0151525\\
-0.01442	-0.01442\\
-0.0142375	-0.0142375\\
-0.01387	-0.01387\\
-0.013275	-0.013275\\
-0.0130925	-0.0130925\\
-0.0134575	-0.0134575\\
-0.013825	-0.013825\\
-0.0140525	-0.0140525\\
-0.01387	-0.01387\\
-0.0137325	-0.0137325\\
-0.013915	-0.013915\\
-0.01419	-0.01419\\
-0.0142825	-0.0142825\\
-0.0140525	-0.0140525\\
-0.0136425	-0.0136425\\
-0.0134575	-0.0134575\\
-0.0137325	-0.0137325\\
-0.0140075	-0.0140075\\
-0.01387	-0.01387\\
-0.0134575	-0.0134575\\
-0.0137775	-0.0137775\\
-0.013595	-0.013595\\
-0.013825	-0.013825\\
-0.01419	-0.01419\\
-0.014465	-0.014465\\
-0.01419	-0.01419\\
-0.014465	-0.014465\\
-0.014695	-0.014695\\
-0.01451	-0.01451\\
-0.0143275	-0.0143275\\
-0.014695	-0.014695\\
-0.0151525	-0.0151525\\
-0.0152425	-0.0152425\\
-0.01538	-0.01538\\
-0.0152425	-0.0152425\\
-0.015105	-0.015105\\
-0.01506	-0.01506\\
-0.01538	-0.01538\\
-0.0151975	-0.0151975\\
-0.0154275	-0.0154275\\
-0.0155175	-0.0155175\\
-0.0154725	-0.0154725\\
-0.0163425	-0.0163425\\
-0.015565	-0.015565\\
-0.0148775	-0.0148775\\
-0.014695	-0.014695\\
-0.0148775	-0.0148775\\
-0.0151525	-0.0151525\\
-0.0154275	-0.0154275\\
-0.01561	-0.01561\\
-0.0157025	-0.0157025\\
-0.015335	-0.015335\\
-0.015105	-0.015105\\
-0.0151975	-0.0151975\\
-0.014785	-0.014785\\
-0.0142825	-0.0142825\\
-0.01451	-0.01451\\
-0.0146025	-0.0146025\\
-0.0148775	-0.0148775\\
-0.01529	-0.01529\\
-0.0149225	-0.0149225\\
-0.014695	-0.014695\\
-0.01497	-0.01497\\
-0.015335	-0.015335\\
-0.0157475	-0.0157475\\
-0.0160675	-0.0160675\\
-0.0162975	-0.0162975\\
-0.016205	-0.016205\\
-0.0160675	-0.0160675\\
-0.0155175	-0.0155175\\
-0.0148775	-0.0148775\\
-0.0152425	-0.0152425\\
-0.0157475	-0.0157475\\
-0.0151975	-0.0151975\\
-0.0154275	-0.0154275\\
-0.015975	-0.015975\\
-0.01648	-0.01648\\
-0.01616	-0.01616\\
-0.015335	-0.015335\\
-0.0146025	-0.0146025\\
-0.0146475	-0.0146475\\
-0.0146025	-0.0146025\\
-0.01419	-0.01419\\
-0.0140525	-0.0140525\\
-0.0140075	-0.0140075\\
-0.014145	-0.014145\\
-0.0146475	-0.0146475\\
-0.01474	-0.01474\\
-0.015105	-0.015105\\
-0.0155175	-0.0155175\\
-0.01561	-0.01561\\
-0.015105	-0.015105\\
-0.0148325	-0.0148325\\
-0.015105	-0.015105\\
-0.01474	-0.01474\\
-0.0145575	-0.0145575\\
-0.0142375	-0.0142375\\
-0.01451	-0.01451\\
-0.0145575	-0.0145575\\
-0.0140525	-0.0140525\\
-0.0134125	-0.0134125\\
-0.0130925	-0.0130925\\
-0.013505	-0.013505\\
-0.0139625	-0.0139625\\
-0.014145	-0.014145\\
-0.013825	-0.013825\\
-0.0137775	-0.0137775\\
-0.01419	-0.01419\\
-0.0146025	-0.0146025\\
-0.014465	-0.014465\\
-0.0142375	-0.0142375\\
-0.01442	-0.01442\\
-0.0146025	-0.0146025\\
-0.01474	-0.01474\\
-0.014465	-0.014465\\
-0.01419	-0.01419\\
-0.0142825	-0.0142825\\
-0.01419	-0.01419\\
-0.0142825	-0.0142825\\
-0.0146475	-0.0146475\\
-0.01529	-0.01529\\
-0.0151525	-0.0151525\\
-0.015015	-0.015015\\
-0.0154725	-0.0154725\\
-0.01538	-0.01538\\
-0.0148775	-0.0148775\\
-0.014465	-0.014465\\
-0.0141	-0.0141\\
-0.0139625	-0.0139625\\
-0.013825	-0.013825\\
-0.0136875	-0.0136875\\
-0.0143275	-0.0143275\\
-0.0137775	-0.0137775\\
-0.0133675	-0.0133675\\
-0.0136425	-0.0136425\\
-0.0140075	-0.0140075\\
-0.0139625	-0.0139625\\
-0.0136425	-0.0136425\\
-0.0131825	-0.0131825\\
-0.0134125	-0.0134125\\
-0.0141	-0.0141\\
-0.014695	-0.014695\\
-0.0146025	-0.0146025\\
-0.0143275	-0.0143275\\
-0.014375	-0.014375\\
-0.01442	-0.01442\\
-0.0145575	-0.0145575\\
-0.014785	-0.014785\\
-0.015015	-0.015015\\
-0.0151525	-0.0151525\\
-0.015015	-0.015015\\
-0.014695	-0.014695\\
-0.0142375	-0.0142375\\
-0.01451	-0.01451\\
-0.01419	-0.01419\\
-0.013915	-0.013915\\
-0.014465	-0.014465\\
-0.0148325	-0.0148325\\
-0.0148775	-0.0148775\\
-0.01497	-0.01497\\
-0.0154275	-0.0154275\\
-0.01497	-0.01497\\
-0.01538	-0.01538\\
-0.01561	-0.01561\\
-0.0154725	-0.0154725\\
-0.0155175	-0.0155175\\
-0.015565	-0.015565\\
-0.0163875	-0.0163875\\
-0.0167075	-0.0167075\\
-0.01625	-0.01625\\
-0.0163425	-0.0163425\\
-0.0166175	-0.0166175\\
-0.0166625	-0.0166625\\
-0.0168	-0.0168\\
-0.016755	-0.016755\\
-0.01712	-0.01712\\
-0.0169825	-0.0169825\\
-0.01648	-0.01648\\
-0.015885	-0.015885\\
-0.015565	-0.015565\\
-0.015335	-0.015335\\
-0.01497	-0.01497\\
-0.0152425	-0.0152425\\
-0.0157025	-0.0157025\\
-0.015335	-0.015335\\
-0.015015	-0.015015\\
-0.01451	-0.01451\\
-0.014465	-0.014465\\
-0.01387	-0.01387\\
-0.0136425	-0.0136425\\
-0.01355	-0.01355\\
-0.0137325	-0.0137325\\
-0.013595	-0.013595\\
-0.01332	-0.01332\\
-0.012955	-0.012955\\
-0.0125875	-0.0125875\\
-0.012955	-0.012955\\
-0.0128175	-0.0128175\\
-0.012725	-0.012725\\
-0.01355	-0.01355\\
-0.014145	-0.014145\\
-0.014375	-0.014375\\
-0.01419	-0.01419\\
-0.014375	-0.014375\\
-0.01497	-0.01497\\
-0.01451	-0.01451\\
-0.0137775	-0.0137775\\
-0.0133675	-0.0133675\\
-0.012955	-0.012955\\
-0.0131825	-0.0131825\\
-0.013595	-0.013595\\
-0.0142375	-0.0142375\\
-0.0139625	-0.0139625\\
-0.014375	-0.014375\\
-0.01506	-0.01506\\
-0.0151975	-0.0151975\\
-0.015015	-0.015015\\
-0.014375	-0.014375\\
-0.01442	-0.01442\\
-0.014785	-0.014785\\
-0.0151975	-0.0151975\\
-0.0148775	-0.0148775\\
-0.0140525	-0.0140525\\
-0.0142375	-0.0142375\\
-0.014145	-0.014145\\
-0.0133675	-0.0133675\\
-0.01268	-0.01268\\
-0.0130925	-0.0130925\\
-0.0142825	-0.0142825\\
-0.0146475	-0.0146475\\
-0.0143275	-0.0143275\\
-0.013825	-0.013825\\
-0.0136875	-0.0136875\\
-0.012635	-0.012635\\
-0.0124975	-0.0124975\\
-0.012635	-0.012635\\
-0.012955	-0.012955\\
-0.0134575	-0.0134575\\
-0.0142375	-0.0142375\\
-0.0145575	-0.0145575\\
-0.01474	-0.01474\\
-0.0148325	-0.0148325\\
-0.0149225	-0.0149225\\
-0.0148775	-0.0148775\\
-0.014785	-0.014785\\
-0.01497	-0.01497\\
-0.015655	-0.015655\\
-0.0158375	-0.0158375\\
-0.015105	-0.015105\\
-0.014145	-0.014145\\
-0.01474	-0.01474\\
-0.0151975	-0.0151975\\
-0.015105	-0.015105\\
-0.01451	-0.01451\\
-0.0142825	-0.0142825\\
-0.014785	-0.014785\\
-0.0155175	-0.0155175\\
-0.0157925	-0.0157925\\
-0.015975	-0.015975\\
-0.0161125	-0.0161125\\
-0.015885	-0.015885\\
-0.0157925	-0.0157925\\
-0.01497	-0.01497\\
-0.01442	-0.01442\\
-0.0146475	-0.0146475\\
-0.0148775	-0.0148775\\
-0.01497	-0.01497\\
-0.01506	-0.01506\\
-0.01474	-0.01474\\
-0.0148325	-0.0148325\\
-0.0142825	-0.0142825\\
-0.0136425	-0.0136425\\
-0.01323	-0.01323\\
-0.0134125	-0.0134125\\
-0.01332	-0.01332\\
-0.0127725	-0.0127725\\
-0.0130475	-0.0130475\\
-0.01387	-0.01387\\
-0.013825	-0.013825\\
-0.013595	-0.013595\\
-0.013825	-0.013825\\
-0.0145575	-0.0145575\\
-0.014465	-0.014465\\
-0.0137775	-0.0137775\\
-0.0130925	-0.0130925\\
-0.0137775	-0.0137775\\
-0.014785	-0.014785\\
-0.01538	-0.01538\\
-0.016205	-0.016205\\
-0.016755	-0.016755\\
-0.0168925	-0.0168925\\
-0.0164325	-0.0164325\\
-0.0163425	-0.0163425\\
-0.01657	-0.01657\\
-0.016525	-0.016525\\
-0.0163875	-0.0163875\\
-0.016525	-0.016525\\
-0.0166175	-0.0166175\\
-0.0166625	-0.0166625\\
-0.01703	-0.01703\\
-0.0168	-0.0168\\
-0.015975	-0.015975\\
-0.015015	-0.015015\\
-0.014375	-0.014375\\
-0.0142825	-0.0142825\\
-0.0142375	-0.0142375\\
-0.0148325	-0.0148325\\
-0.014695	-0.014695\\
-0.0146475	-0.0146475\\
-0.01497	-0.01497\\
-0.0146025	-0.0146025\\
-0.0148775	-0.0148775\\
-0.0151525	-0.0151525\\
-0.0146475	-0.0146475\\
-0.0141	-0.0141\\
-0.014145	-0.014145\\
-0.015105	-0.015105\\
-0.015885	-0.015885\\
-0.01593	-0.01593\\
-0.01616	-0.01616\\
-0.01648	-0.01648\\
-0.0169825	-0.0169825\\
-0.017075	-0.017075\\
-0.01648	-0.01648\\
-0.0157925	-0.0157925\\
-0.0148775	-0.0148775\\
-0.0146025	-0.0146025\\
-0.0146475	-0.0146475\\
-0.0148325	-0.0148325\\
-0.0145575	-0.0145575\\
-0.014375	-0.014375\\
-0.014465	-0.014465\\
-0.0146025	-0.0146025\\
-0.01474	-0.01474\\
-0.01506	-0.01506\\
-0.0148325	-0.0148325\\
-0.014145	-0.014145\\
-0.013505	-0.013505\\
-0.0131375	-0.0131375\\
-0.013	-0.013\\
-0.0130475	-0.0130475\\
-0.01355	-0.01355\\
-0.0137775	-0.0137775\\
-0.0140075	-0.0140075\\
-0.0146475	-0.0146475\\
-0.0145575	-0.0145575\\
-0.015015	-0.015015\\
-0.015975	-0.015975\\
-0.015885	-0.015885\\
-0.0157475	-0.0157475\\
-0.0158375	-0.0158375\\
-0.0161125	-0.0161125\\
-0.0158375	-0.0158375\\
-0.0155175	-0.0155175\\
-0.0149225	-0.0149225\\
-0.014785	-0.014785\\
-0.015565	-0.015565\\
-0.01657	-0.01657\\
-0.017165	-0.017165\\
-0.017075	-0.017075\\
-0.016845	-0.016845\\
-0.0164325	-0.0164325\\
-0.0163425	-0.0163425\\
-0.0166625	-0.0166625\\
-0.0161125	-0.0161125\\
-0.0157025	-0.0157025\\
-0.01538	-0.01538\\
-0.015885	-0.015885\\
-0.0160225	-0.0160225\\
-0.0154275	-0.0154275\\
-0.01506	-0.01506\\
-0.014785	-0.014785\\
-0.015105	-0.015105\\
-0.0157025	-0.0157025\\
-0.015885	-0.015885\\
-0.01561	-0.01561\\
-0.01529	-0.01529\\
-0.0152425	-0.0152425\\
-0.015105	-0.015105\\
-0.0146475	-0.0146475\\
-0.0142825	-0.0142825\\
-0.0139625	-0.0139625\\
-0.0136425	-0.0136425\\
-0.013595	-0.013595\\
-0.0139625	-0.0139625\\
-0.0145575	-0.0145575\\
-0.014695	-0.014695\\
-0.0143275	-0.0143275\\
-0.0140525	-0.0140525\\
-0.013915	-0.013915\\
-0.0139625	-0.0139625\\
-0.0143275	-0.0143275\\
-0.01451	-0.01451\\
-0.014465	-0.014465\\
-0.0141	-0.0141\\
-0.0148325	-0.0148325\\
-0.0154275	-0.0154275\\
-0.0151525	-0.0151525\\
-0.01419	-0.01419\\
-0.013915	-0.013915\\
-0.0143275	-0.0143275\\
-0.01442	-0.01442\\
-0.014375	-0.014375\\
-0.013915	-0.013915\\
-0.0142825	-0.0142825\\
-0.0148325	-0.0148325\\
-0.0145575	-0.0145575\\
-0.0136425	-0.0136425\\
-0.0134125	-0.0134125\\
-0.0143275	-0.0143275\\
-0.0148325	-0.0148325\\
-0.01442	-0.01442\\
-0.0141	-0.0141\\
-0.014695	-0.014695\\
-0.015335	-0.015335\\
-0.01538	-0.01538\\
-0.0157475	-0.0157475\\
-0.0163425	-0.0163425\\
-0.0160225	-0.0160225\\
-0.015655	-0.015655\\
-0.01593	-0.01593\\
-0.015655	-0.015655\\
-0.0157475	-0.0157475\\
-0.0160675	-0.0160675\\
-0.015885	-0.015885\\
-0.015105	-0.015105\\
-0.01529	-0.01529\\
-0.016205	-0.016205\\
-0.01657	-0.01657\\
-0.0163425	-0.0163425\\
-0.0160675	-0.0160675\\
-0.0162975	-0.0162975\\
-0.0161125	-0.0161125\\
-0.015565	-0.015565\\
-0.0152425	-0.0152425\\
-0.01538	-0.01538\\
-0.0155175	-0.0155175\\
-0.0154725	-0.0154725\\
-0.01561	-0.01561\\
-0.0152425	-0.0152425\\
-0.0151975	-0.0151975\\
-0.0155175	-0.0155175\\
-0.0151525	-0.0151525\\
-0.01474	-0.01474\\
-0.014465	-0.014465\\
-0.014145	-0.014145\\
-0.0140525	-0.0140525\\
-0.01451	-0.01451\\
-0.014695	-0.014695\\
-0.0148775	-0.0148775\\
-0.015335	-0.015335\\
-0.0157025	-0.0157025\\
-0.015565	-0.015565\\
-0.015335	-0.015335\\
-0.01451	-0.01451\\
-0.0145575	-0.0145575\\
-0.01529	-0.01529\\
-0.015105	-0.015105\\
-0.0143275	-0.0143275\\
-0.0146025	-0.0146025\\
-0.01529	-0.01529\\
-0.0154725	-0.0154725\\
-0.015885	-0.015885\\
-0.016525	-0.016525\\
-0.0163425	-0.0163425\\
-0.0160675	-0.0160675\\
-0.016205	-0.016205\\
-0.0158375	-0.0158375\\
-0.0154275	-0.0154275\\
-0.0154725	-0.0154725\\
-0.0157025	-0.0157025\\
-0.0158375	-0.0158375\\
-0.015335	-0.015335\\
-0.01497	-0.01497\\
-0.01474	-0.01474\\
-0.0148775	-0.0148775\\
-0.014695	-0.014695\\
-0.01529	-0.01529\\
-0.01561	-0.01561\\
-0.01529	-0.01529\\
-0.01506	-0.01506\\
-0.015335	-0.015335\\
-0.0149225	-0.0149225\\
-0.01419	-0.01419\\
-0.0141	-0.0141\\
-0.0143275	-0.0143275\\
-0.0142825	-0.0142825\\
-0.0140075	-0.0140075\\
-0.0133675	-0.0133675\\
-0.0134575	-0.0134575\\
-0.0139625	-0.0139625\\
-0.0146025	-0.0146025\\
-0.01497	-0.01497\\
-0.0154275	-0.0154275\\
-0.0157925	-0.0157925\\
-0.0151975	-0.0151975\\
-0.014695	-0.014695\\
-0.0154725	-0.0154725\\
-0.016205	-0.016205\\
-0.0161125	-0.0161125\\
-0.015975	-0.015975\\
-0.01657	-0.01657\\
-0.016525	-0.016525\\
-0.016205	-0.016205\\
-0.015885	-0.015885\\
-0.0157025	-0.0157025\\
-0.015975	-0.015975\\
-0.0157025	-0.0157025\\
-0.01538	-0.01538\\
-0.015885	-0.015885\\
-0.0162975	-0.0162975\\
-0.0164325	-0.0164325\\
-0.01561	-0.01561\\
-0.0146475	-0.0146475\\
-0.01529	-0.01529\\
-0.0152425	-0.0152425\\
-0.014375	-0.014375\\
-0.0137775	-0.0137775\\
-0.0140075	-0.0140075\\
-0.01451	-0.01451\\
-0.0152425	-0.0152425\\
-0.015105	-0.015105\\
-0.01474	-0.01474\\
-0.0142375	-0.0142375\\
-0.0136425	-0.0136425\\
-0.01355	-0.01355\\
-0.0137325	-0.0137325\\
-0.0142825	-0.0142825\\
-0.0143275	-0.0143275\\
-0.0137325	-0.0137325\\
-0.0134125	-0.0134125\\
-0.01355	-0.01355\\
-0.0137775	-0.0137775\\
-0.0134575	-0.0134575\\
-0.013595	-0.013595\\
-0.0131825	-0.0131825\\
-0.0137325	-0.0137325\\
-0.01442	-0.01442\\
-0.015105	-0.015105\\
-0.015885	-0.015885\\
-0.015975	-0.015975\\
-0.0151525	-0.0151525\\
-0.0145575	-0.0145575\\
-0.01442	-0.01442\\
-0.013915	-0.013915\\
-0.0134575	-0.0134575\\
-0.0139625	-0.0139625\\
-0.0142375	-0.0142375\\
-0.01419	-0.01419\\
-0.0142825	-0.0142825\\
-0.014465	-0.014465\\
-0.0146025	-0.0146025\\
-0.014695	-0.014695\\
-0.015015	-0.015015\\
-0.015105	-0.015105\\
-0.0155175	-0.0155175\\
-0.01506	-0.01506\\
-0.0141	-0.0141\\
-0.0137775	-0.0137775\\
-0.0142375	-0.0142375\\
-0.01419	-0.01419\\
-0.0140525	-0.0140525\\
-0.0134125	-0.0134125\\
-0.0133675	-0.0133675\\
-0.0131375	-0.0131375\\
-0.0125875	-0.0125875\\
-0.01268	-0.01268\\
-0.01332	-0.01332\\
-0.0137325	-0.0137325\\
-0.01442	-0.01442\\
-0.01506	-0.01506\\
-0.01497	-0.01497\\
-0.013915	-0.013915\\
-0.014145	-0.014145\\
-0.014465	-0.014465\\
-0.01387	-0.01387\\
-0.013825	-0.013825\\
-0.0145575	-0.0145575\\
-0.014785	-0.014785\\
-0.01538	-0.01538\\
-0.015975	-0.015975\\
-0.015565	-0.015565\\
-0.0151525	-0.0151525\\
-0.01497	-0.01497\\
-0.0151525	-0.0151525\\
-0.01561	-0.01561\\
-0.01538	-0.01538\\
-0.0148325	-0.0148325\\
};
\end{axis}

\begin{axis}[%
width=4.927cm,
height=2.746cm,
at={(0cm,7.627cm)},
scale only axis,
xmin=-0.018,
xmax=-0.012,
xlabel style={font=\color{white!15!black}},
xlabel={$u(t-1)$},
ymin=-0.0732425,
ymax=0.0061025,
ylabel style={font=\color{white!15!black}},
ylabel={$\delta^4 y(t)$},
axis background/.style={fill=white},
title style={font=\bfseries},
title={C5, R = 0.6806},
axis x line*=bottom,
axis y line*=left
]
\addplot[only marks, mark=*, mark options={}, mark size=1.5000pt, color=mycolor1, fill=mycolor1] table[row sep=crcr]{%
x	y\\
-0.015655	-0.0396725\\
-0.015655	-0.03662\\
-0.0158375	-0.03662\\
-0.015655	-0.01831\\
-0.01497	-0.0213625\\
-0.0148325	-0.027465\\
-0.0148775	-0.024415\\
-0.01497	-0.024415\\
-0.0148325	-0.01526\\
-0.0139625	-0.0061025\\
-0.0131825	-0.009155\\
-0.0128175	-0.0061025\\
-0.0137325	-0.0305175\\
-0.014785	-0.03357\\
-0.01506	-0.0305175\\
-0.0151525	-0.027465\\
-0.0149225	-0.01526\\
-0.0143275	-0.01526\\
-0.0148775	-0.0030525\\
-0.0148775	-0.01526\\
-0.01442	-0.027465\\
-0.0149225	-0.0305175\\
-0.0149225	-0.0213625\\
-0.01474	-0.03662\\
-0.0151525	-0.03662\\
-0.0152425	-0.024415\\
-0.01497	-0.03357\\
-0.01529	-0.0305175\\
-0.01538	-0.024415\\
-0.0151975	-0.024415\\
-0.01497	-0.0213625\\
-0.014695	-0.0213625\\
-0.014695	-0.0305175\\
-0.0148775	-0.027465\\
-0.0148775	-0.027465\\
-0.0149225	-0.027465\\
-0.01497	-0.027465\\
-0.015105	-0.03662\\
-0.0154725	-0.0396725\\
-0.0155175	-0.024415\\
-0.015105	-0.0213625\\
-0.0149225	-0.0213625\\
-0.0143275	-0.01526\\
-0.014145	-0.01831\\
-0.0143275	-0.01831\\
-0.0142375	-0.0122075\\
-0.014145	-0.024415\\
-0.01442	-0.027465\\
-0.0146475	-0.0213625\\
-0.0146475	-0.03357\\
-0.0151525	-0.0396725\\
-0.015105	-0.0213625\\
-0.014465	-0.01526\\
-0.0140525	-0.01526\\
-0.0142375	-0.024415\\
-0.014375	-0.01526\\
-0.0139625	-0.0122075\\
-0.0137325	-0.01831\\
-0.0140075	-0.01831\\
-0.01387	-0.009155\\
-0.0137325	-0.01526\\
-0.01387	-0.0213625\\
-0.0140075	-0.027465\\
-0.014465	-0.027465\\
-0.014695	-0.024415\\
-0.0148775	-0.03357\\
-0.01497	-0.027465\\
-0.015105	-0.0305175\\
-0.01497	-0.027465\\
-0.0148775	-0.027465\\
-0.014785	-0.01831\\
-0.01474	-0.0213625\\
-0.014695	-0.03662\\
-0.0151525	-0.05188\\
-0.01593	-0.05188\\
-0.016205	-0.0488275\\
-0.01625	-0.03357\\
-0.0157925	-0.0396725\\
-0.0160675	-0.061035\\
-0.01648	-0.0579825\\
-0.0166175	-0.0396725\\
-0.0163875	-0.0549325\\
-0.0166175	-0.07019\\
-0.017165	-0.0579825\\
-0.0169825	-0.042725\\
-0.0166175	-0.03662\\
-0.01616	-0.03357\\
-0.015655	-0.024415\\
-0.01538	-0.027465\\
-0.01538	-0.0305175\\
-0.0151525	-0.0122075\\
-0.0145575	-0.0122075\\
-0.0142375	-0.01831\\
-0.0143275	-0.027465\\
-0.014695	-0.024415\\
-0.014785	-0.024415\\
-0.014785	-0.027465\\
-0.015015	-0.03357\\
-0.0151975	-0.03662\\
-0.01561	-0.0396725\\
-0.01561	-0.03662\\
-0.0157025	-0.042725\\
-0.015565	-0.0305175\\
-0.01529	-0.0305175\\
-0.01538	-0.027465\\
-0.0151525	-0.0213625\\
-0.0149225	-0.024415\\
-0.015015	-0.0305175\\
-0.01506	-0.0213625\\
-0.01474	-0.024415\\
-0.014695	-0.01831\\
-0.01442	-0.01526\\
-0.014375	-0.0213625\\
-0.01442	-0.01526\\
-0.014145	-0.0122075\\
-0.014145	-0.0213625\\
-0.01442	-0.03357\\
-0.0148325	-0.0305175\\
-0.015105	-0.027465\\
-0.01529	-0.01831\\
-0.0148325	-0.0213625\\
-0.01497	-0.0488275\\
-0.01561	-0.03662\\
-0.0155175	-0.027465\\
-0.01561	-0.042725\\
-0.0157025	-0.0305175\\
-0.0151525	-0.01526\\
-0.0146475	-0.024415\\
-0.014785	-0.0213625\\
-0.0149225	-0.03662\\
-0.0155175	-0.0396725\\
-0.0160225	-0.0488275\\
-0.0161125	-0.03662\\
-0.0158375	-0.03662\\
-0.0157475	-0.03357\\
-0.0157475	-0.03357\\
-0.01561	-0.03662\\
-0.01561	-0.027465\\
-0.0152425	-0.01831\\
-0.01497	-0.01831\\
-0.014785	-0.01526\\
-0.0146025	-0.0213625\\
-0.0146475	-0.027465\\
-0.015015	-0.0396725\\
-0.01561	-0.03662\\
-0.015565	-0.0213625\\
-0.015105	-0.0213625\\
-0.0148325	-0.024415\\
-0.01474	-0.01526\\
-0.0142375	-0.0061025\\
-0.0136875	-0.0122075\\
-0.013595	-0.01831\\
-0.0141	-0.0213625\\
-0.0140525	-0.01526\\
-0.0140075	-0.01831\\
-0.014145	-0.01526\\
-0.0140525	-0.0122075\\
-0.013825	-0.0122075\\
-0.0134125	-0.0030525\\
-0.0130925	-0.009155\\
-0.013275	-0.0305175\\
-0.01419	-0.03662\\
-0.01497	-0.0396725\\
-0.01538	-0.0305175\\
-0.015015	-0.01831\\
-0.0146025	-0.01526\\
-0.014145	-0.009155\\
-0.013595	-0.01831\\
-0.0140525	-0.0213625\\
-0.0142825	-0.024415\\
-0.0146025	-0.0305175\\
-0.01506	-0.0457775\\
-0.01593	-0.0457775\\
-0.016525	-0.0488275\\
-0.0164325	-0.0457775\\
-0.0163875	-0.0396725\\
-0.015975	-0.0396725\\
-0.015885	-0.042725\\
-0.015975	-0.0457775\\
-0.0160225	-0.0305175\\
-0.0158375	-0.027465\\
-0.01561	-0.0305175\\
-0.015565	-0.027465\\
-0.0155175	-0.0305175\\
-0.015565	-0.024415\\
-0.0152425	-0.0396725\\
-0.01561	-0.0457775\\
-0.0160675	-0.0305175\\
-0.0157925	-0.01831\\
-0.0152425	-0.024415\\
-0.015015	-0.042725\\
-0.015565	-0.0457775\\
-0.0160225	-0.0457775\\
-0.0164325	-0.0549325\\
-0.0166175	-0.05188\\
-0.0166625	-0.042725\\
-0.0164325	-0.0396725\\
-0.016205	-0.0457775\\
-0.0163425	-0.05188\\
-0.01657	-0.042725\\
-0.0161125	-0.0213625\\
-0.0154725	-0.03662\\
-0.0155175	-0.0305175\\
-0.0155175	-0.01831\\
-0.0149225	-0.027465\\
-0.01497	-0.024415\\
-0.0151525	-0.0213625\\
-0.01497	-0.01831\\
-0.0146475	-0.0213625\\
-0.014785	-0.0213625\\
-0.014695	-0.024415\\
-0.0148775	-0.03357\\
-0.01529	-0.03662\\
-0.015335	-0.01831\\
-0.015015	-0.0213625\\
-0.0149225	-0.0305175\\
-0.01529	-0.042725\\
-0.0160225	-0.0396725\\
-0.01593	-0.0213625\\
-0.015335	-0.0213625\\
-0.0148775	-0.024415\\
-0.0149225	-0.0213625\\
-0.0148775	-0.01831\\
-0.01451	-0.01526\\
-0.014465	-0.0213625\\
-0.0146475	-0.024415\\
-0.0149225	-0.024415\\
-0.01506	-0.024415\\
-0.01474	-0.0305175\\
-0.0152425	-0.03662\\
-0.01561	-0.0396725\\
-0.015565	-0.027465\\
-0.01538	-0.03357\\
-0.01538	-0.042725\\
-0.015655	-0.0305175\\
-0.015335	-0.009155\\
-0.01442	-0.024415\\
-0.0143275	-0.0213625\\
-0.01442	-0.024415\\
-0.01474	-0.0213625\\
-0.01474	-0.0213625\\
-0.014465	-0.0213625\\
-0.01474	-0.027465\\
-0.0148775	-0.01831\\
-0.0145575	-0.01831\\
-0.0146475	-0.024415\\
-0.0146025	-0.01526\\
-0.014465	-0.0213625\\
-0.0145575	-0.01526\\
-0.01419	-0.009155\\
-0.014145	-0.01526\\
-0.013915	-0.0122075\\
-0.013825	-0.027465\\
-0.014375	-0.0305175\\
-0.014695	-0.0305175\\
-0.015105	-0.0457775\\
-0.015655	-0.027465\\
-0.01538	-0.0122075\\
-0.01474	-0.01526\\
-0.0142825	-0.01526\\
-0.014145	-0.0213625\\
-0.014375	-0.01526\\
-0.0140525	-0.0213625\\
-0.013915	-0.01831\\
-0.01419	-0.03357\\
-0.014785	-0.024415\\
-0.015015	-0.03662\\
-0.015335	-0.027465\\
-0.015105	-0.024415\\
-0.0148775	-0.01526\\
-0.0141	-0.009155\\
-0.013825	-0.009155\\
-0.0134125	-0.0030525\\
-0.013275	-0.0061025\\
-0.0134125	-0.0213625\\
-0.013915	-0.0213625\\
-0.0142825	-0.01831\\
-0.01442	-0.024415\\
-0.0146025	-0.0396725\\
-0.0151975	-0.027465\\
-0.01538	-0.024415\\
-0.015105	-0.024415\\
-0.0151525	-0.03357\\
-0.01529	-0.03662\\
-0.015565	-0.0305175\\
-0.01561	-0.0305175\\
-0.01506	-0.01526\\
-0.0142825	-0.0061025\\
-0.0136425	-0.0122075\\
-0.0134125	-0.0122075\\
-0.013505	-0.01526\\
-0.01355	-0.009155\\
-0.013505	-0.0122075\\
-0.0136875	-0.024415\\
-0.0142375	-0.01831\\
-0.0143275	-0.01831\\
-0.014375	-0.027465\\
-0.01451	-0.01831\\
-0.014145	-0.0122075\\
-0.0134575	-0.024415\\
-0.013825	-0.0305175\\
-0.01442	-0.024415\\
-0.0146025	-0.027465\\
-0.01474	-0.01831\\
-0.0146475	-0.01831\\
-0.014375	-0.027465\\
-0.0145575	-0.03662\\
-0.01506	-0.03357\\
-0.0151975	-0.0213625\\
-0.0149225	-0.03662\\
-0.01538	-0.03357\\
-0.01538	-0.03357\\
-0.0151975	-0.03357\\
-0.01538	-0.03357\\
-0.01538	-0.027465\\
-0.0151975	-0.027465\\
-0.01497	-0.0305175\\
-0.0155175	-0.0457775\\
-0.0161125	-0.042725\\
-0.015975	-0.027465\\
-0.015335	-0.027465\\
-0.01506	-0.027465\\
-0.01506	-0.03357\\
-0.0152425	-0.03357\\
-0.0154275	-0.024415\\
-0.0152425	-0.0305175\\
-0.0151525	-0.03662\\
-0.0154725	-0.03662\\
-0.015565	-0.024415\\
-0.01538	-0.0213625\\
-0.0149225	-0.027465\\
-0.015105	-0.0457775\\
-0.0157025	-0.03662\\
-0.0158375	-0.03357\\
-0.0157925	-0.024415\\
-0.015335	-0.027465\\
-0.01497	-0.024415\\
-0.01497	-0.01526\\
-0.01442	-0.0122075\\
-0.0137325	-0.009155\\
-0.01332	-0.009155\\
-0.013	-0.01831\\
-0.0134575	-0.027465\\
-0.014145	-0.024415\\
-0.01451	-0.01526\\
-0.01442	-0.01831\\
-0.014375	-0.024415\\
-0.0145575	-0.01526\\
-0.014145	-0.01526\\
-0.0140075	-0.024415\\
-0.0140075	-0.01526\\
-0.0137775	-0.01831\\
-0.01387	-0.0305175\\
-0.014375	-0.0305175\\
-0.0148325	-0.0213625\\
-0.01451	-0.0122075\\
-0.014145	-0.01526\\
-0.013915	-0.01831\\
-0.013915	-0.01526\\
-0.0136875	-0.01831\\
-0.0142375	-0.03662\\
-0.0152425	-0.024415\\
-0.0152425	-0.042725\\
-0.0155175	-0.05188\\
-0.0158375	-0.0457775\\
-0.0160225	-0.03357\\
-0.0157925	-0.027465\\
-0.01538	-0.027465\\
-0.0151975	-0.0305175\\
-0.0151975	-0.03357\\
-0.01529	-0.03662\\
-0.0154725	-0.0396725\\
-0.015565	-0.05188\\
-0.01593	-0.05188\\
-0.01625	-0.061035\\
-0.0166175	-0.0396725\\
-0.0163425	-0.05188\\
-0.01625	-0.05188\\
-0.0164325	-0.03662\\
-0.01625	-0.042725\\
-0.0162975	-0.0457775\\
-0.01625	-0.024415\\
-0.01561	-0.01831\\
-0.014785	-0.0213625\\
-0.0145575	-0.024415\\
-0.0148775	-0.01831\\
-0.014785	-0.01526\\
-0.014375	-0.01831\\
-0.01451	-0.01526\\
-0.0143275	-0.009155\\
-0.013915	-0.0122075\\
-0.0139625	-0.01831\\
-0.0140525	-0.0061025\\
-0.01387	-0.027465\\
-0.0142375	-0.0213625\\
-0.014695	-0.01831\\
-0.0145575	-0.0396725\\
-0.01506	-0.0488275\\
-0.0157925	-0.0488275\\
-0.01593	-0.0488275\\
-0.01625	-0.03662\\
-0.015975	-0.042725\\
-0.01593	-0.0305175\\
-0.015565	-0.01831\\
-0.0148775	-0.027465\\
-0.0148775	-0.01831\\
-0.014695	-0.009155\\
-0.0143275	-0.01526\\
-0.014465	-0.0122075\\
-0.0140525	-0.0061025\\
-0.0134125	-0.0122075\\
-0.0133675	-0.027465\\
-0.0140075	-0.024415\\
-0.0145575	-0.027465\\
-0.0149225	-0.027465\\
-0.015105	-0.0305175\\
-0.015105	-0.03662\\
-0.0151975	-0.027465\\
-0.0151525	-0.027465\\
-0.0148775	-0.0213625\\
-0.0148325	-0.01831\\
-0.014695	-0.03357\\
-0.0149225	-0.024415\\
-0.01474	-0.01526\\
-0.01442	-0.027465\\
-0.014695	-0.03357\\
-0.015105	-0.027465\\
-0.0149225	-0.01831\\
-0.0146475	-0.01831\\
-0.01451	-0.01831\\
-0.01451	-0.0122075\\
-0.0142375	-0.01526\\
-0.0139625	-0.01831\\
-0.0142375	-0.027465\\
-0.01451	-0.0213625\\
-0.014695	-0.01831\\
-0.0145575	-0.027465\\
-0.014695	-0.0213625\\
-0.01474	-0.01526\\
-0.014375	-0.01831\\
-0.014145	-0.03357\\
-0.014785	-0.03662\\
-0.015335	-0.03357\\
-0.0154725	-0.0305175\\
-0.015335	-0.0305175\\
-0.0152425	-0.0213625\\
-0.015015	-0.024415\\
-0.0148775	-0.0213625\\
-0.0146475	-0.0305175\\
-0.0148325	-0.01831\\
-0.014785	-0.0122075\\
-0.014375	-0.024415\\
-0.0145575	-0.024415\\
-0.0148325	-0.027465\\
-0.01497	-0.0305175\\
-0.0151975	-0.027465\\
-0.0151975	-0.042725\\
-0.01561	-0.05188\\
-0.0161125	-0.0579825\\
-0.0163875	-0.0396725\\
-0.0160225	-0.024415\\
-0.0152425	-0.01526\\
-0.0146025	-0.01526\\
-0.0139625	-0.024415\\
-0.0140075	-0.01526\\
-0.0140525	-0.024415\\
-0.0145575	-0.01526\\
-0.0146475	-0.01831\\
-0.01451	-0.0213625\\
-0.014695	-0.027465\\
-0.0149225	-0.03357\\
-0.0152425	-0.03662\\
-0.01538	-0.0488275\\
-0.015975	-0.03662\\
-0.0160675	-0.03357\\
-0.0157925	-0.027465\\
-0.01529	-0.027465\\
-0.0151525	-0.027465\\
-0.0151525	-0.0305175\\
-0.01538	-0.024415\\
-0.015105	-0.027465\\
-0.01497	-0.024415\\
-0.01497	-0.01831\\
-0.014465	-0.0305175\\
-0.0146475	-0.03357\\
-0.0151975	-0.0213625\\
-0.015015	-0.024415\\
-0.0148775	-0.0457775\\
-0.01561	-0.0305175\\
-0.01561	-0.0305175\\
-0.0154275	-0.042725\\
-0.0157925	-0.03662\\
-0.015885	-0.027465\\
-0.0154275	-0.01831\\
-0.014785	-0.01831\\
-0.0146025	-0.0305175\\
-0.01506	-0.042725\\
-0.0158375	-0.05188\\
-0.0160675	-0.0457775\\
-0.01616	-0.03357\\
-0.015975	-0.024415\\
-0.0154275	-0.0213625\\
-0.015105	-0.01526\\
-0.014695	-0.01831\\
-0.014465	-0.0122075\\
-0.0142825	-0.0213625\\
-0.01442	-0.0213625\\
-0.0146475	-0.01526\\
-0.01419	-0.024415\\
-0.014375	-0.027465\\
-0.014785	-0.0305175\\
-0.01497	-0.03662\\
-0.0154275	-0.042725\\
-0.0158375	-0.0488275\\
-0.0162975	-0.0457775\\
-0.015975	-0.03357\\
-0.0157475	-0.03662\\
-0.0157475	-0.0305175\\
-0.01561	-0.01831\\
-0.0151975	-0.03357\\
-0.0151525	-0.0488275\\
-0.0157475	-0.0457775\\
-0.0161125	-0.0305175\\
-0.01561	-0.01526\\
-0.0148775	-0.0122075\\
-0.0142375	-0.0030525\\
-0.0133675	0.0061025\\
-0.0130475	-0.0122075\\
-0.0134575	-0.01526\\
-0.0137775	-0.0213625\\
-0.0141	-0.01526\\
-0.0141	-0.0122075\\
-0.01387	-0.0122075\\
-0.0137775	-0.01831\\
-0.0137325	-0.0122075\\
-0.0136875	-0.0122075\\
-0.0137325	-0.0213625\\
-0.014145	-0.0305175\\
-0.01474	-0.03357\\
-0.01506	-0.027465\\
-0.015105	-0.03662\\
-0.01529	-0.042725\\
-0.0157025	-0.042725\\
-0.015885	-0.05188\\
-0.016205	-0.05188\\
-0.01625	-0.03662\\
-0.01593	-0.027465\\
-0.0155175	-0.027465\\
-0.0152425	-0.0457775\\
-0.0157475	-0.0457775\\
-0.0160675	-0.0305175\\
-0.0157925	-0.0213625\\
-0.01529	-0.0122075\\
-0.014375	-0.0122075\\
-0.0136425	-0.0122075\\
-0.013505	-0.0061025\\
-0.013	-0.01526\\
-0.0133675	-0.0213625\\
-0.013825	-0.01831\\
-0.0139625	-0.01831\\
-0.0139625	-0.01526\\
-0.013505	-0.009155\\
-0.0131375	-0.0122075\\
-0.013275	-0.0122075\\
-0.01332	-0.01526\\
-0.0134575	-0.0061025\\
-0.01332	-0.0061025\\
-0.0130925	-0.01831\\
-0.0134125	-0.024415\\
-0.01451	-0.03357\\
-0.0151975	-0.0396725\\
-0.0155175	-0.027465\\
-0.01529	-0.0457775\\
-0.015565	-0.0579825\\
-0.01625	-0.0549325\\
-0.0164325	-0.0488275\\
-0.0164325	-0.03662\\
-0.01616	-0.03662\\
-0.0161125	-0.042725\\
-0.0161125	-0.03357\\
-0.0157025	-0.024415\\
-0.0154725	-0.01831\\
-0.0149225	-0.01831\\
-0.01451	-0.03357\\
-0.015105	-0.0213625\\
-0.0149225	-0.0305175\\
-0.0148775	-0.027465\\
-0.015015	-0.0213625\\
-0.01497	-0.0213625\\
-0.0149225	-0.027465\\
-0.0149225	-0.024415\\
-0.01506	-0.0305175\\
-0.0151975	-0.027465\\
-0.0151525	-0.0213625\\
-0.0148775	-0.0305175\\
-0.0149225	-0.01526\\
-0.014375	-0.0061025\\
-0.01332	-0.01831\\
-0.01323	-0.01831\\
-0.0136875	-0.0061025\\
-0.01332	-0.0030525\\
-0.01245	-0.0061025\\
-0.01268	-0.01526\\
-0.013275	-0.01526\\
-0.0136875	-0.0213625\\
-0.0140075	-0.01831\\
-0.0140075	-0.03357\\
-0.01451	-0.0213625\\
-0.014695	-0.01526\\
-0.0142825	-0.027465\\
-0.01451	-0.01831\\
-0.0140525	-0.0122075\\
-0.013595	-0.0122075\\
-0.013275	-0.0061025\\
-0.0127725	-0.009155\\
-0.013	-0.009155\\
-0.012955	-0.01526\\
-0.013275	-0.024415\\
-0.0139625	-0.0213625\\
-0.014145	-0.0122075\\
-0.013915	-0.01831\\
-0.01387	-0.01831\\
-0.0136875	-0.01831\\
-0.013825	-0.01526\\
-0.0136425	-0.0061025\\
-0.01355	-0.01831\\
-0.0134575	-0.009155\\
-0.0131825	-0.0122075\\
-0.01332	-0.0122075\\
-0.013275	-0.01831\\
-0.0137325	-0.01526\\
-0.01387	-0.0122075\\
-0.013595	-0.009155\\
-0.01355	-0.0122075\\
-0.01332	-0.0122075\\
-0.01332	-0.03357\\
-0.0142375	-0.03357\\
-0.0148775	-0.01831\\
-0.0146475	-0.0305175\\
-0.01451	-0.01831\\
-0.014145	-0.0305175\\
-0.0145575	-0.0213625\\
-0.014695	-0.01831\\
-0.0142375	-0.0213625\\
-0.0142825	-0.01831\\
-0.014145	-0.0122075\\
-0.014375	-0.01526\\
-0.0140075	-0.024415\\
-0.0137775	-0.01831\\
-0.0139625	-0.0305175\\
-0.01419	-0.01526\\
-0.0140075	-0.0213625\\
-0.0140075	-0.01526\\
-0.0140525	-0.0305175\\
-0.014465	-0.01831\\
-0.01451	-0.03662\\
-0.0149225	-0.042725\\
-0.015565	-0.03357\\
-0.0155175	-0.024415\\
-0.01497	-0.024415\\
-0.0145575	-0.0305175\\
-0.0148775	-0.03357\\
-0.0152425	-0.0213625\\
-0.01506	-0.03357\\
-0.0151525	-0.027465\\
-0.014785	-0.0122075\\
-0.013825	-0.009155\\
-0.013595	-0.0305175\\
-0.0142375	-0.027465\\
-0.014465	-0.0213625\\
-0.0143275	-0.027465\\
-0.014785	-0.0305175\\
-0.0149225	-0.024415\\
-0.014785	-0.01831\\
-0.01451	-0.024415\\
-0.0145575	-0.0305175\\
-0.0148775	-0.0213625\\
-0.014785	-0.027465\\
-0.01474	-0.024415\\
-0.014695	-0.01526\\
-0.01442	-0.0213625\\
-0.014375	-0.0122075\\
-0.01419	-0.0213625\\
-0.0141	-0.027465\\
-0.0146475	-0.03662\\
-0.0149225	-0.03357\\
-0.01506	-0.03357\\
-0.015015	-0.0213625\\
-0.014695	-0.01526\\
-0.0142825	-0.01831\\
-0.0142825	-0.0213625\\
-0.014465	-0.0305175\\
-0.014695	-0.0305175\\
-0.01506	-0.03662\\
-0.01529	-0.027465\\
-0.0151525	-0.0213625\\
-0.0145575	-0.0305175\\
-0.01497	-0.0457775\\
-0.015565	-0.0305175\\
-0.015335	-0.03662\\
-0.0151975	-0.03662\\
-0.01538	-0.024415\\
-0.0151975	-0.024415\\
-0.01497	-0.0305175\\
-0.01497	-0.024415\\
-0.0149225	-0.027465\\
-0.015015	-0.03662\\
-0.0152425	-0.024415\\
-0.01506	-0.0305175\\
-0.015105	-0.027465\\
-0.01506	-0.027465\\
-0.01506	-0.01526\\
-0.0149225	-0.024415\\
-0.015105	-0.0213625\\
-0.0148775	-0.024415\\
-0.01474	-0.024415\\
-0.0148775	-0.024415\\
-0.0148775	-0.01526\\
-0.0142375	-0.009155\\
-0.0134575	-0.0061025\\
-0.01291	-0.01526\\
-0.0130925	-0.027465\\
-0.0142375	-0.0305175\\
-0.0149225	-0.027465\\
-0.015015	-0.0213625\\
-0.01474	-0.0305175\\
-0.01474	-0.0213625\\
-0.014695	-0.024415\\
-0.01451	-0.01526\\
-0.0140525	-0.0213625\\
-0.0141	-0.01831\\
-0.014145	-0.01831\\
-0.014145	-0.01831\\
-0.0142375	-0.01831\\
-0.01419	-0.01526\\
-0.0141	-0.01526\\
-0.0137325	-0.0122075\\
-0.0137775	-0.027465\\
-0.01451	-0.01831\\
-0.01442	-0.024415\\
-0.014465	-0.01831\\
-0.014465	-0.0305175\\
-0.014695	-0.024415\\
-0.0148325	-0.03662\\
-0.01506	-0.0213625\\
-0.01497	-0.03357\\
-0.015015	-0.0305175\\
-0.01506	-0.01831\\
-0.014465	-0.027465\\
-0.01474	-0.024415\\
-0.0146025	-0.027465\\
-0.0146025	-0.0305175\\
-0.014785	-0.03357\\
-0.01506	-0.01831\\
-0.0149225	-0.03357\\
-0.015015	-0.03662\\
-0.01538	-0.03662\\
-0.01529	-0.024415\\
-0.01506	-0.024415\\
-0.0148775	-0.027465\\
-0.0149225	-0.01831\\
-0.0148775	-0.024415\\
-0.0146475	-0.01831\\
-0.014465	-0.0213625\\
-0.014465	-0.01831\\
-0.014465	-0.03357\\
-0.01506	-0.0396725\\
-0.01561	-0.0396725\\
-0.015565	-0.03662\\
-0.015565	-0.0396725\\
-0.0155175	-0.0213625\\
-0.015015	-0.01831\\
-0.0146025	-0.01526\\
-0.014375	-0.0122075\\
-0.0141	-0.0122075\\
-0.014145	-0.0213625\\
-0.0146475	-0.027465\\
-0.0149225	-0.03357\\
-0.0151525	-0.0305175\\
-0.01506	-0.01831\\
-0.0146475	-0.0305175\\
-0.0151525	-0.0396725\\
-0.0155175	-0.0396725\\
-0.0157475	-0.03357\\
-0.015565	-0.05188\\
-0.0161125	-0.042725\\
-0.0160675	-0.027465\\
-0.015335	-0.01526\\
-0.014785	-0.01831\\
-0.014465	-0.027465\\
-0.0148325	-0.03357\\
-0.0152425	-0.042725\\
-0.015655	-0.0305175\\
-0.0155175	-0.027465\\
-0.0152425	-0.01526\\
-0.01451	-0.024415\\
-0.0142375	-0.01831\\
-0.0143275	-0.01831\\
-0.01442	-0.0122075\\
-0.0141	-0.0122075\\
-0.0141	-0.009155\\
-0.0140075	-0.009155\\
-0.013505	-0.01526\\
-0.013915	-0.027465\\
-0.01419	-0.0213625\\
-0.01451	-0.024415\\
-0.01451	-0.0213625\\
-0.0145575	-0.0213625\\
-0.0146025	-0.0305175\\
-0.0148325	-0.01831\\
-0.0146025	-0.0305175\\
-0.01474	-0.03357\\
-0.015105	-0.0213625\\
-0.0149225	-0.027465\\
-0.0149225	-0.0213625\\
-0.0148775	-0.027465\\
-0.0148775	-0.0396725\\
-0.01529	-0.027465\\
-0.0151525	-0.0213625\\
-0.0149225	-0.024415\\
-0.01497	-0.027465\\
-0.0149225	-0.01831\\
-0.0145575	-0.0213625\\
-0.0148325	-0.03357\\
-0.015335	-0.03662\\
-0.015335	-0.0396725\\
-0.0154725	-0.05188\\
-0.0161125	-0.0396725\\
-0.015885	-0.03662\\
-0.015565	-0.027465\\
-0.0152425	-0.027465\\
-0.01538	-0.042725\\
-0.0158375	-0.03357\\
-0.0155175	-0.027465\\
-0.01529	-0.03662\\
-0.0155175	-0.024415\\
-0.0151975	-0.01831\\
-0.01451	-0.024415\\
-0.014465	-0.01526\\
-0.014375	-0.009155\\
-0.0140075	-0.01526\\
-0.013915	-0.0122075\\
-0.013825	-0.0122075\\
-0.0136875	-0.0122075\\
-0.01387	-0.0213625\\
-0.01419	-0.024415\\
-0.014465	-0.0305175\\
-0.0148325	-0.0305175\\
-0.01497	-0.0305175\\
-0.0151525	-0.03662\\
-0.0154275	-0.03662\\
-0.0154275	-0.027465\\
-0.0151525	-0.027465\\
-0.01506	-0.024415\\
-0.015015	-0.027465\\
-0.015015	-0.03662\\
-0.015335	-0.027465\\
-0.0151975	-0.027465\\
-0.0151975	-0.024415\\
-0.0151525	-0.027465\\
-0.01506	-0.03662\\
-0.0155175	-0.0305175\\
-0.0154725	-0.0305175\\
-0.0154275	-0.03357\\
-0.015335	-0.024415\\
-0.0148775	-0.01526\\
-0.0143275	-0.0122075\\
-0.013825	-0.01831\\
-0.014145	-0.01831\\
-0.01419	-0.01831\\
-0.01442	-0.01526\\
-0.0143275	-0.0213625\\
-0.0145575	-0.03662\\
-0.01529	-0.0457775\\
-0.0157925	-0.0305175\\
-0.015655	-0.03662\\
-0.01561	-0.03357\\
-0.015565	-0.027465\\
-0.0151525	-0.027465\\
-0.015105	-0.027465\\
-0.0151975	-0.027465\\
-0.0151525	-0.03357\\
-0.0151525	-0.03662\\
-0.0154725	-0.0488275\\
-0.015975	-0.042725\\
-0.0160675	-0.0396725\\
-0.0160225	-0.0488275\\
-0.0161125	-0.03357\\
-0.015655	-0.0396725\\
-0.01561	-0.03357\\
-0.0154275	-0.027465\\
-0.015335	-0.03357\\
-0.01561	-0.042725\\
-0.0157925	-0.01831\\
-0.015015	-0.027465\\
-0.01474	-0.01831\\
-0.0146475	-0.0213625\\
-0.0148775	-0.027465\\
-0.01497	-0.01831\\
-0.0146475	-0.0213625\\
-0.014695	-0.0396725\\
-0.01529	-0.05188\\
-0.01625	-0.0549325\\
-0.016525	-0.0488275\\
-0.0164325	-0.0396725\\
-0.016205	-0.0488275\\
-0.0162975	-0.0549325\\
-0.016525	-0.042725\\
-0.01625	-0.0457775\\
-0.01616	-0.042725\\
-0.01625	-0.0305175\\
-0.0158375	-0.024415\\
-0.0155175	-0.03357\\
-0.0157025	-0.027465\\
-0.01538	-0.01526\\
-0.0149225	-0.024415\\
-0.01506	-0.0213625\\
-0.0148325	-0.0213625\\
-0.014785	-0.0213625\\
-0.014785	-0.024415\\
-0.0149225	-0.0305175\\
-0.015335	-0.0213625\\
-0.0151975	-0.01831\\
-0.01497	-0.0213625\\
-0.01497	-0.027465\\
-0.0151525	-0.027465\\
-0.015335	-0.024415\\
-0.0152425	-0.024415\\
-0.015015	-0.03662\\
-0.01538	-0.03357\\
-0.0154725	-0.024415\\
-0.0151525	-0.03357\\
-0.01561	-0.03357\\
-0.015655	-0.027465\\
-0.015335	-0.0213625\\
-0.0149225	-0.01831\\
-0.014375	-0.0122075\\
-0.0139625	-0.01526\\
-0.01442	-0.0213625\\
-0.0145575	-0.01526\\
-0.0142825	-0.0122075\\
-0.01387	-0.0122075\\
-0.01387	-0.01831\\
-0.0142825	-0.01831\\
-0.0146025	-0.0305175\\
-0.0149225	-0.03662\\
-0.01529	-0.03357\\
-0.015335	-0.03357\\
-0.0154275	-0.027465\\
-0.015015	-0.0122075\\
-0.0140075	-0.01526\\
-0.0139625	-0.01831\\
-0.013915	-0.0213625\\
-0.0140075	-0.0213625\\
-0.0146025	-0.027465\\
-0.0148775	-0.03662\\
-0.0154275	-0.03662\\
-0.0152425	-0.024415\\
-0.01497	-0.03662\\
-0.0154725	-0.03357\\
-0.01529	-0.01526\\
-0.0148775	-0.0213625\\
-0.014465	-0.01831\\
-0.0145575	-0.027465\\
-0.0149225	-0.042725\\
-0.015335	-0.0305175\\
-0.0151525	-0.024415\\
-0.0148325	-0.0213625\\
-0.014785	-0.027465\\
-0.0148775	-0.0305175\\
-0.0152425	-0.0488275\\
-0.01593	-0.0396725\\
-0.0160225	-0.042725\\
-0.015975	-0.0305175\\
-0.0154725	-0.01831\\
-0.0149225	-0.0213625\\
-0.0149225	-0.0122075\\
-0.0139625	-0.01526\\
-0.0139625	-0.0213625\\
-0.0140525	-0.009155\\
-0.0141	-0.0213625\\
-0.01474	-0.03357\\
-0.01529	-0.0396725\\
-0.015655	-0.05188\\
-0.01616	-0.0457775\\
-0.01616	-0.024415\\
-0.0155175	-0.027465\\
-0.0151525	-0.0213625\\
-0.015015	-0.03357\\
-0.01506	-0.0396725\\
-0.0154725	-0.0396725\\
-0.01593	-0.0396725\\
-0.0157475	-0.024415\\
-0.01529	-0.027465\\
-0.015335	-0.03662\\
-0.01561	-0.03357\\
-0.01538	-0.01831\\
-0.0149225	-0.01526\\
-0.0145575	-0.0122075\\
-0.014145	-0.009155\\
-0.0137775	-0.0122075\\
-0.0134575	-0.009155\\
-0.0137775	-0.0213625\\
-0.0142825	-0.0213625\\
-0.01442	-0.01526\\
-0.0142375	-0.0213625\\
-0.0141	-0.01526\\
-0.0134125	-0.009155\\
-0.0125875	-0.0061025\\
-0.0124975	-0.01526\\
-0.0130475	-0.0213625\\
-0.0136425	-0.0213625\\
-0.014145	-0.01831\\
-0.01442	-0.027465\\
-0.01474	-0.03662\\
-0.0151975	-0.0488275\\
-0.015975	-0.0549325\\
-0.016205	-0.05188\\
-0.0163875	-0.0488275\\
-0.0163425	-0.03357\\
-0.0157025	-0.027465\\
-0.015335	-0.027465\\
-0.01529	-0.0305175\\
-0.0155175	-0.027465\\
-0.0154725	-0.01831\\
-0.015105	-0.01831\\
-0.014695	-0.01831\\
-0.0143275	-0.0213625\\
-0.0146475	-0.027465\\
-0.014785	-0.01831\\
-0.0142825	-0.0061025\\
-0.0137325	-0.0122075\\
-0.013915	-0.01526\\
-0.0142825	-0.01831\\
-0.014145	-0.01526\\
-0.014375	-0.0213625\\
-0.0143275	-0.01526\\
-0.0146025	-0.03662\\
-0.0151975	-0.027465\\
-0.0151525	-0.0305175\\
-0.0154275	-0.05188\\
-0.0160225	-0.05188\\
-0.0163425	-0.0396725\\
-0.016205	-0.03357\\
-0.01561	-0.0213625\\
-0.0151525	-0.01831\\
-0.01451	-0.01526\\
-0.0146025	-0.027465\\
-0.01442	-0.01831\\
-0.0148325	-0.027465\\
-0.0149225	-0.0213625\\
-0.0146475	-0.027465\\
-0.0148325	-0.0122075\\
-0.01451	-0.01831\\
-0.0146025	-0.024415\\
-0.014465	-0.009155\\
-0.013915	-0.009155\\
-0.01387	-0.01526\\
-0.0136425	-0.0061025\\
-0.01355	-0.009155\\
-0.01355	-0.009155\\
-0.0131825	-0.0061025\\
-0.01323	-0.01526\\
-0.0134575	-0.01526\\
-0.013505	-0.01526\\
-0.013505	-0.01831\\
-0.0139625	-0.027465\\
-0.014465	-0.01831\\
-0.014465	-0.01831\\
-0.0142825	-0.0305175\\
-0.014785	-0.03662\\
-0.0152425	-0.024415\\
-0.0151525	-0.027465\\
-0.0151975	-0.024415\\
-0.014465	-0.0061025\\
-0.0136875	-0.024415\\
-0.0140525	-0.0213625\\
-0.0146025	-0.0305175\\
-0.0148325	-0.03357\\
-0.01529	-0.0396725\\
-0.0154725	-0.027465\\
-0.015105	-0.0213625\\
-0.0146475	-0.01831\\
-0.0145575	-0.0213625\\
-0.0146475	-0.01526\\
-0.014465	-0.0122075\\
-0.013825	-0.0122075\\
-0.0137775	-0.0122075\\
-0.0137325	-0.009155\\
-0.0134575	-0.0122075\\
-0.0131375	-0.009155\\
-0.0134125	-0.0213625\\
-0.0139625	-0.0213625\\
-0.0141	-0.0122075\\
-0.0139625	-0.01831\\
-0.014375	-0.03662\\
-0.015015	-0.01526\\
-0.014785	-0.024415\\
-0.01497	-0.042725\\
-0.0151525	-0.024415\\
-0.01497	-0.01526\\
-0.014375	-0.024415\\
-0.0142825	-0.027465\\
-0.01451	-0.0213625\\
-0.0140525	-0.009155\\
-0.013915	-0.01526\\
-0.0137325	-0.0030525\\
-0.013275	-0.0061025\\
-0.0130475	-0.01526\\
-0.0133675	-0.01526\\
-0.0137775	-0.01831\\
-0.0137325	-0.01526\\
-0.01387	-0.0213625\\
-0.0140525	-0.01831\\
-0.01387	-0.009155\\
-0.0133675	-0.0030525\\
-0.013	-0.01526\\
-0.013	-0.0122075\\
-0.0131825	-0.01831\\
-0.01332	-0.027465\\
-0.01419	-0.01831\\
-0.0146025	-0.0213625\\
-0.014465	-0.024415\\
-0.01442	-0.0305175\\
-0.014695	-0.03662\\
-0.0151525	-0.042725\\
-0.0154275	-0.042725\\
-0.015565	-0.0305175\\
-0.01529	-0.027465\\
-0.0152425	-0.0305175\\
-0.01529	-0.0305175\\
-0.01529	-0.03357\\
-0.0154725	-0.0457775\\
-0.0158375	-0.042725\\
-0.015885	-0.03357\\
-0.0157925	-0.0305175\\
-0.015565	-0.0305175\\
-0.01538	-0.0305175\\
-0.01538	-0.03357\\
-0.0155175	-0.027465\\
-0.015105	-0.027465\\
-0.0149225	-0.0305175\\
-0.0151975	-0.03357\\
-0.01538	-0.0305175\\
-0.0152425	-0.027465\\
-0.01529	-0.027465\\
-0.0151975	-0.024415\\
-0.01497	-0.024415\\
-0.0145575	-0.01831\\
-0.014695	-0.0305175\\
-0.01538	-0.0213625\\
-0.01529	-0.01526\\
-0.014785	-0.0213625\\
-0.01474	-0.0213625\\
-0.0142375	-0.009155\\
-0.0137775	-0.01831\\
-0.0139625	-0.01831\\
-0.0136425	-0.009155\\
-0.013505	-0.01526\\
-0.01355	-0.0122075\\
-0.01332	-0.0061025\\
-0.01355	-0.01526\\
-0.0134575	-0.009155\\
-0.0130925	-0.009155\\
-0.0133675	-0.0213625\\
-0.0134575	-0.009155\\
-0.0136875	-0.0213625\\
-0.0137325	-0.0122075\\
-0.0137775	-0.01831\\
-0.0140525	-0.0213625\\
-0.0141	-0.01831\\
-0.0142375	-0.03357\\
-0.01506	-0.0122075\\
-0.01506	-0.0213625\\
-0.0148325	-0.027465\\
-0.0148775	-0.027465\\
-0.0145575	-0.01526\\
-0.013915	-0.01831\\
-0.01419	-0.0305175\\
-0.01419	-0.009155\\
-0.0137325	-0.01526\\
-0.013915	-0.0213625\\
-0.0141	-0.01526\\
-0.01419	-0.024415\\
-0.01387	-0.0030525\\
-0.0133675	-0.009155\\
-0.0130925	-0.0122075\\
-0.012955	-0.0122075\\
-0.0134125	-0.01831\\
-0.01355	-0.009155\\
-0.0137325	-0.0213625\\
-0.0141	-0.027465\\
-0.0146025	-0.024415\\
-0.01451	-0.0213625\\
-0.014785	-0.03662\\
-0.015105	-0.027465\\
-0.01506	-0.027465\\
-0.01497	-0.0457775\\
-0.015655	-0.0457775\\
-0.0155175	-0.03357\\
-0.015655	-0.042725\\
-0.01561	-0.03357\\
-0.0157025	-0.042725\\
-0.01593	-0.03357\\
-0.0154275	-0.0122075\\
-0.01474	-0.01831\\
-0.0143275	-0.0305175\\
-0.01451	-0.027465\\
-0.0148775	-0.027465\\
-0.0149225	-0.0213625\\
-0.0146475	-0.0122075\\
-0.0139625	-0.01526\\
-0.01387	-0.024415\\
-0.0146025	-0.0396725\\
-0.0151975	-0.03357\\
-0.01538	-0.024415\\
-0.014785	-0.024415\\
-0.01442	-0.01831\\
-0.0146025	-0.0305175\\
-0.0152425	-0.0396725\\
-0.015565	-0.03662\\
-0.015655	-0.03357\\
-0.015335	-0.03357\\
-0.01561	-0.0488275\\
-0.0158375	-0.042725\\
-0.0158375	-0.0488275\\
-0.01625	-0.05188\\
-0.0162975	-0.0396725\\
-0.0161125	-0.0457775\\
-0.016205	-0.0396725\\
-0.0161125	-0.0213625\\
-0.0155175	-0.0213625\\
-0.0151525	-0.027465\\
-0.0152425	-0.03357\\
-0.0154725	-0.03357\\
-0.015565	-0.027465\\
-0.0152425	-0.024415\\
-0.01538	-0.03357\\
-0.01538	-0.024415\\
-0.015015	-0.0305175\\
-0.0151975	-0.024415\\
-0.0151975	-0.03662\\
-0.0155175	-0.0457775\\
-0.0157025	-0.027465\\
-0.0154275	-0.0213625\\
-0.01506	-0.01831\\
-0.014465	-0.01831\\
-0.01451	-0.0305175\\
-0.01442	-0.009155\\
-0.01419	-0.0122075\\
-0.0141	-0.03357\\
-0.0142825	-0.027465\\
-0.014695	-0.027465\\
-0.0148325	-0.024415\\
-0.01497	-0.0213625\\
-0.01451	-0.009155\\
-0.014145	-0.01526\\
-0.0137775	-0.0122075\\
-0.0137325	-0.0213625\\
-0.014145	-0.01831\\
-0.0142825	-0.01831\\
-0.0143275	-0.03357\\
-0.0148325	-0.03357\\
-0.0152425	-0.0396725\\
-0.0155175	-0.03662\\
-0.015655	-0.042725\\
-0.0158375	-0.03662\\
-0.0154725	-0.024415\\
-0.01506	-0.0213625\\
-0.014785	-0.0213625\\
-0.014695	-0.024415\\
-0.01497	-0.024415\\
-0.0149225	-0.01526\\
-0.014465	-0.0122075\\
-0.0142375	-0.027465\\
-0.0146475	-0.0305175\\
-0.015015	-0.024415\\
-0.01497	-0.03662\\
-0.0154725	-0.042725\\
-0.0158375	-0.0305175\\
-0.01561	-0.0213625\\
-0.015105	-0.01831\\
-0.0146025	-0.01831\\
-0.01442	-0.024415\\
-0.0149225	-0.0213625\\
-0.0149225	-0.024415\\
-0.0149225	-0.0396725\\
-0.015105	-0.0305175\\
-0.0152425	-0.0305175\\
-0.01529	-0.0305175\\
-0.015335	-0.027465\\
-0.015105	-0.027465\\
-0.01506	-0.027465\\
-0.014785	-0.01831\\
-0.0145575	-0.0305175\\
-0.0148775	-0.03662\\
-0.015335	-0.03662\\
-0.015565	-0.0305175\\
-0.015105	-0.027465\\
-0.0157025	-0.042725\\
-0.0160225	-0.027465\\
-0.0154725	-0.0213625\\
-0.0148325	-0.027465\\
-0.0149225	-0.01831\\
-0.0148775	-0.024415\\
-0.014785	-0.027465\\
-0.01497	-0.0213625\\
-0.0146025	-0.024415\\
-0.01474	-0.0396725\\
-0.015565	-0.0457775\\
-0.015885	-0.027465\\
-0.0155175	-0.0305175\\
-0.0154275	-0.03357\\
-0.01561	-0.03662\\
-0.01561	-0.024415\\
-0.01538	-0.024415\\
-0.0151975	-0.0213625\\
-0.0148325	-0.027465\\
-0.015015	-0.0213625\\
-0.0152425	-0.0396725\\
-0.0157025	-0.05188\\
-0.0160675	-0.0579825\\
-0.01657	-0.0549325\\
-0.01648	-0.03662\\
-0.0160675	-0.03357\\
-0.0157475	-0.0305175\\
-0.0157475	-0.0305175\\
-0.0155175	-0.0213625\\
-0.0149225	-0.01526\\
-0.014695	-0.01831\\
-0.01474	-0.01831\\
-0.01442	-0.009155\\
-0.0139625	-0.0122075\\
-0.0141	-0.027465\\
-0.014465	-0.01831\\
-0.0142375	-0.009155\\
-0.01387	-0.009155\\
-0.0136875	-0.0213625\\
-0.01442	-0.027465\\
-0.015015	-0.024415\\
-0.01442	-0.01526\\
-0.01442	-0.0305175\\
-0.014465	-0.01526\\
-0.014465	-0.01831\\
-0.0146025	-0.0213625\\
-0.0145575	-0.01526\\
-0.0142375	-0.01526\\
-0.013915	-0.0122075\\
-0.01355	-0.01526\\
-0.013595	-0.0213625\\
-0.0142375	-0.024415\\
-0.0146475	-0.0213625\\
-0.01442	-0.01526\\
-0.0140525	-0.0122075\\
-0.013595	-0.0122075\\
-0.013595	-0.0213625\\
-0.0141	-0.024415\\
-0.0145575	-0.0213625\\
-0.01474	-0.0213625\\
-0.014465	-0.01526\\
-0.0142825	-0.01831\\
-0.014375	-0.0213625\\
-0.014695	-0.03662\\
-0.0152425	-0.042725\\
-0.01561	-0.03357\\
-0.0154725	-0.01831\\
-0.0149225	-0.024415\\
-0.014785	-0.024415\\
-0.01474	-0.0213625\\
-0.014695	-0.01831\\
-0.0143275	-0.0213625\\
-0.0142825	-0.0213625\\
-0.01442	-0.01831\\
-0.01442	-0.009155\\
-0.0141	-0.009155\\
-0.01355	-0.0122075\\
-0.0136425	-0.01831\\
-0.0140525	-0.024415\\
-0.0146475	-0.0305175\\
-0.01497	-0.027465\\
-0.01529	-0.03662\\
-0.0152425	-0.03357\\
-0.015335	-0.042725\\
-0.0157925	-0.0579825\\
-0.01648	-0.042725\\
-0.01657	-0.03662\\
-0.0160675	-0.042725\\
-0.015975	-0.0457775\\
-0.01616	-0.0488275\\
-0.01657	-0.0457775\\
-0.0161125	-0.0213625\\
-0.0151525	-0.0122075\\
-0.014465	-0.01526\\
-0.0142375	-0.009155\\
-0.013825	-0.009155\\
-0.01323	-0.009155\\
-0.0130925	-0.01526\\
-0.0134575	-0.01526\\
-0.01387	-0.0213625\\
-0.0140075	-0.0122075\\
-0.013825	-0.01526\\
-0.0137325	-0.01526\\
-0.013915	-0.0213625\\
-0.01419	-0.01831\\
-0.0142825	-0.01831\\
-0.0143275	-0.01526\\
-0.0140525	-0.01526\\
-0.0136875	-0.01526\\
-0.0134575	-0.01831\\
-0.0137325	-0.024415\\
-0.0140075	-0.0122075\\
-0.01387	-0.009155\\
-0.0134575	-0.01526\\
-0.0137325	-0.0122075\\
-0.013505	-0.0213625\\
-0.013505	-0.024415\\
-0.0137325	-0.01831\\
-0.0143275	-0.024415\\
-0.0145575	-0.01831\\
-0.0142375	-0.024415\\
-0.0145575	-0.03662\\
-0.014695	-0.01831\\
-0.0145575	-0.01831\\
-0.0143275	-0.027465\\
-0.014695	-0.03662\\
-0.0151525	-0.03662\\
-0.0152425	-0.03662\\
-0.0154275	-0.027465\\
-0.01529	-0.0305175\\
-0.015105	-0.0305175\\
-0.015015	-0.03357\\
-0.01529	-0.0305175\\
-0.0151975	-0.03357\\
-0.01538	-0.0396725\\
-0.0154725	-0.03357\\
-0.0154725	-0.0549325\\
-0.0163425	-0.05188\\
-0.0163425	-0.0305175\\
-0.0154725	-0.0213625\\
-0.014785	-0.0213625\\
-0.0146475	-0.0213625\\
-0.0148775	-0.0305175\\
-0.0151525	-0.0305175\\
-0.0154275	-0.03357\\
-0.01561	-0.0396725\\
-0.0157025	-0.027465\\
-0.015335	-0.024415\\
-0.015105	-0.0305175\\
-0.0151525	-0.024415\\
-0.0152425	-0.024415\\
-0.01474	-0.009155\\
-0.0142375	-0.024415\\
-0.0145575	-0.0305175\\
-0.0146475	-0.027465\\
-0.01497	-0.03662\\
-0.015335	-0.03357\\
-0.01529	-0.0213625\\
-0.015015	-0.0213625\\
-0.014695	-0.0305175\\
-0.0148775	-0.0305175\\
-0.01529	-0.0396725\\
-0.0158375	-0.0488275\\
-0.0161125	-0.05188\\
-0.0163875	-0.0457775\\
-0.0162975	-0.042725\\
-0.0161125	-0.027465\\
-0.015565	-0.0213625\\
-0.0148775	-0.03357\\
-0.015335	-0.0488275\\
-0.0157475	-0.027465\\
-0.01529	-0.0305175\\
-0.01561	-0.0549325\\
-0.0161125	-0.0549325\\
-0.01648	-0.0488275\\
-0.0161125	-0.024415\\
-0.015335	-0.01526\\
-0.0146475	-0.024415\\
-0.0146025	-0.027465\\
-0.0146475	-0.01831\\
-0.0146025	-0.0122075\\
-0.0142375	-0.01526\\
-0.01419	-0.01526\\
-0.0139625	-0.0213625\\
-0.0140525	-0.01831\\
-0.0140075	-0.01526\\
-0.014145	-0.0213625\\
-0.014695	-0.024415\\
-0.01474	-0.0305175\\
-0.01506	-0.03662\\
-0.0155175	-0.03662\\
-0.01561	-0.024415\\
-0.01506	-0.0305175\\
-0.0148325	-0.024415\\
-0.015015	-0.0213625\\
-0.01474	-0.01831\\
-0.0145575	-0.01831\\
-0.01419	-0.0213625\\
-0.014375	-0.024415\\
-0.01442	-0.01526\\
-0.0139625	-0.009155\\
-0.0133675	-0.009155\\
-0.0130475	-0.0122075\\
-0.0134575	-0.01831\\
-0.013915	-0.01831\\
-0.014145	-0.01526\\
-0.0137325	-0.027465\\
-0.0137325	-0.0213625\\
-0.01419	-0.024415\\
-0.0146475	-0.01526\\
-0.014465	-0.0122075\\
-0.0142375	-0.01831\\
-0.01442	-0.027465\\
-0.0146025	-0.024415\\
-0.0146025	-0.024415\\
-0.014785	-0.027465\\
-0.014785	-0.01831\\
-0.01451	-0.01831\\
-0.01419	-0.0213625\\
-0.0143275	-0.01831\\
-0.0142375	-0.01831\\
-0.0143275	-0.0305175\\
-0.0146475	-0.0396725\\
-0.015335	-0.0305175\\
-0.0152425	-0.0305175\\
-0.015015	-0.0396725\\
-0.0155175	-0.03357\\
-0.015335	-0.0213625\\
-0.0148325	-0.01831\\
-0.014375	-0.0122075\\
-0.0140075	-0.0122075\\
-0.013915	-0.009155\\
-0.0137775	-0.009155\\
-0.0136425	-0.027465\\
-0.0142375	-0.01526\\
-0.0137325	-0.0122075\\
-0.01332	-0.01526\\
-0.0136425	-0.0213625\\
-0.0140075	-0.01831\\
-0.013915	-0.01526\\
-0.013595	-0.0122075\\
-0.0131825	-0.01526\\
-0.013275	-0.027465\\
-0.0140525	-0.027465\\
-0.0146025	-0.01831\\
-0.0145575	-0.0122075\\
-0.0143275	-0.024415\\
-0.014375	-0.0213625\\
-0.01442	-0.024415\\
-0.01451	-0.027465\\
-0.014785	-0.03357\\
-0.015105	-0.03357\\
-0.0151975	-0.0305175\\
-0.01506	-0.0213625\\
-0.0146475	-0.01831\\
-0.0143275	-0.01526\\
-0.0142375	-0.024415\\
-0.01451	-0.01526\\
-0.01419	-0.01526\\
-0.013915	-0.0213625\\
-0.014375	-0.0305175\\
-0.01474	-0.024415\\
-0.014785	-0.027465\\
-0.0149225	-0.0396725\\
-0.01538	-0.027465\\
-0.0148775	-0.027465\\
-0.0151975	-0.03662\\
-0.015565	-0.0305175\\
-0.0154275	-0.0396725\\
-0.0155175	-0.03357\\
-0.0155175	-0.0549325\\
-0.0163425	-0.0549325\\
-0.01657	-0.0396725\\
-0.01616	-0.0488275\\
-0.01625	-0.0549325\\
-0.01657	-0.05188\\
-0.0166175	-0.0579825\\
-0.0167075	-0.0549325\\
-0.016755	-0.0640875\\
-0.017075	-0.05188\\
-0.0169825	-0.03357\\
-0.01648	-0.024415\\
-0.0157925	-0.0305175\\
-0.0155175	-0.0213625\\
-0.01538	-0.01831\\
-0.01497	-0.0305175\\
-0.0151975	-0.03662\\
-0.0157025	-0.024415\\
-0.01538	-0.01831\\
-0.01497	-0.024415\\
-0.0149225	-0.01831\\
-0.01451	-0.0213625\\
-0.01451	-0.01526\\
-0.0137775	-0.024415\\
-0.01355	-0.0122075\\
-0.0134125	-0.0122075\\
-0.0136425	-0.0030525\\
-0.01355	-0.009155\\
-0.01332	-0.0061025\\
-0.01291	-0.009155\\
-0.0125875	-0.0122075\\
-0.01291	-0.0122075\\
-0.0127725	-0.0122075\\
-0.012725	-0.009155\\
-0.0133675	-0.0213625\\
-0.0140075	-0.024415\\
-0.0142825	-0.01526\\
-0.0141	-0.0305175\\
-0.01419	-0.03662\\
-0.0148775	-0.0213625\\
-0.01442	-0.0061025\\
-0.0136875	-0.009155\\
-0.01332	-0.0061025\\
-0.01291	-0.01526\\
-0.013	-0.01831\\
-0.01355	-0.0213625\\
-0.0142825	-0.009155\\
-0.0140525	-0.01831\\
-0.01442	-0.03662\\
-0.015105	-0.03662\\
-0.0152425	-0.024415\\
-0.015015	-0.01831\\
-0.01442	-0.0305175\\
-0.014375	-0.03357\\
-0.014785	-0.03662\\
-0.0151975	-0.01831\\
-0.01474	-0.0122075\\
-0.0139625	-0.01526\\
-0.0140075	-0.01526\\
-0.0139625	-0.009155\\
-0.01323	-0.009155\\
-0.0125875	-0.0213625\\
-0.012955	-0.03662\\
-0.014145	-0.027465\\
-0.0145575	-0.01526\\
-0.0143275	-0.009155\\
-0.0137775	-0.0122075\\
-0.0136875	-0.01831\\
-0.01268	-0.0122075\\
-0.0125875	-0.009155\\
-0.012635	-0.01526\\
-0.012955	-0.01526\\
-0.01355	-0.0305175\\
-0.0142375	-0.024415\\
-0.0146025	-0.027465\\
-0.01474	-0.0305175\\
-0.0148775	-0.027465\\
-0.0149225	-0.0305175\\
-0.0148775	-0.027465\\
-0.0148325	-0.03357\\
-0.0149225	-0.0457775\\
-0.01561	-0.0396725\\
-0.0157925	-0.01831\\
-0.01506	-0.0122075\\
-0.014145	-0.0213625\\
-0.014695	-0.03662\\
-0.0151525	-0.027465\\
-0.015105	-0.0122075\\
-0.0146025	-0.027465\\
-0.0143275	-0.024415\\
-0.01497	-0.042725\\
-0.0157025	-0.05188\\
-0.015885	-0.0488275\\
-0.0161125	-0.0488275\\
-0.016205	-0.0396725\\
-0.015885	-0.03662\\
-0.0158375	-0.01831\\
-0.015015	-0.024415\\
-0.014465	-0.0213625\\
-0.0146475	-0.027465\\
-0.0149225	-0.027465\\
-0.015015	-0.0305175\\
-0.015105	-0.01831\\
-0.01474	-0.027465\\
-0.0148325	-0.0213625\\
-0.0148325	-0.01526\\
-0.0142825	-0.009155\\
-0.0136875	-0.0122075\\
-0.01323	-0.0122075\\
-0.0134125	-0.0122075\\
-0.01332	-0.009155\\
-0.0127725	-0.01526\\
-0.0130475	-0.024415\\
-0.013915	-0.01526\\
-0.013825	-0.01526\\
-0.013595	-0.01526\\
-0.01387	-0.03357\\
-0.0146025	-0.01526\\
-0.01451	-0.009155\\
-0.0137775	-0.0030525\\
-0.0130925	-0.01526\\
-0.013595	-0.027465\\
-0.0146475	-0.03662\\
-0.015335	-0.0549325\\
-0.0160225	-0.0579825\\
-0.0166625	-0.061035\\
-0.0168	-0.042725\\
-0.0163875	-0.0457775\\
-0.016205	-0.05188\\
-0.01657	-0.0457775\\
-0.01648	-0.042725\\
-0.0163875	-0.0457775\\
-0.0164325	-0.05188\\
-0.01657	-0.05188\\
-0.0166625	-0.07019\\
-0.017075	-0.0488275\\
-0.0168	-0.0305175\\
-0.0160225	-0.01831\\
-0.015015	-0.01526\\
-0.01442	-0.01526\\
-0.0143275	-0.01526\\
-0.0143275	-0.0122075\\
-0.0142825	-0.0305175\\
-0.014785	-0.01831\\
-0.014785	-0.027465\\
-0.01474	-0.027465\\
-0.015015	-0.01831\\
-0.0146025	-0.0305175\\
-0.014465	-0.0305175\\
-0.014785	-0.0305175\\
-0.015105	-0.01526\\
-0.0146475	-0.0122075\\
-0.0140525	-0.01526\\
-0.014145	-0.0213625\\
-0.0141	-0.03357\\
-0.0151975	-0.0488275\\
-0.0160225	-0.0457775\\
-0.0160675	-0.0457775\\
-0.01625	-0.0579825\\
-0.016525	-0.0732425\\
-0.01703	-0.0549325\\
-0.01712	-0.03662\\
-0.016525	-0.024415\\
-0.0158375	-0.01526\\
-0.0149225	-0.027465\\
-0.0146025	-0.0213625\\
-0.0146475	-0.024415\\
-0.0148775	-0.01831\\
-0.0146475	-0.0213625\\
-0.01442	-0.01831\\
-0.014465	-0.024415\\
-0.0146475	-0.024415\\
-0.014785	-0.027465\\
-0.015015	-0.01831\\
-0.0149225	-0.01526\\
-0.0141	-0.009155\\
-0.01355	-0.009155\\
-0.0130925	-0.0030525\\
-0.013	-0.009155\\
-0.0130475	-0.01526\\
-0.013595	-0.01831\\
-0.0137775	-0.01831\\
-0.0140075	-0.0305175\\
-0.0146025	-0.01831\\
-0.01451	-0.0305175\\
-0.01506	-0.0457775\\
-0.01593	-0.0305175\\
-0.015885	-0.0305175\\
-0.0157025	-0.042725\\
-0.01593	-0.042725\\
-0.01616	-0.0305175\\
-0.0157925	-0.03662\\
-0.0154725	-0.0213625\\
-0.01497	-0.0305175\\
-0.01474	-0.042725\\
-0.015565	-0.061035\\
-0.0166175	-0.0732425\\
-0.0172125	-0.0579825\\
-0.017165	-0.0457775\\
-0.0169375	-0.0396725\\
-0.0163875	-0.0457775\\
-0.0162975	-0.0488275\\
-0.0166625	-0.03357\\
-0.0161125	-0.0305175\\
-0.01561	-0.0213625\\
-0.015335	-0.03662\\
-0.0158375	-0.0305175\\
-0.0160225	-0.0213625\\
-0.015335	-0.0213625\\
-0.01506	-0.01831\\
-0.014695	-0.0305175\\
-0.015015	-0.03662\\
-0.01561	-0.0396725\\
-0.0157925	-0.024415\\
-0.015565	-0.024415\\
-0.0152425	-0.0305175\\
-0.0152425	-0.01831\\
-0.01506	-0.01831\\
-0.0146025	-0.0122075\\
-0.0142375	-0.0122075\\
-0.0139625	-0.009155\\
-0.013595	-0.0122075\\
-0.01355	-0.027465\\
-0.0140075	-0.0213625\\
-0.0146025	-0.027465\\
-0.01474	-0.01831\\
-0.0143275	-0.01526\\
-0.014145	-0.009155\\
-0.013915	-0.0213625\\
-0.0140075	-0.0213625\\
-0.0143275	-0.024415\\
-0.01451	-0.01831\\
-0.01451	-0.0122075\\
-0.0140525	-0.0213625\\
-0.0146475	-0.03662\\
-0.01529	-0.0213625\\
-0.0151525	-0.0061025\\
-0.014145	-0.027465\\
-0.013915	-0.024415\\
-0.0142825	-0.024415\\
-0.014465	-0.01831\\
-0.0143275	-0.009155\\
-0.0140075	-0.027465\\
-0.0142825	-0.027465\\
-0.0149225	-0.01526\\
-0.0146025	-0.009155\\
-0.0136875	-0.027465\\
-0.0134575	-0.03357\\
-0.014375	-0.0305175\\
-0.0148775	-0.01526\\
-0.014465	-0.01831\\
-0.014145	-0.0305175\\
-0.01474	-0.03662\\
-0.015335	-0.0305175\\
-0.01538	-0.0457775\\
-0.0157475	-0.0549325\\
-0.016205	-0.03357\\
-0.015885	-0.03357\\
-0.0155175	-0.042725\\
-0.0158375	-0.027465\\
-0.015565	-0.042725\\
-0.015655	-0.042725\\
-0.0160675	-0.03357\\
-0.015885	-0.01526\\
-0.01506	-0.024415\\
-0.01538	-0.0488275\\
-0.01625	-0.0549325\\
-0.0166175	-0.0396725\\
-0.0163875	-0.0396725\\
-0.0161125	-0.0488275\\
-0.0163425	-0.03662\\
-0.0161125	-0.027465\\
-0.01561	-0.024415\\
-0.01529	-0.0305175\\
-0.0154275	-0.0305175\\
-0.0155175	-0.0305175\\
-0.0154725	-0.03357\\
-0.015565	-0.0213625\\
-0.015335	-0.027465\\
-0.0152425	-0.03357\\
-0.015565	-0.0213625\\
-0.0151975	-0.01831\\
-0.014785	-0.01526\\
-0.01451	-0.01526\\
-0.0141	-0.009155\\
-0.0140525	-0.01831\\
-0.0145575	-0.01831\\
-0.0146475	-0.0213625\\
-0.01497	-0.0305175\\
-0.0154275	-0.0396725\\
-0.0157475	-0.027465\\
-0.015655	-0.0305175\\
-0.01538	-0.0213625\\
-0.0145575	-0.03357\\
-0.0145575	-0.03662\\
-0.01529	-0.0213625\\
-0.0151525	-0.009155\\
-0.014375	-0.027465\\
-0.01474	-0.03357\\
-0.01538	-0.03357\\
-0.015565	-0.0457775\\
-0.015975	-0.061035\\
-0.01657	-0.0396725\\
-0.0164325	-0.0396725\\
-0.0161125	-0.042725\\
-0.01625	-0.0305175\\
-0.015885	-0.024415\\
-0.0154275	-0.0305175\\
-0.0154725	-0.03662\\
-0.0157475	-0.03662\\
-0.015885	-0.03357\\
-0.0158375	-0.024415\\
-0.01538	-0.0213625\\
-0.015015	-0.01526\\
-0.01474	-0.027465\\
-0.0148775	-0.01831\\
-0.0148775	-0.01831\\
-0.014695	-0.03662\\
-0.0152425	-0.0396725\\
-0.015565	-0.0213625\\
-0.015335	-0.01831\\
-0.015015	-0.0305175\\
-0.01529	-0.01831\\
-0.0148775	-0.01526\\
-0.01419	-0.01831\\
-0.0140525	-0.024415\\
-0.0143275	-0.01526\\
-0.0142825	-0.009155\\
-0.0140075	-0.009155\\
-0.013275	-0.0061025\\
-0.01332	-0.01526\\
-0.013825	-0.024415\\
-0.0146025	-0.024415\\
-0.01497	-0.03357\\
-0.0154275	-0.0396725\\
-0.0157925	-0.027465\\
-0.0152425	-0.0213625\\
-0.014695	-0.0396725\\
-0.0154275	-0.05188\\
-0.016205	-0.0396725\\
-0.0161125	-0.03662\\
-0.0160225	-0.0579825\\
-0.0164325	-0.042725\\
-0.0164325	-0.0396725\\
-0.01616	-0.03357\\
-0.0157925	-0.03357\\
-0.0157025	-0.0396725\\
-0.015975	-0.027465\\
-0.0157025	-0.024415\\
-0.01538	-0.042725\\
-0.015885	-0.0457775\\
-0.0162975	-0.0488275\\
-0.0164325	-0.0305175\\
-0.0154725	-0.0122075\\
-0.014465	-0.01831\\
-0.015105	-0.027465\\
-0.015105	-0.0061025\\
-0.0142375	-0.009155\\
-0.0137325	-0.0213625\\
-0.01387	-0.0213625\\
-0.014465	-0.03357\\
-0.0151975	-0.0213625\\
-0.015105	-0.01526\\
-0.014695	-0.01526\\
-0.0142825	-0.009155\\
-0.0136425	-0.01526\\
-0.013595	-0.009155\\
-0.01355	-0.0122075\\
-0.0136425	-0.0213625\\
-0.0142375	-0.01526\\
-0.0143275	-0.0030525\\
-0.0137325	-0.0122075\\
-0.0134125	-0.01526\\
-0.01355	-0.01526\\
-0.0137775	-0.009155\\
-0.0134125	-0.009155\\
-0.0136875	-0.009155\\
-0.01355	-0.009155\\
-0.0131825	-0.0213625\\
-0.0137325	-0.027465\\
-0.014375	-0.03662\\
-0.01506	-0.0488275\\
-0.0158375	-0.0396725\\
-0.015975	-0.01831\\
-0.0151975	-0.01831\\
-0.01451	-0.01526\\
-0.014375	-0.0122075\\
-0.013825	-0.009155\\
-0.0134125	-0.01831\\
-0.013915	-0.0213625\\
-0.01419	-0.01526\\
-0.014145	-0.0213625\\
-0.0142375	-0.0213625\\
-0.014465	-0.024415\\
-0.0146025	-0.024415\\
-0.014695	-0.027465\\
-0.01497	-0.0305175\\
-0.01506	-0.03662\\
-0.0157025	-0.024415\\
-0.0151525	-0.01526\\
-0.01419	-0.01526\\
-0.013825	-0.024415\\
-0.0143275	-0.0122075\\
-0.0143275	-0.01831\\
-0.0141	-0.0061025\\
-0.0134575	-0.01526\\
-0.0134125	-0.01526\\
-0.01323	-0.009155\\
-0.01268	-0.009155\\
-0.01268	-0.01526\\
-0.01323	-0.01831\\
-0.0136425	-0.027465\\
-0.0142375	-0.03357\\
-0.0149225	-0.0213625\\
-0.0148775	-0.0061025\\
-0.013825	-0.01526\\
-0.014145	-0.024415\\
-0.01442	-0.0122075\\
-0.01387	-0.0030525\\
-0.01387	-0.0305175\\
-0.0146025	-0.01831\\
-0.014785	-0.03662\\
-0.01538	-0.0457775\\
-0.01593	-0.03357\\
-0.0154725	-0.0213625\\
-0.0151525	-0.0213625\\
-0.01497	-0.027465\\
-0.015105	-0.03662\\
-0.01561	-0.024415\\
-0.015335	-0.0213625\\
-0.0148775	-0.024415\\
};
\addplot [color=mycolor2, line width=2.0pt, forget plot]
  table[row sep=crcr]{%
-0.015655	-0.015655\\
-0.0158375	-0.0158375\\
-0.015655	-0.015655\\
-0.01497	-0.01497\\
-0.0148325	-0.0148325\\
-0.0148775	-0.0148775\\
-0.01497	-0.01497\\
-0.0148325	-0.0148325\\
-0.0139625	-0.0139625\\
-0.0131825	-0.0131825\\
-0.0128175	-0.0128175\\
-0.0137325	-0.0137325\\
-0.014785	-0.014785\\
-0.01506	-0.01506\\
-0.0151525	-0.0151525\\
-0.0149225	-0.0149225\\
-0.0143275	-0.0143275\\
-0.0148775	-0.0148775\\
-0.01442	-0.01442\\
-0.0149225	-0.0149225\\
-0.01474	-0.01474\\
-0.0151525	-0.0151525\\
-0.0152425	-0.0152425\\
-0.01497	-0.01497\\
-0.01529	-0.01529\\
-0.01538	-0.01538\\
-0.0151975	-0.0151975\\
-0.01497	-0.01497\\
-0.014695	-0.014695\\
-0.0148775	-0.0148775\\
-0.0149225	-0.0149225\\
-0.01497	-0.01497\\
-0.015105	-0.015105\\
-0.0154725	-0.0154725\\
-0.0155175	-0.0155175\\
-0.015105	-0.015105\\
-0.0149225	-0.0149225\\
-0.0143275	-0.0143275\\
-0.014145	-0.014145\\
-0.0143275	-0.0143275\\
-0.0142375	-0.0142375\\
-0.014145	-0.014145\\
-0.01442	-0.01442\\
-0.0146475	-0.0146475\\
-0.0151525	-0.0151525\\
-0.015105	-0.015105\\
-0.014465	-0.014465\\
-0.0140525	-0.0140525\\
-0.0142375	-0.0142375\\
-0.014375	-0.014375\\
-0.0139625	-0.0139625\\
-0.0137325	-0.0137325\\
-0.0140075	-0.0140075\\
-0.01387	-0.01387\\
-0.0137325	-0.0137325\\
-0.01387	-0.01387\\
-0.0140075	-0.0140075\\
-0.014465	-0.014465\\
-0.014695	-0.014695\\
-0.0148775	-0.0148775\\
-0.01497	-0.01497\\
-0.015105	-0.015105\\
-0.01497	-0.01497\\
-0.0148775	-0.0148775\\
-0.014785	-0.014785\\
-0.01474	-0.01474\\
-0.014695	-0.014695\\
-0.0151525	-0.0151525\\
-0.01593	-0.01593\\
-0.016205	-0.016205\\
-0.01625	-0.01625\\
-0.0157925	-0.0157925\\
-0.0160675	-0.0160675\\
-0.01648	-0.01648\\
-0.0166175	-0.0166175\\
-0.0163875	-0.0163875\\
-0.0166175	-0.0166175\\
-0.017165	-0.017165\\
-0.0169825	-0.0169825\\
-0.0166175	-0.0166175\\
-0.01616	-0.01616\\
-0.015655	-0.015655\\
-0.01538	-0.01538\\
-0.0151525	-0.0151525\\
-0.0145575	-0.0145575\\
-0.0142375	-0.0142375\\
-0.0143275	-0.0143275\\
-0.014695	-0.014695\\
-0.014785	-0.014785\\
-0.015015	-0.015015\\
-0.0151975	-0.0151975\\
-0.01561	-0.01561\\
-0.0157025	-0.0157025\\
-0.015565	-0.015565\\
-0.01529	-0.01529\\
-0.01538	-0.01538\\
-0.0151525	-0.0151525\\
-0.0149225	-0.0149225\\
-0.015015	-0.015015\\
-0.01506	-0.01506\\
-0.01474	-0.01474\\
-0.014695	-0.014695\\
-0.01442	-0.01442\\
-0.014375	-0.014375\\
-0.01442	-0.01442\\
-0.014145	-0.014145\\
-0.01442	-0.01442\\
-0.0148325	-0.0148325\\
-0.015105	-0.015105\\
-0.01529	-0.01529\\
-0.0148325	-0.0148325\\
-0.01497	-0.01497\\
-0.01561	-0.01561\\
-0.0155175	-0.0155175\\
-0.01561	-0.01561\\
-0.0157025	-0.0157025\\
-0.0151525	-0.0151525\\
-0.0146475	-0.0146475\\
-0.014785	-0.014785\\
-0.0149225	-0.0149225\\
-0.0155175	-0.0155175\\
-0.0160225	-0.0160225\\
-0.0161125	-0.0161125\\
-0.0158375	-0.0158375\\
-0.0157475	-0.0157475\\
-0.01561	-0.01561\\
-0.0152425	-0.0152425\\
-0.01497	-0.01497\\
-0.014785	-0.014785\\
-0.0146025	-0.0146025\\
-0.0146475	-0.0146475\\
-0.015015	-0.015015\\
-0.01561	-0.01561\\
-0.015565	-0.015565\\
-0.015105	-0.015105\\
-0.0148325	-0.0148325\\
-0.01474	-0.01474\\
-0.0142375	-0.0142375\\
-0.0136875	-0.0136875\\
-0.013595	-0.013595\\
-0.0141	-0.0141\\
-0.0140525	-0.0140525\\
-0.0140075	-0.0140075\\
-0.014145	-0.014145\\
-0.0140525	-0.0140525\\
-0.013825	-0.013825\\
-0.0134125	-0.0134125\\
-0.0130925	-0.0130925\\
-0.013275	-0.013275\\
-0.01419	-0.01419\\
-0.01497	-0.01497\\
-0.01538	-0.01538\\
-0.015015	-0.015015\\
-0.0146025	-0.0146025\\
-0.014145	-0.014145\\
-0.013595	-0.013595\\
-0.0140525	-0.0140525\\
-0.0142825	-0.0142825\\
-0.0146025	-0.0146025\\
-0.01506	-0.01506\\
-0.01593	-0.01593\\
-0.016525	-0.016525\\
-0.0164325	-0.0164325\\
-0.0163875	-0.0163875\\
-0.015975	-0.015975\\
-0.015885	-0.015885\\
-0.015975	-0.015975\\
-0.0160225	-0.0160225\\
-0.0158375	-0.0158375\\
-0.01561	-0.01561\\
-0.015565	-0.015565\\
-0.0155175	-0.0155175\\
-0.015565	-0.015565\\
-0.0152425	-0.0152425\\
-0.01561	-0.01561\\
-0.0160675	-0.0160675\\
-0.0157925	-0.0157925\\
-0.0152425	-0.0152425\\
-0.015015	-0.015015\\
-0.015565	-0.015565\\
-0.0160225	-0.0160225\\
-0.0164325	-0.0164325\\
-0.0166175	-0.0166175\\
-0.0166625	-0.0166625\\
-0.0164325	-0.0164325\\
-0.016205	-0.016205\\
-0.0163425	-0.0163425\\
-0.01657	-0.01657\\
-0.0161125	-0.0161125\\
-0.0154725	-0.0154725\\
-0.0155175	-0.0155175\\
-0.0149225	-0.0149225\\
-0.01497	-0.01497\\
-0.0151525	-0.0151525\\
-0.01497	-0.01497\\
-0.0146475	-0.0146475\\
-0.014785	-0.014785\\
-0.014695	-0.014695\\
-0.0148775	-0.0148775\\
-0.01529	-0.01529\\
-0.015335	-0.015335\\
-0.015015	-0.015015\\
-0.0149225	-0.0149225\\
-0.01529	-0.01529\\
-0.0160225	-0.0160225\\
-0.01593	-0.01593\\
-0.015335	-0.015335\\
-0.0148775	-0.0148775\\
-0.0149225	-0.0149225\\
-0.0148775	-0.0148775\\
-0.01451	-0.01451\\
-0.014465	-0.014465\\
-0.0146475	-0.0146475\\
-0.0149225	-0.0149225\\
-0.01506	-0.01506\\
-0.01474	-0.01474\\
-0.0152425	-0.0152425\\
-0.01561	-0.01561\\
-0.015565	-0.015565\\
-0.01538	-0.01538\\
-0.015655	-0.015655\\
-0.015335	-0.015335\\
-0.01442	-0.01442\\
-0.0143275	-0.0143275\\
-0.01442	-0.01442\\
-0.01474	-0.01474\\
-0.014465	-0.014465\\
-0.01474	-0.01474\\
-0.0148775	-0.0148775\\
-0.0145575	-0.0145575\\
-0.0146475	-0.0146475\\
-0.0146025	-0.0146025\\
-0.014465	-0.014465\\
-0.0145575	-0.0145575\\
-0.01419	-0.01419\\
-0.014145	-0.014145\\
-0.013915	-0.013915\\
-0.013825	-0.013825\\
-0.014375	-0.014375\\
-0.014695	-0.014695\\
-0.015105	-0.015105\\
-0.015655	-0.015655\\
-0.01538	-0.01538\\
-0.01474	-0.01474\\
-0.0142825	-0.0142825\\
-0.014145	-0.014145\\
-0.014375	-0.014375\\
-0.0140525	-0.0140525\\
-0.013915	-0.013915\\
-0.01419	-0.01419\\
-0.014785	-0.014785\\
-0.015015	-0.015015\\
-0.015335	-0.015335\\
-0.015105	-0.015105\\
-0.0148775	-0.0148775\\
-0.0141	-0.0141\\
-0.013825	-0.013825\\
-0.0134125	-0.0134125\\
-0.013275	-0.013275\\
-0.0134125	-0.0134125\\
-0.013915	-0.013915\\
-0.0142825	-0.0142825\\
-0.01442	-0.01442\\
-0.0146025	-0.0146025\\
-0.0151975	-0.0151975\\
-0.01538	-0.01538\\
-0.015105	-0.015105\\
-0.0151525	-0.0151525\\
-0.01529	-0.01529\\
-0.015565	-0.015565\\
-0.01561	-0.01561\\
-0.01506	-0.01506\\
-0.0142825	-0.0142825\\
-0.0136425	-0.0136425\\
-0.0134125	-0.0134125\\
-0.013505	-0.013505\\
-0.01355	-0.01355\\
-0.013505	-0.013505\\
-0.0136875	-0.0136875\\
-0.0142375	-0.0142375\\
-0.0143275	-0.0143275\\
-0.014375	-0.014375\\
-0.01451	-0.01451\\
-0.014145	-0.014145\\
-0.0134575	-0.0134575\\
-0.013825	-0.013825\\
-0.01442	-0.01442\\
-0.0146025	-0.0146025\\
-0.01474	-0.01474\\
-0.0146475	-0.0146475\\
-0.014375	-0.014375\\
-0.0145575	-0.0145575\\
-0.01506	-0.01506\\
-0.0151975	-0.0151975\\
-0.0149225	-0.0149225\\
-0.01538	-0.01538\\
-0.0151975	-0.0151975\\
-0.01538	-0.01538\\
-0.0151975	-0.0151975\\
-0.01497	-0.01497\\
-0.0155175	-0.0155175\\
-0.0161125	-0.0161125\\
-0.015975	-0.015975\\
-0.015335	-0.015335\\
-0.01506	-0.01506\\
-0.0152425	-0.0152425\\
-0.0154275	-0.0154275\\
-0.0152425	-0.0152425\\
-0.0151525	-0.0151525\\
-0.0154725	-0.0154725\\
-0.015565	-0.015565\\
-0.01538	-0.01538\\
-0.0149225	-0.0149225\\
-0.015105	-0.015105\\
-0.0157025	-0.0157025\\
-0.0158375	-0.0158375\\
-0.0157925	-0.0157925\\
-0.015335	-0.015335\\
-0.01497	-0.01497\\
-0.01442	-0.01442\\
-0.0137325	-0.0137325\\
-0.01332	-0.01332\\
-0.013	-0.013\\
-0.0134575	-0.0134575\\
-0.014145	-0.014145\\
-0.01451	-0.01451\\
-0.01442	-0.01442\\
-0.014375	-0.014375\\
-0.0145575	-0.0145575\\
-0.014145	-0.014145\\
-0.0140075	-0.0140075\\
-0.0137775	-0.0137775\\
-0.01387	-0.01387\\
-0.014375	-0.014375\\
-0.0148325	-0.0148325\\
-0.01451	-0.01451\\
-0.014145	-0.014145\\
-0.013915	-0.013915\\
-0.0136875	-0.0136875\\
-0.0142375	-0.0142375\\
-0.0152425	-0.0152425\\
-0.0155175	-0.0155175\\
-0.0158375	-0.0158375\\
-0.0160225	-0.0160225\\
-0.0157925	-0.0157925\\
-0.01538	-0.01538\\
-0.0151975	-0.0151975\\
-0.01529	-0.01529\\
-0.0154725	-0.0154725\\
-0.015565	-0.015565\\
-0.01593	-0.01593\\
-0.01625	-0.01625\\
-0.0166175	-0.0166175\\
-0.0163425	-0.0163425\\
-0.01625	-0.01625\\
-0.0164325	-0.0164325\\
-0.01625	-0.01625\\
-0.0162975	-0.0162975\\
-0.01625	-0.01625\\
-0.01561	-0.01561\\
-0.014785	-0.014785\\
-0.0145575	-0.0145575\\
-0.0148775	-0.0148775\\
-0.014785	-0.014785\\
-0.014375	-0.014375\\
-0.01451	-0.01451\\
-0.0143275	-0.0143275\\
-0.013915	-0.013915\\
-0.0139625	-0.0139625\\
-0.0140525	-0.0140525\\
-0.01387	-0.01387\\
-0.0142375	-0.0142375\\
-0.014695	-0.014695\\
-0.0145575	-0.0145575\\
-0.01506	-0.01506\\
-0.0157925	-0.0157925\\
-0.01593	-0.01593\\
-0.01625	-0.01625\\
-0.015975	-0.015975\\
-0.01593	-0.01593\\
-0.015565	-0.015565\\
-0.0148775	-0.0148775\\
-0.014695	-0.014695\\
-0.0143275	-0.0143275\\
-0.014465	-0.014465\\
-0.0140525	-0.0140525\\
-0.0134125	-0.0134125\\
-0.0133675	-0.0133675\\
-0.0140075	-0.0140075\\
-0.0145575	-0.0145575\\
-0.0149225	-0.0149225\\
-0.015105	-0.015105\\
-0.0151975	-0.0151975\\
-0.0151525	-0.0151525\\
-0.0148775	-0.0148775\\
-0.0148325	-0.0148325\\
-0.014695	-0.014695\\
-0.0149225	-0.0149225\\
-0.01474	-0.01474\\
-0.01442	-0.01442\\
-0.014695	-0.014695\\
-0.015105	-0.015105\\
-0.0149225	-0.0149225\\
-0.0146475	-0.0146475\\
-0.01451	-0.01451\\
-0.0142375	-0.0142375\\
-0.0139625	-0.0139625\\
-0.0142375	-0.0142375\\
-0.01451	-0.01451\\
-0.014695	-0.014695\\
-0.0145575	-0.0145575\\
-0.014695	-0.014695\\
-0.01474	-0.01474\\
-0.014375	-0.014375\\
-0.014145	-0.014145\\
-0.014785	-0.014785\\
-0.015335	-0.015335\\
-0.0154725	-0.0154725\\
-0.015335	-0.015335\\
-0.0152425	-0.0152425\\
-0.015015	-0.015015\\
-0.0148775	-0.0148775\\
-0.0146475	-0.0146475\\
-0.0148325	-0.0148325\\
-0.014785	-0.014785\\
-0.014375	-0.014375\\
-0.0145575	-0.0145575\\
-0.0148325	-0.0148325\\
-0.01497	-0.01497\\
-0.0151975	-0.0151975\\
-0.01561	-0.01561\\
-0.0161125	-0.0161125\\
-0.0163875	-0.0163875\\
-0.0160225	-0.0160225\\
-0.0152425	-0.0152425\\
-0.0146025	-0.0146025\\
-0.0139625	-0.0139625\\
-0.0140075	-0.0140075\\
-0.0140525	-0.0140525\\
-0.0145575	-0.0145575\\
-0.0146475	-0.0146475\\
-0.01451	-0.01451\\
-0.014695	-0.014695\\
-0.0149225	-0.0149225\\
-0.0152425	-0.0152425\\
-0.01538	-0.01538\\
-0.015975	-0.015975\\
-0.0160675	-0.0160675\\
-0.0157925	-0.0157925\\
-0.01529	-0.01529\\
-0.0151525	-0.0151525\\
-0.01538	-0.01538\\
-0.015105	-0.015105\\
-0.01497	-0.01497\\
-0.014465	-0.014465\\
-0.0146475	-0.0146475\\
-0.0151975	-0.0151975\\
-0.015015	-0.015015\\
-0.0148775	-0.0148775\\
-0.01561	-0.01561\\
-0.0154275	-0.0154275\\
-0.0157925	-0.0157925\\
-0.015885	-0.015885\\
-0.0154275	-0.0154275\\
-0.014785	-0.014785\\
-0.0146025	-0.0146025\\
-0.01506	-0.01506\\
-0.0158375	-0.0158375\\
-0.0160675	-0.0160675\\
-0.01616	-0.01616\\
-0.015975	-0.015975\\
-0.0154275	-0.0154275\\
-0.015105	-0.015105\\
-0.014695	-0.014695\\
-0.014465	-0.014465\\
-0.0142825	-0.0142825\\
-0.01442	-0.01442\\
-0.0146475	-0.0146475\\
-0.01419	-0.01419\\
-0.014375	-0.014375\\
-0.014785	-0.014785\\
-0.01497	-0.01497\\
-0.0154275	-0.0154275\\
-0.0158375	-0.0158375\\
-0.0162975	-0.0162975\\
-0.015975	-0.015975\\
-0.0157475	-0.0157475\\
-0.01561	-0.01561\\
-0.0151975	-0.0151975\\
-0.0151525	-0.0151525\\
-0.0157475	-0.0157475\\
-0.0161125	-0.0161125\\
-0.01561	-0.01561\\
-0.0148775	-0.0148775\\
-0.0142375	-0.0142375\\
-0.0133675	-0.0133675\\
-0.0130475	-0.0130475\\
-0.0134575	-0.0134575\\
-0.0137775	-0.0137775\\
-0.0141	-0.0141\\
-0.01387	-0.01387\\
-0.0137775	-0.0137775\\
-0.0137325	-0.0137325\\
-0.0136875	-0.0136875\\
-0.0137325	-0.0137325\\
-0.014145	-0.014145\\
-0.01474	-0.01474\\
-0.01506	-0.01506\\
-0.015105	-0.015105\\
-0.01529	-0.01529\\
-0.0157025	-0.0157025\\
-0.015885	-0.015885\\
-0.016205	-0.016205\\
-0.01625	-0.01625\\
-0.01593	-0.01593\\
-0.0155175	-0.0155175\\
-0.0152425	-0.0152425\\
-0.0157475	-0.0157475\\
-0.0160675	-0.0160675\\
-0.0157925	-0.0157925\\
-0.01529	-0.01529\\
-0.014375	-0.014375\\
-0.0136425	-0.0136425\\
-0.013505	-0.013505\\
-0.013	-0.013\\
-0.0133675	-0.0133675\\
-0.013825	-0.013825\\
-0.0139625	-0.0139625\\
-0.013505	-0.013505\\
-0.0131375	-0.0131375\\
-0.013275	-0.013275\\
-0.01332	-0.01332\\
-0.0134575	-0.0134575\\
-0.01332	-0.01332\\
-0.0130925	-0.0130925\\
-0.0134125	-0.0134125\\
-0.01451	-0.01451\\
-0.0151975	-0.0151975\\
-0.0155175	-0.0155175\\
-0.01529	-0.01529\\
-0.015565	-0.015565\\
-0.01625	-0.01625\\
-0.0164325	-0.0164325\\
-0.01616	-0.01616\\
-0.0161125	-0.0161125\\
-0.0157025	-0.0157025\\
-0.0154725	-0.0154725\\
-0.0149225	-0.0149225\\
-0.01451	-0.01451\\
-0.015105	-0.015105\\
-0.0149225	-0.0149225\\
-0.0148775	-0.0148775\\
-0.015015	-0.015015\\
-0.01497	-0.01497\\
-0.0149225	-0.0149225\\
-0.01506	-0.01506\\
-0.0151975	-0.0151975\\
-0.0151525	-0.0151525\\
-0.0148775	-0.0148775\\
-0.0149225	-0.0149225\\
-0.014375	-0.014375\\
-0.01332	-0.01332\\
-0.01323	-0.01323\\
-0.0136875	-0.0136875\\
-0.01332	-0.01332\\
-0.01245	-0.01245\\
-0.01268	-0.01268\\
-0.013275	-0.013275\\
-0.0136875	-0.0136875\\
-0.0140075	-0.0140075\\
-0.01451	-0.01451\\
-0.014695	-0.014695\\
-0.0142825	-0.0142825\\
-0.01451	-0.01451\\
-0.0140525	-0.0140525\\
-0.013595	-0.013595\\
-0.013275	-0.013275\\
-0.0127725	-0.0127725\\
-0.013	-0.013\\
-0.012955	-0.012955\\
-0.013275	-0.013275\\
-0.0139625	-0.0139625\\
-0.014145	-0.014145\\
-0.013915	-0.013915\\
-0.01387	-0.01387\\
-0.0136875	-0.0136875\\
-0.013825	-0.013825\\
-0.0136425	-0.0136425\\
-0.01355	-0.01355\\
-0.0134575	-0.0134575\\
-0.0131825	-0.0131825\\
-0.01332	-0.01332\\
-0.013275	-0.013275\\
-0.0137325	-0.0137325\\
-0.01387	-0.01387\\
-0.013595	-0.013595\\
-0.01355	-0.01355\\
-0.01332	-0.01332\\
-0.0142375	-0.0142375\\
-0.0148775	-0.0148775\\
-0.0146475	-0.0146475\\
-0.01451	-0.01451\\
-0.014145	-0.014145\\
-0.0145575	-0.0145575\\
-0.014695	-0.014695\\
-0.0142375	-0.0142375\\
-0.0142825	-0.0142825\\
-0.014145	-0.014145\\
-0.014375	-0.014375\\
-0.0140075	-0.0140075\\
-0.0137775	-0.0137775\\
-0.0139625	-0.0139625\\
-0.01419	-0.01419\\
-0.0140075	-0.0140075\\
-0.0140525	-0.0140525\\
-0.014465	-0.014465\\
-0.01451	-0.01451\\
-0.0149225	-0.0149225\\
-0.015565	-0.015565\\
-0.0155175	-0.0155175\\
-0.01497	-0.01497\\
-0.0145575	-0.0145575\\
-0.0148775	-0.0148775\\
-0.0152425	-0.0152425\\
-0.01506	-0.01506\\
-0.0151525	-0.0151525\\
-0.014785	-0.014785\\
-0.013825	-0.013825\\
-0.013595	-0.013595\\
-0.0142375	-0.0142375\\
-0.014465	-0.014465\\
-0.0143275	-0.0143275\\
-0.014785	-0.014785\\
-0.0149225	-0.0149225\\
-0.014785	-0.014785\\
-0.01451	-0.01451\\
-0.0145575	-0.0145575\\
-0.0148775	-0.0148775\\
-0.014785	-0.014785\\
-0.01474	-0.01474\\
-0.014695	-0.014695\\
-0.01442	-0.01442\\
-0.014375	-0.014375\\
-0.01419	-0.01419\\
-0.0141	-0.0141\\
-0.0146475	-0.0146475\\
-0.0149225	-0.0149225\\
-0.01506	-0.01506\\
-0.015015	-0.015015\\
-0.014695	-0.014695\\
-0.0142825	-0.0142825\\
-0.014465	-0.014465\\
-0.014695	-0.014695\\
-0.01506	-0.01506\\
-0.01529	-0.01529\\
-0.0151525	-0.0151525\\
-0.0145575	-0.0145575\\
-0.01497	-0.01497\\
-0.015565	-0.015565\\
-0.015335	-0.015335\\
-0.0151975	-0.0151975\\
-0.01538	-0.01538\\
-0.0151975	-0.0151975\\
-0.01497	-0.01497\\
-0.0149225	-0.0149225\\
-0.015015	-0.015015\\
-0.0152425	-0.0152425\\
-0.01506	-0.01506\\
-0.015105	-0.015105\\
-0.01506	-0.01506\\
-0.0149225	-0.0149225\\
-0.015105	-0.015105\\
-0.0148775	-0.0148775\\
-0.01474	-0.01474\\
-0.0148775	-0.0148775\\
-0.0142375	-0.0142375\\
-0.0134575	-0.0134575\\
-0.01291	-0.01291\\
-0.0130925	-0.0130925\\
-0.0142375	-0.0142375\\
-0.0149225	-0.0149225\\
-0.015015	-0.015015\\
-0.01474	-0.01474\\
-0.014695	-0.014695\\
-0.01451	-0.01451\\
-0.0140525	-0.0140525\\
-0.0141	-0.0141\\
-0.014145	-0.014145\\
-0.0142375	-0.0142375\\
-0.01419	-0.01419\\
-0.0141	-0.0141\\
-0.0137325	-0.0137325\\
-0.0137775	-0.0137775\\
-0.01451	-0.01451\\
-0.01442	-0.01442\\
-0.014465	-0.014465\\
-0.014695	-0.014695\\
-0.0148325	-0.0148325\\
-0.01506	-0.01506\\
-0.01497	-0.01497\\
-0.015015	-0.015015\\
-0.01506	-0.01506\\
-0.014465	-0.014465\\
-0.01474	-0.01474\\
-0.0146025	-0.0146025\\
-0.014785	-0.014785\\
-0.01506	-0.01506\\
-0.0149225	-0.0149225\\
-0.015015	-0.015015\\
-0.01538	-0.01538\\
-0.01529	-0.01529\\
-0.01506	-0.01506\\
-0.0148775	-0.0148775\\
-0.0149225	-0.0149225\\
-0.0148775	-0.0148775\\
-0.0146475	-0.0146475\\
-0.014465	-0.014465\\
-0.01506	-0.01506\\
-0.01561	-0.01561\\
-0.015565	-0.015565\\
-0.0155175	-0.0155175\\
-0.015015	-0.015015\\
-0.0146025	-0.0146025\\
-0.014375	-0.014375\\
-0.0141	-0.0141\\
-0.014145	-0.014145\\
-0.0146475	-0.0146475\\
-0.0149225	-0.0149225\\
-0.0151525	-0.0151525\\
-0.01506	-0.01506\\
-0.0146475	-0.0146475\\
-0.0151525	-0.0151525\\
-0.0155175	-0.0155175\\
-0.0157475	-0.0157475\\
-0.015565	-0.015565\\
-0.0161125	-0.0161125\\
-0.0160675	-0.0160675\\
-0.015335	-0.015335\\
-0.014785	-0.014785\\
-0.014465	-0.014465\\
-0.0148325	-0.0148325\\
-0.0152425	-0.0152425\\
-0.015655	-0.015655\\
-0.0155175	-0.0155175\\
-0.0152425	-0.0152425\\
-0.01451	-0.01451\\
-0.0142375	-0.0142375\\
-0.0143275	-0.0143275\\
-0.01442	-0.01442\\
-0.0141	-0.0141\\
-0.0140075	-0.0140075\\
-0.013505	-0.013505\\
-0.013915	-0.013915\\
-0.01419	-0.01419\\
-0.01451	-0.01451\\
-0.0145575	-0.0145575\\
-0.0146025	-0.0146025\\
-0.0148325	-0.0148325\\
-0.0146025	-0.0146025\\
-0.01474	-0.01474\\
-0.015105	-0.015105\\
-0.0149225	-0.0149225\\
-0.0148775	-0.0148775\\
-0.01529	-0.01529\\
-0.0151525	-0.0151525\\
-0.0149225	-0.0149225\\
-0.01497	-0.01497\\
-0.0149225	-0.0149225\\
-0.0145575	-0.0145575\\
-0.0148325	-0.0148325\\
-0.015335	-0.015335\\
-0.0154725	-0.0154725\\
-0.0161125	-0.0161125\\
-0.015885	-0.015885\\
-0.015565	-0.015565\\
-0.0152425	-0.0152425\\
-0.01538	-0.01538\\
-0.0158375	-0.0158375\\
-0.0155175	-0.0155175\\
-0.01529	-0.01529\\
-0.0155175	-0.0155175\\
-0.0151975	-0.0151975\\
-0.01451	-0.01451\\
-0.014465	-0.014465\\
-0.014375	-0.014375\\
-0.0140075	-0.0140075\\
-0.013915	-0.013915\\
-0.013825	-0.013825\\
-0.0136875	-0.0136875\\
-0.01387	-0.01387\\
-0.01419	-0.01419\\
-0.014465	-0.014465\\
-0.0148325	-0.0148325\\
-0.01497	-0.01497\\
-0.0151525	-0.0151525\\
-0.0154275	-0.0154275\\
-0.0151525	-0.0151525\\
-0.01506	-0.01506\\
-0.015015	-0.015015\\
-0.015335	-0.015335\\
-0.0151975	-0.0151975\\
-0.0151525	-0.0151525\\
-0.01506	-0.01506\\
-0.0155175	-0.0155175\\
-0.0154725	-0.0154725\\
-0.0154275	-0.0154275\\
-0.015335	-0.015335\\
-0.0148775	-0.0148775\\
-0.0143275	-0.0143275\\
-0.013825	-0.013825\\
-0.014145	-0.014145\\
-0.01419	-0.01419\\
-0.01442	-0.01442\\
-0.0143275	-0.0143275\\
-0.0145575	-0.0145575\\
-0.01529	-0.01529\\
-0.0157925	-0.0157925\\
-0.015655	-0.015655\\
-0.01561	-0.01561\\
-0.015565	-0.015565\\
-0.0151525	-0.0151525\\
-0.015105	-0.015105\\
-0.0151975	-0.0151975\\
-0.0151525	-0.0151525\\
-0.0154725	-0.0154725\\
-0.015975	-0.015975\\
-0.0160675	-0.0160675\\
-0.0160225	-0.0160225\\
-0.0161125	-0.0161125\\
-0.015655	-0.015655\\
-0.01561	-0.01561\\
-0.0154275	-0.0154275\\
-0.015335	-0.015335\\
-0.01561	-0.01561\\
-0.0157925	-0.0157925\\
-0.015015	-0.015015\\
-0.01474	-0.01474\\
-0.0146475	-0.0146475\\
-0.0148775	-0.0148775\\
-0.01497	-0.01497\\
-0.0146475	-0.0146475\\
-0.014695	-0.014695\\
-0.01529	-0.01529\\
-0.01625	-0.01625\\
-0.016525	-0.016525\\
-0.0164325	-0.0164325\\
-0.016205	-0.016205\\
-0.0162975	-0.0162975\\
-0.016525	-0.016525\\
-0.01625	-0.01625\\
-0.01616	-0.01616\\
-0.01625	-0.01625\\
-0.0158375	-0.0158375\\
-0.0155175	-0.0155175\\
-0.0157025	-0.0157025\\
-0.01538	-0.01538\\
-0.0149225	-0.0149225\\
-0.01506	-0.01506\\
-0.0148325	-0.0148325\\
-0.014785	-0.014785\\
-0.0149225	-0.0149225\\
-0.015335	-0.015335\\
-0.0151975	-0.0151975\\
-0.01497	-0.01497\\
-0.0151525	-0.0151525\\
-0.015335	-0.015335\\
-0.0152425	-0.0152425\\
-0.015015	-0.015015\\
-0.01538	-0.01538\\
-0.0154725	-0.0154725\\
-0.0151525	-0.0151525\\
-0.01561	-0.01561\\
-0.015655	-0.015655\\
-0.015335	-0.015335\\
-0.0149225	-0.0149225\\
-0.014375	-0.014375\\
-0.0139625	-0.0139625\\
-0.01442	-0.01442\\
-0.0145575	-0.0145575\\
-0.0142825	-0.0142825\\
-0.01387	-0.01387\\
-0.0142825	-0.0142825\\
-0.0146025	-0.0146025\\
-0.0149225	-0.0149225\\
-0.01529	-0.01529\\
-0.015335	-0.015335\\
-0.0154275	-0.0154275\\
-0.015015	-0.015015\\
-0.0140075	-0.0140075\\
-0.0139625	-0.0139625\\
-0.013915	-0.013915\\
-0.0140075	-0.0140075\\
-0.0146025	-0.0146025\\
-0.0148775	-0.0148775\\
-0.0154275	-0.0154275\\
-0.0152425	-0.0152425\\
-0.01497	-0.01497\\
-0.0154725	-0.0154725\\
-0.01529	-0.01529\\
-0.0148775	-0.0148775\\
-0.014465	-0.014465\\
-0.0145575	-0.0145575\\
-0.0149225	-0.0149225\\
-0.015335	-0.015335\\
-0.0151525	-0.0151525\\
-0.0148325	-0.0148325\\
-0.014785	-0.014785\\
-0.0148775	-0.0148775\\
-0.0152425	-0.0152425\\
-0.01593	-0.01593\\
-0.0160225	-0.0160225\\
-0.015975	-0.015975\\
-0.0154725	-0.0154725\\
-0.0149225	-0.0149225\\
-0.0139625	-0.0139625\\
-0.0140525	-0.0140525\\
-0.0141	-0.0141\\
-0.01474	-0.01474\\
-0.01529	-0.01529\\
-0.015655	-0.015655\\
-0.01616	-0.01616\\
-0.0155175	-0.0155175\\
-0.0151525	-0.0151525\\
-0.015015	-0.015015\\
-0.01506	-0.01506\\
-0.0154725	-0.0154725\\
-0.01593	-0.01593\\
-0.0157475	-0.0157475\\
-0.01529	-0.01529\\
-0.015335	-0.015335\\
-0.01561	-0.01561\\
-0.01538	-0.01538\\
-0.0149225	-0.0149225\\
-0.0145575	-0.0145575\\
-0.014145	-0.014145\\
-0.0137775	-0.0137775\\
-0.0134575	-0.0134575\\
-0.0137775	-0.0137775\\
-0.0142825	-0.0142825\\
-0.01442	-0.01442\\
-0.0142375	-0.0142375\\
-0.0141	-0.0141\\
-0.0134125	-0.0134125\\
-0.0125875	-0.0125875\\
-0.0124975	-0.0124975\\
-0.0130475	-0.0130475\\
-0.0136425	-0.0136425\\
-0.014145	-0.014145\\
-0.01442	-0.01442\\
-0.01474	-0.01474\\
-0.0151975	-0.0151975\\
-0.015975	-0.015975\\
-0.016205	-0.016205\\
-0.0163875	-0.0163875\\
-0.0163425	-0.0163425\\
-0.0157025	-0.0157025\\
-0.015335	-0.015335\\
-0.01529	-0.01529\\
-0.0155175	-0.0155175\\
-0.0154725	-0.0154725\\
-0.015105	-0.015105\\
-0.014695	-0.014695\\
-0.0143275	-0.0143275\\
-0.0146475	-0.0146475\\
-0.014785	-0.014785\\
-0.0142825	-0.0142825\\
-0.0137325	-0.0137325\\
-0.013915	-0.013915\\
-0.0142825	-0.0142825\\
-0.014145	-0.014145\\
-0.014375	-0.014375\\
-0.0143275	-0.0143275\\
-0.0146025	-0.0146025\\
-0.0151975	-0.0151975\\
-0.0151525	-0.0151525\\
-0.0154275	-0.0154275\\
-0.0160225	-0.0160225\\
-0.0163425	-0.0163425\\
-0.016205	-0.016205\\
-0.01561	-0.01561\\
-0.0151525	-0.0151525\\
-0.01451	-0.01451\\
-0.0146025	-0.0146025\\
-0.01442	-0.01442\\
-0.0148325	-0.0148325\\
-0.0149225	-0.0149225\\
-0.0146475	-0.0146475\\
-0.0148325	-0.0148325\\
-0.01451	-0.01451\\
-0.0146025	-0.0146025\\
-0.014465	-0.014465\\
-0.013915	-0.013915\\
-0.01387	-0.01387\\
-0.0136425	-0.0136425\\
-0.01355	-0.01355\\
-0.0131825	-0.0131825\\
-0.01323	-0.01323\\
-0.0134575	-0.0134575\\
-0.013505	-0.013505\\
-0.0139625	-0.0139625\\
-0.014465	-0.014465\\
-0.0142825	-0.0142825\\
-0.014785	-0.014785\\
-0.0152425	-0.0152425\\
-0.0151525	-0.0151525\\
-0.0151975	-0.0151975\\
-0.014465	-0.014465\\
-0.0136875	-0.0136875\\
-0.0140525	-0.0140525\\
-0.0146025	-0.0146025\\
-0.0148325	-0.0148325\\
-0.01529	-0.01529\\
-0.0154725	-0.0154725\\
-0.015105	-0.015105\\
-0.0146475	-0.0146475\\
-0.0145575	-0.0145575\\
-0.0146475	-0.0146475\\
-0.014465	-0.014465\\
-0.013825	-0.013825\\
-0.0137775	-0.0137775\\
-0.0137325	-0.0137325\\
-0.0134575	-0.0134575\\
-0.0131375	-0.0131375\\
-0.0134125	-0.0134125\\
-0.0139625	-0.0139625\\
-0.0141	-0.0141\\
-0.0139625	-0.0139625\\
-0.014375	-0.014375\\
-0.015015	-0.015015\\
-0.014785	-0.014785\\
-0.01497	-0.01497\\
-0.0151525	-0.0151525\\
-0.01497	-0.01497\\
-0.014375	-0.014375\\
-0.0142825	-0.0142825\\
-0.01451	-0.01451\\
-0.0140525	-0.0140525\\
-0.013915	-0.013915\\
-0.0137325	-0.0137325\\
-0.013275	-0.013275\\
-0.0130475	-0.0130475\\
-0.0133675	-0.0133675\\
-0.0137775	-0.0137775\\
-0.0137325	-0.0137325\\
-0.01387	-0.01387\\
-0.0140525	-0.0140525\\
-0.01387	-0.01387\\
-0.0133675	-0.0133675\\
-0.013	-0.013\\
-0.0131825	-0.0131825\\
-0.01332	-0.01332\\
-0.01419	-0.01419\\
-0.0146025	-0.0146025\\
-0.014465	-0.014465\\
-0.01442	-0.01442\\
-0.014695	-0.014695\\
-0.0151525	-0.0151525\\
-0.0154275	-0.0154275\\
-0.015565	-0.015565\\
-0.01529	-0.01529\\
-0.0152425	-0.0152425\\
-0.01529	-0.01529\\
-0.0154725	-0.0154725\\
-0.0158375	-0.0158375\\
-0.015885	-0.015885\\
-0.0157925	-0.0157925\\
-0.015565	-0.015565\\
-0.01538	-0.01538\\
-0.0155175	-0.0155175\\
-0.015105	-0.015105\\
-0.0149225	-0.0149225\\
-0.0151975	-0.0151975\\
-0.01538	-0.01538\\
-0.0152425	-0.0152425\\
-0.01529	-0.01529\\
-0.0151975	-0.0151975\\
-0.01497	-0.01497\\
-0.0145575	-0.0145575\\
-0.014695	-0.014695\\
-0.01538	-0.01538\\
-0.01529	-0.01529\\
-0.014785	-0.014785\\
-0.01474	-0.01474\\
-0.0142375	-0.0142375\\
-0.0137775	-0.0137775\\
-0.0139625	-0.0139625\\
-0.0136425	-0.0136425\\
-0.013505	-0.013505\\
-0.01355	-0.01355\\
-0.01332	-0.01332\\
-0.01355	-0.01355\\
-0.0134575	-0.0134575\\
-0.0130925	-0.0130925\\
-0.0133675	-0.0133675\\
-0.0134575	-0.0134575\\
-0.0136875	-0.0136875\\
-0.0137325	-0.0137325\\
-0.0137775	-0.0137775\\
-0.0140525	-0.0140525\\
-0.0141	-0.0141\\
-0.0142375	-0.0142375\\
-0.01506	-0.01506\\
-0.0148325	-0.0148325\\
-0.0148775	-0.0148775\\
-0.0145575	-0.0145575\\
-0.013915	-0.013915\\
-0.01419	-0.01419\\
-0.0137325	-0.0137325\\
-0.013915	-0.013915\\
-0.0141	-0.0141\\
-0.01419	-0.01419\\
-0.01387	-0.01387\\
-0.0133675	-0.0133675\\
-0.0130925	-0.0130925\\
-0.012955	-0.012955\\
-0.0134125	-0.0134125\\
-0.01355	-0.01355\\
-0.0137325	-0.0137325\\
-0.0141	-0.0141\\
-0.0146025	-0.0146025\\
-0.01451	-0.01451\\
-0.014785	-0.014785\\
-0.015105	-0.015105\\
-0.01506	-0.01506\\
-0.01497	-0.01497\\
-0.015655	-0.015655\\
-0.0155175	-0.0155175\\
-0.015655	-0.015655\\
-0.01561	-0.01561\\
-0.0157025	-0.0157025\\
-0.01593	-0.01593\\
-0.0154275	-0.0154275\\
-0.01474	-0.01474\\
-0.0143275	-0.0143275\\
-0.01451	-0.01451\\
-0.0148775	-0.0148775\\
-0.0149225	-0.0149225\\
-0.0146475	-0.0146475\\
-0.0139625	-0.0139625\\
-0.01387	-0.01387\\
-0.0146025	-0.0146025\\
-0.0151975	-0.0151975\\
-0.01538	-0.01538\\
-0.014785	-0.014785\\
-0.01442	-0.01442\\
-0.0146025	-0.0146025\\
-0.0152425	-0.0152425\\
-0.015565	-0.015565\\
-0.015655	-0.015655\\
-0.015335	-0.015335\\
-0.01561	-0.01561\\
-0.0158375	-0.0158375\\
-0.01625	-0.01625\\
-0.0162975	-0.0162975\\
-0.0161125	-0.0161125\\
-0.016205	-0.016205\\
-0.0161125	-0.0161125\\
-0.0155175	-0.0155175\\
-0.0151525	-0.0151525\\
-0.0152425	-0.0152425\\
-0.0154725	-0.0154725\\
-0.015565	-0.015565\\
-0.0152425	-0.0152425\\
-0.01538	-0.01538\\
-0.015015	-0.015015\\
-0.0151975	-0.0151975\\
-0.0155175	-0.0155175\\
-0.0157025	-0.0157025\\
-0.0154275	-0.0154275\\
-0.01506	-0.01506\\
-0.014465	-0.014465\\
-0.01451	-0.01451\\
-0.01442	-0.01442\\
-0.01419	-0.01419\\
-0.0141	-0.0141\\
-0.0142825	-0.0142825\\
-0.014695	-0.014695\\
-0.0148325	-0.0148325\\
-0.01497	-0.01497\\
-0.01451	-0.01451\\
-0.014145	-0.014145\\
-0.0137775	-0.0137775\\
-0.0137325	-0.0137325\\
-0.014145	-0.014145\\
-0.0142825	-0.0142825\\
-0.0143275	-0.0143275\\
-0.0148325	-0.0148325\\
-0.0152425	-0.0152425\\
-0.0155175	-0.0155175\\
-0.015655	-0.015655\\
-0.0158375	-0.0158375\\
-0.0154725	-0.0154725\\
-0.01506	-0.01506\\
-0.014785	-0.014785\\
-0.014695	-0.014695\\
-0.01497	-0.01497\\
-0.0149225	-0.0149225\\
-0.014465	-0.014465\\
-0.0142375	-0.0142375\\
-0.0146475	-0.0146475\\
-0.015015	-0.015015\\
-0.01497	-0.01497\\
-0.0154725	-0.0154725\\
-0.0158375	-0.0158375\\
-0.01561	-0.01561\\
-0.015105	-0.015105\\
-0.0146025	-0.0146025\\
-0.01442	-0.01442\\
-0.0149225	-0.0149225\\
-0.015105	-0.015105\\
-0.0152425	-0.0152425\\
-0.01529	-0.01529\\
-0.015335	-0.015335\\
-0.015105	-0.015105\\
-0.01506	-0.01506\\
-0.014785	-0.014785\\
-0.0145575	-0.0145575\\
-0.0148775	-0.0148775\\
-0.015335	-0.015335\\
-0.015565	-0.015565\\
-0.015105	-0.015105\\
-0.0157025	-0.0157025\\
-0.0160225	-0.0160225\\
-0.0154725	-0.0154725\\
-0.0148325	-0.0148325\\
-0.0149225	-0.0149225\\
-0.0148775	-0.0148775\\
-0.014785	-0.014785\\
-0.01497	-0.01497\\
-0.0146025	-0.0146025\\
-0.01474	-0.01474\\
-0.015565	-0.015565\\
-0.015885	-0.015885\\
-0.0155175	-0.0155175\\
-0.0154275	-0.0154275\\
-0.01561	-0.01561\\
-0.01538	-0.01538\\
-0.0151975	-0.0151975\\
-0.0148325	-0.0148325\\
-0.015015	-0.015015\\
-0.0152425	-0.0152425\\
-0.0157025	-0.0157025\\
-0.0160675	-0.0160675\\
-0.01657	-0.01657\\
-0.01648	-0.01648\\
-0.0160675	-0.0160675\\
-0.0157475	-0.0157475\\
-0.0155175	-0.0155175\\
-0.0149225	-0.0149225\\
-0.014695	-0.014695\\
-0.01474	-0.01474\\
-0.01442	-0.01442\\
-0.0139625	-0.0139625\\
-0.0141	-0.0141\\
-0.014465	-0.014465\\
-0.0142375	-0.0142375\\
-0.01387	-0.01387\\
-0.0136875	-0.0136875\\
-0.01442	-0.01442\\
-0.015015	-0.015015\\
-0.01442	-0.01442\\
-0.014465	-0.014465\\
-0.0146025	-0.0146025\\
-0.0145575	-0.0145575\\
-0.0142375	-0.0142375\\
-0.013915	-0.013915\\
-0.01355	-0.01355\\
-0.013595	-0.013595\\
-0.0142375	-0.0142375\\
-0.0146475	-0.0146475\\
-0.01442	-0.01442\\
-0.0140525	-0.0140525\\
-0.013595	-0.013595\\
-0.0141	-0.0141\\
-0.0145575	-0.0145575\\
-0.01474	-0.01474\\
-0.014465	-0.014465\\
-0.0142825	-0.0142825\\
-0.014375	-0.014375\\
-0.014695	-0.014695\\
-0.0152425	-0.0152425\\
-0.01561	-0.01561\\
-0.0154725	-0.0154725\\
-0.0149225	-0.0149225\\
-0.014785	-0.014785\\
-0.01474	-0.01474\\
-0.014695	-0.014695\\
-0.0143275	-0.0143275\\
-0.0142825	-0.0142825\\
-0.01442	-0.01442\\
-0.0141	-0.0141\\
-0.01355	-0.01355\\
-0.0136425	-0.0136425\\
-0.0140525	-0.0140525\\
-0.0146475	-0.0146475\\
-0.01497	-0.01497\\
-0.01529	-0.01529\\
-0.0152425	-0.0152425\\
-0.015335	-0.015335\\
-0.0157925	-0.0157925\\
-0.01648	-0.01648\\
-0.01657	-0.01657\\
-0.0160675	-0.0160675\\
-0.015975	-0.015975\\
-0.01616	-0.01616\\
-0.01657	-0.01657\\
-0.0161125	-0.0161125\\
-0.0151525	-0.0151525\\
-0.014465	-0.014465\\
-0.0142375	-0.0142375\\
-0.013825	-0.013825\\
-0.01323	-0.01323\\
-0.0130925	-0.0130925\\
-0.0134575	-0.0134575\\
-0.01387	-0.01387\\
-0.0140075	-0.0140075\\
-0.013825	-0.013825\\
-0.0137325	-0.0137325\\
-0.013915	-0.013915\\
-0.01419	-0.01419\\
-0.0142825	-0.0142825\\
-0.0143275	-0.0143275\\
-0.0140525	-0.0140525\\
-0.0136875	-0.0136875\\
-0.0134575	-0.0134575\\
-0.0137325	-0.0137325\\
-0.0140075	-0.0140075\\
-0.01387	-0.01387\\
-0.0134575	-0.0134575\\
-0.0137325	-0.0137325\\
-0.013505	-0.013505\\
-0.0137325	-0.0137325\\
-0.0143275	-0.0143275\\
-0.0145575	-0.0145575\\
-0.0142375	-0.0142375\\
-0.0145575	-0.0145575\\
-0.014695	-0.014695\\
-0.0145575	-0.0145575\\
-0.0143275	-0.0143275\\
-0.014695	-0.014695\\
-0.0151525	-0.0151525\\
-0.0152425	-0.0152425\\
-0.0154275	-0.0154275\\
-0.01529	-0.01529\\
-0.015105	-0.015105\\
-0.015015	-0.015015\\
-0.01529	-0.01529\\
-0.0151975	-0.0151975\\
-0.01538	-0.01538\\
-0.0154725	-0.0154725\\
-0.0163425	-0.0163425\\
-0.0154725	-0.0154725\\
-0.014785	-0.014785\\
-0.0146475	-0.0146475\\
-0.0148775	-0.0148775\\
-0.0151525	-0.0151525\\
-0.0154275	-0.0154275\\
-0.01561	-0.01561\\
-0.0157025	-0.0157025\\
-0.015335	-0.015335\\
-0.015105	-0.015105\\
-0.0151525	-0.0151525\\
-0.0152425	-0.0152425\\
-0.01474	-0.01474\\
-0.0142375	-0.0142375\\
-0.0145575	-0.0145575\\
-0.0146475	-0.0146475\\
-0.01497	-0.01497\\
-0.015335	-0.015335\\
-0.01529	-0.01529\\
-0.015015	-0.015015\\
-0.014695	-0.014695\\
-0.0148775	-0.0148775\\
-0.01529	-0.01529\\
-0.0158375	-0.0158375\\
-0.0161125	-0.0161125\\
-0.0163875	-0.0163875\\
-0.0162975	-0.0162975\\
-0.0161125	-0.0161125\\
-0.015565	-0.015565\\
-0.0148775	-0.0148775\\
-0.015335	-0.015335\\
-0.0157475	-0.0157475\\
-0.01529	-0.01529\\
-0.01561	-0.01561\\
-0.0161125	-0.0161125\\
-0.01648	-0.01648\\
-0.0161125	-0.0161125\\
-0.015335	-0.015335\\
-0.0146475	-0.0146475\\
-0.0146025	-0.0146025\\
-0.0146475	-0.0146475\\
-0.0146025	-0.0146025\\
-0.0142375	-0.0142375\\
-0.01419	-0.01419\\
-0.0139625	-0.0139625\\
-0.0140525	-0.0140525\\
-0.0140075	-0.0140075\\
-0.014145	-0.014145\\
-0.014695	-0.014695\\
-0.01474	-0.01474\\
-0.01506	-0.01506\\
-0.0155175	-0.0155175\\
-0.01561	-0.01561\\
-0.01506	-0.01506\\
-0.0148325	-0.0148325\\
-0.015015	-0.015015\\
-0.01474	-0.01474\\
-0.0145575	-0.0145575\\
-0.01419	-0.01419\\
-0.014375	-0.014375\\
-0.01442	-0.01442\\
-0.0139625	-0.0139625\\
-0.0133675	-0.0133675\\
-0.0130475	-0.0130475\\
-0.0134575	-0.0134575\\
-0.013915	-0.013915\\
-0.014145	-0.014145\\
-0.0137325	-0.0137325\\
-0.01419	-0.01419\\
-0.0146475	-0.0146475\\
-0.014465	-0.014465\\
-0.0142375	-0.0142375\\
-0.01442	-0.01442\\
-0.0146025	-0.0146025\\
-0.014785	-0.014785\\
-0.01451	-0.01451\\
-0.01419	-0.01419\\
-0.0143275	-0.0143275\\
-0.0142375	-0.0142375\\
-0.0143275	-0.0143275\\
-0.0146475	-0.0146475\\
-0.015335	-0.015335\\
-0.0152425	-0.0152425\\
-0.015015	-0.015015\\
-0.0155175	-0.0155175\\
-0.015335	-0.015335\\
-0.0148325	-0.0148325\\
-0.014375	-0.014375\\
-0.0140075	-0.0140075\\
-0.013915	-0.013915\\
-0.0137775	-0.0137775\\
-0.0136425	-0.0136425\\
-0.0142375	-0.0142375\\
-0.0137325	-0.0137325\\
-0.01332	-0.01332\\
-0.0136425	-0.0136425\\
-0.0140075	-0.0140075\\
-0.013915	-0.013915\\
-0.013595	-0.013595\\
-0.0131825	-0.0131825\\
-0.013275	-0.013275\\
-0.0140525	-0.0140525\\
-0.0146025	-0.0146025\\
-0.0145575	-0.0145575\\
-0.0143275	-0.0143275\\
-0.014375	-0.014375\\
-0.01442	-0.01442\\
-0.01451	-0.01451\\
-0.014785	-0.014785\\
-0.015105	-0.015105\\
-0.0151975	-0.0151975\\
-0.01506	-0.01506\\
-0.0146475	-0.0146475\\
-0.0143275	-0.0143275\\
-0.0142375	-0.0142375\\
-0.01451	-0.01451\\
-0.01419	-0.01419\\
-0.013915	-0.013915\\
-0.014375	-0.014375\\
-0.01474	-0.01474\\
-0.014785	-0.014785\\
-0.0149225	-0.0149225\\
-0.01538	-0.01538\\
-0.0148775	-0.0148775\\
-0.0151975	-0.0151975\\
-0.015565	-0.015565\\
-0.0154275	-0.0154275\\
-0.0155175	-0.0155175\\
-0.0163425	-0.0163425\\
-0.01657	-0.01657\\
-0.01616	-0.01616\\
-0.01625	-0.01625\\
-0.01657	-0.01657\\
-0.0166175	-0.0166175\\
-0.0167075	-0.0167075\\
-0.016755	-0.016755\\
-0.017075	-0.017075\\
-0.0169825	-0.0169825\\
-0.01648	-0.01648\\
-0.0157925	-0.0157925\\
-0.0155175	-0.0155175\\
-0.01538	-0.01538\\
-0.01497	-0.01497\\
-0.0151975	-0.0151975\\
-0.0157025	-0.0157025\\
-0.01538	-0.01538\\
-0.01497	-0.01497\\
-0.0149225	-0.0149225\\
-0.01451	-0.01451\\
-0.0137775	-0.0137775\\
-0.01355	-0.01355\\
-0.0134125	-0.0134125\\
-0.0136425	-0.0136425\\
-0.01355	-0.01355\\
-0.01332	-0.01332\\
-0.01291	-0.01291\\
-0.0125875	-0.0125875\\
-0.01291	-0.01291\\
-0.0127725	-0.0127725\\
-0.012725	-0.012725\\
-0.0133675	-0.0133675\\
-0.0140075	-0.0140075\\
-0.0142825	-0.0142825\\
-0.0141	-0.0141\\
-0.01419	-0.01419\\
-0.0148775	-0.0148775\\
-0.01442	-0.01442\\
-0.0136875	-0.0136875\\
-0.01332	-0.01332\\
-0.01291	-0.01291\\
-0.013	-0.013\\
-0.01355	-0.01355\\
-0.0142825	-0.0142825\\
-0.0140525	-0.0140525\\
-0.01442	-0.01442\\
-0.015105	-0.015105\\
-0.0152425	-0.0152425\\
-0.015015	-0.015015\\
-0.01442	-0.01442\\
-0.014375	-0.014375\\
-0.014785	-0.014785\\
-0.0151975	-0.0151975\\
-0.01474	-0.01474\\
-0.0139625	-0.0139625\\
-0.0140075	-0.0140075\\
-0.0139625	-0.0139625\\
-0.01323	-0.01323\\
-0.0125875	-0.0125875\\
-0.012955	-0.012955\\
-0.014145	-0.014145\\
-0.0145575	-0.0145575\\
-0.0143275	-0.0143275\\
-0.0137775	-0.0137775\\
-0.0136875	-0.0136875\\
-0.01268	-0.01268\\
-0.0125875	-0.0125875\\
-0.012635	-0.012635\\
-0.012955	-0.012955\\
-0.01355	-0.01355\\
-0.0142375	-0.0142375\\
-0.0146025	-0.0146025\\
-0.01474	-0.01474\\
-0.0148775	-0.0148775\\
-0.0149225	-0.0149225\\
-0.0148775	-0.0148775\\
-0.0148325	-0.0148325\\
-0.0149225	-0.0149225\\
-0.01561	-0.01561\\
-0.0157925	-0.0157925\\
-0.01506	-0.01506\\
-0.014145	-0.014145\\
-0.014695	-0.014695\\
-0.0151525	-0.0151525\\
-0.015105	-0.015105\\
-0.0146025	-0.0146025\\
-0.0143275	-0.0143275\\
-0.01497	-0.01497\\
-0.0157025	-0.0157025\\
-0.015885	-0.015885\\
-0.0161125	-0.0161125\\
-0.016205	-0.016205\\
-0.015885	-0.015885\\
-0.0158375	-0.0158375\\
-0.015015	-0.015015\\
-0.014465	-0.014465\\
-0.0146475	-0.0146475\\
-0.0149225	-0.0149225\\
-0.015015	-0.015015\\
-0.015105	-0.015105\\
-0.01474	-0.01474\\
-0.0148325	-0.0148325\\
-0.0142825	-0.0142825\\
-0.0136875	-0.0136875\\
-0.01323	-0.01323\\
-0.0134125	-0.0134125\\
-0.01332	-0.01332\\
-0.0127725	-0.0127725\\
-0.0130475	-0.0130475\\
-0.013915	-0.013915\\
-0.013825	-0.013825\\
-0.013595	-0.013595\\
-0.01387	-0.01387\\
-0.0146025	-0.0146025\\
-0.01451	-0.01451\\
-0.0137775	-0.0137775\\
-0.0130925	-0.0130925\\
-0.013595	-0.013595\\
-0.0146475	-0.0146475\\
-0.015335	-0.015335\\
-0.0160225	-0.0160225\\
-0.0166625	-0.0166625\\
-0.0168	-0.0168\\
-0.0163875	-0.0163875\\
-0.016205	-0.016205\\
-0.01657	-0.01657\\
-0.01648	-0.01648\\
-0.0163875	-0.0163875\\
-0.0164325	-0.0164325\\
-0.01657	-0.01657\\
-0.0166625	-0.0166625\\
-0.017075	-0.017075\\
-0.0168	-0.0168\\
-0.0160225	-0.0160225\\
-0.015015	-0.015015\\
-0.01442	-0.01442\\
-0.0143275	-0.0143275\\
-0.0142825	-0.0142825\\
-0.014785	-0.014785\\
-0.01474	-0.01474\\
-0.015015	-0.015015\\
-0.0146025	-0.0146025\\
-0.014465	-0.014465\\
-0.014785	-0.014785\\
-0.015105	-0.015105\\
-0.0146475	-0.0146475\\
-0.0140525	-0.0140525\\
-0.014145	-0.014145\\
-0.0141	-0.0141\\
-0.0151975	-0.0151975\\
-0.0160225	-0.0160225\\
-0.0160675	-0.0160675\\
-0.01625	-0.01625\\
-0.016525	-0.016525\\
-0.01703	-0.01703\\
-0.01712	-0.01712\\
-0.016525	-0.016525\\
-0.0158375	-0.0158375\\
-0.0149225	-0.0149225\\
-0.0146025	-0.0146025\\
-0.0146475	-0.0146475\\
-0.0148775	-0.0148775\\
-0.0146475	-0.0146475\\
-0.01442	-0.01442\\
-0.014465	-0.014465\\
-0.0146475	-0.0146475\\
-0.014785	-0.014785\\
-0.015015	-0.015015\\
-0.0149225	-0.0149225\\
-0.0141	-0.0141\\
-0.01355	-0.01355\\
-0.0130925	-0.0130925\\
-0.013	-0.013\\
-0.0130475	-0.0130475\\
-0.013595	-0.013595\\
-0.0137775	-0.0137775\\
-0.0140075	-0.0140075\\
-0.0146025	-0.0146025\\
-0.01451	-0.01451\\
-0.01506	-0.01506\\
-0.01593	-0.01593\\
-0.015885	-0.015885\\
-0.0157025	-0.0157025\\
-0.01593	-0.01593\\
-0.01616	-0.01616\\
-0.0157925	-0.0157925\\
-0.0154725	-0.0154725\\
-0.01497	-0.01497\\
-0.01474	-0.01474\\
-0.015565	-0.015565\\
-0.0166175	-0.0166175\\
-0.0172125	-0.0172125\\
-0.017165	-0.017165\\
-0.0169375	-0.0169375\\
-0.0163875	-0.0163875\\
-0.0162975	-0.0162975\\
-0.0166625	-0.0166625\\
-0.0161125	-0.0161125\\
-0.01561	-0.01561\\
-0.015335	-0.015335\\
-0.0158375	-0.0158375\\
-0.0160225	-0.0160225\\
-0.015335	-0.015335\\
-0.01506	-0.01506\\
-0.014695	-0.014695\\
-0.015015	-0.015015\\
-0.01561	-0.01561\\
-0.0157925	-0.0157925\\
-0.015565	-0.015565\\
-0.0152425	-0.0152425\\
-0.01506	-0.01506\\
-0.0146025	-0.0146025\\
-0.0142375	-0.0142375\\
-0.0139625	-0.0139625\\
-0.013595	-0.013595\\
-0.01355	-0.01355\\
-0.0140075	-0.0140075\\
-0.0146025	-0.0146025\\
-0.01474	-0.01474\\
-0.0143275	-0.0143275\\
-0.014145	-0.014145\\
-0.013915	-0.013915\\
-0.0140075	-0.0140075\\
-0.0143275	-0.0143275\\
-0.01451	-0.01451\\
-0.0140525	-0.0140525\\
-0.0146475	-0.0146475\\
-0.01529	-0.01529\\
-0.0151525	-0.0151525\\
-0.014145	-0.014145\\
-0.013915	-0.013915\\
-0.0142825	-0.0142825\\
-0.014465	-0.014465\\
-0.0143275	-0.0143275\\
-0.0140075	-0.0140075\\
-0.0142825	-0.0142825\\
-0.0149225	-0.0149225\\
-0.0146025	-0.0146025\\
-0.0136875	-0.0136875\\
-0.0134575	-0.0134575\\
-0.014375	-0.014375\\
-0.0148775	-0.0148775\\
-0.014465	-0.014465\\
-0.014145	-0.014145\\
-0.01474	-0.01474\\
-0.015335	-0.015335\\
-0.01538	-0.01538\\
-0.0157475	-0.0157475\\
-0.016205	-0.016205\\
-0.015885	-0.015885\\
-0.0155175	-0.0155175\\
-0.0158375	-0.0158375\\
-0.015565	-0.015565\\
-0.015655	-0.015655\\
-0.0160675	-0.0160675\\
-0.015885	-0.015885\\
-0.01506	-0.01506\\
-0.01538	-0.01538\\
-0.01625	-0.01625\\
-0.0166175	-0.0166175\\
-0.0163875	-0.0163875\\
-0.0161125	-0.0161125\\
-0.0163425	-0.0163425\\
-0.0161125	-0.0161125\\
-0.01561	-0.01561\\
-0.01529	-0.01529\\
-0.0154275	-0.0154275\\
-0.0155175	-0.0155175\\
-0.0154725	-0.0154725\\
-0.015565	-0.015565\\
-0.015335	-0.015335\\
-0.0152425	-0.0152425\\
-0.015565	-0.015565\\
-0.0151975	-0.0151975\\
-0.014785	-0.014785\\
-0.01451	-0.01451\\
-0.0141	-0.0141\\
-0.0140525	-0.0140525\\
-0.0145575	-0.0145575\\
-0.0146475	-0.0146475\\
-0.01497	-0.01497\\
-0.0154275	-0.0154275\\
-0.0157475	-0.0157475\\
-0.015655	-0.015655\\
-0.01538	-0.01538\\
-0.0145575	-0.0145575\\
-0.01529	-0.01529\\
-0.0151525	-0.0151525\\
-0.014375	-0.014375\\
-0.01474	-0.01474\\
-0.01538	-0.01538\\
-0.015565	-0.015565\\
-0.015975	-0.015975\\
-0.01657	-0.01657\\
-0.0164325	-0.0164325\\
-0.0161125	-0.0161125\\
-0.01625	-0.01625\\
-0.015885	-0.015885\\
-0.0154275	-0.0154275\\
-0.0154725	-0.0154725\\
-0.0157475	-0.0157475\\
-0.015885	-0.015885\\
-0.0158375	-0.0158375\\
-0.01538	-0.01538\\
-0.015015	-0.015015\\
-0.01474	-0.01474\\
-0.0148775	-0.0148775\\
-0.014695	-0.014695\\
-0.0152425	-0.0152425\\
-0.015565	-0.015565\\
-0.015335	-0.015335\\
-0.015015	-0.015015\\
-0.01529	-0.01529\\
-0.0148775	-0.0148775\\
-0.01419	-0.01419\\
-0.0140525	-0.0140525\\
-0.0143275	-0.0143275\\
-0.0142825	-0.0142825\\
-0.0140075	-0.0140075\\
-0.013275	-0.013275\\
-0.01332	-0.01332\\
-0.013825	-0.013825\\
-0.0146025	-0.0146025\\
-0.01497	-0.01497\\
-0.0154275	-0.0154275\\
-0.0157925	-0.0157925\\
-0.0152425	-0.0152425\\
-0.014695	-0.014695\\
-0.0154275	-0.0154275\\
-0.016205	-0.016205\\
-0.0161125	-0.0161125\\
-0.0160225	-0.0160225\\
-0.0164325	-0.0164325\\
-0.01616	-0.01616\\
-0.0157925	-0.0157925\\
-0.0157025	-0.0157025\\
-0.015975	-0.015975\\
-0.0157025	-0.0157025\\
-0.01538	-0.01538\\
-0.015885	-0.015885\\
-0.0162975	-0.0162975\\
-0.0164325	-0.0164325\\
-0.0154725	-0.0154725\\
-0.014465	-0.014465\\
-0.015105	-0.015105\\
-0.0142375	-0.0142375\\
-0.0137325	-0.0137325\\
-0.01387	-0.01387\\
-0.014465	-0.014465\\
-0.0151975	-0.0151975\\
-0.015105	-0.015105\\
-0.014695	-0.014695\\
-0.0142825	-0.0142825\\
-0.0136425	-0.0136425\\
-0.013595	-0.013595\\
-0.01355	-0.01355\\
-0.0136425	-0.0136425\\
-0.0142375	-0.0142375\\
-0.0143275	-0.0143275\\
-0.0137325	-0.0137325\\
-0.0134125	-0.0134125\\
-0.01355	-0.01355\\
-0.0137775	-0.0137775\\
-0.0134125	-0.0134125\\
-0.0136875	-0.0136875\\
-0.01355	-0.01355\\
-0.0131825	-0.0131825\\
-0.0137325	-0.0137325\\
-0.014375	-0.014375\\
-0.01506	-0.01506\\
-0.0158375	-0.0158375\\
-0.015975	-0.015975\\
-0.0151975	-0.0151975\\
-0.01451	-0.01451\\
-0.014375	-0.014375\\
-0.013825	-0.013825\\
-0.0134125	-0.0134125\\
-0.013915	-0.013915\\
-0.01419	-0.01419\\
-0.014145	-0.014145\\
-0.0142375	-0.0142375\\
-0.014465	-0.014465\\
-0.0146025	-0.0146025\\
-0.014695	-0.014695\\
-0.01497	-0.01497\\
-0.01506	-0.01506\\
-0.0157025	-0.0157025\\
-0.0151525	-0.0151525\\
-0.01419	-0.01419\\
-0.013825	-0.013825\\
-0.0143275	-0.0143275\\
-0.0141	-0.0141\\
-0.0134575	-0.0134575\\
-0.0134125	-0.0134125\\
-0.01323	-0.01323\\
-0.01268	-0.01268\\
-0.01323	-0.01323\\
-0.0136425	-0.0136425\\
-0.0142375	-0.0142375\\
-0.0149225	-0.0149225\\
-0.0148775	-0.0148775\\
-0.013825	-0.013825\\
-0.014145	-0.014145\\
-0.01442	-0.01442\\
-0.01387	-0.01387\\
-0.0146025	-0.0146025\\
-0.014785	-0.014785\\
-0.01538	-0.01538\\
-0.01593	-0.01593\\
-0.0154725	-0.0154725\\
-0.0151525	-0.0151525\\
-0.01497	-0.01497\\
-0.015105	-0.015105\\
-0.01561	-0.01561\\
-0.015335	-0.015335\\
-0.0148775	-0.0148775\\
};
\end{axis}

\begin{axis}[%
width=4.927cm,
height=2.746cm,
at={(6.484cm,7.627cm)},
scale only axis,
xmin=-0.018,
xmax=-0.012,
xlabel style={font=\color{white!15!black}},
xlabel={$u(t-1)$},
ymin=-1.0101325,
ymax=0,
ylabel style={font=\color{white!15!black}},
ylabel={$\delta^4 y(t)$},
axis background/.style={fill=white},
title style={font=\bfseries},
title={C6, R = 0.6803},
axis x line*=bottom,
axis y line*=left
]
\addplot[only marks, mark=*, mark options={}, mark size=1.5000pt, color=mycolor1, fill=mycolor1] table[row sep=crcr]{%
x	y\\
-0.015655	-0.4638675\\
-0.015655	-0.5493175\\
-0.0158375	-0.424195\\
-0.0157025	-0.216675\\
-0.0149225	-0.268555\\
-0.0148325	-0.2563475\\
-0.0148775	-0.3173825\\
-0.015015	-0.2594\\
-0.0148775	-0.10376\\
-0.0140075	-0.079345\\
-0.013275	-0.07019\\
-0.01291	-0.2227775\\
-0.0137775	-0.3936775\\
-0.014785	-0.41809\\
-0.015105	-0.4364025\\
-0.0151525	-0.31128\\
-0.0149225	-0.1861575\\
-0.0143275	-0.3967275\\
-0.0149225	-0.27771\\
-0.0148325	-0.20752\\
-0.014375	-0.357055\\
-0.0148325	-0.3082275\\
-0.0148325	-0.2594\\
-0.0146475	-0.43335\\
-0.015105	-0.4119875\\
-0.0152425	-0.31128\\
-0.0149225	-0.45166\\
-0.0152425	-0.4486075\\
-0.01538	-0.357055\\
-0.0151525	-0.29297\\
-0.0149225	-0.2227775\\
-0.0146475	-0.2716075\\
-0.0146475	-0.3448475\\
-0.0148775	-0.3265375\\
-0.0148775	-0.3417975\\
-0.0149225	-0.3509525\\
-0.015015	-0.3784175\\
-0.015105	-0.5340575\\
-0.0154725	-0.479125\\
-0.0155175	-0.3173825\\
-0.015105	-0.2899175\\
-0.0149225	-0.1617425\\
-0.014375	-0.17395\\
-0.014145	-0.24109\\
-0.014375	-0.1861575\\
-0.0142375	-0.1953125\\
-0.014145	-0.27771\\
-0.01442	-0.320435\\
-0.014695	-0.2990725\\
-0.014695	-0.476075\\
-0.0151525	-0.3479\\
-0.015105	-0.2044675\\
-0.01451	-0.1586925\\
-0.014145	-0.2136225\\
-0.0142375	-0.253295\\
-0.01442	-0.128175\\
-0.0139625	-0.146485\\
-0.0136875	-0.2044675\\
-0.0140075	-0.164795\\
-0.013915	-0.14038\\
-0.0136875	-0.1861575\\
-0.01387	-0.216675\\
-0.0140525	-0.3479\\
-0.01451	-0.3265375\\
-0.01474	-0.408935\\
-0.0149225	-0.36621\\
-0.0149225	-0.427245\\
-0.015105	-0.3479\\
-0.01497	-0.3326425\\
-0.0149225	-0.2899175\\
-0.014785	-0.27771\\
-0.014695	-0.27771\\
-0.014695	-0.4913325\\
-0.01529	-0.701905\\
-0.0160225	-0.72937\\
-0.0162975	-0.701905\\
-0.0162975	-0.4394525\\
-0.0158375	-0.6011975\\
-0.0160225	-0.7598875\\
-0.01648	-0.7934575\\
-0.0166175	-0.598145\\
-0.0163875	-0.7843025\\
-0.01657	-1.0101325\\
-0.01712	-0.7232675\\
-0.0169825	-0.5950925\\
-0.0166625	-0.3967275\\
-0.0161125	-0.305175\\
-0.01561	-0.2563475\\
-0.015335	-0.3234875\\
-0.015335	-0.2319325\\
-0.015105	-0.1525875\\
-0.0145575	-0.149535\\
-0.01419	-0.18921\\
-0.0142825	-0.29602\\
-0.014695	-0.268555\\
-0.014785	-0.27771\\
-0.014785	-0.36621\\
-0.01506	-0.38147\\
-0.0151975	-0.5188\\
-0.01561	-0.41809\\
-0.01561	-0.52185\\
-0.0157025	-0.372315\\
-0.0155175	-0.3265375\\
-0.0152425	-0.3784175\\
-0.01538	-0.2716075\\
-0.015105	-0.2471925\\
-0.0149225	-0.305175\\
-0.015015	-0.3082275\\
-0.015105	-0.2136225\\
-0.014785	-0.2380375\\
-0.01474	-0.1678475\\
-0.014465	-0.19226\\
-0.01442	-0.2044675\\
-0.014465	-0.1434325\\
-0.01419	-0.18921\\
-0.014145	-0.234985\\
-0.014375	-0.3784175\\
-0.0149225	-0.38147\\
-0.0151525	-0.427245\\
-0.01529	-0.24414\\
-0.0148325	-0.3509525\\
-0.0149225	-0.527955\\
-0.015565	-0.4028325\\
-0.0154725	-0.46692\\
-0.015565	-0.479125\\
-0.015655	-0.2807625\\
-0.0151525	-0.18921\\
-0.0146475	-0.268555\\
-0.01474	-0.3082275\\
-0.0149225	-0.512695\\
-0.0155175	-0.631715\\
-0.0160225	-0.631715\\
-0.01616	-0.460815\\
-0.015885	-0.4730225\\
-0.0157925	-0.41809\\
-0.0157475	-0.408935\\
-0.0157025	-0.405885\\
-0.015655	-0.2716075\\
-0.0152425	-0.24109\\
-0.0149225	-0.216675\\
-0.014785	-0.1953125\\
-0.0146475	-0.2197275\\
-0.0146475	-0.3326425\\
-0.01497	-0.5096425\\
-0.01561	-0.4028325\\
-0.015565	-0.2746575\\
-0.015105	-0.2319325\\
-0.0148775	-0.2227775\\
-0.01474	-0.131225\\
-0.0142375	-0.10376\\
-0.0137325	-0.1190175\\
-0.013595	-0.22583\\
-0.0141	-0.1678475\\
-0.0141	-0.1953125\\
-0.0141	-0.2136225\\
-0.014145	-0.201415\\
-0.0141	-0.13733\\
-0.01387	-0.094605\\
-0.0134575	-0.076295\\
-0.0130475	-0.13733\\
-0.013275	-0.2990725\\
-0.01419	-0.4302975\\
-0.01497	-0.48523\\
-0.015335	-0.3417975\\
-0.015105	-0.2288825\\
-0.0146025	-0.1617425\\
-0.01419	-0.1098625\\
-0.0136425	-0.2471925\\
-0.0141	-0.2288825\\
-0.0142825	-0.3448475\\
-0.0146025	-0.4211425\\
-0.015105	-0.6896975\\
-0.01593	-0.790405\\
-0.01657	-0.650025\\
-0.01648	-0.6225575\\
-0.0163875	-0.39978\\
-0.01593	-0.4638675\\
-0.0157925	-0.4882825\\
-0.015975	-0.531005\\
-0.0160225	-0.3967275\\
-0.0157925	-0.3601075\\
-0.015565	-0.357055\\
-0.015565	-0.3234875\\
-0.0154275	-0.390625\\
-0.0155175	-0.2746575\\
-0.01529	-0.4882825\\
-0.0157025	-0.5859375\\
-0.0160675	-0.4394525\\
-0.015885	-0.26245\\
-0.01529	-0.27771\\
-0.015105	-0.4821775\\
-0.015655	-0.5859375\\
-0.0160675	-0.7232675\\
-0.0163875	-0.7598875\\
-0.0166175	-0.756835\\
-0.0167075	-0.5706775\\
-0.0164325	-0.512695\\
-0.01625	-0.58899\\
-0.0163425	-0.6896975\\
-0.01657	-0.427245\\
-0.01616	-0.26245\\
-0.0154275	-0.38147\\
-0.015565	-0.305175\\
-0.0154275	-0.198365\\
-0.01497	-0.2746575\\
-0.01497	-0.305175\\
-0.0151525	-0.2380375\\
-0.01497	-0.18921\\
-0.014695	-0.253295\\
-0.014785	-0.201415\\
-0.014695	-0.2899175\\
-0.0148775	-0.405885\\
-0.01538	-0.3692625\\
-0.015335	-0.250245\\
-0.015015	-0.253295\\
-0.0149225	-0.3936775\\
-0.01529	-0.653075\\
-0.0160675	-0.4638675\\
-0.015975	-0.29602\\
-0.01538	-0.2136225\\
-0.0149225	-0.253295\\
-0.0148775	-0.22583\\
-0.0148775	-0.1678475\\
-0.0145575	-0.19226\\
-0.014465	-0.2319325\\
-0.0146475	-0.302125\\
-0.0148775	-0.3356925\\
-0.01506	-0.2227775\\
-0.01474	-0.390625\\
-0.0151525	-0.4486075\\
-0.0155175	-0.427245\\
-0.0154725	-0.3509525\\
-0.01538	-0.375365\\
-0.015335	-0.5065925\\
-0.0157025	-0.31433\\
-0.015335	-0.1342775\\
-0.014375	-0.1953125\\
-0.0142825	-0.2136225\\
-0.01442	-0.2990725\\
-0.01474	-0.253295\\
-0.01474	-0.1861575\\
-0.014465	-0.29602\\
-0.01474	-0.2807625\\
-0.0148325	-0.198365\\
-0.0145575	-0.2563475\\
-0.0146025	-0.22583\\
-0.0145575	-0.201415\\
-0.014465	-0.2380375\\
-0.0145575	-0.13733\\
-0.014145	-0.1708975\\
-0.0141	-0.1190175\\
-0.01387	-0.128175\\
-0.0137775	-0.2655025\\
-0.014375	-0.305175\\
-0.0146475	-0.41809\\
-0.015105	-0.5493175\\
-0.01561	-0.357055\\
-0.01538	-0.20752\\
-0.01474	-0.1586925\\
-0.0142375	-0.15564\\
-0.014145	-0.2288825\\
-0.014375	-0.13733\\
-0.0140525	-0.1770025\\
-0.0139625	-0.2136225\\
-0.01419	-0.375365\\
-0.0148325	-0.3326425\\
-0.01497	-0.476075\\
-0.015335	-0.31128\\
-0.01506	-0.29297\\
-0.0148775	-0.14038\\
-0.0142375	-0.15564\\
-0.013915	-0.08545\\
-0.013505	-0.1159675\\
-0.0133675	-0.128175\\
-0.0134125	-0.2380375\\
-0.0139625	-0.250245\\
-0.0142375	-0.29297\\
-0.01442	-0.3082275\\
-0.0146025	-0.494385\\
-0.0151975	-0.427245\\
-0.0154275	-0.320435\\
-0.015105	-0.3845225\\
-0.0151525	-0.408935\\
-0.01529	-0.5340575\\
-0.01561	-0.4028325\\
-0.0154725	-0.26245\\
-0.015015	-0.1342775\\
-0.0142375	-0.1007075\\
-0.013595	-0.10376\\
-0.0133675	-0.131225\\
-0.0134575	-0.13733\\
-0.01355	-0.128175\\
-0.0134575	-0.1770025\\
-0.0136425	-0.2716075\\
-0.01419	-0.2471925\\
-0.0143275	-0.2594\\
-0.0142825	-0.29602\\
-0.014465	-0.17395\\
-0.0141	-0.0915525\\
-0.0134125	-0.2044675\\
-0.013825	-0.31433\\
-0.014465	-0.31433\\
-0.0145575	-0.357055\\
-0.014695	-0.29602\\
-0.014695	-0.2197275\\
-0.014375	-0.3082275\\
-0.0145575	-0.4302975\\
-0.01506	-0.4302975\\
-0.0151975	-0.302125\\
-0.0149225	-0.5188\\
-0.01538	-0.3875725\\
-0.015335	-0.3784175\\
-0.0151975	-0.43335\\
-0.01538	-0.4302975\\
-0.01538	-0.3265375\\
-0.0151525	-0.2807625\\
-0.01497	-0.4821775\\
-0.01538	-0.6652825\\
-0.0160225	-0.5065925\\
-0.01593	-0.31128\\
-0.015335	-0.29602\\
-0.015105	-0.2899175\\
-0.01506	-0.375365\\
-0.0152425	-0.43335\\
-0.0154275	-0.31433\\
-0.0152425	-0.3356925\\
-0.0151975	-0.442505\\
-0.0154725	-0.4821775\\
-0.01561	-0.302125\\
-0.0152425	-0.24414\\
-0.0149225	-0.3265375\\
-0.015105	-0.546265\\
-0.015655	-0.4882825\\
-0.0157925	-0.5096425\\
-0.0157925	-0.2899175\\
-0.01529	-0.2716075\\
-0.01497	-0.253295\\
-0.01497	-0.1586925\\
-0.014465	-0.10376\\
-0.013825	-0.08545\\
-0.0133675	-0.0732425\\
-0.0130475	-0.1953125\\
-0.0136425	-0.2716075\\
-0.0142375	-0.32959\\
-0.0145575	-0.234985\\
-0.01442	-0.268555\\
-0.01442	-0.2990725\\
-0.0145575	-0.1800525\\
-0.01419	-0.198365\\
-0.0140525	-0.183105\\
-0.0140525	-0.13733\\
-0.0137775	-0.18921\\
-0.013915	-0.302125\\
-0.014375	-0.390625\\
-0.0148325	-0.2380375\\
-0.01451	-0.183105\\
-0.01419	-0.146485\\
-0.013915	-0.1861575\\
-0.0139625	-0.112915\\
-0.0136875	-0.29297\\
-0.0143275	-0.4882825\\
-0.0151975	-0.408935\\
-0.0152425	-0.54016\\
-0.0155175	-0.6042475\\
-0.015885	-0.6225575\\
-0.0160225	-0.4547125\\
-0.0157925	-0.32959\\
-0.01538	-0.3265375\\
-0.0151975	-0.3234875\\
-0.0151975	-0.3875725\\
-0.015335	-0.460815\\
-0.0155175	-0.4547125\\
-0.015565	-0.6195075\\
-0.01593	-0.67749\\
-0.01625	-0.8209225\\
-0.0166175	-0.5432125\\
-0.0162975	-0.5950925\\
-0.01625	-0.6622325\\
-0.01648	-0.512695\\
-0.016205	-0.598145\\
-0.0162975	-0.5004875\\
-0.01616	-0.2838125\\
-0.0155175	-0.1861575\\
-0.014785	-0.198365\\
-0.0145575	-0.29297\\
-0.0149225	-0.24414\\
-0.0148775	-0.1617425\\
-0.014465	-0.22583\\
-0.01451	-0.1586925\\
-0.0143275	-0.12207\\
-0.013915	-0.1617425\\
-0.0140075	-0.1800525\\
-0.0140525	-0.12207\\
-0.01387	-0.250245\\
-0.0142375	-0.338745\\
-0.014785	-0.24414\\
-0.0146025	-0.43335\\
-0.01506	-0.6042475\\
-0.0158375	-0.5523675\\
-0.015975	-0.6988525\\
-0.0162975	-0.46997\\
-0.0160225	-0.5157475\\
-0.01593	-0.3356925\\
-0.01561	-0.2136225\\
-0.0148775	-0.2716075\\
-0.0148775	-0.2197275\\
-0.014785	-0.1586925\\
-0.014375	-0.2319325\\
-0.01451	-0.1251225\\
-0.0141	-0.0823975\\
-0.013505	-0.1098625\\
-0.0133675	-0.2471925\\
-0.0141	-0.31433\\
-0.0146025	-0.3936775\\
-0.01497	-0.38147\\
-0.015105	-0.36621\\
-0.015105	-0.4211425\\
-0.0151975	-0.338745\\
-0.015105	-0.2716075\\
-0.0148325	-0.2807625\\
-0.014785	-0.2227775\\
-0.0146475	-0.36621\\
-0.01497	-0.24109\\
-0.01474	-0.201415\\
-0.01442	-0.302125\\
-0.014695	-0.4119875\\
-0.0151525	-0.3082275\\
-0.01497	-0.234985\\
-0.014695	-0.201415\\
-0.01451	-0.2288825\\
-0.01451	-0.1525875\\
-0.0142825	-0.1525875\\
-0.0140075	-0.2197275\\
-0.0142375	-0.2746575\\
-0.01451	-0.31128\\
-0.0146475	-0.24109\\
-0.0146025	-0.320435\\
-0.014695	-0.286865\\
-0.01474	-0.1525875\\
-0.014145	-0.17395\\
-0.0140075	-0.338745\\
-0.014695	-0.476075\\
-0.01529	-0.46692\\
-0.01538	-0.3936775\\
-0.01529	-0.38147\\
-0.0152425	-0.286865\\
-0.01497	-0.2746575\\
-0.0148325	-0.2197275\\
-0.0146475	-0.31433\\
-0.0148325	-0.29297\\
-0.0148775	-0.1770025\\
-0.01442	-0.268555\\
-0.0146475	-0.3173825\\
-0.014785	-0.375365\\
-0.015015	-0.4119875\\
-0.0151975	-0.408935\\
-0.0152425	-0.55542\\
-0.01561	-0.668335\\
-0.0160675	-0.756835\\
-0.01648	-0.476075\\
-0.0160675	-0.2655025\\
-0.01529	-0.1708975\\
-0.0145575	-0.112915\\
-0.013915	-0.1770025\\
-0.0140525	-0.1617425\\
-0.0140075	-0.2990725\\
-0.01451	-0.2563475\\
-0.0146025	-0.234985\\
-0.01451	-0.31128\\
-0.01474	-0.36316\\
-0.01497	-0.4302975\\
-0.0152425	-0.4577625\\
-0.0154275	-0.6469725\\
-0.015975	-0.579835\\
-0.0161125	-0.442505\\
-0.0157925	-0.302125\\
-0.015335	-0.302125\\
-0.0151525	-0.3082275\\
-0.0151525	-0.3936775\\
-0.01538	-0.2716075\\
-0.0151525	-0.2838125\\
-0.01497	-0.268555\\
-0.01497	-0.146485\\
-0.0142825	-0.2563475\\
-0.01451	-0.38147\\
-0.01506	-0.268555\\
-0.0149225	-0.2899175\\
-0.0148325	-0.54016\\
-0.015565	-0.39978\\
-0.0155175	-0.390625\\
-0.01538	-0.5493175\\
-0.0157475	-0.5157475\\
-0.015885	-0.3234875\\
-0.01538	-0.2136225\\
-0.0148325	-0.2136225\\
-0.0146475	-0.36621\\
-0.015105	-0.59204\\
-0.0158375	-0.616455\\
-0.0161125	-0.6286625\\
-0.016205	-0.4821775\\
-0.015975	-0.305175\\
-0.01538	-0.2594\\
-0.015105	-0.1800525\\
-0.014695	-0.17395\\
-0.014465	-0.146485\\
-0.0142375	-0.2227775\\
-0.014465	-0.2563475\\
-0.0146475	-0.146485\\
-0.0142375	-0.22583\\
-0.014375	-0.3173825\\
-0.014785	-0.36316\\
-0.015015	-0.4638675\\
-0.01538	-0.582885\\
-0.0157925	-0.704955\\
-0.0162975	-0.494385\\
-0.0160225	-0.4302975\\
-0.0157475	-0.4455575\\
-0.0157925	-0.390625\\
-0.0157025	-0.2594\\
-0.0152425	-0.320435\\
-0.0151975	-0.53711\\
-0.0157925	-0.616455\\
-0.0161125	-0.354005\\
-0.01561	-0.198365\\
-0.0148775	-0.131225\\
-0.0142825	-0.07019\\
-0.0133675	-0.0885\\
-0.0130925	-0.1617425\\
-0.013505	-0.1861575\\
-0.0137325	-0.2563475\\
-0.014145	-0.2044675\\
-0.014145	-0.1617425\\
-0.013825	-0.15564\\
-0.0137775	-0.1525875\\
-0.0136875	-0.13733\\
-0.0136875	-0.164795\\
-0.0137325	-0.2563475\\
-0.0141	-0.372315\\
-0.014695	-0.3936775\\
-0.01497	-0.3875725\\
-0.01506	-0.4577625\\
-0.0152425	-0.564575\\
-0.015655	-0.564575\\
-0.015885	-0.67749\\
-0.01616	-0.6195075\\
-0.016205	-0.460815\\
-0.01593	-0.31433\\
-0.0154725	-0.305175\\
-0.0152425	-0.512695\\
-0.0157475	-0.59204\\
-0.0160675	-0.41504\\
-0.0157475	-0.268555\\
-0.0152425	-0.1342775\\
-0.014375	-0.1007075\\
-0.0136425	-0.1068125\\
-0.013505	-0.0640875\\
-0.013	-0.1525875\\
-0.0133675	-0.2105725\\
-0.01387	-0.2227775\\
-0.0140075	-0.1953125\\
-0.0139625	-0.1190175\\
-0.013505	-0.0885\\
-0.0131375	-0.131225\\
-0.01323	-0.14038\\
-0.0133675	-0.15564\\
-0.0134575	-0.1190175\\
-0.013275	-0.094605\\
-0.0130475	-0.17395\\
-0.0134575	-0.4119875\\
-0.0146475	-0.476075\\
-0.01538	-0.5432125\\
-0.015565	-0.3692625\\
-0.015335	-0.52185\\
-0.01561	-0.72937\\
-0.016205	-0.653075\\
-0.0163425	-0.683595\\
-0.0163875	-0.5004875\\
-0.0161125	-0.5065925\\
-0.0160675	-0.531005\\
-0.0160675	-0.3479\\
-0.0157025	-0.3326425\\
-0.0154275	-0.201415\\
-0.0148775	-0.18921\\
-0.014465	-0.338745\\
-0.015015	-0.234985\\
-0.0148325	-0.26245\\
-0.01474	-0.2899175\\
-0.0148775	-0.27771\\
-0.0148775	-0.2838125\\
-0.0148325	-0.302125\\
-0.0149225	-0.3417975\\
-0.01506	-0.3692625\\
-0.0151525	-0.31433\\
-0.01506	-0.2319325\\
-0.014785	-0.302125\\
-0.0149225	-0.1434325\\
-0.0143275	-0.0671375\\
-0.01332	-0.1159675\\
-0.0131825	-0.183105\\
-0.0136875	-0.094605\\
-0.013275	-0.042725\\
-0.0124975	-0.0976575\\
-0.012635	-0.1770025\\
-0.013275	-0.2105725\\
-0.013595	-0.26245\\
-0.0139625	-0.2197275\\
-0.0139625	-0.357055\\
-0.01451	-0.3082275\\
-0.0146475	-0.20752\\
-0.01419	-0.31433\\
-0.01451	-0.1770025\\
-0.0140075	-0.14038\\
-0.0136425	-0.0976575\\
-0.01323	-0.061035\\
-0.012725	-0.12207\\
-0.012955	-0.0915525\\
-0.0127725	-0.164795\\
-0.0131825	-0.27771\\
-0.013915	-0.2655025\\
-0.0141	-0.19226\\
-0.01387	-0.216675\\
-0.01387	-0.1617425\\
-0.0136875	-0.2319325\\
-0.013915	-0.1678475\\
-0.0136875	-0.1586925\\
-0.013505	-0.146485\\
-0.0134575	-0.1098625\\
-0.013275	-0.1434325\\
-0.013275	-0.128175\\
-0.01323	-0.22583\\
-0.0137325	-0.20752\\
-0.013825	-0.1678475\\
-0.0136425	-0.1617425\\
-0.013505	-0.128175\\
-0.01332	-0.1525875\\
-0.0133675	-0.338745\\
-0.0142375	-0.4486075\\
-0.01497	-0.3082275\\
-0.01474	-0.286865\\
-0.0145575	-0.201415\\
-0.01419	-0.3417975\\
-0.0146475	-0.31433\\
-0.01474	-0.201415\\
-0.0142825	-0.2563475\\
-0.0143275	-0.19226\\
-0.014145	-0.2716075\\
-0.0143275	-0.1617425\\
-0.0139625	-0.1617425\\
-0.0136875	-0.201415\\
-0.013915	-0.26245\\
-0.014145	-0.1861575\\
-0.0140075	-0.2044675\\
-0.013915	-0.2136225\\
-0.0140075	-0.338745\\
-0.014465	-0.2899175\\
-0.01451	-0.4455575\\
-0.01497	-0.579835\\
-0.01561	-0.4577625\\
-0.015565	-0.29297\\
-0.015015	-0.216675\\
-0.01451	-0.338745\\
-0.0148325	-0.43335\\
-0.0151975	-0.3326425\\
-0.015015	-0.408935\\
-0.0151525	-0.2471925\\
-0.01474	-0.1251225\\
-0.013915	-0.131225\\
-0.0136425	-0.2990725\\
-0.014375	-0.268555\\
-0.0145575	-0.250245\\
-0.01442	-0.390625\\
-0.0148325	-0.36621\\
-0.01497	-0.3326425\\
-0.0148325	-0.22583\\
-0.01451	-0.286865\\
-0.01451	-0.372315\\
-0.0148775	-0.302125\\
-0.01474	-0.31433\\
-0.014785	-0.2807625\\
-0.014695	-0.201415\\
-0.01442	-0.24414\\
-0.014465	-0.1770025\\
-0.01419	-0.198365\\
-0.014145	-0.338745\\
-0.0146025	-0.390625\\
-0.01497	-0.4119875\\
-0.015105	-0.36621\\
-0.015015	-0.26245\\
-0.014695	-0.1800525\\
-0.0142825	-0.2136225\\
-0.0142375	-0.253295\\
-0.01442	-0.3234875\\
-0.0146475	-0.39978\\
-0.01497	-0.476075\\
-0.01529	-0.36621\\
-0.0151525	-0.2105725\\
-0.0146025	-0.3875725\\
-0.01497	-0.55542\\
-0.01561	-0.3875725\\
-0.01538	-0.3875725\\
-0.0152425	-0.4547125\\
-0.0154275	-0.3448475\\
-0.0152425	-0.2838125\\
-0.0149225	-0.3234875\\
-0.015015	-0.2899175\\
-0.0149225	-0.32959\\
-0.01497	-0.41809\\
-0.0152425	-0.31433\\
-0.01506	-0.36621\\
-0.015105	-0.3448475\\
-0.015105	-0.354005\\
-0.015105	-0.2746575\\
-0.0149225	-0.3601075\\
-0.01506	-0.2563475\\
-0.0148775	-0.2655025\\
-0.01474	-0.2990725\\
-0.0148775	-0.2899175\\
-0.0148325	-0.146485\\
-0.0142375	-0.0885\\
-0.0134575	-0.07019\\
-0.01291	-0.1190175\\
-0.0130925	-0.305175\\
-0.01419	-0.4028325\\
-0.0148775	-0.4028325\\
-0.015015	-0.253295\\
-0.0146025	-0.2990725\\
-0.0146025	-0.26245\\
-0.0146025	-0.2319325\\
-0.014465	-0.146485\\
-0.0140075	-0.2044675\\
-0.0141	-0.2044675\\
-0.014145	-0.201415\\
-0.0141	-0.2380375\\
-0.0142375	-0.198365\\
-0.014145	-0.18921\\
-0.0140525	-0.1251225\\
-0.0137775	-0.1800525\\
-0.013825	-0.3479\\
-0.01451	-0.24109\\
-0.01442	-0.2990725\\
-0.01451	-0.268555\\
-0.014465	-0.36316\\
-0.014785	-0.338745\\
-0.0148325	-0.46692\\
-0.0151525	-0.3448475\\
-0.01506	-0.3784175\\
-0.015015	-0.3875725\\
-0.015105	-0.1800525\\
-0.01442	-0.32959\\
-0.01474	-0.2380375\\
-0.0145575	-0.2838125\\
-0.0146025	-0.320435\\
-0.01474	-0.41504\\
-0.015015	-0.3082275\\
-0.0148775	-0.3967275\\
-0.01506	-0.50354\\
-0.01538	-0.4119875\\
-0.015335	-0.357055\\
-0.015105	-0.2838125\\
-0.0149225	-0.3417975\\
-0.01497	-0.2838125\\
-0.0148325	-0.24109\\
-0.014695	-0.2044675\\
-0.014465	-0.24109\\
-0.0145575	-0.20752\\
-0.014465	-0.4302975\\
-0.01506	-0.564575\\
-0.01561	-0.479125\\
-0.01561	-0.46997\\
-0.0155175	-0.4547125\\
-0.015565	-0.253295\\
-0.01497	-0.2227775\\
-0.0146475	-0.1800525\\
-0.014375	-0.15564\\
-0.0141	-0.1708975\\
-0.014145	-0.2899175\\
-0.0146025	-0.36621\\
-0.0149225	-0.41809\\
-0.0151525	-0.354005\\
-0.01506	-0.234985\\
-0.014695	-0.424195\\
-0.0151525	-0.50354\\
-0.0155175	-0.5615225\\
-0.0157475	-0.4455575\\
-0.015565	-0.720215\\
-0.01616	-0.5249025\\
-0.0161125	-0.29297\\
-0.015335	-0.2105725\\
-0.014785	-0.17395\\
-0.01442	-0.31433\\
-0.0148325	-0.405885\\
-0.0152425	-0.546265\\
-0.0157025	-0.41504\\
-0.01561	-0.3173825\\
-0.0152425	-0.17395\\
-0.0145575	-0.1800525\\
-0.0142825	-0.1953125\\
-0.0143275	-0.2197275\\
-0.01442	-0.14038\\
-0.0140525	-0.1770025\\
-0.0141	-0.149535\\
-0.0139625	-0.0915525\\
-0.01355	-0.20752\\
-0.013915	-0.234985\\
-0.0142375	-0.29602\\
-0.014465	-0.268555\\
-0.01451	-0.2807625\\
-0.01451	-0.2838125\\
-0.0146025	-0.36621\\
-0.014785	-0.250245\\
-0.0146025	-0.357055\\
-0.0148325	-0.424195\\
-0.0151525	-0.3265375\\
-0.01497	-0.3479\\
-0.01497	-0.2990725\\
-0.0148775	-0.338745\\
-0.0149225	-0.479125\\
-0.015335	-0.372315\\
-0.0151975	-0.286865\\
-0.0149225	-0.320435\\
-0.01497	-0.2899175\\
-0.0149225	-0.198365\\
-0.01451	-0.305175\\
-0.01474	-0.4547125\\
-0.01529	-0.427245\\
-0.015335	-0.46997\\
-0.01538	-0.6988525\\
-0.0160675	-0.46997\\
-0.0158375	-0.39978\\
-0.015565	-0.302125\\
-0.0152425	-0.38147\\
-0.015335	-0.5615225\\
-0.0157925	-0.372315\\
-0.0155175	-0.3234875\\
-0.0152425	-0.46692\\
-0.015565	-0.2990725\\
-0.0152425	-0.1708975\\
-0.0146025	-0.2197275\\
-0.01451	-0.1708975\\
-0.014375	-0.131225\\
-0.0140075	-0.1434325\\
-0.0139625	-0.1251225\\
-0.01387	-0.12207\\
-0.0136875	-0.1617425\\
-0.01387	-0.2380375\\
-0.01419	-0.268555\\
-0.01442	-0.3692625\\
-0.0148325	-0.3479\\
-0.0149225	-0.41504\\
-0.0151525	-0.48523\\
-0.0154275	-0.4364025\\
-0.0154725	-0.31433\\
-0.015105	-0.3356925\\
-0.01506	-0.2990725\\
-0.015015	-0.3265375\\
-0.015015	-0.4577625\\
-0.01538	-0.3509525\\
-0.0151975	-0.3967275\\
-0.0151975	-0.3417975\\
-0.0151975	-0.3234875\\
-0.01506	-0.50354\\
-0.0154725	-0.41504\\
-0.0154725	-0.41809\\
-0.0154275	-0.38147\\
-0.015335	-0.2288825\\
-0.0148775	-0.146485\\
-0.0142825	-0.1159675\\
-0.01387	-0.2105725\\
-0.014145	-0.198365\\
-0.0142825	-0.2655025\\
-0.014375	-0.198365\\
-0.0143275	-0.2716075\\
-0.014465	-0.479125\\
-0.01529	-0.58899\\
-0.0157925	-0.4638675\\
-0.0157025	-0.476075\\
-0.01561	-0.424195\\
-0.01561	-0.29602\\
-0.0151975	-0.3326425\\
-0.0151525	-0.36316\\
-0.0152425	-0.31433\\
-0.0151525	-0.3601075\\
-0.0151975	-0.4547125\\
-0.0154725	-0.650025\\
-0.0160225	-0.5767825\\
-0.0161125	-0.53711\\
-0.015975	-0.598145\\
-0.0161125	-0.405885\\
-0.0157475	-0.43335\\
-0.0157025	-0.3479\\
-0.0155175	-0.354005\\
-0.0154275	-0.46692\\
-0.0157025	-0.5340575\\
-0.015885	-0.2105725\\
-0.01506	-0.250245\\
-0.014785	-0.2136225\\
-0.014695	-0.2990725\\
-0.0149225	-0.3082275\\
-0.015015	-0.18921\\
-0.0146025	-0.2471925\\
-0.014695	-0.43335\\
-0.01529	-0.7598875\\
-0.0162975	-0.7232675\\
-0.01657	-0.653075\\
-0.0164325	-0.527955\\
-0.01625	-0.6011975\\
-0.0163425	-0.7263175\\
-0.01657	-0.5432125\\
-0.0163425	-0.546265\\
-0.01625	-0.564575\\
-0.0162975	-0.38147\\
-0.01593	-0.32959\\
-0.01561	-0.424195\\
-0.0157925	-0.2899175\\
-0.0154725	-0.2044675\\
-0.01497	-0.286865\\
-0.0151525	-0.2105725\\
-0.0148775	-0.2471925\\
-0.0148775	-0.2380375\\
-0.0148775	-0.2899175\\
-0.015015	-0.3967275\\
-0.015335	-0.3173825\\
-0.0152425	-0.22583\\
-0.0149225	-0.2807625\\
-0.015015	-0.3234875\\
-0.015105	-0.3845225\\
-0.01529	-0.3234875\\
-0.0152425	-0.2655025\\
-0.015015	-0.41809\\
-0.01538	-0.39978\\
-0.0154725	-0.27771\\
-0.0151525	-0.4577625\\
-0.0155175	-0.4211425\\
-0.01561	-0.2990725\\
-0.01529	-0.2044675\\
-0.0148775	-0.146485\\
-0.01442	-0.112915\\
-0.0139625	-0.2471925\\
-0.01442	-0.24109\\
-0.0146025	-0.164795\\
-0.0143275	-0.1190175\\
-0.013915	-0.1434325\\
-0.01387	-0.2563475\\
-0.014375	-0.2990725\\
-0.0146475	-0.3967275\\
-0.01506	-0.45166\\
-0.015335	-0.4302975\\
-0.01538	-0.4547125\\
-0.0154725	-0.2807625\\
-0.015105	-0.1159675\\
-0.0141	-0.1770025\\
-0.0140075	-0.15564\\
-0.0140075	-0.1953125\\
-0.0141	-0.3173825\\
-0.0146475	-0.3479\\
-0.0149225	-0.50354\\
-0.0154275	-0.3326425\\
-0.0152425	-0.31128\\
-0.01497	-0.479125\\
-0.0154725	-0.357055\\
-0.01529	-0.2380375\\
-0.0148775	-0.17395\\
-0.014465	-0.234985\\
-0.0145575	-0.3234875\\
-0.0148325	-0.476075\\
-0.0154275	-0.3265375\\
-0.0151975	-0.2655025\\
-0.0148775	-0.2471925\\
-0.0148325	-0.305175\\
-0.0149225	-0.408935\\
-0.0152425	-0.6378175\\
-0.01593	-0.5493175\\
-0.0160225	-0.53711\\
-0.015975	-0.3448475\\
-0.0155175	-0.22583\\
-0.0149225	-0.268555\\
-0.0149225	-0.1251225\\
-0.014145	-0.1861575\\
-0.014145	-0.17395\\
-0.014145	-0.198365\\
-0.0141	-0.3356925\\
-0.014785	-0.4638675\\
-0.01529	-0.5432125\\
-0.0157025	-0.6958\\
-0.01616	-0.598145\\
-0.016205	-0.3234875\\
-0.0155175	-0.29602\\
-0.0151975	-0.2471925\\
-0.015015	-0.3417975\\
-0.0151975	-0.4486075\\
-0.0155175	-0.598145\\
-0.015975	-0.4486075\\
-0.0158375	-0.305175\\
-0.01538	-0.3479\\
-0.01538	-0.460815\\
-0.01561	-0.3326425\\
-0.01538	-0.2227775\\
-0.015015	-0.183105\\
-0.0146475	-0.1251225\\
-0.014145	-0.12207\\
-0.013825	-0.0915525\\
-0.013505	-0.1617425\\
-0.0137775	-0.2471925\\
-0.0142375	-0.268555\\
-0.01442	-0.201415\\
-0.0142825	-0.183105\\
-0.014145	-0.08545\\
-0.0134575	-0.042725\\
-0.012635	-0.076295\\
-0.0124975	-0.1434325\\
-0.0130475	-0.2197275\\
-0.0136875	-0.27771\\
-0.014145	-0.3265375\\
-0.014465	-0.375365\\
-0.014785	-0.476075\\
-0.0152425	-0.6652825\\
-0.015975	-0.653075\\
-0.016205	-0.701905\\
-0.0163425	-0.6225575\\
-0.0162975	-0.3417975\\
-0.015655	-0.3082275\\
-0.015335	-0.31433\\
-0.01529	-0.41504\\
-0.0155175	-0.36316\\
-0.0155175	-0.2563475\\
-0.015105	-0.198365\\
-0.014695	-0.1586925\\
-0.01442	-0.27771\\
-0.014695	-0.2716075\\
-0.014785	-0.14038\\
-0.0142825	-0.10376\\
-0.0137775	-0.17395\\
-0.0139625	-0.24109\\
-0.0143275	-0.19226\\
-0.01419	-0.2655025\\
-0.01442	-0.216675\\
-0.014375	-0.305175\\
-0.0146025	-0.4547125\\
-0.0152425	-0.36621\\
-0.0152425	-0.476075\\
-0.0154275	-0.650025\\
-0.0160225	-0.704955\\
-0.0163425	-0.5584725\\
-0.01616	-0.36316\\
-0.0157025	-0.268555\\
-0.0151975	-0.1617425\\
-0.0145575	-0.253295\\
-0.0146475	-0.18921\\
-0.014465	-0.3173825\\
-0.0148775	-0.2899175\\
-0.01497	-0.2319325\\
-0.014695	-0.2990725\\
-0.0148325	-0.17395\\
-0.014465	-0.2471925\\
-0.0145575	-0.18921\\
-0.01442	-0.1159675\\
-0.01387	-0.13733\\
-0.0137775	-0.1068125\\
-0.013595	-0.1190175\\
-0.01355	-0.1159675\\
-0.0134575	-0.076295\\
-0.0131375	-0.1068125\\
-0.0131375	-0.15564\\
-0.0134125	-0.1525875\\
-0.013505	-0.1525875\\
-0.013505	-0.24414\\
-0.0139625	-0.3265375\\
-0.014465	-0.2716075\\
-0.01442	-0.2471925\\
-0.0143275	-0.390625\\
-0.0148325	-0.476075\\
-0.01529	-0.375365\\
-0.0151975	-0.390625\\
-0.0151525	-0.1861575\\
-0.01442	-0.1098625\\
-0.0136875	-0.22583\\
-0.0140525	-0.3326425\\
-0.014695	-0.36316\\
-0.0149225	-0.5065925\\
-0.01538	-0.4577625\\
-0.0155175	-0.3265375\\
-0.0151525	-0.2227775\\
-0.014695	-0.234985\\
-0.0145575	-0.2563475\\
-0.0146475	-0.2044675\\
-0.014465	-0.12207\\
-0.0139625	-0.1617425\\
-0.01387	-0.131225\\
-0.0137775	-0.10376\\
-0.013505	-0.0823975\\
-0.0131825	-0.1525875\\
-0.0134575	-0.234985\\
-0.0139625	-0.250245\\
-0.014145	-0.17395\\
-0.0139625	-0.3082275\\
-0.0143275	-0.4302975\\
-0.015015	-0.2746575\\
-0.014695	-0.36621\\
-0.0148775	-0.408935\\
-0.015105	-0.29602\\
-0.0148775	-0.17395\\
-0.0142825	-0.216675\\
-0.0142375	-0.2716075\\
-0.014465	-0.149535\\
-0.0140075	-0.17395\\
-0.0139625	-0.1342775\\
-0.013825	-0.079345\\
-0.01332	-0.079345\\
-0.013	-0.14038\\
-0.01332	-0.19226\\
-0.0136425	-0.1678475\\
-0.0136875	-0.20752\\
-0.013825	-0.250245\\
-0.0140525	-0.183105\\
-0.01387	-0.094605\\
-0.013275	-0.0823975\\
-0.012955	-0.10376\\
-0.013	-0.1251225\\
-0.0131375	-0.1525875\\
-0.01332	-0.320435\\
-0.01419	-0.3265375\\
-0.0146025	-0.2746575\\
-0.01442	-0.268555\\
-0.01442	-0.354005\\
-0.0146475	-0.4638675\\
-0.0151525	-0.50354\\
-0.0154275	-0.512695\\
-0.015565	-0.3784175\\
-0.01529	-0.3936775\\
-0.0152425	-0.405885\\
-0.01529	-0.41504\\
-0.01529	-0.4638675\\
-0.0154725	-0.616455\\
-0.0158375	-0.54016\\
-0.01593	-0.4882825\\
-0.0157925	-0.3936775\\
-0.015565	-0.3692625\\
-0.0154275	-0.357055\\
-0.01538	-0.41809\\
-0.0154725	-0.2594\\
-0.015105	-0.302125\\
-0.01497	-0.372315\\
-0.0151975	-0.4486075\\
-0.0154275	-0.3448475\\
-0.01529	-0.3875725\\
-0.015335	-0.3234875\\
-0.0151975	-0.2807625\\
-0.015015	-0.1770025\\
-0.0145575	-0.27771\\
-0.01474	-0.442505\\
-0.01529	-0.354005\\
-0.01529	-0.2044675\\
-0.014695	-0.24414\\
-0.0146025	-0.1342775\\
-0.0142825	-0.1251225\\
-0.01387	-0.1770025\\
-0.0140075	-0.112915\\
-0.0137325	-0.1159675\\
-0.013505	-0.13733\\
-0.013595	-0.0915525\\
-0.01332	-0.15564\\
-0.013505	-0.128175\\
-0.013505	-0.076295\\
-0.0130925	-0.1434325\\
-0.01332	-0.1617425\\
-0.0134575	-0.201415\\
-0.0137325	-0.18921\\
-0.0137775	-0.19226\\
-0.0137325	-0.26245\\
-0.0140525	-0.2319325\\
-0.014145	-0.26245\\
-0.01419	-0.45166\\
-0.01497	-0.320435\\
-0.0149225	-0.2807625\\
-0.0146475	-0.305175\\
-0.01474	-0.216675\\
-0.014465	-0.128175\\
-0.013915	-0.2319325\\
-0.0140525	-0.183105\\
-0.0141	-0.1098625\\
-0.013595	-0.2044675\\
-0.01387	-0.201415\\
-0.0140075	-0.2471925\\
-0.01419	-0.1525875\\
-0.01387	-0.0915525\\
-0.0133675	-0.0732425\\
-0.013	-0.0915525\\
-0.012955	-0.17395\\
-0.0134125	-0.1708975\\
-0.013595	-0.201415\\
-0.0136875	-0.2746575\\
-0.0141	-0.375365\\
-0.0146475	-0.2746575\\
-0.01451	-0.38147\\
-0.014785	-0.442505\\
-0.0151525	-0.3845225\\
-0.01506	-0.36316\\
-0.015015	-0.5767825\\
-0.0155175	-0.41809\\
-0.0155175	-0.52185\\
-0.01561	-0.4455575\\
-0.01561	-0.5096425\\
-0.015655	-0.5859375\\
-0.01593	-0.3417975\\
-0.01538	-0.2044675\\
-0.01474	-0.164795\\
-0.0143275	-0.2716075\\
-0.0145575	-0.3356925\\
-0.0149225	-0.3448475\\
-0.01497	-0.22583\\
-0.0146475	-0.1251225\\
-0.0140075	-0.1678475\\
-0.013915	-0.320435\\
-0.0146025	-0.4486075\\
-0.0152425	-0.442505\\
-0.015335	-0.2594\\
-0.0148775	-0.2044675\\
-0.01451	-0.2746575\\
-0.014695	-0.4394525\\
-0.0152425	-0.5004875\\
-0.015565	-0.5096425\\
-0.015655	-0.375365\\
-0.0154275	-0.52185\\
-0.015565	-0.5676275\\
-0.015885	-0.531005\\
-0.015885	-0.7263175\\
-0.01625	-0.6286625\\
-0.0163425	-0.5249025\\
-0.0161125	-0.6042475\\
-0.016205	-0.5188\\
-0.01616	-0.2655025\\
-0.015335	-0.250245\\
-0.01497	-0.3173825\\
-0.015105	-0.3875725\\
-0.01538	-0.3967275\\
-0.0154725	-0.2899175\\
-0.0152425	-0.372315\\
-0.01529	-0.3265375\\
-0.01529	-0.24109\\
-0.01497	-0.3234875\\
-0.015105	-0.305175\\
-0.0151525	-0.4394525\\
-0.0154275	-0.4547125\\
-0.01561	-0.3356925\\
-0.015335	-0.24414\\
-0.015015	-0.146485\\
-0.014375	-0.24109\\
-0.01451	-0.1861575\\
-0.014375	-0.164795\\
-0.01419	-0.1342775\\
-0.0140075	-0.2380375\\
-0.014375	-0.3082275\\
-0.014695	-0.3448475\\
-0.0149225	-0.3417975\\
-0.015015	-0.201415\\
-0.0145575	-0.15564\\
-0.014145	-0.1068125\\
-0.013825	-0.13733\\
-0.0137325	-0.2288825\\
-0.014145	-0.2380375\\
-0.0142825	-0.2288825\\
-0.0142825	-0.375365\\
-0.0148325	-0.4364025\\
-0.01529	-0.4974375\\
-0.0155175	-0.4974375\\
-0.01561	-0.57373\\
-0.0158375	-0.3448475\\
-0.0154275	-0.2838125\\
-0.01506	-0.2197275\\
-0.014785	-0.234985\\
-0.0146475	-0.320435\\
-0.0149225	-0.2655025\\
-0.0148775	-0.17395\\
-0.014465	-0.1617425\\
-0.0142375	-0.31128\\
-0.01474	-0.3845225\\
-0.015105	-0.3356925\\
-0.015015	-0.527955\\
-0.0155175	-0.5767825\\
-0.015885	-0.39978\\
-0.01561	-0.27771\\
-0.015105	-0.19226\\
-0.0146475	-0.1770025\\
-0.01442	-0.32959\\
-0.0148775	-0.2655025\\
-0.0148775	-0.305175\\
-0.0148775	-0.3479\\
-0.01506	-0.390625\\
-0.0151525	-0.3784175\\
-0.0151975	-0.4211425\\
-0.015335	-0.2899175\\
-0.0151525	-0.3265375\\
-0.01506	-0.2288825\\
-0.0148325	-0.2105725\\
-0.0145575	-0.3234875\\
-0.0149225	-0.4577625\\
-0.01538	-0.50354\\
-0.01561	-0.2655025\\
-0.01506	-0.5523675\\
-0.01561	-0.5188\\
-0.015975	-0.3479\\
-0.0154725	-0.19226\\
-0.014785	-0.3082275\\
-0.0149225	-0.268555\\
-0.01497	-0.26245\\
-0.0148325	-0.3082275\\
-0.015015	-0.201415\\
-0.014695	-0.2716075\\
-0.014785	-0.5188\\
-0.0155175	-0.5493175\\
-0.015885	-0.36316\\
-0.0154725	-0.39978\\
-0.0154725	-0.442505\\
-0.015565	-0.45166\\
-0.01561	-0.32959\\
-0.01538	-0.320435\\
-0.0152425	-0.2105725\\
-0.0148775	-0.3326425\\
-0.015015	-0.3356925\\
-0.0151975	-0.57373\\
-0.0157925	-0.6286625\\
-0.0161125	-0.8300775\\
-0.0166175	-0.5950925\\
-0.01648	-0.48523\\
-0.0161125	-0.3692625\\
-0.0157925	-0.405885\\
-0.0157025	-0.305175\\
-0.0155175	-0.2105725\\
-0.015015	-0.2044675\\
-0.01474	-0.2136225\\
-0.01474	-0.15564\\
-0.014465	-0.1159675\\
-0.0140525	-0.1861575\\
-0.01419	-0.250245\\
-0.01451	-0.1708975\\
-0.0142825	-0.1159675\\
-0.013915	-0.1159675\\
-0.0137325	-0.2807625\\
-0.01442	-0.3936775\\
-0.01506	-0.183105\\
-0.01451	-0.2380375\\
-0.01442	-0.24414\\
-0.0146025	-0.2288825\\
-0.01451	-0.2838125\\
-0.0146475	-0.2471925\\
-0.0146025	-0.1708975\\
-0.0142825	-0.12207\\
-0.013915	-0.1007075\\
-0.0136425	-0.1342775\\
-0.0136875	-0.2655025\\
-0.0142825	-0.3173825\\
-0.0146475	-0.24109\\
-0.01451	-0.1586925\\
-0.0141	-0.0976575\\
-0.013595	-0.14038\\
-0.013595	-0.24109\\
-0.01419	-0.3082275\\
-0.0145575	-0.3326425\\
-0.014695	-0.22583\\
-0.014465	-0.216675\\
-0.0143275	-0.2380375\\
-0.014375	-0.3326425\\
-0.014695	-0.4638675\\
-0.0151975	-0.54016\\
-0.015655	-0.4119875\\
-0.0155175	-0.2563475\\
-0.01497	-0.268555\\
-0.014785	-0.250245\\
-0.01474	-0.2594\\
-0.014785	-0.164795\\
-0.01442	-0.2197275\\
-0.01442	-0.234985\\
-0.014465	-0.2288825\\
-0.01451	-0.14038\\
-0.014145	-0.0915525\\
-0.013595	-0.14038\\
-0.0136425	-0.2105725\\
-0.0140525	-0.3356925\\
-0.0146025	-0.372315\\
-0.0149225	-0.4486075\\
-0.0151975	-0.39978\\
-0.01529	-0.45166\\
-0.01538	-0.598145\\
-0.0158375	-0.79956\\
-0.01648	-0.634765\\
-0.01648	-0.43335\\
-0.0160225	-0.4730225\\
-0.015975	-0.5523675\\
-0.0161125	-0.720215\\
-0.01648	-0.45166\\
-0.01616	-0.2227775\\
-0.0151525	-0.149535\\
-0.014465	-0.15564\\
-0.01419	-0.1068125\\
-0.01387	-0.07019\\
-0.01332	-0.0976575\\
-0.0131825	-0.149535\\
-0.013505	-0.20752\\
-0.01387	-0.216675\\
-0.0140075	-0.164795\\
-0.01387	-0.15564\\
-0.0137325	-0.1953125\\
-0.01387	-0.250245\\
-0.014145	-0.26245\\
-0.0142825	-0.24109\\
-0.0142825	-0.1708975\\
-0.0140075	-0.1190175\\
-0.0136425	-0.112915\\
-0.0134575	-0.1800525\\
-0.0137325	-0.2197275\\
-0.0140075	-0.1617425\\
-0.01387	-0.10376\\
-0.0134575	-0.19226\\
-0.0137775	-0.12207\\
-0.013595	-0.17395\\
-0.0136425	-0.198365\\
-0.013825	-0.302125\\
-0.0142825	-0.3173825\\
-0.0145575	-0.2105725\\
-0.0142375	-0.31433\\
-0.014465	-0.3082275\\
-0.0146475	-0.253295\\
-0.01451	-0.2105725\\
-0.0142825	-0.357055\\
-0.014695	-0.4394525\\
-0.0151525	-0.4394525\\
-0.0152425	-0.48523\\
-0.0154275	-0.3692625\\
-0.01529	-0.3417975\\
-0.0151525	-0.3173825\\
-0.01506	-0.43335\\
-0.01529	-0.338745\\
-0.0151525	-0.460815\\
-0.01538	-0.4302975\\
-0.0154725	-0.4455575\\
-0.0154725	-0.7598875\\
-0.0162975	-0.579835\\
-0.0163875	-0.31433\\
-0.0155175	-0.20752\\
-0.0148775	-0.2197275\\
-0.014695	-0.2807625\\
-0.0148775	-0.36621\\
-0.0151525	-0.4302975\\
-0.0154275	-0.4730225\\
-0.01561	-0.4821775\\
-0.0157025	-0.31433\\
-0.015335	-0.27771\\
-0.015105	-0.3265375\\
-0.0151975	-0.3234875\\
-0.0151525	-0.19226\\
-0.01474	-0.1434325\\
-0.0142375	-0.2594\\
-0.0145575	-0.253295\\
-0.014695	-0.357055\\
-0.015015	-0.4364025\\
-0.01538	-0.3875725\\
-0.015335	-0.2655025\\
-0.01497	-0.2136225\\
-0.014695	-0.338745\\
-0.01497	-0.4211425\\
-0.015335	-0.5767825\\
-0.0157925	-0.6103525\\
-0.0161125	-0.7171625\\
-0.0163875	-0.59204\\
-0.0162975	-0.54016\\
-0.01616	-0.3082275\\
-0.0155175	-0.2136225\\
-0.0149225	-0.3845225\\
-0.015335	-0.5065925\\
-0.0157925	-0.2899175\\
-0.015335	-0.46692\\
-0.01561	-0.6225575\\
-0.0161125	-0.772095\\
-0.01657	-0.4730225\\
-0.016205	-0.2807625\\
-0.0154725	-0.1586925\\
-0.0146475	-0.2380375\\
-0.014695	-0.2136225\\
-0.014695	-0.234985\\
-0.014695	-0.128175\\
-0.0142375	-0.19226\\
-0.0142375	-0.131225\\
-0.0140525	-0.1861575\\
-0.0141	-0.146485\\
-0.0140525	-0.198365\\
-0.01419	-0.31128\\
-0.014695	-0.2807625\\
-0.014785	-0.405885\\
-0.01506	-0.479125\\
-0.0154725	-0.476075\\
-0.015565	-0.2594\\
-0.015015	-0.2746575\\
-0.0148325	-0.3265375\\
-0.01506	-0.2136225\\
-0.014695	-0.201415\\
-0.01451	-0.13733\\
-0.0141	-0.22583\\
-0.0143275	-0.198365\\
-0.014375	-0.1190175\\
-0.0139625	-0.0671375\\
-0.0133675	-0.079345\\
-0.0130475	-0.146485\\
-0.0134125	-0.2136225\\
-0.013915	-0.2288825\\
-0.0141	-0.14038\\
-0.0137775	-0.1678475\\
-0.0137775	-0.2563475\\
-0.014145	-0.338745\\
-0.0146025	-0.234985\\
-0.01442	-0.2197275\\
-0.0142375	-0.26245\\
-0.014375	-0.31433\\
-0.0146025	-0.2899175\\
-0.0146475	-0.3448475\\
-0.01474	-0.3173825\\
-0.014785	-0.2227775\\
-0.014465	-0.1800525\\
-0.01419	-0.2380375\\
-0.0143275	-0.1953125\\
-0.0142375	-0.24414\\
-0.0143275	-0.3234875\\
-0.014695	-0.5096425\\
-0.01529	-0.3479\\
-0.0152425	-0.3509525\\
-0.01506	-0.5157475\\
-0.0155175	-0.38147\\
-0.0154275	-0.24109\\
-0.0148775	-0.17395\\
-0.014465	-0.13733\\
-0.0140525	-0.146485\\
-0.0139625	-0.1159675\\
-0.013825	-0.1251225\\
-0.0136875	-0.2655025\\
-0.0142825	-0.112915\\
-0.0137325	-0.1251225\\
-0.0133675	-0.164795\\
-0.0136875	-0.234985\\
-0.0140525	-0.17395\\
-0.0140075	-0.1342775\\
-0.0136425	-0.07019\\
-0.0131825	-0.13733\\
-0.01332	-0.253295\\
-0.0140075	-0.3479\\
-0.0146025	-0.2716075\\
-0.01451	-0.2227775\\
-0.0142825	-0.253295\\
-0.014375	-0.2563475\\
-0.01442	-0.2990725\\
-0.0145575	-0.3601075\\
-0.014785	-0.4211425\\
-0.015015	-0.4119875\\
-0.0151525	-0.3509525\\
-0.01506	-0.2288825\\
-0.0146475	-0.183105\\
-0.0143275	-0.198365\\
-0.0142375	-0.2807625\\
-0.01451	-0.1617425\\
-0.01419	-0.15564\\
-0.013915	-0.286865\\
-0.01442	-0.3479\\
-0.014785	-0.3326425\\
-0.0148775	-0.3936775\\
-0.01497	-0.512695\\
-0.0154275	-0.26245\\
-0.0148775	-0.460815\\
-0.01529	-0.4638675\\
-0.01561	-0.427245\\
-0.0154725	-0.460815\\
-0.0155175	-0.4547125\\
-0.015565	-0.78125\\
-0.0163425	-0.7171625\\
-0.0166625	-0.4882825\\
-0.016205	-0.59204\\
-0.01625	-0.7080075\\
-0.016525	-0.6896975\\
-0.01657	-0.7781975\\
-0.016755	-0.704955\\
-0.016755	-0.930785\\
-0.01703	-0.701905\\
-0.0169825	-0.4913325\\
-0.01648	-0.29602\\
-0.0157925	-0.31128\\
-0.0155175	-0.2319325\\
-0.0152425	-0.198365\\
-0.0148775	-0.3082275\\
-0.0151525	-0.4302975\\
-0.01561	-0.26245\\
-0.0152425	-0.2319325\\
-0.0149225	-0.2288825\\
-0.0149225	-0.149535\\
-0.01451	-0.201415\\
-0.014465	-0.0915525\\
-0.01387	-0.13733\\
-0.0136425	-0.1098625\\
-0.01355	-0.164795\\
-0.0137325	-0.112915\\
-0.013595	-0.1007075\\
-0.01332	-0.0579825\\
-0.01291	-0.0579825\\
-0.012635	-0.1159675\\
-0.012955	-0.076295\\
-0.0128175	-0.094605\\
-0.0127725	-0.2044675\\
-0.0134575	-0.2807625\\
-0.0141	-0.2899175\\
-0.0143275	-0.2044675\\
-0.0141	-0.268555\\
-0.0142375	-0.41504\\
-0.0149225	-0.201415\\
-0.014375	-0.12207\\
-0.0136875	-0.094605\\
-0.01332	-0.07019\\
-0.0128625	-0.12207\\
-0.0130925	-0.2044675\\
-0.0136425	-0.3082275\\
-0.0142825	-0.1800525\\
-0.0140075	-0.3234875\\
-0.01442	-0.43335\\
-0.01506	-0.4394525\\
-0.0152425	-0.3265375\\
-0.015015	-0.18921\\
-0.014375	-0.250245\\
-0.014375	-0.338745\\
-0.01474	-0.4455575\\
-0.0152425	-0.234985\\
-0.01474	-0.1342775\\
-0.0139625	-0.2105725\\
-0.0141	-0.1586925\\
-0.0140075	-0.079345\\
-0.01332	-0.0549325\\
-0.012635	-0.13733\\
-0.0130475	-0.3173825\\
-0.01419	-0.3173825\\
-0.0146025	-0.2319325\\
-0.0143275	-0.1434325\\
-0.013825	-0.1525875\\
-0.0136875	-0.0457775\\
-0.01268	-0.0885\\
-0.0125425	-0.076295\\
-0.012635	-0.146485\\
-0.012955	-0.2105725\\
-0.013505	-0.32959\\
-0.0142375	-0.338745\\
-0.0145575	-0.372315\\
-0.014695	-0.3845225\\
-0.0148775	-0.372315\\
-0.0149225	-0.36316\\
-0.0148775	-0.31433\\
-0.014785	-0.3875725\\
-0.01497	-0.59204\\
-0.015655	-0.5340575\\
-0.0158375	-0.27771\\
-0.015105	-0.146485\\
-0.01419	-0.338745\\
-0.014785	-0.3967275\\
-0.01529	-0.357055\\
-0.0151525	-0.1953125\\
-0.0145575	-0.20752\\
-0.0142825	-0.357055\\
-0.0149225	-0.5432125\\
-0.01561	-0.5615225\\
-0.0158375	-0.62561\\
-0.0160675	-0.598145\\
-0.01616	-0.4455575\\
-0.0158375	-0.4577625\\
-0.0157925	-0.2105725\\
-0.01497	-0.198365\\
-0.014465	-0.253295\\
-0.0146475	-0.3173825\\
-0.0148775	-0.3356925\\
-0.015015	-0.357055\\
-0.015105	-0.216675\\
-0.01474	-0.31128\\
-0.0148775	-0.2563475\\
-0.0148325	-0.149535\\
-0.0142825	-0.0976575\\
-0.0136425	-0.0823975\\
-0.01323	-0.1342775\\
-0.0134575	-0.1007075\\
-0.013275	-0.0579825\\
-0.0128175	-0.12207\\
-0.0130475	-0.250245\\
-0.013915	-0.1617425\\
-0.0137775	-0.164795\\
-0.01355	-0.216675\\
-0.013825	-0.36621\\
-0.0145575	-0.24414\\
-0.014465	-0.131225\\
-0.0137775	-0.0732425\\
-0.0130925	-0.2044675\\
-0.0136425	-0.39978\\
-0.01474	-0.50354\\
-0.01538	-0.701905\\
-0.0161125	-0.83313\\
-0.0167075	-0.808715\\
-0.016845	-0.5188\\
-0.0164325	-0.531005\\
-0.01625	-0.69275\\
-0.01657	-0.579835\\
-0.016525	-0.53711\\
-0.0163425	-0.6073\\
-0.01648	-0.65918\\
-0.0166175	-0.67749\\
-0.0166625	-0.894165\\
-0.01703	-0.5950925\\
-0.0168	-0.338745\\
-0.015975	-0.1770025\\
-0.015015	-0.149535\\
-0.014375	-0.1678475\\
-0.0143275	-0.1678475\\
-0.0142825	-0.17395\\
-0.0142825	-0.3173825\\
-0.0148325	-0.234985\\
-0.014785	-0.268555\\
-0.01474	-0.3417975\\
-0.015015	-0.1770025\\
-0.01451	-0.2136225\\
-0.014465	-0.29297\\
-0.014785	-0.3509525\\
-0.015015	-0.19226\\
-0.0146025	-0.128175\\
-0.0140525	-0.1708975\\
-0.0141	-0.1617425\\
-0.0141	-0.41809\\
-0.015105	-0.5676275\\
-0.015885	-0.53711\\
-0.015975	-0.6225575\\
-0.01616	-0.686645\\
-0.01648	-0.9002675\\
-0.0169825	-0.76599\\
-0.01703	-0.476075\\
-0.0164325	-0.31433\\
-0.0157925	-0.17395\\
-0.0148775	-0.1953125\\
-0.0145575	-0.2044675\\
-0.0146475	-0.2746575\\
-0.0148325	-0.1800525\\
-0.0146025	-0.1708975\\
-0.0143275	-0.18921\\
-0.01442	-0.2380375\\
-0.0145575	-0.268555\\
-0.01474	-0.338745\\
-0.01506	-0.2380375\\
-0.0148325	-0.112915\\
-0.0141	-0.08545\\
-0.013505	-0.07019\\
-0.0130925	-0.0823975\\
-0.013	-0.0976575\\
-0.0130475	-0.1800525\\
-0.01355	-0.183105\\
-0.0137325	-0.2380375\\
-0.0140525	-0.3509525\\
-0.0146025	-0.250245\\
-0.0145575	-0.4211425\\
-0.015015	-0.6103525\\
-0.01593	-0.476075\\
-0.015885	-0.4302975\\
-0.0157475	-0.5188\\
-0.015885	-0.598145\\
-0.0161125	-0.372315\\
-0.0157025	-0.3265375\\
-0.0154725	-0.201415\\
-0.0149225	-0.2319325\\
-0.014785	-0.46997\\
-0.015565	-0.7873525\\
-0.01657	-0.927735\\
-0.017165	-0.7354725\\
-0.017075	-0.6195075\\
-0.0168	-0.4394525\\
-0.0164325	-0.479125\\
-0.0163425	-0.64087\\
-0.0166625	-0.3692625\\
-0.01616	-0.320435\\
-0.0157025	-0.24109\\
-0.0154275	-0.4638675\\
-0.015885	-0.43335\\
-0.0160675	-0.2319325\\
-0.01538	-0.22583\\
-0.01506	-0.164795\\
-0.01474	-0.2990725\\
-0.01506	-0.45166\\
-0.0157025	-0.46692\\
-0.0158375	-0.354005\\
-0.0157025	-0.26245\\
-0.0152425	-0.2899175\\
-0.01529	-0.2227775\\
-0.01506	-0.1617425\\
-0.0146475	-0.131225\\
-0.0142825	-0.112915\\
-0.0140075	-0.0915525\\
-0.0136425	-0.1159675\\
-0.013595	-0.19226\\
-0.0140075	-0.2838125\\
-0.0146025	-0.2838125\\
-0.01474	-0.1800525\\
-0.0143275	-0.15564\\
-0.0141	-0.128175\\
-0.0139625	-0.1708975\\
-0.0139625	-0.2380375\\
-0.014375	-0.2594\\
-0.014465	-0.250245\\
-0.01451	-0.13733\\
-0.0140075	-0.3082275\\
-0.0146475	-0.4302975\\
-0.015335	-0.31128\\
-0.015105	-0.1251225\\
-0.014145	-0.1525875\\
-0.01387	-0.2380375\\
-0.0143275	-0.2380375\\
-0.01442	-0.2105725\\
-0.0143275	-0.128175\\
-0.013915	-0.234985\\
-0.0142375	-0.3601075\\
-0.0149225	-0.216675\\
-0.0146025	-0.0823975\\
-0.013595	-0.1068125\\
-0.0134125	-0.27771\\
-0.0143275	-0.3479\\
-0.0148325	-0.2044675\\
-0.01442	-0.1708975\\
-0.0141	-0.3326425\\
-0.014695	-0.4486075\\
-0.015335	-0.3936775\\
-0.015335	-0.55542\\
-0.0157025	-0.6713875\\
-0.0162975	-0.43335\\
-0.01593	-0.3479\\
-0.015565	-0.494385\\
-0.0158375	-0.3326425\\
-0.015565	-0.4486075\\
-0.0157475	-0.5523675\\
-0.0161125	-0.4302975\\
-0.015885	-0.2105725\\
-0.015105	-0.357055\\
-0.01529	-0.5950925\\
-0.0160675	-0.701905\\
-0.01648	-0.512695\\
-0.0162975	-0.4394525\\
-0.0160675	-0.57373\\
-0.0162975	-0.4394525\\
-0.0160675	-0.2807625\\
-0.0155175	-0.253295\\
-0.01529	-0.3265375\\
-0.0154275	-0.3417975\\
-0.0154725	-0.3417975\\
-0.0154725	-0.3875725\\
-0.01561	-0.2655025\\
-0.01529	-0.286865\\
-0.01529	-0.408935\\
-0.015565	-0.2319325\\
-0.0151525	-0.19226\\
-0.014785	-0.1525875\\
-0.014465	-0.12207\\
-0.014145	-0.1434325\\
-0.0141	-0.2563475\\
-0.01451	-0.234985\\
-0.014695	-0.3326425\\
-0.01497	-0.424195\\
-0.0154275	-0.5065925\\
-0.0157475	-0.3936775\\
-0.015655	-0.3417975\\
-0.0154275	-0.1525875\\
-0.0146025	-0.2380375\\
-0.0146025	-0.41809\\
-0.015335	-0.27771\\
-0.015105	-0.14038\\
-0.014375	-0.2838125\\
-0.01474	-0.4302975\\
-0.01538	-0.4455575\\
-0.01561	-0.582885\\
-0.015975	-0.76294\\
-0.01657	-0.5340575\\
-0.0163875	-0.4547125\\
-0.0160675	-0.54016\\
-0.01625	-0.3509525\\
-0.015885	-0.2838125\\
-0.0155175	-0.3234875\\
-0.0154725	-0.4119875\\
-0.0157475	-0.4364025\\
-0.0158375	-0.442505\\
-0.0158375	-0.24414\\
-0.015335	-0.216675\\
-0.01497	-0.17395\\
-0.014695	-0.253295\\
-0.0148775	-0.2319325\\
-0.0149225	-0.1861575\\
-0.014695	-0.3784175\\
-0.0152425	-0.405885\\
-0.015565	-0.2807625\\
-0.01529	-0.234985\\
-0.015015	-0.3417975\\
-0.0152425	-0.1708975\\
-0.014785	-0.1098625\\
-0.0141	-0.13733\\
-0.0140075	-0.1953125\\
-0.0142825	-0.1708975\\
-0.0142375	-0.1190175\\
-0.013915	-0.0671375\\
-0.0133675	-0.12207\\
-0.0134125	-0.1953125\\
-0.013915	-0.31128\\
-0.0145575	-0.357055\\
-0.01497	-0.46997\\
-0.0154275	-0.5249025\\
-0.0157925	-0.268555\\
-0.0151975	-0.20752\\
-0.014695	-0.4455575\\
-0.0154725	-0.653075\\
-0.016205	-0.5065925\\
-0.0161125	-0.4730225\\
-0.015975	-0.7080075\\
-0.01648	-0.5676275\\
-0.0164325	-0.4638675\\
-0.0161125	-0.3326425\\
-0.0157475	-0.3509525\\
-0.015655	-0.4638675\\
-0.01593	-0.3234875\\
-0.015655	-0.2807625\\
-0.015335	-0.4730225\\
-0.015885	-0.59204\\
-0.01625	-0.6195075\\
-0.0164325	-0.27771\\
-0.015565	-0.131225\\
-0.0146025	-0.354005\\
-0.0152425	-0.27771\\
-0.0152425	-0.131225\\
-0.01442	-0.094605\\
-0.013825	-0.1708975\\
-0.0140525	-0.2594\\
-0.0145575	-0.41809\\
-0.0152425	-0.29297\\
-0.0151525	-0.216675\\
-0.01474	-0.131225\\
-0.01419	-0.08545\\
-0.013595	-0.12207\\
-0.013595	-0.1007075\\
-0.01355	-0.14038\\
-0.013595	-0.24109\\
-0.01419	-0.2197275\\
-0.0143275	-0.1098625\\
-0.0136875	-0.0976575\\
-0.0134125	-0.131225\\
-0.01355	-0.1678475\\
-0.0137325	-0.0976575\\
-0.0134575	-0.15564\\
-0.013595	-0.112915\\
-0.013595	-0.08545\\
-0.01323	-0.198365\\
-0.0137775	-0.302125\\
-0.01442	-0.4394525\\
-0.0151525	-0.598145\\
-0.01593	-0.5188\\
-0.0160225	-0.250245\\
-0.0151525	-0.1770025\\
-0.0145575	-0.1861575\\
-0.01442	-0.10376\\
-0.013825	-0.0915525\\
-0.0134125	-0.2136225\\
-0.0140525	-0.2288825\\
-0.0142825	-0.2197275\\
-0.0142825	-0.2594\\
-0.014375	-0.305175\\
-0.0146025	-0.3265375\\
-0.0148325	-0.338745\\
-0.0149225	-0.41504\\
-0.0151975	-0.38147\\
-0.01529	-0.5584725\\
-0.0157925	-0.2899175\\
-0.0152425	-0.128175\\
-0.01419	-0.1251225\\
-0.01387	-0.2380375\\
-0.0143275	-0.1861575\\
-0.0142825	-0.1678475\\
-0.0141	-0.079345\\
-0.0134125	-0.12207\\
-0.0134125	-0.094605\\
-0.0131825	-0.0457775\\
-0.012635	-0.0823975\\
-0.01268	-0.17395\\
-0.0134125	-0.2197275\\
-0.013825	-0.3265375\\
-0.014465	-0.4577625\\
-0.0151525	-0.320435\\
-0.01506	-0.128175\\
-0.0139625	-0.234985\\
-0.01419	-0.2655025\\
-0.01451	-0.1251225\\
-0.01387	-0.1708975\\
-0.013825	-0.3479\\
-0.0146475	-0.320435\\
-0.014785	-0.5188\\
-0.0154275	-0.6011975\\
-0.0160225	-0.372315\\
-0.01561	-0.286865\\
-0.0151525	-0.250245\\
-0.0149225	-0.3417975\\
-0.0151525	-0.479125\\
-0.01561	-0.3356925\\
-0.015335	-0.2105725\\
-0.0148325	-0.253295\\
};
\addplot [color=mycolor2, line width=2.0pt, forget plot]
  table[row sep=crcr]{%
-0.015655	-0.015655\\
-0.0158375	-0.0158375\\
-0.0157025	-0.0157025\\
-0.0149225	-0.0149225\\
-0.0148325	-0.0148325\\
-0.0148775	-0.0148775\\
-0.015015	-0.015015\\
-0.0148775	-0.0148775\\
-0.0140075	-0.0140075\\
-0.013275	-0.013275\\
-0.01291	-0.01291\\
-0.0137775	-0.0137775\\
-0.014785	-0.014785\\
-0.015105	-0.015105\\
-0.0151525	-0.0151525\\
-0.0149225	-0.0149225\\
-0.0143275	-0.0143275\\
-0.0149225	-0.0149225\\
-0.0148325	-0.0148325\\
-0.014375	-0.014375\\
-0.0148325	-0.0148325\\
-0.0146475	-0.0146475\\
-0.015105	-0.015105\\
-0.0152425	-0.0152425\\
-0.0149225	-0.0149225\\
-0.0152425	-0.0152425\\
-0.01538	-0.01538\\
-0.0151525	-0.0151525\\
-0.0149225	-0.0149225\\
-0.0146475	-0.0146475\\
-0.0148775	-0.0148775\\
-0.0149225	-0.0149225\\
-0.015015	-0.015015\\
-0.015105	-0.015105\\
-0.0154725	-0.0154725\\
-0.0155175	-0.0155175\\
-0.015105	-0.015105\\
-0.0149225	-0.0149225\\
-0.014375	-0.014375\\
-0.014145	-0.014145\\
-0.014375	-0.014375\\
-0.0142375	-0.0142375\\
-0.014145	-0.014145\\
-0.01442	-0.01442\\
-0.014695	-0.014695\\
-0.0151525	-0.0151525\\
-0.015105	-0.015105\\
-0.01451	-0.01451\\
-0.014145	-0.014145\\
-0.0142375	-0.0142375\\
-0.01442	-0.01442\\
-0.0139625	-0.0139625\\
-0.0136875	-0.0136875\\
-0.0140075	-0.0140075\\
-0.013915	-0.013915\\
-0.0136875	-0.0136875\\
-0.01387	-0.01387\\
-0.0140525	-0.0140525\\
-0.01451	-0.01451\\
-0.01474	-0.01474\\
-0.0149225	-0.0149225\\
-0.015105	-0.015105\\
-0.01497	-0.01497\\
-0.0149225	-0.0149225\\
-0.014785	-0.014785\\
-0.014695	-0.014695\\
-0.01529	-0.01529\\
-0.0160225	-0.0160225\\
-0.0162975	-0.0162975\\
-0.0158375	-0.0158375\\
-0.0160225	-0.0160225\\
-0.01648	-0.01648\\
-0.0166175	-0.0166175\\
-0.0163875	-0.0163875\\
-0.01657	-0.01657\\
-0.01712	-0.01712\\
-0.0169825	-0.0169825\\
-0.0166625	-0.0166625\\
-0.0161125	-0.0161125\\
-0.01561	-0.01561\\
-0.015335	-0.015335\\
-0.015105	-0.015105\\
-0.0145575	-0.0145575\\
-0.01419	-0.01419\\
-0.0142825	-0.0142825\\
-0.014695	-0.014695\\
-0.014785	-0.014785\\
-0.01506	-0.01506\\
-0.0151975	-0.0151975\\
-0.01561	-0.01561\\
-0.0157025	-0.0157025\\
-0.0155175	-0.0155175\\
-0.0152425	-0.0152425\\
-0.01538	-0.01538\\
-0.015105	-0.015105\\
-0.0149225	-0.0149225\\
-0.015015	-0.015015\\
-0.015105	-0.015105\\
-0.014785	-0.014785\\
-0.01474	-0.01474\\
-0.014465	-0.014465\\
-0.01442	-0.01442\\
-0.014465	-0.014465\\
-0.01419	-0.01419\\
-0.014145	-0.014145\\
-0.014375	-0.014375\\
-0.0149225	-0.0149225\\
-0.0151525	-0.0151525\\
-0.01529	-0.01529\\
-0.0148325	-0.0148325\\
-0.0149225	-0.0149225\\
-0.015565	-0.015565\\
-0.0154725	-0.0154725\\
-0.015565	-0.015565\\
-0.015655	-0.015655\\
-0.0151525	-0.0151525\\
-0.0146475	-0.0146475\\
-0.01474	-0.01474\\
-0.0149225	-0.0149225\\
-0.0155175	-0.0155175\\
-0.0160225	-0.0160225\\
-0.01616	-0.01616\\
-0.015885	-0.015885\\
-0.0157925	-0.0157925\\
-0.0157475	-0.0157475\\
-0.0157025	-0.0157025\\
-0.015655	-0.015655\\
-0.0152425	-0.0152425\\
-0.0149225	-0.0149225\\
-0.014785	-0.014785\\
-0.0146475	-0.0146475\\
-0.01497	-0.01497\\
-0.01561	-0.01561\\
-0.015565	-0.015565\\
-0.015105	-0.015105\\
-0.0148775	-0.0148775\\
-0.01474	-0.01474\\
-0.0142375	-0.0142375\\
-0.0137325	-0.0137325\\
-0.013595	-0.013595\\
-0.0141	-0.0141\\
-0.014145	-0.014145\\
-0.0141	-0.0141\\
-0.01387	-0.01387\\
-0.0134575	-0.0134575\\
-0.0130475	-0.0130475\\
-0.013275	-0.013275\\
-0.01419	-0.01419\\
-0.01497	-0.01497\\
-0.015335	-0.015335\\
-0.015105	-0.015105\\
-0.0146025	-0.0146025\\
-0.01419	-0.01419\\
-0.0136425	-0.0136425\\
-0.0141	-0.0141\\
-0.0142825	-0.0142825\\
-0.0146025	-0.0146025\\
-0.015105	-0.015105\\
-0.01593	-0.01593\\
-0.01657	-0.01657\\
-0.01648	-0.01648\\
-0.0163875	-0.0163875\\
-0.01593	-0.01593\\
-0.0157925	-0.0157925\\
-0.015975	-0.015975\\
-0.0160225	-0.0160225\\
-0.0157925	-0.0157925\\
-0.015565	-0.015565\\
-0.0154275	-0.0154275\\
-0.0155175	-0.0155175\\
-0.01529	-0.01529\\
-0.0157025	-0.0157025\\
-0.0160675	-0.0160675\\
-0.015885	-0.015885\\
-0.01529	-0.01529\\
-0.015105	-0.015105\\
-0.015655	-0.015655\\
-0.0160675	-0.0160675\\
-0.0163875	-0.0163875\\
-0.0166175	-0.0166175\\
-0.0167075	-0.0167075\\
-0.0164325	-0.0164325\\
-0.01625	-0.01625\\
-0.0163425	-0.0163425\\
-0.01657	-0.01657\\
-0.01616	-0.01616\\
-0.0154275	-0.0154275\\
-0.015565	-0.015565\\
-0.0154275	-0.0154275\\
-0.01497	-0.01497\\
-0.0151525	-0.0151525\\
-0.01497	-0.01497\\
-0.014695	-0.014695\\
-0.014785	-0.014785\\
-0.014695	-0.014695\\
-0.0148775	-0.0148775\\
-0.01538	-0.01538\\
-0.015335	-0.015335\\
-0.015015	-0.015015\\
-0.0149225	-0.0149225\\
-0.01529	-0.01529\\
-0.0160675	-0.0160675\\
-0.015975	-0.015975\\
-0.01538	-0.01538\\
-0.0149225	-0.0149225\\
-0.0148775	-0.0148775\\
-0.0145575	-0.0145575\\
-0.014465	-0.014465\\
-0.0146475	-0.0146475\\
-0.0148775	-0.0148775\\
-0.01506	-0.01506\\
-0.01474	-0.01474\\
-0.0151525	-0.0151525\\
-0.0155175	-0.0155175\\
-0.0154725	-0.0154725\\
-0.01538	-0.01538\\
-0.015335	-0.015335\\
-0.0157025	-0.0157025\\
-0.015335	-0.015335\\
-0.014375	-0.014375\\
-0.0142825	-0.0142825\\
-0.01442	-0.01442\\
-0.01474	-0.01474\\
-0.014465	-0.014465\\
-0.01474	-0.01474\\
-0.0148325	-0.0148325\\
-0.0145575	-0.0145575\\
-0.0146025	-0.0146025\\
-0.0145575	-0.0145575\\
-0.014465	-0.014465\\
-0.0145575	-0.0145575\\
-0.014145	-0.014145\\
-0.0141	-0.0141\\
-0.01387	-0.01387\\
-0.0137775	-0.0137775\\
-0.014375	-0.014375\\
-0.0146475	-0.0146475\\
-0.015105	-0.015105\\
-0.01561	-0.01561\\
-0.01538	-0.01538\\
-0.01474	-0.01474\\
-0.0142375	-0.0142375\\
-0.014145	-0.014145\\
-0.014375	-0.014375\\
-0.0140525	-0.0140525\\
-0.0139625	-0.0139625\\
-0.01419	-0.01419\\
-0.0148325	-0.0148325\\
-0.01497	-0.01497\\
-0.015335	-0.015335\\
-0.01506	-0.01506\\
-0.0148775	-0.0148775\\
-0.0142375	-0.0142375\\
-0.013915	-0.013915\\
-0.013505	-0.013505\\
-0.0133675	-0.0133675\\
-0.0134125	-0.0134125\\
-0.0139625	-0.0139625\\
-0.0142375	-0.0142375\\
-0.01442	-0.01442\\
-0.0146025	-0.0146025\\
-0.0151975	-0.0151975\\
-0.0154275	-0.0154275\\
-0.015105	-0.015105\\
-0.0151525	-0.0151525\\
-0.01529	-0.01529\\
-0.01561	-0.01561\\
-0.0154725	-0.0154725\\
-0.015015	-0.015015\\
-0.0142375	-0.0142375\\
-0.013595	-0.013595\\
-0.0133675	-0.0133675\\
-0.0134575	-0.0134575\\
-0.01355	-0.01355\\
-0.0134575	-0.0134575\\
-0.0136425	-0.0136425\\
-0.01419	-0.01419\\
-0.0143275	-0.0143275\\
-0.0142825	-0.0142825\\
-0.014465	-0.014465\\
-0.0141	-0.0141\\
-0.0134125	-0.0134125\\
-0.013825	-0.013825\\
-0.014465	-0.014465\\
-0.0145575	-0.0145575\\
-0.014695	-0.014695\\
-0.014375	-0.014375\\
-0.0145575	-0.0145575\\
-0.01506	-0.01506\\
-0.0151975	-0.0151975\\
-0.0149225	-0.0149225\\
-0.01538	-0.01538\\
-0.015335	-0.015335\\
-0.0151975	-0.0151975\\
-0.01538	-0.01538\\
-0.0151525	-0.0151525\\
-0.01497	-0.01497\\
-0.01538	-0.01538\\
-0.0160225	-0.0160225\\
-0.01593	-0.01593\\
-0.015335	-0.015335\\
-0.015105	-0.015105\\
-0.01506	-0.01506\\
-0.0152425	-0.0152425\\
-0.0154275	-0.0154275\\
-0.0152425	-0.0152425\\
-0.0151975	-0.0151975\\
-0.0154725	-0.0154725\\
-0.01561	-0.01561\\
-0.0152425	-0.0152425\\
-0.0149225	-0.0149225\\
-0.015105	-0.015105\\
-0.015655	-0.015655\\
-0.0157925	-0.0157925\\
-0.01529	-0.01529\\
-0.01497	-0.01497\\
-0.014465	-0.014465\\
-0.013825	-0.013825\\
-0.0133675	-0.0133675\\
-0.0130475	-0.0130475\\
-0.0136425	-0.0136425\\
-0.0142375	-0.0142375\\
-0.0145575	-0.0145575\\
-0.01442	-0.01442\\
-0.0145575	-0.0145575\\
-0.01419	-0.01419\\
-0.0140525	-0.0140525\\
-0.0137775	-0.0137775\\
-0.013915	-0.013915\\
-0.014375	-0.014375\\
-0.0148325	-0.0148325\\
-0.01451	-0.01451\\
-0.01419	-0.01419\\
-0.013915	-0.013915\\
-0.0139625	-0.0139625\\
-0.0136875	-0.0136875\\
-0.0143275	-0.0143275\\
-0.0151975	-0.0151975\\
-0.0152425	-0.0152425\\
-0.0155175	-0.0155175\\
-0.015885	-0.015885\\
-0.0160225	-0.0160225\\
-0.0157925	-0.0157925\\
-0.01538	-0.01538\\
-0.0151975	-0.0151975\\
-0.015335	-0.015335\\
-0.0155175	-0.0155175\\
-0.015565	-0.015565\\
-0.01593	-0.01593\\
-0.01625	-0.01625\\
-0.0166175	-0.0166175\\
-0.0162975	-0.0162975\\
-0.01625	-0.01625\\
-0.01648	-0.01648\\
-0.016205	-0.016205\\
-0.0162975	-0.0162975\\
-0.01616	-0.01616\\
-0.0155175	-0.0155175\\
-0.014785	-0.014785\\
-0.0145575	-0.0145575\\
-0.0149225	-0.0149225\\
-0.0148775	-0.0148775\\
-0.014465	-0.014465\\
-0.01451	-0.01451\\
-0.0143275	-0.0143275\\
-0.013915	-0.013915\\
-0.0140075	-0.0140075\\
-0.0140525	-0.0140525\\
-0.01387	-0.01387\\
-0.0142375	-0.0142375\\
-0.014785	-0.014785\\
-0.0146025	-0.0146025\\
-0.01506	-0.01506\\
-0.0158375	-0.0158375\\
-0.015975	-0.015975\\
-0.0162975	-0.0162975\\
-0.0160225	-0.0160225\\
-0.01593	-0.01593\\
-0.01561	-0.01561\\
-0.0148775	-0.0148775\\
-0.014785	-0.014785\\
-0.014375	-0.014375\\
-0.01451	-0.01451\\
-0.0141	-0.0141\\
-0.013505	-0.013505\\
-0.0133675	-0.0133675\\
-0.0141	-0.0141\\
-0.0146025	-0.0146025\\
-0.01497	-0.01497\\
-0.015105	-0.015105\\
-0.0151975	-0.0151975\\
-0.015105	-0.015105\\
-0.0148325	-0.0148325\\
-0.014785	-0.014785\\
-0.0146475	-0.0146475\\
-0.01497	-0.01497\\
-0.01474	-0.01474\\
-0.01442	-0.01442\\
-0.014695	-0.014695\\
-0.0151525	-0.0151525\\
-0.01497	-0.01497\\
-0.014695	-0.014695\\
-0.01451	-0.01451\\
-0.0142825	-0.0142825\\
-0.0140075	-0.0140075\\
-0.0142375	-0.0142375\\
-0.01451	-0.01451\\
-0.0146475	-0.0146475\\
-0.0146025	-0.0146025\\
-0.014695	-0.014695\\
-0.01474	-0.01474\\
-0.014145	-0.014145\\
-0.0140075	-0.0140075\\
-0.014695	-0.014695\\
-0.01529	-0.01529\\
-0.01538	-0.01538\\
-0.01529	-0.01529\\
-0.0152425	-0.0152425\\
-0.01497	-0.01497\\
-0.0148325	-0.0148325\\
-0.0146475	-0.0146475\\
-0.0148325	-0.0148325\\
-0.0148775	-0.0148775\\
-0.01442	-0.01442\\
-0.0146475	-0.0146475\\
-0.014785	-0.014785\\
-0.015015	-0.015015\\
-0.0151975	-0.0151975\\
-0.0152425	-0.0152425\\
-0.01561	-0.01561\\
-0.0160675	-0.0160675\\
-0.01648	-0.01648\\
-0.0160675	-0.0160675\\
-0.01529	-0.01529\\
-0.0145575	-0.0145575\\
-0.013915	-0.013915\\
-0.0140525	-0.0140525\\
-0.0140075	-0.0140075\\
-0.01451	-0.01451\\
-0.0146025	-0.0146025\\
-0.01451	-0.01451\\
-0.01474	-0.01474\\
-0.01497	-0.01497\\
-0.0152425	-0.0152425\\
-0.0154275	-0.0154275\\
-0.015975	-0.015975\\
-0.0161125	-0.0161125\\
-0.0157925	-0.0157925\\
-0.015335	-0.015335\\
-0.0151525	-0.0151525\\
-0.01538	-0.01538\\
-0.0151525	-0.0151525\\
-0.01497	-0.01497\\
-0.0142825	-0.0142825\\
-0.01451	-0.01451\\
-0.01506	-0.01506\\
-0.0149225	-0.0149225\\
-0.0148325	-0.0148325\\
-0.015565	-0.015565\\
-0.0155175	-0.0155175\\
-0.01538	-0.01538\\
-0.0157475	-0.0157475\\
-0.015885	-0.015885\\
-0.01538	-0.01538\\
-0.0148325	-0.0148325\\
-0.0146475	-0.0146475\\
-0.015105	-0.015105\\
-0.0158375	-0.0158375\\
-0.0161125	-0.0161125\\
-0.016205	-0.016205\\
-0.015975	-0.015975\\
-0.01538	-0.01538\\
-0.015105	-0.015105\\
-0.014695	-0.014695\\
-0.014465	-0.014465\\
-0.0142375	-0.0142375\\
-0.014465	-0.014465\\
-0.0146475	-0.0146475\\
-0.0142375	-0.0142375\\
-0.014375	-0.014375\\
-0.014785	-0.014785\\
-0.015015	-0.015015\\
-0.01538	-0.01538\\
-0.0157925	-0.0157925\\
-0.0162975	-0.0162975\\
-0.0160225	-0.0160225\\
-0.0157475	-0.0157475\\
-0.0157925	-0.0157925\\
-0.0157025	-0.0157025\\
-0.0152425	-0.0152425\\
-0.0151975	-0.0151975\\
-0.0157925	-0.0157925\\
-0.0161125	-0.0161125\\
-0.01561	-0.01561\\
-0.0148775	-0.0148775\\
-0.0142825	-0.0142825\\
-0.0133675	-0.0133675\\
-0.0130925	-0.0130925\\
-0.013505	-0.013505\\
-0.0137325	-0.0137325\\
-0.014145	-0.014145\\
-0.013825	-0.013825\\
-0.0137775	-0.0137775\\
-0.0136875	-0.0136875\\
-0.0137325	-0.0137325\\
-0.0141	-0.0141\\
-0.014695	-0.014695\\
-0.01497	-0.01497\\
-0.01506	-0.01506\\
-0.0152425	-0.0152425\\
-0.015655	-0.015655\\
-0.015885	-0.015885\\
-0.01616	-0.01616\\
-0.016205	-0.016205\\
-0.01593	-0.01593\\
-0.0154725	-0.0154725\\
-0.0152425	-0.0152425\\
-0.0157475	-0.0157475\\
-0.0160675	-0.0160675\\
-0.0157475	-0.0157475\\
-0.0152425	-0.0152425\\
-0.014375	-0.014375\\
-0.0136425	-0.0136425\\
-0.013505	-0.013505\\
-0.013	-0.013\\
-0.0133675	-0.0133675\\
-0.01387	-0.01387\\
-0.0140075	-0.0140075\\
-0.0139625	-0.0139625\\
-0.013505	-0.013505\\
-0.0131375	-0.0131375\\
-0.01323	-0.01323\\
-0.0133675	-0.0133675\\
-0.0134575	-0.0134575\\
-0.013275	-0.013275\\
-0.0130475	-0.0130475\\
-0.0134575	-0.0134575\\
-0.0146475	-0.0146475\\
-0.01538	-0.01538\\
-0.015565	-0.015565\\
-0.015335	-0.015335\\
-0.01561	-0.01561\\
-0.016205	-0.016205\\
-0.0163425	-0.0163425\\
-0.0163875	-0.0163875\\
-0.0161125	-0.0161125\\
-0.0160675	-0.0160675\\
-0.0157025	-0.0157025\\
-0.0154275	-0.0154275\\
-0.0148775	-0.0148775\\
-0.014465	-0.014465\\
-0.015015	-0.015015\\
-0.0148325	-0.0148325\\
-0.01474	-0.01474\\
-0.0148775	-0.0148775\\
-0.0148325	-0.0148325\\
-0.0149225	-0.0149225\\
-0.01506	-0.01506\\
-0.0151525	-0.0151525\\
-0.01506	-0.01506\\
-0.014785	-0.014785\\
-0.0149225	-0.0149225\\
-0.0143275	-0.0143275\\
-0.01332	-0.01332\\
-0.0131825	-0.0131825\\
-0.0136875	-0.0136875\\
-0.013275	-0.013275\\
-0.0124975	-0.0124975\\
-0.012635	-0.012635\\
-0.013275	-0.013275\\
-0.013595	-0.013595\\
-0.0139625	-0.0139625\\
-0.01451	-0.01451\\
-0.0146475	-0.0146475\\
-0.01419	-0.01419\\
-0.01451	-0.01451\\
-0.0140075	-0.0140075\\
-0.0136425	-0.0136425\\
-0.01323	-0.01323\\
-0.012725	-0.012725\\
-0.012955	-0.012955\\
-0.0127725	-0.0127725\\
-0.0131825	-0.0131825\\
-0.013915	-0.013915\\
-0.0141	-0.0141\\
-0.01387	-0.01387\\
-0.0136875	-0.0136875\\
-0.013915	-0.013915\\
-0.0136875	-0.0136875\\
-0.013505	-0.013505\\
-0.0134575	-0.0134575\\
-0.013275	-0.013275\\
-0.01323	-0.01323\\
-0.0137325	-0.0137325\\
-0.013825	-0.013825\\
-0.0136425	-0.0136425\\
-0.013505	-0.013505\\
-0.01332	-0.01332\\
-0.0133675	-0.0133675\\
-0.0142375	-0.0142375\\
-0.01497	-0.01497\\
-0.01474	-0.01474\\
-0.0145575	-0.0145575\\
-0.01419	-0.01419\\
-0.0146475	-0.0146475\\
-0.01474	-0.01474\\
-0.0142825	-0.0142825\\
-0.0143275	-0.0143275\\
-0.014145	-0.014145\\
-0.0143275	-0.0143275\\
-0.0139625	-0.0139625\\
-0.0136875	-0.0136875\\
-0.013915	-0.013915\\
-0.014145	-0.014145\\
-0.0140075	-0.0140075\\
-0.013915	-0.013915\\
-0.0140075	-0.0140075\\
-0.014465	-0.014465\\
-0.01451	-0.01451\\
-0.01497	-0.01497\\
-0.01561	-0.01561\\
-0.015565	-0.015565\\
-0.015015	-0.015015\\
-0.01451	-0.01451\\
-0.0148325	-0.0148325\\
-0.0151975	-0.0151975\\
-0.015015	-0.015015\\
-0.0151525	-0.0151525\\
-0.01474	-0.01474\\
-0.013915	-0.013915\\
-0.0136425	-0.0136425\\
-0.014375	-0.014375\\
-0.0145575	-0.0145575\\
-0.01442	-0.01442\\
-0.0148325	-0.0148325\\
-0.01497	-0.01497\\
-0.0148325	-0.0148325\\
-0.01451	-0.01451\\
-0.0148775	-0.0148775\\
-0.01474	-0.01474\\
-0.014785	-0.014785\\
-0.014695	-0.014695\\
-0.01442	-0.01442\\
-0.014465	-0.014465\\
-0.01419	-0.01419\\
-0.014145	-0.014145\\
-0.0146025	-0.0146025\\
-0.01497	-0.01497\\
-0.015105	-0.015105\\
-0.015015	-0.015015\\
-0.014695	-0.014695\\
-0.0142825	-0.0142825\\
-0.0142375	-0.0142375\\
-0.01442	-0.01442\\
-0.0146475	-0.0146475\\
-0.01497	-0.01497\\
-0.01529	-0.01529\\
-0.0151525	-0.0151525\\
-0.0146025	-0.0146025\\
-0.01497	-0.01497\\
-0.01561	-0.01561\\
-0.01538	-0.01538\\
-0.0152425	-0.0152425\\
-0.0154275	-0.0154275\\
-0.0152425	-0.0152425\\
-0.0149225	-0.0149225\\
-0.015015	-0.015015\\
-0.0149225	-0.0149225\\
-0.01497	-0.01497\\
-0.0152425	-0.0152425\\
-0.01506	-0.01506\\
-0.015105	-0.015105\\
-0.0149225	-0.0149225\\
-0.01506	-0.01506\\
-0.0148775	-0.0148775\\
-0.01474	-0.01474\\
-0.0148775	-0.0148775\\
-0.0148325	-0.0148325\\
-0.0142375	-0.0142375\\
-0.0134575	-0.0134575\\
-0.01291	-0.01291\\
-0.0130925	-0.0130925\\
-0.01419	-0.01419\\
-0.0148775	-0.0148775\\
-0.015015	-0.015015\\
-0.0146025	-0.0146025\\
-0.014465	-0.014465\\
-0.0140075	-0.0140075\\
-0.0141	-0.0141\\
-0.014145	-0.014145\\
-0.0141	-0.0141\\
-0.0142375	-0.0142375\\
-0.014145	-0.014145\\
-0.0140525	-0.0140525\\
-0.0137775	-0.0137775\\
-0.013825	-0.013825\\
-0.01451	-0.01451\\
-0.01442	-0.01442\\
-0.01451	-0.01451\\
-0.014465	-0.014465\\
-0.014785	-0.014785\\
-0.0148325	-0.0148325\\
-0.0151525	-0.0151525\\
-0.01506	-0.01506\\
-0.015015	-0.015015\\
-0.015105	-0.015105\\
-0.01442	-0.01442\\
-0.01474	-0.01474\\
-0.0145575	-0.0145575\\
-0.0146025	-0.0146025\\
-0.01474	-0.01474\\
-0.015015	-0.015015\\
-0.0148775	-0.0148775\\
-0.01506	-0.01506\\
-0.01538	-0.01538\\
-0.015335	-0.015335\\
-0.015105	-0.015105\\
-0.0149225	-0.0149225\\
-0.01497	-0.01497\\
-0.0148325	-0.0148325\\
-0.014695	-0.014695\\
-0.014465	-0.014465\\
-0.0145575	-0.0145575\\
-0.014465	-0.014465\\
-0.01506	-0.01506\\
-0.01561	-0.01561\\
-0.0155175	-0.0155175\\
-0.015565	-0.015565\\
-0.01497	-0.01497\\
-0.0146475	-0.0146475\\
-0.014375	-0.014375\\
-0.0141	-0.0141\\
-0.014145	-0.014145\\
-0.0146025	-0.0146025\\
-0.0149225	-0.0149225\\
-0.0151525	-0.0151525\\
-0.01506	-0.01506\\
-0.014695	-0.014695\\
-0.0151525	-0.0151525\\
-0.0155175	-0.0155175\\
-0.0157475	-0.0157475\\
-0.015565	-0.015565\\
-0.01616	-0.01616\\
-0.0161125	-0.0161125\\
-0.015335	-0.015335\\
-0.014785	-0.014785\\
-0.01442	-0.01442\\
-0.0148325	-0.0148325\\
-0.0152425	-0.0152425\\
-0.0157025	-0.0157025\\
-0.01561	-0.01561\\
-0.0152425	-0.0152425\\
-0.0145575	-0.0145575\\
-0.0142825	-0.0142825\\
-0.0143275	-0.0143275\\
-0.01442	-0.01442\\
-0.0140525	-0.0140525\\
-0.0141	-0.0141\\
-0.0139625	-0.0139625\\
-0.01355	-0.01355\\
-0.013915	-0.013915\\
-0.0142375	-0.0142375\\
-0.014465	-0.014465\\
-0.01451	-0.01451\\
-0.0146025	-0.0146025\\
-0.014785	-0.014785\\
-0.0146025	-0.0146025\\
-0.0148325	-0.0148325\\
-0.0151525	-0.0151525\\
-0.01497	-0.01497\\
-0.0148775	-0.0148775\\
-0.0149225	-0.0149225\\
-0.015335	-0.015335\\
-0.0151975	-0.0151975\\
-0.0149225	-0.0149225\\
-0.01497	-0.01497\\
-0.0149225	-0.0149225\\
-0.01451	-0.01451\\
-0.01474	-0.01474\\
-0.01529	-0.01529\\
-0.015335	-0.015335\\
-0.01538	-0.01538\\
-0.0160675	-0.0160675\\
-0.0158375	-0.0158375\\
-0.015565	-0.015565\\
-0.0152425	-0.0152425\\
-0.015335	-0.015335\\
-0.0157925	-0.0157925\\
-0.0155175	-0.0155175\\
-0.0152425	-0.0152425\\
-0.015565	-0.015565\\
-0.0152425	-0.0152425\\
-0.0146025	-0.0146025\\
-0.01451	-0.01451\\
-0.014375	-0.014375\\
-0.0140075	-0.0140075\\
-0.0139625	-0.0139625\\
-0.01387	-0.01387\\
-0.0136875	-0.0136875\\
-0.01387	-0.01387\\
-0.01419	-0.01419\\
-0.01442	-0.01442\\
-0.0148325	-0.0148325\\
-0.0149225	-0.0149225\\
-0.0151525	-0.0151525\\
-0.0154275	-0.0154275\\
-0.0154725	-0.0154725\\
-0.015105	-0.015105\\
-0.01506	-0.01506\\
-0.015015	-0.015015\\
-0.01538	-0.01538\\
-0.0151975	-0.0151975\\
-0.01506	-0.01506\\
-0.0154725	-0.0154725\\
-0.0154275	-0.0154275\\
-0.015335	-0.015335\\
-0.0148775	-0.0148775\\
-0.0142825	-0.0142825\\
-0.01387	-0.01387\\
-0.014145	-0.014145\\
-0.0142825	-0.0142825\\
-0.014375	-0.014375\\
-0.0143275	-0.0143275\\
-0.014465	-0.014465\\
-0.01529	-0.01529\\
-0.0157925	-0.0157925\\
-0.0157025	-0.0157025\\
-0.01561	-0.01561\\
-0.0151975	-0.0151975\\
-0.0151525	-0.0151525\\
-0.0152425	-0.0152425\\
-0.0151525	-0.0151525\\
-0.0151975	-0.0151975\\
-0.0154725	-0.0154725\\
-0.0160225	-0.0160225\\
-0.0161125	-0.0161125\\
-0.015975	-0.015975\\
-0.0161125	-0.0161125\\
-0.0157475	-0.0157475\\
-0.0157025	-0.0157025\\
-0.0155175	-0.0155175\\
-0.0154275	-0.0154275\\
-0.0157025	-0.0157025\\
-0.015885	-0.015885\\
-0.01506	-0.01506\\
-0.014785	-0.014785\\
-0.014695	-0.014695\\
-0.0149225	-0.0149225\\
-0.015015	-0.015015\\
-0.0146025	-0.0146025\\
-0.014695	-0.014695\\
-0.01529	-0.01529\\
-0.0162975	-0.0162975\\
-0.01657	-0.01657\\
-0.0164325	-0.0164325\\
-0.01625	-0.01625\\
-0.0163425	-0.0163425\\
-0.01657	-0.01657\\
-0.0163425	-0.0163425\\
-0.01625	-0.01625\\
-0.0162975	-0.0162975\\
-0.01593	-0.01593\\
-0.01561	-0.01561\\
-0.0157925	-0.0157925\\
-0.0154725	-0.0154725\\
-0.01497	-0.01497\\
-0.0151525	-0.0151525\\
-0.0148775	-0.0148775\\
-0.015015	-0.015015\\
-0.015335	-0.015335\\
-0.0152425	-0.0152425\\
-0.0149225	-0.0149225\\
-0.015015	-0.015015\\
-0.015105	-0.015105\\
-0.01529	-0.01529\\
-0.0152425	-0.0152425\\
-0.015015	-0.015015\\
-0.01538	-0.01538\\
-0.0154725	-0.0154725\\
-0.0151525	-0.0151525\\
-0.0155175	-0.0155175\\
-0.01561	-0.01561\\
-0.01529	-0.01529\\
-0.0148775	-0.0148775\\
-0.01442	-0.01442\\
-0.0139625	-0.0139625\\
-0.01442	-0.01442\\
-0.0146025	-0.0146025\\
-0.0143275	-0.0143275\\
-0.013915	-0.013915\\
-0.01387	-0.01387\\
-0.014375	-0.014375\\
-0.0146475	-0.0146475\\
-0.01506	-0.01506\\
-0.015335	-0.015335\\
-0.01538	-0.01538\\
-0.0154725	-0.0154725\\
-0.015105	-0.015105\\
-0.0141	-0.0141\\
-0.0140075	-0.0140075\\
-0.0141	-0.0141\\
-0.0146475	-0.0146475\\
-0.0149225	-0.0149225\\
-0.0154275	-0.0154275\\
-0.0152425	-0.0152425\\
-0.01497	-0.01497\\
-0.0154725	-0.0154725\\
-0.01529	-0.01529\\
-0.0148775	-0.0148775\\
-0.014465	-0.014465\\
-0.0145575	-0.0145575\\
-0.0148325	-0.0148325\\
-0.0154275	-0.0154275\\
-0.0151975	-0.0151975\\
-0.0148775	-0.0148775\\
-0.0148325	-0.0148325\\
-0.0149225	-0.0149225\\
-0.0152425	-0.0152425\\
-0.01593	-0.01593\\
-0.0160225	-0.0160225\\
-0.015975	-0.015975\\
-0.0155175	-0.0155175\\
-0.0149225	-0.0149225\\
-0.014145	-0.014145\\
-0.0141	-0.0141\\
-0.014785	-0.014785\\
-0.01529	-0.01529\\
-0.0157025	-0.0157025\\
-0.01616	-0.01616\\
-0.016205	-0.016205\\
-0.0155175	-0.0155175\\
-0.0151975	-0.0151975\\
-0.015015	-0.015015\\
-0.0151975	-0.0151975\\
-0.0155175	-0.0155175\\
-0.015975	-0.015975\\
-0.0158375	-0.0158375\\
-0.01538	-0.01538\\
-0.01561	-0.01561\\
-0.01538	-0.01538\\
-0.015015	-0.015015\\
-0.0146475	-0.0146475\\
-0.014145	-0.014145\\
-0.013825	-0.013825\\
-0.013505	-0.013505\\
-0.0137775	-0.0137775\\
-0.0142375	-0.0142375\\
-0.01442	-0.01442\\
-0.0142825	-0.0142825\\
-0.014145	-0.014145\\
-0.0134575	-0.0134575\\
-0.012635	-0.012635\\
-0.0124975	-0.0124975\\
-0.0130475	-0.0130475\\
-0.0136875	-0.0136875\\
-0.014145	-0.014145\\
-0.014465	-0.014465\\
-0.014785	-0.014785\\
-0.0152425	-0.0152425\\
-0.015975	-0.015975\\
-0.016205	-0.016205\\
-0.0163425	-0.0163425\\
-0.0162975	-0.0162975\\
-0.015655	-0.015655\\
-0.015335	-0.015335\\
-0.01529	-0.01529\\
-0.0155175	-0.0155175\\
-0.015105	-0.015105\\
-0.014695	-0.014695\\
-0.01442	-0.01442\\
-0.014695	-0.014695\\
-0.014785	-0.014785\\
-0.0142825	-0.0142825\\
-0.0137775	-0.0137775\\
-0.0139625	-0.0139625\\
-0.0143275	-0.0143275\\
-0.01419	-0.01419\\
-0.01442	-0.01442\\
-0.014375	-0.014375\\
-0.0146025	-0.0146025\\
-0.0152425	-0.0152425\\
-0.0154275	-0.0154275\\
-0.0160225	-0.0160225\\
-0.0163425	-0.0163425\\
-0.01616	-0.01616\\
-0.0157025	-0.0157025\\
-0.0151975	-0.0151975\\
-0.0145575	-0.0145575\\
-0.0146475	-0.0146475\\
-0.014465	-0.014465\\
-0.0148775	-0.0148775\\
-0.01497	-0.01497\\
-0.014695	-0.014695\\
-0.0148325	-0.0148325\\
-0.014465	-0.014465\\
-0.0145575	-0.0145575\\
-0.01442	-0.01442\\
-0.01387	-0.01387\\
-0.0137775	-0.0137775\\
-0.013595	-0.013595\\
-0.01355	-0.01355\\
-0.0134575	-0.0134575\\
-0.0131375	-0.0131375\\
-0.0134125	-0.0134125\\
-0.013505	-0.013505\\
-0.0139625	-0.0139625\\
-0.014465	-0.014465\\
-0.01442	-0.01442\\
-0.0143275	-0.0143275\\
-0.0148325	-0.0148325\\
-0.01529	-0.01529\\
-0.0151975	-0.0151975\\
-0.0151525	-0.0151525\\
-0.01442	-0.01442\\
-0.0136875	-0.0136875\\
-0.0140525	-0.0140525\\
-0.014695	-0.014695\\
-0.0149225	-0.0149225\\
-0.01538	-0.01538\\
-0.0155175	-0.0155175\\
-0.0151525	-0.0151525\\
-0.014695	-0.014695\\
-0.0145575	-0.0145575\\
-0.0146475	-0.0146475\\
-0.014465	-0.014465\\
-0.0139625	-0.0139625\\
-0.01387	-0.01387\\
-0.0137775	-0.0137775\\
-0.013505	-0.013505\\
-0.0131825	-0.0131825\\
-0.0134575	-0.0134575\\
-0.0139625	-0.0139625\\
-0.014145	-0.014145\\
-0.0139625	-0.0139625\\
-0.0143275	-0.0143275\\
-0.015015	-0.015015\\
-0.014695	-0.014695\\
-0.0148775	-0.0148775\\
-0.015105	-0.015105\\
-0.0148775	-0.0148775\\
-0.0142825	-0.0142825\\
-0.0142375	-0.0142375\\
-0.014465	-0.014465\\
-0.0140075	-0.0140075\\
-0.0139625	-0.0139625\\
-0.013825	-0.013825\\
-0.01332	-0.01332\\
-0.013	-0.013\\
-0.01332	-0.01332\\
-0.0136425	-0.0136425\\
-0.0136875	-0.0136875\\
-0.013825	-0.013825\\
-0.0140525	-0.0140525\\
-0.01387	-0.01387\\
-0.013275	-0.013275\\
-0.012955	-0.012955\\
-0.013	-0.013\\
-0.0131375	-0.0131375\\
-0.01332	-0.01332\\
-0.01419	-0.01419\\
-0.0146025	-0.0146025\\
-0.01442	-0.01442\\
-0.0146475	-0.0146475\\
-0.0151525	-0.0151525\\
-0.0154275	-0.0154275\\
-0.015565	-0.015565\\
-0.01529	-0.01529\\
-0.0152425	-0.0152425\\
-0.01529	-0.01529\\
-0.0154725	-0.0154725\\
-0.0158375	-0.0158375\\
-0.01593	-0.01593\\
-0.0157925	-0.0157925\\
-0.015565	-0.015565\\
-0.0154275	-0.0154275\\
-0.01538	-0.01538\\
-0.0154725	-0.0154725\\
-0.015105	-0.015105\\
-0.01497	-0.01497\\
-0.0151975	-0.0151975\\
-0.0154275	-0.0154275\\
-0.01529	-0.01529\\
-0.015335	-0.015335\\
-0.0151975	-0.0151975\\
-0.015015	-0.015015\\
-0.0145575	-0.0145575\\
-0.01474	-0.01474\\
-0.01529	-0.01529\\
-0.014695	-0.014695\\
-0.0146025	-0.0146025\\
-0.0142825	-0.0142825\\
-0.01387	-0.01387\\
-0.0140075	-0.0140075\\
-0.0137325	-0.0137325\\
-0.013505	-0.013505\\
-0.013595	-0.013595\\
-0.01332	-0.01332\\
-0.013505	-0.013505\\
-0.0130925	-0.0130925\\
-0.01332	-0.01332\\
-0.0134575	-0.0134575\\
-0.0137325	-0.0137325\\
-0.0137775	-0.0137775\\
-0.0137325	-0.0137325\\
-0.0140525	-0.0140525\\
-0.014145	-0.014145\\
-0.01419	-0.01419\\
-0.01497	-0.01497\\
-0.0149225	-0.0149225\\
-0.0146475	-0.0146475\\
-0.01474	-0.01474\\
-0.014465	-0.014465\\
-0.013915	-0.013915\\
-0.0140525	-0.0140525\\
-0.0141	-0.0141\\
-0.013595	-0.013595\\
-0.01387	-0.01387\\
-0.0140075	-0.0140075\\
-0.01419	-0.01419\\
-0.01387	-0.01387\\
-0.0133675	-0.0133675\\
-0.013	-0.013\\
-0.012955	-0.012955\\
-0.0134125	-0.0134125\\
-0.013595	-0.013595\\
-0.0136875	-0.0136875\\
-0.0141	-0.0141\\
-0.0146475	-0.0146475\\
-0.01451	-0.01451\\
-0.014785	-0.014785\\
-0.0151525	-0.0151525\\
-0.01506	-0.01506\\
-0.015015	-0.015015\\
-0.0155175	-0.0155175\\
-0.01561	-0.01561\\
-0.015655	-0.015655\\
-0.01593	-0.01593\\
-0.01538	-0.01538\\
-0.01474	-0.01474\\
-0.0143275	-0.0143275\\
-0.0145575	-0.0145575\\
-0.0149225	-0.0149225\\
-0.01497	-0.01497\\
-0.0146475	-0.0146475\\
-0.0140075	-0.0140075\\
-0.013915	-0.013915\\
-0.0146025	-0.0146025\\
-0.0152425	-0.0152425\\
-0.015335	-0.015335\\
-0.0148775	-0.0148775\\
-0.01451	-0.01451\\
-0.014695	-0.014695\\
-0.0152425	-0.0152425\\
-0.015565	-0.015565\\
-0.015655	-0.015655\\
-0.0154275	-0.0154275\\
-0.015565	-0.015565\\
-0.015885	-0.015885\\
-0.01625	-0.01625\\
-0.0163425	-0.0163425\\
-0.0161125	-0.0161125\\
-0.016205	-0.016205\\
-0.01616	-0.01616\\
-0.015335	-0.015335\\
-0.01497	-0.01497\\
-0.015105	-0.015105\\
-0.01538	-0.01538\\
-0.0154725	-0.0154725\\
-0.0152425	-0.0152425\\
-0.01529	-0.01529\\
-0.01497	-0.01497\\
-0.015105	-0.015105\\
-0.0151525	-0.0151525\\
-0.0154275	-0.0154275\\
-0.01561	-0.01561\\
-0.015335	-0.015335\\
-0.015015	-0.015015\\
-0.014375	-0.014375\\
-0.01451	-0.01451\\
-0.014375	-0.014375\\
-0.01419	-0.01419\\
-0.0140075	-0.0140075\\
-0.014375	-0.014375\\
-0.014695	-0.014695\\
-0.0149225	-0.0149225\\
-0.015015	-0.015015\\
-0.0145575	-0.0145575\\
-0.014145	-0.014145\\
-0.013825	-0.013825\\
-0.0137325	-0.0137325\\
-0.014145	-0.014145\\
-0.0142825	-0.0142825\\
-0.0148325	-0.0148325\\
-0.01529	-0.01529\\
-0.0155175	-0.0155175\\
-0.01561	-0.01561\\
-0.0158375	-0.0158375\\
-0.0154275	-0.0154275\\
-0.01506	-0.01506\\
-0.014785	-0.014785\\
-0.0146475	-0.0146475\\
-0.0149225	-0.0149225\\
-0.0148775	-0.0148775\\
-0.014465	-0.014465\\
-0.0142375	-0.0142375\\
-0.01474	-0.01474\\
-0.015105	-0.015105\\
-0.015015	-0.015015\\
-0.0155175	-0.0155175\\
-0.015885	-0.015885\\
-0.01561	-0.01561\\
-0.015105	-0.015105\\
-0.0146475	-0.0146475\\
-0.01442	-0.01442\\
-0.0148775	-0.0148775\\
-0.01506	-0.01506\\
-0.0151525	-0.0151525\\
-0.0151975	-0.0151975\\
-0.015335	-0.015335\\
-0.0151525	-0.0151525\\
-0.01506	-0.01506\\
-0.0148325	-0.0148325\\
-0.0145575	-0.0145575\\
-0.0149225	-0.0149225\\
-0.01538	-0.01538\\
-0.01561	-0.01561\\
-0.01506	-0.01506\\
-0.01561	-0.01561\\
-0.015975	-0.015975\\
-0.0154725	-0.0154725\\
-0.014785	-0.014785\\
-0.0149225	-0.0149225\\
-0.01497	-0.01497\\
-0.0148325	-0.0148325\\
-0.015015	-0.015015\\
-0.014695	-0.014695\\
-0.014785	-0.014785\\
-0.0155175	-0.0155175\\
-0.015885	-0.015885\\
-0.0154725	-0.0154725\\
-0.015565	-0.015565\\
-0.01561	-0.01561\\
-0.01538	-0.01538\\
-0.0152425	-0.0152425\\
-0.0148775	-0.0148775\\
-0.015015	-0.015015\\
-0.0151975	-0.0151975\\
-0.0157925	-0.0157925\\
-0.0161125	-0.0161125\\
-0.0166175	-0.0166175\\
-0.01648	-0.01648\\
-0.0161125	-0.0161125\\
-0.0157925	-0.0157925\\
-0.0157025	-0.0157025\\
-0.0155175	-0.0155175\\
-0.015015	-0.015015\\
-0.01474	-0.01474\\
-0.014465	-0.014465\\
-0.0140525	-0.0140525\\
-0.01419	-0.01419\\
-0.01451	-0.01451\\
-0.0142825	-0.0142825\\
-0.013915	-0.013915\\
-0.0137325	-0.0137325\\
-0.01442	-0.01442\\
-0.01506	-0.01506\\
-0.01451	-0.01451\\
-0.01442	-0.01442\\
-0.0146025	-0.0146025\\
-0.01451	-0.01451\\
-0.0146475	-0.0146475\\
-0.0146025	-0.0146025\\
-0.0142825	-0.0142825\\
-0.013915	-0.013915\\
-0.0136425	-0.0136425\\
-0.0136875	-0.0136875\\
-0.0142825	-0.0142825\\
-0.0146475	-0.0146475\\
-0.01451	-0.01451\\
-0.0141	-0.0141\\
-0.013595	-0.013595\\
-0.01419	-0.01419\\
-0.0145575	-0.0145575\\
-0.014695	-0.014695\\
-0.014465	-0.014465\\
-0.0143275	-0.0143275\\
-0.014375	-0.014375\\
-0.014695	-0.014695\\
-0.0151975	-0.0151975\\
-0.015655	-0.015655\\
-0.0155175	-0.0155175\\
-0.01497	-0.01497\\
-0.014785	-0.014785\\
-0.01474	-0.01474\\
-0.014785	-0.014785\\
-0.01442	-0.01442\\
-0.014465	-0.014465\\
-0.01451	-0.01451\\
-0.014145	-0.014145\\
-0.013595	-0.013595\\
-0.0136425	-0.0136425\\
-0.0140525	-0.0140525\\
-0.0146025	-0.0146025\\
-0.0149225	-0.0149225\\
-0.0151975	-0.0151975\\
-0.01529	-0.01529\\
-0.01538	-0.01538\\
-0.0158375	-0.0158375\\
-0.01648	-0.01648\\
-0.0160225	-0.0160225\\
-0.015975	-0.015975\\
-0.0161125	-0.0161125\\
-0.01648	-0.01648\\
-0.01616	-0.01616\\
-0.0151525	-0.0151525\\
-0.014465	-0.014465\\
-0.01419	-0.01419\\
-0.01387	-0.01387\\
-0.01332	-0.01332\\
-0.0131825	-0.0131825\\
-0.013505	-0.013505\\
-0.01387	-0.01387\\
-0.0140075	-0.0140075\\
-0.01387	-0.01387\\
-0.0137325	-0.0137325\\
-0.01387	-0.01387\\
-0.014145	-0.014145\\
-0.0142825	-0.0142825\\
-0.0140075	-0.0140075\\
-0.0136425	-0.0136425\\
-0.0134575	-0.0134575\\
-0.0137325	-0.0137325\\
-0.0140075	-0.0140075\\
-0.01387	-0.01387\\
-0.0134575	-0.0134575\\
-0.0137775	-0.0137775\\
-0.013595	-0.013595\\
-0.0136425	-0.0136425\\
-0.013825	-0.013825\\
-0.0142825	-0.0142825\\
-0.0145575	-0.0145575\\
-0.0142375	-0.0142375\\
-0.014465	-0.014465\\
-0.0146475	-0.0146475\\
-0.01451	-0.01451\\
-0.0142825	-0.0142825\\
-0.014695	-0.014695\\
-0.0151525	-0.0151525\\
-0.0152425	-0.0152425\\
-0.0154275	-0.0154275\\
-0.01529	-0.01529\\
-0.0151525	-0.0151525\\
-0.01506	-0.01506\\
-0.01529	-0.01529\\
-0.0151525	-0.0151525\\
-0.01538	-0.01538\\
-0.0154725	-0.0154725\\
-0.0162975	-0.0162975\\
-0.0163875	-0.0163875\\
-0.0155175	-0.0155175\\
-0.0148775	-0.0148775\\
-0.014695	-0.014695\\
-0.0148775	-0.0148775\\
-0.0151525	-0.0151525\\
-0.0154275	-0.0154275\\
-0.01561	-0.01561\\
-0.0157025	-0.0157025\\
-0.015335	-0.015335\\
-0.015105	-0.015105\\
-0.0151975	-0.0151975\\
-0.0151525	-0.0151525\\
-0.01474	-0.01474\\
-0.0142375	-0.0142375\\
-0.0145575	-0.0145575\\
-0.014695	-0.014695\\
-0.015015	-0.015015\\
-0.01538	-0.01538\\
-0.015335	-0.015335\\
-0.01497	-0.01497\\
-0.014695	-0.014695\\
-0.01497	-0.01497\\
-0.015335	-0.015335\\
-0.0157925	-0.0157925\\
-0.0161125	-0.0161125\\
-0.0163875	-0.0163875\\
-0.0162975	-0.0162975\\
-0.01616	-0.01616\\
-0.0155175	-0.0155175\\
-0.0149225	-0.0149225\\
-0.015335	-0.015335\\
-0.0157925	-0.0157925\\
-0.015335	-0.015335\\
-0.01561	-0.01561\\
-0.0161125	-0.0161125\\
-0.01657	-0.01657\\
-0.016205	-0.016205\\
-0.0154725	-0.0154725\\
-0.0146475	-0.0146475\\
-0.014695	-0.014695\\
-0.0142375	-0.0142375\\
-0.0140525	-0.0140525\\
-0.0141	-0.0141\\
-0.0140525	-0.0140525\\
-0.01419	-0.01419\\
-0.014695	-0.014695\\
-0.014785	-0.014785\\
-0.01506	-0.01506\\
-0.0154725	-0.0154725\\
-0.015565	-0.015565\\
-0.015015	-0.015015\\
-0.0148325	-0.0148325\\
-0.01506	-0.01506\\
-0.014695	-0.014695\\
-0.01451	-0.01451\\
-0.0141	-0.0141\\
-0.0143275	-0.0143275\\
-0.014375	-0.014375\\
-0.0139625	-0.0139625\\
-0.0133675	-0.0133675\\
-0.0130475	-0.0130475\\
-0.0134125	-0.0134125\\
-0.013915	-0.013915\\
-0.0141	-0.0141\\
-0.0137775	-0.0137775\\
-0.014145	-0.014145\\
-0.0146025	-0.0146025\\
-0.01442	-0.01442\\
-0.0142375	-0.0142375\\
-0.014375	-0.014375\\
-0.0146025	-0.0146025\\
-0.0146475	-0.0146475\\
-0.01474	-0.01474\\
-0.014785	-0.014785\\
-0.014465	-0.014465\\
-0.01419	-0.01419\\
-0.0143275	-0.0143275\\
-0.0142375	-0.0142375\\
-0.0143275	-0.0143275\\
-0.014695	-0.014695\\
-0.01529	-0.01529\\
-0.0152425	-0.0152425\\
-0.01506	-0.01506\\
-0.0155175	-0.0155175\\
-0.0154275	-0.0154275\\
-0.0148775	-0.0148775\\
-0.014465	-0.014465\\
-0.0140525	-0.0140525\\
-0.0139625	-0.0139625\\
-0.013825	-0.013825\\
-0.0136875	-0.0136875\\
-0.0142825	-0.0142825\\
-0.0137325	-0.0137325\\
-0.0133675	-0.0133675\\
-0.0136875	-0.0136875\\
-0.0140525	-0.0140525\\
-0.0140075	-0.0140075\\
-0.0136425	-0.0136425\\
-0.0131825	-0.0131825\\
-0.01332	-0.01332\\
-0.0140075	-0.0140075\\
-0.0146025	-0.0146025\\
-0.01451	-0.01451\\
-0.0142825	-0.0142825\\
-0.014375	-0.014375\\
-0.01442	-0.01442\\
-0.0145575	-0.0145575\\
-0.014785	-0.014785\\
-0.015015	-0.015015\\
-0.0151525	-0.0151525\\
-0.01506	-0.01506\\
-0.0146475	-0.0146475\\
-0.0143275	-0.0143275\\
-0.0142375	-0.0142375\\
-0.01451	-0.01451\\
-0.01419	-0.01419\\
-0.013915	-0.013915\\
-0.01442	-0.01442\\
-0.014785	-0.014785\\
-0.0148775	-0.0148775\\
-0.01497	-0.01497\\
-0.0154275	-0.0154275\\
-0.0148775	-0.0148775\\
-0.01529	-0.01529\\
-0.01561	-0.01561\\
-0.0154725	-0.0154725\\
-0.0155175	-0.0155175\\
-0.015565	-0.015565\\
-0.0163425	-0.0163425\\
-0.0166625	-0.0166625\\
-0.016205	-0.016205\\
-0.01625	-0.01625\\
-0.016525	-0.016525\\
-0.01657	-0.01657\\
-0.016755	-0.016755\\
-0.01703	-0.01703\\
-0.0169825	-0.0169825\\
-0.01648	-0.01648\\
-0.0157925	-0.0157925\\
-0.0155175	-0.0155175\\
-0.0152425	-0.0152425\\
-0.0148775	-0.0148775\\
-0.0151525	-0.0151525\\
-0.01561	-0.01561\\
-0.0152425	-0.0152425\\
-0.0149225	-0.0149225\\
-0.01451	-0.01451\\
-0.014465	-0.014465\\
-0.01387	-0.01387\\
-0.0136425	-0.0136425\\
-0.01355	-0.01355\\
-0.0137325	-0.0137325\\
-0.013595	-0.013595\\
-0.01332	-0.01332\\
-0.01291	-0.01291\\
-0.012635	-0.012635\\
-0.012955	-0.012955\\
-0.0128175	-0.0128175\\
-0.0127725	-0.0127725\\
-0.0134575	-0.0134575\\
-0.0141	-0.0141\\
-0.0143275	-0.0143275\\
-0.0141	-0.0141\\
-0.0142375	-0.0142375\\
-0.0149225	-0.0149225\\
-0.014375	-0.014375\\
-0.0136875	-0.0136875\\
-0.01332	-0.01332\\
-0.0128625	-0.0128625\\
-0.0130925	-0.0130925\\
-0.0136425	-0.0136425\\
-0.0142825	-0.0142825\\
-0.0140075	-0.0140075\\
-0.01442	-0.01442\\
-0.01506	-0.01506\\
-0.0152425	-0.0152425\\
-0.015015	-0.015015\\
-0.014375	-0.014375\\
-0.01474	-0.01474\\
-0.0152425	-0.0152425\\
-0.01474	-0.01474\\
-0.0139625	-0.0139625\\
-0.0141	-0.0141\\
-0.0140075	-0.0140075\\
-0.01332	-0.01332\\
-0.012635	-0.012635\\
-0.0130475	-0.0130475\\
-0.01419	-0.01419\\
-0.0146025	-0.0146025\\
-0.0143275	-0.0143275\\
-0.013825	-0.013825\\
-0.0136875	-0.0136875\\
-0.01268	-0.01268\\
-0.0125425	-0.0125425\\
-0.012635	-0.012635\\
-0.012955	-0.012955\\
-0.013505	-0.013505\\
-0.0142375	-0.0142375\\
-0.0145575	-0.0145575\\
-0.014695	-0.014695\\
-0.0148775	-0.0148775\\
-0.0149225	-0.0149225\\
-0.0148775	-0.0148775\\
-0.014785	-0.014785\\
-0.01497	-0.01497\\
-0.015655	-0.015655\\
-0.0158375	-0.0158375\\
-0.015105	-0.015105\\
-0.01419	-0.01419\\
-0.014785	-0.014785\\
-0.01529	-0.01529\\
-0.0151525	-0.0151525\\
-0.0145575	-0.0145575\\
-0.0142825	-0.0142825\\
-0.0149225	-0.0149225\\
-0.01561	-0.01561\\
-0.0158375	-0.0158375\\
-0.0160675	-0.0160675\\
-0.01616	-0.01616\\
-0.0158375	-0.0158375\\
-0.0157925	-0.0157925\\
-0.01497	-0.01497\\
-0.014465	-0.014465\\
-0.0146475	-0.0146475\\
-0.0148775	-0.0148775\\
-0.015015	-0.015015\\
-0.015105	-0.015105\\
-0.01474	-0.01474\\
-0.0148775	-0.0148775\\
-0.0148325	-0.0148325\\
-0.0142825	-0.0142825\\
-0.0136425	-0.0136425\\
-0.01323	-0.01323\\
-0.0134575	-0.0134575\\
-0.013275	-0.013275\\
-0.0128175	-0.0128175\\
-0.0130475	-0.0130475\\
-0.013915	-0.013915\\
-0.0137775	-0.0137775\\
-0.01355	-0.01355\\
-0.013825	-0.013825\\
-0.0145575	-0.0145575\\
-0.014465	-0.014465\\
-0.0137775	-0.0137775\\
-0.0130925	-0.0130925\\
-0.0136425	-0.0136425\\
-0.01474	-0.01474\\
-0.01538	-0.01538\\
-0.0161125	-0.0161125\\
-0.0167075	-0.0167075\\
-0.016845	-0.016845\\
-0.0164325	-0.0164325\\
-0.01625	-0.01625\\
-0.01657	-0.01657\\
-0.016525	-0.016525\\
-0.0163425	-0.0163425\\
-0.01648	-0.01648\\
-0.0166175	-0.0166175\\
-0.0166625	-0.0166625\\
-0.01703	-0.01703\\
-0.0168	-0.0168\\
-0.015975	-0.015975\\
-0.015015	-0.015015\\
-0.014375	-0.014375\\
-0.0143275	-0.0143275\\
-0.0142825	-0.0142825\\
-0.0148325	-0.0148325\\
-0.014785	-0.014785\\
-0.01474	-0.01474\\
-0.015015	-0.015015\\
-0.01451	-0.01451\\
-0.014465	-0.014465\\
-0.014785	-0.014785\\
-0.015015	-0.015015\\
-0.0146025	-0.0146025\\
-0.0140525	-0.0140525\\
-0.0141	-0.0141\\
-0.015105	-0.015105\\
-0.015885	-0.015885\\
-0.015975	-0.015975\\
-0.01616	-0.01616\\
-0.01648	-0.01648\\
-0.0169825	-0.0169825\\
-0.01703	-0.01703\\
-0.0164325	-0.0164325\\
-0.0157925	-0.0157925\\
-0.0148775	-0.0148775\\
-0.0145575	-0.0145575\\
-0.0146475	-0.0146475\\
-0.0148325	-0.0148325\\
-0.0146025	-0.0146025\\
-0.0143275	-0.0143275\\
-0.01442	-0.01442\\
-0.0145575	-0.0145575\\
-0.01474	-0.01474\\
-0.01506	-0.01506\\
-0.0148325	-0.0148325\\
-0.0141	-0.0141\\
-0.013505	-0.013505\\
-0.0130925	-0.0130925\\
-0.013	-0.013\\
-0.0130475	-0.0130475\\
-0.01355	-0.01355\\
-0.0137325	-0.0137325\\
-0.0140525	-0.0140525\\
-0.0146025	-0.0146025\\
-0.0145575	-0.0145575\\
-0.015015	-0.015015\\
-0.01593	-0.01593\\
-0.015885	-0.015885\\
-0.0157475	-0.0157475\\
-0.015885	-0.015885\\
-0.0161125	-0.0161125\\
-0.0157025	-0.0157025\\
-0.0154725	-0.0154725\\
-0.0149225	-0.0149225\\
-0.014785	-0.014785\\
-0.015565	-0.015565\\
-0.01657	-0.01657\\
-0.017165	-0.017165\\
-0.017075	-0.017075\\
-0.0168	-0.0168\\
-0.0164325	-0.0164325\\
-0.0163425	-0.0163425\\
-0.0166625	-0.0166625\\
-0.01616	-0.01616\\
-0.0157025	-0.0157025\\
-0.0154275	-0.0154275\\
-0.015885	-0.015885\\
-0.0160675	-0.0160675\\
-0.01538	-0.01538\\
-0.01506	-0.01506\\
-0.01474	-0.01474\\
-0.01506	-0.01506\\
-0.0157025	-0.0157025\\
-0.0158375	-0.0158375\\
-0.0157025	-0.0157025\\
-0.0152425	-0.0152425\\
-0.01529	-0.01529\\
-0.01506	-0.01506\\
-0.0146475	-0.0146475\\
-0.0142825	-0.0142825\\
-0.0140075	-0.0140075\\
-0.0136425	-0.0136425\\
-0.013595	-0.013595\\
-0.0140075	-0.0140075\\
-0.0146025	-0.0146025\\
-0.01474	-0.01474\\
-0.0143275	-0.0143275\\
-0.0141	-0.0141\\
-0.0139625	-0.0139625\\
-0.014375	-0.014375\\
-0.014465	-0.014465\\
-0.01451	-0.01451\\
-0.0140075	-0.0140075\\
-0.0146475	-0.0146475\\
-0.015335	-0.015335\\
-0.015105	-0.015105\\
-0.014145	-0.014145\\
-0.01387	-0.01387\\
-0.0143275	-0.0143275\\
-0.01442	-0.01442\\
-0.0143275	-0.0143275\\
-0.013915	-0.013915\\
-0.0142375	-0.0142375\\
-0.0149225	-0.0149225\\
-0.0146025	-0.0146025\\
-0.013595	-0.013595\\
-0.0134125	-0.0134125\\
-0.0143275	-0.0143275\\
-0.0148325	-0.0148325\\
-0.01442	-0.01442\\
-0.0141	-0.0141\\
-0.014695	-0.014695\\
-0.015335	-0.015335\\
-0.0157025	-0.0157025\\
-0.0162975	-0.0162975\\
-0.01593	-0.01593\\
-0.015565	-0.015565\\
-0.0158375	-0.0158375\\
-0.015565	-0.015565\\
-0.0157475	-0.0157475\\
-0.0161125	-0.0161125\\
-0.015885	-0.015885\\
-0.015105	-0.015105\\
-0.01529	-0.01529\\
-0.0160675	-0.0160675\\
-0.01648	-0.01648\\
-0.0162975	-0.0162975\\
-0.0160675	-0.0160675\\
-0.0162975	-0.0162975\\
-0.0160675	-0.0160675\\
-0.0155175	-0.0155175\\
-0.01529	-0.01529\\
-0.0154275	-0.0154275\\
-0.0154725	-0.0154725\\
-0.01561	-0.01561\\
-0.01529	-0.01529\\
-0.015565	-0.015565\\
-0.0151525	-0.0151525\\
-0.014785	-0.014785\\
-0.014465	-0.014465\\
-0.014145	-0.014145\\
-0.0141	-0.0141\\
-0.01451	-0.01451\\
-0.014695	-0.014695\\
-0.01497	-0.01497\\
-0.0154275	-0.0154275\\
-0.0157475	-0.0157475\\
-0.015655	-0.015655\\
-0.0154275	-0.0154275\\
-0.0146025	-0.0146025\\
-0.015335	-0.015335\\
-0.015105	-0.015105\\
-0.014375	-0.014375\\
-0.01474	-0.01474\\
-0.01538	-0.01538\\
-0.01561	-0.01561\\
-0.015975	-0.015975\\
-0.01657	-0.01657\\
-0.0163875	-0.0163875\\
-0.0160675	-0.0160675\\
-0.01625	-0.01625\\
-0.015885	-0.015885\\
-0.0155175	-0.0155175\\
-0.0154725	-0.0154725\\
-0.0157475	-0.0157475\\
-0.0158375	-0.0158375\\
-0.015335	-0.015335\\
-0.01497	-0.01497\\
-0.014695	-0.014695\\
-0.0148775	-0.0148775\\
-0.0149225	-0.0149225\\
-0.014695	-0.014695\\
-0.0152425	-0.0152425\\
-0.015565	-0.015565\\
-0.01529	-0.01529\\
-0.015015	-0.015015\\
-0.0152425	-0.0152425\\
-0.014785	-0.014785\\
-0.0141	-0.0141\\
-0.0140075	-0.0140075\\
-0.0142825	-0.0142825\\
-0.0142375	-0.0142375\\
-0.013915	-0.013915\\
-0.0133675	-0.0133675\\
-0.0134125	-0.0134125\\
-0.013915	-0.013915\\
-0.0145575	-0.0145575\\
-0.01497	-0.01497\\
-0.0154275	-0.0154275\\
-0.0157925	-0.0157925\\
-0.0151975	-0.0151975\\
-0.014695	-0.014695\\
-0.0154725	-0.0154725\\
-0.016205	-0.016205\\
-0.0161125	-0.0161125\\
-0.015975	-0.015975\\
-0.01648	-0.01648\\
-0.0164325	-0.0164325\\
-0.0161125	-0.0161125\\
-0.0157475	-0.0157475\\
-0.015655	-0.015655\\
-0.01593	-0.01593\\
-0.015655	-0.015655\\
-0.015335	-0.015335\\
-0.015885	-0.015885\\
-0.01625	-0.01625\\
-0.0164325	-0.0164325\\
-0.015565	-0.015565\\
-0.0146025	-0.0146025\\
-0.0152425	-0.0152425\\
-0.01442	-0.01442\\
-0.013825	-0.013825\\
-0.0140525	-0.0140525\\
-0.0145575	-0.0145575\\
-0.0152425	-0.0152425\\
-0.0151525	-0.0151525\\
-0.01474	-0.01474\\
-0.01419	-0.01419\\
-0.013595	-0.013595\\
-0.01355	-0.01355\\
-0.013595	-0.013595\\
-0.01419	-0.01419\\
-0.0143275	-0.0143275\\
-0.0136875	-0.0136875\\
-0.0134125	-0.0134125\\
-0.01355	-0.01355\\
-0.0137325	-0.0137325\\
-0.0134575	-0.0134575\\
-0.013595	-0.013595\\
-0.01323	-0.01323\\
-0.0137775	-0.0137775\\
-0.01442	-0.01442\\
-0.0151525	-0.0151525\\
-0.01593	-0.01593\\
-0.0160225	-0.0160225\\
-0.0151525	-0.0151525\\
-0.0145575	-0.0145575\\
-0.01442	-0.01442\\
-0.013825	-0.013825\\
-0.0134125	-0.0134125\\
-0.0140525	-0.0140525\\
-0.0142825	-0.0142825\\
-0.014375	-0.014375\\
-0.0146025	-0.0146025\\
-0.0148325	-0.0148325\\
-0.0149225	-0.0149225\\
-0.0151975	-0.0151975\\
-0.01529	-0.01529\\
-0.0157925	-0.0157925\\
-0.0152425	-0.0152425\\
-0.01419	-0.01419\\
-0.01387	-0.01387\\
-0.0143275	-0.0143275\\
-0.0142825	-0.0142825\\
-0.0141	-0.0141\\
-0.0134125	-0.0134125\\
-0.0131825	-0.0131825\\
-0.012635	-0.012635\\
-0.01268	-0.01268\\
-0.0134125	-0.0134125\\
-0.013825	-0.013825\\
-0.014465	-0.014465\\
-0.0151525	-0.0151525\\
-0.01506	-0.01506\\
-0.0139625	-0.0139625\\
-0.01419	-0.01419\\
-0.01451	-0.01451\\
-0.01387	-0.01387\\
-0.013825	-0.013825\\
-0.0146475	-0.0146475\\
-0.014785	-0.014785\\
-0.0154275	-0.0154275\\
-0.0160225	-0.0160225\\
-0.01561	-0.01561\\
-0.0151525	-0.0151525\\
-0.0149225	-0.0149225\\
-0.0151525	-0.0151525\\
-0.01561	-0.01561\\
-0.015335	-0.015335\\
-0.0148325	-0.0148325\\
};
\end{axis}

\begin{axis}[%
width=4.927cm,
height=2.746cm,
at={(0cm,3.814cm)},
scale only axis,
xmin=-0.018,
xmax=-0.012,
xlabel style={font=\color{white!15!black}},
xlabel={$u(t-1)$},
ymin=-1,
ymax=0,
ylabel style={font=\color{white!15!black}},
ylabel={$\delta^4 y(t)$},
axis background/.style={fill=white},
title style={font=\bfseries},
title={C7, R = 0.6813},
axis x line*=bottom,
axis y line*=left
]
\addplot[only marks, mark=*, mark options={}, mark size=1.5000pt, color=mycolor1, fill=mycolor1] table[row sep=crcr]{%
x	y\\
-0.01561	-0.3692625\\
-0.0157025	-0.4394525\\
-0.0158375	-0.3356925\\
-0.015655	-0.1708975\\
-0.0149225	-0.201415\\
-0.014785	-0.201415\\
-0.0148325	-0.250245\\
-0.01497	-0.2044675\\
-0.0148775	-0.08545\\
-0.0140075	-0.061035\\
-0.0131825	-0.0579825\\
-0.0128175	-0.183105\\
-0.0137775	-0.320435\\
-0.014785	-0.3326425\\
-0.015015	-0.3448475\\
-0.0151975	-0.2319325\\
-0.0148325	-0.14038\\
-0.0142825	-0.31433\\
-0.0149225	-0.216675\\
-0.014785	-0.1708975\\
-0.014375	-0.286865\\
-0.014785	-0.250245\\
-0.0148325	-0.2136225\\
-0.014695	-0.357055\\
-0.015105	-0.3326425\\
-0.01529	-0.2563475\\
-0.01497	-0.375365\\
-0.01529	-0.36316\\
-0.0154275	-0.2838125\\
-0.0151975	-0.2319325\\
-0.01497	-0.1800525\\
-0.0146475	-0.216675\\
-0.014695	-0.2807625\\
-0.0149225	-0.2594\\
-0.0148775	-0.2746575\\
-0.0149225	-0.2807625\\
-0.01497	-0.305175\\
-0.01506	-0.4394525\\
-0.0155175	-0.3875725\\
-0.015565	-0.2563475\\
-0.0151975	-0.2319325\\
-0.01497	-0.1342775\\
-0.014375	-0.146485\\
-0.014145	-0.198365\\
-0.014375	-0.149535\\
-0.0142825	-0.15564\\
-0.014145	-0.2227775\\
-0.014465	-0.2563475\\
-0.014695	-0.2380375\\
-0.014695	-0.3784175\\
-0.0151975	-0.27771\\
-0.015105	-0.1617425\\
-0.01451	-0.128175\\
-0.014145	-0.17395\\
-0.0142375	-0.2044675\\
-0.014465	-0.1098625\\
-0.0140075	-0.112915\\
-0.0137325	-0.1678475\\
-0.0140075	-0.128175\\
-0.013915	-0.1068125\\
-0.0136875	-0.1525875\\
-0.013825	-0.1770025\\
-0.0140075	-0.2746575\\
-0.0145575	-0.26245\\
-0.014695	-0.3234875\\
-0.0149225	-0.302125\\
-0.01497	-0.3448475\\
-0.015105	-0.2746575\\
-0.015015	-0.268555\\
-0.0148775	-0.2319325\\
-0.0148325	-0.2288825\\
-0.01474	-0.2197275\\
-0.014695	-0.3936775\\
-0.01529	-0.5615225\\
-0.0160225	-0.579835\\
-0.0162975	-0.564575\\
-0.0162975	-0.3509525\\
-0.0158375	-0.512695\\
-0.0160675	-0.631715\\
-0.016525	-0.653075\\
-0.0166625	-0.5004875\\
-0.01648	-0.67749\\
-0.0167075	-0.8392325\\
-0.0172575	-0.58899\\
-0.01703	-0.479125\\
-0.0167075	-0.320435\\
-0.01616	-0.2471925\\
-0.015655	-0.2105725\\
-0.015335	-0.26245\\
-0.0154275	-0.1861575\\
-0.0151525	-0.1251225\\
-0.0146025	-0.12207\\
-0.0142825	-0.1586925\\
-0.0143275	-0.2380375\\
-0.014695	-0.216675\\
-0.014785	-0.2227775\\
-0.01474	-0.2990725\\
-0.01506	-0.3082275\\
-0.0151525	-0.41809\\
-0.015655	-0.3356925\\
-0.015565	-0.424195\\
-0.0157475	-0.305175\\
-0.015565	-0.2716075\\
-0.015335	-0.31433\\
-0.0154275	-0.2227775\\
-0.0151975	-0.201415\\
-0.01497	-0.24414\\
-0.01506	-0.2471925\\
-0.015105	-0.1708975\\
-0.014785	-0.183105\\
-0.014695	-0.13733\\
-0.014465	-0.1525875\\
-0.014375	-0.1617425\\
-0.01442	-0.112915\\
-0.01419	-0.146485\\
-0.014145	-0.1800525\\
-0.014375	-0.29297\\
-0.0149225	-0.305175\\
-0.015105	-0.3417975\\
-0.0152425	-0.19226\\
-0.014785	-0.2746575\\
-0.01497	-0.43335\\
-0.01561	-0.32959\\
-0.0155175	-0.3936775\\
-0.015655	-0.3936775\\
-0.0157025	-0.2136225\\
-0.0151525	-0.149535\\
-0.0146475	-0.2136225\\
-0.014695	-0.2471925\\
-0.0149225	-0.4028325\\
-0.0155175	-0.5065925\\
-0.0160225	-0.4974375\\
-0.0161125	-0.36316\\
-0.0158375	-0.375365\\
-0.0157475	-0.338745\\
-0.0157025	-0.32959\\
-0.01561	-0.3173825\\
-0.015655	-0.2136225\\
-0.0152425	-0.19226\\
-0.0149225	-0.17395\\
-0.014785	-0.15564\\
-0.0146025	-0.1770025\\
-0.0146475	-0.268555\\
-0.01497	-0.4119875\\
-0.01561	-0.31433\\
-0.0155175	-0.216675\\
-0.015105	-0.183105\\
-0.0148325	-0.1770025\\
-0.014695	-0.1068125\\
-0.0142375	-0.079345\\
-0.0136875	-0.0976575\\
-0.013595	-0.1800525\\
-0.014145	-0.1434325\\
-0.0140525	-0.149535\\
-0.0140075	-0.1678475\\
-0.014145	-0.1586925\\
-0.0141	-0.1098625\\
-0.013825	-0.0732425\\
-0.0134575	-0.0640875\\
-0.0130925	-0.1098625\\
-0.013275	-0.24109\\
-0.0142375	-0.3509525\\
-0.01497	-0.3875725\\
-0.015335	-0.253295\\
-0.01506	-0.17395\\
-0.0145575	-0.1251225\\
-0.014145	-0.08545\\
-0.013595	-0.20752\\
-0.01419	-0.2044675\\
-0.01442	-0.2990725\\
-0.014785	-0.36316\\
-0.0151975	-0.5767825\\
-0.0160225	-0.6561275\\
-0.0166175	-0.527955\\
-0.016525	-0.5065925\\
-0.0163875	-0.3417975\\
-0.015975	-0.3784175\\
-0.01593	-0.39978\\
-0.0160225	-0.4302975\\
-0.0161125	-0.3234875\\
-0.015885	-0.29602\\
-0.0157025	-0.2899175\\
-0.01561	-0.2594\\
-0.0155175	-0.3082275\\
-0.015565	-0.2197275\\
-0.015335	-0.3845225\\
-0.015655	-0.4730225\\
-0.0161125	-0.3326425\\
-0.0157925	-0.1953125\\
-0.0152425	-0.2136225\\
-0.01497	-0.36621\\
-0.01561	-0.4547125\\
-0.015975	-0.57373\\
-0.0163875	-0.6042475\\
-0.01657	-0.6011975\\
-0.0166625	-0.45166\\
-0.0164325	-0.408935\\
-0.016205	-0.46997\\
-0.0163875	-0.55542\\
-0.016525	-0.3479\\
-0.0160675	-0.2044675\\
-0.0154275	-0.305175\\
-0.0155175	-0.2563475\\
-0.0155175	-0.164795\\
-0.01497	-0.2197275\\
-0.015015	-0.2471925\\
-0.0151525	-0.19226\\
-0.01497	-0.146485\\
-0.014695	-0.201415\\
-0.014785	-0.164795\\
-0.014695	-0.2288825\\
-0.0148775	-0.3326425\\
-0.015335	-0.305175\\
-0.0154275	-0.198365\\
-0.01497	-0.20752\\
-0.0149225	-0.3173825\\
-0.0152425	-0.5249025\\
-0.0161125	-0.375365\\
-0.0160225	-0.2319325\\
-0.0154275	-0.17395\\
-0.01497	-0.20752\\
-0.0149225	-0.1861575\\
-0.0148775	-0.14038\\
-0.0146025	-0.15564\\
-0.014465	-0.1861575\\
-0.0146475	-0.24109\\
-0.0149225	-0.268555\\
-0.015015	-0.1800525\\
-0.01474	-0.305175\\
-0.0151975	-0.3601075\\
-0.0155175	-0.3448475\\
-0.0154725	-0.2655025\\
-0.01529	-0.29602\\
-0.015335	-0.3967275\\
-0.015655	-0.253295\\
-0.01529	-0.10376\\
-0.0142825	-0.146485\\
-0.01419	-0.164795\\
-0.014375	-0.2288825\\
-0.014695	-0.2044675\\
-0.014695	-0.149535\\
-0.01442	-0.24109\\
-0.01474	-0.2288825\\
-0.014785	-0.164795\\
-0.01451	-0.20752\\
-0.0146025	-0.19226\\
-0.0146025	-0.1678475\\
-0.014465	-0.198365\\
-0.0145575	-0.1159675\\
-0.01419	-0.1342775\\
-0.0140525	-0.1007075\\
-0.013915	-0.1098625\\
-0.013825	-0.216675\\
-0.01442	-0.250245\\
-0.014695	-0.3448475\\
-0.0151525	-0.442505\\
-0.015655	-0.2899175\\
-0.01538	-0.1708975\\
-0.0146475	-0.12207\\
-0.0142375	-0.12207\\
-0.0141	-0.183105\\
-0.014375	-0.1159675\\
-0.0140525	-0.131225\\
-0.0139625	-0.1770025\\
-0.01419	-0.2990725\\
-0.0148775	-0.268555\\
-0.01497	-0.375365\\
-0.01529	-0.2563475\\
-0.015105	-0.2288825\\
-0.0148325	-0.1190175\\
-0.0142375	-0.1190175\\
-0.013915	-0.079345\\
-0.013505	-0.08545\\
-0.0133675	-0.10376\\
-0.013505	-0.18921\\
-0.0140525	-0.20752\\
-0.0142375	-0.2319325\\
-0.014465	-0.24414\\
-0.0146025	-0.38147\\
-0.0151525	-0.3265375\\
-0.01529	-0.24414\\
-0.015015	-0.2990725\\
-0.015105	-0.3234875\\
-0.0151975	-0.41809\\
-0.0155175	-0.3326425\\
-0.0154725	-0.2105725\\
-0.015015	-0.1098625\\
-0.01419	-0.079345\\
-0.013595	-0.08545\\
-0.0134125	-0.10376\\
-0.0134575	-0.1098625\\
-0.013505	-0.1007075\\
-0.0134575	-0.14038\\
-0.0136425	-0.2197275\\
-0.01419	-0.198365\\
-0.0142825	-0.2044675\\
-0.0142825	-0.24109\\
-0.01451	-0.146485\\
-0.0141	-0.0732425\\
-0.0133675	-0.1617425\\
-0.0136875	-0.24414\\
-0.0143275	-0.24109\\
-0.014465	-0.2746575\\
-0.0146475	-0.234985\\
-0.0146025	-0.17395\\
-0.0142825	-0.24414\\
-0.01451	-0.3448475\\
-0.015015	-0.3479\\
-0.0151975	-0.24414\\
-0.0148775	-0.405885\\
-0.015335	-0.320435\\
-0.015335	-0.29602\\
-0.0151975	-0.3448475\\
-0.015335	-0.3448475\\
-0.015335	-0.2594\\
-0.0151525	-0.22583\\
-0.0149225	-0.3845225\\
-0.0154275	-0.53711\\
-0.0160225	-0.4119875\\
-0.01593	-0.2319325\\
-0.01529	-0.2319325\\
-0.015015	-0.2288825\\
-0.015015	-0.2838125\\
-0.0151975	-0.3417975\\
-0.0154275	-0.250245\\
-0.0151975	-0.26245\\
-0.0151525	-0.3509525\\
-0.0154275	-0.3845225\\
-0.015565	-0.2594\\
-0.0152425	-0.198365\\
-0.0148775	-0.26245\\
-0.015105	-0.4394525\\
-0.0157025	-0.39978\\
-0.0158375	-0.405885\\
-0.0157925	-0.2380375\\
-0.01529	-0.2105725\\
-0.0149225	-0.20752\\
-0.01497	-0.131225\\
-0.01442	-0.08545\\
-0.013825	-0.07019\\
-0.01332	-0.0579825\\
-0.013	-0.149535\\
-0.013595	-0.216675\\
-0.0142825	-0.26245\\
-0.0145575	-0.19226\\
-0.014375	-0.2197275\\
-0.01442	-0.24414\\
-0.0146025	-0.149535\\
-0.0142375	-0.1586925\\
-0.0140525	-0.1525875\\
-0.0140075	-0.112915\\
-0.0137775	-0.1434325\\
-0.01387	-0.24414\\
-0.014375	-0.31128\\
-0.0148325	-0.1953125\\
-0.0145575	-0.14038\\
-0.0141	-0.1159675\\
-0.0139625	-0.149535\\
-0.0139625	-0.10376\\
-0.0136875	-0.2288825\\
-0.0142825	-0.3967275\\
-0.0152425	-0.305175\\
-0.0151975	-0.41809\\
-0.0154725	-0.48523\\
-0.0158375	-0.494385\\
-0.0160225	-0.3509525\\
-0.0157025	-0.2655025\\
-0.015335	-0.2655025\\
-0.0152425	-0.26245\\
-0.0152425	-0.302125\\
-0.015335	-0.375365\\
-0.0155175	-0.3692625\\
-0.01561	-0.5004875\\
-0.01593	-0.5523675\\
-0.01625	-0.65918\\
-0.0166175	-0.4455575\\
-0.0163875	-0.476075\\
-0.01625	-0.531005\\
-0.01648	-0.408935\\
-0.016205	-0.4730225\\
-0.0162975	-0.408935\\
-0.016205	-0.2288825\\
-0.0155175	-0.146485\\
-0.014785	-0.15564\\
-0.0145575	-0.2380375\\
-0.0148775	-0.18921\\
-0.014785	-0.128175\\
-0.014375	-0.1800525\\
-0.01442	-0.131225\\
-0.0143275	-0.1007075\\
-0.01387	-0.12207\\
-0.013915	-0.14038\\
-0.0140075	-0.1007075\\
-0.0137775	-0.19226\\
-0.0142375	-0.2746575\\
-0.014695	-0.1953125\\
-0.0145575	-0.3356925\\
-0.01506	-0.4882825\\
-0.0157925	-0.4455575\\
-0.01593	-0.55542\\
-0.01616	-0.375365\\
-0.01593	-0.4028325\\
-0.015885	-0.27771\\
-0.01561	-0.1678475\\
-0.0148775	-0.2136225\\
-0.0148775	-0.1678475\\
-0.014695	-0.12207\\
-0.014375	-0.1800525\\
-0.014465	-0.10376\\
-0.0140075	-0.0640875\\
-0.0134125	-0.0915525\\
-0.01332	-0.198365\\
-0.014145	-0.26245\\
-0.014695	-0.320435\\
-0.015015	-0.31128\\
-0.015105	-0.29297\\
-0.01506	-0.3417975\\
-0.0152425	-0.27771\\
-0.015105	-0.22583\\
-0.0148775	-0.22583\\
-0.0148775	-0.183105\\
-0.014695	-0.2899175\\
-0.01497	-0.1953125\\
-0.0148325	-0.1678475\\
-0.014465	-0.250245\\
-0.014785	-0.3448475\\
-0.0151975	-0.2594\\
-0.015105	-0.19226\\
-0.01474	-0.1617425\\
-0.0145575	-0.1861575\\
-0.01451	-0.128175\\
-0.0142825	-0.1251225\\
-0.0140525	-0.1800525\\
-0.0142375	-0.216675\\
-0.01451	-0.2471925\\
-0.014695	-0.198365\\
-0.0145575	-0.253295\\
-0.014695	-0.2319325\\
-0.01474	-0.131225\\
-0.0142375	-0.13733\\
-0.0141	-0.2807625\\
-0.01474	-0.3967275\\
-0.015335	-0.3845225\\
-0.0154725	-0.3234875\\
-0.015335	-0.305175\\
-0.01529	-0.2319325\\
-0.01506	-0.2227775\\
-0.0148775	-0.1770025\\
-0.014695	-0.2563475\\
-0.0148325	-0.22583\\
-0.0148775	-0.13733\\
-0.01442	-0.2105725\\
-0.0145575	-0.253295\\
-0.0148325	-0.29602\\
-0.01497	-0.3234875\\
-0.0151975	-0.3234875\\
-0.0151975	-0.4394525\\
-0.01561	-0.54016\\
-0.0160675	-0.6103525\\
-0.0164325	-0.3845225\\
-0.0161125	-0.2136225\\
-0.0152425	-0.13733\\
-0.0146475	-0.094605\\
-0.0140075	-0.1434325\\
-0.0140525	-0.1342775\\
-0.0140525	-0.2380375\\
-0.01451	-0.2136225\\
-0.014695	-0.18921\\
-0.0145575	-0.2563475\\
-0.014785	-0.29602\\
-0.015015	-0.354005\\
-0.0152425	-0.372315\\
-0.0154275	-0.52185\\
-0.015975	-0.4547125\\
-0.0161125	-0.3417975\\
-0.0157925	-0.22583\\
-0.01529	-0.2319325\\
-0.015105	-0.2380375\\
-0.015105	-0.3082275\\
-0.01529	-0.2197275\\
-0.01506	-0.2227775\\
-0.01497	-0.2197275\\
-0.0149225	-0.1190175\\
-0.01442	-0.2044675\\
-0.0146025	-0.320435\\
-0.015105	-0.22583\\
-0.01497	-0.24109\\
-0.0148775	-0.427245\\
-0.01561	-0.3234875\\
-0.015565	-0.305175\\
-0.01538	-0.442505\\
-0.0157475	-0.41504\\
-0.0158375	-0.2563475\\
-0.0154275	-0.1617425\\
-0.014785	-0.164795\\
-0.0146025	-0.286865\\
-0.015105	-0.476075\\
-0.0158375	-0.494385\\
-0.01616	-0.5157475\\
-0.016205	-0.375365\\
-0.01593	-0.24414\\
-0.0154275	-0.20752\\
-0.01506	-0.146485\\
-0.014695	-0.14038\\
-0.01442	-0.1190175\\
-0.0142825	-0.1708975\\
-0.01442	-0.2044675\\
-0.0146025	-0.1159675\\
-0.01419	-0.1770025\\
-0.014375	-0.2594\\
-0.01474	-0.286865\\
-0.01497	-0.36316\\
-0.015335	-0.4638675\\
-0.0157925	-0.5584725\\
-0.01625	-0.4028325\\
-0.0160675	-0.3448475\\
-0.0157925	-0.3601075\\
-0.0157925	-0.302125\\
-0.015655	-0.2105725\\
-0.0151975	-0.2594\\
-0.0152425	-0.41809\\
-0.0157025	-0.48523\\
-0.0160675	-0.2899175\\
-0.01561	-0.15564\\
-0.0148325	-0.10376\\
-0.0142375	-0.0579825\\
-0.01332	-0.079345\\
-0.0130925	-0.128175\\
-0.0134575	-0.1525875\\
-0.0137325	-0.201415\\
-0.0141	-0.1678475\\
-0.0141	-0.128175\\
-0.013825	-0.1251225\\
-0.0137775	-0.12207\\
-0.0137325	-0.112915\\
-0.0136425	-0.131225\\
-0.0136875	-0.2044675\\
-0.014145	-0.3082275\\
-0.01474	-0.3326425\\
-0.01506	-0.31433\\
-0.015105	-0.38147\\
-0.015335	-0.4547125\\
-0.015655	-0.4577625\\
-0.0158375	-0.5584725\\
-0.01616	-0.50354\\
-0.0162975	-0.36621\\
-0.01593	-0.253295\\
-0.0154275	-0.24109\\
-0.0151975	-0.408935\\
-0.0157475	-0.479125\\
-0.0160675	-0.32959\\
-0.0157925	-0.2136225\\
-0.0152425	-0.112915\\
-0.014375	-0.079345\\
-0.0136875	-0.0885\\
-0.013505	-0.0549325\\
-0.013	-0.1159675\\
-0.01332	-0.1708975\\
-0.01387	-0.1800525\\
-0.0139625	-0.15564\\
-0.013915	-0.0915525\\
-0.0134575	-0.07019\\
-0.0131375	-0.10376\\
-0.01323	-0.1098625\\
-0.0133675	-0.12207\\
-0.0134125	-0.0915525\\
-0.01323	-0.076295\\
-0.013	-0.13733\\
-0.0134125	-0.320435\\
-0.0146025	-0.3845225\\
-0.01529	-0.427245\\
-0.015565	-0.31128\\
-0.01538	-0.4302975\\
-0.01561	-0.6042475\\
-0.0162975	-0.55542\\
-0.0164325	-0.5615225\\
-0.0164325	-0.4119875\\
-0.016205	-0.408935\\
-0.0160675	-0.4302975\\
-0.0161125	-0.2838125\\
-0.0157025	-0.2655025\\
-0.0154725	-0.164795\\
-0.0149225	-0.149535\\
-0.0145575	-0.29297\\
-0.01506	-0.198365\\
-0.0148775	-0.2044675\\
-0.01474	-0.24109\\
-0.0149225	-0.216675\\
-0.0148775	-0.2197275\\
-0.0148325	-0.234985\\
-0.0148775	-0.268555\\
-0.015015	-0.2990725\\
-0.0151525	-0.2594\\
-0.015105	-0.1861575\\
-0.0148325	-0.24109\\
-0.0148775	-0.12207\\
-0.014375	-0.05188\\
-0.01332	-0.08545\\
-0.0131825	-0.146485\\
-0.0136425	-0.076295\\
-0.01323	-0.03357\\
-0.0124975	-0.0823975\\
-0.0125875	-0.146485\\
-0.01323	-0.17395\\
-0.013595	-0.20752\\
-0.0140075	-0.1800525\\
-0.0140075	-0.29297\\
-0.0145575	-0.26245\\
-0.014695	-0.1770025\\
-0.0142825	-0.268555\\
-0.0146025	-0.149535\\
-0.014145	-0.1159675\\
-0.0136875	-0.0823975\\
-0.01332	-0.0549325\\
-0.0128175	-0.10376\\
-0.013	-0.08545\\
-0.01291	-0.1342775\\
-0.013275	-0.2288825\\
-0.0139625	-0.2197275\\
-0.014145	-0.149535\\
-0.013915	-0.1708975\\
-0.01387	-0.131225\\
-0.0136875	-0.1861575\\
-0.01387	-0.1342775\\
-0.0136875	-0.1251225\\
-0.013505	-0.1159675\\
-0.0134575	-0.0915525\\
-0.01323	-0.112915\\
-0.01332	-0.10376\\
-0.01332	-0.18921\\
-0.0137775	-0.183105\\
-0.013915	-0.13733\\
-0.0136875	-0.128175\\
-0.01355	-0.1007075\\
-0.0133675	-0.12207\\
-0.0134125	-0.2716075\\
-0.01419	-0.3601075\\
-0.0149225	-0.24109\\
-0.014695	-0.22583\\
-0.0145575	-0.1586925\\
-0.01419	-0.27771\\
-0.0145575	-0.253295\\
-0.01474	-0.1678475\\
-0.0142825	-0.20752\\
-0.0143275	-0.1617425\\
-0.014145	-0.2227775\\
-0.0143275	-0.131225\\
-0.0140075	-0.13733\\
-0.0137325	-0.164795\\
-0.0139625	-0.2105725\\
-0.01419	-0.15564\\
-0.0140075	-0.164795\\
-0.013915	-0.1708975\\
-0.0140525	-0.2807625\\
-0.01451	-0.2380375\\
-0.0146025	-0.36621\\
-0.015015	-0.476075\\
-0.015655	-0.3784175\\
-0.015565	-0.2380375\\
-0.01506	-0.1800525\\
-0.0146025	-0.2838125\\
-0.0148775	-0.354005\\
-0.0152425	-0.2716075\\
-0.015105	-0.3234875\\
-0.0151975	-0.2044675\\
-0.0148775	-0.1068125\\
-0.0140075	-0.1068125\\
-0.0137325	-0.2471925\\
-0.01442	-0.2227775\\
-0.0146025	-0.198365\\
-0.01442	-0.31128\\
-0.0148775	-0.29602\\
-0.01497	-0.2594\\
-0.0148775	-0.1678475\\
-0.014465	-0.2105725\\
-0.014465	-0.2899175\\
-0.0148775	-0.2380375\\
-0.014785	-0.24414\\
-0.01474	-0.2227775\\
-0.014695	-0.15564\\
-0.014375	-0.18921\\
-0.014375	-0.13733\\
-0.014145	-0.15564\\
-0.014145	-0.2655025\\
-0.0146025	-0.3082275\\
-0.0149225	-0.32959\\
-0.015015	-0.29297\\
-0.015015	-0.20752\\
-0.014695	-0.146485\\
-0.0142825	-0.1678475\\
-0.0142375	-0.201415\\
-0.01442	-0.26245\\
-0.014695	-0.3265375\\
-0.015015	-0.3875725\\
-0.01529	-0.286865\\
-0.015105	-0.17395\\
-0.0146025	-0.3173825\\
-0.01497	-0.4455575\\
-0.015655	-0.320435\\
-0.0154275	-0.31433\\
-0.0152425	-0.3692625\\
-0.0154275	-0.27771\\
-0.01529	-0.22583\\
-0.01497	-0.250245\\
-0.015015	-0.2319325\\
-0.0149225	-0.2594\\
-0.01497	-0.32959\\
-0.0152425	-0.253295\\
-0.01506	-0.2807625\\
-0.01506	-0.2655025\\
-0.015105	-0.2716075\\
-0.01506	-0.216675\\
-0.0149225	-0.286865\\
-0.01506	-0.2044675\\
-0.0148325	-0.216675\\
-0.01474	-0.24109\\
-0.0148775	-0.234985\\
-0.0148325	-0.1068125\\
-0.014145	-0.0671375\\
-0.0134575	-0.0457775\\
-0.0128625	-0.0976575\\
-0.0130475	-0.24109\\
-0.0141	-0.31433\\
-0.0148325	-0.31433\\
-0.01497	-0.20752\\
-0.0146475	-0.24414\\
-0.0146475	-0.2136225\\
-0.0146025	-0.18921\\
-0.014465	-0.1159675\\
-0.0140525	-0.164795\\
-0.0140525	-0.164795\\
-0.0141	-0.1617425\\
-0.0140525	-0.19226\\
-0.01419	-0.164795\\
-0.014145	-0.149535\\
-0.0140525	-0.1007075\\
-0.0136875	-0.146485\\
-0.0137775	-0.2746575\\
-0.01451	-0.1953125\\
-0.014375	-0.2319325\\
-0.014465	-0.20752\\
-0.014465	-0.29602\\
-0.01474	-0.2746575\\
-0.0148775	-0.38147\\
-0.0151525	-0.27771\\
-0.01506	-0.31128\\
-0.015015	-0.31128\\
-0.015105	-0.1586925\\
-0.01451	-0.2838125\\
-0.01474	-0.19226\\
-0.0146475	-0.24109\\
-0.0146475	-0.2563475\\
-0.01474	-0.3234875\\
-0.01497	-0.24414\\
-0.0148325	-0.31433\\
-0.015015	-0.3967275\\
-0.015335	-0.32959\\
-0.015335	-0.286865\\
-0.015105	-0.22583\\
-0.0149225	-0.268555\\
-0.01497	-0.2227775\\
-0.0148325	-0.19226\\
-0.0146475	-0.1708975\\
-0.01451	-0.2044675\\
-0.0145575	-0.17395\\
-0.01451	-0.354005\\
-0.01506	-0.4547125\\
-0.015655	-0.3936775\\
-0.01561	-0.375365\\
-0.015565	-0.3692625\\
-0.015565	-0.2044675\\
-0.01497	-0.1800525\\
-0.0146025	-0.13733\\
-0.01442	-0.1251225\\
-0.01419	-0.13733\\
-0.014145	-0.234985\\
-0.0145575	-0.29602\\
-0.0149225	-0.3356925\\
-0.0151525	-0.286865\\
-0.015105	-0.19226\\
-0.014695	-0.3509525\\
-0.015105	-0.408935\\
-0.015565	-0.4577625\\
-0.0157475	-0.375365\\
-0.015655	-0.59204\\
-0.01616	-0.3967275\\
-0.0160225	-0.234985\\
-0.01529	-0.1708975\\
-0.01474	-0.14038\\
-0.01442	-0.253295\\
-0.014785	-0.3265375\\
-0.0152425	-0.4302975\\
-0.015655	-0.3234875\\
-0.0155175	-0.2563475\\
-0.0151975	-0.1342775\\
-0.0146025	-0.1342775\\
-0.0142375	-0.1617425\\
-0.0143275	-0.1770025\\
-0.01442	-0.1098625\\
-0.0141	-0.1525875\\
-0.0140525	-0.112915\\
-0.0140075	-0.0732425\\
-0.0134575	-0.164795\\
-0.01387	-0.18921\\
-0.014145	-0.24109\\
-0.01451	-0.2227775\\
-0.0145575	-0.234985\\
-0.0145575	-0.234985\\
-0.0145575	-0.302125\\
-0.014785	-0.198365\\
-0.0146025	-0.2990725\\
-0.014785	-0.3417975\\
-0.0151525	-0.2594\\
-0.01497	-0.2716075\\
-0.0149225	-0.234985\\
-0.0148325	-0.2716075\\
-0.0149225	-0.3784175\\
-0.01529	-0.29297\\
-0.0151975	-0.234985\\
-0.01497	-0.2563475\\
-0.01497	-0.234985\\
-0.0148775	-0.15564\\
-0.0145575	-0.250245\\
-0.014785	-0.36316\\
-0.0152425	-0.3479\\
-0.015335	-0.3784175\\
-0.0154275	-0.5676275\\
-0.0161125	-0.3479\\
-0.0157925	-0.305175\\
-0.0154725	-0.2288825\\
-0.0151525	-0.2990725\\
-0.0152425	-0.4302975\\
-0.0157475	-0.286865\\
-0.0154725	-0.2563475\\
-0.0151975	-0.357055\\
-0.0155175	-0.2288825\\
-0.0151975	-0.1342775\\
-0.0145575	-0.1770025\\
-0.014465	-0.131225\\
-0.014375	-0.1068125\\
-0.0140525	-0.1159675\\
-0.013915	-0.1007075\\
-0.013825	-0.0976575\\
-0.0136875	-0.1342775\\
-0.01387	-0.19226\\
-0.014145	-0.22583\\
-0.014465	-0.2990725\\
-0.0148325	-0.2807625\\
-0.0149225	-0.338745\\
-0.015105	-0.3936775\\
-0.0154275	-0.354005\\
-0.0154275	-0.24109\\
-0.015105	-0.2655025\\
-0.015015	-0.2380375\\
-0.01497	-0.2594\\
-0.01497	-0.3692625\\
-0.015335	-0.2746575\\
-0.0152425	-0.305175\\
-0.0151975	-0.2655025\\
-0.0151975	-0.2563475\\
-0.01506	-0.3967275\\
-0.0154725	-0.3265375\\
-0.0154275	-0.3234875\\
-0.01538	-0.305175\\
-0.015335	-0.183105\\
-0.0148325	-0.1190175\\
-0.0142375	-0.094605\\
-0.013825	-0.17395\\
-0.01419	-0.15564\\
-0.0142375	-0.216675\\
-0.01442	-0.164795\\
-0.0143275	-0.234985\\
-0.0146025	-0.38147\\
-0.0151975	-0.4730225\\
-0.0157925	-0.375365\\
-0.0157025	-0.3845225\\
-0.015655	-0.3448475\\
-0.01561	-0.24109\\
-0.0151975	-0.2594\\
-0.0151525	-0.29297\\
-0.0151975	-0.250245\\
-0.015105	-0.286865\\
-0.0151975	-0.36316\\
-0.0154725	-0.5065925\\
-0.01593	-0.4547125\\
-0.0160225	-0.427245\\
-0.015975	-0.479125\\
-0.0160675	-0.32959\\
-0.0157475	-0.3601075\\
-0.015655	-0.2716075\\
-0.0154725	-0.2807625\\
-0.01538	-0.3692625\\
-0.015655	-0.4302975\\
-0.0158375	-0.198365\\
-0.0151525	-0.2227775\\
-0.0148325	-0.1678475\\
-0.01474	-0.2471925\\
-0.0148775	-0.250245\\
-0.01506	-0.1525875\\
-0.0146025	-0.2044675\\
-0.0146475	-0.3479\\
-0.01529	-0.5859375\\
-0.016205	-0.5676275\\
-0.01648	-0.5157475\\
-0.0164325	-0.41504\\
-0.016205	-0.48523\\
-0.01625	-0.582885\\
-0.016525	-0.427245\\
-0.0162975	-0.442505\\
-0.01625	-0.460815\\
-0.01625	-0.31433\\
-0.015885	-0.2655025\\
-0.01561	-0.3448475\\
-0.0157475	-0.2288825\\
-0.0154725	-0.1678475\\
-0.01497	-0.24414\\
-0.015105	-0.183105\\
-0.01497	-0.216675\\
-0.0149225	-0.2044675\\
-0.01497	-0.250245\\
-0.01506	-0.3326425\\
-0.01538	-0.2563475\\
-0.01529	-0.18921\\
-0.0149225	-0.2227775\\
-0.01497	-0.26245\\
-0.0151525	-0.31128\\
-0.015335	-0.268555\\
-0.01529	-0.216675\\
-0.015015	-0.3417975\\
-0.01538	-0.3234875\\
-0.0154725	-0.22583\\
-0.0151525	-0.372315\\
-0.0155175	-0.3356925\\
-0.01561	-0.24414\\
-0.01529	-0.1617425\\
-0.0148325	-0.1159675\\
-0.01442	-0.0823975\\
-0.01387	-0.18921\\
-0.0143275	-0.1800525\\
-0.01451	-0.131225\\
-0.01419	-0.0915525\\
-0.013915	-0.1190175\\
-0.013825	-0.2044675\\
-0.0143275	-0.234985\\
-0.0146025	-0.31128\\
-0.01497	-0.357055\\
-0.01529	-0.3448475\\
-0.01538	-0.36316\\
-0.0154275	-0.2197275\\
-0.01506	-0.0915525\\
-0.0141	-0.1190175\\
-0.013915	-0.1159675\\
-0.0139625	-0.1617425\\
-0.0140525	-0.250245\\
-0.0146025	-0.27771\\
-0.0148775	-0.4211425\\
-0.0154725	-0.2655025\\
-0.0152425	-0.26245\\
-0.015015	-0.3875725\\
-0.0155175	-0.2655025\\
-0.01529	-0.1861575\\
-0.0148775	-0.1342775\\
-0.01442	-0.19226\\
-0.014465	-0.250245\\
-0.0148325	-0.3784175\\
-0.01538	-0.2471925\\
-0.0151525	-0.2136225\\
-0.0148325	-0.201415\\
-0.0148325	-0.253295\\
-0.0149225	-0.3326425\\
-0.0152425	-0.5188\\
-0.01593	-0.460815\\
-0.0160675	-0.45166\\
-0.015975	-0.2807625\\
-0.015565	-0.18921\\
-0.01497	-0.2197275\\
-0.0149225	-0.0915525\\
-0.014145	-0.1678475\\
-0.014145	-0.131225\\
-0.01419	-0.164795\\
-0.01419	-0.29297\\
-0.0149225	-0.3845225\\
-0.0154275	-0.442505\\
-0.0157475	-0.5859375\\
-0.0162975	-0.4913325\\
-0.0163425	-0.2655025\\
-0.01561	-0.2471925\\
-0.0152425	-0.198365\\
-0.01506	-0.2716075\\
-0.0151525	-0.354005\\
-0.0155175	-0.479125\\
-0.015975	-0.3509525\\
-0.0157925	-0.24414\\
-0.015335	-0.2716075\\
-0.015335	-0.3601075\\
-0.015655	-0.2563475\\
-0.01538	-0.183105\\
-0.01497	-0.1434325\\
-0.0146025	-0.1007075\\
-0.0141	-0.0915525\\
-0.01387	-0.0732425\\
-0.0134575	-0.13733\\
-0.0137325	-0.201415\\
-0.0142375	-0.216675\\
-0.01442	-0.1617425\\
-0.0142825	-0.146485\\
-0.014145	-0.0671375\\
-0.0134575	-0.0396725\\
-0.012635	-0.061035\\
-0.0124975	-0.1159675\\
-0.0130925	-0.1770025\\
-0.0137325	-0.2197275\\
-0.0141	-0.2563475\\
-0.01442	-0.2990725\\
-0.01474	-0.3845225\\
-0.0151975	-0.531005\\
-0.01593	-0.5340575\\
-0.01625	-0.5767825\\
-0.0163425	-0.50354\\
-0.0163425	-0.27771\\
-0.015655	-0.2563475\\
-0.01529	-0.253295\\
-0.01529	-0.338745\\
-0.0154725	-0.2838125\\
-0.0154275	-0.201415\\
-0.015105	-0.1525875\\
-0.014695	-0.1251225\\
-0.014375	-0.234985\\
-0.01474	-0.22583\\
-0.0148325	-0.1190175\\
-0.0142825	-0.094605\\
-0.0137775	-0.146485\\
-0.0139625	-0.1953125\\
-0.0143275	-0.149535\\
-0.0142375	-0.216675\\
-0.01442	-0.1678475\\
-0.014375	-0.2471925\\
-0.0145575	-0.354005\\
-0.0151975	-0.2807625\\
-0.0151975	-0.3845225\\
-0.0154275	-0.5157475\\
-0.015975	-0.5615225\\
-0.0162975	-0.4394525\\
-0.016205	-0.27771\\
-0.01561	-0.20752\\
-0.015105	-0.12207\\
-0.01451	-0.2044675\\
-0.0146475	-0.1434325\\
-0.01451	-0.2746575\\
-0.0148775	-0.2319325\\
-0.0149225	-0.1861575\\
-0.0146475	-0.24109\\
-0.0148325	-0.131225\\
-0.014465	-0.2105725\\
-0.0145575	-0.146485\\
-0.014375	-0.094605\\
-0.01387	-0.112915\\
-0.013825	-0.0823975\\
-0.0136425	-0.1007075\\
-0.01355	-0.0885\\
-0.013505	-0.0671375\\
-0.01323	-0.0915525\\
-0.01323	-0.12207\\
-0.0134575	-0.1251225\\
-0.01355	-0.12207\\
-0.01355	-0.198365\\
-0.0140075	-0.2594\\
-0.01451	-0.2105725\\
-0.014465	-0.1953125\\
-0.0143275	-0.320435\\
-0.0148775	-0.3845225\\
-0.01529	-0.2990725\\
-0.0151525	-0.31433\\
-0.0151525	-0.14038\\
-0.01442	-0.0915525\\
-0.0136875	-0.183105\\
-0.0140525	-0.2655025\\
-0.014695	-0.2899175\\
-0.0148325	-0.405885\\
-0.01538	-0.36621\\
-0.0154725	-0.2563475\\
-0.015105	-0.183105\\
-0.01474	-0.198365\\
-0.0146475	-0.2105725\\
-0.014695	-0.1617425\\
-0.01451	-0.0976575\\
-0.0139625	-0.131225\\
-0.01387	-0.0976575\\
-0.0137775	-0.0823975\\
-0.0134575	-0.061035\\
-0.0131375	-0.1342775\\
-0.013505	-0.1861575\\
-0.0139625	-0.198365\\
-0.01419	-0.14038\\
-0.01387	-0.250245\\
-0.0143275	-0.3417975\\
-0.01497	-0.216675\\
-0.014695	-0.302125\\
-0.0148775	-0.3265375\\
-0.0151525	-0.2471925\\
-0.0148775	-0.14038\\
-0.0143275	-0.1800525\\
-0.0142375	-0.2197275\\
-0.014465	-0.1190175\\
-0.0140525	-0.1251225\\
-0.013915	-0.10376\\
-0.0137325	-0.061035\\
-0.013275	-0.0671375\\
-0.0130475	-0.1159675\\
-0.01332	-0.1617425\\
-0.0136875	-0.13733\\
-0.0136875	-0.1800525\\
-0.01387	-0.198365\\
-0.0140525	-0.1434325\\
-0.013825	-0.079345\\
-0.01332	-0.0640875\\
-0.013	-0.08545\\
-0.013	-0.0976575\\
-0.0131825	-0.12207\\
-0.013275	-0.26245\\
-0.01419	-0.253295\\
-0.0145575	-0.2227775\\
-0.01442	-0.2136225\\
-0.01442	-0.29297\\
-0.0146475	-0.375365\\
-0.015105	-0.39978\\
-0.0154275	-0.41504\\
-0.0155175	-0.302125\\
-0.0152425	-0.3173825\\
-0.0151975	-0.32959\\
-0.01529	-0.3326425\\
-0.015335	-0.3784175\\
-0.0154275	-0.5004875\\
-0.015885	-0.424195\\
-0.015885	-0.390625\\
-0.0157475	-0.320435\\
-0.015565	-0.2990725\\
-0.0154275	-0.286865\\
-0.015335	-0.3417975\\
-0.0154725	-0.20752\\
-0.0151525	-0.2471925\\
-0.015015	-0.2990725\\
-0.0151975	-0.3509525\\
-0.0154275	-0.268555\\
-0.0152425	-0.31128\\
-0.01529	-0.2471925\\
-0.0151975	-0.22583\\
-0.015015	-0.13733\\
-0.0145575	-0.2136225\\
-0.0146475	-0.3448475\\
-0.0152425	-0.2746575\\
-0.0151975	-0.1617425\\
-0.0146475	-0.201415\\
-0.0146025	-0.10376\\
-0.0141	-0.1007075\\
-0.0137325	-0.14038\\
-0.013915	-0.0823975\\
-0.0136425	-0.0915525\\
-0.0134575	-0.1098625\\
-0.01355	-0.0671375\\
-0.01323	-0.1251225\\
-0.0134125	-0.1007075\\
-0.0134575	-0.0579825\\
-0.0130475	-0.1098625\\
-0.013275	-0.128175\\
-0.0134575	-0.1586925\\
-0.0136875	-0.15564\\
-0.0137325	-0.1586925\\
-0.0137775	-0.2105725\\
-0.0141	-0.19226\\
-0.0141	-0.216675\\
-0.0142375	-0.36621\\
-0.01497	-0.253295\\
-0.0149225	-0.2319325\\
-0.014695	-0.250245\\
-0.01474	-0.17395\\
-0.014465	-0.1007075\\
-0.01387	-0.198365\\
-0.01419	-0.1525875\\
-0.0141	-0.094605\\
-0.0136425	-0.1678475\\
-0.013915	-0.15564\\
-0.0140075	-0.201415\\
-0.014145	-0.12207\\
-0.013825	-0.0732425\\
-0.0133675	-0.0579825\\
-0.013	-0.0732425\\
-0.01291	-0.14038\\
-0.0134125	-0.13733\\
-0.01355	-0.164795\\
-0.0136875	-0.2197275\\
-0.0141	-0.29297\\
-0.01451	-0.2105725\\
-0.014465	-0.31128\\
-0.014695	-0.354005\\
-0.015105	-0.31128\\
-0.01506	-0.29297\\
-0.015015	-0.4882825\\
-0.01561	-0.338745\\
-0.015565	-0.43335\\
-0.015655	-0.354005\\
-0.01561	-0.4119875\\
-0.015655	-0.460815\\
-0.01593	-0.253295\\
-0.015335	-0.1586925\\
-0.014695	-0.128175\\
-0.0142375	-0.2105725\\
-0.0145575	-0.26245\\
-0.0148775	-0.2716075\\
-0.0149225	-0.1861575\\
-0.0146475	-0.1007075\\
-0.0140075	-0.131225\\
-0.013915	-0.26245\\
-0.0146475	-0.36316\\
-0.0152425	-0.357055\\
-0.01538	-0.2105725\\
-0.0148325	-0.1617425\\
-0.014465	-0.2227775\\
-0.0146475	-0.354005\\
-0.0151975	-0.4028325\\
-0.0155175	-0.4119875\\
-0.01561	-0.2899175\\
-0.015335	-0.41504\\
-0.01561	-0.4455575\\
-0.015885	-0.427245\\
-0.0157925	-0.6011975\\
-0.0162975	-0.5096425\\
-0.0163425	-0.424195\\
-0.0161125	-0.4913325\\
-0.01625	-0.4119875\\
-0.01616	-0.2288825\\
-0.0154275	-0.2105725\\
-0.015015	-0.26245\\
-0.0151975	-0.31433\\
-0.0154275	-0.320435\\
-0.0154725	-0.24109\\
-0.0152425	-0.31128\\
-0.01538	-0.2655025\\
-0.01529	-0.1953125\\
-0.015015	-0.2655025\\
-0.0151525	-0.2380375\\
-0.0151525	-0.375365\\
-0.0155175	-0.3784175\\
-0.015655	-0.2716075\\
-0.0154275	-0.198365\\
-0.01506	-0.112915\\
-0.01442	-0.1953125\\
-0.0145575	-0.1342775\\
-0.014375	-0.131225\\
-0.014145	-0.1068125\\
-0.0140075	-0.19226\\
-0.0143275	-0.2471925\\
-0.01474	-0.2807625\\
-0.0149225	-0.2746575\\
-0.01497	-0.1617425\\
-0.0145575	-0.12207\\
-0.0141	-0.0885\\
-0.0137775	-0.1098625\\
-0.0136875	-0.1770025\\
-0.0140525	-0.18921\\
-0.0142825	-0.1861575\\
-0.0142825	-0.302125\\
-0.0148325	-0.3509525\\
-0.0152425	-0.405885\\
-0.0155175	-0.39978\\
-0.01561	-0.46997\\
-0.0158375	-0.2746575\\
-0.01538	-0.2288825\\
-0.01506	-0.1800525\\
-0.0148325	-0.1953125\\
-0.01474	-0.26245\\
-0.01497	-0.216675\\
-0.0148775	-0.1434325\\
-0.01451	-0.1251225\\
-0.0142825	-0.24414\\
-0.014695	-0.29297\\
-0.015105	-0.268555\\
-0.015015	-0.4211425\\
-0.0155175	-0.460815\\
-0.0158375	-0.3265375\\
-0.01561	-0.2288825\\
-0.015105	-0.15564\\
-0.0146475	-0.1434325\\
-0.014375	-0.268555\\
-0.0148775	-0.2136225\\
-0.0148775	-0.2594\\
-0.01497	-0.2838125\\
-0.01506	-0.3234875\\
-0.0151975	-0.305175\\
-0.0152425	-0.338745\\
-0.015335	-0.2288825\\
-0.01506	-0.2746575\\
-0.01506	-0.18921\\
-0.014785	-0.1708975\\
-0.0146025	-0.268555\\
-0.0149225	-0.3692625\\
-0.01538	-0.405885\\
-0.01561	-0.2288825\\
-0.0151525	-0.476075\\
-0.0157475	-0.427245\\
-0.0160225	-0.2838125\\
-0.015565	-0.164795\\
-0.0148325	-0.24109\\
-0.0148775	-0.20752\\
-0.0149225	-0.201415\\
-0.014785	-0.2471925\\
-0.01497	-0.1617425\\
-0.014695	-0.234985\\
-0.0148325	-0.427245\\
-0.01561	-0.4486075\\
-0.01593	-0.29297\\
-0.0155175	-0.3356925\\
-0.0155175	-0.36621\\
-0.01561	-0.372315\\
-0.015655	-0.2655025\\
-0.0154275	-0.2655025\\
-0.0152425	-0.17395\\
-0.0148775	-0.2655025\\
-0.01506	-0.2746575\\
-0.0151975	-0.4394525\\
-0.0157475	-0.4882825\\
-0.0161125	-0.65918\\
-0.01657	-0.476075\\
-0.0164325	-0.3875725\\
-0.0160675	-0.29602\\
-0.0157925	-0.3265375\\
-0.0157475	-0.24414\\
-0.0154725	-0.1678475\\
-0.015015	-0.1708975\\
-0.01474	-0.1708975\\
-0.014785	-0.12207\\
-0.01442	-0.0885\\
-0.0140075	-0.149535\\
-0.014145	-0.2044675\\
-0.014465	-0.13733\\
-0.0142825	-0.0976575\\
-0.013825	-0.094605\\
-0.0136875	-0.22583\\
-0.01442	-0.320435\\
-0.01506	-0.149535\\
-0.014465	-0.1861575\\
-0.01442	-0.198365\\
-0.01451	-0.183105\\
-0.01442	-0.2288825\\
-0.0146025	-0.198365\\
-0.0145575	-0.13733\\
-0.0142375	-0.10376\\
-0.0139625	-0.08545\\
-0.0136425	-0.1098625\\
-0.0136875	-0.216675\\
-0.0142825	-0.2594\\
-0.014695	-0.1800525\\
-0.014465	-0.12207\\
-0.0140525	-0.0732425\\
-0.013595	-0.1068125\\
-0.013595	-0.201415\\
-0.014145	-0.253295\\
-0.0145575	-0.268555\\
-0.01474	-0.19226\\
-0.014465	-0.1770025\\
-0.0143275	-0.1953125\\
-0.014375	-0.2716075\\
-0.014695	-0.3784175\\
-0.0152425	-0.4394525\\
-0.01561	-0.32959\\
-0.0154725	-0.20752\\
-0.0149225	-0.216675\\
-0.014785	-0.2044675\\
-0.014785	-0.2105725\\
-0.01474	-0.13733\\
-0.014375	-0.17395\\
-0.01442	-0.19226\\
-0.014465	-0.19226\\
-0.01451	-0.112915\\
-0.0141	-0.0732425\\
-0.013595	-0.1190175\\
-0.0136875	-0.1770025\\
-0.014145	-0.2746575\\
-0.0146025	-0.29602\\
-0.01497	-0.372315\\
-0.0152425	-0.3265375\\
-0.01529	-0.36621\\
-0.01538	-0.4882825\\
-0.015885	-0.650025\\
-0.01648	-0.531005\\
-0.016525	-0.357055\\
-0.0160225	-0.3875725\\
-0.0160225	-0.45166\\
-0.0161125	-0.59204\\
-0.01648	-0.3875725\\
-0.01625	-0.19226\\
-0.01529	-0.1251225\\
-0.01451	-0.131225\\
-0.0142825	-0.094605\\
-0.013915	-0.0579825\\
-0.013275	-0.076295\\
-0.0131825	-0.12207\\
-0.013505	-0.1708975\\
-0.01387	-0.1770025\\
-0.0140525	-0.13733\\
-0.013915	-0.12207\\
-0.0137325	-0.1678475\\
-0.013915	-0.2044675\\
-0.01419	-0.2136225\\
-0.0143275	-0.198365\\
-0.0143275	-0.1434325\\
-0.0140075	-0.094605\\
-0.0136875	-0.094605\\
-0.013505	-0.1434325\\
-0.0137775	-0.183105\\
-0.0140075	-0.1342775\\
-0.013825	-0.0885\\
-0.0134575	-0.146485\\
-0.0137325	-0.0976575\\
-0.01355	-0.1251225\\
-0.01355	-0.1586925\\
-0.013825	-0.2319325\\
-0.0142375	-0.250245\\
-0.0145575	-0.1708975\\
-0.01419	-0.2655025\\
-0.0146025	-0.2563475\\
-0.014695	-0.20752\\
-0.01451	-0.1770025\\
-0.0143275	-0.2990725\\
-0.014785	-0.36621\\
-0.0151975	-0.3601075\\
-0.015335	-0.3967275\\
-0.0154275	-0.302125\\
-0.01529	-0.27771\\
-0.0151525	-0.2563475\\
-0.01506	-0.3479\\
-0.01529	-0.2716075\\
-0.0151975	-0.3692625\\
-0.01538	-0.354005\\
-0.0154725	-0.36621\\
-0.0154725	-0.6225575\\
-0.0163425	-0.476075\\
-0.0163425	-0.2563475\\
-0.015565	-0.17395\\
-0.0148775	-0.183105\\
-0.01474	-0.2288825\\
-0.0149225	-0.2990725\\
-0.0151975	-0.3479\\
-0.0154725	-0.3845225\\
-0.015655	-0.3936775\\
-0.0157025	-0.24109\\
-0.01529	-0.216675\\
-0.015015	-0.253295\\
-0.0151525	-0.2563475\\
-0.0151525	-0.1525875\\
-0.014695	-0.1098625\\
-0.0142375	-0.198365\\
-0.01451	-0.201415\\
-0.0146475	-0.2746575\\
-0.0149225	-0.3479\\
-0.015335	-0.3082275\\
-0.015335	-0.2197275\\
-0.015015	-0.1770025\\
-0.014695	-0.268555\\
-0.01497	-0.3479\\
-0.015335	-0.4638675\\
-0.0157925	-0.4974375\\
-0.0160675	-0.579835\\
-0.0163425	-0.460815\\
-0.01625	-0.4211425\\
-0.0161125	-0.2227775\\
-0.01538	-0.1678475\\
-0.0148325	-0.3173825\\
-0.015335	-0.408935\\
-0.0157925	-0.234985\\
-0.015335	-0.3784175\\
-0.01561	-0.5096425\\
-0.01616	-0.616455\\
-0.016525	-0.3845225\\
-0.0161125	-0.216675\\
-0.01538	-0.1251225\\
-0.0146475	-0.18921\\
-0.0146475	-0.17395\\
-0.014695	-0.18921\\
-0.014695	-0.1068125\\
-0.0142375	-0.1525875\\
-0.01419	-0.112915\\
-0.0140525	-0.146485\\
-0.0141	-0.12207\\
-0.0141	-0.15564\\
-0.01419	-0.253295\\
-0.014695	-0.2288825\\
-0.01474	-0.3234875\\
-0.01506	-0.390625\\
-0.0155175	-0.3875725\\
-0.015565	-0.2136225\\
-0.01506	-0.216675\\
-0.0148325	-0.2655025\\
-0.015015	-0.1770025\\
-0.01474	-0.1586925\\
-0.0145575	-0.112915\\
-0.01419	-0.183105\\
-0.014375	-0.17395\\
-0.014465	-0.0976575\\
-0.0140075	-0.0549325\\
-0.0133675	-0.0640875\\
-0.0130475	-0.1159675\\
-0.0134125	-0.1800525\\
-0.013915	-0.18921\\
-0.0141	-0.112915\\
-0.0137775	-0.1434325\\
-0.013825	-0.2105725\\
-0.014145	-0.2746575\\
-0.0146475	-0.19226\\
-0.014465	-0.1800525\\
-0.0142375	-0.2105725\\
-0.01442	-0.253295\\
-0.0146475	-0.2319325\\
-0.0146475	-0.2746575\\
-0.01474	-0.2594\\
-0.01474	-0.1770025\\
-0.014465	-0.14038\\
-0.014145	-0.1800525\\
-0.0143275	-0.149535\\
-0.01419	-0.18921\\
-0.0142825	-0.2563475\\
-0.0146475	-0.3936775\\
-0.01529	-0.2746575\\
-0.0151525	-0.268555\\
-0.015015	-0.41504\\
-0.0154725	-0.31433\\
-0.01538	-0.18921\\
-0.0148775	-0.13733\\
-0.01442	-0.1098625\\
-0.0140075	-0.1159675\\
-0.0139625	-0.0976575\\
-0.0137325	-0.1007075\\
-0.013595	-0.2105725\\
-0.0142825	-0.0671375\\
-0.013595	-0.094605\\
-0.01323	-0.1342775\\
-0.01355	-0.1800525\\
-0.0140075	-0.14038\\
-0.013915	-0.1007075\\
-0.013595	-0.061035\\
-0.0131375	-0.1068125\\
-0.013275	-0.2105725\\
-0.0140075	-0.27771\\
-0.0146025	-0.2197275\\
-0.01451	-0.183105\\
-0.0142825	-0.20752\\
-0.014375	-0.2105725\\
-0.014375	-0.24109\\
-0.01451	-0.29297\\
-0.014785	-0.3479\\
-0.015105	-0.3326425\\
-0.0151975	-0.286865\\
-0.01506	-0.1861575\\
-0.0146475	-0.146485\\
-0.0142825	-0.1525875\\
-0.0142375	-0.2288825\\
-0.01451	-0.1434325\\
-0.01419	-0.1251225\\
-0.013915	-0.2319325\\
-0.01442	-0.2838125\\
-0.0148325	-0.2716075\\
-0.014785	-0.31128\\
-0.015015	-0.4119875\\
-0.0154275	-0.2136225\\
-0.0148775	-0.36316\\
-0.01529	-0.38147\\
-0.01561	-0.3509525\\
-0.0155175	-0.38147\\
-0.015565	-0.372315\\
-0.01561	-0.634765\\
-0.0163425	-0.58899\\
-0.0166625	-0.3967275\\
-0.016205	-0.4730225\\
-0.01625	-0.582885\\
-0.01657	-0.5584725\\
-0.0166175	-0.634765\\
-0.016755	-0.5676275\\
-0.016755	-0.7781975\\
-0.01712	-0.5859375\\
-0.01703	-0.39978\\
-0.016525	-0.2471925\\
-0.0158375	-0.250245\\
-0.0155175	-0.198365\\
-0.015335	-0.1617425\\
-0.0149225	-0.250245\\
-0.0151975	-0.357055\\
-0.015655	-0.2197275\\
-0.01529	-0.19226\\
-0.015015	-0.183105\\
-0.01497	-0.12207\\
-0.0145575	-0.1586925\\
-0.014465	-0.08545\\
-0.01387	-0.1007075\\
-0.013595	-0.0885\\
-0.013505	-0.12207\\
-0.0136875	-0.094605\\
-0.013595	-0.079345\\
-0.01332	-0.042725\\
-0.013	-0.0457775\\
-0.012635	-0.1007075\\
-0.012955	-0.07019\\
-0.0128625	-0.0732425\\
-0.0127725	-0.164795\\
-0.013505	-0.234985\\
-0.014145	-0.234985\\
-0.014375	-0.1708975\\
-0.01419	-0.216675\\
-0.0142825	-0.338745\\
-0.01497	-0.1770025\\
-0.01442	-0.094605\\
-0.0137325	-0.0732425\\
-0.0134125	-0.061035\\
-0.013	-0.1007075\\
-0.0131375	-0.1678475\\
-0.0136425	-0.250245\\
-0.0142825	-0.15564\\
-0.0140525	-0.250245\\
-0.014465	-0.3601075\\
-0.0151525	-0.36316\\
-0.0152425	-0.2594\\
-0.015015	-0.15564\\
-0.014465	-0.20752\\
-0.01451	-0.2838125\\
-0.0148325	-0.372315\\
-0.0152425	-0.20752\\
-0.01474	-0.1068125\\
-0.0139625	-0.1678475\\
-0.014145	-0.1342775\\
-0.0140075	-0.061035\\
-0.013275	-0.0396725\\
-0.012635	-0.1068125\\
-0.013	-0.2319325\\
-0.014145	-0.2471925\\
-0.01451	-0.1770025\\
-0.0142825	-0.1098625\\
-0.0137325	-0.1190175\\
-0.013595	-0.05188\\
-0.012635	-0.07019\\
-0.0125425	-0.0732425\\
-0.012635	-0.1159675\\
-0.012955	-0.17395\\
-0.01355	-0.2594\\
-0.01419	-0.268555\\
-0.0145575	-0.302125\\
-0.014695	-0.31128\\
-0.0148325	-0.302125\\
-0.0148775	-0.2899175\\
-0.0148775	-0.253295\\
-0.014785	-0.3082275\\
-0.01497	-0.46997\\
-0.01561	-0.424195\\
-0.0157925	-0.2197275\\
-0.015105	-0.112915\\
-0.014145	-0.2594\\
-0.014695	-0.305175\\
-0.0151975	-0.286865\\
-0.0151525	-0.1586925\\
-0.01451	-0.1678475\\
-0.0142825	-0.2838125\\
-0.0148325	-0.4394525\\
-0.015565	-0.45166\\
-0.0158375	-0.5096425\\
-0.0160675	-0.4882825\\
-0.01616	-0.3479\\
-0.0157925	-0.3601075\\
-0.0157025	-0.149535\\
-0.014785	-0.1525875\\
-0.0142825	-0.19226\\
-0.01451	-0.2471925\\
-0.014785	-0.26245\\
-0.0149225	-0.27771\\
-0.01506	-0.1770025\\
-0.0146475	-0.2471925\\
-0.0148325	-0.216675\\
-0.014785	-0.1190175\\
-0.0142825	-0.0823975\\
-0.0136875	-0.0671375\\
-0.01323	-0.1068125\\
-0.0134575	-0.0885\\
-0.01332	-0.0488275\\
-0.0128175	-0.1007075\\
-0.0131375	-0.2136225\\
-0.0139625	-0.14038\\
-0.01387	-0.128175\\
-0.0136425	-0.1800525\\
-0.0139625	-0.2990725\\
-0.0146025	-0.2044675\\
-0.01451	-0.1068125\\
-0.0136875	-0.0549325\\
-0.013	-0.1434325\\
-0.013595	-0.3173825\\
-0.014695	-0.4028325\\
-0.015335	-0.5584725\\
-0.0160675	-0.6652825\\
-0.0166625	-0.653075\\
-0.016845	-0.4211425\\
-0.0164325	-0.427245\\
-0.016205	-0.55542\\
-0.0166175	-0.4882825\\
-0.01648	-0.4486075\\
-0.0163875	-0.5065925\\
-0.01648	-0.5523675\\
-0.0166175	-0.5584725\\
-0.0166625	-0.753785\\
-0.01712	-0.5096425\\
-0.016845	-0.2746575\\
-0.0161125	-0.149535\\
-0.01506	-0.1190175\\
-0.014375	-0.13733\\
-0.0142825	-0.1342775\\
-0.0142825	-0.14038\\
-0.0142375	-0.253295\\
-0.0148325	-0.198365\\
-0.014785	-0.2105725\\
-0.01474	-0.27771\\
-0.015015	-0.1586925\\
-0.0146025	-0.1770025\\
-0.01451	-0.253295\\
-0.0148325	-0.2899175\\
-0.015105	-0.1617425\\
-0.014695	-0.1159675\\
-0.014145	-0.146485\\
-0.01419	-0.13733\\
-0.01419	-0.3234875\\
-0.0151525	-0.46692\\
-0.015975	-0.43335\\
-0.0160225	-0.4974375\\
-0.01616	-0.564575\\
-0.01648	-0.74463\\
-0.017075	-0.64392\\
-0.017165	-0.3936775\\
-0.016525	-0.250245\\
-0.0158375	-0.13733\\
-0.0148775	-0.1525875\\
-0.0145575	-0.1678475\\
-0.0146025	-0.216675\\
-0.014785	-0.15564\\
-0.0146475	-0.14038\\
-0.014375	-0.1586925\\
-0.014465	-0.1953125\\
-0.0146025	-0.2227775\\
-0.014785	-0.27771\\
-0.01506	-0.198365\\
-0.0148775	-0.1007075\\
-0.014145	-0.07019\\
-0.01355	-0.0579825\\
-0.0131825	-0.061035\\
-0.013	-0.08545\\
-0.0130925	-0.1434325\\
-0.01355	-0.149535\\
-0.0137775	-0.18921\\
-0.0140075	-0.286865\\
-0.014695	-0.2044675\\
-0.014695	-0.375365\\
-0.0151975	-0.52185\\
-0.0160675	-0.405885\\
-0.0160675	-0.3601075\\
-0.0158375	-0.427245\\
-0.015975	-0.4974375\\
-0.016205	-0.32959\\
-0.015885	-0.27771\\
-0.015565	-0.1800525\\
-0.015015	-0.198365\\
-0.014785	-0.357055\\
-0.0154725	-0.6195075\\
-0.016525	-0.7507325\\
-0.01712	-0.598145\\
-0.01703	-0.5004875\\
-0.0168	-0.36316\\
-0.0164325	-0.3845225\\
-0.0162975	-0.52185\\
-0.0166625	-0.2899175\\
-0.0161125	-0.2471925\\
-0.01561	-0.18921\\
-0.01538	-0.372315\\
-0.0158375	-0.3417975\\
-0.0160675	-0.19226\\
-0.01529	-0.1800525\\
-0.01506	-0.1342775\\
-0.01474	-0.2380375\\
-0.015015	-0.36316\\
-0.015655	-0.372315\\
-0.0157925	-0.2838125\\
-0.015655	-0.20752\\
-0.01529	-0.234985\\
-0.0152425	-0.1861575\\
-0.015105	-0.1342775\\
-0.014695	-0.1098625\\
-0.0143275	-0.0976575\\
-0.0140075	-0.0732425\\
-0.0136875	-0.0885\\
-0.013595	-0.164795\\
-0.0140525	-0.2380375\\
-0.0146475	-0.24109\\
-0.01474	-0.146485\\
-0.0143275	-0.1342775\\
-0.0141	-0.10376\\
-0.013915	-0.13733\\
-0.0140525	-0.1953125\\
-0.014375	-0.216675\\
-0.01451	-0.201415\\
-0.0145575	-0.1159675\\
-0.0140525	-0.24109\\
-0.0146475	-0.357055\\
-0.01529	-0.253295\\
-0.015105	-0.1007075\\
-0.014145	-0.1251225\\
-0.01387	-0.19226\\
-0.0143275	-0.183105\\
-0.014465	-0.1678475\\
-0.0143275	-0.10376\\
-0.0139625	-0.18921\\
-0.0142375	-0.29297\\
-0.0148325	-0.1770025\\
-0.0145575	-0.0640875\\
-0.013595	-0.0976575\\
-0.0134125	-0.2288825\\
-0.0143275	-0.27771\\
-0.0148775	-0.164795\\
-0.01451	-0.149535\\
-0.01419	-0.2838125\\
-0.014785	-0.3784175\\
-0.015335	-0.32959\\
-0.01538	-0.4486075\\
-0.0157475	-0.5493175\\
-0.01625	-0.3509525\\
-0.015975	-0.27771\\
-0.015565	-0.39978\\
-0.0158375	-0.2746575\\
-0.01561	-0.357055\\
-0.015655	-0.4364025\\
-0.0160225	-0.3448475\\
-0.0158375	-0.17395\\
-0.01506	-0.27771\\
-0.0152425	-0.5065925\\
-0.0161125	-0.582885\\
-0.01657	-0.424195\\
-0.0163425	-0.357055\\
-0.0160675	-0.479125\\
-0.0163425	-0.3845225\\
-0.016205	-0.24109\\
-0.01561	-0.2136225\\
-0.01529	-0.268555\\
-0.0154275	-0.2838125\\
-0.0155175	-0.2746575\\
-0.0155175	-0.31433\\
-0.01561	-0.2197275\\
-0.015335	-0.22583\\
-0.0152425	-0.3265375\\
-0.015565	-0.1953125\\
-0.0151975	-0.15564\\
-0.014785	-0.1251225\\
-0.014465	-0.1007075\\
-0.01419	-0.1159675\\
-0.0140525	-0.2136225\\
-0.0145575	-0.198365\\
-0.014695	-0.268555\\
-0.015015	-0.3479\\
-0.01538	-0.4119875\\
-0.0157925	-0.3082275\\
-0.01561	-0.268555\\
-0.01538	-0.10376\\
-0.01442	-0.1678475\\
-0.014375	-0.3234875\\
-0.0151975	-0.2136225\\
-0.01506	-0.10376\\
-0.014375	-0.2197275\\
-0.014695	-0.3448475\\
-0.01538	-0.357055\\
-0.015565	-0.4638675\\
-0.015975	-0.6134025\\
-0.01657	-0.43335\\
-0.0163875	-0.3692625\\
-0.0160675	-0.427245\\
-0.016205	-0.2838125\\
-0.0158375	-0.22583\\
-0.0154725	-0.2563475\\
-0.0154725	-0.32959\\
-0.015655	-0.354005\\
-0.0158375	-0.357055\\
-0.0158375	-0.198365\\
-0.015335	-0.1770025\\
-0.01497	-0.13733\\
-0.01474	-0.198365\\
-0.0148325	-0.18921\\
-0.0148775	-0.149535\\
-0.0146475	-0.2990725\\
-0.0151975	-0.32959\\
-0.0155175	-0.2227775\\
-0.0152425	-0.1861575\\
-0.01497	-0.2716075\\
-0.0151975	-0.1586925\\
-0.0148775	-0.0885\\
-0.01419	-0.1190175\\
-0.0140525	-0.1617425\\
-0.0142825	-0.14038\\
-0.0142825	-0.10376\\
-0.0140075	-0.0579825\\
-0.0134125	-0.1068125\\
-0.0134125	-0.1617425\\
-0.0139625	-0.2563475\\
-0.0146025	-0.29602\\
-0.015015	-0.38147\\
-0.0154275	-0.43335\\
-0.0157925	-0.2288825\\
-0.0151975	-0.17395\\
-0.014695	-0.3417975\\
-0.01529	-0.527955\\
-0.01616	-0.4119875\\
-0.0161125	-0.3784175\\
-0.01593	-0.5767825\\
-0.0164325	-0.4638675\\
-0.0164325	-0.372315\\
-0.01616	-0.2807625\\
-0.0157925	-0.286865\\
-0.015655	-0.3875725\\
-0.01593	-0.268555\\
-0.0157025	-0.22583\\
-0.01538	-0.390625\\
-0.0158375	-0.4913325\\
-0.0162975	-0.512695\\
-0.0164325	-0.2319325\\
-0.01561	-0.1098625\\
-0.0145575	-0.27771\\
-0.0151525	-0.2197275\\
-0.0152425	-0.094605\\
-0.014375	-0.0732425\\
-0.0137325	-0.1342775\\
-0.013915	-0.2044675\\
-0.01442	-0.338745\\
-0.0152425	-0.24109\\
-0.0151525	-0.1770025\\
-0.014785	-0.1068125\\
-0.01419	-0.07019\\
-0.01355	-0.0976575\\
-0.013595	-0.0885\\
-0.013505	-0.112915\\
-0.013595	-0.198365\\
-0.01419	-0.1800525\\
-0.0143275	-0.0823975\\
-0.0136875	-0.076295\\
-0.0134125	-0.10376\\
-0.01355	-0.1342775\\
-0.0137775	-0.0823975\\
-0.0134575	-0.128175\\
-0.0136425	-0.094605\\
-0.013505	-0.0640875\\
-0.0131375	-0.1586925\\
-0.0137325	-0.24109\\
-0.014375	-0.36316\\
-0.0151975	-0.476075\\
-0.0158375	-0.4028325\\
-0.01593	-0.201415\\
-0.0151525	-0.1434325\\
-0.0145575	-0.146485\\
-0.014375	-0.0823975\\
-0.013825	-0.076295\\
-0.0134125	-0.1617425\\
-0.01387	-0.183105\\
-0.01419	-0.1678475\\
-0.014145	-0.183105\\
-0.0142825	-0.2227775\\
-0.01442	-0.234985\\
-0.0145575	-0.24414\\
-0.014695	-0.305175\\
-0.01497	-0.29602\\
-0.01506	-0.427245\\
-0.015565	-0.2227775\\
-0.0151525	-0.10376\\
-0.014145	-0.1007075\\
-0.013825	-0.19226\\
-0.014375	-0.1586925\\
-0.0143275	-0.1434325\\
-0.014145	-0.0671375\\
-0.0133675	-0.112915\\
-0.01332	-0.07019\\
-0.0131825	-0.03357\\
-0.0125875	-0.0640875\\
-0.012635	-0.131225\\
-0.013275	-0.1617425\\
-0.0136875	-0.24109\\
-0.0143275	-0.3417975\\
-0.01497	-0.2471925\\
-0.0148775	-0.1098625\\
-0.0140075	-0.201415\\
-0.01419	-0.22583\\
-0.01451	-0.1068125\\
-0.0139625	-0.1434325\\
-0.013915	-0.2807625\\
-0.0146025	-0.24414\\
-0.014785	-0.405885\\
-0.015335	-0.4730225\\
-0.015975	-0.29297\\
-0.0154725	-0.2319325\\
-0.0151525	-0.201415\\
-0.01497	-0.29297\\
-0.0152425	-0.39978\\
-0.01561	-0.27771\\
-0.0154275	-0.1800525\\
-0.0148325	-0.20752\\
};
\addplot [color=mycolor2, line width=2.0pt, forget plot]
  table[row sep=crcr]{%
-0.01561	-0.01561\\
-0.0157025	-0.0157025\\
-0.0158375	-0.0158375\\
-0.015655	-0.015655\\
-0.0149225	-0.0149225\\
-0.014785	-0.014785\\
-0.0148325	-0.0148325\\
-0.01497	-0.01497\\
-0.0148775	-0.0148775\\
-0.0140075	-0.0140075\\
-0.0131825	-0.0131825\\
-0.0128175	-0.0128175\\
-0.0137775	-0.0137775\\
-0.014785	-0.014785\\
-0.015015	-0.015015\\
-0.0151975	-0.0151975\\
-0.0148325	-0.0148325\\
-0.0142825	-0.0142825\\
-0.0149225	-0.0149225\\
-0.014785	-0.014785\\
-0.014375	-0.014375\\
-0.014785	-0.014785\\
-0.0148325	-0.0148325\\
-0.014695	-0.014695\\
-0.015105	-0.015105\\
-0.01529	-0.01529\\
-0.01497	-0.01497\\
-0.01529	-0.01529\\
-0.0154275	-0.0154275\\
-0.0151975	-0.0151975\\
-0.01497	-0.01497\\
-0.0146475	-0.0146475\\
-0.014695	-0.014695\\
-0.0149225	-0.0149225\\
-0.0148775	-0.0148775\\
-0.0149225	-0.0149225\\
-0.01497	-0.01497\\
-0.01506	-0.01506\\
-0.0155175	-0.0155175\\
-0.015565	-0.015565\\
-0.0151975	-0.0151975\\
-0.01497	-0.01497\\
-0.014375	-0.014375\\
-0.014145	-0.014145\\
-0.014375	-0.014375\\
-0.0142825	-0.0142825\\
-0.014145	-0.014145\\
-0.014465	-0.014465\\
-0.014695	-0.014695\\
-0.0151975	-0.0151975\\
-0.015105	-0.015105\\
-0.01451	-0.01451\\
-0.014145	-0.014145\\
-0.0142375	-0.0142375\\
-0.014465	-0.014465\\
-0.0140075	-0.0140075\\
-0.0137325	-0.0137325\\
-0.0140075	-0.0140075\\
-0.013915	-0.013915\\
-0.0136875	-0.0136875\\
-0.013825	-0.013825\\
-0.0140075	-0.0140075\\
-0.0145575	-0.0145575\\
-0.014695	-0.014695\\
-0.0149225	-0.0149225\\
-0.01497	-0.01497\\
-0.015105	-0.015105\\
-0.015015	-0.015015\\
-0.0148775	-0.0148775\\
-0.0148325	-0.0148325\\
-0.01474	-0.01474\\
-0.014695	-0.014695\\
-0.01529	-0.01529\\
-0.0160225	-0.0160225\\
-0.0162975	-0.0162975\\
-0.0158375	-0.0158375\\
-0.0160675	-0.0160675\\
-0.016525	-0.016525\\
-0.0166625	-0.0166625\\
-0.01648	-0.01648\\
-0.0167075	-0.0167075\\
-0.0172575	-0.0172575\\
-0.01703	-0.01703\\
-0.0167075	-0.0167075\\
-0.01616	-0.01616\\
-0.015655	-0.015655\\
-0.015335	-0.015335\\
-0.0154275	-0.0154275\\
-0.0151525	-0.0151525\\
-0.0146025	-0.0146025\\
-0.0142825	-0.0142825\\
-0.0143275	-0.0143275\\
-0.014695	-0.014695\\
-0.014785	-0.014785\\
-0.01474	-0.01474\\
-0.01506	-0.01506\\
-0.0151525	-0.0151525\\
-0.015655	-0.015655\\
-0.015565	-0.015565\\
-0.0157475	-0.0157475\\
-0.015565	-0.015565\\
-0.015335	-0.015335\\
-0.0154275	-0.0154275\\
-0.0151975	-0.0151975\\
-0.01497	-0.01497\\
-0.01506	-0.01506\\
-0.015105	-0.015105\\
-0.014785	-0.014785\\
-0.014695	-0.014695\\
-0.014465	-0.014465\\
-0.014375	-0.014375\\
-0.01442	-0.01442\\
-0.01419	-0.01419\\
-0.014145	-0.014145\\
-0.014375	-0.014375\\
-0.0149225	-0.0149225\\
-0.015105	-0.015105\\
-0.0152425	-0.0152425\\
-0.014785	-0.014785\\
-0.01497	-0.01497\\
-0.01561	-0.01561\\
-0.0155175	-0.0155175\\
-0.015655	-0.015655\\
-0.0157025	-0.0157025\\
-0.0151525	-0.0151525\\
-0.0146475	-0.0146475\\
-0.014695	-0.014695\\
-0.0149225	-0.0149225\\
-0.0155175	-0.0155175\\
-0.0160225	-0.0160225\\
-0.0161125	-0.0161125\\
-0.0158375	-0.0158375\\
-0.0157475	-0.0157475\\
-0.0157025	-0.0157025\\
-0.01561	-0.01561\\
-0.015655	-0.015655\\
-0.0152425	-0.0152425\\
-0.0149225	-0.0149225\\
-0.014785	-0.014785\\
-0.0146025	-0.0146025\\
-0.0146475	-0.0146475\\
-0.01497	-0.01497\\
-0.01561	-0.01561\\
-0.0155175	-0.0155175\\
-0.015105	-0.015105\\
-0.0148325	-0.0148325\\
-0.014695	-0.014695\\
-0.0142375	-0.0142375\\
-0.0136875	-0.0136875\\
-0.013595	-0.013595\\
-0.014145	-0.014145\\
-0.0140525	-0.0140525\\
-0.0140075	-0.0140075\\
-0.014145	-0.014145\\
-0.0141	-0.0141\\
-0.013825	-0.013825\\
-0.0134575	-0.0134575\\
-0.0130925	-0.0130925\\
-0.013275	-0.013275\\
-0.0142375	-0.0142375\\
-0.01497	-0.01497\\
-0.015335	-0.015335\\
-0.01506	-0.01506\\
-0.0145575	-0.0145575\\
-0.014145	-0.014145\\
-0.013595	-0.013595\\
-0.01419	-0.01419\\
-0.01442	-0.01442\\
-0.014785	-0.014785\\
-0.0151975	-0.0151975\\
-0.0160225	-0.0160225\\
-0.0166175	-0.0166175\\
-0.016525	-0.016525\\
-0.0163875	-0.0163875\\
-0.015975	-0.015975\\
-0.01593	-0.01593\\
-0.0160225	-0.0160225\\
-0.0161125	-0.0161125\\
-0.015885	-0.015885\\
-0.0157025	-0.0157025\\
-0.01561	-0.01561\\
-0.0155175	-0.0155175\\
-0.015565	-0.015565\\
-0.015335	-0.015335\\
-0.015655	-0.015655\\
-0.0161125	-0.0161125\\
-0.0157925	-0.0157925\\
-0.0152425	-0.0152425\\
-0.01497	-0.01497\\
-0.01561	-0.01561\\
-0.015975	-0.015975\\
-0.0163875	-0.0163875\\
-0.01657	-0.01657\\
-0.0166625	-0.0166625\\
-0.0164325	-0.0164325\\
-0.016205	-0.016205\\
-0.0163875	-0.0163875\\
-0.016525	-0.016525\\
-0.0160675	-0.0160675\\
-0.0154275	-0.0154275\\
-0.0155175	-0.0155175\\
-0.01497	-0.01497\\
-0.015015	-0.015015\\
-0.0151525	-0.0151525\\
-0.01497	-0.01497\\
-0.014695	-0.014695\\
-0.014785	-0.014785\\
-0.014695	-0.014695\\
-0.0148775	-0.0148775\\
-0.015335	-0.015335\\
-0.0154275	-0.0154275\\
-0.01497	-0.01497\\
-0.0149225	-0.0149225\\
-0.0152425	-0.0152425\\
-0.0161125	-0.0161125\\
-0.0160225	-0.0160225\\
-0.0154275	-0.0154275\\
-0.01497	-0.01497\\
-0.0149225	-0.0149225\\
-0.0148775	-0.0148775\\
-0.0146025	-0.0146025\\
-0.014465	-0.014465\\
-0.0146475	-0.0146475\\
-0.0149225	-0.0149225\\
-0.015015	-0.015015\\
-0.01474	-0.01474\\
-0.0151975	-0.0151975\\
-0.0155175	-0.0155175\\
-0.0154725	-0.0154725\\
-0.01529	-0.01529\\
-0.015335	-0.015335\\
-0.015655	-0.015655\\
-0.01529	-0.01529\\
-0.0142825	-0.0142825\\
-0.01419	-0.01419\\
-0.014375	-0.014375\\
-0.014695	-0.014695\\
-0.01442	-0.01442\\
-0.01474	-0.01474\\
-0.014785	-0.014785\\
-0.01451	-0.01451\\
-0.0146025	-0.0146025\\
-0.014465	-0.014465\\
-0.0145575	-0.0145575\\
-0.01419	-0.01419\\
-0.0140525	-0.0140525\\
-0.013915	-0.013915\\
-0.013825	-0.013825\\
-0.01442	-0.01442\\
-0.014695	-0.014695\\
-0.0151525	-0.0151525\\
-0.015655	-0.015655\\
-0.01538	-0.01538\\
-0.0146475	-0.0146475\\
-0.0142375	-0.0142375\\
-0.0141	-0.0141\\
-0.014375	-0.014375\\
-0.0140525	-0.0140525\\
-0.0139625	-0.0139625\\
-0.01419	-0.01419\\
-0.0148775	-0.0148775\\
-0.01497	-0.01497\\
-0.01529	-0.01529\\
-0.015105	-0.015105\\
-0.0148325	-0.0148325\\
-0.0142375	-0.0142375\\
-0.013915	-0.013915\\
-0.013505	-0.013505\\
-0.0133675	-0.0133675\\
-0.013505	-0.013505\\
-0.0140525	-0.0140525\\
-0.0142375	-0.0142375\\
-0.014465	-0.014465\\
-0.0146025	-0.0146025\\
-0.0151525	-0.0151525\\
-0.01529	-0.01529\\
-0.015015	-0.015015\\
-0.015105	-0.015105\\
-0.0151975	-0.0151975\\
-0.0155175	-0.0155175\\
-0.0154725	-0.0154725\\
-0.015015	-0.015015\\
-0.01419	-0.01419\\
-0.013595	-0.013595\\
-0.0134125	-0.0134125\\
-0.0134575	-0.0134575\\
-0.013505	-0.013505\\
-0.0134575	-0.0134575\\
-0.0136425	-0.0136425\\
-0.01419	-0.01419\\
-0.0142825	-0.0142825\\
-0.01451	-0.01451\\
-0.0141	-0.0141\\
-0.0133675	-0.0133675\\
-0.0136875	-0.0136875\\
-0.0143275	-0.0143275\\
-0.014465	-0.014465\\
-0.0146475	-0.0146475\\
-0.0146025	-0.0146025\\
-0.0142825	-0.0142825\\
-0.01451	-0.01451\\
-0.015015	-0.015015\\
-0.0151975	-0.0151975\\
-0.0148775	-0.0148775\\
-0.015335	-0.015335\\
-0.0151975	-0.0151975\\
-0.015335	-0.015335\\
-0.0151525	-0.0151525\\
-0.0149225	-0.0149225\\
-0.0154275	-0.0154275\\
-0.0160225	-0.0160225\\
-0.01593	-0.01593\\
-0.01529	-0.01529\\
-0.015015	-0.015015\\
-0.0151975	-0.0151975\\
-0.0154275	-0.0154275\\
-0.0151975	-0.0151975\\
-0.0151525	-0.0151525\\
-0.0154275	-0.0154275\\
-0.015565	-0.015565\\
-0.0152425	-0.0152425\\
-0.0148775	-0.0148775\\
-0.015105	-0.015105\\
-0.0157025	-0.0157025\\
-0.0158375	-0.0158375\\
-0.0157925	-0.0157925\\
-0.01529	-0.01529\\
-0.0149225	-0.0149225\\
-0.01497	-0.01497\\
-0.01442	-0.01442\\
-0.013825	-0.013825\\
-0.01332	-0.01332\\
-0.013	-0.013\\
-0.013595	-0.013595\\
-0.0142825	-0.0142825\\
-0.0145575	-0.0145575\\
-0.014375	-0.014375\\
-0.01442	-0.01442\\
-0.0146025	-0.0146025\\
-0.0142375	-0.0142375\\
-0.0140525	-0.0140525\\
-0.0140075	-0.0140075\\
-0.0137775	-0.0137775\\
-0.01387	-0.01387\\
-0.014375	-0.014375\\
-0.0148325	-0.0148325\\
-0.0145575	-0.0145575\\
-0.0141	-0.0141\\
-0.0139625	-0.0139625\\
-0.0136875	-0.0136875\\
-0.0142825	-0.0142825\\
-0.0152425	-0.0152425\\
-0.0151975	-0.0151975\\
-0.0154725	-0.0154725\\
-0.0158375	-0.0158375\\
-0.0160225	-0.0160225\\
-0.0157025	-0.0157025\\
-0.015335	-0.015335\\
-0.0152425	-0.0152425\\
-0.015335	-0.015335\\
-0.0155175	-0.0155175\\
-0.01561	-0.01561\\
-0.01593	-0.01593\\
-0.01625	-0.01625\\
-0.0166175	-0.0166175\\
-0.0163875	-0.0163875\\
-0.01625	-0.01625\\
-0.01648	-0.01648\\
-0.016205	-0.016205\\
-0.0162975	-0.0162975\\
-0.016205	-0.016205\\
-0.0155175	-0.0155175\\
-0.014785	-0.014785\\
-0.0145575	-0.0145575\\
-0.0148775	-0.0148775\\
-0.014785	-0.014785\\
-0.014375	-0.014375\\
-0.01442	-0.01442\\
-0.0143275	-0.0143275\\
-0.01387	-0.01387\\
-0.013915	-0.013915\\
-0.0140075	-0.0140075\\
-0.0137775	-0.0137775\\
-0.0142375	-0.0142375\\
-0.014695	-0.014695\\
-0.0145575	-0.0145575\\
-0.01506	-0.01506\\
-0.0157925	-0.0157925\\
-0.01593	-0.01593\\
-0.01616	-0.01616\\
-0.01593	-0.01593\\
-0.015885	-0.015885\\
-0.01561	-0.01561\\
-0.0148775	-0.0148775\\
-0.014695	-0.014695\\
-0.014375	-0.014375\\
-0.014465	-0.014465\\
-0.0140075	-0.0140075\\
-0.0134125	-0.0134125\\
-0.01332	-0.01332\\
-0.014145	-0.014145\\
-0.014695	-0.014695\\
-0.015015	-0.015015\\
-0.015105	-0.015105\\
-0.01506	-0.01506\\
-0.0152425	-0.0152425\\
-0.015105	-0.015105\\
-0.0148775	-0.0148775\\
-0.014695	-0.014695\\
-0.01497	-0.01497\\
-0.0148325	-0.0148325\\
-0.014465	-0.014465\\
-0.014785	-0.014785\\
-0.0151975	-0.0151975\\
-0.015105	-0.015105\\
-0.01474	-0.01474\\
-0.0145575	-0.0145575\\
-0.01451	-0.01451\\
-0.0142825	-0.0142825\\
-0.0140525	-0.0140525\\
-0.0142375	-0.0142375\\
-0.01451	-0.01451\\
-0.014695	-0.014695\\
-0.0145575	-0.0145575\\
-0.014695	-0.014695\\
-0.01474	-0.01474\\
-0.0142375	-0.0142375\\
-0.0141	-0.0141\\
-0.01474	-0.01474\\
-0.015335	-0.015335\\
-0.0154725	-0.0154725\\
-0.015335	-0.015335\\
-0.01529	-0.01529\\
-0.01506	-0.01506\\
-0.0148775	-0.0148775\\
-0.014695	-0.014695\\
-0.0148325	-0.0148325\\
-0.0148775	-0.0148775\\
-0.01442	-0.01442\\
-0.0145575	-0.0145575\\
-0.0148325	-0.0148325\\
-0.01497	-0.01497\\
-0.0151975	-0.0151975\\
-0.01561	-0.01561\\
-0.0160675	-0.0160675\\
-0.0164325	-0.0164325\\
-0.0161125	-0.0161125\\
-0.0152425	-0.0152425\\
-0.0146475	-0.0146475\\
-0.0140075	-0.0140075\\
-0.0140525	-0.0140525\\
-0.01451	-0.01451\\
-0.014695	-0.014695\\
-0.0145575	-0.0145575\\
-0.014785	-0.014785\\
-0.015015	-0.015015\\
-0.0152425	-0.0152425\\
-0.0154275	-0.0154275\\
-0.015975	-0.015975\\
-0.0161125	-0.0161125\\
-0.0157925	-0.0157925\\
-0.01529	-0.01529\\
-0.015105	-0.015105\\
-0.01529	-0.01529\\
-0.01506	-0.01506\\
-0.01497	-0.01497\\
-0.0149225	-0.0149225\\
-0.01442	-0.01442\\
-0.0146025	-0.0146025\\
-0.015105	-0.015105\\
-0.01497	-0.01497\\
-0.0148775	-0.0148775\\
-0.01561	-0.01561\\
-0.015565	-0.015565\\
-0.01538	-0.01538\\
-0.0157475	-0.0157475\\
-0.0158375	-0.0158375\\
-0.0154275	-0.0154275\\
-0.014785	-0.014785\\
-0.0146025	-0.0146025\\
-0.015105	-0.015105\\
-0.0158375	-0.0158375\\
-0.01616	-0.01616\\
-0.016205	-0.016205\\
-0.01593	-0.01593\\
-0.0154275	-0.0154275\\
-0.01506	-0.01506\\
-0.014695	-0.014695\\
-0.01442	-0.01442\\
-0.0142825	-0.0142825\\
-0.01442	-0.01442\\
-0.0146025	-0.0146025\\
-0.01419	-0.01419\\
-0.014375	-0.014375\\
-0.01474	-0.01474\\
-0.01497	-0.01497\\
-0.015335	-0.015335\\
-0.0157925	-0.0157925\\
-0.01625	-0.01625\\
-0.0160675	-0.0160675\\
-0.0157925	-0.0157925\\
-0.015655	-0.015655\\
-0.0151975	-0.0151975\\
-0.0152425	-0.0152425\\
-0.0157025	-0.0157025\\
-0.0160675	-0.0160675\\
-0.01561	-0.01561\\
-0.0148325	-0.0148325\\
-0.0142375	-0.0142375\\
-0.01332	-0.01332\\
-0.0130925	-0.0130925\\
-0.0134575	-0.0134575\\
-0.0137325	-0.0137325\\
-0.0141	-0.0141\\
-0.013825	-0.013825\\
-0.0137775	-0.0137775\\
-0.0137325	-0.0137325\\
-0.0136425	-0.0136425\\
-0.0136875	-0.0136875\\
-0.014145	-0.014145\\
-0.01474	-0.01474\\
-0.01506	-0.01506\\
-0.015105	-0.015105\\
-0.015335	-0.015335\\
-0.015655	-0.015655\\
-0.0158375	-0.0158375\\
-0.01616	-0.01616\\
-0.0162975	-0.0162975\\
-0.01593	-0.01593\\
-0.0154275	-0.0154275\\
-0.0151975	-0.0151975\\
-0.0157475	-0.0157475\\
-0.0160675	-0.0160675\\
-0.0157925	-0.0157925\\
-0.0152425	-0.0152425\\
-0.014375	-0.014375\\
-0.0136875	-0.0136875\\
-0.013505	-0.013505\\
-0.013	-0.013\\
-0.01332	-0.01332\\
-0.01387	-0.01387\\
-0.0139625	-0.0139625\\
-0.013915	-0.013915\\
-0.0134575	-0.0134575\\
-0.0131375	-0.0131375\\
-0.01323	-0.01323\\
-0.0133675	-0.0133675\\
-0.0134125	-0.0134125\\
-0.01323	-0.01323\\
-0.013	-0.013\\
-0.0134125	-0.0134125\\
-0.0146025	-0.0146025\\
-0.01529	-0.01529\\
-0.015565	-0.015565\\
-0.01538	-0.01538\\
-0.01561	-0.01561\\
-0.0162975	-0.0162975\\
-0.0164325	-0.0164325\\
-0.016205	-0.016205\\
-0.0160675	-0.0160675\\
-0.0161125	-0.0161125\\
-0.0157025	-0.0157025\\
-0.0154725	-0.0154725\\
-0.0149225	-0.0149225\\
-0.0145575	-0.0145575\\
-0.01506	-0.01506\\
-0.0148775	-0.0148775\\
-0.01474	-0.01474\\
-0.0149225	-0.0149225\\
-0.0148775	-0.0148775\\
-0.0148325	-0.0148325\\
-0.0148775	-0.0148775\\
-0.015015	-0.015015\\
-0.0151525	-0.0151525\\
-0.015105	-0.015105\\
-0.0148325	-0.0148325\\
-0.0148775	-0.0148775\\
-0.014375	-0.014375\\
-0.01332	-0.01332\\
-0.0131825	-0.0131825\\
-0.0136425	-0.0136425\\
-0.01323	-0.01323\\
-0.0124975	-0.0124975\\
-0.0125875	-0.0125875\\
-0.01323	-0.01323\\
-0.013595	-0.013595\\
-0.0140075	-0.0140075\\
-0.0145575	-0.0145575\\
-0.014695	-0.014695\\
-0.0142825	-0.0142825\\
-0.0146025	-0.0146025\\
-0.014145	-0.014145\\
-0.0136875	-0.0136875\\
-0.01332	-0.01332\\
-0.0128175	-0.0128175\\
-0.013	-0.013\\
-0.01291	-0.01291\\
-0.013275	-0.013275\\
-0.0139625	-0.0139625\\
-0.014145	-0.014145\\
-0.013915	-0.013915\\
-0.01387	-0.01387\\
-0.0136875	-0.0136875\\
-0.01387	-0.01387\\
-0.0136875	-0.0136875\\
-0.013505	-0.013505\\
-0.0134575	-0.0134575\\
-0.01323	-0.01323\\
-0.01332	-0.01332\\
-0.0137775	-0.0137775\\
-0.013915	-0.013915\\
-0.0136875	-0.0136875\\
-0.01355	-0.01355\\
-0.0133675	-0.0133675\\
-0.0134125	-0.0134125\\
-0.01419	-0.01419\\
-0.0149225	-0.0149225\\
-0.014695	-0.014695\\
-0.0145575	-0.0145575\\
-0.01419	-0.01419\\
-0.0145575	-0.0145575\\
-0.01474	-0.01474\\
-0.0142825	-0.0142825\\
-0.0143275	-0.0143275\\
-0.014145	-0.014145\\
-0.0143275	-0.0143275\\
-0.0140075	-0.0140075\\
-0.0137325	-0.0137325\\
-0.0139625	-0.0139625\\
-0.01419	-0.01419\\
-0.0140075	-0.0140075\\
-0.013915	-0.013915\\
-0.0140525	-0.0140525\\
-0.01451	-0.01451\\
-0.0146025	-0.0146025\\
-0.015015	-0.015015\\
-0.015655	-0.015655\\
-0.015565	-0.015565\\
-0.01506	-0.01506\\
-0.0146025	-0.0146025\\
-0.0148775	-0.0148775\\
-0.0152425	-0.0152425\\
-0.015105	-0.015105\\
-0.0151975	-0.0151975\\
-0.0148775	-0.0148775\\
-0.0140075	-0.0140075\\
-0.0137325	-0.0137325\\
-0.01442	-0.01442\\
-0.0146025	-0.0146025\\
-0.01442	-0.01442\\
-0.0148775	-0.0148775\\
-0.01497	-0.01497\\
-0.0148775	-0.0148775\\
-0.014465	-0.014465\\
-0.0148775	-0.0148775\\
-0.014785	-0.014785\\
-0.01474	-0.01474\\
-0.014695	-0.014695\\
-0.014375	-0.014375\\
-0.014145	-0.014145\\
-0.0146025	-0.0146025\\
-0.0149225	-0.0149225\\
-0.015015	-0.015015\\
-0.014695	-0.014695\\
-0.0142825	-0.0142825\\
-0.0142375	-0.0142375\\
-0.01442	-0.01442\\
-0.014695	-0.014695\\
-0.015015	-0.015015\\
-0.01529	-0.01529\\
-0.015105	-0.015105\\
-0.0146025	-0.0146025\\
-0.01497	-0.01497\\
-0.015655	-0.015655\\
-0.0154275	-0.0154275\\
-0.0152425	-0.0152425\\
-0.0154275	-0.0154275\\
-0.01529	-0.01529\\
-0.01497	-0.01497\\
-0.015015	-0.015015\\
-0.0149225	-0.0149225\\
-0.01497	-0.01497\\
-0.0152425	-0.0152425\\
-0.01506	-0.01506\\
-0.015105	-0.015105\\
-0.01506	-0.01506\\
-0.0149225	-0.0149225\\
-0.01506	-0.01506\\
-0.0148325	-0.0148325\\
-0.01474	-0.01474\\
-0.0148775	-0.0148775\\
-0.0148325	-0.0148325\\
-0.014145	-0.014145\\
-0.0134575	-0.0134575\\
-0.0128625	-0.0128625\\
-0.0130475	-0.0130475\\
-0.0141	-0.0141\\
-0.0148325	-0.0148325\\
-0.01497	-0.01497\\
-0.0146475	-0.0146475\\
-0.0146025	-0.0146025\\
-0.014465	-0.014465\\
-0.0140525	-0.0140525\\
-0.0141	-0.0141\\
-0.0140525	-0.0140525\\
-0.01419	-0.01419\\
-0.014145	-0.014145\\
-0.0140525	-0.0140525\\
-0.0136875	-0.0136875\\
-0.0137775	-0.0137775\\
-0.01451	-0.01451\\
-0.014375	-0.014375\\
-0.014465	-0.014465\\
-0.01474	-0.01474\\
-0.0148775	-0.0148775\\
-0.0151525	-0.0151525\\
-0.01506	-0.01506\\
-0.015015	-0.015015\\
-0.015105	-0.015105\\
-0.01451	-0.01451\\
-0.01474	-0.01474\\
-0.0146475	-0.0146475\\
-0.01474	-0.01474\\
-0.01497	-0.01497\\
-0.0148325	-0.0148325\\
-0.015015	-0.015015\\
-0.015335	-0.015335\\
-0.015105	-0.015105\\
-0.0149225	-0.0149225\\
-0.01497	-0.01497\\
-0.0148325	-0.0148325\\
-0.0146475	-0.0146475\\
-0.01451	-0.01451\\
-0.0145575	-0.0145575\\
-0.01451	-0.01451\\
-0.01506	-0.01506\\
-0.015655	-0.015655\\
-0.01561	-0.01561\\
-0.015565	-0.015565\\
-0.01497	-0.01497\\
-0.0146025	-0.0146025\\
-0.01442	-0.01442\\
-0.01419	-0.01419\\
-0.014145	-0.014145\\
-0.0145575	-0.0145575\\
-0.0149225	-0.0149225\\
-0.0151525	-0.0151525\\
-0.015105	-0.015105\\
-0.014695	-0.014695\\
-0.015105	-0.015105\\
-0.015565	-0.015565\\
-0.0157475	-0.0157475\\
-0.015655	-0.015655\\
-0.01616	-0.01616\\
-0.0160225	-0.0160225\\
-0.01529	-0.01529\\
-0.01474	-0.01474\\
-0.01442	-0.01442\\
-0.014785	-0.014785\\
-0.0152425	-0.0152425\\
-0.015655	-0.015655\\
-0.0155175	-0.0155175\\
-0.0151975	-0.0151975\\
-0.0146025	-0.0146025\\
-0.0142375	-0.0142375\\
-0.0143275	-0.0143275\\
-0.01442	-0.01442\\
-0.0141	-0.0141\\
-0.0140525	-0.0140525\\
-0.0140075	-0.0140075\\
-0.0134575	-0.0134575\\
-0.01387	-0.01387\\
-0.014145	-0.014145\\
-0.01451	-0.01451\\
-0.0145575	-0.0145575\\
-0.014785	-0.014785\\
-0.0146025	-0.0146025\\
-0.014785	-0.014785\\
-0.0151525	-0.0151525\\
-0.01497	-0.01497\\
-0.0149225	-0.0149225\\
-0.0148325	-0.0148325\\
-0.0149225	-0.0149225\\
-0.01529	-0.01529\\
-0.0151975	-0.0151975\\
-0.01497	-0.01497\\
-0.0148775	-0.0148775\\
-0.0145575	-0.0145575\\
-0.014785	-0.014785\\
-0.0152425	-0.0152425\\
-0.015335	-0.015335\\
-0.0154275	-0.0154275\\
-0.0161125	-0.0161125\\
-0.0157925	-0.0157925\\
-0.0154725	-0.0154725\\
-0.0151525	-0.0151525\\
-0.0152425	-0.0152425\\
-0.0157475	-0.0157475\\
-0.0154725	-0.0154725\\
-0.0151975	-0.0151975\\
-0.0155175	-0.0155175\\
-0.0151975	-0.0151975\\
-0.0145575	-0.0145575\\
-0.014465	-0.014465\\
-0.014375	-0.014375\\
-0.0140525	-0.0140525\\
-0.013915	-0.013915\\
-0.013825	-0.013825\\
-0.0136875	-0.0136875\\
-0.01387	-0.01387\\
-0.014145	-0.014145\\
-0.014465	-0.014465\\
-0.0148325	-0.0148325\\
-0.0149225	-0.0149225\\
-0.015105	-0.015105\\
-0.0154275	-0.0154275\\
-0.015105	-0.015105\\
-0.015015	-0.015015\\
-0.01497	-0.01497\\
-0.015335	-0.015335\\
-0.0152425	-0.0152425\\
-0.0151975	-0.0151975\\
-0.01506	-0.01506\\
-0.0154725	-0.0154725\\
-0.0154275	-0.0154275\\
-0.01538	-0.01538\\
-0.015335	-0.015335\\
-0.0148325	-0.0148325\\
-0.0142375	-0.0142375\\
-0.013825	-0.013825\\
-0.01419	-0.01419\\
-0.0142375	-0.0142375\\
-0.01442	-0.01442\\
-0.0143275	-0.0143275\\
-0.0146025	-0.0146025\\
-0.0151975	-0.0151975\\
-0.0157925	-0.0157925\\
-0.0157025	-0.0157025\\
-0.015655	-0.015655\\
-0.01561	-0.01561\\
-0.0151975	-0.0151975\\
-0.0151525	-0.0151525\\
-0.0151975	-0.0151975\\
-0.015105	-0.015105\\
-0.0151975	-0.0151975\\
-0.0154725	-0.0154725\\
-0.01593	-0.01593\\
-0.0160225	-0.0160225\\
-0.015975	-0.015975\\
-0.0160675	-0.0160675\\
-0.0157475	-0.0157475\\
-0.015655	-0.015655\\
-0.0154725	-0.0154725\\
-0.01538	-0.01538\\
-0.015655	-0.015655\\
-0.0158375	-0.0158375\\
-0.0151525	-0.0151525\\
-0.0148325	-0.0148325\\
-0.01474	-0.01474\\
-0.0148775	-0.0148775\\
-0.01506	-0.01506\\
-0.0146025	-0.0146025\\
-0.0146475	-0.0146475\\
-0.01529	-0.01529\\
-0.016205	-0.016205\\
-0.01648	-0.01648\\
-0.0164325	-0.0164325\\
-0.016205	-0.016205\\
-0.01625	-0.01625\\
-0.016525	-0.016525\\
-0.0162975	-0.0162975\\
-0.01625	-0.01625\\
-0.015885	-0.015885\\
-0.01561	-0.01561\\
-0.0157475	-0.0157475\\
-0.0154725	-0.0154725\\
-0.01497	-0.01497\\
-0.015105	-0.015105\\
-0.01497	-0.01497\\
-0.0149225	-0.0149225\\
-0.01497	-0.01497\\
-0.01506	-0.01506\\
-0.01538	-0.01538\\
-0.01529	-0.01529\\
-0.0149225	-0.0149225\\
-0.01497	-0.01497\\
-0.0151525	-0.0151525\\
-0.015335	-0.015335\\
-0.01529	-0.01529\\
-0.015015	-0.015015\\
-0.01538	-0.01538\\
-0.0154725	-0.0154725\\
-0.0151525	-0.0151525\\
-0.0155175	-0.0155175\\
-0.01561	-0.01561\\
-0.01529	-0.01529\\
-0.0148325	-0.0148325\\
-0.01442	-0.01442\\
-0.01387	-0.01387\\
-0.0143275	-0.0143275\\
-0.01451	-0.01451\\
-0.01419	-0.01419\\
-0.013915	-0.013915\\
-0.013825	-0.013825\\
-0.0143275	-0.0143275\\
-0.0146025	-0.0146025\\
-0.01497	-0.01497\\
-0.01529	-0.01529\\
-0.01538	-0.01538\\
-0.0154275	-0.0154275\\
-0.01506	-0.01506\\
-0.0141	-0.0141\\
-0.013915	-0.013915\\
-0.0139625	-0.0139625\\
-0.0140525	-0.0140525\\
-0.0146025	-0.0146025\\
-0.0148775	-0.0148775\\
-0.0154725	-0.0154725\\
-0.0152425	-0.0152425\\
-0.015015	-0.015015\\
-0.0155175	-0.0155175\\
-0.01529	-0.01529\\
-0.0148775	-0.0148775\\
-0.01442	-0.01442\\
-0.014465	-0.014465\\
-0.0148325	-0.0148325\\
-0.01538	-0.01538\\
-0.0151525	-0.0151525\\
-0.0148325	-0.0148325\\
-0.0149225	-0.0149225\\
-0.0152425	-0.0152425\\
-0.01593	-0.01593\\
-0.0160675	-0.0160675\\
-0.015975	-0.015975\\
-0.015565	-0.015565\\
-0.01497	-0.01497\\
-0.0149225	-0.0149225\\
-0.014145	-0.014145\\
-0.01419	-0.01419\\
-0.0149225	-0.0149225\\
-0.0154275	-0.0154275\\
-0.0157475	-0.0157475\\
-0.0162975	-0.0162975\\
-0.0163425	-0.0163425\\
-0.01561	-0.01561\\
-0.0152425	-0.0152425\\
-0.01506	-0.01506\\
-0.0151525	-0.0151525\\
-0.0155175	-0.0155175\\
-0.015975	-0.015975\\
-0.0157925	-0.0157925\\
-0.015335	-0.015335\\
-0.015655	-0.015655\\
-0.01538	-0.01538\\
-0.01497	-0.01497\\
-0.0146025	-0.0146025\\
-0.0141	-0.0141\\
-0.01387	-0.01387\\
-0.0134575	-0.0134575\\
-0.0137325	-0.0137325\\
-0.0142375	-0.0142375\\
-0.01442	-0.01442\\
-0.0142825	-0.0142825\\
-0.014145	-0.014145\\
-0.0134575	-0.0134575\\
-0.012635	-0.012635\\
-0.0124975	-0.0124975\\
-0.0130925	-0.0130925\\
-0.0137325	-0.0137325\\
-0.0141	-0.0141\\
-0.01442	-0.01442\\
-0.01474	-0.01474\\
-0.0151975	-0.0151975\\
-0.01593	-0.01593\\
-0.01625	-0.01625\\
-0.0163425	-0.0163425\\
-0.015655	-0.015655\\
-0.01529	-0.01529\\
-0.0154725	-0.0154725\\
-0.0154275	-0.0154275\\
-0.015105	-0.015105\\
-0.014695	-0.014695\\
-0.014375	-0.014375\\
-0.01474	-0.01474\\
-0.0148325	-0.0148325\\
-0.0142825	-0.0142825\\
-0.0137775	-0.0137775\\
-0.0139625	-0.0139625\\
-0.0143275	-0.0143275\\
-0.0142375	-0.0142375\\
-0.01442	-0.01442\\
-0.014375	-0.014375\\
-0.0145575	-0.0145575\\
-0.0151975	-0.0151975\\
-0.0154275	-0.0154275\\
-0.015975	-0.015975\\
-0.0162975	-0.0162975\\
-0.016205	-0.016205\\
-0.01561	-0.01561\\
-0.015105	-0.015105\\
-0.01451	-0.01451\\
-0.0146475	-0.0146475\\
-0.01451	-0.01451\\
-0.0148775	-0.0148775\\
-0.0149225	-0.0149225\\
-0.0146475	-0.0146475\\
-0.0148325	-0.0148325\\
-0.014465	-0.014465\\
-0.0145575	-0.0145575\\
-0.014375	-0.014375\\
-0.01387	-0.01387\\
-0.013825	-0.013825\\
-0.0136425	-0.0136425\\
-0.01355	-0.01355\\
-0.013505	-0.013505\\
-0.01323	-0.01323\\
-0.0134575	-0.0134575\\
-0.01355	-0.01355\\
-0.0140075	-0.0140075\\
-0.01451	-0.01451\\
-0.014465	-0.014465\\
-0.0143275	-0.0143275\\
-0.0148775	-0.0148775\\
-0.01529	-0.01529\\
-0.0151525	-0.0151525\\
-0.01442	-0.01442\\
-0.0136875	-0.0136875\\
-0.0140525	-0.0140525\\
-0.014695	-0.014695\\
-0.0148325	-0.0148325\\
-0.01538	-0.01538\\
-0.0154725	-0.0154725\\
-0.015105	-0.015105\\
-0.01474	-0.01474\\
-0.0146475	-0.0146475\\
-0.014695	-0.014695\\
-0.01451	-0.01451\\
-0.0139625	-0.0139625\\
-0.01387	-0.01387\\
-0.0137775	-0.0137775\\
-0.0134575	-0.0134575\\
-0.0131375	-0.0131375\\
-0.013505	-0.013505\\
-0.0139625	-0.0139625\\
-0.01419	-0.01419\\
-0.01387	-0.01387\\
-0.0143275	-0.0143275\\
-0.01497	-0.01497\\
-0.014695	-0.014695\\
-0.0148775	-0.0148775\\
-0.0151525	-0.0151525\\
-0.0148775	-0.0148775\\
-0.0143275	-0.0143275\\
-0.0142375	-0.0142375\\
-0.014465	-0.014465\\
-0.0140525	-0.0140525\\
-0.013915	-0.013915\\
-0.0137325	-0.0137325\\
-0.013275	-0.013275\\
-0.0130475	-0.0130475\\
-0.01332	-0.01332\\
-0.0136875	-0.0136875\\
-0.01387	-0.01387\\
-0.0140525	-0.0140525\\
-0.013825	-0.013825\\
-0.01332	-0.01332\\
-0.013	-0.013\\
-0.0131825	-0.0131825\\
-0.013275	-0.013275\\
-0.01419	-0.01419\\
-0.0145575	-0.0145575\\
-0.01442	-0.01442\\
-0.0146475	-0.0146475\\
-0.015105	-0.015105\\
-0.0154275	-0.0154275\\
-0.0155175	-0.0155175\\
-0.0152425	-0.0152425\\
-0.0151975	-0.0151975\\
-0.01529	-0.01529\\
-0.015335	-0.015335\\
-0.0154275	-0.0154275\\
-0.015885	-0.015885\\
-0.0157475	-0.0157475\\
-0.015565	-0.015565\\
-0.0154275	-0.0154275\\
-0.015335	-0.015335\\
-0.0154725	-0.0154725\\
-0.0151525	-0.0151525\\
-0.015015	-0.015015\\
-0.0151975	-0.0151975\\
-0.0154275	-0.0154275\\
-0.0152425	-0.0152425\\
-0.01529	-0.01529\\
-0.0151975	-0.0151975\\
-0.015015	-0.015015\\
-0.0145575	-0.0145575\\
-0.0146475	-0.0146475\\
-0.0152425	-0.0152425\\
-0.0151975	-0.0151975\\
-0.0146475	-0.0146475\\
-0.0146025	-0.0146025\\
-0.0141	-0.0141\\
-0.0137325	-0.0137325\\
-0.013915	-0.013915\\
-0.0136425	-0.0136425\\
-0.0134575	-0.0134575\\
-0.01355	-0.01355\\
-0.01323	-0.01323\\
-0.0134125	-0.0134125\\
-0.0134575	-0.0134575\\
-0.0130475	-0.0130475\\
-0.013275	-0.013275\\
-0.0134575	-0.0134575\\
-0.0136875	-0.0136875\\
-0.0137325	-0.0137325\\
-0.0137775	-0.0137775\\
-0.0141	-0.0141\\
-0.0142375	-0.0142375\\
-0.01497	-0.01497\\
-0.0149225	-0.0149225\\
-0.014695	-0.014695\\
-0.01474	-0.01474\\
-0.014465	-0.014465\\
-0.01387	-0.01387\\
-0.01419	-0.01419\\
-0.0141	-0.0141\\
-0.0136425	-0.0136425\\
-0.013915	-0.013915\\
-0.0140075	-0.0140075\\
-0.014145	-0.014145\\
-0.013825	-0.013825\\
-0.0133675	-0.0133675\\
-0.013	-0.013\\
-0.01291	-0.01291\\
-0.0134125	-0.0134125\\
-0.01355	-0.01355\\
-0.0136875	-0.0136875\\
-0.0141	-0.0141\\
-0.01451	-0.01451\\
-0.014465	-0.014465\\
-0.014695	-0.014695\\
-0.015105	-0.015105\\
-0.01506	-0.01506\\
-0.015015	-0.015015\\
-0.01561	-0.01561\\
-0.015565	-0.015565\\
-0.015655	-0.015655\\
-0.01561	-0.01561\\
-0.015655	-0.015655\\
-0.01593	-0.01593\\
-0.015335	-0.015335\\
-0.014695	-0.014695\\
-0.0142375	-0.0142375\\
-0.0145575	-0.0145575\\
-0.0148775	-0.0148775\\
-0.0149225	-0.0149225\\
-0.0146475	-0.0146475\\
-0.0140075	-0.0140075\\
-0.013915	-0.013915\\
-0.0146475	-0.0146475\\
-0.0152425	-0.0152425\\
-0.01538	-0.01538\\
-0.0148325	-0.0148325\\
-0.014465	-0.014465\\
-0.0146475	-0.0146475\\
-0.0151975	-0.0151975\\
-0.0155175	-0.0155175\\
-0.01561	-0.01561\\
-0.015335	-0.015335\\
-0.01561	-0.01561\\
-0.015885	-0.015885\\
-0.0157925	-0.0157925\\
-0.0162975	-0.0162975\\
-0.0163425	-0.0163425\\
-0.0161125	-0.0161125\\
-0.01625	-0.01625\\
-0.01616	-0.01616\\
-0.0154275	-0.0154275\\
-0.015015	-0.015015\\
-0.0151975	-0.0151975\\
-0.0154275	-0.0154275\\
-0.0154725	-0.0154725\\
-0.0152425	-0.0152425\\
-0.01538	-0.01538\\
-0.01529	-0.01529\\
-0.015015	-0.015015\\
-0.0151525	-0.0151525\\
-0.0155175	-0.0155175\\
-0.015655	-0.015655\\
-0.0154275	-0.0154275\\
-0.01506	-0.01506\\
-0.01442	-0.01442\\
-0.0145575	-0.0145575\\
-0.014375	-0.014375\\
-0.014145	-0.014145\\
-0.0140075	-0.0140075\\
-0.0143275	-0.0143275\\
-0.01474	-0.01474\\
-0.0149225	-0.0149225\\
-0.01497	-0.01497\\
-0.0145575	-0.0145575\\
-0.0141	-0.0141\\
-0.0137775	-0.0137775\\
-0.0136875	-0.0136875\\
-0.0140525	-0.0140525\\
-0.0142825	-0.0142825\\
-0.0148325	-0.0148325\\
-0.0152425	-0.0152425\\
-0.0155175	-0.0155175\\
-0.01561	-0.01561\\
-0.0158375	-0.0158375\\
-0.01538	-0.01538\\
-0.01506	-0.01506\\
-0.0148325	-0.0148325\\
-0.01474	-0.01474\\
-0.01497	-0.01497\\
-0.0148775	-0.0148775\\
-0.01451	-0.01451\\
-0.0142825	-0.0142825\\
-0.014695	-0.014695\\
-0.015105	-0.015105\\
-0.015015	-0.015015\\
-0.0155175	-0.0155175\\
-0.0158375	-0.0158375\\
-0.01561	-0.01561\\
-0.015105	-0.015105\\
-0.0146475	-0.0146475\\
-0.014375	-0.014375\\
-0.0148775	-0.0148775\\
-0.01497	-0.01497\\
-0.01506	-0.01506\\
-0.0151975	-0.0151975\\
-0.0152425	-0.0152425\\
-0.015335	-0.015335\\
-0.01506	-0.01506\\
-0.014785	-0.014785\\
-0.0146025	-0.0146025\\
-0.0149225	-0.0149225\\
-0.01538	-0.01538\\
-0.01561	-0.01561\\
-0.0151525	-0.0151525\\
-0.0157475	-0.0157475\\
-0.0160225	-0.0160225\\
-0.015565	-0.015565\\
-0.0148325	-0.0148325\\
-0.0148775	-0.0148775\\
-0.0149225	-0.0149225\\
-0.014785	-0.014785\\
-0.01497	-0.01497\\
-0.014695	-0.014695\\
-0.0148325	-0.0148325\\
-0.01561	-0.01561\\
-0.01593	-0.01593\\
-0.0155175	-0.0155175\\
-0.01561	-0.01561\\
-0.015655	-0.015655\\
-0.0154275	-0.0154275\\
-0.0152425	-0.0152425\\
-0.0148775	-0.0148775\\
-0.01506	-0.01506\\
-0.0151975	-0.0151975\\
-0.0157475	-0.0157475\\
-0.0161125	-0.0161125\\
-0.01657	-0.01657\\
-0.0164325	-0.0164325\\
-0.0160675	-0.0160675\\
-0.0157925	-0.0157925\\
-0.0157475	-0.0157475\\
-0.0154725	-0.0154725\\
-0.015015	-0.015015\\
-0.01474	-0.01474\\
-0.014785	-0.014785\\
-0.01442	-0.01442\\
-0.0140075	-0.0140075\\
-0.014145	-0.014145\\
-0.014465	-0.014465\\
-0.0142825	-0.0142825\\
-0.013825	-0.013825\\
-0.0136875	-0.0136875\\
-0.01442	-0.01442\\
-0.01506	-0.01506\\
-0.014465	-0.014465\\
-0.01442	-0.01442\\
-0.01451	-0.01451\\
-0.01442	-0.01442\\
-0.0146025	-0.0146025\\
-0.0145575	-0.0145575\\
-0.0142375	-0.0142375\\
-0.0139625	-0.0139625\\
-0.0136425	-0.0136425\\
-0.0136875	-0.0136875\\
-0.0142825	-0.0142825\\
-0.014695	-0.014695\\
-0.014465	-0.014465\\
-0.0140525	-0.0140525\\
-0.013595	-0.013595\\
-0.014145	-0.014145\\
-0.0145575	-0.0145575\\
-0.01474	-0.01474\\
-0.014465	-0.014465\\
-0.0143275	-0.0143275\\
-0.014375	-0.014375\\
-0.014695	-0.014695\\
-0.0152425	-0.0152425\\
-0.01561	-0.01561\\
-0.0154725	-0.0154725\\
-0.0149225	-0.0149225\\
-0.014785	-0.014785\\
-0.01474	-0.01474\\
-0.014375	-0.014375\\
-0.01442	-0.01442\\
-0.014465	-0.014465\\
-0.01451	-0.01451\\
-0.0141	-0.0141\\
-0.013595	-0.013595\\
-0.0136875	-0.0136875\\
-0.014145	-0.014145\\
-0.0146025	-0.0146025\\
-0.01497	-0.01497\\
-0.0152425	-0.0152425\\
-0.01529	-0.01529\\
-0.01538	-0.01538\\
-0.015885	-0.015885\\
-0.01648	-0.01648\\
-0.016525	-0.016525\\
-0.0160225	-0.0160225\\
-0.0161125	-0.0161125\\
-0.01648	-0.01648\\
-0.01625	-0.01625\\
-0.01529	-0.01529\\
-0.01451	-0.01451\\
-0.0142825	-0.0142825\\
-0.013915	-0.013915\\
-0.013275	-0.013275\\
-0.0131825	-0.0131825\\
-0.013505	-0.013505\\
-0.01387	-0.01387\\
-0.0140525	-0.0140525\\
-0.013915	-0.013915\\
-0.0137325	-0.0137325\\
-0.013915	-0.013915\\
-0.01419	-0.01419\\
-0.0143275	-0.0143275\\
-0.0140075	-0.0140075\\
-0.0136875	-0.0136875\\
-0.013505	-0.013505\\
-0.0137775	-0.0137775\\
-0.0140075	-0.0140075\\
-0.013825	-0.013825\\
-0.0134575	-0.0134575\\
-0.0137325	-0.0137325\\
-0.01355	-0.01355\\
-0.013825	-0.013825\\
-0.0142375	-0.0142375\\
-0.0145575	-0.0145575\\
-0.01419	-0.01419\\
-0.0146025	-0.0146025\\
-0.014695	-0.014695\\
-0.01451	-0.01451\\
-0.0143275	-0.0143275\\
-0.014785	-0.014785\\
-0.0151975	-0.0151975\\
-0.015335	-0.015335\\
-0.0154275	-0.0154275\\
-0.01529	-0.01529\\
-0.0151525	-0.0151525\\
-0.01506	-0.01506\\
-0.01529	-0.01529\\
-0.0151975	-0.0151975\\
-0.01538	-0.01538\\
-0.0154725	-0.0154725\\
-0.0163425	-0.0163425\\
-0.015565	-0.015565\\
-0.0148775	-0.0148775\\
-0.01474	-0.01474\\
-0.0149225	-0.0149225\\
-0.0151975	-0.0151975\\
-0.0154725	-0.0154725\\
-0.015655	-0.015655\\
-0.0157025	-0.0157025\\
-0.01529	-0.01529\\
-0.015015	-0.015015\\
-0.0151525	-0.0151525\\
-0.014695	-0.014695\\
-0.0142375	-0.0142375\\
-0.01451	-0.01451\\
-0.0146475	-0.0146475\\
-0.0149225	-0.0149225\\
-0.015335	-0.015335\\
-0.015015	-0.015015\\
-0.014695	-0.014695\\
-0.01497	-0.01497\\
-0.015335	-0.015335\\
-0.0157925	-0.0157925\\
-0.0160675	-0.0160675\\
-0.0163425	-0.0163425\\
-0.01625	-0.01625\\
-0.0161125	-0.0161125\\
-0.01538	-0.01538\\
-0.0148325	-0.0148325\\
-0.015335	-0.015335\\
-0.0157925	-0.0157925\\
-0.015335	-0.015335\\
-0.01561	-0.01561\\
-0.01616	-0.01616\\
-0.016525	-0.016525\\
-0.0161125	-0.0161125\\
-0.01538	-0.01538\\
-0.0146475	-0.0146475\\
-0.014695	-0.014695\\
-0.0142375	-0.0142375\\
-0.01419	-0.01419\\
-0.0140525	-0.0140525\\
-0.0141	-0.0141\\
-0.01419	-0.01419\\
-0.014695	-0.014695\\
-0.01474	-0.01474\\
-0.01506	-0.01506\\
-0.0155175	-0.0155175\\
-0.015565	-0.015565\\
-0.01506	-0.01506\\
-0.0148325	-0.0148325\\
-0.015015	-0.015015\\
-0.01474	-0.01474\\
-0.0145575	-0.0145575\\
-0.01419	-0.01419\\
-0.014375	-0.014375\\
-0.014465	-0.014465\\
-0.0140075	-0.0140075\\
-0.0133675	-0.0133675\\
-0.0130475	-0.0130475\\
-0.0134125	-0.0134125\\
-0.013915	-0.013915\\
-0.0141	-0.0141\\
-0.0137775	-0.0137775\\
-0.013825	-0.013825\\
-0.014145	-0.014145\\
-0.0146475	-0.0146475\\
-0.014465	-0.014465\\
-0.0142375	-0.0142375\\
-0.01442	-0.01442\\
-0.0146475	-0.0146475\\
-0.01474	-0.01474\\
-0.014465	-0.014465\\
-0.014145	-0.014145\\
-0.0143275	-0.0143275\\
-0.01419	-0.01419\\
-0.0142825	-0.0142825\\
-0.0146475	-0.0146475\\
-0.01529	-0.01529\\
-0.0151525	-0.0151525\\
-0.015015	-0.015015\\
-0.0154725	-0.0154725\\
-0.01538	-0.01538\\
-0.0148775	-0.0148775\\
-0.01442	-0.01442\\
-0.0140075	-0.0140075\\
-0.0139625	-0.0139625\\
-0.0137325	-0.0137325\\
-0.013595	-0.013595\\
-0.0142825	-0.0142825\\
-0.013595	-0.013595\\
-0.01323	-0.01323\\
-0.01355	-0.01355\\
-0.0140075	-0.0140075\\
-0.013915	-0.013915\\
-0.013595	-0.013595\\
-0.0131375	-0.0131375\\
-0.013275	-0.013275\\
-0.0140075	-0.0140075\\
-0.0146025	-0.0146025\\
-0.01451	-0.01451\\
-0.0142825	-0.0142825\\
-0.014375	-0.014375\\
-0.01451	-0.01451\\
-0.014785	-0.014785\\
-0.015105	-0.015105\\
-0.0151975	-0.0151975\\
-0.01506	-0.01506\\
-0.0146475	-0.0146475\\
-0.0142825	-0.0142825\\
-0.0142375	-0.0142375\\
-0.01451	-0.01451\\
-0.01419	-0.01419\\
-0.013915	-0.013915\\
-0.01442	-0.01442\\
-0.0148325	-0.0148325\\
-0.014785	-0.014785\\
-0.015015	-0.015015\\
-0.0154275	-0.0154275\\
-0.0148775	-0.0148775\\
-0.01529	-0.01529\\
-0.01561	-0.01561\\
-0.0155175	-0.0155175\\
-0.015565	-0.015565\\
-0.01561	-0.01561\\
-0.0163425	-0.0163425\\
-0.0166625	-0.0166625\\
-0.016205	-0.016205\\
-0.01625	-0.01625\\
-0.01657	-0.01657\\
-0.0166175	-0.0166175\\
-0.016755	-0.016755\\
-0.01712	-0.01712\\
-0.01703	-0.01703\\
-0.016525	-0.016525\\
-0.0158375	-0.0158375\\
-0.0155175	-0.0155175\\
-0.015335	-0.015335\\
-0.0149225	-0.0149225\\
-0.0151975	-0.0151975\\
-0.015655	-0.015655\\
-0.01529	-0.01529\\
-0.015015	-0.015015\\
-0.01497	-0.01497\\
-0.0145575	-0.0145575\\
-0.014465	-0.014465\\
-0.01387	-0.01387\\
-0.013595	-0.013595\\
-0.013505	-0.013505\\
-0.0136875	-0.0136875\\
-0.013595	-0.013595\\
-0.01332	-0.01332\\
-0.013	-0.013\\
-0.012635	-0.012635\\
-0.012955	-0.012955\\
-0.0128625	-0.0128625\\
-0.0127725	-0.0127725\\
-0.013505	-0.013505\\
-0.014145	-0.014145\\
-0.014375	-0.014375\\
-0.01419	-0.01419\\
-0.0142825	-0.0142825\\
-0.01497	-0.01497\\
-0.01442	-0.01442\\
-0.0137325	-0.0137325\\
-0.0134125	-0.0134125\\
-0.013	-0.013\\
-0.0131375	-0.0131375\\
-0.0136425	-0.0136425\\
-0.0142825	-0.0142825\\
-0.0140525	-0.0140525\\
-0.014465	-0.014465\\
-0.0151525	-0.0151525\\
-0.0152425	-0.0152425\\
-0.015015	-0.015015\\
-0.014465	-0.014465\\
-0.01451	-0.01451\\
-0.0148325	-0.0148325\\
-0.0152425	-0.0152425\\
-0.01474	-0.01474\\
-0.0139625	-0.0139625\\
-0.014145	-0.014145\\
-0.0140075	-0.0140075\\
-0.013275	-0.013275\\
-0.012635	-0.012635\\
-0.013	-0.013\\
-0.014145	-0.014145\\
-0.01451	-0.01451\\
-0.0142825	-0.0142825\\
-0.0137325	-0.0137325\\
-0.013595	-0.013595\\
-0.012635	-0.012635\\
-0.0125425	-0.0125425\\
-0.012635	-0.012635\\
-0.012955	-0.012955\\
-0.01355	-0.01355\\
-0.01419	-0.01419\\
-0.0145575	-0.0145575\\
-0.014695	-0.014695\\
-0.0148325	-0.0148325\\
-0.0148775	-0.0148775\\
-0.014785	-0.014785\\
-0.01497	-0.01497\\
-0.01561	-0.01561\\
-0.0157925	-0.0157925\\
-0.015105	-0.015105\\
-0.014145	-0.014145\\
-0.014695	-0.014695\\
-0.0151975	-0.0151975\\
-0.0151525	-0.0151525\\
-0.01451	-0.01451\\
-0.0142825	-0.0142825\\
-0.0148325	-0.0148325\\
-0.015565	-0.015565\\
-0.0158375	-0.0158375\\
-0.0160675	-0.0160675\\
-0.01616	-0.01616\\
-0.0157925	-0.0157925\\
-0.0157025	-0.0157025\\
-0.014785	-0.014785\\
-0.0142825	-0.0142825\\
-0.01451	-0.01451\\
-0.014785	-0.014785\\
-0.0149225	-0.0149225\\
-0.01506	-0.01506\\
-0.0146475	-0.0146475\\
-0.0148325	-0.0148325\\
-0.014785	-0.014785\\
-0.0142825	-0.0142825\\
-0.0136875	-0.0136875\\
-0.01323	-0.01323\\
-0.0134575	-0.0134575\\
-0.01332	-0.01332\\
-0.0128175	-0.0128175\\
-0.0131375	-0.0131375\\
-0.0139625	-0.0139625\\
-0.01387	-0.01387\\
-0.0136425	-0.0136425\\
-0.0139625	-0.0139625\\
-0.0146025	-0.0146025\\
-0.01451	-0.01451\\
-0.0136875	-0.0136875\\
-0.013	-0.013\\
-0.013595	-0.013595\\
-0.014695	-0.014695\\
-0.015335	-0.015335\\
-0.0160675	-0.0160675\\
-0.0166625	-0.0166625\\
-0.016845	-0.016845\\
-0.0164325	-0.0164325\\
-0.016205	-0.016205\\
-0.0166175	-0.0166175\\
-0.01648	-0.01648\\
-0.0163875	-0.0163875\\
-0.01648	-0.01648\\
-0.0166175	-0.0166175\\
-0.0166625	-0.0166625\\
-0.01712	-0.01712\\
-0.016845	-0.016845\\
-0.0161125	-0.0161125\\
-0.01506	-0.01506\\
-0.014375	-0.014375\\
-0.0142825	-0.0142825\\
-0.0142375	-0.0142375\\
-0.0148325	-0.0148325\\
-0.014785	-0.014785\\
-0.01474	-0.01474\\
-0.015015	-0.015015\\
-0.0146025	-0.0146025\\
-0.01451	-0.01451\\
-0.0148325	-0.0148325\\
-0.015105	-0.015105\\
-0.014695	-0.014695\\
-0.014145	-0.014145\\
-0.01419	-0.01419\\
-0.0151525	-0.0151525\\
-0.015975	-0.015975\\
-0.0160225	-0.0160225\\
-0.01616	-0.01616\\
-0.01648	-0.01648\\
-0.017075	-0.017075\\
-0.017165	-0.017165\\
-0.016525	-0.016525\\
-0.0158375	-0.0158375\\
-0.0148775	-0.0148775\\
-0.0145575	-0.0145575\\
-0.0146025	-0.0146025\\
-0.014785	-0.014785\\
-0.0146475	-0.0146475\\
-0.014375	-0.014375\\
-0.014465	-0.014465\\
-0.0146025	-0.0146025\\
-0.014785	-0.014785\\
-0.01506	-0.01506\\
-0.0148775	-0.0148775\\
-0.014145	-0.014145\\
-0.01355	-0.01355\\
-0.0131825	-0.0131825\\
-0.013	-0.013\\
-0.0130925	-0.0130925\\
-0.01355	-0.01355\\
-0.0137775	-0.0137775\\
-0.0140075	-0.0140075\\
-0.014695	-0.014695\\
-0.0151975	-0.0151975\\
-0.0160675	-0.0160675\\
-0.0158375	-0.0158375\\
-0.015975	-0.015975\\
-0.016205	-0.016205\\
-0.015885	-0.015885\\
-0.015565	-0.015565\\
-0.015015	-0.015015\\
-0.014785	-0.014785\\
-0.0154725	-0.0154725\\
-0.016525	-0.016525\\
-0.01712	-0.01712\\
-0.01703	-0.01703\\
-0.0168	-0.0168\\
-0.0164325	-0.0164325\\
-0.0162975	-0.0162975\\
-0.0166625	-0.0166625\\
-0.0161125	-0.0161125\\
-0.01561	-0.01561\\
-0.01538	-0.01538\\
-0.0158375	-0.0158375\\
-0.0160675	-0.0160675\\
-0.01529	-0.01529\\
-0.01506	-0.01506\\
-0.01474	-0.01474\\
-0.015015	-0.015015\\
-0.015655	-0.015655\\
-0.0157925	-0.0157925\\
-0.015655	-0.015655\\
-0.01529	-0.01529\\
-0.0152425	-0.0152425\\
-0.015105	-0.015105\\
-0.014695	-0.014695\\
-0.0143275	-0.0143275\\
-0.0140075	-0.0140075\\
-0.0136875	-0.0136875\\
-0.013595	-0.013595\\
-0.0140525	-0.0140525\\
-0.0146475	-0.0146475\\
-0.01474	-0.01474\\
-0.0143275	-0.0143275\\
-0.0141	-0.0141\\
-0.013915	-0.013915\\
-0.0140525	-0.0140525\\
-0.014375	-0.014375\\
-0.01451	-0.01451\\
-0.0145575	-0.0145575\\
-0.0140525	-0.0140525\\
-0.0146475	-0.0146475\\
-0.01529	-0.01529\\
-0.015105	-0.015105\\
-0.014145	-0.014145\\
-0.01387	-0.01387\\
-0.0143275	-0.0143275\\
-0.014465	-0.014465\\
-0.0143275	-0.0143275\\
-0.0139625	-0.0139625\\
-0.0142375	-0.0142375\\
-0.0148325	-0.0148325\\
-0.0145575	-0.0145575\\
-0.013595	-0.013595\\
-0.0134125	-0.0134125\\
-0.0143275	-0.0143275\\
-0.0148775	-0.0148775\\
-0.01451	-0.01451\\
-0.01419	-0.01419\\
-0.014785	-0.014785\\
-0.015335	-0.015335\\
-0.01538	-0.01538\\
-0.0157475	-0.0157475\\
-0.01625	-0.01625\\
-0.015975	-0.015975\\
-0.015565	-0.015565\\
-0.0158375	-0.0158375\\
-0.01561	-0.01561\\
-0.015655	-0.015655\\
-0.0160225	-0.0160225\\
-0.0158375	-0.0158375\\
-0.01506	-0.01506\\
-0.0152425	-0.0152425\\
-0.0161125	-0.0161125\\
-0.01657	-0.01657\\
-0.0163425	-0.0163425\\
-0.0160675	-0.0160675\\
-0.0163425	-0.0163425\\
-0.016205	-0.016205\\
-0.01561	-0.01561\\
-0.01529	-0.01529\\
-0.0154275	-0.0154275\\
-0.0155175	-0.0155175\\
-0.01561	-0.01561\\
-0.015335	-0.015335\\
-0.0152425	-0.0152425\\
-0.015565	-0.015565\\
-0.0151975	-0.0151975\\
-0.014785	-0.014785\\
-0.014465	-0.014465\\
-0.01419	-0.01419\\
-0.0140525	-0.0140525\\
-0.0145575	-0.0145575\\
-0.014695	-0.014695\\
-0.015015	-0.015015\\
-0.01538	-0.01538\\
-0.0157925	-0.0157925\\
-0.01561	-0.01561\\
-0.01538	-0.01538\\
-0.01442	-0.01442\\
-0.014375	-0.014375\\
-0.0151975	-0.0151975\\
-0.01506	-0.01506\\
-0.014375	-0.014375\\
-0.014695	-0.014695\\
-0.01538	-0.01538\\
-0.015565	-0.015565\\
-0.015975	-0.015975\\
-0.01657	-0.01657\\
-0.0163875	-0.0163875\\
-0.0160675	-0.0160675\\
-0.016205	-0.016205\\
-0.0158375	-0.0158375\\
-0.0154725	-0.0154725\\
-0.015655	-0.015655\\
-0.0158375	-0.0158375\\
-0.015335	-0.015335\\
-0.01497	-0.01497\\
-0.01474	-0.01474\\
-0.0148325	-0.0148325\\
-0.0148775	-0.0148775\\
-0.0146475	-0.0146475\\
-0.0151975	-0.0151975\\
-0.0155175	-0.0155175\\
-0.0152425	-0.0152425\\
-0.01497	-0.01497\\
-0.0151975	-0.0151975\\
-0.0148775	-0.0148775\\
-0.01419	-0.01419\\
-0.0140525	-0.0140525\\
-0.0142825	-0.0142825\\
-0.0140075	-0.0140075\\
-0.0134125	-0.0134125\\
-0.0139625	-0.0139625\\
-0.0146025	-0.0146025\\
-0.015015	-0.015015\\
-0.0154275	-0.0154275\\
-0.0157925	-0.0157925\\
-0.0151975	-0.0151975\\
-0.014695	-0.014695\\
-0.01529	-0.01529\\
-0.01616	-0.01616\\
-0.0161125	-0.0161125\\
-0.01593	-0.01593\\
-0.0164325	-0.0164325\\
-0.01616	-0.01616\\
-0.0157925	-0.0157925\\
-0.015655	-0.015655\\
-0.01593	-0.01593\\
-0.0157025	-0.0157025\\
-0.01538	-0.01538\\
-0.0158375	-0.0158375\\
-0.0162975	-0.0162975\\
-0.0164325	-0.0164325\\
-0.01561	-0.01561\\
-0.0145575	-0.0145575\\
-0.0151525	-0.0151525\\
-0.0152425	-0.0152425\\
-0.014375	-0.014375\\
-0.0137325	-0.0137325\\
-0.013915	-0.013915\\
-0.01442	-0.01442\\
-0.0152425	-0.0152425\\
-0.0151525	-0.0151525\\
-0.014785	-0.014785\\
-0.01419	-0.01419\\
-0.01355	-0.01355\\
-0.013595	-0.013595\\
-0.013505	-0.013505\\
-0.013595	-0.013595\\
-0.01419	-0.01419\\
-0.0143275	-0.0143275\\
-0.0136875	-0.0136875\\
-0.0134125	-0.0134125\\
-0.01355	-0.01355\\
-0.0137775	-0.0137775\\
-0.0134575	-0.0134575\\
-0.0136425	-0.0136425\\
-0.013505	-0.013505\\
-0.0131375	-0.0131375\\
-0.0137325	-0.0137325\\
-0.014375	-0.014375\\
-0.0151975	-0.0151975\\
-0.0158375	-0.0158375\\
-0.01593	-0.01593\\
-0.0151525	-0.0151525\\
-0.0145575	-0.0145575\\
-0.014375	-0.014375\\
-0.013825	-0.013825\\
-0.0134125	-0.0134125\\
-0.01387	-0.01387\\
-0.01419	-0.01419\\
-0.014145	-0.014145\\
-0.0142825	-0.0142825\\
-0.01442	-0.01442\\
-0.0145575	-0.0145575\\
-0.014695	-0.014695\\
-0.01497	-0.01497\\
-0.01506	-0.01506\\
-0.015565	-0.015565\\
-0.0151525	-0.0151525\\
-0.014145	-0.014145\\
-0.013825	-0.013825\\
-0.014375	-0.014375\\
-0.0143275	-0.0143275\\
-0.014145	-0.014145\\
-0.0133675	-0.0133675\\
-0.01332	-0.01332\\
-0.0131825	-0.0131825\\
-0.0125875	-0.0125875\\
-0.012635	-0.012635\\
-0.013275	-0.013275\\
-0.0136875	-0.0136875\\
-0.0143275	-0.0143275\\
-0.01497	-0.01497\\
-0.0148775	-0.0148775\\
-0.0140075	-0.0140075\\
-0.01419	-0.01419\\
-0.01451	-0.01451\\
-0.0139625	-0.0139625\\
-0.013915	-0.013915\\
-0.0146025	-0.0146025\\
-0.014785	-0.014785\\
-0.015335	-0.015335\\
-0.015975	-0.015975\\
-0.0154725	-0.0154725\\
-0.0151525	-0.0151525\\
-0.01497	-0.01497\\
-0.0152425	-0.0152425\\
-0.01561	-0.01561\\
-0.0154275	-0.0154275\\
-0.0148325	-0.0148325\\
};
\end{axis}

\begin{axis}[%
width=4.927cm,
height=2.746cm,
at={(6.484cm,3.814cm)},
scale only axis,
xmin=-0.018,
xmax=-0.012,
xlabel style={font=\color{white!15!black}},
xlabel={$u(t-1)$},
ymin=-0.6988525,
ymax=0,
ylabel style={font=\color{white!15!black}},
ylabel={$\delta^4 y(t)$},
axis background/.style={fill=white},
title style={font=\bfseries},
title={C8, R = 0.6793},
axis x line*=bottom,
axis y line*=left
]
\addplot[only marks, mark=*, mark options={}, mark size=1.5000pt, color=mycolor1, fill=mycolor1] table[row sep=crcr]{%
x	y\\
-0.015655	-0.31433\\
-0.015655	-0.3845225\\
-0.015885	-0.2899175\\
-0.0157475	-0.1525875\\
-0.01497	-0.1861575\\
-0.0148325	-0.183105\\
-0.0148325	-0.216675\\
-0.015015	-0.183105\\
-0.0148775	-0.08545\\
-0.0140075	-0.05188\\
-0.01323	-0.0579825\\
-0.0128625	-0.146485\\
-0.0137325	-0.250245\\
-0.014695	-0.2746575\\
-0.015015	-0.286865\\
-0.015105	-0.20752\\
-0.0148775	-0.131225\\
-0.0142825	-0.26245\\
-0.0148775	-0.2044675\\
-0.014785	-0.149535\\
-0.014375	-0.24414\\
-0.0148325	-0.20752\\
-0.0148775	-0.1800525\\
-0.0146475	-0.29602\\
-0.015105	-0.268555\\
-0.0151975	-0.20752\\
-0.0149225	-0.2990725\\
-0.0151975	-0.3082275\\
-0.015335	-0.2380375\\
-0.0151525	-0.201415\\
-0.0149225	-0.15564\\
-0.0146475	-0.18921\\
-0.014695	-0.2380375\\
-0.0148775	-0.2197275\\
-0.0149225	-0.2288825\\
-0.0149225	-0.24109\\
-0.01497	-0.2563475\\
-0.015015	-0.354005\\
-0.0154275	-0.320435\\
-0.0155175	-0.2136225\\
-0.0151525	-0.198365\\
-0.0149225	-0.1190175\\
-0.014375	-0.12207\\
-0.0141	-0.1708975\\
-0.014375	-0.131225\\
-0.0142375	-0.1434325\\
-0.014145	-0.18921\\
-0.014465	-0.2227775\\
-0.014695	-0.2044675\\
-0.014695	-0.320435\\
-0.0151975	-0.2471925\\
-0.015105	-0.1434325\\
-0.01451	-0.112915\\
-0.014145	-0.15564\\
-0.0142375	-0.1800525\\
-0.014465	-0.1007075\\
-0.0139625	-0.1068125\\
-0.0137325	-0.14038\\
-0.0140075	-0.1159675\\
-0.013915	-0.1007075\\
-0.0136875	-0.131225\\
-0.01387	-0.15564\\
-0.0140075	-0.2319325\\
-0.0146025	-0.22583\\
-0.014695	-0.268555\\
-0.0149225	-0.2471925\\
-0.01497	-0.2899175\\
-0.01506	-0.2288825\\
-0.01497	-0.2288825\\
-0.0148775	-0.198365\\
-0.014785	-0.18921\\
-0.01474	-0.19226\\
-0.014695	-0.3326425\\
-0.0151975	-0.4730225\\
-0.01593	-0.48523\\
-0.01625	-0.4730225\\
-0.0162975	-0.2990725\\
-0.0158375	-0.427245\\
-0.0160675	-0.5340575\\
-0.01648	-0.5493175\\
-0.0166625	-0.424195\\
-0.01648	-0.5493175\\
-0.0166625	-0.6988525\\
-0.0172125	-0.494385\\
-0.01703	-0.4028325\\
-0.0167075	-0.2746575\\
-0.01616	-0.2136225\\
-0.015655	-0.183105\\
-0.015335	-0.2288825\\
-0.01538	-0.15564\\
-0.01506	-0.1159675\\
-0.014465	-0.1068125\\
-0.014145	-0.13733\\
-0.0142825	-0.198365\\
-0.014695	-0.1861575\\
-0.01474	-0.18921\\
-0.01474	-0.250245\\
-0.015015	-0.2594\\
-0.0151975	-0.354005\\
-0.01561	-0.286865\\
-0.015655	-0.357055\\
-0.0157475	-0.26245\\
-0.015565	-0.22583\\
-0.01529	-0.2594\\
-0.01538	-0.19226\\
-0.0151525	-0.17395\\
-0.0149225	-0.2105725\\
-0.01506	-0.216675\\
-0.015105	-0.149535\\
-0.0148325	-0.1617425\\
-0.01474	-0.12207\\
-0.014465	-0.1342775\\
-0.0143275	-0.1434325\\
-0.01442	-0.10376\\
-0.01419	-0.131225\\
-0.01419	-0.164795\\
-0.01442	-0.253295\\
-0.01497	-0.26245\\
-0.015105	-0.286865\\
-0.01529	-0.1708975\\
-0.0148325	-0.234985\\
-0.01497	-0.375365\\
-0.01561	-0.286865\\
-0.015565	-0.31433\\
-0.01561	-0.320435\\
-0.015655	-0.201415\\
-0.0151975	-0.1342775\\
-0.0146025	-0.18921\\
-0.01474	-0.2136225\\
-0.0149225	-0.3448475\\
-0.0155175	-0.4302975\\
-0.0160225	-0.427245\\
-0.01616	-0.3082275\\
-0.0158375	-0.3265375\\
-0.0158375	-0.2899175\\
-0.0157475	-0.2838125\\
-0.015655	-0.27771\\
-0.015655	-0.19226\\
-0.0152425	-0.1708975\\
-0.01497	-0.1525875\\
-0.014785	-0.1434325\\
-0.0146025	-0.15564\\
-0.0146475	-0.2288825\\
-0.015015	-0.3509525\\
-0.015565	-0.27771\\
-0.01561	-0.198365\\
-0.0151525	-0.1617425\\
-0.0148775	-0.15564\\
-0.01474	-0.1007075\\
-0.0142375	-0.0732425\\
-0.0136875	-0.0915525\\
-0.0136425	-0.1586925\\
-0.0141	-0.128175\\
-0.0140525	-0.131225\\
-0.0140525	-0.146485\\
-0.014145	-0.13733\\
-0.014145	-0.094605\\
-0.013825	-0.07019\\
-0.0134575	-0.0579825\\
-0.0130475	-0.0976575\\
-0.01323	-0.20752\\
-0.01419	-0.29297\\
-0.01497	-0.3265375\\
-0.01529	-0.216675\\
-0.015015	-0.146485\\
-0.01451	-0.112915\\
-0.0141	-0.0823975\\
-0.0136425	-0.1678475\\
-0.0141	-0.164795\\
-0.0142825	-0.2288825\\
-0.0146025	-0.29297\\
-0.01506	-0.46692\\
-0.01593	-0.54016\\
-0.016525	-0.442505\\
-0.01648	-0.4211425\\
-0.0163425	-0.2746575\\
-0.015885	-0.305175\\
-0.0157925	-0.32959\\
-0.01593	-0.36621\\
-0.0160225	-0.2716075\\
-0.0158375	-0.250245\\
-0.015565	-0.2471925\\
-0.015565	-0.2227775\\
-0.0154725	-0.2655025\\
-0.0155175	-0.1953125\\
-0.01529	-0.3265375\\
-0.015655	-0.39978\\
-0.0161125	-0.2838125\\
-0.0158375	-0.1770025\\
-0.01529	-0.183105\\
-0.01506	-0.3082275\\
-0.0155175	-0.3845225\\
-0.015975	-0.4730225\\
-0.0162975	-0.5065925\\
-0.016525	-0.5065925\\
-0.0166625	-0.3845225\\
-0.0163875	-0.3448475\\
-0.016205	-0.3967275\\
-0.0162975	-0.46997\\
-0.016525	-0.2746575\\
-0.0160225	-0.1800525\\
-0.015335	-0.253295\\
-0.0155175	-0.216675\\
-0.0154275	-0.1342775\\
-0.0148775	-0.1861575\\
-0.0149225	-0.2105725\\
-0.015105	-0.1678475\\
-0.0149225	-0.128175\\
-0.014695	-0.1770025\\
-0.014785	-0.146485\\
-0.014695	-0.198365\\
-0.014785	-0.268555\\
-0.01529	-0.250245\\
-0.015335	-0.17395\\
-0.015015	-0.1770025\\
-0.0148775	-0.268555\\
-0.0152425	-0.442505\\
-0.0160225	-0.320435\\
-0.0160225	-0.198365\\
-0.01538	-0.1525875\\
-0.01497	-0.1770025\\
-0.0149225	-0.1617425\\
-0.0148775	-0.12207\\
-0.01451	-0.1342775\\
-0.014465	-0.164795\\
-0.0146025	-0.20752\\
-0.0148325	-0.2288825\\
-0.01506	-0.1586925\\
-0.014785	-0.26245\\
-0.015105	-0.31128\\
-0.0155175	-0.29602\\
-0.0155175	-0.2288825\\
-0.01529	-0.250245\\
-0.01529	-0.338745\\
-0.01561	-0.2197275\\
-0.0154275	-0.1068125\\
-0.014465	-0.1434325\\
-0.0142825	-0.15564\\
-0.014465	-0.2044675\\
-0.014785	-0.1800525\\
-0.014785	-0.131225\\
-0.01451	-0.198365\\
-0.014785	-0.201415\\
-0.0148325	-0.1434325\\
-0.01451	-0.1770025\\
-0.0146475	-0.1586925\\
-0.0146025	-0.14038\\
-0.01442	-0.1678475\\
-0.0145575	-0.10376\\
-0.01419	-0.12207\\
-0.0141	-0.0915525\\
-0.01387	-0.1007075\\
-0.0137775	-0.18921\\
-0.014375	-0.2105725\\
-0.0146475	-0.2838125\\
-0.015015	-0.36621\\
-0.015565	-0.2471925\\
-0.015335	-0.146485\\
-0.014695	-0.1159675\\
-0.0142375	-0.1098625\\
-0.0140525	-0.1586925\\
-0.014375	-0.1068125\\
-0.0141	-0.1190175\\
-0.0139625	-0.1525875\\
-0.01419	-0.2563475\\
-0.0148325	-0.234985\\
-0.01497	-0.3356925\\
-0.01538	-0.2227775\\
-0.0151975	-0.2044675\\
-0.0148775	-0.1159675\\
-0.0142825	-0.112915\\
-0.0139625	-0.07019\\
-0.01355	-0.0885\\
-0.0134125	-0.094605\\
-0.013505	-0.1678475\\
-0.0140075	-0.1770025\\
-0.0142825	-0.198365\\
-0.01442	-0.2136225\\
-0.0145575	-0.31433\\
-0.015105	-0.2807625\\
-0.015335	-0.2105725\\
-0.015015	-0.2563475\\
-0.01506	-0.2746575\\
-0.0152425	-0.3601075\\
-0.015565	-0.2899175\\
-0.0154725	-0.18921\\
-0.015015	-0.1007075\\
-0.01419	-0.0732425\\
-0.01355	-0.0732425\\
-0.0133675	-0.0915525\\
-0.0134575	-0.0976575\\
-0.013505	-0.094605\\
-0.0134575	-0.1251225\\
-0.013595	-0.1861575\\
-0.01419	-0.1708975\\
-0.0143275	-0.1770025\\
-0.0142825	-0.20752\\
-0.014465	-0.128175\\
-0.014145	-0.076295\\
-0.013505	-0.146485\\
-0.013825	-0.22583\\
-0.014465	-0.2197275\\
-0.0146025	-0.24414\\
-0.014785	-0.20752\\
-0.014695	-0.1525875\\
-0.014375	-0.2136225\\
-0.0145575	-0.29602\\
-0.015015	-0.29602\\
-0.0151975	-0.2105725\\
-0.0149225	-0.3417975\\
-0.015335	-0.250245\\
-0.01529	-0.253295\\
-0.015105	-0.29297\\
-0.015335	-0.2899175\\
-0.01538	-0.216675\\
-0.0151525	-0.19226\\
-0.01497	-0.320435\\
-0.01538	-0.4455575\\
-0.0160225	-0.3479\\
-0.01593	-0.198365\\
-0.01529	-0.19226\\
-0.01497	-0.198365\\
-0.015015	-0.2471925\\
-0.0151525	-0.286865\\
-0.015335	-0.2136225\\
-0.0151525	-0.22583\\
-0.0151525	-0.302125\\
-0.0154275	-0.32959\\
-0.015565	-0.2197275\\
-0.0152425	-0.1708975\\
-0.01497	-0.2288825\\
-0.015105	-0.390625\\
-0.0157925	-0.3448475\\
-0.015885	-0.3509525\\
-0.0158375	-0.2105725\\
-0.01538	-0.19226\\
-0.015015	-0.1800525\\
-0.0149225	-0.1190175\\
-0.014465	-0.076295\\
-0.013825	-0.0671375\\
-0.0133675	-0.0579825\\
-0.0130475	-0.13733\\
-0.013595	-0.1770025\\
-0.01419	-0.2197275\\
-0.0145575	-0.164795\\
-0.01442	-0.183105\\
-0.014375	-0.20752\\
-0.0145575	-0.1342775\\
-0.01419	-0.14038\\
-0.0140525	-0.1342775\\
-0.0140525	-0.1007075\\
-0.0137775	-0.128175\\
-0.01387	-0.2105725\\
-0.014375	-0.2655025\\
-0.0148325	-0.1586925\\
-0.014465	-0.12207\\
-0.0140525	-0.1007075\\
-0.01387	-0.1251225\\
-0.0139625	-0.0885\\
-0.0136425	-0.19226\\
-0.0142375	-0.3265375\\
-0.0151525	-0.26245\\
-0.0151975	-0.3479\\
-0.0154275	-0.405885\\
-0.0157925	-0.4119875\\
-0.015975	-0.31128\\
-0.0157025	-0.22583\\
-0.01529	-0.2227775\\
-0.0151975	-0.2227775\\
-0.0151525	-0.2655025\\
-0.01529	-0.31433\\
-0.0155175	-0.31433\\
-0.015565	-0.4211425\\
-0.01593	-0.4577625\\
-0.01625	-0.5493175\\
-0.01657	-0.3692625\\
-0.0162975	-0.41504\\
-0.0162975	-0.460815\\
-0.0164325	-0.3509525\\
-0.01625	-0.405885\\
-0.0162975	-0.3479\\
-0.016205	-0.201415\\
-0.015565	-0.131225\\
-0.0148325	-0.14038\\
-0.0146025	-0.2044675\\
-0.0148775	-0.164795\\
-0.014785	-0.1098625\\
-0.014375	-0.15564\\
-0.01451	-0.1190175\\
-0.0143275	-0.0915525\\
-0.0139625	-0.1159675\\
-0.0140075	-0.131225\\
-0.0140525	-0.0915525\\
-0.01387	-0.1770025\\
-0.0142375	-0.2319325\\
-0.01474	-0.1708975\\
-0.0146025	-0.286865\\
-0.015105	-0.4119875\\
-0.015885	-0.375365\\
-0.0160225	-0.4913325\\
-0.0163425	-0.32959\\
-0.0160675	-0.3601075\\
-0.0160225	-0.24109\\
-0.015655	-0.1586925\\
-0.01497	-0.19226\\
-0.0149225	-0.149535\\
-0.014785	-0.112915\\
-0.014375	-0.1617425\\
-0.014465	-0.0915525\\
-0.01419	-0.07019\\
-0.013505	-0.08545\\
-0.0134125	-0.1770025\\
-0.0141	-0.216675\\
-0.0146475	-0.268555\\
-0.01497	-0.2594\\
-0.015105	-0.250245\\
-0.01506	-0.286865\\
-0.0151975	-0.234985\\
-0.015105	-0.19226\\
-0.0148775	-0.19226\\
-0.0148325	-0.15564\\
-0.0146475	-0.24414\\
-0.01497	-0.1708975\\
-0.01474	-0.14038\\
-0.01442	-0.2044675\\
-0.01474	-0.2838125\\
-0.0151525	-0.216675\\
-0.015015	-0.164795\\
-0.014695	-0.14038\\
-0.01451	-0.164795\\
-0.0145575	-0.112915\\
-0.0142375	-0.112915\\
-0.0140075	-0.149535\\
-0.0142375	-0.18921\\
-0.014465	-0.2105725\\
-0.0146475	-0.164795\\
-0.0145575	-0.216675\\
-0.014695	-0.198365\\
-0.014695	-0.1190175\\
-0.0142375	-0.1251225\\
-0.0141	-0.2380375\\
-0.014695	-0.32959\\
-0.01538	-0.32959\\
-0.0155175	-0.268555\\
-0.015335	-0.2594\\
-0.0152425	-0.198365\\
-0.01506	-0.19226\\
-0.0148775	-0.1525875\\
-0.014695	-0.2227775\\
-0.0148325	-0.18921\\
-0.0148775	-0.1251225\\
-0.01442	-0.1861575\\
-0.0146025	-0.2136225\\
-0.014785	-0.253295\\
-0.01497	-0.2746575\\
-0.0151975	-0.2746575\\
-0.0151975	-0.372315\\
-0.015655	-0.4547125\\
-0.0161125	-0.512695\\
-0.0164325	-0.3234875\\
-0.0161125	-0.183105\\
-0.01529	-0.1190175\\
-0.0145575	-0.08545\\
-0.0140075	-0.13733\\
-0.0141	-0.1159675\\
-0.0140525	-0.201415\\
-0.01451	-0.1800525\\
-0.0146025	-0.1617425\\
-0.014465	-0.2136225\\
-0.014695	-0.2471925\\
-0.0149225	-0.29297\\
-0.0151975	-0.3082275\\
-0.01538	-0.4394525\\
-0.01593	-0.375365\\
-0.0160675	-0.29297\\
-0.0157475	-0.201415\\
-0.015335	-0.201415\\
-0.015105	-0.20752\\
-0.0151525	-0.2655025\\
-0.015335	-0.18921\\
-0.015105	-0.198365\\
-0.01497	-0.1861575\\
-0.015015	-0.1068125\\
-0.014375	-0.1861575\\
-0.0145575	-0.268555\\
-0.015105	-0.183105\\
-0.0149225	-0.201415\\
-0.014785	-0.3692625\\
-0.01561	-0.2746575\\
-0.0155175	-0.2655025\\
-0.01538	-0.3692625\\
-0.0157475	-0.3509525\\
-0.0158375	-0.2197275\\
-0.0154275	-0.14038\\
-0.0148325	-0.146485\\
-0.0146025	-0.24414\\
-0.01506	-0.3784175\\
-0.0157025	-0.405885\\
-0.0160225	-0.41809\\
-0.0161125	-0.3234875\\
-0.015975	-0.20752\\
-0.01538	-0.1800525\\
-0.01506	-0.1342775\\
-0.014695	-0.1251225\\
-0.014465	-0.1068125\\
-0.0142375	-0.1525875\\
-0.01442	-0.1800525\\
-0.0146025	-0.10376\\
-0.01419	-0.164795\\
-0.01442	-0.2227775\\
-0.0148325	-0.253295\\
-0.015015	-0.320435\\
-0.01538	-0.4028325\\
-0.015885	-0.4821775\\
-0.0162975	-0.3417975\\
-0.0160675	-0.29602\\
-0.0158375	-0.31128\\
-0.0158375	-0.2563475\\
-0.015655	-0.183105\\
-0.0151975	-0.2227775\\
-0.0151525	-0.3417975\\
-0.0157025	-0.405885\\
-0.0160675	-0.234985\\
-0.01561	-0.13733\\
-0.0148325	-0.094605\\
-0.0142375	-0.0579825\\
-0.0133675	-0.0823975\\
-0.0130475	-0.112915\\
-0.0134575	-0.131225\\
-0.0136875	-0.183105\\
-0.01419	-0.146485\\
-0.01419	-0.112915\\
-0.01387	-0.112915\\
-0.0137775	-0.1098625\\
-0.0137325	-0.1007075\\
-0.0136425	-0.1190175\\
-0.0137325	-0.1770025\\
-0.014145	-0.26245\\
-0.014785	-0.27771\\
-0.01506	-0.268555\\
-0.015105	-0.3082275\\
-0.0152425	-0.38147\\
-0.0157025	-0.3875725\\
-0.0158375	-0.4577625\\
-0.0161125	-0.41809\\
-0.01625	-0.31128\\
-0.01593	-0.2136225\\
-0.0154275	-0.20752\\
-0.0151975	-0.354005\\
-0.0157025	-0.3967275\\
-0.0160675	-0.2899175\\
-0.0157925	-0.19226\\
-0.015335	-0.0976575\\
-0.0142825	-0.076295\\
-0.013595	-0.0732425\\
-0.0134575	-0.0488275\\
-0.01291	-0.112915\\
-0.013275	-0.1434325\\
-0.013825	-0.15564\\
-0.0139625	-0.1342775\\
-0.013915	-0.08545\\
-0.0134575	-0.0671375\\
-0.0131375	-0.0915525\\
-0.0131825	-0.10376\\
-0.0133675	-0.1098625\\
-0.0134125	-0.0885\\
-0.013275	-0.07019\\
-0.0130925	-0.1251225\\
-0.0134575	-0.2807625\\
-0.0146475	-0.320435\\
-0.015335	-0.36316\\
-0.015565	-0.253295\\
-0.015335	-0.3326425\\
-0.0154725	-0.4882825\\
-0.01625	-0.4364025\\
-0.0162975	-0.460815\\
-0.0163875	-0.338745\\
-0.0161125	-0.3448475\\
-0.0160225	-0.36316\\
-0.0161125	-0.2380375\\
-0.0157025	-0.22583\\
-0.0154275	-0.14038\\
-0.0148775	-0.13733\\
-0.014465	-0.2563475\\
-0.015105	-0.1708975\\
-0.0149225	-0.198365\\
-0.014785	-0.2105725\\
-0.0149225	-0.198365\\
-0.0148775	-0.198365\\
-0.0149225	-0.2105725\\
-0.0149225	-0.234985\\
-0.01506	-0.2563475\\
-0.0151975	-0.2227775\\
-0.015105	-0.1678475\\
-0.0148325	-0.2105725\\
-0.01497	-0.0976575\\
-0.0142375	-0.05188\\
-0.01323	-0.0915525\\
-0.0131375	-0.128175\\
-0.0136425	-0.0671375\\
-0.01323	-0.03357\\
-0.01245	-0.079345\\
-0.0125875	-0.1251225\\
-0.0131825	-0.146485\\
-0.013595	-0.1800525\\
-0.0139625	-0.15564\\
-0.0139625	-0.2288825\\
-0.014375	-0.201415\\
-0.0145575	-0.1434325\\
-0.01419	-0.216675\\
-0.014465	-0.12207\\
-0.0140075	-0.10376\\
-0.01355	-0.07019\\
-0.01323	-0.0549325\\
-0.01268	-0.094605\\
-0.01291	-0.07019\\
-0.0128625	-0.1251225\\
-0.01323	-0.18921\\
-0.013915	-0.183105\\
-0.0141	-0.13733\\
-0.01387	-0.1525875\\
-0.01387	-0.112915\\
-0.0137325	-0.164795\\
-0.01387	-0.1190175\\
-0.0137325	-0.1159675\\
-0.01355	-0.1068125\\
-0.013505	-0.0823975\\
-0.01323	-0.1068125\\
-0.01332	-0.094605\\
-0.013275	-0.1617425\\
-0.0137325	-0.1586925\\
-0.01387	-0.1251225\\
-0.0136425	-0.1159675\\
-0.013595	-0.0915525\\
-0.0133675	-0.1098625\\
-0.0133675	-0.234985\\
-0.01419	-0.302125\\
-0.0149225	-0.2044675\\
-0.014695	-0.201415\\
-0.01451	-0.13733\\
-0.014145	-0.234985\\
-0.01451	-0.2044675\\
-0.0146475	-0.13733\\
-0.01419	-0.1770025\\
-0.0143275	-0.131225\\
-0.014145	-0.18921\\
-0.0142825	-0.1068125\\
-0.0139625	-0.1251225\\
-0.0136875	-0.1434325\\
-0.013915	-0.1861575\\
-0.01419	-0.146485\\
-0.0140525	-0.1525875\\
-0.0140075	-0.1525875\\
-0.0140075	-0.2380375\\
-0.014465	-0.198365\\
-0.0145575	-0.31433\\
-0.01497	-0.3936775\\
-0.01561	-0.31433\\
-0.015565	-0.2044675\\
-0.015015	-0.15564\\
-0.0145575	-0.2227775\\
-0.014785	-0.2807625\\
-0.0151975	-0.2197275\\
-0.015015	-0.268555\\
-0.015105	-0.1617425\\
-0.01474	-0.08545\\
-0.013915	-0.094605\\
-0.013595	-0.2105725\\
-0.0143275	-0.1770025\\
-0.01451	-0.17395\\
-0.014375	-0.26245\\
-0.014785	-0.24414\\
-0.0149225	-0.2197275\\
-0.014785	-0.15564\\
-0.014465	-0.183105\\
-0.01451	-0.250245\\
-0.0148325	-0.2044675\\
-0.014785	-0.2197275\\
-0.01474	-0.1953125\\
-0.0146475	-0.146485\\
-0.014375	-0.1678475\\
-0.01442	-0.12207\\
-0.014145	-0.14038\\
-0.014145	-0.2380375\\
-0.014695	-0.2746575\\
-0.01497	-0.2899175\\
-0.015105	-0.2563475\\
-0.01506	-0.183105\\
-0.01474	-0.128175\\
-0.0143275	-0.15564\\
-0.0142825	-0.1770025\\
-0.01442	-0.2288825\\
-0.014695	-0.27771\\
-0.015015	-0.32959\\
-0.015335	-0.268555\\
-0.0152425	-0.15564\\
-0.0146475	-0.2746575\\
-0.01497	-0.3784175\\
-0.01561	-0.268555\\
-0.01538	-0.2716075\\
-0.01529	-0.31128\\
-0.0154275	-0.234985\\
-0.0152425	-0.198365\\
-0.01497	-0.2227775\\
-0.015015	-0.1953125\\
-0.0149225	-0.2288825\\
-0.01497	-0.286865\\
-0.01529	-0.216675\\
-0.01506	-0.2563475\\
-0.015105	-0.2319325\\
-0.015105	-0.24109\\
-0.015105	-0.19226\\
-0.0149225	-0.250245\\
-0.01506	-0.17395\\
-0.0148325	-0.19226\\
-0.01474	-0.2105725\\
-0.0148325	-0.201415\\
-0.0148325	-0.10376\\
-0.0142375	-0.07019\\
-0.01355	-0.0549325\\
-0.013	-0.094605\\
-0.0130925	-0.2105725\\
-0.0142375	-0.2716075\\
-0.0149225	-0.27771\\
-0.015015	-0.1861575\\
-0.014695	-0.216675\\
-0.014695	-0.1800525\\
-0.0146475	-0.164795\\
-0.014465	-0.1068125\\
-0.0140075	-0.1525875\\
-0.0140525	-0.146485\\
-0.014145	-0.1434325\\
-0.014145	-0.1678475\\
-0.0142375	-0.1434325\\
-0.014145	-0.1342775\\
-0.0140525	-0.094605\\
-0.0137325	-0.1251225\\
-0.0137775	-0.2288825\\
-0.01451	-0.1770025\\
-0.014465	-0.216675\\
-0.0145575	-0.18921\\
-0.01451	-0.253295\\
-0.014785	-0.2319325\\
-0.0148325	-0.3234875\\
-0.0151525	-0.2288825\\
-0.01506	-0.2655025\\
-0.015015	-0.2594\\
-0.015105	-0.13733\\
-0.01451	-0.24109\\
-0.014785	-0.1678475\\
-0.0146025	-0.2105725\\
-0.014695	-0.22583\\
-0.0148325	-0.2899175\\
-0.01506	-0.2136225\\
-0.0149225	-0.27771\\
-0.01506	-0.3417975\\
-0.0154275	-0.27771\\
-0.015335	-0.2380375\\
-0.0151525	-0.18921\\
-0.0149225	-0.2288825\\
-0.0149225	-0.201415\\
-0.0149225	-0.17395\\
-0.014695	-0.14038\\
-0.01451	-0.1770025\\
-0.0145575	-0.146485\\
-0.014465	-0.2990725\\
-0.015105	-0.3784175\\
-0.015655	-0.3265375\\
-0.01561	-0.3173825\\
-0.0155175	-0.31128\\
-0.0155175	-0.17395\\
-0.01497	-0.1586925\\
-0.0146025	-0.1251225\\
-0.0143275	-0.1068125\\
-0.0141	-0.12207\\
-0.0141	-0.201415\\
-0.0145575	-0.250245\\
-0.0148775	-0.286865\\
-0.015105	-0.24109\\
-0.01506	-0.1678475\\
-0.014695	-0.29297\\
-0.0151525	-0.3448475\\
-0.0155175	-0.38147\\
-0.0157025	-0.29602\\
-0.01561	-0.46997\\
-0.0160225	-0.3326425\\
-0.0160225	-0.1953125\\
-0.01529	-0.1434325\\
-0.014695	-0.12207\\
-0.01442	-0.2197275\\
-0.014785	-0.27771\\
-0.0151975	-0.3692625\\
-0.01561	-0.2746575\\
-0.0155175	-0.2197275\\
-0.0151975	-0.1190175\\
-0.0145575	-0.1342775\\
-0.0142825	-0.14038\\
-0.0143275	-0.1586925\\
-0.014375	-0.1007075\\
-0.0141	-0.131225\\
-0.0141	-0.1007075\\
-0.0140075	-0.0671375\\
-0.013505	-0.146485\\
-0.0139625	-0.1586925\\
-0.01419	-0.201415\\
-0.01442	-0.1800525\\
-0.014465	-0.19226\\
-0.01451	-0.19226\\
-0.0145575	-0.2471925\\
-0.01474	-0.1678475\\
-0.0146025	-0.253295\\
-0.014785	-0.29297\\
-0.0151525	-0.2227775\\
-0.01497	-0.234985\\
-0.01497	-0.201415\\
-0.0148775	-0.234985\\
-0.0149225	-0.3326425\\
-0.015335	-0.250245\\
-0.0152425	-0.198365\\
-0.0149225	-0.22583\\
-0.01497	-0.2044675\\
-0.0149225	-0.14038\\
-0.0145575	-0.2136225\\
-0.01474	-0.3082275\\
-0.01529	-0.29297\\
-0.015335	-0.3265375\\
-0.0154275	-0.4821775\\
-0.0161125	-0.31433\\
-0.015885	-0.268555\\
-0.0155175	-0.20752\\
-0.01529	-0.2655025\\
-0.01538	-0.38147\\
-0.0157925	-0.253295\\
-0.0155175	-0.22583\\
-0.0152425	-0.31128\\
-0.0155175	-0.198365\\
-0.0152425	-0.1190175\\
-0.0145575	-0.15564\\
-0.0145575	-0.1159675\\
-0.01442	-0.0976575\\
-0.0140075	-0.1068125\\
-0.0139625	-0.094605\\
-0.013825	-0.094605\\
-0.0136875	-0.1190175\\
-0.013825	-0.164795\\
-0.01419	-0.1953125\\
-0.01442	-0.2563475\\
-0.0148775	-0.24109\\
-0.0149225	-0.2807625\\
-0.01506	-0.3265375\\
-0.0154275	-0.2990725\\
-0.01538	-0.216675\\
-0.015105	-0.2319325\\
-0.01506	-0.20752\\
-0.01497	-0.2288825\\
-0.015015	-0.3173825\\
-0.01538	-0.234985\\
-0.0152425	-0.2655025\\
-0.0151975	-0.2288825\\
-0.0151525	-0.2197275\\
-0.015105	-0.3509525\\
-0.0155175	-0.2838125\\
-0.0154725	-0.2899175\\
-0.0154725	-0.2594\\
-0.0154275	-0.1617425\\
-0.0149225	-0.1068125\\
-0.0143275	-0.08545\\
-0.013915	-0.1525875\\
-0.01419	-0.13733\\
-0.0142375	-0.1861575\\
-0.014465	-0.14038\\
-0.014375	-0.198365\\
-0.0145575	-0.31128\\
-0.0151975	-0.3875725\\
-0.0157475	-0.305175\\
-0.01561	-0.320435\\
-0.01561	-0.2838125\\
-0.0155175	-0.20752\\
-0.0151525	-0.2197275\\
-0.015105	-0.2471925\\
-0.0151975	-0.2105725\\
-0.0151525	-0.253295\\
-0.0151975	-0.3173825\\
-0.0154725	-0.4455575\\
-0.015975	-0.390625\\
-0.0161125	-0.3692625\\
-0.015975	-0.4119875\\
-0.0160675	-0.2746575\\
-0.0157475	-0.302125\\
-0.0157025	-0.234985\\
-0.0154725	-0.2380375\\
-0.01538	-0.3173825\\
-0.01561	-0.3601075\\
-0.0158375	-0.14038\\
-0.015015	-0.1861575\\
-0.01474	-0.1434325\\
-0.014695	-0.20752\\
-0.0148325	-0.2105725\\
-0.01497	-0.128175\\
-0.0146025	-0.1708975\\
-0.014695	-0.2990725\\
-0.01529	-0.4913325\\
-0.016205	-0.476075\\
-0.016525	-0.4364025\\
-0.0163875	-0.3448475\\
-0.01616	-0.4028325\\
-0.01625	-0.4913325\\
-0.016525	-0.3601075\\
-0.0162975	-0.372315\\
-0.016205	-0.3845225\\
-0.01625	-0.26245\\
-0.015885	-0.2288825\\
-0.015565	-0.29602\\
-0.0157475	-0.198365\\
-0.0154275	-0.149535\\
-0.01497	-0.201415\\
-0.01506	-0.149535\\
-0.0148775	-0.183105\\
-0.0148775	-0.1678475\\
-0.0148775	-0.2105725\\
-0.015015	-0.27771\\
-0.015335	-0.2136225\\
-0.0152425	-0.1617425\\
-0.0149225	-0.19226\\
-0.0149225	-0.2227775\\
-0.015105	-0.268555\\
-0.01529	-0.2227775\\
-0.0152425	-0.1861575\\
-0.015015	-0.2899175\\
-0.01538	-0.2716075\\
-0.0154725	-0.18921\\
-0.0151525	-0.3173825\\
-0.0155175	-0.2838125\\
-0.01561	-0.20752\\
-0.01529	-0.1434325\\
-0.0148775	-0.1068125\\
-0.01442	-0.079345\\
-0.0140075	-0.17395\\
-0.014465	-0.1586925\\
-0.0146025	-0.1190175\\
-0.0143275	-0.08545\\
-0.01387	-0.1068125\\
-0.01387	-0.1770025\\
-0.014375	-0.2044675\\
-0.0146475	-0.26245\\
-0.015015	-0.305175\\
-0.015335	-0.29297\\
-0.0154275	-0.305175\\
-0.0154725	-0.1800525\\
-0.01497	-0.08545\\
-0.0140075	-0.1251225\\
-0.0139625	-0.10376\\
-0.0139625	-0.1342775\\
-0.0141	-0.216675\\
-0.0146025	-0.2380375\\
-0.0148775	-0.354005\\
-0.0154725	-0.2319325\\
-0.01529	-0.2197275\\
-0.01506	-0.320435\\
-0.0154275	-0.234985\\
-0.0152425	-0.164795\\
-0.0148325	-0.1251225\\
-0.01442	-0.164795\\
-0.01451	-0.2197275\\
-0.0148325	-0.3234875\\
-0.01538	-0.216675\\
-0.0151975	-0.1861575\\
-0.0148325	-0.1800525\\
-0.014785	-0.216675\\
-0.01497	-0.2838125\\
-0.0152425	-0.4486075\\
-0.0160225	-0.3784175\\
-0.0160675	-0.3784175\\
-0.0160675	-0.24414\\
-0.015565	-0.1617425\\
-0.015015	-0.19226\\
-0.01497	-0.0915525\\
-0.014145	-0.13733\\
-0.014145	-0.128175\\
-0.01419	-0.13733\\
-0.014145	-0.234985\\
-0.014785	-0.320435\\
-0.01538	-0.36621\\
-0.0157475	-0.4730225\\
-0.01616	-0.4028325\\
-0.01625	-0.216675\\
-0.015565	-0.2044675\\
-0.0152425	-0.1708975\\
-0.015015	-0.2288825\\
-0.0151525	-0.302125\\
-0.0154725	-0.4028325\\
-0.01593	-0.2838125\\
-0.0157025	-0.201415\\
-0.01529	-0.2319325\\
-0.015335	-0.31128\\
-0.01561	-0.2197275\\
-0.01538	-0.1586925\\
-0.0149225	-0.128175\\
-0.0145575	-0.094605\\
-0.0141	-0.08545\\
-0.013825	-0.0732425\\
-0.0134575	-0.1190175\\
-0.0137325	-0.1617425\\
-0.01419	-0.183105\\
-0.01442	-0.14038\\
-0.0142825	-0.131225\\
-0.0141	-0.0671375\\
-0.0134125	-0.0396725\\
-0.0125875	-0.0579825\\
-0.0124975	-0.10376\\
-0.0130475	-0.15564\\
-0.0136875	-0.19226\\
-0.0141	-0.216675\\
-0.01442	-0.253295\\
-0.01474	-0.32959\\
-0.0152425	-0.4547125\\
-0.015975	-0.4486075\\
-0.01625	-0.48523\\
-0.0163875	-0.424195\\
-0.0163425	-0.234985\\
-0.01561	-0.216675\\
-0.01529	-0.2136225\\
-0.0152425	-0.2838125\\
-0.0155175	-0.24109\\
-0.0154725	-0.1770025\\
-0.015015	-0.14038\\
-0.014695	-0.1159675\\
-0.01442	-0.19226\\
-0.01474	-0.19226\\
-0.0148325	-0.1068125\\
-0.0142375	-0.0823975\\
-0.0137775	-0.1251225\\
-0.013915	-0.1678475\\
-0.0142825	-0.128175\\
-0.014145	-0.183105\\
-0.01442	-0.13733\\
-0.0142825	-0.2044675\\
-0.01451	-0.305175\\
-0.0152425	-0.24109\\
-0.0151525	-0.320435\\
-0.01538	-0.442505\\
-0.015975	-0.4730225\\
-0.0163425	-0.372315\\
-0.01616	-0.2380375\\
-0.015655	-0.183105\\
-0.015105	-0.112915\\
-0.014465	-0.1708975\\
-0.01451	-0.1190175\\
-0.014375	-0.2136225\\
-0.014785	-0.198365\\
-0.0148775	-0.15564\\
-0.0146475	-0.2044675\\
-0.0148325	-0.1190175\\
-0.01442	-0.17395\\
-0.01451	-0.13733\\
-0.014375	-0.08545\\
-0.01387	-0.0976575\\
-0.013825	-0.0823975\\
-0.013595	-0.094605\\
-0.01355	-0.0823975\\
-0.013505	-0.061035\\
-0.0131375	-0.079345\\
-0.0131375	-0.112915\\
-0.0134125	-0.1098625\\
-0.01355	-0.1098625\\
-0.013505	-0.1708975\\
-0.0139625	-0.22583\\
-0.01451	-0.1800525\\
-0.014465	-0.1678475\\
-0.014375	-0.26245\\
-0.014785	-0.31433\\
-0.0152425	-0.250245\\
-0.015105	-0.26245\\
-0.0151525	-0.128175\\
-0.01442	-0.0823975\\
-0.0136875	-0.15564\\
-0.0140525	-0.2288825\\
-0.0146475	-0.24109\\
-0.0148775	-0.338745\\
-0.015335	-0.302125\\
-0.0154275	-0.216675\\
-0.0151525	-0.1525875\\
-0.0146475	-0.1617425\\
-0.01451	-0.183105\\
-0.0146025	-0.14038\\
-0.01442	-0.0885\\
-0.0139625	-0.112915\\
-0.01387	-0.094605\\
-0.0137325	-0.079345\\
-0.013505	-0.0640875\\
-0.0131375	-0.112915\\
-0.0134125	-0.15564\\
-0.01387	-0.1708975\\
-0.0141	-0.1251225\\
-0.01387	-0.2044675\\
-0.0143275	-0.2899175\\
-0.015015	-0.1861575\\
-0.014695	-0.250245\\
-0.0148325	-0.2807625\\
-0.0151525	-0.20752\\
-0.0148775	-0.12207\\
-0.0143275	-0.15564\\
-0.0142375	-0.1861575\\
-0.014465	-0.112915\\
-0.0140525	-0.12207\\
-0.0139625	-0.10376\\
-0.0137775	-0.061035\\
-0.013275	-0.061035\\
-0.0130925	-0.1007075\\
-0.0133675	-0.13733\\
-0.0137325	-0.12207\\
-0.0136875	-0.149535\\
-0.01387	-0.17395\\
-0.0141	-0.1251225\\
-0.01387	-0.076295\\
-0.0134125	-0.061035\\
-0.0130475	-0.079345\\
-0.0130475	-0.0915525\\
-0.01323	-0.1098625\\
-0.0133675	-0.216675\\
-0.0142375	-0.2227775\\
-0.0146025	-0.19226\\
-0.014465	-0.1861575\\
-0.01442	-0.253295\\
-0.014695	-0.320435\\
-0.0151975	-0.3448475\\
-0.0154725	-0.3509525\\
-0.015565	-0.2594\\
-0.015335	-0.2716075\\
-0.0152425	-0.27771\\
-0.015335	-0.2807625\\
-0.015335	-0.3173825\\
-0.0154725	-0.424195\\
-0.015885	-0.3601075\\
-0.015885	-0.32959\\
-0.0157475	-0.26245\\
-0.015565	-0.253295\\
-0.0154275	-0.24109\\
-0.01538	-0.286865\\
-0.0155175	-0.1800525\\
-0.015105	-0.20752\\
-0.01497	-0.2563475\\
-0.0151975	-0.2990725\\
-0.0154275	-0.2288825\\
-0.0152425	-0.2655025\\
-0.01529	-0.2136225\\
-0.0151975	-0.18921\\
-0.015015	-0.1251225\\
-0.0145575	-0.1861575\\
-0.01474	-0.305175\\
-0.015335	-0.24109\\
-0.0152425	-0.14038\\
-0.014695	-0.17395\\
-0.0146475	-0.10376\\
-0.014145	-0.0885\\
-0.0137775	-0.1251225\\
-0.0139625	-0.079345\\
-0.0136875	-0.0885\\
-0.013505	-0.1068125\\
-0.013595	-0.07019\\
-0.013275	-0.1098625\\
-0.01355	-0.1007075\\
-0.0134575	-0.061035\\
-0.0130925	-0.0976575\\
-0.01332	-0.1159675\\
-0.0134575	-0.1342775\\
-0.0136875	-0.1342775\\
-0.0137325	-0.1342775\\
-0.0137325	-0.1861575\\
-0.0141	-0.1617425\\
-0.0141	-0.1800525\\
-0.01419	-0.31128\\
-0.015015	-0.2227775\\
-0.01497	-0.19226\\
-0.014695	-0.2136225\\
-0.014785	-0.1525875\\
-0.014465	-0.094605\\
-0.01387	-0.164795\\
-0.01419	-0.1434325\\
-0.014145	-0.0823975\\
-0.0136875	-0.146485\\
-0.01387	-0.13733\\
-0.0140075	-0.1678475\\
-0.014145	-0.112915\\
-0.013825	-0.07019\\
-0.01332	-0.0579825\\
-0.013	-0.0671375\\
-0.01291	-0.12207\\
-0.0134125	-0.12207\\
-0.013595	-0.149535\\
-0.0137325	-0.19226\\
-0.0141	-0.2563475\\
-0.0146475	-0.19226\\
-0.0145575	-0.2563475\\
-0.014785	-0.305175\\
-0.0151525	-0.26245\\
-0.015105	-0.2471925\\
-0.015015	-0.405885\\
-0.015655	-0.29602\\
-0.015565	-0.357055\\
-0.0157025	-0.3082275\\
-0.0157025	-0.36316\\
-0.0157475	-0.405885\\
-0.015975	-0.24109\\
-0.0154725	-0.146485\\
-0.014785	-0.12207\\
-0.0143275	-0.1861575\\
-0.0146475	-0.234985\\
-0.0149225	-0.2380375\\
-0.01497	-0.164795\\
-0.01474	-0.0915525\\
-0.0140075	-0.1159675\\
-0.013915	-0.2319325\\
-0.01474	-0.31433\\
-0.015335	-0.31128\\
-0.0154275	-0.1861575\\
-0.0149225	-0.1434325\\
-0.01451	-0.19226\\
-0.0146475	-0.305175\\
-0.0151975	-0.3417975\\
-0.0155175	-0.3479\\
-0.01561	-0.24414\\
-0.015335	-0.3448475\\
-0.01561	-0.38147\\
-0.0158375	-0.357055\\
-0.0158375	-0.494385\\
-0.01625	-0.424195\\
-0.0162975	-0.354005\\
-0.0161125	-0.4119875\\
-0.016205	-0.3509525\\
-0.0161125	-0.1953125\\
-0.0154275	-0.1800525\\
-0.015015	-0.2227775\\
-0.0151975	-0.2716075\\
-0.0154275	-0.27771\\
-0.0154725	-0.2044675\\
-0.0152425	-0.2563475\\
-0.015335	-0.2288825\\
-0.015335	-0.164795\\
-0.015015	-0.2227775\\
-0.015105	-0.2105725\\
-0.0151525	-0.31128\\
-0.0154725	-0.320435\\
-0.0157025	-0.2288825\\
-0.0154275	-0.1770025\\
-0.01506	-0.1068125\\
-0.014465	-0.1617425\\
-0.0146025	-0.13733\\
-0.01442	-0.1159675\\
-0.01419	-0.0976575\\
-0.0141	-0.164795\\
-0.014375	-0.216675\\
-0.01474	-0.2380375\\
-0.0149225	-0.2380375\\
-0.01497	-0.1434325\\
-0.0145575	-0.1068125\\
-0.01419	-0.0732425\\
-0.0137775	-0.094605\\
-0.0136875	-0.1586925\\
-0.014145	-0.164795\\
-0.0142825	-0.1617425\\
-0.0143275	-0.2563475\\
-0.0148325	-0.305175\\
-0.0152425	-0.3356925\\
-0.0154725	-0.338745\\
-0.01561	-0.3936775\\
-0.0158375	-0.24414\\
-0.0154275	-0.201415\\
-0.015015	-0.1525875\\
-0.0148325	-0.1678475\\
-0.014695	-0.22583\\
-0.01497	-0.18921\\
-0.0149225	-0.128175\\
-0.014465	-0.1159675\\
-0.0142375	-0.20752\\
-0.014695	-0.2563475\\
-0.01506	-0.22583\\
-0.015015	-0.3692625\\
-0.01561	-0.4028325\\
-0.01593	-0.2807625\\
-0.015655	-0.198365\\
-0.0151525	-0.13733\\
-0.014695	-0.1251225\\
-0.01442	-0.22583\\
-0.0148775	-0.19226\\
-0.0148775	-0.216675\\
-0.01497	-0.2471925\\
-0.01506	-0.268555\\
-0.0151975	-0.2594\\
-0.0152425	-0.29297\\
-0.015335	-0.2044675\\
-0.01506	-0.2227775\\
-0.015015	-0.164795\\
-0.0148325	-0.146485\\
-0.0145575	-0.22583\\
-0.0149225	-0.31433\\
-0.01538	-0.338745\\
-0.015655	-0.201415\\
-0.015105	-0.3875725\\
-0.0157475	-0.36316\\
-0.0160225	-0.24109\\
-0.0155175	-0.14038\\
-0.014785	-0.20752\\
-0.0149225	-0.18921\\
-0.01497	-0.1770025\\
-0.0148325	-0.2136225\\
-0.01497	-0.146485\\
-0.014695	-0.198365\\
-0.0148325	-0.36316\\
-0.015655	-0.3784175\\
-0.01593	-0.2471925\\
-0.0155175	-0.2716075\\
-0.0154725	-0.305175\\
-0.01561	-0.3082275\\
-0.01561	-0.22583\\
-0.0154275	-0.2197275\\
-0.0152425	-0.1525875\\
-0.0148775	-0.2227775\\
-0.015015	-0.2380375\\
-0.0151975	-0.36621\\
-0.0157025	-0.41504\\
-0.0160675	-0.5706775\\
-0.0166625	-0.408935\\
-0.016525	-0.32959\\
-0.01616	-0.2594\\
-0.0157925	-0.2746575\\
-0.0157925	-0.2136225\\
-0.015565	-0.149535\\
-0.01497	-0.1434325\\
-0.014785	-0.149535\\
-0.014785	-0.1159675\\
-0.014465	-0.0823975\\
-0.0139625	-0.128175\\
-0.0141	-0.1678475\\
-0.014465	-0.1190175\\
-0.0142825	-0.08545\\
-0.01387	-0.0823975\\
-0.0137325	-0.19226\\
-0.014375	-0.2746575\\
-0.01506	-0.131225\\
-0.014465	-0.15564\\
-0.014375	-0.1708975\\
-0.014465	-0.1525875\\
-0.014465	-0.19226\\
-0.0146025	-0.1678475\\
-0.0145575	-0.1190175\\
-0.0142375	-0.08545\\
-0.01387	-0.0732425\\
-0.013595	-0.0976575\\
-0.0136425	-0.183105\\
-0.0143275	-0.2197275\\
-0.0146475	-0.15564\\
-0.014465	-0.1098625\\
-0.0141	-0.0732425\\
-0.01355	-0.10376\\
-0.01355	-0.1617425\\
-0.0141	-0.2105725\\
-0.01451	-0.2227775\\
-0.014695	-0.1586925\\
-0.01442	-0.149535\\
-0.0142825	-0.1678475\\
-0.014375	-0.2288825\\
-0.014695	-0.3173825\\
-0.0151975	-0.36621\\
-0.01561	-0.27771\\
-0.0154725	-0.17395\\
-0.0149225	-0.18921\\
-0.014785	-0.17395\\
-0.01474	-0.1800525\\
-0.01474	-0.1251225\\
-0.01442	-0.149535\\
-0.014375	-0.1678475\\
-0.01451	-0.1586925\\
-0.01451	-0.10376\\
-0.014145	-0.0732425\\
-0.013595	-0.10376\\
-0.0136875	-0.1586925\\
-0.014145	-0.234985\\
-0.014695	-0.250245\\
-0.01497	-0.305175\\
-0.0152425	-0.2746575\\
-0.01529	-0.3082275\\
-0.015335	-0.408935\\
-0.0158375	-0.5432125\\
-0.01648	-0.4486075\\
-0.01657	-0.2990725\\
-0.0160675	-0.3265375\\
-0.01593	-0.3845225\\
-0.01616	-0.494385\\
-0.01648	-0.320435\\
-0.01616	-0.15564\\
-0.0151975	-0.1098625\\
-0.014465	-0.112915\\
-0.01419	-0.0823975\\
-0.013915	-0.0549325\\
-0.013275	-0.07019\\
-0.0131825	-0.1068125\\
-0.0134575	-0.1525875\\
-0.013915	-0.149535\\
-0.0140525	-0.1190175\\
-0.01387	-0.112915\\
-0.0137325	-0.14038\\
-0.013915	-0.17395\\
-0.01419	-0.183105\\
-0.0143275	-0.17395\\
-0.0142825	-0.12207\\
-0.0140075	-0.0885\\
-0.0136425	-0.08545\\
-0.0134575	-0.1251225\\
-0.0137325	-0.1586925\\
-0.0140075	-0.1159675\\
-0.0137775	-0.0823975\\
-0.0134125	-0.131225\\
-0.0137325	-0.1007075\\
-0.013595	-0.1098625\\
-0.013595	-0.14038\\
-0.01387	-0.198365\\
-0.0142825	-0.216675\\
-0.0145575	-0.149535\\
-0.0142375	-0.22583\\
-0.0145575	-0.2197275\\
-0.014695	-0.17395\\
-0.0145575	-0.1525875\\
-0.014375	-0.24414\\
-0.014695	-0.3082275\\
-0.0151975	-0.2990725\\
-0.01529	-0.3326425\\
-0.0154275	-0.2563475\\
-0.01529	-0.2319325\\
-0.0151525	-0.2197275\\
-0.015015	-0.29602\\
-0.01529	-0.216675\\
-0.015105	-0.302125\\
-0.015335	-0.2899175\\
-0.0154275	-0.29602\\
-0.0154725	-0.5096425\\
-0.01625	-0.4028325\\
-0.0163425	-0.2136225\\
-0.015565	-0.146485\\
-0.0148775	-0.15564\\
-0.01474	-0.198365\\
-0.0149225	-0.253295\\
-0.0151525	-0.286865\\
-0.0154275	-0.320435\\
-0.01561	-0.32959\\
-0.015655	-0.216675\\
-0.01529	-0.19226\\
-0.01506	-0.2227775\\
-0.0151975	-0.2227775\\
-0.0151975	-0.14038\\
-0.01474	-0.1007075\\
-0.0142375	-0.1770025\\
-0.01451	-0.1770025\\
-0.0146475	-0.234985\\
-0.01497	-0.2899175\\
-0.01538	-0.268555\\
-0.01529	-0.1861575\\
-0.015015	-0.1525875\\
-0.014695	-0.22583\\
-0.01497	-0.29602\\
-0.015335	-0.3845225\\
-0.0157925	-0.424195\\
-0.0161125	-0.48523\\
-0.0163425	-0.3936775\\
-0.01625	-0.354005\\
-0.0161125	-0.2105725\\
-0.015565	-0.15564\\
-0.0149225	-0.26245\\
-0.015335	-0.3479\\
-0.0157475	-0.2044675\\
-0.01538	-0.31433\\
-0.01561	-0.4302975\\
-0.01616	-0.5157475\\
-0.016525	-0.3234875\\
-0.01616	-0.1861575\\
-0.0154275	-0.1068125\\
-0.0145575	-0.1617425\\
-0.01451	-0.149535\\
-0.0146025	-0.15564\\
-0.0146025	-0.094605\\
-0.01419	-0.131225\\
-0.01419	-0.10376\\
-0.0140075	-0.131225\\
-0.0141	-0.1098625\\
-0.0140525	-0.13733\\
-0.014145	-0.2136225\\
-0.0146475	-0.198365\\
-0.014785	-0.2807625\\
-0.015105	-0.3326425\\
-0.0155175	-0.3326425\\
-0.01561	-0.1861575\\
-0.01506	-0.19226\\
-0.0148325	-0.2288825\\
-0.01506	-0.15564\\
-0.01474	-0.14038\\
-0.01451	-0.1068125\\
-0.01419	-0.1586925\\
-0.01442	-0.149535\\
-0.014465	-0.08545\\
-0.0139625	-0.061035\\
-0.01332	-0.061035\\
-0.013	-0.10376\\
-0.0133675	-0.1525875\\
-0.01387	-0.164795\\
-0.0140525	-0.10376\\
-0.0137775	-0.128175\\
-0.0137775	-0.183105\\
-0.01419	-0.234985\\
-0.0146475	-0.164795\\
-0.01451	-0.15564\\
-0.0143275	-0.1861575\\
-0.014465	-0.2197275\\
-0.0146475	-0.201415\\
-0.0146475	-0.2319325\\
-0.014785	-0.216675\\
-0.014785	-0.15564\\
-0.014465	-0.12207\\
-0.01419	-0.15564\\
-0.0142825	-0.1342775\\
-0.01419	-0.1617425\\
-0.0143275	-0.2197275\\
-0.014695	-0.3479\\
-0.015335	-0.2471925\\
-0.0152425	-0.2380375\\
-0.01506	-0.3509525\\
-0.015565	-0.2716075\\
-0.0154275	-0.1678475\\
-0.0148775	-0.12207\\
-0.01442	-0.1007075\\
-0.0141	-0.1068125\\
-0.0140075	-0.0885\\
-0.0137775	-0.0915525\\
-0.0136425	-0.1678475\\
-0.01419	-0.0885\\
-0.0136875	-0.08545\\
-0.013275	-0.1190175\\
-0.0136425	-0.164795\\
-0.0140075	-0.1251225\\
-0.0139625	-0.094605\\
-0.0136425	-0.0579825\\
-0.0131825	-0.0976575\\
-0.01332	-0.1800525\\
-0.0141	-0.24109\\
-0.0146475	-0.198365\\
-0.0145575	-0.1586925\\
-0.0143275	-0.1800525\\
-0.014375	-0.1800525\\
-0.014375	-0.201415\\
-0.01451	-0.2471925\\
-0.01474	-0.29297\\
-0.01506	-0.286865\\
-0.0151975	-0.2380375\\
-0.01506	-0.1617425\\
-0.014695	-0.131225\\
-0.0143275	-0.14038\\
-0.0142825	-0.19226\\
-0.0145575	-0.1190175\\
-0.0142375	-0.1098625\\
-0.0139625	-0.201415\\
-0.01442	-0.24109\\
-0.0148325	-0.2288825\\
-0.0148325	-0.26245\\
-0.01497	-0.3448475\\
-0.01538	-0.1861575\\
-0.0149225	-0.3479\\
-0.01538	-0.3234875\\
-0.015655	-0.302125\\
-0.015565	-0.320435\\
-0.015565	-0.31433\\
-0.01561	-0.531005\\
-0.0163425	-0.4974375\\
-0.0166625	-0.3356925\\
-0.01625	-0.39978\\
-0.0162975	-0.48523\\
-0.01657	-0.476075\\
-0.0166175	-0.531005\\
-0.016755	-0.48523\\
-0.0168	-0.634765\\
-0.017075	-0.4882825\\
-0.0169825	-0.338745\\
-0.016525	-0.2044675\\
-0.0158375	-0.2105725\\
-0.0155175	-0.1708975\\
-0.015335	-0.1434325\\
-0.0149225	-0.2136225\\
-0.0151975	-0.302125\\
-0.015655	-0.19226\\
-0.015335	-0.1678475\\
-0.015015	-0.164795\\
-0.015015	-0.1159675\\
-0.01451	-0.14038\\
-0.01451	-0.0732425\\
-0.01387	-0.0976575\\
-0.0136425	-0.0732425\\
-0.01355	-0.1159675\\
-0.0136875	-0.0823975\\
-0.013595	-0.076295\\
-0.01332	-0.0488275\\
-0.012955	-0.0457775\\
-0.01268	-0.08545\\
-0.01291	-0.0640875\\
-0.0128625	-0.07019\\
-0.0127725	-0.1434325\\
-0.013505	-0.1953125\\
-0.0141	-0.201415\\
-0.014375	-0.146485\\
-0.0141	-0.1861575\\
-0.0142375	-0.286865\\
-0.0149225	-0.1342775\\
-0.0143275	-0.094605\\
-0.0136425	-0.07019\\
-0.01323	-0.05188\\
-0.0128175	-0.08545\\
-0.0130475	-0.14038\\
-0.013595	-0.2105725\\
-0.0142375	-0.1251225\\
-0.0140075	-0.2136225\\
-0.014375	-0.29602\\
-0.01506	-0.302125\\
-0.0151975	-0.2227775\\
-0.015015	-0.1434325\\
-0.014465	-0.1800525\\
-0.014465	-0.24109\\
-0.0148325	-0.31128\\
-0.0151975	-0.1708975\\
-0.014785	-0.094605\\
-0.0139625	-0.1434325\\
-0.014145	-0.1190175\\
-0.0139625	-0.0579825\\
-0.01332	-0.0457775\\
-0.012635	-0.0976575\\
-0.013	-0.2136225\\
-0.01419	-0.2136225\\
-0.0145575	-0.1586925\\
-0.0142825	-0.10376\\
-0.013825	-0.1098625\\
-0.0136425	-0.0396725\\
-0.01268	-0.076295\\
-0.0125875	-0.0640875\\
-0.01268	-0.1068125\\
-0.012955	-0.146485\\
-0.013505	-0.2227775\\
-0.01419	-0.234985\\
-0.0146025	-0.250245\\
-0.014695	-0.26245\\
-0.0148775	-0.2563475\\
-0.0148775	-0.250245\\
-0.0148775	-0.216675\\
-0.014785	-0.26245\\
-0.01497	-0.3936775\\
-0.015565	-0.3601075\\
-0.0157925	-0.1861575\\
-0.01506	-0.1007075\\
-0.0140525	-0.216675\\
-0.0146475	-0.2655025\\
-0.015105	-0.24109\\
-0.01506	-0.14038\\
-0.01451	-0.1434325\\
-0.0142825	-0.24414\\
-0.0148325	-0.36621\\
-0.0155175	-0.375365\\
-0.0158375	-0.424195\\
-0.0160225	-0.4119875\\
-0.016205	-0.302125\\
-0.015885	-0.3082275\\
-0.0157925	-0.146485\\
-0.01497	-0.14038\\
-0.014465	-0.1800525\\
-0.0146475	-0.2197275\\
-0.0148775	-0.2288825\\
-0.015015	-0.24109\\
-0.01506	-0.1586925\\
-0.014695	-0.216675\\
-0.0148325	-0.17395\\
-0.014785	-0.10376\\
-0.0142825	-0.0732425\\
-0.013595	-0.061035\\
-0.0131825	-0.094605\\
-0.0134125	-0.076295\\
-0.013275	-0.0457775\\
-0.0128175	-0.0976575\\
-0.0130925	-0.1861575\\
-0.013915	-0.12207\\
-0.013825	-0.112915\\
-0.013595	-0.1525875\\
-0.01387	-0.250245\\
-0.0145575	-0.1708975\\
-0.01451	-0.094605\\
-0.0137775	-0.061035\\
-0.0130925	-0.1342775\\
-0.013595	-0.2716075\\
-0.014695	-0.3417975\\
-0.01538	-0.4730225\\
-0.0160675	-0.5584725\\
-0.0166625	-0.5523675\\
-0.016845	-0.357055\\
-0.0163875	-0.3692625\\
-0.01625	-0.479125\\
-0.0166175	-0.39978\\
-0.016525	-0.3692625\\
-0.0163875	-0.4211425\\
-0.01648	-0.4577625\\
-0.0166175	-0.4638675\\
-0.0166625	-0.634765\\
-0.01712	-0.4211425\\
-0.016845	-0.2380375\\
-0.0160675	-0.14038\\
-0.0151525	-0.112915\\
-0.01451	-0.1251225\\
-0.01442	-0.12207\\
-0.0143275	-0.12207\\
-0.0143275	-0.216675\\
-0.0148325	-0.164795\\
-0.0148775	-0.183105\\
-0.01474	-0.2319325\\
-0.01506	-0.1342775\\
-0.0146475	-0.1617425\\
-0.01451	-0.2105725\\
-0.0148325	-0.2471925\\
-0.0151525	-0.14038\\
-0.0146475	-0.0976575\\
-0.014145	-0.1251225\\
-0.014145	-0.12207\\
-0.014145	-0.2838125\\
-0.01506	-0.3875725\\
-0.01593	-0.36316\\
-0.015975	-0.4211425\\
-0.016205	-0.4730225\\
-0.0164325	-0.616455\\
-0.0169825	-0.52185\\
-0.017075	-0.338745\\
-0.01657	-0.2227775\\
-0.015885	-0.128175\\
-0.01497	-0.146485\\
-0.0146475	-0.146485\\
-0.014695	-0.19226\\
-0.0148775	-0.1342775\\
-0.014695	-0.1251225\\
-0.014465	-0.1342775\\
-0.014465	-0.17395\\
-0.0146475	-0.19226\\
-0.014785	-0.2380375\\
-0.01506	-0.1708975\\
-0.0148775	-0.0823975\\
-0.014145	-0.0640875\\
-0.013595	-0.0579825\\
-0.0131375	-0.0640875\\
-0.013	-0.076295\\
-0.0130475	-0.1342775\\
-0.013595	-0.131225\\
-0.0137775	-0.164795\\
-0.0140075	-0.2380375\\
-0.0146025	-0.1708975\\
-0.01451	-0.2838125\\
-0.01497	-0.4119875\\
-0.015885	-0.3234875\\
-0.015885	-0.2899175\\
-0.0157025	-0.3509525\\
-0.015885	-0.405885\\
-0.01616	-0.2716075\\
-0.0157925	-0.234985\\
-0.0155175	-0.149535\\
-0.01497	-0.164795\\
-0.0148325	-0.32959\\
-0.015565	-0.53711\\
-0.0166175	-0.64087\\
-0.0172575	-0.5065925\\
-0.017165	-0.4211425\\
-0.016845	-0.3082275\\
-0.0164325	-0.3326425\\
-0.0162975	-0.4394525\\
-0.0166625	-0.234985\\
-0.0161125	-0.20752\\
-0.01561	-0.1586925\\
-0.015335	-0.2899175\\
-0.0157475	-0.2807625\\
-0.015975	-0.1525875\\
-0.01529	-0.15564\\
-0.01497	-0.1159675\\
-0.014695	-0.2105725\\
-0.015015	-0.2990725\\
-0.01561	-0.3173825\\
-0.0158375	-0.2380375\\
-0.01561	-0.17395\\
-0.0151975	-0.201415\\
-0.0151975	-0.1525875\\
-0.015015	-0.1159675\\
-0.0146475	-0.0915525\\
-0.0142375	-0.0823975\\
-0.0139625	-0.0732425\\
-0.0136425	-0.08545\\
-0.01355	-0.149535\\
-0.0140075	-0.2105725\\
-0.0146025	-0.20752\\
-0.014785	-0.131225\\
-0.01442	-0.1190175\\
-0.01419	-0.1007075\\
-0.0139625	-0.131225\\
-0.0140075	-0.1708975\\
-0.014375	-0.183105\\
-0.01451	-0.17395\\
-0.01451	-0.1007075\\
-0.0140525	-0.20752\\
-0.0146025	-0.302125\\
-0.015335	-0.2105725\\
-0.0151525	-0.0885\\
-0.014145	-0.1007075\\
-0.013915	-0.1678475\\
-0.0143275	-0.1678475\\
-0.014465	-0.146485\\
-0.014375	-0.094605\\
-0.013915	-0.1708975\\
-0.0142375	-0.24414\\
-0.0148775	-0.149535\\
-0.0145575	-0.061035\\
-0.013595	-0.094605\\
-0.0134125	-0.2044675\\
-0.0143275	-0.24109\\
-0.0148775	-0.149535\\
-0.0145575	-0.1342775\\
-0.01419	-0.24414\\
-0.01474	-0.31433\\
-0.015335	-0.2716075\\
-0.01538	-0.38147\\
-0.0157475	-0.460815\\
-0.0162975	-0.2899175\\
-0.015975	-0.2380375\\
-0.015565	-0.3479\\
-0.015885	-0.234985\\
-0.015655	-0.3173825\\
-0.0157475	-0.38147\\
-0.0161125	-0.2990725\\
-0.015885	-0.1434325\\
-0.0151525	-0.253295\\
-0.01529	-0.4302975\\
-0.01616	-0.4882825\\
-0.01657	-0.357055\\
-0.0163425	-0.302125\\
-0.0161125	-0.3784175\\
-0.0162975	-0.29602\\
-0.0161125	-0.19226\\
-0.015565	-0.183105\\
-0.0152425	-0.2288825\\
-0.01538	-0.2380375\\
-0.0154725	-0.234985\\
-0.0154725	-0.26245\\
-0.015565	-0.1800525\\
-0.015335	-0.198365\\
-0.0151975	-0.2746575\\
-0.015565	-0.1617425\\
-0.0151975	-0.1342775\\
-0.01474	-0.1098625\\
-0.014465	-0.0885\\
-0.0141	-0.1068125\\
-0.0140525	-0.183105\\
-0.0145575	-0.164795\\
-0.0146475	-0.234985\\
-0.0149225	-0.286865\\
-0.0154275	-0.3479\\
-0.0157475	-0.2563475\\
-0.015565	-0.2319325\\
-0.015335	-0.1098625\\
-0.0145575	-0.14038\\
-0.014465	-0.268555\\
-0.0152425	-0.183105\\
-0.01506	-0.1068125\\
-0.0143275	-0.1953125\\
-0.014695	-0.2899175\\
-0.01538	-0.302125\\
-0.015565	-0.3936775\\
-0.01593	-0.512695\\
-0.01657	-0.3417975\\
-0.0163425	-0.29602\\
-0.0160225	-0.357055\\
-0.01616	-0.2319325\\
-0.0157925	-0.19226\\
-0.01538	-0.2136225\\
-0.0154725	-0.27771\\
-0.015655	-0.29602\\
-0.0158375	-0.2990725\\
-0.0158375	-0.1678475\\
-0.015335	-0.149535\\
-0.01497	-0.131225\\
-0.01474	-0.183105\\
-0.0148775	-0.164795\\
-0.0149225	-0.131225\\
-0.014695	-0.26245\\
-0.01529	-0.27771\\
-0.015565	-0.1953125\\
-0.01529	-0.1617425\\
-0.01497	-0.24109\\
-0.0151975	-0.1342775\\
-0.0148775	-0.0823975\\
-0.01419	-0.10376\\
-0.0140525	-0.1434325\\
-0.0143275	-0.12207\\
-0.0142825	-0.0885\\
-0.0139625	-0.0549325\\
-0.01332	-0.1007075\\
-0.0134575	-0.13733\\
-0.013915	-0.216675\\
-0.0146025	-0.24109\\
-0.015015	-0.3173825\\
-0.0154275	-0.357055\\
-0.0157925	-0.2044675\\
-0.015335	-0.15564\\
-0.014785	-0.320435\\
-0.0154725	-0.4547125\\
-0.0162975	-0.3479\\
-0.016205	-0.3265375\\
-0.0160675	-0.4913325\\
-0.01648	-0.3875725\\
-0.0164325	-0.320435\\
-0.01616	-0.234985\\
-0.0157925	-0.2471925\\
-0.0157025	-0.3265375\\
-0.01593	-0.2136225\\
-0.01561	-0.1861575\\
-0.015335	-0.320435\\
-0.0157925	-0.3967275\\
-0.01625	-0.424195\\
-0.0163875	-0.183105\\
-0.01561	-0.0976575\\
-0.0145575	-0.234985\\
-0.0152425	-0.183105\\
-0.0151975	-0.0885\\
-0.0142825	-0.0732425\\
-0.0137325	-0.1251225\\
-0.0139625	-0.1678475\\
-0.014465	-0.2746575\\
-0.0151525	-0.19226\\
-0.01506	-0.1525875\\
-0.014695	-0.0915525\\
-0.01419	-0.0671375\\
-0.01355	-0.0915525\\
-0.01355	-0.076295\\
-0.013505	-0.1007075\\
-0.013595	-0.1708975\\
-0.0142375	-0.1525875\\
-0.014375	-0.0823975\\
-0.0137325	-0.076295\\
-0.0134125	-0.0976575\\
-0.01355	-0.1190175\\
-0.0137775	-0.0732425\\
-0.0134575	-0.1190175\\
-0.0136425	-0.0823975\\
-0.013505	-0.061035\\
-0.0131375	-0.14038\\
-0.0136875	-0.1953125\\
-0.0143275	-0.2899175\\
-0.01506	-0.390625\\
-0.0157925	-0.3448475\\
-0.015885	-0.1678475\\
-0.0151525	-0.128175\\
-0.014465	-0.128175\\
-0.014375	-0.0732425\\
-0.0137775	-0.076295\\
-0.0134125	-0.1434325\\
-0.013915	-0.15564\\
-0.01419	-0.1434325\\
-0.014145	-0.164795\\
-0.0142375	-0.19226\\
-0.01442	-0.201415\\
-0.0146475	-0.20752\\
-0.014695	-0.2594\\
-0.01497	-0.250245\\
-0.015105	-0.36316\\
-0.01561	-0.19226\\
-0.015105	-0.094605\\
-0.014145	-0.0915525\\
-0.013825	-0.1678475\\
-0.0142825	-0.1251225\\
-0.0142825	-0.1190175\\
-0.0140525	-0.0579825\\
-0.0134125	-0.10376\\
-0.0133675	-0.0640875\\
-0.0131825	-0.0396725\\
-0.012635	-0.0640875\\
-0.01268	-0.1190175\\
-0.013275	-0.146485\\
-0.0136875	-0.2105725\\
-0.0142825	-0.29602\\
-0.01506	-0.2105725\\
-0.01497	-0.0885\\
-0.0139625	-0.164795\\
-0.01419	-0.1861575\\
-0.01451	-0.0976575\\
-0.01387	-0.128175\\
-0.01387	-0.2319325\\
-0.0146025	-0.2044675\\
-0.01474	-0.3448475\\
-0.01538	-0.3936775\\
-0.015885	-0.268555\\
-0.01561	-0.20752\\
-0.0151975	-0.1861575\\
-0.01497	-0.24414\\
-0.0151525	-0.3326425\\
-0.01561	-0.234985\\
-0.01538	-0.15564\\
-0.0148775	-0.1800525\\
};
\addplot [color=mycolor2, line width=2.0pt, forget plot]
  table[row sep=crcr]{%
-0.015655	-0.015655\\
-0.015885	-0.015885\\
-0.0157475	-0.0157475\\
-0.01497	-0.01497\\
-0.0148325	-0.0148325\\
-0.015015	-0.015015\\
-0.0148775	-0.0148775\\
-0.0140075	-0.0140075\\
-0.01323	-0.01323\\
-0.0128625	-0.0128625\\
-0.0137325	-0.0137325\\
-0.014695	-0.014695\\
-0.015015	-0.015015\\
-0.015105	-0.015105\\
-0.0148775	-0.0148775\\
-0.0142825	-0.0142825\\
-0.0148775	-0.0148775\\
-0.014785	-0.014785\\
-0.014375	-0.014375\\
-0.0148325	-0.0148325\\
-0.0148775	-0.0148775\\
-0.0146475	-0.0146475\\
-0.015105	-0.015105\\
-0.0151975	-0.0151975\\
-0.0149225	-0.0149225\\
-0.0151975	-0.0151975\\
-0.015335	-0.015335\\
-0.0151525	-0.0151525\\
-0.0149225	-0.0149225\\
-0.0146475	-0.0146475\\
-0.014695	-0.014695\\
-0.0148775	-0.0148775\\
-0.0149225	-0.0149225\\
-0.01497	-0.01497\\
-0.015015	-0.015015\\
-0.0154275	-0.0154275\\
-0.0155175	-0.0155175\\
-0.0151525	-0.0151525\\
-0.0149225	-0.0149225\\
-0.014375	-0.014375\\
-0.0141	-0.0141\\
-0.014375	-0.014375\\
-0.0142375	-0.0142375\\
-0.014145	-0.014145\\
-0.014465	-0.014465\\
-0.014695	-0.014695\\
-0.0151975	-0.0151975\\
-0.015105	-0.015105\\
-0.01451	-0.01451\\
-0.014145	-0.014145\\
-0.0142375	-0.0142375\\
-0.014465	-0.014465\\
-0.0139625	-0.0139625\\
-0.0137325	-0.0137325\\
-0.0140075	-0.0140075\\
-0.013915	-0.013915\\
-0.0136875	-0.0136875\\
-0.01387	-0.01387\\
-0.0140075	-0.0140075\\
-0.0146025	-0.0146025\\
-0.014695	-0.014695\\
-0.0149225	-0.0149225\\
-0.01497	-0.01497\\
-0.01506	-0.01506\\
-0.01497	-0.01497\\
-0.0148775	-0.0148775\\
-0.014785	-0.014785\\
-0.01474	-0.01474\\
-0.014695	-0.014695\\
-0.0151975	-0.0151975\\
-0.01593	-0.01593\\
-0.01625	-0.01625\\
-0.0162975	-0.0162975\\
-0.0158375	-0.0158375\\
-0.0160675	-0.0160675\\
-0.01648	-0.01648\\
-0.0166625	-0.0166625\\
-0.01648	-0.01648\\
-0.0166625	-0.0166625\\
-0.0172125	-0.0172125\\
-0.01703	-0.01703\\
-0.0167075	-0.0167075\\
-0.01616	-0.01616\\
-0.015655	-0.015655\\
-0.015335	-0.015335\\
-0.01538	-0.01538\\
-0.01506	-0.01506\\
-0.014465	-0.014465\\
-0.014145	-0.014145\\
-0.0142825	-0.0142825\\
-0.014695	-0.014695\\
-0.01474	-0.01474\\
-0.015015	-0.015015\\
-0.0151975	-0.0151975\\
-0.01561	-0.01561\\
-0.015655	-0.015655\\
-0.0157475	-0.0157475\\
-0.015565	-0.015565\\
-0.01529	-0.01529\\
-0.01538	-0.01538\\
-0.0151525	-0.0151525\\
-0.0149225	-0.0149225\\
-0.01506	-0.01506\\
-0.015105	-0.015105\\
-0.0148325	-0.0148325\\
-0.01474	-0.01474\\
-0.014465	-0.014465\\
-0.0143275	-0.0143275\\
-0.01442	-0.01442\\
-0.01419	-0.01419\\
-0.01442	-0.01442\\
-0.01497	-0.01497\\
-0.015105	-0.015105\\
-0.01529	-0.01529\\
-0.0148325	-0.0148325\\
-0.01497	-0.01497\\
-0.01561	-0.01561\\
-0.015565	-0.015565\\
-0.01561	-0.01561\\
-0.015655	-0.015655\\
-0.0151975	-0.0151975\\
-0.0146025	-0.0146025\\
-0.01474	-0.01474\\
-0.0149225	-0.0149225\\
-0.0155175	-0.0155175\\
-0.0160225	-0.0160225\\
-0.01616	-0.01616\\
-0.0158375	-0.0158375\\
-0.0157475	-0.0157475\\
-0.015655	-0.015655\\
-0.0152425	-0.0152425\\
-0.01497	-0.01497\\
-0.014785	-0.014785\\
-0.0146025	-0.0146025\\
-0.0146475	-0.0146475\\
-0.015015	-0.015015\\
-0.015565	-0.015565\\
-0.01561	-0.01561\\
-0.0151525	-0.0151525\\
-0.0148775	-0.0148775\\
-0.01474	-0.01474\\
-0.0142375	-0.0142375\\
-0.0136875	-0.0136875\\
-0.0136425	-0.0136425\\
-0.0141	-0.0141\\
-0.0140525	-0.0140525\\
-0.014145	-0.014145\\
-0.013825	-0.013825\\
-0.0134575	-0.0134575\\
-0.0130475	-0.0130475\\
-0.01323	-0.01323\\
-0.01419	-0.01419\\
-0.01497	-0.01497\\
-0.01529	-0.01529\\
-0.015015	-0.015015\\
-0.01451	-0.01451\\
-0.0141	-0.0141\\
-0.0136425	-0.0136425\\
-0.0141	-0.0141\\
-0.0142825	-0.0142825\\
-0.0146025	-0.0146025\\
-0.01506	-0.01506\\
-0.01593	-0.01593\\
-0.016525	-0.016525\\
-0.01648	-0.01648\\
-0.0163425	-0.0163425\\
-0.015885	-0.015885\\
-0.0157925	-0.0157925\\
-0.01593	-0.01593\\
-0.0160225	-0.0160225\\
-0.0158375	-0.0158375\\
-0.015565	-0.015565\\
-0.0154725	-0.0154725\\
-0.0155175	-0.0155175\\
-0.01529	-0.01529\\
-0.015655	-0.015655\\
-0.0161125	-0.0161125\\
-0.0158375	-0.0158375\\
-0.01529	-0.01529\\
-0.01506	-0.01506\\
-0.0155175	-0.0155175\\
-0.015975	-0.015975\\
-0.0162975	-0.0162975\\
-0.016525	-0.016525\\
-0.0166625	-0.0166625\\
-0.0163875	-0.0163875\\
-0.016205	-0.016205\\
-0.0162975	-0.0162975\\
-0.016525	-0.016525\\
-0.0160225	-0.0160225\\
-0.015335	-0.015335\\
-0.0155175	-0.0155175\\
-0.0154275	-0.0154275\\
-0.0148775	-0.0148775\\
-0.0149225	-0.0149225\\
-0.015105	-0.015105\\
-0.0149225	-0.0149225\\
-0.014695	-0.014695\\
-0.014785	-0.014785\\
-0.014695	-0.014695\\
-0.014785	-0.014785\\
-0.01529	-0.01529\\
-0.015335	-0.015335\\
-0.015015	-0.015015\\
-0.0148775	-0.0148775\\
-0.0152425	-0.0152425\\
-0.0160225	-0.0160225\\
-0.01538	-0.01538\\
-0.01497	-0.01497\\
-0.0149225	-0.0149225\\
-0.0148775	-0.0148775\\
-0.01451	-0.01451\\
-0.014465	-0.014465\\
-0.0146025	-0.0146025\\
-0.0148325	-0.0148325\\
-0.01506	-0.01506\\
-0.014785	-0.014785\\
-0.015105	-0.015105\\
-0.0155175	-0.0155175\\
-0.01529	-0.01529\\
-0.01561	-0.01561\\
-0.0154275	-0.0154275\\
-0.014465	-0.014465\\
-0.0142825	-0.0142825\\
-0.014465	-0.014465\\
-0.014785	-0.014785\\
-0.01451	-0.01451\\
-0.014785	-0.014785\\
-0.0148325	-0.0148325\\
-0.01451	-0.01451\\
-0.0146475	-0.0146475\\
-0.0146025	-0.0146025\\
-0.01442	-0.01442\\
-0.0145575	-0.0145575\\
-0.01419	-0.01419\\
-0.0141	-0.0141\\
-0.01387	-0.01387\\
-0.0137775	-0.0137775\\
-0.014375	-0.014375\\
-0.0146475	-0.0146475\\
-0.015015	-0.015015\\
-0.015565	-0.015565\\
-0.015335	-0.015335\\
-0.014695	-0.014695\\
-0.0142375	-0.0142375\\
-0.0140525	-0.0140525\\
-0.014375	-0.014375\\
-0.0141	-0.0141\\
-0.0139625	-0.0139625\\
-0.01419	-0.01419\\
-0.0148325	-0.0148325\\
-0.01497	-0.01497\\
-0.01538	-0.01538\\
-0.0151975	-0.0151975\\
-0.0148775	-0.0148775\\
-0.0142825	-0.0142825\\
-0.0139625	-0.0139625\\
-0.01355	-0.01355\\
-0.0134125	-0.0134125\\
-0.013505	-0.013505\\
-0.0140075	-0.0140075\\
-0.0142825	-0.0142825\\
-0.01442	-0.01442\\
-0.0145575	-0.0145575\\
-0.015105	-0.015105\\
-0.015335	-0.015335\\
-0.015015	-0.015015\\
-0.01506	-0.01506\\
-0.0152425	-0.0152425\\
-0.015565	-0.015565\\
-0.0154725	-0.0154725\\
-0.015015	-0.015015\\
-0.01419	-0.01419\\
-0.01355	-0.01355\\
-0.0133675	-0.0133675\\
-0.0134575	-0.0134575\\
-0.013505	-0.013505\\
-0.0134575	-0.0134575\\
-0.013595	-0.013595\\
-0.01419	-0.01419\\
-0.0143275	-0.0143275\\
-0.0142825	-0.0142825\\
-0.014465	-0.014465\\
-0.014145	-0.014145\\
-0.013505	-0.013505\\
-0.013825	-0.013825\\
-0.014465	-0.014465\\
-0.0146025	-0.0146025\\
-0.014785	-0.014785\\
-0.014695	-0.014695\\
-0.014375	-0.014375\\
-0.0145575	-0.0145575\\
-0.015015	-0.015015\\
-0.0151975	-0.0151975\\
-0.0149225	-0.0149225\\
-0.015335	-0.015335\\
-0.01529	-0.01529\\
-0.015105	-0.015105\\
-0.015335	-0.015335\\
-0.01538	-0.01538\\
-0.0151525	-0.0151525\\
-0.01497	-0.01497\\
-0.01538	-0.01538\\
-0.0160225	-0.0160225\\
-0.01593	-0.01593\\
-0.01529	-0.01529\\
-0.01497	-0.01497\\
-0.015015	-0.015015\\
-0.0151525	-0.0151525\\
-0.015335	-0.015335\\
-0.0151525	-0.0151525\\
-0.0154275	-0.0154275\\
-0.015565	-0.015565\\
-0.0152425	-0.0152425\\
-0.01497	-0.01497\\
-0.015105	-0.015105\\
-0.0157925	-0.0157925\\
-0.015885	-0.015885\\
-0.0158375	-0.0158375\\
-0.01538	-0.01538\\
-0.015015	-0.015015\\
-0.0149225	-0.0149225\\
-0.014465	-0.014465\\
-0.013825	-0.013825\\
-0.0133675	-0.0133675\\
-0.0130475	-0.0130475\\
-0.013595	-0.013595\\
-0.01419	-0.01419\\
-0.0145575	-0.0145575\\
-0.01442	-0.01442\\
-0.014375	-0.014375\\
-0.0145575	-0.0145575\\
-0.01419	-0.01419\\
-0.0140525	-0.0140525\\
-0.0137775	-0.0137775\\
-0.01387	-0.01387\\
-0.014375	-0.014375\\
-0.0148325	-0.0148325\\
-0.014465	-0.014465\\
-0.0140525	-0.0140525\\
-0.01387	-0.01387\\
-0.0139625	-0.0139625\\
-0.0136425	-0.0136425\\
-0.0142375	-0.0142375\\
-0.0151525	-0.0151525\\
-0.0151975	-0.0151975\\
-0.0154275	-0.0154275\\
-0.0157925	-0.0157925\\
-0.015975	-0.015975\\
-0.0157025	-0.0157025\\
-0.01529	-0.01529\\
-0.0151975	-0.0151975\\
-0.0151525	-0.0151525\\
-0.01529	-0.01529\\
-0.0155175	-0.0155175\\
-0.015565	-0.015565\\
-0.01593	-0.01593\\
-0.01625	-0.01625\\
-0.01657	-0.01657\\
-0.0162975	-0.0162975\\
-0.0164325	-0.0164325\\
-0.01625	-0.01625\\
-0.0162975	-0.0162975\\
-0.016205	-0.016205\\
-0.015565	-0.015565\\
-0.0148325	-0.0148325\\
-0.0146025	-0.0146025\\
-0.0148775	-0.0148775\\
-0.014785	-0.014785\\
-0.014375	-0.014375\\
-0.01451	-0.01451\\
-0.0143275	-0.0143275\\
-0.0139625	-0.0139625\\
-0.0140075	-0.0140075\\
-0.0140525	-0.0140525\\
-0.01387	-0.01387\\
-0.0142375	-0.0142375\\
-0.01474	-0.01474\\
-0.0146025	-0.0146025\\
-0.015105	-0.015105\\
-0.015885	-0.015885\\
-0.0160225	-0.0160225\\
-0.0163425	-0.0163425\\
-0.0160675	-0.0160675\\
-0.0160225	-0.0160225\\
-0.015655	-0.015655\\
-0.01497	-0.01497\\
-0.0149225	-0.0149225\\
-0.014785	-0.014785\\
-0.014375	-0.014375\\
-0.014465	-0.014465\\
-0.01419	-0.01419\\
-0.013505	-0.013505\\
-0.0134125	-0.0134125\\
-0.0141	-0.0141\\
-0.0146475	-0.0146475\\
-0.01497	-0.01497\\
-0.015105	-0.015105\\
-0.01506	-0.01506\\
-0.0151975	-0.0151975\\
-0.015105	-0.015105\\
-0.0148775	-0.0148775\\
-0.0148325	-0.0148325\\
-0.0146475	-0.0146475\\
-0.01497	-0.01497\\
-0.01474	-0.01474\\
-0.01442	-0.01442\\
-0.01474	-0.01474\\
-0.0151525	-0.0151525\\
-0.015015	-0.015015\\
-0.014695	-0.014695\\
-0.01451	-0.01451\\
-0.0145575	-0.0145575\\
-0.0142375	-0.0142375\\
-0.0140075	-0.0140075\\
-0.0142375	-0.0142375\\
-0.014465	-0.014465\\
-0.0146475	-0.0146475\\
-0.0145575	-0.0145575\\
-0.014695	-0.014695\\
-0.0142375	-0.0142375\\
-0.0141	-0.0141\\
-0.014695	-0.014695\\
-0.01538	-0.01538\\
-0.0155175	-0.0155175\\
-0.015335	-0.015335\\
-0.0152425	-0.0152425\\
-0.01506	-0.01506\\
-0.0148775	-0.0148775\\
-0.014695	-0.014695\\
-0.0148325	-0.0148325\\
-0.0148775	-0.0148775\\
-0.01442	-0.01442\\
-0.0146025	-0.0146025\\
-0.014785	-0.014785\\
-0.01497	-0.01497\\
-0.0151975	-0.0151975\\
-0.015655	-0.015655\\
-0.0161125	-0.0161125\\
-0.0164325	-0.0164325\\
-0.0161125	-0.0161125\\
-0.01529	-0.01529\\
-0.0145575	-0.0145575\\
-0.0140075	-0.0140075\\
-0.0141	-0.0141\\
-0.0140525	-0.0140525\\
-0.01451	-0.01451\\
-0.0146025	-0.0146025\\
-0.014465	-0.014465\\
-0.014695	-0.014695\\
-0.0149225	-0.0149225\\
-0.0151975	-0.0151975\\
-0.01538	-0.01538\\
-0.01593	-0.01593\\
-0.0160675	-0.0160675\\
-0.0157475	-0.0157475\\
-0.015335	-0.015335\\
-0.015105	-0.015105\\
-0.0151525	-0.0151525\\
-0.015335	-0.015335\\
-0.015105	-0.015105\\
-0.01497	-0.01497\\
-0.015015	-0.015015\\
-0.014375	-0.014375\\
-0.0145575	-0.0145575\\
-0.015105	-0.015105\\
-0.0149225	-0.0149225\\
-0.014785	-0.014785\\
-0.01561	-0.01561\\
-0.0155175	-0.0155175\\
-0.01538	-0.01538\\
-0.0157475	-0.0157475\\
-0.0158375	-0.0158375\\
-0.0154275	-0.0154275\\
-0.0148325	-0.0148325\\
-0.0146025	-0.0146025\\
-0.01506	-0.01506\\
-0.0157025	-0.0157025\\
-0.0160225	-0.0160225\\
-0.0161125	-0.0161125\\
-0.015975	-0.015975\\
-0.01538	-0.01538\\
-0.01506	-0.01506\\
-0.014695	-0.014695\\
-0.014465	-0.014465\\
-0.0142375	-0.0142375\\
-0.01442	-0.01442\\
-0.0146025	-0.0146025\\
-0.01419	-0.01419\\
-0.01442	-0.01442\\
-0.0148325	-0.0148325\\
-0.015015	-0.015015\\
-0.01538	-0.01538\\
-0.015885	-0.015885\\
-0.0162975	-0.0162975\\
-0.0160675	-0.0160675\\
-0.0158375	-0.0158375\\
-0.015655	-0.015655\\
-0.0151975	-0.0151975\\
-0.0151525	-0.0151525\\
-0.0157025	-0.0157025\\
-0.0160675	-0.0160675\\
-0.01561	-0.01561\\
-0.0148325	-0.0148325\\
-0.0142375	-0.0142375\\
-0.0133675	-0.0133675\\
-0.0130475	-0.0130475\\
-0.0134575	-0.0134575\\
-0.0136875	-0.0136875\\
-0.01419	-0.01419\\
-0.01387	-0.01387\\
-0.0137775	-0.0137775\\
-0.0137325	-0.0137325\\
-0.0136425	-0.0136425\\
-0.0137325	-0.0137325\\
-0.014145	-0.014145\\
-0.014785	-0.014785\\
-0.01506	-0.01506\\
-0.015105	-0.015105\\
-0.0152425	-0.0152425\\
-0.0157025	-0.0157025\\
-0.0158375	-0.0158375\\
-0.0161125	-0.0161125\\
-0.01625	-0.01625\\
-0.01593	-0.01593\\
-0.0154275	-0.0154275\\
-0.0151975	-0.0151975\\
-0.0157025	-0.0157025\\
-0.0160675	-0.0160675\\
-0.0157925	-0.0157925\\
-0.015335	-0.015335\\
-0.0142825	-0.0142825\\
-0.013595	-0.013595\\
-0.0134575	-0.0134575\\
-0.01291	-0.01291\\
-0.013275	-0.013275\\
-0.013825	-0.013825\\
-0.0139625	-0.0139625\\
-0.013915	-0.013915\\
-0.0134575	-0.0134575\\
-0.0131375	-0.0131375\\
-0.0131825	-0.0131825\\
-0.0133675	-0.0133675\\
-0.0134125	-0.0134125\\
-0.013275	-0.013275\\
-0.0130925	-0.0130925\\
-0.0134575	-0.0134575\\
-0.0146475	-0.0146475\\
-0.015335	-0.015335\\
-0.015565	-0.015565\\
-0.015335	-0.015335\\
-0.0154725	-0.0154725\\
-0.01625	-0.01625\\
-0.0162975	-0.0162975\\
-0.0163875	-0.0163875\\
-0.0161125	-0.0161125\\
-0.0160225	-0.0160225\\
-0.0161125	-0.0161125\\
-0.0157025	-0.0157025\\
-0.0154275	-0.0154275\\
-0.0148775	-0.0148775\\
-0.014465	-0.014465\\
-0.015105	-0.015105\\
-0.0149225	-0.0149225\\
-0.014785	-0.014785\\
-0.0149225	-0.0149225\\
-0.0148775	-0.0148775\\
-0.0149225	-0.0149225\\
-0.01506	-0.01506\\
-0.0151975	-0.0151975\\
-0.015105	-0.015105\\
-0.0148325	-0.0148325\\
-0.01497	-0.01497\\
-0.0142375	-0.0142375\\
-0.01323	-0.01323\\
-0.0131375	-0.0131375\\
-0.0136425	-0.0136425\\
-0.01323	-0.01323\\
-0.01245	-0.01245\\
-0.0125875	-0.0125875\\
-0.0131825	-0.0131825\\
-0.013595	-0.013595\\
-0.0139625	-0.0139625\\
-0.014375	-0.014375\\
-0.0145575	-0.0145575\\
-0.01419	-0.01419\\
-0.014465	-0.014465\\
-0.0140075	-0.0140075\\
-0.01355	-0.01355\\
-0.01323	-0.01323\\
-0.01268	-0.01268\\
-0.01291	-0.01291\\
-0.0128625	-0.0128625\\
-0.01323	-0.01323\\
-0.013915	-0.013915\\
-0.0141	-0.0141\\
-0.01387	-0.01387\\
-0.0137325	-0.0137325\\
-0.01387	-0.01387\\
-0.0137325	-0.0137325\\
-0.01355	-0.01355\\
-0.013505	-0.013505\\
-0.01323	-0.01323\\
-0.01332	-0.01332\\
-0.013275	-0.013275\\
-0.0137325	-0.0137325\\
-0.01387	-0.01387\\
-0.0136425	-0.0136425\\
-0.013595	-0.013595\\
-0.0133675	-0.0133675\\
-0.01419	-0.01419\\
-0.0149225	-0.0149225\\
-0.014695	-0.014695\\
-0.01451	-0.01451\\
-0.014145	-0.014145\\
-0.01451	-0.01451\\
-0.0146475	-0.0146475\\
-0.01419	-0.01419\\
-0.0143275	-0.0143275\\
-0.014145	-0.014145\\
-0.0142825	-0.0142825\\
-0.0139625	-0.0139625\\
-0.0136875	-0.0136875\\
-0.013915	-0.013915\\
-0.01419	-0.01419\\
-0.0140525	-0.0140525\\
-0.0140075	-0.0140075\\
-0.014465	-0.014465\\
-0.0145575	-0.0145575\\
-0.01497	-0.01497\\
-0.01561	-0.01561\\
-0.015565	-0.015565\\
-0.015015	-0.015015\\
-0.0145575	-0.0145575\\
-0.014785	-0.014785\\
-0.0151975	-0.0151975\\
-0.015015	-0.015015\\
-0.015105	-0.015105\\
-0.01474	-0.01474\\
-0.013915	-0.013915\\
-0.013595	-0.013595\\
-0.0143275	-0.0143275\\
-0.01451	-0.01451\\
-0.014375	-0.014375\\
-0.014785	-0.014785\\
-0.0149225	-0.0149225\\
-0.014785	-0.014785\\
-0.014465	-0.014465\\
-0.01451	-0.01451\\
-0.0148325	-0.0148325\\
-0.014785	-0.014785\\
-0.01474	-0.01474\\
-0.0146475	-0.0146475\\
-0.014375	-0.014375\\
-0.01442	-0.01442\\
-0.014145	-0.014145\\
-0.014695	-0.014695\\
-0.01497	-0.01497\\
-0.015105	-0.015105\\
-0.01506	-0.01506\\
-0.01474	-0.01474\\
-0.0143275	-0.0143275\\
-0.0142825	-0.0142825\\
-0.01442	-0.01442\\
-0.014695	-0.014695\\
-0.015015	-0.015015\\
-0.015335	-0.015335\\
-0.0152425	-0.0152425\\
-0.0146475	-0.0146475\\
-0.01497	-0.01497\\
-0.01561	-0.01561\\
-0.01538	-0.01538\\
-0.01529	-0.01529\\
-0.0154275	-0.0154275\\
-0.0152425	-0.0152425\\
-0.01497	-0.01497\\
-0.015015	-0.015015\\
-0.0149225	-0.0149225\\
-0.01497	-0.01497\\
-0.01529	-0.01529\\
-0.01506	-0.01506\\
-0.015105	-0.015105\\
-0.0149225	-0.0149225\\
-0.01506	-0.01506\\
-0.0148325	-0.0148325\\
-0.01474	-0.01474\\
-0.0148325	-0.0148325\\
-0.0142375	-0.0142375\\
-0.01355	-0.01355\\
-0.013	-0.013\\
-0.0130925	-0.0130925\\
-0.0142375	-0.0142375\\
-0.0149225	-0.0149225\\
-0.015015	-0.015015\\
-0.014695	-0.014695\\
-0.0146475	-0.0146475\\
-0.014465	-0.014465\\
-0.0140075	-0.0140075\\
-0.0140525	-0.0140525\\
-0.014145	-0.014145\\
-0.0142375	-0.0142375\\
-0.014145	-0.014145\\
-0.0140525	-0.0140525\\
-0.0137325	-0.0137325\\
-0.0137775	-0.0137775\\
-0.01451	-0.01451\\
-0.014465	-0.014465\\
-0.0145575	-0.0145575\\
-0.01451	-0.01451\\
-0.014785	-0.014785\\
-0.0148325	-0.0148325\\
-0.0151525	-0.0151525\\
-0.01506	-0.01506\\
-0.015015	-0.015015\\
-0.015105	-0.015105\\
-0.01451	-0.01451\\
-0.014785	-0.014785\\
-0.0146025	-0.0146025\\
-0.014695	-0.014695\\
-0.0148325	-0.0148325\\
-0.01506	-0.01506\\
-0.0149225	-0.0149225\\
-0.01506	-0.01506\\
-0.0154275	-0.0154275\\
-0.015335	-0.015335\\
-0.0151525	-0.0151525\\
-0.0149225	-0.0149225\\
-0.014695	-0.014695\\
-0.01451	-0.01451\\
-0.0145575	-0.0145575\\
-0.014465	-0.014465\\
-0.015105	-0.015105\\
-0.015655	-0.015655\\
-0.01561	-0.01561\\
-0.0155175	-0.0155175\\
-0.01497	-0.01497\\
-0.0146025	-0.0146025\\
-0.0143275	-0.0143275\\
-0.0141	-0.0141\\
-0.0145575	-0.0145575\\
-0.0148775	-0.0148775\\
-0.015105	-0.015105\\
-0.01506	-0.01506\\
-0.014695	-0.014695\\
-0.0151525	-0.0151525\\
-0.0155175	-0.0155175\\
-0.0157025	-0.0157025\\
-0.01561	-0.01561\\
-0.0160225	-0.0160225\\
-0.01529	-0.01529\\
-0.014695	-0.014695\\
-0.01442	-0.01442\\
-0.014785	-0.014785\\
-0.0151975	-0.0151975\\
-0.01561	-0.01561\\
-0.0155175	-0.0155175\\
-0.0151975	-0.0151975\\
-0.0145575	-0.0145575\\
-0.0142825	-0.0142825\\
-0.0143275	-0.0143275\\
-0.014375	-0.014375\\
-0.0141	-0.0141\\
-0.0140075	-0.0140075\\
-0.013505	-0.013505\\
-0.0139625	-0.0139625\\
-0.01419	-0.01419\\
-0.01442	-0.01442\\
-0.014465	-0.014465\\
-0.01451	-0.01451\\
-0.0145575	-0.0145575\\
-0.01474	-0.01474\\
-0.0146025	-0.0146025\\
-0.014785	-0.014785\\
-0.0151525	-0.0151525\\
-0.01497	-0.01497\\
-0.0148775	-0.0148775\\
-0.0149225	-0.0149225\\
-0.015335	-0.015335\\
-0.0152425	-0.0152425\\
-0.0149225	-0.0149225\\
-0.01497	-0.01497\\
-0.0149225	-0.0149225\\
-0.0145575	-0.0145575\\
-0.01474	-0.01474\\
-0.01529	-0.01529\\
-0.015335	-0.015335\\
-0.0154275	-0.0154275\\
-0.0161125	-0.0161125\\
-0.015885	-0.015885\\
-0.0155175	-0.0155175\\
-0.01529	-0.01529\\
-0.01538	-0.01538\\
-0.0157925	-0.0157925\\
-0.0155175	-0.0155175\\
-0.0152425	-0.0152425\\
-0.0155175	-0.0155175\\
-0.0152425	-0.0152425\\
-0.0145575	-0.0145575\\
-0.01442	-0.01442\\
-0.0140075	-0.0140075\\
-0.0139625	-0.0139625\\
-0.013825	-0.013825\\
-0.0136875	-0.0136875\\
-0.013825	-0.013825\\
-0.01419	-0.01419\\
-0.01442	-0.01442\\
-0.0148775	-0.0148775\\
-0.0149225	-0.0149225\\
-0.01506	-0.01506\\
-0.0154275	-0.0154275\\
-0.01538	-0.01538\\
-0.015105	-0.015105\\
-0.01506	-0.01506\\
-0.01497	-0.01497\\
-0.015015	-0.015015\\
-0.01538	-0.01538\\
-0.0152425	-0.0152425\\
-0.0151975	-0.0151975\\
-0.0151525	-0.0151525\\
-0.015105	-0.015105\\
-0.0155175	-0.0155175\\
-0.0154725	-0.0154725\\
-0.0154275	-0.0154275\\
-0.0149225	-0.0149225\\
-0.0143275	-0.0143275\\
-0.013915	-0.013915\\
-0.01419	-0.01419\\
-0.0142375	-0.0142375\\
-0.014465	-0.014465\\
-0.014375	-0.014375\\
-0.0145575	-0.0145575\\
-0.0151975	-0.0151975\\
-0.0157475	-0.0157475\\
-0.01561	-0.01561\\
-0.0155175	-0.0155175\\
-0.0151525	-0.0151525\\
-0.015105	-0.015105\\
-0.0151975	-0.0151975\\
-0.0151525	-0.0151525\\
-0.0151975	-0.0151975\\
-0.0154725	-0.0154725\\
-0.015975	-0.015975\\
-0.0161125	-0.0161125\\
-0.015975	-0.015975\\
-0.0160675	-0.0160675\\
-0.0157475	-0.0157475\\
-0.0157025	-0.0157025\\
-0.0154725	-0.0154725\\
-0.01538	-0.01538\\
-0.01561	-0.01561\\
-0.0158375	-0.0158375\\
-0.015015	-0.015015\\
-0.01474	-0.01474\\
-0.014695	-0.014695\\
-0.0148325	-0.0148325\\
-0.01497	-0.01497\\
-0.0146025	-0.0146025\\
-0.014695	-0.014695\\
-0.01529	-0.01529\\
-0.016205	-0.016205\\
-0.016525	-0.016525\\
-0.0163875	-0.0163875\\
-0.01616	-0.01616\\
-0.01625	-0.01625\\
-0.016525	-0.016525\\
-0.0162975	-0.0162975\\
-0.016205	-0.016205\\
-0.01625	-0.01625\\
-0.015885	-0.015885\\
-0.015565	-0.015565\\
-0.0157475	-0.0157475\\
-0.0154275	-0.0154275\\
-0.01497	-0.01497\\
-0.01506	-0.01506\\
-0.0148775	-0.0148775\\
-0.015015	-0.015015\\
-0.015335	-0.015335\\
-0.0152425	-0.0152425\\
-0.0149225	-0.0149225\\
-0.015105	-0.015105\\
-0.01529	-0.01529\\
-0.0152425	-0.0152425\\
-0.015015	-0.015015\\
-0.01538	-0.01538\\
-0.0154725	-0.0154725\\
-0.0151525	-0.0151525\\
-0.0155175	-0.0155175\\
-0.01561	-0.01561\\
-0.01529	-0.01529\\
-0.0148775	-0.0148775\\
-0.01442	-0.01442\\
-0.0140075	-0.0140075\\
-0.014465	-0.014465\\
-0.0146025	-0.0146025\\
-0.0143275	-0.0143275\\
-0.01387	-0.01387\\
-0.014375	-0.014375\\
-0.0146475	-0.0146475\\
-0.015015	-0.015015\\
-0.015335	-0.015335\\
-0.0154275	-0.0154275\\
-0.0154725	-0.0154725\\
-0.01497	-0.01497\\
-0.0140075	-0.0140075\\
-0.0139625	-0.0139625\\
-0.0141	-0.0141\\
-0.0146025	-0.0146025\\
-0.0148775	-0.0148775\\
-0.0154725	-0.0154725\\
-0.01529	-0.01529\\
-0.01506	-0.01506\\
-0.0154275	-0.0154275\\
-0.0152425	-0.0152425\\
-0.0148325	-0.0148325\\
-0.01442	-0.01442\\
-0.01451	-0.01451\\
-0.0148325	-0.0148325\\
-0.01538	-0.01538\\
-0.0151975	-0.0151975\\
-0.0148325	-0.0148325\\
-0.014785	-0.014785\\
-0.01497	-0.01497\\
-0.0152425	-0.0152425\\
-0.0160225	-0.0160225\\
-0.0160675	-0.0160675\\
-0.015565	-0.015565\\
-0.015015	-0.015015\\
-0.01497	-0.01497\\
-0.014145	-0.014145\\
-0.01419	-0.01419\\
-0.014145	-0.014145\\
-0.014785	-0.014785\\
-0.01538	-0.01538\\
-0.0157475	-0.0157475\\
-0.01616	-0.01616\\
-0.01625	-0.01625\\
-0.015565	-0.015565\\
-0.0152425	-0.0152425\\
-0.015015	-0.015015\\
-0.0151525	-0.0151525\\
-0.0154725	-0.0154725\\
-0.01593	-0.01593\\
-0.0157025	-0.0157025\\
-0.01529	-0.01529\\
-0.015335	-0.015335\\
-0.01561	-0.01561\\
-0.01538	-0.01538\\
-0.0149225	-0.0149225\\
-0.0145575	-0.0145575\\
-0.0141	-0.0141\\
-0.013825	-0.013825\\
-0.0134575	-0.0134575\\
-0.0137325	-0.0137325\\
-0.01419	-0.01419\\
-0.01442	-0.01442\\
-0.0142825	-0.0142825\\
-0.0141	-0.0141\\
-0.0134125	-0.0134125\\
-0.0125875	-0.0125875\\
-0.0124975	-0.0124975\\
-0.0130475	-0.0130475\\
-0.0136875	-0.0136875\\
-0.0141	-0.0141\\
-0.01442	-0.01442\\
-0.01474	-0.01474\\
-0.0152425	-0.0152425\\
-0.015975	-0.015975\\
-0.01625	-0.01625\\
-0.0163875	-0.0163875\\
-0.0163425	-0.0163425\\
-0.01561	-0.01561\\
-0.01529	-0.01529\\
-0.0152425	-0.0152425\\
-0.0155175	-0.0155175\\
-0.0154725	-0.0154725\\
-0.015015	-0.015015\\
-0.014695	-0.014695\\
-0.01442	-0.01442\\
-0.01474	-0.01474\\
-0.0148325	-0.0148325\\
-0.0142375	-0.0142375\\
-0.0137775	-0.0137775\\
-0.013915	-0.013915\\
-0.0142825	-0.0142825\\
-0.014145	-0.014145\\
-0.01442	-0.01442\\
-0.0142825	-0.0142825\\
-0.01451	-0.01451\\
-0.0152425	-0.0152425\\
-0.0151525	-0.0151525\\
-0.01538	-0.01538\\
-0.015975	-0.015975\\
-0.0163425	-0.0163425\\
-0.01616	-0.01616\\
-0.015655	-0.015655\\
-0.015105	-0.015105\\
-0.014465	-0.014465\\
-0.01451	-0.01451\\
-0.014375	-0.014375\\
-0.014785	-0.014785\\
-0.0148775	-0.0148775\\
-0.0146475	-0.0146475\\
-0.0148325	-0.0148325\\
-0.01442	-0.01442\\
-0.01451	-0.01451\\
-0.014375	-0.014375\\
-0.01387	-0.01387\\
-0.013825	-0.013825\\
-0.013595	-0.013595\\
-0.01355	-0.01355\\
-0.013505	-0.013505\\
-0.0131375	-0.0131375\\
-0.0134125	-0.0134125\\
-0.01355	-0.01355\\
-0.013505	-0.013505\\
-0.0139625	-0.0139625\\
-0.01451	-0.01451\\
-0.014465	-0.014465\\
-0.014375	-0.014375\\
-0.014785	-0.014785\\
-0.0152425	-0.0152425\\
-0.015105	-0.015105\\
-0.0151525	-0.0151525\\
-0.01442	-0.01442\\
-0.0136875	-0.0136875\\
-0.0140525	-0.0140525\\
-0.0146475	-0.0146475\\
-0.0148775	-0.0148775\\
-0.015335	-0.015335\\
-0.0154275	-0.0154275\\
-0.0151525	-0.0151525\\
-0.0146475	-0.0146475\\
-0.01451	-0.01451\\
-0.0146025	-0.0146025\\
-0.01442	-0.01442\\
-0.0139625	-0.0139625\\
-0.01387	-0.01387\\
-0.0137325	-0.0137325\\
-0.013505	-0.013505\\
-0.0131375	-0.0131375\\
-0.0134125	-0.0134125\\
-0.01387	-0.01387\\
-0.0141	-0.0141\\
-0.01387	-0.01387\\
-0.0143275	-0.0143275\\
-0.015015	-0.015015\\
-0.014695	-0.014695\\
-0.0148325	-0.0148325\\
-0.0151525	-0.0151525\\
-0.0148775	-0.0148775\\
-0.0143275	-0.0143275\\
-0.0142375	-0.0142375\\
-0.014465	-0.014465\\
-0.0140525	-0.0140525\\
-0.0139625	-0.0139625\\
-0.0137775	-0.0137775\\
-0.013275	-0.013275\\
-0.0130925	-0.0130925\\
-0.0133675	-0.0133675\\
-0.0137325	-0.0137325\\
-0.0136875	-0.0136875\\
-0.01387	-0.01387\\
-0.0141	-0.0141\\
-0.01387	-0.01387\\
-0.0134125	-0.0134125\\
-0.0130475	-0.0130475\\
-0.01323	-0.01323\\
-0.0133675	-0.0133675\\
-0.0142375	-0.0142375\\
-0.0146025	-0.0146025\\
-0.014465	-0.014465\\
-0.01442	-0.01442\\
-0.014695	-0.014695\\
-0.0151975	-0.0151975\\
-0.0154725	-0.0154725\\
-0.015565	-0.015565\\
-0.015335	-0.015335\\
-0.0152425	-0.0152425\\
-0.015335	-0.015335\\
-0.0154725	-0.0154725\\
-0.015885	-0.015885\\
-0.0157475	-0.0157475\\
-0.015565	-0.015565\\
-0.0154275	-0.0154275\\
-0.01538	-0.01538\\
-0.0155175	-0.0155175\\
-0.015105	-0.015105\\
-0.01497	-0.01497\\
-0.0151975	-0.0151975\\
-0.0154275	-0.0154275\\
-0.0152425	-0.0152425\\
-0.01529	-0.01529\\
-0.0151975	-0.0151975\\
-0.015015	-0.015015\\
-0.0145575	-0.0145575\\
-0.01474	-0.01474\\
-0.015335	-0.015335\\
-0.0152425	-0.0152425\\
-0.014695	-0.014695\\
-0.0146475	-0.0146475\\
-0.014145	-0.014145\\
-0.0137775	-0.0137775\\
-0.0139625	-0.0139625\\
-0.0136875	-0.0136875\\
-0.013505	-0.013505\\
-0.013595	-0.013595\\
-0.013275	-0.013275\\
-0.01355	-0.01355\\
-0.0134575	-0.0134575\\
-0.0130925	-0.0130925\\
-0.01332	-0.01332\\
-0.0134575	-0.0134575\\
-0.0136875	-0.0136875\\
-0.0137325	-0.0137325\\
-0.0141	-0.0141\\
-0.01419	-0.01419\\
-0.015015	-0.015015\\
-0.01497	-0.01497\\
-0.014695	-0.014695\\
-0.014785	-0.014785\\
-0.014465	-0.014465\\
-0.01387	-0.01387\\
-0.01419	-0.01419\\
-0.014145	-0.014145\\
-0.0136875	-0.0136875\\
-0.01387	-0.01387\\
-0.0140075	-0.0140075\\
-0.014145	-0.014145\\
-0.013825	-0.013825\\
-0.01332	-0.01332\\
-0.013	-0.013\\
-0.01291	-0.01291\\
-0.0134125	-0.0134125\\
-0.013595	-0.013595\\
-0.0137325	-0.0137325\\
-0.0141	-0.0141\\
-0.0146475	-0.0146475\\
-0.0145575	-0.0145575\\
-0.014785	-0.014785\\
-0.0151525	-0.0151525\\
-0.015105	-0.015105\\
-0.015015	-0.015015\\
-0.015655	-0.015655\\
-0.015565	-0.015565\\
-0.0157025	-0.0157025\\
-0.0157475	-0.0157475\\
-0.015975	-0.015975\\
-0.0154725	-0.0154725\\
-0.014785	-0.014785\\
-0.0143275	-0.0143275\\
-0.0146475	-0.0146475\\
-0.0149225	-0.0149225\\
-0.01497	-0.01497\\
-0.01474	-0.01474\\
-0.0140075	-0.0140075\\
-0.013915	-0.013915\\
-0.01474	-0.01474\\
-0.015335	-0.015335\\
-0.0154275	-0.0154275\\
-0.0149225	-0.0149225\\
-0.01451	-0.01451\\
-0.0146475	-0.0146475\\
-0.0151975	-0.0151975\\
-0.0155175	-0.0155175\\
-0.01561	-0.01561\\
-0.015335	-0.015335\\
-0.01561	-0.01561\\
-0.0158375	-0.0158375\\
-0.01625	-0.01625\\
-0.0162975	-0.0162975\\
-0.0161125	-0.0161125\\
-0.016205	-0.016205\\
-0.0161125	-0.0161125\\
-0.0154275	-0.0154275\\
-0.015015	-0.015015\\
-0.0151975	-0.0151975\\
-0.0154275	-0.0154275\\
-0.0154725	-0.0154725\\
-0.0152425	-0.0152425\\
-0.015335	-0.015335\\
-0.015015	-0.015015\\
-0.015105	-0.015105\\
-0.0151525	-0.0151525\\
-0.0154725	-0.0154725\\
-0.0157025	-0.0157025\\
-0.0154275	-0.0154275\\
-0.01506	-0.01506\\
-0.014465	-0.014465\\
-0.0146025	-0.0146025\\
-0.01442	-0.01442\\
-0.01419	-0.01419\\
-0.0141	-0.0141\\
-0.014375	-0.014375\\
-0.01474	-0.01474\\
-0.0149225	-0.0149225\\
-0.01497	-0.01497\\
-0.0145575	-0.0145575\\
-0.01419	-0.01419\\
-0.0137775	-0.0137775\\
-0.0136875	-0.0136875\\
-0.014145	-0.014145\\
-0.0142825	-0.0142825\\
-0.0143275	-0.0143275\\
-0.0148325	-0.0148325\\
-0.0152425	-0.0152425\\
-0.0154725	-0.0154725\\
-0.01561	-0.01561\\
-0.0158375	-0.0158375\\
-0.0154275	-0.0154275\\
-0.015015	-0.015015\\
-0.0148325	-0.0148325\\
-0.014695	-0.014695\\
-0.01497	-0.01497\\
-0.0149225	-0.0149225\\
-0.014465	-0.014465\\
-0.0142375	-0.0142375\\
-0.014695	-0.014695\\
-0.01506	-0.01506\\
-0.015015	-0.015015\\
-0.01561	-0.01561\\
-0.01593	-0.01593\\
-0.015655	-0.015655\\
-0.0151525	-0.0151525\\
-0.014695	-0.014695\\
-0.01442	-0.01442\\
-0.0148775	-0.0148775\\
-0.01497	-0.01497\\
-0.01506	-0.01506\\
-0.0151975	-0.0151975\\
-0.0152425	-0.0152425\\
-0.015335	-0.015335\\
-0.01506	-0.01506\\
-0.015015	-0.015015\\
-0.0148325	-0.0148325\\
-0.0145575	-0.0145575\\
-0.0149225	-0.0149225\\
-0.01538	-0.01538\\
-0.015655	-0.015655\\
-0.015105	-0.015105\\
-0.0157475	-0.0157475\\
-0.0160225	-0.0160225\\
-0.0155175	-0.0155175\\
-0.014785	-0.014785\\
-0.0149225	-0.0149225\\
-0.01497	-0.01497\\
-0.0148325	-0.0148325\\
-0.01497	-0.01497\\
-0.014695	-0.014695\\
-0.0148325	-0.0148325\\
-0.015655	-0.015655\\
-0.01593	-0.01593\\
-0.0155175	-0.0155175\\
-0.0154725	-0.0154725\\
-0.01561	-0.01561\\
-0.0154275	-0.0154275\\
-0.0152425	-0.0152425\\
-0.0148775	-0.0148775\\
-0.015015	-0.015015\\
-0.0151975	-0.0151975\\
-0.0157025	-0.0157025\\
-0.0160675	-0.0160675\\
-0.0166625	-0.0166625\\
-0.016525	-0.016525\\
-0.01616	-0.01616\\
-0.0157925	-0.0157925\\
-0.015565	-0.015565\\
-0.01497	-0.01497\\
-0.014785	-0.014785\\
-0.014465	-0.014465\\
-0.0139625	-0.0139625\\
-0.0141	-0.0141\\
-0.014465	-0.014465\\
-0.0142825	-0.0142825\\
-0.01387	-0.01387\\
-0.0137325	-0.0137325\\
-0.014375	-0.014375\\
-0.01506	-0.01506\\
-0.014465	-0.014465\\
-0.014375	-0.014375\\
-0.014465	-0.014465\\
-0.0146025	-0.0146025\\
-0.0145575	-0.0145575\\
-0.0142375	-0.0142375\\
-0.01387	-0.01387\\
-0.013595	-0.013595\\
-0.0136425	-0.0136425\\
-0.0143275	-0.0143275\\
-0.0146475	-0.0146475\\
-0.014465	-0.014465\\
-0.0141	-0.0141\\
-0.01355	-0.01355\\
-0.0141	-0.0141\\
-0.01451	-0.01451\\
-0.014695	-0.014695\\
-0.01442	-0.01442\\
-0.0142825	-0.0142825\\
-0.014375	-0.014375\\
-0.014695	-0.014695\\
-0.0151975	-0.0151975\\
-0.01561	-0.01561\\
-0.0154725	-0.0154725\\
-0.0149225	-0.0149225\\
-0.014785	-0.014785\\
-0.01474	-0.01474\\
-0.01442	-0.01442\\
-0.014375	-0.014375\\
-0.01451	-0.01451\\
-0.014145	-0.014145\\
-0.013595	-0.013595\\
-0.0136875	-0.0136875\\
-0.014145	-0.014145\\
-0.014695	-0.014695\\
-0.01497	-0.01497\\
-0.0152425	-0.0152425\\
-0.01529	-0.01529\\
-0.015335	-0.015335\\
-0.0158375	-0.0158375\\
-0.01648	-0.01648\\
-0.01657	-0.01657\\
-0.0160675	-0.0160675\\
-0.01593	-0.01593\\
-0.01616	-0.01616\\
-0.01648	-0.01648\\
-0.01616	-0.01616\\
-0.0151975	-0.0151975\\
-0.014465	-0.014465\\
-0.01419	-0.01419\\
-0.013915	-0.013915\\
-0.013275	-0.013275\\
-0.0131825	-0.0131825\\
-0.0134575	-0.0134575\\
-0.013915	-0.013915\\
-0.0140525	-0.0140525\\
-0.01387	-0.01387\\
-0.0137325	-0.0137325\\
-0.013915	-0.013915\\
-0.01419	-0.01419\\
-0.0143275	-0.0143275\\
-0.0142825	-0.0142825\\
-0.0140075	-0.0140075\\
-0.0136425	-0.0136425\\
-0.0134575	-0.0134575\\
-0.0137325	-0.0137325\\
-0.0140075	-0.0140075\\
-0.0137775	-0.0137775\\
-0.0134125	-0.0134125\\
-0.0137325	-0.0137325\\
-0.013595	-0.013595\\
-0.01387	-0.01387\\
-0.0142825	-0.0142825\\
-0.0145575	-0.0145575\\
-0.0142375	-0.0142375\\
-0.0145575	-0.0145575\\
-0.014695	-0.014695\\
-0.0145575	-0.0145575\\
-0.014375	-0.014375\\
-0.014695	-0.014695\\
-0.0151975	-0.0151975\\
-0.01529	-0.01529\\
-0.0154275	-0.0154275\\
-0.01529	-0.01529\\
-0.0151525	-0.0151525\\
-0.015015	-0.015015\\
-0.01529	-0.01529\\
-0.015105	-0.015105\\
-0.015335	-0.015335\\
-0.0154275	-0.0154275\\
-0.0154725	-0.0154725\\
-0.01625	-0.01625\\
-0.0163425	-0.0163425\\
-0.015565	-0.015565\\
-0.0148775	-0.0148775\\
-0.01474	-0.01474\\
-0.0149225	-0.0149225\\
-0.0151525	-0.0151525\\
-0.0154275	-0.0154275\\
-0.01561	-0.01561\\
-0.015655	-0.015655\\
-0.01529	-0.01529\\
-0.01506	-0.01506\\
-0.0151975	-0.0151975\\
-0.01474	-0.01474\\
-0.0142375	-0.0142375\\
-0.01451	-0.01451\\
-0.0146475	-0.0146475\\
-0.01497	-0.01497\\
-0.01538	-0.01538\\
-0.01529	-0.01529\\
-0.015015	-0.015015\\
-0.014695	-0.014695\\
-0.01497	-0.01497\\
-0.015335	-0.015335\\
-0.0157925	-0.0157925\\
-0.0161125	-0.0161125\\
-0.0163425	-0.0163425\\
-0.01625	-0.01625\\
-0.0161125	-0.0161125\\
-0.015565	-0.015565\\
-0.0149225	-0.0149225\\
-0.015335	-0.015335\\
-0.0157475	-0.0157475\\
-0.01538	-0.01538\\
-0.01561	-0.01561\\
-0.01616	-0.01616\\
-0.016525	-0.016525\\
-0.01616	-0.01616\\
-0.0154275	-0.0154275\\
-0.0145575	-0.0145575\\
-0.01451	-0.01451\\
-0.0146025	-0.0146025\\
-0.01419	-0.01419\\
-0.0140075	-0.0140075\\
-0.0141	-0.0141\\
-0.0140525	-0.0140525\\
-0.014145	-0.014145\\
-0.0146475	-0.0146475\\
-0.014785	-0.014785\\
-0.015105	-0.015105\\
-0.0155175	-0.0155175\\
-0.01561	-0.01561\\
-0.01506	-0.01506\\
-0.0148325	-0.0148325\\
-0.01506	-0.01506\\
-0.01474	-0.01474\\
-0.01451	-0.01451\\
-0.01419	-0.01419\\
-0.01442	-0.01442\\
-0.014465	-0.014465\\
-0.0139625	-0.0139625\\
-0.01332	-0.01332\\
-0.013	-0.013\\
-0.0133675	-0.0133675\\
-0.01387	-0.01387\\
-0.0140525	-0.0140525\\
-0.0137775	-0.0137775\\
-0.01419	-0.01419\\
-0.0146475	-0.0146475\\
-0.01451	-0.01451\\
-0.0143275	-0.0143275\\
-0.014465	-0.014465\\
-0.0146475	-0.0146475\\
-0.014785	-0.014785\\
-0.014465	-0.014465\\
-0.01419	-0.01419\\
-0.0142825	-0.0142825\\
-0.01419	-0.01419\\
-0.0143275	-0.0143275\\
-0.014695	-0.014695\\
-0.015335	-0.015335\\
-0.0152425	-0.0152425\\
-0.01506	-0.01506\\
-0.015565	-0.015565\\
-0.0154275	-0.0154275\\
-0.0148775	-0.0148775\\
-0.01442	-0.01442\\
-0.0141	-0.0141\\
-0.0140075	-0.0140075\\
-0.0137775	-0.0137775\\
-0.0136425	-0.0136425\\
-0.01419	-0.01419\\
-0.0136875	-0.0136875\\
-0.013275	-0.013275\\
-0.0136425	-0.0136425\\
-0.0140075	-0.0140075\\
-0.0139625	-0.0139625\\
-0.0136425	-0.0136425\\
-0.0131825	-0.0131825\\
-0.01332	-0.01332\\
-0.0141	-0.0141\\
-0.0146475	-0.0146475\\
-0.0145575	-0.0145575\\
-0.0143275	-0.0143275\\
-0.014375	-0.014375\\
-0.01451	-0.01451\\
-0.01474	-0.01474\\
-0.01506	-0.01506\\
-0.0151975	-0.0151975\\
-0.01506	-0.01506\\
-0.014695	-0.014695\\
-0.0143275	-0.0143275\\
-0.0142825	-0.0142825\\
-0.0145575	-0.0145575\\
-0.0142375	-0.0142375\\
-0.0139625	-0.0139625\\
-0.01442	-0.01442\\
-0.0148325	-0.0148325\\
-0.01497	-0.01497\\
-0.01538	-0.01538\\
-0.0149225	-0.0149225\\
-0.01538	-0.01538\\
-0.015655	-0.015655\\
-0.015565	-0.015565\\
-0.01561	-0.01561\\
-0.0163425	-0.0163425\\
-0.0166625	-0.0166625\\
-0.01625	-0.01625\\
-0.0162975	-0.0162975\\
-0.01657	-0.01657\\
-0.0166175	-0.0166175\\
-0.016755	-0.016755\\
-0.0168	-0.0168\\
-0.017075	-0.017075\\
-0.0169825	-0.0169825\\
-0.016525	-0.016525\\
-0.0158375	-0.0158375\\
-0.0155175	-0.0155175\\
-0.015335	-0.015335\\
-0.0149225	-0.0149225\\
-0.0151975	-0.0151975\\
-0.015655	-0.015655\\
-0.015335	-0.015335\\
-0.015015	-0.015015\\
-0.01451	-0.01451\\
-0.01387	-0.01387\\
-0.0136425	-0.0136425\\
-0.01355	-0.01355\\
-0.0136875	-0.0136875\\
-0.013595	-0.013595\\
-0.01332	-0.01332\\
-0.012955	-0.012955\\
-0.01268	-0.01268\\
-0.01291	-0.01291\\
-0.0128625	-0.0128625\\
-0.0127725	-0.0127725\\
-0.013505	-0.013505\\
-0.0141	-0.0141\\
-0.014375	-0.014375\\
-0.0141	-0.0141\\
-0.0142375	-0.0142375\\
-0.0149225	-0.0149225\\
-0.0143275	-0.0143275\\
-0.0136425	-0.0136425\\
-0.01323	-0.01323\\
-0.0128175	-0.0128175\\
-0.0130475	-0.0130475\\
-0.013595	-0.013595\\
-0.0142375	-0.0142375\\
-0.0140075	-0.0140075\\
-0.014375	-0.014375\\
-0.01506	-0.01506\\
-0.0151975	-0.0151975\\
-0.015015	-0.015015\\
-0.014465	-0.014465\\
-0.0148325	-0.0148325\\
-0.0151975	-0.0151975\\
-0.014785	-0.014785\\
-0.0139625	-0.0139625\\
-0.014145	-0.014145\\
-0.0139625	-0.0139625\\
-0.01332	-0.01332\\
-0.012635	-0.012635\\
-0.013	-0.013\\
-0.01419	-0.01419\\
-0.0145575	-0.0145575\\
-0.0142825	-0.0142825\\
-0.013825	-0.013825\\
-0.0136425	-0.0136425\\
-0.01268	-0.01268\\
-0.0125875	-0.0125875\\
-0.01268	-0.01268\\
-0.012955	-0.012955\\
-0.013505	-0.013505\\
-0.01419	-0.01419\\
-0.0146025	-0.0146025\\
-0.014695	-0.014695\\
-0.0148775	-0.0148775\\
-0.014785	-0.014785\\
-0.01497	-0.01497\\
-0.015565	-0.015565\\
-0.0157925	-0.0157925\\
-0.01506	-0.01506\\
-0.0140525	-0.0140525\\
-0.0146475	-0.0146475\\
-0.015105	-0.015105\\
-0.01506	-0.01506\\
-0.01451	-0.01451\\
-0.0142825	-0.0142825\\
-0.0148325	-0.0148325\\
-0.0155175	-0.0155175\\
-0.0158375	-0.0158375\\
-0.0160225	-0.0160225\\
-0.016205	-0.016205\\
-0.015885	-0.015885\\
-0.0157925	-0.0157925\\
-0.01497	-0.01497\\
-0.014465	-0.014465\\
-0.0146475	-0.0146475\\
-0.0148775	-0.0148775\\
-0.015015	-0.015015\\
-0.01506	-0.01506\\
-0.014695	-0.014695\\
-0.0148325	-0.0148325\\
-0.014785	-0.014785\\
-0.0142825	-0.0142825\\
-0.013595	-0.013595\\
-0.0131825	-0.0131825\\
-0.0134125	-0.0134125\\
-0.013275	-0.013275\\
-0.0128175	-0.0128175\\
-0.0130925	-0.0130925\\
-0.013915	-0.013915\\
-0.013825	-0.013825\\
-0.013595	-0.013595\\
-0.01387	-0.01387\\
-0.0145575	-0.0145575\\
-0.01451	-0.01451\\
-0.0137775	-0.0137775\\
-0.0130925	-0.0130925\\
-0.013595	-0.013595\\
-0.014695	-0.014695\\
-0.01538	-0.01538\\
-0.0160675	-0.0160675\\
-0.0166625	-0.0166625\\
-0.016845	-0.016845\\
-0.0163875	-0.0163875\\
-0.01625	-0.01625\\
-0.0166175	-0.0166175\\
-0.016525	-0.016525\\
-0.0163875	-0.0163875\\
-0.01648	-0.01648\\
-0.0166175	-0.0166175\\
-0.0166625	-0.0166625\\
-0.01712	-0.01712\\
-0.016845	-0.016845\\
-0.0160675	-0.0160675\\
-0.0151525	-0.0151525\\
-0.01451	-0.01451\\
-0.01442	-0.01442\\
-0.0143275	-0.0143275\\
-0.0148325	-0.0148325\\
-0.0148775	-0.0148775\\
-0.01474	-0.01474\\
-0.01506	-0.01506\\
-0.0146475	-0.0146475\\
-0.01451	-0.01451\\
-0.0148325	-0.0148325\\
-0.0151525	-0.0151525\\
-0.0146475	-0.0146475\\
-0.014145	-0.014145\\
-0.01506	-0.01506\\
-0.01593	-0.01593\\
-0.015975	-0.015975\\
-0.016205	-0.016205\\
-0.0164325	-0.0164325\\
-0.0169825	-0.0169825\\
-0.017075	-0.017075\\
-0.01657	-0.01657\\
-0.015885	-0.015885\\
-0.01497	-0.01497\\
-0.0146475	-0.0146475\\
-0.014695	-0.014695\\
-0.0148775	-0.0148775\\
-0.014695	-0.014695\\
-0.014465	-0.014465\\
-0.0146475	-0.0146475\\
-0.014785	-0.014785\\
-0.01506	-0.01506\\
-0.0148775	-0.0148775\\
-0.014145	-0.014145\\
-0.013595	-0.013595\\
-0.0131375	-0.0131375\\
-0.013	-0.013\\
-0.0130475	-0.0130475\\
-0.013595	-0.013595\\
-0.0137775	-0.0137775\\
-0.0140075	-0.0140075\\
-0.0146025	-0.0146025\\
-0.01451	-0.01451\\
-0.01497	-0.01497\\
-0.015885	-0.015885\\
-0.0157025	-0.0157025\\
-0.015885	-0.015885\\
-0.01616	-0.01616\\
-0.0157925	-0.0157925\\
-0.0155175	-0.0155175\\
-0.01497	-0.01497\\
-0.0148325	-0.0148325\\
-0.015565	-0.015565\\
-0.0166175	-0.0166175\\
-0.0172575	-0.0172575\\
-0.017165	-0.017165\\
-0.016845	-0.016845\\
-0.0164325	-0.0164325\\
-0.0162975	-0.0162975\\
-0.0166625	-0.0166625\\
-0.0161125	-0.0161125\\
-0.01561	-0.01561\\
-0.015335	-0.015335\\
-0.0157475	-0.0157475\\
-0.015975	-0.015975\\
-0.01529	-0.01529\\
-0.01497	-0.01497\\
-0.014695	-0.014695\\
-0.015015	-0.015015\\
-0.01561	-0.01561\\
-0.0158375	-0.0158375\\
-0.01561	-0.01561\\
-0.0151975	-0.0151975\\
-0.015015	-0.015015\\
-0.0146475	-0.0146475\\
-0.0142375	-0.0142375\\
-0.0139625	-0.0139625\\
-0.0136425	-0.0136425\\
-0.01355	-0.01355\\
-0.0140075	-0.0140075\\
-0.0146025	-0.0146025\\
-0.014785	-0.014785\\
-0.01442	-0.01442\\
-0.01419	-0.01419\\
-0.0139625	-0.0139625\\
-0.0140075	-0.0140075\\
-0.014375	-0.014375\\
-0.01451	-0.01451\\
-0.0140525	-0.0140525\\
-0.0146025	-0.0146025\\
-0.015335	-0.015335\\
-0.0151525	-0.0151525\\
-0.014145	-0.014145\\
-0.013915	-0.013915\\
-0.0143275	-0.0143275\\
-0.014465	-0.014465\\
-0.014375	-0.014375\\
-0.013915	-0.013915\\
-0.0142375	-0.0142375\\
-0.0148775	-0.0148775\\
-0.0145575	-0.0145575\\
-0.013595	-0.013595\\
-0.0134125	-0.0134125\\
-0.0143275	-0.0143275\\
-0.0148775	-0.0148775\\
-0.0145575	-0.0145575\\
-0.01419	-0.01419\\
-0.01474	-0.01474\\
-0.015335	-0.015335\\
-0.01538	-0.01538\\
-0.0157475	-0.0157475\\
-0.0162975	-0.0162975\\
-0.015975	-0.015975\\
-0.015565	-0.015565\\
-0.015885	-0.015885\\
-0.015655	-0.015655\\
-0.0157475	-0.0157475\\
-0.0161125	-0.0161125\\
-0.015885	-0.015885\\
-0.0151525	-0.0151525\\
-0.01529	-0.01529\\
-0.01616	-0.01616\\
-0.01657	-0.01657\\
-0.0163425	-0.0163425\\
-0.0161125	-0.0161125\\
-0.0162975	-0.0162975\\
-0.0161125	-0.0161125\\
-0.015565	-0.015565\\
-0.0152425	-0.0152425\\
-0.01538	-0.01538\\
-0.0154725	-0.0154725\\
-0.015565	-0.015565\\
-0.015335	-0.015335\\
-0.0151975	-0.0151975\\
-0.015565	-0.015565\\
-0.0151975	-0.0151975\\
-0.01474	-0.01474\\
-0.014465	-0.014465\\
-0.0141	-0.0141\\
-0.0140525	-0.0140525\\
-0.0145575	-0.0145575\\
-0.0146475	-0.0146475\\
-0.0149225	-0.0149225\\
-0.0154275	-0.0154275\\
-0.0157475	-0.0157475\\
-0.015565	-0.015565\\
-0.015335	-0.015335\\
-0.0145575	-0.0145575\\
-0.014465	-0.014465\\
-0.0152425	-0.0152425\\
-0.01506	-0.01506\\
-0.0143275	-0.0143275\\
-0.014695	-0.014695\\
-0.01538	-0.01538\\
-0.015565	-0.015565\\
-0.01593	-0.01593\\
-0.01657	-0.01657\\
-0.0163425	-0.0163425\\
-0.0160225	-0.0160225\\
-0.01616	-0.01616\\
-0.0157925	-0.0157925\\
-0.01538	-0.01538\\
-0.0154725	-0.0154725\\
-0.015655	-0.015655\\
-0.0158375	-0.0158375\\
-0.015335	-0.015335\\
-0.01497	-0.01497\\
-0.01474	-0.01474\\
-0.0148775	-0.0148775\\
-0.0149225	-0.0149225\\
-0.014695	-0.014695\\
-0.01529	-0.01529\\
-0.015565	-0.015565\\
-0.01529	-0.01529\\
-0.01497	-0.01497\\
-0.0151975	-0.0151975\\
-0.0148775	-0.0148775\\
-0.01419	-0.01419\\
-0.0140525	-0.0140525\\
-0.0143275	-0.0143275\\
-0.0142825	-0.0142825\\
-0.0139625	-0.0139625\\
-0.01332	-0.01332\\
-0.0134575	-0.0134575\\
-0.013915	-0.013915\\
-0.0146025	-0.0146025\\
-0.015015	-0.015015\\
-0.0154275	-0.0154275\\
-0.0157925	-0.0157925\\
-0.015335	-0.015335\\
-0.014785	-0.014785\\
-0.0154725	-0.0154725\\
-0.0162975	-0.0162975\\
-0.016205	-0.016205\\
-0.0160675	-0.0160675\\
-0.01648	-0.01648\\
-0.0164325	-0.0164325\\
-0.01616	-0.01616\\
-0.0157925	-0.0157925\\
-0.0157025	-0.0157025\\
-0.01593	-0.01593\\
-0.01561	-0.01561\\
-0.015335	-0.015335\\
-0.0157925	-0.0157925\\
-0.01625	-0.01625\\
-0.0163875	-0.0163875\\
-0.01561	-0.01561\\
-0.0145575	-0.0145575\\
-0.0152425	-0.0152425\\
-0.0151975	-0.0151975\\
-0.0142825	-0.0142825\\
-0.0137325	-0.0137325\\
-0.0139625	-0.0139625\\
-0.014465	-0.014465\\
-0.0151525	-0.0151525\\
-0.01506	-0.01506\\
-0.014695	-0.014695\\
-0.01419	-0.01419\\
-0.01355	-0.01355\\
-0.013505	-0.013505\\
-0.013595	-0.013595\\
-0.0142375	-0.0142375\\
-0.014375	-0.014375\\
-0.0137325	-0.0137325\\
-0.0134125	-0.0134125\\
-0.01355	-0.01355\\
-0.0137775	-0.0137775\\
-0.0134575	-0.0134575\\
-0.0136425	-0.0136425\\
-0.013505	-0.013505\\
-0.0131375	-0.0131375\\
-0.0136875	-0.0136875\\
-0.0143275	-0.0143275\\
-0.01506	-0.01506\\
-0.0157925	-0.0157925\\
-0.015885	-0.015885\\
-0.0151525	-0.0151525\\
-0.014465	-0.014465\\
-0.014375	-0.014375\\
-0.0137775	-0.0137775\\
-0.0134125	-0.0134125\\
-0.013915	-0.013915\\
-0.01419	-0.01419\\
-0.014145	-0.014145\\
-0.0142375	-0.0142375\\
-0.01442	-0.01442\\
-0.0146475	-0.0146475\\
-0.014695	-0.014695\\
-0.01497	-0.01497\\
-0.015105	-0.015105\\
-0.01561	-0.01561\\
-0.015105	-0.015105\\
-0.014145	-0.014145\\
-0.013825	-0.013825\\
-0.0142825	-0.0142825\\
-0.0140525	-0.0140525\\
-0.0134125	-0.0134125\\
-0.0133675	-0.0133675\\
-0.0131825	-0.0131825\\
-0.012635	-0.012635\\
-0.01268	-0.01268\\
-0.013275	-0.013275\\
-0.0136875	-0.0136875\\
-0.0142825	-0.0142825\\
-0.01506	-0.01506\\
-0.01497	-0.01497\\
-0.0139625	-0.0139625\\
-0.01419	-0.01419\\
-0.01451	-0.01451\\
-0.01387	-0.01387\\
-0.0146025	-0.0146025\\
-0.01474	-0.01474\\
-0.01538	-0.01538\\
-0.015885	-0.015885\\
-0.01561	-0.01561\\
-0.0151975	-0.0151975\\
-0.01497	-0.01497\\
-0.0151525	-0.0151525\\
-0.01561	-0.01561\\
-0.01538	-0.01538\\
-0.0148775	-0.0148775\\
};
\end{axis}

\begin{axis}[%
width=4.927cm,
height=2.746cm,
at={(0cm,0cm)},
scale only axis,
xmin=-0.018,
xmax=-0.012,
xlabel style={font=\color{white!15!black}},
xlabel={$u(t-1)$},
ymin=-0.4,
ymax=0,
ylabel style={font=\color{white!15!black}},
ylabel={$\delta^4 y(t)$},
axis background/.style={fill=white},
title style={font=\bfseries},
title={C10, R = 0.6878},
axis x line*=bottom,
axis y line*=left
]
\addplot[only marks, mark=*, mark options={}, mark size=1.5000pt, color=mycolor1, fill=mycolor1] table[row sep=crcr]{%
x	y\\
-0.015655	-0.19226\\
-0.0157025	-0.22583\\
-0.0158375	-0.1708975\\
-0.01561	-0.1007075\\
-0.0148775	-0.12207\\
-0.014785	-0.1159675\\
-0.0148775	-0.13733\\
-0.0149225	-0.112915\\
-0.0148775	-0.0549325\\
-0.0139625	-0.042725\\
-0.0131825	-0.0457775\\
-0.0128625	-0.1007075\\
-0.0137325	-0.15564\\
-0.01474	-0.1678475\\
-0.015015	-0.17395\\
-0.0151525	-0.131225\\
-0.0149225	-0.08545\\
-0.0143275	-0.1617425\\
-0.01497	-0.131225\\
-0.014785	-0.0976575\\
-0.014375	-0.1525875\\
-0.0148325	-0.1342775\\
-0.014785	-0.112915\\
-0.014695	-0.183105\\
-0.015105	-0.1678475\\
-0.0151975	-0.131225\\
-0.0149225	-0.18921\\
-0.0151975	-0.1861575\\
-0.015335	-0.146485\\
-0.0151525	-0.1251225\\
-0.0149225	-0.10376\\
-0.014695	-0.1190175\\
-0.014695	-0.146485\\
-0.0149225	-0.13733\\
-0.0149225	-0.149535\\
-0.0149225	-0.149535\\
-0.01497	-0.164795\\
-0.015015	-0.2136225\\
-0.0154725	-0.19226\\
-0.0155175	-0.13733\\
-0.015105	-0.1251225\\
-0.0149225	-0.0885\\
-0.014375	-0.08545\\
-0.014145	-0.112915\\
-0.0143275	-0.08545\\
-0.01419	-0.0915525\\
-0.014145	-0.128175\\
-0.01451	-0.13733\\
-0.014695	-0.128175\\
-0.0146475	-0.1953125\\
-0.0151975	-0.15564\\
-0.015105	-0.0915525\\
-0.01451	-0.076295\\
-0.014145	-0.1007075\\
-0.0142375	-0.1159675\\
-0.014465	-0.0732425\\
-0.0140075	-0.0732425\\
-0.0137325	-0.0915525\\
-0.0140075	-0.079345\\
-0.013915	-0.0671375\\
-0.0137325	-0.0885\\
-0.01387	-0.1007075\\
-0.0140075	-0.146485\\
-0.0146025	-0.14038\\
-0.01474	-0.164795\\
-0.0149225	-0.1525875\\
-0.015015	-0.1770025\\
-0.01506	-0.1434325\\
-0.01497	-0.146485\\
-0.0148775	-0.128175\\
-0.014785	-0.1251225\\
-0.014695	-0.12207\\
-0.01474	-0.201415\\
-0.01529	-0.27771\\
-0.015975	-0.286865\\
-0.0162975	-0.2807625\\
-0.0162975	-0.1861575\\
-0.0158375	-0.250245\\
-0.0161125	-0.31433\\
-0.016525	-0.320435\\
-0.0166625	-0.24109\\
-0.0163875	-0.31128\\
-0.0166625	-0.3967275\\
-0.017165	-0.2899175\\
-0.0169825	-0.24414\\
-0.0166625	-0.1708975\\
-0.01616	-0.1342775\\
-0.01561	-0.1190175\\
-0.015335	-0.14038\\
-0.01538	-0.112915\\
-0.015105	-0.076295\\
-0.014465	-0.076295\\
-0.014145	-0.0915525\\
-0.0142375	-0.1251225\\
-0.014695	-0.12207\\
-0.01474	-0.1190175\\
-0.01474	-0.15564\\
-0.015015	-0.1617425\\
-0.0151525	-0.2136225\\
-0.015565	-0.1800525\\
-0.01561	-0.2136225\\
-0.0157025	-0.1586925\\
-0.0154725	-0.13733\\
-0.0152425	-0.1617425\\
-0.015335	-0.12207\\
-0.015105	-0.1098625\\
-0.0149225	-0.1342775\\
-0.015015	-0.13733\\
-0.01506	-0.1007075\\
-0.014785	-0.1098625\\
-0.014695	-0.0823975\\
-0.01451	-0.0915525\\
-0.01442	-0.1007075\\
-0.01442	-0.076295\\
-0.01419	-0.08545\\
-0.01419	-0.1098625\\
-0.01442	-0.1586925\\
-0.0149225	-0.164795\\
-0.0151525	-0.1800525\\
-0.01529	-0.112915\\
-0.0148325	-0.14038\\
-0.01497	-0.2227775\\
-0.015655	-0.1770025\\
-0.015565	-0.2044675\\
-0.01561	-0.201415\\
-0.0157025	-0.1251225\\
-0.0151525	-0.0915525\\
-0.0146475	-0.1190175\\
-0.01474	-0.131225\\
-0.0149225	-0.2044675\\
-0.015565	-0.2594\\
-0.0160225	-0.2563475\\
-0.01616	-0.19226\\
-0.015885	-0.201415\\
-0.0158375	-0.1800525\\
-0.0157025	-0.17395\\
-0.015655	-0.1708975\\
-0.015655	-0.128175\\
-0.0152425	-0.112915\\
-0.0149225	-0.0976575\\
-0.014785	-0.094605\\
-0.0146025	-0.10376\\
-0.0146475	-0.14038\\
-0.015015	-0.2105725\\
-0.01561	-0.1770025\\
-0.0155175	-0.1251225\\
-0.0151525	-0.1068125\\
-0.0148775	-0.1007075\\
-0.01474	-0.07019\\
-0.0142375	-0.0579825\\
-0.0136875	-0.061035\\
-0.0136425	-0.10376\\
-0.014145	-0.0823975\\
-0.0141	-0.08545\\
-0.0140525	-0.094605\\
-0.014145	-0.0885\\
-0.0141	-0.0671375\\
-0.013825	-0.0549325\\
-0.0134125	-0.0457775\\
-0.0130475	-0.0640875\\
-0.013275	-0.128175\\
-0.01419	-0.183105\\
-0.01497	-0.1953125\\
-0.015335	-0.13733\\
-0.015015	-0.0976575\\
-0.01451	-0.079345\\
-0.0141	-0.061035\\
-0.013595	-0.10376\\
-0.0141	-0.1068125\\
-0.0142825	-0.13733\\
-0.0146025	-0.183105\\
-0.01506	-0.2716075\\
-0.01593	-0.31433\\
-0.016525	-0.2594\\
-0.0164325	-0.253295\\
-0.0163875	-0.1678475\\
-0.015885	-0.18921\\
-0.0158375	-0.198365\\
-0.015885	-0.216675\\
-0.0160675	-0.17395\\
-0.0157925	-0.15564\\
-0.01561	-0.1525875\\
-0.015565	-0.14038\\
-0.0154275	-0.1678475\\
-0.0155175	-0.131225\\
-0.01529	-0.19226\\
-0.0157475	-0.24414\\
-0.0160675	-0.1800525\\
-0.0157925	-0.112915\\
-0.0151975	-0.1190175\\
-0.015015	-0.1953125\\
-0.015565	-0.234985\\
-0.015975	-0.2807625\\
-0.0163425	-0.2990725\\
-0.01657	-0.2990725\\
-0.0167075	-0.2288825\\
-0.0163875	-0.2105725\\
-0.01625	-0.24109\\
-0.0163425	-0.27771\\
-0.016525	-0.1861575\\
-0.0161125	-0.1159675\\
-0.01538	-0.15564\\
-0.0154725	-0.13733\\
-0.0154275	-0.08545\\
-0.0148775	-0.1190175\\
-0.0149225	-0.1342775\\
-0.01506	-0.1068125\\
-0.0149225	-0.08545\\
-0.0146475	-0.1098625\\
-0.01474	-0.1007075\\
-0.014695	-0.131225\\
-0.014785	-0.164795\\
-0.01529	-0.15564\\
-0.015335	-0.112915\\
-0.015015	-0.112915\\
-0.0149225	-0.1678475\\
-0.0152425	-0.26245\\
-0.0160675	-0.198365\\
-0.015975	-0.128175\\
-0.01538	-0.1007075\\
-0.0148775	-0.1190175\\
-0.0148775	-0.1007075\\
-0.0148325	-0.079345\\
-0.01451	-0.0885\\
-0.01442	-0.1068125\\
-0.0146475	-0.128175\\
-0.0148325	-0.1434325\\
-0.015015	-0.1068125\\
-0.014695	-0.1586925\\
-0.015105	-0.18921\\
-0.0155175	-0.1770025\\
-0.0155175	-0.1434325\\
-0.015335	-0.15564\\
-0.015335	-0.2044675\\
-0.0157025	-0.131225\\
-0.01538	-0.07019\\
-0.014465	-0.0915525\\
-0.0143275	-0.1068125\\
-0.01442	-0.128175\\
-0.01474	-0.1159675\\
-0.01474	-0.0915525\\
-0.014465	-0.131225\\
-0.01474	-0.131225\\
-0.0148325	-0.094605\\
-0.0145575	-0.1190175\\
-0.014695	-0.10376\\
-0.0146475	-0.094605\\
-0.01451	-0.112915\\
-0.0145575	-0.0732425\\
-0.01419	-0.079345\\
-0.014145	-0.0732425\\
-0.01387	-0.0671375\\
-0.0137775	-0.1190175\\
-0.01442	-0.1342775\\
-0.014695	-0.17395\\
-0.01506	-0.2227775\\
-0.01561	-0.15564\\
-0.01538	-0.0915525\\
-0.014695	-0.079345\\
-0.0142375	-0.0732425\\
-0.0140525	-0.1007075\\
-0.014375	-0.079345\\
-0.0140525	-0.076295\\
-0.0139625	-0.0976575\\
-0.01419	-0.1525875\\
-0.01474	-0.1342775\\
-0.0149225	-0.198365\\
-0.015335	-0.13733\\
-0.015105	-0.12207\\
-0.0148775	-0.0823975\\
-0.0142825	-0.0732425\\
-0.013915	-0.0579825\\
-0.01355	-0.0579825\\
-0.01332	-0.0640875\\
-0.0134575	-0.1007075\\
-0.0140075	-0.1159675\\
-0.0142375	-0.12207\\
-0.01442	-0.128175\\
-0.0145575	-0.198365\\
-0.0152425	-0.183105\\
-0.01538	-0.1342775\\
-0.01506	-0.1586925\\
-0.0151525	-0.17395\\
-0.01529	-0.2136225\\
-0.01561	-0.1800525\\
-0.015565	-0.1190175\\
-0.015015	-0.07019\\
-0.0142375	-0.0549325\\
-0.0136425	-0.0549325\\
-0.0134575	-0.0671375\\
-0.013505	-0.0671375\\
-0.01355	-0.0640875\\
-0.0134575	-0.0823975\\
-0.0136425	-0.1190175\\
-0.01419	-0.1098625\\
-0.0143275	-0.112915\\
-0.0142825	-0.131225\\
-0.014465	-0.0885\\
-0.0141	-0.05188\\
-0.013505	-0.0885\\
-0.013825	-0.14038\\
-0.01451	-0.13733\\
-0.0146025	-0.149535\\
-0.014785	-0.1342775\\
-0.014695	-0.1007075\\
-0.014375	-0.128175\\
-0.0145575	-0.1800525\\
-0.01506	-0.1770025\\
-0.0152425	-0.128175\\
-0.01497	-0.2044675\\
-0.01538	-0.15564\\
-0.015335	-0.1586925\\
-0.0151525	-0.183105\\
-0.015335	-0.1770025\\
-0.01538	-0.13733\\
-0.0151525	-0.1190175\\
-0.01497	-0.1953125\\
-0.0154275	-0.2716075\\
-0.0160225	-0.2136225\\
-0.015975	-0.1251225\\
-0.01529	-0.128175\\
-0.015015	-0.128175\\
-0.015015	-0.1525875\\
-0.0151975	-0.1800525\\
-0.01538	-0.13733\\
-0.0151975	-0.14038\\
-0.015105	-0.183105\\
-0.0154275	-0.198365\\
-0.01561	-0.14038\\
-0.01529	-0.112915\\
-0.0149225	-0.1434325\\
-0.01506	-0.2197275\\
-0.015655	-0.1953125\\
-0.0157475	-0.201415\\
-0.0157475	-0.13733\\
-0.0152425	-0.112915\\
-0.01497	-0.1190175\\
-0.0149225	-0.079345\\
-0.01442	-0.05188\\
-0.0137775	-0.05188\\
-0.01332	-0.042725\\
-0.013	-0.079345\\
-0.013595	-0.12207\\
-0.01419	-0.13733\\
-0.0145575	-0.1007075\\
-0.01442	-0.112915\\
-0.01442	-0.131225\\
-0.0146025	-0.0885\\
-0.01419	-0.0885\\
-0.0140525	-0.0885\\
-0.0140525	-0.0732425\\
-0.0137775	-0.079345\\
-0.01387	-0.128175\\
-0.01442	-0.1617425\\
-0.0148325	-0.10376\\
-0.01442	-0.0823975\\
-0.0140525	-0.07019\\
-0.01387	-0.0823975\\
-0.0139625	-0.0671375\\
-0.0136425	-0.112915\\
-0.0142375	-0.201415\\
-0.0151975	-0.1586925\\
-0.0151525	-0.2044675\\
-0.0154725	-0.2471925\\
-0.0158375	-0.24414\\
-0.015975	-0.19226\\
-0.0157475	-0.146485\\
-0.015335	-0.13733\\
-0.0152425	-0.1434325\\
-0.0151525	-0.164795\\
-0.01529	-0.19226\\
-0.0155175	-0.19226\\
-0.015565	-0.250245\\
-0.015975	-0.2716075\\
-0.0162975	-0.3265375\\
-0.01657	-0.2288825\\
-0.0162975	-0.2471925\\
-0.0162975	-0.2716075\\
-0.01648	-0.2136225\\
-0.016205	-0.24109\\
-0.0162975	-0.2136225\\
-0.016205	-0.128175\\
-0.0155175	-0.08545\\
-0.0148325	-0.0915525\\
-0.0145575	-0.128175\\
-0.0149225	-0.1098625\\
-0.014785	-0.079345\\
-0.014375	-0.10376\\
-0.01442	-0.0823975\\
-0.0142825	-0.0579825\\
-0.013915	-0.079345\\
-0.013915	-0.08545\\
-0.0140525	-0.0640875\\
-0.013825	-0.1068125\\
-0.0142825	-0.146485\\
-0.01474	-0.112915\\
-0.0145575	-0.1708975\\
-0.01506	-0.250245\\
-0.0158375	-0.2197275\\
-0.015885	-0.2746575\\
-0.01625	-0.2044675\\
-0.015975	-0.2044675\\
-0.01593	-0.15564\\
-0.015565	-0.094605\\
-0.0148775	-0.1159675\\
-0.0149225	-0.1068125\\
-0.014695	-0.0732425\\
-0.0143275	-0.1007075\\
-0.014465	-0.0549325\\
-0.01419	-0.0488275\\
-0.01355	-0.061035\\
-0.0134125	-0.112915\\
-0.0141	-0.1434325\\
-0.0146475	-0.164795\\
-0.01497	-0.164795\\
-0.015105	-0.15564\\
-0.01506	-0.17395\\
-0.0152425	-0.1434325\\
-0.01506	-0.12207\\
-0.0148775	-0.1251225\\
-0.0148325	-0.1098625\\
-0.0146025	-0.149535\\
-0.01497	-0.1159675\\
-0.014785	-0.0915525\\
-0.01442	-0.128175\\
-0.014695	-0.17395\\
-0.0151525	-0.13733\\
-0.015015	-0.10376\\
-0.01474	-0.094605\\
-0.01451	-0.10376\\
-0.0145575	-0.0823975\\
-0.0142825	-0.076295\\
-0.0140075	-0.10376\\
-0.01419	-0.1190175\\
-0.01451	-0.131225\\
-0.014695	-0.1068125\\
-0.0145575	-0.1342775\\
-0.014695	-0.128175\\
-0.014695	-0.0823975\\
-0.0142375	-0.0823975\\
-0.0141	-0.146485\\
-0.01474	-0.201415\\
-0.015335	-0.1953125\\
-0.0155175	-0.1678475\\
-0.015335	-0.15564\\
-0.01529	-0.1251225\\
-0.015015	-0.12207\\
-0.0148775	-0.10376\\
-0.0146475	-0.13733\\
-0.0148775	-0.1251225\\
-0.0148325	-0.0823975\\
-0.01442	-0.112915\\
-0.0146025	-0.131225\\
-0.014785	-0.1525875\\
-0.015015	-0.1708975\\
-0.0151975	-0.1708975\\
-0.0152425	-0.2197275\\
-0.015655	-0.268555\\
-0.0161125	-0.2990725\\
-0.01648	-0.201415\\
-0.0160675	-0.1159675\\
-0.0152425	-0.0823975\\
-0.0145575	-0.061035\\
-0.0140075	-0.08545\\
-0.0140075	-0.0732425\\
-0.0140525	-0.128175\\
-0.0145575	-0.12207\\
-0.0146025	-0.1007075\\
-0.014465	-0.131225\\
-0.01474	-0.1586925\\
-0.0149225	-0.1770025\\
-0.0151975	-0.18921\\
-0.01538	-0.2594\\
-0.01593	-0.2319325\\
-0.0160225	-0.1800525\\
-0.0157925	-0.128175\\
-0.01538	-0.131225\\
-0.0151525	-0.13733\\
-0.0151525	-0.164795\\
-0.01538	-0.128175\\
-0.0151525	-0.1251225\\
-0.015015	-0.12207\\
-0.01497	-0.076295\\
-0.014375	-0.112915\\
-0.0145575	-0.1708975\\
-0.0151525	-0.12207\\
-0.0149225	-0.131225\\
-0.0149225	-0.216675\\
-0.0155175	-0.1586925\\
-0.0154725	-0.1586925\\
-0.015335	-0.2136225\\
-0.0157475	-0.2044675\\
-0.0157475	-0.1342775\\
-0.015335	-0.0885\\
-0.014785	-0.0915525\\
-0.0145575	-0.1525875\\
-0.015015	-0.22583\\
-0.0157025	-0.24414\\
-0.0160675	-0.250245\\
-0.0161125	-0.198365\\
-0.015885	-0.1342775\\
-0.01538	-0.1159675\\
-0.01506	-0.094605\\
-0.0146475	-0.08545\\
-0.01442	-0.0732425\\
-0.0142825	-0.1007075\\
-0.014465	-0.1068125\\
-0.0146025	-0.076295\\
-0.0142375	-0.1068125\\
-0.014465	-0.14038\\
-0.0148325	-0.15564\\
-0.01506	-0.1953125\\
-0.01538	-0.24109\\
-0.015885	-0.286865\\
-0.0162975	-0.2105725\\
-0.0160675	-0.183105\\
-0.0158375	-0.19226\\
-0.0157925	-0.1586925\\
-0.015655	-0.1159675\\
-0.01529	-0.1434325\\
-0.0152425	-0.22583\\
-0.0158375	-0.253295\\
-0.01616	-0.1525875\\
-0.0157025	-0.0915525\\
-0.0149225	-0.0671375\\
-0.0142375	-0.0457775\\
-0.0133675	-0.0549325\\
-0.0130925	-0.076295\\
-0.01355	-0.079345\\
-0.013825	-0.1068125\\
-0.014145	-0.094605\\
-0.0141	-0.076295\\
-0.013825	-0.0732425\\
-0.0137325	-0.0732425\\
-0.0137325	-0.0671375\\
-0.0136875	-0.079345\\
-0.0137325	-0.1098625\\
-0.0141	-0.1586925\\
-0.01474	-0.1708975\\
-0.015015	-0.164795\\
-0.01506	-0.1861575\\
-0.01529	-0.2288825\\
-0.015655	-0.2319325\\
-0.015885	-0.2655025\\
-0.01616	-0.250245\\
-0.01625	-0.1861575\\
-0.01593	-0.13733\\
-0.0154275	-0.128175\\
-0.0152425	-0.2105725\\
-0.0157025	-0.24414\\
-0.0160675	-0.17395\\
-0.0157475	-0.1159675\\
-0.01529	-0.0671375\\
-0.0142825	-0.0549325\\
-0.013595	-0.05188\\
-0.0134125	-0.03662\\
-0.012955	-0.0640875\\
-0.01332	-0.1007075\\
-0.0137775	-0.1007075\\
-0.013915	-0.0885\\
-0.01387	-0.061035\\
-0.0134575	-0.0488275\\
-0.0131375	-0.07019\\
-0.0131825	-0.07019\\
-0.013275	-0.076295\\
-0.0134125	-0.061035\\
-0.013275	-0.0549325\\
-0.0130475	-0.0823975\\
-0.0134125	-0.17395\\
-0.0146475	-0.198365\\
-0.015335	-0.216675\\
-0.015565	-0.1586925\\
-0.015335	-0.201415\\
-0.0155175	-0.286865\\
-0.016205	-0.2594\\
-0.0163425	-0.2655025\\
-0.0163875	-0.2044675\\
-0.0161125	-0.2044675\\
-0.0160225	-0.216675\\
-0.0161125	-0.149535\\
-0.0157025	-0.14038\\
-0.0154275	-0.1007075\\
-0.0148775	-0.0885\\
-0.014465	-0.15564\\
-0.015105	-0.1190175\\
-0.0149225	-0.12207\\
-0.014785	-0.1342775\\
-0.0149225	-0.1251225\\
-0.0148325	-0.1251225\\
-0.0148775	-0.131225\\
-0.0149225	-0.1434325\\
-0.01506	-0.1617425\\
-0.0151525	-0.14038\\
-0.015105	-0.10376\\
-0.014785	-0.131225\\
-0.0149225	-0.079345\\
-0.0143275	-0.0457775\\
-0.01323	-0.0640875\\
-0.0131375	-0.08545\\
-0.013595	-0.0549325\\
-0.01323	-0.024415\\
-0.01245	-0.0549325\\
-0.0125875	-0.0823975\\
-0.01323	-0.094605\\
-0.013595	-0.1098625\\
-0.0139625	-0.0976575\\
-0.0140075	-0.146485\\
-0.014375	-0.12207\\
-0.0145575	-0.0915525\\
-0.01419	-0.13733\\
-0.01442	-0.0976575\\
-0.0140075	-0.0671375\\
-0.01355	-0.0579825\\
-0.0131825	-0.042725\\
-0.012725	-0.0640875\\
-0.01291	-0.0488275\\
-0.0128175	-0.0732425\\
-0.0130475	-0.1098625\\
-0.013825	-0.1068125\\
-0.0140075	-0.0823975\\
-0.013825	-0.1007075\\
-0.013825	-0.0732425\\
-0.0136875	-0.1098625\\
-0.013825	-0.079345\\
-0.0136875	-0.079345\\
-0.01355	-0.076295\\
-0.0134575	-0.061035\\
-0.01323	-0.0671375\\
-0.01332	-0.0671375\\
-0.013275	-0.10376\\
-0.013595	-0.0915525\\
-0.0137775	-0.076295\\
-0.01355	-0.0823975\\
-0.013505	-0.0671375\\
-0.013275	-0.079345\\
-0.01332	-0.146485\\
-0.01419	-0.18921\\
-0.0149225	-0.128175\\
-0.014695	-0.12207\\
-0.01451	-0.0976575\\
-0.01419	-0.146485\\
-0.0146475	-0.146485\\
-0.014785	-0.094605\\
-0.0143275	-0.1159675\\
-0.014375	-0.0976575\\
-0.014145	-0.128175\\
-0.0143275	-0.0823975\\
-0.0140075	-0.0885\\
-0.0137325	-0.094605\\
-0.0139625	-0.1190175\\
-0.01419	-0.0915525\\
-0.0141	-0.10376\\
-0.0140075	-0.10376\\
-0.0140525	-0.1434325\\
-0.01451	-0.128175\\
-0.0145575	-0.1800525\\
-0.01497	-0.234985\\
-0.015655	-0.1953125\\
-0.015565	-0.1251225\\
-0.01506	-0.1007075\\
-0.0145575	-0.13733\\
-0.014785	-0.17395\\
-0.0151975	-0.14038\\
-0.01497	-0.1586925\\
-0.015105	-0.1098625\\
-0.01474	-0.0579825\\
-0.013915	-0.061035\\
-0.013595	-0.131225\\
-0.0142825	-0.112915\\
-0.0145575	-0.1068125\\
-0.014375	-0.1617425\\
-0.0148325	-0.1525875\\
-0.0149225	-0.14038\\
-0.014785	-0.0976575\\
-0.0145575	-0.1342775\\
-0.0145575	-0.164795\\
-0.0148775	-0.131225\\
-0.0148325	-0.1342775\\
-0.014785	-0.12207\\
-0.01474	-0.094605\\
-0.01442	-0.1068125\\
-0.014465	-0.0885\\
-0.01419	-0.0915525\\
-0.014145	-0.1434325\\
-0.014695	-0.1678475\\
-0.01497	-0.1770025\\
-0.01506	-0.15564\\
-0.01506	-0.112915\\
-0.014695	-0.0885\\
-0.0142825	-0.0976575\\
-0.0142375	-0.1159675\\
-0.01442	-0.1434325\\
-0.014695	-0.1708975\\
-0.015015	-0.198365\\
-0.015335	-0.164795\\
-0.0151975	-0.1007075\\
-0.0146475	-0.1708975\\
-0.015015	-0.2288825\\
-0.015565	-0.164795\\
-0.01538	-0.1617425\\
-0.0152425	-0.18921\\
-0.0154275	-0.149535\\
-0.0152425	-0.1251225\\
-0.01497	-0.14038\\
-0.015015	-0.128175\\
-0.0149225	-0.1434325\\
-0.015015	-0.1770025\\
-0.01529	-0.14038\\
-0.015105	-0.1586925\\
-0.0151525	-0.146485\\
-0.0151525	-0.149535\\
-0.015105	-0.12207\\
-0.0149225	-0.1586925\\
-0.01506	-0.112915\\
-0.0148775	-0.12207\\
-0.01474	-0.131225\\
-0.0148775	-0.1251225\\
-0.0148325	-0.076295\\
-0.0142375	-0.0488275\\
-0.0134575	-0.0396725\\
-0.012955	-0.07019\\
-0.0130475	-0.1342775\\
-0.014145	-0.1678475\\
-0.0148775	-0.164795\\
-0.01506	-0.1159675\\
-0.014695	-0.13733\\
-0.014695	-0.1251225\\
-0.014695	-0.1098625\\
-0.014465	-0.076295\\
-0.0140075	-0.1068125\\
-0.0140525	-0.094605\\
-0.01419	-0.0915525\\
-0.0141	-0.10376\\
-0.0142375	-0.094605\\
-0.014145	-0.08545\\
-0.0140525	-0.0640875\\
-0.0136875	-0.0885\\
-0.0137775	-0.146485\\
-0.01451	-0.1068125\\
-0.014375	-0.131225\\
-0.01442	-0.112915\\
-0.01442	-0.149535\\
-0.014695	-0.1434325\\
-0.014785	-0.19226\\
-0.015105	-0.146485\\
-0.015015	-0.1617425\\
-0.015015	-0.164795\\
-0.01506	-0.094605\\
-0.01451	-0.1525875\\
-0.01474	-0.1159675\\
-0.0146025	-0.128175\\
-0.0146475	-0.1434325\\
-0.014785	-0.17395\\
-0.01506	-0.1342775\\
-0.0149225	-0.1770025\\
-0.01506	-0.2044675\\
-0.01538	-0.17395\\
-0.015335	-0.146485\\
-0.0151525	-0.1251225\\
-0.0149225	-0.146485\\
-0.01497	-0.131225\\
-0.0148775	-0.1098625\\
-0.014695	-0.094605\\
-0.01451	-0.112915\\
-0.0145575	-0.0976575\\
-0.01451	-0.1800525\\
-0.015105	-0.22583\\
-0.0157025	-0.198365\\
-0.015655	-0.18921\\
-0.0155175	-0.1861575\\
-0.0155175	-0.1159675\\
-0.01506	-0.1068125\\
-0.014695	-0.0885\\
-0.01442	-0.076295\\
-0.014145	-0.0823975\\
-0.014145	-0.131225\\
-0.0145575	-0.149535\\
-0.0149225	-0.1770025\\
-0.0151525	-0.149535\\
-0.01506	-0.1068125\\
-0.014695	-0.1800525\\
-0.0151525	-0.2105725\\
-0.0155175	-0.22583\\
-0.0157475	-0.183105\\
-0.015655	-0.2899175\\
-0.0161125	-0.2136225\\
-0.0161125	-0.128175\\
-0.015335	-0.0976575\\
-0.014785	-0.08545\\
-0.014465	-0.14038\\
-0.0148325	-0.17395\\
-0.0152425	-0.2227775\\
-0.015655	-0.1708975\\
-0.015565	-0.1342775\\
-0.0152425	-0.079345\\
-0.01451	-0.0915525\\
-0.01419	-0.08545\\
-0.0142375	-0.0976575\\
-0.014375	-0.0640875\\
-0.0140525	-0.0885\\
-0.0140525	-0.0671375\\
-0.0139625	-0.0488275\\
-0.0134575	-0.0976575\\
-0.01387	-0.1068125\\
-0.014145	-0.12207\\
-0.01442	-0.1159675\\
-0.014465	-0.12207\\
-0.01451	-0.12207\\
-0.0145575	-0.15564\\
-0.01474	-0.1098625\\
-0.0145575	-0.15564\\
-0.014785	-0.17395\\
-0.0151525	-0.14038\\
-0.01497	-0.146485\\
-0.0149225	-0.13733\\
-0.0148775	-0.1525875\\
-0.0149225	-0.201415\\
-0.01538	-0.15564\\
-0.0152425	-0.1251225\\
-0.0149225	-0.1434325\\
-0.01497	-0.128175\\
-0.0149225	-0.0915525\\
-0.01451	-0.14038\\
-0.01474	-0.1861575\\
-0.01529	-0.17395\\
-0.015335	-0.1953125\\
-0.0154275	-0.2838125\\
-0.0160675	-0.198365\\
-0.015975	-0.1800525\\
-0.01561	-0.14038\\
-0.015335	-0.1708975\\
-0.01538	-0.2319325\\
-0.0158375	-0.1617425\\
-0.015565	-0.1434325\\
-0.01529	-0.19226\\
-0.01561	-0.128175\\
-0.0152425	-0.0823975\\
-0.0145575	-0.1068125\\
-0.01451	-0.0823975\\
-0.0142825	-0.0640875\\
-0.0140075	-0.07019\\
-0.013915	-0.0640875\\
-0.0137775	-0.0640875\\
-0.0136425	-0.079345\\
-0.013825	-0.1068125\\
-0.014145	-0.12207\\
-0.01442	-0.1586925\\
-0.0148325	-0.149535\\
-0.0149225	-0.1678475\\
-0.015105	-0.198365\\
-0.0154275	-0.183105\\
-0.01538	-0.13733\\
-0.015105	-0.146485\\
-0.01506	-0.1342775\\
-0.015015	-0.1434325\\
-0.015015	-0.19226\\
-0.01538	-0.149535\\
-0.0151975	-0.1586925\\
-0.0152425	-0.15564\\
-0.0152425	-0.1434325\\
-0.015105	-0.2105725\\
-0.015565	-0.17395\\
-0.015565	-0.183105\\
-0.0154725	-0.164795\\
-0.01538	-0.1068125\\
-0.0149225	-0.0732425\\
-0.0143275	-0.061035\\
-0.01387	-0.10376\\
-0.01419	-0.08545\\
-0.0142375	-0.1190175\\
-0.01442	-0.0885\\
-0.014375	-0.112915\\
-0.01451	-0.198365\\
-0.0151975	-0.234985\\
-0.0157925	-0.18921\\
-0.015655	-0.198365\\
-0.01561	-0.1770025\\
-0.01561	-0.131225\\
-0.0151975	-0.13733\\
-0.015105	-0.149535\\
-0.0152425	-0.1342775\\
-0.015105	-0.15564\\
-0.0151975	-0.19226\\
-0.0154725	-0.2655025\\
-0.0160675	-0.234985\\
-0.0160675	-0.2197275\\
-0.0160225	-0.24414\\
-0.0161125	-0.1708975\\
-0.0157475	-0.1861575\\
-0.0157025	-0.149535\\
-0.0155175	-0.149535\\
-0.0154275	-0.18921\\
-0.015655	-0.2136225\\
-0.0158375	-0.094605\\
-0.015015	-0.12207\\
-0.014785	-0.094605\\
-0.0146475	-0.1342775\\
-0.0148775	-0.1342775\\
-0.015015	-0.0885\\
-0.0145575	-0.1159675\\
-0.014695	-0.1861575\\
-0.0152425	-0.29602\\
-0.0162975	-0.29602\\
-0.01657	-0.2655025\\
-0.01648	-0.2136225\\
-0.016205	-0.24414\\
-0.0162975	-0.2899175\\
-0.016525	-0.216675\\
-0.0163425	-0.2227775\\
-0.01625	-0.2288825\\
-0.0162975	-0.1617425\\
-0.01593	-0.14038\\
-0.01561	-0.1800525\\
-0.0157475	-0.12207\\
-0.0154725	-0.094605\\
-0.015015	-0.131225\\
-0.015105	-0.0976575\\
-0.0148775	-0.1190175\\
-0.0148325	-0.10376\\
-0.0148325	-0.128175\\
-0.015015	-0.164795\\
-0.015335	-0.131225\\
-0.0152425	-0.10376\\
-0.0148775	-0.1251225\\
-0.01497	-0.1434325\\
-0.015105	-0.164795\\
-0.01529	-0.14038\\
-0.0151975	-0.1159675\\
-0.015015	-0.1770025\\
-0.01538	-0.164795\\
-0.0154725	-0.1190175\\
-0.015105	-0.1953125\\
-0.0155175	-0.1678475\\
-0.01561	-0.1342775\\
-0.01529	-0.094605\\
-0.0148775	-0.0732425\\
-0.014375	-0.0579825\\
-0.013915	-0.10376\\
-0.0142825	-0.0915525\\
-0.014465	-0.079345\\
-0.0142375	-0.061035\\
-0.013825	-0.079345\\
-0.013825	-0.1159675\\
-0.0143275	-0.128175\\
-0.0146475	-0.1586925\\
-0.01497	-0.183105\\
-0.01529	-0.1800525\\
-0.015335	-0.1861575\\
-0.0154275	-0.1190175\\
-0.0149225	-0.0579825\\
-0.0140525	-0.0885\\
-0.0139625	-0.0671375\\
-0.0139625	-0.094605\\
-0.0140525	-0.131225\\
-0.0146025	-0.146485\\
-0.0148775	-0.2136225\\
-0.0154725	-0.14038\\
-0.0152425	-0.1434325\\
-0.01506	-0.1861575\\
-0.0154275	-0.1434325\\
-0.0152425	-0.10376\\
-0.0148325	-0.08545\\
-0.014465	-0.112915\\
-0.01451	-0.13733\\
-0.0148775	-0.198365\\
-0.01538	-0.1342775\\
-0.0151525	-0.1190175\\
-0.0148325	-0.112915\\
-0.0148325	-0.1251225\\
-0.0149225	-0.1708975\\
-0.0152425	-0.2594\\
-0.01593	-0.2227775\\
-0.0161125	-0.22583\\
-0.0160675	-0.149535\\
-0.01561	-0.1098625\\
-0.01497	-0.12207\\
-0.01497	-0.0549325\\
-0.01419	-0.0976575\\
-0.014145	-0.076295\\
-0.01419	-0.0915525\\
-0.01419	-0.146485\\
-0.01474	-0.19226\\
-0.015335	-0.216675\\
-0.0157025	-0.27771\\
-0.016205	-0.234985\\
-0.01625	-0.1342775\\
-0.0155175	-0.1342775\\
-0.0151525	-0.1068125\\
-0.015015	-0.1434325\\
-0.015105	-0.183105\\
-0.0155175	-0.2380375\\
-0.01593	-0.1800525\\
-0.0157925	-0.131225\\
-0.01529	-0.149535\\
-0.015335	-0.18921\\
-0.01561	-0.13733\\
-0.01538	-0.1007075\\
-0.01497	-0.0885\\
-0.0146025	-0.0640875\\
-0.0141	-0.061035\\
-0.01387	-0.0488275\\
-0.0134575	-0.0823975\\
-0.0137325	-0.10376\\
-0.01419	-0.1159675\\
-0.01442	-0.0915525\\
-0.0142375	-0.0885\\
-0.0141	-0.0488275\\
-0.0134125	-0.03357\\
-0.0125875	-0.05188\\
-0.0124975	-0.0732425\\
-0.0130925	-0.0976575\\
-0.0136875	-0.12207\\
-0.014145	-0.131225\\
-0.014465	-0.15564\\
-0.01474	-0.198365\\
-0.0152425	-0.268555\\
-0.0160225	-0.2594\\
-0.01625	-0.2807625\\
-0.0163875	-0.2471925\\
-0.0163875	-0.1525875\\
-0.015655	-0.13733\\
-0.01529	-0.1342775\\
-0.01529	-0.17395\\
-0.0155175	-0.149535\\
-0.0154275	-0.112915\\
-0.015105	-0.0915525\\
-0.014785	-0.079345\\
-0.01442	-0.1251225\\
-0.014695	-0.1190175\\
-0.0148325	-0.07019\\
-0.0142375	-0.0549325\\
-0.0137775	-0.0823975\\
-0.0139625	-0.1098625\\
-0.0142825	-0.0823975\\
-0.01419	-0.1159675\\
-0.01442	-0.0915525\\
-0.0143275	-0.128175\\
-0.0146475	-0.19226\\
-0.01529	-0.1525875\\
-0.0151975	-0.201415\\
-0.0154725	-0.26245\\
-0.0160675	-0.2838125\\
-0.0163425	-0.2197275\\
-0.01616	-0.149535\\
-0.01561	-0.112915\\
-0.0151525	-0.079345\\
-0.014465	-0.1007075\\
-0.0145575	-0.076295\\
-0.01442	-0.13733\\
-0.014785	-0.1190175\\
-0.0148775	-0.1007075\\
-0.0146475	-0.128175\\
-0.0148325	-0.0823975\\
-0.014375	-0.112915\\
-0.01451	-0.08545\\
-0.014375	-0.0579825\\
-0.013825	-0.0640875\\
-0.0137775	-0.0549325\\
-0.01355	-0.0579825\\
-0.01355	-0.061035\\
-0.0134575	-0.042725\\
-0.0131825	-0.0549325\\
-0.0131825	-0.0732425\\
-0.0134125	-0.076295\\
-0.013505	-0.0732425\\
-0.013505	-0.1098625\\
-0.0139625	-0.13733\\
-0.014465	-0.112915\\
-0.01442	-0.1068125\\
-0.0143275	-0.15564\\
-0.01474	-0.1861575\\
-0.0152425	-0.149535\\
-0.015105	-0.164795\\
-0.015105	-0.0823975\\
-0.01442	-0.0549325\\
-0.0136875	-0.1007075\\
-0.0140075	-0.14038\\
-0.0146475	-0.146485\\
-0.0148325	-0.20752\\
-0.015335	-0.1861575\\
-0.0154725	-0.1342775\\
-0.015105	-0.1007075\\
-0.0146475	-0.1068125\\
-0.01451	-0.1159675\\
-0.0146025	-0.0915525\\
-0.014465	-0.061035\\
-0.013915	-0.0823975\\
-0.013825	-0.0640875\\
-0.0136875	-0.0549325\\
-0.013505	-0.0457775\\
-0.0131375	-0.07019\\
-0.0133675	-0.1007075\\
-0.01387	-0.1098625\\
-0.0141	-0.079345\\
-0.01387	-0.131225\\
-0.014375	-0.1770025\\
-0.01497	-0.1190175\\
-0.014695	-0.1617425\\
-0.0148775	-0.1770025\\
-0.015105	-0.128175\\
-0.0148775	-0.08545\\
-0.014375	-0.1068125\\
-0.0143275	-0.1190175\\
-0.0145575	-0.0732425\\
-0.0140525	-0.08545\\
-0.0139625	-0.07019\\
-0.0137775	-0.0488275\\
-0.013275	-0.0488275\\
-0.0130925	-0.0732425\\
-0.01332	-0.0915525\\
-0.0137325	-0.0823975\\
-0.0136875	-0.0976575\\
-0.01387	-0.1068125\\
-0.0140525	-0.0823975\\
-0.013915	-0.05188\\
-0.01332	-0.0457775\\
-0.012955	-0.0549325\\
-0.0130475	-0.0640875\\
-0.0131375	-0.0732425\\
-0.01332	-0.13733\\
-0.0142375	-0.1342775\\
-0.0145575	-0.1190175\\
-0.01442	-0.1190175\\
-0.01442	-0.1525875\\
-0.0146475	-0.19226\\
-0.0151975	-0.20752\\
-0.0155175	-0.2136225\\
-0.015565	-0.1617425\\
-0.015335	-0.164795\\
-0.01529	-0.1678475\\
-0.01538	-0.17395\\
-0.015335	-0.19226\\
-0.0154725	-0.253295\\
-0.015885	-0.216675\\
-0.01593	-0.198365\\
-0.0157925	-0.1586925\\
-0.015565	-0.1586925\\
-0.0154275	-0.149535\\
-0.01538	-0.1770025\\
-0.0154725	-0.1159675\\
-0.01506	-0.1342775\\
-0.015015	-0.15564\\
-0.0152425	-0.1800525\\
-0.0154275	-0.14038\\
-0.0152425	-0.164795\\
-0.01529	-0.131225\\
-0.0151975	-0.12207\\
-0.015015	-0.0823975\\
-0.0145575	-0.128175\\
-0.014695	-0.183105\\
-0.01529	-0.1434325\\
-0.0152425	-0.0915525\\
-0.014695	-0.1098625\\
-0.014695	-0.07019\\
-0.0141	-0.0640875\\
-0.0137775	-0.0823975\\
-0.013915	-0.0579825\\
-0.0136875	-0.0640875\\
-0.0134575	-0.07019\\
-0.01355	-0.0488275\\
-0.01332	-0.0732425\\
-0.01355	-0.0640875\\
-0.0134575	-0.0488275\\
-0.0130925	-0.07019\\
-0.013275	-0.079345\\
-0.013505	-0.0885\\
-0.0137325	-0.0915525\\
-0.0137775	-0.0885\\
-0.0137775	-0.1159675\\
-0.014145	-0.1007075\\
-0.014145	-0.1190175\\
-0.0142375	-0.18921\\
-0.015015	-0.14038\\
-0.0149225	-0.1251225\\
-0.014695	-0.131225\\
-0.01474	-0.1007075\\
-0.014465	-0.0671375\\
-0.01387	-0.1068125\\
-0.01419	-0.0885\\
-0.014145	-0.0640875\\
-0.0136875	-0.094605\\
-0.0139625	-0.0915525\\
-0.0140525	-0.1068125\\
-0.0142375	-0.076295\\
-0.013915	-0.0549325\\
-0.0134125	-0.0457775\\
-0.0130475	-0.05188\\
-0.012955	-0.08545\\
-0.0134575	-0.079345\\
-0.013595	-0.0885\\
-0.0137325	-0.1190175\\
-0.014145	-0.1586925\\
-0.0146475	-0.12207\\
-0.0145575	-0.1586925\\
-0.0148325	-0.1861575\\
-0.0151525	-0.1617425\\
-0.01506	-0.149535\\
-0.015015	-0.24414\\
-0.015655	-0.1770025\\
-0.01561	-0.216675\\
-0.0157025	-0.19226\\
-0.0157025	-0.216675\\
-0.0157475	-0.2380375\\
-0.015975	-0.1525875\\
-0.0154725	-0.0976575\\
-0.01474	-0.0823975\\
-0.014375	-0.1190175\\
-0.0145575	-0.1434325\\
-0.0149225	-0.146485\\
-0.01497	-0.10376\\
-0.014695	-0.0640875\\
-0.0140525	-0.08545\\
-0.0140075	-0.1434325\\
-0.01474	-0.19226\\
-0.015335	-0.18921\\
-0.0154275	-0.12207\\
-0.0149225	-0.0976575\\
-0.01451	-0.12207\\
-0.0146475	-0.1800525\\
-0.0152425	-0.2044675\\
-0.0155175	-0.20752\\
-0.01561	-0.1525875\\
-0.015335	-0.20752\\
-0.015565	-0.22583\\
-0.0158375	-0.216675\\
-0.0158375	-0.286865\\
-0.01625	-0.2471925\\
-0.0162975	-0.2136225\\
-0.01616	-0.2471925\\
-0.016205	-0.2105725\\
-0.0161125	-0.128175\\
-0.0154275	-0.1190175\\
-0.01506	-0.14038\\
-0.0151975	-0.1678475\\
-0.01538	-0.1678475\\
-0.0154725	-0.128175\\
-0.0151975	-0.1586925\\
-0.015335	-0.14038\\
-0.01529	-0.1098625\\
-0.015015	-0.13733\\
-0.0151525	-0.131225\\
-0.015105	-0.18921\\
-0.0155175	-0.19226\\
-0.0157025	-0.1434325\\
-0.0154275	-0.112915\\
-0.01506	-0.076295\\
-0.01442	-0.1098625\\
-0.0145575	-0.0885\\
-0.014465	-0.079345\\
-0.0142375	-0.0671375\\
-0.0141	-0.1068125\\
-0.014375	-0.1342775\\
-0.01474	-0.146485\\
-0.0149225	-0.146485\\
-0.01497	-0.094605\\
-0.0146475	-0.076295\\
-0.0142375	-0.0579825\\
-0.0137325	-0.07019\\
-0.0137325	-0.1007075\\
-0.01419	-0.1068125\\
-0.0143275	-0.1007075\\
-0.0142825	-0.15564\\
-0.0148325	-0.183105\\
-0.0152425	-0.201415\\
-0.0155175	-0.2044675\\
-0.01561	-0.2380375\\
-0.0158375	-0.149535\\
-0.0154725	-0.1190175\\
-0.01506	-0.1007075\\
-0.014785	-0.1098625\\
-0.014695	-0.13733\\
-0.0149225	-0.1159675\\
-0.0148775	-0.08545\\
-0.014465	-0.076295\\
-0.0142375	-0.131225\\
-0.01474	-0.15564\\
-0.01506	-0.1434325\\
-0.015015	-0.2136225\\
-0.0155175	-0.2288825\\
-0.015885	-0.164795\\
-0.01561	-0.1190175\\
-0.0151525	-0.0915525\\
-0.0146475	-0.0823975\\
-0.01442	-0.14038\\
-0.0148325	-0.12207\\
-0.0148325	-0.1342775\\
-0.0149225	-0.149535\\
-0.015105	-0.1678475\\
-0.0151975	-0.1586925\\
-0.01529	-0.1800525\\
-0.015335	-0.128175\\
-0.0151525	-0.14038\\
-0.01506	-0.1068125\\
-0.014785	-0.0976575\\
-0.0145575	-0.13733\\
-0.0149225	-0.18921\\
-0.0154275	-0.2044675\\
-0.01561	-0.1190175\\
-0.015015	-0.2288825\\
-0.0155175	-0.1953125\\
-0.015885	-0.1342775\\
-0.015335	-0.08545\\
-0.014695	-0.1251225\\
-0.0148775	-0.112915\\
-0.0148775	-0.112915\\
-0.014785	-0.1342775\\
-0.0149225	-0.094605\\
-0.0146475	-0.12207\\
-0.0148325	-0.216675\\
-0.015565	-0.2288825\\
-0.015885	-0.1586925\\
-0.015565	-0.1770025\\
-0.0155175	-0.1953125\\
-0.01561	-0.19226\\
-0.0157025	-0.1434325\\
-0.01538	-0.14038\\
-0.01529	-0.10376\\
-0.0149225	-0.13733\\
-0.015105	-0.149535\\
-0.0151975	-0.2319325\\
-0.0157925	-0.253295\\
-0.01616	-0.32959\\
-0.0166175	-0.22583\\
-0.0164325	-0.19226\\
-0.0160675	-0.15564\\
-0.0157475	-0.1678475\\
-0.0157025	-0.131225\\
-0.0155175	-0.094605\\
-0.01497	-0.094605\\
-0.01474	-0.0976575\\
-0.01474	-0.079345\\
-0.01442	-0.0579825\\
-0.0139625	-0.08545\\
-0.014145	-0.112915\\
-0.014465	-0.0732425\\
-0.0142825	-0.0640875\\
-0.013915	-0.061035\\
-0.0137325	-0.1251225\\
-0.01442	-0.1678475\\
-0.01506	-0.0885\\
-0.014465	-0.10376\\
-0.014375	-0.1098625\\
-0.01451	-0.1007075\\
-0.01442	-0.1190175\\
-0.0146475	-0.112915\\
-0.0145575	-0.0823975\\
-0.0142375	-0.0640875\\
-0.013915	-0.0579825\\
-0.013595	-0.0671375\\
-0.0136425	-0.1190175\\
-0.0142825	-0.13733\\
-0.0146475	-0.10376\\
-0.014465	-0.076295\\
-0.0140525	-0.0579825\\
-0.01355	-0.07019\\
-0.013595	-0.1068125\\
-0.0141	-0.128175\\
-0.0145575	-0.14038\\
-0.0146475	-0.1007075\\
-0.014465	-0.0976575\\
-0.0142825	-0.1068125\\
-0.014375	-0.1434325\\
-0.014695	-0.19226\\
-0.0151975	-0.2227775\\
-0.015655	-0.17395\\
-0.0155175	-0.1159675\\
-0.01497	-0.12207\\
-0.0148775	-0.1159675\\
-0.014785	-0.1159675\\
-0.01474	-0.0885\\
-0.01442	-0.0976575\\
-0.01442	-0.1098625\\
-0.01451	-0.10376\\
-0.014465	-0.0732425\\
-0.014145	-0.05188\\
-0.013595	-0.0732425\\
-0.0136875	-0.1007075\\
-0.0141	-0.1434325\\
-0.0146475	-0.1525875\\
-0.0149225	-0.1861575\\
-0.0152425	-0.164795\\
-0.0152425	-0.183105\\
-0.01538	-0.24109\\
-0.0158375	-0.3173825\\
-0.01648	-0.2655025\\
-0.01648	-0.183105\\
-0.0160225	-0.201415\\
-0.01593	-0.2288825\\
-0.0161125	-0.305175\\
-0.0166175	-0.201415\\
-0.0162975	-0.1098625\\
-0.015335	-0.076295\\
-0.0146025	-0.079345\\
-0.0143275	-0.0640875\\
-0.0139625	-0.0457775\\
-0.01332	-0.0549325\\
-0.0131825	-0.0732425\\
-0.013505	-0.094605\\
-0.0139625	-0.094605\\
-0.0140525	-0.076295\\
-0.01387	-0.0732425\\
-0.013825	-0.0915525\\
-0.0139625	-0.1098625\\
-0.01419	-0.112915\\
-0.0143275	-0.1068125\\
-0.0142825	-0.0823975\\
-0.0140075	-0.061035\\
-0.0136425	-0.0579825\\
-0.0134575	-0.0885\\
-0.0136875	-0.0976575\\
-0.0139625	-0.076295\\
-0.0137775	-0.0579825\\
-0.0134575	-0.0885\\
-0.013825	-0.07019\\
-0.013595	-0.076295\\
-0.0136425	-0.094605\\
-0.013915	-0.1251225\\
-0.0143275	-0.13733\\
-0.0145575	-0.1007075\\
-0.0142375	-0.14038\\
-0.0145575	-0.13733\\
-0.014695	-0.1159675\\
-0.0145575	-0.094605\\
-0.014375	-0.146485\\
-0.01474	-0.183105\\
-0.0151975	-0.183105\\
-0.01529	-0.198365\\
-0.0154725	-0.15564\\
-0.0152425	-0.1434325\\
-0.015105	-0.13733\\
-0.01506	-0.1800525\\
-0.01529	-0.149535\\
-0.0151975	-0.1953125\\
-0.0154275	-0.1770025\\
-0.0155175	-0.1861575\\
-0.0154725	-0.31128\\
-0.0162975	-0.2471925\\
-0.0163875	-0.1342775\\
-0.015565	-0.094605\\
-0.0149225	-0.10376\\
-0.014695	-0.128175\\
-0.0148775	-0.1586925\\
-0.0151975	-0.1770025\\
-0.0154275	-0.1953125\\
-0.015565	-0.201415\\
-0.0157025	-0.13733\\
-0.01529	-0.1190175\\
-0.015105	-0.14038\\
-0.0151975	-0.14038\\
-0.0151975	-0.0915525\\
-0.01474	-0.07019\\
-0.0142375	-0.112915\\
-0.0146475	-0.1159675\\
-0.01474	-0.149535\\
-0.015015	-0.183105\\
-0.0154275	-0.1617425\\
-0.01538	-0.1190175\\
-0.01497	-0.1007075\\
-0.014695	-0.1434325\\
-0.01497	-0.1770025\\
-0.015335	-0.2288825\\
-0.0158375	-0.250245\\
-0.0160675	-0.2838125\\
-0.0163425	-0.2319325\\
-0.0162975	-0.2136225\\
-0.0161125	-0.131225\\
-0.015655	-0.1007075\\
-0.01497	-0.164795\\
-0.015335	-0.2136225\\
-0.0157925	-0.131225\\
-0.015335	-0.1861575\\
-0.015655	-0.26245\\
-0.01616	-0.2990725\\
-0.01657	-0.198365\\
-0.01616	-0.1159675\\
-0.0154275	-0.076295\\
-0.0146475	-0.1007075\\
-0.014695	-0.094605\\
-0.01474	-0.10376\\
-0.014695	-0.0732425\\
-0.0142825	-0.0885\\
-0.0142375	-0.076295\\
-0.0140525	-0.0885\\
-0.014145	-0.079345\\
-0.0141	-0.08545\\
-0.01419	-0.1342775\\
-0.01474	-0.1251225\\
-0.014785	-0.1678475\\
-0.015105	-0.201415\\
-0.0155175	-0.198365\\
-0.015655	-0.1159675\\
-0.015105	-0.1190175\\
-0.0148775	-0.1434325\\
-0.015105	-0.1007075\\
-0.014785	-0.0885\\
-0.0145575	-0.07019\\
-0.01419	-0.10376\\
-0.0143275	-0.0885\\
-0.014375	-0.0579825\\
-0.013915	-0.03662\\
-0.013275	-0.042725\\
-0.012955	-0.0671375\\
-0.0133675	-0.094605\\
-0.01387	-0.1007075\\
-0.0140525	-0.07019\\
-0.0137775	-0.079345\\
-0.0137325	-0.1159675\\
-0.01419	-0.1434325\\
-0.0146025	-0.10376\\
-0.01451	-0.10376\\
-0.0142825	-0.12207\\
-0.01442	-0.13733\\
-0.014695	-0.1251225\\
-0.0146475	-0.1434325\\
-0.014785	-0.1342775\\
-0.014785	-0.1007075\\
-0.014465	-0.079345\\
-0.01419	-0.1068125\\
-0.0143275	-0.0915525\\
-0.0142825	-0.1068125\\
-0.014375	-0.1434325\\
-0.01474	-0.20752\\
-0.01538	-0.1586925\\
-0.0152425	-0.1434325\\
-0.015105	-0.2136225\\
-0.015565	-0.1708975\\
-0.0154275	-0.10376\\
-0.0149225	-0.08545\\
-0.01442	-0.07019\\
-0.0140525	-0.07019\\
-0.0139625	-0.0671375\\
-0.0137775	-0.0640875\\
-0.0136875	-0.112915\\
-0.0142825	-0.07019\\
-0.0136875	-0.0579825\\
-0.01332	-0.0823975\\
-0.0136425	-0.10376\\
-0.0140525	-0.0823975\\
-0.013915	-0.0640875\\
-0.0136425	-0.042725\\
-0.0131825	-0.0640875\\
-0.013275	-0.1159675\\
-0.0140525	-0.149535\\
-0.0146475	-0.1251225\\
-0.0145575	-0.1007075\\
-0.0143275	-0.112915\\
-0.014375	-0.1159675\\
-0.014375	-0.1251225\\
-0.01451	-0.1525875\\
-0.014785	-0.1770025\\
-0.015105	-0.1708975\\
-0.0151975	-0.146485\\
-0.01506	-0.10376\\
-0.014695	-0.08545\\
-0.0143275	-0.0915525\\
-0.0142825	-0.12207\\
-0.0145575	-0.0885\\
-0.01419	-0.07019\\
-0.013915	-0.1251225\\
-0.01442	-0.1525875\\
-0.014785	-0.1434325\\
-0.0148325	-0.164795\\
-0.01497	-0.20752\\
-0.0154275	-0.12207\\
-0.0149225	-0.2044675\\
-0.01538	-0.19226\\
-0.015655	-0.183105\\
-0.0155175	-0.19226\\
-0.01561	-0.19226\\
-0.015565	-0.305175\\
-0.0163425	-0.2990725\\
-0.0167075	-0.198365\\
-0.016205	-0.234985\\
-0.0163425	-0.2899175\\
-0.01657	-0.2807625\\
-0.0166175	-0.31128\\
-0.016755	-0.2838125\\
-0.016755	-0.357055\\
-0.017075	-0.2899175\\
-0.0169825	-0.198365\\
-0.016525	-0.1342775\\
-0.0157925	-0.131225\\
-0.0155175	-0.112915\\
-0.015335	-0.094605\\
-0.0149225	-0.131225\\
-0.0151975	-0.183105\\
-0.015655	-0.112915\\
-0.0152425	-0.10376\\
-0.0149225	-0.1068125\\
-0.0148775	-0.079345\\
-0.014465	-0.08545\\
-0.014465	-0.0640875\\
-0.01387	-0.061035\\
-0.013595	-0.0579825\\
-0.01355	-0.0732425\\
-0.0136875	-0.061035\\
-0.013595	-0.05188\\
-0.01332	-0.042725\\
-0.01291	-0.03357\\
-0.012635	-0.061035\\
-0.01291	-0.05188\\
-0.0128175	-0.0457775\\
-0.0127725	-0.094605\\
-0.0134575	-0.128175\\
-0.0141	-0.1251225\\
-0.0143275	-0.1007075\\
-0.0140525	-0.12207\\
-0.01419	-0.1678475\\
-0.0148325	-0.08545\\
-0.0142825	-0.061035\\
-0.01355	-0.0488275\\
-0.01323	-0.03662\\
-0.0128175	-0.05188\\
-0.0130475	-0.0915525\\
-0.013595	-0.1251225\\
-0.01419	-0.0915525\\
-0.0139625	-0.1190175\\
-0.014375	-0.1861575\\
-0.015015	-0.183105\\
-0.0152425	-0.1342775\\
-0.01506	-0.094605\\
-0.01442	-0.1098625\\
-0.01442	-0.149535\\
-0.0148325	-0.183105\\
-0.0151975	-0.1159675\\
-0.01474	-0.0640875\\
-0.0139625	-0.0915525\\
-0.0141	-0.08545\\
-0.0139625	-0.042725\\
-0.01332	-0.0305175\\
-0.0125875	-0.0671375\\
-0.0131375	-0.1342775\\
-0.014145	-0.1342775\\
-0.0145575	-0.1098625\\
-0.0143275	-0.0732425\\
-0.013825	-0.07019\\
-0.013595	-0.042725\\
-0.01268	-0.0457775\\
-0.0125425	-0.0549325\\
-0.01268	-0.0671375\\
-0.012955	-0.0915525\\
-0.0134575	-0.13733\\
-0.0142375	-0.1434325\\
-0.0146025	-0.149535\\
-0.01474	-0.1586925\\
-0.0148775	-0.15564\\
-0.0148775	-0.149535\\
-0.0149225	-0.1342775\\
-0.01474	-0.1617425\\
-0.0149225	-0.234985\\
-0.015655	-0.216675\\
-0.015885	-0.1251225\\
-0.015105	-0.0732425\\
-0.014145	-0.131225\\
-0.0146475	-0.1678475\\
-0.0151525	-0.149535\\
-0.01506	-0.094605\\
-0.01451	-0.094605\\
-0.0142375	-0.149535\\
-0.0148325	-0.2227775\\
-0.015565	-0.2227775\\
-0.0158375	-0.2471925\\
-0.0160675	-0.24414\\
-0.01616	-0.183105\\
-0.01593	-0.19226\\
-0.0157925	-0.1098625\\
-0.015015	-0.0915525\\
-0.01451	-0.112915\\
-0.014695	-0.13733\\
-0.0148775	-0.1434325\\
-0.015015	-0.1525875\\
-0.015105	-0.1068125\\
-0.01474	-0.128175\\
-0.0148775	-0.12207\\
-0.0148775	-0.0732425\\
-0.0142375	-0.0549325\\
-0.013595	-0.042725\\
-0.0131825	-0.0640875\\
-0.0133675	-0.0579825\\
-0.013275	-0.03357\\
-0.0127725	-0.0640875\\
-0.0130475	-0.1159675\\
-0.013915	-0.0885\\
-0.013825	-0.076295\\
-0.013595	-0.1007075\\
-0.01387	-0.1525875\\
-0.0145575	-0.12207\\
-0.01451	-0.061035\\
-0.0137775	-0.0457775\\
-0.0130925	-0.0823975\\
-0.0136425	-0.1708975\\
-0.014695	-0.2044675\\
-0.015335	-0.2746575\\
-0.0161125	-0.3265375\\
-0.0167075	-0.3173825\\
-0.0168925	-0.2197275\\
-0.01648	-0.2197275\\
-0.0162975	-0.2746575\\
-0.01657	-0.2380375\\
-0.016525	-0.2227775\\
-0.0163875	-0.2471925\\
-0.016525	-0.268555\\
-0.01657	-0.2716075\\
-0.0166625	-0.36316\\
-0.01712	-0.2563475\\
-0.0168925	-0.1434325\\
-0.0160675	-0.08545\\
-0.0151525	-0.0732425\\
-0.01451	-0.0823975\\
-0.01442	-0.0823975\\
-0.0143275	-0.079345\\
-0.0143275	-0.1342775\\
-0.0148325	-0.112915\\
-0.0148325	-0.112915\\
-0.01474	-0.149535\\
-0.015015	-0.08545\\
-0.01451	-0.094605\\
-0.01442	-0.128175\\
-0.014785	-0.1434325\\
-0.01506	-0.0885\\
-0.0146475	-0.061035\\
-0.0140075	-0.079345\\
-0.014145	-0.079345\\
-0.0141	-0.146485\\
-0.01506	-0.2288825\\
-0.01593	-0.216675\\
-0.01593	-0.24109\\
-0.01616	-0.27771\\
-0.01648	-0.3509525\\
-0.01703	-0.302125\\
-0.01712	-0.201415\\
-0.016525	-0.13733\\
-0.015885	-0.08545\\
-0.0149225	-0.0976575\\
-0.0146025	-0.0976575\\
-0.0146475	-0.1190175\\
-0.0148775	-0.0915525\\
-0.014695	-0.0823975\\
-0.01442	-0.0915525\\
-0.014465	-0.1098625\\
-0.0146475	-0.1190175\\
-0.0148325	-0.146485\\
-0.015105	-0.112915\\
-0.0148325	-0.0579825\\
-0.01419	-0.0457775\\
-0.01355	-0.0457775\\
-0.0131825	-0.042725\\
-0.013	-0.0549325\\
-0.0130475	-0.0823975\\
-0.013595	-0.0885\\
-0.0137775	-0.1068125\\
-0.0140075	-0.1434325\\
-0.0146025	-0.112915\\
-0.01451	-0.1586925\\
-0.01497	-0.24414\\
-0.015885	-0.1953125\\
-0.01593	-0.1708975\\
-0.0157025	-0.2105725\\
-0.01593	-0.24109\\
-0.01616	-0.15564\\
-0.0157475	-0.146485\\
-0.0154275	-0.1007075\\
-0.0148775	-0.10376\\
-0.01474	-0.19226\\
-0.0155175	-0.31128\\
-0.01657	-0.3692625\\
-0.0172575	-0.2899175\\
-0.01712	-0.2471925\\
-0.016845	-0.18921\\
-0.0164325	-0.201415\\
-0.0162975	-0.2594\\
-0.0166625	-0.1586925\\
-0.0161125	-0.128175\\
-0.01561	-0.10376\\
-0.015335	-0.1800525\\
-0.0157925	-0.1800525\\
-0.0160675	-0.1007075\\
-0.01538	-0.10376\\
-0.015015	-0.08545\\
-0.014695	-0.1251225\\
-0.015015	-0.19226\\
-0.015655	-0.18921\\
-0.0158375	-0.146485\\
-0.01561	-0.1159675\\
-0.0152425	-0.131225\\
-0.0152425	-0.10376\\
-0.015105	-0.076295\\
-0.0146475	-0.0671375\\
-0.0143275	-0.061035\\
-0.0140075	-0.05188\\
-0.0136875	-0.0579825\\
-0.013595	-0.094605\\
-0.0140075	-0.128175\\
-0.0146475	-0.128175\\
-0.014785	-0.079345\\
-0.014465	-0.0732425\\
-0.014145	-0.0671375\\
-0.0140525	-0.0823975\\
-0.0140075	-0.112915\\
-0.014375	-0.1190175\\
-0.01451	-0.112915\\
-0.0145575	-0.07019\\
-0.0140525	-0.1190175\\
-0.0146025	-0.17395\\
-0.015335	-0.1342775\\
-0.0151525	-0.0671375\\
-0.01419	-0.0671375\\
-0.013915	-0.10376\\
-0.01442	-0.1068125\\
-0.01451	-0.094605\\
-0.014375	-0.0671375\\
-0.013915	-0.10376\\
-0.0142825	-0.15564\\
-0.0148775	-0.10376\\
-0.0145575	-0.042725\\
-0.0136425	-0.0640875\\
-0.0134125	-0.128175\\
-0.0143275	-0.13733\\
-0.0148775	-0.0976575\\
-0.01442	-0.0823975\\
-0.014145	-0.146485\\
-0.014695	-0.19226\\
-0.015335	-0.1617425\\
-0.01538	-0.2197275\\
-0.0157025	-0.2716075\\
-0.01625	-0.1800525\\
-0.01593	-0.149535\\
-0.0155175	-0.201415\\
-0.015885	-0.13733\\
-0.0155175	-0.183105\\
-0.015655	-0.2197275\\
-0.0160225	-0.17395\\
-0.015885	-0.0915525\\
-0.01506	-0.146485\\
-0.0152425	-0.250245\\
-0.0161125	-0.286865\\
-0.01657	-0.20752\\
-0.0163425	-0.1800525\\
-0.0160675	-0.2319325\\
-0.01625	-0.1861575\\
-0.0161125	-0.12207\\
-0.0155175	-0.112915\\
-0.01529	-0.1434325\\
-0.01538	-0.146485\\
-0.0154725	-0.1434325\\
-0.0155175	-0.1617425\\
-0.015565	-0.12207\\
-0.01529	-0.1251225\\
-0.0152425	-0.1708975\\
-0.0155175	-0.1098625\\
-0.0151975	-0.0885\\
-0.01474	-0.076295\\
-0.01451	-0.0579825\\
-0.014145	-0.0671375\\
-0.0140525	-0.1159675\\
-0.0145575	-0.10376\\
-0.0146475	-0.1434325\\
-0.01497	-0.17395\\
-0.01538	-0.20752\\
-0.0157475	-0.1586925\\
-0.015565	-0.1434325\\
-0.01538	-0.0640875\\
-0.014695	-0.1190175\\
-0.0146475	-0.1800525\\
-0.01538	-0.128175\\
-0.0151975	-0.0671375\\
-0.01442	-0.128175\\
-0.014785	-0.1861575\\
-0.0154275	-0.1861575\\
-0.01561	-0.2319325\\
-0.0160225	-0.302125\\
-0.0166175	-0.2197275\\
-0.0164325	-0.1861575\\
-0.0161125	-0.216675\\
-0.01625	-0.1525875\\
-0.015885	-0.1190175\\
-0.0154275	-0.1342775\\
-0.0154725	-0.1708975\\
-0.0157475	-0.1800525\\
-0.0158375	-0.1800525\\
-0.015885	-0.112915\\
-0.015335	-0.0976575\\
-0.01497	-0.0823975\\
-0.014785	-0.1159675\\
-0.0149225	-0.1098625\\
-0.0149225	-0.08545\\
-0.014695	-0.1586925\\
-0.0152425	-0.1708975\\
-0.015565	-0.1251225\\
-0.0152425	-0.1007075\\
-0.01497	-0.146485\\
-0.0152425	-0.0823975\\
-0.014785	-0.0549325\\
-0.014145	-0.07019\\
-0.0140525	-0.0885\\
-0.0143275	-0.08545\\
-0.0143275	-0.0640875\\
-0.0139625	-0.0457775\\
-0.01332	-0.07019\\
-0.0134125	-0.094605\\
-0.013915	-0.1342775\\
-0.0145575	-0.149535\\
-0.015015	-0.18921\\
-0.01538	-0.216675\\
-0.0157925	-0.12207\\
-0.015335	-0.0976575\\
-0.014785	-0.19226\\
-0.0155175	-0.2746575\\
-0.0162975	-0.20752\\
-0.016205	-0.1953125\\
-0.0160225	-0.2838125\\
-0.016525	-0.2319325\\
-0.01648	-0.1861575\\
-0.01616	-0.1434325\\
-0.0157475	-0.146485\\
-0.015655	-0.18921\\
-0.01593	-0.131225\\
-0.01561	-0.112915\\
-0.01529	-0.18921\\
-0.0157475	-0.2380375\\
-0.016205	-0.24109\\
-0.0163875	-0.12207\\
-0.015565	-0.0640875\\
-0.01451	-0.14038\\
-0.0151525	-0.1190175\\
-0.0151975	-0.0549325\\
-0.0142825	-0.05188\\
-0.0137325	-0.079345\\
-0.013915	-0.10376\\
-0.01442	-0.1678475\\
-0.015105	-0.1251225\\
-0.015105	-0.0976575\\
-0.01474	-0.07019\\
-0.014145	-0.0488275\\
-0.01355	-0.0640875\\
-0.013505	-0.0579825\\
-0.0134575	-0.0671375\\
-0.01355	-0.1068125\\
-0.014145	-0.0976575\\
-0.0142375	-0.061035\\
-0.0136425	-0.05188\\
-0.0133675	-0.0671375\\
-0.0134575	-0.076295\\
-0.0137325	-0.05188\\
-0.0134575	-0.079345\\
-0.013595	-0.061035\\
-0.0134575	-0.0488275\\
-0.0131375	-0.0915525\\
-0.0136875	-0.1190175\\
-0.0143275	-0.17395\\
-0.01506	-0.2319325\\
-0.0158375	-0.2044675\\
-0.01593	-0.1068125\\
-0.0151525	-0.0823975\\
-0.01451	-0.08545\\
-0.01442	-0.0549325\\
-0.0137775	-0.0549325\\
-0.0134125	-0.0885\\
-0.0139625	-0.0976575\\
-0.0142375	-0.094605\\
-0.01419	-0.1068125\\
-0.0143275	-0.12207\\
-0.014465	-0.131225\\
-0.0146025	-0.131225\\
-0.01474	-0.15564\\
-0.015015	-0.15564\\
-0.015105	-0.216675\\
-0.01561	-0.1190175\\
-0.0151525	-0.0640875\\
-0.01419	-0.061035\\
-0.013825	-0.1068125\\
-0.0142825	-0.079345\\
-0.0142825	-0.08545\\
-0.014145	-0.0457775\\
-0.0134575	-0.07019\\
-0.0134125	-0.0457775\\
-0.01323	-0.03357\\
-0.01268	-0.0457775\\
-0.012725	-0.079345\\
-0.01332	-0.0915525\\
-0.0137325	-0.128175\\
-0.0142825	-0.17395\\
-0.015015	-0.131225\\
-0.0148775	-0.0640875\\
-0.01387	-0.0976575\\
-0.0140525	-0.112915\\
-0.01442	-0.0640875\\
-0.013825	-0.0823975\\
-0.013825	-0.146485\\
-0.0145575	-0.1251225\\
-0.01474	-0.198365\\
-0.015335	-0.2319325\\
-0.01593	-0.1617425\\
-0.015565	-0.128175\\
-0.0151975	-0.1159675\\
-0.015015	-0.149535\\
-0.0151975	-0.201415\\
-0.015655	-0.146485\\
-0.0154275	-0.0976575\\
-0.0148325	-0.1190175\\
};
\addplot [color=mycolor2, line width=2.0pt, forget plot]
  table[row sep=crcr]{%
-0.015655	-0.015655\\
-0.0157025	-0.0157025\\
-0.0158375	-0.0158375\\
-0.01561	-0.01561\\
-0.0148775	-0.0148775\\
-0.014785	-0.014785\\
-0.0148775	-0.0148775\\
-0.0149225	-0.0149225\\
-0.0148775	-0.0148775\\
-0.0139625	-0.0139625\\
-0.0131825	-0.0131825\\
-0.0128625	-0.0128625\\
-0.0137325	-0.0137325\\
-0.01474	-0.01474\\
-0.015015	-0.015015\\
-0.0151525	-0.0151525\\
-0.0149225	-0.0149225\\
-0.0143275	-0.0143275\\
-0.01497	-0.01497\\
-0.014785	-0.014785\\
-0.014375	-0.014375\\
-0.0148325	-0.0148325\\
-0.014785	-0.014785\\
-0.014695	-0.014695\\
-0.015105	-0.015105\\
-0.0151975	-0.0151975\\
-0.0149225	-0.0149225\\
-0.0151975	-0.0151975\\
-0.015335	-0.015335\\
-0.0151525	-0.0151525\\
-0.0149225	-0.0149225\\
-0.014695	-0.014695\\
-0.0149225	-0.0149225\\
-0.01497	-0.01497\\
-0.015015	-0.015015\\
-0.0154725	-0.0154725\\
-0.0155175	-0.0155175\\
-0.015105	-0.015105\\
-0.0149225	-0.0149225\\
-0.014375	-0.014375\\
-0.014145	-0.014145\\
-0.0143275	-0.0143275\\
-0.01419	-0.01419\\
-0.014145	-0.014145\\
-0.01451	-0.01451\\
-0.014695	-0.014695\\
-0.0146475	-0.0146475\\
-0.0151975	-0.0151975\\
-0.015105	-0.015105\\
-0.01451	-0.01451\\
-0.014145	-0.014145\\
-0.0142375	-0.0142375\\
-0.014465	-0.014465\\
-0.0140075	-0.0140075\\
-0.0137325	-0.0137325\\
-0.0140075	-0.0140075\\
-0.013915	-0.013915\\
-0.0137325	-0.0137325\\
-0.01387	-0.01387\\
-0.0140075	-0.0140075\\
-0.0146025	-0.0146025\\
-0.01474	-0.01474\\
-0.0149225	-0.0149225\\
-0.015015	-0.015015\\
-0.01506	-0.01506\\
-0.01497	-0.01497\\
-0.0148775	-0.0148775\\
-0.014785	-0.014785\\
-0.014695	-0.014695\\
-0.01474	-0.01474\\
-0.01529	-0.01529\\
-0.015975	-0.015975\\
-0.0162975	-0.0162975\\
-0.0158375	-0.0158375\\
-0.0161125	-0.0161125\\
-0.016525	-0.016525\\
-0.0166625	-0.0166625\\
-0.0163875	-0.0163875\\
-0.0166625	-0.0166625\\
-0.017165	-0.017165\\
-0.0169825	-0.0169825\\
-0.0166625	-0.0166625\\
-0.01616	-0.01616\\
-0.01561	-0.01561\\
-0.015335	-0.015335\\
-0.01538	-0.01538\\
-0.015105	-0.015105\\
-0.014465	-0.014465\\
-0.014145	-0.014145\\
-0.0142375	-0.0142375\\
-0.014695	-0.014695\\
-0.01474	-0.01474\\
-0.015015	-0.015015\\
-0.0151525	-0.0151525\\
-0.015565	-0.015565\\
-0.01561	-0.01561\\
-0.0157025	-0.0157025\\
-0.0154725	-0.0154725\\
-0.0152425	-0.0152425\\
-0.015335	-0.015335\\
-0.015105	-0.015105\\
-0.0149225	-0.0149225\\
-0.015015	-0.015015\\
-0.01506	-0.01506\\
-0.014785	-0.014785\\
-0.014695	-0.014695\\
-0.01451	-0.01451\\
-0.01442	-0.01442\\
-0.01419	-0.01419\\
-0.01442	-0.01442\\
-0.0149225	-0.0149225\\
-0.0151525	-0.0151525\\
-0.01529	-0.01529\\
-0.0148325	-0.0148325\\
-0.01497	-0.01497\\
-0.015655	-0.015655\\
-0.015565	-0.015565\\
-0.01561	-0.01561\\
-0.0157025	-0.0157025\\
-0.0151525	-0.0151525\\
-0.0146475	-0.0146475\\
-0.01474	-0.01474\\
-0.0149225	-0.0149225\\
-0.015565	-0.015565\\
-0.0160225	-0.0160225\\
-0.01616	-0.01616\\
-0.015885	-0.015885\\
-0.0158375	-0.0158375\\
-0.0157025	-0.0157025\\
-0.015655	-0.015655\\
-0.0152425	-0.0152425\\
-0.0149225	-0.0149225\\
-0.014785	-0.014785\\
-0.0146025	-0.0146025\\
-0.0146475	-0.0146475\\
-0.015015	-0.015015\\
-0.01561	-0.01561\\
-0.0155175	-0.0155175\\
-0.0151525	-0.0151525\\
-0.0148775	-0.0148775\\
-0.01474	-0.01474\\
-0.0142375	-0.0142375\\
-0.0136875	-0.0136875\\
-0.0136425	-0.0136425\\
-0.014145	-0.014145\\
-0.0141	-0.0141\\
-0.0140525	-0.0140525\\
-0.014145	-0.014145\\
-0.0141	-0.0141\\
-0.013825	-0.013825\\
-0.0134125	-0.0134125\\
-0.0130475	-0.0130475\\
-0.013275	-0.013275\\
-0.01419	-0.01419\\
-0.01497	-0.01497\\
-0.015335	-0.015335\\
-0.015015	-0.015015\\
-0.01451	-0.01451\\
-0.0141	-0.0141\\
-0.013595	-0.013595\\
-0.0141	-0.0141\\
-0.0142825	-0.0142825\\
-0.0146025	-0.0146025\\
-0.01506	-0.01506\\
-0.01593	-0.01593\\
-0.016525	-0.016525\\
-0.0164325	-0.0164325\\
-0.0163875	-0.0163875\\
-0.015885	-0.015885\\
-0.0158375	-0.0158375\\
-0.015885	-0.015885\\
-0.0160675	-0.0160675\\
-0.0157925	-0.0157925\\
-0.01561	-0.01561\\
-0.015565	-0.015565\\
-0.0154275	-0.0154275\\
-0.0155175	-0.0155175\\
-0.01529	-0.01529\\
-0.0157475	-0.0157475\\
-0.0160675	-0.0160675\\
-0.0157925	-0.0157925\\
-0.0151975	-0.0151975\\
-0.015015	-0.015015\\
-0.015565	-0.015565\\
-0.015975	-0.015975\\
-0.0163425	-0.0163425\\
-0.01657	-0.01657\\
-0.0167075	-0.0167075\\
-0.0163875	-0.0163875\\
-0.01625	-0.01625\\
-0.0163425	-0.0163425\\
-0.016525	-0.016525\\
-0.0161125	-0.0161125\\
-0.01538	-0.01538\\
-0.0154725	-0.0154725\\
-0.0154275	-0.0154275\\
-0.0148775	-0.0148775\\
-0.0149225	-0.0149225\\
-0.01506	-0.01506\\
-0.0149225	-0.0149225\\
-0.0146475	-0.0146475\\
-0.01474	-0.01474\\
-0.014695	-0.014695\\
-0.014785	-0.014785\\
-0.01529	-0.01529\\
-0.015335	-0.015335\\
-0.015015	-0.015015\\
-0.0149225	-0.0149225\\
-0.0152425	-0.0152425\\
-0.0160675	-0.0160675\\
-0.015975	-0.015975\\
-0.01538	-0.01538\\
-0.0148775	-0.0148775\\
-0.0148325	-0.0148325\\
-0.01451	-0.01451\\
-0.01442	-0.01442\\
-0.0146475	-0.0146475\\
-0.0148325	-0.0148325\\
-0.015015	-0.015015\\
-0.014695	-0.014695\\
-0.015105	-0.015105\\
-0.0155175	-0.0155175\\
-0.015335	-0.015335\\
-0.0157025	-0.0157025\\
-0.01538	-0.01538\\
-0.014465	-0.014465\\
-0.0143275	-0.0143275\\
-0.01442	-0.01442\\
-0.01474	-0.01474\\
-0.014465	-0.014465\\
-0.01474	-0.01474\\
-0.0148325	-0.0148325\\
-0.0145575	-0.0145575\\
-0.014695	-0.014695\\
-0.0146475	-0.0146475\\
-0.01451	-0.01451\\
-0.0145575	-0.0145575\\
-0.01419	-0.01419\\
-0.014145	-0.014145\\
-0.01387	-0.01387\\
-0.0137775	-0.0137775\\
-0.01442	-0.01442\\
-0.014695	-0.014695\\
-0.01506	-0.01506\\
-0.01561	-0.01561\\
-0.01538	-0.01538\\
-0.014695	-0.014695\\
-0.0142375	-0.0142375\\
-0.0140525	-0.0140525\\
-0.014375	-0.014375\\
-0.0140525	-0.0140525\\
-0.0139625	-0.0139625\\
-0.01419	-0.01419\\
-0.01474	-0.01474\\
-0.0149225	-0.0149225\\
-0.015335	-0.015335\\
-0.015105	-0.015105\\
-0.0148775	-0.0148775\\
-0.0142825	-0.0142825\\
-0.013915	-0.013915\\
-0.01355	-0.01355\\
-0.01332	-0.01332\\
-0.0134575	-0.0134575\\
-0.0140075	-0.0140075\\
-0.0142375	-0.0142375\\
-0.01442	-0.01442\\
-0.0145575	-0.0145575\\
-0.0152425	-0.0152425\\
-0.01538	-0.01538\\
-0.01506	-0.01506\\
-0.0151525	-0.0151525\\
-0.01529	-0.01529\\
-0.01561	-0.01561\\
-0.015565	-0.015565\\
-0.015015	-0.015015\\
-0.0142375	-0.0142375\\
-0.0136425	-0.0136425\\
-0.0134575	-0.0134575\\
-0.013505	-0.013505\\
-0.01355	-0.01355\\
-0.0134575	-0.0134575\\
-0.0136425	-0.0136425\\
-0.01419	-0.01419\\
-0.0143275	-0.0143275\\
-0.0142825	-0.0142825\\
-0.014465	-0.014465\\
-0.0141	-0.0141\\
-0.013505	-0.013505\\
-0.013825	-0.013825\\
-0.01451	-0.01451\\
-0.0146025	-0.0146025\\
-0.014785	-0.014785\\
-0.014695	-0.014695\\
-0.014375	-0.014375\\
-0.0145575	-0.0145575\\
-0.01506	-0.01506\\
-0.0152425	-0.0152425\\
-0.01497	-0.01497\\
-0.01538	-0.01538\\
-0.015335	-0.015335\\
-0.0151525	-0.0151525\\
-0.015335	-0.015335\\
-0.01538	-0.01538\\
-0.0151525	-0.0151525\\
-0.01497	-0.01497\\
-0.0154275	-0.0154275\\
-0.0160225	-0.0160225\\
-0.015975	-0.015975\\
-0.01529	-0.01529\\
-0.015015	-0.015015\\
-0.0151975	-0.0151975\\
-0.01538	-0.01538\\
-0.0151975	-0.0151975\\
-0.015105	-0.015105\\
-0.0154275	-0.0154275\\
-0.01561	-0.01561\\
-0.01529	-0.01529\\
-0.0149225	-0.0149225\\
-0.01506	-0.01506\\
-0.015655	-0.015655\\
-0.0157475	-0.0157475\\
-0.0152425	-0.0152425\\
-0.01497	-0.01497\\
-0.0149225	-0.0149225\\
-0.01442	-0.01442\\
-0.0137775	-0.0137775\\
-0.01332	-0.01332\\
-0.013	-0.013\\
-0.013595	-0.013595\\
-0.01419	-0.01419\\
-0.0145575	-0.0145575\\
-0.01442	-0.01442\\
-0.0146025	-0.0146025\\
-0.01419	-0.01419\\
-0.0140525	-0.0140525\\
-0.0137775	-0.0137775\\
-0.01387	-0.01387\\
-0.01442	-0.01442\\
-0.0148325	-0.0148325\\
-0.01442	-0.01442\\
-0.0140525	-0.0140525\\
-0.01387	-0.01387\\
-0.0139625	-0.0139625\\
-0.0136425	-0.0136425\\
-0.0142375	-0.0142375\\
-0.0151975	-0.0151975\\
-0.0151525	-0.0151525\\
-0.0154725	-0.0154725\\
-0.0158375	-0.0158375\\
-0.015975	-0.015975\\
-0.0157475	-0.0157475\\
-0.015335	-0.015335\\
-0.0152425	-0.0152425\\
-0.0151525	-0.0151525\\
-0.01529	-0.01529\\
-0.0155175	-0.0155175\\
-0.015565	-0.015565\\
-0.015975	-0.015975\\
-0.0162975	-0.0162975\\
-0.01657	-0.01657\\
-0.0162975	-0.0162975\\
-0.01648	-0.01648\\
-0.016205	-0.016205\\
-0.0162975	-0.0162975\\
-0.016205	-0.016205\\
-0.0155175	-0.0155175\\
-0.0148325	-0.0148325\\
-0.0145575	-0.0145575\\
-0.0149225	-0.0149225\\
-0.014785	-0.014785\\
-0.014375	-0.014375\\
-0.01442	-0.01442\\
-0.0142825	-0.0142825\\
-0.013915	-0.013915\\
-0.0140525	-0.0140525\\
-0.013825	-0.013825\\
-0.0142825	-0.0142825\\
-0.01474	-0.01474\\
-0.0145575	-0.0145575\\
-0.01506	-0.01506\\
-0.0158375	-0.0158375\\
-0.015885	-0.015885\\
-0.01625	-0.01625\\
-0.015975	-0.015975\\
-0.01593	-0.01593\\
-0.015565	-0.015565\\
-0.0148775	-0.0148775\\
-0.0149225	-0.0149225\\
-0.014695	-0.014695\\
-0.0143275	-0.0143275\\
-0.014465	-0.014465\\
-0.01419	-0.01419\\
-0.01355	-0.01355\\
-0.0134125	-0.0134125\\
-0.0141	-0.0141\\
-0.0146475	-0.0146475\\
-0.01497	-0.01497\\
-0.015105	-0.015105\\
-0.01506	-0.01506\\
-0.0152425	-0.0152425\\
-0.01506	-0.01506\\
-0.0148775	-0.0148775\\
-0.0148325	-0.0148325\\
-0.0146025	-0.0146025\\
-0.01497	-0.01497\\
-0.014785	-0.014785\\
-0.01442	-0.01442\\
-0.014695	-0.014695\\
-0.0151525	-0.0151525\\
-0.015015	-0.015015\\
-0.01474	-0.01474\\
-0.01451	-0.01451\\
-0.0145575	-0.0145575\\
-0.0142825	-0.0142825\\
-0.0140075	-0.0140075\\
-0.01419	-0.01419\\
-0.01451	-0.01451\\
-0.014695	-0.014695\\
-0.0145575	-0.0145575\\
-0.014695	-0.014695\\
-0.0142375	-0.0142375\\
-0.0141	-0.0141\\
-0.01474	-0.01474\\
-0.015335	-0.015335\\
-0.0155175	-0.0155175\\
-0.015335	-0.015335\\
-0.01529	-0.01529\\
-0.015015	-0.015015\\
-0.0148775	-0.0148775\\
-0.0146475	-0.0146475\\
-0.0148775	-0.0148775\\
-0.0148325	-0.0148325\\
-0.01442	-0.01442\\
-0.0146025	-0.0146025\\
-0.014785	-0.014785\\
-0.015015	-0.015015\\
-0.0151975	-0.0151975\\
-0.0152425	-0.0152425\\
-0.015655	-0.015655\\
-0.0161125	-0.0161125\\
-0.01648	-0.01648\\
-0.0160675	-0.0160675\\
-0.0152425	-0.0152425\\
-0.0145575	-0.0145575\\
-0.0140075	-0.0140075\\
-0.0140525	-0.0140525\\
-0.0145575	-0.0145575\\
-0.0146025	-0.0146025\\
-0.014465	-0.014465\\
-0.01474	-0.01474\\
-0.0149225	-0.0149225\\
-0.0151975	-0.0151975\\
-0.01538	-0.01538\\
-0.01593	-0.01593\\
-0.0160225	-0.0160225\\
-0.0157925	-0.0157925\\
-0.01538	-0.01538\\
-0.0151525	-0.0151525\\
-0.01538	-0.01538\\
-0.0151525	-0.0151525\\
-0.015015	-0.015015\\
-0.01497	-0.01497\\
-0.014375	-0.014375\\
-0.0145575	-0.0145575\\
-0.0151525	-0.0151525\\
-0.0149225	-0.0149225\\
-0.0155175	-0.0155175\\
-0.0154725	-0.0154725\\
-0.015335	-0.015335\\
-0.0157475	-0.0157475\\
-0.015335	-0.015335\\
-0.014785	-0.014785\\
-0.0145575	-0.0145575\\
-0.015015	-0.015015\\
-0.0157025	-0.0157025\\
-0.0160675	-0.0160675\\
-0.0161125	-0.0161125\\
-0.015885	-0.015885\\
-0.01538	-0.01538\\
-0.01506	-0.01506\\
-0.0146475	-0.0146475\\
-0.01442	-0.01442\\
-0.0142825	-0.0142825\\
-0.014465	-0.014465\\
-0.0146025	-0.0146025\\
-0.0142375	-0.0142375\\
-0.014465	-0.014465\\
-0.0148325	-0.0148325\\
-0.01506	-0.01506\\
-0.01538	-0.01538\\
-0.015885	-0.015885\\
-0.0162975	-0.0162975\\
-0.0160675	-0.0160675\\
-0.0158375	-0.0158375\\
-0.0157925	-0.0157925\\
-0.015655	-0.015655\\
-0.01529	-0.01529\\
-0.0152425	-0.0152425\\
-0.0158375	-0.0158375\\
-0.01616	-0.01616\\
-0.0157025	-0.0157025\\
-0.0149225	-0.0149225\\
-0.0142375	-0.0142375\\
-0.0133675	-0.0133675\\
-0.0130925	-0.0130925\\
-0.01355	-0.01355\\
-0.013825	-0.013825\\
-0.014145	-0.014145\\
-0.0141	-0.0141\\
-0.013825	-0.013825\\
-0.0137325	-0.0137325\\
-0.0136875	-0.0136875\\
-0.0137325	-0.0137325\\
-0.0141	-0.0141\\
-0.01474	-0.01474\\
-0.015015	-0.015015\\
-0.01506	-0.01506\\
-0.01529	-0.01529\\
-0.015655	-0.015655\\
-0.015885	-0.015885\\
-0.01616	-0.01616\\
-0.01625	-0.01625\\
-0.01593	-0.01593\\
-0.0154275	-0.0154275\\
-0.0152425	-0.0152425\\
-0.0157025	-0.0157025\\
-0.0160675	-0.0160675\\
-0.0157475	-0.0157475\\
-0.01529	-0.01529\\
-0.0142825	-0.0142825\\
-0.013595	-0.013595\\
-0.0134125	-0.0134125\\
-0.012955	-0.012955\\
-0.01332	-0.01332\\
-0.0137775	-0.0137775\\
-0.013915	-0.013915\\
-0.01387	-0.01387\\
-0.0134575	-0.0134575\\
-0.0131375	-0.0131375\\
-0.0131825	-0.0131825\\
-0.013275	-0.013275\\
-0.0134125	-0.0134125\\
-0.013275	-0.013275\\
-0.0130475	-0.0130475\\
-0.0134125	-0.0134125\\
-0.0146475	-0.0146475\\
-0.015335	-0.015335\\
-0.015565	-0.015565\\
-0.015335	-0.015335\\
-0.0155175	-0.0155175\\
-0.016205	-0.016205\\
-0.0163425	-0.0163425\\
-0.0163875	-0.0163875\\
-0.0161125	-0.0161125\\
-0.0160225	-0.0160225\\
-0.0161125	-0.0161125\\
-0.0157025	-0.0157025\\
-0.0154275	-0.0154275\\
-0.0148775	-0.0148775\\
-0.014465	-0.014465\\
-0.015105	-0.015105\\
-0.0149225	-0.0149225\\
-0.014785	-0.014785\\
-0.0149225	-0.0149225\\
-0.0148325	-0.0148325\\
-0.0148775	-0.0148775\\
-0.0149225	-0.0149225\\
-0.01506	-0.01506\\
-0.0151525	-0.0151525\\
-0.015105	-0.015105\\
-0.014785	-0.014785\\
-0.0149225	-0.0149225\\
-0.0143275	-0.0143275\\
-0.01323	-0.01323\\
-0.0131375	-0.0131375\\
-0.013595	-0.013595\\
-0.01323	-0.01323\\
-0.01245	-0.01245\\
-0.0125875	-0.0125875\\
-0.01323	-0.01323\\
-0.013595	-0.013595\\
-0.0139625	-0.0139625\\
-0.0140075	-0.0140075\\
-0.014375	-0.014375\\
-0.0145575	-0.0145575\\
-0.01419	-0.01419\\
-0.01442	-0.01442\\
-0.0140075	-0.0140075\\
-0.01355	-0.01355\\
-0.0131825	-0.0131825\\
-0.012725	-0.012725\\
-0.01291	-0.01291\\
-0.0128175	-0.0128175\\
-0.0130475	-0.0130475\\
-0.013825	-0.013825\\
-0.0140075	-0.0140075\\
-0.013825	-0.013825\\
-0.0136875	-0.0136875\\
-0.013825	-0.013825\\
-0.0136875	-0.0136875\\
-0.01355	-0.01355\\
-0.0134575	-0.0134575\\
-0.01323	-0.01323\\
-0.01332	-0.01332\\
-0.013275	-0.013275\\
-0.013595	-0.013595\\
-0.0137775	-0.0137775\\
-0.01355	-0.01355\\
-0.013505	-0.013505\\
-0.013275	-0.013275\\
-0.01332	-0.01332\\
-0.01419	-0.01419\\
-0.0149225	-0.0149225\\
-0.014695	-0.014695\\
-0.01451	-0.01451\\
-0.01419	-0.01419\\
-0.0146475	-0.0146475\\
-0.014785	-0.014785\\
-0.0143275	-0.0143275\\
-0.014375	-0.014375\\
-0.014145	-0.014145\\
-0.0143275	-0.0143275\\
-0.0140075	-0.0140075\\
-0.0137325	-0.0137325\\
-0.0139625	-0.0139625\\
-0.01419	-0.01419\\
-0.0141	-0.0141\\
-0.0140075	-0.0140075\\
-0.0140525	-0.0140525\\
-0.01451	-0.01451\\
-0.0145575	-0.0145575\\
-0.01497	-0.01497\\
-0.015655	-0.015655\\
-0.015565	-0.015565\\
-0.01506	-0.01506\\
-0.0145575	-0.0145575\\
-0.014785	-0.014785\\
-0.0151975	-0.0151975\\
-0.01497	-0.01497\\
-0.015105	-0.015105\\
-0.01474	-0.01474\\
-0.013915	-0.013915\\
-0.013595	-0.013595\\
-0.0142825	-0.0142825\\
-0.0145575	-0.0145575\\
-0.014375	-0.014375\\
-0.0148325	-0.0148325\\
-0.0149225	-0.0149225\\
-0.014785	-0.014785\\
-0.0145575	-0.0145575\\
-0.0148775	-0.0148775\\
-0.0148325	-0.0148325\\
-0.014785	-0.014785\\
-0.01474	-0.01474\\
-0.01442	-0.01442\\
-0.014465	-0.014465\\
-0.01419	-0.01419\\
-0.014145	-0.014145\\
-0.014695	-0.014695\\
-0.01497	-0.01497\\
-0.01506	-0.01506\\
-0.014695	-0.014695\\
-0.0142825	-0.0142825\\
-0.0142375	-0.0142375\\
-0.01442	-0.01442\\
-0.014695	-0.014695\\
-0.015015	-0.015015\\
-0.015335	-0.015335\\
-0.0151975	-0.0151975\\
-0.0146475	-0.0146475\\
-0.015015	-0.015015\\
-0.015565	-0.015565\\
-0.01538	-0.01538\\
-0.0152425	-0.0152425\\
-0.0154275	-0.0154275\\
-0.0152425	-0.0152425\\
-0.01497	-0.01497\\
-0.015015	-0.015015\\
-0.0149225	-0.0149225\\
-0.015015	-0.015015\\
-0.01529	-0.01529\\
-0.015105	-0.015105\\
-0.0151525	-0.0151525\\
-0.015105	-0.015105\\
-0.0149225	-0.0149225\\
-0.01506	-0.01506\\
-0.0148775	-0.0148775\\
-0.01474	-0.01474\\
-0.0148775	-0.0148775\\
-0.0148325	-0.0148325\\
-0.0142375	-0.0142375\\
-0.0134575	-0.0134575\\
-0.012955	-0.012955\\
-0.0130475	-0.0130475\\
-0.014145	-0.014145\\
-0.0148775	-0.0148775\\
-0.01506	-0.01506\\
-0.014695	-0.014695\\
-0.014465	-0.014465\\
-0.0140075	-0.0140075\\
-0.0140525	-0.0140525\\
-0.01419	-0.01419\\
-0.0141	-0.0141\\
-0.0142375	-0.0142375\\
-0.014145	-0.014145\\
-0.0140525	-0.0140525\\
-0.0136875	-0.0136875\\
-0.0137775	-0.0137775\\
-0.01451	-0.01451\\
-0.014375	-0.014375\\
-0.01442	-0.01442\\
-0.014695	-0.014695\\
-0.014785	-0.014785\\
-0.015105	-0.015105\\
-0.015015	-0.015015\\
-0.01506	-0.01506\\
-0.01451	-0.01451\\
-0.01474	-0.01474\\
-0.0146025	-0.0146025\\
-0.0146475	-0.0146475\\
-0.014785	-0.014785\\
-0.01506	-0.01506\\
-0.0149225	-0.0149225\\
-0.01506	-0.01506\\
-0.01538	-0.01538\\
-0.015335	-0.015335\\
-0.0151525	-0.0151525\\
-0.0149225	-0.0149225\\
-0.01497	-0.01497\\
-0.0148775	-0.0148775\\
-0.014695	-0.014695\\
-0.01451	-0.01451\\
-0.0145575	-0.0145575\\
-0.01451	-0.01451\\
-0.015105	-0.015105\\
-0.0157025	-0.0157025\\
-0.015655	-0.015655\\
-0.0155175	-0.0155175\\
-0.01506	-0.01506\\
-0.014695	-0.014695\\
-0.01442	-0.01442\\
-0.014145	-0.014145\\
-0.0145575	-0.0145575\\
-0.0149225	-0.0149225\\
-0.0151525	-0.0151525\\
-0.01506	-0.01506\\
-0.014695	-0.014695\\
-0.0151525	-0.0151525\\
-0.0155175	-0.0155175\\
-0.0157475	-0.0157475\\
-0.015655	-0.015655\\
-0.0161125	-0.0161125\\
-0.015335	-0.015335\\
-0.014785	-0.014785\\
-0.014465	-0.014465\\
-0.0148325	-0.0148325\\
-0.0152425	-0.0152425\\
-0.015655	-0.015655\\
-0.015565	-0.015565\\
-0.0152425	-0.0152425\\
-0.01451	-0.01451\\
-0.01419	-0.01419\\
-0.0142375	-0.0142375\\
-0.014375	-0.014375\\
-0.0140525	-0.0140525\\
-0.0139625	-0.0139625\\
-0.0134575	-0.0134575\\
-0.01387	-0.01387\\
-0.014145	-0.014145\\
-0.01442	-0.01442\\
-0.014465	-0.014465\\
-0.01451	-0.01451\\
-0.0145575	-0.0145575\\
-0.01474	-0.01474\\
-0.0145575	-0.0145575\\
-0.014785	-0.014785\\
-0.0151525	-0.0151525\\
-0.01497	-0.01497\\
-0.0149225	-0.0149225\\
-0.0148775	-0.0148775\\
-0.0149225	-0.0149225\\
-0.01538	-0.01538\\
-0.0152425	-0.0152425\\
-0.0149225	-0.0149225\\
-0.01497	-0.01497\\
-0.0149225	-0.0149225\\
-0.01451	-0.01451\\
-0.01474	-0.01474\\
-0.01529	-0.01529\\
-0.015335	-0.015335\\
-0.0154275	-0.0154275\\
-0.0160675	-0.0160675\\
-0.015975	-0.015975\\
-0.01561	-0.01561\\
-0.015335	-0.015335\\
-0.01538	-0.01538\\
-0.0158375	-0.0158375\\
-0.015565	-0.015565\\
-0.01529	-0.01529\\
-0.01561	-0.01561\\
-0.0152425	-0.0152425\\
-0.0145575	-0.0145575\\
-0.01451	-0.01451\\
-0.0142825	-0.0142825\\
-0.0140075	-0.0140075\\
-0.013915	-0.013915\\
-0.0137775	-0.0137775\\
-0.0136425	-0.0136425\\
-0.013825	-0.013825\\
-0.014145	-0.014145\\
-0.01442	-0.01442\\
-0.0148325	-0.0148325\\
-0.0149225	-0.0149225\\
-0.015105	-0.015105\\
-0.0154275	-0.0154275\\
-0.01538	-0.01538\\
-0.015105	-0.015105\\
-0.01506	-0.01506\\
-0.015015	-0.015015\\
-0.01538	-0.01538\\
-0.0151975	-0.0151975\\
-0.0152425	-0.0152425\\
-0.015105	-0.015105\\
-0.015565	-0.015565\\
-0.0154725	-0.0154725\\
-0.01538	-0.01538\\
-0.0149225	-0.0149225\\
-0.0143275	-0.0143275\\
-0.01387	-0.01387\\
-0.01419	-0.01419\\
-0.0142375	-0.0142375\\
-0.01442	-0.01442\\
-0.014375	-0.014375\\
-0.01451	-0.01451\\
-0.0151975	-0.0151975\\
-0.0157925	-0.0157925\\
-0.015655	-0.015655\\
-0.01561	-0.01561\\
-0.0151975	-0.0151975\\
-0.015105	-0.015105\\
-0.0152425	-0.0152425\\
-0.015105	-0.015105\\
-0.0151975	-0.0151975\\
-0.0154725	-0.0154725\\
-0.0160675	-0.0160675\\
-0.0160225	-0.0160225\\
-0.0161125	-0.0161125\\
-0.0157475	-0.0157475\\
-0.0157025	-0.0157025\\
-0.0155175	-0.0155175\\
-0.0154275	-0.0154275\\
-0.015655	-0.015655\\
-0.0158375	-0.0158375\\
-0.015015	-0.015015\\
-0.014785	-0.014785\\
-0.0146475	-0.0146475\\
-0.0148775	-0.0148775\\
-0.015015	-0.015015\\
-0.0145575	-0.0145575\\
-0.014695	-0.014695\\
-0.0152425	-0.0152425\\
-0.0162975	-0.0162975\\
-0.01657	-0.01657\\
-0.01648	-0.01648\\
-0.016205	-0.016205\\
-0.0162975	-0.0162975\\
-0.016525	-0.016525\\
-0.0163425	-0.0163425\\
-0.01625	-0.01625\\
-0.0162975	-0.0162975\\
-0.01593	-0.01593\\
-0.01561	-0.01561\\
-0.0157475	-0.0157475\\
-0.0154725	-0.0154725\\
-0.015015	-0.015015\\
-0.015105	-0.015105\\
-0.0148775	-0.0148775\\
-0.0148325	-0.0148325\\
-0.015015	-0.015015\\
-0.015335	-0.015335\\
-0.0152425	-0.0152425\\
-0.0148775	-0.0148775\\
-0.01497	-0.01497\\
-0.015105	-0.015105\\
-0.01529	-0.01529\\
-0.0151975	-0.0151975\\
-0.015015	-0.015015\\
-0.01538	-0.01538\\
-0.0154725	-0.0154725\\
-0.015105	-0.015105\\
-0.0155175	-0.0155175\\
-0.01561	-0.01561\\
-0.01529	-0.01529\\
-0.0148775	-0.0148775\\
-0.014375	-0.014375\\
-0.013915	-0.013915\\
-0.0142825	-0.0142825\\
-0.014465	-0.014465\\
-0.0142375	-0.0142375\\
-0.013825	-0.013825\\
-0.0143275	-0.0143275\\
-0.0146475	-0.0146475\\
-0.01497	-0.01497\\
-0.01529	-0.01529\\
-0.015335	-0.015335\\
-0.0154275	-0.0154275\\
-0.0149225	-0.0149225\\
-0.0140525	-0.0140525\\
-0.0139625	-0.0139625\\
-0.0140525	-0.0140525\\
-0.0146025	-0.0146025\\
-0.0148775	-0.0148775\\
-0.0154725	-0.0154725\\
-0.0152425	-0.0152425\\
-0.01506	-0.01506\\
-0.0154275	-0.0154275\\
-0.0152425	-0.0152425\\
-0.0148325	-0.0148325\\
-0.014465	-0.014465\\
-0.01451	-0.01451\\
-0.0148775	-0.0148775\\
-0.01538	-0.01538\\
-0.0151525	-0.0151525\\
-0.0148325	-0.0148325\\
-0.0149225	-0.0149225\\
-0.0152425	-0.0152425\\
-0.01593	-0.01593\\
-0.0161125	-0.0161125\\
-0.0160675	-0.0160675\\
-0.01561	-0.01561\\
-0.01497	-0.01497\\
-0.01419	-0.01419\\
-0.014145	-0.014145\\
-0.01419	-0.01419\\
-0.01474	-0.01474\\
-0.015335	-0.015335\\
-0.0157025	-0.0157025\\
-0.016205	-0.016205\\
-0.01625	-0.01625\\
-0.0155175	-0.0155175\\
-0.0151525	-0.0151525\\
-0.015015	-0.015015\\
-0.015105	-0.015105\\
-0.0155175	-0.0155175\\
-0.01593	-0.01593\\
-0.0157925	-0.0157925\\
-0.01529	-0.01529\\
-0.015335	-0.015335\\
-0.01561	-0.01561\\
-0.01538	-0.01538\\
-0.01497	-0.01497\\
-0.0146025	-0.0146025\\
-0.0141	-0.0141\\
-0.01387	-0.01387\\
-0.0134575	-0.0134575\\
-0.0137325	-0.0137325\\
-0.01419	-0.01419\\
-0.01442	-0.01442\\
-0.0142375	-0.0142375\\
-0.0141	-0.0141\\
-0.0134125	-0.0134125\\
-0.0125875	-0.0125875\\
-0.0124975	-0.0124975\\
-0.0130925	-0.0130925\\
-0.0136875	-0.0136875\\
-0.014145	-0.014145\\
-0.014465	-0.014465\\
-0.01474	-0.01474\\
-0.0152425	-0.0152425\\
-0.0160225	-0.0160225\\
-0.01625	-0.01625\\
-0.0163875	-0.0163875\\
-0.015655	-0.015655\\
-0.01529	-0.01529\\
-0.0155175	-0.0155175\\
-0.0154275	-0.0154275\\
-0.015105	-0.015105\\
-0.014785	-0.014785\\
-0.01442	-0.01442\\
-0.014695	-0.014695\\
-0.0148325	-0.0148325\\
-0.0142375	-0.0142375\\
-0.0137775	-0.0137775\\
-0.0139625	-0.0139625\\
-0.0142825	-0.0142825\\
-0.01419	-0.01419\\
-0.01442	-0.01442\\
-0.0143275	-0.0143275\\
-0.0146475	-0.0146475\\
-0.01529	-0.01529\\
-0.0151975	-0.0151975\\
-0.0154725	-0.0154725\\
-0.0160675	-0.0160675\\
-0.0163425	-0.0163425\\
-0.01616	-0.01616\\
-0.01561	-0.01561\\
-0.0151525	-0.0151525\\
-0.014465	-0.014465\\
-0.0145575	-0.0145575\\
-0.01442	-0.01442\\
-0.014785	-0.014785\\
-0.0148775	-0.0148775\\
-0.0146475	-0.0146475\\
-0.0148325	-0.0148325\\
-0.014375	-0.014375\\
-0.01451	-0.01451\\
-0.014375	-0.014375\\
-0.013825	-0.013825\\
-0.0137775	-0.0137775\\
-0.01355	-0.01355\\
-0.0134575	-0.0134575\\
-0.0131825	-0.0131825\\
-0.0134125	-0.0134125\\
-0.013505	-0.013505\\
-0.0139625	-0.0139625\\
-0.014465	-0.014465\\
-0.01442	-0.01442\\
-0.0143275	-0.0143275\\
-0.01474	-0.01474\\
-0.0152425	-0.0152425\\
-0.015105	-0.015105\\
-0.01442	-0.01442\\
-0.0136875	-0.0136875\\
-0.0140075	-0.0140075\\
-0.0146475	-0.0146475\\
-0.0148325	-0.0148325\\
-0.015335	-0.015335\\
-0.0154725	-0.0154725\\
-0.015105	-0.015105\\
-0.0146475	-0.0146475\\
-0.01451	-0.01451\\
-0.0146025	-0.0146025\\
-0.014465	-0.014465\\
-0.013915	-0.013915\\
-0.013825	-0.013825\\
-0.0136875	-0.0136875\\
-0.013505	-0.013505\\
-0.0131375	-0.0131375\\
-0.0133675	-0.0133675\\
-0.01387	-0.01387\\
-0.0141	-0.0141\\
-0.01387	-0.01387\\
-0.014375	-0.014375\\
-0.01497	-0.01497\\
-0.014695	-0.014695\\
-0.0148775	-0.0148775\\
-0.015105	-0.015105\\
-0.0148775	-0.0148775\\
-0.014375	-0.014375\\
-0.0143275	-0.0143275\\
-0.0145575	-0.0145575\\
-0.0140525	-0.0140525\\
-0.0139625	-0.0139625\\
-0.0137775	-0.0137775\\
-0.013275	-0.013275\\
-0.0130925	-0.0130925\\
-0.01332	-0.01332\\
-0.0137325	-0.0137325\\
-0.0136875	-0.0136875\\
-0.01387	-0.01387\\
-0.0140525	-0.0140525\\
-0.013915	-0.013915\\
-0.01332	-0.01332\\
-0.012955	-0.012955\\
-0.0130475	-0.0130475\\
-0.0131375	-0.0131375\\
-0.01332	-0.01332\\
-0.0142375	-0.0142375\\
-0.0145575	-0.0145575\\
-0.01442	-0.01442\\
-0.0146475	-0.0146475\\
-0.0151975	-0.0151975\\
-0.0155175	-0.0155175\\
-0.015565	-0.015565\\
-0.015335	-0.015335\\
-0.01529	-0.01529\\
-0.01538	-0.01538\\
-0.015335	-0.015335\\
-0.0154725	-0.0154725\\
-0.015885	-0.015885\\
-0.01593	-0.01593\\
-0.0157925	-0.0157925\\
-0.015565	-0.015565\\
-0.0154275	-0.0154275\\
-0.01538	-0.01538\\
-0.0154725	-0.0154725\\
-0.01506	-0.01506\\
-0.015015	-0.015015\\
-0.0152425	-0.0152425\\
-0.0154275	-0.0154275\\
-0.0152425	-0.0152425\\
-0.01529	-0.01529\\
-0.0151975	-0.0151975\\
-0.015015	-0.015015\\
-0.0145575	-0.0145575\\
-0.014695	-0.014695\\
-0.01529	-0.01529\\
-0.0152425	-0.0152425\\
-0.014695	-0.014695\\
-0.0141	-0.0141\\
-0.0137775	-0.0137775\\
-0.013915	-0.013915\\
-0.0136875	-0.0136875\\
-0.0134575	-0.0134575\\
-0.01355	-0.01355\\
-0.01332	-0.01332\\
-0.01355	-0.01355\\
-0.0134575	-0.0134575\\
-0.0130925	-0.0130925\\
-0.013275	-0.013275\\
-0.013505	-0.013505\\
-0.0137325	-0.0137325\\
-0.0137775	-0.0137775\\
-0.014145	-0.014145\\
-0.0142375	-0.0142375\\
-0.015015	-0.015015\\
-0.0149225	-0.0149225\\
-0.014695	-0.014695\\
-0.01474	-0.01474\\
-0.014465	-0.014465\\
-0.01387	-0.01387\\
-0.01419	-0.01419\\
-0.014145	-0.014145\\
-0.0136875	-0.0136875\\
-0.0139625	-0.0139625\\
-0.0140525	-0.0140525\\
-0.0142375	-0.0142375\\
-0.013915	-0.013915\\
-0.0134125	-0.0134125\\
-0.0130475	-0.0130475\\
-0.012955	-0.012955\\
-0.0134575	-0.0134575\\
-0.013595	-0.013595\\
-0.0137325	-0.0137325\\
-0.014145	-0.014145\\
-0.0146475	-0.0146475\\
-0.0145575	-0.0145575\\
-0.0148325	-0.0148325\\
-0.0151525	-0.0151525\\
-0.01506	-0.01506\\
-0.015015	-0.015015\\
-0.015655	-0.015655\\
-0.01561	-0.01561\\
-0.0157025	-0.0157025\\
-0.0157475	-0.0157475\\
-0.015975	-0.015975\\
-0.0154725	-0.0154725\\
-0.01474	-0.01474\\
-0.014375	-0.014375\\
-0.0145575	-0.0145575\\
-0.0149225	-0.0149225\\
-0.01497	-0.01497\\
-0.014695	-0.014695\\
-0.0140525	-0.0140525\\
-0.0140075	-0.0140075\\
-0.01474	-0.01474\\
-0.015335	-0.015335\\
-0.0154275	-0.0154275\\
-0.0149225	-0.0149225\\
-0.01451	-0.01451\\
-0.0146475	-0.0146475\\
-0.0152425	-0.0152425\\
-0.0155175	-0.0155175\\
-0.01561	-0.01561\\
-0.015335	-0.015335\\
-0.015565	-0.015565\\
-0.0158375	-0.0158375\\
-0.01625	-0.01625\\
-0.0162975	-0.0162975\\
-0.01616	-0.01616\\
-0.016205	-0.016205\\
-0.0161125	-0.0161125\\
-0.0154275	-0.0154275\\
-0.01506	-0.01506\\
-0.0151975	-0.0151975\\
-0.01538	-0.01538\\
-0.0154725	-0.0154725\\
-0.0151975	-0.0151975\\
-0.015335	-0.015335\\
-0.01529	-0.01529\\
-0.015015	-0.015015\\
-0.0151525	-0.0151525\\
-0.015105	-0.015105\\
-0.0155175	-0.0155175\\
-0.0157025	-0.0157025\\
-0.0154275	-0.0154275\\
-0.01506	-0.01506\\
-0.01442	-0.01442\\
-0.0145575	-0.0145575\\
-0.014465	-0.014465\\
-0.0142375	-0.0142375\\
-0.0141	-0.0141\\
-0.014375	-0.014375\\
-0.01474	-0.01474\\
-0.0149225	-0.0149225\\
-0.01497	-0.01497\\
-0.0146475	-0.0146475\\
-0.0142375	-0.0142375\\
-0.0137325	-0.0137325\\
-0.01419	-0.01419\\
-0.0143275	-0.0143275\\
-0.0142825	-0.0142825\\
-0.0148325	-0.0148325\\
-0.0152425	-0.0152425\\
-0.0155175	-0.0155175\\
-0.01561	-0.01561\\
-0.0158375	-0.0158375\\
-0.0154725	-0.0154725\\
-0.01506	-0.01506\\
-0.014785	-0.014785\\
-0.014695	-0.014695\\
-0.0149225	-0.0149225\\
-0.0148775	-0.0148775\\
-0.014465	-0.014465\\
-0.0142375	-0.0142375\\
-0.01474	-0.01474\\
-0.01506	-0.01506\\
-0.015015	-0.015015\\
-0.0155175	-0.0155175\\
-0.015885	-0.015885\\
-0.01561	-0.01561\\
-0.0151525	-0.0151525\\
-0.0146475	-0.0146475\\
-0.01442	-0.01442\\
-0.0148325	-0.0148325\\
-0.0149225	-0.0149225\\
-0.015105	-0.015105\\
-0.0151975	-0.0151975\\
-0.01529	-0.01529\\
-0.015335	-0.015335\\
-0.0151525	-0.0151525\\
-0.01506	-0.01506\\
-0.014785	-0.014785\\
-0.0145575	-0.0145575\\
-0.0149225	-0.0149225\\
-0.0154275	-0.0154275\\
-0.01561	-0.01561\\
-0.015015	-0.015015\\
-0.0155175	-0.0155175\\
-0.015885	-0.015885\\
-0.015335	-0.015335\\
-0.014695	-0.014695\\
-0.0148775	-0.0148775\\
-0.014785	-0.014785\\
-0.0149225	-0.0149225\\
-0.0146475	-0.0146475\\
-0.0148325	-0.0148325\\
-0.015565	-0.015565\\
-0.015885	-0.015885\\
-0.015565	-0.015565\\
-0.0155175	-0.0155175\\
-0.01561	-0.01561\\
-0.0157025	-0.0157025\\
-0.01538	-0.01538\\
-0.01529	-0.01529\\
-0.0149225	-0.0149225\\
-0.015105	-0.015105\\
-0.0151975	-0.0151975\\
-0.0157925	-0.0157925\\
-0.01616	-0.01616\\
-0.0166175	-0.0166175\\
-0.0164325	-0.0164325\\
-0.0160675	-0.0160675\\
-0.0157475	-0.0157475\\
-0.0157025	-0.0157025\\
-0.0155175	-0.0155175\\
-0.01497	-0.01497\\
-0.01474	-0.01474\\
-0.01442	-0.01442\\
-0.0139625	-0.0139625\\
-0.014145	-0.014145\\
-0.014465	-0.014465\\
-0.0142825	-0.0142825\\
-0.013915	-0.013915\\
-0.0137325	-0.0137325\\
-0.01442	-0.01442\\
-0.01506	-0.01506\\
-0.014465	-0.014465\\
-0.014375	-0.014375\\
-0.01451	-0.01451\\
-0.01442	-0.01442\\
-0.0146475	-0.0146475\\
-0.0145575	-0.0145575\\
-0.0142375	-0.0142375\\
-0.013915	-0.013915\\
-0.013595	-0.013595\\
-0.0136425	-0.0136425\\
-0.0142825	-0.0142825\\
-0.0146475	-0.0146475\\
-0.014465	-0.014465\\
-0.0140525	-0.0140525\\
-0.01355	-0.01355\\
-0.013595	-0.013595\\
-0.0141	-0.0141\\
-0.0145575	-0.0145575\\
-0.0146475	-0.0146475\\
-0.014465	-0.014465\\
-0.0142825	-0.0142825\\
-0.014375	-0.014375\\
-0.014695	-0.014695\\
-0.0151975	-0.0151975\\
-0.015655	-0.015655\\
-0.0155175	-0.0155175\\
-0.01497	-0.01497\\
-0.0148775	-0.0148775\\
-0.014785	-0.014785\\
-0.01474	-0.01474\\
-0.01442	-0.01442\\
-0.01451	-0.01451\\
-0.014465	-0.014465\\
-0.014145	-0.014145\\
-0.013595	-0.013595\\
-0.0136875	-0.0136875\\
-0.0141	-0.0141\\
-0.0146475	-0.0146475\\
-0.0149225	-0.0149225\\
-0.0152425	-0.0152425\\
-0.01538	-0.01538\\
-0.0158375	-0.0158375\\
-0.01648	-0.01648\\
-0.0160225	-0.0160225\\
-0.01593	-0.01593\\
-0.0161125	-0.0161125\\
-0.0166175	-0.0166175\\
-0.0162975	-0.0162975\\
-0.015335	-0.015335\\
-0.0146025	-0.0146025\\
-0.0143275	-0.0143275\\
-0.0139625	-0.0139625\\
-0.01332	-0.01332\\
-0.0131825	-0.0131825\\
-0.013505	-0.013505\\
-0.0139625	-0.0139625\\
-0.0140525	-0.0140525\\
-0.01387	-0.01387\\
-0.013825	-0.013825\\
-0.0139625	-0.0139625\\
-0.01419	-0.01419\\
-0.0143275	-0.0143275\\
-0.0142825	-0.0142825\\
-0.0140075	-0.0140075\\
-0.0136425	-0.0136425\\
-0.0134575	-0.0134575\\
-0.0136875	-0.0136875\\
-0.0139625	-0.0139625\\
-0.0137775	-0.0137775\\
-0.0134575	-0.0134575\\
-0.013825	-0.013825\\
-0.013595	-0.013595\\
-0.0136425	-0.0136425\\
-0.013915	-0.013915\\
-0.0143275	-0.0143275\\
-0.0145575	-0.0145575\\
-0.0142375	-0.0142375\\
-0.0145575	-0.0145575\\
-0.014695	-0.014695\\
-0.0145575	-0.0145575\\
-0.014375	-0.014375\\
-0.01474	-0.01474\\
-0.0151975	-0.0151975\\
-0.01529	-0.01529\\
-0.0154725	-0.0154725\\
-0.0152425	-0.0152425\\
-0.015105	-0.015105\\
-0.01506	-0.01506\\
-0.01529	-0.01529\\
-0.0151975	-0.0151975\\
-0.0154275	-0.0154275\\
-0.0155175	-0.0155175\\
-0.0154725	-0.0154725\\
-0.0162975	-0.0162975\\
-0.0163875	-0.0163875\\
-0.015565	-0.015565\\
-0.0149225	-0.0149225\\
-0.014695	-0.014695\\
-0.0148775	-0.0148775\\
-0.0151975	-0.0151975\\
-0.0154275	-0.0154275\\
-0.015565	-0.015565\\
-0.0157025	-0.0157025\\
-0.01529	-0.01529\\
-0.015105	-0.015105\\
-0.0151975	-0.0151975\\
-0.01474	-0.01474\\
-0.0142375	-0.0142375\\
-0.0146475	-0.0146475\\
-0.01474	-0.01474\\
-0.015015	-0.015015\\
-0.0154275	-0.0154275\\
-0.01538	-0.01538\\
-0.01497	-0.01497\\
-0.014695	-0.014695\\
-0.01497	-0.01497\\
-0.015335	-0.015335\\
-0.0158375	-0.0158375\\
-0.0160675	-0.0160675\\
-0.0163425	-0.0163425\\
-0.0162975	-0.0162975\\
-0.0161125	-0.0161125\\
-0.015655	-0.015655\\
-0.01497	-0.01497\\
-0.015335	-0.015335\\
-0.0157925	-0.0157925\\
-0.015335	-0.015335\\
-0.015655	-0.015655\\
-0.01616	-0.01616\\
-0.01657	-0.01657\\
-0.01616	-0.01616\\
-0.0154275	-0.0154275\\
-0.0146475	-0.0146475\\
-0.014695	-0.014695\\
-0.01474	-0.01474\\
-0.014695	-0.014695\\
-0.0142825	-0.0142825\\
-0.0142375	-0.0142375\\
-0.0140525	-0.0140525\\
-0.014145	-0.014145\\
-0.0141	-0.0141\\
-0.01419	-0.01419\\
-0.01474	-0.01474\\
-0.014785	-0.014785\\
-0.015105	-0.015105\\
-0.0155175	-0.0155175\\
-0.015655	-0.015655\\
-0.015105	-0.015105\\
-0.0148775	-0.0148775\\
-0.015105	-0.015105\\
-0.014785	-0.014785\\
-0.0145575	-0.0145575\\
-0.01419	-0.01419\\
-0.0143275	-0.0143275\\
-0.014375	-0.014375\\
-0.013915	-0.013915\\
-0.013275	-0.013275\\
-0.012955	-0.012955\\
-0.0133675	-0.0133675\\
-0.01387	-0.01387\\
-0.0140525	-0.0140525\\
-0.0137775	-0.0137775\\
-0.0137325	-0.0137325\\
-0.01419	-0.01419\\
-0.0146025	-0.0146025\\
-0.01451	-0.01451\\
-0.0142825	-0.0142825\\
-0.01442	-0.01442\\
-0.014695	-0.014695\\
-0.0146475	-0.0146475\\
-0.014785	-0.014785\\
-0.014465	-0.014465\\
-0.01419	-0.01419\\
-0.0143275	-0.0143275\\
-0.0142825	-0.0142825\\
-0.014375	-0.014375\\
-0.01474	-0.01474\\
-0.01538	-0.01538\\
-0.0152425	-0.0152425\\
-0.015105	-0.015105\\
-0.015565	-0.015565\\
-0.0154275	-0.0154275\\
-0.0149225	-0.0149225\\
-0.01442	-0.01442\\
-0.0140525	-0.0140525\\
-0.0139625	-0.0139625\\
-0.0137775	-0.0137775\\
-0.0136875	-0.0136875\\
-0.0142825	-0.0142825\\
-0.0136875	-0.0136875\\
-0.01332	-0.01332\\
-0.0136425	-0.0136425\\
-0.0140525	-0.0140525\\
-0.013915	-0.013915\\
-0.0136425	-0.0136425\\
-0.0131825	-0.0131825\\
-0.013275	-0.013275\\
-0.0140525	-0.0140525\\
-0.0146475	-0.0146475\\
-0.0145575	-0.0145575\\
-0.0143275	-0.0143275\\
-0.014375	-0.014375\\
-0.01451	-0.01451\\
-0.014785	-0.014785\\
-0.015105	-0.015105\\
-0.0151975	-0.0151975\\
-0.01506	-0.01506\\
-0.014695	-0.014695\\
-0.0143275	-0.0143275\\
-0.0142825	-0.0142825\\
-0.0145575	-0.0145575\\
-0.01419	-0.01419\\
-0.013915	-0.013915\\
-0.01442	-0.01442\\
-0.014785	-0.014785\\
-0.0148325	-0.0148325\\
-0.01497	-0.01497\\
-0.0154275	-0.0154275\\
-0.0149225	-0.0149225\\
-0.01538	-0.01538\\
-0.015655	-0.015655\\
-0.0155175	-0.0155175\\
-0.01561	-0.01561\\
-0.015565	-0.015565\\
-0.0163425	-0.0163425\\
-0.0167075	-0.0167075\\
-0.016205	-0.016205\\
-0.0163425	-0.0163425\\
-0.01657	-0.01657\\
-0.0166175	-0.0166175\\
-0.016755	-0.016755\\
-0.017075	-0.017075\\
-0.0169825	-0.0169825\\
-0.016525	-0.016525\\
-0.0157925	-0.0157925\\
-0.0155175	-0.0155175\\
-0.015335	-0.015335\\
-0.0149225	-0.0149225\\
-0.0151975	-0.0151975\\
-0.015655	-0.015655\\
-0.0152425	-0.0152425\\
-0.0149225	-0.0149225\\
-0.0148775	-0.0148775\\
-0.014465	-0.014465\\
-0.01387	-0.01387\\
-0.013595	-0.013595\\
-0.01355	-0.01355\\
-0.0136875	-0.0136875\\
-0.013595	-0.013595\\
-0.01332	-0.01332\\
-0.01291	-0.01291\\
-0.012635	-0.012635\\
-0.01291	-0.01291\\
-0.0128175	-0.0128175\\
-0.0127725	-0.0127725\\
-0.0134575	-0.0134575\\
-0.0141	-0.0141\\
-0.0143275	-0.0143275\\
-0.0140525	-0.0140525\\
-0.01419	-0.01419\\
-0.0148325	-0.0148325\\
-0.0142825	-0.0142825\\
-0.01355	-0.01355\\
-0.01323	-0.01323\\
-0.0128175	-0.0128175\\
-0.0130475	-0.0130475\\
-0.013595	-0.013595\\
-0.01419	-0.01419\\
-0.0139625	-0.0139625\\
-0.014375	-0.014375\\
-0.015015	-0.015015\\
-0.0152425	-0.0152425\\
-0.01506	-0.01506\\
-0.01442	-0.01442\\
-0.0148325	-0.0148325\\
-0.0151975	-0.0151975\\
-0.01474	-0.01474\\
-0.0139625	-0.0139625\\
-0.0141	-0.0141\\
-0.0139625	-0.0139625\\
-0.01332	-0.01332\\
-0.0125875	-0.0125875\\
-0.0131375	-0.0131375\\
-0.014145	-0.014145\\
-0.0145575	-0.0145575\\
-0.0143275	-0.0143275\\
-0.013825	-0.013825\\
-0.013595	-0.013595\\
-0.01268	-0.01268\\
-0.0125425	-0.0125425\\
-0.01268	-0.01268\\
-0.012955	-0.012955\\
-0.0134575	-0.0134575\\
-0.0142375	-0.0142375\\
-0.0146025	-0.0146025\\
-0.01474	-0.01474\\
-0.0148775	-0.0148775\\
-0.0149225	-0.0149225\\
-0.01474	-0.01474\\
-0.0149225	-0.0149225\\
-0.015655	-0.015655\\
-0.015885	-0.015885\\
-0.015105	-0.015105\\
-0.014145	-0.014145\\
-0.0146475	-0.0146475\\
-0.0151525	-0.0151525\\
-0.01506	-0.01506\\
-0.01451	-0.01451\\
-0.0142375	-0.0142375\\
-0.0148325	-0.0148325\\
-0.015565	-0.015565\\
-0.0158375	-0.0158375\\
-0.0160675	-0.0160675\\
-0.01616	-0.01616\\
-0.01593	-0.01593\\
-0.0157925	-0.0157925\\
-0.015015	-0.015015\\
-0.01451	-0.01451\\
-0.014695	-0.014695\\
-0.0148775	-0.0148775\\
-0.015015	-0.015015\\
-0.015105	-0.015105\\
-0.01474	-0.01474\\
-0.0148775	-0.0148775\\
-0.0142375	-0.0142375\\
-0.013595	-0.013595\\
-0.0131825	-0.0131825\\
-0.0133675	-0.0133675\\
-0.013275	-0.013275\\
-0.0127725	-0.0127725\\
-0.0130475	-0.0130475\\
-0.013915	-0.013915\\
-0.013825	-0.013825\\
-0.013595	-0.013595\\
-0.01387	-0.01387\\
-0.0145575	-0.0145575\\
-0.01451	-0.01451\\
-0.0137775	-0.0137775\\
-0.0130925	-0.0130925\\
-0.0136425	-0.0136425\\
-0.014695	-0.014695\\
-0.015335	-0.015335\\
-0.0161125	-0.0161125\\
-0.0167075	-0.0167075\\
-0.0168925	-0.0168925\\
-0.01648	-0.01648\\
-0.0162975	-0.0162975\\
-0.01657	-0.01657\\
-0.016525	-0.016525\\
-0.0163875	-0.0163875\\
-0.016525	-0.016525\\
-0.01657	-0.01657\\
-0.0166625	-0.0166625\\
-0.01712	-0.01712\\
-0.0168925	-0.0168925\\
-0.0160675	-0.0160675\\
-0.0151525	-0.0151525\\
-0.01451	-0.01451\\
-0.01442	-0.01442\\
-0.0143275	-0.0143275\\
-0.0148325	-0.0148325\\
-0.01474	-0.01474\\
-0.015015	-0.015015\\
-0.01451	-0.01451\\
-0.01442	-0.01442\\
-0.014785	-0.014785\\
-0.01506	-0.01506\\
-0.0146475	-0.0146475\\
-0.0140075	-0.0140075\\
-0.014145	-0.014145\\
-0.0141	-0.0141\\
-0.01506	-0.01506\\
-0.01593	-0.01593\\
-0.01616	-0.01616\\
-0.01648	-0.01648\\
-0.01703	-0.01703\\
-0.01712	-0.01712\\
-0.016525	-0.016525\\
-0.015885	-0.015885\\
-0.0149225	-0.0149225\\
-0.0146025	-0.0146025\\
-0.0146475	-0.0146475\\
-0.0148775	-0.0148775\\
-0.014695	-0.014695\\
-0.01442	-0.01442\\
-0.014465	-0.014465\\
-0.0146475	-0.0146475\\
-0.0148325	-0.0148325\\
-0.015105	-0.015105\\
-0.0148325	-0.0148325\\
-0.01419	-0.01419\\
-0.01355	-0.01355\\
-0.0131825	-0.0131825\\
-0.013	-0.013\\
-0.0130475	-0.0130475\\
-0.013595	-0.013595\\
-0.0137775	-0.0137775\\
-0.0140075	-0.0140075\\
-0.0146025	-0.0146025\\
-0.01451	-0.01451\\
-0.01497	-0.01497\\
-0.015885	-0.015885\\
-0.01593	-0.01593\\
-0.0157025	-0.0157025\\
-0.01593	-0.01593\\
-0.01616	-0.01616\\
-0.0157475	-0.0157475\\
-0.0154275	-0.0154275\\
-0.0148775	-0.0148775\\
-0.01474	-0.01474\\
-0.0155175	-0.0155175\\
-0.01657	-0.01657\\
-0.0172575	-0.0172575\\
-0.01712	-0.01712\\
-0.016845	-0.016845\\
-0.0164325	-0.0164325\\
-0.0162975	-0.0162975\\
-0.0166625	-0.0166625\\
-0.0161125	-0.0161125\\
-0.01561	-0.01561\\
-0.015335	-0.015335\\
-0.0157925	-0.0157925\\
-0.0160675	-0.0160675\\
-0.01538	-0.01538\\
-0.015015	-0.015015\\
-0.014695	-0.014695\\
-0.015015	-0.015015\\
-0.015655	-0.015655\\
-0.0158375	-0.0158375\\
-0.01561	-0.01561\\
-0.0152425	-0.0152425\\
-0.015105	-0.015105\\
-0.0146475	-0.0146475\\
-0.0143275	-0.0143275\\
-0.0140075	-0.0140075\\
-0.0136875	-0.0136875\\
-0.013595	-0.013595\\
-0.0140075	-0.0140075\\
-0.0146475	-0.0146475\\
-0.014785	-0.014785\\
-0.014465	-0.014465\\
-0.014145	-0.014145\\
-0.0140525	-0.0140525\\
-0.0140075	-0.0140075\\
-0.014375	-0.014375\\
-0.01451	-0.01451\\
-0.0145575	-0.0145575\\
-0.0140525	-0.0140525\\
-0.0146025	-0.0146025\\
-0.015335	-0.015335\\
-0.0151525	-0.0151525\\
-0.01419	-0.01419\\
-0.013915	-0.013915\\
-0.01442	-0.01442\\
-0.01451	-0.01451\\
-0.014375	-0.014375\\
-0.013915	-0.013915\\
-0.0142825	-0.0142825\\
-0.0148775	-0.0148775\\
-0.0145575	-0.0145575\\
-0.0136425	-0.0136425\\
-0.0134125	-0.0134125\\
-0.0143275	-0.0143275\\
-0.0148775	-0.0148775\\
-0.01442	-0.01442\\
-0.014145	-0.014145\\
-0.014695	-0.014695\\
-0.015335	-0.015335\\
-0.01538	-0.01538\\
-0.0157025	-0.0157025\\
-0.01625	-0.01625\\
-0.01593	-0.01593\\
-0.0155175	-0.0155175\\
-0.015885	-0.015885\\
-0.0155175	-0.0155175\\
-0.015655	-0.015655\\
-0.0160225	-0.0160225\\
-0.015885	-0.015885\\
-0.01506	-0.01506\\
-0.0152425	-0.0152425\\
-0.0161125	-0.0161125\\
-0.01657	-0.01657\\
-0.0163425	-0.0163425\\
-0.0160675	-0.0160675\\
-0.01625	-0.01625\\
-0.0161125	-0.0161125\\
-0.0155175	-0.0155175\\
-0.01529	-0.01529\\
-0.01538	-0.01538\\
-0.0154725	-0.0154725\\
-0.0155175	-0.0155175\\
-0.015565	-0.015565\\
-0.01529	-0.01529\\
-0.0152425	-0.0152425\\
-0.0155175	-0.0155175\\
-0.0151975	-0.0151975\\
-0.01474	-0.01474\\
-0.01451	-0.01451\\
-0.014145	-0.014145\\
-0.0140525	-0.0140525\\
-0.0145575	-0.0145575\\
-0.0146475	-0.0146475\\
-0.01497	-0.01497\\
-0.01538	-0.01538\\
-0.0157475	-0.0157475\\
-0.015565	-0.015565\\
-0.01538	-0.01538\\
-0.014695	-0.014695\\
-0.0146475	-0.0146475\\
-0.01538	-0.01538\\
-0.0151975	-0.0151975\\
-0.01442	-0.01442\\
-0.014785	-0.014785\\
-0.0154275	-0.0154275\\
-0.01561	-0.01561\\
-0.0160225	-0.0160225\\
-0.0166175	-0.0166175\\
-0.0164325	-0.0164325\\
-0.0161125	-0.0161125\\
-0.01625	-0.01625\\
-0.015885	-0.015885\\
-0.0154275	-0.0154275\\
-0.0154725	-0.0154725\\
-0.0157475	-0.0157475\\
-0.0158375	-0.0158375\\
-0.015885	-0.015885\\
-0.015335	-0.015335\\
-0.01497	-0.01497\\
-0.014785	-0.014785\\
-0.0149225	-0.0149225\\
-0.014695	-0.014695\\
-0.0152425	-0.0152425\\
-0.015565	-0.015565\\
-0.0152425	-0.0152425\\
-0.01497	-0.01497\\
-0.0152425	-0.0152425\\
-0.014785	-0.014785\\
-0.014145	-0.014145\\
-0.0140525	-0.0140525\\
-0.0143275	-0.0143275\\
-0.0139625	-0.0139625\\
-0.01332	-0.01332\\
-0.0134125	-0.0134125\\
-0.013915	-0.013915\\
-0.0145575	-0.0145575\\
-0.015015	-0.015015\\
-0.01538	-0.01538\\
-0.0157925	-0.0157925\\
-0.015335	-0.015335\\
-0.014785	-0.014785\\
-0.0155175	-0.0155175\\
-0.0162975	-0.0162975\\
-0.016205	-0.016205\\
-0.0160225	-0.0160225\\
-0.016525	-0.016525\\
-0.01648	-0.01648\\
-0.01616	-0.01616\\
-0.0157475	-0.0157475\\
-0.015655	-0.015655\\
-0.01593	-0.01593\\
-0.01561	-0.01561\\
-0.01529	-0.01529\\
-0.0157475	-0.0157475\\
-0.016205	-0.016205\\
-0.0163875	-0.0163875\\
-0.015565	-0.015565\\
-0.01451	-0.01451\\
-0.0151525	-0.0151525\\
-0.0151975	-0.0151975\\
-0.0142825	-0.0142825\\
-0.0137325	-0.0137325\\
-0.013915	-0.013915\\
-0.01442	-0.01442\\
-0.015105	-0.015105\\
-0.01474	-0.01474\\
-0.014145	-0.014145\\
-0.01355	-0.01355\\
-0.013505	-0.013505\\
-0.0134575	-0.0134575\\
-0.01355	-0.01355\\
-0.014145	-0.014145\\
-0.0142375	-0.0142375\\
-0.0136425	-0.0136425\\
-0.0133675	-0.0133675\\
-0.0134575	-0.0134575\\
-0.0137325	-0.0137325\\
-0.0134575	-0.0134575\\
-0.013595	-0.013595\\
-0.0134575	-0.0134575\\
-0.0131375	-0.0131375\\
-0.0136875	-0.0136875\\
-0.0143275	-0.0143275\\
-0.01506	-0.01506\\
-0.0158375	-0.0158375\\
-0.01593	-0.01593\\
-0.0151525	-0.0151525\\
-0.01451	-0.01451\\
-0.01442	-0.01442\\
-0.0137775	-0.0137775\\
-0.0134125	-0.0134125\\
-0.0139625	-0.0139625\\
-0.0142375	-0.0142375\\
-0.01419	-0.01419\\
-0.0143275	-0.0143275\\
-0.014465	-0.014465\\
-0.0146025	-0.0146025\\
-0.01474	-0.01474\\
-0.015015	-0.015015\\
-0.015105	-0.015105\\
-0.01561	-0.01561\\
-0.0151525	-0.0151525\\
-0.01419	-0.01419\\
-0.013825	-0.013825\\
-0.0142825	-0.0142825\\
-0.014145	-0.014145\\
-0.0134575	-0.0134575\\
-0.0134125	-0.0134125\\
-0.01323	-0.01323\\
-0.01268	-0.01268\\
-0.012725	-0.012725\\
-0.01332	-0.01332\\
-0.0137325	-0.0137325\\
-0.0142825	-0.0142825\\
-0.015015	-0.015015\\
-0.0148775	-0.0148775\\
-0.01387	-0.01387\\
-0.0140525	-0.0140525\\
-0.01442	-0.01442\\
-0.013825	-0.013825\\
-0.0145575	-0.0145575\\
-0.01474	-0.01474\\
-0.015335	-0.015335\\
-0.01593	-0.01593\\
-0.015565	-0.015565\\
-0.0151975	-0.0151975\\
-0.015015	-0.015015\\
-0.0151975	-0.0151975\\
-0.015655	-0.015655\\
-0.0154275	-0.0154275\\
-0.0148325	-0.0148325\\
};
\end{axis}
\end{tikzpicture}%
%	\caption{Scatter plots of the input term $u(t)$ for training datasets. Correlation coefficient with the output signal is shown for each figure.}\label{fig:regr_u}
%\end{figure}
%\begin{figure}[!t]
%	\centering
%	% This file was created by matlab2tikz.
%
\definecolor{mycolor1}{rgb}{0.00000,0.44700,0.74100}%
\definecolor{mycolor2}{rgb}{0.85000,0.32500,0.09800}%
%
\begin{tikzpicture}

\begin{axis}[%
width=4.927cm,
height=3cm,
at={(0cm,14.516cm)},
scale only axis,
xmin=0,
xmax=400,
xlabel style={font=\color{white!15!black}},
xlabel={y(t-1)u(t)},
ymin=-61.8348889862102,
ymax=0,
ylabel style={font=\color{white!15!black}},
ylabel={y(t)},
axis background/.style={fill=white},
title style={font=\small},
title={C1, R = -0.798},
axis x line*=bottom,
axis y line*=left
]
\addplot[only marks, mark=*, mark options={}, mark size=1.5000pt, color=mycolor1, fill=mycolor1] table[row sep=crcr]{%
x	y\\
120.154712	-23.193\\
144.399618	-28.076\\
174.267732	-26.855\\
165.21196	-20.752\\
129.949024	-29.297\\
184.014457	-28.076\\
177.86146	-32.959\\
207.015479	-25.635\\
153.015315	-14.648\\
86.906584	-20.752\\
123.121616	-17.09\\
102.33492	-18.311\\
109.298359	-20.752\\
116.646992	-7.324\\
38.758608	-4.883\\
25.035141	-4.883\\
26.910213	-13.428\\
79.413192	-23.193\\
139.714632	-24.414\\
148.412706	-25.635\\
153.015315	-19.531\\
111.932161	-12.207\\
73.095516	-24.414\\
144.848262	-19.531\\
112.30325	-14.648\\
86.364608	-20.752\\
123.121616	-18.311\\
107.284149	-17.09\\
103.25778	-29.297\\
178.096463	-25.635\\
153.50238	-18.311\\
111.642167	-28.076\\
172.723552	-28.076\\
170.674004	-21.973\\
131.574324	-18.311\\
107.284149	-15.869\\
93.563624	-15.869\\
94.722061	-21.973\\
131.156837	-21.973\\
131.156837	-21.973\\
131.969838	-21.973\\
132.365352	-23.193\\
143.958951	-30.518\\
189.425226	-29.297\\
177.569117	-19.531\\
116.951628	-18.311\\
105.947446	-10.986\\
62.158788	-14.648\\
84.489664	-15.869\\
90.373955	-12.207\\
69.286932	-13.428\\
77.694408	-18.311\\
107.632058	-19.531\\
114.803218	-18.311\\
110.982971	-29.297\\
176.485128	-21.973\\
127.531292	-12.207\\
69.067206	-12.207\\
69.958317	-13.428\\
77.694408	-17.09\\
95.44765	-9.766\\
53.820426	-10.986\\
61.554558	-14.648\\
81.530768	-9.766\\
53.46885	-7.324\\
40.633552	-13.428\\
75.237084	-13.428\\
78.432948	-21.973\\
129.552808	-20.752\\
123.868688	-25.635\\
153.50238	-21.973\\
132.365352	-25.635\\
153.50238	-20.752\\
123.495152	-20.752\\
123.121616	-18.311\\
107.632058	-18.311\\
107.961656	-17.09\\
104.19773	-30.518\\
195.589862	-41.504\\
270.564576	-41.504\\
270.564576	-42.725\\
269.893825	-28.076\\
180.444452	-37.842\\
250.13562	-46.387\\
309.169355	-46.387\\
304.901751	-35.4\\
235.3038	-47.607\\
326.869662	-58.594\\
398.029042	-43.945\\
292.102415	-34.18\\
221.55476	-24.414\\
153.344334	-20.752\\
127.666304	-17.09\\
105.46239	-21.973\\
133.178353	-17.09\\
99.51507	-12.207\\
69.518865	-9.766\\
55.968946	-12.207\\
71.752746	-18.311\\
107.961656	-18.311\\
107.961656	-18.311\\
109.975866	-21.973\\
133.178353	-23.193\\
144.817092	-30.518\\
190.554392	-28.076\\
176.345356	-30.518\\
189.425226	-25.635\\
156.296595	-20.752\\
127.666304	-24.414\\
147.509388	-18.311\\
109.298359	-13.428\\
80.648568	-19.531\\
117.654744	-20.752\\
123.121616	-12.207\\
71.972472	-15.869\\
91.818034	-12.207\\
70.19025	-13.428\\
77.694408	-13.428\\
76.217328	-9.766\\
55.431816	-10.986\\
63.1695	-17.09\\
102.01021	-23.193\\
140.572773	-25.635\\
156.78366	-26.855\\
158.33708	-15.869\\
94.436419	-20.752\\
129.575488	-32.959\\
203.983251	-24.414\\
151.537698	-28.076\\
175.306544	-29.297\\
178.096463	-18.311\\
106.954551	-12.207\\
71.972472	-17.09\\
102.01021	-18.311\\
112.997181	-30.518\\
195.01002	-36.621\\
236.022345	-36.621\\
231.334857	-28.076\\
176.850724	-28.076\\
175.811912	-25.635\\
160.06494	-25.635\\
160.06494	-25.635\\
155.835165	-20.752\\
123.495152	-15.869\\
93.849266	-14.648\\
85.558968	-13.428\\
78.674652	-13.428\\
80.406864	-20.752\\
129.949024	-31.738\\
197.600788	-26.855\\
162.25791	-17.09\\
101.39497	-15.869\\
93.563624	-15.869\\
90.659597	-9.766\\
53.644638	-7.324\\
39.967068	-7.324\\
41.439192	-15.869\\
89.50116	-10.986\\
61.752306	-14.648\\
82.878384	-14.648\\
82.61472	-13.428\\
74.498544	-9.766\\
52.755932	-6.104\\
31.966648	-4.883\\
25.92873	-7.324\\
41.71018	-18.311\\
109.975866	-26.855\\
165.21196	-28.076\\
168.119088	-20.752\\
120.444608	-13.428\\
75.73392	-10.986\\
59.54412	-6.104\\
34.310584	-12.207\\
69.518865	-14.648\\
85.558968	-18.311\\
110.305464	-26.855\\
170.63667	-39.063\\
258.909564	-47.607\\
313.825344	-42.725\\
280.062375	-37.842\\
241.81038	-28.076\\
178.394904	-30.518\\
195.01002	-31.738\\
203.408842	-34.18\\
216.5303	-24.414\\
152.441016	-23.193\\
144.817092	-23.193\\
143.124003	-21.973\\
136.386411	-23.193\\
141.848388	-18.311\\
115.670587	-30.518\\
197.268352	-37.842\\
240.448068	-28.076\\
171.179372	-18.311\\
110.635062	-18.311\\
114.333884	-30.518\\
195.589862	-35.4\\
232.6842	-40.283\\
267.761101	-45.166\\
301.844378	-45.166\\
297.734272	-34.18\\
222.81942	-32.959\\
216.046245	-36.621\\
242.723988	-41.504\\
268.281856	-29.297\\
180.791787	-17.09\\
106.70996	-23.193\\
143.541477	-20.752\\
123.868688	-13.428\\
80.890272	-18.311\\
111.312569	-19.531\\
117.303186	-14.648\\
86.100944	-13.428\\
79.413192	-15.869\\
93.277982	-13.428\\
79.668324	-17.09\\
104.83006	-25.635\\
156.78366	-25.635\\
153.50238	-15.869\\
94.436419	-14.648\\
89.850832	-24.414\\
156.908778	-40.283\\
257.40837	-29.297\\
181.319133	-20.752\\
124.636512	-14.648\\
87.433912	-18.311\\
108.968761	-15.869\\
92.405187	-12.207\\
70.849428	-13.428\\
78.929784	-15.869\\
94.722061	-18.311\\
110.305464	-20.752\\
122.353792	-15.869\\
96.182009	-23.193\\
143.958951	-29.297\\
181.846479	-25.635\\
156.78366	-20.752\\
127.292768	-23.193\\
145.234566	-30.518\\
186.648088	-21.973\\
126.34475	-8.545\\
48.817585	-13.428\\
77.452704	-13.428\\
78.929784	-17.09\\
101.07026	-17.09\\
98.88274	-13.428\\
79.171488	-17.09\\
101.39497	-19.531\\
113.729013	-13.428\\
78.674652	-15.869\\
92.405187	-15.869\\
91.818034	-12.207\\
71.301087	-15.869\\
89.50116	-9.766\\
54.54311	-10.986\\
60.543846	-8.545\\
46.937685	-8.545\\
48.971395	-14.648\\
85.822632	-20.752\\
125.383584	-24.414\\
152.441016	-35.4\\
217.1436	-24.414\\
143.505492	-13.428\\
76.714164	-12.207\\
68.84748	-10.986\\
62.960766	-13.428\\
75.478788	-10.986\\
61.35681	-12.207\\
69.286932	-14.648\\
86.628272	-24.414\\
146.191032	-21.973\\
134.782382	-26.855\\
161.77452	-23.193\\
138.021543	-17.09\\
97.32755	-13.428\\
74.740248	-10.986\\
59.346372	-8.545\\
45.690115	-6.104\\
32.74796	-9.766\\
54.718898	-14.648\\
83.684024	-18.311\\
105.947446	-18.311\\
106.624953	-19.531\\
119.080507	-29.297\\
180.791787	-25.635\\
154.42524	-18.311\\
111.312569	-24.414\\
149.316024	-24.414\\
152.001564	-31.738\\
195.855198	-26.855\\
161.29113	-17.09\\
97.00284	-9.766\\
52.93172	-6.104\\
32.638088	-7.324\\
39.425092	-9.766\\
52.755932	-8.545\\
45.997735	-8.545\\
46.937685	-12.207\\
69.067206	-17.09\\
97.32755	-15.869\\
90.945239	-15.869\\
91.818034	-19.531\\
109.783751	-13.428\\
72.282924	-4.883\\
27.00299	-9.766\\
56.506076	-21.973\\
127.948779	-19.531\\
114.803218	-21.973\\
129.157294	-18.311\\
105.28825	-14.648\\
85.295304	-19.531\\
117.303186	-25.635\\
155.835165	-25.635\\
152.553885	-19.531\\
120.154712	-26.855\\
164.72857	-26.855\\
162.768155	-23.193\\
142.265862	-28.076\\
172.218184	-26.855\\
162.768155	-20.752\\
123.868688	-19.531\\
119.451596	-29.297\\
187.764473	-40.283\\
256.683276	-30.518\\
186.068246	-18.311\\
109.975866	-17.09\\
102.33492	-18.311\\
111.312569	-21.973\\
135.595383	-26.855\\
162.768155	-19.531\\
118.377391	-19.531\\
120.525801	-26.855\\
167.68262	-28.076\\
171.179372	-20.752\\
123.868688	-14.648\\
88.239552	-20.752\\
129.949024	-34.18\\
216.5303	-30.518\\
192.232882	-31.738\\
194.109608	-21.973\\
131.574324	-17.09\\
102.01021	-17.09\\
99.19036	-10.986\\
60.950328	-6.104\\
32.638088	-6.104\\
31.7408	-4.883\\
26.553754	-10.986\\
62.56527	-19.531\\
113.729013	-18.311\\
105.617848	-15.869\\
91.24675	-14.648\\
85.295304	-20.752\\
117.414816	-14.648\\
82.336408	-9.766\\
54.894686	-13.428\\
74.25684	-8.545\\
47.40766	-10.986\\
63.367248	-18.311\\
108.968761	-23.193\\
134.612172	-17.09\\
96.69522	-9.766\\
54.181768	-10.986\\
61.35681	-12.207\\
66.833325	-9.766\\
55.793158	-13.428\\
81.870516	-31.738\\
192.364018	-25.635\\
158.193585	-31.738\\
201.06023	-37.842\\
241.129224	-36.621\\
230.675679	-26.855\\
164.24518	-20.752\\
126.151408	-18.311\\
111.312569	-21.973\\
134.782382	-23.193\\
143.958951	-29.297\\
182.403122	-29.297\\
186.680484	-36.621\\
238.732299	-42.725\\
283.993075	-46.387\\
303.231819	-37.842\\
245.973	-35.4\\
233.3568	-40.283\\
261.114406	-31.738\\
206.900022	-34.18\\
221.55476	-31.738\\
197.600788	-18.311\\
108.639163	-12.207\\
71.301087	-13.428\\
80.151732	-18.311\\
108.291254	-17.09\\
98.2675	-8.545\\
49.44137	-14.648\\
83.947688	-13.428\\
74.740248	-6.104\\
34.09084	-10.986\\
61.752306	-12.207\\
67.724436	-7.324\\
41.571024	-18.311\\
107.284149	-20.752\\
120.838896	-14.648\\
87.975888	-23.193\\
146.510181	-37.842\\
241.129224	-32.959\\
214.2335	-37.842\\
241.81038	-32.959\\
210.014748	-29.297\\
181.846479	-20.752\\
123.868688	-14.648\\
87.170248	-17.09\\
100.13031	-13.428\\
76.714164	-10.986\\
63.564996	-14.648\\
82.072744	-13.428\\
72.04122	-3.662\\
19.511136	-7.324\\
41.036372	-15.869\\
92.690829	-20.752\\
123.868688	-21.973\\
132.365352	-23.193\\
139.297158	-21.973\\
133.178353	-26.855\\
162.25791	-21.973\\
130.365809	-17.09\\
101.39497	-17.09\\
100.13031	-14.648\\
87.712224	-20.752\\
122.353792	-18.311\\
105.617848	-13.428\\
79.171488	-18.311\\
111.312569	-26.855\\
160.80774	-20.752\\
121.980256	-15.869\\
92.103676	-13.428\\
78.191244	-14.648\\
83.684024	-12.207\\
68.395821	-8.545\\
48.817585	-14.648\\
85.016992	-19.531\\
114.803218	-18.311\\
106.954551	-15.869\\
93.277982	-19.531\\
115.506334	-18.311\\
104.610743	-10.986\\
62.158788	-10.986\\
64.773456	-21.973\\
135.177896	-30.518\\
188.875902	-29.297\\
180.235144	-23.193\\
141.407721	-21.973\\
132.365352	-19.531\\
116.228981	-18.311\\
107.961656	-15.869\\
94.722061	-19.531\\
115.877423	-17.09\\
98.57512	-13.428\\
78.432948	-17.09\\
101.39497	-19.531\\
117.303186	-21.973\\
133.573867	-25.635\\
155.835165	-25.635\\
160.06494	-32.959\\
211.827493	-41.504\\
273.594368	-45.166\\
290.281882	-31.738\\
194.680892	-17.09\\
100.45502	-13.428\\
75.478788	-8.545\\
48.1938	-10.986\\
61.96104	-13.428\\
78.432948	-17.09\\
100.45502	-18.311\\
106.624953	-13.428\\
79.171488	-19.531\\
116.951628	-24.414\\
148.412706	-25.635\\
157.70652	-26.855\\
171.12006	-39.063\\
251.761035	-35.4\\
223.6218	-25.635\\
156.78366	-20.752\\
125.777872	-20.752\\
125.777872	-19.531\\
120.154712	-23.193\\
140.132106	-19.531\\
117.303186	-15.869\\
95.023572	-17.09\\
98.57512	-10.986\\
64.169226	-14.648\\
88.503216	-28.076\\
167.585644	-19.531\\
116.228981	-15.869\\
99.086036	-34.18\\
212.80468	-28.076\\
173.256996	-21.973\\
139.616442	-34.18\\
217.79496	-31.738\\
195.855198	-20.752\\
122.353792	-15.869\\
92.976471	-13.428\\
80.890272	-23.193\\
146.927655	-35.4\\
228.153	-37.842\\
245.291844	-36.621\\
234.00819	-29.297\\
180.791787	-20.752\\
125.383584	-15.869\\
93.277982	-14.648\\
84.489664	-10.986\\
62.960766	-10.986\\
63.564996	-13.428\\
78.674652	-15.869\\
90.373955	-10.986\\
62.960766	-13.428\\
79.413192	-21.973\\
131.969838	-21.973\\
134.782382	-26.855\\
169.643035	-35.4\\
230.7726	-43.945\\
284.06048	-28.076\\
178.394904	-28.076\\
178.394904	-28.076\\
176.345356	-24.414\\
149.316024	-18.311\\
111.642167	-20.752\\
131.090384	-34.18\\
220.93952	-37.842\\
236.966604	-24.414\\
145.287714	-12.207\\
69.518865	-10.986\\
59.346372	-3.662\\
19.24381	-8.545\\
46.3139	-9.766\\
54.00598	-12.207\\
69.286932	-13.428\\
76.217328	-13.428\\
74.25684	-7.324\\
40.50172	-8.545\\
46.783875	-10.986\\
59.950602	-8.545\\
46.937685	-12.207\\
68.84748	-14.648\\
86.628272	-24.414\\
147.509388	-28.076\\
169.635192	-24.414\\
149.755476	-28.076\\
176.345356	-35.4\\
225.5688	-35.4\\
228.8256	-40.283\\
261.8395	-40.283\\
255.958182	-29.297\\
180.791787	-19.531\\
118.728949	-18.311\\
114.663482	-30.518\\
195.01002	-37.842\\
237.685602	-26.855\\
163.251545	-15.869\\
90.659597	-9.766\\
53.293062	-4.883\\
26.377966	-6.104\\
31.7408	-6.104\\
32.638088	-6.104\\
33.864992	-12.207\\
67.944162	-14.648\\
81.00344	-12.207\\
65.490555	-9.766\\
51.144542	-4.883\\
25.840836	-8.545\\
45.52776	-8.545\\
45.997735	-8.545\\
45.22014	-7.324\\
38.223956	-4.883\\
26.285189	-12.207\\
71.520813	-25.635\\
157.24509	-24.414\\
152.441016	-31.738\\
195.252176	-25.635\\
160.06494	-32.959\\
215.452983	-46.387\\
304.901751	-42.725\\
281.6432	-39.063\\
252.503232	-34.18\\
219.67486	-31.738\\
205.154432	-32.959\\
208.202003	-20.752\\
128.807664	-20.752\\
123.495152	-13.428\\
77.694408	-14.648\\
88.239552	-23.193\\
138.021543	-20.752\\
122.727328	-15.869\\
95.023572	-19.531\\
116.228981	-17.09\\
101.70259	-17.09\\
102.33492	-20.752\\
125.010048	-19.531\\
118.377391	-24.414\\
147.509388	-20.752\\
122.727328	-15.869\\
94.722061	-18.311\\
104.940341	-14.648\\
78.044544	-2.441\\
12.871393	-9.766\\
53.293062	-13.428\\
71.30268	-8.545\\
42.40029	-4.883\\
24.31734	-8.545\\
44.903975	-12.207\\
66.381666	-12.207\\
67.944162	-14.648\\
81.530768	-13.428\\
77.694408	-19.531\\
114.080571	-19.531\\
111.229045	-12.207\\
70.629702	-19.531\\
109.783751	-17.09\\
92.93542	-7.324\\
38.758608	-7.324\\
37.147328	-3.662\\
18.84099	-10.986\\
56.52297	-8.545\\
45.057785	-7.324\\
40.765384	-19.531\\
110.15484	-20.752\\
115.132096	-10.986\\
60.950328	-14.648\\
80.461464	-12.207\\
67.944162	-13.428\\
73.760004	-14.648\\
79.39216	-8.545\\
45.997735	-10.986\\
57.929178	-7.324\\
39.022272	-7.324\\
38.89044	-9.766\\
53.293062	-14.648\\
80.725128	-14.648\\
79.655824	-9.766\\
52.93172	-9.766\\
52.033248	-9.766\\
52.033248	-9.766\\
55.08024	-18.311\\
108.639163	-26.855\\
157.343445	-18.311\\
106.277044	-17.09\\
96.69522	-15.869\\
92.103676	-21.973\\
129.157294	-21.973\\
125.136235	-13.428\\
76.955868	-14.648\\
82.336408	-13.428\\
76.217328	-17.09\\
94.81532	-12.207\\
66.833325	-8.545\\
47.56147	-10.986\\
62.158788	-15.869\\
88.628365	-12.207\\
67.944162	-14.648\\
82.072744	-14.648\\
84.753328	-21.973\\
127.531292	-18.311\\
109.646268	-25.635\\
159.60351	-35.4\\
219.7278	-28.076\\
169.635192	-18.311\\
107.284149	-15.869\\
94.436419	-23.193\\
141.848388	-28.076\\
169.635192	-20.752\\
125.777872	-23.193\\
137.163402	-18.311\\
102.266935	-7.324\\
40.0989	-8.545\\
49.13375	-20.752\\
121.212432	-18.311\\
105.617848	-18.311\\
108.639163	-23.193\\
138.439017	-23.193\\
137.604069	-19.531\\
113.357924	-15.869\\
92.405187	-18.311\\
109.298359	-23.193\\
137.604069	-19.531\\
115.506334	-18.311\\
107.632058	-19.531\\
112.654808	-13.428\\
77.452704	-14.648\\
83.142048	-13.428\\
75.975624	-13.428\\
78.674652	-20.752\\
124.262976	-25.635\\
154.42524	-23.193\\
139.297158	-23.193\\
136.328454	-17.09\\
97.32755	-12.207\\
69.286932	-15.869\\
91.532392	-15.869\\
93.277982	-21.973\\
131.574324	-24.414\\
149.316024	-28.076\\
170.674004	-23.193\\
135.470313	-13.428\\
80.406864	-24.414\\
152.001564	-36.621\\
225.292392	-24.414\\
148.852158	-23.193\\
143.124003	-29.297\\
178.623809	-21.973\\
131.156837	-20.752\\
124.636512	-20.752\\
124.262976	-18.311\\
109.646268	-20.752\\
126.524944	-25.635\\
154.42524	-19.531\\
118.006302	-21.973\\
132.760866	-21.973\\
132.760866	-20.752\\
123.495152	-19.531\\
116.951628	-21.973\\
129.552808	-17.09\\
100.13031	-18.311\\
107.961656	-17.09\\
101.07026	-17.09\\
97.00284	-8.545\\
45.843925	-2.441\\
12.649262	-2.441\\
12.739579	-8.545\\
48.1938	-19.531\\
116.228981	-25.635\\
153.50238	-23.193\\
135.887787	-17.09\\
100.13031	-20.752\\
121.585968	-19.531\\
113.006366	-15.869\\
88.914007	-12.207\\
68.84748	-14.648\\
82.878384	-14.648\\
82.878384	-12.207\\
69.738591	-15.869\\
90.072444	-13.428\\
75.73392	-10.986\\
60.14835	-8.545\\
46.937685	-14.648\\
84.753328	-21.973\\
126.34475	-17.09\\
98.88274	-17.09\\
98.88274	-17.09\\
100.45502	-23.193\\
137.604069	-20.752\\
125.777872	-28.076\\
168.624456	-21.973\\
131.574324	-23.193\\
140.132106	-23.193\\
134.612172	-15.869\\
93.849266	-20.752\\
121.212432	-19.531\\
114.432129	-20.752\\
122.727328	-21.973\\
131.969838	-25.635\\
153.015315	-19.531\\
117.654744	-24.414\\
150.658794	-30.518\\
186.648088	-25.635\\
155.373735	-23.193\\
138.439017	-18.311\\
109.298359	-21.973\\
130.365809	-19.531\\
114.803218	-14.648\\
85.016992	-13.428\\
77.936112	-15.869\\
91.818034	-13.428\\
80.890272	-26.855\\
167.68262	-35.4\\
221.0376	-29.297\\
182.403122	-29.297\\
181.846479	-29.297\\
175.430436	-18.311\\
107.284149	-14.648\\
84.489664	-12.207\\
69.067206	-10.986\\
62.158788	-10.986\\
63.971478	-18.311\\
109.298359	-23.193\\
140.572773	-25.635\\
154.42524	-21.973\\
128.739807	-15.869\\
95.594856	-25.635\\
158.655015	-30.518\\
191.683558	-31.738\\
197.600788	-25.635\\
165.217575	-42.725\\
274.593575	-32.959\\
202.170506	-18.311\\
108.291254	-13.428\\
77.452704	-13.428\\
79.171488	-19.531\\
118.377391	-26.855\\
167.68262	-31.738\\
196.997766	-26.855\\
163.251545	-19.531\\
114.080571	-13.428\\
76.47246	-13.428\\
76.955868	-13.428\\
77.211	-15.869\\
88.914007	-9.766\\
54.54311	-9.766\\
54.357556	-8.545\\
45.690115	-4.883\\
26.910213	-10.986\\
61.96104	-14.648\\
84.489664	-15.869\\
91.532392	-15.869\\
92.103676	-17.09\\
99.51507	-18.311\\
108.291254	-23.193\\
135.470313	-14.648\\
86.628272	-21.973\\
133.178353	-26.855\\
161.29113	-21.973\\
131.574324	-23.193\\
138.021543	-21.973\\
131.156837	-20.752\\
127.292768	-29.297\\
178.096463	-23.193\\
138.439017	-17.09\\
102.33492	-20.752\\
123.495152	-20.752\\
120.838896	-13.428\\
79.171488	-20.752\\
126.919232	-26.855\\
165.21196	-28.076\\
173.256996	-29.297\\
188.819165	-45.166\\
287.797752	-30.518\\
190.005068	-26.855\\
164.24518	-18.311\\
112.319674	-26.855\\
170.126425	-35.4\\
220.4004	-24.414\\
148.852158	-19.531\\
121.600006	-26.855\\
163.734935	-20.752\\
121.212432	-10.986\\
63.762744	-15.869\\
91.24675	-13.428\\
75.237084	-8.545\\
47.723825	-8.545\\
47.25385	-8.545\\
46.783875	-7.324\\
40.362564	-10.986\\
62.356536	-14.648\\
84.489664	-17.09\\
101.39497	-23.193\\
138.439017	-21.973\\
133.178353	-25.635\\
158.193585	-28.076\\
173.256996	-28.076\\
169.635192	-19.531\\
117.654744	-21.973\\
131.969838	-19.531\\
117.654744	-20.752\\
127.666304	-29.297\\
179.180452	-23.193\\
141.407721	-26.855\\
163.251545	-23.193\\
140.132106	-19.531\\
120.877359	-29.297\\
181.319133	-25.635\\
158.193585	-26.855\\
164.72857	-24.414\\
144.848262	-15.869\\
90.659597	-9.766\\
54.181768	-8.545\\
48.50142	-10.986\\
62.960766	-12.207\\
70.629702	-18.311\\
105.28825	-13.428\\
78.191244	-21.973\\
134.386868	-31.738\\
200.488946	-36.621\\
230.016501	-26.855\\
168.16601	-29.297\\
182.930468	-26.855\\
163.734935	-18.311\\
111.312569	-21.973\\
133.969381	-23.193\\
140.572773	-19.531\\
118.728949	-23.193\\
143.541477	-26.855\\
171.60345	-39.063\\
251.057901	-34.18\\
218.4102	-31.738\\
204.55141	-37.842\\
239.72907	-28.076\\
176.850724	-30.518\\
188.875902	-24.414\\
150.194928	-23.193\\
144.817092	-30.518\\
193.33153	-34.18\\
205.90032	-9.766\\
57.580336	-20.752\\
121.980256	-14.648\\
87.170248	-23.193\\
139.297158	-20.752\\
121.212432	-13.428\\
78.674652	-17.09\\
103.89011	-26.855\\
173.080475	-43.945\\
289.68544	-42.725\\
280.062375	-37.842\\
244.610688	-32.959\\
214.2335	-39.063\\
258.20643	-45.166\\
294.437154	-32.959\\
213.640238	-34.18\\
222.81942	-36.621\\
233.349012	-25.635\\
160.06494	-21.973\\
138.803441	-26.855\\
165.722205	-19.531\\
116.951628	-13.428\\
80.890272	-19.531\\
116.228981	-13.428\\
79.668324	-19.531\\
115.877423	-15.869\\
95.023572	-19.531\\
119.451596	-24.414\\
148.852158	-21.973\\
131.156837	-15.869\\
95.023572	-19.531\\
118.006302	-21.973\\
134.386868	-23.193\\
141.407721	-19.531\\
117.303186	-15.869\\
97.927599	-26.855\\
166.205595	-26.855\\
162.25791	-18.311\\
112.997181	-24.414\\
151.537698	-21.973\\
133.969381	-18.311\\
108.291254	-13.428\\
76.955868	-8.545\\
47.40766	-4.883\\
27.984473	-18.311\\
106.277044	-13.428\\
76.47246	-12.207\\
67.944162	-7.324\\
40.50172	-12.207\\
69.958317	-17.09\\
100.13031	-20.752\\
123.868688	-23.193\\
141.848388	-28.076\\
172.218184	-26.855\\
165.722205	-26.855\\
161.77452	-18.311\\
102.926131	-7.324\\
41.036372	-15.869\\
88.628365	-10.986\\
61.752306	-14.648\\
86.100944	-20.752\\
123.868688	-20.752\\
130.343312	-36.621\\
228.002346	-28.076\\
171.712816	-25.635\\
160.52637	-34.18\\
212.15526	-23.193\\
139.297158	-13.428\\
78.191244	-13.428\\
78.674652	-17.09\\
102.33492	-20.752\\
128.434128	-30.518\\
186.648088	-21.973\\
131.574324	-19.531\\
116.228981	-17.09\\
102.33492	-20.752\\
126.919232	-25.635\\
163.80765	-40.283\\
258.898841	-32.959\\
211.234231	-32.959\\
205.202734	-21.973\\
131.156837	-14.648\\
87.170248	-14.648\\
82.878384	-7.324\\
41.571024	-9.766\\
55.61737	-8.545\\
48.663775	-13.428\\
79.413192	-23.193\\
143.124003	-31.738\\
199.917662	-31.738\\
205.725716	-43.945\\
286.477455	-35.4\\
220.4004	-20.752\\
126.151408	-21.973\\
132.760866	-19.531\\
119.080507	-21.973\\
136.386411	-29.297\\
187.20783	-36.621\\
231.334857	-26.855\\
165.21196	-19.531\\
120.154712	-21.973\\
137.594926	-29.297\\
180.235144	-19.531\\
116.951628	-14.648\\
85.558968	-13.428\\
75.975624	-9.766\\
54.00598	-7.324\\
39.564248	-6.104\\
33.529272	-12.207\\
69.518865	-17.09\\
98.88274	-18.311\\
104.610743	-10.986\\
62.158788	-12.207\\
65.710281	-6.104\\
30.849616	-2.441\\
12.202559	-6.104\\
31.856776	-10.986\\
59.741868	-13.428\\
75.478788	-17.09\\
98.57512	-19.531\\
114.803218	-23.193\\
140.990247	-29.297\\
186.680484	-40.283\\
261.114406	-40.283\\
263.329971	-42.725\\
279.293325	-37.842\\
236.966604	-21.973\\
134.386868	-20.752\\
126.524944	-21.973\\
135.990897	-28.076\\
173.256996	-21.973\\
132.365352	-17.09\\
100.45502	-12.207\\
69.958317	-12.207\\
71.520813	-19.531\\
115.506334	-17.09\\
97.32755	-8.545\\
46.937685	-6.104\\
33.75512	-10.986\\
62.763018	-14.648\\
82.878384	-9.766\\
56.1545	-18.311\\
104.610743	-12.207\\
71.081361	-21.973\\
133.573867	-29.297\\
177.569117	-21.973\\
135.595383	-29.297\\
188.291819	-40.283\\
264.055065	-42.725\\
276.94345	-31.738\\
197.600788	-24.414\\
147.973254	-17.09\\
98.88274	-10.986\\
63.971478	-18.311\\
105.28825	-9.766\\
57.756124	-20.752\\
124.262976	-14.648\\
86.100944	-14.648\\
86.906584	-19.531\\
112.30325	-12.207\\
70.849428	-20.752\\
119.697536	-13.428\\
74.740248	-9.766\\
54.181768	-10.986\\
60.14835	-7.324\\
39.827912	-8.545\\
45.997735	-6.104\\
32.07652	-7.324\\
38.758608	-6.104\\
32.857832	-9.766\\
52.755932	-9.766\\
52.755932	-9.766\\
54.894686	-14.648\\
85.016992	-20.752\\
120.071072	-15.869\\
91.24675	-15.869\\
94.150777	-25.635\\
156.296595	-30.518\\
185.518922	-21.973\\
133.178353	-25.635\\
148.32411	-9.766\\
53.644638	-7.324\\
41.168204	-12.207\\
71.301087	-20.752\\
123.121616	-20.752\\
126.919232	-30.518\\
188.326578	-29.297\\
177.569117	-19.531\\
114.803218	-14.648\\
85.016992	-18.311\\
106.954551	-17.09\\
98.88274	-14.648\\
81.80908	-7.324\\
40.633552	-13.428\\
73.5183	-10.986\\
59.346372	-7.324\\
38.355788	-7.324\\
39.425092	-12.207\\
67.944162	-14.648\\
82.61472	-15.869\\
88.326854	-8.545\\
49.13375	-19.531\\
117.303186	-25.635\\
151.14396	-17.09\\
101.70259	-25.635\\
155.373735	-25.635\\
152.553885	-19.531\\
111.932161	-12.207\\
69.518865	-19.531\\
113.357924	-17.09\\
96.06289	-9.766\\
54.357556	-14.648\\
80.725128	-7.324\\
38.89044	-4.883\\
25.572271	-4.883\\
26.016624	-8.545\\
46.783875	-13.428\\
73.276596	-12.207\\
67.50471	-13.428\\
75.73392	-15.869\\
88.041212	-9.766\\
52.033248	-7.324\\
37.952968	-6.104\\
31.856776	-7.324\\
38.48762	-9.766\\
52.218802	-9.766\\
55.08024	-15.869\\
92.103676	-18.311\\
105.28825	-15.869\\
91.24675	-17.09\\
100.13031	-21.973\\
132.760866	-28.076\\
173.256996	-30.518\\
189.425226	-30.518\\
186.648088	-21.973\\
133.969381	-24.414\\
149.316024	-24.414\\
149.755476	-25.635\\
159.116445	-29.297\\
186.153138	-37.842\\
240.448068	-32.959\\
207.608741	-29.297\\
182.930468	-24.414\\
150.658794	-24.414\\
150.658794	-23.193\\
143.541477	-29.297\\
177.012474	-17.09\\
102.33492	-21.973\\
133.573867	-23.193\\
142.683336	-26.855\\
163.734935	-20.752\\
126.524944	-25.635\\
155.835165	-18.311\\
109.646268	-18.311\\
106.277044	-12.207\\
71.752746	-20.752\\
126.919232	-28.076\\
170.674004	-20.752\\
121.585968	-13.428\\
78.191244	-13.428\\
75.237084	-9.766\\
53.107508	-7.324\\
40.50172	-9.766\\
52.93172	-4.883\\
26.109401	-7.324\\
39.425092	-8.545\\
45.057785	-4.883\\
26.285189	-9.766\\
52.570378	-7.324\\
38.223956	-4.883\\
25.92873	-7.324\\
39.29326	-12.207\\
66.613599	-10.986\\
60.14835	-13.428\\
73.760004	-13.428\\
75.478788	-15.869\\
89.50116	-13.428\\
75.975624	-17.09\\
102.64254	-29.297\\
173.262458	-20.752\\
120.838896	-13.428\\
78.929784	-19.531\\
112.30325	-13.428\\
74.001708	-8.545\\
48.1938	-15.869\\
89.50116	-8.545\\
46.46771	-6.104\\
33.974864	-12.207\\
68.395821	-15.869\\
89.786802	-15.869\\
88.041212	-9.766\\
52.218802	-6.104\\
31.856776	-6.104\\
31.7408	-6.104\\
32.857832	-12.207\\
66.381666	-12.207\\
66.833325	-12.207\\
68.84748	-18.311\\
107.284149	-24.414\\
142.162722	-15.869\\
93.849266	-23.193\\
140.572773	-28.076\\
169.635192	-25.635\\
154.42524	-21.973\\
138.803441	-36.621\\
230.675679	-30.518\\
193.911372	-32.959\\
208.202003	-28.076\\
177.86146	-35.4\\
227.5158	-37.842\\
235.604292	-25.635\\
152.092455	-13.428\\
77.452704	-13.428\\
78.674652	-18.311\\
109.298359	-21.973\\
131.574324	-23.193\\
136.745928	-14.648\\
82.336408	-8.545\\
47.723825	-13.428\\
78.674652	-21.973\\
134.386868	-28.076\\
172.723552	-25.635\\
152.553885	-15.869\\
91.818034	-13.428\\
78.674652	-17.09\\
103.89011	-26.855\\
167.19923	-30.518\\
191.103716	-30.518\\
186.648088	-23.193\\
144.399618	-31.738\\
201.06023	-34.18\\
215.91506	-30.518\\
198.367	-42.725\\
277.7125	-36.621\\
235.363167	-30.518\\
198.367	-37.842\\
244.610688	-36.621\\
225.988191	-18.311\\
110.305464	-17.09\\
104.19773	-20.752\\
128.060592	-25.635\\
158.655015	-26.855\\
163.251545	-18.311\\
112.649272	-24.414\\
149.316024	-21.973\\
131.574324	-14.648\\
88.503216	-19.531\\
117.654744	-18.311\\
114.004286	-26.855\\
168.676255	-30.518\\
188.326578	-23.193\\
139.714632	-17.09\\
98.57512	-10.986\\
63.762744	-15.869\\
91.532392	-12.207\\
69.286932	-12.207\\
68.615547	-7.324\\
42.113	-15.869\\
93.849266	-19.531\\
116.580539	-23.193\\
138.879684	-21.973\\
127.948779	-13.428\\
75.975624	-9.766\\
54.00598	-7.324\\
40.230732	-9.766\\
55.431816	-14.648\\
83.947688	-17.09\\
98.2675	-14.648\\
86.906584	-24.414\\
148.852158	-26.855\\
166.688985	-30.518\\
190.554392	-30.518\\
193.911372	-35.4\\
219.0906	-23.193\\
139.714632	-19.531\\
115.506334	-14.648\\
86.100944	-17.09\\
102.01021	-20.752\\
123.495152	-15.869\\
91.818034	-10.986\\
62.56527	-10.986\\
64.575708	-18.311\\
110.305464	-23.193\\
139.297158	-20.752\\
128.807664	-31.738\\
201.663252	-36.621\\
228.661524	-25.635\\
154.88667	-18.311\\
106.954551	-13.428\\
77.452704	-10.986\\
64.773456	-20.752\\
122.727328	-14.648\\
87.433912	-17.09\\
102.95016	-23.193\\
140.990247	-25.635\\
156.78366	-24.414\\
150.194928	-28.076\\
169.129824	-18.311\\
109.975866	-21.973\\
130.365809	-15.869\\
92.405187	-14.648\\
87.433912	-20.752\\
127.666304	-28.076\\
175.306544	-30.518\\
184.969598	-18.311\\
115.670587	-29.297\\
187.764473	-30.518\\
190.005068	-21.973\\
130.365809	-14.648\\
87.975888	-21.973\\
131.574324	-17.09\\
101.70259	-17.09\\
102.64254	-19.531\\
115.506334	-13.428\\
80.151732	-18.311\\
115.011391	-34.18\\
218.4102	-36.621\\
228.661524	-25.635\\
159.116445	-28.076\\
175.811912	-28.076\\
175.811912	-28.076\\
172.723552	-20.752\\
126.919232	-20.752\\
123.495152	-14.648\\
87.975888	-20.752\\
125.777872	-20.752\\
130.343312	-32.959\\
211.234231	-37.842\\
250.816776	-47.607\\
312.063885	-37.842\\
243.210534	-30.518\\
192.782206	-23.193\\
146.092707	-25.635\\
158.193585	-19.531\\
116.951628	-14.648\\
86.364608	-13.428\\
79.413192	-15.869\\
91.532392	-10.986\\
61.35681	-7.324\\
41.30736	-12.207\\
70.849428	-17.09\\
97.32755	-12.207\\
67.944162	-7.324\\
40.230732	-7.324\\
42.376664	-18.311\\
110.305464	-25.635\\
148.32411	-12.207\\
70.19025	-12.207\\
70.409976	-15.869\\
91.532392	-14.648\\
85.558968	-18.311\\
106.624953	-17.09\\
97.63517	-10.986\\
61.148076	-8.545\\
46.3139	-7.324\\
39.967068	-9.766\\
55.793158	-18.311\\
107.284149	-18.311\\
105.617848	-15.869\\
88.914007	-10.986\\
59.346372	-7.324\\
39.69608	-8.545\\
48.1938	-17.09\\
99.19036	-19.531\\
114.803218	-19.531\\
112.654808	-15.869\\
90.659597	-13.428\\
76.955868	-14.648\\
85.822632	-20.752\\
126.151408	-28.076\\
175.306544	-32.959\\
203.389989	-24.414\\
145.287714	-17.09\\
100.76264	-19.531\\
115.154776	-18.311\\
107.961656	-17.09\\
98.2675	-12.207\\
70.629702	-13.428\\
78.191244	-15.869\\
92.405187	-14.648\\
82.61472	-10.986\\
59.741868	-6.104\\
33.639144	-8.545\\
48.1938	-13.428\\
78.929784	-20.752\\
124.636512	-24.414\\
149.316024	-28.076\\
171.712816	-25.635\\
158.193585	-29.297\\
185.596495	-36.621\\
240.709833	-47.607\\
313.825344	-37.842\\
241.81038	-28.076\\
178.900272	-31.738\\
204.55141	-36.621\\
241.405632	-43.945\\
283.225525	-30.518\\
184.969598	-13.428\\
77.452704	-9.766\\
55.431816	-10.986\\
61.148076	-4.883\\
26.109401	-4.883\\
25.840836	-6.104\\
32.973808	-9.766\\
54.357556	-13.428\\
75.478788	-14.648\\
81.267104	-10.986\\
60.543846	-8.545\\
47.723825	-13.428\\
75.975624	-15.869\\
90.945239	-17.09\\
97.94279	-15.869\\
89.199649	-13.428\\
73.5183	-8.545\\
46.16009	-8.545\\
47.091495	-10.986\\
61.554558	-14.648\\
81.267104	-10.986\\
59.137638	-7.324\\
40.362564	-10.986\\
59.741868	-10.986\\
59.950602	-8.545\\
47.40766	-13.428\\
76.47246	-18.311\\
106.277044	-18.311\\
103.933236	-13.428\\
77.936112	-19.531\\
114.803218	-20.752\\
120.444608	-17.09\\
97.94279	-14.648\\
86.100944	-21.973\\
133.178353	-26.855\\
163.734935	-29.297\\
180.791787	-29.297\\
179.180452	-24.414\\
147.509388	-21.973\\
132.365352	-20.752\\
126.919232	-28.076\\
170.674004	-21.973\\
135.595383	-26.855\\
166.688985	-28.076\\
173.762364	-26.855\\
175.551135	-47.607\\
310.350033	-37.842\\
235.604292	-20.752\\
123.868688	-14.648\\
86.628272	-14.648\\
87.433912	-18.311\\
111.312569	-24.414\\
151.098246	-26.855\\
168.16601	-29.297\\
183.457814	-31.738\\
194.680892	-20.752\\
125.383584	-19.531\\
118.728949	-20.752\\
126.524944	-20.752\\
122.727328	-14.648\\
83.684024	-12.207\\
71.081361	-17.09\\
100.45502	-17.09\\
102.33492	-21.973\\
135.177896	-28.076\\
171.712816	-25.635\\
153.50238	-18.311\\
107.632058	-14.648\\
87.712224	-21.973\\
135.177896	-28.076\\
176.850724	-34.18\\
219.67486	-37.842\\
247.373154	-42.725\\
277.7125	-35.4\\
228.153	-32.959\\
204.576513	-20.752\\
123.495152	-12.207\\
74.658012	-24.414\\
153.783786	-32.959\\
202.170506	-19.531\\
121.951564	-25.635\\
165.70464	-36.621\\
241.405632	-45.166\\
291.09487	-31.738\\
194.680892	-18.311\\
106.954551	-12.207\\
71.520813	-15.869\\
93.277982	-14.648\\
85.822632	-13.428\\
76.47246	-10.986\\
62.763018	-12.207\\
69.067206	-9.766\\
55.61737	-12.207\\
68.84748	-10.986\\
62.356536	-13.428\\
78.929784	-20.752\\
122.353792	-20.752\\
125.383584	-24.414\\
151.537698	-30.518\\
190.554392	-31.738\\
191.189712	-19.531\\
116.228981	-17.09\\
102.95016	-21.973\\
129.948322	-14.648\\
85.295304	-13.428\\
76.47246	-10.986\\
63.1695	-14.648\\
84.489664	-14.648\\
82.072744	-8.545\\
45.690115	-4.883\\
25.3916	-6.104\\
32.74796	-8.545\\
47.56147	-14.648\\
82.878384	-14.648\\
80.725128	-9.766\\
54.00598	-13.428\\
76.47246	-17.09\\
100.45502	-20.752\\
120.071072	-17.09\\
97.63517	-14.648\\
84.489664	-17.09\\
100.13031	-20.752\\
121.585968	-18.311\\
107.961656	-20.752\\
122.727328	-20.752\\
120.444608	-13.428\\
76.47246	-12.207\\
69.958317	-15.869\\
90.373955	-13.428\\
77.211	-14.648\\
86.100944	-20.752\\
127.292768	-30.518\\
185.518922	-24.414\\
146.630484	-20.752\\
129.575488	-31.738\\
196.997766	-23.193\\
138.879684	-15.869\\
91.532392	-14.648\\
82.61472	-9.766\\
54.718898	-9.766\\
53.820426	-7.324\\
40.0989	-7.324\\
41.842012	-17.09\\
93.87537	-13.428\\
71.799516	-7.324\\
40.0989	-13.428\\
75.478788	-14.648\\
82.072744	-10.986\\
60.14835	-9.766\\
51.681672	-4.883\\
26.109401	-6.104\\
34.310584	-17.09\\
99.82269	-20.752\\
120.838896	-15.869\\
90.659597	-12.207\\
70.19025	-17.09\\
98.2675	-18.311\\
106.624953	-20.752\\
122.727328	-21.973\\
132.760866	-26.855\\
163.251545	-26.855\\
161.77452	-21.973\\
128.739807	-15.869\\
90.945239	-12.207\\
69.518865	-13.428\\
78.191244	-15.869\\
90.072444	-13.428\\
74.740248	-8.545\\
49.28756	-18.311\\
108.291254	-23.193\\
137.163402	-20.752\\
124.262976	-24.414\\
150.194928	-30.518\\
181.612618	-20.752\\
126.919232	-23.193\\
144.817092	-30.518\\
188.875902	-24.414\\
151.537698	-29.297\\
182.403122	-29.297\\
189.903154	-45.166\\
298.54726	-42.725\\
274.593575	-30.518\\
197.268352	-32.959\\
217.265728	-42.725\\
282.41225	-41.504\\
276.62416	-46.387\\
309.169355	-43.945\\
299.35334	-53.711\\
363.892025	-46.387\\
305.783104	-29.297\\
184.541803	-19.531\\
121.228917	-20.752\\
126.919232	-14.648\\
87.712224	-14.648\\
88.781528	-19.531\\
122.303122	-29.297\\
179.180452	-20.752\\
124.262976	-14.648\\
87.433912	-17.09\\
99.19036	-12.207\\
70.849428	-12.207\\
68.395821	-3.662\\
20.181282	-9.766\\
52.93172	-9.766\\
53.644638	-9.766\\
53.293062	-10.986\\
58.533408	-6.104\\
31.7408	-4.883\\
24.678682	-3.662\\
18.976484	-7.324\\
37.68198	-7.324\\
37.27916	-6.104\\
32.973808	-13.428\\
75.73392	-18.311\\
105.28825	-17.09\\
96.3876	-13.428\\
76.714164	-15.869\\
94.722061	-26.855\\
154.89964	-18.311\\
100.582323	-4.883\\
26.016624	-7.324\\
37.952968	-4.883\\
25.572271	-4.883\\
26.646531	-15.869\\
90.659597	-18.311\\
101.919026	-12.207\\
69.958317	-18.311\\
109.975866	-29.297\\
178.096463	-25.635\\
153.96381	-18.311\\
105.28825	-13.428\\
77.211	-14.648\\
86.364608	-23.193\\
140.572773	-26.855\\
157.85369	-18.311\\
102.596533	-10.986\\
61.96104	-12.207\\
68.395821	-12.207\\
64.81917	-4.883\\
24.678682	-3.662\\
19.111978	-7.324\\
41.71018	-20.752\\
121.212432	-20.752\\
118.929712	-13.428\\
74.001708	-10.986\\
59.54412	-10.986\\
55.31451	-10.986\\
55.523244	-4.883\\
24.766576	-2.441\\
12.6932	-7.324\\
39.69608	-15.869\\
90.659597	-21.973\\
128.739807	-23.193\\
137.163402	-21.973\\
130.761323	-23.193\\
138.439017	-23.193\\
138.439017	-24.414\\
144.384396	-19.531\\
116.580539	-25.635\\
159.60351	-35.4\\
222.9846	-34.18\\
205.90032	-18.311\\
103.27404	-8.545\\
50.22751	-14.648\\
89.045192	-28.076\\
169.635192	-21.973\\
127.531292	-15.869\\
90.945239	-13.428\\
79.668324	-24.414\\
152.441016	-34.18\\
216.5303	-37.842\\
243.210534	-37.842\\
243.89169	-39.063\\
248.206302	-29.297\\
185.069149	-28.076\\
167.585644	-18.311\\
105.617848	-12.207\\
71.752746	-18.311\\
109.298359	-19.531\\
117.303186	-20.752\\
125.383584	-23.193\\
136.745928	-15.869\\
94.436419	-17.09\\
101.07026	-20.752\\
118.556176	-9.766\\
53.293062	-6.104\\
32.302368	-6.104\\
32.74796	-8.545\\
45.22014	-6.104\\
31.295208	-1.221\\
6.394377	-4.883\\
27.54012	-17.09\\
95.12294	-13.428\\
73.5183	-10.986\\
61.148076	-15.869\\
93.277982	-24.414\\
142.602174	-20.752\\
115.132096	-7.324\\
38.355788	-6.104\\
33.4194	-8.545\\
50.38132	-26.855\\
165.722205	-31.738\\
203.408842	-41.504\\
275.877088	-50.049\\
336.32928	-47.607\\
312.063885	-34.18\\
221.55476	-34.18\\
225.9298	-43.945\\
289.68544	-36.621\\
239.391477	-32.959\\
217.265728	-37.842\\
250.816776	-41.504\\
276.62416	-41.504\\
284.966464	-56.152\\
379.419064	-41.504\\
267.49328	-21.973\\
132.365352	-13.428\\
77.452704	-9.766\\
55.968946	-12.207\\
69.958317	-12.207\\
69.738591	-10.986\\
65.377686	-20.752\\
122.727328	-18.311\\
107.632058	-19.531\\
117.303186	-20.752\\
121.212432	-14.648\\
85.016992	-13.428\\
79.668324	-21.973\\
132.365352	-23.193\\
135.887787	-13.428\\
75.73392	-7.324\\
41.439192	-9.766\\
54.894686	-12.207\\
72.643857	-29.297\\
184.541803	-35.4\\
224.259	-32.959\\
211.827493	-35.4\\
231.4098	-43.945\\
297.727375	-52.49\\
357.56188	-46.387\\
304.901751	-30.518\\
192.782206	-19.531\\
116.580539	-13.428\\
78.191244	-13.428\\
78.432948	-14.648\\
86.906584	-17.09\\
100.45502	-9.766\\
56.506076	-12.207\\
71.081361	-13.428\\
78.674652	-15.869\\
93.849266	-19.531\\
117.654744	-20.752\\
123.495152	-18.311\\
103.27404	-8.545\\
46.3139	-4.883\\
25.748059	-4.883\\
25.3916	-3.662\\
19.177894	-6.104\\
33.193552	-10.986\\
60.75258	-14.648\\
82.072744	-13.428\\
78.674652	-23.193\\
135.052839	-18.311\\
109.975866	-23.193\\
147.785796	-40.283\\
255.958182	-30.518\\
192.232882	-25.635\\
162.88479	-32.959\\
212.420755	-36.621\\
231.334857	-26.855\\
166.205595	-21.973\\
131.574324	-15.869\\
93.563624	-15.869\\
98.800394	-30.518\\
202.853146	-48.828\\
336.18078	-57.373\\
391.85759	-46.387\\
313.436959	-39.063\\
256.057965	-30.518\\
199.496166	-30.518\\
203.40247	-37.842\\
243.89169	-28.076\\
175.811912	-19.531\\
119.803154	-19.531\\
123.728885	-25.635\\
164.756145	-29.297\\
179.707798	-15.869\\
95.594856	-14.648\\
86.100944	-14.648\\
87.712224	-24.414\\
152.001564	-28.076\\
177.356092	-29.297\\
182.930468	-20.752\\
126.151408	-17.09\\
103.89011	-17.09\\
102.95016	-15.869\\
92.976471	-10.986\\
62.56527	-9.766\\
54.357556	-8.545\\
46.630065	-6.104\\
33.309528	-7.324\\
41.30736	-12.207\\
71.520813	-18.311\\
107.961656	-18.311\\
105.28825	-13.428\\
75.73392	-9.766\\
54.357556	-8.545\\
47.877635	-10.986\\
62.960766	-15.869\\
92.103676	-18.311\\
106.277044	-15.869\\
89.199649	-10.986\\
63.564996	-24.414\\
148.412706	-26.855\\
161.29113	-20.752\\
116.646992	-8.545\\
47.091495	-13.428\\
76.47246	-17.09\\
98.2675	-17.09\\
97.94279	-12.207\\
67.944162	-8.545\\
48.663775	-12.207\\
72.424131	-26.855\\
155.38303	-13.428\\
72.538056	-7.324\\
39.161428	-9.766\\
55.793158	-18.311\\
108.968761	-21.973\\
126.740264	-12.207\\
69.067206	-10.986\\
64.575708	-20.752\\
126.919232	-29.297\\
179.707798	-25.635\\
160.52637	-37.842\\
245.291844	-40.283\\
255.958182	-30.518\\
188.875902	-23.193\\
146.927655	-30.518\\
190.005068	-25.635\\
161.013435	-26.855\\
172.597085	-36.621\\
232.689834	-28.076\\
167.585644	-14.648\\
88.503216	-25.635\\
163.80765	-36.621\\
241.405632	-41.504\\
269.776	-31.738\\
203.408842	-28.076\\
181.988632	-34.18\\
220.2901	-30.518\\
189.425226	-17.09\\
104.19773	-17.09\\
105.13768	-21.973\\
135.990897	-23.193\\
143.541477	-20.752\\
129.201952	-24.414\\
150.194928	-15.869\\
97.054804	-19.531\\
121.600006	-28.076\\
170.674004	-19.531\\
115.506334	-13.428\\
77.936112	-10.986\\
61.96104	-8.545\\
48.031445	-8.545\\
49.911345	-17.09\\
100.45502	-18.311\\
109.646268	-19.531\\
120.525801	-26.855\\
169.159645	-31.738\\
198.172072	-28.076\\
172.723552	-21.973\\
128.344293	-13.428\\
77.936112	-15.869\\
97.054804	-29.297\\
177.569117	-20.752\\
119.697536	-7.324\\
43.182304	-17.09\\
104.83006	-25.635\\
159.60351	-28.076\\
179.40564	-35.4\\
234.6312	-46.387\\
304.066785	-36.621\\
236.022345	-29.297\\
189.903154	-34.18\\
216.5303	-25.635\\
158.655015	-18.311\\
113.326779	-21.973\\
137.594926	-29.297\\
185.069149	-28.076\\
177.86146	-28.076\\
172.218184	-18.311\\
109.646268	-14.648\\
86.100944	-12.207\\
72.643857	-17.09\\
101.70259	-18.311\\
107.632058	-13.428\\
81.870516	-24.414\\
151.537698	-30.518\\
186.648088	-18.311\\
109.646268	-13.428\\
81.628812	-23.193\\
137.604069	-14.648\\
82.878384	-6.104\\
34.200712	-12.207\\
69.958317	-13.428\\
76.714164	-10.986\\
61.35681	-7.324\\
39.022272	-6.104\\
32.857832	-7.324\\
40.90454	-10.986\\
63.971478	-20.752\\
124.262976	-25.635\\
157.70652	-29.297\\
184.541803	-32.959\\
200.357761	-18.311\\
107.632058	-12.207\\
75.097464	-28.076\\
181.483264	-41.504\\
267.49328	-31.738\\
202.80582	-29.297\\
192.569181	-45.166\\
297.734272	-34.18\\
221.55476	-30.518\\
193.911372	-25.635\\
161.013435	-23.193\\
148.643937	-31.738\\
199.346378	-24.414\\
150.194928	-20.752\\
131.46392	-30.518\\
198.367	-40.283\\
264.780159	-40.283\\
254.467711	-15.869\\
93.277982	-9.766\\
60.265986	-28.076\\
173.256996	-20.752\\
120.444608	-8.545\\
47.40766	-8.545\\
48.1938	-14.648\\
85.558968	-19.531\\
119.451596	-28.076\\
170.674004	-18.311\\
107.961656	-13.428\\
76.47246	-10.986\\
59.741868	-6.104\\
33.08368	-8.545\\
46.16009	-6.104\\
33.193552	-7.324\\
41.439192	-17.09\\
97.94279	-14.648\\
80.1978	-6.104\\
32.638088	-6.104\\
32.973808	-8.545\\
47.091495	-10.986\\
58.742142	-8.545\\
46.16009	-9.766\\
52.033248	-9.766\\
50.968754	-2.441\\
13.274158	-12.207\\
69.958317	-18.311\\
110.305464	-24.414\\
154.223238	-34.18\\
217.79496	-30.518\\
184.389756	-14.648\\
85.016992	-12.207\\
69.958317	-12.207\\
67.944162	-3.662\\
19.712546	-4.883\\
27.271555	-13.428\\
76.47246	-18.311\\
103.933236	-14.648\\
83.684024	-14.648\\
84.753328	-18.311\\
106.954551	-19.531\\
114.803218	-19.531\\
117.303186	-25.635\\
155.373735	-20.752\\
129.949024	-34.18\\
205.90032	-18.311\\
102.926131	-12.207\\
67.053051	-8.545\\
48.50142	-15.869\\
90.373955	-13.428\\
75.478788	-9.766\\
52.39459	-3.662\\
19.511136	-10.986\\
57.929178	-8.545\\
43.18643	-2.441\\
12.336814	-7.324\\
38.758608	-9.766\\
53.46885	-14.648\\
83.42036	-19.531\\
116.951628	-26.855\\
159.814105	-21.973\\
122.301718	-7.324\\
41.168204	-13.428\\
77.452704	-21.973\\
120.697689	-10.986\\
60.14835	-8.545\\
49.44137	-21.973\\
128.739807	-18.311\\
111.990076	-28.076\\
177.86146	-36.621\\
227.306547	-23.193\\
140.132106	-18.311\\
109.298359	-15.869\\
};
\addplot [color=mycolor2, line width=2.0pt, forget plot]
  table[row sep=crcr]{%
120.154712	-18.6663597217865\\
144.399618	-22.4328714905209\\
174.267732	-27.0729638419164\\
165.21196	-25.666125151225\\
129.949024	-20.1879326004216\\
184.014457	-28.5871439513018\\
177.86146	-27.6312592135557\\
207.015479	-32.1604149739207\\
153.015315	-23.7713433388486\\
86.906584	-13.5011730470933\\
123.121616	-19.1272762884543\\
102.33492	-15.8980068032641\\
109.298359	-16.9797958992649\\
116.646992	-18.1214259257377\\
38.758608	-6.0212546574429\\
25.035141	-3.88927691484663\\
26.910213	-4.18057442514526\\
79.413192	-12.3370543181635\\
139.714632	-21.705046234966\\
148.412706	-23.0563155732065\\
153.015315	-23.7713433388486\\
111.932161	-17.3889641686538\\
73.095516	-11.3555862520448\\
144.848262	-22.5025695502277\\
112.30325	-17.446613849199\\
86.364608	-13.4169756085727\\
123.121616	-19.1272762884543\\
107.284149	-16.6668829240732\\
103.25778	-16.0413756020912\\
178.096463	-27.6677675656685\\
153.50238	-23.8470102048962\\
111.642167	-17.3439128158516\\
172.723552	-26.8330712994151\\
170.674004	-26.5146684702771\\
131.574324	-20.4404273544835\\
107.284149	-16.6668829240732\\
93.563624	-14.5353622291398\\
94.722061	-14.7153285525332\\
131.156837	-20.3755696190568\\
131.969838	-20.5018715248877\\
132.365352	-20.563315695292\\
143.958951	-22.3644126793549\\
189.425226	-29.4277215603222\\
177.569117	-27.5858429372457\\
116.951628	-18.1687519528702\\
105.947446	-16.4592224950824\\
62.158788	-9.65653595572901\\
84.489664	-13.1256979834205\\
90.373955	-14.0398385167828\\
69.286932	-10.7639124192729\\
77.694408	-12.0700365716764\\
107.632058	-16.7209315288791\\
114.803218	-17.8349906444507\\
110.982971	-17.2415049330616\\
176.485128	-27.4174423121544\\
127.531292	-19.812331390351\\
69.067206	-10.7297773904591\\
69.958317	-10.8682138962038\\
77.694408	-12.0700365716764\\
95.44765	-14.8280507675735\\
53.820426	-8.36114885029053\\
61.554558	-9.56266718981083\\
81.530768	-12.6660254812272\\
53.46885	-8.30653056710953\\
40.633552	-6.31253228259509\\
75.237084	-11.6882846372456\\
78.432948	-12.1847707596201\\
129.552808	-20.1263793724173\\
123.868688	-19.2433359456909\\
153.50238	-23.8470102048962\\
132.365352	-20.563315695292\\
153.50238	-23.8470102048962\\
123.495152	-19.1853061170726\\
123.121616	-19.1272762884543\\
107.632058	-16.7209315288791\\
107.961656	-16.7721354702741\\
104.19773	-16.1873993786742\\
195.589862	-30.3854144482205\\
270.564576	-42.0329391958314\\
269.893825	-41.9287362125165\\
180.444452	-28.0325340119215\\
250.13562	-38.8592455879058\\
309.169355	-48.0302961017684\\
304.901751	-47.367312269606\\
235.3038	-36.5550822068742\\
326.869662	-50.7800867021408\\
398.029042	-61.8348889862102\\
292.102415	-45.3789007791268\\
221.55476	-34.4191316295117\\
153.344334	-23.8224573310265\\
127.666304	-19.8333058699766\\
105.46239	-16.3838677326224\\
133.178353	-20.689617601123\\
99.51507	-15.4599354735149\\
69.518865	-10.7999438385763\\
55.968946	-8.69492724750771\\
71.752746	-11.1469832981829\\
107.961656	-16.7721354702741\\
109.975866	-17.0850484454658\\
133.178353	-20.689617601123\\
144.817092	-22.4977272063625\\
190.554392	-29.6031405546402\\
176.345356	-27.3957283536454\\
189.425226	-29.4277215603222\\
156.296595	-24.2810990680114\\
127.666304	-19.8333058699766\\
147.509388	-22.9159826769721\\
109.298359	-16.9797958992649\\
80.648568	-12.5289733234512\\
117.654744	-18.2779829265347\\
123.121616	-19.1272762884543\\
71.972472	-11.1811183269967\\
91.818034	-14.2641800979992\\
70.19025	-10.9042453155073\\
77.694408	-12.0700365716764\\
76.217328	-11.840568195789\\
55.431816	-8.61148264820342\\
63.1695	-9.81355280053762\\
102.01021	-15.8475622258991\\
140.572773	-21.8383607619736\\
156.78366	-24.356765934059\\
158.33708	-24.5980939355694\\
94.436419	-14.6709532947102\\
129.575488	-20.1299027718033\\
203.983251	-31.6893501470459\\
151.537698	-23.5417915385577\\
175.306544	-27.2343460977809\\
178.096463	-27.6677675656685\\
106.954551	-16.6156789826782\\
71.972472	-11.1811183269967\\
102.01021	-15.8475622258991\\
112.997181	-17.5544179082533\\
195.01002	-30.2953344241113\\
236.022345	-36.6667101174492\\
231.334857	-35.9384961694223\\
176.850724	-27.4742386402822\\
175.811912	-27.3128563844177\\
160.06494	-24.8665216632217\\
155.835165	-24.2094146685978\\
123.495152	-19.1853061170726\\
93.849266	-14.5797374869628\\
85.558968	-13.2918172540151\\
78.674652	-12.2223201302198\\
80.406864	-12.4914239528514\\
129.949024	-20.1879326004216\\
197.600788	-30.6978172451237\\
162.25791	-25.207205488248\\
101.39497	-15.751983026681\\
93.563624	-14.5353622291398\\
90.659597	-14.0842137746058\\
53.644638	-8.33383970870003\\
39.967068	-6.20899218927928\\
41.439192	-6.43769063715268\\
89.50116	-13.9042474512125\\
61.752306	-9.5933878768386\\
82.878384	-12.8753812743054\\
82.61472	-12.8344203582683\\
74.498544	-11.5735504493019\\
52.755932	-8.19577682621474\\
31.966648	-4.96610528821979\\
25.92873	-4.02809838459832\\
41.71018	-6.47978935641295\\
109.975866	-17.0850484454658\\
165.21196	-25.666125151225\\
168.119088	-26.1177553545023\\
120.444608	-18.7113958500234\\
75.73392	-11.7654694545895\\
59.54412	-9.2503402050285\\
34.310584	-5.33024208995292\\
69.518865	-10.7999438385763\\
85.558968	-13.2918172540151\\
110.305464	-17.1362523868608\\
170.63667	-26.5088685323283\\
258.909564	-40.2223015359972\\
313.825344	-48.7536165948896\\
280.062375	-43.5084479773699\\
241.81038	-37.5658970206835\\
178.394904	-27.7141311827834\\
195.01002	-30.2953344241113\\
203.408842	-31.6001141541917\\
216.5303	-33.6385681692312\\
152.441016	-23.6821244347921\\
144.817092	-22.4977272063625\\
143.124003	-22.2347012476718\\
136.386411	-21.187998094402\\
141.848388	-22.0365310048227\\
115.670587	-17.969738766235\\
197.268352	-30.6461724127473\\
240.448068	-37.3542581642289\\
171.179372	-26.5931787569139\\
110.635062	-17.1874563282558\\
114.333884	-17.7620783372441\\
195.589862	-30.3854144482205\\
232.6842	-36.1481202566247\\
267.761101	-41.597411766653\\
301.844378	-46.8923410989233\\
297.734272	-46.2538250073473\\
222.81942	-34.6155999834599\\
216.046245	-33.5633689148305\\
242.723988	-37.7078284962689\\
268.281856	-41.6783124653864\\
180.791787	-28.0864934442737\\
106.70996	-16.5776811088145\\
143.541477	-22.2995569635133\\
123.868688	-19.2433359456909\\
80.890272	-12.5665226940509\\
111.312569	-17.2927088744566\\
117.303186	-18.2233674397025\\
86.100944	-13.3760146925357\\
79.413192	-12.3370543181635\\
93.277982	-14.4909869713168\\
79.668324	-12.3766897649077\\
104.83006	-16.2856335556483\\
156.78366	-24.356765934059\\
153.50238	-23.8470102048962\\
94.436419	-14.6709532947102\\
89.850832	-13.9585699428401\\
156.908778	-24.3762033540053\\
257.40837	-39.9890869849425\\
181.319133	-28.1684180726965\\
124.636512	-19.3626194822951\\
87.433912	-13.5830948791673\\
108.968761	-16.9285919578699\\
92.405187	-14.3553959057465\\
70.849428	-11.0066504019485\\
78.929784	-12.261955576964\\
94.722061	-14.7153285525332\\
110.305464	-17.1362523868608\\
122.353792	-19.0079927518501\\
96.182009	-14.9421354258508\\
143.958951	-22.3644126793549\\
181.846479	-28.2503427011193\\
156.78366	-24.356765934059\\
127.292768	-19.7752760413583\\
145.234566	-22.562582922204\\
186.648088	-28.9962856553779\\
126.34475	-19.6279988791383\\
48.817585	-7.58394396017434\\
77.452704	-12.0324872010766\\
78.929784	-12.261955576964\\
101.07026	-15.701538449316\\
98.88274	-15.3617012965408\\
79.171488	-12.2995049475638\\
101.39497	-15.751983026681\\
113.729013	-17.668109990241\\
78.674652	-12.2223201302198\\
92.405187	-14.3553959057465\\
91.818034	-14.2641800979992\\
71.301087	-11.0768168500657\\
89.50116	-13.9042474512125\\
54.54311	-8.47341976571813\\
60.543846	-9.40565034500223\\
46.937685	-7.29189640700817\\
48.971395	-7.60783875997885\\
85.822632	-13.3327781700521\\
125.383584	-19.4786791395316\\
152.441016	-23.6821244347921\\
217.1436	-33.7338459841984\\
143.505492	-22.2939665963658\\
76.714164	-11.917753013133\\
68.84748	-10.6956423616454\\
62.960766	-9.78112540867497\\
75.478788	-11.7258340078453\\
61.35681	-9.53194650278306\\
69.286932	-10.7639124192729\\
86.628272	-13.4579365246097\\
146.191032	-22.7111725040895\\
134.782382	-20.9388078477625\\
161.77452	-25.1321095433972\\
138.021543	-21.4420202762753\\
97.32755	-15.1200983207397\\
74.740248	-11.6110998199016\\
59.346372	-9.21961951800073\\
45.690115	-7.09808303081607\\
32.74796	-5.08748422213084\\
54.718898	-8.50072890730862\\
83.684024	-13.0005396288629\\
105.947446	-16.4592224950824\\
106.624953	-16.5644750412832\\
119.080507	-18.4994790675767\\
180.791787	-28.0864934442737\\
154.42524	-23.9903790037232\\
111.312569	-17.2927088744566\\
149.316024	-23.1966484694408\\
152.001564	-23.6138543771646\\
195.855198	-30.426635113983\\
161.29113	-25.0570135985464\\
97.00284	-15.0696537433746\\
52.93172	-8.22308596780524\\
32.638088	-5.0704153095496\\
39.425092	-6.12479475075872\\
52.755932	-8.19577682621474\\
45.997735	-7.14587263042508\\
46.937685	-7.29189640700817\\
69.067206	-10.7297773904591\\
97.32755	-15.1200983207397\\
90.945239	-14.1285890324288\\
91.818034	-14.2641800979992\\
109.783751	-17.0552028602344\\
72.282924	-11.2293478854708\\
27.00299	-4.19498758320692\\
56.506076	-8.77837184681201\\
127.948779	-19.8771891257778\\
114.803218	-17.8349906444507\\
129.157294	-20.064935202013\\
105.28825	-16.3568146122924\\
85.295304	-13.2508563379781\\
117.303186	-18.2233674397025\\
155.835165	-24.2094146685978\\
152.553885	-23.6996589394351\\
120.154712	-18.6663597217865\\
164.72857	-25.5910292063742\\
162.768155	-25.2864734300349\\
142.265862	-22.1013867206642\\
172.218184	-26.7545610127784\\
162.768155	-25.2864734300349\\
123.868688	-19.2433359456909\\
119.451596	-18.5571287481219\\
187.764473	-29.1697190867526\\
256.683276	-39.8764416694919\\
186.068246	-28.9062056312686\\
109.975866	-17.0850484454658\\
102.33492	-15.8980068032641\\
111.312569	-17.2927088744566\\
135.595383	-21.0651097535934\\
162.768155	-25.2864734300349\\
118.377391	-18.3902480939122\\
120.525801	-18.7240094023317\\
167.68262	-26.0499488693513\\
171.179372	-26.5931787569139\\
123.868688	-19.2433359456909\\
88.239552	-13.7082532337249\\
129.949024	-20.1879326004216\\
216.5303	-33.6385681692312\\
192.232882	-29.863898519167\\
194.109608	-30.1554529828423\\
131.574324	-20.4404273544835\\
102.01021	-15.8475622258991\\
99.19036	-15.4094908961498\\
60.950328	-9.46879842389264\\
32.638088	-5.0704153095496\\
31.7408	-4.93101919013613\\
26.553754	-4.12519755469787\\
62.56527	-9.71968403461943\\
113.729013	-17.668109990241\\
105.617848	-16.4080185536874\\
91.24675	-14.1754295823531\\
85.295304	-13.2508563379781\\
117.414816	-18.2407094623419\\
82.336408	-12.7911838357848\\
54.894686	-8.52803804889912\\
74.25684	-11.5360010787021\\
47.40766	-7.36490829529971\\
63.367248	-9.84427348756539\\
108.968761	-16.9285919578699\\
134.612172	-20.9123652635695\\
96.69522	-15.0218641437656\\
54.181768	-8.41728430800433\\
61.35681	-9.53194650278306\\
66.833325	-10.3827379308526\\
55.793158	-8.66761810591722\\
81.870516	-12.7188062525944\\
192.364018	-29.8842708517017\\
158.193585	-24.5758015989336\\
201.06023	-31.2352509232024\\
241.129224	-37.4600775924562\\
230.675679	-35.836091082981\\
164.24518	-25.5159332615234\\
126.151408	-19.5979626761358\\
111.312569	-17.2927088744566\\
134.782382	-20.9388078477625\\
143.958951	-22.3644126793549\\
182.403122	-28.3368186977878\\
186.680484	-29.0013184616614\\
238.732299	-37.0877088061522\\
283.993075	-44.1190928612628\\
303.231819	-47.1078838135424\\
245.973	-38.2125713042946\\
233.3568	-36.2526104870942\\
261.114406	-40.5648297083563\\
206.900022	-32.142478416473\\
221.55476	-34.4191316295117\\
197.600788	-30.6978172451237\\
108.639163	-16.8773880164749\\
71.301087	-11.0768168500657\\
80.151732	-12.4517885061072\\
108.291254	-16.8233394116691\\
98.2675	-15.2661220973228\\
49.44137	-7.68085064827039\\
83.947688	-13.0415005449\\
74.740248	-11.6110998199016\\
34.09084	-5.29610426479044\\
61.752306	-9.5933878768386\\
67.724436	-10.5211744365973\\
41.571024	-6.45817109517119\\
107.284149	-16.6668829240732\\
120.838896	-18.7726495580094\\
87.975888	-13.6672923176879\\
146.510181	-22.7607531650531\\
241.129224	-37.4600775924562\\
214.2335	-33.2817540726771\\
241.81038	-37.5658970206835\\
210.014748	-32.6263595309382\\
181.846479	-28.2503427011193\\
123.868688	-19.2433359456909\\
87.170248	-13.5421339631303\\
100.13031	-15.5555146727329\\
76.714164	-11.917753013133\\
63.564996	-9.87499417459316\\
82.072744	-12.7502229197478\\
72.04122	-11.191798514871\\
19.511136	-3.03110778673997\\
41.036372	-6.37511145987389\\
92.690829	-14.3997711635695\\
123.868688	-19.2433359456909\\
132.365352	-20.563315695292\\
139.297158	-21.6401905191244\\
133.178353	-20.689617601123\\
162.25791	-25.207205488248\\
130.365809	-20.2526812782482\\
101.39497	-15.751983026681\\
100.13031	-15.5555146727329\\
87.712224	-13.6263314016508\\
122.353792	-19.0079927518501\\
105.617848	-16.4080185536874\\
79.171488	-12.2995049475638\\
111.312569	-17.2927088744566\\
160.80774	-24.9819176536956\\
121.980256	-18.9499629232319\\
92.103676	-14.3085553558222\\
78.191244	-12.1472213890203\\
83.684024	-13.0005396288629\\
68.395821	-10.6254759135282\\
48.817585	-7.58394396017434\\
85.016992	-13.2076198154946\\
114.803218	-17.8349906444507\\
106.954551	-16.6156789826782\\
93.277982	-14.4909869713168\\
115.506334	-17.9442216181153\\
104.610743	-16.2515620660916\\
62.158788	-9.65653595572901\\
64.773456	-10.0627317064295\\
135.177896	-21.0002520181667\\
188.875902	-29.3423825901134\\
180.235144	-28.0000174476053\\
141.407721	-21.9680721936566\\
132.365352	-20.563315695292\\
116.228981	-18.0564867854927\\
107.961656	-16.7721354702741\\
94.722061	-14.7153285525332\\
115.877423	-18.0018712986605\\
98.57512	-15.3139116969318\\
78.432948	-12.1847707596201\\
101.39497	-15.751983026681\\
117.303186	-18.2233674397025\\
133.573867	-20.7510617715272\\
155.835165	-24.2094146685978\\
160.06494	-24.8665216632217\\
211.827493	-32.9079743730916\\
273.594368	-42.5036255835129\\
290.281882	-45.0960760500943\\
194.680892	-30.2442034984884\\
100.45502	-15.6059592500979\\
75.478788	-11.7258340078453\\
48.1938	-7.48703727207829\\
61.96104	-9.62581526870124\\
78.432948	-12.1847707596201\\
100.45502	-15.6059592500979\\
106.624953	-16.5644750412832\\
79.171488	-12.2995049475638\\
116.951628	-18.1687519528702\\
148.412706	-23.0563155732065\\
157.70652	-24.500134732886\\
171.12006	-26.5839644771791\\
251.761035	-39.1117582075289\\
223.6218	-34.7402518881939\\
156.78366	-24.356765934059\\
125.777872	-19.5399328475176\\
120.154712	-18.6663597217865\\
140.132106	-21.7699019508075\\
117.303186	-18.2233674397025\\
95.023572	-14.7621691024575\\
98.57512	-15.3139116969318\\
64.169226	-9.96886294051134\\
88.503216	-13.7492141497619\\
167.585644	-26.0348833852745\\
116.228981	-18.0564867854927\\
99.086036	-15.3932838803849\\
212.80468	-33.059783018411\\
173.256996	-26.9159432686429\\
139.616442	-21.6897921527034\\
217.79496	-33.8350365231793\\
195.855198	-30.426635113983\\
122.353792	-19.0079927518501\\
92.976471	-14.4441464213925\\
80.890272	-12.5665226940509\\
146.927655	-22.8256088808947\\
228.153	-35.4441860724093\\
245.291844	-38.1067518760673\\
234.00819	-36.3538056866564\\
180.791787	-28.0864934442737\\
125.383584	-19.4786791395316\\
93.277982	-14.4909869713168\\
84.489664	-13.1256979834205\\
62.960766	-9.78112540867497\\
63.564996	-9.87499417459316\\
78.674652	-12.2223201302198\\
90.373955	-14.0398385167828\\
62.960766	-9.78112540867497\\
79.413192	-12.3370543181635\\
131.969838	-20.5018715248877\\
134.782382	-20.9388078477625\\
169.643035	-26.3545046456906\\
230.7726	-35.8511480226588\\
284.06048	-44.1295644104522\\
178.394904	-27.7141311827834\\
176.345356	-27.3957283536454\\
149.316024	-23.1966484694408\\
111.642167	-17.3439128158516\\
131.090384	-20.365245965644\\
220.93952	-34.3235524302937\\
236.966604	-36.8134033088451\\
145.287714	-22.5708396078552\\
69.518865	-10.7999438385763\\
59.346372	-9.21961951800073\\
19.24381	-2.98957796909131\\
46.3139	-7.19498971891212\\
54.00598	-8.38997516641383\\
69.286932	-10.7639124192729\\
76.217328	-11.840568195789\\
74.25684	-11.5360010787021\\
40.50172	-6.29205182457658\\
46.783875	-7.26800160720366\\
59.950602	-9.31348828391892\\
46.937685	-7.29189640700817\\
68.84748	-10.6956423616454\\
86.628272	-13.4579365246097\\
147.509388	-22.9159826769721\\
169.635192	-26.3532862144126\\
149.755476	-23.2649185270683\\
176.345356	-27.3957283536454\\
225.5688	-35.0427236079739\\
228.8256	-35.5486763028788\\
261.8395	-40.6774750238069\\
255.958182	-39.7637963540414\\
180.791787	-28.0864934442737\\
118.728949	-18.4448635807445\\
114.663482	-17.8132822786391\\
195.01002	-30.2953344241113\\
237.685602	-36.9251015941961\\
163.251545	-25.3615693748857\\
90.659597	-14.0842137746058\\
53.293062	-8.27922142551904\\
26.377966	-4.09788841310737\\
31.7408	-4.93101919013613\\
32.638088	-5.0704153095496\\
33.864992	-5.26101816670678\\
67.944162	-10.555309465411\\
81.00344	-12.5841036491532\\
65.490555	-10.1741349769907\\
51.144542	-7.94544302830185\\
25.840836	-4.01444381380307\\
45.52776	-7.07286074213354\\
45.997735	-7.14587263042508\\
45.22014	-7.02507114252453\\
38.223956	-5.9381950221456\\
26.285189	-4.08347525504572\\
71.520813	-11.1109518788795\\
157.24509	-24.4284503334725\\
152.441016	-23.6821244347921\\
195.252176	-30.3329540141344\\
160.06494	-24.8665216632217\\
215.452983	-33.4712040573985\\
304.901751	-47.367312269606\\
281.6432	-43.7540334198051\\
252.503232	-39.2270605203207\\
219.67486	-34.1270840763455\\
205.154432	-31.8712962853324\\
208.202003	-32.3447446887848\\
128.807664	-20.0106192351991\\
123.495152	-19.1853061170726\\
77.694408	-12.0700365716764\\
88.239552	-13.7082532337249\\
138.021543	-21.4420202762753\\
122.727328	-19.0660225804684\\
95.023572	-14.7621691024575\\
116.228981	-18.0564867854927\\
101.70259	-15.79977262629\\
102.33492	-15.8980068032641\\
125.010048	-19.4206493109134\\
118.377391	-18.3902480939122\\
147.509388	-22.9159826769721\\
122.727328	-19.0660225804684\\
94.722061	-14.7153285525332\\
104.940341	-16.3027660074866\\
78.044544	-12.1244311469599\\
12.871393	-1.99960574045972\\
53.293062	-8.27922142551904\\
71.30268	-11.0770643269274\\
42.40029	-6.58699981277526\\
24.31734	-3.77776458668543\\
44.903975	-6.97595405403749\\
66.381666	-10.3125714827354\\
67.944162	-10.555309465411\\
81.530768	-12.6660254812272\\
77.694408	-12.0700365716764\\
114.080571	-17.7227254770733\\
111.229045	-17.2797331949893\\
70.629702	-10.9725153731348\\
109.783751	-17.0552028602344\\
92.93542	-14.4377690374332\\
38.758608	-6.0212546574429\\
37.147328	-5.77093794832774\\
18.84099	-2.92699879181252\\
56.52297	-8.78099637543757\\
45.057785	-6.99984885384199\\
40.765384	-6.33301274061361\\
110.15484	-17.1128525407795\\
115.132096	-17.886082731897\\
60.950328	-9.46879842389264\\
80.461464	-12.4999062106326\\
67.944162	-10.555309465411\\
73.760004	-11.4588162613582\\
79.39216	-12.333786940038\\
45.997735	-7.14587263042508\\
57.929178	-8.99945459430171\\
39.022272	-6.06221557347993\\
38.89044	-6.04173511546142\\
53.293062	-8.27922142551904\\
80.725128	-12.5408671266696\\
79.655824	-12.374747856075\\
52.93172	-8.22308596780524\\
52.033248	-8.08350591078714\\
55.08024	-8.55686436502242\\
108.639163	-16.8773880164749\\
157.343445	-24.4437300489316\\
106.277044	-16.5104264364774\\
96.69522	-15.0218641437656\\
92.103676	-14.3085553558222\\
129.157294	-20.064935202013\\
125.136235	-19.440252802903\\
76.955868	-11.9553023837327\\
82.336408	-12.7911838357848\\
76.217328	-11.840568195789\\
94.81532	-14.7298165905994\\
66.833325	-10.3827379308526\\
47.56147	-7.38880309510422\\
62.158788	-9.65653595572901\\
88.628365	-13.7686563856421\\
67.944162	-10.555309465411\\
82.072744	-12.7502229197478\\
84.753328	-13.1666588994575\\
127.531292	-19.812331390351\\
109.646268	-17.0338445040708\\
159.60351	-24.7948372638082\\
219.7278	-34.1353084486337\\
169.635192	-26.3532862144126\\
107.284149	-16.6668829240732\\
94.436419	-14.6709532947102\\
141.848388	-22.0365310048227\\
169.635192	-26.3532862144126\\
125.777872	-19.5399328475176\\
137.163402	-21.3087057492677\\
102.266935	-15.8874451495049\\
40.0989	-6.22947264729779\\
49.13375	-7.63306104866138\\
121.212432	-18.8306793866276\\
105.617848	-16.4080185536874\\
108.639163	-16.8773880164749\\
138.439017	-21.5068759921168\\
137.604069	-21.3771645604338\\
113.357924	-17.6104603096958\\
92.405187	-14.3553959057465\\
109.298359	-16.9797958992649\\
137.604069	-21.3771645604338\\
115.506334	-17.9442216181153\\
107.632058	-16.7209315288791\\
112.654808	-17.5012293360313\\
77.452704	-12.0324872010766\\
83.142048	-12.9163421903424\\
75.975624	-11.8030188251893\\
78.674652	-12.2223201302198\\
124.262976	-19.3045896536768\\
154.42524	-23.9903790037232\\
139.297158	-21.6401905191244\\
136.328454	-21.1789943175847\\
97.32755	-15.1200983207397\\
69.286932	-10.7639124192729\\
91.532392	-14.2198048401761\\
93.277982	-14.4909869713168\\
131.574324	-20.4404273544835\\
149.316024	-23.1966484694408\\
170.674004	-26.5146684702771\\
135.470313	-21.0456797905771\\
80.406864	-12.4914239528514\\
152.001564	-23.6138543771646\\
225.292392	-34.9997828770438\\
148.852158	-23.124585630834\\
143.124003	-22.2347012476718\\
178.623809	-27.7496921940912\\
131.156837	-20.3755696190568\\
124.636512	-19.3626194822951\\
124.262976	-19.3045896536768\\
109.646268	-17.0338445040708\\
126.524944	-19.6559925047541\\
154.42524	-23.9903790037232\\
118.006302	-18.332598413367\\
132.760866	-20.6247598656962\\
123.495152	-19.1853061170726\\
116.951628	-18.1687519528702\\
129.552808	-20.1263793724173\\
100.13031	-15.5555146727329\\
107.961656	-16.7721354702741\\
101.07026	-15.701538449316\\
97.00284	-15.0696537433746\\
45.843925	-7.12197783062057\\
12.649262	-1.96509708838655\\
12.739579	-1.97912807878993\\
48.1938	-7.48703727207829\\
116.228981	-18.0564867854927\\
153.50238	-23.8470102048962\\
135.887787	-21.1105355064186\\
100.13031	-15.5555146727329\\
121.585968	-18.8887092152459\\
113.006366	-17.5558448228635\\
88.914007	-13.8130316434651\\
68.84748	-10.6956423616454\\
82.878384	-12.8753812743054\\
69.738591	-10.8340788673901\\
90.072444	-13.9929979668585\\
75.73392	-11.7654694545895\\
60.14835	-9.34420897094669\\
46.937685	-7.29189640700817\\
84.753328	-13.1666588994575\\
126.34475	-19.6279988791383\\
98.88274	-15.3617012965408\\
100.45502	-15.6059592500979\\
137.604069	-21.3771645604338\\
125.777872	-19.5399328475176\\
168.624456	-26.196265641139\\
131.574324	-20.4404273544835\\
140.132106	-21.7699019508075\\
134.612172	-20.9123652635695\\
93.849266	-14.5797374869628\\
121.212432	-18.8306793866276\\
114.432129	-17.7773409639055\\
122.727328	-19.0660225804684\\
131.969838	-20.5018715248877\\
153.015315	-23.7713433388486\\
117.654744	-18.2779829265347\\
150.658794	-23.4052514233027\\
186.648088	-28.9962856553779\\
155.373735	-24.1377302691843\\
138.439017	-21.5068759921168\\
109.298359	-16.9797958992649\\
130.365809	-20.2526812782482\\
114.803218	-17.8349906444507\\
85.016992	-13.2076198154946\\
77.936112	-12.1075859422762\\
91.818034	-14.2641800979992\\
80.890272	-12.5665226940509\\
167.68262	-26.0499488693513\\
221.0376	-34.3387894237585\\
182.403122	-28.3368186977878\\
181.846479	-28.2503427011193\\
175.430436	-27.2535930553089\\
107.284149	-16.6668829240732\\
84.489664	-13.1256979834205\\
69.067206	-10.7297773904591\\
62.158788	-9.65653595572901\\
63.971478	-9.93814225348357\\
109.298359	-16.9797958992649\\
140.572773	-21.8383607619736\\
154.42524	-23.9903790037232\\
128.739807	-20.0000774665863\\
95.594856	-14.8509196181035\\
158.655015	-24.6474859983471\\
191.683558	-29.7785595489582\\
197.600788	-30.6978172451237\\
165.217575	-25.6669974566727\\
274.593575	-42.6588550954319\\
202.170506	-31.4077353048925\\
108.291254	-16.8233394116691\\
77.452704	-12.0324872010766\\
79.171488	-12.2995049475638\\
118.377391	-18.3902480939122\\
167.68262	-26.0499488693513\\
196.997766	-30.6041361452751\\
163.251545	-25.3615693748857\\
114.080571	-17.7227254770733\\
76.47246	-11.8802036425332\\
76.955868	-11.9553023837327\\
77.211	-11.9949378304769\\
88.914007	-13.8130316434651\\
54.54311	-8.47341976571813\\
54.357556	-8.44459344959482\\
45.690115	-7.09808303081607\\
26.910213	-4.18057442514526\\
61.96104	-9.62581526870124\\
84.489664	-13.1256979834205\\
91.532392	-14.2198048401761\\
92.103676	-14.3085553558222\\
99.51507	-15.4599354735149\\
108.291254	-16.8233394116691\\
135.470313	-21.0456797905771\\
86.628272	-13.4579365246097\\
133.178353	-20.689617601123\\
161.29113	-25.0570135985464\\
131.574324	-20.4404273544835\\
138.021543	-21.4420202762753\\
131.156837	-20.3755696190568\\
127.292768	-19.7752760413583\\
178.096463	-27.6677675656685\\
138.439017	-21.5068759921168\\
102.33492	-15.8980068032641\\
123.495152	-19.1853061170726\\
120.838896	-18.7726495580094\\
79.171488	-12.2995049475638\\
126.919232	-19.7172462127401\\
165.21196	-25.666125151225\\
173.256996	-26.9159432686429\\
188.819165	-29.3335683435982\\
287.797752	-44.7101597310099\\
190.005068	-29.5178015844314\\
164.24518	-25.5159332615234\\
112.319674	-17.4491653620524\\
170.126425	-26.4296005905414\\
220.4004	-34.2397986791032\\
148.852158	-23.124585630834\\
121.600006	-18.8908900565414\\
163.734935	-25.4366653197365\\
121.212432	-18.8306793866276\\
63.762744	-9.90571486162093\\
91.24675	-14.1754295823531\\
75.237084	-11.6882846372456\\
47.723825	-7.41402538378675\\
47.25385	-7.34101349549521\\
46.783875	-7.26800160720366\\
40.362564	-6.27043356333482\\
62.356536	-9.68725664275678\\
84.489664	-13.1256979834205\\
101.39497	-15.751983026681\\
138.439017	-21.5068759921168\\
133.178353	-20.689617601123\\
158.193585	-24.5758015989336\\
173.256996	-26.9159432686429\\
169.635192	-26.3532862144126\\
117.654744	-18.2779829265347\\
131.969838	-20.5018715248877\\
117.654744	-18.2779829265347\\
127.666304	-19.8333058699766\\
179.180452	-27.8361681907597\\
141.407721	-21.9680721936566\\
163.251545	-25.3615693748857\\
140.132106	-21.7699019508075\\
120.877359	-18.7786248891639\\
181.319133	-28.1684180726965\\
158.193585	-24.5758015989336\\
164.72857	-25.5910292063742\\
144.848262	-22.5025695502277\\
90.659597	-14.0842137746058\\
54.181768	-8.41728430800433\\
48.50142	-7.5348268716873\\
62.960766	-9.78112540867497\\
70.629702	-10.9725153731348\\
105.28825	-16.3568146122924\\
78.191244	-12.1472213890203\\
134.386868	-20.8773636773582\\
200.488946	-31.1465004075564\\
230.016501	-35.7336859965397\\
168.16601	-26.1250448142021\\
182.930468	-28.4187433262105\\
163.734935	-25.4366653197365\\
111.312569	-17.2927088744566\\
133.969381	-20.8125059419315\\
140.572773	-21.8383607619736\\
118.728949	-18.4448635807445\\
143.541477	-22.2995569635133\\
171.60345	-26.6590604220299\\
251.057901	-39.0025244375157\\
218.4102	-33.9306157223974\\
204.55141	-31.7776151854838\\
239.72907	-37.2425598788779\\
176.850724	-27.4742386402822\\
188.875902	-29.3423825901134\\
150.194928	-23.3331885846959\\
144.817092	-22.4977272063625\\
193.33153	-30.0345764595845\\
205.90032	-31.9871720049643\\
57.580336	-8.9452610454206\\
121.980256	-18.9499629232319\\
87.170248	-13.5421339631303\\
139.297158	-21.6401905191244\\
121.212432	-18.8306793866276\\
78.674652	-12.2223201302198\\
103.89011	-16.1396097790652\\
173.080475	-26.8885202535184\\
289.68544	-45.0034171710552\\
280.062375	-43.5084479773699\\
244.610688	-38.0009324478401\\
214.2335	-33.2817540726771\\
258.20643	-40.1130677659839\\
294.437154	-45.7416088020172\\
213.640238	-33.1895892152451\\
222.81942	-34.6155999834599\\
233.349012	-36.2514006002151\\
160.06494	-24.8665216632217\\
138.803441	-21.5634902468724\\
165.722205	-25.745393093012\\
116.951628	-18.1687519528702\\
80.890272	-12.5665226940509\\
116.228981	-18.0564867854927\\
79.668324	-12.3766897649077\\
115.877423	-18.0018712986605\\
95.023572	-14.7621691024575\\
119.451596	-18.5571287481219\\
148.852158	-23.124585630834\\
131.156837	-20.3755696190568\\
95.023572	-14.7621691024575\\
118.006302	-18.332598413367\\
134.386868	-20.8773636773582\\
141.407721	-21.9680721936566\\
117.303186	-18.2233674397025\\
97.927599	-15.2133175569915\\
166.205595	-25.8204890378627\\
162.25791	-25.207205488248\\
112.997181	-17.5544179082533\\
151.537698	-23.5417915385577\\
133.969381	-20.8125059419315\\
108.291254	-16.8233394116691\\
76.955868	-11.9553023837327\\
47.40766	-7.36490829529971\\
27.984473	-4.34746362375386\\
106.277044	-16.5104264364774\\
76.47246	-11.8802036425332\\
67.944162	-10.555309465411\\
40.50172	-6.29205182457658\\
69.958317	-10.8682138962038\\
100.13031	-15.5555146727329\\
123.868688	-19.2433359456909\\
141.848388	-22.0365310048227\\
172.218184	-26.7545610127784\\
165.722205	-25.745393093012\\
161.77452	-25.1321095433972\\
102.926131	-15.9898530322949\\
41.036372	-6.37511145987389\\
88.628365	-13.7686563856421\\
61.752306	-9.5933878768386\\
86.100944	-13.3760146925357\\
123.868688	-19.2433359456909\\
130.343312	-20.2491863084075\\
228.002346	-35.4207815657469\\
171.712816	-26.6760507261416\\
160.52637	-24.9382060626352\\
212.15526	-32.9588938636808\\
139.297158	-21.6401905191244\\
78.191244	-12.1472213890203\\
78.674652	-12.2223201302198\\
102.33492	-15.8980068032641\\
128.434128	-19.9525894065808\\
186.648088	-28.9962856553779\\
131.574324	-20.4404273544835\\
116.228981	-18.0564867854927\\
102.33492	-15.8980068032641\\
126.919232	-19.7172462127401\\
163.80765	-25.447961791798\\
258.898841	-40.2206356889241\\
211.234231	-32.8158095156596\\
205.202734	-31.8788001317673\\
131.156837	-20.3755696190568\\
87.170248	-13.5421339631303\\
82.878384	-12.8753812743054\\
41.571024	-6.45817109517119\\
55.61737	-8.64030896432672\\
48.663775	-7.56004916036984\\
79.413192	-12.3370543181635\\
143.124003	-22.2347012476718\\
199.917662	-31.0577498919104\\
205.725716	-31.9600468009784\\
286.477455	-44.5050480185238\\
220.4004	-34.2397986791032\\
126.151408	-19.5979626761358\\
132.760866	-20.6247598656962\\
119.080507	-18.4994790675767\\
136.386411	-21.187998094402\\
187.20783	-29.0832430900841\\
231.334857	-35.9384961694223\\
165.21196	-25.666125151225\\
120.154712	-18.6663597217865\\
137.594926	-21.3757441706372\\
180.235144	-28.0000174476053\\
116.951628	-18.1687519528702\\
85.558968	-13.2918172540151\\
75.975624	-11.8030188251893\\
54.00598	-8.38997516641383\\
39.564248	-6.14641301200049\\
33.529272	-5.20886315604188\\
69.518865	-10.7999438385763\\
98.88274	-15.3617012965408\\
104.610743	-16.2515620660916\\
62.158788	-9.65653595572901\\
65.710281	-10.2082700058044\\
30.849616	-4.79257134364385\\
12.202559	-1.89570056828336\\
31.856776	-4.94903637563855\\
59.741868	-9.28106089205627\\
75.478788	-11.7258340078453\\
98.57512	-15.3139116969318\\
114.803218	-17.8349906444507\\
140.990247	-21.9032164778151\\
186.680484	-29.0013184616614\\
261.114406	-40.5648297083563\\
263.329971	-40.9090237277886\\
279.293325	-43.3889739783473\\
236.966604	-36.8134033088451\\
134.386868	-20.8773636773582\\
126.524944	-19.6559925047541\\
135.990897	-21.1265539239977\\
173.256996	-26.9159432686429\\
132.365352	-20.563315695292\\
100.45502	-15.6059592500979\\
69.958317	-10.8682138962038\\
71.520813	-11.1109518788795\\
115.506334	-17.9442216181153\\
97.32755	-15.1200983207397\\
46.937685	-7.29189640700817\\
33.75512	-5.24394925412554\\
62.763018	-9.7504047216472\\
82.878384	-12.8753812743054\\
56.1545	-8.72375356363102\\
104.610743	-16.2515620660916\\
71.081361	-11.042681821252\\
133.573867	-20.7510617715272\\
177.569117	-27.5858429372457\\
135.595383	-21.0651097535934\\
188.291819	-29.2516437151754\\
264.055065	-41.0216690432391\\
276.94345	-43.0239145368896\\
197.600788	-30.6978172451237\\
147.973254	-22.9880455155789\\
98.88274	-15.3617012965408\\
63.971478	-9.93814225348357\\
105.28825	-16.3568146122924\\
57.756124	-8.9725701870111\\
124.262976	-19.3045896536768\\
86.100944	-13.3760146925357\\
86.906584	-13.5011730470933\\
112.30325	-17.446613849199\\
70.849428	-11.0066504019485\\
119.697536	-18.5953361927869\\
74.740248	-11.6110998199016\\
54.181768	-8.41728430800433\\
60.14835	-9.34420897094669\\
39.827912	-6.18737392803751\\
45.997735	-7.14587263042508\\
32.07652	-4.98317420080103\\
38.758608	-6.0212546574429\\
32.857832	-5.10455313471208\\
52.755932	-8.19577682621474\\
54.894686	-8.52803804889912\\
85.016992	-13.2076198154946\\
120.071072	-18.6533660214052\\
91.24675	-14.1754295823531\\
94.150777	-14.6265780368871\\
156.296595	-24.2810990680114\\
185.518922	-28.8208666610598\\
133.178353	-20.689617601123\\
148.32411	-23.0425519448112\\
53.644638	-8.33383970870003\\
41.168204	-6.3955919178924\\
71.301087	-11.0768168500657\\
123.121616	-19.1272762884543\\
126.919232	-19.7172462127401\\
188.326578	-29.2570436199046\\
177.569117	-27.5858429372457\\
114.803218	-17.8349906444507\\
85.016992	-13.2076198154946\\
106.954551	-16.6156789826782\\
98.88274	-15.3617012965408\\
81.80908	-12.7092620037107\\
40.633552	-6.31253228259509\\
73.5183	-11.4212668907584\\
59.346372	-9.21961951800073\\
38.355788	-5.95867548016411\\
39.425092	-6.12479475075872\\
67.944162	-10.555309465411\\
82.61472	-12.8344203582683\\
88.326854	-13.7218158357178\\
49.13375	-7.63306104866138\\
117.303186	-18.2233674397025\\
151.14396	-23.4806232745604\\
101.70259	-15.79977262629\\
155.373735	-24.1377302691843\\
152.553885	-23.6996589394351\\
111.932161	-17.3889641686538\\
69.518865	-10.7999438385763\\
113.357924	-17.6104603096958\\
96.06289	-14.9236299667915\\
54.357556	-8.44459344959482\\
80.725128	-12.5408671266696\\
38.89044	-6.04173511546142\\
25.572271	-3.97272151415092\\
26.016624	-4.04175295539357\\
46.783875	-7.26800160720366\\
73.276596	-11.3837175201587\\
67.50471	-10.4870394077835\\
75.73392	-11.7654694545895\\
88.041212	-13.6774405778948\\
52.033248	-8.08350591078714\\
37.952968	-5.89609630288532\\
31.856776	-4.94903637563855\\
38.48762	-5.97915593818263\\
52.218802	-8.11233222691044\\
55.08024	-8.55686436502242\\
92.103676	-14.3085553558222\\
105.28825	-16.3568146122924\\
91.24675	-14.1754295823531\\
100.13031	-15.5555146727329\\
132.760866	-20.6247598656962\\
173.256996	-26.9159432686429\\
189.425226	-29.4277215603222\\
186.648088	-28.9962856553779\\
133.969381	-20.8125059419315\\
149.316024	-23.1966484694408\\
149.755476	-23.2649185270683\\
159.116445	-24.7191703977606\\
186.153138	-28.9193938332386\\
240.448068	-37.3542581642289\\
207.608741	-32.2525798313528\\
182.930468	-28.4187433262105\\
150.658794	-23.4052514233027\\
143.541477	-22.2995569635133\\
177.012474	-27.4993669405772\\
102.33492	-15.8980068032641\\
133.573867	-20.7510617715272\\
142.683336	-22.1662424365057\\
163.734935	-25.4366653197365\\
126.524944	-19.6559925047541\\
155.835165	-24.2094146685978\\
109.646268	-17.0338445040708\\
106.277044	-16.5104264364774\\
71.752746	-11.1469832981829\\
126.919232	-19.7172462127401\\
170.674004	-26.5146684702771\\
121.585968	-18.8887092152459\\
78.191244	-12.1472213890203\\
75.237084	-11.6882846372456\\
53.107508	-8.25039510939573\\
40.50172	-6.29205182457658\\
52.93172	-8.22308596780524\\
26.109401	-4.05616611345522\\
39.425092	-6.12479475075872\\
45.057785	-6.99984885384199\\
26.285189	-4.08347525504572\\
52.570378	-8.16695051009144\\
38.223956	-5.9381950221456\\
25.92873	-4.02809838459832\\
39.29326	-6.10431429274021\\
66.613599	-10.3486029020388\\
60.14835	-9.34420897094669\\
73.760004	-11.4588162613582\\
75.478788	-11.7258340078453\\
89.50116	-13.9042474512125\\
75.975624	-11.8030188251893\\
102.64254	-15.9457964028731\\
173.262458	-26.9167918051264\\
120.838896	-18.7726495580094\\
78.929784	-12.261955576964\\
112.30325	-17.446613849199\\
74.001708	-11.4963656319579\\
48.1938	-7.48703727207829\\
89.50116	-13.9042474512125\\
46.46771	-7.21888451871662\\
33.974864	-5.27808707928802\\
68.395821	-10.6254759135282\\
89.786802	-13.9486227090355\\
88.041212	-13.6774405778948\\
52.218802	-8.11233222691044\\
31.856776	-4.94903637563855\\
31.7408	-4.93101919013613\\
32.857832	-5.10455313471208\\
66.381666	-10.3125714827354\\
66.833325	-10.3827379308526\\
68.84748	-10.6956423616454\\
107.284149	-16.6668829240732\\
142.162722	-22.0853636425039\\
93.849266	-14.5797374869628\\
140.572773	-21.8383607619736\\
169.635192	-26.3532862144126\\
154.42524	-23.9903790037232\\
138.803441	-21.5634902468724\\
230.675679	-35.836091082981\\
193.911372	-30.1246564836937\\
208.202003	-32.3447446887848\\
177.86146	-27.6312592135557\\
227.5158	-35.345195327754\\
235.604292	-36.6017644523905\\
152.092455	-23.6279745400216\\
77.452704	-12.0324872010766\\
78.674652	-12.2223201302198\\
109.298359	-16.9797958992649\\
131.574324	-20.4404273544835\\
136.745928	-21.2438500334262\\
82.336408	-12.7911838357848\\
47.723825	-7.41402538378675\\
78.674652	-12.2223201302198\\
134.386868	-20.8773636773582\\
172.723552	-26.8330712994151\\
152.553885	-23.6996589394351\\
91.818034	-14.2641800979992\\
78.674652	-12.2223201302198\\
103.89011	-16.1396097790652\\
167.19923	-25.9748529245005\\
191.103716	-29.6884795248489\\
186.648088	-28.9962856553779\\
144.399618	-22.4328714905209\\
201.06023	-31.2352509232024\\
215.91506	-33.5429889700132\\
198.367	-30.8168503531648\\
277.7125	-43.1433885359122\\
235.363167	-36.5643050310079\\
198.367	-30.8168503531648\\
244.610688	-38.0009324478401\\
225.988191	-35.1078771349541\\
110.305464	-17.1362523868608\\
104.19773	-16.1873993786742\\
128.060592	-19.8945595779625\\
158.655015	-24.6474859983471\\
163.251545	-25.3615693748857\\
112.649272	-17.5003693034474\\
149.316024	-23.1966484694408\\
131.574324	-20.4404273544835\\
88.503216	-13.7492141497619\\
117.654744	-18.2779829265347\\
114.004286	-17.7108743958491\\
168.676255	-26.204312755989\\
188.326578	-29.2570436199046\\
139.714632	-21.705046234966\\
98.57512	-15.3139116969318\\
63.762744	-9.90571486162093\\
91.532392	-14.2198048401761\\
69.286932	-10.7639124192729\\
68.615547	-10.659610942342\\
42.113	-6.54236853369174\\
93.849266	-14.5797374869628\\
116.580539	-18.111102272325\\
138.879684	-21.5753348032829\\
127.948779	-19.8771891257778\\
75.975624	-11.8030188251893\\
54.00598	-8.38997516641383\\
40.230732	-6.2499531053163\\
55.431816	-8.61148264820342\\
83.947688	-13.0415005449\\
98.2675	-15.2661220973228\\
86.906584	-13.5011730470933\\
148.852158	-23.124585630834\\
166.688985	-25.8955849827135\\
190.554392	-29.6031405546402\\
193.911372	-30.1246564836937\\
219.0906	-34.0363177039784\\
139.714632	-21.705046234966\\
115.506334	-17.9442216181153\\
86.100944	-13.3760146925357\\
102.01021	-15.8475622258991\\
123.495152	-19.1853061170726\\
91.818034	-14.2641800979992\\
62.56527	-9.71968403461943\\
64.575708	-10.0320110194018\\
110.305464	-17.1362523868608\\
139.297158	-21.6401905191244\\
128.807664	-20.0106192351991\\
201.663252	-31.328932023051\\
228.661524	-35.5231866521882\\
154.88667	-24.0620634031367\\
106.954551	-16.6156789826782\\
77.452704	-12.0324872010766\\
64.773456	-10.0627317064295\\
122.727328	-19.0660225804684\\
87.433912	-13.5830948791673\\
102.95016	-15.9935860024821\\
140.990247	-21.9032164778151\\
156.78366	-24.356765934059\\
150.194928	-23.3331885846959\\
169.129824	-26.2747759277758\\
109.975866	-17.0850484454658\\
130.365809	-20.2526812782482\\
92.405187	-14.3553959057465\\
87.433912	-13.5830948791673\\
127.666304	-19.8333058699766\\
175.306544	-27.2343460977809\\
184.969598	-28.7355276908511\\
115.670587	-17.969738766235\\
187.764473	-29.1697190867526\\
190.005068	-29.5178015844314\\
130.365809	-20.2526812782482\\
87.975888	-13.6672923176879\\
131.574324	-20.4404273544835\\
101.70259	-15.79977262629\\
102.64254	-15.9457964028731\\
115.506334	-17.9442216181153\\
80.151732	-12.4517885061072\\
115.011391	-17.867330883445\\
218.4102	-33.9306157223974\\
228.661524	-35.5231866521882\\
159.116445	-24.7191703977606\\
175.811912	-27.3128563844177\\
172.723552	-26.8330712994151\\
126.919232	-19.7172462127401\\
123.495152	-19.1853061170726\\
87.975888	-13.6672923176879\\
125.777872	-19.5399328475176\\
130.343312	-20.2491863084075\\
211.234231	-32.8158095156596\\
250.816776	-38.965065016133\\
312.063885	-48.4799691716477\\
243.210534	-37.7834147342618\\
192.782206	-29.9492374893757\\
146.092707	-22.6958974492116\\
158.193585	-24.5758015989336\\
116.951628	-18.1687519528702\\
86.364608	-13.4169756085727\\
79.413192	-12.3370543181635\\
91.532392	-14.2198048401761\\
61.35681	-9.53194650278306\\
41.30736	-6.41721017913416\\
70.849428	-11.0066504019485\\
97.32755	-15.1200983207397\\
67.944162	-10.555309465411\\
40.230732	-6.2499531053163\\
42.376664	-6.58332944972877\\
110.305464	-17.1362523868608\\
148.32411	-23.0425519448112\\
70.19025	-10.9042453155073\\
70.409976	-10.938380344321\\
91.532392	-14.2198048401761\\
85.558968	-13.2918172540151\\
106.624953	-16.5644750412832\\
97.63517	-15.1678879203487\\
61.148076	-9.49951911092041\\
46.3139	-7.19498971891212\\
39.967068	-6.20899218927928\\
55.793158	-8.66761810591722\\
107.284149	-16.6668829240732\\
105.617848	-16.4080185536874\\
88.914007	-13.8130316434651\\
59.346372	-9.21961951800073\\
39.69608	-6.166893470019\\
48.1938	-7.48703727207829\\
99.19036	-15.4094908961498\\
114.803218	-17.8349906444507\\
112.654808	-17.5012293360313\\
90.659597	-14.0842137746058\\
76.955868	-11.9553023837327\\
85.822632	-13.3327781700521\\
126.151408	-19.5979626761358\\
175.306544	-27.2343460977809\\
203.389989	-31.5971852896139\\
145.287714	-22.5708396078552\\
100.76264	-15.653748849707\\
115.154776	-17.889606131283\\
107.961656	-16.7721354702741\\
98.2675	-15.2661220973228\\
70.629702	-10.9725153731348\\
78.191244	-12.1472213890203\\
92.405187	-14.3553959057465\\
82.61472	-12.8344203582683\\
59.741868	-9.28106089205627\\
33.639144	-5.22593206862312\\
48.1938	-7.48703727207829\\
78.929784	-12.261955576964\\
124.636512	-19.3626194822951\\
149.316024	-23.1966484694408\\
171.712816	-26.6760507261416\\
158.193585	-24.5758015989336\\
185.596495	-28.8329178365701\\
240.709833	-37.3949240654761\\
313.825344	-48.7536165948896\\
241.81038	-37.5658970206835\\
178.900272	-27.7926414694202\\
204.55141	-31.7776151854838\\
241.405632	-37.5030183233863\\
283.225525	-43.9998518913002\\
184.969598	-28.7355276908511\\
77.452704	-12.0324872010766\\
55.431816	-8.61148264820342\\
61.148076	-9.49951911092041\\
26.109401	-4.05616611345522\\
25.840836	-4.01444381380307\\
32.973808	-5.1225703202145\\
54.357556	-8.44459344959482\\
75.478788	-11.7258340078453\\
81.267104	-12.6250645651902\\
60.543846	-9.40565034500223\\
47.723825	-7.41402538378675\\
75.975624	-11.8030188251893\\
90.945239	-14.1285890324288\\
97.94279	-15.2156775199577\\
89.199649	-13.8574069012882\\
73.5183	-11.4212668907584\\
46.16009	-7.17109491910761\\
47.091495	-7.31579120681267\\
61.554558	-9.56266718981083\\
81.267104	-12.6250645651902\\
59.137638	-9.18719212613808\\
40.362564	-6.27043356333482\\
59.741868	-9.28106089205627\\
59.950602	-9.31348828391892\\
47.40766	-7.36490829529971\\
76.47246	-11.8802036425332\\
106.277044	-16.5104264364774\\
103.933236	-16.1463095198907\\
77.936112	-12.1075859422762\\
114.803218	-17.8349906444507\\
120.444608	-18.7113958500234\\
97.94279	-15.2156775199577\\
86.100944	-13.3760146925357\\
133.178353	-20.689617601123\\
163.734935	-25.4366653197365\\
180.791787	-28.0864934442737\\
179.180452	-27.8361681907597\\
147.509388	-22.9159826769721\\
132.365352	-20.563315695292\\
126.919232	-19.7172462127401\\
170.674004	-26.5146684702771\\
135.595383	-21.0651097535934\\
166.688985	-25.8955849827135\\
173.762364	-26.9944535552796\\
175.551135	-27.2723439716447\\
310.350033	-48.2137176247096\\
235.604292	-36.6017644523905\\
123.868688	-19.2433359456909\\
86.628272	-13.4579365246097\\
87.433912	-13.5830948791673\\
111.312569	-17.2927088744566\\
151.098246	-23.4735214809302\\
168.16601	-26.1250448142021\\
183.457814	-28.5006679546333\\
194.680892	-30.2442034984884\\
125.383584	-19.4786791395316\\
118.728949	-18.4448635807445\\
126.524944	-19.6559925047541\\
122.727328	-19.0660225804684\\
83.684024	-13.0005396288629\\
71.081361	-11.042681821252\\
100.45502	-15.6059592500979\\
102.33492	-15.8980068032641\\
135.177896	-21.0002520181667\\
171.712816	-26.6760507261416\\
153.50238	-23.8470102048962\\
107.632058	-16.7209315288791\\
87.712224	-13.6263314016508\\
135.177896	-21.0002520181667\\
176.850724	-27.4742386402822\\
219.67486	-34.1270840763455\\
247.373154	-38.4300890178729\\
277.7125	-43.1433885359122\\
228.153	-35.4441860724093\\
204.576513	-31.781515004478\\
123.495152	-19.1853061170726\\
74.658012	-11.5983242347204\\
153.783786	-23.890727388654\\
202.170506	-31.4077353048925\\
121.951564	-18.9455055433737\\
165.70464	-25.7426643227203\\
241.405632	-37.5030183233863\\
291.09487	-45.2223759363401\\
194.680892	-30.2442034984884\\
106.954551	-16.6156789826782\\
71.520813	-11.1109518788795\\
93.277982	-14.4909869713168\\
85.822632	-13.3327781700521\\
76.47246	-11.8802036425332\\
62.763018	-9.7504047216472\\
69.067206	-10.7297773904591\\
55.61737	-8.64030896432672\\
68.84748	-10.6956423616454\\
62.356536	-9.68725664275678\\
78.929784	-12.261955576964\\
122.353792	-19.0079927518501\\
125.383584	-19.4786791395316\\
151.537698	-23.5417915385577\\
190.554392	-29.6031405546402\\
191.189712	-29.701839236207\\
116.228981	-18.0564867854927\\
102.95016	-15.9935860024821\\
129.948322	-20.1878235428215\\
85.295304	-13.2508563379781\\
76.47246	-11.8802036425332\\
63.1695	-9.81355280053762\\
84.489664	-13.1256979834205\\
82.072744	-12.7502229197478\\
45.690115	-7.09808303081607\\
25.3916	-3.94465378529403\\
32.74796	-5.08748422213084\\
47.56147	-7.38880309510422\\
82.878384	-12.8753812743054\\
80.725128	-12.5408671266696\\
54.00598	-8.38997516641383\\
76.47246	-11.8802036425332\\
100.45502	-15.6059592500979\\
120.071072	-18.6533660214052\\
97.63517	-15.1678879203487\\
84.489664	-13.1256979834205\\
100.13031	-15.5555146727329\\
121.585968	-18.8887092152459\\
107.961656	-16.7721354702741\\
122.727328	-19.0660225804684\\
120.444608	-18.7113958500234\\
76.47246	-11.8802036425332\\
69.958317	-10.8682138962038\\
90.373955	-14.0398385167828\\
77.211	-11.9949378304769\\
86.100944	-13.3760146925357\\
127.292768	-19.7752760413583\\
185.518922	-28.8208666610598\\
146.630484	-22.7794425617171\\
129.575488	-20.1299027718033\\
196.997766	-30.6041361452751\\
138.879684	-21.5753348032829\\
91.532392	-14.2198048401761\\
82.61472	-12.8344203582683\\
54.718898	-8.50072890730862\\
53.820426	-8.36114885029053\\
40.0989	-6.22947264729779\\
41.842012	-6.50026981443147\\
93.87537	-14.5837928140163\\
71.799516	-11.1542491442713\\
40.0989	-6.22947264729779\\
75.478788	-11.7258340078453\\
82.072744	-12.7502229197478\\
60.14835	-9.34420897094669\\
51.681672	-8.02888762760615\\
26.109401	-4.05616611345522\\
34.310584	-5.33024208995292\\
99.82269	-15.5077250731239\\
120.838896	-18.7726495580094\\
90.659597	-14.0842137746058\\
70.19025	-10.9042453155073\\
98.2675	-15.2661220973228\\
106.624953	-16.5644750412832\\
122.727328	-19.0660225804684\\
132.760866	-20.6247598656962\\
163.251545	-25.3615693748857\\
161.77452	-25.1321095433972\\
128.739807	-20.0000774665863\\
90.945239	-14.1285890324288\\
69.518865	-10.7999438385763\\
78.191244	-12.1472213890203\\
90.072444	-13.9929979668585\\
74.740248	-11.6110998199016\\
49.28756	-7.65695584846589\\
108.291254	-16.8233394116691\\
137.163402	-21.3087057492677\\
124.262976	-19.3045896536768\\
150.194928	-23.3331885846959\\
181.612618	-28.2140117617975\\
126.919232	-19.7172462127401\\
144.817092	-22.4977272063625\\
188.875902	-29.3423825901134\\
151.537698	-23.5417915385577\\
182.403122	-28.3368186977878\\
189.903154	-29.5019689686894\\
298.54726	-46.3801248935931\\
274.593575	-42.6588550954319\\
197.268352	-30.6461724127473\\
217.265728	-33.7528188995519\\
282.41225	-43.8735074188276\\
276.62416	-42.9743119711944\\
309.169355	-48.0302961017684\\
299.35334	-46.5053516033416\\
363.892025	-56.5316109994864\\
305.783104	-47.5042328436395\\
184.541803	-28.6690685797246\\
121.228917	-18.8332403759962\\
126.919232	-19.7172462127401\\
87.712224	-13.6263314016508\\
88.781528	-13.7924506722454\\
122.303122	-19.0001210302059\\
179.180452	-27.8361681907597\\
124.262976	-19.3045896536768\\
87.433912	-13.5830948791673\\
99.19036	-15.4094908961498\\
70.849428	-11.0066504019485\\
68.395821	-10.6254759135282\\
20.181282	-3.13521678166741\\
52.93172	-8.22308596780524\\
53.644638	-8.33383970870003\\
53.293062	-8.27922142551904\\
58.533408	-9.0933233602199\\
31.7408	-4.93101919013613\\
24.678682	-3.83390004439923\\
18.976484	-2.94804815144266\\
37.68198	-5.85399758362505\\
37.27916	-5.79141840634626\\
32.973808	-5.1225703202145\\
75.73392	-11.7654694545895\\
105.28825	-16.3568146122924\\
96.3876	-14.9740745441566\\
76.714164	-11.917753013133\\
94.722061	-14.7153285525332\\
154.89964	-24.0640783277415\\
100.582323	-15.6257361157082\\
26.016624	-4.04175295539357\\
37.952968	-5.89609630288532\\
25.572271	-3.97272151415092\\
26.646531	-4.13961071275952\\
90.659597	-14.0842137746058\\
101.919026	-15.833396544699\\
69.958317	-10.8682138962038\\
109.975866	-17.0850484454658\\
178.096463	-27.6677675656685\\
153.96381	-23.9186946043097\\
105.28825	-16.3568146122924\\
77.211	-11.9949378304769\\
86.364608	-13.4169756085727\\
140.572773	-21.8383607619736\\
157.85369	-24.5229979907186\\
102.596533	-15.9386490908999\\
61.96104	-9.62581526870124\\
68.395821	-10.6254759135282\\
64.81917	-10.0698335000597\\
24.678682	-3.83390004439923\\
19.111978	-2.9690975110728\\
41.71018	-6.47978935641295\\
121.212432	-18.8306793866276\\
118.929712	-18.4760526561827\\
74.001708	-11.4963656319579\\
59.54412	-9.2503402050285\\
55.31451	-8.5932588436012\\
55.523244	-8.62568623546385\\
24.766576	-3.84755461519448\\
12.6932	-1.97192297560981\\
39.69608	-6.166893470019\\
90.659597	-14.0842137746058\\
128.739807	-20.0000774665863\\
137.163402	-21.3087057492677\\
130.761323	-20.3141254486525\\
138.439017	-21.5068759921168\\
144.384396	-22.4305067116208\\
116.580539	-18.111102272325\\
159.60351	-24.7948372638082\\
222.9846	-34.6412611435386\\
205.90032	-31.9871720049643\\
103.27404	-16.0439016371007\\
50.22751	-7.80297962504897\\
89.045192	-13.8334115882825\\
169.635192	-26.3532862144126\\
127.531292	-19.812331390351\\
90.945239	-14.1285890324288\\
79.668324	-12.3766897649077\\
152.441016	-23.6821244347921\\
216.5303	-33.6385681692312\\
243.210534	-37.7834147342618\\
243.89169	-37.8892341624891\\
248.206302	-38.5595208146841\\
185.069149	-28.7509932081473\\
167.585644	-26.0348833852745\\
105.617848	-16.4080185536874\\
71.752746	-11.1469832981829\\
109.298359	-16.9797958992649\\
117.303186	-18.2233674397025\\
125.383584	-19.4786791395316\\
136.745928	-21.2438500334262\\
94.436419	-14.6709532947102\\
101.07026	-15.701538449316\\
118.556176	-18.4180228275644\\
53.293062	-8.27922142551904\\
32.302368	-5.01826029888469\\
32.74796	-5.08748422213084\\
45.22014	-7.02507114252453\\
31.295208	-4.86179526688999\\
6.394377	-0.993383774068867\\
27.54012	-4.27843218251121\\
95.12294	-14.7776061902084\\
73.5183	-11.4212668907584\\
61.148076	-9.49951911092041\\
93.277982	-14.4909869713168\\
142.602174	-22.1536337001314\\
115.132096	-17.886082731897\\
38.355788	-5.95867548016411\\
33.4194	-5.19179424346064\\
50.38132	-7.82687442485348\\
165.722205	-25.745393093012\\
203.408842	-31.6001141541917\\
275.877088	-42.8582523139579\\
336.32928	-52.249663961988\\
312.063885	-48.4799691716477\\
221.55476	-34.4191316295117\\
225.9298	-35.0988059350621\\
289.68544	-45.0034171710552\\
239.391477	-37.1901138925935\\
217.265728	-33.7528188995519\\
250.816776	-38.965065016133\\
276.62416	-42.9743119711944\\
284.966464	-44.2703114770024\\
379.419064	-58.9437785338583\\
267.49328	-41.5558050494145\\
132.365352	-20.563315695292\\
77.452704	-12.0324872010766\\
55.968946	-8.69492724750771\\
69.958317	-10.8682138962038\\
69.738591	-10.8340788673901\\
65.377686	-10.1566004723477\\
122.727328	-19.0660225804684\\
107.632058	-16.7209315288791\\
117.303186	-18.2233674397025\\
121.212432	-18.8306793866276\\
85.016992	-13.2076198154946\\
79.668324	-12.3766897649077\\
132.365352	-20.563315695292\\
135.887787	-21.1105355064186\\
75.73392	-11.7654694545895\\
41.439192	-6.43769063715268\\
54.894686	-8.52803804889912\\
72.643857	-11.2854198039276\\
184.541803	-28.6690685797246\\
224.259	-34.8392426328492\\
211.827493	-32.9079743730916\\
231.4098	-35.9501387673141\\
297.727375	-46.2527535397298\\
357.56188	-55.5482058404688\\
304.901751	-47.367312269606\\
192.782206	-29.9492374893757\\
116.580539	-18.111102272325\\
78.191244	-12.1472213890203\\
78.432948	-12.1847707596201\\
86.906584	-13.5011730470933\\
100.45502	-15.6059592500979\\
56.506076	-8.77837184681201\\
71.081361	-11.042681821252\\
78.674652	-12.2223201302198\\
93.849266	-14.5797374869628\\
117.654744	-18.2779829265347\\
123.495152	-19.1853061170726\\
103.27404	-16.0439016371007\\
46.3139	-7.19498971891212\\
25.748059	-4.00003065574142\\
25.3916	-3.94465378529403\\
19.177894	-2.97933774008206\\
33.193552	-5.15670814537698\\
60.75258	-9.43807773686487\\
82.072744	-12.7502229197478\\
78.674652	-12.2223201302198\\
135.052839	-20.9808240747355\\
109.975866	-17.0850484454658\\
147.785796	-22.9589234079022\\
255.958182	-39.7637963540414\\
192.232882	-29.863898519167\\
162.88479	-25.304592992971\\
212.420755	-33.0001392305237\\
231.334857	-35.9384961694223\\
166.205595	-25.8204890378627\\
131.574324	-20.4404273544835\\
93.563624	-14.5353622291398\\
98.800394	-15.3489086225618\\
202.853146	-31.5137852765364\\
336.18078	-52.2265940850556\\
391.85759	-60.8761372142635\\
313.436959	-48.6932799339308\\
256.057965	-39.7792979131656\\
199.496166	-30.9922693474828\\
203.40247	-31.5991242467452\\
243.89169	-37.8892341624891\\
175.811912	-27.3128563844177\\
119.803154	-18.6117442349542\\
123.728885	-19.2216171712479\\
164.756145	-25.5953130572591\\
179.707798	-27.9180928191825\\
95.594856	-14.8509196181035\\
86.100944	-13.3760146925357\\
87.712224	-13.6263314016508\\
152.001564	-23.6138543771646\\
177.356092	-27.552748926919\\
182.930468	-28.4187433262105\\
126.151408	-19.5979626761358\\
103.89011	-16.1396097790652\\
102.95016	-15.9935860024821\\
92.976471	-14.4441464213925\\
62.56527	-9.71968403461943\\
54.357556	-8.44459344959482\\
46.630065	-7.24410680739916\\
33.309528	-5.1747253308794\\
41.30736	-6.41721017913416\\
71.520813	-11.1109518788795\\
107.961656	-16.7721354702741\\
105.28825	-16.3568146122924\\
75.73392	-11.7654694545895\\
54.357556	-8.44459344959482\\
47.877635	-7.43792018359125\\
62.960766	-9.78112540867497\\
92.103676	-14.3085553558222\\
106.277044	-16.5104264364774\\
89.199649	-13.8574069012882\\
63.564996	-9.87499417459316\\
148.412706	-23.0563155732065\\
161.29113	-25.0570135985464\\
116.646992	-18.1214259257377\\
47.091495	-7.31579120681267\\
76.47246	-11.8802036425332\\
98.2675	-15.2661220973228\\
97.94279	-15.2156775199577\\
67.944162	-10.555309465411\\
48.663775	-7.56004916036984\\
72.424131	-11.2512847751138\\
155.38303	-24.1391742725923\\
72.538056	-11.268983332215\\
39.161428	-6.0838338347217\\
55.793158	-8.66761810591722\\
108.968761	-16.9285919578699\\
126.740264	-19.6894430495425\\
69.067206	-10.7297773904591\\
64.575708	-10.0320110194018\\
126.919232	-19.7172462127401\\
179.707798	-27.9180928191825\\
160.52637	-24.9382060626352\\
245.291844	-38.1067518760673\\
255.958182	-39.7637963540414\\
188.875902	-29.3423825901134\\
146.927655	-22.8256088808947\\
190.005068	-29.5178015844314\\
161.013435	-25.0138729286829\\
172.597085	-26.8134243086676\\
232.689834	-36.1489955137738\\
167.585644	-26.0348833852745\\
88.503216	-13.7492141497619\\
163.80765	-25.447961791798\\
241.405632	-37.5030183233863\\
269.776	-41.9104317798595\\
203.408842	-31.6001141541917\\
181.988632	-28.2724265544228\\
220.2901	-34.2226632755635\\
189.425226	-29.4277215603222\\
104.19773	-16.1873993786742\\
105.13768	-16.3334231552573\\
135.990897	-21.1265539239977\\
143.541477	-22.2995569635133\\
129.201952	-20.071872943185\\
150.194928	-23.3331885846959\\
97.054804	-15.0777264914212\\
121.600006	-18.8908900565414\\
170.674004	-26.5146684702771\\
115.506334	-17.9442216181153\\
77.936112	-12.1075859422762\\
61.96104	-9.62581526870124\\
48.031445	-7.46181498339576\\
49.911345	-7.75386253656194\\
100.45502	-15.6059592500979\\
109.646268	-17.0338445040708\\
120.525801	-18.7240094023317\\
169.159645	-26.2794087008398\\
198.172072	-30.7865677607697\\
172.723552	-26.8330712994151\\
128.344293	-19.938633296182\\
77.936112	-12.1075859422762\\
97.054804	-15.0777264914212\\
177.569117	-27.5858429372457\\
119.697536	-18.5953361927869\\
43.182304	-6.70848780428635\\
104.83006	-16.2856335556483\\
159.60351	-24.7948372638082\\
179.40564	-27.871151756057\\
234.6312	-36.4505919764048\\
304.066785	-47.2375980415742\\
236.022345	-36.6667101174492\\
189.903154	-29.5019689686894\\
216.5303	-33.6385681692312\\
158.655015	-24.6474859983471\\
113.326779	-17.6056218496483\\
137.594926	-21.3757441706372\\
185.069149	-28.7509932081473\\
177.86146	-27.6312592135557\\
172.218184	-26.7545610127784\\
109.646268	-17.0338445040708\\
86.100944	-13.3760146925357\\
72.643857	-11.2854198039276\\
101.70259	-15.79977262629\\
107.632058	-16.7209315288791\\
81.870516	-12.7188062525944\\
151.537698	-23.5417915385577\\
186.648088	-28.9962856553779\\
109.646268	-17.0338445040708\\
81.628812	-12.6812568819946\\
137.604069	-21.3771645604338\\
82.878384	-12.8753812743054\\
34.200712	-5.31317317737168\\
69.958317	-10.8682138962038\\
76.714164	-11.917753013133\\
61.35681	-9.53194650278306\\
39.022272	-6.06221557347993\\
32.857832	-5.10455313471208\\
40.90454	-6.35463100185537\\
63.971478	-9.93814225348357\\
124.262976	-19.3045896536768\\
157.70652	-24.500134732886\\
184.541803	-28.6690685797246\\
200.357761	-31.1261204627391\\
107.632058	-16.7209315288791\\
75.097464	-11.6665942923479\\
181.483264	-28.193916267786\\
267.49328	-41.5558050494145\\
202.80582	-31.5064330543431\\
192.569181	-29.916143479049\\
297.734272	-46.2538250073473\\
221.55476	-34.4191316295117\\
193.911372	-30.1246564836937\\
161.013435	-25.0138729286829\\
148.643937	-23.0922379349098\\
199.346378	-30.9689993762643\\
150.194928	-23.3331885846959\\
131.46392	-20.4232757942623\\
198.367	-30.8168503531648\\
264.780159	-41.1343143586896\\
254.467711	-39.5322476500597\\
93.277982	-14.4909869713168\\
60.265986	-9.36248404194209\\
173.256996	-26.9159432686429\\
120.444608	-18.7113958500234\\
47.40766	-7.36490829529971\\
48.1938	-7.48703727207829\\
85.558968	-13.2918172540151\\
119.451596	-18.5571287481219\\
170.674004	-26.5146684702771\\
107.961656	-16.7721354702741\\
76.47246	-11.8802036425332\\
59.741868	-9.28106089205627\\
33.08368	-5.13963923279574\\
46.16009	-7.17109491910761\\
33.193552	-5.15670814537698\\
41.439192	-6.43769063715268\\
97.94279	-15.2156775199577\\
80.1978	-12.4589452945956\\
32.638088	-5.0704153095496\\
32.973808	-5.1225703202145\\
47.091495	-7.31579120681267\\
58.742142	-9.12575075208255\\
46.16009	-7.17109491910761\\
52.033248	-8.08350591078714\\
50.968754	-7.91813388671135\\
13.274158	-2.06217637333965\\
69.958317	-10.8682138962038\\
110.305464	-17.1362523868608\\
154.223238	-23.9589974462815\\
217.79496	-33.8350365231793\\
184.389756	-28.6454476667418\\
85.016992	-13.2076198154946\\
69.958317	-10.8682138962038\\
67.944162	-10.555309465411\\
19.712546	-3.06239737537936\\
27.271555	-4.23670988285906\\
76.47246	-11.8802036425332\\
103.933236	-16.1463095198907\\
83.684024	-13.0005396288629\\
84.753328	-13.1666588994575\\
106.954551	-16.6156789826782\\
114.803218	-17.8349906444507\\
117.303186	-18.2233674397025\\
155.373735	-24.1377302691843\\
129.949024	-20.1879326004216\\
205.90032	-31.9871720049643\\
102.926131	-15.9898530322949\\
67.053051	-10.4168729596663\\
48.50142	-7.5348268716873\\
90.373955	-14.0398385167828\\
75.478788	-11.7258340078453\\
52.39459	-8.13964136850094\\
19.511136	-3.03110778673997\\
57.929178	-8.99945459430171\\
43.18643	-6.70912878955385\\
12.336814	-1.91655744591\\
38.758608	-6.0212546574429\\
53.46885	-8.30653056710953\\
83.42036	-12.9595787128259\\
116.951628	-18.1687519528702\\
159.814105	-24.8275537670579\\
122.301718	-18.9999029150058\\
41.168204	-6.3955919178924\\
77.452704	-12.0324872010766\\
120.697689	-18.7507126683664\\
60.14835	-9.34420897094669\\
49.44137	-7.68085064827039\\
128.739807	-20.0000774665863\\
111.990076	-17.3979614206574\\
177.86146	-27.6312592135557\\
227.306547	-35.3126873078366\\
140.132106	-21.7699019508075\\
109.298359	-16.9797958992649\\
};
\end{axis}

\begin{axis}[%
width=4.927cm,
height=3cm,
at={(7cm,14.516cm)},
scale only axis,
xmin=0,
xmax=400,
xlabel style={font=\color{white!15!black}},
xlabel={y(t-1)u(t)},
ymin=-51.7743180767417,
ymax=0,
ylabel style={font=\color{white!15!black}},
ylabel={y(t)},
axis background/.style={fill=white},
title style={font=\small},
title={C2, R = -0.7896},
axis x line*=bottom,
axis y line*=left
]
\addplot[only marks, mark=*, mark options={}, mark size=1.5000pt, color=mycolor1, fill=mycolor1] table[row sep=crcr]{%
x	y\\
112.319674	-19.531\\
121.951564	-24.414\\
150.658794	-19.531\\
119.803154	-20.752\\
129.575488	-25.635\\
160.52637	-24.414\\
154.223238	-28.076\\
175.811912	-23.193\\
138.439017	-13.428\\
79.668324	-17.09\\
101.70259	-17.09\\
102.01021	-18.311\\
108.968761	-15.869\\
88.914007	-8.545\\
45.057785	-6.104\\
31.40508	-7.324\\
40.362564	-9.766\\
57.756124	-21.973\\
132.365352	-23.193\\
140.572773	-23.193\\
138.021543	-19.531\\
111.932161	-10.986\\
65.784168	-17.09\\
101.39497	-19.531\\
112.30325	-13.428\\
79.668324	-18.311\\
108.968761	-19.531\\
114.803218	-14.648\\
88.503216	-24.414\\
149.316024	-21.973\\
131.574324	-15.869\\
97.054804	-23.193\\
142.683336	-25.635\\
155.373735	-19.531\\
116.580539	-17.09\\
100.13031	-14.648\\
86.100944	-17.09\\
102.01021	-20.752\\
123.868688	-18.311\\
109.646268	-18.311\\
109.646268	-19.531\\
117.654744	-20.752\\
128.434128	-28.076\\
174.267732	-25.635\\
154.88667	-17.09\\
102.01021	-15.869\\
91.24675	-12.207\\
68.84748	-10.986\\
63.1695	-15.869\\
90.373955	-12.207\\
69.067206	-12.207\\
70.629702	-15.869\\
93.277982	-18.311\\
107.632058	-15.869\\
96.182009	-25.635\\
154.88667	-21.973\\
127.531292	-12.207\\
69.067206	-9.766\\
55.793158	-13.428\\
77.452704	-15.869\\
88.628365	-10.986\\
60.346098	-10.986\\
61.554558	-12.207\\
68.176095	-9.766\\
53.46885	-8.545\\
47.40766	-12.207\\
68.615547	-14.648\\
85.295304	-17.09\\
100.76264	-18.311\\
108.968761	-23.193\\
138.439017	-21.973\\
133.178353	-21.973\\
132.365352	-17.09\\
101.70259	-17.09\\
101.07026	-17.09\\
100.76264	-15.869\\
93.277982	-17.09\\
104.19773	-24.414\\
156.00546	-35.4\\
230.1	-36.621\\
239.391477	-34.18\\
216.5303	-24.414\\
156.908778	-29.297\\
193.125824	-36.621\\
243.419787	-40.283\\
264.055065	-31.738\\
210.962486	-37.842\\
259.823172	-48.828\\
330.8097	-39.063\\
260.354895	-30.518\\
196.68851	-25.635\\
160.06494	-19.531\\
119.451596	-14.648\\
89.850832	-19.531\\
118.377391	-15.869\\
92.103676	-9.766\\
55.431816	-9.766\\
55.793158	-12.207\\
71.752746	-17.09\\
100.76264	-15.869\\
93.563624	-15.869\\
95.309214	-19.531\\
118.728949	-19.531\\
121.600006	-28.076\\
174.801176	-23.193\\
145.675233	-25.635\\
159.60351	-23.193\\
141.848388	-18.311\\
112.649272	-20.752\\
125.777872	-17.09\\
102.01021	-14.648\\
88.239552	-18.311\\
110.305464	-17.09\\
101.07026	-15.869\\
93.277982	-13.428\\
77.452704	-10.986\\
63.1695	-12.207\\
70.19025	-13.428\\
75.73392	-10.986\\
62.158788	-9.766\\
56.1545	-14.648\\
87.170248	-19.531\\
118.006302	-23.193\\
141.407721	-23.193\\
137.604069	-17.09\\
102.64254	-18.311\\
114.333884	-30.518\\
190.005068	-23.193\\
144.817092	-24.414\\
153.783786	-26.855\\
162.768155	-17.09\\
100.13031	-10.986\\
64.773456	-17.09\\
102.01021	-18.311\\
114.004286	-26.855\\
172.113695	-34.18\\
220.93952	-31.738\\
202.234536	-25.635\\
162.397725	-24.414\\
153.344334	-23.193\\
145.234566	-23.193\\
145.234566	-21.973\\
134.386868	-18.311\\
109.646268	-14.648\\
86.628272	-13.428\\
78.432948	-12.207\\
71.520813	-13.428\\
80.406864	-19.531\\
121.600006	-25.635\\
159.116445	-21.973\\
132.760866	-14.648\\
87.170248	-13.428\\
79.171488	-14.648\\
83.42036	-12.207\\
66.833325	-8.545\\
46.46771	-8.545\\
48.1938	-13.428\\
75.478788	-12.207\\
68.615547	-12.207\\
68.84748	-12.207\\
68.84748	-12.207\\
67.50471	-8.545\\
45.997735	-7.324\\
38.223956	-4.883\\
25.92873	-7.324\\
41.71018	-17.09\\
102.64254	-24.414\\
149.755476	-24.414\\
147.069936	-19.531\\
113.729013	-12.207\\
68.84748	-12.207\\
66.381666	-8.545\\
48.34761	-10.986\\
62.763018	-14.648\\
85.558968	-14.648\\
88.239552	-24.414\\
155.566008	-34.18\\
225.9298	-40.283\\
265.545536	-32.959\\
216.639507	-31.738\\
201.663252	-23.193\\
147.368322	-25.635\\
163.34622	-26.855\\
172.113695	-28.076\\
177.86146	-21.973\\
137.199412	-20.752\\
129.201952	-19.531\\
120.525801	-19.531\\
121.228917	-21.973\\
134.386868	-19.531\\
122.674211	-23.193\\
149.478885	-29.297\\
185.596495	-24.414\\
148.852158	-14.648\\
88.239552	-15.869\\
99.086036	-25.635\\
164.294715	-32.959\\
216.046245	-35.4\\
235.3038	-39.063\\
261.058029	-37.842\\
248.735466	-30.518\\
198.367	-26.855\\
175.551135	-30.518\\
202.273304	-35.4\\
228.8256	-25.635\\
158.193585	-14.648\\
91.198448	-19.531\\
120.877359	-20.752\\
124.262976	-12.207\\
73.095516	-13.428\\
81.387108	-18.311\\
109.646268	-14.648\\
86.100944	-12.207\\
71.972472	-18.311\\
107.632058	-10.986\\
65.179938	-14.648\\
89.587168	-24.414\\
149.755476	-21.973\\
131.969838	-14.648\\
87.170248	-14.648\\
89.308856	-20.752\\
133.74664	-34.18\\
218.4102	-28.076\\
172.723552	-15.869\\
94.722061	-14.648\\
87.712224	-14.648\\
87.170248	-12.207\\
71.301087	-10.986\\
63.762744	-10.986\\
64.366974	-13.428\\
79.910028	-17.09\\
103.25778	-19.531\\
115.506334	-14.648\\
88.781528	-19.531\\
121.228917	-26.855\\
166.205595	-23.193\\
142.265862	-17.09\\
104.52244	-21.973\\
137.594926	-25.635\\
157.24509	-21.973\\
126.34475	-8.545\\
48.663775	-12.207\\
70.409976	-13.428\\
79.171488	-15.869\\
93.563624	-15.869\\
91.818034	-10.986\\
64.971204	-17.09\\
101.70259	-15.869\\
92.690829	-10.986\\
64.366974	-14.648\\
85.822632	-14.648\\
84.753328	-12.207\\
71.081361	-14.648\\
83.142048	-10.986\\
61.96104	-9.766\\
54.181768	-12.207\\
67.50471	-8.545\\
48.817585	-17.09\\
99.82269	-18.311\\
109.975866	-20.752\\
129.575488	-29.297\\
180.235144	-20.752\\
121.980256	-10.986\\
62.763018	-10.986\\
61.96104	-8.545\\
49.13375	-13.428\\
75.478788	-12.207\\
68.176095	-9.766\\
55.431816	-15.869\\
94.722061	-19.531\\
117.654744	-20.752\\
127.666304	-23.193\\
140.572773	-23.193\\
138.439017	-17.09\\
97.63517	-15.869\\
88.326854	-10.986\\
59.54412	-10.986\\
58.742142	-7.324\\
39.564248	-8.545\\
47.723825	-10.986\\
62.960766	-17.09\\
98.57512	-18.311\\
106.624953	-17.09\\
104.19773	-24.414\\
150.194928	-25.635\\
154.42524	-15.869\\
96.182009	-19.531\\
119.080507	-23.193\\
144.817092	-26.855\\
166.205595	-25.635\\
153.96381	-17.09\\
97.00284	-10.986\\
59.54412	-7.324\\
39.29326	-6.104\\
32.857832	-8.545\\
46.16009	-9.766\\
52.570378	-8.545\\
46.630065	-10.986\\
62.356536	-17.09\\
97.63517	-15.869\\
90.945239	-13.428\\
77.694408	-15.869\\
89.50116	-12.207\\
65.490555	-6.104\\
33.639144	-9.766\\
56.330288	-20.752\\
120.838896	-17.09\\
100.45502	-18.311\\
107.284149	-15.869\\
91.24675	-10.986\\
64.169226	-17.09\\
102.64254	-23.193\\
140.990247	-23.193\\
137.604069	-17.09\\
104.52244	-26.855\\
164.72857	-25.635\\
155.373735	-18.311\\
112.319674	-23.193\\
142.683336	-25.635\\
155.373735	-19.531\\
116.580539	-15.869\\
97.927599	-24.414\\
156.469326	-35.4\\
225.5688	-29.297\\
179.180452	-17.09\\
102.95016	-17.09\\
102.64254	-18.311\\
111.312569	-20.752\\
128.060592	-23.193\\
140.990247	-19.531\\
118.377391	-17.09\\
105.46239	-24.414\\
152.441016	-25.635\\
156.78366	-18.311\\
109.298359	-15.869\\
95.594856	-18.311\\
115.340989	-28.076\\
177.356092	-26.855\\
169.159645	-24.414\\
149.755476	-19.531\\
116.951628	-17.09\\
101.70259	-18.311\\
105.947446	-13.428\\
74.25684	-6.104\\
32.638088	-6.104\\
31.966648	-6.104\\
33.309528	-9.766\\
55.431816	-19.531\\
114.080571	-15.869\\
91.532392	-14.648\\
84.489664	-13.428\\
78.432948	-17.09\\
97.00284	-13.428\\
75.478788	-9.766\\
54.894686	-12.207\\
67.272777	-9.766\\
54.357556	-8.545\\
49.44137	-17.09\\
101.70259	-20.752\\
120.838896	-17.09\\
96.69522	-10.986\\
61.35681	-10.986\\
61.35681	-13.428\\
73.5183	-9.766\\
55.793158	-10.986\\
66.981642	-28.076\\
170.674004	-20.752\\
128.434128	-25.635\\
162.397725	-35.4\\
226.206	-30.518\\
192.232882	-24.414\\
149.755476	-18.311\\
111.312569	-18.311\\
111.312569	-19.531\\
119.803154	-21.973\\
136.386411	-25.635\\
159.60351	-25.635\\
163.34622	-31.738\\
206.900022	-34.18\\
227.19446	-41.504\\
271.311648	-32.959\\
214.2335	-28.076\\
184.543548	-35.4\\
230.1	-26.855\\
175.067745	-29.297\\
189.903154	-28.076\\
174.267732	-17.09\\
101.07026	-10.986\\
63.971478	-12.207\\
72.424131	-17.09\\
100.76264	-14.648\\
83.947688	-9.766\\
56.506076	-13.428\\
76.714164	-13.428\\
74.740248	-7.324\\
40.90454	-8.545\\
48.031445	-12.207\\
67.724436	-7.324\\
41.71018	-14.648\\
86.364608	-20.752\\
120.838896	-13.428\\
81.387108	-21.973\\
140.011956	-30.518\\
195.01002	-29.297\\
190.4305	-35.4\\
227.5158	-29.297\\
187.20783	-28.076\\
175.811912	-23.193\\
138.879684	-12.207\\
72.863583	-14.648\\
86.364608	-15.869\\
91.532392	-10.986\\
63.762744	-12.207\\
68.615547	-13.428\\
72.538056	-3.662\\
19.580714	-8.545\\
48.1938	-17.09\\
99.82269	-19.531\\
116.580539	-20.752\\
125.010048	-23.193\\
139.714632	-20.752\\
126.151408	-23.193\\
140.132106	-18.311\\
108.639163	-15.869\\
94.150777	-15.869\\
92.976471	-12.207\\
73.095516	-18.311\\
107.961656	-17.09\\
98.88274	-12.207\\
71.752746	-17.09\\
103.58249	-23.193\\
139.297158	-17.09\\
100.45502	-13.428\\
77.936112	-12.207\\
71.081361	-14.648\\
83.684024	-7.324\\
41.168204	-8.545\\
48.971395	-12.207\\
71.081361	-17.09\\
100.45502	-17.09\\
99.51507	-14.648\\
86.364608	-15.869\\
93.849266	-18.311\\
104.610743	-12.207\\
69.067206	-9.766\\
57.756124	-18.311\\
112.649272	-26.855\\
166.205595	-26.855\\
164.72857	-21.973\\
133.969381	-19.531\\
117.654744	-17.09\\
101.70259	-14.648\\
86.100944	-13.428\\
79.668324	-17.09\\
101.39497	-17.09\\
98.88274	-10.986\\
64.366974	-14.648\\
86.906584	-18.311\\
109.975866	-19.531\\
118.728949	-21.973\\
133.969381	-21.973\\
137.199412	-29.297\\
188.819165	-35.4\\
233.3568	-36.621\\
234.00819	-26.855\\
164.24518	-14.648\\
85.558968	-10.986\\
61.148076	-7.324\\
41.036372	-10.986\\
61.554558	-10.986\\
63.971478	-14.648\\
85.822632	-13.428\\
77.936112	-12.207\\
71.972472	-15.869\\
94.722061	-20.752\\
126.151408	-20.752\\
127.666304	-24.414\\
155.566008	-31.738\\
203.980126	-29.297\\
184.541803	-26.855\\
164.72857	-18.311\\
110.982971	-17.09\\
103.58249	-20.752\\
127.292768	-20.752\\
125.777872	-17.09\\
102.33492	-14.648\\
87.433912	-18.311\\
105.617848	-12.207\\
71.752746	-13.428\\
81.628812	-20.752\\
124.636512	-17.09\\
102.01021	-18.311\\
114.663482	-32.959\\
205.795996	-25.635\\
158.193585	-20.752\\
131.090384	-28.076\\
178.394904	-28.076\\
172.723552	-19.531\\
115.506334	-13.428\\
78.432948	-13.428\\
80.890272	-19.531\\
123.377327	-30.518\\
196.139186	-34.18\\
220.93952	-31.738\\
202.80582	-25.635\\
158.193585	-17.09\\
102.95016	-14.648\\
86.100944	-13.428\\
77.452704	-10.986\\
62.763018	-8.545\\
49.28756	-12.207\\
71.081361	-14.648\\
83.42036	-10.986\\
62.960766	-13.428\\
79.413192	-19.531\\
116.951628	-20.752\\
127.666304	-21.973\\
139.198955	-29.297\\
190.4305	-34.18\\
219.05962	-25.635\\
161.474865	-23.193\\
146.510181	-24.414\\
152.880468	-20.752\\
126.919232	-14.648\\
89.308856	-18.311\\
116.000185	-29.297\\
189.375808	-31.738\\
199.346378	-19.531\\
116.228981	-12.207\\
69.738591	-8.545\\
45.843925	-7.324\\
38.355788	-9.766\\
52.93172	-8.545\\
47.091495	-12.207\\
69.286932	-14.648\\
82.878384	-13.428\\
74.498544	-9.766\\
54.00598	-10.986\\
60.346098	-10.986\\
59.950602	-9.766\\
53.46885	-10.986\\
61.96104	-14.648\\
86.628272	-21.973\\
132.365352	-18.311\\
110.635062	-20.752\\
126.919232	-21.973\\
138.012413	-30.518\\
193.911372	-31.738\\
205.154432	-34.18\\
222.17	-34.18\\
217.17972	-24.414\\
151.098246	-17.09\\
103.89011	-15.869\\
99.958831	-26.855\\
172.597085	-31.738\\
199.346378	-23.193\\
140.990247	-18.311\\
104.940341	-9.766\\
53.293062	-4.883\\
26.285189	-6.104\\
31.630928	-6.104\\
32.522112	-7.324\\
40.362564	-14.648\\
81.530768	-10.986\\
60.950328	-12.207\\
65.710281	-8.545\\
44.750165	-4.883\\
25.840836	-8.545\\
45.52776	-8.545\\
45.843925	-8.545\\
45.22014	-9.766\\
51.144542	-7.324\\
39.29326	-10.986\\
64.169226	-24.414\\
149.755476	-26.855\\
167.19923	-25.635\\
156.78366	-21.973\\
136.803898	-30.518\\
198.367	-40.283\\
264.055065	-34.18\\
224.66514	-31.738\\
205.725716	-26.855\\
172.113695	-28.076\\
180.94982	-29.297\\
184.014457	-20.752\\
128.060592	-17.09\\
102.01021	-14.648\\
85.295304	-12.207\\
73.534968	-20.752\\
123.868688	-18.311\\
108.639163	-15.869\\
95.023572	-18.311\\
109.298359	-18.311\\
109.298359	-14.648\\
87.712224	-17.09\\
103.25778	-17.09\\
104.19773	-20.752\\
125.383584	-19.531\\
115.877423	-13.428\\
80.648568	-17.09\\
98.57512	-9.766\\
52.218802	-4.883\\
25.840836	-10.986\\
60.346098	-13.428\\
71.30268	-8.545\\
42.5541	-2.441\\
12.290435	-8.545\\
45.37395	-12.207\\
66.613599	-12.207\\
68.176095	-15.869\\
88.628365	-13.428\\
77.694408	-18.311\\
107.284149	-17.09\\
97.32755	-10.986\\
63.762744	-17.09\\
96.06289	-14.648\\
79.934136	-9.766\\
51.681672	-7.324\\
37.147328	-6.104\\
31.521056	-7.324\\
37.550148	-8.545\\
45.37395	-6.104\\
34.09084	-13.428\\
75.975624	-17.09\\
94.81532	-10.986\\
60.950328	-14.648\\
80.1978	-10.986\\
60.950328	-14.648\\
80.461464	-10.986\\
59.54412	-9.766\\
52.755932	-9.766\\
51.681672	-7.324\\
39.022272	-8.545\\
45.37395	-10.986\\
60.346098	-14.648\\
81.00344	-12.207\\
66.613599	-8.545\\
46.46771	-8.545\\
45.690115	-8.545\\
45.690115	-10.986\\
62.356536	-21.973\\
131.156837	-24.414\\
143.041626	-18.311\\
105.947446	-17.09\\
96.69522	-13.428\\
77.936112	-19.531\\
114.803218	-17.09\\
97.32755	-9.766\\
55.793158	-14.648\\
82.61472	-13.428\\
76.955868	-15.869\\
88.914007	-12.207\\
66.833325	-14.648\\
81.267104	-12.207\\
69.067206	-15.869\\
88.914007	-12.207\\
68.176095	-13.428\\
75.237084	-14.648\\
84.489664	-19.531\\
113.357924	-18.311\\
109.298359	-21.973\\
136.803898	-30.518\\
189.425226	-24.414\\
146.630484	-14.648\\
85.558968	-14.648\\
87.433912	-15.869\\
97.340446	-23.193\\
140.132106	-18.311\\
111.312569	-23.193\\
138.021543	-17.09\\
95.75527	-8.545\\
46.630065	-9.766\\
56.1545	-19.531\\
113.729013	-14.648\\
84.489664	-13.428\\
79.910028	-20.752\\
123.868688	-21.973\\
130.365809	-18.311\\
106.277044	-14.648\\
85.558968	-18.311\\
109.298359	-20.752\\
122.727328	-18.311\\
108.291254	-15.869\\
93.563624	-15.869\\
91.818034	-12.207\\
70.409976	-14.648\\
82.878384	-12.207\\
68.84748	-13.428\\
78.674652	-19.531\\
116.580539	-20.752\\
125.010048	-19.531\\
117.303186	-19.531\\
114.803218	-14.648\\
83.684024	-10.986\\
62.763018	-13.428\\
77.452704	-14.648\\
86.100944	-18.311\\
109.646268	-20.752\\
126.919232	-24.414\\
147.973254	-18.311\\
106.954551	-12.207\\
73.095516	-21.973\\
136.803898	-30.518\\
187.197412	-20.752\\
126.524944	-20.752\\
127.666304	-25.635\\
156.296595	-18.311\\
109.298359	-17.09\\
102.64254	-18.311\\
109.298359	-15.869\\
95.023572	-18.311\\
111.642167	-21.973\\
132.365352	-17.09\\
103.25778	-20.752\\
125.383584	-20.752\\
125.383584	-18.311\\
109.646268	-14.648\\
88.239552	-21.973\\
130.365809	-14.648\\
86.364608	-17.09\\
101.70259	-17.09\\
101.39497	-17.09\\
97.32755	-10.986\\
59.137638	-6.104\\
31.630928	-4.883\\
25.484377	-10.986\\
62.158788	-20.752\\
123.495152	-21.973\\
131.969838	-19.531\\
114.803218	-13.428\\
78.929784	-17.09\\
100.45502	-14.648\\
85.016992	-14.648\\
82.61472	-9.766\\
55.08024	-15.869\\
89.786802	-14.648\\
82.878384	-12.207\\
69.738591	-13.428\\
75.975624	-12.207\\
68.615547	-9.766\\
53.644638	-7.324\\
40.362564	-8.545\\
49.44137	-18.311\\
105.28825	-13.428\\
77.694408	-18.311\\
105.947446	-15.869\\
93.277982	-21.973\\
129.948322	-19.531\\
118.006302	-25.635\\
153.96381	-15.869\\
95.309214	-23.193\\
139.714632	-21.973\\
127.531292	-13.428\\
79.668324	-17.09\\
100.45502	-14.648\\
86.100944	-19.531\\
115.877423	-20.752\\
125.010048	-24.414\\
145.287714	-17.09\\
102.64254	-24.414\\
150.658794	-25.635\\
157.24509	-23.193\\
140.132106	-18.311\\
108.968761	-14.648\\
87.712224	-19.531\\
116.580539	-17.09\\
100.45502	-14.648\\
85.016992	-12.207\\
71.081361	-13.428\\
77.936112	-12.207\\
73.534968	-25.635\\
160.06494	-30.518\\
191.103716	-25.635\\
159.60351	-23.193\\
143.958951	-25.635\\
153.50238	-14.648\\
85.558968	-13.428\\
77.211	-9.766\\
55.256028	-9.766\\
55.08024	-12.207\\
71.081361	-18.311\\
109.298359	-19.531\\
118.006302	-23.193\\
139.714632	-18.311\\
107.632058	-15.869\\
95.594856	-19.531\\
120.877359	-28.076\\
175.811912	-28.076\\
174.267732	-23.193\\
149.061411	-37.842\\
243.210534	-25.635\\
156.78366	-15.869\\
93.563624	-13.428\\
77.452704	-12.207\\
72.424131	-19.531\\
119.080507	-23.193\\
145.675233	-28.076\\
173.762364	-24.414\\
148.412706	-18.311\\
106.624953	-10.986\\
62.56527	-14.648\\
83.684024	-10.986\\
63.1695	-13.428\\
75.478788	-8.545\\
48.1938	-10.986\\
61.35681	-8.545\\
46.16009	-7.324\\
40.362564	-9.766\\
55.08024	-13.428\\
77.452704	-15.869\\
91.818034	-17.09\\
98.88274	-17.09\\
99.51507	-15.869\\
93.563624	-19.531\\
114.080571	-12.207\\
72.192198	-21.973\\
133.178353	-23.193\\
138.879684	-20.752\\
124.262976	-18.311\\
108.639163	-18.311\\
109.298359	-19.531\\
119.451596	-24.414\\
148.412706	-19.531\\
116.580539	-15.869\\
95.023572	-18.311\\
109.298359	-17.09\\
99.51507	-13.428\\
79.171488	-20.752\\
126.151408	-25.635\\
156.78366	-23.193\\
143.124003	-25.635\\
163.80765	-32.959\\
208.202003	-24.414\\
151.537698	-23.193\\
141.407721	-19.531\\
119.803154	-23.193\\
146.510181	-29.297\\
181.319133	-20.752\\
126.151408	-19.531\\
121.228917	-24.414\\
147.973254	-17.09\\
98.88274	-10.986\\
63.564996	-15.869\\
90.659597	-8.545\\
47.723825	-8.545\\
47.56147	-8.545\\
47.25385	-8.545\\
46.630065	-9.766\\
54.00598	-10.986\\
62.356536	-14.648\\
84.489664	-17.09\\
100.76264	-19.531\\
116.228981	-20.752\\
125.383584	-21.973\\
135.177896	-25.635\\
158.193585	-24.414\\
147.509388	-17.09\\
102.95016	-20.752\\
124.262976	-17.09\\
102.64254	-19.531\\
120.154712	-24.414\\
149.316024	-21.973\\
134.386868	-23.193\\
140.990247	-20.752\\
125.010048	-19.531\\
120.877359	-26.855\\
166.205595	-21.973\\
135.595383	-23.193\\
142.265862	-21.973\\
130.761323	-14.648\\
83.684024	-9.766\\
54.00598	-8.545\\
48.50142	-13.428\\
76.47246	-12.207\\
70.629702	-17.09\\
98.57512	-14.648\\
85.295304	-18.311\\
112.319674	-25.635\\
162.397725	-30.518\\
192.232882	-21.973\\
137.594926	-26.855\\
167.19923	-23.193\\
140.990247	-18.311\\
110.635062	-19.531\\
119.080507	-19.531\\
118.006302	-18.311\\
111.312569	-20.752\\
128.434128	-25.635\\
163.80765	-32.959\\
212.420755	-29.297\\
187.20783	-29.297\\
188.291819	-30.518\\
192.232882	-23.193\\
145.675233	-25.635\\
159.116445	-19.531\\
120.154712	-18.311\\
114.663482	-24.414\\
154.66269	-29.297\\
175.957782	-9.766\\
57.580336	-19.531\\
114.803218	-12.207\\
72.643857	-18.311\\
109.646268	-18.311\\
106.954551	-10.986\\
64.773456	-12.207\\
74.658012	-23.193\\
149.919552	-37.842\\
248.735466	-35.4\\
232.047	-34.18\\
220.93952	-26.855\\
175.067745	-31.738\\
210.359464	-36.621\\
238.732299	-28.076\\
181.988632	-30.518\\
198.367	-31.738\\
201.663252	-23.193\\
144.817092	-18.311\\
115.340989	-24.414\\
150.658794	-17.09\\
102.33492	-13.428\\
81.131976	-18.311\\
108.968761	-12.207\\
72.424131	-17.09\\
101.70259	-14.648\\
87.712224	-18.311\\
112.319674	-20.752\\
126.151408	-15.869\\
94.722061	-12.207\\
73.095516	-15.869\\
95.880498	-18.311\\
112.319674	-20.752\\
126.151408	-18.311\\
109.975866	-18.311\\
112.997181	-24.414\\
151.098246	-23.193\\
140.572773	-15.869\\
98.800394	-25.635\\
160.52637	-20.752\\
126.919232	-17.09\\
101.70259	-14.648\\
84.753328	-9.766\\
54.54311	-7.324\\
42.244832	-15.869\\
92.690829	-13.428\\
76.714164	-10.986\\
61.148076	-7.324\\
40.633552	-12.207\\
70.19025	-14.648\\
86.100944	-15.869\\
95.309214	-19.531\\
119.803154	-23.193\\
142.683336	-21.973\\
135.990897	-24.414\\
146.630484	-17.09\\
96.3876	-8.545\\
47.877635	-14.648\\
82.336408	-9.766\\
54.894686	-13.428\\
78.674652	-17.09\\
101.70259	-18.311\\
113.326779	-28.076\\
171.712816	-18.311\\
110.305464	-21.973\\
135.595383	-23.193\\
141.848388	-18.311\\
108.639163	-14.648\\
84.489664	-13.428\\
77.936112	-17.09\\
101.39497	-18.311\\
112.649272	-26.855\\
163.251545	-15.869\\
94.150777	-15.869\\
94.150777	-15.869\\
95.023572	-17.09\\
104.19773	-20.752\\
132.231744	-32.959\\
211.234231	-28.076\\
179.40564	-28.076\\
174.801176	-19.531\\
116.951628	-13.428\\
80.151732	-15.869\\
90.659597	-9.766\\
55.793158	-6.104\\
34.982024	-17.09\\
97.94279	-17.09\\
102.01021	-19.531\\
121.228917	-25.635\\
162.397725	-28.076\\
183.027444	-35.4\\
230.7726	-31.738\\
198.172072	-18.311\\
111.642167	-18.311\\
110.635062	-13.428\\
81.870516	-18.311\\
113.656377	-24.414\\
156.469326	-29.297\\
186.153138	-23.193\\
142.683336	-18.311\\
112.997181	-20.752\\
129.949024	-25.635\\
158.193585	-18.311\\
109.975866	-13.428\\
78.674652	-12.207\\
69.067206	-7.324\\
40.633552	-8.545\\
46.3139	-4.883\\
26.910213	-12.207\\
69.738591	-15.869\\
91.818034	-15.869\\
90.945239	-12.207\\
69.067206	-13.428\\
72.282924	-7.324\\
37.015496	-3.662\\
18.372254	-7.324\\
38.223956	-10.986\\
60.14835	-12.207\\
69.067206	-14.648\\
84.489664	-18.311\\
108.291254	-20.752\\
126.524944	-24.414\\
156.469326	-34.18\\
221.55476	-32.959\\
215.452983	-36.621\\
239.391477	-31.738\\
198.743356	-20.752\\
126.919232	-18.311\\
111.990076	-18.311\\
113.326779	-21.973\\
135.990897	-17.09\\
103.25778	-15.869\\
93.849266	-12.207\\
70.19025	-12.207\\
71.752746	-17.09\\
101.39497	-14.648\\
83.42036	-8.545\\
46.937685	-7.324\\
40.765384	-10.986\\
62.763018	-14.648\\
82.878384	-13.428\\
77.211	-14.648\\
83.947688	-12.207\\
71.301087	-15.869\\
96.753293	-23.193\\
140.572773	-17.09\\
105.46239	-24.414\\
156.469326	-32.959\\
214.859721	-35.4\\
228.8256	-26.855\\
167.68262	-19.531\\
118.377391	-14.648\\
85.295304	-9.766\\
57.043206	-15.869\\
91.532392	-10.986\\
64.971204	-17.09\\
102.01021	-14.648\\
85.822632	-12.207\\
72.643857	-17.09\\
98.57512	-12.207\\
71.520813	-12.207\\
70.849428	-13.428\\
74.99538	-9.766\\
54.357556	-9.766\\
53.46885	-8.545\\
46.630065	-8.545\\
46.3139	-7.324\\
38.619452	-6.104\\
32.302368	-6.104\\
32.857832	-9.766\\
52.755932	-10.986\\
59.346372	-8.545\\
47.723825	-15.869\\
92.103676	-17.09\\
98.57512	-14.648\\
83.947688	-14.648\\
86.628272	-20.752\\
126.524944	-25.635\\
155.835165	-18.311\\
110.982971	-21.973\\
127.135778	-9.766\\
53.644638	-6.104\\
34.310584	-14.648\\
86.100944	-18.311\\
108.968761	-19.531\\
120.154712	-26.855\\
166.205595	-25.635\\
155.373735	-17.09\\
100.45502	-13.428\\
78.191244	-15.869\\
92.976471	-14.648\\
84.753328	-10.986\\
61.148076	-7.324\\
40.50172	-12.207\\
67.053051	-7.324\\
39.69608	-6.104\\
32.07652	-7.324\\
39.425092	-9.766\\
54.54311	-13.428\\
75.73392	-13.428\\
74.740248	-9.766\\
56.1545	-17.09\\
102.64254	-23.193\\
136.745928	-14.648\\
87.170248	-20.752\\
125.777872	-23.193\\
138.439017	-15.869\\
90.373955	-10.986\\
62.356536	-14.648\\
84.753328	-14.648\\
82.072744	-9.766\\
54.181768	-12.207\\
67.272777	-7.324\\
38.758608	-6.104\\
31.7408	-6.104\\
32.638088	-8.545\\
46.783875	-12.207\\
66.833325	-10.986\\
60.75258	-12.207\\
68.615547	-14.648\\
81.267104	-9.766\\
52.033248	-7.324\\
38.0848	-6.104\\
31.7408	-8.545\\
44.903975	-8.545\\
45.52776	-9.766\\
55.431816	-18.311\\
106.624953	-18.311\\
105.947446	-14.648\\
84.489664	-14.648\\
86.100944	-19.531\\
118.006302	-24.414\\
150.658794	-25.635\\
159.60351	-26.855\\
164.24518	-21.973\\
133.969381	-21.973\\
134.782382	-21.973\\
134.386868	-23.193\\
143.124003	-25.635\\
162.397725	-30.518\\
193.33153	-26.855\\
169.159645	-24.414\\
151.537698	-20.752\\
127.666304	-19.531\\
119.803154	-19.531\\
120.877359	-24.414\\
147.069936	-15.869\\
95.023572	-19.531\\
118.728949	-21.973\\
135.177896	-24.414\\
148.852158	-17.09\\
104.83006	-20.752\\
126.151408	-18.311\\
109.975866	-17.09\\
99.51507	-12.207\\
71.752746	-17.09\\
104.52244	-24.414\\
148.852158	-19.531\\
114.803218	-10.986\\
64.575708	-14.648\\
82.878384	-12.207\\
67.053051	-9.766\\
54.357556	-10.986\\
60.346098	-7.324\\
39.564248	-9.766\\
52.93172	-8.545\\
45.52776	-6.104\\
33.08368	-7.324\\
39.425092	-9.766\\
51.144542	-7.324\\
39.022272	-8.545\\
45.997735	-9.766\\
53.820426	-10.986\\
60.346098	-13.428\\
74.25684	-12.207\\
68.615547	-15.869\\
89.50116	-13.428\\
76.217328	-14.648\\
87.712224	-24.414\\
145.287714	-17.09\\
100.45502	-15.869\\
93.849266	-18.311\\
106.624953	-13.428\\
74.740248	-8.545\\
48.34761	-19.531\\
110.15484	-10.986\\
59.950602	-8.545\\
47.56147	-10.986\\
61.752306	-13.428\\
76.217328	-12.207\\
67.944162	-10.986\\
58.93989	-6.104\\
31.7408	-6.104\\
31.630928	-6.104\\
32.74796	-12.207\\
66.16194	-10.986\\
59.950602	-12.207\\
68.615547	-15.869\\
92.690829	-19.531\\
113.729013	-15.869\\
93.849266	-20.752\\
125.777872	-24.414\\
147.509388	-20.752\\
124.636512	-20.752\\
129.949024	-31.738\\
198.172072	-23.193\\
146.092707	-30.518\\
191.683558	-24.414\\
154.223238	-26.855\\
171.60345	-30.518\\
188.875902	-20.752\\
122.353792	-10.986\\
63.1695	-12.207\\
71.301087	-15.869\\
94.722061	-18.311\\
109.646268	-18.311\\
107.632058	-15.869\\
88.628365	-9.766\\
54.181768	-10.986\\
63.971478	-18.311\\
110.982971	-24.414\\
149.755476	-23.193\\
137.604069	-17.09\\
98.57512	-12.207\\
71.520813	-14.648\\
89.045192	-24.414\\
151.537698	-25.635\\
160.52637	-26.855\\
164.72857	-20.752\\
128.807664	-26.855\\
169.643035	-29.297\\
185.069149	-26.855\\
174.07411	-35.4\\
230.1	-31.738\\
203.980126	-26.855\\
174.07411	-31.738\\
204.55141	-26.855\\
165.21196	-17.09\\
102.64254	-14.648\\
89.045192	-18.311\\
112.997181	-20.752\\
128.060592	-21.973\\
133.573867	-18.311\\
112.319674	-20.752\\
127.292768	-18.311\\
109.975866	-14.648\\
88.781528	-17.09\\
103.58249	-17.09\\
106.07763	-25.635\\
161.013435	-24.414\\
150.658794	-20.752\\
125.010048	-14.648\\
84.489664	-12.207\\
70.849428	-13.428\\
77.452704	-14.648\\
83.142048	-9.766\\
54.894686	-8.545\\
48.971395	-14.648\\
86.364608	-17.09\\
102.01021	-20.752\\
124.636512	-19.531\\
113.006366	-12.207\\
69.067206	-8.545\\
47.091495	-9.766\\
53.644638	-8.545\\
48.34761	-13.428\\
76.955868	-14.648\\
83.947688	-13.428\\
79.910028	-19.531\\
119.080507	-25.635\\
158.655015	-24.414\\
152.880468	-26.855\\
170.63667	-30.518\\
189.425226	-17.09\\
103.25778	-17.09\\
101.39497	-14.648\\
86.364608	-15.869\\
95.023572	-18.311\\
109.298359	-15.869\\
92.103676	-10.986\\
62.763018	-9.766\\
57.404548	-17.09\\
102.95016	-20.752\\
125.010048	-18.311\\
114.004286	-26.855\\
170.63667	-31.738\\
198.172072	-21.973\\
132.760866	-14.648\\
85.822632	-13.428\\
77.211	-10.986\\
65.179938	-18.311\\
108.639163	-17.09\\
102.01021	-15.869\\
95.880498	-18.311\\
111.312569	-23.193\\
141.407721	-20.752\\
127.292768	-24.414\\
147.069936	-17.09\\
102.95016	-18.311\\
108.639163	-14.648\\
85.558968	-12.207\\
72.863583	-17.09\\
105.13768	-24.414\\
152.441016	-25.635\\
154.88667	-19.531\\
124.451532	-35.4\\
228.8256	-26.855\\
168.16601	-20.752\\
124.262976	-13.428\\
80.648568	-17.09\\
102.95016	-17.09\\
102.01021	-15.869\\
95.594856	-17.09\\
100.76264	-14.648\\
87.170248	-13.428\\
84.086136	-28.076\\
179.40564	-30.518\\
190.554392	-18.311\\
114.004286	-23.193\\
145.234566	-24.414\\
153.344334	-24.414\\
150.658794	-18.311\\
111.642167	-18.311\\
108.968761	-15.869\\
95.594856	-18.311\\
111.312569	-21.973\\
138.803441	-28.076\\
181.483264	-32.959\\
219.078473	-42.725\\
282.41225	-35.4\\
228.8256	-24.414\\
154.223238	-21.973\\
138.407927	-21.973\\
136.386411	-18.311\\
109.646268	-13.428\\
79.413192	-12.207\\
71.972472	-13.428\\
77.936112	-8.545\\
47.877635	-8.545\\
48.50142	-9.766\\
56.681864	-14.648\\
83.684024	-12.207\\
67.944162	-8.545\\
47.091495	-7.324\\
42.376664	-14.648\\
88.239552	-23.193\\
134.612172	-12.207\\
70.409976	-12.207\\
70.849428	-17.09\\
98.88274	-14.648\\
85.558968	-17.09\\
99.82269	-15.869\\
90.659597	-10.986\\
60.950328	-9.766\\
53.107508	-6.104\\
33.4194	-8.545\\
48.817585	-15.869\\
92.976471	-18.311\\
105.617848	-13.428\\
75.478788	-9.766\\
53.107508	-7.324\\
39.564248	-12.207\\
68.84748	-13.428\\
77.936112	-18.311\\
107.632058	-18.311\\
105.617848	-14.648\\
83.947688	-12.207\\
69.958317	-13.428\\
78.929784	-18.311\\
111.312569	-25.635\\
160.52637	-28.076\\
173.762364	-23.193\\
138.879684	-14.648\\
86.906584	-14.648\\
86.628272	-15.869\\
93.563624	-14.648\\
84.226	-12.207\\
69.958317	-12.207\\
70.629702	-14.648\\
84.753328	-14.648\\
82.61472	-10.986\\
59.950602	-8.545\\
46.937685	-8.545\\
48.1938	-13.428\\
78.929784	-17.09\\
102.64254	-21.973\\
134.386868	-23.193\\
141.848388	-23.193\\
142.683336	-23.193\\
146.927655	-30.518\\
201.174656	-40.283\\
265.545536	-34.18\\
219.05962	-23.193\\
148.20327	-26.855\\
174.07411	-29.297\\
193.65317	-36.621\\
236.718144	-28.076\\
170.168636	-14.648\\
84.753328	-9.766\\
55.61737	-12.207\\
67.724436	-9.766\\
51.681672	-6.104\\
32.186392	-7.324\\
39.29326	-9.766\\
54.181768	-12.207\\
68.176095	-12.207\\
67.50471	-10.986\\
60.346098	-9.766\\
54.181768	-10.986\\
61.96104	-14.648\\
83.684024	-14.648\\
83.684024	-13.428\\
74.99538	-10.986\\
59.950602	-7.324\\
39.425092	-7.324\\
40.50172	-10.986\\
61.752306	-12.207\\
67.724436	-9.766\\
52.570378	-7.324\\
40.230732	-10.986\\
59.741868	-13.428\\
73.021464	-9.766\\
54.00598	-13.428\\
76.955868	-14.648\\
85.295304	-19.531\\
111.229045	-13.428\\
78.191244	-15.869\\
93.849266	-17.09\\
99.51507	-15.869\\
90.945239	-13.428\\
78.929784	-18.311\\
110.982971	-25.635\\
156.78366	-23.193\\
143.124003	-25.635\\
156.296595	-21.973\\
132.760866	-19.531\\
117.303186	-18.311\\
111.642167	-21.973\\
133.178353	-19.531\\
120.154712	-20.752\\
128.434128	-23.193\\
143.541477	-23.193\\
151.195167	-40.283\\
263.329971	-34.18\\
212.80468	-17.09\\
102.01021	-12.207\\
71.972472	-14.648\\
87.170248	-17.09\\
103.58249	-20.752\\
127.666304	-23.193\\
144.817092	-24.414\\
152.880468	-26.855\\
164.24518	-19.531\\
118.006302	-15.869\\
96.182009	-19.531\\
118.728949	-20.752\\
121.585968	-12.207\\
69.286932	-10.986\\
63.762744	-13.428\\
78.674652	-13.428\\
79.910028	-17.09\\
104.52244	-23.193\\
141.848388	-21.973\\
131.574324	-14.648\\
86.100944	-12.207\\
73.095516	-18.311\\
111.990076	-23.193\\
146.510181	-26.855\\
173.080475	-31.738\\
208.04259	-36.621\\
238.732299	-30.518\\
197.268352	-29.297\\
181.846479	-19.531\\
116.228981	-13.428\\
82.125648	-21.973\\
138.803441	-28.076\\
171.712816	-18.311\\
114.333884	-20.752\\
134.140928	-35.4\\
233.994	-37.842\\
242.529378	-25.635\\
156.78366	-17.09\\
99.51507	-9.766\\
57.043206	-12.207\\
71.301087	-14.648\\
85.558968	-12.207\\
69.286932	-10.986\\
62.356536	-10.986\\
61.554558	-12.207\\
69.067206	-8.545\\
47.877635	-12.207\\
68.84748	-15.869\\
92.976471	-18.311\\
107.961656	-17.09\\
102.95016	-19.531\\
120.877359	-28.076\\
175.306544	-24.414\\
147.069936	-17.09\\
101.70259	-15.869\\
95.309214	-19.531\\
115.154776	-14.648\\
85.295304	-10.986\\
62.356536	-10.986\\
63.564996	-13.428\\
77.694408	-12.207\\
68.395821	-9.766\\
52.218802	-6.104\\
31.966648	-6.104\\
32.857832	-8.545\\
47.56147	-14.648\\
82.61472	-13.428\\
74.001708	-10.986\\
60.75258	-8.545\\
48.663775	-15.869\\
92.976471	-18.311\\
106.624953	-10.986\\
62.960766	-13.428\\
77.694408	-15.869\\
93.277982	-17.09\\
100.13031	-17.09\\
100.76264	-19.531\\
115.506334	-19.531\\
113.006366	-13.428\\
76.217328	-9.766\\
55.968946	-14.648\\
83.684024	-10.986\\
62.960766	-13.428\\
78.674652	-19.531\\
119.803154	-25.635\\
156.296595	-21.973\\
132.760866	-18.311\\
113.656377	-26.855\\
165.722205	-24.414\\
145.727166	-14.648\\
84.489664	-10.986\\
61.752306	-10.986\\
61.35681	-9.766\\
53.820426	-8.545\\
46.783875	-7.324\\
41.842012	-14.648\\
80.461464	-13.428\\
71.799516	-8.545\\
46.630065	-12.207\\
68.615547	-13.428\\
75.237084	-9.766\\
53.293062	-8.545\\
44.903975	-4.883\\
26.109401	-7.324\\
41.30736	-14.648\\
85.822632	-19.531\\
114.080571	-17.09\\
97.94279	-12.207\\
70.19025	-14.648\\
84.489664	-14.648\\
85.295304	-15.869\\
93.849266	-20.752\\
125.383584	-23.193\\
140.990247	-23.193\\
139.714632	-19.531\\
114.803218	-10.986\\
63.1695	-10.986\\
62.763018	-12.207\\
71.081361	-15.869\\
90.072444	-14.648\\
81.530768	-8.545\\
49.28756	-17.09\\
101.07026	-20.752\\
122.727328	-18.311\\
109.975866	-20.752\\
128.434128	-25.635\\
153.015315	-19.531\\
119.080507	-18.311\\
114.663482	-26.855\\
165.722205	-21.973\\
136.803898	-24.414\\
152.001564	-25.635\\
167.575995	-37.842\\
252.21693	-39.063\\
252.503232	-25.635\\
166.6275	-32.959\\
217.85899	-36.621\\
242.723988	-35.4\\
237.2508	-37.842\\
253.617084	-36.621\\
249.462252	-43.945\\
298.518385	-41.504\\
273.594368	-24.414\\
154.223238	-19.531\\
121.228917	-18.311\\
111.990076	-15.869\\
94.722061	-12.207\\
74.426079	-17.09\\
107.01758	-23.193\\
142.265862	-18.311\\
110.305464	-13.428\\
80.406864	-15.869\\
92.405187	-12.207\\
70.629702	-10.986\\
61.148076	-10.986\\
59.950602	-7.324\\
39.564248	-8.545\\
46.630065	-15.869\\
86.00998	-8.545\\
45.37395	-8.545\\
44.12638	-6.104\\
30.73364	-3.662\\
18.910568	-7.324\\
37.418316	-8.545\\
43.49405	-6.104\\
32.857832	-12.207\\
69.286932	-14.648\\
84.226	-14.648\\
83.142048	-12.207\\
69.958317	-15.869\\
95.309214	-21.973\\
127.531292	-17.09\\
94.18299	-6.104\\
32.522112	-10.986\\
56.929452	-6.104\\
31.966648	-7.324\\
40.0989	-13.428\\
76.714164	-15.869\\
89.199649	-15.869\\
91.532392	-12.207\\
73.754694	-28.076\\
171.179372	-23.193\\
139.714632	-17.09\\
98.57512	-13.428\\
77.211	-14.648\\
86.906584	-20.752\\
125.777872	-23.193\\
136.745928	-18.311\\
102.266935	-6.104\\
34.310584	-12.207\\
68.176095	-9.766\\
51.681672	-6.104\\
30.849616	-3.662\\
19.0424	-7.324\\
41.571024	-19.531\\
113.729013	-19.531\\
111.932161	-10.986\\
60.75258	-8.545\\
46.630065	-8.545\\
43.34024	-9.766\\
48.996022	-6.104\\
30.959488	-3.662\\
19.0424	-9.766\\
53.107508	-14.648\\
83.42036	-18.311\\
106.954551	-20.752\\
122.353792	-19.531\\
116.228981	-20.752\\
123.868688	-20.752\\
123.868688	-19.531\\
115.506334	-18.311\\
109.298359	-21.973\\
136.803898	-30.518\\
191.683558	-26.855\\
161.77452	-17.09\\
96.3876	-8.545\\
50.22751	-12.207\\
73.754694	-26.855\\
162.25791	-21.973\\
127.948779	-13.428\\
76.714164	-15.869\\
94.150777	-20.752\\
129.201952	-29.297\\
185.596495	-29.297\\
187.764473	-31.738\\
205.154432	-29.297\\
186.153138	-26.855\\
169.643035	-24.414\\
146.191032	-17.09\\
98.57512	-13.428\\
78.674652	-15.869\\
94.436419	-19.531\\
117.303186	-17.09\\
102.95016	-19.531\\
114.803218	-13.428\\
79.413192	-18.311\\
108.639163	-17.09\\
97.63517	-7.324\\
39.967068	-4.883\\
25.748059	-4.883\\
26.109401	-7.324\\
38.89044	-8.545\\
43.810215	-3.662\\
19.177894	-7.324\\
40.90454	-17.09\\
95.12294	-3.662\\
20.04945	-10.986\\
61.148076	-14.648\\
85.822632	-20.752\\
121.212432	-18.311\\
101.25983	-7.324\\
38.355788	-7.324\\
39.967068	-7.324\\
43.182304	-24.414\\
150.194928	-26.855\\
173.080475	-34.18\\
228.42494	-42.725\\
288.692825	-40.283\\
264.780159	-29.297\\
190.4305	-29.297\\
194.180516	-36.621\\
242.06481	-32.959\\
216.046245	-29.297\\
193.125824	-30.518\\
202.273304	-35.4\\
235.3038	-35.4\\
242.4192	-46.387\\
312.555606	-36.621\\
235.363167	-18.311\\
110.982971	-10.986\\
63.564996	-9.766\\
56.330288	-10.986\\
63.1695	-12.207\\
69.958317	-10.986\\
65.179938	-17.09\\
101.07026	-18.311\\
107.961656	-14.648\\
88.239552	-19.531\\
113.729013	-12.207\\
70.409976	-17.09\\
101.07026	-18.311\\
109.975866	-19.531\\
114.080571	-10.986\\
61.752306	-6.104\\
34.42656	-10.986\\
61.96104	-13.428\\
80.890272	-17.09\\
108.58986	-30.518\\
195.01002	-30.518\\
196.68851	-32.959\\
216.639507	-35.4\\
240.4722	-45.166\\
308.48378	-40.283\\
265.545536	-26.855\\
169.643035	-17.09\\
102.01021	-13.428\\
78.191244	-13.428\\
78.432948	-14.648\\
87.170248	-15.869\\
93.563624	-8.545\\
49.44137	-12.207\\
71.081361	-12.207\\
71.752746	-14.648\\
86.628272	-18.311\\
110.305464	-18.311\\
109.298359	-15.869\\
89.786802	-7.324\\
39.69608	-4.883\\
25.660165	-7.324\\
38.0848	-4.883\\
25.572271	-6.104\\
33.193552	-10.986\\
60.543846	-14.648\\
82.072744	-15.869\\
92.690829	-19.531\\
113.357924	-17.09\\
102.33492	-19.531\\
124.099974	-34.18\\
217.17972	-26.855\\
168.676255	-21.973\\
139.198955	-28.076\\
180.94982	-31.738\\
199.346378	-20.752\\
128.060592	-18.311\\
108.968761	-13.428\\
78.929784	-15.869\\
98.498883	-28.076\\
186.087728	-41.504\\
284.966464	-46.387\\
316.82321	-36.621\\
246.752298	-31.738\\
208.613874	-26.855\\
175.067745	-28.076\\
186.621172	-34.18\\
219.67486	-23.193\\
144.817092	-18.311\\
112.319674	-18.311\\
116.000185	-24.414\\
156.908778	-25.635\\
157.70652	-14.648\\
88.239552	-14.648\\
86.100944	-13.428\\
80.890272	-17.09\\
107.01758	-26.855\\
169.643035	-25.635\\
159.60351	-20.752\\
126.151408	-18.311\\
111.312569	-17.09\\
102.95016	-14.648\\
85.822632	-8.545\\
48.817585	-8.545\\
47.877635	-7.324\\
39.967068	-7.324\\
39.69608	-6.104\\
34.310584	-10.986\\
63.971478	-17.09\\
100.45502	-18.311\\
104.940341	-13.428\\
75.73392	-9.766\\
54.181768	-8.545\\
47.877635	-10.986\\
62.960766	-14.648\\
84.753328	-17.09\\
98.88274	-14.648\\
82.072744	-9.766\\
56.506076	-20.752\\
125.777872	-21.973\\
131.156837	-15.869\\
88.914007	-4.883\\
26.734425	-10.986\\
62.356536	-13.428\\
77.211	-14.648\\
83.684024	-9.766\\
54.357556	-7.324\\
41.71018	-13.428\\
79.668324	-21.973\\
127.948779	-13.428\\
73.021464	-3.662\\
19.64663	-9.766\\
55.431816	-14.648\\
86.628272	-20.752\\
119.324	-14.648\\
82.336408	-12.207\\
71.520813	-20.752\\
126.524944	-26.855\\
164.24518	-23.193\\
145.675233	-29.297\\
189.903154	-35.4\\
224.259	-24.414\\
151.098246	-19.531\\
123.025769	-26.855\\
166.688985	-19.531\\
122.303122	-23.193\\
148.643937	-29.297\\
186.153138	-21.973\\
132.365352	-10.986\\
66.981642	-20.752\\
133.74664	-34.18\\
226.54504	-37.842\\
248.05431	-29.297\\
188.819165	-24.414\\
159.154866	-31.738\\
204.55141	-26.855\\
167.19923	-17.09\\
104.52244	-17.09\\
105.13768	-19.531\\
121.228917	-21.973\\
136.386411	-19.531\\
121.951564	-21.973\\
135.595383	-14.648\\
89.587168	-19.531\\
122.303122	-24.414\\
149.316024	-18.311\\
108.639163	-14.648\\
85.558968	-12.207\\
69.286932	-8.545\\
48.1938	-9.766\\
56.867418	-17.09\\
100.76264	-15.869\\
95.023572	-19.531\\
120.154712	-25.635\\
161.474865	-26.855\\
167.68262	-20.752\\
127.292768	-18.311\\
106.624953	-10.986\\
63.762744	-14.648\\
89.587168	-24.414\\
147.509388	-18.311\\
105.617848	-8.545\\
50.38132	-15.869\\
97.626088	-25.635\\
160.06494	-23.193\\
149.061411	-30.518\\
203.40247	-39.063\\
257.503296	-30.518\\
196.68851	-26.855\\
175.067745	-29.297\\
186.153138	-20.752\\
128.434128	-14.648\\
90.656472	-17.09\\
107.34229	-24.414\\
154.66269	-24.414\\
154.66269	-25.635\\
158.655015	-13.428\\
80.890272	-14.648\\
86.628272	-12.207\\
72.863583	-17.09\\
102.01021	-14.648\\
86.100944	-12.207\\
74.658012	-20.752\\
129.575488	-24.414\\
148.852158	-15.869\\
95.309214	-12.207\\
74.426079	-20.752\\
123.495152	-13.428\\
75.73392	-9.766\\
54.894686	-10.986\\
62.960766	-12.207\\
69.738591	-10.986\\
61.148076	-6.104\\
32.522112	-6.104\\
32.74796	-10.986\\
61.148076	-12.207\\
71.081361	-20.752\\
124.262976	-19.531\\
120.525801	-21.973\\
139.198955	-28.076\\
172.218184	-14.648\\
86.628272	-14.648\\
90.656472	-25.635\\
166.6275	-36.621\\
237.377322	-26.855\\
172.597085	-25.635\\
168.98592	-39.063\\
257.503296	-31.738\\
205.154432	-25.635\\
161.936295	-20.752\\
129.949024	-20.752\\
132.60528	-25.635\\
161.013435	-18.311\\
112.649272	-17.09\\
107.95753	-28.076\\
182.494	-31.738\\
208.04259	-31.738\\
197.600788	-18.311\\
106.624953	-8.545\\
51.791245	-15.869\\
97.054804	-15.869\\
91.24675	-8.545\\
47.091495	-7.324\\
41.036372	-10.986\\
63.971478	-13.428\\
81.628812	-23.193\\
140.572773	-17.09\\
100.76264	-10.986\\
62.356536	-9.766\\
53.107508	-7.324\\
39.69608	-8.545\\
46.16009	-7.324\\
39.967068	-7.324\\
41.71018	-14.648\\
83.947688	-12.207\\
66.833325	-7.324\\
39.29326	-8.545\\
46.3139	-9.766\\
53.644638	-10.986\\
59.137638	-8.545\\
46.46771	-9.766\\
52.755932	-8.545\\
44.903975	-4.883\\
26.822319	-13.428\\
76.714164	-17.09\\
103.25778	-23.193\\
146.927655	-30.518\\
194.460696	-26.855\\
162.768155	-13.428\\
78.191244	-10.986\\
63.1695	-12.207\\
67.724436	-7.324\\
39.29326	-9.766\\
54.718898	-10.986\\
62.56527	-14.648\\
83.142048	-13.428\\
76.714164	-13.428\\
77.694408	-15.869\\
92.690829	-18.311\\
107.632058	-17.09\\
102.33492	-21.973\\
133.178353	-21.973\\
137.199412	-30.518\\
184.389756	-19.531\\
110.857956	-9.766\\
54.181768	-8.545\\
48.817585	-15.869\\
90.945239	-10.986\\
61.96104	-9.766\\
52.570378	-4.883\\
26.197295	-10.986\\
58.33566	-7.324\\
37.015496	-3.662\\
18.573664	-6.104\\
32.522112	-10.986\\
60.346098	-13.428\\
77.211	-15.869\\
95.880498	-24.414\\
145.727166	-17.09\\
95.44765	-7.324\\
41.30736	-13.428\\
77.694408	-18.311\\
101.25983	-10.986\\
60.950328	-13.428\\
78.432948	-20.752\\
122.727328	-17.09\\
104.19773	-25.635\\
162.397725	-30.518\\
188.875902	-20.752\\
125.383584	-17.09\\
102.33492	-15.869\\
};
\addplot [color=mycolor2, line width=2.0pt, forget plot]
  table[row sep=crcr]{%
112.319674	-17.5789117669522\\
121.951564	-19.0863782546041\\
150.658794	-23.5792853764999\\
119.803154	-18.7501352039945\\
129.575488	-20.2795822814779\\
160.52637	-25.1236385755458\\
154.223238	-24.1371488775481\\
175.811912	-27.5159460365525\\
138.439017	-21.6667942336317\\
79.668324	-12.4687188659127\\
101.70259	-15.9172546750849\\
102.01021	-15.9653996228503\\
108.968761	-17.0544675456688\\
88.914007	-13.9157409226381\\
45.057785	-7.05189748796193\\
31.40508	-4.91514185087535\\
40.362564	-6.31705849897643\\
57.756124	-9.03928734512843\\
132.365352	-20.7162179246494\\
140.572773	-22.000743818067\\
138.021543	-21.6014562714596\\
111.932161	-17.5182629367611\\
65.784168	-10.2957393282174\\
101.39497	-15.8691097273196\\
112.30325	-17.5763412818664\\
79.668324	-12.4687188659127\\
108.968761	-17.0544675456688\\
114.803218	-17.9676059225758\\
88.503216	-13.8514489024916\\
149.316024	-23.3691313178858\\
131.574324	-20.5924158254981\\
97.054804	-15.1898396364188\\
142.683336	-22.3310635157149\\
155.373735	-24.3172107004764\\
116.580539	-18.2457706280801\\
100.13031	-15.6711804976177\\
86.100944	-13.4754744536322\\
102.01021	-15.9653996228503\\
123.868688	-19.3864232284019\\
109.646268	-17.1605027160923\\
117.654744	-18.4138921533849\\
128.434128	-20.1009504708627\\
174.267732	-27.2742697299395\\
154.88667	-24.2409811998479\\
102.01021	-15.9653996228503\\
91.24675	-14.2808335365285\\
68.84748	-10.7751717325765\\
63.1695	-9.88652323601374\\
90.373955	-14.1442342592226\\
69.067206	-10.8095605785315\\
70.629702	-11.0541034831006\\
93.277982	-14.5987373091677\\
107.632058	-16.8452630202389\\
96.182009	-15.0532403591129\\
154.88667	-24.2409811998479\\
127.531292	-19.959649540947\\
69.067206	-10.8095605785315\\
55.793158	-8.73206773803158\\
77.452704	-12.1219569220604\\
88.628365	-13.8710357046107\\
60.346098	-9.44463863224756\\
61.554558	-9.63377212024086\\
68.176095	-10.6700947032695\\
53.46885	-8.36829526793679\\
47.40766	-7.41967139450271\\
68.615547	-10.7388723951795\\
85.295304	-13.3493854616366\\
100.76264	-15.7701451124686\\
108.968761	-17.0544675456688\\
138.439017	-21.6667942336317\\
133.178353	-20.8434589709993\\
132.365352	-20.7162179246494\\
101.70259	-15.9172546750849\\
101.07026	-15.818290060234\\
100.76264	-15.7701451124686\\
93.277982	-14.5987373091677\\
104.19773	-16.3077636958482\\
156.00546	-24.4160806280723\\
230.1	-36.0124584903594\\
239.391477	-37.4666476680067\\
216.5303	-33.8886937881577\\
156.908778	-24.5574569947763\\
193.125824	-30.2257093447043\\
243.419787	-38.0971098438489\\
264.055065	-41.326693035557\\
210.962486	-33.0172871364539\\
259.823172	-40.664368520137\\
330.8097	-51.7743180767417\\
260.354895	-40.7475873487588\\
196.68851	-30.7832977049355\\
160.06494	-25.0514211538978\\
119.451596	-18.6951136138948\\
89.850832	-14.0623613981932\\
118.377391	-18.52699208859\\
92.103676	-14.4149491906107\\
55.431816	-8.67551487503365\\
55.793158	-8.73206773803158\\
71.752746	-11.2298686957597\\
100.76264	-15.7701451124686\\
93.563624	-14.6434425271951\\
95.309214	-14.916641081807\\
118.728949	-18.5820136786897\\
121.600006	-19.0313566645044\\
174.801176	-27.3577579085876\\
145.675233	-22.7993189112818\\
159.60351	-24.9792037322498\\
141.848388	-22.2003875913707\\
112.649272	-17.6304964444555\\
125.777872	-19.6852255297946\\
102.01021	-15.9653996228503\\
88.239552	-13.8101834142021\\
110.305464	-17.2636720710988\\
101.07026	-15.818290060234\\
93.277982	-14.5987373091677\\
77.452704	-12.1219569220604\\
63.1695	-9.88652323601374\\
70.19025	-10.9853257911906\\
75.73392	-11.8529537171326\\
62.158788	-9.72833886423752\\
56.1545	-8.78862060102951\\
87.170248	-13.6428289339172\\
118.006302	-18.4689137434847\\
141.407721	-22.1314197424113\\
137.604069	-21.5361183092875\\
102.64254	-16.0643642377013\\
114.333884	-17.8941514628056\\
190.005068	-29.7372865028593\\
144.817092	-22.6650131001502\\
153.783786	-24.0683711856381\\
162.768155	-25.4744955475441\\
100.13031	-15.6711804976177\\
64.773456	-10.1375549564412\\
102.01021	-15.9653996228503\\
114.004286	-17.8425667853023\\
172.113695	-26.937146009604\\
220.93952	-34.5787713727942\\
202.234536	-31.6512943633946\\
162.397725	-25.4165203411182\\
153.344334	-23.999593493728\\
145.234566	-22.7303510623223\\
134.386868	-21.0326010669249\\
109.646268	-17.1605027160923\\
86.628272	-13.5580054302111\\
78.432948	-12.2753728123708\\
71.520813	-11.1935693583627\\
80.406864	-12.5843061805301\\
121.600006	-19.0313566645044\\
159.116445	-24.9029742316213\\
132.760866	-20.778118974225\\
87.170248	-13.6428289339172\\
79.171488	-12.3909601269882\\
83.42036	-13.0559419893561\\
66.833325	-10.4599406446554\\
46.46771	-7.2725618318864\\
48.1938	-7.54270848323635\\
75.478788	-11.8130235539011\\
68.615547	-10.7388723951795\\
68.84748	-10.7751717325765\\
67.50471	-10.5650176739624\\
45.997735	-7.19900705057824\\
38.223956	-5.98234953840646\\
25.92873	-4.05805003404058\\
41.71018	-6.52797099467806\\
102.64254	-16.0643642377013\\
149.755476	-23.4379090097958\\
147.069936	-23.0176008925677\\
113.729013	-17.799484397271\\
68.84748	-10.7751717325765\\
66.381666	-10.3892524613034\\
48.34761	-7.56678095711902\\
62.763018	-9.82290560823417\\
85.558968	-13.3906509499261\\
88.239552	-13.8101834142021\\
155.566008	-24.3473029361622\\
225.9298	-35.3597894143208\\
265.545536	-41.5599634615395\\
216.639507	-33.9057855419793\\
201.663252	-31.5618839273398\\
147.368322	-23.0643006467576\\
163.34622	-25.5649672633947\\
172.113695	-26.937146009604\\
177.86146	-27.8367164071479\\
137.199412	-21.4727863083517\\
129.201952	-20.2211209616402\\
120.525801	-18.8632351391996\\
121.228917	-18.9732783193991\\
134.386868	-21.0326010669249\\
122.674211	-19.1994781898092\\
149.478885	-23.3946203444055\\
185.596495	-29.0473101788079\\
148.852158	-23.2965326430918\\
88.239552	-13.8101834142021\\
99.086036	-15.5077434090581\\
164.294715	-25.7134141856712\\
216.046245	-33.8129353761866\\
235.3038	-36.8268940900645\\
261.058029	-40.8576333460998\\
248.735466	-38.9290553863764\\
198.367	-31.0459945821692\\
175.551135	-27.4751323864537\\
202.273304	-31.657361860095\\
228.8256	-35.8130048741051\\
158.193585	-24.7585393883253\\
91.198448	-14.2732738938948\\
120.877359	-18.9182567292993\\
124.262976	-19.4481323993417\\
73.095516	-11.4400227543738\\
81.387108	-12.7377220708405\\
109.646268	-17.1605027160923\\
86.100944	-13.4754744536322\\
71.972472	-11.2642575417147\\
107.632058	-16.8452630202389\\
65.179938	-10.2011725842208\\
89.587168	-14.0210959099038\\
149.755476	-23.4379090097958\\
131.969838	-20.6543168750737\\
87.170248	-13.6428289339172\\
89.308856	-13.9775378944871\\
133.74664	-20.9324003529989\\
218.4102	-34.1829129133903\\
172.723552	-27.0325934233266\\
94.722061	-14.8247470225284\\
87.712224	-13.7276524376232\\
87.170248	-13.6428289339172\\
71.301087	-11.1591805124077\\
63.762744	-9.979370584665\\
64.366974	-10.0739373286616\\
79.910028	-12.5065474416057\\
103.25778	-16.1606541332319\\
115.506334	-18.0776491027753\\
88.781528	-13.8950069179082\\
121.228917	-18.9732783193991\\
166.205595	-26.0124819243937\\
142.265862	-22.2657255535428\\
104.52244	-16.3585833629339\\
137.594926	-21.5346873579273\\
157.24509	-24.6100924660488\\
126.34475	-19.7739463922201\\
48.663775	-7.61626326454451\\
70.409976	-11.0197146371456\\
79.171488	-12.3909601269882\\
93.563624	-14.6434425271951\\
91.818034	-14.3702439725833\\
64.971204	-10.1685040726583\\
101.70259	-15.9172546750849\\
92.690829	-14.5068432498892\\
64.366974	-10.0739373286616\\
85.822632	-13.4319164382155\\
84.753328	-13.2645619579305\\
71.081361	-11.1247916664527\\
83.142048	-13.0123839739395\\
61.96104	-9.69738974802043\\
54.181768	-8.47987253817595\\
67.50471	-10.5650176739624\\
48.817585	-7.64033573842718\\
99.82269	-15.6230355498523\\
109.975866	-17.2120873935955\\
129.575488	-20.2795822814779\\
180.235144	-28.2082166093175\\
121.980256	-19.0908687781113\\
62.763018	-9.82290560823417\\
61.96104	-9.69738974802043\\
49.13375	-7.68981804585266\\
75.478788	-11.8130235539011\\
68.176095	-10.6700947032695\\
55.431816	-8.67551487503365\\
94.722061	-14.8247470225284\\
117.654744	-18.4138921533849\\
127.666304	-19.9807799800852\\
140.572773	-22.000743818067\\
138.439017	-21.6667942336317\\
97.63517	-15.2806714768544\\
88.326854	-13.8238468633596\\
59.54412	-9.31912277203382\\
58.742142	-9.19360691182008\\
39.564248	-6.19211577054449\\
47.723825	-7.4691537019282\\
62.960766	-9.85385472445126\\
98.57512	-15.4277810394707\\
106.624953	-16.6876431723122\\
104.19773	-16.3077636958482\\
150.194928	-23.5066867017059\\
154.42524	-24.1687637781999\\
96.182009	-15.0532403591129\\
119.080507	-18.6370352687895\\
144.817092	-22.6650131001502\\
166.205595	-26.0124819243937\\
153.96381	-24.0965463565519\\
97.00284	-15.1817068620034\\
59.54412	-9.31912277203382\\
39.29326	-6.14970401869144\\
32.857832	-5.14250895690224\\
46.16009	-7.22441688412106\\
52.570378	-8.22767733832032\\
46.630065	-7.29797166542922\\
62.356536	-9.7592879804546\\
97.63517	-15.2806714768544\\
90.945239	-14.2336446952774\\
77.694408	-12.1597854977534\\
89.50116	-14.0076349819166\\
65.490555	-10.2497865860413\\
33.639144	-5.2647904256898\\
56.330288	-8.81613280465012\\
120.838896	-18.9122369674961\\
100.45502	-15.7220001647033\\
107.284149	-16.7908125273187\\
91.24675	-14.2808335365285\\
64.169226	-10.0429882124446\\
102.64254	-16.0643642377013\\
140.990247	-22.0660817802391\\
137.604069	-21.5361183092875\\
104.52244	-16.3585833629339\\
164.72857	-25.7813159030911\\
155.373735	-24.3172107004764\\
112.319674	-17.5789117669522\\
142.683336	-22.3310635157149\\
155.373735	-24.3172107004764\\
116.580539	-18.2457706280801\\
97.927599	-15.3264389137247\\
156.469326	-24.4886793028662\\
225.5688	-35.3032900770108\\
179.180452	-28.0431490218768\\
102.95016	-16.1125091854666\\
102.64254	-16.0643642377013\\
111.312569	-17.4212919190255\\
128.060592	-20.042489151025\\
140.990247	-22.0660817802391\\
118.377391	-18.52699208859\\
105.46239	-16.5056929255502\\
152.441016	-23.858217127024\\
156.78366	-24.5378750444008\\
109.298359	-17.1060522231721\\
95.594856	-14.9613462998344\\
115.340989	-18.0517713107323\\
177.356092	-27.7576223431655\\
169.159645	-26.4748139669989\\
149.755476	-23.4379090097958\\
116.951628	-18.3038489731854\\
101.70259	-15.9172546750849\\
105.947446	-16.5816080018887\\
74.25684	-11.6217790878977\\
32.638088	-5.10811729380573\\
31.966648	-5.00303165656641\\
33.309528	-5.21320293104505\\
55.431816	-8.67551487503365\\
114.080571	-17.8545059873707\\
91.532392	-14.3255387545559\\
84.489664	-13.2232964696411\\
78.432948	-12.2753728123708\\
97.00284	-15.1817068620034\\
75.478788	-11.8130235539011\\
54.894686	-8.59144980841511\\
67.272777	-10.5287183365654\\
54.357556	-8.50738474179656\\
49.44137	-7.737962993618\\
101.70259	-15.9172546750849\\
120.838896	-18.9122369674961\\
96.69522	-15.133561914238\\
61.35681	-9.60282300402378\\
73.5183	-11.5061917732803\\
55.793158	-8.73206773803158\\
66.981642	-10.4831534208653\\
170.674004	-26.7118230527312\\
128.434128	-20.1009504708627\\
162.397725	-25.4165203411182\\
226.206	-35.403016885138\\
192.232882	-30.0859569035513\\
149.755476	-23.4379090097958\\
111.312569	-17.4212919190255\\
119.803154	-18.7501352039945\\
136.386411	-21.3455452620017\\
159.60351	-24.9792037322498\\
163.34622	-25.5649672633947\\
206.900022	-32.3814795911754\\
227.19446	-35.5577186440227\\
271.311648	-42.4624053087832\\
214.2335	-33.5292265362643\\
184.543548	-28.8825156975822\\
230.1	-36.0124584903594\\
175.067745	-27.3994780522092\\
189.903154	-29.7213361608576\\
174.267732	-27.2742697299395\\
101.07026	-15.818290060234\\
63.971478	-10.0120390962275\\
72.424131	-11.3349457250667\\
100.76264	-15.7701451124686\\
83.947688	-13.138472965935\\
56.506076	-8.84364500827073\\
76.714164	-12.006369607443\\
74.740248	-11.6974362392837\\
40.90454	-6.40188200268252\\
48.031445	-7.51729864969354\\
67.724436	-10.5994065199174\\
41.71018	-6.52797099467806\\
86.364608	-13.5167399419216\\
120.838896	-18.9122369674961\\
81.387108	-12.7377220708405\\
140.011956	-21.9129715497785\\
195.01002	-30.5206008277017\\
190.4305	-29.803869954578\\
227.5158	-35.6080108796216\\
187.20783	-29.2994967707313\\
175.811912	-27.5159460365525\\
138.879684	-21.7357620825912\\
72.863583	-11.4037234169768\\
86.364608	-13.5167399419216\\
91.532392	-14.3255387545559\\
63.762744	-9.979370584665\\
68.615547	-10.7388723951795\\
72.538056	-11.3527758829699\\
19.580714	-3.06453563727336\\
48.1938	-7.54270848323635\\
99.82269	-15.6230355498523\\
116.580539	-18.2457706280801\\
125.010048	-19.5650550390171\\
139.714632	-21.8664380069354\\
126.151408	-19.7436868496323\\
140.132106	-21.9317759691076\\
108.639163	-17.0028828681655\\
94.150777	-14.7353365864737\\
92.976471	-14.5515484679166\\
73.095516	-11.4400227543738\\
107.961656	-16.8968476977421\\
98.88274	-15.475925987236\\
71.752746	-11.2298686957597\\
103.58249	-16.2114738003176\\
139.297158	-21.8011000447633\\
100.45502	-15.7220001647033\\
77.936112	-12.1976140734463\\
71.081361	-11.1247916664527\\
83.684024	-13.0972074776456\\
41.168204	-6.44314749097197\\
48.971395	-7.66440821230985\\
71.081361	-11.1247916664527\\
100.45502	-15.7220001647033\\
99.51507	-15.574890602087\\
86.364608	-13.5167399419216\\
93.849266	-14.6881477452225\\
104.610743	-16.3724034764588\\
69.067206	-10.8095605785315\\
57.756124	-9.03928734512843\\
112.649272	-17.6304964444555\\
166.205595	-26.0124819243937\\
164.72857	-25.7813159030911\\
133.969381	-20.9672610701506\\
117.654744	-18.4138921533849\\
101.70259	-15.9172546750849\\
86.100944	-13.4754744536322\\
79.668324	-12.4687188659127\\
101.39497	-15.8691097273196\\
98.88274	-15.475925987236\\
64.366974	-10.0739373286616\\
86.906584	-13.6015634456277\\
109.975866	-17.2120873935955\\
118.728949	-18.5820136786897\\
133.969381	-20.9672610701506\\
137.199412	-21.4727863083517\\
188.819165	-29.5516833626546\\
233.3568	-36.5221732874538\\
234.00819	-36.6241209421084\\
164.24518	-25.7056615688467\\
85.558968	-13.3906509499261\\
61.148076	-9.5701544924613\\
41.036372	-6.42251474682724\\
61.554558	-9.63377212024086\\
63.971478	-10.0120390962275\\
85.822632	-13.4319164382155\\
77.936112	-12.1976140734463\\
71.972472	-11.2642575417147\\
94.722061	-14.8247470225284\\
126.151408	-19.7436868496323\\
127.666304	-19.9807799800852\\
155.566008	-24.3473029361622\\
203.980126	-31.9244929180065\\
184.541803	-28.8822425913672\\
164.72857	-25.7813159030911\\
110.982971	-17.3697072415222\\
103.58249	-16.2114738003176\\
127.292768	-19.9223186602475\\
125.777872	-19.6852255297946\\
102.33492	-16.0162192899359\\
87.433912	-13.6840944222066\\
105.617848	-16.5300233243855\\
71.752746	-11.2298686957597\\
81.628812	-12.7755506465335\\
124.636512	-19.5065937191794\\
102.01021	-15.9653996228503\\
114.663482	-17.9457361403089\\
205.795996	-32.2086908449899\\
158.193585	-24.7585393883253\\
131.090384	-20.5166754119308\\
178.394904	-27.920204585796\\
172.723552	-27.0325934233266\\
115.506334	-18.0776491027753\\
78.432948	-12.2753728123708\\
80.890272	-12.6599633319161\\
123.377327	-19.3095213700087\\
196.139186	-30.6973241814771\\
220.93952	-34.5787713727942\\
202.80582	-31.7407047994494\\
158.193585	-24.7585393883253\\
102.95016	-16.1125091854666\\
86.100944	-13.4754744536322\\
77.452704	-12.1219569220604\\
62.763018	-9.82290560823417\\
49.28756	-7.71389051973533\\
71.081361	-11.1247916664527\\
83.42036	-13.0559419893561\\
62.960766	-9.85385472445126\\
79.413192	-12.4287887026812\\
116.951628	-18.3038489731854\\
127.666304	-19.9807799800852\\
139.198955	-21.7857305034286\\
190.4305	-29.803869954578\\
219.05962	-34.2845522475615\\
161.474865	-25.2720854978222\\
146.510181	-22.929994835626\\
152.880468	-23.9269948189341\\
126.919232	-19.8638573404098\\
89.308856	-13.9775378944871\\
116.000185	-18.1549406657389\\
189.375808	-29.6388023671372\\
199.346378	-31.1992749366732\\
116.228981	-18.1907490379803\\
69.738591	-10.9146376078386\\
45.843925	-7.17493457669557\\
38.355788	-6.00298228255118\\
52.93172	-8.28423020131825\\
47.091495	-7.37018908707722\\
69.286932	-10.8439494244866\\
82.878384	-12.97111848565\\
74.498544	-11.6596076635907\\
54.00598	-8.45236033455533\\
60.346098	-9.44463863224756\\
59.950602	-9.38274039981338\\
53.46885	-8.36829526793679\\
61.96104	-9.69738974802043\\
86.628272	-13.5580054302111\\
132.365352	-20.7162179246494\\
110.635062	-17.3152567486021\\
126.919232	-19.8638573404098\\
138.012413	-21.6000273547016\\
193.911372	-30.3486537807851\\
205.154432	-32.1082810365635\\
222.17	-34.7713511638555\\
217.17972	-33.9903331223289\\
151.098246	-23.6480630684099\\
103.89011	-16.2596187480829\\
99.958831	-15.644342686364\\
172.597085	-27.0128003438485\\
199.346378	-31.1992749366732\\
140.990247	-22.0660817802391\\
104.940341	-16.423988153962\\
53.293062	-8.34078306431618\\
26.285189	-4.11383866916016\\
31.630928	-4.95048883794676\\
32.522112	-5.08996613828258\\
40.362564	-6.31705849897643\\
81.530768	-12.7602059899484\\
60.950328	-9.53920537624421\\
65.710281	-10.2841754319963\\
44.750165	-7.00375254019659\\
25.840836	-4.04429393223027\\
45.52776	-7.12545226927009\\
45.843925	-7.17493457669557\\
45.22014	-7.07730732150475\\
51.144542	-8.004522797842\\
39.29326	-6.14970401869144\\
64.169226	-10.0429882124446\\
149.755476	-23.4379090097958\\
167.19923	-26.1679936114518\\
156.78366	-24.5378750444008\\
136.803898	-21.410885258776\\
198.367	-31.0459945821692\\
264.055065	-41.326693035557\\
224.66514	-35.1618601846188\\
205.725716	-32.1976914726183\\
172.113695	-26.937146009604\\
180.94982	-28.3200690203738\\
184.014457	-28.7997087976468\\
128.060592	-20.042489151025\\
102.01021	-15.9653996228503\\
85.295304	-13.3493854616366\\
73.534968	-11.5088004462838\\
123.868688	-19.3864232284019\\
108.639163	-17.0028828681655\\
95.023572	-14.8719358637796\\
109.298359	-17.1060522231721\\
87.712224	-13.7276524376232\\
103.25778	-16.1606541332319\\
104.19773	-16.3077636958482\\
125.383584	-19.6235163588548\\
115.877423	-18.1357274478806\\
80.648568	-12.6221347562231\\
98.57512	-15.4277810394707\\
52.218802	-8.17265293107909\\
25.840836	-4.04429393223027\\
60.346098	-9.44463863224756\\
71.30268	-11.159429829428\\
42.5541	-6.66005110753848\\
12.290435	-1.92354967521061\\
45.37395	-7.10137979538742\\
66.613599	-10.4255517987003\\
68.176095	-10.6700947032695\\
88.628365	-13.8710357046107\\
77.694408	-12.1597854977534\\
107.284149	-16.7908125273187\\
97.32755	-15.232526529089\\
63.762744	-9.979370584665\\
96.06289	-15.0345972993871\\
79.934136	-12.5103205330845\\
51.681672	-8.08858786446055\\
37.147328	-5.81384879455788\\
31.521056	-4.9332930063985\\
37.550148	-5.87689329055564\\
45.37395	-7.10137979538742\\
34.09084	-5.33548439983262\\
75.975624	-11.8907822928255\\
94.81532	-14.8393427890054\\
60.950328	-9.53920537624421\\
80.1978	-12.551586021374\\
60.950328	-9.53920537624421\\
80.461464	-12.5928515096634\\
59.54412	-9.31912277203382\\
52.755932	-8.25671799769763\\
51.681672	-8.08858786446055\\
39.022272	-6.1072922668384\\
45.37395	-7.10137979538742\\
60.346098	-9.44463863224756\\
81.00344	-12.6776750133695\\
66.613599	-10.4255517987003\\
46.46771	-7.2725618318864\\
45.690115	-7.1508621028129\\
62.356536	-9.7592879804546\\
131.156837	-20.5270758287238\\
143.041626	-22.3871387167254\\
105.947446	-16.5816080018887\\
96.69522	-15.133561914238\\
77.936112	-12.1976140734463\\
114.803218	-17.9676059225758\\
97.32755	-15.232526529089\\
55.793158	-8.73206773803158\\
82.61472	-12.9298529973606\\
76.955868	-12.0441981831359\\
88.914007	-13.9157409226381\\
66.833325	-10.4599406446554\\
81.267104	-12.7189405016589\\
69.067206	-10.8095605785315\\
88.914007	-13.9157409226381\\
68.176095	-10.6700947032695\\
75.237084	-11.7751949782081\\
84.489664	-13.2232964696411\\
113.357924	-17.7414060521657\\
109.298359	-17.1060522231721\\
136.803898	-21.410885258776\\
189.425226	-29.6465366725422\\
146.630484	-22.9488232006576\\
85.558968	-13.3906509499261\\
87.433912	-13.6840944222066\\
97.340446	-15.2345448544462\\
140.132106	-21.9317759691076\\
111.312569	-17.4212919190255\\
138.021543	-21.6014562714596\\
95.75527	-14.9864523516217\\
46.630065	-7.29797166542922\\
56.1545	-8.78862060102951\\
113.729013	-17.799484397271\\
84.489664	-13.2232964696411\\
79.910028	-12.5065474416057\\
123.868688	-19.3864232284019\\
130.365809	-20.4032737295725\\
106.277044	-16.633192679392\\
85.558968	-13.3906509499261\\
109.298359	-17.1060522231721\\
122.727328	-19.2077914177867\\
108.291254	-16.9484323752454\\
93.563624	-14.6434425271951\\
91.818034	-14.3702439725833\\
70.409976	-11.0197146371456\\
82.878384	-12.97111848565\\
68.84748	-10.7751717325765\\
78.674652	-12.3132013880638\\
116.580539	-18.2457706280801\\
125.010048	-19.5650550390171\\
117.303186	-18.3588705632851\\
114.803218	-17.9676059225758\\
83.684024	-13.0972074776456\\
62.763018	-9.82290560823417\\
77.452704	-12.1219569220604\\
86.100944	-13.4754744536322\\
109.646268	-17.1605027160923\\
126.919232	-19.8638573404098\\
147.973254	-23.1589772592717\\
106.954551	-16.7392278498154\\
73.095516	-11.4400227543738\\
136.803898	-21.410885258776\\
187.197412	-29.2978662718501\\
126.524944	-19.80214816947\\
127.666304	-19.9807799800852\\
156.296595	-24.4616455437724\\
109.298359	-17.1060522231721\\
102.64254	-16.0643642377013\\
109.298359	-17.1060522231721\\
95.023572	-14.8719358637796\\
111.642167	-17.4728765965288\\
132.365352	-20.7162179246494\\
103.25778	-16.1606541332319\\
125.383584	-19.6235163588548\\
109.646268	-17.1605027160923\\
88.239552	-13.8101834142021\\
130.365809	-20.4032737295725\\
86.364608	-13.5167399419216\\
101.70259	-15.9172546750849\\
101.39497	-15.8691097273196\\
97.32755	-15.232526529089\\
59.137638	-9.25550514425425\\
31.630928	-4.95048883794676\\
25.484377	-3.9885052971107\\
62.158788	-9.72833886423752\\
123.495152	-19.3279619085642\\
131.969838	-20.6543168750737\\
114.803218	-17.9676059225758\\
78.929784	-12.3531315512953\\
100.45502	-15.7220001647033\\
85.016992	-13.30582744622\\
82.61472	-12.9298529973606\\
55.08024	-8.62049046779242\\
89.786802	-14.052340199944\\
82.878384	-12.97111848565\\
69.738591	-10.9146376078386\\
75.975624	-11.8907822928255\\
68.615547	-10.7388723951795\\
53.644638	-8.3958074715574\\
40.362564	-6.31705849897643\\
49.44137	-7.737962993618\\
105.28825	-16.4784386468822\\
77.694408	-12.1597854977534\\
105.947446	-16.5816080018887\\
93.277982	-14.5987373091677\\
129.948322	-20.3379337327982\\
118.006302	-18.4689137434847\\
153.96381	-24.0965463565519\\
95.309214	-14.916641081807\\
139.714632	-21.8664380069354\\
127.531292	-19.959649540947\\
79.668324	-12.4687188659127\\
100.45502	-15.7220001647033\\
86.100944	-13.4754744536322\\
115.877423	-18.1357274478806\\
125.010048	-19.5650550390171\\
145.287714	-22.7386691420435\\
102.64254	-16.0643642377013\\
150.658794	-23.5792853764999\\
157.24509	-24.6100924660488\\
140.132106	-21.9317759691076\\
108.968761	-17.0544675456688\\
87.712224	-13.7276524376232\\
116.580539	-18.2457706280801\\
100.45502	-15.7220001647033\\
85.016992	-13.30582744622\\
71.081361	-11.1247916664527\\
77.936112	-12.1976140734463\\
73.534968	-11.5088004462838\\
160.06494	-25.0514211538978\\
191.103716	-29.9092335497759\\
159.60351	-24.9792037322498\\
143.958951	-22.5307072890186\\
153.50238	-24.0243289349039\\
85.558968	-13.3906509499261\\
77.211	-12.0841283463674\\
55.256028	-8.64800267141303\\
55.08024	-8.62049046779242\\
71.081361	-11.1247916664527\\
109.298359	-17.1060522231721\\
118.006302	-18.4689137434847\\
139.714632	-21.8664380069354\\
107.632058	-16.8452630202389\\
95.594856	-14.9613462998344\\
120.877359	-18.9182567292993\\
175.811912	-27.5159460365525\\
174.267732	-27.2742697299395\\
149.061411	-23.3292823822334\\
243.210534	-38.0643601047073\\
156.78366	-24.5378750444008\\
93.563624	-14.6434425271951\\
77.452704	-12.1219569220604\\
72.424131	-11.3349457250667\\
119.080507	-18.6370352687895\\
145.675233	-22.7993189112818\\
173.762364	-27.1951756659571\\
148.412706	-23.2277549511817\\
106.624953	-16.6876431723122\\
62.56527	-9.79195649201708\\
83.684024	-13.0972074776456\\
63.1695	-9.88652323601374\\
75.478788	-11.8130235539011\\
48.1938	-7.54270848323635\\
61.35681	-9.60282300402378\\
46.16009	-7.22441688412106\\
40.362564	-6.31705849897643\\
55.08024	-8.62049046779242\\
77.452704	-12.1219569220604\\
91.818034	-14.3702439725833\\
98.88274	-15.475925987236\\
99.51507	-15.574890602087\\
93.563624	-14.6434425271951\\
114.080571	-17.8545059873707\\
72.192198	-11.2986463876698\\
133.178353	-20.8434589709993\\
138.879684	-21.7357620825912\\
124.262976	-19.4481323993417\\
108.639163	-17.0028828681655\\
109.298359	-17.1060522231721\\
119.451596	-18.6951136138948\\
148.412706	-23.2277549511817\\
116.580539	-18.2457706280801\\
95.023572	-14.8719358637796\\
109.298359	-17.1060522231721\\
99.51507	-15.574890602087\\
79.171488	-12.3909601269882\\
126.151408	-19.7436868496323\\
156.78366	-24.5378750444008\\
143.124003	-22.4000313646744\\
163.80765	-25.6371846850427\\
208.202003	-32.5852498507049\\
151.537698	-23.71684076032\\
141.407721	-22.1314197424113\\
119.803154	-18.7501352039945\\
146.510181	-22.929994835626\\
181.319133	-28.3778694075205\\
126.151408	-19.7436868496323\\
121.228917	-18.9732783193991\\
147.973254	-23.1589772592717\\
98.88274	-15.475925987236\\
63.564996	-9.94842146844791\\
90.659597	-14.18893947725\\
47.723825	-7.4691537019282\\
47.56147	-7.44374386838538\\
47.25385	-7.39559892062004\\
46.630065	-7.29797166542922\\
54.00598	-8.45236033455533\\
62.356536	-9.7592879804546\\
84.489664	-13.2232964696411\\
100.76264	-15.7701451124686\\
116.228981	-18.1907490379803\\
125.383584	-19.6235163588548\\
135.177896	-21.1564031660762\\
158.193585	-24.7585393883253\\
147.509388	-23.0863785844777\\
102.95016	-16.1125091854666\\
124.262976	-19.4481323993417\\
102.64254	-16.0643642377013\\
120.154712	-18.8051567940943\\
149.316024	-23.3691313178858\\
134.386868	-21.0326010669249\\
140.990247	-22.0660817802391\\
125.010048	-19.5650550390171\\
120.877359	-18.9182567292993\\
166.205595	-26.0124819243937\\
135.595383	-21.2217431628505\\
142.265862	-22.2657255535428\\
130.761323	-20.4651747791481\\
83.684024	-13.0972074776456\\
54.00598	-8.45236033455533\\
48.50142	-7.59085343100169\\
76.47246	-11.96854103175\\
70.629702	-11.0541034831006\\
98.57512	-15.4277810394707\\
85.295304	-13.3493854616366\\
112.319674	-17.5789117669522\\
162.397725	-25.4165203411182\\
192.232882	-30.0859569035513\\
137.594926	-21.5346873579273\\
167.19923	-26.1679936114518\\
140.990247	-22.0660817802391\\
110.635062	-17.3152567486021\\
119.080507	-18.6370352687895\\
118.006302	-18.4689137434847\\
111.312569	-17.4212919190255\\
128.434128	-20.1009504708627\\
163.80765	-25.6371846850427\\
212.420755	-33.2455176963421\\
187.20783	-29.2994967707313\\
188.291819	-29.4691495689343\\
192.232882	-30.0859569035513\\
145.675233	-22.7993189112818\\
159.116445	-24.9029742316213\\
120.154712	-18.8051567940943\\
114.663482	-17.9457361403089\\
154.66269	-24.2059265694582\\
175.957782	-27.5387758380301\\
57.580336	-9.01177514150782\\
114.803218	-17.9676059225758\\
72.643857	-11.3693345710218\\
109.646268	-17.1605027160923\\
106.954551	-16.7392278498154\\
64.773456	-10.1375549564412\\
74.658012	-11.6845656589429\\
149.919552	-23.463588193365\\
248.735466	-38.9290553863764\\
232.047	-36.3171792929702\\
220.93952	-34.5787713727942\\
175.067745	-27.3994780522092\\
210.359464	-32.9229094539516\\
238.732299	-37.3634811301416\\
181.988632	-28.4826512630043\\
198.367	-31.0459945821692\\
201.663252	-31.5618839273398\\
144.817092	-22.6650131001502\\
115.340989	-18.0517713107323\\
150.658794	-23.5792853764999\\
102.33492	-16.0162192899359\\
81.131976	-12.697791907609\\
108.968761	-17.0544675456688\\
72.424131	-11.3349457250667\\
101.70259	-15.9172546750849\\
87.712224	-13.7276524376232\\
112.319674	-17.5789117669522\\
126.151408	-19.7436868496323\\
94.722061	-14.8247470225284\\
73.095516	-11.4400227543738\\
95.880498	-15.0060515178618\\
112.319674	-17.5789117669522\\
126.151408	-19.7436868496323\\
109.975866	-17.2120873935955\\
112.997181	-17.6849469373756\\
151.098246	-23.6480630684099\\
140.572773	-22.000743818067\\
98.800394	-15.4630381910307\\
160.52637	-25.1236385755458\\
126.919232	-19.8638573404098\\
101.70259	-15.9172546750849\\
84.753328	-13.2645619579305\\
54.54311	-8.53642540117388\\
42.244832	-6.61164823482055\\
92.690829	-14.5068432498892\\
76.714164	-12.006369607443\\
61.148076	-9.5701544924613\\
40.633552	-6.35947025082947\\
70.19025	-10.9853257911906\\
86.100944	-13.4754744536322\\
95.309214	-14.916641081807\\
119.803154	-18.7501352039945\\
142.683336	-22.3310635157149\\
135.990897	-21.2836442124261\\
146.630484	-22.9488232006576\\
96.3876	-15.0854169664727\\
47.877635	-7.49322617581087\\
82.336408	-12.8862949819439\\
54.894686	-8.59144980841511\\
78.674652	-12.3132013880638\\
101.70259	-15.9172546750849\\
113.326779	-17.7365316148789\\
171.712816	-26.8744052953617\\
110.305464	-17.2636720710988\\
135.595383	-21.2217431628505\\
141.848388	-22.2003875913707\\
108.639163	-17.0028828681655\\
84.489664	-13.2232964696411\\
77.936112	-12.1976140734463\\
101.39497	-15.8691097273196\\
112.649272	-17.6304964444555\\
163.251545	-25.5501498817886\\
94.150777	-14.7353365864737\\
95.023572	-14.8719358637796\\
104.19773	-16.3077636958482\\
132.231744	-20.695307222546\\
211.234231	-33.0598173647566\\
179.40564	-28.0783927137608\\
174.801176	-27.3577579085876\\
116.951628	-18.3038489731854\\
80.151732	-12.5443760172986\\
90.659597	-14.18893947725\\
55.793158	-8.73206773803158\\
34.982024	-5.47496170016844\\
97.94279	-15.3288164246197\\
102.01021	-15.9653996228503\\
121.228917	-18.9732783193991\\
162.397725	-25.4165203411182\\
183.027444	-28.6452335056349\\
230.7726	-36.1177256767159\\
198.172072	-31.0154868181161\\
111.642167	-17.4728765965288\\
110.635062	-17.3152567486021\\
81.870516	-12.8133792222265\\
113.656377	-17.7881162923822\\
156.469326	-24.4886793028662\\
186.153138	-29.1344291832905\\
142.683336	-22.3310635157149\\
112.997181	-17.6849469373756\\
129.949024	-20.3380436013156\\
158.193585	-24.7585393883253\\
109.975866	-17.2120873935955\\
78.674652	-12.3132013880638\\
69.067206	-10.8095605785315\\
40.633552	-6.35947025082947\\
46.3139	-7.24848935800373\\
26.910213	-4.21165983758901\\
69.738591	-10.9146376078386\\
91.818034	-14.3702439725833\\
90.945239	-14.2336446952774\\
69.067206	-10.8095605785315\\
72.282924	-11.3128457197384\\
37.015496	-5.79321605041315\\
18.372254	-2.87540214928005\\
38.223956	-5.98234953840646\\
60.14835	-9.41368951603047\\
69.067206	-10.8095605785315\\
84.489664	-13.2232964696411\\
108.291254	-16.9484323752454\\
126.524944	-19.80214816947\\
156.469326	-24.4886793028662\\
221.55476	-34.6750612683249\\
215.452983	-33.7200852103938\\
239.391477	-37.4666476680067\\
198.743356	-31.1048972541709\\
126.919232	-19.8638573404098\\
111.990076	-17.5273270894489\\
113.326779	-17.7365316148789\\
135.990897	-21.2836442124261\\
103.25778	-16.1606541332319\\
93.849266	-14.6881477452225\\
70.19025	-10.9853257911906\\
71.752746	-11.2298686957597\\
101.39497	-15.8691097273196\\
83.42036	-13.0559419893561\\
46.937685	-7.34611661319456\\
40.765384	-6.3801029949742\\
62.763018	-9.82290560823417\\
82.878384	-12.97111848565\\
77.211	-12.0841283463674\\
83.947688	-13.138472965935\\
71.301087	-11.1591805124077\\
96.753293	-15.1426507951677\\
140.572773	-22.000743818067\\
105.46239	-16.5056929255502\\
156.469326	-24.4886793028662\\
214.859721	-33.6272350446011\\
228.8256	-35.8130048741051\\
167.68262	-26.2436479456963\\
118.377391	-18.52699208859\\
85.295304	-13.3493854616366\\
57.043206	-8.92771007488928\\
91.532392	-14.3255387545559\\
64.971204	-10.1685040726583\\
102.01021	-15.9653996228503\\
85.822632	-13.4319164382155\\
72.643857	-11.3693345710218\\
98.57512	-15.4277810394707\\
71.520813	-11.1935693583627\\
70.849428	-11.0884923290557\\
74.99538	-11.7373664025151\\
54.357556	-8.50738474179656\\
53.46885	-8.36829526793679\\
46.630065	-7.29797166542922\\
46.3139	-7.24848935800373\\
38.619452	-6.04424777084063\\
32.302368	-5.05557447518607\\
32.857832	-5.14250895690224\\
52.755932	-8.25671799769763\\
59.346372	-9.28817365581673\\
47.723825	-7.4691537019282\\
92.103676	-14.4149491906107\\
98.57512	-15.4277810394707\\
83.947688	-13.138472965935\\
86.628272	-13.5580054302111\\
126.524944	-19.80214816947\\
155.835165	-24.3894281221244\\
110.982971	-17.3697072415222\\
127.135778	-19.8977484913714\\
53.644638	-8.3958074715574\\
34.310584	-5.36987606292912\\
86.100944	-13.4754744536322\\
108.968761	-17.0544675456688\\
120.154712	-18.8051567940943\\
166.205595	-26.0124819243937\\
155.373735	-24.3172107004764\\
100.45502	-15.7220001647033\\
78.191244	-12.2375442366778\\
92.976471	-14.5515484679166\\
84.753328	-13.2645619579305\\
61.148076	-9.5701544924613\\
40.50172	-6.33883750668475\\
67.053051	-10.4943294906104\\
39.69608	-6.21274851468921\\
32.07652	-5.02022748811467\\
39.425092	-6.17033676283617\\
54.54311	-8.53642540117388\\
75.73392	-11.8529537171326\\
74.740248	-11.6974362392837\\
56.1545	-8.78862060102951\\
102.64254	-16.0643642377013\\
136.745928	-21.4018124981559\\
87.170248	-13.6428289339172\\
125.777872	-19.6852255297946\\
138.439017	-21.6667942336317\\
90.373955	-14.1442342592226\\
62.356536	-9.7592879804546\\
84.753328	-13.2645619579305\\
82.072744	-12.8450294936545\\
54.181768	-8.47987253817595\\
67.272777	-10.5287183365654\\
38.758608	-6.06602677854895\\
31.7408	-4.96768466949501\\
32.638088	-5.10811729380573\\
46.783875	-7.32204413931188\\
66.833325	-10.4599406446554\\
60.75258	-9.50825626002713\\
68.615547	-10.7388723951795\\
81.267104	-12.7189405016589\\
52.033248	-8.14361227170177\\
38.0848	-5.96057053069814\\
31.7408	-4.96768466949501\\
44.903975	-7.02782501407926\\
45.52776	-7.12545226927009\\
55.431816	-8.67551487503365\\
106.624953	-16.6876431723122\\
105.947446	-16.5816080018887\\
84.489664	-13.2232964696411\\
86.100944	-13.4754744536322\\
118.006302	-18.4689137434847\\
150.658794	-23.5792853764999\\
159.60351	-24.9792037322498\\
164.24518	-25.7056615688467\\
133.969381	-20.9672610701506\\
134.782382	-21.0945021165005\\
134.386868	-21.0326010669249\\
143.124003	-22.4000313646744\\
162.397725	-25.4165203411182\\
193.33153	-30.257903950468\\
169.159645	-26.4748139669989\\
151.537698	-23.71684076032\\
127.666304	-19.9807799800852\\
119.803154	-18.7501352039945\\
120.877359	-18.9182567292993\\
147.069936	-23.0176008925677\\
95.023572	-14.8719358637796\\
118.728949	-18.5820136786897\\
135.177896	-21.1564031660762\\
148.852158	-23.2965326430918\\
104.83006	-16.4067283106992\\
126.151408	-19.7436868496323\\
109.975866	-17.2120873935955\\
99.51507	-15.574890602087\\
71.752746	-11.2298686957597\\
104.52244	-16.3585833629339\\
148.852158	-23.2965326430918\\
114.803218	-17.9676059225758\\
64.575708	-10.1066058402241\\
82.878384	-12.97111848565\\
67.053051	-10.4943294906104\\
54.357556	-8.50738474179656\\
60.346098	-9.44463863224756\\
39.564248	-6.19211577054449\\
52.93172	-8.28423020131825\\
45.52776	-7.12545226927009\\
33.08368	-5.17785594397364\\
39.425092	-6.17033676283617\\
51.144542	-8.004522797842\\
39.022272	-6.1072922668384\\
45.997735	-7.19900705057824\\
53.820426	-8.42331967517802\\
60.346098	-9.44463863224756\\
74.25684	-11.6217790878977\\
68.615547	-10.7388723951795\\
89.50116	-14.0076349819166\\
76.217328	-11.9286108685185\\
87.712224	-13.7276524376232\\
145.287714	-22.7386691420435\\
100.45502	-15.7220001647033\\
93.849266	-14.6881477452225\\
106.624953	-16.6876431723122\\
74.740248	-11.6974362392837\\
48.34761	-7.56678095711902\\
110.15484	-17.2400982312568\\
59.950602	-9.38274039981338\\
47.56147	-7.44374386838538\\
61.752306	-9.66472123645795\\
76.217328	-11.9286108685185\\
67.944162	-10.6337953658725\\
58.93989	-9.22455602803717\\
31.7408	-4.96768466949501\\
31.630928	-4.95048883794676\\
32.74796	-5.12531312535398\\
66.16194	-10.3548636153483\\
59.950602	-9.38274039981338\\
68.615547	-10.7388723951795\\
92.690829	-14.5068432498892\\
113.729013	-17.799484397271\\
93.849266	-14.6881477452225\\
125.777872	-19.6852255297946\\
147.509388	-23.0863785844777\\
124.636512	-19.5065937191794\\
129.949024	-20.3380436013156\\
198.172072	-31.0154868181161\\
146.092707	-22.8646568734539\\
191.683558	-29.999983380093\\
154.223238	-24.1371488775481\\
171.60345	-26.8572886567904\\
188.875902	-29.5605631490839\\
122.353792	-19.149330097949\\
63.1695	-9.88652323601374\\
71.301087	-11.1591805124077\\
94.722061	-14.8247470225284\\
109.646268	-17.1605027160923\\
107.632058	-16.8452630202389\\
88.628365	-13.8710357046107\\
54.181768	-8.47987253817595\\
63.971478	-10.0120390962275\\
110.982971	-17.3697072415222\\
149.755476	-23.4379090097958\\
137.604069	-21.5361183092875\\
98.57512	-15.4277810394707\\
71.520813	-11.1935693583627\\
89.045192	-13.9362724061977\\
151.537698	-23.71684076032\\
160.52637	-25.1236385755458\\
164.72857	-25.7813159030911\\
128.807664	-20.1594117907004\\
169.643035	-26.5504683012433\\
185.069149	-28.9647763850876\\
174.07411	-27.2439663651511\\
230.1	-36.0124584903594\\
203.980126	-31.9244929180065\\
174.07411	-27.2439663651511\\
204.55141	-32.0139033540613\\
165.21196	-25.8569702373356\\
102.64254	-16.0643642377013\\
89.045192	-13.9362724061977\\
112.997181	-17.6849469373756\\
128.060592	-20.042489151025\\
133.573867	-20.9053600205749\\
112.319674	-17.5789117669522\\
127.292768	-19.9223186602475\\
109.975866	-17.2120873935955\\
88.781528	-13.8950069179082\\
103.58249	-16.2114738003176\\
106.07763	-16.6019828210809\\
161.013435	-25.1998680761742\\
150.658794	-23.5792853764999\\
125.010048	-19.5650550390171\\
84.489664	-13.2232964696411\\
70.849428	-11.0884923290557\\
77.452704	-12.1219569220604\\
83.142048	-13.0123839739395\\
54.894686	-8.59144980841511\\
48.971395	-7.66440821230985\\
86.364608	-13.5167399419216\\
102.01021	-15.9653996228503\\
124.636512	-19.5065937191794\\
113.006366	-17.6863844620659\\
69.067206	-10.8095605785315\\
47.091495	-7.37018908707722\\
53.644638	-8.3958074715574\\
48.34761	-7.56678095711902\\
76.955868	-12.0441981831359\\
83.947688	-13.138472965935\\
79.910028	-12.5065474416057\\
119.080507	-18.6370352687895\\
158.655015	-24.8307568099733\\
152.880468	-23.9269948189341\\
170.63667	-26.7059799883014\\
189.425226	-29.6465366725422\\
103.25778	-16.1606541332319\\
101.39497	-15.8691097273196\\
86.364608	-13.5167399419216\\
95.023572	-14.8719358637796\\
109.298359	-17.1060522231721\\
92.103676	-14.4149491906107\\
62.763018	-9.82290560823417\\
57.404548	-8.98426293788721\\
102.95016	-16.1125091854666\\
125.010048	-19.5650550390171\\
114.004286	-17.8425667853023\\
170.63667	-26.7059799883014\\
198.172072	-31.0154868181161\\
132.760866	-20.778118974225\\
85.822632	-13.4319164382155\\
77.211	-12.0841283463674\\
65.179938	-10.2011725842208\\
108.639163	-17.0028828681655\\
102.01021	-15.9653996228503\\
95.880498	-15.0060515178618\\
111.312569	-17.4212919190255\\
141.407721	-22.1314197424113\\
127.292768	-19.9223186602475\\
147.069936	-23.0176008925677\\
102.95016	-16.1125091854666\\
108.639163	-17.0028828681655\\
85.558968	-13.3906509499261\\
72.863583	-11.4037234169768\\
105.13768	-16.4548732584646\\
152.441016	-23.858217127024\\
154.88667	-24.2409811998479\\
124.451532	-19.4776428953135\\
228.8256	-35.8130048741051\\
168.16601	-26.3193022799408\\
124.262976	-19.4481323993417\\
80.648568	-12.6221347562231\\
102.95016	-16.1125091854666\\
102.01021	-15.9653996228503\\
95.594856	-14.9613462998344\\
100.76264	-15.7701451124686\\
87.170248	-13.6428289339172\\
84.086136	-13.1601411660787\\
179.40564	-28.0783927137608\\
190.554392	-29.8232600263176\\
114.004286	-17.8425667853023\\
145.234566	-22.7303510623223\\
153.344334	-23.999593493728\\
150.658794	-23.5792853764999\\
111.642167	-17.4728765965288\\
108.968761	-17.0544675456688\\
95.594856	-14.9613462998344\\
111.312569	-17.4212919190255\\
138.803441	-21.7238294538529\\
181.483264	-28.4035571990219\\
219.078473	-34.2875028902383\\
282.41225	-44.199736767901\\
228.8256	-35.8130048741051\\
154.223238	-24.1371488775481\\
138.407927	-21.6619284042773\\
136.386411	-21.3455452620017\\
109.646268	-17.1605027160923\\
79.413192	-12.4287887026812\\
71.972472	-11.2642575417147\\
77.936112	-12.1976140734463\\
47.877635	-7.49322617581087\\
48.50142	-7.59085343100169\\
56.681864	-8.87115721189135\\
83.684024	-13.0972074776456\\
67.944162	-10.6337953658725\\
47.091495	-7.37018908707722\\
42.376664	-6.63228097896527\\
88.239552	-13.8101834142021\\
134.612172	-21.0678629137207\\
70.409976	-11.0197146371456\\
70.849428	-11.0884923290557\\
98.88274	-15.475925987236\\
85.558968	-13.3906509499261\\
99.82269	-15.6230355498523\\
90.659597	-14.18893947725\\
60.950328	-9.53920537624421\\
53.107508	-8.31174240493886\\
33.4194	-5.2303987625933\\
48.817585	-7.64033573842718\\
92.976471	-14.5515484679166\\
105.617848	-16.5300233243855\\
75.478788	-11.8130235539011\\
53.107508	-8.31174240493886\\
39.564248	-6.19211577054449\\
68.84748	-10.7751717325765\\
77.936112	-12.1976140734463\\
107.632058	-16.8452630202389\\
105.617848	-16.5300233243855\\
83.947688	-13.138472965935\\
69.958317	-10.9490264537936\\
78.929784	-12.3531315512953\\
111.312569	-17.4212919190255\\
160.52637	-25.1236385755458\\
173.762364	-27.1951756659571\\
138.879684	-21.7357620825912\\
86.906584	-13.6015634456277\\
86.628272	-13.5580054302111\\
93.563624	-14.6434425271951\\
84.226	-13.1820309813516\\
69.958317	-10.9490264537936\\
70.629702	-11.0541034831006\\
84.753328	-13.2645619579305\\
82.61472	-12.9298529973606\\
59.950602	-9.38274039981338\\
46.937685	-7.34611661319456\\
48.1938	-7.54270848323635\\
78.929784	-12.3531315512953\\
102.64254	-16.0643642377013\\
134.386868	-21.0326010669249\\
141.848388	-22.2003875913707\\
142.683336	-22.3310635157149\\
146.927655	-22.9953327977981\\
201.174656	-31.4854148131784\\
265.545536	-41.5599634615395\\
219.05962	-34.2845522475615\\
148.20327	-23.1949765711018\\
174.07411	-27.2439663651511\\
193.65317	-30.3082431384247\\
236.718144	-37.0482500422205\\
170.168636	-26.6327289887487\\
84.753328	-13.2645619579305\\
55.61737	-8.70455553441096\\
67.724436	-10.5994065199174\\
51.681672	-8.08858786446055\\
32.186392	-5.03742331966292\\
39.29326	-6.14970401869144\\
54.181768	-8.47987253817595\\
68.176095	-10.6700947032695\\
67.50471	-10.5650176739624\\
60.346098	-9.44463863224756\\
54.181768	-8.47987253817595\\
61.96104	-9.69738974802043\\
83.684024	-13.0972074776456\\
74.99538	-11.7373664025151\\
59.950602	-9.38274039981338\\
39.425092	-6.17033676283617\\
40.50172	-6.33883750668475\\
61.752306	-9.66472123645795\\
67.724436	-10.5994065199174\\
52.570378	-8.22767733832032\\
40.230732	-6.29642575483171\\
59.741868	-9.35007188825091\\
73.021464	-11.4284330343558\\
54.00598	-8.45236033455533\\
76.955868	-12.0441981831359\\
85.295304	-13.3493854616366\\
111.229045	-17.4082197565616\\
78.191244	-12.2375442366778\\
93.849266	-14.6881477452225\\
99.51507	-15.574890602087\\
90.945239	-14.2336446952774\\
78.929784	-12.3531315512953\\
110.982971	-17.3697072415222\\
156.78366	-24.5378750444008\\
143.124003	-22.4000313646744\\
156.296595	-24.4616455437724\\
132.760866	-20.778118974225\\
117.303186	-18.3588705632851\\
111.642167	-17.4728765965288\\
133.178353	-20.8434589709993\\
120.154712	-18.8051567940943\\
128.434128	-20.1009504708627\\
143.541477	-22.4653693268465\\
151.195167	-23.6632319666687\\
263.329971	-41.2132101256195\\
212.80468	-33.305604976333\\
102.01021	-15.9653996228503\\
71.972472	-11.2642575417147\\
87.170248	-13.6428289339172\\
103.58249	-16.2114738003176\\
127.666304	-19.9807799800852\\
144.817092	-22.6650131001502\\
152.880468	-23.9269948189341\\
164.24518	-25.7056615688467\\
118.006302	-18.4689137434847\\
96.182009	-15.0532403591129\\
118.728949	-18.5820136786897\\
121.585968	-19.0291596071715\\
69.286932	-10.8439494244866\\
63.762744	-9.979370584665\\
78.674652	-12.3132013880638\\
79.910028	-12.5065474416057\\
104.52244	-16.3585833629339\\
141.848388	-22.2003875913707\\
131.574324	-20.5924158254981\\
86.100944	-13.4754744536322\\
73.095516	-11.4400227543738\\
111.990076	-17.5273270894489\\
146.510181	-22.929994835626\\
173.080475	-27.088454678093\\
208.04259	-32.560300463285\\
238.732299	-37.3634811301416\\
197.268352	-30.8740475352526\\
181.846479	-28.4604032012409\\
116.228981	-18.1907490379803\\
82.125648	-12.8533093854579\\
138.803441	-21.7238294538529\\
171.712816	-26.8744052953617\\
114.333884	-17.8941514628056\\
134.140928	-20.9941095239387\\
233.994	-36.6219000955809\\
242.529378	-37.9577538371043\\
156.78366	-24.5378750444008\\
99.51507	-15.574890602087\\
57.043206	-8.92771007488928\\
71.301087	-11.1591805124077\\
85.558968	-13.3906509499261\\
69.286932	-10.8439494244866\\
62.356536	-9.7592879804546\\
61.554558	-9.63377212024086\\
69.067206	-10.8095605785315\\
47.877635	-7.49322617581087\\
68.84748	-10.7751717325765\\
92.976471	-14.5515484679166\\
107.961656	-16.8968476977421\\
102.95016	-16.1125091854666\\
120.877359	-18.9182567292993\\
175.306544	-27.4368519725701\\
147.069936	-23.0176008925677\\
101.70259	-15.9172546750849\\
95.309214	-14.916641081807\\
115.154776	-18.0226275126755\\
85.295304	-13.3493854616366\\
62.356536	-9.7592879804546\\
63.564996	-9.94842146844791\\
77.694408	-12.1597854977534\\
68.395821	-10.7044835492245\\
52.218802	-8.17265293107909\\
31.966648	-5.00303165656641\\
32.857832	-5.14250895690224\\
47.56147	-7.44374386838538\\
82.61472	-12.9298529973606\\
74.001708	-11.5818489246662\\
60.75258	-9.50825626002713\\
48.663775	-7.61626326454451\\
92.976471	-14.5515484679166\\
106.624953	-16.6876431723122\\
62.960766	-9.85385472445126\\
77.694408	-12.1597854977534\\
93.277982	-14.5987373091677\\
100.13031	-15.6711804976177\\
100.76264	-15.7701451124686\\
115.506334	-18.0776491027753\\
113.006366	-17.6863844620659\\
76.217328	-11.9286108685185\\
55.968946	-8.75957994165219\\
83.684024	-13.0972074776456\\
62.960766	-9.85385472445126\\
78.674652	-12.3132013880638\\
119.803154	-18.7501352039945\\
156.296595	-24.4616455437724\\
132.760866	-20.778118974225\\
113.656377	-17.7881162923822\\
165.722205	-25.9368275901492\\
145.727166	-22.8074468339536\\
84.489664	-13.2232964696411\\
61.752306	-9.66472123645795\\
61.35681	-9.60282300402378\\
53.820426	-8.42331967517802\\
46.783875	-7.32204413931188\\
41.842012	-6.54860373882278\\
80.461464	-12.5928515096634\\
71.799516	-11.2371885683525\\
46.630065	-7.29797166542922\\
68.615547	-10.7388723951795\\
75.237084	-11.7751949782081\\
53.293062	-8.34078306431618\\
44.903975	-7.02782501407926\\
26.109401	-4.08632646553955\\
41.30736	-6.46492649868029\\
85.822632	-13.4319164382155\\
114.080571	-17.8545059873707\\
97.94279	-15.3288164246197\\
70.19025	-10.9853257911906\\
84.489664	-13.2232964696411\\
85.295304	-13.3493854616366\\
93.849266	-14.6881477452225\\
125.383584	-19.6235163588548\\
140.990247	-22.0660817802391\\
139.714632	-21.8664380069354\\
114.803218	-17.9676059225758\\
63.1695	-9.88652323601374\\
62.763018	-9.82290560823417\\
71.081361	-11.1247916664527\\
90.072444	-14.0970454179714\\
81.530768	-12.7602059899484\\
49.28756	-7.71389051973533\\
101.07026	-15.818290060234\\
122.727328	-19.2077914177867\\
109.975866	-17.2120873935955\\
128.434128	-20.1009504708627\\
153.015315	-23.9480994342754\\
119.080507	-18.6370352687895\\
114.663482	-17.9457361403089\\
165.722205	-25.9368275901492\\
136.803898	-21.410885258776\\
152.001564	-23.7894394351139\\
167.575995	-26.2269602951681\\
252.21693	-39.4739318652364\\
252.503232	-39.5187403784511\\
166.6275	-26.0785133728916\\
217.85899	-34.0966442161088\\
242.723988	-37.988211831658\\
237.2508	-37.1316148926752\\
253.617084	-39.6930669708649\\
249.462252	-39.0428031076123\\
298.518385	-46.7204734829276\\
273.594368	-42.8196689300136\\
154.223238	-24.1371488775481\\
121.228917	-18.9732783193991\\
111.990076	-17.5273270894489\\
94.722061	-14.8247470225284\\
74.426079	-11.6482663215459\\
107.01758	-16.7490923836972\\
142.265862	-22.2657255535428\\
110.305464	-17.2636720710988\\
80.406864	-12.5843061805301\\
92.405187	-14.4621380318618\\
70.629702	-11.0541034831006\\
61.148076	-9.5701544924613\\
59.950602	-9.38274039981338\\
39.564248	-6.19211577054449\\
46.630065	-7.29797166542922\\
86.00998	-13.4612378726929\\
45.37395	-7.10137979538742\\
44.12638	-6.90612528500577\\
30.73364	-4.81005621363603\\
18.910568	-2.95965252120434\\
37.418316	-5.85626054641092\\
43.49405	-6.80716067015479\\
32.857832	-5.14250895690224\\
69.286932	-10.8439494244866\\
84.226	-13.1820309813516\\
83.142048	-13.0123839739395\\
69.958317	-10.9490264537936\\
95.309214	-14.916641081807\\
127.531292	-19.959649540947\\
94.18299	-14.7403781741544\\
32.522112	-5.08996613828258\\
56.929452	-8.90990667983012\\
31.966648	-5.00303165656641\\
40.0989	-6.27579301068698\\
76.714164	-12.006369607443\\
89.199649	-13.9604461406655\\
91.532392	-14.3255387545559\\
73.754694	-11.5431892922389\\
171.179372	-26.7909171167136\\
139.714632	-21.8664380069354\\
98.57512	-15.4277810394707\\
77.211	-12.0841283463674\\
86.906584	-13.6015634456277\\
125.777872	-19.6852255297946\\
136.745928	-21.4018124981559\\
102.266935	-16.0055791031021\\
34.310584	-5.36987606292912\\
68.176095	-10.6700947032695\\
51.681672	-8.08858786446055\\
30.849616	-4.82820736915918\\
19.0424	-2.98028526534907\\
41.571024	-6.50619198696974\\
113.729013	-17.799484397271\\
111.932161	-17.5182629367611\\
60.75258	-9.50825626002713\\
46.630065	-7.29797166542922\\
43.34024	-6.78308819627213\\
48.996022	-7.66826253136783\\
30.959488	-4.84540320070744\\
19.0424	-2.98028526534907\\
53.107508	-8.31174240493886\\
83.42036	-13.0559419893561\\
106.954551	-16.7392278498154\\
122.353792	-19.149330097949\\
116.228981	-18.1907490379803\\
123.868688	-19.3864232284019\\
115.506334	-18.0776491027753\\
109.298359	-17.1060522231721\\
136.803898	-21.410885258776\\
191.683558	-29.999983380093\\
161.77452	-25.318983860486\\
96.3876	-15.0854169664727\\
50.22751	-7.86100008235165\\
73.754694	-11.5431892922389\\
162.25791	-25.3946381947305\\
127.948779	-20.0249895377213\\
76.714164	-12.006369607443\\
94.150777	-14.7353365864737\\
129.201952	-20.2211209616402\\
185.596495	-29.0473101788079\\
187.764473	-29.3866157752139\\
205.154432	-32.1082810365635\\
186.153138	-29.1344291832905\\
169.643035	-26.5504683012433\\
146.191032	-22.8800455087475\\
98.57512	-15.4277810394707\\
78.674652	-12.3132013880638\\
94.436419	-14.780041804501\\
117.303186	-18.3588705632851\\
102.95016	-16.1125091854666\\
114.803218	-17.9676059225758\\
79.413192	-12.4287887026812\\
108.639163	-17.0028828681655\\
97.63517	-15.2806714768544\\
39.967068	-6.25516026654226\\
25.748059	-4.02977360254162\\
26.109401	-4.08632646553955\\
38.89044	-6.08665952269367\\
43.810215	-6.85664297758028\\
19.177894	-3.00149114127559\\
40.90454	-6.40188200268252\\
95.12294	-14.8874877367708\\
20.04945	-3.13789650534349\\
61.148076	-9.5701544924613\\
85.822632	-13.4319164382155\\
121.212432	-18.9706982873338\\
101.25983	-15.8479592551754\\
38.355788	-6.00298228255118\\
39.967068	-6.25516026654226\\
43.182304	-6.75836997096081\\
150.194928	-23.5066867017059\\
173.080475	-27.088454678093\\
228.42494	-35.7502984350841\\
288.692825	-45.18269611811\\
264.780159	-41.4401759454945\\
190.4305	-29.803869954578\\
194.180516	-30.3907769321451\\
242.06481	-37.8850452937929\\
216.046245	-33.8129353761866\\
193.125824	-30.2257093447043\\
202.273304	-31.657361860095\\
235.3038	-36.8268940900645\\
242.4192	-37.940510114151\\
312.555606	-48.9174088961501\\
235.363167	-36.8361854921644\\
110.982971	-17.3697072415222\\
63.564996	-9.94842146844791\\
56.330288	-8.81613280465012\\
63.1695	-9.88652323601374\\
69.958317	-10.9490264537936\\
65.179938	-10.2011725842208\\
101.07026	-15.818290060234\\
107.961656	-16.8968476977421\\
88.239552	-13.8101834142021\\
113.729013	-17.799484397271\\
70.409976	-11.0197146371456\\
101.07026	-15.818290060234\\
109.975866	-17.2120873935955\\
114.080571	-17.8545059873707\\
61.752306	-9.66472123645795\\
34.42656	-5.38802721845228\\
61.96104	-9.69738974802043\\
80.890272	-12.6599633319161\\
108.58986	-16.9951665611645\\
195.01002	-30.5206008277017\\
196.68851	-30.7832977049355\\
216.639507	-33.9057855419793\\
240.4722	-37.6357893115403\\
308.48378	-48.2801361242902\\
265.545536	-41.5599634615395\\
169.643035	-26.5504683012433\\
102.01021	-15.9653996228503\\
78.191244	-12.2375442366778\\
78.432948	-12.2753728123708\\
87.170248	-13.6428289339172\\
93.563624	-14.6434425271951\\
49.44137	-7.737962993618\\
71.081361	-11.1247916664527\\
71.752746	-11.2298686957597\\
86.628272	-13.5580054302111\\
110.305464	-17.2636720710988\\
109.298359	-17.1060522231721\\
89.786802	-14.052340199944\\
39.69608	-6.21274851468921\\
25.660165	-4.01601750073131\\
38.0848	-5.96057053069814\\
25.572271	-4.002261398921\\
33.193552	-5.19505177552189\\
60.543846	-9.47558774846464\\
82.072744	-12.8450294936545\\
92.690829	-14.5068432498892\\
113.357924	-17.7414060521657\\
102.33492	-16.0162192899359\\
124.099974	-19.4226213052138\\
217.17972	-33.9903331223289\\
168.676255	-26.3991596327544\\
139.198955	-21.7857305034286\\
180.94982	-28.3200690203738\\
199.346378	-31.1992749366732\\
128.060592	-20.042489151025\\
108.968761	-17.0544675456688\\
78.929784	-12.3531315512953\\
98.498883	-15.4158493497795\\
186.087728	-29.1241920041951\\
284.966464	-44.5994913339614\\
316.82321	-49.5853224637437\\
246.752298	-38.6186740075002\\
208.613874	-32.6497108993397\\
175.067745	-27.3994780522092\\
186.621172	-29.2076801828432\\
219.67486	-34.3808421430922\\
144.817092	-22.6650131001502\\
112.319674	-17.5789117669522\\
116.000185	-18.1549406657389\\
156.908778	-24.5574569947763\\
157.70652	-24.6823098876968\\
88.239552	-13.8101834142021\\
86.100944	-13.4754744536322\\
80.890272	-12.6599633319161\\
107.01758	-16.7490923836972\\
169.643035	-26.5504683012433\\
159.60351	-24.9792037322498\\
126.151408	-19.7436868496323\\
111.312569	-17.4212919190255\\
102.95016	-16.1125091854666\\
85.822632	-13.4319164382155\\
48.817585	-7.64033573842718\\
47.877635	-7.49322617581087\\
39.967068	-6.25516026654226\\
39.69608	-6.21274851468921\\
34.310584	-5.36987606292912\\
63.971478	-10.0120390962275\\
100.45502	-15.7220001647033\\
104.940341	-16.423988153962\\
75.73392	-11.8529537171326\\
54.181768	-8.47987253817595\\
47.877635	-7.49322617581087\\
62.960766	-9.85385472445126\\
84.753328	-13.2645619579305\\
98.88274	-15.475925987236\\
82.072744	-12.8450294936545\\
56.506076	-8.84364500827073\\
125.777872	-19.6852255297946\\
131.156837	-20.5270758287238\\
88.914007	-13.9157409226381\\
26.734425	-4.18414763396839\\
62.356536	-9.7592879804546\\
77.211	-12.0841283463674\\
83.684024	-13.0972074776456\\
54.357556	-8.50738474179656\\
41.71018	-6.52797099467806\\
79.668324	-12.4687188659127\\
127.948779	-20.0249895377213\\
73.021464	-11.4284330343558\\
19.64663	-3.07485200934572\\
55.431816	-8.67551487503365\\
86.628272	-13.5580054302111\\
119.324	-18.6751438370432\\
82.336408	-12.8862949819439\\
71.520813	-11.1935693583627\\
126.524944	-19.80214816947\\
164.24518	-25.7056615688467\\
145.675233	-22.7993189112818\\
189.903154	-29.7213361608576\\
224.259	-35.0982960825272\\
151.098246	-23.6480630684099\\
123.025769	-19.254499779909\\
166.688985	-26.0881362586382\\
122.303122	-19.1413998447039\\
148.643937	-23.2639444200613\\
186.153138	-29.1344291832905\\
132.365352	-20.7162179246494\\
66.981642	-10.4831534208653\\
133.74664	-20.9324003529989\\
226.54504	-35.4560793098515\\
248.05431	-38.8224491187734\\
188.819165	-29.5516833626546\\
159.154866	-24.9089874200944\\
204.55141	-32.0139033540613\\
167.19923	-26.1679936114518\\
104.52244	-16.3585833629339\\
105.13768	-16.4548732584646\\
121.228917	-18.9732783193991\\
136.386411	-21.3455452620017\\
121.951564	-19.0863782546041\\
135.595383	-21.2217431628505\\
89.587168	-14.0210959099038\\
122.303122	-19.1413998447039\\
149.316024	-23.3691313178858\\
108.639163	-17.0028828681655\\
85.558968	-13.3906509499261\\
69.286932	-10.8439494244866\\
48.1938	-7.54270848323635\\
56.867418	-8.90019787126866\\
100.76264	-15.7701451124686\\
95.023572	-14.8719358637796\\
120.154712	-18.8051567940943\\
161.474865	-25.2720854978222\\
167.68262	-26.2436479456963\\
127.292768	-19.9223186602475\\
106.624953	-16.6876431723122\\
63.762744	-9.979370584665\\
89.587168	-14.0210959099038\\
147.509388	-23.0863785844777\\
105.617848	-16.5300233243855\\
50.38132	-7.88507255623432\\
97.626088	-15.2792500724736\\
160.06494	-25.0514211538978\\
149.061411	-23.3292823822334\\
203.40247	-31.8340852138704\\
257.503296	-40.3012896928759\\
196.68851	-30.7832977049355\\
175.067745	-27.3994780522092\\
186.153138	-29.1344291832905\\
128.434128	-20.1009504708627\\
90.656472	-14.1884503901888\\
107.34229	-16.7999120507828\\
154.66269	-24.2059265694582\\
158.655015	-24.8307568099733\\
80.890272	-12.6599633319161\\
86.628272	-13.5580054302111\\
72.863583	-11.4037234169768\\
102.01021	-15.9653996228503\\
86.100944	-13.4754744536322\\
74.658012	-11.6845656589429\\
129.575488	-20.2795822814779\\
148.852158	-23.2965326430918\\
95.309214	-14.916641081807\\
74.426079	-11.6482663215459\\
123.495152	-19.3279619085642\\
75.73392	-11.8529537171326\\
54.894686	-8.59144980841511\\
62.960766	-9.85385472445126\\
69.738591	-10.9146376078386\\
61.148076	-9.5701544924613\\
32.522112	-5.08996613828258\\
32.74796	-5.12531312535398\\
61.148076	-9.5701544924613\\
71.081361	-11.1247916664527\\
124.262976	-19.4481323993417\\
120.525801	-18.8632351391996\\
139.198955	-21.7857305034286\\
172.218184	-26.9534993593441\\
86.628272	-13.5580054302111\\
90.656472	-14.1884503901888\\
166.6275	-26.0785133728916\\
237.377322	-37.1514165800856\\
172.597085	-27.0128003438485\\
168.98592	-26.4476246390926\\
257.503296	-40.3012896928759\\
205.154432	-32.1082810365635\\
161.936295	-25.3443029194702\\
129.949024	-20.3380436013156\\
132.60528	-20.7537685423837\\
161.013435	-25.1998680761742\\
112.649272	-17.6304964444555\\
107.95753	-16.8962019463135\\
182.494	-28.5617453269868\\
208.04259	-32.560300463285\\
197.600788	-30.9260763820613\\
106.624953	-16.6876431723122\\
51.791245	-8.10573690015878\\
97.054804	-15.1898396364188\\
91.24675	-14.2808335365285\\
47.091495	-7.37018908707722\\
41.036372	-6.42251474682724\\
63.971478	-10.0120390962275\\
81.628812	-12.7755506465335\\
140.572773	-22.000743818067\\
100.76264	-15.7701451124686\\
62.356536	-9.7592879804546\\
53.107508	-8.31174240493886\\
39.69608	-6.21274851468921\\
46.16009	-7.22441688412106\\
39.967068	-6.25516026654226\\
41.71018	-6.52797099467806\\
83.947688	-13.138472965935\\
66.833325	-10.4599406446554\\
39.29326	-6.14970401869144\\
46.3139	-7.24848935800373\\
53.644638	-8.3958074715574\\
59.137638	-9.25550514425425\\
46.46771	-7.2725618318864\\
52.755932	-8.25671799769763\\
44.903975	-7.02782501407926\\
26.822319	-4.1979037357787\\
76.714164	-12.006369607443\\
103.25778	-16.1606541332319\\
146.927655	-22.9953327977981\\
194.460696	-30.4346273042434\\
162.768155	-25.4744955475441\\
78.191244	-12.2375442366778\\
63.1695	-9.88652323601374\\
67.724436	-10.5994065199174\\
39.29326	-6.14970401869144\\
54.718898	-8.56393760479449\\
62.56527	-9.79195649201708\\
83.142048	-13.0123839739395\\
76.714164	-12.006369607443\\
77.694408	-12.1597854977534\\
92.690829	-14.5068432498892\\
107.632058	-16.8452630202389\\
102.33492	-16.0162192899359\\
133.178353	-20.8434589709993\\
137.199412	-21.4727863083517\\
184.389756	-28.858446040841\\
110.857956	-17.3501414114563\\
54.181768	-8.47987253817595\\
48.817585	-7.64033573842718\\
90.945239	-14.2336446952774\\
61.96104	-9.69738974802043\\
52.570378	-8.22767733832032\\
26.197295	-4.10008256734985\\
58.33566	-9.12998928404051\\
37.015496	-5.79321605041315\\
18.573664	-2.90692439727894\\
32.522112	-5.08996613828258\\
60.346098	-9.44463863224756\\
77.211	-12.0841283463674\\
95.880498	-15.0060515178618\\
145.727166	-22.8074468339536\\
95.44765	-14.9383074038564\\
41.30736	-6.46492649868029\\
77.694408	-12.1597854977534\\
101.25983	-15.8479592551754\\
60.950328	-9.53920537624421\\
78.432948	-12.2753728123708\\
122.727328	-19.2077914177867\\
104.19773	-16.3077636958482\\
162.397725	-25.4165203411182\\
188.875902	-29.5605631490839\\
125.383584	-19.6235163588548\\
102.33492	-16.0162192899359\\
};
\end{axis}

\begin{axis}[%
width=4.927cm,
height=3cm,
at={(0cm,9.677cm)},
scale only axis,
xmin=-6.170934,
xmax=200,
xlabel style={font=\color{white!15!black}},
xlabel={y(t-1)u(t)},
ymin=-28.1670486811626,
ymax=1.221,
ylabel style={font=\color{white!15!black}},
ylabel={y(t)},
axis background/.style={fill=white},
title style={font=\small},
title={C4, R = -0.7299},
axis x line*=bottom,
axis y line*=left
]
\addplot[only marks, mark=*, mark options={}, mark size=1.5000pt, color=mycolor1, fill=mycolor1] table[row sep=crcr]{%
x	y\\
67.388124	-10.986\\
68.398836	-15.869\\
98.213241	-10.986\\
67.388124	-10.986\\
68.398836	-14.648\\
91.725776	-14.648\\
92.267752	-17.09\\
106.70996	-10.986\\
65.377686	-7.324\\
43.314136	-14.648\\
86.628272	-8.545\\
51.16746	-9.766\\
57.941678	-6.104\\
34.200712	-3.662\\
19.44522	-3.662\\
18.84099	-3.662\\
20.181282	-2.441\\
14.436074	-13.428\\
80.890272	-13.428\\
81.387108	-13.428\\
80.151732	-9.766\\
56.1545	-7.324\\
43.716956	-10.986\\
64.773456	-12.207\\
69.958317	-6.104\\
36.215032	-9.766\\
58.293254	-13.428\\
79.171488	-7.324\\
44.390764	-15.869\\
96.753293	-13.428\\
80.406864	-10.986\\
67.190376	-17.09\\
105.13768	-13.428\\
81.628812	-10.986\\
65.575434	-8.545\\
50.22751	-8.545\\
50.22751	-10.986\\
65.575434	-12.207\\
72.863583	-10.986\\
65.784168	-9.766\\
58.654596	-10.986\\
66.179664	-10.986\\
67.992354	-15.869\\
98.498883	-14.648\\
88.503216	-10.986\\
65.377686	-10.986\\
62.960766	-6.104\\
34.42656	-4.883\\
27.984473	-8.545\\
48.50142	-7.324\\
41.30736	-7.324\\
42.244832	-10.986\\
64.366974	-9.766\\
57.218994	-9.766\\
59.191726	-15.869\\
95.880498	-8.545\\
49.59518	-6.104\\
34.536432	-7.324\\
41.571024	-8.545\\
49.13375	-9.766\\
54.54311	-4.883\\
26.734425	-7.324\\
41.036372	-7.324\\
40.765384	-7.324\\
40.0989	-6.104\\
33.75512	-7.324\\
41.036372	-8.545\\
50.065155	-9.766\\
57.756124	-10.986\\
65.981916	-10.986\\
66.179664	-12.207\\
74.206353	-13.428\\
80.890272	-10.986\\
65.784168	-12.207\\
72.643857	-9.766\\
57.580336	-9.766\\
57.580336	-9.766\\
59.543302	-17.09\\
109.2051	-20.752\\
135.282288	-19.531\\
127.674147	-19.531\\
123.728885	-15.869\\
102.275705	-20.752\\
137.17072	-23.193\\
154.163871	-23.193\\
152.888256	-14.648\\
97.365256	-23.193\\
159.243138	-26.855\\
181.459235	-21.973\\
146.450045	-15.869\\
102.275705	-14.648\\
91.198448	-10.986\\
66.981642	-8.545\\
52.56884	-10.986\\
66.377412	-8.545\\
49.59518	-6.104\\
34.76228	-6.104\\
34.872152	-8.545\\
50.22751	-9.766\\
57.756124	-10.986\\
64.773456	-9.766\\
58.830384	-12.207\\
73.986627	-10.986\\
68.596584	-15.869\\
98.800394	-13.428\\
84.582972	-18.311\\
113.656377	-12.207\\
74.658012	-10.986\\
67.585872	-12.207\\
73.754694	-9.766\\
58.293254	-8.545\\
51.47508	-10.986\\
66.377412	-10.986\\
64.773456	-8.545\\
50.22751	-8.545\\
49.13375	-7.324\\
41.842012	-7.324\\
42.113	-7.324\\
41.30736	-4.883\\
27.628014	-6.104\\
34.982024	-8.545\\
50.851295	-13.428\\
81.387108	-12.207\\
74.658012	-13.428\\
79.413192	-8.545\\
51.005105	-15.869\\
99.086036	-17.09\\
106.07763	-13.428\\
83.347596	-14.648\\
92.004088	-14.648\\
88.781528	-10.986\\
64.366974	-7.324\\
43.314136	-8.545\\
51.16746	-8.545\\
53.038815	-17.09\\
109.83743	-20.752\\
134.514464	-19.531\\
123.728885	-14.648\\
92.531416	-15.869\\
99.958831	-13.428\\
84.086136	-12.207\\
76.220508	-12.207\\
74.206353	-9.766\\
58.117466	-8.545\\
50.38132	-8.545\\
49.911345	-9.766\\
57.218994	-8.545\\
51.16746	-13.428\\
83.602728	-15.869\\
98.498883	-10.986\\
66.377412	-8.545\\
50.697485	-9.766\\
57.580336	-8.545\\
48.663775	-6.104\\
33.4194	-3.662\\
19.913956	-4.883\\
27.54012	-8.545\\
47.723825	-9.766\\
54.357556	-4.883\\
27.447343	-7.324\\
41.168204	-7.324\\
40.50172	-3.662\\
19.64663	-3.662\\
19.0424	-2.441\\
12.917772	-4.883\\
27.628014	-10.986\\
65.784168	-12.207\\
74.877738	-14.648\\
88.239552	-13.428\\
78.432948	-8.545\\
48.34761	-7.324\\
39.967068	-4.883\\
27.447343	-7.324\\
41.973844	-6.104\\
35.763336	-10.986\\
66.179664	-12.207\\
77.783004	-21.973\\
145.637044	-23.193\\
153.30573	-21.973\\
144.846016	-18.311\\
117.00729	-13.428\\
85.563216	-17.09\\
109.83743	-15.869\\
101.990063	-19.531\\
123.728885	-12.207\\
76.440234	-13.428\\
84.086136	-12.207\\
75.549123	-13.428\\
83.602728	-13.428\\
82.609056	-10.986\\
69.200814	-18.311\\
118.362304	-17.09\\
108.26515	-13.428\\
81.870516	-7.324\\
44.251608	-12.207\\
76.440234	-15.869\\
101.704421	-17.09\\
112.33257	-19.531\\
129.822557	-21.973\\
146.845559	-20.752\\
136.402896	-15.869\\
103.1485	-14.648\\
95.753976	-18.311\\
121.03571	-19.531\\
126.248384	-13.428\\
82.864188	-9.766\\
60.803116	-13.428\\
83.105892	-9.766\\
58.478808	-7.324\\
43.856112	-9.766\\
59.191726	-10.986\\
65.575434	-7.324\\
43.050472	-7.324\\
43.314136	-9.766\\
57.404548	-6.104\\
36.324904	-10.986\\
67.388124	-13.428\\
82.609056	-13.428\\
80.648568	-8.545\\
50.851295	-9.766\\
59.728856	-12.207\\
78.906048	-18.311\\
118.014395	-15.869\\
98.213241	-9.766\\
58.830384	-8.545\\
51.32127	-9.766\\
58.478808	-7.324\\
42.911316	-8.545\\
49.757535	-7.324\\
43.050472	-8.545\\
51.005105	-10.986\\
66.179664	-10.986\\
65.179938	-6.104\\
36.880368	-10.986\\
67.992354	-13.428\\
83.105892	-13.428\\
82.125648	-9.766\\
59.904644	-12.207\\
76.220508	-13.428\\
82.125648	-10.986\\
63.1695	-4.883\\
27.808685	-10.986\\
63.1695	-8.545\\
50.22751	-8.545\\
50.22751	-6.104\\
35.098	-6.104\\
35.879312	-10.986\\
64.971204	-10.986\\
63.762744	-6.104\\
35.653464	-8.545\\
49.757535	-8.545\\
49.44137	-4.883\\
28.433709	-8.545\\
48.50142	-4.883\\
27.54012	-7.324\\
40.633552	-4.883\\
26.822319	-4.883\\
27.984473	-10.986\\
64.169226	-12.207\\
73.534968	-12.207\\
76.220508	-15.869\\
97.340446	-10.986\\
64.575708	-6.104\\
34.872152	-7.324\\
41.30736	-4.883\\
28.165144	-7.324\\
41.439192	-6.104\\
34.42656	-3.662\\
20.85509	-8.545\\
51.16746	-9.766\\
59.006172	-9.766\\
60.265986	-17.09\\
103.58249	-10.986\\
65.575434	-13.428\\
76.714164	-7.324\\
40.90454	-8.545\\
46.3139	-3.662\\
19.580714	-1.221\\
6.572643	-6.104\\
34.200712	-9.766\\
55.61737	-8.545\\
49.28756	-9.766\\
56.867418	-9.766\\
59.543302	-17.09\\
105.13768	-10.986\\
66.179664	-10.986\\
66.783894	-10.986\\
67.190376	-13.428\\
83.602728	-15.869\\
98.213241	-15.869\\
95.594856	-10.986\\
62.158788	-6.104\\
33.08368	-3.662\\
19.580714	-4.883\\
26.197295	-6.104\\
32.973808	-4.883\\
26.285189	-6.104\\
33.4194	-6.104\\
34.646304	-9.766\\
55.793158	-9.766\\
55.793158	-8.545\\
49.44137	-9.766\\
54.894686	-6.104\\
32.74796	-3.662\\
20.25086	-7.324\\
42.244832	-10.986\\
63.971478	-8.545\\
50.22751	-12.207\\
71.752746	-8.545\\
48.971395	-7.324\\
42.508496	-8.545\\
51.62889	-12.207\\
74.206353	-14.648\\
87.433912	-10.986\\
67.388124	-17.09\\
104.83006	-9.766\\
59.367514	-12.207\\
74.877738	-13.428\\
82.609056	-13.428\\
81.387108	-9.766\\
58.478808	-9.766\\
59.904644	-12.207\\
77.783004	-18.311\\
116.000185	-15.869\\
96.753293	-8.545\\
51.32127	-12.207\\
73.095516	-9.766\\
59.367514	-12.207\\
75.097464	-13.428\\
81.387108	-9.766\\
59.191726	-9.766\\
60.080432	-14.648\\
91.198448	-13.428\\
81.387108	-12.207\\
72.643857	-8.545\\
51.791245	-9.766\\
61.340246	-15.869\\
100.244473	-14.648\\
92.79508	-15.869\\
97.054804	-12.207\\
73.095516	-10.986\\
65.784168	-9.766\\
56.506076	-8.545\\
47.40766	-4.883\\
26.109401	-3.662\\
19.111978	-3.662\\
19.983534	-7.324\\
41.71018	-10.986\\
64.169226	-9.766\\
56.330288	-8.545\\
49.13375	-10.986\\
63.971478	-9.766\\
55.08024	-8.545\\
47.877635	-3.662\\
20.518186	-6.104\\
33.529272	-6.104\\
33.75512	-6.104\\
35.098	-8.545\\
50.697485	-10.986\\
63.762744	-7.324\\
41.439192	-6.104\\
33.864992	-7.324\\
40.765384	-7.324\\
39.967068	-6.104\\
34.872152	-8.545\\
51.945055	-15.869\\
96.753293	-14.648\\
90.920136	-17.09\\
108.58986	-17.09\\
109.52981	-17.09\\
107.64991	-13.428\\
82.367352	-10.986\\
66.783894	-9.766\\
59.367514	-10.986\\
67.388124	-12.207\\
75.768849	-14.648\\
91.462112	-14.648\\
93.879032	-17.09\\
111.71733	-19.531\\
130.525673	-21.973\\
144.428529	-18.311\\
119.699007	-21.973\\
145.24153	-19.531\\
126.599942	-14.648\\
95.490312	-17.09\\
110.77738	-14.648\\
91.198448	-10.986\\
65.377686	-8.545\\
49.911345	-8.545\\
51.005105	-10.986\\
65.179938	-7.324\\
42.244832	-7.324\\
42.508496	-7.324\\
41.973844	-7.324\\
40.90454	-3.662\\
20.45227	-7.324\\
41.168204	-7.324\\
40.765384	-6.104\\
34.872152	-8.545\\
50.38132	-12.207\\
71.301087	-8.545\\
51.791245	-12.207\\
77.331345	-18.311\\
117.00729	-17.09\\
111.40971	-19.531\\
125.174179	-17.09\\
109.2051	-14.648\\
91.198448	-14.648\\
87.433912	-7.324\\
43.716956	-9.766\\
57.756124	-8.545\\
49.13375	-6.104\\
35.543592	-7.324\\
41.30736	-8.545\\
45.997735	-2.441\\
13.095965	-4.883\\
27.54012	-8.545\\
49.911345	-9.766\\
58.478808	-12.207\\
73.534968	-12.207\\
73.534968	-10.986\\
66.981642	-12.207\\
73.986627	-14.648\\
87.170248	-10.986\\
65.179938	-9.766\\
57.404548	-7.324\\
43.987944	-10.986\\
64.773456	-10.986\\
63.564996	-7.324\\
43.050472	-10.986\\
66.783894	-13.428\\
80.890272	-10.986\\
64.773456	-8.545\\
49.59518	-8.545\\
49.757535	-8.545\\
48.663775	-6.104\\
34.200712	-7.324\\
41.71018	-6.104\\
35.427616	-9.766\\
57.404548	-8.545\\
49.59518	-10.986\\
64.366974	-10.986\\
64.575708	-8.545\\
48.34761	-6.104\\
34.310584	-6.104\\
35.763336	-10.986\\
67.190376	-14.648\\
90.392808	-14.648\\
89.587168	-12.207\\
74.206353	-10.986\\
65.981916	-9.766\\
57.941678	-8.545\\
50.065155	-8.545\\
50.697485	-9.766\\
58.117466	-12.207\\
70.409976	-7.324\\
42.779484	-7.324\\
43.585124	-10.986\\
65.981916	-10.986\\
66.586146	-13.428\\
81.870516	-13.428\\
84.086136	-14.648\\
94.40636	-20.752\\
136.797184	-21.973\\
142.033472	-14.648\\
89.587168	-9.766\\
57.218994	-7.324\\
40.90454	-6.104\\
34.310584	-7.324\\
41.168204	-4.883\\
28.433709	-7.324\\
42.911316	-7.324\\
42.508496	-7.324\\
43.182304	-10.986\\
65.784168	-12.207\\
74.206353	-13.428\\
82.609056	-13.428\\
85.80492	-18.311\\
117.684797	-17.09\\
107.64991	-12.207\\
74.658012	-12.207\\
73.754694	-9.766\\
59.191726	-9.766\\
59.904644	-13.428\\
80.890272	-9.766\\
58.293254	-8.545\\
51.005105	-10.986\\
62.763018	-6.104\\
35.207872	-10.986\\
65.981916	-14.648\\
86.906584	-8.545\\
50.697485	-8.545\\
53.20117	-15.869\\
98.800394	-13.428\\
82.864188	-10.986\\
69.200814	-17.09\\
108.26515	-17.09\\
105.77001	-10.986\\
65.179938	-8.545\\
50.065155	-9.766\\
59.006172	-12.207\\
77.331345	-18.311\\
118.362304	-19.531\\
126.599942	-18.311\\
117.00729	-15.869\\
97.927599	-9.766\\
59.006172	-9.766\\
57.404548	-7.324\\
42.508496	-6.104\\
34.872152	-6.104\\
35.207872	-7.324\\
42.779484	-9.766\\
55.431816	-8.545\\
48.971395	-6.104\\
35.989184	-12.207\\
73.315242	-10.986\\
67.388124	-13.428\\
84.824676	-17.09\\
111.085	-19.531\\
125.525737	-15.869\\
100.244473	-12.207\\
77.111619	-15.869\\
99.371678	-12.207\\
74.206353	-9.766\\
59.367514	-10.986\\
69.200814	-15.869\\
102.275705	-18.311\\
114.663482	-12.207\\
72.643857	-8.545\\
48.50142	-4.883\\
26.285189	-2.441\\
12.827455	0\\
-0	-6.104\\
33.639144	-8.545\\
48.34761	-8.545\\
48.34761	-7.324\\
40.50172	-6.104\\
33.639144	-6.104\\
33.529272	-7.324\\
40.0989	-4.883\\
26.734425	-6.104\\
34.42656	-8.545\\
50.38132	-10.986\\
65.981916	-13.428\\
80.648568	-10.986\\
67.190376	-12.207\\
76.672167	-17.09\\
108.26515	-17.09\\
110.46976	-19.531\\
126.599942	-20.752\\
131.858208	-13.428\\
82.864188	-10.986\\
66.981642	-10.986\\
69.59631	-14.648\\
94.40636	-17.09\\
107.64991	-13.428\\
81.870516	-9.766\\
55.968946	-7.324\\
39.967068	-3.662\\
19.712546	-4.883\\
25.303706	-6.104\\
32.74796	-1.221\\
6.774108	-9.766\\
54.54311	-7.324\\
40.90454	-6.104\\
32.857832	-4.883\\
25.660165	-3.662\\
19.379304	-6.104\\
32.638088	-6.104\\
32.74796	-4.883\\
25.92873	-3.662\\
19.177894	-3.662\\
19.64663	-4.883\\
28.970839	-12.207\\
75.329397	-13.428\\
84.341268	-15.869\\
98.213241	-14.648\\
92.004088	-14.648\\
96.281304	-23.193\\
153.723204	-20.752\\
137.17072	-18.311\\
118.691902	-18.311\\
118.014395	-15.869\\
102.577216	-15.869\\
99.958831	-14.648\\
90.656472	-10.986\\
65.784168	-8.545\\
49.59518	-8.545\\
51.62889	-12.207\\
72.863583	-14.648\\
86.906584	-8.545\\
51.16746	-9.766\\
58.293254	-9.766\\
58.293254	-9.766\\
58.293254	-10.986\\
66.179664	-10.986\\
66.783894	-12.207\\
73.534968	-10.986\\
64.971204	-9.766\\
58.117466	-9.766\\
55.793158	-10.986\\
58.533408	-2.441\\
12.871393	-4.883\\
26.734425	-8.545\\
45.22014	-7.324\\
36.47352	1.221\\
-6.170934	-2.441\\
12.917772	-7.324\\
39.827912	-7.324\\
40.90454	-8.545\\
47.723825	-8.545\\
49.59518	-8.545\\
49.911345	-10.986\\
62.56527	-7.324\\
42.508496	-8.545\\
48.031445	-10.986\\
59.741868	-4.883\\
25.660165	-4.883\\
24.766576	-3.662\\
19.0424	-3.662\\
18.976484	-2.441\\
13.095965	-3.662\\
20.584102	-10.986\\
62.356536	-12.207\\
68.176095	-8.545\\
47.56147	-8.545\\
47.091495	-7.324\\
40.765384	-6.104\\
33.529272	-8.545\\
46.46771	-6.104\\
32.973808	-6.104\\
32.302368	-6.104\\
32.638088	-3.662\\
19.44522	-4.883\\
26.734425	-7.324\\
40.50172	-7.324\\
39.967068	-3.662\\
19.913956	-4.883\\
25.92873	-6.104\\
32.41224	-6.104\\
34.646304	-9.766\\
58.117466	-13.428\\
79.413192	-6.104\\
35.653464	-7.324\\
41.71018	-9.766\\
57.218994	-8.545\\
50.38132	-12.207\\
69.738591	-7.324\\
41.973844	-7.324\\
41.439192	-10.986\\
63.1695	-7.324\\
40.90454	-9.766\\
53.644638	-6.104\\
33.974864	-7.324\\
41.571024	-8.545\\
47.877635	-8.545\\
47.56147	-7.324\\
41.036372	-8.545\\
49.44137	-8.545\\
49.44137	-10.986\\
65.575434	-10.986\\
68.794332	-18.311\\
114.333884	-13.428\\
80.648568	-10.986\\
64.366974	-9.766\\
58.293254	-9.766\\
59.728856	-13.428\\
81.387108	-9.766\\
59.543302	-10.986\\
64.971204	-14.648\\
82.072744	-4.883\\
26.734425	-3.662\\
21.0565	-10.986\\
64.169226	-7.324\\
42.376664	-9.766\\
57.941678	-10.986\\
65.981916	-12.207\\
72.424131	-10.986\\
63.762744	-9.766\\
56.867418	-8.545\\
51.005105	-12.207\\
72.192198	-8.545\\
50.53513	-10.986\\
64.773456	-10.986\\
63.564996	-7.324\\
42.376664	-8.545\\
48.817585	-4.883\\
27.808685	-6.104\\
35.879312	-10.986\\
65.981916	-13.428\\
81.131976	-10.986\\
66.179664	-13.428\\
79.171488	-9.766\\
55.968946	-7.324\\
41.973844	-7.324\\
42.376664	-7.324\\
43.050472	-9.766\\
58.654596	-12.207\\
74.877738	-13.428\\
81.387108	-12.207\\
71.301087	-7.324\\
43.987944	-8.545\\
53.20117	-19.531\\
119.803154	-12.207\\
74.426079	-13.428\\
82.864188	-15.869\\
96.467651	-10.986\\
65.575434	-9.766\\
58.654596	-9.766\\
58.293254	-9.766\\
58.654596	-8.545\\
52.098865	-13.428\\
80.890272	-10.986\\
66.377412	-9.766\\
59.191726	-10.986\\
66.377412	-10.986\\
65.575434	-8.545\\
51.47508	-9.766\\
58.117466	-4.883\\
28.878062	-9.766\\
58.117466	-12.207\\
72.643857	-9.766\\
55.61737	-7.324\\
39.564248	-3.662\\
18.976484	-2.441\\
12.827455	-3.662\\
20.785512	-9.766\\
58.293254	-13.428\\
80.648568	-10.986\\
64.366974	-10.986\\
64.169226	-13.428\\
78.191244	-12.207\\
70.409976	-7.324\\
40.90454	-6.104\\
34.310584	-3.662\\
20.65368	-8.545\\
48.1938	-4.883\\
27.808685	-8.545\\
48.50142	-7.324\\
41.168204	-7.324\\
40.0989	-6.104\\
33.75512	-4.883\\
28.340932	-9.766\\
56.1545	-10.986\\
63.564996	-6.104\\
35.317744	-10.986\\
64.773456	-8.545\\
50.53513	-9.766\\
59.367514	-14.648\\
88.781528	-8.545\\
51.62889	-12.207\\
73.986627	-12.207\\
71.301087	-3.662\\
21.928056	-12.207\\
72.192198	-13.428\\
79.413192	-9.766\\
58.117466	-15.869\\
95.880498	-13.428\\
80.406864	-13.428\\
80.890272	-10.986\\
67.794606	-15.869\\
97.626088	-13.428\\
81.387108	-12.207\\
72.863583	-10.986\\
65.981916	-8.545\\
50.851295	-9.766\\
57.404548	-8.545\\
49.59518	-6.104\\
35.427616	-8.545\\
49.28756	-7.324\\
44.119776	-10.986\\
68.794332	-15.869\\
99.086036	-13.428\\
83.602728	-14.648\\
90.920136	-14.648\\
87.712224	-10.986\\
64.366974	-7.324\\
42.244832	-7.324\\
41.439192	-4.883\\
27.628014	-6.104\\
35.543592	-9.766\\
58.293254	-12.207\\
73.986627	-12.207\\
73.754694	-12.207\\
71.972472	-8.545\\
51.791245	-9.766\\
60.803116	-17.09\\
107.34229	-14.648\\
91.462112	-13.428\\
86.54346	-17.09\\
109.83743	-23.193\\
142.265862	-7.324\\
43.182304	-7.324\\
42.244832	-8.545\\
50.697485	-8.545\\
52.26122	-12.207\\
76.000782	-17.09\\
105.77001	-14.648\\
88.781528	-10.986\\
63.762744	-9.766\\
55.61737	-6.104\\
35.098	-6.104\\
35.098	-8.545\\
48.1938	-6.104\\
34.42656	-3.662\\
20.45227	-6.104\\
33.08368	-3.662\\
20.45227	-4.883\\
27.715908	-12.207\\
70.629702	-7.324\\
42.508496	-9.766\\
56.681864	-7.324\\
42.647652	-9.766\\
57.941678	-9.766\\
57.043206	-10.986\\
65.377686	-13.428\\
81.870516	-10.986\\
65.784168	-13.428\\
80.151732	-12.207\\
72.643857	-13.428\\
79.910028	-9.766\\
59.728856	-14.648\\
88.781528	-13.428\\
80.151732	-7.324\\
43.856112	-9.766\\
58.117466	-10.986\\
63.762744	-8.545\\
50.22751	-13.428\\
81.628812	-13.428\\
81.870516	-14.648\\
90.392808	-13.428\\
86.301756	-20.752\\
131.46392	-20.752\\
129.201952	-10.986\\
66.981642	-12.207\\
74.877738	-10.986\\
69.59631	-15.869\\
99.086036	-7.324\\
44.925416	-10.986\\
68.794332	-15.869\\
97.626088	-4.883\\
28.702274	-7.324\\
42.779484	-9.766\\
56.330288	-10.986\\
61.752306	-6.104\\
34.09084	-6.104\\
33.75512	-4.883\\
26.822319	-4.883\\
27.090884	-6.104\\
34.646304	-7.324\\
42.244832	-7.324\\
43.453292	-9.766\\
58.478808	-13.428\\
81.131976	-10.986\\
67.794606	-14.648\\
90.392808	-13.428\\
81.131976	-12.207\\
73.315242	-8.545\\
51.16746	-10.986\\
65.981916	-9.766\\
59.904644	-15.869\\
95.880498	-13.428\\
81.131976	-13.428\\
81.131976	-12.207\\
73.315242	-7.324\\
45.196404	-12.207\\
75.329397	-14.648\\
90.114496	-10.986\\
67.190376	-13.428\\
79.668324	-9.766\\
55.61737	-4.883\\
27.090884	-6.104\\
34.646304	-6.104\\
34.76228	-9.766\\
56.506076	-6.104\\
35.098	-8.545\\
49.59518	-8.545\\
52.26122	-14.648\\
92.79508	-20.752\\
130.343312	-12.207\\
76.440234	-14.648\\
91.462112	-14.648\\
89.308856	-10.986\\
66.586146	-9.766\\
59.728856	-10.986\\
66.586146	-10.986\\
66.586146	-12.207\\
75.329397	-14.648\\
93.60072	-18.311\\
118.014395	-19.531\\
124.80309	-14.648\\
94.142696	-19.531\\
123.025769	-15.869\\
99.371678	-14.648\\
90.920136	-12.207\\
75.097464	-10.986\\
68.794332	-13.428\\
85.06638	-18.311\\
109.975866	-7.324\\
43.182304	-8.545\\
50.065155	-12.207\\
72.643857	-9.766\\
58.478808	-8.545\\
50.065155	-3.662\\
21.455658	-7.324\\
44.925416	-13.428\\
87.040296	-20.752\\
136.797184	-21.973\\
143.637501	-19.531\\
126.599942	-14.648\\
95.490312	-17.09\\
112.9649	-21.973\\
143.241987	-17.09\\
110.77738	-14.648\\
95.212	-18.311\\
116.348094	-13.428\\
83.602728	-13.428\\
84.582972	-15.869\\
98.213241	-12.207\\
73.095516	-7.324\\
44.119776	-10.986\\
65.377686	-10.986\\
65.179938	-6.104\\
36.215032	-9.766\\
58.478808	-7.324\\
44.925416	-13.428\\
82.125648	-9.766\\
58.293254	-8.545\\
51.005105	-9.766\\
59.191726	-12.207\\
74.658012	-12.207\\
74.426079	-10.986\\
66.179664	-7.324\\
45.057248	-13.428\\
82.864188	-14.648\\
88.781528	-9.766\\
60.441774	-13.428\\
84.086136	-12.207\\
74.658012	-9.766\\
58.117466	-9.766\\
56.1545	-6.104\\
33.974864	-4.883\\
28.07725	-6.104\\
35.653464	-9.766\\
55.61737	-4.883\\
27.090884	-6.104\\
33.864992	-4.883\\
27.984473	-12.207\\
71.301087	-10.986\\
65.981916	-12.207\\
74.877738	-14.648\\
90.114496	-13.428\\
82.864188	-14.648\\
88.239552	-10.986\\
61.96104	-4.883\\
27.271555	-8.545\\
47.723825	-7.324\\
41.30736	-7.324\\
42.911316	-10.986\\
65.575434	-10.986\\
67.585872	-15.869\\
96.467651	-15.869\\
95.309214	-9.766\\
60.441774	-15.869\\
97.054804	-17.09\\
101.70259	-6.104\\
35.317744	-7.324\\
42.508496	-9.766\\
58.117466	-10.986\\
67.794606	-14.648\\
89.587168	-4.883\\
29.327298	-10.986\\
65.377686	-9.766\\
58.654596	-10.986\\
67.388124	-13.428\\
85.80492	-18.311\\
117.684797	-15.869\\
101.704421	-14.648\\
91.198448	-13.428\\
80.648568	-8.545\\
51.16746	-9.766\\
55.793158	0\\
-0	-7.324\\
41.71018	-10.986\\
62.56527	-7.324\\
43.585124	-8.545\\
52.731195	-17.09\\
107.95753	-15.869\\
103.450011	-18.311\\
119.0215	-18.311\\
113.326779	-10.986\\
66.586146	-10.986\\
65.784168	-8.545\\
51.791245	-8.545\\
52.885005	-15.869\\
101.40291	-17.09\\
108.58986	-14.648\\
90.114496	-9.766\\
60.080432	-10.986\\
69.003066	-15.869\\
97.927599	-13.428\\
80.406864	-7.324\\
42.779484	-7.324\\
41.439192	-7.324\\
40.50172	-7.324\\
39.69608	-4.883\\
26.822319	-6.104\\
34.872152	-10.986\\
63.367248	-9.766\\
55.793158	-6.104\\
34.42656	-7.324\\
39.564248	-6.104\\
30.849616	-1.221\\
6.125757	-1.221\\
6.327222	-7.324\\
40.230732	-7.324\\
41.30736	-7.324\\
42.244832	-10.986\\
64.773456	-10.986\\
66.981642	-13.428\\
85.563216	-18.311\\
118.691902	-20.752\\
135.655824	-18.311\\
119.699007	-17.09\\
107.01758	-12.207\\
74.426079	-10.986\\
66.981642	-12.207\\
75.768849	-12.207\\
75.329397	-12.207\\
73.534968	-8.545\\
50.22751	-7.324\\
42.113	-6.104\\
35.763336	-9.766\\
57.941678	-10.986\\
62.158788	-7.324\\
40.0989	-3.662\\
20.382692	-6.104\\
34.982024	-7.324\\
41.571024	-4.883\\
28.07725	-8.545\\
48.971395	-8.545\\
49.911345	-8.545\\
52.098865	-14.648\\
89.308856	-10.986\\
67.794606	-14.648\\
93.879032	-19.531\\
127.322589	-21.973\\
142.428986	-14.648\\
92.004088	-10.986\\
66.981642	-8.545\\
49.59518	-7.324\\
42.779484	-9.766\\
56.506076	-8.545\\
50.697485	-8.545\\
51.16746	-9.766\\
57.404548	-7.324\\
43.453292	-9.766\\
56.1545	-8.545\\
49.28756	-7.324\\
41.571024	-9.766\\
53.644638	-4.883\\
26.822319	-2.441\\
13.23022	-1.221\\
6.595842	-3.662\\
19.712546	-6.104\\
32.07652	-2.441\\
12.827455	-2.441\\
13.095965	-7.324\\
39.564248	-6.104\\
32.857832	-6.104\\
34.09084	-9.766\\
56.506076	-10.986\\
63.367248	-8.545\\
48.971395	-7.324\\
43.182304	-13.428\\
81.870516	-19.531\\
118.377391	-13.428\\
81.387108	-10.986\\
63.1695	-9.766\\
53.46885	-3.662\\
20.518186	-7.324\\
43.050472	-8.545\\
50.697485	-9.766\\
60.265986	-13.428\\
83.602728	-13.428\\
81.387108	-10.986\\
64.575708	-9.766\\
57.043206	-9.766\\
57.218994	-8.545\\
49.59518	-7.324\\
40.90454	-4.883\\
27.00299	-4.883\\
26.822319	-4.883\\
26.377966	-3.662\\
19.24381	-6.104\\
32.74796	-7.324\\
40.90454	-10.986\\
61.96104	-9.766\\
54.54311	-4.883\\
28.07725	-9.766\\
58.830384	-12.207\\
72.643857	-6.104\\
36.550752	-13.428\\
81.628812	-12.207\\
73.095516	-7.324\\
41.973844	-8.545\\
48.663775	-12.207\\
70.629702	-9.766\\
54.718898	-6.104\\
33.974864	-9.766\\
53.820426	-4.883\\
26.016624	-3.662\\
19.111978	-2.441\\
13.005648	-4.883\\
26.646531	-7.324\\
40.0989	-7.324\\
40.50172	-8.545\\
48.1938	-8.545\\
47.40766	-4.883\\
26.016624	-3.662\\
19.0424	-2.441\\
12.6932	-4.883\\
25.748059	-4.883\\
26.016624	-4.883\\
27.447343	-8.545\\
49.44137	-10.986\\
63.1695	-6.104\\
34.982024	-8.545\\
49.911345	-10.986\\
66.377412	-13.428\\
82.609056	-13.428\\
83.347596	-13.428\\
82.125648	-10.986\\
66.981642	-12.207\\
74.658012	-13.428\\
82.125648	-14.648\\
90.392808	-13.428\\
85.321512	-18.311\\
116.677692	-15.869\\
100.244473	-15.869\\
98.800394	-12.207\\
75.097464	-13.428\\
82.367352	-13.428\\
83.105892	-13.428\\
80.406864	-9.766\\
58.117466	-7.324\\
44.251608	-10.986\\
67.585872	-13.428\\
81.870516	-12.207\\
74.877738	-10.986\\
66.783894	-8.545\\
51.32127	-12.207\\
71.081361	-8.545\\
50.22751	-14.648\\
89.587168	-15.869\\
97.054804	-10.986\\
64.575708	-6.104\\
35.763336	-9.766\\
55.793158	-2.441\\
13.49873	-3.662\\
20.518186	-7.324\\
40.0989	-4.883\\
26.285189	-3.662\\
19.84804	-8.545\\
45.52776	-4.883\\
26.377966	-7.324\\
39.425092	-6.104\\
31.856776	-4.883\\
25.92873	-4.883\\
26.197295	-6.104\\
33.08368	-6.104\\
33.309528	-7.324\\
40.0989	-6.104\\
34.200712	-7.324\\
41.168204	-7.324\\
41.439192	-7.324\\
43.716956	-14.648\\
87.433912	-9.766\\
57.404548	-8.545\\
50.38132	-10.986\\
63.564996	-7.324\\
40.633552	-6.104\\
34.536432	-10.986\\
62.158788	-8.545\\
46.783875	-4.883\\
27.178778	-10.986\\
61.752306	-8.545\\
48.663775	-8.545\\
47.091495	-7.324\\
39.161428	-2.441\\
12.6932	-3.662\\
18.976484	-4.883\\
26.109401	-4.883\\
26.285189	-7.324\\
39.827912	-8.545\\
48.031445	-8.545\\
49.911345	-9.766\\
56.681864	-7.324\\
43.050472	-10.986\\
66.377412	-14.648\\
88.239552	-12.207\\
73.095516	-10.986\\
68.190102	-14.648\\
90.656472	-12.207\\
76.220508	-17.09\\
106.70996	-15.869\\
99.371678	-19.531\\
124.099974	-20.752\\
128.060592	-13.428\\
79.171488	-6.104\\
34.982024	-6.104\\
35.763336	-8.545\\
50.851295	-12.207\\
73.315242	-12.207\\
72.192198	-8.545\\
47.877635	-7.324\\
40.765384	-10.986\\
64.169226	-13.428\\
81.628812	-14.648\\
89.850832	-13.428\\
79.668324	-9.766\\
56.506076	-6.104\\
35.653464	-10.986\\
66.586146	-12.207\\
75.329397	-14.648\\
91.198448	-14.648\\
89.587168	-12.207\\
75.768849	-19.531\\
123.728885	-19.531\\
123.377327	-17.09\\
110.77738	-20.752\\
135.282288	-18.311\\
117.684797	-15.869\\
102.577216	-17.09\\
109.83743	-15.869\\
97.054804	-10.986\\
65.784168	-8.545\\
51.62889	-9.766\\
59.904644	-12.207\\
75.549123	-12.207\\
74.206353	-8.545\\
52.26122	-13.428\\
82.125648	-12.207\\
73.095516	-8.545\\
51.62889	-9.766\\
59.191726	-12.207\\
75.329397	-13.428\\
84.086136	-14.648\\
89.850832	-10.986\\
65.981916	-9.766\\
56.1545	-8.545\\
49.59518	-12.207\\
70.19025	-8.545\\
48.50142	-7.324\\
41.168204	-2.441\\
14.079688	-7.324\\
43.314136	-9.766\\
58.478808	-12.207\\
73.534968	-12.207\\
71.081361	-8.545\\
48.50142	-6.104\\
33.75512	-6.104\\
33.529272	-7.324\\
41.571024	-8.545\\
48.971395	-9.766\\
55.968946	-6.104\\
36.215032	-13.428\\
81.870516	-13.428\\
83.105892	-14.648\\
91.462112	-15.869\\
100.831626	-17.09\\
105.46239	-12.207\\
73.534968	-12.207\\
72.192198	-7.324\\
43.050472	-8.545\\
51.005105	-12.207\\
72.424131	-10.986\\
63.762744	-6.104\\
34.76228	-6.104\\
36.099056	-8.545\\
51.62889	-12.207\\
73.534968	-10.986\\
68.398836	-18.311\\
116.677692	-18.311\\
114.333884	-12.207\\
73.986627	-8.545\\
49.911345	-7.324\\
42.113	-8.545\\
50.851295	-13.428\\
79.910028	-12.207\\
73.095516	-13.428\\
81.131976	-14.648\\
89.308856	-12.207\\
74.426079	-10.986\\
67.388124	-13.428\\
80.890272	-9.766\\
58.830384	-12.207\\
72.424131	-7.324\\
42.779484	-7.324\\
43.585124	-10.986\\
67.388124	-13.428\\
83.844432	-14.648\\
88.781528	-8.545\\
53.824955	-14.648\\
94.142696	-13.428\\
83.602728	-12.207\\
72.424131	-7.324\\
43.856112	-17.09\\
102.64254	-8.545\\
50.697485	-10.986\\
65.784168	-12.207\\
71.752746	-8.545\\
50.53513	-6.104\\
38.003504	-17.09\\
108.58986	-20.752\\
128.434128	-12.207\\
75.549123	-15.869\\
98.800394	-15.869\\
99.371678	-13.428\\
82.609056	-12.207\\
74.658012	-10.986\\
64.971204	-8.545\\
51.005105	-7.324\\
44.251608	-10.986\\
68.794332	-17.09\\
109.2051	-17.09\\
112.65728	-20.752\\
136.02936	-15.869\\
101.704421	-14.648\\
92.531416	-10.986\\
69.003066	-13.428\\
83.347596	-9.766\\
58.478808	-8.545\\
50.38132	-8.545\\
50.38132	-7.324\\
42.244832	-7.324\\
40.90454	-3.662\\
20.65368	-8.545\\
49.44137	-10.986\\
62.763018	-6.104\\
33.864992	-6.104\\
33.529272	-6.104\\
35.427616	-6.104\\
36.880368	-14.648\\
85.016992	-6.104\\
35.427616	-3.662\\
21.323826	-7.324\\
42.647652	-8.545\\
50.22751	-9.766\\
57.218994	-8.545\\
48.817585	-7.324\\
40.765384	-6.104\\
33.309528	-4.883\\
26.734425	-4.883\\
27.984473	-10.986\\
64.575708	-10.986\\
63.762744	-8.545\\
48.34761	-6.104\\
33.309528	-4.883\\
26.646531	-8.545\\
48.34761	-10.986\\
63.971478	-10.986\\
64.575708	-9.766\\
56.506076	-6.104\\
34.872152	-7.324\\
42.244832	-7.324\\
42.911316	-10.986\\
66.981642	-14.648\\
91.462112	-15.869\\
98.213241	-13.428\\
79.910028	-7.324\\
43.182304	-12.207\\
71.752746	-9.766\\
57.580336	-9.766\\
56.1545	-6.104\\
35.098	-9.766\\
56.506076	-8.545\\
49.44137	-8.545\\
48.34761	-6.104\\
33.193552	-4.883\\
26.822319	-7.324\\
41.30736	-9.766\\
57.404548	-13.428\\
80.406864	-13.428\\
82.125648	-12.207\\
74.658012	-13.428\\
82.864188	-13.428\\
85.06638	-17.09\\
112.65728	-21.973\\
144.428529	-23.193\\
148.20327	-13.428\\
85.563216	-17.09\\
110.14505	-15.869\\
104.608448	-20.752\\
133.74664	-14.648\\
88.781528	-7.324\\
42.244832	-6.104\\
34.76228	-7.324\\
40.633552	-6.104\\
32.41224	-3.662\\
19.177894	-2.441\\
13.139903	-4.883\\
27.00299	-8.545\\
48.031445	-9.766\\
54.181768	-4.883\\
26.822319	-7.324\\
40.765384	-6.104\\
34.646304	-8.545\\
48.817585	-8.545\\
48.817585	-7.324\\
41.168204	-4.883\\
26.646531	-3.662\\
19.712546	-3.662\\
20.115366	-6.104\\
34.200712	-7.324\\
40.633552	-3.662\\
19.712546	-3.662\\
20.181282	-9.766\\
53.107508	-7.324\\
39.827912	-9.766\\
54.00598	-7.324\\
41.571024	-8.545\\
49.44137	-12.207\\
69.286932	-9.766\\
56.506076	-12.207\\
71.752746	-10.986\\
63.762744	-7.324\\
41.973844	-6.104\\
35.879312	-13.428\\
81.387108	-13.428\\
81.870516	-13.428\\
82.609056	-14.648\\
89.308856	-10.986\\
66.377412	-10.986\\
66.179664	-10.986\\
67.585872	-12.207\\
74.206353	-8.545\\
52.731195	-17.09\\
106.07763	-13.428\\
83.105892	-14.648\\
95.753976	-23.193\\
151.612641	-15.869\\
98.800394	-8.545\\
50.851295	-7.324\\
43.050472	-6.104\\
36.324904	-9.766\\
59.191726	-13.428\\
82.864188	-15.869\\
99.086036	-14.648\\
92.004088	-15.869\\
97.340446	-9.766\\
59.006172	-9.766\\
59.367514	-10.986\\
66.783894	-12.207\\
72.192198	-6.104\\
34.872152	-4.883\\
28.340932	-7.324\\
42.779484	-8.545\\
50.851295	-10.986\\
67.190376	-13.428\\
82.125648	-10.986\\
65.575434	-8.545\\
50.22751	-7.324\\
43.856112	-10.986\\
67.388124	-14.648\\
92.267752	-15.869\\
101.990063	-17.09\\
111.40971	-20.752\\
134.514464	-14.648\\
94.142696	-18.311\\
113.656377	-12.207\\
72.643857	-8.545\\
52.098865	-14.648\\
92.267752	-15.869\\
96.467651	-10.986\\
67.794606	-10.986\\
70.20054	-15.869\\
104.608448	-19.531\\
126.248384	-17.09\\
104.83006	-7.324\\
42.779484	-6.104\\
35.653464	-12.207\\
71.520813	-6.104\\
35.653464	-8.545\\
48.50142	-6.104\\
34.646304	-9.766\\
54.894686	-3.662\\
20.584102	-4.883\\
27.359449	-7.324\\
41.439192	-9.766\\
57.218994	-10.986\\
64.773456	-12.207\\
73.754694	-13.428\\
83.347596	-15.869\\
99.086036	-13.428\\
81.131976	-8.545\\
50.697485	-7.324\\
44.251608	-10.986\\
64.773456	-8.545\\
49.757535	-7.324\\
41.71018	-4.883\\
28.340932	-4.883\\
28.433709	-6.104\\
34.310584	-4.883\\
26.197295	-3.662\\
19.177894	-4.883\\
26.377966	-4.883\\
27.271555	-8.545\\
48.34761	-7.324\\
40.50172	-6.104\\
33.639144	-4.883\\
27.715908	-9.766\\
57.043206	-9.766\\
56.506076	-7.324\\
41.71018	-9.766\\
56.330288	-8.545\\
49.911345	-10.986\\
64.169226	-9.766\\
57.580336	-10.986\\
64.773456	-9.766\\
56.506076	-7.324\\
41.571024	-7.324\\
41.842012	-7.324\\
41.571024	-8.545\\
48.817585	-8.545\\
50.065155	-9.766\\
59.728856	-14.648\\
88.781528	-9.766\\
58.654596	-10.986\\
67.992354	-15.869\\
97.626088	-10.986\\
65.377686	-9.766\\
56.506076	-8.545\\
48.1938	-7.324\\
40.90454	-4.883\\
27.00299	-3.662\\
20.04945	-6.104\\
34.982024	-10.986\\
60.543846	-7.324\\
39.161428	-9.766\\
53.293062	-7.324\\
41.036372	-8.545\\
47.723825	-2.441\\
13.320537	-4.883\\
25.748059	-2.441\\
13.095965	-2.441\\
13.76724	-8.545\\
50.22751	-10.986\\
64.169226	-7.324\\
41.973844	-7.324\\
42.113	-9.766\\
56.330288	-9.766\\
56.867418	-9.766\\
57.756124	-12.207\\
73.315242	-12.207\\
73.986627	-13.428\\
80.648568	-12.207\\
71.752746	-8.545\\
48.663775	-7.324\\
41.71018	-7.324\\
42.508496	-9.766\\
55.431816	-8.545\\
47.56147	-8.545\\
49.44137	-8.545\\
50.697485	-12.207\\
72.643857	-10.986\\
65.784168	-12.207\\
75.329397	-15.869\\
95.023572	-8.545\\
52.56884	-8.545\\
53.35498	-13.428\\
83.105892	-13.428\\
83.347596	-15.869\\
98.800394	-14.648\\
96.01764	-21.973\\
146.845559	-25.635\\
166.6275	-15.869\\
103.735653	-18.311\\
121.713217	-21.973\\
146.450045	-20.752\\
139.45344	-24.414\\
163.622628	-23.193\\
158.825664	-25.635\\
174.138555	-20.752\\
136.797184	-14.648\\
93.073392	-10.986\\
68.398836	-13.428\\
82.367352	-9.766\\
58.478808	-8.545\\
52.098865	-9.766\\
61.340246	-13.428\\
82.367352	-10.986\\
65.981916	-8.545\\
51.32127	-8.545\\
49.59518	-8.545\\
49.44137	-8.545\\
47.40766	-6.104\\
33.309528	-10.986\\
59.54412	-7.324\\
40.230732	-7.324\\
39.827912	-6.104\\
32.522112	-7.324\\
37.952968	-3.662\\
18.43817	-3.662\\
18.976484	-3.662\\
18.775074	-3.662\\
18.63958	-6.104\\
33.08368	-6.104\\
34.536432	-9.766\\
56.1545	-10.986\\
62.356536	-8.545\\
49.13375	-8.545\\
51.16746	-12.207\\
70.849428	-9.766\\
53.820426	-4.883\\
26.109401	-7.324\\
37.952968	-4.883\\
25.748059	-6.104\\
33.193552	-8.545\\
48.663775	-8.545\\
47.723825	-4.883\\
28.07725	-9.766\\
58.830384	-14.648\\
89.045192	-15.869\\
95.309214	-10.986\\
63.1695	-7.324\\
42.244832	-8.545\\
50.53513	-12.207\\
74.206353	-13.428\\
79.910028	-7.324\\
41.168204	-6.104\\
34.76228	-6.104\\
34.536432	-8.545\\
45.690115	-2.441\\
12.380752	-2.441\\
12.783517	-7.324\\
41.842012	-10.986\\
64.366974	-10.986\\
62.960766	-7.324\\
40.50172	-6.104\\
33.4194	-6.104\\
30.849616	-3.662\\
18.306338	-7.324\\
37.015496	-1.221\\
6.327222	-3.662\\
19.712546	-7.324\\
41.71018	-10.986\\
63.971478	-10.986\\
64.773456	-10.986\\
65.179938	-12.207\\
72.863583	-10.986\\
65.377686	-10.986\\
64.971204	-8.545\\
51.16746	-10.986\\
68.794332	-17.09\\
108.26515	-18.311\\
110.635062	-8.545\\
48.34761	-4.883\\
28.790168	-4.883\\
29.683757	-14.648\\
88.503216	-10.986\\
63.762744	-8.545\\
48.817585	-9.766\\
57.756124	-10.986\\
68.190102	-17.09\\
107.95753	-18.311\\
117.00729	-18.311\\
118.014395	-19.531\\
124.099974	-14.648\\
92.531416	-13.428\\
80.406864	-9.766\\
56.330288	-9.766\\
57.218994	-8.545\\
50.851295	-10.986\\
65.784168	-9.766\\
58.830384	-12.207\\
71.972472	-8.545\\
50.697485	-9.766\\
57.941678	-8.545\\
48.817585	-6.104\\
33.309528	-4.883\\
25.840836	-4.883\\
26.197295	-4.883\\
26.016624	-4.883\\
24.947247	-2.441\\
12.739579	-4.883\\
27.090884	-7.324\\
40.50172	-6.104\\
33.193552	-6.104\\
33.75512	-8.545\\
49.757535	-12.207\\
70.629702	-8.545\\
47.091495	-4.883\\
25.572271	-1.221\\
6.728931	-7.324\\
43.314136	-13.428\\
82.609056	-14.648\\
94.948336	-20.752\\
139.079904	-24.414\\
164.965398	-24.414\\
160.473222	-15.869\\
103.735653	-17.09\\
113.27252	-21.973\\
145.24153	-18.311\\
120.028605	-15.869\\
104.89409	-18.311\\
121.713217	-20.752\\
138.31208	-19.531\\
133.045172	-26.855\\
180.4656	-20.752\\
132.60528	-12.207\\
73.315242	-7.324\\
42.113	-6.104\\
34.872152	-6.104\\
34.872152	-6.104\\
34.76228	-6.104\\
36.215032	-10.986\\
64.575708	-12.207\\
71.520813	-8.545\\
51.16746	-10.986\\
64.169226	-6.104\\
35.653464	-3.662\\
21.792562	-9.766\\
59.191726	-12.207\\
71.520813	-9.766\\
55.08024	-6.104\\
34.536432	-6.104\\
34.536432	-7.324\\
44.251608	-10.986\\
69.805044	-17.09\\
108.89748	-15.869\\
102.577216	-17.09\\
112.65728	-20.752\\
140.968336	-25.635\\
175.08705	-21.973\\
144.846016	-17.09\\
107.95753	-10.986\\
65.377686	-8.545\\
49.911345	-4.883\\
28.609497	-7.324\\
43.453292	-9.766\\
56.867418	-8.545\\
49.13375	-6.104\\
35.317744	-7.324\\
42.779484	-8.545\\
50.38132	-8.545\\
51.47508	-10.986\\
65.179938	-9.766\\
55.256028	-4.883\\
26.377966	-4.883\\
25.660165	-2.441\\
12.6932	-2.441\\
12.739579	-3.662\\
19.84804	-7.324\\
40.362564	-8.545\\
47.877635	-7.324\\
42.911316	-10.986\\
63.971478	-8.545\\
51.32127	-8.545\\
54.60255	-19.531\\
124.099974	-14.648\\
92.267752	-12.207\\
77.331345	-17.09\\
110.14505	-18.311\\
116.000185	-13.428\\
83.347596	-12.207\\
72.863583	-10.986\\
64.971204	-9.766\\
60.803116	-14.648\\
97.086944	-24.414\\
167.626524	-25.635\\
175.08705	-20.752\\
139.826976	-19.531\\
128.377263	-14.648\\
95.753976	-17.09\\
113.90485	-19.531\\
125.877295	-14.648\\
92.004088	-10.986\\
67.585872	-9.766\\
62.053164	-12.207\\
78.234663	-13.428\\
82.864188	-8.545\\
51.47508	-7.324\\
43.314136	-7.324\\
44.251608	-8.545\\
53.671145	-14.648\\
93.073392	-17.09\\
106.70996	-12.207\\
74.658012	-10.986\\
66.981642	-9.766\\
59.006172	-9.766\\
57.218994	-6.104\\
34.872152	-4.883\\
27.271555	-6.104\\
33.309528	-6.104\\
33.193552	-4.883\\
27.271555	-7.324\\
42.647652	-9.766\\
57.404548	-10.986\\
62.960766	-8.545\\
48.031445	-3.662\\
20.382692	-3.662\\
20.45227	-6.104\\
34.982024	-8.545\\
49.59518	-9.766\\
56.506076	-8.545\\
48.1938	-4.883\\
28.970839	-10.986\\
67.794606	-14.648\\
88.781528	-10.986\\
62.356536	-7.324\\
40.765384	-9.766\\
55.968946	-8.545\\
49.28756	-8.545\\
49.13375	-6.104\\
33.974864	-6.104\\
34.872152	-4.883\\
28.970839	-10.986\\
63.971478	-9.766\\
53.293062	-3.662\\
19.64663	-6.104\\
34.982024	-9.766\\
57.941678	-12.207\\
70.409976	-7.324\\
41.30736	-6.104\\
35.879312	-10.986\\
67.388124	-13.428\\
82.609056	-12.207\\
76.891893	-17.09\\
111.71733	-19.531\\
125.174179	-12.207\\
76.440234	-12.207\\
77.783004	-17.09\\
107.01758	-13.428\\
84.582972	-10.986\\
70.607022	-18.311\\
116.348094	-14.648\\
88.503216	-7.324\\
44.793584	-8.545\\
55.38869	-19.531\\
129.451468	-21.973\\
143.637501	-17.09\\
109.83743	-12.207\\
79.577433	-17.09\\
110.14505	-14.648\\
91.198448	-9.766\\
59.543302	-8.545\\
52.56884	-12.207\\
75.768849	-12.207\\
75.549123	-12.207\\
76.220508	-12.207\\
74.426079	-9.766\\
59.367514	-9.766\\
60.617562	-12.207\\
73.986627	-10.986\\
64.773456	-6.104\\
35.317744	-6.104\\
34.536432	-4.883\\
27.447343	-6.104\\
35.427616	-10.986\\
64.575708	-8.545\\
50.851295	-13.428\\
82.367352	-13.428\\
84.341268	-15.869\\
98.800394	-13.428\\
82.367352	-9.766\\
56.681864	-9.766\\
56.867418	-6.104\\
37.332064	-14.648\\
88.503216	-10.986\\
62.960766	-7.324\\
42.779484	-13.428\\
82.125648	-10.986\\
67.992354	-13.428\\
85.321512	-15.869\\
104.89409	-20.752\\
135.655824	-17.09\\
109.83743	-14.648\\
94.948336	-15.869\\
100.530115	-14.648\\
90.392808	-7.324\\
45.328236	-12.207\\
76.672167	-13.428\\
85.06638	-13.428\\
85.06638	-15.869\\
97.340446	-12.207\\
73.095516	-8.545\\
50.38132	-8.545\\
50.851295	-8.545\\
50.851295	-8.545\\
50.22751	-6.104\\
37.332064	-10.986\\
68.596584	-13.428\\
82.125648	-7.324\\
44.119776	-7.324\\
44.925416	-10.986\\
65.575434	-9.766\\
55.431816	-3.662\\
20.65368	-6.104\\
34.982024	-9.766\\
55.793158	-6.104\\
34.200712	-4.883\\
26.109401	-4.883\\
26.285189	-2.441\\
13.632985	-8.545\\
49.911345	-8.545\\
51.16746	-9.766\\
60.265986	-13.428\\
84.824676	-17.09\\
103.89011	-13.428\\
78.929784	-6.104\\
37.777656	-12.207\\
79.125774	-20.752\\
133.74664	-15.869\\
101.40291	-14.648\\
97.086944	-19.531\\
129.09991	-17.09\\
110.77738	-13.428\\
85.321512	-13.428\\
84.341268	-12.207\\
78.00273	-15.869\\
99.673189	-14.648\\
90.114496	-10.986\\
69.805044	-14.648\\
95.490312	-20.752\\
136.402896	-18.311\\
114.333884	-9.766\\
57.218994	-7.324\\
44.793584	-6.104\\
37.216088	-14.648\\
84.226	-2.441\\
13.452351	-3.662\\
20.518186	-7.324\\
42.508496	-10.986\\
66.981642	-12.207\\
73.754694	-12.207\\
71.972472	-9.766\\
55.61737	-3.662\\
19.983534	-4.883\\
26.646531	-3.662\\
19.84804	-7.324\\
40.230732	-3.662\\
20.921006	-8.545\\
48.971395	-8.545\\
46.937685	-7.324\\
39.29326	-6.104\\
33.08368	-6.104\\
33.639144	-6.104\\
32.857832	-7.324\\
39.827912	-3.662\\
19.913956	0\\
-0	-2.441\\
13.408413	-6.104\\
35.207872	-10.986\\
66.377412	-10.986\\
69.805044	-18.311\\
117.00729	-14.648\\
88.781528	-9.766\\
56.867418	-6.104\\
35.207872	-8.545\\
47.56147	-1.221\\
6.572643	-3.662\\
20.45227	-6.104\\
34.76228	-12.207\\
69.286932	-7.324\\
41.842012	-7.324\\
42.376664	-9.766\\
57.043206	-10.986\\
64.575708	-9.766\\
58.654596	-10.986\\
66.377412	-13.428\\
83.347596	-17.09\\
102.95016	-14.648\\
82.61472	-2.441\\
13.452351	-3.662\\
20.85509	-8.545\\
48.50142	-8.545\\
48.031445	-4.883\\
26.197295	-6.104\\
32.638088	-3.662\\
19.24381	-1.221\\
6.147735	-1.221\\
6.192912	-2.441\\
13.005648	-4.883\\
26.822319	-4.883\\
28.165144	-8.545\\
51.47508	-13.428\\
80.406864	-13.428\\
74.740248	-6.104\\
34.536432	-6.104\\
35.317744	-13.428\\
74.498544	-7.324\\
40.50172	-3.662\\
21.323826	-12.207\\
72.192198	-8.545\\
52.56884	-13.428\\
85.80492	-21.973\\
136.803898	-10.986\\
66.586146	-8.545\\
51.16746	-10.986\\
};
\addplot [color=mycolor2, line width=2.0pt, forget plot]
  table[row sep=crcr]{%
67.388124	-10.460336004614\\
68.398836	-10.6172239916412\\
98.213241	-15.2451714038239\\
67.388124	-10.460336004614\\
68.398836	-10.6172239916412\\
91.725776	-14.2381532574488\\
92.267752	-14.3222815983184\\
106.70996	-16.5640764333924\\
65.377686	-10.1482653347666\\
43.314136	-6.7234460527429\\
86.628272	-13.4468921054858\\
51.16746	-7.94247995540948\\
57.941678	-8.99400939772642\\
34.200712	-5.30881285724819\\
19.44522	-3.01838844606567\\
18.84099	-2.92459671469075\\
20.181282	-3.1326438279224\\
14.436074	-2.24084268360806\\
80.890272	-12.5562098245178\\
81.387108	-12.6333312992036\\
80.151732	-12.4415697945795\\
56.1545	-8.71659430927472\\
43.716956	-6.78597387365952\\
64.773456	-10.0544736033916\\
69.958317	-10.8592947644202\\
36.215032	-5.62148611137846\\
58.293254	-9.04858285774971\\
79.171488	-12.2894112093886\\
44.390764	-6.89056586501094\\
96.753293	-15.0185506623225\\
80.406864	-12.4811727140127\\
67.190376	-10.4296405288913\\
105.13768	-16.3200189330926\\
81.628812	-12.6708498544561\\
65.575434	-10.1789608104893\\
50.22751	-7.79657601501285\\
65.575434	-10.1789608104893\\
72.863583	-11.3102653025344\\
65.784168	-10.2113615904188\\
58.654596	-9.10467224721808\\
66.179664	-10.2727525418642\\
67.992354	-10.554127735989\\
98.498883	-15.2895102445524\\
88.503216	-13.7379306901159\\
65.377686	-10.1482653347666\\
62.960766	-9.77309840926688\\
34.42656	-5.34387016149916\\
27.984473	-4.34389582490899\\
48.50142	-7.5286433244663\\
41.30736	-6.41194381763104\\
42.244832	-6.55746310994607\\
64.366974	-9.99137734773943\\
57.218994	-8.88183061878967\\
59.191726	-9.1880483666981\\
95.880498	-14.8830708712076\\
49.59518	-7.69842245510966\\
34.536432	-5.3609250662699\\
41.571024	-6.45287111859464\\
49.13375	-7.62679688436949\\
54.54311	-8.46646595083466\\
26.734425	-4.14985685593731\\
41.036372	-6.36987964719623\\
40.765384	-6.32781547676141\\
40.0989	-6.22436035488119\\
33.75512	-5.23964574345574\\
41.036372	-6.36987964719623\\
50.065155	-7.77137442530798\\
57.756124	-8.96520673826969\\
65.981916	-10.2420570661415\\
66.179664	-10.2727525418642\\
74.206353	-11.5186970638476\\
80.890272	-12.5562098245178\\
65.784168	-10.2113615904188\\
72.643857	-11.2761582870467\\
57.580336	-8.93792000825804\\
59.543302	-9.24262182672139\\
109.2051	-16.9513850751725\\
135.282288	-20.9992221767884\\
127.674147	-19.8182468578957\\
123.728885	-19.2058427175722\\
102.275705	-15.8757682497406\\
137.17072	-21.292354439112\\
154.163871	-23.9301199486125\\
152.888256	-23.732112336581\\
97.365256	-15.1135427502814\\
159.243138	-24.7185502583381\\
181.459235	-28.1670486811626\\
146.450045	-22.7327396529223\\
102.275705	-15.8757682497406\\
91.198448	-14.1562986555216\\
66.981642	-10.3972397489618\\
52.56884	-8.16000946654628\\
66.377412	-10.3034480175869\\
49.59518	-7.69842245510966\\
34.76228	-5.39598237052087\\
34.872152	-5.41303727529162\\
50.22751	-7.79657601501285\\
57.756124	-8.96520673826969\\
64.773456	-10.0544736033916\\
58.830384	-9.13195897722972\\
73.986627	-11.48459004836\\
68.596584	-10.6479194673639\\
98.800394	-15.3363123542103\\
84.582972	-13.1294099742094\\
113.656377	-17.642335502426\\
74.658012	-11.5888059290166\\
67.585872	-10.4910314803367\\
73.754694	-11.4485881986786\\
58.293254	-9.04858285774971\\
51.47508	-7.99023033590293\\
66.377412	-10.3034480175869\\
64.773456	-10.0544736033916\\
50.22751	-7.79657601501285\\
49.13375	-7.62679688436949\\
41.842012	-6.49493528902945\\
42.113	-6.53699945946427\\
41.30736	-6.41194381763104\\
27.628014	-4.28856440016316\\
34.982024	-5.43009218006236\\
50.851295	-7.89340317545789\\
81.387108	-12.6333312992036\\
74.658012	-11.5888059290166\\
79.413192	-12.3269297646411\\
51.005105	-7.91727836570461\\
99.086036	-15.3806511949388\\
106.07763	-16.4659228734892\\
83.347596	-12.9376484695853\\
92.004088	-14.2813542973548\\
88.781528	-13.7811317300219\\
64.366974	-9.99137734773943\\
43.314136	-6.7234460527429\\
51.16746	-7.94247995540948\\
53.038815	-8.2329614367446\\
109.83743	-17.0495386350757\\
134.514464	-20.8800365316678\\
123.728885	-19.2058427175722\\
92.531416	-14.363208899282\\
99.958831	-15.5161309860537\\
84.086136	-13.0522884995236\\
76.220508	-11.8313447058175\\
74.206353	-11.5186970638476\\
58.117466	-9.02129612773806\\
50.38132	-7.82045120525957\\
49.911345	-7.74749923506126\\
57.218994	-8.88183061878967\\
51.16746	-7.94247995540948\\
83.602728	-12.9772513890186\\
98.498883	-15.2895102445524\\
66.377412	-10.3034480175869\\
50.697485	-7.86952798521117\\
57.580336	-8.93792000825804\\
48.663775	-7.55384491417118\\
33.4194	-5.18753353443403\\
19.913956	-3.09114809222319\\
27.54012	-4.27492103515734\\
47.723825	-7.40794097377454\\
54.357556	-8.43766329137793\\
27.447343	-4.26051970542897\\
41.168204	-6.39034329767803\\
40.50172	-6.28688817579781\\
19.64663	-3.04965235652398\\
19.0424	-2.95586062514906\\
12.917772	-2.00516392993809\\
27.628014	-4.28856440016316\\
65.784168	-10.2113615904188\\
74.877738	-11.6229129445042\\
88.239552	-13.6970033891523\\
78.432948	-12.1747711794503\\
48.34761	-7.50476813421958\\
39.967068	-6.20389670439939\\
27.447343	-4.26051970542897\\
41.973844	-6.51539893951126\\
35.763336	-5.55137150287652\\
66.179664	-10.2727525418642\\
77.783004	-12.0738834826184\\
145.637044	-22.6065413982849\\
153.30573	-23.7969148277913\\
144.846016	-22.4837539072864\\
117.00729	-18.1624816917194\\
85.563216	-13.2815685594003\\
109.83743	-17.0495386350757\\
101.990063	-15.8314294090121\\
123.728885	-19.2058427175722\\
76.440234	-11.8654517213051\\
84.086136	-13.0522884995236\\
75.549123	-11.7271288251609\\
83.602728	-12.9772513890186\\
82.609056	-12.823008439647\\
69.200814	-10.7417111987388\\
118.362304	-18.3728140305593\\
108.26515	-16.8054811347759\\
81.870516	-12.7083684097087\\
44.251608	-6.86896534505793\\
76.440234	-11.8654517213051\\
101.704421	-15.7870905682836\\
112.33257	-17.4368472768559\\
129.822557	-20.1517342610422\\
146.845559	-22.7941333984216\\
136.402896	-21.1731687939914\\
103.1485	-16.0112480408556\\
95.753976	-14.8634314666149\\
121.03571	-18.7877940504327\\
126.248384	-19.596932490353\\
82.864188	-12.8626113590802\\
60.803116	-9.43817672513816\\
83.105892	-12.9001299143328\\
58.478808	-9.07738551720644\\
43.856112	-6.80757439361253\\
59.191726	-9.1880483666981\\
65.575434	-10.1789608104893\\
43.050472	-6.6825187517793\\
43.314136	-6.7234460527429\\
57.404548	-8.9106332782464\\
36.324904	-5.6385410161492\\
67.388124	-10.460336004614\\
82.609056	-12.823008439647\\
80.648568	-12.5186912692652\\
50.851295	-7.89340317545789\\
59.728856	-9.27142448617812\\
78.906048	-12.248208228444\\
118.014395	-18.3188097813977\\
98.213241	-15.2451714038239\\
58.830384	-9.13195897722972\\
51.32127	-7.9663551456562\\
58.478808	-9.07738551720644\\
42.911316	-6.66091823182629\\
49.757535	-7.72362404481453\\
43.050472	-6.6825187517793\\
51.005105	-7.91727836570461\\
66.179664	-10.2727525418642\\
65.179938	-10.1175698590439\\
36.880368	-5.7247630347124\\
67.992354	-10.554127735989\\
83.105892	-12.9001299143328\\
82.125648	-12.7479713291419\\
59.904644	-9.29871121618976\\
76.220508	-11.8313447058175\\
82.125648	-12.7479713291419\\
63.1695	-9.8054991891964\\
27.808685	-4.31660909489735\\
63.1695	-9.8054991891964\\
50.22751	-7.79657601501285\\
35.098	-5.44809457954258\\
35.879312	-5.56937390235675\\
64.971204	-10.0851690791144\\
63.762744	-9.89758561636451\\
35.653464	-5.53431659810578\\
49.757535	-7.72362404481453\\
49.44137	-7.67454726486294\\
28.433709	-4.41362857938319\\
48.50142	-7.5286433244663\\
27.54012	-4.27492103515734\\
40.633552	-6.30735182627961\\
26.822319	-4.16350022094313\\
27.984473	-4.34389582490899\\
64.169226	-9.96068187201673\\
73.534968	-11.414481183191\\
76.220508	-11.8313447058175\\
97.340446	-15.1096916127089\\
64.575708	-10.0237781276689\\
34.872152	-5.41303727529162\\
41.30736	-6.41194381763104\\
28.165144	-4.37194051964318\\
41.439192	-6.43240746811284\\
34.42656	-5.34387016149916\\
20.85509	-3.23723581927383\\
51.16746	-7.94247995540948\\
59.006172	-9.15924570724137\\
60.265986	-9.35480060565814\\
103.58249	-16.078614231709\\
65.575434	-10.1789608104893\\
76.714164	-11.9079725643211\\
40.90454	-6.34941599671442\\
46.3139	-7.18908506317959\\
19.580714	-3.03942053128308\\
6.572643	-1.02023991969823\\
34.200712	-5.30881285724819\\
55.61737	-8.6332181897947\\
49.28756	-7.65067207461622\\
56.867418	-8.82725715876638\\
59.543302	-9.24262182672139\\
105.13768	-16.3200189330926\\
66.179664	-10.2727525418642\\
66.783894	-10.3665442732391\\
67.190376	-10.4296405288913\\
83.602728	-12.9772513890186\\
98.213241	-15.2451714038239\\
95.594856	-14.8387320304791\\
62.158788	-9.64861120216926\\
33.08368	-5.13542132541231\\
19.580714	-3.03942053128308\\
26.197295	-4.06648073645729\\
32.973808	-5.11836642064157\\
26.285189	-4.08012410146311\\
33.4194	-5.18753353443403\\
34.646304	-5.37797997104065\\
55.793158	-8.66050491980634\\
49.44137	-7.67454726486294\\
54.894686	-8.52103941085795\\
32.74796	-5.0833091163906\\
20.25086	-3.1434440878989\\
42.244832	-6.55746310994607\\
63.971478	-9.92998639629403\\
50.22751	-7.79657601501285\\
71.752746	-11.1378353909025\\
48.971395	-7.60159529466462\\
42.508496	-6.59839041090967\\
51.62889	-8.01410552614965\\
74.206353	-11.5186970638476\\
87.433912	-13.571947747319\\
67.388124	-10.460336004614\\
104.83006	-16.2722685525991\\
59.367514	-9.21533509670974\\
74.877738	-11.6229129445042\\
82.609056	-12.823008439647\\
81.387108	-12.6333312992036\\
58.478808	-9.07738551720644\\
59.904644	-9.29871121618976\\
77.783004	-12.0738834826184\\
116.000185	-18.0061536020411\\
96.753293	-15.0185506623225\\
51.32127	-7.9663551456562\\
73.095516	-11.3462671522157\\
59.367514	-9.21533509670974\\
75.097464	-11.6570199599919\\
81.387108	-12.6333312992036\\
59.191726	-9.1880483666981\\
60.080432	-9.3259979462014\\
91.198448	-14.1562986555216\\
81.387108	-12.6333312992036\\
72.643857	-11.2761582870467\\
51.791245	-8.03930711585452\\
61.340246	-9.52155284461818\\
100.244473	-15.5604698267822\\
92.79508	-14.4041362002456\\
97.054804	-15.0653527719804\\
73.095516	-11.3462671522157\\
65.784168	-10.2113615904188\\
56.506076	-8.771167769298\\
47.40766	-7.35886419382295\\
26.109401	-4.05283737145147\\
19.111978	-2.96666088512557\\
19.983534	-3.1019483521997\\
41.71018	-6.47447163854765\\
64.169226	-9.96068187201673\\
56.330288	-8.74388103928636\\
49.13375	-7.62679688436949\\
63.971478	-9.92998639629403\\
55.08024	-8.54984207031468\\
47.877635	-7.43181616402126\\
20.518186	-3.18493982359811\\
33.529272	-5.20458843920477\\
33.75512	-5.23964574345574\\
35.098	-5.44809457954258\\
50.697485	-7.86952798521117\\
63.762744	-9.89758561636451\\
41.439192	-6.43240746811284\\
33.864992	-5.25670064822648\\
40.765384	-6.32781547676141\\
39.967068	-6.20389670439939\\
34.872152	-5.41303727529162\\
51.945055	-8.06318230610124\\
96.753293	-15.0185506623225\\
90.920136	-14.1130976156156\\
108.58986	-16.8558843141857\\
109.52981	-17.0017882545823\\
107.64991	-16.709980373789\\
82.367352	-12.7854898843944\\
66.783894	-10.3665442732391\\
59.367514	-9.21533509670974\\
67.388124	-10.460336004614\\
75.768849	-11.7612358406485\\
91.462112	-14.1972259564852\\
93.879032	-14.5723928819849\\
111.71733	-17.341346515869\\
130.525673	-20.260875592981\\
144.428529	-22.4189493981483\\
119.699007	-18.580304040496\\
145.24153	-22.5451476527857\\
126.599942	-19.6515031563225\\
95.490312	-14.8225041656513\\
110.77738	-17.1954425754724\\
91.198448	-14.1562986555216\\
65.377686	-10.1482653347666\\
49.911345	-7.74749923506126\\
51.005105	-7.91727836570461\\
65.179938	-10.1175698590439\\
42.244832	-6.55746310994607\\
42.508496	-6.59839041090967\\
41.973844	-6.51539893951126\\
40.90454	-6.34941599671442\\
20.45227	-3.17470799835721\\
41.168204	-6.39034329767803\\
40.765384	-6.32781547676141\\
34.872152	-5.41303727529162\\
50.38132	-7.82045120525957\\
71.301087	-11.0677265257335\\
51.791245	-8.03930711585452\\
77.331345	-12.0037746174494\\
117.00729	-18.1624816917194\\
111.40971	-17.2935961353756\\
125.174179	-19.4301887887798\\
109.2051	-16.9513850751725\\
91.198448	-14.1562986555216\\
87.433912	-13.571947747319\\
43.716956	-6.78597387365952\\
57.756124	-8.96520673826969\\
49.13375	-7.62679688436949\\
35.543592	-5.51726169333504\\
41.30736	-6.41194381763104\\
45.997735	-7.140008283228\\
13.095965	-2.03282397659067\\
27.54012	-4.27492103515734\\
49.911345	-7.74749923506126\\
58.478808	-9.07738551720644\\
73.534968	-11.414481183191\\
66.981642	-10.3972397489618\\
73.986627	-11.48459004836\\
87.170248	-13.5310204463554\\
65.179938	-10.1175698590439\\
57.404548	-8.9106332782464\\
43.987944	-6.82803804409433\\
64.773456	-10.0544736033916\\
63.564996	-9.86689014064181\\
43.050472	-6.6825187517793\\
66.783894	-10.3665442732391\\
80.890272	-12.5562098245178\\
64.773456	-10.0544736033916\\
49.59518	-7.69842245510966\\
49.757535	-7.72362404481453\\
48.663775	-7.55384491417118\\
34.200712	-5.30881285724819\\
41.71018	-6.47447163854765\\
35.427616	-5.49925929385481\\
57.404548	-8.9106332782464\\
49.59518	-7.69842245510966\\
64.366974	-9.99137734773943\\
64.575708	-10.0237781276689\\
48.34761	-7.50476813421958\\
34.310584	-5.32586776201893\\
35.763336	-5.55137150287652\\
67.190376	-10.4296405288913\\
90.392808	-14.0312430136883\\
89.587168	-13.9061873718551\\
74.206353	-11.5186970638476\\
65.981916	-10.2420570661415\\
57.941678	-8.99400939772642\\
50.065155	-7.77137442530798\\
50.697485	-7.86952798521117\\
58.117466	-9.02129612773806\\
70.409976	-10.9294036295893\\
42.779484	-6.64045458134449\\
43.585124	-6.76551022317771\\
65.981916	-10.2420570661415\\
66.586146	-10.3358487975164\\
81.870516	-12.7083684097087\\
84.086136	-13.0522884995236\\
94.40636	-14.6542474839121\\
136.797184	-21.2343722333777\\
142.033472	-22.0471761615139\\
89.587168	-13.9061873718551\\
57.218994	-8.88183061878967\\
40.90454	-6.34941599671442\\
34.310584	-5.32586776201893\\
41.168204	-6.39034329767803\\
28.433709	-4.41362857938319\\
42.911316	-6.66091823182629\\
42.508496	-6.59839041090967\\
43.182304	-6.7029824022611\\
65.784168	-10.2113615904188\\
74.206353	-11.5186970638476\\
82.609056	-12.823008439647\\
85.80492	-13.3190871146528\\
117.684797	-18.2676478611394\\
107.64991	-16.709980373789\\
74.658012	-11.5888059290166\\
73.754694	-11.4485881986786\\
59.191726	-9.1880483666981\\
59.904644	-9.29871121618976\\
80.890272	-12.5562098245178\\
58.293254	-9.04858285774971\\
51.005105	-7.91727836570461\\
62.763018	-9.74240293354418\\
35.207872	-5.46514948431333\\
65.981916	-10.2420570661415\\
86.906584	-13.4900931453918\\
50.697485	-7.86952798521117\\
53.20117	-8.25816302644947\\
98.800394	-15.3363123542103\\
82.864188	-12.8626113590802\\
69.200814	-10.7417111987388\\
108.26515	-16.8054811347759\\
105.77001	-16.4181724929958\\
65.179938	-10.1175698590439\\
50.065155	-7.77137442530798\\
59.006172	-9.15924570724137\\
77.331345	-12.0037746174494\\
118.362304	-18.3728140305593\\
126.599942	-19.6515031563225\\
117.00729	-18.1624816917194\\
97.927599	-15.2008325630953\\
59.006172	-9.15924570724137\\
57.404548	-8.9106332782464\\
42.508496	-6.59839041090967\\
34.872152	-5.41303727529162\\
35.207872	-5.46514948431333\\
42.779484	-6.64045458134449\\
55.431816	-8.60441553033797\\
48.971395	-7.60159529466462\\
35.989184	-5.58642880712749\\
73.315242	-11.3803741677034\\
67.388124	-10.460336004614\\
84.824676	-13.166928529462\\
111.085	-17.2431929559658\\
125.525737	-19.4847594547492\\
100.244473	-15.5604698267822\\
77.111619	-11.9696676019617\\
99.371678	-15.4249900356673\\
74.206353	-11.5186970638476\\
59.367514	-9.21533509670974\\
69.200814	-10.7417111987388\\
102.275705	-15.8757682497406\\
114.663482	-17.7986635921044\\
72.643857	-11.2761582870467\\
48.50142	-7.5286433244663\\
26.285189	-4.08012410146311\\
12.827455	-1.99114445423746\\
-0	0\\
33.639144	-5.22164334397551\\
48.34761	-7.50476813421958\\
40.50172	-6.28688817579781\\
33.639144	-5.22164334397551\\
33.529272	-5.20458843920477\\
40.0989	-6.22436035488119\\
26.734425	-4.14985685593731\\
34.42656	-5.34387016149916\\
50.38132	-7.82045120525957\\
65.981916	-10.2420570661415\\
80.648568	-12.5186912692652\\
67.190376	-10.4296405288913\\
76.672167	-11.9014535709865\\
108.26515	-16.8054811347759\\
110.46976	-17.1476921949789\\
126.599942	-19.6515031563225\\
131.858208	-20.4677186242236\\
82.864188	-12.8626113590802\\
66.981642	-10.3972397489618\\
69.59631	-10.8031021501842\\
94.40636	-14.6542474839121\\
107.64991	-16.709980373789\\
81.870516	-12.7083684097087\\
55.968946	-8.68779164981799\\
39.967068	-6.20389670439939\\
19.712546	-3.05988418176488\\
25.303706	-3.92777319223144\\
32.74796	-5.0833091163906\\
6.774108	-1.05151236754334\\
54.54311	-8.46646595083466\\
40.90454	-6.34941599671442\\
32.857832	-5.10036402116135\\
25.660165	-3.98310461697727\\
19.379304	-3.00815662082477\\
32.638088	-5.06625421161986\\
32.74796	-5.0833091163906\\
25.92873	-4.02479267671728\\
19.177894	-2.97689271036647\\
19.64663	-3.04965235652398\\
28.970839	-4.49700469886321\\
75.329397	-11.6930218096732\\
84.341268	-13.0918914189569\\
98.213241	-15.2451714038239\\
92.004088	-14.2813542973548\\
96.281304	-14.9452860685421\\
153.723204	-23.8617173190016\\
137.17072	-21.292354439112\\
118.691902	-18.4239759508177\\
118.014395	-18.3188097813977\\
102.577216	-15.9225703593985\\
99.958831	-15.5161309860537\\
90.656472	-14.0721703146519\\
65.784168	-10.2113615904188\\
49.59518	-7.69842245510966\\
51.62889	-8.01410552614965\\
72.863583	-11.3102653025344\\
86.906584	-13.4900931453918\\
51.16746	-7.94247995540948\\
58.293254	-9.04858285774971\\
66.179664	-10.2727525418642\\
66.783894	-10.3665442732391\\
73.534968	-11.414481183191\\
64.971204	-10.0851690791144\\
58.117466	-9.02129612773806\\
55.793158	-8.66050491980634\\
58.533408	-9.08586081391973\\
12.871393	-1.99796473971344\\
26.734425	-4.14985685593731\\
45.22014	-7.01930593253624\\
36.47352	-5.66160996663166\\
-6.170934	0.957884553994958\\
12.917772	-2.00516392993809\\
39.827912	-6.18229618444638\\
40.90454	-6.34941599671442\\
47.723825	-7.40794097377454\\
49.59518	-7.69842245510966\\
49.911345	-7.74749923506126\\
62.56527	-9.71170745782148\\
42.508496	-6.59839041090967\\
48.031445	-7.45569135426799\\
59.741868	-9.27344427666957\\
25.660165	-3.98310461697727\\
24.766576	-3.84439707275142\\
19.0424	-2.95586062514906\\
18.976484	-2.94562879990816\\
13.095965	-2.03282397659067\\
20.584102	-3.19517164883901\\
62.356536	-9.67930667789196\\
68.176095	-10.5826489721318\\
47.56147	-7.38273938406967\\
47.091495	-7.30978741387135\\
40.765384	-6.32781547676141\\
33.529272	-5.20458843920477\\
46.46771	-7.21296025342631\\
32.973808	-5.11836642064157\\
32.302368	-5.01414200259815\\
32.638088	-5.06625421161986\\
19.44522	-3.01838844606567\\
26.734425	-4.14985685593731\\
40.50172	-6.28688817579781\\
39.967068	-6.20389670439939\\
19.913956	-3.09114809222319\\
25.92873	-4.02479267671728\\
32.41224	-5.03119690736889\\
34.646304	-5.37797997104065\\
58.117466	-9.02129612773806\\
79.413192	-12.3269297646411\\
35.653464	-5.53431659810578\\
41.71018	-6.47447163854765\\
57.218994	-8.88183061878967\\
50.38132	-7.82045120525957\\
69.738591	-10.8251877489326\\
41.973844	-6.51539893951126\\
41.439192	-6.43240746811284\\
63.1695	-9.8054991891964\\
40.90454	-6.34941599671442\\
53.644638	-8.32700044188627\\
33.974864	-5.27375555299722\\
41.571024	-6.45287111859464\\
47.877635	-7.43181616402126\\
47.56147	-7.38273938406967\\
41.036372	-6.36987964719623\\
49.44137	-7.67454726486294\\
65.575434	-10.1789608104893\\
68.794332	-10.6786149430866\\
114.333884	-17.747501671846\\
80.648568	-12.5186912692652\\
64.366974	-9.99137734773943\\
58.293254	-9.04858285774971\\
59.728856	-9.27142448617812\\
81.387108	-12.6333312992036\\
59.543302	-9.24262182672139\\
64.971204	-10.0851690791144\\
82.072744	-12.7397592943925\\
26.734425	-4.14985685593731\\
21.0565	-3.26849972973213\\
64.169226	-9.96068187201673\\
42.376664	-6.57792676042787\\
57.941678	-8.99400939772642\\
65.981916	-10.2420570661415\\
72.424131	-11.2420512715591\\
63.762744	-9.89758561636451\\
56.867418	-8.82725715876638\\
51.005105	-7.91727836570461\\
72.192198	-11.2060494218777\\
50.53513	-7.84432639550629\\
64.773456	-10.0544736033916\\
63.564996	-9.86689014064181\\
42.376664	-6.57792676042787\\
48.817585	-7.5777201044179\\
27.808685	-4.31660909489735\\
35.879312	-5.56937390235675\\
65.981916	-10.2420570661415\\
81.131976	-12.5937283797703\\
66.179664	-10.2727525418642\\
79.171488	-12.2894112093886\\
55.968946	-8.68779164981799\\
41.973844	-6.51539893951126\\
42.376664	-6.57792676042787\\
43.050472	-6.6825187517793\\
58.654596	-9.10467224721808\\
74.877738	-11.6229129445042\\
81.387108	-12.6333312992036\\
71.301087	-11.0677265257335\\
43.987944	-6.82803804409433\\
53.20117	-8.25816302644947\\
119.803154	-18.5964702809136\\
74.426079	-11.5528040793352\\
82.864188	-12.8626113590802\\
96.467651	-14.974211821594\\
65.575434	-10.1789608104893\\
58.654596	-9.10467224721808\\
58.293254	-9.04858285774971\\
58.654596	-9.10467224721808\\
52.098865	-8.08705749634797\\
80.890272	-12.5562098245178\\
66.377412	-10.3034480175869\\
59.191726	-9.1880483666981\\
66.377412	-10.3034480175869\\
65.575434	-10.1789608104893\\
51.47508	-7.99023033590293\\
58.117466	-9.02129612773806\\
28.878062	-4.48260336913484\\
58.117466	-9.02129612773806\\
72.643857	-11.2761582870467\\
55.61737	-8.6332181897947\\
39.564248	-6.14136888348278\\
18.976484	-2.94562879990816\\
12.827455	-1.99114445423746\\
20.785512	-3.22643555929732\\
58.293254	-9.04858285774971\\
80.648568	-12.5186912692652\\
64.366974	-9.99137734773943\\
64.169226	-9.96068187201673\\
78.191244	-12.1372526241977\\
70.409976	-10.9294036295893\\
40.90454	-6.34941599671442\\
34.310584	-5.32586776201893\\
20.65368	-3.20597190881552\\
48.1938	-7.48089294397286\\
27.808685	-4.31660909489735\\
48.50142	-7.5286433244663\\
41.168204	-6.39034329767803\\
40.0989	-6.22436035488119\\
33.75512	-5.23964574345574\\
28.340932	-4.39922724965482\\
56.1545	-8.71659430927472\\
63.564996	-9.86689014064181\\
35.317744	-5.48220438908407\\
64.773456	-10.0544736033916\\
50.53513	-7.84432639550629\\
59.367514	-9.21533509670974\\
88.781528	-13.7811317300219\\
51.62889	-8.01410552614965\\
73.986627	-11.48459004836\\
71.301087	-11.0677265257335\\
21.928056	-3.40378719680626\\
72.192198	-11.2060494218777\\
79.413192	-12.3269297646411\\
58.117466	-9.02129612773806\\
95.880498	-14.8830708712076\\
80.406864	-12.4811727140127\\
80.890272	-12.5562098245178\\
67.794606	-10.5234322602663\\
97.626088	-15.1540304534374\\
81.387108	-12.6333312992036\\
72.863583	-11.3102653025344\\
65.981916	-10.2420570661415\\
50.851295	-7.89340317545789\\
57.404548	-8.9106332782464\\
49.59518	-7.69842245510966\\
35.427616	-5.49925929385481\\
49.28756	-7.65067207461622\\
44.119776	-6.84850169457613\\
68.794332	-10.6786149430866\\
99.086036	-15.3806511949388\\
83.602728	-12.9772513890186\\
90.920136	-14.1130976156156\\
87.712224	-13.6151487872251\\
64.366974	-9.99137734773943\\
42.244832	-6.55746310994607\\
41.439192	-6.43240746811284\\
27.628014	-4.28856440016316\\
35.543592	-5.51726169333504\\
58.293254	-9.04858285774971\\
73.986627	-11.48459004836\\
73.754694	-11.4485881986786\\
71.972472	-11.1719424063901\\
51.791245	-8.03930711585452\\
60.803116	-9.43817672513816\\
107.34229	-16.6622299932956\\
91.462112	-14.1972259564852\\
86.54346	-13.4337271445912\\
109.83743	-17.0495386350757\\
142.265862	-22.0832489491183\\
43.182304	-6.7029824022611\\
42.244832	-6.55746310994607\\
50.697485	-7.86952798521117\\
52.26122	-8.11225908605284\\
76.000782	-11.7972376903299\\
105.77001	-16.4181724929958\\
88.781528	-13.7811317300219\\
63.762744	-9.89758561636451\\
55.61737	-8.6332181897947\\
35.098	-5.44809457954258\\
48.1938	-7.48089294397286\\
34.42656	-5.34387016149916\\
20.45227	-3.17470799835721\\
33.08368	-5.13542132541231\\
20.45227	-3.17470799835721\\
27.715908	-4.30220776516898\\
70.629702	-10.9635106450769\\
42.508496	-6.59839041090967\\
56.681864	-8.79845449930965\\
42.647652	-6.61999093086268\\
57.941678	-8.99400939772642\\
57.043206	-8.85454388877803\\
65.377686	-10.1482653347666\\
81.870516	-12.7083684097087\\
65.784168	-10.2113615904188\\
80.151732	-12.4415697945795\\
72.643857	-11.2761582870467\\
79.910028	-12.4040512393269\\
59.728856	-9.27142448617812\\
88.781528	-13.7811317300219\\
80.151732	-12.4415697945795\\
43.856112	-6.80757439361253\\
58.117466	-9.02129612773806\\
63.762744	-9.89758561636451\\
50.22751	-7.79657601501285\\
81.628812	-12.6708498544561\\
81.870516	-12.7083684097087\\
90.392808	-14.0312430136883\\
86.301756	-13.3962085893386\\
131.46392	-20.4065151848373\\
129.201952	-20.0554007167793\\
66.981642	-10.3972397489618\\
74.877738	-11.6229129445042\\
69.59631	-10.8031021501842\\
99.086036	-15.3806511949388\\
44.925416	-6.97355733640936\\
68.794332	-10.6786149430866\\
97.626088	-15.1540304534374\\
28.702274	-4.4553166391232\\
42.779484	-6.64045458134449\\
56.330288	-8.74388103928636\\
61.752306	-9.58551494651704\\
34.09084	-5.29175795247745\\
33.75512	-5.23964574345574\\
26.822319	-4.16350022094313\\
27.090884	-4.20518828068314\\
34.646304	-5.37797997104065\\
42.244832	-6.55746310994607\\
43.453292	-6.74504657269591\\
58.478808	-9.07738551720644\\
81.131976	-12.5937283797703\\
67.794606	-10.5234322602663\\
90.392808	-14.0312430136883\\
81.131976	-12.5937283797703\\
73.315242	-11.3803741677034\\
51.16746	-7.94247995540948\\
65.981916	-10.2420570661415\\
59.904644	-9.29871121618976\\
95.880498	-14.8830708712076\\
81.131976	-12.5937283797703\\
73.315242	-11.3803741677034\\
45.196404	-7.01562150684417\\
75.329397	-11.6930218096732\\
90.114496	-13.9880419737823\\
67.190376	-10.4296405288913\\
79.668324	-12.3665326840744\\
55.61737	-8.6332181897947\\
27.090884	-4.20518828068314\\
34.646304	-5.37797997104065\\
34.76228	-5.39598237052087\\
56.506076	-8.771167769298\\
35.098	-5.44809457954258\\
49.59518	-7.69842245510966\\
52.26122	-8.11225908605284\\
92.79508	-14.4041362002456\\
130.343312	-20.2325685676343\\
76.440234	-11.8654517213051\\
91.462112	-14.1972259564852\\
89.308856	-13.8629863319491\\
66.586146	-10.3358487975164\\
59.728856	-9.27142448617812\\
66.586146	-10.3358487975164\\
75.329397	-11.6930218096732\\
93.60072	-14.5291918420788\\
118.014395	-18.3188097813977\\
124.80309	-19.3725864191454\\
94.142696	-14.6133201829485\\
123.025769	-19.0967013856333\\
99.371678	-15.4249900356673\\
90.920136	-14.1130976156156\\
75.097464	-11.6570199599919\\
68.794332	-10.6786149430866\\
85.06638	-13.2044470847145\\
109.975866	-17.0710273928743\\
43.182304	-6.7029824022611\\
50.065155	-7.77137442530798\\
72.643857	-11.2761582870467\\
58.478808	-9.07738551720644\\
50.065155	-7.77137442530798\\
21.455658	-3.33045911591314\\
44.925416	-6.97355733640936\\
87.040296	-13.5108486192769\\
136.797184	-21.2343722333777\\
143.637501	-22.2961619071498\\
126.599942	-19.6515031563225\\
95.490312	-14.8225041656513\\
112.9649	-17.5350008367591\\
143.241987	-22.2347681616505\\
110.77738	-17.1954425754724\\
95.212	-14.7793031257453\\
116.348094	-18.0601578512027\\
83.602728	-12.9772513890186\\
84.582972	-13.1294099742094\\
98.213241	-15.2451714038239\\
73.095516	-11.3462671522157\\
44.119776	-6.84850169457613\\
65.377686	-10.1482653347666\\
65.179938	-10.1175698590439\\
36.215032	-5.62148611137846\\
58.478808	-9.07738551720644\\
44.925416	-6.97355733640936\\
82.125648	-12.7479713291419\\
58.293254	-9.04858285774971\\
51.005105	-7.91727836570461\\
59.191726	-9.1880483666981\\
74.658012	-11.5888059290166\\
74.426079	-11.5528040793352\\
66.179664	-10.2727525418642\\
45.057248	-6.99402098689116\\
82.864188	-12.8626113590802\\
88.781528	-13.7811317300219\\
60.441774	-9.38208733566978\\
84.086136	-13.0522884995236\\
74.658012	-11.5888059290166\\
58.117466	-9.02129612773806\\
56.1545	-8.71659430927472\\
33.974864	-5.27375555299722\\
28.07725	-4.35829715463736\\
35.653464	-5.53431659810578\\
55.61737	-8.6332181897947\\
27.090884	-4.20518828068314\\
33.864992	-5.25670064822648\\
27.984473	-4.34389582490899\\
71.301087	-11.0677265257335\\
65.981916	-10.2420570661415\\
74.877738	-11.6229129445042\\
90.114496	-13.9880419737823\\
82.864188	-12.8626113590802\\
88.239552	-13.6970033891523\\
61.96104	-9.61791572644656\\
27.271555	-4.23323297541733\\
47.723825	-7.40794097377454\\
41.30736	-6.41194381763104\\
42.911316	-6.66091823182629\\
65.575434	-10.1789608104893\\
67.585872	-10.4910314803367\\
96.467651	-14.974211821594\\
95.309214	-14.7943931897505\\
60.441774	-9.38208733566978\\
97.054804	-15.0653527719804\\
101.70259	-15.7868063509158\\
35.317744	-5.48220438908407\\
42.508496	-6.59839041090967\\
58.117466	-9.02129612773806\\
67.794606	-10.5234322602663\\
89.587168	-13.9061873718551\\
29.327298	-4.55233612360904\\
65.377686	-10.1482653347666\\
58.654596	-9.10467224721808\\
67.388124	-10.460336004614\\
85.80492	-13.3190871146528\\
117.684797	-18.2676478611394\\
101.704421	-15.7870905682836\\
91.198448	-14.1562986555216\\
80.648568	-12.5186912692652\\
51.16746	-7.94247995540948\\
55.793158	-8.66050491980634\\
-0	0\\
41.71018	-6.47447163854765\\
62.56527	-9.71170745782148\\
43.585124	-6.76551022317771\\
52.731195	-8.18521105625116\\
107.95753	-16.7577307542825\\
103.450011	-16.0580501505134\\
119.0215	-18.4751378710761\\
113.326779	-17.5911735821677\\
66.586146	-10.3358487975164\\
65.784168	-10.2113615904188\\
51.791245	-8.03930711585452\\
52.885005	-8.20908624649788\\
101.40291	-15.7402884586257\\
108.58986	-16.8558843141857\\
90.114496	-13.9880419737823\\
60.080432	-9.3259979462014\\
69.003066	-10.7110157230161\\
97.927599	-15.2008325630953\\
80.406864	-12.4811727140127\\
42.779484	-6.64045458134449\\
41.439192	-6.43240746811284\\
40.50172	-6.28688817579781\\
39.69608	-6.16183253396458\\
26.822319	-4.16350022094313\\
34.872152	-5.41303727529162\\
63.367248	-9.83619466491911\\
55.793158	-8.66050491980634\\
34.42656	-5.34387016149916\\
39.564248	-6.14136888348278\\
30.849616	-4.78863826174056\\
6.125757	-0.950871944478177\\
6.327222	-0.982144392323283\\
40.230732	-6.244824005363\\
41.30736	-6.41194381763104\\
42.244832	-6.55746310994607\\
64.773456	-10.0544736033916\\
66.981642	-10.3972397489618\\
85.563216	-13.2815685594003\\
118.691902	-18.4239759508177\\
135.655824	-21.0572043825227\\
119.699007	-18.580304040496\\
107.01758	-16.6118268138858\\
74.426079	-11.5528040793352\\
66.981642	-10.3972397489618\\
75.768849	-11.7612358406485\\
75.329397	-11.6930218096732\\
73.534968	-11.414481183191\\
50.22751	-7.79657601501285\\
42.113	-6.53699945946427\\
35.763336	-5.55137150287652\\
57.941678	-8.99400939772642\\
62.158788	-9.64861120216926\\
40.0989	-6.22436035488119\\
20.382692	-3.16390773838071\\
34.982024	-5.43009218006236\\
41.571024	-6.45287111859464\\
28.07725	-4.35829715463736\\
48.971395	-7.60159529466462\\
49.911345	-7.74749923506126\\
52.098865	-8.08705749634797\\
89.308856	-13.8629863319491\\
67.794606	-10.5234322602663\\
93.879032	-14.5723928819849\\
127.322589	-19.7636761919263\\
142.428986	-22.1085699070131\\
92.004088	-14.2813542973548\\
66.981642	-10.3972397489618\\
49.59518	-7.69842245510966\\
42.779484	-6.64045458134449\\
56.506076	-8.771167769298\\
50.697485	-7.86952798521117\\
51.16746	-7.94247995540948\\
57.404548	-8.9106332782464\\
43.453292	-6.74504657269591\\
56.1545	-8.71659430927472\\
49.28756	-7.65067207461622\\
41.571024	-6.45287111859464\\
53.644638	-8.32700044188627\\
26.822319	-4.16350022094313\\
13.23022	-2.05366373776728\\
6.595842	-1.02384098945009\\
19.712546	-3.05988418176488\\
32.07652	-4.97908469834718\\
12.827455	-1.99114445423746\\
13.095965	-2.03282397659067\\
39.564248	-6.14136888348278\\
32.857832	-5.10036402116135\\
34.09084	-5.29175795247745\\
56.506076	-8.771167769298\\
63.367248	-9.83619466491911\\
48.971395	-7.60159529466462\\
43.182304	-6.7029824022611\\
81.870516	-12.7083684097087\\
118.377391	-18.3751559133709\\
81.387108	-12.6333312992036\\
63.1695	-9.8054991891964\\
53.46885	-8.29971371187462\\
20.518186	-3.18493982359811\\
43.050472	-6.6825187517793\\
50.697485	-7.86952798521117\\
60.265986	-9.35480060565814\\
83.602728	-12.9772513890186\\
81.387108	-12.6333312992036\\
64.575708	-10.0237781276689\\
57.043206	-8.85454388877803\\
57.218994	-8.88183061878967\\
49.59518	-7.69842245510966\\
40.90454	-6.34941599671442\\
27.00299	-4.19154491567732\\
26.822319	-4.16350022094313\\
26.377966	-4.09452543119148\\
19.24381	-2.98712453560737\\
32.74796	-5.0833091163906\\
40.90454	-6.34941599671442\\
61.96104	-9.61791572644656\\
54.54311	-8.46646595083466\\
28.07725	-4.35829715463736\\
58.830384	-9.13195897722972\\
72.643857	-11.2761582870467\\
36.550752	-5.67359832040017\\
81.628812	-12.6708498544561\\
73.095516	-11.3462671522157\\
41.973844	-6.51539893951126\\
48.663775	-7.55384491417118\\
70.629702	-10.9635106450769\\
54.718898	-8.4937526808463\\
33.974864	-5.27375555299722\\
53.820426	-8.35428717189791\\
26.016624	-4.0384360417231\\
19.111978	-2.96666088512557\\
13.005648	-2.01880450089005\\
26.646531	-4.13621349093149\\
40.0989	-6.22436035488119\\
40.50172	-6.28688817579781\\
48.1938	-7.48089294397286\\
47.40766	-7.35886419382295\\
26.016624	-4.0384360417231\\
19.0424	-2.95586062514906\\
12.6932	-1.97030469306086\\
25.748059	-3.99674798198309\\
26.016624	-4.0384360417231\\
27.447343	-4.26051970542897\\
49.44137	-7.67454726486294\\
63.1695	-9.8054991891964\\
34.982024	-5.43009218006236\\
49.911345	-7.74749923506126\\
66.377412	-10.3034480175869\\
82.609056	-12.823008439647\\
83.347596	-12.9376484695853\\
82.125648	-12.7479713291419\\
66.981642	-10.3972397489618\\
74.658012	-11.5888059290166\\
82.125648	-12.7479713291419\\
90.392808	-14.0312430136883\\
85.321512	-13.2440500041477\\
116.677692	-18.111319771461\\
100.244473	-15.5604698267822\\
98.800394	-15.3363123542103\\
75.097464	-11.6570199599919\\
82.367352	-12.7854898843944\\
83.105892	-12.9001299143328\\
80.406864	-12.4811727140127\\
58.117466	-9.02129612773806\\
44.251608	-6.86896534505793\\
67.585872	-10.4910314803367\\
81.870516	-12.7083684097087\\
74.877738	-11.6229129445042\\
66.783894	-10.3665442732391\\
51.32127	-7.9663551456562\\
71.081361	-11.0336195102459\\
50.22751	-7.79657601501285\\
89.587168	-13.9061873718551\\
97.054804	-15.0653527719804\\
64.575708	-10.0237781276689\\
35.763336	-5.55137150287652\\
55.793158	-8.66050491980634\\
13.49873	-2.09534326012049\\
20.518186	-3.18493982359811\\
40.0989	-6.22436035488119\\
26.285189	-4.08012410146311\\
19.84804	-3.08091626698229\\
45.52776	-7.06705631302968\\
26.377966	-4.09452543119148\\
39.425092	-6.11976836352977\\
31.856776	-4.9449748888057\\
25.92873	-4.02479267671728\\
26.197295	-4.06648073645729\\
33.08368	-5.13542132541231\\
33.309528	-5.17047862966328\\
40.0989	-6.22436035488119\\
34.200712	-5.30881285724819\\
41.168204	-6.39034329767803\\
41.439192	-6.43240746811284\\
43.716956	-6.78597387365952\\
87.433912	-13.571947747319\\
57.404548	-8.9106332782464\\
50.38132	-7.82045120525957\\
63.564996	-9.86689014064181\\
40.633552	-6.30735182627961\\
34.536432	-5.3609250662699\\
62.158788	-9.64861120216926\\
46.783875	-7.26203703337791\\
27.178778	-4.21883164568896\\
61.752306	-9.58551494651704\\
48.663775	-7.55384491417118\\
47.091495	-7.30978741387135\\
39.161428	-6.07884106256616\\
12.6932	-1.97030469306086\\
18.976484	-2.94562879990816\\
26.109401	-4.05283737145147\\
26.285189	-4.08012410146311\\
39.827912	-6.18229618444638\\
48.031445	-7.45569135426799\\
49.911345	-7.74749923506126\\
56.681864	-8.79845449930965\\
43.050472	-6.6825187517793\\
66.377412	-10.3034480175869\\
88.239552	-13.6970033891523\\
73.095516	-11.3462671522157\\
68.190102	-10.5848232117117\\
90.656472	-14.0721703146519\\
76.220508	-11.8313447058175\\
106.70996	-16.5640764333924\\
99.371678	-15.4249900356673\\
124.099974	-19.2634450872066\\
128.060592	-19.8782328659244\\
79.171488	-12.2894112093886\\
34.982024	-5.43009218006236\\
35.763336	-5.55137150287652\\
50.851295	-7.89340317545789\\
73.315242	-11.3803741677034\\
72.192198	-11.2060494218777\\
47.877635	-7.43181616402126\\
40.765384	-6.32781547676141\\
64.169226	-9.96068187201673\\
81.628812	-12.6708498544561\\
89.850832	-13.9471146728187\\
79.668324	-12.3665326840744\\
56.506076	-8.771167769298\\
35.653464	-5.53431659810578\\
66.586146	-10.3358487975164\\
75.329397	-11.6930218096732\\
91.198448	-14.1562986555216\\
89.587168	-13.9061873718551\\
75.768849	-11.7612358406485\\
123.728885	-19.2058427175722\\
123.377327	-19.1512720516027\\
110.77738	-17.1954425754724\\
135.282288	-20.9992221767884\\
117.684797	-18.2676478611394\\
102.577216	-15.9225703593985\\
109.83743	-17.0495386350757\\
97.054804	-15.0653527719804\\
65.784168	-10.2113615904188\\
51.62889	-8.01410552614965\\
59.904644	-9.29871121618976\\
75.549123	-11.7271288251609\\
74.206353	-11.5186970638476\\
52.26122	-8.11225908605284\\
82.125648	-12.7479713291419\\
73.095516	-11.3462671522157\\
51.62889	-8.01410552614965\\
59.191726	-9.1880483666981\\
75.329397	-11.6930218096732\\
84.086136	-13.0522884995236\\
89.850832	-13.9471146728187\\
65.981916	-10.2420570661415\\
56.1545	-8.71659430927472\\
49.59518	-7.69842245510966\\
70.19025	-10.8952966141016\\
48.50142	-7.5286433244663\\
41.168204	-6.39034329767803\\
14.079688	-2.18552259030289\\
43.314136	-6.7234460527429\\
58.478808	-9.07738551720644\\
73.534968	-11.414481183191\\
71.081361	-11.0336195102459\\
48.50142	-7.5286433244663\\
33.75512	-5.23964574345574\\
33.529272	-5.20458843920477\\
41.571024	-6.45287111859464\\
48.971395	-7.60159529466462\\
55.968946	-8.68779164981799\\
36.215032	-5.62148611137846\\
81.870516	-12.7083684097087\\
83.105892	-12.9001299143328\\
91.462112	-14.1972259564852\\
100.831626	-15.6516107771686\\
105.46239	-16.3704221125023\\
73.534968	-11.414481183191\\
72.192198	-11.2060494218777\\
43.050472	-6.6825187517793\\
51.005105	-7.91727836570461\\
72.424131	-11.2420512715591\\
63.762744	-9.89758561636451\\
34.76228	-5.39598237052087\\
36.099056	-5.60348371189823\\
51.62889	-8.01410552614965\\
73.534968	-11.414481183191\\
68.398836	-10.6172239916412\\
116.677692	-18.111319771461\\
114.333884	-17.747501671846\\
73.986627	-11.48459004836\\
49.911345	-7.74749923506126\\
42.113	-6.53699945946427\\
50.851295	-7.89340317545789\\
79.910028	-12.4040512393269\\
73.095516	-11.3462671522157\\
81.131976	-12.5937283797703\\
89.308856	-13.8629863319491\\
74.426079	-11.5528040793352\\
67.388124	-10.460336004614\\
80.890272	-12.5562098245178\\
58.830384	-9.13195897722972\\
72.424131	-11.2420512715591\\
42.779484	-6.64045458134449\\
43.585124	-6.76551022317771\\
67.388124	-10.460336004614\\
83.844432	-13.0147699442711\\
88.781528	-13.7811317300219\\
53.824955	-8.35499018689451\\
94.142696	-14.6133201829485\\
83.602728	-12.9772513890186\\
72.424131	-11.2420512715591\\
43.856112	-6.80757439361253\\
102.64254	-15.9327102913124\\
50.697485	-7.86952798521117\\
65.784168	-10.2113615904188\\
71.752746	-11.1378353909025\\
50.53513	-7.84432639550629\\
38.003504	-5.89910206125776\\
108.58986	-16.8558843141857\\
128.434128	-19.9362150716587\\
75.549123	-11.7271288251609\\
98.800394	-15.3363123542103\\
99.371678	-15.4249900356673\\
82.609056	-12.823008439647\\
74.658012	-11.5888059290166\\
64.971204	-10.0851690791144\\
51.005105	-7.91727836570461\\
44.251608	-6.86896534505793\\
68.794332	-10.6786149430866\\
109.2051	-16.9513850751725\\
112.65728	-17.4872504562656\\
136.02936	-21.1151865882571\\
101.704421	-15.7870905682836\\
92.531416	-14.363208899282\\
69.003066	-10.7110157230161\\
83.347596	-12.9376484695853\\
58.478808	-9.07738551720644\\
50.38132	-7.82045120525957\\
42.244832	-6.55746310994607\\
40.90454	-6.34941599671442\\
20.65368	-3.20597190881552\\
49.44137	-7.67454726486294\\
62.763018	-9.74240293354418\\
33.864992	-5.25670064822648\\
33.529272	-5.20458843920477\\
35.427616	-5.49925929385481\\
36.880368	-5.7247630347124\\
85.016992	-13.1967808218193\\
35.427616	-5.49925929385481\\
21.323826	-3.30999546543134\\
42.647652	-6.61999093086268\\
50.22751	-7.79657601501285\\
57.218994	-8.88183061878967\\
48.817585	-7.5777201044179\\
40.765384	-6.32781547676141\\
33.309528	-5.17047862966328\\
26.734425	-4.14985685593731\\
27.984473	-4.34389582490899\\
64.575708	-10.0237781276689\\
63.762744	-9.89758561636451\\
48.34761	-7.50476813421958\\
33.309528	-5.17047862966328\\
26.646531	-4.13621349093149\\
48.34761	-7.50476813421958\\
63.971478	-9.92998639629403\\
64.575708	-10.0237781276689\\
56.506076	-8.771167769298\\
34.872152	-5.41303727529162\\
42.244832	-6.55746310994607\\
42.911316	-6.66091823182629\\
66.981642	-10.3972397489618\\
91.462112	-14.1972259564852\\
98.213241	-15.2451714038239\\
79.910028	-12.4040512393269\\
43.182304	-6.7029824022611\\
71.752746	-11.1378353909025\\
57.580336	-8.93792000825804\\
56.1545	-8.71659430927472\\
35.098	-5.44809457954258\\
56.506076	-8.771167769298\\
49.44137	-7.67454726486294\\
48.34761	-7.50476813421958\\
33.193552	-5.15247623018306\\
26.822319	-4.16350022094313\\
41.30736	-6.41194381763104\\
57.404548	-8.9106332782464\\
80.406864	-12.4811727140127\\
82.125648	-12.7479713291419\\
74.658012	-11.5888059290166\\
82.864188	-12.8626113590802\\
85.06638	-13.2044470847145\\
112.65728	-17.4872504562656\\
144.428529	-22.4189493981483\\
148.20327	-23.0048843796651\\
85.563216	-13.2815685594003\\
110.14505	-17.0972890155692\\
104.608448	-16.2378687823569\\
133.74664	-20.7608508865472\\
88.781528	-13.7811317300219\\
42.244832	-6.55746310994607\\
34.76228	-5.39598237052087\\
40.633552	-6.30735182627961\\
32.41224	-5.03119690736889\\
19.177894	-2.97689271036647\\
13.139903	-2.03964426206665\\
27.00299	-4.19154491567732\\
48.031445	-7.45569135426799\\
54.181768	-8.41037656136629\\
26.822319	-4.16350022094313\\
40.765384	-6.32781547676141\\
34.646304	-5.37797997104065\\
48.817585	-7.5777201044179\\
41.168204	-6.39034329767803\\
26.646531	-4.13621349093149\\
19.712546	-3.05988418176488\\
20.115366	-3.1224120026815\\
34.200712	-5.30881285724819\\
40.633552	-6.30735182627961\\
19.712546	-3.05988418176488\\
20.181282	-3.1326438279224\\
53.107508	-8.24362432240625\\
39.827912	-6.18229618444638\\
54.00598	-8.38308983135464\\
41.571024	-6.45287111859464\\
49.44137	-7.67454726486294\\
69.286932	-10.7550788837636\\
56.506076	-8.771167769298\\
71.752746	-11.1378353909025\\
63.762744	-9.89758561636451\\
41.973844	-6.51539893951126\\
35.879312	-5.56937390235675\\
81.387108	-12.6333312992036\\
81.870516	-12.7083684097087\\
82.609056	-12.823008439647\\
89.308856	-13.8629863319491\\
66.377412	-10.3034480175869\\
66.179664	-10.2727525418642\\
67.585872	-10.4910314803367\\
74.206353	-11.5186970638476\\
52.731195	-8.18521105625116\\
106.07763	-16.4659228734892\\
83.105892	-12.9001299143328\\
95.753976	-14.8634314666149\\
151.612641	-23.5341047245494\\
98.800394	-15.3363123542103\\
50.851295	-7.89340317545789\\
43.050472	-6.6825187517793\\
36.324904	-5.6385410161492\\
59.191726	-9.1880483666981\\
82.864188	-12.8626113590802\\
99.086036	-15.3806511949388\\
92.004088	-14.2813542973548\\
97.340446	-15.1096916127089\\
59.006172	-9.15924570724137\\
59.367514	-9.21533509670974\\
66.783894	-10.3665442732391\\
72.192198	-11.2060494218777\\
34.872152	-5.41303727529162\\
28.340932	-4.39922724965482\\
42.779484	-6.64045458134449\\
50.851295	-7.89340317545789\\
67.190376	-10.4296405288913\\
82.125648	-12.7479713291419\\
65.575434	-10.1789608104893\\
50.22751	-7.79657601501285\\
43.856112	-6.80757439361253\\
67.388124	-10.460336004614\\
92.267752	-14.3222815983184\\
101.990063	-15.8314294090121\\
111.40971	-17.2935961353756\\
134.514464	-20.8800365316678\\
94.142696	-14.6133201829485\\
113.656377	-17.642335502426\\
72.643857	-11.2761582870467\\
52.098865	-8.08705749634797\\
92.267752	-14.3222815983184\\
96.467651	-14.974211821594\\
67.794606	-10.5234322602663\\
70.20054	-10.8968938815591\\
104.608448	-16.2378687823569\\
126.248384	-19.596932490353\\
104.83006	-16.2722685525991\\
42.779484	-6.64045458134449\\
35.653464	-5.53431659810578\\
71.520813	-11.1018335412211\\
35.653464	-5.53431659810578\\
48.50142	-7.5286433244663\\
34.646304	-5.37797997104065\\
54.894686	-8.52103941085795\\
20.584102	-3.19517164883901\\
27.359449	-4.24687634042315\\
41.439192	-6.43240746811284\\
57.218994	-8.88183061878967\\
64.773456	-10.0544736033916\\
73.754694	-11.4485881986786\\
83.347596	-12.9376484695853\\
99.086036	-15.3806511949388\\
81.131976	-12.5937283797703\\
50.697485	-7.86952798521117\\
44.251608	-6.86896534505793\\
64.773456	-10.0544736033916\\
49.757535	-7.72362404481453\\
41.71018	-6.47447163854765\\
28.340932	-4.39922724965482\\
28.433709	-4.41362857938319\\
34.310584	-5.32586776201893\\
26.197295	-4.06648073645729\\
19.177894	-2.97689271036647\\
26.377966	-4.09452543119148\\
27.271555	-4.23323297541733\\
48.34761	-7.50476813421958\\
40.50172	-6.28688817579781\\
33.639144	-5.22164334397551\\
27.715908	-4.30220776516898\\
57.043206	-8.85454388877803\\
56.506076	-8.771167769298\\
41.71018	-6.47447163854765\\
56.330288	-8.74388103928636\\
49.911345	-7.74749923506126\\
64.169226	-9.96068187201673\\
57.580336	-8.93792000825804\\
64.773456	-10.0544736033916\\
56.506076	-8.771167769298\\
41.571024	-6.45287111859464\\
41.842012	-6.49493528902945\\
41.571024	-6.45287111859464\\
48.817585	-7.5777201044179\\
50.065155	-7.77137442530798\\
59.728856	-9.27142448617812\\
88.781528	-13.7811317300219\\
58.654596	-9.10467224721808\\
67.992354	-10.554127735989\\
97.626088	-15.1540304534374\\
65.377686	-10.1482653347666\\
56.506076	-8.771167769298\\
48.1938	-7.48089294397286\\
40.90454	-6.34941599671442\\
27.00299	-4.19154491567732\\
20.04945	-3.1121801774406\\
34.982024	-5.43009218006236\\
60.543846	-9.3979314837672\\
39.161428	-6.07884106256616\\
53.293062	-8.27242698186298\\
41.036372	-6.36987964719623\\
47.723825	-7.40794097377454\\
13.320537	-2.0676832134679\\
25.748059	-3.99674798198309\\
13.095965	-2.03282397659067\\
13.76724	-2.1370227824737\\
50.22751	-7.79657601501285\\
64.169226	-9.96068187201673\\
41.973844	-6.51539893951126\\
42.113	-6.53699945946427\\
56.330288	-8.74388103928636\\
56.867418	-8.82725715876638\\
57.756124	-8.96520673826969\\
73.315242	-11.3803741677034\\
73.986627	-11.48459004836\\
80.648568	-12.5186912692652\\
71.752746	-11.1378353909025\\
48.663775	-7.55384491417118\\
41.71018	-6.47447163854765\\
42.508496	-6.59839041090967\\
55.431816	-8.60441553033797\\
47.56147	-7.38273938406967\\
49.44137	-7.67454726486294\\
50.697485	-7.86952798521117\\
72.643857	-11.2761582870467\\
65.784168	-10.2113615904188\\
75.329397	-11.6930218096732\\
95.023572	-14.750054349022\\
52.56884	-8.16000946654628\\
53.35498	-8.28203821669619\\
83.105892	-12.9001299143328\\
83.347596	-12.9376484695853\\
98.800394	-15.3363123542103\\
96.01764	-14.9043587675785\\
146.845559	-22.7941333984216\\
166.6275	-25.8647894339487\\
103.735653	-16.102388991242\\
121.713217	-18.8929602198527\\
146.450045	-22.7327396529223\\
139.45344	-21.6466901408219\\
163.622628	-25.3983575331162\\
158.825664	-24.6537477671278\\
174.138555	-27.0306945576636\\
136.797184	-21.2343722333777\\
93.073392	-14.4473372401516\\
68.398836	-10.6172239916412\\
82.367352	-12.7854898843944\\
58.478808	-9.07738551720644\\
52.098865	-8.08705749634797\\
61.340246	-9.52155284461818\\
82.367352	-12.7854898843944\\
65.981916	-10.2420570661415\\
51.32127	-7.9663551456562\\
49.59518	-7.69842245510966\\
49.44137	-7.67454726486294\\
47.40766	-7.35886419382295\\
33.309528	-5.17047862966328\\
59.54412	-9.24274880094687\\
40.230732	-6.244824005363\\
39.827912	-6.18229618444638\\
32.522112	-5.04825181213963\\
37.952968	-5.89125759981632\\
18.43817	-2.86206889377414\\
18.976484	-2.94562879990816\\
18.775074	-2.91436488944985\\
18.63958	-2.89333280423245\\
33.08368	-5.13542132541231\\
34.536432	-5.3609250662699\\
56.1545	-8.71659430927472\\
62.356536	-9.67930667789196\\
49.13375	-7.62679688436949\\
51.16746	-7.94247995540948\\
70.849428	-10.9976176605645\\
53.820426	-8.35428717189791\\
26.109401	-4.05283737145147\\
37.952968	-5.89125759981632\\
25.748059	-3.99674798198309\\
33.193552	-5.15247623018306\\
48.663775	-7.55384491417118\\
47.723825	-7.40794097377454\\
28.07725	-4.35829715463736\\
58.830384	-9.13195897722972\\
89.045192	-13.8220590309855\\
95.309214	-14.7943931897505\\
63.1695	-9.8054991891964\\
42.244832	-6.55746310994607\\
50.53513	-7.84432639550629\\
74.206353	-11.5186970638476\\
79.910028	-12.4040512393269\\
41.168204	-6.39034329767803\\
34.76228	-5.39598237052087\\
34.536432	-5.3609250662699\\
45.690115	-7.09225790273455\\
12.380752	-1.92180488523167\\
12.783517	-1.98432416876148\\
41.842012	-6.49493528902945\\
64.366974	-9.99137734773943\\
62.960766	-9.77309840926688\\
40.50172	-6.28688817579781\\
33.4194	-5.18753353443403\\
30.849616	-4.78863826174056\\
18.306338	-2.84160524329234\\
37.015496	-5.74573830750129\\
6.327222	-0.982144392323283\\
19.712546	-3.05988418176488\\
41.71018	-6.47447163854765\\
63.971478	-9.92998639629403\\
64.773456	-10.0544736033916\\
65.179938	-10.1175698590439\\
72.863583	-11.3102653025344\\
65.377686	-10.1482653347666\\
64.971204	-10.0851690791144\\
51.16746	-7.94247995540948\\
68.794332	-10.6786149430866\\
108.26515	-16.8054811347759\\
110.635062	-17.173351233391\\
48.34761	-7.50476813421958\\
28.790168	-4.46896000412902\\
29.683757	-4.60766754835487\\
88.503216	-13.7379306901159\\
63.762744	-9.89758561636451\\
48.817585	-7.5777201044179\\
57.756124	-8.96520673826969\\
68.190102	-10.5848232117117\\
107.95753	-16.7577307542825\\
117.00729	-18.1624816917194\\
118.014395	-18.3188097813977\\
124.099974	-19.2634450872066\\
92.531416	-14.363208899282\\
80.406864	-12.4811727140127\\
56.330288	-8.74388103928636\\
57.218994	-8.88183061878967\\
50.851295	-7.89340317545789\\
65.784168	-10.2113615904188\\
58.830384	-9.13195897722972\\
71.972472	-11.1719424063901\\
50.697485	-7.86952798521117\\
57.941678	-8.99400939772642\\
48.817585	-7.5777201044179\\
33.309528	-5.17047862966328\\
25.840836	-4.01114931171146\\
26.197295	-4.06648073645729\\
26.016624	-4.0384360417231\\
24.947247	-3.87244176748561\\
12.739579	-1.9775038832855\\
27.090884	-4.20518828068314\\
40.50172	-6.28688817579781\\
33.193552	-5.15247623018306\\
33.75512	-5.23964574345574\\
49.757535	-7.72362404481453\\
70.629702	-10.9635106450769\\
47.091495	-7.30978741387135\\
25.572271	-3.96946125197145\\
6.728931	-1.04449975802656\\
43.314136	-6.7234460527429\\
82.609056	-12.823008439647\\
94.948336	-14.7383758247817\\
139.079904	-21.5887079350876\\
164.965398	-25.6067892944295\\
160.473222	-24.909490311127\\
103.735653	-16.102388991242\\
113.27252	-17.5827512172525\\
145.24153	-22.5451476527857\\
120.028605	-18.6314659607544\\
104.89409	-16.2822076230854\\
121.713217	-18.8929602198527\\
138.31208	-21.469522289967\\
133.045172	-20.6519653657619\\
180.4656	-28.0128114751239\\
132.60528	-20.5836830356923\\
73.315242	-11.3803741677034\\
42.113	-6.53699945946427\\
34.872152	-5.41303727529162\\
34.76228	-5.39598237052087\\
36.215032	-5.62148611137846\\
64.575708	-10.0237781276689\\
71.520813	-11.1018335412211\\
51.16746	-7.94247995540948\\
64.169226	-9.96068187201673\\
35.653464	-5.53431659810578\\
21.792562	-3.38275511158886\\
59.191726	-9.1880483666981\\
71.520813	-11.1018335412211\\
55.08024	-8.54984207031468\\
34.536432	-5.3609250662699\\
44.251608	-6.86896534505793\\
69.805044	-10.8355029301137\\
108.89748	-16.9036346946791\\
102.577216	-15.9225703593985\\
112.65728	-17.4872504562656\\
140.968336	-21.8818401974112\\
175.08705	-27.1779248975184\\
144.846016	-22.4837539072864\\
107.95753	-16.7577307542825\\
65.377686	-10.1482653347666\\
49.911345	-7.74749923506126\\
28.609497	-4.44091530939483\\
43.453292	-6.74504657269591\\
56.867418	-8.82725715876638\\
49.13375	-7.62679688436949\\
35.317744	-5.48220438908407\\
42.779484	-6.64045458134449\\
50.38132	-7.82045120525957\\
51.47508	-7.99023033590293\\
65.179938	-10.1175698590439\\
55.256028	-8.57712880032632\\
26.377966	-4.09452543119148\\
25.660165	-3.98310461697727\\
12.6932	-1.97030469306086\\
12.739579	-1.9775038832855\\
19.84804	-3.08091626698229\\
40.362564	-6.2652876558448\\
47.877635	-7.43181616402126\\
42.911316	-6.66091823182629\\
63.971478	-9.92998639629403\\
51.32127	-7.9663551456562\\
54.60255	-8.47569253758627\\
124.099974	-19.2634450872066\\
92.267752	-14.3222815983184\\
77.331345	-12.0037746174494\\
110.14505	-17.0972890155692\\
116.000185	-18.0061536020411\\
83.347596	-12.9376484695853\\
72.863583	-11.3102653025344\\
64.971204	-10.0851690791144\\
60.803116	-9.43817672513816\\
97.086944	-15.0703417103754\\
167.626524	-26.0198631486684\\
175.08705	-27.1779248975184\\
139.826976	-21.7046723465562\\
128.377263	-19.9273881898345\\
95.753976	-14.8634314666149\\
113.90485	-17.6809047771557\\
125.877295	-19.5393301207186\\
92.004088	-14.2813542973548\\
67.585872	-10.4910314803367\\
62.053164	-9.63221569410984\\
78.234663	-12.1439923477874\\
82.864188	-12.8626113590802\\
51.47508	-7.99023033590293\\
43.314136	-6.7234460527429\\
44.251608	-6.86896534505793\\
53.671145	-8.33111499664779\\
93.073392	-14.4473372401516\\
106.70996	-16.5640764333924\\
74.658012	-11.5888059290166\\
66.981642	-10.3972397489618\\
59.006172	-9.15924570724137\\
57.218994	-8.88183061878967\\
34.872152	-5.41303727529162\\
27.271555	-4.23323297541733\\
33.309528	-5.17047862966328\\
33.193552	-5.15247623018306\\
27.271555	-4.23323297541733\\
42.647652	-6.61999093086268\\
57.404548	-8.9106332782464\\
62.960766	-9.77309840926688\\
48.031445	-7.45569135426799\\
20.382692	-3.16390773838071\\
20.45227	-3.17470799835721\\
34.982024	-5.43009218006236\\
49.59518	-7.69842245510966\\
56.506076	-8.771167769298\\
48.1938	-7.48089294397286\\
28.970839	-4.49700469886321\\
67.794606	-10.5234322602663\\
88.781528	-13.7811317300219\\
62.356536	-9.67930667789196\\
40.765384	-6.32781547676141\\
55.968946	-8.68779164981799\\
49.28756	-7.65067207461622\\
49.13375	-7.62679688436949\\
33.974864	-5.27375555299722\\
34.872152	-5.41303727529162\\
28.970839	-4.49700469886321\\
63.971478	-9.92998639629403\\
53.293062	-8.27242698186298\\
19.64663	-3.04965235652398\\
34.982024	-5.43009218006236\\
57.941678	-8.99400939772642\\
70.409976	-10.9294036295893\\
41.30736	-6.41194381763104\\
35.879312	-5.56937390235675\\
67.388124	-10.460336004614\\
82.609056	-12.823008439647\\
76.891893	-11.9355605864741\\
111.71733	-17.341346515869\\
125.174179	-19.4301887887798\\
76.440234	-11.8654517213051\\
77.783004	-12.0738834826184\\
107.01758	-16.6118268138858\\
84.582972	-13.1294099742094\\
70.607022	-10.9599901372114\\
116.348094	-18.0601578512027\\
88.503216	-13.7379306901159\\
44.793584	-6.95309368592756\\
55.38869	-8.59772128773619\\
129.451468	-20.0941318914078\\
143.637501	-22.2961619071498\\
109.83743	-17.0495386350757\\
79.577433	-12.3524241091006\\
110.14505	-17.0972890155692\\
91.198448	-14.1562986555216\\
59.543302	-9.24262182672139\\
52.56884	-8.16000946654628\\
75.768849	-11.7612358406485\\
75.549123	-11.7271288251609\\
76.220508	-11.8313447058175\\
74.426079	-11.5528040793352\\
59.367514	-9.21533509670974\\
60.617562	-9.40937406568142\\
73.986627	-11.48459004836\\
64.773456	-10.0544736033916\\
35.317744	-5.48220438908407\\
34.536432	-5.3609250662699\\
27.447343	-4.26051970542897\\
35.427616	-5.49925929385481\\
64.575708	-10.0237781276689\\
50.851295	-7.89340317545789\\
82.367352	-12.7854898843944\\
84.341268	-13.0918914189569\\
98.800394	-15.3363123542103\\
82.367352	-12.7854898843944\\
56.681864	-8.79845449930965\\
56.867418	-8.82725715876638\\
37.332064	-5.79487764321434\\
88.503216	-13.7379306901159\\
62.960766	-9.77309840926688\\
42.779484	-6.64045458134449\\
82.125648	-12.7479713291419\\
67.992354	-10.554127735989\\
85.321512	-13.2440500041477\\
104.89409	-16.2822076230854\\
135.655824	-21.0572043825227\\
109.83743	-17.0495386350757\\
94.948336	-14.7383758247817\\
100.530115	-15.6048086675108\\
90.392808	-14.0312430136883\\
45.328236	-7.03608515732597\\
76.672167	-11.9014535709865\\
85.06638	-13.2044470847145\\
97.340446	-15.1096916127089\\
73.095516	-11.3462671522157\\
50.38132	-7.82045120525957\\
50.851295	-7.89340317545789\\
50.22751	-7.79657601501285\\
37.332064	-5.79487764321434\\
68.596584	-10.6479194673639\\
82.125648	-12.7479713291419\\
44.119776	-6.84850169457613\\
44.925416	-6.97355733640936\\
65.575434	-10.1789608104893\\
55.431816	-8.60441553033797\\
20.65368	-3.20597190881552\\
34.982024	-5.43009218006236\\
55.793158	-8.66050491980634\\
34.200712	-5.30881285724819\\
26.109401	-4.05283737145147\\
26.285189	-4.08012410146311\\
13.632985	-2.11618302129709\\
49.911345	-7.74749923506126\\
51.16746	-7.94247995540948\\
60.265986	-9.35480060565814\\
84.824676	-13.166928529462\\
103.89011	-16.1263646122025\\
78.929784	-12.2518926541361\\
37.777656	-5.86404475700679\\
79.125774	-12.2823152439316\\
133.74664	-20.7608508865472\\
101.40291	-15.7402884586257\\
97.086944	-15.0703417103754\\
129.09991	-20.0395612254384\\
110.77738	-17.1954425754724\\
85.321512	-13.2440500041477\\
84.341268	-13.0918914189569\\
78.00273	-12.107990498106\\
99.673189	-15.4717921453252\\
90.114496	-13.9880419737823\\
69.805044	-10.8355029301137\\
95.490312	-14.8225041656513\\
136.402896	-21.1731687939914\\
114.333884	-17.747501671846\\
57.218994	-8.88183061878967\\
44.793584	-6.95309368592756\\
37.216088	-5.77687524373411\\
84.226	-13.0739989189285\\
13.452351	-2.08814406989584\\
20.518186	-3.18493982359811\\
42.508496	-6.59839041090967\\
66.981642	-10.3972397489618\\
73.754694	-11.4485881986786\\
71.972472	-11.1719424063901\\
55.61737	-8.6332181897947\\
19.983534	-3.1019483521997\\
26.646531	-4.13621349093149\\
19.84804	-3.08091626698229\\
40.230732	-6.244824005363\\
20.921006	-3.24746764451473\\
48.971395	-7.60159529466462\\
46.937685	-7.28591222362463\\
39.29326	-6.09930471304797\\
33.08368	-5.13542132541231\\
33.639144	-5.22164334397551\\
32.857832	-5.10036402116135\\
39.827912	-6.18229618444638\\
19.913956	-3.09114809222319\\
-0	0\\
13.408413	-2.08132378441986\\
35.207872	-5.46514948431333\\
66.377412	-10.3034480175869\\
69.805044	-10.8355029301137\\
117.00729	-18.1624816917194\\
88.781528	-13.7811317300219\\
56.867418	-8.82725715876638\\
35.207872	-5.46514948431333\\
47.56147	-7.38273938406967\\
6.572643	-1.02023991969823\\
20.45227	-3.17470799835721\\
34.76228	-5.39598237052087\\
69.286932	-10.7550788837636\\
41.842012	-6.49493528902945\\
42.376664	-6.57792676042787\\
57.043206	-8.85454388877803\\
64.575708	-10.0237781276689\\
58.654596	-9.10467224721808\\
66.377412	-10.3034480175869\\
83.347596	-12.9376484695853\\
102.95016	-15.9804606718059\\
82.61472	-12.8238876352621\\
13.452351	-2.08814406989584\\
20.85509	-3.23723581927383\\
48.50142	-7.5286433244663\\
48.031445	-7.45569135426799\\
26.197295	-4.06648073645729\\
32.638088	-5.06625421161986\\
19.24381	-2.98712453560737\\
6.147735	-0.954283484243098\\
6.192912	-0.961296093759879\\
13.005648	-2.01880450089005\\
26.822319	-4.16350022094313\\
28.165144	-4.37194051964318\\
51.47508	-7.99023033590293\\
80.406864	-12.4811727140127\\
74.740248	-11.6015710297586\\
34.536432	-5.3609250662699\\
35.317744	-5.48220438908407\\
74.498544	-11.5640524745061\\
40.50172	-6.28688817579781\\
21.323826	-3.30999546543134\\
72.192198	-11.2060494218777\\
52.56884	-8.16000946654628\\
85.80492	-13.3190871146528\\
136.803898	-21.235414415468\\
66.586146	-10.3358487975164\\
51.16746	-7.94247995540948\\
};
\end{axis}

\begin{axis}[%
width=4.927cm,
height=3cm,
at={(7cm,9.677cm)},
scale only axis,
xmin=-13.139903,
xmax=201.153202,
xlabel style={font=\color{white!15!black}},
xlabel={y(t-1)u(t)},
ymin=-31.1421088442355,
ymax=2.441,
ylabel style={font=\color{white!15!black}},
ylabel={y(t)},
axis background/.style={fill=white},
title style={font=\small},
title={C5, R = -0.748},
axis x line*=bottom,
axis y line*=left
]
\addplot[only marks, mark=*, mark options={}, mark size=1.5000pt, color=mycolor1, fill=mycolor1] table[row sep=crcr]{%
x	y\\
74.877738	-10.986\\
68.398836	-15.869\\
98.213241	-10.986\\
67.388124	-12.207\\
76.440234	-13.428\\
84.086136	-15.869\\
100.530115	-14.648\\
91.725776	-14.648\\
87.712224	-7.324\\
43.453292	-8.545\\
50.851295	-10.986\\
65.784168	-9.766\\
57.941678	-9.766\\
54.54311	-6.104\\
32.186392	-2.441\\
12.515007	-3.662\\
20.115366	-2.441\\
14.436074	-12.207\\
73.534968	-13.428\\
81.387108	-12.207\\
72.863583	-10.986\\
62.960766	-6.104\\
36.324904	-6.104\\
36.324904	-1.221\\
7.042728	-6.104\\
36.434776	-10.986\\
65.575434	-12.207\\
71.972472	-8.545\\
51.791245	-14.648\\
89.308856	-14.648\\
87.712224	-9.766\\
59.728856	-13.428\\
82.609056	-12.207\\
74.206353	-9.766\\
58.478808	-9.766\\
57.404548	-8.545\\
50.22751	-8.545\\
50.851295	-12.207\\
72.643857	-10.986\\
65.575434	-10.986\\
65.784168	-10.986\\
66.377412	-10.986\\
67.992354	-14.648\\
90.920136	-15.869\\
95.880498	-9.766\\
58.293254	-8.545\\
48.971395	-8.545\\
48.34761	-6.104\\
34.982024	-7.324\\
41.71018	-7.324\\
41.439192	-4.883\\
28.165144	-9.766\\
57.218994	-10.986\\
64.366974	-8.545\\
51.791245	-13.428\\
81.131976	-15.869\\
91.818034	-8.545\\
48.031445	-6.104\\
34.76228	-6.104\\
35.098	-9.766\\
54.54311	-6.104\\
33.529272	-4.883\\
27.359449	-7.324\\
40.633552	-7.324\\
40.230732	-3.662\\
20.316776	-6.104\\
34.200712	-8.545\\
49.44137	-10.986\\
64.575708	-10.986\\
65.377686	-9.766\\
58.478808	-13.428\\
81.131976	-10.986\\
65.784168	-12.207\\
72.643857	-10.986\\
64.971204	-10.986\\
64.773456	-7.324\\
43.050472	-8.545\\
51.791245	-14.648\\
93.337056	-20.752\\
134.514464	-20.752\\
134.888	-19.531\\
123.377327	-13.428\\
86.301756	-15.869\\
104.608448	-24.414\\
162.279858	-23.193\\
152.030115	-15.869\\
105.481243	-21.973\\
150.866618	-28.076\\
190.720268	-23.193\\
154.163871	-17.09\\
110.46976	-14.648\\
91.725776	-13.428\\
82.609056	-9.766\\
60.080432	-10.986\\
66.586146	-12.207\\
71.081361	-4.883\\
27.808685	-4.883\\
27.984473	-7.324\\
43.050472	-10.986\\
64.971204	-9.766\\
57.756124	-9.766\\
58.654596	-10.986\\
66.783894	-13.428\\
83.844432	-14.648\\
91.462112	-15.869\\
99.673189	-14.648\\
91.198448	-17.09\\
104.52244	-12.207\\
75.097464	-12.207\\
73.986627	-10.986\\
65.575434	-8.545\\
51.32127	-9.766\\
58.830384	-12.207\\
71.972472	-8.545\\
50.22751	-9.766\\
56.330288	-7.324\\
42.113	-6.104\\
35.207872	-8.545\\
48.34761	-6.104\\
34.536432	-4.883\\
28.165144	-8.545\\
50.697485	-13.428\\
81.131976	-12.207\\
74.658012	-10.986\\
65.179938	-7.324\\
43.856112	-8.545\\
53.35498	-19.531\\
121.228917	-14.648\\
91.462112	-10.986\\
69.003066	-17.09\\
103.58249	-12.207\\
71.520813	-6.104\\
36.099056	-9.766\\
58.293254	-8.545\\
53.038815	-14.648\\
93.879032	-15.869\\
102.275705	-19.531\\
123.728885	-14.648\\
92.267752	-14.648\\
92.267752	-13.428\\
83.844432	-13.428\\
83.844432	-14.648\\
89.308856	-10.986\\
65.784168	-7.324\\
43.314136	-7.324\\
42.779484	-6.104\\
35.763336	-8.545\\
51.32127	-10.986\\
68.596584	-15.869\\
98.800394	-14.648\\
88.503216	-8.545\\
50.697485	-8.545\\
50.38132	-9.766\\
55.61737	-6.104\\
33.4194	-2.441\\
13.274158	-4.883\\
27.54012	-7.324\\
41.168204	-8.545\\
47.877635	-6.104\\
34.536432	-7.324\\
41.168204	-6.104\\
33.75512	-4.883\\
26.197295	-4.883\\
25.572271	-1.221\\
6.48351	-3.662\\
20.785512	-12.207\\
73.095516	-14.648\\
90.114496	-15.869\\
95.309214	-12.207\\
71.301087	-7.324\\
41.439192	-6.104\\
33.193552	-3.662\\
20.584102	-7.324\\
41.842012	-8.545\\
49.911345	-9.766\\
58.830384	-12.207\\
77.783004	-18.311\\
121.03571	-18.311\\
120.358203	-19.531\\
128.025705	-18.311\\
117.00729	-15.869\\
100.831626	-15.869\\
101.40291	-17.09\\
109.52981	-18.311\\
116.000185	-12.207\\
76.220508	-10.986\\
68.398836	-12.207\\
75.768849	-10.986\\
68.398836	-12.207\\
74.426079	-9.766\\
60.978904	-15.869\\
101.990063	-18.311\\
115.670587	-12.207\\
74.426079	-7.324\\
43.987944	-9.766\\
60.803116	-17.09\\
109.52981	-18.311\\
120.358203	-18.311\\
121.713217	-21.973\\
146.450045	-20.752\\
136.402896	-17.09\\
110.77738	-15.869\\
103.735653	-18.311\\
121.365308	-20.752\\
133.74664	-17.09\\
105.77001	-8.545\\
53.038815	-14.648\\
90.920136	-12.207\\
72.863583	-7.324\\
43.856112	-10.986\\
66.586146	-9.766\\
58.478808	-8.545\\
50.065155	-7.324\\
43.314136	-8.545\\
50.22751	-8.545\\
50.851295	-9.766\\
59.728856	-13.428\\
82.367352	-14.648\\
87.975888	-7.324\\
43.716956	-8.545\\
52.26122	-12.207\\
78.234663	-17.09\\
108.89748	-15.869\\
97.340446	-8.545\\
50.851295	-8.545\\
51.005105	-9.766\\
58.117466	-8.545\\
49.59518	-7.324\\
42.376664	-6.104\\
35.763336	-8.545\\
51.005105	-9.766\\
58.830384	-9.766\\
57.580336	-9.766\\
59.543302	-12.207\\
76.220508	-14.648\\
91.198448	-15.869\\
97.626088	-10.986\\
67.585872	-13.428\\
84.086136	-17.09\\
104.83006	-12.207\\
70.409976	-3.662\\
20.986922	-9.766\\
56.330288	-8.545\\
50.38132	-9.766\\
57.580336	-8.545\\
49.44137	-8.545\\
50.38132	-8.545\\
50.851295	-10.986\\
63.971478	-7.324\\
42.911316	-7.324\\
42.779484	-9.766\\
56.506076	-6.104\\
35.543592	-8.545\\
48.50142	-6.104\\
34.536432	-3.662\\
20.382692	-6.104\\
33.75512	-4.883\\
28.07725	-10.986\\
64.575708	-12.207\\
73.754694	-12.207\\
76.440234	-18.311\\
112.649272	-10.986\\
64.773456	-4.883\\
27.896579	-6.104\\
34.536432	-6.104\\
35.098	-8.545\\
48.031445	-6.104\\
33.974864	-8.545\\
48.50142	-7.324\\
43.314136	-13.428\\
80.648568	-9.766\\
59.904644	-14.648\\
88.503216	-10.986\\
65.377686	-9.766\\
55.08024	-6.104\\
33.75512	-3.662\\
19.64663	-3.662\\
19.44522	-1.221\\
6.550665	-2.441\\
13.586606	-8.545\\
48.817585	-8.545\\
49.28756	-7.324\\
42.779484	-9.766\\
59.367514	-15.869\\
97.626088	-10.986\\
66.377412	-9.766\\
59.191726	-9.766\\
59.728856	-13.428\\
83.602728	-14.648\\
91.462112	-12.207\\
73.534968	-12.207\\
69.738591	-6.104\\
33.309528	-2.441\\
13.095965	-4.883\\
26.377966	-4.883\\
26.46586	-6.104\\
32.973808	-3.662\\
20.04945	-4.883\\
27.808685	-9.766\\
55.968946	-7.324\\
42.113	-7.324\\
42.508496	-10.986\\
62.158788	-7.324\\
39.425092	-4.883\\
27.00299	-9.766\\
56.330288	-12.207\\
71.301087	-9.766\\
57.580336	-10.986\\
64.366974	-7.324\\
42.113	-7.324\\
42.647652	-10.986\\
66.179664	-14.648\\
89.045192	-13.428\\
80.151732	-8.545\\
52.56884	-14.648\\
90.114496	-13.428\\
81.628812	-13.428\\
82.609056	-13.428\\
82.609056	-13.428\\
81.628812	-10.986\\
65.784168	-10.986\\
68.190102	-12.207\\
78.674115	-18.311\\
117.00729	-17.09\\
104.83006	-10.986\\
66.179664	-10.986\\
66.179664	-10.986\\
66.981642	-13.428\\
82.864188	-13.428\\
81.870516	-9.766\\
59.191726	-12.207\\
75.549123	-14.648\\
91.198448	-14.648\\
90.114496	-9.766\\
58.293254	-8.545\\
51.62889	-10.986\\
69.003066	-18.311\\
116.000185	-14.648\\
92.531416	-13.428\\
82.367352	-9.766\\
58.478808	-10.986\\
65.784168	-9.766\\
56.330288	-6.104\\
33.529272	-4.883\\
26.016624	-3.662\\
19.0424	-3.662\\
19.712546	-7.324\\
41.439192	-10.986\\
63.762744	-9.766\\
56.330288	-6.104\\
35.098	-7.324\\
42.647652	-9.766\\
55.256028	-6.104\\
34.200712	-6.104\\
34.200712	-9.766\\
53.820426	-6.104\\
33.864992	-7.324\\
42.113	-12.207\\
72.424131	-12.207\\
70.849428	-8.545\\
48.34761	-4.883\\
27.178778	-6.104\\
33.974864	-7.324\\
40.0989	-6.104\\
34.76228	-7.324\\
44.654428	-14.648\\
89.308856	-9.766\\
60.617562	-17.09\\
108.26515	-20.752\\
132.999568	-18.311\\
115.670587	-13.428\\
82.609056	-10.986\\
66.783894	-10.986\\
66.783894	-12.207\\
74.658012	-13.428\\
83.105892	-14.648\\
91.198448	-15.869\\
101.117268	-20.752\\
134.888	-20.752\\
137.938544	-24.414\\
159.594318	-15.869\\
103.1485	-20.752\\
136.402896	-20.752\\
134.888	-14.648\\
95.490312	-17.09\\
111.085	-18.311\\
114.333884	-9.766\\
57.756124	-7.324\\
42.647652	-8.545\\
50.851295	-9.766\\
57.756124	-7.324\\
42.113	-6.104\\
35.427616	-7.324\\
41.973844	-6.104\\
33.974864	-3.662\\
20.45227	-4.883\\
27.447343	-7.324\\
40.633552	-2.441\\
13.901495	-10.986\\
64.575708	-8.545\\
49.757535	-7.324\\
44.119776	-15.869\\
100.244473	-19.531\\
124.451532	-19.531\\
126.9515	-19.531\\
124.80309	-14.648\\
93.337056	-17.09\\
106.40234	-12.207\\
72.643857	-7.324\\
43.585124	-10.986\\
64.575708	-7.324\\
41.973844	-3.662\\
21.188332	-6.104\\
34.310584	-4.883\\
26.197295	-2.441\\
13.052027	-4.883\\
27.359449	-10.986\\
63.971478	-9.766\\
58.293254	-10.986\\
66.377412	-10.986\\
66.377412	-12.207\\
74.206353	-14.648\\
88.781528	-10.986\\
65.377686	-10.986\\
65.179938	-8.545\\
50.22751	-7.324\\
43.716956	-13.428\\
79.171488	-9.766\\
56.330288	-6.104\\
35.879312	-10.986\\
66.377412	-13.428\\
80.151732	-10.986\\
64.366974	-7.324\\
42.508496	-7.324\\
42.508496	-7.324\\
41.71018	-4.883\\
27.271555	-6.104\\
34.76228	-7.324\\
42.508496	-10.986\\
64.575708	-8.545\\
49.757535	-7.324\\
43.050472	-10.986\\
64.773456	-8.545\\
49.13375	-6.104\\
34.536432	-7.324\\
43.314136	-13.428\\
82.367352	-14.648\\
90.656472	-13.428\\
82.367352	-12.207\\
74.426079	-12.207\\
73.315242	-8.545\\
50.851295	-9.766\\
57.218994	-8.545\\
50.697485	-12.207\\
72.192198	-7.324\\
42.113	-4.883\\
28.433709	-9.766\\
57.941678	-9.766\\
58.478808	-10.986\\
66.783894	-12.207\\
74.206353	-10.986\\
68.596584	-17.09\\
110.14505	-20.752\\
136.02936	-23.193\\
148.643937	-15.869\\
96.753293	-9.766\\
57.043206	-6.104\\
34.09084	-6.104\\
34.200712	-9.766\\
54.894686	-6.104\\
35.543592	-9.766\\
57.218994	-6.104\\
35.427616	-7.324\\
43.050472	-8.545\\
51.005105	-10.986\\
66.981642	-13.428\\
82.609056	-14.648\\
93.60072	-19.531\\
125.525737	-14.648\\
92.531416	-13.428\\
82.125648	-10.986\\
66.586146	-10.986\\
66.586146	-10.986\\
67.585872	-12.207\\
73.754694	-9.766\\
58.478808	-10.986\\
65.784168	-9.766\\
56.506076	-7.324\\
42.911316	-12.207\\
74.206353	-13.428\\
80.648568	-8.545\\
50.851295	-9.766\\
60.978904	-18.311\\
114.333884	-12.207\\
75.329397	-12.207\\
77.111619	-17.09\\
108.58986	-14.648\\
90.392808	-10.986\\
64.971204	-7.324\\
42.779484	-7.324\\
44.119776	-12.207\\
77.331345	-17.09\\
109.83743	-20.752\\
134.140928	-18.311\\
117.00729	-13.428\\
82.864188	-9.766\\
59.006172	-8.545\\
50.22751	-6.104\\
35.317744	-7.324\\
41.842012	-4.883\\
28.165144	-8.545\\
50.065155	-8.545\\
48.50142	-6.104\\
35.098	-9.766\\
57.756124	-10.986\\
65.784168	-12.207\\
75.329397	-14.648\\
92.79508	-17.09\\
111.40971	-19.531\\
124.80309	-18.311\\
115.340989	-13.428\\
84.582972	-14.648\\
91.462112	-12.207\\
74.206353	-7.324\\
44.390764	-13.428\\
84.582972	-19.531\\
125.877295	-18.311\\
114.333884	-12.207\\
72.643857	-6.104\\
34.76228	-4.883\\
26.109401	-1.221\\
6.372399	2.441\\
-13.139903	-4.883\\
26.910213	-6.104\\
34.42656	-8.545\\
48.1938	-6.104\\
33.864992	-4.883\\
26.910213	-4.883\\
26.822319	-7.324\\
40.0989	-4.883\\
26.822319	-4.883\\
27.628014	-8.545\\
50.38132	-12.207\\
73.534968	-13.428\\
81.131976	-10.986\\
67.190376	-14.648\\
92.004088	-17.09\\
108.58986	-17.09\\
110.77738	-20.752\\
134.888	-20.752\\
132.231744	-14.648\\
90.920136	-10.986\\
66.981642	-10.986\\
69.200814	-18.311\\
117.684797	-18.311\\
115.670587	-12.207\\
74.658012	-8.545\\
49.13375	-4.883\\
26.646531	-4.883\\
26.377966	-4.883\\
25.3916	-2.441\\
13.052027	-6.104\\
33.75512	-8.545\\
47.723825	-7.324\\
40.90454	-7.324\\
39.564248	-6.104\\
32.07652	-3.662\\
19.44522	-4.883\\
26.016624	-4.883\\
26.285189	-6.104\\
32.522112	-2.441\\
12.783517	-2.441\\
13.095965	-7.324\\
42.508496	-9.766\\
59.367514	-13.428\\
83.347596	-15.869\\
97.054804	-10.986\\
68.398836	-18.311\\
119.0215	-23.193\\
152.447589	-21.973\\
144.428529	-19.531\\
126.248384	-14.648\\
94.40636	-14.648\\
94.40636	-17.09\\
107.34229	-13.428\\
83.105892	-9.766\\
58.293254	-7.324\\
42.508496	-7.324\\
44.251608	-13.428\\
80.151732	-8.545\\
50.851295	-12.207\\
73.315242	-10.986\\
65.784168	-8.545\\
51.005105	-8.545\\
51.005105	-10.986\\
66.179664	-9.766\\
59.367514	-12.207\\
73.986627	-10.986\\
65.377686	-8.545\\
51.005105	-12.207\\
70.19025	-6.104\\
32.522112	-2.441\\
12.917772	-7.324\\
40.0989	-7.324\\
39.022272	-2.441\\
12.15618	-1.221\\
6.192912	-2.441\\
12.96171	-6.104\\
33.4194	-6.104\\
34.200712	-8.545\\
47.877635	-7.324\\
42.508496	-13.428\\
78.929784	-8.545\\
48.817585	-6.104\\
35.427616	-10.986\\
61.752306	-7.324\\
39.827912	-4.883\\
25.92873	-4.883\\
24.947247	-2.441\\
12.6932	-3.662\\
18.976484	-3.662\\
19.44522	-6.104\\
34.09084	-9.766\\
55.256028	-8.545\\
47.56147	-4.883\\
27.090884	-7.324\\
40.0989	-7.324\\
40.50172	-7.324\\
39.967068	-6.104\\
33.08368	-2.441\\
13.139903	-7.324\\
38.619452	-3.662\\
19.511136	-4.883\\
25.92873	-4.883\\
26.822319	-7.324\\
40.633552	-6.104\\
33.193552	-4.883\\
26.46586	-3.662\\
19.511136	-4.883\\
26.016624	-4.883\\
27.808685	-13.428\\
79.910028	-13.428\\
78.674652	-7.324\\
42.508496	-12.207\\
69.067206	-7.324\\
42.647652	-12.207\\
71.752746	-8.545\\
48.663775	-7.324\\
41.842012	-8.545\\
48.34761	-7.324\\
42.113	-4.883\\
27.359449	-6.104\\
33.639144	-9.766\\
54.54311	-7.324\\
41.571024	-12.207\\
68.395821	-6.104\\
34.200712	-8.545\\
48.031445	-6.104\\
35.317744	-12.207\\
70.849428	-7.324\\
43.716956	-14.648\\
91.198448	-17.09\\
106.07763	-13.428\\
80.406864	-9.766\\
56.867418	-9.766\\
58.117466	-12.207\\
74.426079	-13.428\\
80.890272	-8.545\\
51.791245	-13.428\\
79.413192	-10.986\\
60.75258	-4.883\\
26.553754	-3.662\\
20.85509	-12.207\\
70.629702	-10.986\\
62.960766	-8.545\\
50.53513	-10.986\\
65.575434	-12.207\\
72.192198	-9.766\\
56.681864	-7.324\\
42.647652	-9.766\\
58.117466	-12.207\\
72.192198	-8.545\\
50.38132	-10.986\\
64.575708	-9.766\\
56.330288	-6.104\\
35.098	-8.545\\
48.50142	-4.883\\
27.54012	-8.545\\
50.065155	-10.986\\
65.575434	-14.648\\
88.239552	-13.428\\
80.648568	-13.428\\
78.929784	-8.545\\
48.817585	-6.104\\
34.872152	-7.324\\
42.376664	-8.545\\
50.22751	-12.207\\
73.534968	-12.207\\
74.658012	-14.648\\
88.781528	-10.986\\
63.971478	-8.545\\
51.16746	-12.207\\
76.000782	-18.311\\
112.319674	-12.207\\
74.206353	-14.648\\
90.114496	-14.648\\
89.045192	-9.766\\
58.478808	-9.766\\
58.478808	-12.207\\
72.863583	-9.766\\
58.654596	-10.986\\
66.981642	-14.648\\
88.239552	-9.766\\
59.006172	-12.207\\
73.534968	-10.986\\
66.179664	-10.986\\
65.575434	-6.104\\
36.880368	-9.766\\
58.117466	-8.545\\
50.38132	-9.766\\
58.117466	-9.766\\
58.117466	-9.766\\
55.61737	-6.104\\
32.857832	-3.662\\
18.910568	-2.441\\
12.783517	-6.104\\
34.76228	-10.986\\
65.575434	-12.207\\
73.315242	-10.986\\
64.773456	-8.545\\
50.38132	-12.207\\
71.752746	-8.545\\
49.59518	-9.766\\
54.894686	-6.104\\
34.42656	-8.545\\
48.34761	-7.324\\
41.439192	-7.324\\
41.71018	-7.324\\
41.571024	-7.324\\
41.30736	-6.104\\
33.529272	-6.104\\
33.639144	-4.883\\
28.340932	-10.986\\
63.367248	-7.324\\
42.376664	-9.766\\
56.506076	-7.324\\
43.050472	-12.207\\
72.424131	-9.766\\
58.830384	-14.648\\
87.712224	-8.545\\
51.32127	-13.428\\
80.890272	-12.207\\
70.629702	-7.324\\
43.182304	-10.986\\
64.169226	-9.766\\
57.043206	-10.986\\
64.971204	-12.207\\
73.534968	-13.428\\
80.151732	-7.324\\
43.987944	-13.428\\
82.609056	-14.648\\
89.587168	-14.648\\
88.239552	-9.766\\
58.117466	-9.766\\
58.293254	-10.986\\
65.377686	-7.324\\
42.911316	-9.766\\
56.506076	-7.324\\
42.376664	-8.545\\
49.44137	-7.324\\
44.119776	-13.428\\
83.844432	-15.869\\
98.800394	-15.869\\
98.800394	-14.648\\
90.920136	-15.869\\
95.309214	-8.545\\
49.911345	-7.324\\
42.113	-6.104\\
34.42656	-4.883\\
27.628014	-4.883\\
28.609497	-8.545\\
51.005105	-10.986\\
66.586146	-13.428\\
80.890272	-12.207\\
71.520813	-7.324\\
44.390764	-12.207\\
75.768849	-15.869\\
99.958831	-15.869\\
98.800394	-13.428\\
86.54346	-20.752\\
133.373104	-17.09\\
104.83006	-10.986\\
64.971204	-6.104\\
35.317744	-7.324\\
43.453292	-10.986\\
66.981642	-13.428\\
84.086136	-17.09\\
106.07763	-12.207\\
74.426079	-10.986\\
63.762744	-6.104\\
34.76228	-9.766\\
55.968946	-7.324\\
42.244832	-7.324\\
41.30736	-4.883\\
27.54012	-4.883\\
27.359449	-3.662\\
19.782124	-3.662\\
20.382692	-6.104\\
34.646304	-10.986\\
63.762744	-8.545\\
49.59518	-9.766\\
56.867418	-8.545\\
49.911345	-8.545\\
50.697485	-12.207\\
71.301087	-7.324\\
43.182304	-12.207\\
73.754694	-13.428\\
80.151732	-8.545\\
51.005105	-10.986\\
65.377686	-8.545\\
50.851295	-10.986\\
67.190376	-15.869\\
96.182009	-10.986\\
65.575434	-8.545\\
51.16746	-9.766\\
58.293254	-10.986\\
63.971478	-7.324\\
43.453292	-8.545\\
52.41503	-13.428\\
82.367352	-14.648\\
90.656472	-15.869\\
102.275705	-20.752\\
131.858208	-15.869\\
98.800394	-14.648\\
89.308856	-10.986\\
67.585872	-10.986\\
69.59631	-17.09\\
106.07763	-13.428\\
82.125648	-10.986\\
68.190102	-14.648\\
89.045192	-9.766\\
56.681864	-7.324\\
42.376664	-9.766\\
56.1545	-6.104\\
34.200712	-3.662\\
20.382692	-6.104\\
33.75512	-4.883\\
26.734425	-4.883\\
27.090884	-4.883\\
27.715908	-8.545\\
49.44137	-9.766\\
57.941678	-12.207\\
73.095516	-12.207\\
73.986627	-12.207\\
75.329397	-14.648\\
90.392808	-14.648\\
88.781528	-10.986\\
66.179664	-10.986\\
65.981916	-9.766\\
58.654596	-10.986\\
67.388124	-14.648\\
89.045192	-10.986\\
66.783894	-10.986\\
66.586146	-9.766\\
58.830384	-10.986\\
68.190102	-14.648\\
90.656472	-12.207\\
75.329397	-12.207\\
74.877738	-13.428\\
79.910028	-9.766\\
55.968946	-6.104\\
33.75512	-4.883\\
27.628014	-7.324\\
41.571024	-7.324\\
42.244832	-7.324\\
41.973844	-6.104\\
35.543592	-8.545\\
52.26122	-14.648\\
92.531416	-18.311\\
114.663482	-12.207\\
76.220508	-14.648\\
91.198448	-13.428\\
81.387108	-10.986\\
66.377412	-10.986\\
66.783894	-10.986\\
66.586146	-10.986\\
66.586146	-13.428\\
83.105892	-14.648\\
93.60072	-19.531\\
125.525737	-17.09\\
109.52981	-15.869\\
102.275705	-19.531\\
122.303122	-13.428\\
83.844432	-15.869\\
97.927599	-13.428\\
82.367352	-10.986\\
68.596584	-13.428\\
84.824676	-17.09\\
102.64254	-7.324\\
43.182304	-10.986\\
64.366974	-7.324\\
43.585124	-8.545\\
51.16746	-10.986\\
64.366974	-7.324\\
43.050472	-8.545\\
52.26122	-15.869\\
103.1485	-20.752\\
137.17072	-21.973\\
144.428529	-19.531\\
126.599942	-15.869\\
103.450011	-19.531\\
129.09991	-21.973\\
142.8245	-17.09\\
110.46976	-18.311\\
119.0215	-17.09\\
108.26515	-12.207\\
75.768849	-9.766\\
61.340246	-13.428\\
82.609056	-10.986\\
65.575434	-6.104\\
36.770496	-9.766\\
57.941678	-8.545\\
50.53513	-8.545\\
50.53513	-8.545\\
51.005105	-9.766\\
59.904644	-12.207\\
74.206353	-8.545\\
51.16746	-7.324\\
43.856112	-8.545\\
51.791245	-10.986\\
67.388124	-10.986\\
66.981642	-9.766\\
58.654596	-9.766\\
60.080432	-14.648\\
90.656472	-13.428\\
81.387108	-9.766\\
60.978904	-13.428\\
84.086136	-13.428\\
82.367352	-10.986\\
65.575434	-8.545\\
49.13375	-7.324\\
40.90454	-4.883\\
28.165144	-6.104\\
35.543592	-8.545\\
48.817585	-6.104\\
33.864992	-4.883\\
27.090884	-4.883\\
27.896579	-7.324\\
42.779484	-7.324\\
43.716956	-12.207\\
74.658012	-14.648\\
89.850832	-13.428\\
82.864188	-13.428\\
80.648568	-10.986\\
61.554558	-4.883\\
27.271555	-6.104\\
33.974864	-7.324\\
41.036372	-8.545\\
49.911345	-8.545\\
50.851295	-10.986\\
67.794606	-14.648\\
89.308856	-14.648\\
87.712224	-9.766\\
60.441774	-14.648\\
89.587168	-13.428\\
79.910028	-6.104\\
35.317744	-8.545\\
49.757535	-7.324\\
43.716956	-10.986\\
67.388124	-17.09\\
103.58249	-12.207\\
72.424131	-9.766\\
57.756124	-8.545\\
50.851295	-10.986\\
66.981642	-12.207\\
77.783004	-19.531\\
125.174179	-15.869\\
101.40291	-17.09\\
105.77001	-12.207\\
72.863583	-7.324\\
43.716956	-8.545\\
47.723825	-4.883\\
27.271555	-6.104\\
34.310584	-8.545\\
48.1938	-3.662\\
21.591152	-8.545\\
52.26122	-13.428\\
84.086136	-15.869\\
102.577216	-20.752\\
134.140928	-18.311\\
113.656377	-9.766\\
59.191726	-10.986\\
65.981916	-8.545\\
51.47508	-13.428\\
83.105892	-15.869\\
101.117268	-15.869\\
99.958831	-15.869\\
97.054804	-9.766\\
59.904644	-10.986\\
68.596584	-14.648\\
90.114496	-13.428\\
80.151732	-7.324\\
42.647652	-6.104\\
34.536432	-4.883\\
26.910213	-3.662\\
19.712546	-4.883\\
26.910213	-3.662\\
20.921006	-8.545\\
49.28756	-8.545\\
48.663775	-6.104\\
34.42656	-8.545\\
45.843925	-6.104\\
30.73364	-3.662\\
18.306338	-2.441\\
12.739579	-6.104\\
33.309528	-8.545\\
48.34761	-8.545\\
49.28756	-7.324\\
43.182304	-10.986\\
66.783894	-14.648\\
93.60072	-19.531\\
126.599942	-21.973\\
144.033015	-20.752\\
135.655824	-19.531\\
122.674211	-13.428\\
82.367352	-10.986\\
67.190376	-10.986\\
68.190102	-12.207\\
75.549123	-10.986\\
66.377412	-7.324\\
43.050472	-7.324\\
41.973844	-7.324\\
42.911316	-8.545\\
50.53513	-10.986\\
62.763018	-7.324\\
40.230732	-2.441\\
13.586606	-4.883\\
27.896579	-6.104\\
34.536432	-7.324\\
42.113	-6.104\\
34.982024	-8.545\\
49.911345	-6.104\\
37.106216	-14.648\\
88.781528	-10.986\\
67.794606	-12.207\\
78.234663	-20.752\\
135.655824	-20.752\\
134.514464	-15.869\\
99.086036	-13.428\\
81.387108	-8.545\\
49.59518	-7.324\\
42.779484	-6.104\\
35.207872	-10.986\\
65.179938	-7.324\\
43.716956	-10.986\\
64.366974	-8.545\\
50.697485	-10.986\\
63.762744	-4.883\\
28.521603	-7.324\\
42.376664	-9.766\\
54.357556	-3.662\\
20.316776	-3.662\\
19.983534	-6.104\\
33.08368	-2.441\\
13.23022	-3.662\\
19.309726	-3.662\\
19.379304	-2.441\\
13.139903	-6.104\\
32.973808	-6.104\\
32.973808	-6.104\\
34.09084	-7.324\\
42.376664	-10.986\\
63.564996	-7.324\\
41.842012	-7.324\\
43.314136	-12.207\\
74.426079	-14.648\\
88.781528	-9.766\\
59.367514	-10.986\\
63.564996	-9.766\\
53.46885	-2.441\\
13.720861	-9.766\\
57.043206	-8.545\\
50.697485	-12.207\\
74.658012	-13.428\\
83.105892	-15.869\\
95.880498	-10.986\\
64.366974	-8.545\\
49.757535	-7.324\\
42.911316	-8.545\\
49.44137	-6.104\\
33.75512	-4.883\\
26.910213	-4.883\\
26.822319	-4.883\\
26.285189	-3.662\\
19.24381	-4.883\\
26.197295	-3.662\\
20.45227	-8.545\\
48.1938	-8.545\\
47.723825	-4.883\\
28.07725	-7.324\\
43.987944	-14.648\\
86.628272	-6.104\\
36.550752	-9.766\\
59.191726	-17.09\\
102.33492	-9.766\\
56.1545	-6.104\\
34.872152	-9.766\\
56.681864	-10.986\\
61.752306	-8.545\\
47.56147	-3.662\\
20.115366	-6.104\\
32.41224	-1.221\\
6.372399	-2.441\\
13.052027	-6.104\\
33.639144	-6.104\\
33.529272	-7.324\\
40.633552	-6.104\\
34.310584	-8.545\\
47.40766	-7.324\\
39.161428	-3.662\\
19.0424	-1.221\\
6.3492	-6.104\\
32.186392	-4.883\\
26.016624	-7.324\\
41.571024	-10.986\\
64.169226	-7.324\\
42.376664	-8.545\\
49.28756	-9.766\\
57.404548	-12.207\\
73.986627	-14.648\\
90.392808	-17.09\\
106.40234	-17.09\\
104.52244	-12.207\\
74.426079	-10.986\\
67.190376	-12.207\\
74.658012	-12.207\\
75.549123	-13.428\\
85.06638	-18.311\\
116.348094	-17.09\\
107.95753	-13.428\\
83.602728	-12.207\\
75.097464	-12.207\\
75.097464	-12.207\\
75.768849	-13.428\\
81.131976	-10.986\\
65.575434	-10.986\\
66.783894	-12.207\\
75.097464	-13.428\\
81.870516	-12.207\\
74.658012	-10.986\\
66.783894	-10.986\\
65.784168	-9.766\\
56.867418	-9.766\\
57.404548	-7.324\\
45.057248	-12.207\\
74.658012	-8.545\\
50.53513	-6.104\\
35.989184	-8.545\\
48.663775	-8.545\\
47.091495	-3.662\\
20.45227	-7.324\\
39.967068	-7.324\\
39.564248	-3.662\\
19.84804	-6.104\\
32.522112	-4.883\\
26.46586	-2.441\\
13.139903	-6.104\\
31.966648	-3.662\\
19.580714	-3.662\\
19.712546	-8.545\\
46.783875	-3.662\\
20.115366	-8.545\\
47.091495	-4.883\\
27.447343	-7.324\\
41.30736	-8.545\\
48.663775	-7.324\\
44.119776	-13.428\\
80.890272	-4.883\\
28.970839	-8.545\\
50.851295	-10.986\\
63.971478	-10.986\\
61.148076	-6.104\\
34.646304	-7.324\\
41.571024	-12.207\\
67.053051	-3.662\\
20.382692	-6.104\\
34.42656	-8.545\\
48.50142	-6.104\\
33.864992	-9.766\\
52.218802	-1.221\\
6.394377	-3.662\\
18.976484	-4.883\\
26.197295	-4.883\\
26.46586	-7.324\\
40.230732	-3.662\\
20.65368	-8.545\\
49.911345	-10.986\\
63.762744	-9.766\\
57.756124	-8.545\\
51.62889	-14.648\\
88.239552	-10.986\\
65.784168	-10.986\\
68.794332	-18.311\\
113.656377	-18.311\\
114.663482	-13.428\\
83.844432	-17.09\\
107.34229	-13.428\\
85.563216	-17.09\\
105.46239	-13.428\\
79.171488	-4.883\\
27.984473	-7.324\\
42.508496	-12.207\\
72.643857	-10.986\\
65.575434	-10.986\\
64.366974	-8.545\\
47.723825	-4.883\\
27.090884	-6.104\\
35.653464	-9.766\\
59.367514	-15.869\\
97.626088	-13.428\\
79.413192	-9.766\\
56.330288	-9.766\\
57.043206	-7.324\\
44.654428	-12.207\\
76.000782	-15.869\\
99.371678	-14.648\\
89.850832	-13.428\\
83.844432	-13.428\\
85.06638	-19.531\\
123.728885	-17.09\\
111.085	-19.531\\
127.322589	-20.752\\
133.74664	-15.869\\
102.862858	-18.311\\
118.014395	-15.869\\
98.498883	-8.545\\
51.791245	-8.545\\
52.098865	-10.986\\
67.992354	-13.428\\
83.602728	-13.428\\
81.870516	-10.986\\
67.585872	-9.766\\
60.080432	-13.428\\
80.648568	-9.766\\
59.367514	-12.207\\
74.206353	-9.766\\
60.617562	-14.648\\
92.004088	-18.311\\
112.997181	-10.986\\
66.179664	-8.545\\
49.44137	-7.324\\
42.508496	-7.324\\
42.244832	-12.207\\
69.286932	-3.662\\
20.65368	-4.883\\
27.896579	-13.428\\
78.929784	-10.986\\
65.179938	-10.986\\
65.784168	-9.766\\
56.681864	-8.545\\
48.34761	-3.662\\
20.181282	-6.104\\
33.529272	-4.883\\
27.628014	-8.545\\
48.817585	-7.324\\
41.973844	-7.324\\
43.453292	-13.428\\
81.870516	-13.428\\
83.347596	-15.869\\
99.371678	-14.648\\
92.79508	-17.09\\
105.77001	-14.648\\
88.239552	-9.766\\
57.756124	-8.545\\
50.22751	-8.545\\
51.16746	-9.766\\
58.293254	-9.766\\
56.506076	-6.104\\
34.76228	-4.883\\
28.609497	-10.986\\
65.981916	-12.207\\
73.095516	-9.766\\
60.441774	-14.648\\
92.79508	-17.09\\
106.70996	-12.207\\
73.754694	-8.545\\
49.911345	-7.324\\
42.244832	-7.324\\
43.716956	-9.766\\
58.293254	-8.545\\
51.005105	-9.766\\
59.006172	-15.869\\
96.753293	-12.207\\
74.658012	-12.207\\
74.877738	-12.207\\
73.754694	-10.986\\
66.179664	-10.986\\
64.971204	-10.986\\
63.971478	-7.324\\
43.585124	-12.207\\
74.877738	-14.648\\
91.198448	-14.648\\
88.503216	-12.207\\
76.672167	-10.986\\
70.409274	-17.09\\
105.77001	-10.986\\
65.179938	-8.545\\
51.005105	-10.986\\
65.377686	-7.324\\
43.314136	-9.766\\
58.478808	-10.986\\
64.169226	-8.545\\
50.38132	-9.766\\
60.803116	-15.869\\
100.831626	-18.311\\
113.656377	-10.986\\
67.794606	-12.207\\
76.220508	-13.428\\
83.844432	-14.648\\
90.114496	-9.766\\
59.367514	-9.766\\
57.941678	-8.545\\
51.32127	-10.986\\
66.981642	-8.545\\
53.671145	-15.869\\
101.990063	-20.752\\
137.544256	-23.193\\
152.888256	-21.973\\
141.220471	-14.648\\
92.267752	-13.428\\
84.582972	-12.207\\
75.768849	-12.207\\
72.863583	-8.545\\
50.22751	-6.104\\
35.989184	-7.324\\
42.244832	-7.324\\
40.90454	-3.662\\
20.65368	-4.883\\
28.253038	-10.986\\
62.56527	-7.324\\
40.633552	-3.662\\
20.04945	-3.662\\
21.122416	-8.545\\
51.32127	-10.986\\
63.367248	-9.766\\
56.330288	-6.104\\
35.317744	-12.207\\
70.629702	-6.104\\
35.653464	-7.324\\
42.647652	-8.545\\
48.663775	-6.104\\
33.974864	-6.104\\
33.08368	-4.883\\
26.553754	-6.104\\
34.76228	-8.545\\
50.065155	-9.766\\
56.330288	-8.545\\
48.031445	-6.104\\
33.193552	-4.883\\
26.553754	-4.883\\
27.54012	-8.545\\
49.757535	-9.766\\
57.580336	-8.545\\
49.44137	-8.545\\
48.817585	-6.104\\
35.098	-7.324\\
43.050472	-8.545\\
52.098865	-14.648\\
91.462112	-17.09\\
105.77001	-13.428\\
80.151732	-7.324\\
43.314136	-9.766\\
57.580336	-9.766\\
57.404548	-8.545\\
48.971395	-7.324\\
41.842012	-8.545\\
49.28756	-8.545\\
49.28756	-7.324\\
41.30736	-3.662\\
19.84804	-3.662\\
19.983534	-4.883\\
27.447343	-7.324\\
42.911316	-9.766\\
58.478808	-12.207\\
74.658012	-10.986\\
66.981642	-14.648\\
89.850832	-13.428\\
84.824676	-17.09\\
112.65728	-23.193\\
153.723204	-17.09\\
109.83743	-14.648\\
93.60072	-17.09\\
110.46976	-18.311\\
121.365308	-19.531\\
125.877295	-18.311\\
110.982971	-8.545\\
49.44137	-4.883\\
27.808685	-6.104\\
33.75512	-3.662\\
19.379304	-3.662\\
19.177894	-3.662\\
19.712546	-6.104\\
33.864992	-6.104\\
34.200712	-8.545\\
47.25385	-4.883\\
26.822319	-6.104\\
33.974864	-6.104\\
34.646304	-8.545\\
48.817585	-7.324\\
41.973844	-7.324\\
41.168204	-6.104\\
33.4194	-6.104\\
32.857832	-6.104\\
33.529272	-7.324\\
41.036372	-9.766\\
54.181768	-4.883\\
26.285189	-3.662\\
20.115366	-6.104\\
32.973808	-4.883\\
26.377966	-8.545\\
46.937685	-9.766\\
55.968946	-7.324\\
42.647652	-9.766\\
55.61737	-7.324\\
42.647652	-9.766\\
57.404548	-14.648\\
85.295304	-7.324\\
41.973844	-7.324\\
43.050472	-10.986\\
66.586146	-14.648\\
89.308856	-14.648\\
90.392808	-14.648\\
89.587168	-10.986\\
66.377412	-12.207\\
73.315242	-12.207\\
74.658012	-13.428\\
81.628812	-12.207\\
75.097464	-13.428\\
83.105892	-15.869\\
98.213241	-13.428\\
87.778836	-21.973\\
143.637501	-20.752\\
128.434128	-12.207\\
72.192198	-8.545\\
50.065155	-8.545\\
50.851295	-8.545\\
51.791245	-12.207\\
75.329397	-12.207\\
76.220508	-13.428\\
84.341268	-15.869\\
97.340446	-10.986\\
66.377412	-9.766\\
59.191726	-12.207\\
74.426079	-9.766\\
57.580336	-9.766\\
55.61737	-3.662\\
21.323826	-9.766\\
57.218994	-12.207\\
73.095516	-10.986\\
67.388124	-14.648\\
89.587168	-13.428\\
80.648568	-8.545\\
50.22751	-8.545\\
50.851295	-12.207\\
74.658012	-12.207\\
77.331345	-15.869\\
102.275705	-19.531\\
128.025705	-20.752\\
135.282288	-18.311\\
118.014395	-17.09\\
106.40234	-10.986\\
65.377686	-8.545\\
52.41503	-13.428\\
84.582972	-19.531\\
119.451596	-10.986\\
68.596584	-12.207\\
78.674115	-21.973\\
144.846016	-21.973\\
141.615985	-19.531\\
119.803154	-9.766\\
57.218994	-6.104\\
35.653464	-9.766\\
57.218994	-10.986\\
64.169226	-7.324\\
41.71018	-4.883\\
27.715908	-6.104\\
34.09084	-6.104\\
34.310584	-8.545\\
47.877635	-7.324\\
41.439192	-6.104\\
35.879312	-8.545\\
50.38132	-9.766\\
58.830384	-12.207\\
75.768849	-14.648\\
91.462112	-14.648\\
88.239552	-9.766\\
57.941678	-12.207\\
73.315242	-9.766\\
57.580336	-8.545\\
49.757535	-7.324\\
41.571024	-7.324\\
42.113	-8.545\\
49.28756	-9.766\\
54.54311	-6.104\\
32.638088	-3.662\\
19.111978	-3.662\\
19.712546	-4.883\\
27.178778	-7.324\\
41.439192	-7.324\\
40.230732	-6.104\\
33.529272	-10.986\\
62.356536	-8.545\\
50.065155	-9.766\\
56.506076	-6.104\\
34.76228	-4.883\\
28.165144	-7.324\\
42.779484	-10.986\\
64.169226	-9.766\\
57.756124	-9.766\\
57.756124	-10.986\\
63.762744	-7.324\\
41.571024	-7.324\\
41.973844	-8.545\\
48.663775	-7.324\\
41.973844	-7.324\\
42.911316	-12.207\\
74.877738	-15.869\\
96.753293	-12.207\\
73.315242	-12.207\\
75.768849	-15.869\\
97.340446	-13.428\\
79.668324	-8.545\\
49.13375	-7.324\\
41.036372	-4.883\\
27.178778	-4.883\\
26.910213	-3.662\\
19.983534	-3.662\\
20.85509	-10.986\\
60.346098	-6.104\\
32.522112	-4.883\\
26.646531	-6.104\\
34.200712	-8.545\\
47.56147	-7.324\\
39.827912	-6.104\\
32.186392	-4.883\\
25.92873	-6.104\\
34.310584	-10.986\\
64.169226	-10.986\\
63.971478	-7.324\\
41.973844	-4.883\\
28.07725	-9.766\\
56.330288	-8.545\\
49.59518	-9.766\\
57.756124	-10.986\\
66.377412	-13.428\\
81.628812	-13.428\\
80.890272	-12.207\\
71.520813	-8.545\\
48.971395	-7.324\\
41.71018	-6.104\\
35.427616	-9.766\\
55.431816	-6.104\\
33.974864	-6.104\\
35.098	-8.545\\
50.38132	-12.207\\
72.192198	-9.766\\
58.293254	-10.986\\
67.585872	-15.869\\
94.436419	-10.986\\
66.783894	-10.986\\
68.398836	-14.648\\
90.392808	-12.207\\
75.768849	-15.869\\
98.498883	-13.428\\
87.778836	-21.973\\
145.637044	-21.973\\
142.033472	-15.869\\
103.1485	-19.531\\
129.451468	-21.973\\
146.054531	-20.752\\
138.685616	-23.193\\
155.439486	-21.973\\
150.07559	-25.635\\
174.138555	-20.752\\
136.797184	-13.428\\
84.824676	-9.766\\
60.617562	-12.207\\
75.097464	-8.545\\
51.16746	-7.324\\
44.522596	-12.207\\
76.672167	-14.648\\
90.114496	-9.766\\
58.478808	-7.324\\
43.716956	-9.766\\
56.681864	-7.324\\
42.508496	-8.545\\
47.091495	-6.104\\
33.08368	-9.766\\
52.39459	-4.883\\
26.646531	-4.883\\
26.46586	-1.221\\
6.505488	-3.662\\
18.910568	-2.441\\
12.290435	-3.662\\
18.910568	-4.883\\
24.947247	-4.883\\
24.85447	-4.883\\
26.109401	-3.662\\
20.518186	-8.545\\
48.817585	-9.766\\
55.08024	-6.104\\
34.646304	-12.207\\
72.643857	-14.648\\
84.489664	-8.545\\
46.783875	-2.441\\
13.005648	-3.662\\
18.910568	-2.441\\
12.6932	-6.104\\
33.08368	-7.324\\
41.842012	-8.545\\
48.031445	-3.662\\
21.122416	-7.324\\
44.251608	-14.648\\
89.308856	-14.648\\
87.975888	-9.766\\
56.330288	-7.324\\
42.113	-12.207\\
72.192198	-13.428\\
81.628812	-14.648\\
86.364608	-7.324\\
40.90454	-4.883\\
27.359449	-6.104\\
34.09084	-6.104\\
32.302368	-3.662\\
18.43817	-3.662\\
18.976484	-8.545\\
48.34761	-14.648\\
85.295304	-10.986\\
62.960766	-6.104\\
33.639144	-3.662\\
20.04945	-4.883\\
24.766576	-7.324\\
36.87634	-4.883\\
24.678682	-3.662\\
18.976484	-6.104\\
33.08368	-6.104\\
34.76228	-12.207\\
71.301087	-9.766\\
57.580336	-10.986\\
65.377686	-12.207\\
72.863583	-10.986\\
65.377686	-12.207\\
72.424131	-10.986\\
65.575434	-13.428\\
83.844432	-18.311\\
115.670587	-15.869\\
95.594856	-7.324\\
41.439192	-4.883\\
28.702274	-8.545\\
51.791245	-14.648\\
88.503216	-10.986\\
64.169226	-4.883\\
27.984473	-10.986\\
65.784168	-9.766\\
61.340246	-17.09\\
108.58986	-20.752\\
133.74664	-19.531\\
126.599942	-19.531\\
124.099974	-15.869\\
100.530115	-14.648\\
87.975888	-7.324\\
42.376664	-9.766\\
57.218994	-8.545\\
51.005105	-10.986\\
65.981916	-10.986\\
66.377412	-12.207\\
71.972472	-7.324\\
43.453292	-10.986\\
65.179938	-8.545\\
48.817585	-6.104\\
33.4194	-3.662\\
19.379304	-4.883\\
26.197295	-4.883\\
26.016624	-4.883\\
24.947247	-3.662\\
19.111978	-6.104\\
33.974864	-9.766\\
54.00598	-6.104\\
33.193552	-6.104\\
33.864992	-6.104\\
35.653464	-13.428\\
77.936112	-6.104\\
33.639144	-3.662\\
19.177894	-1.221\\
6.639798	-6.104\\
35.763336	-10.986\\
67.388124	-14.648\\
93.879032	-21.973\\
146.450045	-23.193\\
155.85696	-24.414\\
160.03377	-17.09\\
110.77738	-18.311\\
121.365308	-20.752\\
136.797184	-18.311\\
120.028605	-17.09\\
112.33257	-18.311\\
121.365308	-20.752\\
138.31208	-20.752\\
141.73616	-28.076\\
188.67072	-19.531\\
125.174179	-12.207\\
73.315242	-7.324\\
42.244832	-6.104\\
34.982024	-6.104\\
34.982024	-6.104\\
34.872152	-4.883\\
28.878062	-12.207\\
72.192198	-7.324\\
43.182304	-10.986\\
65.981916	-10.986\\
64.169226	-7.324\\
42.376664	-12.207\\
72.192198	-12.207\\
73.754694	-12.207\\
71.520813	-6.104\\
34.310584	-4.883\\
27.628014	-6.104\\
34.42656	-8.545\\
51.945055	-13.428\\
86.060052	-19.531\\
125.525737	-18.311\\
119.0215	-18.311\\
121.03571	-23.193\\
157.990716	-29.297\\
200.625856	-21.973\\
145.24153	-14.648\\
92.79508	-9.766\\
58.293254	-6.104\\
35.653464	-10.986\\
64.366974	-8.545\\
50.851295	-9.766\\
57.218994	-7.324\\
42.244832	-8.545\\
49.44137	-7.324\\
42.911316	-9.766\\
57.756124	-9.766\\
58.654596	-10.986\\
65.575434	-7.324\\
41.30736	-6.104\\
33.08368	-3.662\\
19.177894	-3.662\\
19.0424	-1.221\\
6.372399	-3.662\\
19.913956	-6.104\\
33.639144	-7.324\\
41.036372	-7.324\\
42.779484	-12.207\\
70.849428	-7.324\\
44.119776	-12.207\\
77.783004	-18.311\\
116.348094	-12.207\\
76.672167	-12.207\\
77.783004	-17.09\\
110.46976	-17.09\\
107.95753	-12.207\\
75.549123	-14.648\\
87.712224	-8.545\\
50.38132	-12.207\\
76.000782	-17.09\\
113.59723	-24.414\\
168.09039	-29.297\\
201.153202	-23.193\\
157.132575	-18.311\\
120.028605	-15.869\\
103.450011	-18.311\\
122.042815	-19.531\\
125.877295	-13.428\\
83.844432	-12.207\\
74.877738	-8.545\\
54.132575	-14.648\\
93.879032	-12.207\\
74.877738	-8.545\\
51.47508	-8.545\\
50.22751	-7.324\\
43.987944	-12.207\\
76.220508	-14.648\\
92.531416	-15.869\\
98.800394	-9.766\\
59.543302	-9.766\\
59.543302	-12.207\\
73.534968	-7.324\\
42.779484	-7.324\\
41.71018	-4.883\\
27.271555	-4.883\\
26.553754	-3.662\\
19.84804	-4.883\\
27.359449	-10.986\\
64.169226	-8.545\\
50.38132	-10.986\\
62.960766	-7.324\\
41.439192	-6.104\\
33.974864	-3.662\\
20.518186	-8.545\\
48.971395	-8.545\\
49.59518	-9.766\\
56.681864	-7.324\\
41.168204	-4.883\\
28.609497	-8.545\\
52.26122	-14.648\\
88.781528	-8.545\\
48.34761	-2.441\\
13.586606	-10.986\\
62.763018	-9.766\\
56.506076	-9.766\\
55.968946	-7.324\\
41.036372	-3.662\\
20.921006	-10.986\\
65.575434	-10.986\\
64.169226	-6.104\\
33.4194	-3.662\\
19.712546	-10.986\\
63.1695	-13.428\\
79.910028	-12.207\\
70.629702	-6.104\\
34.536432	-7.324\\
43.182304	-12.207\\
74.877738	-14.648\\
90.114496	-12.207\\
76.891893	-18.311\\
118.691902	-21.973\\
139.616442	-13.428\\
83.347596	-13.428\\
85.06638	-17.09\\
106.40234	-10.986\\
68.794332	-17.09\\
109.83743	-17.09\\
108.58986	-13.428\\
80.890272	-6.104\\
37.551808	-9.766\\
63.479	-19.531\\
129.822557	-21.973\\
144.033015	-15.869\\
102.275705	-15.869\\
103.735653	-19.531\\
125.877295	-14.648\\
91.462112	-10.986\\
67.190376	-9.766\\
60.265986	-12.207\\
75.768849	-12.207\\
75.549123	-12.207\\
76.000782	-13.428\\
82.367352	-8.545\\
52.098865	-10.986\\
68.398836	-13.428\\
81.628812	-8.545\\
50.53513	-7.324\\
42.508496	-6.104\\
34.42656	-6.104\\
34.310584	-3.662\\
21.323826	-7.324\\
42.911316	-7.324\\
43.856112	-8.545\\
52.731195	-12.207\\
76.891893	-15.869\\
99.371678	-10.986\\
67.585872	-12.207\\
71.081361	-8.545\\
49.757535	-13.428\\
82.125648	-14.648\\
88.781528	-8.545\\
49.13375	-3.662\\
21.591152	-10.986\\
67.585872	-13.428\\
83.602728	-13.428\\
85.80492	-18.311\\
121.365308	-24.414\\
160.473222	-15.869\\
102.275705	-15.869\\
103.1485	-17.09\\
108.58986	-12.207\\
75.329397	-9.766\\
60.441774	-12.207\\
76.891893	-14.648\\
93.073392	-14.648\\
92.79508	-13.428\\
82.609056	-9.766\\
58.654596	-8.545\\
50.38132	-6.104\\
36.324904	-10.986\\
65.377686	-7.324\\
43.050472	-7.324\\
44.654428	-14.648\\
91.198448	-15.869\\
97.340446	-8.545\\
51.32127	-7.324\\
44.793584	-12.207\\
72.643857	-7.324\\
41.571024	-6.104\\
34.310584	-7.324\\
41.973844	-9.766\\
55.793158	-6.104\\
34.200712	-3.662\\
19.44522	-3.662\\
19.511136	-2.441\\
13.49873	-6.104\\
35.653464	-9.766\\
58.478808	-9.766\\
60.265986	-13.428\\
84.824676	-15.869\\
96.753293	-10.986\\
64.575708	-8.545\\
52.731195	-15.869\\
102.862858	-20.752\\
133.74664	-15.869\\
101.704421	-14.648\\
96.281304	-23.193\\
152.447589	-17.09\\
110.46976	-15.869\\
100.244473	-13.428\\
84.341268	-13.428\\
85.80492	-15.869\\
99.673189	-10.986\\
67.585872	-9.766\\
62.053164	-17.09\\
111.40971	-18.311\\
120.358203	-19.531\\
120.877359	-12.207\\
70.629702	-4.883\\
29.503086	-7.324\\
44.251608	-10.986\\
62.56527	-2.441\\
13.408413	-3.662\\
20.316776	-8.545\\
49.44137	-8.545\\
51.945055	-13.428\\
81.131976	-8.545\\
50.22751	-6.104\\
34.872152	-6.104\\
33.309528	-3.662\\
19.913956	-6.104\\
33.08368	-3.662\\
19.983534	-4.883\\
27.808685	-8.545\\
48.971395	-6.104\\
33.529272	-1.221\\
6.550665	-4.883\\
26.46586	-6.104\\
33.639144	-6.104\\
32.74796	-3.662\\
20.04945	-3.662\\
19.84804	-3.662\\
19.309726	-3.662\\
20.115366	-8.545\\
49.13375	-10.986\\
66.179664	-14.648\\
92.79508	-19.531\\
124.80309	-15.869\\
96.467651	-7.324\\
42.508496	-7.324\\
42.113	-6.104\\
33.75512	-4.883\\
26.197295	-3.662\\
20.382692	-7.324\\
41.571024	-8.545\\
48.34761	-6.104\\
34.76228	-8.545\\
49.44137	-8.545\\
49.911345	-9.766\\
57.404548	-9.766\\
58.478808	-10.986\\
66.179664	-12.207\\
76.672167	-14.648\\
88.781528	-9.766\\
55.431816	-6.104\\
33.75512	-6.104\\
34.982024	-9.766\\
55.968946	-4.883\\
27.54012	-7.324\\
39.425092	-2.441\\
13.095965	-6.104\\
32.302368	-6.104\\
30.959488	-3.662\\
18.573664	-3.662\\
19.379304	-6.104\\
33.309528	-7.324\\
41.71018	-10.986\\
65.575434	-13.428\\
79.910028	-8.545\\
47.25385	-2.441\\
13.811178	-6.104\\
35.207872	-9.766\\
54.181768	-4.883\\
27.090884	-1.221\\
7.131861	-12.207\\
72.192198	-7.324\\
45.057248	-14.648\\
93.337056	-18.311\\
113.326779	-13.428\\
81.387108	-8.545\\
51.16746	-8.545\\
};
\addplot [color=mycolor2, line width=2.0pt, forget plot]
  table[row sep=crcr]{%
74.877738	-11.5924113741234\\
68.398836	-10.5893616127026\\
98.213241	-15.2051640777119\\
67.388124	-10.4328853408798\\
76.440234	-11.8343136655952\\
84.086136	-13.0180358729919\\
100.530115	-15.5638575589441\\
91.725776	-14.2007886109313\\
87.712224	-13.579419067751\\
43.453292	-6.72734580235192\\
50.851295	-7.87268881635963\\
65.784168	-10.1845643008132\\
57.941678	-8.97040675938952\\
54.54311	-8.44424772479192\\
32.186392	-4.98302842311817\\
12.515007	-1.9375466376139\\
20.115366	-3.11421797508167\\
14.436074	-2.23496212499485\\
73.534968	-11.3845265923899\\
81.387108	-12.6001781261904\\
72.863583	-11.2805842015231\\
62.960766	-9.74745123713438\\
36.324904	-5.62374400644344\\
7.042728	-1.09034009777456\\
36.434776	-5.64075415467331\\
65.575434	-10.1522485490238\\
71.972472	-11.1426243009181\\
51.791245	-8.0182098665696\\
89.308856	-13.8266062217899\\
87.712224	-13.579419067751\\
59.728856	-9.24709383792792\\
82.609056	-12.7893575040955\\
74.206353	-11.4884689832567\\
58.478808	-9.05356407807593\\
57.404548	-8.88724944070312\\
50.22751	-7.77611575576574\\
50.851295	-7.87268881635963\\
72.643857	-11.2465666917849\\
65.575434	-10.1522485490238\\
65.784168	-10.1845643008132\\
66.377412	-10.276409069057\\
67.992354	-10.5264309381652\\
90.920136	-14.0760611478842\\
95.880498	-14.8440137918138\\
58.293254	-9.0248370043479\\
48.971395	-7.58164671593968\\
48.34761	-7.48507365534579\\
34.982024	-5.41584219474497\\
41.71018	-6.45748092775901\\
41.439192	-6.41552714473406\\
28.165144	-4.36046740166516\\
57.218994	-8.85852236697509\\
64.366974	-9.96515735445304\\
51.791245	-8.0182098665696\\
81.131976	-12.5606791351992\\
91.818034	-14.2150717973245\\
48.031445	-7.4361256657297\\
34.76228	-5.38182189828523\\
35.098	-5.43379735120984\\
54.54311	-8.44424772479192\\
33.529272	-5.19093023481664\\
27.359449	-4.23573142363555\\
40.633552	-6.29079968168691\\
40.230732	-6.22843595016334\\
20.316776	-3.14539984084346\\
34.200712	-5.29488114066587\\
49.44137	-7.65440724104467\\
64.575708	-9.99747310624253\\
65.377686	-10.1216336262758\\
58.478808	-9.05356407807593\\
81.131976	-12.5606791351992\\
65.784168	-10.1845643008132\\
72.643857	-11.2465666917849\\
64.971204	-10.0587029517384\\
64.773456	-10.0280880289905\\
43.050472	-6.66498207082835\\
51.791245	-8.0182098665696\\
93.337056	-14.4502435370256\\
134.514464	-20.8252418423446\\
134.888	-20.8830718875717\\
123.377327	-19.1010140934512\\
86.301756	-13.3610534263365\\
104.608448	-16.1952563580994\\
162.279858	-25.1238208033252\\
152.030115	-23.5369774970405\\
105.481243	-16.330380614728\\
150.866618	-23.3568473780383\\
190.720268	-29.5269042988327\\
154.163871	-23.8673210408586\\
110.46976	-17.1026921555868\\
91.725776	-14.2007886109313\\
82.609056	-12.7893575040955\\
60.080432	-9.30152408288629\\
66.586146	-10.3087248208465\\
71.081361	-11.0046644003131\\
27.808685	-4.30528118108236\\
27.984473	-4.33249630356155\\
43.050472	-6.66498207082835\\
64.971204	-10.0587029517384\\
57.756124	-8.94167968566149\\
58.654596	-9.08077920055511\\
66.783894	-10.3393397435945\\
83.844432	-12.9806157762634\\
91.462112	-14.1599687139341\\
99.673189	-15.4311901069815\\
91.198448	-14.1191488169368\\
104.52244	-16.1819407833492\\
75.097464	-11.6264288838617\\
73.986627	-11.4544514735184\\
65.575434	-10.1522485490238\\
51.32127	-7.94544934146461\\
58.830384	-9.1079943230343\\
71.972472	-11.1426243009181\\
50.22751	-7.77611575576574\\
56.330288	-8.72093480333032\\
42.113	-6.51984465928258\\
35.207872	-5.45080749943971\\
48.34761	-7.48507365534579\\
34.536432	-5.34685659359048\\
28.165144	-4.36046740166516\\
50.697485	-7.84887628087072\\
81.131976	-12.5606791351992\\
74.658012	-11.5583938643852\\
65.179938	-10.0910187035279\\
43.856112	-6.78970953387549\\
53.35498	-8.26030397737347\\
121.228917	-18.7684018486705\\
91.462112	-14.1599687139341\\
69.003066	-10.682907209988\\
103.58249	-16.0364197331392\\
71.520813	-11.0726994197896\\
36.099056	-5.58877870174869\\
58.293254	-9.0248370043479\\
53.038815	-8.21135598775739\\
93.879032	-14.5341511030755\\
102.275705	-15.8341060722012\\
123.728885	-19.155441551688\\
92.267752	-14.2846961769812\\
83.844432	-12.9806157762634\\
89.308856	-13.8266062217899\\
65.784168	-10.1845643008132\\
43.314136	-6.7058019678256\\
42.779484	-6.6230282878034\\
35.763336	-5.53680324882408\\
51.32127	-7.94544934146461\\
68.596584	-10.6199765354505\\
98.800394	-15.296065850353\\
88.503216	-13.7018787587427\\
50.697485	-7.84887628087072\\
50.38132	-7.79992829125464\\
55.61737	-8.61056236216473\\
33.4194	-5.17392008658676\\
13.274158	-2.05507677303382\\
27.54012	-4.26370252173916\\
41.168204	-6.37357336170911\\
47.877635	-7.4123131302408\\
34.536432	-5.34685659359048\\
41.168204	-6.37357336170911\\
33.75512	-5.22589553951138\\
26.197295	-4.05580922502316\\
25.572271	-3.95904434509716\\
6.48351	-1.00376316213296\\
20.785512	-3.21796854661634\\
73.095516	-11.3164915729135\\
90.114496	-13.951333684837\\
95.309214	-14.7555688238387\\
71.301087	-11.0386819100514\\
41.439192	-6.41552714473406\\
33.193552	-5.13895478189202\\
20.584102	-3.18678668085456\\
41.842012	-6.47789087625763\\
49.911345	-7.72716776614965\\
58.830384	-9.1079943230343\\
77.783004	-12.0421984473287\\
121.03571	-18.7384899538379\\
120.358203	-18.6335997680146\\
128.025705	-19.8206660412494\\
117.00729	-18.1148185786723\\
100.831626	-15.6105368475976\\
101.40291	-15.6989818155727\\
109.52981	-16.9571711053769\\
116.000185	-17.9589007348809\\
76.220508	-11.800296155857\\
68.398836	-10.5893616127026\\
75.768849	-11.7303712747284\\
68.398836	-10.5893616127026\\
74.426079	-11.5224864929949\\
60.978904	-9.44062359777991\\
101.990063	-15.7898835882137\\
115.670587	-17.9078730769129\\
74.426079	-11.5224864929949\\
43.987944	-6.81011948237412\\
60.803116	-9.41340847530072\\
109.52981	-16.9571711053769\\
120.358203	-18.6335997680146\\
121.713217	-18.8433801396612\\
146.450045	-22.6730829849439\\
136.402896	-21.1176048487706\\
110.77738	-17.1503172265646\\
103.735653	-16.0601321014708\\
121.365308	-18.789517611806\\
133.74664	-20.7063689716\\
105.77001	-16.375086904537\\
53.038815	-8.21135598775739\\
90.920136	-14.0760611478842\\
72.863583	-11.2805842015231\\
43.856112	-6.78970953387549\\
66.586146	-10.3087248208465\\
58.478808	-9.05356407807593\\
50.065155	-7.75098030163856\\
43.314136	-6.7058019678256\\
50.22751	-7.77611575576574\\
50.851295	-7.87268881635963\\
59.728856	-9.24709383792792\\
82.367352	-12.751937407367\\
87.975888	-13.6202389647482\\
43.716956	-6.76816569934917\\
52.26122	-8.09097039167459\\
78.234663	-12.1121233284573\\
108.89748	-16.8592751261447\\
97.340446	-15.0700398210834\\
50.851295	-7.87268881635963\\
51.005105	-7.89650135184853\\
58.117466	-8.99762188186871\\
49.59518	-7.67821977653357\\
42.376664	-6.56066455627983\\
35.763336	-5.53680324882408\\
51.005105	-7.89650135184853\\
58.830384	-9.1079943230343\\
57.580336	-8.91446456318231\\
59.543302	-9.21836676419989\\
76.220508	-11.800296155857\\
91.198448	-14.1191488169368\\
97.626088	-15.1142623050709\\
67.585872	-10.4635002636278\\
84.086136	-13.0180358729919\\
104.83006	-16.229565854327\\
70.409976	-10.9007220094464\\
20.986922	-3.24915041237813\\
56.330288	-8.72093480333032\\
50.38132	-7.79992829125464\\
57.580336	-8.91446456318231\\
49.44137	-7.65440724104467\\
50.38132	-7.79992829125464\\
50.851295	-7.87268881635963\\
63.971478	-9.90392750895717\\
42.911316	-6.64343823630202\\
42.779484	-6.6230282878034\\
56.506076	-8.7481499258095\\
35.543592	-5.50278295236433\\
48.50142	-7.50888619083469\\
34.536432	-5.34685659359048\\
20.382692	-3.15560481509277\\
33.75512	-5.22589553951138\\
28.07725	-4.34685984042556\\
64.575708	-9.99747310624253\\
73.754694	-11.4185441021281\\
76.440234	-11.8343136655952\\
112.649272	-17.4401195455387\\
64.773456	-10.0280880289905\\
27.896579	-4.31888874232196\\
34.536432	-5.34685659359048\\
35.098	-5.43379735120984\\
48.031445	-7.4361256657297\\
33.974864	-5.25991583597112\\
48.50142	-7.50888619083469\\
43.314136	-6.7058019678256\\
80.648568	-12.4858389417422\\
59.904644	-9.2743089604071\\
88.503216	-13.7018787587427\\
65.377686	-10.1216336262758\\
55.08024	-8.52740504347832\\
33.75512	-5.22589553951138\\
19.64663	-3.04164926930879\\
19.44522	-3.010467403547\\
6.550665	-1.01415995571437\\
13.586606	-2.10344930465359\\
48.817585	-7.55783418045077\\
49.28756	-7.63059470555576\\
42.779484	-6.6230282878034\\
59.367514	-9.1911516417207\\
97.626088	-15.1142623050709\\
66.377412	-10.276409069057\\
59.191726	-9.16393651924151\\
59.728856	-9.24709383792792\\
83.602728	-12.9431956795349\\
91.462112	-14.1599687139341\\
73.534968	-11.3845265923899\\
69.738591	-10.7967796185796\\
33.309528	-5.15690993835689\\
13.095965	-2.02748931359441\\
26.377966	-4.08378032312677\\
26.46586	-4.09738788436636\\
32.973808	-5.10493448543227\\
20.04945	-3.10401300083236\\
27.808685	-4.30528118108236\\
55.968946	-8.6649926071231\\
42.113	-6.51984465928258\\
42.508496	-6.58107450477845\\
62.158788	-9.62329071710109\\
39.425092	-6.1037084871162\\
27.00299	-4.18054520305276\\
56.330288	-8.72093480333032\\
71.301087	-11.0386819100514\\
57.580336	-8.91446456318231\\
64.366974	-9.96515735445304\\
42.113	-6.51984465928258\\
42.647652	-6.60261833930478\\
66.179664	-10.2457941463091\\
89.045192	-13.7857863247926\\
80.151732	-12.4089198540225\\
52.56884	-8.1385954626524\\
90.114496	-13.951333684837\\
81.628812	-12.6375982229189\\
82.609056	-12.7893575040955\\
81.628812	-12.6375982229189\\
65.784168	-10.1845643008132\\
68.190102	-10.5570458609131\\
78.674115	-12.1801583479337\\
117.00729	-18.1148185786723\\
104.83006	-16.229565854327\\
66.179664	-10.2457941463091\\
66.981642	-10.3699546663424\\
82.864188	-12.8288564950867\\
81.870516	-12.6750183196474\\
59.191726	-9.16393651924151\\
75.549123	-11.6963537649902\\
91.198448	-14.1191488169368\\
90.114496	-13.951333684837\\
58.293254	-9.0248370043479\\
51.62889	-7.99307441244242\\
69.003066	-10.682907209988\\
116.000185	-17.9589007348809\\
92.531416	-14.3255160739785\\
82.367352	-12.751937407367\\
58.478808	-9.05356407807593\\
65.784168	-10.1845643008132\\
56.330288	-8.72093480333032\\
33.529272	-5.19093023481664\\
26.016624	-4.02783812691955\\
19.0424	-2.94810367202343\\
19.712546	-3.0518542435581\\
41.439192	-6.41552714473406\\
63.762744	-9.87161175716768\\
56.330288	-8.72093480333032\\
35.098	-5.43379735120984\\
42.647652	-6.60261833930478\\
55.256028	-8.55462016595751\\
34.200712	-5.29488114066587\\
53.820426	-8.33236333237749\\
33.864992	-5.24290568774125\\
42.113	-6.51984465928258\\
72.424131	-11.2125491820467\\
70.849428	-10.9687570289228\\
48.34761	-7.48507365534579\\
27.178778	-4.20776032553195\\
33.974864	-5.25991583597112\\
40.0989	-6.20802600166472\\
34.76228	-5.38182189828523\\
44.654428	-6.91330311089494\\
89.308856	-13.8266062217899\\
60.617562	-9.38468140157269\\
108.26515	-16.7613791469125\\
132.999568	-20.5907088811457\\
115.670587	-17.9078730769129\\
82.609056	-12.7893575040955\\
66.783894	-10.3393397435945\\
74.658012	-11.5583938643852\\
83.105892	-12.8662765918152\\
91.198448	-14.1191488169368\\
101.117268	-15.6547593315851\\
134.888	-20.8830718875717\\
137.938544	-21.3553505902599\\
159.594318	-24.7080512398581\\
103.1485	-15.9692303288298\\
136.402896	-21.1176048487706\\
134.888	-20.8830718875717\\
95.490312	-14.7836060291698\\
111.085	-17.1979422975425\\
114.333884	-17.7009275751534\\
57.756124	-8.94167968566149\\
42.647652	-6.60261833930478\\
50.851295	-7.87268881635963\\
57.756124	-8.94167968566149\\
42.113	-6.51984465928258\\
35.427616	-5.48482779589946\\
41.973844	-6.49830082475625\\
33.974864	-5.25991583597112\\
20.45227	-3.16637673235593\\
27.447343	-4.24933898487515\\
40.633552	-6.29079968168691\\
13.901495	-2.15219974667664\\
64.575708	-9.99747310624253\\
49.757535	-7.70335523066075\\
44.119776	-6.83052943087274\\
100.244473	-15.5196350749566\\
124.451532	-19.2673202158415\\
126.9515	-19.6543599188591\\
124.80309	-19.3217476740784\\
93.337056	-14.4502435370256\\
106.40234	-16.4729828837691\\
72.643857	-11.2465666917849\\
43.585124	-6.74775575085054\\
64.575708	-9.99747310624253\\
41.973844	-6.49830082475625\\
21.188332	-3.28033227813991\\
34.310584	-5.31189128889574\\
26.197295	-4.05580922502316\\
13.052027	-2.02068692633538\\
27.359449	-4.23573142363555\\
63.971478	-9.90392750895717\\
58.293254	-9.0248370043479\\
66.377412	-10.276409069057\\
74.206353	-11.4884689832567\\
88.781528	-13.7449664277954\\
65.377686	-10.1216336262758\\
65.179938	-10.0910187035279\\
50.22751	-7.77611575576574\\
43.716956	-6.76816569934917\\
79.171488	-12.2571605728458\\
56.330288	-8.72093480333032\\
35.879312	-5.55475840528895\\
66.377412	-10.276409069057\\
80.151732	-12.4089198540225\\
64.366974	-9.96515735445304\\
42.508496	-6.58107450477845\\
41.71018	-6.45748092775901\\
27.271555	-4.22212386239596\\
34.76228	-5.38182189828523\\
42.508496	-6.58107450477845\\
64.575708	-9.99747310624253\\
49.757535	-7.70335523066075\\
43.050472	-6.66498207082835\\
64.773456	-10.0280880289905\\
49.13375	-7.60678217006686\\
34.536432	-5.34685659359048\\
43.314136	-6.7058019678256\\
82.367352	-12.751937407367\\
90.656472	-14.0352412508869\\
82.367352	-12.751937407367\\
74.426079	-11.5224864929949\\
73.315242	-11.3505090826517\\
50.851295	-7.87268881635963\\
57.218994	-8.85852236697509\\
50.697485	-7.84887628087072\\
72.192198	-11.1766418106563\\
42.113	-6.51984465928258\\
28.433709	-4.40204606100836\\
57.941678	-8.97040675938952\\
58.478808	-9.05356407807593\\
66.783894	-10.3393397435945\\
74.206353	-11.4884689832567\\
68.596584	-10.6199765354505\\
110.14505	-17.0524212473325\\
136.02936	-21.0597748035435\\
148.643937	-23.0127366557639\\
96.753293	-14.9791380484423\\
57.043206	-8.83130724449591\\
34.09084	-5.27787099243599\\
34.200712	-5.29488114066587\\
54.894686	-8.4986779697503\\
35.543592	-5.50278295236433\\
57.218994	-8.85852236697509\\
35.427616	-5.48482779589946\\
43.050472	-6.66498207082835\\
51.005105	-7.89650135184853\\
66.981642	-10.3699546663424\\
82.609056	-12.7893575040955\\
93.60072	-14.4910634340228\\
125.525737	-19.4336263382319\\
92.531416	-14.3255160739785\\
82.125648	-12.7145173106385\\
66.586146	-10.3087248208465\\
67.585872	-10.4635002636278\\
73.754694	-11.4185441021281\\
58.478808	-9.05356407807593\\
65.784168	-10.1845643008132\\
56.506076	-8.7481499258095\\
42.911316	-6.64343823630202\\
74.206353	-11.4884689832567\\
80.648568	-12.4858389417422\\
50.851295	-7.87268881635963\\
60.978904	-9.44062359777991\\
114.333884	-17.7009275751534\\
75.329397	-11.662336255252\\
77.111619	-11.938256056462\\
108.58986	-16.8116500551669\\
90.392808	-13.9944213538897\\
64.971204	-10.0587029517384\\
42.779484	-6.6230282878034\\
44.119776	-6.83052943087274\\
77.331345	-11.9722735662002\\
109.83743	-17.0047961763547\\
134.140928	-20.7674117971175\\
117.00729	-18.1148185786723\\
82.864188	-12.8288564950867\\
59.006172	-9.13520944551348\\
50.22751	-7.77611575576574\\
35.317744	-5.46781764766959\\
41.842012	-6.47789087625763\\
28.165144	-4.36046740166516\\
50.065155	-7.75098030163856\\
48.50142	-7.50888619083469\\
35.098	-5.43379735120984\\
57.756124	-8.94167968566149\\
65.784168	-10.1845643008132\\
75.329397	-11.662336255252\\
92.79508	-14.3663359709757\\
111.40971	-17.2482132057968\\
124.80309	-19.3217476740784\\
115.340989	-17.8568454189448\\
84.582972	-13.0949549607116\\
91.462112	-14.1599687139341\\
74.206353	-11.4884689832567\\
44.390764	-6.87248321389769\\
84.582972	-13.0949549607116\\
125.877295	-19.4880537964687\\
114.333884	-17.7009275751534\\
72.643857	-11.2465666917849\\
34.76228	-5.38182189828523\\
26.109401	-4.04220166378357\\
6.372399	-0.986561194570979\\
-13.139903	2.03429170085345\\
26.910213	-4.16618166618875\\
34.42656	-5.32984644536061\\
48.1938	-7.46126111985688\\
33.864992	-5.24290568774125\\
26.910213	-4.16618166618875\\
26.822319	-4.15257410494915\\
40.0989	-6.20802600166472\\
26.822319	-4.15257410494915\\
27.628014	-4.27731008297876\\
50.38132	-7.79992829125464\\
73.534968	-11.3845265923899\\
81.131976	-12.5606791351992\\
67.190376	-10.4022704181319\\
92.004088	-14.243876279984\\
108.58986	-16.8116500551669\\
110.77738	-17.1503172265646\\
134.888	-20.8830718875717\\
132.231744	-20.4718360104011\\
90.920136	-14.0760611478842\\
66.981642	-10.3699546663424\\
69.200814	-10.7135221327359\\
117.684797	-18.2197087644956\\
115.670587	-17.9078730769129\\
74.658012	-11.5583938643852\\
49.13375	-7.60678217006686\\
26.646531	-4.12535898246997\\
26.377966	-4.08378032312677\\
25.3916	-3.93107324699355\\
13.052027	-2.02068692633538\\
33.75512	-5.22589553951138\\
47.723825	-7.38850059475189\\
40.90454	-6.33275346471186\\
39.564248	-6.12525232164252\\
32.07652	-4.9660182748883\\
19.44522	-3.010467403547\\
26.016624	-4.02783812691955\\
26.285189	-4.06941678626275\\
32.522112	-5.03500387604279\\
12.783517	-1.97911678197464\\
13.095965	-2.02748931359441\\
42.508496	-6.58107450477845\\
59.367514	-9.1911516417207\\
83.347596	-12.9036966885437\\
97.054804	-15.0258173370958\\
68.398836	-10.5893616127026\\
119.0215	-18.4266542662551\\
152.447589	-23.6016099295267\\
144.428529	-22.360116198055\\
126.248384	-19.5455050023854\\
94.40636	-14.61579089707\\
107.34229	-16.6185039339791\\
83.105892	-12.8662765918152\\
58.293254	-9.0248370043479\\
42.508496	-6.58107450477845\\
44.251608	-6.85093937937136\\
80.151732	-12.4089198540225\\
50.851295	-7.87268881635963\\
73.315242	-11.3505090826517\\
65.784168	-10.1845643008132\\
51.005105	-7.89650135184853\\
66.179664	-10.2457941463091\\
59.367514	-9.1911516417207\\
73.986627	-11.4544514735184\\
65.377686	-10.1216336262758\\
51.005105	-7.89650135184853\\
70.19025	-10.8667044997081\\
32.522112	-5.03500387604279\\
12.917772	-1.99990185415501\\
40.0989	-6.20802600166472\\
39.022272	-6.04134475559262\\
12.15618	-1.88199380833181\\
6.192912	-0.958773400817016\\
12.96171	-2.00670424141404\\
33.4194	-5.17392008658676\\
34.200712	-5.29488114066587\\
47.877635	-7.4123131302408\\
42.508496	-6.58107450477845\\
78.929784	-12.2197404761173\\
48.817585	-7.55783418045077\\
35.427616	-5.48482779589946\\
61.752306	-9.56036004256367\\
39.827912	-6.16607221863977\\
25.92873	-4.01423056567996\\
24.947247	-3.86227946517117\\
12.6932	-1.9651340970533\\
18.976484	-2.93789869777412\\
19.44522	-3.010467403547\\
34.09084	-5.27787099243599\\
55.256028	-8.55462016595751\\
47.56147	-7.36336514062472\\
27.090884	-4.19415276429235\\
40.0989	-6.20802600166472\\
40.50172	-6.27038973318829\\
39.967068	-6.18761605316609\\
33.08368	-5.12194463366215\\
13.139903	-2.03429170085345\\
38.619452	-5.97898102406905\\
19.511136	-3.02067237779631\\
25.92873	-4.01423056567996\\
26.822319	-4.15257410494915\\
40.633552	-6.29079968168691\\
33.193552	-5.13895478189202\\
26.46586	-4.09738788436636\\
19.511136	-3.02067237779631\\
26.016624	-4.02783812691955\\
27.808685	-4.30528118108236\\
79.910028	-12.371499757294\\
78.674652	-12.1802414851261\\
42.508496	-6.58107450477845\\
69.067206	-10.6928372277128\\
42.647652	-6.60261833930478\\
71.752746	-11.1086067911799\\
48.663775	-7.53402164496187\\
41.842012	-6.47789087625763\\
48.34761	-7.48507365534579\\
42.113	-6.51984465928258\\
27.359449	-4.23573142363555\\
33.639144	-5.20794038304651\\
54.54311	-8.44424772479192\\
41.571024	-6.43593709323268\\
68.395821	-10.588894836846\\
34.200712	-5.29488114066587\\
48.031445	-7.4361256657297\\
35.317744	-5.46781764766959\\
70.849428	-10.9687570289228\\
43.716956	-6.76816569934917\\
91.198448	-14.1191488169368\\
106.07763	-16.4227119755148\\
80.406864	-12.4484188450137\\
56.867418	-8.80409212201672\\
58.117466	-8.99762188186871\\
74.426079	-11.5224864929949\\
80.890272	-12.5232590384707\\
51.791245	-8.0182098665696\\
79.413192	-12.2945806695743\\
60.75258	-9.40558459978244\\
26.553754	-4.11099544560595\\
20.85509	-3.2287404638795\\
70.629702	-10.9347395191846\\
62.960766	-9.74745123713438\\
50.53513	-7.82374082674354\\
65.575434	-10.1522485490238\\
72.192198	-11.1766418106563\\
56.681864	-8.77536504828869\\
42.647652	-6.60261833930478\\
58.117466	-8.99762188186871\\
72.192198	-11.1766418106563\\
50.38132	-7.79992829125464\\
64.575708	-9.99747310624253\\
56.330288	-8.72093480333032\\
35.098	-5.43379735120984\\
48.50142	-7.50888619083469\\
27.54012	-4.26370252173916\\
50.065155	-7.75098030163856\\
65.575434	-10.1522485490238\\
88.239552	-13.6610588617455\\
80.648568	-12.4858389417422\\
78.929784	-12.2197404761173\\
48.817585	-7.55783418045077\\
34.872152	-5.3988320465151\\
42.376664	-6.56066455627983\\
50.22751	-7.77611575576574\\
73.534968	-11.3845265923899\\
74.658012	-11.5583938643852\\
88.781528	-13.7449664277954\\
63.971478	-9.90392750895717\\
51.16746	-7.92163680597571\\
76.000782	-11.7662786461188\\
112.319674	-17.3890918875706\\
74.206353	-11.4884689832567\\
90.114496	-13.951333684837\\
89.045192	-13.7857863247926\\
58.478808	-9.05356407807593\\
72.863583	-11.2805842015231\\
58.654596	-9.08077920055511\\
66.981642	-10.3699546663424\\
88.239552	-13.6610588617455\\
59.006172	-9.13520944551348\\
73.534968	-11.3845265923899\\
66.179664	-10.2457941463091\\
65.575434	-10.1522485490238\\
36.880368	-5.7097397558278\\
58.117466	-8.99762188186871\\
50.38132	-7.79992829125464\\
58.117466	-8.99762188186871\\
55.61737	-8.61056236216473\\
32.857832	-5.0869793289674\\
18.910568	-2.9276937235248\\
12.783517	-1.97911678197464\\
34.76228	-5.38182189828523\\
65.575434	-10.1522485490238\\
73.315242	-11.3505090826517\\
64.773456	-10.0280880289905\\
50.38132	-7.79992829125464\\
71.752746	-11.1086067911799\\
49.59518	-7.67821977653357\\
54.894686	-8.4986779697503\\
34.42656	-5.32984644536061\\
48.34761	-7.48507365534579\\
41.439192	-6.41552714473406\\
41.71018	-6.45748092775901\\
41.571024	-6.43593709323268\\
41.30736	-6.39511719623543\\
33.529272	-5.19093023481664\\
33.639144	-5.20794038304651\\
28.340932	-4.38768252414434\\
63.367248	-9.81038191167181\\
42.376664	-6.56066455627983\\
56.506076	-8.7481499258095\\
43.050472	-6.66498207082835\\
72.424131	-11.2125491820467\\
58.830384	-9.1079943230343\\
87.712224	-13.579419067751\\
51.32127	-7.94544934146461\\
80.890272	-12.5232590384707\\
70.629702	-10.9347395191846\\
43.182304	-6.68539201932697\\
64.169226	-9.9345424317051\\
57.043206	-8.83130724449591\\
64.971204	-10.0587029517384\\
73.534968	-11.3845265923899\\
80.151732	-12.4089198540225\\
43.987944	-6.81011948237412\\
82.609056	-12.7893575040955\\
89.587168	-13.8696938908425\\
88.239552	-13.6610588617455\\
58.117466	-8.99762188186871\\
58.293254	-9.0248370043479\\
65.377686	-10.1216336262758\\
42.911316	-6.64343823630202\\
56.506076	-8.7481499258095\\
42.376664	-6.56066455627983\\
49.44137	-7.65440724104467\\
44.119776	-6.83052943087274\\
83.844432	-12.9806157762634\\
98.800394	-15.296065850353\\
90.920136	-14.0760611478842\\
95.309214	-14.7555688238387\\
49.911345	-7.72716776614965\\
42.113	-6.51984465928258\\
34.42656	-5.32984644536061\\
27.628014	-4.27731008297876\\
28.609497	-4.42926118348755\\
51.005105	-7.89650135184853\\
66.586146	-10.3087248208465\\
80.890272	-12.5232590384707\\
71.520813	-11.0726994197896\\
44.390764	-6.87248321389769\\
75.768849	-11.7303712747284\\
99.958831	-15.4754125909691\\
98.800394	-15.296065850353\\
86.54346	-13.398473523065\\
133.373104	-20.6485389263729\\
104.83006	-16.229565854327\\
64.971204	-10.0587029517384\\
35.317744	-5.46781764766959\\
43.453292	-6.72734580235192\\
66.981642	-10.3699546663424\\
84.086136	-13.0180358729919\\
106.07763	-16.4227119755148\\
74.426079	-11.5224864929949\\
63.762744	-9.87161175716768\\
34.76228	-5.38182189828523\\
55.968946	-8.6649926071231\\
42.244832	-6.5402546077812\\
41.30736	-6.39511719623543\\
27.54012	-4.26370252173916\\
27.359449	-4.23573142363555\\
19.782124	-3.06262616082126\\
20.382692	-3.15560481509277\\
34.646304	-5.36386674182036\\
63.762744	-9.87161175716768\\
49.59518	-7.67821977653357\\
56.867418	-8.80409212201672\\
49.911345	-7.72716776614965\\
50.697485	-7.84887628087072\\
71.301087	-11.0386819100514\\
43.182304	-6.68539201932697\\
73.754694	-11.4185441021281\\
80.151732	-12.4089198540225\\
51.005105	-7.89650135184853\\
65.377686	-10.1216336262758\\
50.851295	-7.87268881635963\\
67.190376	-10.4022704181319\\
96.182009	-14.8906930804673\\
65.575434	-10.1522485490238\\
51.16746	-7.92163680597571\\
58.293254	-9.0248370043479\\
63.971478	-9.90392750895717\\
43.453292	-6.72734580235192\\
52.41503	-8.11478292716349\\
82.367352	-12.751937407367\\
90.656472	-14.0352412508869\\
102.275705	-15.8341060722012\\
131.858208	-20.414005965174\\
98.800394	-15.296065850353\\
89.308856	-13.8266062217899\\
67.585872	-10.4635002636278\\
69.59631	-10.7747519782318\\
106.07763	-16.4227119755148\\
82.125648	-12.7145173106385\\
68.190102	-10.5570458609131\\
89.045192	-13.7857863247926\\
56.681864	-8.77536504828869\\
42.376664	-6.56066455627983\\
56.1545	-8.69371968085113\\
34.200712	-5.29488114066587\\
20.382692	-3.15560481509277\\
33.75512	-5.22589553951138\\
26.734425	-4.13896654370956\\
27.090884	-4.19415276429235\\
27.715908	-4.29091764421835\\
49.44137	-7.65440724104467\\
57.941678	-8.97040675938952\\
73.095516	-11.3164915729135\\
73.986627	-11.4544514735184\\
75.329397	-11.662336255252\\
90.392808	-13.9944213538897\\
88.781528	-13.7449664277954\\
66.179664	-10.2457941463091\\
65.981916	-10.2151792235612\\
58.654596	-9.08077920055511\\
67.388124	-10.4328853408798\\
89.045192	-13.7857863247926\\
66.783894	-10.3393397435945\\
66.586146	-10.3087248208465\\
58.830384	-9.1079943230343\\
68.190102	-10.5570458609131\\
90.656472	-14.0352412508869\\
75.329397	-11.662336255252\\
74.877738	-11.5924113741234\\
79.910028	-12.371499757294\\
55.968946	-8.6649926071231\\
33.75512	-5.22589553951138\\
27.628014	-4.27731008297876\\
41.571024	-6.43593709323268\\
42.244832	-6.5402546077812\\
41.973844	-6.49830082475625\\
35.543592	-5.50278295236433\\
52.26122	-8.09097039167459\\
92.531416	-14.3255160739785\\
114.663482	-17.7519552331215\\
76.220508	-11.800296155857\\
91.198448	-14.1191488169368\\
81.387108	-12.6001781261904\\
66.377412	-10.276409069057\\
66.783894	-10.3393397435945\\
66.586146	-10.3087248208465\\
83.105892	-12.8662765918152\\
93.60072	-14.4910634340228\\
125.525737	-19.4336263382319\\
109.52981	-16.9571711053769\\
102.275705	-15.8341060722012\\
122.303122	-18.9347079710609\\
83.844432	-12.9806157762634\\
97.927599	-15.1609415937244\\
82.367352	-12.751937407367\\
68.596584	-10.6199765354505\\
84.824676	-13.1323750574401\\
102.64254	-15.8908986829292\\
43.182304	-6.68539201932697\\
64.366974	-9.96515735445304\\
43.585124	-6.74775575085054\\
51.16746	-7.92163680597571\\
64.366974	-9.96515735445304\\
43.050472	-6.66498207082835\\
52.26122	-8.09097039167459\\
103.1485	-15.9692303288298\\
137.17072	-21.2364777195153\\
144.428529	-22.360116198055\\
126.599942	-19.5999324606222\\
103.450011	-16.0159096174833\\
129.09991	-19.9869721636398\\
142.8245	-22.1117838562844\\
110.46976	-17.1026921555868\\
119.0215	-18.4266542662551\\
108.26515	-16.7613791469125\\
75.768849	-11.7303712747284\\
61.340246	-9.49656579398712\\
82.609056	-12.7893575040955\\
65.575434	-10.1522485490238\\
36.770496	-5.69272960759793\\
57.941678	-8.97040675938952\\
50.53513	-7.82374082674354\\
51.005105	-7.89650135184853\\
59.904644	-9.2743089604071\\
74.206353	-11.4884689832567\\
51.16746	-7.92163680597571\\
43.856112	-6.78970953387549\\
51.791245	-8.0182098665696\\
67.388124	-10.4328853408798\\
66.981642	-10.3699546663424\\
58.654596	-9.08077920055511\\
60.080432	-9.30152408288629\\
90.656472	-14.0352412508869\\
81.387108	-12.6001781261904\\
60.978904	-9.44062359777991\\
84.086136	-13.0180358729919\\
82.367352	-12.751937407367\\
65.575434	-10.1522485490238\\
49.13375	-7.60678217006686\\
40.90454	-6.33275346471186\\
28.165144	-4.36046740166516\\
35.543592	-5.50278295236433\\
48.817585	-7.55783418045077\\
33.864992	-5.24290568774125\\
27.090884	-4.19415276429235\\
27.896579	-4.31888874232196\\
42.779484	-6.6230282878034\\
43.716956	-6.76816569934917\\
74.658012	-11.5583938643852\\
89.850832	-13.9105137878398\\
82.864188	-12.8288564950867\\
80.648568	-12.4858389417422\\
61.554558	-9.52974511981573\\
27.271555	-4.22212386239596\\
33.974864	-5.25991583597112\\
41.036372	-6.35316341321049\\
49.911345	-7.72716776614965\\
50.851295	-7.87268881635963\\
67.794606	-10.4958160154173\\
89.308856	-13.8266062217899\\
87.712224	-13.579419067751\\
60.441774	-9.3574662790935\\
89.587168	-13.8696938908425\\
79.910028	-12.371499757294\\
35.317744	-5.46781764766959\\
49.757535	-7.70335523066075\\
43.716956	-6.76816569934917\\
67.388124	-10.4328853408798\\
103.58249	-16.0364197331392\\
72.424131	-11.2125491820467\\
57.756124	-8.94167968566149\\
50.851295	-7.87268881635963\\
66.981642	-10.3699546663424\\
77.783004	-12.0421984473287\\
125.174179	-19.3791988799951\\
101.40291	-15.6989818155727\\
105.77001	-16.375086904537\\
72.863583	-11.2805842015231\\
43.716956	-6.76816569934917\\
47.723825	-7.38850059475189\\
27.271555	-4.22212386239596\\
34.310584	-5.31189128889574\\
48.1938	-7.46126111985688\\
21.591152	-3.34269600966349\\
52.26122	-8.09097039167459\\
84.086136	-13.0180358729919\\
102.577216	-15.8807853608547\\
134.140928	-20.7674117971175\\
113.656377	-17.5960373893301\\
59.191726	-9.16393651924151\\
65.981916	-10.2151792235612\\
51.47508	-7.96926187695352\\
83.105892	-12.8662765918152\\
101.117268	-15.6547593315851\\
99.958831	-15.4754125909691\\
97.054804	-15.0258173370958\\
59.904644	-9.2743089604071\\
68.596584	-10.6199765354505\\
90.114496	-13.951333684837\\
80.151732	-12.4089198540225\\
42.647652	-6.60261833930478\\
34.536432	-5.34685659359048\\
26.910213	-4.16618166618875\\
19.712546	-3.0518542435581\\
26.910213	-4.16618166618875\\
20.921006	-3.23894543812882\\
49.28756	-7.63059470555576\\
48.663775	-7.53402164496187\\
34.42656	-5.32984644536061\\
45.843925	-7.09745849433194\\
30.73364	-4.75811646318983\\
18.306338	-2.83414812623944\\
12.739579	-1.97231439471561\\
33.309528	-5.15690993835689\\
48.34761	-7.48507365534579\\
49.28756	-7.63059470555576\\
43.182304	-6.68539201932697\\
66.783894	-10.3393397435945\\
93.60072	-14.4910634340228\\
126.599942	-19.5999324606222\\
144.033015	-22.2988835658376\\
135.655824	-21.0019447583164\\
122.674211	-18.9921591769775\\
82.367352	-12.751937407367\\
67.190376	-10.4022704181319\\
68.190102	-10.5570458609131\\
75.549123	-11.6963537649902\\
66.377412	-10.276409069057\\
43.050472	-6.66498207082835\\
41.973844	-6.49830082475625\\
42.911316	-6.64343823630202\\
50.53513	-7.82374082674354\\
62.763018	-9.71683631438645\\
40.230732	-6.22843595016334\\
13.586606	-2.10344930465359\\
27.896579	-4.31888874232196\\
34.536432	-5.34685659359048\\
42.113	-6.51984465928258\\
34.982024	-5.41584219474497\\
49.911345	-7.72716776614965\\
37.106216	-5.74470506052254\\
88.781528	-13.7449664277954\\
67.794606	-10.4958160154173\\
78.234663	-12.1121233284573\\
135.655824	-21.0019447583164\\
134.514464	-20.8252418423446\\
99.086036	-15.3402883343405\\
81.387108	-12.6001781261904\\
49.59518	-7.67821977653357\\
42.779484	-6.6230282878034\\
35.207872	-5.45080749943971\\
65.179938	-10.0910187035279\\
43.716956	-6.76816569934917\\
64.366974	-9.96515735445304\\
50.697485	-7.84887628087072\\
63.762744	-9.87161175716768\\
28.521603	-4.41565362224795\\
42.376664	-6.56066455627983\\
54.357556	-8.41552065106389\\
20.316776	-3.14539984084346\\
19.983534	-3.09380802658305\\
33.08368	-5.12194463366215\\
13.23022	-2.04827438577479\\
19.309726	-2.98949051203453\\
19.379304	-3.00026242929769\\
13.139903	-2.03429170085345\\
32.973808	-5.10493448543227\\
34.09084	-5.27787099243599\\
42.376664	-6.56066455627983\\
63.564996	-9.84099683441974\\
41.842012	-6.47789087625763\\
43.314136	-6.7058019678256\\
74.426079	-11.5224864929949\\
88.781528	-13.7449664277954\\
59.367514	-9.1911516417207\\
63.564996	-9.84099683441974\\
53.46885	-8.27793308741912\\
13.720861	-2.12423437683396\\
57.043206	-8.83130724449591\\
50.697485	-7.84887628087072\\
74.658012	-11.5583938643852\\
83.105892	-12.8662765918152\\
95.880498	-14.8440137918138\\
64.366974	-9.96515735445304\\
49.757535	-7.70335523066075\\
42.911316	-6.64343823630202\\
49.44137	-7.65440724104467\\
33.75512	-5.22589553951138\\
26.910213	-4.16618166618875\\
26.822319	-4.15257410494915\\
26.285189	-4.06941678626275\\
19.24381	-2.97928553778521\\
26.197295	-4.05580922502316\\
20.45227	-3.16637673235593\\
48.1938	-7.46126111985688\\
47.723825	-7.38850059475189\\
28.07725	-4.34685984042556\\
43.987944	-6.81011948237412\\
86.628272	-13.4116039356512\\
36.550752	-5.65870931113818\\
59.191726	-9.16393651924151\\
102.33492	-15.8432736119514\\
56.1545	-8.69371968085113\\
34.872152	-5.3988320465151\\
56.681864	-8.77536504828869\\
61.752306	-9.56036004256367\\
47.56147	-7.36336514062472\\
20.115366	-3.11421797508167\\
32.41224	-5.01799372781291\\
6.372399	-0.986561194570979\\
13.052027	-2.02068692633538\\
33.639144	-5.20794038304651\\
33.529272	-5.19093023481664\\
40.633552	-6.29079968168691\\
34.310584	-5.31189128889574\\
47.40766	-7.33955260513581\\
39.161428	-6.06288859011895\\
19.0424	-2.94810367202343\\
6.3492	-0.982969574970127\\
32.186392	-4.98302842311817\\
26.016624	-4.02783812691955\\
41.571024	-6.43593709323268\\
64.169226	-9.9345424317051\\
42.376664	-6.56066455627983\\
49.28756	-7.63059470555576\\
57.404548	-8.88724944070312\\
73.986627	-11.4544514735184\\
90.392808	-13.9944213538897\\
106.40234	-16.4729828837691\\
104.52244	-16.1819407833492\\
74.426079	-11.5224864929949\\
67.190376	-10.4022704181319\\
74.658012	-11.5583938643852\\
75.549123	-11.6963537649902\\
85.06638	-13.1697951541686\\
116.348094	-18.0127632627362\\
107.95753	-16.7137540759347\\
83.602728	-12.9431956795349\\
75.097464	-11.6264288838617\\
75.768849	-11.7303712747284\\
81.131976	-12.5606791351992\\
65.575434	-10.1522485490238\\
66.783894	-10.3393397435945\\
75.097464	-11.6264288838617\\
81.870516	-12.6750183196474\\
74.658012	-11.5583938643852\\
66.783894	-10.3393397435945\\
65.784168	-10.1845643008132\\
56.867418	-8.80409212201672\\
57.404548	-8.88724944070312\\
45.057248	-6.97566684241851\\
74.658012	-11.5583938643852\\
50.53513	-7.82374082674354\\
35.989184	-5.57176855351882\\
48.663775	-7.53402164496187\\
47.091495	-7.29060461551973\\
20.45227	-3.16637673235593\\
39.967068	-6.18761605316609\\
39.564248	-6.12525232164252\\
19.84804	-3.07283113507057\\
32.522112	-5.03500387604279\\
26.46586	-4.09738788436636\\
13.139903	-2.03429170085345\\
31.966648	-4.94900812665842\\
19.580714	-3.03144429505947\\
19.712546	-3.0518542435581\\
46.783875	-7.24297954454192\\
20.115366	-3.11421797508167\\
47.091495	-7.29060461551973\\
27.447343	-4.24933898487515\\
41.30736	-6.39511719623543\\
48.663775	-7.53402164496187\\
44.119776	-6.83052943087274\\
80.890272	-12.5232590384707\\
28.970839	-4.48520337969476\\
50.851295	-7.87268881635963\\
63.971478	-9.90392750895717\\
61.148076	-9.46681444527831\\
34.646304	-5.36386674182036\\
41.571024	-6.43593709323268\\
67.053051	-10.3810100551125\\
20.382692	-3.15560481509277\\
34.42656	-5.32984644536061\\
48.50142	-7.50888619083469\\
33.864992	-5.24290568774125\\
52.218802	-8.08440332756713\\
6.394377	-0.989963781561261\\
18.976484	-2.93789869777412\\
26.197295	-4.05580922502316\\
26.46586	-4.09738788436636\\
40.230732	-6.22843595016334\\
20.65368	-3.19755859811772\\
49.911345	-7.72716776614965\\
63.762744	-9.87161175716768\\
57.756124	-8.94167968566149\\
51.62889	-7.99307441244242\\
88.239552	-13.6610588617455\\
65.784168	-10.1845643008132\\
68.794332	-10.6505914581985\\
113.656377	-17.5960373893301\\
114.663482	-17.7519552331215\\
83.844432	-12.9806157762634\\
107.34229	-16.6185039339791\\
85.563216	-13.2467142418883\\
105.46239	-16.3274618335592\\
79.171488	-12.2571605728458\\
27.984473	-4.33249630356155\\
42.508496	-6.58107450477845\\
72.643857	-11.2465666917849\\
65.575434	-10.1522485490238\\
64.366974	-9.96515735445304\\
47.723825	-7.38850059475189\\
27.090884	-4.19415276429235\\
35.653464	-5.5197931005942\\
59.367514	-9.1911516417207\\
97.626088	-15.1142623050709\\
79.413192	-12.2945806695743\\
56.330288	-8.72093480333032\\
57.043206	-8.83130724449591\\
44.654428	-6.91330311089494\\
76.000782	-11.7662786461188\\
99.371678	-15.384510818328\\
89.850832	-13.9105137878398\\
83.844432	-12.9806157762634\\
85.06638	-13.1697951541686\\
123.728885	-19.155441551688\\
111.085	-17.1979422975425\\
127.322589	-19.7118111247757\\
133.74664	-20.7063689716\\
102.862858	-15.9250078448423\\
118.014395	-18.2707364224637\\
98.498883	-15.2493865616995\\
51.791245	-8.0182098665696\\
52.098865	-8.06583493754741\\
67.992354	-10.5264309381652\\
83.602728	-12.9431956795349\\
81.870516	-12.6750183196474\\
67.585872	-10.4635002636278\\
60.080432	-9.30152408288629\\
80.648568	-12.4858389417422\\
59.367514	-9.1911516417207\\
74.206353	-11.4884689832567\\
60.617562	-9.38468140157269\\
92.004088	-14.243876279984\\
112.997181	-17.4939820733939\\
66.179664	-10.2457941463091\\
49.44137	-7.65440724104467\\
42.508496	-6.58107450477845\\
42.244832	-6.5402546077812\\
69.286932	-10.726854737451\\
20.65368	-3.19755859811772\\
27.896579	-4.31888874232196\\
78.929784	-12.2197404761173\\
65.179938	-10.0910187035279\\
65.784168	-10.1845643008132\\
56.681864	-8.77536504828869\\
48.34761	-7.48507365534579\\
20.181282	-3.12442294933098\\
33.529272	-5.19093023481664\\
27.628014	-4.27731008297876\\
48.817585	-7.55783418045077\\
41.973844	-6.49830082475625\\
43.453292	-6.72734580235192\\
81.870516	-12.6750183196474\\
83.347596	-12.9036966885437\\
99.371678	-15.384510818328\\
92.79508	-14.3663359709757\\
105.77001	-16.375086904537\\
88.239552	-13.6610588617455\\
57.756124	-8.94167968566149\\
50.22751	-7.77611575576574\\
51.16746	-7.92163680597571\\
58.293254	-9.0248370043479\\
56.506076	-8.7481499258095\\
34.76228	-5.38182189828523\\
28.609497	-4.42926118348755\\
65.981916	-10.2151792235612\\
73.095516	-11.3164915729135\\
60.441774	-9.3574662790935\\
92.79508	-14.3663359709757\\
106.70996	-16.5206079547469\\
73.754694	-11.4185441021281\\
49.911345	-7.72716776614965\\
42.244832	-6.5402546077812\\
43.716956	-6.76816569934917\\
58.293254	-9.0248370043479\\
51.005105	-7.89650135184853\\
59.006172	-9.13520944551348\\
96.753293	-14.9791380484423\\
74.658012	-11.5583938643852\\
74.877738	-11.5924113741234\\
73.754694	-11.4185441021281\\
66.179664	-10.2457941463091\\
64.971204	-10.0587029517384\\
63.971478	-9.90392750895717\\
43.585124	-6.74775575085054\\
74.877738	-11.5924113741234\\
91.198448	-14.1191488169368\\
88.503216	-13.7018787587427\\
76.672167	-11.8702210369855\\
70.409274	-10.9006133273066\\
105.77001	-16.375086904537\\
65.179938	-10.0910187035279\\
51.005105	-7.89650135184853\\
65.377686	-10.1216336262758\\
43.314136	-6.7058019678256\\
58.478808	-9.05356407807593\\
64.169226	-9.9345424317051\\
50.38132	-7.79992829125464\\
60.803116	-9.41340847530072\\
100.831626	-15.6105368475976\\
113.656377	-17.5960373893301\\
67.794606	-10.4958160154173\\
76.220508	-11.800296155857\\
83.844432	-12.9806157762634\\
90.114496	-13.951333684837\\
59.367514	-9.1911516417207\\
57.941678	-8.97040675938952\\
51.32127	-7.94544934146461\\
66.981642	-10.3699546663424\\
53.671145	-8.30925196698955\\
101.990063	-15.7898835882137\\
137.544256	-21.2943077647424\\
152.888256	-23.6698330527065\\
141.220471	-21.8634515145138\\
92.267752	-14.2846961769812\\
84.582972	-13.0949549607116\\
75.768849	-11.7303712747284\\
72.863583	-11.2805842015231\\
50.22751	-7.77611575576574\\
35.989184	-5.57176855351882\\
42.244832	-6.5402546077812\\
40.90454	-6.33275346471186\\
20.65368	-3.19755859811772\\
28.253038	-4.37407496290475\\
62.56527	-9.68622139163851\\
40.633552	-6.29079968168691\\
20.04945	-3.10401300083236\\
21.122416	-3.2701273038906\\
51.32127	-7.94544934146461\\
63.367248	-9.81038191167181\\
56.330288	-8.72093480333032\\
35.317744	-5.46781764766959\\
70.629702	-10.9347395191846\\
35.653464	-5.5197931005942\\
42.647652	-6.60261833930478\\
48.663775	-7.53402164496187\\
33.974864	-5.25991583597112\\
33.08368	-5.12194463366215\\
26.553754	-4.11099544560595\\
34.76228	-5.38182189828523\\
50.065155	-7.75098030163856\\
56.330288	-8.72093480333032\\
48.031445	-7.4361256657297\\
33.193552	-5.13895478189202\\
26.553754	-4.11099544560595\\
27.54012	-4.26370252173916\\
49.757535	-7.70335523066075\\
57.580336	-8.91446456318231\\
49.44137	-7.65440724104467\\
48.817585	-7.55783418045077\\
35.098	-5.43379735120984\\
43.050472	-6.66498207082835\\
52.098865	-8.06583493754741\\
91.462112	-14.1599687139341\\
105.77001	-16.375086904537\\
80.151732	-12.4089198540225\\
43.314136	-6.7058019678256\\
57.580336	-8.91446456318231\\
57.404548	-8.88724944070312\\
48.971395	-7.58164671593968\\
41.842012	-6.47789087625763\\
49.28756	-7.63059470555576\\
41.30736	-6.39511719623543\\
19.84804	-3.07283113507057\\
19.983534	-3.09380802658305\\
27.447343	-4.24933898487515\\
42.911316	-6.64343823630202\\
58.478808	-9.05356407807593\\
74.658012	-11.5583938643852\\
66.981642	-10.3699546663424\\
89.850832	-13.9105137878398\\
84.824676	-13.1323750574401\\
112.65728	-17.4413593269846\\
153.723204	-23.7990979176788\\
109.83743	-17.0047961763547\\
93.60072	-14.4910634340228\\
110.46976	-17.1026921555868\\
121.365308	-18.789517611806\\
125.877295	-19.4880537964687\\
110.982971	-17.1821463858111\\
49.44137	-7.65440724104467\\
27.808685	-4.30528118108236\\
33.75512	-5.22589553951138\\
19.379304	-3.00026242929769\\
19.177894	-2.9690805635359\\
19.712546	-3.0518542435581\\
33.864992	-5.24290568774125\\
34.200712	-5.29488114066587\\
47.25385	-7.31574006964691\\
26.822319	-4.15257410494915\\
33.974864	-5.25991583597112\\
34.646304	-5.36386674182036\\
48.817585	-7.55783418045077\\
41.973844	-6.49830082475625\\
41.168204	-6.37357336170911\\
33.4194	-5.17392008658676\\
32.857832	-5.0869793289674\\
33.529272	-5.19093023481664\\
41.036372	-6.35316341321049\\
54.181768	-8.38830552858471\\
26.285189	-4.06941678626275\\
20.115366	-3.11421797508167\\
32.973808	-5.10493448543227\\
26.377966	-4.08378032312677\\
46.937685	-7.26679208003082\\
55.968946	-8.6649926071231\\
42.647652	-6.60261833930478\\
55.61737	-8.61056236216473\\
42.647652	-6.60261833930478\\
57.404548	-8.88724944070312\\
85.295304	-13.2052366786096\\
41.973844	-6.49830082475625\\
43.050472	-6.66498207082835\\
66.586146	-10.3087248208465\\
89.308856	-13.8266062217899\\
90.392808	-13.9944213538897\\
89.587168	-13.8696938908425\\
66.377412	-10.276409069057\\
73.315242	-11.3505090826517\\
74.658012	-11.5583938643852\\
81.628812	-12.6375982229189\\
75.097464	-11.6264288838617\\
83.105892	-12.8662765918152\\
98.213241	-15.2051640777119\\
87.778836	-13.5897317952329\\
143.637501	-22.2376509336202\\
128.434128	-19.8838972172587\\
72.192198	-11.1766418106563\\
50.065155	-7.75098030163856\\
50.851295	-7.87268881635963\\
51.791245	-8.0182098665696\\
75.329397	-11.662336255252\\
76.220508	-11.800296155857\\
84.341268	-13.0575348639831\\
97.340446	-15.0700398210834\\
66.377412	-10.276409069057\\
59.191726	-9.16393651924151\\
74.426079	-11.5224864929949\\
57.580336	-8.91446456318231\\
55.61737	-8.61056236216473\\
21.323826	-3.30130916965239\\
57.218994	-8.85852236697509\\
73.095516	-11.3164915729135\\
67.388124	-10.4328853408798\\
89.587168	-13.8696938908425\\
80.648568	-12.4858389417422\\
50.22751	-7.77611575576574\\
50.851295	-7.87268881635963\\
74.658012	-11.5583938643852\\
77.331345	-11.9722735662002\\
102.275705	-15.8341060722012\\
128.025705	-19.8206660412494\\
135.282288	-20.9441147130893\\
118.014395	-18.2707364224637\\
106.40234	-16.4729828837691\\
65.377686	-10.1216336262758\\
52.41503	-8.11478292716349\\
84.582972	-13.0949549607116\\
119.451596	-18.4932408098065\\
68.596584	-10.6199765354505\\
78.674115	-12.1801583479337\\
144.846016	-22.4247506431733\\
141.615985	-21.9246841467312\\
119.803154	-18.5476682680433\\
57.218994	-8.85852236697509\\
35.653464	-5.5197931005942\\
57.218994	-8.85852236697509\\
64.169226	-9.9345424317051\\
41.71018	-6.45748092775901\\
27.715908	-4.29091764421835\\
34.09084	-5.27787099243599\\
34.310584	-5.31189128889574\\
47.877635	-7.4123131302408\\
41.439192	-6.41552714473406\\
35.879312	-5.55475840528895\\
50.38132	-7.79992829125464\\
58.830384	-9.1079943230343\\
75.768849	-11.7303712747284\\
91.462112	-14.1599687139341\\
88.239552	-13.6610588617455\\
57.941678	-8.97040675938952\\
73.315242	-11.3505090826517\\
57.580336	-8.91446456318231\\
49.757535	-7.70335523066075\\
41.571024	-6.43593709323268\\
42.113	-6.51984465928258\\
49.28756	-7.63059470555576\\
54.54311	-8.44424772479192\\
32.638088	-5.05295903250766\\
19.111978	-2.95887558928659\\
19.712546	-3.0518542435581\\
27.178778	-4.20776032553195\\
41.439192	-6.41552714473406\\
40.230732	-6.22843595016334\\
33.529272	-5.19093023481664\\
62.356536	-9.65390563984903\\
50.065155	-7.75098030163856\\
56.506076	-8.7481499258095\\
34.76228	-5.38182189828523\\
28.165144	-4.36046740166516\\
42.779484	-6.6230282878034\\
64.169226	-9.9345424317051\\
57.756124	-8.94167968566149\\
63.762744	-9.87161175716768\\
41.571024	-6.43593709323268\\
41.973844	-6.49830082475625\\
48.663775	-7.53402164496187\\
41.973844	-6.49830082475625\\
42.911316	-6.64343823630202\\
74.877738	-11.5924113741234\\
96.753293	-14.9791380484423\\
73.315242	-11.3505090826517\\
75.768849	-11.7303712747284\\
97.340446	-15.0700398210834\\
79.668324	-12.3340796605655\\
49.13375	-7.60678217006686\\
41.036372	-6.35316341321049\\
27.178778	-4.20776032553195\\
26.910213	-4.16618166618875\\
19.983534	-3.09380802658305\\
20.85509	-3.2287404638795\\
60.346098	-9.34265392524501\\
32.522112	-5.03500387604279\\
26.646531	-4.12535898246997\\
34.200712	-5.29488114066587\\
47.56147	-7.36336514062472\\
39.827912	-6.16607221863977\\
32.186392	-4.98302842311817\\
25.92873	-4.01423056567996\\
34.310584	-5.31189128889574\\
64.169226	-9.9345424317051\\
63.971478	-9.90392750895717\\
41.973844	-6.49830082475625\\
28.07725	-4.34685984042556\\
56.330288	-8.72093480333032\\
49.59518	-7.67821977653357\\
57.756124	-8.94167968566149\\
66.377412	-10.276409069057\\
81.628812	-12.6375982229189\\
80.890272	-12.5232590384707\\
71.520813	-11.0726994197896\\
48.971395	-7.58164671593968\\
41.71018	-6.45748092775901\\
35.427616	-5.48482779589946\\
55.431816	-8.5818352884367\\
33.974864	-5.25991583597112\\
35.098	-5.43379735120984\\
50.38132	-7.79992829125464\\
72.192198	-11.1766418106563\\
58.293254	-9.0248370043479\\
67.585872	-10.4635002636278\\
94.436419	-14.6204445672102\\
66.783894	-10.3393397435945\\
68.398836	-10.5893616127026\\
90.392808	-13.9944213538897\\
75.768849	-11.7303712747284\\
98.498883	-15.2493865616995\\
87.778836	-13.5897317952329\\
145.637044	-22.5472159076082\\
142.033472	-21.9893185918496\\
103.1485	-15.9692303288298\\
129.451468	-20.0413996218766\\
146.054531	-22.6118503527265\\
138.685616	-21.4710106807141\\
155.439486	-24.0648090290107\\
150.07559	-23.2343821136035\\
174.138555	-26.9597589293552\\
136.797184	-21.1786476742881\\
84.824676	-13.1323750574401\\
60.617562	-9.38468140157269\\
75.097464	-11.6264288838617\\
51.16746	-7.92163680597571\\
44.522596	-6.89289316239631\\
76.672167	-11.8702210369855\\
90.114496	-13.951333684837\\
58.478808	-9.05356407807593\\
43.716956	-6.76816569934917\\
56.681864	-8.77536504828869\\
42.508496	-6.58107450477845\\
47.091495	-7.29060461551973\\
33.08368	-5.12194463366215\\
52.39459	-8.11161845004631\\
26.646531	-4.12535898246997\\
26.46586	-4.09738788436636\\
6.505488	-1.00716574912324\\
18.910568	-2.9276937235248\\
12.290435	-1.90277888051219\\
18.910568	-2.9276937235248\\
24.947247	-3.86227946517117\\
24.85447	-3.84791592830715\\
26.109401	-4.04220166378357\\
20.518186	-3.17658170660524\\
48.817585	-7.55783418045077\\
55.08024	-8.52740504347832\\
34.646304	-5.36386674182036\\
72.643857	-11.2465666917849\\
84.489664	-13.0805092155624\\
46.783875	-7.24297954454192\\
13.005648	-2.01350662867307\\
18.910568	-2.9276937235248\\
12.6932	-1.9651340970533\\
33.08368	-5.12194463366215\\
41.842012	-6.47789087625763\\
48.031445	-7.4361256657297\\
21.122416	-3.2701273038906\\
44.251608	-6.85093937937136\\
89.308856	-13.8266062217899\\
87.975888	-13.6202389647482\\
56.330288	-8.72093480333032\\
42.113	-6.51984465928258\\
72.192198	-11.1766418106563\\
81.628812	-12.6375982229189\\
86.364608	-13.3707840386539\\
40.90454	-6.33275346471186\\
27.359449	-4.23573142363555\\
34.09084	-5.27787099243599\\
32.302368	-5.00098357958304\\
18.43817	-2.85455807473807\\
18.976484	-2.93789869777412\\
48.34761	-7.48507365534579\\
85.295304	-13.2052366786096\\
62.960766	-9.74745123713438\\
33.639144	-5.20794038304651\\
20.04945	-3.10401300083236\\
24.766576	-3.83430836706756\\
36.87634	-5.70911614947614\\
24.678682	-3.82070080582797\\
18.976484	-2.93789869777412\\
33.08368	-5.12194463366215\\
34.76228	-5.38182189828523\\
71.301087	-11.0386819100514\\
57.580336	-8.91446456318231\\
65.377686	-10.1216336262758\\
72.863583	-11.2805842015231\\
65.377686	-10.1216336262758\\
72.424131	-11.2125491820467\\
65.575434	-10.1522485490238\\
83.844432	-12.9806157762634\\
115.670587	-17.9078730769129\\
95.594856	-14.7997913078263\\
41.439192	-6.41552714473406\\
28.702274	-4.44362472035156\\
51.791245	-8.0182098665696\\
88.503216	-13.7018787587427\\
64.169226	-9.9345424317051\\
27.984473	-4.33249630356155\\
65.784168	-10.1845643008132\\
61.340246	-9.49656579398712\\
108.58986	-16.8116500551669\\
133.74664	-20.7063689716\\
126.599942	-19.5999324606222\\
124.099974	-19.2128927576047\\
100.530115	-15.5638575589441\\
87.975888	-13.6202389647482\\
42.376664	-6.56066455627983\\
57.218994	-8.85852236697509\\
51.005105	-7.89650135184853\\
65.981916	-10.2151792235612\\
66.377412	-10.276409069057\\
71.972472	-11.1426243009181\\
43.453292	-6.72734580235192\\
65.179938	-10.0910187035279\\
48.817585	-7.55783418045077\\
33.4194	-5.17392008658676\\
19.379304	-3.00026242929769\\
26.197295	-4.05580922502316\\
26.016624	-4.02783812691955\\
24.947247	-3.86227946517117\\
19.111978	-2.95887558928659\\
33.974864	-5.25991583597112\\
54.00598	-8.36109040610552\\
33.193552	-5.13895478189202\\
33.864992	-5.24290568774125\\
35.653464	-5.5197931005942\\
77.936112	-12.0659023006779\\
33.639144	-5.20794038304651\\
19.177894	-2.9690805635359\\
6.639798	-1.02795933628607\\
35.763336	-5.53680324882408\\
67.388124	-10.4328853408798\\
93.879032	-14.5341511030755\\
146.450045	-22.6730829849439\\
155.85696	-24.1294414614969\\
160.03377	-24.7760862593346\\
110.77738	-17.1503172265646\\
121.365308	-18.789517611806\\
136.797184	-21.1786476742881\\
120.028605	-18.5825721100465\\
112.33257	-17.3910884187302\\
121.365308	-18.789517611806\\
138.31208	-21.413180635487\\
141.73616	-21.9432893834023\\
188.67072	-29.2095976576116\\
125.174179	-19.3791988799951\\
73.315242	-11.3505090826517\\
42.244832	-6.5402546077812\\
34.982024	-5.41584219474497\\
34.872152	-5.3988320465151\\
28.878062	-4.47083984283075\\
72.192198	-11.1766418106563\\
43.182304	-6.68539201932697\\
65.981916	-10.2151792235612\\
64.169226	-9.9345424317051\\
42.376664	-6.56066455627983\\
72.192198	-11.1766418106563\\
73.754694	-11.4185441021281\\
71.520813	-11.0726994197896\\
34.310584	-5.31189128889574\\
27.628014	-4.27731008297876\\
34.42656	-5.32984644536061\\
51.945055	-8.04202240205851\\
86.060052	-13.323633329608\\
125.525737	-19.4336263382319\\
119.0215	-18.4266542662551\\
121.03571	-18.7384899538379\\
157.990716	-24.459785005315\\
200.625856	-31.0604662635195\\
145.24153	-22.4859832753908\\
92.79508	-14.3663359709757\\
58.293254	-9.0248370043479\\
35.653464	-5.5197931005942\\
64.366974	-9.96515735445304\\
50.851295	-7.87268881635963\\
57.218994	-8.85852236697509\\
42.244832	-6.5402546077812\\
49.44137	-7.65440724104467\\
42.911316	-6.64343823630202\\
57.756124	-8.94167968566149\\
58.654596	-9.08077920055511\\
65.575434	-10.1522485490238\\
41.30736	-6.39511719623543\\
33.08368	-5.12194463366215\\
19.177894	-2.9690805635359\\
19.0424	-2.94810367202343\\
6.372399	-0.986561194570979\\
19.913956	-3.08303610931988\\
33.639144	-5.20794038304651\\
41.036372	-6.35316341321049\\
42.779484	-6.6230282878034\\
70.849428	-10.9687570289228\\
44.119776	-6.83052943087274\\
77.783004	-12.0421984473287\\
116.348094	-18.0127632627362\\
76.672167	-11.8702210369855\\
77.783004	-12.0421984473287\\
110.46976	-17.1026921555868\\
107.95753	-16.7137540759347\\
75.549123	-11.6963537649902\\
87.712224	-13.579419067751\\
50.38132	-7.79992829125464\\
76.000782	-11.7662786461188\\
113.59723	-17.5868803771946\\
168.09039	-26.0233949497359\\
201.153202	-31.1421088442355\\
157.132575	-24.326929449649\\
120.028605	-18.5825721100465\\
103.450011	-16.0159096174833\\
122.042815	-18.8944077976293\\
125.877295	-19.4880537964687\\
83.844432	-12.9806157762634\\
74.877738	-11.5924113741234\\
54.132575	-8.38068957345627\\
93.879032	-14.5341511030755\\
74.877738	-11.5924113741234\\
51.47508	-7.96926187695352\\
50.22751	-7.77611575576574\\
43.987944	-6.81011948237412\\
76.220508	-11.800296155857\\
92.531416	-14.3255160739785\\
98.800394	-15.296065850353\\
59.543302	-9.21836676419989\\
73.534968	-11.3845265923899\\
42.779484	-6.6230282878034\\
41.71018	-6.45748092775901\\
27.271555	-4.22212386239596\\
26.553754	-4.11099544560595\\
19.84804	-3.07283113507057\\
27.359449	-4.23573142363555\\
64.169226	-9.9345424317051\\
50.38132	-7.79992829125464\\
62.960766	-9.74745123713438\\
41.439192	-6.41552714473406\\
33.974864	-5.25991583597112\\
20.518186	-3.17658170660524\\
48.971395	-7.58164671593968\\
49.59518	-7.67821977653357\\
56.681864	-8.77536504828869\\
41.168204	-6.37357336170911\\
28.609497	-4.42926118348755\\
52.26122	-8.09097039167459\\
88.781528	-13.7449664277954\\
48.34761	-7.48507365534579\\
13.586606	-2.10344930465359\\
62.763018	-9.71683631438645\\
56.506076	-8.7481499258095\\
55.968946	-8.6649926071231\\
41.036372	-6.35316341321049\\
20.921006	-3.23894543812882\\
65.575434	-10.1522485490238\\
64.169226	-9.9345424317051\\
33.4194	-5.17392008658676\\
19.712546	-3.0518542435581\\
63.1695	-9.77976698892387\\
79.910028	-12.371499757294\\
70.629702	-10.9347395191846\\
34.536432	-5.34685659359048\\
43.182304	-6.68539201932697\\
74.877738	-11.5924113741234\\
90.114496	-13.951333684837\\
76.891893	-11.9042385467238\\
118.691902	-18.375626608287\\
139.616442	-21.6151191727432\\
83.347596	-12.9036966885437\\
85.06638	-13.1697951541686\\
106.40234	-16.4729828837691\\
68.794332	-10.6505914581985\\
109.83743	-17.0047961763547\\
108.58986	-16.8116500551669\\
80.890272	-12.5232590384707\\
37.551808	-5.81369066167703\\
63.479	-9.82768311748389\\
129.822557	-20.0988508277933\\
144.033015	-22.2988835658376\\
102.275705	-15.8341060722012\\
103.735653	-16.0601321014708\\
125.877295	-19.4880537964687\\
91.462112	-14.1599687139341\\
67.190376	-10.4022704181319\\
60.265986	-9.33025115661432\\
75.768849	-11.7303712747284\\
75.549123	-11.6963537649902\\
76.000782	-11.7662786461188\\
82.367352	-12.751937407367\\
52.098865	-8.06583493754741\\
68.398836	-10.5893616127026\\
81.628812	-12.6375982229189\\
50.53513	-7.82374082674354\\
42.508496	-6.58107450477845\\
34.42656	-5.32984644536061\\
34.310584	-5.31189128889574\\
21.323826	-3.30130916965239\\
42.911316	-6.64343823630202\\
43.856112	-6.78970953387549\\
52.731195	-8.16373091677958\\
76.891893	-11.9042385467238\\
99.371678	-15.384510818328\\
67.585872	-10.4635002636278\\
71.081361	-11.0046644003131\\
49.757535	-7.70335523066075\\
82.125648	-12.7145173106385\\
88.781528	-13.7449664277954\\
49.13375	-7.60678217006686\\
21.591152	-3.34269600966349\\
67.585872	-10.4635002636278\\
83.602728	-12.9431956795349\\
85.80492	-13.2841343386168\\
121.365308	-18.789517611806\\
160.473222	-24.844121278811\\
102.275705	-15.8341060722012\\
103.1485	-15.9692303288298\\
108.58986	-16.8116500551669\\
75.329397	-11.662336255252\\
60.441774	-9.3574662790935\\
76.891893	-11.9042385467238\\
93.073392	-14.4094236400284\\
92.79508	-14.3663359709757\\
82.609056	-12.7893575040955\\
58.654596	-9.08077920055511\\
50.38132	-7.79992829125464\\
36.324904	-5.62374400644344\\
65.377686	-10.1216336262758\\
43.050472	-6.66498207082835\\
44.654428	-6.91330311089494\\
91.198448	-14.1191488169368\\
97.340446	-15.0700398210834\\
51.32127	-7.94544934146461\\
44.793584	-6.93484694542126\\
72.643857	-11.2465666917849\\
41.571024	-6.43593709323268\\
34.310584	-5.31189128889574\\
41.973844	-6.49830082475625\\
55.793158	-8.63777748464391\\
34.200712	-5.29488114066587\\
19.44522	-3.010467403547\\
19.511136	-3.02067237779631\\
13.49873	-2.08984453013553\\
35.653464	-5.5197931005942\\
58.478808	-9.05356407807593\\
60.265986	-9.33025115661432\\
84.824676	-13.1323750574401\\
96.753293	-14.9791380484423\\
64.575708	-9.99747310624253\\
52.731195	-8.16373091677958\\
102.862858	-15.9250078448423\\
133.74664	-20.7063689716\\
101.704421	-15.7456611042262\\
96.281304	-14.9060657201615\\
152.447589	-23.6016099295267\\
110.46976	-17.1026921555868\\
100.244473	-15.5196350749566\\
84.341268	-13.0575348639831\\
85.80492	-13.2841343386168\\
99.673189	-15.4311901069815\\
67.585872	-10.4635002636278\\
62.053164	-9.60693823515271\\
111.40971	-17.2482132057968\\
120.358203	-18.6335997680146\\
120.877359	-18.7139743904337\\
70.629702	-10.9347395191846\\
29.503086	-4.56760472275674\\
44.251608	-6.85093937937136\\
62.56527	-9.68622139163851\\
13.408413	-2.07586184521419\\
20.316776	-3.14539984084346\\
49.44137	-7.65440724104467\\
51.945055	-8.04202240205851\\
81.131976	-12.5606791351992\\
50.22751	-7.77611575576574\\
34.872152	-5.3988320465151\\
33.309528	-5.15690993835689\\
19.913956	-3.08303610931988\\
33.08368	-5.12194463366215\\
19.983534	-3.09380802658305\\
27.808685	-4.30528118108236\\
48.971395	-7.58164671593968\\
33.529272	-5.19093023481664\\
6.550665	-1.01415995571437\\
26.46586	-4.09738788436636\\
33.639144	-5.20794038304651\\
32.74796	-5.06996918073753\\
20.04945	-3.10401300083236\\
19.84804	-3.07283113507057\\
19.309726	-2.98949051203453\\
20.115366	-3.11421797508167\\
49.13375	-7.60678217006686\\
66.179664	-10.2457941463091\\
92.79508	-14.3663359709757\\
124.80309	-19.3217476740784\\
96.467651	-14.9349155644548\\
42.508496	-6.58107450477845\\
42.113	-6.51984465928258\\
33.75512	-5.22589553951138\\
26.197295	-4.05580922502316\\
20.382692	-3.15560481509277\\
41.571024	-6.43593709323268\\
48.34761	-7.48507365534579\\
34.76228	-5.38182189828523\\
49.44137	-7.65440724104467\\
49.911345	-7.72716776614965\\
57.404548	-8.88724944070312\\
58.478808	-9.05356407807593\\
66.179664	-10.2457941463091\\
76.672167	-11.8702210369855\\
88.781528	-13.7449664277954\\
55.431816	-8.5818352884367\\
33.75512	-5.22589553951138\\
34.982024	-5.41584219474497\\
55.968946	-8.6649926071231\\
27.54012	-4.26370252173916\\
39.425092	-6.1037084871162\\
13.095965	-2.02748931359441\\
32.302368	-5.00098357958304\\
30.959488	-4.79308176788458\\
18.573664	-2.87553496625054\\
19.379304	-3.00026242929769\\
33.309528	-5.15690993835689\\
41.71018	-6.45748092775901\\
65.575434	-10.1522485490238\\
79.910028	-12.371499757294\\
47.25385	-7.31574006964691\\
13.811178	-2.1382170617553\\
35.207872	-5.45080749943971\\
54.181768	-8.38830552858471\\
27.090884	-4.19415276429235\\
7.131861	-1.10413947834625\\
72.192198	-11.1766418106563\\
45.057248	-6.97566684241851\\
93.337056	-14.4502435370256\\
113.326779	-17.545009731362\\
81.387108	-12.6001781261904\\
51.16746	-7.92163680597571\\
};
\end{axis}

\begin{axis}[%
width=4.927cm,
height=3cm,
at={(0cm,4.839cm)},
scale only axis,
xmin=0,
xmax=3000,
xlabel style={font=\color{white!15!black}},
xlabel={y(t-1)u(t)},
ymin=-500,
ymax=0,
ylabel style={font=\color{white!15!black}},
ylabel={y(t)},
axis background/.style={fill=white},
title style={font=\small},
title={C6, R = -0.8014},
axis x line*=bottom,
axis y line*=left
]
\addplot[only marks, mark=*, mark options={}, mark size=1.5000pt, color=mycolor1, fill=mycolor1] table[row sep=crcr]{%
x	y\\
748.77738	-140.381\\
871.344867	-178.223\\
1103.022147	-147.705\\
906.02247	-139.16\\
871.41992	-192.871\\
1207.758202	-185.547\\
1175.440245	-219.727\\
1380.105287	-169.678\\
1012.807982	-86.67\\
514.21311	-107.422\\
639.268322	-102.539\\
615.849234	-126.953\\
755.497303	-103.76\\
581.36728	-41.504\\
220.38624	-31.738\\
163.895032	-28.076\\
154.726836	-89.111\\
527.002454	-157.471\\
951.439782	-167.236\\
1013.617396	-174.561\\
1041.954609	-124.512\\
713.578272	-74.463\\
444.469647	-158.691\\
941.513703	-111.084\\
638.733	-83.008\\
492.486464	-142.822\\
847.362926	-123.291\\
722.361969	-103.76\\
626.91792	-173.34\\
1056.85398	-164.795\\
983.661355	-124.512\\
759.149664	-180.664\\
1111.444928	-179.443\\
1087.604023	-142.822\\
852.504518	-117.188\\
686.604492	-89.111\\
522.101349	-108.643\\
646.534493	-137.939\\
820.874989	-130.615\\
779.640935	-136.719\\
821.134314	-140.381\\
848.182002	-151.367\\
936.810363	-213.623\\
1325.957961	-191.65\\
1157.9493	-126.953\\
757.782457	-115.967\\
666.81025	-64.697\\
366.055626	-69.58\\
400.085	-96.436\\
549.20302	-74.463\\
421.311654	-78.125\\
450.625	-111.084\\
652.951752	-128.174\\
753.406772	-119.629\\
725.071369	-190.43\\
1150.57806	-139.16\\
807.68464	-81.787\\
462.750846	-63.477\\
361.501515	-85.449\\
492.869832	-101.318\\
565.86103	-51.27\\
280.70325	-58.594\\
328.302182	-81.787\\
455.226442	-65.918\\
360.90105	-56.152\\
311.531296	-74.463\\
418.556523	-86.67\\
503.03268	-139.16\\
820.48736	-130.615\\
779.640935	-163.574\\
976.373206	-146.484\\
885.056328	-170.898\\
1023.337224	-139.16\\
830.64604	-133.057\\
786.899098	-115.967\\
681.654026	-111.084\\
652.951752	-111.084\\
679.389744	-196.533\\
1259.579997	-280.762\\
1830.287478	-291.748\\
1901.905212	-280.762\\
1778.62727	-175.781\\
1126.580429	-240.479\\
1585.237568	-303.955\\
2020.388885	-317.383\\
2080.445565	-239.258\\
1585.802024	-313.721\\
2148.361408	-404.053\\
2744.732029	-289.307\\
1928.231155	-238.037\\
1534.148465	-158.691\\
990.866604	-122.07\\
748.77738	-102.539\\
628.974226	-129.395\\
781.80459	-92.773\\
540.217179	-61.035\\
346.43466	-59.814\\
341.717382	-75.684\\
444.870552	-118.408\\
700.264912	-107.422\\
635.293708	-111.084\\
669.170016	-146.484\\
890.476236	-152.588\\
952.759472	-207.52\\
1295.75488	-167.236\\
1050.409316	-208.74\\
1295.64918	-148.926\\
908.001822	-130.615\\
803.54348	-151.367\\
914.559414	-108.643\\
648.490067	-98.877\\
593.855262	-122.07\\
737.54694	-123.291\\
729.142974	-85.449\\
503.807304	-95.215\\
550.91399	-67.139\\
387.257752	-76.904\\
444.966544	-81.787\\
464.223012	-57.373\\
324.616434	-75.684\\
435.183	-93.994\\
561.050186	-151.367\\
917.435387	-152.588\\
933.228208	-170.898\\
1013.937834	-97.656\\
582.908664	-140.381\\
874.012106	-211.182\\
1307.005398	-161.133\\
1003.214058	-186.768\\
1169.541216	-191.65\\
1161.59065	-112.305\\
657.994995	-75.684\\
446.232864	-107.422\\
641.201918	-123.291\\
765.267237	-205.078\\
1314.344902	-252.686\\
1633.362304	-252.686\\
1605.566844	-184.326\\
1164.387342	-189.209\\
1191.827491	-167.236\\
1050.409316	-163.574\\
1024.300388	-162.354\\
989.872338	-108.643\\
648.490067	-96.436\\
570.322504	-86.67\\
507.79953	-78.125\\
457.734375	-87.891\\
526.291308	-133.057\\
830.807908	-203.857\\
1269.213682	-161.133\\
973.565586	-109.863\\
653.794713	-92.773\\
546.989608	-89.111\\
507.487145	-52.49\\
288.32757	-41.504\\
225.698752	-47.607\\
268.50348	-90.332\\
509.47248	-67.139\\
378.66396	-78.125\\
442.03125	-85.449\\
481.93236	-80.566\\
446.980168	-54.932\\
295.698956	-37.842\\
197.497398	-30.518\\
162.05058	-54.932\\
311.794032	-119.629\\
716.338452	-172.119\\
1055.777946	-194.092\\
1172.703864	-136.719\\
798.575679	-91.553\\
519.654828	-64.697\\
353.051529	-43.945\\
247.8498	-98.877\\
564.884301	-91.553\\
534.761073	-137.939\\
833.427438	-168.457\\
1073.408004	-275.879\\
1828.526012	-316.162\\
2084.139904	-260.01\\
1704.36555	-249.023\\
1586.774556	-159.912\\
1010.164104	-185.547\\
1185.64533	-195.313\\
1251.761017	-212.402\\
1341.743434	-158.691\\
988.010166	-144.043\\
896.811718	-142.822\\
881.354562	-129.395\\
803.154765	-156.25\\
955.625	-109.863\\
690.049503	-195.313\\
1255.276651	-234.375\\
1489.21875	-175.781\\
1075.076596	-104.98\\
634.28916	-111.084\\
695.608008	-192.871\\
1239.581917	-234.375\\
1536.328125	-289.307\\
1923.023629	-303.955\\
2031.331265	-302.734\\
1989.870582	-228.271\\
1483.7615	-205.078\\
1340.594886	-235.596\\
1561.530288	-275.879\\
1783.281856	-170.898\\
1054.611558	-104.98\\
653.60548	-152.588\\
941.620548	-122.07\\
730.95516	-79.346\\
475.123848	-109.863\\
665.879643	-122.07\\
730.95516	-95.215\\
559.67377	-75.684\\
447.595176	-101.318\\
595.547204	-80.566\\
479.448266	-115.967\\
713.428984	-162.354\\
995.879436	-147.705\\
887.11623	-100.098\\
597.484962	-101.318\\
619.660888	-157.471\\
1012.066117	-261.23\\
1669.2597	-185.547\\
1141.485144	-118.408\\
706.777352	-85.449\\
508.506999	-101.318\\
602.943418	-90.332\\
526.003236	-67.139\\
388.466254	-76.904\\
450.580536	-92.773\\
552.092123	-120.85\\
728.0004	-134.277\\
791.697192	-89.111\\
540.101771	-156.25\\
969.84375	-179.443\\
1110.572727	-170.898\\
1051.364496	-140.381\\
861.097054	-150.146\\
943.067026	-202.637\\
1242.975358	-125.732\\
722.959	-53.711\\
306.850943	-78.125\\
450.625	-85.449\\
503.807304	-119.629\\
705.332584	-101.318\\
586.225948	-74.463\\
439.033848	-118.408\\
702.514664	-112.305\\
653.952015	-79.346\\
463.459986	-102.539\\
597.084597	-90.332\\
522.660952	-80.566\\
469.135818	-95.215\\
538.72647	-54.932\\
309.81648	-68.359\\
379.255732	-47.607\\
262.362177	-51.27\\
294.8025	-106.201\\
622.231659	-122.07\\
737.54694	-167.236\\
1044.221584	-219.727\\
1351.760504	-142.822\\
842.078512	-83.008\\
472.73056	-63.477\\
359.152866	-62.256\\
357.972	-91.553\\
514.619413	-54.932\\
306.79522	-70.801\\
401.866476	-85.449\\
506.968917	-150.146\\
899.074248	-133.057\\
816.171638	-190.43\\
1147.15032	-124.512\\
740.970912	-117.188\\
667.38566	-56.152\\
312.542032	-62.256\\
336.306912	-34.18\\
182.76046	-46.387\\
248.866255	-51.27\\
286.34295	-95.215\\
542.249425	-100.098\\
577.365264	-117.188\\
684.495108	-123.291\\
749.485989	-197.754\\
1220.339934	-170.898\\
1032.565716	-128.174\\
776.862614	-153.809\\
940.695844	-163.574\\
1021.356056	-213.623\\
1322.112747	-161.133\\
967.764798	-104.98\\
597.8611	-53.711\\
292.080418	-40.283\\
215.393201	-41.504\\
223.416032	-52.49\\
284.4958	-54.932\\
295.698956	-51.27\\
279.78039	-70.801\\
401.866476	-108.643\\
622.633033	-98.877\\
564.884301	-103.76\\
600.35536	-118.408\\
667.82112	-69.58\\
373.2967	-36.621\\
202.51413	-81.787\\
473.219582	-125.732\\
732.137436	-125.732\\
739.052696	-142.822\\
839.507716	-118.408\\
680.846	-87.891\\
511.789293	-123.291\\
742.704984	-172.119\\
1046.311401	-172.119\\
1027.378311	-120.85\\
743.4692	-207.52\\
1272.92768	-155.029\\
942.421291	-151.367\\
931.209784	-173.34\\
1066.38768	-172.119\\
1043.213259	-130.615\\
782.12262	-112.305\\
690.90036	-192.871\\
1236.110239	-266.113\\
1695.672036	-202.637\\
1242.975358	-124.512\\
752.301504	-118.408\\
713.289792	-115.967\\
707.050799	-150.146\\
926.550966	-173.34\\
1056.85398	-125.732\\
764.324828	-134.277\\
831.040353	-177.002\\
1105.200488	-192.871\\
1175.934487	-120.85\\
721.35365	-97.656\\
590.037552	-130.615\\
817.91113	-218.506\\
1380.302402	-195.313\\
1233.792221	-203.857\\
1246.789412	-115.967\\
694.410396	-108.643\\
650.554284	-101.318\\
586.225948	-63.477\\
351.02781	-41.504\\
221.921888	-34.18\\
178.38542	-29.297\\
159.873729	-78.125\\
444.921875	-108.643\\
632.628189	-131.836\\
760.430048	-93.994\\
542.157392	-107.422\\
625.518306	-119.629\\
679.014204	-72.021\\
404.830041	-79.346\\
446.003866	-73.242\\
403.636662	-54.932\\
305.751512	-75.684\\
435.183	-120.85\\
717.00305	-156.25\\
906.875	-95.215\\
540.44034	-73.242\\
407.664972	-58.594\\
327.24749	-74.463\\
407.684925	-45.166\\
258.846346	-117.188\\
712.385852	-195.313\\
1190.823361	-163.574\\
1015.303818	-216.064\\
1372.870656	-241.699\\
1549.048891	-249.023\\
1573.078291	-181.885\\
1118.95652	-131.836\\
801.431044	-130.615\\
794.008585	-129.395\\
793.70893	-155.029\\
962.265003	-184.326\\
1147.613676	-181.885\\
1158.97122	-247.803\\
1610.7195	-270.996\\
1801.310412	-328.369\\
2140.637511	-217.285\\
1412.3525	-238.037\\
1569.139904	-264.893\\
1717.036426	-205.078\\
1336.903482	-239.258\\
1546.563712	-200.195\\
1242.610365	-113.525\\
671.38685	-74.463\\
433.598049	-79.346\\
473.616274	-117.188\\
697.385788	-97.656\\
565.037616	-64.697\\
375.501388	-90.332\\
517.692692	-63.477\\
353.312982	-48.828\\
273.583284	-64.697\\
363.661837	-72.021\\
399.572508	-48.828\\
278.07546	-100.098\\
591.979572	-135.498\\
791.443818	-97.656\\
588.279744	-173.34\\
1098.1089	-241.699\\
1544.45661	-220.947\\
1440.353493	-279.541\\
1791.578269	-187.988\\
1197.859536	-206.299\\
1288.130956	-134.277\\
799.082427	-85.449\\
508.506999	-108.643\\
642.514702	-87.891\\
505.37325	-63.477\\
368.420508	-92.773\\
523.23972	-50.049\\
270.364698	-32.959\\
176.231773	-43.945\\
247.8498	-98.877\\
577.540557	-125.732\\
752.883216	-157.471\\
951.439782	-152.588\\
921.936696	-146.484\\
890.476236	-168.457\\
1017.817194	-135.498\\
803.909634	-108.643\\
642.514702	-112.305\\
657.994995	-89.111\\
533.596668	-146.484\\
863.669664	-96.436\\
556.242848	-80.566\\
473.566948	-120.85\\
732.47185	-164.795\\
986.79246	-123.291\\
724.704498	-93.994\\
545.541176	-80.566\\
467.605064	-91.553\\
523.042289	-61.035\\
341.979105	-61.035\\
347.594325	-87.891\\
510.119364	-109.863\\
643.687317	-124.512\\
727.274592	-96.436\\
566.850808	-128.174\\
755.713904	-114.746\\
649.232868	-61.035\\
341.979105	-69.58\\
408.99124	-135.498\\
828.705768	-190.43\\
1171.52536	-186.768\\
1142.273088	-157.471\\
960.100687	-152.588\\
913.696944	-114.746\\
680.788018	-109.863\\
643.687317	-87.891\\
521.457303	-125.732\\
748.231132	-117.188\\
675.940384	-70.801\\
414.823059	-107.422\\
635.293708	-126.953\\
762.479718	-150.146\\
912.737534	-164.795\\
1004.755115	-163.574\\
1021.356056	-222.168\\
1427.873736	-267.334\\
1762.265728	-302.734\\
1945.671418	-190.43\\
1164.66988	-106.201\\
618.408423	-68.359\\
380.486194	-45.166\\
253.878086	-70.801\\
396.698003	-64.697\\
375.501388	-119.629\\
698.752989	-102.539\\
595.136356	-93.994\\
554.188624	-124.512\\
745.577856	-145.264\\
885.674608	-172.119\\
1062.146349	-183.105\\
1170.04095	-258.789\\
1667.895105	-231.934\\
1465.127078	-177.002\\
1085.730268	-120.85\\
732.47185	-120.85\\
732.47185	-123.291\\
758.486232	-157.471\\
954.431731	-108.643\\
650.554284	-113.525\\
679.7877	-107.422\\
613.701886	-58.594\\
340.079576	-102.539\\
617.694936	-152.588\\
910.797772	-107.422\\
637.334726	-115.967\\
722.010542	-216.064\\
1341.109248	-159.912\\
983.778624	-156.25\\
984.21875	-219.727\\
1396.145358	-206.299\\
1269.151448	-129.395\\
767.700535	-85.449\\
500.645691	-85.449\\
516.282858	-146.484\\
927.97614	-236.816\\
1526.27912	-246.582\\
1598.344524	-251.465\\
1606.86135	-192.871\\
1186.542392	-122.07\\
737.54694	-103.76\\
609.90128	-72.021\\
416.713506	-69.58\\
396.2581	-58.594\\
339.024884	-89.111\\
522.101349	-102.539\\
583.959605	-58.594\\
336.9155	-90.332\\
534.223448	-126.953\\
762.479718	-145.264\\
893.664128	-185.547\\
1172.100399	-233.154\\
1519.930926	-281.982\\
1807.222638	-197.754\\
1245.652446	-172.119\\
1087.275723	-178.223\\
1119.418663	-156.25\\
952.65625	-103.76\\
630.75704	-128.174\\
809.675158	-214.844\\
1384.66958	-246.582\\
1539.658008	-141.602\\
842.673502	-79.346\\
453.303698	-52.49\\
280.66403	-28.076\\
147.034012	-35.4\\
191.2308	-64.697\\
355.380621	-74.463\\
421.311654	-102.539\\
580.165662	-81.787\\
452.28211	-64.697\\
356.545167	-62.256\\
340.8516	-61.035\\
334.166625	-54.932\\
301.741476	-65.918\\
371.77752	-102.539\\
602.724242	-148.926\\
891.768888	-157.471\\
948.605304	-155.029\\
945.211813	-183.105\\
1146.60351	-225.83\\
1434.92382	-225.83\\
1459.76512	-270.996\\
1756.596072	-247.803\\
1579.000716	-184.326\\
1140.793614	-125.732\\
766.588004	-122.07\\
768.91893	-205.078\\
1318.036306	-236.816\\
1491.703984	-166.016\\
1012.199552	-107.422\\
617.6765	-53.711\\
293.100927	-40.283\\
217.608766	-42.725\\
222.17	-25.635\\
137.070345	-61.035\\
338.62218	-84.229\\
471.935087	-89.111\\
497.684935	-78.125\\
422.03125	-47.607\\
250.174785	-35.4\\
187.3368	-52.49\\
280.66403	-56.152\\
302.266216	-62.256\\
330.57936	-47.607\\
248.460933	-37.842\\
203.703486	-69.58\\
407.66922	-164.795\\
1013.81884	-190.43\\
1185.61718	-217.285\\
1332.82619	-147.705\\
922.27002	-208.74\\
1353.05268	-291.748\\
1907.156676	-261.23\\
1712.36265	-273.438\\
1762.30791	-200.195\\
1286.653265	-202.637\\
1302.347999	-212.402\\
1334.096962	-139.16\\
858.75636	-133.057\\
791.822207	-80.566\\
466.154876	-75.684\\
454.558104	-135.498\\
803.909634	-93.994\\
554.188624	-104.98\\
624.73598	-115.967\\
690.119617	-111.084\\
659.061372	-113.525\\
677.630725	-120.85\\
728.0004	-136.719\\
828.653859	-147.705\\
889.77492	-125.732\\
743.579048	-92.773\\
553.762037	-120.85\\
692.59135	-57.373\\
305.683344	-26.855\\
141.606415	-46.387\\
253.968825	-73.242\\
388.91502	-37.842\\
189.172158	-17.09\\
86.37286	-39.063\\
207.42453	-70.801\\
385.015838	-84.229\\
470.418965	-104.98\\
586.3133	-87.891\\
510.119364	-142.822\\
836.794098	-123.291\\
699.799716	-83.008\\
481.778432	-125.732\\
704.476396	-70.801\\
386.361057	-56.152\\
297.156384	-39.063\\
198.83067	-24.414\\
126.513348	-48.828\\
249.462252	-36.621\\
193.102533	-65.918\\
366.899588	-111.084\\
626.51376	-106.201\\
589.203148	-76.904\\
426.663392	-86.67\\
474.51825	-64.697\\
360.103502	-92.773\\
507.932175	-67.139\\
362.684878	-63.477\\
341.696691	-58.594\\
311.13414	-43.945\\
233.34795	-57.373\\
303.617916	-51.27\\
281.62611	-90.332\\
499.53596	-83.008\\
452.974656	-67.139\\
362.684878	-64.697\\
344.705616	-51.27\\
274.14069	-61.035\\
347.594325	-135.498\\
811.362024	-179.443\\
1057.995928	-123.291\\
717.923493	-114.746\\
651.298296	-80.566\\
472.036194	-136.719\\
806.095224	-125.732\\
718.306916	-80.566\\
461.723746	-102.539\\
580.165662	-76.904\\
440.736824	-108.643\\
606.771155	-64.697\\
354.216075	-64.697\\
360.103502	-80.566\\
455.842428	-104.98\\
588.20294	-74.463\\
414.461058	-81.787\\
458.252561	-85.449\\
494.407914	-135.498\\
786.430392	-115.967\\
694.410396	-178.223\\
1112.824412	-231.934\\
1444.021084	-183.105\\
1099.72863	-117.188\\
680.159152	-86.67\\
514.21311	-135.498\\
823.692342	-173.34\\
1041.08004	-133.057\\
806.458477	-163.574\\
964.432304	-98.877\\
550.349382	-50.049\\
273.117393	-52.49\\
301.8175	-119.629\\
696.599667	-107.422\\
619.610096	-100.098\\
593.881434	-156.25\\
935.625	-146.484\\
869.089572	-133.057\\
772.262828	-90.332\\
524.286928	-114.746\\
682.853446	-148.926\\
878.067696	-120.85\\
714.7069	-125.732\\
739.052696	-112.305\\
647.77524	-80.566\\
466.154876	-97.656\\
554.295456	-70.801\\
400.592058	-79.346\\
463.459986	-135.498\\
811.362024	-156.25\\
944.0625	-164.795\\
989.75877	-146.484\\
861.032952	-104.98\\
599.75074	-72.021\\
410.159595	-85.449\\
492.869832	-101.318\\
593.622162	-129.395\\
774.81726	-159.912\\
978.021792	-190.43\\
1154.19623	-146.484\\
855.613044	-84.229\\
504.363252	-155.029\\
968.001076	-222.168\\
1366.777536	-155.029\\
945.211813	-155.029\\
956.683959	-181.885\\
1108.952845	-137.939\\
823.357891	-113.525\\
681.83115	-129.395\\
772.358755	-115.967\\
694.410396	-131.836\\
803.804092	-167.236\\
1007.429664	-125.732\\
759.672744	-146.484\\
885.056328	-137.939\\
833.427438	-141.602\\
845.222338	-109.863\\
661.814712	-144.043\\
857.199893	-102.539\\
604.569944	-106.201\\
632.002151	-119.629\\
709.758857	-115.967\\
660.432065	-58.594\\
315.411502	-35.4\\
182.8056	-28.076\\
147.034012	-47.607\\
270.217332	-122.07\\
726.43857	-161.133\\
967.764798	-161.133\\
941.177853	-101.318\\
591.798438	-119.629\\
698.752989	-104.98\\
607.41428	-92.773\\
519.807119	-58.594\\
330.47016	-81.787\\
462.750846	-81.787\\
461.27868	-80.566\\
458.82337	-95.215\\
538.72647	-79.346\\
446.003866	-75.684\\
417.094524	-50.049\\
276.77097	-72.021\\
418.009884	-139.16\\
802.67488	-96.436\\
559.714544	-119.629\\
692.173394	-107.422\\
635.293708	-145.264\\
861.851312	-135.498\\
821.253378	-186.768\\
1125.090432	-137.939\\
828.461634	-151.367\\
914.559414	-155.029\\
894.207272	-72.021\\
424.635816	-131.836\\
767.681028	-95.215\\
556.150815	-113.525\\
669.3434	-128.174\\
769.813044	-166.016\\
987.961216	-123.291\\
742.704984	-158.691\\
976.267032	-201.416\\
1235.485744	-164.795\\
995.69139	-142.822\\
852.504518	-113.525\\
679.7877	-136.719\\
811.153827	-113.525\\
667.29995	-96.436\\
557.978696	-81.787\\
476.245701	-96.436\\
557.978696	-83.008\\
500.040192	-172.119\\
1074.711036	-225.83\\
1410.08252	-191.65\\
1189.57155	-187.988\\
1170.413288	-181.885\\
1089.12738	-101.318\\
593.622162	-89.111\\
512.38825	-72.021\\
406.19844	-62.256\\
352.244448	-68.359\\
399.284919	-115.967\\
692.207023	-146.484\\
887.839524	-167.236\\
1007.429664	-141.602\\
832.336556	-93.994\\
569.697634	-169.678\\
1053.191346	-201.416\\
1268.719384	-224.609\\
1398.415634	-178.223\\
1152.033472	-288.086\\
1856.71427	-209.961\\
1287.900774	-117.188\\
693.049832	-84.229\\
485.832872	-69.58\\
412.81814	-125.732\\
766.588004	-162.354\\
1019.745474	-218.506\\
1364.351464	-166.016\\
1012.199552	-126.953\\
739.247319	-69.58\\
397.51054	-72.021\\
412.752351	-78.125\\
450.625	-87.891\\
494.035311	-56.152\\
316.69728	-70.801\\
395.423585	-59.814\\
324.19188	-36.621\\
203.832486	-83.008\\
472.73056	-93.994\\
543.849284	-118.408\\
687.240032	-107.422\\
623.477288	-112.305\\
655.973505	-113.525\\
671.38685	-146.484\\
855.613044	-100.098\\
593.881434	-142.822\\
865.644142	-169.678\\
1016.031864	-130.615\\
782.12262	-139.16\\
828.14116	-119.629\\
714.065501	-135.498\\
831.144732	-191.65\\
1165.04035	-148.926\\
888.939294	-114.746\\
687.099048	-128.174\\
765.070606	-115.967\\
673.072468	-79.346\\
467.824016	-122.07\\
746.58012	-181.885\\
1115.68259	-170.898\\
1051.364496	-187.988\\
1208.198876	-279.541\\
1770.892235	-187.988\\
1170.413288	-159.912\\
974.983464	-120.85\\
741.2939	-152.588\\
963.898396	-224.609\\
1394.148063	-148.926\\
908.001822	-129.395\\
805.61327	-186.768\\
1138.724496	-119.629\\
698.752989	-68.359\\
396.755636	-87.891\\
505.37325	-68.359\\
383.015477	-52.49\\
293.15665	-57.373\\
318.305404	-50.049\\
274.018275	-48.828\\
270.897744	-64.697\\
367.220172	-95.215\\
549.20012	-107.422\\
637.334726	-147.705\\
881.651145	-139.16\\
843.44876	-166.016\\
1024.484736	-194.092\\
1201.235388	-174.561\\
1054.697562	-125.732\\
757.409568	-134.277\\
806.467662	-119.629\\
718.491774	-130.615\\
803.54348	-183.105\\
1113.095295	-140.381\\
853.376099	-158.691\\
964.682589	-136.719\\
823.595256	-129.395\\
800.825655	-201.416\\
1246.563624	-166.016\\
1024.484736	-167.236\\
1025.825624	-152.588\\
908.051188	-91.553\\
523.042289	-58.594\\
325.079512	-46.387\\
262.457646	-84.229\\
481.200277	-79.346\\
456.2395	-106.201\\
608.637931	-79.346\\
459.095956	-108.643\\
664.460588	-191.65\\
1210.65305	-235.596\\
1479.778476	-185.547\\
1158.555468	-190.43\\
1189.04492	-169.678\\
1031.472562	-118.408\\
717.670888	-133.057\\
811.248529	-145.264\\
880.445104	-125.732\\
764.324828	-144.043\\
891.482127	-181.885\\
1165.700965	-260.01\\
1675.76445	-230.713\\
1474.25607	-214.844\\
1384.66958	-239.258\\
1507.086142	-162.354\\
1019.745474	-173.34\\
1075.92138	-139.16\\
858.75636	-141.602\\
889.402162	-186.768\\
1186.723872	-213.623\\
1286.864952	-84.229\\
498.130306	-100.098\\
588.376044	-85.449\\
510.045081	-119.629\\
718.491774	-123.291\\
720.142731	-75.684\\
444.870552	-98.877\\
604.731732	-173.34\\
1130.00346	-303.955\\
2014.61374	-289.307\\
1901.614911	-261.23\\
1697.995	-211.182\\
1380.496734	-240.479\\
1593.894812	-290.527\\
1899.174999	-217.285\\
1412.3525	-218.506\\
1424.440614	-225.83\\
1438.98876	-152.588\\
952.759472	-131.836\\
832.808012	-169.678\\
1050.137142	-115.967\\
694.410396	-81.787\\
495.711007	-114.746\\
682.853446	-84.229\\
501.246779	-98.877\\
588.417027	-95.215\\
571.86129	-115.967\\
711.341578	-158.691\\
967.539027	-126.953\\
757.782457	-90.332\\
542.533992	-112.305\\
678.54681	-129.395\\
791.37982	-153.809\\
937.773473	-129.395\\
777.14637	-106.201\\
653.348552	-167.236\\
1035.023604	-159.912\\
969.226632	-111.084\\
689.498388	-183.105\\
1143.30762	-168.457\\
1030.283012	-119.629\\
711.912179	-81.787\\
471.747416	-58.594\\
327.24749	-45.166\\
260.517488	-98.877\\
577.540557	-96.436\\
552.674716	-65.918\\
366.899588	-47.607\\
264.123636	-57.373\\
329.89475	-102.539\\
600.776001	-119.629\\
720.645096	-158.691\\
973.410594	-180.664\\
1111.444928	-172.119\\
1065.244491	-181.885\\
1098.94917	-112.305\\
633.4002	-46.387\\
259.906361	-70.801\\
396.698003	-62.256\\
351.12384	-78.125\\
457.734375	-126.953\\
757.782457	-139.16\\
858.75636	-201.416\\
1228.033352	-133.057\\
796.745316	-124.512\\
770.604768	-191.65\\
1172.1314	-142.822\\
849.933722	-95.215\\
550.91399	-69.58\\
405.16434	-93.994\\
557.666402	-129.395\\
798.496545	-190.43\\
1157.62397	-130.615\\
777.289865	-106.201\\
630.090533	-98.877\\
590.196813	-122.07\\
744.26079	-163.574\\
1042.293528	-255.127\\
1635.108943	-219.727\\
1404.05553	-214.844\\
1333.536708	-137.939\\
823.357891	-90.332\\
539.191708	-107.422\\
607.793676	-50.049\\
283.177242	-74.463\\
421.311654	-69.58\\
392.4312	-79.346\\
469.252244	-134.277\\
821.238132	-185.547\\
1165.420707	-217.285\\
1404.53024	-278.32\\
1804.07024	-239.258\\
1485.074406	-129.395\\
786.592205	-118.408\\
711.158448	-98.877\\
601.073283	-136.719\\
848.614833	-179.443\\
1146.64077	-239.258\\
1515.69943	-179.443\\
1103.933336	-122.07\\
750.97464	-139.16\\
868.91504	-184.326\\
1133.973552	-133.057\\
799.140342	-89.111\\
522.101349	-73.242\\
414.403236	-50.049\\
276.77097	-48.828\\
263.768856	-36.621\\
201.818331	-64.697\\
368.449415	-98.877\\
570.322536	-107.422\\
613.701886	-80.566\\
455.842428	-73.242\\
394.261686	-34.18\\
172.74572	-17.09\\
85.43291	-30.518\\
159.273442	-57.373\\
314.117175	-87.891\\
497.287278	-111.084\\
642.732024	-130.615\\
772.45711	-150.146\\
915.440162	-190.43\\
1216.8477	-266.113\\
1724.944466	-261.23\\
1707.66051	-280.762\\
1830.287478	-249.023\\
1559.382026	-136.719\\
838.634346	-123.291\\
754.047756	-125.732\\
780.418524	-166.016\\
1030.461312	-145.264\\
877.685088	-102.539\\
602.724242	-79.346\\
457.667728	-63.477\\
373.117806	-111.084\\
656.950776	-108.643\\
620.677459	-56.152\\
309.453672	-41.504\\
231.79984	-69.58\\
398.76298	-96.436\\
547.370736	-76.904\\
443.582272	-106.201\\
610.65575	-86.67\\
506.23947	-122.07\\
744.26079	-181.885\\
1108.952845	-146.484\\
903.952764	-190.43\\
1220.46587	-260.01\\
1699.68537	-281.982\\
1822.731648	-223.389\\
1403.106309	-145.264\\
883.059856	-107.422\\
625.518306	-64.697\\
379.059723	-101.318\\
586.225948	-75.684\\
450.395484	-126.953\\
760.194564	-115.967\\
681.654026	-92.773\\
550.422209	-119.629\\
692.173394	-69.58\\
405.16434	-98.877\\
570.322536	-75.684\\
419.894832	-46.387\\
255.638757	-54.932\\
298.720216	-42.725\\
231.5695	-47.607\\
256.268481	-46.387\\
243.763685	-30.518\\
160.37209	-42.725\\
229.219625	-62.256\\
336.306912	-61.035\\
329.71107	-61.035\\
340.880475	-97.656\\
565.037616	-130.615\\
753.38732	-108.643\\
622.633033	-98.877\\
586.637241	-156.25\\
955.625	-190.43\\
1157.62397	-150.146\\
910.034906	-156.25\\
901.25	-74.463\\
407.684925	-43.945\\
247.014845	-90.332\\
530.971496	-133.057\\
794.217233	-145.264\\
893.664128	-202.637\\
1257.767859	-183.105\\
1109.799405	-130.615\\
767.75497	-89.111\\
518.893353	-93.994\\
550.710846	-102.539\\
593.290654	-81.787\\
456.780395	-48.828\\
270.897744	-64.697\\
356.545167	-52.49\\
283.55098	-41.504\\
218.850592	-32.959\\
177.418297	-61.035\\
340.880475	-93.994\\
531.818052	-100.098\\
559.04733	-69.58\\
398.76298	-123.291\\
740.485746	-172.119\\
1011.715482	-109.863\\
653.794713	-146.484\\
885.056328	-163.574\\
973.428874	-118.408\\
676.464904	-69.58\\
396.2581	-86.67\\
501.47262	-108.643\\
608.726729	-59.814\\
334.06119	-69.58\\
384.7774	-53.711\\
286.172208	-31.738\\
165.0376	-31.738\\
169.100064	-56.152\\
306.421464	-76.904\\
421.0494	-67.139\\
371.27867	-83.008\\
466.587968	-100.098\\
555.343704	-73.242\\
388.91502	-37.842\\
196.097244	-32.959\\
171.3868	-41.504\\
218.10352	-50.049\\
266.661072	-61.035\\
346.43466	-128.174\\
748.664334	-130.615\\
753.38732	-109.863\\
633.689784	-107.422\\
629.385498	-141.602\\
858.249722	-185.547\\
1145.010537	-201.416\\
1254.016016	-205.078\\
1254.257048	-151.367\\
922.884599	-157.471\\
963.092636	-162.354\\
992.957064	-166.016\\
1027.473024	-185.547\\
1175.440245	-246.582\\
1571.220504	-216.064\\
1364.876288	-195.313\\
1216.018738	-157.471\\
971.753541	-147.705\\
908.68116	-142.822\\
883.925358	-167.236\\
1010.439912	-103.76\\
621.31488	-120.85\\
734.64715	-148.926\\
919.022346	-179.443\\
1097.473388	-137.939\\
846.117826	-155.029\\
942.421291	-129.395\\
777.14637	-112.305\\
653.952015	-70.801\\
417.442696	-111.084\\
679.389744	-177.002\\
1082.544232	-141.602\\
832.336556	-81.787\\
477.717867	-97.656\\
557.908728	-53.711\\
297.988628	-50.049\\
280.424547	-70.801\\
388.909893	-45.166\\
243.986732	-46.387\\
252.252506	-54.932\\
292.677696	-36.621\\
197.826642	-62.256\\
336.306912	-51.27\\
268.50099	-30.518\\
162.599904	-57.373\\
308.838859	-64.697\\
355.380621	-80.566\\
443.999226	-75.684\\
415.732212	-76.904\\
432.277384	-104.98\\
593.97684	-92.773\\
526.579548	-104.98\\
628.62024	-180.664\\
1078.383416	-128.174\\
750.971466	-112.305\\
662.15028	-122.07\\
706.29702	-86.67\\
482.40522	-51.27\\
288.18867	-92.773\\
523.23972	-73.242\\
398.289996	-43.945\\
243.80686	-81.787\\
458.252561	-80.566\\
457.292616	-98.877\\
548.569596	-61.035\\
326.354145	-36.621\\
190.4292	-29.297\\
151.817054	-36.621\\
196.471665	-69.58\\
378.37604	-68.359\\
374.265525	-80.566\\
454.39224	-109.863\\
643.687317	-150.146\\
871.447384	-109.863\\
649.729782	-152.588\\
924.835868	-177.002\\
1066.260048	-153.809\\
923.776854	-145.264\\
901.653648	-230.713\\
1432.035591	-167.236\\
1044.221584	-208.74\\
1303.37256	-178.223\\
1116.032426	-203.857\\
1298.976804	-234.375\\
1441.875	-136.719\\
806.095224	-81.787\\
468.721297	-65.918\\
383.840514	-108.643\\
648.490067	-134.277\\
804.050676	-137.939\\
808.184601	-90.332\\
506.130196	-50.049\\
278.572734	-67.139\\
392.158899	-128.174\\
781.476878	-179.443\\
1100.703362	-177.002\\
1053.338902	-103.76\\
602.22304	-81.787\\
480.743986	-109.863\\
669.834711	-175.781\\
1094.412506	-200.195\\
1253.62109	-203.857\\
1258.001547	-150.146\\
934.808996	-208.74\\
1326.33396	-227.051\\
1442.682054	-212.402\\
1380.613	-290.527\\
1899.174999	-251.465\\
1620.691925	-209.961\\
1360.967202	-241.699\\
1562.342336	-207.52\\
1272.92768	-106.201\\
635.931588	-100.098\\
604.792116	-126.953\\
781.014856	-155.029\\
959.474481	-158.691\\
967.539027	-115.967\\
709.254172	-148.926\\
910.831416	-130.615\\
782.12262	-96.436\\
582.666312	-129.395\\
784.263095	-122.07\\
753.29397	-175.781\\
1097.576564	-181.885\\
1115.68259	-134.277\\
806.467662	-97.656\\
561.522	-58.594\\
340.079576	-96.436\\
554.507	-74.463\\
422.651988	-65.918\\
369.338554	-53.711\\
308.83825	-95.215\\
559.67377	-123.291\\
735.923979	-137.939\\
828.461634	-136.719\\
796.114737	-80.566\\
455.842428	-62.256\\
344.27568	-42.725\\
234.688425	-54.932\\
310.805256	-91.553\\
523.042289	-95.215\\
543.963295	-91.553\\
543.183949	-150.146\\
918.292936	-174.561\\
1083.500127	-198.975\\
1242.3999	-198.975\\
1260.506625	-229.492\\
1416.195132	-137.939\\
830.944536	-113.525\\
671.38685	-87.891\\
514.953369	-93.994\\
561.050186	-128.174\\
762.763474	-106.201\\
614.478986	-69.58\\
396.2581	-64.697\\
381.453512	-124.512\\
752.301504	-153.809\\
923.776854	-134.277\\
833.457339	-211.182\\
1341.850428	-230.713\\
1440.571972	-159.912\\
966.188304	-111.084\\
650.841156	-76.904\\
443.582272	-70.801\\
421.336751	-131.836\\
784.556036	-106.201\\
632.002151	-122.07\\
735.34968	-139.16\\
843.44876	-156.25\\
949.84375	-151.367\\
928.485178	-168.457\\
1021.017877	-115.967\\
698.585208	-130.615\\
774.938795	-91.553\\
533.113119	-84.229\\
502.762901	-129.395\\
796.03804	-183.105\\
1143.30762	-201.416\\
1213.329984	-106.201\\
663.119044	-220.947\\
1411.85133	-207.52\\
1284.34128	-139.16\\
822.99224	-76.904\\
459.039976	-123.291\\
738.266508	-107.422\\
637.334726	-104.98\\
630.50988	-123.291\\
724.704498	-80.566\\
476.467324	-108.643\\
674.347101	-207.52\\
1318.58208	-219.727\\
1359.890403	-145.264\\
899.038896	-159.912\\
995.612112	-177.002\\
1105.200488	-180.664\\
1111.444928	-131.836\\
803.804092	-128.174\\
762.763474	-84.229\\
505.879374	-133.057\\
808.853503	-134.277\\
848.227809	-229.492\\
1479.07594	-251.465\\
1671.487855	-332.031\\
2188.748352	-238.037\\
1534.148465	-194.092\\
1226.079164	-147.705\\
927.735105	-162.354\\
1007.731278	-122.07\\
733.15242	-84.229\\
496.614184	-81.787\\
482.216152	-85.449\\
494.407914	-62.256\\
349.940976	-46.387\\
263.292612	-74.463\\
432.183252	-100.098\\
571.859874	-68.359\\
380.486194	-46.387\\
254.803791	-46.387\\
267.560216	-112.305\\
676.52532	-157.471\\
913.961684	-73.242\\
422.459856	-95.215\\
556.150815	-97.656\\
566.795424	-91.553\\
536.409027	-113.525\\
663.099525	-98.877\\
564.884301	-68.359\\
380.486194	-48.828\\
266.454396	-40.283\\
220.549425	-53.711\\
306.850943	-106.201\\
622.231659	-126.953\\
736.835212	-96.436\\
543.89904	-63.477\\
345.187926	-39.063\\
212.424594	-56.152\\
318.718752	-96.436\\
561.546828	-123.291\\
724.704498	-133.057\\
769.867802	-90.332\\
517.692692	-86.67\\
498.3525	-95.215\\
559.67377	-133.057\\
808.853503	-185.547\\
1161.895314	-216.064\\
1341.109248	-164.795\\
986.79246	-102.539\\
606.415646	-107.422\\
633.360112	-100.098\\
591.979572	-103.76\\
598.48768	-65.918\\
380.215024	-87.891\\
508.537326	-93.994\\
545.541176	-91.553\\
518.006874	-56.152\\
305.354576	-36.621\\
199.840797	-56.152\\
315.630392	-84.229\\
491.981589	-134.277\\
801.499413	-148.926\\
905.321154	-179.443\\
1097.473388	-159.912\\
983.778624	-180.664\\
1144.50644	-239.258\\
1577.188736	-319.824\\
2108.279808	-253.906\\
1627.283554	-173.34\\
1107.6426	-189.209\\
1219.452005	-220.947\\
1456.482624	-288.086\\
1862.187904	-180.664\\
1095.004504	-89.111\\
515.596246	-59.814\\
339.504264	-62.256\\
345.396288	-42.725\\
227.6388	-28.076\\
148.044748	-39.063\\
211.018326	-59.814\\
331.848072	-83.008\\
465.093824	-86.67\\
480.84516	-65.918\\
362.087574	-62.256\\
345.396288	-78.125\\
442.03125	-100.098\\
571.859874	-104.98\\
599.75074	-96.436\\
540.330908	-68.359\\
373.035063	-47.607\\
256.268481	-45.166\\
248.096838	-72.021\\
403.533663	-87.891\\
487.619268	-64.697\\
348.263951	-41.504\\
228.728544	-76.904\\
418.203952	-48.828\\
266.454396	-69.58\\
384.7774	-79.346\\
453.303698	-120.85\\
703.70955	-126.953\\
722.997335	-84.229\\
487.348994	-125.732\\
736.663788	-123.291\\
715.580964	-101.318\\
578.829734	-84.229\\
495.098062	-142.822\\
865.644142	-175.781\\
1071.736757	-175.781\\
1084.744551	-194.092\\
1187.066672	-147.705\\
895.240005	-136.719\\
823.595256	-126.953\\
776.444548	-173.34\\
1050.61374	-135.498\\
833.583696	-184.326\\
1140.793614	-172.119\\
1065.244491	-178.223\\
1161.835737	-303.955\\
1992.425025	-231.934\\
1439.614338	-125.732\\
748.231132	-83.008\\
487.921024	-87.891\\
523.039341	-112.305\\
680.680605	-146.484\\
903.952764	-172.119\\
1074.711036	-189.209\\
1188.421729	-192.871\\
1183.070714	-125.732\\
759.672744	-111.084\\
675.279636	-130.615\\
791.657515	-129.395\\
762.91292	-76.904\\
437.96828	-57.373\\
334.082979	-103.76\\
609.90128	-101.318\\
608.515908	-142.822\\
878.640944	-174.561\\
1070.757174	-155.029\\
928.313652	-106.201\\
624.249478	-85.449\\
511.668612	-135.498\\
831.144732	-168.457\\
1064.142869	-230.713\\
1486.945285	-244.141\\
1600.344255	-286.865\\
1870.072935	-236.816\\
1530.778624	-216.064\\
1341.109248	-123.291\\
735.923979	-85.449\\
524.144166	-153.809\\
971.611453	-202.637\\
1242.975358	-115.967\\
724.097948	-186.768\\
1203.71976	-249.023\\
1650.524444	-308.838\\
2001.887916	-189.209\\
1171.014501	-112.305\\
657.994995	-63.477\\
373.117806	-95.215\\
559.67377	-85.449\\
502.269222	-93.994\\
535.29583	-51.27\\
291.98265	-76.904\\
432.277384	-52.49\\
296.0436	-74.463\\
418.556523	-58.594\\
332.579544	-79.346\\
466.395788	-124.512\\
736.363968	-112.305\\
676.52532	-162.354\\
1004.808906	-191.65\\
1193.2129	-190.43\\
1143.72258	-103.76\\
615.60808	-109.863\\
661.814712	-130.615\\
767.75497	-85.449\\
495.945996	-80.566\\
454.39224	-54.932\\
314.815292	-90.332\\
519.409	-79.346\\
443.14741	-47.607\\
254.554629	-26.855\\
140.156245	-31.738\\
170.27437	-58.594\\
326.134204	-85.449\\
481.93236	-91.553\\
504.548583	-56.152\\
309.453672	-67.139\\
379.872462	-102.539\\
598.930299	-135.498\\
781.552464	-93.994\\
535.29583	-87.891\\
505.37325	-104.98\\
613.18818	-125.732\\
736.663788	-115.967\\
683.741432	-137.939\\
815.771246	-126.953\\
734.550058	-89.111\\
505.794036	-72.021\\
412.752351	-95.215\\
542.249425	-78.125\\
447.734375	-97.656\\
574.021968	-129.395\\
791.37982	-203.857\\
1242.916129	-139.16\\
838.29984	-140.381\\
871.344867	-206.299\\
1273.071129	-152.588\\
908.051188	-96.436\\
557.978696	-69.58\\
391.10918	-54.932\\
306.79522	-58.594\\
324.02482	-46.387\\
253.968825	-50.049\\
285.929937	-106.201\\
583.362093	-45.166\\
241.502602	-50.049\\
274.018275	-65.918\\
370.525078	-93.994\\
526.648382	-69.58\\
379.69806	-53.711\\
283.218103	-28.076\\
149.588928	-54.932\\
307.783996	-101.318\\
591.798438	-139.16\\
807.68464	-108.643\\
620.677459	-89.111\\
512.38825	-101.318\\
584.402224	-102.539\\
597.084597	-119.629\\
707.485906	-144.043\\
865.122258	-168.457\\
1021.017877	-164.795\\
992.72508	-140.381\\
822.492279	-91.553\\
524.690243	-73.242\\
417.11319	-79.346\\
460.524184	-112.305\\
637.44318	-64.697\\
360.103502	-62.256\\
359.092608	-114.746\\
678.607844	-139.16\\
828.14116	-133.057\\
796.745316	-157.471\\
971.753541	-205.078\\
1220.419178	-104.98\\
642.05768	-184.326\\
1150.931544	-185.547\\
1148.350383	-170.898\\
1060.763886	-184.326\\
1147.613676	-181.885\\
1188.982245	-312.5\\
2082.8125	-286.865\\
1859.45893	-195.313\\
1269.5345	-236.816\\
1565.35376	-283.203\\
1877.069484	-275.879\\
1848.941058	-311.279\\
2086.191858	-281.982\\
1920.861384	-372.314\\
2529.129002	-280.762\\
1850.783104	-196.533\\
1241.498961	-118.408\\
734.958456	-124.512\\
759.149664	-92.773\\
552.092123	-79.346\\
480.916106	-123.291\\
769.829004	-172.119\\
1049.409543	-104.98\\
626.62562	-92.773\\
553.762037	-91.553\\
531.373612	-59.814\\
346.083804	-80.566\\
446.980168	-36.621\\
199.840797	-54.932\\
297.73144	-43.945\\
241.389885	-65.918\\
358.462084	-45.166\\
240.644448	-40.283\\
208.021412	-23.193\\
117.217422	-23.193\\
120.186126	-46.387\\
237.826149	-30.518\\
155.916462	-37.842\\
203.703486	-81.787\\
461.27868	-112.305\\
643.619955	-115.967\\
654.05388	-81.787\\
465.776965	-107.422\\
641.201918	-166.016\\
954.592	-80.566\\
441.09885	-48.828\\
260.155584	-37.842\\
194.69709	-28.076\\
147.034012	-48.828\\
266.454396	-81.787\\
467.249131	-123.291\\
690.799473	-72.021\\
415.417128	-129.395\\
779.47548	-173.34\\
1056.85398	-175.781\\
1055.740686	-130.615\\
751.03625	-75.684\\
435.183	-100.098\\
590.177808	-135.498\\
826.131306	-178.223\\
1050.802808	-93.994\\
524.95649	-53.711\\
302.93004	-84.229\\
471.935087	-63.477\\
338.205456	-31.738\\
160.403852	-21.973\\
114.677087	-54.932\\
311.794032	-126.953\\
741.532473	-126.953\\
727.567643	-92.773\\
513.03469	-57.373\\
314.117175	-61.035\\
309.56952	-18.311\\
91.866287	-35.4\\
178.9116	-30.518\\
158.144276	-58.594\\
316.524788	-84.229\\
479.684155	-131.836\\
767.681028	-135.498\\
796.457244	-148.926\\
886.258626	-153.809\\
918.085921	-148.926\\
886.258626	-145.264\\
859.091296	-125.732\\
752.883216	-155.029\\
970.791598	-236.816\\
1500.22936	-213.623\\
1290.710166	-111.084\\
630.512784	-58.594\\
346.524916	-135.498\\
828.705768	-158.691\\
961.826151	-142.822\\
831.652506	-78.125\\
446.328125	-83.008\\
495.474752	-142.822\\
891.780568	-217.285\\
1376.500475	-224.609\\
1443.562043	-250.244\\
1617.577216	-239.258\\
1515.69943	-178.223\\
1125.834691	-183.105\\
1096.43274	-84.229\\
487.348994	-79.346\\
464.888214	-101.318\\
602.943418	-126.953\\
762.479718	-134.277\\
811.301634	-142.822\\
842.078512	-86.67\\
515.77317	-124.512\\
738.729696	-102.539\\
585.805307	-59.814\\
326.404998	-39.063\\
206.721396	-32.959\\
177.418297	-53.711\\
285.20541	-40.283\\
206.530941	-23.193\\
121.044267	-48.828\\
271.776648	-100.098\\
551.640078	-64.697\\
350.65774	-65.918\\
364.52654	-86.67\\
504.67941	-146.484\\
847.556424	-97.656\\
538.182216	-52.49\\
274.89013	-29.297\\
159.873729	-81.787\\
482.216152	-159.912\\
983.778624	-201.416\\
1298.12612	-280.762\\
1876.332446	-333.252\\
2245.451976	-323.486\\
2126.273478	-207.52\\
1348.88	-212.402\\
1407.800456	-277.1\\
1831.631	-231.934\\
1516.152558	-214.844\\
1416.251648	-242.92\\
1614.68924	-263.672\\
1757.37388	-270.996\\
1846.024752	-357.666\\
2403.51552	-238.037\\
1521.05643	-135.498\\
813.800988	-70.801\\
407.10575	-59.814\\
342.794034	-67.139\\
383.565107	-67.139\\
383.565107	-69.58\\
412.81814	-126.953\\
750.800042	-93.994\\
554.188624	-107.422\\
645.176532	-136.719\\
793.517076	-70.801\\
409.654586	-85.449\\
505.345386	-117.188\\
703.831128	-140.381\\
819.965421	-76.904\\
432.277384	-51.27\\
289.1628	-68.359\\
385.54476	-64.697\\
390.899274	-167.236\\
1062.617544	-227.051\\
1450.85589	-214.844\\
1388.751616	-249.023\\
1641.559616	-274.658\\
1865.751794	-360.107\\
2453.048884	-306.396\\
2013.940908	-190.43\\
1202.94631	-125.732\\
748.231132	-69.58\\
405.16434	-78.125\\
457.734375	-81.787\\
485.242271	-109.863\\
641.709783	-72.021\\
412.752351	-68.359\\
394.294712	-75.684\\
440.707932	-95.215\\
561.38764	-107.422\\
647.110128	-135.498\\
803.909634	-95.215\\
537.0126	-45.166\\
243.986732	-34.18\\
179.00066	-28.076\\
145.9952	-32.959\\
172.013021	-39.063\\
211.72146	-72.021\\
395.611353	-73.242\\
411.693282	-95.215\\
556.150815	-140.381\\
817.438563	-100.098\\
601.188588	-168.457\\
1073.408004	-244.141\\
1551.271914	-190.43\\
1199.51857	-172.119\\
1093.644126	-207.52\\
1337.4664	-239.258\\
1502.779498	-148.926\\
921.703014	-130.615\\
779.640935	-80.566\\
476.467324	-92.773\\
577.604698	-187.988\\
1245.984464	-314.941\\
2162.384906	-371.094\\
2534.57202	-294.189\\
1976.95008	-247.803\\
1628.809119	-175.781\\
1149.080397	-191.65\\
1277.34725	-256.348\\
1657.033472	-147.705\\
927.735105	-128.174\\
790.961754	-96.436\\
612.754344	-185.547\\
1192.510569	-173.34\\
1066.38768	-92.773\\
558.864552	-90.332\\
532.597472	-65.918\\
397.090032	-119.629\\
751.389749	-180.664\\
1144.50644	-186.768\\
1173.089808	-141.602\\
863.347394	-104.98\\
642.05768	-115.967\\
698.585208	-89.111\\
522.101349	-64.697\\
369.613961	-52.49\\
294.10147	-45.166\\
246.470862	-36.621\\
199.144998	-46.387\\
259.906361	-76.904\\
449.196264	-113.525\\
669.3434	-113.525\\
650.611775	-72.021\\
406.19844	-62.256\\
347.69976	-51.27\\
286.34295	-68.359\\
393.06425	-95.215\\
550.91399	-103.76\\
602.22304	-100.098\\
560.849094	-54.932\\
321.846588	-123.291\\
756.266994	-172.119\\
1039.942998	-124.512\\
704.488896	-50.049\\
277.671852	-61.035\\
349.791585	-95.215\\
549.20012	-95.215\\
545.677165	-84.229\\
468.818614	-51.27\\
291.98265	-93.994\\
561.050186	-144.043\\
841.355163	-86.67\\
471.31146	-32.959\\
176.825035	-42.725\\
244.856975	-111.084\\
659.061372	-139.16\\
802.67488	-81.787\\
461.27868	-68.359\\
401.814202	-133.057\\
816.171638	-179.443\\
1100.703362	-157.471\\
989.075351	-222.168\\
1448.313192	-268.555\\
1711.23246	-173.34\\
1079.21484	-139.16\\
881.5786	-197.754\\
1231.216404	-133.057\\
838.126043	-179.443\\
1156.510135	-220.947\\
1403.897238	-172.119\\
1039.942998	-84.229\\
515.144564	-142.822\\
917.916994	-238.037\\
1569.139904	-280.762\\
1830.287478	-205.078\\
1318.036306	-175.781\\
1145.916339	-229.492\\
1474.945084	-175.781\\
1091.072667	-112.305\\
686.85738	-101.318\\
625.233378	-130.615\\
808.376235	-136.719\\
846.153891	-136.719\\
853.673436	-155.029\\
948.157364	-106.201\\
649.525316	-114.746\\
714.408596	-163.574\\
991.422014	-92.773\\
548.659522	-76.904\\
444.966544	-61.035\\
345.33603	-48.828\\
275.38992	-57.373\\
332.992892	-102.539\\
602.724242	-93.994\\
562.836072	-133.057\\
821.094747	-169.678\\
1068.801722	-202.637\\
1268.912894	-157.471\\
971.753541	-136.719\\
798.575679	-61.035\\
356.505435	-95.215\\
584.04881	-167.236\\
1010.439912	-111.084\\
638.733	-56.152\\
331.072192	-113.525\\
698.4058	-172.119\\
1074.711036	-178.223\\
1138.84497	-233.154\\
1545.344712	-305.176\\
2000.42868	-213.623\\
1372.955021	-181.885\\
1182.2525	-216.064\\
1372.870656	-140.381\\
871.344867	-113.525\\
702.606225	-129.395\\
815.059105	-164.795\\
1043.976325	-174.561\\
1105.843935	-177.002\\
1085.730268	-97.656\\
584.764128	-86.67\\
509.44626	-69.58\\
414.07058	-101.318\\
604.767142	-92.773\\
545.319694	-74.463\\
454.000911	-151.367\\
942.410942	-162.354\\
992.957064	-112.305\\
674.50383	-93.994\\
573.081418	-136.719\\
808.556166	-68.359\\
385.54476	-43.945\\
246.223835	-54.932\\
313.826516	-78.125\\
444.921875	-68.359\\
380.486194	-47.607\\
254.554629	-26.855\\
144.077075	-48.828\\
271.776648	-78.125\\
454.921875	-124.512\\
745.577856	-142.822\\
881.354562	-187.988\\
1187.520196	-209.961\\
1276.352919	-107.422\\
631.426516	-83.008\\
513.736512	-178.223\\
1155.241486	-261.23\\
1683.62735	-202.637\\
1294.85043	-189.209\\
1247.265728	-283.203\\
1861.493319	-227.051\\
1463.343695	-185.547\\
1168.760553	-133.057\\
833.202934	-140.381\\
894.507732	-185.547\\
1161.895314	-129.395\\
793.70893	-112.305\\
713.58597	-189.209\\
1229.8585	-236.816\\
1556.591568	-247.803\\
1542.821478	-111.084\\
648.841644	-52.49\\
320.03153	-141.602\\
863.347394	-111.084\\
640.732512	-52.49\\
290.2697	-37.842\\
212.709882	-68.359\\
398.054457	-103.76\\
632.62472	-167.236\\
1013.617396	-117.188\\
690.940448	-86.67\\
491.93892	-52.49\\
285.44062	-34.18\\
185.87084	-48.828\\
264.64776	-40.283\\
219.058954	-56.152\\
318.718752	-96.436\\
552.674716	-87.891\\
481.203225	-43.945\\
235.764925	-39.063\\
211.72146	-52.49\\
288.32757	-67.139\\
361.409237	-39.063\\
212.424594	-62.256\\
338.548128	-45.166\\
239.018472	-34.18\\
188.36598	-79.346\\
457.667728	-120.85\\
732.47185	-175.781\\
1120.076532	-239.258\\
1533.404522	-207.52\\
1257.77872	-100.098\\
582.870654	-70.801\\
408.380168	-74.463\\
411.78039	-41.504\\
222.66896	-36.621\\
205.846641	-85.449\\
488.170137	-91.553\\
523.042289	-87.891\\
505.37325	-103.76\\
606.06216	-122.07\\
724.24131	-130.615\\
779.640935	-135.498\\
823.692342	-166.016\\
1015.353856	-152.588\\
963.898396	-223.389\\
1362.002733	-115.967\\
658.228692	-51.27\\
284.44596	-50.049\\
286.830819	-95.215\\
543.963295	-74.463\\
419.97132	-67.139\\
360.200735	-31.738\\
170.27437	-48.828\\
257.470044	-37.842\\
191.253468	-18.311\\
92.873392	-32.959\\
176.825035	-69.58\\
384.7774	-87.891\\
508.537326	-130.615\\
791.657515	-183.105\\
1103.02452	-128.174\\
715.85179	-51.27\\
291.00852	-93.994\\
545.541176	-106.201\\
589.203148	-50.049\\
276.77097	-68.359\\
400.515381	-139.16\\
822.99224	-128.174\\
790.961754	-207.52\\
1329.99568	-240.479\\
1501.550876	-148.926\\
902.640486	-114.746\\
684.918874	-100.098\\
};
\addplot [color=mycolor2, line width=2.0pt, forget plot]
  table[row sep=crcr]{%
748.77738	-115.588447766479\\
871.344867	-134.509138945702\\
1103.022147	-170.273062767706\\
906.02247	-139.862305868336\\
871.41992	-134.5207248456\\
1207.758202	-186.441123323481\\
1175.440245	-181.452213956836\\
1380.105287	-213.046270012377\\
1012.807982	-156.346740235235\\
514.21311	-79.3788605180268\\
639.268322	-98.6835807543511\\
615.849234	-95.068386035159\\
755.497303	-116.625799440591\\
581.36728	-89.745421365987\\
220.38624	-34.0209307480558\\
163.895032	-25.3004068385684\\
154.726836	-23.8851162959257\\
527.002454	-81.3531461473704\\
951.439782	-146.873357131404\\
1013.617396	-156.47168913241\\
1041.954609	-160.846092729775\\
713.578272	-110.154776337349\\
444.469647	-68.6125916037217\\
941.513703	-145.341072510283\\
638.733	-98.6009433234031\\
492.486464	-76.0249273552598\\
847.362926	-130.807056846725\\
722.361969	-111.510711931826\\
626.91792	-96.7770544160795\\
1056.85398	-163.14610233555\\
983.661355	-151.847387740694\\
759.149664	-117.189612864913\\
1111.444928	-171.573283911762\\
1087.604023	-167.892973480512\\
852.504518	-131.600762231266\\
686.604492	-105.990845316096\\
522.101349	-80.5965646394053\\
646.534493	-99.8052565014147\\
820.874989	-126.718125204096\\
779.640935	-120.352841710918\\
821.134314	-126.758157094773\\
848.182002	-130.933497262757\\
936.810363	-144.615019901805\\
1325.957961	-204.68757017686\\
1157.9493	-178.752144167711\\
757.782457	-116.978557698015\\
666.81025	-102.935216542302\\
366.055626	-56.5078523145647\\
400.085	-61.7609524003672\\
549.20302	-84.7802381402899\\
421.311654	-65.0377020093579\\
450.625	-69.562790845484\\
652.951752	-100.795886061733\\
753.406772	-116.303085053442\\
725.071369	-111.928960865011\\
1150.57806	-177.614248963512\\
807.68464	-124.681936602341\\
462.750846	-71.4346525689182\\
361.501515	-55.8048361237627\\
492.869832	-76.0841077114337\\
565.86103	-87.3517281054094\\
280.70325	-43.3320420957506\\
328.302182	-50.6798691164094\\
455.226442	-70.2731135027569\\
360.90105	-55.7121425954298\\
311.531296	-48.0909545308639\\
418.556523	-64.6123936010252\\
503.03268	-77.6529422630419\\
820.48736	-126.658287079153\\
779.640935	-120.352841710918\\
976.373206	-150.722319259057\\
885.056328	-136.62577138671\\
1023.337224	-157.972134873809\\
830.64604	-128.226478218362\\
786.899098	-121.473281266405\\
681.654026	-105.226643971418\\
652.951752	-100.795886061733\\
679.389744	-104.877107715814\\
1259.579997	-194.440831921146\\
1830.287478	-282.540704619633\\
1901.905212	-293.596303956265\\
1778.62727	-274.565940138883\\
1126.580429	-173.909744805864\\
1585.237568	-244.712453554924\\
2020.388885	-311.88657849386\\
2080.445565	-321.157502809453\\
1585.802024	-244.799588389142\\
2148.361408	-331.64164279659\\
2744.732029	-423.703123573321\\
1928.231155	-297.660228653561\\
1534.148465	-236.825850311712\\
990.866604	-152.959658984359\\
748.77738	-115.588447766479\\
628.974226	-97.0944855044397\\
781.80459	-120.68684421905\\
540.217179	-83.3930976619459\\
346.43466	-53.4789720836757\\
341.717382	-52.7507678720274\\
444.870552	-68.6744791391756\\
700.264912	-108.0995985795\\
635.293708	-98.0700212705819\\
669.170016	-103.29949262264\\
890.476236	-137.462440294573\\
952.759472	-147.07707711909\\
1295.75488	-200.025133324727\\
1050.409316	-162.151242276962\\
1295.64918	-200.008816460389\\
908.001822	-140.167857600232\\
803.54348	-124.042667482923\\
914.559414	-141.180150306464\\
648.490067	-100.107137633498\\
593.855262	-91.6731858707264\\
737.54694	-113.854809488925\\
729.142974	-112.557492808469\\
503.807304	-77.7725206426167\\
550.91399	-85.0443598562464\\
387.257752	-59.7808155465593\\
444.966544	-68.6892973836557\\
464.223012	-71.6619102122976\\
324.616434	-50.1109017550044\\
435.183	-67.1790158302585\\
561.050186	-86.609080149876\\
917.435387	-141.624113043277\\
933.228208	-144.062044148038\\
1013.937834	-156.521155011074\\
582.908664	-89.9833641627795\\
874.012106	-134.920879503132\\
1307.005398	-201.761871034657\\
1003.214058	-154.865730241117\\
1169.541216	-180.541583342648\\
1161.59065	-179.314257828615\\
657.994995	-101.5744093527\\
446.232864	-68.884778666992\\
641.201918	-98.9820691518605\\
765.267237	-118.133980024039\\
1314.344902	-202.894867165945\\
1633.362304	-252.141448716893\\
1605.566844	-247.850675301228\\
1164.387342	-179.745981990957\\
1191.827491	-183.981906197683\\
1050.409316	-162.151242276962\\
1024.300388	-158.120818093519\\
989.872338	-152.806174562051\\
648.490067	-100.107137633498\\
570.322504	-88.0404439616637\\
507.79953	-78.3887988833843\\
457.734375	-70.6602620602792\\
526.291308	-81.243366839834\\
830.807908	-128.251465713127\\
1269.213682	-195.927979804033\\
973.565586	-150.288908145973\\
653.794713	-100.926013595123\\
546.989608	-84.4385546650924\\
507.487145	-78.3405761429277\\
288.32757	-44.5090051526139\\
225.698752	-34.8410209807773\\
268.50348	-41.4487687556718\\
509.47248	-78.6470514680056\\
378.66396	-58.4541955066912\\
442.03125	-68.236177289138\\
481.93236	-74.3956947802507\\
446.980168	-69.0001396697105\\
295.698956	-45.6469228947705\\
197.497398	-30.4875898798365\\
162.05058	-25.0156795626727\\
311.794032	-48.1315129761689\\
716.338452	-110.580864157678\\
1055.777946	-162.979995421631\\
1172.703864	-181.029799978081\\
798.575679	-123.27579014162\\
519.654828	-80.2188962263743\\
353.051529	-54.5004153553516\\
247.8498	-38.2604689009598\\
564.884301	-87.2009508623809\\
534.761073	-82.5508408841178\\
833.427438	-128.655841452386\\
1073.408004	-165.701540025787\\
1828.526012	-282.268787857493\\
2084.139904	-321.727796359898\\
1704.36555	-263.102189800606\\
1586.774556	-244.949717743054\\
1010.164104	-155.938605905504\\
1185.64533	-183.027568616288\\
1251.761017	-193.233819282333\\
1341.743434	-207.124366973966\\
988.010166	-152.518711857242\\
896.811718	-138.440445973954\\
881.354562	-136.054331333413\\
803.154765	-123.982661712619\\
955.625	-147.519427465666\\
690.049503	-106.522650208531\\
1255.276651	-193.776526217437\\
1489.21875	-229.890069191507\\
1075.076596	-165.95911986779\\
634.28916	-97.9149496202779\\
695.608008	-107.380714276722\\
1239.581917	-191.35373676142\\
1536.328125	-237.162323504931\\
1923.023629	-296.856345065299\\
2031.331265	-313.575749070929\\
1989.870582	-307.175480954779\\
1483.7615	-229.047635814883\\
1340.594886	-206.947066104507\\
1561.530288	-241.052770758463\\
1783.281856	-275.284466612981\\
1054.611558	-162.79994059891\\
653.60548	-100.89680177687\\
941.620548	-145.357566127787\\
730.95516	-112.837239195578\\
475.123848	-73.3446676596405\\
665.879643	-102.791559132925\\
730.95516	-112.837239195578\\
559.67377	-86.396603393539\\
447.595176	-69.0950781948085\\
595.547204	-91.9343702423629\\
479.448266	-74.0122262390856\\
713.428984	-110.131730811868\\
995.879436	-153.733487742106\\
887.11623	-136.94375758807\\
597.484962	-92.2335011260537\\
619.660888	-95.6567894525844\\
1012.066117	-156.232218848648\\
1669.2597	-257.682914569532\\
1141.485144	-176.210579362661\\
706.777352	-109.104921190571\\
508.506999	-78.4980105739844\\
602.943418	-93.0761206723888\\
526.003236	-81.1988972869143\\
388.466254	-59.9673714896658\\
450.580536	-69.5559269565915\\
552.092123	-85.2262277496548\\
728.0004	-112.381114142865\\
791.697192	-122.213961009826\\
540.101771	-83.3752821777497\\
969.84375	-149.714369895256\\
1110.572727	-171.438642611927\\
1051.364496	-162.298693009965\\
861.097054	-132.927188382945\\
943.067026	-145.58085832546\\
1242.975358	-191.877580814745\\
722.959	-111.60287535503\\
306.850943	-47.3684504158636\\
450.625	-69.562790845484\\
503.807304	-77.7725206426167\\
705.332584	-108.881892964875\\
586.225948	-90.4954518921932\\
439.033848	-67.7734696089032\\
702.514664	-108.446891845143\\
653.952015	-100.950296238397\\
463.459986	-71.5441222110822\\
597.084597	-92.1716969501716\\
522.660952	-80.6829503180639\\
469.135818	-72.4202979986886\\
538.72647	-83.1629775434916\\
309.81648	-47.8262391095125\\
379.255732	-58.5455471002872\\
262.362177	-40.5007384064729\\
294.8025	-45.5085373608339\\
622.231659	-96.0536382856155\\
737.54694	-113.854809488925\\
1044.221584	-161.196044702651\\
1351.760504	-208.67069783731\\
842.078512	-129.991304090391\\
472.73056	-72.9752168023268\\
359.152866	-55.4422761700175\\
357.972	-55.2599863845539\\
514.619413	-79.4415813404599\\
306.79522	-47.3598484798985\\
401.866476	-62.0359580977525\\
506.968917	-78.2605775055368\\
899.074248	-138.789711774058\\
816.171638	-125.992070897553\\
1147.15032	-177.08511018805\\
740.970912	-114.383366599819\\
667.38566	-103.024042340872\\
312.542032	-48.2469814201133\\
336.306912	-51.9155559042365\\
182.76046	-28.2126549876381\\
248.866255	-38.4173786298227\\
286.34295	-44.2026402017839\\
542.249425	-83.7068146182723\\
577.365264	-89.1276318470561\\
684.495108	-105.665220599303\\
749.485989	-115.697835438397\\
1220.339934	-188.383359976108\\
1032.565716	-159.396733284495\\
776.862614	-119.923953472084\\
940.695844	-145.214819962026\\
1021.356056	-157.66630280676\\
1322.112747	-204.093986116414\\
967.764798	-149.393442953445\\
597.8611	-92.2915653901825\\
292.080418	-45.0883307126669\\
215.393201	-33.250157427356\\
223.416032	-34.4886384589047\\
284.4958	-43.9174964367681\\
295.698956	-45.6469228947705\\
279.78039	-43.1895805874906\\
401.866476	-62.0359580977525\\
622.633033	-96.1155982204012\\
564.884301	-87.2009508623809\\
600.35536	-92.6766032524721\\
667.82112	-103.091264117075\\
373.2967	-57.6256538483426\\
202.51413	-31.2620206789351\\
473.219582	-73.0507069218382\\
732.137436	-113.019746642146\\
739.052696	-114.087252406411\\
839.507716	-129.59445139812\\
680.846	-105.101909339216\\
511.789293	-79.0046969118826\\
742.704984	-114.651054561754\\
1046.311401	-161.518648869921\\
1027.378311	-158.595955766501\\
743.4692	-114.769026262767\\
1272.92768	-196.501308106002\\
942.421291	-145.481176486356\\
931.209784	-143.750460888012\\
1066.38768	-164.617815576235\\
1043.213259	-161.040389998453\\
782.12262	-120.735938375771\\
690.90036	-106.653996644104\\
1236.110239	-190.817815295463\\
1695.672036	-261.760175717734\\
1242.975358	-191.877580814745\\
752.301504	-116.132465299295\\
713.289792	-110.110243801642\\
707.050799	-109.14713308842\\
926.550966	-143.031281121862\\
1056.85398	-163.14610233555\\
764.324828	-117.988500744909\\
831.040353	-128.287348149562\\
1105.200488	-170.609332347453\\
1175.934487	-181.528509885542\\
721.35365	-111.355058153845\\
590.037552	-91.0838475911398\\
817.91113	-126.260595542598\\
1380.302402	-213.07669857164\\
1233.792221	-190.459983836245\\
1246.789412	-192.466354719157\\
694.410396	-107.195839418314\\
650.554284	-100.425789939585\\
586.225948	-90.4954518921932\\
351.02781	-54.1880147084123\\
221.921888	-34.2579880809518\\
178.38542	-27.5372819114426\\
159.873729	-24.6796399935969\\
444.921875	-68.6824018489998\\
632.628189	-97.6585462288251\\
760.430048	-117.387265201987\\
542.157392	-83.6926075229494\\
625.518306	-96.5609965942845\\
679.014204	-104.819135764693\\
404.830041	-62.4934423745946\\
446.003866	-68.8494283424916\\
403.636662	-62.3092209626082\\
305.751512	-47.1987317169409\\
435.183	-67.1790158302585\\
717.00305	-110.683457870123\\
906.875	-139.993910564213\\
540.44034	-83.4275469312227\\
407.664972	-62.931069475209\\
327.24749	-50.517056758013\\
407.684925	-62.9341496113882\\
258.846346	-39.958001060562\\
712.385852	-109.970702965787\\
1190.823361	-183.826899073862\\
1015.303818	-156.732021383979\\
1372.870656	-211.929463081787\\
1549.048891	-239.126022777391\\
1573.078291	-242.835431102146\\
1118.95652	-172.732845194897\\
801.431044	-123.71657163018\\
794.008585	-122.570769770593\\
793.70893	-122.524512154858\\
962.265003	-148.544441923147\\
1147.613676	-177.156638257986\\
1158.97122	-178.909897526314\\
1610.7195	-248.646088630774\\
1801.310412	-278.067527185018\\
2140.637511	-330.449308080312\\
1412.3525	-218.024258657634\\
1569.139904	-242.227463965059\\
1717.036426	-265.058187574847\\
1336.903482	-206.377225628772\\
1546.563712	-238.742386745235\\
1242.610365	-191.821236999557\\
671.38685	-103.641704350532\\
433.598049	-66.9343476140847\\
473.616274	-73.1119440982622\\
697.385788	-107.655149424149\\
565.037616	-87.2246180341501\\
375.501388	-57.9659906033463\\
517.692692	-79.9160021211241\\
353.312982	-54.5407757444887\\
273.583284	-42.2329359527604\\
363.661837	-56.1383241180926\\
399.572508	-61.6818392268726\\
278.07546	-42.926391263782\\
591.979572	-91.3836363979697\\
791.443818	-122.174847772505\\
588.279744	-90.812495512914\\
1098.1089	-169.514606904332\\
1544.45661	-238.417114299817\\
1440.353493	-222.346760115664\\
1791.578269	-276.565180382272\\
1197.859536	-184.913070435587\\
1288.130956	-198.848231398216\\
799.082427	-123.354016616261\\
508.506999	-78.4980105739844\\
642.514702	-99.1847230632443\\
505.37325	-78.0142550649708\\
368.420508	-56.8729181496609\\
523.23972	-80.7722944896745\\
270.364698	-41.7360841919033\\
176.231773	-27.2048243340423\\
247.8498	-38.2604689009598\\
577.540557	-89.1546917533987\\
752.883216	-116.222263934942\\
951.439782	-146.873357131404\\
921.936696	-142.318978211649\\
890.476236	-137.462440294573\\
1017.817194	-157.120010175111\\
803.909634	-124.099190521191\\
642.514702	-99.1847230632443\\
657.994995	-101.5744093527\\
533.596668	-82.3710921762688\\
863.669664	-133.324321101629\\
556.242848	-85.8669734140812\\
473.566948	-73.1043296644842\\
732.47185	-113.071369990024\\
986.79246	-152.330735096545\\
724.704498	-111.872327143757\\
545.541176	-84.2149608292646\\
467.605064	-72.1839961505046\\
523.042289	-80.7418171496259\\
341.979105	-52.7911699409504\\
347.594325	-53.6579890797274\\
510.119364	-78.7469106777549\\
643.687317	-99.3657391453242\\
727.274592	-112.269071468584\\
566.850808	-87.5045197170543\\
755.713904	-116.659236045439\\
649.232868	-100.221803510009\\
341.979105	-52.7911699409504\\
408.99124	-63.1358049059753\\
828.705768	-127.926959249554\\
1171.52536	-180.84787481356\\
1142.273088	-176.332213945008\\
960.100687	-148.210337376724\\
913.696944	-141.047011176987\\
680.788018	-105.092958682376\\
643.687317	-99.3657391453242\\
521.457303	-80.4971435305169\\
748.231132	-115.504123693527\\
675.940384	-104.344631470087\\
414.823059	-64.0360603408618\\
635.293708	-98.0700212705819\\
762.479718	-117.703671893831\\
912.737534	-140.898907460671\\
1004.755115	-155.103622559057\\
1021.356056	-157.66630280676\\
1427.873736	-220.4202653007\\
1762.265728	-272.040215813655\\
1945.671418	-300.352474683762\\
1164.66988	-179.789597262635\\
618.408423	-95.4634469597438\\
380.486194	-58.7354929993148\\
253.878086	-39.1910528636223\\
396.698003	-61.2381031045995\\
375.501388	-57.9659906033463\\
698.752989	-107.866203664829\\
595.136356	-91.8709478102002\\
554.188624	-85.5498637231812\\
745.577856	-115.094538598507\\
885.674608	-136.721215009065\\
1062.146349	-163.963083101873\\
1170.04095	-180.618727069074\\
1667.895105	-257.47226261597\\
1465.127078	-226.171047964425\\
1085.730268	-167.603722712888\\
732.47185	-113.071369990024\\
758.486232	-117.087199147396\\
954.431731	-147.335223034333\\
650.554284	-100.425789939585\\
679.7877	-104.938540015384\\
613.701886	-94.7369008317271\\
340.079576	-52.4979404518366\\
617.694936	-95.3533062730266\\
910.797772	-140.599467220347\\
637.334726	-98.3850923568418\\
722.010542	-111.456462294326\\
1341.109248	-207.026467949111\\
983.778624	-151.865490506674\\
984.21875	-151.933432571326\\
1396.145358	-215.522368995167\\
1269.151448	-195.91837276775\\
767.700535	-118.509607208147\\
500.645691	-77.2844637796966\\
516.282858	-79.6983666422473\\
927.97614	-143.251284630057\\
1526.27912	-235.61106284913\\
1598.344524	-246.735768814506\\
1606.86135	-248.050507645475\\
1186.542392	-183.166047698189\\
737.54694	-113.854809488925\\
609.90128	-94.1502028893935\\
416.713506	-64.3278878454731\\
396.2581	-61.1701954643636\\
339.024884	-52.3351280934401\\
522.101349	-80.5965646394053\\
583.959605	-90.1455974808908\\
336.9155	-52.0095033766472\\
534.223448	-82.4678479400328\\
762.479718	-117.703671893831\\
893.664128	-137.954553835619\\
1172.100399	-180.936643341016\\
1519.930926	-234.63109482368\\
1807.222638	-278.980195014519\\
1245.652446	-192.290841758144\\
1087.275723	-167.84229394823\\
1119.418663	-172.804186014536\\
952.65625	-147.061142782564\\
630.75704	-97.3696977483197\\
809.675158	-124.989212024939\\
1384.66958	-213.750857994217\\
1539.658008	-237.676356136651\\
842.673502	-130.083152445287\\
453.303698	-69.9763003239022\\
280.66403	-43.3259877209224\\
147.034012	-22.6975783055276\\
191.2308	-29.520217794429\\
355.380621	-54.8599563032704\\
421.311654	-65.0377020093579\\
580.165662	-89.5599281030518\\
452.28211	-69.8185982159982\\
356.545167	-55.0397267772297\\
340.8516	-52.61711747051\\
334.166625	-51.5851607043912\\
301.741476	-46.5797041539986\\
371.77752	-57.3911386736482\\
602.724242	-93.0422865658782\\
891.768888	-137.661986437623\\
948.605304	-146.435799960208\\
945.211813	-145.911948188404\\
1146.60351	-177.000699359396\\
1434.92382	-221.508583788877\\
1459.76512	-225.343324773576\\
1756.596072	-271.16499341256\\
1579.000716	-243.749673346968\\
1140.793614	-176.103828168756\\
766.588004	-118.337866267759\\
768.91893	-118.697690329483\\
1318.036306	-203.464707641681\\
1491.703984	-230.273713713991\\
1012.199552	-156.252817153217\\
617.6765	-95.3504603154964\\
293.100927	-45.2458662557969\\
217.608766	-33.5921732602538\\
222.17	-34.296289025556\\
137.070345	-21.1594912407286\\
338.62218	-52.2729628471164\\
471.935087	-72.8524199714314\\
497.684935	-76.8274131269976\\
422.03125	-65.1487857398239\\
250.174785	-38.6193758530239\\
187.3368	-28.9191026597776\\
280.66403	-43.3259877209224\\
302.266216	-46.6607080460779\\
330.57936	-51.0313961220836\\
248.460933	-38.3548092439452\\
203.703486	-31.4456210621114\\
407.66922	-62.9317252371741\\
1013.81884	-156.502785957573\\
1185.61718	-183.023223112682\\
1332.82619	-205.747815785527\\
922.27002	-142.370433296689\\
1353.05268	-208.870170500442\\
1907.156676	-294.406970235021\\
1712.36265	-264.336698748557\\
1762.30791	-272.046727431172\\
1286.653265	-198.620120862922\\
1302.347999	-201.042910318938\\
1334.096962	-205.943984322223\\
858.75636	-132.565856439188\\
791.822207	-122.233259522553\\
466.154876	-71.9601312416988\\
454.558104	-70.1699424480926\\
803.909634	-124.099190521191\\
554.188624	-85.5498637231812\\
624.73598	-96.440229260224\\
690.119617	-106.533473677085\\
659.061372	-101.739025519609\\
677.630725	-104.60556869603\\
728.0004	-112.381114142865\\
828.653859	-127.918946079157\\
889.77492	-137.354178439983\\
743.579048	-114.785983452112\\
553.762037	-85.4840116682389\\
692.59135	-106.915034055903\\
305.683344	-47.1882086515842\\
141.606415	-21.8597224499835\\
253.968825	-39.2050602047119\\
388.91502	-60.0366473074668\\
189.172158	-29.2024260987358\\
86.37286	-13.3333419027046\\
207.42453	-32.0200370521227\\
385.015838	-59.4347322296649\\
470.418965	-72.6183768589049\\
586.3133	-90.508936383524\\
510.119364	-78.7469106777549\\
836.794098	-129.175551334057\\
699.799716	-108.027786469541\\
481.778432	-74.3719329799306\\
704.476396	-108.749723585084\\
386.361057	-59.6423931182937\\
297.156384	-45.8719054393172\\
198.83067	-30.6934065151233\\
126.513348	-19.5298120745319\\
249.462252	-38.5093824348836\\
193.102533	-29.8091564267676\\
366.899588	-56.6381343719018\\
626.51376	-96.714664407651\\
589.203148	-90.9550409982241\\
426.663392	-65.8638475431251\\
474.51825	-73.2511817522664\\
360.103502	-55.5890254476612\\
507.932175	-78.4092752359873\\
362.684878	-55.9875113700613\\
341.696691	-52.7475738111\\
311.13414	-48.0296457269559\\
233.34795	-36.0218244439888\\
303.617916	-46.8693694039383\\
281.62611	-43.4745036040106\\
499.53596	-77.113155074126\\
452.974656	-69.9255062493937\\
362.684878	-55.9875113700613\\
344.705616	-53.2120602919761\\
274.14069	-42.3189824814572\\
347.594325	-53.6579890797274\\
811.362024	-125.2496128166\\
1057.995928	-163.322384365798\\
717.923493	-110.825546267114\\
651.298296	-100.540642757655\\
472.036194	-72.8680278163002\\
806.095224	-124.436579126004\\
718.306916	-110.884735113615\\
461.723746	-71.2760995759032\\
580.165662	-89.5599281030518\\
440.736824	-68.0363572944574\\
606.771155	-93.6670068157287\\
354.216075	-54.680185829311\\
360.103502	-55.5890254476612\\
455.842428	-70.3682030013017\\
588.20294	-90.8006393118863\\
414.461058	-63.9801783994496\\
458.252561	-70.7402542141479\\
494.407914	-76.3215407798813\\
786.430392	-121.40092731923\\
694.410396	-107.195839418314\\
1112.824412	-171.786234273963\\
1444.021084	-222.912924588651\\
1099.72863	-169.76464394004\\
680.159152	-104.995880903673\\
514.21311	-79.3788605180268\\
823.692342	-127.153038796278\\
1041.08004	-160.711085882781\\
806.458477	-124.492654338127\\
964.432304	-148.879007262758\\
550.349382	-84.9572015578527\\
273.117393	-42.1610165559453\\
301.8175	-46.5914399467558\\
696.599667	-107.533796257542\\
619.610096	-95.6489487130058\\
593.881434	-91.6772260313078\\
935.625	-144.432035916352\\
869.089572	-134.160990009492\\
772.262828	-119.213886450832\\
524.286928	-80.9339515461532\\
682.853446	-105.411797930022\\
878.067696	-135.546939217806\\
714.7069	-110.329002164825\\
739.052696	-114.087252406411\\
647.77524	-99.9967900915466\\
466.154876	-71.9601312416988\\
554.295456	-85.5663553338811\\
400.592058	-61.8392267295778\\
463.459986	-71.5441222110822\\
811.362024	-125.2496128166\\
944.0625	-145.734529226218\\
989.75877	-152.788643117877\\
861.032952	-132.91729298429\\
599.75074	-92.5832683185448\\
410.159595	-63.3161633736553\\
492.869832	-76.0841077114337\\
593.622162	-91.6372023222191\\
774.81726	-119.608213039336\\
978.021792	-150.976810783293\\
1154.19623	-178.172784337611\\
855.613044	-132.080624076427\\
504.363252	-77.8583421004696\\
968.001076	-149.429917088469\\
1366.777536	-210.988870721939\\
945.211813	-145.911948188404\\
956.683959	-147.682898519049\\
1108.952845	-171.188582112043\\
823.357891	-127.101409736725\\
681.83115	-105.253986528457\\
772.358755	-119.228694661289\\
694.410396	-107.195839418314\\
803.804092	-124.082898047246\\
1007.429664	-155.516491558099\\
759.672744	-117.270360503494\\
885.056328	-136.62577138671\\
833.427438	-128.655841452386\\
845.222338	-130.476615181637\\
661.814712	-102.164057452028\\
857.199893	-132.325585286059\\
604.569944	-93.3272068037458\\
632.002151	-97.5619050072876\\
709.758857	-109.565174857633\\
660.432065	-101.950618818854\\
315.411502	-48.6899402915638\\
182.8056	-28.2196232303649\\
147.034012	-22.6975783055276\\
270.217332	-41.7133353647505\\
726.43857	-112.140015105692\\
967.764798	-149.393442953445\\
941.177853	-145.289227487691\\
591.798438	-91.3556748189251\\
698.752989	-107.866203664829\\
607.41428	-93.7662857502363\\
519.807119	-80.2424053236962\\
330.47016	-51.0145389642244\\
462.750846	-71.4346525689182\\
461.27868	-71.2073949255388\\
458.82337	-70.8283697582915\\
538.72647	-83.1629775434916\\
446.003866	-68.8494283424916\\
417.094524	-64.38670543314\\
276.77097	-42.7250176936737\\
418.009884	-64.528009169569\\
802.67488	-123.908582067937\\
559.714544	-86.4028976586905\\
692.173394	-106.850514364784\\
635.293708	-98.0700212705819\\
861.851312	-133.043622871705\\
821.253378	-126.776536954144\\
1125.090432	-173.679734598549\\
828.461634	-127.889272387129\\
914.559414	-141.180150306464\\
894.207272	-138.038398745402\\
424.635816	-65.5508514927255\\
767.681028	-118.506595920799\\
556.150815	-85.8527663187583\\
669.3434	-103.326257837459\\
769.813044	-118.835714329869\\
987.961216	-152.511155466425\\
742.704984	-114.651054561754\\
976.267032	-150.705929223539\\
1235.485744	-190.721412266184\\
995.69139	-153.704459160542\\
852.504518	-131.600762231266\\
679.7877	-104.938540015384\\
811.153827	-125.217473533681\\
667.29995	-103.010811324387\\
557.978696	-86.1349355363859\\
476.245701	-73.5178476332292\\
557.978696	-86.1349355363859\\
500.040192	-77.1909931549107\\
1074.711036	-165.902688525052\\
1410.08252	-217.673842804179\\
1189.57155	-183.633657538725\\
1170.413288	-180.676204728808\\
1089.12738	-168.128133456932\\
593.622162	-91.6372023222191\\
512.38825	-79.0971576508927\\
406.19844	-62.7046815500291\\
352.244448	-54.375826602401\\
399.284919	-61.6374442344588\\
692.207023	-106.855705659305\\
887.839524	-137.055412177234\\
1007.429664	-155.516491558099\\
832.336556	-128.487442458981\\
569.697634	-87.9439830437927\\
1053.191346	-162.580703072559\\
1268.719384	-195.851675230631\\
1398.415634	-215.872830542018\\
1152.033472	-177.838920298991\\
1856.71427	-286.620197334448\\
1287.900774	-198.812698300136\\
693.049832	-106.985809728519\\
485.832872	-74.9978151695906\\
412.81814	-63.7265618419788\\
766.588004	-118.337866267759\\
1019.745474	-157.417677943947\\
1364.351464	-210.614359012398\\
1012.199552	-156.252817153217\\
739.247319	-114.117296276686\\
397.51054	-61.3635340979648\\
412.752351	-63.7164060218469\\
450.625	-69.562790845484\\
494.035311	-76.2640222122089\\
316.69728	-48.8884252981385\\
395.423585	-61.0413717364248\\
324.19188	-50.045363533413\\
203.832486	-31.4655347376045\\
472.73056	-72.9752168023268\\
543.849284	-83.953784176107\\
687.240032	-106.088953357359\\
623.477288	-96.2459255080246\\
655.973505	-101.262352795548\\
671.38685	-103.641704350532\\
855.613044	-132.080624076427\\
593.881434	-91.6772260313078\\
865.644142	-133.629120436204\\
1016.031864	-156.844409537374\\
782.12262	-120.735938375771\\
828.14116	-127.83980095116\\
714.065501	-110.229989672208\\
831.144732	-128.303461091688\\
1165.04035	-179.846786559999\\
888.939294	-137.225183207443\\
687.099048	-106.067189716849\\
765.070606	-118.103626179652\\
673.072468	-103.90191248896\\
467.824016	-72.2177956782299\\
746.58012	-115.249257668696\\
1115.68259	-172.227450004145\\
1051.364496	-162.298693009965\\
1208.198876	-186.509149982661\\
1770.892235	-273.37188605425\\
1170.413288	-180.676204728808\\
974.983464	-150.507785373731\\
741.2939	-114.433226120906\\
963.898396	-148.796588110392\\
1394.148063	-215.214047409943\\
908.001822	-140.167857600232\\
805.61327	-124.362180090666\\
1138.724496	-175.78441929737\\
698.752989	-107.866203664829\\
396.755636	-61.2469998864576\\
505.37325	-78.0142550649708\\
383.015477	-59.125937347316\\
293.15665	-45.254468191762\\
318.305404	-49.136670720531\\
274.018275	-42.3000853296317\\
270.897744	-41.8183702776931\\
367.220172	-56.687622788524\\
549.20012	-84.7797904685153\\
637.334726	-98.3850923568418\\
881.651145	-136.100114725806\\
843.44876	-130.202828695174\\
1024.484736	-158.149275816386\\
1201.235388	-185.434199282414\\
1054.697562	-162.813217000051\\
757.409568	-116.920994980643\\
806.467662	-124.494072222696\\
718.491774	-110.913271564966\\
803.54348	-124.042667482923\\
1113.095295	-171.828050368216\\
853.376099	-131.735307821963\\
964.682589	-148.917643652454\\
823.595256	-127.13805167148\\
800.825655	-123.623117986048\\
1246.563624	-192.431499921\\
1024.484736	-158.149275816386\\
1025.825624	-158.356268130375\\
908.051188	-140.175478208793\\
523.042289	-80.7418171496259\\
325.079512	-50.182386910198\\
262.457646	-40.515475915664\\
481.200277	-74.2826834368709\\
456.2395	-70.4294988381652\\
608.637931	-93.9551802380718\\
459.095956	-70.8704487439345\\
664.460588	-102.572500212175\\
1210.65305	-186.887999786069\\
1479.778476	-228.432778082967\\
1158.555468	-178.845718065744\\
1189.04492	-183.552361888145\\
1031.472562	-159.22798356341\\
717.670888	-110.786551739998\\
811.248529	-125.232092641406\\
880.445104	-135.91393868623\\
764.324828	-117.988500744909\\
891.482127	-137.617719263219\\
1165.700965	-179.948765418416\\
1675.76445	-258.687050078552\\
1474.25607	-227.580286602153\\
1384.66958	-213.750857994217\\
1507.086142	-232.648250944962\\
1019.745474	-157.417677943947\\
1075.92138	-166.08952881692\\
858.75636	-132.565856439188\\
889.402162	-137.296635945026\\
1186.723872	-183.194062689107\\
1286.864952	-198.652798895667\\
498.130306	-76.896164860083\\
588.376044	-90.8273613032239\\
510.045081	-78.735443642432\\
718.491774	-110.913271564966\\
720.142731	-111.16812909947\\
444.870552	-68.6744791391756\\
604.731732	-93.3521819489448\\
1130.00346	-174.438156654986\\
2014.61374	-310.995071800407\\
1901.614911	-293.551490313557\\
1697.995	-262.118770688882\\
1380.496734	-213.106697520368\\
1593.894812	-246.048868653222\\
1899.174999	-293.174842129063\\
1412.3525	-218.024258657634\\
1424.440614	-219.890295708171\\
1438.98876	-222.136086859101\\
952.759472	-147.07707711909\\
832.808012	-128.560220922495\\
1050.137142	-162.109226891585\\
694.410396	-107.195839418314\\
495.711007	-76.5226986956899\\
682.853446	-105.411797930022\\
501.246779	-77.3772534802763\\
588.417027	-90.8336878316172\\
571.86129	-88.2779857062938\\
711.341578	-109.809498829649\\
967.539027	-149.358590779571\\
757.782457	-116.978557698015\\
542.533992	-83.7507431058229\\
678.54681	-104.746984350403\\
791.37982	-122.164968428287\\
937.773473	-144.763694785558\\
777.14637	-119.967756765908\\
653.348552	-100.857139910071\\
1035.023604	-159.776156416513\\
969.226632	-149.61910565035\\
689.498388	-106.437574818846\\
1143.30762	-176.491914212722\\
1030.283012	-159.044353232535\\
711.912179	-109.89758226492\\
471.747416	-72.8234492784588\\
327.24749	-50.517056758013\\
260.517488	-40.2159745449872\\
577.540557	-89.1546917533987\\
552.674716	-85.3161623848993\\
366.899588	-56.6381343719018\\
264.123636	-40.772654088026\\
329.89475	-50.9257131656549\\
600.776001	-92.7415374259068\\
720.645096	-111.245678972253\\
973.410594	-150.264982096422\\
1111.444928	-171.573283911762\\
1065.244491	-164.441341973342\\
1098.94917	-169.644319029189\\
633.4002	-97.7777212406939\\
259.906361	-40.1216351282194\\
396.698003	-61.2381031045995\\
351.12384	-54.2028388189363\\
457.734375	-70.6602620602792\\
757.782457	-116.978557698015\\
858.75636	-132.565856439188\\
1228.033352	-189.570989662035\\
796.745316	-122.993237778701\\
770.604768	-118.957932429219\\
1172.1314	-180.941428952287\\
849.933722	-131.203909538995\\
550.91399	-85.0443598562464\\
405.16434	-62.5450479699718\\
557.666402	-86.0867268435608\\
798.496545	-123.263574259476\\
1157.62397	-178.701923113074\\
777.289865	-119.989908028426\\
630.090533	-97.2668093443517\\
590.196813	-91.1084326444166\\
744.26079	-114.891223676593\\
1042.293528	-160.8984115126\\
1635.108943	-252.411076641308\\
1404.05553	-216.743458904488\\
1333.536708	-205.857498148969\\
823.357891	-127.101409736725\\
539.191708	-83.2347961369725\\
607.793676	-93.8248529504485\\
283.177242	-43.7139511954441\\
421.311654	-65.0377020093579\\
392.4312	-60.5794385283601\\
469.252244	-72.4382706311146\\
821.238132	-126.774183435566\\
1165.420707	-179.905502109375\\
1404.53024	-216.816739686607\\
1804.07024	-278.493560667256\\
1485.074406	-229.250308559355\\
786.592205	-121.425906323668\\
711.158448	-109.781229128928\\
601.073283	-92.787428722635\\
848.614833	-131.000313201341\\
1146.64077	-177.006451169852\\
1515.69943	-233.977880574112\\
1103.933336	-170.413722628628\\
750.97464	-115.927637864261\\
868.91504	-134.134047578397\\
1133.973552	-175.051018079526\\
799.140342	-123.36295693034\\
522.101349	-80.5965646394053\\
414.403236	-63.9712524417414\\
276.77097	-42.7250176936737\\
263.768856	-40.7178868493328\\
201.818331	-31.1546104813041\\
368.449415	-56.8773805110367\\
570.322536	-88.0404489014901\\
613.701886	-94.7369008317271\\
455.842428	-70.3682030013017\\
394.261686	-60.862009878737\\
172.74572	-26.6666838054092\\
85.43291	-13.1882422183657\\
159.273442	-24.5869739430487\\
314.117175	-48.4901355794714\\
497.287278	-76.7660269839313\\
642.732024	-99.2182709685583\\
772.45711	-119.243877681081\\
915.440162	-141.316111003078\\
1216.8477	-187.844265289117\\
1724.944466	-266.278948368229\\
1707.66051	-263.610831383573\\
1830.287478	-282.540704619633\\
1559.382026	-240.721144461237\\
838.634346	-129.459629640249\\
754.047756	-116.402033482684\\
780.418524	-120.47287779629\\
1030.461312	-159.071877328197\\
877.685088	-135.487876182511\\
602.724242	-93.0422865658782\\
457.667728	-70.6499737910499\\
373.117806	-57.5980380571514\\
656.950776	-101.413213706888\\
620.677459	-95.8137170883183\\
309.453672	-47.7702325918513\\
231.79984	-35.7828433574184\\
398.76298	-61.5568727315659\\
547.370736	-84.4973892334127\\
443.582272	-68.4756078999181\\
610.65575	-94.2666701045041\\
506.23947	-78.1479730803632\\
744.26079	-114.891223676593\\
1108.952845	-171.188582112043\\
903.952764	-139.542806227637\\
1220.46587	-188.402800663216\\
1699.68537	-262.379712391543\\
1822.731648	-281.374314335131\\
1403.106309	-216.596928059797\\
883.059856	-136.317576847648\\
625.518306	-96.5609965942845\\
379.059723	-58.5152892737777\\
586.225948	-90.4954518921932\\
450.395484	-69.5273605575423\\
760.194564	-117.350913636407\\
681.654026	-105.226643971418\\
550.422209	-84.9684438310707\\
692.173394	-106.850514364784\\
405.16434	-62.5450479699718\\
570.322536	-88.0404489014901\\
419.894832	-64.8189877958738\\
255.638757	-39.4628469019484\\
298.720216	-46.1133135243846\\
231.5695	-35.7472858689449\\
256.268481	-39.5600571297484\\
243.763685	-37.6296970549336\\
160.37209	-24.7565717705922\\
229.219625	-35.3845366580977\\
336.306912	-51.9155559042365\\
329.71107	-50.897358561666\\
340.880475	-52.6215748920593\\
565.037616	-87.2246180341501\\
753.38732	-116.300082256421\\
622.633033	-96.1155982204012\\
586.637241	-90.5589430188178\\
955.625	-147.519427465666\\
1157.62397	-178.701923113074\\
910.034906	-140.481703918264\\
901.25	-139.125581690968\\
407.684925	-62.9341496113882\\
247.014845	-38.131577250407\\
530.971496	-81.965845483854\\
794.217233	-122.602978674192\\
893.664128	-137.954553835619\\
1257.767859	-194.161092943776\\
1109.799405	-171.319265221542\\
767.75497	-118.518010316096\\
518.893353	-80.1013476523737\\
550.710846	-85.0130006028017\\
593.290654	-91.5860275723326\\
456.780395	-70.5129965707685\\
270.897744	-41.8183702776931\\
356.545167	-55.0397267772297\\
283.55098	-43.771644972587\\
218.850592	-33.7838734151597\\
177.418297	-27.3879875425752\\
340.880475	-52.6215748920593\\
531.818052	-82.0965279758751\\
559.04733	-86.2999001154313\\
398.76298	-61.5568727315659\\
740.485746	-114.308471729398\\
1011.715482	-156.178091471854\\
653.794713	-100.926013595123\\
885.056328	-136.62577138671\\
973.428874	-150.267803972298\\
676.464904	-104.42560140086\\
396.2581	-61.1701954643636\\
501.47262	-77.4121164600208\\
608.726729	-93.9688879478116\\
334.06119	-51.5688847479911\\
384.7774	-59.3979246563531\\
286.172208	-44.1762828313883\\
165.0376	-25.4767845779543\\
169.100064	-26.1039054291039\\
306.421464	-47.3021519241032\\
421.0494	-64.9972179701892\\
371.27867	-57.3141314099295\\
466.587968	-72.0269874707426\\
555.343704	-85.7281729347203\\
388.91502	-60.0366473074668\\
196.097244	-30.2714486984695\\
171.3868	-26.4569078991995\\
218.10352	-33.6685482261832\\
266.661072	-41.1643570111923\\
346.43466	-53.4789720836757\\
748.664334	-115.570996903225\\
753.38732	-116.300082256421\\
633.689784	-97.8224242004148\\
629.385498	-97.1579733893032\\
858.249722	-132.4876469452\\
1145.010537	-176.754792790472\\
1254.016016	-193.581922525149\\
1254.257048	-193.619130533144\\
922.884599	-142.465305597238\\
963.092636	-148.672203279653\\
992.957064	-153.282362411268\\
1027.473024	-158.610576572292\\
1175.440245	-181.452213956836\\
1571.220504	-242.548645307934\\
1364.876288	-210.695375871522\\
1216.018738	-187.716298775442\\
971.753541	-150.009183524974\\
908.68116	-140.272726720248\\
883.925358	-136.451184025683\\
1010.439912	-155.981182269926\\
621.31488	-95.9121154987561\\
734.64715	-113.407170131885\\
919.022346	-141.869091233562\\
1097.473388	-169.416503185417\\
846.117826	-130.614851285822\\
942.421291	-145.481176486356\\
777.14637	-119.967756765908\\
653.952015	-100.950296238397\\
417.442696	-64.4404525976653\\
679.389744	-104.877107715814\\
1082.544232	-167.111895681778\\
832.336556	-128.487442458981\\
477.717867	-73.7451052766086\\
557.908728	-86.1241346057897\\
297.988628	-46.0003785939456\\
280.424547	-43.2890188314021\\
388.909893	-60.0358558546431\\
243.986732	-37.6641287260786\\
252.252506	-38.9401127658856\\
292.677696	-45.1805322651565\\
197.826642	-30.5384151369996\\
336.306912	-51.9155559042365\\
268.50099	-41.4483843754239\\
162.599904	-25.1004784764444\\
308.838859	-47.6753241688209\\
355.380621	-54.8599563032704\\
443.999226	-68.5399729127207\\
415.732212	-64.1764059053235\\
432.277384	-66.7304771160609\\
593.97684	-91.6919538152155\\
526.579548	-81.2878623268427\\
628.62024	-97.039840835191\\
1078.383416	-166.469592273945\\
750.971466	-115.927147895222\\
662.15028	-102.215858942399\\
706.29702	-109.030772542688\\
482.40522	-74.4686899786511\\
288.18867	-44.4875632183039\\
523.23972	-80.7722944896745\\
398.289996	-61.4838583913379\\
243.80686	-37.6363619614406\\
458.252561	-70.7402542141479\\
457.292616	-70.5920679101075\\
548.569596	-84.6824567450533\\
326.354145	-50.3791514678319\\
190.4292	-29.3964751411325\\
151.817054	-23.4359344780683\\
196.471665	-30.3292479100338\\
378.37604	-58.4097494179472\\
374.265525	-57.7752109542308\\
454.39224	-70.1443380924959\\
643.687317	-99.3657391453242\\
871.447384	-134.524964451675\\
649.729782	-100.298511914226\\
924.835868	-142.76652216829\\
1066.260048	-164.598113078324\\
923.776854	-142.60304262458\\
901.653648	-139.187892662173\\
1432.035591	-221.062729098523\\
1044.221584	-161.196044702651\\
1303.37256	-201.201071367596\\
1116.032426	-172.281454039647\\
1298.976804	-200.522500371234\\
1441.875	-222.581634508365\\
806.095224	-124.436579126004\\
468.721297	-72.3563085670679\\
383.840514	-59.2532979603996\\
648.490067	-100.107137633498\\
804.050676	-124.120963115136\\
808.184601	-124.759115370661\\
506.130196	-78.1311044991552\\
278.572734	-43.0031552410466\\
392.158899	-60.5374035380466\\
781.476878	-120.636255456079\\
1100.703362	-169.915112907023\\
1053.338902	-162.603481229931\\
602.22304	-92.9649162249132\\
480.743986	-74.2122459879995\\
669.834711	-103.402101308934\\
1094.412506	-168.943996124405\\
1253.62109	-193.520957965398\\
1258.001547	-194.197167261595\\
934.808996	-144.306069723661\\
1326.33396	-204.745612983617\\
1442.682054	-222.706219093337\\
1380.613	-213.124645453661\\
1899.174999	-293.174842129063\\
1620.691925	-250.185527664332\\
1360.967202	-210.091931917425\\
1562.342336	-241.178126265105\\
1272.92768	-196.501308106002\\
635.931588	-98.1684905366557\\
604.792116	-93.3615034015105\\
781.014856	-120.56493331516\\
959.474481	-148.113670221098\\
967.539027	-149.358590779571\\
709.254172	-109.487266847429\\
910.831416	-140.604660830411\\
782.12262	-120.735938375771\\
582.666312	-89.9459523869415\\
784.263095	-121.066362597097\\
753.29397	-116.285671856365\\
1097.576564	-169.432430420942\\
1115.68259	-172.227450004145\\
806.467662	-124.494072222696\\
561.522	-86.6819138776984\\
340.079576	-52.4979404518366\\
554.507	-85.5990112917765\\
422.651988	-65.2446088026008\\
369.338554	-57.014636522775\\
308.83825	-47.6752301577482\\
559.67377	-86.396603393539\\
735.923979	-113.604273685111\\
828.461634	-127.889272387129\\
796.114737	-122.895895564912\\
455.842428	-70.3682030013017\\
344.27568	-53.1456912533188\\
234.688425	-36.2287530033421\\
310.805256	-47.9788760428407\\
523.042289	-80.7418171496259\\
543.963295	-83.9713840060034\\
543.183949	-83.8510766932838\\
918.292936	-141.756492520063\\
1083.500127	-167.259456789029\\
1242.3999	-191.788747606436\\
1260.506625	-194.583875093974\\
1416.195132	-218.61744413583\\
830.944536	-128.272556919757\\
671.38685	-103.641704350532\\
514.953369	-79.493133987072\\
561.050186	-86.609080149876\\
762.763474	-117.747475187654\\
614.478986	-94.8568614303757\\
396.2581	-61.1701954643636\\
381.453512	-58.8848174702498\\
752.301504	-116.132465299295\\
923.776854	-142.60304262458\\
833.457339	-128.660457257122\\
1341.850428	-207.140883592537\\
1440.571972	-222.380486626579\\
966.188304	-149.150080240788\\
650.841156	-100.470074249012\\
443.582272	-68.4756078999181\\
421.336751	-65.0415762226436\\
784.556036	-121.11158377549\\
632.002151	-97.5619050072876\\
735.34968	-113.515619391142\\
843.44876	-130.202828695174\\
949.84375	-146.626978345942\\
928.485178	-143.329864611031\\
1021.017877	-157.614098257423\\
698.585208	-107.840303382753\\
774.938795	-119.626974345934\\
533.113119	-82.2964469214549\\
502.762901	-77.6112965928028\\
796.03804	-122.884055881429\\
1143.30762	-176.491914212722\\
1213.329984	-187.301236956552\\
663.119044	-102.365406631743\\
1411.85133	-217.946893256496\\
1284.34128	-198.263220715365\\
822.99224	-127.044964346355\\
459.039976	-70.861807134988\\
738.266508	-113.965888897043\\
637.334726	-98.3850923568418\\
630.50988	-97.3315437635533\\
724.704498	-111.872327143757\\
476.467324	-73.5520594820958\\
674.347101	-104.098677046594\\
1318.58208	-203.548958543452\\
1359.890403	-209.925706910779\\
899.038896	-138.784254500755\\
995.612112	-153.69222104918\\
1105.200488	-170.609332347453\\
1111.444928	-171.573283911762\\
803.804092	-124.082898047246\\
762.763474	-117.747475187654\\
505.879374	-78.0923852129961\\
808.853503	-124.862373489766\\
848.227809	-130.940568469992\\
1479.07594	-228.324327897493\\
1671.487855	-258.02687391541\\
2188.748352	-337.876158277001\\
1534.148465	-236.825850311712\\
1226.079164	-189.269322486187\\
927.735105	-143.214076158953\\
1007.731278	-155.563051583836\\
733.15242	-113.17642929336\\
496.614184	-76.6621217475566\\
482.216152	-74.4395036313789\\
494.407914	-76.3215407798813\\
349.940976	-54.0202406030569\\
263.292612	-40.6443692642822\\
432.183252	-66.7159459989949\\
571.859874	-88.2777671189721\\
380.486194	-58.7354929993148\\
254.803791	-39.3339535533302\\
267.560216	-41.3031574905531\\
676.52532	-104.434927793252\\
913.961684	-141.087878978926\\
422.459856	-65.2149494669432\\
556.150815	-85.8527663187583\\
566.795424	-87.4959701123759\\
536.409027	-82.8052348467807\\
663.099525	-102.362393491961\\
564.884301	-87.2009508623809\\
380.486194	-58.7354929993148\\
266.454396	-41.1324525244\\
220.549425	-34.0461215475545\\
306.850943	-47.3684504158636\\
622.231659	-96.0536382856155\\
736.835212	-113.744940338294\\
543.89904	-83.9614649888034\\
345.187926	-53.2865142828835\\
212.424594	-32.7918948191042\\
318.718752	-49.2004790766373\\
561.546828	-86.6857465655677\\
724.704498	-111.872327143757\\
769.867802	-118.844167299192\\
517.692692	-79.9160021211241\\
498.3525	-76.9304648539785\\
559.67377	-86.396603393539\\
808.853503	-124.862373489766\\
1161.895314	-179.361288681564\\
1341.109248	-207.026467949111\\
986.79246	-152.330735096545\\
606.415646	-93.6121270416134\\
633.360112	-97.7715328730725\\
591.979572	-91.3836363979697\\
598.48768	-92.3882902800309\\
380.215024	-58.6936326009934\\
508.537326	-78.5026921401602\\
545.541176	-84.2149608292646\\
518.006874	-79.9645022637114\\
305.354576	-47.1374568743399\\
199.840797	-30.8493393933\\
315.630392	-48.7237302483753\\
491.981589	-75.9469900148368\\
801.499413	-123.727125723822\\
905.321154	-139.754044013746\\
1097.473388	-169.416503185417\\
983.778624	-151.865490506674\\
1144.50644	-176.67697554958\\
1577.188736	-243.469958759992\\
2108.279808	-325.454263140441\\
1627.283554	-251.203074647873\\
1107.6426	-170.986320145016\\
1219.452005	-188.246290751559\\
1456.482624	-224.836607253023\\
1862.187904	-287.46515990223\\
1095.004504	-169.035382605525\\
515.596246	-79.5923746379244\\
339.504264	-52.4091297814857\\
345.396288	-53.3186790367835\\
227.6388	-35.1405053708005\\
148.044748	-22.853605194777\\
211.018326	-32.5748098221407\\
331.848072	-51.2272466574493\\
465.093824	-71.7963370927896\\
480.84516	-74.22786417563\\
362.087574	-55.8953058039627\\
345.396288	-53.3186790367835\\
442.03125	-68.236177289138\\
571.859874	-88.2777671189721\\
599.75074	-92.5832683185448\\
540.330908	-83.4106539596215\\
373.035063	-57.5852650552032\\
256.268481	-39.5600571297484\\
248.096838	-38.2986040526378\\
403.533663	-62.2933210504988\\
487.619268	-75.2735803652971\\
348.263951	-53.7613589624075\\
228.728544	-35.3087286916262\\
418.203952	-64.5579673647286\\
266.454396	-41.1324525244\\
384.7774	-59.3979246563531\\
453.303698	-69.9763003239022\\
703.70955	-108.631345892082\\
722.997335	-111.608793112782\\
487.348994	-75.2318582821171\\
736.663788	-113.718477687847\\
715.580964	-110.463931055183\\
578.829734	-89.3537014621673\\
495.098062	-76.4280786350301\\
865.644142	-133.629120436204\\
1071.736757	-165.443550332556\\
1084.744551	-167.451557996097\\
1187.066672	-183.246980580262\\
895.240005	-138.197821302247\\
823.595256	-127.13805167148\\
776.444548	-119.859416800311\\
1050.61374	-162.182799123466\\
833.583696	-128.679962933822\\
1140.793614	-176.103828168756\\
1065.244491	-164.441341973342\\
1161.835737	-179.352091805248\\
1992.425025	-307.56980924135\\
1439.614338	-222.232657070633\\
748.231132	-115.504123693527\\
487.921024	-75.3201623115148\\
523.039341	-80.7413620681115\\
680.680605	-105.076377382951\\
903.952764	-139.542806227637\\
1074.711036	-165.902688525052\\
1188.421729	-183.456160156794\\
1183.070714	-182.630126232232\\
759.672744	-117.270360503494\\
675.279636	-104.242632080516\\
791.657515	-122.207836088101\\
762.91292	-117.770545103528\\
437.96828	-67.6089783269822\\
334.082979	-51.5722483067145\\
609.90128	-94.1502028893935\\
608.515908	-93.9363435991207\\
878.640944	-135.635431269349\\
1070.757174	-165.292332518754\\
928.313652	-143.303386214887\\
624.249478	-96.3651281520478\\
511.668612	-78.9860674369045\\
831.144732	-128.303461091688\\
1064.142869	-164.271285050675\\
1486.945285	-229.539115360074\\
1600.344255	-247.044466444021\\
1870.072935	-288.682368806004\\
1530.778624	-236.305649380416\\
1341.109248	-207.026467949111\\
735.923979	-113.604273685111\\
524.144166	-80.9119134365351\\
971.611453	-149.987249460451\\
1242.975358	-191.877580814745\\
724.097948	-111.778694276545\\
1203.71976	-185.817710738322\\
1650.524444	-254.7907610171\\
2001.887916	-309.030591726624\\
1171.014501	-180.769013725585\\
657.994995	-101.5744093527\\
373.117806	-57.5980380571514\\
559.67377	-86.396603393539\\
502.269222	-77.5350875741691\\
535.29583	-82.6333910962546\\
291.98265	-45.0732383078172\\
432.277384	-66.7304771160609\\
296.0436	-45.7001254434266\\
418.556523	-64.6123936010252\\
332.579544	-51.3401636810173\\
466.395788	-71.9973207253452\\
736.363968	-113.672194601131\\
676.52532	-104.434927793252\\
1004.808906	-155.111926252999\\
1193.2129	-184.19577119963\\
1143.72258	-176.555971412588\\
615.60808	-95.0311591940748\\
661.814712	-102.164057452028\\
767.75497	-118.518010316096\\
495.945996	-76.5589738483289\\
454.39224	-70.1443380924959\\
314.815292	-48.597903605783\\
519.409	-80.1809478618851\\
443.14741	-68.4084784367222\\
254.554629	-39.2954905206697\\
140.156245	-21.63586031983\\
170.27437	-26.2851825501394\\
326.134204	-50.3451992685945\\
481.93236	-74.3956947802507\\
504.548583	-77.8869515686309\\
309.453672	-47.7702325918513\\
379.872462	-58.6407514497976\\
598.930299	-92.4566171880392\\
781.552464	-120.647923634962\\
535.29583	-82.6333910962546\\
505.37325	-78.0142550649708\\
613.18818	-94.6576002535655\\
736.663788	-113.718477687847\\
683.741432	-105.548875953637\\
815.771246	-125.930262553693\\
734.550058	-113.39218208087\\
505.794036	-78.0792116219943\\
412.752351	-63.7164060218469\\
542.249425	-83.7068146182723\\
447.734375	-69.1165662856222\\
574.021968	-88.6115286561933\\
791.37982	-122.164968428287\\
1242.916129	-191.868437659041\\
838.29984	-129.407992090369\\
871.344867	-134.509138945702\\
1273.071129	-196.52345226752\\
908.051188	-140.175478208793\\
557.978696	-86.1349355363859\\
391.10918	-60.3753588595589\\
306.79522	-47.3598484798985\\
324.02482	-50.0195745518016\\
253.968825	-39.2050602047119\\
285.929937	-44.1388835594861\\
583.362093	-90.0533598059201\\
241.502602	-37.2806546276087\\
274.018275	-42.3000853296317\\
370.525078	-57.1977997313079\\
526.648382	-81.2984882023379\\
379.69806	-58.6138290867484\\
283.218103	-43.7202588907489\\
149.588928	-23.091979608908\\
307.783996	-47.5124854132267\\
591.798438	-91.3556748189251\\
807.68464	-124.681936602341\\
620.677459	-95.8137170883183\\
512.38825	-79.0971576508927\\
584.402224	-90.2139243888991\\
597.084597	-92.1716969501716\\
707.485906	-109.214300372162\\
865.122258	-133.548557423638\\
1021.017877	-157.614098257423\\
992.72508	-153.24655113921\\
822.492279	-126.967785578036\\
524.690243	-80.9962111122888\\
417.11319	-64.3895868956729\\
460.524184	-71.0909236968192\\
637.44318	-98.4018343549963\\
360.103502	-55.5890254476612\\
359.092608	-55.4329741680185\\
678.607844	-104.756406143194\\
828.14116	-127.83980095116\\
796.745316	-122.993237778701\\
971.753541	-150.009183524974\\
1220.419178	-188.395592838905\\
642.05768	-99.1141727702118\\
1150.931544	-177.668816139233\\
1148.350383	-177.270363406292\\
1060.763886	-163.749672872701\\
1147.613676	-177.156638257986\\
1188.982245	-183.542686774877\\
2082.8125	-321.522885565292\\
1859.45893	-287.043889338935\\
1269.5345	-195.977504343137\\
1565.35376	-241.642998515555\\
1877.069484	-289.762423118852\\
1848.941058	-285.420249882456\\
2086.191858	-322.044555631857\\
1920.861384	-296.522560218272\\
2529.129002	-390.420575395004\\
1850.783104	-285.704605745149\\
1241.498961	-191.649670033684\\
734.958456	-113.455226307568\\
759.149664	-117.189612864913\\
552.092123	-85.2262277496548\\
480.916106	-74.2388160796729\\
769.829004	-118.838178068326\\
1049.409543	-161.99690774139\\
626.62562	-96.7319321885863\\
553.762037	-85.4840116682389\\
531.373612	-82.0279199608662\\
346.083804	-53.4248105912044\\
446.980168	-69.0001396697105\\
199.840797	-30.8493393933\\
297.73144	-45.9606765910563\\
241.389885	-37.2632545519455\\
358.462084	-55.3356404445566\\
240.644448	-37.1481817572282\\
208.021412	-32.1121774742596\\
117.217422	-18.0948039057594\\
120.186126	-18.5530814878602\\
237.826149	-36.7131221314261\\
155.916462	-24.0687583588879\\
203.703486	-31.4456210621114\\
461.27868	-71.2073949255388\\
643.619955	-99.355340501847\\
654.05388	-100.966021095406\\
465.776965	-71.9017932803091\\
641.201918	-98.9820691518605\\
954.592	-147.359963692144\\
441.09885	-68.092243095109\\
260.155584	-40.1601075774241\\
194.69709	-30.0553075171026\\
147.034012	-22.6975783055276\\
266.454396	-41.1324525244\\
467.249131	-72.1290509236885\\
690.799473	-106.638422760543\\
415.417128	-64.1277665213773\\
779.47548	-120.327300492479\\
1056.85398	-163.14610233555\\
1055.740686	-162.974243611175\\
751.03625	-115.937148573929\\
435.183	-67.1790158302585\\
590.177808	-91.1054988505968\\
826.131306	-127.529540638412\\
1050.802808	-162.211985470738\\
524.95649	-81.0373115491803\\
302.93004	-46.7631822764695\\
471.935087	-72.8524199714314\\
338.205456	-52.2086333393166\\
160.403852	-24.7614748571117\\
114.677087	-17.7026534651881\\
311.794032	-48.1315129761689\\
741.532473	-114.47005453411\\
727.567643	-112.31430962763\\
513.03469	-79.1969483205497\\
314.117175	-48.4901355794714\\
309.56952	-47.7881159986616\\
91.866287	-14.1813599075333\\
178.9116	-27.6185080957135\\
158.144276	-24.4126650647401\\
316.524788	-48.8617977809823\\
479.684155	-74.0486403243444\\
767.681028	-118.506595920799\\
796.457244	-122.948768225781\\
886.258626	-136.811369620958\\
918.085921	-141.724535701984\\
886.258626	-136.811369620958\\
859.091296	-132.617560367986\\
752.883216	-116.222263934942\\
970.791598	-149.860688790518\\
1500.22936	-231.589772404847\\
1290.710166	-199.246382956112\\
630.512784	-97.3319920528062\\
346.524916	-53.4929048642594\\
828.705768	-127.926959249554\\
961.826151	-148.476696525337\\
831.652506	-128.381845949516\\
446.328125	-68.899484067311\\
495.474752	-76.4862281111657\\
891.780568	-137.663789474288\\
1376.500475	-212.489796707094\\
1443.562043	-222.842062623442\\
1617.577216	-249.704711352074\\
1515.69943	-233.977880574112\\
1125.834691	-173.794625545904\\
1096.43274	-169.255858793367\\
487.348994	-75.2318582821171\\
464.888214	-71.7645971639669\\
602.943418	-93.0761206723888\\
762.479718	-117.703671893831\\
811.301634	-125.240290437817\\
842.078512	-129.991304090391\\
515.77317	-79.619686321048\\
738.729696	-114.03739103289\\
585.805307	-90.4305177187584\\
326.404998	-50.3870016239547\\
206.721396	-31.911494553641\\
177.418297	-27.3879875425752\\
285.20541	-44.0270386326336\\
206.530941	-31.8820940957647\\
121.044267	-18.6855523514362\\
271.776648	-41.954046316806\\
551.640078	-85.1564457540093\\
350.65774	-54.1308871588795\\
364.52654	-56.2718079548359\\
504.67941	-77.907147277342\\
847.556424	-130.836927051225\\
538.182216	-83.0789612834776\\
274.89013	-42.4346732175931\\
159.873729	-24.6796399935969\\
482.216152	-74.4395036313789\\
983.778624	-151.865490506674\\
1298.12612	-200.391180641597\\
1876.332446	-289.648646874215\\
2245.451976	-346.629472754656\\
2126.273478	-328.231938375398\\
1348.88	-208.226035651942\\
1407.800456	-217.321561548749\\
1831.631	-282.748103543089\\
1516.152558	-234.04782975201\\
1416.251648	-218.62616848687\\
1614.68924	-249.258895717223\\
1757.37388	-271.285063304869\\
1846.024752	-284.970060957475\\
2403.51552	-371.029675254668\\
1521.05643	-234.804838400596\\
813.800988	-125.626114658734\\
407.10575	-62.8447426113595\\
342.794034	-52.916970186345\\
383.565107	-59.2107834981785\\
412.81814	-63.7265618419788\\
750.800042	-115.900685244775\\
554.188624	-85.5498637231812\\
645.176532	-99.5956286356298\\
793.517076	-122.494895733943\\
409.654586	-63.2382053477089\\
505.345386	-78.0099537110643\\
703.831128	-108.650113836572\\
819.965421	-126.57771557626\\
432.277384	-66.7304771160609\\
289.1628	-44.6379392548006\\
385.54476	-59.5163816953172\\
390.899274	-60.3429557590314\\
1062.617544	-164.035821274927\\
1450.85589	-223.968010702931\\
1388.751616	-214.381000166737\\
1641.559616	-253.406864306687\\
1865.751794	-288.015316095664\\
2453.048884	-378.676119725803\\
2013.940908	-310.891207008861\\
1202.94631	-185.698313588631\\
748.231132	-115.504123693527\\
405.16434	-62.5450479699718\\
457.734375	-70.6602620602792\\
485.242271	-74.9066443427698\\
641.709783	-99.0604680573202\\
412.752351	-63.7164060218469\\
394.294712	-60.8671080884024\\
440.707932	-68.0318972486253\\
561.38764	-86.6611727812701\\
647.110128	-99.8941170331392\\
803.909634	-124.099190521191\\
537.0126	-82.8984081557604\\
243.986732	-37.6641287260786\\
179.00066	-27.6322562502826\\
145.9952	-22.5372173360213\\
172.013021	-26.5535773703697\\
211.72146	-32.6833523206225\\
395.611353	-61.0703574032464\\
411.693282	-63.5529179878099\\
556.150815	-85.8527663187583\\
817.438563	-126.187645574484\\
601.188588	-92.8052283067647\\
1073.408004	-165.701540025787\\
1551.271914	-239.469189898598\\
1199.51857	-185.169174813168\\
1093.644126	-168.825381628472\\
1337.4664	-206.46412304258\\
1502.779498	-231.983436130386\\
921.703014	-142.282904820048\\
779.640935	-120.352841710918\\
476.467324	-73.5520594820958\\
577.604698	-89.164593172467\\
1245.984464	-192.342095236515\\
2162.384906	-333.806444257441\\
2534.57202	-391.260811783802\\
1976.95008	-305.180948520395\\
1628.809119	-251.43857547232\\
1149.080397	-177.383055359016\\
1277.34725	-197.183555259482\\
1657.033472	-255.795556919173\\
927.735105	-143.214076158953\\
790.961754	-122.100431756514\\
612.754344	-94.5906291735561\\
1192.510569	-184.087352659919\\
1066.38768	-164.617815576235\\
558.864552	-86.2716847528013\\
532.597472	-82.2168467119434\\
397.090032	-61.2986204556839\\
751.389749	-115.991718065193\\
1144.50644	-176.67697554958\\
1173.089808	-181.089377990287\\
863.347394	-133.274572417899\\
642.05768	-99.1141727702118\\
698.585208	-107.840303382753\\
522.101349	-80.5965646394053\\
369.613961	-57.0571509849961\\
294.10147	-45.4003196559431\\
246.470862	-38.0476028245484\\
199.144998	-30.741929195669\\
259.906361	-40.1216351282194\\
449.196264	-69.3422374728539\\
669.3434	-103.326257837459\\
650.611775	-100.434664800963\\
406.19844	-62.7046815500291\\
347.69976	-53.6742650361275\\
286.34295	-44.2026402017839\\
393.06425	-60.6771621893748\\
550.91399	-85.0443598562464\\
602.22304	-92.9649162249132\\
560.849094	-86.5780376628042\\
321.846588	-49.6833217983393\\
756.266994	-116.74461631504\\
1039.942998	-160.53556118968\\
704.488896	-108.751653204802\\
277.671852	-42.8640864673602\\
349.791585	-53.9971791775097\\
549.20012	-84.7797904685153\\
545.677165	-84.2359533937346\\
468.818614	-72.3713313512381\\
291.98265	-45.0732383078172\\
561.050186	-86.609080149876\\
841.355163	-129.879641010901\\
471.31146	-72.7561509349452\\
176.825035	-27.2964059383088\\
244.856975	-37.798467770281\\
659.061372	-101.739025519609\\
802.67488	-123.908582067937\\
461.27868	-71.2073949255388\\
401.814202	-62.02788858246\\
816.171638	-125.992070897553\\
1100.703362	-169.915112907023\\
989.075351	-152.683144015615\\
1448.313192	-223.575495487049\\
1711.23246	-264.162231795801\\
1079.21484	-166.59793884552\\
881.5786	-136.088915984809\\
1231.216404	-190.062356054327\\
838.126043	-129.381163120914\\
1156.510135	-178.529980874757\\
1403.897238	-216.719023435332\\
1039.942998	-160.53556118968\\
515.144564	-79.5226486784356\\
917.916994	-141.698458512371\\
1569.139904	-242.227463965059\\
1830.287478	-282.540704619633\\
1318.036306	-203.464707641681\\
1145.916339	-176.894621062479\\
1474.945084	-227.686649402201\\
1091.072667	-168.428426589171\\
686.85738	-106.029883529802\\
625.233378	-96.5170123793163\\
808.376235	-124.788697830269\\
846.153891	-130.620418624633\\
853.673436	-131.781207609018\\
948.157364	-146.366651651678\\
649.525316	-100.266948584199\\
714.408596	-110.282953102389\\
991.422014	-153.04539739138\\
548.659522	-84.6963385836765\\
444.966544	-68.6892973836557\\
345.33603	-53.3093770347845\\
275.38992	-42.5118255887147\\
332.992892	-51.4039720371236\\
602.724242	-93.0422865658782\\
562.836072	-86.8847666170979\\
821.094747	-126.752049153701\\
1068.801722	-164.990470219759\\
1268.912894	-195.881547287567\\
971.753541	-150.009183524974\\
798.575679	-123.27579014162\\
356.505435	-55.0335933651779\\
584.04881	-90.1593680190487\\
1010.439912	-155.981182269926\\
638.733	-98.6009433234031\\
331.072192	-51.1074743896852\\
698.4058	-107.812608245599\\
1074.711036	-165.902688525052\\
1138.84497	-175.803016817845\\
1545.344712	-238.554210230304\\
2000.42868	-308.805330081881\\
1372.955021	-211.94248647119\\
1182.2525	-182.503818882775\\
1372.870656	-211.929463081787\\
871.344867	-134.509138945702\\
702.606225	-108.461026078025\\
815.059105	-125.820329648427\\
1043.976325	-161.158184174451\\
1105.843935	-170.708660988964\\
1085.730268	-167.603722712888\\
584.764128	-90.2697913564623\\
509.44626	-78.6430038976844\\
414.07058	-63.91990047558\\
604.767142	-93.3576481756829\\
545.319694	-84.1807707465083\\
454.000911	-70.0839288001158\\
942.410942	-145.479578915598\\
992.957064	-153.282362411268\\
674.50383	-104.122871236101\\
573.081418	-88.4663363501079\\
808.556166	-124.816473702712\\
385.54476	-59.5163816953172\\
246.223835	-38.0094693709358\\
313.826516	-48.4452666724548\\
444.921875	-68.6824018489998\\
380.486194	-58.7354929993148\\
254.554629	-39.2954905206697\\
144.077075	-22.2411171902449\\
271.776648	-41.954046316806\\
454.921875	-70.2260976236569\\
745.577856	-115.094538598507\\
881.354562	-136.054331333413\\
1187.520196	-183.316990888513\\
1276.352919	-197.030060803151\\
631.426516	-97.4730444755631\\
513.736512	-79.3052882861458\\
1155.241486	-178.334140064675\\
1683.62735	-259.900842629207\\
1294.85043	-199.885513760388\\
1247.265728	-192.539883418817\\
1861.493319	-287.357937109265\\
1463.343695	-225.895747884256\\
1168.760553	-180.421072725195\\
833.202934	-128.621184864767\\
894.507732	-138.084780628647\\
1161.895314	-179.361288681564\\
793.70893	-122.524512154858\\
713.58597	-110.155964674356\\
1229.8585	-189.852736987607\\
1556.591568	-240.290382638841\\
1542.821478	-238.164699663877\\
648.841644	-100.161410426434\\
320.03153	-49.4031320618036\\
863.347394	-133.274572417899\\
640.732512	-98.9096071459807\\
290.2697	-44.8088109400974\\
212.709882	-32.8359346071203\\
398.054457	-61.4474983354312\\
632.62472	-97.6580107207608\\
1013.617396	-156.47168913241\\
690.940448	-106.660185011726\\
491.93892	-75.9404032193359\\
285.44062	-44.0633479009493\\
185.87084	-28.6928030339959\\
264.64776	-40.8535628884457\\
219.058954	-33.8160381690596\\
318.718752	-49.2004790766373\\
552.674716	-85.3161623848993\\
481.203225	-74.2831385183853\\
235.764925	-36.394931853484\\
211.72146	-32.6833523206225\\
288.32757	-44.5090051526139\\
361.409237	-55.7905912078933\\
212.424594	-32.7918948191042\\
338.548128	-52.2615314711659\\
239.018472	-36.8971805291388\\
188.36598	-29.0779767415137\\
457.667728	-70.6499737910499\\
732.47185	-113.071369990024\\
1120.076532	-172.905740974094\\
1533.404522	-236.711008145144\\
1257.77872	-194.162769551757\\
582.870654	-89.97749657514\\
408.380168	-63.0414739795342\\
411.78039	-63.5663648129638\\
222.66896	-34.3733132699283\\
205.846641	-31.776458993905\\
488.170137	-75.358617780066\\
523.042289	-80.7418171496259\\
505.37325	-78.0142550649708\\
606.06216	-93.5575595571534\\
724.24131	-111.80082500791\\
779.640935	-120.352841710918\\
823.692342	-127.153038796278\\
1015.353856	-156.739745728896\\
963.898396	-148.796588110392\\
1362.002733	-210.251786400347\\
658.228692	-101.610485059845\\
284.44596	-43.9098026570272\\
286.830819	-44.2779523331725\\
543.963295	-83.9713840060034\\
419.97132	-64.830795216115\\
360.200735	-55.6040352647869\\
170.27437	-26.2851825501394\\
257.470044	-39.7455419023568\\
191.253468	-29.523717044011\\
92.873392	-14.3368262808469\\
176.825035	-27.2964059383088\\
384.7774	-59.3979246563531\\
508.537326	-78.5026921401602\\
791.657515	-122.207836088101\\
1103.02452	-170.273429086714\\
715.85179	-110.50573835037\\
291.00852	-44.9228622713206\\
545.541176	-84.2149608292646\\
589.203148	-90.9550409982241\\
276.77097	-42.7250176936737\\
400.515381	-61.8273901334864\\
822.99224	-127.044964346355\\
790.961754	-122.100431756514\\
1329.99568	-205.310871152815\\
1501.550876	-231.793774271381\\
902.640486	-139.34023042726\\
684.918874	-105.730637177667\\
};
\end{axis}

\begin{axis}[%
width=4.927cm,
height=3cm,
at={(7cm,4.839cm)},
scale only axis,
xmin=0,
xmax=2286.740716,
xlabel style={font=\color{white!15!black}},
xlabel={y(t-1)u(t)},
ymin=-400,
ymax=0,
ylabel style={font=\color{white!15!black}},
ylabel={y(t)},
axis background/.style={fill=white},
title style={font=\small},
title={C7, R = -0.8022},
axis x line*=bottom,
axis y line*=left
]
\addplot[only marks, mark=*, mark options={}, mark size=1.5000pt, color=mycolor1, fill=mycolor1] table[row sep=crcr]{%
x	y\\
599.021904	-112.305\\
699.21093	-144.043\\
888.889353	-117.188\\
720.940576	-111.084\\
693.608496	-156.25\\
981.40625	-147.705\\
935.711175	-175.781\\
1100.740622	-134.277\\
801.499413	-68.359\\
404.275126	-80.566\\
477.998078	-80.566\\
482.429208	-100.098\\
595.683198	-81.787\\
458.252561	-34.18\\
180.23114	-24.414\\
125.170578	-23.193\\
127.816623	-73.242\\
433.153188	-128.174\\
769.813044	-133.057\\
808.853503	-137.939\\
818.392087	-92.773\\
530.012149	-56.152\\
335.171288	-125.732\\
743.579048	-86.67\\
498.3525	-68.359\\
404.275126	-114.746\\
680.788018	-100.098\\
588.376044	-85.449\\
516.282858	-142.822\\
873.499352	-133.057\\
796.745316	-102.539\\
627.128524	-150.146\\
926.550966	-145.264\\
883.059856	-113.525\\
679.7877	-92.773\\
543.557007	-72.021\\
423.339438	-86.67\\
517.33323	-112.305\\
668.327055	-103.76\\
619.34344	-109.863\\
657.859644	-112.305\\
676.52532	-122.07\\
757.68849	-175.781\\
1094.412506	-155.029\\
942.421291	-102.539\\
614.003532	-92.773\\
533.44475	-53.711\\
303.896838	-58.594\\
336.9155	-79.346\\
453.303698	-59.814\\
338.427612	-62.256\\
360.213216	-89.111\\
523.794458	-102.539\\
602.724242	-95.215\\
578.811985	-151.367\\
914.559414	-111.084\\
644.731536	-64.697\\
366.055626	-51.27\\
291.98265	-69.58\\
402.58988	-81.787\\
458.252561	-43.945\\
241.389885	-45.166\\
253.065098	-67.139\\
373.695674	-51.27\\
280.70325	-42.725\\
236.26925	-61.035\\
341.979105	-70.801\\
412.274223	-109.863\\
645.774714	-104.98\\
626.62562	-129.395\\
774.81726	-120.85\\
730.1757	-137.939\\
828.461634	-109.863\\
653.794713	-107.422\\
637.334726	-92.773\\
546.989608	-91.553\\
538.148534	-87.891\\
537.541356	-157.471\\
1009.231639	-224.609\\
1464.226071	-231.934\\
1511.977746	-225.83\\
1430.63305	-140.381\\
902.228687	-205.078\\
1355.56558	-252.686\\
1684.15219	-261.23\\
1722.02816	-200.195\\
1337.903185	-270.996\\
1870.685388	-335.693\\
2286.740716	-235.596\\
1574.488068	-191.65\\
1238.8256	-128.174\\
802.625588	-98.877\\
606.511518	-84.229\\
519.777159	-104.98\\
636.28378	-74.463\\
434.938383	-50.049\\
285.929937	-48.828\\
279.833268	-63.477\\
373.117806	-95.215\\
563.10151	-86.67\\
511.00632	-89.111\\
536.804664	-119.629\\
725.071369	-123.291\\
772.048242	-167.236\\
1041.211336	-134.277\\
845.810823	-169.678\\
1056.415228	-122.07\\
748.77738	-108.643\\
670.435953	-125.732\\
764.324828	-89.111\\
533.596668	-80.566\\
485.329584	-97.656\\
590.037552	-98.877\\
584.758578	-68.359\\
401.814202	-73.242\\
423.778212	-54.932\\
315.859	-61.035\\
352.04988	-64.697\\
367.220172	-45.166\\
255.549228	-58.594\\
336.9155	-72.021\\
429.893349	-117.188\\
708.049896	-122.07\\
744.26079	-136.719\\
808.556166	-76.904\\
460.501152	-109.863\\
685.984572	-173.34\\
1075.92138	-131.836\\
825.557032	-157.471\\
989.075351	-157.471\\
954.431731	-85.449\\
500.645691	-59.814\\
351.586692	-85.449\\
510.045081	-98.877\\
613.729539	-161.133\\
1032.701397	-202.637\\
1305.995465	-198.975\\
1260.506625	-145.264\\
915.017936	-150.146\\
943.067026	-135.498\\
846.049512	-131.836\\
825.557032	-126.953\\
774.032441	-85.449\\
510.045081	-76.904\\
454.810256	-69.58\\
406.41678	-62.256\\
364.757904	-70.801\\
423.956388	-107.422\\
670.742968	-164.795\\
1022.882565	-125.732\\
759.672744	-86.67\\
514.21311	-73.242\\
430.516476	-70.801\\
403.211695	-42.725\\
233.919375	-31.738\\
172.591244	-39.063\\
221.018454	-72.021\\
404.830041	-57.373\\
321.460919	-59.814\\
338.427612	-67.139\\
378.66396	-63.477\\
351.02781	-43.945\\
236.555935	-29.297\\
153.428389	-25.635\\
136.12185	-43.945\\
250.266775	-96.436\\
577.458768	-140.381\\
861.097054	-155.029\\
933.894696	-101.318\\
589.974714	-69.58\\
393.68364	-50.049\\
272.166462	-34.18\\
194.00568	-83.008\\
478.790144	-81.787\\
483.688318	-119.629\\
727.224691	-145.264\\
930.996976	-230.713\\
1533.549311	-262.451\\
1734.80111	-211.182\\
1384.29801	-202.637\\
1294.85043	-136.719\\
871.173468	-151.367\\
970.111103	-159.912\\
1030.63284	-172.119\\
1093.644126	-129.395\\
812.729995	-118.408\\
739.339552	-115.967\\
719.807169	-103.76\\
646.00976	-123.291\\
756.266994	-87.891\\
550.373442	-153.809\\
991.299005	-189.209\\
1195.233253	-133.057\\
811.248529	-78.125\\
467.8125	-85.449\\
533.543556	-146.484\\
936.03276	-181.885\\
1192.256175	-229.492\\
1521.072976	-241.699\\
1610.923835	-240.479\\
1580.668467	-180.664\\
1171.064048	-163.574\\
1072.22757	-187.988\\
1242.60068	-222.168\\
1427.873736	-139.16\\
858.75636	-81.787\\
507.651909	-122.07\\
757.68849	-102.539\\
614.003532	-65.918\\
395.903508	-87.891\\
532.707351	-98.877\\
592.075476	-76.904\\
452.041712	-58.594\\
346.524916	-80.566\\
473.566948	-65.918\\
392.278018	-91.553\\
561.586102	-133.057\\
821.094747	-122.07\\
730.95516	-79.346\\
473.616274	-83.008\\
506.099776	-126.953\\
818.212085	-209.961\\
1345.640049	-150.146\\
926.550966	-92.773\\
555.524724	-69.58\\
415.32302	-83.008\\
493.980608	-74.463\\
434.938383	-56.152\\
324.895472	-62.256\\
364.757904	-74.463\\
444.469647	-96.436\\
579.194616	-107.422\\
633.360112	-72.021\\
437.815659	-122.07\\
757.68849	-144.043\\
891.482127	-137.939\\
843.634924	-106.201\\
651.436934	-118.408\\
741.470896	-158.691\\
970.554156	-101.318\\
578.829734	-41.504\\
235.576704	-58.594\\
336.9155	-65.918\\
387.466004	-91.553\\
538.148534	-81.787\\
471.747416	-59.814\\
352.663344	-96.436\\
570.322504	-91.553\\
531.373612	-65.918\\
385.027038	-83.008\\
484.849728	-76.904\\
444.966544	-67.139\\
390.950397	-79.346\\
450.367896	-46.387\\
260.741327	-53.711\\
298.955426	-40.283\\
222.76499	-43.945\\
253.47476	-86.67\\
509.44626	-100.098\\
606.693978	-137.939\\
863.774018	-177.002\\
1088.916304	-115.967\\
679.450653	-68.359\\
389.304505	-48.828\\
275.38992	-48.828\\
280.761	-73.242\\
411.693282	-46.387\\
259.071395	-52.49\\
297.93324	-70.801\\
421.336751	-119.629\\
716.338452	-107.422\\
656.992952	-150.146\\
907.182132	-102.539\\
608.363887	-91.553\\
521.394335	-47.607\\
264.980562	-47.607\\
257.173014	-31.738\\
169.703086	-34.18\\
184.64036	-41.504\\
233.293984	-75.684\\
431.02038	-83.008\\
480.284288	-92.773\\
541.887093	-97.656\\
591.893016	-152.588\\
933.228208	-130.615\\
784.47369	-97.656\\
590.037552	-119.629\\
727.224691	-129.395\\
803.154765	-167.236\\
1035.023604	-133.057\\
799.140342	-84.229\\
478.083804	-43.945\\
238.97291	-31.738\\
170.27437	-34.18\\
183.99094	-41.504\\
224.204608	-43.945\\
236.555935	-40.283\\
219.824331	-56.152\\
318.718752	-87.891\\
502.121283	-79.346\\
453.303698	-81.787\\
474.691748	-96.436\\
543.89904	-58.594\\
313.302118	-29.297\\
160.401075	-64.697\\
370.778507	-97.656\\
565.037616	-96.436\\
565.018524	-109.863\\
641.709783	-93.994\\
536.987722	-69.58\\
403.84232	-97.656\\
586.521936	-137.939\\
838.531181	-139.16\\
828.14116	-97.656\\
599.021904	-162.354\\
995.879436	-128.174\\
779.169746	-118.408\\
726.314672	-137.939\\
846.117826	-137.939\\
836.048279	-103.76\\
619.34344	-90.332\\
557.438772	-153.809\\
985.761881	-214.844\\
1368.985968	-164.795\\
1007.88622	-92.773\\
557.194638	-92.773\\
557.194638	-91.553\\
556.550687	-113.525\\
700.562775	-136.719\\
831.114801	-100.098\\
606.693978	-104.98\\
647.83158	-140.381\\
874.012106	-153.809\\
937.773473	-103.76\\
617.47576	-79.346\\
479.408532	-104.98\\
659.37938	-175.781\\
1113.572635	-159.912\\
1010.164104	-162.354\\
992.957064	-95.215\\
568.338335	-84.229\\
504.363252	-83.008\\
478.790144	-52.49\\
290.2697	-34.18\\
182.11104	-28.076\\
145.9952	-23.193\\
126.123534	-59.814\\
341.717382	-86.67\\
504.67941	-104.98\\
603.635	-76.904\\
443.582272	-87.891\\
513.371331	-97.656\\
556.15092	-59.814\\
336.214494	-63.477\\
355.661631	-61.035\\
336.363885	-45.166\\
250.580968	-57.373\\
329.89475	-97.656\\
579.393048	-124.512\\
725.033376	-78.125\\
440.625	-56.152\\
313.60892	-46.387\\
259.071395	-59.814\\
327.48165	-41.504\\
237.112352	-91.553\\
558.198641	-158.691\\
964.682589	-122.07\\
755.49123	-167.236\\
1059.44006	-194.092\\
1243.935628	-197.754\\
1242.092874	-140.381\\
861.097054	-106.201\\
647.507497	-106.201\\
647.507497	-104.98\\
643.94732	-120.85\\
750.11595	-150.146\\
937.511624	-147.705\\
941.17626	-200.195\\
1301.2675	-220.947\\
1468.634709	-263.672\\
1728.36996	-178.223\\
1158.4495	-190.43\\
1255.31456	-212.402\\
1376.789764	-163.574\\
1066.338906	-189.209\\
1226.452738	-163.574\\
1015.303818	-91.553\\
541.444442	-58.594\\
341.192862	-62.256\\
370.485456	-95.215\\
563.10151	-75.684\\
435.183	-51.27\\
295.72536	-72.021\\
412.752351	-52.49\\
291.21452	-40.283\\
224.215178	-48.828\\
273.583284	-56.152\\
309.453672	-40.283\\
229.411685	-76.904\\
452.041712	-109.863\\
639.732249	-78.125\\
470.625	-134.277\\
848.227809	-195.313\\
1244.534436	-178.223\\
1152.033472	-222.168\\
1415.654496	-150.146\\
954.027684	-161.133\\
1006.114452	-111.084\\
661.060884	-67.139\\
399.544189	-85.449\\
502.269222	-67.139\\
386.04925	-48.828\\
282.518808	-72.021\\
403.533663	-41.504\\
222.66896	-25.635\\
136.58328	-36.621\\
207.201618	-79.346\\
466.395788	-104.98\\
630.50988	-128.174\\
774.427308	-124.512\\
750.060288	-117.188\\
714.495236	-136.719\\
826.056198	-111.084\\
661.060884	-90.332\\
537.565732	-90.332\\
530.971496	-73.242\\
438.573096	-115.967\\
688.032211	-78.125\\
452.03125	-67.139\\
397.060046	-100.098\\
608.495742	-137.939\\
833.427438	-103.76\\
611.76896	-76.904\\
447.811992	-64.697\\
375.501388	-74.463\\
425.407119	-51.27\\
288.18867	-50.049\\
285.029055	-72.021\\
418.009884	-86.67\\
509.44626	-98.877\\
575.760771	-79.346\\
466.395788	-101.318\\
597.370928	-92.773\\
528.342235	-52.49\\
296.0436	-54.932\\
323.879072	-112.305\\
688.87887	-158.691\\
982.138599	-153.809\\
943.464406	-129.395\\
791.37982	-122.07\\
735.34968	-92.773\\
552.092123	-89.111\\
523.794458	-70.801\\
420.062333	-102.539\\
610.209589	-90.332\\
521.034976	-54.932\\
319.869036	-84.229\\
499.730657	-101.318\\
606.692184	-118.408\\
719.802232	-129.395\\
786.592205	-129.395\\
807.94238	-175.781\\
1129.744487	-216.064\\
1420.188672	-244.141\\
1573.488745	-153.809\\
937.773473	-85.449\\
500.645691	-54.932\\
307.783996	-37.842\\
212.709882	-57.373\\
322.493633	-53.711\\
311.738644	-95.215\\
559.67377	-85.449\\
497.569527	-75.684\\
447.595176	-102.539\\
615.849234	-118.408\\
721.933576	-141.602\\
873.825942	-148.926\\
951.63714	-208.74\\
1345.3293	-181.885\\
1148.967545	-136.719\\
836.173404	-90.332\\
545.785944	-92.773\\
560.534466	-95.215\\
582.33494	-123.291\\
742.704984	-87.891\\
526.291308	-89.111\\
531.903559	-87.891\\
506.955288	-47.607\\
278.072487	-81.787\\
494.157054	-128.174\\
767.505912	-90.332\\
537.565732	-96.436\\
602.146384	-170.898\\
1064.010948	-129.395\\
796.03804	-122.07\\
768.91893	-177.002\\
1121.30767	-166.016\\
1024.484736	-102.539\\
606.415646	-64.697\\
377.895177	-65.918\\
398.276556	-114.746\\
726.91591	-190.43\\
1230.93952	-197.754\\
1281.841428	-206.299\\
1314.537228	-150.146\\
926.550966	-97.656\\
588.279744	-83.008\\
487.921024	-58.594\\
337.970192	-56.152\\
320.796376	-47.607\\
274.597176	-68.359\\
399.284919	-81.787\\
464.223012	-46.387\\
266.72525	-70.801\\
417.442696	-103.76\\
621.31488	-114.746\\
703.851964	-145.264\\
917.632688	-185.547\\
1206.0555	-223.389\\
1435.721103	-161.133\\
1017.877161	-137.939\\
871.360663	-144.043\\
901.997266	-120.85\\
734.64715	-84.229\\
513.544213	-103.76\\
651.71656	-167.236\\
1074.825772	-194.092\\
1211.910448	-115.967\\
688.032211	-62.256\\
354.54792	-41.504\\
221.133312	-23.193\\
121.461741	-31.738\\
170.845654	-51.27\\
281.62611	-61.035\\
344.2374	-80.566\\
454.39224	-67.139\\
371.27867	-51.27\\
282.54897	-50.049\\
274.919157	-48.828\\
266.454396	-45.166\\
247.28385	-52.49\\
296.98842	-81.787\\
482.216152	-123.291\\
742.704984	-133.057\\
803.930394	-125.732\\
771.240088	-152.588\\
955.506056	-181.885\\
1152.241475	-183.105\\
1183.59072	-223.389\\
1456.272891	-201.416\\
1283.422752	-146.484\\
903.952764	-101.318\\
615.912122	-96.436\\
607.450364	-163.574\\
1051.290098	-191.65\\
1210.65305	-131.836\\
803.804092	-85.449\\
491.33175	-45.166\\
247.28385	-31.738\\
171.448676	-35.4\\
184.08	-21.973\\
117.072144	-46.387\\
257.355076	-68.359\\
381.785015	-72.021\\
400.868886	-62.256\\
335.124048	-36.621\\
192.443355	-28.076\\
148.578192	-41.504\\
221.921888	-43.945\\
235.764925	-48.828\\
258.397776	-36.621\\
190.4292	-30.518\\
163.72907	-54.932\\
320.857812	-128.174\\
783.912184	-153.809\\
957.614834	-170.898\\
1051.364496	-124.512\\
777.452928	-172.119\\
1122.043761	-241.699\\
1588.687527	-222.168\\
1460.310264	-224.609\\
1455.915538	-164.795\\
1059.137465	-163.574\\
1054.23443	-172.119\\
1081.079439	-113.525\\
702.606225	-106.201\\
633.913769	-65.918\\
383.840514	-59.814\\
360.319536	-117.188\\
697.385788	-79.346\\
467.824016	-81.787\\
488.186603	-96.436\\
573.890636	-86.67\\
514.21311	-87.891\\
523.039341	-93.994\\
564.527964	-107.422\\
651.084742	-119.629\\
722.798418	-103.76\\
615.60808	-74.463\\
443.129313	-96.436\\
554.507	-48.828\\
260.155584	-20.752\\
109.425296	-34.18\\
186.52026	-58.594\\
310.079448	-30.518\\
152.559482	-13.428\\
67.60998	-32.959\\
174.419028	-58.594\\
318.634172	-69.58\\
389.85674	-83.008\\
465.093824	-72.021\\
419.378283	-117.188\\
688.831064	-104.98\\
599.75074	-70.801\\
413.548641	-107.422\\
607.793676	-59.814\\
327.48165	-46.387\\
247.149936	-32.959\\
168.980793	-21.973\\
114.2596	-41.504\\
214.326656	-34.18\\
181.4958	-53.711\\
299.975935	-91.553\\
518.006874	-87.891\\
489.201306	-59.814\\
331.848072	-68.359\\
374.265525	-52.49\\
291.21452	-74.463\\
407.684925	-53.711\\
290.146822	-50.049\\
269.413767	-46.387\\
245.480004	-36.621\\
195.116688	-45.166\\
240.644448	-41.504\\
228.728544	-75.684\\
421.257144	-73.242\\
400.99995	-54.932\\
297.73144	-51.27\\
274.14069	-40.283\\
216.118295	-48.828\\
277.147728	-108.643\\
648.490067	-144.043\\
846.684754	-96.436\\
561.546828	-90.332\\
512.724432	-63.477\\
369.626571	-111.084\\
654.951264	-101.318\\
578.829734	-67.139\\
384.773609	-83.008\\
469.659264	-64.697\\
370.778507	-89.111\\
499.288933	-52.49\\
288.32757	-54.932\\
306.79522	-65.918\\
374.150568	-84.229\\
471.935087	-62.256\\
346.516896	-65.918\\
370.525078	-68.359\\
396.755636	-112.305\\
655.973505	-95.215\\
571.86129	-146.484\\
917.282808	-190.43\\
1185.61718	-151.367\\
911.834808	-95.215\\
556.150815	-72.021\\
428.596971	-113.525\\
692.161925	-141.602\\
855.559284	-108.643\\
660.440797	-129.395\\
770.029645	-81.787\\
458.252561	-42.725\\
234.688425	-42.725\\
246.4378	-98.877\\
577.540557	-89.111\\
513.992248	-79.346\\
472.188046	-124.512\\
745.577856	-118.408\\
704.646008	-103.76\\
600.35536	-67.139\\
388.466254	-84.229\\
501.246779	-115.967\\
685.828838	-95.215\\
561.38764	-97.656\\
574.021968	-89.111\\
512.38825	-62.256\\
357.972	-75.684\\
428.220072	-54.932\\
310.805256	-62.256\\
363.637296	-106.201\\
633.913769	-123.291\\
740.485746	-131.836\\
791.807016	-117.188\\
688.831064	-83.008\\
474.224704	-58.594\\
333.69283	-67.139\\
387.257752	-80.566\\
473.566948	-104.98\\
630.50988	-130.615\\
798.84134	-155.029\\
936.685218	-114.746\\
670.231386	-69.58\\
416.64504	-126.953\\
794.979686	-178.223\\
1099.814133	-128.174\\
781.476878	-125.732\\
775.892172	-147.705\\
903.36378	-111.084\\
665.170992	-90.332\\
542.533992	-100.098\\
597.484962	-92.773\\
555.524724	-103.76\\
632.62472	-131.836\\
794.180064	-101.318\\
610.339632	-112.305\\
678.54681	-106.201\\
639.754824	-108.643\\
648.490067	-86.67\\
522.10008	-114.746\\
680.788018	-81.787\\
482.216152	-86.67\\
515.77317	-96.436\\
572.154788	-93.994\\
531.818052	-42.725\\
229.988675	-26.855\\
138.168975	-18.311\\
95.565109	-39.063\\
220.31532	-96.436\\
572.154788	-125.732\\
752.883216	-125.732\\
736.663788	-83.008\\
486.343872	-97.656\\
570.408696	-85.449\\
494.407914	-75.684\\
425.419764	-46.387\\
260.741327	-65.918\\
371.77752	-65.918\\
370.525078	-64.697\\
367.220172	-76.904\\
435.122832	-65.918\\
370.525078	-59.814\\
327.48165	-40.283\\
221.999613	-58.594\\
340.079576	-109.863\\
631.71225	-78.125\\
452.03125	-92.773\\
536.784578	-83.008\\
489.415168	-118.408\\
704.646008	-109.863\\
665.879643	-152.588\\
919.190112	-111.084\\
667.170504	-124.512\\
752.301504	-124.512\\
722.667648	-63.477\\
374.260392	-113.525\\
665.142975	-76.904\\
450.580536	-96.436\\
568.586656	-102.539\\
614.003532	-129.395\\
767.700535	-97.656\\
586.521936	-125.732\\
771.240088	-158.691\\
973.410594	-131.836\\
796.553112	-114.746\\
684.918874	-90.332\\
540.908016	-107.422\\
637.334726	-89.111\\
522.101349	-76.904\\
446.350816	-68.359\\
398.054457	-81.787\\
474.691748	-69.58\\
419.14992	-141.602\\
886.711724	-181.885\\
1135.68994	-157.471\\
980.414446	-150.146\\
934.808996	-147.705\\
884.45754	-81.787\\
477.717867	-72.021\\
415.417128	-54.932\\
311.794032	-50.049\\
283.177242	-54.932\\
319.869036	-93.994\\
561.050186	-118.408\\
717.670888	-134.277\\
811.301634	-114.746\\
674.476988	-76.904\\
464.653968	-140.381\\
874.012106	-163.574\\
1030.352626	-183.105\\
1146.60351	-150.146\\
970.543744	-236.816\\
1517.753744	-158.691\\
970.554156	-93.994\\
554.188624	-68.359\\
394.294712	-56.152\\
332.082928	-101.318\\
617.735846	-130.615\\
817.91113	-172.119\\
1068.342633	-129.395\\
786.592205	-102.539\\
598.930299	-53.711\\
305.884145	-53.711\\
307.817741	-64.697\\
373.172296	-70.801\\
399.31764	-43.945\\
247.014845	-61.035\\
341.979105	-45.166\\
243.128578	-29.297\\
162.539756	-65.918\\
372.964044	-75.684\\
439.269936	-96.436\\
561.546828	-89.111\\
518.893353	-93.994\\
547.327062	-93.994\\
555.880516	-120.85\\
705.88485	-79.346\\
469.252244	-119.629\\
725.071369	-136.719\\
818.673372	-103.76\\
619.34344	-108.643\\
644.578919	-93.994\\
561.050186	-108.643\\
664.460588	-151.367\\
920.159993	-117.188\\
701.721744	-93.994\\
562.836072	-102.539\\
610.209589	-93.994\\
547.327062	-62.256\\
368.181984	-100.098\\
610.297506	-145.264\\
891.049376	-139.16\\
858.75636	-151.367\\
975.560315	-227.051\\
1434.281167	-139.16\\
861.26124	-122.07\\
739.86627	-91.553\\
558.198641	-119.629\\
753.543071	-172.119\\
1065.244491	-114.746\\
697.540934	-102.539\\
636.459573	-142.822\\
868.214938	-91.553\\
533.113119	-53.711\\
310.771846	-70.801\\
407.10575	-52.49\\
295.04629	-42.725\\
237.80735	-46.387\\
256.52011	-40.283\\
220.549425	-39.063\\
216.721524	-53.711\\
303.896838	-76.904\\
444.966544	-90.332\\
535.939756	-119.629\\
714.065501	-112.305\\
678.54681	-135.498\\
836.158158	-157.471\\
971.753541	-141.602\\
855.559284	-96.436\\
579.194616	-106.201\\
635.931588	-95.215\\
570.14742	-103.76\\
636.46384	-147.705\\
900.557385	-109.863\\
667.857177	-122.07\\
742.06353	-106.201\\
639.754824	-102.539\\
634.613871	-158.691\\
979.282161	-130.615\\
803.54348	-129.395\\
793.70893	-122.07\\
724.24131	-73.242\\
417.11319	-47.607\\
263.26671	-37.842\\
214.791192	-69.58\\
396.2581	-62.256\\
359.092608	-86.67\\
496.70577	-65.918\\
385.027038	-93.994\\
571.389526	-152.588\\
963.898396	-189.209\\
1188.421729	-150.146\\
940.214252	-153.809\\
960.383396	-137.939\\
838.531181	-96.436\\
584.498596	-103.76\\
630.75704	-117.188\\
708.049896	-100.098\\
608.495742	-114.746\\
710.162994	-145.264\\
925.622208	-202.637\\
1298.700533	-181.885\\
1162.24515	-170.898\\
1098.361446	-191.65\\
1207.20335	-131.836\\
825.557032	-144.043\\
891.482127	-108.643\\
668.371736	-112.305\\
703.25391	-147.705\\
935.711175	-172.119\\
1043.213259	-79.346\\
470.759818	-89.111\\
525.398456	-67.139\\
399.544189	-98.877\\
595.635048	-100.098\\
584.672418	-61.035\\
357.604065	-81.787\\
500.209292	-139.16\\
902.03512	-234.375\\
1545	-227.051\\
1492.406223	-206.299\\
1337.230118	-166.016\\
1079.104	-194.092\\
1282.94812	-233.154\\
1519.930926	-170.898\\
1110.837	-177.002\\
1150.513	-184.326\\
1171.207404	-125.732\\
785.070608	-106.201\\
668.960099	-137.939\\
853.704471	-91.553\\
548.219364	-67.139\\
405.653838	-97.656\\
584.764128	-73.242\\
437.181498	-86.67\\
518.97996	-81.787\\
492.684888	-100.098\\
615.802896	-133.057\\
813.776612	-102.539\\
612.055291	-75.684\\
453.195792	-89.111\\
540.101771	-104.98\\
643.94732	-124.512\\
761.515392	-107.422\\
645.176532	-86.67\\
533.19384	-136.719\\
846.153891	-129.395\\
784.263095	-90.332\\
560.690724	-148.926\\
929.893944	-134.277\\
821.238132	-97.656\\
579.393048	-64.697\\
373.172296	-46.387\\
257.355076	-32.959\\
188.888029	-75.684\\
439.269936	-72.021\\
408.791196	-52.49\\
292.15934	-36.621\\
202.51413	-47.607\\
272.835717	-81.787\\
477.717867	-93.994\\
562.836072	-124.512\\
761.515392	-142.822\\
878.640944	-137.939\\
851.221569	-145.264\\
875.070336	-87.891\\
495.70524	-36.621\\
203.832486	-47.607\\
265.885095	-46.387\\
260.741327	-64.697\\
377.895177	-100.098\\
595.683198	-111.084\\
687.498876	-168.457\\
1027.082329	-106.201\\
637.843206	-104.98\\
651.61086	-155.029\\
948.157364	-106.201\\
632.002151	-74.463\\
429.502584	-53.711\\
310.771846	-76.904\\
456.271432	-100.098\\
615.802896	-151.367\\
917.435387	-98.877\\
586.637241	-85.449\\
506.968917	-80.566\\
480.898454	-101.318\\
617.735846	-133.057\\
847.839204	-207.52\\
1333.73104	-184.326\\
1177.84314	-180.664\\
1124.814064	-112.305\\
672.48234	-75.684\\
451.757796	-87.891\\
497.287278	-36.621\\
207.201618	-67.139\\
381.080964	-52.49\\
297.93324	-65.918\\
393.464542	-117.188\\
723.167148	-153.809\\
968.842891	-177.002\\
1153.876038	-234.375\\
1532.109375	-196.533\\
1227.152052	-106.201\\
647.507497	-98.877\\
595.635048	-79.346\\
480.916106	-108.643\\
674.347101	-141.602\\
904.83678	-191.65\\
1210.65305	-140.381\\
861.097054	-97.656\\
599.021904	-108.643\\
680.322466	-144.043\\
886.152536	-102.539\\
614.003532	-73.242\\
427.806522	-57.373\\
323.58372	-40.283\\
223.490084	-36.621\\
197.130843	-29.297\\
160.928421	-54.932\\
312.83774	-80.566\\
464.704688	-86.67\\
495.14571	-64.697\\
366.055626	-58.594\\
315.411502	-26.855\\
135.72517	-15.869\\
79.329131	-24.414\\
127.856118	-46.387\\
254.803791	-70.801\\
399.31764	-87.891\\
506.955288	-102.539\\
604.569944	-119.629\\
727.224691	-153.809\\
980.070948	-212.402\\
1380.613	-213.623\\
1396.453551	-230.713\\
1508.170881	-201.416\\
1261.266992	-111.084\\
679.389744	-102.539\\
627.128524	-101.318\\
627.057102	-135.498\\
836.158158	-113.525\\
685.91805	-80.566\\
473.566948	-61.035\\
350.95125	-50.049\\
295.088904	-93.994\\
557.666402	-90.332\\
516.066716	-47.607\\
262.362177	-37.842\\
211.34757	-58.594\\
335.802214	-78.125\\
444.921875	-59.814\\
345.007152	-86.67\\
498.3525	-67.139\\
390.950397	-98.877\\
601.073283	-141.602\\
860.798558	-112.305\\
693.034155	-153.809\\
982.83951	-206.299\\
1344.863181	-224.609\\
1455.915538	-175.781\\
1097.576564	-111.084\\
671.169528	-83.008\\
481.778432	-48.828\\
286.083252	-81.787\\
474.691748	-57.373\\
341.426723	-109.863\\
655.772247	-92.773\\
543.557007	-74.463\\
441.788979	-96.436\\
557.978696	-52.49\\
305.64927	-84.229\\
484.31675	-58.594\\
325.079512	-37.842\\
209.26626	-45.166\\
246.470862	-32.959\\
178.63778	-40.283\\
217.608766	-35.4\\
187.3368	-26.855\\
142.11666	-36.621\\
197.130843	-48.828\\
264.64776	-50.049\\
271.26558	-48.828\\
273.583284	-79.346\\
460.524184	-103.76\\
600.35536	-84.229\\
482.716399	-78.125\\
464.921875	-128.174\\
783.912184	-153.809\\
932.236349	-119.629\\
725.071369	-125.732\\
725.222176	-56.152\\
307.4322	-36.621\\
205.846641	-73.242\\
430.516476	-106.201\\
630.090533	-115.967\\
713.428984	-162.354\\
1004.808906	-146.484\\
885.056328	-102.539\\
604.569944	-73.242\\
429.124878	-79.346\\
466.395788	-84.229\\
488.865116	-64.697\\
361.332745	-39.063\\
216.721524	-52.49\\
289.27239	-39.063\\
210.276129	-32.959\\
173.199545	-24.414\\
131.884428	-53.711\\
299.975935	-74.463\\
422.651988	-79.346\\
440.211608	-56.152\\
321.807112	-100.098\\
599.386824	-136.719\\
803.634282	-86.67\\
515.77317	-120.85\\
732.47185	-130.615\\
777.289865	-98.877\\
566.664087	-56.152\\
319.78564	-72.021\\
416.713506	-87.891\\
494.035311	-47.607\\
264.980562	-50.049\\
274.919157	-41.504\\
220.38624	-24.414\\
127.416666	-26.855\\
143.08344	-46.387\\
253.968825	-64.697\\
354.216075	-54.932\\
304.762736	-72.021\\
404.830041	-79.346\\
438.78338	-57.373\\
305.683344	-31.738\\
165.0376	-25.635\\
133.302	-34.18\\
180.23114	-39.063\\
207.42453	-48.828\\
277.147728	-104.98\\
611.29854	-101.318\\
584.402224	-89.111\\
513.992248	-85.449\\
500.645691	-117.188\\
708.049896	-150.146\\
926.550966	-159.912\\
992.573784	-166.016\\
1012.199552	-120.85\\
734.64715	-126.953\\
776.444548	-131.836\\
808.682024	-133.057\\
821.094747	-151.367\\
961.785918	-200.195\\
1272.03903	-169.678\\
1068.801722	-156.25\\
972.8125	-128.174\\
790.961754	-119.629\\
733.804286	-114.746\\
710.162994	-136.719\\
828.653859	-83.008\\
498.546048	-98.877\\
601.073283	-119.629\\
738.230559	-140.381\\
855.902957	-107.422\\
656.992952	-124.512\\
756.908448	-98.877\\
593.855262	-90.332\\
526.003236	-54.932\\
321.846588	-85.449\\
520.982553	-137.939\\
838.531181	-109.863\\
643.687317	-64.697\\
377.895177	-80.566\\
454.39224	-41.504\\
227.981472	-40.283\\
224.215178	-56.152\\
306.421464	-32.959\\
177.418297	-36.621\\
198.48582	-43.945\\
232.55694	-26.855\\
144.077075	-50.049\\
269.413767	-40.283\\
210.236977	-23.193\\
123.15483	-43.945\\
236.555935	-51.27\\
280.70325	-63.477\\
348.679161	-62.256\\
343.092816	-63.477\\
358.01028	-84.229\\
475.05156	-76.904\\
437.96828	-86.67\\
518.97996	-146.484\\
874.362996	-101.318\\
595.547204	-92.773\\
546.989608	-100.098\\
579.167028	-69.58\\
386.02984	-40.283\\
228.646308	-79.346\\
447.51144	-61.035\\
333.067995	-37.842\\
210.628572	-67.139\\
376.179817	-62.256\\
352.244448	-80.566\\
445.52998	-48.828\\
261.083316	-29.297\\
152.3444	-23.193\\
119.768652	-29.297\\
157.178405	-56.152\\
304.34384	-54.932\\
300.7527	-65.918\\
371.77752	-87.891\\
510.119364	-117.188\\
678.049768	-84.229\\
495.098062	-124.512\\
752.301504	-141.602\\
853.010448	-124.512\\
747.819072	-117.188\\
731.721872	-195.313\\
1216.018738	-135.498\\
848.488476	-173.34\\
1082.33496	-141.602\\
886.711724	-164.795\\
1050.07374	-184.326\\
1130.655684	-101.318\\
595.547204	-63.477\\
361.501515	-51.27\\
298.54521	-84.229\\
501.246779	-104.98\\
626.62562	-108.643\\
636.539337	-74.463\\
417.216189	-40.283\\
224.215178	-52.49\\
307.53891	-104.98\\
640.06306	-145.264\\
893.664128	-142.822\\
847.362926	-84.229\\
487.348994	-64.697\\
379.059723	-89.111\\
541.705769	-141.602\\
878.923614	-161.133\\
1006.114452	-164.795\\
1010.85253	-115.967\\
724.097948	-166.016\\
1054.865664	-178.223\\
1125.834691	-170.898\\
1114.084062	-240.479\\
1572.011223	-203.857\\
1313.858365	-169.678\\
1102.907	-196.533\\
1270.389312	-164.795\\
1016.949945	-91.553\\
549.867318	-84.229\\
512.028091	-104.98\\
647.83158	-125.732\\
778.155348	-128.174\\
781.476878	-96.436\\
593.274272	-124.512\\
761.515392	-106.201\\
637.843206	-78.125\\
473.515625	-106.201\\
643.684261	-95.215\\
590.999505	-150.146\\
940.214252	-151.367\\
934.085757	-108.643\\
654.465432	-79.346\\
457.667728	-45.166\\
263.001618	-78.125\\
449.21875	-53.711\\
303.896838	-52.49\\
294.10147	-42.725\\
244.856975	-76.904\\
453.425984	-98.877\\
590.196813	-112.305\\
672.48234	-109.863\\
639.732249	-64.697\\
364.89108	-48.828\\
269.091108	-35.4\\
193.815	-43.945\\
247.014845	-70.801\\
404.486113	-75.684\\
432.382692	-74.463\\
441.788979	-120.85\\
736.82245	-140.381\\
871.344867	-162.354\\
1013.738376	-159.912\\
1013.04252	-187.988\\
1156.502176	-109.863\\
661.814712	-91.553\\
543.183949	-72.021\\
424.635816	-78.125\\
467.8125	-104.98\\
624.73598	-86.67\\
503.03268	-57.373\\
327.771949	-50.049\\
294.188022	-97.656\\
590.037552	-117.188\\
703.831128	-107.422\\
666.768354	-168.457\\
1067.175095	-184.326\\
1150.931544	-130.615\\
789.17583	-91.553\\
536.409027	-62.256\\
357.972	-57.373\\
341.426723	-107.422\\
639.268322	-85.449\\
511.668612	-103.76\\
625.05024	-113.525\\
690.118475	-129.395\\
788.921315	-122.07\\
748.77738	-135.498\\
816.239952	-91.553\\
551.515272	-109.863\\
649.729782	-75.684\\
442.070244	-68.359\\
408.034871	-107.422\\
660.860144	-147.705\\
922.27002	-162.354\\
984.027594	-91.553\\
576.692347	-190.43\\
1220.46587	-170.898\\
1064.010948	-113.525\\
673.543825	-65.918\\
392.278018	-96.436\\
575.626484	-83.008\\
490.909312	-80.566\\
482.429208	-98.877\\
581.199006	-64.697\\
383.847301	-93.994\\
586.898536	-170.898\\
1088.962056	-179.443\\
1113.802701	-117.188\\
727.385916	-134.277\\
838.425588	-146.484\\
917.282808	-148.926\\
919.022346	-106.201\\
647.507497	-106.201\\
632.002151	-69.58\\
419.14992	-106.201\\
645.595879	-109.863\\
692.027037	-175.781\\
1132.908545	-195.313\\
1294.534564	-263.672\\
1733.116056	-190.43\\
1223.89361	-155.029\\
979.318193	-118.408\\
745.851992	-130.615\\
808.376235	-97.656\\
586.521936	-67.139\\
395.851544	-68.359\\
404.275126	-68.359\\
394.294712	-48.828\\
273.583284	-35.4\\
200.2932	-59.814\\
346.083804	-81.787\\
467.249131	-54.932\\
303.77396	-39.063\\
213.869925	-37.842\\
218.272656	-90.332\\
544.159968	-128.174\\
741.614764	-59.814\\
345.007152	-74.463\\
432.183252	-79.346\\
457.667728	-73.242\\
427.806522	-91.553\\
533.113119	-79.346\\
451.87547	-54.932\\
306.79522	-41.504\\
226.487328	-34.18\\
187.1355	-43.945\\
251.057785	-86.67\\
509.44626	-103.76\\
600.35536	-72.021\\
404.830041	-48.828\\
265.526664	-29.297\\
159.317086	-42.725\\
241.73805	-80.566\\
469.135818	-101.318\\
597.370928	-107.422\\
621.543692	-76.904\\
440.736824	-70.801\\
407.10575	-78.125\\
459.21875	-108.643\\
662.396371	-151.367\\
945.135548	-175.781\\
1087.908609	-131.836\\
786.929084	-83.008\\
490.909312	-86.67\\
512.56638	-81.787\\
482.216152	-84.229\\
484.31675	-54.932\\
316.847776	-69.58\\
402.58988	-76.904\\
446.350816	-76.904\\
433.73856	-45.166\\
245.612708	-29.297\\
160.401075	-47.607\\
269.360406	-70.801\\
413.548641	-109.863\\
657.859644	-118.408\\
721.933576	-148.926\\
910.831416	-130.615\\
803.54348	-146.484\\
930.759336	-195.313\\
1287.503296	-260.01\\
1718.6661	-212.402\\
1361.284418	-142.822\\
915.346198	-155.029\\
999.161905	-180.664\\
1190.937088	-236.816\\
1539.304	-155.029\\
948.157364	-76.904\\
446.350816	-50.049\\
285.929937	-52.49\\
292.15934	-37.842\\
200.94102	-23.193\\
122.296689	-30.518\\
164.858236	-48.828\\
270.897744	-68.359\\
384.245939	-70.801\\
394.078366	-54.932\\
301.741476	-48.828\\
271.776648	-67.139\\
381.080964	-81.787\\
468.721297	-85.449\\
489.708219	-79.346\\
444.575638	-57.373\\
314.117175	-37.842\\
204.422484	-37.842\\
208.547262	-57.373\\
321.460919	-73.242\\
405.02826	-53.711\\
289.126313	-35.4\\
194.4522	-58.594\\
317.57948	-39.063\\
211.72146	-50.049\\
276.77097	-63.477\\
361.501515	-92.773\\
540.217179	-100.098\\
568.156248	-68.359\\
399.284919	-106.201\\
624.249478	-102.539\\
595.136356	-83.008\\
475.718848	-70.801\\
418.717114	-119.629\\
727.224691	-146.484\\
898.532856	-144.043\\
888.889353	-158.691\\
970.554156	-120.85\\
732.47185	-111.084\\
669.170016	-102.539\\
627.128524	-139.16\\
845.95364	-108.643\\
668.371736	-147.705\\
914.146245	-141.602\\
876.374778	-146.484\\
957.565908	-249.023\\
1627.863351	-190.43\\
1185.61718	-102.539\\
610.209589	-69.58\\
410.24368	-73.242\\
437.181498	-91.553\\
556.550687	-119.629\\
740.383881	-139.16\\
871.41992	-153.809\\
966.074329	-157.471\\
963.092636	-96.436\\
579.194616	-86.67\\
525.30687	-101.318\\
614.088398	-102.539\\
602.724242	-61.035\\
347.594325	-43.945\\
255.05678	-79.346\\
464.888214	-80.566\\
480.898454	-109.863\\
673.899642	-139.16\\
853.60744	-123.291\\
740.485746	-87.891\\
516.623298	-70.801\\
423.956388	-107.422\\
658.926548	-139.16\\
879.07372	-185.547\\
1192.510569	-198.975\\
1300.699575	-231.934\\
1507.571	-184.326\\
1187.98107	-168.457\\
1036.347464	-89.111\\
528.695563	-67.139\\
411.830626	-126.953\\
801.962101	-163.574\\
1003.362916	-93.994\\
586.898536	-151.367\\
978.436288	-203.857\\
1347.49477	-246.582\\
1589.22099	-153.809\\
946.232968	-86.67\\
507.79953	-50.049\\
293.237091	-75.684\\
444.870552	-69.58\\
408.99124	-75.684\\
431.02038	-42.725\\
242.5071	-61.035\\
343.077735	-45.166\\
254.73624	-58.594\\
330.47016	-48.828\\
277.147728	-62.256\\
365.940768	-101.318\\
597.370928	-91.553\\
551.515272	-129.395\\
803.154765	-156.25\\
972.8125	-155.029\\
933.894696	-85.449\\
506.968917	-86.67\\
520.54002	-106.201\\
626.161096	-70.801\\
412.274223	-63.477\\
360.295452	-45.166\\
259.7045	-73.242\\
423.778212	-69.58\\
389.85674	-39.063\\
208.869861	-21.973\\
114.677087	-25.635\\
137.531775	-46.387\\
258.190042	-72.021\\
406.19844	-75.684\\
417.094524	-45.166\\
249.76798	-57.373\\
324.616434	-84.229\\
493.497711	-109.863\\
635.667318	-76.904\\
437.96828	-72.021\\
415.417128	-84.229\\
493.497711	-101.318\\
593.622162	-92.773\\
546.989608	-109.863\\
647.752248	-103.76\\
600.35536	-70.801\\
400.592058	-56.152\\
321.807112	-72.021\\
408.791196	-59.814\\
341.717382	-75.684\\
443.432556	-102.539\\
627.128524	-157.471\\
954.431731	-109.863\\
659.837178	-107.422\\
664.834758	-166.016\\
1021.330432	-125.732\\
748.231132	-75.684\\
436.545312	-54.932\\
307.783996	-43.945\\
245.432825	-46.387\\
254.803791	-39.063\\
212.424594	-40.283\\
230.136779	-84.229\\
458.037302	-26.855\\
142.11666	-37.842\\
205.10364	-53.711\\
300.942733	-72.021\\
400.868886	-56.152\\
305.354576	-40.283\\
211.687165	-24.414\\
129.63834	-42.725\\
239.388175	-84.229\\
491.981589	-111.084\\
644.731536	-87.891\\
502.121283	-73.242\\
421.1415	-83.008\\
477.296	-84.229\\
488.865116	-96.436\\
570.322504	-117.188\\
708.049896	-139.16\\
845.95364	-133.057\\
801.535368	-114.746\\
672.296814	-74.463\\
425.407119	-58.594\\
333.69283	-61.035\\
354.24714	-91.553\\
519.654828	-57.373\\
319.338118	-50.049\\
288.682632	-92.773\\
550.422209	-113.525\\
671.38685	-108.643\\
652.509858	-124.512\\
768.363552	-164.795\\
980.695045	-85.449\\
522.606084	-145.264\\
907.028416	-152.588\\
947.113716	-140.381\\
874.012106	-152.588\\
952.759472	-148.926\\
973.529262	-253.906\\
1692.28349	-235.596\\
1527.133272	-158.691\\
1031.4915	-189.209\\
1254.077252	-233.154\\
1549.774638	-223.389\\
1497.153078	-253.906\\
1701.678012	-227.051\\
1554.845248	-311.279\\
2120.432548	-234.375\\
1549.21875	-159.912\\
1013.04252	-98.877\\
613.729539	-100.098\\
614.001132	-79.346\\
473.616274	-64.697\\
393.293063	-100.098\\
626.813676	-142.822\\
873.499352	-87.891\\
527.873346	-76.904\\
460.501152	-73.242\\
426.488166	-48.828\\
282.518808	-63.477\\
352.170396	-34.18\\
185.87084	-40.283\\
217.608766	-35.4\\
193.815	-48.828\\
265.526664	-37.842\\
201.622176	-31.738\\
165.0376	-17.09\\
86.37286	-18.311\\
94.887602	-40.283\\
207.256035	-28.076\\
143.440284	-29.297\\
158.262394	-65.918\\
372.964044	-93.994\\
540.4655	-93.994\\
533.509944	-68.359\\
390.534967	-86.67\\
518.97996	-135.498\\
781.552464	-70.801\\
388.909893	-37.842\\
203.02233	-29.297\\
152.3444	-24.414\\
128.29557	-40.283\\
219.824331	-67.139\\
383.565107	-100.098\\
562.650858	-62.256\\
360.213216	-100.098\\
606.693978	-144.043\\
878.230171	-145.264\\
872.455584	-103.76\\
600.35536	-62.256\\
361.333824	-83.008\\
492.486464	-113.525\\
692.161925	-148.926\\
878.067696	-83.008\\
463.59968	-42.725\\
241.73805	-67.139\\
376.179817	-53.711\\
285.20541	-24.414\\
123.388356	-15.869\\
82.5188	-42.725\\
241.73805	-92.773\\
538.454492	-98.877\\
564.884301	-70.801\\
388.909893	-43.945\\
238.97291	-47.607\\
240.605778	-20.752\\
104.112784	-28.076\\
141.896104	-29.297\\
151.817054	-46.387\\
251.41754	-69.58\\
394.93608	-103.76\\
604.19448	-107.422\\
631.426516	-120.85\\
717.00305	-124.512\\
740.970912	-120.85\\
719.17835	-115.967\\
685.828838	-101.318\\
606.692184	-123.291\\
769.829004	-187.988\\
1187.520196	-169.678\\
1025.194476	-87.891\\
497.287278	-45.166\\
265.485748	-103.76\\
630.75704	-122.07\\
739.86627	-114.746\\
665.985784	-63.477\\
362.644101	-67.139\\
398.335687	-113.525\\
706.80665	-175.781\\
1113.572635	-180.664\\
1161.127528	-203.857\\
1317.731648	-195.313\\
1233.792221	-139.16\\
874.06396	-144.043\\
851.870302	-59.814\\
341.717382	-61.035\\
354.24714	-76.904\\
454.810256	-98.877\\
590.196813	-104.98\\
632.39952	-111.084\\
650.841156	-70.801\\
420.062333	-98.877\\
584.758578	-86.67\\
495.14571	-47.607\\
260.648325	-32.959\\
174.419028	-26.855\\
144.560465	-42.725\\
227.6388	-35.4\\
181.4958	-19.531\\
102.635405	-40.283\\
224.980555	-85.449\\
474.071052	-56.152\\
306.421464	-51.27\\
286.34295	-72.021\\
420.674661	-119.629\\
694.326716	-81.787\\
447.783825	-42.725\\
222.17	-21.973\\
119.489174	-57.373\\
337.238494	-126.953\\
778.729702	-161.133\\
1035.601791	-223.389\\
1488.887685	-266.113\\
1793.069394	-261.23\\
1717.06479	-168.457\\
1091.938274	-170.898\\
1135.959006	-222.168\\
1464.531456	-195.313\\
1280.276715	-179.443\\
1182.888256	-202.637\\
1346.928139	-220.947\\
1472.611755	-223.389\\
1529.767872	-301.514\\
2031.601332	-203.857\\
1313.858365	-109.863\\
661.814712	-59.814\\
343.9305	-47.607\\
271.978791	-54.932\\
313.826516	-53.711\\
305.884145	-56.152\\
333.149816	-101.318\\
599.194652	-79.346\\
467.824016	-84.229\\
505.879374	-111.084\\
648.841644	-63.477\\
368.420508	-70.801\\
420.062333	-101.318\\
612.163356	-115.967\\
681.654026	-64.697\\
366.055626	-46.387\\
263.292612	-58.594\\
332.579544	-54.932\\
332.942852	-129.395\\
826.83405	-186.768\\
1196.996112	-173.34\\
1120.46976	-198.975\\
1311.6432	-225.83\\
1542.4189	-297.852\\
2045.051832	-257.568\\
1702.52448	-157.471\\
997.578785	-100.098\\
595.683198	-54.932\\
319.869036	-61.035\\
356.505435	-67.139\\
397.060046	-86.67\\
507.79953	-62.256\\
357.972	-56.152\\
324.895472	-63.477\\
370.769157	-78.125\\
462.03125	-89.111\\
536.804664	-111.084\\
661.060884	-79.346\\
448.939668	-40.283\\
218.33386	-28.076\\
148.044748	-23.193\\
120.6036	-24.414\\
127.856118	-34.18\\
185.2556	-57.373\\
316.182603	-59.814\\
335.137842	-75.684\\
444.870552	-114.746\\
674.476988	-81.787\\
497.183173	-150.146\\
964.988342	-208.74\\
1341.57198	-162.354\\
1028.51259	-144.043\\
920.43477	-170.898\\
1107.760836	-198.975\\
1264.28715	-131.836\\
820.810936	-111.084\\
667.170504	-72.021\\
425.932194	-79.346\\
491.072394	-142.822\\
944.05342	-247.803\\
1696.954944	-300.293\\
2045.595916	-239.258\\
1607.81376	-200.195\\
1315.881735	-145.264\\
946.976016	-153.809\\
1025.136985	-208.74\\
1345.3293	-115.967\\
724.097948	-98.877\\
608.291304	-75.684\\
479.45814	-148.926\\
957.147402	-136.719\\
836.173404	-76.904\\
463.269696	-72.021\\
424.635816	-53.711\\
322.588266	-95.215\\
596.23633	-145.264\\
917.632688	-148.926\\
932.574612	-113.525\\
694.3189	-83.008\\
506.099776	-93.994\\
567.911748	-74.463\\
437.693514	-53.711\\
307.817741	-43.945\\
246.223835	-39.063\\
213.869925	-29.297\\
159.317086	-35.4\\
198.9834	-65.918\\
386.213562	-95.215\\
561.38764	-96.436\\
552.674716	-58.594\\
330.47016	-53.711\\
298.955426	-41.504\\
233.293984	-54.932\\
315.859	-78.125\\
453.4375	-86.67\\
504.67941	-80.566\\
452.861486	-46.387\\
271.781433	-96.436\\
589.802576	-142.822\\
862.930524	-101.318\\
573.257244	-40.283\\
223.490084	-50.049\\
286.830819	-76.904\\
444.966544	-73.242\\
419.749902	-67.139\\
374.971315	-41.504\\
236.36528	-75.684\\
449.033172	-117.188\\
682.385724	-70.801\\
385.015838	-25.635\\
137.531775	-39.063\\
223.870053	-91.553\\
544.831903	-111.084\\
644.731536	-65.918\\
374.150568	-59.814\\
353.739996	-113.525\\
696.36235	-151.367\\
931.209784	-131.836\\
830.434964	-179.443\\
1166.3795	-219.727\\
1404.05553	-140.381\\
874.012106	-111.084\\
703.71714	-159.912\\
998.490528	-109.863\\
687.962106	-142.822\\
915.346198	-174.561\\
1105.843935	-137.939\\
830.944536	-69.58\\
424.22926	-111.084\\
715.93638	-202.637\\
1343.078036	-233.154\\
1524.127698	-169.678\\
1090.520506	-142.822\\
933.627414	-191.65\\
1242.2753	-153.809\\
960.383396	-96.436\\
589.802576	-85.449\\
527.305779	-107.422\\
666.768354	-113.525\\
704.649675	-109.863\\
685.984572	-125.732\\
771.240088	-87.891\\
535.871427	-90.332\\
562.407032	-130.615\\
794.008585	-78.125\\
462.03125	-62.256\\
360.213216	-50.049\\
284.078124	-40.283\\
226.430743	-46.387\\
270.111501	-85.449\\
502.269222	-79.346\\
476.552076	-107.422\\
660.860144	-139.16\\
879.07372	-164.795\\
1028.97998	-123.291\\
758.486232	-107.422\\
619.610096	-41.504\\
238.648	-67.139\\
408.137981	-129.395\\
779.47548	-85.449\\
491.33175	-41.504\\
243.960512	-87.891\\
540.705432	-137.939\\
858.808214	-142.822\\
912.63258	-185.547\\
1229.805516	-245.361\\
1608.341355	-173.34\\
1114.05618	-147.705\\
957.42381	-170.898\\
1082.63883	-113.525\\
702.606225	-90.332\\
559.064748	-102.539\\
642.099218	-131.836\\
835.18106	-141.602\\
897.04867	-142.822\\
876.070148	-79.346\\
475.123848	-70.801\\
417.442696	-54.932\\
325.911556	-79.346\\
472.188046	-75.684\\
443.432556	-59.814\\
363.609306	-119.629\\
742.537203	-131.836\\
803.804092	-89.111\\
533.596668	-74.463\\
452.660577	-108.643\\
646.534493	-63.477\\
360.295452	-35.4\\
198.9834	-47.607\\
271.978791	-64.697\\
369.613961	-56.152\\
314.619656	-41.504\\
222.66896	-23.193\\
124.430445	-42.725\\
238.619125	-64.697\\
377.895177	-102.539\\
615.849234	-118.408\\
730.695768	-152.588\\
963.898396	-173.34\\
1053.73386	-91.553\\
538.148534	-69.58\\
425.55128	-136.719\\
883.751616	-211.182\\
1361.06799	-164.795\\
1050.07374	-151.367\\
994.935291	-230.713\\
1516.476549	-185.547\\
1199.375808	-148.926\\
940.765542	-112.305\\
703.25391	-114.746\\
731.161512	-155.029\\
973.737149	-107.422\\
660.860144	-90.332\\
572.25322	-156.25\\
1018.59375	-196.533\\
1291.811409	-205.078\\
1280.507032	-92.773\\
540.217179	-43.945\\
266.350645	-111.084\\
677.279148	-87.891\\
505.37325	-37.842\\
207.866106	-29.297\\
163.067102	-53.711\\
309.805048	-81.787\\
498.655339	-135.498\\
821.253378	-96.436\\
570.322504	-70.801\\
401.866476	-42.725\\
231.5695	-28.076\\
152.677288	-39.063\\
211.018326	-35.4\\
192.5052	-45.166\\
256.362216	-79.346\\
454.731926	-72.021\\
394.314975	-32.959\\
176.825035	-30.518\\
165.40756	-41.504\\
228.728544	-53.711\\
289.126313	-32.959\\
179.857263	-51.27\\
276.96054	-37.842\\
198.85971	-25.635\\
140.813055	-63.477\\
364.99275	-96.436\\
586.234444	-145.264\\
920.24744	-190.43\\
1213.41996	-161.133\\
976.627113	-80.566\\
469.135818	-57.373\\
329.89475	-58.594\\
324.02482	-32.959\\
176.825035	-30.518\\
169.313864	-64.697\\
367.220172	-73.242\\
414.403236	-67.139\\
383.565107	-73.242\\
422.459856	-89.111\\
518.893353	-93.994\\
552.496732	-97.656\\
584.764128	-122.07\\
735.34968	-118.408\\
737.208208	-170.898\\
1035.812778	-89.111\\
504.190038	-41.504\\
229.51712	-40.283\\
231.62725	-76.904\\
440.736824	-63.477\\
359.152866	-57.373\\
306.773431	-26.855\\
143.08344	-45.166\\
238.160318	-28.076\\
141.36266	-13.428\\
67.865112	-25.635\\
136.12185	-52.49\\
287.38275	-64.697\\
370.778507	-96.436\\
577.458768	-136.719\\
813.614769	-98.877\\
554.007831	-43.945\\
249.43182	-80.566\\
467.605064	-90.332\\
504.50422	-42.725\\
237.80735	-57.373\\
335.115693	-112.305\\
664.17177	-97.656\\
599.021904	-162.354\\
1037.44206	-189.209\\
1171.014501	-117.188\\
710.276468	-92.773\\
555.524724	-80.566\\
};
\addplot [color=mycolor2, line width=2.0pt, forget plot]
  table[row sep=crcr]{%
599.021904	-92.3995941487465\\
699.21093	-107.853829258917\\
888.889353	-137.111873392099\\
720.940576	-111.205644038959\\
693.608496	-106.98964946117\\
981.40625	-151.382676642561\\
935.711175	-144.334175817462\\
1100.740622	-169.7900962497\\
801.499413	-123.631907242675\\
404.275126	-62.3597523185619\\
477.998078	-73.7315749493543\\
482.429208	-74.415080194966\\
595.683198	-91.8845961539787\\
458.252561	-70.6858136092862\\
180.23114	-27.8007933896286\\
125.170578	-19.3076589175344\\
127.816623	-19.7158134148352\\
433.153188	-66.8142158211225\\
769.813044	-118.744260203232\\
808.853503	-124.766281339508\\
818.392087	-126.237615333254\\
530.012149	-81.7547858174899\\
335.171288	-51.7004316114501\\
743.579048	-114.697645935684\\
498.3525	-76.8712603588085\\
404.275126	-62.3597523185619\\
680.788018	-105.0120807678\\
588.376044	-90.757461971616\\
516.282858	-79.6370320126974\\
873.499352	-134.737953779388\\
796.745316	-122.898584086359\\
627.128524	-96.7350621233751\\
926.550966	-142.921206461474\\
883.059856	-136.212668950167\\
679.7877	-104.857780939025\\
543.557007	-83.8440906887228\\
423.339438	-65.3004372580647\\
517.33323	-79.7990527098658\\
668.327055	-103.089967542534\\
619.34344	-95.5342068671475\\
657.859644	-101.475361262313\\
676.52532	-104.354556289065\\
757.68849	-116.874038327614\\
1094.412506	-168.813979439578\\
942.421291	-145.369216424409\\
614.003532	-94.7105219088898\\
533.44475	-82.2842671889667\\
303.896838	-46.8763233978292\\
336.9155	-51.9694776677515\\
453.303698	-69.9224476461314\\
338.427612	-52.2027221187048\\
360.213216	-55.5631684637274\\
523.794458	-80.7957021494204\\
602.724242	-92.9706826620665\\
578.811985	-89.2821984393251\\
914.559414	-141.071500247703\\
644.731536	-99.450340402407\\
366.055626	-56.4643647459432\\
291.98265	-45.0385506411724\\
402.58988	-62.0998018135787\\
458.252561	-70.6858136092862\\
241.389885	-37.2345773279312\\
253.065098	-39.0354880051477\\
373.695674	-57.6428480864738\\
280.70325	-43.2986944267636\\
236.26925	-36.4447154003049\\
341.979105	-52.750542673564\\
412.274223	-63.5936192463337\\
645.774714	-99.6112514194854\\
626.62562	-96.6574887906047\\
774.81726	-119.516164409647\\
730.1757	-112.630169091908\\
828.461634	-127.790850782324\\
653.794713	-100.848342496999\\
637.334726	-98.3093767123791\\
546.989608	-84.3735720601996\\
538.148534	-83.0098295259382\\
537.541356	-82.9161718476439\\
1009.231639	-155.674764516915\\
1464.226071	-225.858009196294\\
1511.977746	-233.223742169429\\
1430.63305	-220.676259604329\\
902.228687	-139.169476026634\\
1355.56558	-209.097044027309\\
1684.15219	-259.78178394078\\
1722.02816	-265.624181743966\\
1337.903185	-206.372605874384\\
1870.685388	-288.554674673784\\
2286.740716	-352.731532304391\\
1574.488068	-242.866007910195\\
1238.8256	-191.089811402084\\
802.625588	-123.805620611494\\
606.511518	-93.5548729278858\\
519.777159	-80.1760306571168\\
636.28378	-98.1472674752647\\
434.938383	-67.0895835370188\\
285.929937	-44.1049149577954\\
279.833268	-43.1644990272633\\
373.117806	-57.5537114984542\\
563.10151	-86.8588454630973\\
511.00632	-78.82312192618\\
536.804664	-82.8025365342137\\
725.071369	-111.842822068951\\
772.048242	-119.089041231543\\
1041.211336	-160.607657628283\\
845.810823	-130.466976666378\\
1056.415228	-162.952869783132\\
748.77738	-115.499492685933\\
670.435953	-103.415266697707\\
764.324828	-117.897698620734\\
533.596668	-82.3077006585112\\
485.329584	-74.8624654466391\\
590.037552	-91.0137508716542\\
584.758578	-90.1994650302439\\
401.814202	-61.9801528793552\\
423.778212	-65.3681184885043\\
315.859	-48.7214961812631\\
352.04988	-54.3039675425874\\
367.220172	-56.6439968713279\\
255.549228	-39.4186669878861\\
336.9155	-51.9694776677515\\
429.893349	-66.3113831223865\\
708.049896	-109.217246632541\\
744.26079	-114.802805168917\\
808.556166	-124.720416876219\\
460.501152	-71.0326605182542\\
685.984572	-105.813653260168\\
1075.92138	-165.961708885956\\
825.557032	-127.342813666867\\
989.075351	-152.565641430917\\
954.431731	-147.221836126857\\
500.645691	-77.224986852432\\
351.586692	-54.2325204336774\\
510.045081	-78.6748500635205\\
613.729539	-94.6682582757396\\
1032.701397	-159.294992925073\\
1305.995465	-201.450815271195\\
1260.506625	-194.434126355096\\
915.017936	-141.142227622487\\
943.067026	-145.468821549913\\
846.049512	-130.503794630096\\
825.557032	-127.342813666867\\
774.032441	-119.395105469071\\
510.045081	-78.6748500635205\\
454.810256	-70.1548353882691\\
406.41678	-62.6901041121869\\
364.757904	-56.2641901190769\\
423.956388	-65.3956022749521\\
670.742968	-103.462624000016\\
1022.882565	-157.780430459566\\
759.672744	-117.180111048935\\
514.21311	-79.3177717754454\\
430.516476	-66.4075009463236\\
403.211695	-62.1957172604963\\
233.919375	-36.0822453556364\\
172.591244	-26.622333495216\\
221.018454	-34.0922682670105\\
404.830041	-62.4453483884964\\
321.460919	-49.5855965398606\\
338.427612	-52.2027221187048\\
378.66396	-58.4092100624707\\
351.02781	-54.1463124509105\\
236.555935	-36.4889367843171\\
153.428389	-23.6664482213926\\
136.12185	-20.9969011329785\\
250.266775	-38.6038445080226\\
577.458768	-89.0734636655877\\
861.097054	-132.824889675956\\
933.894696	-144.053982520257\\
589.974714	-91.0040580613273\\
393.68364	-60.7260073731815\\
272.166462	-41.981888244441\\
194.00568	-29.9255268878308\\
478.790144	-73.8537517453116\\
483.688318	-74.6092988908296\\
727.224691	-112.174973660642\\
930.996976	-143.607006958648\\
1533.549311	-236.551172832388\\
1734.80111	-267.594419206405\\
1384.29801	-213.529043680709\\
1294.85043	-199.731684962442\\
871.173468	-134.379184365111\\
970.111103	-149.640391441166\\
1030.63284	-158.975916400496\\
1093.644126	-168.695456227525\\
812.729995	-125.364233242651\\
739.339552	-114.043700383477\\
719.807169	-111.03081512852\\
646.00976	-99.6475074476227\\
756.266994	-116.654771465072\\
550.373442	-84.8955310840331\\
991.299005	-152.908641788258\\
1195.233253	-184.365658004863\\
811.248529	-125.13571596101\\
467.8125	-72.1604416283757\\
533.543556	-82.2995080912417\\
936.03276	-144.383780553592\\
1192.256175	-183.906441410091\\
1521.072976	-234.626688464177\\
1610.923835	-248.486253281553\\
1580.668467	-243.819339258288\\
1171.064048	-180.637539353466\\
1072.22757	-165.391935823263\\
1242.60068	-191.672120425427\\
1427.873736	-220.250633275765\\
858.75636	-132.463835807672\\
507.651909	-78.305701422959\\
757.68849	-116.874038327614\\
614.003532	-94.7105219088898\\
395.903508	-61.068423737081\\
532.707351	-82.1705228202371\\
592.075476	-91.328102232178\\
452.041712	-69.7277853250331\\
346.524916	-53.4517375525361\\
473.566948	-73.0480697037425\\
392.278018	-60.509189087474\\
561.586102	-86.625092601583\\
821.094747	-126.654502738297\\
730.95516	-112.750401402571\\
473.616274	-73.0556782775702\\
506.099776	-78.0662837016583\\
818.212085	-126.209849884888\\
1345.640049	-207.566023158144\\
926.550966	-142.921206461474\\
555.524724	-85.6901203352231\\
415.32302	-64.063898552584\\
493.980608	-76.1968926207264\\
434.938383	-67.0895835370188\\
324.895472	-50.1153790088541\\
364.757904	-56.2641901190769\\
444.469647	-68.5597884150771\\
579.194616	-89.3412195683901\\
633.360112	-97.696289414493\\
437.815659	-67.5334055957426\\
757.68849	-116.874038327614\\
891.482127	-137.511810796256\\
843.634924	-130.131342554894\\
651.436934	-100.484653254188\\
741.470896	-114.372461851591\\
970.554156	-149.708732710681\\
578.829734	-89.2849362363666\\
235.576704	-36.3378896416773\\
336.9155	-51.9694776677515\\
387.466004	-59.7669321889047\\
538.148534	-83.0098295259382\\
471.747416	-72.7674054789153\\
352.663344	-54.3985948412661\\
570.322504	-87.9726893985113\\
531.373612	-81.9647925431346\\
385.027038	-59.39071978826\\
484.849728	-74.7884472857776\\
444.966544	-68.63643516343\\
390.950397	-60.3044025166254\\
450.367896	-69.4695979064313\\
260.741327	-40.219552293042\\
298.955426	-46.114106757214\\
222.76499	-34.3616728021177\\
253.47476	-39.0986786869666\\
509.44626	-78.5824814589698\\
606.693978	-93.5830175246623\\
863.774018	-133.237813453032\\
1088.916304	-167.966186010375\\
679.450653	-104.80579117738\\
389.304505	-60.0505223967213\\
275.38992	-42.4791091456578\\
280.761	-43.307602409137\\
411.693282	-63.5040086456763\\
259.071395	-39.9619639844582\\
297.93324	-45.956433772447\\
421.336751	-64.991521232171\\
716.338452	-110.49576283598\\
656.992952	-101.34167334787\\
907.182132	-139.933548766849\\
608.363887	-93.8406022854782\\
521.394335	-80.4254813117077\\
264.980562	-40.873457585796\\
257.173014	-39.6691372401132\\
169.703086	-26.1768328795366\\
184.64036	-28.4809190007156\\
233.293984	-35.9857783079401\\
431.02038	-66.4852285298711\\
480.284288	-74.0842246183033\\
541.887093	-83.586505156653\\
591.893016	-91.2999576354015\\
933.228208	-143.951176228377\\
784.47369	-121.005676240463\\
590.037552	-91.0137508716542\\
727.224691	-112.174973660642\\
803.154765	-123.88724657493\\
1035.023604	-159.653195159242\\
799.140342	-123.268018707861\\
478.083804	-73.7447982514658\\
238.97291	-36.8617570561242\\
170.27437	-26.2649538804402\\
183.99094	-28.3807454610982\\
224.204608	-34.5837349972411\\
236.555935	-36.4889367843171\\
219.824331	-33.9080738663936\\
318.718752	-49.1626151493702\\
502.121283	-77.4525980650082\\
453.303698	-69.9224476461314\\
474.691748	-73.2215709777435\\
543.89904	-83.896849544742\\
313.302118	-48.3270951459944\\
160.401075	-24.7419904548643\\
370.778507	-57.1928728100036\\
565.037616	-87.1574913180058\\
565.018524	-87.1545463621708\\
641.709783	-98.9842326541705\\
536.987722	-82.8307733729549\\
403.84232	-62.2929916567596\\
586.521936	-90.4714643719224\\
838.531181	-129.344086231393\\
828.14116	-127.741417418806\\
599.021904	-92.3995941487465\\
995.879436	-153.615176829131\\
779.169746	-120.18753875715\\
726.314672	-112.034602522782\\
846.117826	-130.514332117678\\
836.048279	-128.961096668609\\
619.34344	-95.5342068671475\\
557.438772	-85.9853637264918\\
985.761881	-152.054536108758\\
1368.985968	-211.167149304325\\
1007.88622	-155.467232590737\\
557.194638	-85.9477058672929\\
556.550687	-85.8483759251749\\
700.562775	-108.062352400589\\
831.114801	-128.200103853658\\
606.693978	-93.5830175246623\\
647.83158	-99.9285245982278\\
874.012106	-134.817046482312\\
937.773473	-144.652286886425\\
617.47576	-95.246115775908\\
479.408532	-73.9491385748164\\
659.37938	-101.709781721191\\
1113.572635	-171.769444225783\\
1010.164104	-155.818597967717\\
992.957064	-153.164398677366\\
568.338335	-87.6666297885066\\
504.363252	-77.7984235253307\\
478.790144	-73.8537517453116\\
290.2697	-44.7743267726624\\
182.11104	-28.0907694253634\\
145.9952	-22.5198730423472\\
126.123534	-19.4546531210078\\
341.717382	-52.7101716974479\\
504.67941	-77.8471911424943\\
603.635	-93.1111677912509\\
443.582272	-68.422910131812\\
513.371331	-79.1879267106096\\
556.15092	-85.7867115547948\\
336.214494	-51.8613469475503\\
355.661631	-54.8610829407688\\
336.363885	-51.8843906253813\\
250.580968	-38.6523090224094\\
329.89475	-50.8865215249327\\
579.393048	-89.3718278585773\\
725.033376	-111.836961619179\\
440.625	-67.966748627929\\
313.60892	-48.3744195928881\\
259.071395	-39.9619639844582\\
327.48165	-50.5142989748867\\
237.112352	-36.5747645389186\\
558.198641	-86.1025741101812\\
964.682589	-148.803038938559\\
755.49123	-116.535109265282\\
1059.44006	-163.41945246951\\
1243.935628	-191.878037191719\\
1242.092874	-191.593790955347\\
861.097054	-132.824889675956\\
647.507497	-99.8785345436562\\
643.94732	-99.3293744750493\\
750.11595	-115.705968148539\\
937.511624	-144.611896474711\\
941.17626	-145.177169425235\\
1301.2675	-200.721522996184\\
1468.634709	-226.538045033429\\
1728.36996	-266.602409321722\\
1158.4495	-178.69173552261\\
1255.31456	-193.633246294467\\
1376.789764	-212.370890901092\\
1066.338906	-164.483604825607\\
1226.452738	-189.181287824525\\
1015.303818	-156.611402845918\\
541.444442	-83.5182258959508\\
341.192862	-52.6292640798813\\
370.485456	-57.1476694655448\\
563.10151	-86.8588454630973\\
435.183	-67.1273158993431\\
295.72536	-45.6158665668626\\
412.752351	-63.6673708618524\\
291.21452	-44.9200659918139\\
224.215178	-34.5853654279543\\
273.583284	-42.2004341388512\\
309.453672	-47.7334693601444\\
229.411685	-35.3869306705353\\
452.041712	-69.7277853250331\\
639.732249	-98.6791964980714\\
470.625	-72.5942719387668\\
848.227809	-130.839798634944\\
1244.534436	-191.970403791009\\
1152.033472	-177.702058218177\\
1415.654496	-218.365806011074\\
954.027684	-147.159511563151\\
1006.114452	-155.193936000024\\
661.060884	-101.969155019766\\
399.544189	-61.6300016102417\\
502.269222	-77.4754177707109\\
386.04925	-59.5483967835473\\
282.518808	-43.5787456590029\\
403.533663	-62.2453810746744\\
222.66896	-34.3468600999997\\
136.58328	-21.0680770690225\\
207.201618	-31.9610105779431\\
466.395788	-71.9419127015509\\
630.50988	-97.2566389137832\\
774.427308	-119.456014010643\\
750.060288	-115.697382241789\\
714.495236	-110.21144533575\\
826.056198	-127.419810410233\\
661.060884	-101.969155019766\\
537.565732	-82.9199318645848\\
530.971496	-81.9027658376793\\
438.573096	-67.6502408415423\\
688.032211	-106.129503166991\\
452.03125	-69.7261715534038\\
397.060046	-61.2468206222433\\
608.495742	-93.8609410216832\\
833.427438	-128.556829907892\\
611.76896	-94.365837441565\\
447.811992	-69.0753477284225\\
375.501388	-57.921380874064\\
425.407119	-65.6193786589606\\
288.18867	-44.453326278144\\
285.029055	-43.9659532092849\\
418.009884	-64.4783494123524\\
509.44626	-78.5824814589698\\
575.760771	-88.8115463089466\\
466.395788	-71.9419127015509\\
597.370928	-92.1449298178921\\
528.342235	-81.49720028542\\
296.0436	-45.6649553341439\\
323.879072	-49.958598519083\\
688.87887	-106.260100975618\\
982.138599	-151.495642044867\\
943.464406	-145.530117723689\\
791.37982	-122.070952159219\\
735.34968	-113.428259527235\\
552.092123	-85.1606389637462\\
523.794458	-80.7957021494204\\
420.062333	-64.7949412654126\\
610.209589	-94.1253032531402\\
521.034976	-80.3700499067258\\
319.869036	-49.340047350173\\
499.730657	-77.083842146925\\
606.692184	-93.5827407987683\\
719.802232	-111.030053592434\\
786.592205	-121.332458825358\\
807.94238	-124.62573990879\\
1129.744487	-174.263964962763\\
1420.188672	-219.065206182256\\
1573.488745	-242.711861560943\\
937.773473	-144.652286886425\\
500.645691	-77.224986852432\\
307.783996	-47.4759205397595\\
212.709882	-32.810664580018\\
322.493633	-49.7448934768082\\
311.738644	-48.0859280666314\\
559.67377	-86.3301139046476\\
497.569527	-76.7504861651666\\
447.595176	-69.0419036919504\\
615.849234	-94.9952228765518\\
721.933576	-111.358815060548\\
873.825942	-134.788330540658\\
951.63714	-146.790768293631\\
1345.3293	-207.518089883433\\
1148.967545	-177.229136596116\\
836.173404	-128.980397297084\\
545.785944	-84.187905952919\\
560.534466	-86.4628769314325\\
582.33494	-89.8256169855095\\
742.704984	-114.562820884512\\
526.291308	-81.1808432020425\\
531.903559	-82.0465372796683\\
506.955288	-78.198247092415\\
278.072487	-42.8928971898373\\
494.157054	-76.2241095533298\\
767.505912	-118.388383299526\\
537.565732	-82.9199318645848\\
602.146384	-92.8815476165547\\
1064.010948	-164.124515495218\\
796.03804	-122.789486213786\\
768.91893	-118.60634242398\\
1121.30767	-172.962579384871\\
1024.484736	-158.027566581248\\
606.415646	-93.5400845973905\\
377.895177	-58.2906246873549\\
398.276556	-61.4344682350055\\
726.91591	-112.127343951461\\
1230.93952	-189.87337743438\\
1281.841428	-197.725036295584\\
1314.537228	-202.768388851133\\
926.550966	-142.921206461474\\
588.279744	-90.7426076217883\\
487.921024	-75.2621970802604\\
337.970192	-52.1321647282767\\
320.796376	-49.4830902657421\\
274.597176	-42.3568277676736\\
399.284919	-61.5900090112817\\
464.223012	-71.6067603152433\\
266.72525	-41.1425770654673\\
417.442696	-64.3908602226315\\
621.31488	-95.8383030190114\\
703.851964	-108.56971235963\\
917.632688	-141.545555150223\\
1206.0555	-186.034998013801\\
1435.721103	-221.461095733138\\
1017.877161	-157.008343003227\\
871.360663	-134.408059339317\\
901.997266	-139.133779157563\\
734.64715	-113.31989372885\\
513.544213	-79.2145938934437\\
651.71656	-100.527785781966\\
1074.825772	-165.792710500638\\
1211.910448	-186.93812829226\\
688.032211	-106.129503166991\\
354.54792	-54.6892921536343\\
221.133312	-34.1099852027584\\
121.461741	-18.7355679284145\\
170.845654	-26.3530748813439\\
281.62611	-43.4410462988516\\
344.2374	-53.0988864320723\\
454.39224	-70.090356095459\\
371.27867	-57.2700233413942\\
282.54897	-43.5833981709396\\
274.919157	-42.4064935871119\\
266.454396	-41.1007976255062\\
247.28385	-38.1437260089566\\
296.98842	-45.8106945532955\\
482.216152	-74.3822161414155\\
742.704984	-114.562820884512\\
803.930394	-124.006887950865\\
771.240088	-118.964382849085\\
955.506056	-147.387551592887\\
1152.241475	-177.734142842551\\
1183.59072	-182.56978824304\\
1456.272891	-224.631225001451\\
1283.422752	-197.968956751317\\
903.952764	-139.43541624354\\
615.912122	-95.0049233994175\\
607.450364	-93.6996906528953\\
1051.290098	-162.162314498262\\
1210.65305	-186.744173673726\\
803.804092	-123.987405769225\\
491.33175	-75.7883042159897\\
247.28385	-38.1437260089566\\
171.448676	-26.4460914934088\\
184.08	-28.3944830353003\\
117.072144	-18.058469180325\\
257.355076	-39.6972204450804\\
381.785015	-58.8906352213676\\
400.868886	-61.8343371518183\\
335.124048	-51.6931448047433\\
192.443355	-29.6845370426106\\
148.578192	-22.9183015654041\\
221.921888	-34.2316236635039\\
235.764925	-36.3669228771802\\
258.397776	-39.8580577302875\\
190.4292	-29.373852068806\\
163.72907	-25.2553362695594\\
320.857812	-49.4925668164796\\
783.912184	-120.919063503657\\
957.614834	-147.712832237966\\
1051.364496	-162.173790447572\\
777.452928	-119.922718246635\\
1122.043761	-173.076121993584\\
1588.687527	-245.056285494321\\
1460.310264	-225.253992923853\\
1455.915538	-224.576103023528\\
1059.137465	-163.372776955636\\
1054.23443	-162.616479997091\\
1081.079439	-166.757343494662\\
702.606225	-108.377556150906\\
633.913769	-97.78169143695\\
383.840514	-59.2076975392978\\
360.319536	-55.5795684063411\\
697.385788	-107.572299687231\\
467.824016	-72.162217980324\\
488.186603	-75.3031628473727\\
573.890636	-88.5230765320495\\
514.21311	-79.3177717754454\\
523.039341	-80.6792247654233\\
564.527964	-87.0788771009919\\
651.084742	-100.430327364525\\
722.798418	-111.492217611054\\
615.60808	-94.9580246846684\\
443.129313	-68.3530408541002\\
554.507	-85.5331356174231\\
260.155584	-40.1292009801534\\
109.425296	-16.8789369345106\\
186.52026	-28.7708950364504\\
310.079448	-47.8299957943898\\
152.559482	-23.5324186414776\\
67.60998	-10.4288919498424\\
174.419028	-26.9042705974552\\
318.634172	-49.1495686186492\\
389.85674	-60.1357050745734\\
465.093824	-71.7410837428885\\
419.378283	-64.6894260213866\\
688.831064	-106.252726862972\\
599.75074	-92.5120176680725\\
413.548641	-63.790199213092\\
607.793676	-93.7526467956583\\
327.48165	-50.5142989748867\\
247.149936	-38.1230696704017\\
168.980793	-26.0654186230448\\
114.2596	-17.6246320828998\\
214.326656	-33.0600532257966\\
181.4958	-27.9958681773047\\
299.975935	-46.271521063428\\
518.006874	-79.902962820306\\
489.201306	-75.4596815735752\\
331.848072	-51.1878229612185\\
374.265525	-57.7307480460139\\
291.21452	-44.9200659918139\\
407.684925	-62.8857164638209\\
290.146822	-44.7553727456828\\
269.413767	-41.5572829017701\\
245.480004	-37.8654813618179\\
195.116688	-30.096900735115\\
240.644448	-37.1195930914559\\
228.728544	-35.2815556404657\\
421.257144	-64.9792417905641\\
400.99995	-61.8545538756587\\
297.73144	-45.9253059656427\\
274.14069	-42.2864144474713\\
216.118295	-33.3364149337001\\
277.147728	-42.7502523955238\\
648.490067	-100.030096729641\\
846.684754	-130.601781202197\\
561.546828	-86.619034556566\\
512.724432	-79.088142037201\\
369.626571	-57.0151857869171\\
654.951264	-101.026741387421\\
578.829734	-89.2849362363666\\
384.773609	-59.3516281680886\\
469.659264	-72.4453064103629\\
370.778507	-57.1928728100036\\
499.288933	-77.0157058766931\\
288.32757	-44.4747517110732\\
306.79522	-47.3234010734529\\
374.150568	-57.7130158394391\\
471.935087	-72.7963538764909\\
346.516896	-53.4505004613043\\
370.525078	-57.1537811898322\\
396.755636	-61.1998651432082\\
655.973505	-101.184422855981\\
571.86129	-88.210048334691\\
917.282808	-141.491585888356\\
1185.61718	-182.882371272656\\
911.834808	-140.651227655108\\
556.150815	-85.7866953584632\\
428.596971	-66.1114158085646\\
692.161925	-106.766514760394\\
855.559284	-131.970684350454\\
660.440797	-101.873506118191\\
770.029645	-118.777671075787\\
458.252561	-70.6858136092862\\
234.688425	-36.2008719157097\\
246.4378	-38.0132221390521\\
577.540557	-89.0860796823901\\
513.992248	-79.2837036403294\\
472.188046	-72.8353729987971\\
745.577856	-115.005963622814\\
704.646008	-108.692194263625\\
600.35536	-92.6052807728791\\
388.466254	-59.9212215286269\\
501.246779	-77.3177051434941\\
685.828838	-105.789631169658\\
561.38764	-86.5944796838725\\
574.021968	-88.5433345950982\\
512.38825	-79.0362857024782\\
357.972	-55.21745915424\\
428.220072	-66.0532788449536\\
310.805256	-47.9419522423629\\
363.637296	-56.0913354643332\\
633.913769	-97.78169143695\\
740.485746	-114.220501698602\\
791.807016	-122.136847474162\\
688.831064	-106.252726862972\\
474.224704	-73.1495290778373\\
333.69283	-51.472378316147\\
387.257752	-59.7348091560871\\
473.566948	-73.0480697037425\\
630.50988	-97.2566389137832\\
798.84134	-123.221897417028\\
936.685218	-144.48442270707\\
670.231386	-103.383712079003\\
416.64504	-64.2678211648305\\
794.979686	-122.626234286915\\
1099.814133	-169.647184601543\\
781.476878	-120.543415660856\\
775.892172	-119.681970420884\\
903.36378	-139.344565003883\\
665.170992	-102.603142372435\\
542.533992	-83.6862898300615\\
597.484962	-92.1625196509997\\
555.524724	-85.6901203352231\\
632.62472	-97.5828546270729\\
794.180064	-122.502891972086\\
610.339632	-94.1453624869372\\
678.54681	-104.666372692319\\
639.754824	-98.6826787093628\\
648.490067	-100.030096729641\\
522.10008	-80.5343430263413\\
680.788018	-105.0120807678\\
482.216152	-74.3822161414155\\
515.77317	-79.5584122426556\\
572.154788	-88.2553206292472\\
531.818052	-82.0333477584769\\
229.988675	-35.4759318263727\\
138.168975	-21.3126717549018\\
95.565109	-14.7409923200083\\
220.31532	-33.9838093011557\\
572.154788	-88.2553206292472\\
752.883216	-116.132821079282\\
736.663788	-113.630961707334\\
486.343872	-75.0189201587692\\
570.408696	-87.9859845814849\\
494.407914	-76.262804903255\\
425.419764	-65.6213291600361\\
260.741327	-40.219552293042\\
371.77752	-57.3469713415146\\
370.525078	-57.1537811898322\\
367.220172	-56.6439968713279\\
435.122832	-67.1180349385909\\
370.525078	-57.1537811898322\\
327.48165	-50.5142989748867\\
221.999613	-34.2436128051484\\
340.079576	-52.457538849327\\
631.71225	-97.4421054205582\\
452.03125	-69.7261715534038\\
536.784578	-82.7994382531063\\
489.415168	-75.492669953252\\
704.646008	-108.692194263625\\
665.879643	-102.712452339827\\
919.190112	-141.785789012385\\
667.170504	-102.911568652112\\
752.301504	-116.043091551276\\
722.667648	-111.472046236942\\
374.260392	-57.7299562767754\\
665.142975	-102.598820728415\\
450.580536	-69.5023977916585\\
568.586656	-87.704933495709\\
614.003532	-94.7105219088898\\
767.700535	-118.418404048503\\
586.521936	-90.4714643719224\\
771.240088	-118.964382849085\\
973.410594	-150.149340491713\\
796.553112	-122.868936470011\\
684.918874	-105.649268515591\\
540.908016	-83.4354817686328\\
637.334726	-98.3093767123791\\
522.101349	-80.5345387705774\\
446.350816	-68.849960195048\\
398.054457	-61.4002092916784\\
474.691748	-73.2215709777435\\
419.14992	-64.6542008511922\\
886.711724	-136.77597242677\\
1135.68994	-175.181055707796\\
980.414446	-151.229690104902\\
934.808996	-144.195014005694\\
884.45754	-136.428262793206\\
477.717867	-73.6883521848725\\
415.417128	-64.0784147847086\\
311.794032	-48.0944717086694\\
283.177242	-43.6803096151245\\
319.869036	-49.340047350173\\
561.050186	-86.5424271421613\\
717.670888	-110.70129212432\\
811.301634	-125.143907448524\\
674.476988	-104.038599486454\\
464.653968	-71.6732356131082\\
874.012106	-134.817046482312\\
1030.352626	-158.932693173262\\
1146.60351	-176.864482360445\\
970.543744	-149.707126651591\\
1517.753744	-234.114694349041\\
970.554156	-149.708732710681\\
554.188624	-85.484025872036\\
394.294712	-60.8202657040015\\
332.082928	-51.2240496817082\\
617.735846	-95.286234243502\\
817.91113	-126.163427342287\\
1068.342633	-164.792681272308\\
786.592205	-121.332458825358\\
598.930299	-92.3854640063169\\
305.884145	-47.1828670467722\\
307.817741	-47.4811257322303\\
373.172296	-57.5621166232945\\
399.31764	-61.5950562509568\\
247.014845	-38.1022317786822\\
341.979105	-52.750542673564\\
243.128578	-37.5027720741942\\
162.539756	-25.0718836609292\\
372.964044	-57.5299935904769\\
439.269936	-67.7577289530065\\
561.546828	-86.619034556566\\
518.893353	-80.0397028948749\\
547.327062	-84.4256246019108\\
555.880516	-85.7450015276834\\
705.88485	-108.88328660474\\
469.252244	-72.382523259097\\
725.071369	-111.842822068951\\
818.673372	-126.281003763071\\
619.34344	-95.5342068671475\\
644.578919	-99.4267991115693\\
561.050186	-86.5424271421613\\
664.460588	-102.493562003431\\
920.159993	-141.935393910259\\
701.721744	-108.24112426939\\
562.836072	-86.8179014453446\\
610.209589	-94.1253032531402\\
547.327062	-84.4256246019108\\
368.181984	-56.7923571196827\\
610.297506	-94.1388645187041\\
891.049376	-137.445058618247\\
858.75636	-132.463835807672\\
975.560315	-150.480936626356\\
1434.281167	-221.238984486268\\
861.26124	-132.850215494033\\
739.86627	-114.124947044252\\
558.198641	-86.1025741101812\\
753.543071	-116.234604225758\\
1065.244491	-164.314790461465\\
697.540934	-107.596231078284\\
636.459573	-98.1743836821108\\
868.214938	-133.922828813751\\
533.113119	-82.2331128495301\\
310.771846	-47.9367987239023\\
407.10575	-62.7963782700358\\
295.04629	-45.5111194917062\\
237.80735	-36.6819685204515\\
256.52011	-39.5684262907886\\
220.549425	-34.0199201793119\\
216.721524	-33.4294634756759\\
303.896838	-46.8763233978292\\
444.966544	-68.63643516343\\
535.939756	-82.669123803156\\
714.065501	-110.145158378084\\
678.54681	-104.666372692319\\
836.158158	-128.978045589738\\
971.753541	-149.893738778887\\
855.559284	-131.970684350454\\
579.194616	-89.3412195683901\\
635.931588	-98.0929415856017\\
570.14742	-87.9456825554662\\
636.46384	-98.1750418701764\\
900.557385	-138.911676394485\\
667.857177	-103.017488495926\\
742.06353	-114.463876106584\\
639.754824	-98.6826787093628\\
634.613871	-97.8896827144488\\
979.282161	-151.055034263835\\
803.54348	-123.947206165722\\
793.70893	-122.430219186502\\
724.24131	-111.714784823222\\
417.11319	-64.3400336660961\\
263.26671	-40.6090945830851\\
214.791192	-33.131708264754\\
396.2581	-61.1231198286088\\
359.092608	-55.3903138089837\\
496.70577	-76.6172509767533\\
385.027038	-59.39071978826\\
571.389526	-88.1372783711172\\
963.898396	-148.68207655897\\
1188.421729	-183.314975135119\\
940.214252	-145.028778943728\\
960.383396	-148.139885077716\\
838.531181	-129.344086231393\\
584.498596	-90.1593626047307\\
630.75704	-97.2947635358334\\
708.049896	-109.217246632541\\
608.495742	-93.8609410216832\\
710.162994	-109.543193640977\\
925.622208	-142.777944818303\\
1298.700533	-200.325566341829\\
1162.24515	-179.277217484436\\
1098.361446	-169.423106503014\\
1207.20335	-186.212054768213\\
825.557032	-127.342813666867\\
891.482127	-137.511810796256\\
668.371736	-103.096859621503\\
703.25391	-108.477462065426\\
935.711175	-144.334175817462\\
1043.213259	-160.916455806583\\
470.759818	-72.6150677200241\\
525.398456	-81.0431200872716\\
399.544189	-61.6300016102417\\
595.635048	-91.8771689790648\\
584.672418	-90.1861747832952\\
357.604065	-55.1607048945943\\
500.209292	-77.1576719675878\\
902.03512	-139.139618166477\\
1545	-238.317450508143\\
1492.406223	-230.204819539059\\
1337.230118	-206.268784766643\\
1079.104	-166.452630493941\\
1282.94812	-197.895744396514\\
1519.930926	-234.450526364272\\
1110.837	-171.347470401368\\
1150.513	-177.467524230728\\
1171.207404	-180.659652127858\\
785.070608	-121.097751305785\\
668.960099	-103.187615071427\\
853.704471	-131.684578003963\\
548.219364	-84.5632628787543\\
405.653838	-62.5724197158596\\
584.764128	-90.2003211220565\\
437.181498	-67.4355857687318\\
518.97996	-80.0530620919209\\
492.684888	-75.9970268039157\\
615.802896	-94.9880752040459\\
813.776612	-125.525674728151\\
612.055291	-94.4100042208022\\
453.195792	-69.9058030617854\\
540.101771	-83.3111178509079\\
643.94732	-99.3293744750493\\
761.515392	-117.464340934724\\
645.176532	-99.5189813811813\\
533.19384	-82.2455641265026\\
846.153891	-130.519895171951\\
784.263095	-120.973191798075\\
560.690724	-86.4869798493493\\
929.893944	-143.436863415561\\
821.238132	-126.676619985961\\
579.393048	-89.3718278585773\\
373.172296	-57.5621166232945\\
257.355076	-39.6972204450804\\
188.888029	-29.1361252445231\\
439.269936	-67.7577289530065\\
408.791196	-63.0563596251744\\
292.15934	-45.0658052109655\\
202.51413	-31.237961911634\\
272.835717	-42.0851213482208\\
477.717867	-73.6883521848725\\
562.836072	-86.8179014453446\\
761.515392	-117.464340934724\\
878.640944	-135.531048340549\\
851.221569	-131.301588441179\\
875.070336	-134.980279282087\\
495.70524	-76.4629184468136\\
203.832486	-31.4413193490335\\
265.885095	-41.0129825038934\\
260.741327	-40.219552293042\\
377.895177	-58.2906246873549\\
595.683198	-91.8845961539787\\
687.498876	-106.047235828824\\
1027.082329	-158.4282473199\\
637.843206	-98.3878101474822\\
651.61086	-100.511481474834\\
948.157364	-146.254010141748\\
632.002151	-97.4868228750694\\
429.502584	-66.2511073175012\\
310.771846	-47.9367987239023\\
456.271432	-70.3802229216436\\
615.802896	-94.9880752040459\\
917.435387	-141.515121317664\\
586.637241	-90.4892502577675\\
506.968917	-78.2003493762551\\
480.898454	-74.1789602010274\\
617.735846	-95.286234243502\\
847.839204	-130.779856011737\\
1333.73104	-205.729049266261\\
1177.84314	-181.68321956201\\
1124.814064	-173.503443383937\\
672.48234	-103.730923482557\\
451.757796	-69.6839910614224\\
497.287278	-76.7069490376013\\
207.201618	-31.9610105779431\\
381.080964	-58.7820348075504\\
297.93324	-45.956433772447\\
393.464542	-60.6922113364363\\
723.167148	-111.549094500068\\
968.842891	-149.44476875473\\
1153.876038	-177.986275455402\\
1532.109375	-236.329061585517\\
1227.152052	-189.289157552411\\
647.507497	-99.8785345436562\\
595.635048	-91.8771689790648\\
480.916106	-74.1816830357435\\
674.347101	-104.018564315778\\
904.83678	-139.571776398445\\
1210.65305	-186.744173673726\\
861.097054	-132.824889675956\\
599.021904	-92.3995941487465\\
680.322466	-104.940269010053\\
886.152536	-136.689717243265\\
614.003532	-94.7105219088898\\
427.806522	-65.9894884361137\\
323.58372	-49.9130402435862\\
223.490084	-34.473519115036\\
197.130843	-30.4075857089197\\
160.928421	-24.8233339851269\\
312.83774	-48.2554644786597\\
464.704688	-71.681059212519\\
495.14571	-76.3766105095432\\
366.055626	-56.4643647459432\\
315.411502	-48.6524692670446\\
135.72517	-20.9357129347471\\
79.329131	-12.2365800977\\
127.856118	-19.7219055492739\\
254.803791	-39.3036827514108\\
399.31764	-61.5950562509568\\
506.955288	-78.198247092415\\
604.569944	-93.2553836297285\\
727.224691	-112.174973660642\\
980.070948	-151.176705271494\\
1380.613	-212.960628024853\\
1396.453551	-215.404045325154\\
1508.170881	-232.636530285139\\
1261.266992	-194.551413555672\\
679.389744	-104.796395916802\\
627.128524	-96.7350621233751\\
627.057102	-96.7240452243782\\
836.158158	-128.978045589738\\
685.91805	-105.803392189978\\
473.566948	-73.0480697037425\\
350.95125	-54.1345030114212\\
295.088904	-45.5176927343185\\
557.666402	-86.0204758308667\\
516.066716	-79.603691941249\\
262.362177	-40.4695696649877\\
211.34757	-32.6005268954636\\
335.802214	-51.7977524371972\\
444.921875	-68.6295449354709\\
345.007152	-53.2176212761911\\
498.3525	-76.8712603588085\\
390.950397	-60.3044025166254\\
601.073283	-92.7160209534753\\
860.798558	-132.778846436017\\
693.034155	-106.901056915641\\
982.83951	-151.603758111244\\
1344.863181	-207.446190665496\\
1455.915538	-224.576103023528\\
1097.576564	-169.302037844639\\
671.169528	-103.528421211465\\
481.778432	-74.3146974912949\\
286.083252	-44.1285639156754\\
474.691748	-73.2215709777435\\
341.426723	-52.6653373208478\\
655.772247	-101.153378653098\\
543.557007	-83.8440906887228\\
441.788979	-68.1462932931232\\
557.978696	-86.0686474230278\\
305.64927	-47.1466373955178\\
484.31675	-74.7062350151388\\
325.079512	-50.1437673218584\\
209.26626	-32.2794832107276\\
246.470862	-38.0183219782422\\
178.63778	-27.5550163715434\\
217.608766	-33.5663212435877\\
187.3368	-28.8968469659248\\
142.11666	-21.9216052336133\\
197.130843	-30.4075857089197\\
264.64776	-40.8221226186996\\
271.26558	-41.8429264959305\\
273.583284	-42.2004341388512\\
460.524184	-71.0362132221507\\
600.35536	-92.6052807728791\\
482.716399	-74.4593796298714\\
464.921875	-71.7145604760294\\
783.912184	-120.919063503657\\
932.236349	-143.798181206925\\
725.071369	-111.842822068951\\
725.222176	-111.866084165882\\
307.4322	-47.4216557334042\\
205.846641	-31.7520043228381\\
430.516476	-66.4075009463236\\
630.090533	-97.1919543131888\\
713.428984	-110.046975136242\\
1004.808906	-154.992554515078\\
885.056328	-136.520626307481\\
604.569944	-93.2553836297285\\
429.124878	-66.1928458735131\\
466.395788	-71.9419127015509\\
488.865116	-75.4078240048462\\
361.332745	-55.7358566818827\\
216.721524	-33.4294634756759\\
289.27239	-44.6204909302247\\
210.276129	-32.435256288674\\
173.199545	-26.7161643971329\\
131.884428	-20.3432754968833\\
299.975935	-46.271521063428\\
422.651988	-65.1943975613968\\
440.211608	-67.9029825907119\\
321.807112	-49.6389970791122\\
599.386824	-92.4558833422996\\
803.634282	-123.961212444778\\
515.77317	-79.5584122426556\\
732.47185	-112.984352013581\\
777.289865	-119.89756565218\\
566.664087	-87.4083757335692\\
319.78564	-49.327183452372\\
416.713506	-64.2783820985305\\
494.035311	-76.2053306009821\\
264.980562	-40.873457585796\\
274.919157	-42.4064935871119\\
220.38624	-33.9947487662626\\
127.416666	-19.6541197368075\\
143.08344	-22.0707317998283\\
253.968825	-39.1748885971189\\
354.216075	-54.6381048045314\\
304.762736	-47.0098888371561\\
404.830041	-62.4453483884964\\
438.78338	-67.6826773119389\\
305.683344	-47.1518933364942\\
165.0376	-25.4571780388237\\
133.302	-20.5619370793763\\
180.23114	-27.8007933896286\\
207.42453	-31.9953949271519\\
277.147728	-42.7502523955238\\
611.29854	-94.2932747910355\\
584.402224	-90.1444971488469\\
513.992248	-79.2837036403294\\
500.645691	-77.224986852432\\
708.049896	-109.217246632541\\
926.550966	-142.921206461474\\
992.573784	-153.105277439547\\
1012.199552	-156.132567403317\\
734.64715	-113.31989372885\\
776.444548	-119.767174848095\\
808.682024	-124.739830570514\\
821.094747	-126.654502738297\\
961.785918	-148.356225186015\\
1272.03903	-196.213008787347\\
1068.801722	-164.863496107283\\
972.8125	-150.057084027477\\
790.961754	-122.00646515387\\
733.804286	-113.189881301921\\
710.162994	-109.543193640977\\
828.653859	-127.820501637938\\
498.546048	-76.9011152882007\\
601.073283	-92.7160209534753\\
738.230559	-113.872637351509\\
855.902957	-132.023696177748\\
656.992952	-101.34167334787\\
756.908448	-116.753716243\\
593.855262	-91.6026356056214\\
526.003236	-81.1364078722025\\
321.846588	-49.6450862827861\\
520.982553	-80.3619636182416\\
838.531181	-129.344086231393\\
643.687317	-99.2892688102697\\
377.895177	-58.2906246873549\\
454.39224	-70.090356095459\\
227.981472	-35.1663192039699\\
224.215178	-34.5853654279543\\
306.421464	-47.2657489200341\\
177.418297	-27.366910171221\\
198.48582	-30.6165919640247\\
232.55694	-35.8720886982363\\
144.077075	-22.2240007706605\\
269.413767	-41.5572829017701\\
210.236977	-32.4292170622518\\
123.15483	-18.9967282222419\\
236.555935	-36.4889367843171\\
280.70325	-43.2986944267636\\
348.679161	-53.7840315176946\\
343.092816	-52.9223334606986\\
358.01028	-55.2233638739847\\
475.05156	-73.2770722583275\\
437.96828	-67.5569475035834\\
518.97996	-80.0530620919209\\
874.362996	-134.871171537464\\
595.547204	-91.8636189738076\\
546.989608	-84.3735720601996\\
579.167028	-89.3369640979534\\
386.02984	-59.5454027759652\\
228.646308	-35.268870673566\\
447.51144	-69.0289873488852\\
333.067995	-51.3759970318827\\
210.628572	-32.489620895282\\
376.179817	-58.0260290744723\\
352.244448	-54.3339798077722\\
445.52998	-68.7233456042355\\
261.083316	-40.272304362027\\
152.3444	-23.4992420758529\\
119.768652	-18.474407634587\\
157.178405	-24.2448911032597\\
304.34384	-46.9452738036622\\
300.7527	-46.3913376682461\\
371.77752	-57.3469713415146\\
510.119364	-78.6863082739904\\
678.049768	-104.589703577604\\
495.098062	-76.3692607685193\\
752.301504	-116.043091551276\\
853.010448	-131.577524416938\\
747.819072	-115.351672932301\\
731.721872	-112.868667324327\\
1216.018738	-187.571835217016\\
848.488476	-130.880006722239\\
1082.33496	-166.951008584487\\
886.711724	-136.77597242677\\
1050.07374	-161.974690331618\\
1130.655684	-174.404517808039\\
595.547204	-91.8636189738076\\
361.501515	-55.7618895855217\\
298.54521	-46.0508306204648\\
501.246779	-77.3177051434941\\
626.62562	-96.6574887906047\\
636.539337	-98.1866873410896\\
417.216189	-64.3559213418792\\
224.215178	-34.5853654279543\\
307.53891	-47.4381158338208\\
640.06306	-98.7302243518708\\
893.664128	-137.848386145982\\
847.362926	-130.706389760155\\
487.348994	-75.1739610082771\\
379.059723	-58.4702568127397\\
541.705769	-83.5585357887591\\
878.923614	-135.574650407691\\
1006.114452	-155.193936000024\\
1010.85253	-155.924788213143\\
724.097948	-111.692671123325\\
1054.865664	-162.713848332077\\
1125.834691	-173.660875891743\\
1114.084062	-171.848332237926\\
1572.011223	-242.483952644367\\
1313.858365	-202.663673705888\\
1102.907	-170.124261739537\\
1270.389312	-195.95853850397\\
1016.949945	-156.865319214755\\
549.867318	-84.8174610637606\\
512.028091	-78.9807308968746\\
647.83158	-99.9285245982278\\
778.155348	-120.031067077435\\
781.476878	-120.543415660856\\
593.274272	-91.513017446676\\
761.515392	-117.464340934724\\
637.843206	-98.3878101474822\\
473.515625	-73.0401530911131\\
643.684261	-99.2887974198951\\
590.999505	-91.1621328693685\\
940.214252	-145.028778943728\\
934.085757	-144.083453827967\\
654.465432	-100.951801423916\\
457.667728	-70.5956026646046\\
263.001618	-40.5682039361013\\
449.21875	-69.2923412430128\\
303.896838	-46.8763233978292\\
294.10147	-45.3653802725547\\
244.856975	-37.769378654457\\
453.425984	-69.9413103566511\\
590.196813	-91.0383170046544\\
672.48234	-103.730923482557\\
639.732249	-98.6791964980714\\
364.89108	-56.2847326205584\\
269.091108	-41.507512500305\\
193.815	-29.8961143496671\\
247.014845	-38.1022317786822\\
404.486113	-62.3922972272547\\
432.382692	-66.6953662144256\\
441.788979	-68.1462932931232\\
736.82245	-113.655435444119\\
871.344867	-134.405622794043\\
1013.738376	-156.369932201026\\
1013.04252	-156.262595872326\\
1156.502176	-178.391359282485\\
661.814712	-102.085433574512\\
543.183949	-83.7865462023463\\
424.635816	-65.5004045718866\\
467.8125	-72.1604416283757\\
624.73598	-96.3660103523016\\
503.03268	-77.5931817604391\\
327.771949	-50.5590778212071\\
294.188022	-45.378730985808\\
590.037552	-91.0137508716542\\
703.831128	-108.56649839044\\
666.768354	-102.84953670213\\
1067.175095	-164.612587628599\\
1150.931544	-177.532084967948\\
789.17583	-121.730984989157\\
536.409027	-82.7415092195427\\
357.972	-55.21745915424\\
341.426723	-52.6653373208478\\
639.268322	-98.6076353978371\\
511.668612	-78.9252809817994\\
625.05024	-96.4144852014904\\
690.118475	-106.451311010076\\
788.921315	-121.691725852642\\
748.77738	-115.499492685933\\
816.239952	-125.905646837235\\
551.515272	-85.0716592487669\\
649.729782	-100.221323731684\\
442.070244	-68.189678637924\\
408.034871	-62.9396959062388\\
660.860144	-101.938190718786\\
922.27002	-142.260867214559\\
984.027594	-151.787020991419\\
576.692347	-88.9552426308072\\
1220.46587	-188.257808783561\\
1064.010948	-164.124515495218\\
673.543825	-103.89465836861\\
392.278018	-60.509189087474\\
575.626484	-88.7908324348519\\
490.909312	-75.7231428262436\\
482.429208	-74.415080194966\\
581.199006	-89.6503982833571\\
383.847301	-59.2087444393215\\
586.898536	-90.5295552145511\\
1088.962056	-167.973243291926\\
1113.802701	-171.804932085051\\
727.385916	-112.199842742168\\
838.425588	-129.327798429094\\
917.282808	-141.491585888356\\
919.022346	-141.759910976526\\
647.507497	-99.8785345436562\\
632.002151	-97.4868228750694\\
419.14992	-64.6542008511922\\
645.595879	-99.5836659817756\\
692.027037	-106.745708181582\\
1132.908545	-174.752023367825\\
1294.534564	-199.682962386505\\
1733.116056	-267.334498317571\\
1223.89361	-188.786540342011\\
979.318193	-151.060592227833\\
745.851992	-115.048249313825\\
808.376235	-124.692662379658\\
586.521936	-90.4714643719224\\
395.851544	-61.0604082497035\\
404.275126	-62.3597523185619\\
394.294712	-60.8202657040015\\
273.583284	-42.2004341388512\\
200.2932	-30.8953817334094\\
346.083804	-53.3836956837798\\
467.249131	-72.0735415223723\\
303.77396	-46.8573693708495\\
213.869925	-32.9896021141538\\
218.272656	-33.6687267919488\\
544.159968	-83.9370978914903\\
741.614764	-114.394653602381\\
345.007152	-53.2176212761911\\
432.183252	-66.6646024394551\\
457.667728	-70.5956026646046\\
427.806522	-65.9894884361137\\
533.113119	-82.2331128495301\\
451.87547	-69.7021423673584\\
306.79522	-47.3234010734529\\
226.487328	-34.9358463309783\\
187.1355	-28.8657962845091\\
251.057785	-38.7258584151595\\
509.44626	-78.5824814589698\\
600.35536	-92.6052807728791\\
404.830041	-62.4453483884964\\
265.526664	-40.9576942436325\\
159.317086	-24.5747843093246\\
241.73805	-37.2882820497152\\
469.135818	-72.3645644581308\\
597.370928	-92.1449298178921\\
621.543692	-95.8735974478047\\
440.736824	-67.9839975668194\\
407.10575	-62.7963782700358\\
459.21875	-70.834849013292\\
662.396371	-102.175154927227\\
945.135548	-145.787892675713\\
1087.908609	-167.810748273618\\
786.929084	-121.384422672873\\
490.909312	-75.7231428262436\\
512.56638	-79.0637623933902\\
482.216152	-74.3822161414155\\
484.31675	-74.7062350151388\\
316.847776	-48.8740156475696\\
402.58988	-62.0998018135787\\
446.350816	-68.849960195048\\
433.73856	-66.9045099069728\\
245.612708	-37.8859510569326\\
160.401075	-24.7419904548643\\
269.360406	-41.5490519260571\\
413.548641	-63.790199213092\\
657.859644	-101.475361262313\\
721.933576	-111.358815060548\\
910.831416	-140.496453659444\\
803.54348	-123.947206165722\\
930.759336	-143.570350803995\\
1287.503296	-198.598383834013\\
1718.6661	-265.105581376552\\
1361.284418	-209.979179232505\\
915.346198	-141.192862291056\\
999.161905	-154.121500222951\\
1190.937088	-183.702971215373\\
1539.304	-237.438838082192\\
948.157364	-146.254010141748\\
446.350816	-68.849960195048\\
285.929937	-44.1049149577954\\
292.15934	-45.0658052109655\\
200.94102	-30.9953084717836\\
122.296689	-18.8643593061924\\
164.858236	-25.4295110024529\\
270.897744	-41.7861875071117\\
384.245939	-59.2702346605743\\
394.078366	-60.7868941653946\\
301.741476	-46.5438571345527\\
271.776648	-41.9217591320446\\
381.080964	-58.7820348075504\\
468.721297	-72.3006242717864\\
489.708219	-75.5378732977108\\
444.575638	-68.5761376091851\\
314.117175	-48.4528183215664\\
204.422484	-31.5323269989784\\
208.547262	-32.1685772105461\\
321.460919	-49.5855965398606\\
405.02826	-62.4759238232681\\
289.126313	-44.5979584394688\\
194.4522	-29.9944029447893\\
317.57948	-48.9868815581241\\
211.72146	-32.6581997184865\\
276.77097	-42.6921371812723\\
361.501515	-55.7618895855217\\
540.217179	-83.3289196245831\\
568.156248	-87.6385427272699\\
399.284919	-61.5900090112817\\
624.249478	-96.290967040776\\
595.136356	-91.8002453505673\\
475.718848	-73.3800019508289\\
418.717114	-64.5874401893899\\
727.224691	-112.174973660642\\
898.532856	-138.59939122312\\
888.889353	-137.111873392099\\
970.554156	-149.708732710681\\
732.47185	-112.984352013581\\
669.170016	-103.219994931788\\
627.128524	-96.7350621233751\\
845.95364	-130.489006299601\\
668.371736	-103.096859621503\\
914.146245	-141.007768608409\\
876.374778	-135.181490474174\\
957.565908	-147.70528536445\\
1627.863351	-251.09918678703\\
1185.61718	-182.882371272656\\
610.209589	-94.1253032531402\\
410.24368	-63.280406410795\\
437.181498	-67.4355857687318\\
556.550687	-85.8483759251749\\
740.383881	-114.2047889432\\
871.41992	-134.417199777611\\
966.074329	-149.01771591498\\
963.092636	-148.557787452872\\
579.194616	-89.3412195683901\\
525.30687	-81.0289928756067\\
614.088398	-94.7236125553331\\
602.724242	-92.9706826620665\\
347.594325	-53.6166947217468\\
255.05678	-39.3427065012403\\
464.888214	-71.7093682406238\\
480.898454	-74.1789602010274\\
673.899642	-103.94954341734\\
853.60744	-131.669610896817\\
740.485746	-114.220501698602\\
516.623298	-79.6895451472288\\
423.956388	-65.3956022749521\\
658.926548	-101.639932033328\\
879.07372	-135.597804374828\\
1192.510569	-183.945681882262\\
1300.699575	-200.633920123641\\
1507.571	-232.543998174764\\
1187.98107	-183.247003141965\\
1036.347464	-159.857401592919\\
528.695563	-81.5517014039658\\
411.830626	-63.5251940643964\\
801.962101	-123.703277226197\\
1003.362916	-154.769509434004\\
586.898536	-90.5295552145511\\
978.436288	-150.924557696318\\
1347.49477	-207.852115313564\\
1589.22099	-245.138572576587\\
946.232968	-145.957170563439\\
507.79953	-78.3284720769146\\
293.237091	-45.2320491401581\\
444.870552	-68.6216283228415\\
408.99124	-63.0872165676142\\
431.02038	-66.4852285298711\\
242.5071	-37.4069086097885\\
343.077735	-52.9200072047302\\
254.73624	-39.2932629571717\\
330.47016	-50.9752789645424\\
277.147728	-42.7502523955238\\
365.940768	-56.4466478101953\\
597.370928	-92.1449298178921\\
551.515272	-85.0716592487669\\
803.154765	-123.88724657493\\
972.8125	-150.057084027477\\
933.894696	-144.053982520257\\
506.968917	-78.2003493762551\\
520.54002	-80.2937025591311\\
626.161096	-96.5858356026565\\
412.274223	-63.5936192463337\\
360.295452	-55.5758534306271\\
259.7045	-40.0596209226485\\
423.778212	-65.3681184885043\\
389.85674	-60.1357050745734\\
208.869861	-32.2183383569645\\
114.677087	-17.6890297770488\\
137.531775	-21.2143831597796\\
258.190042	-39.8260145993723\\
406.19844	-62.6564249975306\\
417.094524	-64.3371544210921\\
249.76798	-38.526904991695\\
324.616434	-50.0723371805338\\
493.497711	-76.1224053832519\\
635.667318	-98.0521777327565\\
437.96828	-67.5569475035834\\
415.417128	-64.0784147847086\\
493.497711	-76.1224053832519\\
593.622162	-91.5666797494962\\
546.989608	-84.3735720601996\\
647.752248	-99.9162875755846\\
600.35536	-92.6052807728791\\
400.592058	-61.7916362177152\\
321.807112	-49.6389970791122\\
408.791196	-63.0563596251744\\
341.717382	-52.7101716974479\\
443.432556	-68.3998163224784\\
627.128524	-96.7350621233751\\
954.431731	-147.221836126857\\
659.837178	-101.780397418413\\
664.834758	-102.551278016672\\
1021.330432	-157.541012738265\\
748.231132	-115.415233507483\\
436.545312	-67.3374535838975\\
307.783996	-47.4759205397595\\
245.432825	-37.8582039644085\\
254.803791	-39.3036827514108\\
212.424594	-32.7666586843413\\
230.136779	-35.4987769834536\\
458.037302	-70.6526097412739\\
142.11666	-21.9216052336133\\
205.10364	-31.6373958412556\\
300.942733	-46.4206504061571\\
400.868886	-61.8343371518183\\
305.354576	-47.1011806170323\\
211.687165	-32.6529096880884\\
129.63834	-19.9968146776102\\
239.388175	-36.9258120050467\\
491.981589	-75.8885423866828\\
644.731536	-99.450340402407\\
502.121283	-77.4525980650082\\
421.1415	-64.9614036137055\\
477.296	-73.62327887232\\
488.865116	-75.4078240048462\\
570.322504	-87.9726893985113\\
708.049896	-109.217246632541\\
845.95364	-130.489006299601\\
801.535368	-123.637453329363\\
672.296814	-103.702305952898\\
425.407119	-65.6193786589606\\
333.69283	-51.472378316147\\
354.24714	-54.6428966049198\\
519.654828	-80.1571610053122\\
319.338118	-49.2581528361349\\
288.682632	-44.5295203004663\\
550.422209	-84.9030534316764\\
671.38685	-103.56194329883\\
652.509858	-100.65015261488\\
768.363552	-118.520674935936\\
980.695045	-151.272972718685\\
522.606084	-80.6123945365206\\
907.028416	-139.909837954407\\
947.113716	-146.093026626805\\
874.012106	-134.817046482312\\
952.759472	-146.963888876715\\
973.529262	-150.167645122921\\
1692.28349	-261.036043284027\\
1527.133272	-235.561493831196\\
1031.4915	-159.108365372699\\
1254.077252	-193.442390574044\\
1549.774638	-239.05394212967\\
1497.153078	-230.937025611248\\
1701.678012	-262.485155602333\\
1554.845248	-239.836087662175\\
2120.432548	-327.078368164301\\
1549.21875	-238.968195973729\\
1013.04252	-156.262595872326\\
613.729539	-94.6682582757396\\
614.001132	-94.710151707025\\
473.616274	-73.0556782775702\\
393.293063	-60.6657605674423\\
626.813676	-96.6864965747294\\
873.499352	-134.737953779388\\
527.873346	-81.4248737928302\\
460.501152	-71.0326605182542\\
426.488166	-65.7861309987143\\
282.518808	-43.5787456590029\\
352.170396	-54.3225572292317\\
185.87084	-28.6707214968329\\
217.608766	-33.5663212435877\\
193.815	-29.8961143496671\\
265.526664	-40.9576942436325\\
201.622176	-31.1003773140609\\
165.0376	-25.4571780388237\\
86.37286	-13.3230807691241\\
94.887602	-14.6364863388164\\
207.256035	-31.9694044424766\\
143.440284	-22.1257752641061\\
158.262394	-24.4120972487995\\
372.964044	-57.5299935904769\\
540.4655	-83.3672233317855\\
533.509944	-82.2943234141242\\
390.534967	-60.2403221163247\\
518.97996	-80.0530620919209\\
781.552464	-120.555074860089\\
388.909893	-59.9896531890968\\
203.02233	-31.3163521565196\\
152.3444	-23.4992420758529\\
128.29557	-19.7896913617404\\
219.824331	-33.9080738663936\\
383.565107	-59.1652157955488\\
562.650858	-86.7893320419282\\
360.213216	-55.5631684637274\\
606.693978	-93.5830175246623\\
878.230171	-135.467686286117\\
872.455584	-134.576951754351\\
600.35536	-92.6052807728791\\
361.333824	-55.7360231184712\\
492.486464	-75.9664197477347\\
692.161925	-106.766514760394\\
878.067696	-135.442624391119\\
463.59968	-71.5106108698969\\
241.73805	-37.2882820497152\\
376.179817	-58.0260290744723\\
285.20541	-43.9931561050677\\
123.388356	-19.0327497891981\\
82.5188	-12.7285890194119\\
241.73805	-37.2882820497152\\
538.454492	-83.0570237851761\\
564.884301	-87.1338423601257\\
388.909893	-59.9896531890968\\
238.97291	-36.8617570561242\\
240.605778	-37.1136282139082\\
104.112784	-16.0594778305404\\
141.896104	-21.8875842992351\\
151.817054	-23.4178985455903\\
251.41754	-38.7813509034492\\
394.93608	-60.9191972163624\\
604.19448	-93.1974680159825\\
631.426516	-97.3980307290349\\
717.00305	-110.598277593891\\
740.970912	-114.29533893109\\
719.17835	-110.93381930916\\
685.828838	-105.789631169658\\
606.692184	-93.5827407987683\\
769.829004	-118.746722045633\\
1187.520196	-183.175912969353\\
1025.194476	-158.137044527735\\
497.287278	-76.7069490376013\\
265.485748	-40.9513829188396\\
630.75704	-97.2947635358334\\
739.86627	-114.124947044252\\
665.985784	-102.728824671551\\
362.644101	-55.938134363843\\
398.335687	-61.4435892377019\\
706.80665	-109.025474971004\\
1113.572635	-171.769444225783\\
1161.127528	-179.104823422513\\
1317.731648	-203.261130618287\\
1233.792221	-190.313408780258\\
874.06396	-134.825045002104\\
851.870302	-131.401656010512\\
341.717382	-52.7101716974479\\
354.24714	-54.6428966049198\\
454.810256	-70.1548353882691\\
590.196813	-91.0383170046544\\
632.39952	-97.5481173520862\\
650.841156	-100.392754034752\\
420.062333	-64.7949412654126\\
584.758578	-90.1994650302439\\
495.14571	-76.3766105095432\\
260.648325	-40.2052066622769\\
174.419028	-26.9042705974552\\
144.560465	-22.298564053768\\
227.6388	-35.1134617817042\\
181.4958	-27.9958681773047\\
102.635405	-15.8315909718257\\
224.980555	-34.7034254249236\\
474.071052	-73.1258281374454\\
306.421464	-47.2657489200341\\
286.34295	-44.168622533968\\
420.674661	-64.8893933352085\\
694.326716	-107.100435454247\\
447.783825	-69.071002946786\\
222.17	-34.2698951322939\\
119.489174	-18.4312979359248\\
337.238494	-52.0192997432269\\
778.729702	-120.119661628224\\
1035.601791	-159.74238095326\\
1488.887685	-229.662082318557\\
1793.069394	-276.582347289489\\
1717.06479	-264.858578064789\\
1091.938274	-168.43232723103\\
1135.959006	-175.222559347368\\
1464.531456	-225.905115069837\\
1280.276715	-197.483678099508\\
1182.888256	-182.461432625206\\
1346.928139	-207.764712041526\\
1472.611755	-227.151507469205\\
1529.767872	-235.967882928354\\
2031.601332	-313.376084071966\\
1313.858365	-202.663673705888\\
661.814712	-102.085433574512\\
343.9305	-53.0515468686024\\
271.978791	-41.9529398468654\\
313.826516	-48.4079839449662\\
305.884145	-47.1828670467722\\
333.149816	-51.3886179847099\\
599.194652	-92.4262406619765\\
467.824016	-72.162217980324\\
505.879374	-78.0322865218998\\
648.841644	-100.084327755076\\
368.420508	-56.8291496320225\\
420.062333	-64.7949412654126\\
612.163356	-94.4266733310217\\
681.654026	-105.145663174713\\
366.055626	-56.4643647459432\\
263.292612	-40.6130899867117\\
332.579544	-51.3006530855927\\
332.942852	-51.3566936268931\\
826.83405	-127.539794685645\\
1196.996112	-184.637580375404\\
1120.46976	-172.833331116292\\
1311.6432	-202.321982783393\\
1542.4189	-237.919313827556\\
2045.051832	-315.45083414838\\
1702.52448	-262.615723949062\\
997.578785	-153.877302732822\\
595.683198	-91.8845961539787\\
319.869036	-49.340047350173\\
356.505435	-54.9912403634281\\
397.060046	-61.2468206222433\\
507.79953	-78.3284720769146\\
357.972	-55.21745915424\\
324.895472	-50.1153790088541\\
370.769157	-57.1914305652384\\
462.03125	-71.268679323683\\
536.804664	-82.8025365342137\\
661.060884	-101.969155019766\\
448.939668	-69.2492926276583\\
218.33386	-33.678167556506\\
148.044748	-22.8360174139032\\
120.6036	-18.6031990123649\\
127.856118	-19.7219055492739\\
185.2556	-28.5758202487743\\
316.182603	-48.7714121954616\\
335.137842	-51.6952725399616\\
444.870552	-68.6216283228415\\
674.476988	-104.038599486454\\
497.183173	-76.6908907604588\\
964.988342	-148.850201576388\\
1341.57198	-206.938520353891\\
1028.51259	-158.648866190503\\
920.43477	-141.977778476019\\
1107.760836	-170.872969714103\\
1264.28715	-195.01727527392\\
820.810936	-126.610724671018\\
667.170504	-102.911568652112\\
425.932194	-65.7003718857085\\
491.072394	-75.7482983514629\\
944.05342	-145.620973590869\\
1696.954944	-261.756618693377\\
2045.595916	-315.534759528148\\
1607.81376	-248.006521796188\\
1315.881735	-202.975780100603\\
946.976016	-146.071786294808\\
1025.136985	-158.128176496313\\
1345.3293	-207.518089883433\\
724.097948	-111.692671123325\\
608.291304	-93.8294063013292\\
479.45814	-73.9567906473632\\
957.147402	-147.640730488759\\
836.173404	-128.980397297084\\
463.269696	-71.4597105814902\\
424.635816	-65.5004045718866\\
322.588266	-49.7594906905907\\
596.23633	-91.9699171947777\\
917.632688	-141.545555150223\\
932.574612	-143.850358537515\\
694.3189	-107.099229830173\\
506.099776	-78.0662837016583\\
567.911748	-87.6008284122866\\
437.693514	-67.5145646345826\\
307.817741	-47.4811257322303\\
246.223835	-37.9802178715454\\
213.869925	-32.9896021141538\\
159.317086	-24.5747843093246\\
198.9834	-30.6933440656582\\
386.213562	-59.5737420372223\\
561.38764	-86.5944796838725\\
552.674716	-85.2505043866873\\
330.47016	-50.9752789645424\\
298.955426	-46.114106757214\\
233.293984	-35.9857783079401\\
315.859	-48.7214961812631\\
453.4375	-69.9430867085993\\
504.67941	-77.8471911424943\\
452.861486	-69.8542361015204\\
271.781433	-41.9224972220127\\
589.802576	-90.9775056410712\\
862.930524	-133.107703848114\\
573.257244	-88.4253753238863\\
223.490084	-34.473519115036\\
286.830819	-44.2438767063058\\
444.966544	-68.63643516343\\
419.749902	-64.746748540895\\
374.971315	-57.8396167019325\\
236.36528	-36.4595281024228\\
449.033172	-69.2637156923135\\
682.385724	-105.258528159763\\
385.015838	-59.3889921795573\\
137.531775	-21.2143831597796\\
223.870053	-34.5321296285325\\
544.831903	-84.0407443873525\\
644.731536	-99.450340402407\\
374.150568	-57.7130158394391\\
353.739996	-54.5646692488547\\
696.36235	-107.414433580491\\
931.209784	-143.639832758005\\
830.434964	-128.095238468156\\
1166.3795	-179.914944184442\\
1404.05553	-216.576656492854\\
874.012106	-134.817046482312\\
703.71714	-108.548915652868\\
998.490528	-154.017939799022\\
687.962106	-106.118689416267\\
915.346198	-141.192862291056\\
1105.843935	-170.577286245367\\
830.944536	-128.173840345108\\
424.22926	-65.4376929929812\\
715.93638	-110.433742917559\\
1343.078036	-207.170830662138\\
1524.127698	-235.097881706281\\
1090.520506	-168.213635415385\\
933.627414	-144.012754064071\\
1242.2753	-191.621930307597\\
960.383396	-148.139885077716\\
589.802576	-90.9775056410712\\
527.305779	-81.3373261420648\\
666.768354	-102.84953670213\\
704.649675	-108.692759901224\\
685.984572	-105.813653260168\\
771.240088	-118.964382849085\\
535.871427	-82.6585840018125\\
562.407032	-86.7517216919685\\
794.008585	-122.476441203092\\
462.03125	-71.268679323683\\
360.213216	-55.5631684637274\\
284.078124	-43.819271363635\\
226.430743	-34.9271180507602\\
270.111501	-41.6649089134289\\
502.269222	-77.4754177707109\\
476.552076	-73.5085280172703\\
660.860144	-101.938190718786\\
879.07372	-135.597804374828\\
1028.97998	-158.720961461178\\
758.486232	-116.997090650982\\
619.610096	-95.5753387623466\\
238.648	-36.81163943616\\
408.137981	-62.9556007038581\\
779.47548	-120.234698464214\\
491.33175	-75.7883042159897\\
243.960512	-37.6310985401302\\
540.705432	-83.4042330292194\\
858.808214	-132.471834327464\\
912.63258	-140.774284605999\\
1229.805516	-189.698456436227\\
1608.341355	-248.087903734894\\
1114.05618	-171.844031417761\\
957.42381	-147.683366637536\\
1082.63883	-166.997880768103\\
702.606225	-108.377556150906\\
559.064748	-86.2361717879206\\
642.099218	-99.0443033055224\\
835.18106	-128.827327464005\\
897.04867	-138.370454379366\\
876.070148	-135.134501059968\\
475.123848	-73.2882227384973\\
417.442696	-64.3908602226315\\
325.911556	-50.2721107553798\\
472.188046	-72.8353729987971\\
443.432556	-68.3998163224784\\
363.609306	-56.0870179850842\\
742.537203	-114.536940534891\\
803.804092	-123.987405769225\\
533.596668	-82.3077006585112\\
452.660577	-69.8232457321584\\
646.534493	-99.7284479206049\\
360.295452	-55.5758534306271\\
198.9834	-30.6933440656582\\
271.978791	-41.9529398468654\\
369.613961	-57.0132406846188\\
314.619656	-48.5303264062582\\
222.66896	-34.3468600999997\\
124.430445	-19.1934928271804\\
238.619125	-36.8071854449733\\
377.895177	-58.2906246873549\\
615.849234	-94.9952228765518\\
730.695768	-112.710389985016\\
963.898396	-148.68207655897\\
1053.73386	-162.539266685634\\
538.148534	-83.0098295259382\\
425.55128	-65.6416156052277\\
883.751616	-136.319373467684\\
1361.06799	-209.945795045335\\
1050.07374	-161.974690331618\\
994.935291	-153.469541729254\\
1516.476549	-233.917686027875\\
1199.375808	-185.004650332494\\
940.765542	-145.113815854596\\
703.25391	-108.477462065426\\
731.161512	-112.782231358912\\
973.737149	-150.199711854205\\
660.860144	-101.938190718786\\
572.25322	-88.2705038417316\\
1018.59375	-157.118877413287\\
1291.811409	-199.262913611787\\
1280.507032	-197.519204675721\\
540.217179	-83.3289196245831\\
266.350645	-41.0847939531388\\
677.279148	-104.47083484381\\
505.37325	-77.9542165016273\\
207.866106	-32.0635083682688\\
163.067102	-25.1532271911917\\
309.805048	-47.7876693811733\\
498.655339	-76.9179735098729\\
821.253378	-126.678971693307\\
570.322504	-87.9726893985113\\
401.866476	-61.9882161844736\\
231.5695	-35.7197753109679\\
152.677288	-23.5505903085161\\
211.018326	-32.5497407526318\\
192.5052	-29.6940766819159\\
256.362216	-39.5440710186005\\
454.731926	-70.1427529249045\\
394.314975	-60.8233912874964\\
176.825035	-27.2753990467399\\
165.40756	-25.514244656293\\
228.728544	-35.2815556404657\\
289.126313	-44.5979584394688\\
179.857263	-27.7431225718657\\
276.96054	-42.7213785010734\\
198.85971	-30.6742647870476\\
140.813055	-21.7205231494258\\
364.99275	-56.3004152970588\\
586.234444	-90.427118507533\\
920.24744	-141.948882677958\\
1213.41996	-187.170971691193\\
976.627113	-150.645491046788\\
469.135818	-72.3645644581308\\
329.89475	-50.8865215249327\\
324.02482	-49.9810802613332\\
176.825035	-27.2753990467399\\
169.313864	-26.1167950836003\\
367.220172	-56.6439968713279\\
414.403236	-63.9220211558862\\
383.565107	-59.1652157955488\\
422.459856	-65.1647610511049\\
518.893353	-80.0397028948749\\
552.496732	-85.2230502163887\\
584.764128	-90.2003211220565\\
735.34968	-113.428259527235\\
737.208208	-113.714938915363\\
1035.812778	-159.774925861953\\
504.190038	-77.7717051312386\\
229.51712	-35.4031941012113\\
231.62725	-35.7286832933412\\
440.736824	-67.9839975668194\\
359.152866	-55.3996086523059\\
306.773431	-47.3200401032722\\
143.08344	-22.0707317998283\\
238.160318	-36.7364141087175\\
141.36266	-21.8053001477342\\
67.865112	-10.4682462590871\\
136.12185	-20.9969011329785\\
287.38275	-44.3290124919216\\
370.778507	-57.1928728100036\\
577.458768	-89.0734636655877\\
813.614769	-125.500710319645\\
554.007831	-85.4561384113048\\
249.43182	-38.4750520504893\\
467.605064	-72.1284444641922\\
504.50422	-77.8201679488667\\
237.80735	-36.6819685204515\\
335.115693	-51.6918560395013\\
664.17177	-102.449011602512\\
599.021904	-92.3995941487465\\
1037.44206	-160.02624387645\\
1171.014501	-180.629896690217\\
710.276468	-109.560697093649\\
555.524724	-85.6901203352231\\
};
\end{axis}

\begin{axis}[%
width=4.927cm,
height=3cm,
at={(0cm,0cm)},
scale only axis,
xmin=0,
xmax=2000,
xlabel style={font=\color{white!15!black}},
xlabel={y(t-1)u(t)},
ymin=-294.863639463692,
ymax=0,
ylabel style={font=\color{white!15!black}},
ylabel={y(t)},
axis background/.style={fill=white},
title style={font=\small},
title={C8, R = -0.801},
axis x line*=bottom,
axis y line*=left
]
\addplot[only marks, mark=*, mark options={}, mark size=1.5000pt, color=mycolor1, fill=mycolor1] table[row sep=crcr]{%
x	y\\
515.144564	-96.436\\
600.410536	-122.07\\
755.49123	-100.098\\
614.001132	-93.994\\
588.590428	-130.615\\
817.91113	-125.732\\
798.901128	-153.809\\
968.842891	-115.967\\
694.410396	-61.035\\
362.120655	-74.463\\
441.788979	-73.242\\
439.891452	-86.67\\
515.77317	-73.242\\
410.374926	-34.18\\
180.88056	-20.752\\
106.76904	-23.193\\
127.399149	-58.594\\
344.415532	-100.098\\
601.188588	-109.863\\
663.792246	-114.746\\
682.853446	-83.008\\
474.224704	-52.49\\
312.36799	-104.98\\
620.85172	-81.787\\
470.27525	-59.814\\
354.876462	-97.656\\
581.150856	-83.008\\
486.343872	-72.021\\
435.150882	-118.408\\
719.802232	-107.422\\
641.201918	-83.008\\
504.605632	-119.629\\
733.804286	-123.291\\
747.266751	-95.215\\
568.338335	-80.566\\
472.036194	-62.256\\
365.940768	-75.684\\
450.395484	-95.215\\
568.338335	-87.891\\
524.621379	-91.553\\
548.219364	-96.436\\
579.194616	-102.539\\
632.768169	-141.602\\
878.923614	-128.174\\
776.862614	-85.449\\
510.045081	-79.346\\
456.2395	-47.607\\
268.50348	-48.828\\
280.761	-68.359\\
389.304505	-52.49\\
296.98842	-57.373\\
331.960178	-75.684\\
444.870552	-89.111\\
523.794458	-81.787\\
497.183173	-128.174\\
774.427308	-98.877\\
573.882108	-57.373\\
324.616434	-45.166\\
257.22037	-62.256\\
360.213216	-72.021\\
402.237285	-40.283\\
221.274519	-42.725\\
239.388175	-56.152\\
312.542032	-46.387\\
253.968825	-40.283\\
223.490084	-52.49\\
294.10147	-62.256\\
363.637296	-92.773\\
545.319694	-90.332\\
539.191708	-107.422\\
643.242936	-98.877\\
595.635048	-115.967\\
694.410396	-91.553\\
544.831903	-91.553\\
541.444442	-79.346\\
467.824016	-75.684\\
444.870552	-76.904\\
467.499416	-133.057\\
847.839204	-189.209\\
1229.8585	-194.092\\
1265.285748	-189.209\\
1198.639015	-119.629\\
768.855583	-170.898\\
1126.559616	-213.623\\
1423.797295	-219.727\\
1448.440384	-169.678\\
1130.90387	-219.727\\
1512.820395	-279.541\\
1904.233292	-197.754\\
1321.589982	-161.133\\
1041.563712	-109.863\\
687.962106	-85.449\\
524.144166	-73.242\\
450.584784	-91.553\\
551.515272	-62.256\\
360.213216	-46.387\\
262.457646	-42.725\\
244.087925	-54.932\\
322.890296	-79.346\\
467.824016	-74.463\\
439.033848	-75.684\\
454.558104	-100.098\\
608.495742	-103.76\\
647.87744	-141.602\\
886.711724	-114.746\\
722.785054	-142.822\\
889.209772	-104.98\\
642.05768	-90.332\\
555.722464	-103.76\\
628.88936	-76.904\\
459.039976	-69.58\\
419.14992	-84.229\\
508.911618	-86.67\\
514.21311	-59.814\\
352.663344	-64.697\\
374.336842	-48.828\\
279.833268	-53.711\\
309.805048	-57.373\\
325.649148	-41.504\\
235.576704	-52.49\\
302.76232	-65.918\\
394.716984	-101.318\\
612.163356	-104.98\\
642.05768	-114.746\\
680.788018	-68.359\\
409.333692	-93.994\\
586.898536	-150.146\\
934.808996	-114.746\\
716.474024	-125.732\\
787.333784	-128.174\\
779.169746	-80.566\\
470.586006	-53.711\\
316.680056	-75.684\\
451.757796	-85.449\\
530.381943	-137.939\\
884.051051	-172.119\\
1112.577216	-170.898\\
1082.63883	-123.291\\
781.048485	-130.615\\
822.743885	-115.967\\
726.185354	-113.525\\
710.89355	-111.084\\
677.279148	-76.904\\
460.501152	-68.359\\
404.275126	-61.035\\
356.505435	-57.373\\
336.148407	-62.256\\
373.909536	-91.553\\
570.008978	-140.381\\
876.538964	-111.084\\
673.280124	-79.346\\
472.188046	-64.697\\
381.453512	-62.256\\
354.54792	-40.283\\
220.549425	-29.297\\
159.873729	-36.621\\
206.54244	-63.477\\
356.804217	-51.27\\
288.18867	-52.49\\
296.98842	-58.594\\
331.524852	-54.932\\
303.77396	-37.842\\
203.703486	-28.076\\
146.528644	-23.193\\
122.737356	-39.063\\
221.721588	-83.008\\
497.051904	-117.188\\
716.721808	-130.615\\
784.47369	-86.67\\
503.03268	-58.594\\
330.47016	-45.166\\
246.470862	-32.959\\
185.88876	-67.139\\
383.565107	-65.918\\
385.027038	-91.553\\
551.515272	-117.188\\
746.721936	-186.768\\
1234.53648	-216.064\\
1424.293888	-177.002\\
1157.062074	-168.457\\
1070.375778	-109.863\\
694.004571	-122.07\\
777.83004	-131.836\\
844.936924	-146.484\\
927.97614	-108.643\\
676.411318	-100.098\\
623.210148	-98.877\\
611.949753	-89.111\\
553.111977	-106.201\\
649.525316	-78.125\\
489.21875	-130.615\\
841.813675	-159.912\\
1013.04252	-113.525\\
694.3189	-70.801\\
426.505224	-73.242\\
454.613094	-123.291\\
787.82949	-153.809\\
1002.680871	-189.209\\
1250.67149	-202.637\\
1350.575605	-202.637\\
1328.285535	-153.809\\
996.989938	-137.939\\
899.224341	-158.691\\
1048.94751	-187.988\\
1204.815092	-109.863\\
673.899642	-72.021\\
447.034347	-101.318\\
625.233378	-86.67\\
515.77317	-53.711\\
320.600959	-74.463\\
449.905446	-84.229\\
502.762901	-67.139\\
394.643042	-51.27\\
303.21078	-70.801\\
416.168278	-58.594\\
346.524916	-79.346\\
485.280136	-107.422\\
658.926548	-100.098\\
601.188588	-69.58\\
414.07058	-70.801\\
431.673697	-107.422\\
688.467598	-177.002\\
1134.405818	-128.174\\
788.526448	-79.346\\
475.123848	-61.035\\
364.317915	-70.801\\
421.336751	-64.697\\
375.501388	-48.828\\
282.518808	-53.711\\
313.725951	-65.918\\
391.091494	-83.008\\
500.040192	-91.553\\
541.444442	-63.477\\
383.528034	-104.98\\
651.61086	-124.512\\
772.845984	-118.408\\
724.183328	-91.553\\
559.938148	-100.098\\
625.011912	-135.498\\
836.158158	-87.891\\
508.537326	-42.725\\
244.087925	-57.373\\
331.960178	-62.256\\
368.181984	-81.787\\
483.688318	-72.021\\
418.009884	-52.49\\
310.42586	-79.346\\
470.759818	-80.566\\
467.605064	-57.373\\
336.148407	-70.801\\
413.548641	-63.477\\
366.135336	-56.152\\
326.973096	-67.139\\
381.080964	-41.504\\
234.08256	-48.828\\
270.897744	-36.621\\
201.818331	-40.283\\
231.62725	-75.684\\
443.432556	-84.229\\
505.879374	-113.525\\
706.80665	-146.484\\
898.532856	-98.877\\
581.199006	-58.594\\
333.69283	-46.387\\
260.741327	-43.945\\
252.68375	-63.477\\
358.01028	-42.725\\
238.619125	-47.607\\
270.217332	-61.035\\
362.120655	-102.539\\
614.003532	-93.994\\
578.251088	-134.277\\
816.269883	-89.111\\
530.299561	-81.787\\
467.249131	-46.387\\
259.071395	-45.166\\
244.79972	-28.076\\
150.62774	-35.4\\
191.2308	-37.842\\
212.028726	-67.139\\
383.565107	-70.801\\
408.380168	-79.346\\
462.031758	-85.449\\
516.282858	-125.732\\
771.240088	-112.305\\
674.50383	-84.229\\
507.395496	-102.539\\
625.180283	-109.863\\
684.007038	-144.043\\
891.482127	-115.967\\
696.497802	-75.684\\
429.582384	-40.283\\
218.33386	-29.297\\
156.651059	-29.297\\
157.705751	-36.621\\
197.826642	-39.063\\
210.276129	-37.842\\
205.784796	-50.049\\
284.078124	-74.463\\
426.747453	-68.359\\
390.534967	-70.801\\
409.654586	-83.008\\
469.659264	-51.27\\
276.96054	-30.518\\
168.76454	-58.594\\
339.024884	-90.332\\
527.629212	-87.891\\
519.787374	-97.656\\
574.021968	-83.008\\
477.296	-61.035\\
355.406805	-85.449\\
513.206694	-118.408\\
719.802232	-118.408\\
706.777352	-84.229\\
516.660686	-136.719\\
836.173404	-100.098\\
604.792116	-101.318\\
621.484612	-117.188\\
720.940576	-115.967\\
702.875987	-86.67\\
518.97996	-76.904\\
473.113408	-128.174\\
821.467166	-178.223\\
1135.636956	-139.16\\
851.10256	-79.346\\
475.123848	-76.904\\
461.885424	-79.346\\
480.916106	-98.877\\
606.511518	-114.746\\
695.475506	-85.449\\
517.906389	-90.332\\
557.438772	-120.85\\
752.4121	-131.836\\
803.804092	-87.891\\
526.291308	-68.359\\
413.025078	-91.553\\
578.340301	-156.25\\
992.8125	-137.939\\
873.843565	-140.381\\
863.623912	-84.229\\
505.879374	-76.904\\
459.039976	-72.021\\
416.713506	-47.607\\
263.26671	-30.518\\
163.179746	-26.855\\
140.156245	-23.193\\
126.123534	-54.932\\
311.794032	-70.801\\
412.274223	-87.891\\
506.955288	-65.918\\
379.0285	-73.242\\
426.488166	-83.008\\
471.153408	-53.711\\
301.909531	-56.152\\
315.630392	-53.711\\
296.001321	-40.283\\
223.490084	-51.27\\
294.8025	-84.229\\
499.730657	-106.201\\
614.478986	-63.477\\
356.804217	-48.828\\
270.897744	-40.283\\
224.980555	-50.049\\
273.117393	-35.4\\
201.603	-76.904\\
466.115144	-130.615\\
794.008585	-104.98\\
647.83158	-139.16\\
879.07372	-162.354\\
1037.44206	-164.795\\
1035.077395	-124.512\\
761.515392	-90.332\\
549.128228	-89.111\\
540.101771	-89.111\\
545.002876	-106.201\\
659.189607	-125.732\\
782.807432	-125.732\\
801.164304	-168.457\\
1094.9705	-183.105\\
1213.61994	-219.727\\
1432.400313	-147.705\\
962.888895	-166.016\\
1091.223168	-184.326\\
1198.119	-140.381\\
915.143739	-162.354\\
1052.378628	-139.16\\
866.41016	-80.566\\
477.998078	-52.49\\
306.59409	-56.152\\
334.160552	-81.787\\
483.688318	-65.918\\
379.0285	-43.945\\
255.05678	-62.256\\
356.789136	-47.607\\
265.885095	-36.621\\
205.187463	-46.387\\
260.741327	-52.49\\
291.21452	-36.621\\
208.556595	-70.801\\
417.442696	-92.773\\
541.887093	-68.359\\
413.025078	-114.746\\
729.096084	-164.795\\
1056.171155	-150.146\\
981.504402	-196.533\\
1263.117591	-131.836\\
844.936924	-144.043\\
901.997266	-96.436\\
577.458768	-63.477\\
378.894213	-76.904\\
454.810256	-59.814\\
343.9305	-45.166\\
261.330476	-64.697\\
367.220172	-36.621\\
197.826642	-28.076\\
150.62774	-34.18\\
192.7752	-70.801\\
414.823059	-86.67\\
518.97996	-107.422\\
649.043724	-103.76\\
625.05024	-100.098\\
608.495742	-114.746\\
693.295332	-93.994\\
559.358294	-76.904\\
456.271432	-76.904\\
450.580536	-62.256\\
372.788928	-97.656\\
575.779776	-68.359\\
394.294712	-56.152\\
331.072192	-81.787\\
495.711007	-113.525\\
681.83115	-86.67\\
509.44626	-65.918\\
382.588072	-56.152\\
326.973096	-65.918\\
375.40301	-45.166\\
253.065098	-45.166\\
257.22037	-59.814\\
346.083804	-75.684\\
443.432556	-84.229\\
490.465467	-65.918\\
387.466004	-86.67\\
509.44626	-79.346\\
451.87547	-47.607\\
268.50348	-50.049\\
294.188022	-95.215\\
585.76268	-131.836\\
818.306052	-131.836\\
808.682024	-107.422\\
654.951934	-103.76\\
625.05024	-79.346\\
472.188046	-76.904\\
452.041712	-61.035\\
362.120655	-89.111\\
530.299561	-75.684\\
436.545312	-50.049\\
292.336209	-74.463\\
440.374182	-85.449\\
511.668612	-101.318\\
615.912122	-109.863\\
667.857177	-109.863\\
687.962106	-148.926\\
959.82807	-181.885\\
1195.530105	-205.078\\
1321.72771	-129.395\\
791.37982	-73.242\\
426.488166	-47.607\\
266.742021	-34.18\\
192.7752	-54.932\\
308.772772	-46.387\\
269.230148	-80.566\\
470.586006	-72.021\\
416.713506	-64.697\\
380.288966	-85.449\\
510.045081	-98.877\\
601.073283	-117.188\\
720.940576	-123.291\\
785.610252	-175.781\\
1129.744487	-150.146\\
945.769654	-117.188\\
718.831192	-80.566\\
486.779772	-80.566\\
488.310526	-83.008\\
509.171072	-106.201\\
641.666442	-75.684\\
453.195792	-79.346\\
476.552076	-74.463\\
428.16225	-42.725\\
248.787675	-74.463\\
449.905446	-107.422\\
641.201918	-73.242\\
433.153188	-80.566\\
503.054104	-147.705\\
916.804935	-109.863\\
675.877176	-106.201\\
668.960099	-147.705\\
935.711175	-140.381\\
866.291151	-87.891\\
521.457303	-56.152\\
327.983832	-58.594\\
352.970256	-97.656\\
613.377336	-151.367\\
970.111103	-162.354\\
1046.37153	-167.236\\
1068.63804	-129.395\\
796.03804	-83.008\\
500.040192	-72.021\\
423.339438	-53.711\\
310.771846	-50.049\\
285.029055	-42.725\\
246.4378	-61.035\\
356.505435	-72.021\\
408.791196	-41.504\\
239.395072	-65.918\\
391.091494	-89.111\\
535.200666	-101.318\\
623.308336	-128.174\\
814.417596	-161.133\\
1050.426027	-192.871\\
1239.581917	-136.719\\
866.114865	-118.408\\
750.11468	-124.512\\
779.694144	-102.539\\
623.334581	-73.242\\
443.919762	-89.111\\
559.706191	-136.719\\
878.693013	-162.354\\
1013.738376	-93.994\\
557.666402	-54.932\\
312.83774	-37.842\\
202.341174	-23.193\\
121.044267	-32.959\\
177.418297	-45.166\\
247.28385	-52.49\\
297.93324	-73.242\\
415.721592	-58.594\\
325.079512	-45.166\\
248.909826	-45.166\\
248.096838	-43.945\\
239.807865	-40.283\\
221.274519	-47.607\\
269.360406	-70.801\\
418.717114	-104.98\\
632.39952	-111.084\\
671.169528	-107.422\\
654.951934	-123.291\\
774.390771	-152.588\\
966.64498	-155.029\\
999.161905	-183.105\\
1190.1825	-167.236\\
1065.627792	-124.512\\
768.363552	-85.449\\
519.444471	-83.008\\
521.373248	-141.602\\
910.076054	-158.691\\
1002.451047	-115.967\\
711.341578	-76.904\\
439.352552	-39.063\\
212.424594	-30.518\\
164.278394	-29.297\\
151.289708	-19.531\\
103.70961	-45.166\\
249.76798	-57.373\\
320.428205	-62.256\\
346.516896	-53.711\\
289.126313	-34.18\\
179.6159	-26.855\\
141.606415	-36.621\\
195.812487	-41.504\\
222.66896	-43.945\\
233.34795	-35.4\\
185.3898	-28.076\\
151.133108	-50.049\\
293.237091	-112.305\\
688.87887	-128.174\\
798.011324	-145.264\\
891.049376	-101.318\\
627.057102	-133.057\\
864.8705	-195.313\\
1273.245447	-174.561\\
1144.247355	-184.326\\
1187.98107	-135.498\\
868.406682	-137.939\\
889.016855	-145.264\\
912.403184	-95.215\\
587.571765	-90.332\\
537.565732	-56.152\\
324.895472	-54.932\\
331.899144	-102.539\\
612.055291	-68.359\\
404.275126	-79.346\\
473.616274	-84.229\\
501.246779	-79.346\\
473.616274	-79.346\\
473.616274	-84.229\\
507.395496	-93.994\\
571.389526	-102.539\\
619.540638	-89.111\\
528.695563	-67.139\\
402.028332	-84.229\\
479.684155	-39.063\\
206.721396	-20.752\\
109.05176	-36.621\\
199.840797	-51.27\\
271.32084	-26.855\\
133.7379	-13.428\\
67.60998	-31.738\\
167.354474	-50.049\\
272.166462	-58.594\\
327.24749	-72.021\\
402.237285	-62.256\\
357.972	-91.553\\
533.113119	-80.566\\
457.292616	-57.373\\
331.960178	-86.67\\
485.61201	-48.828\\
264.64776	-41.504\\
219.639168	-28.076\\
142.401472	-21.973\\
113.468572	-37.842\\
194.69709	-28.076\\
148.578192	-50.049\\
278.572734	-75.684\\
426.85776	-73.242\\
406.346616	-54.932\\
304.762736	-61.035\\
335.265255	-45.166\\
250.580968	-65.918\\
362.087574	-47.607\\
258.02994	-46.387\\
250.582574	-42.725\\
226.1007	-32.959\\
175.605552	-42.725\\
226.86975	-37.842\\
207.866106	-64.697\\
358.938956	-63.477\\
346.393989	-50.049\\
272.166462	-46.387\\
248.031289	-36.621\\
195.812487	-43.945\\
249.43182	-93.994\\
561.050186	-120.85\\
710.3563	-81.787\\
474.691748	-80.566\\
455.842428	-54.932\\
318.825328	-93.994\\
550.710846	-81.787\\
464.223012	-54.932\\
314.815292	-70.801\\
400.592058	-52.49\\
299.87537	-75.684\\
422.69514	-42.725\\
233.919375	-50.049\\
278.572734	-57.373\\
325.649148	-74.463\\
418.556523	-58.594\\
328.302182	-61.035\\
341.979105	-61.035\\
353.14851	-95.215\\
554.436945	-79.346\\
475.123848	-125.732\\
785.070608	-157.471\\
980.414446	-125.732\\
755.146392	-81.787\\
476.245701	-62.256\\
368.181984	-89.111\\
541.705769	-112.305\\
674.50383	-87.891\\
531.037422	-107.422\\
633.360112	-64.697\\
360.103502	-34.18\\
185.87084	-37.842\\
216.872502	-84.229\\
488.865116	-70.801\\
407.10575	-69.58\\
411.49612	-104.98\\
626.62562	-97.656\\
577.537584	-87.891\\
508.537326	-62.256\\
361.333824	-73.242\\
434.544786	-100.098\\
591.979572	-81.787\\
482.216152	-87.891\\
514.953369	-78.125\\
449.21875	-58.594\\
337.970192	-67.139\\
379.872462	-48.828\\
276.268824	-56.152\\
330.061456	-95.215\\
570.14742	-109.863\\
663.792246	-115.967\\
698.585208	-102.539\\
604.569944	-73.242\\
419.749902	-51.27\\
292.90551	-62.256\\
359.092608	-70.801\\
416.168278	-91.553\\
549.867318	-111.084\\
681.389256	-131.836\\
803.804092	-107.422\\
629.385498	-62.256\\
372.788928	-109.863\\
685.984572	-151.367\\
931.209784	-107.422\\
656.992952	-108.643\\
670.435953	-124.512\\
759.149664	-93.994\\
562.836072	-79.346\\
476.552076	-89.111\\
531.903559	-78.125\\
467.8125	-91.553\\
559.938148	-114.746\\
691.229904	-86.67\\
523.66014	-102.539\\
619.540638	-92.773\\
560.534466	-96.436\\
575.626484	-76.904\\
463.269696	-100.098\\
593.881434	-69.58\\
410.24368	-76.904\\
456.271432	-84.229\\
499.730657	-80.566\\
458.82337	-41.504\\
224.95168	-28.076\\
145.9952	-21.973\\
115.072601	-37.842\\
215.51019	-84.229\\
502.762901	-108.643\\
652.509858	-111.084\\
652.951752	-74.463\\
437.693514	-86.67\\
507.79953	-72.021\\
416.713506	-65.918\\
369.338554	-42.725\\
240.157225	-61.035\\
345.33603	-58.594\\
331.524852	-57.373\\
326.739235	-67.139\\
379.872462	-57.373\\
322.493633	-53.711\\
295.034523	-37.842\\
208.547262	-50.049\\
290.484396	-91.553\\
529.725658	-70.801\\
412.274223	-86.67\\
503.03268	-75.684\\
447.595176	-101.318\\
601.119694	-92.773\\
562.297153	-129.395\\
779.47548	-91.553\\
549.867318	-106.201\\
641.666442	-103.76\\
602.22304	-54.932\\
324.867848	-96.436\\
563.282676	-67.139\\
394.643042	-84.229\\
499.730657	-90.332\\
544.159968	-115.967\\
692.207023	-85.449\\
514.744776	-111.084\\
685.499364	-136.719\\
838.634346	-111.084\\
673.280124	-95.215\\
568.338335	-75.684\\
451.757796	-91.553\\
546.479857	-80.566\\
473.566948	-69.58\\
403.84232	-56.152\\
326.973096	-70.801\\
409.654586	-58.594\\
354.024948	-119.629\\
749.116798	-151.367\\
945.135548	-130.615\\
810.727305	-126.953\\
787.997271	-124.512\\
745.577856	-69.58\\
406.41678	-63.477\\
363.786687	-50.049\\
282.27636	-42.725\\
240.969	-48.828\\
284.325444	-80.566\\
479.448266	-100.098\\
604.792116	-114.746\\
691.229904	-96.436\\
566.850808	-67.139\\
406.929479	-117.188\\
727.385916	-137.939\\
866.394859	-152.588\\
952.759472	-118.408\\
758.876872	-187.988\\
1204.815092	-133.057\\
813.776612	-78.125\\
459.21875	-57.373\\
330.927464	-48.828\\
288.768792	-87.891\\
534.289389	-111.084\\
693.608496	-147.705\\
916.804935	-109.863\\
667.857177	-87.891\\
511.789293	-47.607\\
271.978791	-53.711\\
307.817741	-56.152\\
322.874	-63.477\\
358.01028	-40.283\\
227.19612	-52.49\\
294.10147	-40.283\\
217.608766	-26.855\\
149.985175	-58.594\\
332.579544	-63.477\\
366.135336	-80.566\\
466.154876	-72.021\\
418.009884	-76.904\\
447.811992	-76.904\\
453.425984	-98.877\\
577.540557	-67.139\\
397.060046	-101.318\\
614.088398	-117.188\\
701.721744	-89.111\\
533.596668	-93.994\\
559.358294	-80.566\\
480.898454	-93.994\\
576.559196	-133.057\\
811.248529	-100.098\\
597.484962	-79.346\\
475.123848	-90.332\\
539.191708	-81.787\\
476.245701	-56.152\\
331.072192	-85.449\\
522.606084	-123.291\\
756.266994	-117.188\\
723.167148	-130.615\\
841.813675	-192.871\\
1225.502334	-125.732\\
780.418524	-107.422\\
656.992952	-83.008\\
510.665216	-106.201\\
670.871717	-152.588\\
947.113716	-101.318\\
617.735846	-90.332\\
560.690724	-124.512\\
759.149664	-79.346\\
462.031758	-47.607\\
277.215561	-62.256\\
359.092608	-46.387\\
259.906361	-39.063\\
218.166855	-42.725\\
236.26925	-37.842\\
207.18495	-37.842\\
209.26626	-47.607\\
270.217332	-65.918\\
380.215024	-78.125\\
464.921875	-102.539\\
612.055291	-96.436\\
580.930464	-112.305\\
693.034155	-130.615\\
803.54348	-119.629\\
722.798418	-86.67\\
522.10008	-92.773\\
555.524724	-83.008\\
498.546048	-91.553\\
563.234056	-126.953\\
774.032441	-93.994\\
571.389526	-106.201\\
643.684261	-91.553\\
553.163226	-87.891\\
545.539437	-140.381\\
868.818009	-113.525\\
702.606225	-115.967\\
715.632357	-103.76\\
619.34344	-64.697\\
370.778507	-42.725\\
237.80735	-34.18\\
194.00568	-61.035\\
347.594325	-54.932\\
317.836552	-74.463\\
428.16225	-56.152\\
326.973096	-79.346\\
482.344334	-124.512\\
784.301088	-155.029\\
968.001076	-122.07\\
762.20508	-128.174\\
795.576018	-113.525\\
688.075025	-83.008\\
501.534336	-87.891\\
534.289389	-98.877\\
599.293497	-84.229\\
512.028091	-101.318\\
627.057102	-126.953\\
811.22967	-178.223\\
1148.647235	-156.25\\
998.4375	-147.705\\
949.300035	-164.795\\
1038.043705	-109.863\\
690.049503	-120.85\\
747.94065	-93.994\\
578.251088	-95.215\\
594.52246	-126.953\\
804.247255	-144.043\\
865.122258	-56.152\\
331.072192	-74.463\\
437.693514	-57.373\\
340.394009	-83.008\\
497.051904	-84.229\\
491.981589	-51.27\\
301.36506	-68.359\\
418.083644	-119.629\\
775.435178	-196.533\\
1299.08313	-190.43\\
1248.26865	-174.561\\
1128.362304	-137.939\\
896.6035	-161.133\\
1065.08913	-196.533\\
1281.198627	-144.043\\
933.686726	-148.926\\
968.019	-153.809\\
977.302386	-104.98\\
653.60548	-91.553\\
576.692347	-118.408\\
730.695768	-79.346\\
475.123848	-59.814\\
360.319536	-80.566\\
479.448266	-59.814\\
355.953114	-73.242\\
435.863142	-67.139\\
403.236834	-84.229\\
516.660686	-111.084\\
677.279148	-85.449\\
510.045081	-64.697\\
386.176393	-76.904\\
464.653968	-89.111\\
545.002876	-107.422\\
654.951934	-89.111\\
535.200666	-74.463\\
458.096376	-115.967\\
717.719763	-108.643\\
658.485223	-75.684\\
469.770588	-126.953\\
792.694532	-113.525\\
694.3189	-83.008\\
493.980608	-57.373\\
330.927464	-42.725\\
239.388175	-31.738\\
183.636068	-69.58\\
406.41678	-63.477\\
363.786687	-47.607\\
264.123636	-34.18\\
189.63064	-42.725\\
245.66875	-70.801\\
414.823059	-81.787\\
491.212722	-104.98\\
643.94732	-122.07\\
753.29397	-117.188\\
725.276532	-122.07\\
730.95516	-72.021\\
403.533663	-34.18\\
190.8953	-50.049\\
279.523665	-41.504\\
234.08256	-53.711\\
313.725951	-86.67\\
515.77317	-95.215\\
589.285635	-141.602\\
866.037832	-92.773\\
558.864552	-87.891\\
542.375361	-128.174\\
781.476878	-93.994\\
557.666402	-65.918\\
380.215024	-50.049\\
290.484396	-65.918\\
391.091494	-87.891\\
540.705432	-129.395\\
786.592205	-86.67\\
514.21311	-74.463\\
440.374182	-72.021\\
431.261748	-86.67\\
528.42699	-113.525\\
727.581725	-179.443\\
1153.280161	-151.367\\
972.835709	-151.367\\
942.410942	-97.656\\
586.521936	-64.697\\
387.405636	-76.904\\
435.122832	-36.621\\
207.201618	-54.932\\
311.794032	-51.27\\
290.08566	-54.932\\
324.867848	-93.994\\
578.251088	-128.174\\
807.368026	-146.484\\
946.872576	-189.209\\
1229.8585	-161.133\\
1003.214058	-86.67\\
528.42699	-81.787\\
491.212722	-68.359\\
414.323899	-91.553\\
566.621517	-120.85\\
770.0562	-161.133\\
1012.076373	-113.525\\
694.3189	-80.566\\
494.191844	-92.773\\
579.274612	-124.512\\
765.997824	-87.891\\
524.621379	-63.477\\
369.626571	-51.27\\
289.1628	-37.842\\
209.26626	-34.18\\
183.99094	-29.297\\
160.928421	-47.607\\
270.217332	-64.697\\
373.172296	-73.242\\
418.431546	-56.152\\
316.69728	-52.49\\
281.60885	-26.855\\
135.214925	-15.869\\
79.329131	-23.193\\
121.044267	-41.504\\
227.2344	-62.256\\
351.12384	-76.904\\
443.582272	-86.67\\
511.00632	-101.318\\
617.735846	-131.836\\
842.43204	-181.885\\
1182.2525	-179.443\\
1176.248865	-194.092\\
1268.779404	-169.678\\
1059.469432	-93.994\\
574.867304	-86.67\\
528.42699	-85.449\\
530.381943	-113.525\\
702.606225	-96.436\\
579.194616	-70.801\\
416.168278	-56.152\\
323.884736	-46.387\\
273.497752	-76.904\\
456.271432	-76.904\\
437.96828	-42.725\\
235.457475	-32.959\\
183.449794	-50.049\\
285.929937	-67.139\\
379.872462	-51.27\\
295.72536	-73.242\\
418.431546	-54.932\\
318.825328	-81.787\\
498.655339	-122.07\\
739.86627	-96.436\\
593.274272	-128.174\\
819.03186	-177.002\\
1157.062074	-189.209\\
1223.046976	-148.926\\
932.574612	-95.215\\
575.28903	-73.242\\
423.778212	-45.166\\
262.143464	-68.359\\
393.06425	-47.607\\
281.547798	-85.449\\
508.506999	-79.346\\
464.888214	-62.256\\
369.364848	-81.787\\
471.747416	-47.607\\
276.311028	-69.58\\
400.085	-54.932\\
304.762736	-34.18\\
189.0154	-39.063\\
212.424594	-32.959\\
178.63778	-37.842\\
204.422484	-32.959\\
173.199545	-24.414\\
128.29557	-31.738\\
170.27437	-45.166\\
244.79972	-43.945\\
237.39089	-43.945\\
245.432825	-68.359\\
396.755636	-90.332\\
522.660952	-72.021\\
414.12075	-67.139\\
397.060046	-104.98\\
640.06306	-125.732\\
759.672744	-100.098\\
606.693978	-104.98\\
605.52464	-51.27\\
280.70325	-32.959\\
185.262539	-62.256\\
364.757904	-91.553\\
544.831903	-96.436\\
591.538424	-135.498\\
836.158158	-120.85\\
732.47185	-86.67\\
507.79953	-61.035\\
354.24714	-64.697\\
377.895177	-73.242\\
422.459856	-56.152\\
313.60892	-35.4\\
196.3992	-45.166\\
248.096838	-37.842\\
204.422484	-31.738\\
166.78319	-25.635\\
137.531775	-45.166\\
250.580968	-62.256\\
351.12384	-68.359\\
379.255732	-50.049\\
286.830819	-81.787\\
491.212722	-115.967\\
681.654026	-74.463\\
441.788979	-100.098\\
606.693978	-112.305\\
668.327055	-83.008\\
475.718848	-48.828\\
278.07546	-62.256\\
360.213216	-74.463\\
418.556523	-45.166\\
252.25211	-48.828\\
269.091108	-41.504\\
220.38624	-24.414\\
127.856118	-24.414\\
130.541658	-40.283\\
221.274519	-54.932\\
300.7527	-48.828\\
270.897744	-59.814\\
337.35096	-69.58\\
386.02984	-50.049\\
268.512885	-30.518\\
159.273442	-24.414\\
127.416666	-31.738\\
167.957496	-36.621\\
195.812487	-43.945\\
250.266775	-86.67\\
506.23947	-89.111\\
515.596246	-76.904\\
443.582272	-74.463\\
437.693514	-101.318\\
615.912122	-128.174\\
793.268886	-137.939\\
858.808214	-140.381\\
861.097054	-103.76\\
632.62472	-108.643\\
666.416162	-111.084\\
681.389256	-112.305\\
695.055645	-126.953\\
806.659362	-169.678\\
1078.134012	-144.043\\
907.326857	-131.836\\
820.810936	-104.98\\
647.83158	-101.318\\
623.308336	-96.436\\
598.578252	-114.746\\
693.295332	-72.021\\
431.261748	-83.008\\
504.605632	-102.539\\
632.768169	-119.629\\
729.378013	-91.553\\
559.938148	-106.201\\
645.595879	-85.449\\
513.206694	-75.684\\
440.707932	-50.049\\
295.088904	-74.463\\
456.756042	-122.07\\
744.26079	-96.436\\
566.850808	-56.152\\
328.994568	-69.58\\
393.68364	-41.504\\
228.728544	-35.4\\
197.709	-50.049\\
274.018275	-31.738\\
171.448676	-35.4\\
192.5052	-42.725\\
226.86975	-28.076\\
152.17192	-43.945\\
236.555935	-40.283\\
210.962071	-24.414\\
130.077792	-39.063\\
210.276129	-46.387\\
253.968825	-53.711\\
295.034523	-53.711\\
295.034523	-53.711\\
302.93004	-74.463\\
419.97132	-64.697\\
367.220172	-72.021\\
432.558126	-124.512\\
745.577856	-89.111\\
523.794458	-76.904\\
454.810256	-85.449\\
494.407914	-61.035\\
338.62218	-37.842\\
214.791192	-65.918\\
372.964044	-57.373\\
314.117175	-32.959\\
182.856532	-58.594\\
328.302182	-54.932\\
310.805256	-67.139\\
371.27867	-45.166\\
240.644448	-28.076\\
145.9952	-23.193\\
119.768652	-26.855\\
144.077075	-48.828\\
265.526664	-48.828\\
268.212204	-59.814\\
337.35096	-76.904\\
450.580536	-102.539\\
597.084597	-76.904\\
454.810256	-102.539\\
621.488879	-122.07\\
737.54694	-104.98\\
630.50988	-98.877\\
619.167774	-162.354\\
1010.816004	-118.408\\
743.720648	-142.822\\
897.064982	-123.291\\
776.610009	-145.264\\
928.23696	-162.354\\
1004.808906	-96.436\\
570.322504	-58.594\\
335.802214	-48.828\\
286.083252	-74.463\\
444.469647	-93.994\\
562.836072	-95.215\\
561.38764	-65.918\\
369.338554	-36.621\\
203.832486	-46.387\\
273.497752	-92.773\\
569.069582	-125.732\\
775.892172	-124.512\\
743.212128	-74.463\\
432.183252	-57.373\\
336.148407	-76.904\\
467.499416	-122.07\\
757.68849	-136.719\\
853.673436	-139.16\\
853.60744	-97.656\\
609.764064	-137.939\\
873.843565	-152.588\\
966.64498	-142.822\\
928.343	-197.754\\
1289.158326	-169.678\\
1093.57471	-141.602\\
917.864164	-164.795\\
1062.103775	-140.381\\
866.291151	-78.125\\
469.21875	-72.021\\
437.815659	-89.111\\
549.903981	-108.643\\
672.391527	-111.084\\
677.279148	-81.787\\
501.681458	-102.539\\
628.974226	-91.553\\
549.867318	-65.918\\
398.276556	-89.111\\
540.101771	-84.229\\
521.293281	-124.512\\
782.059872	-128.174\\
790.961754	-91.553\\
551.515272	-70.801\\
409.654586	-42.725\\
249.556725	-64.697\\
373.172296	-54.932\\
311.794032	-46.387\\
261.62268	-39.063\\
224.61225	-65.918\\
388.652528	-86.67\\
517.33323	-95.215\\
570.14742	-95.215\\
554.436945	-57.373\\
325.649148	-42.725\\
235.457475	-29.297\\
160.401075	-37.842\\
214.110036	-63.477\\
362.644101	-65.918\\
377.776058	-64.697\\
383.847301	-102.539\\
625.180283	-122.07\\
755.49123	-134.277\\
838.425588	-135.498\\
858.37983	-157.471\\
971.753541	-97.656\\
586.521936	-80.566\\
477.998078	-61.035\\
358.76373	-67.139\\
402.028332	-90.332\\
539.191708	-75.684\\
437.907624	-51.27\\
291.98265	-46.387\\
272.662786	-83.008\\
500.040192	-102.539\\
615.849234	-90.332\\
564.033008	-147.705\\
941.17626	-161.133\\
1009.014846	-112.305\\
680.680605	-79.346\\
466.395788	-54.932\\
316.847776	-50.049\\
297.841599	-90.332\\
537.565732	-76.904\\
460.501152	-86.67\\
522.10008	-98.877\\
601.073283	-107.422\\
654.951934	-103.76\\
636.46384	-117.188\\
705.940512	-81.787\\
491.212722	-89.111\\
528.695563	-65.918\\
383.840514	-58.594\\
349.747586	-90.332\\
555.722464	-125.732\\
787.333784	-135.498\\
818.678916	-80.566\\
507.485234	-155.029\\
993.580861	-145.264\\
901.653648	-96.436\\
570.322504	-56.152\\
335.171288	-83.008\\
497.051904	-75.684\\
449.033172	-70.801\\
423.956388	-85.449\\
502.269222	-58.594\\
347.638202	-79.346\\
496.864652	-145.264\\
925.622208	-151.367\\
939.534969	-98.877\\
611.949753	-108.643\\
678.366892	-122.07\\
762.20508	-123.291\\
760.828761	-90.332\\
550.754204	-87.891\\
523.039341	-61.035\\
366.57621	-89.111\\
541.705769	-95.215\\
598.045415	-146.484\\
941.452668	-166.016\\
1106.49664	-228.271\\
1508.87131	-163.574\\
1057.342336	-131.836\\
832.808012	-103.76\\
655.45192	-109.863\\
684.007038	-85.449\\
511.668612	-59.814\\
353.739996	-57.373\\
339.303922	-59.814\\
346.083804	-46.387\\
259.071395	-32.959\\
185.88876	-51.27\\
296.64822	-67.139\\
383.565107	-47.607\\
264.123636	-34.18\\
187.75074	-32.959\\
189.51425	-76.904\\
463.269696	-109.863\\
635.667318	-52.49\\
301.8175	-62.256\\
360.213216	-68.359\\
395.525174	-61.035\\
356.505435	-76.904\\
447.811992	-67.139\\
382.356605	-47.607\\
264.123636	-34.18\\
185.87084	-29.297\\
159.873729	-39.063\\
223.870053	-73.242\\
429.124878	-87.891\\
508.537326	-62.256\\
351.12384	-43.945\\
238.1819	-29.297\\
158.78974	-41.504\\
234.08256	-64.697\\
375.501388	-84.229\\
495.098062	-89.111\\
513.992248	-63.477\\
362.644101	-59.814\\
343.9305	-67.139\\
394.643042	-91.553\\
556.550687	-126.953\\
792.694532	-146.484\\
906.589476	-111.084\\
663.060396	-69.58\\
411.49612	-75.684\\
446.232864	-69.58\\
410.24368	-72.021\\
415.417128	-50.049\\
287.78175	-59.814\\
347.160456	-67.139\\
389.674756	-63.477\\
359.152866	-41.504\\
225.698752	-29.297\\
160.401075	-41.504\\
234.829632	-63.477\\
373.117806	-93.994\\
562.836072	-100.098\\
610.297506	-122.07\\
746.58012	-109.863\\
673.899642	-123.291\\
781.048485	-163.574\\
1078.279808	-217.285\\
1440.16498	-179.443\\
1153.280161	-119.629\\
762.275988	-130.615\\
844.29536	-153.809\\
1013.908928	-197.754\\
1278.281856	-128.174\\
779.169746	-62.256\\
360.213216	-43.945\\
249.43182	-45.166\\
251.393956	-32.959\\
175.01229	-21.973\\
115.863629	-28.076\\
151.133108	-42.725\\
237.80735	-61.035\\
343.077735	-59.814\\
331.848072	-47.607\\
261.505251	-45.166\\
251.393956	-56.152\\
318.718752	-69.58\\
398.76298	-73.242\\
418.431546	-69.58\\
389.85674	-48.828\\
266.454396	-35.4\\
190.5582	-34.18\\
187.75074	-50.049\\
280.424547	-63.477\\
349.821747	-46.387\\
248.866255	-32.959\\
181.043787	-52.49\\
285.44062	-40.283\\
219.058954	-43.945\\
243.80686	-56.152\\
320.796376	-79.346\\
462.031758	-86.67\\
493.58565	-59.814\\
348.296922	-90.332\\
530.971496	-87.891\\
511.789293	-69.58\\
400.085	-61.035\\
358.76373	-97.656\\
593.650824	-123.291\\
754.047756	-119.629\\
738.230559	-133.057\\
813.776612	-102.539\\
621.488879	-92.773\\
557.194638	-87.891\\
537.541356	-118.408\\
715.421136	-86.67\\
531.63378	-120.85\\
745.76535	-115.967\\
717.719763	-118.408\\
769.652	-203.857\\
1332.613209	-161.133\\
1003.214058	-85.449\\
508.506999	-58.594\\
345.470224	-62.256\\
371.606064	-79.346\\
480.916106	-101.318\\
625.233378	-114.746\\
716.474024	-128.174\\
802.625588	-131.836\\
806.308976	-86.67\\
522.10008	-76.904\\
467.499416	-89.111\\
541.705769	-89.111\\
525.398456	-56.152\\
319.78564	-40.283\\
233.802532	-70.801\\
414.823059	-70.801\\
423.956388	-93.994\\
578.251088	-115.967\\
709.254172	-107.422\\
645.176532	-74.463\\
437.693514	-61.035\\
365.47758	-90.332\\
554.096488	-118.408\\
747.983336	-153.809\\
991.299005	-169.678\\
1109.185086	-194.092\\
1261.598	-157.471\\
1014.900595	-141.602\\
881.614052	-84.229\\
502.762901	-62.256\\
381.878304	-104.98\\
661.26902	-139.16\\
856.11232	-81.787\\
510.678028	-125.732\\
812.731648	-172.119\\
1137.70659	-206.299\\
1333.516736	-129.395\\
798.496545	-74.463\\
433.598049	-42.725\\
247.9759	-64.697\\
377.895177	-59.814\\
349.373574	-62.256\\
353.365056	-37.842\\
214.791192	-52.49\\
294.10147	-41.504\\
234.08256	-52.49\\
295.04629	-43.945\\
248.64081	-54.932\\
321.846588	-85.449\\
505.345386	-79.346\\
479.408532	-112.305\\
697.077135	-133.057\\
830.807908	-133.057\\
801.535368	-74.463\\
441.788979	-76.904\\
463.269696	-91.553\\
539.796488	-62.256\\
361.333824	-56.152\\
318.718752	-42.725\\
246.4378	-63.477\\
367.277922	-59.814\\
334.06119	-34.18\\
182.11104	-24.414\\
126.9528	-24.414\\
130.541658	-41.504\\
230.264192	-61.035\\
343.077735	-65.918\\
363.274098	-41.504\\
228.728544	-51.27\\
291.00852	-73.242\\
429.124878	-93.994\\
545.541176	-65.918\\
377.776058	-62.256\\
360.213216	-74.463\\
436.278717	-87.891\\
514.953369	-80.566\\
476.467324	-92.773\\
548.659522	-86.67\\
501.47262	-62.256\\
353.365056	-48.828\\
278.954364	-62.256\\
353.365056	-53.711\\
307.817741	-64.697\\
380.288966	-87.891\\
539.123394	-139.16\\
848.45852	-98.877\\
595.635048	-95.215\\
592.80859	-140.381\\
866.291151	-108.643\\
646.534493	-67.139\\
387.257752	-48.828\\
275.38992	-40.283\\
225.705649	-42.725\\
235.457475	-35.4\\
193.1778	-36.621\\
207.860796	-67.139\\
367.586025	-35.4\\
187.974	-34.18\\
186.52026	-47.607\\
266.742021	-65.918\\
368.15203	-50.049\\
273.117393	-37.842\\
199.540866	-23.193\\
123.572304	-39.063\\
220.31532	-72.021\\
421.971039	-96.436\\
561.546828	-79.346\\
454.731926	-63.477\\
364.99275	-72.021\\
414.12075	-72.021\\
418.009884	-80.566\\
475.017136	-98.877\\
595.635048	-117.188\\
712.385852	-114.746\\
691.229904	-95.215\\
559.67377	-64.697\\
370.778507	-52.49\\
299.87537	-56.152\\
326.973096	-76.904\\
437.96828	-47.607\\
265.885095	-43.945\\
253.47476	-80.566\\
477.998078	-96.436\\
572.154788	-91.553\\
548.219364	-104.98\\
645.83696	-137.939\\
823.357891	-74.463\\
458.096376	-139.16\\
871.41992	-129.395\\
805.61327	-120.85\\
752.4121	-128.174\\
800.318456	-125.732\\
821.910084	-212.402\\
1415.65933	-198.975\\
1293.3375	-134.277\\
875.351763	-159.912\\
1059.896736	-194.092\\
1290.129524	-190.43\\
1276.26186	-212.402\\
1427.34144	-194.092\\
1325.64836	-253.906\\
1724.783458	-195.313\\
1291.01893	-135.498\\
858.37983	-81.787\\
507.651909	-84.229\\
516.660686	-68.359\\
408.034871	-57.373\\
348.770467	-85.449\\
535.081638	-120.85\\
741.2939	-76.904\\
461.885424	-67.139\\
403.236834	-65.918\\
382.588072	-46.387\\
269.230148	-56.152\\
311.531296	-29.297\\
159.873729	-39.063\\
211.72146	-29.297\\
160.401075	-46.387\\
252.252506	-32.959\\
175.605552	-30.518\\
158.144276	-19.531\\
99.061232	-18.311\\
94.558004	-34.18\\
175.8561	-25.635\\
130.969215	-28.076\\
151.666552	-57.373\\
323.58372	-78.125\\
449.21875	-80.566\\
454.39224	-58.594\\
333.69283	-74.463\\
444.469647	-114.746\\
657.609326	-53.711\\
293.100927	-37.842\\
200.259864	-28.076\\
143.945652	-20.752\\
108.304688	-34.18\\
185.87084	-56.152\\
319.78564	-84.229\\
471.935087	-50.049\\
287.78175	-85.449\\
514.744776	-118.408\\
719.802232	-120.85\\
725.8251	-89.111\\
515.596246	-57.373\\
331.960178	-72.021\\
427.300593	-96.436\\
586.234444	-124.512\\
736.363968	-68.359\\
381.785015	-37.842\\
214.110036	-57.373\\
320.428205	-47.607\\
253.650096	-23.193\\
117.217422	-18.311\\
95.2172	-39.063\\
221.721588	-85.449\\
497.569527	-85.449\\
488.170137	-63.477\\
351.02781	-41.504\\
226.487328	-43.945\\
222.88904	-15.869\\
79.900415	-30.518\\
154.787296	-25.635\\
132.84057	-42.725\\
230.80045	-58.594\\
332.579544	-89.111\\
520.497351	-93.994\\
552.496732	-100.098\\
595.683198	-104.98\\
624.73598	-102.539\\
610.209589	-100.098\\
591.979572	-86.67\\
518.97996	-104.98\\
653.60548	-157.471\\
994.744307	-144.043\\
867.715032	-74.463\\
418.556523	-40.283\\
236.018097	-86.67\\
523.66014	-106.201\\
639.754824	-96.436\\
559.714544	-56.152\\
320.796376	-57.373\\
340.394009	-97.656\\
606.150792	-146.484\\
927.97614	-150.146\\
962.285714	-169.678\\
1099.852796	-164.795\\
1047.10743	-120.85\\
763.40945	-123.291\\
738.266508	-58.594\\
339.024884	-56.152\\
328.994568	-72.021\\
428.596971	-87.891\\
527.873346	-91.553\\
551.515272	-96.436\\
566.850808	-63.477\\
376.609041	-86.67\\
512.56638	-69.58\\
397.51054	-41.504\\
225.698752	-29.297\\
154.483081	-24.414\\
130.98111	-37.842\\
200.94102	-30.518\\
156.465786	-18.311\\
95.894707	-39.063\\
217.424658	-74.463\\
411.78039	-48.828\\
265.526664	-45.166\\
250.580968	-61.035\\
355.406805	-100.098\\
580.968792	-68.359\\
376.726449	-37.842\\
198.178554	-24.414\\
132.763332	-53.711\\
315.713258	-108.643\\
668.371736	-136.719\\
878.693013	-189.209\\
1261.077985	-223.389\\
1505.195082	-220.947\\
1448.307585	-142.822\\
928.343	-147.705\\
981.795135	-191.65\\
1266.8065	-159.912\\
1048.22316	-147.705\\
973.67136	-168.457\\
1119.733679	-183.105\\
1220.394825	-185.547\\
1270.625856	-253.906\\
1710.818628	-168.457\\
1082.673139	-95.215\\
577.098115	-56.152\\
325.906208	-45.166\\
260.517488	-50.049\\
286.830819	-48.828\\
279.833268	-48.828\\
289.696524	-86.67\\
515.77317	-65.918\\
388.652528	-73.242\\
441.209808	-92.773\\
543.557007	-53.711\\
311.738644	-64.697\\
383.847301	-84.229\\
510.511969	-98.877\\
579.320343	-56.152\\
317.708016	-39.063\\
221.018454	-50.049\\
283.177242	-48.828\\
294.139872	-113.525\\
723.3813	-155.029\\
990.63531	-145.264\\
941.601248	-168.457\\
1107.267861	-189.209\\
1285.296737	-246.582\\
1684.15506	-208.74\\
1383.52872	-135.498\\
860.954292	-89.111\\
533.596668	-51.27\\
300.39093	-58.594\\
344.415532	-58.594\\
348.692894	-76.904\\
452.041712	-53.711\\
310.771846	-50.049\\
289.583514	-53.711\\
314.692749	-69.58\\
411.49612	-76.904\\
463.269696	-95.215\\
566.624465	-68.359\\
386.775222	-32.959\\
179.231042	-25.635\\
134.711925	-23.193\\
120.6036	-25.635\\
133.789065	-30.518\\
165.956884	-53.711\\
296.001321	-52.49\\
294.10147	-65.918\\
385.027038	-95.215\\
552.62786	-68.359\\
409.333692	-113.525\\
721.33785	-164.795\\
1047.10743	-129.395\\
812.729995	-115.967\\
736.854318	-140.381\\
907.422784	-162.354\\
1025.590218	-108.643\\
674.347101	-93.994\\
562.836072	-59.814\\
354.876462	-65.918\\
410.405468	-131.836\\
876.313892	-214.844\\
1483.068132	-256.348\\
1760.085368	-202.637\\
1365.368106	-168.457\\
1107.267861	-123.291\\
803.734029	-133.057\\
886.824905	-175.781\\
1132.908545	-93.994\\
586.898536	-83.008\\
509.171072	-63.477\\
399.841623	-115.967\\
741.02913	-112.305\\
686.85738	-61.035\\
365.47758	-62.256\\
365.940768	-46.387\\
278.600322	-84.229\\
525.925876	-119.629\\
757.849715	-126.953\\
792.694532	-95.215\\
578.811985	-69.58\\
422.97682	-80.566\\
483.879396	-61.035\\
357.604065	-46.387\\
264.173965	-36.621\\
204.528285	-32.959\\
179.857263	-29.297\\
158.78974	-34.18\\
191.51054	-59.814\\
349.373574	-84.229\\
498.130306	-83.008\\
478.790144	-52.49\\
297.93324	-47.607\\
265.885095	-40.283\\
225.705649	-52.49\\
301.8175	-68.359\\
396.755636	-73.242\\
425.096568	-69.58\\
391.10918	-40.283\\
235.293003	-83.008\\
509.171072	-120.85\\
732.47185	-84.229\\
476.567682	-35.4\\
197.0364	-40.283\\
230.861873	-67.139\\
388.466254	-67.139\\
386.04925	-58.594\\
326.134204	-37.842\\
215.51019	-68.359\\
406.804409	-97.656\\
568.650888	-59.814\\
325.268532	-24.414\\
130.98111	-37.842\\
216.872502	-81.787\\
486.714437	-96.436\\
561.546828	-59.814\\
339.504264	-53.711\\
316.680056	-97.656\\
599.021904	-125.732\\
773.503264	-108.643\\
684.342257	-152.588\\
994.721172	-184.326\\
1177.84314	-115.967\\
722.010542	-95.215\\
604.99611	-139.16\\
871.41992	-93.994\\
592.068206	-126.953\\
818.212085	-152.588\\
969.544152	-119.629\\
725.071369	-57.373\\
350.893268	-101.318\\
654.919552	-172.119\\
1140.804732	-195.313\\
1276.761081	-142.822\\
920.48779	-120.85\\
787.82115	-151.367\\
975.560315	-118.408\\
737.208208	-76.904\\
468.883688	-73.242\\
450.584784	-91.553\\
566.621517	-95.215\\
589.285635	-93.994\\
585.206644	-104.98\\
643.94732	-72.021\\
437.815659	-79.346\\
494.008196	-109.863\\
667.857177	-64.697\\
381.453512	-53.711\\
310.771846	-43.945\\
247.8498	-35.4\\
198.9834	-42.725\\
248.787675	-73.242\\
429.124878	-65.918\\
393.464542	-93.994\\
580.036974	-114.746\\
722.785054	-139.16\\
866.41016	-102.539\\
628.974226	-92.773\\
540.217179	-43.945\\
254.26577	-56.152\\
342.358744	-107.422\\
647.110128	-73.242\\
419.749902	-42.725\\
251.13755	-78.125\\
480.625	-115.967\\
722.010542	-120.85\\
770.0562	-157.471\\
1043.717788	-205.078\\
1340.594886	-136.719\\
876.232071	-118.408\\
765.389312	-142.822\\
902.206574	-92.773\\
570.739496	-76.904\\
475.958856	-85.449\\
535.081638	-111.084\\
703.71714	-118.408\\
750.11468	-119.629\\
733.804286	-67.139\\
402.028332	-59.814\\
352.663344	-52.49\\
312.36799	-73.242\\
437.181498	-65.918\\
387.466004	-52.49\\
321.02884	-104.98\\
653.60548	-111.084\\
679.389744	-78.125\\
467.8125	-64.697\\
393.293063	-96.436\\
573.890636	-53.711\\
304.863636	-32.959\\
185.262539	-41.504\\
237.859424	-57.373\\
327.771949	-48.828\\
272.70438	-35.4\\
188.6112	-21.973\\
118.280659	-40.283\\
224.215178	-54.932\\
320.857812	-86.67\\
520.54002	-96.436\\
595.106556	-126.953\\
801.962101	-142.822\\
876.070148	-81.787\\
483.688318	-62.256\\
385.302384	-128.174\\
835.566306	-181.885\\
1178.97857	-139.16\\
894.38132	-130.615\\
861.01408	-196.533\\
1291.811409	-155.029\\
1002.107456	-128.174\\
809.675158	-93.994\\
590.376314	-98.877\\
630.044244	-130.615\\
815.56006	-85.449\\
524.144166	-74.463\\
470.382771	-128.174\\
833.131	-158.691\\
1040.219505	-169.678\\
1059.469432	-73.242\\
426.488166	-39.063\\
238.167111	-93.994\\
571.389526	-73.242\\
418.431546	-35.4\\
194.4522	-29.297\\
163.623745	-50.049\\
289.583514	-67.139\\
406.929479	-109.863\\
661.814712	-76.904\\
452.041712	-61.035\\
346.43466	-36.621\\
198.48582	-26.855\\
145.5541	-36.621\\
197.826642	-30.518\\
165.956884	-40.283\\
229.411685	-68.359\\
393.06425	-61.035\\
335.265255	-32.959\\
176.825035	-30.518\\
165.40756	-39.063\\
215.276193	-47.607\\
256.268481	-29.297\\
159.873729	-47.607\\
257.173014	-32.959\\
173.199545	-24.414\\
133.66665	-56.152\\
321.807112	-78.125\\
470.625	-115.967\\
732.563539	-156.25\\
992.8125	-137.939\\
836.048279	-67.139\\
388.466254	-51.27\\
294.8025	-51.27\\
282.54897	-29.297\\
157.178405	-30.518\\
169.863188	-57.373\\
325.649148	-62.256\\
352.244448	-57.373\\
326.739235	-65.918\\
380.215024	-76.904\\
450.580536	-80.566\\
473.566948	-83.008\\
497.051904	-103.76\\
626.91792	-100.098\\
625.011912	-145.264\\
877.685088	-76.904\\
435.122832	-37.842\\
209.26626	-36.621\\
209.215773	-67.139\\
383.565107	-50.049\\
281.325429	-47.607\\
255.411555	-23.193\\
124.012971	-41.504\\
218.850592	-25.635\\
129.55929	-15.869\\
80.487568	-25.635\\
136.12185	-47.607\\
260.648325	-58.594\\
334.747522	-84.229\\
507.395496	-118.408\\
709.027104	-84.229\\
470.418965	-35.4\\
200.9304	-65.918\\
382.588072	-74.463\\
413.120724	-39.063\\
216.721524	-51.27\\
299.46807	-92.773\\
546.989608	-81.787\\
503.153624	-137.939\\
876.464406	-157.471\\
983.248924	-107.422\\
653.018338	-83.008\\
497.051904	-74.463\\
};
\addplot [color=mycolor2, line width=2.0pt, forget plot]
  table[row sep=crcr]{%
515.144564	-79.7682729469771\\
600.410536	-92.9714003851719\\
755.49123	-116.985085071552\\
614.001132	-95.0758550314993\\
588.590428	-91.1410994034716\\
817.91113	-126.650580873082\\
798.901128	-123.706950804497\\
968.842891	-150.02181828716\\
694.410396	-107.526938797992\\
362.120655	-56.0730739804101\\
441.788979	-68.4094258671791\\
439.891452	-68.1156006727814\\
515.77317	-79.8656103130064\\
410.374926	-63.5450733549108\\
180.88056	-28.0087006428783\\
106.76904	-16.532799761829\\
127.399149	-19.7272975409765\\
344.415532	-53.3314997064675\\
601.188588	-93.0918789238971\\
663.792246	-102.785828987249\\
682.853446	-105.737386881\\
474.224704	-73.4319805942307\\
312.36799	-48.3690537132769\\
620.85172	-96.1366438112316\\
470.27525	-72.8204219237532\\
354.876462	-54.9513368897231\\
581.150856	-89.9891085811348\\
486.343872	-75.308589935515\\
435.150882	-67.3815405504188\\
719.802232	-111.458772784448\\
641.201918	-99.2877983842012\\
504.605632	-78.1363574360805\\
733.804286	-113.626940214222\\
747.266751	-115.711554238528\\
568.338335	-88.0051360349995\\
472.036194	-73.0930977344919\\
365.940768	-56.6646046647411\\
450.395484	-69.742112044878\\
568.338335	-88.0051360349995\\
524.621379	-81.2357234810916\\
548.219364	-84.8897861268515\\
579.194616	-89.6861918909962\\
632.768169	-97.981862848408\\
878.923614	-136.098143396298\\
776.862614	-120.294366603962\\
510.045081	-78.9786364444118\\
456.2395	-70.6470367902249\\
268.50348	-41.5767929560317\\
280.761	-43.4748256042283\\
389.304505	-60.2823948547534\\
296.98842	-45.9875829120687\\
331.960178	-51.4028331787484\\
444.870552	-68.8865963031076\\
523.794458	-81.1076777544292\\
497.183173	-76.9870164999123\\
774.427308	-119.91726827605\\
573.882108	-88.86356923749\\
324.616434	-50.2656809756928\\
257.22037	-39.8296441728199\\
360.213216	-55.7777139486546\\
402.237285	-62.2849890721763\\
221.274519	-34.263559127459\\
239.388175	-37.0683932591759\\
312.542032	-48.3960034876323\\
253.968825	-39.3261541873187\\
223.490084	-34.6066313561155\\
294.10147	-45.5405491439238\\
363.637296	-56.3079203550107\\
545.319694	-84.4407827127028\\
539.191708	-83.4918862397056\\
643.242936	-99.6038426410783\\
595.635048	-92.2319333367745\\
694.410396	-107.526938797992\\
544.831903	-84.3652500402293\\
541.444442	-83.8407139536072\\
467.824016	-72.4408572026375\\
444.870552	-68.8865963031076\\
467.499416	-72.3905940664073\\
847.839204	-131.284834910574\\
1229.8585	-190.439141495357\\
1265.285748	-195.924922741462\\
1198.639015	-185.604917134321\\
768.855583	-119.054506807435\\
1126.559616	-174.443682841871\\
1423.797295	-220.469862608756\\
1448.440384	-224.285755829768\\
1130.90387	-175.116374864732\\
1512.820395	-234.254767731789\\
1904.233292	-294.863639463692\\
1321.589982	-204.643429777445\\
1041.563712	-161.282374456896\\
687.962106	-106.52844440883\\
524.144166	-81.1618287736676\\
450.584784	-69.7714244653598\\
551.515272	-85.4001455624839\\
360.213216	-55.7777139486546\\
262.457646	-40.6406174231688\\
244.087925	-37.7961325521456\\
322.890296	-49.9983947481939\\
467.824016	-72.4408572026375\\
439.033848	-67.9828037945202\\
454.558104	-70.3866786996366\\
608.495742	-94.2233652977639\\
647.87744	-100.321478826881\\
886.711724	-137.304104067604\\
722.785054	-111.920651985115\\
889.209772	-137.690917767338\\
642.05768	-99.4203100354231\\
555.722464	-86.0516140302679\\
628.88936	-97.3812433007247\\
459.039976	-71.0806803721641\\
419.14992	-64.903849444995\\
508.911618	-78.8031237975205\\
514.21311	-79.6240406632612\\
352.663344	-54.6086435701681\\
374.336842	-57.964706363019\\
279.833268	-43.3311696587535\\
309.805048	-47.9721914123029\\
325.649148	-50.4255930042475\\
235.576704	-36.478200757295\\
302.76232	-46.8816504483584\\
394.716984	-61.1204976561096\\
612.163356	-94.7912820633889\\
642.05768	-99.4203100354231\\
680.788018	-105.41756282389\\
409.333692	-63.3838420351648\\
586.898536	-90.8791160452374\\
934.808996	-144.751792714671\\
716.474024	-110.943411810614\\
787.333784	-121.915789414189\\
779.169746	-120.651617651458\\
470.586006	-72.868541366644\\
316.680056	-49.0367615407313\\
451.757796	-69.9530611318899\\
530.381943	-82.127725986005\\
884.051051	-136.89210847468\\
1112.577216	-172.278558762926\\
1082.63883	-167.642707949522\\
781.048485	-120.942533591994\\
822.743885	-127.398915509349\\
726.185354	-112.44717614446\\
710.89355	-110.079295590986\\
677.279148	-104.874226992639\\
460.501152	-71.3069381920788\\
404.275126	-62.6005413820916\\
356.505435	-55.2035774683256\\
336.148407	-52.0513652945536\\
373.909536	-57.8985395740788\\
570.008978	-88.2638290624211\\
876.538964	-135.728888966811\\
673.280124	-104.254992586909\\
472.188046	-73.1166114675875\\
381.453512	-59.0666969782855\\
354.54792	-54.9004633490474\\
220.549425	-34.151280943535\\
159.873729	-24.7558688242764\\
206.54244	-31.982349966241\\
356.804217	-55.2498427806149\\
288.18867	-44.6249734449033\\
296.98842	-45.9875829120687\\
331.524852	-51.3354245218089\\
303.77396	-47.0382992442178\\
203.703486	-31.5427482051402\\
146.528644	-22.689430668519\\
122.737356	-19.0054357522024\\
221.721588	-34.3327861454852\\
497.051904	-76.9666899699375\\
716.721808	-110.981780267014\\
784.47369	-121.472914200532\\
503.03268	-77.8927915067534\\
330.47016	-51.1721092794278\\
246.470862	-38.1651217297767\\
185.88876	-28.7842023029775\\
383.565107	-59.3936698284027\\
385.027038	-59.6200445573373\\
551.515272	-85.4001455624839\\
746.721936	-115.627191605856\\
1234.53648	-191.163509782548\\
1424.293888	-220.546758239101\\
1157.062074	-179.166878164762\\
1070.375778	-165.743818691129\\
694.004571	-107.464098264226\\
777.83004	-120.444169021801\\
844.936924	-130.835427347363\\
927.97614	-143.693749671019\\
676.411318	-104.739846655256\\
623.210148	-96.5018378588384\\
611.949753	-94.758206411238\\
553.111977	-85.647389557969\\
649.525316	-100.576646466679\\
489.21875	-75.7537543981129\\
841.813675	-130.351803533538\\
1013.04252	-156.865971009748\\
694.3189	-107.512770973246\\
426.505224	-66.0427916722491\\
454.613094	-70.39519370229\\
787.82949	-121.992547695792\\
1002.680871	-155.261507130338\\
1250.67149	-193.661957736047\\
1350.575605	-209.131748685537\\
1328.285535	-205.680211948042\\
996.989938	-154.380286733985\\
899.224341	-139.241637563807\\
1048.94751	-162.425728877015\\
1204.815092	-186.561260324768\\
673.899642	-104.350922709001\\
447.034347	-69.221652134467\\
625.233378	-96.8151277082378\\
515.77317	-79.8656103130064\\
320.600959	-49.6438991921006\\
449.905446	-69.6662314325799\\
502.762901	-77.8510172041377\\
394.643042	-61.1090480047875\\
303.21078	-46.9510928577047\\
416.168278	-64.4421529630611\\
346.524916	-53.6581301912298\\
485.280136	-75.1438742624374\\
658.926548	-102.032393246556\\
601.188588	-93.0918789238971\\
414.07058	-64.1173320131417\\
431.673697	-66.8431110268431\\
688.467598	-106.606718017146\\
1134.405818	-175.658638849313\\
788.526448	-122.100469121857\\
475.123848	-73.5712097912811\\
364.317915	-56.4133117460084\\
421.336751	-65.2424723176551\\
375.501388	-58.1450321000626\\
282.518808	-43.747015816707\\
313.725951	-48.5793290636722\\
391.091494	-60.5591036395621\\
500.040192	-77.4294155609392\\
541.444442	-83.8407139536072\\
383.528034	-59.3879292084105\\
651.61086	-100.899585413648\\
772.845984	-119.672405972796\\
724.183328	-112.137169657786\\
559.938148	-86.7043974535444\\
625.011912	-96.7808344989699\\
836.158158	-129.476067176719\\
508.537326	-78.745166034779\\
244.087925	-37.7961325521456\\
331.960178	-51.4028331787484\\
368.181984	-57.0116488579923\\
483.688318	-74.8973870012307\\
418.009884	-64.727318992822\\
310.42586	-48.0683219056158\\
470.759818	-72.8954555263311\\
467.605064	-72.4069532771789\\
336.148407	-52.0513652945536\\
413.548641	-64.0365116463491\\
366.135336	-56.6947328159735\\
326.973096	-50.6306015645856\\
381.080964	-59.0090092676376\\
234.08256	-36.2468379617942\\
270.897744	-41.9475360786537\\
201.818331	-31.2508387702046\\
231.62725	-35.8666420868176\\
443.432556	-68.6639278223728\\
505.879374	-78.333591778866\\
706.80665	-109.44645390442\\
898.532856	-139.134563612002\\
581.199006	-89.996564434522\\
333.69283	-51.6711280755924\\
260.741327	-40.3748516323138\\
252.68375	-39.1271649704639\\
358.01028	-55.436597274981\\
238.619125	-36.949308647599\\
270.217332	-41.8421767408574\\
362.120655	-56.0730739804101\\
614.003532	-95.0762266628208\\
578.251088	-89.5400899920405\\
816.269883	-126.396439709963\\
530.299561	-82.1149694312025\\
467.249131	-72.3518383392004\\
259.071395	-40.1162687006712\\
244.79972	-37.9063514340094\\
150.62774	-23.3241608615835\\
191.2308	-29.6113978798945\\
212.028726	-32.8318815146573\\
383.565107	-59.3936698284027\\
408.380168	-63.2361922917551\\
462.031758	-71.5439469964312\\
516.282858	-79.9445336567492\\
771.240088	-119.423738784196\\
674.50383	-104.444479036027\\
507.395496	-78.5683577881932\\
625.180283	-96.806906139482\\
684.007038	-105.916016430753\\
891.482127	-138.042783722139\\
696.497802	-107.850166068928\\
429.582384	-66.5192787710852\\
218.33386	-33.8082087148785\\
156.651059	-24.2568500281117\\
157.705751	-24.4201652704929\\
197.826642	-30.632740162701\\
210.276129	-32.5604982066855\\
205.784796	-31.8650315325195\\
284.078124	-43.9884702607426\\
426.747453	-66.080299956987\\
390.534967	-60.4729274460415\\
409.654586	-63.4335313106961\\
469.659264	-72.7250387190893\\
276.96054	-42.8863381158811\\
168.76454	-26.1325787580106\\
339.024884	-52.4967773565194\\
527.629212	-81.7014755446675\\
519.787374	-80.4871952868447\\
574.021968	-88.8852260527492\\
477.296	-73.9075596738712\\
355.406805	-55.0334585855264\\
513.206694	-79.4682007849116\\
719.802232	-111.458772784448\\
706.777352	-109.441917215063\\
516.660686	-80.0030389563044\\
836.173404	-129.478427964689\\
604.792116	-93.6498722041602\\
621.484612	-96.2346448488625\\
720.940576	-111.635041236539\\
702.875987	-108.837804952343\\
518.97996	-80.3621701485939\\
473.113408	-73.2599004271324\\
821.467166	-127.201220188878\\
1135.636956	-175.849276116757\\
851.10256	-131.790153786716\\
475.123848	-73.5712097912811\\
461.885424	-71.5212877056822\\
480.916106	-74.4681199974875\\
606.511518	-93.9161153864168\\
695.475506	-107.691867230001\\
517.906389	-80.1959315613302\\
557.438772	-86.3173781178126\\
752.4121	-116.508292925339\\
803.804092	-124.466157050534\\
526.291308	-81.4943059481951\\
413.025078	-63.955439809029\\
578.340301	-89.5539043024918\\
992.8125	-153.733425565509\\
873.843565	-135.311516178359\\
863.623912	-133.729039866083\\
505.879374	-78.333591778866\\
459.039976	-71.0806803721641\\
416.713506	-64.5265795472895\\
263.26671	-40.765898057953\\
163.179746	-25.2677936020041\\
140.156245	-21.7026877262814\\
126.123534	-19.5297731709139\\
311.794032	-48.2801783924376\\
412.274223	-63.8391726274081\\
506.955288	-78.5001931712073\\
379.0285	-58.6911926390498\\
426.488166	-66.0401503026317\\
471.153408	-72.9564015145901\\
301.909531	-46.749599155436\\
315.630392	-48.8742248659686\\
296.001321	-45.8347342013179\\
223.490084	-34.6066313561155\\
294.8025	-45.6491011044644\\
499.730657	-77.3814851854832\\
614.478986	-95.1498489954554\\
356.804217	-55.2498427806149\\
270.897744	-41.9475360786537\\
224.980555	-34.837425400848\\
273.117393	-42.2912406999423\\
201.603	-31.2174955435023\\
466.115144	-72.1762445528039\\
794.008585	-122.949358212626\\
647.83158	-100.314377571713\\
879.07372	-136.121386767607\\
1037.44206	-160.644151548795\\
1035.077395	-160.277991724292\\
761.515392	-117.917904773582\\
549.128228	-85.030520430754\\
540.101771	-83.6328062044225\\
545.002876	-84.3917245910325\\
659.189607	-102.073126981471\\
782.807432	-121.214900174504\\
801.164304	-124.057395424339\\
1094.9705	-169.552222457089\\
1213.61994	-187.924659198799\\
1432.400313	-221.802008837114\\
962.888895	-149.099863536507\\
1091.223168	-168.971961647428\\
1198.119	-185.524394691971\\
915.143739	-141.706698778771\\
1052.378628	-162.957025092221\\
866.41016	-134.160480293672\\
477.998078	-74.0162739134222\\
306.59409	-47.4749861769871\\
334.160552	-51.7435531359864\\
483.688318	-74.8973870012307\\
379.0285	-58.6911926390498\\
255.05678	-39.4946200849692\\
356.789136	-55.2475075422986\\
265.885095	-41.1713455069923\\
205.187463	-31.7725366774554\\
260.741327	-40.3748516323138\\
291.21452	-45.0935153757789\\
208.556595	-32.2942345847061\\
417.442696	-64.6394919820021\\
541.887093	-83.9092568603091\\
413.025078	-63.955439809029\\
729.096084	-112.897892159615\\
1056.171155	-163.544284184204\\
981.504402	-151.982407480855\\
1263.117591	-195.589191470176\\
844.936924	-130.835427347363\\
901.997266	-139.671014973022\\
577.458768	-89.4174021051091\\
378.894213	-58.6703987826882\\
454.810256	-70.4257235250424\\
343.9305	-53.2563942551674\\
261.330476	-40.4660792245717\\
367.220172	-56.8627157477525\\
197.826642	-30.632740162701\\
150.62774	-23.3241608615835\\
192.7752	-29.8505426352671\\
414.823059	-64.2338506652901\\
518.97996	-80.3621701485939\\
649.043724	-100.50207368694\\
625.05024	-96.786769451174\\
608.495742	-94.2233652977639\\
693.295332	-107.354275169719\\
559.358294	-86.6146091584254\\
456.271432	-70.6519813449572\\
450.580536	-69.7707666779208\\
372.788928	-57.7250174774532\\
575.779776	-89.1574162652279\\
394.294712	-61.0551103638662\\
331.072192	-51.2653317576501\\
495.711007	-76.7590569182379\\
681.83115	-105.579088042073\\
509.44626	-78.8859111779284\\
382.588072	-59.24237949166\\
326.973096	-50.6306015645856\\
375.40301	-58.1297986225023\\
253.065098	-39.1862153292906\\
257.22037	-39.8296441728199\\
346.083804	-53.5898255931128\\
443.432556	-68.6639278223728\\
490.465467	-75.9468040173721\\
387.466004	-59.9977096230147\\
509.44626	-78.8859111779284\\
451.87547	-69.9712825252749\\
268.50348	-41.5767929560317\\
294.188022	-45.5539514081474\\
585.76268	-90.7032328509494\\
818.306052	-126.711733116724\\
808.682024	-125.221487181889\\
654.951934	-101.416938641058\\
625.05024	-96.786769451174\\
472.188046	-73.1166114675875\\
452.041712	-69.9970244978355\\
362.120655	-56.0730739804101\\
530.299561	-82.1149694312025\\
436.545312	-67.5974629935904\\
292.336209	-45.2672048613456\\
440.374182	-68.1903496677056\\
511.668612	-79.2300343489928\\
615.912122	-95.3717649227642\\
667.857177	-103.415268853606\\
687.962106	-106.52844440883\\
959.82807	-148.625905853352\\
1195.530105	-185.123513662794\\
1321.72771	-204.664756460214\\
791.37982	-122.542303458121\\
426.488166	-66.0401503026317\\
266.742021	-41.3040373994052\\
192.7752	-29.8505426352671\\
308.772772	-47.8123472064644\\
269.230148	-41.689314868164\\
470.586006	-72.868541366644\\
416.713506	-64.5265795472895\\
380.288966	-58.8863712412419\\
510.045081	-78.9786364444118\\
601.073283	-93.0740243615957\\
720.940576	-111.635041236539\\
785.610252	-121.648906716367\\
1129.744487	-174.936848599568\\
945.769654	-146.449010971685\\
718.831192	-111.308410751777\\
486.779772	-75.3760874742789\\
488.310526	-75.6131191958961\\
509.171072	-78.8432993112219\\
641.666442	-99.3597282458592\\
453.195792	-70.1757296126247\\
476.552076	-73.7923657325375\\
428.16225	-66.2993761564605\\
248.787675	-38.5238718451154\\
449.905446	-69.6662314325799\\
641.201918	-99.2877983842012\\
433.153188	-67.0722048582799\\
503.054104	-77.8961089356831\\
916.804935	-141.963928972403\\
675.877176	-104.65713669804\\
668.960099	-103.586052337085\\
935.711175	-144.891491870497\\
866.291151	-134.142052180364\\
521.457303	-80.7457777539482\\
327.983832	-50.7871103793138\\
352.970256	-54.656167783559\\
613.377336	-94.9792624765767\\
970.111103	-150.218196329442\\
1046.37153	-162.026847688887\\
1068.63804	-165.474736245577\\
796.03804	-123.263611980766\\
500.040192	-77.4294155609392\\
423.339438	-65.5525811577891\\
310.771846	-48.1218965866132\\
285.029055	-44.1357184874787\\
246.4378	-38.1600021986305\\
356.505435	-55.2035774683256\\
408.791196	-63.2998384912574\\
239.395072	-37.069461234686\\
391.091494	-60.5591036395621\\
535.200666	-82.8738878178125\\
623.308336	-96.5170419155857\\
814.417596	-126.109619765975\\
1050.426027	-162.654671888074\\
1239.581917	-191.944777457446\\
866.114865	-134.114754930723\\
750.11468	-116.152545745925\\
779.694144	-120.732818785508\\
623.334581	-96.5211058589324\\
443.919762	-68.7393699097116\\
559.706191	-86.6684797508625\\
878.693013	-136.062435665313\\
1013.738376	-156.973721795098\\
557.666402	-86.3526258001912\\
312.83774	-48.4417928021375\\
202.341174	-31.3317991181283\\
121.044267	-18.7432670428466\\
177.418297	-27.4725817370439\\
247.28385	-38.2910099817715\\
297.93324	-46.1338848725524\\
415.721592	-64.3729852512\\
325.079512	-50.3373869294797\\
248.909826	-38.5427864857612\\
248.096838	-38.4168982337664\\
239.807865	-37.1333807380559\\
221.274519	-34.263559127459\\
269.360406	-41.7094848484445\\
418.717114	-64.8368310009431\\
632.39952	-97.9247788838111\\
671.169528	-103.92817442833\\
654.951934	-101.416938641058\\
774.390771	-119.911610653719\\
966.64498	-149.681479716566\\
999.161905	-154.716607969994\\
1190.1825	-184.29545636575\\
1065.627792	-165.008610227984\\
768.363552	-118.978317586294\\
519.444471	-80.434097997249\\
521.373248	-80.7327621411452\\
910.076054	-140.921986081522\\
1002.451047	-155.225919714993\\
711.341578	-110.148671106694\\
439.352552	-68.0321539564706\\
212.424594	-32.8931802429789\\
164.278394	-25.4379152720382\\
151.289708	-23.4266642126742\\
103.70961	-16.0590580893804\\
249.76798	-38.6756685295336\\
320.428205	-49.6171488598877\\
346.516896	-53.6568883232305\\
289.126313	-44.7701640728895\\
179.6159	-27.8128726149519\\
141.606415	-21.9272413068944\\
195.812487	-30.3208555442359\\
222.66896	-34.4794832739408\\
233.34795	-36.1330862596806\\
185.3898	-28.7069401512417\\
151.133108	-23.4024152689476\\
293.237091	-45.4067031814113\\
688.87887	-106.670401999166\\
798.011324	-123.569167872672\\
891.049376	-137.975773794638\\
627.057102	-97.0975247749609\\
864.8705	-133.922069510159\\
1273.245447	-197.157453348943\\
1144.247355	-177.182565266275\\
1187.98107	-183.954572890731\\
868.406682	-134.469634505849\\
889.016855	-137.661045267486\\
912.403184	-141.282333747004\\
587.571765	-90.9833631214578\\
537.565732	-83.2401097357159\\
324.895472	-50.3088890009775\\
331.899144	-51.3933822845504\\
612.055291	-94.7745485889073\\
404.275126	-62.6005413820916\\
473.616274	-73.3377674088439\\
501.246779	-77.6162511948104\\
473.616274	-73.3377674088439\\
507.395496	-78.5683577881932\\
571.389526	-88.4776019280907\\
619.540638	-95.9336275044694\\
528.695563	-81.8665961410392\\
402.028332	-62.2526334557107\\
479.684155	-74.2773568399337\\
206.721396	-32.0100606557272\\
109.05176	-16.8862706993997\\
199.840797	-30.9446247811662\\
271.32084	-42.0130509643175\\
133.7379	-20.7088302120869\\
67.60998	-10.469160921942\\
167.354474	-25.9142351367795\\
272.166462	-42.1439924732061\\
327.24749	-50.6730904832632\\
402.237285	-62.2849890721763\\
357.972	-55.4306697554034\\
533.113119	-82.5506387135365\\
457.292616	-70.8101079947049\\
331.960178	-51.4028331787484\\
485.61201	-75.1952637512645\\
264.64776	-40.9797486565074\\
219.639168	-34.0103309386197\\
142.401472	-22.0503530083787\\
113.468572	-17.5701980661873\\
194.69709	-30.1481403521171\\
148.578192	-23.0067957650512\\
278.572734	-43.1359805270072\\
426.85776	-66.0973805970614\\
406.346616	-62.9213041179805\\
304.762736	-47.1914076323545\\
335.265255	-51.9146124008752\\
250.580968	-38.8015567815285\\
362.087574	-56.0679515071827\\
258.02994	-39.9550031598744\\
250.582574	-38.8018054648211\\
226.1007	-35.0108758035979\\
175.605552	-27.1918847287702\\
226.86975	-35.1299604151747\\
207.866106	-32.1873148598988\\
358.938956	-55.5803993954423\\
346.393989	-53.6378566187183\\
272.166462	-42.1439924732061\\
248.031289	-38.4067482081448\\
195.812487	-30.3208555442359\\
249.43182	-38.6236153691049\\
561.050186	-86.8765925166596\\
710.3563	-109.996104371209\\
474.691748	-73.5043006687763\\
455.842428	-70.585551626857\\
318.825328	-49.3689491525208\\
550.710846	-85.2755831052284\\
464.223012	-71.8832547546475\\
314.815292	-48.7480095784109\\
400.592058	-62.0302316204491\\
299.87537	-46.4346166802136\\
422.69514	-65.4528139423028\\
233.919375	-36.2215693546293\\
278.572734	-43.1359805270072\\
325.649148	-50.4255930042475\\
418.556523	-64.8119640652982\\
328.302182	-50.8364057256443\\
341.979105	-52.954227795759\\
353.14851	-54.6837697708837\\
554.436945	-85.8525560616187\\
475.123848	-73.5712097912811\\
785.070608	-121.565344794346\\
980.414446	-151.81363173559\\
755.146392	-116.931688154203\\
476.245701	-73.7449246716548\\
368.181984	-57.0116488579923\\
541.705769	-83.8811794945858\\
674.50383	-104.444479036027\\
531.037422	-82.2292245389103\\
633.360112	-98.0735230814626\\
360.103502	-55.7607251324859\\
185.87084	-28.7814274557771\\
216.872502	-33.5819227129219\\
488.865116	-75.698995451971\\
407.10575	-63.0388532728141\\
411.49612	-63.7186861915174\\
626.62562	-97.0307113475213\\
577.537584	-89.4296064777065\\
508.537326	-78.745166034779\\
361.333824	-55.9512360452802\\
434.544786	-67.2876887764921\\
591.979572	-91.665896096558\\
482.216152	-74.6694274195563\\
514.953369	-79.7386670925977\\
449.21875	-69.5598990401068\\
337.970192	-52.3334621141382\\
379.872462	-58.8218771029411\\
276.268824	-42.7792283945607\\
330.061456	-51.1088229429218\\
570.14742	-88.2852663055079\\
663.792246	-102.785828987249\\
698.585208	-108.173393339864\\
604.569944	-93.6154696733452\\
419.749902	-64.9967544881302\\
292.90551	-45.3553590625748\\
359.092608	-55.604191852029\\
416.168278	-64.4421529630611\\
549.867318	-85.1449658446677\\
681.389256	-105.510662354084\\
803.804092	-124.466157050534\\
629.385498	-97.4580684759649\\
372.788928	-57.7250174774532\\
685.984572	-106.222230419792\\
931.209784	-144.194467751401\\
656.992952	-101.732982897935\\
670.435953	-103.814582992223\\
759.149664	-117.551580347372\\
562.836072	-87.1531305059068\\
476.552076	-73.7923657325375\\
531.903559	-82.3633427213657\\
467.8125	-72.43907399168\\
559.938148	-86.7043974535444\\
691.229904	-107.03445111261\\
523.66014	-81.0868790978298\\
619.540638	-95.9336275044694\\
560.534466	-86.7967351395288\\
575.626484	-89.1336795533394\\
463.269696	-71.7356372192857\\
593.881434	-91.9603925500302\\
410.24368	-63.5247503864028\\
456.271432	-70.6519813449572\\
499.730657	-77.3814851854832\\
458.82337	-71.0471397163221\\
224.95168	-34.8329542115115\\
145.9952	-22.6068287940791\\
115.072601	-17.8185761565885\\
215.51019	-33.37097362591\\
502.762901	-77.8510172041377\\
652.509858	-101.038792003126\\
652.951752	-101.107217691116\\
437.693514	-67.7752579213347\\
507.79953	-78.6309209920862\\
416.713506	-64.5265795472895\\
369.338554	-57.1907395402775\\
240.157225	-37.1874778707528\\
345.33603	-53.4740354932008\\
331.524852	-51.3354245218089\\
326.739235	-50.5943890343886\\
379.872462	-58.8218771029411\\
322.493633	-49.936972916997\\
295.034523	-45.6850290270076\\
208.547262	-32.2927894034047\\
290.484396	-44.9804583145437\\
529.725658	-82.0261026269143\\
412.274223	-63.8391726274081\\
503.03268	-77.8927915067534\\
447.595176	-69.3084944771314\\
601.119694	-93.0812109371212\\
562.297153	-87.0696808475147\\
779.47548	-120.698959455808\\
549.867318	-85.1449658446677\\
641.666442	-99.3597282458592\\
602.22304	-93.2520600754671\\
324.867848	-50.3046115244673\\
563.282676	-87.2222855203644\\
394.643042	-61.1090480047875\\
499.730657	-77.3814851854832\\
544.159968	-84.2612033352298\\
692.207023	-107.185754456449\\
514.744776	-79.7063672208304\\
685.499364	-106.147097715528\\
838.634346	-129.8594959345\\
673.280124	-104.254992586909\\
568.338335	-88.0051360349995\\
451.757796	-69.9530611318899\\
546.479857	-84.6204297580456\\
473.566948	-73.3301294561091\\
403.84232	-62.5335229380397\\
326.973096	-50.6306015645856\\
409.654586	-63.4335313106961\\
354.024948	-54.8194830259402\\
749.116798	-115.998027326616\\
945.135548	-146.350821950544\\
810.727305	-125.538191548901\\
787.997271	-122.018527976937\\
745.577856	-115.450034954906\\
406.41678	-62.932168759664\\
363.786687	-56.3310530111554\\
282.27636	-43.709473620611\\
240.969	-37.3131782940839\\
284.325444	-44.0267668684211\\
479.448266	-74.2408302812701\\
604.792116	-93.6498722041602\\
691.229904	-107.03445111261\\
566.850808	-87.7747978580213\\
406.929479	-63.0115583458688\\
727.385916	-112.633078828868\\
866.394859	-134.158110989151\\
952.759472	-147.531359013455\\
758.876872	-117.509339492601\\
1204.815092	-186.561260324768\\
813.776612	-126.010365711405\\
459.21875	-71.1083628796084\\
330.927464	-51.2429211501937\\
288.768792	-44.7148032388533\\
534.289389	-82.7327798695855\\
693.608496	-107.402767482703\\
916.804935	-141.963928972403\\
667.857177	-103.415268853606\\
511.789293	-79.2487213654543\\
271.978791	-42.1149322974838\\
307.817741	-47.6644641095541\\
322.874	-49.995871371521\\
358.01028	-55.436597274981\\
227.19612	-35.1804976295046\\
294.10147	-45.5405491439238\\
217.608766	-33.6959305309545\\
149.985175	-23.2246619948806\\
332.579544	-51.4987397641901\\
366.135336	-56.6947328159735\\
466.154876	-72.182396909331\\
418.009884	-64.727318992822\\
447.811992	-69.3420676507139\\
453.425984	-70.211374011439\\
577.540557	-89.430066836006\\
397.060046	-61.4833123341806\\
614.088398	-95.0893678560411\\
701.721744	-108.659074597593\\
533.596668	-82.6255145276492\\
559.358294	-86.6146091584254\\
480.898454	-74.465386649118\\
576.559196	-89.2781066338063\\
811.248529	-125.618901200529\\
597.484962	-92.5183858302933\\
475.123848	-73.5712097912811\\
539.191708	-83.4918862397056\\
476.245701	-73.7449246716548\\
331.072192	-51.2653317576501\\
522.606084	-80.9236623377488\\
756.266994	-117.10520932175\\
723.167148	-111.979817859344\\
841.813675	-130.351803533538\\
1225.502334	-189.76460494237\\
780.418524	-120.844986409114\\
656.992952	-101.732982897935\\
510.665216	-79.0746621067227\\
670.871717	-103.882059471879\\
947.113716	-146.65713411219\\
617.735846	-95.6541619894873\\
560.690724	-86.8209311257921\\
759.149664	-117.551580347372\\
462.031758	-71.5439469964312\\
277.215561	-42.9258271955625\\
359.092608	-55.604191852029\\
259.906361	-40.2455601664925\\
218.166855	-33.7823485945269\\
236.26925	-36.5854390011142\\
207.18495	-32.0818403163928\\
209.26626	-32.4041236437721\\
270.217332	-41.8421767408574\\
380.215024	-58.8749215899198\\
464.921875	-71.991471163074\\
612.055291	-94.7745485889073\\
580.930464	-89.9549816768833\\
693.034155	-107.313832855698\\
803.54348	-124.42580222472\\
722.798418	-111.92272135219\\
522.10008	-80.8453094480845\\
555.524724	-86.0209947063056\\
498.546048	-77.1980527654383\\
563.234056	-87.2147568891768\\
774.032441	-119.856124548959\\
571.389526	-88.4776019280907\\
643.684261	-99.6721802214751\\
553.163226	-85.6553252803001\\
545.539437	-84.4748091216511\\
868.818009	-134.53332700442\\
702.606225	-108.796033282116\\
715.632357	-110.813082719173\\
619.34344	-95.9030921072472\\
370.778507	-57.4137110553857\\
237.80735	-36.8236082242679\\
194.00568	-30.0410780137901\\
347.594325	-53.8237243078435\\
317.836552	-49.2158407643841\\
428.16225	-66.2993761564605\\
326.973096	-50.6306015645856\\
482.344334	-74.6892759387438\\
784.301088	-121.446187404969\\
968.001076	-149.891466278455\\
762.20508	-118.024700466435\\
795.576018	-123.19206954476\\
688.075025	-106.54592950766\\
501.534336	-77.66077835644\\
534.289389	-82.7327798695855\\
599.293497	-92.7984309352906\\
512.028091	-79.2856983722488\\
627.057102	-97.0975247749609\\
811.22967	-125.615980952574\\
1148.647235	-177.863870774089\\
998.4375	-154.604436475229\\
949.300035	-146.995677703502\\
1038.043705	-160.737314101467\\
690.049503	-106.851670286149\\
747.94065	-115.815905061825\\
578.251088	-89.5400899920405\\
594.52246	-92.0596531081482\\
804.247255	-124.534779238585\\
865.122258	-133.96105332609\\
331.072192	-51.2653317576501\\
437.693514	-67.7752579213347\\
340.394009	-52.7087814119451\\
497.051904	-76.9666899699375\\
491.981589	-76.1815700266993\\
301.36506	-46.6652897899203\\
418.083644	-64.7387404621022\\
775.435178	-120.073333301042\\
1299.08313	-201.158325131144\\
1248.26865	-193.289886650837\\
1128.362304	-174.722822560061\\
896.6035	-138.835809812049\\
1065.08913	-164.925200365112\\
1281.198627	-198.388974512848\\
933.686726	-144.578013263355\\
968.019	-149.894241745041\\
977.302386	-151.331740497955\\
653.60548	-101.208445108003\\
576.692347	-89.2987245846757\\
730.695768	-113.145597442478\\
475.123848	-73.5712097912811\\
360.319536	-55.7941772161962\\
479.448266	-74.2408302812701\\
355.953114	-55.1180525586958\\
435.863142	-67.4918314358511\\
403.236834	-62.4397656204073\\
516.660686	-80.0030389563044\\
677.279148	-104.874226992639\\
510.045081	-78.9786364444118\\
386.176393	-59.7980180229624\\
464.653968	-71.9499867328891\\
545.002876	-84.3917245910325\\
654.951934	-101.416938641058\\
535.200666	-82.8738878178125\\
458.096376	-70.9345673242687\\
717.719763	-111.136309990109\\
658.485223	-101.964055666159\\
469.770588	-72.7422768379361\\
792.694532	-122.745881857257\\
694.3189	-107.512770973246\\
493.980608	-76.491110890297\\
330.927464	-51.2429211501937\\
239.388175	-37.0683932591759\\
183.636068	-28.435381092624\\
406.41678	-62.932168759664\\
363.786687	-56.3310530111554\\
264.123636	-40.8985899503659\\
189.63064	-29.3636188901528\\
245.66875	-38.0409175870536\\
414.823059	-64.2338506652901\\
491.212722	-76.0625137520107\\
643.94732	-99.7129139563907\\
753.29397	-116.644847305953\\
725.276532	-112.306448344106\\
730.95516	-113.185763355704\\
403.533663	-62.4857285177088\\
190.8953	-29.5594469180792\\
279.523665	-43.2832287537434\\
234.08256	-36.2468379617942\\
313.725951	-48.5793290636722\\
515.77317	-79.8656103130064\\
589.285635	-91.2487496935185\\
866.037832	-134.102826649228\\
558.864552	-86.5381549951211\\
542.375361	-83.9848633945077\\
781.476878	-121.008868698953\\
557.666402	-86.3526258001912\\
380.215024	-58.8749215899198\\
290.484396	-44.9804583145437\\
391.091494	-60.5591036395621\\
540.705432	-83.7262809274042\\
786.592205	-121.800958587626\\
514.21311	-79.6240406632612\\
440.374182	-68.1903496677056\\
431.261748	-66.7793222138212\\
528.42699	-81.8250085831625\\
727.581725	-112.663399144463\\
1153.280161	-178.581262612298\\
972.835709	-150.640091716231\\
942.410942	-145.928926563755\\
586.521936	-90.8208008970418\\
387.405636	-59.9883618565084\\
435.122832	-67.377197109349\\
207.201618	-32.0844212959205\\
311.794032	-48.2801783924376\\
290.08566	-44.9187154867929\\
324.867848	-50.3046115244673\\
578.251088	-89.5400899920405\\
807.368026	-125.018019343071\\
946.872576	-146.619794455164\\
1229.8585	-190.439141495357\\
1003.214058	-155.344069209257\\
528.42699	-81.8250085831625\\
491.212722	-76.0625137520107\\
414.323899	-64.1565575442775\\
566.621517	-87.7392929757989\\
770.0562	-119.240418008394\\
1012.076373	-156.716366640434\\
694.3189	-107.512770973246\\
494.191844	-76.5238200210571\\
579.274612	-89.6985789823267\\
765.997824	-118.611993160084\\
524.621379	-81.2357234810916\\
369.626571	-57.2353379312437\\
289.1628	-44.7758139529007\\
209.26626	-32.4041236437721\\
183.99094	-28.4903317385892\\
160.928421	-24.9191840666575\\
270.217332	-41.8421767408574\\
373.172296	-57.7843806259754\\
418.431546	-64.7926118287712\\
316.69728	-49.0394286148484\\
281.60885	-43.6061121108605\\
135.214925	-20.9375421923409\\
79.329131	-12.2838290772578\\
121.044267	-18.7432670428466\\
227.2344	-35.1864251490822\\
351.12384	-54.3702569426913\\
443.582272	-68.6871108035923\\
511.00632	-79.1274808276737\\
617.735846	-95.6541619894873\\
842.43204	-130.447555117749\\
1182.2525	-183.067524541026\\
1176.248865	-182.137883370719\\
1268.779404	-196.465902739828\\
1059.469432	-164.05501045092\\
574.867304	-89.0161232755721\\
528.42699	-81.8250085831625\\
530.381943	-82.127725986005\\
702.606225	-108.796033282116\\
579.194616	-89.6861918909962\\
416.168278	-64.4421529630611\\
323.884736	-50.1523801862493\\
273.497752	-42.3501379156951\\
456.271432	-70.6519813449572\\
437.96828	-67.8178044428672\\
235.457475	-36.4597385777831\\
183.449794	-28.4065372373001\\
285.929937	-44.2752168075445\\
379.872462	-58.8218771029411\\
295.72536	-45.7920026383566\\
418.431546	-64.7926118287712\\
318.825328	-49.3689491525208\\
498.655339	-77.2149760815867\\
739.86627	-114.565616516186\\
593.274272	-91.8663757098583\\
819.03186	-126.824121860966\\
1157.062074	-179.166878164762\\
1223.046976	-189.384401634767\\
932.574612	-144.405806431915\\
575.28903	-89.0814260216899\\
423.778212	-65.6205237250604\\
262.143464	-40.5919674765666\\
393.06425	-60.8645777725781\\
281.547798	-43.5966584294275\\
508.506999	-78.740470008493\\
464.888214	-71.9862588789439\\
369.364848	-57.1948110710971\\
471.747416	-73.0483815054276\\
276.311028	-42.7857635313489\\
400.085	-61.9517155226961\\
304.762736	-47.1914076323545\\
189.0154	-29.2683512008913\\
212.424594	-32.8931802429789\\
178.63778	-27.6614142698826\\
204.422484	-31.6540824455076\\
173.199545	-26.8193232450614\\
128.29557	-19.8661050913235\\
170.27437	-26.3663704738901\\
244.79972	-37.9063514340094\\
237.39089	-36.759120899208\\
245.432825	-38.0043854539202\\
396.755636	-61.4361755464423\\
522.660952	-80.9321584491434\\
414.12075	-64.1251006562244\\
397.060046	-61.4833123341806\\
640.06306	-99.1114503410685\\
759.672744	-117.632577393889\\
606.693978	-93.9443686576324\\
605.52464	-93.7633008967169\\
280.70325	-43.4658832255552\\
185.262539	-28.6872342455738\\
364.757904	-56.4814424516363\\
544.831903	-84.3652500402293\\
591.538424	-91.5975859239712\\
836.158158	-129.476067176719\\
732.47185	-113.420617317777\\
507.79953	-78.6309209920862\\
354.24714	-54.8538886536829\\
377.895177	-58.5157016706522\\
422.459856	-65.4163810657015\\
313.60892	-48.5612072365122\\
196.3992	-30.4117059307025\\
248.096838	-38.4168982337664\\
204.422484	-31.6540824455076\\
166.78319	-25.825773875171\\
137.531775	-21.2962980369958\\
250.580968	-38.8015567815285\\
351.12384	-54.3702569426913\\
379.255732	-58.7263786925675\\
286.830819	-44.4147151276102\\
491.212722	-76.0625137520107\\
681.654026	-105.551661031162\\
441.788979	-68.4094258671791\\
606.693978	-93.9443686576324\\
668.327055	-103.488027762804\\
475.718848	-73.6633433897315\\
278.07546	-43.0589794462748\\
360.213216	-55.7777139486546\\
418.556523	-64.8119640652982\\
252.25211	-39.0603270772957\\
269.091108	-41.6677850269395\\
220.38624	-34.1260123363701\\
127.856118	-19.7980575382038\\
130.541658	-20.2139036961573\\
221.274519	-34.263559127459\\
300.7527	-46.5704680582445\\
270.897744	-41.9475360786537\\
337.35096	-52.237576278112\\
386.02984	-59.7753248208553\\
268.512885	-41.5782492862727\\
159.273442	-24.6629165529941\\
127.416666	-19.7300099850841\\
167.957496	-26.0076109129219\\
195.812487	-30.3208555442359\\
250.266775	-38.752905131616\\
506.23947	-78.3893513423409\\
515.596246	-79.8382142713724\\
443.582272	-68.6871108035923\\
437.693514	-67.7752579213347\\
615.912122	-95.3717649227642\\
793.268886	-122.834818497264\\
858.808214	-132.983346444587\\
861.097054	-133.337765042028\\
632.62472	-97.9596502894767\\
666.416162	-103.192132891638\\
681.389256	-105.510662354084\\
695.055645	-107.62685327239\\
806.659362	-124.908285285236\\
1078.134012	-166.945153171869\\
907.326857	-140.496282867306\\
820.810936	-127.099605346339\\
647.83158	-100.314377571713\\
623.308336	-96.5170419155857\\
598.578252	-92.6876778334022\\
693.295332	-107.354275169719\\
431.261748	-66.7793222138212\\
504.605632	-78.1363574360805\\
632.768169	-97.981862848408\\
729.378013	-112.941547845796\\
559.938148	-86.7043974535444\\
645.595879	-99.9681873562691\\
513.206694	-79.4682007849116\\
440.707932	-68.242029648349\\
295.088904	-45.6934497282132\\
456.756042	-70.7270214510832\\
744.26079	-115.246092047383\\
566.850808	-87.7747978580213\\
328.994568	-50.943619194042\\
393.68364	-60.960488074333\\
228.728544	-35.417787944583\\
197.709	-30.6145237244004\\
274.018275	-42.430739020008\\
171.448676	-26.548207511641\\
192.5052	-29.8087341116006\\
226.86975	-35.1299604151747\\
152.17192	-23.5632715507516\\
236.555935	-36.6298311366969\\
210.962071	-32.666713844985\\
130.077792	-20.1420757234199\\
210.276129	-32.5604982066855\\
253.968825	-39.3261541873187\\
295.034523	-45.6850290270076\\
302.93004	-46.9076212838746\\
419.97132	-65.0310402647717\\
367.220172	-56.8627157477525\\
432.558126	-66.9800616593537\\
745.577856	-115.450034954906\\
523.794458	-81.1076777544292\\
454.810256	-70.4257235250424\\
494.407914	-76.5572776792372\\
338.62218	-52.4344200983171\\
214.791192	-33.2596393855426\\
372.964044	-57.752133556825\\
314.117175	-48.6399086853867\\
182.856532	-28.3146727618651\\
328.302182	-50.8364057256443\\
310.805256	-48.1270700043009\\
371.27867	-57.4911594873214\\
240.644448	-37.2629225904801\\
145.9952	-22.6068287940791\\
119.768652	-18.545742672784\\
144.077075	-22.3098140738647\\
265.526664	-41.1158437627467\\
268.212204	-41.5316899207002\\
337.35096	-52.237576278112\\
450.580536	-69.7707666779208\\
597.084597	-92.4563907577831\\
454.810256	-70.4257235250424\\
621.488879	-96.2353055783829\\
737.54694	-114.206476652499\\
630.50988	-97.6321749628436\\
619.167774	-95.8758908623642\\
1010.816004	-156.521203058341\\
743.720648	-115.162453011863\\
897.064982	-138.907268631007\\
776.610009	-120.255251633144\\
928.23696	-143.734136704881\\
1004.808906	-155.591025655007\\
570.322504	-88.3123774297955\\
335.802214	-51.9977585603547\\
286.083252	-44.2989570808998\\
444.469647	-68.82451761355\\
562.836072	-87.1531305059068\\
561.38764	-86.9288460483091\\
369.338554	-57.1907395402775\\
203.832486	-31.5627233886698\\
273.497752	-42.3501379156951\\
569.069582	-88.1183669887239\\
775.892172	-120.144097169429\\
743.212128	-115.083710528697\\
432.183252	-66.9220137760168\\
336.148407	-52.0513652945536\\
467.499416	-72.3905940664073\\
757.68849	-117.32532283715\\
853.673436	-132.188244638901\\
853.60744	-132.178025396945\\
609.764064	-94.4197603731484\\
873.843565	-135.311516178359\\
966.64498	-149.681479716566\\
928.343	-143.750556615435\\
1289.158326	-199.62150512033\\
1093.57471	-169.336089422835\\
917.864164	-142.127946752828\\
1062.103775	-164.462928938555\\
866.291151	-134.142052180364\\
469.21875	-72.6568267191099\\
437.815659	-67.7941716329023\\
549.903981	-85.1506429776425\\
672.391527	-104.11739655467\\
677.279148	-104.874226992639\\
501.681458	-77.6835596661395\\
628.974226	-97.394384493945\\
549.867318	-85.1449658446677\\
398.276556	-61.6716845087198\\
540.101771	-83.6328062044225\\
521.293281	-80.7203795403599\\
782.059872	-121.099143211718\\
790.961754	-122.477567449769\\
551.515272	-85.4001455624839\\
409.654586	-63.4335313106961\\
249.556725	-38.6429564566922\\
373.172296	-57.7843806259754\\
311.794032	-48.2801783924376\\
261.62268	-40.5113259573475\\
224.61225	-34.7803947034073\\
388.652528	-60.1814385738848\\
517.33323	-80.1071799627517\\
570.14742	-88.2852663055079\\
554.436945	-85.8525560616187\\
325.649148	-50.4255930042475\\
235.457475	-36.4597385777831\\
160.401075	-24.8375264454669\\
214.110036	-33.1541648420366\\
362.644101	-56.1541277007033\\
377.776058	-58.4972565242425\\
383.847301	-59.4373665488751\\
625.180283	-96.806906139482\\
755.49123	-116.985085071552\\
838.425588	-129.827170513079\\
858.37983	-132.917012731245\\
971.753541	-150.472521914605\\
586.521936	-90.8208008970418\\
477.998078	-74.0162739134222\\
358.76373	-55.5532662829683\\
402.028332	-62.2526334557107\\
539.191708	-83.4918862397056\\
437.907624	-67.8084120806023\\
291.98265	-45.2124575286825\\
272.662786	-42.2208464498738\\
500.040192	-77.4294155609392\\
615.849234	-95.3620269433703\\
564.033008	-87.3384717173265\\
941.17626	-145.737740520727\\
1009.014846	-156.242300255118\\
680.680605	-105.400930309251\\
466.395788	-72.2197012613812\\
316.847776	-49.0627323762474\\
297.841599	-46.1196945950809\\
537.565732	-83.2401097357159\\
460.501152	-71.3069381920788\\
522.10008	-80.8453094480845\\
601.073283	-93.0740243615957\\
654.951934	-101.416938641058\\
636.46384	-98.5541241390274\\
705.940512	-109.312335567118\\
491.212722	-76.0625137520107\\
528.695563	-81.8665961410392\\
383.840514	-59.4363156064673\\
349.747586	-54.1571489873944\\
555.722464	-86.0516140302679\\
787.333784	-121.915789414189\\
818.678916	-126.769469758829\\
507.485234	-78.5822533929962\\
993.580861	-153.852403487927\\
901.653648	-139.617806968262\\
570.322504	-88.3123774297955\\
335.171288	-51.9000619507146\\
497.051904	-76.9666899699375\\
449.033172	-69.5311629578661\\
423.956388	-65.6481136343671\\
502.269222	-77.7745727961556\\
347.638202	-53.8305185026321\\
496.864652	-76.9376946748501\\
925.622208	-143.329251812754\\
939.534969	-145.483592544367\\
611.949753	-94.758206411238\\
678.366892	-105.042660217703\\
762.20508	-118.024700466435\\
760.828761	-117.811582446124\\
550.754204	-85.2822969347437\\
523.039341	-80.9907506175199\\
366.57621	-56.7630005606511\\
541.705769	-83.8811794945858\\
598.045415	-92.6051699507173\\
941.452668	-145.780541300022\\
1106.49664	-171.337003556992\\
1508.87131	-233.643266199627\\
1057.342336	-163.725637327005\\
832.808012	-128.957309182914\\
655.45192	-101.494359665184\\
684.007038	-105.916016430753\\
511.668612	-79.2300343489928\\
353.739996	-54.7753592391408\\
339.303922	-52.539985381804\\
346.083804	-53.5898255931128\\
259.071395	-40.1162687006712\\
185.88876	-28.7842023029775\\
296.64822	-45.9349041722488\\
383.565107	-59.3936698284027\\
264.123636	-40.8985899503659\\
187.75074	-29.072523172965\\
189.51425	-29.3455963195249\\
463.269696	-71.7356372192857\\
635.667318	-98.4307855875906\\
301.8175	-46.7353484878747\\
360.213216	-55.7777139486546\\
395.525174	-61.2456429551542\\
356.505435	-55.2035774683256\\
447.811992	-69.3420676507139\\
382.356605	-59.2065376637062\\
264.123636	-40.8985899503659\\
185.87084	-28.7814274557771\\
159.873729	-24.7558688242764\\
223.870053	-34.6654681817786\\
429.124878	-66.4484356213497\\
508.537326	-78.745166034779\\
351.12384	-54.3702569426913\\
238.1819	-36.8816059373764\\
158.78974	-24.5880170473846\\
234.08256	-36.2468379617942\\
375.501388	-58.1450321000626\\
495.098062	-76.6641446014276\\
513.992248	-79.5898409812092\\
362.644101	-56.1541277007033\\
343.9305	-53.2563942551674\\
394.643042	-61.1090480047875\\
556.550687	-86.1798613669222\\
792.694532	-122.745881857257\\
906.589476	-140.382102085862\\
663.060396	-102.672504661155\\
411.49612	-63.7186861915174\\
446.232864	-69.0975453901195\\
410.24368	-63.5247503864028\\
415.417128	-64.325840101757\\
287.78175	-44.5619633543463\\
347.160456	-53.7565412620855\\
389.674756	-60.3397268832574\\
359.152866	-55.6135225854331\\
225.698752	-34.9486356092619\\
160.401075	-24.8375264454669\\
234.829632	-36.3625193595446\\
373.117806	-57.7759430465139\\
562.836072	-87.1531305059068\\
610.297506	-94.5023619378955\\
746.58012	-115.60523191107\\
673.899642	-104.350922709001\\
781.048485	-120.942533591994\\
1078.279808	-166.967729155263\\
1440.16498	-223.004339444641\\
1153.280161	-178.581262612298\\
762.275988	-118.035680313829\\
844.29536	-130.736083481891\\
1013.908928	-157.000131155574\\
1278.281856	-197.937323070687\\
779.169746	-120.651617651458\\
360.213216	-55.7777139486546\\
249.43182	-38.6236153691049\\
251.393956	-38.9274450335233\\
175.01229	-27.1000202533352\\
115.863629	-17.9410639819918\\
151.133108	-23.4024152689476\\
237.80735	-36.8236082242679\\
343.077735	-53.1243466785581\\
331.848072	-51.3854739700293\\
261.505251	-40.4931425013266\\
251.393956	-38.9274450335233\\
318.718752	-49.3524462443049\\
398.76298	-61.7470055061863\\
418.431546	-64.7926118287712\\
389.85674	-60.3679064475941\\
266.454396	-41.2594997082216\\
190.5582	-29.5072482020496\\
187.75074	-29.072523172965\\
280.424547	-43.4227270738091\\
349.821747	-54.1686325500745\\
248.866255	-38.5360396739662\\
181.043787	-28.0339757535913\\
285.44062	-44.1994478394891\\
219.058954	-33.9204868988024\\
243.80686	-37.7526106532407\\
320.796376	-49.674158807913\\
462.031758	-71.5439469964312\\
493.58565	-76.4299530721848\\
348.296922	-53.9325189126678\\
530.971496	-82.219016136202\\
511.789293	-79.2487213654543\\
400.085	-61.9517155226961\\
358.76373	-55.5532662829683\\
593.650824	-91.9246834254274\\
754.047756	-116.761568342326\\
738.230559	-114.312332582649\\
813.776612	-126.010365711405\\
621.488879	-96.2353055783829\\
557.194638	-86.2795748507133\\
537.541356	-83.2363352002607\\
715.421136	-110.780375911109\\
531.63378	-82.32156841875\\
745.76535	-115.479067722819\\
717.719763	-111.136309990109\\
769.652	-119.177829100002\\
1332.613209	-206.350336617857\\
1003.214058	-155.344069209257\\
508.506999	-78.740470008493\\
345.470224	-53.4948149488486\\
371.606064	-57.5418552643483\\
480.916106	-74.4681199974875\\
625.233378	-96.8151277082378\\
716.474024	-110.943411810614\\
802.625588	-124.283669967664\\
806.308976	-124.854029280149\\
522.10008	-80.8453094480845\\
467.499416	-72.3905940664073\\
541.705769	-83.8811794945858\\
525.398456	-81.3560510445924\\
319.78564	-49.5176499931847\\
233.802532	-36.2034766385894\\
414.823059	-64.2338506652901\\
423.956388	-65.6481136343671\\
578.251088	-89.5400899920405\\
709.254172	-109.825443835758\\
645.176532	-99.9032529896988\\
437.693514	-67.7752579213347\\
365.47758	-56.592881677852\\
554.096488	-85.7998375262782\\
747.983336	-115.822514834571\\
991.299005	-153.499066337633\\
1109.185086	-171.753299698537\\
1261.598	-195.353888298742\\
1014.900595	-157.153687204607\\
881.614052	-136.51474799184\\
502.762901	-77.8510172041377\\
381.878304	-59.1324744834164\\
661.26902	-102.39511656526\\
856.11232	-132.565897007174\\
510.678028	-79.0766459985939\\
812.731648	-125.848556814647\\
1137.70659	-176.169751457757\\
1333.516736	-206.490244506608\\
798.496545	-123.644302589939\\
433.598049	-67.1410899754903\\
247.9759	-38.3981714217842\\
377.895177	-58.5157016706522\\
349.373574	-54.0992345816405\\
353.365056	-54.7173011359426\\
214.791192	-33.2596393855426\\
294.10147	-45.5405491439238\\
234.08256	-36.2468379617942\\
295.04629	-45.6868511044076\\
248.64081	-38.5011303309365\\
321.846588	-49.836780338494\\
505.345386	-78.2509056679932\\
479.408532	-74.2346776150502\\
697.077135	-107.939873689081\\
830.807908	-128.64760031099\\
801.535368	-124.114853342953\\
441.788979	-68.4094258671791\\
463.269696	-71.7356372192857\\
539.796488	-83.585534235791\\
361.333824	-55.9512360452802\\
318.718752	-49.3524462443049\\
246.4378	-38.1600021986305\\
367.277922	-56.8716581264256\\
334.06119	-51.7281672895843\\
182.11104	-28.1992360214013\\
126.9528	-19.6581820123467\\
130.541658	-20.2139036961573\\
230.264192	-35.6555774844033\\
343.077735	-53.1243466785581\\
363.274098	-56.2516804580527\\
228.728544	-35.417787944583\\
291.00852	-45.0616170206852\\
429.124878	-66.4484356213497\\
545.541176	-84.4750783995128\\
377.776058	-58.4972565242425\\
360.213216	-55.7777139486546\\
436.278717	-67.5561817218612\\
514.953369	-79.7386670925977\\
476.467324	-73.7792421918049\\
548.659522	-84.9579430015182\\
501.47262	-77.6512218570081\\
353.365056	-54.7173011359426\\
278.954364	-43.1950745525141\\
353.365056	-54.7173011359426\\
307.817741	-47.6644641095541\\
380.288966	-58.8863712412419\\
539.123394	-83.4813080638325\\
848.45852	-131.380733753697\\
595.635048	-92.2319333367745\\
592.80859	-91.7942665360876\\
866.291151	-134.142052180364\\
646.534493	-100.113528340094\\
387.257752	-59.9654625538643\\
275.38992	-42.6431332883213\\
225.705649	-34.949703584772\\
235.457475	-36.4597385777831\\
193.1778	-29.9128837894455\\
207.860796	-32.1864926256\\
367.586025	-56.9193667618597\\
187.974	-29.1070941766457\\
186.52026	-28.881987794442\\
266.742021	-41.3040373994052\\
368.15203	-57.0070105894075\\
273.117393	-42.2912406999423\\
199.540866	-30.8981815503816\\
123.572304	-19.1347244307888\\
220.31532	-34.1150306308204\\
421.971039	-65.3406895208381\\
561.546828	-86.9534957344774\\
454.731926	-70.4135944077876\\
364.99275	-56.5178075055214\\
414.12075	-64.1251006562244\\
418.009884	-64.727318992822\\
475.017136	-73.554685823957\\
595.635048	-92.2319333367745\\
712.385852	-110.310373159447\\
691.229904	-107.03445111261\\
559.67377	-86.6634594762484\\
370.778507	-57.4137110553857\\
299.87537	-46.4346166802136\\
326.973096	-50.6306015645856\\
437.96828	-67.8178044428672\\
265.885095	-41.1713455069923\\
253.47476	-39.2496500086323\\
477.998078	-74.0162739134222\\
572.154788	-88.5960999815652\\
548.219364	-84.8897861268515\\
645.83696	-100.005517877358\\
823.357891	-127.493992118172\\
458.096376	-70.9345673242687\\
871.41992	-134.93622351413\\
805.61327	-124.746301721757\\
752.4121	-116.508292925339\\
800.318456	-123.926418920168\\
821.910084	-127.269804439565\\
1415.65933	-219.209728155793\\
1293.3375	-200.268635102128\\
875.351763	-135.545055184939\\
1059.896736	-164.121176930168\\
1290.129524	-199.771891618729\\
1276.26186	-197.624533994494\\
1427.34144	-221.018660646201\\
1325.64836	-205.271854935448\\
1724.783458	-267.076481568337\\
1291.01893	-199.909612921693\\
858.37983	-132.917012731245\\
507.651909	-78.6080624140411\\
516.660686	-80.0030389563044\\
408.034871	-63.1827242999163\\
348.770467	-54.0058456435554\\
535.081638	-82.8554567624236\\
741.2939	-114.786679859305\\
461.885424	-71.5212877056822\\
403.236834	-62.4397656204073\\
382.588072	-59.24237949166\\
269.230148	-41.689314868164\\
311.531296	-48.2394946729041\\
159.873729	-24.7558688242764\\
211.72146	-32.7843024856465\\
160.401075	-24.8375264454669\\
252.252506	-39.0603883964637\\
175.605552	-27.1918847287702\\
158.144276	-24.4880692810146\\
99.061232	-15.339273564847\\
94.558004	-14.6419649929439\\
175.8561	-27.2306811805761\\
130.969215	-20.2801093515399\\
151.666552	-23.4850171433875\\
323.58372	-50.1057689471381\\
449.21875	-69.5598990401068\\
454.39224	-70.3609952590091\\
333.69283	-51.6711280755924\\
444.469647	-68.82451761355\\
657.609326	-101.828426182996\\
293.100927	-45.3856186783871\\
200.259864	-31.009515790749\\
143.945652	-22.2894636975468\\
108.304688	-16.7705893016493\\
185.87084	-28.7814274557771\\
319.78564	-49.5176499931847\\
471.935087	-73.0774416811498\\
287.78175	-44.5619633543463\\
514.744776	-79.7063672208304\\
719.802232	-111.458772784448\\
725.8251	-112.391392115257\\
515.596246	-79.8382142713724\\
331.960178	-51.4028331787484\\
427.300593	-66.1659516858052\\
586.234444	-90.7762838004272\\
736.363968	-114.023297715985\\
381.785015	-59.1180290191041\\
214.110036	-33.1541648420366\\
320.428205	-49.6171488598877\\
253.650096	-39.2768001542086\\
117.217422	-18.1506939326589\\
95.2172	-14.7440391098583\\
221.721588	-34.3327861454852\\
497.569527	-77.046842019737\\
488.170137	-75.5913804668998\\
351.02781	-54.3553870444406\\
226.487328	-35.0707437513318\\
222.88904	-34.5135618661205\\
79.900415	-12.3722903388664\\
154.787296	-23.9682530670217\\
132.84057	-20.5698819063769\\
230.80045	-35.7386150965676\\
332.579544	-51.4987397641901\\
520.497351	-80.5971326579825\\
552.496732	-85.5521210944756\\
595.683198	-92.2393891901617\\
624.73598	-96.7381074265538\\
610.209589	-94.4887483083578\\
591.979572	-91.665896096558\\
518.97996	-80.3621701485939\\
653.60548	-101.208445108003\\
994.744307	-154.032558893949\\
867.715032	-134.362535004389\\
418.556523	-64.8119640652982\\
236.018097	-36.546548867246\\
523.66014	-81.0868790978298\\
639.754824	-99.0637211110652\\
559.714544	-86.6697731827076\\
320.796376	-49.674158807913\\
340.394009	-52.7087814119451\\
606.150792	-93.86025826972\\
927.97614	-143.693749671019\\
962.285714	-149.006463139789\\
1099.852796	-170.308228338063\\
1047.10743	-162.140799142836\\
763.40945	-118.211192805874\\
738.266508	-114.317899155305\\
339.024884	-52.4967773565194\\
328.994568	-50.943619194042\\
428.596971	-66.3666911313377\\
527.873346	-81.7392788117669\\
551.515272	-85.4001455624839\\
566.850808	-87.7747978580213\\
376.609041	-58.3165481617841\\
512.56638	-79.3690504774189\\
397.51054	-61.5530697010718\\
225.698752	-34.9486356092619\\
154.483081	-23.9211464743283\\
130.98111	-20.2819512492769\\
200.94102	-31.1149903342549\\
156.465786	-24.2281611740181\\
95.894707	-14.8489486189092\\
217.424658	-33.6674220728982\\
411.78039	-63.7627043730829\\
265.526664	-41.1158437627467\\
250.580968	-38.8015567815285\\
355.406805	-55.0334585855264\\
580.968792	-89.9609166290874\\
376.726449	-58.3347283660309\\
198.178554	-30.6872324633697\\
132.763332	-20.5579218813733\\
315.713258	-48.887056366421\\
668.371736	-103.494946454085\\
878.693013	-136.062435665313\\
1261.077985	-195.273365856393\\
1505.195082	-233.074015587251\\
1448.307585	-224.265192384826\\
928.343	-143.750556615435\\
981.795135	-152.0274264346\\
1266.8065	-196.160405689547\\
1048.22316	-162.3135658988\\
973.67136	-150.769489251826\\
1119.733679	-173.386711180349\\
1220.394825	-188.973725642727\\
1270.625856	-196.751819155165\\
1710.818628	-264.914078140358\\
1082.673139	-167.648020574109\\
577.098115	-89.3615562921983\\
325.906208	-50.4653978157057\\
260.517488	-40.3401909725768\\
286.830819	-44.4147151276102\\
279.833268	-43.3311696587535\\
289.696524	-44.8584591843281\\
515.77317	-79.8656103130064\\
388.652528	-60.1814385738848\\
441.209808	-68.3197433321403\\
543.557007	-84.1678370047168\\
311.738644	-48.2716017609234\\
383.847301	-59.4373665488751\\
510.511969	-79.0509323629215\\
579.320343	-89.7056602623111\\
317.708016	-49.1959374295767\\
221.018454	-34.2239083881528\\
283.177242	-43.8489719406768\\
294.139872	-45.5464955547602\\
723.3813	-112.012978522159\\
990.63531	-153.396295566837\\
941.601248	-145.803548375749\\
1107.267861	-171.456424340068\\
1285.296737	-199.023552027378\\
1684.15506	-260.785321052349\\
1383.52872	-214.234419383181\\
860.954292	-133.315658862562\\
533.596668	-82.6255145276492\\
300.39093	-46.5144492819229\\
344.415532	-53.3314997064675\\
348.692894	-53.9938337450133\\
452.041712	-69.9970244978355\\
310.771846	-48.1218965866132\\
289.583514	-44.8409599944779\\
314.692749	-48.7290342379825\\
411.49612	-63.7186861915174\\
463.269696	-71.7356372192857\\
566.624465	-87.7397494629388\\
386.775222	-59.8907445282168\\
179.231042	-27.7532787453176\\
134.711925	-20.8596544612139\\
120.6036	-18.6750313513704\\
133.789065	-20.7167529273217\\
165.956884	-25.6978233790347\\
296.001321	-45.8347342013179\\
294.10147	-45.5405491439238\\
385.027038	-59.6200445573373\\
552.62786	-85.5724257911102\\
409.333692	-63.3838420351648\\
721.33785	-111.696557678876\\
1047.10743	-162.140799142836\\
812.729995	-125.848300853574\\
736.854318	-114.099226640355\\
907.422784	-140.511136816379\\
1025.590218	-158.808936671947\\
674.347101	-104.420210117117\\
562.836072	-87.1531305059068\\
354.876462	-54.9513368897231\\
410.405468	-63.5498026731694\\
876.313892	-135.694037381483\\
1483.068132	-229.647737391905\\
1760.085368	-272.542854678371\\
1365.368106	-211.422313974966\\
1107.267861	-171.456424340068\\
803.734029	-124.455308048336\\
886.824905	-137.321629736186\\
1132.908545	-175.426791539477\\
586.898536	-90.8791160452374\\
509.171072	-78.8432993112219\\
399.841623	-61.9140294743095\\
741.02913	-114.745681182226\\
686.85738	-106.357381582475\\
365.47758	-56.592881677852\\
365.940768	-56.6646046647411\\
278.600322	-43.1402524290477\\
525.925876	-81.4377201244154\\
757.849715	-117.350287945402\\
792.694532	-122.745881857257\\
578.811985	-89.626942864259\\
422.97682	-65.4964310717339\\
483.879396	-74.9269747385831\\
357.604065	-55.3736963511247\\
264.173965	-40.9063832140237\\
204.528285	-31.6704653477759\\
179.857263	-27.8502468027213\\
158.78974	-24.5880170473846\\
191.51054	-29.6547146073407\\
349.373574	-54.0992345816405\\
498.130306	-77.1336766200821\\
478.790144	-74.138922469372\\
297.93324	-46.1338848725524\\
265.885095	-41.1713455069923\\
225.705649	-34.949703584772\\
301.8175	-46.7353484878747\\
396.755636	-61.4361755464423\\
425.096568	-65.8246663844194\\
391.10918	-60.5618422527087\\
235.293003	-36.434270683322\\
509.171072	-78.8432993112219\\
732.47185	-113.420617317777\\
476.567682	-73.7947822652054\\
197.0364	-30.5103740465555\\
230.861873	-35.748126226009\\
388.466254	-60.1525947185608\\
386.04925	-59.7783303891678\\
326.134204	-50.5007021718608\\
215.51019	-33.37097362591\\
406.804409	-62.9921917086282\\
568.650888	-88.0535337368423\\
325.268532	-50.3666559929739\\
130.98111	-20.2819512492769\\
216.872502	-33.5819227129219\\
486.714437	-75.3659705857835\\
561.546828	-86.9534957344774\\
339.504264	-52.5710076160574\\
316.680056	-49.0367615407313\\
599.021904	-92.7563757413344\\
773.503264	-119.774183404039\\
684.342257	-105.967923880735\\
994.721172	-154.028976522856\\
1177.84314	-182.384751089491\\
722.010542	-111.800721602589\\
604.99611	-93.6814599374077\\
871.41992	-134.93622351413\\
592.068206	-91.679620750953\\
818.212085	-126.697182666563\\
969.544152	-150.130406017215\\
725.071369	-112.274679595436\\
350.893268	-54.3345537022512\\
654.919552	-101.411924405453\\
1140.804732	-176.649487543421\\
1276.761081	-197.701836561136\\
920.48779	-142.534205751766\\
787.82115	-121.991256276949\\
975.560315	-151.06198710302\\
737.208208	-114.154025227171\\
468.883688	-72.6049435800108\\
450.584784	-69.7714244653598\\
566.621517	-87.7392929757989\\
589.285635	-91.2487496935185\\
585.206644	-90.6171326870033\\
643.94732	-99.7129139563907\\
437.815659	-67.7941716329023\\
494.008196	-76.4953827923374\\
667.857177	-103.415268853606\\
381.453512	-59.0666969782855\\
310.771846	-48.1218965866132\\
247.8498	-38.3786452927681\\
198.9834	-30.8118599561065\\
248.787675	-38.5238718451154\\
429.124878	-66.4484356213497\\
393.464542	-60.9265615413023\\
580.036974	-89.8166279812877\\
722.785054	-111.920651985115\\
866.41016	-134.160480293672\\
628.974226	-97.394384493945\\
540.217179	-83.6506767159014\\
254.26577	-39.3721350468008\\
342.358744	-53.0130135221154\\
647.110128	-100.202663338319\\
419.749902	-64.9967544881302\\
251.13755	-38.8877414916002\\
480.625	-74.4230432860413\\
722.010542	-111.800721602589\\
770.0562	-119.240418008394\\
1043.717788	-161.61592533625\\
1340.594886	-207.586270439165\\
876.232071	-135.681367695502\\
765.389312	-118.517767277294\\
902.206574	-139.703425559954\\
570.739496	-88.3769471331316\\
475.958856	-73.7005077606506\\
535.081638	-82.8554567624236\\
703.71714	-108.968054452742\\
750.11468	-116.152545745925\\
733.804286	-113.626940214222\\
402.028332	-62.2526334557107\\
352.663344	-54.6086435701681\\
312.36799	-48.3690537132769\\
437.181498	-67.6959740952101\\
387.466004	-59.9977096230147\\
321.02884	-49.7101550177116\\
653.60548	-101.208445108003\\
679.389744	-105.201045151219\\
467.8125	-72.43907399168\\
393.293063	-60.9000086382289\\
573.890636	-88.8648897674523\\
304.863636	-47.207031632495\\
185.262539	-28.6872342455738\\
237.859424	-36.8316716948657\\
327.771949	-50.7543010629433\\
272.70438	-42.2272871303678\\
188.6112	-29.2057622924987\\
118.280659	-18.3153323373909\\
224.215178	-34.7189095400394\\
320.857812	-49.6836719503573\\
520.54002	-80.6037397983392\\
595.106556	-92.150098261628\\
801.962101	-124.180931404916\\
876.070148	-135.656294504474\\
483.688318	-74.8973870012307\\
385.302384	-59.6626808897725\\
835.566306	-129.384421034685\\
1178.97857	-182.56056831922\\
894.38132	-138.491713274563\\
861.01408	-133.324916818166\\
1291.811409	-200.032325429199\\
1002.107456	-155.172715891085\\
809.675158	-125.375270390567\\
590.376314	-91.4176373927188\\
630.044244	-97.5600729120065\\
815.56006	-126.286526185168\\
524.144166	-81.1618287736676\\
470.382771	-72.8370711618019\\
833.131	-129.007322706773\\
1040.219505	-161.074228863666\\
1059.469432	-164.05501045092\\
426.488166	-66.0401503026317\\
238.167111	-36.8793159142042\\
571.389526	-88.4776019280907\\
418.431546	-64.7926118287712\\
194.4522	-30.1102200211515\\
163.623745	-25.3365452416316\\
289.583514	-44.8409599944779\\
406.929479	-63.0115583458688\\
661.814712	-102.479614998211\\
452.041712	-69.9970244978355\\
346.43466	-53.644154376\\
198.48582	-30.7348114923805\\
145.5541	-22.5385260541187\\
197.826642	-30.632740162701\\
165.956884	-25.6978233790347\\
229.411685	-35.5235698581611\\
393.06425	-60.8645777725781\\
335.265255	-51.9146124008752\\
176.825035	-27.3807172616089\\
165.40756	-25.6127625440176\\
215.276193	-33.3347400366048\\
256.268481	-39.6822476032479\\
159.873729	-24.7558688242764\\
257.173014	-39.8223112674615\\
173.199545	-26.8193232450614\\
133.66665	-20.6977974072304\\
321.807112	-49.8306676226412\\
470.625	-72.8745794465398\\
732.563539	-113.434815027875\\
992.8125	-153.733425565509\\
836.048279	-129.459052810897\\
388.466254	-60.1525947185608\\
294.8025	-45.6491011044644\\
282.54897	-43.7516862933397\\
157.178405	-24.3385076493023\\
169.863188	-26.3027004280447\\
325.649148	-50.4255930042475\\
352.244448	-54.543779039317\\
326.739235	-50.5943890343886\\
380.215024	-58.8749215899198\\
450.580536	-69.7707666779208\\
473.566948	-73.3301294561091\\
497.051904	-76.9666899699375\\
626.91792	-97.07597294555\\
625.011912	-96.7808344989699\\
877.685088	-135.90636212377\\
435.122832	-67.377197109349\\
209.26626	-32.4041236437721\\
209.215773	-32.3963059143856\\
383.565107	-59.3936698284027\\
281.325429	-43.5622253938749\\
255.411555	-39.5495557108351\\
124.012971	-19.2029601222649\\
218.850592	-33.8882227965498\\
129.55929	-20.0617875636489\\
80.487568	-12.4632088577419\\
136.12185	-21.0779762491048\\
260.648325	-40.3604506089137\\
334.747522	-51.8344433179736\\
507.395496	-78.5683577881932\\
709.027104	-109.790283177048\\
470.418965	-72.8426756718226\\
200.9304	-31.1133458656574\\
382.588072	-59.24237949166\\
413.120724	-63.9702502462683\\
216.721524	-33.5585443155658\\
299.46807	-46.3715477480307\\
546.989608	-84.6993628571105\\
503.153624	-77.9115192478139\\
876.464406	-135.717343930117\\
983.248924	-152.252540404277\\
653.018338	-101.117528292437\\
497.051904	-76.9666899699375\\
};
\end{axis}

\begin{axis}[%
width=4.927cm,
height=3cm,
at={(7cm,0cm)},
scale only axis,
xmin=0,
xmax=1077.987963,
xlabel style={font=\color{white!15!black}},
xlabel={y(t-1)u(t)},
ymin=-200,
ymax=0,
ylabel style={font=\color{white!15!black}},
ylabel={y(t)},
axis background/.style={fill=white},
title style={font=\small},
title={C10, R = -0.8059},
axis x line*=bottom,
axis y line*=left
]
\addplot[only marks, mark=*, mark options={}, mark size=1.5000pt, color=mycolor1, fill=mycolor1] table[row sep=crcr]{%
x	y\\
336.952888	-63.477\\
396.350388	-75.684\\
467.045964	-62.256\\
381.878304	-59.814\\
374.555268	-80.566\\
506.035046	-76.904\\
487.18684	-90.332\\
564.033008	-68.359\\
406.804409	-40.283\\
238.233662	-48.828\\
290.575428	-46.387\\
276.884003	-54.932\\
326.900332	-45.166\\
252.25211	-21.973\\
115.863629	-17.09\\
87.92805	-18.311\\
100.582323	-40.283\\
237.508568	-62.256\\
373.909536	-67.139\\
406.929479	-69.58\\
415.32302	-52.49\\
300.82019	-34.18\\
204.66984	-64.697\\
382.618058	-52.49\\
301.8175	-39.063\\
231.760779	-61.035\\
360.96099	-53.711\\
315.713258	-45.166\\
272.892972	-73.242\\
445.238118	-67.139\\
400.752691	-52.49\\
319.08671	-75.684\\
464.245656	-74.463\\
451.320243	-58.594\\
349.747586	-50.049\\
294.188022	-41.504\\
243.960512	-47.607\\
284.166183	-58.594\\
349.747586	-54.932\\
327.889108	-59.814\\
358.166232	-59.814\\
359.242884	-65.918\\
407.966502	-85.449\\
530.381943	-76.904\\
464.653968	-54.932\\
327.889108	-50.049\\
287.78175	-35.4\\
200.2932	-34.18\\
195.88558	-45.166\\
256.362216	-34.18\\
193.39044	-36.621\\
212.548284	-51.27\\
301.36506	-54.932\\
321.846588	-51.27\\
311.67033	-78.125\\
472.03125	-62.256\\
361.333824	-36.621\\
207.201618	-30.518\\
173.80001	-40.283\\
233.077438	-46.387\\
259.906361	-29.297\\
160.928421	-29.297\\
164.151091	-36.621\\
203.832486	-31.738\\
174.336834	-26.855\\
148.99154	-35.4\\
198.3462	-40.283\\
235.293003	-58.594\\
345.470224	-56.152\\
335.171288	-65.918\\
395.903508	-61.035\\
367.67484	-70.801\\
423.956388	-57.373\\
341.426723	-58.594\\
346.524916	-51.27\\
301.36506	-50.049\\
295.088904	-48.828\\
298.632048	-80.566\\
514.81674	-111.084\\
724.156596	-114.746\\
748.029174	-112.305\\
711.452175	-74.463\\
479.914035	-100.098\\
661.64778	-125.732\\
838.00378	-128.174\\
840.18057	-96.436\\
642.74594	-124.512\\
854.899392	-158.691\\
1077.987963	-115.967\\
772.920055	-97.656\\
631.248384	-68.359\\
426.833596	-53.711\\
329.463274	-47.607\\
292.878264	-56.152\\
339.270384	-45.166\\
261.330476	-30.518\\
172.670844	-30.518\\
173.80001	-36.621\\
215.258238	-50.049\\
295.088904	-48.828\\
287.889888	-47.607\\
285.927642	-62.256\\
377.333616	-64.697\\
402.803522	-85.449\\
533.543556	-72.021\\
452.363901	-85.449\\
528.843861	-63.477\\
387.019269	-54.932\\
336.952888	-64.697\\
390.899274	-48.828\\
291.454332	-43.945\\
263.93367	-53.711\\
323.555064	-54.932\\
324.867848	-40.283\\
236.783474	-43.945\\
255.05678	-32.959\\
190.107512	-36.621\\
211.229928	-40.283\\
228.646308	-30.518\\
173.220168	-34.18\\
197.15024	-43.945\\
262.307705	-63.477\\
384.734097	-65.918\\
403.154488	-72.021\\
427.300593	-45.166\\
270.454008	-56.152\\
351.623824	-89.111\\
554.805086	-70.801\\
442.081444	-81.787\\
513.704147	-80.566\\
488.310526	-50.049\\
293.237091	-36.621\\
215.917416	-47.607\\
284.166183	-52.49\\
326.80274	-81.787\\
524.172883	-103.76\\
670.70464	-102.539\\
651.532806	-76.904\\
487.18684	-80.566\\
506.035046	-72.021\\
450.995502	-69.58\\
435.70996	-68.359\\
416.784823	-51.27\\
306.03063	-45.166\\
267.111724	-39.063\\
228.166983	-37.842\\
221.716278	-41.504\\
249.273024	-56.152\\
350.613088	-84.229\\
522.809403	-70.801\\
429.124861	-50.049\\
297.841599	-42.725\\
251.9066	-40.283\\
229.411685	-28.076\\
153.7161	-23.193\\
126.564201	-24.414\\
138.134412	-41.504\\
234.08256	-32.959\\
185.262539	-34.18\\
193.39044	-37.842\\
213.42888	-35.4\\
195.762	-26.855\\
144.077075	-21.973\\
114.677087	-18.311\\
97.23141	-25.635\\
145.50426	-51.27\\
307.00476	-73.242\\
449.266428	-78.125\\
469.21875	-54.932\\
318.825328	-39.063\\
220.31532	-31.738\\
172.591244	-24.414\\
137.69496	-41.504\\
237.112352	-42.725\\
249.556725	-54.932\\
330.910368	-73.242\\
466.698024	-108.643\\
718.13023	-125.732\\
826.436436	-103.76\\
680.1468	-101.318\\
643.774572	-67.139\\
425.325565	-75.684\\
480.896136	-79.346\\
509.956742	-86.67\\
547.49439	-69.58\\
434.45752	-62.256\\
387.605856	-61.035\\
376.646985	-56.152\\
348.535464	-67.139\\
410.622124	-52.49\\
330.63451	-76.904\\
494.262008	-97.656\\
616.892952	-72.021\\
437.815659	-45.166\\
271.266996	-47.607\\
296.401182	-78.125\\
499.21875	-93.994\\
614.438778	-112.305\\
744.35754	-119.629\\
799.480607	-119.629\\
784.168095	-91.553\\
595.0945	-84.229\\
550.604973	-96.436\\
637.44196	-111.084\\
715.93638	-74.463\\
458.096376	-46.387\\
287.089143	-62.256\\
384.181776	-54.932\\
326.900332	-34.18\\
204.02042	-47.607\\
286.784568	-53.711\\
320.600959	-42.725\\
250.325775	-34.18\\
201.52528	-43.945\\
258.30871	-40.283\\
238.233662	-52.49\\
321.02884	-65.918\\
404.341012	-62.256\\
373.909536	-45.166\\
269.595854	-45.166\\
275.377102	-67.139\\
431.502353	-104.98\\
670.8222	-79.346\\
488.136592	-51.27\\
305.10777	-40.283\\
239.724133	-47.607\\
282.452331	-40.283\\
233.802532	-31.738\\
183.064784	-35.4\\
207.4086	-42.725\\
253.487425	-51.27\\
307.92762	-57.373\\
337.238494	-42.725\\
258.14445	-63.477\\
394.001739	-75.684\\
469.770588	-70.801\\
434.293334	-57.373\\
351.925982	-62.256\\
391.029936	-81.787\\
503.153624	-52.49\\
303.70714	-28.076\\
160.903556	-36.621\\
211.229928	-42.725\\
251.9066	-51.27\\
302.28792	-46.387\\
268.395182	-36.621\\
215.917416	-52.49\\
311.42317	-52.49\\
305.64927	-37.842\\
222.435276	-47.607\\
278.929413	-41.504\\
240.889216	-37.842\\
220.353966	-45.166\\
256.362216	-29.297\\
165.762426	-31.738\\
176.082424	-29.297\\
161.455767	-26.855\\
154.89964	-47.607\\
279.833946	-53.711\\
323.555064	-69.58\\
434.45752	-89.111\\
548.210872	-62.256\\
365.940768	-36.621\\
208.556595	-31.738\\
178.399298	-29.297\\
168.45775	-40.283\\
226.430743	-31.738\\
177.25673	-30.518\\
173.220168	-39.063\\
230.315448	-61.035\\
364.317915	-53.711\\
329.463274	-79.346\\
479.408532	-54.932\\
326.900332	-48.828\\
278.954364	-32.959\\
183.449794	-29.297\\
158.78974	-23.193\\
123.572304	-23.193\\
124.847919	-25.635\\
143.632905	-40.283\\
229.411685	-46.387\\
267.560216	-48.828\\
284.325444	-51.27\\
312.59319	-79.346\\
488.136592	-73.242\\
441.209808	-53.711\\
325.542371	-63.477\\
388.225332	-69.58\\
434.45752	-85.449\\
532.005474	-72.021\\
432.558126	-47.607\\
271.121865	-28.076\\
153.210732	-21.973\\
118.280659	-21.973\\
118.698146	-26.855\\
145.5541	-26.855\\
144.560465	-25.635\\
139.890195	-32.959\\
187.075284	-47.607\\
272.835717	-43.945\\
251.057785	-45.166\\
261.330476	-52.49\\
296.0436	-35.4\\
191.2308	-20.752\\
114.75856	-35.4\\
205.4616	-56.152\\
327.983832	-54.932\\
324.867848	-59.814\\
351.586692	-53.711\\
308.83825	-40.283\\
234.567909	-51.27\\
308.85048	-72.021\\
439.112037	-70.801\\
423.956388	-51.27\\
315.41304	-81.787\\
501.681458	-62.256\\
377.333616	-63.477\\
389.367918	-73.242\\
450.584784	-70.801\\
429.124861	-54.932\\
328.932816	-47.607\\
293.782797	-78.125\\
500.703125	-108.643\\
694.22877	-85.449\\
522.606084	-50.049\\
300.594294	-51.27\\
307.92762	-51.27\\
311.67033	-61.035\\
375.48732	-72.021\\
437.815659	-54.932\\
331.899144	-56.152\\
346.513992	-73.242\\
457.323048	-79.346\\
485.280136	-56.152\\
335.171288	-45.166\\
272.079984	-57.373\\
359.269726	-87.891\\
553.625409	-78.125\\
492.109375	-80.566\\
491.210902	-54.932\\
328.932816	-45.166\\
269.595854	-47.607\\
274.597176	-31.738\\
174.908118	-20.752\\
110.566656	-20.752\\
107.9104	-17.09\\
92.93542	-31.738\\
180.144888	-48.828\\
284.325444	-54.932\\
316.847776	-40.283\\
232.352344	-45.166\\
263.814606	-52.49\\
297.93324	-35.4\\
198.9834	-35.4\\
198.9834	-35.4\\
195.0894	-29.297\\
162.539756	-31.738\\
183.064784	-51.27\\
304.18491	-64.697\\
373.172296	-41.504\\
233.293984	-32.959\\
182.856532	-28.076\\
156.80446	-32.959\\
179.857263	-26.855\\
152.939225	-45.166\\
274.564114	-80.566\\
488.310526	-63.477\\
392.859153	-81.787\\
518.120645	-98.877\\
631.82403	-97.656\\
615.135144	-76.904\\
471.729136	-58.594\\
357.247618	-54.932\\
332.942852	-57.373\\
350.893268	-65.918\\
409.153026	-76.904\\
478.804304	-76.904\\
491.41656	-100.098\\
652.538862	-108.643\\
720.085804	-130.615\\
851.479185	-91.553\\
596.834007	-98.877\\
651.797184	-108.643\\
704.223926	-85.449\\
557.042031	-96.436\\
625.098152	-85.449\\
530.381943	-51.27\\
304.18491	-34.18\\
199.03014	-36.621\\
218.590749	-51.27\\
303.21078	-43.945\\
252.68375	-31.738\\
183.064784	-41.504\\
237.112352	-32.959\\
183.449794	-23.193\\
129.092238	-31.738\\
178.399298	-34.18\\
189.0154	-25.635\\
146.452755	-42.725\\
251.9066	-58.594\\
341.192862	-45.166\\
272.079984	-68.359\\
433.054265	-100.098\\
636.022692	-87.891\\
571.2915	-109.863\\
702.02457	-81.787\\
521.146764	-81.787\\
509.205862	-62.256\\
370.485456	-37.842\\
225.878898	-46.387\\
272.662786	-42.725\\
244.856975	-29.297\\
169.512442	-40.283\\
228.646308	-21.973\\
119.09366	-19.531\\
104.783815	-24.414\\
137.69496	-45.166\\
264.627594	-57.373\\
343.549524	-65.918\\
398.276556	-65.918\\
397.090032	-62.256\\
379.574832	-69.58\\
419.14992	-57.373\\
341.426723	-48.828\\
289.696524	-50.049\\
292.336209	-43.945\\
263.14266	-59.814\\
353.739996	-46.387\\
267.560216	-36.621\\
215.258238	-51.27\\
310.74747	-69.58\\
417.89748	-54.932\\
323.879072	-41.504\\
240.889216	-37.842\\
220.353966	-41.504\\
237.112352	-32.959\\
184.669277	-30.518\\
173.220168	-41.504\\
240.889216	-47.607\\
279.833946	-52.49\\
305.64927	-42.725\\
251.13755	-53.711\\
315.713258	-51.27\\
291.98265	-32.959\\
185.88876	-32.959\\
194.326264	-58.594\\
359.415596	-80.566\\
500.073162	-78.125\\
479.21875	-67.139\\
410.622124	-62.256\\
373.909536	-50.049\\
297.841599	-48.828\\
286.083252	-41.504\\
246.990304	-54.932\\
325.911556	-50.049\\
288.682632	-32.959\\
192.513519	-45.166\\
267.111724	-52.49\\
315.25494	-61.035\\
371.031765	-68.359\\
416.784823	-68.359\\
428.064058	-87.891\\
566.457495	-107.422\\
708.125824	-119.629\\
768.855583	-80.566\\
491.210902	-46.387\\
270.111501	-32.959\\
184.669277	-24.414\\
136.791642	-34.18\\
192.12578	-29.297\\
170.596431	-51.27\\
299.46807	-48.828\\
282.518808	-40.283\\
237.508568	-52.49\\
313.31281	-63.477\\
385.876683	-70.801\\
435.567752	-75.684\\
482.258448	-103.76\\
664.99784	-92.773\\
586.047041	-72.021\\
443.073192	-51.27\\
310.74747	-52.49\\
318.14189	-54.932\\
337.941664	-65.918\\
399.528998	-51.27\\
307.92762	-50.049\\
299.693412	-48.828\\
280.761	-30.518\\
177.706314	-45.166\\
273.751126	-68.359\\
408.034871	-48.828\\
291.454332	-52.49\\
325.80543	-86.67\\
536.40063	-63.477\\
389.367918	-63.477\\
399.841623	-85.449\\
538.243251	-81.787\\
501.681458	-53.711\\
317.646854	-35.4\\
206.1342	-36.621\\
219.945726	-61.035\\
383.360835	-90.332\\
580.563764	-97.656\\
629.39292	-100.098\\
636.022692	-79.346\\
488.136592	-53.711\\
323.555064	-46.387\\
271.781433	-37.842\\
218.272656	-34.18\\
195.27034	-29.297\\
169.512442	-40.283\\
235.293003	-42.725\\
243.318875	-30.518\\
176.577148	-42.725\\
253.487425	-56.152\\
338.259648	-62.256\\
382.998912	-78.125\\
496.40625	-96.436\\
628.666284	-114.746\\
737.472542	-84.229\\
533.590715	-73.242\\
462.669714	-76.904\\
481.572848	-63.477\\
388.225332	-46.387\\
282.821539	-57.373\\
363.457955	-90.332\\
583.906048	-101.318\\
636.378358	-61.035\\
364.317915	-36.621\\
208.556595	-26.855\\
143.593685	-18.311\\
95.894707	-21.973\\
119.09366	-30.518\\
168.76454	-31.738\\
179.573604	-42.725\\
240.969	-37.842\\
209.26626	-30.518\\
167.635374	-29.297\\
160.928421	-29.297\\
160.401075	-26.855\\
147.514515	-31.738\\
179.00232	-43.945\\
259.09972	-63.477\\
381.242862	-68.359\\
411.794616	-65.918\\
403.154488	-74.463\\
466.287306	-91.553\\
581.727762	-92.773\\
599.684672	-106.201\\
690.3065	-100.098\\
637.824456	-74.463\\
459.511173	-54.932\\
334.920404	-51.27\\
322.02687	-84.229\\
541.339783	-97.656\\
615.135144	-69.58\\
425.55128	-46.387\\
265.008931	-26.855\\
146.03749	-21.973\\
117.885145	-20.752\\
107.536864	-14.648\\
78.044544	-25.635\\
141.274485	-40.283\\
224.215178	-40.283\\
223.490084	-35.4\\
190.5582	-24.414\\
128.29557	-19.531\\
102.986963	-28.076\\
149.08356	-28.076\\
150.62774	-30.518\\
162.05058	-24.414\\
127.416666	-21.973\\
117.885145	-32.959\\
193.106781	-69.58\\
426.80372	-79.346\\
494.008196	-86.67\\
531.63378	-63.477\\
394.001739	-80.566\\
522.228812	-114.746\\
750.094602	-103.76\\
680.1468	-106.201\\
684.465445	-81.787\\
524.172883	-81.787\\
527.117215	-86.67\\
544.37427	-59.814\\
369.112194	-56.152\\
334.160552	-40.283\\
233.077438	-35.4\\
213.8868	-62.256\\
371.606064	-47.607\\
281.547798	-48.828\\
291.454332	-53.711\\
318.667363	-50.049\\
297.841599	-50.049\\
298.742481	-52.49\\
316.19976	-57.373\\
347.737753	-64.697\\
390.899274	-56.152\\
332.082928	-41.504\\
247.737376	-52.49\\
300.82019	-31.738\\
167.957496	-18.311\\
96.224305	-25.635\\
139.40313	-34.18\\
180.88056	-21.973\\
109.42554	-9.766\\
49.17181	-21.973\\
116.281116	-32.959\\
179.231042	-37.842\\
211.34757	-43.945\\
246.223835	-39.063\\
224.61225	-58.594\\
341.192862	-48.828\\
277.147728	-36.621\\
211.229928	-54.932\\
307.783996	-39.063\\
211.72146	-26.855\\
141.606415	-23.193\\
118.05237	-17.09\\
88.25276	-25.635\\
131.430645	-19.531\\
101.932289	-29.297\\
162.01241	-43.945\\
246.223835	-42.725\\
236.26925	-32.959\\
182.26327	-40.283\\
220.549425	-29.297\\
162.01241	-43.945\\
240.598875	-31.738\\
172.01996	-31.738\\
170.845654	-30.518\\
161.501256	-24.414\\
130.077792	-26.855\\
142.60005	-26.855\\
146.03749	-41.504\\
228.728544	-36.621\\
198.48582	-30.518\\
164.858236	-32.959\\
175.01229	-26.855\\
143.08344	-31.738\\
180.144888	-58.594\\
349.747586	-75.684\\
444.870552	-51.27\\
297.57108	-48.828\\
277.147728	-39.063\\
228.870117	-58.594\\
346.524916	-58.594\\
335.802214	-37.842\\
217.5915	-46.387\\
262.457646	-39.063\\
223.870053	-51.27\\
287.26581	-32.959\\
181.043787	-35.4\\
197.709	-37.842\\
214.791192	-47.607\\
268.50348	-36.621\\
205.187463	-41.504\\
233.293984	-41.504\\
240.889216	-57.373\\
334.082979	-51.27\\
307.00476	-72.021\\
450.995502	-93.994\\
585.206644	-78.125\\
470.625	-50.049\\
291.435327	-40.283\\
238.233662	-54.932\\
333.931628	-69.58\\
416.64504	-56.152\\
339.270384	-63.477\\
374.260392	-43.945\\
244.59787	-23.193\\
126.123534	-24.414\\
139.477182	-52.49\\
305.64927	-45.166\\
259.7045	-42.725\\
253.487425	-64.697\\
386.176393	-61.035\\
360.96099	-56.152\\
326.973096	-39.063\\
227.463849	-53.711\\
319.634161	-65.918\\
391.091494	-52.49\\
310.42586	-53.711\\
316.680056	-48.828\\
281.639904	-37.842\\
218.953812	-42.725\\
242.5071	-35.4\\
200.2932	-36.621\\
215.258238	-57.373\\
343.549524	-67.139\\
404.445336	-70.801\\
426.505224	-62.256\\
365.940768	-45.166\\
258.033358	-35.4\\
201.603	-39.063\\
225.315384	-46.387\\
272.662786	-57.373\\
344.582238	-68.359\\
419.314106	-79.346\\
482.344334	-65.918\\
386.213562	-40.283\\
241.939698	-68.359\\
425.603134	-91.553\\
563.234056	-65.918\\
401.902046	-64.697\\
399.245187	-75.684\\
461.445348	-59.814\\
358.166232	-50.049\\
300.594294	-56.152\\
335.171288	-51.27\\
307.92762	-57.373\\
350.893268	-70.801\\
427.779642	-56.152\\
340.337272	-63.477\\
384.734097	-58.594\\
354.024948	-59.814\\
357.029766	-48.828\\
294.139872	-63.477\\
377.751627	-45.166\\
266.298736	-48.828\\
290.575428	-52.49\\
311.42317	-50.049\\
285.029055	-30.518\\
164.278394	-19.531\\
101.209642	-15.869\\
82.820311	-28.076\\
158.854008	-53.711\\
319.634161	-67.139\\
404.445336	-65.918\\
387.466004	-46.387\\
272.662786	-54.932\\
322.890296	-50.049\\
289.583514	-43.945\\
246.223835	-30.518\\
171.541678	-42.725\\
242.5071	-37.842\\
213.42888	-36.621\\
208.556595	-41.504\\
234.829632	-37.842\\
212.709882	-34.18\\
187.1355	-25.635\\
141.274485	-35.4\\
205.4616	-58.594\\
336.9155	-42.725\\
246.4378	-52.49\\
302.76232	-45.166\\
265.485748	-59.814\\
353.739996	-57.373\\
346.647666	-76.904\\
461.885424	-58.594\\
351.915564	-64.697\\
389.734728	-65.918\\
382.588072	-37.842\\
223.116432	-61.035\\
356.505435	-46.387\\
271.781433	-51.27\\
303.21078	-57.373\\
345.614952	-69.58\\
415.32302	-53.711\\
323.555064	-70.801\\
435.567752	-81.787\\
501.681458	-69.58\\
421.72438	-58.594\\
349.747586	-50.049\\
299.693412	-58.594\\
348.692894	-52.49\\
308.53622	-43.945\\
255.05678	-37.842\\
220.353966	-45.166\\
262.143464	-39.063\\
236.018646	-72.021\\
452.363901	-90.332\\
565.658984	-79.346\\
492.500622	-75.684\\
469.770588	-74.463\\
448.565112	-46.387\\
272.662786	-42.725\\
246.4378	-35.4\\
200.2932	-30.518\\
172.670844	-32.959\\
191.920257	-52.49\\
313.31281	-59.814\\
362.532654	-70.801\\
426.505224	-59.814\\
351.586692	-42.725\\
258.956225	-72.021\\
447.034347	-84.229\\
530.558471	-90.332\\
565.658984	-73.242\\
472.04469	-115.967\\
747.407315	-85.449\\
524.144166	-51.27\\
303.21078	-39.063\\
226.018518	-34.18\\
202.78994	-56.152\\
342.358744	-69.58\\
435.70996	-89.111\\
554.805086	-68.359\\
416.784823	-53.711\\
311.738644	-31.738\\
180.144888	-36.621\\
208.556595	-34.18\\
196.535	-39.063\\
219.573123	-25.635\\
144.094335	-35.4\\
197.709	-26.855\\
144.560465	-19.531\\
108.357988	-39.063\\
221.018454	-42.725\\
246.4378	-48.828\\
282.518808	-46.387\\
269.230148	-48.828\\
284.325444	-48.828\\
287.889888	-62.256\\
362.516688	-43.945\\
259.89073	-62.256\\
377.333616	-69.58\\
416.64504	-56.152\\
335.171288	-58.594\\
348.692894	-54.932\\
327.889108	-61.035\\
375.48732	-80.566\\
491.210902	-62.256\\
371.606064	-50.049\\
299.693412	-57.373\\
342.459437	-51.27\\
297.57108	-36.621\\
215.917416	-56.152\\
343.425632	-74.463\\
456.756042	-69.58\\
429.37818	-78.125\\
502.109375	-113.525\\
725.42475	-79.346\\
495.436424	-72.021\\
441.776814	-56.152\\
345.447104	-68.359\\
433.054265	-92.773\\
577.604698	-64.697\\
395.686852	-57.373\\
358.237012	-76.904\\
468.883688	-51.27\\
298.54521	-32.959\\
191.294036	-42.725\\
244.087925	-32.959\\
184.669277	-25.635\\
142.68441	-28.076\\
154.726836	-25.635\\
139.890195	-25.635\\
141.76155	-31.738\\
179.573604	-42.725\\
246.4378	-48.828\\
289.696524	-63.477\\
378.894213	-59.814\\
361.396188	-67.139\\
414.314769	-79.346\\
488.136592	-73.242\\
442.528164	-54.932\\
330.910368	-58.594\\
351.915564	-53.711\\
322.588266	-57.373\\
352.958696	-76.904\\
467.499416	-59.814\\
364.685958	-63.477\\
387.019269	-62.256\\
376.150752	-57.373\\
357.204298	-84.229\\
524.409754	-69.58\\
430.63062	-73.242\\
450.584784	-65.918\\
393.464542	-42.725\\
244.856975	-29.297\\
162.539756	-24.414\\
138.573864	-41.504\\
236.36528	-34.18\\
197.15024	-47.607\\
273.74025	-35.4\\
205.4616	-45.166\\
274.564114	-79.346\\
501.228682	-93.994\\
588.590428	-75.684\\
472.570896	-79.346\\
495.436424	-70.801\\
430.399279	-52.49\\
317.14458	-54.932\\
334.920404	-59.814\\
361.396188	-53.711\\
326.509169	-62.256\\
385.302384	-76.904\\
494.262008	-106.201\\
682.553827	-93.994\\
602.407546	-87.891\\
566.457495	-97.656\\
615.135144	-68.359\\
429.362879	-74.463\\
462.191841	-59.814\\
369.112194	-59.814\\
374.555268	-75.684\\
479.45814	-85.449\\
513.206694	-37.842\\
223.797588	-48.828\\
286.083252	-37.842\\
225.197742	-53.711\\
322.588266	-53.711\\
312.759153	-35.4\\
208.0812	-46.387\\
282.821539	-74.463\\
485.424297	-118.408\\
784.808224	-118.408\\
780.545536	-106.201\\
688.394882	-85.449\\
557.042031	-97.656\\
645.50616	-115.967\\
758.076279	-86.67\\
563.355	-89.111\\
580.914609	-91.553\\
583.375716	-64.697\\
403.968068	-56.152\\
353.701448	-72.021\\
445.737969	-48.828\\
293.260968	-37.842\\
228.641364	-52.49\\
312.36799	-39.063\\
231.760779	-47.607\\
282.452331	-41.504\\
249.273024	-51.27\\
314.49018	-65.918\\
401.902046	-52.49\\
312.36799	-41.504\\
248.525952	-50.049\\
302.396058	-57.373\\
350.893268	-65.918\\
400.715522	-56.152\\
337.248912	-46.387\\
285.372824	-70.801\\
438.187389	-65.918\\
398.276556	-47.607\\
295.496649	-78.125\\
487.8125	-67.139\\
410.622124	-53.711\\
319.634161	-37.842\\
217.5915	-29.297\\
163.067102	-23.193\\
132.501609	-41.504\\
240.142144	-36.621\\
208.556595	-31.738\\
175.51114	-24.414\\
135.00942	-31.738\\
181.890478	-46.387\\
271.781433	-51.27\\
307.00476	-63.477\\
388.225332	-73.242\\
449.266428	-72.021\\
444.441591	-74.463\\
444.469647	-47.607\\
267.598947	-23.193\\
129.532905	-35.4\\
197.709	-26.855\\
150.951955	-37.842\\
221.035122	-52.49\\
312.36799	-58.594\\
362.638266	-85.449\\
520.982553	-56.152\\
338.259648	-57.373\\
354.048783	-74.463\\
454.000911	-57.373\\
340.394009	-41.504\\
240.142144	-34.18\\
198.38072	-45.166\\
268.782866	-54.932\\
337.941664	-79.346\\
480.916106	-53.711\\
318.667363	-47.607\\
282.452331	-45.166\\
269.595854	-50.049\\
305.148753	-68.359\\
435.583548	-103.76\\
668.7332	-89.111\\
572.716397	-90.332\\
564.033008	-59.814\\
358.166232	-43.945\\
263.14266	-48.828\\
277.147728	-21.973\\
124.323234	-39.063\\
221.721588	-30.518\\
173.220168	-36.621\\
215.917416	-58.594\\
359.415596	-76.904\\
483.034024	-86.67\\
561.79494	-111.084\\
722.046	-93.994\\
583.420758	-53.711\\
325.542371	-53.711\\
322.588266	-42.725\\
258.14445	-57.373\\
356.114211	-73.242\\
466.698024	-95.215\\
601.473155	-72.021\\
440.480436	-52.49\\
321.97366	-59.814\\
373.478616	-75.684\\
465.607968	-54.932\\
328.932816	-40.283\\
235.293003	-35.4\\
199.656	-25.635\\
142.22298	-24.414\\
131.420562	-19.531\\
107.283783	-32.959\\
187.075284	-41.504\\
239.395072	-46.387\\
264.173965	-36.621\\
206.54244	-35.4\\
189.921	-19.531\\
98.338585	-13.428\\
67.126572	-20.752\\
108.678224	-29.297\\
160.401075	-39.063\\
221.018454	-48.828\\
282.518808	-52.49\\
309.48104	-62.256\\
379.574832	-79.346\\
508.528514	-107.422\\
698.243	-103.76\\
680.1468	-112.305\\
736.159275	-98.877\\
619.167774	-61.035\\
373.29006	-54.932\\
335.964112	-53.711\\
333.384177	-69.58\\
429.37818	-59.814\\
361.396188	-45.166\\
267.111724	-36.621\\
211.229928	-31.738\\
186.555964	-50.049\\
296.940717	-47.607\\
271.121865	-28.076\\
154.726836	-21.973\\
122.719205	-32.959\\
188.294767	-43.945\\
249.43182	-32.959\\
190.107512	-46.387\\
265.843897	-36.621\\
214.562439	-51.27\\
313.56732	-76.904\\
467.499416	-61.035\\
377.745615	-80.566\\
517.797682	-104.98\\
686.25426	-113.525\\
733.8256	-87.891\\
548.791404	-59.814\\
362.532654	-45.166\\
261.330476	-31.738\\
184.810374	-40.283\\
232.352344	-30.518\\
180.483452	-54.932\\
326.900332	-47.607\\
278.929413	-40.283\\
238.999039	-51.27\\
294.8025	-32.959\\
191.294036	-45.166\\
259.7045	-34.18\\
189.0154	-23.193\\
127.816623	-25.635\\
138.9417	-21.973\\
119.09366	-23.193\\
124.847919	-24.414\\
128.735022	-17.09\\
90.11557	-21.973\\
117.885145	-29.297\\
158.262394	-30.518\\
164.858236	-29.297\\
163.623745	-43.945\\
254.26577	-54.932\\
316.847776	-45.166\\
258.846346	-42.725\\
251.9066	-62.256\\
379.574832	-74.463\\
449.905446	-59.814\\
361.396188	-65.918\\
380.215024	-32.959\\
180.450525	-21.973\\
123.114719	-40.283\\
236.018097	-56.152\\
333.149816	-58.594\\
359.415596	-83.008\\
513.736512	-74.463\\
449.905446	-53.711\\
314.692749	-40.283\\
233.802532	-42.725\\
249.556725	-46.387\\
268.395182	-36.621\\
203.832486	-24.414\\
135.00942	-32.959\\
180.450525	-25.635\\
138.48027	-21.973\\
115.468115	-18.311\\
97.908917	-28.076\\
155.765648	-40.283\\
227.19612	-43.945\\
243.80686	-31.738\\
182.4935	-52.49\\
314.31012	-70.801\\
416.168278	-47.607\\
283.309257	-64.697\\
390.899274	-70.801\\
421.336751	-51.27\\
294.8025	-34.18\\
195.88558	-42.725\\
248.787675	-47.607\\
267.598947	-29.297\\
163.623745	-34.18\\
188.36598	-28.076\\
149.08356	-19.531\\
102.283847	-19.531\\
104.061168	-29.297\\
160.928421	-36.621\\
200.499975	-32.959\\
182.856532	-39.063\\
219.573123	-42.725\\
237.80735	-32.959\\
175.605552	-20.752\\
107.536864	-18.311\\
95.565109	-21.973\\
115.468115	-25.635\\
136.58328	-29.297\\
166.846415	-54.932\\
319.869036	-53.711\\
309.805048	-47.607\\
274.597176	-47.607\\
278.929413	-61.035\\
371.031765	-76.904\\
477.343128	-83.008\\
516.807808	-85.449\\
524.144166	-64.697\\
395.686852	-65.918\\
405.527536	-67.139\\
411.830626	-69.58\\
430.63062	-76.904\\
488.648016	-101.318\\
645.598296	-86.67\\
547.49439	-79.346\\
494.008196	-63.477\\
391.716567	-63.477\\
390.510504	-59.814\\
370.188846	-70.801\\
426.505224	-46.387\\
278.600322	-53.711\\
327.475967	-62.256\\
384.181776	-72.021\\
439.112037	-56.152\\
343.425632	-65.918\\
400.715522	-52.49\\
315.25494	-48.828\\
284.325444	-32.959\\
193.733002	-51.27\\
313.56732	-73.242\\
446.556474	-57.373\\
337.238494	-36.621\\
215.258238	-43.945\\
247.8498	-28.076\\
154.726836	-25.635\\
142.68441	-32.959\\
180.450525	-23.193\\
124.847919	-25.635\\
138.9417	-28.076\\
149.588928	-19.531\\
105.85802	-29.297\\
157.705751	-25.635\\
134.250495	-19.531\\
103.70961	-28.076\\
151.666552	-31.738\\
174.336834	-35.4\\
195.0894	-36.621\\
201.818331	-35.4\\
200.2932	-46.387\\
262.457646	-40.283\\
229.411685	-47.607\\
285.927642	-75.684\\
451.757796	-56.152\\
330.061456	-50.049\\
295.088904	-52.49\\
303.70714	-40.283\\
223.490084	-26.855\\
152.42898	-42.725\\
241.73805	-35.4\\
193.815	-25.635\\
143.171475	-37.842\\
212.709882	-36.621\\
208.556595	-42.725\\
237.80735	-30.518\\
163.72907	-21.973\\
114.677087	-18.311\\
94.887602	-20.752\\
111.708016	-34.18\\
185.87084	-31.738\\
174.336834	-35.4\\
200.2932	-47.607\\
278.929413	-63.477\\
369.626571	-48.828\\
289.696524	-63.477\\
384.734097	-74.463\\
448.565112	-64.697\\
388.570182	-59.814\\
374.555268	-97.656\\
609.764064	-70.801\\
444.701081	-86.67\\
544.37427	-76.904\\
484.418296	-86.67\\
553.8213	-95.215\\
589.285635	-61.035\\
359.86236	-39.063\\
224.61225	-32.959\\
191.920257	-47.607\\
284.166183	-57.373\\
343.549524	-58.594\\
344.415532	-41.504\\
233.293984	-25.635\\
143.632905	-34.18\\
201.52528	-57.373\\
351.925982	-76.904\\
474.574584	-75.684\\
451.757796	-48.828\\
283.397712	-39.063\\
228.870117	-48.828\\
297.704316	-72.021\\
447.034347	-81.787\\
510.678028	-83.008\\
509.171072	-61.035\\
380.00391	-83.008\\
525.85568	-90.332\\
572.25322	-86.67\\
563.355	-114.746\\
748.029174	-98.877\\
639.140928	-85.449\\
553.880418	-98.877\\
637.262265	-84.229\\
519.777159	-51.27\\
308.85048	-47.607\\
289.402953	-56.152\\
345.447104	-67.139\\
415.523271	-67.139\\
408.137981	-51.27\\
314.49018	-63.477\\
388.225332	-56.152\\
337.248912	-43.945\\
266.350645	-54.932\\
331.899144	-52.49\\
325.80543	-75.684\\
475.371204	-76.904\\
474.574584	-57.373\\
345.614952	-45.166\\
260.517488	-30.518\\
177.706314	-43.945\\
254.26577	-35.4\\
201.603	-31.738\\
179.00232	-26.855\\
154.41625	-42.725\\
251.9066	-53.711\\
320.600959	-58.594\\
350.860872	-58.594\\
343.302246	-37.842\\
215.51019	-30.518\\
167.635374	-23.193\\
127.399149	-28.076\\
159.359376	-40.283\\
230.861873	-42.725\\
244.087925	-40.283\\
238.999039	-62.256\\
379.574832	-73.242\\
454.613094	-80.566\\
503.054104	-81.787\\
518.120645	-95.215\\
589.285635	-59.814\\
360.319536	-47.607\\
281.547798	-40.283\\
236.783474	-43.945\\
262.307705	-54.932\\
326.900332	-46.387\\
268.395182	-34.18\\
194.6551	-30.518\\
179.934128	-52.49\\
316.19976	-62.256\\
373.909536	-57.373\\
356.114211	-85.449\\
542.942946	-91.553\\
571.656932	-65.918\\
399.528998	-47.607\\
278.929413	-36.621\\
211.229928	-32.959\\
195.545747	-56.152\\
333.149816	-48.828\\
291.454332	-53.711\\
324.521862	-59.814\\
363.609306	-67.139\\
410.622124	-63.477\\
389.367918	-72.021\\
436.519281	-51.27\\
308.85048	-56.152\\
332.082928	-42.725\\
248.787675	-39.063\\
233.167047	-54.932\\
338.985372	-75.684\\
472.570896	-81.787\\
491.212722	-47.607\\
295.496649	-91.553\\
581.727762	-78.125\\
479.21875	-53.711\\
315.713258	-34.18\\
203.40518	-50.049\\
297.841599	-45.166\\
267.111724	-45.166\\
269.595854	-53.711\\
314.692749	-37.842\\
224.516586	-48.828\\
304.003128	-86.67\\
550.70118	-91.553\\
570.008978	-63.477\\
394.001739	-70.801\\
442.081444	-78.125\\
490.703125	-76.904\\
473.113408	-57.373\\
350.893268	-56.152\\
335.171288	-41.504\\
250.767168	-54.932\\
333.931628	-59.814\\
377.845038	-92.773\\
599.684672	-101.318\\
673.460746	-131.836\\
866.558028	-90.332\\
580.563764	-76.904\\
484.418296	-62.256\\
391.029936	-67.139\\
416.731773	-52.49\\
314.31012	-37.842\\
223.116432	-37.842\\
223.116432	-39.063\\
225.315384	-31.738\\
177.25673	-23.193\\
131.225994	-34.18\\
197.76548	-45.166\\
258.033358	-29.297\\
163.067102	-25.635\\
140.813055	-24.414\\
140.819952	-50.049\\
301.495176	-67.139\\
388.466254	-35.4\\
203.55	-41.504\\
240.889216	-43.945\\
253.47476	-40.283\\
236.018097	-47.607\\
277.215561	-45.166\\
257.22037	-32.959\\
183.449794	-25.635\\
139.40313	-23.193\\
126.564201	-26.855\\
153.422615	-47.607\\
278.929413	-54.932\\
317.836552	-41.504\\
233.293984	-30.518\\
165.40756	-23.193\\
126.123534	-28.076\\
158.34864	-42.725\\
248.787675	-51.27\\
300.39093	-56.152\\
324.895472	-40.283\\
230.136779	-39.063\\
224.61225	-42.725\\
251.13755	-57.373\\
348.770467	-76.904\\
481.572848	-89.111\\
553.111977	-69.58\\
416.64504	-46.387\\
276.049037	-48.828\\
288.768792	-46.387\\
273.497752	-46.387\\
267.560216	-35.4\\
204.1872	-39.063\\
226.721652	-43.945\\
254.26577	-41.504\\
234.829632	-29.297\\
159.317086	-20.752\\
113.6172	-29.297\\
165.23508	-40.283\\
236.018097	-57.373\\
342.459437	-61.035\\
372.130395	-74.463\\
454.000911	-65.918\\
405.527536	-73.242\\
463.98807	-96.436\\
635.706112	-126.953\\
836.874176	-106.201\\
680.642209	-73.242\\
466.698024	-80.566\\
519.24787	-91.553\\
608.552791	-122.07\\
795.77433	-80.566\\
494.191844	-43.945\\
256.682745	-30.518\\
174.898658	-31.738\\
177.25673	-25.635\\
136.58328	-18.311\\
96.553903	-21.973\\
118.698146	-29.297\\
163.623745	-37.842\\
212.709882	-37.842\\
209.947416	-30.518\\
168.76454	-29.297\\
163.623745	-36.621\\
207.860796	-43.945\\
251.848795	-45.166\\
258.033358	-42.725\\
239.388175	-32.959\\
179.857263	-24.414\\
131.420562	-23.193\\
126.981675	-35.4\\
197.709	-39.063\\
215.276193	-30.518\\
164.278394	-23.193\\
128.25729	-35.4\\
192.5052	-28.076\\
153.210732	-30.518\\
169.863188	-37.842\\
216.872502	-50.049\\
291.435327	-54.932\\
312.83774	-40.283\\
234.567909	-56.152\\
330.061456	-54.932\\
319.869036	-46.387\\
266.72525	-37.842\\
223.116432	-58.594\\
356.192926	-73.242\\
447.948072	-73.242\\
453.294738	-79.346\\
483.772562	-62.256\\
376.150752	-57.373\\
345.614952	-54.932\\
335.964112	-72.021\\
437.815659	-59.814\\
369.112194	-78.125\\
484.921875	-70.801\\
438.187389	-74.463\\
485.424297	-124.512\\
816.17616	-98.877\\
615.608202	-53.711\\
320.600959	-37.842\\
222.435276	-41.504\\
246.990304	-51.27\\
311.67033	-63.477\\
391.716567	-70.801\\
440.807026	-78.125\\
490.703125	-80.566\\
492.741656	-54.932\\
331.899144	-47.607\\
289.402953	-56.152\\
341.348008	-56.152\\
331.072192	-36.621\\
208.556595	-28.076\\
164.497284	-45.166\\
266.298736	-46.387\\
278.600322	-59.814\\
369.112194	-73.242\\
450.584784	-64.697\\
387.405636	-47.607\\
279.833946	-40.283\\
241.214604	-57.373\\
351.925982	-70.801\\
448.524335	-91.553\\
588.411131	-100.098\\
654.340626	-113.525\\
740.069475	-92.773\\
597.921985	-85.449\\
535.081638	-52.49\\
314.31012	-40.283\\
247.095922	-65.918\\
416.404006	-85.449\\
524.144166	-52.49\\
328.69238	-74.463\\
481.328832	-104.98\\
695.80744	-119.629\\
773.281856	-79.346\\
489.644166	-46.387\\
271.781433	-30.518\\
179.384804	-40.283\\
237.508568	-37.842\\
222.435276	-41.504\\
237.112352	-29.297\\
166.846415	-35.4\\
198.9834	-30.518\\
172.670844	-35.4\\
199.656	-31.738\\
180.144888	-34.18\\
201.52528	-53.711\\
317.646854	-50.049\\
302.396058	-67.139\\
416.731773	-80.566\\
504.504292	-79.346\\
479.408532	-46.387\\
276.049037	-47.607\\
287.641494	-57.373\\
339.303922	-40.283\\
234.567909	-35.4\\
200.9304	-28.076\\
160.903556	-41.504\\
238.648	-35.4\\
197.0364	-23.193\\
123.15483	-14.648\\
75.905936	-17.09\\
91.38023	-26.855\\
148.99154	-37.842\\
212.709882	-40.283\\
221.999613	-28.076\\
154.221468	-31.738\\
180.144888	-46.387\\
270.946467	-57.373\\
332.992892	-41.504\\
237.112352	-41.504\\
239.395072	-48.828\\
287.010984	-54.932\\
321.846588	-50.049\\
295.989786	-57.373\\
339.303922	-53.711\\
310.771846	-40.283\\
228.646308	-31.738\\
181.890478	-42.725\\
244.087925	-36.621\\
210.57075	-42.725\\
251.9066	-57.373\\
352.958696	-83.008\\
506.099776	-63.477\\
383.528034	-57.373\\
357.204298	-85.449\\
527.305779	-68.359\\
408.034871	-41.504\\
239.395072	-34.18\\
192.12578	-28.076\\
156.80446	-28.076\\
154.726836	-26.855\\
147.031125	-25.635\\
146.452755	-45.166\\
247.28385	-28.076\\
149.588928	-23.193\\
126.564201	-32.959\\
185.262539	-41.504\\
231.011264	-32.959\\
179.857263	-25.635\\
135.173355	-17.09\\
90.7479	-25.635\\
144.094335	-46.387\\
271.781433	-59.814\\
348.296922	-50.049\\
286.830819	-40.283\\
231.62725	-45.166\\
259.7045	-46.387\\
269.230148	-50.049\\
295.989786	-61.035\\
368.77347	-70.801\\
430.399279	-68.359\\
411.794616	-58.594\\
344.415532	-41.504\\
237.859424	-34.18\\
195.27034	-36.621\\
213.244083	-48.828\\
277.147728	-35.4\\
197.0364	-28.076\\
161.942368	-50.049\\
295.989786	-61.035\\
362.120655	-57.373\\
343.549524	-65.918\\
406.779978	-83.008\\
495.474752	-48.828\\
300.389856	-81.787\\
512.150194	-76.904\\
477.343128	-73.242\\
457.323048	-76.904\\
478.804304	-76.904\\
502.721448	-122.07\\
815.79381	-119.629\\
775.435178	-79.346\\
518.684802	-93.994\\
622.992232	-115.967\\
770.832649	-112.305\\
752.66811	-124.512\\
834.479424	-113.525\\
775.37575	-142.822\\
970.189846	-115.967\\
766.54187	-79.346\\
501.228682	-53.711\\
333.384177	-52.49\\
321.97366	-45.166\\
269.595854	-37.842\\
230.041518	-52.49\\
328.69238	-73.242\\
446.556474	-45.166\\
269.595854	-41.504\\
246.990304	-42.725\\
247.20685	-31.738\\
183.636068	-34.18\\
189.63064	-25.635\\
139.40313	-24.414\\
132.32388	-23.193\\
126.981675	-29.297\\
159.317086	-24.414\\
130.077792	-20.752\\
107.163328	-17.09\\
86.37286	-13.428\\
69.342192	-24.414\\
125.170578	-20.752\\
106.021968	-18.311\\
98.568113	-37.842\\
213.42888	-51.27\\
293.82837	-50.049\\
281.325429	-40.283\\
228.646308	-48.828\\
289.696524	-67.139\\
383.565107	-34.18\\
185.2556	-24.414\\
129.198888	-19.531\\
100.135437	-14.648\\
76.447912	-20.752\\
112.849376	-36.621\\
207.860796	-50.049\\
279.523665	-36.621\\
210.57075	-47.607\\
285.927642	-74.463\\
454.000911	-73.242\\
441.209808	-53.711\\
309.805048	-37.842\\
218.272656	-43.945\\
260.725685	-59.814\\
363.609306	-73.242\\
431.834832	-46.387\\
259.071395	-25.635\\
144.5814	-36.621\\
204.528285	-34.18\\
182.11104	-17.09\\
86.04815	-12.207\\
64.147785	-26.855\\
151.94559	-53.711\\
312.759153	-53.711\\
307.817741	-43.945\\
243.01585	-29.297\\
159.317086	-28.076\\
142.401472	-17.09\\
85.74053	-18.311\\
92.873392	-21.973\\
113.864086	-26.855\\
144.560465	-36.621\\
208.556595	-54.932\\
320.857812	-57.373\\
338.271208	-59.814\\
355.953114	-63.477\\
377.751627	-62.256\\
371.606064	-59.814\\
352.663344	-53.711\\
320.600959	-64.697\\
405.132614	-93.994\\
597.237876	-86.67\\
523.66014	-50.049\\
283.177242	-29.297\\
171.651123	-52.49\\
318.14189	-67.139\\
404.445336	-59.814\\
347.160456	-37.842\\
215.51019	-37.842\\
224.516586	-59.814\\
372.401964	-89.111\\
564.518185	-89.111\\
572.716397	-98.877\\
639.140928	-97.656\\
622.264032	-73.242\\
462.669714	-76.904\\
461.885424	-43.945\\
255.05678	-36.621\\
215.258238	-45.166\\
268.782866	-54.932\\
329.921592	-57.373\\
346.647666	-61.035\\
359.86236	-42.725\\
254.256475	-51.27\\
305.10777	-48.828\\
278.07546	-29.297\\
159.317086	-21.973\\
115.863629	-17.09\\
91.38023	-25.635\\
136.12185	-23.193\\
118.493037	-13.428\\
70.080732	-25.635\\
142.68441	-46.387\\
256.52011	-35.4\\
192.5052	-30.518\\
169.313864	-40.283\\
234.567909	-61.035\\
354.24714	-48.828\\
269.091108	-24.414\\
127.856118	-18.311\\
99.923127	-32.959\\
193.733002	-68.359\\
419.314106	-81.787\\
527.117215	-109.863\\
734.214429	-130.615\\
882.565555	-126.953\\
836.874176	-87.891\\
572.961429	-87.891\\
582.541548	-109.863\\
726.19443	-95.215\\
624.134325	-89.111\\
589.02371	-98.877\\
655.356756	-107.422\\
715.96763	-108.643\\
743.987264	-145.264\\
981.548848	-102.539\\
659.018153	-57.373\\
347.737753	-34.18\\
198.38072	-29.297\\
168.985096	-32.959\\
188.888029	-32.959\\
188.888029	-31.738\\
188.301554	-53.711\\
318.667363	-45.166\\
266.298736	-45.166\\
271.266996	-59.814\\
347.160456	-34.18\\
197.15024	-37.842\\
223.797588	-51.27\\
308.85048	-57.373\\
336.148407	-35.4\\
198.3462	-24.414\\
138.134412	-31.738\\
179.00232	-31.738\\
191.189712	-58.594\\
373.360968	-91.553\\
583.375716	-86.67\\
560.23488	-96.436\\
635.706112	-111.084\\
756.704208	-140.381\\
961.329088	-120.85\\
798.8185	-80.566\\
511.916364	-54.932\\
327.889108	-34.18\\
199.64538	-39.063\\
228.870117	-39.063\\
232.463913	-47.607\\
279.833946	-36.621\\
211.229928	-32.959\\
190.700774	-36.621\\
214.562439	-43.945\\
260.725685	-47.607\\
287.641494	-58.594\\
347.638202	-45.166\\
256.362216	-23.193\\
125.70606	-18.311\\
96.553903	-18.311\\
95.2172	-17.09\\
89.19271	-21.973\\
119.489174	-32.959\\
181.637049	-35.4\\
198.3462	-42.725\\
249.556725	-57.373\\
332.992892	-45.166\\
270.454008	-63.477\\
403.332858	-97.656\\
622.264032	-78.125\\
490.703125	-68.359\\
435.583548	-84.229\\
544.456256	-96.436\\
607.450364	-62.256\\
384.181776	-58.594\\
348.692894	-40.283\\
237.508568	-41.504\\
257.615328	-76.904\\
509.719712	-124.512\\
859.506336	-147.705\\
1011.48384	-115.967\\
781.385646	-98.877\\
649.918521	-75.684\\
493.383996	-80.566\\
536.97239	-103.76\\
668.7332	-63.477\\
396.350388	-51.27\\
314.49018	-41.504\\
262.180768	-72.021\\
462.878967	-72.021\\
443.073192	-40.283\\
241.939698	-41.504\\
243.960512	-34.18\\
205.28508	-50.049\\
313.406838	-76.904\\
487.18684	-75.684\\
472.570896	-58.594\\
357.247618	-46.387\\
282.821539	-52.49\\
317.14458	-41.504\\
243.171936	-30.518\\
174.898658	-26.855\\
150.468565	-24.414\\
133.66665	-20.752\\
112.849376	-23.193\\
129.950379	-37.842\\
221.716278	-51.27\\
303.21078	-51.27\\
296.64822	-31.738\\
179.573604	-29.297\\
164.678437	-26.855\\
150.468565	-32.959\\
189.51425	-45.166\\
262.143464	-47.607\\
277.215561	-45.166\\
253.878086	-28.076\\
163.991916	-47.607\\
292.021338	-69.58\\
421.72438	-53.711\\
304.863636	-26.855\\
149.47493	-26.855\\
154.89964	-41.504\\
240.889216	-42.725\\
245.66875	-37.842\\
210.628572	-26.855\\
153.422615	-41.504\\
246.990304	-62.256\\
362.516688	-41.504\\
226.487328	-17.09\\
91.68785	-25.635\\
146.914185	-51.27\\
305.10777	-54.932\\
316.847776	-39.063\\
221.018454	-32.959\\
193.733002	-58.594\\
359.415596	-76.904\\
473.113408	-64.697\\
406.361857	-87.891\\
571.2915	-108.643\\
692.273196	-72.021\\
447.034347	-59.814\\
380.058156	-80.566\\
500.073162	-54.932\\
343.984184	-73.242\\
469.407978	-87.891\\
558.459414	-69.58\\
419.14992	-36.621\\
223.278237	-58.594\\
377.63833	-100.098\\
663.449544	-114.746\\
750.094602	-83.008\\
533.492416	-72.021\\
468.1365	-92.773\\
597.921985	-74.463\\
462.191841	-48.828\\
298.632048	-45.166\\
277.861232	-57.373\\
355.081497	-58.594\\
363.692958	-57.373\\
357.204298	-64.697\\
395.686852	-48.828\\
297.704316	-50.049\\
310.654143	-68.359\\
415.554361	-43.945\\
259.09972	-35.4\\
205.4616	-30.518\\
172.670844	-23.193\\
130.367853	-26.855\\
156.376665	-46.387\\
271.781433	-41.504\\
248.525952	-57.373\\
352.958696	-69.58\\
438.28442	-83.008\\
516.807808	-63.477\\
390.510504	-57.373\\
337.238494	-25.635\\
150.195465	-47.607\\
292.878264	-72.021\\
437.815659	-51.27\\
295.72536	-26.855\\
158.82047	-51.27\\
316.38717	-74.463\\
464.946972	-74.463\\
477.233367	-92.773\\
616.662131	-120.85\\
794.34705	-87.891\\
566.457495	-74.463\\
484.0095	-86.67\\
550.70118	-61.035\\
376.646985	-47.607\\
294.639723	-53.711\\
338.325589	-68.359\\
433.054265	-72.021\\
457.621434	-72.021\\
441.776814	-45.166\\
270.454008	-39.063\\
231.018582	-32.959\\
196.732271	-46.387\\
276.884003	-43.945\\
258.30871	-34.18\\
208.39546	-63.477\\
395.207802	-68.359\\
416.784823	-50.049\\
299.693412	-40.283\\
245.605451	-58.594\\
346.524916	-32.959\\
186.482022	-21.973\\
123.510233	-28.076\\
160.903556	-35.4\\
202.8774	-34.18\\
190.8953	-25.635\\
136.58328	-18.311\\
98.238515	-28.076\\
156.271016	-37.842\\
220.353966	-53.711\\
322.588266	-59.814\\
367.975728	-75.684\\
478.095828	-86.67\\
531.63378	-48.828\\
288.768792	-39.063\\
242.464041	-76.904\\
501.337176	-109.863\\
712.131966	-83.008\\
531.998272	-78.125\\
516.40625	-113.525\\
748.3568	-92.773\\
599.684672	-74.463\\
469.042437	-57.373\\
359.269726	-58.594\\
373.360968	-75.684\\
472.570896	-52.49\\
321.02884	-45.166\\
284.500634	-75.684\\
490.583688	-95.215\\
624.134325	-96.436\\
600.410536	-48.828\\
283.397712	-25.635\\
155.373735	-56.152\\
341.348008	-47.607\\
271.978791	-21.973\\
120.697689	-20.752\\
115.505632	-31.738\\
183.064784	-41.504\\
250.767168	-67.139\\
405.653838	-50.049\\
295.088904	-39.063\\
221.018454	-28.076\\
152.17192	-19.531\\
105.506462	-25.635\\
137.993205	-23.193\\
125.70606	-26.855\\
151.94559	-42.725\\
243.318875	-39.063\\
213.166791	-24.414\\
130.541658	-20.752\\
111.708016	-26.855\\
147.514515	-30.518\\
164.278394	-20.752\\
112.849376	-31.738\\
170.845654	-24.414\\
128.29557	-19.531\\
106.932225	-36.621\\
209.874951	-47.607\\
286.784568	-69.58\\
440.7893	-92.773\\
591.149556	-81.787\\
495.711007	-42.725\\
247.9759	-32.959\\
190.107512	-34.18\\
188.36598	-21.973\\
117.885145	-21.973\\
122.719205	-35.4\\
201.603	-39.063\\
221.721588	-37.842\\
216.872502	-42.725\\
247.20685	-48.828\\
285.204348	-52.49\\
309.48104	-52.49\\
315.25494	-62.256\\
376.150752	-62.256\\
388.726464	-86.67\\
525.30687	-47.607\\
270.217332	-25.635\\
141.76155	-24.414\\
139.477182	-42.725\\
244.087925	-31.738\\
179.573604	-34.18\\
183.99094	-18.311\\
98.238515	-28.076\\
148.578192	-18.311\\
92.873392	-13.428\\
68.34852	-18.311\\
97.561008	-31.738\\
174.336834	-36.621\\
209.215773	-51.27\\
307.92762	-69.58\\
414.07058	-52.49\\
291.21452	-25.635\\
144.094335	-39.063\\
225.315384	-45.166\\
249.76798	-25.635\\
141.76155	-32.959\\
191.920257	-58.594\\
345.470224	-50.049\\
307.000566	-79.346\\
505.592712	-92.773\\
577.604698	-64.697\\
393.293063	-51.27\\
307.92762	-46.387\\
};
\addplot [color=mycolor2, line width=2.0pt, forget plot]
  table[row sep=crcr]{%
336.952888	-52.7020281256511\\
396.350388	-61.992254822249\\
467.045964	-73.0495876643142\\
381.878304	-59.7287093677735\\
374.555268	-58.5833301085901\\
506.035046	-79.1477805254993\\
487.18684	-76.1997758693408\\
564.033008	-88.2191086945413\\
406.804409	-63.6273442617203\\
238.233662	-37.2615804830284\\
290.575428	-45.4482360129797\\
276.884003	-43.3067916415925\\
326.900332	-51.1297309057303\\
252.25211	-39.4541737715417\\
115.863629	-18.1219643806644\\
87.92805	-13.7526245631516\\
100.582323	-15.731850369804\\
237.508568	-37.1481701941048\\
373.909536	-58.4823326480026\\
406.929479	-63.6469061734664\\
415.32302	-64.9597206635913\\
300.82019	-47.0505957323735\\
204.66984	-32.011940091021\\
382.618058	-59.8444125936621\\
301.8175	-47.2065827012995\\
231.760779	-36.2491717040301\\
360.96099	-56.4570802765841\\
315.713258	-49.3799863284061\\
272.892972	-42.6825636396876\\
445.238118	-69.6386725061931\\
400.752691	-62.6808089340738\\
319.08671	-49.9076202158608\\
464.245656	-72.6115979149089\\
451.320243	-70.5899637229454\\
349.747586	-54.70321748437\\
294.188022	-46.01327355198\\
243.960512	-38.1573039521544\\
284.166183	-44.4457807075537\\
349.747586	-54.70321748437\\
327.889108	-51.2843830912963\\
358.166232	-56.0199585899453\\
359.242884	-56.1883552590534\\
407.966502	-63.8091045616071\\
530.381943	-82.9558228250695\\
464.653968	-72.67546105802\\
327.889108	-51.2843830912963\\
287.78175	-45.0112832466629\\
200.2932	-31.3273998701464\\
195.88558	-30.6380141385507\\
256.362216	-40.0970260209975\\
193.39044	-30.2477550158645\\
212.548284	-33.2441894411864\\
301.36506	-47.1358175989533\\
321.846588	-50.3392864017264\\
311.67033	-48.7476412358008\\
472.03125	-73.8293248096044\\
361.333824	-56.515394387114\\
207.201618	-32.4079296792269\\
173.80001	-27.1836608067845\\
233.077438	-36.4551073173491\\
259.906361	-40.6513576089534\\
160.928421	-25.1704451031701\\
164.151091	-25.6744955239509\\
203.832486	-31.8809714730606\\
174.336834	-27.2676242169646\\
148.99154	-23.3034249332924\\
198.3462	-31.0228740672376\\
235.293003	-36.8016387557269\\
345.470224	-54.0342051076974\\
335.171288	-52.4233721572575\\
395.903508	-61.9223593467462\\
367.67484	-57.5071781511909\\
423.956388	-66.31004594454\\
341.426723	-53.4017703934767\\
346.524916	-54.1991670635893\\
301.36506	-47.1358175989533\\
295.088904	-46.1541784386652\\
298.632048	-46.7083534625078\\
514.81674	-80.5213051357969\\
724.156596	-113.263671714747\\
748.029174	-116.997526867779\\
711.452175	-111.276602374471\\
479.914035	-75.0622531256201\\
661.64778	-103.486811221023\\
838.00378	-131.070248559383\\
840.18057	-131.410715288974\\
642.74594	-100.53041779398\\
854.899392	-133.712852467927\\
1077.987963	-168.605624015721\\
772.920055	-120.890652456701\\
631.248384	-98.7321114393886\\
426.833596	-66.76006344651\\
329.463274	-51.5305947836752\\
292.878264	-45.8084173082377\\
339.270384	-53.064502357874\\
261.330476	-40.8741001686912\\
172.670844	-27.0070505434217\\
173.80001	-27.1836608067845\\
215.258238	-33.668047128755\\
295.088904	-46.1541784386652\\
287.889888	-45.028196863137\\
285.927642	-44.7212864683478\\
377.333616	-59.0178851447792\\
402.803522	-63.0015747054684\\
533.543556	-83.4503234605661\\
452.363901	-70.7531998762131\\
528.843861	-82.7152549483414\\
387.019269	-60.5327959082723\\
336.952888	-52.7020281256511\\
390.899274	-61.1396585882493\\
291.454332	-45.58570337111\\
263.93367	-41.2812597696041\\
323.555064	-50.6065052130517\\
324.867848	-50.8118347465113\\
236.783474	-37.0347599051811\\
255.05678	-39.892845771359\\
190.107512	-29.7342797795565\\
211.229928	-33.0379884039909\\
228.646308	-35.7620444405934\\
173.220168	-27.0929690499225\\
197.15024	-30.8358167075834\\
262.307705	-41.02694631448\\
384.734097	-60.175377398727\\
403.154488	-63.056468492291\\
427.300593	-66.8331053758275\\
270.454008	-42.3010908762743\\
351.623824	-54.9966755652113\\
554.805086	-86.775790586565\\
442.081444	-69.1449443683547\\
513.704147	-80.34728701734\\
488.310526	-76.3755290184766\\
293.237091	-45.8645406160345\\
215.917416	-33.7711476473527\\
284.166183	-44.4457807075537\\
326.80274	-51.1144667649201\\
524.172883	-81.9846779961999\\
670.70464	-104.903373914055\\
651.532806	-101.904751344931\\
487.18684	-76.1997758693408\\
506.035046	-79.1477805254993\\
450.995502	-70.5391717282035\\
435.70996	-68.148395174302\\
416.784823	-65.1883579169398\\
306.03063	-47.8655486982225\\
267.111724	-41.7783319041894\\
228.166983	-35.6870743170807\\
221.716278	-34.6781343481783\\
249.273024	-38.988221765335\\
350.613088	-54.8385886664292\\
522.809403	-81.7714192176945\\
429.124861	-67.1184349482059\\
297.841599	-46.5847211479811\\
251.9066	-39.4001333451611\\
229.411685	-35.8817553011239\\
153.7161	-24.0423825231182\\
126.564201	-19.7956162963724\\
138.134412	-21.6052864528179\\
234.08256	-36.6123161432717\\
185.262539	-28.9764886686697\\
193.39044	-30.2477550158645\\
213.42888	-33.3819214411547\\
195.762	-30.6186852742859\\
144.077075	-22.5347647381243\\
114.677087	-17.936380068782\\
97.23141	-15.207741755627\\
145.50426	-22.7579874695184\\
307.00476	-48.0179101365315\\
449.266428	-70.2687312309572\\
469.21875	-73.3894281374518\\
318.825328	-49.8667380569415\\
220.31532	-34.4590137216803\\
172.591244	-26.9946004900516\\
137.69496	-21.5365527737528\\
237.112352	-37.0861989586013\\
249.556725	-39.0325947878369\\
330.910368	-51.7569314360812\\
466.698024	-72.9951671672088\\
718.13023	-112.321101635254\\
826.436436	-129.26102682383\\
680.1468	-106.380200496075\\
643.774572	-100.691303765062\\
425.325565	-66.5241957777446\\
480.896136	-75.2158612897508\\
509.956742	-79.7611639991327\\
547.49439	-85.6323413984693\\
434.45752	-67.9525039074324\\
387.605856	-60.6245426351089\\
376.646985	-58.9104907654381\\
348.535464	-54.513632263377\\
410.622124	-64.224464305712\\
330.63451	-51.7137851192149\\
494.262008	-77.3063866633391\\
616.892952	-96.486811256593\\
437.815659	-68.4777427236904\\
271.266996	-42.4282484640787\\
296.401182	-46.3594288298257\\
499.21875	-78.0816593070791\\
614.438778	-96.1029595319767\\
744.35754	-116.423255017836\\
799.480607	-125.044927455932\\
784.168095	-122.649932586208\\
595.0945	-93.0773653924589\\
550.604973	-86.1188605487464\\
637.44196	-99.7008344513437\\
715.93638	-111.977966590204\\
458.096376	-71.6498031386834\\
287.089143	-44.9029541748729\\
384.181776	-60.0889901383323\\
326.900332	-51.1297309057303\\
204.02042	-31.910365798815\\
286.784568	-44.8553162979232\\
320.600959	-50.14446042774\\
250.325775	-39.1528801338703\\
201.52528	-31.5201066761289\\
258.30871	-40.4014726816072\\
238.233662	-37.2615804830284\\
321.02884	-50.2113843132431\\
404.341012	-63.2420499888347\\
373.909536	-58.4823326480026\\
269.595854	-42.1668689780363\\
275.377102	-43.0711007135344\\
431.502353	-67.4902930171372\\
670.8222	-104.921761203932\\
488.136592	-76.3483244006013\\
305.10777	-47.7212062829825\\
239.724133	-37.4947016324826\\
282.452331	-44.1777210484028\\
233.802532	-36.5685176062727\\
183.064784	-28.6327428515296\\
207.4086	-32.4403032589586\\
253.487425	-39.647386556452\\
307.92762	-48.1622525517716\\
337.238494	-52.7466991048322\\
258.14445	-40.3757811518764\\
394.001739	-61.6249080207719\\
469.770588	-73.4757398529246\\
434.293334	-67.9268239518718\\
351.925982	-55.0439354047364\\
391.029936	-61.1600951318855\\
503.153624	-78.6971039214577\\
303.70714	-47.502136958212\\
160.903556	-25.166556025569\\
211.229928	-33.0379884039909\\
251.9066	-39.4001333451611\\
302.28792	-47.2801600141933\\
268.395182	-41.979074625273\\
215.917416	-33.7711476473527\\
311.42317	-48.7089835072713\\
305.64927	-47.8059010555942\\
222.435276	-34.7905911757283\\
278.929413	-43.6267095268148\\
240.889216	-37.6769295914094\\
220.353966	-34.4650582538731\\
256.362216	-40.0970260209975\\
165.762426	-25.9265207343412\\
176.082424	-27.5406479438776\\
161.455767	-25.252926081116\\
154.89964	-24.2274972990682\\
279.833946	-43.7681854580333\\
323.555064	-50.6065052130517\\
434.45752	-67.9525039074324\\
548.210872	-85.7444047042319\\
365.940768	-57.2359559282316\\
208.556595	-32.6198585230112\\
178.399298	-27.9030248905076\\
168.45775	-26.3480901771761\\
226.430743	-35.4155130022155\\
177.25673	-27.7243184510736\\
173.220168	-27.0929690499225\\
230.315448	-36.0231107984091\\
364.317915	-56.9821292138875\\
329.463274	-51.5305947836752\\
479.408532	-74.9831885611887\\
326.900332	-51.1297309057303\\
278.954364	-43.6306120554786\\
183.449794	-28.6929613822835\\
158.78974	-24.8359389148338\\
123.572304	-19.3276605510486\\
124.847919	-19.5271765664968\\
143.632905	-22.4652931275038\\
229.411685	-35.8817553011239\\
267.560216	-41.8484795089137\\
284.325444	-44.470690355164\\
312.59319	-48.8919836510409\\
488.136592	-76.3483244006013\\
441.209808	-69.0086137814291\\
325.542371	-50.9173353413524\\
388.225332	-60.7214334549768\\
434.45752	-67.9525039074324\\
532.005474	-83.209755583838\\
432.558126	-67.6554240497589\\
271.121865	-42.4055488573494\\
153.210732	-23.9633390737271\\
118.280659	-18.5000064974619\\
118.698146	-18.5653046812724\\
145.5541	-22.7657828295682\\
144.560465	-22.6103706589605\\
139.890195	-21.8799044434747\\
187.075284	-29.2600159550559\\
272.835717	-42.6736085165003\\
251.057785	-39.2673721384862\\
261.330476	-40.8741001686912\\
296.0436	-46.3035002496224\\
191.2308	-29.9099706784254\\
114.75856	-17.9491230737848\\
205.4616	-32.1357774560498\\
327.983832	-51.2991986548067\\
324.867848	-50.8118347465113\\
351.586692	-54.990867834285\\
308.83825	-48.3046821007715\\
234.567909	-36.6882284668033\\
308.85048	-48.3065949670117\\
439.112037	-68.6805062323311\\
423.956388	-66.31004594454\\
315.41304	-49.33302991983\\
501.681458	-78.4668458150555\\
377.333616	-59.0178851447792\\
389.367918	-60.9001427097494\\
450.584784	-70.4749322681527\\
429.124861	-67.1184349482059\\
328.932816	-51.4476270649493\\
293.782797	-45.9498932394562\\
500.703125	-78.3138269951596\\
694.22877	-108.582729114867\\
522.606084	-81.7396185593886\\
300.594294	-47.015263857297\\
307.92762	-48.1622525517716\\
311.67033	-48.7476412358008\\
375.48732	-58.7291102234605\\
437.815659	-68.4777427236904\\
331.899144	-51.9115836216472\\
346.513992	-54.1974584658127\\
457.323048	-71.5288486804852\\
485.280136	-75.901552671339\\
335.171288	-52.4233721572575\\
272.079984	-42.5554060518831\\
359.269726	-56.1925535546885\\
553.625409	-86.5912800135819\\
492.109375	-76.9696982746935\\
491.210902	-76.8291701741712\\
328.932816	-51.4476270649493\\
269.595854	-42.1668689780363\\
274.597176	-42.9491142772943\\
174.908118	-27.3569774366816\\
110.566656	-17.2934769868219\\
107.9104	-16.8780180802316\\
92.93542	-14.5358158162134\\
180.144888	-28.1760486174206\\
284.325444	-44.470690355164\\
316.847776	-49.5574336858095\\
232.352344	-36.3416970284254\\
263.814606	-41.2626372425381\\
297.93324	-46.5990545065349\\
198.9834	-31.1225370572805\\
195.0894	-30.5134854514629\\
162.539756	-25.4224703135605\\
183.064784	-28.6327428515296\\
304.18491	-47.5768638677424\\
373.172296	-58.367022631086\\
233.293984	-36.4889767803777\\
182.856532	-28.6001706340117\\
156.80446	-24.5254258249525\\
179.857263	-28.1310618510818\\
152.939225	-23.920873286788\\
274.564114	-42.94394312573\\
488.310526	-76.3755290184766\\
392.859153	-61.4461987659992\\
518.120645	-81.0380613365465\\
631.82403	-98.8221469095176\\
615.135144	-96.2118765403324\\
471.729136	-73.7820718520183\\
357.247618	-55.8762802818234\\
332.942852	-52.0748275953002\\
350.893268	-54.8824109773994\\
409.153026	-63.9946860581508\\
478.804304	-74.8886826460167\\
491.41656	-76.8613366701006\\
652.538862	-102.062106255651\\
720.085804	-112.626968477831\\
851.479185	-133.177905738195\\
596.834007	-93.3494376912984\\
651.797184	-101.946102101336\\
704.223926	-110.146048532483\\
557.042031	-87.1256660216897\\
625.098152	-97.7701677630272\\
530.381943	-82.9558228250695\\
304.18491	-47.5768638677424\\
199.03014	-31.1298475534428\\
218.590749	-34.1892775283325\\
303.21078	-47.4245024294334\\
252.68375	-39.5216855936103\\
183.064784	-28.6327428515296\\
237.112352	-37.0861989586013\\
183.449794	-28.6929613822835\\
129.092238	-20.1910207633515\\
178.399298	-27.9030248905076\\
189.0154	-29.563465047319\\
146.452755	-22.906339396293\\
251.9066	-39.4001333451611\\
341.192862	-53.3651927310247\\
272.079984	-42.5554060518831\\
433.054265	-67.7330240124345\\
636.022692	-99.4788499997552\\
571.2915	-89.354392774771\\
702.02457	-109.802052306606\\
521.146764	-81.5113696663732\\
509.205862	-79.6437205811111\\
370.485456	-57.946780151226\\
225.878898	-35.3292001918888\\
272.662786	-42.6465607755538\\
244.856975	-38.2975176731883\\
169.512442	-26.5130521330679\\
228.646308	-35.7620444405934\\
119.09366	-18.6271661185665\\
104.783815	-16.3889960938487\\
137.69496	-21.5365527737528\\
264.627594	-41.3897948303425\\
343.549524	-53.7337928274473\\
398.276556	-62.2935223398336\\
397.090032	-62.1079408432899\\
379.574832	-59.3684285972147\\
419.14992	-65.5582773123595\\
341.426723	-53.4017703934767\\
289.696524	-45.3107686548493\\
292.336209	-45.7236357293493\\
263.14266	-41.1575397103545\\
353.739996	-55.3276611725011\\
267.560216	-41.8484795089137\\
215.258238	-33.668047128755\\
310.74747	-48.6032988205607\\
417.89748	-65.3623860454899\\
323.879072	-50.6571825609454\\
240.889216	-37.6769295914094\\
220.353966	-34.4650582538731\\
237.112352	-37.0861989586013\\
184.669277	-28.8836979203979\\
173.220168	-27.0929690499225\\
240.889216	-37.6769295914094\\
279.833946	-43.7681854580333\\
305.64927	-47.8059010555942\\
251.13755	-39.2798479991277\\
315.713258	-49.3799863284061\\
291.98265	-45.6683363773459\\
185.88876	-29.0744344585122\\
194.326264	-30.3941251006007\\
359.415596	-56.2153687467123\\
500.073162	-78.2152959276826\\
479.21875	-74.9535051939942\\
410.622124	-64.224464305712\\
373.909536	-58.4823326480026\\
297.841599	-46.5847211479811\\
286.083252	-44.7456250714246\\
246.990304	-38.631186767484\\
325.911556	-50.9750787201643\\
288.682632	-45.1521881333481\\
192.513519	-30.1105978142145\\
267.111724	-41.7783319041894\\
315.25494	-49.308301861566\\
371.031765	-58.0322270884942\\
416.784823	-65.1883579169398\\
428.064058	-66.9525171848248\\
566.457495	-88.5983171435998\\
708.125824	-110.75633544636\\
768.855583	-120.254937716485\\
491.210902	-76.8291701741712\\
270.111501	-42.2475201422338\\
184.669277	-28.8836979203979\\
136.791642	-21.3952668778966\\
192.12578	-30.0499524468318\\
170.596431	-26.6825963655124\\
299.46807	-46.8391137454042\\
282.518808	-44.1881185634516\\
237.508568	-37.1481701941048\\
313.31281	-49.0045377641838\\
385.876683	-60.3540866534996\\
435.567752	-68.1261527472963\\
482.258448	-75.428937384056\\
664.99784	-104.010786419427\\
586.047041	-91.6622730882681\\
443.073192	-69.3000613976219\\
310.74747	-48.6032988205607\\
318.14189	-49.7598430874045\\
337.941664	-52.8566803112171\\
399.528998	-62.4894139195186\\
307.92762	-48.1622525517716\\
299.693412	-46.8743589706118\\
280.761	-43.9131838471909\\
177.706314	-27.7946368530125\\
273.751126	-42.8167855379256\\
408.034871	-63.819798000035\\
291.454332	-45.58570337111\\
325.80543	-50.9584797959941\\
536.40063	-83.8971918497905\\
389.367918	-60.9001427097494\\
399.841623	-62.5383108784988\\
538.243251	-84.1853919727908\\
501.681458	-78.4668458150555\\
317.646854	-49.6824156424283\\
206.1342	-32.2409772788728\\
219.945726	-34.4012063721168\\
383.360835	-59.9605886400448\\
580.563764	-90.8046463132314\\
629.39292	-98.4419025722245\\
636.022692	-99.4788499997552\\
488.136592	-76.3483244006013\\
323.555064	-50.6065052130517\\
271.781433	-42.5087103749524\\
218.272656	-34.1395253320178\\
195.27034	-30.5417858617239\\
169.512442	-26.5130521330679\\
235.293003	-36.8016387557269\\
243.318875	-38.0569469811215\\
176.577148	-27.6180265896497\\
253.487425	-39.647386556452\\
338.259648	-52.9064154590918\\
382.998912	-59.9039810939913\\
496.40625	-77.6417626349266\\
628.666284	-98.3282511026187\\
737.472542	-115.346388277222\\
533.590715	-83.4576994915571\\
462.669714	-72.3651084424448\\
481.572848	-75.3217042610595\\
388.225332	-60.7214334549768\\
282.821539	-44.235468024592\\
363.457955	-56.8476248433331\\
583.906048	-91.3274052853163\\
636.378358	-99.5344789027944\\
364.317915	-56.9821292138875\\
208.556595	-32.6198585230112\\
143.593685	-22.4591588172881\\
95.894707	-14.9986711062559\\
119.09366	-18.6271661185665\\
168.76454	-26.3960744971937\\
179.573604	-28.0866953977036\\
240.969	-37.6894084237973\\
209.26626	-32.7308555974443\\
167.635374	-26.2194642338309\\
160.928421	-25.1704451031701\\
160.401075	-25.0879641252242\\
147.514515	-23.0724068418484\\
179.00232	-27.9973421779866\\
259.09972	-40.5251927408568\\
381.242862	-59.6293213424772\\
411.794616	-64.40785108933\\
403.154488	-63.056468492291\\
466.287306	-72.9309277071579\\
581.727762	-90.9867045697975\\
599.684672	-93.7953036635372\\
690.3065	-107.969255863211\\
637.824456	-99.7606597731256\\
459.511173	-71.8710882914199\\
334.920404	-52.3841319664321\\
322.02687	-50.3674838957172\\
541.339783	-84.6697134383958\\
615.135144	-96.2118765403324\\
425.55128	-66.5594993430263\\
265.008931	-41.4494388755936\\
146.03749	-22.8413887504045\\
117.885145	-18.4381450601678\\
107.536864	-16.8195941714924\\
78.044544	-12.2067680658716\\
141.274485	-22.0964180663348\\
224.215178	-35.0689815638377\\
223.490084	-34.955571274914\\
190.5582	-29.8047708556024\\
128.29557	-20.0664157493033\\
102.986963	-16.1079545951284\\
149.08356	-23.3178175703667\\
150.62774	-23.5593392212838\\
162.05058	-25.3459594177394\\
127.416666	-19.928948391173\\
117.885145	-18.4381450601678\\
193.106781	-30.2033885624864\\
426.80372	-66.7553906098959\\
494.008196	-77.2666885107516\\
531.63378	-83.1516197780926\\
394.001739	-61.6249080207719\\
522.228812	-81.680610311461\\
750.094602	-117.320575722453\\
680.1468	-106.380200496075\\
684.465445	-107.05566985206\\
524.172883	-81.9846779961999\\
527.117215	-82.4451942090043\\
544.37427	-85.1443305879034\\
369.112194	-57.7319913925438\\
334.160552	-52.2652852584754\\
233.077438	-36.4551073173491\\
213.8868	-33.4535436577279\\
371.606064	-58.1220518774438\\
281.547798	-44.0362451171843\\
291.454332	-45.58570337111\\
318.667363	-49.8420311137178\\
297.841599	-46.5847211479811\\
298.742481	-46.7256260346663\\
316.19976	-49.4560789900223\\
347.737753	-54.3888641160918\\
390.899274	-61.1396585882493\\
332.082928	-51.9403288554232\\
247.737376	-38.7480345849625\\
300.82019	-47.0505957323735\\
167.957496	-26.2698465967917\\
96.224305	-15.0502227732241\\
139.40313	-21.8037237243202\\
180.88056	-28.2911133870546\\
109.42554	-17.1149976513766\\
49.17181	-7.69084998496636\\
116.281116	-18.1872625644749\\
179.231042	-28.0331160612393\\
211.34757	-33.0563885192995\\
246.223835	-38.5113051097389\\
224.61225	-35.1310866843372\\
341.192862	-53.3651927310247\\
277.147728	-43.3480402637662\\
211.229928	-33.0379884039909\\
307.783996	-48.1397886514547\\
211.72146	-33.1148677963666\\
141.606415	-22.1483344760726\\
118.05237	-18.4643003387458\\
88.25276	-13.8034117092545\\
131.430645	-20.5567656371073\\
101.932289	-15.9429954545752\\
162.01241	-25.3399893356145\\
246.223835	-38.5113051097389\\
236.26925	-36.9543313091487\\
182.26327	-28.5073798857398\\
220.549425	-34.4956295476125\\
162.01241	-25.3399893356145\\
240.598875	-37.631518021742\\
172.01996	-26.9052472703346\\
170.845654	-26.7215767631386\\
161.501256	-25.2600409112385\\
130.077792	-20.3451690032898\\
142.60005	-22.3037466466803\\
146.03749	-22.8413887504045\\
228.728544	-35.7749067846756\\
198.48582	-31.0447117111011\\
164.858236	-25.7850984509657\\
175.01229	-27.373270740195\\
143.08344	-22.3793525675165\\
180.144888	-28.1760486174206\\
349.747586	-54.70321748437\\
444.870552	-69.5811823514566\\
297.57108	-46.5424098918552\\
277.147728	-43.3480402637662\\
228.870117	-35.7970498927882\\
346.524916	-54.1991670635893\\
335.802214	-52.5220538453551\\
217.5915	-34.0329872848652\\
262.457646	-41.0503982422735\\
223.870053	-35.0150013544238\\
287.26581	-44.9305862550077\\
181.043787	-28.3166433476255\\
197.709	-30.9232110771947\\
214.791192	-33.59499753546\\
268.50348	-41.9960132669799\\
205.187463	-32.0929003168449\\
233.293984	-36.4889767803777\\
240.889216	-37.6769295914094\\
334.082979	-52.2531522435246\\
307.00476	-48.0179101365315\\
450.995502	-70.5391717282035\\
585.206644	-91.5308285216594\\
470.625	-73.6093764735281\\
291.435327	-45.5827308426641\\
238.233662	-37.2615804830284\\
333.931628	-52.2294797808662\\
416.64504	-65.1664947786203\\
339.270384	-53.064502357874\\
374.260392	-58.5372092299776\\
244.59787	-38.2569916546148\\
126.123534	-19.7266925819449\\
139.477182	-21.8153060277393\\
305.64927	-47.8059010555942\\
259.7045	-40.6197849930823\\
253.487425	-39.647386556452\\
386.176393	-60.4009636069613\\
360.96099	-56.4570802765841\\
326.973096	-51.1411117560245\\
227.463849	-35.5770987413732\\
319.634161	-49.9932457707289\\
391.091494	-61.1697232774301\\
310.42586	-48.5529965383452\\
316.680056	-49.5312009854172\\
281.639904	-44.0506512053213\\
218.953812	-34.2460633791705\\
242.5071	-37.9299791158641\\
200.2932	-31.3273998701464\\
215.258238	-33.668047128755\\
343.549524	-53.7337928274473\\
404.445336	-63.2583670663194\\
426.505224	-66.708703535389\\
365.940768	-57.2359559282316\\
258.033358	-40.3584055070399\\
201.603	-31.5322626830123\\
225.315384	-35.2410622600447\\
272.662786	-42.6465607755538\\
344.582238	-53.8953172547843\\
419.314106	-65.58395726792\\
482.344334	-75.4423706162639\\
386.213562	-60.4067771249727\\
241.939698	-37.8412330708604\\
425.603134	-66.5676097081953\\
563.234056	-88.0941464452934\\
401.902046	-62.860576912606\\
399.245187	-62.4450236921692\\
461.445348	-72.1736081655037\\
358.166232	-56.0199585899453\\
300.594294	-47.015263857297\\
335.171288	-52.4233721572575\\
307.92762	-48.1622525517716\\
350.893268	-54.8824109773994\\
427.779642	-66.9080323308134\\
340.337272	-53.231371862144\\
384.734097	-60.175377398727\\
354.024948	-55.3722298610427\\
357.029766	-55.8422065503312\\
294.139872	-46.0057425209527\\
377.751627	-59.0832652862274\\
266.298736	-41.651174316385\\
290.575428	-45.4482360129797\\
311.42317	-48.7089835072713\\
285.029055	-44.580740537347\\
164.278394	-25.6944066941038\\
101.209642	-15.8299678953073\\
82.820311	-12.9537348250809\\
158.854008	-24.8459909252608\\
319.634161	-49.9932457707289\\
404.445336	-63.2583670663194\\
387.466004	-60.6026687046577\\
272.662786	-42.6465607755538\\
322.890296	-50.5025303753794\\
289.583514	-45.2930930200333\\
246.223835	-38.5113051097389\\
171.541678	-26.8304402800589\\
242.5071	-37.9299791158641\\
213.42888	-33.3819214411547\\
208.556595	-32.6198585230112\\
234.829632	-36.7291639607502\\
212.709882	-33.2694646136048\\
187.1355	-29.2694342014596\\
141.274485	-22.0964180663348\\
205.4616	-32.1357774560498\\
336.9155	-52.6961803543521\\
246.4378	-38.5447708844792\\
302.76232	-47.3543598297557\\
265.485748	-41.5240167285805\\
353.739996	-55.3276611725011\\
346.647666	-54.2183661094583\\
461.885424	-72.2424394429772\\
351.915564	-55.0423059492589\\
389.734728	-60.9575146202604\\
382.588072	-59.8397225522003\\
223.116432	-34.8971292228809\\
356.505435	-55.7601971416178\\
271.781433	-42.5087103749524\\
303.21078	-47.4245024294334\\
345.614952	-54.0568416821213\\
415.32302	-64.9597206635913\\
323.555064	-50.6065052130517\\
435.567752	-68.1261527472963\\
501.681458	-78.4668458150555\\
421.72438	-65.9609426942581\\
349.747586	-54.70321748437\\
299.693412	-46.8743589706118\\
348.692894	-54.5382555284781\\
308.53622	-48.2574422814328\\
255.05678	-39.892845771359\\
220.353966	-34.4650582538731\\
262.143464	-41.0012577564956\\
236.018646	-36.915134912481\\
452.363901	-70.7531998762131\\
565.658984	-88.4734238701501\\
492.500622	-77.0308923203076\\
469.770588	-73.4757398529246\\
448.565112	-70.1590400044585\\
272.662786	-42.6465607755538\\
246.4378	-38.5447708844792\\
200.2932	-31.3273998701464\\
172.670844	-27.0070505434217\\
191.920257	-30.0178070659427\\
313.31281	-49.0045377641838\\
362.532654	-56.7029006368835\\
426.505224	-66.708703535389\\
351.586692	-54.990867834285\\
258.956225	-40.5027490171339\\
447.034347	-69.9196165629127\\
530.558471	-82.9834331645332\\
565.658984	-88.4734238701501\\
472.04469	-73.8314269291684\\
747.407315	-116.900263328348\\
524.144166	-81.9801864361167\\
303.21078	-47.4245024294334\\
226.018518	-35.3510378357522\\
202.78994	-31.7179092451616\\
342.358744	-53.5475456597083\\
435.70996	-68.148395174302\\
554.805086	-86.775790586565\\
416.784823	-65.1883579169398\\
311.738644	-48.7583260718048\\
180.144888	-28.1760486174206\\
208.556595	-32.6198585230112\\
196.535	-30.7395884307566\\
219.573123	-34.3429283917669\\
144.094335	-22.5374643351239\\
197.709	-30.9232110771947\\
144.560465	-22.6103706589605\\
108.357988	-16.94802429239\\
221.018454	-34.5689892973878\\
246.4378	-38.5447708844792\\
282.518808	-44.1881185634516\\
269.230148	-42.1096697416323\\
284.325444	-44.470690355164\\
287.889888	-45.028196863137\\
362.516688	-56.700403431455\\
259.89073	-40.6489128001063\\
377.333616	-59.0178851447792\\
416.64504	-65.1664947786203\\
335.171288	-52.4233721572575\\
348.692894	-54.5382555284781\\
327.889108	-51.2843830912963\\
375.48732	-58.7291102234605\\
491.210902	-76.8291701741712\\
371.606064	-58.1220518774438\\
299.693412	-46.8743589706118\\
342.459437	-53.5632948208138\\
297.57108	-46.5424098918552\\
215.917416	-33.7711476473527\\
343.425632	-53.7144151639784\\
456.756042	-71.440164572933\\
429.37818	-67.1580559917946\\
502.109375	-78.5337753312359\\
725.42475	-113.462020772303\\
495.436424	-77.4900743753827\\
441.776814	-69.0972978889812\\
345.447104	-54.0305889615427\\
433.054265	-67.7330240124345\\
577.604698	-90.3418255892919\\
395.686852	-61.88847267887\\
358.237012	-56.0310291273515\\
468.883688	-73.3370218587799\\
298.54521	-46.6947713301642\\
191.294036	-29.9198612761002\\
244.087925	-38.1772323271549\\
184.669277	-28.8836979203979\\
142.68441	-22.3169412007293\\
154.726836	-24.2004694219003\\
139.890195	-21.8799044434747\\
141.76155	-22.1725987854893\\
179.573604	-28.0866953977036\\
246.4378	-38.5447708844792\\
289.696524	-45.3107686548493\\
378.894213	-59.261974541\\
361.396188	-56.5251485972695\\
414.314769	-64.8020224379577\\
488.136592	-76.3483244006013\\
442.528164	-69.2148148186246\\
330.910368	-51.7569314360812\\
351.915564	-55.0423059492589\\
322.588266	-50.4552905560406\\
352.958696	-55.2054598320734\\
467.499416	-73.1205110512585\\
364.685958	-57.0396939750996\\
387.019269	-60.5327959082723\\
376.150752	-58.8328761004381\\
357.204298	-55.8695047000145\\
524.409754	-82.021726445846\\
430.63062	-67.3539472586642\\
450.584784	-70.4749322681527\\
393.464542	-61.5408862705175\\
244.856975	-38.2975176731883\\
162.539756	-25.4224703135605\\
138.573864	-21.6740201318831\\
236.36528	-36.9693511411227\\
197.15024	-30.8358167075834\\
273.74025	-42.8150844477189\\
205.4616	-32.1357774560498\\
274.564114	-42.94394312573\\
501.228682	-78.3960281597202\\
588.590428	-92.0600784135289\\
472.570896	-73.9137296023299\\
495.436424	-77.4900743753827\\
430.399279	-67.3177637436304\\
317.14458	-49.6038561184785\\
334.920404	-52.3841319664321\\
361.396188	-56.5251485972695\\
326.509169	-51.0685499983635\\
385.302384	-60.2642618645501\\
494.262008	-77.3063866633391\\
682.553827	-106.756678066593\\
602.407546	-94.2211821386628\\
566.457495	-88.5983171435998\\
615.135144	-96.2118765403324\\
429.362879	-67.1556627974903\\
462.191841	-72.2903654229207\\
369.112194	-57.7319913925438\\
374.555268	-58.5833301085901\\
479.45814	-74.9909476346508\\
513.206694	-80.2694815349391\\
223.797588	-35.0036672700335\\
286.083252	-44.7456250714246\\
225.197742	-35.2226621447361\\
322.588266	-50.4552905560406\\
312.759153	-48.9179415430944\\
208.0812	-32.5455030817817\\
282.821539	-44.235468024592\\
485.424297	-75.9241005625938\\
784.808224	-122.750053694421\\
780.545536	-122.083336444421\\
688.394882	-107.670264077743\\
557.042031	-87.1256660216897\\
645.50616	-100.962137471281\\
758.076279	-118.568971509296\\
563.355	-88.113063018846\\
580.914609	-90.8595211747216\\
583.375716	-91.2444572739612\\
403.968068	-63.1837186734572\\
353.701448	-55.3216319682635\\
445.737969	-69.7168530542721\\
293.260968	-45.8682751628224\\
228.641364	-35.7612711608966\\
312.36799	-48.8567606357275\\
231.760779	-36.2491717040301\\
282.452331	-44.1777210484028\\
249.273024	-38.988221765335\\
314.49018	-49.1886875045899\\
401.902046	-62.860576912606\\
312.36799	-48.8567606357275\\
248.525952	-38.8713739478565\\
302.396058	-47.2970736306674\\
350.893268	-54.8824109773994\\
400.715522	-62.6749954160623\\
337.248912	-52.7483285603097\\
285.372824	-44.6345086579121\\
438.187389	-68.5358841601133\\
398.276556	-62.2935223398336\\
295.496649	-46.2179528986072\\
487.8125	-76.2976339144604\\
410.622124	-64.224464305712\\
319.634161	-49.9932457707289\\
217.5915	-34.0329872848652\\
163.067102	-25.5049512915064\\
132.501609	-20.7242726591856\\
240.142144	-37.5600817739308\\
208.556595	-32.6198585230112\\
175.51114	-27.4512947241606\\
135.00942	-21.1165136239101\\
181.890478	-28.4490723443336\\
271.781433	-42.5087103749524\\
307.00476	-48.0179101365315\\
388.225332	-60.7214334549768\\
449.266428	-70.2687312309572\\
444.441591	-69.5140895456314\\
444.469647	-69.5184777202213\\
267.598947	-41.8545373357614\\
129.532905	-20.259944477779\\
197.709	-30.9232110771947\\
150.951955	-23.6100489455725\\
221.035122	-34.5715963010257\\
312.36799	-48.8567606357275\\
362.638266	-56.7194191674931\\
520.982553	-81.48568580062\\
338.259648	-52.9064154590918\\
354.048783	-55.3759578387069\\
454.000911	-71.0092408544462\\
340.394009	-53.2402459661397\\
240.142144	-37.5600817739308\\
198.38072	-31.0282732612368\\
268.782866	-42.0397113902318\\
337.941664	-52.8566803112171\\
480.916106	-75.2189847516327\\
318.667363	-49.8420311137178\\
282.452331	-44.1777210484028\\
269.595854	-42.1668689780363\\
305.148753	-47.7276163399833\\
435.583548	-68.1286233634148\\
668.7332	-104.59502550682\\
572.716397	-89.5772576453346\\
564.033008	-88.2191086945413\\
358.166232	-56.0199585899453\\
263.14266	-41.1575397103545\\
277.147728	-43.3480402637662\\
124.323234	-19.4451117894556\\
221.721588	-34.6789648730953\\
173.220168	-27.0929690499225\\
215.917416	-33.7711476473527\\
359.415596	-56.2153687467123\\
483.034024	-75.5502434467765\\
561.79494	-87.8690576135631\\
722.046	-112.933558236824\\
583.420758	-91.2515021898393\\
325.542371	-50.9173353413524\\
322.588266	-50.4552905560406\\
258.14445	-40.3757811518764\\
356.114211	-55.699006693381\\
466.698024	-72.9951671672088\\
601.473155	-94.0750361861689\\
440.480436	-68.8945343803406\\
321.97366	-50.3591614416993\\
373.478616	-58.414933439482\\
465.607968	-72.8246740092142\\
328.932816	-51.4476270649493\\
235.293003	-36.8016387557269\\
199.656	-31.2277368801035\\
142.22298	-22.2447699931093\\
131.420562	-20.5551885782112\\
107.283783	-16.7800103529377\\
187.075284	-29.2600159550559\\
239.395072	-37.4432339564523\\
264.173965	-41.3188437592343\\
206.54244	-32.3048291606291\\
189.921	-29.7051078655595\\
98.338585	-15.3809124571348\\
67.126572	-10.4991131149543\\
108.678224	-16.9981116704179\\
160.401075	-25.0879641252242\\
221.018454	-34.5689892973878\\
282.518808	-44.1881185634516\\
309.48104	-48.405219409889\\
379.574832	-59.3684285972147\\
508.528514	-79.5377781345016\\
698.243	-109.210585619136\\
680.1468	-106.380200496075\\
736.159275	-115.140983198841\\
619.167774	-96.8426109463848\\
373.29006	-58.3854418281347\\
335.964112	-52.5473759400851\\
333.384177	-52.143854225998\\
429.37818	-67.1580559917946\\
361.396188	-56.5251485972695\\
267.111724	-41.7783319041894\\
211.229928	-33.0379884039909\\
186.555964	-29.1787903053555\\
296.940717	-46.4438162612959\\
271.121865	-42.4055488573494\\
154.726836	-24.2004694219003\\
122.719205	-19.1942292937627\\
188.294767	-29.4507524931703\\
249.43182	-39.0130586833621\\
190.107512	-29.7342797795565\\
265.843897	-41.5800339919529\\
214.562439	-33.5592188035685\\
313.56732	-49.0443450893498\\
467.499416	-73.1205110512585\\
377.745615	-59.082324963101\\
517.797682	-80.9875474347053\\
686.25426	-107.33545430205\\
733.8256	-114.775978446348\\
548.791404	-85.8352043824108\\
362.532654	-56.7029006368835\\
261.330476	-40.8741001686912\\
184.810374	-28.9057665784426\\
232.352344	-36.3416970284254\\
180.483452	-28.2290026358777\\
326.900332	-51.1297309057303\\
278.929413	-43.6267095268148\\
238.999039	-37.3812913435589\\
294.8025	-46.109382646135\\
191.294036	-29.9198612761002\\
259.7045	-40.6197849930823\\
189.0154	-29.563465047319\\
127.816623	-19.9915047479033\\
138.9417	-21.7315525167001\\
119.09366	-18.6271661185665\\
124.847919	-19.5271765664968\\
128.735022	-20.1351494283685\\
90.11557	-14.0947695474243\\
117.885145	-18.4381450601678\\
158.262394	-24.7534579368878\\
164.858236	-25.7850984509657\\
163.623745	-25.5920145460049\\
254.26577	-39.7691257121094\\
316.847776	-49.5574336858095\\
258.846346	-40.4855630948443\\
251.9066	-39.4001333451611\\
379.574832	-59.3684285972147\\
449.905446	-70.3686785702089\\
361.396188	-56.5251485972695\\
380.215024	-59.4685595591129\\
180.450525	-28.2238525993536\\
123.114719	-19.2560907310568\\
236.018097	-36.9150490446506\\
333.149816	-52.1071983596932\\
359.415596	-56.2153687467123\\
513.736512	-80.3523491527335\\
449.905446	-70.3686785702089\\
314.692749	-49.2203708571166\\
233.802532	-36.5685176062727\\
249.556725	-39.0325947878369\\
268.395182	-41.979074625273\\
203.832486	-31.8809714730606\\
135.00942	-21.1165136239101\\
180.450525	-28.2238525993536\\
138.48027	-21.6593813090801\\
115.468115	-18.0601029433703\\
97.908917	-15.3137090710617\\
155.765648	-24.3629476234264\\
227.19612	-35.5352238627461\\
243.80686	-38.1332715953652\\
182.4935	-28.5433896318126\\
314.31012	-49.1605247331098\\
416.168278	-65.0919255280572\\
283.309257	-44.3117508779783\\
390.899274	-61.1396585882493\\
421.336751	-65.9003145317231\\
294.8025	-46.109382646135\\
195.88558	-30.6380141385507\\
248.787675	-38.9123094418035\\
267.598947	-41.8545373357614\\
163.623745	-25.5920145460049\\
188.36598	-29.461890755113\\
149.08356	-23.3178175703667\\
102.283847	-15.9979818347596\\
104.061168	-16.2759685345807\\
160.928421	-25.1704451031701\\
200.499975	-31.3597410734831\\
182.856532	-28.6001706340117\\
219.573123	-34.3429283917669\\
237.80735	-37.1949020012155\\
175.605552	-27.4660614884669\\
107.536864	-16.8195941714924\\
95.565109	-14.9471194392876\\
115.468115	-18.0601029433703\\
136.58328	-21.3626774555311\\
166.846415	-26.0960649667857\\
319.869036	-50.0299820305945\\
309.805048	-48.4558967577826\\
274.597176	-42.9491142772943\\
278.929413	-43.6267095268148\\
371.031765	-58.0322270884942\\
477.343128	-74.6601434602996\\
516.807808	-80.8327235134785\\
524.144166	-81.9801864361167\\
395.686852	-61.88847267887\\
405.527536	-63.4276314853784\\
411.830626	-64.4134833308106\\
430.63062	-67.3539472586642\\
488.648016	-76.4283150550578\\
645.598296	-100.976548251649\\
547.49439	-85.6323413984693\\
494.008196	-77.2666885107516\\
391.716567	-61.2674895112265\\
390.510504	-61.0788519645221\\
370.188846	-57.9003880616519\\
426.505224	-66.708703535389\\
278.600322	-43.5752371585533\\
327.475967	-51.2197646553746\\
384.181776	-60.0889901383323\\
439.112037	-68.6805062323311\\
343.425632	-53.7144151639784\\
400.715522	-62.6749954160623\\
315.25494	-49.308301861566\\
284.325444	-44.470690355164\\
193.733002	-30.3013343523289\\
313.56732	-49.0443450893498\\
446.556474	-69.8448735433886\\
337.238494	-52.7466991048322\\
215.258238	-33.668047128755\\
247.8498	-38.765618564863\\
154.726836	-24.2004694219003\\
142.68441	-22.3169412007293\\
180.450525	-28.2238525993536\\
124.847919	-19.5271765664968\\
138.9417	-21.7315525167001\\
149.588928	-23.3968610197577\\
105.85802	-16.557010033301\\
157.705751	-24.6663946823893\\
134.250495	-20.9978119058964\\
103.70961	-16.2209821543963\\
151.666552	-23.7218174228099\\
174.336834	-27.2676242169646\\
195.0894	-30.5134854514629\\
201.818331	-31.5659421106786\\
200.2932	-31.3273998701464\\
262.457646	-41.0503982422735\\
229.411685	-35.8817553011239\\
285.927642	-44.7212864683478\\
451.757796	-70.6584003837775\\
330.061456	-51.6241550578589\\
295.088904	-46.1541784386652\\
303.70714	-47.502136958212\\
223.490084	-34.955571274914\\
152.42898	-23.8410670370165\\
241.73805	-37.8096937698307\\
193.815	-30.3141594713771\\
143.171475	-22.3931219198838\\
212.709882	-33.2694646136048\\
208.556595	-32.6198585230112\\
237.80735	-37.1949020012155\\
163.72907	-25.6084881876029\\
114.677087	-17.936380068782\\
94.887602	-14.841152123853\\
111.708016	-17.4719944857475\\
185.87084	-29.0716316324269\\
174.336834	-27.2676242169646\\
200.2932	-31.3273998701464\\
278.929413	-43.6267095268148\\
369.626571	-57.8124439189551\\
289.696524	-45.3107686548493\\
384.734097	-60.175377398727\\
448.565112	-70.1590400044585\\
388.570182	-60.7753706522716\\
374.555268	-58.5833301085901\\
609.764064	-95.371798240647\\
444.701081	-69.5546757811717\\
544.37427	-85.1443305879034\\
484.418296	-75.766754254298\\
553.8213	-86.6219188754502\\
589.285635	-92.1688141453537\\
359.86236	-56.2852460789212\\
224.61225	-35.1310866843372\\
191.920257	-30.0178070659427\\
284.166183	-44.4457807075537\\
343.549524	-53.7337928274473\\
344.415532	-53.8692431518055\\
233.293984	-36.4889767803777\\
143.632905	-22.4652931275038\\
201.52528	-31.5201066761289\\
351.925982	-55.0439354047364\\
474.574584	-74.2271218452568\\
451.757796	-70.6584003837775\\
283.397712	-44.3255859215819\\
228.870117	-35.7970498927882\\
297.704316	-46.5632490289257\\
447.034347	-69.9196165629127\\
510.678028	-79.8739786875133\\
509.171072	-79.6382791570313\\
380.00391	-59.4355397027414\\
525.85568	-82.2478804140518\\
572.25322	-89.5048131934527\\
563.355	-88.113063018846\\
748.029174	-116.997526867779\\
639.140928	-99.9665661382037\\
553.880418	-86.6311653861931\\
637.262265	-99.6727287686762\\
519.777159	-81.2971528906706\\
308.85048	-48.3065949670117\\
289.402953	-45.2648518882927\\
345.447104	-54.0305889615427\\
415.523271	-64.9910414630562\\
408.137981	-63.835925198565\\
314.49018	-49.1886875045899\\
388.225332	-60.7214334549768\\
337.248912	-52.7483285603097\\
266.350645	-41.6592932839778\\
331.899144	-51.9115836216472\\
325.80543	-50.9584797959941\\
475.371204	-74.3517193517351\\
474.574584	-74.2271218452568\\
345.614952	-54.0568416821213\\
260.517488	-40.7469425808868\\
177.706314	-27.7946368530125\\
254.26577	-39.7691257121094\\
201.603	-31.5322626830123\\
179.00232	-27.9973421779866\\
154.41625	-24.151891378232\\
251.9066	-39.4001333451611\\
320.600959	-50.14446042774\\
350.860872	-54.877343993367\\
343.302246	-53.6951166428085\\
215.51019	-33.70745436301\\
167.635374	-26.2194642338309\\
127.399149	-19.926208597393\\
159.359376	-24.9250343746518\\
230.861873	-36.1085758789712\\
244.087925	-38.1772323271549\\
238.999039	-37.3812913435589\\
379.574832	-59.3684285972147\\
454.613094	-71.1049909929167\\
503.054104	-78.6815382265909\\
518.120645	-81.0380613365465\\
589.285635	-92.1688141453537\\
360.319536	-56.3567519281614\\
281.547798	-44.0362451171843\\
236.783474	-37.0347599051811\\
262.307705	-41.02694631448\\
326.900332	-51.1297309057303\\
268.395182	-41.979074625273\\
194.6551	-30.4455575848972\\
179.934128	-28.1430841293769\\
316.19976	-49.4560789900223\\
373.909536	-58.4823326480026\\
356.114211	-55.699006693381\\
542.942946	-84.9204604850155\\
571.656932	-89.4115491554636\\
399.528998	-62.4894139195186\\
278.929413	-43.6267095268148\\
211.229928	-33.0379884039909\\
195.545747	-30.5848616387151\\
333.149816	-52.1071983596932\\
291.454332	-45.58570337111\\
324.521862	-50.7577198700629\\
363.609306	-56.8712973059916\\
410.622124	-64.224464305712\\
389.367918	-60.9001427097494\\
436.519281	-68.2749792150498\\
308.85048	-48.3065949670117\\
332.082928	-51.9403288554232\\
248.787675	-38.9123094418035\\
233.167047	-36.469122855445\\
338.985372	-53.0199242848701\\
472.570896	-73.9137296023299\\
491.212722	-76.8294548361955\\
295.496649	-46.2179528986072\\
581.727762	-90.9867045697975\\
479.21875	-74.9535051939942\\
315.713258	-49.3799863284061\\
203.40518	-31.8141375219883\\
297.841599	-46.5847211479811\\
267.111724	-41.7783319041894\\
269.595854	-42.1668689780363\\
314.692749	-49.2203708571166\\
224.516586	-35.1161240975835\\
304.003128	-47.5484317621932\\
550.70118	-86.1339080648843\\
570.008978	-89.1537964512999\\
394.001739	-61.6249080207719\\
442.081444	-69.1449443683547\\
490.703125	-76.7497499386172\\
473.113408	-73.9985826595398\\
350.893268	-54.8824109773994\\
335.171288	-52.4233721572575\\
250.767168	-39.2219174002921\\
333.931628	-52.2294797808662\\
377.845038	-59.0978754864203\\
599.684672	-93.7953036635372\\
673.460746	-105.334450130055\\
866.558028	-135.536352975745\\
580.563764	-90.8046463132314\\
484.418296	-75.766754254298\\
391.029936	-61.1600951318855\\
416.731773	-65.1800604881548\\
314.31012	-49.1605247331098\\
223.116432	-34.8971292228809\\
225.315384	-35.2410622600447\\
177.25673	-27.7243184510736\\
131.225994	-20.5247566437375\\
197.76548	-30.9320449844101\\
258.033358	-40.3584055070399\\
163.067102	-25.5049512915064\\
140.813055	-22.0242468587147\\
140.819952	-22.0253256026606\\
301.495176	-47.1561687439822\\
388.466254	-60.7591155122384\\
203.55	-31.8367884859212\\
240.889216	-37.6769295914094\\
253.47476	-39.6454056528599\\
236.018097	-36.9150490446506\\
277.215561	-43.3586498676638\\
257.22037	-40.2312479192355\\
183.449794	-28.6929613822835\\
139.40313	-21.8037237243202\\
126.564201	-19.7956162963724\\
153.422615	-23.9964792076242\\
278.929413	-43.6267095268148\\
317.836552	-49.7120858713755\\
233.293984	-36.4889767803777\\
165.40756	-25.8710169574665\\
126.123534	-19.7266925819449\\
158.34864	-24.7669474758697\\
248.787675	-38.9123094418035\\
300.39093	-46.9834561606443\\
324.895472	-50.8161553529723\\
230.136779	-35.9951655900476\\
224.61225	-35.1310866843372\\
251.13755	-39.2798479991277\\
348.770467	-54.5503885434289\\
481.572848	-75.3217042610595\\
553.111977	-86.5109752924524\\
416.64504	-65.1664947786203\\
276.049037	-43.1761965252332\\
288.768792	-45.1656642212673\\
273.497752	-42.7771558919131\\
267.560216	-41.8484795089137\\
204.1872	-31.936451475964\\
226.721652	-35.4610134114597\\
254.26577	-39.7691257121094\\
234.829632	-36.7291639607502\\
159.317086	-24.9184198927797\\
113.6172	-17.7706055748593\\
165.23508	-25.8440397563953\\
236.018097	-36.9150490446506\\
342.459437	-53.5632948208138\\
372.130395	-58.2040612861572\\
454.000911	-71.0092408544462\\
405.527536	-63.4276314853784\\
463.98807	-72.5713094796403\\
635.706112	-99.4293344482992\\
836.874176	-130.893569789445\\
680.642209	-106.457686281126\\
466.698024	-72.9951671672088\\
519.24787	-81.2143680125526\\
608.552791	-95.182345809796\\
795.77433	-124.465237173842\\
494.191844	-77.2954124730796\\
256.682745	-40.1471592264831\\
174.898658	-27.3554978197861\\
177.25673	-27.7243184510736\\
136.58328	-21.3626774555311\\
96.553903	-15.1017744401923\\
118.698146	-18.5653046812724\\
163.623745	-25.5920145460049\\
212.709882	-33.2694646136048\\
209.947416	-32.8373936445969\\
168.76454	-26.3960744971937\\
163.623745	-25.5920145460049\\
207.860796	-32.5110301978246\\
251.848795	-39.3910921977358\\
258.033358	-40.3584055070399\\
239.388175	-37.4421552125064\\
179.857263	-28.1310618510818\\
131.420562	-20.5551885782112\\
126.981675	-19.8609124468827\\
197.709	-30.9232110771947\\
215.276193	-33.67085542911\\
164.278394	-25.6944066941038\\
128.25729	-20.0604284623309\\
192.5052	-30.1092966585112\\
153.210732	-23.9633390737271\\
169.863188	-26.5679115101953\\
216.872502	-33.9205304573152\\
291.435327	-45.5827308426641\\
312.83774	-48.9302331554586\\
234.567909	-36.6882284668033\\
330.061456	-51.6241550578589\\
319.869036	-50.0299820305945\\
266.72525	-41.7178843925544\\
223.116432	-34.8971292228809\\
356.192926	-55.7113183259316\\
447.948072	-70.0625301937616\\
453.294738	-70.8987899557212\\
483.772562	-75.665756480895\\
376.150752	-58.8328761004381\\
345.614952	-54.0568416821213\\
335.964112	-52.5473759400851\\
437.815659	-68.4777427236904\\
369.112194	-57.7319913925438\\
484.921875	-75.8455178903036\\
438.187389	-68.5358841601133\\
485.424297	-75.9241005625938\\
816.17616	-127.65624059529\\
615.608202	-96.2858664567538\\
320.600959	-50.14446042774\\
222.435276	-34.7905911757283\\
246.990304	-38.631186767484\\
311.67033	-48.7476412358008\\
391.716567	-61.2674895112265\\
440.807026	-68.9456155729302\\
490.703125	-76.7497499386172\\
492.741656	-77.0685918952322\\
331.899144	-51.9115836216472\\
289.402953	-45.2648518882927\\
341.348008	-53.3894587609262\\
331.072192	-51.782241956641\\
208.556595	-32.6198585230112\\
164.497284	-25.7286427767944\\
266.298736	-41.651174316385\\
278.600322	-43.5752371585533\\
369.112194	-57.7319913925438\\
450.584784	-70.4749322681527\\
387.405636	-60.5932266842828\\
279.833946	-43.7681854580333\\
241.214604	-37.7278227819368\\
351.925982	-55.0439354047364\\
448.524335	-70.1526621674451\\
588.411131	-92.0320349811282\\
654.340626	-102.343916029021\\
740.069475	-115.75256860949\\
597.921985	-93.5196058340807\\
535.081638	-83.6908913372942\\
314.31012	-49.1605247331098\\
247.095922	-38.6477062365398\\
416.404006	-65.1287952036956\\
524.144166	-81.9801864361167\\
328.69238	-51.4100210218326\\
481.328832	-75.2835382783566\\
695.80744	-108.829645267552\\
773.281856	-120.947240921015\\
489.644166	-76.5841205910453\\
271.781433	-42.5087103749524\\
179.384804	-28.0571656228761\\
237.508568	-37.1481701941048\\
222.435276	-34.7905911757283\\
237.112352	-37.0861989586013\\
166.846415	-26.0960649667857\\
198.9834	-31.1225370572805\\
172.670844	-27.0070505434217\\
199.656	-31.2277368801035\\
180.144888	-28.1760486174206\\
201.52528	-31.5201066761289\\
317.646854	-49.6824156424283\\
302.396058	-47.2970736306674\\
416.731773	-65.1800604881548\\
504.504292	-78.9083588044382\\
479.408532	-74.9831885611887\\
276.049037	-43.1761965252332\\
287.641494	-44.9893461274987\\
339.303922	-53.0697479595062\\
234.567909	-36.6882284668033\\
200.9304	-31.4270628601893\\
160.903556	-25.166556025569\\
238.648	-37.3263861389738\\
197.0364	-30.8180112543717\\
123.15483	-19.2623644005383\\
75.905936	-11.8722732952978\\
91.38023	-14.292572116457\\
148.99154	-23.3034249332924\\
212.709882	-33.2694646136048\\
221.999613	-34.7224501254599\\
154.221468	-24.1214259725093\\
180.144888	-28.1760486174206\\
270.946467	-42.3781152585931\\
332.992892	-52.0826542368911\\
237.112352	-37.0861989586013\\
239.395072	-37.4432339564523\\
287.010984	-44.8907295050066\\
321.846588	-50.3392864017264\\
295.989786	-46.2950833253504\\
339.303922	-53.0697479595062\\
310.771846	-48.6071114147937\\
228.646308	-35.7620444405934\\
181.890478	-28.4490723443336\\
244.087925	-38.1772323271549\\
210.57075	-32.9348878853932\\
251.9066	-39.4001333451611\\
352.958696	-55.2054598320734\\
506.099776	-79.1579047962863\\
383.528034	-59.9867398520225\\
357.204298	-55.8695047000145\\
527.305779	-82.4746870716133\\
408.034871	-63.819798000035\\
239.395072	-37.4432339564523\\
192.12578	-30.0499524468318\\
156.80446	-24.5254258249525\\
154.726836	-24.2004694219003\\
147.031125	-22.9968009210122\\
146.452755	-22.906339396293\\
247.28385	-38.677099623848\\
149.588928	-23.3968610197577\\
126.564201	-19.7956162963724\\
185.262539	-28.9764886686697\\
231.011264	-36.1319417825266\\
179.857263	-28.1310618510818\\
135.173355	-21.1421543211365\\
90.7479	-14.1936708319407\\
144.094335	-22.5374643351239\\
271.781433	-42.5087103749524\\
348.296922	-54.4763224564548\\
286.830819	-44.8625503107174\\
231.62725	-36.2282867395018\\
259.7045	-40.6197849930823\\
269.230148	-42.1096697416323\\
295.989786	-46.2950833253504\\
368.77347	-57.6790123488538\\
430.399279	-67.3177637436304\\
411.794616	-64.40785108933\\
344.415532	-53.8692431518055\\
237.859424	-37.2030467760798\\
195.27034	-30.5417858617239\\
213.244083	-33.353017766373\\
277.147728	-43.3480402637662\\
197.0364	-30.8180112543717\\
161.942368	-25.3290342270951\\
295.989786	-46.2950833253504\\
362.120655	-56.6384608185617\\
343.549524	-53.7337928274473\\
406.779978	-63.6235230650634\\
495.474752	-77.496069169925\\
300.389856	-46.9832881787684\\
512.150194	-80.1042367939154\\
477.343128	-74.6601434602996\\
457.323048	-71.5288486804852\\
478.804304	-74.8886826460167\\
502.721448	-78.6295082648588\\
815.79381	-127.596438109033\\
775.435178	-121.28403707457\\
518.684802	-81.1262998385453\\
622.992232	-97.4407856475359\\
770.832649	-120.564166073472\\
752.66811	-117.723092204215\\
834.479424	-130.519012123514\\
775.37575	-121.274742077438\\
970.189846	-151.745167861903\\
766.54187	-119.893055174613\\
501.228682	-78.3960281597202\\
333.384177	-52.143854225998\\
321.97366	-50.3591614416993\\
269.595854	-42.1668689780363\\
230.041518	-35.9802660355992\\
328.69238	-51.4100210218326\\
446.556474	-69.8448735433886\\
269.595854	-42.1668689780363\\
246.990304	-38.631186767484\\
247.20685	-38.6650562305126\\
183.636068	-28.7220960712465\\
189.63064	-29.6596933241457\\
139.40313	-21.8037237243202\\
132.32388	-20.6964744740674\\
126.981675	-19.8609124468827\\
159.317086	-24.9184198927797\\
130.077792	-20.3451690032898\\
107.163328	-16.7611702627531\\
86.37286	-13.5093808633951\\
69.342192	-10.845653155756\\
125.170578	-19.5776429203955\\
106.021968	-16.5826527638276\\
98.568113	-15.4168124049982\\
213.42888	-33.3819214411547\\
293.82837	-45.957021207826\\
281.325429	-44.0014648920856\\
228.646308	-35.7620444405934\\
289.696524	-45.3107686548493\\
383.565107	-59.9925383548942\\
185.2556	-28.9754033556002\\
129.198888	-20.2077016451595\\
100.135437	-15.661953955855\\
76.447912	-11.9570425179775\\
112.849376	-17.650511984673\\
207.860796	-32.5110301978246\\
279.523665	-43.7196551187152\\
210.57075	-32.9348878853932\\
285.927642	-44.7212864683478\\
454.000911	-71.0092408544462\\
441.209808	-69.0086137814291\\
309.805048	-48.4558967577826\\
218.272656	-34.1395253320178\\
260.725685	-40.7795061959809\\
363.609306	-56.8712973059916\\
431.834832	-67.5422952947055\\
259.071395	-40.5207624925941\\
144.5814	-22.6136450542784\\
204.528285	-31.9897997982471\\
182.11104	-28.4835699407081\\
86.04815	-13.4585937172921\\
64.147785	-10.0332078746517\\
151.94559	-23.7654611161803\\
312.759153	-48.9179415430944\\
307.817741	-48.145066629482\\
243.01585	-38.0095515361157\\
159.317086	-24.9184198927797\\
142.401472	-22.2726875173069\\
85.74053	-13.4104795788788\\
92.873392	-14.5261141590471\\
113.864086	-17.8092204476774\\
144.560465	-22.6103706589605\\
208.556595	-32.6198585230112\\
320.857812	-50.1846342161604\\
338.271208	-52.9082235321692\\
355.953114	-55.6738098812232\\
377.751627	-59.0832652862274\\
371.606064	-58.1220518774438\\
352.663344	-55.159264503393\\
320.600959	-50.14446042774\\
405.132614	-63.365862641446\\
597.237876	-93.4126059149733\\
523.66014	-81.9044810399797\\
283.177242	-44.2911027147163\\
171.651123	-26.8475583214043\\
318.14189	-49.7598430874045\\
404.445336	-63.2583670663194\\
347.160456	-54.2985704168408\\
215.51019	-33.70745436301\\
224.516586	-35.1161240975835\\
372.401964	-58.246536770374\\
564.518185	-88.2949941159475\\
572.716397	-89.5772576453346\\
639.140928	-99.9665661382037\\
622.264032	-97.3268895562784\\
462.669714	-72.3651084424448\\
461.885424	-72.2424394429772\\
255.05678	-39.892845771359\\
215.258238	-33.668047128755\\
268.782866	-42.0397113902318\\
329.921592	-51.6022792505152\\
346.647666	-54.2183661094583\\
359.86236	-56.2852460789212\\
254.256475	-39.7676719024854\\
305.10777	-47.7212062829825\\
278.07546	-43.4931446973482\\
159.317086	-24.9184198927797\\
115.863629	-18.1219643806644\\
91.38023	-14.292572116457\\
136.12185	-21.290506247911\\
118.493037	-18.5332240531733\\
70.080732	-10.9611665026899\\
142.68441	-22.3169412007293\\
256.52011	-40.121721859274\\
192.5052	-30.1092966585112\\
169.313864	-26.4819930036945\\
234.567909	-36.6882284668033\\
354.24714	-55.4069824019774\\
269.091108	-42.0879228142381\\
127.856118	-19.9976820702382\\
99.923127	-15.6287470358675\\
193.733002	-30.3013343523289\\
419.314106	-65.58395726792\\
527.117215	-82.4451942090043\\
734.214429	-114.83679429813\\
882.565555	-138.040053547014\\
836.874176	-130.893569789445\\
572.961429	-89.6155825382664\\
582.541548	-91.1139869709511\\
726.19443	-113.582404655191\\
624.134325	-97.6194177933096\\
589.02371	-92.1278470570502\\
655.356756	-102.502846590968\\
715.96763	-111.982854331006\\
743.987264	-116.365340998218\\
981.548848	-153.521803303245\\
659.018153	-103.075517295227\\
347.737753	-54.3888641160918\\
198.38072	-31.0282732612368\\
168.985096	-26.430571155122\\
188.888029	-29.5435432414421\\
188.301554	-29.4518140322685\\
318.667363	-49.8420311137178\\
266.298736	-41.651174316385\\
271.266996	-42.4282484640787\\
347.160456	-54.2985704168408\\
197.15024	-30.8358167075834\\
223.797588	-35.0036672700335\\
308.85048	-48.3065949670117\\
336.148407	-52.5762010981987\\
198.3462	-31.0228740672376\\
138.134412	-21.6052864528179\\
179.00232	-27.9973421779866\\
191.189712	-29.9035441986155\\
373.360968	-58.3965323857272\\
583.375716	-91.2444572739612\\
560.23488	-87.6250522082802\\
635.706112	-99.4293344482992\\
756.704208	-118.354369032191\\
961.329088	-150.359277032766\\
798.8185	-124.941368819164\\
511.916364	-80.0676639801023\\
327.889108	-51.2843830912963\\
199.64538	-31.2260758302695\\
228.870117	-35.7970498927882\\
232.463913	-36.3591472797375\\
279.833946	-43.7681854580333\\
211.229928	-33.0379884039909\\
190.700774	-29.8270705278283\\
214.562439	-33.5592188035685\\
260.725685	-40.7795061959809\\
287.641494	-44.9893461274987\\
347.638202	-54.3732935725863\\
256.362216	-40.0970260209975\\
125.70606	-19.6613964314346\\
96.553903	-15.1017744401923\\
95.2172	-14.8927037908212\\
89.19271	-13.9504271321843\\
119.489174	-18.6890275558606\\
181.637049	-28.4094340958973\\
198.3462	-31.0228740672376\\
249.556725	-39.0325947878369\\
332.992892	-52.0826542368911\\
270.454008	-42.3010908762743\\
403.332858	-63.0843669347486\\
622.264032	-97.3268895562784\\
490.703125	-76.7497499386172\\
435.583548	-68.1286233634148\\
544.456256	-85.1571538300592\\
607.450364	-95.0099177320748\\
384.181776	-60.0889901383323\\
348.692894	-54.5382555284781\\
237.508568	-37.1481701941048\\
257.615328	-40.2930223938453\\
509.719712	-79.7240906806615\\
859.506336	-134.433414009045\\
1011.48384	-158.203866720743\\
781.385646	-122.214736122019\\
649.918521	-101.652264731809\\
493.383996	-77.1690588208822\\
536.97239	-83.9866195195754\\
668.7332	-104.59502550682\\
396.350388	-61.992254822249\\
314.49018	-49.1886875045899\\
262.180768	-41.0070923895474\\
462.878967	-72.397837224076\\
443.073192	-69.3000613976219\\
241.939698	-37.8412330708604\\
243.960512	-38.1573039521544\\
205.28508	-32.1081683678477\\
313.406838	-49.019244467931\\
487.18684	-76.1997758693408\\
472.570896	-73.9137296023299\\
357.247618	-55.8762802818234\\
282.821539	-44.235468024592\\
317.14458	-49.6038561184785\\
243.171936	-38.0339645892604\\
174.898658	-27.3554978197861\\
150.468565	-23.5344430247363\\
133.66665	-20.9064940489887\\
112.849376	-17.650511984673\\
129.950379	-20.3252406282893\\
221.716278	-34.6781343481783\\
303.21078	-47.4245024294334\\
296.64822	-46.3980674766151\\
179.573604	-28.0866953977036\\
164.678437	-25.7569765018968\\
150.468565	-23.5344430247363\\
189.51425	-29.6414890312846\\
262.143464	-41.0012577564956\\
277.215561	-43.3586498676638\\
253.878086	-39.7084889471506\\
163.991916	-25.6495993274033\\
292.021338	-45.6743874786622\\
421.72438	-65.9609426942581\\
304.863636	-47.6830218441703\\
149.47493	-23.3790308541286\\
154.89964	-24.2274972990682\\
240.889216	-37.6769295914094\\
245.66875	-38.4244855384458\\
210.628572	-32.9439316917495\\
153.422615	-23.9964792076242\\
246.990304	-38.631186767484\\
362.516688	-56.700403431455\\
226.487328	-35.42436333224\\
91.68785	-14.3406862548704\\
146.914185	-22.978510603913\\
305.10777	-47.7212062829825\\
316.847776	-49.5574336858095\\
221.018454	-34.5689892973878\\
193.733002	-30.3013343523289\\
359.415596	-56.2153687467123\\
473.113408	-73.9985826595398\\
406.361857	-63.5581257187676\\
571.2915	-89.354392774771\\
692.273196	-108.27686227229\\
447.034347	-69.9196165629127\\
380.058156	-59.4440241951424\\
500.073162	-78.2152959276826\\
343.984184	-53.8017770007869\\
469.407978	-73.4190248547774\\
558.459414	-87.347355644753\\
419.14992	-65.5582773123595\\
223.278237	-34.9224367716943\\
377.63833	-59.0655447623999\\
663.449544	-103.768620994394\\
750.094602	-117.320575722453\\
533.492416	-83.4423247704989\\
468.1365	-73.2201558980075\\
597.921985	-93.5196058340807\\
462.191841	-72.2903654229207\\
298.632048	-46.7083534625078\\
277.861232	-43.4596377873813\\
355.081497	-55.537482266044\\
363.692958	-56.8843811233849\\
357.204298	-55.8695047000145\\
395.686852	-61.88847267887\\
297.704316	-46.5632490289257\\
310.654143	-48.5887017586151\\
415.554361	-64.995904178625\\
259.09972	-40.5251927408568\\
205.4616	-32.1357774560498\\
172.670844	-27.0070505434217\\
130.367853	-20.3905367787996\\
156.376665	-24.4585153905122\\
271.781433	-42.5087103749524\\
248.525952	-38.8713739478565\\
352.958696	-55.2054598320734\\
438.28442	-68.5510605562006\\
516.807808	-80.8327235134785\\
390.510504	-61.0788519645221\\
337.238494	-52.7466991048322\\
150.195465	-23.4917280803222\\
292.878264	-45.8084173082377\\
437.815659	-68.4777427236904\\
295.72536	-46.253725061375\\
158.82047	-24.8407453236285\\
316.38717	-49.4853913581389\\
464.946972	-72.7212891414075\\
477.233367	-74.6429759941193\\
616.662131	-96.4507090735662\\
794.34705	-124.241999583716\\
566.457495	-88.5983171435998\\
484.0095	-75.7028154098573\\
550.70118	-86.1339080648843\\
376.646985	-58.9104907654381\\
294.639723	-46.0839230690317\\
338.325589	-52.9167291396104\\
433.054265	-67.7330240124345\\
457.621434	-71.5755185501446\\
441.776814	-69.0972978889812\\
270.454008	-42.3010908762743\\
231.018582	-36.1330863741166\\
196.732271	-30.7704431352588\\
276.884003	-43.3067916415925\\
258.30871	-40.4014726816072\\
208.39546	-32.5946557673606\\
395.207802	-61.8135455674763\\
416.784823	-65.1883579169398\\
299.693412	-46.8743589706118\\
245.605451	-38.4145850870856\\
346.524916	-54.1991670635893\\
186.482022	-29.1672252067841\\
123.510233	-19.317952168351\\
160.903556	-25.166556025569\\
202.8774	-31.7315886630981\\
190.8953	-29.8574958931784\\
136.58328	-21.3626774555311\\
98.238515	-15.3652607380299\\
156.271016	-24.4419910728175\\
220.353966	-34.4650582538731\\
322.588266	-50.4552905560406\\
367.975728	-57.5542393529298\\
478.095828	-74.7778715403456\\
531.63378	-83.1516197780926\\
288.768792	-45.1656642212673\\
242.464041	-37.9232443564663\\
501.337176	-78.4129974573374\\
712.131966	-111.382926925105\\
531.998272	-83.2086291355419\\
516.40625	-80.7699167480114\\
748.3568	-117.048770098751\\
599.684672	-93.7953036635372\\
469.042437	-73.3618514256448\\
359.269726	-56.1925535546885\\
373.360968	-58.3965323857272\\
472.570896	-73.9137296023299\\
321.02884	-50.2113843132431\\
284.500634	-44.4980914211175\\
490.583688	-76.731069071477\\
624.134325	-97.6194177933096\\
600.410536	-93.9088343863943\\
283.397712	-44.3255859215819\\
155.373735	-24.3016494102804\\
341.348008	-53.3894587609262\\
271.978791	-42.5395786869249\\
120.697689	-18.8780486142594\\
115.505632	-18.0659708912633\\
183.064784	-28.6327428515296\\
250.767168	-39.2219174002921\\
405.653838	-63.447386091418\\
295.088904	-46.1541784386652\\
221.018454	-34.5689892973878\\
152.17192	-23.800860872201\\
105.506462	-16.5020236531166\\
137.993205	-21.5832005899256\\
125.70606	-19.6613964314346\\
151.94559	-23.7654611161803\\
243.318875	-38.0569469811215\\
213.166791	-33.3409287019875\\
130.541658	-20.4177212200809\\
111.708016	-17.4719944857475\\
147.514515	-23.0724068418484\\
164.278394	-25.6944066941038\\
112.849376	-17.650511984673\\
170.845654	-26.7215767631386\\
128.29557	-20.0664157493033\\
106.932225	-16.7250239727533\\
209.874951	-32.8260595602067\\
286.784568	-44.8553162979232\\
440.7893	-68.9428430899398\\
591.149556	-92.4603457524844\\
495.711007	-77.5330212724244\\
247.9759	-38.785341576546\\
190.107512	-29.7342797795565\\
188.36598	-29.461890755113\\
117.885145	-18.4381450601678\\
122.719205	-19.1942292937627\\
201.603	-31.5322626830123\\
221.721588	-34.6789648730953\\
216.872502	-33.9205304573152\\
247.20685	-38.6650562305126\\
285.204348	-44.6081577132943\\
309.48104	-48.405219409889\\
315.25494	-49.308301861566\\
376.150752	-58.8328761004381\\
388.726464	-60.7998143613267\\
525.30687	-82.1620423011117\\
270.217332	-42.2640729261309\\
141.76155	-22.1725987854893\\
139.477182	-21.8153060277393\\
244.087925	-38.1772323271549\\
179.573604	-28.0866953977036\\
183.99094	-28.7776007865675\\
98.238515	-15.3652607380299\\
148.578192	-23.2387741209756\\
92.873392	-14.5261141590471\\
68.34852	-10.6902351980631\\
97.561008	-15.2592934225952\\
174.336834	-27.2676242169646\\
209.215773	-32.7229590416089\\
307.92762	-48.1622525517716\\
414.07058	-64.7638293967217\\
291.21452	-45.5481949264016\\
144.094335	-22.5374643351239\\
225.315384	-35.2410622600447\\
249.76798	-39.0656366976948\\
141.76155	-22.1725987854893\\
191.920257	-30.0178070659427\\
345.470224	-54.0342051076974\\
307.000566	-48.017254162614\\
505.592712	-79.0785960794264\\
577.604698	-90.3418255892919\\
393.293063	-61.5140656335596\\
307.92762	-48.1622525517716\\
};
\end{axis}
\end{tikzpicture}%
%	\caption{Scatter plots of the identified significant term $y(t-1)u(t)$ for training datasets. Correlation coefficient with the output signal is shown for each figure.}\label{fig:regr_first}
%\end{figure}
\end{document}